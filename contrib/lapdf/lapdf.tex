\documentclass[a4paper,11pt]{article}
\usepackage[color]{lapdf}
\usepackage{shortvrb}
\textheight24.62cm
\textwidth15.92cm
\oddsidemargin0cm
\topmargin-1cm
\parindent0cm
\parskip0cm
\unitlength1cm
\pagestyle{headings}

\MakeShortVerb{\|}

\newdimen\a
\newdimen\x
\newdimen\y
\Defnum(\n,0)

\title{\Huge{\bf\Lapdf} \\ \vspace{0.5cm}
 \Large Drawing in \TeX{} with PDF commands \\ \vspace{0.5cm}
 \normalsize Integral/Rational Bezier Curves \\
 Many Useful Drawing Primitives \\
 A Rich Set Of Math Functions \\
 Function, Parametric \& Polar Plots and Grids \\ \vspace{0.8cm}
 \begin{lapdf}(6,6.3)(-3,-3.3)
  \Whilenum{\n<360}{%
  \Nextcol(0,23) \Rad(\n,\a)
  \Cos(\Np\a,\x) \Add(\x,\x)
  \Sin(\Np\a,\y) \Add(\y,\y)
  \Circle(64)(\Np\x,\Np\y,1.5) \Stroke
  \Add(\n,15)}
 \end{lapdf}}
\author{Detlef Reimers, detlefreimers@gmx.de}
\date{\today}

% ---------------------------------------------------------------------------
\begin{document}
\maketitle

\begin{abstract}
This package started as a project for drawing arbitrary bezier curves.
It implements the integral and rational bezier curves of degree one up
to seven. Using quadratic rational bezier curves makes it possible to
exactly draw arbitrary conics. Also implemented are most elementary math
functions as power series and two loop commands for programming constructs.

This basic functionality gives the ability to draw many geometric
shapes like arbitray circles, rotated ellipses, rectangles and polygons.
The package also implements function, parametric and polar plotting
routines together with the possibility to draw special grids.

You are invited to extend this package to your own needs. This is not
very complicated, because you can use all of the math capabilities in
your \TeX{} documents directly and so, you can program your drawings
(see example above). The package also defines a rainbow color palette
and you can step through the colors.

This package needs the standard \LaTeX{} |calc| package for multiplication
and division of dimensions. No other packages are needed. Most of the drawing
primitives rely on the graphic capabilities of PDF, which is mostly like
PostScript without programming. With the help of the \pdfTeX{} engine this
programming possibility is back at the users hand. Now you can write your
PDF-graphics directly in your \pdfTeX{} source file.

The author hopes that \Lapdf{} may be useful and helps, to produce beautiful
and sophisticated drawings. I will be thankful for contructive criticism and
help from the users to further improve this package.
\end{abstract}

\tableofcontents
\parskip0.2cm

\section{Introduction}

\subsection{What is it for?}

\Lapdf{} is intended as a practical extension to the standard \LaTeX{}
|picture| environment. It gives the user the ability to quickly
draw many useful geometric shapes like arbitrary lines, circles,
rotated ellipses, arcs, vectors, rectangles, triangles and equilateral
polygons.

You can also draw bezier curves of degree one to seven. You may
choose the integral bezier form or the more general rational form.
The letter one enables you to exactly draw all kind of conic curves
like parabolas, hyperbolas, ellipses and circles. There is no need to
approximate them with quadratic or cubic bezier curves.

This package also implements a rich set of basic math functions like: |Sin|,
|Cos|, |Tan|, |Asin|, |Acos|, |Atan|, |Sinh|, |Cosh|, |Tanh|, |Asinh|, |Acosh|,
|Atanh|, |Exp|, |Pow|, |Ln|, |Log|, |Sqrt|, |Hypot|, |Len|. They can be used in
your documents to program special plot functions or in loop constructs in
connection with drawing functions. Al these functions are based on the ability
to do floating point multiplication and division with dimension values and
with dimension registers. These calculations are provided by the help of the
|calc| package, which belongs to the standard \LaTeX{} tools.

\subsection{How does it work?}

\TeX{} and also \LaTeX{} were made to typeset arbirary complicated and
structured text documents, with the special ability to integrate math formulas
into the document. |Donald Knuth|, the inventor of this marvellous
typesetting machine, only provided a small peephole to implement
inline graphics in \TeX{} documents. The basic shape of these drawing
capabilities is the |dot|. If you want to draw a line with pure \TeX{}
commands, you have to draw lots of evenly spaced dots. This uses much of
\TeX{} memory and it takes a lot of time to place them. |Leslie Lamport|,
the inventor of the \LaTeX{} macro package, used this technique together with
some specially designed graphic fonts for his implementation of the |picture|
environment, which allows to produce simple to moderate comlicated graphics
from inside the \LaTeX{} document. This approach is nice for simple
drawings, but you are limited to only a small set of line or vector slopes,
you only can draw circles of limited radii and so on. This is the price for
general portability, because the line and circle fonts are limited to special
drawing primitives.

Other new drawing packages avoid these limitations by using the |Postscript|
programming language for calculations and drawing. With the help of |DVIPS|
you can produce arbitrary sophisticated drawings to be integrated into you
\LaTeX{} file. But you are limited to output devices which can handle postscript.

So, there were other approaches to implement more general drawing capabilities.
The first one uses so called |Tpic| specials. |Tpic| is a simple graphic
language from Unix systems, which was used together with |Groff| to produce
documents with inline graphics. Another approach was done by
|Eberhard Matthes| in his |EmTeX| distribution. He implemented a small but
powerful set of graphic primitives into his DVI drivers, which could set
points, draw arbitrary lines and set the linewidth.

\Lapdf{} evolved out of my former |Ladraw| style, which had very similar
capabilities, but no way to fill or clip graphics. \Lapdf{} uses the |PDF|
capabilities directly, so it does not rely on other packages or tools with the
exception of the |calc| style to perform floating point arithmetic in \TeX{}.

The ability to draw arbitrary lines frees us from the use of single dots as
basic drawing primitve. We don't need to calculate the number of dots to draw
a smooth bezier curve, because we use straight line segments to draw them. This
way, we never have white spaces in curve shapes, but we have to take care to use
enough line segments to produce a smooth curve.

The Bezier curve drawing is invoked by two general calling routines |Curve|
for integral curves and |Rcurve| for rational curves. As an example, the
command:
\begin{verbatim}
  \Curve(64)(0,0)(3,6)(6,-7)(9,0)
\end{verbatim}
will draw an integral cubic bezier curve (4 points), consisting of 64 line segments
and the command:
\begin{verbatim}
  \Rcurve(96)(0,0,1)(5,8,2)(10,0,1)
\end{verbatim}
will draw a  rational quadratic bezier curve (3 points) with the weights (will be
explained later) $w_0=1$, $w_1=2$ and $w_2=1$, consisting of 96 line segments.

This package uses round brackets around command parameters with the only
exception of the |Text| command, where the last argument (the actual text)
cannot be bracketed with round brackets. This convention makes some functions
a little more static, but you don't have to remember complicated calling
syntaxes. My first decision was to make this style very easy to use. All
\Lapdf{} commands begin with a capital letter to distinguish them from other,
equally spelled \TeX{} commands. I hope that the names of the graphic macros
helps guessing what they do. The standiest one seems to be |concat|, which
is a native |PDF| matrix calculation commands, the others tell the user
what they mean.

\subsection{How to use it?}

Here are some simple examples that will show you the basic usage of the
\Lapdf{} package. First, we want to draw an ellipse at (1,2), rotated by an
angle of $30\deg$ with diameters $a=3cm$ and $b=2cm$ with red color and linewidth
of $0.3pt$. We also want to draw two axes with tick marks. The center point will
be marked and also the two main axes of the ellipse will be drawn dashed.
\vspace{0.1cm}

\begin{minipage}[c][7cm]{8cm}
\small{
\begin{verbatim}
  \documentclass{article}
  \usepackage[color]{Lapdf}
  \unitlength1cm
  \begin{document}
  \begin{lapdf}(6,6)(-2,-1)
   \Lingrid(10)(1,3)(-2,4)(-1,5)
   \Red
   \Ellipse(50)(1,2)(3,2,30) \Stroke
   \Blue
   \Dash(3)
   \Line(-2,0.5)(4,3.5) \Stroke
   \Line(-0.5,5)(2.5,-1) \Stroke
   \Dash(0)
   \Point(1)(1,2)
   \Text(1.1,2.2,cb){C}
  \end{lapdf}
  \end{document}
\end{verbatim}
}
\end{minipage}
\begin{minipage}[c][7cm]{7cm}
\begin{lapdf}(6,6.3)(-2,-1)
 \Lingrid(10)(1,3)(-2,4)(-1,5)
 \Red
 \Ellipse(50)(1,2)(3,2,30) \Stroke
 \Blue
 \Dash(3)
 \Line(-2,0.5)(4,3.5) \Stroke
 \Line(-0.5,5)(2.5,-1) \Stroke
 \Dash(0)
 \Point(1)(1,2)
 \Text(1.1,2.2,cb){C}
\end{lapdf}
\end{minipage}
\vspace{0.1cm}

From this example you can see how the package is invoked in the preamble
of the document. If you only want to draw in black and white, you would
use the option |black| instead of |color| or simply no option. This means that
all color commands will be silently ignored. The ellipse is drawn with 50
line segments in it's first part and with $2\cdot 50=100$ segments in
the second part. |Line| draws a line (solid or dashed) between two points. The
|Point| command draws a circle filled with a shade of gray (black..white).
|Puttext| puts the bracketed text at a specific point with optional position
parameters (here centered and bottom).

Our next example shows how to plot the function $f(x)=2\sin x$ from $x=-4$
to $x=4$. We also want to write the term into the picture. We have to
supply the function definition in |Fx|.
\vspace{0.1cm}

\begin{minipage}[c][5.7cm]{7.3cm}
\small{
\begin{verbatim}
  \documentclass{article}
  \usepackage[color]{Lapdf}
  \unitlength1cm
  \begin{document}
  \begin{lapdf}(8,4)(-7,-2)
   \Lingrid(10)(1,3)(-4,4)(-2,2)
   \Red
   \def\Fx(#1,#2){\Sin(#1,#2) #2=2#2}
   \Fplot(50)(-4,4) \Stroke
   \Text(-2,1.2,cb){$y=2\sin x$}
  \end{lapdf}
  \end{document}
\end{verbatim}
}
\end{minipage}
\begin{minipage}[c][5.7cm]{8.5cm}
\begin{lapdf}(8,4)(-4,-2)
 \Lingrid(10)(1,3)(-4,4)(-2,2)
 \Red
 \def\Fx(#1,#2){\Sin(#1,#2) #2=2#2}
 \Fplot(50)(-4,4) \Stroke
 \Text(-2,1.2,cb){$y=2\sin x$}
\end{lapdf}
\end{minipage}
\vspace{0.1cm}

The function definition has two parameters. The first is the
$x$-value and the second is the register for the function value
$f(x)$. The |Sin| function also takes two parameters, the
same way as |Fx| does. So, $\sin x$ is in register |\#2|.
At last we have to multiply this value with 2 to get $2\sin x$.
The function is plotted with 50 line segments. The last introductory
example will draw several bezier curves (integral and rational) and
axes without a grid. All curves are drawn in different colors.
\vspace{0.1cm}

\begin{minipage}[c][7.8cm]{9cm}
\small{
\begin{verbatim}
  \documentclass{article}
  \usepackage[color]{Lapdf}
  \unitlength1cm
  \begin{document}
  \begin{lapdf}(6,6)(0,0)
   \Lingrid(10)(0,2)(0,6)(0,6)
   \Stepcol(0,23,4)
   \Curve(32)(0,0)(6,6) \Stroke
   \Stepcol(0,23,4)
   \Curve(64)(0,0)(3,6)(6,0) \Stroke
   \Stepcol(0,23,4)
   \Curve(64)(0,3)(2,6)(4,0)(6,3) \Stroke
   \Stepcol(0,23,4)
   \Rcurve(64)(0,0,1)(3,6,5)(6,0,1) \Stroke
   \Stepcol(0,23,4)
   \Rcurve(64)(0,3,1)(2,6,10)(4,0,10)(6,3,1)
   \Stroke
  \end{lapdf}
  \end{document}
\end{verbatim}
}
\end{minipage}
\begin{minipage}[c][7.8cm]{6.5cm}
\begin{lapdf}(6,6)(0,0)
 \Lingrid(10)(0,2)(0,6)(0,6)
 \Stepcol(0,23,4)
 \Curve(64)(0,0)(6,6) \Stroke
 \Stepcol(0,23,4)
 \Curve(64)(0,0)(3,6)(6,0) \Stroke
 \Stepcol(0,23,4)
 \Curve(64)(0,3)(2,6)(4,0)(6,3) \Stroke
 \Stepcol(0,23,4)
 \Rcurve(64)(0,0,1)(3,6,5)(6,0,1) \Stroke
 \Stepcol(0,23,4)
 \Rcurve(64)(0,3,1)(2,6,10)(4,0,10)(6,3,1)
 \Stroke
\end{lapdf}
\end{minipage}
\vspace{0.1cm}

The grid is disabled with the first 0 value in |Lingrid|. The next
value 2 only shows ticks mark without text. The first three curves are
integral bezier curves with degree 1, 2 and 3. The next two are
rational bezier curves with degree 2 and 3. All curves are drawn with
64 line segments. The rational form needs a so called |weight| parameter.
Usually, the first and the last values are set to 1, the other may have
arbitrary values. If $w>1$, the curve is pulled into the direction of
the bezier point, otherwise it is pushed away from this point. This allows
much more control over the shape of the bezier curve, compared to the
integral form. The |Stepcol| command cycles between color 0 and color
23 in step of length 4. Please look at the file |colors.tex| to see a
whole color circle with all possible 96 colors.

\end{document}
