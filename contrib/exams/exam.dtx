% \CheckSum{1040}
%
%\iffalse   % this is a METACOMMENT !
%
% File: exam.dtx Copyright (C) 1993-1996 Hans van der Meer
%
%    \begin{macrocode}
%<*driver>
\documentclass{ltxdoc}
\setlength\overfullrule{5pt}
\usepackage{verbatim}
\IfFileExists{exam.cls}{\def\noexamples{}}%
  {\ClassWarningNoLine{ltxdoc}{Install exam.cls first}%
    \def\noexamples{\fbox{\huge Install exam.cls first}\end{document}}}
\makeatletter
  \let\old@maketitle=\maketitle
  \let\LoadClass=\@gobble
  \let\@oldtableofcontents=\tableofcontents
  % exam.cls
%
% A LaTeX2e document class for preparing exams.

%% exam.cls
%% Copyright (c) 1994, 1997, 2000, 2004, 2008, 2011, 2015 Philip S. Hirschhorn
%
% This work may be distributed and/or modified under the
% conditions of the LaTeX Project Public License, either version 1.3
% of this license or (at your option) any later version.
% The latest version of this license is in
%   http://www.latex-project.org/lppl.txt
% and version 1.3 or later is part of all distributions of LaTeX
% version 2003/12/01 or later.
%
% This work has the LPPL maintenance status "author-maintained".
% 
% This work consists of the files exam.cls and examdoc.tex.


% The user's guide for exam.cls is in the file examdoc.tex.


%%% Philip Hirschhorn
%%% Department of Mathematics
%%% Wellesley College
%%% Wellesley, MA 02481
%%% psh@math.mit.edu

% The newest version of this documentclass should always be available
% from my web page: http://www-math.mit.edu/~psh/


\def\fileversion{2.5}
\def\filedate{2015/05/07}
%---------------------------------------------------------------------
%---------------------------------------------------------------------
% 
% If there's some feature that you'd like that this file doesn't
% provide, tell me about it.
% 
%
%
%
%
% Thanks to:
%
% Piet van Oostrum, from whose excellent ``fancyheadings.sty'' we
% shamelessly stole most of the code for setting the headers and
% footers.
%
% Mate Wierdl <mw@wierdlmpc.msci.memphis.edu>, who contributed the
% code so that if the number of points is ``1'', then the default
% value of \pointname will print ``1 point'' instead of ``1 points''.
%
% Tom Brikowski <brikowi@utdallas.edu>, who contributed the code for
% making the number of points and number of questions available as
% macros (as well as the idea of putting the number of points in a 
% box, instead of in parentheses).  (I changed his code to make this
% all optional, so if there are errors there, it's my fault and not
% his.)
%
% Ottmar Beucher <beucher@fh-karlsruhe.de>, Dan Drake
% <drake@math.umn.edu>, and Justus Piater <Justus.Piater@ULg.ac.be> who
% contributed ideas and code for the \pointsofquestion and \gradetable
% commands for printing a Grading Table.  (I changed all the code to
% make this compatible with hyperref.sty, so if there are errors there,
% it's my fault and not theirs.)
%
% Justus Piater <Justus.Piater@ULg.ac.be>, who contributed the code for
% the solution environment.  (I changed his code to allow page breaks
% inside solutions so, once again, if it's buggy, it's my fault.) 
%
% Donald Arseneau <asnd@triumf.ca>, who created the excellent
% ``framed.sty'' and generously allowed me to include basically the
% whole thing in exam.cls, making the few changes needed for it to
% work well with question environments:
% framed.sty   v 0.8a   21-Jul-2003
% Copyright (C) 1992-2003 by Donald Arseneau
% These macros may be freely transmitted, reproduced, or modified
% provided that this notice is left intact.
%
%--------------------------------------------------------------------
%--------------------------------------------------------------------
%                    Changelog since version 2.4:

%--------------------------------------------------------------------
% Version 2.5 2015/05/07

% No longer betatest.

%--------------------------------------------------------------------
% Version 2.408$\beta$ 2013/11/17

% New commands:
%
%   \firstqinrange{whatever}
%   \lastqinrange{whatever}
%   \numqinrange{whatever}
%
% where ``whatever'' is the name of a grading range.
%
% \firstqinrange{whatever} prints the number of the first question in
% the range.
%
% \lastqinrange{whatever} prints the number of the last question in the
% range.
%
% \numqinrange{whatever} prints the number of questions in the range.
%
% We changed a couple of internal command name related to grading
% ranges.  If the user defines the range `myrange', then we now use

%   \range@myrange@firstp
%   \range@myrange@lastp
%   \range@myrange@firstq
%   \range@myrange@lastq
%
% where we used to use
%
%   \tbl@myrange@firstp
%   \tbl@myrange@lastp
%   \tbl@myrange@firstq
%   \tbl@myrange@lastq

%--------------------------------------------------------------------
% Version 2.407$\beta$ 2012/12/19

% New environment:
%
%   solutionbox
%
% The solutionbox environment is different from the other solution
% environments (solution, solutionorbox, solutionorlines,
% solutionordottedlines, and solutionorgrid), in that
%
%   (1) The box is always printed, whether answers are being printed
%   or not.
%
%   (2) The argument giving the size of the box is a required
%   argument, not an optional argument, and so it should be enclosed
%   in braces, not in brackets.  It can be either a length or
%   \stretch{number}.
%
%   (3) We make no use of the TheSolution environment; the solutionbox
%   environment is completely freestanding.
%
% If answers are not being printed then only the box is printed, with
% nothing in it.  If answers are being printed, then the solution is
% printed inside of the box.
%
% Note: It's the user's responsibility to be sure that the box is
% large enough to hold the solution!  If the solution takes up too
% much vertical space, then it will spill out of the bottom of the
% box, overwriting whatever follows the box.
%
%--------------------------------------------------------------------
% Version 2.406$\beta$, 2012/12/16
%
% New command:
%
%   \noquestionsonthispage
%
%   This command tells the \ifcontinuation and \ifincomplete commands
%   to assume that no part of any question is on this page.  This is
%   similar to the job done by the \nomorequestions command for the
%   pages that follow the end of all of the questions.
%
%
% If you give the command \noquestionsonthispage on a page, then
%
%   (1) \ifcontinuation on that page will expand to its second
%   argument,
%   (2) \ifincomplete on that page will expand to its second
%   argument, and
%   (3) an \ifincomplete on an earlier page will not assume that a
%   question from that earlier page continues onto this page.
%
% The way that this command affects the \ifincomplete command on
% earlier pages is as follows: If there is a page with no questions or
% parts or subparts or subsubparts, then the last page before that
% with a question (or part, etc.) would normally be deemed incomplete;
% if, however, the page with no questions (or parts, etc.) (along with
% all adjacent pages with no questions or parts etc.) has a
% \noquestionsonthispage command, then that last page with a question
% (or part, etc.) will not be deemed incomplete.
%
% Note that if you're tempted to use this command on a page that follows
% the end of all of the questions, then you should probably use the
% command \nomorequestions instead.
%
%--------------------------------------------------------------------
% Version 2.405$\beta$, 2012/10/21
%
% It is now possible to use a parts, subparts, or subsubparts
% environment inside one of the solution environments (solution,
% solutionorbox, solutionorlines, solutionordottedlines, or
% solutionorgrid) without getting problems from multiply defined
% labels or having its points (if any) counted as being actual points
% on the exam.
%
% Any \part, \subpart, or \subsubpart command inside one of the
% solution environments now writes a \PgInfo command in the .aux file
% of the form question2@object3, but no labels and no other \PgInfo
% commands.  In addition, if there are points assigned to any of these
% commands inside any of the solution environments, those points are
% not added to the points of the question or the points on the page,
% and do not affect any gradetables or pointtables.
%
%--------------------------------------------------------------------
% Version 2.404$\beta$, 2012/09/03
%
% New command:
%
%   \fillwithgrid{length}
%
% New environment:
%
%   solutionorgrid
%
% These are similar to the \fillwithlines command and the
% solutionorlines environment.
%
% By default, the created grids are in black.  However, if you give the
% commands
%
% \usepackage{color}
% \colorgrids
%
% then the grids will be in color, by default a light gray.  That
% default color was defined by the command
%
% \definecolor{GridColor}{gray}{0.8}
%
% You can change the color by redefining the color GridColor, and you
% can return to using black grids by giving the command
%
% \nocolorgrids
%
% The default grid size and grid line thickness were set by the
% commands
%
% \setlength{\gridsize}{5mm}
% \setlength{\gridlinewidth}{0.1pt}
%
% You can change either or both of those by giving new \setlength
% commands.  The period of the grid is \gridsize (both horizontally
% and vertically).  That is, the horizontal distance from the left
% edge of one vertical line to the left edge of the next vertical line
% is \gridsize, as is the vertical distance from the top edge of one
% horizontal line to the top edge of the next horizontal line.  Thus,
% each square has outer side length equal to \gridsize+\gridlinewidth.
%
%--------------------------------------------------------------------
% Version 2.403$\beta$, 2012/08/29:
%
% We changed the code for the command \fillin (which had been modified
% in version 2.402beta) so that if only one optional argument is used,
% a space following that optional argument will not be ignored.  We
% did this in such a way that the second optional argument will be
% recognized even when spaces appear in between the optional
% arguments.
%
%--------------------------------------------------------------------
% Version 2.402$\beta$, 2012/08/21:
%
% We modified the command \fillin that we had created in version
% 2.401beta.  \fillin now takes two optional arguments (and no required
% arguments).
%
% \fillin can take two optional arguments, as in
%
% \fillin[Answer][Length]
%
% The first optional argument is the answer to be printed above the line
% when \printanswers is in effect; the default value is empty.  That
% line is printed a distance of \answerclearance below the baseline.
%
% The second optional argument is the length of the line that we print;
% the default value is \fillinlinelength.  The value of
% \fillinlinelength is set with the command
%
%   \setlength\fillinlinelength{1in}
%
% and can be changed by giving a new \setlength command.
%
% When answers are being printed, the first optional argument is
% printed subject to the declarations in the argument of the last
% \CorrectChoiceEmphasis command.  It is centered on the line unless
% it is too long, in which case it extends to the right of the line.
%
%--------------------------------------------------------------------
% Version 2.401$\beta$, 2012/08/20:
%
%
% New command:
%
%   \fillin[CorrectAnswer]{width}
%
% This is for use in fill in the blank questions.  This command inserts
% a blank line of width ``width''.  If answers are being printed and if
% the optional argument ``CorrectAnswer'' appears, then the optional
% argument is printed subject to the declarations in the argument of the
% last \CorrectChoiceEmphasis command, and it is printed a distance of
% \answerclearance above the line.  It is centered on the line unless it
% is too long, in which case it extends to the right of the line.
%
% Note: We changed this command in version 2.401beta.
%
%--------------------------------------------------------------------
%--------------------------------------------------------------------
%--------------------------------------------------------------------

\NeedsTeXFormat{LaTeX2e}

\ProvidesClass{exam}[\filedate\space Version \fileversion\space by
  Philip Hirschhorn]

\RequirePackage{ifthen}

\newif\ifprintanswers
\printanswersfalse
\DeclareOption{answers}{\printanswerstrue}
\DeclareOption{noanswers}{\printanswersfalse}

\newif\ifcancelspace
\cancelspacefalse
\DeclareOption{cancelspace}{\cancelspacetrue}
\DeclareOption{nocancelspace}{\cancelspacefalse}

% The following keeps track of whether the user has requested that we
% add up the points on the exam.  We make the default false so that
% users who put other than numbers into the points argument of a
% question (or part, or subpart) won't get error messages.
% We use \if@printtotalpoints as a flag to signal that we are counting
% points, so that we will know to print the total on the screen (and
% in the log file).  We use this separate flag so that the user can
% use both \addpoints and \noaddpoints to count some points and not
% others, but still have the total printed when we finish the file no
% matter what the state of \if@addpoints.
\newif\if@addpoints
\newif\if@printtotalpoints
\def\addpoints{\global\@addpointstrue\global\@printtotalpointstrue}
\def\noaddpoints{\global\@addpointsfalse}
\@addpointsfalse
\@printtotalpointsfalse
\DeclareOption{addpoints}{\addpoints}


\DeclareOption*{%
  \PassOptionsToClass{\CurrentOption}{article}%
}
\ProcessOptions\relax
\LoadClass{article}




%                         *****************
%                         ** PAGE LAYOUT **
%                         *****************


% We set the parameters in terms of \paperwidth and \paperheight
% so that the options

% a4paper
% a5paper
% b5paper
% letterpaper
% legalpaper
% executivepaper
% landscape

% will all work:

\setlength{\textwidth}{\paperwidth}
\addtolength{\textwidth}{-2in}
\setlength{\oddsidemargin}{0pt}
\setlength{\evensidemargin}{0pt}

\setlength{\headheight}{15pt}
\setlength{\headsep}{15pt}
\setlength{\topmargin}{0in}
\addtolength{\topmargin}{-\headheight}
\addtolength{\topmargin}{-\headsep}
\setlength{\footskip}{29pt}
\setlength{\textheight}{\paperheight}
\addtolength{\textheight}{-2.2in}

\setlength{\marginparwidth}{.5in}
\setlength{\marginparsep}{5pt}

%--------------------------------------------------------------------

%                          ****************
%                          ** EXTRAWIDTH **
%                          ****************

\newlength\@extrawidth

% \@rightmargin is needed for \pointsinrightmargin and
% \pointsdroppedatright, so that we can right justify the points:
\newlength\@rightmargin
\setlength{\@rightmargin}{1in}

% We put the argument of \extrawidth into a length so that it will
% work correctly even if it's negative:

\def\extrawidth#1{%
  \@extrawidth=#1
  \advance \textwidth by \@extrawidth
  \divide\@extrawidth by 2
  \advance\oddsidemargin by -\@extrawidth
  \advance\evensidemargin by -\@extrawidth
  % Bug fix, 13 April 2004: 
  %\advance\@rightmargin by \@extrawidth
  \advance\@rightmargin by -\@extrawidth
}

%--------------------------------------------------------------------
%--------------------------------------------------------------------
%             Making room for large headers and footers

% The following are used to save the effect of any changes to 
% \topmargin and \textheight caused by \extraheadheight or
% \extrafootheight commands.  They hold the values currently in effect.
% We put them into lengths so that it will work correctly even if the
% argument is negative:

\newlength\@extrahead
\newlength\@extrafoot
\setlength{\@extrahead}{0in}
\setlength{\@extrafoot}{0in}

% The following are used to hold the requested values for extrahead and
% extrafoot, first page and all pages after the first, and then the
% similar things requested for the cover pages:
\newlength\run@exhd
\newlength\fp@exhd
\newlength\run@exft
\newlength\fp@exft
\newlength\covrun@exhd
\newlength\covfp@exhd
\newlength\covrun@exft
\newlength\covfp@exft

\setlength{\run@exhd}{0in}
\setlength{\fp@exhd}{0in}
\setlength{\run@exft}{0in}
\setlength{\fp@exft}{0in}
\setlength{\covrun@exhd}{0in}
\setlength{\covfp@exhd}{0in}
\setlength{\covrun@exft}{0in}
\setlength{\covfp@exft}{0in}

\newcommand*\adj@hdht@ftht{%
  \if@coverpages
    \ifnum\value{page}=1\relax
      \@setheadheight{\covfp@exhd}%
      \@setfootheight{\covfp@exft}%
    \else
      \@setheadheight{\covrun@exhd}%
      \@setfootheight{\covrun@exft}%
    \fi
  \else
    \ifnum\value{page}=1\relax
      \@setheadheight{\fp@exhd}%
      \@setfootheight{\fp@exft}%
    \else
      \@setheadheight{\run@exhd}%
      \@setfootheight{\run@exft}%
    \fi
  \fi
}

\newcommand*\extraheadheight{%
  \@ifnextchar[{\@xtrahd}{\@ytrahd}%
}

\def\@xtrahd[#1]#2{%
  \setlength{\fp@exhd}{#1}%
  \setlength{\run@exhd}{#2}%
  \adj@hdht@ftht
}

\def\@ytrahd#1{%
  \setlength{\fp@exhd}{#1}%
  \setlength{\run@exhd}{#1}%
  \adj@hdht@ftht
}

\newcommand*\extrafootheight{%
  \@ifnextchar[{\@xtraft}{\@ytraft}%
}

\def\@xtraft[#1]#2{%
  \setlength{\fp@exft}{#1}%
  \setlength{\run@exft}{#2}%
  \adj@hdht@ftht
}

\def\@ytraft#1{%
  \setlength{\fp@exft}{#1}%
  \setlength{\run@exft}{#1}%
  \adj@hdht@ftht
}

\newcommand*\coverextraheadheight{%
  \@ifnextchar[{\cov@xtrahd}{\cov@ytrahd}%
}

\def\cov@xtrahd[#1]#2{%
  \setlength{\covfp@exhd}{#1}%
  \setlength{\covrun@exhd}{#2}%
  \adj@hdht@ftht
}

\def\cov@ytrahd#1{%
  \setlength{\covfp@exhd}{#1}%
  \setlength{\covrun@exhd}{#1}%
  \adj@hdht@ftht
}

\newcommand*\coverextrafootheight{%
  \@ifnextchar[{\cov@xtraft}{\cov@ytraft}%
}

\def\cov@xtraft[#1]#2{%
  \setlength{\covfp@exft}{#1}%
  \setlength{\covrun@exft}{#2}%
  \adj@hdht@ftht
}

\def\cov@ytraft#1{%
  \setlength{\covfp@exft}{#1}%
  \setlength{\covrun@exft}{#1}%
  \adj@hdht@ftht
}

\def\@appendoutput#1{%
  \output=\expandafter{\the\output #1}%
}

\@appendoutput{\adj@hdht@ftht}

%--------------------------------------------------------------------
%                 \@setheadheight and \@setfootheight:

\def\@setheadheight#1{%
  \begingroup % Avoid trouble from using \@temp and \@spaces
    % Reset the effect of the most recent change:
    \global\advance\topmargin by -\@extrahead
    \global\advance\textheight by \@extrahead
    % Save the newly set value:
    \def\@temp{#1}
    \def\@spaces{ }
    \ifx\@temp\@empty
      \global\@extrahead=0in
    \else
      \ifx\@temp\@spaces
        \global\@extrahead=0in
      \else
        \global\@extrahead=#1
      \fi
    \fi
    % Set the new values:
    \global\advance\topmargin by \@extrahead
    \global\advance\textheight by -\@extrahead
    % Make it take effect RIGHT NOW!:
    % (The following stuff isn't necessary if \@setheadheight is
    % executed only in the preamble or as we return from the output
    % routine, but we're leaving it in so that this will still work if
    % we use this at some random point in the middle of composing a
    % page).
    % Bugfix, Version 2.306beta, 2009/03/28:
    % We don't do this!!
    % If the user had a figure environment that floated to the
    % top of a page, then this would cause that page to run
    % over the footer and off the bottom of the page, because
    % this somehow caused a full page's worth of stuff to be
    % placed after the figure, as if the figure wasn't taking
    % up space on the page.
    % We *do* need to put \@colht at the correct new value, though,
    % apparently because \@colht is set near the end of the
    % output routine.
    \global\@colht=\textheight
%     \global\@colroom=\textheight
%     \global\vsize=\textheight
%     \global\pagegoal=\textheight
  \endgroup
}

\def\@setfootheight#1{%
  \begingroup % Avoid trouble from using \@temp and \@spaces
    % Reset the effect of the most recent change:
    \global\advance\textheight by \@extrafoot
    % Save the newly set value:
    \def\@temp{#1}
    \def\@spaces{ }
    \ifx\@temp\@empty
      \global\@extrafoot=0in
    \else
      \ifx\@temp\@spaces
        \global\@extrafoot=0in
      \else
        \global\@extrafoot=#1
      \fi
    \fi
    % Set the new values:
    \global\advance\textheight by -\@extrafoot
    % Make it take effect RIGHT NOW!:
    % (The following stuff isn't necessary if \@setfootheight is
    % executed only in the preamble or as we return from the output
    % routine, but we're leaving it in so that this will still work if
    % we use this at some random point in the middle of composing a
    % page). 
    % Bugfix, Version 2.306beta, 2009/03/28:
    % We don't do this!!
    % If the user had a figure environment that floated to the
    % top of a page, then this would cause that page to run
    % over the footer and off the bottom of the page, because
    % this somehow caused a full page's worth of stuff to be
    % placed after the figure, as if the figure wasn't taking
    % up space on the page.
    % We *do* need to put \@colht at the correct new value, though,
    % apparently because \@colht is set near the end of the
    % output routine.
    \global\@colht=\textheight
%     \global\@colroom=\textheight
%     \global\vsize=\textheight
%     \global\pagegoal=\textheight
  \endgroup
}




%---------------------------------------------------------------------
%
%                      *************************
%                      ** HEADERS AND FOOTERS **
%                      *************************
% 
% The pagestyles available are head, foot, headandfoot, and empty.
% \pagestyle{head} prints the head, and gives an empty foot.
% \pagestyle{foot} prints the foot, and gives an empty head.
% \pagestyle{headandfoot} prints both the head and the foot.
% \pagestyle{empty} gives an empty head and an empty foot.
% 

%                            Pagestyles:

\newcommand*\ps@head{%
  \@dohead
  \@nofoot
}

\newcommand*\ps@headandfoot{%
  \@dohead
  \@dofoot
}

\newcommand*\ps@foot{%
  \@nohead
  \@dofoot
}

% \ps@empty is already defined by article.cls, so we'll
% say \def instead of \newcommand*:
\def\ps@empty{%
  \@nohead
  \@nofoot
}

\newif\if@coverpages
\@coverpagesfalse

\newcounter{num@coverpages}
% We'll set this to zero in case there is no coverpages environment:
\setcounter{num@coverpages}{0}

\newenvironment{coverpages}{%
    \ifnum \value{numquestions}>0\relax
      \ClassError{exam}{%
        Coverpages cannot be used after questions have begun.\MessageBreak
      }{%
        All question, part, subpart, and subsubpart environments
        \MessageBreak
        must begin after the cover pages are complete.\MessageBreak
      }%
    \fi
    \@coverpagestrue
    \pagenumbering{roman}%
    \adj@hdht@ftht
  }{%
    \clearpage
    \setcounter{num@coverpages}{\value{page}}%
    \addtocounter{num@coverpages}{-1}%
    \pagenumbering{arabic}%
    % Bugfix, Version 2.307\beta, 2009/06/11:
    % We have to say \@coverpagesfalse before \adj@hdht@ftht
    % because we're still inside the group created by the
    % coverpages environment and we want to set the 
    % extraheadheight and extrafootheight to the values correct
    % for the first non-cover page:
    \@coverpagesfalse
    \adj@hdht@ftht
}

\newcommand*\cover@question@error{%
  \ClassError{exam}{%
    No questions are allowed in the cover pages.\MessageBreak
  }{%
    All question, part, subpart, and subsubpart environments
    \MessageBreak
    must begin after the cover pages are complete.\MessageBreak
  }%
}

\newcommand*\@dohead{%
  \def\@oddhead{%
    \if@coverpages
      \ifnum\value{page}=1\relax
        \cov@fullhead
      \else
        \covrun@fullhead
      \fi
    \else
      \ifnum\value{page}=1\relax
        \@fullhead
      \else
        \run@fullhead
      \fi
    \fi
  }% @oddhead
  \let\@evenhead=\@oddhead
}

\newcommand*\@dofoot{%
  \def\@oddfoot{%
    \if@coverpages
      \ifnum\value{page}=1\relax
        \cov@fullfoot
      \else
        \covrun@fullfoot
      \fi
    \else
      \ifnum\value{page}=1\relax
        \@fullfoot
      \else
        \run@fullfoot
      \fi
    \fi
  }% @oddfoot
  \let\@evenfoot=\@oddfoot
}

\newcommand*\@nohead{%
  \def\@oddhead{}%
  \let\@evenhead=\@oddhead
}

\newcommand*\@nofoot{%
  \def\@oddfoot{}%
  \let\@evenfoot=\@oddfoot
}

%--------------------------------------------------------------------
%       \@fullhead, \run@fullhead, \@fullfoot, and \run@fullfoot:

\newcommand*\@fullhead{%
  \vbox to \headheight{%
    \vss
    \hbox to \textwidth{%
      \normalfont\rlap{\parbox[b]{\textwidth}{\raggedright\@lhead\strut}}%
        \hss\parbox[b]{\textwidth}{\centering\@chead\strut}\hss
        \llap{\parbox[b]{\textwidth}{\raggedleft\@rhead\strut}}%
    }% hbox
    \if@headrule
      \hrule
    \else
      % an invisible hrule, to keep positioning constant:
      \hrule width 0pt
    \fi
  }% vbox
}

\newcommand*\run@fullhead{%
  \vbox to \headheight{%
    \vss
    \hbox to \textwidth{%
      \normalfont\rlap{\parbox[b]{\textwidth}{\raggedright\run@lhead\strut}}%
        \hss\parbox[b]{\textwidth}{\centering\run@chead\strut}\hss
        \llap{\parbox[b]{\textwidth}{\raggedleft\run@rhead\strut}}%
    }% hbox
    \ifrun@headrule
      \hrule
    \else
      % an invisible hrule, to keep positioning constant:
      \hrule width 0pt
    \fi
  }% vbox
}

% We arrange it so that the very top of first line of text in the
% foot is at a fixed position on the page, whether or not there's
% a footrule:

\newcommand*\@fullfoot{%
  \vbox to 0pt{%
    \if@footrule
      \hrule
    \else
      % an invisible hrule, to keep positioning constant:
      \hrule width 0pt
    \fi
    \vskip 3pt
    \hbox to \textwidth{%
      \normalfont\rlap{\parbox[t]{\textwidth}{\raggedright\@lfoot}}%
        \hss\parbox[t]{\textwidth}{\centering\@cfoot}\hss
        \llap{\parbox[t]{\textwidth}{\raggedleft\@rfoot}}%
    }% hbox
    \vss
  }% vbox
}

\newcommand*\run@fullfoot{%
  \vbox to 0pt{%
    \ifrun@footrule
      \hrule
    \else
      % an invisible hrule, to keep positioning constant:
      \hrule width 0pt
    \fi
    \vskip 3pt
    \hbox to \textwidth{%
      \normalfont\rlap{\parbox[t]{\textwidth}{\raggedright\run@lfoot}}%
        \hss\parbox[t]{\textwidth}{\centering\run@cfoot}\hss
        \llap{\parbox[t]{\textwidth}{\raggedleft\run@rfoot}}%
    }% hbox
    \vss
  }% vbox
}

%--------------------------------------------------------------------
%       \cov@fullhead, \covrun@fullhead, \cov@fullfoot, and
%       \covrun@fullfoot: 

\newcommand*\cov@fullhead{%
  \vbox to \headheight{%
    \vss
    \hbox to \textwidth{%
      \normalfont\rlap{\parbox[b]{\textwidth}{\raggedright\cov@lhead\strut}}%
        \hss\parbox[b]{\textwidth}{\centering\cov@chead\strut}\hss
        \llap{\parbox[b]{\textwidth}{\raggedleft\cov@rhead\strut}}%
    }% hbox
    \ifcov@headrule
      \hrule
    \else
      % an invisible hrule, to keep positioning constant:
      \hrule width 0pt
    \fi
  }% vbox
}

\newcommand*\covrun@fullhead{%
  \vbox to \headheight{%
    \vss
    \hbox to \textwidth{%
      \normalfont\rlap{\parbox[b]{\textwidth}{\raggedright\covrun@lhead\strut}}%
        \hss\parbox[b]{\textwidth}{\centering\covrun@chead\strut}\hss
        \llap{\parbox[b]{\textwidth}{\raggedleft\covrun@rhead\strut}}%
    }% hbox
    \ifcovrun@headrule
      \hrule
    \else
      % an invisible hrule, to keep positioning constant:
      \hrule width 0pt
    \fi
  }% vbox
}

% We arrange it so that the very top of first line of text in the
% foot is at a fixed position on the page, whether or not there's
% a footrule:

\newcommand*\cov@fullfoot{%
  \vbox to 0pt{%
    \ifcov@footrule
      \hrule
    \else
      % an invisible hrule, to keep positioning constant:
      \hrule width 0pt
    \fi
    \vskip 3pt
    \hbox to \textwidth{%
      \normalfont\rlap{\parbox[t]{\textwidth}{\raggedright\cov@lfoot}}%
        \hss\parbox[t]{\textwidth}{\centering\cov@cfoot}\hss
        \llap{\parbox[t]{\textwidth}{\raggedleft\cov@rfoot}}%
    }% hbox
    \vss
  }% vbox
}

\newcommand*\covrun@fullfoot{%
  \vbox to 0pt{%
    \ifcovrun@footrule
      \hrule
    \else
      % an invisible hrule, to keep positioning constant:
      \hrule width 0pt
    \fi
    \vskip 3pt
    \hbox to \textwidth{%
      \normalfont\rlap{\parbox[t]{\textwidth}{\raggedright\covrun@lfoot}}%
        \hss\parbox[t]{\textwidth}{\centering\covrun@cfoot}\hss
        \llap{\parbox[t]{\textwidth}{\raggedleft\covrun@rfoot}}%
    }% hbox
    \vss
  }% vbox
}

%--------------------------------------------------------------------
%--------------------------------------------------------------------
%
%            ********************************************
%            ** COMMANDS TO DEFINE HEADERS AND FOOTERS **
%            ********************************************
%
% \lhead[#1]{#2} sets the first page left head to #1, and the
%   running left head to #2
% 
% \lhead{#1} sets both the first page left head and the running
%   left head to #1
% 
% \chead, \rhead, \lfoot, \cfoot, and \rfoot work similarly.
% 
% 
% \@lhead is the left head for Page 1
% \run@lhead is the running left head
% (i.e., for all pages other than the first)
% 
% \@chead is the center head for Page 1
% \run@chead is the running center head
% (i.e., for all pages other than the first)
% 
% etc.
% 
% Alternative commands are:
% \firstpageheader{LEFT}{CENTER}{RIGHT}
% \runningheader{LEFT}{CENTER}{RIGHT}
% or
% \header{LEFT}{CENTER}{RIGHT}
% which is equivalent to the two commands
%          \firstpageheader{LEFT}{CENTER}{RIGHT}
%          \runningheader{LEFT}{CENTER}{RIGHT}
% 
% Alternative commands are:
% \firstpagefooter{LEFT}{CENTER}{RIGHT}
% \runningfoother{LEFT}{CENTER}{RIGHT}
% or
% \footer{LEFT}{CENTER}{RIGHT}
% which is equivalent to the two commands
%          \firstpagefooter{LEFT}{CENTER}{RIGHT}
%          \runningfoother{LEFT}{CENTER}{RIGHT}

\def\firstpageheader#1#2#3{%
  \def\@lhead{#1}%
  \def\@chead{#2}%
  \def\@rhead{#3}%
}

\def\runningheader#1#2#3{%
  \def\run@lhead{#1}%
  \def\run@chead{#2}%
  \def\run@rhead{#3}%
}

\def\header#1#2#3{%
  \firstpageheader{#1}{#2}{#3}%
  \runningheader{#1}{#2}{#3}%
}

\def\firstpagefooter#1#2#3{%
  \def\@lfoot{#1}%
  \def\@cfoot{#2}%
  \def\@rfoot{#3}%
}

\def\runningfooter#1#2#3{%
  \def\run@lfoot{#1}%
  \def\run@cfoot{#2}%
  \def\run@rfoot{#3}%
}

\def\footer#1#2#3{%
  \firstpagefooter{#1}{#2}{#3}%
  \runningfooter{#1}{#2}{#3}%
}

\def\lhead{\@ifnextchar[{\@xlhead}{\@ylhead}}
\def\@xlhead[#1]#2{\def\@lhead{#1}\def\run@lhead{#2}}
\def\@ylhead#1{\def\run@lhead{#1}\def\@lhead{#1}}

\def\chead{\@ifnextchar[{\@xchead}{\@ychead}}
\def\@xchead[#1]#2{\def\@chead{#1}\def\run@chead{#2}}
\def\@ychead#1{\def\run@chead{#1}\def\@chead{#1}}

\def\rhead{\@ifnextchar[{\@xrhead}{\@yrhead}}
\def\@xrhead[#1]#2{\def\@rhead{#1}\def\run@rhead{#2}}
\def\@yrhead#1{\def\run@rhead{#1}\def\@rhead{#1}}

\def\lfoot{\@ifnextchar[{\@xlfoot}{\@ylfoot}}
\def\@xlfoot[#1]#2{\def\@lfoot{#1}\def\run@lfoot{#2}}
\def\@ylfoot#1{\def\run@lfoot{#1}\def\@lfoot{#1}}

\def\cfoot{\@ifnextchar[{\@xcfoot}{\@ycfoot}}
\def\@xcfoot[#1]#2{\def\@cfoot{#1}\def\run@cfoot{#2}}
\def\@ycfoot#1{\def\run@cfoot{#1}\def\@cfoot{#1}}

\def\rfoot{\@ifnextchar[{\@xrfoot}{\@yrfoot}}
\def\@xrfoot[#1]#2{\def\@rfoot{#1}\def\run@rfoot{#2}}
\def\@yrfoot#1{\def\run@rfoot{#1}\def\@rfoot{#1}}


%                    Initialize head and foot:

\pagestyle{headandfoot}

\lhead{}
\chead{}
\rhead{}
\lfoot{}
\cfoot[]{Page \thepage}
\rfoot{}

%--------------------------------------------------------------------
%                    Coverpage headers and footers
%
% \coverlhead[#1]{#2} sets the first cover page left head to #1, and the
%   running cover left head to #2
% 
% \coverlhead{#1} sets both the first cover page left head and the running
%   cover left head to #1
% 
% \coverchead, \coverrhead, \coverlfoot, \covercfoot, and \coverrfoot
% work similarly.
% 
% 
% \cov@lhead is the left head for Page 1
% \covrun@lhead is the running left head
% (i.e., for all pages other than the first)
% 
% \cov@chead is the center head for Page 1
% \covrun@chead is the running center head
% (i.e., for all pages other than the first)
% 
% etc.
% 
% Alternative commands are:
% \coverfirstpageheader{LEFT}{CENTER}{RIGHT}
% \coverrunningheader{LEFT}{CENTER}{RIGHT}
% or
% \coverheader{LEFT}{CENTER}{RIGHT}
% which is equivalent to the two commands
%          \coverfirstpageheader{LEFT}{CENTER}{RIGHT}
%          \coverrunningheader{LEFT}{CENTER}{RIGHT}
% 
% Alternative commands are:
% \coverfirstpagefooter{LEFT}{CENTER}{RIGHT}
% \coverrunningfoother{LEFT}{CENTER}{RIGHT}
% or
% \coverfooter{LEFT}{CENTER}{RIGHT}
% which is equivalent to the two commands
%          \coverfirstpagefooter{LEFT}{CENTER}{RIGHT}
%          \coverrunningfoother{LEFT}{CENTER}{RIGHT}

\def\coverfirstpageheader#1#2#3{%
  \def\cov@lhead{#1}%
  \def\cov@chead{#2}%
  \def\cov@rhead{#3}%
}

\def\coverrunningheader#1#2#3{%
  \def\covrun@lhead{#1}%
  \def\covrun@chead{#2}%
  \def\covrun@rhead{#3}%
}

\def\coverheader#1#2#3{%
  \coverfirstpageheader{#1}{#2}{#3}%
  \coverrunningheader{#1}{#2}{#3}%
}

\def\coverfirstpagefooter#1#2#3{%
  \def\cov@lfoot{#1}%
  \def\cov@cfoot{#2}%
  \def\cov@rfoot{#3}%
}

\def\coverrunningfooter#1#2#3{%
  \def\covrun@lfoot{#1}%
  \def\covrun@cfoot{#2}%
  \def\covrun@rfoot{#3}%
}

\def\coverfooter#1#2#3{%
  \coverfirstpagefooter{#1}{#2}{#3}%
  \coverrunningfooter{#1}{#2}{#3}%
}

\def\coverlhead{\@ifnextchar[{\cov@xlhead}{\cov@ylhead}}
\def\cov@xlhead[#1]#2{\def\cov@lhead{#1}\def\covrun@lhead{#2}}
\def\cov@ylhead#1{\def\covrun@lhead{#1}\def\cov@lhead{#1}}

\def\coverchead{\@ifnextchar[{\cov@xchead}{\cov@ychead}}
\def\cov@xchead[#1]#2{\def\cov@chead{#1}\def\covrun@chead{#2}}
\def\cov@ychead#1{\def\covrun@chead{#1}\def\cov@chead{#1}}

\def\coverrhead{\@ifnextchar[{\cov@xrhead}{\cov@yrhead}}
\def\cov@xrhead[#1]#2{\def\cov@rhead{#1}\def\covrun@rhead{#2}}
\def\cov@yrhead#1{\def\covrun@rhead{#1}\def\cov@rhead{#1}}

\def\coverlfoot{\@ifnextchar[{\cov@xlfoot}{\cov@ylfoot}}
\def\cov@xlfoot[#1]#2{\def\cov@lfoot{#1}\def\covrun@lfoot{#2}}
\def\cov@ylfoot#1{\def\covrun@lfoot{#1}\def\cov@lfoot{#1}}

\def\covercfoot{\@ifnextchar[{\cov@xcfoot}{\cov@ycfoot}}
\def\cov@xcfoot[#1]#2{\def\cov@cfoot{#1}\def\covrun@cfoot{#2}}
\def\cov@ycfoot#1{\def\covrun@cfoot{#1}\def\cov@cfoot{#1}}

\def\coverrfoot{\@ifnextchar[{\cov@xrfoot}{\cov@yrfoot}}
\def\cov@xrfoot[#1]#2{\def\cov@rfoot{#1}\def\covrun@rfoot{#2}}
\def\cov@yrfoot#1{\def\covrun@rfoot{#1}\def\cov@rfoot{#1}}


%                 Initialize coverpage head and foot:

\coverlhead{}
\coverchead{}
\coverrhead{}
\coverlfoot{}
\covercfoot{}
\coverrfoot{}




%--------------------------------------------------------------------
%--------------------------------------------------------------------

%                      Headrules and footrules:

\newif\if@headrule
\newif\ifrun@headrule

\def\firstpageheadrule{\@headruletrue}
\def\nofirstpageheadrule{\@headrulefalse}

\def\runningheadrule{\run@headruletrue}
\def\norunningheadrule{\run@headrulefalse}

\def\headrule{\@headruletrue\run@headruletrue}
\def\noheadrule{\@headrulefalse\run@headrulefalse}

\newif\if@footrule
\newif\ifrun@footrule

\def\firstpagefootrule{\@footruletrue}
\def\nofirstpagefootrule{\@footrulefalse}

\def\runningfootrule{\run@footruletrue}
\def\norunningfootrule{\run@footrulefalse}

\def\footrule{\@footruletrue\run@footruletrue}
\def\nofootrule{\@footrulefalse\run@footrulefalse}

%                             Initialize:

\noheadrule
\nofootrule

%                 Cover page headrules and footrules:

\newif\ifcov@headrule
\newif\ifcovrun@headrule

\def\coverfirstpageheadrule{\cov@headruletrue}
\def\nocoverfirstpageheadrule{\cov@headrulefalse}

\def\coverrunningheadrule{\covrun@headruletrue}
\def\nocoverrunningheadrule{\covrun@headrulefalse}

\def\coverheadrule{\cov@headruletrue\covrun@headruletrue}
\def\nocoverheadrule{\cov@headrulefalse\covrun@headrulefalse}

\newif\ifcov@footrule
\newif\ifcovrun@footrule

\def\coverfirstpagefootrule{\cov@footruletrue}
\def\nocoverfirstpagefootrule{\cov@footrulefalse}

\def\coverrunningfootrule{\covrun@footruletrue}
\def\nocoverrunningfootrule{\covrun@footrulefalse}

\def\coverfootrule{\cov@footruletrue\covrun@footruletrue}
\def\nocoverfootrule{\cov@footrulefalse\covrun@footrulefalse}

%                             Initialize:

\nocoverheadrule
\nocoverfootrule

%--------------------------------------------------------------------
%--------------------------------------------------------------------

%                \numpages, \iflastpage, and \oddeven
%     Also: \numpoints, \numquestions, \numparts, and \numsubparts
%                    Also also: \pointsofquestion
%     Also: \numcoverpages and \totalnumpages

% Make the number of pages available as the macro \numpages,
% the number of points as \numpoints,
% the number of questions as \numquestions,
% the number of parts as \numparts, and
% the number of subparts as \numsubparts

% This was previously done with \pageref commands.  When I stopped
% using \pageref for this (in order to make this compatible with
% hyperref.sty), this stuff was created:

% \gdef commands for exam@lastpage, exam@numpoints,
% exam@numbonuspoints, exam@numquestions, exam@numparts,
% exam@numsubparts and exam@numsubsubparts are written to the .aux
% file via \AtEndDocument.

% \gdef commands for pointsofq@i, pointsofq@ii, etc. and
% bonuspointsofq@i, bonuspointsofq@ii, etc.  are written to the .aux
% file as each question is completed (see the definition of the
% questions environment).

% \gdef commands for pointsonpage@i, pointsonpage@ii, etc. and
% bonuspointsonpage@i, bonuspointsonpage@ii, etc. are written to the
% .aux file as we encounter points defined for a later page, and for
% the last such page with AtEndDocument.

\def\numpages{\@ifundefined{exam@lastpage}%
  {\mbox{\normalfont\bf ??}}%
  \exam@lastpage
}% numpages

% Change 2011/04/01: We added a ``0'' in front of the
% mbox when \exam@lastcoverpage isn't defined.  This is
% so that the construction \romannumeral\numcoverpages
% won't generate an error on the first run of latex.
\def\numcoverpages{\@ifundefined{exam@lastcoverpage}%
  {0\mbox{\normalfont\bf ??}}%
  \exam@lastcoverpage
}% numpages

\def\totalnumpages{\@ifundefined{exam@totalpages}%
  {\mbox{\normalfont\bf ??}}%
  \exam@totalpages
}% numpages

\def\numpoints{\@ifundefined{exam@numpoints}%
  {\mbox{\normalfont\bf ??}}%
  \exam@numpoints
}% numpoints
\def\numbonuspoints{\@ifundefined{exam@numbonuspoints}%
  {\mbox{\normalfont\bf ??}}%
  \exam@numbonuspoints
}% numbonuspoints

\def\numquestions{\@ifundefined{exam@numquestions}%
  {\mbox{\normalfont\bf ??}}%
  \exam@numquestions
}% numquestions

\def\numparts{\@ifundefined{exam@numparts}%
  {\mbox{\normalfont\bf ??}}%
  \exam@numparts
}% numparts

\def\numsubparts{\@ifundefined{exam@numsubparts}%
  {\mbox{\normalfont\bf ??}}%
  \exam@numsubparts
}% numsubparts

\def\numsubsubparts{\@ifundefined{exam@numsubsubparts}%
  {\mbox{\normalfont\bf ??}}%
  \exam@numsubsubparts
}% numsubsubparts

\def\pointsofquestion#1{\@ifundefined{pointsofq@\romannumeral #1}%
  {\mbox{\normalfont\bf ??}}%
  {\csname pointsofq@\romannumeral #1\endcsname}%
}% pointsofquestion
\def\bonuspointsofquestion#1{\@ifundefined{bonuspointsofq@\romannumeral #1}%
  {\mbox{\normalfont\bf ??}}%
  {\csname bonuspointsofq@\romannumeral #1\endcsname}%
}% bonuspointsofquestion

% For use in \combinedgradetable and \combinedpointtable, we're
% changing the defintions of \pointsonpage and \bonuspointsonpage
% so that when, e.g.,  pointsonpage@ii is undefined, we just produce
% 0.  This comes up because the final page of the exam may have either
% points with no bonus point or bonus points with no points, in which
% case the combined table will try to list both points and bonus points
% for that last page, and one of those will be undefined.
%
% \def\pointsonpage#1{\@ifundefined{pointsonpage@\romannumeral #1}%
%   {\mbox{\normalfont\bf ??}}%
%   {\csname pointsonpage@\romannumeral #1\endcsname}%
% }% pointsonpage
% \def\bonuspointsonpage#1{\@ifundefined{bonuspointsonpage@\romannumeral #1}%
%   {\mbox{\normalfont\bf ??}}%
%   {\csname bonuspointsonpage@\romannumeral #1\endcsname}%
% }% bonuspointsonpage
%
% spanish.ldf redefines \@roman, so we'll avoid using \roman:
\def\pointsonpage#1{\@ifundefined{pointsonpage@\romannumeral #1}%
  {0}%
  {\csname pointsonpage@\romannumeral #1\endcsname}%
}% pointsonpage
\def\bonuspointsonpage#1{\@ifundefined{bonuspointsonpage@\romannumeral #1}%
  {0}%
  {\csname bonuspointsonpage@\romannumeral #1\endcsname}%
}% bonuspointsonpage


\newif\if@pointschanged
\@pointschangedfalse

\newcommand*{\CheckIfChanged@hlf}[2]{%
  % The first argument is the name of a half counter.
  % The second argument expands to the name (without the escape
  % character, and not assumed to be defined) of the control sequence
  % holding the previous value.
  \@ifundefined{#2}%
    {\global\@pointschangedtrue}%
    {%
    % OK; it's defined.  See if it's changed:
    \begingroup
      \set@hlfcntr{tmp@hlfcntr}{\csname #2\endcsname}%
      \edef\othpt@check{\prtaux@hlfcntr{tmp@hlfcntr}}%
      \edef\pt@check{\prtaux@hlfcntr{#1}}%
      \ifx \pt@check \othpt@check
        % Do nothing
      \else
        \global\@pointschangedtrue
      \fi
    \endgroup
    }%
}% CheckIfChanged@hlf


%%%\let\@realenddocument=\enddocument
%%%\def\enddocument{\clearpage
%%%   \if@filesw 
%%%     {\advance\c@page-1 \immediate\write\@mainaux
%%%       {\string\newlabel{@lastpage}{{}{\arabic{page}}}}%
%%%     }
%%%   \fi
%%%   \@realenddocument
%%%}

\AtEndDocument{%
  \clearpage
  \if@filesw
    \advance\c@page-1
    \immediate\write\@mainaux
      {\string\gdef\string\exam@lastpage{\arabic{page}}}%
    \immediate\write\@mainaux
      {\string\gdef\string\exam@lastcoverpage{\arabic{num@coverpages}}}%
    % We can now trash the value of num@coverpages:
    \addtocounter{num@coverpages}{\value{page}}%
    \immediate\write\@mainaux
      {\string\gdef\string\exam@totalpages{\arabic{num@coverpages}}}%
    \advance\c@page+1 % In case some other package looks at \c@page
    %
    \immediate\write\@mainaux
        {\string\gdef\string\exam@numpoints{%
                            \prtaux@hlfcntr{numpoints}}}%
    % See if this has changed from the last run of LaTeX:
    \CheckIfChanged@hlf{numpoints}{exam@numpoints}%
    \immediate\write\@mainaux
        {\string\gdef\string\exam@numbonuspoints{%
                            \prtaux@hlfcntr{numbonuspoints}}}%
    % See if this has changed from the last run of LaTeX:
    \CheckIfChanged@hlf{numbonuspoints}{exam@numbonuspoints}%
    \immediate\write\@mainaux
      {\string\gdef\string\exam@numquestions{\thenumquestions}}%
    \immediate\write\@mainaux
      {\string\gdef\string\exam@numparts{\thenumparts}}%
    \immediate\write\@mainaux
      {\string\gdef\string\exam@numsubparts{\thenumsubparts}}%
    \immediate\write\@mainaux
      {\string\gdef\string\exam@numsubsubparts{\thenumsubsubparts}}%
    \ifnum \thepageof@pagepoints > 0\relax
      \immediate\write\@mainaux
          {\string\gdef\string\pointsonpage@\romannumeral
                          \csname c@pageof@pagepoints\endcsname
             {\prtaux@hlfcntr{@pagepoints}}}%
      % See if this has changed from the last run of LaTeX:
      \CheckIfChanged@hlf{@pagepoints}{pointsonpage@\romannumeral
                          \csname c@pageof@pagepoints\endcsname}%
    \fi
    \ifnum \thepageof@pagebonuspoints > 0\relax
      \immediate\write\@mainaux
          {\string\gdef\string\bonuspointsonpage@\romannumeral
                          \csname c@pageof@pagebonuspoints\endcsname
             {\prtaux@hlfcntr{@pagebonuspoints}}}%
      \CheckIfChanged@hlf{@pagebonuspoints}{bonuspointsonpage@\romannumeral
                          \csname c@pageof@pagebonuspoints\endcsname}%
    \fi
    \immediate\write\@mainaux
      {\string\gdef\string\lastpage@withpoints{\page@withpoints}}%
    % See if this has changed from the last run of LaTeX:
    \@ifundefined{lastpage@withpoints}%
      {\global\@pointschangedtrue}%
      {%
      % OK; it's defined.  See if it's changed:
      \begingroup
        \edef\othpt@check{\page@withpoints}%
        \edef\pt@check{\lastpage@withpoints}%
        \ifx \pt@check \othpt@check
          % Do nothing
        \else
          \global\@pointschangedtrue
        \fi
      \endgroup
      }%
    \immediate\write\@mainaux
      {\string\gdef\string\lastpage@withbonuspoints{\page@withbonuspoints}}%
    % See if this has changed from the last run of LaTeX:
    \@ifundefined{lastpage@withbonuspoints}%
      {\global\@pointschangedtrue}%
      {%
      % OK; it's defined.  See if it's changed:
      \begingroup
        \edef\othpt@check{\page@withbonuspoints}%
        \edef\pt@check{\lastpage@withbonuspoints}%
        \ifx \pt@check \othpt@check
          % Do nothing
        \else
          \global\@pointschangedtrue
        \fi
      \endgroup
      }%
  \fi
  % Echo numbers of questions, parts, and subparts:
  \typeout{This exam contains \thenumquestions\space questions
    with \thenumparts\space parts, \thenumsubparts\space subparts,
    and \thenumsubsubparts\space subsubparts.}
  % If counting points, echo total points:
  \if@printtotalpoints
    \begingroup
      \def\typ@expnd{%
        \thenumpoints
        \ifnumpoints@half
          \space and a half%
        \fi
      }
      \typeout{This exam has a total of \typ@expnd\space points.}
      \def\typ@expnd{%
        \thenumbonuspoints
        \ifnumbonuspoints@half
          \space and a half%
        \fi
      }
      \typeout{This exam has a total of \typ@expnd\space bonus points.}
    \endgroup
  \fi
  \if@pointschanged
    \ClassWarningNoLine{exam}{Point totals have changed.
               Rerun to get point totals right}%
  \fi
}% AtEndDocument


% We define \iflastpage so that it can safely be used
% in headers and footers:
\def\iflastpage#1#2{%
  \@ifundefined{exam@lastpage}{\def\@@lastpage{-1}}%
          {\edef\@@lastpage{\exam@lastpage}}%
  \ifnum\value{page}=\@@lastpage\relax
    #1%
  \else
    #2%
  \fi
}% iflastpage


% The macro \oddeven takes two arguments.  If the page number is odd,
% then you get the first argument; otherwise, you get the second
% argument.
\def\oddeven#1#2{%
  \ifodd\value{page}%
    #1%
  \else
    #2%
  \fi
}% oddeven



%--------------------------------------------------------------------
%--------------------------------------------------------------------
%                 \ifcontinuation, \ContinuedQuestion,
%                \ifincomplete, and \IncompleteQuestion

% The commands \ifcontinuation, \ContinuedQuestion, \ifincomplete, and
% \IncompleteQuestion assume that there is only one questions
% environment in the entire document.  (Actually, \ContinuedQuestion
% should work even if there are multiple questions environments, but
% none of the other three will work in general.)



% \PgInfo@write, \PgInfo and \PgInfo@get are our replacements
% for \label, \newlabel, and \pageref.  (We're avoiding using
% \label, \newlabel, and \pageref so that we will be compatible
% with hyperref.sty, which redefines those commands.)

% We use \PgInfo, \PgInfo@write, and \PgInfo@get to know on which page
% each question, part, subpart, subsubpart, and choice appears.

% We use \PgInfo@write to write \PgInfo commands to the .aux file.  The
% \PgInfo command takes two arguments: A question (or part, or subpart,
% or subsubpart) label, and the number of the page on which it appears.

% The label for a question is of the form `question@2' (if it's question
% 2).

% The label for a part is of the form `part@2@1' (if it's part a of
% question 2).

% The label for a subpart is of the form `subpart@2@1@3' (if it's
% subpart iii of part a of question 2).

% The label for a subsubpart is of the form `subsubpart@2@1@3@4' (if
% it's subsubpart $\delta$ of subpart iii of part a of question 2).

% Each question, part, subpart, subsubpart, and choice also gets a
% \PgInfo@write command using a label of the form question2@object3 (if
% it's the third object of the second question).

% Inside one of the solution environments (solution, solutionorbox,
% etc.) each part, subpart, subsubpart, and choice get only the
% \PgInfo@write command using a label of the form question2@object3
% (if it's the third object of the second question).

% When read in from the .aux file, the \PgInfo command defines a
% control sequence of the form `Pg@label' that expands to the page
% number for the corresponding question.  For example,
% \PgInfo{subsubpart@2@1@3@4}{7} defines \csname
% Pg@subsubpart@2@1@3@4\endcsname to expand to 7.

% The \PgInfo@get{label} command returns the value of the macro Pg@label,
% but it *doesn't* check whether that macro is defined.  Thus, it's
% important to check that the macro Pg@label is defined before giving the
% command \PgInfo@get{label}.



% The token list to a \write command isn't expanded until
% it's shipped out.  Since the argument to PgInfo@write
% generally contains macros, we want to expand those macros
% now, rather than waiting until this is shipped out, at which
% point the macros may have different values.  Thus, we use
% \edef to force expansion of the argument, and we put
% a \noexpand in front of \thepage so that the \thepage
% will not be expanded now.  (This may not get shipped out
% until a later page, and so we want the \thepage to be expanded
% only when it's shipped out.)
% We use the \begingroup \endgroup pair so that our use
% of \reserved@a won't affect its use anywhere else.
\def\PgInfo@write#1{%
  \begingroup
    \edef\reserved@a{\write\@mainaux
      {\string\PgInfo{#1}{\noexpand\thepage}}}%
    \reserved@a
  \endgroup
}

%\PgInfo commands are written to the .aux file by the \PgInfo@write
%command; that's the only place that \PgInfo commands appear.
\def\PgInfo#1#2{\expandafter\gdef\csname Pg@#1\endcsname{#2}}

% Note: PgInfo@get assumes that the control sequence being
% constructed is already defined; you have to make sure of this
% *before* calling \Pginfo@get
\def\PgInfo@get#1{\csname Pg@#1\endcsname}

% \set@counter@to@pageof takes two arguments: The first is the name of a
% counter, and the second (expands to) the label of a question, part,
% subpart, subsubpart, or choice.  If that label exists, then we set the
% counter equal to the page on which the question (or part, etc.)
% appears.  If that label doesn't exist, we set the counter equal to -1.
% (No labels exist on the first run of LaTeX on the file, and if a new
% question (or part, etc.) was created by editing the file, then the
% label will not exist until the run of LaTeX after that.)
\def\set@counter@to@pageof#1#2{%
  \@ifundefined{Pg@#2}%
  {\setcounter{#1}{-1}}%
  {\setcounter{#1}{\csname Pg@#2\endcsname}}%
}


%--------------------------------------------------------------------

% \ifcontinuation#1#2 expands to #2 if either:
% (1) The command \noquestionsonthispage has been given on this page,
% or
% (2) The current page is before the page containing question number
% 1, or
% (3) A question begins on this page before any part, subpart,
% subsubpart, or choice begins, or
% (4) The current page is later than a page with the \nomorequestions
% command.
% Otherwise, it expands to #1.
%
% Thus, for example, if there are no questions, parts, subparts,
% subsubparts, or choices on this page, and no \noquestionsonthispage
% command, and we're not before question 1, and not after a
% \nomorequestions command, then \ifcontinuation will expand to #1.


\def\ifcontinuation#1#2{%
  % If there's a \noquestionsonthispage command on this page, then
  % we assume that we're not continuing anything:
  \@ifundefined{No@Questions@Pg@\thepage}%
    {\chk@contin{#1}{#2}}%
    {#2}%
}% \ifcontinuation

\def\chk@contin#1#2{%
  % We check whether we're on a page *before* the page on which the
  % first question appears.  If we don't yet know which page has
  % question number 1, then we must be doing an early run of LaTeX,
  % and we'll assume we're not a continuation.
  \expandafter\ifx\csname Pg@question@1\endcsname\relax
    % No page info yet; assume not a continuation
    #2%
  \else
    % Note: The ``\relax'' at the end of the following \ifnum
    % serves an entirely different purpose from the one at the
    % end of the above \expandafter\ifx.  That one (above)
    % is one of the things being compared, whereas the
    % one we're about to use is just to clearly mark the
    % end of the second number being compared by the \ifnum
    % (since it's conceivable that the ``#2'' would begin
    % with a digit).
    \ifnum \thepage < \csname Pg@question@1\endcsname\relax
      % We're before the page with question 1:
      #2%
    \else
      % The current page begins a new question if Contin@\thepage
      % has been defined as a macro that expands to \relax (Note
      % that this is different from if Contin@\thepage has never
      % been defined at all, in which case it will be let equal to
      % \relax (temporarily) by the \csname command.)
      \expandafter\ifx\csname Contin@\thepage\endcsname\ref@relax
        #2%
      \else
        % See if we're after a \nomorequestions command:
        \@ifundefined{Pg@@endquestions}%
        {#1}%
        {\ifnum \thepage > \PgInfo@get{@endquestions}\relax
          % We're after a \nomorequestions:
          #2%
         \else
           % We actually are incomplete:
           #1%
         \fi
        }%
      \fi
    \fi 
  \fi
}% chk@contin

\def\nomorequestions{%
  \PgInfo@write{@endquestions}%
}% nomorequestions

\def\noquestionsonthispage{%
  \write\@mainaux{\string\expandafter\string\gdef
    \string\csname\space No@Questions@Pg@\thepage\string\endcsname
    {No questions here}}%
}% noquestionsonthispage


%--------------------------------------------------------------------
% \ContinuedQuestion is for use in headers and footers, where we can
% assume that \thepage is the number of the page on which we'll
% actually appear.

% \ContinuedQuestion expands to the number of the question that
% continues onto this page, or to -1 if this page begins with a new
% question.

% ACTUALLY: \ContinuedQuestion expands to a positive number if either
% (1) this page doesn't contain the beginning of any question, part,
% subpart, subsubpart, or choice, or (2) this page has a part, subpart,
% subsubpart, or choice that appears before any question.  That means
% that if the current page actually begins with space for a continuation
% of the previous question (but doesn't begin any part, subpart,
% subsubpart, or choice of that question) and then has a question, then
% we'll be asserting that this page begins with a new question, but the
% actual top of the page will begin with some blank space that's
% intended for the previous question.

% \ContinuedQuestion works by examining the value of the macro
% Contin@\thepage.  If this page starts with a question (i.e., if no
% question continues onto this page), then the macro Contin@\thepage
% will be defined, and will expand to `\relax' (and so an \ifx between
% \csname Contin@\thepage\endcsname and \ref@relax will be true).

% If Contin@\thepage is undefined, then when it is used in an \ifx
% command it will be temporarily set equal to \relax (which is
% *different* from being a macro that expands to \relax); in this case,
% there is no question, part, subpart, subsubpart, or choice that begins
% on this page, and so \ContinuedQuestion will be set equal to the last
% question that was begun on a page before this one.

% The last possibility is that this page begins with either a part,
% subpart, subsubpart, or choice.  In this case, Contin@\thepage is
% defined, and it expands to the number of the question that is
% continued onto this page.


\def\ref@relax{\relax}

\def\ContinuedQuestion{%
  \expandafter\ifx\csname Contin@\thepage\endcsname\relax
    % We get here if there's no question, part, subpart,
    % subsubpart, or choice on this page, and so Contin@\thepage has
    % never been defined at all.  In that case, this page
    % continues whichever question was last begun on or 
    % before this page.
    \find@latestques
    \thelatest@ques
  \else
    \expandafter\ifx\csname Contin@\thepage\endcsname\ref@relax
      % We get here if this page begins with a new question,
      % which is why Contin@\thepage has been defined to be
      % a macro that expands to \relax.
      % ACTUALLY: We get here if this page has a question that
      % appears before any part, subpart, subsubpart, or choice.  That
      % means that if the current page actually begins with space
      % for a continuation of the previous question but doesn't begin
      % any part, subpart, subsubpart, or choice of that question, then
      % we'll be asserting that this page begins with a new question,
      % but the actual top of the page will begin with some space
      % that's intended for the previous question.
      -1\relax
    \else
      % We get here if we didn't get anywhere above.  This happens
      % if Contin@\thepage has been defined to be a macro that expands
      % to something other than \relax, in which case it has been
      % defined to be a macro that expands to the number of the
      % question that continues onto this page.
      \csname Contin@\thepage\endcsname
    \fi
  \fi
}

%--------------------------------------------------------------------
% \find@latestques is for use in headers and footers, where we can
% assume that \thepage actually equals the page on which we'll appear.
% We find the last question that was started on or before the current
% page.

\newcounter{latest@ques}

\newcommand\find@latestques{%
  % \find@latestques is for use in headers and footers.
  % \find@latestques will set the counter latest@ques
  % to the number of the last question
  % that was begun on the exam from page 1 through the current
  % page.  This may well be the value of the question counter,
  % but it may be less than that if the page following this one
  % begins a new question and that question beginning was
  % typeset before the present page was shipped out.
  % Note: This macro is called both by \ContinuedQuestion and by
  % \find@quesend, which is why it has to find the last question
  % begun on or before the current page, rather than just before
  % the current page.
  \ifnum 1 > \value{question}\relax
    % Oops; probably because we're before the first question
    % Just set latest@ques to -1:
    \setcounter{latest@ques}{-1}%
  \else
    % If question latest@ques actually begins on this page (rather
    % than on the next page, but early enough on the next page
    % that the counter was advanced before we ran off into the
    % output routine to output the page and set the header and 
    % footer), then that's the correct question number.
    \expandafter\ifx\csname Pg@question@\arabic{question}\endcsname\relax
      % We don't know what page that question is on;
      % this must be an early run, before the aux file
      % is helpful.  Just set it equal to -1 and wait until the next
      % run to get it right.
      \setcounter{latest@ques}{-1}%
    \else
      % We now know that \PgInfo@get can tell us the page number
      % of \arabic{question} and of all earlier questions.
      % Set latest@ques equal to the current question number, and
      % then call \decr@latest@ques to recursively decrement
      % latest@ques as needed to find a question that begins on
      % or before the current page:
      \setcounter{latest@ques}{\value{question}}%
      \decr@latest@ques
    \fi
  \fi
}

\def\decr@latest@ques{%
  % If we get here, then we've already checked that the reference
  % Pg@question@\thelatest@ques is defined at least for a value of
  % \thelatest@ques greater than or equal to it's present value,
  % so we assume it's defined for all lesser values as well:
  \ifnum \thepage < \PgInfo@get{question@\thelatest@ques}\relax
    % Nope; latest@ques starts on a later page
    % Decrement latest@ques and see if that one's right:
    \addtocounter{latest@ques}{-1}%
    \ifnum \thelatest@ques < 1\relax
      \setcounter{latest@ques}{-1}%
      \let\next@dlq=\relax
    \else
      \let\next@dlq=\decr@latest@ques
    \fi
  \else
    % latest@ques starts on this page or earlier, so
    % that's the correct question number!  Exit:
    \let\next@dlq=\relax
  \fi
  \next@dlq
}

%--------------------------------------------------------------------
%--------------------------------------------------------------------
%--------------------------------------------------------------------
\newcounter{ques@end}
\newcounter{last@object}

\def\find@quesend{%
  % We find the last question started on or before the current page
  % and then find the page containing the last part (or subpart, or
  % subsubpart, or choice) of that question, and set the counter
  % ques@end to that page number.
  % Set latest@ques equal to the correct question number:
  \find@latestques
  \ifnum \value{latest@ques} < 0\relax
    % This must be an early run of LaTeX, before we have
    % \PgInfo commands in the .aux file:
    \setcounter{ques@end}{-1}%
  \else
    % We now know that this question has at least one object (since
    % we know that latest@ques isn't negative).
    % We'll find its highest numbered object by setting last@object
    % equal to 2 and then calling \find@lastobject to recursively
    % test whether that object number exists and, if so, incrementing
    % last@object to test for a higher numbered one:
    \setcounter{last@object}{2}%
    \find@lastobject
    \setcounter{ques@end}{\PgInfo@get{question\thelatest@ques
               @object\thelast@object}}%
  \fi
}% find@quesend

\def\find@lastobject{%
  % We check whether this question has an object numbered last@object
  % and recursively increment last@object to find the highest
  % numbered value for which the object exists:
  \@ifundefined{Pg@question\thelatest@ques @object\thelast@object}%
    {\addtocounter{last@object}{-1}%
      \let\nextfind@lastobject=\relax
    }%
    {\addtocounter{last@object}{1}%
      \let\nextfind@lastobject=\find@lastobject
    }%
  \nextfind@lastobject
}% find@lastobject



%--------------------------------------------------------------------
%--------------------------------------------------------------------
%--------------------------------------------------------------------

\newcounter{incmp@ques}

\def\IncompleteQuestion{%
  \Find@Incmp@ques
  % If there's no incomplete question, the counter incmp@ques will be
  % set to -1:
  \theincmp@ques
}

\def\Find@Incmp@ques{%
  % If we're on the last page, then there's no incomplete question:
  \iflastpage{\setcounter{incmp@ques}{-1}}{\chk@incomp}%
}% Find@Incmp@ques
\newcounter{next@ques}
\newcounter{next@page}

\def\chk@incomp{%
  % If we get here, we're not on the last page.
  % \find@quesend calls \find@latestques to set the counter
  % latest@ques equal to the number of the last question begun on or
  % before the current page, and then it sets the counter ques@end to
  % the page containing the last ques@object of that question:
  \find@quesend
  \ifnum \theques@end > \thepage\relax
    % This question has a part (or sub...) starting on a later page
    \setcounter{incmp@ques}{\value{latest@ques}}%
  \else
    \chk@incompi
  \fi
}% chk@incomp

\def\chk@incompi{%
  % If there are any pages after the current one and before the next
  % question (if there is a next question) that lack a
  % \noquestionsonthispage and that aren't following the page of a
  % \nomorequestions command, then question latest@ques is incomplete.
  % Otherwise, there is no incomplete question:
  \setcounter{next@ques}{\thelatest@ques}%
  \addtocounter{next@ques}{1}%
  % We use next@page as a scratch counter.  We start by setting it
  % to the last page we want to check for a \noquestionsonthispage
  % command:
  \expandafter\ifx\csname Pg@question@\thenext@ques \endcsname\relax
    % This isn't the last page but there is no next question:
%
    \@ifundefined{exam@lastpage}%
      {\setcounter{next@page}{-1}}%
      {\setcounter{next@page}{\exam@lastpage}}%
%
%    \setcounter{next@page}{\exam@lastpage}%
  \else
    \setcounter{next@page}{\PgInfo@get{question@\thenext@ques}}%
    \addtocounter{next@page}{-1}%
  \fi
  % See if that's after a \nomorequestions command:
  \@ifundefined{Pg@@endquestions}%
    {}%
    {\ifnum \PgInfo@get{@endquestions} < \value{next@page}\relax
       \setcounter{next@page}{\PgInfo@get{@endquestions}}%
     \fi
    }%
  % OK, the counter next@page now contains the last page to check.
  \chk@incompii
}% chk@incompi

\def\chk@incompii{%
  \ifnum \value{next@page} > \value{page}\relax
    % We need to check the page next@page:
    \@ifundefined{No@Questions@Pg@\arabic{next@page}}%
    {\setcounter{incmp@ques}{\value{latest@ques}}%
      \let\next@incompii=\relax
    }%
    {\addtocounter{next@page}{-1}%
      \let\next@incompii = \chk@incompii
    }%
  \else
    % There's no incomplete question:
    \setcounter{incmp@ques}{-1}%
    \let\next@incompii=\relax
  \fi
  \next@incompii
}% chk@incompii

\def\ifincomplete#1#2{%
  % We need to pass the arguments to \chk@ifincomp; we save them in
  % macros so they won't be messed up by the call to \Find@Incmp@ques:
  \def\incomp@first{#1}%
  \def\incomp@second{#2}%
  % If there's a \noquestionsonthispage command on this page, then
  % we assume nothing from this page is incomplete:
  \@ifundefined{No@Questions@Pg@\thepage}%
    {\chk@ifincomp}%
    {\incomp@second}%
}% ifincomplete

\def\chk@ifincomp{%
  \Find@Incmp@ques
  % If there's no incomplete question, \Find@Incmp@ques sets the
  % counter incmp@ques to -1:
  \ifnum \theincmp@ques < 0\relax
    \incomp@second
  \else
    % Are we after a page with \nomorequestions?
    \@ifundefined{Pg@@endquestions}%
      {\incomp@first}%
      {\ifnum \thepage < \PgInfo@get{@endquestions}\relax
         \incomp@first
       \else
         \incomp@second
       \fi
      }%
  \fi
}% chk@ifincomp

%--------------------------------------------------------------------
% These are the commands for dealing with hlfcntr's, i.e., the things
% used to count points.
%
% A point value is a nonnegative integer with an optional half integer.
% A hlfcntr consists of a regular counter together with an \if: If the
% regular counter is called ``counter'', then the \if is called
% ``\ifcounter@half''; it's set true by ``\counter@halftrue'' and set
% false by ``\counter@halffalse''.
%
% The commands:
%
% \new@hlfcntr{countername}
% \set@hlfcntr{countername}{value}
% \copy@hlfcntr{tocounter}{fromcounter}
% \addto@hlfcntr{countername}{value}
% \add@hlfcntrtohlfcntr{getsaddedto}{whatsadded}
% \ifhlfcntr@pos{countername}
% \prtaux@hlfcntr{countername}
% \prt@hlfcntr{countername}
%
% ``value'' can be either a (nonnegative) integer, an integer followed by
% ``\half'', or just plain ``\half''.  (Actually, ``value'' can be empty
% (although the braces must be present), in which case it's interpreted
% as ``0''.)
%
% Examples of valid values:
%   0
%   0\half
%   1
%   1\half
%   2
%   2\half
% etc.

% Note on using ``\global'': LaTeX's \setcounter and \addtocounter
% commands are already \global (i.e., you don't have to say
% ``\global''), but \somethingtrue and \somethingfalse (used to set
% \ifsomething) aren't.  Thus, we need to say ``\global'' when setting
% these things.

% To create a hlfcntr:
\newcommand*\new@hlfcntr[1]{%
  \newcounter{#1}%
  \expandafter\newif\csname if#1@half\endcsname
}% new@hlfcntr


% A scratch hlfcntr:
\new@hlfcntr{tmp@hlfcntr}

\newcommand*\horiz@half{$\frac{1}{2}$}
\newcommand*\slanted@half{%
  $\raise0.6ex\hbox{$\scriptstyle 1$}\kern -.2em/\kern -.2em
     \raise-0.5ex\hbox{$\scriptstyle 2$}$%
}% slanted@half
\newcommand*\useslantedhalf{\global\let\half\slanted@half}
\newcommand*\usehorizontalhalf{\global\let\half\horiz@half}
\newcommand*\half{\slanted@half}


\newcommand*\set@hlfcntr[2]{%
  \begingroup
    \global\csname #1@halffalse\endcsname
    % If there as a `\half' present, it will be executed
    % right after the assignment of the digit part of #2
    % to the counter #1.
    \def\half{%
      \global\csname #1@halftrue\endcsname
    }%
    % We insert a `0' in case there are no digits present:
    % We avoid using \setcounter, because calc.sty redefines
    % \setcounter in a way that conflicts with the \half trick
    % we're using:
    %    \setcounter{#1}{0#2}\relax
    \global\csname c@#1\endcsname 0#2\relax
  \endgroup
}% set@hlfcntr

\newcommand*\copy@hlfcntr[2]{%
  % We set #1 to the value of #2
  \setcounter{#1}{\value{#2}}%
  \csname if#2@half\endcsname
    \global\csname #1@halftrue\endcsname
  \else
    \global\csname #1@halffalse\endcsname
  \fi
}% copy@hlfcntr

\newcommand*\addto@hlfcntr[2]{%
  % We add the valueandhalf #2 to hlfcntr #1
  \begingroup
    \def\half{\add@half{#1}}%
    % We insert a `0' in case there are no digits present:
    % We avoid using \addtocounter, because calc.sty redefines
    % \addtocounter in a way that conflicts with the \half trick
    % we're using:
    %    \addtocounter{#1}{0#2}\relax
    \global\advance\csname c@#1\endcsname 0#2\relax
  \endgroup
}% addto@hlfcntr

\newcommand*\add@hlfcntrtohlfcntr[2]{%
  % We add the hlfcntr #2 to the hlfcntr #1
  \addtocounter{#1}{\value{#2}}%
  \csname if#2@half\endcsname
    \add@half{#1}%
  \fi
}% add@hlfcntrtohlfcntr

\newcommand*\add@half[1]{%
  % We add one half to hlfcntr #1:
  \csname if#1@half\endcsname
    \addtocounter{#1}{1}%
    \global\csname #1@halffalse\endcsname
  \else
    \global\csname #1@halftrue\endcsname
  \fi
}% add@half

\newcounter{ifpos@cntr}
\def\ifhlfcntr@pos#1{%
  % The argument must be a hlfcntr (which, of course,
  % can never be negative); we'll be true if and only if
  % that halfcntr is positive:
  \setcounter{ifpos@cntr}{\value{#1}}%
  \csname if#1@half\endcsname
    \addtocounter{ifpos@cntr}{1}%
  \fi
  \ifnum \value{ifpos@cntr} > 0\relax
}% ifhlfnctr@pos

% \prtaux@hlfcntr is used inside the argument of a \write command for
% writing to the .aux file:
\newcommand*\prtaux@hlfcntr[1]{%
  % We don't want a \relax after the 0 in the following
  % line, because it would sometimes appear in the aux file:
  \ifnum \value{#1} = 0 
    % We have to make the following a macro, because if we
    % don't do this part, the \fi will cause confusion, since
    % there's no \if visible until the \csname is expanded:
    \prtaux@halforzero{#1}%
  \else
    \arabic{#1}%
    % We have to make the following a macro, because if we
    % don't do this part, the \fi will cause confusion, since
    % there's no \if visible until the \csname is expanded:
    \prtaux@halforblank{#1}%
  \fi
}% prtaux@hlfcntr
\newcommand*\prtaux@halforzero[1]{%
  \csname if#1@half\endcsname
    \string\half
  \else
    0%
  \fi
}% prtaux@hlforzero
\newcommand*\prtaux@halforblank[1]{%
  \csname if#1@half\endcsname
    \string\half
  \fi
}% prtaux@halforblank

\newcommand*\prt@hlfcntr[1]{%
  % We don't want a \relax after the 0 in the following
  % line, because it would sometimes appear in the aux file:
  \ifnum \value{#1} = 0 
    % We have to make the following a macro, because if we
    % don't do this part, the \fi will cause confusion, since
    % there's no \if visible until the \csname is expanded:
    \prt@halforzero{#1}%
  \else
    \arabic{#1}%
    % We have to make the following a macro, because if we
    % don't do this part, the \fi will cause confusion, since
    % there's no \if visible until the \csname is expanded:
    \prt@halforblank{#1}%
  \fi
}% prt@hlfcntr
\newcommand*\prt@halforzero[1]{%
  \csname if#1@half\endcsname
    \half
  \else
    0%
  \fi
}% prt@hlforzero
\newcommand*\prt@halforblank[1]{%
  \csname if#1@half\endcsname
    \half
  \fi
}% prt@halforblank

% End of the commands for dealing with hlfcntr's

%--------------------------------------------------------------------
%---------------------------------------------------------------------
%
%                    ***************************
%                    ** QUESTION ENVIRONMENTS **
%                    ***************************
%
%
%

% We define the command \part only inside of a parts environment, so
% that we don't interfere with the meaning of the standard article
% documentclass command \part if that is used inside of a questions
% environment.  The commands \question, \subpart, and \subsubpart are
% defined everywhere inside of a questions environment.  If the user
% accidentally gives a \subpart command outside of a subparts
% environment, then an error will be created.




% We use the counter name `partno' for the parts environment so that
% we will not interfere with the counter `part' used by the article
% document class.

\newcounter{question}
\newcounter{partno}
\newcounter{subpart}
\newcounter{subsubpart}
\newcounter{choice}
\new@hlfcntr{numpoints}
\set@hlfcntr{numpoints}{0}
\new@hlfcntr{numbonuspoints}
\set@hlfcntr{numbonuspoints}{0}
\new@hlfcntr{pointsof@thisquestion}
\set@hlfcntr{pointsof@thisquestion}{0}
\new@hlfcntr{bonuspointsof@thisquestion}
\set@hlfcntr{bonuspointsof@thisquestion}{0}
\newcounter{numquestions}
\newcounter{numparts}
\newcounter{numsubparts}
\newcounter{numsubsubparts}
\newcounter{Curr@Page}

% @pagepoints accumulates the points on a single page:
\new@hlfcntr{@pagepoints}
\set@hlfcntr{@pagepoints}{0}
\new@hlfcntr{@pagebonuspoints}
\set@hlfcntr{@pagebonuspoints}{0}
\newcounter{pageof@pagepoints}
\setcounter{pageof@pagepoints}{0}
\newcounter{pageof@pagebonuspoints}
\setcounter{pageof@pagebonuspoints}{0}

% latest@points is a holding area for points until we know
% whether they'll land on the same page as the points
% currently counted in @pagepoints:
\new@hlfcntr{latest@points}
\set@hlfcntr{latest@points}{0}
\new@hlfcntr{latest@bonuspoints}
\set@hlfcntr{latest@bonuspoints}{0}

% Whenever we meet a new page on which points are defined, we'll
% redefine \page@withpoints to expand to that page.  At the end of the
% document, it will hold the last page that has points, and we'll write
% a \gdef\lastpage@withpoints command to the .aux file.
% We initialize \page@withpoints here:
\def\page@withpoints{0}%
\def\page@withbonuspoints{0}%

% \pageinfo@commands is used by each question, part, subpart, and
% subsubpart to insert into everypar the \PgInfo@write command to put
% its page number into the .aux file, the \PgInfo@get command to read
% the page number into the counter Curr@Page, and to test and set
% \Contin@\theCurr@Page.  \temp@toks is used by part, subpart, and
% subsubpart to append all that to \pageinfo@commands, rather than
% deleting whatever may have been put into \pageinfo@commands by the
% current question and/or part and/or subpart.
\newtoks\pageinfo@commands
\newtoks\temp@toks

% \pagepoint@commands holds the commands to manage the counting of the
% number of points defined on each page.
\newtoks\pagepoint@commands

% \point@toks holds the commands to print the points at the proper
% location on the page (except that it's not used by the \qformat
% option).
\newtoks\point@toks

% We'll use \greeknum to number subsubparts
\def\greeknum#1{\expandafter\lc@greek\csname c@#1\endcsname}
\def\lc@greek#1{%
  \ifcase #1\or $\alpha$\or $\beta$\or $\gamma$\or $\delta$\or
  $\epsilon$\or $\zeta$\or $\eta$\or $\theta$\or $\iota$\or
  $\kappa$\or $\lambda$\or $\mu$\or $\nu$\or $\xi$\or o\or $\pi$\or
  $\rho$\or $\sigma$\or $\tau$\or $\upsilon$\or $\phi$\or $\chi$\or
  $\psi$\or $\omega$\else \@ctrerr
  \fi
}% lc@greek


% The following macros are a variation on a trick from Victor
% Eijkhout's ``TeX by Topic'', page 142:
% Both \prepend@toklist and \append@toklist take two arguments,
% both of which should be token lists.
% \prepend@toklist prepends #2 to #1
% \append@toklist appends #2 to #1

\def\prepend@toklist#1#2{%
  \edef\do@it{\noexpand#1={\the#2\the#1}}%
  \do@it
}% prepend@toklist

\def\append@toklist#1#2{%
  \edef\do@it{\noexpand#1={\the#1\the#2}}%
  \do@it
}% append@toklist


% The command \qformat is provided for the user who wants to
% design a nonstandard question line.  If this command is used,
% then the usual line containing the question number and the beginning
% of the question will be replaced by the line specified by the
% \qformat command, and the question will begin on the following
% line.
% Within the argument of the \qformat command:
% \thequestion will be replaced by the question number, and
% \thepoints will be replaced by ``\@points \@pointname'' if the
% number of points has been specified for this question, and otherwise
% it inserts nothing at all.  (The conditional @placepoints is used to
% determine if there were points specified for this question.)
% The argument to the \qformat command *must* contain some
% stretch, i.e., at least one \hfil or \dotfill or ...
%
% The command \noqformat cancels the effect of \qformat and returns us
% to the default situation.
%
% The commands \bonusqformat and \nobonusformat are analogous.
\newif\if@qformat
\@qformatfalse
\newif\if@bonusqformat
\@bonusqformatfalse

\def\qformat#1{%
  \global\@qformattrue
  \gdef\@questionformat{#1}%
}% qformat
\def\bonusqformat#1{%
  \global\@bonusqformattrue
  \gdef\@bonusquestionformat{#1}%
}% bonusqformat

\newcommand\noqformat{%
  \global\@qformatfalse
}% noqformat
\newcommand\nobonusqformat{%
  \global\@bonusqformatfalse
}% nobonusqformat


% \thepoints is for use in either a \qformat command
% or a \pointformat command (or a \bonusqformat command).
% It needs to have the
% \if@placepoints so that if it's used in a \qformat command
% it won't print anything if there are no points:
\newcommand\thepoints{%
  \if@placepoints
    \if@bonus
      \@points \@bonuspointname
    \else
      \@points \@pointname
    \fi
  \fi
}% thepoints

% \themarginpoints is for use only in a \pointformat command,
% and so it doesn't need the \if@placepoints bit in \thepoints:
\newcommand\themarginpoints{%
  \if@bonus
    \@points \@marginbonuspointname
  \else
    \@points \@marginpointname
  \fi
}% themarginpoints

% We define the \subpart and \subsubpart commands when we enter a
% questions environment (rather than waiting until we enter a subparts
% of subsubparts environment) so that we can signal an error if a
% \subpart or \subsubpart command appears outside of the corresponding
% environment.  (We don't do this for the \part command so that the user
% can use the standard sectioning \part command outside of a parts
% environment.)

% The counter ques@object will count the items in each question, where
% an item is defined as either the question itself, or a part, or a
% subpart, or a subsubpart, or a choice.  This will be used by
% \find@quesend to find the last page occupied by the last question
% begun on or before the current page:
\newcounter{ques@object}

% \first@questionobject will be used by the \question command.
% That is, it will be used only once, but we want to keep its
% definition here, near the definitions of \addquestionobject and
% \questionobject@pluspagecheck.
\newcommand{\first@questionobject}{%
  \setcounter{ques@object}{1}%
  % \PgInfo@write expands it's argument, so we don't need edef:
%   \edef\q@object@label{%
%     question\arabic{question}@object\arabic{ques@object}}%
%   \PgInfo@write{\q@object@label}%
  \PgInfo@write{question\arabic{question}@object\arabic{ques@object}}%
}% first@questionobject

% \addquestionobject will be used by each part, subpart, and
% subsubpart, and can also be used by the user to mark the end of a
% question that spills over onto the next page without any part,
% subpart, etc. starting on that page:
\newcommand{\addquestionobject}{%
  \addtocounter{ques@object}{1}%
  % \PgInfo@write expands it's argument, so we don't need edef:
%   \edef\q@object@label{%
%     question\arabic{question}@object\arabic{ques@object}}%
%   \PgInfo@write{\q@object@label}%
  \PgInfo@write{question\arabic{question}@object\arabic{ques@object}}%
}% addquestionobject

% \questionobject@pluspagecheck will be used by each choice, as well
% as by any \part, \subpart, or \subsubpart that's inside of a
% solution.  It uses the questionobject to check if we're the first
% one on the current page, since choices (and questions etc. inside of
% solution environments) don't have labels the way that questions,
% parts, subparts, and subsubparts do (those things use the label to
% check if they're the first thing on the page).
\newcommand{\questionobject@pluspagecheck}{%
  % We don't want to do any of this if we're both inside a solution
  % environment and not printing answers (because we want to avoid
  % incrementing ques@object):
  \if@insolution
    \ifprintanswers
      \doqobj@ppchk
    \fi
  \else
    \doqobj@ppchk
  \fi
}% questionobject@pluspagecheck
\newcommand{\doqobj@ppchk}{%
  \addtocounter{ques@object}{1}%
  % We need the edef because we check the page of \q@object@label:
  \edef\q@object@label{%
    question\arabic{question}@object\arabic{ques@object}}%
  \PgInfo@write{\q@object@label}%
  \set@counter@to@pageof{Curr@Page}{\q@object@label}%
  \expandafter\ifx\csname Contin@\theCurr@Page\endcsname\relax
    % We're the first \question, \part, \subpart, \subsubpart,
    % or choice on this page:
    \global\expandafter\edef\csname
    Contin@\theCurr@Page\endcsname{\arabic{question}}%
  \fi
}% doqobj@ppchk                               



% if@bonus will be true when we're doing a bonusquestion or bonuspart
% or etc., and it will also be used also to distinguish between
% \gradetable and \bonusgradetable (and between \pointtable and
% \bonuspointtable, etc.), and also to distinguish between
% \pointsinrange and \bonuspointsinrange:
\newif\if@bonus
\@bonusfalse

% The following are for advanced users who want to customize the list
% parameters (\topsep, \partopsep, \itemsep, \parsep, etc.) for the
% lists that these environments create.  They are all defined to be
% empty, but the user can change them using \renewcommand.
\newcommand\questionshook{}
\newcommand\partshook{}
\newcommand\subpartshook{}
\newcommand\subsubpartshook{}
\newcommand\choiceshook{}
\newcommand\checkboxeshook{}


\newenvironment{questions}{%
  % \@queslevel is used for two purposes:
  % (1) We check that every \question, \part, \subpart, and 
  % \subsubpart command appears inside the appropriate environment,
  % and generate an error if one appears in the wrong place.
  % (2) If a \qformat is being used and if \@queslevel tells us
  % that we're currently processing a question, then we set
  % \global \point@toks={} to avoid setting the points for a 
  % question other than via the qformat command.
  \def\@queslevel{question}%
  \def\titledquestion##1{%
    \@bonusfalse
    \def\thequestiontitle{##1}%
    \process@question
  }%
  \def\bonustitledquestion##1{%
    \@bonustrue
    \def\thequestiontitle{##1}%
    \process@question
  }%
  \def\question{%
    \@bonusfalse
    \def\thequestiontitle{\csname p@question\endcsname
                          \csname thequestion\endcsname}%
    \process@question
  }%
  \def\bonusquestion{%
    \@bonustrue
    \def\thequestiontitle{\csname p@question\endcsname
                          \csname thequestion\endcsname}%
    \process@question
  }%
  \def\process@question{%
    \if@coverpages
      \cover@question@error
    \fi
    \@checkqueslevel{question}%
    \addtocounter{numquestions}{1}%
    % Write the sum of points of the previous question (if any)
    % to the .aux file.  (At this point, the question counter
    % has not yet been incremented, so \value{question} is the
    % number of the question that was just completed.)
    \if@filesw
      \ifnum \value{question} > 0\relax
        % First do regular points:
        \immediate\write\@mainaux
          {\string\gdef\string\pointsofq@
            \romannumeral \csname c@question\endcsname
              {\prtaux@hlfcntr{pointsof@thisquestion}}}%
        % See if this has changed from the last run of LaTeX:
        \CheckIfChanged@hlf{pointsof@thisquestion}{pointsofq@\romannumeral
                            \csname c@question\endcsname}%
        % Now do bonus points:
        \immediate\write\@mainaux
          {\string\gdef\string\bonuspointsofq@
            \romannumeral \csname c@question\endcsname
              {\prtaux@hlfcntr{bonuspointsof@thisquestion}}}%
        % See if this has changed from the last run of LaTeX:
        \CheckIfChanged@hlf{bonuspointsof@thisquestion}%
                           {bonuspointsofq@\romannumeral
                             \csname c@question\endcsname}%
      \fi
    \fi
    \set@hlfcntr{pointsof@thisquestion}{0}%
    \set@hlfcntr{bonuspointsof@thisquestion}{0}%
    % If there was a question with points immediately preceding
    % this question (i.e., there were no parts in the previous
    % question), then @placepoints will still be true, and we need to
    % cancel it.  (We used to do this inside of the \thepoints macro,
    % but that allowed for an error if the user specified points for a
    % question but had a \qformat that didn't mention \thepoints.)  We
    % also set @placepoints to be false when entering a parts
    % environment. 
    \global \@placepointsfalse
    % point@toks will normally be empty at this point, but it might be
    % nonempty if there were points somewhere in the previous question
    % that never made it onto the page because we never entered
    % horizontal mode (perhaps because the user was weird and let the
    % text of a question (or part, etc.) consist entirely of an
    % enumerate environment, or description environment, or etc.).
    \global \point@toks={}%
    % Important: Don't leave any blank lines inside of
    % \pageinfo@commands!!  This token list will be dumped into
    % horizontal mode by \everypar, and so any blank lines will
    % cause paragraph breaks. 
    \pageinfo@commands={%
      \edef\@queslabel{question@\arabic{question}}%
      \PgInfo@write{\@queslabel}%
      \first@questionobject
      % In addition to the \PgInfo@write we use an actual \label
      % command.  We do this in order to make the question numbers in
      % the grade tables into \ref's, so that if the user says
      % \usepackage{hyperref}, those question numbers will be clickable.
      % 
      % A further purpose of these labels (which we actually do for all
      % questions, parts, subparts, and subsubparts) is that if a
      % question (or part, etc.) is, e.g., moved from one page to
      % another, LaTeX will notice this and warn the user that LaTeX
      % must be run one more time to be sure everything is correct.
      % 
      % We need to do this even though we've already included code to
      % check when point totals change because questions (and parts,
      % etc.) know what page they're on from reading the info written to
      % the .aux file on the previous run.  Thus, if a question (or
      % part, etc.) is moved to a different page, then the pointsonpage
      % totals won't notice until the *second* subsequent run of LaTeX,
      % and so there'll be no warning to the user on the *first* run.
      % Including these labels gives the user a warning on that first
      % run.
      %
      % Further futzing required: Since this is being put into the
      % token list \pageinfo@commands, the contents of which won't
      % actually be  executed until we enter horizontal mode (which
      % may well be in the first part of a parts environment), we need
      % to make sure that \@currentlabel is the number of the
      % question:
      \begingroup % to confine the change to \@currentlabel
%        \def\@currentlabel{\csname p@question\endcsname
%                           \csname thequestion\endcsname}%
        \def\@currentlabel{\thequestiontitle}%
        \label{\@queslabel}%
      \endgroup
      \set@counter@to@pageof{Curr@Page}{\@queslabel}%
      \expandafter\ifx\csname Contin@\theCurr@Page\endcsname\relax
        % We're the first \question, \part, \subpart, \subsubpart,
        % or choice on this page:
        \global\expandafter\edef
               \csname Contin@\theCurr@Page\endcsname{\relax}%
      \fi
      \the\pagepoint@commands
      \global \pageinfo@commands={}%
    }% pageinfo@commands
    \ifhmode
      % Remove any skips at the end of the previous paragraph
      % that might cause a blank line, and then end that paragraph:
      \unskip\unskip \par
    \fi
    \@doitem
  }% process@question
  \def\subpart{%
    \@bonusfalse
    \process@subpart
  }%
  \def\bonussubpart{%
    \@bonustrue
    \process@subpart
  }%
  \def\process@subpart{%
    \if@coverpages
      \cover@question@error
    \fi
    \@checkqueslevel{subpart}%
    \if@insolution
      % We don't count this subpart, so no addtocounter{numsubparts}.
      \temp@toks={%
        \questionobject@pluspagecheck
        \global \pageinfo@commands={}%
        % We omit the pagepoint@commands
      }% temp@toks
    \else
      \addtocounter{numsubparts}{1}%
      % Important: Don't leave any blank lines inside of
      % \pageinfo@commands!!  This token list will be dumped into
      % horizontal mode by \everypar, and so any blank lines will
      % cause paragraph breaks. 
      \temp@toks={%
        \edef\@subpartlabel{subpart@\arabic{question}%
          @\arabic{partno}@\arabic{subpart}}%
        \PgInfo@write{\@subpartlabel}%
        \addquestionobject
        % In addition to the \PgInfo@write we use an actual \label
        % command.  We do this in order to make the question numbers in
        % the grade tables into \ref's, so that if the user says
        % \usepackage{hyperref}, those question numbers will be clickable.
        % 
        % A further purpose of these labels (which we actually do for all
        % questions, parts, subparts, and subsubparts) is that if a
        % question (or part, etc.) is, e.g., moved from one page to
        % another, LaTeX will notice this and warn the user that LaTeX
        % must be run one more time to be sure everything is correct.
        % 
        % We need to do this even though we've already included code to
        % check when point totals change because questions (and parts,
        % etc.) know what page they're on from reading the info written to
        % the .aux file on the previous run.  Thus, if a question (or
        % part, etc.) is moved to a different page, then the pointsonpage
        % totals won't notice until the *second* subsequent run of LaTeX,
        % and so there'll be no warning to the user on the *first* run.
        % Including these labels gives the user a warning on that first
        % run.
        \label{\@subpartlabel}%
        \set@counter@to@pageof{Curr@Page}{\@subpartlabel}%
        \expandafter\ifx\csname Contin@\theCurr@Page\endcsname\relax
          % We're the first \question, \part, \subpart, \subsubpart,
          % or choice on this page:
          \global\expandafter\edef\csname
                         Contin@\theCurr@Page\endcsname{\arabic{question}}%
        \fi
        \the\pagepoint@commands
        \global \pageinfo@commands={}%
      }% temp@toks
    \fi
    \append@toklist \pageinfo@commands \temp@toks
    \ifhmode
      % Remove any skips at the end of the previous paragraph
      % that might cause a blank line, and then end that paragraph:
      \unskip\unskip \par
    \fi
    \@doitem
  }% process@subpart
  \def\subsubpart{%
    \@bonusfalse
    \process@subsubpart
  }%
  \def\bonussubsubpart{%
    \@bonustrue
    \process@subsubpart
  }%
  \def\process@subsubpart{%
    \if@coverpages
      \cover@question@error
    \fi
    \@checkqueslevel{subsubpart}%
    \if@insolution
      % We don't count this subsubpart, so no addtocounter{numsubsubparts}.
      \temp@toks={%
        \questionobject@pluspagecheck
        \global \pageinfo@commands={}%
        % We omit the pagepoint@commands
      }% temp@toks
    \else
      \addtocounter{numsubsubparts}{1}%
      % Important: Don't leave any blank lines inside of
      % \pageinfo@commands!!  This token list will be dumped into
      % horizontal mode by \everypar, and so any blank lines will
      % cause paragraph breaks. 
      \temp@toks={%
        \edef\@subsubpartlabel{subsubpart@\arabic{question}%
          @\arabic{partno}@\arabic{subpart}@\arabic{subsubpart}}%
        \PgInfo@write{\@subsubpartlabel}%
        \addquestionobject
        % In addition to the \PgInfo@write we use an actual \label
        % command.  We do this in order to make the question numbers in
        % the grade tables into \ref's, so that if the user says
        % \usepackage{hyperref}, those question numbers will be clickable.
        % 
        % A further purpose of these labels (which we actually do for all
        % questions, parts, subparts, and subsubparts) is that if a
        % question (or part, etc.) is, e.g., moved from one page to
        % another, LaTeX will notice this and warn the user that LaTeX
        % must be run one more time to be sure everything is correct.
        % 
        % We need to do this even though we've already included code to
        % check when point totals change because questions (and parts,
        % etc.) know what page they're on from reading the info written to
        % the .aux file on the previous run.  Thus, if a question (or
        % part, etc.) is moved to a different page, then the pointsonpage
        % totals won't notice until the *second* subsequent run of LaTeX,
        % and so there'll be no warning to the user on the *first* run.
        % Including these labels gives the user a warning on that first
        % run.
        \label{\@subsubpartlabel}%
        \set@counter@to@pageof{Curr@Page}{\@subsubpartlabel}%
        \expandafter\ifx\csname Contin@\theCurr@Page\endcsname\relax
          % We're the first \question, \part, \subpart, \subsubpart,
          % or choice on this page:
          \global\expandafter\edef\csname
                         Contin@\theCurr@Page\endcsname{\arabic{question}}%
        \fi
        \the\pagepoint@commands
        \global \pageinfo@commands={}%
      }% temp@toks
    \fi
    \append@toklist \pageinfo@commands \temp@toks
    \ifhmode
      % Remove any skips at the end of the previous paragraph
      % that might cause a blank line, and then end that paragraph:
      \unskip\unskip \par
    \fi
    \@doitem
  }% process@subsubpart
  \list{\question@number}%
    {\usecounter{question}%
    % We use the default definition of \makelabel
    % so as not to interfere with \qformat commands.
    % \def\makelabel##1{\hss\llap{##1}}%
    \settowidth{\leftmargin}{10.\hskip\labelsep}%
    \labelwidth\leftmargin\advance\labelwidth-\labelsep
    \partopsep=0pt
    \questionshook
    }%
  }% End of the first argument of \newenvironment{questions}
  {%
    \endlist
    % Write the number of points of the final question
    % to the .aux file:
    \if@filesw
      \ifnum \value{question} > 0\relax
        % First do the regular points:
        \immediate\write\@mainaux
          {\string\gdef\string\pointsofq@\romannumeral
                                \csname c@question\endcsname
            {\prtaux@hlfcntr{pointsof@thisquestion}}}%
        % See if this has changed from the last run of LaTeX:
        \CheckIfChanged@hlf{pointsof@thisquestion}%
                           {pointsofq@\romannumeral
                            \csname c@question\endcsname}%
        % Now do the bonus points:
        \immediate\write\@mainaux
          {\string\gdef\string\bonuspointsofq@\romannumeral
                                \csname c@question\endcsname
            {\prtaux@hlfcntr{bonuspointsof@thisquestion}}}%
        % See if this has changed from the last run of LaTeX:
        \CheckIfChanged@hlf{bonuspointsof@thisquestion}%
                           {bonuspointsofq@\romannumeral
                            \csname c@question\endcsname}%
      \fi
    \fi
  }% End of the second argument of \newenvironment{questions}
  
% \question@number is used as the label in the question list (instead
% of \questionlabel) so that if the user uses a \qformat command,
% we'll use the \@questionformat specified by the \qformat command:

\def\question@number{%
  \if@bonus
    \if@bonusqformat
      \makebox[\hsize][s]{\@bonusquestionformat}\hskip-\labelsep
    \else
      \questionlabel
    \fi
  \else
    \if@qformat
      \makebox[\hsize][s]{\@questionformat}\hskip-\labelsep
    \else
      \questionlabel
    \fi
  \fi
}
\newcommand\questionlabel{\thequestion.}
     
% We want the \part command to be defined *only* inside of a parts
% environment, so that the user can use the standard sectioning \part
% command inside of a questions environment (as long as it's outside of
% a parts environment).

\newenvironment{parts}{%
  \def\@queslevel{part}%
  % If the question numbers are being inserted via a \qformat,
  % and if a question is beginning with a parts environment, then
  % we need to enter horizonal mode to get the qformat printed
  % on the page, rather than saving up the question label (and
  % possible points) to be combined with the label of the first
  % part.  (\if@inlabel tells us if we are still waiting to enter
  % horizontal mode after seeing a \question command.)
  \if@bonus
    \if@bonusqformat
      \if@inlabel
        \leavevmode
        \@inlabelfalse
      \fi
      % The following is just in case the question had points,
      % in which case @placepoints will still be true.
      % (We used to do this inside of the \thepoints macro,
      % but that allowed for an error if the user specified points for
      % a question but had a \qformat that didn't mention \thepoints.)
      % We also set @placepoints to be false in the \question command,
      % in case one queation follows a previous one that had no parts.
      \global \@placepointsfalse
    \fi
  \else
    \if@qformat
      \if@inlabel
        \leavevmode
        \@inlabelfalse
      \fi
      % The following is just in case the question had points,
      % in which case @placepoints will still be true.
      % (We used to do this inside of the \thepoints macro,
      % but that allowed for an error if the user specified points for
      % a question but had a \qformat that didn't mention \thepoints.)
      % We also set @placepoints to be false in the \question command,
      % in case one queation follows a previous one that had no parts.
      \global \@placepointsfalse
    \fi
  \fi
  \def\part{%
    \@bonusfalse
    \process@part
  }%
  \def\bonuspart{%
    \@bonustrue
    \process@part
  }%
  \def\process@part{%
    \if@coverpages
      \cover@question@error
    \fi
    \@checkqueslevel{part}%
    \if@insolution
      % We don't count this part, so no addtocounter{numparts}.
      \temp@toks={%
        \questionobject@pluspagecheck
        \global \pageinfo@commands={}%
        % We omit the pagepoint@commands
      }% temp@toks
    \else
      \addtocounter{numparts}{1}%
      % Important: Don't leave any blank lines inside of
      % \pageinfo@commands!!  This token list will be dumped into
      % horizontal mode by \everypar, and so any blank lines will
      % cause paragraph breaks. 
      \temp@toks={%
        \edef\@partlabel{part@\arabic{question}@\arabic{partno}}%
        \PgInfo@write{\@partlabel}%
        \addquestionobject
        % In addition to the \PgInfo@write we use an actual \label
        % command.  We do this in order to make the question numbers in
        % the grade tables into \ref's, so that if the user says
        % \usepackage{hyperref}, those question numbers will be clickable.
        % 
        % A further purpose of these labels (which we actually do for all
        % questions, parts, subparts, and subsubparts) is that if a
        % question (or part, etc.) is, e.g., moved from one page to
        % another, LaTeX will notice this and warn the user that LaTeX
        % must be run one more time to be sure everything is correct.
        % 
        % We need to do this even though we've already included code to
        % check when point totals change because questions (and parts,
        % etc.) know what page they're on from reading the info written to
        % the .aux file on the previous run.  Thus, if a question (or
        % part, etc.) is moved to a different page, then the pointsonpage
        % totals won't notice until the *second* subsequent run of LaTeX,
        % and so there'll be no warning to the user on the *first* run.
        % Including these labels gives the user a warning on that first
        % run.
        \label{\@partlabel}%
        \set@counter@to@pageof{Curr@Page}{\@partlabel}%
        \expandafter\ifx\csname Contin@\theCurr@Page\endcsname\relax
          \global\expandafter\edef\csname
              Contin@\theCurr@Page\endcsname{\arabic{question}}%
        \fi
        \the\pagepoint@commands
        \global \pageinfo@commands={}%
      }% temp@toks
    \fi
    \append@toklist \pageinfo@commands \temp@toks
    \ifhmode
      % Remove any skips at the end of the previous paragraph
      % that might cause a blank line, and then end that paragraph:
      \unskip\unskip \par
    \fi
    \@doitem
  }% process@part
  \list{\partlabel}%
    {%
    \usecounter{partno}\def\makelabel##1{\hss\llap{##1}}%
    \settowidth{\leftmargin}{(m)\hskip\labelsep}%
    \labelwidth\leftmargin\advance\labelwidth-\labelsep
    \topsep=0pt
    \partopsep=0pt
    \partshook
    }%
  }% newenvironment{parts}
  {\endlist}
\newcommand\partlabel{(\thepartno)}
\def\thepartno{\alph{partno}}

\newenvironment{subparts}{%
  \def\@queslevel{subpart}%
  \list{\subpartlabel}%
    {%
    \usecounter{subpart}\def\makelabel##1{\hss\llap{##1}}%
    \settowidth{\leftmargin}{vii.\hskip\labelsep}%
    \labelwidth\leftmargin\advance\labelwidth-\labelsep
    \topsep=0pt
    \partopsep=0pt
    \subpartshook
    }%
  }%
  {\endlist}
\newcommand\subpartlabel{\thesubpart.}
\def\thesubpart{\roman{subpart}}

\newenvironment{subsubparts}{%
  \def\@queslevel{subsubpart}%
  \list{\subsubpartlabel}%
    {%
    \usecounter{subsubpart}\def\makelabel##1{\hss\llap{##1}}%
    \settowidth{\leftmargin}{($\psi$)\hskip\labelsep}%
    \labelwidth\leftmargin\advance\labelwidth-\labelsep
    \topsep=0pt
    \partopsep=0pt
    \subsubpartshook
    }%
  }%
  {\endlist}
\newcommand\subsubpartlabel{\thesubsubpart)}
\def\thesubsubpart{\greeknum{subsubpart}}




\pagepoint@commands={%
  \ifhlfcntr@pos{latest@points}%
    % We're putting a question (or part, etc.)
    % with points onto this page: 
    \ifnum \theCurr@Page > \thepageof@pagepoints\relax
      % These points go on a later page than
      % the points currently counted in @pagepoints:
      \ifnum \thepageof@pagepoints = 0\relax
        % Do nothing...
      \else
        \immediate\write\@mainaux
          {\string\gdef\string\pointsonpage@
           \romannumeral \csname c@pageof@pagepoints\endcsname
             {\prtaux@hlfcntr{@pagepoints}}}%
        % See if this has changed from the last run of LaTeX:
        \CheckIfChanged@hlf{@pagepoints}{pointsonpage@\romannumeral
                            \csname c@pageof@pagepoints\endcsname}%
      \fi
      % The following is a macro because \theCurr@Page and
      % \thepageof@pagepoints might differ by more than 1:
      \increment@pageof@pagepoints
      % The following label is so that we can make the page
      % numbers in a grade table indexed by page into \pageref's
      % so that \usepackage{hyperref} and pdflatex will
      % make them clickable:
      \label{firstpoints@onpage@\arabic{Curr@Page}}%
    \else
      % These points go on the same page as the points
      % currently counted in @pagepoints:
      \add@hlfcntrtohlfcntr{@pagepoints}{latest@points}%
      \set@hlfcntr{latest@points}{0}%
    \fi
  \fi
  \ifhlfcntr@pos{latest@bonuspoints}%
    % We're putting a question (or part, etc.)
    % with bonus points onto this page: 
    \ifnum \theCurr@Page > \thepageof@pagebonuspoints\relax
      % These bonus points go on a later page than
      % the points currently counted in @pagebonuspoints:
      \ifnum \thepageof@pagebonuspoints = 0\relax
        % Do nothing...
      \else
        \immediate\write\@mainaux
          {\string\gdef\string\bonuspointsonpage@
           \romannumeral \csname c@pageof@pagebonuspoints\endcsname
             {\prtaux@hlfcntr{@pagebonuspoints}}}%
        % See if this has changed from the last run of LaTeX:
        \CheckIfChanged@hlf{@pagebonuspoints}%
                           {bonuspointsonpage@\romannumeral
                            \csname c@pageof@pagebonuspoints\endcsname}%
      \fi
      % The following is a macro because \theCurr@Page and
      % \thepageof@pagebonuspoints might differ by more than 1:
      \increment@pageof@pagebonuspoints
      % The following label is so that we can make the page
      % numbers in a bonus grade table indexed by page into \pageref's
      % so that \usepackage{hyperref} and pdflatex will
      % make them clickable:
      \label{firstbonuspoints@onpage@\arabic{Curr@Page}}%
    \else
      % These points go on the same page as the points
      % currently counted in @pagebonuspoints:
      \add@hlfcntrtohlfcntr{@pagebonuspoints}{latest@bonuspoints}%
      \set@hlfcntr{latest@bonuspoints}{0}%
    \fi
  \fi
}% pagepoint@commands
\def\increment@pageof@pagepoints{%
  \addtocounter{pageof@pagepoints}{1}%
  \ifnum \theCurr@Page > \thepageof@pagepoints\relax
    \immediate\write\@mainaux
        {\string\gdef\string\pointsonpage@
         \romannumeral \csname c@pageof@pagepoints\endcsname{0}}%
    % See if this has changed from the last run of LaTeX:
    \@ifundefined{pointsonpage@\romannumeral
                        \csname c@pageof@pagepoints\endcsname}
      {\global\@pointschangedtrue}%
      {%
      % OK; it's defined.  See if it's changed:
      \begingroup
        \set@hlfcntr{tmp@hlfcntr}{\csname pointsonpage@\romannumeral 
                  \csname c@pageof@pagepoints\endcsname\endcsname}%
        \edef\othpt@check{\prtaux@hlfcntr{tmp@hlfcntr}}%
        \def\pt@check{0}%
        \ifx \pt@check \othpt@check
          % Do nothing
        \else
          \global\@pointschangedtrue
        \fi
      \endgroup
      }%
    \let\next@incr@pageof = \increment@pageof@pagepoints
  \else
    \copy@hlfcntr{@pagepoints}{latest@points}%
    \set@hlfcntr{latest@points}{0}%
    % \page@withpoints will be used to find the last
    % page that has points, which will be written to
    % the .aux file via \AtEndDocument:
    \global\edef\page@withpoints{\thepageof@pagepoints}%
    \let\next@incr@pageof = \relax
  \fi
  \next@incr@pageof
}% increment@pageof@pagepoints
\def\increment@pageof@pagebonuspoints{%
  \addtocounter{pageof@pagebonuspoints}{1}%
  \ifnum \theCurr@Page > \thepageof@pagebonuspoints\relax
    \immediate\write\@mainaux
        {\string\gdef\string\bonuspointsonpage@
         \romannumeral \csname c@pageof@pagebonuspoints\endcsname{0}}%
    % See if this has changed from the last run of LaTeX:
    \@ifundefined{bonuspointsonpage@\romannumeral
                        \csname c@pageof@pagebonuspoints\endcsname}
      {\global\@pointschangedtrue}%
      {%
      % OK; it's defined.  See if it's changed:
      \begingroup
        \set@hlfcntr{tmp@hlfcntr}{\csname bonuspointsonpage@\romannumeral 
                  \csname c@pageof@pagebonuspoints\endcsname\endcsname}%
        \edef\othpt@check{\prtaux@hlfcntr{tmp@hlfcntr}}%
        \def\pt@check{0}%
        \ifx \pt@check \othpt@check
          % Do nothing
        \else
          \global\@pointschangedtrue
        \fi
      \endgroup
      }%
    \let\next@incr@pageof = \increment@pageof@pagebonuspoints
  \else
    \copy@hlfcntr{@pagebonuspoints}{latest@bonuspoints}%
    \set@hlfcntr{latest@bonuspoints}{0}%
    % \page@withbonuspoints will be used to find the last
    % page that has bonus points, which will be written to
    % the .aux file via \AtEndDocument:
    \global\edef\page@withbonuspoints{\thepageof@pagebonuspoints}%
    \let\next@incr@pageof = \relax
  \fi
  \next@incr@pageof
}% increment@pageof@pagebonuspoints






\def\@checkqueslevel#1{%
  \begingroup
    \def\@temp{#1}%
    \ifx\@temp\@queslevel
      % Everything's fine; do nothing.
    \else
      \ClassError{exam}{%
        I found a #1 where I expected to find a
        \@queslevel\MessageBreak
      }{%
        Both #1 and \@queslevel \space can be used only inside the
        correct \MessageBreak \space \space
        environment and outside of any smaller environment
        \MessageBreak
      }%
    \fi
  \endgroup
}

\def\@doitem{\@ifnextchar[{\@readpoints}%
                          {\item@points@pageinfo}%
}

\def\@readpoints[#1]{%
% We use \def for \@points instead of \edef because we don't want
% \half (if present) to be expanded yet, so that the command \points
% can figure out how to deal with it:
  \def\@points{#1}%
  \global \@placepointstrue
  \if@addpoints
    \if@bonus
      \addto@hlfcntr{numbonuspoints}{\@points}%
      \addto@hlfcntr{bonuspointsof@thisquestion}{\@points}%
      % latest@bonuspoints is a holding area for bonus points to be
      % added to @pagepoints after we check whether they're
      % on the same page as the points currently counted
      % by @pagepoints:
      \addto@hlfcntr{latest@bonuspoints}{\@points}%
    \else
      \addto@hlfcntr{numpoints}{\@points}%
      \addto@hlfcntr{pointsof@thisquestion}{\@points}%
      % latest@points is a holding area for points to be
      % added to @pagepoints after we check whether they're
      % on the same page as the points currently counted
      % by @pagepoints:
      \addto@hlfcntr{latest@points}{\@points}%
    \fi
  \fi
  \item@points@pageinfo
}




% Bug fix, 5 April 2004: \item@points@pageinfo
% Appending \point@toks and \pageinfo@commands to \everypar:
% Instead of appending the contents of \point@toks and
% \pageinfo@commands to \everypar using \append@toklist,
% we instead want to append only the two tokens
%   \the\point@toks
% and the two tokens
%   \the\pageinfo@commands
% to \everypar.  We need to do this because if a questions environment
% immediately follows a \section command, then @nobreak will be true,
% and so the \if@nobreak inside of \everypar will *not* execute the
% \everypar={} that we had been counting on to keep the points from
% being inserted a second time in the second paragraph of a question.
% Since we've put the command \global \point@toks={} inside of 
% \point@toks and the command \pageinfo@commands={} inside of
% \pageinfo@commands, when the contents of \point@toks and of
% \pageinfo@commands are executed (when we enter horizontal mode and
% \everypar is dumped in), the contents of \point@toks and
% \pageinfo@commands will be made empty, and so if
% the second paragraph also gets \the\point@toks and
% \the\pageinfo@commands, it won't matter.

\def\item@points@pageinfo{%
  \item
  % Also: We need to do this here, *after* the \item command, rather
  % than inside the macro \@readpoints, because the \item command
  % puts the result of the \qformat command into an \hbox (with the
  % command ``\sbox\@tempboxa{\makelabel{#1}}%''), expanding the
  % argument of \qformat as it does so.  Thus, @placepoints will be
  % true when the argument of \qformat is expanded, and so if the
  % user put a \thepoints command inside that argument it will
  % correctly expand to the number of points.  (When @placepoints is
  % false, \thepoints expands to nothing at all).
  %
  % We also want to define \padded@point@block when @placepoints is
  % true even if qformat and bonusqformat are true just in case the
  % user, for some deranged reason, says \droppoints immediately
  % following a \question.
  %
  \if@placepoints
    % Since we want the user to be able to say \thepoints in the
    % argument to a \pointformat command, we need \@placepointstrue
    % when \point@block is expanded so that \thepoints will actually
    % print something.  (After setting up \point@toks, we do
    % \@placepointsfalse, but \point@block isn't actually expanded
    % until we enter horizontal mode.)  Thus, we define
    % \padded@point@block, and use that instead of \point@block.  We
    % put  \begingroup and \endgroup around this to confine the
    % effect of \@placepointstrue and also to confine the effect of
    % any declarations like, e.g., \bfseries that the user might put
    % in the argument of a \pointformat command.
    %
    % Note: We first tried using an \edef to expand \point@block right
    % here, while @placepoints is true, but that causes problems if
    % the user puts a \boldmath declaration in the argument of a
    % \pointformat command.  Apparently, expanding \boldmath (without 
    % executing anything) gives you bunches of undefined control
    % sequence errors.
    \if@bonus
      \def\padded@point@block{%
        \begingroup
          \@placepointstrue
          \bonuspoint@block
        \endgroup
      }%
    \else
      \def\padded@point@block{%
        \begingroup
          \@placepointstrue
          \point@block
        \endgroup
      }%
    \fi
    % \setup@point@toks puts commands into \point@toks to place
    % \padded@point@block at the correct spot.  It doesn't append
    % anything to \everypar (we do that in this macro, below). 
    %
    % If @qformat is true, and if we're currently doing a question (or
    % if @bonusqformat is true and we're doing a bonusquestion)
    % (rather than a part, subpart, or subsubpart), then we don't want
    % to set the points (if any), since the points of a question will
    % appear only if the user chooses to cause that by putting a
    % \thepoints in the argument of the \qformat command.
    \if@pointsdropped
      % Do nothing!
    \else
      \if@bonus
        \if@bonusqformat
          \ifx\ques@ref\@queslevel
            % Do nothing
          \else
            \setup@point@toks
          \fi
        \else
          \setup@point@toks
        \fi
      \else
        \if@qformat
          \ifx\ques@ref\@queslevel
            % Do nothing
          \else
            \setup@point@toks
          \fi
        \else
          \setup@point@toks
        \fi
      \fi
    \fi
    \global \@placepointsfalse
  \fi
  % We *don't* use \append@toklist; see the Bug fixnote above
  % (Bug fix, 5 April 2004).
  % We can append the tokens ``\the\point@toks'' whether or not we're
  % setting any points because if we're not setting them, \point@toks
  % will be empty.
  % Also: It's important to do this *after* the \item command above,
  % since the \item command discards the previous contents of
  % \everypar.
  % Version 2.218$\beta$, 2007/10/31 changes:
  % Instead of appending
  %   \the \pageinfo@commands \the \point@toks
  % to \everypar, we insert them into the box \@labels.  This corrects
  % the problem that arose when a question (or part, etc.) begins with
  % a list environment (including verbatim, flushleft, center,
  % flushright, and possibly others that are implemented as trivlist
  % environments).  The \item command in those environments throws
  % away the previous contents of \everypar, and so the tokens \the
  % \pageinfo@commands \the \point@toks didn't get inserted where we
  % expected.  List environments *do* preserve the contents of the box
  % \@labels, though.
  \global\setbox\@labels\hbox{\unhbox\@labels
    \the \pageinfo@commands
    \the \point@toks}%
  %   \edef\append@everypar{\noexpand\everypar={\the\everypar
  %                         \noexpand\the \noexpand\pageinfo@commands
  %                         \noexpand\the \noexpand\point@toks}}%
  %   \append@everypar
}







% Initialize \@points:
% (The only reason I think this is necessary is in case the user uses
% a \qformat command, puts \themarginpoints into the format (which is
% *not* the intended use of \themarginpoints), and then doesn't have
% any points for the first question.)
\def\@points{0}


\def\setup@point@toks{%
% We set the token list \point@toks equal to the sequence of commands
% needed to put \padded@point@block at the correct location, followed by the
% tokens ``\global \point@toks={}''.  The \question, \part, \subpart,
% or \subsubpart command then adds the two tokens ``\the\point@toks''
% to \everypar.
%
% Note: It is not the *contents* of \point@toks that is added to
% \everypar; just the two tokens ``\the\point@toks''.  This difference
% is the bug fix of 2 April 2004, described above (the bug was that in
% earlier versions, we used to append the contents).
%
% The result of this is that whenever we finally enter horizontal mode
% (because we finally encountered the text of a question, part,
% subpart, or subsubpart) the contents of \point@toks will be dumped
% into horizontal mode and executed, and so the points will be placed
% and the token list \point@toks will be set to empty.  Thus, in the
% occasional circumstances in which \everypar is *not* set to empty
% after being added to the first paragraph (which occurs when a
% questions environment immediately follows a \section command), and
% so \everypar will still contain ``\the\point@toks'' when it
% encounters a possible second paragraph of the first question, the
% tokens ``\the\point@toks'' will insert an *empty* token list, which
% will do no harm.
%
  \if@pointsinleftmargin
    \point@toks={%
          \llap{\padded@point@block 
                \hskip\@totalleftmargin
                \hskip\marginpointssep
          }%
          \global \point@toks={}%
    }%
  \else
    \if@pointsinrightmargin
      \point@toks={%
            \rlap{\hskip-\@totalleftmargin
                  \hskip\textwidth
                  \hskip\@rightmargin
                  \hskip-\rightpointsmargin
                  \llap{\padded@point@block}%
            }%
            \global \point@toks={}%
      }%
    \else
      % The points just go after the question number:
      \point@toks={%
            \padded@point@block
            \enspace
            \global \point@toks={}%
      }%
    \fi
  \fi
}% setup@point@toks



\def\droppoints{%
  \leavevmode\unskip\nobreak\hfill
  \rlap{\hskip\rightmargin  % Defined by the list environment
        \hskip\@rightmargin % Defined by exam.cls
        \hskip-\rightpointsmargin
        \llap{\padded@point@block}%
  }% rlap
  \par
}


\def\droptotalpoints{%
  \leavevmode\unskip\nobreak\hfill
  \rlap{\hskip\rightmargin  % Defined by the list environment
        \hskip\@rightmargin % Defined by exam.cls
        \hskip-\rightpointsmargin
        \llap{\total@block}%
  }% rlap
  \par
}% droptotalpoints
\def\droptotalbonuspoints{%
  \leavevmode\unskip\nobreak\hfill
  \rlap{\hskip\rightmargin  % Defined by the list environment
        \hskip\@rightmargin % Defined by exam.cls
        \hskip-\rightpointsmargin
        \llap{\bonustotal@block}%
  }% rlap
  \par
}% droptotalbonuspoints

% The following is the default definition;
% it can be changed by a \totalformat command.
\def\total@block{%
  Total for Question \thequestion: \totalpoints\@marginpointname
}% total@block
\def\bonustotal@block{%
  Total for Question \thequestion: \totalbonuspoints\@marginbonuspointname
}% bonustotal@block

\def\totalformat#1{%
  \gdef\total@block{\begingroup #1\endgroup}%
}% totalformat
\def\bonustotalformat#1{%
  \gdef\bonustotal@block{\begingroup #1\endgroup}%
}% bonustotalformat
% The following is for use in the argument to a \totalformat command:
\def\totalpoints{\pointsofquestion{\arabic{question}}}
\def\totalbonuspoints{\bonuspointsofquestion{\arabic{question}}}



% @placepoints is set true when we encounter a question (or part, etc.)
% that has points.  It is set to false (1) when we set \point@toks equal
% to the sequence of commands required to put the properly formatted
% points onto the page (this happens only if @qformat is false or if
% @qformat is true but we're not doing a question), or (2) by a
% \question command or entering a parts environment (since if we're
% doing a question and @qformat is true, we need to leave @placepoints
% true so that the \thepoints command can tell if it should expand to
% points or to nothing, and encountering a \question command or parts
% environment tells us that we no longer have to deal with a possible
% \thepoints, since we won't be expanding a qformat).
\newif\if@placepoints
\@placepointsfalse


% \marginpointssep will be used if the user says
% \pointsinleftmargin.  It will be the distance from whatever encloses
% the points (parentheses, brackets, or a box) to the left margin:
\newlength\marginpointssep
\setlength{\marginpointssep}{5pt}

% \rightpointsmargin will be used if the user says \pointsinrightmargin.
% It will be the distance from whatever encloses the point (parentheses,
% brackets, or a box) to the right edge of the paper:
\newlength\rightpointsmargin
\setlength{\rightpointsmargin}{1cm}


\newif\if@pointsdropped
\newif\if@pointsinleftmargin
\newif\if@pointsinrightmargin
\def\pointsinleftmargin{\global\@pointsinleftmargintrue 
                    \global\@pointsinrightmarginfalse
                    \global\@pointsdroppedfalse
                    \gdef\pt@name{\@marginpointname}%
                    \gdef\bnspt@name{\@marginbonuspointname}}
\def\pointsinrightmargin{\global\@pointsinrightmargintrue
                         \global\@pointsinleftmarginfalse
                         \global\@pointsdroppedfalse
                         \gdef\pt@name{\@marginpointname}%
                         \gdef\bnspt@name{\@marginbonuspointname}}
\def\nopointsinmargin{\global\@pointsinleftmarginfalse
                      \global\@pointsinrightmarginfalse
                      \global\@pointsdroppedfalse
                      \gdef\pt@name{\@pointname}%
                      \gdef\bnspt@name{\@bonuspointname}}
\def\pointsdroppedatright{\global\@pointsdroppedtrue
                          \global\@pointsinleftmarginfalse
                          \global\@pointsinrightmarginfalse
                          \gdef\pt@name{\@marginpointname}%
                          \gdef\bnspt@name{\@marginbonuspointname}}
\let\pointsinmargin\pointsinleftmargin
\let\nopointsinrightmargin\nopointsinmargin
\let\nopointsinleftmargin\nopointsinmargin

\nopointsinmargin


% Will the points be displayed inside parentheses (the default), or
% will they be boxed or bracketed, or customized using pointformat:
\def\boxedpoints{%
  \gdef\point@block{\fbox{\@points\pt@name}}%
  \gdef\bonuspoint@block{\fbox{\@points\bnspt@name}}%
}
\def\noboxedpoints{%
  \gdef\point@block{(\@points\pt@name)}%
  \gdef\bonuspoint@block{(\@points\bnspt@name)}%
}
\def\bracketedpoints{%
  \gdef\point@block{[\@points\pt@name]}%
  \gdef\bonuspoint@block{[\@points\bnspt@name]}%
}
\let\nobracketedpoints=\noboxedpoints

\def\pointformat#1{%
  % We don't have to worry about the user putting things
  % like \bfseries, etc. into \point@block, because
  % \padded@point@block encloses \point@block in a group,
  % which confines the effects of anything here:
  \gdef\point@block{#1}%
}

\def\bonuspointformat#1{%
  % We don't have to worry about the user putting things
  % like \bfseries, etc. into \point@block, because
  % \padded@point@block encloses \point@block in a group,
  % which confines the effects of anything here:
  \gdef\bonuspoint@block{#1}%
}
\def\thebonuspoints{\@points \@bonuspointname}

%Initialize:
\noboxedpoints




\def\pointname#1{\gdef\@pointname{#1}}
\def\bonuspointname#1{\gdef\@bonuspointname{#1}}
% Initialize to leave a space, and then the word `points':
%%\pointname{ points}
% The following improvement was contributed by 
% Mate Wierdl <mw@wierdlmpc.msci.memphis.edu> 
% If the number of points is ``1'', then the default value of
% \pointname will print `` point'' instead of `` points'' (and this
% version of the command doesn't generate an error message if the
% points entry is something other than a number):
% Note the space before the \points in the following; it's
% intentional!)
\pointname{ \points}
\bonuspointname{ \bonuspoints}
\newcommand\point@sing{point}
\newcommand\point@plur{points}
\newcommand\pointpoints[2]{%
  \renewcommand\point@sing{#1}%
  \renewcommand\point@plur{#2}%
}
%\newcommand\bonuspoint@sing{bonus point}
%\newcommand\bonuspoint@plur{bonus points}
\newcommand\bonuspoint@sing{point (bonus)}
\newcommand\bonuspoint@plur{points (bonus)}
\newcommand\bonuspointpoints[2]{%
  \renewcommand\bonuspoint@sing{#1}%
  \renewcommand\bonuspoint@plur{#2}%
}



% The command \points:
% We use \ifthenelse and \equal so that if the user types something
% other than a legit point value, there still won't be any error
% messages.  We rig it so that if the point value looks like it's
% intended to be One or one or ONE, or some strange way of attempting
% one half, then it will expand to the singular value.  Alas, this is
% only useful for English, but I'm hoping few or no users will try doing
% this anyway.

% 0 points, one half point, 1 point, 1 and a half points, etc.:
\newcommand\points{%
  \begingroup
    \let\half=\relax
    \edef\pt@string{\@points}%
    \ifthenelse{\equal{\pt@string}{1} \or \equal{\pt@string}{\half}}
          {\point@sing}{\point@plur}%
  \endgroup
}% \points
\newcommand\bonuspoints{%
  \begingroup
    \let\half=\relax
    \edef\pt@string{\@points}%
    \ifthenelse{\equal{\pt@string}{1} \or \equal{\pt@string}{\half}}
          {\bonuspoint@sing}{\bonuspoint@plur}%
  \endgroup
}% \bonuspoints
%\newcommand\points{%
%  \begingroup
%    \let\half=\relax
%    \edef\pt@string{\@points}%
%    \ifthenelse{\equal{\pt@string}{1} \or \equal{\pt@string}{\half} \or
%      \equal{\pt@string}{0\half} \or \equal{\pt@string}{0 \half}
%      \equal{\pt@string}{one} \or \equal{\pt@string}{One} \or
%      \equal{\pt@string}{ONE}}
%        {\point@sing}{\point@plur}%
%  \endgroup
%}

%\newcommand\points{\ifthenelse{\equal{\@points}{1}}{\point@sing}{\point@plur}}
% If we used the following line instead, then you'd get an error
% message if the point value contained something other than a valid
% integer:
%\pointname{ \ifthenelse{\@points = 1}{point}{points}}

% We used to define a command named \marks that works like \points,
% except that it expands to either ``mark'' or ``marks'', but that
% conflicts with some package or other.  Thus, we'll implement
% \marksnotpoints using the \pointpoints command instead:
\newcommand\marksnotpoints{%
  \pointpoints{mark}{marks}%
  \bonuspointpoints{mark (bonus)}{marks (bonus)}%
}% \marksnotpoints


% \@marginpointname is used in place of \@pointname if any of
% \@pointsinmargin, \@pointsinrightmargin, and \@pointsdropped are
% true:
\def\marginpointname#1{\gdef\@marginpointname{#1}}
\marginpointname{}
\def\marginbonuspointname#1{\gdef\@marginbonuspointname{#1}}
\marginbonuspointname{ (bonus)}


%--------------------------------------------------------------------
%         choices (for multiple choice) and checkboxes


\renewcommand\thechoice{\Alph{choice}}
\newcommand\choicelabel{\thechoice.}

% We will have \@correctchoicetrue when we're printing solutions
% and we're printing the correct choice of a choices or
% oneparchoices environment.
% We'll say \begingroup before saying \@correctchoicetrue
% and we'll say \endgroup at either the next \choice or \correctchoice
% or the end of the choices or oneparchoices environment.
% Thus, we'll never again need to say \@correctchoicefalse
\newif\if@correctchoice
\@correctchoicefalse

\newcommand\CorrectChoiceEmphasis[1]{%
  \def\CorrectChoice@Emphasis{#1}%
}
\CorrectChoiceEmphasis{\bfseries}
\let\correctchoiceemphasis\CorrectChoiceEmphasis

% Note: \do@choice@pageinfo is used in both the choices and
% the checkboxes environments.
\newtoks\choice@toks
\def\do@choice@pageinfo{%
  \choice@toks={%
    \questionobject@pluspagecheck
    \choice@toks={}%
  }%
  % Version 2.217-beta changes:
  % Instead of appending stuff to \everypar, we insert
  % \the \pageinfo@commands and \the \point@toks 
  % into the box \@labels:
  \global\setbox\@labels\hbox{\unhbox\@labels
    \the \choice@toks}%
  %   \edef\append@everypar{\noexpand\everypar={\the\everypar
  %                         \noexpand\the \noexpand\choice@toks}}%
  %   \append@everypar
}% do@choice@pageinfo



% Added 22 April 2004: Increased the \leftmargin by 2.5em,
% so the choices will be visibly indented.
\newenvironment{choices}%
  {\list{\choicelabel}%
     {\usecounter{choice}\def\makelabel##1{\hss\llap{##1}}%
       \settowidth{\leftmargin}{W.\hskip\labelsep\hskip 2.5em}%
       \def\choice{%
         \if@correctchoice
           \endgroup
         \fi
         \item
         \do@choice@pageinfo
       } % choice
       \def\CorrectChoice{%
         \if@correctchoice
           \endgroup
         \fi
         \ifprintanswers
           % We can't say \choice here, because that would
           % insert an \endgroup:
           \begingroup \@correctchoicetrue
           \CorrectChoice@Emphasis
         \fi
         \item
         \do@choice@pageinfo
       } % CorrectChoice
       \let\correctchoice\CorrectChoice
       \labelwidth\leftmargin\advance\labelwidth-\labelsep
       \topsep=0pt
       \partopsep=0pt
       \choiceshook
     }%
  }%
  {\if@correctchoice \endgroup \fi \endlist}

\newenvironment{oneparchoices}%
  {%
    \setcounter{choice}{0}%
    \def\choice{%
      \if@correctchoice \endgroup \fi
      \refstepcounter{choice}%
      \ifnum\value{choice}>1\relax
        \penalty -50\hskip 1em plus 1em\relax
      \fi
      \choicelabel
      % No need to put the following into a token string; we just put
      % the choicelabel onto the page, so we're at the spot whose page
      % number we want to record:
      \questionobject@pluspagecheck
      \nobreak\enskip
    }% choice
    \def\CorrectChoice{%
      \if@correctchoice \endgroup \fi
      \refstepcounter{choice}%
      \ifprintanswers
        \begingroup \@correctchoicetrue 
        \CorrectChoice@Emphasis
      \fi
      \ifnum\value{choice}>1\relax
        \penalty -50\hskip 1em plus 1em\relax
      \fi
      \choicelabel
      % No need to put the following into a token string; we just put
      % the choicelabel onto the page, so we're at the spot whose page
      % number we want to record:
      \questionobject@pluspagecheck
      \nobreak\enskip
    }% CorrectChoice
    \let\correctchoice\CorrectChoice
    \let\par\@empty
    % If we're continuing the paragraph containing the question,
    % then leave a bit of space before the first choice:
    \ifvmode\else\enskip\fi
    \ignorespaces
  }%
  {\if@correctchoice \endgroup \fi}

\newcommand{\checkboxchar}[1]{\def\checkbox@char{#1}}
\newcommand{\checkedchar}[1]{\def\checked@char{#1}}
\checkboxchar{$\bigcirc$}
\checkedchar{$\surd$}

\newenvironment{checkboxes}%
  {\list{\checkbox@char}%
     {%
       \settowidth{\leftmargin}{W.\hskip\labelsep\hskip 2.5em}%
       \def\choice{%
         \if@correctchoice
           \endgroup
         \fi
         \item
         \do@choice@pageinfo
       } % choice
       \def\CorrectChoice{%
         \if@correctchoice
           \endgroup
         \fi
         \ifprintanswers
           % We can't say \choice here, because that would
           % insert an \endgroup:
           \begingroup \@correctchoicetrue
           \CorrectChoice@Emphasis
           \item[\checked@char]
         \else
           \item
         \fi
         \do@choice@pageinfo
       } % CorrectChoice
       \let\correctchoice\CorrectChoice
       \labelwidth\leftmargin\advance\labelwidth-\labelsep
       \topsep=0pt
       \partopsep=0pt
       \checkboxeshook
     }%
  }%
  {\if@correctchoice \endgroup \fi \endlist}

\newenvironment{oneparcheckboxes}%
  {%
    % Although we're not printing numbers for the choices, we use the
    % choice counter to keep track of whether a choice is the first
    % one (in which case we don't leave any additional space) or a
    % later one (in which case we do leave additional space):    
    \setcounter{choice}{0}%
    \def\choice{%
      \if@correctchoice \endgroup \fi
      \stepcounter{choice}%
      \ifnum\value{choice}>1\relax
        \penalty -50\hskip 1em plus 1em\relax
      \fi
      \checkbox@char
      % No need to put the following into a token string; we just put
      % \checkbox@char onto the page, so we're at the spot whose page
      % number we want to record:
      \questionobject@pluspagecheck
      \nobreak\enskip
    }% choice
    \def\CorrectChoice{%
      \if@correctchoice \endgroup \fi
      \stepcounter{choice}%
      \ifprintanswers
        \begingroup \@correctchoicetrue 
        \CorrectChoice@Emphasis
      \fi
      \ifnum\value{choice}>1\relax
        \penalty -50\hskip 1em plus 1em\relax
      \fi
      \ifprintanswers
        \checked@char
      \else
        \checkbox@char
      \fi
      % No need to put the following into a token string; we just put
      % the choicelabel onto the page, so we're at the spot whose page
      % number we want to record:
      \questionobject@pluspagecheck
      \nobreak\enskip
    }% CorrectChoice
    \let\correctchoice\CorrectChoice
    \let\par\@empty
    % If we're continuing the paragraph containing the question,
    % then leave a bit of space before the first choice:
    \ifvmode\else\enskip\fi
    \ignorespaces
  }%
  {\if@correctchoice \endgroup \fi}


%--------------------------------------------------------------------
%             Answer Lines (for short answer questions)

% Note: \ques@ref is also used in \item@points@pageinfo

\def\ques@ref{question}
\def\part@ref{part}
\def\subpart@ref{subpart}
\def\subsubpart@ref{subsubpart}

% Note: \answerclearance is also used by \fillin

\newlength\answerlinelength
\newlength\answerskip
\newlength\answerclearance
\setlength\answerlinelength{1in}
\setlength\answerskip{2ex}
\setlength\answerclearance{0.2ex}

\newcommand\answerline[1][{}]{%
  % One optional argument, the default value of which is empty.
  \ifx\@queslevel\ques@ref
    \let\ans@l=\questionlabel
  \else
    \ifx\@queslevel\part@ref
      \let\ans@l=\partlabel
    \else
      \ifx\@queslevel\subpart@ref
        \let\ans@l=\subpartlabel
      \else
        \ifx\@queslevel\subsubpart@ref
          \let\ans@l=\subsubpartlabel
        \else
          % Oops; no question level defined.
          % We must be outide of the questions environment.
          % Just leave out the label, I guess:
          \def\ans@l{}%
        \fi
      \fi
    \fi
  \fi
  \par \nobreak \vskip \answerskip
  \hfill 
  \ifprintanswers
    \ans@l~\hbox to 0pt{\hbox to \answerlinelength{\hrulefill}\hss}%
    \raise \answerclearance\hbox to \answerlinelength{%
      \CorrectChoice@Emphasis \hfil #1\hss}%
  \else
    \ans@l~\hbox to \answerlinelength{\hrulefill}%
  \fi
  \par
}% answerline

%--------------------------------------------------------------------
%               \fillin, for fill-in-the-blank questions

\newlength\fillinlinelength
\setlength\fillinlinelength{1in}

% \fillin can take two optional arguments.
%
% The first optional argument is the answer to be printed above the line
% when \printanswers is in effect; the default value is empty.  That
% line is printed a distance of \answerclearance below the baseline.
%
% The second optional argument is the length of the line that we print;
% the default value is \fillinlinelength.  The value of
% \fillinlinelength is set with the command
%
%   \setlength\fillinlinelength{1in}
%
% and can be changed by giving a new \setlength command.
%
% When answers are being printed, the optional argument is printed
% subject to the declarations in the argument of the last
% \CorrectChoiceEmphasis command.  It is centered on the line unless it
% is too long, in which case it extends to the right of the line.
%
% \fillin eats (and ignores) space characters appearing before the
% first optional argument.  It also eats (and ignores) space
% characters appearing after the first optional argument and before
% the second optional argument.  However, if exactly one optional
% argument appears, and if there are one or more space characters
% following that one optional argument, then those spaces are replaced
% by a single space character, but not eaten.

\newcommand\fillin[1][{}]{%
   \def\fillin@ans{#1}%
   \fillin@relay
}% fillin

\newcommand\fillin@relay{%
  % We use \exam@ifnextchar, a variation on \@ifnextchar.
  % If \exam@ifnextchar encounters one or more space characters
  % followed by a [, then those spaces are ignored (just as they would
  % be by \@ifnextchar).  However, if one or more space characters are
  % followed by a non-space character other than [, then
  % \exam@ifnextchar inserts a space following the
  % {\@fillin@relay[\fillinlinelength]} that is the third argument to
  % \exam@ifnextchar.
  \exam@ifnextchar[{\@fillin@relay}
                   {\@fillin@relay[\fillinlinelength]}%
}% fillin@relay

\def\@fillin@relay[#1]{%
  % The first argument is in \fillin@ans, the second is #1.
  \leavevmode
  \ifprintanswers
    \rlap{\raise -\answerclearance \hbox to #1{\hrulefill}}%
    \begingroup
      \setbox0 \hbox{\color@begingroup
             \CorrectChoice@Emphasis \fillin@ans \color@endgroup}%
      \ifdim\wd0 > #1\relax
        \hbox{\CorrectChoice@Emphasis \fillin@ans}%
      \else
        \hbox to #1{\CorrectChoice@Emphasis \hfil \fillin@ans \hfil}%
      \fi
    \endgroup
  \else
    \raise -\answerclearance \hbox to #1{\hrulefill}%
  \fi
}% @fillin@relay

% \exam@ifnextchar is used by \fillin.
% \exam@ifnextchar is a variation of \@ifnextchar that does not always
% ignore space tokens.  If \exam@ifnextchar encounters one or more
% space tokens, it makes note of that (with the command
% \@tempswatrue).  If the first non-space character encountered
% matches argument #1, then any spaces that had been encountered are
% ignored.  However, if one or more spaces are encountered and the
% first non-space character found does not match argument #1, then
% \exam@ifnextchar produces argument #3 followed by a space character.
% (This differs from the behavior of \new@ifnextchar in amsgen.sty,
% which does lookahead for any character, including a space.)  This
% code (as well as the idea for it) is due to Dan Luecking and Ulrich
% Diez.
\long\def\exam@ifnextchar#1#2#3{%
  \let\reserved@d=#1%
  \def\reserved@a{#2}%
  \def\reserved@b{#3}%
  % The following says we haven't yet seen any spaces:
  \@tempswafalse
  \futurelet\@let@token\exam@ifnch
}% exam@ifnextchar

\def\exam@ifnch{%
  \ifx\@let@token\@sptoken
    % Signal that we've found a space:
    \@tempswatrue
    \let\reserved@c\exam@xifnch % this gobbles the space
  \else
    \ifx\@let@token\reserved@d
      \let\reserved@c\reserved@a
    \else
      \if@tempswa
        \def\reserved@c{\expandafter\reserved@b\space}%
      \else 
        \let\reserved@c\reserved@b
      \fi
    \fi
  \fi
  \reserved@c
}% exam@ifnch

% The following defines \exam@xifnch so that it will eat a space
% following it and then call \exam@ifnch:
{% keep redefinition of \: local
  \def\:{\exam@xifnch}
  \expandafter\gdef\: {\futurelet\@let@token\exam@ifnch}
}

%--------------------------------------------------------------------
%                            \fillwithlines


% \fillwithlines takes one argument, which is either a length or \fill
% or \stretch{number}, and it fills that much vertical space with
% horizontal lines that run the length of the current line.  That is,
% they extend from the current left margin (which depends on whether
% we're in a question, part, subpart, or subsubpart) to the right
% margin.
%
% The distance between the lines is \linefillheight, whose default value
% is set with the command
%
% \setlength\linefillheight{.25in}
%
% This value can be changed by giving a new \setlength command.
%
% The thickness of the lines is \linefillthickness, whose default value
% is set with the command
%
% \setlength\linefillthickness{.1pt}
%
% This value can be changed by giving a new \setlength command.


\newlength\linefillheight
\newlength\linefillthickness
\setlength\linefillheight{.25in}
\setlength\linefillthickness{0.1pt}

\newcommand\linefill{\leavevmode
    \leaders\hrule height \linefillthickness \hfill\kern\z@}


\def\fillwithlines#1{%
  \begingroup
  \ifhmode
    \par
  \fi
  \hrule height \z@
  \nobreak
  \setbox0=\hbox to \hsize{\hskip \@totalleftmargin
          \vrule height \linefillheight depth \z@ width \z@
          \linefill}%
  % We use \cleaders (rather than \leaders) so that a given
  % vertical space will always produce the same number of lines
  % no matter where on the page it happens to start:
  \cleaders \copy0 \vskip #1 \hbox{}%
  \endgroup
}

%--------------------------------------------------------------------
%                         \fillwithdottedlines


% \fillwithdottedlines is similar to \fillwithlines, except that it
% fills the space with dotted lines (created by \dotfill) rather than
% with solid lines.

% \fillwithdottedlines takes one argument, which is either a length or
% \fill or \stretch{number}, and it fills that much vertical space
% with dotted lines that run the length of the current line.  That is,
% they extend from the current left margin (which depends on whether
% we're in a question, part, subpart, or subsubpart) to the right
% margin.
%
% The distance between the lines is \dottedlinefillheight, whose
% default value is set with the command
%
% \setlength\dottedlinefillheight{.25in}
%
% This value can be changed by giving a new \setlength command.

\newlength\dottedlinefillheight
\setlength\dottedlinefillheight{.25in}

\def\fillwithdottedlines#1{%
  \begingroup
  \ifhmode
    \par
  \fi
  \hrule height \z@
  \nobreak
  \setbox0=\hbox to \hsize{\hskip \@totalleftmargin
          \vrule height \dottedlinefillheight depth \z@ width \z@
          \dotfill}%
  % We use \cleaders (rather than \leaders) so that a given
  % vertical space will always produce the same number of lines
  % no matter where on the page it happens to start:
  \cleaders \copy0 \vskip #1 \hbox{}%
  \endgroup
}% fillwithdottedlines

%--------------------------------------------------------------------
%                            \fillwithgrid


% \fillwithgrid is similar to \fillwithlines, except that it
% fills the space with a grid.

% \fillwithgrid takes one argument, which is either a length or \fill
% or \stretch{number}, and it fills that much vertical space with a
% grid that runs the length of the current line.  That is, it extends
% from the current left margin (which depends on whether we're in a
% question, part, subpart, or subsubpart) to the right margin.
%
% The default grid size and grid line thickness were set by the
% commands
%
% \setlength{\gridsize}{5mm}
% \setlength{\gridlinewidth}{0.1pt}
%
% You can change either or both of those by giving new \setlength
% commands.  The period of the grid is \gridsize (both horizontally
% and vertically).  That is, the horizontal distance from the left
% edge of one vertical line to the left edge of the next vertical line
% is \gridsize, as is the vertical distance from the top edge of one
% horizontal line to the top edge of the next horizontal line.  Thus,
% each square has outer side length equal to \gridsize+\gridlinewidth.

% By default, the created grids are in black.  However, if you give the
% commands
%
% \usepackage{color}
% \colorgrids
%
% then the grids will be in color, by default a light gray.  That
% default color was defined by the command
%
% \definecolor{GridColor}{gray}{0.8}
%
% You can change the color by redefining the color GridColor by giving
% a new \definecolor command.

\newif\if@colorgrids
\newcommand\colorgrids{%
  \@ifundefined{definecolor}
  {%
    \ClassError{exam}{%
      You must load the color package with the command\MessageBreak
      \space\space\protect\usepackage{color}\MessageBreak
      in order to use the command \protect\colorgrids
      }{%
      This command makes use of the package color.sty,\MessageBreak
      and so you have to load color.sty before your\MessageBreak
      \protect\begin{document} command.\MessageBreak
      }%
  }%
  {%
    \definecolor{GridColor}{gray}{0.8}
    \@colorgridstrue
  }%
}% \colorgrids
\newcommand\nocolorgrids{\@colorgridsfalse}
\nocolorgrids

\newlength\gridsize
\newlength\gridlinewidth
\setlength{\gridsize}{5mm}
\setlength{\gridlinewidth}{0.1pt}

\def\fillwithgrid#1{%
  \begingroup
  \ifhmode
    \par
  \fi
  \hrule height \z@
  \nobreak

  % We first set box0 equal to an \hbox which, when printed, is a
  % square with width and height equal to \gridsize+\gridlinewidth,
  % but which has
  % width equal to \gridsize,
  % height equal to \gridsize, and
  % depth equal to 0pt.
  % When we put multiple copies of it together using \leaders or
  % \cleaders, the right edge will coincide with the left edge of the
  % next box and the bottom edge will coincide with the top edge of
  % the box below it.
  \setlength{\@tempdima}{\gridsize}
  \addtolength{\@tempdima}{\gridlinewidth}
  \setlength{\@tempdimb}{\gridsize}
  \addtolength{\@tempdimb}{-\gridlinewidth}
  \setbox0=\hbox{%
    \rlap{\vrule height \gridsize depth \gridlinewidth width \gridlinewidth}%
    \rlap{\vrule height \gridsize depth -\@tempdimb width \@tempdima}%
    \vrule height 0pt depth \gridlinewidth width \@tempdima
    \llap{\vrule height \gridsize depth \gridlinewidth width \gridlinewidth}%
  }%
  \wd0=\gridsize
  \dp0=0pt
  % Now we set box1 equal to an \hbox containing a single line of
  % copies of box0.  We use \leaders (instead of \cleaders) so that
  % if we use it twice on a page, once with a question and once
  % with a part, the boxes will line up vertically.  We add a kern of
  % \gridlinewidth at the right because the rightmost vertical line
  % appears to the right of where the \leaders command thinks that it
  % appears.
  \setbox1=\hbox to \textwidth{%
    \color@begingroup
    \if@colorgrids
      \color{GridColor}%
    \fi
    \hskip \@totalleftmargin \leaders\copy0\hfil \kern\gridlinewidth
    \color@endgroup
  }%
  % Finally: We create the grid, using \cleaders: We use \cleaders
  % (rather than \leaders) so that a given vertical space will always
  % produce the same number of lines no matter where on the page it
  % happens to start.  We add a kern of \gridlinewidth because the
  % bottommost horizontal line appears below where the \cleaders
  % command thinks that it appears.
  \cleaders \copy1 \vskip #1 \kern \gridlinewidth \hbox{}%
  \endgroup
}% fillwithgrid

%--------------------------------------------------------------------
%                            \makeemptybox

% \makeemptybox takes one argument, which is a length, and it creates
% an empty box of width the length of the current line and of height
% equal to the argument. That is, the box extends from the current
% left margin (which depends on whether we're in a question, part,
% subpart, or subsubpart) to the right margin.

% As of version 2.304, the argument of \makeemptybox can be either
% a length, or \fill, or \stretch{number}.

% \newcommand\makeemptybox[1]{
%   \par
%   \begingroup
%     \setlength{\fboxsep}{0pt}%
%     \framebox[\linewidth]{%
%       \vrule height 0pt depth #1 width 0pt
%     }%
%   \endgroup
% }

\newlength\minboxheight
\setlength\minboxheight{.1in}

\newcommand\makeemptybox[1]{%
  \par
  \hbox to \hsize{\hskip\@totalleftmargin \leaders\hrule\hfill}%
  \nointerlineskip
  \begingroup
    \setbox0=\hbox to \hsize{\hskip\@totalleftmargin
                           \vrule height\minboxheight \hfill \vrule}%
    % The vertical size desired may not be an exact multiple of
    % \minboxheight, and so \cleaders might leave a gap between the
    % vertical lines and the horizontal lines above and below it.
    % Thus, we put a single copy of \box0 immediately below the
    % horizontal line above and we'll also put a single copy of \box0
    % immediately above the horizontal line below.
    \copy0
    \nobreak
    \vskip -\minboxheight
    \cleaders \copy0 \vskip #1
    \vskip -\minboxheight
    \nointerlineskip
    \copy0
  \endgroup
  \nointerlineskip
  \hbox to \hsize{\hskip\@totalleftmargin \leaders\hrule\hfill}%
}

%--------------------------------------------------------------------
%                      \uplevel and \fullwidth
%           and the EnvUplevel and EnvFullwidth environments:

% \uplevel is used to print text at the indentation level of the 
% enclosing environment.  For example, to precede a question with
% directions about how that question should be answered, you would
% say \uplevel{Answer this question correctly.}
% 
% \fullwidth is similar, but uses the full page of text on the page.

% The EnvUplevel environment is similar to the \uplevel command, but it
% has the advantage that you can include verbatim material (using, e.g.,
% the \verb command) in the environment.  (You can't include verbatim
% material in the argument of an \uplevel command.)

% The EnvFullwidth environment is similar to the \fullwidth command, but
% it has the advantage that you can include verbatim material (using,
% e.g., the \verb command) in the environment.  (You can't include
% verbatim material in the argument of an \fullwidth command.)


\long\def\uplevel#1{%
  \par\bigskip
  \vbox{%
    % We entered internal vertical mode, and so we get \parshape=0.
    % We set \leftskip to provide the correct left margin for whatever
    % is in the argument of the \uplevel command:
    \leftskip=\@totalleftmargin
    \advance\leftskip-\leftmargin
    % We adjust \@totalleftmargin and linewidth in case there's a
    % solution environment inside of the argument to the \uplevel:
    \advance\@totalleftmargin-\leftmargin
    \advance\linewidth\leftmargin
    #1%
  }% vbox
  \nobreak
}

\newenvironment{EnvUplevel}
  {\par\bigskip\vbox\bgroup
    % We set \leftskip to provide the correct left margin for whatever
    % is inside of the environment:
    \leftskip=\@totalleftmargin
    \advance\leftskip-\leftmargin
    % We adjust \@totalleftmargin (and linewidth?) in case there's a
    % solution environment inside of the environment:
    \advance\@totalleftmargin-\leftmargin
    \advance\linewidth\leftmargin
  }
  {\egroup\nobreak}


\long\def\fullwidth#1{%
  \par\bigskip
  \vbox{%
    % We entered internal vertical mode, and so we get \parshape=0.
    \leftskip=0pt \rightskip=0pt
    \advance\linewidth\@totalleftmargin
    \@totalleftmargin=0pt
    #1%
  }% vbox
  \nobreak
}

\newenvironment{EnvFullwidth}
  {\par\bigskip\vbox\bgroup
    % We entered internal vertical mode, and so we get \parshape=0.
    \leftskip=0pt \rightskip=0pt
    % We adjust \@totalleftmargin (and linewidth?) in case there's a
    % solution environment inside of the environment:
    \advance\linewidth\@totalleftmargin
    \@totalleftmargin=0pt
  }
  {\egroup\nobreak}


%--------------------------------------------------------------------
%--------------------------------------------------------------------
%
%                        ********************
%                        ** GRADING TABLES **
%                        ********************



\newcounter{@iterator}
\newlength\@cellwidth

\def\cellwidth#1{\@cellwidth=#1}
\def\gradetablestretch#1{\def\@gtblstretch{#1}}

% \settabletotalpoints allows the user to specify a total
% number of points to appear in a table that may be different
% from the sum of the points in the table:
\newcommand\prt@tablepoints{\prt@hlfcntr{tbl@points}}
\newcommand\settabletotalpoints[1]{%
  \def\prt@tablepoints{#1}%
}% \settabletotalpoints

% \settabletotalbonuspoints is similar to \settabletotalpoints:
\newcommand\prt@tablebonuspoints{\prt@hlfcntr{tbl@bonuspoints}}
\newcommand\settabletotalbonuspoints[1]{%
  \def\prt@tablebonuspoints{#1}%
}% \settabletotalbonuspoints

% All of the following that begin with `h' are for horizontal tables,
% and all of them that begin with `v' are for vertical tables:
\def\hqword#1{\def\@hqword{#1}}
\def\hpword#1{\def\@hpword{#1}}
\def\hsword#1{\def\@hsword{#1}}
\def\htword#1{\def\@htword{#1}}
\def\hpgword#1{\def\@hpgword{#1}}

\def\vqword#1{\def\@vqword{#1}}
\def\vpword#1{\def\@vpword{#1}}
\def\vsword#1{\def\@vsword{#1}}
\def\vtword#1{\def\@vtword{#1}}
\def\vpgword#1{\def\@vpgword{#1}}


% The following are the versions for bonusgradetable:
\def\bhqword#1{\def\@bhqword{#1}}
\def\bhpword#1{\def\@bhpword{#1}}
\def\bhsword#1{\def\@bhsword{#1}}
\def\bhtword#1{\def\@bhtword{#1}}
\def\bhpgword#1{\def\@bhpgword{#1}}

\def\bvqword#1{\def\@bvqword{#1}}
\def\bvpword#1{\def\@bvpword{#1}}
\def\bvsword#1{\def\@bvsword{#1}}
\def\bvtword#1{\def\@bvtword{#1}}
\def\bvpgword#1{\def\@bvpgword{#1}}

% The following are the versions for combinedgradetable:
\def\chqword#1{\def\@chqword{#1}}
\def\chpword#1{\def\@chpword{#1}}
\def\chbpword#1{\def\@chbpword{#1}}
\def\chsword#1{\def\@chsword{#1}}
\def\chtword#1{\def\@chtword{#1}}
\def\chpgword#1{\def\@chpgword{#1}}

\def\cvqword#1{\def\@cvqword{#1}}
\def\cvpword#1{\def\@cvpword{#1}}
\def\cvbpword#1{\def\@cvbpword{#1}}
\def\cvsword#1{\def\@cvsword{#1}}
\def\cvtword#1{\def\@cvtword{#1}}
\def\cvpgword#1{\def\@cvpgword{#1}}



% Initialize:
\cellwidth{2em}
\gradetablestretch{1.5}

\hqword{Question:}
\hpgword{Page:}
\hpword{Points:}
\hsword{Score:}
\htword{Total}
\vpword{Points}
\vsword{Score}
\vtword{Total:}
\vqword{Question}
\vpgword{Page}

\bhqword{Question:}
\bhpgword{Page:}
\bhpword{Bonus Points:}
\bhsword{Score:}
\bhtword{Total}
\bvqword{Question}
\bvpgword{Page}
\bvpword{Bonus Points}
\bvsword{Score}
\bvtword{Total:}

\chqword{Question:}
\chpgword{Page:}
\chpword{Points:}
\chbpword{Bonus Points:}
\chsword{Score:}
\chtword{Total}
\cvqword{Question}
\cvpgword{Page}
\cvpword{Points}
\cvbpword{Bonus Points}
\cvsword{Score}
\cvtword{Total:}


% The only commands here accessible to the user are \gradetable,
% \bonusgradetable, \combinedgradetable, \pointtable,
% \bonuspointtable, \combinedpointtable, \partialgradetable,
% \partialbonusgradetable, \partialcombinedtable, \partialpointtable,
% \partialbonuspointtable, \partialcombinedpointtable,
% \begingradingrange, \endgradingrange, \pointsinrange,
% \bonuspointsinrange, \firstqinrange, \lastqinrange, and
% \numqinrange.

% The possibilities are

%   \gradetable[v][questions]
%   \gradetable[v][pages]
%   \gradetable[h][questions]
%   \gradetable[h][pages]

%   \bonusgradetable[v][questions]
%   \bonusgradetable[v][pages]
%   \bonusgradetable[h][questions]
%   \bonusgradetable[h][pages]

%   \combinedgradetable[v][questions]
%   \combinedgradetable[v][pages]
%   \combinedgradetable[h][questions]
%   \combinedgradetable[h][pages]

%   \pointtable[v][questions]
%   \pointtable[v][pages]
%   \pointtable[h][questions]
%   \pointtable[h][pages]

%   \bonuspointtable[v][questions]
%   \bonuspointtable[v][pages]
%   \bonuspointtable[h][questions]
%   \bonuspointtable[h][pages]

%   \combinedpointtable[v][questions]
%   \combinedpointtable[v][pages]
%   \combinedpointtable[h][questions]
%   \combinedpointtable[h][pages]

%   \partialgradetable{whatever}[v][questions]
%   \partialgradetable{whatever}[v][pages]
%   \partialgradetable{whatever}[h][questions]
%   \partialgradetable{whatever}[h][pages]

%   \partialbonusgradetable{whatever}[v][questions]
%   \partialbonusgradetable{whatever}[v][pages]
%   \partialbonusgradetable{whatever}[h][questions]
%   \partialbonusgradetable{whatever}[h][pages]

%   \partialcombinedgradetable{whatever}[v][questions]
%   \partialcombinedgradetable{whatever}[v][pages]
%   \partialcombinedgradetable{whatever}[h][questions]
%   \partialcombinedgradetable{whatever}[h][pages]

%   \partialpointtable{whatever}[v][questions]
%   \partialpointtable{whatever}[v][pages]
%   \partialpointtable{whatever}[h][questions]
%   \partialpointtable{whatever}[h][pages]

%   \partialbonuspointtable{whatever}[v][questions]
%   \partialbonuspointtable{whatever}[v][pages]
%   \partialbonuspointtable{whatever}[h][questions]
%   \partialbonuspointtable{whatever}[h][pages]

%   \partialcombinedpointtable{whatever}[v][questions]
%   \partialcombinedpointtable{whatever}[v][pages]
%   \partialcombinedpointtable{whatever}[h][questions]
%   \partialcombinedpointtable{whatever}[h][pages]

%   \begingradingrange{whatever}
%   \endgradingrange{whatever}
%
%   \pointsinrange{whatever}
%   \bonuspointsinrange{whatever}
%
%    \firstqinrange{whatever}
%    \lastqinrange{whatever}
%    \numqinrange{whatever}
%
% where ``whatever'' is a label chosen by the user.

% If one or both optional arguments are omitted, the defaults are
% `[v]' and `[questions]'.

% \@scorestrue means we're doing \gradetable
% \@scoresfalse mans we're doing \pointtable
\newif\if@scores

% \@partialtrue means we're doing \partialgradetable,
% \partialbonusgradetable, \partialcombinedgradetable,
% \partialpointtable, \partialbonuspointtable, or
% \partialcombinedpointtable:
\newif\if@partial

% \@combinedtrue means we're doing \combinedgradetable
% or \combinedpointtable:
\newif\if@combined

% It's OK to use the counter num@cols as a scratch counter
% in \begingradingrange and \endgradingrange because
% it's only used in typesetting tables:
\def\begingradingrange#1{%
  \setcounter{num@cols}{\value{question}}%
  \addtocounter{num@cols}{1}%
  \immediate\write\@mainaux
  {\string\expandafter\string\gdef
    \string\csname\space range@#1@firstq\string\endcsname
      {\arabic{num@cols}}}%
  \write\@mainaux
  {\string\expandafter\string\gdef
    \string\csname\space range@#1@firstp\string\endcsname
      {\thepage}}%
}% begingradingrange

\def\endgradingrange#1{%
  \setcounter{num@cols}{\value{question}}%
  \immediate\write\@mainaux
  {\string\expandafter\string\gdef
    \string\csname\space range@#1@lastq\string\endcsname
      {\arabic{num@cols}}}%
  \write\@mainaux
  {\string\expandafter\string\gdef
    \string\csname\space range@#1@lastp\string\endcsname
      {\thepage}}%
}% endgradingrange




% Now that grading tables may be for only part of the exam,
% we need the counter tbl@points to add up the total points
% for the questions (or pages) that appear on the table:
\new@hlfcntr{tbl@points}

% We'll use the counter tbl@bonuspoints to add up the total bonus
% points for the questions (or pages) that appear on the table:
\new@hlfcntr{tbl@bonuspoints}


\def\partialgradetable#1{%
  \@scorestrue
  \@bonusfalse
  \@partialtrue
  \@combinedfalse
  \def\tbl@range{#1}%
  % If the user doesn't include the optional argument
  % choosing between vertical and horizontal,
  % we give them vertical:
  \@ifnextchar[{\i@gtable}{\i@gtable[v]}%
}% partialgradetable

\def\partialbonusgradetable#1{%
  \@scorestrue
  \@bonustrue
  \@partialtrue
  \@combinedfalse
  \def\tbl@range{#1}%
  % If the user doesn't include the optional argument
  % choosing between vertical and horizontal,
  % we give them vertical:
  \@ifnextchar[{\i@gtable}{\i@gtable[v]}%
}% partialbonusgradetable

\def\partialcombinedgradetable#1{%
  \@scorestrue
  \@bonusfalse
  \@partialtrue
  \@combinedtrue
  \def\tbl@range{#1}%
  % If the user doesn't include the optional argument
  % choosing between vertical and horizontal,
  % we give them vertical:
  \@ifnextchar[{\i@gtable}{\i@gtable[v]}%
}% partialcombinedgradetable

\def\partialpointtable#1{%
  \@scoresfalse
  \@bonusfalse
  \@partialtrue
  \@combinedfalse
  \def\tbl@range{#1}%
  % If the user doesn't include the optional argument
  % choosing between vertical and horizontal,
  % we give them vertical:
  \@ifnextchar[{\i@gtable}{\i@gtable[v]}%
}% partialpointtable

\def\partialbonuspointtable#1{%
  \@scoresfalse
  \@bonustrue
  \@partialtrue
  \@combinedfalse
  \def\tbl@range{#1}%
  % If the user doesn't include the optional argument
  % choosing between vertical and horizontal,
  % we give them vertical:
  \@ifnextchar[{\i@gtable}{\i@gtable[v]}%
}% partialbonuspointtable

\def\partialcombinedpointtable#1{%
  \@scoresfalse
  \@bonusfalse
  \@partialtrue
  \@combinedtrue
  \def\tbl@range{#1}%
  % If the user doesn't include the optional argument
  % choosing between vertical and horizontal,
  % we give them vertical:
  \@ifnextchar[{\i@gtable}{\i@gtable[v]}%
}% partialcombinedpointtable

%--------------------------------------------------------------------

\def\gradetable{%
  \@scorestrue
  \@bonusfalse
  \@partialfalse
  \@combinedfalse
  % If the user doesn't include the optional argument
  % choosing between vertical and horizontal,
  % we give them vertical:
  \@ifnextchar[{\i@gtable}{\i@gtable[v]}%
}% gradetable

\def\bonusgradetable{%
  \@scorestrue
  \@bonustrue
  \@partialfalse
  \@combinedfalse
  % If the user doesn't include the optional argument
  % choosing between vertical and horizontal,
  % we give them vertical:
  \@ifnextchar[{\i@gtable}{\i@gtable[v]}%
}% bonusgradetable

\def\combinedgradetable{%
  \@scorestrue
  \@bonusfalse
  \@partialfalse
  \@combinedtrue
  % If the user doesn't include the optional argument
  % choosing between vertical and horizontal,
  % we give them vertical:
  \@ifnextchar[{\i@gtable}{\i@gtable[v]}%
}% bonusgradetable

\def\pointtable{%
  \@scoresfalse
  \@bonusfalse
  \@partialfalse
  \@combinedfalse
  % If the user doesn't include the optional argument
  % choosing between vertical and horizontal,
  % we give them vertical:
  \@ifnextchar[{\i@gtable}{\i@gtable[v]}%
}% pointtable

\def\bonuspointtable{%
  \@scoresfalse
  \@bonustrue
  \@partialfalse
  \@combinedfalse
  % If the user doesn't include the optional argument
  % choosing between vertical and horizontal,
  % we give them vertical:
  \@ifnextchar[{\i@gtable}{\i@gtable[v]}%
}% bonuspointtable

\def\combinedpointtable{%
  \@scoresfalse
  \@bonusfalse
  \@partialfalse
  \@combinedtrue
  % If the user doesn't include the optional argument
  % choosing between vertical and horizontal,
  % we give them vertical:
  \@ifnextchar[{\i@gtable}{\i@gtable[v]}%
}% bonuspointtable

%--------------------------------------------------------------------

% \i@gtable and \ii@gtable insert any missing optional arguments
% (the defaults being [v] and [questions]) and then make sure
% that the user said \addpoints and that this isn't the
% first run of LaTeX.
% \do@table then branches, depending on whether the user
% selected [questions] or [pages].

\def\i@gtable[#1]{%
  % If the user doesn't include the second optional argument,
  % which chooses between questions and pages,
  % we give them questions:
  \@ifnextchar[{\ii@gtable{#1}}{\ii@gtable{#1}[questions]}%
}
\def\ii@gtable#1[#2]{%
  % We make sure the user said \addpoints, and then make sure
  % that this isn't the first run of LaTeX (by checking that
  % \exam@numpoints is defined).  If both of those are OK,
  % we go on to \do@table to see whether we're doing a table
  % indexed by questions or by pages.
  \if@addpoints
    \@ifundefined{exam@numpoints}%
      {\ClassWarning{exam}%
        {%
          You must run LaTeX again to produce the
            table.\MessageBreak
        }%
        \fbox{Run \LaTeX{} again to produce the table}%
      }%
      {\do@table{#1}{#2}}%
  \else
    \ClassError{exam}{%
      You must give the command \protect\addpoints\MessageBreak
      \space\space in order to create a grade table.\MessageBreak
      }{%
      If you don't give the command \protect\addpoints\MessageBreak
      \space\space then we're not keeping track of point values.
      \MessageBreak
      }%
  \fi
}% ii@gtable


\def\@questionsref{questions}
\def\@pagesref{pages}
\def\do@table#1#2{%
  % The first argument is ``v'' or ``h'';
  % the second argument is ``questions'' or ``pages''.
  % See whether we're doing a table indexed by
  % questions or by pages:
  \begingroup % avoid trouble from using \@temp
    \def\@temp{#2}%
    \ifx\@temp\@questionsref
      \find@qrange{#1}%
    \else
      \ifx\@temp\@pagesref
        \find@prange{#1}%
      \else
        \ClassError{exam}{%
          The second optional argument to a\MessageBreak
           \space\space  grade table or point table command\MessageBreak 
            \space \space must be either `questions' or
            `pages',\MessageBreak
            \space\space not `#2'.\MessageBreak
        }{%
          Grade tables and point tables can be indexed by questions or
          pages;\MessageBreak
          \space\space for others, you're on your own.\MessageBreak
        }%
        \fbox{Error: grade or point table: Invalid second
                          optional argument `#2'.}%
      \fi
    \fi
  \endgroup
}% do@table


\def\range@undefined{%
  \fbox{Warning: grading range `\tbl@range ' not defined;
                               run \LaTeX{} again.}%
  \ClassWarning{exam}{%
    Grading range `\tbl@range ' not defined.\MessageBreak
    \space\space Run LaTeX again to produce the table.\MessageBreak
  }%
}% range@undefined


%--------------------------------------------------------------------
%--------------------------------------------------------------------
% Grade and point tables indexed by question numbers:

% We get to \find@qrange if we know we're doing a table indexed
% by question numbers.  The argument is either ``v'' or ``h''.
% If we're not doing a partial table, then \find@qrange sets
% \tbl@firstq to 1 and \tbl@lastq to \numquestions.  Otherwise,
% \find@qrange makes sure the grading range is defined and that
% its last question isn't before its first question.
% \find@qrange then calls \@tblquestions, passing along the argument
% that is either ``v'' or ``h''.

\def\find@qrange#1{%
  % We're doing a table indexed by question numbers.
  \if@partial
    \@ifundefined{range@\tbl@range @firstq}%
      {%
        \range@undefined
      }%
      {%
        \@ifundefined{range@\tbl@range @lastq}%
          {%
            \range@undefined
          }%
          {%
            \edef\tbl@firstq{\csname range@\tbl@range @firstq\endcsname}%
            \edef\tbl@lastq{\csname range@\tbl@range @lastq\endcsname}%
            % Check that firstq precedes or equals lastq:
            \ifnum \tbl@firstq > \tbl@lastq\relax
              \fbox{Error: Grading Range `\tbl@range ':
                      Last question precedes first question.}%
              \ClassError{exam}{%
                In grading range `\tbl@range ',
                                the last question\MessageBreak
                \space\space comes before the first question.\MessageBreak
                }{%
                  \string\begingradingrange \space must precede
                  \string\endgradingrange \space by at
                             least one question.\MessageBreak
                }%
            \else
              \@tblquestions{#1}%
            \fi
          }%
      }%
  \else
    \def\tbl@firstq{1}%
    % \numquestions is always defined, even if this is the first
    % run of LaTeX and \exam@numquestions isn't defined.
    % If it's the first run of LaTeX, then its value isn't useful,
    % but it's never used until a later run (when its value is useful).
    \def\tbl@lastq{\numquestions}%
    \@tblquestions{#1}%
  \fi
}% find@qrange

\def\@tblquestions#1{%
  % \tbl@firstq and \tbl@lastq have already been set.
  % The argument is either ``v'' or ``h', and we branch accordingly.
  \if v#1%
    \@vtblquestions
  \else
    \if h#1%
      \@htblquestions
    \else
      \ClassError{exam}{%
        The first optional argument to \protect\gradetable,\MessageBreak
        \space\space \protect\bonusgradetable,
        \protect\pointtable, or\MessageBreak
        \space\space  \protect\bonuspointtable
        \space must be either `h' or `v'\MessageBreak
      }{%
        Grade tables and point tables can be either horizontal or
        vertical;\MessageBreak
        \space\space no diagonals allowed.\MessageBreak
      }%
      \if@scores
        \if@bonus
          \fbox{Error: bonusgradetable: Invalid first optional argument
                                                        `#1'.}%
        \else
          \fbox{Error: gradetable: Invalid first optional argument
                                                        `#1'.}%
        \fi
      \else
        \if@bonus
          \fbox{Error: bonuspointtable: Invalid first optional argument
                                                    `#1'.}%
        \else
          \fbox{Error: pointtable: Invalid first optional argument
                                                    `#1'.}%
        \fi
      \fi
    \fi
  \fi
}% @tblquestions

%--------------------------------------------------------------------
% Vertical, indexed by question numbers:

\def\@vtblquestions{%
  % The table is vertical and indexed by question numbers.
  % The question range has already been determined.
  % Branch according to whether it's bonus, non-bonus, or combined,
  % and gradetable or pointtable:
  \begingroup
    % Save the current value of question in @iterator, so that
    % we can restore it after doing the table:
    \setcounter{@iterator}{\value{question}}%
    \renewcommand\arraystretch{\@gtblstretch}%
    \if@bonus
      \if@scores
        \@bvgrdtblquestions
      \else
        \@bvpttblquestions
      \fi
    \else
      \if@combined
        \if@scores
          \@cvgrdtblquestions
        \else
          \@cvpttblquestions
        \fi
      \else
        \if@scores
          \@vgrdtblquestions
        \else
          \@vpttblquestions
        \fi
      \fi
    \fi
    % Restore the saved value of question:
    \setcounter{question}{\value{@iterator}}%
  \endgroup
}% @vtblquestions

\def\@cvgrdtblquestions{%
  % Vertical combined gradetable, indexed by questions
  \set@hlfcntr{tbl@points}{0}%
  \set@hlfcntr{tbl@bonuspoints}{0}%
  \begin{tabular}{|c|c|c|c|}
    \hline
    {\@cvqword}& {\@cvpword}& {\@cvbpword}& {\@cvsword}\\
    \hline
    \setcounter{question}{\tbl@firstq}%
    \addtocounter{question}{-1}\do@cvloop
    {\@cvtword}& \prt@tablepoints&
      \prt@tablebonuspoints&
      \hbox to \@cellwidth{\hfill}\\
    \hline
  \end{tabular}%
}
\def\do@cvloop{%
  % \do@cvloop is used by \@cvgrdtblquestions
  \addtocounter{question}{1}%
  \ref{question@\arabic{question}} &
    \pointsofquestion{\arabic{question}}&
    \bonuspointsofquestion{\arabic{question}}&\\
  \hline
  \@ifundefined{pointsofq@\romannumeral \c@question}%
    {}%
    {\addto@hlfcntr{tbl@points}
       {\csname pointsofq@\romannumeral \c@question\endcsname}}%
  \@ifundefined{bonuspointsofq@\romannumeral \c@question}%
    {}%
    {\addto@hlfcntr{tbl@bonuspoints}
       {\csname bonuspointsofq@\romannumeral \c@question\endcsname}}%
  \ifnum \value{question} < \tbl@lastq\relax
    \let\next@cvloop=\do@cvloop
  \else
    \let\next@cvloop=\relax
  \fi
  \next@cvloop
}% do@cvloop


\def\@cvpttblquestions{%
  % Vertical combined pointtable, indexed by questions
  \set@hlfcntr{tbl@points}{0}%
  \set@hlfcntr{tbl@bonuspoints}{0}%
  \begin{tabular}{|c|c|c|}
    \hline
    {\@cvqword}& {\@cvpword}& {\@cvbpword}\\
    \hline
    \setcounter{question}{\tbl@firstq}%
    \addtocounter{question}{-1}\do@cvptloop
    {\@cvtword}& \prt@tablepoints&
      \prt@tablebonuspoints\\
    \hline
  \end{tabular}%
}
\def\do@cvptloop{%
  % \do@cvptloop is used by \@cvpttblquestions
  \addtocounter{question}{1}%
  \ref{question@\arabic{question}} &
    \pointsofquestion{\arabic{question}}&
    \bonuspointsofquestion{\arabic{question}}\\
  \hline
  \@ifundefined{pointsofq@\romannumeral \c@question}%
    {}%
    {\addto@hlfcntr{tbl@points}
       {\csname pointsofq@\romannumeral \c@question\endcsname}}%
  \@ifundefined{bonuspointsofq@\romannumeral \c@question}%
    {}%
    {\addto@hlfcntr{tbl@bonuspoints}
       {\csname bonuspointsofq@\romannumeral \c@question\endcsname}}%
  \ifnum \value{question} < \tbl@lastq\relax
    \let\next@cvptloop=\do@cvptloop
  \else
    \let\next@cvptloop=\relax
  \fi
  \next@cvptloop
}% do@cvptloop


\def\@bvgrdtblquestions{%
  % Vertical bonus gradetable, indexed by questions:
  \set@hlfcntr{tbl@bonuspoints}{0}%
  \begin{tabular}{|c|c|c|}
    \hline
    {\@bvqword}& {\@bvpword}& {\@bvsword}\\
    \hline
    \setcounter{question}{\tbl@firstq}%
    \addtocounter{question}{-1}\do@bvloop
    {\@bvtword}& \prt@tablebonuspoints&\hbox to \@cellwidth{\hfill}\\
    \hline
  \end{tabular}%
}% @bvgrdtblquestions
\def\do@bvloop{%
  % \do@bvloop is used by \@bvgrdtblquestions
  \addtocounter{question}{1}%
  \ref{question@\arabic{question}} &
        \bonuspointsofquestion{\arabic{question}}&\\
  \hline
  \@ifundefined{bonuspointsofq@\romannumeral \c@question}%
    {}%
    {\addto@hlfcntr{tbl@bonuspoints}
       {\csname bonuspointsofq@\romannumeral \c@question\endcsname}}%
  \ifnum \value{question} < \tbl@lastq\relax
    \let\next@bvloop=\do@bvloop
  \else
    \let\next@bvloop=\relax
  \fi
  \next@bvloop
}% do@bvloop


\def\@bvpttblquestions{%
  % Vertical bonus point table, indexed by questions:
  \set@hlfcntr{tbl@bonuspoints}{0}%
  \begin{tabular}{|c|c|}
    \hline
    {\@bvqword}& {\@bvpword}\\
    \hline
    \setcounter{question}{\tbl@firstq}%
    \addtocounter{question}{-1}\do@bvptloop
    {\@bvtword}& \prt@tablebonuspoints\\
    \hline
  \end{tabular}%
}% @bvpttblquestions
\def\do@bvptloop{%
  % \do@bvptloop is used by \@bvpttblquestions
  \addtocounter{question}{1}%
  \ref{question@\arabic{question}} &
        \bonuspointsofquestion{\arabic{question}}\\
  \hline
  \@ifundefined{bonuspointsofq@\romannumeral \c@question}%
    {}%
    {\addto@hlfcntr{tbl@bonuspoints}
       {\csname bonuspointsofq@\romannumeral \c@question\endcsname}}%
  \ifnum \value{question} < \tbl@lastq\relax
    \let\next@bvptloop=\do@bvptloop
  \else
    \let\next@bvptloop=\relax
  \fi
  \next@bvptloop
}% do@bvptloop

\def\@vgrdtblquestions{%
  % Vertical non-bonus grade table, indexed by questions:
  \set@hlfcntr{tbl@points}{0}%
  \begin{tabular}{|c|c|c|}
    \hline
    {\@vqword}& {\@vpword}& {\@vsword}\\
    \hline
    \setcounter{question}{\tbl@firstq}%
    \addtocounter{question}{-1}\do@vloop
    {\@vtword}& \prt@tablepoints&\hbox to \@cellwidth{\hfill}\\
    \hline
  \end{tabular}%
}% @vgrdtblquestions
\def\do@vloop{%
  % \do@vloop is used by \@vgrdtblquestions
  \addtocounter{question}{1}%
  \ref{question@\arabic{question}} &
        \pointsofquestion{\arabic{question}}&\\
  \hline
  \@ifundefined{pointsofq@\romannumeral \c@question}%
    {}%
    {\addto@hlfcntr{tbl@points}
       {\csname pointsofq@\romannumeral \c@question\endcsname}}%
  \ifnum \value{question} < \tbl@lastq\relax
    \let\next@vloop=\do@vloop
  \else
    \let\next@vloop=\relax
  \fi
  \next@vloop
}% do@vloop


\def\@vpttblquestions{%
  % Vertical non-bonus point table, indexed by questions:
  \set@hlfcntr{tbl@points}{0}%
  \begin{tabular}{|c|c|}
    \hline
    {\@vqword}& {\@vpword}\\
    \hline
    \setcounter{question}{\tbl@firstq}%
    \addtocounter{question}{-1}\do@vptloop
    {\@vtword}& \prt@tablepoints\\
    \hline
  \end{tabular}%
}% @vpttblquestions
\def\do@vptloop{%
  % \do@vptloop is used by \@vpttblquestions
  \addtocounter{question}{1}%
  \ref{question@\arabic{question}} &
        \pointsofquestion{\arabic{question}}\\
  \hline
  \@ifundefined{pointsofq@\romannumeral \c@question}%
    {}%
    {\addto@hlfcntr{tbl@points}
       {\csname pointsofq@\romannumeral \c@question\endcsname}}%
  \ifnum \value{question} < \tbl@lastq\relax
    \let\next@vptloop=\do@vptloop
  \else
    \let\next@vptloop=\relax
  \fi
  \next@vptloop
}% do@vptloop


%--------------------------------------------------------------------
% Horizontal, indexed by question numbers:

\def\@htblquestions{%
  % We get here from \@tblquestions.
  % The table is horizontal and indexed by question numbers.
  % The question range has already been determined.
  % Set num@cols equal to the number of questions on the table, and
  % do either bonus, combined, or non-bonus table, in each case
  % putting in a line for scores only if it's a gradetable:
  \@ifundefined{exam@numquestions}%
  {}%
  {%
    \setcounter{num@cols}{\tbl@lastq}%
    \addtocounter{num@cols}{-\tbl@firstq}%
    \addtocounter{num@cols}{1}%
  }%
  % Do either bonus, combined, or non-bonus table.  In each case, put
  % in a line for scores if it's a gradetable:
  \begingroup
    % Save the current value of question in @iterator, so that
    % we can restore it after doing the table:
    \setcounter{@iterator}{\value{question}}%
    \renewcommand\arraystretch{\@gtblstretch}%
    \if@bonus
      % It's a horizontal bonus table, by questions:
      \begin{tabular}{|l|*{\thenum@cols}{c|}c|}
        \hline
        {\@bhqword}& \setcounter{question}{\tbl@firstq}%
        \addtocounter{question}{-1}\do@qnumloop
        {\@bhtword}\\
        \hline
        \set@hlfcntr{tbl@bonuspoints}{0}%
        {\@bhpword}& \setcounter{question}{\tbl@firstq}%
        \addtocounter{question}{-1}\do@bptloop
        \prt@tablebonuspoints\\
        \hline
        % If it's a grade table, add in the score line:
        \if@scores
          {\@bhsword}& \setcounter{question}{\tbl@firstq}%
          \addtocounter{question}{-1}\do@sloop
          \\
          \hline
        \fi
      \end{tabular}%
    \else
      \if@combined
        % It's a horizontal combined table, by questions:
        \begin{tabular}{|l|*{\thenum@cols}{c|}c|}
          \hline
          {\@chqword}& \setcounter{question}{\tbl@firstq}%
          \addtocounter{question}{-1}\do@qnumloop
          {\@chtword}\\
          \hline
          \set@hlfcntr{tbl@points}{0}%
          {\@chpword}& \setcounter{question}{\tbl@firstq}%
          \addtocounter{question}{-1}\do@ptloop
          \prt@tablepoints\\
          \hline
          \set@hlfcntr{tbl@bonuspoints}{0}%
          {\@chbpword}& \setcounter{question}{\tbl@firstq}%
          \addtocounter{question}{-1}\do@bptloop
          \prt@tablebonuspoints\\
          \hline
          % If it's a grade table, add in the score line:
          \if@scores
            {\@chsword}& \setcounter{question}{\tbl@firstq}%
            \addtocounter{question}{-1}\do@sloop
            \\
            \hline
          \fi
        \end{tabular}%
      \else
        % Horizontal non-bonus table, indexed by question number:
        \set@hlfcntr{tbl@points}{0}%
        \begin{tabular}{|l|*{\thenum@cols}{c|}c|}
          \hline
          {\@hqword}& \setcounter{question}{\tbl@firstq}%
          \addtocounter{question}{-1}\do@qnumloop
          {\@htword}\\
          \hline
          {\@hpword}& \setcounter{question}{\tbl@firstq}%
          \addtocounter{question}{-1}\do@ptloop
          \prt@tablepoints\\
          \hline
          % If it's a grade table, add in the score line:
          \if@scores
            {\@hsword}& \setcounter{question}{\tbl@firstq}%
            \addtocounter{question}{-1}\do@sloop
            \\
            \hline
          \fi
        \end{tabular}%
      \fi
    \fi
    % Restore the saved value of question:
    \setcounter{question}{\value{@iterator}}%
  \endgroup
}% @htblquestions


% \do@qnumloop and \do@sloop are used by non-bonus,
% combined, and bonus tables:
\def\do@qnumloop{%
  \addtocounter{question}{1}%
  \ref{question@\arabic{question}}&
  \ifnum \value{question} < \tbl@lastq\relax
    \let\next@qnloop=\do@qnumloop
  \else
    \let\next@qnloop=\relax
  \fi
  \next@qnloop
}% do@qnumloop
\def\do@sloop{%
  \addtocounter{question}{1}%
  \hbox to \@cellwidth{\hfill}&
  \ifnum \value{question} < \tbl@lastq\relax
    \let\next@sloop=\do@sloop
  \else
    \let\next@sloop=\relax
  \fi
  \next@sloop
}% do@sloop

%  \do@ptloop is used by combined and non-bonus tables:
\def\do@ptloop{%
  \addtocounter{question}{1}%
  \pointsofquestion{\arabic{question}}&
  \@ifundefined{pointsofq@\romannumeral \c@question}%
    {}%
    {\addto@hlfcntr{tbl@points}
       {\csname pointsofq@\romannumeral \c@question\endcsname}}%
  \ifnum \value{question} < \tbl@lastq\relax
    \let\next@ptloop=\do@ptloop
  \else
    \let\next@ptloop=\relax
  \fi
  \next@ptloop
}% do@ptloop

%  \do@bptloop is used by bonus and combined tables:
\def\do@bptloop{%
  \addtocounter{question}{1}%
  \bonuspointsofquestion{\arabic{question}}&
  \@ifundefined{bonuspointsofq@\romannumeral \c@question}%
    {}%
    {\addto@hlfcntr{tbl@bonuspoints}
       {\csname bonuspointsofq@\romannumeral \c@question\endcsname}}%
  \ifnum \value{question} < \tbl@lastq\relax
    \let\next@bptloop=\do@bptloop
  \else
    \let\next@bptloop=\relax
  \fi
  \next@bptloop
}% do@bptloop


%--------------------------------------------------------------------
%--------------------------------------------------------------------
% Grade and point tables indexed by page numbers:


% The only pages listed are those on which there are a nonzero number
% of points.  We check pages \tbl@firstp through \tbl@lastp
% Once we've checked that, e.g., \lastpage@withpoints and
% \pointsonpage@\romannumeral{\lastpage@withpoints} are defined, we
% can safely (we think) check \pointsonpage@\romannumeral{n} for all n
% between \tbl@firstp and \tbl@lastp without generating errors.
%
% Actually: Since we added the notion of half points and half counters
% (a long time ago), there won't be any errors even if
% \pointsonpage@\romannumeral{n} isn't defined, since it's tested by the
% lines:
%   \set@hlfcntr{tmp@hlfcntr}{\csname pointsonpage@\romannumeral
%                          \csname c@@iterator\endcsname\endcsname}%
%   \ifhlfcntr@pos{tmp@hlfcntr}%
% and if 
%   \csname pointsonpage@\romannumeral
%             \csname c@@iterator\endcsname\endcsname
% isn't defined, tmp@hlfcntr gets the value zero (because of the way
% that \set@hlfcntr is written).

% \find@prange makes sure the grading range is defined and that its
% last page isn't before its first page (if it's a partial table).  In
% any case, it then sets \tbl@firstp and \tbl@lastp, and calls
% \@tblpages.

\def\find@prange#1{%
  % We're doing a table indexed by pages.
  % The argument is either ``v'' or ``h''.
  % After determining the first and last page of the range, we
  % call \@tblpages, passing it that argument (i.e., we say
  % \@tblpages{#1}).
  % If not a partial table, we set \tbl@firstp to 1 and \tbl@lastp to
  % the last page with the appropriate points (and so if it's a
  % combined table, it's the last page to have either bonus or
  % non-bonus points).
  \if@partial
    \@ifundefined{range@\tbl@range @firstp}%
      {%
        \range@undefined
      }%
      {%
        \@ifundefined{range@\tbl@range @lastp}%
          {%
            \range@undefined
          }%
          {%
            \edef\tbl@firstp{\csname range@\tbl@range @firstp\endcsname}%
            \edef\tbl@lastp{\csname range@\tbl@range @lastp\endcsname}%
            % Check that firstp precedes or equals lastp:
            \ifnum \tbl@firstp > \tbl@lastp\relax
              \fbox{Error: Grading Range `\tbl@range ':
                      Last page precedes first page.}%
              \ClassError{exam}{%
                In grading range `\tbl@range ', the last page\MessageBreak
                \space\space comes before the first page.\MessageBreak
                }{%
                  \string\begingradingrange \space must precede
                  \string\endgradingrange.\MessageBreak
                }%
            \else
              \@tblpages{#1}%
            \fi
          }%
      }%
  \else
    \def\tbl@firstp{1}%
    % \tbl@lastp isn't used on the first run of LaTeX, and
    % \lastpage@withbonuspoints is defined on the second and later runs.
    \def\tbl@lastp{\lastpage@withpoints}%
    \if@bonus
      \def\tbl@lastp{\lastpage@withbonuspoints}%
    \fi
    \if@combined
      \ifnum \lastpage@withbonuspoints > \lastpage@withpoints\relax
        \def\tbl@lastp{\lastpage@withbonuspoints}%
      \fi        
    \fi
    \@tblpages{#1}%
  \fi
}% find@prange


\def\@tblpages#1{%
  % We get here from \find@prange; the argument is either ``v'' or
  % ``h''.
  % Check that there's enough info from the .aux file to do a page
  % indexed grade table.  If so, call \@whichtblpgs{#1}:
  \@ifundefined{lastpage@withpoints}%
    {\ClassWarning{exam}{%
       You must run LaTeX twice more\MessageBreak
       \space\space to produce the table.\MessageBreak}%
       \fbox{Run \LaTeX{} twice more to produce the table}%
    }%
    {%
      \@ifundefined{pointsonpage@\romannumeral
               \csname lastpage@withpoints\endcsname}%
        {\ClassWarning{exam}{%
           You must run LaTeX again\MessageBreak
           \space\space to produce the table.\MessageBreak}%
           \fbox{Run \LaTeX{} again to produce the table}%
        }%
        {%
          \@whchtblpgs#1
        }%
    }%
}% @tblpages

\def\@whchtblpgs#1{%
  % At this point, we know the table is indexed by pages.
  % It can be vertical or horizontal, 
  % grade or point, and bonus or non-bonus.
  % The argument is either ``v'' or ``h''; test it, and branch:
  \if v#1%
    \@vtblpages
  \else
    \if h#1%
      \@htblpages
    \else
      \ClassError{exam}{%
        The first optional argument to
          \protect\gradetable,\MessageBreak 
          \space\space \protect\bonusgradetable,
            \protect\pointtable, or\MessageBreak
          \space\space
          \protect\bonuspointtable \space must be either `h' or
          `v'\MessageBreak
      }{%
        Grade tables and point tables can be either horizontal or
        vertical;\MessageBreak
        \space\space no diagonals allowed.\MessageBreak
      }%
      \if@scores
        \if@bonus
          \fbox{Error: bonusgradetable: Invalid first optional argument
                                                  `#1'.}%
        \else
          \fbox{Error: gradetable: Invalid first optional argument
                                                  `#1'.}%
        \fi
      \else
        \if@bonus
          \fbox{Error: bonuspointtable: Invalid first optional argument
                                                  `#1'.}%
        \else
          \fbox{Error: pointtable: Invalid first optional argument
                                                  `#1'.}%
        \fi
      \fi
    \fi
  \fi
}% \@whchtblpgs

%--------------------------------------------------------------------
% Vertical, indexed by pages:

\def\@vtblpages{%
  % We get here from \@whchtblpgs.
  % At this point, we know it's a vertical table indexed by pages.
  % It can be grade or point, and bonus, non-bonus, or combined.
  \begingroup
    \renewcommand\arraystretch{\@gtblstretch}%
    \if@bonus
      % It's vertical, bonus, by pages, either point or grade:
      \set@hlfcntr{tbl@bonuspoints}{0}%
      \if@scores
        \begin{tabular}{|c|c|c|}
          \hline
          {\@bvpgword}& {\@bvpword}& {\@bvsword}\\
          \hline
          \setcounter{@iterator}{\tbl@firstp}%
          \addtocounter{@iterator}{-1}\pg@bvloop
          {\@bvtword}& \prt@tablebonuspoints&
                              \hbox to \@cellwidth{\hfill}\\
          \hline
        \end{tabular}%
      \else
        \begin{tabular}{|c|c|}
          \hline
          {\@bvpgword}& {\@bvpword}\\
          \hline
          \setcounter{@iterator}{\tbl@firstp}%
          \addtocounter{@iterator}{-1}\pg@ptbvloop
          {\@bvtword}& \prt@tablebonuspoints\\
          \hline
        \end{tabular}%
      \fi
    \else
      \if@combined
        % It's vertical, combined, by pages, either point or grade:
        \set@hlfcntr{tbl@points}{0}%
        \set@hlfcntr{tbl@bonuspoints}{0}%
        \if@scores
          % Combined gradetable, vertical, by pages:
          \begin{tabular}{|c|c|c|c|}
            \hline
            {\@cvpgword}& {\@cvpword}& {\@cvbpword}& {\@cvsword}\\
            \hline
            \setcounter{@iterator}{\tbl@firstp}%
            \addtocounter{@iterator}{-1}\pg@cvloop
            {\@cvtword}& \prt@tablepoints&
              \prt@tablebonuspoints&
              \hbox to \@cellwidth{\hfill}\\
            \hline
          \end{tabular}%
        \else
          % Combined pointtable, vertical, by pages:
          \begin{tabular}{|c|c|c|}
            \hline
            {\@cvpgword}& {\@cvpword}& {\@cvbpword}\\
            \hline
            \setcounter{@iterator}{\tbl@firstp}%
            \addtocounter{@iterator}{-1}\pg@ptcvloop
            {\@cvtword}& \prt@tablepoints&
              \prt@tablebonuspoints\\
            \hline
          \end{tabular}%
        \fi
      \else
        % It's non-bonus, either point or grade:
        \set@hlfcntr{tbl@points}{0}%
        \if@scores
          \begin{tabular}{|c|c|c|}
            \hline
            {\@vpgword}& {\@vpword}& {\@vsword}\\
            \hline
            \setcounter{@iterator}{\tbl@firstp}%
            \addtocounter{@iterator}{-1}\pg@vloop
            {\@vtword}& \prt@tablepoints&\hbox to \@cellwidth{\hfill}\\
            \hline
          \end{tabular}%
        \else
          \begin{tabular}{|c|c|}
            \hline
            {\@vpgword}& {\@vpword}\\
            \hline
            \setcounter{@iterator}{\tbl@firstp}%
            \addtocounter{@iterator}{-1}\pg@ptvloop
            {\@vtword}& \prt@tablepoints\\
            \hline
          \end{tabular}%
        \fi
      \fi
    \fi
  \endgroup
}% @vtblpages


% \pg@ptvloop and \pg@ptvloopline are used only by non-bonus pointtable:
\def\pg@ptvloop{%
  \addtocounter{@iterator}{1}%
  \set@hlfcntr{tmp@hlfcntr}{\pointsonpage{\the@iterator}}%
  \ifhlfcntr@pos{tmp@hlfcntr}%
    \add@hlfcntrtohlfcntr{tbl@points}{tmp@hlfcntr}%
    \pg@ptvloopline
  \fi
  \ifnum \the@iterator < \tbl@lastp\relax
    \let\next@pg@ptvloop=\pg@ptvloop
  \else
    \let\next@pg@ptvloop=\relax
  \fi
  \next@pg@ptvloop
}% pg@ptvloop
\def\pg@ptvloopline{%
  % We still don't understand why we need to hide this inside of a
  % macro; there's some weird interaction with the \ifnum checking to
  % see if something is ``>0''; checking that something is ``=0''
  % doesn't cause the ``\ifnum incomplete'' (or whatever) error.
  \pageref{firstpoints@onpage@\arabic{@iterator}} &
                         \pointsonpage{\the@iterator}\\
  \hline
}% pg@ptvloopline

% \pg@vloop and \pg@vloopline are used only by non-bonus gradetable:
\def\pg@vloop{%
  \addtocounter{@iterator}{1}%
  \set@hlfcntr{tmp@hlfcntr}{\pointsonpage{\the@iterator}}%
  \ifhlfcntr@pos{tmp@hlfcntr}%
    \add@hlfcntrtohlfcntr{tbl@points}{tmp@hlfcntr}%
    \pg@vloopline
  \fi
  \ifnum \the@iterator < \tbl@lastp\relax
    \let\next@pg@vloop=\pg@vloop
  \else
    \let\next@pg@vloop=\relax
  \fi
  \next@pg@vloop
}% pg@vloop
\def\pg@vloopline{%
  % We still don't understand why we need to hide this inside of a
  % macro; there's some weird interaction with the \ifnum checking to
  % see if something is ``>0''; checking that something is ``=0''
  % doesn't cause the ``\ifnum incomplete'' (or whatever) error.
  \pageref{firstpoints@onpage@\arabic{@iterator}} &
                         \pointsonpage{\the@iterator}&\\
  \hline
}% pg@vloopline

% \pg@cvloop, \pg@noncvloopline, \check@bnspts, and
% \pg@cvloopline are used only by combined gradetable:
\def\pg@cvloop{%
  \addtocounter{@iterator}{1}%
  % Add the bonus points:
  \set@hlfcntr{tmp@hlfcntr}{\bonuspointsonpage{\the@iterator}}%
  \add@hlfcntrtohlfcntr{tbl@bonuspoints}{tmp@hlfcntr}%
  % Add the regular points:
  \set@hlfcntr{tmp@hlfcntr}{\csname pointsonpage@\romannumeral
         \csname c@@iterator\endcsname\endcsname}%
  \add@hlfcntrtohlfcntr{tbl@points}{tmp@hlfcntr}%
  % Print a line of the table, if there are points on this page:
  \ifhlfcntr@pos{tmp@hlfcntr}%
    \pg@noncvloopline
  \else
    \check@bnspts
  \fi
  % Do the tail recursion bit:
  \ifnum \the@iterator < \tbl@lastp\relax
    \let\next@pg@cvloop=\pg@cvloop
  \else
    \let\next@pg@cvloop=\relax
  \fi
  \next@pg@cvloop
}% pg@cvloop
\def\pg@noncvloopline{%
  % We still don't understand why we need to hide this inside of a
  % macro; there's some weird interaction with the \ifnum checking to
  % see if something is ``>0''; checking that something is ``=0''
  % doesn't cause the ``\ifnum incomplete'' (or whatever) error.
  \pageref{firstpoints@onpage@\arabic{@iterator}}&
           \pointsonpage{\the@iterator}&
           \bonuspointsonpage{\the@iterator}&\\
  \hline
}% pg@noncvloopline
\def\check@bnspts{%
  % We still don't understand why we need to hide this inside of a
  % macro; there's some weird interaction with the \ifnum checking to
  % see if something is ``>0''; checking that something is ``=0''
  % doesn't cause the ``\ifnum incomplete'' (or whatever) error.
  \set@hlfcntr{tmp@hlfcntr}{\bonuspointsonpage{\the@iterator}}%
  \ifhlfcntr@pos{tmp@hlfcntr}%
    \pg@cvloopline
  \fi
}% check@bnspts
\def\pg@cvloopline{%
  % We still don't understand why we need to hide this inside of a
  % macro; there's some weird interaction with the \ifnum checking to
  % see if something is ``>0''; checking that something is ``=0''
  % doesn't cause the ``\ifnum incomplete'' (or whatever) error.
  \pageref{firstbonuspoints@onpage@\arabic{@iterator}}&
           \pointsonpage{\the@iterator}&
           \bonuspointsonpage{\the@iterator}&\\
  \hline
}% pg@cvloopline

% \pg@ptcvloop, \pg@ptnoncvloopline, \check@ptbnspts, and
% \pg@ptcvloopline are used only by combined pointtable:
\def\pg@ptcvloop{%
  \addtocounter{@iterator}{1}%
  % Add the bonus points:
  \set@hlfcntr{tmp@hlfcntr}{\bonuspointsonpage{\the@iterator}}%
  \add@hlfcntrtohlfcntr{tbl@bonuspoints}{tmp@hlfcntr}%
  % Add the regular points:
  \set@hlfcntr{tmp@hlfcntr}{\pointsonpage{\the@iterator}}%
  \add@hlfcntrtohlfcntr{tbl@points}{tmp@hlfcntr}%
  % Print a line of the table, if there are points on this page:
  \ifhlfcntr@pos{tmp@hlfcntr}%
    \pg@ptnoncvloopline
  \else
    \check@ptbnspts
  \fi
  % Do the tail recursion bit:
  \ifnum \the@iterator < \tbl@lastp\relax
    \let\next@pg@ptcvloop=\pg@ptcvloop
  \else
    \let\next@pg@ptcvloop=\relax
  \fi
  \next@pg@ptcvloop
}% pg@ptcvloop
\def\pg@ptnoncvloopline{%
  % We still don't understand why we need to hide this inside of a
  % macro; there's some weird interaction with the \ifnum checking to
  % see if something is ``>0''; checking that something is ``=0''
  % doesn't cause the ``\ifnum incomplete'' (or whatever) error.
  \pageref{firstpoints@onpage@\arabic{@iterator}}&
           \pointsonpage{\the@iterator}&
           \bonuspointsonpage{\the@iterator}\\
  \hline
}% pg@ptnoncvloopline
\def\check@ptbnspts{%
  % We still don't understand why we need to hide this inside of a
  % macro; there's some weird interaction with the \ifnum checking to
  % see if something is ``>0''; checking that something is ``=0''
  % doesn't cause the ``\ifnum incomplete'' (or whatever) error.
  \set@hlfcntr{tmp@hlfcntr}{\bonuspointsonpage{\the@iterator}}%
  \ifhlfcntr@pos{tmp@hlfcntr}%
    \pg@ptcvloopline
  \fi
}% check@ptbnspts
\def\pg@ptcvloopline{%
  % We still don't understand why we need to hide this inside of a
  % macro; there's some weird interaction with the \ifnum checking to
  % see if something is ``>0''; checking that something is ``=0''
  % doesn't cause the ``\ifnum incomplete'' (or whatever) error.
  \pageref{firstbonuspoints@onpage@\arabic{@iterator}}&
           \pointsonpage{\the@iterator}&
           \bonuspointsonpage{\the@iterator}\\
  \hline
}% pg@ptcvloopline

% \pg@bvloop and \pg@bvloopline are used only by bonus gradetable:
\def\pg@bvloop{%
  \addtocounter{@iterator}{1}%
  \set@hlfcntr{tmp@hlfcntr}{\bonuspointsonpage{\the@iterator}}%
  \ifhlfcntr@pos{tmp@hlfcntr}%
    \add@hlfcntrtohlfcntr{tbl@bonuspoints}{tmp@hlfcntr}%
    \pg@bvloopline
  \fi
  \ifnum \the@iterator < \tbl@lastp\relax
    \let\next@pg@bvloop=\pg@bvloop
  \else
    \let\next@pg@bvloop=\relax
  \fi
  \next@pg@bvloop
}% pg@bvloop
\def\pg@bvloopline{%
  % We still don't understand why we need to hide this inside of a
  % macro; there's some weird interaction with the \ifnum checking to
  % see if something is ``>0''; checking that something is ``=0''
  % doesn't cause the ``\ifnum incomplete'' (or whatever) error.
  \pageref{firstbonuspoints@onpage@\arabic{@iterator}} &
                         \bonuspointsonpage{\the@iterator}&\\
  \hline
}% pg@bvloopline

% \pg@ptbvloop and \pg@ptbvloopline are used only by bonus pointtable:
\def\pg@ptbvloop{%
  \addtocounter{@iterator}{1}%
  \set@hlfcntr{tmp@hlfcntr}{\bonuspointsonpage{\the@iterator}}%
  \ifhlfcntr@pos{tmp@hlfcntr}%
    \add@hlfcntrtohlfcntr{tbl@bonuspoints}{tmp@hlfcntr}%
    \pg@ptbvloopline
  \fi
  \ifnum \the@iterator < \tbl@lastp\relax
    \let\next@pg@ptbvloop=\pg@ptbvloop
  \else
    \let\next@pg@ptbvloop=\relax
  \fi
  \next@pg@ptbvloop
}% pg@ptbvloop
\def\pg@ptbvloopline{%
  % We still don't understand why we need to hide this inside of a
  % macro; there's some weird interaction with the \ifnum checking to
  % see if something is ``>0''; checking that something is ``=0''
  % doesn't cause the ``\ifnum incomplete'' (or whatever) error.
  \pageref{firstbonuspoints@onpage@\arabic{@iterator}} &
                         \bonuspointsonpage{\the@iterator}\\
  \hline
}% pg@ptbvloopline

%--------------------------------------------------------------------
% Horizontal, indexed by pages:

% For a horizontal table, before we begin we need to know how many
% pages there are with points on them:
\newcounter{num@cols}
\def\count@pgswpts{%
  % We're called by \@htblpages.
  % \count@pgswpts is used for horizontal
  % grade tables and point tables,
  % bonus, non-bonus, and combined.
  \setcounter{num@cols}{0}%
  \setcounter{@iterator}{\tbl@firstp}%
  \addtocounter{@iterator}{-1}%
  \if@bonus
    \docount@pgswbpts
  \else
    \if@combined
      \docount@pgswcpts
    \else
      \docount@pgswpts
    \fi
  \fi
}% count@pgswpts

\def\docount@pgswcpts{%
  % Called by \count@pgswpts
  % \docount@pgswcpts is used for combined gradetable.
  % Count the number of pages in range with any kind of point (bonus
  % or non-bonus):
  \addtocounter{@iterator}{1}%
  \set@hlfcntr{tmp@hlfcntr}{\pointsonpage{\the@iterator}}%
  \ifhlfcntr@pos{tmp@hlfcntr}%
    \addtocounter{num@cols}{1}%
  \else
    \check@bnsptpage
  \fi
  \ifnum \the@iterator < \tbl@lastp\relax
    \let\next@docount=\docount@pgswcpts
  \else
    \let\next@docount=\relax
  \fi
  \next@docount
}% docount@pgswcpts
\def\check@bnsptpage{%
  % We don't understand why we need to hide this inside of a macro:
  \set@hlfcntr{tmp@hlfcntr}{\bonuspointsonpage{\the@iterator}}%
  \ifhlfcntr@pos{tmp@hlfcntr}%
    \addtocounter{num@cols}{1}%
  \fi
}% check@bnsptpage

\def\docount@pgswpts{%
  % Called by \count@pgswpts
  % \docount@pgswpts is used for non-bonus gradetable:
  \addtocounter{@iterator}{1}%
  \set@hlfcntr{tmp@hlfcntr}{\pointsonpage{\the@iterator}}%
  \ifhlfcntr@pos{tmp@hlfcntr}%
    \addtocounter{num@cols}{1}%
  \fi
  \ifnum \the@iterator < \tbl@lastp\relax
    \let\next@docount=\docount@pgswpts
  \else
    \let\next@docount=\relax
  \fi
  \next@docount
}% docount@pgswpts

\def\docount@pgswbpts{%
  % Called by \count@pgswpts
  % \docount@pgswbpts is used for bonusgradetable:
  \addtocounter{@iterator}{1}%
  \set@hlfcntr{tmp@hlfcntr}{\bonuspointsonpage{\the@iterator}}%
  \ifhlfcntr@pos{tmp@hlfcntr}%
    \addtocounter{num@cols}{1}%
  \fi
  \ifnum \the@iterator < \tbl@lastp\relax
    \let\next@docount=\docount@pgswbpts
  \else
    \let\next@docount=\relax
  \fi
  \next@docount
}% docount@pgswbpts

%--------------------------------------------------------------------

\def\@htblpages{%
  % We get here from \@whchtblpgs.
  % Horizontal grade or point table, by pages:
  \begingroup
    \renewcommand\arraystretch{\@gtblstretch}%
    % Set num@cols equal to the number of pages with points (of
    % correct type): 
    \count@pgswpts
    % Suppress ``extra alignment tab'' errors if \thenum@cols = 0:
    \ifnum \thenum@cols = 0\relax
      \setcounter{num@cols}{1}%
    \fi
    \set@hlfcntr{tbl@points}{0}%
    \set@hlfcntr{tbl@bonuspoints}{0}%
    \if@bonus
      % Horizontal bonus grade or point table, by pages:
      \begin{tabular}{|l|*{\thenum@cols}{c|}c|}
        \hline
        {\@bhpgword}& \setcounter{@iterator}{\tbl@firstp}%
                      \addtocounter{@iterator}{-1}\do@bpgnumloop
        {\@bhtword}\\
        \hline
        {\@bhpword}& \setcounter{@iterator}{\tbl@firstp}%
                     \addtocounter{@iterator}{-1}\do@bpgptloop
        \prt@tablebonuspoints\\
        \hline
        % If it's a grade table, do the line for scores:
        \if@scores
          {\@bhsword}& \setcounter{@iterator}{\tbl@firstp}%
                       \addtocounter{@iterator}{-1}\do@bpgsloop
          \\
          \hline
        \fi
      \end{tabular}%
    \else
      \if@combined
        % Horizontal combined grade or point table, by pages:
        \begin{tabular}{|l|*{\thenum@cols}{c|}c|}
          \hline
          {\@chpgword}& \setcounter{@iterator}{\tbl@firstp}%
                        \addtocounter{@iterator}{-1}\do@cpgnumloop
          {\@chtword}\\
          \hline
          {\@chpword}& \setcounter{@iterator}{\tbl@firstp}%
                       \addtocounter{@iterator}{-1}\do@cpgptloop
          \prt@tablepoints\\
          \hline
          {\@chbpword}& \setcounter{@iterator}{\tbl@firstp}%
                       \addtocounter{@iterator}{-1}\do@cbpgptloop
          \prt@tablebonuspoints\\
          \hline
          % If it's a grade table, do the line for scores:
          \if@scores
            {\@chsword}& \setcounter{@iterator}{\tbl@firstp}%
                         \addtocounter{@iterator}{-1}\do@cpgsloop
            \\
            \hline
          \fi
        \end{tabular}%
      \else
        % Horizontal non-bonus grade or point table, by pages:
        \begin{tabular}{|l|*{\thenum@cols}{c|}c|}
          \hline
          {\@hpgword}& \setcounter{@iterator}{\tbl@firstp}%
                       \addtocounter{@iterator}{-1}\do@pgnumloop
          {\@htword}\\
          \hline
          {\@hpword}& \setcounter{@iterator}{\tbl@firstp}%
                      \addtocounter{@iterator}{-1}\do@pgptloop
          \prt@tablepoints\\
          \hline
          % If it's a grade table, do the line for scores:
          \if@scores
            {\@hsword}& \setcounter{@iterator}{\tbl@firstp}%
                        \addtocounter{@iterator}{-1}\do@pgsloop
            \\
            \hline
          \fi
        \end{tabular}%
      \fi
    \fi
  \endgroup
}% @htblpages

%--------------------------------------------------------------------
% \do@pgnumloop, \do@pgptloop, and \do@pgsloop
%  are only for non-bonus gradetable.
% \pg@line is also for combinedgradetable.
\def\do@pgnumloop{%
  \addtocounter{@iterator}{1}%
  \set@hlfcntr{tmp@hlfcntr}{\pointsonpage{\the@iterator}}%
  \ifhlfcntr@pos{tmp@hlfcntr}%
    \pg@line
  \fi
  \ifnum \the@iterator < \tbl@lastp\relax
    \let\next@pgnumloop=\do@pgnumloop
  \else
    \let\next@pgnumloop=\relax
  \fi
  \next@pgnumloop
}% \do@pgnumloop
\def\pg@line{%
  % We still don't know why we need to hide this inside of a macro:
  \pageref{firstpoints@onpage@\arabic{@iterator}}&
}% \pg@line

\def\do@pgptloop{%
  \addtocounter{@iterator}{1}%
  \set@hlfcntr{tmp@hlfcntr}{\pointsonpage{\the@iterator}}%
  \ifhlfcntr@pos{tmp@hlfcntr}%
    \add@hlfcntrtohlfcntr{tbl@points}{tmp@hlfcntr}%
    \pgpt@entry
  \fi
  \ifnum \the@iterator < \tbl@lastp\relax
    \let\next@pgptloop=\do@pgptloop
  \else
    \let\next@pgptloop=\relax
  \fi
  \next@pgptloop
}% \do@pgptloop
% \pgpt@entry is used by different macros:
\def\pgpt@entry{%
  % We still don't know why we need to hide this inside of a macro:
  \pointsonpage{\the@iterator}&
}% \pgpt@entry

\def\do@pgsloop{%
  \addtocounter{@iterator}{1}%
  \set@hlfcntr{tmp@hlfcntr}{\pointsonpage{\the@iterator}}%
  \ifhlfcntr@pos{tmp@hlfcntr}%
    \pg@sentry
  \fi
  \ifnum \the@iterator < \tbl@lastp\relax
    \let\next@pgsloop=\do@pgsloop
  \else
    \let\next@pgsloop=\relax
  \fi
  \next@pgsloop
}% \do@pgsloop
% \pg@sentry is called by several different macros,
% to create different tables:
\def\pg@sentry{%
  % We still don't know why we need to hide this inside of a macro:
  \hbox to \@cellwidth{\hfill}&
}% \pg@sentry

%--------------------------------------------------------------------
% These things are for bonusgradetable:

% \do@bpgnumloop, \do@bpgptloop, \do@bpgsloop,
% and \bpg@sline are only for bonusgradetable.
% \bpg@line is also used for combinedgradetable.
\def\do@bpgnumloop{%
  \addtocounter{@iterator}{1}%
  \set@hlfcntr{tmp@hlfcntr}{\bonuspointsonpage{\the@iterator}}%
  \ifhlfcntr@pos{tmp@hlfcntr}%
    \bpg@line
  \fi
  \ifnum \the@iterator < \tbl@lastp\relax
    \let\next@bpgnumloop=\do@bpgnumloop
  \else
    \let\next@bpgnumloop=\relax
  \fi
  \next@bpgnumloop
}% do@bpgnumloop
\def\bpg@line{%
  % We still don't know why we need to hide this inside of a macro:
  \pageref{firstbonuspoints@onpage@\arabic{@iterator}}&
}% bpg@line

\def\do@bpgptloop{%
  \addtocounter{@iterator}{1}%
  \set@hlfcntr{tmp@hlfcntr}{\bonuspointsonpage{\the@iterator}}%
  \ifhlfcntr@pos{tmp@hlfcntr}%
    \add@hlfcntrtohlfcntr{tbl@bonuspoints}{tmp@hlfcntr}%
    \bpgpt@line
  \fi
  \ifnum \the@iterator < \tbl@lastp\relax
    \let\next@bpgptloop=\do@bpgptloop
  \else
    \let\next@bpgptloop=\relax
  \fi
  \next@bpgptloop
}% do@bpgptloop
% \bpgpt@line is called by several macros:
\def\bpgpt@line{%
  % We still don't know why we need to hide this inside of a macro:
  \bonuspointsonpage{\the@iterator}&
}% bpgpt@line

\def\do@bpgsloop{%
  \addtocounter{@iterator}{1}%
  \set@hlfcntr{tmp@hlfcntr}{\bonuspointsonpage{\the@iterator}}%
  \ifhlfcntr@pos{tmp@hlfcntr}%
    \pg@sentry
  \fi
  \ifnum \the@iterator < \tbl@lastp\relax
    \let\next@bpgsloop=\do@bpgsloop
  \else
    \let\next@bpgsloop=\relax
  \fi
  \next@bpgsloop
}% do@bpgsloop

%--------------------------------------------------------------------
% These things are for combinedgradetable:

% \do@cpgnumloop, \cpg@line, \cbpg@line, \do@cpgptloop, \bpgpt@line,
% \do@cpgsloop, \bpg@sline, and \do@cbpgptloop are only for
% combinedgradetable:
\def\do@cpgnumloop{%
  \addtocounter{@iterator}{1}%
  \set@hlfcntr{tmp@hlfcntr}{\pointsonpage{\the@iterator}}%
  \ifhlfcntr@pos{tmp@hlfcntr}%
    \pg@line
  \else
    \check@bpgpts
  \fi
  \ifnum \the@iterator < \tbl@lastp\relax
    \let\next@cpgnumloop=\do@cpgnumloop
  \else
    \let\next@cpgnumloop=\relax
  \fi
  \next@cpgnumloop
}% do@bpgnumloop
\def\check@bpgpts{%
  \set@hlfcntr{tmp@hlfcntr}{\bonuspointsonpage{\the@iterator}}%
  \ifhlfcntr@pos{tmp@hlfcntr}%
    \bpg@line
  \fi
}% check@bpgpts

\def\do@cpgptloop{%
  \addtocounter{@iterator}{1}%
  \set@hlfcntr{tmp@hlfcntr}{\pointsonpage{\the@iterator}}%
  \ifhlfcntr@pos{tmp@hlfcntr}%
    \add@hlfcntrtohlfcntr{tbl@points}{tmp@hlfcntr}%
    \pgpt@entry
  \else
    \check@cbpts
  \fi
  \ifnum \the@iterator < \tbl@lastp\relax
    \let\next@cpgptloop=\do@cpgptloop
  \else
    \let\next@cpgptloop=\relax
  \fi
  \next@cpgptloop
}% do@cpgptloop
\def\check@cbpts{%
  \set@hlfcntr{tmp@hlfcntr}{\bonuspointsonpage{\the@iterator}}%
  \ifhlfcntr@pos{tmp@hlfcntr}%
    \pgpt@entry
  \fi
}% check@cbpts

\def\do@cbpgptloop{%
  \addtocounter{@iterator}{1}%
  \set@hlfcntr{tmp@hlfcntr}{\bonuspointsonpage{\the@iterator}}%
  \ifhlfcntr@pos{tmp@hlfcntr}%
    \add@hlfcntrtohlfcntr{tbl@bonuspoints}{tmp@hlfcntr}%
    \bpgpt@line
  \else
    \check@cnonpts
  \fi
  \ifnum \the@iterator < \tbl@lastp\relax
    \let\next@cbpgptloop=\do@cbpgptloop
  \else
    \let\next@cbpgptloop=\relax
  \fi
  \next@cbpgptloop
}% do@cbpgptloop
\def\check@cnonpts{%
  \set@hlfcntr{tmp@hlfcntr}{\pointsonpage{\the@iterator}}%
  \ifhlfcntr@pos{tmp@hlfcntr}%
    \bpgpt@line
  \fi
}% check@cnonpts

\def\do@cpgsloop{%
  \addtocounter{@iterator}{1}%
  \set@hlfcntr{tmp@hlfcntr}{\pointsonpage{\the@iterator}}%
  \ifhlfcntr@pos{tmp@hlfcntr}%
    \pg@sentry
  \else
    \check@bnsline
  \fi
  \ifnum \the@iterator < \tbl@lastp\relax
    \let\next@cpgsloop=\do@cpgsloop
  \else
    \let\next@cpgsloop=\relax
  \fi
  \next@cpgsloop
}% do@cpgsloop
\def\check@bnsline{%
  \set@hlfcntr{tmp@hlfcntr}{\bonuspointsonpage{\the@iterator}}%
  \ifhlfcntr@pos{tmp@hlfcntr}%
    \pg@sentry
  \fi
}% check@bnsline

%--------------------------------------------------------------------
% \pointsinrange and \bonuspointsinrange:
% We say either \@bonusfalse or \@bonustrue, and then we check it only
% in \do@countloop:
\def\pointsinrange#1{%
  \@bonusfalse
  \def\tbl@range{#1}%
  \@ifundefined{exam@numpoints}%
    {\mbox{\normalfont\bf ??}}%
    {\read@range}%
}% pointsinrange

\def\bonuspointsinrange#1{%
  \@bonustrue
  \def\tbl@range{#1}%
  \@ifundefined{exam@numpoints}%
    {\mbox{\normalfont\bf ??}}%
    {\read@range}%
}% bonuspointsinrange

\def\bad@range{%
  {\mbox{\normalfont\bf ??}}%
  \ClassWarning{exam}{%
    Grading range `\tbl@range ' not defined.\MessageBreak
    \space\space Run LaTeX again.\MessageBreak
  }%
}% bad@range

\def\read@range{%
  \@ifundefined{range@\tbl@range @firstq}%
  {%
    \bad@range
  }%
  {%
    \@ifundefined{range@\tbl@range @lastq}%
    {%
      \bad@range
    }%
    {%
      \edef\tbl@firstq{\csname range@\tbl@range @firstq\endcsname}%
      \edef\tbl@lastq{\csname range@\tbl@range @lastq\endcsname}%
      % Check that firstq precedes or equals lastq:
      \ifnum \tbl@firstq > \tbl@lastq\relax
        \fbox{Error: Grading Range `\tbl@range ':
          Last question precedes first question.}%
        \ClassError{exam}{%
          In grading range `\tbl@range ',
          the last question\MessageBreak
          \space\space comes before the first question.\MessageBreak
        }{%
          \string\begingradingrange \space must precede
          \string\endgradingrange \space by at
          least one question.\MessageBreak
        }%
      \else
        \count@pointsinrange
      \fi
    }%
  }%
}% read@range

\def\count@pointsinrange{%
  % Used for both \pointsinrange and \bonuspointsinrange:
  \set@hlfcntr{tbl@points}{0}%
  \setcounter{@iterator}{\tbl@firstq}%
  \addtocounter{@iterator}{-1}\do@countloop
  \prt@hlfcntr{tbl@points}%
}% count@pointsinrange
\def\do@countloop{%
  % We check \if@bonus here when needed:
  \addtocounter{@iterator}{1}%
  \if@bonus
    \@ifundefined{bonuspointsofq@\romannumeral \c@@iterator}%
      {}%
      {\addto@hlfcntr{tbl@points}
         {\csname bonuspointsofq@\romannumeral \c@@iterator\endcsname}}%
  \else
    \@ifundefined{pointsofq@\romannumeral \c@@iterator}%
      {}%
      {\addto@hlfcntr{tbl@points}
         {\csname pointsofq@\romannumeral \c@@iterator\endcsname}}%
  \fi
  \ifnum \value{@iterator} < \tbl@lastq\relax
    \let\next@countloop=\do@countloop
  \else
    \let\next@countloop=\relax
  \fi
  \next@countloop
}% do@countloop

%--------------------------------------------------------------------
% \firstqinrange, \lastqinrange, and \numqinrange.

\newcommand{\firstqinrange}[1]{%
  \@ifundefined{range@#1@firstq}%
  {\mbox{\normalfont\bf ??}}%
  {\csname range@#1@firstq\endcsname}%
}% firstqinrange

\newcommand{\lastqinrange}[1]{%
  \@ifundefined{range@#1@lastq}%
  {\mbox{\normalfont\bf ??}}%
  {\csname range@#1@lastq\endcsname}%
}% lastqinrange

\newcommand{\numqinrange}[1]{%
  \@ifundefined{range@#1@firstq}%
  {%
    \mbox{\normalfont\bf ??}%
  }%
  {%
    \@ifundefined{range@#1@lastq}%
    {%
      \mbox{\normalfont\bf ??}%
    }%
    {%
      \setcounter{@iterator}{\csname range@#1@lastq\endcsname}%
      \addtocounter{@iterator}{-\csname range@#1@firstq\endcsname}%
      \stepcounter{@iterator}%
      \arabic{@iterator}%
    }%
  }%
}% numqinrange

%--------------------------------------------------------------------
%--------------------------------------------------------------------
%
%                     ***************************
%                     ** SOLUTION ENVIRONMENTS **
%                     ***************************



% If the documentclass options include ``answers'', then the command
% \printanswerstrue is given at the beginning of the run.

% If the documentclass options include ``noanswers'', then the command
% \printanswersfalse is given at the beginning of the run.

\def\printanswers{\printanswerstrue}
\def\noprintanswers{\printanswersfalse}

% If the documentclass options include ``cancelspace'', then the
% command \cancelspacetrue is given at the beginning of the run.

% If the documentclass options include ``nocancelspace'', then the
% command \cancelspacefalse is given at the beginning of the run.

\def\cancelspace{\cancelspacetrue}
\def\nocancelspace{\cancelspacefalse}

% \if@insolution will be true while we're inside of any of the
% solution environments.  This is used to supress \PgInfo@write and
% \label commands generated if there's a parts (or subparts, or
% subsubparts) environment inside of a solution.  (It won't suppress
% the labels for the question objects, since a question object is
% never a label that's been used before.)
\newif\if@insolution
\@insolutionfalse

\newcommand\SolutionEmphasis[1]{%
  \def\Solution@Emphasis{#1}%
}
\SolutionEmphasis{}

% If printanswers is true, we print the solution using a TheSolution
% environment.  If printanswers is false and cancelspace is false, we
% insert blank vertical space equal to the optional argument (the
% default value of which is 0pt).
\newenvironment{solution}[1][0pt]%
  {%
    \@insolutiontrue % cancelled by the end of the environment
    \@addpointsfalse % cancelled by the end of the environment
    \ifprintanswers
      \begingroup
      \Solution@Emphasis
      \begin{TheSolution}%
    \else
      \ifcancelspace
        % Do nothing
      \else
        \par
        \penalty 0
        \vspace*{#1}%
      \fi
      \setbox\z@\vbox\bgroup
    \fi
  }{%
    \ifprintanswers
      \end{TheSolution}%
      \endgroup
    \else
      \egroup
    \fi
  }%

% If printanswers is true, we print the solution using a TheSolution
% environment.  If printanswers is false and cancelspace is false,
% we insert an empty box of width the current line width and of
% height equal to the optional argument, which can be a length, or
% \fill, or \stretch{number}.  If the optional argument is omitted,
% then the box is entirely omitted when printanswers is false.
\newenvironment{solutionorbox}[1][-1pt]%
  {%
    \@insolutiontrue % cancelled by the end of the environment
    \@addpointsfalse % cancelled by the end of the environment
    \ifprintanswers
      \begingroup
      \Solution@Emphasis
      \begin{TheSolution}%
    \else
      \ifcancelspace
        % Do nothing
      \else
        \par
        % Note: It's important that the following test be
        % ``\ifdim 0pt > #1'' rather than ``\ifdim #1 < 0pt''
        % That's because if the user says
        % ``\begin{solutionorbox}{\stretch{1}}''
        % (or \stretch{anythingelse}), then this will expand to
        % ``\ifdim 0pt > \z@ plus 1fill\relax''.
        % The \ifdim will be ``\ifdim 0pt > \z@'', and we'll have
        % ``plus 1fill\relax'' left over.  This is OK because if the
        % \ifdim is false, that leftover stuff will be ignored,
        % and it will only be true if the user omitted the optional
        % argument, in which case there's no \stretch and thus no
        % left over part.
        % If we said ``\ifdim #1 < 0pt'', then we'd get an error
        % when the user used \stretch, since the leftover stuff
        % would appear when TeX was looking for <, =, or >.
        \ifdim 0pt > #1
          % do nothing
        \else
          \makeemptybox{#1}%
        \fi
      \fi
      \setbox\z@\vbox\bgroup
    \fi
  }{%
    \ifprintanswers
      \end{TheSolution}%
      \endgroup
    \else
      \egroup
    \fi
  }%
  
% If printanswers is true, we print the solution using a TheSolution
% environment.  If printanswers is false and cancelspace is false,
% we insert lined vertical space equal to the optional argument (the
% default value of which is 0pt).
\newenvironment{solutionorlines}[1][0pt]%
  {%
    \@insolutiontrue % cancelled by the end of the environment
    \@addpointsfalse % cancelled by the end of the environment
    \ifprintanswers
      \begingroup
      \Solution@Emphasis
      \begin{TheSolution}%
    \else
      \ifcancelspace
        % Do nothing
      \else
        \par
        \penalty 0
        \fillwithlines{#1}%
      \fi
      \setbox\z@\vbox\bgroup
    \fi
  }{%
    \ifprintanswers
      \end{TheSolution}%
      \endgroup
    \else
      \egroup
    \fi
  }%
  
% If printanswers is true, we print the solution using a TheSolution
% environment.  If printanswers is false and cancelspace is false,
% we insert dotted lined vertical space equal to the optional
% argument (the default value of which is 0pt).
\newenvironment{solutionordottedlines}[1][0pt]%
  {%
    \@insolutiontrue % cancelled by the end of the environment
    \@addpointsfalse % cancelled by the end of the environment
    \ifprintanswers
      \begingroup
      \Solution@Emphasis
      \begin{TheSolution}%
    \else
      \ifcancelspace
        % Do nothing
      \else
        \par
        \penalty 0
        \fillwithdottedlines{#1}%
      \fi
      \setbox\z@\vbox\bgroup
    \fi
  }{%
    \ifprintanswers
      \end{TheSolution}%
      \endgroup
    \else
      \egroup
    \fi
  }%

% If printanswers is true, we print the solution using a TheSolution
% environment.  If printanswers is false and cancelspace is false,
% we insert a grid occupying vertically the optional argument (the
% default value of which is 0pt).
\newenvironment{solutionorgrid}[1][0pt]%
  {%
    \@insolutiontrue % cancelled by the end of the environment
    \@addpointsfalse % cancelled by the end of the environment
    \ifprintanswers
      \begingroup
      \Solution@Emphasis
      \begin{TheSolution}%
    \else
      \ifcancelspace
        % Do nothing
      \else
        \par
        \penalty 0
        \fillwithgrid{#1}%
      \fi
      \setbox\z@\vbox\bgroup
    \fi
  }{%
    \ifprintanswers
      \end{TheSolution}%
      \endgroup
    \else
      \egroup
    \fi
  }%


% The environment TheSolution is called from the solution,
% solutionorbox, solutionorlines, solutionordottedlines, and
% solutionorgrid environments when printanswers is true.  It uses
% Donald Arseneau's framed.sty macros (included at the end of this
% file) to allow the solution to be broken across pages and have each
% piece enclosed in an fbox (or a colorbox, if the user has given the
% command \shadedsolutions), (or no box at all, if the user has given
% the command \unframedsolutions).

% Of course, the user can change TheSolution with a \renewenvironment
% command.
\newcommand{\solutiontitle}{\noindent\textbf{Solution:}\enspace}
\newenvironment{TheSolution}%
  {%
    \vspace{\parskip}%
    % If we don't set \leftskip and \rightskip to 0pt, then if we
    % appear inside of an \uplevel command we'd have indentation
    % inside of the solution box:
    \leftskip=0pt
    \rightskip=0pt
    % If the user said \unframedsolutions, then both
    % \if@framedsolutions and \if@shadedsolutions are false:
    \if@framedsolutions
      % We'll use the default \FrameCommand
    \else
      \if@shadedsolutions
        \def\FrameCommand{\colorbox{SolutionColor}}%
      \else
        % It's \unframedsolutions:
        \def\FrameCommand{}%
      \fi
    \fi
    \MakeFramed{\advance\hsize-\width}%
    \solutiontitle
    \ignorespaces
  }%
  {%
    \unskip
    \endMakeFramed
  }%

\newif\if@framedsolutions
\@framedsolutionstrue
\newif\if@shadedsolutions
\@shadedsolutionsfalse
% If the user said \unframedsolutions, then both
% \if@framedsolutions and \if@shadedsolutions are false.

\def\framedsolutions{\@framedsolutionstrue\@shadedsolutionsfalse}
\def\shadedsolutions{%
  \@ifundefined{definecolor}
  {%
    \ClassError{exam}{%
      You must load the color package with the command\MessageBreak
      \space\space\protect\usepackage{color}\MessageBreak
      in order to use the command \protect\shadedsolutions
      \MessageBreak
      }{%
      This command makes use of the package color.sty,\MessageBreak
      and so you have to load color.sty before your\MessageBreak
      \protect\begin{document} command.\MessageBreak
      }%
  }%
  {%
    \definecolor{SolutionColor}{gray}{0.8}
    \@shadedsolutionstrue
    \@framedsolutionsfalse
  }%
}
\def\unframedsolutions{\@framedsolutionsfalse\@shadedsolutionsfalse}


% The solutionbox environment is different from the other solution
% environments (solution, solutionorbox, solutionorlines,
% solutionordottedlines, and solutionorgrid), in that
%
%   (1) The box is always printed, whether answers are being printed
%   or not.
%
%   (2) The argument giving the size of the box is a required
%   argument, not an optional argument, and so it should be enclosed
%   in braces, not in brackets.  It can be either a length or
%   \stretch{number}.
%
%   (3) We make no use of the TheSolution environment; the solutionbox
%   environment is completely freestanding.
%
% If answers are not being printed then only the box is printed, with
% nothing in it.  If answers are being printed, then the solution is
% printed inside of the box.
%
% Note: It's the user's responsibility to be sure that the box is
% large enough to hold the solution!  If the solution takes up too
% much vertical space, then it will spill out of the bottom of the
% box, overwriting whatever follows the box.

\newbox\exam@box
\newenvironment{solutionbox}[1]{%
  \@insolutiontrue % cancelled by the end of the environment
  \@addpointsfalse % cancelled by the end of the environment
  \def\solutionbox@size{#1}% saved for end of environment
  \@tempdima=\textwidth
  \advance\@tempdima -\@totalleftmargin
  \advance\@tempdima -6\fboxsep
  \advance\@tempdima -2\fboxrule
  % Confine the \Solution@Emphasis, as well as anything the user puts
  % into the solution (e.g., \color{red}, or whatever); don't say
  % \endgroup until after using \box\exam@box:
  \begingroup 
    \Solution@Emphasis
    % We save the solution in a box of the proper width.  We'll either
    % print it (if we're printing solutions) or throw it away by just
    % not using it before the environment ends:
    \setbox\exam@box=\vtop\bgroup
      \hsize=\@tempdima
      \leftskip=0pt
      \rightskip=0pt
      \vskip 2\fboxsep
      \solutiontitle
      \ignorespaces
  }%
  {%
    \unskip
    \egroup
    % OK, the solution is now inside \box\exam@box.
    % Set the height and depth to 0pt, so that if we use it we won't
    % be advancing our position on the page:
    \ht\exam@box=0pt
    \dp\exam@box=0pt
    \par
    \vspace{\parskip}
    \ifprintanswers
      % We enclose the \vtop in an \hbox to avoid having the
      % indentation of the enclosing list environment (implemented via
      % \parshape) shift us to the right when we enter horizontal
      % mode.  If we don't use this \hbox, then we'd have to comment
      % out the \hskip \@totalleftmargin:
      \hbox to \textwidth{%
        \noindent
        \hskip\@totalleftmargin
        \hskip3\fboxsep\hskip\fboxrule
        \box\exam@box\hfill
      }%
      \par\nointerlineskip
    \fi
  \endgroup % Finish confining the \Solution@Emphasis
  \makeemptybox{\solutionbox@size}%  
  }







%--------------------------------------------------------------------
%--------------------------------------------------------------------

% The following stuff is lifted from:
%
% framed.sty   v 0.8a   21-Jul-2003
% Copyright (C) 1992-2003 by Donald Arseneau
% These macros may be freely transmitted, reproduced, or modified
% provided that this notice is left intact.
%
% The modifications I made are marked with ``psh'' in a comment:
% 
%%%%%%%%%%%%%%%%%%%%%%%%%%%%%%%%%%%%%%%%%%%%%%%%%%%%%%%%%%%%%%%%%%%%%%
% Create framed or shaded regions that can break across pages using 
% \begin{framed} ... \end{framed}    -- ordinary frame box (box at margin)
% \begin{shaded} ... \end{shaded}    -- shaded background (into margin)
%    ... leftbar ...                 -- line on left side
% \begin{MakeFramed}{settings} ... \end{MakeFramed}
%                        -- generic frame (for new environments)
%
% The "framed" environment puts the text into "\fbox" with the
% settings "\fboxrule=\FrameRule" and "\fboxsep=\FrameSep".
% You can change these lengths (using "\setlength") and you
% can even change the definition of "\FrameCommand" to use
% much fancier boxes.
%
% In fact, the "shaded" environment just redefines "\FrameCommand"
% to use "\colorbox{shadecolor}" (and you have to define the
% color "shadecolor": \newcolor{shadecolor}...).
%
% A page break is allowed, and even encouraged, before the framed
% environment.  If you want to attach some text (a box title) to the
% frame, then the text should be inserted by \FrameCommand
%
% The contents of the framed regions are restricted: 
% Floats, footnotes, marginpars and head-line entries will be lost.
% (Some of these may be handled in a later version.)
% This package will not work with the page breaking of multicol.sty,
% or other systems that perform column-balancing.
%
% The MakeFramed environment does the work.  Its "settings" argument
% should contain any adjustments to the text width (applied to \hsize,
% and using the "\width" of the frame itself) as well as a `restore' 
% command -- \@parboxrestore or \FrameRestore or something similar.
% 
% Expert commands:
% \MakeFramed, \endMakeFramed: the "MakeFramed" environment
% \FrameCommand: command to draw the frame around its argument
% \FrameRestore: restore some text settings, but fewer than \@parboxrestore
% \FrameRule: length register; \fboxrule for default "framed".
% \FrameSep: length register; \fboxsep for default "framed".
% \FrameHeightAdjust: macro; height of frame above baseline at top of page
% 
% This is still a `pre-production' version because I can think of many
% features/improvements that should be made.  Nevertheless, starting 
% with version 0.5 it should be bug-free.
%
%%%%%%%%%%%%%%%%%%%%%%%%%%%%%%%%%%%%%%%%%%%%%%%%%%%%%%%%%%%%%%%%%%%%%%

%psh: Commented out \ProvidesPackage:
%\ProvidesPackage{framed}[2003/07/21 v 0.8a: 
%   framed or shaded text with page breaks]

%psh: Created \saved@totalleftmargin and \@sollistdepth:
\newdimen\saved@totalleftmargin
\newcount\@sollistdepth

\newenvironment{framed}% using default \FrameCommand
  {\MakeFramed {\advance\hsize-\width \FrameRestore}}%
  {\endMakeFramed}

\newenvironment{shaded}{%
  \def\FrameCommand{\colorbox{shadecolor}}%
  \MakeFramed {\FrameRestore}}%
 {\endMakeFramed}

\newenvironment{leftbar}{%
  \def\FrameCommand{\vrule width 3pt \hspace{10pt}}%
  \MakeFramed {\advance\hsize-\width \FrameRestore}}%
 {\endMakeFramed}

\chardef\FrameRestore=\catcode`\| % for debug
\catcode`\|=\catcode`\% % (debug: insert space after backslash)

\def\MakeFramed#1{\par
 % measure added width and height; call result \width and \height
 \setbox\z@\vbox{\vskip-1in \hbox{\hskip-1in 
   \FrameCommand{\hbox{\vrule \@height .7in \@depth.3in \@width 1in}}}%
   \vskip\z@skip}%
 \def\width{\wd\z@}\def\height{\ht\z@}%
 \edef\fb@frw{\the\width}\edef\fb@frh{\the\height}%
 % insert pre-penalties and skips
 \begingroup
 \skip@\lastskip
 \if@nobreak\else 
    \penalty9999 % updates \page parameters
    \ifdim\pagefilstretch=\z@ \ifdim\pagefillstretch=\z@
       \edef\@tempa{\the\skip@}%
       \ifx\@tempa\zero@glue \penalty-30
       \else \vskip-\skip@ \penalty-30 \vskip\skip@
    \fi\fi\fi
    \penalty\z@
    % Give a stretchy breakpoint that will always be taken in preference
    % to the \penalty 9999 used to update page parameters.  The cube root
    % of 10000/100 indicates a multiplier of 0.21545, but the maximum 
    % calculated badness is really 8192, not 10000, so the multiplier
    % is 0.2301. 
    \advance\skip@ \z@ plus-.5\baselineskip
    \advance\skip@ \z@ plus-.231\height
    \advance\skip@ \z@ plus-.231\skip@
    \advance\skip@ \z@ plus-.231\topsep
    \vskip-\skip@ \penalty 1800 \vskip\skip@
 \fi
 \addvspace{\topsep}%
 \endgroup
 % clear out pending page break
 \penalty\@M \vskip 2\baselineskip \vskip\height
 \penalty9999 \vskip -2\baselineskip \vskip-\height
 \penalty9999 % updates \pagetotal
|\message{After clearout, \pagetotal=\the\pagetotal, \pagegoal=\the\pagegoal. }%
 \fb@adjheight 
%psh: Added commands:
  \advance\hsize-\@totalleftmargin
  \saved@totalleftmargin=\@totalleftmargin
  \@totalleftmargin=0pt
  \parshape 0
  \let\@listdepth=\@sollistdepth
  \@sollistdepth=0
  \leftmargin=0pt
%psh: end of added commands
 \setbox\@tempboxa\vbox\bgroup
   #1% Modifications to \hsize (can use \width and \height)
   \textwidth\hsize \columnwidth\hsize
%psh: added one line:
   \linewidth=\hsize
}

\def\endMakeFramed{\par
     \kern\z@ \penalty-100 % put depth into height
 \egroup
 \begingroup \put@frame \endgroup
%psh: Added one line:
 \@totalleftmargin=\saved@totalleftmargin
}

% \put@frame takes the contents of \@tempboxa and puts all, or a piece,
% of it on the page with a frame (\FrameCommand).  It recurses until
% all of \@tempboxa has been used up. (\@tempboxa must have zero depth.)

\def\put@frame{\relax
 \ifdim\pagegoal=\maxdimen \pagegoal\vsize \fi
|   \message{=============== Entering putframe ====================^^J
|     \pagegoal=\the\pagegoal,  \pagetotal=\the\pagetotal. }%
 \ifinner \else
    \dimen@\pagegoal \advance\dimen@-\pagetotal % natural space left on page
  \ifdim\dimen@<2\baselineskip 
|   \message{Page has only \the\dimen@\space room left; eject. }%
    \eject \fb@adjheight \put@frame
  \else % there's appreciable room left on the page
|    \message{\string\pagetotal=\the\pagetotal,
|        \string\pagegoal=\the\pagegoal, 
|        \string\pagestretch=\the\pagestretch,
|        \string\pageshrink=\the\pageshrink,
|        \string\fb@frh=\fb@frh. \space}
|    \message{Box of size \the\ht\@tempboxa\space + \fb@frh}%
     \begingroup % temporarily set \dimen@ to be...
     \advance\dimen@.8\pageshrink  % maximum space available on page
     \advance\dimen@-\fb@frh\relax % space available for frame's contents
     \expandafter\endgroup
     % restore \dimen@ to real room left on page
     \ifdim\dimen@>\ht\@tempboxa % whole box does fit
|       \message{fits in \the\dimen@. }%
     \else % box must be split
|       \message{must be split to fit in \the\dimen@. }%
        \setbox\@tempboxa\vbox{% simulate frame and flexiblity of the page:
           \vskip \fb@frh \@plus\pagestretch \@minus.8\pageshrink
           \kern137sp\kern-137sp\penalty-30
           \unvbox\@tempboxa}%
        \edef\fb@resto@set{\boxmaxdepth\the\boxmaxdepth \splittopskip\the\splittopskip}%
        \boxmaxdepth\z@ \splittopskip\z@
        \setbox\tw@\vsplit\@tempboxa to\dimen@
        \setbox\tw@\vbox{\unvbox\tw@}% natural-sized
|       \message{Box of size \the\ht\@tempboxa\space split to \the\dimen@. 
|          Natural height of split box is \the\ht\tw@. }%
        % If the split-to size > (\vsize-\topskip), then set box to full size
        \begingroup
          \advance\dimen@\topskip
          \expandafter\endgroup
        \ifdim\dimen@>\pagegoal
|         \message{Frame is big -- Use up the full column. }%
          \dimen@ii\pagegoal
          \advance\dimen@ii -\topskip
          \advance\dimen@ii \FrameHeightAdjust\relax
        \else  % suspect this is wrong:
          % If the split-to size > feasible room_on_page, rebox it smaller.
          \advance\dimen@.8\pageshrink
          \ifdim\ht\tw@>\dimen@
|           \message{Box too tall; rebox it to \the\dimen@. }%
            \dimen@ii\dimen@
          \else % use natural size
            \dimen@ii\ht\tw@
          \fi
        \fi
        % Re-box contents to desired size \dimen@ii
        \advance\dimen@ii -\fb@frh
        \setbox\tw@\vbox to\dimen@ii \bgroup
        % remove simulated frame and page flexibility:
        \vskip -\fb@frh \@plus-\pagestretch \@minus-.8\pageshrink
        \unvbox\tw@ \unpenalty\unpenalty
        \ifdim\lastkern=-137sp % whole box went to next page
|          \message{box split at beginning! }%
           \egroup \fb@resto@set \eject % (\vskip for frame size was discarded) 
           \fb@adjheight
        \else %
           \egroup \fb@resto@set
           \ifvoid\@tempboxa % it all fit after all
|             \message{box split at end! }%
              \setbox\@tempboxa\box\tw@
           \else % it really did split
|             \message{box split as expected. Its reboxed height is \the\ht\tw@. }%
              \ifdim\wd\tw@>\z@
%psh: Changed the command that inserts the box:
%     Instead of \centerline, we shift right by \saved@totalleftmargin:
%              \centerline{\FrameCommand{\box\tw@}}%  ??? \centerline bad idea
       \hbox{\hskip \saved@totalleftmargin\FrameCommand{\box\tw@}}%
              \else
|               \message{Zero width means likely blank. Don't frame it (guess)}%
                \box\tw@
              \fi
              \hrule \@height\z@
              \eject
              \fb@adjheight
              \put@frame
  \fi\fi\fi\fi\fi
  \ifvoid\@tempboxa\else
%psh: Changed the command that inserts the box:
%     Instead of \centerline, we shift right by \saved@totalleftmargin:
%    \centerline{\FrameCommand{\box\@tempboxa}}%
    \hbox{\hskip\saved@totalleftmargin\FrameCommand{\box\@tempboxa}}%
    \nointerlineskip \null %{\showoutput \showlists}
    \penalty-30 \vskip\topsep
  \fi}

\def\fb@adjheight{%
  \vbox to\FrameHeightAdjust{}% get proper baseline skip from above.
  \penalty\@M \nointerlineskip
  \vskip-\FrameHeightAdjust
  \penalty\@M} % useful for tops of pages

\edef\zero@glue{\the\z@skip}

\catcode`\|=\FrameRestore

% Provide configuration commands:
\providecommand\FrameCommand{\fboxrule=\FrameRule \fboxsep=\FrameSep \fbox}
\@ifundefined{FrameRule}{\newdimen\FrameRule \FrameRule=\fboxrule}{}
\@ifundefined{FrameSep} {\newdimen\FrameSep  \FrameSep =3\fboxsep}{}

% Height of frame above first baseline when frame starts a page:
\providecommand\FrameHeightAdjust{6pt}

% \FrameRestore has parts of \@parboxrestore.  See how it is used in the 
% "settings" argument of \MakeFrame.  Previous behavior can be restored by 
% using \@parboxrestore there, or redefining:
% \makeatletter \renewcommand\FrameRestore{\@parboxrestore} \makeatother
\def\FrameRestore{%
  \let\if@nobreak\iffalse
  \let\if@noskipsec\iffalse
%  \let\par\@@par  ??
  \let\-\@dischyph
  \let\'\@acci\let\`\@accii\let\=\@acciii
%  \parindent\z@ \parskip\z@skip    Definitely omit!
%  \everypar{}%  ??
  \linewidth\hsize
%  \@totalleftmargin\z@
%  \leftskip\z@skip \rightskip\z@skip \@rightskip\z@skip
%  \parfillskip\@flushglue \lineskip\normallineskip
%  \baselineskip\normalbaselineskip
  \sloppy
%  \let\\\@normalcr
}

%  Compatibility with previous versions (temporary!):
\let\fram@d=\MakeFramed  \let\endfram@d=\endMakeFramed

%%%%%%%%%%%%%%%%%%%%%%%%%%%%%%%%%%%%%%%%%%%%%%%%%%%%%%%%%%%%%%%%%%%%%%

% This ends the stuff that's lifted from:
% % framed.sty   v 0.8a   21-Jul-2003
% % Copyright (C) 1992-2003 by Donald Arseneau


%--------------------------------------------------------------------
%--------------------------------------------------------------------

\endinput
%---------------------------------------------------------------------
%---------------------------------------------------------------------
%---------------------------------------------------------------------
%---------------------------------------------------------------------
%---------------------------------------------------------------------
%
  \let\tableofcontents=\@oldtableofcontents
  \let\maketitle=\old@maketitle
\makeatother
\noexamples
\ProvidesFile{exam.dtx}[1997/03/14 3.30  Slides and notes]
\GetFileInfo{exam.dtx}
\title{The \textsf{exam} package%
\thanks{This file has version \fileversion\space dated \filedate.}}
\author{Hans van der Meer\\hansm@wins.uva.nl}
\date{Printed \today}
\CodelineNumbered
\DisableCrossrefs
\RecordChanges
\begin{document}
\maketitle
\DocInput{exam.dtx}
\end{document}
%</driver>
%    \end{macrocode}
%\fi
%
% %%%%%%%%%%%%%%%%%%%%%%%%%%%%%%%%%%%%%%%%%%%%%%%%%%%%%%%%%%%%%%%%%%%%
%
% \changes{3.00}{1994/02/13}{First version for LaTeX2E and docstrip}
% \changes{3.01}{1994/03/24}{added mbox{} to Copyright (missing item error)}
% \changes{3.10}{1994/10/19}{updated several features}
% \changes{3.11}{1994/10/21}{added dumpitemno and ignorespace in SRset}
% \changes{3.12}{1994/10/25}{changed pagenumbering index}
% \changes{3.13}{1994/11/10}{help shows class options}
% \changes{3.14}{1994/11/24}{empty default for mainfolder, etc.}
% \changes{3.15}{1994/12/10}{default language initialization added}
% \changes{3.16}{1995/01/19}{require latest latex because of box trouble}
% \changes{3.17}{1995/01/26}{maketitle redefinition better in exam.cfg}
% \changes{3.18}{1995/02/02}{options, documentation, checksquare->boxtimes}
% \changes{3.19}{1995/03/10}{pagestyle tuned, added options}
% \changes{3.20}{1995/07/12}{table of contents problem solved}
% \changes{3.21}{1995/08/07}{writing toc-entry in question displaced}
% \changes{3.22}{1995/08/09}{folders default set to @currdir}
% \changes{3.23}{1995/10/26}{help standard, textbo removed, small changes}
% \changes{3.24}{1995/10/30}{ignorespaces added to longanswer start}
% \changes{3.25}{1995/12/17}{error in altanswer repaired}
% \changes{3.26}{1996/08/10}{cleaning, fixing loose ends, simplified where possible}
% \changes{3.27}{1996/08/19}{dir path separator changed in folder macros}
% \changes{3.28}{1996/08/23}{shortanswer changed, documentation polished}
% \changes{3.29}{1997/03/08}{directory macros changed}
% \changes{3.30}{1997/03/14}{CurrentDirectory for @currdir added}
%
% %%%%%%%%%%%%%%%%%%%%%%%%%%%%%%%%%%%%%%%%%%%%%%%%%%%%%%%%%%%%%%%%%%%%
%
% \DoNotIndex{\#}
% \DoNotIndex{\@@input}
% \DoNotIndex{\@dblarg}
% \DoNotIndex{\@depth}
% \DoNotIndex{\@empty}
% \DoNotIndex{\@firstoftwo}
% \DoNotIndex{\@height}
% \DoNotIndex{\@ifstar}
% \DoNotIndex{\@ifundefined}
% \DoNotIndex{\@m}
% \DoNotIndex{\@makeother}
% \DoNotIndex{\@namedef}
% \DoNotIndex{\@ne}
% \DoNotIndex{\@sanitize}
% \DoNotIndex{\@secondoftwo}
% \DoNotIndex{\@tempdima}
% \DoNotIndex{\@tempdimb}
% \DoNotIndex{\@title}
% \DoNotIndex{\@warning}
% \DoNotIndex{\@whilenum}
% \DoNotIndex{\@width}
% \DoNotIndex{\\}
% \DoNotIndex{\{}
% \DoNotIndex{\}}
% \DoNotIndex{\^}
% \DoNotIndex{\ }
% \DoNotIndex{\addtolength}
% \DoNotIndex{\addvspace}
% \DoNotIndex{\advance}
% \DoNotIndex{\begin}
% \DoNotIndex{\begingroup}
% \DoNotIndex{\bgroup}
% \DoNotIndex{\bigskip}
% \DoNotIndex{\bigskipamount}
% \DoNotIndex{\box}
% \DoNotIndex{\catcode}
% \DoNotIndex{\count@}
% \DoNotIndex{\csname}
% \DoNotIndex{\def}
% \DoNotIndex{\dimen@}
% \DoNotIndex{\divide}
% \DoNotIndex{\do}
% \DoNotIndex{\dotfill}
% \DoNotIndex{\dp}
% \DoNotIndex{\edef}
% \DoNotIndex{\egroup}
% \DoNotIndex{\else}
% \DoNotIndex{\emph}
% \DoNotIndex{\end}
% \DoNotIndex{\endcsname}
% \DoNotIndex{\endgroup}
% \DoNotIndex{\endlist}
% \DoNotIndex{\enskip}
% \DoNotIndex{\enspace}
% \DoNotIndex{\ensuremath}
% \DoNotIndex{\expandafter}
% \DoNotIndex{\f@baselineskip}
% \DoNotIndex{\fbox}
% \DoNotIndex{\fi}
% \DoNotIndex{\footnotesize}
% \DoNotIndex{\gdef}
% \DoNotIndex{\global}
% \DoNotIndex{\hbox}
% \DoNotIndex{\hfil}
% \DoNotIndex{\hfill}
% \DoNotIndex{\hrule}
% \DoNotIndex{\hskip}
% \DoNotIndex{\hspace}
% \DoNotIndex{\hss}
% \DoNotIndex{\ht}
% \DoNotIndex{\if}
% \DoNotIndex{\ifcase}
% \DoNotIndex{\ifdim}
% \DoNotIndex{\ifnum}
% \DoNotIndex{\ifodd}
% \DoNotIndex{\ifx}
% \DoNotIndex{\ignorespaces}
% \DoNotIndex{\item}
% \DoNotIndex{\itemsep}
% \DoNotIndex{\leavevmode}
% \DoNotIndex{\let}
% \DoNotIndex{\list}
% \DoNotIndex{\llap}
% \DoNotIndex{\m@ne}
% \DoNotIndex{\makebox}
% \DoNotIndex{\mbox}
% \DoNotIndex{\medskip}
% \DoNotIndex{\medskipamount}
% \DoNotIndex{\multiply}
% \DoNotIndex{\newcommand}
% \DoNotIndex{\newcount}
% \DoNotIndex{\newcounter}
% \DoNotIndex{\newenvironment}
% \DoNotIndex{\newif}
% \DoNotIndex{\newlength}
% \DoNotIndex{\newpage}
% \DoNotIndex{\newsavebox}
% \DoNotIndex{\newtoks}
% \DoNotIndex{\next}
% \DoNotIndex{\noexpand}
% \DoNotIndex{\noindent}
% \DoNotIndex{\null}
% \DoNotIndex{\number}
% \DoNotIndex{\or}
% \DoNotIndex{\p@}
% \DoNotIndex{\par}
% \DoNotIndex{\parbox}
% \DoNotIndex{\parsep}
% \DoNotIndex{\partopsep}
% \DoNotIndex{\phantom}
% \DoNotIndex{\protect}
% \DoNotIndex{\providecommand}
% \DoNotIndex{\raggedright}
% \DoNotIndex{\relax}
% \DoNotIndex{\renewcommand}
% \DoNotIndex{\rlap}
% \DoNotIndex{\rmfamily}
% \DoNotIndex{\romannumeral}
% \DoNotIndex{\selectfont}
% \DoNotIndex{\setbox}
% \DoNotIndex{\setcounter}
% \DoNotIndex{\setlength}
% \DoNotIndex{\settowidth}
% \DoNotIndex{\sffamily}
% \DoNotIndex{\sloppy}
% \DoNotIndex{\small}
% \DoNotIndex{\smallskip}
% \DoNotIndex{\smallskipamount}
% \DoNotIndex{\space}
% \DoNotIndex{\stepcounter}
% \DoNotIndex{\string}
% \DoNotIndex{\strut}
% \DoNotIndex{\test}
% \DoNotIndex{\textbf}
% \DoNotIndex{\textemdash}
% \DoNotIndex{\textsl}
% \DoNotIndex{\texttt}
% \DoNotIndex{\the}
% \DoNotIndex{\thinspace}
% \DoNotIndex{\tw@}
% \DoNotIndex{\topsep}
% \DoNotIndex{\typeout}
% \DoNotIndex{\undefined}
% \DoNotIndex{\underbar}
% \DoNotIndex{\uppercase}
% \DoNotIndex{\upshape}
% \DoNotIndex{\vadjust}
% \DoNotIndex{\value}
% \DoNotIndex{\vbox}
% \DoNotIndex{\vfil}
% \DoNotIndex{\vfill}
% \DoNotIndex{\vskip}
% \DoNotIndex{\vspace}
% \DoNotIndex{\vss}
% \DoNotIndex{\wd}
% \DoNotIndex{\xdef}
% \DoNotIndex{\z@}
% \DoNotIndex{\z@skip}
%
% \begin{abstract}
% This article describes the use and the implementation of the 
% \emph{exam class}.
% Its purpose is the typesetting of exams.
% Exam questions can be multiple choice or free long\slash short
% answer questions.
% Options are the typesetting of the exam itself, an exam
% showing all the answers and a collection of questions and answers.
% Questions can be parametrized.
% Use of a random generator provides for automatic shuffling
% of multiple choice items.
% \end{abstract}
%
% \tableofcontents
%
% \section{Usage}
%
% \subsection{Exam production}
%
% An exam can be built from the following template.
% For special issues as the use of default names for
% various directories, language selection, etc. see
% the implementation section.
%
% You may customize the typesetting by providing
% a file \emph{exam.cfg} in the search path; this file
% is read just before typesetting begins.
% The example of this file gives a language customization
% and an implementation for a title page.
%
% \medskip\noindent
% \DescribeEnv{exam}
% |\documentclass[options]{exam}|\\
% |\title{title of exam}|\\
% |\author{the examinator}|\\
% |\target{the students}|\\
% |\begin{document}|\\
% |\begin{exam}[startvalue random generator]{date of exam}|\\
% |\question{directory}{file}|\\
% |\question[parameter value]{directory}{file}|\\
% |....|\\
% |\end{exam}|\\
% |... % possibly other exams|\\
% |\end{document}|
%
% \subsection{Format of a problem}
%
% \DescribeEnv{problem}
% A problem is built by environment |problem|.
% In it several elements can be placed. These are:
% \begin{enumerate}
% \item \DescribeMacro{\parameterproblem}
% |\parameterproblem{text}|: used to communicate to the
% maintainer of the problems the possibilities offered
% by the transfer of macro |\parameter| on posing
% the question; an example of this will follow.
% Can also find a place before the problem declaration.
% \item |\begin{problem}[#1]|; the optional parameter
% can have the value |\compact| (no pagebreak within problem, default)
% or the value |\split| (pagebreak may occur in problem).
% \item |\problemdate{date}|: a macro to remember on which
% day the problem was born;
% \item \DescribeMacro{\score}
% |\score{value}|: use this macro for the number of
% points the answer is worth; it is possible to include
% several score items in one problem for partial rewards.
% The score value is not shown when an exam is typeset,
% the student must earn these points!
% \item text of the question.
% \item the answer or multiple choice list;
% see the description below.
% \item |\remark[h]{b}|: a boxed remark with heading h and body b.
% \item |\end{problem}|.
% \end{enumerate}
%
% \subsection{Examples of Question and Typesetting}
%
% \subsubsection{Simple Problem}
%
% \begin{center}
% \textbf{problem --- coding}
% \end{center}
% \medskip
% |\begin{problem}|\\
% |\problemdate{\today}|\\
% |What is the question?|\\
% |\score{2}|\\
% |\shortanswer{To be or not to be.}|\\
% |\end{problem}|
%
% \medskip
% \DescribeMacro{\question}
% This problem is called up with\\
% |\question{}{exampa}|\\
% which we show without |answers| option and with both the |answers| and 
% |series| option set.
% \begin{center}
% \textbf{problem --- result --- without answers}
% \end{center}
% \medskip
% \begin{center}
% \begin{minipage}{.9\linewidth}\setlength\linewidth{.8\linewidth}
% \answersfalse
% \MakePercentComment\question{}{exampa}\MakePercentIgnore
% \end{minipage}
% \end{center}
% \medskip
% \begin{center}
% \textbf{problem --- result --- with answers}
% \end{center}
% \medskip
% \begin{center}
% \begin{minipage}{.9\linewidth}\setlength\linewidth{.8\linewidth}
% \answerstrue\seriestrue
% \addtocounter{problemnum}{-1}
% \MakePercentComment\question{}{exampa}\MakePercentIgnore
% \end{minipage}
% \end{center}
%
% \subsubsection{Parametrized Problem}
%
% \DescribeMacro{\parameter}
% The next example shows the use of |\parameter| for the selection
% of alternate questions. It is given both value 1 and 2 and called
% with respectively:\\
% |\question[1]{}{exampb}|\\
% |\question[2]{}{exampb}|\\
% However, remember that |\parameter| can be defined to anything;
% e.g. the number that goes into a calculation, a word substituted
% at a specific place, etc.
%
% \medskip 
% \begin{center}
% \textbf{parameterproblem --- coding}
% \end{center}
% \medskip
% |\parameterproblem{1= to be\\2= not to be}|\\
% |\problemdate{\today}|\\
% |\begin{problem}|\\
% |\score{2}|\\
% |What is\ifnum\parameter=1\relax \else n't\fi\ the question?|\\
% |\shortanswer{\ifnum\parameter=1\relax To be or n\else N\fi ot to be.}|\\
% |\end{problem}|
% \medskip
% \begin{center}
% \textbf{parameterproblem --- result --- parameter = 1}
% \end{center}
% \medskip
% \begin{center}
% \begin{minipage}{.9\linewidth}\setlength\linewidth{.8\linewidth}
% \answerstrue\seriestrue
% \MakePercentComment\question[1]{}{exampb}\MakePercentIgnore
% \end{minipage}
% \end{center}
% \medskip
% \begin{center}
% \textbf{parameterproblem --- result --- parameter = 2}
% \end{center}
% \medskip
% \begin{center}
% \begin{minipage}{.9\linewidth}\setlength\linewidth{.8\linewidth}
% \answerstrue\seriestrue
% \addtocounter{problemnum}{-1}
% \MakePercentComment\question[2]{}{exampb}\MakePercentIgnore
% \end{minipage}
% \end{center}
%
% \subsection{Answers}
% \DescribeMacro{\answer}
% \DescribeMacro{\altanswer}
% The basic macros for showing and suppressing answers are
% |\answer| that shows its argument when the \emph{answers}
% option is chosen, and |\altanswer| that alternates its
% two arguments. Both macros have a first, optional argument
% for specifying the width of the box wherein the text
% is placed.
%
% \medskip 
% \begin{center}
% \textbf{answer --- coding}
% \end{center}
% \medskip
% |\answer{answer}|
%
% \medskip
% \begin{center}
% \textbf{answer --- result --- 
% left without, right with answers}
% \end{center}
% \medskip
% \begin{center}
% \fbox{\parbox[t]{.4\linewidth}{\strut\answersfalse
% \answer{answer}
% }}\qquad\fbox{\parbox[t]{.4\linewidth}{\strut\answerstrue
% \answer{answer}}}
% \end{center}
%
% \medskip 
% \begin{center}
% \textbf{altanswer --- coding}
% \end{center}
% \medskip
% |\altanswer{answer NO}{answer YES}|
%
% \medskip
% \begin{center}
% \textbf{altanswer --- result --- 
% left without, right with answers}
% \end{center}
% \medskip
% \begin{center}
% \fbox{\parbox[t]{.4\linewidth}{\strut\answersfalse
% \altanswer{answer NO}{answer YES}
% }}\qquad\fbox{\parbox[t]{.4\linewidth}{\strut\answerstrue
% \altanswer{answer NO}{answer YES}}}
% \end{center}
%
% \DescribeMacro{\shortanswer}
% Some questions can be answered by a few words, a short sentence.
% The command |\shortanswer| serves this purpose;
% its argument is the answer.
%
% \medskip 
% \begin{center}
% \textbf{short answer --- coding}
% \end{center}
% |\shortanswer{The answer.}|
%
% \medskip
% \begin{center}
% \textbf{short answer --- result --- 
% left without answers, right with answers}
% \end{center}
% \medskip
% \begin{center}
% \renewcommand*\shortwhite{2mm}
% \fbox{\parbox[t]{.4\linewidth}{\answersfalse
% Answer the next question:
% \shortanswer{The answer.}}}
% \qquad
% \fbox{\parbox[t]{.4\linewidth}{\answerstrue
% Answer the next question:
% \shortanswer{The answer.}}}
% \end{center}
%
% \DescribeEnv{longanswer}
% When however more space is needed by the student, the
% environment |longanswer| can be used. 
% This environment has one optional parameter, meant
% for specifying the amount of white space to be reserved
% for the students answer. 
%
% \medskip 
% \begin{center}
% \textbf{long answer --- coding}
% \end{center}
% \medskip
% |\begin{longanswer}[5mm]|\\
% |The answer.|\\
% |\end{longanswer}|
%
% \medskip
% \begin{center}
% \textbf{long answer --- result --- 
% left without answers, right with answers}
% \end{center}
% \medskip
% \begin{center}
% \fbox{\parbox[t]{.4\linewidth}{\answersfalse
% Answer the next question:
% \begin{longanswer}[5mm]
% The answer.
% \end{longanswer}}}
% \qquad
% \fbox{\parbox[t]{.4\linewidth}{\answerstrue
% Answer the next question:
% \begin{longanswer}[5mm]
% The answer.
% \end{longanswer}}}
% \end{center}
%
% \DescribeMacro{\answerstart}
% The answer is headed by a call to |\answerstart|; redefine
% to your taste.
%
% \DescribeMacro{\shortwhite}
% \DescribeMacro{\longwhite}
% The default of white space reserved for the depth of the short answer
% can be changed by redefinition of |\shortwhite|.
% The default for the white space of the long answer
% can be changed by redefinition of |\longwhite|.
%
% \subsection{Multiple Choice}
%
% \DescribeEnv{choice}
% Multiple choice is provided for by environment |choice|.
% Within this environment a itemized list of alternatives is given.
% \DescribeMacro{\baditem}
% \DescribeMacro{\gooditem}
% However instead of |\item| one codes |\baditem{text}| for wrong answers
% and |\gooditem{text}| for the correct one; the answer being put
% into the argument of these two macros.
% \DescribeMacro{\ordered}
% \DescribeMacro{\random}
% The optional parameter of this environment can be |\ordered| for
% production of the alternatives in the order specified, or
% |\random| for randomization; randomize is the default, unless
% the \emph{series} option is specified in the |\documentclass| call.
%
% \medskip 
% \begin{center}
% \textbf{multiple choice example --- coding}
% \end{center}
% \medskip
% |\begin{choice}[\ordered]|\\
% |\baditem{first wrong answer}|\\
% |\gooditem{the right answer}|\\
% |\baditem{second wrong answer}|\\
% |\end{choice}|
%
% \medskip
% \begin{center}
% \textbf{multiple choice --- result --- 
% left without answers, right with answers}
% \end{center}
% \medskip
% \begin{center}
% \fbox{\parbox[t]{.4\linewidth}{\answersfalse
% Choose appropriate alternative:
% \begin{choice}[\ordered]
% \baditem{first wrong answer}
% \gooditem{the right answer}
% \baditem{2nd wrong answer}
% \end{choice}}}
% \qquad
% \fbox{\parbox[t]{.45\linewidth}{\answerstrue
% Choose appropriate alternative:
% \begin{choice}[\ordered]
% \baditem{first wrong answer}
% \gooditem{the right answer}
% \baditem{2nd wrong answer}
% \end{choice}}}
% \end{center}
%
% \DescribeMacro{\badmark}
% \DescribeMacro{\goodmark}
% The marks for the multiple choice items are produced
% by the macros |\badmark| and |\goodmark|. For their
% redefinition see the implementation section of this
% document.
%
% \subsection{Shuffling it Yourself}
%
% \DescribeMacro{\loaditem}
% The selection of alternatives is implemented by the
% mechanism in macros |\loaditem| and friends.
% Pieces text can be loaded (in this implementation at most 5) and
% selectively dumped into the typeset input stream.
% It is a useful mechanism when one has to produce
% a whole series of variations on the same theme.
%
% Macro |\loaditem| can be used to load from one to five
% items in a data store. 
% \DescribeMacro{\shuffle}
% This data store can be shuffled
% by a call to |\shuffle|. 
% \DescribeMacro{\dumpitem}
% \DescribeMacro{\dumpitems}
% Popping items from the store
% is effected by macros |\dumpitem| (pop one item) and
% |\dumpitems| (all items). Clearing of the store
% is done by |\resetloadcounter|.
% \DescribeMacro{\SRtest}
% With |\SRtest{1}{2}| one can make a random choice between two
% alternatives.
%
% \medskip 
% \begin{center}
% \textbf{load and dump --- coding}
% \end{center}
% \medskip
% |\SRset{349}                     % start random generator|\\
% |\resetloadcounter               % initialize load stack|\\
% |\loaditem{\fbox{item 1}\space}  % load 4 items of text|\\
% |\loaditem{\fbox{item 2}\space}|\\
% |\loaditem{\fbox{item 3}\space}|\\
% |\loaditem{\fbox{item 4}\space}|\\
% |Here comes nr~2: \dumpitemno{2} % dump 2nd item|\\
% |\par|\\
% |\shuffle                        % randomize|\\
% |Here nr~2 again after randomization: \dumpitemno{2}|\\
% |\par|\\
% |Dump the whole lot: \dumpitems|
%
% \medskip
% \begin{center}
% \textbf{load and dump --- result}
% \end{center}
% \medskip
% \begin{center}
% \parbox[t]{.8\linewidth}{%
% \SRset{349}
% \resetloadcounter
% \loaditem{\fbox{item 1}\space}
% \loaditem{\fbox{item 2}\space}
% \loaditem{\fbox{item 3}\space}
% \loaditem{\fbox{item 4}\space}
% Here comes nr~2: \dumpitemno{2}\par
% \shuffle
% Nr~2 after randomization: \dumpitemno{2}\par
% Dump the whole lot: \dumpitems
% }
% \end{center}
%
%
% \section{Summary of Options and Macros}
%
% \subsection{Options}
%
% \begin{flushleft}\sloppy
% |answers| problems with answers\\
% |pages| each problen on a separate page\\
% |questiononly| suppress open space for answers\\
% |nosep| suppress separation between successive problems\\
% |scores| add score values\\
% |series| produce problem collection
% \end{flushleft}
%
% \subsection{Exam Production}
%
% \begin{flushleft}\sloppy
% |\target{name}| exam meant for these people\\
% |\begin{exam}[randomstart]{date}...\end{exam}| dated exam\\
% |\begin{exam}{date}...\end{exam}| randomstart=0, i.e. not random\\
% |\begin{exam}{}...\end{exam}| today's date\\
% |\question[variant]{dir}{file}| call problem variant from dir/file\\
% |\question{dir}{file}| no problem variants
% \end{flushleft}
%
% \subsection{Problem Definition}
%
% \begin{flushleft}\sloppy
% |\parameterproblem[param]{explication}| set default, show explication\\
% |\begin{problem}...\end{problem}| problem definition, kept wholly on page\\
% |\begin{problem}[\split]...\end{problem}| do not confine to one page\\
% |\problemdate{date}| reference date for problem\\
% |\score{value}| set problems worth in points\\
% |\remark[label]{text}| place remark if series option\\[\smallskipamount]
% |\answer[width]{answer}| coding in an answer\\
% |\altanswer[width]{alt}{answer}| alternate text instead of answer\\
% |\shortanswer{answer}| short answer or row of dots\\
% |\begin{longanswer}[height]...\end{longanswer}| long answer or space\\[\smallskipamount]
% |\begin{choice}...\end{choice}| randomly permuted multiple choice\\
% |\begin{choice}[\ordered]...\end{choice}| not permuted\\
% |\baditem{text}| inside a |choice| text for false item\\
% |\gooditem{text}| ditto for good item\\[\smallskipamount]
% |\loaditem{anything}| put `anything' to memory stack\\
% |\dumpitem| pop top from memory stack\\
% |\dumpitems| pop whole memory stack\\
% |\shuffle| randomize memory stack
% 
% \end{flushleft}
%
% \subsection{Styling and Customization}
%
% \begin{flushleft}\sloppy
% |\everyproblem| token register executed at start of each problem\\
% |\scoreboxsize{size}| set size of square score box\\
% |\shortwhite| default vertical space short answers, change |\renewcommand|\\
% |\longwhite| ditto for long answers\\
% |\marksize| size of multiple choice marks, |\selectfont| expression\\
% |\badmark| symbol for multiple choice item box\\
% |\goodmark| ditto for the good one\\
% |\headerfont| font for page header\\
% |\bodyfont| font for exam text\\
% |\titlefont| font for titles\\
% |\Mainfolder{dir}| set |\mainfolder| to dir\\
% |\Commonfolder{dir}| set |\commonfolder| to dir\\
% |\Figuresfolder{dir}| set |\figuresfolder| to dir
% \end{flushleft}
%
% \StopEventually{\PrintChanges}
%
% \newpage
%    \begin{macrocode}
%<*cls>
%    \end{macrocode}
%
% \section{Identification}
%
% This document class can only be used with \LaTeXe, so we make
% sure that an appropriate message is displayed when another \TeX{}
% format is used. We require the latest version that has no known
% troubles with this class.
%    \begin{macrocode}
\NeedsTeXFormat{LaTeX2e}[1995/12/01]
%    \end{macrocode}
%
% Announce the Class name and its version.
%    \begin{macrocode}
\ProvidesClass{exam}[1997/03/14 vs 3.30 exam package]
%    \end{macrocode}
%
% \section{Declaration of Class Options}
%
% In this part we define the options for this class that are additional
% to those of its parent class.
%
% \subsection{Switching answers on and off}
%
% \begin{macro}{\ifanswers}
% The flag |\ifanswers| governs the production of answers in the
% typesetting of problems. With the |answers| options in the
% optional argument of the document class this option is turned on.
% Then also we show the score values.
%
%    \begin{macrocode}
\newif\ifanswers
\answersfalse
\DeclareOption{answers}{\answerstrue\scorestrue}
%    \end{macrocode}
% \end{macro}
%
% \subsection{Each problem on separate page}
%
% \begin{macro}{\ifproblempages}
% The flag |\ifproblempages| governs the typesetting of problems 
% on separate pages, or their collection of more than one to a page.
% If separate pages are chosen, a separator between problems
% is unnecessary.
%
%    \begin{macrocode}
\newif\ifproblempages
\problempagesfalse
\DeclareOption{pages}{\problempagestrue\problemsepfalse}
%    \end{macrocode}
% \end{macro}
%
% \subsection{Suppress prompt for answer}
%
% \begin{macro}{\ifreservespace}
% The flag |\ifreservespace| governs the typesetting of space 
% for answers. If true, all reservation of answer space is
% suppressed. It is set by option |questiononly|;
% this option has no effect when the |answers| option is on.
%
%    \begin{macrocode}
\newif\ifreservespace
\reservespacetrue
\DeclareOption{questiononly}{\reservespacefalse}
%    \end{macrocode}
% \end{macro}
%
% \subsection{Visible separation between problems}
%
% \begin{macro}{\ifproblemsep}
% The value of flag |\ifproblemsep| determines the appearance 
% of a visible separation between successive problems.
%
%    \begin{macrocode}
\newif\ifproblemsep
\problemseptrue
\DeclareOption{nosep}{\problemsepfalse}
%    \end{macrocode}
% \end{macro}
%
% \subsection{Showing score values}
%
% \begin{macro}{\ifscores}
% The flag |\ifshowscores| determines when score values are printed.
%
%    \begin{macrocode}
\newif\ifscores
\scoresfalse
\DeclareOption{scores}{\scorestrue}
%    \end{macrocode}
% \end{macro}
%
% \subsection{Typeset a Catalogue of Problems}
%
% \begin{macro}{\ifseries}
% The flag |\ifseries| initiates the production of a problem catalogue.
% In order to show answers and score values, the respective flags are set.
%
%    \begin{macrocode}
\newif\ifseries
\seriesfalse
\DeclareOption{series}{\seriestrue\answerstrue\scorestrue}
%    \end{macrocode}
% \end{macro}
%
% \subsection{Show Options}
%
% Show options to the user with option |help|.
%
%    \begin{macrocode}
\DeclareOption{help}{\ClassWarningNoLine{exam}{%
    available options for exam:\MessageBreak
    answers:\space show questions with answers;\MessageBreak
    nosep:\space no separators between problems;\MessageBreak
    pages:\space each problem on a page;\MessageBreak
    questiononly:\space suppress answer space in exams;\MessageBreak
    scores:\space typeset score values always;\MessageBreak
    series:\space\space typeset catalogue of problems}}
%    \end{macrocode}
%
% \section{Loading of Parent Class}
%
% Since the \emph{exam class} is implemented as a modification
% of an existing document class, we must load the parent class.
% \begin{macro}{\parentclass}
% In order to make changes in parent class easy, the
% name of this class is parametrized in macro |\parentclass|.
% Obvious candidates are \emph{article} and \emph{report}.
% In order to provide some flexibility, we allow for the case
% that the user has already defined |\parentclass| (before
% the call to |\documentclass|. In that case we refrain
% from redefinition.
%
%    \begin{macrocode}
\providecommand\parentclass{article}
%    \end{macrocode}
% \end{macro}
%
% The options of the |\documentclass| call which are not specific for the
% \emph{exam class} must be passed to the parent class.
% We take the opportunity to select the production of a titlepage 
% (not automatically added if the parent class is \emph{article}.
% After this we process the local options.
%
%    \begin{macrocode}
\DeclareOption*{\PassOptionsToClass{\CurrentOption}{\parentclass}}
\PassOptionsToClass{titlepage}{\parentclass}
\ProcessOptions
%    \end{macrocode}
%
% Then we load the parent class.
%
%    \begin{macrocode}
\LoadClass{\parentclass}
%    \end{macrocode}
%
% A table of contents can be produced when a series is run or when
% producing an exam including answers, otherwise kill the
% corresponding macro.
%
%    \begin{macrocode}
\ifseries\else\ifanswers\else\let\tableofcontents=\relax\fi\fi
%    \end{macrocode}
%
% \section{Helpfull Macros}
%
% \begin{macro}{\@ifemptyarg}
% Testing for the presence or absence of a parameter.
%
%    \begin{macrocode}
\providecommand\@ifemptyarg[1]{% {absence}{presence}
  \ifx\@empty#1\@empty
  \expandafter\@firstoftwo\else\expandafter\@secondoftwo\fi}
%    \end{macrocode}
% \end{macro}
%
% \begin{macro}{\examerror}
% \begin{macro}{\examwarning}
% Define |\examerror| and |\examwarning| to issue proper 
% warnings in case of errors.
% Note that the error macro provides for a help text in its 
% second argument.
%
%    \begin{macrocode}
\newcommand*\examerror[2]{\ClassError{exam}{!!!! #1}{#2}}
\newcommand*\examwarning[1]{\ClassWarning{exam}{!!!! #1}}
%    \end{macrocode}
% \end{macro}
% \end{macro}
%
% \section{Produce an Exam}
%
% \begin{macro}{\examnum}
% First we need a counter for exams, since in one run more than
% one exam can be produced.
% By stepping this counter we will effect the automatic reset of
% the counter that numbers the problems and 
% the counter that remembers the score value.
%
%    \begin{macrocode}
\newcounter{examnum}
%    \end{macrocode}
% \end{macro}
%
% \begin{environment}{exam}
% Exams are produced within the |exam| environment. This environment takes
% 2 parameters. The first one is optional and provides the initial value
% of the random generator.\footnote{Not used when a series
% is run.} The default is 0, which effectively shuts off randomness.
% The second parameter must be present, but can be empty.
% It fixes the date for which the exam is planned; an empty argument
% fills in the current date.
%
%    \begin{macrocode}
\newenvironment{exam}[2][0]{%
  \stepcounter{examnum}%
  \@ifemptyarg{#2}{}{\date{#2}}%
%    \end{macrocode}
%
% When answers are requested we start with a titlepage.\footnote{%
% If not inhibited by the |notitlepage| option.}
% In the case of exam production, typesetting of the titlepage 
% is deferred to the end of the exam,
% so that we may print on it the number of problems.
% We write a few messages to the table of contents (date and initial 
% value of the random generator) when an exam with answers 
% is in production.
% Disable the random generator for a series.
%
%    \begin{macrocode}
  \ifanswers
    \pagenumbering{roman}%
    \maketitle\newpage\mbox{}\newpage
  \fi
  \pagenumbering{arabic}%
  \ifseries\SRset{0}\else
    \SRset{#1}%
    \addtocontents{toc}{\protect\contentsline{section}%
      {\Exam~\theexamnum~\textemdash~\@date~%
      \textemdash~random start #1}{}}%
  \fi
%    \end{macrocode}
%
% In each separate exam the first page gets the number one.
%
%    \begin{macrocode}
  \setcounter{page}{1}}%
%    \end{macrocode}
%
% At the end of the exam produced for the students
% a titlepage is made. If answers are given for the exam
% we also provide the total of the scores.
%
%    \begin{macrocode}
  {\ifseries\else
    \typeout{Total value scores = \thetotalscore}%
    \ifanswers
      \addtocontents{toc}{\protect\contentsline{section}%
        {Total value scores = \thetotalscore}{}}%
    \else\maketitle
    \fi\fi}
%    \end{macrocode}
% \end{environment}
%
% \section{Choosing Problems}
%
% \begin{macro}{\problemnum}
% We start with a counter |\problemnum| with which the problems
% of the exam are neatly numbered. This counter is automatically
% reset each time a new |exam| environment is entered.
% \begin{macro}{\problemid}
% A textual identification of the current problem is collected
% in token register |\problemid|.
%
%    \begin{macrocode}
\newcounter{problemnum}[examnum]
\newtoks\problemid
%    \end{macrocode}
% \end{macro}
% \end{macro}
%
% \begin{macro}{\question}
% \textbf{Each question must reside in its own file} which is called up
% by macro |\question|. Of its three parameters the first is
% optional and provides a means of communication with the
% problem itself. To achieve this the first 
% argument of |\question| is cached 
% in macro |\parameter|.\footnote{As most uses of this mechanism
% boil down to a choice between several alternatives, the
% number~1 is provided by macro {\ttfamily\protect\bslash parameterproblem}
% as a convenient default value. See also the discussion
% under the heading ``Parametrized Problems''.}
% The default behaviour here is not touching the
% definition of |\parameter| in case of an empty argument;
% in many cases a forgotten argument will then lead to
% a ``missing something'' error. The benefit of not
% touching |\parameter| in case of an empty argument
% is that this macro now also can be initialized by
% other means, e.g. by definition earlier in the problem coding.
%
% The second parameter of |\question| is the name of the (sub)directory
% where the file named in the third parameter can be found.
% This second parameter doubles up as section name in the
% series production.\footnote{It is silently assumed
% that all problems of a given category reside in a common
% directory.}
%
%    \begin{macrocode}
\newcommand*\question[3][]{%
  \@ifemptyarg{#1}{}{\renewcommand\parameter{#1}}%
%    \end{macrocode}
%
% When a series is run we look for the start of a new section and
% perform the appropriate actions if indeed a new section is found.
% I.e.\ eject the page and then reset the section name 
% and the problem counter.
% Note the use of uppercase in order to smooth out differences in typing.
% The identification of the problem is set to its file name and,
% in the case of a series, is mentioned in the output.
% Then the problem number is incremented. 
%
%    \begin{macrocode}
  \ifseries
    \uppercase{\def\@namesection{#2}}%
    \ifx\namesection\@namesection
    \else
      \newpage
      \global\let\namesection=\@namesection
      \addcontentsline{toc}{subsection}{\namesection}%
      \setcounter{problemnum}{0}%
    \fi   
  \fi
  \problemid={\MakeUppercase{#3}}%
  \ifseries
    \noindent\underbar{\emph{File\,:}~\texttt{\the\problemid}}\par
    \nopagebreak\medskip\nopagebreak
  \fi
  \stepcounter{problemnum}%
%    \end{macrocode}
%
% If appropriate a summary of this problem is written to the table of contents.
%
%    \begin{macrocode}
  \ifanswers
    \addcontentsline{toc}{subsection}%
      {\hbox to1cm{\theproblemnum:\hss}#3}%
  \fi
%    \end{macrocode}
%
% Reading of the problem itself is surrounded by calculations
% on the score that this question will bring.
% Scores are mentioned on the console except when a series is run.
% In a problem all contributions from the various parts of the
% problem are collected in counter |scorecounter|.
% At the end of the problem |totalscore| is 
% updated with this value.\footnote{%
% Note the resets for |totalscore| with |examnum|
% and |scorecounter| with |problemnum| in their declaration.}
% The code guards against typing errors in the name of the file.
%
%    \begin{macrocode}
  \edef\@curquestion{\mainfolder\@ifemptyarg{#2}{}{#2\@currdir}#3}%
  \IfFileExists{\@curquestion}{\@@input \@curquestion}%
    {\examwarning{File \@curquestion: not found}}%
  \ifseries\else
    \addtocounter{totalscore}{\value{scorecounter}}%
    \typeout{\Problem\space\theproblemnum: score=\thescorecounter}%
  \fi
  }
%    \end{macrocode}
% \end{macro}
%
% \begin{macro}{namesection}
% Macro |\namesection| gets its initial value here:
%
%    \begin{macrocode}
\newcommand*\namesection{\Collection}
%    \end{macrocode}
% \end{macro}
%
% \subsection{Parametrized Problems}
%
% \begin{macro}{\reset@parameter}
% A parametrized problem gets its parameter from the first
% argument of macro |\question|, as already have been mentioned.
% This is effected by definition of macro |\parameter| to
% the value of that argument. 
% We add code here to (re)initialize this macro.
%
%    \begin{macrocode}
\newcommand*\reset@parameter{\gdef\parameter{}}
\reset@parameter
%    \end{macrocode}
% \end{macro}
%
% \begin{macro}{\parameterproblem}
% The first argument of |\parameterproblem| is optional and 
% sets a default by for |\parameter|;
% set to the number~1 if not explicitely given.
% It is recommended that |\parameterproblem| is placed
% before the first use of |\parameter| or even before the
% |\begin{problem}|.
% Furthermore, in typesetting an exam this macro will print a warning
% if |\parameter| has not been set on the |\question| call.
%
% |\parameterproblem| will typeset its second argument in a framed box.
% Usually this tells the reader which selections are available; however,
% only in the case a series is run, otherwise `silence' is the word.
% The description is placed by macro |\remark| which does the
% necessary suppression.
% 
%    \begin{macrocode}
\newcommand\parameterproblem[2][1 ]{%
  \ifx\parameter\@empty
    \ifseries\else\examwarning{\string\parameter\space set to `#1'}\fi
    \renewcommand\parameter{#1}%
  \fi
  \remark[Parameter \Problem]{#2}}
%    \end{macrocode}
% \end{macro}
%
%
% \section{Typesetting a Problem}
%
% Each problem must be enclosed in an environment |problem|.
% Within this environment a default setup exists.
% \begin{macro}{\everyproblem}
% By supplying code in token register |\everyproblem| one
% can influence the typesetting of each problem.
%
%    \begin{macrocode}
\newtoks\everyproblem
%    \end{macrocode}
% \end{macro}
%
% \begin{environment}{problem}
% The |problem| environment also has one optional parameter
% for specific adjustments of the options setting.
% Execution of options occurs in the order:
% default setup, possible modification by |\everyproblem| and
% final customization through the optional parameter.
% This mechanism provides for maximum flexibility.
%
%    \begin{macrocode}
\newenvironment{problem}[1][]{%
%    \end{macrocode}
%
% Choose by default for keeping the whole problem on a page,
% execute any code in the token register and honor the
% option calls from the user.
%
%    \begin{macrocode}
\compact\the\everyproblem#1\relax
%    \end{macrocode}
%
% In order to keep everything on page we will enclose
% the problem in a |\vbox|, coded in
% macro |\@boxing|. Otherwise |\@boxing| is a noop and
% \TeX's pagebuilder can choose its breakpoint freely.
% For the declaration of |\@boxing| see section~\ref{ref:boxing}.
%
% The problem is typeset with a standard opening
% programmed in |\problemstart|, completing the
% opening manoeuvres of the environment.
%
%    \begin{macrocode}
  \@boxing\bgroup\noindent\problemstart\ignorespaces}%
%    \end{macrocode}
%
% After processing the body of the problem some postprocessing follows
% and the possible |\vbox| is closed by an |\egroup|.
%
% In particular a visual separation from the next problem is added,
% if not suppressed.
% In the case of series production the origin date
% of the problem is added too.\footnote{Only if it has been
% provided to it by the proper macro call, of course.}
%
%    \begin{macrocode}
  {\par\ifproblemsep
    \nopagebreak\smallskip\nopagebreak
    \hbox to\linewidth{\hrulefill
      \ifseries
        \emph{\footnotesize\thinspace\the\@problemdate}%
      \fi}\fi
  \egroup\par
%    \end{macrocode}
%
% Start a new page or separate the problem from the next one by a skip.
%
%    \begin{macrocode}
  \ifproblempages\newpage\else\bigskip\fi
%    \end{macrocode}
%  
% The origin date and the communicated value
% in macro |\parameter| are cleared for the next problem.
%
%    \begin{macrocode}
  \reset@problemdate\reset@parameter}
%    \end{macrocode}
% \end{environment}
% 
% \subsection{Code for Options to Problem}
% \label{ref:boxing}
%
% \begin{macro}{\compact}
% \begin{macro}{\split}
% The options to |problem| are |\compact| or |\split|. 
% These options govern the possibility for the problem 
% to be split between successive pages or the necessity 
% to keep everything on page; the last one being the 
% favoured behaviour in this implementation.
% Note the |\noident| before the |\vbox| that prevents
% an unwanted shift to the right.
% 
%    \begin{macrocode}
\newcommand*\compact{\def\@boxing{\noindent\vbox}}
\newcommand*\split{\def\@boxing{}}
%    \end{macrocode}
% \end{macro}
% \end{macro}
%
%
% \subsection{Numbering the Problem}
%
% \begin{macro}{\problemstart}
% \begin{macro}{\@problemstart}
% A problem gets a standard opening clause, coded in
% macro |\@problemstart|. The opening code is used to
% format the first paragraph with a nice indentation.\footnote{%
% This indentation is also used in the left margin in multiple
% choice listings in order to limit the variation in margins.}
%
%    \begin{macrocode}
\newcommand*\problemstart{%
  \hangafter-2\settowidth\hangindent{\@problemstart}%
  \noindent\llap{\@problemstart}}
\newcommand*\@problemstart{%
  \textbf{\Problem\,\ifnum\value{problemnum}<10 \phantom{0}\fi
  \theproblemnum}.\enskip}
%    \end{macrocode}
% \end{macro}
% \end{macro}
%
% \subsection{Date of Origin}
%
% \begin{macro}{\@problemdate}
% \begin{macro}{\problemdate}
% \begin{macro}{\@resetproblemdate}
% The user may specify an original date or date of last change
% for the problem to be printed when a series is produced.
% The global assignments are here just in case things happen in
% a deeper nested level.
%
%    \begin{macrocode}
\newtoks\@problemdate
\newcommand*\problemdate[1]{\global\@problemdate={#1}\ignorespaces}
\newcommand*\reset@problemdate{\global\@problemdate={}}
\reset@problemdate
%    \end{macrocode}
% \end{macro}
% \end{macro}
% \end{macro}
%
% \subsection{Score Values}
%
% Associated with each problem are of course the benefits the
% student receives for a good answer to (part of) the problem.
% \begin{macro}{\score}
% The |\score| macro exists for this purpose.
% If answers are not included, just an empty square is printed
% into which the teacher can express his satisfaction with
% the answer given. When answers are included in the printout
% the each call |\score{value}| shows up in the right margin
% of the document.\footnote{At the end of each problem a summary
% of its total score plus a grand total are presented
% on the console.}
%
% \begin{macro}{\totalscore}
% \begin{macro}{\scorecounter}
% These counters collect the values. Note that |\totalscore|
% is reset for each new exam and |\scorecounter| for each problem.
%
%    \begin{macrocode}
\newcounter{totalscore}[examnum]
\newcounter{scorecounter}[problemnum]
%    \end{macrocode}
% \end{macro}
% \end{macro}
%
% \begin{macro}{\scoreboxsize}
% \begin{macro}{\scorebox}
% The next commands are used for the production of the box
% for the score value.
%
%    \begin{macrocode}
\newcommand*\scoreboxsize{6mm}
\newcommand*\scorebox[1]{%
  \fbox{\vbox to\scoreboxsize{\vss\hbox to
    \scoreboxsize{\hss#1\hss}\vss}}}
%    \end{macrocode}
% \end{macro}
% \end{macro}
%
% Finally the next code puts the score box on paper.
% It takes the value of the score as its argument and adds
% it to the running sum for this problem.
%
%    \begin{macrocode}
\newcommand*\score[1]{%
  \addtocounter{scorecounter}{#1}%
  \rightnote{\scorebox{\ifscores#1\fi}}%
  \ignorespaces}
%    \end{macrocode}
% \end{macro}
%
% \begin{macro}{\leftnote}
% \begin{macro}{\rightnote}
% We do not use |\marginpar| for the placement of the score values,
% because we do not want these items wandering around, as
% |\marginpar|'s sometimes do.
%
% \begin{macro}{\@rlnote}\mbox{}
%    \begin{macrocode}
\providecommand\leftnote{\@rlnote l}
\providecommand\rightnote{\@rlnote r}
\providecommand\@rlnote[2]{%
  \leavevmode\noindent
  \vadjust{\vbox to\z@{%
    \leftskip\z@skip\rightskip\z@skip
    \noindent
    \if#1l\llap{#2\hskip\marginparsep}%
    \else\hfill\rlap{\null\hskip\marginparsep\relax#2}\fi
    \vss\vskip\z@skip
  }}\ignorespaces}
%    \end{macrocode}
% \end{macro}
% \end{macro}
% \end{macro}
%
% \subsection{Adding remarks}
%
% In making a catalogue of problems (option \emph{series} selected) 
% it is useful when remarks can be added that stand out against the rest
% of the text. 
% \begin{macro}{\remark}
% Macro |\remark| provides such a mechanism.
% Its first (optional) argument is set emphasized, its second argument
% hangs on the first. The complete remark is placed
% in a |\parbox| and then boxed and centered.
%
%    \begin{macrocode}
\newcommand\remark[2][]{%
  \ifseries
    \begin{center}%
      \fbox{\parbox{.9\linewidth}{%
        \sloppy\hangafter\@ne
        \sbox\@tempboxa{\emph{#1}\@ifemptyarg{#1}{}{:~}}%
        \hangindent=\wd\@tempboxa
        \strut\usebox{\@tempboxa}#2}}%
      \end{center}%
    \nopagebreak\addvspace{\bigskipamount}\nopagebreak
  \fi}
%    \end{macrocode}
% \end{macro}
%
% \section{Answers}
%
% In this section various ways of typesetting answers are provided.
%
% \begin{macro}{\longwhite}
% \begin{macro}{\shortwhite}
% We start with two definitions for long and short stretches of white
% space. These are meant for leaving room for the students answer.
%
%    \begin{macrocode}
\newcommand*\longwhite{25mm}
\newcommand*\shortwhite{8mm}
%    \end{macrocode}
% \end{macro}
% \end{macro}
%
% \begin{macro}{\answerstart}
% Just as with the typesetting of the problem, we provide
% a macro to start an answer. Note that the text is
% parametrized in order to keep switching to other
% languages simple.
%
%    \begin{macrocode}
\newcommand*\answerstart{\noindent\emph{\Answer}:\enspace}
%    \end{macrocode}
% \end{macro}
%
% \subsection{Switching Answer On and Off}
% 
% \begin{macro}{\answer}
% Macro call |\answer| holds the answer and shows it when
% answers are requested. The optional first argument specifies
% a width for the box into which the typesetting takes places.
% The answer is centered by default; change it with |\hfil|'s.
% Implementation of |\answer| is by the next macro |\altanswer|. 

%    \begin{macrocode}
\newcommand*\answer[2][]{\altanswer[#1]{}{#2}}
%    \end{macrocode}
% \end{macro}
%
% \subsection{Alternating Some Stuff and Answer}
% 
% \begin{macro}{\altanswer}
% With |\altanswer| the text alternates between two possibilities:
% the first one is typeset when answers are suppressed, the second
% one for the opposite case. Optional width argument and placement
% are the same as for |\answer|.
%
%    \begin{macrocode}
\newcommand*\altanswer[3][]{%
  \@ifemptyarg{#1}%
    {\mbox{\ifanswers#3\else#2\fi}}%
    {\makebox[#1]{\ifanswers#3\else#2\fi}}%
  }
%    \end{macrocode}
% \end{macro}
%
% \subsection{Problem with a Short Answer}
% 
% \begin{macro}{\shortanswer}
% A question ``Give a short answer to \ldots'' is formatted
% in |\shortanswer|. Usually the answer will fit on one line.
% In the exam a row of dots is produced, otherwise the answer will show.
% The optional argument provides the width of the box into which
% the data are typeset.
%
%    \begin{macrocode}
\newcommand*\shortanswer[1]{\par
  \ifanswers
    \addvspace{\smallskipamount}%
    \settowidth\@tempdima{\answerstart\quad}%
    \setlength\@tempdimb{\linewidth}%
    \addtolength\@tempdimb{-\@tempdima}%
    \answerstart\parbox[t]{\@tempdimb}{\noindent#1}%
    \par\medskip
  \else\ifreservespace
    \addvspace{\shortwhite}%
    \answerstart\mbox{}\dotfill\quad\mbox{}\par
    \medskip
   \fi\fi
  }
%    \end{macrocode}
% \end{macro}
%
% \subsection{Problem with a Long Answer}
% 
% \begin{environment}{longanswer}
% For elaborate questions, problems, etc.\ an environment is available.
% The |longanswer| environment takes as optional argument the length
% of white to be reserved for the student.
%
% Code for opening of the environment. 
% It opens a box in order to let the answer disappear
% and places a rule in order to guarantee sufficient
% white space.
%
%    \begin{macrocode}
\newenvironment{longanswer}[1][\longwhite]{
  \par
  \ifanswers
    \addvspace{\medskipamount}\answerstart
    \nopagebreak\par\noindent
  \else\ifreservespace
      \addvspace{\medskipamount}\answerstart
      \nopagebreak\par\noindent
      \hrule\@height#1\@width\z@\par
    \fi
    \setbox\z@\vbox\bgroup\leavevmode
   \fi\ignorespaces}
%    \end{macrocode}
%
% Aftermath of |longanswer|. If necessary close the box
% and empty it to get rid of the answer.
%    \begin{macrocode}
  {\ifanswers\par\medskip\else\egroup\setbox\z@\hbox{}\fi}
%    \end{macrocode}
% \end{environment}
%
% \section{Multiple Choice Questions}
%
% Multiple choice problems must be placed
% in an |choice| environment, a modification
% of |itemize|.
%
% We will make it possible to 
% shuffle the items of a multiple
% choice problem randomly. These items are held in a series
% of token registers declared below.
%
% \begin{macro}{\loadcounter}
% \begin{macro}{\resetloadcounter}
% \begin{macro}{\incloadcounter}
% \begin{macro}{\decloadcounter}
% We need a counter into which to keep the number of items
% at any time loaded into the token registers declared above.
% Also we provide for resetting, incrementing and decrementing
% of this register. Note the global assignments.
%
%    \begin{macrocode}
\newcount\loadcounter
\newcommand*\resetloadcounter{\global\loadcounter\z@}
\newcommand*\incloadcounter{\global\advance\loadcounter\@ne}
\newcommand*\decloadcounter{\global\advance\loadcounter\m@ne}
%    \end{macrocode}
% \end{macro}
% \end{macro}
% \end{macro}
% \end{macro}
%
% We want a specific behaviour when the list of items is typeset.
% However, we cannot be sure at which listlevel this will occur.
% \begin{macro}{\@listk}
% Therefore we predeclare a replacement for |\@listi|, |\@listii|,
% or whatsoever, and swap the |\@list..| at the right time.
% Note the choice for the leftside margin, derived from the
% width of the text with which the problem starts. This choice
% diminishes the number of different margins. It is easily adapted
% to your own taste.
%
%    \begin{macrocode}
\newcommand*\@listk{%
  \settowidth{\leftmargin}{\@problemstart}%
  \topsep\medskipamount
  \partopsep\z@
  \itemsep\smallskipamount
  \parsep\z@}
%    \end{macrocode}
% \end{macro}
%
% \subsection{Typesetting Multiple Choice}
%
% \begin{environment}{choice}
% The multiple choice environment |choice| takes one argument,
% the modifier options to the environment typesetting.
% Here the options are |\random| and |\ordered|; the names
% speak for themselves. Note that random permutation is not
% executed if a series is run. Furthermore the counter
% for the number of items loaded is reset.
%
%    \begin{macrocode}
\newenvironment{choice}[1][]{%
  \ifseries\ordered\else\random\fi#1\relax
  \resetloadcounter
%    \end{macrocode}
%
% The following code is taken from \LaTeX's |itemize|.
% I did not find a more elegant way to bend this environment
% to my whims.
%
%    \begin{macrocode}
  \ifnum\@itemdepth>3 \@toodeep \else
  \advance\@itemdepth\@ne
  \expandafter\let
    \csname @list\romannumeral\the\@itemdepth\endcsname=\@listk
  \list{\badmark}{\def\makelabel##1{\hss\llap{##1}}}%
  \fi}%
%    \end{macrocode}
%
% At the end of |choice| we dump all the items that may have been
% collected inbetween and finish the |list|.
%
%    \begin{macrocode}
  {\@dumpitems\endlist}
%    \end{macrocode}
% \end{environment}
% 
% \subsection{Code for Options to Choice}
%
% \begin{macro}{\random}
% The option |\random| codes macros |\@loaditem| and
% |\@dumpitems| so that the items are actually loaded,
% then shuffled and dumped afterwards.
% \begin{macro}{\ordered}
% The |\ordered| option makes them noops and thus the
% items will be typeset on the fly.
%
%    \begin{macrocode}
\newcommand*\random{%
  \def\@loaditem{\loaditem}%
  \def\@dumpitems{\shuffle\dumpitems}}
\newcommand*\ordered{\def\@loaditem{}\def\@dumpitems{}}
%    \end{macrocode}
% \end{macro}
% \end{macro}
% 
% \subsection{Formatting the Item Mark}
%
% \begin{macro}{\marksize}
% \begin{macro}{\badmark}
% \begin{macro}{\goodmark}
% We require two marks: one for the bad guys and one
% for the good guy. We use the two symbols |\square| and |\boxtimes|,
% but provide replacements (later on, after giving exam.cfg a chance
% to define them) in case these are undefined.
% Typeset these marks in fixed size (unchanged baselineskip) 
% provided by macro |\marksize|.
%
%    \begin{macrocode}
\newcommand*\marksize{\fontsize{12}{\f@baselineskip}\selectfont}
\newcommand*\badmark{{\marksize\ensuremath{\square}}}
\newcommand*\goodmark{%
  \ifanswers{\marksize\ensuremath{\boxtimes}}\else\badmark\fi}
%    \end{macrocode}
% \end{macro}
% \end{macro}
% \end{macro}
%
% \begin{macro}{\baditem}
% \begin{macro}{\gooditem}
% Each item can either be right or wrong. We take the precaution
% to suppress the difference when typesetting the actual exam.
% Enclose each item in your list in the argument to
% |\gooditem| and |\baditem|. They
% will load the item in memory prior to (possible) random shuffling.
%
%    \begin{macrocode}
\newcommand*\baditem[1]{\@loaditem{\item[\badmark]#1}}
\newcommand*\gooditem[1]{\@loaditem{\item[\goodmark]#1}}
%    \end{macrocode}
% \end{macro}
% \end{macro}
%
% \subsection{Loading and Dumping Items}
%
% \begin{macro}{\@itemA}
% \begin{macro}{\@itemB}
% \begin{macro}{\@itemC}
% \begin{macro}{\@itemD}
% \begin{macro}{\@itemE}
% This series of token registers
% can hold five alternatives. The mechanism that loads the
% items is sufficiently general to use it for other purposes too.
% Use your imagination!
% That there are five of them is remembered in a definition
% because we will need this number to prevent overfilling the store.
%
% \begin{macro}{\@itemstore}\mbox{}
%    \begin{macrocode}
\newtoks\@itemA
\newtoks\@itemB
\newtoks\@itemC
\newtoks\@itemD
\newtoks\@itemE
\newcommand\@itemstore{5}
%    \end{macrocode}
% \end{macro}
% \end{macro}
% \end{macro}
% \end{macro}
% \end{macro}
% \end{macro}
%
% \begin{macro}{\loaditem}
% According to the value of |loadcounter| the token registers
% |\@itemA|, etc.\ are filled. Argument to macro |\loaditem|
% is the contents of the item.
%
%    \begin{macrocode}
\newcommand\loaditem[1]{%
  \ifcase\loadcounter
    \@itemA={#1}%
    \or\@itemB={#1}%
    \or\@itemC={#1}%
    \or\@itemD={#1}%
    \or\@itemE={#1}%
  \fi
  \ifnum\loadcounter<\@itemstore \incloadcounter
  \else\examwarning{\string\loaditem\space ignored, too many}\fi}
%    \end{macrocode}
% \end{macro}
%
% \begin{macro}{\dumpitemno}
% Produce items that were loaded.
%
%    \begin{macrocode}
\newcommand*\dumpitemno[1]{%
  \ifnum#1>\loadcounter
    \examwarning{\string\dumpitemno[#1] ignored, out range}%
  \else\ifcase#1\relax
    \or\the\@itemA
    \or\the\@itemB
    \or\the\@itemC
    \or\the\@itemD
    \or\the\@itemE
  \fi\fi}
%    \end{macrocode}
% \end{macro}
%
% \begin{macro}{\dumpitem}
% \begin{macro}{\dumpitems}
% With |\dumpitem| the last one comes out and is chopped off
% from the stack, with |\dumpitems| the whole lot is dumped.
% By means of |\dumpitemno| one can peek inside the stack:
% its parameter gives the position to be produced, the item itself
% remains on the stack.
%    \begin{macrocode}
\newcommand*\dumpitem{\dumpitemno{\loadcounter}\decloadcounter}
\newcommand*\dumpitems{\@whilenum\loadcounter>\z@\do{\dumpitem}}
%    \end{macrocode}
% \end{macro}
% \end{macro}
%
% \subsubsection{Shuffling Items}
%
% \begin{macro}{\shuffle}
% This macro permutes |loadcounter| items in the
% token registers |\@itemA|, etc. Undoubtedly it
% can be done better, but who's perfect?
%
%    \begin{macrocode}
\newcommand*\shuffle{%
  \ifcase\loadcounter
    \or
    \or\shuffle@ii
    \or\shuffle@\@itemA\@itemC \shuffle@ii \shuffle@\@itemB\@itemC
    \or\shuffle@iv
    \or\shuffle@\@itemD\@itemE \shuffle@iv \shuffle@\@itemD\@itemE
    \fi
  }
%    \end{macrocode}
% \end{macro}
%
% \begin{macro}{\shuffle@}
% \begin{macro}{\shuffle@ii}
% \begin{macro}{\shuffle@iv}
% \begin{macro}{\@item@}
% Random interchange of two and four items.
%
%    \begin{macrocode}
\newtoks\@item@
\newcommand*\shuffle@[2]{\SRtest{}{\@item@=#1 #1=#2 #2=\@item@}}
\newcommand*\shuffle@ii{\shuffle@\@itemA\@itemB}
\newcommand*\shuffle@iv{%
  \SRtest{\shuffle@\@itemA\@itemB}{\shuffle@\@itemC\@itemD}%
  \SRtest{\shuffle@\@itemA\@itemC}{\shuffle@\@itemB\@itemD}}
%    \end{macrocode}
% \end{macro}
% \end{macro}
% \end{macro}
% \end{macro}
%
% \subsubsection{Random Generator Implementation}
%
% \begin{macro}{\SRset}
% \begin{macro}{\SRbit}
% \begin{macro}{\SRtest}
% \begin{macro}{\SRvalue}
% Not much commentary with these macros. They are
% described in Tugboat~1994, vol.~15.1, p.~57--58.
%
% \begin{macro}{\@SR}
% \begin{macro}{\@SRconst}
% \begin{macro}{\@SRadvance}\mbox{}
%    \begin{macrocode}
\ifx\@SR\undefined\newcount\@SR\fi
\providecommand\@SRconst{2097152}
\providecommand\SRset[1]{\global\@SR#1 \ignorespaces}
\providecommand\@SRadvance{%
  \begingroup
  \ifnum\@SR<\@SRconst\relax\count@\z@\else\count@\@ne\fi
  \ifodd\@SR\advance\count@\@ne\fi
  \global\divide\@SR\tw@
  \ifodd\count@\global\advance\@SR\@SRconst\relax\fi
  \endgroup}
\providecommand\SRbit{\@SRadvance\ifodd\@SR1\else0\fi}
\providecommand\SRtest[2]{\@SRadvance
  \ifodd\@SR#2\else#1\fi\ignorespaces}
\providecommand\SRvalue{\number\@SR }
\SRset{0}
%    \end{macrocode}
% \end{macro}
% \end{macro}
% \end{macro}
% \end{macro}
% \end{macro}
% \end{macro}
% \end{macro}
%
% \section{Page Style}
%
% \begin{macro}{\thehead}
% For a page style |examheadings| is offered.
% Choose it by supplying to |\pagestyle|.
%
% \begin{macro}{\ps@examheadings}\mbox{}
%    \begin{macrocode}
\newcommand*\thehead{%
  \textsl{\@title\enspace:\enspace
  \ifseries\namesection\else\@date\fi}}
\newcommand*\ps@examheadings{%
  \let\@oddfoot\@empty
  \let\@evenfoot\@empty
  \renewcommand*\@oddhead{%
    \vbox{%
    \hbox to\textwidth{\headerfont\thehead\hfil\upshape\thepage}%
    \vskip1.5\p@
    \hrule\@height.5\p@\@width\textwidth
    }}%
  \let\@evenhead\@oddhead}
%    \end{macrocode}
% \end{macro}
% \end{macro}
%
% \section{Titlepage}
%
% \begin{macro}{\target}
% With target we denote the group of students for whom
% the exam is meant. Defined with |\target| and called up
% with |\@target|, just like |\author|, etc.
%
%    \begin{macrocode}
\newcommand*\target[1]{\gdef\@target{#1}}\def\@target{}
%</cls>
%    \end{macrocode}
% \end{macro}
%
% The titlepage is best set by a redefined |\maketitle|. Of course it 
% needs to be suppressed if the \texttt{notitlepage} option is given
% on the |\documentclass| call. Provide two versions, one
% for a real exam and one for collections and/or answers.
% See the example below, supplied in \emph{exam.cfg}.
%
% The titlepage is set by a redefined |\maketitle|. Of course it 
% needs to be suppressed if the notitlepage option is given
% on the |\documentclass| call. Provided are two versions, one
% for a real exam and one for collections and/or answers.
%
% \subsection{Example titlepage}
%
%    \begin{macrocode}
%<*cfg>
%
\if@titlepage
\renewcommand*\maketitle{%
\begin{titlepage}
  \begin{center}\titlefont
    \vspace*{1cm}%
    \mbox{}\rule{2cm}{0.4pt}\mbox{}\par
    \addvspace{1cm}%
    \begin{Large}
      \textbf{\ifseries\Collection\else\Exam\fi}\\[10mm]
    \end{Large}
    \begin{large}
      \@title\\[5mm]
      \ifseries\@author\else\@target\fi\\[5mm]
      \@date\\[10mm]
    \end{large}
    \mbox{}\rule{2cm}{0.4pt}\mbox{}\par
    \addvspace{2cm}%
  \ifseries
    \vfill\vfill
    \begin{flushleft}\emph{Copyright notice, if any.}\end{flushleft}%
  \else\ifanswers
      \begin{huge}\Answers\end{huge}\par
    \else
      \begin{minipage}{.75\textwidth}%
      \raggedright\parindent\medskipamount
        Name:\enspace\dotfill\strut\par
        Address:\enspace\dotfill\strut\par
        City:\enspace\dotfill\strut\par
        Student number:\enspace\dotfill\strut\par
        \vspace{1cm}%
        \begin{itemize}%
        \item Please write legible, what cannot be read
            cannot be given credit.
        \item Put your name and student number on all
            on all separate sheets of paper.
        \item This exam has \theproblemnum\ problems.
        \end{itemize}%
      \end{minipage}\\[10mm]
      Good luck!\par
    \fi
  \fi
  \end{center}%
\end{titlepage}\let\maketitle=\relax}
%    \end{macrocode}
%
% And in case no title page requested:
%
%    \begin{macrocode}
\else\let\maketitle=\relax\fi
%</cfg>
%    \end{macrocode}
%
%
% \section{Language Dependent Items}
%
% \begin{macro}{\Exam}
% \begin{macro}{\Collection}
% \begin{macro}{\Answers}
% \begin{macro}{\Answer}
% \begin{macro}{\Problem}
% Predefine all language specific macros, 
% default is the English language.
%
%    \begin{macrocode}
%<*cls>
\newcommand*\Exam{EXAM}
\newcommand*\Collection{COLLECTION OF EXAMS}
\newcommand*\Answers{ANSWERS}
\newcommand*\Answer{Answer}
\newcommand*\Problem{Problem}
%    \end{macrocode}
% \end{macro}
% \end{macro}
% \end{macro}
% \end{macro}
% \end{macro}
%
% \subsection{Dutch equivalents}
%
%    \begin{macrocode}
%</cls>
%<*cfg>
\renewcommand*\Answers{ANTWOORDEN}
\renewcommand*\Answer{Antwoord}
\renewcommand*\Exam{TENTAMEN}
\renewcommand*\Collection{TENTAMENBUNDEL}
\renewcommand*\Problem{Opgave}
%</cfg>
%<*cls>
%    \end{macrocode}
%
% \section{Initializations}
%
% \subsection{Fonts}
%
% \begin{macro}{\headerfont}
% \begin{macro}{\bodyfont}
% \begin{macro}{\titlefont}
% Fonts for pageheader, body of the text and on the titlepage.
%
%    \begin{macrocode}
\newcommand*\headerfont{\rmfamily\small}
\newcommand*\bodyfont{\sffamily}
\newcommand*\titlefont{\rmfamily\upshape}
%    \end{macrocode}
% \end{macro}
% \end{macro}
% \end{macro}
%
% And initialize to |\bodyfont|.
%
%    \begin{macrocode}
\bodyfont
%    \end{macrocode}
%
% \subsection{Directory Localization}
%
% \begin{macro}{\CurrentDirectory}
% \begin{macro}{\DirectorySeparator}
% We can determine from |\@currdir| which
% character separates directories in a path name. 
% E.g. in UNIX this is |/| from the string |./|, but
% in the MacOS the current directory and the separator are both |:|.
% Therefore we extract from |\@currdir| the last character 
% (of two at most).\\
% Make |\CurrentDirectory| a synonyme for |\@currdir|.
%
%    \begin{macrocode}
\let\CurrentDirectory=\@currdir
\def\DirectorySeparator#1#2`\^^M{\@ifemptyarg{#2}{#1}{#2}}
\edef\DirectorySeparator{%
	\expandafter\DirectorySeparator\CurrentDirectory`\^^M}
%    \end{macrocode}
% \end{macro}
% \end{macro}
%
% \begin{macro}{\LastChar}
% Another macro delivers the last character of a string.
%
%    \begin{macrocode}
\providecommand*{\LastChar}[1]{%
  \@ifemptyarg{#1}{}{\expandafter\@lastchar#1`\^^M}}
\def\@lastchar#1#2`\^^M{\@ifemptyarg{#2}{#1}{\@lastchar#2`\^^M}}
%    \end{macrocode}
% \end{macro}
%
% \begin{macro}{\DirectoryName}
% The next macro ensures that a path name ends correctly, when
% a filename is concatenated with it.
% If the directory separator character isn't the last character,
% it is added.
%
%    \begin{macrocode}
\providecommand*{\DirectoryName}[1]{\@ifemptyarg{#1}{}%
  {\if\LastChar{#1}\DirectorySeparator\relax#1\else
    #1\DirectorySeparator\fi}}
%    \end{macrocode}
% \end{macro}
%
% \begin{macro}{\Setfolder}
% Macro |\Setfolder| can be used to install a standard
% folder (directory) name.
% E.g. a name |\figuresfolder| can be defined as the
% standard place for figures.
% Supply as first argument to |\Setfolder| the macro name
% for the folder. e.g. |\figuresfolder| and as second
% parameter its location on disk.
% Below three of these folders (here initialized 
% with empty names) are defined in the example configuration file.
%    \begin{macrocode}
\newcommand*\Setfolder[2]{\edef#1{\DirectoryName{#2}}}
%</cls>
%<*cfg>
\Setfolder{\mainfolder}{}
\Setfolder{\commonfolder}{}
\Setfolder{\figuresfolder}{}
%</cfg>
%<*cls>
%    \end{macrocode}
% \end{macro}
%
% \subsection{Configuration File}
%
% Last, but not least, see if there is a configuration
% file \texttt{exam.cfg} and read it for the final adjustments.
%
%    \begin{macrocode}
\InputIfFileExists{exam.cfg}{}{}
%    \end{macrocode}
%
% \subsection{Macros Needed but Possibly Missing}
%
% \begin{macro}{\square}
% \begin{macro}{\boxtimes}
% Possibly the following macros are still undefined; here we guarantee
% they are available. You may define them yourselves in
% \emph{exam.cfg} or in your document.
%
%    \begin{macrocode}
\providecommand\square{\bigcirc}
\providecommand\boxtimes{\surd}
%    \end{macrocode}
% \end{macro}
% \end{macro}
%    \begin{macrocode}
%</cls>
%    \end{macrocode}
%
% \section{Coding of Example Questions}
%
%    \begin{macrocode}
%<*exa>
\begin{problem}
\problemdate{\today}
What is the question?
\score{2}
\shortanswer{To be or not to be.}
\end{problem}
%</exa>
%    \end{macrocode}
%
%    \begin{macrocode}
%<*exb>
\parameterproblem{1= to be\\2= not to be}
\problemdate{\today}
\begin{problem}
\score{2}
What is\ifnum\parameter=1\relax\else n't\fi\ the question?
\shortanswer{\ifnum\parameter=1\relax To be or n\else N\fi ot to be.}
\end{problem}
%</exb>
%    \end{macrocode}
% \Finale
%
