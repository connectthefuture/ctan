% This file is part of emf.

% emf is free software: you can redistribute it and/or modify it under
% the terms of the GNU General Public License as published by the Free
% Software Foundation, either version 3 of the License, or (at your
% option) any later version.

% emf is distributed in the hope that it will be useful, but WITHOUT
% ANY WARRANTY; without even the implied warranty of MERCHANTABILITY
% or FITNESS FOR A PARTICULAR PURPOSE. See the GNU General Public
% License for more details.

% You should have received a copy of the GNU General Public License
% along with emf. If not, see <http://www.gnu.org/licenses/>.
\documentclass[11pt,a4paper,british]{scrartcl}
\usepackage[T1]{fontenc} %
\usepackage[utf8]{inputenc} %
\usepackage{babel} %
\usepackage{emf}
\usepackage{listings}
\title{The \texttt{emf} package}
\author{Palle Jørgensen\thanks{\texttt{hamselv@pallej.dk}}}
\setkomafont{descriptionlabel}{\bfseries}

%% Setting up for displaying the various symbols
\makeatletter
%% Calligra
\DeclareFontFamily{T1}{calligra}{}%
\DeclareFontShape{T1}{calligra}{m}{n}{<->s*[1.07]callig15}{}%
\DeclareMathAlphabet{\mathemfcalligra}{T1}{calligra}{m}{n}%
\def\emfcalligra{\mathemfcalligra{\mkern-7muE\mkern4mu}}%
%% Chorus
\def\qzc@scale{1.22}%
\DeclareMathAlphabet{\mathemfchorus}{OT1}{qzc}{m}{n}%
\def\emfchorus{\mathemfchorus{E}}
%% Fourier
\DeclareFontEncoding{FMS}{}{}%
\DeclareFontSubstitution{FMS}{futm}{m}{n}%
\DeclareMathAlphabet{\mathemffourier}{FMS}{futm}{m}{n}%
\def\emffourier{\mathemffourier{E}}%
%% French Cursive
\DeclareMathAlphabet{\mathemffrcursive}{T1}{frc}{m}{sl}%
\def\emffrcursive{\mathemffrcursive{E}}%
%% Miama Nueva
\def\fmm@scale{0.85}%
\DeclareMathAlphabet{\mathemfmiama}{OT1}{fmm}{m}{n}%
\def\emfmiama{\mathemfmiama{E}}
%% RSFS
\DeclareMathAlphabet{\mathemfrsfs}{OMS}{rsfs}{m}{n}%
\def\emfrsfs{\mathemfrsfs{E}}%
\makeatother


\begin{document}
\maketitle

\section{Introduktion}
\label{sec:introduktion}

This package provides \LaTeX\ support for the symbol for the EMF
($\emf$) in electric circuits and electrodynamics. It provides support
for multiple symbols but does not provide any fonts. The fonts
themselves must be aquired otherwise. However the fonts are part of a
normal \TeX Live installation.

\section{Usage}
\label{sec:usage}

\subsection{Loading the package}
\label{sec:loading-package}
\begin{lstlisting}
\usepackage{emf}
\end{lstlisting}

\subsection{Options}
\label{sec:options}
\begin{description}
\item[scale=$\langle num\rangle$] Set a scale factor for the font. As
  this package as much as possible relies on existing font support the
  `scale' option only applies to some of the fonts. Some of the fonts
  are scaled as default. Thus the `scale' option changes this as the
  scale factors are not multiplied.

\item[boondox] Use the calligraphical E from the STIX fonts: $\emf$.
  This is the default of the package. It is possible to scale this
  symbol. The default scale is 1.1.

\item[calligra] Use the calligraphical E from the Calligra font:
  $\emfcalligra$. It is possible to scale this
  symbol. The default scale is 1.07.

\item[chorus] Use the calligraphical E from the \TeX\ Gyre Chorus
  font (Chancery): $\emfchorus$. It is
  possible to scale this symbol. The default scale is 1.22.

\item[cmr] Use the calligraphical E from Computer Modern: $\mathcal{E}$.

\item[fourier] Use the calligraphical E from the fourier fonts
  (Utopia like): $\emffourier$.

\item[frcursive] Use the E from the French Cursive font: $\emffrcursive$.

\item[miama] Use the E from the Miama Nueva font: $\emfmiama$. It is
  possible to scale this symbol. The default scale is 0.85.

\item[rsfs] Use the E from \textsl{Ralph Smiths Formal Script:} $\emfrsfs$.

\item[cal] Use the default calligraphical E of the document. In this
  document from Computer Modern: $\mathcal{E}$. This option depends on
  the math fonts loaded by the user.

\item[u0] Use the notation $U_0$ as a symbol for EMF.

\item[U0] The same as otion `u0'.

\item[text=$\langle \mathit{text}\rangle$] Use $\mathit{text}$ as
  symbol. The default is $\mathit{EMF}$.
\end{description}

\subsection{Command}
\label{sec:command}

To access the selected symbol just type \lstinline|\emf| in your
document. \lstinline|\emf| only applies to math mode.

\section{License}
\label{sec:license}

This file is part of emf.

emf is free software: you can redistribute it and/or modify it under
the terms of the GNU General Public License as published by the Free
Software Foundation, either version 3 of the License, or (at your
option) any later version.

emf is distributed in the hope that it will be useful, but WITHOUT ANY
WARRANTY; without even the implied warranty of MERCHANTABILITY or
FITNESS FOR A PARTICULAR PURPOSE. See the GNU General Public License
for more details.

You should have received a copy of the GNU General Public License
along with emf. If not, see <http://www.gnu.org/licenses/>.
\end{document}

%%% Local Variables:
%%% mode: latex
%%% TeX-master: t
%%% End:
