\documentclass[DIV=10, fontsize=12pt, pagesize=auto]{scrartcl}

\usepackage{lmodern}
\usepackage[T1]{fontenc}
\usepackage{booktabs}
\usepackage{microtype}

\addtokomafont{disposition}{\rmfamily}
\addtokomafont{descriptionlabel}{\ttfamily}

\title{The \textsf{acromake} package}
\author{Paul A. Thompson}
\date{1995/7/16}


\begin{document}

\maketitle

\noindent
This algorithm is presented and discussed in \TeX\ and TUG News, V4, \#1.
The algorithm as presented there has several small problems, which are
solved here.

This file generates acronyms.  There are three parameters which may be
set by the user
%
\begin{enumerate}
\item To generate an acronym \\
  \verb+\acromk{ANSA}{{\tt ANSA}}{another nasty stupid acronym}+
  
  The first argument is the name used in text to implement the acronym.
  The second argument is the name printed in the text when the acronym
  is used.  The final argument is the full, non-acronymized version.
  
  So, if there was an acronym \\
  \verb+\acromake{ANSA}{{\tt ANSA}}{another nasty stupid acronym}+ \\
  it could be used as 
  %
\begin{verbatim}
\ANSA is a new approach to the use of \ANSA, where \ANSA is ...
\end{verbatim}
  
  This would result in
  %
  \begin{quote}
    another nasty stupid acronym ({\tt ANSA}) is a new approach to 
    the use of {\tt ANSA}, where {\tt ANSA} is ...
  \end{quote}
\item To use the acronym
  %
\begin{verbatim}
here \ANSA whatever {foo}lish use of acronym you wish
\end{verbatim}
  
  This will generate the following:\nopagebreak

  \begin{tabular}{@{}ll@{}}
    \toprule
    Usage number   & Resolved text                       \\
    \midrule
    1              & another nasty stupid acronym (ANSA) \\
    2              & ANSA (page 2)                       \\
    3 (and beyond) & ANSA                                \\
    \bottomrule
  \end{tabular}
  \medskip
\item The following may be set by the user:
  %
  \begin{description}
  \item[ACRcnta] count of uses of acronym macro which will be in full form  
    (default is 1)
  \item[ACRcntb] count of uses of acronym macro which will include page of
    acronym def (default is 1 reference, or $\mathrm{ACRcntb}-\mathrm{ACRcnta}$)
  \item[useACRlb] should the page reference be used --- 1--yes, 0--no
  \end{description}
\end{enumerate}

For every acronym, the macro generates the acronym macro and a macro which
acts like a counter for acronym use.  Since counters are quite rare, it is
not responsible to use a counter for every acronym -- rather, a macro is
generated which fills this function.  The acronym macro is called \verb+\foo+,
while the counter macro is called \verb+\fooz+.  Thus, I would avoid calling any
acronym \textsf{fooz}, as this may be confoozing.

The \texttt{.sty} file generates 1 counter if it is included.  For each acronym
specified, the \texttt{.sty} file generates two macros, 1 for the acronym, and 1 for
the count of acronym use

\end{document}
