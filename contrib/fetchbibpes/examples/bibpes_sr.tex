\documentclass{article}
\usepackage{etoolbox}
\usepackage{bibleref-parse}
\usepackage[deffolder=exmpldefs,usesr,
    useverses=verses]{fetchbibpes}
\usepackage{makeidx}
\usepackage[columns=1]{idxlayout}

\makeatletter
\renewcommand{\l@section}{\@dottedtocline{1}{1.5em}{4.75em}}
\makeatother
\renewcommand{\indexname}{Index to Scriptures}

\setbooktitle{Genesis}{Gen}
\setindexbooktitle{Gen}{Genesis (Gen)}
\biblerefmap{Gen}{01}
\setbooktitle{Matthew}{Mat}
\setindexbooktitle{Matt}{Matthew (Mat)}
\biblerefmap{Mat}{40}
\setbooktitle{ICorinthians}{1Co}
\setindexbooktitle{1Cor}{1 Corinthians (1Co)}
\biblerefmap{1Co}{46}


\makeindex

\title{The \textsf{fetchbibpes} package\\[3pt]Illustrating the \textsf{sr} command}
\author{D. P. Story}
\date{\today}

\begin{document}

\maketitle

\tableofcontents

\section{Verses from Genesis}

   Moses, the author of the Book of Genesis, describes creation, \sr{for it is
   written}{Gen 1:1-10}: \fetchverses[showfirst]{Gen 1:1-10}
   
   \sr{}{Php 2:13} ``\fetchverses{Php 2:13}''

\section{Some words from Matthew}

    \sr{In the times of old}{Mat 2:1-6, 2:9-10} \fetchverses[showfirst]{Mat 2:1-6}
    \emph{And the passage continues with verse 9}, \fetchverses[showfirst]{Mat 2:9-10}


\section{Passages from I Corinthians}

Paul greets the Church at Corinthian \sr{}{1Co 1:1-3} \fetchverses{1Co 1:1-3}


% Index to Scriptures
    \newpage
    \printindex

\end{document}
