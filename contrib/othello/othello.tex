\documentclass[12pt,twoside]{report}

\usepackage{othello}
\usepackage{graphicx}
\usepackage{makeidx}
\usepackage{floatfig}

\def\startposition{
\inifulldiagram
\gofontsize{10}
\pos{d}{5}=\black{.}
\pos{d}{4}=\white{.}
\pos{e}{5}=\white{.}
\pos{e}{4}=\black{.}
\centerline{\showfulldiagram}\caption{Othello start position}
}

\def\cornerpositions{
\inifulldiagram
\gofontsize{10}
\pos{b}{2}=\neutral{65}
\pos{b}{7}=\neutral{65}
\pos{g}{2}=\neutral{65}
\pos{g}{7}=\neutral{65}
\pos{a}{2}=\neutral{66}
\pos{a}{7}=\neutral{66}
\pos{b}{1}=\neutral{66}
\pos{b}{8}=\neutral{66}
\pos{g}{8}=\neutral{66}
\pos{g}{1}=\neutral{66}
\pos{h}{7}=\neutral{66}
\pos{h}{2}=\neutral{66}
\centerline{\showfulldiagram}\caption{Special positions around the corners} \label{cornerpositions}
}

\def\makingamove{
\inifulldiagram
\gofontsize{10}
\pos{d}{5}=\black{.}
\pos{d}{4}=\white{.}
\pos{e}{5}=\white{.}
\pos{e}{4}=\black{.}
\pos{c}{4}=\neutral{65}
\pos{d}{3}=\neutral{65}
\pos{e}{6}=\neutral{65}
\pos{f}{5}=\neutral{65}
\centerline{\showfulldiagram}\caption{Black can place his first stone at these
places}\label{makingamove}
}

\def\validmoves{
\inifulldiagram
\gofontsize{10}
\pos{d}{5}=\black{.}
\pos{d}{4}=\white{.}
\pos{e}{5}=\white{.}
\pos{e}{4}=\black{.}
\pos{d}{3}=\black{.}
\pos{c}{3}=\white{.}
\centerline{\showfulldiagram}\label{validmoves}\caption{Opening
c4,c3}
}

\def\fliponlyonce{
\inifulldiagram \gofontsize{10} \pos{c}{3}=\white{.}
\pos{d}{3}=\white{.} \pos{e}{3}=\white{.} \pos{f}{3}=\white{.}
\pos{f}{4}=\white{.} \pos{f}{5}=\white{.} \pos{f}{6}=\white{.}
\pos{e}{6}=\white{.} \pos{d}{6}=\white{.} \pos{c}{6}=\white{.}
\pos{d}{4}=\black{.} \pos{e}{4}=\black{.} \pos{d}{5}=\black{.}
\pos{e}{5}=\black{.}
\centerline{\showfulldiagram}\caption{Stones do not flip recursively}\label{fliponlyonce} }

\def\stablestones{
\inifulldiagram \gofontsize{10} \pos{a}{1}=\black{.}
\pos{b}{1}=\black{.} \pos{a}{2}=\black{.} \pos{a}{3}=\black{.}
\pos{b}{2}=\black{.} \pos{a}{4}=\black{.} \pos{h}{1}=\white{.}
\pos{h}{2}=\white{.} \pos{g}{1}=\white{.} \pos{g}{2}=\white{.}
\pos{g}{3}=\white{.} \pos{h}{3}=\white{.} \pos{f}{1}=\white{.}
\centerline{\showfulldiagram}\caption{Example of stable
stones}\label{stablestones} }

\def\maximumdisk{
\inifulldiagram
\gofontsize{10}
\pos{a}{3}=\white{.}
\pos{a}{4}=\white{.}
\pos{a}{5}=\white{.}
\pos{a}{6}=\white{.}
\pos{b}{3}=\white{.}
\pos{b}{4}=\white{.}
\pos{b}{5}=\white{.}
\pos{b}{6}=\white{.}
\pos{c}{1}=\white{.}
\pos{c}{2}=\white{.}
\pos{c}{3}=\white{.}
\pos{c}{4}=\white{.}
\pos{c}{5}=\white{.}
\pos{c}{6}=\white{.}
\pos{c}{7}=\white{.}
\pos{c}{8}=\white{.}
\pos{d}{1}=\white{.}
\pos{d}{2}=\white{.}
\pos{d}{3}=\white{.}
\pos{d}{4}=\black{.}
\pos{d}{5}=\black{.}
\pos{d}{6}=\white{.}
\pos{d}{7}=\white{.}
\pos{d}{8}=\white{.}
\pos{e}{1}=\white{.}
\pos{e}{2}=\white{.}
\pos{e}{3}=\white{.}
\pos{e}{4}=\white{.}
\pos{e}{5}=\white{.}
\pos{e}{6}=\white{.}
\pos{e}{7}=\white{.}
\pos{e}{8}=\white{.}
\pos{f}{1}=\white{.}
\pos{f}{2}=\white{.}
\pos{f}{3}=\white{.}
\pos{f}{4}=\white{.}
\pos{f}{5}=\white{.}
\pos{f}{6}=\white{.}
\pos{f}{7}=\white{.}
\pos{f}{8}=\white{.}
\pos{g}{3}=\white{.}
\pos{g}{4}=\white{.}
\pos{g}{5}=\white{.}
\pos{g}{6}=\white{.}
\pos{h}{3}=\white{.}
\pos{h}{4}=\white{.}
\pos{h}{5}=\white{.}
\pos{h}{6}=\white{.}
\centerline{\showfulldiagram}\caption{Maximum disk strategy: black to play and win (46-18)}\label{maximumdisk}
}

\def\stabledisks{
\inifulldiagram \gofontsize{10} \pos{a}{1}=\white{.}
\pos{a}{2}=\white{.} \pos{a}{3}=\white{.} \pos{a}{4}=\white{.}
\pos{a}{5}=\white{.} \pos{a}{6}=\white{.} \pos{b}{1}=\white{.}
\pos{b}{2}=\white{.} \pos{b}{3}=\white{.} \pos{b}{4}=\white{.}
\pos{b}{5}=\white{.} \pos{b}{6}=\white{.} \pos{c}{1}=\white{.}
\pos{c}{2}=\white{.} \pos{c}{3}=\white{.} \pos{c}{4}=\white{.}
\pos{c}{5}=\white{.} \pos{c}{6}=\white{.} \pos{c}{7}=\white{.}
\pos{c}{8}=\white{.} \pos{d}{1}=\white{.} \pos{d}{2}=\white{.}
\pos{d}{3}=\white{.} \pos{d}{4}=\black{.} \pos{d}{5}=\black{.}
\pos{d}{6}=\white{.} \pos{d}{7}=\white{.} \pos{d}{8}=\white{.}
\pos{e}{1}=\white{.} \pos{e}{2}=\white{.} \pos{e}{3}=\white{.}
\pos{e}{4}=\white{.} \pos{e}{5}=\white{.} \pos{e}{6}=\white{.}
\pos{e}{7}=\white{.} \pos{e}{8}=\white{.} \pos{f}{1}=\white{.}
\pos{f}{2}=\white{.} \pos{f}{3}=\white{.} \pos{f}{4}=\white{.}
\pos{f}{5}=\white{.} \pos{f}{6}=\white{.} \pos{f}{7}=\white{.}
\pos{f}{8}=\white{.} \pos{g}{3}=\white{.} \pos{g}{4}=\white{.}
\pos{g}{5}=\white{.} \pos{g}{6}=\white{.} \pos{h}{3}=\white{.}
\pos{h}{4}=\white{.} \pos{h}{5}=\white{.} \pos{h}{6}=\white{.}
\centerline{\showfulldiagram}\caption{Examples of stable disks in an end game}\label{stabledisks} }



\def\quietmove{\inifulldiagram
\gofontsize{10}
\pos{b}{5}=\black{.}
\pos{c}{2}=\black{.}
\pos{d}{2}=\black{.}
\pos{e}{2}=\black{.}
\pos{f}{2}=\black{.}
\pos{f}{3}=\black{.}
\pos{g}{3}=\black{.}
\pos{g}{4}=\black{.}
\pos{b}{3}=\white{.}
\pos{b}{4}=\white{.}
\pos{c}{3}=\white{.}
\pos{c}{4}=\white{.}
\pos{c}{5}=\white{.}
\pos{d}{3}=\white{.}
\pos{d}{4}=\white{.}
\pos{d}{5}=\white{.}
\pos{e}{3}=\white{.}
\pos{e}{5}=\white{.}
\pos{f}{4}=\white{.}
\pos{f}{5}=\white{.}
\pos{g}{5}=\white{.}
\centerline{\showfulldiagram}\caption{Example of a quiet move}\label{quietmove}}

\def\frontiermovetwo{\inifulldiagram
\gofontsize{10}
\pos{b}{4}=\black{.}
\pos{b}{5}=\black{.}
\pos{c}{2}=\black{.}
\pos{c}{3}=\black{.}
\pos{d}{2}=\black{.}
\pos{d}{3}=\black{.}
\pos{e}{2}=\black{.}
\pos{e}{3}=\black{.}
\pos{f}{3}=\black{.}
\pos{f}{4}=\black{.}
\pos{c}{4}=\white{.}
\pos{c}{5}=\white{.}
\pos{d}{4}=\white{.}
\pos{d}{5}=\white{.}
\pos{e}{4}=\white{.}
\pos{e}{5}=\white{.}
\centerline{\showfulldiagram}\caption{}\label{frontiermovetwo}}

\def\frontierstones{\inifulldiagram
\gofontsize{10}
\pos{b}{3}=\black{.}
\pos{c}{3}=\black{.}
\pos{c}{4}=\black{.}
\pos{d}{3}=\black{.}
\pos{d}{4}=\black{.}
\pos{d}{5}=\black{.}
\pos{d}{6}=\black{.}
\pos{d}{7}=\black{.}
\pos{e}{3}=\black{.}
\pos{e}{4}=\black{.}
\pos{e}{5}=\black{.}
\pos{f}{3}=\black{.}
\pos{f}{4}=\black{.}
\pos{f}{5}=\black{.}
\pos{b}{4}=\white{.}
\pos{b}{5}=\white{.}
\pos{c}{5}=\white{.}
\pos{c}{6}=\white{.}
\pos{c}{7}=\white{.}
\pos{e}{6}=\white{.}
\pos{e}{7}=\white{.}
\pos{f}{6}=\white{.}
\centerline{\showfulldiagram}\caption{Frontiers in the beginning of the game}\label{frontierstones}}

\def\mobilityexampletwo{\inifulldiagram
\gofontsize{10}
\pos{a}{4}=\black{.}
\pos{b}{3}=\black{.}
\pos{b}{4}=\black{.}
\pos{b}{5}=\black{.}
\pos{b}{6}=\black{.}
\pos{c}{3}=\black{.}
\pos{c}{6}=\black{.}
\pos{d}{3}=\black{.}
\pos{d}{4}=\black{.}
\pos{d}{5}=\black{.}
\pos{d}{6}=\black{.}
\pos{e}{3}=\black{.}
\pos{e}{4}=\black{.}
\pos{e}{5}=\black{.}
\pos{c}{4}=\white{.}
\pos{c}{5}=\white{.}
\centerline{\showfulldiagram}\caption{White to play and win a corner}\label{mobilityexampletwo}}

\def\mobiuitween{\inifulldiagram
\gofontsize{10}
\pos{a}{4}=\white{.}
\pos{b}{3}=\black{.}
\pos{b}{4}=\black{.}
\pos{b}{5}=\black{.}
\pos{b}{6}=\black{.}
\pos{c}{3}=\black{.}
\pos{c}{6}=\black{.}
\pos{d}{3}=\black{.}
\pos{d}{4}=\black{.}
\pos{d}{5}=\black{.}
\pos{d}{6}=\black{.}
\pos{e}{3}=\black{.}
\pos{e}{4}=\black{.}
\pos{e}{5}=\black{.}
\pos{c}{4}=\white{.}
\pos{c}{5}=\white{.}
\pos{b}{7}=\black{.}
\pos{a}{2}=\white{.}
\pos{a}{3}=\white{.}
\pos{a}{7}=\white{.}
\pos{a}{5}=\white{.}
\pos{a}{6}=\white{.}
\centerline{\showfulldiagram}\caption{}\label{mobiuitween}}

\def\frontiermoveone{\inifulldiagram
\gofontsize{10}
\pos{a}{2}=\black{.}
\pos{b}{3}=\black{.}
\pos{c}{1}=\black{.}
\pos{c}{2}=\black{.}
\pos{d}{1}=\black{.}
\pos{d}{2}=\black{.}
\pos{e}{1}=\black{.}
\pos{e}{2}=\black{.}
\pos{f}{1}=\black{.}
\pos{f}{2}=\black{.}
\pos{a}{4}=\white{.}
\pos{a}{5}=\white{.}
\pos{a}{6}=\white{.}
\pos{a}{7}=\white{.}
\pos{b}{4}=\white{.}
\pos{b}{5}=\white{.}
\pos{b}{6}=\white{.}
\pos{b}{8}=\white{.}
\pos{c}{3}=\white{.}
\pos{c}{4}=\white{.}
\pos{c}{5}=\white{.}
\pos{c}{6}=\white{.}
\pos{c}{7}=\white{.}
\pos{c}{8}=\white{.}
\pos{d}{3}=\white{.}
\pos{d}{4}=\white{.}
\pos{d}{5}=\white{.}
\pos{d}{6}=\white{.}
\pos{d}{7}=\white{.}
\pos{d}{8}=\white{.}
\pos{e}{3}=\white{.}
\pos{e}{4}=\white{.}
\pos{e}{5}=\white{.}
\pos{e}{6}=\white{.}
\pos{e}{7}=\white{.}
\pos{f}{3}=\white{.}
\pos{f}{4}=\white{.}
\pos{f}{5}=\white{.}
\pos{f}{6}=\white{.}
\pos{f}{7}=\white{.}
\centerline{\showfulldiagram}\caption{Frontier moves: h6 is devastating for black}\label{frontiermoveone}
}

\def\wedge{
\inifulldiagram
\gofontsize{10}
\pos{a}{2}=\black{.}
\pos{a}{3}=\black{.}
\pos{a}{4}=\black{.}
\pos{a}{5}=\black{.}
\pos{a}{6}=\black{.}
\pos{b}{1}=\black{.}
\pos{b}{3}=\black{.}
\pos{b}{4}=\black{.}
\pos{b}{5}=\black{.}
\pos{b}{6}=\black{.}
\pos{c}{1}=\black{.}
\pos{c}{2}=\black{.}
\pos{c}{3}=\black{.}
\pos{c}{6}=\black{.}
\pos{d}{2}=\black{.}
\pos{d}{5}=\black{.}
\pos{d}{6}=\black{.}
\pos{e}{1}=\black{.}
\pos{e}{2}=\black{.}
\pos{e}{4}=\black{.}
\pos{f}{1}=\black{.}
\pos{g}{1}=\black{.}
\pos{c}{4}=\white{.}
\pos{c}{5}=\white{.}
\pos{d}{3}=\white{.}
\pos{d}{4}=\white{.}
\pos{e}{3}=\white{.}
\pos{e}{5}=\white{.}
\pos{e}{6}=\white{.}
\pos{f}{2}=\white{.}
\pos{f}{3}=\white{.}
\pos{f}{4}=\white{.}
\pos{f}{5}=\white{.}
\centerline{\showfulldiagram}\caption{White plays a4, wedging the first column}\label{wedge}}

\def\wedgetwo{\inifulldiagram
\gofontsize{10}
\pos{a}{3}=\black{.}
\pos{a}{4}=\black{.}
\pos{a}{5}=\black{.}
\pos{a}{6}=\black{.}
\pos{a}{7}=\black{.}
\pos{b}{1}=\black{.}
\pos{b}{3}=\black{.}
\pos{b}{4}=\black{.}
\pos{b}{5}=\black{.}
\pos{b}{6}=\black{.}
\pos{c}{1}=\black{.}
\pos{c}{2}=\black{.}
\pos{c}{3}=\black{.}
\pos{c}{6}=\black{.}
\pos{d}{1}=\black{.}
\pos{d}{5}=\black{.}
\pos{d}{6}=\black{.}
\pos{e}{1}=\black{.}
\pos{e}{4}=\black{.}
\pos{f}{1}=\black{.}
\pos{f}{2}=\black{.}
\pos{g}{1}=\black{.}
\pos{c}{4}=\white{.}
\pos{c}{5}=\white{.}
\pos{d}{2}=\white{.}
\pos{d}{3}=\white{.}
\pos{d}{4}=\white{.}
\pos{e}{2}=\white{.}
\pos{e}{3}=\white{.}
\pos{e}{5}=\white{.}
\pos{e}{6}=\white{.}
\pos{f}{3}=\white{.}
\pos{f}{4}=\white{.}
\pos{f}{5}=\white{.}
\centerline{\showfulldiagram}\caption{Wedging by sacrificing a corner}\label{wedgetwo}
}

\def\tempo{
\inifulldiagram
\gofontsize{10}
\pos{b}{3}=\black{.}
\pos{b}{4}=\black{.}
\pos{b}{5}=\black{.}
\pos{b}{6}=\black{.}
\pos{c}{2}=\black{.}
\pos{c}{3}=\black{.}
\pos{c}{6}=\black{.}
\pos{d}{2}=\black{.}
\pos{d}{4}=\black{.}
\pos{d}{5}=\black{.}
\pos{d}{6}=\black{.}
\pos{e}{2}=\black{.}
\pos{e}{4}=\black{.}
\pos{e}{5}=\black{.}
\pos{e}{6}=\black{.}
\pos{f}{2}=\black{.}
\pos{f}{6}=\black{.}
\pos{a}{3}=\white{.}
\pos{a}{4}=\white{.}
\pos{a}{5}=\white{.}
\pos{c}{4}=\white{.}
\pos{c}{5}=\white{.}
\pos{d}{3}=\white{.}
\pos{e}{3}=\white{.}
\pos{f}{3}=\white{.}
\pos{f}{4}=\white{.}
\pos{f}{5}=\white{.}
\centerline{\showfulldiagram}\caption{White to move: gaining tempo}\label{tempo}
}

\def\parity{
\inifulldiagram
\gofontsize{10}
\pos{b}{5}=\black{.}
\pos{b}{6}=\black{.}
\pos{d}{4}=\black{.}
\pos{d}{6}=\black{.}
\pos{e}{3}=\black{.}
\pos{e}{5}=\black{.}
\pos{f}{2}=\black{.}
\pos{f}{4}=\black{.}
\pos{g}{1}=\black{.}
\pos{g}{6}=\black{.}
\pos{a}{3}=\white{.}
\pos{a}{4}=\white{.}
\pos{a}{5}=\white{.}
\pos{a}{6}=\white{.}
\pos{a}{7}=\white{.}
\pos{b}{1}=\white{.}
\pos{b}{3}=\white{.}
\pos{b}{4}=\white{.}
\pos{c}{1}=\white{.}
\pos{c}{2}=\white{.}
\pos{c}{3}=\white{.}
\pos{c}{4}=\white{.}
\pos{c}{5}=\white{.}
\pos{c}{6}=\white{.}
\pos{c}{7}=\white{.}
\pos{c}{8}=\white{.}
\pos{d}{1}=\white{.}
\pos{d}{2}=\white{.}
\pos{d}{3}=\white{.}
\pos{d}{5}=\white{.}
\pos{d}{7}=\white{.}
\pos{d}{8}=\white{.}
\pos{e}{1}=\white{.}
\pos{e}{2}=\white{.}
\pos{e}{4}=\white{.}
\pos{e}{6}=\white{.}
\pos{e}{7}=\white{.}
\pos{e}{8}=\white{.}
\pos{f}{1}=\white{.}
\pos{f}{3}=\white{.}
\pos{f}{5}=\white{.}
\pos{f}{6}=\white{.}
\pos{f}{7}=\white{.}
\pos{f}{8}=\white{.}
\pos{g}{2}=\white{.}
\pos{g}{3}=\white{.}
\pos{g}{4}=\white{.}
\pos{g}{5}=\white{.}
\pos{g}{8}=\white{.}
\pos{h}{1}=\white{.}
\pos{h}{2}=\white{.}
\pos{h}{3}=\white{.}
\pos{h}{4}=\white{.}
\pos{h}{5}=\white{.}
\pos{h}{6}=\white{.}
\pos{h}{7}=\white{.}
\centerline{\showfulldiagram}\caption{Black to play and win}\label{parity}}

\def\paritytwo{
\inifulldiagram
\gofontsize{10}
\pos{c}{5}=\black{.}
\pos{c}{7}=\black{.}
\pos{d}{1}=\black{.}
\pos{d}{2}=\black{.}
\pos{d}{3}=\black{.}
\pos{d}{4}=\black{.}
\pos{e}{1}=\black{.}
\pos{e}{2}=\black{.}
\pos{f}{2}=\black{.}
\pos{f}{3}=\black{.}
\pos{f}{6}=\black{.}
\pos{f}{8}=\black{.}
\pos{g}{3}=\black{.}
\pos{g}{4}=\black{.}
\pos{g}{5}=\black{.}
\pos{g}{8}=\black{.}
\pos{h}{3}=\black{.}
\pos{h}{4}=\black{.}
\pos{h}{5}=\black{.}
\pos{h}{6}=\black{.}
\pos{h}{7}=\black{.}
\pos{h}{8}=\black{.}
\pos{a}{2}=\white{.}
\pos{a}{3}=\white{.}
\pos{a}{4}=\white{.}
\pos{a}{5}=\white{.}
\pos{a}{6}=\white{.}
\pos{a}{7}=\white{.}
\pos{a}{8}=\white{.}
\pos{b}{3}=\white{.}
\pos{b}{4}=\white{.}
\pos{b}{5}=\white{.}
\pos{b}{6}=\white{.}
\pos{b}{7}=\white{.}
\pos{b}{8}=\white{.}
\pos{c}{1}=\white{.}
\pos{c}{2}=\white{.}
\pos{c}{3}=\white{.}
\pos{c}{4}=\white{.}
\pos{c}{6}=\white{.}
\pos{c}{8}=\white{.}
\pos{d}{5}=\white{.}
\pos{d}{6}=\white{.}
\pos{d}{7}=\white{.}
\pos{d}{8}=\white{.}
\pos{e}{3}=\white{.}
\pos{e}{4}=\white{.}
\pos{e}{5}=\white{.}
\pos{e}{6}=\white{.}
\pos{e}{7}=\white{.}
\pos{e}{8}=\white{.}
\pos{f}{1}=\white{.}
\pos{f}{4}=\white{.}
\pos{f}{5}=\white{.}
\pos{f}{7}=\white{.}
\pos{g}{6}=\white{.}
\centerline{\showfulldiagram}\caption{Black to play and draw}\label{paritytwo}}

\def\overviewboard{
\inifulldiagram
\gofontsize{10}
\pos{d}{5}=\black{.}
\pos{d}{4}=\white{.}
\pos{e}{5}=\white{23}
\pos{e}{4}=\black{62}
\pos{d}{3}=\neutral{43}
\pos{b}{2}=\neutral{65}
\pos{b}{1}=\neutral{66}
\centerline{\showfulldiagram}\caption{All possible characters using  the Othello package}
}

\def\overviewboardcorner{
\inifulldiagram
\gofontsize{10}
\pos{d}{5}=\black{.}
\pos{d}{4}=\white{.}
\pos{e}{5}=\white{23}
\pos{e}{4}=\black{62}
\pos{d}{3}=\neutral{43}
\pos{b}{2}=\neutral{65}
\pos{b}{1}=\neutral{66}
\centerline{\showdiagram a-d:1-4 }\caption{The northwest corner of the overview board}
}


\makeindex

\newcommand{\clearemptydoublepage}{\newpage{\pagestyle{empty}\cleardoublepage}}
\begin{document}
\initfloatingfigs

\title{Introduction to Othello}
\author{M. le Comte}
\maketitle

\clearemptydoublepage
\chapter{preface}
This manual is intended as an introduction to the game Othello. It
has been written using Latex, using a package called othello.sty.
This package is used to create all Othello pictures in this
manual. For more information, see the last chapter. If anyone has
any suggestions or remarks on either the package or the content of
this manual, please mail me at michiel.le.comte@zonnet.nl
\clearemptydoublepage
\tableofcontents
\chapter{A short introduction to Othello}
\section{The history of Othello}
\section{The rules of Othello}

\subsection{Starting the game}
\begin{figure}[h]
\startposition
\end{figure}


Othello is a game that is played on a $8\times 8$\ board using
stones that are black on one side, and white on the other.
One player uses the black side of the stones, the other the white sides.
The starting position is as follows: four stones are placed in the middle, two
white and two black, each color getting a diagonal. Thus the
starting position is as in figure 2.1. The player using the black
stones always makes the first move. The players take
turns placing a stone. Following standard Othello notation, we will number
the rows by $1\ldots 8$\ and the columns by a$\ldots$h. For
example, the begin positions for black are d5 and e4, and for
white d4 and e5. In addition, we will also use ``compass''
directions, i.e. west when we mean the left side of the board,
north for the upper side etc. There are a few places on the Othello
board that are important enough to deserve a special name. The
squares a1,a8,h1 and h8 are called the \index{corners}corners. There are 3 squares
adjacent to a1, two of which are on the edge of the board (a2 and
b1). These two stones are called C squares\index{C square}. The third square,
situated on the diagonal, i.e. square b2, is called an \index{X square}X square.
The same notation also applies for the other corners, so there are
4 X squares and 8 C squares. (A mnemonic: the two diagonals form
an X) Figure \ref{cornerpositions} shows these special stones.

The four squares already occupied in the starting
position are called the center squares, for obvious
reasons.\index{center squares}

\begin{figure}[ht]
\cornerpositions
\end{figure}


\subsection{Making a move}
At his turn, a player must place one of his stones on the board. He may
place a stone on one of the empty squares of the board, adjacent to
a disk of the opponent. In addition, the stone also must flank one or more of
his opponent's disks between his new stone and one or more of his
other stones which already are on the board. He then changes the stones
of the opponent which were flanked to his color, by flipping the stones.

\begin{figure}[h]
\makingamove
\end{figure}
As mentioned, the black player is the first to move. His valid moves are
c4,d3,e6,f5. These squares are shown in figure \ref{makingamove},
where X marks the spot.
Suppose that he chooses c4, then the white disk d4 is flipped to black, and then it is
white's turn to move. White now has 3 different spots to place his disk, c3, e3
and c5. If he chooses c3, the situation then becomes as figure \ref{validmoves}.

\begin{floatingfigure}{4cm}
\validmoves
\end{floatingfigure}
Note that disks which are flipped may not be used to flip even
more disks in the same turn. For example: if we have the board
from figure \ref{fliponlyonce}, and suppose that white has to make
a move. The choices for white are d3 and e3. Both moves will flip
3 of the black stones, and the fourth one will be completely
surrounded by white stones, but it will not be flipped.
\begin{figure} [ht]
\fliponlyonce
\end{figure}

\subsection{Passing} If a player can not make a move that flips at least one of his
opponents disks, then he has to pass. If he is able to make a valid move however, then
passing is not allowed. It is possible that a player has to pass several times
before he can make a move again.

\subsection{The end of Othello}
The game ends when neither player can  make a valid move. This  usually happens
when all 64 squares are filled, but
sometimes it ends earlier. The stones then are counted. The winner is the player who has more discs
then his opponent. If both players have 32 stones, then it is a draw.

Looking at figure \ref{fliponlyonce}, you can see that the white
player can easily win. After either of his moves, the black player
has only remaining stone. It does not matter which move black
makes, the white player can always flank all the stones in the
next turn. This is an example where the game ends before the board
is completely filled.

\clearemptydoublepage
\chapter{The strategy of Othello}
\section{The maximum disk strategy}\index{maximum disk strategy}
Playing Othello is like every game that is not based entirely on luck, if you want
to win, you need a strategy. The most obvious strategy is not always the best
though. When people first learn to play Othello, they are usually tempted to get
as many stones they can, whenever they can.
This strategy is known as the maximum disk strategy.
\begin{floatingfigure}{4cm}
\maximumdisk
\end{floatingfigure}
Having many stones in the beginning is
not at all a guarantee that you will win in the end however! Consider for example
the position in figure \ref{maximumdisk}.
White has 46 stones and black only 2, with 16 remaining moves.
White however is very limited in its moves, which allows black to
force white into making less then perfect moves. If black plays
all moves right, and white plays the best possible moves too, then
white will end up with all the corners, but black will have 46
stones. You may want to try to find the best play yourself, a
solution in which black wins is not that hard, but 46 stones is a
little
puzzle\footnote{Solution:b2,a1,b1,-,a2,-,b7,a8,a7,-,g2,h1,g1,-,g8,h8,b8,g7,h2,-,h7}
(the answer is given at the bottom of the page).

There are several reasons why the maximum disk strategy usually fails. The two most
important reasons are stable disks and mobility. The next sections will deal with these
notions.
\subsection{Stable disks}\index{stable disks}
Let's start with stable disks. Stable disks are disks that can not
be flipped by the opponent. Good examples of stable disks are
corner stones. There is no way that an opponent can flip one of
your cornerstones, so acquiring a corner stone in the beginning of
the game, or in the middle part even, usually means that you will
be able to use that corner to gain a lot of stable disks and thus
win the game. Examples of stable stones are given in figure \ref{stablestones}

\begin{figure}[h]
\stablestones
\end{figure}

Looking back at the example in figure \ref{maximumdisk}, we see that
white has 46 stones, but not one of these disks is stable. Thus black is able to turn a lot of
the disks into black disks.

To see an example of the opposite situation: suppose that white also was able to completely
capture one of the corners, leading to the situation \ref{stabledisks} .
\begin{figure}[ht]
\stabledisks
\end{figure}
Now the white stones in the north-west corner are completely safe,
white has 21 stable stones in this situation. Black can still gain
more stones then one would expect, but the dramatic change to 46
stones is gone. In fact, assuming perfect play, the end result
will be 35 stones for white, and 29 for black. Once more, it is a
nice puzzle to find the best way to play\footnote{Solution:
b7,a8,a7,-,b8,-,g7,h8,g8,-,h7,-,g2,h1,h2,-,g1}.

\section{Mobility}
Besides the lack of stable disks, white has another problem in our
maximum disk problem. Looking at the maximum disk figure once
more, we see that white can't make a move. It is this fact that
allows black to force white into making bad moves. This brings us
to the notion of \index{mobility}mobility. One could say that mobility is the
amount of moves that a player can make. Thus, someone with a low
mobility only has a few options to choose from, while someone with
high mobility can place a stone in a lot of different places.
Before refining the definition, let's look at another example of
why mobility is important.
\begin{floatingfigure}{4cm}
\mobilityexampletwo
\end{floatingfigure}
In figure \ref{mobilityexampletwo} we
see a position in which black has enclosed white. Again, this is
an extreme example of mobility difference, since black can make no
valid moves and white a lot. This situation can be used by white
to force black into making one or more bad moves. As an exercise,
loot at the position now and try to find sequence of moves that
will give white a corner. I hope you have done your ``homework''
by now, so lets look at the position together.

\noindent\textbf{Analysis:} White in total
has 13 squares to place his stones. We want to force black however
so we cant place our stone somewhere where it gives black multiple
options. Basically this only leaves us the north side, since all
the other options offer black at least two moves. So now we only
have to consider the five possible moves on the north side. We can
scratch one move immediately, namely g1.

Suppose we did play g1,
then black would play f1 and would thus have gained a corner in
the next round instead of white.

Another move that does not lead
to our objective is e1. Black then plays f1, and then white either
moves to b1, which opens the move c2 to black, or to one of the
squares towards the east, west or south side and we already know
that this gives black more options. Thus e1 does not lead to a
good way of forcing black.

Starting with either b1 or f1 basically leads to the same end
pattern, but a different corner. Lets look at b1 first. This leads
to the sequence b1,c1,e1,f1,g1,g2. The situation has changed to
figure \ref{mobiuitween}.
\begin{figure}
\mobiuitween
\end{figure}
Technically we do not have a corner yet,
but black was forced to place a stone at the X square g2, and this
enables white to take the h1 corner for sure in this case. g4 for
example provides a place on the diagonal. Of course, sometimes
white wants to wait before actually taking the corner, to make
things even more profitable, but this was a good example of using
black's limited capability to move to get a corner.

If we place f1 first, then the sequence becomes f1,e1,c1,b1,b2 and
again black is forced to take an X square.

The last possibility, and the most effective one, starts at c1.
Black has to take b1, and white can answer with f1. Black's only
option is to take e1, and white now directly can take a1.

We have seen how white can gain a corner from this situation. This
does not mean that c1 is the best move overall though. Try this
position out against a computer to try out several options and you
will find that f5 is the best choice. After all, we're out to have
the most stones in the end, and corners are a possible way to
obtain that goal, but capturing corners is not always the best
way. Playing f5 and subsequent moves will cause black to worsen
its position even further before white will actually grab a
corner.

\subsection{Frontiers}\index{frontiers}
Mobility is al about gaining many possible moves while reducing
the possible moves of the opponent. Our previous examples gave
extreme positions in which one color was completely surrounded by
corners of the opponent. In general, both colors will have stones
adjacent to one or more empty squares. These stones are called
frontier stones. If several frontier discs adjacent to each other
are of the same color, then we call it a frontier.
\begin{figure}
\frontierstones
\end{figure}
Figure \ref{frontierstones}
shows one large black frontier, and two smaller white frontiers.
A large frontier can be very limiting for a player, since he can
not make moves along his entire frontier. If white would play
h3,h4 or h5 in figure \ref{frontierstones}, then that move would
basically limit white to moves on the south and east side of the
board. Similarly, black can only make moves in the north and east
side of the board.

The basic idea of mobility was to gain as
many possible moves while limiting the opponent's moves. Since
frontiers block your moves, a general rule for placing stones is
that you try to place stones in such a way that you have as little
frontier disks as possible. We'll look at two examples to see how
the notion of frontiers works in regular games.

\begin{floatingfigure}{4cm}
\frontiermoveone
\end{floatingfigure}
\textbf{Example 1: }The first example is the situation given by
figure \ref{frontiermoveone}. In this case, white has quite a
large frontier (from c6 to h4) while black hardly has a frontier.
Black would like to make a move which keeps its own frontier small
and, if possible, white's frontier even larger. One move that
would \textbf{not} accomplish this is h6. In this case, black
almost completely takes over the frontier by flipping 5 of the white
frontier stones. In contrast, the move c7 also flips 5 white
stones, but only adds on frontier stone. Even more important,
this frontier stone c7 does not offer white any good
possibilities, since it only opens the option to either an X or a
C square.
\newpage
\begin{figure}[h]
\frontiermovetwo
\end{figure}
\textbf{Example 2: }
Figure \ref{frontiermovetwo} shows a game in an earlier stage of
the game. Lets assume that black is the one to move. He already
has a larger frontier then white has, so it is important that the
frontier does not grow any further. The move e6 is a disastrous as
h6 was in the previous example, so the possible moves are f2 until
f6. The moves f3,f4 and f5 also lead to a large black frontier, so the only real options are to
play either f2 or f6.

There are moves that do not add to the frontier. These moves are
quiet moves. They usually are the best options to play when
possible. One example is figure \ref{quietmove},\index{quiet moves} where black can
play d4. This move does not open any new options for white to
play, whereas white has no official quiet move after that. White
does have one unofficial quiet move, a3. It is unofficial since it
opens two new moves for black, but both options are so bad that
black has gained no real options.
\begin{figure}[h]
\quietmove
\end{figure}

\subsection{Tempo}\index{tempo}
Mobility is about reducing the opponents amount of possible
squares while not running out of your own options, and this leads
to the notion of tempo. Basically, if you can make a move which
does not create a new possibility for your opponent and your opponent can't, then you have
lowered his mobility. This is called: gaining a tempo. Quiet moves
are nice examples of this notion. Gaining tempo's is a good thing
since your opponent is forced to increase his frontier, thus
lowering mobility. A nice example is figure \ref{tempo}.
\begin{figure}[ht]
\tempo
\end{figure}

White can play b1,f1,g1 and black has to make 3 moves as well. The
only option created by whites moves is b2 or g2, both X squares. Black does
not want to play there, so he is forced to flip the three white
stones c6,d6 and e6 in the three turn. After these three moves,
white has no frontier except for c2 and f2, and black is in
serious trouble.

Moves that allow tempo gaining are very strong, but that does not
mean that you have to play them right away. Most of the time,
saving these moves until you really need them tends to make them
even stronger.

\section{Wedges}\index{wedges}
A wedge is a situation where a player places a stone between two
stones of his opponent, in such a way that he ``wedges'' those two
stones. For example, if white plays a4 in figure \ref{wedge}, then
he places a wedge between the black stones on the first column.
\begin{figure}[h]
\wedge
\end{figure}

Wedges are powerful stones when it comes to capturing corners. In
the example, white can take both corners. Another situation in
which wedging often occurs is the one in figure \ref{wedgetwo}.
White can play b2 in this case, thus sacrificing the a1 corner.
However, if black plays a1, then white can make a wedge by playing
a2, and after that a8. In this case, black has gained a corner,
and secured the first column, but white has gained a corner too
and also the first row. This particular type of wedging is often
used when sacrificing a corner. An edge that permits this kind of
wedging is also known as an unbalanced edge.\index{unbalanced edge} When playing Othello
in the begin stage of the game, it is usually wise not to try and
grab edges because they usually end becoming unbalanced edges and pose a
threat in the middle and end stage.
\begin{figure}
\wedgetwo
\end{figure}

\section{Unbalanced edges and stoner traps}

\section{Parity and passing}\index{parity}
In general, if a game is played without either of the players
passing, white places the last stone on the board. This offers a
slight advantage. If white has to pass once however, then parity
changes to black's advantage. In the end game, usually certain
holes on the board remain, consisting out of 1 to 4 empty squares.
For example, look at figure \ref{parity}.
\begin{figure}[ht]
\parity
\end{figure}

Black has to play and there is only one way to win, assuming
perfect play from white of course. Try to find the solution before
reading on. The solution is a1,g2,b1,g7,h8,b2,h1,h2. The finishing
blow of this solution lies in the fact that white has to pass
after the h1 move. Although white has 46 stones with only
two moves left, black still wins, mainly because of the parity
change.

In general, if there is an odd number of empty spaces, parity can
change, while even number of empty squares usually preserve
parity.

One more example: the end position of a game between J. Lysons and
E. Lazard at the Cambridge tournament in 1984 is given in figure
\ref{paritytwo}. Black has to play and draw. Once more, try to
find the solution before looking at the analysis.
\begin{figure}
\paritytwo
\end{figure}

\noindent\textbf{Analysis:} Black has to move first, 8 squares are
empty, so at this moment white gets to make the last move. There
is one empty place with an even amount of squares, and 2 odd. Of
course, we're not only focusing on parity alone, all the other
ideas we have seen so far are in place. White cant play inside the
north-west corner, so we'll stay away from that place. Why? There
are 5 other empty squares. Regardless of what player makes the 5th
move, black can always start in the north-west corner and also
gets the last move. This guarantees our parity advantage, which is
what we want. So lets look at the remaining 5 empty squares. G7 is
a very good winner, since black then immediately gains 8 stable
stones. White has to move, basically has to choose between b7 and
b8. White will play b8, otherwise black can win 33-31 by playing
b2,b8,a8,a1,a7,a2. Black can now take the corner a8. White has to
play a7, otherwise we can win 35-29. And we're done. B7 creates a
few more stables disks. White has to pass, so we can make the
final move. Counting stones shows that the next move is b2. White
has no choice but to play a1 (otherwise black wins with 34-30),
and a2 brings the score to 32-32.

Once more: this example, like the others, has been chosen to illustrate the
definitions that we have discussed, and not every end game works
according
to these rules. It is up to you to determine what plan works best.
There are games where you can gain parity, but it will cause you
to lose the game. In these cases, forget parity. You're in it for
the stones, not for your knowledge of funny concepts :)

\section{Beginning Othello games}
Keeping the idea of mobility as standard strategy in mind, we can
give the following rules of thumb for openings in Othello:

\begin{itemize}
\item Try to have fewer discs than your opponent.
\item Try to occupy the center of the position (the 4 center-squares in the first few moves).
\item Avoid flipping too many frontier discs (those located on the outside boundaries of the position, i.e. avoid
    building walls).
\item Try to group your discs into one connected cluster rather than having several scattered isolated discs.
\item Avoid taking edges too soon (before the mid-game).
\end{itemize}

Mind that these rules are not carved in stone, but they apply very
well for most of the games. In fact, they do not only apply to
openings, but also to the middle section of the game.

The first and third rule immediately follow from mobility
arguments. The fourth rule, combined with the second, is partially fixed on the
idea that if your stones are scattered, then keeping a small
frontier is hard. Why? If you place a stone, then you have other
stones all over the board, causing lots of stones to flip. The
fifth rule has already been mentioned. If you take edges too soon,
then you usually end up with an end game in which you have several
unbalanced edges, and you opponent is playing with a wide smile on
his or her face.

Using these general rules, one nice description of how to play is
``to try to get enclosed''. This also is known as: ``curling into
a ball''.

\subsection{Creeping along the edges}
As with every rule, how flexible they may be, there are always
exceptions. One strategy, that can be wonderful or disastrous
(really, really disastrous), is the ``creeping along edges''
strategy. What is this strategy about? Instead of trying to have
the opponent enclose you while you are a small cluster of stones
in the middle of the board, you try to gain control of one or two
edges. You try to gain a lot of tempi while doing so, to leave
the opponent in a position where he has no free moves when you
have the edges. After that you force the opponent to sacrifice a
good corner and you are set. The one problem with this strategy is
that if the opponent is able to have one move left after you have
the edges, then you basically can just give up. So either you win
by using a daring short term strategy, or you lose big time because
your position offers no long term prospects.


\section{Standard openings of Othello}
Once you have grasped all the ideas presented in this manual, and
are quite familiar with them, you will find that Othello games
generally move into certain stages. There is an opening stage of
about roughly 20-26 moves, a middle stage and and end stage
(starting somewhere between 16 to 10 moves before the end). The
end game is a matter of good calculating using the concepts given
and also practice (the freeware program Icare is \textbf{great}
for practicing Othello games). The middle game is also about
rules, directed more at how to attack edges and gaining tempi.

The opening of Othello roughly obeys the rules mentioned before,
but this may not be enough against really experienced players. For
example: there are positions in which all the rules can not help
you determine what move to make. Your opponent may know what move
is better simply because he has played both options several times
and found out that one move simply works out better in the long
run. Fortunately you can tap into the common Othello openings
source of the last decade. Almost any Othello game played on
tournaments is collected in the Thor database.

Furthermore, Robert Gatliff
has made a huge list containing standard openings and the
percentage in which each is played. This knowledge is collected in
a Java programme, so that you can easily learn standard openings.
The occurrence percentage allows you to learn the openings that
are played most first, and after that to continue into the more
obscure ones. A small tip: try to learn like 3 openings at first,
and learn them well (like 20 moves deep, with several variances),
before going on.

So, to sum it all up, there are three programs that can help you
advance in Othello (besides playing humans of course). Luckily
they are all free to download, so do not hesitate and get hooked.
\begin{itemize}
\item For openings, get the java applet based on Robert Gatliff's list.
\item For end games, use Icare. You can vary from 6 until 12 empty
squares, starting from positions that really have occurred.
\item For games in general, practicing mid game and analyzing
games you played vs someone else, use Wzebra. You can also use it
to import Thor databases and study other games.
\end{itemize}

Also, playing human adversaries is a world apart. There are
several places where you can meet other people. Common places are:
\begin{itemize}
\item
IOS (the Internet Othello Server) which has a very strong player
base
\item Vog (www.vog.ru) which has a very nice graphical layer for
$\$24$ a year.
\item The Zone (www.zone.com/reversi), microsoft's server, for
free. The Zone has several rooms, for beginners and more advanced players.
\item Yahoo
\end{itemize}


\chapter{\LaTeX\ and Othello}
The pictures in this manual have been created using the
\textbf{Othello} package for \LaTeX. I have created this package
by modifying the \textbf{go} package by Hanna Kolodziejska. Thus,
the commands to use are quite similar as for go.  A short list of
possible commands:
\begin{itemize}
\item {\verb|\gofontsize{size}|} This command determines the font for
the board. Standard values are 10,15,20.
\item
{\verb|\inifulldiagram|} This creates a board, you use this to
start. When including the Othello package, a first board is
initialized immediately, so this command is only needed when you
want to make more then one board.
\item
{\verb|\inidiagram{let1}{let2}{num1}{num2}|} This cleans the
current board specified by rows let1 to let2 and columns num1 to
num2. This can be used to make small modifications to a filled
board for example.

\item{\verb|\pos|}\{letter\}\{number\}={\verb|\color|}\{marking\} This command places a
stone onto the position denoted by (letter, number), where letter
denotes the row, and number denotes the column , color is either
black,white or neutral. If the color is either black or white,
then the argument is a dot or a number ranging between 1 and 64. A
dot represents a regular stone, while a number shows a stone that
is numbered. The color neutral is meant to insert a character onto
the board without drawing a stone. Besides the numbers 1 to 64,
neutral also recognizes the arguments 65 and 66 to draw an X or a
C respectively. These options have been added for the example of
special stones (chapter 2).

Example: {\verb|\pos|}\{b\}\{5\}=\hbox{\verb|\black|}\{.\}
gives a plain black stone on row b, column 5.

Example: {\verb|\pos|}\{b\}\{5\}=\hbox{\verb|\white|}\{21\} gives
a white stone on b5, with the number 21 inside.

\textbf{Warning:} Othello moves usually are described in
letter/number combinations where the letter denotes the column and
the number the row, i.e. reversed from Othello package. This is
because I modified an existing package, and do not have enough
\TeX\ knowledge yet to make a clean switch. Thus,
 if you want to make a picture of a written
transcription and it says : black puts a stone on e2, you'd have
to define the position as b5. If anyone with sufficient knowledge
and time wants to look at othello.sty, I'd be more then happy (if
you do not want to translate all positions, just follow the
original transcription since the position will only come out
reflected in the diagonal from the left upper corner to the right
bottom corner. The idea of the game is the same though.)

\item{\verb|\showfulldiagram|} This shows the full diagram made so
far. You can add more moves after this and reuse this command to
show the latest version of the board.
\end{itemize}

\noindent A short example to illustrate the commands, the commands to create
the opening position for Othello:
{\obeylines
{\verb|\inifulldiagram|}
{\verb|\gofontsize{10}|}
{\verb|\pos{d}{4}=\black{.}}|}
{\verb|\pos{e}{5}=\black{.}|}
{\verb|\pos{d}{5}=\white{.}|}
{\verb|\pos{e}{4}=\white{.}|}
{\verb|\showfulldiagram|}}

A last example will show all different kind of characters for an
Othello board.

\begin{figure}
\overviewboard
\end{figure}

In the go package, a feature has been added to show only parts of
the board, since go boards are rather large. Since Othello boards
are only 8x8, this is not necessary in general, but the option
still exists. To use this option, use the command
{\verb|\showdiagram|}row1-row2:column1-column2, instead of
{\verb|\showfulldiagram|}. For example: the northwest corner of
the overview board is given by:

\begin{figure}
\overviewboardcorner
\end{figure}

As mentioned: this package has been created using the go package
as a basis. There still are 2 things to improve, but unfortunately
I am not a \TeX\ or metafont guru (yet). If anyone knows how to
either:
\begin{itemize}
\item make prettier metafont definitions for the X and C
\item change the position definition in such a way that the position
command uses the same co\"ordinates as the standard Othello
notation
\end{itemize}
then I would be more then happy to hear from you.
\printindex
\end{document}
