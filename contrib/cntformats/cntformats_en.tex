% arara: pdflatex
% !arara: biber
% !arara: pdflatex
% arara: pdflatex
% --------------------------------------------------------------------------
% the CNTFORMATS package
% 
%   A different way to read counters
% 
% --------------------------------------------------------------------------
% Clemens Niederberger
% Web:    https://github.com/cgnieder/cntformats/
% E-Mail: contact@mychemistry.eu
% --------------------------------------------------------------------------
% Copyright 2012-2014 Clemens Niederberger
% 
% This work may be distributed and/or modified under the
% conditions of the LaTeX Project Public License, either version 1.3
% of this license or (at your option) any later version.
% The latest version of this license is in
%   http://www.latex-project.org/lppl.txt
% and version 1.3 or later is part of all distributions of LaTeX
% version 2005/12/01 or later.
% 
% This work has the LPPL maintenance status `maintained'.
% 
% The Current Maintainer of this work is Clemens Niederberger.
% --------------------------------------------------------------------------
% If you have any ideas, questions, suggestions or bugs to report, please
% feel free to contact me.
% --------------------------------------------------------------------------
\documentclass[load-preamble+]{cnltx-doc}
\usepackage{cntformats}

\setcnltx{
  package  = {cntformats} ,
  authors  = Clemens Niederberger ,
  email    = {contact@mychemistry.eu} ,
  url      = {https://github.com/cgnieder/cntformats/} ,
  info     = {A different way to read counters} ,
  add-cmds = {
    @cntfmts@parsed@pattern, AddCounterPattern, 
    eSaveCounterPattern,  
    NewCounterPattern, NewPatternFormat,
    ReadCounterFrom, ReadCounterPattern, ReadCounterPatternFrom,
    SaveCounterPattern
  } ,
  add-silent-cmds = {
    alphalph,
    cs,
    ExplSyntaxOff, ExplSyntaxOn,
    int,
    myoddnumber,
    numexpr, ordinalnum,
    relax,
    somesetupcommand,
    tmpa, tmpb, the
  }
}

\defbibheading{bibliography}[\bibname]{\addsec{#1}}

\newcommand*\at{\cnltxat}

\usepackage[biblatex]{embrac}[2012/06/29]
  \ChangeEmph{[}[,.02em]{]}[.055em,-.08em]
  \ChangeEmph{(}[-.01em,.04em]{)}[.04em,-.05em]

\usepackage{booktabs,array,ragged2e}
\usepackage{subcaption}

% ----------------------------------------------------------------------------
% example definitions that have to be done in the preamble:
\usepackage{alphalph}
\NewPatternFormat{aa}{\alphalph}
\usepackage{fmtcount}
\NewPatternFormat{o}{\ordinalnum}
\newcommand*\myoddnumber[1]{\the\numexpr2*(#1)-1\relax}
\NewPatternFormat{x}{\myoddnumber}
\NewCounterPattern{test}{t}

\begin{document}

\section{Motivation}
\cntformats\ provides a way to format counters with what I will call patterns.
This does not in any way effect the usual \LaTeXe\ way of treating counters
and does not use \cs*{the\meta{counter}} nor is it affected by the
redefinition of them.

This package is aimed at package or class authors and probably not very useful
for document authors.

When I first had the idea for this package the idea of what it does already
existed as part of the \pkg{exsheets}~\cite{bnd:exsheets} package.  I can't
recall why I came up with the idea in the first place or why I originally
wanted a new syntax for formatting the \code{question} counter.  I am also not
convinced any more that it is a good idea.  Anyway, here we are.

Until and including v0.6\changedversion{0.7} \cntformats{} used to be part of
the \pkg{exsheets} bundle.  Since version 0.7 it is distributed as a package
of its own.

\section{License and Requirements}\label{sec:license}
\license

\cntformats\ requires the \needpackage{etoolbox} package~\cite{pkg:etoolbox}
and the \pkg*{cnltx-base}\footnote{\CTANurl{cnltx}} package~\cite{bnd:cnltx}.

\section{Example}
A use case typically looks as follows:
\begin{example}[side-by-side]
  \ReadCounterPattern{se.sse}
\end{example}
where the key \code{se} stands for the current value of the \code{section}
counter and \code{sse} for \code{subsection}, respectively. \code{se.sse} is
an example for what will be called \emph{pattern}. The keys for the counters
can have optional arguments that specify the format:
\begin{example}[side-by-side]
  \stepcounter{subsection}
  \ReadCounterPattern{se[A](sse[r])}
\end{example}
\code{A} stands for \cs*{Alph} and \code{r} for \cs*{roman}.  A complete list
can be found in table~\ref{tab:predefined:formats} on
page~\pageref{tab:predefined:formats}.  As you can see you can insert
arbitrary other tokens in a pattern that won't be changed. It is important to
notice, though, that the patterns are only replaced if they're \emph{not}
placed in a braced group!

\begin{example}[side-by-side]
  \ReadCounterPattern{{se[A](sse[r])}}
\end{example}

I would imagine that the argument to \cs{ReadCounterPattern} is usually
supplied by a user setting an option \ldots
\begin{sourcecode}
  \somesetupcommand{
    counter-format = se[A](sse[r])
  }
\end{sourcecode}
\ldots{} and then internally used by the corresponding package or class.


\section{Usage}
In the following description of the available commands the symbol
\textcolor{expandable}{\expandablesign} means that the command is expandable.

In order to make counters known to \cntformats\ the following commands are used:
\begin{commands}
  \command{AddCounterPattern}[\sarg\oarg{module}\marg{counter}\marg{pattern}]
    This command will make the (existing) counter \meta{counter} known to
    \cntformats\ and assign the pattern \meta{pattern} to it.
  \command{NewCounterPattern}[\sarg\oarg{module}\marg{counter}\marg{pattern}]
    This command will create a new counter \meta{counter}, make it known to
    \cntformats\ and assign the pattern \meta{pattern} to it.
  \command{ReadCounterFrom}[\oarg{module}\marg{counter}\marg{internal cmd}]
    If you use one of the commands above with the starred version the number
    for the pattern is \emph{not automatically fetched} from the internal
    \cs*{c@\meta{counter}}.  This can (and must) now be assigned with
    \cs{ReadCounterFrom}  where \meta{internal cmd} is the macro that holds
    the number.
\end{commands}
The commands above can only be used in the document preamble.

After the creation of these pattern markers one wants to be able to use them.
There are a number of macros that allow different aspects of usage.
\begin{commands}
  \command{ReadCounterPattern}[\oarg{module}\marg{pattern}]
    Reads, interprets and prints a pattern.
  \expandable\command{\at cntfmts\at parsed\at pattern}
    After \cs{ReadCounterPattern} has been used the current pattern
    interpretation is stored in this macro.  The \emph{interpretation} is
    \emph{not} what is printed.  See the examples below for details.
  \command{ReadCounterPatternFrom}[\oarg{module}\marg{macro that holds
    pattern}]
    Reads, interprets and prints a pattern that's stored in a macro.\\
    Otherwise the same as \cs{ReadCounterPattern}.
  \command{SaveCounterPattern}[\marg{cmd a}\marg{cmd b}\marg{pattern}]
    Saves the \meta{pattern} in \meta{cmd a} and the interpreted pattern in
    \meta{cmd b}.
  \command{eSaveCounterPattern}[\oarg{module}\marg{cmd a}\marg{cmd
    b}\marg{pattern}]
    Saves the \meta{pattern} in \meta{cmd a} and the expanded pattern in
    \meta{cmd b}.
  \command{SaveCounterPatternFrom}[\oarg{module}\marg{cmd a}\marg{cmd
    b}\marg{macro that holds pattern}]
    Like \cs{SaveCounterPattern} but reads the pattern from a macro.
  \command{eSaveCounterPatternFrom}[\oarg{module}\marg{cmd a}\marg{cmd
    b}\marg{macro that holds pattern}]
    Like \cs{eSaveCounterPattern} but reads the pattern from a macro.
\end{commands}

The optional argument \meta{module} should be specific for a package, say, so
that different patterns for the \code{section} counter for example don't
interfer with each other.  This means: \cs{ReadCounterPattern} and friends
\emph{only} read the patterns known to the specified module!  If you leave the
argument the default module \code{cntfmts} is used.

The \pkg{exsheets} package uses the commands with the module \code{exsheets}.
You can find the following lines in \pkg{exsheets}' code:
\begin{sourcecode}
  \AddCounterPattern*[exsheets]{section}{se}
  \ReadCounterFrom[exsheets]{section} \l__exsheets_counter_sec_int
  \NewCounterPattern*[exsheets]{question}{qu}
  \ReadCounterFrom[exsheets]{question} \l__exsheets_counter_qu_int
\end{sourcecode}

Another example would be the \pkg{tasks} package~\cite{pkg:tasks}:
\begin{sourcecode}
  \NewCounterPattern*[tasks]{task}{tsk}
  \ReadCounterFrom[tasks]{task} \g__tasks_int
\end{sourcecode}

In fact \pkg{exsheets} loads the \pkg{tasks} package and also does the
following
\begin{sourcecode}
  \AddCounterPattern*[tasks]{question}{qu}
  \ReadCounterFrom[tasks]{question}\l__exsheets_counter_qu_int
\end{sourcecode}
so \pkg{tasks} knows about the question counter and uses the same pattern as
\pkg{exsheets} does.

Now let's see a short example that hopefully explains what the other macros
do:
\begin{example}
  % preamble
  % \NewCounterPattern{test}{t}
  \setcounter{test}{11}
  \ReadCounterPattern{t}
  \ReadCounterPattern{t[a]} \\
  \ttfamily\makeatletter
  \meaning\@cntfmts@parsed@pattern
  
  \bigskip
  \SaveCounterPattern\tmpa\tmpb{t[a]}
  \meaning\tmpa \\
  \meaning\tmpb
  
  \bigskip
  \eSaveCounterPattern\tmpa\tmpb{t[a]}
  \meaning\tmpa \\
  \meaning\tmpb
\end{example}
You can see that somehow an additional \cs*{@empty} found its way into the
interpreted pattern. This is due to the fact that reading optional arguments
expandably isn't easy and must have some safety net.  In this case looking for
an optional argument is done by reading the token following the command as
argument.  If there is no option then \cs*{@empty} is found so no other tokens
provided by users get read which might otherwise lead to strange
effects/outcome/errors\footnote{Imagine for example there would be \emph{no}
  token following the command -- nasty error messages could follow.}.  On the
other hand the grabbing of the next token can lead to spaces after the pattern
being ignored, see the next example: if there is no optional argument then the
next token is grabbed ignoring the spaces as is normal for a \TeX{} macro.
The next example demonstrates this.

A word of caution: although I used a one-letter pattern in the above example I
would not recommend this.  The pattern should consist of two or more
characters otherwise it might get inconvenient hiding the characters that you
\emph{don't} want replaced:
\begin{example}
  \setcounter{test}{3}
  Where is the first space: \ReadCounterPattern{t t[r] t}?\par
  \ReadCounterPattern{every instance of the letter t{} is replaced}
\end{example}

\section{Predefined and New Patterns and Format Keys}
\subsection{Predefined Patterns and Format Keys}
\cntformats\ predefines a number of pattern keys. These are listed in
table~\ref{tab:predefined:patterns}.

\subsection{New Patterns and Format Keys}
Table~\ref{tab:predefined:formats} lists the predefined formats.  If you want
you can add own formats.
\begin{commands}
  \command{NewPatternFormat}[\marg{pattern}\marg{format}]
    \meta{format} is a number presentation command like \cs*{@alph}, \ie,
    it needs a mandatory argument that takes a number.  It is used in
    \meta{format} \emph{without} its argument.  This command can only be used
    in the preamble.
\end{commands}

\begin{table}
  \centering
  \caption{Predefined Patterns and Format Keys.}
  \addtocounter{table}{-1}% necessary fix for subcaption and KOMA
  \begin{minipage}{.48\linewidth}
    \centering
    \subcaption{Predefined Patterns for the module \code{cntfmts}.}
    \label{tab:predefined:patterns}
    \begin{tabular}{>{\ttfamily}l>{\ttfamily}l}
      \toprule
        \normalfont\bfseries counter & \normalfont\bfseries pattern \\
      \midrule
        chapter       & ch \\
        section       & se \\
        subsection    & sse \\
        subsubsection & ssse \\
        paragraph     & pg \\
     \bottomrule
    \end{tabular}
  \end{minipage}
  \begin{minipage}{.48\linewidth}
    \centering
    \subcaption{Predefined Format Keys}
    \label{tab:predefined:formats}
    \begin{tabular}{>{\ttfamily}ll}
      \toprule
        \normalfont\bfseries key & \normalfont\bfseries format \\
      \midrule
        1 & \cs*{arabic} \\
        a & \cs*{alph} \\
        A & \cs*{Alph} \\
        r & \cs*{roman} \\
        R & \cs*{Roman} \\
     \bottomrule
    \end{tabular}
  \end{minipage}
\end{table}

Here are now a few examples of possible new patterns. Suppose the following code
in the preamble:
\begin{sourcecode}
  \usepackage{alphalph,fmtcount}
  \newcommand*\myoddnumber[1]{\the\numexpr2*(#1)-1\relax}

  \NewPatternFormat{aa}{\alphalph}
  \NewPatternFormat{o}{\ordinalnum}
  \NewPatternFormat{x}{\myoddnumber}

  \newcounter{test}
  \NewCounterPattern{test}{t}
  \setcounter{test}{4}
\end{sourcecode}

Then we can use the new pattern and the new formats as
follows:\setcounter{test}{4}
\begin{example}
  \ReadCounterPattern{t[aa]}
  \ReadCounterPattern{t[o]}
  \ReadCounterPattern{t[x]}
\end{example}

\end{document}
