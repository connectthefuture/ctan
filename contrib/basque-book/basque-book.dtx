% \iffalse meta-comment
%
% Copyright (C) 2012-2014 by Edorta Ibarra
% -----------------------------------
%
% This file may be distributed and/or modified under the
% conditions of the LaTeX Project Public License, either
% version 1.2 of this license or (at your option) any later
% version. The latest version of this license is in:
%
%     http://www.latex-project.org/lppl.txt
%
% and version 1.2 or later is part of all the distributions of
% LaTeX version 1999/12/01 or later.
%
% This work has the LPPL maintenance status `maintained'.
%
% The Current Maintainer of this work is Edorta Ibarra.
%
% This work consists of the files basque-book.dtx, basque-book.ins
% and the derived files basque-book.cls, basque-book.pdf and
% basque-book_[EUS].pdf.
%
% \fi
%
% \iffalse
%
%    \begin{macrocode}
%<package>\NeedsTeXFormat{LaTeX2e}[1999/12/01]
%<package>\ProvidesClass{basque-book}
%<package>   [2012/07/06 v1.20 Standard LaTeX document class adapted to Basque]
%<package>\RequirePackage{basque-date}
%    \end{macrocode}
%<*driver>
\documentclass{ltxdoc}
\usepackage{array}
\EnableCrossrefs
\CodelineIndex
\RecordChanges
\begin{document}
  \DocInput{basque-book.dtx}
\end{document}
%</driver>
% \fi
% \CheckSum{0}
%% \CharacterTable
%%  {Upper-case    \A\B\C\D\E\F\G\H\I\J\K\L\M\N\O\P\Q\R\S\T\U\V\W\X\Y\Z
%%   Lower-case    \a\b\c\d\e\f\g\h\i\j\k\l\m\n\o\p\q\r\s\t\u\v\w\x\y\z
%%   Digits        \0\1\2\3\4\5\6\7\8\9
%%   Exclamation   \!     Double quote  \"     Hash (number) \#
%%   Dollar        \$     Percent       \%     Ampersand     \&
%%   Acute accent  \'     Left paren    \(     Right paren   \)
%%   Asterisk      \*     Plus          \+     Comma         \,
%%   Minus         \-     Point         \.     Solidus       \/
%%   Colon         \:     Semicolon     \;     Less than     \<
%%   Equals        \=     Greater than  \>     Question mark \?
%%   Commercial at \@     Left bracket  \[     Backslash     \\
%%   Right bracket \]     Circumflex    \^     Underscore    \_
%%   Grave accent  \`     Left brace    \{     Vertical bar  \|
%%   Right brace   \}     Tilde         \~}
%%
% \changes{v1.00}{12/05/03}{First version}
% \changes{v1.05}{12/05/22}{First public version}
% \changes{v1.20}{12/07/05}{Captions of tables and figures improved}
% \changes{v1.22}{12/07/11}{Documentation in Basque included}
% \changes{v1.24}{14/01/25}{Bugs fixed}
%
% \GetFileInfo{basque-book.cls}
%
% \title{The \textsf{basque-book} document class\thanks{This file (\textsf{basque-book.dtx}) 
% has version v1.24 last revised 2014-01-25.}}
% \author{Edorta Ibarra\\\texttt{gautegiz@yahoo.es}}
% \date{2014-01-25}
% \maketitle

% \begin{abstract}
% \noindent The class is derived from the \LaTeX\ \verb|book| 
% document class. The extensions solve grammatical and numeration issues
% that occur when book-type documents are written in Basque.
% The class is useful for writing books, PhD Theses, etc. in Basque.
% \end{abstract}
%
% \tableofcontents
%
% \section{Basque Language and \LaTeX }
%
% \subsection{Introduction}
%
% \LaTeX\ is primarily intended for English and English-like languages.
% Basque does not have an English-like structure nor numeration rules and,
% for that reason, writing grammatically correct documents in Basque using
% \LaTeX\ is not allways straightforward. 
%
% This issues are well known for Basque \LaTeX\ users. Most non-advanced
% Basque \LaTeX\ users do not have the ability to modify the document 
% classes themselves. As stated in \cite{bib1}, one of the possible 
% solutions to address this problems is to redefine \LaTeX\ document 
% classes for Basque. 
%
% \subsection{The basque-book document class}
%
% One of the most common class used in \LaTeX\ is the \verb|book|
% document class. In this sense, a derived class called \verb|basque-book| 
% is provided. This class allows to write grammatically correct books in
% Basque without the need of performing manual corrections or redefinitions 
% on the document class code. The class is useful for writing books, PhD 
% Theses, etc. in Basque.
%
%
% \section{Calling the document class}
%
% The document class \verb|basque-book| is called using the 
% \verb|\documentclass|
% command:\\ \verb|\documentclass[<options>]{basque-book}|. 
%
% The class \verb|basque-book| provides the same options provided by the
% standard \verb|book| document class.
%
% The document class \verb|basque-book| requires the package 
% \verb|basque-date|. This package prints the current date 
% in Basque following the recommendations of the Basque Language Academy. 
% This package and its documentation are available online from 
% CTAN\footnote{http://ctan.org/tex-archive/macros/latex/contrib/basque-date}.
%
% Due to incompatibility reasons, the package \verb|babel| should not be
% used when the document class \verb|basque-book| is called (refer to
% appendix \ref{imple} for more details).
%
%
% \section{Acknowledgements}
%
% I would like to thank Jose Ramon Etxebarria, who tested the package
% \verb|basque-book| and made suggestions that helped to improve it.
%
% \addcontentsline{toc}{section}{References}
%
% \begin{thebibliography}{99}
% \bibitem{bib1} J. Arias, J. Lazaro, J. M. Aguirregabiria, ``Basque: A Case Study in Generalizing LaTeX Language Support," {\it International Conference on TeX, XML and Digital Typography}. pp. 27-33, 2004.
% \end{thebibliography}
% 
% \appendix
%
% \section{Appendices}
%
% \subsection{License}
%
% Copyright 2012-2014 Edorta Ibarra. 
%
% This program can be redistributed and/or modified under the terms of the 
% \LaTeX\ Project Public License Distributed from CTAN archives in directory
% macros/latex/basee/lppl.txe; either version 1.2 of the License, or any later 
% version.
%
%
% \subsection{Version history}
%
% \begin{itemize}
% \item \textbf{Version v1.00 (2012/05/03).} Initial non-public 
% version for development.
% \item \textbf{Version v1.05 (2012/05/22).} First public version. 
% This version included most of the current features.
% \item \textbf{Version v1.20 (2012/07/06).} Second public version. 
% \begin{itemize}
% \item[(a)] Captions for tables and figures 
% were redefined in order to improve the readability of the document. 
% \item[(b)] A minor bug was corrected (a missing \verb|\RequirePackage|).
% \item[(c)] Some minor changes were performed in the documentation.
% \end{itemize}
% \item \textbf{Version v1.24 (2014/01/25).} Third public version.
% \begin{itemize}
% \item[(a)] Documentation in Basque was provided in the \verb|.dtx| file.
% \item[(b)] A bug related with equation numbering was fixed.
% \end{itemize}
% \end{itemize}
%
%
%
% \iffalse
%<basque>
%<basque> %%%Documentation of basque-book document class in Basque
%<basque> \documentclass{ltxdoc}
%<basque> \begin{document}
%<basque> 
%<basque> \title{\textsf{basque-book} dokumentu-klasea\thanks{Fitxategi honek 
%<basque> (\textsf{basque-book.dtx})  1.24. bertsioa du. Azken aldiz errebisatua: 2014-01-25.}}
%<basque> \author{Edorta Ibarra\\\texttt{gautegiz@yahoo.es}}
%<basque> \date{2014ko urtarrilaren 25a}
%<basque> \renewcommand{\contentsname}{Aurkibidea}
%<basque> \renewcommand{\refname}{Bibliografia}
%<basque> \renewcommand\thesection{\arabic{section}.}
%<basque> \renewcommand\thesubsection{\thesection \arabic{subsection}.}
%<basque> \renewcommand{\abstractname}{Laburpena}
%<basque> \maketitle
%<basque>
%<basque> \begin{abstract}
%<basque> \LaTeX -eko \verb|book| dokumentu-klasetik eratorritako klasea da hau. 
%<basque> Liburu-motako dokumentuak euskaraz idaztean gertatzen diren arazo gramatikalak 
%<basque> eta numerazio-arazoak konpontzen dituzte hedapenek. 
%<basque> Klase hau oso baliagarria da liburuak, doktorego-tesiak, etab. euskaraz idazteko.
%<basque> \end{abstract}
%<basque>
%<basque> \tableofcontents
%<basque> 
%<basque> \section{Euskara eta \LaTeX }
%<basque> 
%<basque> \subsection{Sarrera}
%<basque> Oro har, ingelesarekin eta inglesaren antzeko hizkuntzekin lan egiteko dago \LaTeX\
%<basque> diseinatuta. Euskarak ez du ingelesaren antzeko hizkuntzen egiturarik, eta ezta 
%<basque> numerazio-araurik ere. Arrazoi horiek direla-bide, gramatikalki 
%<basque> zuzenak diren eus\-karazko dokumentuak \LaTeX -ekin idaztea ez da beti erraza izaten.
%<basque> 
%<basque> Aipatutako arazo horiek aski ezagunak dira \LaTeX -erabiltzaile euskaldunen artean.
%<basque> Adituak ez diren \LaTeX -erabiltzaile euskaldun gehientsuenek ez dute beraien kabuz
%<basque> dokumentu-klaseak berdefinitzeko gaitasunik; \cite{bib1} erreferentzian aipatzen den
%<basque> bezala, \LaTeX -eko dokumentu klaseak euskararako 
%<basque> berdefinitzea da arazo horri soluzioa emateko aukeretako bat.
%<basque>
%<basque> \subsection{basque-book dokomentu-klasea}
%<basque>
%<basque> \LaTeX -eko klase arruntenetakoa eta erabilienetakoa da \verb|book| dokumentu-klasea.
%<basque> Alde horretatik, \verb|basque-book| izeneko klase eratorria eskeintzen da. 
%<basque> Klase horri esker, gramatika aldetik zuzenak diren euskarazko liburuak idatz daitezke,
%<basque> eta dokumentu-klasearen kodean zuzenketak edota aldaketak egitea ez da beha\-rrezkoa.
%<basque> Klase hori oso baliagarria da liburuak, doktorego-tesiak, etab. euskaraz idazteko.
%<basque> 
%<basque> \section{Dokumentu-klasea nola deitu}
%<basque> 
%<basque> \verb|\usepackage| komandoa erabiliz deitzen da  
%<basque> \verb|basque-book| dokumentu-klasea:\\ \verb|\documentclass[<options>]{basque-book}|. 
%<basque>
%<basque> \verb|book| dokumentu-klasearen aukera berdinak eskaintzen ditu \verb|basque-book|
%<basque> klaseak.
%<basque> 
%<basque> \verb|basque-book| klaseak funtzionatu ahal izateko, beharrezkoa da \verb|basque-date| 
%<basque> paketea erabiltzea. Euskaltzaindiaren gomendioak jarraituz uneko data inprimatzen du
%<basque> \verb|basque-date| paketeak. CTANen dago eskuragai aipatutako 
%<basque> paketea\footnote{http://ctan.org/tex-archive/macros/latex/contrib/basque-date}.
%<basque> 
%<basque> Bateraezintasun-arazoak direla-eta, ez da gomendagarria \verb|babel| paketea eta
%<basque> \verb|basque-book| dokumentu-klasea batera erabiltzea (dokumentazioaren
%<basque> ingelesezko bertsioan daude xehetasun teknikoak).
%<basque> 
%<basque> \section{Eskerrak}
%<basque>
%<basque> Jose Ramon Etxebarriari eskerrak eman nahi dizkiot, \verb|basque-book| dokumentu-klasea
%<basque> probatzeagatik eta hura hobetzeko gomendioak emateagatik.
%<basque>
%<basque> \addcontentsline{toc}{section}{Bibliografia}
%<basque> 
%<basque>  \begin{thebibliography}{99}
%<basque>  \bibitem{bib1} J. Arias, J. Lazaro, J. M. Aguirregabiria, ``Basque: A Case Study in Generalizing LaTeX Language Support," {\it International Conference on TeX, XML and Digital Typography}. pp. 27-33, 2004.
%<basque> \end{thebibliography}
%<basque> 
%<basque> \appendix
%<basque>
%<basque> \renewcommand\thesection{A.}
%<basque> \renewcommand\thesubsection{\thesection \arabic{subsection}.} 
%<basque>
%<basque> \section{Eranskinak}
%<basque> 
%<basque> \subsection{Lizentzia}
%<basque> 
%<basque> Copyright 2012-2014 Edorta Ibarra. 
%<basque>
%<basque> CTAN fitxategietan banatutako \LaTeX\ proiektuko lizentzia
%<basque> publikoaren terminoetan birbanatu edota alda daiteke
%<basque> programa hau:
%<basque>
%<basque> macros/latex/basee/lppl.txe; bai lizentziaren 1.2. bertsioaren
%<basque> terminoetan, edota ondorengo edozein bertsioren terminoetan. 
%<basque> 
%<basque> \subsection{Bertsioen historia}
%<basque>  
%<basque> \begin{itemize}
%<basque> \item \textbf{v1.00. bertsioa (2012/05/03).} Garapenerako bertsio 
%<basque> ez-publikoa.
%<basque> \item \textbf{v1.05. bertsioa (2012/05/22).} Lehenengo bertsio publikoa. 
%<basque> Azkenengo bertsioaren ezaugarri gehientsuenak ditu bertsio horrek.
%<basque> \item \textbf{v1.20. bertsioa (2012/07/06).} Bigarren bertsio publikoa. 
%<basque> \begin{itemize}
%<basque> \item[(a)] Irudien eta taulen oinak berdefinitu dira dokumentua irakurtzeko e\-rra\-zagoa
%<basque> izan dadin.
%<basque> \item[(b)] Programazio-errore txiki bat konpondu da
%<basque> (\verb|\RequirePackage| komando bat faltan zegoen).
%<basque> \item[(c)] Zenbait aldaketa txiki egin dira dokumentazioan.
%<basque> \end{itemize}
%<basque> \item \textbf{v1.24. bertsioa (2014/01/25).} Hirugarren bertsio publikoa.
%<basque> \begin{itemize}
%<basque> \item[(a)] Euskarazko dokumentazioa gehitu da \verb|.dtx| fitxategian.
%<basque> \item[(b)] Ekuazioen numerazioarekin erlazionatutako errore bat arazi da doku\-mentu-klasean.
%<basque> \end{itemize}
%<basque> \end{itemize}
%<basque> \subsection{Inplementazioa}
%<basque> Ingelesezko dokumentazioan daude irakurgai dokumentu-klasearen
%<basque> inplementazioari buruzko xehetasun teknikoak.
%<basque> \end{document}
% \fi
%
% \subsection{Implementation}\label{imple}
%
% No changes from the standard \verb|book| document class are implemented
% at the beginning of the code.
%    \begin{macrocode}
\newcommand\@ptsize{}
\newif\if@restonecol
\newif\if@titlepage
\@titlepagetrue
\newif\if@openright
\newif\if@mainmatter \@mainmattertrue
\if@compatibility\else
\DeclareOption{a4paper}
   {\setlength\paperheight {297mm}%
    \setlength\paperwidth  {210mm}}
\DeclareOption{a5paper}
   {\setlength\paperheight {210mm}%
    \setlength\paperwidth  {148mm}}
\DeclareOption{b5paper}
   {\setlength\paperheight {250mm}%
    \setlength\paperwidth  {176mm}}
\DeclareOption{letterpaper}
   {\setlength\paperheight {11in}%
    \setlength\paperwidth  {8.5in}}
\DeclareOption{legalpaper}
   {\setlength\paperheight {14in}%
    \setlength\paperwidth  {8.5in}}
\DeclareOption{executivepaper}
   {\setlength\paperheight {10.5in}%
    \setlength\paperwidth  {7.25in}}
\DeclareOption{landscape}
   {\setlength\@tempdima   {\paperheight}%
    \setlength\paperheight {\paperwidth}%
    \setlength\paperwidth  {\@tempdima}}
\fi
\if@compatibility
  \renewcommand\@ptsize{0}
\else
\DeclareOption{10pt}{\renewcommand\@ptsize{0}}
\fi
\DeclareOption{11pt}{\renewcommand\@ptsize{1}}
\DeclareOption{12pt}{\renewcommand\@ptsize{2}}
\if@compatibility\else
\DeclareOption{oneside}{\@twosidefalse \@mparswitchfalse}
\fi
\DeclareOption{twoside}{\@twosidetrue  \@mparswitchtrue}
\DeclareOption{draft}{\setlength\overfullrule{5pt}}
\if@compatibility\else
\DeclareOption{final}{\setlength\overfullrule{0pt}}
\fi
\DeclareOption{titlepage}{\@titlepagetrue}
\if@compatibility\else
\DeclareOption{notitlepage}{\@titlepagefalse}
\fi
\if@compatibility
\@openrighttrue
\else
\DeclareOption{openright}{\@openrighttrue}
\DeclareOption{openany}{\@openrightfalse}
\fi
\if@compatibility\else
\DeclareOption{onecolumn}{\@twocolumnfalse}
\fi
\DeclareOption{twocolumn}{\@twocolumntrue}
\DeclareOption{leqno}{\input{leqno.clo}}
\DeclareOption{fleqn}{\input{fleqn.clo}}
\DeclareOption{openbib}{%
  \AtEndOfPackage{%
   \renewcommand\@openbib@code{%
      \advance\leftmargin\bibindent
      \itemindent -\bibindent
      \listparindent \itemindent
      \parsep \z@
      }%
   \renewcommand\newblock{\par}}%
}
\ExecuteOptions{letterpaper,10pt,twoside,onecolumn,final,openright}
\ProcessOptions
\input{bk1\@ptsize.clo}
\setlength\lineskip{1\p@}
\setlength\normallineskip{1\p@}
\renewcommand\baselinestretch{}
\setlength\parskip{0\p@ \@plus \p@}
\@lowpenalty   51
\@medpenalty  151
\@highpenalty 301
\setcounter{topnumber}{2}
\renewcommand\topfraction{.7}
\setcounter{bottomnumber}{1}
\renewcommand\bottomfraction{.3}
\setcounter{totalnumber}{3}
\renewcommand\textfraction{.2}
\renewcommand\floatpagefraction{.5}
\setcounter{dbltopnumber}{2}
\renewcommand\dbltopfraction{.7}
\renewcommand\dblfloatpagefraction{.5}
\if@twoside
%    \end{macrocode}
% \DescribeMacro{\ps@headings} The order of the elements of the document headings are
% redefined in order to comply with the Basque grammatical rules:
%    \begin{macrocode}
  \def\ps@headings{%
      \let\@oddfoot\@empty\let\@evenfoot\@empty
      \def\@evenhead{\thepage\hfil\slshape\leftmark}%
      \def\@oddhead{{\slshape\rightmark}\hfil\thepage}%
      \let\@mkboth\markboth
    \def\chaptermark##1{%
      \markboth {{%
        \ifnum \c@secnumdepth >\m@ne
          \if@mainmatter
            \thechapter\ \@chapapp .\ %
          \fi
        \fi
        ##1}}{}}%
    \def\sectionmark##1{%
      \markright {{%
        \ifnum \c@secnumdepth >\z@
          \thesection \ %
        \fi
        ##1}}}}
\else
  \def\ps@headings{%
    \let\@oddfoot\@empty
    \def\@oddhead{{\slshape\rightmark}\hfil\thepage}%
    \let\@mkboth\markboth
    \def\chaptermark##1{%
      \markright {{%
        \ifnum \c@secnumdepth >\m@ne
          \if@mainmatter
            \thechapter\ \@chapapp .\ %
          \fi
        \fi
        ##1}}}}
\fi
\def\ps@myheadings{%
    \let\@oddfoot\@empty\let\@evenfoot\@empty
    \def\@evenhead{\thepage\hfil\slshape\leftmark}%
    \def\@oddhead{{\slshape\rightmark}\hfil\thepage}%
    \let\@mkboth\@gobbletwo
    \let\chaptermark\@gobble
    \let\sectionmark\@gobble
    }
  \if@titlepage
  \newcommand\maketitle{\begin{titlepage}%
  \let\footnotesize\small
  \let\footnoterule\relax
  \let \footnote \thanks
  \null\vfil
  \vskip 60\p@
  \begin{center}%
    {\LARGE \@title \par}%
    \vskip 3em%
    {\large
     \lineskip .75em%
      \begin{tabular}[t]{c}%
        \@author
      \end{tabular}\par}%
      \vskip 1.5em%
    {\large \@date \par}%       % Set date in \large size.
  \end{center}\par
  \@thanks
  \vfil\null
  \end{titlepage}%
  \setcounter{footnote}{0}%
  \global\let\thanks\relax
  \global\let\maketitle\relax
  \global\let\@thanks\@empty
  \global\let\@author\@empty
  \global\let\@date\@empty
  \global\let\@title\@empty
  \global\let\title\relax
  \global\let\author\relax
  \global\let\date\relax
  \global\let\and\relax
}
\else
\newcommand\maketitle{\par
  \begingroup
    \renewcommand\thefootnote{\@fnsymbol\c@footnote}%
    \def\@makefnmark{\rlap{\@textsuperscript{\normalfont\@thefnmark}}}%
    \long\def\@makefntext##1{\parindent 1em\noindent
            \hb@xt@1.8em{%
                \hss\@textsuperscript{\normalfont\@thefnmark}}##1}%
    \if@twocolumn
      \ifnum \col@number=\@ne
        \@maketitle
      \else
        \twocolumn[\@maketitle]%
      \fi
    \else
      \newpage
      \global\@topnum\z@   % Prevents figures from going at top of page.
      \@maketitle
    \fi
    \thispagestyle{plain}\@thanks
  \endgroup
  \setcounter{footnote}{0}%
  \global\let\thanks\relax
  \global\let\maketitle\relax
  \global\let\@maketitle\relax
  \global\let\@thanks\@empty
  \global\let\@author\@empty
  \global\let\@date\@empty
  \global\let\@title\@empty
  \global\let\title\relax
  \global\let\author\relax
  \global\let\date\relax
  \global\let\and\relax
}
\def\@maketitle{%
  \newpage
  \null
  \vskip 2em%
  \begin{center}%
  \let \footnote \thanks
    {\LARGE \@title \par}%
    \vskip 1.5em%
    {\large
      \lineskip .5em%
      \begin{tabular}[t]{c}%
        \@author
      \end{tabular}\par}%
    \vskip 1em%
    {\large \@date}%
  \end{center}%
  \par
  \vskip 1.5em}
\fi
\newcommand*\chaptermark[1]{}
\setcounter{secnumdepth}{2}
\newcounter {part}
\newcounter {chapter}
\newcounter {section}[chapter]
\newcounter {subsection}[section]
\newcounter {subsubsection}[subsection]
\newcounter {paragraph}[subsubsection]
\newcounter {subparagraph}[paragraph]
%    \end{macrocode}
% \DescribeMacro{\thepart}\DescribeMacro{\thechapter}\DescribeMacro{\thesection}
% \DescribeMacro{\thesubsection}\DescribeMacro{\thesubsubsection}
% \DescribeMacro{\theparagraph}\DescribeMacro{\thesubparagraph}
% The commands that are responsible for numbering the different parts, chapters, sections, etc. of the
% document are redefined in order to comply with the Basque numeration rules:
%    \begin{macrocode}
\renewcommand\thepart          {\Roman{part}.}
\renewcommand\thechapter       {\arabic{chapter}.}
\renewcommand\thesection       {\thechapter \arabic{section}.}
\renewcommand\thesubsection    {\thesection \arabic{subsection}.}
\renewcommand\thesubsubsection {\thesubsection \arabic{subsubsection}.}
\renewcommand\theparagraph     {\thesubsubsection \arabic{paragraph}.}
\renewcommand\thesubparagraph  {\theparagraph \arabic{subparagraph}.}
\newcommand\@chapapp{\chaptername}
\newcommand\frontmatter{%
    \cleardoublepage
  \@mainmatterfalse
  \pagenumbering{roman}}
\newcommand\mainmatter{%
    \cleardoublepage
  \@mainmattertrue
  \pagenumbering{arabic}}
\newcommand\backmatter{%
  \if@openright
    \cleardoublepage
  \else
    \clearpage
  \fi
  \@mainmatterfalse}
\newcommand\part{%
  \if@openright
    \cleardoublepage
  \else
    \clearpage
  \fi
  \thispagestyle{plain}%
  \if@twocolumn
    \onecolumn
    \@tempswatrue
  \else
    \@tempswafalse
  \fi
  \null\vfil
  \secdef\@part\@spart}
%    \end{macrocode}
% \DescribeMacro{\@part} The \verb|part| environment is modified changing the order
% of \verb|\partname| and \verb|\thepart|. Moreover, the extra dot is eliminated:
%    \begin{macrocode}
\def\@part[#1]#2{%
    \ifnum \c@secnumdepth >-2\relax
      \refstepcounter{part}%
      \addcontentsline{toc}{part}{\thepart\hspace{1em}#1}%
    \else
      \addcontentsline{toc}{part}{#1}%
    \fi
    \markboth{}{}%
    {\centering
     \interlinepenalty \@M
     \normalfont
     \ifnum \c@secnumdepth >-2\relax
       \huge\bfseries \thepart\nobreakspace\partname
       \par
       \vskip 20\p@
     \fi
     \Huge \bfseries #2\par}%
    \@endpart}
\def\@spart#1{%
    {\centering
     \interlinepenalty \@M
     \normalfont
     \Huge \bfseries #1\par}%
    \@endpart}
\def\@endpart{\vfil\newpage
              \if@twoside
               \if@openright
                \null
                \thispagestyle{empty}%
                \newpage
               \fi
              \fi
              \if@tempswa
                \twocolumn
              \fi}
\newcommand\chapter{\if@openright\cleardoublepage\else\clearpage\fi
                    \thispagestyle{plain}%
                    \global\@topnum\z@
                    \@afterindentfalse
                    \secdef\@chapter\@schapter}
%    \end{macrocode}
% \DescribeMacro{\@chapter} Similarly, the \verb|chapter| environment is modified
% changing the order of \verb|\@chapapp| and \verb|\thechapter|. 
% As done before, the extra dot is eliminated:
%    \begin{macrocode}
\def\@chapter[#1]#2{\ifnum \c@secnumdepth >\m@ne
                       \if@mainmatter
                         \refstepcounter{chapter}%
                         \typeout{\thechapter\space\@chapapp}%
                         \addcontentsline{toc}{chapter}%
                                   {\protect\numberline{\thechapter}#1}%
                       \else
                         \addcontentsline{toc}{chapter}{#1}%
                       \fi
                    \else
                      \addcontentsline{toc}{chapter}{#1}%
                    \fi
                    \chaptermark{#1}%
                    \addtocontents{lof}{\protect\addvspace{10\p@}}%
                    \addtocontents{lot}{\protect\addvspace{10\p@}}%
                    \if@twocolumn
                      \@topnewpage[\@makechapterhead{#2}]%
                    \else
                      \@makechapterhead{#2}%
                      \@afterheading
                    \fi}
\def\@makechapterhead#1{%
  \vspace*{50\p@}%
  {\parindent \z@ \raggedright \normalfont
    \ifnum \c@secnumdepth >\m@ne
      \if@mainmatter
        \huge\bfseries \thechapter\space\@chapapp 
        \par\nobreak
        \vskip 20\p@
      \fi
    \fi
    \interlinepenalty\@M
    \Huge \bfseries #1\par\nobreak
    \vskip 40\p@
  }}
\def\@schapter#1{\if@twocolumn
                   \@topnewpage[\@makeschapterhead{#1}]%
                 \else
                   \@makeschapterhead{#1}%
                   \@afterheading
                 \fi}
\def\@makeschapterhead#1{%
  \vspace*{50\p@}%
  {\parindent \z@ \raggedright
    \normalfont
    \interlinepenalty\@M
    \Huge \bfseries  #1\par\nobreak
    \vskip 40\p@
  }}
\newcommand\section{\@startsection {section}{1}{\z@}%
                                   {-3.5ex \@plus -1ex \@minus -.2ex}%
                                   {2.3ex \@plus.2ex}%
                                   {\normalfont\Large\bfseries}}
\newcommand\subsection{\@startsection{subsection}{2}{\z@}%
                                     {-3.25ex\@plus -1ex \@minus -.2ex}%
                                     {1.5ex \@plus .2ex}%
                                     {\normalfont\large\bfseries}}
\newcommand\subsubsection{\@startsection{subsubsection}{3}{\z@}%
                                     {-3.25ex\@plus -1ex \@minus -.2ex}%
                                     {1.5ex \@plus .2ex}%
                                     {\normalfont\normalsize\bfseries}}
\newcommand\paragraph{\@startsection{paragraph}{4}{\z@}%
                                    {3.25ex \@plus1ex \@minus.2ex}%
                                    {-1em}%
                                    {\normalfont\normalsize\bfseries}}
\newcommand\subparagraph{\@startsection{subparagraph}{5}{\parindent}%
                                       {3.25ex \@plus1ex \@minus .2ex}%
                                       {-1em}%
                                      {\normalfont\normalsize\bfseries}}
\if@twocolumn
  \setlength\leftmargini  {2em}
\else
  \setlength\leftmargini  {2.5em}
\fi
\leftmargin  \leftmargini
\setlength\leftmarginii  {2.2em}
\setlength\leftmarginiii {1.87em}
\setlength\leftmarginiv  {1.7em}
\if@twocolumn
  \setlength\leftmarginv  {.5em}
  \setlength\leftmarginvi {.5em}
\else
  \setlength\leftmarginv  {1em}
  \setlength\leftmarginvi {1em}
\fi
\setlength  \labelsep  {.5em}
\setlength  \labelwidth{\leftmargini}
\addtolength\labelwidth{-\labelsep}
\@beginparpenalty -\@lowpenalty
\@endparpenalty   -\@lowpenalty
\@itempenalty     -\@lowpenalty
\renewcommand\theenumi{\@arabic\c@enumi}
\renewcommand\theenumii{\@alph\c@enumii}
\renewcommand\theenumiii{\@roman\c@enumiii}
\renewcommand\theenumiv{\@Alph\c@enumiv}
\newcommand\labelenumi{\theenumi.}
\newcommand\labelenumii{(\theenumii)}
\newcommand\labelenumiii{\theenumiii.}
\newcommand\labelenumiv{\theenumiv.}
\renewcommand\p@enumii{\theenumi}
\renewcommand\p@enumiii{\theenumi(\theenumii)}
\renewcommand\p@enumiv{\p@enumiii\theenumiii}
\newcommand\labelitemi{\textbullet}
\newcommand\labelitemii{\normalfont\bfseries \textendash}
\newcommand\labelitemiii{\textasteriskcentered}
\newcommand\labelitemiv{\textperiodcentered}
\newenvironment{description}
               {\list{}{\labelwidth\z@ \itemindent-\leftmargin
                        \let\makelabel\descriptionlabel}}
               {\endlist}
\newcommand*\descriptionlabel[1]{\hspace\labelsep
                                \normalfont\bfseries #1}
\newenvironment{verse}
               {\let\\\@centercr
                \list{}{\itemsep      \z@
                        \itemindent   -1.5em%
                        \listparindent\itemindent
                        \rightmargin  \leftmargin
                        \advance\leftmargin 1.5em}%
                \item\relax}
               {\endlist}
\newenvironment{quotation}
               {\list{}{\listparindent 1.5em%
                        \itemindent    \listparindent
                        \rightmargin   \leftmargin
                        \parsep        \z@ \@plus\p@}%
                \item\relax}
               {\endlist}
\newenvironment{quote}
               {\list{}{\rightmargin\leftmargin}%
                \item\relax}
               {\endlist}
\if@compatibility
\newenvironment{titlepage}
    {%
      \cleardoublepage
      \if@twocolumn
        \@restonecoltrue\onecolumn
      \else
        \@restonecolfalse\newpage
      \fi
      \thispagestyle{empty}%
      \setcounter{page}\z@
    }%
    {\if@restonecol\twocolumn \else \newpage \fi
    }
\else
\newenvironment{titlepage}
    {%
      \cleardoublepage
      \if@twocolumn
        \@restonecoltrue\onecolumn
      \else
        \@restonecolfalse\newpage
      \fi
      \thispagestyle{empty}%
      \setcounter{page}\@ne
    }%
    {\if@restonecol\twocolumn \else \newpage \fi
     \if@twoside\else
        \setcounter{page}\@ne
     \fi
    }
\fi
%    \end{macrocode}
% \DescribeMacro{\appendix} In order to correctly modify the appendix environment, an
% extra dot is added to the \verb|\gdef\thechapter{\@Alph\c@chapter}| code line:
%    \begin{macrocode}
\newcommand\appendix{\par
  \setcounter{chapter}{0}%
  \setcounter{section}{0}%
  \gdef\@chapapp{\appendixname}%
  \gdef\thechapter{\@Alph\c@chapter.}}
\setlength\arraycolsep{5\p@}
\setlength\tabcolsep{6\p@}
\setlength\arrayrulewidth{.4\p@}
\setlength\doublerulesep{2\p@}
\setlength\tabbingsep{\labelsep}
\skip\@mpfootins = \skip\footins
\setlength\fboxsep{3\p@}
\setlength\fboxrule{.4\p@}
\@addtoreset {equation}{chapter}
%    \end{macrocode}
% \DescribeMacro{\theequation} The dot after \verb|\thechapter| is eliminated in order to print the equation numbering correctly (without an extra dot):
%    \begin{macrocode}
\renewcommand\theequation
  {\ifnum \c@chapter>\z@ \thechapter\fi \@arabic\c@equation}
\newcounter{figure}[chapter]
%    \end{macrocode}
% \DescribeMacro{\thefigure} Similarly, an extra dot is added in \verb|\thefigure|:
%    \begin{macrocode}
\renewcommand \thefigure
   {\ifnum \c@chapter>\z@ \thechapter\fi \@arabic\c@figure.}
\def\fps@figure{tbp}
\def\ftype@figure{1}
\def\ext@figure{lof}
%    \end{macrocode}
% \DescribeMacro{\fnum@figure} The order between \verb|\figurename|
% and \verb|\thefigure| is changed. On the other hand, \verb|\textbf{}|
% command is added:
%    \begin{macrocode}
\def\fnum@figure{\textbf{\thefigure\nobreakspace\figurename}}
\newenvironment{figure}
               {\@float{figure}}
               {\end@float}
\newenvironment{figure*}
               {\@dblfloat{figure}}
               {\end@dblfloat}
\newcounter{table}[chapter]
%    \end{macrocode}
% \DescribeMacro{\thetable}\DescribeMacro{\fnum@table} The same changes as in 
% \verb|\thefigure| and \verb|\fnum@figure| are applied to \verb|\thetable| 
% and \verb|\fnum@table|:
%    \begin{macrocode}
\renewcommand \thetable
   {\ifnum \c@chapter>\z@ \thechapter\fi \@arabic\c@table.}
\def\fps@table{tbp}
\def\ftype@table{2}
\def\ext@table{lot}
\def\fnum@table{\textbf{\thetable\nobreakspace\tablename}}
\newenvironment{table}
               {\@float{table}}
               {\end@float}
\newenvironment{table*}
               {\@dblfloat{table}}
               {\end@dblfloat}
\newlength\abovecaptionskip
\newlength\belowcaptionskip
\setlength\abovecaptionskip{10\p@}
%    \end{macrocode}
% \DescribeMacro{\belowcaptionskip} The length of \verb|\belowcaptionskip| 
% is changed from \verb|0p@| to \verb|10p@| in order to improve 
% the readability of the captions of figures and tables:
%    \begin{macrocode}
\setlength\belowcaptionskip{10\p@}
%    \end{macrocode}
% \DescribeMacro{\captionwidth} For the same reason, the width
% for table and figure captions is redefined:
%    \begin{macrocode}
\newlength{\@contcwidth}
\newcommand{\captionwidth}[1]{\setlength{\@contcwidth}{#1}}
\captionwidth{0.85\textwidth}
%    \end{macrocode}
% \DescribeMacro{\@makecaption} The ``:'' is changed by \verb|\textbf{.}| in the
% caption environment. Moreover, the font size of the caption is changed to 
% \verb|\small| and the new caption width is
% applied\footnote{Changes applied in the caption environment are
% the main changes from version v1.05 to version v1.20.}:
%    \begin{macrocode}
\long\def\@makecaption#1#2{%
\centering
\parbox{\@contcwidth}{
  \vskip\abovecaptionskip
  \sbox\@tempboxa{\small #1\textbf{.} #2}%
  \ifdim \wd\@tempboxa >\hsize
    \small #1\textbf{.} #2\par
  \else
    \global \@minipagefalse
    \hb@xt@\hsize{\hfil\box\@tempboxa\hfil}%
  \fi
  \vskip\belowcaptionskip}%end parbox
} %end @makecaption
\DeclareOldFontCommand{\rm}{\normalfont\rmfamily}{\mathrm}
\DeclareOldFontCommand{\sf}{\normalfont\sffamily}{\mathsf}
\DeclareOldFontCommand{\tt}{\normalfont\ttfamily}{\mathtt}
\DeclareOldFontCommand{\bf}{\normalfont\bfseries}{\mathbf}
\DeclareOldFontCommand{\it}{\normalfont\itshape}{\mathit}
\DeclareOldFontCommand{\sl}{\normalfont\slshape}{\@nomath\sl}
\DeclareOldFontCommand{\sc}{\normalfont\scshape}{\@nomath\sc}
\DeclareRobustCommand*\cal{\@fontswitch\relax\mathcal}
\DeclareRobustCommand*\mit{\@fontswitch\relax\mathnormal}
\newcommand\@pnumwidth{1.55em}
\newcommand\@tocrmarg{2.55em}
\newcommand\@dotsep{4.5}
\setcounter{tocdepth}{2}
\newcommand\tableofcontents{%
    \if@twocolumn
      \@restonecoltrue\onecolumn
    \else
      \@restonecolfalse
    \fi
    \chapter*{\contentsname
        \@mkboth{%
           \MakeUppercase\contentsname}{\MakeUppercase\contentsname}}%
    \@starttoc{toc}%
    \if@restonecol\twocolumn\fi
    }
\newcommand*\l@part[2]{%
  \ifnum \c@tocdepth >-2\relax
    \addpenalty{-\@highpenalty}%
    \addvspace{2.25em \@plus\p@}%
    \setlength\@tempdima{3em}%
    \begingroup
      \parindent \z@ \rightskip \@pnumwidth
      \parfillskip -\@pnumwidth
      {\leavevmode
       \large \bfseries #1\hfil \hb@xt@\@pnumwidth{\hss #2}}\par
       \nobreak
         \global\@nobreaktrue
         \everypar{\global\@nobreakfalse\everypar{}}%
    \endgroup
  \fi}
\newcommand*\l@chapter[2]{%
  \ifnum \c@tocdepth >\m@ne
    \addpenalty{-\@highpenalty}%
    \vskip 1.0em \@plus\p@
    \setlength\@tempdima{1.5em}%
    \begingroup
      \parindent \z@ \rightskip \@pnumwidth
      \parfillskip -\@pnumwidth
      \leavevmode \bfseries
      \advance\leftskip\@tempdima
      \hskip -\leftskip
      #1\nobreak\hfil \nobreak\hb@xt@\@pnumwidth{\hss #2}\par
      \penalty\@highpenalty
    \endgroup
  \fi}
\newcommand*\l@section{\@dottedtocline{1}{1.5em}{2.3em}}
\newcommand*\l@subsection{\@dottedtocline{2}{3.8em}{3.2em}}
\newcommand*\l@subsubsection{\@dottedtocline{3}{7.0em}{4.1em}}
\newcommand*\l@paragraph{\@dottedtocline{4}{10em}{5em}}
\newcommand*\l@subparagraph{\@dottedtocline{5}{12em}{6em}}
\newcommand\listoffigures{%
    \if@twocolumn
      \@restonecoltrue\onecolumn
    \else
      \@restonecolfalse
    \fi
    \chapter*{\listfigurename}%
      \@mkboth{\MakeUppercase\listfigurename}%
              {\MakeUppercase\listfigurename}%
    \@starttoc{lof}%
    \if@restonecol\twocolumn\fi
    }
\newcommand*\l@figure{\@dottedtocline{1}{1.5em}{2.3em}}
\newcommand\listoftables{%
    \if@twocolumn
      \@restonecoltrue\onecolumn
    \else
      \@restonecolfalse
    \fi
    \chapter*{\listtablename}%
      \@mkboth{%
          \MakeUppercase\listtablename}%
         {\MakeUppercase\listtablename}%
    \@starttoc{lot}%
    \if@restonecol\twocolumn\fi
    }
\let\l@table\l@figure
\newdimen\bibindent
\setlength\bibindent{1.5em}
\newenvironment{thebibliography}[1]
     {\chapter*{\bibname}%
      \@mkboth{\MakeUppercase\bibname}{\MakeUppercase\bibname}%
      \list{\@biblabel{\@arabic\c@enumiv}}%
           {\settowidth\labelwidth{\@biblabel{#1}}%
            \leftmargin\labelwidth
            \advance\leftmargin\labelsep
            \@openbib@code
            \usecounter{enumiv}%
            \let\p@enumiv\@empty
            \renewcommand\theenumiv{\@arabic\c@enumiv}}%
      \sloppy
      \clubpenalty4000
      \@clubpenalty \clubpenalty
      \widowpenalty4000%
      \sfcode`\.\@m}
     {\def\@noitemerr
       {\@latex@warning{Empty `thebibliography' environment}}%
      \endlist}
\newcommand\newblock{\hskip .11em\@plus.33em\@minus.07em}
\let\@openbib@code\@empty
\newenvironment{theindex}
               {\if@twocolumn
                  \@restonecolfalse
                \else
                  \@restonecoltrue
                \fi
                \twocolumn[\@makeschapterhead{\indexname}]%
                \@mkboth{\MakeUppercase\indexname}%
                        {\MakeUppercase\indexname}%
                \thispagestyle{plain}\parindent\z@
                \parskip\z@ \@plus .3\p@\relax
                \columnseprule \z@
                \columnsep 35\p@
                \let\item\@idxitem}
               {\if@restonecol\onecolumn\else\clearpage\fi}
\newcommand\@idxitem{\par\hangindent 40\p@}
\newcommand\subitem{\@idxitem \hspace*{20\p@}}
\newcommand\subsubitem{\@idxitem \hspace*{30\p@}}
\newcommand\indexspace{\par \vskip 10\p@ \@plus5\p@ \@minus3\p@\relax}
\renewcommand\footnoterule{%
  \kern-3\p@
  \hrule\@width.4\columnwidth
  \kern2.6\p@}
\@addtoreset{footnote}{chapter}
\newcommand\@makefntext[1]{%
    \parindent 1em%
    \noindent
    \hb@xt@1.8em{\hss\@makefnmark}#1}
%    \end{macrocode}
% \DescribeMacro{\contentsname}\DescribeMacro{\listfigurename}
% \DescribeMacro{\listtablename}\DescribeMacro{\bibname}\DescribeMacro{\indexname} 
% \DescribeMacro{\figurename}
% The names for the table of contents, list of figures, list of tables, bibliography,
% etc. are defined in Basque taking into account the uppercase and lowercase 
% letters\footnote{Uppercase and lowercase letters are not correctly addressed
% by the basque babel package. For this reason, the use of this package is not
% recommended.}:
%    \begin{macrocode}
\newcommand\contentsname{Aurkibidea}
\newcommand\listfigurename{Irudien zerrenda}
\newcommand\listtablename{Taulen zerrenda}
\newcommand\bibname{Bibliografia}
\newcommand\indexname{Indizea}
\newcommand\figurename{irudia}
\newcommand\tablename{taula}
\newcommand\partname{atala}
\newcommand\chaptername{k\vspace{0.01cm}apitulua}
\newcommand\appendixname{eranskina}
%    \end{macrocode}
% \DescribeMacro{\today}Finally, the current date is defined using the
% command \verb|\eusdata| provided by the package \verb|basque-date| 
% (available from CTAN):
%    \begin{macrocode}
\def\today{\eusdata}
\setlength\columnsep{10\p@}
\setlength\columnseprule{0\p@}
\pagestyle{headings}
\pagenumbering{arabic}
\if@twoside
\else
  \raggedbottom
\fi
\if@twocolumn
  \twocolumn
  \sloppy
  \flushbottom
\else
  \onecolumn
\fi     
%    \end{macrocode}
%
% \Finale
\endinput