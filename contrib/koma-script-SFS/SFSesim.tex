%%
%% SFSesim.lco
%%
%% Copyright (c) Hannu V�is�nen  2005
%%
%% This work may be distributed and/or modified under the
%% conditions of the LaTeX Project Public License, either version 1.3
%% of this license or (at your option) any later version.
%% The latest version of this license is in
%% http://www.latex-project.org/lppl.txt
%% and version 1.3 or later is part of all distributions of LaTeX
%% version 2003/12/01 or later.
%%
%% This work has the LPPL maintenance status "maintained".
%% Maintainer is Hannu V�is�nen <hvaisane@joyx.joensuu.fi>
%%
%% This work consists of the file SFS.lco and SFSesim.tex.
%%
\documentclass{scrlttr2}
\usepackage[latin9]{inputenc}
\usepackage[T1]{fontenc}
\usepackage{lmodern}
\usepackage[finnish]{babel}


% Jos haluat kappaleet sisennettyin�, kommentoi seuraavat kaksi rivi�.
\setlength{\parindent}{0pt}
\setlength{\parskip}{1ex plus 0.5ex minus 0.2ex}


\LoadLetterOption{SFS}


% Kappale 6.4.1, sivu 154.
%
\setkomavar{fromname}{Liisa L�hett�j�}
\setkomavar{fromaddress}{Asuntokatu 456 \\ 12345 Olematon}
\setkomavar{fromphone}{123 456 789}
\setkomavar{fromfax}{987 654 321}
\setkomavar{fromemail}{Etunimi.Sukunimi@osoite.fi}
\setkomavar{fromurl}{www.yritys.osoite.fi}
\KOMAoptions{foldmarks=false}
%\KOMAoptions{fromphone,fromemail,fromurl}


% L�hett�j�n tiedot kirjeen alareunaan.
\firstfoot{%
  \parbox[t]{\textwidth}{
    \begin{tabular}[t]{llll}
       Liisa L�hett�j�  & Puhelin & Faksi& S�hk�posti \\
       Asuntokatu 4456  & \usekomavar{fromphone}
                             & \usekomavar{fromfax}
                            & \usekomavar{fromemail} \\
       12345 Olematon
    \end{tabular}%
  }
}


\begin{document}
\begin{letter}{Ville Vastaanottaja \\
Asuntokatu 123 \\
12345 Olematon}


\opening{Aloitusfraasi tulee t�h�n}

\C{0}

Otsikko alkaa sarakkeesta 0

\C{2}

Leip�teksti alkaa sarakkeesta 2.

T�m� kirje on esimerkki SFS-standardin mukaisesta kirjeest�.

Kirje k�ytt�� \KOMAScript'in scrlttr2-pakettia, johon on
m��ritelty konfigurointitiedosto SFS.lco. Enemm�n kommentteja
on sen alussa.

\C{0}

Toinen otsikko

\C{2}

Sarakkeet m��ritell��n komennolla \verb=\C{n}=,
miss� n voi olla 0, 1, 2, 3, 4, 5, 6, 7 tai 8.

\verb=\C{0}= on sama kuin vasen marginaali, ja
\verb=\C{8}= on sama kuin oikea marginaali.

\C{0}


%\cc{Huu Haa Hoo}
%\encl{Juupa joo.}

\closing{Lopetusfraasi tulee t�h�n}
\end{letter}
\end{document}
