\documentclass{book}
\usepackage[repeat]{poetry}
\usepackage{drama}
\usepackage{example}

\hfuzz 1pt

\HouseStyle {arden}
%\HouseStyle {penguin}


\begin{document}

\ExampleTitle {William Shakespeare}{Antony and Cleopatra}
              {\ifthenelse {\equal {\housestyle}{arden}}
                           {The Arden Shakespeare\\[.5ex]
                             University Paperbacks, Methuen, 1965}
                           {New Penguin Shakespeare\\[.5ex]
                             Penguin Books}
              }


\FrontMatter {Antony and Cleopatra}

\parindent 1em 

\chapter [Reviser's preface]{Preface to revised edition}

\noindent\lorem[32421] 

\section {The Text}

\noindent\lorem[41322] 

\section {The Apparatus Criticus}

\noindent\lorem[3225] 

\section {Introduction, Notes, and Appendices}

\noindent\lorem[3422] 

\chapter {Introduction}

\noindent\lorem[223113] 

%%%%%%%%%%%%%%%%%%%%%%%%%%%%%%%%%%%%%%%%%%%%%%%%%%%%%%%%%%%%%%%%%
%%%%%%%%%%%%%%%%%%%%%%%%%%%%%%%%%%%%%%%%%%%%%%%%%%%%%%%%%%%%%%%%%
%%%%%%%%%%%%%%%%%%%%%%%%%%%%%%%%%%%%%%%%%%%%%%%%%%%%%%%%%%%%%%%%%

\MainMatter {Antony and Cleopatra}

\begin {DramatisPersonae}

\begin {Characters} {triumvirs}
  \persona{Antony,}  
  \persona{Octavius C\ae sar,} 
  \persona{Lepidus,}
\end{Characters}

\persona{Sextus Pompeius.}

\begin{Characters}  {friends of Antony}
  \persona{Domitius Enobarbus,}
  \persona{Ventidius,}
  \persona{Eros,}
  \persona{Scarus,} 
  \persona{Decretas,}
  \persona{Demetrius,}
  \persona{Philo}
\end{Characters}

\begin{Characters}  {friends of C\ae sar}
  \persona{M\ae cenas,}
  \persona{Agrippa,}
  \persona{Dolabella,}
  \persona{Proculeius,\kern .4em}
  \persona{Thidias,} 
  \persona{Gallus,}
\end{Characters}

\begin{Characters}  {friends of Pompey}
  \persona{Menas,}
  \persona{Menecrates,}
  \persona{Varrius,}
\end{Characters}

\persona{Tarus,} lieutenant-general to C\ae sar. 
\persona{Canidius,} lieutenant-general to Antony.
\persona{Silius,} an officer in Ventidius' army.

\begin{unnamed}
A `schoolmaster' acting as ambassador from Antony to C\ae sar.
\end{unnamed}

\begin{Characters} [.85]{attendants on Cleopatra} 
\persona{Alexas,}
\persona{Mardian,} a eunuch, 
\persona{Diomedes,}
\end{Characters}

\persona{Seleucus} treasurer to Cleopatra.

\begin{unnamed}
  A soothsayer.
  A Clown.
\end{unnamed}

\ifthenelse {\equal {\housestyle}{penguin}}{\newpage}{}

\persona{Cleopatra,} queen of Egypt.
\persona{Octavia,} C\ae sar's sister.

\begin{Characters} [.8]{attendants on Cleopatra} 
  \persona{Charmian,} 
  \persona{Iras,}
\end{Characters} 

\Facies \titulus {#1}
\spatium{.5\leading minus .25\leading}
\titulus{Officers, Soldiers, messengers, and other attendants.}
\spatium{.5\leading minus .25\leading \penalty 1000}
\titulus{\textsc{Scene:} In several parts of the Roman empire.}

\ifthenelse {\equal {\housestyle}{penguin}}{\newpage}{}

\end {DramatisPersonae}

%%%%%%%%%%%%%%%%%%%%%%%%%%%%%%%%%%%%%%%%%%%%%%%%%%%%%%%%%%%%%%%%%
%%%%%%%%%%%%%%%%%%%%%%%%%%%%%%%%%%%%%%%%%%%%%%%%%%%%%%%%%%%%%%%%%
%%%%%%%%%%%%%%%%%%%%%%%%%%%%%%%%%%%%%%%%%%%%%%%%%%%%%%%%%%%%%%%%%

\ifthenelse {\equal {\housestyle}{arden}}
  {\persona*[1]{Ant\\Antony}
  \persona*[2]{Cleo\\Cleopatra}
  \persona*[3]{C\ae{}s\\Caesar}
  \persona*[4]{Agr\\Agrippa}
  \persona*[5]{Ale\\Alexas}
  \persona*[6]{Can\\Canidius}
  \persona*[7]{Char\\Charmian}
  \persona*[8]{Dec\\Decretas}
  \persona*[9]{Dem\\Demetrius}
  \persona*[10]{Dio\\Diomedes}
  \persona*[11]{Dol\\Dolabella}
  \persona*[12]{Egyp\\Egyptian}
  \persona*[13]{Eno\\Enobarbus}
  \persona*[14]{Eros\\Eros}
  \persona*[15]{Gal\\Gallus}
  \persona*[16]{Iras\\Iras}
  \persona*[17]{Lep\\Lepidus}
  \persona*[18]{Mar\\Mardian}
  \persona*[19]{M{\ae}c\\Mecaenas}
  \persona*[20]{Men\\Menas}
  \persona*[21]{Mene\\Menecrates}
  \persona*[22]{Oct\\Octavia}
  \persona*[23]{Phi\\Philo}
  \persona*[24]{Pom\\Pompey}
  \persona*[25]{Pro\\Proculeius}
  \persona*[26]{Scar\\Scarus}
  \persona*[27]{Sel\\Seleucus}
  \persona*[28]{Sil\\Silius}
  \persona*[29]{Taur\\Taurus}
  \persona*[30]{Thid\\Thidias}
  \persona*[31]{Var\\Varius}
  \persona*[32]{Ven\\Ventidius}
  \persona*[50]{Amb\\Ambassador}
  \persona*[51]{Att\\Attendant}
  \persona*[52]{First Att}
  \persona*[53]{Sec. Att}
  \persona*[54]{Capt}
  \persona*[55]{Clown}
  \persona*[56]{Guard}
  \persona*[57]{First Guard}
  \persona*[58]{Sec. Guard}
  \persona*[59]{Third Guard}
  \persona*[60]{Mess\\Messenger}
  \persona*[61]{Sec Mess}
  \persona*[62]{Sent\\Sentry}
  \persona*[63]{Ser\\Servant}
  \persona*[64]{First Ser}
  \persona*[65]{Sec. Ser}
  \persona*[66]{Sold}
  \persona*[67]{First Sold}
  \persona*[68]{Sec. Sold}
  \persona*[69]{Third Sold}
  \persona*[70]{Fourth Sold}
  \persona*[71]{Sooth\\Soothsayer}
  \persona*[72]{First Watch}
  \persona*[73]{Sec. Watch}
  \persona*[99]{All}
  }
  {\persona*[1]{Antony}
  \persona*[2]{Cleopatra}
  \persona*[3]{Caesar}
  \persona*[4]{Agrippa}
  \persona*[5]{Alexas}
  \persona*[6]{Canidius}
  \persona*[7]{Charmian}
  \persona*[8]{Decretas}
  \persona*[9]{Demetrius}
  \persona*[10]{Diomedes}
  \persona*[11]{Dolabella}
  \persona*[12]{Egyptian}
  \persona*[13]{Enobarbus}
  \persona*[14]{Eros}
  \persona*[15]{Gallus}
  \persona*[16]{Iras}
  \persona*[17]{Lepidus}
  \persona*[18]{Mardian}
  \persona*[19]{Mecaenas}
  \persona*[20]{Menas}
  \persona*[21]{Menecrates}
  \persona*[22]{Octavia}
  \persona*[23]{Philo}
  \persona*[24]{Pompey}
  \persona*[25]{Proculeius}
  \persona*[26]{Scarus}
  \persona*[27]{Seleucus}
  \persona*[28]{Silius}
  \persona*[29]{Taurus}
  \persona*[30]{Thidias}
  \persona*[31]{Varius}
  \persona*[32]{Ventidius}
  \persona*[50]{Ambassador}
  \persona*[51]{Attendant}
  \persona*[52]{First Attendant}
  \persona*[53]{Sec. Attendant}
  \persona*[54]{Captain}
  \persona*[55]{Clown}
  \persona*[56]{Guard}
  \persona*[57]{First Guard}
  \persona*[58]{Sec. Guard}
  \persona*[59]{Third Guard}
  \persona*[60]{Messenger}
  \persona*[61]{Sec Messenger}
  \persona*[62]{Sentry}
  \persona*[63]{Servant}
  \persona*[64]{First Servant}
  \persona*[65]{Sec. Servant}
  \persona*[66]{Soldier}
  \persona*[67]{First Soldier}
  \persona*[68]{Sec. Soldier}
  \persona*[69]{Third Soldier}
  \persona*[70]{Fourth Soldier}
  \persona*[71]{Soothsayer}
  \persona*[72]{First Watch}
  \persona*[73]{Sec. Watch}
  \persona*[99]{All}
  }

%%%%%%%%%%%%%%%%%%%%%%%%%%%%%%%%%%%%%%%%%%%%%%%%%%%%%%%%%%%%%%%%%
%%%%%%%%%%%%%%%%%%%%%%%%%%%%%%%%%%%%%%%%%%%%%%%%%%%%%%%%%%%%%%%%%
%%%%%%%%%%%%%%%%%%%%%%%%%%%%%%%%%%%%%%%%%%%%%%%%%%%%%%%%%%%%%%%%%

\thispagestyle{empty}
\pagestyle{MainMatterPage}
\Drama
\Versus
\numerus{1}

\Act
\Scene {Alexandria. A room in Cleopatra's palace.}

	\(Enter \9 and \23\)

\23 Nay, but this dotage of our general's
	O'erflows the measure: those his goodly eyes,
	That o'er the files and musters of the war
   Have glow'd like plated Mars, now bend, now turn, 
	The office and devotion of their view
	Upon a tawny front: his captain's heart, 
	Which in the scuffles of great fights hath burst
	The buckles on his breast, reneges all temper,
	And is become the bellows and the fan 
	To cool a gipsy's lust. \\

	\(Flourish. Enter \1, \2, her Ladies,
	  the Train, with Eunuchs fanning her.\)

		  Look, where they come:
	Take but good note, and you shall see in him. 
	The triple pillar of the world transform'd 
	Into a strumpet's fool: behold and see. 

\2 If it be love indeed, tell me how much.

\1	There's beggary in the love that can be reckon'd.

\2	I'll set a bourn how far to be beloved.

\1	Then must thou needs find out new heaven, new earth.

	\(Enter an Attendant.\)

\51	News, my good lord, from Rome. \\

\1	Grates me: the sum. 

\2	Nay, hear them, Antony:
	Fulvia perchance is angry; or, who knows
	If the scarce-bearded Caesar have not sent
	His powerful mandate to you, `Do this, or this;
	Take in that kingdom, and enfranchise that;
	Perform 't, or else we damn thee.' \\

\1	How, my love!

\2	Perchance! nay, and most like:
	You must not stay here longer, your dismission
	Is come from Caesar; therefore hear it, Antony.
	Where's Fulvia's process? Caesar's I would say? both?
	Call in the messengers. As I am Egypt's queen,
	Thou blushest, Antony; and that blood of thine
	Is Caesar's homager: else so thy cheek pays shame
	When shrill-tongued Fulvia scolds. The messengers!

\1	Let Rome in Tiber melt, and the wide arch
	Of the ranged empire fall! Here is my space.
	Kingdoms are clay: our dungy earth alike
	Feeds beast as man: the nobleness of life
	Is to do thus; when suche a mutual pair	\[r]Embracing.\]
	And such a twain can do't, in which I bind,
	On pain of punishment, the world to weet
	We stand up peerless. \\

\2	Excellent falsehood!
	Why did he marry Fulvia, and not love her?
	I'll seem the fool I am not; Antony 
	Will be himself. \\

\1	                  But stirr'd by Cleopatra.
	Now, for the love of Love and her soft hours, 
        Let's not confound the time with conference harsh: 
	There's not a minute of our lives should stretch 
	Without some pleasure now. What sport tonight?

\2	Hear the ambassadors. \\

\1	Fie, wrangling queen!
	Whom every thing becomes, to chide, to laugh,
	To weep; whose every passion fully strives
	To make itself, in thee, fair and admired!
	No messenger, but thine; and all alone
	To-night we'll wander through the streets and note
	The qualities of people. Come, my queen;
	Last night you did desire it: speak not to us.
	\[r]Exeunt \1 and \2 with their train.\]

\9	Is Caesar with Antonius prized so slight?

\23	Sir, sometimes, when he is not Antony,
	He comes too short of that great property
	Which still should go with Antony. \\

\9	I am full sorry
	That he approves the common liar, who
	Thus speaks of him at Rome: but I will hope
	Of better deeds to-morrow. Rest you happy! \[r]Exeunt.\]

\Scene{The same. Another room.}

\(Enter \7, \16, \5, and a \71\)

\begin{PROSE}

\7	Lord Alexas, sweet Alexas, most any thing Alexas,
	almost most absolute Alexas, where's the soothsayer
	that you praised so to the queen? O, that I knew
	this husband, which, you say, must charge his horns
	with garlands! 

\5	Soothsayer! 

\71	Your will?

\7	Is this the man? Is't you, sir, that know things?

\end{PROSE}

\71	In nature's infinite book of secrecy
	A little I can read.\\

\5	Show him your hand.

	\(Enter \13\)


\13	Bring in the banquet quickly; wine enough
	Cleopatra's health to drink. 

\begin{PROSE}

\7	Good sir, give me good fortune.

\71	I make not, but foresee. 

\7	Pray, then, foresee me one.

\71	You shall be yet far fairer than you are.

\7	He means in flesh. 

\16	No, you shall paint when you are old.

\7	Wrinkles forbid! 

\5	Vex not his prescience; be attentive.

\7	Hush!

\71	You shall be more beloving than beloved.

\7	I had rather heat my liver with drinking.

\5	Nay, hear him.

\7	Good now, some excellent fortune! Let me be married
	to three kings in a forenoon, and widow them all:
	let me have a child at fifty, to whom Herod of Jewry
	may do homage: find me to marry me with Octavius
	Caesar,  and companion me with my mistress.

\71	You shall outlive the lady whom you serve.

\7	O excellent! I love long life better than figs.

\71	You have seen and proved a fairer former fortune
	Than that which is to approach.

\7	Then belike my children shall have no names: 
	pri\-thee, how many boys and wenches must I have?

\71	If every of your wishes had a womb.
	And fertile every wish, a million.

\7	Out, fool! I forgive thee for a witch.

\5	You think none but your sheets are privy to your wishes.

\7	Nay, come, tell Iras hers.

\5	We'll know all our fortunes.

\13	Mine, and most of our fortunes, to-night, shall
	be--drunk to bed.

\16	There's a palm presages chastity, if nothing else.

\7	E'en as the o'erflowing Nilus presageth famine.

\16	Go, you wild bedfellow, you cannot soothsay.

\7	Nay, if an oily palm be not a fruitful
	prognostication, I cannot scratch mine ear. Prithee,
	tell her but a worky-day fortune.

\71	Your fortunes are alike.

\16	But how, but how? give me particulars.

\71	I have said.

\16	Am I not an inch of fortune better than she?

\7	Well, if you were but an inch of fortune better than
	I, where would you choose it?

\16	Not in my husband's nose.

\7	Our worser thoughts heavens mend! Alexas,---come,
	his fortune, his fortune! O, let him marry a woman
	that cannot go, sweet Isis, I beseech thee! and let
	her die too, and give him a worse! and let worst
	follow worse, till the worst of all follow him
	laughing to his grave, fifty-fold a cuckold! Good
	Isis, hear me this prayer, though thou deny me a
	matter of more weight; good Isis, I beseech thee!

\16	Amen. Dear goddess, hear that prayer of the people!
	for, as it is a heartbreaking to see a handsome man
	loose-wived, so it is a deadly sorrow to behold a
	foul knave uncuckolded: therefore, dear Isis, keep
	decorum, and fortune him accordingly!

\7	Amen.

\5	Lo, now, if it lay in their hands to make me a
	cuckold, they would make themselves whores, but
	they'ld do't!

\end{PROSE}

\13	Hush! here comes Antony. \\

\7	Not he; the queen.

	\(Enter \2.\)

\2	Saw you my lord? \\

\13	                  No, lady. \\

\2	Was he not here? 

\7	No, madam.

\2	He was disposed to mirth; but on the sudden
	A Roman thought hath struck him. Enobarbus!

\13	Madam?

\2	Seek him, and bring him hither. 	Where's Alexas?

\5	Here, at your service. My lord approaches.

\2	We will not look upon him: go with us. \[r]Exeunt.\]

	\(Enter \1 with a Messenger and Attendants.\)

\60	Fulvia thy wife first came into the field.

\1	Against my brother Lucius?

\60	Ay:
	But soon that war had end, and the time's state
	Made friends of them, joining their force 'gainst Caesar;
	Whose better issue in the war, from Italy,
	Upon the first encounter, drave them. \\

\1	Well, what worst?

\60	The nature of bad news infects the teller.

\1	When it concerns the fool or coward. On:
	Things that are past are done with me. 'Tis thus:
	Who tells me true, though in his tale lie death,
	I hear him as he flatter'd. \\

\60	Labienus---
	This is stiff news---hath, with his Parthian force,
	Extended Asia from Euphrates;
	His conquering banner shook from Syria
	To Lydia and to Ionia;
        Whilst--- \\

\1	Antony, thou wouldst say,--- \\

\60	O, my lord!

\1	Speak to me home, mince not the general tongue:
	Name Cleopatra as she is call'd in Rome;
	Rail thou in Fulvia's phrase; and taunt my faults
	With such full licence as both truth and malice
	Have power to utter. O, then we bring forth weeds,
	When our quick minds lie still; and our ills told us
	Is as our earing. Fare thee well awhile.

\60	At your noble pleasure. \[r]Exit.\]

\1	From Sicyon, ho, the news! Speak there!

\52	The man from Sicyon,---is there such an one?

\53	He stays upon your will. \\

\1	Let him appear.
	These strong Egyptian fetters I must break,
	Or lose myself in dotage. \\

	\(Enter another Messenger.\)

		    What are you?

\61	Fulvia thy wife is dead. \\

\1	Where died she?

\61	In Sicyon: 
	Her length of sickness, with what else more serious
	Importeth thee to know, this bears. \\   \[r]Gives a letter.\]

\1	Forbear me.
	\[r]Exit Second Messenger.\]
	There's a great spirit gone! Thus did I desire it:
	What our contempt doth often hurl from us, 
        We wish it ours again; the present pleasure, 
	By revolution lowering, does become
	The opposite of itself: she's good, being gone;
	The hand could pluck her back that shoved her on.
	I must from this enchanting queen break off:
	Ten thousand harms, more than the ills I know,
	My idleness doth hatch. How now! Enobarbus!

	\(Re-enter \13.\)

\begin{PROSE}

\13	What's your pleasure, sir?

\1	I must with haste from hence.

\13	Why, then, we kill all our women:
	we see how mortal an unkindness is to them;
	if they suffer our departure, death's the word.

\1	I must be gone.

\13	Under a compelling occasion, let women die; it were
	pity to cast them away for nothing; though, between
	them and a great cause, they should be esteemed
	nothing. Cleopatra, catching but the least noise of
	this, dies instantly; I have seen her die twenty
	times upon far poorer moment: I do think there is
	mettle in death, which commits some loving act upon
	her, she hath such a celerity in dying.

\1	She is cunning past man's thought.

\13	Alack, sir, no; her passions are made of nothing but
	the finest part of pure love: we cannot call her
	winds and waters sighs and tears; they are greater
	storms and tempests than almanacs can report: this
	cannot be cunning in her; if it be, she makes a
	shower of rain as well as Jove.

\1	Would I had never seen her.

\13	O, sir, you had then left unseen a wonderful piece
	of work; which not to have been blest withal would
	have discredited your travel.

\1	Fulvia is dead.

\13	Sir?

\1	Fulvia is dead.

\13	Fulvia!

\1	Dead.

\13	Why, sir, give the gods a thankful sacrifice. When
	it pleaseth their deities to take the wife of a man
	from him, it shows to man the tailors of the earth;
	comforting therein, that when old robes are worn
	out, there are members to make new. If there were
	no more women but Fulvia, then had you indeed a cut,
	and the case to be lamented: this grief is crowned
	with consolation; your old smock brings forth a new
	petticoat: and indeed the tears live in an onion
	that should water this sorrow.

\1	The business she hath broached in the state
	Cannot endure my absence.

\13	And the business you have broached here cannot be
	without you; especially that of Cleopatra's, which
	wholly depends on your abode.

\end{PROSE}

\1	No more light answers. Let our officers
	Have notice what we purpose. I shall break
	The cause of our expedience to the queen,
	And get her leave to part. For not alone
	The death of Fulvia, with more urgent touches,
	Do strongly speak to us; but the letters too
	Of many our contriving friends in Rome
	Petition us at home: Sextus Pompeius 
	Hath given the dare to Caesar, and commands
	The empire of the sea: our slippery people,
	Whose love is never link'd to the deserver
	Till his deserts are past, begin to throw
	Pompey the Great and all his dignities 
	Upon his son; who, high in name and power,
	Higher than both in blood and life, stands up
	For the main soldier: whose quality, going on,
	The sides o' the world may danger: much is breeding,
	Which, like the courser's hair, hath yet but life,
	And not a serpent's poison. Say, our pleasure,
	To such whose place is under us, requires
	Our quick remove from hence.

\13	I shall do't. 	\[r]Exeunt.\]


\Scene{The same. Another room.}


	\(Enter \2, \7, \16, and \9.\)

\2	Where is he? \\

\7	                  I did not see him since.

\2	See where he is, who's with him, what he does:
	I did not send you: if you find him sad,
	Say I am dancing; if in mirth, report
	That I am sudden sick: quick, and return. \[r]Exit \9\]

\7	Madam, methinks, if you did love him dearly,
	You do not hold the method to enforce
	The like from him. \\

\2	                  What should I do, I do not?

\7	In each thing give him way, cross him nothing.

\2	Thou teachest like a fool; the way to lose him.

\7	Tempt him not so too far; I wish, forbear:
	In time we hate that which we often fear.

	\(Enter \1.\)

	But here comes Antony. \\

\2	I am sick and sullen.

\1	I am sorry to give breathing to my purpose,---

\2	Help me away, dear Charmian; I shall fall:
	It cannot be thus long, the sides of nature
	Will not sustain it. \\

\1	Now, my dearest queen,---

\2	Pray you, stand further from me. \\

\1	What's the matter?

\2	I know, by that same eye, there's some good news.
	What says the married woman? You may go:
	Would she had never given you leave to come!
	Let her not say 'tis I that keep you here:
	I have no power upon you; hers you are.

\1	The gods best know,--- \\

\2	O, never was there queen
	So mightily betray'd! yet at the first
	I saw the treasons planted. \\

\1	Cleopatra,---

\2	Why should I think you can be mine and true,
	Though you in swearing shake the throned gods,
	Who have been false to Fulvia? Riotous madness,
	To be entangled with those mouth-made vows,
	Which break themselves in swearing! \\

\1	Most sweet queen,---

\2	Nay, pray you, seek no colour for your going,
	But bid farewell, and go: when you sued staying,
	Then was the time for words: no going then;
	Eternity was in our lips and eyes,
	Bliss in our brows' bent; none our parts so poor,
	But was a race of heaven: they are so still,
	Or thou, the greatest soldier of the world,
	Art turn'd the greatest liar. \\

\1	How now, lady!

\2	I would I had thy inches; thou shouldst know
	There were a heart in Egypt. \\

\1	Hear me, queen:
	The strong necessity of time commands
	Our services awhile; but my full heart
	Remains in use with you. Our Italy
	Shines o'er with civil swords: Sextus Pompeius
	Makes his approaches to the port of Rome:
	Equality of two domestic powers
	Breed scrupulous faction: the hated, grown to strength,
	Are newly grown to love: the condemn'd Pompey,
	Rich in his father's honour, creeps apace,
	Into the hearts of such as have not thrived
	Upon the present state, whose numbers threaten;
	And quietness, grown sick of rest, would purge
	By any desperate change: my more particular,
	And that which most with you should safe my going,
	Is Fulvia's death.

\2	Though age from folly could not give me freedom,
	It does from childishness: can Fulvia die?

\1	She's dead, my queen:
	Look here, and at thy sovereign leisure read
	The garboils she awaked; at the last, best:
	See when and where she died. \\

\2	O most false love!
	Where be the sacred vials thou shouldst fill
	With sorrowful water? Now I see, I see,
	In Fulvia's death, how mine received shall be.

\1	Quarrel no more, but be prepared to know
	The purposes I bear; which are, or cease,
	As you shall give the advice. By the fire
	That quickens Nilus' slime, I go from hence
	Thy soldier, servant; making peace or war
	As thou affect'st. \\

\2	                  Cut my lace, Charmian, come;
	But let it be: I am quickly ill, and well,
	So Antony loves. \\

\1	                  My precious queen, forbear;
	And give true evidence to his love, which stands
	An honourable trial. \\

\2	So Fulvia told me.
	I prithee, turn aside and weep for her,
	Then bid adieu to me, and say the tears
	Belong to Egypt: good now, play one scene
	Of excellent dissembling; and let it look
	Life perfect honour. \\

\1	You'll heat my blood: no more.

\2	You can do better yet; but this is meetly.

\1	Now, by my sword,--- \\

\2	And target. Still he mends;
	But this is not the best. Look, prithee, Charmian,
	How this Herculean Roman does become
	The carriage of his chafe. \\

\1	I'll leave you, lady.

\2	Courteous lord, one word.
	Sir, you and I must part, but that's not it:
	Sir, you and I have loved, but there's not it;
	That you know well: something it is I would,
	O, my oblivion is a very Antony,
	And I am all forgotten. \\

\1	But that your royalty
	Holds idleness your subject, I should take you
	For idleness itself. \\

\2	'Tis sweating labour
	To bear such idleness so near the heart
	As Cleopatra this. But, sir, forgive me;
	Since my becomings kill me, when they do not
	Eye well to you: your honour calls you hence;
	Therefore be deaf to my unpitied folly.
	And all the gods go with you! upon your sword
	Sit laurel victory! and smooth success
	Be strew'd before your feet! \\

\1	Let us go. Come;
	Our separation so abides, and flies,
	That thou, residing here, go'st yet with me,
	And I, hence fleeting, here remain with thee.
   Away! \[r]Exeunt.\]

\Scene{Caesar's house.}


	\(Enter \3, reading a letter, \17,
	and their Train.\)

\3	You may see, Lepidus, and henceforth know,
	It is not Caesar's natural vice to hate
	Our great competitor: from Alexandria
	This is the news: he fishes, drinks, and wastes
	The lamps of night in revel; is not more man-like
	Than Cleopatra; nor the queen of Ptolemy
	More womanly than he; hardly gave audience, or
	Vouchsafed to think he had partners: you shall find there
	A man who is the abstract of all faults
	That all men follow. \\

\17	I must not think there are
	Evils enow to darken all his goodness:
	His faults in him seem as the spots of heaven,
	More fiery by night's blackness; hereditary,
	Rather than purchased; what he cannot change,
	Than what he chooses.

\3	You are too indulgent. Let us grant, it is not
	Amiss to tumble on the bed of Ptolemy;
	To give a kingdom for a mirth; to sit
	And keep the turn of tippling with a slave;
	To reel the streets at noon, and stand the buffet
	With knaves that smell of sweat: say this becomes him,---
	As his composure must be rare indeed
	Whom these things cannot blemish,---yet must Antony
	No way excuse his soils, when we do bear
	So great weight in his lightness. If he fill'd
	His vacancy with his voluptuousness,
	Full surfeits, and the dryness of his bones,
	Call on him for't: but to confound such time,
	That drums him from his sport, and speaks as loud
	As his own state and ours,---'tis to be chid
	As we rate boys, who, being mature in knowledge,
	Pawn their experience to their present pleasure,
	And so rebel to judgment. \\

	\(Enter a Messenger.\)

\17	Here's more news.

\60	Thy biddings have been done; and every hour,
	Most noble Caesar, shalt thou have report
	How 'tis abroad. Pompey is strong at sea;
	And it appears he is beloved of those
	That only have fear'd Caesar: to the ports
	The discontents repair, and men's reports
	Give him much wrong'd. \\

\3	I should have known no less.
	It hath been taught us from the primal state,
	That he which is was wish'd until he were;
	And the ebb'd man, ne'er loved till ne'er worth love,
	Comes dear'd by being lack'd. This common body,
	Like to a vagabond flag upon the stream,
	Goes to and back, lackeying the varying tide,
	To rot itself with motion. \\

\61	Caesar, I bring thee word,
	Menecrates and Menas, famous pirates,
	Make the sea serve them, which they ear wound
	With keels of every kind: many hot inroads
	They make in Italy; the borders maritime
	Lack blood to think on't, and flush youth revolt:
	No vessel can peep forth, but 'tis as soon
	Taken as seen; for Pompey's name strikes more
	Than could his war resisted. \\

\3	Antony,
	Leave thy lascivious wassails. When thou once
	Wast beaten from Modena, where thou slew'st
	Hirtius and Pansa, consuls, at thy heel
	Did famine follow; whom thou fought'st against,
	Though daintily brought up, with patience more
	Than savages could suffer: thou didst drink
	The stale of horses, and the gilded puddle
	Which beasts would cough at: thy palate then did deign
	The roughest berry on the rudest hedge;
	Yea, like the stag, when snow the pasture sheets,
	The barks of trees thou browsed'st; on the Alps
	It is reported thou didst eat strange flesh, 
	Which some did die to look on: and all this---
	It wounds thine honour that I speak it now---
	Was borne so like a soldier, that thy cheek
	So much as lank'd not. \\

\17	'Tis pity of him.

\3	Let his shames quickly
	Drive him to Rome: 'tis time we twain
	Did show ourselves i' the field; and to that end
	Assemble we immediate council: Pompey
	Thrives in our idleness. \\

\17	To-morrow, Caesar,
	I shall be furnish'd to inform you rightly
	Both what by sea and land I can be able
	To front this present time. \\

\3	Till which encounter,
	It is my business too. Farewell.

\17	Farewell, my lord: what you shall know meantime
	Of stirs abroad, I shall beseech you, sir,
	To let me be partaker. \\

\3	Doubt not, sir;
	I knew it for my bond. 	\[r]Exeunt.\]



\Scene{Alexandria. Cleopatra's palace.}


	\(Enter \2, \7, \16, and \18.\)

\2	Charmian! 

\7	Madam? 

\2	Ha, ha! 
	Give me to drink mandragora. \\

\7	Why, madam?

\2	That I might sleep out this great gap of time
	My Antony is away. \\

\7	                  You think of him too much.

\2	O, 'tis treason!  \\

\7	                  Madam, I trust, not so.

\2	Thou, eunuch Mardian! \\

\18	What's your highness' pleasure?

\2	Not now to hear thee sing; I take no pleasure
	In aught an eunuch has: 'tis well for thee,
	That, being unseminar'd, thy freer thoughts
	May not fly forth of Egypt. Hast thou affections?

\18	Yes, gracious madam.

\2	Indeed!

\18	Not in deed, madam; for I can do nothing
	But what indeed is honest to be done:
	Yet have I fierce affections, and think
	What Venus did with Mars. \\

\2	O Charmian,
	Where think'st thou he is now? Stands he, or sits he?
	Or does he walk? or is he on his horse?
	O happy horse, to bear the weight of Antony!
	Do bravely, horse! for wot'st thou whom thou movest?
	The demi-Atlas of this earth, the arm
	And burgonet of men. He's speaking now,
	Or murmuring `Where's my serpent of old Nile?'
	For so he calls me: now I feed myself
	With most delicious poison. Think on me,
	That am with Phoebus' amorous pinches black,
	And wrinkled deep in time? Broad-fronted Caesar,
	When thou wast here above the ground, I was
	A morsel for a monarch: and great Pompey
	Would stand and make his eyes grow in my brow;
	There would he anchor his aspect and die
	With looking on his life. \\

	\(Enter \5, from Caesar.\)

\5	Sovereign of Egypt, hail!

\2	How much unlike art thou Mark Antony!
	Yet, coming from him, that great medicine hath
	With his tinct gilded thee.
	How goes it with my brave Mark Antony?

\5	Last thing he did, dear queen,
	He kiss'd,---the last of many doubled kisses,---
	This orient pearl. His speech sticks in my heart.

\2	Mine ear must pluck it thence. \\

\5	`Good friend,' quoth he,
	`Say, the firm Roman to great Egypt sends
	This treasure of an oyster; at whose foot,
	To mend the petty present, I will piece
	Her opulent throne with kingdoms; all the east,
	Say thou, shall call her mistress.' So he nodded,
	And soberly did mount an arm-gaunt steed,
	Who neigh'd so high, that what I would have spoke
	Was beastly dumb'd by him. \\

\2	What, was he sad or merry?

\5	Like to the time o' the year between the extremes
	Of hot and cold, he was nor sad nor merry.

\2	O well-divided disposition! Note him,
	Note him good Charmian, 'tis the man; but note him:
	He was not sad, for he would shine on those
	That make their looks by his; he was not merry,
	Which seem'd to tell them his remembrance lay
	In Egypt with his joy; but between both:
	O heavenly mingle! Be'st thou sad or merry,
	The violence of either thee becomes,
	So does it no man else. Met'st thou my posts?

\5	Ay, madam, twenty several messengers:
	Why do you send so thick? \\

\2	Who's born that day
	When I forget to send to Antony,  
	Shall die a beggar. Ink and paper, Charmian.
	Welcome, my good Alexas. Did I, Charmian,
	Ever love Caesar so? \\

\7	O that brave Caesar!

\2	Be choked with such another emphasis!
	Say, the brave Antony. \\

\7	The valiant Caesar!

\2	By Isis, I will give thee bloody teeth,
	If thou with Caesar paragon again
	My man of men. \\

\7	                  By your most gracious pardon,
	I sing but after you.  \\

\2	My salad days,
	When I was green in judgment: cold in blood,
	To say as I said then! But, come, away;
	Get me ink and paper:
	He shall have every day a several greeting,
	Or I'll unpeople Egypt.  \[r]Exeunt.\]

\Act
\Scene{Messina. Pompey's house.}


	\(Enter \24, \21, and \20, in warlike manner.\)

\24	If the great gods be just, they shall assist
	The deeds of justest men. \\

\21	Know, worthy Pompey,
	That what they do delay, they not deny.

\24	Whiles we are suitors to their throne, decays
	The thing we sue for. \\

\21	We, ignorant of ourselves,
	Beg often our own harms, which the wise powers
	Deny us for our good; so find we profit
	By losing of our prayers. \\

\24	I shall do well:
	The people love me, and the sea is mine;
	My powers are crescent, and my auguring hope
	Says it will come to the full. Mark Antony
	In Egypt sits at dinner, and will make
	No wars without doors: Caesar gets money where
	He loses hearts: Lepidus flatters both,
	Of both is flatter'd; but he neither loves,
	Nor either cares for him. \\

\20	Caesar and Lepidus
	Are in the field: a mighty strength they carry.

\24	Where have you this? 'tis false. \\

\20	From Silvius, sir.

\24	He dreams: I know they are in Rome together,
	Looking for Antony. But all the charms of love,
	Salt Cleopatra, soften thy waned lip!
	Let witchcraft join with beauty, lust with both!
	Tie up the libertine in a field of feasts,
	Keep his brain fuming; Epicurean cooks
	Sharpen with cloyless sauce his appetite;
	That sleep and feeding may prorogue his honour
	Even till a Lethe'd dulness! \\

	\(Enter \31.\)

		       How now, Varrius!

\31	This is most certain that I shall deliver:
	Mark Antony is every hour in Rome
	Expected: since he went from Egypt 'tis
	A space for further travel. \\

\24	I could have given less matter
	A better ear. Menas, I did not think
	This amorous surfeiter would have donn'd his helm
	For such a petty war: his soldiership
	Is twice the other twain: but let us rear
	The higher our opinion, that our stirring
	Can from the lap of Egypt's widow pluck
	The ne'er-lust-wearied Antony. \\

\20	I cannot hope
	Caesar and Antony shall well greet together:
	His wife that's dead did trespasses to Caesar;
	His brother warr'd upon him; although, I think,
	Not moved by Antony. \\

\24	I know not, Menas,
	How lesser enmities may give way to greater.
	Were't not that we stand up against them all,
	'Twere pregnant they should square between themselves;
	For they have entertained cause enough
	To draw their swords: but how the fear of us
	May cement their divisions and bind up
	The petty difference, we yet not know.
	Be't as our gods will have't! It only stands
	Our lives upon to use our strongest hands.
	Come, Menas.  	\[r]Exeunt.\]


\Scene{Rome. The house of Lepidus.}


	\(Enter \13 and \17.\)

\17	Good Enobarbus, 'tis a worthy deed,
	And shall become you well, to entreat your captain
	To soft and gentle speech. \\

\13	I shall entreat him
	To answer like himself: if Caesar move him,
	Let Antony look over Caesar's head
	And speak as loud as Mars. By Jupiter,
	Were I the wearer of Antonius' beard,
	I would not shave't to-day.  \\

\17	'Tis not a time
	For private stomaching.        \\

\13	Every time
	Serves for the matter that is then born in't.

\17	But small to greater matters must give way.

\13	Not if the small come first.  \\

\17	Your speech is passion:
	But, pray you, stir no embers up. Here comes
	The noble Antony. \\

	\(Enter \1 and \32.\)

\13	                  And yonder, Caesar.

	\(Enter \3, \19, and \4.\)

\1	If we compose well here, to Parthia:
	Hark, Ventidius. \\

\3	                  I do not know,
	Mecaenas; ask Agrippa. \\

\17	Noble friends,
	That which combined us was most great, and let not
	A leaner action rend us. What's amiss,
	May it be gently heard: when we debate
	Our trivial difference loud, we do commit
	Murder in healing wounds: then, noble partners,
	The rather, for I earnestly beseech,
	Touch you the sourest points with sweetest terms,
	Nor curstness grow to the matter. \\

\1	'Tis spoken well.
	Were we before our armies, and to fight.
	I should do thus. \[r]Flourish.\]

\3	Welcome to Rome. \\

\1	                  Thank you.  \\

\3	Sit. \\

\1	Sit, sir. \\

\3	Nay, then.

\1	I learn, you take things ill which are not so,
	Or being, concern you not.\\

\3	I must be laugh'd at,
	If, or for nothing or a little, I
	Should say myself offended, and with you
	Chiefly i' the world; more laugh'd at, that I should
	Once name you derogately, when to sound your name
	It not concern'd me. \\

\1	My being in Egypt, Caesar,
	What was't to you?

\3	No more than my residing here at Rome
	Might be to you in Egypt: yet, if you there
	Did practise on my state, your being in Egypt
	Might be my question. \\

\1	How intend you, practised?

\3	You may be pleased to catch at mine intent
	By what did here befal me. Your wife and brother
	Made wars upon me; and their contestation
	Was theme for you, you were the word of war.

\1	You do mistake your business; my brother never
	Did urge me in his act: I did inquire it;
	And have my learning from some true reports,
	That drew their swords with you. Did he not rather
	Discredit my authority with yours;
	And make the wars alike against my stomach,
	Having alike your cause? Of this my letters
	Before did satisfy you. If you'll patch a quarrel,
	As matter whole you have not to make it with,
	It must not be with this. \\

\3	You praise yourself
	By laying defects of judgment to me; but
	You patch'd up your excuses.\\

\1	Not so, not so;
	I know you could not lack, I am certain on't,
	Very necessity of this thought, that I,
	Your partner in the cause 'gainst which he fought,
	Could not with graceful eyes attend those wars
	Which fronted mine own peace. As for my wife,
	I would you had her spirit in such another:
	The third o' the world is yours; which with a snaffle
	You may pace easy, but not such a wife.

\13	Would we had all such wives, that the men might
        go to wars with the women!

\1	So much uncurbable, her garboils, Caesar
	Made out of her impatience, which not wanted
	Shrewdness of policy too, I grieving grant
	Did you too much disquiet: for that you must
	But say, I could not help it. \\

\3	I wrote to you
	When rioting in Alexandria; you
	Did pocket up my letters, and with taunts
	Did gibe my missive out of audience. \\

\1	Sir,
	He fell upon me ere admitted: then
	Three kings I had newly feasted, and did want
	Of what I was i' the morning: but next day
	I told him of myself; which was as much
	As to have ask'd him pardon. Let this fellow
	Be nothing of our strife; if we contend,
	Out of our question wipe him. \\

\3	You have broken
	The article of your oath; which you shall never
	Have tongue to charge me with. \\

\17	Soft, Caesar! 
\1	No, Lepidus, let him speak:
	The honour is sacred which he talks on now,
	Supposing that I lack'd it. But, on, Caesar;
	The article of my oath.

\3	To lend me arms and aid when I required them;
	The which you both denied. \\

\1	Neglected, rather;
	And then when poison'd hours had bound me up
	From mine own knowledge. As nearly as I may,
	I'll play the penitent to you: but mine honesty
	Shall not make poor my greatness, nor my power
	Work without it. Truth is, that Fulvia,
	To have me out of Egypt, made wars here;
	For which myself, the ignorant motive, do
	So far ask pardon as befits mine honour
	To stoop in such a case. \\

\17	'Tis noble spoken.

\19	If it might please you, to enforce no further
	The griefs between ye: to forget them quite
	Were to remember that the present need
	Speaks to atone you.\\

\17	Worthily spoken, Mecaenas.

\13	Or, if you borrow one another's love for the
	instant, you may, when you hear no more words of
	Pompey, return it again: you shall have time to
	wrangle in when you have nothing else to do.

\1	Thou art a soldier only: speak no more.

\13	That truth should be silent I had almost forgot.

\1	You wrong this presence; therefore speak no more.

\13	Go to, then; your considerate stone.

\3	I do not much dislike the matter, but
	The manner of his speech; for't cannot be
	We shall remain in friendship, our conditions
	So differing in their acts. Yet if I knew  
	What hoop should hold us stanch, from edge to edge
	O' the world I would pursue it. \\

\4	Give me leave, Caesar,---  

\3	Speak, Agrippa.

\4	Thou hast a sister by the mother's side,
	Admired Octavia: great Mark Antony
	Is now a widower. \\

\3	                  Say not so, Agrippa:
	If Cleopatra heard you, your reproof
	Were well deserved of rashness.

\1	I am not married, Caesar: let me hear
	Agrippa further speak.

\4	To hold you in perpetual amity,
	To make you brothers, and to knit your hearts
	With an unslipping knot, take Antony
	Octavia to his wife; whose beauty claims
	No worse a husband than the best of men;
	Whose virtue and whose general graces speak
	That which none else can utter. By this marriage,
	All little jealousies, which now seem great,
	And all great fears, which now import their dangers,
	Would then be nothing: truths would be tales,
	Where now half tales be truths: her love to both
	Would, each to other and all loves to both,
	Draw after her. Pardon what I have spoke;
	For 'tis a studied, not a present thought,
	By duty ruminated.  \\

\1	                  Will Caesar speak?

\3	Not till he hears how Antony is touch'd
	With what is spoke already. \\

\1	What power is in Agrippa,
	If I would say, `Agrippa, be it so,'
	To make this good? \\

\3	                  The power of Caesar, and
	His power unto Octavia. \\

\1	May I never
	To this good purpose, that so fairly shows,
	Dream of impediment! Let me have thy hand:
	Further this act of grace: and from this hour
	The heart of brothers govern in our loves
	And sway our great designs! \\

\3	There is my hand.
	A sister I bequeath you, whom no brother
	Did ever love so dearly: let her live
	To join our kingdoms and our hearts; and never
	Fly off our loves again! \\

\17	Happily, amen!

\1	I did not think to draw my sword 'gainst Pompey;
	For he hath laid strange courtesies and great
	Of late upon me: I must thank him only,
	Lest my remembrance suffer ill report;
	At heel of that, defy him. \\

\17	Time calls upon's:
	Of us must Pompey presently be sought,
	Or else he seeks out us. \\

\1	Where lies he?

\3	About the mount Misenum. \\

\1	What is his strength?

\3      By land, great and increasing: but by sea
	He is an absolute master. \\

\1	So is the fame.
	Would we had spoke together! Haste we for it:
	Yet, ere we put ourselves in arms, dispatch we
	The business we have talk'd of. \\

\3	With most gladness:
	And do invite you to my sister's view,
	Whither straight I'll lead you. \\

\1	Let us, Lepidus,
	Not lack your company. \\

\17	Noble Antony,
	Not sickness should detain me.

	\(Flourish. Exeunt \3, \1, and \17.\)

\begin{PROSE}

\19	Welcome from Egypt, sir.

\13	Half the heart of Caesar, worthy Mecaenas! My
	honourable friend, Agrippa!

\4	Good Enobarbus!

\19	We have cause to be glad that matters are so well
	digested. You stayed well by 't in Egypt.

\13	Ay, sir; we did sleep day out of countenance, and
	made the night light with drinking.

\19	Eight wild-boars roasted whole at a breakfast, and
	but twelve persons there; is this true?

\13	This was but as a fly by an eagle: we had much more
	monstrous matter of feast, which worthily deserved noting.

\19 She's a most triumphant lady, if report be square to
	her.

\13	When she first met Mark Antony, she pursed up
	his heart, upon the river of Cydnus.

\4	There she appeared indeed; or my reporter devised
	well for her.

\end{PROSE}

\13	I will tell you.
	The barge she sat in, like a burnish'd throne,
	Burn'd on the water: the poop was beaten gold;
	Purple the sails, and so perfumed that
	The winds were love-sick with them; the oars were silver,
	Which to the tune of flutes kept stroke, and made
	The water which they beat to follow faster,
	As amorous of their strokes. For her own person,
	It beggar'd all description: she did lie
	In her pavilion---cloth-of-gold of tissue---
	O'er-picturing that Venus where we see
	The fancy outwork nature: on each side her
	Stood pretty dimpled boys, like smiling Cupids,
	With divers-colour'd fans, whose wind did seem
	To glow the delicate cheeks which they did cool,
	And what they undid did. \\

\4	O, rare for Antony!

\13	Her gentlewomen, like the Nereides,
	So many mermaids, tended her i' the eyes,
	And made their bends adornings: at the helm
	A seeming mermaid steers: the silken tackle
	Swell with the touches of those flower-soft hands,
	That yarely frame the office. From the barge
	A strange invisible perfume hits the sense
	Of the adjacent wharfs. The city cast
	Her people out upon her; and Antony,
	Enthroned i' the market-place, did sit alone,
	Whistling to the air; which, but for vacancy,
	Had gone to gaze on Cleopatra too,
	And made a gap in nature.  \\

\4	Rare Egyptian!

\13	Upon her landing, Antony sent to her,
	Invited her to supper: she replied,
	It should be better he became her guest;
	Which she entreated: our courteous Antony,
	Whom ne'er the word of `No' woman heard speak,
	Being barber'd ten times o'er, goes to the feast,
	And for his ordinary pays his heart
	For what his eyes eat only.  \\

\4	Royal wench!
	She made great Caesar lay his sword to bed:
	He plough'd her, and she cropp'd. \\

\13	I saw her once
	Hop forty paces through the public street;
	And having lost her breath, she spoke, and panted,
	That she did make defect perfection,
	And, breathless, power breathe forth.

\19	Now Antony must leave her utterly.  

\13	Never; he will not:
	Age cannot wither her, nor custom stale
	Her infinite variety: other women cloy
	The appetites they feed: but she makes hungry
	Where most she satisfies; for vilest things
	Become themselves in her: that the holy priests
	Bless her when she is riggish.

\19	If beauty, wisdom, modesty, can settle
	The heart of Antony, Octavia is
	A blessed lottery to him. \\

\4	Let us go.
	Good Enobarbus, make yourself my guest
	Whilst you abide here.  \\[.]

\13	Humbly, sir, I thank you.
     \[r]Exeunt.\]


\Scene{The same. Caesar's house.}


	\(Enter \1, \3, \22 between them,\\ and Attendants.\)

\1	The world and my great office will sometimes
	Divide me from your bosom. \\

\22	All which time
	Before the gods my knee shall bow my prayers
	To them for you. \\

\1	                  Good night, sir. My Octavia,
	Read not my blemishes in the world's report:
	I have not kept my square; but that to come
	Shall all be done by the rule. Good night, dear lady.
	Good night, sir. 

\3	Good night. \[r]Exeunt \3 and \22.\]

	\(Enter Soothsayer.\)

\1	Now, sirrah; you do wish yourself in Egypt?

\71	Would I had never come from thence, nor you Thither!

\1	If you can, your reason?  \\

\71	I see it in
	My motion, have it not in my tongue: but yet
	Hie you to Egypt again. \\

\1	Say to me,
	Whose fortunes shall rise higher, Caesar's or mine?

\71	Caesar's.
	Therefore, O Antony, stay not by his side:
	Thy demon, that's thy spirit which keeps thee, is
	Noble, courageous high, unmatchable,
	Where Caesar's is not; but, near him, thy angel
	Becomes a fear, as being o'erpower'd: therefore
	Make space enough between you. \\

\1	Speak this no more.

\71	To none but thee; no more, but when to thee.
	If thou dost play with him at any game,
	Thou art sure to lose; and, of that natural luck,
	He beats thee 'gainst the odds: thy lustre thickens,
	When he shines by: I say again, thy spirit
	Is all afraid to govern thee near him;
	But, he away, 'tis noble. \\

\1	Get thee gone:
	Say to Ventidius I would speak with him:

	\(Exit Soothsayer.\)

	He shall to Parthia. Be it art or hap,
	He hath spoken true: the very dice obey him;
	And in our sports my better cunning faints
	Under his chance: if we draw lots, he speeds;
	His cocks do win the battle still of mine,
	When it is all to nought; and his quails ever
	Beat mine, inhoop'd, at odds. I will to Egypt:
	And though I make this marriage for my peace,
	I' the east my pleasure lies. \\

	\(Enter \32.\)

		        O, come, Ventidius,
	You must to Parthia: your commission's ready;
	Follow me, and receive't.  \[r]Exeunt.\]



\Scene{The same. A street.}



	\(Enter \17, \19, and \4.\)

\17	Trouble yourselves no further: pray you, hasten
	Your generals after. \\

\4	Sir, Mark Antony
	Will e'en but kiss Octavia, and we'll follow.

\17	Till I shall see you in your soldier's dress,
	Which will become you both, farewell. \\

\19	We shall,
	As I conceive the journey, be at the Mount
	Before you, Lepidus. \\

\17	Your way is shorter;
	My purposes do draw me much about:
	You'll win two days upon me. \\

\19 \4	Sir, good success!

\17	Farewell. 	\[r]Exeunt.\]




\Scene{Alexandria. Cleopatra's palace.}

	\(Enter \2, \7, \16, and \5.\)

\2	Give me some music; music, moody food
	Of us that trade in love. \\

\99	The music, ho!

	\(Enter \18.\)

\2	Let it alone; let's to billiards: come, Charmian.

\7	My arm is sore; best play with Mardian.

\2	As well a woman with an eunuch play'd
	As with a woman. Come, you'll play with me, sir?

\18	As well as I can, madam.

\2	And when good will is show'd, though't come 	too short,
	The actor may plead pardon. I'll none now:
	Give me mine angle; we'll to the river: there,
	My music playing far off, I will betray
	Tawny-finn'd fishes; my bended hook shall pierce
	Their slimy jaws; and, as I draw them up,
	I'll think them every one an Antony,
	And say `Ah, ha! you're caught.' \\

\7	'Twas merry when
	You wager'd on your angling; when your diver
	Did hang a salt-fish on his hook, which he
	With fervency drew up. \\

\2	That time,---O times!---
	I laugh'd him out of patience; and that night
	I laugh'd him into patience; and next morn,
	Ere the ninth hour, I drunk him to his bed;
	Then put my tires and mantles on him, whilst
	I wore his sword Philippan. \\


	\(Enter a \60.\)

		      O, from Italy
	Ram thou thy fruitful tidings in mine ears,
	That long time have been barren. \\

\60	Madam, madam,---

\2	Antonius dead!---If thou say so, villain,
	Thou kill'st thy mistress: but well and free,
	If thou so yield him, there is gold, and here
	My bluest veins to kiss; a hand that kings
	Have lipp'd, and trembled kissing.

\60	First, madam, he is well. \\

\2	Why, there's more gold.
	But, sirrah, mark, we use
	To say the dead are well: bring it to that,
	The gold I give thee will I melt and pour
	Down thy ill-uttering throat.  \\

\60	Good madam, hear me.

\2	Well, go to, I will;
	But there's no goodness in thy face: if Antony
	Be free and healthful,---so tart a favour
	To trumpet such good tidings! If not well,
	Thou shouldst come like a Fury crown'd with snakes,
	Not like a formal man. \\

\60	Will't please you hear me?

\2	I have a mind to strike thee ere thou speak'st:
	Yet if thou say Antony lives, is well,
	Or friends with Caesar, or not captive to him,
	I'll set thee in a shower of gold, and hail
	Rich pearls upon thee. \\

\60	Madam, he's well. \\

\2	Well said. 

\60	And friends with Caesar. \\

\2	Thou'rt an honest man.

\60	Caesar and he are greater friends than ever.

\2	Make thee a fortune from me. \\

\60	But yet, madam,---

\2	I do not like `But yet,' it does allay
	The good precedence; fie upon `But yet'!
	`But yet' is as a gaoler to bring forth
	Some monstrous malefactor. Prithee, friend,
	Pour out the pack of matter to mine ear,
	The good and bad together: he's friends with Caesar:
	In state of health thou say'st; and thou say'st free.

\60	Free, madam! no; I made no such report:
	He's bound unto Octavia. \\

\2	For what good turn?

\60	For the best turn i' the bed. \\

\2	I am pale, Charmian.

\60	Madam, he's married to Octavia.

\2	The most infectious pestilence upon thee!
	\[r]Strikes him down.\]

\60	Good madam, patience. \\

\2	What say you? Hence,
	\[r]Strikes him again.\]

	Horrible villain! or I'll spurn thine eyes
	Like balls before me; I'll unhair thy head:
	\[r]She hales him up and down.\]

	Thou shalt be whipp'd with wire, and stew'd in brine,
	Smarting in lingering pickle. \\

\60	Gracious madam,
	I that do bring the news made not the match.

\2	Say 'tis not so, a province I will give thee,
	And make thy fortunes proud: the blow thou hadst
	Shall make thy peace for moving me to rage;
	And I will boot thee with what gift beside
	Thy modesty can beg. \\

\60	He's married, madam.

\2	Rogue, thou hast lived too long. \\ 	\[r]Draws a knife.\]

\60	Nay, then I'll run.
	What mean you, madam? I have made no fault.	\[r]Exit.\]

\7	Good madam, keep yourself within yourself:
	The man is innocent.

\2	Some innocents 'scape not the thunderbolt.
	Melt Egypt into Nile! and kindly creatures
	Turn all to serpents! Call the slave again:
	Though I am mad, I will not bite him: call.

\7	He is afeard to come. \\

\2	I will not hurt him.
	These hands do lack nobility, that they strike
	A meaner than myself; since I myself
	Have given myself the cause. \\

	\(Enter \60 again.\)

		       Come hither, sir.
	Though it be honest, it is never good
	To bring bad news: give to a gracious message.
	An host of tongues; but let ill tidings tell
	Themselves when they be felt.  \\

\60	I have done my duty.

\2	Is he married?
	I cannot hate thee worser than I do,
	If thou again say `Yes.' \\

\60	He's married, madam.

\2	The gods confound thee! dost thou hold there still?

\60	Should I lie, madam? \\

\2	O, I would thou didst,
	So half my Egypt were submerged and made
	A cistern for scaled snakes! Go, get thee hence:
	Hadst thou Narcissus in thy face, to me
	Thou wouldst appear most ugly. He is married?

\60	I crave your highness' pardon. \\

\2	He is married?

\60	Take no offence that I would not offend you:
	To punish me for what you make me do.
	Seems much unequal: he's married to Octavia.

\2	O, that his fault should make a knave of thee,
	That art not what thou'rt sure of! Get thee hence:
	The merchandise which thou hast brought from Rome
	Are all too dear for me: lie they upon thy hand,
	And be undone by 'em! 	\[r]Exit \60\]

\7	Good your highness, patience.

\2	In praising Antony, I have dispraised Caesar.

\7	Many times, madam. \\

\2	                  I am paid for't now.
	Lead me from hence:
	I faint: O Iras, Charmian! 'tis no matter.
	Go to the fellow, good Alexas; bid him
	Report the feature of Octavia, her years,
	Her inclination, let him not leave out
	The colour of her hair: bring me word quickly. \[r]Exit \5\]
	Let him for ever go:---let him not---Charmian,
	Though he be painted one way like a Gorgon,
	The other way's a Mars. Bid you Alexas
	\[To \18\] Bring me word how tall she is. Pity me, Charmian,
	But do not speak to me. Lead me to my chamber. 	\[r]Exeunt.\]


\Scene{Near Misenum.}

	\(Flourish. Enter \24 and \20 at one door,
	with drum and trumpet: at another, \3, \1, \17, \13, \19,
	with Soldiers marching.\)

\24	Your hostages I have, so have you mine;
	And we shall talk before we fight. \\

\3	Most meet
	That first we come to words; and therefore have we
	Our written purposes before us sent;
	Which, if thou hast consider'd, let us know
	If 'twill tie up thy discontented sword,
	And carry back to Sicily much tall youth
	That else must perish here. \\

\24	To you all three,
	The senators alone of this great world,
	Chief factors for the gods, I do not know
	Wherefore my father should revengers want,
	Having a son and friends; since Julius Caesar,
	Who at Philippi the good Brutus ghosted,
	There saw you labouring for him. What was't
	That moved pale Cassius to conspire; and what
	Made the all-honour'd, honest Roman, Brutus,
	With the arm'd rest, courtiers and beauteous freedom,
	To drench the Capitol; but that they would
	Have one man but a man? And that is it
	Hath made me rig my navy; at whose burthen
	The anger'd ocean foams; with which I meant
	To scourge the ingratitude that despiteful Rome
	Cast on my noble father. \\

\3	Take your time.

\1	Thou canst not fear us, Pompey, with thy sails;
	We'll speak with thee at sea: at land, thou know'st
	How much we do o'er-count thee. \\

\24	At land, indeed,
	Thou dost o'er-count me of my father's house:
	But, since the cuckoo builds not for himself,
	Remain in't as thou mayst. \\

\17	Be pleased to tell us---
	For this is from the present---how you take
	The offers we have sent you. \\

\3	There's the point.

\1	Which do not be entreated to, but weigh
	What it is worth embraced. \\

\3	And what may follow,
	To try a larger fortune.  \\

\24	You have made me offer
	Of Sicily, Sardinia; and I must
	Rid all the sea of pirates; then, to send
	Measures of wheat to Rome; this 'greed upon
	To part with unhack'd edges, and bear back
	Our targes undinted.  \\

\3 \1 \17 That's our offer. 

\24	Know, then,
	I came before you here a man prepared
	To take this offer: but Mark Antony
	Put me to some impatience: though I lose
	The praise of it by telling, you must know,
	When Caesar and your brother were at blows,
	Your mother came to Sicily and did find
	Her welcome friendly. \\

\1	I have heard it, Pompey;
	And am well studied for a liberal thanks
	Which I do owe you. \\

\24	Let me have your hand:
	I did not think, sir, to have met you here.

\1	The beds i' the east are soft; and thanks to you,
	That call'd me timelier than my purpose hither;
	For I have gain'd by 't. \\

\3	Since I saw you last,
	There is a change upon you. \\

\24	Well, I know not
	What counts harsh fortune casts upon my face;
	But in my bosom shall she never come,
	To make my heart her vassal. \\

\17	Well met here.

\24	I hope so, Lepidus. Thus we are agreed:
	I crave our composition may be written,
	And seal'd between us. \\

\3	That's the next to do.

\24	We'll feast each other ere we part; and let's
	Draw lots who shall begin. \\

\1	That will I, Pompey.

\24	No, Antony, take the lot: but, first
	Or last, your fine Egyptian cookery
	Shall have the fame. I have heard that Julius Caesar
	Grew fat with feasting there. \\

\1	You have heard much.

\24	I have fair meanings, sir. \\

\1	And fair words to them.

\24	Then so much have I heard:
	And I have heard, Apollodorus carried---

\13	No more of that: he did so.  \\

\24	What, I pray you?

\13	A certain queen to Caesar in a mattress.

\24	I know thee now: how farest thou, soldier? \\

\13	Well;
	And well am like to do; for, I perceive,
	Four feasts are toward. \\

\24	Let me shake thy hand;
	I never hated thee: I have seen thee fight,
	When I have envied thy behavior. \\

\13	Sir,
	I never loved you much; but I ha' praised ye,
	When you have well deserved ten times as much
	As I have said you did. \\

\24	Enjoy thy plainness,
	It nothing ill becomes thee.
	Aboard my galley I invite you all:
	Will you lead, lords? \\

\3 \1 \17  Show us the way, sir. \\

\24	Come. \[r]Exeunt all but \20 and \13\]

\begin{PROSE}

\20	\[Aside.\]  Thy father, Pompey, would ne'er have
	made this treaty.---You and I have known, sir.

\13	At sea, I think.

\20	We have, sir.

\13	You have done well by water.

\20	And you by land.

\13	I will praise any man that will praise me; though it
	cannot be denied what I have done by land.

\20	Nor what I have done by water.

\13	Yes, something you can deny for your own
	safety: you have been a great thief by sea.

\20	And you by land.

\13	There I deny my land service. But give me your
	hand, Menas: if our eyes had authority, here they
	might take two thieves kissing.

\20	All men's faces are true, whatsome'er their hands are.

\13	But there is never a fair woman has a true face.

\20	No slander; they steal hearts.

\13	We came hither to fight with you.

\20	For my part, I am sorry it is turned to a drinking.
	Pompey doth this day laugh away his fortune.

\13	If he do, sure, he cannot weep't back again.

\20	You've said, sir. We looked not for Mark Antony
	here: pray you, is he married to Cleopatra?

\13	Caesar's sister is called Octavia.

\20	True, sir; she was the wife of Caius Marcellus.

\13	But she is now the wife of Marcus Antonius.

\20	Pray ye, sir?

\13	'Tis true.

\20	Then is Caesar and he for ever knit together.

\13	If I were bound to divine of this unity, I would
	not prophesy so.

\20	I think the policy of that purpose made more in the
	marriage than the love of the parties.

\13	I think so too. But you shall find, the band that
	seems to tie their friendship together will be the
	very strangler of their amity: Octavia is of a
	holy, cold, and still conversation.

\20	Who would not have his wife so?

\13	Not he that himself is not so; which is Mark Antony.
	He will to his Egyptian dish again: then shall the
	sighs of Octavia blow the fire up in Caesar; and, as
	I said before, that which is the strength of their
	amity shall prove the immediate author of their
	variance. Antony will use his affection where it is:
	he married but his occasion here.

\20	And thus it may be. Come, sir, will you aboard?
	I have a health for you.

\13	I shall take it, sir: we have used our throats in Egypt.

\20	Come, let's away. \[r]Exeunt.\]
\end{PROSE}


\Scene{On board Pompey's galley, off Misenum.}


	\(Music plays. Enter two or three Servants with a banquet.\)

\begin{PROSE}

\64	Here they'll be, man. Some o' their plants are
	ill-rooted already: the least wind i' the world
	will blow them down.

\65	Lepidus is high-coloured.

\64 They have made him drink alms-drink.

\65	As they pinch one another by the disposition, he
	cries out `No more;' reconciles them to his
	entreaty, and himself to the drink.

\64	But it raises the greater war between him and
	his discretion.

\65	Why, this is to have a name in great men's
	fellowship: I had as lief have a reed that will do
	me no service as a partisan I could not heave.

\64 To be called into a huge sphere, and not to be seen
	to move in't, are the holes where eyes should be,
	which pitifully disaster the cheeks.

\end{PROSE}

	\(A sennet sounded. Enter \3, \1, \17, \24,
          \4, \19, \13, \20, with other captains.\)

\1 \[To \3\]  Thus do they, sir: they take the flow o' the Nile
	By certain scales i' the pyramid; they know,
	By the height, the lowness, or the mean, if dearth
	Or foison follow: the higher Nilus swells,
	The more it promises: as it ebbs, the seedsman
	Upon the slime and ooze scatters his grain,
	And shortly comes to harvest.

\17	You've strange serpents there.

\1	Ay, Lepidus.

\begin{PROSE}

\17	Your serpent of Egypt is bred now of your mud by the
	operation of your sun: so is your crocodile.

\1	They are so.

\24	Sit,---and some wine! A health to Lepidus!

\17	I am not so well as I should be, but I'll ne'er out.

\13	Not till you have slept; I fear me you'll be in till then.

\17	Nay, certainly, I have heard the Ptolemies'
	pyramises are very goodly things; without
	contradiction, I have heard that.

\end{PROSE}

\20	\[Aside to \24\]  Pompey, a word. \\

\24	\[Aside to \20\]   Say in mine ear: 	what is't?

\20	\[Aside to \24\]  Forsake thy seat, I do beseech thee, captain,
	And hear me speak a word. \\


\24	\[Aside to \20\]  Forbear me till anon.
	This wine for Lepidus!

\begin{PROSE}

\17	What manner o' thing is your crocodile?

\1	It is shaped, sir, like itself; and it is as broad
	as it hath breadth: it is just so high as it is,
	and moves with its own organs: it lives by that
	which nourisheth it; and the elements once out of
	it, it transmigrates.

\17	What colour is it of?

\1	Of it own colour too.

\17	'Tis a strange serpent.

\1	'Tis so. And the tears of it are wet.

\3	Will this description satisfy him?

\1	With the health that Pompey gives him, else he is a
	very epicure.

\end{PROSE}

\24	\[Aside to \20\]  Go hang, sir, hang! Tell me of that? away!
	Do as I bid you. Where's this cup I call'd for?

\20	\[Aside to \24\]  If for the sake of merit thou wilt hear me,
	Rise from thy stool. \\

\24	\[Aside to \20\]  I think thou'rt mad. The matter?
	\[r]Rises, and walks aside.\]

\20	I have ever held my cap off to thy fortunes.

\24	Thou hast served me with much faith. What's else to say?
	Be jolly, lords. \\

\1	                  These quick-sands, Lepidus,
	Keep off them, for you sink.

\20	Wilt thou be lord of all the world? \\

\24	What say'st thou?

\20	Wilt thou be lord of the whole world? That's twice.

\24	How should that be? \\

\20	But entertain it,
	And, though thou think me poor, I am the man
	Will give thee all the world. \\

\24	Hast thou drunk well?

\20	Now, Pompey, I have kept me from the cup.
	Thou art, if thou darest be, the earthly Jove:
	Whate'er the ocean pales, or sky inclips,
	Is thine, if thou wilt ha't. \\

\24	Show me which way.

\20	These three world-sharers, these competitors,
	Are in thy vessel: let me cut the cable;
	And, when we are put off, fall to their throats:
	All there is thine. \\

\24	Ah, this thou shouldst have done,
	And not have spoke on't! In me 'tis villany;
	In thee't had been good service. Thou must know,
	'Tis not my profit that does lead mine honour;
	Mine honour, it. Repent that e'er thy tongue
	Hath so betray'd thine act: being done unknown,
	I should have found it afterwards well done;
	But must condemn it now. Desist, and drink.

\20	\[Aside.\]  For this,
	I'll never follow thy pall'd fortunes more.
	Who seeks, and will not take when once 'tis offer'd,
	Shall never find it more. \\

\24	This health to Lepidus!

\1	Bear him ashore. I'll pledge it for him, Pompey.

\13	Here's to thee, Menas! \\

\20	Enobarbus, welcome!

\24	Fill till the cup be hid.

\13	There's a strong fellow, Menas.
	\[r]Pointing to the Attendant who carries off Lepidus.\]

\20	Why?

\13	A' bears the third part of the world, man; see'st not?

\20	The third part, then, is drunk: would it were all,
	That it might go on wheels!

\13	Drink thou; increase the reels.

\20	Come.

\24	This is not yet an Alexandrian feast.

\1	It ripens towards it. Strike the vessels, ho?
	Here is to Caesar!  \\

\3	                  I could well forbear't.
	It's monstrous labour, when I wash my brain,
	And it grows fouler. \\

\1	Be a child o' the time.

\3	Possess it, I'll make answer:
	But I had rather fast from all four days
	Than drink so much in one. \\

\13	\[To \1\] Ha, my brave emperor!
	Shall we dance now the Egyptian Bacchanals,
	And celebrate our drink? \\

\24	Let's ha't, good soldier.

\1	Come, let's all take hands,
	Till that the conquering wine hath steep'd our sense
	In soft and delicate Lethe. \\

\13	All take hands.
	Make battery to our ears with the loud music:
	The while I'll place you: then the boy shall sing;
	The holding every man shall bear as loud
	As his strong sides can volley. 
	\[r]Music plays. Enobarbus places them hand in hand.\]
   \spatium {1ex} \numerus{}
  	\hskip 8em \textsc{the song} 
   {\Locus \textus {+\widthof{As his}}
    \Forma \strophae {000011}
   \numerus{-1}
	Come, thou monarch of the vine, 
	Plumpy Bacchus with pink eyne!
	In thy fats our cares be drown'd,
	With thy grapes our hairs be crown'd:
	Cup us, till the world go round,
	Cup us, till the world go round!
   }
   \spatium {1ex} \numerus{+0}
\3	What would you more? Pompey, good night. Good brother,
	Let me request you off: our graver business
	Frowns at this levity. Gentle lords, let's part;
	You see we have burnt our cheeks: strong Enobarb
	Is weaker than the wine; and mine own tongue
	Splits what it speaks: the wild disguise hath almost
	Antick'd us all. What needs more words? Good night.
	Good Antony, your hand. \\

\24	I'll try you on the shore.

\1	And shall, sir; give's your hand. \\

\24	O Antony,
	You have my father's house. But, what? we are friends.
	Come, down into the boat. \\

\13	Take heed you fall not.
	\[r]Exeunt all but \13 and \20\]

	Menas, I'll not on shore. \\

\20	No, to my cabin.
	These drums! these trumpets, flutes! what!
	Let Neptune hear we bid a loud farewell
	To these great fellows: sound and be hang'd, sound out!
	\[r]Sound a flourish, with drums.\]

\13	Ho! says a' There's my cap.

\20	Ho! Noble captain, come. \[r]Exeunt.\]

\Act



\Scene{A plain in Syria.}


	\(Enter \32 as it were in triumph, with \28,
	and other Romans, Officers, and Soldiers; the dead
	body of \persona{Pacorus} borne before him.\)

\32	Now, darting Parthia, art thou struck; and now
	Pleased fortune does of Marcus Crassus' death
	Make me revenger. Bear the king's son's body
	Before our army. Thy Pacorus, Orodes,
	Pays this for Marcus Crassus. \\

\28	Noble Ventidius,
	Whilst yet with Parthian blood thy sword is warm,
	The fugitive Parthians follow; spur through Media,
	Mesopotamia, and the shelters whither
	The routed fly: so thy grand captain Antony
	Shall set thee on triumphant chariots and
	Put garlands on thy head. \\

\32	O Silius, Silius,
	I have done enough; a lower place, note well,
	May make too great an act: for learn this, Silius;
	Better to leave undone, than by our deed
	Acquire too high a fame when him we serve's away.
	Caesar and Antony have ever won
	More in their officer than person: Sossius,
	One of my place in Syria, his lieutenant,
	For quick accumulation of renown,
	Which he achieved by the minute, lost his favour.
	Who does i' the wars more than his captain can
	Becomes his captain's captain: and ambition,
	The soldier's virtue, rather makes choice of loss,
	Than gain which darkens him.
	I could do more to do Antonius good,
	But 'twould offend him; and in his offence
	Should my performance perish. \\

\28	Thou hast, Ventidius, that
	Without the which a soldier, and his sword,
	Grants scarce distinction. Thou wilt write to Antony!

\32	I'll humbly signify what in his name,
	That magical word of war, we have effected;
	How, with his banners and his well-paid ranks,
	The ne'er-yet-beaten horse of Parthia
	We have jaded out o' the field. \\

\28	Where is he now?

\32	He purposeth to Athens: whither, with what haste
	The weight we must convey with's will permit,
	We shall appear before him. On there; pass along!
	\[r]Exeunt.\]



\Scene{Rome. An ante-chamber in Caesar's house.}


	\(Enter \4 at one door, \13 at another.\)

\4	What, are the brothers parted?

\13	They have dispatch'd with Pompey, he is gone;
	The other three are sealing. Octavia weeps
	To part from Rome; Caesar is sad; and Lepidus,
	Since Pompey's feast, as Menas says, is troubled
	With the green sickness. \\

\4	'Tis a noble Lepidus.

\13	A very fine one: O, how he loves Caesar!

\4	Nay, but how dearly he adores Mark Antony!

\13	Caesar? Why, he's the Jupiter of men.

\4	What's Antony? The god of Jupiter.

\13	Spake you of Caesar? How! the non-pareil!

\4	O Antony! O thou Arabian bird!

\13	Would you praise Caesar, say `Caesar:' go no further.

\4	Indeed, he plied them both with excellent praises.

\13	But he loves Caesar best; yet he loves Antony:
	Ho! hearts, tongues, figures, scribes, bards, poets, cannot
	Think, speak, cast, write, sing, number, ho!
	His love to Antony. But as for Caesar,
	Kneel down, kneel down, and wonder. \\

\4	Both he loves.

\13	They are his shards, and he their beetle. So;
        \[r]Trumpets within.\]
	This is to horse. Adieu, noble Agrippa.

\4	Good fortune, worthy soldier; and farewell.

	\(Enter \3, \1, \17, and \22\)

\1	No further, sir.

\3	You take from me a great part of myself;
	Use me well in 't. Sister, prove such a wife
	As my thoughts make thee, and as my farthest band
	Shall pass on thy approof. Most noble Antony,
	Let not the piece of virtue, which is set
	Betwixt us as the cement of our love,
	To keep it builded, be the ram to batter
	The fortress of it; for better might we
	Have loved without this mean, if on both parts
	This be not cherish'd. \\

\1	Make me not offended
	In your distrust. \\

\3	                  I have said. \\

\1	You shall not find,
	Though you be therein curious, the least cause
	For what you seem to fear: so, the gods keep you,
	And make the hearts of Romans serve your ends!
	We will here part.

\3	Farewell, my dearest sister, fare thee well:
	The elements be kind to thee, and make
	Thy spirits all of comfort! fare thee well.

\22	My noble brother!

\1	The April 's in her eyes: it is love's spring,
	And these the showers to bring it on. Be cheerful.

\22	Sir, look well to my husband's house; and--- \\

\3	What,
        Octavia? \\

\22	       I'll tell you in your ear.

\1	Her tongue will not obey her heart, nor can
	Her heart inform her tongue,---the swan's 	down-feather,
	That stands upon the swell at full of tide,
	And neither way inclines.

\13	\[Aside to \4\]  Will Caesar weep? \\

\4	\[Aside to \13\]  He has a cloud in 's face.  

\13	\[Aside to \4\]  He were the worse for that, were he a horse;
	So is he, being a man. \\

\4 \[Aside to \13\]  Why, Enobarbus,
	When Antony found Julius Caesar dead,
	He cried almost to roaring; and he wept
	When at Philippi he found Brutus slain.

\13	\[Aside to \4\]  That year, indeed, he was troubled with a rheum;
	What willingly he did confound he wail'd,
	Believe't, till I wept too. \\

\3	No, sweet Octavia,
	You shall hear from me still; the time shall not
	Out-go my thinking on you. \\

\1	Come, sir, come;
	I'll wrestle with you in my strength of love:
	Look, here I have you; thus I let you go,
	And give you to the gods. \\

\3	Adieu; be happy!

\17	Let all the number of the stars give light
	To thy fair way! \\

\3	Farewell, farewell! \\  \[r]Kisses \22\]

\1	Farewell!
	\[r]Trumpets sound. Exeunt.\]





\Scene{Alexandria. Cleopatra's palace.}


	\(Enter \2, \7, \16, and \5\)

\2	Where is the fellow? \\

\5	Half afeard to come.

\2	Go to, go to. 	Come hither, sir. \\

	\(Enter the \60 as before.\)


\5	Good majesty,
	Herod of Jewry dare not look upon you
	But when you are well pleased. \\

\2	That Herod's head
	I'll have: but how, when Antony is gone
	Through whom I might command it? Come thou near.

\60	Most gracious majesty!  \\

\2	Didst thou behold
        Octavia? \\

\60	Ay, dread queen. \\

\2	Where? \\

\60	Madam, in Rome;
	I look'd her in the face, and saw her led
	Between her brother and Mark Antony.

\2	Is she as tall as me? \\

\60	She is not, madam.

\2	Didst hear her speak? is she shrill-tongued or low?

\60	Madam, I heard her speak; she is low-voiced.

\2	That's not so good: he cannot like her long.

\7	Like her! O Isis! 'tis impossible.

\2	I think so, Charmian: dull of tongue, and dwarfish!
	What majesty is in her gait? Remember,
	If e'er thou look'dst on majesty. \\

\60	She creeps:
	Her motion and her station are as one;
	She shows a body rather than a life,
	A statue than a breather. \\

\2	Is this certain?

\60	Or I have no observance. \\

\7	Three in Egypt
	Cannot make better note. \\

\2	He's very knowing;
	I do perceive't: there's nothing in her yet:
	The fellow has good judgment. \\

\7	Excellent.

\2	Guess at her years, I prithee. \\

\60	Madam,
	She was a widow,--- \\

\2	                  Widow! Charmian, hark.

\60	And I do think she's thirty.

\2	Bear'st thou her face in mind? is't long or round?

\60	Round even to faultiness.

\2	For the most part, too, they are foolish that are so.
	Her hair, what colour? \\

\60	Brown, madam: and her forehead
	As low as she would wish it. \\

\2	There's gold for thee.
	Thou must not take my former sharpness ill:
	I will employ thee back again; I find thee
	Most fit for business: go make thee ready;
	Our letters are prepared. \\	\[r]Exit \60\]

\7	A proper man.

\2	Indeed, he is so: I repent me much
	That so I harried him. Why, methinks, by him,
	This creature's no such thing. \\

\7	Nothing, madam.

\2	The man hath seen some majesty, and should know.

\7	Hath he seen majesty? Isis else defend,
	And serving you so long!

\2	I have one thing more to ask him yet, good Charmian:
	But 'tis no matter; thou shalt bring him to me
	Where I will write. All may be well enough.

\7	I warrant you, madam. \[r]Exeunt.\]


\Scene{Athens. A room in Antony's house.}


	\(Enter \1 and \22\)

\1	Nay, nay, Octavia, not only that,---
	That were excusable, that, and thousands more
	Of semblable import,---but he hath waged
	New wars 'gainst Pompey; made his will, and read it
	To public ear:
	Spoke scantly of me: when perforce he could not
	But pay me terms of honour, cold and sickly
	He vented them; most narrow measure lent me:
	When the best hint was given him, he not took't,
	Or did it from his teeth. \\

\22	O my good lord,
	Believe not all; or, if you must believe,
	Stomach not all. A more unhappy lady,
	If this division chance, ne'er stood between,
	Praying for both parts:
	The good gods me presently,
	When I shall pray, `O bless my lord and husband!'
	Undo that prayer, by crying out as loud,
	`O, bless my brother!' Husband win, win brother,
	Prays, and destroys the prayer; no midway
	'Twixt these extremes at all. \\

\1	Gentle Octavia,
	Let your best love draw to that point, which seeks
	Best to preserve it: if I lose mine honour,
	I lose myself: better I were not yours
	Than yours so branchless. But, as you requested,
	Yourself shall go between 's: the mean time, lady,
	I'll raise the preparation of a war
	Shall stain your brother: make your soonest haste;
	So your desires are yours. \\

\22	Thanks to my lord.
	The Jove of power make me most weak, most weak,
	Your reconciler! Wars 'twixt you twain would be
	As if the world should cleave, and that slain men
	Should solder up the rift.

\1	When it appears to you where this begins,
	Turn your displeasure that way: for our faults
	Can never be so equal, that your love
	Can equally move with them. Provide your going;
	Choose your own company, and command what cost
	Your heart has mind to. \[r]Exeunt.\]


\Scene{The same. Another room.}


	\(Enter \13 and \14, meeting.\)

\begin{PROSE}

\13	How now, friend Eros!

\14	There's strange news come, sir.

\13	What, man?

\14	Caesar and Lepidus have made wars upon Pompey.

\13	This is old: what is the success?

\14	Caesar, having made use of him in the wars 'gainst
	Pompey, presently denied him rivality; would not let
	him partake in the glory of the action: and not
	resting here, accuses him of letters he had formerly
	wrote to Pompey; upon his own appeal, seizes him: so
	the poor third is up, till death enlarge his confine.

\13	Then, world, thou hast a pair of chaps, no more;
	And throw between them all the food thou hast,
	They'll grind the one the other. Where's Antony?

\end{PROSE}

\14	He's walking in the garden---thus; and spurns
	The rush that lies before him; cries, `Fool Lepidus!'
	And threats the throat of that his officer
	That murder'd Pompey. \\

\13	Our great navy's rigg'd.

\14	For Italy and Caesar. More, Domitius;
	My lord desires you presently: my news
	I might have told hereafter. \\

\13	'Twill be naught:
	But let it be. Bring me to Antony.

\14	Come, sir. 	\[r]Exeunt.\]




\Scene{Rome. Caesar's house.}


	\(Enter \3, \4, and \19\)

\3	Contemning Rome, he has done all this, and more,
	In Alexandria: here's the manner of 't:
	I' the market-place, on a tribunal silver'd,
	Cleopatra and himself in chairs of gold
	Were publicly enthroned: at the feet sat
	Caesarion, whom they call my father's son,
	And all the unlawful issue that their lust
	Since then hath made between them. Unto her
	He gave the stablishment of Egypt; made her
	Of lower Syria, Cyprus, Lydia,
	Absolute queen. \\

\19	                  This in the public eye?

\3	I' the common show-place, where they exercise.
	His sons he there proclaim'd the kings of kings:
	Great Media, Parthia, and Armenia.
	He gave to Alexander; to Ptolemy he assign'd
	Syria, Cilicia, and Phoenicia: she
	In the habiliments of the goddess Isis
	That day appear'd; and oft before gave audience,
	As 'tis reported, so. \\

\19	Let Rome be thus Inform'd.

\4	Who, queasy with his insolence
	Already, will their good thoughts call from him.

\3	The people know it; and have now received
	His accusations. \\

\4	                  Who does he accuse?

\3	Caesar: and that, having in Sicily
	Sextus Pompeius spoil'd, we had not rated him
	His part o' the isle: then does he say, he lent me
	Some shipping unrestored: lastly, he frets
	That Lepidus of the triumvirate
	Should be deposed; and, being, that we detain
	All his revenue. \\

\4	                  Sir, this should be answer'd.

\3	'Tis done already, and the messenger gone.
	I have told him, Lepidus was grown too cruel;
	That he his high authority abused,
	And did deserve his change: for what I have conquer'd,
	I grant him part; but then, in his Armenia,
	And other of his conquer'd kingdoms, I
	Demand the like. \\

\19	                  He'll never yield to that.

\3	Nor must not then be yielded to in this.


	\(Enter \22 with her train.\)

\22	Hail, Caesar, and my lord! hail, most dear Caesar!

\3	That ever I should call thee castaway!

\22	You have not call'd me so, nor have you cause.

\3	Why have you stol'n upon us thus! You come not
	Like Caesar's sister: the wife of Antony
	Should have an army for an usher, and
	The neighs of horse to tell of her approach
	Long ere she did appear; the trees by the way
	Should have borne men; and expectation fainted,
	Longing for what it had not; nay, the dust
	Should have ascended to the roof of heaven,
	Raised by your populous troops: but you are come
	A market-maid to Rome; and have prevented
	The ostentation of our love, which, left unshown,
	Is often left unloved; we should have met you
	By sea and land; supplying every stage
	With an augmented greeting. \\

\22	Good my lord,
	To come thus was I not constrain'd, but did
	On my free will. My lord, Mark Antony,
	Hearing that you prepared for war, acquainted
	My grieved ear withal; whereon, I begg'd
	His pardon for return. \\

\3	Which soon he granted,
	Being an obstruct 'tween his lust and him.

\22	Do not say so, my lord.  \\

\3	I have eyes upon him,
	And his affairs come to me on the wind.
	Where is he now? \\

\22	                  My lord, in Athens.

\3	No, my most wronged sister; Cleopatra
	Hath nodded him to her. He hath given his empire
	Up to a whore; who now are levying
	The kings o' the earth for war; he hath assembled
	Bocchus, the king of Libya; Archelaus,
	Of Cappadocia; Philadelphos, king
	Of Paphlagonia; the Thracian king, Adallas;
	King Malchus of Arabia; King of Pont;
	Herod of Jewry; Mithridates, king
	Of Comagene; Polemon and Amyntas,
	The kings of Mede and Lycaonia,
	With a more larger list of sceptres. \\

\22	Ay me, most wretched,
	That have my heart parted betwixt two friends
	That do afflict each other! \\

\3	Welcome hither:
	Your letters did withhold our breaking forth;
	Till we perceived, both how you were wrong led,
	And we in negligent danger. Cheer your heart;
	Be you not troubled with the time, which drives
	O'er your content these strong necessities;
	But let determined things to destiny
	Hold unbewail'd their way. Welcome to Rome;
	Nothing more dear to me. You are abused
	Beyond the mark of thought: and the high gods,
	To do you justice, make them ministers
	Of us and those that love you. Best of comfort;
	And ever welcome to us. \\

\4	Welcome, lady.

\19	Welcome, dear madam.
	Each heart in Rome does love and pity you:
	Only the adulterous Antony, most large
	In his abominations, turns you off;
	And gives his potent regiment to a trull,
	That noises it against us. \\

\22	Is it so, sir?

\3	Most certain. Sister, welcome: pray you,
	Be ever known to patience: my dear'st sister! \[r]Exeunt.\]






\Scene{Near Actium. Antony's camp.}


	\(Enter \2 and \13\)

\2	I will be even with thee, doubt it not.

\13	But why, why, why?

\2	Thou hast forspoke my being in these wars,
	And say'st it is not fit. \\

\13	Well, is it, is it?

\2	If not denounced against us, why should not we
	Be there in person? \\

\13	\[Aside.\]  Well, I could reply:
	If we should serve with horse and mares together,
	The horse were merely lost; the mares would bear
	A soldier and his horse. \\

\2	What is't you say?

\13	Your presence needs must puzzle Antony;
	Take from his heart, take from his brain, from's time,
	What should not then be spared. He is already
	Traduced for levity; and 'tis said in Rome
	That Photinus an eunuch and your maids
	Manage this war. \\

\2	                  Sink Rome, and their tongues rot
	That speak against us! A charge we bear i' the war,
	And, as the president of my kingdom, will
	Appear there for a man. Speak not against it:
	I will not stay behind. \\


	\(Enter \1 and \6\)

\13	Nay, I have done.
	Here comes the emperor. \\


\1	Is it not strange, Canidius,
	That from Tarentum and Brundusium
	He could so quickly cut the Ionian sea,
	And take in Toryne? You have heard on't, sweet?

\2	Celerity is never more admired
	Than by the negligent. \\

\1	A good rebuke,
	Which might have well becomed the best of men,
	To taunt at slackness. Canidius, we
	Will fight with him by sea. \\

\2	By sea! what else?

\6	Why will my lord do so? \\

\1	For that he dares us to't.

\13	So hath my lord dared him to single fight.

\22	Ay, and to wage this battle at Pharsalia.
	Where Caesar fought with Pompey: but these offers,
	Which serve not for his vantage, be shakes off;
	And so should you. \\

\13	                  Your ships are not well mann'd;
	Your mariners are muleters, reapers, people
	Ingross'd by swift impress; in Caesar's fleet
	Are those that often have 'gainst Pompey fought:
	Their ships are yare; yours, heavy: no disgrace
	Shall fall you for refusing him at sea,
	Being prepared for land. \\

\1	By sea, by sea.

\13	Most worthy sir, you therein throw away
	The absolute soldiership you have by land;
	Distract your army, which doth most consist
	Of war-mark'd footmen; leave unexecuted
	Your own renowned knowledge; quite forego
	The way which promises assurance; and
	Give up yourself merely to chance and hazard,
	From firm security. \\

\1	I'll fight at sea.

\2	I have sixty sails, Caesar none better.

\1	Our overplus of shipping will we burn;
	And, with the rest full-mann'd, from the head of Actium
	Beat the approaching Caesar. But if we fail,
	We then can do't at land. \\


	\(Enter a Messenger.\)

		    Thy business?

\60	The news is true, my lord; he is descried;
	Caesar has taken Toryne.

\1	Can he be there in person? 'tis impossible;
	Strange that power should be. Canidius,
	Our nineteen legions thou shalt hold by land,
	And our twelve thousand horse. We'll to our ship:
	Away, my Thetis! \\


	\(Enter a Soldier.\)

	How now, worthy soldier?

\66	O noble emperor, do not fight by sea;
	Trust not to rotten planks: do you misdoubt
	This sword and these my wounds? Let the Egyptians
	And the Phoenicians go a-ducking; we
	Have used to conquer, standing on the earth,
	And fighting foot to foot. \\

\1	Well, well: away! \[r]Exeunt Antony, Cleopatra and Enobarbus.\]

\66	By Hercules, I think I am i' the right.

\6	Soldier, thou art: but his whole action grows
	Not in the power on't: so our leader's led,
	And we are women's men. \\

\66	You keep by land
	The legions and the horse whole, do you not?

\6	Marcus Octavius, Marcus Justeius,
	Publicola, and Caelius, are for sea:
	But we keep whole by land. This speed of Caesar's
	Carries beyond belief. \\

\66	While he was yet in Rome,
	His power went out in such distractions as
	Beguiled all spies. \\

\6	Who's his lieutenant, hear you?

\66	They say, one Taurus. \\

\6	Well I know the man.

	\(Enter a Messenger.\)

\60	The emperor calls Canidius.

\6	With news the time's with labour, and throes forth,
	Each minute, some. 	\[r]Exeunt.\]




\Scene{A plain near Actium.}


	\(Enter \3, and \29, with his army, marching.\)

\3	Taurus!

\29	My lord?

\3	Strike not by land; keep whole: provoke not battle,
	Till we have done at sea. Do not exceed
	The prescript of this scroll: our fortune lies
	Upon this jump. \[r]Exeunt.\]


\Scene{Another part of the plain.}


	\(Enter \1 and \13\)

\1	Set we our squadrons on yond side o' the hill,
	In eye of Caesar's battle; from which place
	We may the number of the ships behold,
	And so proceed accordingly. \[r]Exeunt.\]



\Scene{Another part of the plain.}


	\(\6 marcheth with his land army one way over
	  the stage; and \29, the lieutenant of \3,
          the other way. After their going in, is
	  heard the noise of a sea-fight.\)

	\(Alarum. Enter \13\)

\13	Naught, naught all, naught! I can behold no longer:
	The Antoniad, the Egyptian admiral,
	With all their sixty, fly and turn the rudder:
	To see't mine eyes are blasted. \\


	\(Enter \26\)

\26	Gods and goddesses,
	All the whole synod of them! \\

\13	What's thy passion!

\26	The greater cantle of the world is lost
	With very ignorance; we have kiss'd away
	Kingdoms and provinces. \\

\13	How appears the fight?

\26	On our side like the token'd pestilence,
	Where death is sure. Yon ribaudred nag of Egypt,---
	Whom leprosy o'ertake!---i' the midst o' the fight,
	When vantage like a pair of twins appear'd,
	Both as the same, or rather ours the elder,
	The breese upon her, like a cow in June,
	Hoists sails and flies.

\13	That I beheld:
	Mine eyes did sicken at the sight, and could not
	Endure a further view. \\

\26	She once being loof'd,
	The noble ruin of her magic, Antony,
	Claps on his sea-wing, and, like a doting mallard,
	Leaving the fight in height, flies after her:
	I never saw an action of such shame;
	Experience, manhood, honour, ne'er before
	Did violate so itself. \\

\13	Alack, alack!

	\(Enter \6\)

\6	Our fortune on the sea is out of breath,
	And sinks most lamentably. Had our general
	Been what he knew himself, it had gone well:
	O, he has given example for our flight,
	Most grossly, by his own!

\13	Ay, are you thereabouts? Why, then, good night
   indeed. \\

\6	Toward Peloponnesus are they fled.

\26	'Tis easy to't; and there I will attend
	What further comes. \\

\6	To Caesar will I render
	My legions and my horse: six kings already
	Show me the way of yielding. \\

\13	I'll yet follow
	The wounded chance of Antony, though my reason
	Sits in the wind against me. 	\[r]Exeunt.\]


\Scene{Alexandria. Cleopatra's palace.}


	\(Enter \1 with Attendants.\)

\1	Hark! the land bids me tread no more upon't;
	It is ashamed to bear me! Friends, come hither:
	I am so lated in the world, that I
	Have lost my way for ever: I have a ship
	Laden with gold; take that, divide it; fly,
	And make your peace with Caesar. \\

\99	Fly! not we.

\1	I have fled myself; and have instructed cowards
	To run and show their shoulders. Friends, be gone;
	I have myself resolved upon a course
	Which has no need of you; be gone:
	My treasure's in the harbour, take it. O,
	I follow'd that I blush to look upon:
	My very hairs do mutiny; for the white
	Reprove the brown for rashness, and they them
	For fear and doting. Friends, be gone: you shall
	Have letters from me to some friends that will
	Sweep your way for you. Pray you, look not sad,
	Nor make replies of loathness: take the hint
	Which my despair proclaims; let that be left
	Which leaves itself: to the sea-side straightway:
	I will possess you of that ship and treasure.
	Leave me, I pray, a little: pray you now:
	Nay, do so; for, indeed, I have lost command,
	Therefore I pray you: I'll see you by and by. \[r]Sits down.\]


	\(Enter \2 led by \7 and \16; \14 following.\)

\14	Nay, gentle madam, to him, comfort him.

\16	Do, most dear queen.

\7	Do! why: what else?

\2	Let me sit down. O Juno!

\1	No, no, no, no, no.

\14	See you here, sir?

\1	O fie, fie, fie!

\7	Madam!

\16	Madam, O good empress!

\14	Sir, sir,---

\1	Yes, my lord, yes; he at Philippi kept
	His sword e'en like a dancer; while I struck
	The lean and wrinkled Cassius; and 'twas I
	That the mad Brutus ended: he alone
	Dealt on lieutenantry, and no practise had
	In the brave squares of war: yet now---No matter.

\2	Ah, stand by.

\14	The queen, my lord, the queen.

\16	Go to him, madam, speak to him:
	He is unqualitied with very shame.

\2	Well then, sustain him: O!

\14	Most noble sir, arise; the queen approaches:
	Her head's declined, and death will seize her, but
	Your comfort makes the rescue.

\1	I have offended reputation,
	A most unnoble swerving. \\

\14	Sir, the queen.

\1	O, whither hast thou led me, Egypt? See,
	How I convey my shame out of thine eyes
	By looking back what I have left behind
	'Stroy'd in dishonour. \\

\2	O my lord, my lord,
	Forgive my fearful sails! I little thought
	You would have follow'd. \\

\1	Egypt, thou knew'st too well
	My heart was to thy rudder tied by the strings,
	And thou shouldst tow me after: o'er my spirit
	Thy full supremacy thou knew'st, and that
	Thy beck might from the bidding of the gods
	Command me. \\

\2	          O, my pardon! \\

\1	Now I must
	To the young man send humble treaties, dodge
	And palter in the shifts of lowness; who
	With half the bulk o' the world play'd as I pleased,
	Making and marring fortunes. You did know
	How much you were my conqueror; and that
	My sword, made weak by my affection, would
	Obey it on all cause. \\

\2	Pardon, pardon!

\1	Fall not a tear, I say; one of them rates
	All that is won and lost: give me a kiss;
	Even this repays me. We sent our schoolmaster;
	Is he come back? Love, I am full of lead.
	Some wine, within there, and our viands! Fortune knows
	We scorn her most when most she offers blows. \[r]Exeunt.\]



\Scene{Egypt. Caesar's camp.}


	\(Enter \3, \11, \30, with others.\)

\3	Let him appear that's come from Antony.
	Know you him? \\

\11	                  Caesar, 'tis his schoolmaster:
	An argument that he is pluck'd, when hither
	He sends so poor a pinion off his wing,
	Which had superfluous kings for messengers
	Not many moons gone by. \\

	\(Enter Ambassador from \1\)


\3	Approach, and speak.

\50	Such as I am, I come from Antony:
	I was of late as petty to his ends
	As is the morn-dew on the myrtle-leaf
	To his grand sea. \\

\3	                  Be't so: declare thine office.

\50	Lord of his fortunes he salutes thee, and
	Requires to live in Egypt: which not granted,
	He lessens his requests; and to thee sues
	To let him breathe between the heavens and earth,
	A private man in Athens: this for him.
	Next, Cleopatra does confess thy greatness;
	Submits her to thy might; and of thee craves
	The circle of the Ptolemies for her heirs,
	Now hazarded to thy grace. \\

\3	For Antony,
	I have no ears to his request. The queen
	Of audience nor desire shall fail, so she
	From Egypt drive her all-disgraced friend,
	Or take his life there: this if she perform,
	She shall not sue unheard. So to them both.

\50	Fortune pursue thee! \\

\3	Bring him through the bands. 	\[r]Exit ambassador.\]
	\[To \30\] 	To try eloquence, now 'tis time: dispatch;
	From Antony win Cleopatra: promise,
	And in our name, what she requires; add more,
	From thine invention, offers: women are not
	In their best fortunes strong; but want will perjure
	The ne'er touch'd vestal: try thy cunning, Thidias;
	Make thine own edict for thy pains, which we
	Will answer as a law. \\

\30	Caesar, I go.

\3	Observe how Antony becomes his flaw,
	And what thou think'st his very action speaks
	In every power that moves. \\

\30	Caesar, I shall.  	\[r]Exeunt.\]



\Scene{Alexandria. Cleopatra's palace.}


	\(Enter \2, \13, \7, and \16\)

\2	What shall we do, Enobarbus?  \\

\13	Think, and die.

\2	Is Antony or we in fault for this?

\13	Antony only, that would make his will
	Lord of his reason. What though you fled
	From that great face of war, whose several ranges
	Frighted each other? why should he follow?
	The itch of his affection should not then
	Have nick'd his captainship; at such a point,
	When half to half the world opposed, he being
	The meered question: 'twas a shame no less
	Than was his loss, to course your flying flags,
	And leave his navy gazing. \\

\2	Prithee, peace.

	\(Enter \1 with the Ambassador.\)

\1	Is that his answer?

\50	Ay, my lord.

\1	The queen shall then have courtesy, so she
	Will yield us up. \\

\50	                  He says so. \\

\1	Let her know't.
	To the boy Caesar send this grizzled head,
	And he will fill thy wishes to the brim
	With principalities. \\

\2	That head, my lord?

\1	To him again: tell him he wears the rose
	Of youth upon him; from which the world should note
	Something particular: his coin, ships, legions,
	May be a coward's; whose ministers would prevail
	Under the service of a child as soon
	As i' the command of Caesar: I dare him therefore
	To lay his gay comparisons apart,
	And answer me declined, sword against sword,
	Ourselves alone. I'll write it: follow me. \[r]Exeunt \1 and Ambassador.\]

\13	\[Aside.\]  Yes, like enough, high-battled Caesar will
	Unstate his happiness, and be staged to the show,
	Against a sworder! I see men's judgments are
	A parcel of their fortunes; and things outward
	Do draw the inward quality after them,
	To suffer all alike. That he should dream,
	Knowing all measures, the full Caesar will
	Answer his emptiness! Caesar, thou hast subdued
	His judgment too. \\

	\(Enter a Servant.\)

\63	                  A messenger from C\ae{}sar.

\2	What, no more ceremony? See, my women!
	Against the blown rose may they stop their nose
	That kneel'd unto the buds. Admit him, sir. 	\[r]Exit Servant.\]

\13	\[Aside.\]  Mine honesty and I begin to square.
	The loyalty well held to fools does make
	Our faith mere folly: yet he that can endure
	To follow with allegiance a fall'n lord
	Does conquer him that did his master conquer
	And earns a place i' the story.\\

	\(Enter \30\)

\2	Caesar's will?

\30	Hear it apart. \\

\2	                  None but friends: say boldly.

\30	So, haply, are they friends to Antony.

\13	He needs as many, sir, as Caesar has;
	Or needs not us. If Caesar please, our master
	Will leap to be his friend: for us, you know,
	Whose he is we are, and that is, Caesar's. \\

\30	So.
	Thus then, thou most renown'd: Caesar entreats,
	Not to consider in what case thou stand'st,
	Further than he is Caesar. \\

\2	Go on: right royal.

\30	He knows that you embrace not Antony
	As you did love, but as you fear'd him. \\

\2	O!

\30	The scars upon your honour, therefore, he
	Does pity, as constrained blemishes,
	Not as deserved. \\

\2	                  He is a god, and knows
	What is most right: mine honour was not yielded,
	But conquer'd merely. \\

\13	\[Aside.\]             To be sure of that,
	I will ask Antony. Sir, sir, thou art so leaky,
	That we must leave thee to thy sinking, for
	Thy dearest quit thee. \\	\[r]Exit.\]

\30	Shall I say to Caesar
	What you require of him? for he partly begs
	To be desired to give. It much would please him,
	That of his fortunes you should make a staff
	To lean upon: but it would warm his spirits,
	To hear from me you had left Antony,
	And put yourself under his shrowd,
	The universal landlord. \\

\2	What's your name?

\30	My name is Thidias. \\

\2	Most kind messenger,
	Say to great Caesar this: in deputation
	I kiss his conquering hand: tell him, I am prompt
	To lay my crown at 's feet, and there to kneel:
	Tell him from his all-obeying breath I hear
	The doom of Egypt. \\

\30	'Tis your noblest course.
	Wisdom and fortune combating together,
	If that the former dare but what it can,
	No chance may shake it. Give me grace to lay
	My duty on your hand. \\

\2	Your Caesar's father oft,
	When he hath mused of taking kingdoms in,
	Bestow'd his lips on that unworthy place,
	As it rain'd kisses. \\

	\(Re-enter \1 and \13\)

\1	Favours, by Jove that thunders!
	What art thou, fellow? \\

\30	One that but performs
	The bidding of the fullest man, and worthiest
	To have command obey'd. \\

\13	\[Aside.\]               You will be whipp'd.

\1	Approach, there! Ah, you kite! Now, gods 	and devils!
	Authority melts from me: of late, when I cried `Ho!'
	Like boys unto a muss, kings would start forth,
	And cry `Your will?' Have you no ears? I am
	Antony yet. \\

	\(Enter Servants.\)

	Take hence this Jack, and whip him.

\13	\[Aside.\]  'Tis better playing with a lion's whelp
	Than with an old one dying. \\

\1	Moon and stars!
	Whip him. Were't twenty of the greatest tributaries
	That do acknowledge Caesar, should I find them
	So saucy with the hand of she here,---what's her name,
	Since she was Cleopatra? Whip him, fellows,
	Till, like a boy, you see him cringe his face,
	And whine aloud for mercy: take him hence.

\30	Mark Antony! \\


\1	                  Tug him away: being whipp'd,
	Bring him again: this Jack of Caesar's shall
	Bear us an errand to him.  \[r]Exeunt Servants with Thidias.\]
	You were half blasted ere I knew you: ha!
	Have I my pillow left unpress'd in Rome,
	Forborne the getting of a lawful race,
	And by a gem of women, to be abused
	By one that looks on feeders? \\

\2	Good my lord,---

\1	You have been a boggler ever:
	But when we in our viciousness grow hard---
	O misery on't!---the wise gods seel our eyes;
	In our own filth drop our clear judgments; make us
	Adore our errors; laugh at's, while we strut
	To our confusion. \\

\2	                  O, is't come to this?

\1	I found you as a morsel cold upon
	Dead Caesar's trencher; nay, you were a fragment
	Of Cneius Pompey's; besides what hotter hours,
	Unregister'd in vulgar fame, you have
	Luxuriously pick'd out: for, I am sure,
	Though you can guess what temperance should be,
	You know not what it is. \\

\2	Wherefore is this?

\1	To let a fellow that will take rewards
	And say `God quit you!' be familiar with
	My playfellow, your hand; this kingly seal
	And plighter of high hearts! O, that I were
	Upon the hill of Basan, to outroar
	The horned herd! for I have savage cause;
	And to proclaim it civilly, were like
	A halter'd neck which does the hangman thank
	For being yare about him. \\

	\(Enter a Servant with Thidias.\)

		    Is he whipp'd?

\63	Soundly, my lord. \\

\1	                  Cried he? and begg'd a' pardon?

\63	He did ask favour.

\1	If that thy father live, let him repent
	Thou wast not made his daughter; and be thou sorry
	To follow Caesar in his triumph, since
	Thou hast been whipp'd for following him: henceforth
	The white hand of a lady fever thee,
	Shake thou to look on 't. Get thee back to Caesar,
	Tell him thy entertainment: look, thou say
	He makes me angry with him; for he seems
	Proud and disdainful, harping on what I am,
	Not what he knew I was: he makes me angry;
	And at this time most easy 'tis to do't,
	When my good stars, that were my former guides,
	Have empty left their orbs, and shot their fires
	Into the abysm of hell. If he mislike
	My speech and what is done, tell him he has
	Hipparchus, my enfranched bondman, whom
	He may at pleasure whip, or hang, or torture,
	As he shall like, to quit me: urge it thou:
	Hence with thy stripes, begone! \[r]Exit Thidias.\]

\2	Have you done yet? \\

\1	                  Alack, our terrene moon
	Is now eclipsed; and it portends alone
	The fall of Antony! \\

\2	I must stay his time.

\1	To flatter Caesar, would you mingle eyes
	With one that ties his points? \\

\2	Not know me yet?

\1	Cold-hearted toward me? \\

\2	Ah, dear, if I be so,
	From my cold heart let heaven engender hail,
	And poison it in the source; and the first stone
	Drop in my neck: as it determines, so
	Dissolve my life! The next Caesarion smite!
	Till by degrees the memory of my womb,
	Together with my brave Egyptians all,
	By the discandying of this pelleted storm,
	Lie graveless, till the flies and gnats of Nile
	Have buried them for prey! \\

\1	I am satisfied.
	Caesar sits down in Alexandria; where
	I will oppose his fate. Our force by land
	Hath nobly held; our sever'd navy too
	Have knit again, and fleet, threatening most sea-like.
	Where hast thou been, my heart? Dost thou hear, lady?
	If from the field I shall return once more
	To kiss these lips, I will appear in blood;
	I and my sword will earn our chronicle:
	There's hope in't yet.

\2	That's my brave lord!

\1	I will be treble-sinew'd, hearted, breathed,
	And fight maliciously: for when mine hours
	Were nice and lucky, men did ransom lives
	Of me for jests; but now I'll set my teeth,
	And send to darkness all that stop me. Come,
	Let's have one other gaudy night: call to me
	All my sad captains; fill our bowls once more;
	Let's mock the midnight bell. \\

\2	It is my birth-day:
	I had thought to have held it poor: but, since my lord
	Is Antony again, I will be Cleopatra.

\1	We will yet do well.

\2	Call all his noble captains to my lord.

\1	Do so, we'll speak to them; and to-night I'll force
	The wine peep through their scars. Come on, my queen;
	There's sap in't yet. The next time I do fight,
	I'll make death love me; for I will contend
	Even with his pestilent scythe. 	\[r]Exeunt all but Enobarbus.\]

\13	Now he'll outstare the lightning. To be furious,
	Is to be frighted out of fear; and in that mood
	The dove will peck the estridge; and I see still,
	A diminution in our captain's brain
	Restores his heart: when valour preys on reason,
	It eats the sword it fights with. I will seek
	Some way to leave him. 	\[r]Exit.\]




\Act


\Scene{Before Alexandria. Caesar's camp.}


	\(Enter \3, \4, and \19, with his Army; \3 reading a letter.\)

\3	He calls me boy; and chides, as he had power
	To beat me out of Egypt; my messenger
	He hath whipp'd with rods; dares me to personal combat,
	Caesar to Antony: let the old ruffian know
	I have many other ways to die; meantime
	Laugh at his challenge. \\

\19	Caesar must think,
	When one so great begins to rage, he's hunted
	Even to falling. Give him no breath, but now
	Make boot of his distraction: never anger
	Made good guard for itself. \\

\3	Let our best heads
	Know, that to-morrow the last of many battles
	We mean to fight: within our files there are,
	Of those that served Mark Antony but late,
	Enough to fetch him in. See it done:
	And feast the army; we have store to do't,
	And they have earn'd the waste. Poor Antony! \[r]Exeunt.\]


\Scene{Alexandria. Cleopatra's palace.}


	\(Enter \1, \2, \13, \7, \16,\\ \5, with others.\)

\1	He will not fight with me, Domitius. \\

\13	No.

\1	Why should he not?

\13	He thinks, being twenty times of better fortune,
	He is twenty men to one. \\

\1	To-morrow, soldier,
	By sea and land I'll fight: or I will live,
	Or bathe my dying honour in the blood
	Shall make it live again. Woo't thou fight well?

\13	I'll strike, and cry `Take all.' \\

\1	Well said; come on.
	Call forth my household servants: let's to-night

	\(Enter three or four Servitors.\)

	Be bounteous at our meal. Give me thy hand,
	Thou hast been rightly honest;---so hast thou;---
	Thou,---and thou,---and thou:---you have served me well,
	And kings have been your fellows. \\

\2	\[Aside to \13\]  What means this?

\13	\[Aside to \2\]  'Tis one of those odd tricks which sorrow shoots
	Out of the mind. \\

\1	                  And thou art honest too.
	I wish I could be made so many men,
	And all of you clapp'd up together in
	An Antony, that I might do you service
	So good as you have done. \\

\99	The gods forbid!

\1	Well, my good fellows, wait on me to-night:
	Scant not my cups; and make as much of me
	As when mine empire was your fellow too,
	And suffer'd my command. \\

\2 \relax \[lb]Aside to \13\]  What does he mean?

\13	\[Aside to \2\]  To make his followers weep. \\

\1	Tend me to-night;
	May be it is the period of your duty:
	Haply you shall not see me more; or if,
	A mangled shadow: perchance to-morrow
	You'll serve another master. I look on you
	As one that takes his leave. Mine honest friends,
	I turn you not away; but, like a master
	Married to your good service, stay till death:
	Tend me to-night two hours, I ask no more,
	And the gods yield you for't! \\

\13	What mean you, sir,
	To give them this discomfort? Look, they weep;
	And I, an ass, am onion-eyed: for shame,
	Transform us not to women. \\

\1	Ho, ho, ho!
	Now the witch take me, if I meant it thus!
	Grace grow where those drops fall, my hearty friends,
	You take me in too dolorous a sense;
	For I spake to you for your comfort; did desire you
	To burn this night with torches: know, my hearts,
	I hope well of to-morrow; and will lead you
	Where rather I'll expect victorious life
	Than death and honour. Let's to supper, come,
	And drown consideration. \[r]Exeunt.\]


\Scene{The same. Before the palace.}


	\(Enter two Soldiers to their guard.\)

\67	Brother, good night: to-morrow is the day.

\68	It will determine one way: fare you well.
	Heard you of nothing strange about the streets?

\67	Nothing. What news?

\68	Belike 'tis but a rumour. Good night to you.

\67	Well, sir, good night. \\

	\(Enter two other Soldiers.\)

\69	Soldiers, have careful watch.

\67	And you. Good night, good night. \[r]They place themselves in every corner of the stage.\]

\68	Here we: and if to-morrow
	Our navy thrive, I have an absolute hope
	Our landmen will stand up. \\

\67	'Tis a brave army,
	And full of purpose. \\  \[r]Music of the hautboys as under the stage.\]

\68	Peace! what noise? \\

\67	List, list!

\68	Hark! \\

\67	    Music i' the air. \\

\69 	Under the earth.

\70	It signs well, does it not? \\

\69	No. \\

\67	Peace, I say!
	What should this mean?

\68	'Tis the god Hercules, whom Antony loved,
	Now leaves him. \\

\67	Walk; let's see if other watchmen
	Do hear what we do? \\

\68	How now, masters! \\

\99	\[Speaking together.\]  How now!
	How now! do you hear this? \\

\67	Ay; is't not strange?

\69	Do you hear, masters? do you hear?

\67	Follow the noise so far as we have quarter;
	Let's see how it will give off. \\

\99	Content. 'Tis strange. 	\[r]Exeunt.\]



\Scene{The same. A room in the palace.}


	\(Enter \1 and \2, \7, and others\\ attending.\)

\1	Eros! mine armour, Eros! \\

\2	Sleep a little.

\1	No, my chuck. Eros, come; mine armour, Eros!


	\(Enter \14 with armour.\)

	Come good fellow, put mine iron on:
	If fortune be not ours to-day, it is
	Because we brave her: come. \\

\2	Nay, I'll help too.
	What's this for? \\

\1	                  Ah, let be, let be! thou art
	The armourer of my heart: false, false; this, this.

\2	Sooth, la, I'll help: thus it must be. \\

\1	Well, well;
	We shall thrive now. Seest thou, my good fellow?
	Go put on thy defences. \\

\14	Briefly, sir.

\2	Is not this buckled well? \\

\1	Rarely, rarely:
	He that unbuckles this, till we do please
	To daff't for our repose, shall hear a storm.
	Thou fumblest, Eros; and my queen's a squire
	More tight at this than thou: dispatch. O love,
	That thou couldst see my wars to-day, and knew'st
	The royal occupation! thou shouldst see
	A workman in't. \\

	\(Enter an armed Soldier.\)

	Good morrow to thee; welcome:
	Thou look'st like him that knows a warlike charge:
	To business that we love we rise betime,
	And go to't with delight. \\

\66	A thousand, sir,
	Early though't be, have on their riveted trim,
	And at the port expect you. 	\[r]Shout. Trumpets flourish.\]

	\(Enter Captains and Soldiers.\)

\54	The morn is fair. Good morrow, general.

\99	Good morrow, general. \\

\1	'Tis well blown, lads:
	This morning, like the spirit of a youth
	That means to be of note, begins betimes.
	So, so; come, give me that: this way; well said.
	Fare thee well, dame, whate'er becomes of me:
	This is a soldier's kiss: rebukeable \[r]Kisses her.\]

	And worthy shameful cheque it were, to stand
	On more mechanic compliment; I'll leave thee
	Now, like a man of steel. You that will fight,
	Follow me close; I'll bring you to't. Adieu.
  \[r]Exeunt \1, \14, Captains, and Soldiers.\]

\7	Please you, retire to your chamber. \\

\2	Lead me.
	He goes forth gallantly. That he and Caesar might
	Determine this great war in single fight!
	Then Antony,---but now---Well, on. \[r]Exeunt.\]

\Scene{Alexandria. Antony's camp.}


	\(Trumpets sound. Enter \1 and \14; a soldier meeting them.\)

\66	The gods make this a happy day to Antony!

\1	Would thou and those thy scars had once prevail'd
	To make me fight at land! \\

\66	Hadst thou done so,
	The kings that have revolted, and the soldier
	That has this morning left thee, would have still
	Follow'd thy heels. \\

\1	Who's gone this morning? \\

\66	Who!
	One ever near thee: call for Enobarbus,
	He shall not hear thee; or from Caesar's camp
	Say `I am none of thine.' \\

\1	What say'st thou? \\

\66	Sir,
	He is with Caesar. \\

\14	                  Sir, his chests and treasure
	He has not with him. \\

\1	Is he gone? \\

\66	Most certain.

\1	Go, Eros, send his treasure after; do it;
	Detain no jot, I charge thee: write to him---
	I will subscribe---gentle adieus and greetings;
	Say that I wish he never find more cause
	To change a master. O, my fortunes have
	Corrupted honest men! Dispatch.---Enobarbus! \[r]Exeunt.\]



\Scene{Alexandria. Caesar's camp.}


	\(Flourish. Enter \3, \4, with \13,\\ and others.\)

\3	Go forth, Agrippa, and begin the fight:
	Our will is Antony be took alive;
	Make it so known.

\4	Caesar, I shall.  	\[r]Exit.\]

\3	The time of universal peace is near:
	Prove this a prosperous day, the three-nook'd world
	Shall bear the olive freely. \\

	\(Enter a Messenger.\)

\60	Antony
	Is come into the field. \\

\3	Go charge Agrippa
	Plant those that have revolted in the van,
	That Antony may seem to spend his fury
	Upon himself. 	\[r]Exeunt all but \13\]

\13	Alexas did revolt; and went to Jewry on
	Affairs of Antony; there did persuade
	Great Herod to incline himself to Caesar,
	And leave his master Antony: for this pains
	Caesar hath hang'd him. Canidius and the rest
	That fell away have entertainment, but
	No honourable trust. I have done ill;
	Of which I do accuse myself so sorely,
	That I will joy no more. \\

	\(Enter a Soldier of \3's.\)

\66	Enobarbus, Antony
	Hath after thee sent all thy treasure, with
	His bounty overplus: the messenger
	Came on my guard; and at thy tent is now
	Unloading of his mules. \\

\13	I give it you.

\66	Mock not, Enobarbus.
	I tell you true: best you safed the bringer
	Out of the host; I must attend mine office,
	Or would have done't myself. Your emperor
	Continues still a Jove. \[r]Exit.\]

\13	I am alone the villain of the earth,
	And feel I am so most. O Antony,
	Thou mine of bounty, how wouldst thou have paid
	My better service, when my turpitude
	Thou dost so crown with gold! This blows my heart:
	If swift thought break it not, a swifter mean
	Shall outstrike thought: but thought will do't, I feel.
	I fight against thee! No: I will go seek
	Some ditch wherein to die; the foul'st best fits
	My latter part of life. 	\[r]Exit.\]



\Scene{Field of battle between the camps.}


	\(Alarum. Drums and trumpets. Enter \4 and others.\)

\4	Retire, we have engaged ourselves too far:
	Caesar himself has work, and our oppression
	Exceeds what we expected. 	\[r]Exeunt.\]


	\(Alarums. Enter \1 and \26 wounded.\)

\26	O my brave emperor, this is fought indeed!
	Had we done so at first, we had droven them home
	With clouts about their heads. \\

\1	Thou bleed'st apace.

\26	I had a wound here that was like a T,
	But now 'tis made an H. \\

\1	They do retire.

\26	We'll beat 'em into bench-holes: I have yet
	Room for six scotches more.

	\(Enter \14\)

\14	They are beaten, sir, and our advantage serves
	For a fair victory. \\

\26	Let us score their backs,
	And snatch 'em up, as we take hares, behind:
	'Tis sport to maul a runner. \\

\1	I will reward thee
	Once for thy spritely comfort, and ten-fold
	For thy good valour. Come thee on. \\

\26	I'll halt after. 	\[r]Exeunt.\]

\newpage

\Scene{Under the walls of Alexandria.}


	\(Alarum. Enter \1, in a march; \26, with others.\)

\1	We have beat him to his camp: run one before,
	And let the queen know of our gests. To-morrow,
	Before the sun shall see 's, we'll spill the blood
	That has to-day escaped. I thank you all;
	For doughty-handed are you, and have fought
	Not as you served the cause, but as 't had been
	Each man's like mine; you have shown all Hectors.
	Enter the city, clip your wives, your friends,
	Tell them your feats; whilst they with joyful tears
	Wash the congealment from your wounds, and kiss
	The honour'd gashes whole. \\
   \(Enter \2, attended.\)
	\[l]To \26\] Give me thy hand
	To this great fairy I'll commend thy acts,
	Make her thanks bless thee. O thou day o' the world,
	Chain mine arm'd neck; leap thou, attire and all,
	Through proof of harness to my heart, and there
	Ride on the pants triumphing! \\

\2	Lord of lords!
	O infinite virtue, comest thou smiling from
	The world's great snare uncaught? \\

\1	My nightingale,
	We have beat them to their beds. What, girl! though grey
	Do something mingle with our younger brown, yet ha' we
	A brain that nourishes our nerves, and can
	Get goal for goal of youth. Behold this man;
	Commend unto his lips thy favouring hand:
	Kiss it, my warrior: he hath fought to-day
	As if a god, in hate of mankind, had
	Destroy'd in such a shape. \\

\2	I'll give thee, friend,
	An armour all of gold; it was a king's.

\1	He has deserved it, were it carbuncled
	Like holy Phoebus' car. Give me thy hand:
	Through Alexandria make a jolly march;
	Bear our hack'd targets like the men that owe them:
	Had our great palace the capacity
	To camp this host, we all would sup together,
	And drink carouses to the next day's fate,
	Which promises royal peril. Trumpeters,
	With brazen din blast you the city's ear;
	Make mingle with rattling tabourines;
	That heaven and earth may strike their sounds together,
	Applauding our approach. 	\[r]Exeunt.\]


\Scene{Caesar's camp.}


	\(Enter a Sentry and his company. \13 follows.\)

\62	If we be not relieved within this hour,
	We must return to the court of guard: the night
	Is shiny; and they say we shall embattle
	By the second hour i' the morn. \\

\72	This last day was
	A shrewd one to's. \\

\13	                  O, bear me witness, night,---

\73	What man is this? \\

\72	                  Stand close, and list him.

\13	Be witness to me, O thou blessed moon,
	When men revolted shall upon record
	Bear hateful memory, poor Enobarbus did
	Before thy face repent! \\

\62	Enobarbus! \\

\73	Peace!
	Hark further.

\13	O sovereign mistress of true melancholy,
	The poisonous damp of night disponge upon me,
	That life, a very rebel to my will,
	May hang no longer on me: throw my heart
	Against the flint and hardness of my fault:
	Which, being dried with grief, will break to powder,
	And finish all foul thoughts. O Antony,
	Nobler than my revolt is infamous,
	Forgive me in thine own particular;
	But let the world rank me in register
	A master-leaver and a fugitive:
	O Antony! O Antony! \\

\72	Let's speak To him.

\62	Let's hear him, for the things he speaks
	May concern Caesar. \\

\72	Let's do so. But he sleeps.

\62	Swoons rather; for so bad a prayer as his
	Was never yet for sleep. \\

\72	Go we to him.

\73	Awake, sir, awake; speak to us. \\

\72	Hear you, sir?

\62	The hand of death hath raught him. \\
	\[l]Drums afar off.\] Hark! the drums
	Demurely wake the sleepers. Let us bear him
	To the court of guard; he is of note: our hour
	Is fully out. 

\72	Come on, then; He may recover yet. 	\[r]Exeunt with the body.\]



\Scene{Between the two camps.}


	\(Enter \1 and \26, with their Army.\)

\1	Their preparation is to-day by sea;
	We please them not by land.

\26 For both, my lord. \\

\1	I would they'ld fight i' the fire or i' the air;
	We'ld fight there too. But this it is; our foot
	Upon the hills adjoining to the city
	Shall stay with us: order for sea is given;
	They have put forth the haven 
	Where their appointment we may best discover,
	And look on their endeavour. 	\[r]Exeunt.\]




\Scene{Another part of the same.}


	\(Enter \3, and his Army.\)

\3	But being charged, we will be still by land,
	Which, as I take't, we shall; for his best force
	Is forth to man his galleys. To the vales,
	And hold our best advantage.  \[r]Exeunt.\]

\Scene{Another part of the same.}


	\(Enter \1 and \26\)

\1	Yet they are not join'd: where yond pine 	does stand,
	I shall discover all: I'll bring thee word
	Straight, how 'tis like to go. \\ \[r]Exit.\]

\26 Swallows have built
	In Cleopatra's sails their nests: the augurers
	Say they know not, they cannot tell; look grimly,
	And dare not speak their knowledge. Antony
	Is valiant, and dejected; and, by starts,
	His fretted fortunes give him hope, and fear,
	Of what he has, and has not. \\

	\(Re-enter \1\)

\1	All is lost;
	This foul Egyptian hath betrayed me:
	My fleet hath yielded to the foe; and yonder
	They cast their caps up and carouse together
	Like friends long lost. Triple-turn'd whore! 'tis thou
	Hast sold me to this novice; and my heart
	Makes only wars on thee. Bid them all fly;
	For when I am revenged upon my charm,
	I have done all. Bid them all fly; begone. \[r]Exit \26\]
	O sun, thy uprise shall I see no more:
	Fortune and Antony part here; even here
	Do we shake hands. All come to this? The hearts
	That spaniel'd me at heels, to whom I gave
	Their wishes, do discandy, melt their sweets
	On blossoming Caesar; and this pine is bark'd,
	That overtopp'd them all. Betray'd I am:
	O this false soul of Egypt! this grave charm,---
	Whose eye beck'd forth my wars, and call'd them home;
	Whose bosom was my crownet, my chief end,---
	Like a right gipsy, hath, at fast and loose,
	Beguiled me to the very heart of loss.
	What, Eros, Eros! \\

	\(Enter \2\)

	Ah, thou spell! Avaunt!

\2	Why is my lord enraged against his love?

\1	Vanish, or I shall give thee thy deserving,
	And blemish Caesar's triumph. Let him take thee,
	And hoist thee up to the shouting plebeians:
	Follow his chariot, like the greatest spot
	Of all thy sex; most monster-like, be shown
	For poor'st diminutives, for doits; and let
	Patient Octavia plough thy visage up
	With her prepared nails.  \\	\[r]Exit \2\]

		'Tis well thou'rt gone,
	If it be well to live; but better 'twere
	Thou fell'st into my fury, for one death
	Might have prevented many. Eros, ho!
	The shirt of Nessus is upon me: teach me,
	Alcides, thou mine ancestor, thy rage:
	Let me lodge Lichas on the horns o' the moon;
	And with those hands, that grasp'd the heaviest club,
	Subdue my worthiest self. The witch shall die:
	To the young Roman boy she hath sold me, and I fall
	Under this plot; she dies for't. Eros, ho! \[r]Exit.\]


\Scene{Alexandria. Cleopatra's palace.}


	\(Enter \2, \7, \16, and \18\)

\2	Help me, my women! O, he is more mad
	Than Telamon for his shield; the boar of Thessaly
	Was never so emboss'd. \\

\7	To the monument!
	There lock yourself, and send him word you are dead.
	The soul and body rive not more in parting
	Than greatness going off. \\

\2	To the monument!
	Mardian, go tell him I have slain myself;
	Say, that the last I spoke was `Antony,'
	And word it, prithee, piteously: hence, Mardian,
	And bring me how he takes my death to the monument! \[r]Exeunt.\]


\Scene{The same. Another room.}


	\(Enter \1 and \14\)

\1	Eros, thou yet behold'st me? \\

\14	Ay, noble lord.

\1	Sometimes we see a cloud that's dragonish;
	A vapour sometime like a bear or lion,
	A tower'd citadel, a pendent rock,
	A forked mountain, or blue promontory
	With trees upon't, that nod unto the world,
	And mock our eyes with air: thou hast seen these signs;
	They are black vesper's pageants. \\

\14	Ay, my lord,

\1	That which is now a horse, even with a thought
	The rack dislimns, and makes it indistinct,
	As water is in water. \\

\14	It does, my lord.

\1	My good knave Eros, now thy captain is
	Even such a body: here I am Antony:
	Yet cannot hold this visible shape, my knave.
	I made these wars for Egypt: and the queen,---
	Whose heart I thought I had, for she had mine;
	Which whilst it was mine had annex'd unto't
	A million more, now lost,---she, Eros, has
	Pack'd cards with Caesar, and false-play'd my glory
	Unto an enemy's triumph.
	Nay, weep not, gentle Eros; there is left us
	Ourselves to end ourselves. \\


	\(Enter \18\)

		      O! thy vile lady!
	She has robb'd me of my sword. \\

\18	No, Antony;
	My mistress loved thee, and her fortunes mingled
	With thine entirely. \\

\1	Hence, saucy eunuch; peace!
	She hath betray'd me and shall die the death.

\18	Death of one person can be paid but once,
	And that she has discharged: what thou wouldst do
	Is done unto thy hand: the last she spake
	Was `Antony! most noble Antony!'
	Then in the midst a tearing groan did break
	The name of Antony; it was divided
	Between her heart and lips: she render'd life,
	Thy name so buried in her. \\

\1	Dead, then? \\

\18	Dead.

\1	Unarm, Eros; the long day's task is done,
	And we must sleep. \[To \18\] That thou depart'st hence safe,
	Does pay thy labour richly; go.\\ \[r]Exit \18\]
		         Off, pluck off:
	The seven-fold shield of Ajax cannot keep
	The battery from my heart. O, cleave, my sides!
	Heart, once be stronger than thy continent,
	Crack thy frail case! Apace, Eros, apace.
	No more a soldier: bruised pieces, go;
	You have been nobly borne. From me awhile. \[r]Exit \14\]
	I will o'ertake thee, Cleopatra, and
	Weep for my pardon. So it must be, for now
	All length is torture: since the torch is out,
	Lie down, and stray no farther: now all labour
	Mars what it does; yea, very force entangles
	Itself with strength: seal then, and all is done.
	Eros!---I come, my queen:---Eros!---Stay for me:
	Where souls do couch on flowers, we'll hand in hand,
	And with our sprightly port make the ghosts gaze:
	Dido and her \AE{}neas shall want troops, 
	And all the haunt be ours. Come, Eros, Eros! 

	\(Re-enter \14\)

\14 What would my lord? \\


\1	Since Cleopatra died,
	I have lived in such dishonour, that the gods
	Detest my baseness. I, that with my sword
	Quarter'd the world, and o'er green Neptune's back
	With ships made cities, condemn myself to lack
	The courage of a woman; less noble mind
	Than she which by her death our Caesar tells
	`I am conqueror of myself.' Thou art sworn, Eros,
	That, when the exigent should come, which now
	Is come indeed, when I should see behind me
	The inevitable prosecution of
	Disgrace and horror, that, on my command,
	Thou then wouldst kill me: do't; the time is come:
	Thou strikest not me, 'tis Caesar thou defeat'st.
	Put colour in thy cheek. \\



\14	The gods withhold me!
	Shall I do that which all the Parthian darts,
	Though enemy, lost aim, and could not? \\

\1	Eros,
	Wouldst thou be window'd in great Rome and see
	Thy master thus with pleach'd arms, bending down
	His corrigible neck, his face subdued
	To penetrative shame, whilst the wheel'd seat
	Of fortunate Caesar, drawn before him, branded
	His baseness that ensued? \\

\14	I would not see't.

\1	Come, then; for with a wound I must be cured.
	Draw that thy honest sword, which thou hast worn
	Most useful for thy country. \\

\14	O, sir, pardon me!

\1	When I did make thee free, sworest thou not then
	To do this when I bade thee? Do it at once;
	Or thy precedent services are all
	But accidents unpurposed. Draw, and come.

\14	Turn from me, then, that noble countenance,
	Wherein the worship of the whole world lies.

\1	Lo thee! \[r]Turning from him.\]

\14	My sword is drawn. \\

\1	                  Then let it do at once
	The thing why thou hast drawn it. \\

\14	My dear master,
	My captain, and my emperor, let me say,
	Before I strike this bloody stroke, farewell.

\1	'Tis said, man; and farewell.

\14	Farewell, great chief. Shall I strike now? \\

\1	Now, Eros.

\14	Why, there then: thus I do escape the sorrow
	Of Antony's death. \\ \[r]Kills himself.\]

\1	                  Thrice-nobler than myself!
	Thou teachest me, O valiant Eros, what
	I should, and thou couldst not. My queen and Eros
	Have by their brave instruction got upon me
	A nobleness in record: but I will be
	A bridegroom in my death, and run into't
	As to a lover's bed. Come, then; and, Eros,
	Thy master dies thy scholar: to do thus \[r]Falling on his sword.\]
	I learn'd of thee. How! not dead? not dead?
	The guard, ho! O, dispatch me! \\



	\(Enter \8 and Guard.\)

\57	What's the noise?

\1	I have done my work in, friends: O, make an end
	Of what I have begun. \\

\58	The star is fall'n.

\57	And time is at his period. \\

\99	Alas, and woe!

\1	Let him that loves me strike me dead. \\

\57	Not I.

\58	Nor I.

\59	Nor any one. 	\[r]Exeunt Guards.\]

\8	Thy death and fortunes bid thy followers fly.
	This sword but shown to Caesar, with this tidings,
	Shall enter me with him.


	\(Enter \10\)

\10	Where's Antony? \\

\8	                  There, Diomed there. \\

\10	Lives he?
	Wilt thou not answer, man?  \[r]Exit \8\]

\1	Art thou there, Diomed? Draw thy sword, and give me
	Sufficing strokes for death. \\

\10	Most absolute lord,
	My mistress Cleopatra sent me to thee.

\1	When did she send thee? \\

\10	Now, my lord. \\

\1	Where is she?

\10	Lock'd in her monument. She had a prophesying fear
	Of what hath come to pass: for when she saw---
	Which never shall be found---you did suspect
	She had disposed with Caesar, and that your rage
	Would not be purged, she sent you word she was dead;
	But, fearing since how it might work, hath sent
	Me to proclaim the truth; and I am come,
	I dread, too late.

\1	Too late, good Diomed: call my guard, I prithee.

\10	What, ho, the emperor's guard! The guard, what, ho!
	Come, your lord calls!

	\(Enter four or five of the Guard of \1\)

\1	Bear me, good friends, where Cleopatra bides;
	'Tis the last service that I shall command you.

\57	Woe, woe are we, sir, you may not live to wear
	All your true followers out. \\

\99	Most heavy day!

\1	Nay, good my fellows, do not please sharp fate
	To grace it with your sorrows: bid that welcome
	Which comes to punish us, and we punish it
	Seeming to bear it lightly. Take me up:
	I have led you oft: carry me now, good friends,
	And have my thanks for all. \[r]Exeunt, bearing \1\]


\Scene{The same. A monument.}


	\(Enter \2 and her maids aloft,\\ with \7 and \16\)

\2	O Charmian, I will never go from hence.

\7	Be comforted, dear madam. \\

\2	No, I will not:
	All strange and terrible events are welcome,
	But comforts we despise; our size of sorrow,
	Proportion'd to our cause, must be as great
	As that which makes it. \\

	\(Enter, below, \10\)

		  How now! is he dead?

\10	His death's upon him, but not dead.
	Look out o' the other side your monument;
	His guard have brought him thither. \\

	\(Enter, below, \1, borne by the Guard.\)

\2	O sun,
	Burn the great sphere thou movest in! darkling stand
	The varying shore o' the world. O Antony,
	Antony, Antony! Help, Charmian, help, Iras, help;
	Help, friends below; let's draw him hither. \\

\1	Peace!
	Not Caesar's valour hath o'erthrown Antony,
	But Antony's hath triumph'd on itself.

\2	So it should be, that none but Antony
	Should conquer Antony; but woe 'tis so!

\1	I am dying, Egypt, dying; only
	I here importune death awhile, until
	Of many thousand kisses the poor last
	I lay up thy lips. \\

\2	                  I dare not, dear,---
	Dear my lord, pardon,---I dare not,
	Lest I be taken: not the imperious show
	Of the full-fortuned Caesar ever shall
	Be brooch'd with me; if knife, drugs, serpents, have
	Edge, sting, or operation, I am safe:
	Your wife Octavia, with her modest eyes
	And still conclusion, shall acquire no honour
	Demuring upon me. But come, come, Antony,---
	Help me, my women,---we must draw thee up:
	Assist, good friends. \\

\1	O, quick, or I am gone.

\2	Here's sport indeed! How heavy weighs my lord!
	Our strength is all gone into heaviness,
	That makes the weight: had I great Juno's power,
	The strong-wing'd Mercury should fetch thee up,
	And set thee by Jove's side. Yet come a little,---
	Wishes were ever fools,---O, come, come, come;
	\[r]They heave \1 aloft to \2\]
	And welcome, welcome! die where thou hast lived:
	Quicken with kissing: had my lips that power,
	Thus would I wear them out. \\

\99	A heavy sight!

\1	I am dying, Egypt, dying:
	Give me some wine, and let me speak a little.

\2	No, let me speak; and let me rail so high,
	That the false housewife Fortune break her wheel,
	Provoked by my offence. \\

\1	One word, sweet queen:
	Of Caesar seek your honour, with your safety. O!

\2	They do not go together. \\

\1	Gentle, hear me:
	None about Caesar trust but Proculeius.

\2	My resolution and my hands I'll trust;
	None about Caesar.

\1	The miserable change now at my end
	Lament nor sorrow at; but please your thoughts
	In feeding them with those my former fortunes
	Wherein I lived, the greatest prince o' the world,
	The noblest; and do now not basely die,
	Not cowardly put off my helmet to
	My countryman,---a Roman by a Roman
	Valiantly vanquish'd. Now my spirit is going;
	I can no more. \\

\2	                  Noblest of men, woo't die?
	Hast thou no care of me? shall I abide
	In this dull world, which in thy absence is
	No better than a sty? O, see, my women, 
	The crown o' the earth doth melt. \\ \[r]\1 dies.\]
                                        My lord!
	O, wither'd is the garland of the war,
	The soldier's pole is fall'n: young boys and girls
	Are level now with men; the odds is gone,
	And there is nothing left remarkable
	Beneath the visiting moon. \\  \[r]Faints.\]

\7	O, quietness, lady!

\16	She is dead too, our sovereign. \\

\7	Lady! \\

\16	Madam!

\7	O madam, madam, madam! \\

\16	Royal Egypt,
        Empress!

\7	Peace, peace, Iras!

\2	No more, but e'en a woman, and commanded
	By such poor passion as the maid that milks
	And does the meanest chares. It were for me
	To throw my sceptre at the injurious gods;
	To tell them that this world did equal theirs
	Till they had stol'n our jewel. All's but naught;
	Patience is scottish, and impatience does
	Become a dog that's mad: then is it sin
	To rush into the secret house of death,
	Ere death dare come to us? How do you, women?
	What, what! good cheer! Why, how now, Charmian!
	My noble girls! Ah, women, women, look,
	Our lamp is spent, it's out! Good sirs, take heart:
	We'll bury him; and then, what's brave, what's noble,
	Let's do it after the high Roman fashion,
	And make death proud to take us. Come, away:
	This case of that huge spirit now is cold:
	Ah, women, women! come; we have no friend
	But resolution, and the briefest end.
	\[r]Exeunt; those above bearing off \1's body.\]

\Act



\Scene{Alexandria. Caesar's camp.}


	\(Enter \3, \4, \11, \19, \15,\\ \25, and others, his council of war.\)

\3	Go to him, Dolabella, bid him yield;
	Being so frustrate, tell him he mocks
	The pauses that he makes. \\

\11	Caesar, I shall.  \[r]Exit.\]

	\(Enter \8, with the sword of Antony.\)

\3	Wherefore is that? and what art thou that darest
	Appear thus to us? \\

\8	                  I am call'd Dercetas;
	Mark Antony I served, who best was worthy
	Best to be served: whilst he stood up and spoke,
	He was my master; and I wore my life
	To spend upon his haters. If thou please
	To take me to thee, as I was to him
	I'll be to Caesar; if thou pleasest not,
	I yield thee up my life. \\

\3	What is't thou say'st?

\8	I say, O Caesar, Antony is dead.

\3	The breaking of so great a thing should make
	A greater crack: the round world
	Should have shook lions into civil streets,
	And citizens to their dens: the death of Antony
	Is not a single doom; in the name lay
	A moiety of the world. \\

\8	He is dead, Caesar:
	Not by a public minister of justice,
	Nor by a hired knife; but that self hand,
	Which writ his honour in the acts it did,
	Hath, with the courage which the heart did lend it,
	Splitted the heart. This is his sword;
	I robb'd his wound of it; behold it stain'd
	With his most noble blood. \\

\3	Look you sad, friends?
	The gods rebuke me, but it is tidings
	To wash the eyes of kings. \\

\4	And strange it is,
	That nature must compel us to lament
	Our most persisted deeds. \\

\19	His taints and honours
	Waged equal with him. \\

\4	A rarer spirit never
	Did steer humanity: but you, gods, will give us
	Some faults to make us men. Caesar is touch'd.

\19	When such a spacious mirror's set before him,
	He needs must see himself. \\

\3	O Antony!
	I have follow'd thee to this; but we do lance
	Diseases in our bodies: I must perforce
	Have shown to thee such a declining day,
	Or look on thine; we could not stall together
	In the whole world: but yet let me lament,
	With tears as sovereign as the blood of hearts,
	That thou, my brother, my competitor
	In top of all design, my mate in empire,
	Friend and companion in the front of war,
	The arm of mine own body, and the heart
	Where mine his thoughts did kindle,---that our stars,
	Unreconciliable, should divide
	Our equalness to this. Hear me, good friends---
	But I will tell you at some meeter season:
	The business of this man looks out of him;
	We'll hear him what he says. \\
	\(Enter an Egyptian.\)
   Whence are you?

\12	A poor Egyptian yet. The queen my mistress,
	Confined in all she has, her monument,
	Of thy intents desires instruction,
	That she preparedly may frame herself
	To the way she's forced to. \\

\3	Bid her have good heart:
	She soon shall know of us, by some of ours,
	How honourable and how kindly we
	Determine for her; for Caesar cannot live
	To be ungentle. \\

\12	So the gods preserve thee!  \[r]Exit.\]

\3	Come hither, Proculeius. Go and say,
	We purpose her no shame: give her what comforts
	The quality of her passion shall require,
	Lest, in her greatness, by some mortal stroke
	She do defeat us; for her life in Rome
	Would be eternal in our triumph: go,
	And with your speediest bring us what she says,
	And how you find of her. \\

\25	Caesar, I shall. 	\[r]Exit.\]

\3	Gallus, go you along. \[Exit \15\] Where's Dolabella,
	To second Proculeius? \\

\99	Dolabella!

\3	Let him alone, for I remember now
	How he's employ'd: he shall in time be ready.
	Go with me to my tent; where you shall see
	How hardly I was drawn into this war;
	How calm and gentle I proceeded still
	In all my writings: go with me, and see
	What I can show in this. \[r]Exeunt.\]



\Scene{Alexandria. A room in the monument.}


	\(Enter \2, \7, and \16\)

\2	My desolation does begin to make
	A better life. 'Tis paltry to be Caesar;
	Not being Fortune, he's but Fortune's knave,
	A minister of her will: and it is great
	To do that thing that ends all other deeds;
	Which shackles accidents and bolts up change;
	Which sleeps, and never palates more the dug,
	The beggar's nurse and Caesar's.


	\(Enter, to the gates of the monument, \25, \15 and Soldiers.\)

\25	Caesar sends greeting to the Queen of Egypt;
	And bids thee study on what fair demands
	Thou mean'st to have him grant thee. \\

\2	What's thy name?

\25	My name is Proculeius. \\

\2	Antony
	Did tell me of you, bade me trust you; but
	I do not greatly care to be deceived,
	That have no use for trusting. If your master
	Would have a queen his beggar, you must tell him,
	That majesty, to keep decorum, must
	No less beg than a kingdom: if he please
	To give me conquer'd Egypt for my son,
	He gives me so much of mine own, as I
	Will kneel to him with thanks. \\

\25	Be of good cheer;
	You're fall'n into a princely hand, fear nothing:
	Make your full reference freely to my lord,
	Who is so full of grace, that it flows over
	On all that need: let me report to him
	Your sweet dependency; and you shall find
	A conqueror that will pray in aid for kindness,
	Where he for grace is kneel'd to. \\

\2	Pray you, tell him
	I am his fortune's vassal, and I send him
	The greatness he has got. I hourly learn
	A doctrine of obedience; and would gladly
	Look him i' the face. \\

\25	This I'll report, dear lady.
	Have comfort, for I know your plight is pitied
	Of him that caused it. 

\15	You see how easily she may be surprised:
	\[To \25 and the Guard.\] Guard her till Caesar come. \[r]Exit.\]

\16	Royal queen!

\7	O Cleopatra! thou art taken, queen:

\2	Quick, quick, good hands. \\	\[r]Drawing a dagger.\]

\25	Hold, worthy lady, hold:
	   \[r]Seizes and disarms her.\]
	Do not yourself such wrong, who are in this
	Relieved, but not betray'd. \\

\2	What, of death too,
	That rids our dogs of languish? \\

\25	Cleopatra,
	Do not abuse my master's bounty by
	The undoing of yourself: let the world see
	His nobleness well acted, which your death
	Will never let come forth. \\

\2	Where art thou, death?
	Come hither, come! come, come, and take a queen
	Worthy many babes and beggars! \\

\25	O, temperance, lady!

\2	Sir, I will eat no meat, I'll not drink, sir;
	If idle talk will once be necessary,
	I'll not sleep neither: this mortal house I'll ruin,
	Do Caesar what he can. Know, sir, that I
	Will not wait pinion'd at your master's court;
	Nor once be chastised with the sober eye
	Of dull Octavia. Shall they hoist me up
	And show me to the shouting varletry
	Of censuring Rome? Rather a ditch in Egypt
	Be gentle grave unto me! rather on Nilus' mud
	Lay me stark naked, and let the water-flies
	Blow me into abhorring! rather make
	My country's high pyramides my gibbet,
	And hang me up in chains! \\

\25	You do extend
	These thoughts of horror further than you shall
	Find cause in Caesar. \\

	\(Enter \11\)

\11	Proculeius,
	What thou hast done thy master Caesar knows,
	And he hath sent for thee: for the queen,
	I'll take her to my guard. \\

\25	So, Dolabella,
	It shall content me best: be gentle to her.
	\[To \2\] To Caesar I will speak what you shall please,
	If you'll employ me to him. \\

\2	Say, I would die. 	\[r]Exeunt \25 and Soldiers.\]

\11	Most noble empress, you have heard of me?

\2	I cannot tell. \\

\11	                  Assuredly you know me.

\2	No matter, sir, what I have heard or known.
	You laugh when boys or women tell their dreams;
	Is't not your trick? \\

\11	I understand not, madam.

\2	I dream'd there was an Emperor Antony:
	O, such another sleep, that I might see
	But such another man! \\

\11	If it might please ye,---

\2	His face was as the heavens; and therein stuck
	A sun and moon, which kept their course, and lighted
	The little O, the earth. \\

\11	Most sovereign creature,---

\2	His legs bestrid the ocean: his rear'd arm
	Crested the world: his voice was propertied
	As all the tuned spheres, and that to friends;
	But when he meant to quail and shake the orb,
	He was as rattling thunder. For his bounty,
	There was no winter in't; an autumn 'twas
	That grew the more by reaping: his delights
	Were dolphin-like; they show'd his back above
	The element they lived in: in his livery
	Walk'd crowns and crownets; realms and islands were
	As plates dropp'd from his pocket. \\

\11	Cleopatra!

\2	Think you there was, or might be, such a man
	As this I dream'd of? \\

\11	Gentle madam, no.

\2	You lie, up to the hearing of the gods.
	But, if there be, or ever were, one such,
	It's past the size of dreaming: nature wants stuff
	To vie strange forms with fancy; yet, to imagine
	And Antony, were nature's piece 'gainst fancy,
	Condemning shadows quite. \\

\11	Hear me, good madam.
	Your loss is as yourself, great; and you bear it
	As answering to the weight: would I might never
	O'ertake pursued success, but I do feel,
	By the rebound of yours, a grief that smites
	My very heart at root. \\

\2	I thank you, sir,
	Know you what Caesar means to do with me?

\11	I am loath to tell you what I would you knew.

\2	Nay, pray you, sir,--- \\

\11	Though he be honourable,---

\2	He'll lead me, then, in triumph?

\11	Madam, he will; I know't.

        \[c]Flourish, and shout within,\\
        ``\textup{Make way there: Octavius Caesar!}''\]

	\(Enter \3, \15, \25, \19, \27, and others of his Train.\)

\3	Which is the Queen of Egypt?

\11	It is the emperor, madam.  \[r]\2 kneels.\]

\3	Arise, you shall not kneel:
	I pray you, rise; rise, Egypt. \\

\2	Sir, the gods
	Will have it thus; my master and my lord
	I must obey. \\

\3	                  Take to you no hard thoughts:
	The record of what injuries you did us,
	Though written in our flesh, we shall remember
	As things but done by chance. \\

\2	Sole sir o' the world,
	I cannot project mine own cause so well
	To make it clear; but do confess I have
	Been laden with like frailties which before
	Have often shamed our sex. \\

\3	Cleopatra, know,
	We will extenuate rather than enforce:
	If you apply yourself to our intents,
	Which towards you are most gentle, you shall find
	A benefit in this change; but if you seek
	To lay on me a cruelty, by taking
	Antony's course, you shall bereave yourself
	Of my good purposes, and put your children
	To that destruction which I'll guard them from,
	If thereon you rely. I'll take my leave.

\2	And may, through all the world: 'tis yours; and we,
	Your scutcheons and your signs of conquest, shall
	Hang in what place you please. Here, my good lord.

\3	You shall advise me in all for Cleopatra.

\2	This is the brief of money, plate, and jewels,
	I am possess'd of: 'tis exactly valued;
	Not petty things admitted. Where's Seleucus?

	\(Enter \27.\)

\27	Here, madam.

\2	This is my treasurer: let him speak, my lord,
	Upon his peril, that I have reserved
	To myself nothing. Speak the truth, Seleucus.

\27	Madam,
	I had rather seal my lips, than, to my peril,
	Speak that which is not. \\

\2	What have I kept back?

\27	Enough to purchase what you have made known.

\3	Nay, blush not, Cleopatra; I approve
	Your wisdom in the deed. \\

\2	See, Caesar! O, behold,
	How pomp is follow'd! mine will now be yours;
	And, should we shift estates, yours would be mine.
	The ingratitude of this Seleucus does
	Even make me wild: O slave, of no more trust
	Than love that's hired! What, goest thou back? thou shalt
	Go back, I warrant thee; but I'll catch thine eyes,
	Though they had wings: slave, soulless villain, dog!
	O rarely base! \\

\3	                  Good queen, let us entreat you.

\2	O Caesar, what a wounding shame is this,
	That thou, vouchsafing here to visit me,
	Doing the honour of thy lordliness
	To one so meek, that mine own servant should
	Parcel the sum of my disgraces by
	Addition of his envy! Say, good Caesar,
	That I some lady trifles have reserved,
	Immoment toys, things of such dignity
	As we greet modern friends withal; and say,
	Some nobler token I have kept apart
	For Livia and Octavia, to induce
	Their mediation; must I be unfolded
	With one that I have bred? The gods! it smites me
	Beneath the fall I have. \[To \27.\] Prithee, go hence;
	Or I shall show the cinders of my spirits
	Through the ashes of my chance: wert thou a man,
	Thou wouldst have mercy on me. \\

\3	Forbear, Seleucus.

	\[r]Exit \27\]

\2	Be it known, that we, the greatest, are misthought
	For things that others do; and, when we fall,
	We answer others' merits in our name,
	Are therefore to be pitied. \\

\3	Cleopatra,
	Not what you have reserved, nor what acknowledged,
	Put we i' the roll of conquest: still be't yours,
	Bestow it at your pleasure; and believe,
	Caesar's no merchant, to make prize with you
	Of things that merchants sold. Therefore be cheer'd;
	Make not your thoughts your prisons: no, dear queen;
	For we intend so to dispose you as
	Yourself shall give us counsel. Feed, and sleep:
	Our care and pity is so much upon you,
	That we remain your friend; and so, adieu.

\2	My master, and my lord! \\

\3	Not so. Adieu. 	\[r]Flourish. Exeunt \3 and his train.\]

\2	He words me, girls, he words me, that I should not
	Be noble to myself: but, hark thee, Charmian.
	\[r]Whispers \7\]

\16	Finish, good lady; the bright day is done,
	And we are for the dark. \\

\2	Hie thee again:
	I have spoke already, and it is provided;
	Go put it to the haste. \\

\7	Madam, I will.

	\(Re-enter \11\)

\11	Where is the queen? \\

\7	Behold, sir. \\ 	\[r]Exit.\]

\2	Dolabella!

\11	Madam, as thereto sworn by your command,
	Which my love makes religion to obey,
	I tell you this: Caesar through Syria
	Intends his journey; and within three days
	You with your children will he send before:
	Make your best use of this: I have perform'd
	Your pleasure and my promise. \\

\2	Dolabella,
	I shall remain your debtor. \\

\11	I your servant,
	Adieu, good queen; I must attend on Caesar.

\2	Farewell, and thanks. \\ \[r]Exit \11\]
		Now, Iras, what think'st thou?
	Thou, an Egyptian puppet, shalt be shown
	In Rome, as well as I	mechanic slaves
	With greasy aprons, rules, and hammers, shall
	Uplift us to the view; in their thick breaths,
	Rank of gross diet, shall be enclouded,
	And forced to drink their vapour. \\

\16	The gods forbid!

\2	Nay, 'tis most certain, Iras: saucy lictors
	Will catch at us, like strumpets; and scald rhymers
	Ballad us out o' tune: the quick comedians
	Extemporally will stage us, and present
	Our Alexandrian revels; Antony
	Shall be brought drunken forth, and I shall see
	Some squeaking Cleopatra boy my greatness
	I' the posture of a whore. \\

\16	O the good gods!

\2	Nay, that's certain.

\16	I'll never see 't; for, I am sure, my nails
	Are stronger than mine eyes. \\

\2	Why, that's the way
	To fool their preparation, and to conquer
	Their most absurd intents. \\

	\(Re-enter \7\)

		     Now, Charmian!
	Show me, my women, like a queen: go fetch
	My best attires: I am again for Cydnus,
	To meet Mark Antony: sirrah Iras, go.
	Now, noble Charmian, we'll dispatch indeed;
	And, when thou hast done this chare, I'll give thee leave
	To play till doomsday. Bring our crown and all.
	\[r]Exit \16. A noise within.\]
	Wherefore's this noise? \\

	\(Enter a Guardsman.\)

\56	Here is a rural fellow
	That will not be denied your highness presence:
	He brings you figs.

\2	Let him come in. \\	\[r]Exit Guardsman.\]
	What poor an instrument
	May do a noble deed! he brings me liberty.
	My resolution's placed, and I have nothing
	Of woman in me: now from head to foot
	I am marble-constant; now the fleeting moon
	No planet is of mine. \\

	\(Re-enter Guardsman, with Clown bringing in a basket.\)

\56	This is the man.

\2	Avoid, and leave him. \[r]Exit Guardsman.\]

	Hast thou the pretty worm of Nilus there,
	That kills and pains not?

\begin{PROSE}

\55	Truly, I have him: but I would not be the party
	that should desire you to touch him, for his biting
	is immortal; those that do die of it do seldom or
	never recover.

\2	Rememberest thou any that have died on't?

\55	Very many, men and women too. I heard of one of
	them no longer than yesterday: a very honest woman,
	but something given to lie; as a woman should not
	do, but in the way of honesty: how she died of the
	biting of it, what pain she felt: truly, she makes
	a very good report o' the worm; but he that will
	believe all that they say, shall never be saved by
	half that they do: but this is most fallible, the
	worm's an odd worm.

\2	Get thee hence; farewell.

\55	I wish you all joy of the worm.
   \[r]Setting down his basket.\]

\2	Farewell.

\55	You must think this, look you, that the worm will
	do his kind.

\2	Ay, ay; farewell.

\55	Look you, the worm is not to be trusted but in the
	keeping of wise people; for, indeed, there is no
	goodness in worm.


\2	Take thou no care; it shall be heeded.

\55	Very good. Give it nothing, I pray you, for it is
	not worth the feeding.

\2	Will it eat me?

\55	You must not think I am so simple but I know the
	devil himself will not eat a woman: I know that a
	woman is a dish for the gods, if the devil dress her
	not. But, truly, these same whoreson devils do the
	gods great harm in their women; for in every ten
	that they make, the devils mar five.

\2	Well, get thee gone; farewell.

\55	Yes, forsooth: I wish you joy o' the worm. \[r]Exit.\]

\end{PROSE}

	\(Re-enter \16 with a robe, crown and other jewels.\)

\2	Give me my robe, put on my crown; I have
	Immortal longings in me: now no more
	The juice of Egypt's grape shall moist this lip:
	Yare, yare, good Iras; quick. Methinks I hear
	Antony call; I see him rouse himself
	To praise my noble act; I hear him mock
	The luck of Caesar, which the gods give men
	To excuse their after wrath: husband, I come:
	Now to that name my courage prove my title!
	I am fire and air; my other elements
	I give to baser life. So; have you done?
	Come then, and take the last warmth of my lips.
	Farewell, kind Charmian; Iras, long farewell.
	\[r]Kisses them. \16 falls and dies.\]
	Have I the aspic in my lips? Dost fall?
	If thou and nature can so gently part,
	The stroke of death is as a lover's pinch,
	Which hurts, and is desired. Dost thou lie still?
	If thus thou vanishest, thou tell'st the world
	It is not worth leave-taking.

\7	Dissolve, thick cloud, and rain; that I may say,
	The gods themselves do weep! \\

\2	This proves me base:
	If she first meet the curled Antony,
	He'll make demand of her, and spend that kiss
	Which is my heaven to have. Come, thou mortal wretch,
	\[To an asp, which she applies to her breast.\] %
        With thy sharp teeth this knot intrinsicate
	Of life at once untie: poor venomous fool
	Be angry, and dispatch. O, couldst thou speak,
	That I might hear thee call great Caesar ass
	Unpolicied! \\

\7	          O eastern star! \\

\2	Peace, peace!
	Dost thou not see my baby at my breast,
	That sucks the nurse asleep? \\

\7	O, break! O, break!

\2	As sweet as balm, as soft as air, as gentle,---
	O Antony!---Nay, I will take thee too.
	\[r]Applying another asp to her arm.\]
	What should I stay---  \[r]Dies.\]

\7	In this vile world? So, fare thee well.
	Now boast thee, death, in thy possession lies
	A lass unparallel'd. Downy windows, close;
	And golden Phoebus never be beheld
	Of eyes again so royal! Your crown's awry;
	I'll mend it, and then play.

	\(Enter the Guard, rushing in.\)

\57	Where is the queen? \\

\7	Speak softly, wake her not.

\57	Caesar hath sent--- \\

\7	                  Too slow a messenger.
	\[r]Applies an asp.\]
	O, come apace, dispatch! I partly feel thee.

\57	Approach, ho! All's not well: Caesar's beguiled.

\58	There's Dolabella sent from Caesar; call him.

\57	What work is here! Charmian, is this well done?

\7	It is well done, and fitting for a princess
	Descended of so many royal kings.
	Ah, soldier! \[r]Dies.\]

	\(Re-enter \11\)

\11	How goes it here? \\

\58

\11	Caesar, thy thoughts
	Touch their effects in this: thyself art coming
	To see perform'd the dreaded act which thou
	So sought'st to hinder.
	\[r]Within  `\textup{A way there, a way for Caesar!}'\]

	\(Re-enter \3 and all his train marching.\)

\11	O sir, you are too sure an augurer;
	That you did fear is done. \\

\3	Bravest at the last,
	She levell'd at our purposes, and, being royal,
	Took her own way. The manner of their deaths?
	I do not see them bleed. \\

\11	Who was last with them?

\57	A simple countryman, that brought her figs:
	This was his basket. \\

\3	Poison'd, then. \\

\57	O Caesar,
	This Charmian lived but now; she stood and spake:
	I found her trimming up the diadem
	On her dead mistress; tremblingly she stood
	And on the sudden dropp'd. \\

\3	O noble weakness!
	If they had swallow'd poison, 'twould appear
	By external swelling: but she looks like sleep,
	As she would catch another Antony
	In her strong toil of grace. \\

\11	Here, on her breast,
	There is a vent of blood and something blown:
	The like is on her arm.

\57	This is an aspic's trail: and these fig-leaves
	Have slime upon them, such as the aspic leaves
	Upon the caves of Nile. \\

\3	Most probable
	That so she died; for her physician tells me
	She hath pursued conclusions infinite
	Of easy ways to die. Take up her bed;
	And bear her women from the monument:
        She shall be buried by her Antony: 
	No grave upon the earth shall clip in it
	A pair so famous. High events as these
	Strike those that make them; and their story is
	No less in pity than his glory which
	Brought them to be lamented. Our army shall
	In solemn show attend this funeral;
	And then to Rome. Come, Dolabella, see
	High order in this great solemnity. \[r]Exeunt.\]

\endVersus
\endDrama

%%%%%%%%%%%%%%%%%%%%%%%%%%%%%%%%%%%%%%%%%%%%%%%%%%%%%%%%%%%%%%%%%
%%%%%%%%%%%%%%%%%%%%%%%%%%%%%%%%%%%%%%%%%%%%%%%%%%%%%%%%%%%%%%%%%
%%%%%%%%%%%%%%%%%%%%%%%%%%%%%%%%%%%%%%%%%%%%%%%%%%%%%%%%%%%%%%%%%

\BackMatter {Antony and Cleopatra}

\chapter {Appendices}

\section{`An arm-gaunt steed'}

\noindent\lorem [2314]

\section{Mislineation}

\noindent\lorem [14312]

\section{Punctuation}

\noindent\lorem [22315]

\section{Scenes \textsc{iv}.xv and \textsc{v}.ii}

\noindent\lorem [331]

\newpage

\newcommand {\GlossWidth}{.66in}
\newcommand {\GlossSep}  {10pt}


\Locus  \textus   {\leftmargin \\ \leftmargin + \GlossWidth + \GlossSep}
\Modus            {\measure {\textwidth - (\GlossWidth + \GlossSep)}}

\Novus \textus \gloss
\Locus         {\textrightmargin + \GlossSep \\ \leftmargin }
\Facies        {\RelSize{-3}#1}
\Modus         {\rangedleft{\GlossWidth} \milestone}

\Facies \commissura {\llap{\P\kern 5em}}

\numerus {}

\section{Extracts from North's \textit{Plutarch}}

\prosa 

%LXVI. 
\noindent When the skirmish began, and that they came to join, there was no
great hurt at the first meeting, neither did the ships vehemently hit
one against the other, as they do commonly in fight by sea. For on the
other side Antonius' ships, for their heaviness, could not have the
strength and swiftness to make their blows of any force: and Caesar's
ships on the other side took great heed not to rush and shock with
the forecastles of Antonius' ships, whose prows were armed with great
brazen spurs. Furthermore they durst not flank them, because their
points were easily broken, which way soever they came to set upon his
ships, that were made of great main square pieces of timber, bound
together with great iron pins: so that the battle was much like unto a
battle by land, or to speak more properly, to the assault of a city.
For there were always three or four of Caesar's ships about one of
Antonius' ships, and the soldiers fought with their pikes, halbards
and darts, and threw halbards and darts with fire. Antonius' ships on
the other side bestowed among them, with their crossbows and engines
of battery, great store of shot from their high towers of wood that
were set upon their ships. Now Publicola seeing Agrippa put forth his
left wing of Caesar's army, to compass in Antonius' ships that fought,
he was driven also to loof off to have more room, and to go a little
at one side, to put those farther off that were afraid, and in the
midst of the battle, for they were sore distressed by Arruntius.
Howbeit the battle was yet of even hand, and
the victory doubtful, being indifferent to both: when suddenly they
saw the threescore ships of 
\gloss{Cleopatra flieth.} 
Cleopatra busily about their yard-masts,
and hoising sail to fly. So they fled through the middest of them that
were in fight, for they had been placed behind the great ships, and
did marvellously disorder the other ships. For the enemies themselves
wondered much to see them sail in that sort, with full sail towards
Peloponnesus.
There \gloss{The soul of a lover liveth in another body.} Antonius
shewed plainly, that he had not only lost the courage and heart of an
emperor, but also of a valiant man; and that he was not his own man
(proving that true which an old man spake in mirth, that the soul of
a lover lived in another body, and not in his own); he was so carried
away with the vain love of this woman, as if he had been glued unto
her, and that she could not have removed without moving of him also.
For \gloss{Antonius flieth after Cleopatra.} when he saw Cleopatra's
ship under sail, he forgot, forsook, and betrayed them that fought for
him, and imbarked upon a galley with five banks of oars, to follow her
that had already begun to overthrow him, and would in the end be his
utter destruction.
%LXVII.
When she knew his galley afar off, she lift up a sign in the poop
of her ship; and so Antonius, coming to it, was plucked up where Cleopatra
was: howbeit he saw her not at his first coming, nor she him, but went and sat
down alone in the prow of his ship, and said never a word, clapping his head
between both his hands. In the meantime came certain light brigantines of
Caesar's, that followed him hard. So Antonius straight turned the prow of his
ship, and presently put the rest to flight, saving one Eurycles a
Lacedaemonian, that followed him near, and pressed upon him with great
courage, shaking a dart in his hand over the prow, as though he would have
thrown it unto Antonius. Antonius seeing him, came to the forecastle of his
ship, and asked him what he was that durst follow Antonius so near?
``I am,'' answered he, ``Eurycles the son of Lachares, who through Caesar's good fortune
seeketh to revenge the death of my father.'' This Lachares was condemned of
felony, beheaded by Antonius. But yet Eurycles durst not venture upon
Antonius' ship, but set upon the other admiral galley (for there were two),
and fell upon him with such a blow of his brazen spur that was so heavy and
big, that he turned her round, and took her, with another that was loden
with very rich stuff and carriage. After Eurycles had left Antonius, he
turned again to his place, and sat down, speaking never a word, as he did
before: and so lived three days alone, without speaking to any man. But when
he arrived at the head of Taenarus, there Cleopatra's women first brought
Antonius and Cleopatra to speak together, and afterwards to sup and lie
together. Then began there again a great number of merchants' ships to gather
about them, and some of their friends that had escaped from this overthrow,
who brought news' that his army by sea was overthrown, but that they thought
the army by land was yet whole. Then Antonius sent unto Canidius, to return
with his army into Asia by 
\gloss{Antonius licenceth his\\ friends to\\ depart, and giveth
them\\ a shippe\\ loaden with\\ gold and\\ silver.} Macedon. 
Now for himself, he determined to cross over into Af\-ri\-ca, and took
one of his carects or hulks loden with gold and silver, and other rich
carriage and gave it unto his friends, commanding them to depart, and
seek to save themselves. They answered him weeping, that they would
neither do it, nor yet forsake him. Then Antonius very courteously
and lovingly did comfort them, and prayed them to depart; and wrote
unto Theophilus, governor of Corinth, that he would see them safe, and
help to hide them in some secret place, until they had made their way
and peace with Caesar. This Theophilus was the father of Hipparchus,
who was had in great estimation about Antonius. He was the first of
all his enfranchised bondmen that revolted from him, and yielded unto
Caesar, and afterwards went and dwelt at Corinth.
%LXVIII.
And thus it stood with Antonius. Now \gloss{Antonius' navy overthrown
by Caesar.} for his army by sea, that fought before the head or
foreland of Actium, they held out a long time, and nothing troubled
them more than a great boisterous wind that rose full in the prows of
their ships, and yet with much ado his navy was at length overthrown,
five hours within night. There were not slain above five thousand
men: but yet there were three hundred ships taken, as Octavius Caesar
writeth himself in his Commentaries. Many plainly saw Antonius fly,
and yet could very hardly believe it, that he, that had nineteen
legions whole by land, and twelve thousand horsemen upon the sea-side,
would so have forsaken them, and have fled so cowardly, as it he
had not oftentimes proved both the one and the other fortune, and
that he had not been thoroughly acquainted with the diverse changes
and fortunes of battles. And yet his soldiers still wished for him,
and ever hoped that he would come by some means or other unto them.
Furthermore, they shewed themselves so valiant and faithful unto him,
that after they certainly knew he was fled, they kept themselves whole
together seven days.
In the end Canidius, Antonius' lieutenant, flying by night, 
\gloss{Antonius' legions do yield themselves unto Octavius Caesar.}
and forsaking his camp, when they saw themselves thus destitute of their heads
and leaders, they yielded themselves unto the stronger. This done,
Caesar sailed towards Athens, and there made peace with the Grecians,
and divided the rest of the corn that was taken up for Antonius'
army, unto the towns and cities of Greece, the which had been brought
to extreme misery and poverty, clean without money, slaves, horse,
and other beasts of carriage. So that my grandfather Nicarchus told
that all the citizens of our city of Chaer\-o\-ne\-a (not one excepted)
were driven themselves to carry a certain measure of corn on their
shoulders to the sea-side, that lieth directly over against the ile of
Anticyra, and yet were they driven thither with whips. They carried
it thus but once: for the second time that they were charged again to
make the like carriage, all the corn being ready to be carried, news
came that Antonius had lost the battle, and so scaped our poor city.
For Antonius' soldiers and deputies fled immediately, and the citizens
divided the corn amongst them. 
%LXIX.
Antonius being arrived in Lybia, he sent Cleopatra before into Egypt
from the city of Paraetonium; and he himself remained very solitary,
having only two of his friends with him, with whom he wandered up and
down, both of them orators, the one Aristocrates a Grecian, and the
other Lucilius a Roman: of whom we have written in another place,
\gloss{Lucilius spoken of in Brutus' life.} that, at the battle where
Brutus was overthrown by the city of Philippes, he came and willingly
put himself into the hands of those that followed Brutus, saying that
it was he: because Brutus in the meantime might have liberty to save
himself.
And afterwards, because Antonius saved his life, he still remained with
him, and was very faithful \gloss{The fidelity of Lucilius unto Antonius.} 
and friend\-ly unto him till his death. But when Antonius
heard that he whom he had trusted with the government of Lybia, and
unto whom he had given the charge of his army there, had yielded
unto Caesar, he was so mad withal, that he would have slain himself
for anger, had not his friends about him withstood him, and kept him
from it. So \gloss {The wonderful attempt of Cleopatra.} he went unto
Alexandria, and there found Cleopatra about a wonderful enterprise,
and of great attempt. Betwixt the Red Sea and the sea between the
lands that point upon the coast of Egypt, there is a little piece of
land that divideth both the seas, and separateth Africk from Asia:
the which streight is so narrow at the end where the two seas are
narrowest, that it is not above three hundred furlongs over. Cleopatra
went about to lift her ships out of the one sea, and to hale them over
the bank into the other sea: that when her ships were come into the
gulf of Arabia, she might then carry all her gold and silver away, and
so with a great company of men go and dwell in some place about the
Ocean Sea, far from the sea Mediterraneum, to escape the danger and
bondage of this war. But now, because the Arabians dwelling about the
city of Petra, did burn the first ships that were brought to land, and
that Antonius thought that his army by land which he left at Actium
was yet whole, she left off her enterprise, and determined to keep
all the ports and passages of her realm. Antonius,
he forsook the city and company of his friends, and built him a house
in the sea by the ile of Pharos, upon certain forced mounts which he
caused to be cast into the sea, and dwelt there as a man that banished
himself from all men's company: saying that he would lead Timon's
life, \gloss{Antonius followeth the life and example of Timon Misanthropos
the Athenian.} because he had the like wrong offered him, that was before
offered unto Timon: and that for the unthankfulness of those he had
done good unto, and whom he took to be his friends, he was angry with
all men and would trust no man.
% LXX.
This Timon was a citizen of Athens, that lived about the war of
Peloponnesus, as appeareth by Plato \gloss{Plato and Aristophanes'
testimony of Timon Misanthropos, what he was.} and Aristophanes'
comedies: in the which they mocked him, calling him a viper and
malicious man unto mankind, to shun all other men's companies but the
company of young Alcibiades, a bold and insolent youth, whom he would
greatly feast and make much of, and kissed him very gladly. Apemantus
wondering at it, asked him the cause what he meant to make so much
of that young man alone, and to hate all others: Timon answered
him, ``I do it,'' said he, ``because I know that one day he shall do
great mischief unto the Athenians.'' This Timon sometimes would have
Apemantus in his company, because he was much like of his nature and
conditions, and also followed him in manner of life. On a time when
they solemnly celebrated the feast called Choe at Athens (to wit,
the feasts of the dead where they make sprinklings and sacrifices
for the dead) and that they two then feasted together by themselves,
Apemantus said unto the other: ``O, here is a trim banquet, Timon!''
Timon answered again: ``Yea,'' said he, ``so thou wert not here.'' It
is reported of him also, that this Timon on a time (the people being
assembled in the marI;et-place about dispatch of some affairs) got up
into the pulpit for orations, where the orators commonly use to speak
unto the people: and silence being made, every man listening to hear
what he would say, because it was a wonder to see him in that place,
at length he began to speak in this manner: ``My lords of Athens, I
have a little yard at my house where there groweth a fig-tree, on the
which many citizens have hanged themselves: and because I mean to make
some building on the place, I thought good to let you all understand
it, that, before the figtree be cut down, if any of you be desperate,
you may there in time go hang yourselves.'' He died in the city of
Hales, and was buried upon the sea-side. Now it chanced so, that
the sea getting in, it compassed his tomb round about, that no man
could come to it: and upon the same was written this epitaph: Here
\gloss{The epitaph of Timon Misanthropos.} dies a wretched corse, of
wretched soul bereft: Seek not my name: a plague consume you wicked
wretches left! It is reported that Timon himself, when he lived, made
this epitaph: for that which is commonly rehearsed was not his, but
made by the poet Callimachus: Here lie I, Timon, who alive all living
men did hate: Pass by and curse thy fill: but pass, and stay not here
thy gate.
%LXXI. 
Many other things could we tell you of this Timon, but this Little
shall suffice at this present. But now to return to Antonius again.
Canidius himself came to bring him news, that he had lost all his
army by land at Actium: on the other side he was advertised also,
that Herodes king of Jurie, who had also certain legions and bands
with him, was revolted unto Caesar, and all the other kings in like
manner: so that, saving those that were about him, he had none left
him. All this notwithstanding did nothing trouble him: and it
seemed that he was contented to forgo all his hope, and so to be rid
of all his cares and troubles. Thereupon he left his solitary house
\gloss{Antonius' rioting in Alexandria after his great loss
and overthrow.} he had built by the sea, which he called Timoneon and Cleopatra received
him into her royal palace. He was no sooner come thither, but he
straight set all the city on rioting and banqueting again, and himself
to liberality and gifts. He caused the son of Julius Caesar
\gloss{Toga virilis. Antyllus the eldest son of Antonius by his wife Fulvia.}
and Cleopatra to be enrolled (according to the manner of the Romans)
amongst the number of young men: and gave Antyllus, his eldest son he
had by Fulvia, the man's gown, the which was a plain gown without gard
or embroderie, of purple. For these things, there was kept great feasting, banqueting
and dancing in Alexandria many days together. Indeed they did break
their first order \gloss {An order erected by Antonius
and Cleopatra, called Synapothanumenon, revoking the former called
Amimetobion.} they had set down, which they called Amimetobion (as
much to say, 'no life comparable'), and did set up another, which they
called Synapothanumenon (signifying the order and agreement of those
that will die together), the which in exceeding sumptuousness and cost
was not inferior to the first. For their friends made themselves to be
enrolled in this order of those that would die together, and so made
great feasts one to another: for every man, when it came to his turn,
feasted their whole company and fraternity.
Cleopatra \gloss {Cleopatra very busy in proving the force of poison.}
in the meantime was very careful in gathering all sorts of poisons
together, to destroy men. Now to make proof of those poisons which
made men die with least pain, she tried it upon condemned men in
prison. For when she saw the poisons that were sudden and vehement,
and brought speedy death with grievous torments; and in contrary
manner, that such as were more mild and gentle had not that quick
speed and force to make one die suddenly: she afterwards went about to
prove the stinging of snakes and adders, and made some to be applied
unto men in her sight, some in one sort, some in another. So
\gloss {The property of the biting of an aspick.} when she had daily made
divers and sundry proofs, she found none of them all she had proved so
fit as the biting of an aspick, the which causeth only a heaviness of
the head, without swooning or complaining, and bringeth a great desire
also to sleep, with a little sweat in the face; and so by little and
little taketh away the senses and vital powers, no living creature
perceiving that the patients feel any pain. For they are so sorry when
any body awaketh them and taketh them up, as those that be taken out
of a sound sleep are very heavy and desirous to sleep
% LXXII. 
This notwithstanding, they sent ambassadors unto Octavius Caesar
in Asia, Cleopatra requesting the realm of Egypt for their children,
and Antonius 
\gloss {Antonius and Cleopatra send ambassadors unto Octavius
Caesar.}praying that he might be suffered to live at Athens
like a private man, if Caesar would not let him remain in Egypt. And
because they had no other men of estimation about them, for that
some were fled, and those that remained they did not greatly trust,
they were enforced to send Euphronius, the schoolmaster of their
children. For Alexas Laodicean, who was brought into Antonius' house
and favour by means of Timagenes, and afterwards was in greater credit
with him than any other Grecian (for that he had ever been one of
Cleopatra's ministers to win Antonius, and to overthrow all his good
determinations to use his wife Octavia well?: him Antonius had sent
unto Herodes king of Jurie, hoping still to keep him his friend, that
he should not revolt from him. But he remained there, and betrayed
Antonius. For where he should have kept Herodes from revolting from
him, he persuaded him to turn to Caesar: and trusting king Herodes,
he presumed to come in Caesar's presence. Howbeit Herodes did him no
pleasure, for he was presently taken prisoner, and sent in chains
\gloss {Alexas' treason justly punished.}
to his own country, and there by Caesar's commandment put to death.
Thus was Alexas, in Antonius' life-time, put to death for betraying of
him.
%LXXIII.
Furthermore, Caesar would not grant unto Antonius' requests: but
for Cleopatra, he made her answer, that he would deny her nothing
reasonable, so that she would either put Antonius to death, or drive
him out of her country. Therewithal he sent Thyreus one of his men
unto her, a very wise and discreet man: who bringing letters of
credit from a young lord unto a noble lady, and that besides greatly
liked her beauty, might easily by his eloquence have persuaded her.
He was longer in talk with her than any man else was, and the queen
herself also did him great honour: insomuch as he made Antonius
jealous of him. Whereupon Antonius caused him to be taken and well
favouredly whipped, and so sent him unto Caesar: and bad him tell
him, that he made him angry with him, because he shewed himself
proud and disdainful towards him; and now specially, when he was
easy to be angered, by reason of his present misery. ``To be short,
if this mislike thee,'' said he, ``thou hast Hipparchus, one of my
enfranchised bondmen, with thee: hang him if thou wilt, or whip him at
thy pleasure, that we may cry quittance.'' From henceforth Cleopatra,
to clear herself of the suspicion he had of her, made more of him than
ever she did. For first of all, where she did solemnize the day of
her birth very meanly and sparingly, fit for her present misfortune,
she now in contrary manner did keep it with such solemnity, that she
exceeded all measure of sumptuousness and magnificence: so that the
guests that were bidden to the feasts, and came poor, went away rich.
Now things passing thus, Agrippa by divers letters sent one after
another unto Caesar, prayed him to return to Rome, because the affairs
there did of necessity require his person and presence.
%LXXIV.
Thereupon he did defer the war till the next year following: but when
winter was done, he returned again through Syria by the coast of
Africa, to make wars a\-gainst Antonius and his other captains. When
the city of Pelusium was taken, \gloss {Pelusium was yielded up to
Octavius Caesar.} there ran a rumour in the city, that Seleucus (by
Cleopatra's consent) had surrendered the same. But to clear herself
that she did not, Cleopatra brought Seleucus' wife and children unto
Antonius, to be revenged of them at his pleasure.
Furthermore, \gloss {Cleopatra's monuments set up by the temple
of Isis.} Cleopatra had long before made many sumptuous tombs and
monuments, as well for excellency of workmanship, as for height and
greatness of building, joining hard to the temple of Isis. Thither she
caused to be brought all the treasure and precious things she had of
the ancient kings her predecessors: as gold, silver, emeralds, pearls,
ebony, ivory, and cinnamon, and besides all that, a marvellous number
of torches, faggots, and flax. So Octavius Caesar, being afraid to
lose such a treasure and mass of riches, and that this woman for spite
would set it on fire and burn it every whit, he always sent some one
or other unto her from him, to put her in good comfort, whilst he in
the meantime drew near the city with his army. So Caesar came and
pitched his camp hard by the city, in the place where they run and
manage their horses. Antonius made a sally upon him, and fought very
valiantly, so that he drave Caesar's horsemen back, fighting with his
men even into their camp. Then he came again to the palace, greatly
boasting of this victory, and sweetly kissed Cleopatra, armed as he
was when he came from the fight, recommending one of his men of arms
unto her, that had valiantly fought in this skirmish Cleopatra, to
reward his manliness, gave him an armour and headpiece of clean gold:
howbeit the man-at-arms, when he had, received this rich gift, stole
away by night and went to Caesar.
%LXXV.
Antonius sent again to challenge Caesar to fight with him hand to hand.
Caesar answered him, ``That he had many other ways to die than so.'' Then
Antonius, seeing there was no way more honourable for him to die than fighting
valiantly, he determined to set up his rest, both by sea and land. So
being at supper (as it is reported) he commanded his officers and household
servants that waited on him at his board, that they should fill his cups full,
and make as much of him as they could: ``For,'' said he, ``you know not whether
you shall do so much for me tomorrow or not, or whether you shall serve
another master: and it may be you shall see me no more, but a dead body.'' This
notwithstanding, perceiving that his friends and men fell a-weeping to hear
him say so, to salve that he had spoken, he added this more unto it, 'that
he would not lead them to battle, where he thought not rather safely to return
with victory, than valiantly to die with honour.' 
Furthermore, the selfsame night, within a little of midnight, when
all the city was quiet, full of fear and sorrow, thinking what would
be the issue and end of this war, it is said that suddenly they heard
a marvellous sweet \gloss {Strange noises heard, and nothing seen.}
harmony of sundry sorts of instruments of music, with the cry of a
multitude of people, as they had been dancing, and had sung as they
use in Bacchus' feasts, with movings and turnings after the manner
of the Satyrs: and it seemed, that this dance went through the city
unto the gate that opened to the enemies, and that all the troupe,
that made this noise they heard, went out of the city at that gate.
Now such as in reason sought the depth of the interpretation of this
wonder, thought that it was the god unto whom Antonius bare singular
devotion to counterfeit and resemble him, that did forsake them.
%LXXVI.
The next morning by break of day, he went to set those few footmen
he had in order upon the hills adjoining unto the city: and there he
stood to behold his galleys which departed from the haven, and rowed
against the galleys of the enemies, and so stood still, looking what
exploits his soldiers in them would do. But when by force of rowing
they were come near unto them, they first saluted Caesar's men; and
then Caesar's men resaluted them also, and of two armies made but one:
and then did all together row toward the city. When \gloss {Antonius'
navy do yeild themselves unto Caesar. Antonius overthrown by Octavius
Caesar. Cleopatra flieth into her tomb or monument.} Antonius saw
that his men did forsake him, and yielded unto Caesar, and that his
footmen were broken and overthrown, he then fled into the city, crying
out that Cleopatra had betrayed him unto them with whom he had made
war for her sake. Then she, being afraid of his fury, fled into the
tomb which he had caused to be made, and there she locked the doors
unto her, and shut all the springs of the locks with great bolts, and
in the meantime sent unto Antonius to tell him that she was dead.
Antonius believing it, said unto himself: ``What doest thou look for
further, Antonius, sith spiteful fortune hath taken from thee the
only joy thou hadst, for whom thou yet reservedst thy life ?'' When
he had said these words, he went into a chamber and unarmed himself,
and being naked, said thus: ``O Cleopatra, it grieveth me not that I
have lost thy company, for I will not be long from thee: but I am
sorry that, having been so great a captain and emperor, I am indeed
condemned to be judged of less courage and noble mind than a woman
'' Now he had a man of his called Eros, whom he loved and trusted
much, and whom he had long before caused to swear unto him, that he
should kill him when he did command him: and then he willed him to
keep his promise. His man, drawing his sword, lift it up as though
he had meant to have stricken his master: but turning his head at
one side, he thrust his sword into himself, and fell down dead at
his master's foot. \gloss {Eros, Antonius' servant, slew himself.}
Then said Antonius: ``O noble Eros, I thank thee for this, and it is
valiantly done of thee, to shew me what I should do to myself, which
thou couldest not do for me.'' Therewithal \gloss {Antonius did thrust
his sword into himself, but died not presently.} he took his sword,
and thrust it into his belly, and so fell down upon a little bed.
The wound he had killed him not presently, for the blood stinted a
little when he was laid: and when he came somewhat to himself again,
he prayed them that were about him to despatch him. But they all fled
out of the chamber, and left him crying out, tormenting himself: until
at last there came a secretary unto him (called Diomedes) who was
commanded to bring him into the tomb or monument where Cleopatra was.
%LXXVII. 
When he heard that she was alive, he very earnestly prayed his men
to carry his body thither, and so he was carried \gloss {Antonius
carried unto Cleopatra's tomb.} in his men's arms into the entry of
the monument. Notwithstanding, Cleopatra would not open the gates,
but came to the high windows, and cast out certain chains and ropes,
in the which Antonius was trussed: and Cleopatra her own self, with
two women only, which she had suffered to come with her into these
monuments, trised Antonius up. They that were present to behold it
said they
\gloss {A lamentable sight to see Antonius and Cleopatra.}
never saw so pitiful a sight. For they plucked up poor Antonius, all
bloody as he was, and drawing on with pangs of death: who holding up
his hands to Cleopatra, raised up himself as well as he could. It was
a hard thing for these women to do, to lift him up: but Cleopatra,
stooping down with her head, putting to all her strength to her
uttermost power, did lift him up with much ado, and never let go her
hold, with the help of the women beneath that bad her be of good
courage, and were as sorry to see her labour so as she herself. So
when she had gotten him in after that sort, and laid him on a bed, she
rent her garments upon him, clapping her breast, and scratching her
face and stomach. Then she dried up his blood that had bewrayed his
face, and called him her lord, her husband, and emperor, forgetting
her own misery and calamity for the pity and compassion she took of
him. Antonius made her cease her lamenting, and called for wine,
either because he was athirst, or else for that he thought thereby
to hasten his death. When he had drunk, he earnestly prayed her, and
persuaded her, that she would seek to save her life, if she could
possible, without reproach and dishonour: and that chiefly she should
trust Proculeius above any man else about Caesar. And as for himself,
that she should not lament nor sorrow for the miserable change of
his fortune at the end of his days: but rather that she should think
him the more fortunate, for the former triumphs and honours he had
received; considering that while he lived, he was the noblest and
greatest prince of the world; and that now he was overcome, not
cowardly, but valiantly, a Roman by another Roman.
%  LXXVIII.
\gloss {The death of Antonius.} 
As Antonius gave the last gasp, Proculeius came that was sent from
Caesar. For after Antonius had thrust his sword in himself, as they
carried him into the tombs and monuments of Cleopatra, one of his
guard (called Dercetaeus) took his sword with which he had stricken
himself, and hid it: then he secretly stole away, and brought 
\gloss {Octavius Caesar lamenteth Antonius' death.} 
Octavius Caesar the first news of his death, and shewed him his sword that
was bloodied. Caesar hearing this news, straight withdrew himself into a secret
place of his tent, and there burst out with tears, lamenting his hard
and miserable fortune, that had been his friend and brother-in-law,
his equal in the empire, and companion with him in sundry great
exploits and battles. Then he called for all his friends and shewed
them the letters Antonius had written to him, and his answers also
sent him again, during their quarrel and strife: and now fiercely and
proudly the other answered him, to all just and reasonable matters
he wrote unto him. After \gloss {Proculeius sent by Octavius Caesar
to bring Cleopatra alive.} this, he sent Proculeius, and commanded
him to do what he could possible to get Cleopatra alive, fearing lest
otherwise all the treasure would be lost: and furthermore, he thought
that if he could take Cleopatra, and bring her alive to Rome, she
would marvellously beautify and set out his triumph. But Cleopatra
would never put herself into Proculeius' hands, although they spake
together. For Proculeius came to the gates that were thick and strong,
and surely barred, but yet there were some cranewes through the
which her voice might be heard; and so they without understood,
that Cleopatra demanded the kingdom of Egypt for her sons: and that
Proculeius answered her that she should be of good cheer, and not be
afraid to refer all unto Caesar.
% LXXIX. 
After he had viewed the place very well, he came and reported her
answer unto Caesar: who immediately sent Gallus to speak once again with her,
and bad him purposely hold her in talk, whilst Proculeius did set up a ladder
against that high window by the which Antonius was trised up, and came
down into the monument with two of his men, hard by the gate where Cleopatra
stood to hear what Gallus said unto her. One of her women which was shut up in
her monuments with her, saw Proculeius by chance as he came down, and
skreeked out: ``O poor Cleopatra, thou art taken.'' Then when she saw
Proculeius behind her as she came from the gate, she thought to have stabbed
herself in with a short dagger she wore of purpose by her side. 
But \gloss {Cleopatra taken.}  Proculeius came suddenly upon her, and taking her by
both the hands, said unto her: ``Cleopatra, first thou shalt do thyself great
wrong, and secondly unto Caesar, to deprive him of the occasion and
opportunity openly to shew his bounty and mercy, and to give his enemies cause
to accuse the most courteous and noble prince that ever was, and to appeach
him, as though he were a cruel and merciless man, that were not to be
trusted.'' So even as he spake the word, he took her dagger from her, and shook
her clothes for fear of any poison hidden about her. Afterwards, Caesar sent
one of his infranchised men called Epaphroditus, whom he straightly
charged to look well unto her, and to beware in any case that she made not
herself away: and for the rest, to use her with all the courtesy possible. 
% LXXX. 
And \gloss {Caesar took the city of Alexandria.} for himself, he in
the meantime entered the city of Alexandria, and (as he went) talked
with the philosopher Arrius, and held him by the hand, to the end
that his countrymen should reverence him the more, because they saw
Caesar so highly esteem and honour him. Then \gloss {Caesar greatly
honoured Arrius the philosopher.} he went into the show-place of
exercises, and so up to his chair of state which was prepared for
him of a great height: and there, according to his commandment, all
the people of Alexandria were assembled, who, quaking for fear, fell
down on their knees before him and craved mercy. Caesar bad them
all stand up, and told them openly that he forgave the people, and
pardoned the felonies and offences they had committed against him in
this war: first, for the founder's sake of the same city, which was
Alexander the Great: secondly, for the beauty of the city, which he
much esteemed and wondered at: thirdly, for the love he bare unto
his very friend Arrius. Thus \gloss {Philostratus the eloquentest
orator in his time for present speech upon a sudden.} did Caesar
honour Arrius, who craved pardon for himself and many others, and
specially for Philostratus, the eloquentest man of all the sophisters
and orators of his time,for present and sudden speech: howbeit, he
falsely named himself an Academic philosopher. Therefore Caesar, that
hated his nature and conditions, would not hear his suit. Thereupon he
let his grey beard grow long, and followed Arrius step by step in a
long mourning gown, still buzzing in his ears this Greek verse:

\begin{versus}
\Locus \textus {+ 2em}
A wise man, if that he be wise indeed,
May by a wise man have the better speed.
\end{versus}

\noindent  Caesar understanding this, not for the desire he had to deliver Philostratus
of his fear, but to rid Arrius of malice and envy that might have fallen out
against him, he pardoned him. 
% LXXXI. 
Now touching Antonius' sons, Antyllus, 
\gloss {Antyllus, Antonius' eldest son by Fulvia, slain.}  
his eldest son by Fulvia, was
slain, because his schoolmaster The\-o\-do\-rus did betray him unto the soldiers,
who strake off his head. And the villain took a precious stone of great
value from his neck, the which he did sew in his girdle, and afterwards denied
that he had it: but it was found about him, and so Caesar trussed him up
for it. For Cleopatra's children, they were very honourably kept, with
their governors and train that waited on them. But for Caesarion, who was said
to be Julius Caesar's son, his mother Cleopatra had sent him unto the Indians
through Ethiopia, with a great sum of money. But one of his governors also,
called Rhodon, even such another as Theodorus, persuaded him to return into
his country, and told him that Caesar sent for him to give him his mother's
kingdom. 
So, \gloss {The saying of Arrius the philosopher.} as Caesar was
determining with himself what he should do, Arrius said unto him:

\begin{versus}
\Locus \textus +
Too many Caesars is not good,
\end{versus}

\noindent alluding unto a certain verse of Homer, that saith: 

\begin{versus}
\Locus \textus + 
Too many lords cloth not well.
\end{versus}
%  LXXXII.
\noindent Therefore \gloss {Caesarion, Cleopatra's son, put to death.} Caesar
did put Caesarion to death, after the death of his mother Cleopatra.
Many princes, great kings, and captains, did crave Antonius' body of
Octavius Caesar, to give him honourable burial: but Caesar would never
take it from Cleopatra, 
\gloss {Cleopatra burieth Antonius.} 
who did sumptuously and royally bury him with her own hands, whom
Caesar suffered to take as much as she would to bestow upon his
funerals. Now was she altogether overcome with sorrow and passion
of mind, for she had knocked her breast so pitifully, that she
had martyred it, and in divers places had raised ulcers and
inflammations, so that she fell into a fever withal; whereof she was
very glad, hoping thereby to have good colour to abstain from meat,
and that so she might have died easily without any trouble. She \gloss
{Olympus, Cleopatra's physician.} had a physician called Olympus, whom
she made privy to her intent, to the end he should help to rid her
out of her life: as Olympus writeth himself, who wrote a book of all
these things. But Caesar mistrusted the matter by many conjectures
he had, and therefore did put her in fear, and threatened her to put
her children to shameful death. With these threats, Cleopatra for
fear yielded straight, as she would have yielded unto strokes: and
afterwards suffered \gloss {Caesar came to see Cleopatra.}herself to be cured and dieted as they listed.
% LXXXIII. 
Shortly  after, Caesar came
himself in person to see her, and to comfort her. Cleopatra,
being laid upon a little low bed in poor estate (when she saw Caesar
come into her chamber), suddenly rose up,
\gloss {Cleopatra a martyred creature through her own passion and fury.}
naked in her smock, and
fell down at his feet marvellously disfigured: both for that she had
plucked her hair from her head, as also for that she had martyred
all her face with her nails; and besides, her voice was small and
trembling, her eyes sunk into her head with continual blubbering ;
and moreover, they might see the most part of her stomach torn in
sunder. To be short, her body was not much better than her mind:
yet her good grace and comeliness and the force of her beauty was
not altogether defaced. But notwithstanding this ugly and pitiful
state of hers, yet she shewed herself within, by her outward looks
and countenance. When Caesar had made her lie down again, and sat by
her bedside, Cleopatra began to clear and excuse herself for that
she had done, laying all to the fear she had of Antonius: Caesar,
in contrary manner, reproved her in every point. Then she suddenly
altered her speech, and prayed him to pardon her, as though she were
afraid to die, and desirous to live. At length, she gave him a brief
and memorial of all the ready money and treasure she had. But \gloss
{Seleucus,\\ one of Cleopatra's treasurers.} by chance there stood one
Seleucus by, one of her treasurers, who, to seem a good servant, came
straight to Caesar to disprove Cleopatra, that she had not set in all,
but kept many things back of purpose. Cleopatra's words unto Caesar.
Cleopatra was in such a rage with him, that she
flew upon him, and took him by the hair of the head, and boxed
\gloss {Cleopatra beat her treasurer before Octavius Caesar.}
him wellfavouredly. Caesar fell a-laughing and parted the fray. ``Alas,''
said she, ``O Caesar: is not this a great shame and reproach, that thou
having vouchsafed to take the pains to come unto me, and done me this
honour, poor wretch and caitiff creature, brought into this pitiful
and miserable state: and that mine own servants should come now to
accuse me? though it may be I have reserved some jewels and trifles
meet for women, but not for me (poor soul) to set out myself withal,
but meaning to give some pretty presents and gifts unto Octavia and
Livia, that they, making means and intercession for me to thee, thou
mightest yet extend thy favour and mercy upon me.'' Caesar was glad to
hear her say so, persuading himself thereby that she had vet a desire
to save her life. So he made her answer, that he did not only give
her that to dispose of at her pleasure which she had kept back, but
further promised to use her more honourably and bountifully than she
would think for: and so he took his leave of her, supposing he had
deceived her, but indeed he was deceived himself.
% LXXXIV.
There was a young gentleman, Cornelius Dolabella, that was one of
Caesar's very great familiars, and besides did bear no ill will unto
Cleopatra. He \gloss {Cleopatra finely deceiveth Octavius Caesar, as
though she desired to live.} He sent her word secretly (as she had
requested him) that Caesar determined to take his journey through
Syria, and that within three days he would send her away before with
her children. When this was told Cleopatra, she requested Caesar
that it would please him to suffer her to offer the last oblations
of the dead unto the soul of Antonius. Thus
being granted her, she was carried
to the place where his tomb was, and there falling down on her knees,
embracing the tomb with her women, the tears running down her cheeks,
she began to speak
\gloss {Cleopatra's lamentation over Antonius' tomb.}
in this sort: ``O my dear lord Antonius, it is not
long sithence I buried thee here, being a free woman: and now I offer
unto thee the funeral sprinklings and oblations, being a captive and
prisoner; and yet I am forbidden and kept from tearing and murdering
this captive body of mine with blows, which they carefully guard and
keep only to triumph of thee: look therefore henceforth for no other
honours, offerings, nor sacrifices from me: for these are the last
which Cleopatra can give thee, sith now they carry her away. Whilst
we lived together, nothing could sever our companies: but now, at our
death, I fear me they will make us change our countries For as thou,
being a Roman, hast been buried in Egypt: even so, wretched creature,
I, an Egyptian, shall be buried in Italy, which shall be all the good
that I have received by thy country If therefore the gods where thou
art now have any power and authority, sith our gods here have forsaken
us, suffer not thy true friend and lover to be carried away alive,
that in me they triumph of thee: but receive me with thee, and let
me be buried in one self tomb with thee. For though my griefs and
miseries be infinite, yet none hath grieved me more, nor that I could
less bear withal, than this small time which I have been driven to
live alone without thee.''
% LXXXV.
Then having ended these doleful plaints, and crowned the tomb with
garlands and sundry nosegays, and marvellous lovingly embraced the same, she
commanded they should prepare her bath; and when she had bathed and washed
herself, she fell to her meat, and was sumptuously served. Now whilst she was
at dinner, there came a countryman and brought her a basket. The soldiers that
warded at the gates, asked him straight what he had in his basket. He
opened his basket, and took out the leaves that covered the figs, and shewed
them that they were figs he brought. They all of them marvelled to see so
goodly figs. The countryman laughed to hear them, and bade them take some if
they would. They believed he told them truly, and so bade him carry them in.
After Cleopatra had dined, she sent a certain table written and sealed
unto Caesar, and commanded them all to go out of the tombs where she was, but
the two women; then she shut the doors to her. Caesar, when he had received
this table, and began to read her lamentation and petition, requesting him
that he would let her be buried with Antonius, found straight what she meant,
and thought to have gone thither himself: howbeit, he sent one before in all
haste that might be, to see what it was. 
Her \gloss {The death of Cleopatra.}  death was very sudden: for those whom Caesar
sent unto her ran thither in all haste possible, and found the soldiers
standing at the gate, mistrusting nothing, nor understanding of her death. 
But \gloss{Cleopatra's two waiting women dead with her.}  when they had opened the
doors, they found Cleopatra stark-dead, laid upon a bed of gold, attired and
arrayed in her royal robes, and one of her two women, which was called Iras,
dead at her feet: and her other woman (called Charmion) half dead, and
trembling, trimming the diadem which Cleopatra wore upon her head. One of the
soldiers seeing her, angrily said unto her: ``Is that well done, Charmion?''
``Very well,'' said she again, ``and meet for a princess descended from the race
of so many noble kings:'' she said no more, but fell down dead hard by the
bed. 
% LXXXVI. 
Some report that this aspick was brought unto her in the basket
with figs, and that she had commanded them to hide it under the fig-leaves,
that when she should think to take out the figs, the aspick should bite her 
before she should see her: howbeit, that when she would have taken
away the leaves for the figs, she perceived it, and said, ``Art thou here,
then?''
And \gloss{Cleopatra \\killed with\\ the biting of\\ an aspick.} so, her arm being naked,
she put it to the aspick to be bitten. Others say again, she kept it in a box,
and that she did prick and thrust it with a spindle of gold, so that the
aspick, being angered withal, leapt out with great fury, and bit her in the
arm.
Howbeit few can tell the troth. For they report also, that she had
hidden poison in a hollow razor which she carried in the hair of her head; and
yet was there no mark seen on her body, or any sign discerned that she was
poisoned, neither also did they kind this serpent in her tomb: but it was
reported only, that there was seen certain fresh steps or tracks where it had
gone, on the tomb-side toward the sea, and specially by the door-side. Some
say also that they found two little pretty bitings in her arm, scant
to be discerned: the which it seemeth Caesar himself gave credit unto, because
in his triumph he carried Cleopatra's  
\gloss{The image\\ of Cleopatra,\\ carried in\\ triumph at\\ Rome
with\\ an aspick\\ biting of her\\ arm.}
image, with an aspick biting of her arm. And thus goeth the report of her death. 
Now Caesar, though he was marvellous sorry for the death of
Cleopatra, yet he wondered at her noble mind and courage, and therefore
commanded she should be nobly buried, and laid by Antonius: and willed also
that her two women should have honourable burial. 
\gloss{The age of Cleopatra and Antonius.}  Cleopatra died being eight and thirty
years old, after she had reigned two and twenty years, and governed about
fourteen of them with Antonius. And for Antonius, some say that he lived three
and fifty years: and others say, six and fifty. All his statues, images, and
metals, were plucked down and overthrown, saving those of Cleopatra, which
stood still in their places, by means of Archibius one of her friends, who
gave Caesar a thousand talents that they should not be handled as those of
Antonius were. 
%LXXXVII. 
Antonius left seven children by three wives, of the which Caesar did
put Antyllus (the eldest son he had by Fulvia) to death. 
Octavia \gloss{Of Antonius' issue came emperors.} his wife took all the rest, and
brought them up with hers, and married Cleopatra, Antonius' daughter, unto
king Juba, a marvellous courteous and goodly prince. And Antonius (the son of
Fulvia) came to be so great, that next unto Agrippa, who was in greatest
estimation about Caesar, and next unto the children of Livia, which were the
second in estimation, he had the third place. Furthermore, Octavia having had
two daughters by her first husband Marcellus, and a son also called Marcellus,
Caesar married his daughter unto that Marcellus, and so did adopt him for his son. And Octavia
also married one of her daughters unto Agrippa. But when Marcellus was dead,
after he had been married a while, Octavia, perceiving that her brother Caesar
was very busy to choose some one among his friends, whom he trusted best, to
make his son-in-law, she persuaded him that Agrippa should marry his daughter
(Marcellus' widow), and leave her own daughter. Caesar first was contented
withal, and then Agrippa: and so she afterwards took away her daughter and
married her unto Antonius; and Agrippa married Julia, Caesar's daughter. Now
there remained two daughters more of Octavia and Antonius: Domitius Aenobarbus
married the one; and the other, which was Antonia, so fair and virtuous a
young lady, was married unto Drusus, the son of Livia, and son-in-law of
Caesar. Of this marriage came Germanicus and Clodius: of the which,
Clodius afterwards came to be emperor. And of the sons of Germanicus, the one
whose name was Caius came also to be emperor: who after he had
licentiously reigned a time, was slain, with his wife and daughter. Agrippina
also (having a son by her first husband Aenobarbus, called Lucius Domitius)
was afterwards married unto Clodius, who adopted her son, and called him Nero
Germanicus. This Nero was emperor in our time, who slew his own mother, and
had almost destroyed the empire of Rome through his madness and wicked life,
being the fifth emperor of Rome after Antonius.

\endprosa
\end{document}

