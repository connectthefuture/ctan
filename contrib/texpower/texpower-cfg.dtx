% \iffalse meta-comment
% --------------------------------------------------------------
% Part of the TeXPower bundle
% Copyright (C) 1999-2004 Stephan Lehmke
%
% This program is free software; you can redistribute it and/or
% modify it under the terms of the GNU General Public License
% as published by the Free Software Foundation; either version 2
% of the License, or (at your option) any later version.
%
% This program is distributed in the hope that it will be useful,
% but WITHOUT ANY WARRANTY; without even the implied warranty of
% MERCHANTABILITY or FITNESS FOR A PARTICULAR PURPOSE.  See the
% GNU General Public License for more details.
% --------------------------------------------------------------
%
% texpower-cfg.dtx,v 1.3 2005/03/28 22:30:59 hansfn Exp
%
% \fi
%
% \iffalse
%
%<*driver>
\documentclass{article}
\begin{document}
\typeout{***********************************************************}
\typeout{*}
\typeout{*\space\space\space\space\space\space\space Do NOT compile this dtx file!}
\typeout{*}
\typeout{***********************************************************}
\end{document}
%</driver>
%
% \fi
%<*tpcolors>
%<<KEEPCOMMENTS
%=======================================================================================================================
% File: tpcolors.cfg
%
% Color configuration for the texpower package.
%
%=======================================================================================================================

%=======================================================================================================================
% Standard colors.
%
% For every of the colors described in the "Defined colors for emphasis and highlighting elements" section of the
% documentation, there is one default value which is defined here. Further default colors are for background, panel and
% navigation elements. 
%
% Note that this file is loaded _only_ if TeXPower's color management is active.
%
% Redefine colors for individual taste. Please remember, however, that a document transferred into another configuration
% where a different tpcolors.cfg is present, will look different on compilation. So for a `stable' color configuration,
% it might be better to put the respective \defineTPcolor commands into the preamble of the document or into a style
% file. 

% Color definitions for white background.

\defineTPcolor[whitebg]{pagecolor}{rgb}{1,1,1}%                 Page background (for background style `plain').

\defineTPcolor[whitebg]{bgndstartcolor}{rgb}{1,1,0.9}%          Start color for gradient background.
\defineTPcolor[whitebg]{bgndendcolor}{rgb}{1,1,1}%              End color for gradient background.
\defineTPcolor[whitebg]{bgndmidcolor}{rgb}{0.9,1,0.9}%          Middle color for gradient background (`double' gradient).

\defineTPcolor[whitebg]{textcolor}{rgb}{0,0,0.5}%               Normal text.

\defineTPcolor[whitebg]{emcolor}{rgb}{0,0,0.8}%                 Emphasised text (if coloremph option is given).
\defineTPcolor[whitebg]{altemcolor}{rgb}{0,0.5,0.8}%            Double emphasised text (if coloremph option is given).

\defineTPcolor[whitebg]{mathcolor}{rgb}{0,0.5,0}%               Math (if colormath option is given).

\defineTPcolor[whitebg]{codecolor}{rgb}{0,0.5,0}%               \code (additional emphasising command).
\defineTPcolor[whitebg]{underlcolor}{rgb}{0.7,0,0.3}%           \underl (additional emphasising command).
\defineTPcolor[whitebg]{conceptcolor}{rgb}{0.6,0,0}%            \concept (additional emphasising command).
\defineTPcolor[whitebg]{inactivecolor}{rgb}{0.7,0.7,0.7}%       \inactive (additional emphasising command).
\defineTPcolor[whitebg]{presentcolor}{rgb}{1,1,0.8}%            Background color for \present (emphasising command).
\defineTPcolor[whitebg]{highlightcolor}{rgb}{1,1,0.7}%          Background color for \highlight (emphasising command).

\defineTPcolor[whitebg]{buttoncolor}{rgb}{0.8,0.8,0.9}%         Background color for buttons.
\defineTPcolor[whitebg]{buttonshadowcolor}{rgb}{.001,0,.502}%   Button shadow.
\defineTPcolor[whitebg]{buttonframecolor}{gray}{0}%             Button frame.
\defineTPcolor[whitebg]{buttontextcolor}{gray}{0}%              Button text.

\defineTPcolor[whitebg]{toppanelcolor}{gray}{0.9}%              Top panel background.
\defineTPcolor[whitebg]{bottompanelcolor}{gray}{0.9}%           Bottom panel background.
\defineTPcolor[whitebg]{leftpanelcolor}{gray}{0.9}%             Left panel background.
\defineTPcolor[whitebg]{rightpanelcolor}{gray}{0.9}%            Right panel background.
\defineTPcolor[whitebg]{toppaneltextcolor}{gray}{0}%            Top panel text.
\defineTPcolor[whitebg]{bottompaneltextcolor}{gray}{0}%         Bottom panel text.
\defineTPcolor[whitebg]{leftpaneltextcolor}{gray}{0}%           Left panel text.
\defineTPcolor[whitebg]{rightpaneltextcolor}{gray}{0}%          Right panel text.


% Color definitions for `light' background.

\defineTPcolor[lightbg]{pagecolor}{rgb}{1,1,0.9}%               Page background (for background style `plain').

\defineTPcolor[lightbg]{bgndstartcolor}{rgb}{1,1,0.8}%          Start color for gradient background.
\defineTPcolor[lightbg]{bgndendcolor}{rgb}{1,1,1}%              End color for gradient background.
\defineTPcolor[lightbg]{bgndmidcolor}{rgb}{0.8,1,0.8}%          Middle color for gradient background (`double' gradient).

\defineTPcolor[lightbg]{textcolor}{rgb}{0,0,0.5}%               Normal text.

\defineTPcolor[lightbg]{emcolor}{rgb}{0,0,0.8}%                 Emphasised text (if coloremph option is given).
\defineTPcolor[lightbg]{altemcolor}{rgb}{0,0.5,0.8}%            Double emphasised text (if coloremph option is given).

\defineTPcolor[lightbg]{mathcolor}{rgb}{0,0.4,0}%               Math (if colormath option is given).

\defineTPcolor[lightbg]{codecolor}{rgb}{0,0.5,0}%               \code (additional emphasising command).
\defineTPcolor[lightbg]{underlcolor}{rgb}{0.7,0,0.3}%           \underl (additional emphasising command).
\defineTPcolor[lightbg]{conceptcolor}{rgb}{0.6,0,0}%            \concept (additional emphasising command).
\defineTPcolor[lightbg]{inactivecolor}{rgb}{0.7,0.7,0.7}%       \inactive (additional emphasising command).
\defineTPcolor[lightbg]{presentcolor}{rgb}{1,1,1}%              Background color for \present (emphasising command).
\defineTPcolor[lightbg]{highlightcolor}{rgb}{1,1,0.6}%          Background color for \highlight (emphasising command).

\defineTPcolor[lightbg]{buttoncolor}{rgb}{0.8,0.8,0.9}%         Background color for buttons.
\defineTPcolor[lightbg]{buttonshadowcolor}{rgb}{.001,0,.502}%   Button shadow.
\defineTPcolor[lightbg]{buttonframecolor}{gray}{0}%             Button frame.
\defineTPcolor[lightbg]{buttontextcolor}{gray}{0}%              Button text.

\defineTPcolor[lightbg]{toppanelcolor}{gray}{0.9}%              Top panel background.
\defineTPcolor[lightbg]{bottompanelcolor}{gray}{0.9}%           Bottom panel background.
\defineTPcolor[lightbg]{leftpanelcolor}{gray}{0.9}%             Left panel background.
\defineTPcolor[lightbg]{rightpanelcolor}{gray}{0.9}%            Right panel background.
\defineTPcolor[lightbg]{toppaneltextcolor}{gray}{0}%            Top panel text.
\defineTPcolor[lightbg]{bottompaneltextcolor}{gray}{0}%         Bottom panel text.
\defineTPcolor[lightbg]{leftpaneltextcolor}{gray}{0}%           Left panel text.
\defineTPcolor[lightbg]{rightpaneltextcolor}{gray}{0}%          Right panel text.


% Color definitions for `dark' background.

\defineTPcolor[darkbg]{pagecolor}{rgb}{0,0,0.4}%               Page background (for background style `plain').

\defineTPcolor[darkbg]{bgndstartcolor}{rgb}{0,0,0.2}%          Start color for gradient background.
\defineTPcolor[darkbg]{bgndendcolor}{rgb}{0,0,0.6}%            End color for gradient background.
\defineTPcolor[darkbg]{bgndmidcolor}{rgb}{0,0.3,0.4}%          Middle color for gradient background (`double' gradient).

\defineTPcolor[darkbg]{textcolor}{rgb}{1,1,0.6}%               Normal text.

\defineTPcolor[darkbg]{emcolor}{rgb}{1,1,0}%                   Emphasised text (if coloremph option is given).
\defineTPcolor[darkbg]{altemcolor}{rgb}{1,0.6,0}%              Double emphasised text (if coloremph option is given).

\defineTPcolor[darkbg]{mathcolor}{rgb}{1,0.5,1}%               Math (if colormath option is given).

\defineTPcolor[darkbg]{codecolor}{rgb}{0,0.7,0}%               \code (additional emphasising command).
\defineTPcolor[darkbg]{underlcolor}{rgb}{0.3,1,0.7}%           \underl (additional emphasising command).
\defineTPcolor[darkbg]{conceptcolor}{rgb}{0.4,1,1}%            \concept (additional emphasising command).
\defineTPcolor[darkbg]{inactivecolor}{rgb}{0.4,0.4,0.4}%       \inactive (additional emphasising command).
\defineTPcolor[darkbg]{presentcolor}{rgb}{0,0,0}%              Background color for \present (emphasising command).
\defineTPcolor[darkbg]{highlightcolor}{rgb}{0,0,0.1}%          Background color for \highlight (emphasising command).

\defineTPcolor[darkbg]{buttoncolor}{rgb}{0.4,0.4,0.6}%         Background color for buttons.
\defineTPcolor[darkbg]{buttonshadowcolor}{gray}{0}%            Button shadow.
\defineTPcolor[darkbg]{buttonframecolor}{gray}{0}%             Button frame.
\defineTPcolor[darkbg]{buttontextcolor}{gray}{0}%              Button text.

\defineTPcolor[darkbg]{toppanelcolor}{gray}{0.4}%              Top panel background.
\defineTPcolor[darkbg]{bottompanelcolor}{gray}{0.4}%           Bottom panel background.
\defineTPcolor[darkbg]{leftpanelcolor}{gray}{0.4}%             Left panel background.
\defineTPcolor[darkbg]{rightpanelcolor}{gray}{0.4}%            Right panel background.
\defineTPcolor[darkbg]{toppaneltextcolor}{gray}{1}%            Top panel text.
\defineTPcolor[darkbg]{bottompaneltextcolor}{gray}{1}%         Bottom panel text.
\defineTPcolor[darkbg]{leftpaneltextcolor}{gray}{1}%           Left panel text.
\defineTPcolor[darkbg]{rightpaneltextcolor}{gray}{1}%          Right panel text.


% Color definitions for black background.

\defineTPcolor[blackbg]{pagecolor}{rgb}{0,0,0}%                 Page background (for background style `plain').

\defineTPcolor[blackbg]{bgndstartcolor}{rgb}{0,0,0}%            Start color for gradient background.
\defineTPcolor[blackbg]{bgndendcolor}{rgb}{0,0,0.2}%            End color for gradient background.
\defineTPcolor[blackbg]{bgndmidcolor}{rgb}{0,0.2,0.2}%          Middle color for gradient background (`double' gradient).

\defineTPcolor[blackbg]{textcolor}{rgb}{1,1,0.6}%               Normal text.

\defineTPcolor[blackbg]{emcolor}{rgb}{1,1,0}%                   Emphasised text (if coloremph option is given).
\defineTPcolor[blackbg]{altemcolor}{rgb}{1,0.5,0}%              Double emphasised text (if coloremph option is given).

\defineTPcolor[blackbg]{mathcolor}{rgb}{1,0.5,1}%               Math (if colormath option is given).

\defineTPcolor[blackbg]{codecolor}{rgb}{0,0.7,0}%               \code (additional emphasising command).
\defineTPcolor[blackbg]{underlcolor}{rgb}{0.3,1,0.7}%           \underl (additional emphasising command).
\defineTPcolor[blackbg]{conceptcolor}{rgb}{0.4,1,1}%            \concept (additional emphasising command).
\defineTPcolor[blackbg]{inactivecolor}{rgb}{0.4,0.4,0.4}%       \inactive (additional emphasising command).
\defineTPcolor[blackbg]{presentcolor}{rgb}{0,0,0.3}%            Background color for \present (emphasising command).
\defineTPcolor[blackbg]{highlightcolor}{rgb}{0,0,0.4}%          Background color for \highlight (emphasising command).

\defineTPcolor[blackbg]{buttoncolor}{rgb}{0.4,0.4,0.6}%         Background color for buttons.
\defineTPcolor[blackbg]{buttonshadowcolor}{gray}{0}%            Button shadow.
\defineTPcolor[blackbg]{buttonframecolor}{gray}{0}%             Button frame.
\defineTPcolor[blackbg]{buttontextcolor}{gray}{0}%              Button text.

\defineTPcolor[blackbg]{toppanelcolor}{gray}{0.4}%              Top panel background.
\defineTPcolor[blackbg]{bottompanelcolor}{gray}{0.4}%           Bottom panel background.
\defineTPcolor[blackbg]{leftpanelcolor}{gray}{0.4}%             Left panel background.
\defineTPcolor[blackbg]{rightpanelcolor}{gray}{0.4}%            Right panel background.
\defineTPcolor[blackbg]{toppaneltextcolor}{gray}{1}%            Top panel text.
\defineTPcolor[blackbg]{bottompaneltextcolor}{gray}{1}%         Bottom panel text.
\defineTPcolor[blackbg]{leftpaneltextcolor}{gray}{1}%           Left panel text.
\defineTPcolor[blackbg]{rightpaneltextcolor}{gray}{1}%          Right panel text.


% The default definitions of TeXPower's standard colors do not include dimmed and enhanced variants, leaving the
% variants to be calculated automatically. This works quite well for the dimmed variant, but the automatically
% calculated enhanced variant might not be to everybody's taste. The following example (if uncommented) will define
% custom enhanced versions of all standard `text body' colors in all color sets which look slightly better than the
% automatically calculated ones. Note that background and panels are not affected by dimming and enhancing.

%% Enhanced color variants for white background.

% \defineTPcolor[whitebg]{etextcolor}{rgb}{0,0,0.7}%              Normal text. 

% \defineTPcolor[whitebg]{eemcolor}{rgb}{0,0,1}%                  Emphasised text (if coloremph option is given).
% \defineTPcolor[whitebg]{ealtemcolor}{rgb}{0,0.6,0.9}%           Double emphasised text (if coloremph option is given).

% \defineTPcolor[whitebg]{emathcolor}{rgb}{0,0.6,0}%              Math (if colormath option is given).

% \defineTPcolor[whitebg]{ecodecolor}{rgb}{0,0.6,0}%              \code (additional emphasising command).
% \defineTPcolor[whitebg]{eunderlcolor}{rgb}{0.8,0,0.4}%          \underl (additional emphasising command).
% \defineTPcolor[whitebg]{econceptcolor}{rgb}{0.8,0,0}%           \concept (additional emphasising command).
% \defineTPcolor[whitebg]{einactivecolor}{rgb}{0.6,0.6,0.6}%      \inactive (additional emphasising command).
% \defineTPcolor[whitebg]{epresentcolor}{rgb}{1,1,0.8}%           Background color for \present (emphasising command).
% \defineTPcolor[whitebg]{ehighlightcolor}{rgb}{1,1,0.7}%         Background color for \highlight (emphasising command).


%% Enhanced color variants for `light' background.

% \defineTPcolor[lightbg]{etextcolor}{rgb}{0,0,0.7}%              Normal text. 

% \defineTPcolor[lightbg]{eemcolor}{rgb}{0,0,1}%                  Emphasised text (if coloremph option is given).
% \defineTPcolor[lightbg]{ealtemcolor}{rgb}{0,0.6,0.9}%           Double emphasised text (if coloremph option is given).

% \defineTPcolor[lightbg]{emathcolor}{rgb}{0,0.6,0}%              Math (if colormath option is given).

% \defineTPcolor[lightbg]{ecodecolor}{rgb}{0,0.65,0}%             \code (additional emphasising command).
% \defineTPcolor[lightbg]{eunderlcolor}{rgb}{0.8,0,0.4}%          \underl (additional emphasising command).
% \defineTPcolor[lightbg]{econceptcolor}{rgb}{0.8,0,0}%           \concept (additional emphasising command).
% \defineTPcolor[lightbg]{einactivecolor}{rgb}{0.6,0.6,0.6}%      \inactive (additional emphasising command).
% \defineTPcolor[lightbg]{epresentcolor}{rgb}{.95,1,.95}%         Background color for \present (emphasising command).
% \defineTPcolor[lightbg]{ehighlightcolor}{rgb}{1,1,0.4}%         Background color for \highlight (emphasising command).


%% Enhanced color variants for `dark' background.

% \defineTPcolor[darkbg]{etextcolor}{rgb}{1,1,0.8}%               Normal text. 

% \defineTPcolor[darkbg]{eemcolor}{rgb}{1,1,0.3}%                 Emphasised text (if coloremph option is given).
% \defineTPcolor[darkbg]{ealtemcolor}{rgb}{1,0.8,0}%              Double emphasised text (if coloremph option is given).

% \defineTPcolor[darkbg]{emathcolor}{rgb}{1,0.6,1}%               Math (if colormath option is given).

% \defineTPcolor[darkbg]{ecodecolor}{rgb}{0,0.9,0}%               \code (additional emphasising command).
% \defineTPcolor[darkbg]{eunderlcolor}{rgb}{0.4,1,0.8}%           \underl (additional emphasising command).
% \defineTPcolor[darkbg]{econceptcolor}{rgb}{0.7,1,1}%            \concept (additional emphasising command).
% \defineTPcolor[darkbg]{einactivecolor}{rgb}{0.5,0.5,0.5}%       \inactive (additional emphasising command).
% \defineTPcolor[darkbg]{epresentcolor}{rgb}{0.1,0,0}%            Background color for \present (emphasising command).
% \defineTPcolor[darkbg]{ehighlightcolor}{rgb}{0,0,0}%            Background color for \highlight (emphasising command).


%% Enhanced color variants for black background.

% \defineTPcolor[blackbg]{etextcolor}{rgb}{1,1,0.8}%              Normal text. 

% \defineTPcolor[blackbg]{eemcolor}{rgb}{1,1,0.3}%                Emphasised text (if coloremph option is given).
% \defineTPcolor[blackbg]{ealtemcolor}{rgb}{1,0.4,0}%             Double emphasised text (if coloremph option is given).

% \defineTPcolor[blackbg]{emathcolor}{rgb}{1,0.6,1}%              Math (if colormath option is given).

% \defineTPcolor[blackbg]{ecodecolor}{rgb}{0,0.9,0}%              \code (additional emphasising command).
% \defineTPcolor[blackbg]{eunderlcolor}{rgb}{0.4,1,0.8}%          \underl (additional emphasising command).
% \defineTPcolor[blackbg]{econceptcolor}{rgb}{0.7,1,1}%           \concept (additional emphasising command).
% \defineTPcolor[blackbg]{einactivecolor}{rgb}{0.5,0.5,0.5}%      \inactive (additional emphasising command).
% \defineTPcolor[blackbg]{epresentcolor}{rgb}{0,0,0.4}%           Background color for \present (emphasising command).
% \defineTPcolor[blackbg]{ehighlightcolor}{rgb}{0,0,0.5}%         Background color for \highlight (emphasising command).



% Remember that the color definitions above only represent _one_ example for a possible color configuration.
% The following example (if uncommented) completely redefines the standard colors for the darkbackground option. This
% works for one or all colors of this or other background options by analogy. (Comment out the definition of the
% darkbackground colors above if convenient.)

%% Color definitions for `dark' background.

% \defineTPcolor[darkbg]{pagecolor}{rgb}{0,0.1,0}%              Page background (for background style `plain').

% \defineTPcolor[darkbg]{bgndstartcolor}{rgb}{0,0.05,0}%        Start color for gradient background.
% \defineTPcolor[darkbg]{bgndendcolor}{rgb}{0,0.4,0}%           End color for gradient background.
% \defineTPcolor[darkbg]{bgndmidcolor}{rgb}{0,0.2,0.5}%         Middle color for gradient background (`double' gradient).

% \defineTPcolor[darkbg]{textcolor}{rgb}{1,0.6,1}%              Normal text.

% \defineTPcolor[darkbg]{emcolor}{rgb}{1,0,1}%                  Emphasised text (if coloremph option is given).
% \defineTPcolor[darkbg]{altemcolor}{rgb}{1,0,0.5}%             Double emphasised text (if coloremph option is given).

% \defineTPcolor[darkbg]{mathcolor}{rgb}{0.5,1,1}%              Math (if colormath option is given).

% \defineTPcolor[darkbg]{codecolor}{rgb}{0.8,0,0}%              \code (additional emphasising command).
% \defineTPcolor[darkbg]{underlcolor}{rgb}{1,0.7,0.3}%          \underl (additional emphasising command).
% \defineTPcolor[darkbg]{conceptcolor}{rgb}{1,1,0.4}%           \concept (additional emphasising command).
% \defineTPcolor[darkbg]{inactivecolor}{rgb}{0.4,0.4,0.4}%      \inactive (additional emphasising command).
% \defineTPcolor[darkbg]{presentcolor}{rgb}{0,0,0}%             Background color for \present (emphasising command).
% \defineTPcolor[darkbg]{highlightcolor}{rgb}{0,0.3,0}%         Background color for \highlight (emphasising command).

% \defineTPcolor[darkbg]{buttoncolor}{rgb}{0.4,0.4,0.6}%        Background color for buttons.
% \defineTPcolor[darkbg]{buttonshadowcolor}{gray}{0}%           Button shadow.
% \defineTPcolor[darkbg]{buttonframecolor}{gray}{0}%            Button frame.
% \defineTPcolor[darkbg]{buttontextcolor}{gray}{0}%             Button text.

% \defineTPcolor[darkbg]{toppanelcolor}{gray}{0.4}%             Top panel background.
% \defineTPcolor[darkbg]{bottompanelcolor}{gray}{0.4}%          Bottom panel background.
% \defineTPcolor[darkbg]{leftpanelcolor}{gray}{0.4}%            Left panel background.
% \defineTPcolor[darkbg]{rightpanelcolor}{gray}{0.4}%           Right panel background.
% \defineTPcolor[darkbg]{toppaneltextcolor}{gray}{1}%           Top panel text.
% \defineTPcolor[darkbg]{bottompaneltextcolor}{gray}{1}%        Bottom panel text.
% \defineTPcolor[darkbg]{leftpaneltextcolor}{gray}{1}%          Left panel text.
% \defineTPcolor[darkbg]{rightpaneltextcolor}{gray}{1}%         Right panel text.


% In the usual standard defaults given above, no dimmed or enhanced variants are included, to demonstrate TeXPower's
% automatic dimming and enhancing capabilities. It is possible, however, to provide `hand-tuned' dimmed and enhanced
% variants for nicer results. This is done by prefixing "d" or "e" to the color name. In the following, custom dimmed
% and enhanced variants of some of the colors defined above are specified. Note that background and panels are not
% affected by dimming and enhancing, so only the `text body' colors are specified.

%% Dimmed color variants for `dark' background.

% \defineTPcolor[darkbg]{dtextcolor}{rgb}{0.5,0.4,0.5}%            Normal text. 

% \defineTPcolor[darkbg]{demcolor}{rgb}{0.5,0,0.5}%                Emphasised text (if coloremph option is given).
% \defineTPcolor[darkbg]{daltemcolor}{rgb}{0.5,0,0.2}%             Double emphasised text (if coloremph option is given).

% \defineTPcolor[darkbg]{dmathcolor}{rgb}{0.2,0.5,0.5}%            Math (if colormath option is given).

% \defineTPcolor[darkbg]{dcodecolor}{rgb}{0.5,0,0}%                \code (additional emphasising command).
% \defineTPcolor[darkbg]{dunderlcolor}{rgb}{0.5,0.3,0.1}%          \underl (additional emphasising command).
% \defineTPcolor[darkbg]{dconceptcolor}{rgb}{0.5,0.5,0.2}%         \concept (additional emphasising command).
% \defineTPcolor[darkbg]{dinactivecolor}{rgb}{0.2,0.2,0.2}%        \inactive (additional emphasising command).
% \defineTPcolor[darkbg]{dpresentcolor}{rgb}{0,0.05,0}%            Background color for \present (emphasising command).
% \defineTPcolor[darkbg]{dhighlightcolor}{rgb}{0,0.15,0}%          Background color for \highlight (emphasising command).


%% Enhanced color variants for `dark' background.

% \defineTPcolor[darkbg]{etextcolor}{rgb}{1,0.8,1}%                Normal text. 

% \defineTPcolor[darkbg]{eemcolor}{rgb}{1,0.2,1}%                  Emphasised text (if coloremph option is given).
% \defineTPcolor[darkbg]{ealtemcolor}{rgb}{1,0.2,0.7}%             Double emphasised text (if coloremph option is given).

% \defineTPcolor[darkbg]{emathcolor}{rgb}{0.7,1,1}%                Math (if colormath option is given).

% \defineTPcolor[darkbg]{ecodecolor}{rgb}{1,0,0}%                  \code (additional emphasising command).
% \defineTPcolor[darkbg]{eunderlcolor}{rgb}{1,0.9,0.5}%            \underl (additional emphasising command).
% \defineTPcolor[darkbg]{econceptcolor}{rgb}{1,1,0.6}%             \concept (additional emphasising command).
% \defineTPcolor[darkbg]{einactivecolor}{rgb}{0.5,0.5,0.5}%        \inactive (additional emphasising command).
% \defineTPcolor[darkbg]{epresentcolor}{rgb}{0.2,0,0}%             Background color for \present (emphasising command).
% \defineTPcolor[darkbg]{ehighlightcolor}{rgb}{0,0.4,0}%           Background color for \highlight (emphasising command).



% The following example (if uncommented) shows what is neccessary for defining a completely new standard color which is
% recognized by TeXPower commands like \dimcolors, \enhancecolors (and display commands using those) as well as the
% commands \whitebackground, \lightbackground, \darkbackground, and \blackbackground. 
%
% If you don't want to give different definitions of your color for different backgrounds, you can define it just in the
% current color set (leaving out the optional argument to \defineTPcolor), but then it will stay the same on all
% backgrounds and not be adapted when the background color changes.
%
% You don't have to define dimmed and enhanced variants of your color (with prefix d and e, respectively). If you just
% leave them out, TeXPower will calculate appropriate variants automatically.

% \defineTPcolor[whitebg]{mycolor}{rgb}{1,0.5,0}%               Variants of mycolor for white backgrounds.
% \defineTPcolor[whitebg]{dmycolor}{rgb}{0.9,0.8,0.6}%              
% \defineTPcolor[whitebg]{emycolor}{rgb}{1,0.7,0}%              
%   %
% \defineTPcolor[lightbg]{mycolor}{rgb}{1,0.5,0}%               Variants of mycolor for light backgrounds.
% \defineTPcolor[lightbg]{dmycolor}{rgb}{0.9,0.8,0.6}%              
% \defineTPcolor[lightbg]{emycolor}{rgb}{1,0.7,0}%              
%   %
% \defineTPcolor[darkbg]{mycolor}{rgb}{0,0.5,1}%                Variants of mycolor for dark backgrounds.
% \defineTPcolor[darkbg]{dmycolor}{rgb}{0.1,0.2,0.4}%              
% \defineTPcolor[darkbg]{emycolor}{rgb}{0,0.3,1}%              
%   %
% \defineTPcolor[blackbg]{mycolor}{rgb}{0,0.5,1}%               Variants of mycolor for black backgrounds.
% \defineTPcolor[blackbg]{dmycolor}{rgb}{0.1,0.2,0.4}%              
% \defineTPcolor[blackbg]{emycolor}{rgb}{0,0.3,1}%              

%KEEPCOMMENTS
%</tpcolors>
%=======================================================================================================================
%<*tpoptions>
%<<KEEPCOMMENTS
%=======================================================================================================================
% File: tpoptions.cfg
%
% System-specific configuration of options defined by the texpower package.
%
%=======================================================================================================================

% The following example (if uncommented) will turn on the display option by default.

%\ExecuteOptions{display}

%KEEPCOMMENTS
%</tpoptions>
%=======================================================================================================================
%<*tpsettings>
%<<KEEPCOMMENTS
%=======================================================================================================================
% File: tpsettings.cfg
%
% System-specific configuration of defaults defined by the texpower package.
%
%=======================================================================================================================


%KEEPCOMMENTS
%</tpsettings>
\endinput
