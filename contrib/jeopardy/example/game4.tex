\documentclass{article}
\usepackage{amsmath}
\usepackage[pdftex,designi]{web}
\usepackage[pdftex]{exerquiz}
\usepackage[pdftex]{eforms}
\usepackage[ImplMulti]{dljslib}

\everymath{\displaystyle}
\parindent 0pt
\definecolor{pozadi}{rgb}{1,1,0.9}
\pagecolor{pozadi}

\usepackage[picture,finetune]{jeopardy}
\def\JeopardyPictureFile{aleq.jpg}
\def\ChampionMsg{Niels Abel and Evariste Galois}

\def\logoB{\rotatebox{90}{%
    \vbox{\normalsize\hbox{\color{red}{Robert Ma\v{r}\'{i}k}}
      \hbox{\color{green}Jeopardy game}}}
  \vfill}

\let\rmdefault\sfdefault
\hypersetup{pdfpagemode=Window, pdfnewwindow=true, pdfmenubar=true,%
  pdftoolbar=true,colorlinks,  pdfwindowui=false,
  pdfpagemode=Window}
\pagestyle{empty}

\begin{insDLJS}[dljslibRMbbbb]{dljsliRMbbbb}{RMbbbb}


function ProcRespSetFormulaEmpty(flag,CorrAns,n,epsilon,a,indepVar,oComp)
{
    ok2Continue = true;
    if (!ProcessIt) return null;
    var fieldname = event.target.name;
    var UserAns = event.value;
    this.getField(fieldname).strokeColor = \eqDefaultColor;
    UserAns = stripWhiteSpace(UserAns);
    if (!ok2Continue) return null;
    if (UserAns == "empty") {
      success = (CorrAns == UserAns);
      return notifyField(success, flag, fieldname);
    }
    if (CorrAns == "empty") {
      success = (CorrAns == UserAns);
      return notifyField(success, flag, fieldname);
    }
    var aUserAns = UserAns.split(",");
    var aCorrAns = CorrAns.split(",");
    var numCorrect = 0;
    var match = 0;
    if ( aUserAns.length != aCorrAns.length ) return notifyField(false, flag, fieldname);
    for ( var i=0; i< aCorrAns.length; i++) {
        match = 0;
        for ( var j=i; j< aUserAns.length; j++) {
            var retn = _ProcResp(flag,aCorrAns[i],aUserAns[j],n,epsilon,a,indepVar,oComp);
            if ( retn == null ) return syntaxError(), null;
            if (retn==1) {
               var temp=aUserAns[j];
               aUserAns[j]=aUserAns[i];
               aUserAns[i]=temp;
               match = match + 1;
            }
        }
        numCorrect += (match) ? 1 : 0;
    }
    var success = (numCorrect == aCorrAns.length) ? true : false;
    if ( success == null ) return syntaxError(), null;
    return notifyField(success, flag, fieldname);
}

\end{insDLJS}


\def\quest#1#2{
  \begin{question}
    Solve the equation in the set of \textit{real} numbers.
    \begin{equation*}
      #1=0
    \end{equation*}
    \bigskip
    \begin{itemize}
    \item Write the solution into the field.
    \item If there is no solution, write the word \textit{empty}.
    \end{itemize}
    \bigskip
    $x={}$\RespBoxMath{#2}{10}{1.0E-2}{0}{1}*{ProcRespSetFormulaEmpty}
  \end{question}
}

\title{Algebraic equations}

%\def\title{Algebraic equations}

\author{Robert Ma\v r\'\i k}

\begin{document}

\maketitle
You will solve algebraic equations. Write the solutions as comma
separated list, In the case of repeted solution write each solution
only \textbf{once}!

\SetGameWidth{0.5\linewidth}
\Celltoks{\BC{}\BG{1 1 1}}
\AditionalShift=4.6pt

\newpage
\MakeGameBoard
\begin{category}{Linear}
  \quest{x+1}{-1}
  \quest{2x-3}{3/2}
  \quest{5x+15}{-3}
  \quest{7x+13}{-13/7}
  \quest{3x+1}{-1/3}
  \quest{2x+7}{-7/2}
  \quest{5x-6}{6/5}
  \quest{7x-13}{13/7}
\end{category}
\begin{category}{Quadratic1}
  \quest{x^2+x}{-1,0}
  \quest{x^2-9}{3,-3}
  \quest{x^2-7}{sqrt(7),-sqrt(7)}
  \quest{x^2+4x}{-4,0}
  \quest{x^2-x}{1,0}
  \quest{x^2+9}{empty}
  \quest{x^2}{0}
  \quest{x^2-4x}{4,0}
\end{category}
\begin{category}{Quadratic2}
  \quest{x^2+x+1}{empty}
  \quest{x^2-6x+9}{3}
  \quest{x^2-7x+6}{1,6}
  \quest{x^2+4x+4}{-2}
  \quest{x^2-3x+2}{1,2}
  \quest{x^2+2x+9}{empty}
  \quest{x^2-5x+6}{2,3}
  \quest{x^2+4x-21}{-7,3}
\end{category}
\begin{category}{Higher order}
  \quest{x^8-2}{2^(1/8),-2^(1/8)}
  \quest{x^3-2x}{0,sqrt(2),-sqrt(2)}
  \quest{x^4-1}{-1,1}
  \quest{x^3-8}{2}
  \quest{x^5-x^3}{1,-1,0}
  \quest{x^6-x^5}{1,0}
  \quest{x^3+2x^2+x}{0,-1}
  \quest{x^3+4x^2+5x}{0}
\end{category}

\newpage

\newpage
\textbf{\'Evariste Galois} (1811--1832) was a French mathematician
born in Bourg-la-Reine. He died in a duel at the age of twenty.

\medskip
\textbf{Niels Henrik Abel} (1802--1829), Norwegian mathematician, was
born in Nedstrand. In early April 1829 he obtained a position 
in Berlin, but the letter bringing the offer did not reach Norway
until two days after Abel's death from tuberculosis.

\medskip
Both mathematicians proved the impossibility of solving the $5$-degree
polynomial equation by radicals.

\newpage

\end{document}
