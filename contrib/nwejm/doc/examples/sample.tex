% This is an example of the usage of the `nwejmart' class dedicated to articles
% submitted to the North-Western European Journal of Mathematics.
%
% The language of the article is by default English. Should it be French, German
% or Dutch instead, it would be specified as \documentclass' option.
\documentclass[
% french  % If the language of the article will be French
% german  % If the language of the article will be German
% dutch   % If the language of the article will be Dutch
]{nwejmart}
%
% The following package should not be used for a real article! ;)
\usepackage{lipsum}
%
% Replace below the examples of simple and sophisticated bibliographic files
% `sample.bib' and `biblatex-examples.bib' by your own bibtex file(s),
% preferrably at `biblatex' format (don't forget the `.bib' extension
% below). This will require an extra `biber' compilation. See `biblatex'
% package's documentation for more details.
\addbibresource{sample.bib}
\addbibresource{biblatex-examples.bib}
%
% Should acronyms be used in the article, define them thanks to \newacronym
% command from `glossaries' package as follows:
%  - 1st argument: ⟨label⟩      of the acronym (also called key),
%  - 2nd argument: ⟨short form⟩ of the acronym (lowercase!),
%  - 3rd argument: ⟨long form⟩  of the acronym,
% and use them with \gls{⟨label⟩} (or, if needed, with \acrshort{⟨label⟩}).
% See `glossaries' package's documentation for more details.
\newacronym{nwejm}{nwejm}{North-Western European Journal of Mathematics}
%
\begin{document}
%
% Title of the article. A short form (that will be displayed in the headers and
% in the volume's TOC) may be specified as optional argument.
\title{Article's Title}
%
% Subtitle of the article, if any. A short form may be specified as optional
% argument.
% \subtitle{Article's Subtitle}
%
% Author(s) of the article:
% - one \author command per author,
% - mandatory argument entered as `⟨Last Name⟩, ⟨First Name⟩'.
% Use the key-value `affiliation={⟨affiliation⟩}' optional argument to specify
% one or more affiliations. An affiliation can be tagged
% (`affiliation=[⟨tag⟩]{⟨affiliation⟩}') and reused later
% (affiliationtagged={⟨tag⟩}).
\author[affiliation={Affiliation 1}]{Last1, First1}
\author[affiliation=[aff2]{Affiliation 2}]{Last2, First2}
\author[affiliation={Affiliation 3},affiliation={Affiliation 3 bis}]{Last3, First3}
\author[affiliation={Affiliation 4},affiliationtagged={aff2}]{Last4, First4}
%
% The abstract is entered as usually.
\begin{abstract}
  \lipsum[1]
\end{abstract}
%
% The keywords are entered thanks to \keywords command, as a comma separated list.
\keywords{foo,bar,baz}
%
% The Mathematical Subject Classification (MSC) are entered thanks to \msc
% command,as a comma separated list.
\msc{11B13,11B30,11P70}
%
\maketitle
%
% Acknowledgments, if any, are entered thanks to \acknowledgments command (and
% will be displayed just before the bibliography, thanks to the
% \printbibliography command).
\acknowledgments{Thanks to mum, daddy and all my buddies.}
%
% Unnumbered sections, if needed, are entered as usually with the starred
% version of the \section command. Note that:
% - their titles will automatically be displayed in the headers (and in the
%   volume's TOC),
% - no need to use the starred versions of the subsequent \subsection commands
% (if any)
\section*{Recommendations for \LaTeX}
Don't use:
\begin{itemize}
\item \verb"$$...$$" but \verb"\[...\]"
\item \verb"$a \over b$" but \verb"$\frac{a}{b}$"
\item \verb"{\cal ...}" but \verb"\mathcal{...}"
\item \verb"{\bf ...}" but \verb"\textbf{...}"
\item \verb"{\it ...}" but \verb"\emph{...}"
\item \verb"\'e" for instance to get an accent but type it directly (\verb"é")
  using the UTF8 encoding.
\end{itemize}
%
% Use mainly the \autocite command (from `biblatex' package) to cite
% references. Depending on the context, \textcite command (among others) may be
% used. See `biblatex' package's documentation for more details.
More generally, it is worth having a look at documents that highlight obsolete
commands and
packages\autocite{ensenbach2016,ensenbach2011,trettin2007,ensenbach2011a,trettin2007a}.
%
\section*{Introduction}
%
\subsection{Citations tests}
%
\begin{enumerate}
\item It\footnote{Foo bar.} is well known\autocite{baez/article}
  that... Moreover, it is well known\autocite{companion} that...
\item \textcite{baez/article} have proved... Moreover, \textcite{companion}
  have proved...
\end{enumerate}
%
\subsection{Cross-references tests}
%
% The cross-references are entered thanks to the \vref command (from `varioref'
% package) and the `cleveref' features. Note that:
% - the name of the object referenced is automatically added,
% - the page of the object referenced is automatically added (if not on the
%   same page).
Cf. \vref{thm-bolzano-weierstrass} \& \vref{rmk-euler} \&
\vref{eq-euler} \& \vref{sec-first-numbered}.
%
\subsection{Acronyms tests}
%
% As said above, use \gls{⟨label⟩} to display the acronym labelled ⟨label⟩. Note
% that, automatically:
% - the first occurrence of this command displays the /complete/ form of the
%   acronym (long form followed by the short one in parentheses),
% - the subsequent occurrences of this command display only the short form of the
%   acronym,
% - if an occurrence should be displayed as the short form of an acronym,
%   regardless it is the first one or not, the command \acrshort{⟨label⟩} is to
%   be used.
\begin{enumerate}
\item The present article is published in the \gls{nwejm}.
\item Moreover, the present article is published in the \gls{nwejm}.
\end{enumerate}
%
\subsection{Miscellaneaous}
%
% Use:
% - the \century command to display centuries, even negative ones,
% - the \aside command for interpolated clauses,
% - the \ie command for "that is",
% - the \acrlong when need to display (only) the long form of an acronym.
\begin{itemize}
\item It has been proved in the \century{19} \aside{more than 100 years ago}
  that...
\item This has been conceptualized in the \century{-3} \aside*{more than 2000
    years ago}.
\item \acrshort{nwejm} \ie{} \acrlong*{nwejm}.
\item \acrshort{nwejm} \ie*{} \acrlong*{nwejm}.
\end{itemize}
%
\subsection{Theorems tests}
%
% The theorems and the like are entered as usually. Note that, should one of
% them be unnumbered, the environment used would be starred.
\begin{theorem}[Bolzano–Weierstrass]\label{thm-bolzano-weierstrass}
  A subset of $\bbR^n$ ($n\in\bbN^*$) is sequentially compact if and only if it is
  closed and bounded.
\end{theorem}
\begin{proof}[not that easy!]
  ...
\end{proof}
\begin{definition}
  In Cartesian space $\bbR^n$ with the $p$-norm $L_p$, an open ball is the set
  \[
    B(r)=\set{x\in \bbR^n}[\sum _{i=1}^n\left|x_i\right|^p<r^p]
  \]
\end{definition}
\begin{remark}[Euler's identity]\label{rmk-euler}
  One of the most beautiful mathematical equation:
  \begin{equation*}
    \E[\I\pi]+1=0
  \end{equation*}
\end{remark}
\begin{lemma*}[Zorn]
  Suppose a partially ordered set $P$ has the property that every chain has an
  upper bound in $P$. Then the set $P$ contains at least one maximal element.
\end{lemma*}
%
\lipsum[2-6]
%
\begin{equation}\label{eq-euler}
  \E[\I\pi]+1=0
\end{equation}
%
\lipsum[8-15]
%
\section{First (numbered) section}\label{sec-first-numbered}
\lipsum[2]
\subsection{First subsection}
\lipsum[3-8]
\subsection{Second subsection}
\lipsum[9-15]
\section{Second (numbered) section}
\lipsum[16-38]
%
% The \printbibliography command (from `biblatex' package) displays the list of
% references (preceded by the acknowledgments, if any).
\printbibliography
%
\end{document}
