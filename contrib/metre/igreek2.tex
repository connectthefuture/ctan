\documentclass[11pt]{article}
\usepackage{ibycus4}
\usepackage{metre}
\thispagestyle{empty}
\begin{document}
\begin{center}
{\Large\bf Metrische Formen in Archilochos' Fragmenten}\\
\vspace{2ex}
{\small Rainer Thiel}
\end{center}
\vspace{2ex}
\begin{enumerate}
\item Elegisches Distichon (\sigla{6da_\cc hem\c hem}, frgg.~1--17\,W.):\\
  \metra{\n1\m\mbb\n2\m\mbb\n3\m\mbb\n4\m\mbb\n5\m\mbb\n6\m\a\cc}
  \metra{\m\mbb\m\mbb\m\c\m\bb\m\bb\m}

\item Iamb. Trimeter, stichisch (\sigla{3ia}, frgg. 18--87):
  \metra{\a\m\b\n1\m\s\a\m\b\n2\m\s\a\m\b\n3\m}\\ 
  (= Horazens Epodisches System VII, Hor. epod. 17)
\item Troch. Tetrameter, stichisch \sigla{4tr_}, frgg. 88--167):\par
  \metra{\m\b\m\a\s\m\b\m\a\c\m\b\m\a\s\m\b\m}
\item In Archilochos' Epoden-Fragmenten vorfindliche Systeme:
\begin{enumerate}
\item \sigla{enopl\c ith} (frgg. 168--171):
  \metra{\bbmb\m\bbmb\m\bbmb\m\bm\c\m\b\m\b\m\bm}
\item \sigla{3ia\cc 2ia} (frgg. 172--181):
  \metra{\a\m\b\m\s\a\m\b\m\s\a\m\b\m\cc\a\m\b\m\s\a\m\b\m}\\
  (= Horazens Epodisches System I, vgl. Hor. epod. 1--10)
\item \sigla{3ia\cc hem} (frgg. 182--187): %
  \metra{\a\m\b\m\s\a\m\b\m\s\a\m\b\m\cc\m\bb\m\bb\m}
\item \sigla{alcm\c ith\cc 3ia_} (frgg. 188--192):\par
\metra{\m\mbb\m\mbb\m\mbb\m\mbb\c\m\b\m\b\m\bm\cc\a\m\b\m\s\a\m\b\m\s\b\m
\m}
\item \sigla{6da_\cc 2ia} (frg. 193,  evtl. frg. 194):\par
  \metra{\m\mbb\m\mbb\m\mbb\m\mbb\m\mbb\m\a\cc\a\m\b\m\s\a\m\b\m}\\
(= Horazens Epodisches System V, vgl. Hor. epod. 14. 15)
\item \Angus\sigla{6da_\cc<<}\Angud\sigla{alcm_} (frg. 195, nicht ganz
  sicher):\\\Angus\metra{\m\mbb\m\mbb\m\mbb\m\mbb\m\mbb\m\a}\Angud
  \metra{\m\mbb\m\mbb\m\bb\m\m} \\(vgl. Horazens Epodisches System
  III, Hor. epod. 12)
\item
\begin{enumerate}
\item \sigla{hem 2ia} (evtl. nur ein Teil von
  \addtocounter{enumiii}{1}(\roman{enumiii})\addtocounter{enumiii}{-1};
  frg. 196)\par Heph. 19,9, p. 50,14--17:\par {\greek {\small Tri'ton de'
      e)stin para` )Arxilo'xw| a)suna'rthton \textit{e)k daktulikou=
        penq\-hmi\-me\-rou=s} (\metra{\m\bb\m\bb\m}) \textit{kai`
        i)ambikou= dime'trou a)katalh'ktou} 
(\metra{\a\m\b\m\s\a\m\b\m});}\\
    a)lla' m' o( lusimelh's, w)tai=re, da'mnatai po'qos.}

\item K\"olner Epodenfragment (frg. 196\,a): \sigla{3ia\cc hem\cc2ia}\\
\metra{\a\m\b\m\s\a\m\b\m\s\a\m\b\m\cc\m\bb\m\bb\m\cc\a\m\b\m\s\a
\m\b\m}\\ (= Horazens Epodisches System II, vgl. Hor. epod. 11)

\end{enumerate}
\item \sigla{3tr_} (frg. 197): 
\metra{\m\b\m\a\s\m\b\m\a\s\m\b\m}

\end{enumerate}
\end{enumerate}

\end{document}


