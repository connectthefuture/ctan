\documentclass[11pt]{book}
\usepackage [latin] {babel}[2005/05/21 v3.8g]
\usepackage[pagestyles,outermarks,clearempty]{titlesec}[2005/01/22 v2.6]
\usepackage{titletoc}[2005/01/22 v1.5]
\usepackage [repeat]{TEXNIKA}
\usepackage{example}

\TextHeight {\textheight - 8\leading}
\TextWidth  {4in}

\parindent  0em
\parskip    0pt 
\hfuzz      1pt

\newpagestyle {MainMatterPage} {
  \sethead   []
             [\textsc{\MakeLowercase {C. Ivli Caesaris}}]
             []
             {}
             {\textsc{\MakeLowercase {De Bello Gallico LIB.\space
                      \romannumeral\chaptertitle}}}
             {}
  \Capita    {chapter}{section}
}

\Locus \numerus  {\leftmargin - .75em  \\  \rightmargin + .75em}
\Facies          {\oldstylenums{#1}}
\Modus           {\pagewise \aligned{right} \\
                  \pagewise \aligned{left}}

\Novus \numerus \Nchap
\Facies         {\textbf{\oldstylenums{#1}}\Nsect{0}}
\Locus          {\rightmargin + .5em \\ \leftmargin - .5em}
\Modus          {\milestone \aligned{left} \\ \aligned{right}}
\Progressio     {0}

\Novus \numerus \Nsect
\Facies         {\RelSize{-1}\oldstylenums{#1}}
\Locus          {\rightmargin + .5em \\ \leftmargin - .5em}
\Modus          {\milestone \aligned{left} \\ \aligned{right}}
\Progressio     {0}

% The commands to set the milestones

\newcommand {\chap} {\par \hskip 1em \Nchap*{=+1}\Nsect{1}}
\newcommand {\sect} {\Nsect*{=+1}}

\Novus  \titulus \Liber
\Facies     {\ifthenelse {\value{Nliber} > 0}{\newpage}{} 
             \thispagestyle{empty} 
             {\RelSize{+1}C. IVLI CAESARIS}\\[1.5ex plus .25ex minus .125ex]
             {\RelSize{-1}\textsc{commentariorum}}\\[2ex plus .25ex minus .125ex]
             {\RelSize{+2}\LetterSpace{DE BELLO GALLICO}}\\[2.5ex plus .25ex minus .125ex]
             \textsc{LIBER \Nliber*{=+1}}%
             \vspace{3ex}%
             \Nchap{0}\numerus{1}%
            }

\Novus \numerus \Nliber
\Facies         {\ordinal {#1}\MakeUppercase{\theordinal}}
\Caput          {\chapter \headline}

\Facies \incipit {\textsc{#1#2}}

\begin{document}

\ExampleTitle [l]{C. IVLI CAESARIS}
                 {Commentarii\\[1ex]de Bello Gallico}
                 {Commentariorum pars prior\\[.75ex] Oxford Classical Texts}

\pagestyle {MainMatterPage} 
\thispagestyle {empty} 

\numerus{1} 
\Liber
\prosa  
\chap \incipit{Gallia} est omnis divisa in partes tres, quarum unam
incolunt Belgae, aliam Aquitani, tertiam qui ipsorum lingua Celtae,
nostra Galli appellantur. \sect Hi omnes lingua, institutis, legibus
inter se differunt. Gallos ab Aquitanis Garumna flumen, a Belgis
Matrona et Sequana dividit. \sect  Horum omnium fortissimi sunt
Belgae, propterea quod a cultu atque humanitate provinciae longissime
absunt, minimeque ad eos mercatores saepe commeant atque ea quae ad
effeminandos animos pertinent important, proximique sunt Germanis,
qui trans Rhenum incolunt, quibuscum continenter bellum gerunt.

\sect Qua de causa Helvetii quoque reliquos Gallos virtute praecedunt,
quod fere cotidianis proeliis cum Germanis contendunt, cum aut suis
finibus eos prohibent aut ipsi in eorum finibus bellum gerunt. \sect
Eorum una, pars, quam Gallos obtinere dictum est, initium capit
a flumine Rhodano, continetur Garumna flumine, Oceano, finibus
Belgarum, attingit etiam ab Sequanis et Helvetiis flumen Rhenum,
vergit ad septentriones. \sect Belgae ab extremis Galliae finibus
oriuntur, pertinent ad inferiorem partem fluminis Rheni, spectant in
septentrionem et orientem solem. \sect Aquitania a Garumna flumine ad
Pyrenaeos montes et eam partem Oceani quae est ad Hispaniam pertinet;
spectat  inter occasum solis et septentriones.

\chap Apud Helvetios longe nobilissimus fuit et ditissimus Orgetorix. Is
M. Messala, [et P.] M. Pisone consulibus regni cupiditate inductus
coniurationem nobilitatis fecit et civitati persuasit ut de finibus suis
cum omnibus copiis exirent: \sect perfacile esse, cum virtute omnibus
praestarent, totius Galliae imperio potiri. \sect Id hoc facilius iis
persuasit, quod undique loci natura Helvetii continentur: una ex parte
flumine Rheno latissimo atque altissimo, qui agrum Helvetium a
Germanis dividit; altera ex parte monte Iura altissimo, qui est inter
Sequanos et Helvetios; tertia lacu Lemanno et flumine Rhodano, qui
provinciam nostram ab Helvetiis dividit. \sect His rebus fiebat ut et minus
late vagarentur et minus facile finitimis bellum inferre possent; qua
ex parte homines bellandi cupidi magno dolore adficiebantur. \sect Pro
multitudine autem hominum et pro gloria belli atque fortitudinis angustos
se fines habere arbitrabantur, qui in longitudinem milia passuum
\textsc{ccxl}, in latitudinem \textsc{clxxx} patebant.

\chap His rebus adducti et auctoritate Orgetorigis permoti
constituerunt ea quae ad proficiscendum pertinerent comparare,
iumentorum et carrorum quam maximum numerum coemere, sementes
quam maximas facere, ut in itinere copia frumenti suppeteret, cum
proximis civitatibus pacem et amicitiam confirmare. \sect Ad eas res
conficiendas biennium sibi satis esse duxerunt; in tertium annum
profectionem lege confirmant. \sect Ad eas res conficiendas Orgetorix
deligitur. Is sibi legationem ad civitates suscipit. In eo itinere
persuadet Castico, Catamantaloedis filio, Sequano, cuius pater regnum
in Sequanis multos annos obtinuerat et a senatu populi Romani amicus
appellatus erat, ut regnum in civitate sua occuparet, quod pater ante
habuerit; \sect itemque Dumnorigi Haeduo, fratri Diviciaci, qui eo
tempore principatum in civitate obtinebat ac maxime plebi acceptus
erat, ut idem conaretur persuadet eique filiam suam in matrimonium
dat. \sect Perfacile factu esse illis probat conata perficere,
propterea quod ipse suae civitatis imperium obtenturus esset: non
esse dubium quin totius Galliae plurimum Helvetii possent; se suis
copiis suoque exercitu illis regna conciliaturum confirmat. \sect Hac
oratione adducti inter se fidem et ius iurandum dant et regno occupato
per tres potentissimos ac firmissimos populos totius Galliae sese
potiri posse sperant.

\chap Ea res est Helvetiis per indicium enuntiata. Moribus suis Orgetoricem
ex vinculis causam dicere coegerunt; damnatum poenam sequi oportebat, ut
igni cremaretur. \sect Die constituta causae dictionis Orgetorix ad
iudicium omnem suam familiam, ad hominum milia decem, undique coegit, et
omnes clientes obaeratosque suos, quorum magnum numerum habebat, eodem
conduxit; per eos ne causam diceret se eripuit. \sect Cum civitas ob eam
rem incitata armis ius suum exequi conaretur multitudinemque hominum ex
agris magistratus cogerent, Orgetorix mortuus est; \sect neque abest
suspicio, ut Helvetii arbitrantur, quin ipse sibi mortem consciverit.
\chap Post eius mortem nihilo minus Helvetii id quod constituerant
facere conantur, ut e finibus suis exeant. \sect Ubi iam se ad eam rem
paratos esse arbitrati sunt, oppida sua omnia, numero ad duodecim,
vicos ad quadringentos, reliqua privata aedificia incendunt; \sect
frumentum omne, praeter quod secum portaturi erant, comburunt, ut
domum reditionis spe sublata paratiores ad omnia pericula subeunda
essent; trium mensum molita cibaria sibi quemque domo efferre iubent.
Persuadent Rauracis et Tulingis et Latobrigis finitimis, uti eodem usi
consilio oppidis suis vicisque exustis una cum iis proficiscantur,
Boiosque, qui trans Rhenum incoluerant et in agrum Noricum transierant
Noreiamque oppugnabant, receptos ad se socios sibi adsciscunt.
\chap Erant omnino itinera duo, quibus itineribus domo exire possent:
unum per Sequanos, angustum et difficile, inter montem Iuram et flumen
Rhodanum, vix qua singuli carri ducerentur, mons autem altissimus
impendebat, ut facile perpauci prohibere possent; \sect alterum per
provinciam nostram, multo facilius atque expeditius, propterea quod
inter fines Helvetiorum et Allobrogum, qui nuper pacati erant,
Rhodanus fluit isque non nullis locis vado transitur. \sect Extremum
oppidum Allobrogum est proximumque Helvetiorum finibus Genava. Ex eo
oppido pons ad Helvetios pertinet. Allobrogibus sese vel persuasuros,
quod nondum bono animo in populum Romanum viderentur, existimabant vel
vi coacturos ut per suos fines eos ire paterentur. Omnibus rebus ad
profectionem comparatis diem dicunt, qua die ad ripam Rhodani omnes
conveniant. Is dies erat a.~d. V. Kal.~Apr. L.~Pisone, A.~Gabinio
consulibus.

\chap \looseness=-1 Caesari  cum id nuntiatum esset, eos per provincia nostram iter
facere conari, maturat ab urbe proficisci et quam maximis potest
itineribus in Galliam ulteriorem contendit et ad Genavam pervenit.
\sect Provinciae toti quam maximum potest militum numerum imperat (erat
omnino in Gallia ulteriore legio una), pontem, qui erat ad Genavam,
iubet rescindi. \sect Ubi de eius aventu Helvetii certiores facti
sunt, legatos ad eum mittunt nobilissimos civitatis, cuius legationis
Nammeius et Verucloetius principem locum obtinebant, qui dicerent
sibi esse in animo sine ullo maleficio iter per provinciam facere,
propterea quod aliud iter haberent nullum: rogare ut eius voluntate id
sibi facere liceat. Caesar, quod memoria tenebat L.~Cassium consulem
occisum exercitumque eius ab Helvetiis pulsum et sub iugum missum,
concedendum non putabat; \sect neque homines inimico animo, data
facultate per provinciam itineris faciundi, temperaturos ab iniuria
et maleficio existimabat. \sect Tamen, ut spatium intercedere posset
dum milites quos imperaverat convenirent, legatis respondit diem se ad
deliberandum sumpturum: si quid vellent, ad Id.~April. reverterentur.  

\chap Interea ea legione quam secum habebat militibusque, qui ex
provincia convenerant, a lacu Lemanno, qui in flumen Rhodanum influit,
ad montem Iuram, qui fines Sequanorum ab Helvetiis dividit, milia
passuum \textsc{xviii} murum in altitudinem pedum sedecim fossamque perducit.
\sect Eo opere perfecto praesidia disponit, castella communit, quo
facilius, si se invito transire conentur, prohibere possit. \sect
Ubi ea dies quam constituerat cum legatis venit et legati ad eum
reverterunt, negat se more et exemplo populi Romani posse iter ulli
per provinciam dare et, si vim lacere conentur, prohibiturum ostendit.
\sect Helvetii ea spe deiecti navibus iunctis ratibusque compluribus
factis, alii vadis Rhodani, qua minima altitudo fluminis erat,
non numquam interdiu, saepius noctu si perrumpere possent conati,
operis munitione et militum concursu et telis repulsi, hoc conatu
destiterunt.
\chap Relinquebatur una per Sequanos via, qua Sequanis invitis propter
angustias ire non poterant. \sect His cum sua sponte persuadere non
possent, legatos ad Dumnorigem Haeduum mittunt, ut eo deprecatore a
Sequanis impetrarent. \sect Dumnorix gratia et largitione apud Sequanos
plurimum poterat et Helvetiis erat amicus, quod ex ea civitate
Orgetorigis filiam in matrimonium duxerat, et cupiditate regni
adductus novis rebus studebat et quam plurimas civitates suo beneficio
habere obstrictas volebat. \sect Itaque rem suscipit et a Sequanis
impetrat ut per fines suos Helvetios ire patiantur, obsidesque uti
inter sese dent perficit: Sequani, ne itinere Helvetios prohibeant,
Helvetii, ut sine maleficio et iniuria transeant.
\chap Caesari renuntiatur Helvetiis esse in animo per agrum
Sequanorum et Haeduorum iter in Santonum fines facere, qui non longe
a Tolosatium finibus absunt, quae civitas est in provincia. \sect
Id si fieret, intellegebat magno cum periculo provinciae futurum
ut homines bellicosos, populi Romani inimicos, locis patentibus
maximeque frumentariis finitimos haberet. \sect Ob eas causas ei
munitioni quam fecerat T.~Labienum legatum praeficit; ipse in
Italiam magnis itineribus contendit duasque ibi legiones conscribit
et tres, quae circum Aquileiam hiemabant, ex hibernis educit et,
qua proximum iter in ulteriorem Galliam per Alpes erat, cum his
quinque legionibus ire contendit. \sect Ibi Ceutrones et Graioceli et
Caturiges locis superioribus occupatis itinere exercitum prohibere
conantur. Compluribus his proeliis pulsis ab Ocelo, quod est
$\langle$oppidum$\rangle$ citerioris provinciae extremum, in fines
Vocontiorum ulterioris provinciae die septimo pervenit; inde in
Allobrogum fines, ab Allobrogibus in Segusiavos exercitum ducit. Hi
sunt extra provinciam trans Rhodanum primi.
\chap Helvetii iam per angustias et fines Sequanorum suas copias
traduxerant et in Haeduorum fines pervenerant eorumque agros
populabantur. \sect Haedui, cum se suaque ab iis defendere non possent,
legatos ad Caesarem mittunt rogatum auxilium: \sect ita se omni tempore
de populo Romano meritos esse ut paene in conspectu exercitus nostri
agri vastari, liberi [eorum] in servitutem abduci, oppida expugnari
non debuerint. Eodem tempore $\langle$quo$\rangle$ Haedui Ambarri,
necessarii et consanguinei Haeduorum, Caesarem certiorem faciunt sese
depopulatis agris non facile ab oppidis vim hostium prohibere. \sect
Item Allobroges, qui trans Rhodanum vicos possessionesque habebant,
fuga se ad Caesarem recipiunt et demonstrant sibi praeter agri solum
nihil esse reliqui. \sect Quibus rebus adductus Caesar non expectandum
sibi statuit dum, omnibus, fortunis sociorum consumptis, in Santonos
Helvetii pervenirent.
\chap Flumen est Arar, quod per fines Haeduorum et Sequanorum in
Rhodanum influit, incredibili lenitate, ita ut oculis in utram partem
fluat iudicari non possit. Id Helvetii ratibus ac lintribus iunctis
transibant. \sect Ubi per exploratores Caesar certior factus est tres
iam partes copiarum Helvetios id flumen traduxisse, quartam vero
partem citra flumen Ararim reliquam esse, de tertia vigilia cum
legionibus tribus e castris profectus ad eam partem pervenit quae
nondum flumen transierat. \sect Eos impeditos et inopinantes adgressus
magnam partem eorum concidit; reliqui sese fugae mandarunt atque in
proximas silvas abdiderunt. \sect Is pagus appellabatur Tigurinus;
nam omnis civitas Helvetia in quattuor pagos divisa est. \sect Hic
pagus unus, cum domo exisset, patrum nostrorum memoria L.~Cassium
consulem interfecerat et eius exercitum sub iugum miserat. \sect
Ita sive casu sive consilio deorum immortalium quae pars civitatis
Helvetiae insignem calamitatem populo Romano intulerat, ea princeps
poenam persolvit. \sect Qua in re Caesar non solum publicas, sed
etiam privatas iniurias ultus est, quod eius soceri L.~Pisonis avum,
L.~Pisonem legatum, Tigurini eodem proelio quo Cassium interfecerant.
\chap Hoc proelio facto, reliquas copias Helvetiorum ut consequi
posset, pontem in Arari faciendum curat atque ita exercitum traducit.
\sect Helvetii repentino eius adventu commoti cum id quod ipsi
diebus \textsc{xx} aegerrime confecerant, ut flumen transirent, illum uno
die fecisse intellegerent, legatos ad eum mittunt; cuius legationis
Divico princeps fuit, qui bello Cassiano dux Helvetiorum fuerat.
\sect Is ita cum Caesare egit: si pacem populus Romanus cum Helvetiis
faceret, in eam partem ituros atque ibi futuros Helvetios ubi
eos Caesar constituisset atque esse voluisset; \sect sin bello
persequi perseveraret, reminisceretur et veteris incommodi populi
Romani et pristinae virtutis Helvetiorum. Quod improviso unum pagum
adortus esset, cum ii qui flumen transissent suis auxilium ferre non
possent, ne ob eam rem aut suae magnopere virtuti tribueret aut ipsos
despiceret. Se ita a patribus maioribusque suis didicisse, ut magis
virtute contenderent quam dolo aut insidiis niterentur. \sect Quare ne
committeret ut is locus ubi constitissent ex calamitate populi Romani
et internecione exercitus nomen caperet aut memoriam proderet.

\chap His Caesar ita respondit: eo sibi minus dubitationis dari, quod
eas res quas legati Helvetii commemorassent memoria teneret, atque eo
gravius ferre quo minus merito populi Romani accidissent; \sect qui si
alicuius iniuriae sibi conscius fuisset, non fuisse difficile cavere;
sed eo deceptum, quod neque commissum a se intellegeret quare timeret
neque sine causa timendum putaret. \sect Quod si veteris contumeliae
oblivisci vellet, num etiam recentium iniuriarum, quod eo invito iter
per provinciam per vim temptassent, quod Haeduos, quod Ambarros, quod
Allobrogas vexassent, memoriam deponere posse? \sect Quod sua victoria
tam insolenter gloriarentur quodque tam diu se impune iniurias tulisse
admirarentur, eodem pertinere. \sect Consuesse enim deos immortales,
quo gravius homines ex commutatione rerum doleant, quos pro scelere
eorum ulcisci velint, his secundiores interdum res et diuturniorem
impunitatem concedere. \sect Cum ea ita sint, tamen, si obsides ab
iis sibi dentur, uti ea quae polliceantur facturos intellegat, et si
Haeduis de iniuriis quas ipsis sociisque eorum intulerint, item si
Allobrogibus satis faciunt, sese cum iis pacem esse facturum. \sect
Divico respondit: ita Helvetios a maioribus suis institutos esse uti
obsides accipere, non dare, consuerint; eius rem populum Romanum esse
testem. Hoc responso dato discessit.

\chap Postero die castra ex eo loco movent. Idem facit Caesar
equitatumque omnem, ad numerum quattuor milium, quem ex omni provincia
et Haeduis atque eorum sociis coactum habebat, praemittit, qui videant
quas in partes hostes iter faciant. Qui cupidius novissimum agmen
insecuti alieno loco cum equitatu Helvetiorum proelium committunt;
et pauci de nostris cadunt. \sect Quo proelio sublati Helvetii, quod
quingentis equitibus tantam multitudinem equitum propulerant, audacius
subsistere non numquam et novissimo agmine proelio nostros lacessere
coeperunt. Caesar suos a proelio continebat, ac satis habebat in
praesentia hostem rapinis, pabulationibus populationibusque prohibere.
\sect ita dies circiter \textsc{xv} iter fecerunt uti inter novissimum hostium
agmen et nostrum primum non amplius quinis aut senis milibus passuum
interesset.

\chap Interim cotidie Caesar Haeduos frumentum, quod essent publice
polliciti, flagitare. \sect Nam propter frigora [quod Gallia sub
septentrionibus, ut ante dictum est, posita est,] non modo frumenta
in agris matura non erant, sed ne pabuli quidem satis magna copia
suppetebat; \sect eo autem frumento quod flumine Arari navibus
subvexerat propterea uti minus poterat quod iter ab Arari Helvetii
averterant, a quibus discedere nolebat. \sect Diem ex die ducere
Haedui: conferri, comportari, adesse dicere. \sect Ubi se diutius
duci intellexit et diem instare quo die frumentum militibus metiri
oporteret, convocatis eorum principibus, quorum magnam copiam in
castris habebat, in his Diviciaco et Lisco, qui summo magistratui
praeerat, quem vergobretum appellant Haedui, qui creatur annuus
et vitae necisque in suos habet potestatem, graviter eos accusat,
\sect quod, cum neque emi neque ex agris sumi possit, tam necessario
tempore, tam propinquis hostibus ab iis non sublevetur, praesertim cum
magna ex parte eorum precibus adductus bellum susceperit[; multo etiam
gravius quod sit destitutus queritur].

\chap Tum demum Liscus oratione Caesaris adductus quod antea tacuerat
proponit: esse non nullos, quorum auctoritas apud plebem plurimum
valeat, qui privatim plus possint quam ipsi magistratus. \sect Hos
seditiosa atque improba oratione multitudinem deterrere, ne frumentum
conferant quod debeant: \sect praestare, si iam principatum Galliae
obtinere non possint, Gallorum quam Romanorum imperia perferre, \sect
neque dubitare [debeant] quin, si Helvetios superaverint Romani, una
cum reliqua Gallia Haeduis libertatem sint erepturi. \sect Ab isdem
nostra consilia quaeque in castris gerantur hostibus enuntiari; hos a
se coerceri non posse. \sect Quin etiam, quod necessariam rem coactus
Caesari enuntiarit, intellegere sese quanto id cum periculo fecerit,
et ob eam causam quam diu potuerit tacuisse.

\chap Caesar hac oratione Lisci Dumnorigem, Diviciaci fratrem,
designari sentiebat, sed, quod pluribus praesentibus eas res iactari
nolebat, celeriter concilium dimittit, Liscum retinet. \sect Quaerit
ex solo ea quae in conventu dixerat. Dicit liberius atque audacius.
Eadem secreto ab aliis quaerit; reperit esse vera: \sect ipsum esse
Dumnorigem, summa audacia, magna apud plebem propter liberalitatem
gratia, cupidum rerum novarum. Complures annos portoria reliquaque
omnia Haeduorum vectigalia parvo pretio redempta habere, propterea
quod illo licente contra liceri audeat nemo. \sect His rebus et suam
rem familiarem auxisse et facultates ad largiendum magnas comparasse;
\sect magnum umerum equitatus suo sumptu semper alere et circum se
habere, \sect neque solum domi, sed etiam apud finitimas civitates
largiter posse, atque huius potentiae causa matrem in Biturigibus
homini illic nobilissimo ac potentissimo conlocasse; \sect ipsum ex
Helvetiis uxorem habere, sororum ex matre et propinquas suas nuptum in
alias civitates conlocasse. \sect Favere et cupere Helvetiis propter
eam adfinitatem, odisse etiam suo nomine Caesarem et Romanos, quod
eorum adventu potentia eius deminuta et Diviciacus frater in antiquum
locum gratiae atque honoris sit restitutus. \sect Si quid accidat
Romanis, summam in spem per Helvetios regni obtinendi venire; imperio
populi Romani non modo de regno, sed etiam de ea quam habeat gratia
desperare. \sect Reperiebat etiam in quaerendo Caesar, quod proelium
equestre adversum paucis ante diebus esset factum, initium eius fugae
factum a Dumnorige atque eius equitibus (nam equitatui, quem auxilio
Caesari Haedui miserant, Dumnorix praeerat): eorum fuga reliquum esse
equitatum perterritum.

\chap Quibus rebus cognitis, cum ad has suspiciones certissimae res
accederent, quod per fines Sequanorum Helvetios traduxisset, quod
obsides inter eos dandos curasset, quod ea omnia non modo iniussu
suo et civitatis sed etiam inscientibus ipsis fecisset, quod a
magistratu Haeduorum accusaretur, satis esse causae arbitrabatur quare
in eum aut ipse animadverteret aut civitatem animadvertere iuberet.
\sect His omnibus rebus unum repugnabat, quod Diviciaci fratris
summum in populum Romanum studium, summum in se voluntatem, egregiam
fidem, iustitiam, temperantiam cognoverat; nam ne eius supplicio
Diviciaci animum offenderet verebatur. \sect Itaque prius quam quicquam
conaretur, Diviciacum ad se vocari iubet et, cotidianis interpretibus
remotis, per C.~Valerium Troucillum, principem Galliae provinciae,
familiarem suum, cui summam omnium rerum fidem habebat, cum eo
conloquitur; \sect simul commonefacit quae ipso praesente in concilio
[Gallorum] de Dumnorige sint dicta, et ostendit quae separatim
quisque de eo apud se dixerit. \sect Petit atque hortatur ut sine eius
offensione animi vel ipse de eo causa cognita statuat vel civitatem
statuere iubeat.

\chap Diviciacus multis cum lacrimis Caesarem complexus obsecrare
coepit ne quid gravius in fratrem statueret: \sect scire se illa esse
vera, nec quemquam ex eo plus quam se doloris capere, propterea quod,
cum ipse gratia plurimum domi atque in reliqua Gallia, ille minimum
propter adulescentiam posset, per se crevisset; \sect quibus opibus
ac nervis non solum ad minuendam gratiam, sed paene ad perniciem
suam uteretur. Sese tamen et amore fraterno et existimatione vulgi
commoveri. \sect Quod si quid ei a Caesare gravius accidisset, cum
ipse eum locum amicitiae apud eum teneret, neminem existimaturum non
sua voluntate factum; qua ex re futurum uti totius Galliae animi a se
averterentur. \sect Haec cum pluribus verbis flens a Caesare peteret,
Caesar eius dextram prendit; consolatus rogat finem orandi faciat;
tanti eius apud se gratiam esse ostendit uti et rei publicae iniuriam
et suum dolorem eius voluntati ac precibus condonet. Dumnorigem ad se
vocat, fratrem adhibet; quae in eo reprehendat ostendit; quae ipse
intellegat, quae civitas queratur proponit; monet ut in reliquum
tempus omnes suspiciones vitet; praeterita se Diviciaco fratri
condonare dicit. Dumnorigi custodes ponit, ut quae agat, quibuscum
loquatur scire possit.

\chap Eodem die ab exploratoribus certior factus hostes
sub monte consedisse milia passuum ab ipsius castris
octo, qualis esset natura montis et qualis in circuitu
ascensus qui cognoscerent misit.\sect  Renuntiatum est
facilem esse. De tertia vigilia T.~Labienum, legatum
pro praetore, cum duabus legionibus et iis ducibus
qui iter cognoverant summum iugum montis ascendere
iubet; quid sui consilii sit ostendit. \sect Ipse de quarta
vigilia eodem itinere quo hostes ierant ad eos
contendit equitatumque omnem ante se mittit. \sect
P.~Considius, qui rei militaris peritissimus habebatur et
in exercitu L.~Sullae et postea in M.~Crassi fuerat, cum
exploratoribus praemittitur.

\chap Prima luce, cum summus mons a [Lucio] Labieno teneretur, ipse
ab hostium castris non longius mille et quingentis passibus abesset
neque, ut postea ex captivis comperit, aut ipsius adventus aut Labieni
cognitus esset, \sect Considius equo admisso ad eum accurrit, dicit
montem, quem a Labieno occupari voluerit, ab hostibus teneri: id se a
Gallicis armis atque insignibus cognovisse. \sect Caesar suas copias
in proximum collem subducit, aciem instruit. Labienus, ut erat ei
praeceptum a Caesare ne proelium committeret, nisi ipsius copiae prope
hostium castra visae essent, ut undique uno tempore in hostes impetus
fieret, monte occupato nostros expectabat proelioque abstinebat. \sect
Multo denique die per exploratores Caesar cognovit et montem a suis
teneri et Helvetios castra, movisse et Considium timore perterritum
quod non vidisset pro viso sibi renuntiavisse. Eo die quo consuerat
intervallo hostes sequitur et milia passuum tria ab eorum castris
castra ponit.

\chap Postridie eius diei, quod omnino biduum supererat, cum exercitui
frumentum metiri oporteret, et quod a Bibracte, oppido Haeduorum longe
maximo et copiosissimo, non amplius milibus passuum \textsc{xviii} aberat,
rei frumentariae prospiciendum existimavit; $\langle$itaque$\rangle$
iter ab Helvetiis avertit ac Bibracte ire contendit. \sect Ea res per
fugitivos L.~Aemilii, decurionis equitum Gallorum, hostibus nuntiatur.
\sect Helvetii, seu quod timore perterritos Romanos discedere a se
existimarent, eo magis quod pridie superioribus locis occupatis
proelium non commisissent, sive eo quod re frumentaria intercludi
posse confiderent, commutato consilio atque itinere converso nostros a
novissimo agmine insequi ac lacessere coeperunt.

\chap Postquam id animum advertit, copias suas Caesar in proximum
collem subduxit equitatumque, qui sustineret hostium petum, misit.
\sect Ipse interim in colle medio triplicem aciem instruxit legionum
quattuor veteranarum; in summo iugo duas legiones quas in Gallia
citeriore proxime conscripserat et omnia auxilia conlocavit, \sect
ita ut supra se totum montem hominibus compleret; impedimenta
sarcinasque in unum locum conterri et eum ab iis qui in superiore acie
constiterant muniri iussit. \sect Helvetii cum omnibus suis carris
secuti impedimenta in unum locum contulerunt; ipsi concertissima acie,
reiecto nostro equitatu, phalange facta sub primam nostram aciem
successerunt.

\chap Caesar primum suo, deinde omnium ex conspectu remotis equis, ut
aequato omnium periculo spem fugae tolleret, cohortatus suos proelium
commisit. \sect Milites loco superiore pilis missis facile hostium
phalangem perfregerunt. Ea disiecta gladiis destrictis in eos impetum
fecerunt. \sect Gallis magno ad pugnam erat impedimento quod pluribus
eorum scutis uno ictu pilorum transfixis et conligatis, cum ferrum
se inflexisset, neque evellere neque sinistra impedita satis commode
pugnare poterant, \sect multi ut diu iactato bracchio praeoptarent
scutum manu emittere et nudo corpore pugnare. \sect Tandem vulneribus
defessi et pedem referre et, quod mons suberit circiter mille passuum
$\langle$spatio$\rangle$, eo se recipere coeperunt. \sect Capto monte
et succedentibus nostris, Boi et Tulingi, qui hominum milibus circiter
\textsc{xv} agmen hostium claudebant et novissimis praesidio erant, ex itinere
nostros $\langle$ab$\rangle$ latere aperto adgressi circumvenire, et
id conspicati Helvetii, qui in montem sese receperant, rursus instare
et proelium redintegrare coeperunt. \sect Romani [conversa] signa
bipertito intulerunt: prima et secunda acies, ut victis ac submotis
resisteret, tertia, ut venientes sustineret.

\chap Ita ancipiti proelio diu atque acriter pugnatum est. Diutius
cum sustinere nostrorum impetus non possent, alteri se, ut coeperant,
in montem receperunt, alteri ad impedimenta et carros suos se
contulerunt. \sect Nam hoc toto proelio, cum ab hora septima ad
vesperum pugnatum sit, aversum hostem videre nemo potuit. \sect Ad
multam noctem etiam ad impedimenta pugnatum est, propterea quod pro
vallo carros obiecerunt et e loco superiore in nostros venientes tela
coiciebant et non nulli inter carros rotasque mataras ac tragulas
subiciebant nostrosque vulnerabant. \sect Diu cum esset pugnatum,
impedimentis castrisque nostri potiti sunt. Ibi Orgetorigis filia
atque unus e filiis captus est. \sect Ex eo proelio circiter hominum
milia \textsc{cxxx} superfuerunt eaque tota nocte continenter ierunt [nullam
partem noctis itinere intermisso]; in fines Lingonum die quarto
pervenerunt, cum et propter vulnera militum et propter sepulturam
occisorum nostri [triduum morati] eos sequi non potuissent. \sect
Caesar ad Lingonas litteras nuntiosque misit, ne eos frumento neve
alia re iuvarent: qui si iuvissent, se eodem loco quo Helvetios
habiturum. Ipse triduo intermisso cum omnibus copiis eos sequi coepit.

\chap Helvetii omnium rerum inopia adducti legatos de deditione ad
eum miserunt. \sect Qui cum eum in itinere convenissent seque ad pedes
proiecissent suppliciterque locuti flentes pacem petissent, atque
eos in eo loco quo tum essent suum adventum expectare iussisset,
paruerunt. \sect Eo postquam Caesar pervenit, obsides, arma, servos
qui ad eos perfugissent, poposcit. \sect Dum ea conquiruntur et
conferuntur, [nocte intermissa] circiter hominum milia \textsc{vi} eius pagi
qui Verbigenus appellatur, sive timore perterriti, ne armis traditis
supplicio adficerentur, sive spe salutis inducti, quod in tanta
multitudine dediticiorum suam fugam aut occultari aut omnino ignorari
posse existimarent, prima nocte e castris Helvetiorum egressi ad
Rhenum finesque Germanorum contenderunt.

\chap Quod ubi Caesar resciit, quorum per fines ierant his uti
conquirerent et reducerent, si sibi purgati esse vellent, imperavit;
reductos in hostium numero habuit; \sect reliquos omnes obsidibus,
armis, perfugis traditis in deditionem accepit. \sect Helvetios,
Tulingos, Latobrigos in fines suos, unde erant profecti, reverti
iussit, et, quod omnibus frugibus amissis domi nihil erat quo famem
tolerarent, Allobrogibus imperavit ut iis frumenti copiam facerent;
ipsos oppida vicosque, quos incenderant, restituere iussit. Id ea
maxime ratione fecit, quod noluit eum locum unde Helvetii discesserant
vacare, ne propter bonitatem agrorum Germani, qui trans Rhenum
incolunt, $\langle$ex$\rangle$ suis finibus in Helvetiorum fines
transirent et finitimi Galliae provinciae Allobrogibusque essent.
\sect Boios petentibus Haeduis, quod egregia virtute erant cogniti, ut
in finibus suis conlocarent, concessit; quibus illi agros dederunt
quosque postea in parem iuris libertatisque condicionem atque ipsi
erant receperunt.

\chap In castris Hevetiorum tabulae repertae sunt litteris Graecis
confectae et ad Caesarem relatae, quibus in tabulis nominatim ratio
confecta erat, qui numerus domo exisset eorum qui arma ferre possent,
et item separatim, $\langle$quot$\rangle$ pueri, senes mulieresque.
\sect [Quarum omnium rerum] summa erat capitum Helvetiorum milium
\textsc{cclxiii}, Tulingorum milium \textsc{xxxvi}, Latobrigorum 
\textsc{xiiii}, Rauracorum \textsc{xxiii}, Boiorum \textsc{xxxii}; ex his
qui arma ferre possent ad milia nonaginta
duo. \sect Summa omnium fuerunt ad milia \textsc{ccclxviii}. Eorum qui domum
redierunt censu habito, ut Caesar imperaverat, repertus est numerus
milium C et X.

\chap Bello Helvetiorum confecto totius fere Galliae legati, principes
civitatum, ad Caesarem gratulatum convenerunt: \sect in\-tel\-le\-ge\-re
sese, tametsi pro veteribus Helvetiorum iniuriis populi Romani ab
his poenas bello repetisset, tamen eam rem non minus ex usu [terrae]
Galliae quam populi Romani accidisse, \sect propterea quod eo consilio
florentissimis rebus domos suas Helvetii reliquissent uti toti Galliae
bellum inferrent imperioque potirentur, locumque domicilio ex magna
copia deligerent quem ex omni Gallia oportunissimum ac fructuosissimum
iudicassent, reliquasque civitates stipendiarias haberent. \sect
Petierunt uti sibi concilium totius Galliae in diem certam indicere
idque Caesaris facere voluntate liceret: sese habere quasdam res quas
ex communi consensu ab eo petere vellent. \sect Ea re permissa diem
concilio constituerunt et iure iurando ne quis enuntiaret, nisi quibus
communi consilio mandatum esset, inter se sanxerunt.

\chap Eo concilio dimisso, idem princeps civitatum qui ante fuerant
ad Caesarem reverterunt petieruntque uti sibi secreto in occulto
de sua omniumque salute cum eo agere liceret. \sect Ea re impetrata
sese omnes flentes Caesari ad pedes proiecerunt: non minus se id
contendere et laborare ne ea quae dixissent enuntiarentur quam uti
ea quae vellent impetrarent, propterea quod, si enuntiatum esset,
summum in cruciatum se venturos viderent. \sect Locutus est pro his
Diviciacus Haeduus: Galliae totius lactiones esse duas; harum alterius
principatum tenere Haeduos, alterius Arvernos. \sect Hi cum tantopere
de potentatu inter se multos annos contenderent, factum esse uti ab
Arvernis Sequanisque Germani mercede arcesserentur. \sect Horum primo
circiter milia \textsc{xv} Rhenum transisse; postea quam agros et cultum et
copias Gallorum homines feri ac barbari adamassent, traductos plures;
nunc esse in gallia ad c et \textsc{xx} milium numerum. \sect cum his haeduos
eorumque clientes semel atque iterum armis contendisse; magnam
calamitatem pulsos accepisse, omnem nobilitatem, omnem senatum,
omnem equitatum amisisse. \sect Quibus proeliis calamitatibusque
fractos, qui et sua virtute et populi Romani hospitio atque amicitia
plurimum ante in Gallia potuissent, coactos esse Sequanis obsides dare
nobilissimos civitatis et iure iurando civitatem obstringere sese
neque obsides repetituros neque auxilium a populo Romano imploraturos
neque recusaturos quo minus perpetuo sub illorum dicione atque
imperio essent. \sect Unum se esse ex omni civitate Haeduorum qui
adduci non potuerit ut iuraret aut liberos suos obsides daret. \sect
Ob eam rem se ex civitate profugisse et Romam ad senatum venisse
auxilium postulatum, quod solus neque iure iurando neque obsidibus
teneretur. \sect Sed peius victoribus Sequanis quam Haeduis victis
accidisse, propterea quod Ariovistus, rex Germanorum, in eorum finibus
consedisset tertiamque partem agri Sequani, qui esset optimus totius
Galliae, occupavisset et nunc de altera parte tertia Sequanos decedere
iuberet, propterea quod paucis mensibus ante Harudum milia hominum
\textsc{xxiiii} ad eum venissent, quibus locus ac sedes pararentur. \sect
Futurum esse paucis annis uti omnes ex Galliae finibus pellerentur
atque omnes Germani Rhenum transirent; neque enim conferendum esse
Gallicum cum Germanorum agro neque hanc consuetudinem victus cum
illa comparandam. \sect Ariovistum autem, ut semel Gallorum copias
proelio vicerit, quod proelium factum sit ad Magetobrigam, superbe et
crudeliter imperare, obsides nobilissimi cuiusque liberos poscere et
in eos omnia exempla cruciatusque edere, si qua res non ad nutum aut
ad voluntatem eius facta sit. \sect Hominem esse barbarum, iracundum,
temerarium: non posse eius imperia, diutius sustineri. \sect Nisi quid
in Caesare populoque Romano sit auxilii, omnibus Gallis idem esse
faciendum quod Helvetii fecerint, ut domo emigrent, aliud domicilium,
alias sedes, remotas a Germanis, petant fortunamque, quaecumque
accidat, experiantur. Haec si enuntiata Ariovisto sint, non dubitare
quin de omnibus obsidibus qui apud eum sint gravissimum supplicium
sumat. \sect Caesarem vel auctoritate sua atque exercitus vel recenti
victoria vel nomine populi Romani deterrere posse ne maior multitudo
Germanorum Rhenum traducatur, Galliamque omnem ab Ariovisti iniuria
posse defendere.

\chap Hac oratione ab Diviciaco habita omnes qui aderant magno fletu
auxilium a Caesare petere coeperunt. \sect Animadvertit Caesar unos
ex omnibus Sequanos nihil earum rerum facere quas ceteri facerent
sed tristes capite demisso terram intueri. Eius rei quae causa
esset miratus ex ipsis quaesiit. \sect Nihil Sequani respondere, sed
in eadem tristitia taciti permanere. Cum ab his saepius quaereret
neque ullam omnino vocem exprimere posset, idem Diviacus Haeduus
respondit: \sect hoc esse miseriorem et graviorem fortunam Sequanorum
quam reliquorum, quod soli ne in occulto quidem queri neque auxilium
implorare auderent absentisque Ariovisti crudelitatem, \sect velut si
cora adesset, horrerent, propterea quod reliquis tamen fugae facultas
daretur, Sequanis vero, qui intra fines suos Ariovistum recepissent,
quorum oppida omnia in potestate eius essent, omnes cruciatus essent
perferendi.

\chap His rebus cognitis Caesar Gallorum animos verbis confirmavit
pollicitusque est sibi eam rem curae futuram; magnam se habere spem
et beneficio suo et auctoritate adductum Ariovistum finem iniuriis
facturum. Hac oratione habita, concilium dimisit. \sect Et secundum
ea multae res eum hortabantur quare sibi eam rem cogitandam et
suscipiendam putaret, in primis quod Haeduos, fratres consanguineosque
saepe numero a senatu appellatos, in servitute atque [in] dicione
videbat Germanorum teneri eorumque obsides esse apud Ariovistum
ac Sequanos intellegebat; quod in tanto imperio populi Romani
turpissimum sibi et rei publicae esse arbitrabatur. \sect Paulatim
autem Germanos consuescere Rhenum transire et in Galliam magnam
eorum multitudinem venire populo Romano periculosum videbat, neque
sibi homines feros ac barbaros temperaturos existimabat quin, cum
omnem Galliam occupavissent, ut ante Cimbri Teutonique fecissent, in
provinciam exirent atque inde in Italiam contenderent [, praesertim
cum Sequanos a provincia nostra Rhodanus divideret]; quibus rebus quam
maturrime occurrendum putabat. \sect Ipse autem Ariovistus tantos sibi
spiritus, tantam arrogantiam sumpserat, ut ferendus non videretur.

\chap Quam ob rem placuit ei ut ad Ariovistum legatos mitteret, qui ab
eo postularent uti aliquem locum medium utrisque conloquio deligeret:
velle sese de re publica et summis utriusque rebus cum eo agere.
\sect Ei legationi Ariovistus respondit: si quid ipsi a Caesare opus
esset, sese ad eum venturum fuisse; si quid ille se velit, illum ad se
venire oportere. \sect Praeterea se neque sine exercitu in eas partes
Galliae venire audere quas Caesar possideret, neque exercitum sine
magno commeatu atque molimento in unum locum contrahere posse. \sect
Sibi autem mirum videri quid in sua Gallia, quam bello vicisset, aut
Caesari aut omnino populo Romano negotii esset.

\chap His responsis ad Caesarem relatis, iterum ad eum Caesar legatos
cum his mandatis mittit: \sect quoniam tanto suo populique Romani
beneficio adtectus, cum in consulatu suo rex atque amicus a senatu
appellatus esset, hanc sibi populoque Romano gratiam referret ut in
conloquium venire invitatus gravaretur neque de communi re dicendum
sibi et cognoscendum putaret, haec esse quae ab eo postularet:
\sect primum ne quam multitudinem hominum amplius trans Rhenum in
Galliam traduceret; deinde obsides quos haberet ab Haeduis redderet
Sequanisque permitteret ut quos illi haberent voluntate eius reddere
illis liceret; neve Haeduos iniuria lacesseret neve his sociisque
eorum bellum inferret. \sect Si [id] ita fecisset, sibi populoque
Romano perpetuam gratiam atque amicitiam cum eo futuram; si non
impetraret, sese, quoniam M.~Messala, M.~Pisone consulibus senatus
censuisset uti quicumque Galliam provinciam obtineret, quod commodo
rei publicae lacere posset, Haeduos ceterosque amicos populi Romani
defenderet, se Haeduorum iniurias non neglecturum.

\chap Ad haec Ariovistus respondit: ius esse belli ut qui vicissent
iis quos vicissent quem ad modum vellent imperarent. Item populum
Romanum victis non ad alterius praescriptum, sed ad suum arbitrium
imperare consuesse. \sect Si ipse populo Romano non praescriberet quem
ad modum suo iure uteretur, non oportere se a populo Romano in suo
iure impediri. \sect Haeduos sibi, quoniam belli fortunam temptassent
et armis congressi ac superati essent, stipendiarios esse factos.
\sect Magnam Caesarem iniuriam facere, qui suo adventu vectigalia
sibi deteriora faceret. \sect Haeduis se obsides redditurum non esse
neque his neque eorum sociis iniuria bellum inlaturum, si in eo
manerent quod convenisset stipendiumque quotannis penderent; si id non
fecissent, longe iis fraternum nomen populi Romani afuturum. \sect Quod
sibi Caesar denuntiaret se Haeduorum iniurias non neglecturum, neminem
secum sine sua pernicie contendisse. \sect Cum vellet, congrederetur:
intellecturum quid invicti Germani, exercitatissimi in armis, qui
inter annos \textsc{xiiii} tectum non subissent, virtute possent.

\chap Haec eodem tempore Caesari mandata referebantur et legati ab
Haeduis et a Treveris veniebant: \sect Haedui questum quod Harudes,
qui nuper in Galliam transportati essent, fines eorum popularentur:
sese ne obsidibus quidem datis pacem Ariovisti redimere potuisse; \sect
Treveri autem, pagos centum Sueborum ad ripas Rheni consedisse, qui
Rhemum transire conarentur; his praeesse Nasuam et Cimberium fratres.
Quibus rebus Caesar vehementer commotus maturandum sibi existimavit,
ne, si nova manus Sueborum cum veteribus copiis Ariovisti sese
coniunxisset, minus facile resisti posset. \sect Itaque re frumentaria
quam celerrime potuit comparata magnis itineribus ad Ariovistum
contendit.

\chap Cum tridui viam processisset, nuntiatum est ei Ariovistum cum
suis omnibus copiis ad occupandum Vesontionem, quod est oppidum
maximum Sequanorum, contendere [triduique viam a suis finibus
processisse]. Id ne accideret, magnopere sibi praecavendum Caesar
existimabat. Namque omnium rerum quae ad bellum usui erant summa
erat in eo oppido facultas, \sect idque natura loci sic muniebatur ut
magnam ad ducendum bellum daret facultatem, propterea quod flumen
[alduas] Dubis ut circino circumductum paene totum oppidum cingit,
\sect reliquum spatium, quod est non amplius pedum \textsc{mdc}, qua flumen
intermittit, mons continet magna altitudine, ita ut radices eius
montis ex utraque parte ripae fluminis contingant, \sect hunc murus
circumdatus arcem efficit et cum oppido coniungit. \sect Huc Caesar
magnis nocturnis diurnisque itineribus contendit occupatoque oppido
ibi praesidium conlocat.

\chap \looseness=1 Dum paucos dies ad Vesontionem rei frumentariae commeatusque
causa moratur, ex percontatione nostrorum vocibusque Gallorum ac
mercatorum, qui ingenti magnitudine corporum Germanos, incredibili
virtute atque exercitatione in armis esse praedicabant (saepe numero
sese cum his congressos ne vultum quidem atque aciem oculorum dicebant
ferre potuisse), tantus subito timor omnem exercitum occupavit ut non
mediocriter omnium mentes animosque perturbaret. \sect Hic primum ortus
est a tribunis militum, praefectis, reliquisque qui ex urbe amicitiae
causa Caesarem secuti non magnum in re militari usum habebant:
\sect quorum alius alia causa inlata, quam sibi ad proficiscendum
necessariam esse diceret, petebat ut eius voluntate discedere liceret;
non nulli pudore adducti, ut timoris suspicionem vitarent, remanebant.
\sect Hi neque vultum fingere neque interdum lacrimas tenere poterant:
abditi in tabernaculis aut suum fatum querebantur aut cum familiaribus
suis commune periculum miserabantur. Vulgo totis castris testamenta
obsignabantur. \sect Horum vocibus ac timore paulatim etiam ii qui
magnum in castris usum habebant, milites centurionesque quique
equitatui praeerant, perturbabantur. \sect Qui se ex his minus timidos
existimari volebant, non se hostem vereri, sed angustias itineris et
magnitudinem silvarum quae intercederent inter ipsos atque Ariovistum,
aut rem frumentariam, ut satis commode supportari posset, timere
dicebant. \sect Non nulli etiam Caesari nuntiabant, cum castra moveri
ac signa ferri iussisset, non fore dicto audientes milites neque
propter timorem signa laturos.

\chap Haec cum animadvertisset, convocato consilio omniumque ordinum
ad id consilium adhibitis centurionibus, vehementer eos incusavit:
primum, quod aut quam in partem aut quo consilio ducerentur sibi
quaerendum aut cogitandum putarent. \sect Ariovistum se consule
cupidissime populi Romani amicitiam adpetisse; cur hunc tam temere
quisquam ab officio discessurum iudicaret? \sect Sibi quidem persuaderi
cognitis suis poslulatis atque aequitate condicionum perspecta
eum neque suam neque populi Romani gratiam epudiaturum. \sect
Quod si furore atque amentia impulsum bellum intulisset, quid
tandem vererentur? Aut cur de sua virtute aut de ipsius diligentia
desperarent? \sect Factum eius hostis periculum patrum nostrorum
emoria Cimbris et Teutonis a C.~Mario pulsis [cum non minorem laudem
exercitus quam ipse imperator meritus videbatur]; factum etiam nuper
in Italia servili tumultu, quos tamen aliquid usus ac disciplina, quam
a nobis accepissent, sublevarint. \sect Ex quo iudicari posse quantum
haberet in se boni constantia, propterea quod quos aliquam diu inermes
sine causa timuissent hos postea armatos ac victores superassent.
\sect Denique hos esse eosdem Germanos quibuscum saepe numero Helvetii
congressi non solum in suis sed etiam in illorum finibus plerumque
superarint, qui tamen pares esse nostro exercitui non potuerint.
\sect Si quos adversum proelium et fuga Gallorum commoveret, hos, si
quaererent, reperire posse diuturnitate belli defatigatis Gallis
Ariovistum, cum multos menses castris se ac paludibus tenuisset
neque sui potestatem fecisset, desperantes iam de pugna et dispersos
subito adortum magis ratione et consilio quam virtute vicisse. \sect
Cui rationi contra homines barbaros atque imperitos locus fuisset,
hac ne ipsum quidem sperare nostros exercitus capi posse. \sect Qui
suum timorem in rei frumentariae simulationem angustiasque itineris
conferrent, facere arroganter, cum aut de officio imperatoris
desperare aut praescribere viderentur. \sect Haec sibi esse curae;
frumentum Sequanos, Leucos, Lingones subministrare, iamque esse in
agris frumenta matura; de itinere ipsos brevi tempore iudicaturos.
\sect Quod non fore dicto audientes neque signa laturi dicantur, nihil
se ea re commoveri: scire enim, quibuscumque exercitus dicto audiens
non fuerit, aut male re gesta fortunam defuisse aut aliquo facinore
comperto avaritiam esse convictam. \sect Suam innocentiam perpetua
vita, felicitatem Helvetiorum bello esse perspectam. \sect Itaque se
quod in longiorem diem conlaturus fuisset repraesentaturum et proxima
nocte de quarta, vigilia castra moturum, ut quam primum intellegere
posset utrum apud eos pudor atque officium an timor plus valeret. \sect
Quod si praeterea nemo sequatur, tamen se cum sola decima legione
iturum, de qua non dubitet, sibique eam praetoriam cohortem futuram.
Huic legioni Caesar et indulserat praecipue et propter virtutem
confidebat maxime.

\chap \looseness=1 Hac oratione habita mirum in modum conversae sunt omnium mentes
summaque alacritas et cupiditas belli gerendi innata est, \sect
princepsque X. legio per tribunos militum ei gratias egit quod de se
optimum iudicium fecisset, seque esse ad bellum gerendum paratissimam
confirmavit. \sect Deinde reliquae legiones cum tribunis militum et
primorum ordinum centurionibus egerunt uti Caesari satis facerent:
se neque umquam dubitasse neque timuisse neque de summa belli suum
iudicium sed imperatoris esse existimavisse. \sect Eorum satisfactione
accepta et itinere exquisito per Diviciacum, quod ex Gallis ei maximam
fidem habebat, ut milium amplius quinquaginta circuitu locis apertis
exercitum duceret, de quarta vigilia, ut dixerat, profectus est. \sect
Septimo die, cum iter non intermitteret, ab exploratoribus certior
factus est Ariovisti copias a nostris milia passuum \textsc{iiii}
et \textsc{xx} abesse.

\chap Cognito Caesaris adventu Ariovistus legatos ad eum mittit:
quod antea de conloquio postulasset, id per se fieri licere, quoniam
propius accessisset seque id sine periculo facere posse existimaret.
\sect Non respuit condicionem Caesar iamque eum ad sanitatem reverti
arbitrabatur, cum id quod antea petenti denegasset ultro polliceretur,
\sect magnamque in spem veniebat pro suis tantis populique Romani
in eum beneficiis cognitis suis postulatis fore uti pertinacia
desisteret. \sect Dies conloquio dictus est ex eo die quintus. \sect
Interim saepe cum legati ultro citroque inter eos mitterentur,
Ariovistus postulavit ne quem peditem ad conloquium Caesar adduceret:
vereri se ne per insidias ab eo circumveniretur; uterque cum equitatu
veniret: alia ratione sese non esse venturum. \sect Caesar, quod neque
conloquium interposita causa tolli volebat neque salutem suam Gallorum
equitatui committere audebat, commodissimum esse statuit omnibus equis
Gallis equitibus detractis eo legionarios milites legionis X., cui
quam maxime confidebat, imponere, ut praesidium quam amicissimum, si
quid opus facto esset, haberet. \sect Quod cum fieret, non inridicule
quidam ex militibus X. legionis dixit: plus quam pollicitus esset
Caesarem facere; pollicitum se in cohortis praetoriae loco X. legionem
habiturum ad equum rescribere.


\chap Planities erat magna et in ea tumulus terrenus satis grandis.
Hic locus aequum fere spatium a castris Ariovisti et Caesaris aberat.
Eo, ut erat dictum, ad conloquium venerunt. \sect Legionem Caesar,
quam equis devexerat, passibus \textsc{cc} ab eo tumulo constituit. item
equites Ariovisti pari intervallo constiterunt. \sect Ariovistus ex
equis ut conloquerentur et praeter se denos ad conloquium adducerent
postulavit. \sect Ubi eo ventum est, Caesar initio orationis sua
senatusque in eum beneficia commemoravit, quod rex appellatus esset a
senatu, quod amicus, quod munera amplissime missa; quam rem et paucis
contigisse et pro magnis hominum officiis consuesse tribui docebat;
\sect illum, cum neque aditum neque causam postulandi iustam haberet,
beneficio ac liberalitate sua ac senatus ea praemia consecutum. \sect
Docebat etiam quam veteres quamque iustae causae necessitudinis
ipsis cum Haeduis intercederent, \sect quae senatus consulta quotiens
quamque honorifica in eos facta essent, ut omni tempore totius Galliae
principatum Haedui tenuissent, prius etiam quam nostram amicitiam
adpetissent. \sect Populi Romani hanc esse consuetudinem, ut socios
atque amicos non modo sui nihil deperdere, sed gratia, dignitate,
honore auctiores velit esse; quod vero ad amicitiam populi Romani
attulissent, id iis eripi quis pati posset? \sect Postulavit deinde
eadem quae legatis in mandatis dederat: ne aut Haeduis aut eorum
sociis bellum inferret, obsides redderet, si nullam partem Germanorum
domum remittere posset, at ne quos amplius Rhenum transire pateretur.

\chap Ariovistus ad postulata Caesaris pauca respondit, de suis
virtutibus multa praedicavit: \sect transisse Rhenum sese non sua
sponte, sed rogatum et arcessitum a Gallis; non sine magna spe
magnisque praemiis domum propinquosque reliquisse; sedes habere in
Gallia ab ipsis concessas, obsides ipsorum voluntate datos; stipendium
capere iure belli, quod victores victis imponere consuerint. \sect Non
sese Gallis sed Gallos sibi bellum intulisse: omnes Galliae civitates
ad se oppugnandum venisse ac contra se castra habuisse; eas omnes
copias a se uno proelio pulsas ac superatas esse. \sect Si iterum
experiri velint, se iterum paratum esse decertare; si pace uti velint,
iniquum esse de stipendio recusare, quod sua voluntate ad id tempus
pependerint. \sect Amicitiam populi Romani sibi ornamento et praesidio,
non detrimento esse oportere, atque se hac spe petisse. Si per populum
Romanum stipendium remittatur et dediticii subtrahantur, non minus
libenter sese recusaturum populi Romani amicitiam quam adpetierit.
\sect Quod multitudinem Germanorum in Galliam traducat, id se sui
muniendi, non Galliae oppugnandae causa facere; eius rei testimonium
esse quod nisi rogatus non venerit et quod bellum non intulerit sed
defenderit. \sect Se prius in Galliam venisse quam populum Romanum.
Numquam ante hoc tempus exercitum populi Romani Galliae provinciae
finibus egressum. \sect Quid sibi vellet? Cur in suas possessiones
veniret? Provinciam suam hanc esse Galliam, sicut illam nostram. Ut
ipsi concedi non oporteret, si in nostros fines impetum faceret, sic
item nos esse iniquos, quod in suo iure se interpellaremus. \sect Quod
fratres a senatu Haeduos appellatos diceret, non se tam barbarum neque
tam imperitum esse rerum ut non sciret neque bello Allobrogum proximo
Haeduos Romanis auxilium tulisse neque ipsos in iis contentionibus
quas Haedui secum et cum Sequanis habuissent auxilio populi Romani
usos esse. \sect Debere se suspicari simulata Caesarem amicitia, quod
exercitum in Gallia habeat, sui opprimendi causa habere. \sect Qui nisi
decedat atque exercitum deducat ex his regionibus, sese illum non pro
amico sed pro hoste habiturum. \sect Quod si eum interfecerit, multis
sese nobilibus principibusque populi Romani gratum esse facturum (id
se ab ipsis per eorum nuntios compertum habere), quorum omnium gratiam
atque amicitiam eius morte redimere posset. \sect Quod si decessisset
et liberam possessionem Galliae sibi tradidisset, magno se illum
praemio remuneraturum et quaecumque bella geri vellet sine ullo eius
labore et periculo confecturum.

\chap Multa a Caesare in eam sententiam dicta sunt quare negotio desistere
non posset: neque suam neque populi Romani consuetudinem pati ut
optime meritos socios desereret, neque se iudicare Galliam potius
esse Ariovisti quam populi Romani. \sect Bello superatos esse Arvernos
et Rutenos a Q. Fabio Maximo, quibus populus Romanus ignovisset
neque in provinciam redegisset neque stipendium posuisset. \sect Quod
si antiquissimum quodque tempus spectari oporteret, populi Romani
iustissimum esse in Gallia imperium; si iudicium senatus observari
oporteret, liberam debere esse Galliam, quam bello victam suis legibus
uti voluisset.

\chap Dum haec in conloquio geruntur, Caesari nuntiatum est equites
Ariovisti propius tumulum accedere et ad nostros adequitare,
lapides telaque in nostros coicere. \sect Caesar loquendi finem
fecit seque ad suos recepit suisque imperavit ne quod omnino telum
in hostes reicerent. \sect Nam etsi sine ullo periculo legionis
delectae cum equitatu proelium fore videbat, tamen committendum non
putabat ut, pulsis hostibus, dici posset eos ab se per fidem in
conloquio circumventos. \sect Postea quam in vulgus militum elatum
est qua arrogantia in conloquio Ariovistus usus omni Gallia Romanis
interdixisset, impetumque in nostros eius equites fecissent, eaque res
conloquium ut diremisset, multo maior alacritas studiumque pugnandi
maius exercitui iniectum est.

\chap Biduo post Ariovistus ad Caesarem legatos misit: velle se de iis
rebus quae inter eos egi coeptae neque perfectae essent agere cum eo:
uti aut iterum conloquio diem constitueret aut, si id minus vellet, ex
suis legatis aliquem ad se mitteret. \sect Conloquendi Caesari causa
visa non est, et eo magis quod pridie eius diei Germani retineri non
potuerant quin tela in nostros coicerent. \sect Legatum ex suis sese
magno cum periculo ad eum missurum et hominibus feris obiecturum
existimabat. \sect Commodissimum visum est C. Valerium Procillum, C.
Valerii Caburi filium, summa virtute et humanitate adulescentem,
cuius pater a C. Valerio Flacco civitate donatus erat, et propter
fidem et propter linguae Gallicae scientiam, qua multa iam Ariovistus
longinqua consuetudine utebatur, et quod in eo peccandi Germanis causa
non esset, ad eum mittere, et una M. Metium, qui hospitio Ariovisti
utebatur. \sect His mandavit quae diceret Ariovistus cognogcerent et
ad se referrent. Quos cum apud se in castris Ariovistus conspexisset,
exercitu suo praesente conclamavit: quid ad se venirent? an speculandi
causa? Conantes dicere prohibuit et in catenas coniecit.

\chap Eodem die castra promovit et milibus passuum \textsc{vi} a Caesaris
castris sub monte consedit. \sect Postridie eius diei praeter castra
Caesaris suas copias traduxit et milibus passuum duobus ultra eum
castra fecit eo consilio uti frumento commeatuque qui ex Sequanis
et Haeduis supportaretur Caesarem intercluderet. \sect Ex eo die
dies continuos V Caesar pro castris suas copias produxit et aciem
instructam habuit, ut, si vellet Ariovistus proelio contendere, ei
potestas non deesset. \sect Ariovistus his omnibus diebus exercitum
castris continuit, equestri proelio cotidie contendit. Genus hoc
erat pugnae, quo se Germani exercuerant: \sect equitum milia erant
\textsc{vi}, totidem numero pedites velocissimi ac fortissimi, quos ex omni
copia singuli singulos suae salutis causa delegerant: \sect cum his
in proeliis versabantur, ad eos se equites recipiebant; hi, si quid
erat durius, concurrebant, si qui graviore vulnere accepto equo
deciderat, circumsistebant; \sect si quo erat longius prodeundum aut
celerius recipiendum, tanta erat horum exercitatione celeritas ut
iubis sublevati equorum cursum adaequarent.

\chap Ubi eum castris se tenere Caesar intellexit, ne diutius commeatu
prohiberetur, ultra eum locum, quo in loco Germani consederant,
circiter passus \textsc{dc} ab his, castris idoneum locum delegit acieque
triplici instructa ad eum locum venit. \sect Primam et secundam aciem
in armis esse, tertiam castra munire iussit. \sect [Hic locus ab hoste
circiter passus \textsc{dc}, uti dictum est, aberat.] Eo circiter hominum
\textsc{xvi} milia expedita cum omni equitatu Ariovistus misit, quae copiae
nostros terrerent et munitione prohiberent. \sect Nihilo setius Caesar,
ut ante constituerat, duas acies hostem propulsare, tertiam opus
perficere iussit. Munitis castris duas ibi legiones reliquit et partem
auxiliorum, quattuor reliquas legiones in castra maiora reduxit.

\chap Proximo die instituto suo Caesar ex castris utrisque copias
suas eduxit paulumque a maioribus castris progressus aciem instruxit
hostibusque pugnandi potestatem fecit. \sect Ubi ne tum quidem eos
prodire intellexit, circiter meridiem exercitum in castra reduxit.
Tum demum Ariovistus partem suarum copiarum, quae castra minora
oppugnaret, misit. Acriter utrimque usque ad vesperum pugnatum est.
Solis occasu suas copias Ariovistus multis et inlatis et acceptis
vulneribus in castra reduxit. \sect Cum ex captivis quaereret Caesar
quam ob rem Ariovistus proelio non decertaret, hanc reperiebat causam,
quod apud Germanos ea consuetudo esset ut matres familiae eorum
sortibus et vaticinationibus declararent utrum proelium committi ex
usu esset necne; eas ita dicere: \sect non esse fas Germanos superare,
si ante novam lunam proelio contendissent.

\chap Postridie eius diei Caesar praesidio utrisque castris quod
satis esse visum est reliquit, alarios omnes in conspectu hostium
pro castris minoribus constituit, quod minus multitudine militum
legionariorum pro hostium numero valebat, ut ad speciem alariis
uteretur; ipse triplici instructa acie usque ad castra hostium
accessit. \sect Tum demum necessario Germani suas copias castris
eduxerunt generatimque constituerunt paribus intervallis, Harudes,
Marcomanos, Tribocos, Vangiones, Nemetes, Sedusios, Suebos, omnemque
aciem suam raedis et carris circumdederunt, ne qua spes in fuga
relinqueretur. \sect Eo mulieres imposuerunt, quae ad proelium
proficiscentes milites passis manibus flentes implorabant ne se in
servitutem Romanis traderent.

\chap {\DriveOut Caesar singulis legionibus singulos legatos et quaestorem}
praefecit, uti eos testes suae quisque virtutis haberet; \sect
ipse a dextro cornu, quod eam partem minime firmam hostium esse
animadverterat, proelium commisit. \sect Ita nostri acriter in hostes
signo dato impetum fecerunt itaque hostes repente celeriterque
procurrerunt, ut spatium pila in hostes coiciendi non daretur. \sect
Relictis pilis comminus gladiis pugnatum est. At Germani celeriter ex
consuetudine sua phalange facta impetus gladiorum exceperunt. \sect
Reperti sunt complures nostri qui in phalanga insilirent et scuta
manibus revellerent et desuper vulnerarent. \sect Cum hostium acies a
sinistro cornu pulsa atque in fugam coniecta esset, a dextro cornu
vehementer multitudine suorum nostram aciem premebant. \sect Id cum
animadvertisset P. Crassus adulescens, qui equitatui praeerat, quod
expeditior erat quam ii qui inter aciem versabantur, tertiam aciem
laborantibus nostris subsidio misit.

\chap Ita proelium restitutum est, atque omnes hostes terga verterunt
nec prius fugere destiterunt quam ad flumen Rhenum milia passuum
ex eo loco circiter \textsc{l} pervenerunt. \sect Ibi perpauci aut viribus
confisi tranare contenderunt aut lintribus inventis sibi salutem
reppererunt. \sect In his fuit Ariovistus, qui naviculam deligatam ad
ripam nactus ea profugit; reliquos omnes consecuti equites nostri
interfecerunt. \sect Duae fuerunt Ariovisti uxores, una Sueba natione,
quam domo secum eduxerat, altera Norica, regis Voccionis soror, quam
in Gallia duxerat a fratre missam: utraque in ea fuga periit; duae
filiae: harum altera occisa, altera capta est. \sect C. Valerius
Procillus, cum a custodibus in fuga trinis catenis vinctus traheretur,
in ipsum Caesarem hostes equitatu insequentem incidit. \sect Quae
quidem res Caesari non minorem quam ipsa victoria voluptatem attulit,
quod hominem honestissimum provinciae Galliae, suum familiarem et
hospitem, ereptum ex manibus hostium sibi restitutum videbat neque
eius calamitate de tanta voluptate et gratulatione quicquam fortuna
deminuerat. \sect Is se praesente de se ter sortibus consultum dicebat,
utrum igni statim necaretur an in aliud tempus reservaretur: sortium
beneficio se esse incolumem. \sect Item M. Metius repertus et ad eum
reductus est.

\chap Hoc proelio trans Rhenum nuntiato, Suebi, qui ad ripas Rheni
venerant, domum reverti coeperunt; quos ubi qui proximi Rhenum
incolunt perterritos senserunt, insecuti magnum ex iis numerum
occiderunt. \sect Caesar una aestate duobus maximis bellis confectis
maturius paulo quam tempus anni postulabat in hiberna in Sequanos
exercitum deduxit; hibernis Labienum praeposuit; \sect ipse in
citeriorem Galliam ad conventus agendos profectus est.

\Liber

\chap \incipit{Cum} esset Caesar in citeriore Gallia [in hibernis],
ita uti supra demonstravimus, crebri ad eum rumores
adferebantur litterisque item Labieni certior fiebat
omnes Belgas, quam tertiam esse Galliae partem
dixeramus, contra populum Romanum coniurare obsidesque
inter se dare. \sect Coniurandi has esse causas:
primum quod vererentur ne, omni pacata Gallia, ad
eos exercitus noster adduceretur; \sect deinde quod ab non
nullis Gallis sollicitarentur, partim qui, ut Germanos
diutius in Gallia versari noluerant, ita populi Romani
exercitum hiemare atque inveterascere in Gallia moleste
ferebant, partim qui mobilitate et levitate animi
novis imperiis studebant; \sect ab non nullis etiam quod
in Gallia a potentioribus atque iis qui ad conducendos
homines facultates habebant vulgo regna occupabantur;
qui minus facile eam rem imperio nostro
consequi poterant.

\chap His nuntiis litterisque commotus Caesar duas legiones
in citeriore Gallia novas conscripsit et inita
aestate in ulteriorem Galliam qui deduceret Q.~Pedium
legatum misit. \sect Ipse, cum primum pabuli copia esse
inciperet, ad exercitum venit. \sect Dat negotium Senonibus reliquisque Gallis
qui finitimi Belgis erant uti ea
quae apud eos gerantur cognoscant seque de his rebus
certiorem faciant. \sect Hi constanter omnes nuntiaverunt
manus cogi, exercitum in unum locum conduci. Tum
vero dubitandum non existimavit quin ad eos
proficisceretur. Re frumentaria provisa castra movet
diebusque circiter \textsc{xv} ad fines Belgarum pervenit.

\chap Eo cum de improviso celeriusque omnium opinione
venisset, Remi, qui proximi Galliae ex Belgis sunt,
ad eum legatos Iccium et Andebrogium, primos civitatis,
miserunt, \sect qui dicerent se suaque omnia in fidem
atque potestatem populi Romani permittere, neque
se cum reliquis Belgis consensisse neque contra populum
Romanum coniurasse, \sect paratosque esse et obsides
dare et imperata facere et oppidis recipere et frumento
ceterisque rebus iuvare; \sect reliquos omnes Belgas in
armis esse, Germanosque qui cis Rhenum incolant
sese cum his coniunxisse, \sect tantumque esse eorum
omnium furorem ut ne Suessiones quidem, fratres
consanguineosque suos, qui eodem iure et isdem legibus
utantur, unum imperium unumque magistratum cum
ipsis habeant, deterrere potuelint quin cum iis consentirent.

\chap Cum ab iis quaereret quae civitates quantaeque in
armis essent et quid in bello possent, sic reperiebat:
plerosque Belgos esse ortos a Germanis Rhenumque
antiquitus traductos propter loci fertilitatem ibi con\-se\-disse Gallosque
qui ea loca incolerent expulisse,
solosque esse qui, \sect patrum nostrorum memoria omni
Gallia vexata, Teutonos Cimbrosque intra suos fines
ingredi prohibuerint; \sect qua ex re fieri uti earum rerum
memoria magnam sibi auctoritatem Illagnosque spiritus
in re militari sumerent. \sect De numero eorum omnia se
habere explorata Remi dicebant, propterea quod propinquitatibus
adfinitatibus quo coniuncti quantam quisque
multitudinem in communi Belgarum concilio ad
id bellum pollicitus sit cognoverint. \sect Plurimum inter
eos Bellovacos et virtute et auctoritate et hominum
numero valere: hos posse conficere armata milia
centum, pollicitos ex eo numero electa milia \textsc{lx} totiusque
belli imperium sibi postulare. \sect Suessiones
suos esse finitimos; fines latissimos teracissimosque
agros possidere. \sect Apud eos fuisse regem nostra etiam
memoria Diviciacum, totius Galliae potentissimum,
qui cum magnae partis harum regionum, tum etiam
Britanniae imperium obtinuerit; nunc esse regem
Galbam: ad hunc propter iustitiam prudentiamque
summam totius belli omnium voluntate deferri;
oppida habere numero \textsc{xii}, polliceri milia armata L;
totidem Nervios, \sect qui maxime feri inter ipsos habeantur
longissimeque absint; \sect \textsc{xv} milia atrebates, ambianos
x milia, morinos \textsc{xxv} milia, menapios \textsc{vii} milia,
Caletos X milia, Veliocasses et Viromanduos totidem,
atuatucos \textsc{xviiii} milia; \sect condrusos, eburones, caerosos,
Paemanos, qui uno nomine Germani appellantur,
arbitrari ad \textsc{xl} milia.

\chap Caesar Remos cohortatus liberaliterque oratione
prosecutus omnem senatum ad se convenire principumque
liberos obsides ad se adduci iussit. Quae
omnia ab his diligenter ad diem facta sunt. \sect Ipse
Diviciacum Haeduum magnopere cohortatus docet
quanto opere rei publicae comnlunisque salutis intersit
manus hostium distineri, ne cum tanta multitudine
uno tempore confligendum sit. \sect Id fieri posse, si suas
copias Haedui in fines Bellovacorum introduxerint et
eorum agros populari coeperint. \sect His $\langle$datis$\rangle$ mandatis
eum a se dimittit. Postquam omnes Belgarum copias
in unum locum coactas ad se venire vidit neque iam
longe abesse ab iis quos miserat exploratoribus et ab
Remis cognovit, flumen Axonam, quod est in extremis
Remorum finibus, exercitum traducere maturavit atque
ibi castra posuit. \sect Quae res et latus ullum castrorum
ripis fluminis muniebat et post eum quae erant tuta
ab hostibus reddebat et commeatus ab Remis reliquisque
civitatibus ut sine periculo ad eum portari
possent efficiebat. \sect In eo flumine pons erat. Ibi praesidium
ponit et in altera parte fluminis Q.~Titurium
Sabinum legatum cum sex cohortibus relinquit; castra
in altitudinem pedum \textsc{xii} vallo fossaque duodeviginti
pedum muniri iubet.

\chap Ab his castris oppidum Remorum nomine Bibrax
aberat milia passuum \textsc{viii}. Id ex itinere magno
impetu Belgae oppugnare coeperunt. Aegre eo die
sustentatum est. \sect Gallorum eadem atque Belgarum
oppugnatio est haec: ubi circumiecta multitudine
hominum totis moenibus undique in murum lapides
iaci coepti sunt murusque defensoribus nudatus est,
testudine facta portas succedunt murumque subruunt.
Quod tum facile fiebat. \sect Nam cum tanta multitudo
lapides ac tela $\dag$coicerent$\dag$, in muro consistendi
potestas erat nulli. \sect Cum finem oppugnandi nox fecisset,
Iccius Remus, summa nobilitate et gratia inter suos,
qui tum oppido praeerat, unus ex iis qui legati de
pace ad Caesarem venerant, nuntium ad eum mittit,
nisi subsidium sibi submittatur, sese diutius sustinere
non posse.

\chap Eo de media nocte Caesar isdem ducibus usus qui
nuntii ab Iccio venerant, Numidas et Cretas sagittarios
et funditores Baleares subsidio oppidanis mittit;
\sect quorum adventu et Remis cum spe delensionis studium
propugnandi accessit et hostibus eadem de causa spes
potiundi oppidi discessit. \sect Itaque paulisper apud oppidum
morati agrosque Remorum depopulati, omnibus
vicis aedificiisque quo adire potuerant incensis, ad
castra Caesaris omnibus copiis contenderunt et a
milibus passuum minus duobus castra posuerunt;
\sect quae castra, ut fumo atque ignibus significabatur,
amplius milibus passuum \textsc{viii} latitudinem patebant.

\chap Caesar primo et propter multitudinem hostium et
propter eximiam opinionem virtutis proelio supersedere
statuit; \sect cotidie tamen equestribus proeliis quid
hostis virtute posset et quid nostri auderent periclitabatur.
\sect Ubi nostros non esse inferiores intellexit,
loco pro castris ad aciem instruendam natura oportuno
atque idoneo, quod is collis ubi castra posita erant
paululum ex planitie editus tantum adversus in latitudinem
patebat quantum loci acies instructa occupare
poterat, atque ex utraque parte lateris deiectus habebat
et in fronte leniter fastigatus paulatim ad planitiem
redibat, ab utroque latere eius collis transversam
fossam obduxit circiter passuum \textsc{cccc} \sect et ad extremas
fossas castella constituit ibique tormenta conlocavit,
ne, cum aciem instruxisset, hostes, quod tantum
multitudine poterant, ab lateribus pugnantes suos
circumvenire possent. \sect Hoc facto, duabus legionibus
quas proxime conscripserat in castris relictis ut, si
quo opus esset, subsidio duci possent, reliquas \textsc{vi}
legiones pro castris in acie constituit. Hostes item
suas copias ex castris eductas instruxerunt.

\chap Palus erat non magna inter nostrum atque hostium
exercitum. Hanc si nostri transirent hostes expectabant;
nostri autem, si ab illis initium transeundi
fieret, ut impeditos adgrederentur parati in armis
erant. \sect Interim proelio equestri inter duas acies
contendebatur. Ubi neutri transeundi initium faciunt,
secundiore equitum proelio nostris Caesar suos in
castra reduxit. \sect Hostes protinus ex eo loco ad flumen
Axonam contenderunt, quod esse post nostra castra
demonstratum est. \sect Ibi vadis repertis partem suarum
copiarum traducere conati sunt eo consilio ut, si
possent, castellum, cui praeerat Q.~Titurius legatus,
expugnarent pontemque interscinderent, \sect si minus
potuissent, agros Remorum popularentur, qui magno
nobis usui ad bellum gerendum erant, commeatuque
nostros prohiberent.

\chap $\langle$Caesar$\rangle$ certior factus ab Titurio onlnem
equitatum et levis armaturae Numidas, funditores
sagittariosque pontem traducit atque ad eos contendit.
Acriter in eo loco pugnatum est. \sect Hostes impeditos
nostri in flumine adgressi magnum eorum numerum
occiderunt; \sect per eorum corpora reliquos audacissime
transire conantes multitudine telorum reppulerunt
primosque, qui transierant, equitatu circumventos
interfecerunt. \sect Hostes, ubi et de expugnando oppido
et de flumine transeundo spem se fefellisse intellexerunt
neque nostros in locum iniquiorum progredi
pugnandi causa viderunt atque ipsos res frumentaria
deficere coepit, concilio convocato constituerunt
optimum esse domum suam quemque reverti, et quorum
in fines primum Romani exercitum introduxissent,
ad eos defendendos undique convenirent, ut potius
in suis quam in alienis finibus decertarent et domesticis
copiis rei frumentariae uterentur. \sect Ad eam sententiam
cum reliquis causis haec quoque ratio eos
deduxit, quod Diviciacum atque Haeduos finibus Bellovacorum
adpropinquare cognoverant. His persuaderi
ut diutius morarentur neque suis auxilium terrent
non poterat.

\chap Ea re constituta, secunda vigilia magno cum,
strepitu ac tumultu castris egressi nullo certo ordine
neque imperio, cum sibi quisque primum itineris
locum peteret et domum pervenire properaret, fecerunt
ut consimilis fugae profectio videretur. \sect Hac re statim
Caesar per speculatores cognita insidias veritus, quod
qua de causa discederent nondum perspexerat, exercitum
equitatumque castris continuit. \sect Prima luce,
confirmata re ab exploratoribus, omnem equitatum,
qui novissimum agmen moraretur, praemisit. His
Q.~Pedium et L.~Aurunculeium Cottam legatos praefecit;
T.~Labienum legatum cum legionibus tribus
subsequi iussit. \sect Hi novissimos adorti et multa milia
passuum prosecuti magnam multitudinem eorum
fugientium conciderunt, cum ab extremo agmine, ad
quos ventum erat, consisterent fortiterque impetum
nostrorum militum sustinerent, \sect priores, quod abesse
a periculo viderentur neque ulla necessitate neque
imperio continerentur, exaudito clamore perturbatis
ordinibus omnes in fuga sibi praesidium ponerent.
\sect Ita sine ullo periculo tantam eorum multitudinem
nostri interfecerunt quantum fuit diei spatium; sub
occasum solis sequi destiterunt seque in castra, ut erat
imperatum, receperunt.

\chap Postridie eius diei Caesar, prius quam se hostes ex
terrore ac fuga reciperent, in fines Suessionum, qui
proximi Remis erant, exercitum duxit et magno itinere
[confecto] ad oppidum Noviodunum contendit. \sect Id ex
itinere oppugnare conatus, quod vacuum ab
defensoribus esse audiebat, propter latitudinem fossae
murique altitudinem paucis defendentihus expugnare non
potuit. \sect Castris munitis vineas agere quaeque ad
oppugnandum usui erant comparare coepit. \sect Interim
omnis ex fuga Suessionum multitudo in oppidum
proxima nocte convenit. \sect Celeriter vineis ad oppidum
actis, aggere iacto turribusque constitutis, magnitudine
operum, quae neque viderant ante Galli neque
audierant, et celeritate Romanorum permoti legatos ad
Caesarem de deditione mittunt et petentibus Remis ut
conservarentur impetrant.
\chap Caesar, obsidibus acceptis primis civitatis
atque ipsius Galbae regis duobus filiis armisque omnibus ex
oppido traditis, in deditionem Suessiones accipit
exercitumque in Bellovacos ducit. \sect Qui cum se suaque
omnia in oppidum Bratuspantium contulissent atque
ab eo oppido Caesar cum exercitu circiter milia
passuum V abesset, omnes maiores natu ex oppido
egressi manus ad Caesarem tendere et voce significare
coeperunt sese in eius fidem ac potestatem venire
neque contra populum Romanum armis contendere.
\sect Item, cum ad oppidum accessisset castraque ibi poneret,
pueri mulieresque ex muro passis mallibus suo more
pacem ab Romanis petierunt.

\chap Pro his Diviciacus (nam post discessum Belgarum
dimissis Haeduorum copiis ad Cum reverterat) facit
verba: \sect Bellovacos omni tempore in fide atque amicitia
civitatis Haeduae fuisse; \sect impulsos ab suis principibus,
qui dicerent Haeduos a Caesare in servitutem redacto.
omnes indignitates contumeliasque perferre, et ab
Haeduis defecisse et populo Romano bellum intulisse.
\sect Qui eius consilii principes fuissent, quod intellegerent
quantam calamitatem civitati intulissent, in Britanniam
profugisse. \sect Petere non solum Bellovacos, sed
etiam pro his Haeduos, ut sua clementia ac
mansuetudine in eos utatur. \sect Quod si fecerit, Haeduorum
auctoritatem apud omnes Belgas amplificaturum,
quorum auxiliis atque opibus, si qua bella inciderint,
sustentare consuerint.

\chap Caesar honoris Diviciaci atque Haeduorum causa
sese eos in fidem recepturum et conservaturum dixit,
et quod erat civitas magna inter Belgas auctoritate
atque hominum multitudine praestabat, \textsc{dc} obsides
poposcit. \sect His traditis omnibusque armis ex oppido
conlatis, ab eo loco in fines Ambianorum pervenit;
qui se suaque omnia sine mora dediderunt. \sect Eorum
fines Nervii attingebant. Quorum de natura
moribusque Caesar cum quaereret, sic reperiebat: \sect nullum
esse aditum ad eos mercatoribus; nihil pati vini
reliquarumque rerum ad luxuriam pertinentium
inferri, quod his rebus relanguescere animos eorum
et remitti virtutem existimarent; \sect esse homines feros
magnaeque virtutis; increpitare atque incusare reliquos
Belgas, qui se populo Romano dedidissent
patriamque virtutem proiecissent; \sect confirmare sese
neque legatos missuros neque ullam condicionem pacis
accepturos.

\chap Cum per eorum fines triduum iter fecisset, inveniebat
ex captivis Sabim flumen a castris suis non
amplius milibus passuum X abesse; \sect trans id flumen
omnes Nervios consedisse adventumque ibi Romanorum
expectare una cum Atrebatibus et Viromanduis,
finitimis suis \sect (nam his utrisque persuaserant
uti eandem belli fortunam experirentur); \sect expectari
etiam ab iis Atuatucorum copias atque esse in itinere;
\sect mulieres quique per aetatem ad pugnam inutiles
viderentur in eum locum coniecisse quo propter
paludes exercitui aditus non esset.

\chap His rebus cognitis, exploratores centurionesque
praemittit qui locum castris idoneum deligant. \sect Cum ex
dediticiis Belgis reliquisque Gallis complures Caesarem
secuti una iter facerent, quidam ex his, ut postea ex
captivis cognitum est, eorum dierum consuetudine
itineris nostri exercitus perspecta, nocte ad Nervios
pervenerunt atque his demonstrarunt inter singulas
legiones impedimentorum magnum numerum intercedere,
neque esse quicquam negotii, cum prima legio
in castra venisset reliquaeque legiones magnum spatium
abessent, hanc sub sarcinis adoriri; \sect qua pulsa
impedimentisque direptis, futurum ut reliquae contra
consistere non auderent. \sect Adiuvabat etiam eorum
consilium qui rem deferebant quod Nervii antiquitus,
cum equitatu nihil possent (neque enim ad hoc
tempus ei rei student, sed quicquid possunt,
pedestribus valent copiis), quo facilius finitimorum
equitatum, si praedandi causa ad eos venissent, impedirent,
teneris arboribus incisis atque inflexis crebrisque in
latitudinem ramis enatis [et] rubis sentibusque
interiectis effecerant ut instar muri hae saepes
munimentum praeberent, quo non modo non intrari sed ne
perspici quidem posset. \sect His rebus cum iter agminis
nostri impediretur, non omittendum sibi consilium
Nervii existimaverunt.

\chap Loci natura erat haec, quem locum nostri castris
delegerant. Collis ab summo aequaliter declivis ad
flumen Sabim, quod supra nominavimus, vergebat.
\sect Ab eo flumine pari acclivitate collis nascebatur
adversus huic et contrarius, passus circiter \textsc{cc} infimus
apertus, ab superiore parte silvestris, ut non facile
introrsus perspici posset. \sect Intra eas silvas hostes in
occulto sese continebant; in aperto loco secundum
flumen paucae stationes equitum videbantur. Fluminis
erat altitudo pedum circiter trium.

\chap Caesar equitatu praemisso subsequebatur omnibus
copiis; sed ratio ordoque agminis aliter se habebat
ac Belgae ad Nervios detulerant. \sect Nam quod hostibus
adpropinquabat, consuetudine sua caesar \textsc{vi} legiones
expeditas ducebat; \sect {\DriveOut post eas totius exercitus
impedimenta conlocarat; inde duae legiones quae proxime
conscriptae erant totum agmen claudebant praesidioque
impedimentis erant}. \sect Equites nostri cum funditoribus
sagittariisque flumen transgressi cum hostium
equitatu proelium commiserunt. \sect Cum se illi identidem
in silvis ad suos reciperent ac rursus ex silva in
nostros impetum facerent, neque nostri longius quam
quem ad finem porrecta [ac] loca aperta pertinebant
cedentes insequi auderent, interim legiones \textsc{vi} quae
primae venerant, opere dimenso, castra munire
coeperunt. \sect Ubi prima impedimenta nostri exercitus ab
iis qui in silvis abditi latebant visa sunt, quod tempus
inter eos committendi proelii convenerat, ut intra
silvas aciem ordinesque constituerant atque ipsi sese
confirmaverant, subito omnibus copiis provolaverunt
impetumque in nostros equites fecerunt. \sect His facile
pulsis ac proturbatis, incredibili celeritate ad flumen
decucurrerunt, ut paene uno tempore et ad silvas et
in flumine [et iam in manibus nostris] hostes
viderentur. \sect Eadem autem celeritate adverso colle ad
nostra castra atque eos qui in opere occupati erant
contenderunt.

\chap Caesari omnia uno tempore erant agenda: vexillum
proponendum, quod erat insigne, cum ad arma concurri
oporteret; signum tuba dandum; ab opere
revocandi milites; qui paulo longius aggeris petendi
causa processerant arcessendi; acies instruenda;
milites cohortandi; signum dandum. Quarum rerum
magnam partem temporis brevitas et incursus hostium
impediebat. \sect His difficultatibus duae res erant
subsidio, scientia atque usus militum, quod superioribus
proeliis exercitati quid fieri oporteret non minus
commode ipsi sibi praescribere quam ab aliis doceri
poterant, et quod ab opere singulisque legionibus
singulos legatos Caesar discedere nisi munitis castris
vetuerat. \sect Hi propter propinquitatem et celeritatem
hostium nihil iam Caesaris imperium expectabant, sed
per se quae videbantur administrabant.

\chap Caesar, necessariis rebus imperatis, ad cohortandos
milites, quam [in] partem fors obtulit, decucurrit et
ad legionem decimam devenit. \sect Milites non longiore
oratione cohortatus quam uti suae pristinae virtutis
memoriam retinerent neu perturbarentur animo
hostiumque impetum fortiter sustinerent, \sect quod non
longius hostes aberant quam quo telum adigi posset,
proelii committendi signum dedit. \sect Atque in alteram
item cohortandi causa profectus pugnantibus
occurrit. \sect Temporis tanta fuit exiguitas hostiumque
tam paratus ad dimicandum animus ut non modo ad
insignia accommodanda sed etiam ad galeas induendas
scutisque tegimenta detrahenda tempus defuerit.
\sect Quam quisque ab opere in partem casu devenit
quaeque prima signa conspexit, ad haec constitit, ne
in quaerendis suis pugnandi tempus dimitteret.

\chap Instructo exercitu magis ut loci natura [delectusque
collis] et necessitas temporis quam ut rei militaris
ratio atque ordo postulabat, cum diversae legiones
aliae alia in parte hostibus resisterent saepibusque
densissimis, ut ante demonstravimus, interiectis
prospectus impediretur, neque certa subsidia conlocari
neque quid in quaque parte opus esset provideri
neque ab uno omnia imperia administrari poterant.
\sect Itaque in tanta rerum iniquitate fortunae quoque
eventus varii sequebantur.

\chap Legionis \textsc{viiii}. et \textsc{x}. milites, ut in sinistra parte
aciei constiterant, pilis emissis cursu ac lassitudine
exanimatos vulneribusque confectos Atrebates (nam
his ea pars obvenerat) celeriter ex loco superiore in
flumen compulerunt et transire conantes insecuti
gladiis magnam partem eorum impeditam interfecerunt.
\sect Ipsi transire flumen non dubitaverunt et
in locum iniquum progressi rursus resistentes hostes
redintegrato proelio in fugam coniecerunt. \sect Item alia
in parte diversae duae legiones, \textsc{xi}. et \textsc{viii}., profligatis
Viromanduis, quibuscum erant congressae, ex
loco superiore in ipsis fluminis ripis proeliabantur.
\sect At totis fere castris a fronte et a sinistra parte nudatis,
cum in dextro cornu legio \textsc{xii}. et non magno
ab ea intervallo \textsc{vii}. constitisset, omnes Nervii
confertissimo agmine duce Boduognato, qui summam
imperii tenebat, ad eum locum contenderunt; quorum
pars $\langle$ab$\rangle$ aperto latere legiones circumvenire, pars
summum castrorum locum petere coepit.

\chap Eodem tempore equites nostri levisque armaturae
pedites, qui cum iis una fuerant, quos primo hostium
impetu pulsos dixeram, cum se in castra reciperent,
adversis hostibus occurrebant ac rursus aliam in
partem fugam petebant; \sect et calones, qui ab decumana
porta ac summo iugo collis nostros victores flumen
transire conspexerant, praedandi causa egressi, cum
respexissent et hostes in nostris castris versari
vidissent, praecipites fugae sese mandabant. \sect Simul eorum
qui cum impedimentis veniebant clamor fremitusque
oriebatur, aliique aliam in partem perterriti ferebantur.
\sect Quibus omnibus rebus permoti equites Treveri, quorum
inter Gallos virtutis opinio est singularis, qui auxilii
causa a civitate missi ad Caesarem venerant, cum
multitudine hostium castra [nostra] compleri, legiones
premi et paene circumventas teneri, calones, equites,
tunditores, Numidas diversos dissipatosque in omnes
partes fugere vidissent, desperatis nostris rebus
domum contenderunt: \sect Romanos pulsos superatosque,
castris impedimentisque eorum hostes potitos civitati
renuntiaverunt.

\chap Caesar ab X. legionis cohortatione ad dextrum
cornu prolectus, ubi suos urgeri signisque in unum
locum conlatis \textsc{xii}. legionis confertos milites sibi
ipsos ad pugnam esse impedimento vidit, quartae
cohortis omnibus centurionibus occisis signiferoque
interfecto, signo amisso, reliquarum cohortium omnibus
fere centurionibus aut vulneratis aut occisis, in
his primipilo P.~Sextio Baculo, fortissimo viro, multis
gravibusque vulneribus confecto, ut iam se sustinere
non posset, reliquos esse tardiores et non nullos ab
novissimis deserto $\langle$loco$\rangle$ proelio excedere ac tela
vitare, hostes neque a fronte ex inferiore loco
subeuntes intermittere et ab utroque latere instare et
rem esse in angusto vidit, neque ullum esse subsidium
quod submitti posset, \sect scuto ab novissimis [uni] militi
detracto, quod ipse eo sine scuto venerat, in primam
aciem processit centurionibusque nominatim appellatis
reliquos cohortatus milites signa inferre et manipulos
laxare iussit, quo facilius gladiis uti possent. \sect Cuius
adventu spe inlata militibus ac redintegrato animo,
cum pro se quisque in conspectu imperatoris etiam
in extremis suis rebus operam navare cuperet, paulum
hostium impetus tardatus est.

\chap Caesar, cum \textsc{vii}. legionem, quae iuxta constiterat,
item urgeri ab hoste vidisset, tribunos militum monuit
ut paulatim sese legiones coniungerent et conversa
signa in hostes inferrent. \sect Quo facto cum aliis alii
subsidium ferrent neque timerent ne aversi ab hoste
circumvenirentur, audacius resistere ac fortius
pugnare coeperunt. \sect Interim milites legionum duarum
quae in novissimo agmine praesidio impedimentis
fuerant, proelio nuntiato, cursu incitato in summo
colle ab hostibus conspiciebantur, \sect et T.~Labienus
castris hostium potitus et ex loco superiore quae res
in nostris castris gererentur conspicatus X.~legionem
subsidio nostris misit. \sect Qui cum ex equitum et
calonum fuga quo in loco res esset quantoque in
periculo et castra et legiones et imperator
versaretur cognovissent, nihil ad celeritatem sibi reliqui
fecerunt.

\chap Horum adventu tanta rerum commutatio est facta
ut nostri, etiam qui vulneribus confecti procubuissent,
scutis innixi proelium redintegrarent, calones perterritos
hostes conspicati etiam inermes armatis occurrerent,
\sect equites vero, ut turpitudinem fugae virtute
delerent, omnibus in locis pugnae se legionariis
militibus praeferrent. \sect At hostes, etiam in extrema
spe salutis, tantam virtutem praestiterunt ut, cum
primi eorum cecidissent, proximi iacentibus insisterent
atque ex eorum corporibus pugnarent, \sect his deiectis
et coacervatis cadaveribus qui superessent ut ex
tumulo tela in nostros coicerent et pila intercepta
remitterent: \sect ut non nequiquam tantae virtutis
homines iudicari deberet ausos esse transire latissimum
flumen, ascendere altissimas ripas, subire iniquissimum
locum; quae facilia ex difficillimis animi magnitudo
redegerat.

\chap Hoc proelio facto et prope ad internecionem gente
ac nomine Nerviorum redacto, maiores natu, quos
una cum pueris mulieribusque in aestuaria ac paludes
coniectos dixeramus, hac pugna nuntiata, cum
victoribus nihil impeditum, victis nihil tutum
arbitrarentur, \sect omnium qui supererant consensu legatos ad
Caesarem miserunt seque ei dediderunt; et in
commemoranda civitatis calamitate ex \textsc{dc} ad tres
senatores, ex hominum milibus \textsc{lx} vix ad d, qui arma
ferre possent, sese redactos esse dixerunt. \sect Quos Caesar,
ut in miseros ac supplices usus misericordia videretur,
diligentissime conservavit suisque finibus atque oppidis
uti iussit et finitimis imperavit ut ab iniuria et
maleficio se suosque prohiberent.

\chap Atuatuci, de quibus supra diximus, cum omnibus
copiis auxilio Nerviis venirent, hac pugna nuntiata
ex itinere domum reverterunt; \sect cunctis oppidis castellisque
desertis sua omnia in unum oppidum egregie
natura munitum contulerunt. \sect Quod cum ex omnibus
in circuitu partibus altissimas rupes deiectusque
haberet, una ex parte leniter acclivis aditus in
latitudinem non amplius pedum \textsc{cc} relinquebatur; quem
locum duplici altissimo muro munierant; tum magni
ponderis saxa et praeacutas trabes in muro
conlocabant. \sect Ipsi erant ex Cimbris Teutonisque prognati,
qui, cum iter in provinciam nostram atque Italiam;
facerent, iis impedimentis quae secum agere ac portare
non poterant citra flumen Rhenum depositis custodiae
[ex suis] ac praesidio \textsc{vi} milia hominum una reliquerant.
\sect Hi post eorum obitum multos annos a finitimis
exagitati, cum alias bellum inferrent, alias
inlatum defenderent, consensu eorum omnium pace
facta hunc sibi domicilio locum delegerant.

\chap Ac primo adventu exercitus nostri crebras ex oppido
excursiones faciebant parvulisque proeliis cum
nostris contendebant; \sect postea vallo pedum \textsc{xii} in
circuitu $\dag$\textsc{xv}$\dag$ milium crebrisque castellis
circummuniti oppido sese continebant. \sect Ubi vineis actis aggere
extructo turrim procul constitui viderunt, primum
inridere ex muro atque increpitare vocibus, quod tanta
machinatio a tanto spatio institueretur: \sect quibusnam
mallibus aut quibus viribus praesertim homines tantulae
staturae (nam plerumque omnibus Gallis prae
magnitudine corporum quorum brevitas nostra
contemptui est) tanti oneris turrim in muro sese $\langle$posse$\rangle$
conlocare confiderent?

\chap Ubi vero moveri et adpropinquare muris viderunt,
nova atque inusitata specie commoti legatos ad
Caesarem de pace miserunt, qui ad hunc modum locuti,
\sect non se existimare Romanos sine ope divina bellum
gerere, qui tantae altitudinis machinationes tanta
celeritate promovere possent, \sect se suaque omnia eorum
potestati permittere dixerunt. \sect Unum petere ac deprecari:
si forte pro sua clementia ac mansuetudine,
quam ipsi ab aliis audirent, statuisset Atuatucos esse
conservandos, ne se armis despoliaret. \sect Sibi omnes
fere finitimos esse inimicos ac suae virtuti invidere;
a quibus se defelldere traditis armis non possent.
\sect Sibi praestare, si in eum casum deducerentur, quamvis
fortunam a populo Romano pati quam ab his per
cruciatum interfici inter quos dominari consuessent.

\chap Ad haec Caesar respondit: se magis consuetudine
sua quam merito eorum civitatem conservaturum, si
prius quam murum aries attigisset se dedidissent;
\sect sed deditionis nullam esse condicionem nisi armis
traditis. Se id quod in Nerviis fecisset facturum
finitimisque imperaturum ne quam dediticiis populi
Romani iniuriam inferrent. \sect Re renuntiata ad suos
illi se quae imperarentur facere dixerunt. \sect Armorum
magna multitudine de muro in fossam, quae erat
ante oppidum, iacta, sic ut prope summam muri
aggerisque altitudinem acervi armorum adaequarent,
et tamen circiter parte tertia, ut postea perspectum
est, celata atque in oppido retenta, portis patefactis eo
die pace sunt usi.

\chap \looseness=1 Sub vesperum Caesar portas claudi militesque ex
oppido exire iussit, ne quam noctu oppidani a
militibus iniuriam acciperent. \sect Illi ante inito, ut
intellectum est, consilio, quod deditione facta nostros
praesidia deducturos aut denique indiligentius
servaturos crediderant, partim cum iis quae retinuerant
et celaverant armis, partim scutis ex cortice factis
aut viminibus intextis, quae subito, ut temporis
exiguitas postulabat, pellibus induxerant, tertia
vigilia, qua minime arduus ad nostras munitiones accensus
videbatur, omnibus copiis repente ex oppido eruptionem
fecerunt. \sect Celeriter, ut ante Caesar imperaverat,
ignibus significatione facta, ex proximis castellis
eo concursum est, \sect pugnatumque ab hostibus ita acriter
est ut a viris fortibus in extrema spe salutis iniquo
loco contra eos qui ex vallo turribusque tela iacerent
pugnari debuit, cum in una virtute omnis spes consisteret.
\sect occisis ad hominum milibus \textsc{iiii} reliqui
in oppidum reiecti sunt. \sect Postridie eius diei refractis
portis, cum iam defenderet nemo, atque intromissis
militibus nostris, sectionem eius oppidi universa
Caesar vendidit. \sect Ab iis qui emerant capitum numerus
ad eum relatus est milium \textsc{liii}.

\chap Eodem tempore a P.~Crasso, quem cum legione una
miserat ad Venetos, Venellos, Osismos, Coriosolitas,
Esuvios, Aulercos, Redones, quae sunt maritimae
civitates Oceanumque attingunt, certior factus est omnes
eas civitates in dicionem potestatemque populi Romani
esse redactas.

\chap His rebus gestis omni Gallia pacata, tanta huius
belli ad barbaros opinio perlata est uti ab iis
nationibus quae trans Rhenum incolerent legationes ad
Caesarem mitterentur, quae se obsides daturas,
imperata facturas pollicerentur. \sect Quas legationes Caesar,
quod in Italiam Illyricumque properabat, inita proxima
aestate ad se reverti iussit. \sect Ipse in Carnutes, Andes,
Turonos quaeque civitates propinquae iis locis erant
ubi bellum gesserat, legionibus in hiberna deductis,
in Italiam profectus est. \sect Ob easque res ex litteris
caesaris dierum \textsc{xv} supplicatio decreta est, quod ante
id tempus accidit nulli.

\Liber

\chap \incipit{Cum} in Italiam proficisceretur Caesar, Ser.~Galbam
cum legione \textsc{xii}. et parte equitatus in Nantuates,
Veragros Sedunosque misit, qui a finibus Allobrogum
et lacu Lemanno et flumine Rhodano ad summas Alpes
pertinent. \sect Causa mittendi fuit quod iter per Alpes,
quo magno cum periculo magnisque cum portoriis
mercatores ire consuerant, patefieri volebat. \sect Huic
permisit, si opus esse arbitraretur, uti in his locis
legionem hiemandi causa conlocaret. \sect Galba secundis
aliquot proeliis factis castellisque compluribus eorum
expugnatis, missis ad eum undique legatis obsidibusque
datis et pace facta, constituit cohortes duas in
Nantuatibus conlocare et ipse cum reliquis eius legionis
cohortibus in vico Veragrorum, qui appellatur
Octodurus hiemare; \sect qui vicus positus in valle non
magna adiecta planitie altissimis montibus undique
continetur. \sect Cum hic in duas partes flumine divideretur,
alteram partem eius vici Gallis [ad hiemandum]
concessit, alteram vacuam ab his relictam cohortibus
attribuit. Eum locum vallo fossaque munivit.

\chap Cum dies hibernorum complures transissent
frumentumque eo comportari iussisset, subito per
exploratores certior factus est ex ea parte vici, quam Gallis
concesserat, omnes noctu discessisse montesque qui
impenderent a maxima multitudine Sedunorum et
Veragrorum teneri. \sect Id aliquot de causis acciderat,
ut subito Galli belli renovandi legionisque
opprimendae consilium caperent: \sect primum, quod legionem
neque eam plenissimam detractis cohortibus duabus
et compluribus singillatim, qui commeatus petendi
causa missi erant, absentibus propter paucitatem
despiciebant; \sect tum etiam, quod propter iniquitatem loci,
cum ipsi ex montibus in vallem decurrerent et tela
coicerent, ne primum quidem impetum suum posse
sustineri existimabant. \sect Accedebat quod suos ab se
liberos abstraetos obsidum nomine dolebant, et
Romanos non solum itinerum causa sed etiam perpetuae
possessionis culmina Alpium oceupare conari et ea
loca finitimae provinciae adiungere sibi persuasum
habebant.

\chap His nuntiis acceptis Galba, cum neque opus hibernorum
munitionesque plene essent perfectae neque
de frumento reliquoque commeatu satis esset provisum
quod deditione facta obsidibusque acceptis nihil de
bello timendum existimaverat, consilio celeriter
convocato sententias exquirere coepit. \sect Quo in consilio,
cum tantum repentini periculi praeter opinionem
accidisset ac iam omnia fere superiora loca multitudine
armatorum completa conspicerentur neque subsidio
veniri neque commeatus supportari interclusis itineribus
possent, \sect prope iam desperata salute non nullae
eius modi sententiae dicebantur, ut impedimentis
relictis eruptione facta isdem itineribus quibus eo
pervenissent ad salutem contenderent. \sect Maiori tamen
parti placuit, hoc reservato ad extremum $\langle$casum$\rangle$
consilio interim rei eventum experiri et castra defendere.

\chap Brevi spatio interiecto, vix ut iis rebus quas
constituissent conlocandis atque administrandis tempus
daretur, hostes ex omnibus partibus signo dato
decurrere, lapides gaesaque in vallum coicere. \sect Nostri
primo integris viribus fortiter propugnare neque
ullum flustra telum ex loco superiore mittere, et
quaecumque pars castrorum nudata defensoribus
premi videbatur, eo occurrere et auxilium ferre, \sect sed
hoc superari quod diuturnitate pugnae hostes defessi
proelio excedebant, alii integris viribus succedebant;
\sect quarum rerum a nostris propter paucitatem fieri nihil
poterat, ac non modo defesso ex pugna excedendi, sed
ne saucio quidem eius loci ubi constiterat relinquendi
ac sui recipiendi facultas dabatur.

\chap Cum iam amplius horis sex continenter pugnaretur,
ac non solum vires sed etiam tela nostros deficerent,
atque hostes acrius instarent languidioribusque nostris
vallum scindere et fossas complere coepissent, resque
esset iam ad extremum perducta casum, \sect P.~Sextius
Baculus, primi pili centurio, quem Nervico proelio
compluribus confectum vulneribus diximus, et item
C.~Volusenus, tribunus militum, vir et consilii magni
et virtutis, ad Galbam accurrunt atque unam esse
spem salutis docent, si eruptione facta extremum
auxilium experirentur. \sect Itaque convocatis centurionibus
celeriter milites certiores facit, paulisper
intermitterent proelium ac tantum modo tela missa
exciperent seque ex labore reficerent, post dato signo
ex castris erumperent, atque omnem spem salutis in
virtute ponerent.

\chap Quod iussi sunt faciunt, ac subito omnibus portis
eruptione facta neque cognoscendi quid fieret neque
sui colligendi hostibus facultatem relinquunt. \sect Ita
commutata fortuna eos qui in spem potiundorum
castrorum venerant undique circumventos intercipiunt,
et ex hominum milibus amplius \textsc{xxx}, quem numerum
barbarorum ad castra venisse constabat, plus tertia
parte interfecta reliquos perterritos in fugam coiciunt
ac ne in locis quidem superioribus consistere patiuntur.
\sect Sic omnibus hostium copiis fusis armisque exutis se
intra munitiones suas recipiunt. \sect Quo proelio facto,
quod saepius fortunam temptare Galba nolebat atque
alio se in hiberna consilio venisse meminerat, aliis
occurrisse rebus videbat, maxime frumenti [commeatusque]
inopia permotus postero die omnibus eius vici
aedificiis incensis in provinciam reverti contendit, \sect ac
nullo hoste prohibente aut iter demorante incolumem
legionem in Nantuates, inde in Allobroges perduxit
ibique hiemavit.

\chap His rebus gestis cum omnibus de causis Caesar
pacatam Galliam existimaret, [\,superatis Belgis,
expulsis Germanis, victis in Alpibus Sedunis,\,] atque ita
inita hieme in Illyricum profectus esset, quod eas
quoque nationes adire et regiones cognoseere volebat,
subitum bellum in Gallia eoortum est. \sect Eius belli
haec fuit causa. P.~Crassus adulescens eum legione
\textsc{vii}. proximus mare Oeeanum in Andibus hiemabat.
\sect Is, quod in his loeis inopia frumenti erat, praefectos
tribunosque militum eomplures in finitimas civitates
frumenti causa dimisit; \sect quo in numero est T.~Terrasidius
missus in Esuvios, M.~Trebius Gallus in Coriosolites,
Q.~Velanius eum T.~Silio in Venetos.

\chap Huius est civitatis longe amplissima auctoritas
omnis orae maritimae regionum earum, quod et naves
habent Veneti plurimas, quibus in Britanniam
navigare consuerunt, et scientia atque usu rerum
nauticarum ceteros antecedunt et in magno impetu maris
atque aperto paucis portibus interiectis, quos tenent
ipsi, omnes fere qui eo mari uti consuerunt habent
vectigales. \sect Ab his fit initium retinendi Silii atque
Velanii, quod per eos suos se obsides, quos Crasso
dedissent, recuperaturos existimabant.  \sect Horum
auctoritate finitimi adducti, ut sunt Gallorum subita et
repentina consilia, eadem de causa Trebium
Terrasidiumque retinent et celeriter missis legatis per suos
principes inter se coniurant nihil nisi communi
consilio acturos eundemque omnes fortunae exitum esse
laturos, \sect reliquasque civitates sollicitant, ut in ea
libertate quam a maioribus acceperint permanere
quam Romanorum servitutem perferre malint. \sect Omni
ora maritima celeriter ad suam sententiam perducta
communem legationem ad P.~Crassum mittunt, si
velit suos recuperare, obsides sibi remittat.

\chap Quibus de rebus Caesar a Crasso certior factus, 
quod ipse aberat longius, naves interim longas
aedificari in flumine Ligeri, quod influit in Oceanum,
remiges ex provincia institui, nautas gubernatoresque
comparari iubet. \sect His rebus celeriter administratis
ipse, cum primum per anni tempus potuit, ad
exercitum contendit. \sect Veneti reliquaeque item civitates
cognito Caesaris adventu [certiores facti], simul quod
quantum in se facinus admisissent intellegebant,
[legatos, quod nomen ad omnes nationes sanctum
inviolatumque semper fuisset, retentos ab se et in
vincula coniectos,] pro magnitudine periculi bellum
parare et maxime ea quae ad usum navium pertinent
providere instituunt, boc maiore spe quod multum
natura loci confidebant.  \sect Pedestria esse itinera
concisa aestuariis, navigationem impeditam propter
inscientiam locorum paucitatemque portuum sciebant,
\sect neque nostros exercitus propter inopiam frumenti
diutius apud se morari posse confidebant; \sect ac iam
ut omnia contra opinionem acciderent, tamen sc
plurimum navibus posse, [quam] Romanos neque
ullam facultatem habere navium, neque eorum
locorum ubi bellum gesturi essent vada, portus, insulas
novisse; \sect ac longe aliam esse navigationem in concluso
mari atque in vastissimo atque apertissimo Oceano
perspiciebant. \sect His initis consiliis oppida muniunt,
\sect frumenta ex agris in oppida comportant, naves in
Venetiam, ubi Caesarem primum bellum gesturum
\sect constabat, quam plurimas possunt cogunt.  Socios
sibi ad id bellum Osismos, Lexovios, Namnetes,
Ambiliatos, Morinos, Diablintes, Menapios adsciscunt;
auxilia ex Britannia, quae contra eas regiones posita
est, arcessunt. 

\chap Erant hae difficultates belli gerendi quas supra
ostendiIrIus, sed tamen multa Caesarem ad id bellum
incitabant: \sect iniuria retentorum equitum Romanorum,
rebellio facta post deditionem, defectio datis obsidibus,
tot civitatum coniuratio, in primis ne hac parte
neglecta reliquae nationes sibi idem licere arbitrarentur.
\sect Itaque cum intellegeret omnes fere Gallos
novis rebus studere et ad bellum mobiliter celeriterque
excitari, omnes autem homines natura libertati studere
et condicionem servitutis odisse, prius quam  plures
civitates conspirarent, partiendum sibi ac latius
distribuendum exercitum putavit.

\chap Itaque T.~Labienum legatum in Treveros, qui proximi
flumini Rheno sunt, cum equitatu mittit.  \sect Huic
imandat, Remos reliquosque Belgas adeat atque in
officio contineat Germanosque, qui auxilio a Belgis
arcessiti dicebantur, si per vim navibus flumen transire
conentur, prohibeat.  \sect P.~Crassum cum cohortibus legionariis
\textsc{xii} et magno numero equitatus in aquitaniam proficisci iubet,
ne ex his nationibus auxilia in
Galliam mittantur ac tantae nationes coniungantur.
\sect Q.~Titurium Sabinum legatum cum legionibus tribus
in Venellos, Coriosolites Lexoviosque mittit, qui eam
manum distinendam curet. \sect D.~Brutum adulescentem
classi Gallicisque navibus, quas ex Pictonibus et
Santonis reliquisque pacatis regionibus convenirel
iusserat, praeficit et, cum primum possit, in Venetos
proficisci iubet. Ipse eo pedestribus copiis contendit.

\chap Erant eius modi fere situs oppidorum ut posita
in extremis lingulis promunturiisque neque pedibus
aditum haberent, cum ex alto se acstus incitavisset,
quod [bis] accidit semper horarum \textsc{xii} spatio, neque
navibus, quod rursus minuente aestu naves in vadis
adflictarentur. \sect Ita utraque re oppidorum oppugnatio
impediebatur. \sect Ac si quando magnitudine operis forte
superati, extruso mari aggere ac molibus atque his
oppidi moenibus adaequatis, suis fortunis desperare
coeperant, magno numero navium adpulso, cuius rei
summam facultatem habebant, omnia sua deportabant
seque in proxima oppida recipiebant: \sect ibi se rursus
isdem oportunitatibus loci defendebant.  \sect Haec eo
facilius Inagnaln partem aestatis faciebant quod
nostrae naves tempestatibus detinebantur summaque
erat vasto atque aperto mari, rnagnis aestibus, raris
ac prope nullis portibus difficultas navigandi.

\chap Namque ipsorum naves ad hunc modum factae
armataeque erant: carinae aliquanto planiores quam
nostrarum navium, quo facilius vada ac decessum
aestus excipere possent; \sect prorae admodum erectae
atque item puppes, ad magnitudinem fluctuum
tempestatumque accommodatae; \sect naves totae factae ex
robore ad quamvis vim et contumeliam perferendam;
\sect transtra ex pedalibus in altitudinem trabibus, confixa
clavis ferreis digiti pollicis crassitudine; \sect ancorae pro
funibus ferreis catenis revinctae; \sect pelles pro velis
alutaeque tenuiter confectae, [hae] sive propter inopiam
lini atque eius usus inscientiam, sive eo, quod est
magis veri simile, quod tantas tempestates Oceani
tantosque impetus ventorum sustineri ac tanta onera
navium regi velis non satis commode posse arbitrabantur.
\sect Cum his navibus nostrae classi eius modi
congressus erat ut una celeritate et pulsu remorum
praestaret, reliqua pro loci natura, pro vi tempestatum
illis essent aptiora et accommodatiora. \sect Neque enim
iis nostrae rostro nocere poterant (tanta in iis erat 
firmitudo), neque propter altitudinem facile telum
adigebatur, et eadem de causa minus commode copulis
continebautur. \sect Accedebat ut, cum [saevire ventus
coepisset et] se vento dedissent, et tempestatem ferrent
facilius et in vadis consisterent tutius et ab aestu
relictae nihil saxa et cotes timerent; quarum rerum
omnium nostris navibus casus erat extimescendus.

\chap Compluribus expugnatis oppidis Caesar, ubi intellexit
frustra tantum laborem sumi neque hostium
fugam captis oppidis reprimi neque iis noceri posse,
statuit expectandam classem.  \sect Quae ubi convenit ac
primum ab hostibus visa est, circiter \textsc{ccxx} naves
eorum paratissimae atque omni genere armorum
ornatissimae profectae ex portu nostris adversae
constiterunt; \sect neque satis Bruto, qui classi praeerat, vel
tribunis militum centurionibusque, quibus singulae
naves erant attributae, constabat quid agerent aut
quam rationem pugnae insisterent.  \sect Rostro enim
noceri non posse cognoverant; turribus autem excitatis
tamen has altitudo puppium ex barbaris navibus
superabat, ut neque ex inferiore loco satis commode
tela adigi possent et missa a Gallis gravius acciderent.
\sect Una erat magno usui res praeparata a nostris, falces
praeacutae insertae adfixaeque longuriis, non absimili
forma muralium falcium. \sect His cum funes qui antemnas
ad malos destinabant comprehensi adductique erant,
navigio remis incitato praerumpebantur.  \sect Quibus
abscisis antemnae necessario concidebant, ut, cum
omnis Gallicis navibus spes in velis armamentisque
consisteret, his ereptis omnis usus navium uno tempore
eriperetur.  \sect Reliquum erat certamen positum in
virtute, qua nostri milites facile superabant, atque
eo magis quod in conspectu Caesaris atque omnis
exercitus res gerebatur, ut nullum paulo fortius
factum latere posset; \sect omnes enim colles ac loca
superiora, unde erat propinquus despectus in mare,
ab exercitu tenebantur.
\chap Deiectis, ut diximus, antemnis, cum singulas binae
ac ternae naves circumsteterant, milites summa vi
transcendere in hostium naves contendebant.  \sect Quod
postquam barbari fieri animadverterunt, expugnatis
compluribus navibus, cum ei rei nullum reperiretur
auxilium, fuga salutem petere contenderunt. \sect Ac iam
conversis in eam partem navibus quo ventus ferebat,
tanta subito malacia ac tranquillitas extitit ut se ex
loco movere non possent. \sect Quae quidem res ad negotium
conficiendum maximae fuit oportunitati: \sect nam
singulas nostri consectati expugnaverunt, ut perpaucae
ex omni numero noctis interventu ad terram per venirent,
cum ab hora fere \textsc{iiii}. usque ad solis occasum
pugnaretur.                         

\chap Quo proelio bellum Venetorum totiusque orae maritimae
confectum est.  \sect Nam cum omnis iuventus,
omnes etiam gravioris aetatis in quibus aliquid consilii
aut dignitatis fuit eo convenerant, tum navium quod
ubique fuerat in unum locum coegerant; \sect quibus
amissis reliqui neque quo se reciperent neque quem
ad modum oppida defenderent habebant. \sect Itaque se
suaque omnia Caesari dediderunt. In quos eo gravius
Caesar vindicandum statuit quo diligentius in reliquum
tempus a barbaris ius legatorum conservaretur.
Itaque omni senatu necato reliquos sub corona
vendidit.
\chap Dum haec in Venetis geruntur, Q.~Titurius Sabinus
cum iis copiis quas a Caesare acceperat in fines
Venellorum pervenit.  \sect His praeerat Viridovix ac
summam imperii tenebat earum omnium civitatum
quae defecerant, ex quibus exercitum [magnasque
scopias] coegerat; \sect atque his paucis diebus Aulerci
Eburovices Lexoviique, senatu suo interfecto quod
auctores belli esse nolebant, portas clauserunt seque
cum Viridovice coniunxerunt; \sect magnaque praeterea
multitudo undique ex Gallia perditorum hominum
latronumque convenerat, quos spes praedandi studiumque
bellandi ab agri cultura et cotidiano labore
revocabat.  \sect Sabinus idoneo omnibus rebus loco castris
sese tenebat, cum Viridovix contra eum duorum milium
spatio consedisset cotidieque productis copiis pugnandi
potestatem faceret, ut iam non solum hostibus in
contemptionem Sabinus veniret, sed etiam nostrorum
militum vocibus non nihil carperetur; \sect tantamque
opinionem timoris praebuit ut iam ad vallum castrorum
hostes accedere auderent. \sect Id ea de causa faciebat
quod cum tanta multitudine hostium, praesertim eo
absente qui summam imperii teneret, nisi aequo loco
aut oportunitate aliqua data legato dimicandum non
existimabat.

\chap Hac confirmata opinione timoris idoneum quendam 
hominem et callidum deligit, Gallum, ex iis quos
auxilii causa secum habebat.  \sect Huic magnis praemiis
pollicitationibusque persuadet uti ad hostes transeat,
et quid fieri velit edocet. \sect Qui ubi pro perfuga ad
eos venit, timorem Romanorum proponit, quibus
angustiis ipse Caesar a Venetis prematur docet, \sect eque
longius abesse quin proxima nocte Sabinus clam ex
castris exercitum educat et ad Caesarem auxilii ferendi
causa proficiscatur.  \sect Quod ubi auditum est, conclamant
omnes occasionem negotii bene gerendi amittendam non
esse: \sect ad castra iri oportere.  Multae res
ad hoc consilium Gallos hortabantur: superiorum
dierum Sabini cunctatio, perfugae confirmatio, inopia
cibariorum, cui rei paum diligenter ab iis erat
provisum, spes Venetici belli, et quod fere libenter.
homines id quod volunt credunt.  \sect His rebus adducti
non prius Viridovicem reliquosque duces ex concilio
dimittunt quam ab iis sit concessum arma uti capiant
et ad castra contendant. \sect Qua re concessa laeti, ut
explorata victoria, sarmentis virgultisque collectis,
quibus fossas Romanorum compleant, ad castra pergunt. 

\chap Locus erat castrorum editus et paulatim ab imo 
acclivis circiter passus mille. Huc magno cursu
contenderunt, ut quam minimum spatii ad se colligendos
armandosque Romanis daretur, exanimatique pervenerunt.
\sect Sabinus suos hortatus cupientibus signum dat.
Impeditis hostibus propter ea quae ferebant onera
subito duabus portis eruptionem fieri iubet.  \sect Factum
est oportunitate loci, hostium inscientia ac defatigatione,
virtute militum et superiorum pugnarum exercitatione,
ut ne unum quidem nostrorum impetum
ferrent ac statim terga verterent.  \sect Quos impeditos
integris viribus milites nostri consecuti magnum
numerum eorum occiderunt; reliquos equites consectati
paucos, qui ex fuga evaserant, reliquerunt.
\sect Sic uno tempore et de navali pugna Sabinus et de
Sabini victoria Caesar est certior factus, civitatesque
omnes se statim Titurio dediderunt. \sect Nam ut ad bella 
suscipienda Gallorum alacer ac promptus est animus,
sic mollis ac minime resistens ad calamitates ferendas
mens eorum est.

\chap Eodem fere tempore P.~Crassus., cum in Aquitaniam
pervenisset, quae [pars], ut ante dictum est,
[et regionum latitudine et multitudine hominum] tertia pars
Galliae est [aestimanda], cum intellegeret in iis locis
sibi bellum gerendum ubi paucis ante annis L.~Valerius
Praeconinus legatus exercitu pulso interfectus esset
atque unde L.~Manlius proconsul impedimentis amissis
profugisset, non mediocrem sibi diligentiam adhibendam
intellegebat.  \sect Itaque re frumentaria provisa,
auxiliis equitatuque comparato, multis praeterea viris
fortibus Tolosa et Carcasone et Narbone, quae sunt
civitates Galliae provinciae finitimae, ex his regionibus
nominatim evocatis, in Sotiatium fines exercitum introduxit.
\sect Cuius adventu cognito Sotiates magnis
copiis coactis, equitatuque, quo plurimum valebant,
in itinere agmen nostrum adorti primum equestre
proelium commiserunt, \sect deinde equitatu suo pulso
atque insequentibus nostris subito pedestres copias,
quas in convalle in insidiis conlocaverant, ostenderunt.
Hi nostros disiectos adorti proelium renovarunt.

\chap Pugnatum est diu atque acriter, cum Sotiates
superioribus victoriis freti in sua virtute totius
Aquitaniae salutem positam putarent, nostri autem quid
sine imperatore et sine reliquis legionibus adulescentulo
duce efficere possent perspici cuperent;
tandem confecti vulneribus hostes terga verterunt.
\sect Quorum magno numero interfecto Crassus ex itinere
oppidum Sotiatium oppugnare coepit.  Quibus fortiter
resistentibus vineas turresque egit.  \sect Illi alias eruptione
temptata, alias cuniculis ad aggerem vineasque actis
(cuius rei sunt longe peritissimi Aquitani, propterea
quod multis locis apud eos aerariae secturaeque sunt),
ubi diligentia nostrorum nihil his rebus profici posse
intellexerunt, legatos ad Crassum mittunt seque in
deditionem ut recipiat petunt. Qua re impetrata arma
tradere iussi faciunt.

\chap Atque in eam rem omnium nostrorum intentis
animis alia ex parte oppidi Adiatunnus, qui summam
imperii tenebat, cum \textsc{dc} devotis, quos illi soldurios
appellant, \sect quorum haec est condicio, ut omnibus in
vita commodis una cum iis fruantur quorum se
amicitiae dediderint, si quid his per vim accidat, aut
eundem casum una ferant aut sibi mortem consciscant;
\sect neque adhuc hominum memoria repertus est quisquam
qui, eo interfecto cuius se amicitiae devovisset, mortem
recusaret---\sect cum his Adiatunnus eruptionem facere
conatus clamore ab ea parte munitionis sublato cum
ad arma milites concurrissent vehementerque ibi
pugnatum esset, repulsus in oppidum tamen uti eadem
deditionis condicione uteretur a Crasso impetravit.

\chap Armis obsidibusque acceptis, Crassus in fines Vocatium
et Tarusatium profectus est. \sect Tum vero barbari
commoti, quod oppidum et natura loci et manu
munitum paucis diebus quibus eo ventum erat
expugnatum cognoverant, legatos quoque versus
dimittere, coniurare, obsides inter se dare, copias parare
coeperunt. \sect Mittuntur etiam ad eas civitates legati
quae sunt citerioris Hispaniae finitimae Aquitaniae:
inde auxilia ducesque arcessuntur.  \sect Quorum adventu
magna cum auctoritate et magna [cum] hominum
multitudine bellum gerere conantur. \sect Duces vero ii
deliguntur qui una cum Q.~Sertorio omnes annos
fuerant summamque scientiam rei militaris habere
existimabantur.  \sect Hi consuetudine populi Romani loca
capere, castra munire, commeatibus nostros
intercludere instituunt. \sect Quod ubi Crassus animadvertit,
suas copias propter exiguitatem non facile diduci,
hostem et vagari et vias obsidere et castris satis
praesidii relinquere, ob eam causam minus commode
frumentum commeatumque sibi supportari, in dies
hostium numcrum augeri, non cunctandum existimavit
quin pugna decertaret. \sect Hac re ad consilium
delata, ubi omnes idem sentire intellexit, posterum
diem pugnae constituit.

\chap Prima luce productis omnibus copiis duplici acie
instituta, auxiliis in mediam aciem coniectis, quid
hostes consilii caperent expectabat.  \sect Illi, etsi propter
multitudinem et veterem belli gloriam paucitatemque
nostrorum se tuto dimicaturos existimabant, tamen
tutius esse arbitrabantur obsessis viis commeatu intercluso
sine vulnere victoria potiri, \sect et si propter inopiam
rei frumentariae Romani se recipere coepissent,
impeditos in agmine et sub sarcinis infirmiores animo
adoriri cogitabant.  \sect Hoc consilio probato ab ducibus,
productis Romanorum copiis, sese castris tenebant.
\sect Hac re perspecta Crassus, cum sua cunctatione atque
opinione timoris hostes nostros milites alacriores ad
pugnaudum effecissent atque omnium voces audirentur
expectari diutius non oportere quin ad castra iretur,
cohortatus suos omnibus cupientibus ad hostium castra
contendit.

\chap {\DriveOut Ibi cum alii fossas complerent, alii multis telis
coniectis defensores vallo munitionibusque depellerent,
auxiliaresque, quibus ad pugnam non multum Crassus
confidebat, lapidibus telisque subministrandis et ad
aggerem caespitibus comportandis speciem atque
opinionem pugnantium praeberent, cum item ab hostibus
constanter ac non timide pugnaretur telaque ex loco
superiore missa non frustra acciderent,} \sect equites
circumitis hostium castris Crasso renuntiaverunt non
eadem esse diligentia ab decumana porta castra munita
facilemque aditum habere.

\chap Crassus equitum praefectos cohortatus, ut magnis
praemiis pollicitationibusque suos excitarent, quid fieri
vellet ostendit. \sect Illi, ut erat imperatum, eductis iis
cohortibus quae praesidio castris relictae intritae ab
labore erant, et longiore itinere circumductis, ne ex
hostium castris conspici possent, omnium oculis
mentibusque ad pugnam intentis celeriter ad eas quas
diximus munitiones pervenerunt \sect atque his prorutis
prius in hostium castris constiterunt quam plane ab
his videri aut quid rei gereretur cognosci posset.
\sect Tum vero clamore ab ea parte audito nostri redintegratis
viribus, quod plerumque in spe victoriae accidere
consuevit, acrius impugnare coeperunt.  \sect Hostes undique
circumventi desperatis omnibus rebus se per
munitiones deicere et fuga salutem petere contenderunt.
\sect Quos equitatus apertissimis campis consectatus ex
milium L numero, quae ex Aquitania Cantabrisque
convenisse constabat, vix quarta parte relicta, multa
nocte se in castra recepit.

\chap Hac audita pugna maxima pars Aquitaniae sese
Crasso dedidit obsidesque ultro misit; quo in numero
fuerunt Tarbelli, Bigerriones, Ptianii, Vocates,
Tarusates, Elusates, Gates, Ausci, Garumni, Sibusates,
Cocosates: \sect paucae ultimae nationes anni tempore confisae,
quod hiems suberat, id facere neglexerunt.

\chap Eodem fere tempore Caesar, etsi pmpe exacta iam
aestas erat, tamen, quod omni Gallia pacata Morini
Menapiique supererant, qui in armis essent neque ad
eum umquam legatos de pace misissent, arbitratus id
bellum celeriter confici posse eo exercitum duxit; qui
longe alia ratione ac reliqui Galli bellum gerere coeperunt.
\sect Nam quod intellegebant maximas nationes,
quae proelio contendissent, pulsas superatasque esse,
continentesque silvas ac paludes habebant, eo se suaque
omnia contulerunt.  \sect Ad quarum initium silvarum cum
Caesar pervenisset castraque munire instituisset neque
hostis interim visus esset, dispersis in opere nostris
subito ex omnibus partibus silvae evolaverunt et in
nostros impetum fecerunt.  \sect Nostri celeriter arma
ceperunt eosque in silvas repulerunt et compluribus
interfectis longius impeditioribus locis secuti paucos
ex suis deperdiderunt.

\chap Reliquis deinceps diebus Caesar silvas caedere
instituit, et ne quis inermibus imprudentibusque
militibus ab latere impetus fieri posset, omnem eam
materiam quae erat caesa conversam ad hostem conlocabat
et pro vallo ad utrumque latus extruebat.
\sect Incredibili celeritate magno spatio paucis diebus
confecto, cum iam pecus atque extrema impedimenta a
nostris tenerentur, ipsi densiores silvas peterent, eius
modi sunt tempestates consecutae uti opus necessario
intermitteretur et continuatione imbrium diutius sub
pellibus milites contineri non possent. \sect Itaque vastatis
omnibus eorum agris, vicis aedificiisque incensis,
Caesar exercitum reduxit et in Aulercis Lexoviisque,
reliquis item civitatibus quae proxime bellum fecerant,
in hibernis conlocavit.

\Liber

\chap \incipit{Ea} {\DriveOut quae secuta est hieme, qui fuit annus Cn.~Pompeio,
M.~Crasso consulibus, Usipetes Germani et item Tencteri magna [cum]
multitudine hominum flumen Rhenum transierunt, non longe a mari, quo
Rhenus influit}.  \sect Causa transeundi fuit quod ab Suebis complures
annos exagitati bello premebantur et agri cultura prohibebantur.
\sect Sueborum gens est longe maxima et bellicosissima Germanorum omnium.
\sect Hi centum pagos habere dicuntur, ex quibus quotannis singula milia
armatorum bellandi causa ex finibus educunt. Reliqui, qui domi
manserunt, se atque illos alunt; \sect hi rursus in vicem anno post in
armis sunt, illi domi remanent. \sect Sic neque agri cultura nec ratio
atque usus belli intermittitur. \sect Sed privati ac separati agri apud eos
nihil est, neque longius anno remanere uno in loco colendi causa licet.
\sect Neque multum frumento, sed maximam partem lacte atque pecore vivunt
multum sunt in venationibus; \sect quae res et cibi genere et cotidiana
exercitatione et libertate vitae, quod a pueris nullo officio aut
disciplina adsuefacti nihil omnino contra voluntatem faciunt, et vires
alit et immani corporum magnitudine homines efficit. \sect Atque in eam
se consuetudinem adduxerunt ut locis frigidissimis neque vestitus
praeter pelles habeant quicquam, quarum propter exiguitatem magna est
corporis pars aperta, et laventur in fluminibus.

\chap Mercatoribus est aditus magis eo ut quae bello ceperint quibus
vendant habeant, quam quo ullam rem ad se importari desiderent. \sect
Quin etiam iumentis, quibus maxime Galli delectantur quaeque impenso
parant pretio, Germani importatis non utuntur, sed quae sunt apud eos
nata, parva atque deformia, haec cotidiana exercitatione summi ut sint
laboris efficiunt.  \sect Equestribus proeliis saepe ex equis desiliunt
ac pedibus proeliantur, equos eodem remanere vestigio adsue fecerunt, ad
quos se celeriter, cum usus est, recipiunt: \sect neque eorum moribus
turpius quicquam aut inertius habetur quam ephippiis uti. \sect Itaque ad
quemvis numerum ephippiatorum equitum quamvis pauci adire audent. \sect
Vinum omnino ad se importari non patiuntur, quod ea re ad laborem
ferendum remollescere homines atque effeminari arbitrantur.

\chap Publice maximam putant esse laudem quam latissime a suis finibus
vacare agros: hac re significari magnum numerum civitatum suam vim
sustinere non posse.  \sect Itaque una ex parte a Suebis circiter milia
passuum C agri vacare dicuntur. \sect Ad alteram partem succedunt Ubii,
quorum fuit civitas ampla atque florens, ut est captus Germanorum; ii
paulo, quamquam sunt eiusdem generis, sunt ceteris humaniores, propterea
quod Rhenum attingunt multum ad eos mercatores ventitant et ipsi propter
propinquitatem [quod] Gallicis sunt moribus adsuefacti. \sect Hos cum
Suebi multis saepe bellis experti propter amplitudinem gravitatem
civitatis finibus expellere non potuissent, tamen vectigales sibi
fecerunt ac multo humiliores infirmiores redegerunt.

\chap In eadem causa fuerunt Usipetes et Tencteri, quos supra diximus;
qui complures annos Sueborum vim sustinuerunt, \sect ad extremum tamen
agris expulsi et multis locis Germaniae triennium vagati ad Rhenum
pervenerunt, quas regiones Menapii incolebant. Hi ad utramque ripam
fluminis agros, aedificia vicosque habebant; \sect sed tantae
multitudinis adventu perterriti ex iis aedificiis quae trans flumen
habuerant demigraverant, et cis Rhenum dispositis praesidiis Germanos
transire prohibebant. \sect Illi omnia experti, cum neque vi contendere
propter inopiam navium neque clam transire propter custodias Menapiorum
possent, \sect reverti se in suas sedes regiones simulaverunt et tridui
viam progressi rursus reverterunt atque omni hoc itinere una nocte
equitatu confecto inscios inopinantes Menapios oppresserunt, \sect qui de
Germanorum discessu per exploratores certiores facti sine metu trans
Rhenum in suos vicos remigraverant. \sect His interfectis navibus eorum
occupatis, prius quam ea pars Menapiorum quae citra Rhenum erat certior
fieret, flumen transierunt atque omnibus eorum aedificiis occupatis
reliquam partem hiemis se eorum copiis aluerunt.

\chap His de rebus Caesar certior factus et infirmitatem Gallorum
veritus, quod sunt in consiliis capiendis mobiles et novis plerumque
rebus student, nihil his committendum existimavit. \sect Est enim hoc
Gallicae consuetudinis, uti et viatores etiam invitos consistere cogant
et quid quisque eorum de quaque re audierit aut cognoverit quaerant et
mercatores in oppidis vulgus circumsistat quibus ex regionibus veniant
quas ibi res cognoverint pronuntiare cogat. \sect His rebus atque
auditionibus permoti de summis saepe rebus consula ineunt, quorum eos in
vestigio paenitere necesse est, cum incertis rumoribus serviant et pleri
ad voluntatem eorum ficta respondeant.

\chap Qua consuetudine cognita Caesar, ne graviori bello, occurreret,
maturius quam consuerat ad exercitum proficiscitur. \sect Eo cum
venisset, ea quas fore suspicatus erat facta cognovit: \sect missas
legationes ab non nullis civitatibus ad Germanos invitatos eos uti ab
Rheno discederent: omnia quae[que] postulassent ab se fore parata. \sect
Qua spe adducti Germani latius iam vagabantur et in fines Eburonum et
Condrusorum, qui sunt Treverorum clientes, pervenerant. \sect Principibus
Gallice evocatis Caesar ea quae cognoverat dissimulanda sibi
existimavit, eorumque animis permulsis et confirmatis equitatu imperato
bellum cum Germanis gerere constituit.

\chap Re frumentaria comparata equitibusque delectis iter in ea loca
facere coepit, quibus in locis esse Germanos audiebat. \sect A quibus cum
paucorum dierum iter abesset, legati ab iis venerunt, quorum haec fuit
oratio: \sect Germanos neque priores populo Romano bellum inferre neque
tamen recusare, si lacessantur, quin armis contendant, quod Germanorum
consuetudo [haec] sit a maioribus tradita, Quicumque bellum inferant,
resistere neque deprecari. Haec tamen dicere venisse invitos, eiectos
domo; \sect si suam gratiam Romani velint, posse iis utiles esse amicos;
vel sibi agros attribuant vel patiantur eos tenere quos armis
possederint: \sect sese unis Suebis concedere, quibus ne di quidem
immortales pares esse possint; reliquum quidem in terris esse neminem
quem non superare possint.

\chap Ad haec Caesar quae visum est respondit; sed exitus fuit
orationis: sibi nullam cum iis amicitiam esse posse, si in Gallia
remanerent; \sect neque verum esse, qui suos fines tueri non potuerint
alienos occupare; neque ullos in Gallia vacare agros qui dari tantae
praesertim multitudini sine iniuria possint; \sect sed licere, si velint,
in Ubiorum finibus considere, quorum sint legati apud se et de Sueborum
iniuriis querantur et a se auxilium petant: hoc se Ubiis imperaturus.

\chap Legati haec se ad suos relaturos dixerunt et re deliberata post
diem tertium ad Caesarem reversuros: interea ne propius se castra
moveret petierunt. \sect Ne id quidem Caesar ab se impetrari posse dixit.
\sect Cognoverat enim magnam partem equitatus ab iis aliquot diebus ante
praedandi frumentandi causa ad Ambivaritos trans Mosam missam: hos
expectari equites atque eius rei causa moram interponi arbitrabatur.

\chap [Mosa profluit ex monte Vosego, qui est in finibus Lingonum, et
parte quadam ex Rheno recepta, quae appellatur Vacalus insulam efficit
Batavorum, in Oceanum influit \sect neque longius ab Oceano milibus
passuum \textsc{lxxx} in rhenum influit. \sect rhenus autem oritur ex lepontiis, qui
Alpes incolunt, et longo spatio per fines Nantuatium, Helvetiorum,
Sequanorum, Mediomatricorum, Tribocorum, Treverorum citatus fertur et,
\sect ubi Oceano adpropinquavit, in plures diffluit partes multis
ingentibus insulis effectis, quarum pars magna a feris barbaris
nationibus incolitur, \sect ex quibus sunt qui piscibus atque ovis avium
vivere existimantur, multis capitibus in Oceanum influit.]

\chap Caesar cum ab hoste non amplius passuum \textsc{xii}: milibus abesset, ut
erat constitutum, ad eum legati revertuntur; qui in itinere congressi
magnopere ne longius progrederetur orabant. \sect Cum id non
impetrassent, petebant uti ad eos [equites] qui agmen antecessissent
praemitteret eos pugna prohiberet, sibique ut potestatem faceret in
Ubios legatos mittendi; \sect quorum si principes ac senatus sibi iure
iurando fidem fecisset, ea condicione quae a Caesare ferretur se usuros
ostendebant: ad has res conficiendas sibi tridui spatium daret. \sect
Haec omnia Caesar eodem illo pertinere arbitrabatur ut tridui mora
interposita equites eorum qui abessent reverterentur; tamen sese non
longius milibus passuum \textsc{iiii} aquationis causa processurum eo die dixit:
\sect huc postero die quam frequentissimi convenirent, ut de eorum
postulatis cognosceret. \sect Interim ad praefectos, qui cum omni
equitatu antecesserant, mittit qui nuntiarent ne hostes proelio
lacesserent, et si ipsi lacesserentur, sustinerent quoad ipse cum
exercitu propius accessisset.

\chap At hostes, ubi primum nostros equites conspexerunt, quorum erat V
milium numerus, cum ipsi non amplius \textsc{dccc} equites haberent, quod ii qui
frumentandi causa erant trans Mosam profecti nondum redierant, nihil
timentibus nostris, quod legati eorum paulo ante a Caesare discesserant
atque is dies indutiis erat ab his petitus, impetu facto celeriter
nostros perturbaverunt; \sect rursus his resistentibus consuetudine sua
ad pedes desiluerunt subfossis equis compluribus nostris deiectis
reliquos in fugam coniecerunt atque ita perterritos egerunt ut non prius
fuga desisterent quam in conspectum agminis nostri venissent. \sect In eo
proelio ex equitibus nostris interficiuntur \textsc{iiii} et
\textsc{lxx}, \sect in his vir
fortissimus Piso Aquitanus, amplissimo genere natus, cuius avus in
civitate sua regnum obtinuerat amicus a senatu nostro appellatus.  \sect
Hic cum fratri intercluso ab hostibus auxilium ferret, illum ex periculo
eripuit, ipse equo vulnerato deiectus, quoad potuit, fortissime
restitit; \sect cum circumventus multis vulneribus acceptis cecidisset
atque id frater, qui iam proelio excesserat, procul animadvertisset,
incitato equo se hostibus obtulit atque interfectus est.

\chap Hoc facto proelio Caesar neque iam sibi legatos audiendos neque
condiciones accipiendas arbitrabatur ab iis qui per dolum atque insidias
petita pace ultro bellum intulissent; \sect expectare vero dum hostium
copiae augerentur equitatus reverteretur summae dementiae esse
iudicabat, \sect et cognita Gallorum infirmitate quantum iam apud eos
hostes uno proelio auctoritatis essent consecuti sentiebat; quibus ad
consilia capienda nihil spatii dandum existimabat. \sect His constitutis
rebus et consilio cum legatis et quaestore communicato, ne quem diem
pugnae praetermitteret, oportunissima res accidit, quod postridie eius
diei mane eadem et perfidia et simulatione usi Germani frequentes,
omnibus principibus maioribusque natu adhibitis, ad eum in castra
venerunt, \sect simul, ut dicebatur, sui purgandi causa, quod contra
atque esset dictum et ipsi petissent, proelium pridie commisissent,
simul ut, si quid possent, de indutiis fallendo impetrarent. \sect Quos
sibi Caesar oblatos gavisus illos retineri iussit; ipse omnes copias
castris D eduxit equitatumque, quod recenti proelio perterritum esse
existimabat, agmen subsequi iussit.

\chap Acie triplici instituta et celeriter \textsc{viii} milium itinere confecto,
prius ad hostium castra pervenit quam quid ageretur Germani sentire
possent.  \sect Qui omnibus rebus subito perterriti et celeritate
adventus nostri et discessu suorum, neque consilii habendi neque arma
capiendi spatio dato perturbantur, copiasne adversus hostem ducere an
castra defendere an fuga salutem petere praestaret. \sect Quorum timor
cum fremitu et concursu significaretur, milites nostri pristini diei
perfidia incitati in castra inruperunt.  \sect Quo loco qui celeriter
arma capere potuerunt paulisper nostris restiterunt atque inter carros
impedimenta proelium commiserunt; \sect at reliqua multitudo puerorum
mulierum (nam eum omnibus suis domo excesserant Rhenum transierant)
passim fugere coepit, ad quos consectandos Caesar equitatum misit.

\chap Germani post tergum clamore audito, eum suos interfiei viderent,
armis abiectis signis militaribus relictis se ex castris eiecerunt, \sect
et eum ad confluentem Mosae et Rheni pervenissent, reliqua fuga
desperata, magno numero interfecto, reliqui se in flumen
praecipitaverunt atque ibi timore, lassitudine, vi fluminis oppressi
perierunt. \sect Nostri ad unum omnes incolumes, perpaucis vulneratis, ex
tanti belli timore, cum hostium numerus capitum \textsc{ccccxxx} milium fuisset,
se in castra receperunt.  \sect Caesar iis quos in castris retinuerat
discedendi potestatem fecit. \sect Illi supplicia cruciatusque Gallorum
veriti, quorum agros vexaverant, remanere se apud eum velle dixerunt.
His Caesar libertatem concessit.

\chap Germanico bello confecto multis de causis Caesar statuit sibi
Rhenum esse transeundum; quarum illa fuit iustissima quod, cum videret
Germanos tam facile impelli ut in Galliam venirent, suis quoque rebus
eos timere voluit, cum intellegerent et posse et audere populi Romani
exercitum Rhenum transire. \sect Accessit etiam quod illa pars equitatus
Usipetum et Tencterorum, quam supra commemoravi praedandi frumentandi
causa Mosam transisse neque proelio interfuisse, post fugam suorum se
trans Rhenum in fines Sugambrorum receperat seque cum his coniunxerat.
\sect Ad quos cum Caesar nuntios misisset, qui postularent eos qui sibi
Galliae bellum intulissent sibi dederent, responderunt: \sect populi
Romani imperium Rhenum finire; si se invito Germanos in Galliam transire
non aequum existimaret, cur sui quicquam esse imperii aut potestatis
trans Rhenum postularet? \sect Ubii autem, qui uni ex Transrhenanis ad
Caesarem legatos miserant, amicitiam fecerant, obsides dederant,
magnopere orabant ut sibi auxilium ferret, quod graviter ab Suebis
premerentur; \sect vel, si id facere occupationibus rei publicae
prohiberetur, exercitum modo Rhenum transportaret: id sibi
$\langle$ad$\rangle$ auxilium spemque reliqui temporis satis futurum.
\sect Tantum esse nomen atque opinionem eius exercitus Ariovisto pulso et
hoc novissimo proelio facto etiam ad ultimas Germanorum nationes, uti
opinione et amicitia populi Romani tuti esse possint.  \sect Navium
magnam copiam ad transportandum exercitum pollicebantur.

\chap Caesar his de causis quas commemoravi Rhenum transire decrevat;
sed navibus transire neque satis tutum esse arbitrabatur neque suae
neque populi Romani dignitatis esse statuebat. \sect Itaque, etsi summa
difficultas faciendi pontis proponebatur propter latitudinem,
rapiditatem altitudinemque fluminis, tamen id sibi contendendum aut
aliter non traducendum exercitum existimabat. \sect Rationem pontis hanc
instituit. Tigna bina sesquipedalia. paulum ab imo praeacuta dimensa ad
altitudinem fluminis intervallo pedum duorum inter se iungebat. \sect
Haec cum machinationibus immissa in flumen defixerat fistucisque
adegerat, non sublicae modo derecte ad perpendiculum, sed prone ac
fastigate, ut secundum naturam fluminis procumberent, \sect iis item
contraria duo ad eundem modum iuncta intervallo pedum quadragenum ab
inferiore parte contra vim atque impetu fluminis conversa statuebat.
\sect Haec utraque insuper bipedalibus trabibus immissis, quantum eorum
tignorum iunctura distabat, binis utrimque fibulis ab extrema parte
distinebantur; \sect quibus disclusis atque in contrariam partem
revinctis, tanta erat operis firmitudo atque ea rerum natura ut, quo
maior vis aquae se incitavisset, hoc artius inligata tenerentur. \sect
Haec derecta materia iniecta contexebantur ac longuriis cratibusque
consternebantur; \sect ac nihilo setius sublicae et ad inferiorem partem
fluminis oblique agebantur, quae pro ariete subiectae et cum omni opere
coniunctae vim fluminis exciperent, \sect et aliae item supra pontem
mediocri spatio, ut, si arborum trunci sive naves deiciendi operis causa
essent a barbaris missae, his defensoribus earum rerum vis minueretur
neu ponti nocerent.

\chap Diebus X, quibus materia coepta erat comportari, omni opere
effecto exercitus traducitur. \sect Caesar ad utramque partem pontis
firmo praesidio relicto in fines Sugambrorum contendit. \sect Interim a
compluribus civitatibus ad eum legati veniunt; quibus pacem atque
amicitiam petentibus liberaliter respondet obsidesque ad se adduci
iubet. \sect At Sugambri, ex eo tempore quo pons institui coeptus est
fuga comparata, hortantibus iis quos ex Tencteris atque Usipetibus apud
se habebant, finibus suis excesserant suaque omnia exportaverant seque
in solitudinem ac silvas abdiderant.

\chap Caesar paucos dies in eorum finibus moratus, omnibus vicis
aedificiisque incensis frumentisque succisis, se in fines Ubiorum
recepit atque his auxilium suum pollicitus, si a Suebis premerentur,
haec ab iis cognovit: \sect Suebos, postea quam per exploratores pontem
fieri comperissent, more suo concilio habito nuntios in omnes partes
dimisisse, uti de oppidis demigrarent, liberos, uxores suaque omnia in
silvis deponerent atque omnes qui arma ferre possent unum in locum
convenirent. \sect Hunc esse delectum medium fere regionum earum quas
Suebi obtinerent; hic Romanorum adventum expectare atque ibi decertare
constituisse. \sect Quod ubi Caesar comperit, omnibus iis rebus
confectis, quarum rerum causa exercitum traducere constituerat, ut
Germanis metum iniceret, ut Sugambros ulcisceretur, ut Ubios obsidione
liberaret, diebus omnino \textsc{xviii} trans rhenum consumptis, satis et ad
laudem et ad utilitatem profectum arbitratus se in Galliam recepit
pontemque rescidit.

\chap Exigua parte aestatis reliqua Caesar, etsi in his locis, quod
omnis Gallia ad septentriones vergit, maturae sunt hiemes, tamen in
Britanniam proficisci contendit, quod omnibus fere Gallicis bellis
hostibus nostris inde subministrata auxilia intellegebat, \sect et si
tempus anni ad bellum gerendum deficeret, tamen magno sibi usui fore
arbitrabatur, si modo insulam adiisset, genus hominum perspexisset,
loca, portus, aditus cognovisset; quae omnia fere Gallis erant
incognita. \sect Neque enim temere praeter mercatores illo adit quisquam,
neque his ipsis quicquam praeter oram maritimam atque eas regiones quae
sunt contra Galliam notum est.  \sect Itaque vocatis ad se undique
mercatoribus, neque quanta esset insulae magnitudo neque quae aut
quantae nationes incolerent, neque quem usum belli haberent aut quibus
institutis uterentur, neque qui essent ad maiorem navium multitudinem
idonei portus reperire poterat.

\chap Ad haec cognoscenda, prius quam periculum faceret, idoneum esse
arbitratus C.~Volusenum cum navi longa praemittit. \sect Huic mandat ut
exploratis omnibus rebus ad se quam primum revertatur. \sect Ipse cum
omnibus copiis in Morinos proficiscitur, quod inde erat brevissimus in
Britanniam traiectus. \sect Huc naves undique ex finitimis regionibus et
quam superiore aestate ad Veneticum bellum fecerat classem iubet
convenire.
\sect Interim, consilio eius cognito et per mercatores perlato ad
Britannos, a compluribus insulae civitatibus ad eum legati veniunt, qui
polliceantur obsides dare atque imperio populi Romani obtemperare. \sect
Quibus auditis, liberaliter pollicitus hortatusque ut in ea sententia
permanerent, \sect eos domum remittit et cum iis una Commium, quem ipse
Atrebatibus superatis regem ibi constituerat, cuius et virtutem et
consilium probabat et quem sibi fidelem esse arbitrabatur cuiusque
auctoritas in his regionibus magni habebatur, mittit.  \sect Huic imperat
quas possit adeat civitates horteturque ut populi Romani fidem sequantur
seque celeriter eo venturum nuntiet. \sect Volusenus perspectis
regionibus omnibus quantum ei facultatis dari potuit, qui navi egredi ac
se barbaris committere non auderet, V.~die ad Caesarem revertitur
quaeque ibi perspexisset renuntiat.

\chap Dum in his locis Caesar navium parandarum causa moratur, ex magna
parte Morinorum ad eum legati venerunt, qui se de superioris temporis
consilio excusarent, quod homines barbari et nostrae consuetudinis
imperiti bellum populo Romano fecissent, seque ea quae imperasset
facturos pollicerentur. \sect Hoc sibi Caesar satis oportune accidisse
arbitratus, quod neque post tergum hostem relinquere volebat neque belli
gerendi propter anni tempus facultatem habebat neque has tantularum
rerum occupationes Britanniae anteponendas iudicabat, magnum iis numerum
obsidum imperat.

\chap Quibus adductis eos in fidem recipit. Navibus
circiter \textsc{lxxx} onerariis coactis contractisque, quot satis esse ad duas
transportandas legiones existimabat, quod praeterea navium longarum
habebat quaestori, legatis praefectisque distribuit. \sect Huc accedebant
\textsc{xviii} onerariae naves, quae ex eo loco a milibus passuum
\textsc{viii} vento
tenebantur quo minus in eundem portum venire possent: has equitibus
tribuit. \sect Reliquum exercitum Q.~Titurio Sabino et L.~Aurunculeio
Cottae legatis in Menapios atque in eos pagos Morinorum a quibus ad eum
legati non venerant ducendum dedit. \sect P.~Sulpicium Rufum legatum cum
eo praesidio quod satis esse arbitrabatur portum tenere iussit.

\chap His constitutis rebus, nactus idoneam ad navigandum tempestatem
\textsc{iii}.  fere vigilia solvit equitesque in ulteriorem portum progredi et
naves conscendere et se sequi iussit. \sect A quibus cum paulo tardius
esset administratum, ipse hora diei circiter \textsc{iiii}. cum primis navibus
Britanniam attigit atque ibi in omnibus collibus expositas hostium
copias armatas conspexit.  \sect Cuius loci haec erat natura atque ita
montibus angustis mare continebatur, uti ex locis superioribus in litus
telum adigi posset. \sect Hunc ad egrediendum nequaquam idoneum locum
arbitratus, dum reliquae naves eo convenirent ad horam nonam in ancoris
expectavit.  \sect Interim legatis tribunisque militum convocatis et quae
ex Voluseno cognovisset et quae fieri vellet ostendit monuitque, ut rei
militaris ratio, maximeque ut maritimae res postularent, ut, cum celerem
atque instabilem motum haberent, ad nutum et ad tempus D omnes res ab
iis administrarentur. \sect his dimissis, et \textsc{vii} ab eo loco progressus
aperto ac plano litore naves constituit.

\chap At barbari, consilio Romanorum cognito praemisso equitatu et
essedariis, quo plerumque genere in proeliis uti consuerunt, reliquis
copiis subsecuti nostros navibus egredi prohibebant. \sect Erat ob has
causas summa difficultas, quod naves propter magnitudinem nisi in alto
constitui non poterant, militibus autem, ignotis locis, impeditis
manibus, magno et gravi onere armorum oppressis simul et de navibus
desiliendum et in auctibus consistendum et cum hostibus erat pugnandum,
\sect cum illi aut ex arido aut paulum in aquam progressi omnibus membris
expeditis, notissimis locis, audacter tela coicerent et equos
insuefactos incitarent.  \sect Quibus rebus nostri perterriti atque huius
omnino generis pugnae imperiti, non eadem alacritate ac studio quo in
pedestribus uti proeliis consuerant utebantur.

\chap Quod ubi Caesar animadvertit, naves longas, quarum et species erat
barbaris inusitatior et motus ad usum expeditior, paulum removeri ab
onerariis navibus et remis incitari et ad latus apertum hostium
constitui atque inde fundis, sagittis, tormentis hostes propelli ac
submoveri iussit; quae res magno usui nostris fuit. \sect Nam et navium
figura et remorum motu et inusitato genere tormentorum permoti barbari
constiterunt ac paulum modo pedem rettulerunt. \sect Atque nostris
militibus cunctantibus, maxime propter altitudinem maris, qui X legionis
aquilam gerebat, obtestatus deos, ut ea res legioni feliciter eveniret,
' desilite', inquit, ' milites, nisi vultis aquilam hostibus prodere;
ego certe meum rei publicae atque imperatori officium praestitero.' \sect
Hoc cum voce magna dixisset, se ex navi proiecit atque in hostes aquilam
ferre coepit.  \sect Tum nostri cohortati inter se, ne tantum dedecus
admitteretur, universi ex navi desiluerunt. \sect Hos item ex proximis
primi navibus cum conspexissent, subsecuti hostibus adpropinquaverunt.

\chap Pugnatum est ab utrisque acriter. Nostri tamen, quod neque ordines
servare neque firmiter insistere neque signa subsequi poterant atque
alius alia ex navi quibuscumque signis occurrerat se adgregabat,
magnopere perturbabantur; \sect hostes vero, notis omnibus vadii, ubi ex
litore aliquos singulares ex navi egredientes conspexerant, incitatis
equis impeditos adoriebantur, \sect plures paucos circumsistebant, alii
ab latere aperto in universos tela coiciebant. \sect Quod cum
animadvertisset Caesar, scaphas longarum navium, item speculatoria
navigia militibus compleri iussit, et quos laborantes conspexerat, his
subsidia submittebat.  \sect Nostri, simul in arido constiterunt, suis
omnibus consecutis, in hostes impetum fecerunt atque eos in fugam
dederunt; neque longius prosequi potuerunt, quod equites cursum tenere
atque insulam capere non potuerant. Hoc unum ad pristinam fortunam
Caesari defuit.

\chap Hostes proelio superati, simul atque se ex fuga: receperunt,
statim ad Caesarem legatos de pace miserunt; obsides sese daturos
quaeque imperasset facturos polliciti sunt. \sect Una cum his legatis
Commius Atrebas venit, quem supra demonstraveram a Caesare in Britanniam
praemissum.  \sect Hunc illi e navi egressum, cum ad eos oratoris modo
Caesaris mandata deferret, comprehenderant atque in vincula coniecerant;
\sect tum proelio facto remiserunt et in petenda pace eius rei culpam in
multitudinem contulerunt et propter imprudentiam ut ignosceretur
petiverunt. \sect Caesar questus quod, cum ultro in continentem legatis
missis pacem ab se petissent, bellum sine causa intulissent, ignoscere
$\langle$se$\rangle$ imprudentiae dixit obsidesque imperavit; \sect
quorum illi partem statim dederunt, partem ex longinquioribus locis
arcessitam paucis diebus sese daturos dixerunt.  \sect Interea suos in
agros remigrare iusserunt, principesque undique convenire et se
civitatesque suas Caesari commendare coeperunt.

\chap His rebus pace confirmata, post diem quartum quam est in
Britanniam ventum naves \textsc{xviii}, de quibus supra demonstratum est, quae
equites sustulerant, ex superiore portu leni vento solverunt. \sect Quae
cum adpropinquarent Britanniae et ex castris viderentur, tanta tempestas
subito coorta est ut nulla earum cursum tenere posset, sed aliae eodem
unde erant profectae referrentur, aliae ad inferiorem partem insulae,
quae est propius solis occasum, magno suo cum periculo deicerentur; \sect
quae tamen ancoris iactis cum fluctibus complerentur, necessario adversa
nocte in altum provectae continentem petierunt.

\chap Eadem nocte accidit ut esset luna plena, qui dies a maritimos
aestus maximos in Oceano efficere consuevit, nostrisque id erat
incognitum.  \sect Ita uno tempore et longas naves, [quibus Caesar
exercitum transportandum curaverat,] quas Caesar in aridum subduxerat,
aestus complebat, et onerarias, quae ad ancoras erant deligatae,
tempestas adflictabat, neque ulla nostris facultas aut administrandi `
aut auxiliandi dabatur.  \sect Compluribus navibus fractis, reliquae cum
essent funibus, ancoris reliquisque armamentis amissis ad navigandum
inutiles, magna, id quod necesse erat accidere, totius exercitus
perturbatio facta est. \sect Neque enim naves erant aliae quibus
reportari possent, et omnia deerant quae ad reficiendas naves erant
usui, et, quod omnibus constabat hiemari in Gallia oportere, frumentum
in his locis in hiemem provisum non erat.  Quibus rebus cognitis,
principes Britanniae, qui post proelium ad Caesarem convenerant, inter
se conlocuti, cum et equites et naves et frumentum Romanis deesse
intellegerent et paucitatem militum ex castrorum exiguitate
cognoscerent, quae hoc erant etiam angustior quod sine impedimentis
Caesar legiones transportaverat, \sect optimum factu esse duxerunt
rebellione facta frumento commeatuque nostros prohibere et rem in hiemem
producere, quod his superatis aut reditu interclusis neminem postea
belli inferendi causa in Britanniam transiturum confidebant.  Itaque
rursus coniuratione facta paulatim ex castris discedere et suos clam ex
agris deducere coeperunt.

\chap At Caesar, etsi nondum eorum consilia cognoverat, tamen et ex
eventu navium suarum et ex eo quod obsides dare intermiserant fore id
quod accidit suspicabatur. \sect Itaque ad omnes casus subsidia
comparabat.  Nam et frumentum ex agris cotidie in castra conferebat et,
quae gravissime adflictae erant naves, earum materia atque aere ad
reliquas reficiendas utebatur et quae ad eas res erant usui ex
continenti comportari iubebat. \sect Itaque, cum summo studio a militibus
administraretur, \textsc{xii} navibus amissis, reliquis ut navigari
$\langle$satis$\rangle$ commode posset effecit.

\chap Dum ea geruntur, legione ex consuetudine una frumentatum missa,
quae appellabatur \textsc{vii}., neque ulla ad id tempus belli suspicione
interposita, cum pars hominum in agris remaneret, pars etiam in castra
ventitaret, ii qui pro portis castrorum in statione erant Caesari
nuntiaverunt pulverem maiorem quam consuetudo ferret in ea parte videri
quam in partem legio iter fecisset. \sect Caesar---id quod
erat---suspicatus aliquid novi a barbaris initum consilii, cohortes quae
in statione erant secum in eam partem proficisci, ex reliquis duas in
stationem succedere, reliquas armari et confestim sese subsequi iussit.
\sect Cum paulo longius a castris processisset, suos ab hostibus premi
atque aegre sustinere et conferta legione ex omnibus partibus tela coici
animadvertit. \sect Nam quod omni ex reliquis partibus demesso frumento
pars una erat reliqua, suspicati hostes huc nostros esse venturos noctu
in silvis delituerant; \sect tum dispersos depositis armis in metendo
occupatos Subito adorti paucis interfectis reliquos incertis ordinibus
perturbaverant, simul equitatu atque essedis circumdederant.

\chap Genus hoc est ex essedis pugnae. Primo per omnes partes
perequitant et tela coiciunt atque ipso terrore equorum et strepitu
rotarum ordines plerumque perturbant, et cum se inter equitum turmas
insinuaverunt, ex essedis desiliunt et pedibus proeliantur. \sect Aurigae
interim paulatim ex proelio excedunt atque ita currus conlocant ut, si
illi a multitudine hostium premantur, expeditum ad quos receptum
habeant. \sect Ita mobilitatem equitum, stabilitatem peditum in proeliis
praestant, ac tantum usu cotidiano et exercitatione efficiunt uti in
declivi ac praecipiti loco incitatos equos sustinere et brevi moderari
ac flectere et per temonem percurrere et in iugo insistere et se inde in
currus citissime recipere consuerint.

\chap Quibus rebus perturbatis nostris [novitate pugnae] tempore
oportunissimo Caesar auxilium tulit: nam\-que eius adventu hostes
constiterunt, nostri se ex timore receperunt. \sect Quo facto, ad
lacessendum hostem et committendum proelium alienum esse tempus
arbitratus suo se loco continuit et brevi tempore intermisso in castra
legiones reduxit.  \sect Dum haec geruntur, nostris omnibus occupatis qui
erant in agris reliqui discesserunt. \sect Secutae sunt continuos
complures dies tempeststes, quae et nostros in castris continerent et
hostem a pugna prohiberent. \sect Interim barbari nuntios in omnes partes
dimiserunt paucitatemque nostrorum militum suis praedicaverunt et quanta
praedae faciendae atque in perpetuum sui liberandi facultas daretur, si
Romanos castris expulissent, demonstra; verunt. His rebus celeriter
magna multitudine peditatus equitatusque coacta ad castra venerunt.

\chap Caesar, etsi idem quod superioribus diebus acciderat fore videbat,
ut, si essent hostes pulsi, celeritate periculum effugerent, tamen
nactus equites circiter \textsc{xxx}, quos Commius Atrebas, de quo ante dictum
est, secum transportaverat, legiones in acie pro castris constituit.
\sect Commisso proelio diutius nostrorum militum impetum hostes ferre non
potuerunt ac terga verterunt.  \sect Quos tanto spatio secuti quantum
cursu et viribus efficere potuerunt, complures ex iis occiderunt, deinde
omnibus longe lateque aedificiis incensis se in castra receperunt.

\chap Eodem die legati ab hostibus missi ad Caesarem de pace venerunt.
\sect His Caesar numerum obsidum quem ante imperaverat duplicavit eosque
in continentem adduci iussit, quod propinqua die aequinoctii infirmis
navibus hiemi navigationem subiciendam non existimabat. \sect Ipse
idoneam tempestatem nactus paulo post mediam noctem naves solvit, \sect
quae omnes incolumes ad continentem pervenerunt; sed ex iis onerariae
duae eosdem portus quos reliquae capere non potuerunt et paulo infra
delatae sunt.

\chap Quibus ex navibus cum essent expositi milites circiter
\textsc{ccc} atque
in castra contenderent, Morini, quos Caesar in Britanniam proficiscens
pacatos reliquerat, spe praedae adducti primo non ita magno suorum
numero circumsteterunt ac, si sese interfici nollent, arma ponere
iusserunt. \sect Cum illi orbe facto sese defenderent, celeriter ad
clamorem hominum circiter milia \textsc{vi} convenerunt; Qua re nuntiata, Caesar
omnem ex castris equitatum suis auxilio misit. \sect Interim nostri
milites impetum hostium sustinuerunt atque amplius horis
\textsc{iiii} fortissime
pugnaverunt et paucis vulneribus acceptis complures ex iis occiderunt.
\sect Postea vero quam equitatus noster in conspectum venit, hostes
abiectis armis terga verterunt magnusque eorum numerus est occisus.

\chap Caesar postero die T.~Labienum legatum cum iis legionibus quas ex
Britannia reduxerat in Morinos qui rebellionem fecerant misit. \sect Qui
cum propter siccitates paludum quo se reciperent non haberent, quo
perfugio superiore anno erant usi, omnes fere in potestatem Labieni
venerunt. \sect At Q.~Titurius et L.~Cotta legati, qui in Menapiorum
fines legiones duxerant, omnibus eorum agris vastatis, frumentis
succisis, aedificiis incensis, quod Menapii se omnes in densissimas
silvas abdiderant, se ad Caesarem receperunt. \sect Caesar in Belgis
omnium legionum hiberna constituit. Eo duae omnino civitates ex
Britannia obsides miserunt, reliquae neglexerunt. \sect His rebus gestis
ex litteris Caesaris dierum \textsc{xx} supplicatio a senatu decreta est.

\endprosa
\end{document}


