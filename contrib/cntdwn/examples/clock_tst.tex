\documentclass{article}
\usepackage[designvi,usesf,tight]{web}
\usepackage{eforms}
\usepackage[longcount]{cntdwn}

\title{The \texttt{cntdwn} Package
    \texorpdfstring{\\[6pt]}{: }Clock Timers}
\author{D. P. Story}
\subject{Demo file illustrating clocks, local and foreign time zones}
\keywords{AeB, AcroTeX, clocks, countdown, cntdwn package}

\university{Acro\negthinspace\TeX.Net}
\email{dpstory@acrotex.net}
\version{1.0}
\copyrightyears{2010}
\revisionLabel{}

\setClockTimer{LocalClock}{}
\setClockTimer{CESTClock}{tzoffset=+0200} % this is summer time, CET is +0100

\parskip6pt\parindent0pt

\begin{document}

\maketitle

The clocks below show local time and date (the time/date your computer
shows), and the time and date of Central Europe during the summer (CEST), the
time when this demo file was originally created.

\begin{tabular}{rcc}
Local Time: &%
\cntdwnclocktime{LocalClock}{1in}{11bp}&%
\cntdwnclockdate{LocalClock}{1in}{11bp}\\[3bp]
CEST: &%
\cntdwnclocktime{CESTClock}{1in}{11bp}&%
\cntdwnclockdate{CESTClock}{1in}{11bp}%
\end{tabular}

In the preamble we have
\begin{verbatim}
\setClockTimer{LocalClock}{}
\setClockTimer{CESTClock}{tzoffset=+0200}
\end{verbatim}
and in the body of the document we have
\begin{verbatim}
\begin{tabular}{rcc}
Local Time: &\cntdwnclocktime{LocalClock}{1in}{11bp}&%
\cntdwnclockdate{LocalClock}{1in}{11bp}\\[3bp]
CEST: &\cntdwnclocktime{CESTClock}{1in}{11bp}&%
\cntdwnclockdate{CESTClock}{1in}{11bp}%
\end{tabular}
\end{verbatim}
These commands are available through the \texttt{longcount} option of the \texttt{cntdwn}
package.
\end{document}
