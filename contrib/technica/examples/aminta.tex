\documentclass{book}
\usepackage{poetry}
\usepackage{drama}
\usepackage[italian]{babel}[2005/05/21 v3.8g]
\usepackage[pagestyles,outermarks]{titlesec}[2005/01/22 v2.6]
\usepackage{example}

\TextHeight {6.5in}
\TextWidth  {4.3in}
\headsep =  1cm
\hfuzz 1pt

%%%%%%%%%%%%%%%%%%%%%%%%%%%%%%%%%%%%%%%%%%%%%%%%%%%%%%%%%%%%%%%%%%%%%%
%%%%%%%%%%%%%%%%%%%%%%%%%%%%%%%%%%%%%%%%%%%%%%%%%%%%%%%%%%%%%%%%%%%%%%
%%%%%%%%%%%%%%%%%%%%%%%%%%%%%%%%%%%%%%%%%%%%%%%%%%%%%%%%%%%%%%%%%%%%%%

\newpagestyle {maintext} {

  \sethead [] 
           [\textsl{\LetterSpace{TORQUATO TASSO}}]
           []
           {}
           {\textsl{\LetterSpace{AMINTA}}}
           {}

  \setfoot [\textit{\thepage}][][]
           {}{}{\textit{\thepage}}
}

%%%%%%%%%%%%%%%%%%%%%%%%%%%%%%%%%%%%%%%%%%%%%%%%%%%%%%%%%%%%%%%%%%%%%%
%%%%%%%%%%%%%%%%%%%%%%%%%%%%%%%%%%%%%%%%%%%%%%%%%%%%%%%%%%%%%%%%%%%%%%
%%%%%%%%%%%%%%%%%%%%%%%%%%%%%%%%%%%%%%%%%%%%%%%%%%%%%%%%%%%%%%%%%%%%%%

\Facies \personae {\textsc{#1}{#2}\\\textsc{#1}}
\Locus            {\centre}
\SpatiumSupra     {.5\leading plus .25\leading minus .25\leading 
                   \penalty -50}
\SpatiumInfra     {.125\leading plus .125\leading minus .0625\leading 
                   \penalty 10000}

\Forma \(          {\centeredfinal}
\Facies            {\itshape}
\SpatiumInfra      {2ex plus 1ex minus .5ex}
           


\Facies \numeri  {\RelSize{-1}\oldstylenums{#1#2}}
\Locus           {\leftmargin + 6em}
\Progressio      {5\\}


\Facies \tituli {\textsc{\MakeUppercase{#1}}} 
\SpatiumSupra   {2\leading plus .5\leading minus .25\leading 
                 \penalty -10000}
\SpatiumInfra   {1\leading plus .25\leading minus .125\leading}

\Novus \numerus \Natto
\Facies         {\ordinal {#1}\MakeUppercase{\theordinal}}

\Novus \titulus \Atto
\Facies         {\thispagestyle{empty}\Nscena{0}\textsc{ATTO \Natto*{=+1}}}
\SpatiumSupra   {2\leading plus .5\leading \penalty -10000}
\SpatiumInfra   {1\leading plus .25\leading}

\Novus \numerus \Nscena
\Facies         {\ordinal [f]{#1}\MakeUppercase{\theordinal}}

\Novus \titulus \Scena
\Facies         {\itshape SCENA \Nscena*{=+1}}
\SpatiumSupra   {1\leading plus .5\leading}
\SpatiumInfra   {.5\leading plus .25\leading}


%%%%%%%%%%%%%%%%%%%%%%%%%%%%%%%%%%%%%%%%%%%%%%%%%%%%%%%%%%%%%%%%%%%%%%
%%%%%%%%%%%%%%%%%%%%%%%%%%%%%%%%%%%%%%%%%%%%%%%%%%%%%%%%%%%%%%%%%%%%%%
%%%%%%%%%%%%%%%%%%%%%%%%%%%%%%%%%%%%%%%%%%%%%%%%%%%%%%%%%%%%%%%%%%%%%%

\begin{document}

\ExampleTitle [i]{Torquato Tasso}
                 {Aminta}
                 {Aminta\\[.5ex]Biblioteca Universale Rizzoli, 1976}

\Drama


\persona*[1]{Amore}
\persona*[2]{Dafne}
\persona*[3]{Silvia}
\persona*[4]{Aminta}
\persona*[5]{Tirsi}
\persona*[6]{Satiro}
\persona*[7]{Nerina}
\persona*[8]{Elpino}
\persona*[9]{Coro}
\persona*[10]{Ergasto}

\Versus

\cleardoublepage
\titulus{{\RelSize{3}AMINTA}\\[4ex]\RelSize{2}\itshape FAVOLA BOSCARECCIA}
\cleardoublepage
\titulus{INTERLOCUTORI}

{\Facies* \textus {\itshape }
 \Locus           {\leftmargin + 1.25in}
 \Facies  \personae {\textsc{#1} }
 \Locus             {}

\1, in abito pastorale; 
\2, compagna di Silvia; 
\3, amata da Aminta;  
\4, innamorato di Silvia; 
\5, compagno d'Aminta;  
\6, innamorato di Silvia; 
\7, messaggera;  
\10, nunzio; 
\8, pastore; 
\9 de' pastori. 

}

\Locus \textus   {+7em}

\cleardoublepage
\pagenumbering{arabic}
\pagestyle {maintext} 
\thispagestyle{empty}

\titulus{PROLOGO}
\numerus{1}

\persona{Amore \textit{in abito pastorale}}
	Chi crederia che sotto umane forme
	e sotto queste pastorali spoglie
	fosse nascosto un Dio? non mica un Dio
	selvaggio, o de la plebe de gli Dei,
	ma tra' grandi e celesti il pi\`u potente,
	che fa spesso cader di mano a Marte
	la sanguinosa spada, ed a Nettuno
	scotitor de la terra il gran tridente,
	ed i folgori eterni al sommo Giove.
	In questo aspetto, certo, e in questi panni
	non riconoscer\`a s\`{\i} di leggiero
	Venere madre me suo figlio Amore.
	Io da lei son constretto di fuggire
	e celarmi da lei, perch'ella vuole
	ch'io di me stesso e de le mie saette
	faccia a suo senno; e, qual femina, e quale
	vana ed ambiziosa, mi rispinge
	pur tra le corti e tra corone e scettri,
	e quivi vuol che impieghi ogni mia prova,
	e solo al volgo de' ministri miei,
	miei minori fratelli, ella consente
	l'albergar tra le selve ed oprar l'armi
	ne' rozzi petti. Io, che non son fanciullo,
	se ben ho volto fanciullesco ed atti,
	voglio dispor di me come a me piace;
	ch'a me fu, non a lei, concessa in sorte
	la face onnipotente, e l'arco d'oro.
	Per\`o spesso celandomi, e fuggendo
	l'imperio no, che in me non ha, ma i preghi,
	c'han forza porti da importuna madre,
	ricovero ne' boschi, e ne le case
	de le genti minute; ella mi segue,
	dar promettendo, a chi m'insegna a lei,
	o dolci baci, o cosa altra pi\`u cara:
	quasi io di dare in cambio non sia buono,
	a chi mi tace, o mi nasconde a lei,
	o dolci baci, o cosa altra pi\`u cara.
	Questo io so certo almen: che i baci miei
	saran sempre pi\`u cari a le fanciulle,
	se io, che son l'Amor, d'amor m'intendo;
	onde sovente ella mi cerca in vano,
	che rivelarmi altri non vuole, e tace.
	Ma per istarne anco pi\`u occulto, ond'ella
	ritrovar non mi possa ai contrasegni,
	deposto ho l'ali, la faretra e l'arco.
	Non per\`o disarmato io qui ne vengo,
	ch\'e questa, che par verga, \`e la mia face
	(cos\`{\i} l'ho trasformata), e tutta spira
	d'invisibili fiamme; e questo dardo,
	se bene egli non ha la punta d'oro,
	\`e di tempre divine, e imprime amore
	dovunque fiede. Io voglio oggi con questo
	far cupa e immedicabile ferita
	nel duro sen de la pi\`u cruda ninfa
	che mai seguisse il coro di Diana.
	N\'e la piaga di Silvia fia minore
	(ch\'e questo \`e 'l nome de l'alpestre ninfa)
	che fosse quella che pur feci io stesso
	nel molle sen d'Aminta, or son molt'anni,
	quando lei tenerella ei tenerello
	seguiva ne le caccie e ne i diporti.
	E, perch\'e il colpo mio pi\`u in lei s'interni,
	aspetter\`o che la piet\`a mollisca
	quel duro gelo che d'intorno al core
	l'ha ristretto il rigor de l'onestate
	e del virginal fasto; ed in quel punto
	ch'ei fia pi\`u molle, lancerogli il dardo.
	E, per far s\`{\i} bell'opra a mio grand'agio,
	io ne vo a mescolarmi infra la turba
	de' pastori festanti e coronati,
	che gi\`a qui s'\`e inviata, ove a diporto
	si sta ne' d\`{\i} solenni, esser fingendo
	uno di loro schiera: e in questo luogo,
	in questo luogo a punto io far\`o il colpo,
	che veder non potrallo occhio mortale.
	Queste selve oggi ragionar d'Amore
	s'udranno in nuova guisa; e ben parrassi
	che la mia deit\`a sia qui presente
	in se medesma, e non ne' suoi ministri.
	Spirer\`o nobil sensi a' rozzi petti,
	raddolcir\`o de le lor lingue il suono;
	perch\'e, ovunque i' mi sia, io sono Amore,
	ne' pastori non men che ne gli eroi,
	e la disagguaglianza de' soggetti
	come a me piace agguaglio; e questa \`e pure
	suprema gloria e gran miracol mio:
	render simili a le pi\`u dotte cetre
	le rustiche sampogne; e, se mia madre,
	che si sdegna vedermi errar fra' boschi,
	ci\`o non conosce, \`e cieca ella, e non io,
	cui cieco a torto il cieco volgo appella.

\newpage

\Atto

\Scena
\(\2, \3\)

\2 Vorrai dunque pur, Silvia,
	dai piaceri di Venere lontana
	menarne tu questa tua giovinezza?
	N\'e 'l dolce nome di madre udirai,
	n\'e intorno ti vedrai vezzosamente
	scherzar i figli pargoletti? Ah, cangia,
	cangia, prego, consiglio,
	pazzarella che sei.
\3 Altri segua i diletti de l'amore,
	se pur v'\`e ne l'amor alcun diletto:
	me questa vita giova, e 'l mio trastullo
	\`e la cura de l'arco e de gli strali;
	seguir le fere fugaci, e le forti
	atterrar combattendo; e, se non mancano
	saette a la faretra, o fere al bosco,
	non tem'io che a me manchino diporti.
	\2 Insipidi diporti veramente,
	ed insipida vita: e, s'a te piace,
	\`e sol perch\'e non hai provata l'altra.
	Cos\`{\i} la gente prima, che gi\`a visse
	nel mondo ancora semplice ed infante,
	stim\`o dolce bevanda e dolce cibo
	l'acqua e le ghiande, ed or l'acqua e le ghiande
	sono cibo e bevanda d'animali,
	poi che s'\`e posto in uso il grano e l'uva.
	Forse, se tu gustassi anco una volta
	la millesima parte de le gioie
	che gusta un cor amato riamando,
	diresti, ripentita, sospirando:
	``Perduto \`e tutto il tempo,
	che in amar non si spende''.
	O mia fuggita etate,
	quante vedove notti,
	quanti d\`{\i} solitari
	ho consumati indarno,
	che si poteano impiegar in quest'uso,
	il qual pi\`u replicato \`e pi\`u soave!
	Cangia, cangia consiglio,
	pazzarella che sei,
	ch\'e 'l pentirsi da sezzo nulla giova.

	\3 Quando io dir\`o, pentita, sospirando,
	queste parole che tu fingi ed orni
	come a te piace, torneranno i fiumi,
	a le lor fonti, e i lupi fuggiranno
	da gli agni, e 'l veltro le timide lepri,
	amer\`a l'orso il mare, e 'l delfin l'alpi.

	\2 Conosco la ritrosa fanciullezza:
	qual tu sei, tal io fui: cos\`{\i} portava
	la vita e 'l volto, e cos\`{\i} biondo il crine,
	e cos\`{\i} vermigliuzza avea la bocca,
	e cos\`{\i} mista col candor la rosa
	ne le guancie pienotte e delicate.
	Era il mio sommo gusto (or me n'avveggio,
	gusto di sciocca) sol tender le reti,
	ed invescar le panie, ed aguzzare
	il dardo ad una cote, e spiar l'orme
	e 'l covil de le fere: e, se talora
	vedea guatarmi da cupido amante,
	chinava gli occhi rustica e selvaggia,
	piena di sdegno e di vergogna, e m'era
	mal grata la mia grazia, e dispiacente
	quanto di me piaceva altrui: pur come
	fosse mia colpa e mia onta e mio scorno
	l'esser guardata, amata e desiata.
	Ma che non puote il tempo? e che non puote,
	servendo, meritando, supplicando,
	fare un fedele ed importuno amante?
	Fui vinta, io te 'l confesso, e furon l'armi
	del vincitore umilt\`a, sofferenza,
	pianti, sospiri, e dimandar mercede.
	Mostrommi l'ombra d'una breve notte
	allora quel che 'l lungo corso e 'l lume
	di mille giorni non m'avea mostrato;
	ripresi allor me stessa e la mia cieca
	simplicitate, e dissi sospirando:
	``Eccoti, Cinzia, il corno, eccoti l'arco,
	ch'io rinunzio i tuoi strali e la tua vita''.
	Cos\`{\i} spero veder ch'anco il tuo Aminta
	pur un giorno domestichi la tua
	rozza salvatichezza, ed ammollisca
	questo tuo cor di ferro e di macigno.
	Forse ch'ei non \`e bello? o ch'ei non t'ama?
	o ch'altri lui non ama? o ch'ei si cambia
	per l'amor d'altri? over per l'odio tuo?
	forse ch'in gentilezza egli ti cede?
	Se tu sei figlia di Cidippe, a cui
	fu padre il Dio di questo nobil fiume,
	ed egli \`e figlio di Silvano, a cui
	Pane fu padre, il gran Dio de' pastori.
	Non \`e men di te bella, se ti guardi
	dentro lo specchio mai d'alcuna fonte,
	la candida Amarilli; e pur ei sprezza
	le sue dolci lusinghe, e segue i tuoi
	dispettosi fastidi. Or fingi (e voglia
	pur Dio che questo fingere sia vano)
	ch'egli, teco sdegnato, al fin procuri
	ch'a lui piaccia colei cui tanto ei piace:
	qual animo fia il tuo? o con quali occhi
	il vedrai fatto altrui? fatto felice
	ne l'altrui braccia, e te schernir ridendo?

	\3 Faccia Aminta di s\'e e de' suoi amori
	quel ch'a lui piace: a me nulla ne cale;
	e, pur che non sia mio, sia di chi vuole;
	ma esser non pu\`o mio, s'io lui non voglio;
	n\'e, s'anco egli mio fosse, io sarei sua.

	\2 Onde nasce il tuo odio? \\

   \3 Dal suo amore.

	\2 Piacevol padre di figlio crudele.
	Ma quando mai dai mansueti agnelli
	nacquer le tigri? o dai bei cigni i corvi?
	O me inganni, o te stessa. \\

   \3 Odio il suo amore,
	ch'odia la mia onestate, ed amai lui,
	mentr'ei volse di me quel ch'io voleva.
	\2 Tu volevi il tuo peggio: egli a te brama
	quel ch'a s\'e brama. \\

   \3 Dafne, o taci, o parla
	d'altro, se vuoi risposta. \\

   \2 Or guata modi!
	guata che dispettosa giovinetta!
	Or rispondimi almen: s'altri t'amasse,
	gradiresti il suo amore in questa guisa?

	\3 In questa guisa gradirei ciascuno
	insidiator di mia virginitate,
	che tu dimandi amante, ed io nimico.

	\2 Stimi dunque nemico
	il monton de l'agnella?
	de la giovenca il toro?
	Stimi dunque nemico
	il tortore a la fida tortorella?
	Stimi dunque stagione
	di nimicizia e d'ira
	la dolce primavera,
	ch'or allegra e ridente
	riconsiglia ad amare
	il mondo e gli animali
	e gli uomini e le donne? e non t'accorgi
	come tutte le cose
	or sono innamorate
	d'un amor pien di gioia e di salute?
	Mira l\`a quel colombo
	con che dolce susurro lusingando
	bacia la sua compagna.
	Odi quell'usignuolo
	che va di ramo in ramo
	cantando: ``Io amo, io amo''; e, se no 'l sai,
	la biscia lascia il suo veleno e corre
	cupida al suo amatore;
	van le tigri in amore;
	ama il leon superbo; e tu sol, fiera
	pi\`u che tutte le fere,
	albergo gli dineghi nel tuo petto.
	Ma che dico leoni e tigri e serpi,
	che pur han sentimento? amano ancora
	gli alberi. Veder puoi con quanto affetto
	e con quanti iterati abbracciamenti
	la vite s'avviticchia al suo marito;
	l'abete ama l'abete, il pino il pino,
	l'orno per l'orno e per la salce il salce
	e l'un per l'altro faggio arde e sospira.
	Quella quercia, che pare
	s\`{\i} ruvida e selvaggia,
	sent'anch'ella il potere
	de l'amoroso foco; e, se tu avessi
	spirto e senso d'amore, intenderesti
	i suoi muti sospiri. Or tu da meno
	esser vuoi de le piante,
	per non esser amante?
	Cangia, cangia consiglio,
	pazzarella che sei.
\newpage
	\3 Or su, quando i sospiri
	udir\`o de le piante,
	io son contenta allor d'esser amante.

	\2 Tu prendi a gabbo i miei fidi consigli
	e burli mie ragioni? O in amore
	sorda non men che sciocca! Ma va pure,
	ch\'e verr\`a tempo che ti pentirai
	non averli seguiti. E gi\`a non dico
	allor che fuggirai le fonti, ov'ora
	spesso ti specchi e forse ti vagheggi,
	allor che fuggirai le fonti, solo
	per tema di vederti crespa e brutta;
	questo averratti ben; ma non t'annuncio
	gi\`a questo solo, ch\'e, bench'\`e gran male,
	\`e per\`o mal commune. Or non rammenti
	ci\`o che l'altr'ieri Elpino raccontava,
	il saggio Elpino a la bella Licori,
	Licori ch'in Elpin puote con gli occhi
	quel ch'ei potere in lei dovria col canto,
	se 'l dovere in amor si ritrovasse?
	E 'l raccontava udendo Batto e Tirsi
	gran maestri d'amore, e 'l raccontava
	ne l'antro de l'Aurora, ove su l'uscio
	\`e scritto: ``Lungi, ah lungi ite, profani''.
	Diceva egli, e diceva che glie 'l disse
	quel grande che cant\`o l'armi e gli amori,
	ch'a lui lasci\`o la fistola morendo,
	che l\`a gi\`u ne lo 'nferno \`e un nero speco,
	l\`a dove essala un fumo pien di puzza
	da le triste fornaci d'Acheronte;
	e che quivi punite eternamente
	in tormenti di tenebre e di pianto
	son le femine ingrate e sconoscenti.
	Quivi aspetta ch'albergo s'apparecchi
	a la tua feritate;
	e dritto \`e ben ch'il fumo
	tragga mai sempre il pianto da quegli occhi,
	onde trarlo giamai
	non pot\'e la pietate.
	Segui, segui tuo stile,
	ostinata che sei.

	\3 Ma che fe' allor Licori? e com' rispose
	a queste cose? \\

   \2 Tu de' fatti propri
	nulla ti curi, e vuoi saper gli altrui.
	Con gli occhi gli rispose.

	\3 Come risponder sol pot\'e con gli occhi?

	\2 Risposer questi con dolce sorriso,
	volti ad Elpino: ``Il core e noi siam tuoi;
	tu bramar pi\`u non d\'ei: costei non puote
	pi\`u darti''. E tanto solo basterebbe
	per intiera mercede al casto amante,
	se stimasse veraci come belli
	quegli occhi, e lor prestasse intera fede.

	\3 E perch\'e lor non crede? \\

   \2 Or tu non sai
	ci\`o che Tirsi ne scrisse, allor ch'ardendo
	forsennato egli err\`o per le foreste,
	s\`{\i} ch'insieme movea pietate e riso
	ne le vezzose ninfe e ne' pastori?
	N\'e gi\`a cose scrivea degne di riso,
	se ben cose facea degne di riso.
	Lo scrisse in mille piante, e con le piante
	crebbero i versi; e cos\`{\i} lessi in una:
	``Specchi del cor, fallaci infidi lumi,
	ben riconosco in voi gli inganni vostri:
	ma che pro', se schivarli Amor mi toglie?''

	\3 Io qui trapasso il tempo ragionando,
	n\'e mi sovviene ch'oggi \`e 'l d\`{\i} prescritto
	ch'andar si deve a la caccia ordinata
	ne l'Eliceto. Or, se ti pare, aspetta
	ch'io pria deponga nel solito fonte
	il sudore e la polve, ond'ier mi sparsi
	seguendo in caccia una damma veloce,
	ch'al fin giunsi ed ancisi. \\

   \2 Aspetterotti,
	e forse anch'io mi bagner\`o nel fonte.
	Ma sino a le mie case ir prima voglio,
	ch\'e l'ora non \`e tarda, come pare.
	Tu ne le tue m'aspetta ch'a te venga,
	e pensa in tanto pur quel che pi\`u importa
	de la caccia e del fonte; e, se non sai,
	credi di non saper, e credi a' savi.


\Scena

\(\4, \5\)

	\4 Ho visto al pianto mio
	risponder per pietate i sassi e l'onde,
	e sospirar le fronde
	ho visto al pianto mio;
	ma non ho visto mai,
	n\'e spero di vedere,
	compassion ne la crudele e bella,
	che non so s'io mi chiami o donna o fera:
	ma niega d'esser donna,
	poich\'e nega pietate
	a chi non la negaro
	le cose inanimate.

	\5 Pasce l'agna l'erbette, il lupo l'agne,
	ma il crudo Amor di lagrime si pasce,
	n\'e se ne mostra mai satollo. \\

   \4 Ahi, lasso,
	ch'Amor satollo \`e del mio pianto omai,
	e solo ha sete del mio sangue; e tosto
	voglio ch'egli e quest'empia il sangue mio
	bevan con gli occhi. \\

   \5 Ahi, Aminta, ahi, Aminta,
	che parli? o che vaneggi? Or ti conforta,
	ch'un'altra troverai, se ti disprezza
	questa crudele. \\

   \4 Ohim\`e, come poss'io
	altri trovar, se me trovar non posso?
	Se perduto ho me stesso, quale acquisto
	far\`o mai che mi piaccia? \\

   \5 O miserello,
	non disperar, ch'acquisterai costei.
	La lunga etate insegna a l'uom di porre
	freno ai leoni ed a le tigri ircane.

	\4 Ma il misero non puote a la sua morte
	indugio sostener di lungo tempo.

	\5 Sar\`a corto l'indugio: in breve spazio
	s'adira e in breve spazio anco si placa
	femina, cosa mobil per natura
	pi\`u che fraschetta al vento e pi\`u che cima
	di pieghevole spica. Ma, ti prego,
	fa ch'io sappia pi\`u a dentro de la tua
	dura condizione e de l'amore;
	ch\'e, se ben confessato m'hai pi\`u volte
	d'amare, mi tacesti per\`o dove
	fosse posto l'amore. Ed \`e ben degna
	la fedele amicizia ed il commune
	studio de le Muse ch'a me scuopra
	ci\`o ch'agli altri si cela. \\

   \4 Io son contento,
	Tirsi, a te dir ci\`o che le selve e i monti
	e i fiumi sanno, e gli uomini non sanno.
	Ch'io sono omai s\`{\i} prossimo a la morte,
	ch'\`e ben ragion ch'io lasci chi ridica
	la cagion del morire, e che l'incida
	ne la scorza d'un faggio, presso il luogo
	dove sar\`a sepolto il corpo essangue;
	s\`{\i} che talor passandovi quell'empia
	si goda di calcar l'ossa infelici
	co 'l pi\`e superbo, e tra s\'e dica: ``\`E questo
	pur mio trionfo''; e goda di vedere
	che nota sia la sua vittoria a tutti
	li pastori paesani e pellegrini
	che quivi il caso guidi; e forse (ahi, spero
	troppo alte cose) un giorno esser potrebbe
	ch'ella, commossa da tarda pietate,
	piangesse morto chi gi\`a vivo uccise,
	dicendo: ``Oh pur qui fosse, e fosse mio!''
	Or odi. \\

   \5 Segui pur, ch'io ben t'ascolto,
	e forse a miglior fin che tu non pensi.
	\4 Essendo io fanciulletto, s\`{\i} che a pena
	giunger potea con la man pargoletta
	a c\^orre i frutti dai piegati rami
	degli arboscelli, intrinseco divenni
	de la pi\`u vaga e cara verginella
	che mai spiegasse al vento chioma d'oro.
	La figliuola conosci di Cidippe
	e di Montan, ricchissimo d'armenti,
	Silvia, onor de le selve, ardor de l'alme?
	Di questa parlo, ahi lasso; vissi a questa
	cos\`{\i} unito alcun tempo, che fra due
	tortorelle pi\`u fida compagnia
	non sar\`a mai, n\'e fue.
	Congiunti eran gli alberghi,
	ma pi\`u congiunti i cori;
	conforme era l'etate,
	ma 'l pensier pi\`u conforme;
	seco tendeva insidie con le reti
	ai pesci ed agli augelli, e seguitava
	i cervi seco e le veloci damme:
	e 'l diletto e la preda era commune.
	Ma, mentre io fea rapina d'animali,
	fui non so come a me stesso rapito.
	A poco a poco nacque nel mio petto,
	non so da qual radice,
	com'erba suol che per se stessa germini,
	un incognito affetto,
	che mi fea desiare
	d'esser sempre presente
	a la mia bella Silvia;
	e bevea da' suoi lumi
	un'estranea dolcezza,
	che lasciava nel fine
	un non so che d'amaro;
	sospirava sovente, e non sapeva
	la cagion de' sospiri.
	Cos\`{\i} fui prima amante ch'intendessi
	che cosa fosse Amore.
	Ben me n'accorsi al fin: ed in qual modo,
	ora m'ascolta, e nota. \\

   \5 \`{E} da notare.

	\4 A l'ombra d'un bel faggio Silvia e Filli
	sedean un giorno, ed io con loro insieme,
	quando un'ape ingegnosa, che, cogliendo
	sen' giva il mel per que' prati fioriti,
	a le guancie di Fillide volando,
	a le guancie vermiglie come rosa,
	le morse e le rimorse avidamente:
	ch'a la similitudine ingannata
	forse un fior le credette. Allora Filli
	cominci\`o lamentarsi, impaziente
	de l'acuta puntura:
	ma la mia bella Silvia disse: ``Taci,
	taci, non ti lagnar, Filli, perch'io
	con parole d'incanti leverotti
	il dolor de la picciola ferita.
	A me insegn\`o gi\`a questo secreto
	la saggia Aresia, e n'ebbe per mercede
	quel mio corno d'avolio ornato d'oro''.
	Cos\`{\i} dicendo, avvicin\`o le labra
	de la sua bella e dolcissima bocca
	a la guancia rimorsa, e con soave
	susurro mormor\`o non so che versi.
	Oh mirabili effetti! Sent\`{\i} tosto
	cessar la doglia, o fosse la virtute
	di que' magici detti, o, com'io credo,
	la virt\`u de la bocca,
	che sana ci\`o che tocca.
	Io, che sino a quel punto altro non volsi
	che 'l soave splendor degli occhi belli,
	e le dolci parole, assai pi\`u dolci
	che 'l mormorar d'un lento fiumicello
	che rompa il corso fra minuti sassi,
	o che 'l garrir de l'aura infra le frondi,
	allor sentii nel cor novo desire
	d'appressare a la sua questa mia bocca;
	e fatto non so come astuto e scaltro
	pi\`u de l'usato (guarda quanto Amore
	aguzza l'intelletto!) mi sovvenne
	d'un inganno gentile, co 'l qual io
	recar potessi a fine il mio talento:
	ch\'e, fingendo ch'un'ape avesse morso
	il mio labro di sotto, incominciai
	a lamentarmi di cotal maniera,
	che quella medicina, che la lingua
	non richiedeva, il volto richiedeva.
	La semplicetta Silvia,
	pietosa del mio male,
	s'offr\`{\i} di dar aita
	a la finta ferita, ahi lasso, e fece
	pi\`u cupa e pi\`u mortale
	la mia piaga verace,
	quando le labra sue
	giunse a le labra mie.
	N\'e l'api d'alcun fiore
	coglion s\`{\i} dolce il mel ch'allora io colsi
	da quelle fresche rose,
	se ben gli ardenti baci,
	che spingeva il desire a inumidirsi,
	raffren\`o la temenza
	e la vergogna, o felli
	pi\`u lenti e meno audaci.
	Ma mentre al cor scendeva
	quella dolcezza mista
	d'un secreto veleno,
	tal diletto n'avea
	che, fingendo ch'ancor non mi passasse
	il dolor di quel morso,
	fei s\`{\i} ch'ella pi\`u volte
	vi replic\`o l'incanto.
	Da indi in qua and\`o in guisa crescendo
	il desire e l'affanno impaziente
	che, non potendo pi\`u capir nel petto,
	fu forza che scoppiasse; ed una volta
	che in cerchio sedevam ninfe e pastori,
	e facevamo alcuni nostri giuochi,
	ch\'e ciascun ne l'orecchio del vicino
	mormorando diceva un suo secreto,
	``Silvia,''le dissi ``io per te ardo, e certo
	morr\`o, se non m'aiti.'' A quel parlare
	chin\`o ella il bel volto, e fuor le venne
	un improviso, insolito rossore
	che diede segno di vergogna e d'ira;
	n\'e ebbi altra risposta che un silenzio,
	un silenzio turbato e pien di dure
	minaccie. Indi si tolse, e pi\`u non volle
	n\'e vedermi n\'e udirmi. E gi\`a tre volte
	ha il nudo mietitor tronche le spighe,
	ed altretante il verno ha scossi i boschi
	de le lor verdi chiome; ed ogni cosa
	tentata ho per placarla, fuor che morte.
	Mi resta sol che per placarla io mora;
	e morr\`o volontier, pur ch'io sia certo
	ch'ella o se ne compiaccia, o se ne doglia:
	n\'e so di tai due cose qual pi\`u brami.
	Ben fora la piet\`a premio maggiore
	a la mia fede, e maggior ricompensa
	a la mia morte; ma bramar non deggio
	cosa che turbi il bel lume sereno
	agli occhi cari, e affanni quel bel petto.

	\5 \`E possibil per\`o che, s'ella un giorno
	udisse tai parole, non t'amasse?

	\4 Non so, n\'e 'l credo; ma fugge i miei detti
	come l'aspe l'incanto. \\

   \5 Or ti confida,
	ch'a me d\`a il cuor di far ch'ella t'ascolti.

	\4 O nulla impetrerai, o, se tu impetri
	ch'io parli, io nulla impetrer\`o parlando.

	\5 Perch\'e disperi s\`{\i}? \\

   \4 Giusta cagione
	ho del mio disperar, che il saggio Mopso
	mi predisse la mia cruda ventura,
	Mopso ch'intende il parlar degli augelli
	e la virt\`u de l'erbe e de le fonti.

	\5 Di qual Mopso tu dici? di quel Mopso
	c'ha ne la lingua melate parole,
	e ne le labra un amichevol ghigno,
	e la fraude nel seno, ed il rasoio
	tien sotto il manto? Or su, sta di bon core,
	ch\'e i sciaurati pronostichi infelici,
	ch'ei vende a' mal accorti con quel grave
	suo supercilio, non han mai effetto:
	e per prova so io ci\`o che ti dico;
	anzi da questo sol ch'ei t'ha predetto
	mi giova di sperar felice fine
	a l'amor tuo. \\

   \4 Se sai cosa per prova,
	che conforti mia speme, non tacerla.

	\5 Dirolla volontieri. Allor che prima
	mia sorte mi condusse in queste selve,
	costui conobbi, e lo stimava io tale
	qual tu lo stimi; in tanto un d\`{\i} mi venne
	e bisogno e talento d'irne dove
	siede la gran cittade in ripa al fiume,
	ed a costui ne feci motto; ed egli
	cos\`{\i} mi disse: ``Andrai ne la gran terra,
	ove gli astuti e scaltri cittadini
	e i cortigian malvagi molte volte
	prendonsi a gabbo, e fanno brutti scherni
	di noi rustici incauti; per\`o, figlio,
	va su l'avviso, e non t'appressar troppo
	ove sian drappi colorati e d'oro,
	e pennacchi e divise e foggie nove;
	ma sopra tutto guarda che mal fato
	o giovenil vaghezza non ti meni
	al magazzino de le ciancie: ah fuggi,
	fuggi quell'incantato alloggiamento''.
	``Che luogo \`e questo?'' io chiesi; ed ei soggiunse:
	``Quivi abitan le maghe, che incantando
	fan traveder e traudir ciascuno.
	Ci\`o che diamante sembra ed oro fino,
	\`e vetro e rame; e quelle arche d'argento,
	che stimeresti piene di tesoro,
	sporte son piene di vesciche bugge.
	Quivi le mura son fatte con arte,
	che parlano e rispondono ai parlanti;
	n\'e gi\`a rispondon la parola mozza,
	com'Eco suole ne le nostre selve,
	ma la replican tutta intiera intiera:
	con giunta anco di quel ch'altri non disse.
	I trespidi, le tavole e le panche,
	le scranne, le lettiere, le cortine,
	e gli arnesi di camera e di sala
	han tutti lingua e voce: e gridan sempre.
	Quivi le ciancie in forma di bambine
	vanno trescando, e se un muto v'entrasse,
	un muto ciancerebbe a suo dispetto.
	Ma questo \`e 'l minor mal che ti potesse
	incontrar: tu potresti indi restarne
	converso in selce, in fera, in acqua, o in foco:
	acqua di pianto, e foco di sospiri''.
	Cos\`{\i} diss'egli; ed io n'andai con questo
	fallace antiveder ne la cittade;
	e, come volse il Ciel benigno, a caso
	passai per l\`a dov'\`e 'l felice albergo.
	Quindi uscian fuor voci canore e dolci
	e di cigni e di ninfe e di sirene,
	di sirene celesti; e n'uscian suoni
	soavi e chiari; e tanto altro diletto,
	ch'attonito godendo ed ammirando,
	mi fermai buona pezza. Era su l'uscio,
	quasi per guardia de le cose belle,
	uom d'aspetto magnanimo e robusto,
	di cui, per quanto intesi, in dubbio stassi
	s'egli sia miglior duce o cavaliero;
	che, con fronte benigna insieme e grave,
	con regal cortesia invit\`o dentro,
	ei grande e 'n pregio, me negletto e basso.
	Oh che sentii? che vidi allora? I' vidi
	celesti dee, ninfe leggiadre e belle,
	novi Lini ed Orfei; ed oltre ancora,
	senza vel, senza nube, e quale e quanta
	a gl'immortali appar, vergine Aurora
	sparger d'argento e d'or rugiade e raggi;
	e fecondando illuminar d'intorno
	vidi Febo, e le Muse, e fra le Muse
	Elpin seder accolto; ed in quel punto
	sentii me far di me stesso maggiore,
	pien di nova virt\`u, pieno di nova
	deitade, e cantai guerre ed eroi,
	sdegnando pastoral ruvido carme.
	E se ben poi (come altrui piacque) feci
	ritorno a queste selve, io pur ritenni
	parte di quello spirto; n\'e gi\`a suona
	la mia sampogna umil come soleva,
	ma di voce pi\`u altera e pi\`u sonora
	emula de le trombe, empie le selve.
	Udimmi Mopso poscia, e con maligno
	guardo mirando, affascinommi; ond'io
	roco divenni, e poi gran tempo tacqui:
	quando i pastor credean ch'io fossi stato
	visto dal lupo, e 'l lupo era costui.
	Questo t'ho detto, acci\`o che sappi quanto
	il parlar di costui di fede \`e degno;
	e d\'ei bene sperar, sol perch\'e ei vuole
	che nulla speri. \\

   \4 Piacemi d'udire
	quanto mi narri. A te dunque rimetto
	la cura di mia vita. \\

   \5 Io n'avr\`o cura.
	Tu fra mezz'ora qui trovar ti lassa.

	\9 O bella et\`a de l'oro,
	non gi\`a perch\'e di latte
	sen' corse il fiume e still\`o mele il bosco;
	non perch\'e i frutti loro
	dier da l'aratro intatte
	le terre, e gli angui errar senz'ira o tosco;
	non perch\'e nuvol fosco
	non spieg\`o allor suo velo,
	ma in primavera eterna,
	ch'ora s'accende e verna,
	rise di luce e di sereno il cielo;
	n\'e port\`o peregrino
	o guerra o merce agli altrui lidi il pino;
	ma sol perch\'e quel vano
	nome senza soggetto,
	quell'idolo d'errori, idol d'inganno,
	quel che dal volgo insano
	onor poscia fu detto,
	che di nostra natura 'l feo tiranno,
	non mischiava il suo affanno
	fra le liete dolcezze
	de l'amoroso gregge;
	n\'e fu sua dura legge
	nota a quell'alme in libertate avvezze,
	ma legge aurea e felice
	che natura scolp\`{\i}: ``S'ei piace, ei lice''.
	Allor tra fiori e linfe
	traen dolci carole
	gli Amoretti senz'archi e senza faci;
	sedean pastori e ninfe
	meschiando a le parole
	vezzi e susurri, ed ai susurri i baci
	strettamente tenaci;
	la verginella ignude
	scopria sue fresche rose,
	ch'or tien nel velo ascose,
	e le poma del seno acerbe e crude;
	e spesso in fonte o in lago
	scherzar si vide con l'amata il vago.
	Tu prima, Onor, velasti
	la fonte dei diletti,
	negando l'onde a l'amorosa sete;
	tu a' begli occhi insegnasti
	di starne in s\'e ristretti,
	e tener lor bellezze altrui secrete;
	tu raccogliesti in rete
	le chiome a l'aura sparte;
	tu i dolci atti lascivi
	festi ritrosi e schivi;
	ai detti il fren ponesti, ai passi l'arte;
	opra \`e tua sola, o Onore,
	che furto sia quel che fu don d'Amore.
	E son tuoi fatti egregi
	le pene e i pianti nostri.
	Ma tu, d'Amore e di Natura donno,
	tu domator de' Regi,
	che fai tra questi chiostri,
	che la grandezza tua capir non ponno?
	Vattene, e turba il sonno
	agl'illustri e potenti:
	noi qui, negletta e bassa
	turba, senza te lassa
	viver ne l'uso de l'antiche genti.
	Amiam, ch\'e non ha tregua
	con gli anni umana vita, e si dilegua.
	Amiam, ch\'e 'l Sol si muore e poi rinasce:
	a noi sua breve luce
	s'asconde, e 'l sonno eterna notte adduce.



\Atto

\Scena

\(\6 solo\)

	\6 Picciola \`e l'ape, e fa col picciol morso
	pur gravi e pur moleste le ferite;
	ma qual cosa \`e pi\`u picciola d'Amore,
	se in ogni breve spazio entra, e s'asconde
	in ogni breve spazio? or sotto a l'ombra
	de le palpebre, or tra' minuti rivi
	d'un biondo crine, or dentro le pozzette
	che forma un dolce riso in bella guancia;
	e pur fa tanto grandi e s\`{\i} mortali
	e cos\`{\i} immedicabili le piaghe.
	Ohim\`e, che tutte piaga e tutte sangue
	son le viscere mie; e mille spiedi
	ha ne gli occhi di Silvia il crudo Amore.
	Crudel Amor, Silvia crudele ed empia
	pi\`u che le selve! Oh come a te confassi
	tal nome, e quanto vide chi te 'l pose!
	Celan le selve angui, leoni ed orsi,
	dentro il lor verde: e tu dentro al bel petto
	nascondi odio, disdegno ed impietate,
	fere peggior ch'angui, leoni ed orsi
	ch\'e si placano quei, questi placarsi
	non possono per prego n\'e per dono.
	Ohim\`e, quando ti porto i fior novelli,
	tu li ricusi, ritrosetta, forse
	perch\'e fior via pi\`u belli hai nel bel volto.
	Ohim\`e, quando io ti porgo i vaghi pomi,
	tu li rifiuti, disdegnosa, forse
	perch\'e pomi pi\`u vaghi hai nel bel seno.
	Lasso, quand'io t'offrisco il dolce mele,
	tu lo disprezzi, dispettosa, forse
	perch\'e mel via pi\`u dolce hai ne le labra.
	Ma, se mia povert\`a non pu\`o donarti
	cosa ch'in te non sia pi\`u bella e dolce,
	me medesmo ti dono. Or perch\'e iniqua
	scherni e abborri il dono? non son io
	da disprezzar, se ben me stesso vidi
	nel liquido del mar, quando l'altr'ieri
	taceano i venti ed ei giacea senz'onda.
	Questa mia faccia di color sanguigno,
	queste mie spalle larghe, e queste braccia
	torose e nerborute, e questo petto
	setoso, e queste mie velate coscie
	son di virilit\`a, di robustezza
	indicio; e, se no 'l credi, fanne prova.
	Che vuoi tu far di questi tenerelli,
	che di molle lanugine fiorite
	hanno a pena le guancie? e che con arte
	dispongono i capelli in ordinanza?
	Femine nel sembiante e ne le forze
	sono costoro. Or di' ch'alcun ti segua
	per le selve e pei monti, e 'ncontra gli orsi
	ed incontra i cinghiai per te combatta.
	Non sono io brutto, no, n\'e tu mi sprezzi
	perch\'e s\`{\i} fatto io sia, ma solamente
	perch\'e povero sono. Ahi, ch\'e le ville
	seguon l'essempio de le gran cittadi!
	e veramente il secol d'oro \`e questo,
	poich\'e sol vince l'oro e regna l'oro.
	O chiunque tu fosti, che insegnasti
	primo a vender l'amor, sia maledetto
	il tuo cener sepolto e l'ossa fredde,
	e non si trovi mai pastore o ninfa
	che lor dica passando: ``Abbiate pace'';
	ma le bagni la pioggia e mova il vento,
	e con pi\`e immondo la greggia il calpesti
	e 'l peregrin. Tu prima svergognasti
	la nobilt\`a d'amor; tu le sue liete
	dolcezze inamaristi. Amor venale,
	amor servo de l'oro \`e il maggior mostro
	ed il pi\`u abominabile e il pi\`u sozzo,
	che produca la terra o 'l mar fra l'onde.
	Ma perch\'e in van mi lagno? Usa ciascuno
	quell'armi che gli ha date la natura
	per sua salute: il cervo adopra il corso,
	il leone gli artigli, ed il bavoso
	cinghiale il dente; e son potenza ed armi
	de la donna bellezza e leggiadria;
	io, perch\'e non per mia salute adopro
	la violenza, se mi fe' natura
	atto a far violenza ed a rapire?
	Sforzer\`o, rapir\`o quel che costei
	mi niega, ingrata, in merto de l'amore;
	che, per quanto un caprar test\'e mi ha detto,
	ch'osservato ha suo stile, ella ha per uso
	d'andar sovente a rinfrescarsi a un fonte;
	e mostrato m'ha il loco. Ivi io disegno
	tra i cespugli appiattarmi e tra gli arbusti,
	ed aspettar fin che vi venga; e, come
	veggia l'occasion, correrle addosso.
	Qual contrasto col corso o con le braccia
	potr\`a fare una tenera fanciulla
	contra me s\`{\i} veloce e s\`{\i} possente?
	Pianga e sospiri pure, usi ogni sforzo
	di piet\`a, di bellezza: che, s'io posso
	questa mano ravvoglierle nel crine,
	indi non partir\`a, ch'io pria non tinga
	l'armi mie per vendetta nel suo sangue.


\Scena

\(\2, \5\)

	\2 Tirsi, com'io t'ho detto, io m'era accorta
	ch'Aminta amava Silvia; e Dio sa quanti
	buoni officii n'ho fatti, e son per farli
	tanto pi\`u volontier, quant'or vi aggiungi
	le tue preghiere; ma torrei pi\`u tosto
	a domar un giuvenco, un orso, un tigre,
	che a domar una semplice fanciulla:
	fanciulla tanto sciocca quanto bella,
	che non s'avveggia ancor come sian calde
	l'armi di sua bellezza e come acute,
	ma ridendo e piangendo uccida altrui,
	e l'uccida e non sappia di ferire.

	\5 Ma quale \`e cos\`{\i} semplice fanciulla
	che, uscita da le fascie, non apprenda
	l'arte del parer bella e del piacere,
	de l'uccider piacendo, e del sapere
	qual arme fera, e qual dia morte, e quale
	sani e ritorni in vita? \\

   \2 Chi \`e 'l mastro
	di cotant'arte? \\

   \5 Tu fingi, e mi tenti:
	quel che insegna agli augelli il canto e 'l volo,
	a' pesci il nuoto ed a' montoni il cozzo,
	al toro usar il corno, ed al pavone
	spiegar la pompa de l'occhiute piume.

	\2 Come ha nome 'l gran mastro? \\

   \5 Dafne ha nome.

	\2 Lingua bugiarda! \\

   \5 E perch\'e? tu non sei
	atta a tener mille fanciulle a scola?
	Bench\'e, per dir il ver, non han bisogno
	di maestro: maestra \`e la natura,
	ma la madre e la balia anco v'han parte.

	\2 In somma, tu sei goffo insieme e tristo.
	Ora, per dirti il ver, non mi risolvo
	se Silvia \`e semplicetta come pare
	a le parole, a gli atti. Ier vidi un segno
	che me ne mette in dubbio. Io la trovai
	l\`a presso la cittade in quei gran prati
	ove fra stagni giace un'isoletta,
	sovra essa un lago limpido e tranquillo,
	tutta pendente in atto che parea
	vagheggiar se medesma, e 'nsieme insieme
	chieder consiglio a l'acque in qual maniera
	dispor dovesse in su la fronte i crini,
	e sovra i crini il velo, e sovra 'l velo
	i fior che tenea in grembo; e spesso spesso
	or prendeva un lingustro, or una rosa,
	e l'accostava al bel candido collo,
	a le guancie vermiglie, e de' colori
	fea paragone; e poi, s\`{\i} come lieta
	de la vittoria, lampeggiava un riso
	che parea che dicesse: ``Io pur vi vinco,
	n\'e porto voi per ornamento mio,
	ma porto voi sol per vergogna vostra,
	perch\'e si veggia quanto mi cedete''.
	Ma, mentre ella s'ornava e vagheggiava,
	rivolse gli occhi a caso, e si fu accorta
	ch'io di lei m'era accorta, e vergognando
	rizzossi tosto, e fior lasci\`o cadere.
	In tanto io pi\`u ridea del suo rossore,
	ella pi\`u s'arrossia del riso mio.
	Ma, perch\'e accolta una parte de' crini
	e l'altra aveva sparsa, una o due volte
	con gli occhi al fonte consiglier ricorse,
	e si mir\`o quasi di furto, pure
	temendo ch'io nel suo guatar guatassi;
	ed incolta si vide, e si compiacque
	perch\'e bella si vide ancor che incolta.
	Io me n'avvidi, e tacqui. \\

   \5 Tu mi narri
	quel ch'io credeva a punto. Or non m'apposi?

	\2 Ben t'apponesti; ma pur odo dire
	che non erano pria le pastorelle,
	n\'e le ninfe s\`{\i} accorte; n\'e io tale
	fui in mia fanciullezza. Il mondo invecchia,
	e invecchiando intristisce. \\

   \5 Forse allora
	non usavan s\`{\i} spesso i cittadini
	ne le selve e ne i campi, n\'e s\`{\i} spesso
	le nostre forosette aveano in uso
	d'andare a la cittade. Or son mischiate
	schiatte e costumi. Ma lasciam da parte
	questi discorsi; or non farai ch'un giorno
	Silvia contenta sia che le ragioni
	Aminta, o solo, o almeno in tua presenza?

	\2 Non so. Silvia \`e ritrosa fuor di modo.

	\5 E costui rispettoso \`e fuor di modo.

	\2 \`E spacciato un amante rispettoso:
	consiglial pur che faccia altro mestiero,
	poich'egli \`e tal. Chi imparar vuol d'amare,
	disimpari il rispetto: osi, domandi,
	solleciti, importuni, al fine involi;
	e se questo non basta, anco rapisca.
	Or non sai tu com'\`e fatta la donna?
	Fugge, e fuggendo vuol che altri la giunga;
	niega, e negando vuol ch'altri si toglia;
	pugna, e pugnando vuol ch'altri la vinca.
	Ve', Tirsi, io parlo teco in confidenza:
	non ridir ch'io ci\`o dica. E sovra tutto
	non porlo in rime. Tu sai s'io saprei
	renderti poi per versi altro che versi.

	\5 Non hai cagion di sospettar ch'io dica
	cosa giamai che sia contra tuo grado.
	Ma ti prego, o mia Dafne, per la dolce
	memoria di tua fresca giovanezza,
	che tu m'aiti ad aitar Aminta
	miserel, che si muore. \\

   \2 Oh che gentile
	scongiuro ha ritrovato questo sciocco
	di rammentarmi la mia giovanezza,
	il ben passato e la presente noia!
	Ma che vuoi tu ch'io faccia? \\

   \5 A te non manca
	n\'e saper, n\'e consiglio. Basta sol che
	ti disponga a voler. \\

   \2 Or su, dirotti:
	debbiamo in breve andare Silvia ed io
	al fonte che s'appella di Diana,
	l\`a dove a le dolci acque fa dolce ombra
	quel platano ch'invita al fresco seggio
	le ninfe cacciatrici. Ivi so certo
	che tuffer\`a le belle membra ignude.

	\5 Ma che per\`o? \\

   \2 Ma che per\`o? Da poco
	intenditor! s'hai senno, tanto basti.

	\5 Intendo; ma non so s'egli avr\`a tanto
	d'ardir. \\

   \2 S'ei non l'avr\`a, stiasi, ed aspetti
	ch'altri lui cerchi. \\

   \5 Egli \`e ben tal che 'l merta.

	\2 Ma non vogliamo noi parlar alquanto
	di te medesmo? Or su, Tirsi, non vuoi
	tu inamorarti? sei giovane ancora,
	n\'e passi di quattr'anni il quinto lustro,
	se ben sovviemmi quando eri fanciullo;
	vuoi viver neghittoso e senza gioia?
	ch\'e sol amando uom sa che sia diletto.

	\5 I diletti di Venere non lascia
	l'uom che schiva l'amor, ma coglie e gusta
	le dolcezze d'amor senza l'amaro.

	\2 Insipido \`e quel dolce che condito
	non \`e di qualche amaro, e tosto sazia.

	\5 \`E meglio saziarsi, ch'esser sempre
	famelico nel cibo e dopo 'l cibo.

	\2 Ma non, se 'l cibo si possede e piace,
	e gustato a gustar sempre n'invoglia.

	\5 Ma chi possede s\`{\i} quel che gli piace
	che l'abbia sempre presso a la sua fame?

	\2 Ma chi ritrova il ben, s'egli no 'l cerca?

	\5 Periglioso \`e cercar quel che trovato
	trastulla s\`{\i}, ma pi\`u tormenta assai
	non ritrovato. Allor vedrassi amante
	Tirsi mai pi\`u, ch'Amor nel seggio suo
	non avr\`a pi\`u n\'e pianti n\'e sospiri.
	A bastanza ho gi\`a pianto e sospirato.
	Faccia altri la sua parte. \\

   \2 Ma non hai
	gi\`a goduto a bastanza. \\

   \5 N\'e desio
	goder, se cos\`{\i} caro egli si compra.

	\2 Sar\`a forza l'amar, se non fia voglia.

	\5 Ma non si pu\`o sforzar chi sta lontano.
	\2 Ma chi lung'\`e d'Amor? \\

   \5 Chi teme e fugge.

	\2 E che giova fuggir da lui, c'ha l'ali?

	\5 Amor nascente ha corte l'ali: a pena
	pu\`o su tenerle, e non le spiega a volo.

	\2 Pur non s'accorge l'uom quand'egli nasce;
	e, quando uom se n'accorge, \`e grande, e vola.

	\5 Non, s'altra volta nascer non l'ha visto.

	\2 Vedrem, Tirsi, s'avrai la fuga e gli occhi
	come tu dici. Io ti protesto, poi
	che fai del corridore e del cerviero,
	che, quando ti vedr\`o chieder aita,
	non moverei, per aiutarti, un passo,
	un dito, un detto, una palpebra sola.

	\5 Crudel, daratti il cor vedermi morto?
	Se vuoi pur ch'ami, ama tu me: facciamo
	l'amor d'accordo. \\

   \2 Tu mi scherni, e forse
	non merti amante cos\`{\i} fatta: ahi quanti
	n'inganna il viso colorito e liscio!

	\5 Non burlo io, no; ma tu con tal protesto
	non accetti il mio amor, pur come \`e l'uso
	di tutte quante; ma, se non mi vuoi,
	viver\`o senza amor. \\

   \2 Contento vivi
	pi\`u che mai fossi, o Tirsi, in ozio vivi:
	ch\'e ne l'ozio l'amor sempre germoglia.

	\5 O Dafne, a me quest'ozii ha fatto Dio:
	colui che Dio qui pu\`o stimarsi; a cui
	si pascon gli ampi armenti e l'ampie greggie
	da l'uno a l'altro mare, e per li lieti
	colti di fecondissime campagne,
	e per gli alpestri dossi d'Apennino.
	Egli mi disse, allor che suo mi fece:
	``Tirsi, altri scacci i lupi e i ladri, e guardi
	i miei murati ovili; altri comparta
	le pene e i premii a' miei ministri; ed altri
	pasca e curi le greggi; altri conservi
	le lane e 'l latte, ed altri le dispensi:
	tu canta, or che se' 'n ozio''. Ond'\`e ben giusto
	che non gli scherzi di terreno amore,
	ma canti gli avi del mio vivo e vero
	non so s'io lui mi chiami Apollo o Giove,
	ch\'e ne l'opre e nel volto ambi somiglia,
	gli avi pi\`u degni di Saturno o Celo:
	agreste Musa a regal merto; e pure,
	chiara o roca che suoni, ei non la sprezza.
	Non canto lui, per\`o che lui non posso
	degnamente onorar, se non tacendo
	e riverendo; ma non fian giamai
	gli altari suoi senza i miei fiori, e senza
	soave fumo d'odorati incensi:
	ed allor questa semplice e devota
	religion mi si torr\`a dal core,
	che d'aria pasceransi in aria i cervi,
	e che, mutando i fiumi e letto e corso,
	il Perso bea la Sona, il Gallo il Tigre.

	\2 Oh, tu vai alto; or su, discendi un poco
	al proposito nostro. \\

   \5 Il punto \`e questo:
	che tu, in andando al fonte con colei,
	cerchi d'intenerirla: ed io fra tanto
	procurer\`o ch'Aminta l\`a ne venga.
	N\'e la mia forse men difficil cura
	sar\`a di questa tua. Or vanne. \\

   \2 Io vado,
	ma il proposito nostro altro intendeva.

	\5 Se ben ravviso di lontan la faccia,
	Aminta \`e quel che di l\`a spunta. \`E desso.


\Scena

\(\4, \5\)

	\4 Vorr\`o veder ci\`o che Tirsi avr\`a fatto:
	e, s'avr\`a fatto nulla,
	prima ch'io vada in nulla,
uccider vo' me stesso inanzi a gli occhi
	de la crudel fanciulla.
	A lei, cui tanto piace
	la piaga del mio core,
	colpo de' suoi begli occhi,
altrettanto piacer devr\`a per certo
	la piaga del mio petto,
	colpo de la mia mano.

	\5 Nove, Aminta, t'annuncio di conforto:
	lascia omai questo tanto lamentarti.

	\4 Ohim\`e, che di'? che porte?
	O la vita o la morte?

	\5 Porto salute e vita, s'ardirai
	di farti loro incontra; ma fa d'uopo
	d'esser un uom, Aminta, un uom ardito.

	\4 Qual ardir mi bisogna, e 'ncontra a cui?

	\5 Se la tua donna fosse in mezz'un bosco,
	che, cinto intorno d'altissime rupi,
	desse albergo a le tigri ed a' leoni,
	v'andresti tu? \\

   \4 V'andrei sicuro e baldo
	pi\`u che di festa villanella al ballo.

	\5 E s'ella fosse tra ladroni ed armi,
	v'andresti tu? \\

   \4 V'andrei pi\`u lieto e pronto
	che l'assetato cervo a la fontana.

	\5 Bisogna a maggior prova ardir pi\`u grande.

	\4 Andr\`o per mezzo i rapidi torrenti,
	quando la neve si discioglie e gonfi
	li manda al mare; andr\`o per mezzo 'l foco
	e ne l'inferno, quando ella vi sia,
	s'esser pu\`o inferno ov'\`e cosa s\`{\i} bella.
	Ors\`u, scuoprimi il tutto. \\

   \5 Odi. \\

   \4 Di' tosto.

	\5 Silvia t'attende a un fonte, ignuda e sola.
	Ardirai tu d'andarvi? \\

   \4 Oh, che mi dici?
	Silvia m'attende ignuda e sola? \\

   \5 Sola,
	se non quanto v'\`e Dafne, ch'\`e per noi.

	\4 Ignuda ella m'aspetta? \\

   \5 Ignuda: ma...

	\4 Ohim\`e, che ``ma''? Tu taci; tu m'uccidi.

	\5 Ma non sa gi\`a che tu v'abbi d'andare.

	\4 Dura conclusion, che tutte attosca
	le dolcezze passate. Or, con qual arte,
	crudel, tu mi tormenti?
	Poco dunque ti pare
	che infelice io sia,
	che a crescer vieni la miseria mia?

	\5 S'a mio senno farai, sarai felice.

	\4 E che consigli? \\

   \5 Che tu prenda quello
	che la fortuna amica t'appresenta.

	\4 Tolga Dio che mai faccia
	cosa che le dispiaccia;
	cosa io non feci mai che le spiacesse,
	fuor che l'amarla: e questo a me fu forza,
	forza di sua bellezza, e non mia colpa.
	Non sar\`a dunque ver ch'in quanto io posso,
	non cerchi compiacerla. \\

   \5 Ormai rispondi:
	se fosse in tuo poter di non amarla,
	lasciaresti d'amarla, per piacerle?

	\4 N\'e questo mi consente Amor ch'io dica,
	n\'e ch'imagini pur d'aver gi\`a mai
	a lasciar il suo amor, bench'io potessi.

	\5 Dunque tu l'ameresti al suo dispetto,
	quando potessi far di non amarla.

	\4 Al suo dispetto no, ma l'amerei.

	\5 Dunque fuor di sua voglia. \\

   \4 S\`{\i} per certo.

	\5 Perch\'e dunque non osi oltra sua voglia
	prenderne quel che, se ben grava in prima,
	al fin, al fin le sar\`a caro e dolce
	che l'abbi preso? \\

   \4 Ahi, Tirsi, Amor risponda
	per me; ch\'e quanto a mezz'il cor mi parla,
	non so ridir. Tu troppo scaltro sei
	gi\`a per lungo uso a ragionar d'amore:
	a me lega la lingua
	quel che mi lega il core.

	\5 Dunque andar non vogliamo? \\

   \4 Andare io voglio,
	ma non dove tu stimi. \\

   \5 E dove? \\

   \4 A morte,
	s'altro in mio pro' non hai fatto che quanto
	ora mi narri. \\

   \5 E poco parti questo?
	Credi tu dunque, sciocco, che mai Dafne
	consigliasse l'andar, se non vedesse
	in parte il cor di Silvia? E forse ch'ella
	il sa, n\'e per\`o vuol ch'altri risappia
	ch'ella ci\`o sappia. Or, se 'l consenso espresso
	cerchi di lei, non vedi che tu cerchi
	quel che pi\`u le dispiace? Or dove \`e dunque
	questo tuo desiderio di piacerle?
	E s'ella vuol che 'l tuo diletto sia
	tuo furto o tua rapina, e non suo dono
	n\'e sua mercede, a te, folle, che importa
	pi\`u l'un modo che l'altro? \\

   \4 E chi m'accerta
	che il suo desir sia tale? \\

   \5 Oh mentecatto!
	Ecco, tu chiedi pur quella certezza
	ch'a lei dispiace, e dispiacer le deve
	dirittamente, e tu cercar non d\'ei.
	Ma chi t'accerta ancor che non sia tale?
	Or s'ella fosse tale, e non v'andassi?
	Eguale \`e il dubbio e 'l rischio. Ahi, pur \`e meglio
	come ardito morir, che come vile.
	Tu taci, tu sei vinto. Ora confessa
	questa perdita tua, che fia cagione
	di vittoria maggiore. Andianne. \\

   \4 Aspetta.

	\5 Che ``Aspetta''? non sai ben che 'l tempo fugge?

	\4 Deh, pensiam pria se ci\`o dee farsi, e come.

	\5 Per strada penserem ci\`o che vi resta;
	ma nulla fa chi troppe cose pensa.

	\9 Amore, in quale scola,
	da qual mastro s'apprende
	la tua s\`{\i} lunga e dubbia arte d'amare?
	Chi n'insegna a spiegare
	ci\`o che la mente intende,
	mentre con l'ali tue sovra il ciel vola?
	Non gi\`a la dotta Atene,
	n\'e 'l Liceo ne 'l dimostra;
	non Febo in Elicona,
	che s\`{\i} d'Amor ragiona
	come colui ch'impara:
	freddo ne parla, e poco;
	non ha voce di foco,
	come a te si conviene;
	non alza i suoi pensieri
	a par de' tuoi misteri.
	Amor, degno maestro
	sol tu sei di te stesso,
	e sol tu sei da te medesmo espresso;
	tu di legger insegni
	ai pi\`u rustici ingegni
	quelle mirabil cose
	che con lettre amorose
	scrivi di propria man negli occhi altrui;
	tu in bei facondi detti
	sciogli la lingua de' fedeli tuoi;
	e spesso (oh strana e nova
	eloquenza d'Amore!)
	spesso in un dir confuso
	e 'n parole interrotte
	meglio si esprime il core,
	e pi\`u par che si mova,
	che non si fa con voci adorne e dotte;
	e 'l silenzio ancor suole
	aver prieghi e parole.
	Amor, leggan pur gli altri
	le socratiche carte,
	ch'io in due begli occhi apprender\`o quest'arte;
	e perderan le rime
	de le penne pi\`u saggie
	appo le mie selvaggie,
	che rozza mano in rozza scorza imprime.



\Atto

\Scena

\( \5, \9\)

	\5 Oh crudeltate estrema, oh ingrato core,
	oh donna ingrata, oh tre fiate e quattro
	ingratissimo sesso! E tu, natura,
	negligente maestra, perch\'e solo
	a le donne nel volto e in quel di fuori
	ponesti quanto in loro \`e di gentile,
	di mansueto e di cortese, e tutte
	l'altre parti obliasti? Ahi, miserello,
	forse ha se stesso ucciso; ei non appare;
	io l'ho cerco e ricerco omai tre ore
	nel loco ov'io il lasciai e nei contorni:
	n\'e trovo lui n\'e orme de' suoi passi.
	Ahi, che s'\`e certo ucciso! Io vo' novella
	chiederne a que' pastor che col\`a veggio.
	Amici, avete visto Aminta, o inteso
	novella di lui forse? \\

   \9 Tu mi pari
	cos\`{\i} turbato: e qual cagion t'affanna?
	Ond'\`e questo sudor, e questo ansare?
	Havvi nulla di mal? fa che 'l sappiamo.

	\5 Temo del mal d'Aminta: avetel visto?

	\9 Noi visto non l'abbiam dapoi che teco,
	buona pezza, part\`{\i}; ma che ne temi?

	\5 Ch'egli non s'abbia ucciso di sua mano.

	\9 Ucciso di sua mano? or perch\'e questo?
	che ne stimi cagione? \\

   \5 Odio ed Amore.

	\9 Duo potenti inimici, insieme aggiunti,
	che far non ponno? Ma parla pi\`u chiaro.

	\5 L'amar troppo una ninfa, e l'esser troppo
	odiato da lei. \\

   \9 Deh, narra il tutto;
	questo \`e luogo di passo, e forse intanto
	alcun verr\`a che nova di lui rechi:
	forse arrivar potrebbe anch'egli istesso.

	\5 Dirollo volontier, ch\'e non \`e giusto,
	che tanta ingratitudine e s\`{\i} strana
	senza l'infamia debita si resti.
	Presentito avea Aminta (ed io fui lasso,
	colui che rifer\`{\i}'lo e che 'l condussi:
	or me ne pento) che Silvia dovea
	con Dafne ire a lavarsi ad una fonte.
	L\`a dunque s'invi\`o dubbio ed incerto,
	mosso non dal suo cor, ma sol dal mio
	stimolar importuno; e spesso in forse
	fu di tornar indietro, ed io 'l sospinsi,
	pur mal suo grado, inanzi. Or quando omai
	c'era il fonte vicino, ecco, sentiamo
	un feminil lamento; e quasi a un tempo
	Dafne veggiam, che battea palma a palma;
	la qual, come ci vide, alz\`o la voce:
	``Ah, correte,'' grid\`o ``Silvia \`e sforzata''.
	L'inamorato Aminta, che ci\`o intese,
	si spicc\`o com'un pardo, ed io segu\`{\i}'lo;
	ecco miriamo a un'arbore legata
	la giovinetta, ignuda come nacque,
	ed a legarla fune era il suo crine:
	il suo crine medesmo in mille nodi
	a la pianta era avvolto; e 'l suo bel cinto,
	che del sen virginal fu pria custode,
	di quello stupro era ministro, ed ambe
	le mani al duro tronco le stringea;
	e la pianta medesma avea prestati
	legami contra lei: ch'una ritorta
	d'un pieghevole ramo avea a ciascuna
	de le tenere gambe. A fronte a fronte
	un satiro villan noi le vedemmo,
	che di legarla pur allor finia.
	Ella quanto potea faceva schermo;
	ma che potuto avrebbe a lungo andare?
	Aminta, con un dardo che tenea
	ne la man destra, al satiro avventossi
	come un leone, ed io fra tanto pieno
	m'avea di sassi il grembo, onde fuggissi.
	Come la fuga de l'altro concesse
	spazio a lui di mirare, egli rivolse
	i cupidi occhi in quelle membra belle,
	che, come suole tremolare il latte
	ne' giunchi, s\`{\i} parean morbide e bianche.
	E tutto 'l vidi sfavillar nel viso;
	poscia accostossi pianamente a lei
	tutto modesto, e disse: ``O bella Silvia,
	perdona a queste man, se troppo ardire
	\`e l'appressarsi a le tue dolci membra,
	perch\'e necessit\`a dura le sforza:
	necessit\`a di scioglier questi nodi;
	n\'e questa grazia, che fortuna vuole
	conceder loro, tuo mal grado sia''.

	\9 Parole d'ammollir un cor di sasso.
	Ma che rispose allor? \\

   \5 Nulla rispose,
	ma disdegnosa e vergognosa a terra
	chinava il viso, e 'l delicato seno,
	quanto potea torcendosi, celava.
	Egli, fattosi inanzi, il biondo crine
	cominci\`o a sviluppare, e disse in tanto:
	``Gi\`a di nodi s\`{\i} bei non era degno
	cos\`{\i} ruvido tronco: or, che vantaggio
	hanno i servi d'Amor, se lor commune
	\`e con le piante il prezioso laccio?
	Pianta crudel, potesti quel bel crine
	offender tu, ch'a te feo tanto onore?''
	Quinci con le sue man le man le sciolse,
	in modo tal che parea che temesse
	pur di toccarle, e desiasse insieme;
	si chin\`o poi per islegarle i piedi;
	ma come Silvia in libert\`a le mani
	si vide, disse in atto dispettoso:
	``Pastor, non mi toccar: son di Diana;
	per me stessa sapr\`o sciogliermi i piedi''.
	\9 Or tanto orgoglio alberga in cor di ninfa?
	Ahi d'opra graziosa ingrato merto!
	\5 Ei si trasse in disparte riverente,
	non alzando pur gli occhi per mirarla,
	negando a se medesmo il suo piacere,
	per t\^orre a lei fatica di negarlo.
	Io, che m'era nascoso, e vedea il tutto
	ed udia il tutto, allor fui per gridare;
	pur mi ritenni. Or odi strana cosa.
	Dopo molta fatica ella si sciolse;
	e, sciolta a pena, senza dire ``A Dio'',
	a fuggir cominci\`o com'una cerva;
	e pur nulla cagione avea di tema,
	ch\'e l'era noto il rispetto d'Aminta.

	\9 Perch\'e dunque fuggissi? \\

   \5 A la sua fuga
	volse l'obligo aver, non a l'altrui
	modesto amore. \\

   \9 Ed in quest'anco \`e ingrata.
	Ma che fe' 'l miserello allor? che disse?

	\5 No 'l so, ch'io, pien di mal talento, corsi
	per arrivarla e ritenerla, e 'nvano,
	ch'io la smarrii; e poi tornando dove
	lasciai Aminta al fonte, no 'l trovai;
	ma presago \`e il mio cor di qualche male.
	So ch'egli era disposto di morire,
	prima che ci\`o avvenisse. \\

   \9 \`E uso ed arte
	di ciascun ch'ama minacciarsi morte;
	ma rade volte poi segue l'effetto.

	\5 Dio faccia ch'ei non sia tra questi rari.

	\9 Non sar\`a, no. \\

   \5 Io voglio irmene a l'antro
	del saggio Elpino: ivi, s'\`e vivo, forse
	sar\`a ridotto, ove sovente suole
	raddolcir gli amarissimi martiri
	al dolce suon de la sampogna chiara,
	ch'ad udir trae dagli alti monti i sassi,
	e correr fa di puro latte i fiumi,
	e stillar mele da le dure scorze.


\Scena

\(\4, \2, \7\)

	\4 Dispietata pietate
	fu la tua veramente, o Dafne, allora
	che ritenesti il dardo;
	per\`o che 'l mio morire
	pi\`u amaro sar\`a, quanto pi\`u tardo.
	Ed or perch\'e m'avvolgi
	per s\`{\i} diverse strade e per s\`{\i} varii
	ragionamenti in vano? di che temi?
	ch'io non m'uccida? Temi del mio bene.

	\2 Non disperar, Aminta,
	ch\'e, s'io lei ben conosco,
	sola vergogna fu, non crudeltate,
	quella che mosse Silvia a fuggir via.

	\4 Ohim\`e, che mia salute
	sarebbe il disperare,
	poich\'e sol la speranza
	\`e stata mia rovina; ed anco, ahi lasso,
	tenta di germogliar dentr'al mio petto,
	sol perch\'e io viva: e quale \`e maggior male
	de la vita d'un misero com'io?

	\2 Vivi, misero, vivi
	ne la miseria tua; e questo stato
	sopporta sol per divenir felice,
	quando che sia. Fia premio de la speme,
	se vivendo e sperando ti mantieni,
	quel che vedesti ne la bella ignuda.

	\4 Non pareva ad Amor e a mia fortuna
	ch'a pien misero fossi, s'anco a pieno
	non m'era dimostrato
	quel che m'era negato.

	\7 Dunque a me pur convien esser sinistra
	c\`ornice d'amarissima novella!
	Oh per mai sempre misero Montano,
	qual animo fia 'l tuo quando udirai
	de l'unica tua Silvia il duro caso?
	Padre vecchio, orbo padre: ahi, non pi\`u padre!

	\2 Odo una mesta voce. \\

   \4 Io odo 'l nome
	di Silvia, che gli orecchi e 'l cor mi fere;
	ma chi \`e che la noma? \\

   \2 Ella \`e Nerina,
	ninfa gentil che tanto a Cinzia \`e cara,
	c'ha s\`{\i} begli occhi e cos\`{\i} belle mani
	e modi s\`{\i} avvenenti e graziosi.

	\7 E pur voglio che 'l sappi e che procuri
	di ritrovar le reliquie infelici,
	se nulla ve ne resta. Ahi Silvia, ahi dura
	infelice tua sorte!

	\4 Ohim\`e, che fia? che costei dice? \\

   \7 Dafne!

	\2 Che parli fra te stessa, e perch\'e nomi
	tu Silvia, e poi sospiri? \\

   \7 Ahi, ch'a ragione
	sospiro l'aspro caso! \\

   \4 Ahi, di qual caso
	pu\`o ragionar costei? Io sento, io sento
	che mi s'agghiaccia il core e mi si chiude
	lo spirto. \`E viva?

	\2 Narra, qual aspro caso \`e quel che dici?

	\7 O Dio, perch\'e son io
	la messaggiera? E pur convien narrarlo.
	Venne Silvia al mio albergo ignuda; e quale
	fosse l'occasion, saper la d\'ei;
	poi rivestita mi preg\`o che seco
	ir volessi a la caccia che ordinata
	era nel bosco c'ha nome da l'elci.
	Io la compiacqui: andammo, e ritrovammo
	molte ninfe ridotte; ed indi a poco
	ecco, di non so d'onde, un lupo sbuca,
	grande fuor di misura, e da le labra
	gocciolava una bava sanguinosa;
	Silvia un quadrello adatta su la corda
	d'un arco ch'io le diedi, e tira e 'l coglie
	a sommo 'l capo: ei si rinselva, ed ella,
	vibrando un dardo, dentro 'l bosco il segue.

	\4 Oh dolente principio; ohim\`e, qual fine
	gi\`a mi s'annuncia? \\

   \7 Io con un altro dardo
	seguo la traccia, ma lontana assai,
	ch\'e pi\`u tarda mi mossi. Come furo
	dentro a la selva, pi\`u non la rividi:
	ma pur per l'orme lor tanto m'avvolsi,
	che giunsi nel pi\`u folto e pi\`u deserto;
	quivi il dardo di Silvia in terra scorsi,
	n\'e molto indi lontano un bianco velo,
	ch'io stessa le ravvolsi al crine; e, mentre
	mi guardo intorno, vidi sette lupi
	che leccavan di terra alquanto sangue
	sparto intorno a cert'ossa affatto nude;
	e fu mia sorte ch'io non fui veduta
	da loro, tanto intenti erano al pasto;
	tal che, piena di tema e di pietate,
	indietro ritornai; e questo \`e quanto
	posso dirvi di Silvia; ed ecco 'l velo.

	\4 Poco p\`arti aver detto? Oh velo, oh sangue,
	oh Silvia, tu se' morta! \\

   \2 Oh miserello,
	tramortito \`e d'affanno, e forse morto.

	\7 Egli rispira pure: questo fia
	un breve svenimento; ecco, riviene.

	\4 Dolor, che s\`{\i} mi crucii,
	ch\'e non m'uccidi omai? tu sei pur lento!
	Forse lasci l'officio a la mia mano.
	Io son, io son contento
	ch'ella prenda tal cura,
	poi che tu la ricusi, o che non puoi.
	Ohim\`e, se nulla manca
	a la certezza omai,
	e nulla manca al colmo
	de la miseria mia,
	che bado? che pi\`u aspetto? O Dafne, o Dafne,
	a questo amaro fin tu mi salvasti,
	a questo fine amaro?
	Bello e dolce morir fu certo allora
	che uccidere io mi volsi.
	Tu me 'l negasti, e 'l Ciel, a cui parea
	ch'io precorressi col morir la noia
	ch'apprestata m'avea.
	Or che fatt'ha l'estremo
	de la sua crudeltate,
	ben soffrir\`a ch'io moia,
	e tu soffrir lo dei.

	\2 Aspetta a la tua morte,
	sin che 'l ver meglio intenda.

	\4 Ohim\`e, che vuoi ch'attenda?
	Ohim\`e, che troppo ho atteso, e troppo inteso.

	\7 Deh, foss'io stata muta!

	\4 Ninfa, dammi, ti prego,
	quel velo ch'\`e di lei
	solo e misero avanzo,
	s\`{\i} ch'egli m'accompagne
	per questo breve spazio
	e di via e di vita che mi resta,
	e con la sua presenza
	accresca quel martire,
	ch'\`e ben picciol martire,
	s'ho bisogno d'aiuto al mio morire.

	\7 Debbo darlo o negarlo?
	La cagion perch\'e 'l chiedi
	fa ch'io debba negarlo.

	\4 Crudel, s\`{\i} picciol dono
	mi nieghi al punto estremo?
	E in questo anco maligno
	mi si mostra il mio fato. Io cedo, io cedo:
	a te si resti; e voi restate ancora,
	ch'io vo per non tornare.

	\2 Aminta, aspetta, ascolta...
	Ohim\`e, con quanta furia egli si parte!

	\7 Egli va s\`{\i} veloce,
	che fia vano il seguirlo; ond'\`e pur meglio
	ch'io segua il mio viaggio; e forse \`e meglio
	ch'io taccia e nulla conti
	al misero Montano.

	\9 Non bisogna la morte,
	ch'a stringer nobil core
	prima basta la fede, e poi l'amore.
	N\'e quella che si cerca
	\`e s\`{\i} difficil fama
	seguendo chi ben ama,
	ch'amore \`e merce, e con amar si merca.
	E cercando l'amor si trova spesso
	gloria immortal appresso.



\Atto

\Scena

\(\2, \3, \9\)

	\2 Ne porti il vento, con la ria novella,
	che s'era di te sparta, ogni tuo male
	e presente e futuro. Tu sei viva
	e sana, Dio lodato, ed io per morta
	pur ora ti tenea: in tal maniera
	m'avea Nerina il tuo caso dipinto.
	Ahi, fosse stata muta, ed altri sordo!

	\3 Certo 'l rischio fu grande, ed ella avea
	giusta cagion di sospettarmi morta.

	\2 Ma non giusta cagion avea di dirlo.
	Or narra tu qual fosse 'l rischio, e come
	tu lo fuggisti. \\

   \3 Io, seguitando un lupo,
	mi rinselvai nel pi\`u profondo bosco,
	tanto ch'io ne perdei la traccia. Or, mentre
	cerco di ritornare onde mi tolsi,
	il vidi, e riconobbi a un stral che fitto
	gli aveva di mia man press'un orecchio.
	Il vidi con molt'altri intorno a un corpo
	d'un animal ch'avea di fresco ucciso,
	ma non distinsi ben la forma. Il lupo
	ferito, credo, mi conobbe, e 'ncontro
	mi venne con la bocca sanguinosa.
	Io l'aspettava ardita, e con la destra
	vibrava un dardo. Tu sai ben s'io sono
	maestra di ferire, e se mai soglio
	far colpo in fallo. Or, quando il vidi tanto
	vicin, che giusto spazio mi parea
	a la percossa, lanciai un dardo, e 'n vano:
	ch\'e, colpa di fortuna o pur mia colpa,
	in vece sua colsi una pianta. Allora
	pi\`u ingordo incontro ei mi venia; ed io
	che 'l vidi s\`{\i} vicin, che stimai vano
	l'uso de l'arco, non avendo altr'armi,
	a la fuga ricorsi. Io fuggo, ed egli
	non resta di seguirmi. Or odi caso:
	un vel, ch'aveva involto intorno al crine,
	si spieg\`o in parte, e giva ventilando,
	s\`{\i} ch'ad un ramo avviluppossi. Io sento
	che non so chi mi tien e mi ritarda.
	Io, per la tema del morir, raddoppio
	la forza al corso, e d'altra parte il ramo
	non cede, e non mi lascia; al fin mi svolgo
	del velo, e alquanto de' miei crini ancora
	lascio svelti co 'l velo; e cotant'ali
	m'impenn\`o la paura ai pi\`e fugaci,
	ch'ei non mi giunse e salva uscii del bosco.
	Poi, tornando al mio albergo, io t'incontrai
	tutta turbata, e mi stupii vedendo
	stupirti al mio apparir. \\

   \2 Ohim\`e, tu vivi,
	altri non gi\`a. \\

   \3 Che dici? ti rincresce
	forse ch'io viva sia? M'odii tu tanto?

	\2 Mi piace di tua vita, ma mi duole
	de l'altrui morte. \\

   \3 E di qual morte intendi?

	\2 De la morte d'Aminta. \\

   \3 Ahi, come \`e morto?

	\2 Il come non so dir, n\'e so dir anco
	s'\`e ver l'effetto; ma per certo il credo.

	\3 Ch'\`e ci\`o che tu mi dici? ed a chi rechi
	la cagion di sua morte? \\

   \2 A la tua morte.

	\3 Io non t'intendo. \\

   \2 La dura novella
	de la tua morte, ch'egli ud\`{\i} e credette,
	avr\`a porto al meschino il laccio o 'l ferro
	od altra cosa tal che l'avr\`a ucciso.

	\3 Vano il sospetto in te de la sua morte
	sar\`a, come fu van de la mia morte;
	ch'ognuno a suo poter salva la vita.

	\2 O Silvia, Silvia, tu non sai n\'e credi
	quanto 'l foco d'amor possa in un petto,
	che petto sia di carne e non di pietra,
	com' \`e cotesto tuo: ch\'e, se creduto
	l'avessi, avresti amato chi t'amava
	pi\`u che le care pupille degli occhi,
	pi\`u che lo spirto de la vita sua.
	Il credo io ben, anzi l'ho visto e sollo:
	il vidi, quando tu fuggisti, o fera
	pi\`u che tigre crudel, ed in quel punto,
	ch'abbracciar lo dovevi, il vidi un dardo
	rivolgere in se stesso, e quello al petto
	premersi disperato, n\'e pentirsi
	poscia nel fatto, che le vesti ed anco
	la pelle trapassossi, e nel suo sangue
	lo tinse; e 'l ferro saria giunto a dentro,
	e passato quel cor che tu passasti
	pi\`u duramente, se non ch'io gli tenni
	il braccio, e l'impedii ch'altro non fesse.
	Ahi lassa, e forse quella breve piaga
	solo una prova fu del suo furore
	e de la disperata sua costanza,
	e mostr\`o quella strada al ferro audace,
	che correr poi dovea liberamente.

	\3 Oh, che mi narri? \\

   \2 Il vidi poscia, allora
	ch'intese l'amarissima novella
	de la tua morte, tramortir d'affanno,
	e poi partirsi furioso in fretta,
	per uccider se stesso; e s'avr\`a ucciso
	veracemente. \\

   \3 E ci\`o per fermo tieni?

	\2 Io non v'ho dubbio. \\

   \3 Ohim\`e, tu no 'l seguisti
	per impedirlo? Ohim\`e, cerchiamo, andiamo,
	che, poi ch'egli moria per la mia morte,
	de' per la vita mia restare in vita.

	\2 Io lo seguii, ma correa s\`{\i} veloce
	che mi spar\`{\i} tosto dinanzi, e 'ndarno
	poi mi girai per le sue orme. Or dove
	vuoi tu cercar, se non n'hai traccia alcuna?

	\3 Egli morr\`a, se no 'l troviamo, ahi lassa;
	e sar\`a l'omicida ei di se stesso.

	\2 Crudel, forse t'incresce ch'a te tolga
	la gloria di quest'atto? esser tu dunque
	l'omicida vorresti? e non ti pare
	che la sua cruda morte esser debb'opra
	d'altri che di tua mano? Or ti consola,
	ch\'e, comunque egli muoia, per te muore,
	e tu sei che l'uccidi.

	\3 Ohim\`e, che tu m'accori, e quel cordoglio
	ch'io sento del suo caso inacerbisce
	con l'acerba memoria
	de la mia crudeltate,
	ch'io chiamava onestate; e ben fu tale,
	ma fu troppo severa e rigorosa;
	or me n'accorgo e pento. \\

   \2 Oh, quel ch'io odo!
	Tu sei pietosa, tu, tu senti al core
	spirto alcun di pietate? oh che vegg'io?
	tu piangi, tu, superba? Oh maraviglia!
	Che pianto \`e questo tuo? pianto d'amore?

	\3 Pianto d'amor non gi\`a, ma di pietate.

	\2 La piet\`a messaggiera \`e de l'amore,
	come 'l lampo del tuono. \\

   \9 Anzi sovente
	quando egli vuol ne' petti virginelli
	occulto entrare, onde fu prima escluso
	da severa onest\`a, l'abito prende,
	prende l'aspetto de la sua ministra
	e sua nuncia, pietate; e con tai larve
	le semplici ingannando, \`e dentro accolto.

	\2 Questo \`e pianto d'amor, ch\'e troppo abonda.
	Tu taci? ami tu, Silvia? ami, ma in vano.
	Oh potenza d'Amor, giusto castigo
	manda sovra costei. Misero Aminta!
	Tu, in guisa d'ape che ferendo muore
	e ne le piaghe altrui lascia la vita,
	con la tua morte hai pur trafitto al fine
	quel duro cor, che non potesti mai
	punger vivendo. Or, se tu, spirto errante,
	s\`{\i} come io credo, e de le membra ignudo,
	qui intorno sei, mira il suo pianto, e godi:
	amante in vita, amato in morte; e s'era
	tuo destin che tu fossi in morte amato,
	e se questa crudel volea l'amore
	venderti sol con prezzo cos\`{\i} caro,
	desti quel prezzo tu ch'ella richiese,
	e l'amor suo col tuo morir comprasti.

	\9 Caro prezzo a chi 'l diede; a chi 'l riceve
	prezzo inutile, e infame. \\

   \3 Oh potess'io
	con l'amor mio comprar la vita sua;
	anzi pur con la mia la vita sua,
	s'egli \`e pur morto! \\

   \2 O tardi saggia, e tardi
	pietosa, quando ci\`o nulla rileva!


\Scena

\( \10, \9, \3, \2\)

	\10 Io ho s\`{\i} pieno il petto di pietate
	e s\`{\i} pieno d'orror, che non rimiro
	n\'e odo alcuna cosa, ond'io mi volga,
	la qual non mi spaventi e non m'affanni.

	\9 Or ch'apporta costui,
	ch'\`e s\`{\i} turbato in vista ed in favella?

	\10 Porto l'aspra novella
	de la morte d'Aminta. \\

   \3 Ohim\`e, che dice?

	\10 Il pi\`u nobil pastor di queste selve,
	che fu cos\`{\i} gentil, cos\`{\i} leggiadro,
	cos\`{\i} caro a le ninfe ed a le Muse,
	ed \`e morto fanciullo, ahi, di che morte!

	\9 Contane, prego, il tutto, acci\`o che teco
	pianger possiam la sua sciagura e nostra.

	\3 Ohim\`e, ch'io non ardisco
	appressarmi ad udire
	quel ch'\`e pur forza udire. Empio mio core,
	mio duro alpestre core,
	di che, di che paventi?
	Vattene incontra pure
	a quei coltei pungenti
	che costui porta ne la lingua, e quivi
	mostra la tua fierezza.
	Pastore, io vengo a parte
	di quel dolor che tu prometti altrui,
	ch\'e a me ben si conviene
	pi\`u che forse non pensi; ed io 'l ricevo
	come dovuta cosa. Or tu di lui
	non mi sii dunque scarso.

	\10 Ninfa, io ti credo bene,
	ch'io sentii quel meschino in su la morte
	finir la vita sua
	co 'l chiamar il tuo nome.

	\2 Ora comincia omai
	questa dolente istoria.

	\10 Io era a mezzo 'l colle, ove avea tese
	certe mie reti, quanto assai vicino
	vidi passar Aminta, in volto e in atti
	troppo mutato da quel ch'ei soleva,
	troppo turbato e scuro. Io corsi, e corsi
	tanto che 'l giunsi e lo fermai; ed egli
	mi disse: ``Ergasto, io vo' che tu mi faccia
	un gran piacere: quest'\`e, che tu ne venga
	meco per testimonio d'un mio fatto;
	ma pria voglio da te che tu mi leghi
	di stretto giuramento la tua fede
	di startene in disparte e non por mano,
	per impedirmi in quel che son per fare''.
	Io (chi pensato avria caso s\`{\i} strano,
	n\'e s\`{\i} pazzo furor?), com' egli volse,
	feci scongiuri orribili, chiamando
	e Pane e Pale e Priapo e Pomona,
	ed Ecate notturna. Indi si mosse,
	e mi condusse ov'\`e scosceso il colle,
	e gi\`u per balzi e per dirupi incolti
	strada non gi\`a, ch\'e non v'\`e strada alcuna,
	ma cala un precipizio in una valle.
	Qui ci fermammo. Io, rimirando a basso,
	tutto sentii raccapricciarmi, e 'ndietro
	tosto mi trassi; ed egli un cotal poco
	parve ridesse, e serenossi in viso;
	onde quell'atto pi\`u rassicurommi.
	Indi parlommi s\`{\i}: ``Fa che tu conti
	a le ninfe e ai pastor ci\`o che vedrai''.
	Poi disse, in gi\`u guardando:
	``Se presti a mio volere
	cos\`{\i} aver io potessi
	la gola e i denti de gli avidi lupi,
	com'ho questi dirupi,
	sol vorrei far la morte
	che fece la mia vita:
	vorrei che queste mie membra meschine
	s\`{\i} fosser lacerate,
	ohim\`e, come gi\`a foro
	quelle sue delicate.
	Poi che non posso, e 'l cielo
	dinega al mio desire
	gli animali voraci,
	che ben verriano a tempo, io prender voglio
	altra strada al morire:
	prender\`o quella via
	che, se non la devuta,
	almen fia la pi\`u breve.
	Silvia, io ti seguo, io vengo
	a farti compagnia,
	se non la sdegnerai;
	e morirei contento,
	s'io fossi certo almeno
	che 'l mio venirti dietro
	turbar non ti dovesse,
	e che fosse finita
	l'ira tua con la vita.
	Silvia, io ti seguo, io vengo''. Cos\`{\i} detto,
	precipitossi d'alto
	co 'l capo in giuso; ed io restai di ghiaccio.

	\2 Misero Aminta! \\

   \3 Ohim\`e!

	\9 Perch\'e non l'impedisti?
	Forse ti fu ritegno a ritenerlo
	il fatto giuramento?

	\10 Questo no, ch\'e, sprezzando i giuramenti,
	vani forse in tal caso,
	quand'io m'accorsi del suo pazzo ed empio
	proponimento, con la man vi corsi,
	e, come volse la sua dura sorte,
	lo presi in questa fascia di zendado
	che lo cingeva; la qual, non potendo
	l'impeto e 'l peso sostener del corpo,
	che s'era tutto abandonato, in mano
	spezzata mi rimase. \\

   \9 E che divenne
	de l'infelice corpo? \\

   \10 Io no 'l so dire:
	ch'era s\`{\i} pien d'orrore e di pietate,
	che non mi diede il cor di rimirarvi,
	per non vederlo in pezzi. \\

   \9 O strano caso!

	\3 Ohim\`e, ben son di sasso,
	poi che questa novella non m'uccide.
	Ahi, se la falsa morte
	di chi tanto l'odiava
	a lui tolse la vita,
	ben sarebbe ragione
	che la verace morte
	di chi tanto m'amava
	togliesse a me la vita;
	e vo' che la mi tolga,
	se non potr\`o co 'l duol, almen co 'l ferro,
	o pur con questa fascia,
	che non senza cagione
	non segu\`{\i} le ruine
	del suo dolce signore,
	ma rest\`o sol per fare in me vendetta
	de l'empio mio rigore
	e del suo amaro fine.
	Cinto infelice, cinto
	di signor pi\`u infelice,
	non ti spiaccia restare
	in s\`{\i} odioso albergo,
	ch\'e tu vi resti sol per instrumento
	di vendetta e di pena.
	Dovea certo, io dovea
	esser compagna al mondo
	de l'infelice Aminta.
	Poscia ch'allor non volsi,
	sar\`o per opra tua
	sua compagna a l'inferno.

	\9 Cons\`olati, meschina,
	che questo \`e di fortuna e non tua colpa.

	\3 Pastor, di chi piangete?
	Se piangete il mio affanno,
	io non merto pietate,
	ch\'e non la seppi usare;
	se piangete il morire
	del misero innocente,
	questo \`e picciolo segno
	a s\`{\i} alta cagione. E tu rasciuga,
	Dafne, queste tue lagrime, per Dio.
	Se cagion ne son io,
	ben ti voglio pregare,
	non per piet\`a di me, ma per pietate
	di chi degno ne fue,
	che m'aiuti a cercare
	l'infelici sue membra e a sepelirle.
	Questo sol mi ritiene,
	ch'or ora non m'uccida:
	pagar vo' questo ufficio,
	poi ch'altro non m'avanza,
	a l'amor ch'ei portommi;
	e se ben quest'empia
	mano contaminare
	potesse la piet\`a de l'opra, pure
	so che gli sar\`a cara
	l'opra di questa mano;
	ch\'e so certo ch'ei m'ama,
	come mostr\`o morendo.

	\2 Son contenta aiutarti in questo ufficio;
	ma tu gi\`a non pensare
	d'aver poscia a morire.

	\3 Sin qui vissi a me stessa,
	a la mia feritate: or, quel ch'avanza,
	viver voglio ad Aminta;
	e, se non posso a lui,
	viver\`o al freddo suo
	cadavero infelice.
	Tanto, e non pi\`u, mi lice
	restar nel mondo, e poi finir a un punto
	e l'essequie e la vita.
	Pastor, ma quale strada
	ci conduce a la valle, ove il dirupo
	va a terminare? \\

   \10 Questa vi conduce;
	e quinci poco spazio ella \`e lontana.

	\2 Andiam, che verr\`o teco e guiderotti;
	ch\'e ben rammento il luogo. \\

   \3 A Dio, pastori;
	piagge, a Dio; a Dio, selve; e fiumi, a Dio.
	\10 Costei parla di modo, che dimostra
	d'esser disposta a l'ultima partita.
	\9 Ci\`o che morte rallenta, Amor, restringi,
	amico tu di pace, ella di guerra,
	e del suo trionfar trionfi e regni;
	e mentre due bell'alme annodi e cingi,
	cos\`{\i} rendi sembiante al ciel la terra,
	che d'abitarla tu non fuggi o sdegni.
	Non sono ire l\`a su: gli umani ingegni
	tu placidi ne rendi, e l'odio interno
	sgombri, signor, da' mansueti cori,
	sgombri mille furori;
	e quasi fai col tuo valor superno
	de le cose mortali un giro eterno.



\Atto

\(\8, \9\)

	\8 Veramente la legge con che Amore
	il suo imperio governa eternamente
	non \`e dura, n\'e obliqua; e l'opre sue,
	piene di providenza e di mistero,
	altri a torto condanna. Oh con quant'arte,
	e per che ignote strade egli conduce
	l'uom ad esser beato, e fra le gioie
	del suo amoroso paradiso il pone,
	quando ei pi\`u crede al fondo esser de' mali!
	Ecco, precipitando, Aminta ascende
	al colmo, al sommo d'ogni contentezza.
	Oh fortunato Aminta, oh te felice
	tanto pi\`u, quanto misero pi\`u fosti!
	Or co 'l tuo essempio a me lice sperare,
	quando che sia, che quella bella ed empia,
	che sotto il riso di piet\`a ricopre
	il mortal ferro di sua feritate,
	sani le piaghe mie con piet\`a vera,
	che con finta pietate al cor mi fece.

	\9 Quel che qui viene \`e il saggio Elpino, e parla
	cos\`{\i} d'Aminta come vivo ei fosse,
	chiamandolo felice e fortunato:
	dura condizione degli amanti!
	Forse egli stima fortunato amante
	chi muore, e morto al fin piet\`a ritrova
	nel cor de la sua ninfa; e questo chiama
	paradiso d'Amore, e questo spera.
	Di che lieve merc\'e l'alato Dio
	i suoi servi contenta! Elpin, tu dunque
	in s\`{\i} misero stato sei, che chiami
	fortunata la morte miserabile
	de l'infelice Aminta? e un simil fine
	sortir vorresti? \\

   \8 Amici, state allegri,
	che falso \`e quel romor che a voi pervenne
	de la sua morte. \\

	\9 Oh che ci narri, e quanto
	ci racconsoli! E non \`e dunque il vero
	che si precipitasse? \\

   \8 Anzi \`e pur vero,
	ma fu felice il precipizio, e sotto
	una dolente imagine di morte
	gli rec\`o vita e gioia. Egli or si giace
	nel seno accolto de l'amata ninfa,
	quanto spietata gi\`a, tanto or pietosa;
	e le rasciuga da' begli occhi il pianto
	con la sua bocca. Io a trovar ne vado
	Montano, di lei padre, ed a condurlo
	col\`a dov'essi stanno; e solo il suo
	volere \`e quel che manca, e che prolunga
	il concorde voler d'ambidue loro.

	\9 Pari \`e l'et\`a, la gentilezza \`e pari,
	e concorde il desio; e 'l buon Montano
	vago \`e d'aver nipoti e di munire
	di s\`{\i} dolce presidio la vecchiaia,
	s\`{\i} che far\`a del lor volere il suo.
	Ma tu, deh, Elpin, narra qual dio, qual sorte
	nel periglioso precipizio Aminta
	abbia salvato. \\

   \8 Io son contento: udite,
	udite quel che con quest'occhi ho visto.
	Io era anzi il mio speco, che si giace
	presso la valle, e quasi a pi\`e del colle,
	dove la costa face di s\'e grembo;
	quivi con Tirsi ragionando andava
	pur di colei che ne l'istessa rete
	lui prima, e me dapoi, ravvolse e strinse,
	e proponendo a la sua fuga, al suo
	libero stato, il mio dolce servigio,
	quando ci trasse gli occhi ad alto un grido:
	e 'l veder rovinar un uom dal sommo,
	e 'l vederlo cader sovra una macchia,
	fu tutto un punto. Sporgea fuor del colle,
	poco di sopra a noi, d'erbe e di spini
	e d'altri rami strettamente giunti
	e quasi in un tessuti, un fascio grande.
	Quivi, prima che urtasse in altro luogo,
	a cader venne; e bench'egli co 'l peso
	lo sfondasse, e pi\`u in giuso indi cadesse,
	quasi su' nostri piedi, quel ritegno
	tanto d'impeto tolse a la caduta,
	ch'ella non fu mortal; fu nondimeno
	grave cos\`{\i}, ch'ei giacque un'ora e piue
	stordito affatto e di se stesso fuori.
	Noi muti di pietate e di stupore
	restammo a lo spettacolo improviso,
	riconoscendo lui; ma conoscendo
	ch'egli morto non era, e che non era
	per morir forse, mitighiam l'affanno.
	Allor Tirsi mi di\`e notizia intiera
	de' suoi secreti ed angosciosi amori.
	Ma, mentre procuriam di ravvivarlo
	con diversi argomenti, avendo in tanto
	gi\`a mandato a chiamar Alfesibeo,
	a cui Febo insegn\`o la medica arte,
	allor che diede a me la cetra e 'l plettro,
	sopragiunsero insieme Dafne e Silvia,
	che, come intesi poi, givan cercando
	quel corpo che credean di vita privo.
	Ma, come Silvia il riconobbe, e vide
	le belle guancie tenere d'Aminta
	iscolorite in s\`{\i} leggiadri modi,
	che viola non \`e che impallidisca
	s\`{\i} dolcemente, e lui languir s\`{\i} fatto
	che parea gi\`a negli ultimi sospiri
	essalar l'alma, in guisa di baccante
	gridando e percotendosi il bel petto,
	lasci\`o cadersi in su 'l giacente corpo,
	e giunse viso a viso e bocca a bocca.

	\9 Or non ritenne adunque la vergogna
	lei, ch'\`e tanto severa e schiva tanto?

	\8 La vergogna ritien debile amore:
	ma debil freno \`e di potente amore.
	Poi, s\`{\i} come ne gli occhi avesse un fonte,
	inaffiar cominci\`o co 'l pianto suo
	il colui freddo viso, e fu quell'acqua
	di cotanta virt\`u, ch'egli rivenne;
	e gli occhi aprendo, un doloroso ``ohim\`e''
	spinse dal petto interno;
	ma quell'``ohim\`e'', ch'amaro
	cos\`{\i} dal cor partissi,
	s'incontr\`o ne lo spirto
	de la sua cara Silvia, e fu raccolto
	da la soave bocca, e tutto quivi
	subito raddolcissi.
	Or chi potrebbe dir come in quel punto
	rimanessero entrambi, fatto certo
	ciascun de l'altrui vita, e fatto certo
	Aminta de l'amor de la sua ninfa,
	e vistosi con lei congiunto e stretto?
	Chi \`e servo d'Amor, per s\'e lo stimi.
	Ma non si pu\`o stimar, non che ridire.

	\9 Aminta \`e sano s\`{\i}, ch'egli sia fuori
	del rischio de la vita? \\

   \8 Aminta \`e sano,
	se non ch'alquanto pur graffiat'ha 'l viso,
	ed alquanto dirotta la persona;
	ma sar\`a nulla, ed ei per nulla il tiene.
	Felice lui, che s\`{\i} gran segno ha dato
	d'amore, e de l'amor il dolce or gusta,
	a cui gli affanni scorsi ed i perigli
	fanno soave e dolce condimento;
	ma restate con Dio, ch'io vo' seguire
	il mio viaggio, e ritrovar Montano.

	\9 Non so se il molto amaro,
	che provato ha costui servendo, amando,
	piangendo e disperando,
	raddolcito puot'esser pienamente
	d'alcun dolce presente;
	ma, se pi\`u caro viene
	e pi\`u si gusta dopo 'l male il bene,
	io non ti cheggio, Amore,
	questa beatitudine maggiore;
	bea pur gli altri in tal guisa:
	me la mia ninfa accoglia
	dopo brevi preghiere e servir breve;
	e siano i condimenti
	de le nostre dolcezze
	non s\`{\i} gravi tormenti,
	ma soavi disdegni
	e soavi ripulse,
	risse e guerre a cui segua,
	reintegrando i cori, o pace o tregua.

\newpage
\titulus{EPILOGO}

	\persona{Venere} Scesa dal terzo cielo,
	io che sono di lui regina e dea,
	cerco il mio figlio fuggitivo Amore.
	Quest'ier mentre sedea
	nel mio grembo scherzando,
	o fosse elezion o fosse errore,
	con un suo strale aurato
	mi punse il manco lato,
	e poi fugg\`{\i} da me ratto volando
	per non esser punito;
	n\'e so dove sia gito.
	Io che madre pur sono,
	e son tenera e molle,
	volta l'ira in pietate,
	usat'ho poi per ritrovarlo ogn'arte.
	Cerc'ho tutto il mio cielo in parte in parte,
	e la sfera di Marte, e l'altre rote
	e correnti ed immote;
	n\'e l\`a suso ne' cieli
	\`e luogo alcuno ov'ei s'asconda o celi.
	Tal ch'ora tra voi discendo,
	mansueti mortali,
	dove so che sovente e' fa soggiorno,
	per aver da voi nova
	se 'l fuggitivo mio qua gi\`u si trova.
	N\'e gi\`a trovarlo spero
	tra voi, donne leggiadre,
	perch\'e, se ben d'intorno
	al volto ed a le chiome
	spesso vi scherza e vola,
	e se ben spesso fiede
	le porte di pietate
	ed albergo vi chiede,
	non \`e alcuna di voi che nel suo petto
	dar li voglia ricetto,
	ove sol feritate e sdegno siede.
	Ma ben trovarlo spero
	ne gli uomini cortesi,
	de' qual nessun si sdegna
	d'averlo in sua magione;
	ed a voi mi rivolgo, amica schiera.
	Ditemi, ov'\`e il mio figlio?
	Chi di voi me l'insegna,
	vo' che per guiderdone
	da queste labbra prenda
	un bacio quanto posso
	condirlo pi\`u soave;
	ma chi me 'l riconduce
	dal volontario esiglio,.
	altro premio n'attenda,
	di cui non pu\`o maggiore
	darli, la mia potenza,
	se ben in don li desse
	tutto 'l regno d'Amore;
	e per lo Stige io giuro
	che ferme servar\`o l'alte promesse.
	Ditemi, ov'\`e il mio figlio?
	Ma non risponde alcun: ciascun si tace.
	Non l'avete veduto?
	Forse ch'egli tra voi
	dimora sconosciuto,
	e dagli omeri suoi
	spiccato aver de' l'ali
	e deposto gli strali,
	e la faretra ancor depost'e l'arco,
	onde sempre va carco,
	e gli altri arnesi alteri e trionfali.
	Ma vi dar\`o tai segni
	che conoscer ai segni
	facilmente il potrete,
	ancor che di celarsi a voi s'ingegni.
	Egli, ben che sia vecchio
	e d'astuzia e d'etate,
	picciolo \`e s\`{\i}, ch'ancor fanciuilo sembra
	al viso ed a le membra,
	e 'n guisa di fanciullo
	sempre instabil si move,
	n\'e par che luogo trove in cui s'appaghi,
	ed ha giuoco e trastullo
	di puerili scherzi;
	ma il suo scherzar \`e pieno
	di periglio e di danno.
	Facilmente s'adira,
	facilmente si placa; e nel suo viso
	vedi quasi in un punto
	e le lagrime e 'l riso.
	Crespe ha le chiome e d'oro,
	e 'n quella guisa appunto
	che Fortuna si pinge,
	ha lunghi e folti in su la fronte i crini,
	ma nuda ha poi la testa
	a gli opposti confini.
	Il color del suo volto
	pi\`u che fuoco \`e vivace;
	ne la fronte dimostra
	una lascivia audace;
	gli occhi infiammati e pieni
	d'un ingannevol riso
	volge sovente in biechi; e pur sott'occhio
	quasi di furto mira,
	n\'e mai con dritto guardo i lumi gira.
	Con lingua che dal latte
	par che si discompagni,
	dolcemente favella, ed i suoi detti
	forma tronchi e imperfetti;
	di lusinghe e di vezzi
	\`e pieno il suo parlare,
	e son le voci sue sottili e chiare.
	Ha sempre in bocca il ghigno,
	e gl'inganni e la frode
	sotto quel ghigno asconde,
	come tra fronde e fior angue maligno.
	Questi da prima altrui
	tutto cortese e um\`{\i}le
	a i sembianti ed al volto,
	qual povero peregrin albergo chiede
	per grazia e per mercede;
	ma poi che dentro \`e accolto,
	a poco a poco insuperbisce, e fassi
	oltra modo insolente;
	egli sol vuol le chiavi
	tener de l'altrui core,
	egli scacciarne fuore
	gli antichi albergatori, e 'n quella vece
	ricever nova gente;
	ei far la ragion serva
	e dar legge a la mente:
	cosi divien tiranno
	d'ospite mansueto,
	e persegue ed ancide
	chi li s'oppone e chi li fa divieto.
	Or ch'io v'ho dato i segni
	e degli atti e del viso
	e de' costumi suoi,
	s'egli \`e pur qui fra voi
	datemi, prego, del mio figlio aviso.
	Ma voi non rispondete?
	Forse tenerlo ascoso a me volete?
	Volete, ah folli, ah sciocchi,
	tenere ascoso Amore?
	Ma tosto uscir\`a fuore
	da la lingua e da gli occhi
	per mille, indici aperti:
	tal, io vi rendo certi,
	ch'averr\`a quello a voi, ch'avvenir suole
	a colui che nel seno
	crede nasconder l'angue,
	che co' gridi e co 'l sangue al fin lo scuopre.
	Ma poi che qui no 'l trovo,
	prima ch'al ciel ritorni
   andr\`o cercando in terra altri soggiorni.

\endVersus
\endDrama
\end{document}
