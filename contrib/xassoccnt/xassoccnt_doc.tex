%% LaTeX package xassoccnt - version 1.4 (2017/04/30 -- 00:47:05)
%% Documentation file for xassoccnt.sty
%%
%%
%% -------------------------------------------------------------------------------------------
%% Copyright (c) 2015 -- 2017 by Dr. Christian Hupfer <typography dot with dot latex at gmail dot com>
%% -------------------------------------------------------------------------------------------
%%
%% This work may be distributed and/or modified under the
%% conditions of the LaTeX Project Public License, either version 1.3
%% of this license or (at your option) any later version.
%% The latest version of this license is in
%%   http://www.latex-project.org/lppl.txt
%% and version 1.3 or later is part of all distributions of LaTeX
%% version 2005/12/01 or later.
%%
%%
%% This work has the LPPL maintenance status `author-maintained`
%%
%%

\documentclass[12pt,a4paper,oneside]{article}




\usepackage[lmargin=2cm,rmargin=2cm,headheight=15pt]{geometry}
\usepackage{savesym}
\usepackage{bbding}
\savesymbol{Cross}

\usepackage{graphicx}
\usepackage{blindtext}
\usepackage[x11names]{xcolor}
\usepackage{imakeidx}
\usepackage{fontawesome}
\usepackage[most,documentation]{tcolorbox}
\usepackage[tikz]{bclogo}
\usepackage{marginnote}
\usepackage{fancyhdr}
\usepackage{datetime}
\usepackage{array}
\usepackage{xkeyval}
\usepackage{xparse}
\usepackage{totcount}
\usepackage{enumitem}
\usepackage{microtype}
\usepackage{caption}
\usepackage[T1]{fontenc}
\usepackage[scaled=0.92]{helvet}

\newlist{codeoptionsenum}{enumerate}{1}
\setlist[codeoptionsenum,1]{label={\textcolor{blue}{\#\arabic*}}}

\renewcommand{\rmdefault}{\sfdefault}

\newcolumntype{C}[1]{>{\centering\arraybackslash}p{#1}}

\makeatletter
\define@key{chdoc}{packageauthor}{%
  \def\KVchdocpackageauthor{#1}%
}

\define@key{chdoc}{packageauthormail}{%
  \def\KVchdocpackageauthormail{#1}%
}

\define@key{chdoc}{filepurpose}{%
  \def\KVchdocfilepurpose{#1}%
}


\newcommand{\chdocextractversion}[1]{%
  \@nameuse{#1}%
}


\@namedef{xassoccntversion0.1}{v0.1 2016-11-07}

\@namedef{xassoccntversion0.2}{v0.2 2016-11-14}

\@namedef{xassoccntversion0.3}{v0.3 2016-01-08}

\@namedef{xassoccntversion0.4}{v0.4 2016-01-26}

\@namedef{xassoccntversion0.5}{v0.5 2016-02-27}

\@namedef{xassoccntversion0.6}{v0.6 2016-03-05}

\@namedef{xassoccntversion0.7}{v0.7 2016-05-10}

\@namedef{xassoccntversion0.8}{v0.8 2016-06-10}

\@namedef{xassoccntversion0.9}{v0.9 2016-06-19}

\@namedef{xassoccntversion1.0}{v1.0 2016-07-28}

\@namedef{xassoccntversion1.1}{v1.1 2016-10-29}

\@namedef{xassoccntversion1.2}{v1.2 2017-03-03}

\@namedef{xassoccntversion1.3}{v1.3 2017-03-04}

\@namedef{xassoccntversion1.4}{v1.4 2017-04-07}

\newcommand{\authorname}{Autor}


\makeatother






\fancypagestyle{plain}{%
\fancyfoot[L]{\begin{tabular}[t]{l}\PackageDocName\ \packageversion \tabularnewline \textcopyright\ Dr. Christian Hupfer\end{tabular}}%
\fancyfoot[C]{\thepage}%
\fancyfoot[R]{\today}%
\renewcommand{\headrule}{{\color{blue}%
\hrule width\headwidth height\headrulewidth \vskip-\headrulewidth}}
\renewcommand{\footrule}{{\color{blue}\vskip-\footruleskip\vskip-\footrulewidth
\hrule width\headwidth height\footrulewidth\vskip\footruleskip}}
\renewcommand{\footrulewidth}{2pt}
\renewcommand{\headrulewidth}{2pt}
}



\newtcolorbox{CHPackageTitleBox}[1][]{%
  enhanced jigsaw,
  drop lifted shadow,
  colback=yellow!30!white,
  width=0.8\textwidth,
  #1
}


\presetkeys{chdoc}{packageauthor={Christian Hupfer}}{}%
\NewDocumentCommand{\CHPackageTitlePage}{O{}mO{}}{%
  \setkeys{chdoc}{packageauthor={Christian Hupfer},filepurpose={Documentation},#1}%
  \begin{center}
    \begin{CHPackageTitleBox}[#3]
      \large \bfseries%
      \begin{center}%
        \begin{tabular}{C{0.9\textwidth}}%
          \scshape \PackageDocName \tabularnewline
          \tabularnewline
          #2 \tabularnewline
          \tabularnewline
          \KVchdocfilepurpose \tabularnewline
          \tabularnewline
          Version \packageversion \tabularnewline
          \tabularnewline
          \today \tabularnewline
          \tabularnewline
          \addtocounter{footnote}{2}
          \authorname: \KVchdocpackageauthor\(^\mathrm{\fnsymbol{footnote}}\)
          \tabularnewline
        \end{tabular}
      \end{center}
    \end{CHPackageTitleBox}
    \renewcommand{\thefootnote}{\fnsymbol{footnote}}%
    \footnotetext{\mymailtoaddress}%
  \end{center}
}

\newtcolorbox{docCommandArgs}[1]{colbacktitle={blue},coltitle={white},title={Description of arguments of command \cs{#1}}}


\newcommand{\tcolorboxdoclink}{http://mirrors.ctan.org/macros/latex/contrib/tcolorbox/tcolorbox.pdf}

% 'Stolen' from tcolorbox documentation ;-)

\newtcolorbox{marker}[1][]{enhanced,
  before skip=2mm,after skip=3mm,
  boxrule=0.4pt,left=5mm,right=2mm,top=1mm,bottom=1mm,
  colback=yellow!50,
  colframe=yellow!20!black,
  sharp corners,rounded corners=southeast,arc is angular,arc=3mm,
  underlay={%
    \path[fill=tcbcol@back!80!black] ([yshift=3mm]interior.south east)--++(-0.4,-0.1)--++(0.1,-0.2);
    \path[draw=tcbcol@frame,shorten <=-0.05mm,shorten >=-0.05mm] ([yshift=3mm]interior.south east)--++(-0.4,-0.1)--++(0.1,-0.2);
    \path[fill=yellow!50!black,draw=none] (interior.south west) rectangle node[white]{\Huge\bfseries !} ([xshift=4mm]interior.north west);
    },
    drop fuzzy shadow,#1}


%%%% Documentation macros


\NewDocumentCommand{\packagename}{sm}{%
  \textcolor{blue}{\textbf{\faEnvelopeO~#2}}%
  \IfBooleanF{#1}{%
    \index{Package!#2}
  }%
}

\NewDocumentCommand{\classname}{sm}{%
  \textcolor{brown}{\textbf{\faBriefcase~#2}}%
  \IfBooleanF{#1}{%
    \index{Package!#2}%
  }%
}


\NewDocumentCommand{\CHDocPackage}{sm}{%
  \textcolor{blue}{\textbf{\faEnvelopeO~#2}}%
  \IfBooleanF{#1}{%
    \index{Package!#2}
  }%
}




\NewDocumentCommand{\CHDocClass}{sm}{%
  \textcolor{brown}{\textbf{\faBriefcase~#2}}%
  \IfBooleanF{#1}{%
    \index{Package!#2}%
  }%
}

\NewDocumentCommand{\CHDocKey}{sm}{%
  \textcolor{red}{\textbf{\faKey~#2}}%
  \IfBooleanF{#1}{%
      \index{Option!#2}%
  }%
}

\newcommand{\handrightnote}{\tcbdocmarginnote{\ding{43}}}


\NewDocumentCommand{\CHDocCounter}{sm}{%
  \textcolor{Green4}{\textbf{\faCalculator~#2}}%
  \IfBooleanF{#1}{%
    \index{Counter!#2}%
  }%	
}


\NewDocumentCommand{\CHDocTag}{sm}{%
  \textcolor{violet}{\faTag~#2}%
  \IfBooleanF{#1}{%
    \index{Feature!#2}%
  }%	
}


\NewDocumentCommand{\CHDocFileExt}{sm}{%
    \faFile~#2%
}

\NewDocumentCommand{\CHDocFiles}{sm}{%
    \faFilesO~#2%
}


\NewDocumentCommand{\CHDocConventions}{}{%
  \section*{\centering Typographical conventions}
  Throughout this documentation following symbols and conventions are used:
  \begin{itemize}
  \item \CHDocClass*{foo} means a the class \texttt{foo}
  \item \CHDocPackage*{foo} names a package \texttt{foo}
  \item \CHDocCounter*{foo} indicates a counter named \texttt{foo}
  \item \CHDocFileExt*{foo} will indicate either a file named \texttt{foo} or a file extension \texttt{foo}
  \item \CHDocFiles*{foo} will indicate some files 
  \item \CHDocTag*{foo} names a special feature or tag \texttt{foo}
  \item \CHDocKey*{foo} deals with a command or package option named \texttt{foo}
  \end{itemize}
}



\renewcommand{\tcbdocnew}[1]{#1}%
\renewcommand{\tcbdocupdated}[1]{#1}%

\newcommand{\CHDocNew}[1]{%
  \tcbdocmarginnote[doclang/new={N},
  colframe=blue,
  halign=left,
  colback={blue!20!white},
  fontupper={\tiny}
  ]{%
    \chdocextractversion{xassoccntversion#1}%
  }%
}



\newcommand{\CHDocUpdate}[1]{\tcbdocmarginnote[doclang/updated={},colback={yellow},colframe={yellow!50!red},  fontupper={\tiny}
]{%
  \tcbdocupdated{\chdocextractversion{xassoccntversion#1}}%
}%
}



\newcommand{\CHDocFullVersion}[1]{Version \chdocextractversion{xassoccntversion#1}}


\newcommand{\CHDocExpCommand}[1][Expandable]{%
  \tcbdocmarginnote[doclang/new={N},
  colframe=green!50!blue,
  halign=left,
  colback={green!90!blue},
  fontupper={\tiny}
  ]{%
    #1%
  }%
}


\newcommand{\CHDocExperimentalFeature}[1][Experimental]{%
  \tcbdocmarginnote[doclang/new={N},
  colframe=yellow!50!blue,
  halign=left,
  colback={blue!10!yellow},
  fontupper={\tiny}
  ]{%
    #1%
  }%
}


\usepackage[T1]{fontenc}
\usepackage{graphicx}%
\usepackage[autodefinecounters=true]{xassoccnt}
\usepackage{amsmath}
\usepackage{amssymb}
\usepackage{minitoc}
\usepackage{tocbibind}
\usepackage{bookmark}
\usepackage{cleveref}
\usetikzlibrary{mindmap}
\usepackage{url}

\setcounter{tocdepth}{3}
\newcommand{\PackageDocName}{xassoccnt}%


\newcommand{\mymailtoaddress}{%
  typography.with.latex@gmail.com%
}

\doparttoc

\let\DeclareAssociatedCountersOrig\DeclareAssociatedCounters % -> for use in examples only

\def\packageversion{\xassoccntpackageversion}%

\makeindex[intoc]


\renewcommand{\rmdefault}{\sfdefault}


\newcounter{foobar}

\newtotcounter{totalpages}%
\newtotcounter{anothertotalpages}%

\newtotcounter{totalproofs}%
\newtotcounter{totalsections}%
\newtotcounter{totalequations}

\newtotcounter{totalsubsections}%
\newtotcounter{othertotalsubsections}%

\DeclareDocumentCounter{foocntr}%
\DeclareDocumentCounter{foobarcntr}%


%%%% Declare all the counters here -- 
\DeclareAssociatedCounters{subsection}{totalsubsections}%
\DeclareAssociatedCounters{equation}{totalequations}%
\DeclareAssociatedCounters{section}{totalsections}%
\DeclareAssociatedCounters{proof}{totalproofs}
\DeclareAssociatedCounters{page}{totalpages,anothertotalpages}%

\DeclareTotalDocumentCounter{foototal}%

\RegisterTotalDocumentCounter{section}%

\hypersetup{breaklinks=true,
  pdftitle={\jobname.pdf -- version \packageversion},
  pdfauthor={PACKAGEAUTHOR},
  pdfsubject={Documentation of \PackageDocName\ package},
  pdfkeywords={LaTeX, counters},
  bookmarksopen=true,
  bookmarksopenlevel=2,
  bookmarksnumbered=true,
  pdfcreator={LaTeX}
}



\begin{document}
\yyyymmdddate


\setlength{\parindent}{0em}

\pagestyle{empty}%

\begin{tikzpicture}[mindmap,concept color=red,black,scale=1]
\bfseries
  \node [concept] {Counters}
  child[concept color=red!30!yellow,grow=30] { node [concept] {Associated counters}}
  child[grow=85,white,concept color=red!80!blue] { node [concept] {Backup \& Restore of counters}}
  child[grow=300,concept color=yellow!90!red] { node [concept] {Extensions to \LaTeXe} child[grow=230,concept color={yellow!20!green}] {node [concept] {Query macros}}
    child[grow=290,concept color={yellow!30!cyan}] {node [concept] {Document counters}}
}
  child[grow=-20,white] { node [concept] {Periodically resetting counters}}
  child[grow=130,concept color=red!20!green] { node [concept] {Suspension \\ \& Resuming of counters} child[grow=160,concept color=green!10!yellow] {node[concept] {Cascaded suspension}}}
  child[grow=170,concept color=red!30!yellow] { node [concept] {Total} child[concept color=yellow!30!,grow=150] {node[concept] {Super total counters}}}
  child[grow=260,concept color=red!30!green] { node [concept] {Loops on counters}}
  child[grow=200,concept color=red!30!green] { node [concept] {Counter output}}
  child[grow=230,concept color=red!20!cyan] { node [concept] {Coupled counters} child [grow=250,concept color=cyan!50!blue] {node[concept] {Coupled counter groups}}}
  child[grow=57,concept color=red!20!yellow] { node [concept] {Label}};
\end{tikzpicture}

\CHPackageTitlePage[packageauthor={Christian Hupfer}]{Associated counters stepping simultaneously\\ (and other gadgets)}


\clearpage
\tableofcontents
\clearpage

\CHDocConventions
\clearpage


\pagestyle{plain}


\setcounter{footnote}{0}

\part{Introduction}

\parttoc

\section*{Preface}

This package is the successor and a complete rewrite of \CHDocPackage{assoccnt}. Not all features of that package are implemented yet -- if some functionality of your document depends on \CHDocPackage{assoccnt}, continue using the older version and shift gradually to \CHDocPackage{\PackageDocName} please.

\begin{marker}
Most times class and package authors will benefit of this package, but there might be usual documents that need the features of |\PackageDocName||
\end{marker}

\section{Introduction}

The aim of this package is to provide some additional support for example for a package like \CHDocPackage{totcount}. 

For example, the total number of pages in a document could be achieved by using

\begin{dispExample}%
\regtotcounter{page}
...
The number of pages in the document is \number\totvalue{page} page(s) -- but in fact it has \total{totalpages} pages. 
\end{dispExample}%

This will work, as long there is no reset of the page counter, as it might happen in the case of \cs{setcounter} or  \cs{pagenumbering} being applied in the document. The result is a false page counter total value. 

This package provides associate counters, i.e. counters that are increased simultaneously with a driver counter and are not influenced by a a resetting of the driver counter, as long as not being added to the reset list by definition of the counter or explicitly by \cs{@addtoreset}. 

This package defines some macros to handle associated counters. The only interception to the standard behaviour is within the redefined commands \cs{addtocounter} and \cs{stepcounter}. The usual commands still work, as there is code added to their definition. In a previous version \cs{refstepcounter} was redefined, but since these use \cs{addtocounter} effectively, it was decided to use the basic command. 

Internally, the associated counters are stored in one list per counter -- it is not recommended to operate on those lists directly. 

Please note that this package does not provide means for simultaneous stepping of counters defined by plain \TeX{} \cs{newcount} command.\marginnote{\bcbombe}




\section{Requirements, loading and incompatibilities}%



\subsection{Required packages and \TeX\ engine}

The package does not require features from Xe\LaTeX\ or Lua\LaTeX\ but can be run with those features as well as with \LaTeX\ or pdf\LaTeX. The compilation documentation requires however pdf\LaTeX\ as of version \packageversion. 

\begin{itemize}
\item \CHDocPackage{xcolor}  
\item \CHDocPackage{xparse}
\item \CHDocPackage{l3keys2e}
\end{itemize}

The documentation file requires some more packages such as \CHDocPackage{tcolorbox} but those packages are widely available on CTAN, MikTeX and TeXLive as well.

\subsection{Loading of the package}

Loading is done with

\begin{tcblisting}{listing only}
\usepackage[options]{xassoccnt}
\end{tcblisting}

For the relevant options see \cref{subsection:packageoptions}

Concerning the packages \CHDocPackage{hyperref} and \CHDocPackage{cleveref}: The preferred loading order is the usually recommended on: 
\begin{itemize}
  \item other packages
  \item \CHDocPackage{xassoccnt}
  \item \CHDocPackage{hyperref}
  \item \CHDocPackage{cleveref}
\end{itemize}

For potential problems see \cref{subsection:incompatibilities}

\subsection{Incompatibilities}\label{subsection:incompatibilities}



\begin{itemize}
\item This package cannot be used with plain \TeX\ and will not provide support for counters (or better counter registers) that have defined with the \TeX\ primitive \cs{countdef} directly in a \LaTeXe\ document nor will it hook into plain \TeX\ \cs{advance} commands used otherwise than in the usual \LaTeXe\ wrappers \cs{addtocounter} etc.

\item This package does not work really well with the \CHDocPackage{calc} package if that one is loaded after \CHDocPackage{\PackageDocName}. Load \CHDocPackage{calc} \textbf{before} \CHDocPackage{\PackageDocName}! 

Especially the \CHDocPackage{tcolorbox} bundle loads \CHDocPackage{calc} indirectly, so placing any loading of \CHDocPackage{tcolorbox} modules before \CHDocPackage{\PackageDocName} is mandatory!


\begin{marker}
\tcbdocmarginnote{\bcbombe\tcbdocupdated{2015-11-10}}
  As of version \packageversion\ \CHDocPackage{\PackageDocName} will abort compilation if \CHDocPackage{calc} is loaded after this package, but will issue a warning only if \CHDocPackage{calc} is loaded first. 
\end{marker}


\begin{marker}
\CHDocUpdate{0.7}
Of course any package other that loads \CHDocPackage{calc} must be loaded before \PackageDocName, i.e. \CHDocPackage{mathtools}. 
\end{marker}

\item \CHDocPackage{\PackageDocName} and \CHDocPackage{perpage} are not compatible completely. As of version \packageversion\ it is not possible to use the command \cs{AddAbsoluteCounter} from \CHDocPackage{perpage} -- this feature is provided already by this package.
\tcbdocmarginnote{\tcbdocnew{2015-11-10}}

\begin{marker}
  It is not recommended to use counters under control of \CHDocPackage{perpage} with the association method! 
\end{marker}

% is somewhat similar to \refCom{AddAssociatedCounters} from this package but unless changed manually, the values of and \texttt{perpage} - absolute counter and an associated counter by this package differ by one, since \cs{AddAbsoluteCounter} initializes the counter value to the value of 1.
%\item This package does not work together when the Package \CHDocPackage{xifthen} is included. As of version \packageversion~I could not figure out, where the strange behaviour comes in. It's most likely an incompatibility between \CHDocPackage{etoolbox} and \CHDocPackage{xifthen}
%\marginnote{\bcbombe}
\end{itemize}

\subsection{Package options}\label{subsection:packageoptions}

As of version \packageversion\ \CHDocPackage{\PackageDocName} supports the package options

\begin{docKey}{autodefinecounters}{=\meta{true,false}}{initially false}
  Setting this package option to \texttt{true}, all counters used with the special package macros will be autodefined, unless disabled locally. See the commands 
\begin{itemize}
\item \refCom{DeclareAssociatedCounters} 
\item \refCom{AddAssociatedCounters}
\end{itemize}
\end{docKey}

\begin{docKey}{nonumberofruns}{}{initially not set}\CHDocNew{0.6}
  Using this package option the super total counter \CHDocCounter{numberofruns} counter will not be defined. See \cref{subsection:numberofruns} about this feature. 
\end{docKey}

\begin{docKey}{redefinelabel}{=\meta{true,false}}{initially \meta{true}}\CHDocNew{1.2}
\CHDocExperimentalFeature This is an experimental feature as of version \packageversion. 

Enables the redefinition of the \refCom{label} command and takes care of the optional argument of \refCom{label} if \CHDocPackage{cleveref} is used. This will provide \refCom{label} with a final optional argument that can be used to allow labels for associated counters (either all or only a specified list of counters), for more on this see \refCom{label} in \namecref{subsec::associated_counters_experimental} \nameref{subsec::associated_counters_experimental}. 
\end{docKey}

\begin{docKey}{standardcounterformats}{=\meta{choice=on/off}}{initially \meta{on}}\CHDocNew{1.4}
This key enables (\meta{on}) or disables (\meta{off}) the definition of the standard counter formats to be used in the macro \refCom{CounterFormat}. 
\end{docKey}
\clearpage
\part{Tools for counters}

\parttoc


\section[Additions to standard commands]{Additions and extensions to standard counter related commands}\label{section::additions_to_l2e}

\subsection{Extension of \LaTeXe\ commands}\label{subsection::extension_of_l2e_commands}

\begin{docCommand}[before={\CHDocNew{0.9}}]{addtocounter}{\marg{counter}\marg{increment value}\oarg{options}}
  The \refCom{addtocounter} macro behaves like the usual \cs{addtocounter} counter, but takes care to specific counter features such as \CHDocTag{periodic counters} and has an optional argument in order to perform special settings. 

As of \packageversion, there is only one option used:
\begin{docKey}[][after={\CHDocNew{0.9}}]{wrap}{=\meta{true/false}}{initially true}

  This key determines whether addition of values to a periodic counter (see \nameref{section:periodic_counters}) will lead to a modulo part addition. 
\end{docKey}

\end{docCommand}
  

\subsection[\LaTeXe\ additions]{Additions to \LaTeXe\ commands}

\begin{docCommand}{NewDocumentCounter}{\oarg{options}\marg{counter1,counter2,\dots}\oarg{resetting counter}}
  This command is a new interface to \cs{newcounter} and behaves effectively the same. %The first optional argument is reserved for later purposes but not used as of version \packageversion. 
  
    \begin{docKey}{initial}{=\meta{integer value}}{initially 0}
      This is used for the start value of the new counter. 
    \end{docKey}

\CHDocUpdate{1.2}This command allows multiple counters (specified as a comma separated list) to be defined at once, but all have the same resetting counter then and initial value is the same for all those counters (if those options are given).

\end{docCommand}%

\begin{docCommand}{DeclareDocumentCounter}{\oarg{options}\marg{counter}\oarg{resetting counter}}
  This command is the preamble-only version of \refCom{NewDocumentCounter}. 
  \CHDocUpdate{1.2}This command allows multiple counters (specified as a comma separated list) to be defined at once, but all have the same resetting counter then and initial value is the same for all those counters (if those options are given), i.e. the behaviour is like in \refCom{NewDocumentCounter}.
\end{docCommand}%


\begin{docCommand}{SetDocumentCounter}{\oarg{options}\marg{counter}\marg{counter value}}

This command behaves like the standard macro \cs{setcounter}, but has an additional optional 1st argument. %which is not used as of version \packageversion.
% This optional argument can be used to enable the setting of the driver counter value as well as the associated counter values at once.

\begin{docCommandArgs}{SetDocumentCounter}%


\begin{codeoptionsenum}
  \item \oarg{options}: %As of version \packageversion, this option is not used% the key value argument

    \begin{docKey}{associatedtoo}{=\meta{true/false}}{initially false}
      If enabled (\meta{true}), \refCom{SetDocumentCounter} will use the counter value for \underline{all} counters associated to this driver counter as well. Initially, this option is set to \meta{false}. 
    \end{docKey}
    
    \begin{docKey}{onlycounters}{=\meta{comma separated list of counters}}{initially empty}
      If this key is used, only those associated counters are set as well that are given in the comma separated list. 
      
      Names, that are either not referring to counters at all or to counters that are not associated to the given driver counter will be ignored silently. 
    \end{docKey}

   



  \item \marg{counter} 
    Holds the name of the (driver) counter to be set. 
  \item \marg{counter value}
    Holds the value to be set 
  \end{codeoptionsenum}
\end{docCommandArgs}

Some notes on \refCom{SetDocumentCounter}
\begin{itemize}
    \item The option keys \refKey{associatedtoo} and \refKey{onlycounters} are mutually exclusive! %
    \item The counter to be set can be either a driver counter or an otherwise associated counter. 
\end{itemize}
\end{docCommand}%

\begin{docCommand}{StepDownCounter}{\oarg{options}\marg{counter}}\CHDocNew{0.4}

This macro subtracts the value of 1 from the counter and is the counterpart of \cs{stepcounter}. 

\begin{docCommandArgs}{StepDownCounter}%


\begin{codeoptionsenum}
  \item \oarg{options}: As of version \packageversion, this option is not used% the key value argument
  \item \marg{counter} 
    Holds the name of the first counter.
  \end{codeoptionsenum}
\end{docCommandArgs}

\end{docCommand}%


\begin{docCommand}{SubtractFromCounter}{\oarg{options}\marg{counter}\marg{delta value}}\CHDocNew{0.4}

This macro subtracts the (positive) delta value from the counter and is the counterpart of \cs{addtocounter}

\begin{docCommandArgs}{SubtractFromCounter}%


\begin{codeoptionsenum}
  \item \oarg{options}: As of version \packageversion, this option is not used% the key value argument

  \item \marg{counter 1} 
    Holds the name of the first counter.
  \item \marg{delta value}
    Holds the (positive) value to be subtracted from the counter value. 
  \end{codeoptionsenum}
\end{docCommandArgs}

\end{docCommand}%


\begin{docCommand}{CopyDocumentCounters}{\oarg{options}\marg{source counter}\marg{target counter}}

This document copies the counter value from the source counter in argument 2 to the target counter in argument 3.

\begin{docCommandArgs}{CopyDocumentCounters}%


\begin{codeoptionsenum}
  \item \oarg{options}: As of version \packageversion, this option is not used% the key value argument

  \item \marg{source counter} 
    Holds the name of the source counter.
  \item \marg{target counter}
    Holds the name of the target counter.
  \end{codeoptionsenum}
\end{docCommandArgs}

\end{docCommand}%


\begin{docCommand}{SwapDocumentCounters}{\oarg{options}\marg{counter 1}\marg{counter 2}}

This macro swaps the values of the counters given in arguments 2 and 3

\begin{docCommandArgs}{SwapDocumentCounters}%


\begin{codeoptionsenum}
  \item \oarg{options}: As of version \packageversion, this option is not used% the key value argument

  \item \marg{counter 1} 
    Holds the name of the first counter.
  \item \marg{counter 2}
    Holds the name of the second counter.
  \end{codeoptionsenum}
\end{docCommandArgs}

\end{docCommand}%


\begin{docCommand}{SyncCounters}{\oarg{options}\marg{driver counter}}

This document synchronizes the driver counter value to the associated values. It has the same options as \refCom{SetDocumentCounter}. If the given counter is no driver counter, nothing is done. 

\begin{docCommandArgs}{SyncCounters}%


\begin{codeoptionsenum}
  \item \oarg{options}: see \refCom{SetDocumentCounter}

  \item \marg{source counter} 
    Holds the name of the source counter.
  \end{codeoptionsenum}
\end{docCommandArgs}

\end{docCommand}%

\begin{dispExample}%[breakable=true]
  \SetDocumentCounter{foocntr}{17}
  \SetDocumentCounter{foobarcntr}{20}

\begin{itemize}
\item Displaying counters

  \thefoocntr\ and \thefoobarcntr
\item Swapping counters

  \SwapDocumentCounters{foocntr}{foobarcntr}

  \thefoocntr\ and \thefoobarcntr

\item Step down counters

\StepDownCounter{foocntr}
\StepDownCounter{foobarcntr}

  \thefoocntr\ and \thefoobarcntr

\item Subtracting some value from the counters
  \SubtractFromCounter{foocntr}{5}
  \SubtractFromCounter{foobarcntr}{10}

  \thefoocntr\ and \thefoobarcntr
\end{itemize}
\end{dispExample}


\subsection[\protect\cs{IfIsDocumentCounter}-Queries]{Commands checking whether a name refers to a counter}

\CHDocPackage{\PackageDocName}\ provides three commands that are quite similar -- all check whether \marg{name} is an already defined \LaTeXe\ counter (name), in good tradition with the \CHDocPackage{xparse} - syntax:

\begin{itemize}
\item \begin{docCommand}{IfIsDocumentCounterTF}{\oarg{}\marg{name}\marg{true branch}\marg{false branch}}
This macro performs the full branching 
\end{docCommand}
\item 

\begin{docCommand}{IfIsDocumentCounterT}{\oarg{}\marg{name}\marg{\meta{long} true branch}}
This command executes only if the name is a counter. 
\end{docCommand}

\begin{docCommand}{IfIsDocumentCounterF}{\oarg{}\marg{name}\marg{true branch}}
This command executes only if the name is not a counter. 
\end{docCommand}

\end{itemize}

The optional argument is not used as of version \packageversion\ for none of those three commands. 


\subsection[Information macros]{Information on counters} \label{subsection::informationcmds}

On occasions it might be important to have some information which counter has been changed last. Since there are four commands manipulating counter values, there are four corresponding routines for this:

\begin{docCommand}[before={\CHDocExpCommand\par\CHDocUpdate{1.3}}]{LastAddedToCounter}{}
This command has no arguments and expands to the name of the counter which was used last in \cs{addtocounter}. There is no further typesetting done with the countername. 

\begin{dispExample}
  \newcounter{SomeCounter}
  
  \addtocounter{SomeCounter}{10}

  The last counter something added to was \LastAddedToCounter.
\end{dispExample}%
\end{docCommand}%

\begin{marker}
  Please note that \refCom{LastAddedToCounter} might fail! 
\end{marker}


\begin{docCommand}[before={\CHDocExpCommand\par\CHDocUpdate{1.3}}]{LastSteppedCounter}{}

This command has no arguments and expands to the name of the counter which was stepped last using \cs{stepcounter}. There is no further typesetting done with the countername. 


\begin{dispExample}
  \stepcounter{SomeCounter}
  The last counter being stepped  was \LastSteppedCounter.
\end{dispExample}%

\end{docCommand}%

\begin{docCommand}[before={\CHDocExpCommand\par\CHDocUpdate{1.3}}]{LastRefSteppedCounter}{}

This macro gives the last counter being used in \cs{refstepcounter} and is expandable.%{\ChDocVersion{1.3}}

\begin{dispExample}
  \begin{equation}
    E = mc^2 \label{eq::einstein}
  \end{equation}
  % \stepcounter{SomeCounter}

  The last counter being refstepped  was \LastRefSteppedCounter.
\end{dispExample}%


\end{docCommand}%





\begin{docCommand}[before={\CHDocExpCommand\par\CHDocUpdate{1.3}}]{LastSetCounter}{}
This command has no arguments and expands to the name of the counter which was set last using \cs{setcounter}. There is no further typesetting done with the countername. 

\begin{dispExample}
  \setcounter{SomeCounter}{21}%

  The last counter being set  was \LastSetCounter.
\end{dispExample}%

\end{docCommand}%


\begin{docCommand}{LastCounterValue}{}
This command has no arguments and expands to the value of the very last change of a counter, i.e. using \cs{setcounter} etc.

\begin{dispExample}
    \setcounter{SomeCounter}{100}%

    The last counter being set  was \LastSetCounter and it had the value \LastCounterValue{} then, where as \stepcounter{equation} will yield \fbox{\LastSteppedCounter} and \LastCounterValue!
\end{dispExample}%

The usage of \refCom{LastSetCounter} is best together with one of the other \cs{Last...} macros. 

\end{docCommand}%

\begin{marker}
All of the \cs{Last...} macros are expandable, i.e. it is possible to store the value to an macro defined with \cs{edef}
\end{marker}

\begin{dispExample}
    \setcounter{SomeCounter}{50}%

    \edef\lastcounterset{\LastSetCounter}
    \edef\lastcountervalue{\LastCounterValue}
   
    \setcounter{equation}{81}%


    The last counter being set was \fbox{\LastSetCounter} and it had the value \LastCounterValue{} then, but we changed \lastcounterset{} earlier and it had the value \lastcountervalue{} then.
\end{dispExample}%

\setcounter{equation}{1}




\begin{marker}
Please note, that all of this commands are only working in the current run of compilation, i.e. \underline{after} there has been some operation on the counters. They can't be used for information on the last changed counter in a previous run. 
\end{marker}





\section{Counter reset lists}

The package \CHDocPackage{chngcntr} offers the possibility of add or remove counters to the reset list of a driver counter with the commands \cs{counterwithin} and \cs{counterwithout}, whereas the package \CHDocPackage{remreset} provides \cs{@removefromreset} as a counterpart to the \LaTeXe\ core command \cs{@addtoreset} macro. 

\subsection[Addition and Removal]{Addition and Removal of counters from the reset list}\CHDocNew{1.0}

\begin{docCommand}[before={\CHDocNew{1.0}\CHDocUpdate{1.4}}\par]{RemoveFromReset}{\marg{counter name1, counter name2,\dots}\marg{driver counter name}}
This macro removes the counters given in the comma separated list in the first argument from the reset list of the driver counter given in the 2nd argument.

If the 2nd argument does not point to a \LaTeXe\ counter name an error message is shipped and the compilation fails. 
\end{docCommand}


\begin{docCommand}[before={\CHDocNew{1.0},\CHDocUpdate{1.4}}\par]{RemoveFromFullReset}{\marg{counter name1, counter name2,\dots}\marg{driver counter name}}
This macro removes the counters given in the comma separated list in the first argument and all of its own reset list from the reset list of the driver counter given in the 2nd argument.

If the 2nd argument does not point to a \LaTeXe\ counter name an error message is shipped and the compilation fails. 
\end{docCommand}

\begin{docCommand}[before={\CHDocNew{1.4}}]{ClearCounterResetList}{\marg{driver counter name}}
This macro removes all counters of the given driver counter reset list. The individual counter formatting macros \cs{theX} are reset both for the driver counter as well as the counters in the reset list to use the \cs{arabic} standard output macro. \texttt{X} means some arbitray \LaTeX2e\ counter name.  

If the resetting shall not be applied, use \refCom{ClearCounterResetList*} instead.
\end{docCommand}

\begin{docCommand}[before={\CHDocNew{1.4}}]{ClearCounterResetList*}{\marg{driver counter name}} 
    This behaves like \refCom{ClearCounterResetList} but does \textbf{not} reset the relevant \cs{theX} macros.
\end{docCommand}


\begin{docCommand}[before={\CHDocNew{1.0}\CHDocUpdate{1.4}}]{AddToReset}{\marg{counter name1, counter name2,\dots}\marg{driver counter name}}
This macro adds the counters given in the comma separated list in the first argument to the reset list of the driver counter given in the 2nd argument.

If the 2nd argument does not point to a \LaTeXe\ counter name an error message is shipped and the compilation fails. 

An accidental specificiation of the driver counter to be added to its own reset list is ignored internally. 
\end{docCommand}


\subsection[Information macros about the reset list]{Information macros about the counter reset list}

Sometimes it might be necessary or convenient to know how many counters are on a reset list of some other counters, i.e. added by \cs{newcounter}\textbraceleft counter\textbraceright[resetting counter] or \refCom{NewDocumentCounter}. 

There are some macros that provide this information:

\begin{docCommand}{countersresetlistcount}{\marg{counter name}}
This macro determines the number of counters being in the reset list of the counter specified as mandatory argument. 

Please note: This command isn't expandable. The number is stored internally to another macro, which can be accessed with \refCom{getresetlistcount}, which returns a pure integer number. 
\end{docCommand}

\begin{docCommand}{getresetlistcount}{}
This macro returns the number of counters being in the reset list of the counter specified as mandatory argument. It needs a previous call of \refCom{countersresetlistcount} first!

If the counter has no other counters in its reset list, the value of 0 is returned. 
\end{docCommand}


\begin{docCommand}[before={\CHDocNew{1.0}}\par]{CounterFullResetList}{\marg{counter name}}
  This macro determines the full reset list of a counter as well of the counters being on the reset list, i.e. the list is tracked down until there are no counters left in a recursion. 
  
  The counter names are stored internally in \CHDocPackage{expl3} - \cs{seq} - variable named \cs{xy\_fullresetlist\_seq} -- the \meta{xy} is replaced by the counter name, e.g. if the counter is named \CHDocCounter*{foo}, the identifier would be \cs{foo\_fullresetlist\_seq}. Unless \CHDocTag{expl3} features are not applied, the \refCom{CounterFullResetList} is not really useful on a document or package/class developing level. However, to loop through the full reset list with some action performed on the members of the sequence, the command \refCom{LoopFullCounterResetList} may be very useful. 

\begin{marker}
\begin{itemize}
\item The driver counter \CHDocCounter*{foo} is not added to the relevant sequence. 
\item If the name given to \refCom{CounterFullResetList} does not indicate a \LaTeXe\ counter an error message is shipped and the compilation fails. 
\end{itemize}
\end{marker}
\end{docCommand}

\begin{docCommand}{IfInResetListTF}{\oarg{}\marg{resetting counter}\marg{reset counter}\marg{true branch}\marg{false branch}}
This command sequence tests whether the counter \meta{reset counter} is in the reset list of \meta{resetting counter} and expands the relevant branch then.
See the short-circuit commands \refCom{IfInResetListT} and \refCom{IfInResetListF} as well. 
\end{docCommand}

\begin{docCommand}{IfInResetListT}{\oarg{}\marg{resetting counter}\marg{reset counter}\marg{true branch}}
This command sequence tests whether the counter \meta{reset counter} is in the reset list of \meta{resetting counter} and expands to the true branch.
See the related commands \refCom{IfInResetListTF} and \refCom{IfInResetListF} as well. 
\end{docCommand}

\begin{docCommand}{IfInResetListF}{\oarg{}\marg{resetting counter}\marg{reset counter}\marg{false branch}}
This command sequence tests whether the counter \meta{reset counter} is not in the reset list of \meta{resetting counter} and expands to the false branch.
See the related commands \refCom{IfInResetListTF} and \refCom{IfInResetListT} as well. 
\end{docCommand}

\begin{docCommand}{DisplayResetList}{\oarg{separator={,}}\marg{resetting counter}}\CHDocNew{0.8}

This command displays the reset list of a counter as a separated list. If the counter has no resetting list, nothing is shown. 

\begin{docCommandArgs}{DisplayResetList}%

\begin{codeoptionsenum}
\item \oarg{separator}% 
  This separator is used for display, it defaults to a comma character. 

\item \marg{resetting counter}%

  Contains the name of counter whose resetting list should be displayed. 
\end{codeoptionsenum}
\end{docCommandArgs}

\end{docCommand}

\begin{docCommand}{ShowResetList}{\marg{resetting counter}}\CHDocNew{0.8}
This command displays the reset list of a counter on the terminal as the \cs{show} command would do. This is rather useful for debugging purposes only. 
\end{docCommand}


\begin{docCommand}[before={\CHDocNew{1.3}}]{GetAllResetLists}{}
  This determines all reset lists and stores the information internally. It should be called right before \cs{begin{document}} or at any time inside the document environment, when new counters are added there (which is not recommended)

The information can be retrieved with \refCom{GetParentCounter}. 

\end{docCommand}

\begin{docCommand}[before={\CHDocExpCommand\par\CHDocNew{1.3}}]{GetParentCounter}{\marg{counter}}
This macro tries to detect the counter that was responsible for the resetting of the counter named \marg{counter} and is expandable. 
In order to minimize the amount of searching and maintaining expandability, the counter reset data must be stored beforehand, i.e. with \refCom{GetAllResetLists}. 

\begin{marker} 
If a counter has been added to more than one parent counter as their resetting driver counter, only the most recent addition is in action. This may be correct in some occasions but there is no guarantee that the given counter name really caused the last reset of the counter given as argument. 
\end{marker}
\end{docCommand}



\section[Loops on multiple counters]{Performing the same action for many counters} \CHDocNew{0.7}

Sometimes it might be necessary to set the values of many counters at once. This can be done with consecutive \cs{setcounter} statements, for example. This poses no problem, but might become tedious if there are more than three counters or if this task occurs more than once. \CHDocPackage{\PackageDocName} provides some macros that can do the usual operations like stepping, refstepping, adding to, resetting or setting counter values. 


All macros concerning this feature use the first macro argument having a comma-separated list of counters. Whether there's a second argument depends on the specific nature of the operation that should be performed. 

\begin{marker}
\begin{itemize}
\item As of version \packageversion\ \PackageDocName\ does not check whether the names given in the first argument refer to counters. 
\item All macros use the extended counter macros, i.e. are aware of associated counters and step them too if their driver counter is given in the argument list. If an associated counter itself is given in the list, this one is stepped or operated on too!
\end{itemize}
\end{marker}


\begin{docCommand}[before={\CHDocNew{0.7}}]{LoopAddtoCounters}{\marg{counter1, counter2,\dots}\marg{counter increment/decrement}}

%\begin{docCommandArgs}{LoopAddToCounters}%
The 2nd argument value is added (or subtracted) to the counters given in the list of the 1st argument using the \cs{addtocounter}.

\begin{codeoptionsenum}
  \item \marg{counter1, counter2,\dots} 
    Holds the comma separated list of counter names
  \item \marg{counter increment/decrement}
    Specifies the value to be added or subtracted.

    No check is performed whether \#2 \textbf{is} or \textbf{expands} to an integer value. 
  \end{codeoptionsenum}
%\end{docCommandArgs}

\end{docCommand}


\begin{docCommand}[before={\CHDocNew{0.7}}]{LoopResetCounters}{\marg{counter1, counter2,\dots}}


%\begin{docCommandArgs}{LoopResetCounters}%
All counters given in the first argument are set to zero using the regular \cs{setcounter}. This is a shorthand version of \refCom{LoopSetCounters} for this specific case. 


\begin{codeoptionsenum}
  \item \marg{counter1, counter2,\dots} 
    Holds the comma separated list of counter names
  \end{codeoptionsenum}
%\end{docCommandArgs}

\end{docCommand}


\begin{docCommand}[before={\CHDocNew{0.7}}]{LoopRefstepCounters}{\marg{counter1, counter2,\dots}} 

%\begin{docCommandArgs}{LoopStepCounters}%
All counters given in the first argument are stepped using the regular \cs{refstepcounter} to allow labels -- however, only the last counter will have the correct label reference.

\begin{marker}
  This macro is meant only to complete the number of \cs{Loop...Counters} but is not regarded as being really useful. 
\end{marker}

\begin{codeoptionsenum}
  \item \marg{counter1, counter2,\dots} 
    Holds the comma separated list of counter names
  \end{codeoptionsenum}
%\end{docCommandArgs}

\end{docCommand}



\begin{docCommand}[before={\CHDocNew{0.7}}]{LoopSetCounters}{\marg{counter1, counter2,\dots}\marg{new counter value}} 

%\begin{docCommandArgs}{LoopAddToCounters}%
The 2nd argument value is used as new counter value  added (or subtracted) to the counters given in the list of the 1st argument using the \cs{addtocounter}.

\begin{codeoptionsenum}
  \item \marg{counter1, counter2,\dots} 
    Holds the comma separated list of counter names
  \item \marg{new counter value}
    Specifies the value to be set. 

    No check is performed whether \textbf{is} or \textbf{expands} to an integer value. 
  \end{codeoptionsenum}
%\end{docCommandArgs}

\end{docCommand}


\begin{docCommand}{LoopStepCounters}{\marg{counter1, counter2,\dots}} \CHDocNew{0.7}

\begin{docCommandArgs}{LoopStepCounters}%
All counters given in the first argument are stepped using the regular \cs{stepcounter}.

\begin{codeoptionsenum}
  \item \marg{counter1, counter2,\dots} 
    Holds the comma separated list of counter names
  \end{codeoptionsenum}
\end{docCommandArgs}

\end{docCommand}


A more general command for doing "arbitrary" operations with counters (and more setup, for example) is

\begin{docCommand}{LoopCountersFunction}{\marg{counter1, counter2,\dots}\marg{counter operation macro}} \CHDocNew{0.7}

\begin{docCommandArgs}{LoopAddToCounters}%
The 2nd argument value should hold a macro with any number of arguments, but the last mandatory argument of this macro is reserved for counter name.

\begin{codeoptionsenum}
  \item \marg{counter1, counter2,\dots} 
    Holds the comma separated list of counter names
  \item A macro name that is to be called and that operates on a counter.
  \end{codeoptionsenum}
\end{docCommandArgs}

\end{docCommand}


\begin{dispExample}
  % We assume we have the counters foocntr and foobarcntr
   \newcommand{\showcountervalues}[2]{%
     \textcolor{#1}{\csname the#2\endcsname}% Now, an extra empty line to show the values in rows
     
   }
   % Note that the 2nd argument is not given here -- it's added by the \LoopCountersFunction macro
   \LoopCountersFunction{foocntr,foobarcntr}{\showcountervalues{blue}}
\end{dispExample}


\begin{docCommand}[before={\CHDocNew{1.4}}\par]{LoopCounterResetList}{\marg{counter name}\marg{counter operation macro}}
This macro will perform the same action on the reset list of a the counter name given as first argument, the action is a control sequence name specified by the in the second mandatory argument. The loop provides all counters on the reset list of a counter. 

As of version \packageversion\ the counter operation macro must have two mandatory arguments, the second one is meant for the current counter in the loop. 

Do not confuse this command with \refCom{LoopFullCounterResetList} which tracks all counters recursively on the reset list, so \refCom{LoopCounterResetList} steeps  only level down in the reset list hierarchy.  
\end{docCommand}

\begin{docCommand}[before={\CHDocNew{1.0}}\par]{LoopFullCounterResetList}{\marg{counter name}\marg{counter operation macro}} 
This macro determines the full reset list of a counter, i.e. it cascades down the reset list and tracks the reset lists of all 'sub'-counters too and performs the counter operation macro on this.  
\begin{codeoptionsenum}
  \item \marg{counter name}

    Holds the comma separated list of counter names
  \item \marg{counter operation macro} 
    A macro name that is to be called and that expects the name of a counter as the last argument. 
  \end{codeoptionsenum}

See the macro \refCom{CounterFullResetList} for more information about the internal storage of the full reset list. 

\end{docCommand}



\begin{docCommand}[before={\CHDocNew{1.4}}\par]{CounterWithin}{\marg{counter nameA, counter nameB,\dots}\marg{drivercounter}} 
  This macro sets all counters nameA, nameB, \dots to the reset list of the \marg{drivercounter} and redefines the corresponding macros \cs{thenameA}, etc. to be prepended with \cs{thedrivercounter}, i.e. \verb!\CounterWithin{equation}{section}! would mean that  \verb!\theequation! expands to \verb!\thesection.\arabic{section}!

  The default format for the counter output is arabic numbers, i.e. \cs{arabic} will be used. 

If the macros \cs{thenameA} etc. should not be changed, use the starred version of this command: \refCom{CounterWithin*}. 

    \begin{marker}
      Please note that the redefinition of \cs{thenameA} etc. is only local, i.e. it is group safe. 
    \end{marker}
\end{docCommand}

\begin{docCommand}[before={\CHDocNew{1.4}}\par]{CounterWithin*}{\marg{counter nameA, counter nameB,\dots}\marg{drivercounter}} 
  This macro sets all counters nameA, nameB, \dots to the reset list of the \marg{drivercounter}, but does not change the corresponding macros \cs{thenameA}, etc. at all. 

  The default format for the counter output is arabic numbers, i.e. \cs{arabic} will be used. 

  If the macros \cs{thenameA} etc. should be changed, use non-starred version of this command: \refCom{CounterWithin}. 

\end{docCommand}


\begin{docCommand}[before={\CHDocNew{1.4}}\par]{CounterWithout}{\marg{counter nameA, counter nameB,\dots}\marg{drivercounter}} 
  This macro removes all counters nameA, nameB, \dots from the reset list of the \marg{drivercounter} and redefines the corresponding macros \cs{thenameA}, etc. without \cs{thedrivercounter}, i.e. \verb!\CounterWithout{equation}{section}! would mean that \verb!\theequation! expand to \verb!\arabic{section!

    The default format for the counter output is arabic numbers, i.e. \cs{arabic} will be used. 

    If the macros \cs{thenameA} etc. should not be changed, use the starred version of this command: \refCom{CounterWithout*}. 

    \begin{marker}
      Please note that the redefinition of \cs{thenameA} etc. is only local, i.e. it is group safe. 
      \end{marker}

\end{docCommand}

\begin{docCommand}[before={\CHDocNew{1.4}}\par]{CounterWithout*}{\marg{counter nameA, counter nameB,\dots}\marg{drivercounter}} 
  This macro removes all counters nameA, nameB, \dots from the reset list of the \marg{drivercounter}, but does not redefine the corresponding macros \cs{thenameA}, etc.

  If the macros \cs{thenameA} etc. should be changed, use the non-starred version of this command: \refCom{CounterWithout}. 
\end{docCommand}





\section{Counter output}\CHDocNew{0.7}

Once in a while it might be necessary to provide counter output not only as integer numbers, letters or Roman figures but also using binary, octal or hexdecimal number output. The \CHDocPackage{fmtcount} package has support for this already -- here are some alternatives. 

\subsection{Extra counter output types}


\begin{marker}
None of the commands checks whether the argument refers to counter name. 
\end{marker}
\begin{docCommand}[doc new={\chdocextractversion{xassoccntversion0.7}}]{BinaryValue}{\marg{counter name}}

This command will print the value of the counter using binary digits. 

\end{docCommand}

\begin{docCommand}{hexValue}{\marg{counter name}} \CHDocNew{0.7}

This command will print the value of the counter using lowercase hexadecimal digits. 

\end{docCommand}

\begin{docCommand}{HexValue}{\marg{counter name}} \CHDocNew{0.7}

This command will print the value of the counter using uppercase hexadecimal digits. 

\end{docCommand}

\begin{docCommand}{OctalValue}{\marg{counter name}} \CHDocNew{0.7}

This command will print the value of the counter using octal digits. 

\end{docCommand}


\begin{docCommand}{xalphalph}{\marg{counter name}}\CHDocNew{1.4}
This is allows to use more than 26 characters for the usual alphabet and prints the counter value with style \texttt{aa} etc. in the same manner as the \CHDocPackage{alphalph} does, but with the \cs{int\_to\_alph:n} macro from the \CHDocPackage{expl3} bundle. For usage with uppercase characters see \refCom{xAlphAlph}. 
\end{docCommand}

\begin{docCommand}{xAlphAlph}{\marg{counter name}}\CHDocNew{1.4}
This is allows to use more than 26 characters for the usual alphabet and prints the counter value with style \texttt{AA} etc. in the same manner as the \CHDocPackage{alphalph} does, but with the \cs{int\_to\_Alph:n} macro from the \CHDocPackage{expl3} bundle. For usage with uppercase characters see \refCom{xalphalph}. 
\end{docCommand}

\subsection{Quick counter output changes}\CHDocNew{1.4}


\begin{docCommand}{CounterFormat}{\oarg{options}\marg{counter1!formatname1,counter2!formatname2,\dots }}\CHDocNew{1.4}

\begin{marker}
  This macro needs the package option \refKey{standardcounterformats} to be activated with \meta{standardcounterformats=on}, which is the default. 
\end{marker}

\begin{codeoptionsenum}
\item \oarg{options}
  \begin{docKey}[][]{recursive}{=\meta{true/false}}{default: false}
    If this key is set, the same counter format is used for the relevant counter and its resetting counters, i.e. the macro will pursue the reset counter list chain and recursively adds \cs{the...} to the output format of \cs{thenameA} etc. 
    \begin{marker}
      Since the \meta{recursive} option needs information on the parent counters, the macro \refCom{GetAllResetLists} must have been called before \refCom{CounterFormat} with this option can be applied. In order to provide the most recent information (which includes recently added counters or changed resetting levels), use \refCom{GetAllResetLists} just before \refCom{CounterFormat}. 
      \end{marker}
  \end{docKey}
  \begin{docKey}[][]{separator}{=\meta{separator character/string}}{default: !}
    Specifies the separator that is used to split the counter name from the format, e.g. \meta{chapter!R} where \meta{chapter} is the counter name and \meta{R} will be recognized as a counter format, meaning \cs{Roman} here, see \cref{table--predefined-counter-formats} for a list of predefined counter formats. 
    
    \begin{marker}
      The chosen separator must be the same for all counters in the given list of the 2nd argument (see below) and mustn't occur in the counter name itself\footnote{It is not recommended to use counter names with non alphabetic characters anyway.}.
      \end{marker}
    \end{docKey}
  \item \marg{counter1!formatname1,counter2!formatname2,\dots} 
    A comma separated list of counters with a given format name, each separated with a separator charactor, default is \meta{!}. If the format is omitted, the default format is \cs{arabic}, i.e. arabic numbers are used. 
  \end{codeoptionsenum}
    
\end{docCommand}

\begin{table}[htpb]
Currently following counter formats shorthands and their output macros are stored in \cs{AtBeginDocument} if the package option \refKey{standardcounterformats} is set to on, which is the default behaviour of the package. 

\centering 
\begin{tabular}{ll}
  a & \cs{alph} \tabularnewline
  A & \cs{Alph} \tabularnewline
  aa & \refCom{xalphalph}  \tabularnewline
  AA & \refCom{xAlphAlph}  \tabularnewline
  b & \refCom{BinaryValue}  \tabularnewline
  h & \refCom{hexValue}  \tabularnewline
  H & \refCom{HexValue}  \tabularnewline
  n & \cs{arabic}        \tabularnewline
  o & \refCom{OctalValue}    \tabularnewline
  r & \cs{roman}    \tabularnewline
  R & \cs{Roman}    \tabularnewline
\end{tabular}
\caption[List of predefined counter formats]{List of predefined counter format shorthands -- please note that \meta{n} has been used in order to allow \meta{a} to be used for output with lowercase characters.}\label{table--predefined-counter-formats}
\end{table}

\begin{dispExample*}{title=Simple usages of \refCom{CounterFormat},breakable}
% Assume foobar is a defined counter
\setcounter{foobar}{17}

\CounterFormat{foobar!b}
\thefoobar

\CounterFormat{foobar!h}
\thefoobar


\CounterFormat{foobar!H}
\thefoobar

\CounterFormat{foobar!R}
\thefoobar

\setcounter{foobar}{30}

\CounterFormat{foobar!aa}
\thefoobar

\CounterFormat{foobar!o}
\thefoobar

\CounterFormat{foobar!AA}
\thefoobar


\end{dispExample*}

\begin{dispExample*}{title={Showing the \meta{recursive} option of \refCom{CounterFormat}},breakable}
% All counters are using \arabic by default from \newcounter or \NewDocumentCounter
\NewDocumentCounter{foolevelzero}
\NewDocumentCounter{foolevelone}[foolevelzero]
\NewDocumentCounter{fooleveltwo}[foolevelone]

% Get the current reset lists! (Important}

\GetAllResetLists

% Now change to Hex format (!H) for all counters in the hierarchy. 
\CounterFormat[recursive]{fooleveltwo!H}

\setcounter{foolevelzero}{20}% Should be 14
\setcounter{foolevelone}{15}% Should be F
\setcounter{fooleveltwo}{10}% Should be A

\thefoolevelzero % -> 14

\thefoolevelone % 14.F

\thefooleveltwo % 14.F.A  

\end{dispExample*}


\begin{docCommand}{StoreCounterFormats}{\oarg{options}\marg{formatshorthandA!formatmacroA,formatshorthandB!formatmacroB,\dots}}\CHDocNew{1.4}
  Stores the counter formats separated by the separator charactor as given in the option to the global list. Existing formats will be overwritten if the format shorthand already exists. There is no warning about this!
  The only handled option is \refKey{separator} and has the same meaning as in \refCom{CounterFormat}. 

  The formatmacro must be a command sequence with exactly one mandatory argument, which may not be specified in the format storage process.  

\begin{marker}
  This macro will become a preamble-only command most likely. 
\end{marker}
\end{docCommand}

\begin{dispExample}
  \StoreCounterFormats{foo!\Roman,foobarnice!\OctalValue}
  \setcounter{foobar}{17}
  \CounterFormat{foobar!foo}
  \thefoobar
  
  \CounterFormat{foobar!foobarnice}
  \thefoobar
\end{dispExample}




\begin{docCommand}{AddCounterFormats}{\oarg{options}\marg{formatshorthandA!formatmacroA,formatshorthandB!formatmacroB,\dots}}\CHDocNew{1.4}
  Adds the counter formats separated by the separator charactor as given in the option to the global list, similar to \refCom{StoreCounterFormats}. Existing formats will be overwritten if the format shorthand already exists. There is no warning about this!
  The only handled option is \refKey{separator} and has the same meaning as in \refCom{CounterFormat}. 

  The formatmacro must be a command sequence with exactly one mandatory argument, which may not be specified in the format storage process.  
\end{docCommand}

\begin{docCommand}{RemoveCounterFormats}{\oarg{options}\marg{formatshorthandA,formatshorthandB,\dots}}\CHDocNew{1.4}
  Removes the given counter formats from the global list. 

  The optional argument is ignored as of version \packageversion. 
\end{docCommand}





\clearpage
\part{Features}

\parttoc

\section{Associated counters}
\tcbset{color command={blue}}

The main purpose of this package is co-stepping of counters, but there are some helper commands in addition to macros provided \LaTeXe\ already, see section \nameref{section::additions_to_l2e}. 

\begin{itemize}
  \item Section \nameref{subsection::associatedcounterscmds} describes the most important macros for setting up associated counters
  \item Section \nameref{subsection::drivercounterscmds} informs about the macros for setting up, removing or clearing driver counters
  \item Section \nameref{subsection::querycmds} deals with query command sequences about counters being a driver or an associated counters
  \item Section \nameref{subsection::informationcmds} contains routines that show which counters have been changed last
\end{itemize}



\subsection[Association macros]{Associated counters commands}\label{subsection::associatedcounterscmds}

All macros have the general rule, that the driver counter is specified as 1st mandatory argument to the macro, which is in almost all cases the 2nd argument of the macro.


\begin{docCommand}{DeclareAssociatedCounters}{\oarg{options}\marg{driver counter}\marg{associated counters list}}
This command is the main macro of the package. It declares the counter names being specified in comma - separated - list (CSV) which should be stepped simultaneously when the driver counter is increased by \cs{stepcounter}. If only counter is to be associated, omit a trailing ","! 

%\begin{docCommandArgs}{DeclareAssociatedCounters}

\begin{codeoptionsenum}
  \item \oarg{options}: %As of \packageversion, the optional argument \oarg{options} is not used so far, but is reserved for later purposes.
    \begin{docKey}[][]{autodefine}{=\meta{choice}}{initially none}
      This choice - key can be specified if the specified counters should be defined if they not already available.
      Possible values are
      \begin{itemize}
        \item \texttt{none} -- no counter is autodefined
        \item \texttt{all} -- all counters will be autodefined          
        \item \texttt{driver} -- only driver counters will be autodefined          
        \item \texttt{associated} -- only associated counters will be autodefined          
        \end{itemize}	
      \end{docKey}
      Default is \texttt{none}
      \begin{docKey}[][]{sloppy}{}{}
        If \refKey{autodefine} key is used, the \texttt{sloppy} key disables the check whether a counter is defined already. 
      \end{docKey}

  \item \marg{driver counter} 

    Holds the name of the driver counter to which the list of counters should be associated
\item \marg{associated counters list}

  A comma separated list of counter names that should be associated to the driver counter
\end{codeoptionsenum}
%\end{docCommandArgs}


\begin{itemize}
\item This command is a preamble command, i.e. it can be used in the preamble of the document or within other packages or class files only. 
\item This command should be used as early as possible, i.e. in the preamble of the document, since the driven counters are not increased as long as they are not associated to the driver counter. On the hand, it is possible or may be required to control the starting point of the association at any position in the body of the document, when the association should start later on. Use the command \refCom{AddAssociatedCounters} if counters should be associated within the document body. 
\end{itemize}


% Relax for documentation purposes
\renewcommand{\DeclareAssociatedCounters}[3][]{\relax}%
\begin{dispExample}
%%%% The association of anothertotalpages in this example just takes place here, so the stepping of the counter will start from here and providing a 'wrong' value.
%%%% 
\DeclareAssociatedCounters{page}{totalpages,anothertotalpages}%
This document has \number\totvalue{totalpages} (note: \number\totvalue{anothertotalpages}) pages.
\end{dispExample}

\begin{itemize}
  \item Current version (\packageversion) rules:
    \begin{itemize}
      \item No checking whether the 2nd and 3rd arguments hold counter names is applied.
      \item Mutually cross - association of two counters is not supported! The compilation will stop on this!
        \CHDocUpdate{0.6}

        A driver counter, say, \CHDocCounter{foo}) of, say \CHDocCounter{foobar} can not be an associated counter of \CHDocCounter{foobar}, which in turn can be a driver counter of other counters, of course. 

        A contrary feature are the \CHDocTag{coupled counters} -- If some counters should share a common base, i.e. increasing one arbitrary member counter of a group of counters then all should be increased, this called coupling of counters -- all group members are on an equal footing. See \cref{sec::coupledcounters} about this feature. 

       On the other side, \CHDocTag{associated counters} belong to a hierarchy. The driver counter dominates the associated counters. 
        
      \end{itemize}
  \item A self-association of the driver counter to itself is ignored internally as this would lead to inconsistent counter values. 
  \item The order of the specification of associated counters in the 2nd arguments is of no importance.
  \item Specifing an associated counter name multiple times has no effect, only the first occurence of the name will be used.
\end{itemize}

\end{docCommand}


\begin{docCommand}{AddAssociatedCounters}{\oarg{options}\marg{driver counter}\marg{associated counters list}}
The usage of this macro is similar to \refCom{DeclareAssociatedCounters}; if it is called in the document preamble (or in package file), \refCom{AddAssociatedCounters} falls back to 
\begin{center}\refCom{DeclareAssociatedCounters},\end{center} having the same optional argument functionality with \refKey{autodefine} and \refKey{sloppy}; if it is called in the document body, this command adds some counters to the associated counter list for a specific driver counter -- if this list does not exists, the \LaTeX{} run will issue a warning, but add the driver counter to the driver list and the associated counters analogously. 
\marginnote{\bcbombe}

Using \refCom{AddAssociatedCounters} in the document body automated generation of counters is disabled. 



\begin{docCommandArgs}{AddAssociatedCounters}

\begin{codeoptionsenum}
  \item \oarg{options}: As of version \packageversion, the optional argument \oarg{options} are the same as for \refCom{DeclareAssociatedCounters}, see \refKey{autodefine} and \refKey{sloppy}.

  \item \marg{driver counter} 

    Holds the name of the driver counter to which the list of counters should be associated
  \item \marg{associated counters list}

  A comma separated list of counter names that should be associated to the driver counter
\end{codeoptionsenum}
\end{docCommandArgs}


% macro of the package. It declares the counter names being specified in comma - separated - list (CSV) which should be stepped simultaneously when the driver counter is increased by \cs{stepcounter}.

\end{docCommand}%

\begin{docCommand}{RemoveAssociatedCounter}{\marg{driver counter}\marg{associated counter}}
This command removes a counter from the existing list for a driver counter, i.e. the counter will not be increased any longer by \cs{stepcounter}. It can be increased however manually, of course. 
\end{docCommand}



\begin{dispExample}
\RemoveAssociatedCounter{page}{anothertotalpages}
This document has \number\totvalue{totalpages} (beware: \number\totvalue{anothertotalpages}) pages.
\end{dispExample}



\begin{docCommand}{RemoveAssociatedCounters}{\marg{driver counter}\marg{list of associated counters}}
This command removes the comma-separated-value list of counters from the existing list for a driver counter, i.e. the counters will not be increased any longer by \cs{stepcounter}. They can be increased however manually, of course. 

Take care not to confuse the commands \refCom{RemoveAssociatedCounters}
and{}\linebreak \refCom{RemoveAssociatedCounter}
\end{docCommand}

\begin{docCommand}{ClearAssociatedCounters}{\oarg{options}\marg{driver counter}}
This command clears the internal list for all counters associated to the \marg{driver counter}. The counters will not be increased automatically any longer.

The optional argument is not used as of version \packageversion.

Please note that the driver counter is not removed from the list of driver counters -- this simplifies reassociating of (other) counters to this one later on with the macro \refCom{AddAssociatedCounters} and suppress the relevant warning.

If the driver counter and all its associated counters should be removed, use \refCom{RemoveDriverCounter} instead. 
\end{docCommand}


\begin{docCommand}[before={\CHDocNew{1.2}}]{DeclareTotalAssociatedCounters}{\oarg{options}\marg{driver counter}\marg{associated counters list}}
This command combines the features of \CHDocTag{associated counters} and \CHDocTag{total counters}, i.e. the associated counters are defined with \refCom{NewTotalDocumentCounter} and associated to the driver counter.

See \cref{sec::totalcounters} for more information on \CHDocTag{total counters}. 
\end{docCommand}



\clearpage


\subsection[Driver macros]{Driver counter commands} \label{subsection::drivercounterscmds}


\begin{docCommand}{AddDriverCounter}{\oarg{options}\marg{driver counter name}}

\begin{docCommandArgs}{AddDriverCounter}%

\begin{codeoptionsenum}
\item \oarg{options}: As of \packageversion, the optional argument \oarg{options} is not used so far, but is reserved for later purposes. 

  \item \marg{driver counter name} 

    Holds the name of the driver counter that should be added to the list of driver counters.
\end{codeoptionsenum}
\end{docCommandArgs}

\end{docCommand}%



\begin{docCommand}{RemoveDriverCounter}{\oarg{options}\marg{driver counter}}
This command clears the internal list for all counters associated to the \marg{driver counter}. The counters will not be increased automatically any longer.

The optional argument is not used as of version \packageversion.

If all driver counters should be unregistered, use \refCom{ClearDriverCounters} instead!
\end{docCommand}


\begin{docCommand}{ClearDriverCounters}{\oarg{options}}%

This clears completely the list of driver counters, such that no counters are regarded as being associated -- i.e. no driver is hold as being a driver counter.

The optional argument is not used as of version \packageversion. 

\end{docCommand}


\subsection[Query macros]{Commands for queries} \label{subsection::querycmds}

Sometimes it might be necessary to get information, whether a counter is regarded as a driver or as an associated counter. This section describes some query macros in order to obtain this information.


\begin{docCommand}{IsAssociatedToCounter}{\marg{driver counter}\marg{associated counter}\marg{True branch}\marg{False branch}}
This macro checks, whether a counter is associated to a particular given driver counter and expands the corresponding branch. If the internal driver counter list does not exist, the false branch will be used, since this also means, that the possibly associated counter is not associated at all. 



\begin{docCommandArgs}{IsAssociatedToCounter}%

\begin{codeoptionsenum}
  \item \marg{driver counter} 

    Holds the name of the driver counter to which \marg{associated counter} the could possibly be associated.
\item \marg{associated counter}

  Contains the name of the possibly associated counter.

\item \marg{True branch}

  This code is expanded if the counter is associated to the driver, otherwise it is ignored.

\item \marg{True branch}

  This code is expanded if the counter is \textbf{not} associated to the driver, otherwise it is ignored.

\end{codeoptionsenum}
\end{docCommandArgs}


\begin{dispExample}
% Remove associated counter first for demonstration purposes
\RemoveAssociatedCounter{page}{anothertotalpages}
\IsAssociatedToCounter{page}{totalpages}{Yes, totalpages is associated}{No, totalpages is not associated}

\IsAssociatedToCounter{page}{anothertotalpages}{Yes, anothertotalpages is associated}{No, anotherpages is not associated}
\end{dispExample}

See also

\begin{itemize}
  \item \refCom{IsAssociatedCounter} for checking whether a counter is associated 
  \item \refCom{IsDriverCounter} in order to check whether a counter is a driver. 
  \item \refCom{GetDriverCounter} returns the driver counter name for a given associated counter name
\end{itemize}


\end{docCommand}


\begin{docCommand}{GetDriverCounter}{\marg{counter name}}%

This commands returns the driver counter to which the counter name of the first argument is connected to. If the counter is not defined, the macro returns nothing. 

\begin{itemize}
  \item No check whether the counter name is defined is performed
  \item No check whether the counter is associated at all is performed. Usage of this command in conjunction with \refCom{IsAssociatedCounter} is strongly encouraged. 
\end{itemize} 


\begin{dispExample}%
totalpages is associated to the \textcolor{blue}{\textbf{\GetDriverCounter{totalpages}}} counter. 
% Try with an undefined counter name
humptydumpty is associated to the \textcolor{blue}{\textbf{\GetDriverCounter{humptydumpty}}} counter. 

\end{dispExample}% 

\end{docCommand}%


\begin{docCommand}{IsAssociatedCounter}{\marg{counter name}\marg{True branch}\marg{False branch}}%

This commands tests, whether a given counter name is an associated counter and expands correspondingly the true or the false branch. The command does not tell to which driver the counter it is associated -- this information can be obtained by \refCom{GetDriverCounter}. 

\begin{docCommandArgs}{IfAssociatedCounter}%

\begin{codeoptionsenum}
\item \marg{counter name}%

  Contains the name of the possibly associated counter

\item \marg{True branch}

  This code is expanded if the counter is associated to a driver, otherwise it is ignored

\item \marg{True branch}

  This code is expanded if the counter is \textbf{not} associated a  driver, otherwise it is ignored

\end{codeoptionsenum}
\end{docCommandArgs}


\begin{dispExample}
\IsAssociatedCounter{section}{Yes, section is an associated counter}{No, section counter does not have the associated counter properties}
\IsAssociatedCounter{totalpages}{Yes, totalpages is an associated counter}{No, totalpages counter does not have the associated counter properties}
\end{dispExample}

\end{docCommand}%



\begin{docCommand}{IsDriverCounter}{\marg{driver counter name}\marg{True branch}\marg{False branch}}%

This commands tests, whether a given counter name is a driver counter and expands correspondingly the true or the false branch.

\begin{docCommandArgs}{IfDriverCounter}%

\begin{codeoptionsenum}
\item \marg{driver counter name}%

  Contains the name of the possible driver counter

\item \marg{True branch}

  This code is expanded if the counter is a driver, otherwise it is ignored

\item \marg{True branch}

  This code is expanded if the counter is \textbf{not} a  driver, otherwise it is ignored
\end{codeoptionsenum}
\end{docCommandArgs}


\begin{dispExample}
\IsDriverCounter{section}{Yes, section is a driver counter}{No, section counter does not have driver properties}
\end{dispExample}

\end{docCommand}%



\section[Counter backup/restoration]{Backup and restore of counter values}\CHDocNew{1.0}\label{section::new_backuprestore}

It might be necessary to interrupt the current sectioning, e.g. including another document's structure (an external paper, for example) such that the counting should start again and after finishing of the external structure the old values should be restored. 


\begin{marker}
  Since version \CHDocFullVersion{1.0} the commands and feature behaviour of backup and restoration of counter values has changed. 

  The old behaviour is still available using the macros \cs{Former...} prefixed macros, see \cref{section::old_backuprestore} for this. 
\end{marker}

\subsection{Key philosophy in backup/restore}

The basic idea is to provide a scheme that allows easy storage and restoration of counter values. For a single counter this is quite easy, using some other temporary counter, storing the old value there and copy the values back at the right position -- if many counters should be controlled to have a backup this procedure might get tedious, however. 

\CHDocPackage{xassoccnt} provides some tools to define groups of counters that should be under control of backup and restoration. A backup counter group can be used multiple times, each using an individual ID, this allows maintaining 'complete' counter states and reinjecting them at any position later on.

Those are the basic steps (in pseudo code) to use the backup feature:

\begin{enumerate}
  \item Define a symbolic backup counter group name 

    Note that counter group names are providing something like a namespace. A group of coupled counters (see \namecref{sec::coupledcounters} \nameref{sec::coupledcounters}) may have the same name as a group of counters designed to be backed up -- however, each feature has its own namespace. In each namespace there can be only one counter group with a specific name, duplicates are not allowed within the same feature namespace.
  \item Populate the counter group name with counter names
  \item Define a backup state and a corresponding id 
  \item Restore at any place
\end{enumerate}

\subsection{The default counter group "scratch"}

There is a default counter group for backup named "scratch", if no specific counter group name is given using the \refKey{name} option (see \cref{subsubsec::common_options_backuprestore} for a detailed description of available backup/restore options). 

\subsection{Description of basic backup/restore macros}

\subsubsection{Common options to (most) of the backup/restore macros} \label{subsubsec::common_options_backuprestore}

\begin{itemize}
\item 
  \begin{docKey}[][]{resetbackup}{=\meta{true/false}}{initially true}
    This key decides whether \textbf{all} counters in the backup list should be reset to zero or should keep the current value when the backup command is given. The default value is \meta{true}.
  \end{docKey}
\item \begin{docKey}[][]{cascading}{=\meta{true/false}}{initially false}
    This key decides whether all counters in the reset list of a driver counter shall be added to the counter group -- the driver counter is added itself too. If the driver counter has no reset list, it is added nevertheless. 
    
    Please note that the reset lists of the individual counters are tracked as well -- this is done recursively. 
    
    This provides a very convenient feature to backup and restore the value of a certain sectioning level, say \cs{chapter}. Using \refKey{cascading} this would mean\footnote{Assuming \CHDocClass{book} is used, for example!} that \CHDocCounter{chapter}, \CHDocCounter{section}, \CHDocCounter{subsection},\CHDocCounter{subsubsection}, \CHDocCounter{paragraph}, \CHDocCounter{subparagraph}, \CHDocCounter{equation}, \CHDocCounter{figure} and \CHDocCounter{table} would be added to the counter group.
    \end{docKey}
  \item  \begin{docKey}[][]{backup-id}{=string}{no default value}
      This key declares a string-like backup id under which the backup of a certain group is stored and can be retrieved later on. The id should be a string containing only alphanumeric characters. 
    \end{docKey} 
\item 
  \begin{docKey}[][]{restore-id}{=string}{no default value}
    This key declares a string-like restore id for unique \CHDocPackage{hyperref} names. The id should be a string containing only alphanumeric characters.     \end{docKey} 
  This key is only needed if the same backup-id should be restored more than once and the package \CHDocPackage{hyperref} is used in order to provide unique hyper anchors. 
  \item \begin{docKey}[][]{keep-after-restore}{=\meta{true/false}}{initially false}
  This key decides whether the values of a certain backup id (see \refKey{backup-id}) shall be kept after a restore has been issued. The default operation is to remove the \refKey{backup-id} and the values of the counters belonging to the given \refKey{backup-id}.
  
  This option is useful if the same backup state (i.e. \refKey{backup-id}) is to be used more than once. 

  Reusing the same counter group for restore after having restored them already with the \refKey{keep-after-restore} option being false will be ignored, i.e. the counter values are not changed. 
\end{docKey}

\item \refKey{name} 
  The meaning of this key is basically the same as for the \nameref{sec::coupledcounters} feature. However, depending on the stage of the backup/restore process the precise action differs:
  \begin{itemize}
    \item If \refKey{name} is used in \refCom{AssignBackupCounters} the counters are added to group indicated by the value of \refKey{name}.
    \item If \refKey{name} is used either for \refCom{BackupCounterGroup} or in \refCom{RestoreBackupCounterGroup} to indicate the counter group for a certain backup state, either for storing or restoring. 
     \end{itemize}
\end{itemize}


\subsubsection{Core backup/restore macros}

\begin{docCommand}[before={\CHDocNew{1.0}}]{BackupCounterGroup}{\oarg{options}\marg{counter group name}}
  Concerning the backup feature this performs the storing of counter values at the position where the command is expanded!
  A backup operation needs 
  \begin{itemize}
    \item \refKey{backup-id} to be able to refer to a certain backup state
    \item \refKey{name} to identify the group of counters
  \end{itemize}
\end{docCommand}

\begin{docCommand}[before={\CHDocNew{1.0}}]{RestoreBackupCounterGroup}{\oarg{options}\marg{counter group name}}
  This command restores the value of the given counter group. 
  Useful options are (see \nameref{subsubsec::common_options_backuprestore})
  \begin{itemize}
    \item \refKey{backup-id}
    \item \refKey{keep-after-restore}
    \item \refKey{restore-id}
    \item \refKey{name}
    \end{itemize}
\end{docCommand}


\subsubsection{Declaring counter groups}

\begin{docCommand}[before={\CHDocNew{1.0}}]{DeclareBackupCountersGroupName}{\oarg{}\marg{counter group name}}
  This command declares (better: reserves a name for a backup counter group) -- the name has to be specified as first mandatory argument. 

  A counter group name consists of alphanumeric characters, special symbols etc. are not allowed. The name must not contain commas!
  
  As of version \packageversion\ the optional argument is not used and reserved for later purposes. 
\end{docCommand}

\subsubsection{Populating counter groups}

\begin{docCommand}[before={\CHDocNew{1.0}}]{AssignBackupCounters}{\oarg{options}\marg{counter name1,counter name2,\dots}}
  This macro populates the counter group (given as optional key - value \refKey{name}) with the comma separated list of counters from the first mandatory argument. The meaning of \refKey{name} is the same as in \nameref{sec::coupledcounters}, but the counter group names are not related to that feature -- the namespace is safe then.
  
  Some important notes about the behaviour of \refCom{AssignBackupCounters}
  \begin{itemize}
  \item If \refKey{name} is not given or empty, the default counter group "scratch" is used. 
    
  \item If the counter group given to \refKey{name} does not exist, it will be created automatically. 
  \end{itemize}
  
\end{docCommand}


\begin{docCommand}[before={\CHDocNew{1.0}}]{AddBackupCounter}{\oarg{options}\marg{counter name1,counter name2,\dots}}
  This command is similar to \refCom{AssignBackupCounters}, adding counters to a counter group named by \refKey{name} option. If the counter group name does not exist, no action is performed, i.e. the adding operation is ignored. 
\end{docCommand}


\subsubsection[Clearing backup states]{Clearing backup states}\label{subsubsec::clearingbackupstates}

It might be necessary to remove a certain backup state, i.e. a collection of counter values to be referred to with a \refKey{backup-id}. This can be done either for a certain counter (with \refCom{ClearCounterBackupState}) in a counter group only or for all counters in group (with \refCom{ClearBackupState}).

\begin{docCommand}[before={\CHDocNew{1.0}}]{ClearCounterBackupState}{\oarg{options}\marg{countergroup name}\marg{counter name}}
  This macro removes the given backup-id state value for a specific counter only. Use the \refKey{backup-id} option in the first optional argument to specify which backup-id should be cleared. 

  \begin{itemize}
  \item If the backup-id does not exist, the operation is ignored silently.
  \item The counter is still part of the group, but can't be restored to the previous state that was identified with \refKey{backup-id}.
  \end{itemize}

  \marginnote{\bcbombe}Use this macro with care!
  \smallskip
\end{docCommand}


\begin{docCommand}[before={\CHDocNew{1.0}}]{ClearBackupState}{\oarg{options}\marg{countergroup name}}
  This macro removes the given backup-id state value for a whole counter group. Use the \refKey{backup-id} option in the first optional argument to specify which backup-id should be cleared. 

  \begin{itemize}
  \item If the backup-id does not exist, the operation is ignored silently.
  \item The \refKey{backup-id} is removed from the list of possible backup-ids for the relevant countergroup. 
  \end{itemize}
\end{docCommand}




\subsubsection[Clearing backup groups]{Clearing and deleting backup counter groups}\label{subsubsec::clearingdeletingbackupcountergroups}

From time to time it might be necessary to remove counters from a group or to clear the whole group or remove even the whole group. Those operations can be achieved with \refCom{RemoveCountersFromBackupGroup}, \refCom{ClearBackupCounterGroups} and \refCom{DeleteBackupCounterGroups}. All macros remove at least all stored values belonging to a counter group. A \refCom{RestoreBackupCounterGroup} call does nothing after any of the mentioned macros have been issued. 

\begin{docCommand}[before={\CHDocNew{1.0}}]{RemoveCountersFromBackupGroup}{\oarg{options}\marg{countergroup name}\marg{counter name 1, counter name 2,\dots}}
  This command removes the given counter names from a counter group. The stored values are deleted completely and are lost afterwards. A non-existing counter group name is ignored as well as counters that do not belong to the given group name. 

  As of version \packageversion\ the optional argument is not used and reserved for later purposes. 

  If complete counter groups shall be deleted and be unavailable afterwards, use 

\refCom{DeleteBackupCounterGroups} instead. 
\end{docCommand}


\begin{docCommand}[before={\CHDocNew{1.0}}]{ClearBackupCounterGroups}{\oarg{options}\marg{countergroup name 1, countergroup name 2,\dots}}
  This command removes all names of the counter groups given as comma separated list in the 2nd argument, the internal storage of counter values is removed as well, i.e. older counter values aren't available any longer. The counter group names remain valid, any subsequent call to \refCom{BackupCounterGroup} or \refCom{RestoreBackupCounterGroup} with one of the provided group names will be ignored. 
  
  As of version \packageversion\ the optional argument is not used and reserved for later purposes. 

  If counter groups shall be deleted and be unavailable, use \refCom{DeleteBackupCounterGroups}. 
\end{docCommand}


\begin{docCommand}[before={\CHDocNew{1.0}}]{DeleteBackupCounterGroups}{\oarg{options}\marg{countergroup name 1, countergroup name 2,\dots}}
  This command clears the counter groups given as comma separated list in the 2nd argument with \refCom{ClearBackupCounterGroups} and removes the names as well, i.e. those names are not available any longer.
 
  As of version \packageversion\ the optional argument is not used and reserved for later purposes. 
\end{docCommand}


\subsection{Query for backup features}\label{subsec::querybackupmacros}\CHDocNew{1.0}

As usual, some macros to ask whether a certain counter is under backup control or whether a certain \refKey{backup-id} exists might be useful!


\subsubsection{Querying for backup counter group existence}\label{subsec::querybackupcountergroup}

\begin{docCommand}[after={\CHDocNew{1.0}\par}]{IsBackupCounterGroupTF}{\marg{counter group name}\marg{true branch}\marg{false branch}}
This macro tests if the given name is a backup counter group and expands to the \meta{true}/\meta{false} branch accordingly. There are two short-circuit commands: \refCom{IsBackupCounterGroupT} and \refCom{IsBackupCounterGroupF}.
\end{docCommand}

\begin{docCommand}[after={\CHDocNew{1.0}\par}]{IsBackupCounterGroupT}{\marg{counter group name}\marg{true branch}}
This macro tests if the given name is a backup counter group and expands to the \meta{true} branch accordingly. There are two related macros: \refCom{IsBackupCounterGroupF} and \refCom{IsBackupCounterGroupTF}.
\end{docCommand}

\begin{docCommand}[after={\CHDocNew{1.0}\par}]{IsBackupCounterGroupF}{\marg{counter group name}\marg{false branch}}
This macro tests if the given name is a backup counter group and expands to the \meta{false} branch accordingly. There are two related macros: \refCom{IsBackupCounterGroupT} and \refCom{IsBackupCounterGroupTF}.
\end{docCommand}



\subsubsection{Querying for backup counter entity}\label{subsec::querybackupmacros}

\begin{docCommand}[after={\CHDocNew{1.0}\par}]{IsBackupCounterTF}{\marg{counter name}\marg{true branch}\marg{false branch}}
This macro tests if a counter is under the administration of the backup counter commands and expands to the relevant \meta{true}/\meta{false} branch then. There are two short-circuit commands: \refCom{IsBackupCounterT} and \refCom{IsBackupCounterF}.
\end{docCommand}

\begin{docCommand}[after={\CHDocNew{1.0}\par}]{IsBackupCounterT}{\marg{counter name}\marg{true branch}}
This macro tests if a counter is under the administration of the backup counter commands and expands to the \meta{true} branch then. There are related commands: \refCom{IsBackupCounterTF} and \refCom{IsBackupCounterF}.
\end{docCommand}

\begin{docCommand}[after={\CHDocNew{1.0}\par}]{IsBackupCounterF}{\marg{counter name}\marg{false branch}}
This macro tests if a counter is under the administration of the backup counter commands and expands to the \meta{false} branch if this not the case. There are related commands: \refCom{IsBackupCounterTF} and \refCom{IsBackupCounterT}.
\end{docCommand}


\subsubsection{Querying for backup state entity}\label{subsec::querybackupstate}

\begin{docCommand}[after={\CHDocNew{1.0}\par}]{IsBackupStateTF}{\marg{counter group name}\marg{counter backup-id}\marg{true branch}\marg{false branch}}
This macro tests if the backup-id exists for the given counter group and executes the relevant \meta{true}/\meta{false} branch accordingly. There are two short-circuit commands: \refCom{IsBackupStateT} and \refCom{IsBackupStateF}.
\end{docCommand}

\begin{docCommand}[after={\CHDocNew{1.0}\par}]{IsBackupStateT}{\marg{counter group name}\marg{counter backup-id}\marg{true branch}}
This macro tests if the backup-id exists for the given counter group and executes the relevant \meta{true} branch accordingly. There are two related: \refCom{IsBackupStateTF} and \refCom{IsBackupStateF}.
\end{docCommand}

\begin{docCommand}[after={\CHDocNew{1.0}\par}]{IsBackupStateF}{\marg{counter group name}\marg{counter backup-id}\marg{false branch}}
This macro tests if the backup-id exists for the given counter group and executes the relevant \meta{false} branch accordingly. There are two related: \refCom{IsBackupStateTF} and \refCom{IsBackupStateT}.
\end{docCommand}


\subsection{Some notes on the backup features} \label{subsec::backup_and_hyperref}

Principally backing up counter values and restoring them later on is not really difficult -- with one exception: If the \CHDocPackage{hyperref} package is used, the counter values form up the hypertarget anchors, for example \texttt{chapter.1} for the first chapter. If the chapter counter is reset, there would be a chapter with number one again and as well an anchor name \texttt{chapter.1} -- \CHDocPackage{hyperref} will complain 'only' about this but it will put the wrong hyperlink as well, for example for the table of contents and the bookmarks -- this is an undesirable feature.

However, there is a solution to this problem: The hypertarget anchors are built up from the specifications of a macro \cs{theH...} where the ellipses stands for the counter name. If for example \cs{theHchapter} is changed after a counter was reset the hypertargets will again be correct, since this will provide a different target name. \refCom{BackupCounterGroup} does this resetting automatically in an unique way and \refCom{RestoreBackupCounterGroup} restores as well the old \cs{theH...} macros of all counters that are in the backup list. It tracks the number of calls to \refCom{BackupCounterGroup} and changes the relevant \cs{theH...} macro definitions to use unique anchor names then -- this way multiple \refCom{BackupCounterGroup} calls are possible without destroying the hyperlink facilities with \CHDocPackage{hyperref}. 


\section{Coupled counters}\label{sec::coupledcounters}\CHDocNew{0.5}

\begin{marker}
The features described here are very experimental and not fully implemented so far. 
\end{marker}

Occasionally there are requests where the figure or table environment should use the same counter in the sense of using continued counter values, e.g figure 1 is followed by table 2, the next figure is numbered as 3 etc. 

This can be achieved with the concept of coupled counters. As usual, those counters belonging to a 'group' should be declared first in the preamble. In some sense coupled counters are similar to associated counters. 

\subsection[Common options for coupled counters]{Common options for most of the coupled counter macros}\label{subsection:options_coupledcounters}

\begin{docKey}{name}{=\meta{name of a group}}{}\CHDocNew{0.6}
  This option has the name of the counter group that should be coupled, say ``figuretablegroup'' etc. 
\end{docKey}

\begin{docKey}{multiple}{=\meta{true,false}}{initially false}\CHDocNew{0.6}
  This option allows to add a counter multiple times to a counter group. In general, using this style is not recommended. 
\end{docKey}

\subsection[Macros for coupled counters]{Macros for declaring, adding and removing coupled counters} 



\begin{docCommand}{DeclareCoupledCounters}{\oarg{options}\marg{counter name1, counter name2, \dots}}\CHDocNew{0.5}
\begin{codeoptionsenum}
  \item \oarg{options}: See \cref{subsection:options_coupledcounters} for a explanation about available options. 
  \item \marg{counter name 1, counter name2, \dots}: The list of counters that should should be stepped together for the given counter group. 
\end{codeoptionsenum}

This macro is a preamble-only command. 

\end{docCommand}

\begin{docCommand}{DeclareCoupledCountersGroup}{\marg{counter group name}}\CHDocNew{0.5}
This macro defines a name for a counter group and allocates a new group list for the counter names. If the name already exists, nothing is done. 
\begin{codeoptionsenum}
\item \marg{counter group name}: The name of the counter group. 
\end{codeoptionsenum}

This macro is a preamble-only command and does not add counters to the group container. Use \refCom{DeclareCoupledCounters} or \refCom{AddCoupledCounters} to add counters to the relevant group. 

\end{docCommand}

\begin{docCommand}{RemoveCoupledCounters}{\oarg{options}\marg{counter name1, counter name2, \dots}}\CHDocNew{0.5}

This removes the comma separated counter names from the coupled counter list given in the \refKey{name} option. 
\begin{codeoptionsenum}
  \item \oarg{options}:  As of version \packageversion{} the only recognized option is \refKey{name}.
  \item \marg{counter name 1, counter name2, \dots}: The list of counters that should should removed from the given counter group. 
\end{codeoptionsenum}

\begin{marker}
  \begin{itemize}
  \item The list name itself is still available 
  \item If the list given by the \refKey{name} option does not exist, \refCom{RemoveCoupledCounters} issues a warning on the console and ignores this list then. 
  \end{itemize}
\end{marker}

If all counters from a group name should be removed, this is equal to clearing -- just use \refCom{ClearCoupledCounters} for simpler usage of this feature. 

\end{docCommand}

\begin{docCommand}{AddCoupledCounters}{\oarg{options}\marg{counter name1, counter name2, \dots}}\CHDocNew{0.5}
\CHDocUpdate{0.6}
This adds the listed counter names to coupled counter list. It acts like \refCom{DeclareCoupledCounters}, but does not setup new counter groups. Please use \refCom{DeclareCoupledCounters} first, then apply \cs{AddCoupledCounters} later on.


\begin{codeoptionsenum}
   \item \oarg{options}: See \cref{subsection:options_coupledcounters} for a explanation about available options. 
  \item \marg{counter name 1, counter name2, \dots}: The list of counters that should should be stepped together for the given counter group. 
\end{codeoptionsenum}
\begin{marker}
  If the list given by the \refKey{name} option does not exist, \refCom{AddCoupledCounters} issues a warning on the console and ignores this list then. The counters are not added to any list at all. 
\end{marker}


\end{docCommand}

\begin{docCommand}{ClearCoupledCounters}{\marg{options}}\CHDocNew{0.6}
  This removes all names from the given name of a group of coupled counters. 
  \begin{codeoptionsenum}
  \item \oarg{options}:  As of version \packageversion{} the only recognized option is \refKey{name}.
\end{codeoptionsenum}
After clearing a list, the coupling stops for the counters on that list (unless they are part of another list, which is possible, but not recommended). using \refCom{AddCoupledCounters} with the relevant \refKey{name} option adds counters again to the list and the coupling is active again, however, for different counters (eventually). 

In order to clear all coupled counter lists, use \refCom{ClearAllCoupledCounters} instead. 

\begin{marker}
  \begin{itemize}
  \item The list name itself is still available 
  \item If the list given by the \refKey{name} option does not exist, \refCom{ClearCoupledCounters} issues a warning on the console and ignores this list then. 
  \end{itemize}
\end{marker}
\end{docCommand}


\begin{docCommand}{ClearAllCoupledCounters}{}\CHDocNew{0.6}
  This removes all coupled counter groups, but not the group names, i.e. the list names can be used later on to add counter names again. In order to clear a specific list, use \refCom{ClearCoupledCounters}.
\end{docCommand}


\begin{docCommand}{IsCoupledCounterTF}{\marg{counter name}\marg{true branch}\marg{false branch}}\CHDocNew{0.6}
This macro tests if a counter is under the administration of the coupled counter commands and expands to the relevant branch then. There are two short-circuit commands \refCom{IsCoupledCounterT} and \refCom{IsCoupledCounterF}.
\end{docCommand}

\begin{docCommand}{IsCoupledCounterT}{\marg{counter name}\marg{true branch}}\CHDocNew{0.6}
This macro tests if a counter is under the administration of the coupled counter commands and executes the true branch then. There are two related commands \refCom{IsCoupledCounterTF} and \refCom{IsCoupledCounterF}.
\end{docCommand}

\begin{docCommand}{IsCoupledCounterF}{\marg{counter name}\marg{false branch}}\CHDocNew{0.6}
This macro tests if a counter is under the administration of the coupled counter commands and executes the false branch then if this is not the case. There are two related commands \refCom{IsCoupledCounterTF} and \refCom{IsCoupledCounterT}.
\end{docCommand}

\clearpage
\section{Periodic counters}\label{section:periodic_counters}\CHDocNew{0.9}


It might be very convenient to have counters that are automatically reset not only by a driving counter such as \CHDocCounter*{chapter} but also periodically, i.e. after a certain amount of steps -- this can be achieved with the concept of periodic counters.

\subsection{Commands related to periodic counters setup}

\begin{docCommand}[before={\CHDocNew{0.9}}]{DeclarePeriodicCounter}{\oarg{}\marg{counter name}\marg{counter treshold value}}
This defines the counter given in the first mandatory argument as a periodic counter and is automatically reset if the treshold value is reached. 

\begin{marker}
The command \refCom{DeclarePeriodicCounter} does not define a new counter, however but is the preamble-only version of \refCom{AddPeriodicCounter}. 
\end{marker}

Please note that in case of \cs{addtocounter} applied to a periodic counter the value to be added leads to a modulo division such that the counter might be reset if the addition would increase the counter beyond the treshold value, the module part will be added then. In order to prevent this wrapping, use the \refKey{wrap} option to \refCom{addtocounter}: 

\begin{dispExample}
  \setcounter{foocntr}{3}% 
  \AddPeriodicCounter{foocntr}{8}% 
  
  Value of foocntr is: \thefoocntr  % Should be 3

  \addtocounter{foocntr}{20} % Is it 23? No, it is 23 % 8 = 7
  Value of foocntr is \thefoocntr\ now! 

  Adding a value of 4 again: 
  \addtocounter{foocntr}{4} % Is it 11? No, it is 11 % 8 = 3
  Value of foocntr is \thefoocntr\ now! 

  Now prevent the wrapping
  \addtocounter{foocntr}{10}[wrap=false] % Is it 13? Yes, it is!
  Value of foocntr is \thefoocntr\ now! 

\end{dispExample}
\end{docCommand}

\begin{docCommand}[before={\CHDocNew{0.9}}]{AddPeriodicCounter}{\oarg{}\marg{counter name}\marg{counter treshold value}}
This defines the counter given in the first mandatory argument as a periodic counter and is automatically reset if the treshold value is reached. 
\end{docCommand}


\begin{docCommand}[before={\CHDocNew{0.9}}]{RemovePeriodicCounter}{\oarg{options}\marg{counter name}}
This removes the counter given in the first mandatory argument as a periodic counter. The counter is reset unless the \refKey{reset} is set to \meta{false}. 

\begin{codeoptionsenum}
\item \oarg{options} 

  As of version \packageversion, there is only one option: 
  
  \begin{docKey}[][before={\CHDocNew{0.9}}]{reset}{=\meta{true/false}}{initially true}
      Use `false` to prevent the resetting of the relevant counter after removal! 
    \end{docKey}
  \item \marg{counter name} -- the name of the counter that should be no periodic counter any longer. 
\end{codeoptionsenum}

If all periodic counters should be removed, use the macro \refCom{RemoveAllPeriodicCounters} instead. 

\end{docCommand}

\begin{docCommand}[before={\CHDocNew{1.0}}]{RemoveAllPeriodicCounters}{\oarg{options}}
  This command removes all counters given in the first mandatory argument as a periodic counter. All counters are reset unless the \refKey{reset} option is set to \meta{false}. 

\begin{codeoptionsenum}
\item \oarg{options} 
  As of version \packageversion, there is only one option: \refKey{reset}, having the same meaning as in \refCom{RemovePeriodicCounter}.
\item \marg{counter name} -- the name of the counter that should be no periodic counter any longer. 
\end{codeoptionsenum}

If only a specific counter shall be removed from the periodic counter property use the command \refCom{RemovePeriodicCounter} instead. 

\end{docCommand}


\begin{docCommand}[before={\CHDocNew{0.9}}]{ChangePeriodicCounterCondition}{\oarg{options}\marg{counter name}\marg{new counter treshold value}}
This changes the counter treshold condition -- the counter is reset automatically if not specified otherwise with the \refKey{reset} option. 

\begin{codeoptionsenum}
\item \oarg{options} 
  
  As of version \packageversion, there is only one option: \refKey{reset}, which serves the same functionality as in \refCom{RemovePeriodicCounter}. 
\item \marg{counter name} -- the name of the counter that should be no periodic counter any longer. 
\item \marg{new counter value treshold} -- the new value after which an automatic resetting will occur. 

\end{codeoptionsenum}

\end{docCommand}

\subsection{Commands to query for periodic counter feature}


\begin{docCommand}[after={\CHDocNew{0.9}\par}]{IsPeriodicCounterTF}{\marg{counter name}\marg{true branch}\marg{false branch}}
This macro tests if a counter is under the administration of the periodic counter commands and expands to the relevant branch then. There are two short-circuit commands: \refCom{IsPeriodicCounterT} and \refCom{IsPeriodicCounterF}.
\end{docCommand}

\begin{docCommand}[before={\CHDocNew{0.9}}]{IsPeriodicCounterT}{\marg{counter name}\marg{true branch}}
This macro tests if a counter is under the administration of the periodic counter commands and expands to the \meta{true} branch then. There are two related commands: \refCom{IsPeriodicCounterTF} and \refCom{IsPeriodicCounterF}.
\end{docCommand}

\begin{docCommand}[before={\CHDocNew{0.9}}]{IsPeriodicCounterF}{\marg{counter name}\marg{false branch}}
This macro tests if a counter is under the administration of the periodic counter commands and expands to the \meta{false} branch then if this is not the case. There are two related commands: \refCom{IsPeriodicCounterTF} and \refCom{IsPeriodicCounterT}.
\end{docCommand}



\section[Suspending and Resuming]{Suspending and resuming (associated) counters}\label{section::suspendedresumedcounters}



Rather than removing an associated counter from the list, it is possible to suspend the automatic stepping for a while and then resume it (or completely drop it), for example, if the value of a counter should not be stepped within a specific chapter etc. 



\begin{marker}[before=\CHDocNew{0.8}\par]
  Suspension and resuming counters can cause wrong hyper links if \CHDocPackage{hyperref} is used. 
\end{marker}

\subsection{Macros for suspension and resume}

\begin{docCommand}{SuspendCounters}{\oarg{options}\marg{counters list}}%
\begin{docCommandArgs}{SuspendCounters}%

\begin{codeoptionsenum}
\item \oarg{options}% 
  
  Not used so far, reserved for later usage.

\item \marg{counters list}%

  Contains the name of counters to be suspended, separated by commas (CSV - list)
\end{codeoptionsenum}
\end{docCommandArgs}
\end{docCommand}%

\begin{docCommand}{CascadeSuspendCounters}{\oarg{options}\marg{counters list}}\CHDocNew{0.8}

This macro is more powerful than \refCom{SuspendCounters}, since it tries to detect whether a counter has a reset list and 'mutes' the counters on this list as well and checks whether those counters themselves have reset lists and cascades down to the final state. 

\begin{marker}
  Stated differently: All counters anyhow connected to a counter named \CHDocCounter{foo} will be suspended, e.g. for the \CHDocClass{book} class and \CHDocCounter{chapter}, this means in a standard setup, that \CHDocCounter*{section,figure,table,equation,footnote} will be suspended, as well as in consequence \CHDocCounter*{subsection,subsubsection,paragraph,subparagraph}, assuming hereby no other counters have been added to the reset lists.
\end{marker}


\begin{docCommandArgs}{CascadeSuspendCounters}%

\begin{codeoptionsenum}
\item \oarg{options}% 

  Not used so far, reserved for later usage.

\item \marg{counters list}%

  Contains the name of counters to be suspended, separated by commas (CSV - list)
\end{codeoptionsenum}
\end{docCommandArgs}
\end{docCommand}%


\begin{docCommand}{ResumeSuspendedCounters}{\oarg{options}\marg{counters list}}
  As of version \packageversion\ the optional argument is not used and reserved for later purposes. 
  This command revokes the suspension of the counters in the \marg{counters} list.
\end{docCommand}

\begin{docCommand}{ResumeAllSuspendedCounters}{\oarg{options}} \CHDocNew{0.8}
  As of version \packageversion\ the optional argument is not used and reserved for later purposes. 
  This command revokes all suspended counters.
\end{docCommand}


\subsection{Query suspension}
\begin{docCommand}[before={\CHDocNew{0.1}}\par]{IsSuspendedCounter}{\marg{counter name}\marg{true branch}\marg{false branch}}
See \nameref{section::suspendedresumedcounters} on this topic. 

This command checks, whether a counter is suspended, i.e. not updated at all and expands the corresponding branches.

%\begin{docCommandArgs}{IfSuspendedCounter}%

\begin{codeoptionsenum}
\item \marg{counter name}%

  Contains the name of counter presumed to be suspended

\item \marg{True branch}

  This code is expanded if the counter is suspended, otherwise it is ignored

\item \marg{True branch}

  This code is expanded if the counter is \textbf{not} suspended, otherwise it is ignored

\end{codeoptionsenum}
%\end{docCommandArgs}


\end{docCommand}



\begin{marker}
If a driver counter is suspended, all counters associated to it are suspended too!
\end{marker}

\setcounter{totalequations}{0}
\setcounter{equation}{0}
\renewcommand{\DeclareAssociatedCounters}[3][]{\relax}%
\begin{dispExample}
\textbf{This example shows 4 equations, but only two of them are counted}

\begin{equation}
E_{0} = mc^2
\end{equation}

Now suspend the equations:

\SuspendCounters{equation}
\begin{equation}
E^2 = \left({ pc}\right)^2 + E^{2}_{0}
\end{equation}

\begin{equation}
  m(v) = \frac{m_{0}}{\sqrt{1-\frac{v^2}{c^2}}} 
\end{equation}

And resume it: \ResumeSuspendedCounters{equation}

\begin{equation}
  E = h \nu
\end{equation}

There are \number\totvalue{totalequations}~equations in here!

\end{dispExample}









\section{Total counters}\tcbdocmarginnote{\tcbdocnew{2015-11-27}}\label{sec::totalcounters}



Similarly to the package \CHDocPackage{totcount} or the features of \CHDocPackage{totalcount} by Axel Sommerfeldt this package provides the possibility of defining a counter that stores its finally value to the auxiliary file and starts from this value then, if not set otherwise to another value. 

The declaration of a total counter is a preamble - only event and \refCom{DeclareTotalDocumentCounter} is a preamble-only command in order to prevent counter register confusion. If a certain existing counter should be treated with total counter features, use \refCom{RegisterTotalDocumentCounter} instead.  

Use \refCom{NewTotalDocumentCounter} only in rare cases, if a total counter must be defined within the document body. 

\begin{marker}
\tcbdocmarginnote{\bcbombe}The standard \LaTeXe\ commands \cs{stepcounter}, \cs{addtocounter} and \cs{setcounter} support the specification of a total counter, but \cs{refstepcounter} will fail since the usage of a total counter for labelling purposes is most probably of no use (as of version \packageversion)
\end{marker}

\subsection{Defining total counters}

\begin{docCommand}{RegisterTotalDocumentCounter}{\oarg{options}\marg{total counter name}}\CHDocNew{0.5}
\begin{codeoptionsenum}
  \item \oarg{options}: As of version \packageversion, only this option is used
    \begin{docKey}{supertotal}{=\meta{true,false}}{false}
      Set this key to switch the super total counter on or off. 
    \end{docKey}
    \item \marg{total counter name}: The name of the total counter. This must be the same like the name of an already existing counter. Internally another counter is defined which has a prefix to prevent name clashes with counter names defined by the package \CHDocPackage{totalcount}. If the counter name does not exist, the compilation exits with an error message. 
\end{codeoptionsenum}


\end{docCommand}




\begin{docCommand}{TotalCounterInternalName}{\marg{counter name}}
This command reports the internal name of a total counter or the usual name if this counter is not a total one. 
\end{docCommand}

\begin{dispExample}
\TotalCounterInternalName{chapter}

\TotalCounterInternalName{foototal}
\end{dispExample}

\begin{docCommand}[before={\CHDocExpCommand\par\CHDocNew{1.0}}]{TotalCounterInternalNameExp}{\marg{counter name}}
This command is the expandable version of \refCom{TotalCounterInternalName}
\end{docCommand}


\stepcounter{foototal}

\begin{docCommand}[before={\CHDocExpCommand}]{TotalValue}{\marg{counter name}}\CHDocUpdate{0.9}%
This command prints the value of a total counter or falls back to the value of the counter if this is not a total counter. 

\begin{dispExample}
``Total'' value of the section non-total counter: \TotalValue{section}

Total value of the foototal total counter: \TotalValue{foototal}
\end{dispExample}


\end{docCommand}

\subsection{Queries about total counters}

\begin{docCommand}{IsTotalCounterTF}{\marg{counter name}\marg{true branch}\marg{false branch}}
This macro tests if a counter is under the administration of the total counter commands and expands to the relevant branch then. There are two short-circuit commands \refCom{IsTotalCounterT} and \refCom{IsTotalCounterF}.
\end{docCommand}




\begin{docCommand}{IsTotalCounterT}{\marg{counter name}\marg{true branch}}
This macro checks if a counter is under the administration of the total counter commands and expands to the code in the second argument if this is true.
\end{docCommand}


\begin{docCommand}{IsTotalCounterF}{\marg{counter name}\marg{false branch}}
This macro checks if a counter is under the administration of the total counter commands and expands to the code in the second argument if this is not the case.
\end{docCommand}

\begin{dispExample}
  \IsTotalCounterTF{foototal}{Yes, this is a total counter}{No, this is no total counter}

  \IsTotalCounterTF{section}{Yes, this is a total counter}{No, this is no total counter}

  \IsTotalCounterT{foototal}{Yes, this is a total counter}

  \IsTotalCounterF{page}{No, page isn't a total counter}

\end{dispExample}

\begin{marker}
  The features of using other \CHDocFiles{aux} files or a different external file as provided by \CHDocPackage{totcount} is not (yet) support as of version \packageversion. 
\end{marker}



\section{Super total counters}\label{section:supertotalcounters} \tcbdocmarginnote{\tcbdocnew{2015-11-25}}
In addition to the concept of a total counter, there is also the possibility of using super total counters -- those counters survive the reset at the beginning of a compilation, i.e. the value of a super total counter might be stepped in each run and as such the number of compilation runs etc. can be tracked. The values of the last run are persistent as long as the \CHDocFileExt{.aux} file isn't deleted. 

\subsection{Defining super total counters}

\begin{docCommand}{NewTotalDocumentCounter}{\oarg{options}\marg{total counter name1, total counter name2,\dots}}
This macro defines a new counter (which mustn't exist before of course) and puts it under control of the total counter features. 
\begin{codeoptionsenum}
  \item \oarg{options}: As of version \packageversion, only this option is used

    \refKey{supertotal} -- this has the same meaning as in \refCom{RegisterTotalDocumentCounter} and defaults to \texttt{false}. 

  \item \marg{total counter name1, total counter name2,\dots}: The names of the total counter, separated by a comma. This must not be the same like the name of any already existing counter. Internally another counter is defined which has a prefix to prevent name clashes with counter names defined by the package \CHDocPackage{totcount}. 
  \end{codeoptionsenum}

\end{docCommand}

\begin{docCommand}{DeclareTotalDocumentCounter}{\oarg{options}\marg{total counter name1, total counter name2,\dots}}

This is the preamble - only version of \refCom{NewTotalDocumentCounter} and should be preferred in most cases over that command. 

If an already existing counter should be tracked with total counter features, 
use \refCom{RegisterTotalDocumentCounter} instead. 

\CHDocUpdate{1.2}This command allows multiple counters (specified as a comma separated list) to be defined at once.

\end{docCommand}


\subsection{Queryies about super total counters}

\begin{docCommand}{IsSuperTotalCounterTF}{\marg{counter name}\marg{true branch}\marg{false branch}}
This macro tests if a counter is under the administration of the super total counter commands and expands to the relevant branch then. There are two short-circuit commands \refCom{IsSuperTotalCounterT} and \refCom{IsSuperTotalCounterF}.
\end{docCommand}




\begin{docCommand}{IsSuperTotalCounterT}{\marg{counter name}\marg{true branch}}
This macro checks if a counter is under the administration of the super total counter commands and expands to the code in the second argument if this is true.
\end{docCommand}


\begin{docCommand}{IsSuperTotalCounterF}{\marg{counter name}\marg{false branch}}
This macro checks if a counter is under the administration of the super total counter commands and expands to the code in the second argument if this is not the case.
\end{docCommand}


\begin{dispExample}
  \IsSuperTotalCounterTF{numberofruns}{Yes, this is a super total counter}{No, this is no super total counter}

  \IsSuperTotalCounterT{numberofruns}{Yes, this is a super total counter}

  \IsSuperTotalCounterTF{chapter}{Yes, this is a total counter}{No, this is no super total counter}
\end{dispExample}

\subsection{The \CHDocCounter{numberofruns} counter}\label{subsection:numberofruns}

This package adds a counter of its own: \CHDocCounter{numberofruns} which is a super total counter and is stepped each compilation run. It's added in \cs{AtBeginDocument} and can be retrieved with \refCom{TotalValue}. Use the \refKey{nonumberofruns} package option to prevent the definition of this counter.  



\section{Experimental features}\label{subsec::associated_counters_experimental}


\begin{marker}
\marginnote{\bcbombe}
The content here is only of experimental nature and there is no guarantee that the feature will be maintained in future releases. 
\end{marker}


\subsection{Labels}



\begin{marker}
To enable the redefined \refCom{label} macro, specify the package option \refKey{redefinelabel}
\end{marker}



\begin{docCommand}[before={\CHDocNew{1.2}}]{label}{\oarg{cleveref-counter-overrule}\marg{label name}\oarg{options for associated counters}}
  \begin{docKey}{all}{=\meta{true,false}}{initially \meta{false}}\CHDocNew{1.2}
    This will enable that all associated counters to a driver counter will cause the generation of a label too. By default this option is \meta{false}.
    This option deliberately overrules \refKey{select}, the value of the option \refKey{prefix} is disregarded.
  \end{docKey}
  \begin{docKey}{select}{={counter1, counter2,\dots}}{initially empty}\CHDocNew{1.2}
    Select only some of the associated counters to be able to be labeled. 
    As of version \packageversion{} there is no check whether the given names refer to counters at all or are associated counters to the last counter that has been used with \cs{refstepcounter}. 
  \end{docKey}
  \begin{docKey}{prefix}{=\meta{text}}{initially empty}\CHDocNew{1.2}
    This gives the prefix of the label of the associated counter. If the option \refKey{all} is enabled, the label name is generated from the name of the associated counter, the value of \refKey{prefix-sep} and the value of the 2nd argument. 
    \newcounter{morefoobar}
    \begin{dispExample*}{listing only}
      % Assume that some counter has the associated counters foobar, morefoobar and yetanotherfoobar
      \label{foo}[prefix=assoc,all]
      \end{dispExample*}
      will cause a label named \texttt{foobar::foo}, \texttt{morefoobar::foo} and \texttt{yetanotherfoobar::foo}, whereas
      \begin{dispExample*}{listing only}
        \label{foo}[prefix=assoc,select=morefoobar]
      \end{dispExample*}
      would generate the label \texttt{assocc::foobar} only and will be tied to the value of the counter \texttt{morefoobar}
  \end{docKey}
  \begin{docKey}{prefix-sep}{=\meta{text}}{initially ::}\CHDocNew{1.2}
    Defines the separator between the \refKey{prefix} and the label name for the driver counter, i.e. the  2nd argument of the \refCom{label} command. 
  \end{docKey}
\end{docCommand}


\begin{marker}
\marginnote{\bcbombe}
If the package option \refKey{redefinelabel} is set to \meta{false}, the usage of the third optional argument will leave spurious content at the position \cs{label} was used. The reason is that the content of third optional argument with \texttt{[]} is not recognized as an argument any longer. 
\end{marker}


\begin{docCommand}[before={\CHDocNew{1.3}}]{LaTeXLabel}{\oarg{cleveref-counter-overrule}\marg{label name}}
  This is the default label macro, either with the \CHDocPackage{cleveref} extension or the classical \LaTeX2e\ macro (eventually modified by \CHDocPackage{hyperref}) and is not modified by this package. 


The feature of label hooks from \refCom{RegisterPreLabelHook} or \refCom{RegisterPostLabelHook} is not used here. 
\end{docCommand}


\subsection{Hooks}


This feature is experimental and only realized for the modified \refCom{label} command until now. See \nolinkurl{xassoccnt_getparentcounter_example.tex} as an example of usage. 

\begin{docCommand}[before={\CHDocNew{1.3}}]{RegisterPreLabelHook}{\marg{command name1, command name2,\dots}}

This macro declares a possible list of hooks (commands) that should be executed \textbf{before} the traditional \cs{label} command is applied. If the hook name refers to some unknown macro, nothing is done. 

The hook names must be given with the \textbackslash\ as command sequence indicator, i.e. \cs{zlabel}, more than one macro name is possible by using comma as separator. 

As of version \packageversion\ the hook macro does not allow more than one argument, which is automatically used from the surrounding \cs{label} call and is the usual label name. 


\end{docCommand}

\begin{docCommand}[before={\CHDocNew{1.3}}]{RegisterPostLabelHook}{\marg{command name1, command name2,\dots}}

This macro declares a possible list of hooks (commands) that should be executed after the traditional \cs{label} command is applied. If the hook name refers to some unknown macro, nothing is done. 

The hook names must be given with the \textbackslash\ as command sequence indicator, i.e. \cs{zlabel}, more than one macro name is possible by using comma as separator. 

As of version \packageversion\ the hook macro does not allow more than one argument, which is automatically used from the surrounding \cs{label} call and is the usual label name. 

\end{docCommand}





\clearpage
\part{Meta-Information}

\parttoc

\clearpage
\section{To - Do list}

\begin{itemize}
\item Merging of counter groups, removing counters from counter groups
\item Backup and restoration of individual counters not being member of a counter group
\item Switch to the container support for all features -- this is a major task and will be done in (tiny) steps. 
\item Better counter definition/copy counter routines \(\longrightarrow\) another package perhaps
\item More examples 
\item Some macro names might be non-intuitive
\item Improve documentation
\item Hooks for conditionals on \CHDocCounter*{numberofruns} (see \cref{subsection:numberofruns})


\end{itemize}

Some issues that have been addressed partially are:

\begin{itemize}
\item \CHDocNew{1.0} Add counter group support for the \CHDocTag{backup} feature, i.e. define a symbolic name for a group of counters that should be controlled by the backup feature. This will allow multiple backup groups, which might be necessary. 
\end{itemize}


If you 

\begin{itemize}
  \item find bugs
  \item errors in the documentation
  \item have suggestions
  \item have feature requests
\end{itemize}

don't hesitate and contact me using my mail address: \mymailtoaddress.

\clearpage

\section{Acknowledgments}

I would like to express my gratitudes to the developpers of fine \LaTeX{} packages and of course
to the users at tex.stackexchange.com, especially to

\begin{itemize}
  \item Paulo Roberto Massa Cereda
  \item Enrico Gregorio
  \item Joseph Wright
  \item David Carlisle
  \item Werner Grundlingh
  \item Gonzalo Medina
  \item Cosmo Huber (for providing the bug report with the \CHDocPackage{calc} package.)
\end{itemize}

for their invaluable help on many questions on macros.

\vspace{2\baselineskip}
\begin{marker}
A special gratitude goes to Prof. Dr. Dr. Thomas Sturm for providing the wonderful \CHDocPackage{tcolorbox} package which was used to
write this documentation.
\end{marker}

\clearpage

\section{Version history}


\begin{itemize}[itemsep=15pt]

\item \CHDocFullVersion{1.4}

\begin{itemize}
 \item Improved the core macros \cs{refstepcounter} and \cs{stepcounter} in order to fit the \CHDocPackage{expl3} and \CHDocPackage{xparse} changes of Februar - April 2017. 
  \item Added following experimental features:
    \item \refCom{CounterFormat} with quick and possible recursive change of the counter output
    \item \refCom{StoreCounterFormats}, \refCom{AddCounterFormats} and \refCom{RemoveCounterFormats} for defining own short hand counter formats.
    \item Provided the macros \refCom{xalphalph} and \refCom{xAlphAlph} in order to allow counter output in the same manner as the \CHDocPackage{alphalph} does. 
  \item Added the macros \refCom{CounterWithin}, \refCom{CounterWithin*}, \refCom{CounterWithout} and \refCom{CounterWithout*} which provides a quicker access to add or remove counters from the reset list and changing the corresponding \cs{the...} macros. 
  \item Provided the \refCom{LoopCounterResetList} to perform the same action on all counters being in the reset list of a given counter. 
  \item New macros \refCom{ClearCounterResetList} and \refCom{ClearCounterResetList*} to remove all counters on first level of a driver counter. 
  \item Added the explanation (missing in previous versions) to the documentation that the \refCom{AddToReset}, \refCom{RemoveFromReset} and \refCom{RemoveFromFullReset} macros actually support a comma separated list of counter names for the first argument. 
\end{itemize}

\item   \CHDocFullVersion{1.3}
\begin{itemize}
  \item Provided the \refCom{LaTeXLabel} macro to access the non-xassoccnt version of the \refCom{label} command. 
  \item \CHDocExperimentalFeature Added the concept of label hooks, see \cref{subsec::associated_counters_experimental} for more information.
  \item The macros \cs{Last...} are defined with \CHDocPackage{expl3} methods. 
  \item Added \refCom{GetAllResetLists} and \refCom{GetParentCounter} for information on parent (or driver) counters. 
\end{itemize}

\item   \CHDocFullVersion{1.2}
\begin{itemize}
  \item Corrected some typos in the manual. 
  \item The macros \refCom{NewDocumentCounter}, \refCom{DeclareDocumentCounter} and \refCom{NewTotalDocumentCounter} allow multiple counters to be specified and defined. 
  \item Added the macro \refCom{DeclareTotalAssociatedCounters} in order to combine total counters and the associated feature, i.e. the counters are total ones and associated to a driver counter. 
  \item \CHDocExperimentalFeature An extended version of \refCom{label} is provided to allow labels also for associated counters during the stepping process of the driver counter. 
\end{itemize}

\item   \CHDocFullVersion{1.1}
\begin{itemize}
  \item Added some missing basic functions needed after the more restrictive \CHDocPackage{expl3} update from 2016/10/19
  \item Added a statement about the requirement to load \CHDocPackage{tcolorbox} before \CHDocPackage{xassoccnt} in the documentation (i.e. this file!)
\end{itemize}
\item   \CHDocFullVersion{1.0}
\begin{itemize}
\item \CHDocNew{1.0} Restructured the \CHDocPackage{xassoccnt} manual file.
\item \CHDocNew{1.0} Added some improvements for counter reset lists macros
\item \CHDocNew{1.0} Added new backup/restore features, with cascading counters possibility -- the old backup/restore macros are still available but renamed with a prefix \cs{Former...}
\item \CHDocNew{1.0} Added the \refCom{RemoveAllPeriodicCounters} -- it was missing in the \CHDocTag{periodic counter} features -- see \nameref{section:periodic_counters} for more information on this. 
\item \CHDocNew{1.0} Added the expandable version of \refCom{TotalCounterInternalName} named 

  \refCom{TotalCounterInternalNameExp}.
\end{itemize}
\item   \CHDocFullVersion{0.9}
  \begin{itemize}
  \item \CHDocUpdate{0.9}\refCom{TotalValue} is an expandable command now. 
    \item \CHDocNew{0.9} Added the \CHDocTag{periodic counter} features -- see \nameref{section:periodic_counters} for more information on this. 
  \end{itemize}
\item   \CHDocFullVersion{0.8}
  \begin{itemize}[label={$\checkmark$}]
  \item \CHDocUpdate{0.8}Fixed the \refCom{SuspendCounters} and \refCom{ResumeSuspendedCounters} macros -- the comma separated list of counters was not used (contrary to the purpose and the documentation description). 
  \item \CHDocNew{0.8}Additions of commands
    \begin{itemize}[label={$\triangleright$}]
    \item \refCom{ResumeAllSuspendedCounters}
    \item  \refCom{CascadeSuspendCounters}
    \item  \refCom{DisplayResetList}
    \item  \refCom{ShowResetList}
    \end{itemize}
  \end{itemize}
\item \CHDocFullVersion{0.7}
  \leavevmode
  \begin{itemize}[label={$\checkmark$}]
  \item\CHDocUpdate{0.7} Fixed a small bug in the \CHDocPackage{\PackageDocName} version of \cs{stepcounter}
  \item Added some macros that support the output of binary, octal or hexadecimal (both lower/uppercase) values of counters.  \CHDocNew{0.7}
  \item Added the \cs{Loop...Counters} macros that perform an action in loop on all given counter names. \CHDocNew{0.7}
  \end{itemize}
  
\item   \CHDocFullVersion{0.6}
  \begin{itemize}[label={$\checkmark$}]
  \item The coupled counters allow to specify a counter group to which all relevant counters belong, this allows several coupled counter groups then  \CHDocNew{0.6}
  \item Fixed a small bug within backup counter support -- the resetting was not done any more \CHDocNew{0.6}
  \item Added the \CHDocKey{nonumberofruns}\ package option.
  \end{itemize}
\item
  \CHDocFullVersion{0.5}
  \begin{itemize}[label={$\checkmark$}]
  \item Added support (very experimental!) for the \CHDocTag{coupled counters} feature, see \cref{sec::coupledcounters} about this feature! \CHDocNew{0.5}
  \item Added \cs{RegisterTotalDocumentCounter} and improved \cs{TotalValue} support\CHDocNew{0.5}
  \end{itemize}
\item
  \CHDocFullVersion{0.4}
  \begin{itemize}[label={$\checkmark$}]
  \item Added \cs{BackupCounterValues} and \cs{RestoreCounterValues} support\CHDocNew{0.4}
  \item Added \cs{StepDownCounter} and \cs{SubtractFromCounter} macros\CHDocNew{0.4}
  \end{itemize}
\item 
  \CHDocFullVersion{0.3}\CHDocUpdate{0.3}
  \begin{itemize}[label={$\checkmark$}]
  \item Added the \CHDocTag{totalcounter} features similar to the packages \CHDocPackage{totcount} or \CHDocPackage{totalcount} \tcbdocmarginnote{\tcbdocnew{2015-11-11}}
  \item Added the \CHDocTag{super total counter} features \tcbdocmarginnote{\tcbdocnew{2015-25-11}}
  \item Added the \CHDocCounter*{numberofruns} counter \tcbdocmarginnote{\tcbdocnew{2015-25-11}}
\end{itemize}
\item \CHDocFullVersion{0.2}\CHDocUpdate{0.2}

Improved \cs{stepcounter} to remove some incompatibilities with the \CHDocPackage{perpage}. This is only partially managed so far.  

\item 
\CHDocFullVersion{0.1}

  A major bug fixed due to some error in usage together with \CHDocPackage{calc} when the driven counters are not stepped any longer. 
  The culprit was in \CHDocPackage{assoccnt} that the counter reset list was not really disabled. 
  
  Thanks to this question \url{http://tex.stackexchange.com/questions/269731/calc-breaks-assoccnt} this bug was detected. 
  
  This however lead to some internal inconsistencies and it was decided to rewrite \CHDocPackage{assoccnt} with \CHDocPackage{expl3} and the features of the new \LaTeX\ 3 - Syntax. 
  
\end{itemize}
\clearpage
\phantomsection
\part{Appendix}\label{examplesappendix}
\setcounter{section}{0}
\renewcommand{\theHsection}{appendix.\thesection}
\renewcommand{\thesection}{\Alph{section}}

Note: The \cs{DeclareAssociatedCounters} command has to be used in the preamble of the document. It's missing here for the sake of a compact example. 



\section[Total number of sections]{Example: Total number of sections}
In this example, all sections of this document are counted, i.e. the current one as well as all following ones.
\begin{dispExample}
This document has \total{totalsections} section(s)%
\end{dispExample}



\section[Subsection with suspension]{Example: Total number of subsections with suspension}

In this example, the subsections of this document are counted but later on, the associatedcounter is removed from the list, so it is frozen.


\begin{dispExample}

\subsection{First dummy subsection}
SubSection counter: \thesubsection~-- \number\totvalue{totalsubsections}
\subsection{Second dummy subsection}
SubSection counter: \thesubsection~-- \number\totvalue{totalsubsections}

\RemoveAssociatedCounter{subsection}{totalsubsections}%
\subsection{Third dummy subsection after removing the associated counter}

SubSection counter: \thesubsection~-- \number\totvalue{totalsubsections}

\end{dispExample}



\subsection{Suspension of a non-associated counter}
This example will show the suspension of a non-associated counter


\begin{dispExample}
\setcounter{equation}{0}%
\SuspendCounters{equation}%
\begin{equation}
E_{0} = mc^2
\end{equation}

\begin{equation}
E^2 = \left({ pc}\right)^2 + E^{2}_{0}
\end{equation}

\begin{equation}
  m(v) = \frac{m_{0}}{\sqrt{1-\frac{v^2}{c^2}}} 
\end{equation}


There are \number\value{equation}~equations in here!
\end{dispExample}


\section[Former Backup/Restore Feature]{Former backup and restore of counter values}\label{section::old_backuprestore} \CHDocUpdate{1.0}

\subsection[Macros for backup/restoration]{Description of backup and restoring macros for counter values}

\begin{docCommand}{FormerBackupCounterValues}{\oarg{options}\marg{counter name1, counter name2,...}}
  This macro adds counter names (separated by a comma) to a list and stores the current values of the counters to another list. The values are used from the current state where this command is used, not a previous or a later state is stored.  
  
  \begin{itemize}
  \item All counters in the list will be reset to zero (after storing the values) for the next usage, unless the \refKey{resetbackup} key is set to \meta{false}.
  \item Multiple specification of the same counter name is possible, but only the first occurence will be regarded -- consecutive occurences of the same counter name are not taken into account. \CHDocNew{0.5}
  \end{itemize}

\begin{docKey}[][]{resetbackup}{=\meta{true/false}}{initially true}
This key decides whether \textbf{all} counters in the backup list should be reset to zero or should keep the current value. The default value is \meta{true}.
\end{docKey}

Please note: If a name does not belong to a counter register the compilation aborts with an error message!

Some remarks

\begin{marker}
If a specific counter name is suffixed with an \textasteriskcentered\ at its end the resetting is disabled for this particular counter, regardless whether \refKey{resetbackup} is set to true or not.\CHDocNew{0.4}
\end{marker}

\begin{marker}
  Strangely enough, a counter name like \CHDocCounter{foo*} is possible, but \cs{thefoo*} would fail. Be careful about choosing counter names for new counters -- just restrict yourself to the usual letters (and if really needed, using \makeatletter @\makeatother)
\end{marker}
\end{docCommand}%


\begin{docCommand}{FormerRestoreAllCounterValues}{\oarg{options}} \CHDocNew{0.5}
This macro restores all stored counter values corresponding to the counter names. 

As of version \packageversion\ the optional argument isn't used and reserved for later purposes. 

The backup list is cleared after the restoring has been finished. 
\end{docCommand}

\begin{marker}
The command \refCom{FormerRestoreAllCounterValues} was previously called \refCom{FormerRestoreCounterValues} -- that macro is now reserved for updating only particular
counters, not all in a row.
\end{marker}



\begin{docCommand}{FormerRestoreCounterValues}{\oarg{options}\marg{counter name1,counter name2,...}} \CHDocUpdate{0.5}
This macro restores only the stored counter values given by the counter names. 
As of version \packageversion\ the optional argument isn't used and reserved for later purposes. 

%The backup list is cleared after the restoring has been finished. 
\end{docCommand}

\begin{dispExample}

  \captionof{figure}{A dummy figure}

  \captionof{table}{A dummy table}


\FormerBackupCounterValues{figure,table*}

  \captionof{figure}{Another dummy figure}

  \captionof{table}{Another dummy table}

  \captionof{figure}{Even another dummy figure}

  \captionof{table}{Even another dummy table}

Before restoring: \thefigure\ and \thetable

\FormerRestoreAllCounterValues

Restored the values: \thefigure\ and \thetable

\captionof{figure}{Yet another dummy figure}
\captionof{table}{Yet another dummy table}


\end{dispExample}

\begin{docCommand}{FormerAddBackupCounter}{\oarg{options}\marg{counter name1,counter name2,...}} \CHDocNew{0.5}
This is similar to \refCom{FormerBackupCounterValues}, but adds the counter names to an existing global list and can be applied after \refCom{FormerBackupCounterValues}. 

\end{docCommand}


\begin{docCommand}{FormerRemoveBackupCounters}{\oarg{options}\marg{counter name1, counter name2,...}} \CHDocNew{0.5}
This macro removes the counters from the list of backup counters and immediately restores the counter value unless the starred version \refCom{FormerRemoveBackupCounters*} is used. 

If the package \CHDocPackage{hyperref} is used, the macro \cs{theH...} (see \nameref{subsec::backup_and_hyperref} on this) is restored to the original meaning. 

As of version \packageversion\ the optional argument isn't used and reserved for later purposes. 

\end{docCommand}



\begin{docCommand}{FormerRemoveBackupCounters*}{\oarg{options}\marg{counter name}} \CHDocNew{0.5}

This command is basically similar to \refCom{FormerRemoveBackupCounters}, but does not restore the counter value right at the place the macro is used.  

As of version \packageversion\ the optional argument isn't used and reserved for later purposes. 
\end{docCommand}



\clearpage
\markboth{\indexname}{\indexname}
%%%% Index of commands etc. 
\printindex

\end{document}