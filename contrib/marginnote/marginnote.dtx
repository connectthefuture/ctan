% \CheckSum{859}
% \iffalse meta-comment
% ======================================================================
% marginnote.dtx
% Copyright (c) Markus Kohm, 2005-2017
%
% This file is part of the work marginnote.
%
% This work may be distributed and/or modified under the conditions of
% the LaTeX Project Public License, version 1.3c of the license.
% The latest version of this license is in
%   http://www.latex-project.org/lppl.txt
% and version 1.3c or later is part of all distributions of LaTeX
% version 2005/12/01 or later and of this work.
%
% This work has the LPPL maintenance status "maintained".
%
% The Current Maintainer and author of this work is Markus Kohm.
%
% This work consists of the files marginnote.dtx and the
% derived files README.txt and marginnote.sty.
% ======================================================================
%
%<package>%%% From File: $Id: marginnote.dtx 13 2017-04-22 09:25:22Z mjk $
%<*dtx>
\ifx\ProvidesFile\undefined\def\ProvidesFile#1[#2]{}\fi
\begingroup
  \def\filedate$#1: #2-#3-#4 #5${\gdef\filedate{#2/#3/#4}}
  \filedate$Date: 2017-04-22 11:25:22 +0200 (Sa, 22 Apr 2017) $
  \def\filerevision$#1: #2 ${\gdef\filerevision{#2}}
  \filerevision$Revision: 13 $
\endgroup
\ProvidesFile{marginnote.dtx}[\filedate\space\filerevision\space
%</dtx>
%<package>\NeedsTeXFormat{LaTeX2e}[1995/12/01]
%<package>\ProvidesPackage{marginnote}[%
%<README>LaTeX package marginnote
%<README>Copyright (c) Markus Kohm, 2005-2017
%<README>-------------------------------------------------------------------
%<README>Version:
% \fi^^A meta-comment
% \newcommand*{\packagedateandversion}{%
% \iffalse meta-comment
%<*package|README>
% \fi^^A meta-comment
  2017/04/22 v1.2b
% \iffalse meta-comment
%</package|README>
% \fi^^A meta-comment
% }
% \iffalse meta-comment
%<README>Licence:
%<README>  This work may be distributed and/or modified under the conditions 
%<README>  of the LaTeX Project Public License, version 1.3c of the license.
%<README>  The latest version of this license is in
%<README>    http://www.latex-project.org/lppl.txt
%<README>  and version 1.3c or later is part of all distributions of LaTeX
%<README>  version 2005/12/01 or later and of this work.
%<README>Abstract:
%<README>  This package provides the command \marginnote that may be used
%<README>  instead of \marginpar at almost every place, where \marginpar
%<README>  cannot be used, e.g. inside floats, footnotes, frames made with
%<README>  framed package.  See marginnote.pdf for more information.
%<README>-------------------------------------------------------------------
%<*dtx|package>
  non floating margin notes for LaTeX]
%</dtx|package>
%<*dtx>
\ifx\documentclass\undefined
  \input docstrip.tex

  \keepsilent
  \askforoverwritefalse
  \usedir{tex/latex/marginnote}

  \generate{%
    \file{marginnote.sty}{\from{marginnote.dtx}{package}}%
    \nopreamble\nopostamble
    \file{README.txt}{\from{marginnote.dtx}{README}}%
  }

  \ifToplevel{%
    \Msg{*********************************************************************}
    \Msg{*}
    \Msg{* marginnote}
    \Msg{* ==========}
    \Msg{*}
    \Msg{* To finish the installation run}
    \Msg{* \space\space pdflatex marginnote.dtx}
    \Msg{* After this copy}
    \Msg{* \space\space marginnote.sty to .../tex/latex/marginnote/}
    \Msg{* \space\space marginnote.pdf to .../doc/latex/marginnote/}
    \Msg{* \space\space marginnote.dtx to .../source/latex/marginnote/}
    \Msg{* \space\space README
      \space\space\space\space\space\space\space\space to 
      .../source/latex/marginnote/}
    \Msg{* where .../ is your local TDS tree}
    \Msg{*}
    \Msg{*********************************************************************}
  }
\else
  \let\endbatchfile\relax
\fi
\endbatchfile
\documentclass{ltxdoc}
\usepackage{graphicx}% needed for \reflectbox
\providecommand*{\XeTeX}{%
  X\kern-.1em\lower.5ex\hbox{\reflectbox{E}}\kern-.15em\TeX}
\CodelineIndex
\RecordChanges
\begin{document}
\DocInput{marginnote.dtx}
\end{document}
%</dtx>
%\fi
%
% \GetFileInfo{marginnote.dtx}
%
% \title{Non-Floating Margin Notes with \textsf{marginnote}
%   Package\thanks{This file has revision number \fileversion, last revised
%     \filedate.}}
% \author{Markus Kohm\thanks{Email: \texttt{komascript@gmx.info}}}
% \date{\packagedateandversion}
% \maketitle
%
% \begin{abstract}
%   In \LaTeX{} the command \cs{marginpar}\oarg{left}\marg{right} might be
%   used to create a note in the margin. But there is a problem with this
%   command: it creates a special kind of float. For this it cannot be used
%   e.g. at floats or footnotes. Package \textsl{marginnote} supports another
%   command \cs{marginnote} to create notes in the margin. This does not use a
%   kind of float and for this does not have the disadvantage of
%   \cs{marginpar}. But there might be other problems \dots
% \end{abstract}
%
% \tableofcontents
%
% \changes{v1.0b}{2006/14/03}{spelling fixes}
%
% \section{How to Use \textsf{marginnote} Package}
%
% First of all you have to load. You may use:
% \begin{verbatim}
% \usepackage{marginnote}
% \end{verbatim}\vskip-\baselineskip
% to do so. You may also use one of the following options for a global change
% of the behaviour of \textsf{marginnote}:
% \begin{description}
% \item[\texttt{fulladjust}] adjusts the margin note at the height and depth of
% the current line. Note, that this may sometimes result in extra height and
% depth of the current line, but results in the best vertical
% alignment. This is the default.
% \item[\texttt{heightadjust}] adjusts the margin note at the height of the
% current line but not the depth. Note, that this may sometimes result in
% extra height of the current line and in vertical misplacement.
% \item[\texttt{depthadjust}] adjusts the margin note at the depth of the
% current line but not height. Note, that this may sometimes result in extra
% depth of the current line and very often in vertical misplacement.
% \item[\texttt{noadjust}] does not adjust the margin note at the height or
% depth of the current line. Note, that this often results in vertical
% misplacement but seldom in vertical extra space before or after the current
% line.
% \end{description}
%
% \DescribeMacro{\marginnote}
% The command \cs{marginnote}\oarg{left}\marg{right}\oarg{voffset} may be used
% to set a margin note using \textsf{marginnote}. The first optional argument
% and the mandatory argument are same using \cs{marginpar} from the \LaTeX{}
% kernel. Even \cs{reversemarginpar} will be considered. The note \meta{left}
% or \meta{right} will be put at the current vertical position. Second
% optional argument \meta{voffset} may be used to adjust the vertical position
% of the margin note. Use a negative dimension to move it up or a positive
% dimension to move it down.
%
% \DescribeMacro{\marginnoteleftadjust}
% \DescribeMacro{\marginnoterightadjust}
% At some environments, e.g. \texttt{framed} from the \textsf{framed} package
% the horizontal placement of the margin notes are not correct. In this case
% you may redefine \cs{marginnoteleftadjust} and \cs{marginnoterightadjust} to
% fix this. Note that these are macros not lengths! So you have to use
% \cs{renewcommand}, \cs{def} or \cs{let} to change them. You may e.g. use
% \begin{verbatim}
% \begingroup
%   \makeatletter
%   \g@addto@macro\framed{%
%     \let\marginnoteleftadjust\FrameSep
%     \let\marginnoterightadjust\FrameSep
%   }
% \endgroup
% \end{verbatim}\vskip-\baselineskip
% at your preamble after loading package \textsf{framed} to fix the problem
% using \texttt{framed} environment.
%
% NOTE: \cs{marginnoteleftadjust} and \cs{marginnoterightadjust} will be used
% only, if the correct horizontal position cannot be determined using
% PDF\TeX{} features (\cs{pdfsavepos} and \cs{pdflastxpos}). So if you are
% using PDF\LaTeX{} with PDF output or PDF\LaTeX{} with PDF\TeX-version since
% 1.40 or \XeTeX{} you will not need to use the example code above, but you
% will need at least two \LaTeX{} runs to get the correct horizontal positions
% of the margin notes.
%
% \DescribeMacro{\marginnotetextwidth}
% Package \textsl{marginnote} needs to know the real width of the type area to
% find the right margin. While some environments (e.g. thos of package
% \textsl{framed}) change \cs{textwidth}, \textsl{marginnote} defines it's own
% text width macro. If you change type area after \cs{begin\{document\}} you
% should add
% \begin{verbatim}
%   \edef\marginnotetextwidth{\the\textwidth}
% \end{verbatim}\vskip-\baselineskip
% after changing the type area. Maybe you should do this globally using
% \verb|\xdef| instead of \verb|\edef|. Most users will never need to change
% \cs{marginnotetextwidth}.
%
% \DescribeMacro{\marginnotevadjust}
% At some environments the vertical adjustment of the margin note will be
% wrong, e.g. one base line to low. In this case you may use the additional
% optional argument of \cs{marginnote} at every usage of \cs{marginnote} or
% redefine \cs{marginnotevadjust} at the begin of the environment. The default
% definition is \texttt{0pt}.
%
% \DescribeMacro{\raggedleftmarginnote}
% \DescribeMacro{\raggedrightmarginnote}
% These macros define how the margin note will be aligned. The defaults are:
% \begin{itemize}
% \item align margin notes at the left margin right to the margin,
% \item align margin notes at the right margin left to the margin.
% \end{itemize}
% You may change this using \cs{renewcommand}, e.g. use^^A
% \changes{v1.0a}{2006/02/06}{Example to macros \cs{raggedleftmarginnote} and
%   \cs{raggedrightmarginnote} at documentation fixed [thanks to Susumu
%   Tanimura].}
% \begin{verbatim}
% \renewcommand*{\raggedleftmarginnote}{}
% \renewcommand*{\raggedrightmarginnote}{\centering}
% \end{verbatim}\vskip-\baselineskip
% to get justified text at the left and centered text at the right margin.
%
% \DescribeMacro{\marginfont}
% This macro defines the font that will be used to set margin notes. The
% default is \cs{normalcolor}. You may use \cs{renewcommand} to change this,
% e.g. use
% \begin{verbatim}
% \renewcommand*{\marginfont}{\color{red}\sffamily}
% \end{verbatim}\vskip-\baselineskip
% to get red colored margin notes in sans serif font family. You need to load
% e.g. package \textsf{color} to use \cs{color}.
%
%
% \section{Known Problems Using \textsf{marginnote}}
%
% At double side layout (e.g. using class option \texttt{twoside})
% \cs{marginnote} needs to know the number of the current page to decide
% wether the page is odd or even and so wether to use left or right
% margin. \LaTeX{} uses an asynchronous output. Because of this counter
% \texttt{page} should not be used to get the number of the current page
% unless you are at page head or foot. To solve the problem
% \textsf{marginnote} uses a mechanism similar to labels. But this means, that
% the correct margin won't be known at this \LaTeX{} run but only at the
% next. So after adding or deleting a margin note or after each change of page
% break you need two \LaTeX{} runs to get all margins right.
%
% The command \cs{marginnote} uses \cs{strut} and \cs{vadjust} to put the
% margin note at the correct position. But under some circumstances this may
% fail. You may adjust the vertical position of the margin note using the
% second optional argument of \cs{marginnote}. Sometimes even the text outside
% \cs{marginnote} will be moved because of using \cs{marginnote}. You may use
% one of the package options \texttt{fulladjust}, \texttt{heightadjust},
% \texttt{depthajust} or \texttt{noajust} to change the global adjustment or
% a local redefinition of |\mn@strut| or |\mn@zbox|.
%
% Note: The margin note will be placed at the current vertical line. This
% means, if you are using two \cs{marginnote} commands at the same line, they
% will be put on the same place. This is not a bug but a feature!
%
% Since release~1.1b \cs{marginnote} between paragraphs (in vertical mode)
% will place the note between the paragaphs instead of the end of the previous
% paragraph. You may use \cs{leavevmode} or the third optional argument of
% \cs{marginnote} to place it different.
%
% No page break may occure inside a margin note created with \cs{marginnote}.
%
% \cs{marginnote} somewhat different from \cs{marginpar} if used immediate
% after \cs{item}. This is not a bug, it's a feature!
%
% With math \cs{marginnote} may work or may not depending on the math
% environment.
%
% If you are using \XeTeX{}, PDF\LaTeX{} since version~1.40 or PDF\LaTeX{}
% before version~1.40 with PDF output and the horizontal position of
% a margin note is wrong, do one more PDF\LaTeX{} run.
%
% Sometimes lines are stretched vertically using \cs{marginnote}, e.g.\ if
% you're using \cs{marginnote} at a list \emph{and} upper case umlauts like
% ``\"U'' or if \verb|\lineskiplimit>0pt|. In this case
% \verb|\lineskiplimit=0pt| or \verb|\lineskiplimit=-\maxdimen|, or one of the
% options may help.
%
% You should not use \cs{marginnote} at the optional argument of \cs{item}.
%
%
% \StopEventually{\PrintIndex\PrintChanges}
%
% \section{Implementation}
%
% \iffalse
%<*package>
% \fi
%
% First test $\varepsilon$-\TeX.
%    \begin{macrocode}
\begingroup
  \def\@tempb{}%
  \def\@tempa{%
    \PackageError{marginnote}{seems you are not running e-TeX\@tempb}{%
      Since 2004 the LaTeX team recommends to use e-TeX.\MessageBreak
      marginnote since version 1.1d uses e-TeX features.\MessageBreak
      At actual systems `latex' should already use e-TeX.\MessageBreak
      At deprecated systems it may be called `elatex'.\MessageBreak
      Use either unsupported marginnote up to version 1.1c or\MessageBreak
      ask you administrator for LaTeX using e-TeX\@tempb.\MessageBreak
      Not using e-TeX\@tempb\space is a fatal error!\MessageBreak
      Processing cannot be continued!}%
    \endgroup
    \batchmode \errmessage{}\csname @@end\endcsname\end\relax
    \csname endinput\endcsname
  }%
  \expandafter\ifx\csname eTeXversion\endcsname\relax\else
    \ifnum\eTeXversion <2
      \def\@tempb{ V 2}%
    \else
      \let\@tempa\endgroup
    \fi
  \fi
\@tempa
%    \end{macrocode}
%
% Next declare and process the options.
%
% \begin{macro}{\if@mn@verbose}
% Use verbose output mode by default. But you may change this using option
% \texttt{quiet}.
%    \begin{macrocode}
\newif\if@mn@verbose\@mn@verbosetrue
\DeclareOption{verbose}{\@mn@verbosetrue}
\DeclareOption{quiet}{\@mn@verbosefalse}
%    \end{macrocode}
% \end{macro}
%
% \changes{v1.1e}{2009/06/06}{new options \texttt{fulladjust},
%   \texttt{heightadjust}, \texttt{depthadjust}, and \texttt{noadjust}}
% \begin{macro}{\mn@strut}
%   \changes{v1.1e}{2009/06/06}{new (semi internal)}
% The package needs to adjust the margin note at the current line. Sometimes
% this provocates extra vertical line spacing. To avoid this you may redefine
% \cs{mn@strut}. The default value is \cs{strut}.
%    \begin{macrocode}
\newcommand*{\mn@strut}{}
%    \end{macrocode}
% \begin{macro}{\mn@zbox}
%   \changes{v1.1b}{2009/02/16}{new (internal)}
% This macro is used to set a horizontal box without height, depth and width.
%    \begin{macrocode}
\newcommand{\mn@zbox}[1]{}
%    \end{macrocode}
% The options do redefine both, \cs{mn@strut} and \cs{mn@zbox}.
%    \begin{macrocode}
\DeclareOption{fulladjust}{%
  \renewcommand*{\mn@strut}{\strut}%
  \renewcommand{\mn@zbox}[1]{%
    \bgroup
      \setbox\@tempboxa\vbox{#1}%
      \ht\@tempboxa\ht\strutbox
      \dp\@tempboxa\dp\strutbox
      \wd\@tempboxa\z@
      \box\@tempboxa
    \egroup
  }%
}
\DeclareOption{heightadjust}{%
  \renewcommand*{\mn@strut}{\begingroup\dp\strutbox\z@\strut\endgroup}%
  \renewcommand{\mn@zbox}[1]{%
    \bgroup
      \setbox\@tempboxa\vbox{#1}%
      \ht\@tempboxa\ht\strutbox
      \dp\@tempboxa\dp\z@
      \wd\@tempboxa\z@
      \box\@tempboxa
    \egroup
  }%
}
\DeclareOption{depthadjust}{%
  \renewcommand*{\mn@strut}{\begingroup\ht\strutbox\z@\strut\endgroup}%
  \renewcommand{\mn@zbox}[1]{%
    \bgroup
      \setbox\@tempboxa\vbox{#1}%
      \ht\@tempboxa\ht\z@
      \dp\@tempboxa\dp\strutbox
      \wd\@tempboxa\z@
      \box\@tempboxa
    \egroup
  }%
}
\DeclareOption{noadjust}{%
  \renewcommand*{\mn@strut}{\relax}%
  \renewcommand{\mn@zbox}[1]{%
    \bgroup
      \setbox\@tempboxa\vbox{\kern-\ht\strutbox #1}%
      \ht\@tempboxa\ht\z@
      \dp\@tempboxa\dp\z@
      \wd\@tempboxa\z@
      \box\@tempboxa
    \egroup
  }%
}
%    \end{macrocode}
% \end{macro}
% \end{macro}
%
%    \begin{macrocode}
\ExecuteOptions{verbose,fulladjust}
\ProcessOptions\relax
%    \end{macrocode}
%
% \begin{macro}{\newmarginnote}
%   We need a macro to define a new note at the \texttt{aux} file. This will
%   be done using the mechanism of \LaTeX{} that is used for
%   \cs{newlabel}. But we use another prefix. This will result in the usual
%   ``Labels(s) may have changed. Rerun to get cross-references right.'' if a
%   margin note is new or have moved to another page.
%    \begin{macrocode}
\newcommand*{\newmarginnote}{\@newl@bel{mn}}
%    \end{macrocode}
% \end{macro}
%
% \begin{macro}{\if@mn@pdfmode}
%   \changes{v1.1}{2006/10/23}{new switch}^^A
%   \changes{v1.1a}{2008/11/10}{PDF\TeX\ since 1.40 allows \cs{pdfsavepos} in
%     DVI mode too}^^A
%   \changes{v1.1b}{2009/02/16}{if level fixed}^^A
%   \changes{v1.1c}{2009/02/26}{\protect\XeTeX has working \cs{pdflastxpos}^^A
%     but no \cs{pdftexversion}}^^A
%   \changes{v1.2}{2016/06/02}{addition for lua\TeX{} from 0.85}^^A
% \begin{macro}{\@mn@mode@prefix}
%   \changes{v1.2}{2016/06/02}{(new (internal)}^^A
% We need to know, wether or not PDF\TeX{} and which version of PDF\TeX{} is
% used. With PDF\TeX{} the horizontal output position may be detected using
% \cs{pdfsavepos} and \cs{pdflastxpos}. So the relative position of the margin
% may be calculated. Without PDF\TeX{} only manual adjustment is
% available. While PDF mode or not may change before start of the document,
% setting up the switch is delayed.
%    \begin{macrocode}
\newif\if@mn@pdfmode\@mn@pdfmodefalse
\newcommand*{\@mn@mode@prefix}{pdf}
\AtBeginDocument{%
  \begingroup\expandafter\expandafter\expandafter\endgroup
  \expandafter\ifx\csname pdflastxpos\endcsname\relax
    \begingroup\expandafter\expandafter\expandafter\endgroup
    \expandafter\ifx\csname lastxpos\endcsname\relax\else
      \@mn@pdfmodetrue
      \renewcommand*{\@mn@mode@prefix}{}%
    \fi
  \else % bg or 1
    \begingroup\expandafter\expandafter\expandafter\endgroup
    \expandafter\ifx\csname pdftexversion\endcsname\relax % bg 2
      \begingroup\expandafter\expandafter\expandafter\endgroup
      \expandafter\ifx\csname pdfoutput\endcsname\relax % bg 3
        \begingroup\expandafter\expandafter\expandafter\endgroup
        \expandafter\ifx\csname XeTeXrevision\endcsname\relax\else % bg 4
          \@mn@pdfmodetrue
        \fi % ed 4
      \else % or 3
        \ifcase\pdfoutput\else\@mn@pdfmodetrue\fi % bg ed 4
      \fi % ed 3
    \else % or 2
      \ifnum \pdftexversion<140 % bg 3
        \begingroup\expandafter\expandafter\expandafter\endgroup
        \expandafter\ifx\csname pdfoutput\endcsname\relax % bg 4
        \else % or 4
          \ifcase\pdfoutput\else\@mn@pdfmodetrue\fi % bg ed 5
        \fi % ed 4
      \else % or 3
        \@mn@pdfmodetrue
      \fi % ed 3
    \fi % ed 2
  \fi % ed 1
  \if@mn@verbose
    \if@mn@pdfmode
      \PackageInfo{marginnote}{%
        \string\pdfoutput\space not 0 or unimportant and\MessageBreak
        \string\pdflastxpos\space or \string\lastxpos\space 
        available.\MessageBreak
        Extended position detection mode activated\@gobble
      }%
    \else
      \PackageInfo{marginnote}{%
        either \string\pdflastxpos\space or \string\pdfoutput\space not
        available\MessageBreak
        or \string\pdfoutput\space set to 0.\MessageBreak
        Extended position detection mode deactivated\@gobble
      }%
    \fi
  \fi
}
%    \end{macrocode}
% \end{macro}
% \end{macro}
%
% \begin{macro}{\marginnotetextwidth}
%   \changes{v1.1}{2006/10/23}{new macro}
% Some environments change \cs{textwidth}. But at PDF mode we need to know the
% real text width to find the right margin. So we use our own text width
% macro. Sometimes it may be usefull if the user can set it up. Because of
% this it is a user command.
%    \begin{macrocode}
\newcommand*{\marginnotetextwidth}{}
\let\marginnotetextwidth\textwidth
\AtBeginDocument{\if@mn@pdfmode\edef\marginnotetextwidth{\the\textwidth}\fi}
%    \end{macrocode}
% \end{macro}
%
% \begin{macro}{\@mn@margintest}
%   \changes{v1.1}{2006/10/23}{new PDF mode feature}
% \begin{macro}{\@mn@thispage}
% \begin{macro}{\@mn@atthispage}
% \begin{macro}{\@mn@currpage}
%   \changes{v1.1}{2006/10/23}{new (internal)}
% \begin{macro}{\@mn@currxpos}
%   \changes{v1.1}{2006/10/23}{new (internal)}
% \begin{macro}{\mn@abspage}
%   Macro \cs{@mn@margintest} does the complete test, which margin to use. The
%   result may be found at \cs{if@tempswa}. To avoid changes on the last page
%   if there is a new note on the first page, try to count the notes by
%   page. We know that this can not be successfull, but never the less it may
%   be a good try. \cs{@mn@thispage} saves the page number of the last usage
%   of \cs{@mn@margintest}. \cs{@mn@atthispage} saves the number of margin
%   note at this page. But we need to know the absolut page number to do
%   this. So we increase the absolut page number \texttt{mn@abspage} at every
%   \cs{@outputpage}. \cs{@mn@currpage} is the page from the page label if
%   found. \cs{@mn@currxpos} is somehow special. Using PDF\TeX{} the real $x$
%   position may be written with the page label and used to calculate the
%   correct horizontal offset. In this case \cs{marginnoteleftadjust} and
%   \cs{marginnoterightadjust} will not be used!
%    \begin{macrocode}
\newcommand*{\@mn@thispage}{}
\newcommand*{\@mn@currpage}{}
\newcommand*{\@mn@currxpos}{}
\newcounter{mn@abspage}
\AtBeginDocument{\setcounter{mn@abspage}{1}%
  \g@addto@macro\@outputpage{\stepcounter{mn@abspage}}}
\newcommand*{\@mn@margintest}{%
%    \end{macrocode}
%   \changes{v1.2}{2016/06/02}{addition for lua\TeX{} from 0.85}^^A
%   Number of the next margin note at this page.
%    \begin{macrocode}
  \expandafter\ifx\csname @mn@thispage\endcsname\@empty
    \gdef\@mn@atthispage{1}%
  \else\expandafter\ifnum \@mn@thispage=\value{mn@abspage}%
      \begingroup
        \@tempcnta\@mn@atthispage\advance\@tempcnta by \@ne
        \xdef\@mn@atthispage{\the\@tempcnta}%
      \endgroup
    \else
      \gdef\@mn@atthispage{1}%
    \fi
  \fi
  \xdef\@mn@thispage{\themn@abspage}%
%    \end{macrocode}
%   Use the number of the page and the number of the margin note at this page
%   to save the real number of this page at the \texttt{aux} file. At PDF mode
%   save the current $x$ position too.
%    \begin{macrocode}
  \let\@mn@currpage\relax
  \let\@mn@currxpos\relax
  \if@mn@pdfmode
    \@nameuse{\@mn@mode@prefix savepos}%
    \protected@write\@auxout{\let\themn@abspage\relax}{%
      \string\newmarginnote{note.\@mn@thispage.\@mn@atthispage}{%
        {\themn@abspage}{\noexpand\number\@nameuse{\@mn@mode@prefix lastxpos}sp}}%
    }%
  \else
    \protected@write\@auxout{\let\themn@abspage\relax}{%
      \string\newmarginnote{note.\@mn@thispage.\@mn@atthispage}{%
        {\themn@abspage}{}}%
    }%
  \fi
%    \end{macrocode}
%   If the margin note label was not defined, it seems to be new. In this case
%   the absolut page number will be used for the test instead of the saved
%   real page number.
%    \begin{macrocode}
  \expandafter\ifx\csname mn@note.\@mn@thispage.\@mn@atthispage\endcsname\relax
%    \end{macrocode}
%   If we are not in two side mode, we are on a odd page.
%    \begin{macrocode}
    \if@twoside
      \if@mn@verbose
        \PackageInfo{marginnote}{Suggest that margin
          note \@mn@thispage.\@mn@atthispage\space will be on\MessageBreak
          absolute page \themn@abspage.\MessageBreak
          This may be wrong}%
      \fi
      \ifodd\value{mn@abspage}\@tempswatrue\else\@tempswafalse\fi
    \else
      \if@mn@verbose
        \PackageInfo{marginnote}{right page because not two side mode}%
      \fi
      \@tempswatrue
    \fi
  \else
    \edef\@mn@currpage{\csname
      mn@note.\@mn@thispage.\@mn@atthispage\endcsname}%
    \edef\@mn@currxpos{\expandafter\@secondoftwo\@mn@currpage}%
%    \end{macrocode}
% \changes{v1.1d}{2009/05/06}{take care of \cs{hoffset}}^^A
% Ulrike Fischer suggested a simple change to take care of \cs{hoffset},
% e.g., using package \textsf{crop}.
% \changes{v1.1d}{2009/05/06}{take care of \cs{pdfhorigin}}^^A
% We use this occasion to take care of \cs{pdfhorigin}, too.
% \changes{v1.2a}{2016/10/21}{redefine \cs{@mn@currxpos} only if not empty}^^A
% If \cs{@mn@currxpos} is not empty here, it should be corrected by
% \cs{hoffset} and maybe by \cs{pdfhorigin}.
%    \begin{macrocode}
    \ifx\@mn@currxpos\@empty\else
      \edef\@mn@currxpos{\the\dimexpr \@mn@currxpos -\hoffset\relax}%
      \begingroup\expandafter\expandafter\expandafter\endgroup
      \expandafter\ifx\csname pdfhorigin\endcsname\relax\else
        \begingroup\expandafter\expandafter\expandafter\endgroup
        \expandafter\ifx\csname pdfoutput\endcsname\relax
          \begingroup\expandafter\expandafter\expandafter\endgroup
          \expandafter\ifx\csname outputmode\endcsname\relax\else
            \ifnum \outputmode=1 %
              \edef\@mn@currxpos{\the\dimexpr \@mn@currxpos -\pdfhorigin
                +1in\relax}%
            \fi
          \fi
        \else
          \ifnum \pdfoutput=1 %
            \edef\@mn@currxpos{\the\dimexpr \@mn@currxpos -\pdfhorigin 
              +1in\relax}%
          \fi
        \fi
      \fi
%    \end{macrocode}
% \changes{v1.2b}{2017/04/22}{\textsf{bidi} code added}^^A
% If you are using package \textsf{bidi} and RTL mode is active, the position
% is from right instead of left. So we have to substract \cs{@mn@currxpos}
% from \cs{pdfpagewidth} (or \cs{pagewidth} using Lua\TeX, but this cannot be,
% because \textsf{bidi} is not Lua\TeX-compatible).
%    \begin{macrocode}
      \begingroup\expandafter\expandafter\expandafter\endgroup
      \expandafter\ifx\csname \@mn@mode@prefix pagewidth\endcsname\relax\else
        \@mn@if@RTL{%
          \PackageInfo{marginnote}{Margin note
            \@mn@thispage.\@mn@atthispage\space in RTL mode}%
          \edef\@mn@currxpos{%
            \the\dimexpr\@nameuse{\@mn@mode@prefix pagewidth}
                        -\@mn@currxpos\relax
          }%
        }{}%
      \fi
    \fi
    \edef\@mn@currpage{\expandafter\@firstoftwo\@mn@currpage}%
    \if@mn@verbose
      \PackageInfo{marginnote}{Margin note \@mn@thispage.\@mn@atthispage\space
        is on absolute page \@mn@currpage\MessageBreak}%
    \fi
    \if@twoside
      \ifodd\@mn@currpage\relax
        \@tempswatrue
      \else
        \@tempswafalse
      \fi
    \else
      \if@mn@verbose
        \PackageInfo{marginnote}{right page because not two side mode}%
      \fi
      \@tempswatrue
    \fi
  \fi
}
%    \end{macrocode}
% \begin{macro}{@mn@ifRTL}
%   \changes{1.2b}{2017/04/22}{new internal}
% Test, whether or not \cs{if@RTL} exists and is true or false.
%    \begin{macrocode}
\newcommand*{\@mn@if@RTL}{%
  \begingroup\expandafter\expandafter\expandafter\endgroup
  \expandafter\ifx\csname if@RTL\endcsname\iftrue
    \expandafter\@firstoftwo
  \else
    \expandafter\@secondoftwo
  \fi
}
%    \end{macrocode}
% \end{macro}
% \end{macro}
% \end{macro}
% \end{macro}
% \end{macro}
% \end{macro}
% \end{macro}
%
% \begin{macro}{\marginnote}
% \begin{macro}{\@mn@marginnote}
% \begin{macro}{\@mn@@marginnote}
%   \changes{v1.1g}{2011/04/11}{missing \cs{long} added}
% \begin{macro}{\@mn@@@marginnote}
%   \changes{v1.1}{2006/10/23}{new PDF mode feature}
%   \changes{v1.1g}{2011/04/11}{missing \cs{long} added}
% Command \cs{marginnote} is the main macro of the package. The others are
% helpers to manage the optional arguments.
%    \begin{macrocode}
\newcommand*{\marginnote}{%
  \@dblarg\@mn@marginnote
}
\newcommand{\@mn@marginnote}[2][]{%
  \ifhmode
    \@bsphack
    \begingroup
    \ifdim\@savsk>\z@\else
      \def\:{\@xifnch}\expandafter\def\: { \futurelet\@let@token\@ifnch}%
    \fi
  \else
    \begingroup
  \fi
  \@ifnextchar [{\@mn@@marginnote[{#1}]{#2}}{\@mn@@marginnote[{#1}]{#2}[\z@]}%
}
\newcommand{\@mn@@marginnote}{}
\long\def\@mn@@marginnote[#1]#2[#3]{%
  \endgroup
%    \end{macrocode}
% In horizontal mode the space hack of the \LaTeX{} kernel will be used. In
% vertical mode this should not be used.
%    \begin{macrocode}
  \ifhmode
    \@mn@@@marginnote[{#1}]{#2}[{#3}]%
    \@esphack
  \else
    \@mn@@@marginnote[{#1}]{#2}[{#3}]%
  \fi
}
\newcommand{\@mn@@@marginnote}{}
\long\def\@mn@@@marginnote[#1]#2[#3]{%
%    \end{macrocode}
%   \changes{v1.1b}{2009/02/16}{use \cs{mn@vadjust} instead of \cs{vadjust}}%
%   \changes{v1.1e}{2009/06/06}{use \cs{mn@strut} instead of \cs{strut}}%
% All changes (but change of counters that are global because of using the
% \LaTeX{} commands to change them an \cs{gdef} and \cs{xdef}) should be
% local. In h-mode a \cs{strut} will be used to fix base line. The margin
% note will be put to vertical list using \cs{vadjust}. This also means that
% wie are one line to deep. This will be corrected later using negative kern.
% In v-mode wie use a special kind of vbox to simply set everything. Math
% mode should behave like v-mode. And if we are just after an item we have
% to leave v-mode first.
%    \begin{macrocode}
  \begingroup
    \ifmmode\mn@strut\let\@tempa\mn@vadjust\else
      \if@inlabel\leavevmode\fi
      \ifhmode\mn@strut\let\@tempa\mn@vadjust\else\let\@tempa\mn@vlap\fi
    \fi
    \@tempa{%
%    \end{macrocode}
% Everything will be put upwards using a vbox with zero height and depth and
% \cs{vss}. At this box the margin test will be done. If cs{reversemargin}
% was used, the logic switchs. Then the note will be places to the margin.
%    \begin{macrocode}
      \vbox to\z@{%
        \vss
        \@mn@margintest
        \if@reversemargin\if@tempswa
            \@tempswafalse
          \else
            \@tempswatrue
        \fi\fi
        \if@tempswa
          \rlap{%
%    \end{macrocode}
% If \cs{@mn@currxpos} is neither \cs{relax} nor empty it is the real 
% current $x$ position of the last PDF\LaTeX{} run and may be used to
% calculate the real horizontal offset.
%    \begin{macrocode}
            \ifx\@mn@currxpos\relax
              \kern\marginnoterightadjust
              \if@mn@verbose
                \PackageInfo{marginnote}{%
                  xpos not known,\MessageBreak
                  using \string\marginnoterightadjust}%
              \fi
            \else\ifx\@mn@currxpos\@empty
                \kern\marginnoterightadjust
                \if@mn@verbose
                  \PackageInfo{marginnote}{%
                    xpos not known,\MessageBreak
                    using \string\marginnoterightadjust}%
                \fi
              \else
                \if@mn@verbose
                  \PackageInfo{marginnote}{%
                    xpos seems to be \@mn@currxpos,\MessageBreak
                    \string\marginnoterightadjust
                    \space ignored}%
                \fi
                \begingroup
                  \setlength{\@tempdima}{\@mn@currxpos}%
                  \kern-\@tempdima
                  \if@twoside\ifodd\@mn@currpage\relax
                      \kern\oddsidemargin
                    \else
                      \kern\evensidemargin
                    \fi
                  \else
                    \kern\oddsidemargin
                  \fi
                  \kern 1in
                \endgroup
              \fi
            \fi
            \kern\marginnotetextwidth\kern\marginparsep
            \vbox to\z@{\kern\marginnotevadjust\kern #3
              \vbox to\z@{%
                \hsize\marginparwidth
%    \end{macrocode}
%   \changes{v1.1g}{2011/04/11}{set \cs{linewidth}}
%    \begin{macrocode}
                \linewidth\hsize
%    \end{macrocode}
%   Here's the correction of the vertical position. The remain is simple.
%   \changes{v1.1i}{2012/03/29}{\cs{strut} moved to fix hyphenation (thanks to
%     Ulrike Fischer)}
%   \changes{v1.1i}{2012/03/29}{\cs{ignorespaces} added}
%    \begin{macrocode}
                \kern-\parskip
                \marginfont\raggedrightmarginnote\strut\hspace{\z@}%
                \ignorespaces#2\endgraf
                \vss}%
              \vss}%
          }%
        \else
%    \end{macrocode}
%   Using the left margin.
%   \changes{v1.1f}{2010/01/05}{missing usage of \cs{marginnotevadjust} on
%     left margin fixed}
%    \begin{macrocode}
          \llap{%
            \vbox to\z@{\kern\marginnotevadjust\kern #3
              \vbox to\z@{%
                \hsize\marginparwidth
%    \end{macrocode}
%   \changes{v1.1g}{2011/04/11}{set \cs{linewidth}}
%    \begin{macrocode}
                \linewidth\hsize
%    \end{macrocode}
%   Same like above for left margins.
%    \begin{macrocode}
                \kern-\parskip
                \marginfont\raggedleftmarginnote\strut\hspace{\z@}%
                \ignorespaces#1\endgraf
                \vss}%
              \vss}%
            \ifx\@mn@currxpos\relax
              \kern\marginnoteleftadjust
              \if@mn@verbose
                \PackageInfo{marginnote}{%
                  xpos not known,\MessageBreak
                  using \string\marginnoteleftadjust}%
              \fi
            \else\ifx\@mn@currxpos\@empty
                \kern\marginnoteleftadjust
                \if@mn@verbose
                  \PackageInfo{marginnote}{%
                    xpos not known,\MessageBreak
                    using \string\marginnoteleftadjust}%
                \fi
              \else
                \if@mn@verbose
                  \PackageInfo{marginnote}{%
                    xpos seems to be \@mn@currxpos,\MessageBreak
                    \string\marginnoteleftadjust
                    \space ignored}%
                \fi
               \begingroup
                  \kern\@mn@currxpos
                  \if@twoside\ifodd\@mn@currpage\relax
                      \kern-\oddsidemargin
                    \else
                      \kern-\evensidemargin
                    \fi
                  \else
                    \kern-\oddsidemargin
                  \fi
                  \kern-1in
                \endgroup
              \fi
            \fi
            \kern\marginparsep
          }%
        \fi
      }%
    }%
  \endgroup
}
%    \end{macrocode}
% \end{macro}
% \end{macro}
% \end{macro}
% \end{macro}
%
% \begin{macro}{\marginnoterightadjust}
% \begin{macro}{\marginnoteleftadjust}
% These may be used to define an automatic horizontal adjust. The default is
% zero. They will be used only if not PDF\TeX{} or PDF\TeX{} before version~1.40
% in DVI mode is used, because in this case the save position features are not
% available.
%    \begin{macrocode}
\newcommand*{\marginnoterightadjust}{}
\newcommand*{\marginnoteleftadjust}{}
\let\marginnoterightadjust\z@
\let\marginnoteleftadjust\z@
%    \end{macrocode}
% \end{macro}
% \end{macro}
%
% \begin{macro}{\marginnotevadjust}
% This may be used to define an automatic vertical adjust. The defaul tis
% zero. Values greater than zero will move the margin note down, values less
% than zero will move the margin note up.
%    \begin{macrocode}
\newcommand*{\marginnotevadjust}{}
\let\marginnotevadjust\z@
%    \end{macrocode}
% \end{macro}
%
% \begin{macro}{\mn@vlap}
% This macro is used to set a vertical box without size at vertical mode.
%    \begin{macrocode}
\newcommand{\mn@vlap}[1]{%
  \setbox\@tempboxa\vbox to \ht\strutbox{#1\vss}%
  \box\@tempboxa\vskip-\baselineskip
}
%    \end{macrocode}
% \end{macro}
%
% \begin{macro}{\mn@vadjust}
%   \changes{v1.1b}{2009/02/16}{new (internal)}
% This macro is used to set a vertical box at horizontal mode.
%    \begin{macrocode}
\newcommand{\mn@vadjust}[1]{%
  \mn@zbox{\kern-\parskip
    \leavevmode\vadjust{#1}%
    \kern\parskip
  }%
}
%    \end{macrocode}
% \end{macro}
%
% \begin{macro}{\marginfont}
%  \changes{v1.0a}{2006/02/06}{Use \cs{providecommand} to define it.}
% \begin{macro}{\raggedleftmarginnote}
% \begin{macro}{\raggedrightmarginnote}
%   These are very simple. A class may also define \cs{marginfont}. Use this
%   if available. I don't use \cs{let} for the definitions of the ragged
%   macros, so the meaning may change loading e.g. package \textsf{ragged2e}.
%    \begin{macrocode}
\providecommand*{\marginfont}{}
\newcommand*{\raggedleftmarginnote}{\raggedleft}
\newcommand*{\raggedrightmarginnote}{\raggedright}
%    \end{macrocode}
% \end{macro}
% \end{macro}
% \end{macro}
%
% \Finale
%
\endinput
%
% end of `marginnote.dtx'
%
% \iffalse
%%% Local Variables:
%%% mode: doc-tex
%%% TeX-master: t
%%% End:
% \fi
