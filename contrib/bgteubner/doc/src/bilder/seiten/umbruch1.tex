% ===================================================================
\PassOptionsToClass{a5paper}{bgteubner}
\PassOptionsToClass{ulinevec}{bgteubner}
\documentclass[english,ngerman]{bgteubner}
% ===================================================================
% Einstellung des passenden Zeichensatzes
% F�r Windows:
%\usepackage[ansinew]{inputenc}
% F�r Unix und Linux
\usepackage[latin1]{inputenc}
% ===================================================================
\usepackage{color}
\definecolor{grau}{gray}{0}
% ===================================================================
\providecommand\groesser{}
% ===================================================================
\newcommand\blindtext{%
  Dies ist Blindtext.
  Er erf�llt einzig den Zweck, die Seite zu f�llen.
  Ansonsten hat er keinen Sinn.}
% ===================================================================
\newcommand\linie{%
  \begingroup
  \unitlength1mm%
  \thinlines
  \begin{picture}(0,0)
    \put(0,14.8){{\color{grau}\line(1,0){115}}}%
    \put(0,176){{\color{grau}\line(1,0){115}}}%
    \put(0,14.8){{\color{grau}\line(0,1){161.2}}}%
    \put(115,14.8){{\color{grau}\line(0,1){161.2}}}%
  \end{picture}%
  \endgroup
}%                                
% ===================================================================
\lofoot{\linie}
\lefoot{\linie}
% ===================================================================
\begin{document}
% ===================================================================
\setcounter{page}{8}%
\setcounter{chapter}{3}%
\setcounter{section}{4}%
\markleft{\thechapter\enskip Ein Kapitel}%
% ===================================================================
\noindent
\blindtext\
\blindtext\
\blindtext\
\blindtext\
\blindtext\
\blindtext\
\blindtext\
\blindtext
\groesser

\blindtext\
\blindtext\
\blindtext\
\blindtext\
\blindtext\
\blindtext\
\blindtext\
\blindtext\
\blindtext\
\blindtext\

\blindtext\
\blindtext\
\blindtext\
\blindtext\
\blindtext\
\blindtext\

\section{Ein neuer Abschnitt}
\subsection{Gefolgt von einem Unterabschnitt}
\subsubsection{Und noch einer Ebene}
\blindtext\
\blindtext\
\blindtext\
\blindtext\
\blindtext\
\blindtext\
\blindtext\
\blindtext\
\blindtext\
\blindtext\
\blindtext\
\groesser

\blindtext\
\blindtext\
\blindtext\
\blindtext\
\blindtext\
\blindtext\
\blindtext\
\blindtext\

\blindtext\
\blindtext\
\blindtext\
\blindtext\
\blindtext\
\blindtext\
\blindtext\
\blindtext\

\blindtext\
\blindtext\
\blindtext\
\blindtext\
\blindtext\
\blindtext\
\blindtext\
\blindtext\

% ===================================================================
\end{document}
% ===================================================================

%%% Local Variables: 
%%% mode: latex
%%% TeX-master: "richtlinien"
%%% End: 
