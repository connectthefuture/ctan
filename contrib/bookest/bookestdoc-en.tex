%%%%%%%%%%%%%%%%%%%%%%%%%%%%%%%%%%%%%%%%%%%%%%%%%%%%%%%%%%%%%%%%%%%%%%%%%%%%%%%%%%%%%%%%%%%%%%%%%%%%%%%%%%%%%%%%%%%%%%%%%
%%%%%%%%% THIS DOCUMENT IS OUT OF DATE, NEWEST FEATURES DOCUMENTED ONLY IN THE ITALIAN VERSION OF THIS DOCUMENT %%%%%%%%%
%%%%%%%%%%%%%%%%%%%%%%%%%%%%%%%%%%%%%%%%%%%%%%%%%%%%%%%%%%%%%%%%%%%%%%%%%%%%%%%%%%%%%%%%%%%%%%%%%%%%%%%%%%%%%%%%%%%%%%%%%

\documentclass[a4paper,oneside,centered,noparindent,noparskip]{bookest}

\usepackage[utf8x]{inputenc}
%\usepackage[italian]{babel}
\usepackage{palatino}
\usepackage{guit}

\hyphenation{Su-pe-rio-re}

\hypersetup{
pdftitle={The bookest class},
pdfsubject={An extension for the book class},
pdfauthor={Riccardo Bresciani},
pdfkeywords={TeX, LaTeX, pdfLaTeX, book, bookest},
pdfstartview=FitV,
%colorlinks
}

\setoddheadtext{{\colorA The \texttt{bookest} class --- Version 1.0.4\hfill Riccardo Bresciani}}
\setoddfoot{\hfill{\colorA\thepage}\hfill}

\makeatletter
\renewcommand \thesection{\@arabic\c@section.}
\renewcommand\thesubsection{\thesection\@arabic\c@subsection}
\makeatother

\newcommand{\tA}[1]{\texttt{\colorA #1}}

\shipouttext{60}{15}{\fbox{\itshape \ttfamily out of date}}

\begin{document}

\chapter*{The \texttt{bookest} class\\{\Large Version 1.0.4 --- May 22, 2007}\\{\Large Riccardo Bresciani}}
 
\textit{\sffamily
\begin{abstract}{\today: WARNING}
This document is out of date, the newest features are documented only in the Italian version of this document.
\end{abstract}
\vspace{7ex}}
 
The \texttt{bookest} class in an extension of the standard \texttt{book} class, on which it relies and that is loaded with the default options.

\ppar
The extensions provided by the class involve:
\begin{enumerate}
 \item colors;
 \item document layout;
 \item headings and footers;
 \item title page layout;
 \item \dots
\end{enumerate}

The \texttt{bookest} web page is \url{http://tex.boris-web.net/bookest}.

\section{Colors}
\texttt{bookest} requires the \texttt{color} and the \texttt{pdfcolmk}\footnote{This package, reported by Massimiliano Dominici (\GuIT), allows to bypass some of the limitations that pdf\TeX\ has up to version 1.30, mainly the unsupported \emph{colorstack}. Currently the package is loaded by default, but as soon as the main ditributions (MiK\TeX, te\TeX, \dots) will have pdf\TeX\ versione 1.40 this package will be removed from the class.} packages to provide color support to the document.

\ppar
Specifically, colors \texttt{A} and \texttt{B} are defined: they will be used in the definition of the sectioning commands and in coloring of some text elements --- e.g. the \texttt{footnote} rules or the list labels in \texttt{itemize}, \texttt{enumerate} e \texttt{description}.

\ppar
The default document is black and white, the user can anyway define the colors according to his taste by using the commands listed in \ref{coloricomandi}.

\ppar
\texttt{bookest} provides some preset color combinations that can be activated with the corresponding class options listed in \ref{coloriopzioni}. For these options to be used the \texttt{hyperref} package is required, as link and anchor colors (color \texttt{A}) and citation color (color \texttt{B}) are set.

\subsection{Commands}\label{coloricomandi}
\begin{description}
 \item[\texttt{\textbackslash colorA}, \texttt{\textbackslash colorB}, \texttt{\textbackslash black}] declare the colors used by the class (\texttt{A} and \texttt{B} empty by default and user-definable, plus black);
 \item[\texttt{\textbackslash setcolorA\{\textit{<R> <G> <B>}\}}, \texttt{\textbackslash setcolorB\{\textit{<R> <G> <B>}\}}] specify color \texttt{A} and \texttt{B} in RGB format;
 \item[\texttt{\textbackslash SETcolorA\{\textit{<command>}\}}, \texttt{\textbackslash SETcolorB\{\textit{<command>}\}}] redefine the commands \texttt{\textbackslash colorA} and \texttt{\textbackslash colorB} in \texttt{\textit{<command>}};
\end{description}

\subsection{Options}\label{coloriopzioni}
\begin{description}
 \item[\texttt{noitemcolor}] cancels the list coloration in \texttt{itemize}, \texttt{enumerate} and \texttt{description};
\end{description}
\ppar
\begin{description}
 \item[\texttt{blue}, \texttt{green}, \texttt{red}] are generic color themes;
 \item[\texttt{guitgreen}] is a color theme using the colors of the Gruppo Utilizzatori Italiani di \TeX\ (\GuIT).
 \item[\texttt{sssupcolor1}, \texttt{sssupcolor2}, \texttt{sssupcolor3}] are color themes using the colors of the Scuola Superiore Sant'Anna logo;
 \item[\texttt{enscblue}] is a color theme using the colors of the École Normale Supérieure de Cachan logo.
\end{description}

\section{Document layout}
\texttt{bookest} provides options to control:
\begin{itemize}
 \item doubleside (inherets that ones of \texttt{book});
 \item margins;
 \item line spacing;
 \item paragraph indentation and spacing.
\end{itemize}

Moreover, \texttt{bookest} provides two commands to insert an image or a text as shipout picture.

\subsection{Options}
\begin{description}
 \item[\texttt{oneside}, \texttt{twoside}] toggles duplex options (default: \texttt{twoside});
\end{description}
\ppar
\begin{description}
 \item[\texttt{centered}] sets page margins (2.5 cm horizontal, 3 cm above and below) --- requires the \texttt{geometry} package;
 \item[\texttt{left5mm}] sets page margins allowing 5 mm for binding (3 cm left, 2 cm right and 3 cm above and below) requires the \texttt{geometry} package;
 \item[\texttt{left8mm}] sets page margins allowing 8 mm for binding (3.3 cm left, 1.7 cm right and 3 cm above and below) --- requires the \texttt{geometry} package;
\end{description}
\ppar
\begin{description}
 \item[\texttt{onehalfspacing}] line spacing to 1.5 --- requires the \texttt{setspace} package;
 \item[\texttt{doublespacing}] line spacing to 2 --- requires the \texttt{setspace} package;
\end{description}
\ppar
\begin{description}
 \item[\texttt{noparindent}] sets paragraph indentation to 0;
 \item[\texttt{noparskip}] sets paragraph spacing to 0.
\end{description}

\subsection{Commands}
\begin{description}
 \item [\texttt{\textbackslash shipouttext\{\textit{<rot>}\}\{\textit{<sc>}\}\{\textit{<text>}\}}] puts in every page the text \texttt{\textit{<text>}} as shipout picture, rotating it counterclockwise by \texttt{\textit{<rot>}} degrees and applying a scale factor \texttt{\textit{<sc>}}. The default color is gray 5\% --- requires the \texttt{setspace} package --- requires the \texttt{everyshi} and \texttt{color} packages;
 \item [\texttt{\textbackslash shipoutimage\{\textit{<options>}\}\{\textit{<file>}\}}] puts in everypage the image \texttt{\textit{<file>}} as shipout picture, using \texttt{\textit{<options>}} as \texttt{\textbackslash includegraphics} options --- requires the \texttt{everyshi} and \texttt{graphicx} packages.
\end{description}

\section{Headings and footers}
\texttt{bookest} provides commands to allow the user to set easily headings and footers. Moreover it provides a default setting that is different from \texttt{book}\footnote{To use the default \texttt{book} styles one has to simply use the command \texttt{\textbackslash pagestyle\{\textit{<style>}\}}.}.

\ppar
It also redefines chapter headings and the \texttt{plain} style to include colors.

\subsection{Commands}
\begin{description}
 \item[\texttt{\textbackslash setoddhead}, \texttt{\textbackslash setevenhead}] define odd and even page headings;
 \item[\texttt{\textbackslash oddheadtext}] is the text to be used in odd page headings (default: \texttt{\{\textbackslash colorA\{ \textbackslash slshape\textbackslash rightmark\}\textbackslash hfill\textbackslash thepage\}});
 \item[\texttt{\textbackslash evenheadtext}] is the text to be used in even page headings (default in the \texttt{oneside} case: \texttt{\textbackslash oddheadtext}; in the \texttt{twoside} case: \texttt{\{\textbackslash colorA\textbackslash thepage\textbackslash hfill\textbackslash slshape\textbackslash leftmark\}});
 \item[\texttt{\textbackslash setoddheadtext}, \texttt{\textbackslash setevenheadtext}] set the text in \texttt{\textbackslash oddheadtext} e \texttt{\textbackslash evenheadtext};
 \item[\texttt{\textbackslash setoddfoot}, \texttt{\textbackslash setevenfoot}] define odd and even page footers;
 \item[\texttt{\textbackslash oddfoottext}, \texttt{\textbackslash evenfoottext}] is the text to be used in odd and even page headings (default: (default: empty);
 \item[\texttt{\textbackslash setoddfoottext}, \texttt{\textbackslash setevenfoottext}] set the text in \texttt{\textbackslash oddfoottext} e \texttt{\textbackslash evenfoottext};
 \item[\texttt{\textbackslash setleftmark}, \texttt{\textbackslash setrightmark}] sets the text in \texttt{\textbackslash leftmark} e \texttt{\textbackslash rightmark};
 \item[\texttt{\textbackslash makeheadrule}] defines the horizontal rule in headings (default: \texttt{\{\textbackslash colorB\textbackslash hrule \textbackslash @width \textbackslash textwidth \textbackslash @height 0.4pt \textbackslash vskip-0.4pt\}});
 \item[\texttt{\textbackslash makefootrule}] defines the horizontal rule in headings (default: \texttt{\textbackslash makeheadrule});
\end{description}

\section{Title page layout}
\texttt{bookest} provides commands to allow the user to customize easily the title page of his document, especially starting from preset layouts that can be activated by the options in \ref{copertinaopzioni}.

\ppar
The default layout has author and title centered at top of the page and to the bottom of the page is a footer made by the content of \texttt{\textbackslash titlingpageprefooter} and of \texttt{\textbackslash titlingpagefooter} divided by a horizontal line. Between title and footer is the content of \texttt{\textbackslash titlingpagemiddle}.

The different options allow to vary the logo position; for each option \texttt{\textit{<optlogo>}} in \ref{copertinaopzioni} there exists a variant \texttt{\textit{<optlogo>}-nofooter} where no footer is present.

\subsection{Commands}
\begin{description}
 \item[\texttt{\textbackslash inslogo\{\textit{<file>}\}}] inserts the image \texttt{\textit{<file>}} with \texttt{\textbackslash includegraphics} options previously defined and used for the logo (default: \texttt{width=0.6\textbackslash paperwidth});
 \item[\texttt{\textbackslash setlogooptions\{\textit{<options>}\}}] defines \texttt{\textit{<options>}} as the \texttt{\textbackslash includegraphics} options to be used by \texttt{\textbackslash inslogo};
 \item[\texttt{\textbackslash logo}, \texttt{\textbackslash leftlogo}, \texttt{\textbackslash rightlogo}] are the path (relative or absolute) of the image to be used as logo depending on the chosen options (default for \texttt{\textbackslash logo} is the relative path \texttt{logo}, default for the others is \texttt{\textbackslash logo});
 \item[\texttt{\textbackslash setlogo\{\textit{<path>}\}}, \texttt{\textbackslash setleftlogo\{\textit{<path>}\}}, \texttt{\textbackslash setrightlogo\{\textit{<path>}\}}] set to \texttt{\textit{<path>}} the content of \texttt{\textbackslash logo}, \texttt{\textbackslash leftlogo}, \texttt{\textbackslash rightlogo};
 \item[\texttt{\textbackslash titlingpagemiddle}] is the text to be put in the middle of the \emph{titling page};
 \item[\texttt{\textbackslash settitlingpagemiddle}] sets the text in \texttt{\textbackslash titlingpagemiddle};
 \item[\texttt{\textbackslash titlingpageprefooter}] is the text to be put before the footer in the \emph{titling page};
 \item[\texttt{\textbackslash settitlingpageprefooter}] sets the text in \texttt{\textbackslash titlingpageprefooter};
 \item[\texttt{\textbackslash titlingpagefooter}] is the text to be used as footer in the \emph{titling page} (default: \texttt{\textbackslash today});
 \item[\texttt{\textbackslash settitlingpagefooter}] sets the text in \texttt{\textbackslash titlingpagefooter};
 \item[\texttt{\textbackslash settitlingpagetitle}] defines the format of the title in the \emph{titling page};
 \item[\texttt{\textbackslash titling}] inserts the \emph{titling page}.
\end{description}

\subsection{Options}\label{copertinaopzioni}
\begin{description}
 \item[\texttt{nofooter}] variant without footer of the default layout;
 \item[\texttt{logo}, \texttt{logo-nofooter}] adds to the default layout the logo in \texttt{\textbackslash logo} under the title (and matching \texttt{nofooter} variant) --- requires the \texttt{graphicx} package;
 \item[\texttt{logo-bg}, \texttt{logo-bg-nofooter}] adds to the default layout the logo in \texttt{\textbackslash logo} in the background (and matching \texttt{nofooter} variant) --- requires the \texttt{graphicx} and the \texttt{eso-pic} packages;
 \item[\texttt{logo-topl}, \texttt{logo-topl-nofooter}] adds to the default layout the logo in \texttt{\textbackslash leftlogo} at top left of the page before the title (and matching \texttt{nofooter} variant) --- requires the \texttt{graphicx} package;
 \item[\texttt{logo-topc}, \texttt{logo-topc-nofooter}] adds to the default layout the logo in \texttt{\textbackslash logo} at top center of the page before the title (and matching \texttt{nofooter} variant) --- requires the \texttt{graphicx} package;
 \item[\texttt{logo-topr}, \texttt{logo-topr-nofooter}]  adds to the default layout the logo in \texttt{\textbackslash rightlogo} at top left of the page before the title (and matching \texttt{nofooter} variant) --- requires the \texttt{graphicx} package;
 \item[\texttt{logo-toplr}, \texttt{logo-toplr-nofooter}]  adds to the default layout the logo in \texttt{\textbackslash leftlogo} at top left of the page and \texttt{\textbackslash rightlogo} at top right of the page before the title (and matching \texttt{nofooter} variant) --- requires the \texttt{graphicx} package;
\end{description}

\section{Miscellanea}
\texttt{bookest} provides also other little shortcuts, which can be useful when using the class and that are hereby listed:

\subsection{Commands}
\begin{description}
 \item[\texttt{\textbackslash setbibname\{\textit{<name>}\}}] renames the bibliography title to \texttt{\textit{<name>}};
 \item[\texttt{\textbackslash setcontentsname\{\textit{<name>}\}}] renames the contents title to \texttt{\textit{<name>}};
 \item[\texttt{\textbackslash ppar}] inserts a vertical space of \texttt{1.5ex} --- useful for example with the \texttt{noparskip} option;
 \item[\texttt{\textbackslash dimstleftskip}] sets \texttt{\textbackslash leftskip} to \texttt{1cm};
 \item[\texttt{\textbackslash UCase}] provides the command \texttt{\textbackslash MakeUppercase}, that is instead redefined as a null command to make heading and footer commands more flexible;
 \item[\texttt{\textbackslash epigraph\{\textit{<text1>}\}\{\textit{<text2>}\}\{\textit{<environment>}\}\{\textit{<l>}\}}] makes an epigraph, where \texttt{\textit{<text1>}} is divided from \texttt{\textit{<text2>}} by a horizontal line of color \texttt{B}. The epigraph has length \texttt{\textit{<l>}} and is contained in the environment \texttt{\textit{<environment>}}.
\end{description}

\subsection{Environments}
\begin{description}
 \item[\texttt{abstract}] is an environment of width \texttt{0.9\textbackslash textwidth}, with a parameter \texttt{\textit{<title>}} to be written in bold series before the text contained in the environment;
 \item[\texttt{dimst}] is an environment where the text is in \texttt{slshape} and with 1 cm extra for the left margin.
\end{description}

\section{Contacts}
For comments, suggestions or bug reports, you can contact me at the address \href{mailto:bresciani@sssup.it}{\textit{bresciani@sssup.it}}.

\end{document}
