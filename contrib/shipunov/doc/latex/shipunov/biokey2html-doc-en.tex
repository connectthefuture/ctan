\documentclass[a4paper,12pt]{article}
\usepackage{biokey}

\author{\IN A.V.Shipunova, \IN A.B.Shipunov}

\title{``\texttt{biokey2html}'',\\ the primer of hypertext web/typography identification key}

\date{}
\begin{document}
\maketitle

\subsection*{Example A}

This is the plain text source:

\bigskip\hrule
\begin{verbatim}

1. Leaves pinnatifid ... 2.

-- Leaves entire ... 3.

2. Plants perennial, leaves dissected to numerous lanceolate 
lobes ... Polemonium L.---Jacob's ladder.

-- Annual plants, leaves with narrow lobes ... 
Gilia L.---Gilia.

3(1). The compact inflorescence is surrounded by leaflike 
bracts ... Collomia Nutt.---Collomia.

-- The sparse inflorescence without specialised bracts ...
Plox L.---Phlox, Sweet-william.

\end{verbatim}

\subsection*{Example B}

This is the \TeX/\LaTeX variant (with stable numbers) of the same key. Please look to the Collomia---the italicizing is pretty simple (fo binomials one need to use \verb|\NN| command)q:

\bigskip\hrule
\begin{verbatim}

\Z1. Leaves pinnatifid \T 2.

\AN Leaves entire \T 3.

\Z2. Plants perennial, leaves dissected to numerous lanceolate 
lobes \T Polem\'onium L.---Jacob's ladder.

\AN Annual plants, leaves with narrow lobes \T 
G\'\i{}lia L.---Gilia.

\ZZ3(1). The compact inflorescence is surrounded by leaflike 
bracts \T\KN Coll\'omia Nutt.---Collomia.

-- The sparse inflorescence without specialised bracts \T
Plox L.---Phlox, Sweet-william.

\end{verbatim}

\subsection*{Example C}

This is the typorgaphic PDF output. To produce this, you need special style file, \texttt{biokey.sty}:

\bigskip\hrule\bigskip

\Z1. Leaves pinnatifid \T 2.

\AN Leaves entire \T 3.

\Z2. Plants perennial, leaves dissected to numerous lanceolate 
lobes \T Polem\'onium L.---Jacob's ladder.

\AN Annual plants, leaves with narrow lobes \T G\'\i{}lia L.---Gilia.

\ZZ3(1). The compact inflorescence is surrounded by leaflike 
bracts \T\KN Coll\'omia Nutt.---Collomia.

\AN The sparse inflorescence without specialised bracts \T
Plox L.---Phlox, Sweet-william.

\subsection*{Example D}

This is the variant with \textbf{automatic} numeration (you need to run \LaTeX{} twice to obtain results) \emph{with PDF hyperlinks}. You can produce if from example B manually, or can use the Perl-script (see below):

\bigskip\hrule
\begin{verbatim}

\TEZA{AA} Leaves pinnatifid \SSYLKA{AB}

\AN Leaves entire \SSYLKA{AC}

\TEZA{AB} Plants perennial, leaves dissected to numerous lanceolate 
lobes \T Polem\'onium L.---Jacob's ladder.

\AN Annual plants, leaves with narrow lobes \T G\'\i{}lia L.---Gilia.

\STEZA{AC}{AA} The compact inflorescence is surrounded by leaflike 
bracts \T Coll\'omia Nutt.---Collomia.

\AN The sparse inflorescence without specialised bracts \T
Plox L.---Phlox, Sweet-william.

\end{verbatim}

And output:

\bigskip

\TEZA{AA} Leaves pinnatifid \SSYLKA{AB}

\AN Leaves entire \SSYLKA{AC}

\TEZA{AB} Plants perennial, leaves dissected to numerous lanceolate 
lobes \T Polem\'onium L.---Jacob's ladder.

\AN Annual plants, leaves with narrow lobes \T G\'\i{}lia L.---Gilia.

\STEZA{AC}{AA} The compact inflorescence is surrounded by leaflike 
bracts \T Coll\'omia Nutt.---Collomia.

\AN The sparse inflorescence without specialised bracts \T
Plox L.---Phlox, Sweet-william.

\subsection*{Example F}

Three simple \textsf{Perl} scripts do the main work. You can run them all by command \texttt{biokey2html.sh} under UNIX shell on \textsf{Mac OS X} or \textsf{Linux}, or command \texttt{biokey2html.bat} under \textsf{Windows} console:

\begin{verbatim}
$ chmod 755 biokey2html.sh biokey2html*.pl
$ ./biokey2html.sh biokey2html-ex-en.tex 
Making relative LaTeX key... 
Making HTML title and paragrafs tags... 
Making reference tags... 
\end{verbatim}

If you want only to produce "automatic" PDF-key with hyperlinks, you should use the file with \texttt{.2} extension, which appears after \texttt{biokey2html.sh} run (see \texttt{biokey2html-ex-en.pdf}).

And the HTML result (part of code):

\bigskip\hrule
\begin{verbatim}

<p class="ST"><span class="TEZA"><a name="AA">1</a>.</span> 
Leaves pinnatifid ... <span class="SSYLKA"><a href="#AB">2</a>.
</span>

<p class="ST">-- Leaves entire ... <span class="SSYLKA">
<a href="#AC">3</a>.</span>

<p class="ST"><span class="TEZA"><a name="AB">2</a>.
</span> Plants perennial, leaves dissected to numerous lanceolate 
lobes ... <!--<a href="">--><span class="SP"> Polemonium L.--Jacob's ladder.
</span><!--</a>-->

<p class="ST">-- Annual plants, leaves with narrow lobes ...
<!--<a href="">--><span class="SP"> Glia L.--Gilia.</span>
<!--</a>-->

\end{verbatim}

The commented anchors placed there just in case if you are interested to insert links for images etc. The real examples of keys (which is widely used) are on the web-pages:

\begin{itemize}

\item Abramova L.A. et al. The key for most frequent plants in Chupa Gulf environs. [Electronic resource]. 2005. Mode of access:
\texttt{http://herba.msu.ru\\/shipunov2/belomor/2005/flora/ws\_key/ws\_key.htm}

\item Abramova L.A. et al. The key for most frequent plants in Chupa Gulf environs. [Electronic resource]. 2004. Mode of access:
\texttt{http://herba.msu.ru\\/shipunov/belomor/2004/flora/ws\_key.pdf}

\end{itemize}

\end{document}
