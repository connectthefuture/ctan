\documentclass[a4paper]{article}
\bibliographystyle{apalike}

\usepackage{biocon}
\newanimal{Homo sapiens}{}
\newplant{Arabidopsis thaliana}{}

\newcommand{\biocon}{{\tt biocon.sty}}

\title{The \biocon\ package}
\author{Pieter Edelman}

\begin{document}
  \maketitle

  \begin{abstract}
    \emph{Warning: this documentation is in an early state. Though usable, it is not very good at the moment.}
    The \biocon\ package attempts to automate the typesetting of biological entities. At the moment only species typesetting is done (very basic at the moment). Section \ref{secSpeciesIntro} handles with the conventions on species typesetting, and how this package may help, while section \ref{secSpeciesCommands} handles the commands to properly typeset species.
  \end{abstract}
  
  \tableofcontents
  
  \section{Introduction}
    As a biologist (to-be, that is), I often have to write papers in which species are used. To write the scientific name of a species follows a strict convention, which can tell you a lot about the species (``can'', not ``will''). So it is very important that such a name is typeset correctly (and of course it is nice). That is why I am writing the \biocon---biological conventions---package.
    
    I believe that the use macro's has some advantages. First of all, it can save a lot of typing (``can'', not ``will''). Second, you are sure every instance is typeset correctly.
    
    Currently, the \biocon\ package does not follow all the conventions, but the basics are there. Besides species, also genes and their associated products should be typeset, and maybe even more.
  %end{Introduction}
  
  \section{Typesetting species}
    \subsection{On species conventions}\label{secSpeciesIntro}
      \emph{Note: knowledge of this section is not required for using the package and can be skipped.}
      
      Typesetting biological species follows strict rules, laid down in \cite{ICBN} There is quite some discrepancy between typesetting of different kingdoms\footnote{``Kingdom'' is the lowest biosystematic branche existing. Although arbitary, the kingdoms exist of the Bacteria, Fungi, Planta and Animalia}, but there is one common factor, which is the basic species.
      
      Species names are \emph{always} built up of the name of the genus\footnote{``Genus'' is the taxonomical branch direct between the species} and is followed by the species-specific epiteton (often referred to as species name, which is, strictly spoken wrong, because the species name is the full construction described here). This is followed by the abbreviatiated name of the person who first described the species.
      
      For example, our own species, the human, has the scientific name  ... In this, \emph{Homo} is the genus, \emph{sapiens} is the epiteton, and ... is the author. 
      
      But that's not all folks. The first letter of the genus is always capitalized, while the rest is in lowercase. The epiteton is in lowercase only. This construction should be {\it italic}, but the author not.
      
      So, is that all folks?---No, of course it is not. In biological papers, it is not really nice to read the full name of a species, not even if the author is omitted. Instead, an abbreviated form is prefferd, consisting either of the first letter of the genus followed by the epiteton, or just the genus. And, of course, depending on the nature of the paper, somewhere the full name has to be used.
      
      Got it all? If ``no'', that's ok, because the \biocon\ package is here (this does not mean the package is useless if you got it). This package will help you with typesetting the species properly.
    %end{On species conventions}
    \subsection{The commands}\label{secSpeciesCommands}
      \subsubsection{Setting parameters}
        \begin{description}
          \item{{\tt $\backslash$newbacterium[Abbr]\{Genus epiton\}\{Author\}}, {{\tt $\backslash$newfungus[Abbr]\{Genus epiton\}\{Author\}}}, {{\tt $\backslash$newplant[Abbr]\{Genus epiton\}\{Author\}}}, {{\tt $\backslash$newanimal[Abbr]\{Genus epiton\}\{Author\}}}}\\
	    These commands are used to create new species names. Although a bit arbritary, four different classes of species are distinguished. This is because typesetting of these can differ.
	  
	    Every species of course has a genus and epiteton, and these have to be given as the first mandatory argument. If the case is not correct, the \biocon\ package automatically corrects this. If the epiteton is not known, or if more members of af a genus are targeted, fill in respectively ``sp.'' and ``spp.'' for the genus. \emph{Don't forget the ``.''{!}}
	  
	    The second mandatory argument is the author who first described the species. This field may be left blank if it is not going to be used (just write ``{\tt\{\}}'').
          
	    Every species of a given group has an unique identifier, by which the user can refer to it. By default, the capitalized first letter of the genus followed by the lowercase first letter of the epiteton is chosen (e.g. for \animal{Hs} this becomes ``Hs''). The optional argument speciefies another name. Please note that an identifier only has to be unique within a group, so a bacterium with the identifier ``Hs'' may exist besides an animal with the identifier ``Hs''.
	
	  \item{{\tt $\backslash$setabbreviation\{s|g\}}}\\
	    This command specifies how a species name is abbreviated. If ``{\tt s}'' (standard) is chosen (which is the default), a species name is abbreviated to G.~epiteton (e.g. \plant[ad]{At}). Otherwise if ``{\tt g}'' (genus) is chosen, then the genus name is used as abbreviation (e.g. \plant[gd]{At}).
        \end{description}	
      %End{Setting parameters}
      \subsubsection{Using parameters}
        \begin{description}
	  \item{{\tt $\backslash$bacterium[a|g|l|e(d)]\{Abbr\}}, {\tt $\backslash$fungus[a|g|l|e(d)]\{Abbr\}}, {\tt $\backslash$plant[a|g|l|e(d)]\{Abbr\}}, {\tt $\backslash$animal[a|g|l|e(d)]\{Abbr\}}}\\
	    This command is used to actually display a species name. In its simplest form, just the identifier is given. It then depends on the situation what output is given; if a species name is used the first time in the document, the full name (Genus epiteton). If it is used for the second time or more, it is abbreviated according to how it is specified with {\tt $\backslash$setabbreviation}.
	    
	    However, with the optional arguments, other modes can be forced. ``{\tt a}'' stands for ``abbreviated''. When this option is invoked, a name is always abbreviated to the abbreviation defined with {\tt $\backslash$setabbreviation}. ``{\tt g}'' stand for ``genus'' and with this option, only the genus name is displayed. When ``{\tt s}'' is used the name is abbreviated in the standard way (G. epiteton). With ``{\tt l}'' which stands for ``long'', the full name (Genus epiteton) can be forced. And with ``{\tt e}'' for ``extended'', the complete name inclusive author can be specified.
	\end{description}
      %end{Using parameters} 
    %end{The commands}
  %end{Typesetting species}  
  
  \bibliography{Bibliography}
\end{document}


