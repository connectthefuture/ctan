\documentclass{schuleab}
\usepackage[utf8]{inputenc}
\usepackage[ngerman]{babel}
\usepackage{schulinf}
\definecolor{blau}{rgb}{0,0,0.75}         
\definecolor{orange}{rgb}{0.8,0.3,0}      
\usepackage{hyperref}
\hypersetup{
 pdftitle = {\LaTeX-Klassen und Pakete für den Einsatz im Bereich der Schule},
 pdfsubject = {},
 pdfauthor = {Johannes Pieper, Johannes Kuhaupt, Ludger Humbert, Andr\'e Hilbig},
 colorlinks = true,
 hypertexnames = true,
 linkcolor=blau, %
 filecolor=orange, %
 citecolor=blau,
 menucolor=orange, %
 urlcolor=orange,
 breaklinks=true
}
\usepackage{caption,xparse,xargs}
\usepackage[german=guillemets]{csquotes}
\usepackage{schule,syntaxdi,schulinf,schulphy}
\usepackage{placeins,float,prettyref}
\usepackage{newfloat}

\lstset{  %
	language=[LaTeX]TeX,                 % the language of the code
	basicstyle=\small,            % the size of the fonts that are used for the code
	numbers=left,                    % where to put the line-numbers
	numberstyle=\footnotesize,           % the size of the fonts that are used for the line-numbers
	stepnumber=2,                    % the step between two line-numbers. If it's 1, each line will be numbered
	numbersep=5pt,                   % how far the line-numbers are from the code
	backgroundcolor=\color{white},       % choose the background color. You must add \usepackage{color}
	showspaces=false,                % show spaces adding particular underscores
	showstringspaces=false,          % underline spaces within strings
	showtabs=false,                  % show tabs within strings adding particular underscores
	frame=false,                    % adds a frame around the code
	tabsize=2,                       % sets default tabsize to 2 spaces
	resetmargins=true,
	captionpos=b,                    % sets the caption-position to bottom
	title=,                    % show the filename of files included with \lstinputlisting;
	breaklines=true,
	breakautoindent=true,
	prebreak=\mbox{ $\curvearrowright$},
	postbreak=\mbox{$\rightsquigarrow$ },
	linewidth=\columnwidth,
	breakatwhitespace=true,         % sets if automatic breaks should only happen at whitespace 
	numberstyle=\tiny\color{gray},         % line number style
	keywordstyle=\color{blue},           % keyword style
	commentstyle=\color{OliveGreen},        % comment style
	stringstyle=\color{mauve},          % string literal style
	morekeywords={
	zeitpunkt, punkteitem, scaleSequenzdiagramm, newthread, newthreadtwo, 
	newinst, node, chainin, draw, to, dokName, jahrgang, minisec, subsection, 
	glqq, grqq, euro
}                % if you want to add more keywords to the set
}
\newcommand{\materialsammlung}{\url{http://ddi.uni-wuppertal.de/material/materialsammlung/index.html}}

\dokName{Fahrkartenauskunft}
\jahrgang{EF}

\begin{document}
 \thispagestyle{empty}
 \section*{Arbeitsblatt zur Identifikation von Objekten mit der \enquote{Methode nach Abbott}}
 \footnotesize{entnommen aus: \materialsammlung}
 % Der folgende kenntlich gemachte Abschnitt ist in der Zusammenarbeit von 
 % Informatikreferendaren und ehemaligen Informatikreferendaren der 
 % Studienseminare (heute ZfsL) Arnsberg, Hamm und Solingen entstanden.
 %
 % Der Abschnitt steht unter der Lizenz: Creative Commons by-nc-sa Version 4.0
 % http://creativecommons.org/licenses/by-nc-sa/4.0/deed.de
 %
 % Nach dieser Lizenz darf der Abschnitt beliebig kopiert und bearbeitet werden,
 % sofern das Folgeprodukt wiederum unter gleichen Lizenzbedingungen vertrieben
 % und auf die ursprünglichen Urheber verwiesen wird.
 % Eine kommerzielle Nutzung ist ausdrücklich ausgeschlossen.
 %
 % Die Namensnennung durch einen Verweis und die Lizenzangabe der ursprünglichen
 % Urheber auf den Materialien für Schülerinnen und Schüler ist erforderlich.
 %
 % Die vollständige Sammlung der Dokumente steht unter
 % http://ddi.uni-wuppertal.de/material/materialsammlung/
 % zur Verfügung.
 %
 % Das LaTeX-Paket zum Setzen der Dokumente der Sammlung steht
 % unter  http://www.ctan.org/pkg/
 % zur Verfügung.
	 \begin{lstlisting}[gobble=6,caption={}]
			\documentclass{schuleab}
			\usepackage[utf8]{inputenc}
			\usepackage{schulinf}
			\dokName{Fahrkartenauskunft}
			\jahrgang{EF}

			\begin{document}
				\section*{Problembeschreibung Fahrkartenauskunft}
					\subsection*{Ausgangssituation} 
						Das örtliche Nahverkehrsunternehmen »NahUnt« will an den Bushaltestellen Fahrscheinautomaten installieren. An dem Automaten kann der Kunde eine Entfernungszone per Knopfdruck wählen. Es gibt drei Entfernungszonen mit unterschiedlichen Preisen: 1.Zone: 1,10~\euro, 2.Zone: 1,90~\euro, 3.Zone: 4,20~\euro. In einem Display steht als erstes der Text »Bitte wählen Sie eine Entfernungszone aus«. Nach der Betätigung einer Entfernungszonentaste soll die ausgewählte Zone und der Preis angezeigt werden. 

				\minisec{Aufgabe}
					\begin{enumerate}
						\item Ermitteln Sie die vorkommenden Objekte und die zugehörigen Attribute und Attributwerte und notieren Sie diese mit Objektkarten.
						\item Erstellen Sie das Objektdiagramm.
						\item Fassen Sie die Objekte geeignet zu Klassen zusammen und dokumentieren diese mit Klassenkarten.
						\item Erstellen Sie das Klassendiagramm.
					\end{enumerate}
			\end{document}
		\end{lstlisting}
			\clearpage
				\section*{Problembeschreibung Fahrkartenauskunft}
					\subsection*{Ausgangssituation} 
						Das örtliche Nahverkehrsunternehmen »NahUnt« will an den Bushaltestellen Fahrscheinautomaten installieren. An dem Automaten kann der Kunde eine Entfernungszone per Knopfdruck wählen. Es gibt drei Entfernungszonen mit unterschiedlichen Preisen: 1.Zone: 1,10~\euro, 2.Zone: 1,90~\euro, 3.Zone: 4,20~\euro. In einem Display steht als erstes der Text »Bitte wählen Sie eine Entfernungszone aus«. Nach der Betätigung einer Entfernungszonentaste soll die ausgewählte Zone und der Preis angezeigt werden. 

				\minisec{Aufgabe}
					\begin{enumerate}
						\item Ermitteln Sie die vorkommenden Objekte und die zugehörigen Attribute und Attributwerte und notieren Sie diese mit Objektkarten.
						\item Erstellen Sie das Objektdiagramm.
						\item Fassen Sie die Objekte geeignet zu Klassen zusammen und dokumentieren diese mit Klassenkarten.
						\item Erstellen Sie das Klassendiagramm.
					\end{enumerate}
\end{document}
