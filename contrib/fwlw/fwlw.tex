\documentclass[pagesize=auto]{scrartcl}

\usepackage{fixltx2e}
\setlength\parskip{3pt}

\newcommand*{\mail}[1]{\href{mailto:#1}{\texttt{#1}}}
\newcommand*{\pkg}[1]{\textsf{#1}}
\newcommand*{\cls}[1]{\textsf{#1}}
\newcommand*{\cs}[1]{\texttt{\textbackslash#1}}
\makeatletter
\newcommand*{\cmd}[1]{\cs{\expandafter\@gobble\string#1}}
\makeatother
\newcommand*{\env}[1]{\texttt{#1}}
\newcommand*{\opt}[1]{\texttt{#1}}
\newcommand*{\meta}[1]{\textlangle\textsl{#1}\textrangle}
\newcommand*{\marg}[1]{\texttt{\{}\meta{#1}\texttt{\}}}
\newcommand*{\oarg}[1]{\texttt{[}\meta{#1}\texttt{]}}
\newcommand*{\widevisiblespace}{%
  \vrule height 2pt depth 0pt width 0.2pt
  \kern -.1pt
  \vrule height 0.2pt depth 0pt width 15pt
  \kern -.1pt
  \vrule height 2pt depth 0pt width 0.2pt}

\addtokomafont{title}{\rmfamily}

\title{The \pkg{fwlw} package}
\subtitle{First Word, Last Word}
\author{Donald Arseneau}
\date{1995}


\begin{document}

\maketitle

\noindent
The fwlw package provides a mechanism to determine the first and
last words on the current page, plus the first word on the \emph{next}
page.  (So perhaps it should have been named `fwlwnw'). These words
can be used in head-lines or foot-lines.  The `words' you see may 
not be real words, but any unbreakable object.

Two pagestyles are defined to print these words: 
\verb+\pagestyle{NextWordFoot}+ which helps you read ahead to the 
word on the next page; and \verb+\pagestyle{fwlwhead}+ which is like 
the headers in a lexicon.  Or you can use the
words in your own page-style. The words are made available in the box
registers:
%
\begin{labeling}[\hspace{\labelsep}--]{\quad\cmd{\FirstWordBox}}
\setlength\itemsep{0pt}
\item[\quad\cmd{\FirstWordBox}] first word on this page
\item[\quad\cmd{\NextWordBox}] first word on next page
\item[\quad\cmd{\LastWordBox}] last word on this page
\end{labeling}
%
Use them in your header lines like: \verb+\usebox\LastWordBox+.

Such labelling does not make sense when \cmd{\chapter} generates a page
break, so the last page before a \cmd{\chapter} (or any \cmd{\clearpage}) gets 
a blank ``next word'', and the first page of the chapter gets a blank
``first word''. The fwlwhead page style produces blank headers on 
float pages.

Note that `words' are any unsplittable unit and may be things like:
%
\begin{itemize}
\setlength\itemsep{0pt}
\item two\textvisiblespace words (\texttt{two}%
      \lower 2pt\hbox{\verb+~+}\texttt{words})
      with unbreakable space between them
\item \widevisiblespace Word  (where \widevisiblespace\ represents a 
      paragraph indent)
\item a whole displayed equation
\item the first column of an aligned equation
\item a list item label
\item anomalously blank, if there are \cmd{\write}s or split footnotes etc.
\item blank for the first word of the document (because of a \cmd{\write})
\item partial words like  `par-'  or  `-tial' due to hyphenation.
\end{itemize}
%
In short, a wide range of irregularities will cause non-words to
be captured, or nothing at all.  


\end{document}
