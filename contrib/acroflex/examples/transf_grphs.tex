\documentclass{article}
\usepackage{amsmath}
\usepackage[%
    web={pro,tight,usesf},
    eforms,exerquiz,dljslib={ImplMulti},
    graphicxsp={showembeds}
]{aeb_pro}
\usepackage{acroflex}

\margins{.25in}{.25in}{24pt}{.25in} % left,right,top, bottom
\screensize{6.8in}{5.5in}             % height, width dimensions

\DeclareDocInfo
{
    title=Exploring Graphical Transformations,
    author=D. P. Story,
    university=Northwest Florida State College,
    email=storyd@owc.edu,
    subject=Demo of the acroflex and the rmannot package,
    keywords={Adobe Acrobat, JavaScript, Adobe FLEX 3, ActionScript},
    talksite=\url{http://faculty.owc.edu/math/storyd},
    talkdate={\today},
    copyrightStatus=True,
    copyrightNotice={Copyright (C) \the\year, D. P. Story},
    copyrightInfoURL=http://www.acrotex.net
}
\talkdateLabel{Published:}

\setWindowOptions{showtitle}

\requiredVersionMsg{%
    Version 9 required: This document may be viewed, but version 9
    of Acrobat/Adobe Reader is required to view the rich media
    annotations present in this document.}
\requiredVersionMsgRedirect{%
    Version 9 required: This document may be viewed, but version 9
    of Acrobat/Adobe Reader is required to view the rich media
    annotations present in this document.}
\requiresVersion[warnonly]{9}

\def\AcroTeX{Acro\!\TeX}
\def\myURL{http://www.math.uakron.edu/\string~dpstory}
%
% The following definitions are for the rmannot package.
% The \saveNamedPath associates the name AcroAd with the
% path to a Flash animation, Acro_Advertiser.swf.
% The two \makePoster definitions are for the posters of the
% rich media annotations. The first for the animation on page 1
% and the second for all the AcroFlex graphing screens.
\newcommand{\myRMFiles}{C:/Users/Public/Documents/My TeX Files/%
    tex/latex/aeb/aebpro/rmannot/RMfiles}
\saveNamedPath{AcroAd}{\myRMFiles/Acro_Advertiser.swf}
%
% See the rmannot documentation. You can create your own
% poster graphic and declare it in the preamble.
%
\makePoster[hiresbb]{aflogo}{aflogo1}
\makePoster[hiresbb]{AcroAd_poster}{AcroAd_poster}

\parindent=0pt
\pagestyle{empty}   %\previewtrue
\def\graphAndControls{\kern0pt\noindent
    \graphScreen[poster=aflogo]{\hScreenGraph}{\vScreenGraph}\\[1ex]
    \makebox[\hScreenGraph][l]{%
        \savedelSelBtn[\textSize{7}\CA{S}]{9bp}{11bp}\kern1bp
        \functionSelect{50bp}{11bp}\hfill
        \funcInputField{\hScreenGraph-50bp-6bp-9bp}{11bp}%
    }\\[1ex]
    \makebox[\hScreenGraph][c]{\scriptsize
        \graphBtn[\textSize{7}]{}{9bp}\kern1pt
        \graphClrBtn[\textSize{7}]{}{9bp}\hfill
        \hShiftL{\raisebox{1bp}{\reflectbox{\ding{220}}}}\,%
        \vShiftU{\raisebox{1bp}{\rotatebox{90}{\ding{220}}}}\,\,%
        \amtShift[\textSize{7}]{12bp}{9bp}\,\,%
        \vShiftD{\raisebox{1bp}{\rotatebox[origin=c]{-90}{\ding{220}}}}
        \hShiftR{\raisebox{1bp}{\ding{220}}}\enspace
        \hfill\zoomInOut[\textSize{7}]{}{9bp}%
    }\\[1pt]%
    \makebox[\hScreenGraph][l]{\scriptsize
        $x = $ \strut\domMin[\textSize{7}]{36bp}{9bp}\ldots
            \domMax[\textSize{7}]{36bp}{9bp}\hfill
        $n = $\numPoints[\textSize{7}]{16bp}{9bp}
    }\\[1pt]%
    \makebox[\hScreenGraph][l]{\scriptsize
        $y = $ \rngMin[\textSize{7}]{36bp}{9bp}\ldots
            \rngMax[\textSize{7}]{36bp}{9bp}%
    }%\\[1pt]%%
%    \makebox[\hScreenGraph][l]{\scriptsize
%        $t = $ \strut\domMinP[\textSize{7}]{36bp}{9bp}\ldots
%            \domMaxP[\textSize{7}]{36bp}{9bp}%
%    }
\par
}

%
% Display an advertisement on the opening page through the
% \optionalPageMatter command.
%
\optionalPageMatter{%
\begin{center}
\begin{minipage}{.7\linewidth}
    \resizebox{\linewidth}{!}{%
        \rmAnnot[poster=AcroAd_poster,enabled=pageopen]{612bp}{265bp}{AcroAd}}%
\end{minipage}
\end{center}
}

\begin{document}

\maketitle



% Basic parameters. These are the graphing screen dimensions you wish
% for the AcroFleX graphing widget, and the \graphName is the base name used by
% all the supporting components to the graphing screen.

\dimScreenGraph{160bp}{160bp}
%\dimScreenGraph{186bp}{186bp*3/4}
\graphName{graph1}

\begin{center}\bfseries\Large\color{blue}
    Transforming Graphs
\end{center}

\medskip
This is a demo of graphical transformations, and is based on my recent work over the summer, 2008,
on creating a interactive graphing system using a Flash application, written by myself, for PDF.
Adobe Reader 9.0 or later is required to use this system.
\medskip

\begin{minipage}[t]{\hScreenGraph + 10bp}
%
% Insert the \graphScreen with all possible controls. Plot data entered by the
% user is interactive mode.
\graphAndControls
\end{minipage}\hfill
\begin{minipage}[t]{\linewidth - \hScreenGraph - 10bp}\small\par
\textbf{\textcolor{red}{Instructions:}} This graphing system can graph
a function of $x$.
\begin{questions}
    \item For a function of $x$, enter an algebraic expression to be graphed, e.g.,
        \verb!x^2!, \verb!x^3!, \verb!3x^2-1!, \verb!|x|!, \verb!sqrt(x)!
    \item Set the graph viewing window: The range of the horizontal
        axis ($x$-axis) and range of the vertical axis ($y$-axis).
        For parametric plotting, set the range of the $t$ variable.
    \item Enter the number of points, $n$, to plot.
    \item Click the \textsf{Graph It!} button.
    \item Shift the viewing window horizontally or vertical;
    or zoom in or out.
    \item Click \texttt{Clear} to clear the plot; use
         \textsf{shift-click} to deactivate the graphing screen.
\end{questions}
\end{minipage}

\medskip

\noindent\makebox[\linewidth][c]{\rule{.67\linewidth}{.4pt}}

\medskip

\parskip3pt

\defineGraphJS{graph=c2,xInterval={[d_min,d_max]},yInterval={[r_min,r_max]},
    noquotes,points=n_Pts}{newFunc}{\verticalShift}
\defineGraphJS{graph=c2,xInterval={[d_min,d_max]},yInterval={[r_min,r_max]},
    noquotes,points=n_Pts}{newFunc}{\horizontalShift}
\defineGraphJS{graph=c2,xInterval={[d_min,d_max]},yInterval={[r_min,r_max]},
    noquotes,points=n_Pts}{newFunc}{\verticalShrink}
\defineGraphJS{graph=c2,xInterval={[d_min,d_max]},yInterval={[r_min,r_max]},
    noquotes,points=n_Pts}{newFunc}{\horizontalShrink}
\def\gatherGraphingData{%
    var f = this.getField("graph1theFunction").value;\r\t
    var d_min = this.getField("graph1theDom.min").value;\r\t
    var d_max = this.getField("graph1theDom.max").value;\r\t
    var r_min = this.getField("graph1theRng.min").value;\r\t
    var r_max = this.getField("graph1theRng.max").value;\r\t
    var n_Pts = this.getField("graph1numNodes").value;\r\t
}

Explore graphical transformations by entering numbers into the fields below. Put your mouse
over the fields for additional instructions. Click on the green links to populate the
graphing screen with suggested examples, then manipulate these examples using the controls below.

\smallskip

\begin{center}
\begin{tabular}{lll}
\textbf{Description} & \textbf{Input} & \textbf{Suggested Examples}\\

Vertical Shift & \textField[\textSize{7}\TU{Enter a number (positive
or negative) to shift the graph vertically up or down.}
    \AA{\AAKeystroke{%
        if (event.willCommit) {\r\t
            \gatherGraphingData
            var _C = event.value;\r\t
            if ( isFinite(_C) ) {\r\t\t
                var newFunc  = f+"+"+(_C);\r\t\t
                \verticalShift\r\t
            }\r
        }
}}]{vShift}{20bp}{9bp}&
\sgraphLink{populate,xInterval={[-2,2]},yInterval={[-3,3]},points=40}{x}{$x$},
\sgraphLink{populate,xInterval={[-2,2]},yInterval={[-2,4]},points=40}{x^2}{$x^2$},
\sgraphLink{populate,xInterval={[-2,2]},yInterval={[-4,4]},points=40}{x^3}{$x^3$},
\sgraphLink{populate,xInterval={[-2,3]},yInterval={[-3,6]},points=40}{(x-1)^2+1}{$(x-1)^2+1$}\\
%
%
Horizontal Shift & \textField[\textSize{7}\TU{Enter a number
(positive or negative) to shift the graph horizontally right or
left.}
    \AA{\AAKeystroke{%
        if (event.willCommit) {\r\t
            \gatherGraphingData
            var _C = event.value;\r\t
            if ( isFinite(_C) ) {\r\t\t
                newFunc = f.replace(/x/g, "(x-("+_C+"))");\r\t\t
                \horizontalShift\r\t
            }\r
        }
}}]{hShift}{20bp}{9bp}&
\sgraphLink{populate,xInterval={[-2,4]},yInterval={[-2,2]},points=60}{sqrt(x)}{$\sqrt{x}$},
\sgraphLink{populate,xInterval={[-3,3]},yInterval={[-3,3]},points=60}{x^{1/3}}{$\sqrt[3]{x}$},
\sgraphLink{populate,xInterval={[-2,4]},yInterval={[-2,3]},points=60}{sqrt(x-1)+1}{$\sqrt{x-1}+1$}\\
%
%
Vertical Stretch/Shrink& \textField[\textSize{7}\TU{Enter a positive number
to stretch or shrink the graph vertically.}
    \AA{\AAKeystroke{%
        if (event.willCommit) {\r\t
            var f = this.getField("graph1theFunction").value;\r\t
            var d_min = this.getField("graph1theDom.min").value;\r\t
            var d_max = this.getField("graph1theDom.max").value;\r\t
            var r_min = this.getField("graph1theRng.min").value;\r\t
            var r_max = this.getField("graph1theRng.max").value;\r\t
            var n_Pts = this.getField("graph1numNodes").value;\r\t
            var _C = event.value;\r\t
            if ( isFinite(_C) && (Number(_C) > 0) ) {\r\t\t
                newFunc = String("(_C)*("+f+")")\r\t\t
                \verticalShrink\r\t
            }\r
        }
}}]{vShrink}{20bp}{9bp}&
\sgraphLink{populate,xInterval={[-2,2]},yInterval={[-2,4]},points=40}{|x-1|}{$|x|$},
\sgraphLink{populate,xInterval={[-2,3]},yInterval={[-2,2]},points=40}{|x-1|}{$|x-1|$}\\
%
%
Horizontal Stretch/Shrink& \textField[\textSize{7}\TU{Enter a positive number
to stretch or shrink the graph horizontally.}
    \AA{\AAKeystroke{%
        if (event.willCommit) {\r\t
            var f = this.getField("graph1theFunction").value;\r\t
            var d_min = this.getField("graph1theDom.min").value;\r\t
            var d_max = this.getField("graph1theDom.max").value;\r\t
            var r_min = this.getField("graph1theRng.min").value;\r\t
            var r_max = this.getField("graph1theRng.max").value;\r\t
            var n_Pts = this.getField("graph1numNodes").value;\r\t
            var _C = event.value;\r\t
            if ( isFinite(_C) && (Number(_C) > 0) ) {\r\t\t
                newFunc=f.replace(/x/g, "(("+_C+")*x)");\r\t\t
                \horizontalShrink\r\t
            }\r
        }
}}]{hShrink}{20bp}{9bp}&\\
%
%
Reflect w/resp $y$-axis & \pushButton[\textSize{7}\CA{y-Axis}
    \TU{Press to reflect current graph with respect to the y-axis.}
    \A{\JS{%
            var f = this.getField("graph1theFunction").value;\r\t
            var d_min = this.getField("graph1theDom.min").value;\r\t
            var d_max = this.getField("graph1theDom.max").value;\r\t
            var r_min = this.getField("graph1theRng.min").value;\r\t
            var r_max = this.getField("graph1theRng.max").value;\r\t
            var n_Pts = this.getField("graph1numNodes").value;\r\t
            newFunc=f.replace(/x/g, "((-1)*x)");\r\t\t
            \horizontalShrink\r\t
}}]{y_reflect}{}{9bp}&
Click \graphClrBtn[\textSize{7}]{}{9bp} before clicking on green links\\
%
%
Reflect w/resp $x$-axis & \pushButton[\textSize{7}\CA{x-Axis}
    \TU{Press to reflect current graph with respect to the x-axis.}
    \A{\JS{%
            var f = this.getField("graph1theFunction").value;\r\t
            var d_min = this.getField("graph1theDom.min").value;\r\t
            var d_max = this.getField("graph1theDom.max").value;\r\t
            var r_min = this.getField("graph1theRng.min").value;\r\t
            var r_max = this.getField("graph1theRng.max").value;\r\t
            var n_Pts = this.getField("graph1numNodes").value;\r\t
                newFunc = String("(-1)*("+f+")")\r\t\t
                \verticalShrink\r\t
}}]{x_reflect}{}{9bp}&
\end{tabular}
\end{center}
\end{document}
