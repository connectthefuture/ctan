% \iffalse meta-comment
%
% Copyright (C) 2015 by Moritz Schubotz
% -----------------------------------
%
% This file may be distributed and/or modified under the
% conditions of the LaTeX Project Public License, either version 1.3
% of this license or (at your option) any later version.
% The latest version of this license is in:
%
% http://www.latex-project.org/lppl.txt
%
% and version 1.3 or later is part of all distributions of LaTeX
% version 2005/12/01 or later.
%
% \fi
%
% \iffalse
%<*driver>
\ProvidesFile{texvc.dtx}
%</driver>
%<package>\NeedsTeXFormat{LaTeX2e}[2005/12/01]
%<package>\ProvidesPackage{texvc}
%<*package>
[2015/11/09 v1.0 Initial Version]
%</package>
%<package>\RequirePackage{amsmath}
%<package>\RequirePackage{amsfonts}
%<package>\RequirePackage{amssymb}
%<package>
%<package>\PassOptionsToPackage{dvips,usenames}{color}
%<package>\RequirePackage{color}
%<package>
%%<package>\PassOptionsToPackage{greek}{babel}
%%<package>\RequirePackage{babel}
%%<package>\RequirePackage{teubner}
%<package>
%<package>\RequirePackage{eurosym}
%<package>
%<package>\RequirePackage{cancel}
%<*driver>
\documentclass{ltxdoc}
\usepackage{texvc}[2015/11/09]
\EnableCrossrefs
\CodelineIndex
\RecordChanges
\begin{document}
  \DocInput{texvc.dtx}
  \PrintChanges
  \PrintIndex
\end{document}
%</driver>
% \fi
%
% \CheckSum{0}
%
% \CharacterTable
%  {Upper-case    \A\B\C\D\E\F\G\H\I\J\K\L\M\N\O\P\Q\R\S\T\U\V\W\X\Y\Z
%   Lower-case    \a\b\c\d\e\f\g\h\i\j\k\l\m\n\o\p\q\r\s\t\u\v\w\x\y\z
%   Digits        \0\1\2\3\4\5\6\7\8\9
%   Exclamation   \!     Double quote  \"     Hash (number) \#
%   Dollar        \$     Percent       \%     Ampersand     \&
%   Acute accent  \'     Left paren    \(     Right paren   \)
%   Asterisk      \*     Plus          \+     Comma         \,
%   Minus         \-     Point         \.     Solidus       \/
%   Colon         \:     Semicolon     \;     Less than     \<
%   Equals        \=     Greater than  \>     Question mark \?
%   Commercial at \@     Left bracket  \[     Backslash     \\
%   Right bracket \]     Circumflex    \^     Underscore    \_
%   Grave accent  \`     Left brace    \{     Vertical bar  \|
%   Right brace   \}     Tilde         \~}
%
%
% \changes{v1.0}{2015/11/09}{Initial version}
%
% \GetFileInfo{texvc.sty}
%
% \DoNotIndex{\newcommand,\newenvironment}
%
% \title{The \textsf{texvc} package\thanks{This document
% corresponds to \textsf{texvc}~\fileversion,
% dated \filedate.}}
% \author{Moritz Schubotz \\ \texttt{schubotz@tu-berlin.de}}
% 
% \maketitle
%
% \begin{abstract}
% This package provides all LaTeX command availible in MediaWiki.
% This includes several packages like amsmath, and adds some specific
% commands such as \texttt{\textbackslash Reals}.
% \end{abstract}
%
% \section{Introduction}
%
% \subsection{Arrows}
% The first group of MediaWiki coustom command (other\_delimiters2) defines short hand notations for some arrorws.
%
% \DescribeMacro{\darr} Short hand notation for arrow $\downarrow.$ 
%
%
% \DescribeMacro{\dArr} Short hand notation for arrow $\Downarrow.$ 
%
%
% \DescribeMacro{\Darr} Short hand notation for arrow $\Downarrow.$ 
%
%
% \DescribeMacro{\lang} Short hand notation for arrow $\langle.$ 
%
%
% \DescribeMacro{\rang} Short hand notation for arrow $\rangle.$ 
%
%
% \DescribeMacro{\uarr} Short hand notation for arrow $\uparrow.$ 
%
%
% \DescribeMacro{\uArr} Short hand notation for arrow $\Uparrow.$ 
%
%
% \DescribeMacro{\Uarr} Short hand notation for arrow $\Uparrow.$ 
%
% \subsection{Literals}
% 
% The second group of MediaWiki coustom commands (other\_litereals3) defines short hand notations for some literals.
%
% \DescribeMacro{\C} Short hand notation for literal $\mathbb{C}.$ 
%
%
% \DescribeMacro{\H} Short hand notation for literal $\mathbb{H}.$ 
%
%
% \DescribeMacro{\N} Short hand notation for literal $\mathbb{N}.$ 
%
%
% \DescribeMacro{\Q} Short hand notation for literal $\mathbb{Q}.$ 
%
%
% \DescribeMacro{\R} Short hand notation for literal $\mathbb{R}.$ 
%
%
% \DescribeMacro{\Z} Short hand notation for literal $\mathbb{Z}.$ 
%
%
% \DescribeMacro{\alef} Short hand notation for literal $\aleph.$ 
%
%
% \DescribeMacro{\alefsym} Short hand notation for literal $\aleph.$ 
%
%
% \DescribeMacro{\Alpha} Short hand notation for literal $\mathrm{A}.$ 
%
%
% \DescribeMacro{\and} Short hand notation for literal $\land.$ 
%
%
% \DescribeMacro{\ang} Short hand notation for literal $\angle.$ 
%
%
% \DescribeMacro{\Beta} Short hand notation for literal $\mathrm{B}.$ 
%
%
% \DescribeMacro{\bull} Short hand notation for literal $\bullet.$ 
%
%
% \DescribeMacro{\Chi} Short hand notation for literal $\mathrm{X}.$ 
%
%
% \DescribeMacro{\clubs} Short hand notation for literal $\clubsuit.$ 
%
%
% \DescribeMacro{\cnums} Short hand notation for literal $\mathbb{C}.$ 
%
%
% \DescribeMacro{\Complex} Short hand notation for literal $\mathbb{C}.$ 
%
%
% \DescribeMacro{\Dagger} Short hand notation for literal $\ddagger.$ 
%
%
% \DescribeMacro{\diamonds} Short hand notation for literal $\diamondsuit.$ 
%
%
% \DescribeMacro{\Doteq} Short hand notation for literal $\doteqdot.$ 
%
%
% \DescribeMacro{\doublecap} Short hand notation for literal $\Cap.$ 
%
%
% \DescribeMacro{\doublecup} Short hand notation for literal $\Cup.$ 
%
%
% \DescribeMacro{\empty} Short hand notation for literal $\emptyset.$ 
%
%
% \DescribeMacro{\Epsilon} Short hand notation for literal $\mathrm{E}.$ 
%
%
% \DescribeMacro{\Eta} Short hand notation for literal $\mathrm{H}.$ 
%
%
% \DescribeMacro{\exist} Short hand notation for literal $\exists.$ 
%
%
% \DescribeMacro{\ge} Short hand notation for literal $\geq.$ 
%
%
% \DescribeMacro{\gggtr} Short hand notation for literal $\ggg.$ 
%
%
% \DescribeMacro{\hAar} Short hand notation for literal $\Leftrightarrow.$ 
%
%
% \DescribeMacro{\harr} Short hand notation for literal $\leftrightarrow.$ 
%
%
% \DescribeMacro{\Harr} Short hand notation for literal $\Leftrightarrow.$ 
%
%
% \DescribeMacro{\hearts} Short hand notation for literal $\heartsuit.$ 
%
%
% \DescribeMacro{\image} Short hand notation for literal $\Im.$ 
%
%
% \DescribeMacro{\infin} Short hand notation for literal $\infty.$ 
%
%
% \DescribeMacro{\Iota} Short hand notation for literal $\mathrm{I}.$ 
%
%
% \DescribeMacro{\isin} Short hand notation for literal $\in.$ 
%
%
% \DescribeMacro{\Kappa} Short hand notation for literal $\mathrm{K}.$ 
%
%
% \DescribeMacro{\larr} Short hand notation for literal $\leftarrow.$ 
%
%
% \DescribeMacro{\Larr} Short hand notation for literal $\Leftarrow.$ 
%
%
% \DescribeMacro{\lArr} Short hand notation for literal $\Leftarrow.$ 
%
%
% \DescribeMacro{\le} Short hand notation for literal $\leq.$ 
%
%
% \DescribeMacro{\lrarr} Short hand notation for literal $\leftrightarrow.$ 
%
%
% \DescribeMacro{\Lrarr} Short hand notation for literal $\Leftrightarrow.$ 
%
%
% \DescribeMacro{\lrArr} Short hand notation for literal $\Leftrightarrow.$ 
%
%
% \DescribeMacro{\Mu} Short hand notation for literal $\mathrm{M}.$ 
%
%
% \DescribeMacro{\natnums} Short hand notation for literal $\mathbb{N}.$ 
%
%
% \DescribeMacro{\ne} Short hand notation for literal $\neq.$ 
%
%
% \DescribeMacro{\Nu} Short hand notation for literal $\mathrm{N}.$ 
%
%
% \DescribeMacro{\O} Short hand notation for literal $\emptyset.$ 
%
%
% \DescribeMacro{\omicron} Short hand notation for literal $\mathrm{o}.$ 
%
%
% \DescribeMacro{\Omicron} Short hand notation for literal $\mathrm{O}.$ 
%
%
% \DescribeMacro{\or} Short hand notation for literal $\lor.$ 
%
%
% \DescribeMacro{\part} Short hand notation for literal $\partial.$ 
%
%
% \DescribeMacro{\plusmn} Short hand notation for literal $\pm.$ 
%
%
% \DescribeMacro{\rarr} Short hand notation for literal $\rightarrow.$ 
%
%
% \DescribeMacro{\Rarr} Short hand notation for literal $\Rightarrow.$ 
%
%
% \DescribeMacro{\rArr} Short hand notation for literal $\Rightarrow.$ 
%
%
% \DescribeMacro{\real} Short hand notation for literal $\Re.$ 
%
%
% \DescribeMacro{\reals} Short hand notation for literal $\mathbb{R}.$ 
%
%
% \DescribeMacro{\Reals} Short hand notation for literal $\mathbb{R}.$ 
%
%
% \DescribeMacro{\restriction} Short hand notation for literal $\upharpoonright.$ 
%
%
% \DescribeMacro{\Rho} Short hand notation for literal $\mathrm{P}.$ 
%
%
% \DescribeMacro{\sdot} Short hand notation for literal $\cdot.$ 
%
%
% \DescribeMacro{\sect} Short hand notation for literal $\S.$ 
%
%
% \DescribeMacro{\spades} Short hand notation for literal $\spadesuit.$ 
%
%
% \DescribeMacro{\sub} Short hand notation for literal $\subset.$ 
%
%
% \DescribeMacro{\sube} Short hand notation for literal $\subseteq.$ 
%
%
% \DescribeMacro{\supe} Short hand notation for literal $\supseteq.$ 
%
%
% \DescribeMacro{\Tau} Short hand notation for literal $\mathrm{T}.$ 
%
%
% \DescribeMacro{\thetasym} Short hand notation for literal $\vartheta.$ 
%
%
% \DescribeMacro{\varcoppa} Short hand notation for literal $\mbox{\\coppa}.$ 
%
%
% \DescribeMacro{\weierp} Short hand notation for literal $\wp.$ 
%
%
% \DescribeMacro{\Zeta} Short hand notation for literal $\mathrm{Z}.$ 
%
%
% \StopEventually{}
%
% \section{Implementation}
% \begin{macro}{\darr}
% This macro does the following replacement.
%    \begin{macrocode}
\newcommand{\darr}{\downarrow}
%    \end{macrocode}
% \end{macro}
%
%
% \begin{macro}{\dArr}
% This macro does the following replacement.
%    \begin{macrocode}
\newcommand{\dArr}{\Downarrow}
%    \end{macrocode}
% \end{macro}
%
%
% \begin{macro}{\Darr}
% This macro does the following replacement.
%    \begin{macrocode}
\newcommand{\Darr}{\Downarrow}
%    \end{macrocode}
% \end{macro}
%
%
% \begin{macro}{\lang}
% This macro does the following replacement.
%    \begin{macrocode}
\newcommand{\lang}{\langle}
%    \end{macrocode}
% \end{macro}
%
%
% \begin{macro}{\rang}
% This macro does the following replacement.
%    \begin{macrocode}
\newcommand{\rang}{\rangle}
%    \end{macrocode}
% \end{macro}
%
%
% \begin{macro}{\uarr}
% This macro does the following replacement.
%    \begin{macrocode}
\newcommand{\uarr}{\uparrow}
%    \end{macrocode}
% \end{macro}
%
%
% \begin{macro}{\uArr}
% This macro does the following replacement.
%    \begin{macrocode}
\newcommand{\uArr}{\Uparrow}
%    \end{macrocode}
% \end{macro}
%
%
% \begin{macro}{\Uarr}
% This macro does the following replacement.
%    \begin{macrocode}
\newcommand{\Uarr}{\Uparrow}
%    \end{macrocode}
% \end{macro}
%
%
% \begin{macro}{\C}
% This macro does the following replacement.
%    \begin{macrocode}
\newcommand{\C}{\mathbb{C}}
%    \end{macrocode}
% \end{macro}
%
%
% \begin{macro}{\H}
% This macro does the following replacement.
%    \begin{macrocode}
\renewcommand{\H}{\mathbb{H}}
%    \end{macrocode}
% \end{macro}
%
%
% \begin{macro}{\N}
% This macro does the following replacement.
%    \begin{macrocode}
\newcommand{\N}{\mathbb{N}}
%    \end{macrocode}
% \end{macro}
%
%
% \begin{macro}{\Q}
% This macro does the following replacement.
%    \begin{macrocode}
\newcommand{\Q}{\mathbb{Q}}
%    \end{macrocode}
% \end{macro}
%
%
% \begin{macro}{\R}
% This macro does the following replacement.
%    \begin{macrocode}
\newcommand{\R}{\mathbb{R}}
%    \end{macrocode}
% \end{macro}
%
%
% \begin{macro}{\Z}
% This macro does the following replacement.
%    \begin{macrocode}
\newcommand{\Z}{\mathbb{Z}}
%    \end{macrocode}
% \end{macro}
%
%
% \begin{macro}{\alef}
% This macro does the following replacement.
%    \begin{macrocode}
\newcommand{\alef}{\aleph}
%    \end{macrocode}
% \end{macro}
%
%
% \begin{macro}{\alefsym}
% This macro does the following replacement.
%    \begin{macrocode}
\newcommand{\alefsym}{\aleph}
%    \end{macrocode}
% \end{macro}
%
%
% \begin{macro}{\Alpha}
% This macro does the following replacement.
%    \begin{macrocode}
\newcommand{\Alpha}{\mathrm{A}}
%    \end{macrocode}
% \end{macro}
%
%
% \begin{macro}{\and}
% This macro does the following replacement.
%    \begin{macrocode}
\renewcommand{\and}{\land}
%    \end{macrocode}
% \end{macro}
%
%
% \begin{macro}{\ang}
% This macro does the following replacement.
%    \begin{macrocode}
\newcommand{\ang}{\angle}
%    \end{macrocode}
% \end{macro}
%
%
% \begin{macro}{\Beta}
% This macro does the following replacement.
%    \begin{macrocode}
\newcommand{\Beta}{\mathrm{B}}
%    \end{macrocode}
% \end{macro}
%
%
% \begin{macro}{\bull}
% This macro does the following replacement.
%    \begin{macrocode}
\newcommand{\bull}{\bullet}
%    \end{macrocode}
% \end{macro}
%
%
% \begin{macro}{\Chi}
% This macro does the following replacement.
%    \begin{macrocode}
\newcommand{\Chi}{\mathrm{X}}
%    \end{macrocode}
% \end{macro}
%
%
% \begin{macro}{\clubs}
% This macro does the following replacement.
%    \begin{macrocode}
\newcommand{\clubs}{\clubsuit}
%    \end{macrocode}
% \end{macro}
%
%
% \begin{macro}{\cnums}
% This macro does the following replacement.
%    \begin{macrocode}
\newcommand{\cnums}{\mathbb{C}}
%    \end{macrocode}
% \end{macro}
%
%
% \begin{macro}{\Complex}
% This macro does the following replacement.
%    \begin{macrocode}
\newcommand{\Complex}{\mathbb{C}}
%    \end{macrocode}
% \end{macro}
%
%
% \begin{macro}{\Dagger}
% This macro does the following replacement.
%    \begin{macrocode}
\newcommand{\Dagger}{\ddagger}
%    \end{macrocode}
% \end{macro}
%
%
% \begin{macro}{\diamonds}
% This macro does the following replacement.
%    \begin{macrocode}
\newcommand{\diamonds}{\diamondsuit}
%    \end{macrocode}
% \end{macro}
%
%
% \begin{macro}{\Doteq}
% This macro does the following replacement.
%    \begin{macrocode}
\renewcommand{\Doteq}{\doteqdot}
%    \end{macrocode}
% \end{macro}
%
%
% \begin{macro}{\doublecap}
% This macro does the following replacement.
%    \begin{macrocode}
\renewcommand{\doublecap}{\Cap}
%    \end{macrocode}
% \end{macro}
%
%
% \begin{macro}{\doublecup}
% This macro does the following replacement.
%    \begin{macrocode}
\renewcommand{\doublecup}{\Cup}
%    \end{macrocode}
% \end{macro}
%
%
% \begin{macro}{\empty}
% This macro does the following replacement.
%    \begin{macrocode}
\renewcommand{\empty}{\emptyset}
%    \end{macrocode}
% \end{macro}
%
%
% \begin{macro}{\Epsilon}
% This macro does the following replacement.
%    \begin{macrocode}
\newcommand{\Epsilon}{\mathrm{E}}
%    \end{macrocode}
% \end{macro}
%
%
% \begin{macro}{\Eta}
% This macro does the following replacement.
%    \begin{macrocode}
\newcommand{\Eta}{\mathrm{H}}
%    \end{macrocode}
% \end{macro}
%
%
% \begin{macro}{\exist}
% This macro does the following replacement.
%    \begin{macrocode}
\newcommand{\exist}{\exists}
%    \end{macrocode}
% \end{macro}
%
%
% \begin{macro}{\ge}
% This macro does the following replacement.
%    \begin{macrocode}
\renewcommand{\ge}{\geq}
%    \end{macrocode}
% \end{macro}
%
%
% \begin{macro}{\gggtr}
% This macro does the following replacement.
%    \begin{macrocode}
\renewcommand{\gggtr}{\ggg}
%    \end{macrocode}
% \end{macro}
%
%
% \begin{macro}{\hAar}
% This macro does the following replacement.
%    \begin{macrocode}
\newcommand{\hAar}{\Leftrightarrow}
%    \end{macrocode}
% \end{macro}
%
%
% \begin{macro}{\harr}
% This macro does the following replacement.
%    \begin{macrocode}
\newcommand{\harr}{\leftrightarrow}
%    \end{macrocode}
% \end{macro}
%
%
% \begin{macro}{\Harr}
% This macro does the following replacement.
%    \begin{macrocode}
\newcommand{\Harr}{\Leftrightarrow}
%    \end{macrocode}
% \end{macro}
%
%
% \begin{macro}{\hearts}
% This macro does the following replacement.
%    \begin{macrocode}
\newcommand{\hearts}{\heartsuit}
%    \end{macrocode}
% \end{macro}
%
%
% \begin{macro}{\image}
% This macro does the following replacement.
%    \begin{macrocode}
\newcommand{\image}{\Im}
%    \end{macrocode}
% \end{macro}
%
%
% \begin{macro}{\infin}
% This macro does the following replacement.
%    \begin{macrocode}
\newcommand{\infin}{\infty}
%    \end{macrocode}
% \end{macro}
%
%
%
% \begin{macro}{\Iota}
% This macro does the following replacement.
%    \begin{macrocode}
\newcommand{\Iota}{\mathrm{I}}
%    \end{macrocode}
% \end{macro}
%
%
% \begin{macro}{\isin}
% This macro does the following replacement.
%    \begin{macrocode}
\newcommand{\isin}{\in}
%    \end{macrocode}
% \end{macro}
%
%
% \begin{macro}{\Kappa}
% This macro does the following replacement.
%    \begin{macrocode}
\newcommand{\Kappa}{\mathrm{K}}
%    \end{macrocode}
% \end{macro}
%
%
% \begin{macro}{\larr}
% This macro does the following replacement.
%    \begin{macrocode}
\newcommand{\larr}{\leftarrow}
%    \end{macrocode}
% \end{macro}
%
%
% \begin{macro}{\Larr}
% This macro does the following replacement.
%    \begin{macrocode}
\newcommand{\Larr}{\Leftarrow}
%    \end{macrocode}
% \end{macro}
%
%
% \begin{macro}{\lArr}
% This macro does the following replacement.
%    \begin{macrocode}
\newcommand{\lArr}{\Leftarrow}
%    \end{macrocode}
% \end{macro}
%
%
% \begin{macro}{\le}
% This macro does the following replacement.
%    \begin{macrocode}
\renewcommand{\le}{\leq}
%    \end{macrocode}
% \end{macro}
%
%
% \begin{macro}{\lrarr}
% This macro does the following replacement.
%    \begin{macrocode}
\newcommand{\lrarr}{\leftrightarrow}
%    \end{macrocode}
% \end{macro}
%
%
% \begin{macro}{\Lrarr}
% This macro does the following replacement.
%    \begin{macrocode}
\newcommand{\Lrarr}{\Leftrightarrow}
%    \end{macrocode}
% \end{macro}
%
%
% \begin{macro}{\lrArr}
% This macro does the following replacement.
%    \begin{macrocode}
\newcommand{\lrArr}{\Leftrightarrow}
%    \end{macrocode}
% \end{macro}
%
%
%
% \begin{macro}{\Mu}
% This macro does the following replacement.
%    \begin{macrocode}
\newcommand{\Mu}{\mathrm{M}}
%    \end{macrocode}
% \end{macro}
%
%
% \begin{macro}{\natnums}
% This macro does the following replacement.
%    \begin{macrocode}
\newcommand{\natnums}{\mathbb{N}}
%    \end{macrocode}
% \end{macro}
%
%
% \begin{macro}{\ne}
% This macro does the following replacement.
%    \begin{macrocode}
\renewcommand{\ne}{\neq}
%    \end{macrocode}
% \end{macro}
%
%
% \begin{macro}{\Nu}
% This macro does the following replacement.
%    \begin{macrocode}
\newcommand{\Nu}{\mathrm{N}}
%    \end{macrocode}
% \end{macro}
%
%
% \begin{macro}{\O}
% This macro does the following replacement.
%    \begin{macrocode}
\renewcommand{\O}{\emptyset}
%    \end{macrocode}
% \end{macro}
%
%
% \begin{macro}{\omicron}
% This macro does the following replacement.
%    \begin{macrocode}
\newcommand{\omicron}{\mathrm{o}}
%    \end{macrocode}
% \end{macro}
%
%
% \begin{macro}{\Omicron}
% This macro does the following replacement.
%    \begin{macrocode}
\newcommand{\Omicron}{\mathrm{O}}
%    \end{macrocode}
% \end{macro}
%
%
%
% \begin{macro}{\orMediaWiki}
% This macro does the following replacement.
%    \begin{macrocode}
\newcommand{\orMediaWiki}{\lor}
%    \end{macrocode}
% \end{macro}
%
%
% \begin{macro}{\part}
% This macro does the following replacement.
%    \begin{macrocode}
\renewcommand{\part}{\partial}
%    \end{macrocode}
% \end{macro}
%
%
% \begin{macro}{\plusmn}
% This macro does the following replacement.
%    \begin{macrocode}
\newcommand{\plusmn}{\pm}
%    \end{macrocode}
% \end{macro}
%
%
% \begin{macro}{\rarr}
% This macro does the following replacement.
%    \begin{macrocode}
\newcommand{\rarr}{\rightarrow}
%    \end{macrocode}
% \end{macro}
%
%
% \begin{macro}{\Rarr}
% This macro does the following replacement.
%    \begin{macrocode}
\newcommand{\Rarr}{\Rightarrow}
%    \end{macrocode}
% \end{macro}
%
%
% \begin{macro}{\rArr}
% This macro does the following replacement.
%    \begin{macrocode}
\newcommand{\rArr}{\Rightarrow}
%    \end{macrocode}
% \end{macro}
%
%
% \begin{macro}{\real}
% This macro does the following replacement.
%    \begin{macrocode}
\newcommand{\real}{\Re}
%    \end{macrocode}
% \end{macro}
%
%
% \begin{macro}{\reals}
% This macro does the following replacement.
%    \begin{macrocode}
\newcommand{\reals}{\mathbb{R}}
%    \end{macrocode}
% \end{macro}
%
%
% \begin{macro}{\Reals}
% This macro does the following replacement.
%    \begin{macrocode}
\newcommand{\Reals}{\mathbb{R}}
%    \end{macrocode}
% \end{macro}
%
%
% \begin{macro}{\restriction}
% This macro does the following replacement.
%    \begin{macrocode}
\renewcommand{\restriction}{\upharpoonright}
%    \end{macrocode}
% \end{macro}
%
%
% \begin{macro}{\Rho}
% This macro does the following replacement.
%    \begin{macrocode}
\newcommand{\Rho}{\mathrm{P}}
%    \end{macrocode}
% \end{macro}
%
%
% \begin{macro}{\sdot}
% This macro does the following replacement.
%    \begin{macrocode}
\newcommand{\sdot}{\cdot}
%    \end{macrocode}
% \end{macro}
%
%
% \begin{macro}{\sect}
% This macro does the following replacement.
%    \begin{macrocode}
\newcommand{\sect}{\S}
%    \end{macrocode}
% \end{macro}
%
%
% \begin{macro}{\spades}
% This macro does the following replacement.
%    \begin{macrocode}
\newcommand{\spades}{\spadesuit}
%    \end{macrocode}
% \end{macro}
%
%
% \begin{macro}{\sub}
% This macro does the following replacement.
%    \begin{macrocode}
\newcommand{\sub}{\subset}
%    \end{macrocode}
% \end{macro}
%
%
% \begin{macro}{\sube}
% This macro does the following replacement.
%    \begin{macrocode}
\newcommand{\sube}{\subseteq}
%    \end{macrocode}
% \end{macro}
%
%
% \begin{macro}{\supe}
% This macro does the following replacement.
%    \begin{macrocode}
\newcommand{\supe}{\supseteq}
%    \end{macrocode}
% \end{macro}
%
%
% \begin{macro}{\Tau}
% This macro does the following replacement.
%    \begin{macrocode}
\newcommand{\Tau}{\mathrm{T}}
%    \end{macrocode}
% \end{macro}
%
%
% \begin{macro}{\thetasym}
% This macro does the following replacement.
%    \begin{macrocode}
\newcommand{\thetasym}{\vartheta}
%    \end{macrocode}
% \end{macro}
%
%
% \begin{macro}{\varcoppa}
% This macro does the following replacement.
%    \begin{macrocode}
\newcommand{\varcoppa}{\mbox{\\coppa}}
%    \end{macrocode}
% \end{macro}
%
%
% \begin{macro}{\weierp}
% This macro does the following replacement.
%    \begin{macrocode}
\newcommand{\weierp}{\wp}
%    \end{macrocode}
% \end{macro}
%
%
% \begin{macro}{\Zeta}
% This macro does the following replacement.
%    \begin{macrocode}
\newcommand{\Zeta}{\mathrm{Z}}
%    \end{macrocode}
% \end{macro}
%
%
% \Finale
\endinput