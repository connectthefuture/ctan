\documentclass{article}
%
% the driver line is not necessary if you
% have aebpro.cfg configured to your driver.
%
\usepackage[%
%    driver=dvips,
    web={pro,tight,usesf},
    eforms,graphicxsp={showembeds}
]{aeb_pro}
\usepackage{rmannot}

%\AcroVer{11}

\margins{.25in}{.25in}{24pt}{.25in} % left,right,top, bottom
\screensize{5in}{5.5in}             % height, width

\DeclareDocInfo
{
    title=The \texttt{rmannot} Package\texorpdfstring{\\[1ex]}{: }FLV Movies,
    author=D. P. Story,
    university=Acro\negthinspace\TeX.Net,
    email=dpstory@acrotex.net,
    subject={Demo of the rmannot package, SWF movies},
    keywords={Adobe Acrobat, JavaScript, ActionScript},
    talksite=\url{http://www.acrotex.net},
    talkdate={July 2008},
    copyrightStatus=True,
    copyrightNotice={Copyright (C) 2008--\the\year, D. P. Story},
    copyrightInfoURL=http://www.acrotex.net
}
\talkdateLabel{Published:}

\def\AcroTeX{Acro\!\TeX}

% Place \AcroVer{11} in rmannot.cfg, or uncomment line below
%\AcroVer{11}
% The argument corresponds to the version of Acrobat you have
%
% You can use VPX with this file, the recommended minimal document level code is below
% uncomment the next line to use VPX, assuming you've installed it.
%\useVideoPlayerX
\ifVideoPlayerEx
\begin{insDLJS}{playerJS}{JS for VPX Player}
function vpx_init(r,pg) {
    var rm=this.getAnnotRichMedia(pg,r);
//    var rtn=rm.callAS(\mmSetScaleMode,"maintainAspectRatio");
}
\end{insDLJS}
\fi
%
% Convenience command pointing to the rich media files, this needs
% to be edited to point to its location on your system.
%
\defineRMPath{\myRMFiles}{%
    C:/Users/Public/Documents/My TeX Files/%
    tex/latex/aeb/aebpro/rmannot/RMfiles}
\saveNamedPath{horse1}{\myRMFiles/horse1.flv}
\makePoster[hiresbb]{aebmovie_poster}{aebmovie_poster}
\makePoster[hiresbb]{horse1_poster}{horse1_poster}

\parindent=0pt\parskip6pt\pagestyle{empty}

\begin{document}

\maketitle

\begin{center}
\resizebox{.75\linewidth}{!}{%
    \rmAnnot[poster=aebmovie_poster]{320bp}{240bp}{horse1}%
}%
\end{center}

Video using the default skin.\footnote{Video downloaded from youtube.com.}

\newpage

\begin{center}
\resizebox{.75\linewidth}{!}
    {\rmAnnot[poster=horse1_poster,skin=skin3]{320bp}{240bp}{horse1}}%
\end{center}

Video using \texttt{skin3} and poster style that Acrobat uses.

Here is the same video in a floating window, with \texttt{skin2}.
\raisebox{-2pt}{\resizebox{.25in}{!}
    {\rmAnnot[poster=horse1_poster,windowed,skin=skin2]{320bp}{240bp}{horse1}}}

\newpage

This demo file size is 507KB. The size of the \texttt{horse1.flv}
movie is 391KB, the size of \texttt{movie\_poster} is 192KB, and the
size of \texttt{horse1\_poster} is 202KB; the sum of these three
file sizes is 785KB. Acrobat does do some file compression, so the final
file size is smaller than the sum of the movie and graphic files
that it contains. But wait, the FLV file was used three times in
three different rich media annotations, and \texttt{horse1\_poster}
was used twice.  If the rich media annotations were created by
Acrobat's user interface, Acrobat would have embedded the FLV file
three times and embedded the \texttt{horse1\_poster} twice, swelling
the file size to 1.49MB! The \texttt{rmannot} package, however, only
embeds each of these files once, and uses and re-uses them as
needed.



\end{document}
