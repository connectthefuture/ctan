%
% Requires AeB Pro, Acrobat 7.0 or later and distiller
%
% We choose the pro option, this brings in the use of layers---the questions are hidden
% even if the user tries to peek. The forcredit option forces the user to enter something
% in the name section before being allowed to continue.
%
% Try this file with the various design options: jeopardy,florida,iceland,hornet,qatar,
% norway,germany,bahamas,spain
%
\documentclass[pro,forcredit,design=iceland]{jj_game} % or dvips, dvipsone
\usepackage{amsmath}
\usepackage{graphicx}
\usepackage[%
    dljslib={ImplMulti},
    exerquiz,
    uselayers
]{aeb_pro}

\author{D. P. Story}
\university{Northwest Florida State College}

%
% Require version 7.0 or later (this is an AeB Pro command)
%
\requiresVersion{7}
%
% include a standard footer at the bottom of the first page.
%
\includeFootBanner

\titleBanner{Function Jeopardy!}
\afterGameBoardInsertion{\medskip\gameboardPrintButton}

\GameDesign
{
    Cat: General Functions,
    Cat: Quadratic Functions,
    Cat: Polynomial Functions,
    Cat: Rational Functions,
    NumQuestions: 3,
%    Goal: 1,500,              % specify absolute goal
    GoalPercentage: 85,        % specify relative goal
    ExtraHeight: .7in,
    Champion: You are FuncTerrific!
}

\APScore{align: c}

\begin{document}


\begin{instructions}
%
% Insert the Instruction page here
%
\textcolor{red}{\textbf{Extra Credit:}} Before you begin, enter your
name in the text field below. After you have finished with
\textsf{Function Jeopardy!}, print the next page (the game board page) and
turn it in for extra credit.

\textcolor{red}{\textbf{Name:}} \underbar{\contestantName{1.5in}{11bp}}

\textcolor{blue}{\textbf{Method of Scoring.}} If you answer a
question correctly, the dollar value of that question is added to
your total.  If you miss a question, the dollar value is
\textit{subtracted} from your total.  So think carefully before
you answer!

\textcolor{blue}{\textbf{Instructions.}} Solve the problems in any
order you wish. If your total at the end is more than \$\Goal, you
will be declared  \textbf{FuncTerrific}, a master of functions of
college algebra!

\textcolor{blue}{\textbf{To Begin:}} Go to the next page.

\end{instructions}


\everymath{\displaystyle}

\begin{Questions}

\begin{Category}{General Functions}

\begin{Question}
Given credit for first using the functional notation $f(x)$.
\begin{oAnswer}
Who is\dots\space\RespBoxTxt{2}{1}{3}{Leonhard Euler}{L. Euler}{Euler}
\end{oAnswer}
\end{Question}

\begin{Question}
Given $ f(x) = \frac{x}{x+2} $, the expression that represents $ f(1/x) $. What is \dots
\begin{oAnswer}
\begin{equation*}
f(2x)=\RespBoxMath{1/(2*x+1)}{4}{.0001}{[1,2]}
\end{equation*}
\end{oAnswer}
\end{Question}

\begin{Question}[2]
The axis of symmetry of the graph of the function $ f(x) = 2 - ( x + 1 )^2 $. What is \dots


\Ans0 the $x$-axis &
\Ans0 the $y$-axis \\[1ex]
\Ans1 the line $ x = -1 $ &
\Ans0 the line $ x = 1 $ \\[1ex]
\Ans0 the line $ y = 2 $ &
\Ans0 the line $ y = -2 $

\end{Question}

\end{Category}

\begin{Category}{Quadratic Functions}

\begin{Question}
The number of zeros of the quadratic function
$$ f(x) = x^2 - 2x + 2 $$
What is \dots

\Ans1 $0$
\Ans0 $1$
\Ans0 $2$
\Ans0 $3$
\end{Question}

\begin{Question}
The vertex $V$ of the parabola $ f(x) = 3 - 4x - 4x^2 $. What is \dots

\Ans0 $V(1/4, 7/4)$
\Ans0 $V(-1/4, 15/4)$
\Ans0 $V(1/2, 0)$
\Ans1 $V(-1/2,4)$
\Ans0 $V(3/4, -9/4)$
\Ans0 $V(-1/2, 15/4)$
\Ans0 None of these
\end{Question}

\begin{Question}[4]
The price $p$ and the quantity $x$ sold of a certain product obey the
demand equation
\begin{equation*}
  p = -\frac{1}{6}x + 100
\end{equation*}
Find the quantity $x$ that maximizes revenue.

\Ans0 $100$ &
\Ans0 $200$ &
\Ans1 $300$ &
\Ans0 $400$ \\[3ex]
\Ans0 $500$ &
\Ans0 $600$ &
\Ans0 $700$ &
\Ans0 $800$ \\[3ex]
\Ans0 $900$  &
\Ans0 $1000$ &
\Ans0 $1100$ &
\Ans0 $1200$
\end{Question}

\end{Category}


\begin{Category}{Polynomial Functions}

\begin{Question}[4]
The \textbf{end behavior} of the polynomial function
\begin{equation*}
  f(x) = (2x-1)^2 ( x + 3 )^2 ( 3x^3 + 1 )^2
\end{equation*}
is like that of what function? What is \dots

\Ans0 $y = x$       &
\Ans0 $y = x^{2}$   &
\Ans0 $y = x^{3}$   &
\Ans0 $y = x^{4}$   \\[3ex]
\Ans0 $y = x^{5}$   &
\Ans0 $y = x^{6}$   &
\Ans0 $y = x^{7}$   &
\Ans0 $y = x^{8}$   \\[3ex]
\Ans0 $y = x^{9}$   &
\Ans1 $y = x^{10}$  &
\Ans0 $y = x^{11}$  &
\Ans0 $y = x^{12}$
\end{Question}

\begin{Question}
The multiplicity of the zero $ x = 1/2 $ of the polynomial function
$ f(x) = x^2 (x - 2 ) (2x - 1 )^3$. What is \dots

\Ans0 $1$
\Ans0 $2$
\Ans1 $3$
\Ans0 Don't fool with me, $1/2$ is not a zero of this polynomial!
\Ans0 None of these
\end{Question}

\begin{Question}
The number of times the function
$$
    y = -(x^2 + 0.5)(x-1)^2(x+1)(x-2)
$$
touches but \textit{does not cross} the $x$-axis. What is \dots

\Ans0 $0$ times
\Ans1 $1$ time
\Ans0 $2$ times
\Ans0 $3$ times
\Ans0 $4$ times
\end{Question}

\end{Category}

\begin{Category}{Rational Functions}

\begin{Question}
For a rational function, when the degree of the numerator is greater than the
degree of the denominator, then the $x$-axis is a horizontal asymptote. True or False?

\Ans0 True
\Ans1 False
\end{Question}

\begin{Question}
The asymptotes for the rational function
\begin{equation*}
  R(x) = \frac{3x^2 -1}{(3x-1)(2x+2)}
\end{equation*}
What are \dots

\Ans0 $ y = 1 $, $ x =   -2 $, $ x = 3$
\Ans0 $ y = 1/6 $, $ x = -2 $, $ x = 1/3$
\Ans1 $ y = 1/2 $, $ x = -1 $, $ x = 1/3$
\Ans0 $ y = 1/2 $, $ x = -2 $, $ x = 3$
\Ans0 $ y = 1   $, $ x =  1 $, $ x = 1/3$
\Ans0 $ y = 1/6 $, $ x = -1 $, $ x = 1/3$
\Ans0 None of these
\end{Question}

\begin{Question}
The oblique asymptote of the rational function
\begin{equation*}
  R(x) = \frac{4x^4 - 6x^3 + 5x^2 + x + 4}{2x^3 + 3x}
\end{equation*}
What is \dots

\Ans0 $y = 4$
\Ans0 $y = 2x + 4$
\Ans1 $y = 2x-3$
\Ans0 $y = 4x - 3$
\Ans0 $y = 4x + 4$
\Ans0 $ y = 2x + 3$
\Ans0 $ y = 2x - 4$
\Ans0 None of these
\end{Question}
\end{Category}
\end{Questions}
\end{document}
