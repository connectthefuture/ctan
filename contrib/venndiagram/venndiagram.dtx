%\iffalse
% venndiagram.dtx generated using makedtx version 1.1 (c) Nicola Talbot
% Command line args:
%   -author "Nicola Talbot"
%   -src "venndiagram.sty\Z=>venndiagram.sty"
%   -doc "venndiagram-manual.tex"
%   venndiagram
% Created on 2016/3/16 22:06
%\fi
%\iffalse
%<*package>
%% \CharacterTable
%%  {Upper-case    \A\B\C\D\E\F\G\H\I\J\K\L\M\N\O\P\Q\R\S\T\U\V\W\X\Y\Z
%%   Lower-case    \a\b\c\d\e\f\g\h\i\j\k\l\m\n\o\p\q\r\s\t\u\v\w\x\y\z
%%   Digits        \0\1\2\3\4\5\6\7\8\9
%%   Exclamation   \!     Double quote  \"     Hash (number) \#
%%   Dollar        \$     Percent       \%     Ampersand     \&
%%   Acute accent  \'     Left paren    \(     Right paren   \)
%%   Asterisk      \*     Plus          \+     Comma         \,
%%   Minus         \-     Point         \.     Solidus       \/
%%   Colon         \:     Semicolon     \;     Less than     \<
%%   Equals        \=     Greater than  \>     Question mark \?
%%   Commercial at \@     Left bracket  \[     Backslash     \\
%%   Right bracket \]     Circumflex    \^     Underscore    \_
%%   Grave accent  \`     Left brace    \{     Vertical bar  \|
%%   Right brace   \}     Tilde         \~}
%</package>
%\fi
% \iffalse
% Doc-Source file to use with LaTeX2e
% Copyright (C) 2016 Nicola Talbot, all rights reserved.
% \fi
% \iffalse
%<*driver>
\documentclass[widecs]{nlctdoc}

\usepackage{amsmath}
\usepackage{venndiagram}
\usepackage{alltt}
\usepackage[utf8]{inputenc}
\usepackage[T1]{fontenc}
\usepackage[colorlinks,
            bookmarks,
            hyperindex=false,
            pdfauthor={Nicola L.C. Talbot},
            pdftitle={venndiagram.sty: LaTeX2e Package for Drawing Venn Diagrams},
            pdfkeywords={LaTeX,package,venn diagram}]{hyperref}

\DisableCrossrefs

\setlength\marginparwidth{6pc}

\doxitem{KeyValOption}{keyvaloption}{options}

\CheckSum{2598}

\begin{document}
\DocInput{venndiagram.dtx}
\end{document}
%</driver>
%\fi
%
%\MakeShortVerb{"}
%
%\title{venndiagram v1.1:
%Drawing Simple Venn Diagrams}
%\author{Nicola L. C. Talbot\\\url{http://www.dickimaw-books.com/}}
%
%\date{2016-03-16}
%\maketitle
%
%The \styfmt{venndiagram} package is provided to assist generating
%simple two- and three-set Venn diagrams for lectures or assignment
%sheets. This package requires the \sty{tikz} package.
%As from v1.1, this package no longer requires the
%\styfmt{intersections} library.
%
%\begin{important}
%The aim of this package is to provide very simple Venn diagrams for
%assignments or exam questions. If you require more complex diagrams
%or different layouts it's simpler to directly use the \sty{tikz} package.
%\end{important}
%
%\tableofcontents
%
%\section{Available Commands and Environments}
%
%This package defines two environments:
%\begin{definition}[\DescribeEnv{venndiagram3sets}]
%\cs{begin}\{venndiagram3sets\}\oarg{options}
%\end{definition}
%(for two sets) and
%\begin{definition}[\DescribeEnv{venndiagram2sets}]
%\cs{begin}\{venndiagram2sets\}\oarg{options}
%\end{definition}
%(for three sets).
%
%The optional argument \meta{options} is a comma-separated list of
%\meta{key}=\meta{value} settings.
%
%\begin{important}
%If the \meta{value} contains commas or equal
%signs, make sure you enclose the entire value in braces.
%For example:
%\begin{verbatim}
%\begin{venndiagram3sets}[tikzoptions={scale=2,thick}]
%\end{verbatim}
%\end{important}
%
%The following keys are available:
%\begin{description}
%\item[shade] The name of the colour used to shade regions (default:
%\texttt{lightgray}).
%
%\item[labelA] The label for the first set (default: \verb|$A$|).
%
%\item[labelB] The label for the second set (default: \verb|$B$|).
%
%\item[labelC] (Not available for the 2 set version.) The label for the third set (default: \verb|$C$|).
%
%\item[labelOnlyA] The label for the region given by $A\setminus(B \cup C)$ (for 3 set version)
%or $A\setminus B$ (for 2 set version). (Default: empty.)
%
%\item[labelOnlyB] The label for the region given by $B\setminus(A \cup C)$ (for 3 set version)
%or $B\setminus A$ (for 2 set version). (Default: empty.)
%
%\item[labelOnlyC] (Not available for 2 set version.) The label for the region given by 
%$C\setminus(A \cup B)$. (Default: empty.)
%
%\item[labelOnlyAB] (Not available for 2 set version.) The label for the region given by 
%$(A \cap B) \setminus C$. (Default: empty.)
%
%\item[labelOnlyAC] (Not available for 2 set version.) The label for the region given by 
%$(A \cap C)\setminus B$. (Default: empty.)
%
%\item[labelOnlyBC] (Not available for 2 set version.) The label for the region given by 
%$(B \cap C)\setminus A$. (Default: empty.)
%
%\item[labelABC] (Not available for 2 set version.) The label for the region given by 
%$A \cap B \cap C$. (Default: empty.)
%
%\item[labelNotABC] (Not available for 2 set version.) The label for the region given by 
%$(A \cup B \cup C)^c$. (Default: empty.)
%
%\item[labelAB] (Not available for 3 set version.) The label for the region given by 
%$A \cap B$. (Default: empty.)
%
%\item[labelNotAB] (Not available for 3 set version.) The label for the region given by 
%$(A \cup B)^c$. (Default: empty.)
%
%\item[radius] The radius of each set. (Default: 1.2cm.)
%
%\item[hgap] The horizontal gap between the outer vertical edge and the
%nearest set edge. (Default: 0.5cm.)
%
%\item[vgap] The vertical gap between the outer horizontal edge and
%the nearest set edge. (Default: 0.5cm.)
%
%\item[overlap] The overlap between the sets. (Default: 0.75cm.)
%
%\item[tikzoptions] Any options to pass to \env{tikzpicture}.
%
%\end{description}
%
%Both environments draw the outline of the sets and the rectangular
%outline of the encompassing universal set. Within the Venn diagram environments
%commands are provided to shade various regions. (The commands have a
%cumulative effect, possibly drawing over each other. The set
%outlines and labels are drawn at the end of the environment.)
%Available commands are as follows:
%
%\begin{definition}[\DescribeMacro{\fillA}]
%\cs{fillA}
%\end{definition}
%Shades set $A$.
%\begin{definition}[\DescribeMacro{\fillB}]
%\cs{fillB}
%\end{definition}
%Shades set $B$.
%\begin{definition}[\DescribeMacro{\fillC}]
%\cs{fillC}
%\end{definition}
%(Only for 3 set version.) Shades set $C$.
%
%\begin{definition}[\DescribeMacro{\fillAll}]
%\cs{fillAll}
%\end{definition}
%Shades the entire Venn diagram.
%
%\begin{definition}[\DescribeMacro{\fillNotABC}]
%\cs{fillNotABC}
%\end{definition}
%(Not available for 2 sets version.) Fills $(A\cup B \cup C)^c$.
%
%\begin{definition}[\DescribeMacro{\fillOnlyA}]
%\cs{fillOnlyA}
%\end{definition}
%Shades set $A \setminus (B \cup C)$ (for 3 sets version)
%or $A \setminus B$ (for 2 sets version).
%\begin{definition}[\DescribeMacro{\fillOnlyB}]
%\cs{fillOnlyB}
%\end{definition}
%Shades set $B \setminus (A \cup C)$ (for 3 sets version)
%or $B \setminus A$ (for 2 sets version).
%\begin{definition}[\DescribeMacro{\fillOnlyC}]
%\cs{fillOnlyC}
%\end{definition}
%(Not available for 2 sets version.) Shades $C \setminus (A \cup B)$.
%
%\begin{definition}[\DescribeMacro{\fillNotA}]
%\cs{fillNotA}
%\end{definition}
%Shades everything except $A$ (that is $A^c$).
%\begin{definition}[\DescribeMacro{\fillNotB}]
%\cs{fillNotB}
%\end{definition}
%Shades everything except $B$ (that is $B^c$).
%\begin{definition}[\DescribeMacro{\fillNotC}]
%\cs{fillNotC}
%\end{definition}
%(Not available for 2 set version.) Shades everything except $C$ (that is $C^c$).
%
%\begin{definition}[\DescribeMacro{\fillNotAorB}]
%\cs{fillNotAorB}
%\end{definition}
%(Not available for 3 set version.) Shades $(A\cup B)^c$
%\begin{definition}[\DescribeMacro{\fillNotAorNotB}]
%\cs{fillNotAorNotB}
%\end{definition}
%(Not available for 3 set version.) Shades $(A\cap B)^c$
%
%
%\begin{definition}[\DescribeMacro{\fillANotB}]
%\cs{fillANotB}
%\end{definition}
%Shades $A\setminus B$.
%\begin{definition}[\DescribeMacro{\fillBNotA}]
%\cs{fillBNotA}
%\end{definition}
%Shades $B\setminus A$.
%\begin{definition}[\DescribeMacro{\fillANotC}]
%\cs{fillANotC}
%\end{definition}
%(Not available for 2 set version.) Shades $A\setminus C$.
%\begin{definition}[\DescribeMacro{\fillCNotA}]
%\cs{fillCNotA}
%\end{definition}
%(Not available for 2 set version.) Shades $C\setminus A$.
%\begin{definition}[\DescribeMacro{\fillBNotC}]
%\cs{fillBNotC}
%\end{definition}
%(Not available for 2 set version.) Shades $B\setminus C$.
%\begin{definition}[\DescribeMacro{\fillCNotB}]
%\cs{fillCNotB}
%\end{definition}
%(Not available for 2 set version.) Shades $C\setminus B$.
%
%\begin{definition}[\DescribeMacro{\fillACapB}]
%\cs{fillACapB}
%\end{definition}
%Shades $A \cap B$. (\cs{fillBCapA} is equivalent to \cs{fillACapB}.)
%\begin{definition}[\DescribeMacro{\fillACapC}]
%\cs{fillACapC}
%\end{definition}
%(Not available for 2 set version.) Shades $A \cap C$. (\cs{fillCCapA} is equivalent to \cs{fillACapC}.)
%\begin{definition}[\DescribeMacro{\fillBCapC}]
%\cs{fillBCapC}
%\end{definition}
%(Not available for 2 set version.) Shades $B \cap C$. (\cs{fillCCapB} is equivalent to \cs{fillBCapC}.)
%
%\begin{definition}[\DescribeMacro{\fillACapBNotC}]
%\cs{fillACapBNotC}
%\end{definition}
%(Not available for 2 set version.) Shades $A \cap B \setminus C$.
%(\cs{fillBCapANotC} is equivalent to \cs{fillACapBNotC}.)
%\begin{definition}[\DescribeMacro{\fillACapCNotB}]
%\cs{fillACapCNotB}
%\end{definition}
%(Not available for 2 set version.) Shades $A \cap C \setminus B$.
%(\cs{fillCCapANotB} is equivalent to \cs{fillACapCNotB}.)
%\begin{definition}[\DescribeMacro{\fillBCapCNotA}]
%\cs{fillBCapCNotA}
%\end{definition}
%(Not available for 2 set version.) Shades $B \cap C \setminus A$.
%(\cs{fillCCapBNotA} is equivalent to \cs{fillBCapCNotA}.)
%\begin{definition}[\DescribeMacro{\fillACapBCapC}]
%\cs{fillACapBCapC}
%\end{definition}
%(Not available for 2 set version.) Shades $A \cap B \cap C$.
%(Synonyms: \cs{fillACapCCapB}, \cs{fillBCapACapC}, \cs{fillBCapCCapA},
%\cs{fillCCapACapB}, \cs{fillCCapBCapA}.)
%
%
%\begin{definition}[\DescribeMacro{\setpostvennhook}]
%\cs{setpostvennhook}\marg{cmds}
%\end{definition}
%Sets the hook applied at the very end of the Venn diagram
%environments (after the outline and labels are drawn but before the
%end of the \env{tikzpicture} environment). The Venn diagram
%environments create coordinate nodes \texttt{venn bottom left},
%\texttt{venn top left}, \texttt{venn top right} and \texttt{venn
%bottom right}, which may be referenced within the environment or in
%the hook.
%
%The set labels may also be referenced \emph{but only in
%\cs{setpostvennhook}}: \texttt{labelOnlyA}, \texttt{labelOnlyB},
%\texttt{labelOnlyC} (three set version only), \texttt{labelNotABC}
%(three set version only), \texttt{labelNotAB} (two set version
%only), \texttt{labelA}, \texttt{labelB}, \texttt{labelC} (three set
%version only), \texttt{labelOnlyAB}, \texttt{labelOnlyAC} (three set
%version only), \texttt{labelOnlyBC} (three set version only) and 
%\texttt{labelAB} (two set version only).
%
%\section{Examples}
%
%\begin{enumerate}
%\item (Three sets) $ A \cup B \cup C $
%
%\begin{verbatim}
%\begin{venndiagram3sets}
% \fillA \fillB \fillC
%\end{venndiagram3sets}
%\end{verbatim}
%Produces:
%
%\begin{venndiagram3sets}
% \fillA \fillB \fillC
%\end{venndiagram3sets}
%
%\item (Two sets) $ A \cup B $
%
%\begin{verbatim}
%\begin{venndiagram2sets}
% \fillA \fillB
%\end{venndiagram2sets}
%\end{verbatim}
%Produces:
%
%\begin{venndiagram2sets}
% \fillA \fillB
%\end{venndiagram2sets}
%
%\item (Three sets) $ A \setminus (B \cup C)$
%
%\begin{verbatim}
%\begin{venndiagram3sets}
% \fillOnlyA
%\end{venndiagram3sets}
%\end{verbatim}
%Produces:
%
%\begin{venndiagram3sets}
% \fillOnlyA
%\end{venndiagram3sets}
%
%\item (Two sets) $ A \cap B$:
%
%\begin{verbatim}
%\begin{venndiagram2sets}
%\fillACapB
%\end{venndiagram2sets}
%\end{verbatim}
%
%\begin{venndiagram2sets}
%\fillACapB
%\end{venndiagram2sets}
%
%\item (Three sets) region labels:
%
%\begin{verbatim}
%\begin{venndiagram3sets}[labelOnlyA={1},labelOnlyB={2},labelOnlyC={3},
% labelOnlyAB={4},labelOnlyAC={5},labelOnlyBC={6},labelABC={7},
% labelNotABC={8}]
%\end{venndiagram3sets}
%\end{verbatim}
%Produces:
%
%\begin{venndiagram3sets}[labelOnlyA={1},labelOnlyB={2},labelOnlyC={3},
% labelOnlyAB={4},labelOnlyAC={5},labelOnlyBC={6},labelABC={7},
% labelNotABC={8}]
%\end{venndiagram3sets}
%
%\item Annotating the diagram:
%
%\begin{verbatim}
%\begin{venndiagram3sets}[labelOnlyA={1},labelOnlyB={2},labelOnlyC={3},
% labelOnlyAB={4},labelOnlyAC={5},labelOnlyBC={6},labelABC={7},
% labelNotABC={8}]
%\setpostvennhook
%{
%  \draw[<-] (labelA) -- ++(135:3cm) node[above] {Students who eat
%artichokes};
%  \draw[<-] (labelB) -- ++(45:3cm) node[above] {Students who eat
%broccoli};
%  \draw[<-] (labelC) -- ++(-90:3cm) node[below] {Students who eat
%carrots};
%  \draw[<-] (labelABC) -- ++(0:3cm) 
%    node[right,text width=4cm,align=flush left]
%   {7 students eat artichokes, broccoli and carrots};
%  \draw[<-] (labelNotABC) -- ++(-135:3cm)
%    node[below,text width=4cm,align=flush left]
%   {8 students don't eat artichokes, broccoli or carrots};
%}
%\end{venndiagram3sets}
%\end{verbatim}
%Produces:
%
%\begin{venndiagram3sets}[labelOnlyA={1},labelOnlyB={2},labelOnlyC={3},
% labelOnlyAB={4},labelOnlyAC={5},labelOnlyBC={6},labelABC={7},
% labelNotABC={8}]
%\setpostvennhook
%{
%  \draw[<-] (labelA) -- ++(135:3cm) node[above] {Students who eat
%artichokes};
%  \draw[<-] (labelB) -- ++(45:3cm) node[above] {Students who eat
%broccoli};
%  \draw[<-] (labelC) -- ++(-90:3cm) node[below] {Students who eat
%carrots};
%  \draw[<-] (labelABC) -- ++(0:3cm) 
%    node[right,text width=4cm,align=flush left]
%   {7 students eat artichokes, broccoli and carrots};
%  \draw[<-] (labelNotABC) -- ++(-135:3cm)
%    node[below,text width=4cm,align=flush left]
%   {8 students don't eat artichokes, broccoli or carrots};
%}
%\end{venndiagram3sets}
%
%\end{enumerate}
%
%\StopEventually{\PrintIndex}
%
%
%\section{The Code}
%\iffalse
%    \begin{macrocode}
%<*venndiagram.sty>
%    \end{macrocode}
%\fi
% Package identification:
%    \begin{macrocode}
\NeedsTeXFormat{LaTeX2e}
\ProvidesPackage{venndiagram}[2016/03/16 v1.1 (NLCT) Venn diagrams]
%    \end{macrocode}
% Required packages:
%    \begin{macrocode}
\RequirePackage{xkeyval}
\RequirePackage{tikz}
\RequirePackage{etoolbox}
%    \end{macrocode}
% TiKZ intersections library no longer needed. (Removed in v1.1)
%\changes{1.1}{2016-03-16}{removed intersections library}
%
% The \sty{pgf} package reversed the order of arguments of
% \texttt{atan2} in v3.0, which messes things up a bit.
% In the event that there are users with older versions of
% \sty{pgf}, backward-compatibility is required. Add a switch to
% determine which syntax to use.
%\begin{macro}{\ifvennoldpgf}
%\changes{1.1}{2016-03-16}{new}
% If true use old syntax.
%    \begin{macrocode}
\newif\ifvennoldpgf
%    \end{macrocode}
%\end{macro}
% Try to determine this setting.
%    \begin{macrocode}
\ifdef\pgfversion
{
  \def\@venn@checkversion#1.#2\@venn@end@checkversion{%
  \ifnum#1<3 \vennoldpgftrue \else \vennoldpgffalse\fi}
  \expandafter\@venn@checkversion\pgfversion.0\@venn@end@checkversion
}
{
  \vennoldpgftrue
}
%    \end{macrocode}
%
%\subsection{Initialising the Default Values}
% Set up macros used by the keys for the Venn diagram options. First the 
% default set labels.
%\begin{macro}{\@venn@label@A}
% Set $A$:
%    \begin{macrocode}
\newcommand*{\@venn@label@A}{$A$}
%    \end{macrocode}
%\end{macro}
%\begin{macro}{\@venn@label@B}
% Set $B$:
%    \begin{macrocode}
\newcommand*{\@venn@label@B}{$B$}
%    \end{macrocode}
%\end{macro}
%\begin{macro}{\@venn@label@C}
% Set $C$:
%    \begin{macrocode}
\newcommand*{\@venn@label@C}{$C$}
%    \end{macrocode}
%\end{macro}
% 
%\begin{macro}{\@venn@shade}
% The colour used to shade regions.
%    \begin{macrocode}
\newcommand*{\@venn@shade}{lightgray}
%    \end{macrocode}
%\end{macro}
%
% The default labels for all the other regions are empty.
%\begin{macro}{\@venn@label@OnlyA}
% Only set $A$:
%    \begin{macrocode}
\newcommand*{\@venn@label@OnlyA}{}
%    \end{macrocode}
%\end{macro}
%\begin{macro}{\@venn@label@OnlyB}
% Only set $B$:
%    \begin{macrocode}
\newcommand*{\@venn@label@OnlyB}{}
%    \end{macrocode}
%\end{macro}
%\begin{macro}{\@venn@label@OnlyC}
% Only set $C$:
%    \begin{macrocode}
\newcommand*{\@venn@label@OnlyC}{}
%    \end{macrocode}
%\end{macro}
%\begin{macro}{\@venn@label@OnlyAB}
% Sets $A$ and $B$ but not $C$:
%    \begin{macrocode}
\newcommand*{\@venn@label@OnlyAB}{}
%    \end{macrocode}
%\end{macro}
%\begin{macro}{\@venn@label@OnlyAC}
% Sets $A$ and $C$ but not $B$:
%    \begin{macrocode}
\newcommand*{\@venn@label@OnlyAC}{}
%    \end{macrocode}
%\end{macro}
%\begin{macro}{\@venn@label@OnlyBC}
% Sets $B$ and $C$ but not $A$:
%    \begin{macrocode}
\newcommand*{\@venn@label@OnlyBC}{}
%    \end{macrocode}
%\end{macro}
%\begin{macro}{\@venn@label@ABC}
% Intersection of sets $A$, $B$ and $C$:
%    \begin{macrocode}
\newcommand*{\@venn@label@ABC}{}
%    \end{macrocode}
%\end{macro}
%\begin{macro}{\@venn@label@NotABC}
% Everything except $A$, $B$ or $C$:
%    \begin{macrocode}
\newcommand*{\@venn@label@NotABC}{}
%    \end{macrocode}
%\end{macro}
%\begin{macro}{\@venn@label@NotAB}
% Everything except $A$ or $B$ (two set version only):
%    \begin{macrocode}
\newcommand*{\@venn@label@NotAB}{}
%    \end{macrocode}
%\end{macro}
%\begin{macro}{\@venn@label@AB}
% Intersection of $A$ and $B$ (two set version only):
%    \begin{macrocode}
\newcommand*{\@venn@label@AB}{}
%    \end{macrocode}
%\end{macro}
%
% Now the default dimensions of the diagrams.
%\begin{macro}{\@venn@radius}
% The radius of the sets.
%    \begin{macrocode}
\newcommand*{\@venn@radius}{1.2cm}
%    \end{macrocode}
%\end{macro}
%\begin{macro}{\@venn@hgap}
% The horizontal distance between the edge of the diagram and the
% outer edge of the nearest set.
%    \begin{macrocode}
\newcommand*{\@venn@hgap}{0.5cm}
%    \end{macrocode}
%\end{macro}
%\begin{macro}{\@venn@vgap}
%    \begin{macrocode}
% The vertical distance between the edge of the diagram and the
% outer edge of the nearest set.
\newcommand*{\@venn@vgap}{0.5cm}
%    \end{macrocode}
%\end{macro}
%\begin{macro}{\@venn@overlap}
% The size of the set overlap.
%    \begin{macrocode}
\newcommand*{\@venn@overlap}{0.75cm}
%    \end{macrocode}
%\end{macro}
%
%\begin{macro}{\@venn@tikzoptions}
% Any options to be passed to the \env{tikzpicture} environment.
%    \begin{macrocode}
\newcommand*{\@venn@tikzoptions}{}
%    \end{macrocode}
%\end{macro}
%
% Lengths to store the centres of the sets and the overall width and
% height of the diagram.
%\begin{macro}{\@venn@Ax}
% The $x$-coordinate of set $A$:
%    \begin{macrocode}
\newlength\@venn@Ax
%    \end{macrocode}
%\end{macro}
%\begin{macro}{\@venn@Ay}
% The $y$-coordinate of set $A$:
%    \begin{macrocode}
\newlength\@venn@Ay
%    \end{macrocode}
%\end{macro}
%\begin{macro}{\@venn@Bx}
% The $x$-coordinate of set $B$:
%    \begin{macrocode}
\newlength\@venn@Bx
%    \end{macrocode}
%\end{macro}
%\begin{macro}{\@venn@By}
% The $y$-coordinate of set $B$:
%    \begin{macrocode}
\newlength\@venn@By
%    \end{macrocode}
%\end{macro}
%\begin{macro}{\@venn@Cx}
% The $x$-coordinate of set $C$:
%    \begin{macrocode}
\newlength\@venn@Cx
%    \end{macrocode}
%\end{macro}
%\begin{macro}{\@venn@Cy}
% The $y$-coordinate of set $C$:
%    \begin{macrocode}
\newlength\@venn@Cy
%    \end{macrocode}
%\end{macro}
%
%\begin{macro}{\@venn@w}
% The width of the entire Venn diagram.
%    \begin{macrocode}
\newlength\@venn@w
%    \end{macrocode}
%\end{macro}
%\begin{macro}{\@venn@h}
% The height of the entire Venn diagram.
%    \begin{macrocode}
\newlength\@venn@h
%    \end{macrocode}
%\end{macro}
%
%\subsection{Defining the key=value Options}
%
% Now define the keys for the optional argument of
% \env{venndiagram2sets} and \env{venndiagram3sets}.
% They are all in the family "venn".
%
%\begin{keyvaloption}{shade}
% Option to set the shading.
%    \begin{macrocode}
\define@key{venn}{shade}{\def\@venn@shade{#1}}
%    \end{macrocode}
%\end{keyvaloption}
%\begin{keyvaloption}{labelA}
% Option to set the label for set $A$.
%    \begin{macrocode}
\define@key{venn}{labelA}{\def\@venn@label@A{#1}}
%    \end{macrocode}
%\end{keyvaloption}
%\begin{keyvaloption}{labelB}
% Option to set the label for set $B$.
%    \begin{macrocode}
\define@key{venn}{labelB}{\def\@venn@label@B{#1}}
%    \end{macrocode}
%\end{keyvaloption}
%\begin{keyvaloption}{labelC}
% Option to set the label for set $C$.
%    \begin{macrocode}
\define@key{venn}{labelC}{\def\@venn@label@C{#1}}
%    \end{macrocode}
%\end{keyvaloption}
%
% Now for the region labels.
%
%\begin{keyvaloption}{labelOnlyA}
% Option to set the label for only set $A$.
%    \begin{macrocode}
\define@key{venn}{labelOnlyA}{\def\@venn@label@OnlyA{#1}}
%    \end{macrocode}
%\end{keyvaloption}
%\begin{keyvaloption}{labelOnlyB}
% Option to set the label for only set $B$.
%    \begin{macrocode}
\define@key{venn}{labelOnlyB}{\def\@venn@label@OnlyB{#1}}
%    \end{macrocode}
%\end{keyvaloption}
%\begin{keyvaloption}{labelOnlyC}
% Option to set the label for only set $C$.
%    \begin{macrocode}
\define@key{venn}{labelOnlyC}{\def\@venn@label@OnlyC{#1}}
%    \end{macrocode}
%\end{keyvaloption}
%\begin{keyvaloption}{labelOnlyAB}
% Option to set the label for the intersection of $A$ and $B$.
%    \begin{macrocode}
\define@key{venn}{labelOnlyAB}{\def\@venn@label@OnlyAB{#1}}
%    \end{macrocode}
%\end{keyvaloption}
%\begin{keyvaloption}{labelOnlyAC}
% Option to set the label for the intersection of $A$ and $C$.
%    \begin{macrocode}
\define@key{venn}{labelOnlyAC}{\def\@venn@label@OnlyAC{#1}}
%    \end{macrocode}
%\end{keyvaloption}
%\begin{keyvaloption}{labelOnlyBC}
% Option to set the label for the intersection of $B$ and $C$.
%    \begin{macrocode}
\define@key{venn}{labelOnlyBC}{\def\@venn@label@OnlyBC{#1}}
%    \end{macrocode}
%\end{keyvaloption}
%\begin{keyvaloption}{labelABC}
% Option to set the label for the intersection of $A$, $B$ and $C$.
% (Three set version only)
%    \begin{macrocode}
\define@key{venn}{labelABC}{\def\@venn@label@ABC{#1}}
%    \end{macrocode}
%\end{keyvaloption}
%\begin{keyvaloption}{labelNotABC}
% Option to set the label for the region outside the three sets.
% (Three set version only)
%    \begin{macrocode}
\define@key{venn}{labelNotABC}{\def\@venn@label@NotABC{#1}}
%    \end{macrocode}
%\end{keyvaloption}
%\begin{keyvaloption}{labelAB}
% Option to set the label for the intersection of $A$ and $B$.
% (Two set version only)
%    \begin{macrocode}
\define@key{venn}{labelAB}{\def\@venn@label@AB{#1}}
%    \end{macrocode}
%\end{keyvaloption}
%\begin{keyvaloption}{labelNotAB}
% Option to set the label for the region outside the two sets.
% (Two set version only)
%    \begin{macrocode}
\define@key{venn}{labelNotAB}{\def\@venn@label@NotAB{#1}}
%    \end{macrocode}
%\end{keyvaloption}
%
% Now for the dimension options.
%
%\begin{keyvaloption}{radius}
% Option to set the radius.
%    \begin{macrocode}
\define@key{venn}{radius}{\def\@venn@radius{#1}}
%    \end{macrocode}
%\end{keyvaloption}
%\begin{keyvaloption}{hgap}
% Option to set the horizontal gap between the outer edge of the
% diagram and the nearest set edge.
%    \begin{macrocode}
\define@key{venn}{hgap}{\def\@venn@hgap{#1}}
%    \end{macrocode}
%\end{keyvaloption}
%\begin{keyvaloption}{vgap}
% Option to set the vertical gap between the outer edge of the
% diagram and the nearest set edge.
%    \begin{macrocode}
\define@key{venn}{vgap}{\def\@venn@vgap{#1}}
%    \end{macrocode}
%\end{keyvaloption}
%\begin{keyvaloption}{overlap}
% Option to set the set overlap.
%    \begin{macrocode}
\define@key{venn}{overlap}{\def\@venn@overlap{#1}}
%    \end{macrocode}
%\end{keyvaloption}
%
% Finally the option to set the information to pass to the
% \env{tikzpicture} environment.
%\begin{keyvaloption}{tikzoptions}
%    \begin{macrocode}
\define@key{venn}{tikzoptions}{\def\@venn@tikzoptions{#1}}
%    \end{macrocode}
%\end{keyvaloption}
%
%\subsection{Circle Intersection}
% Previously commands like \cs{fillOnlyA} used \sty{pgf} path operations
% to compute the intersection points of the circles, but the code
% didn't work properly when the co-ordinate system has been scaled.
% Version 1.1 changes this to calculate the co-ordinates in a more
% low-level way. We have two circles centred on $(a_x, a_y)$ and
% $(b_x, b_y)$ both with radius $r$. This gives the equations:
%\begin{align}
%(x-a_x)^2+(y-a_y)^2 &= r^2\label{eq:A}\\
%(x-b_x)^2+(y-b_y)^2 &= r^2
%\end{align}
%Combining:
%\begin{equation}
%(x-a_x)^2-(x-b_x)^2+(y-a_y)^2-(y-b_y)^2 = 0\label{eq:chord}
%\end{equation}
%Re-arranging gives the equation of the chord between the two points
%of intersection:
%\[
%x(b_x-a_x)+y(b_y-a_y) = \frac{b_x^2-a_x^2+b_y^2-a_y^2}{2}
%\]
%Special cases:
%\begin{enumerate}
%\item $b_x=a_x$ (circles vertically stacked):
%\[y(b_y-a_y) = \frac{b_y^2-a_y^2}{2}\]
%Rearranging gives $y=\frac{1}{2}(b_y+a_y)$. Substituting into
%\eqref{eq:A}:
%\begin{align*}
%(x-a_x)^2+\left(\frac{1}{2}(b_y+a_y)-a_y\right)^2 &= r^2\\
%(x-a_x)^2+\frac{1}{4}(b_y-a_y)^2 &= r^2\\
%x^2 - 2x a_x + a_x^2+\frac{(b_y-a_y)^2}{4} -r^2 &=0
%\end{align*}
%This is a quadratic equation in $x$ with solutions given by
%\begin{align*}
%x &= a_x\pm\sqrt{a_x^2-\left(a_x^2+\frac{(b_y-a_y)^2}{4}-r^2\right)}\\
%&= a_x\pm\sqrt{r^2-\frac{1}{4}(b_y-a_y)^2}
%\end{align*}
%If $r^2 < \frac{1}{4}(b_y-a_y)^2$ then no solution exists (circles
%don't overlap). If $r^2 = \frac{1}{4}(b_y-a_y)^2$ then there's only
%one point of intersection.
%\item $b_y=a_y$ (circles horizontally aligned):
%\[x(b_x-a_x) = \frac{b_x^2-a_x^2}{2} \]
%Rearranging gives $x=\frac{1}{2}(b_x+a_x)$. Substituting into
%\eqref{eq:A}:
%\begin{align*}
%\left(\frac{b_x+a_x}{2}-a_x\right)^2+(y-a_y)^2 &= r^2\\
%\frac{1}{4}(b_x-a_x)^2+(y-a_y)^2 &= r^2\\
%y^2 - 2ya_y+a_y^2+\frac{1}{4}(b_x-a_x)^2-r^2 &= 0
%\end{align*}
%This is a quadratic equation in $y$ with solutions given by
%\begin{align*}
%y &= a_y\pm\sqrt{a_y^2-\left(a_y^2+\frac{(b_x-a_x)^2}{4}-r^2\right)}\\
%&= a_y\pm\sqrt{r^2-\frac{1}{4}(b_x-a_x)^2}
%\end{align*}
%If $r^2 < \frac{1}{4}(b_x-a_x)^2$ then no solution exists (circles
%don't overlap). If $r^2 = \frac{1}{4}(b_x-a_x)^2$ then there's only
%one point of intersection.
%\end{enumerate}
%The general case has the chord given by equation~\eqref{eq:chord},
%which can be rewritten in the form $y = mx + c$ where
%\begin{align*}
% m &= \frac{a_x-b_x}{b_y-a_y}\\
% c &= \frac{b_x^2-a_x^2+b_y^2-a_y^2}{2(b_y-a_y)}
%\end{align*}
%Substituting into equation~\eqref{eq:A}:
%\[
%(x-a_x)^2+((mx+c)-a_y)^2 = r^2
%\]
%Rearranging:
%\begin{displaymath}
%(1+m^2)x^2+2x(mc-a_x-a_ym)+a_x^2+c^2-2a_yc+a_y^2-r^2 = 0
%\end{displaymath}
%This is a quadratic solution in $x$ with solutions given by
%\[
%x = \frac{-(mc-a_x-a_ym)\pm
%\sqrt{(mc-a_x-a_ym)^2-(1+m^2)(a_x^2+c^2-2a_yc+a_y^2-r^2)}}{1+m^2}
%\]
%
%Provide command to compute the intersection of two circles of the
%same radius (given by \cs{@venn@radius}.
%\begin{macro}{\@venn@computecircleintersects}
%The four arguments are $a_x$, $a_y$, $b_x$ and $b_y$ (the centre
%co-ordinates in lengths). The results are stored in \cs{@venn@intersect@i@x},
% \cs{@venn@intersect@i@y}, \cs{@venn@intersect@ii@x} and
% \cs{@venn@intersect@ii@y}. The number of intersects (0, 1 or 2)
% is stored in \cs{@venn@intersect@n}.
%    \begin{macrocode}
\newcommand*{\@venn@computecircleintersects}[4]{%
 \ifdim#1=#3\relax
%    \end{macrocode}
% Case 1 ($a_x = b_x$). Compute $y = \frac{1}{2}(b_y+a_y)$.
%    \begin{macrocode}
  \pgfmathsetlength{\@venn@intersect@i@y}{0.5*(#4+#2)}%
  \setlength{\@venn@intersect@ii@y}{\@venn@intersect@i@y}%
%    \end{macrocode}
% Compute $r^2 - \frac{1}{4}(b_y-a_y)^2$
%    \begin{macrocode}
  \pgfmathsetlength{\dimen@i}{#4-#2}%
  \pgfmathsetlength{\dimen@}{\@venn@radius*\@venn@radius-0.25*\dimen@i*\dimen@i}%
  \setlength{\@venn@intersect@i@x}{#1}%
  \setlength{\@venn@intersect@ii@x}{#1}%
  \ifdim\dimen@=0pt\relax
%    \end{macrocode}
% One point of intersection.
%    \begin{macrocode}
   \def\@venn@intersect@n{1}%
  \else
   \ifdim\dimen@>0pt\relax
%    \end{macrocode}
% Two points of intersection.
%    \begin{macrocode}
     \def\@venn@intersect@n{2}%
     \pgfmathsetlength{\dimen@i}{sqrt(\dimen@)}%
     \addtolength{\@venn@intersect@i@x}{\dimen@i}%
     \addtolength{\@venn@intersect@ii@x}{-\dimen@i}%
   \else
%    \end{macrocode}
% No intersection.
%    \begin{macrocode}
     \def\@venn@intersect@n{0}%
   \fi
  \fi
 \else
   \ifdim#2=#4\relax
%    \end{macrocode}
% Case 2 ($a_y = b_y$). Compute $x = \frac{1}{2}(b_x+a_x)$
%    \begin{macrocode}
     \pgfmathsetlength{\@venn@intersect@i@x}{0.5*(#3+#1)}%
     \setlength{\@venn@intersect@ii@x}{\@venn@intersect@i@x}%
%    \end{macrocode}
% Compute $r^2 - \frac{1}{4}(b_x-a_x)^2$
%    \begin{macrocode}
     \setlength{\dimen@i}{#3}%
     \addtolength{\dimen@i}{-#1}%
     \pgfmathsetlength{\dimen@}{\@venn@radius*\@venn@radius
       -0.25*\dimen@i*\dimen@i}%
     \setlength{\@venn@intersect@i@y}{#2}%
     \setlength{\@venn@intersect@ii@y}{#2}%
     \ifdim\dimen@=0pt\relax
%    \end{macrocode}
% One point of intersection.
%    \begin{macrocode}
      \def\@venn@intersect@n{1}%
     \else
      \ifdim\dimen@>0pt\relax
%    \end{macrocode}
% Two points of intersection.
%    \begin{macrocode}
        \def\@venn@intersect@n{2}%
        \pgfmathsetlength{\dimen@i}{sqrt(\dimen@)}%
        \addtolength{\@venn@intersect@i@y}{\dimen@i}%
        \addtolength{\@venn@intersect@ii@y}{-\dimen@i}%
      \else
%    \end{macrocode}
% No intersection.
%    \begin{macrocode}
        \def\@venn@intersect@n{0}%
      \fi
     \fi
   \else
%    \end{macrocode}
% General case. Convert all lengths to scalar to reduce chances of
% exceeding max dimension. Using inches to ensure more reasonable
% values.
%    \begin{macrocode}
    \pgfmathparse{#1/72.27}\let\@vnn@ax\pgfmathresult
    \pgfmathparse{#2/72.27}\let\@vnn@ay\pgfmathresult
    \pgfmathparse{#3/72.27}\let\@vnn@bx\pgfmathresult
    \pgfmathparse{#4/72.27}\let\@vnn@by\pgfmathresult
    \pgfmathparse{\@venn@radius/72.27}\let\@vnn@r\pgfmathresult
    \pgfmathparse{\@vnn@ax*\@vnn@ax}\let\@vnn@ax@sq\pgfmathresult
    \pgfmathparse{\@vnn@ay*\@vnn@ay}\let\@vnn@ay@sq\pgfmathresult
    \pgfmathparse{\@vnn@bx*\@vnn@bx}\let\@vnn@bx@sq\pgfmathresult
    \pgfmathparse{\@vnn@by*\@vnn@by}\let\@vnn@by@sq\pgfmathresult
    \pgfmathparse{\@vnn@r*\@vnn@r}\let\@vnn@r@sq\pgfmathresult
%    \end{macrocode}
% Set
%\[c = \frac{b_x^2-a_x^2+b_y^2-a_y^2}{2(b_y-a_y)}\]
%    \begin{macrocode}
    \pgfmathparse{0.5*(\@vnn@bx@sq-\@vnn@ax@sq+\@vnn@by@sq-\@vnn@ay@sq)
       /(\@vnn@by-\@vnn@ay)}%
    \let\@vnn@c\pgfmathresult
%    \end{macrocode}
% Set
%\[m = \frac{a_x-b_x}{b_y-a_y}\]
%    \begin{macrocode}
    \pgfmathparse{(\@vnn@ax-\@vnn@bx)/(\@vnn@by-\@vnn@ay)}%
    \let\@vnn@m\pgfmathresult
%    \end{macrocode}
% Compute $(1+m^2)$.
%    \begin{macrocode}
    \pgfmathparse{1+\@vnn@m*\@vnn@m}%
    \let\@vnn@one@plus@m@sq\pgfmathresult
%    \end{macrocode}
% Compute \[mc-a_x-a_ym = m(c-a_y)-a_x\]
%    \begin{macrocode}
     \pgfmathparse{\@vnn@m*(\@vnn@c-\@vnn@ay)-\@vnn@ax}%
     \let\@vnn@b\pgfmathresult
%    \end{macrocode}
% Denote this value $b$ and now compute
% \[b^2-(1+m^2)(a_x^2+a_y^2+c^2-2a_yc-r^2)\]
%    \begin{macrocode}
     \pgfmathparse{\@vnn@b*\@vnn@b
      - \@vnn@one@plus@m@sq
       *(\@vnn@ax@sq+\@vnn@ay@sq+\@vnn@c*\@vnn@c
         -2*\@vnn@ay*\@vnn@c-\@vnn@r@sq)}%
     \let\@vnn@root\pgfmathresult
     \setlength\dimen@{\pgfmathresult in}%
     \ifdim\dimen@=0pt\relax
%    \end{macrocode}
% One point of intersection.
%    \begin{macrocode}
       \def\@venn@intersect@n{1}%
%    \end{macrocode}
%\[x_1 = \frac{-b}{1+m^2}\]
%    \begin{macrocode}
        \pgfmathparse{-\@vnn@b/\@vnn@one@plus@m@sq}%
        \setlength{\@venn@intersect@i@x}{\pgfmathresult in}%
        \setlength{\@venn@intersect@ii@x}{\@venn@intersect@i@x}%
%    \end{macrocode}
%\[y_1 = mx_1 + c\]
%    \begin{macrocode}
        \pgfmathparse{\pgfmathresult*\@vnn@m+c}%
        \setlength{\@venn@intersect@i@y}{\pgfmathresult in}%
        \setlength{\@venn@intersect@ii@y}{\@venn@intesect@i@y}%
     \else
       \ifdim\dimen@>0pt\relax
%    \end{macrocode}
% Two points of intersection.
%    \begin{macrocode}
         \def\@venn@intersect@n{2}%
         \pgfmathsqrt{\@vnn@root}%
         \let\@vnn@root\pgfmathresult
%    \end{macrocode}
% First point.
%    \begin{macrocode}
         \pgfmathparse{(-\@vnn@b+\@vnn@root)/\@vnn@one@plus@m@sq}%
         \setlength{\@venn@intersect@i@x}{\pgfmathresult in}%
         \pgfmathparse{\pgfmathresult*\@vnn@m+\@vnn@c}%
         \setlength{\@venn@intersect@i@y}{\pgfmathresult in}%
%    \end{macrocode}
% Second point.
%    \begin{macrocode}
         \pgfmathparse{(-\@vnn@b-\@vnn@root)/\@vnn@one@plus@m@sq}%
         \setlength{\@venn@intersect@ii@x}{\pgfmathresult in}%
         \pgfmathparse{\pgfmathresult*\@vnn@m+\@vnn@c}%
         \setlength{\@venn@intersect@ii@y}{\pgfmathresult in}%
       \else
%    \end{macrocode}
% No intersection.
%    \begin{macrocode}
         \def\@venn@intersect@n{0}%
       \fi
     \fi
   \fi
 \fi
}
%    \end{macrocode}
%\end{macro}
%
%\begin{macro}{\@venn@intersect@i@x}
%\changes{1.1}{2016-03-16}{new}
%    \begin{macrocode}
\newlength\@venn@intersect@i@x
%    \end{macrocode}
%\end{macro}
%
%\begin{macro}{\@venn@intersect@i@y}
%\changes{1.1}{2016-03-16}{new}
%    \begin{macrocode}
\newlength\@venn@intersect@i@y
%    \end{macrocode}
%\end{macro}
%
%\begin{macro}{\@venn@intersect@ii@x}
%\changes{1.1}{2016-03-16}{new}
%    \begin{macrocode}
\newlength\@venn@intersect@ii@x
%    \end{macrocode}
%\end{macro}
%
%\begin{macro}{\@venn@intersect@ii@y}
%\changes{1.1}{2016-03-16}{new}
%    \begin{macrocode}
\newlength\@venn@intersect@ii@y
%    \end{macrocode}
%\end{macro}
%
%\subsection{Environment Definitions}
%
%\begin{environment}{venndiagram3sets}
% Environment to draw Venn diagram with three sets.
%    \begin{macrocode}
\newenvironment{venndiagram3sets}[1][]%
{%
%    \end{macrocode}
% Disable the keys that aren't applicable.
%    \begin{macrocode}
  \disable@keys{venn}{labelAB,labelNotAB}%
%    \end{macrocode}
% Set the key values given in the optional argument.
%    \begin{macrocode}
  \setkeys{venn}{#1}%
%    \end{macrocode}
% Calculate centre of set $C$
%    \begin{macrocode}
  \pgfmathsetlength{\@venn@Cx}{\@venn@hgap + 2*\@venn@radius
    -0.5*\@venn@overlap}%
  \pgfmathsetlength{\@venn@Cy}{\@venn@vgap+\@venn@radius}%
%    \end{macrocode}
% Calculate centre of set $A$
%    \begin{macrocode}
  \pgfmathsetlength{\@venn@Ax}{\@venn@hgap+\@venn@radius}%
  \pgfmathsetlength{\@venn@Ay}{\@venn@Cy 
    + (\@venn@radius - 0.5*\@venn@overlap)*1.73205}%
%    \end{macrocode}
% Calculate centre of set $B$
%    \begin{macrocode}
  \pgfmathsetlength{\@venn@Bx}{\@venn@hgap+3*\@venn@radius
    -\@venn@overlap}%
  \setlength{\@venn@By}{\@venn@Ay}%
%    \end{macrocode}
% Compute dimensions of entire diagram
%    \begin{macrocode}
  \pgfmathsetlength{\@venn@w}{2*\@venn@hgap+4*\@venn@radius
     -\@venn@overlap}%
  \pgfmathsetlength{\@venn@h}{2*\@venn@vgap+4*\@venn@radius
     -\@venn@overlap}%
%    \end{macrocode}
% Define filling commands.
% Fill all of set $A$:
%    \begin{macrocode}
  \def\fillA{\path[fill=\@venn@shade] (\@venn@Ax,\@venn@Ay)
    circle (\@venn@radius);}%
%    \end{macrocode}
% Fill all of set $B$:
%    \begin{macrocode}
  \def\fillB{\path[fill=\@venn@shade] (\@venn@Bx,\@venn@By)
    circle (\@venn@radius);}%
%    \end{macrocode}
% Fill all of set $C$:
%    \begin{macrocode}
  \def\fillC{\path[fill=\@venn@shade] (\@venn@Cx,\@venn@Cy)
    circle (\@venn@radius);}%
%    \end{macrocode}
% Fill everything:
%    \begin{macrocode}
  \def\fillAll{\path[fill=\@venn@shade] (0,0)
    rectangle (\@venn@w,\@venn@h);}%
%    \end{macrocode}
% Fill everything except set $A$:
%    \begin{macrocode}
  \def\fillNotA{\path[fill=\@venn@shade,even odd rule]
    (0,0) rectangle (\@venn@w,\@venn@h)
    (\@venn@Ax,\@venn@Ay) circle (\@venn@radius);}%
%    \end{macrocode}
% Fill everything except set $B$:
%    \begin{macrocode}
  \def\fillNotB{\path[fill=\@venn@shade,even odd rule]
    (0,0) rectangle (\@venn@w,\@venn@h)
    (\@venn@Bx,\@venn@By) circle (\@venn@radius);}%
%    \end{macrocode}
% Fill everything except set $C$:
%    \begin{macrocode}
  \def\fillNotC{\path[fill=\@venn@shade,even odd rule]
    (0,0) rectangle (\@venn@w,\@venn@h)
    (\@venn@Cx,\@venn@Cy) circle (\@venn@radius);}%
%    \end{macrocode}
% Fill only set $A$.
%    \begin{macrocode}
  \let\fillOnlyA\@venn@fillOnlyA@threesets
%    \end{macrocode}
% Fill only set $B$:
%    \begin{macrocode}
  \let\fillOnlyB\@venn@fillOnlyB@threesets
%    \end{macrocode}
% Fill only set $C$:
%    \begin{macrocode}
  \let\fillOnlyC\@venn@fillOnlyC@threesets
%    \end{macrocode}
% Fill everything except $A$, $B$ or $C$.
%    \begin{macrocode}
  \let\fillNotABC\@venn@fillNotABC@threesets
%    \end{macrocode}
% Fill $A$ but not $B$
%    \begin{macrocode}
  \def\fillANotB{%
    \begin{scope}
      \clip (\@venn@Ax,\@venn@Ay) circle (\@venn@radius);
      \path[fill=\@venn@shade,even odd rule] 
        (\@venn@Ax,\@venn@Ay) circle (\@venn@radius)
        (\@venn@Bx,\@venn@By) circle (\@venn@radius);
    \end{scope}
  }%
%    \end{macrocode}
% Fill $B$ but not $A$
%    \begin{macrocode}
  \def\fillBNotA{%
    \begin{scope}
      \clip (\@venn@Bx,\@venn@By) circle (\@venn@radius);
      \path[fill=\@venn@shade,even odd rule] 
        (\@venn@Bx,\@venn@By) circle (\@venn@radius)
        (\@venn@Ax,\@venn@Ay) circle (\@venn@radius);
    \end{scope}
  }%
%    \end{macrocode}
% Fill $A$ but not $C$
%    \begin{macrocode}
  \def\fillANotC{%
    \begin{scope}
      \clip (\@venn@Ax,\@venn@Ay) circle (\@venn@radius);
      \path[fill=\@venn@shade,even odd rule] 
        (\@venn@Ax,\@venn@Ay) circle (\@venn@radius)
        (\@venn@Cx,\@venn@Cy) circle (\@venn@radius);
    \end{scope}
  }%
%    \end{macrocode}
% Fill $C$ but not $A$
%    \begin{macrocode}
  \def\fillCNotA{%
    \begin{scope}
      \clip (\@venn@Cx,\@venn@Cy) circle (\@venn@radius);
      \path[fill=\@venn@shade,even odd rule] 
        (\@venn@Cx,\@venn@Cy) circle (\@venn@radius)
        (\@venn@Ax,\@venn@Ay) circle (\@venn@radius);
    \end{scope}
  }%
%    \end{macrocode}
% Fill $B$ but not $C$
%    \begin{macrocode}
  \def\fillBNotC{%
    \begin{scope}
      \clip (\@venn@Bx,\@venn@By) circle (\@venn@radius);
      \path[fill=\@venn@shade,even odd rule] 
        (\@venn@Bx,\@venn@By) circle (\@venn@radius)
        (\@venn@Cx,\@venn@Cy) circle (\@venn@radius);
    \end{scope}
  }%
%    \end{macrocode}
% Fill $C$ but not $B$
%    \begin{macrocode}
  \def\fillCNotB{%
    \begin{scope}
      \clip (\@venn@Cx,\@venn@Cy) circle (\@venn@radius);
      \path[fill=\@venn@shade,even odd rule] 
        (\@venn@Cx,\@venn@Cy) circle (\@venn@radius)
        (\@venn@Bx,\@venn@By) circle (\@venn@radius);
    \end{scope}
  }%
%    \end{macrocode}
% Fill $A$ intersect $B$
%    \begin{macrocode}
  \def\fillACapB{%
    \begin{scope}
      \clip (\@venn@Ax,\@venn@Ay) circle (\@venn@radius);
      \path[fill=\@venn@shade] 
        (\@venn@Bx,\@venn@By) circle (\@venn@radius);
    \end{scope}
  }%
%    \end{macrocode}
% Define a synonym:
%    \begin{macrocode}
  \let\fillBCapA\fillACapB
%    \end{macrocode}
% Fill $A$ intersect $C$
%    \begin{macrocode}
  \def\fillACapC{%
    \begin{scope}
      \clip (\@venn@Ax,\@venn@Ay) circle (\@venn@radius);
      \path[fill=\@venn@shade] 
        (\@venn@Cx,\@venn@Cy) circle (\@venn@radius);
    \end{scope}
  }%
%    \end{macrocode}
% Define a synonym:
%    \begin{macrocode}
  \let\fillCCapA\fillACapC
%    \end{macrocode}
% Fill $B$ intersect $C$
%    \begin{macrocode}
  \def\fillBCapC{%
    \begin{scope}
      \clip (\@venn@Bx,\@venn@By) circle (\@venn@radius);
      \path[fill=\@venn@shade] 
        (\@venn@Cx,\@venn@Cy) circle (\@venn@radius);
    \end{scope}
  }%
%    \end{macrocode}
% Define a synonym:
%    \begin{macrocode}
  \let\fillCCapB\fillBCapC
%    \end{macrocode}
% Fill $A$ intersect $B$ but not $C$
%    \begin{macrocode}
  \def\fillACapBNotC{%
    \begin{scope}
      \clip (\@venn@Ax,\@venn@Ay) circle (\@venn@radius);
      \clip (\@venn@Bx,\@venn@By) circle (\@venn@radius);
      \path[fill=\@venn@shade,even odd rule] 
        (\@venn@Bx,\@venn@By) circle (\@venn@radius)
        (\@venn@Cx,\@venn@Cy) circle (\@venn@radius);
    \end{scope}
  }%
%    \end{macrocode}
% Define a synonym:
%    \begin{macrocode}
  \let\fillBCapANotC\fillACapBNotC
%    \end{macrocode}
% Fill $A$ intersect $C$ but not $B$
%    \begin{macrocode}
  \def\fillACapCNotB{%
    \begin{scope}
      \clip (\@venn@Ax,\@venn@Ay) circle (\@venn@radius);
      \clip (\@venn@Cx,\@venn@Cy) circle (\@venn@radius);
      \path[fill=\@venn@shade,even odd rule] 
        (\@venn@Cx,\@venn@Cy) circle (\@venn@radius)
        (\@venn@Bx,\@venn@By) circle (\@venn@radius);
    \end{scope}
  }%
%    \end{macrocode}
% Define a synonym:
%    \begin{macrocode}
  \let\fillCCapANotB\fillACapCNotB
%    \end{macrocode}
% Fill $B$ intersect $C$ but not $A$
%    \begin{macrocode}
  \def\fillBCapCNotA{%
    \begin{scope}
      \clip (\@venn@Bx,\@venn@By) circle (\@venn@radius);
      \clip (\@venn@Cx,\@venn@Cy) circle (\@venn@radius);
      \path[fill=\@venn@shade,even odd rule] 
        (\@venn@Cx,\@venn@Cy) circle (\@venn@radius)
        (\@venn@Ax,\@venn@Ay) circle (\@venn@radius);
    \end{scope}
  }%
%    \end{macrocode}
% Define a synonym:
%    \begin{macrocode}
  \let\fillCCapBNotA\fillBCapCNotA
%    \end{macrocode}
% Fill the intersection of all three sets
%    \begin{macrocode}
  \def\fillACapBCapC{%
    \begin{scope}
      \clip (\@venn@Ax,\@venn@Ay) circle (\@venn@radius);
      \clip (\@venn@Cx,\@venn@Cy) circle (\@venn@radius);
      \path[fill=\@venn@shade] 
        (\@venn@Bx,\@venn@By) circle (\@venn@radius);
    \end{scope}
  }%
%    \end{macrocode}
% Define synonyms:
%    \begin{macrocode}
  \let\fillACapCCapB\fillACapBCapC
  \let\fillBCapACapC\fillACapBCapC
  \let\fillBCapCCapA\fillACapBCapC
  \let\fillCCapACapB\fillACapBCapC
  \let\fillCCapBCapA\fillACapBCapC
%    \end{macrocode}
% Start the \env{tikzpicture} environment.
%    \begin{macrocode}
  \ifdefempty{\@venn@tikzoptions}%
  {%
    \def\@venn@dobegin{\begin{tikzpicture}}%
  }%
  {%
    \edef\@venn@dobegin{\noexpand\begin{tikzpicture}%
       [\expandonce\@venn@tikzoptions]}%
  }%
  \@venn@dobegin
%    \end{macrocode}
% coordinates of the Venn diagram corners
%    \begin{macrocode}
  \path (0,0) coordinate (venn bottom left)
        (0,\@venn@h) coordinate (venn top left)
        (\@venn@w,\@venn@h) coordinate (venn top right)
        (\@venn@w,0) coordinate (venn bottom right);
}%
%    \end{macrocode}
% End environment code:
%    \begin{macrocode}
{%
%    \end{macrocode}
% Draw outlines
%    \begin{macrocode}
    \draw (0,0) rectangle (\@venn@w,\@venn@h);
    \draw (\@venn@Ax,\@venn@Ay) circle (\@venn@radius);
    \draw (\@venn@Bx,\@venn@By) circle (\@venn@radius);
    \draw (\@venn@Cx,\@venn@Cy) circle (\@venn@radius);
%    \end{macrocode}
% Draw labels
%    \begin{macrocode}
    \draw (\@venn@Ax,\@venn@Ay) node[above,left] (labelOnlyA) {\@venn@label@OnlyA};
    \draw (\@venn@Bx,\@venn@By) node[above,right] (labelOnlyB) {\@venn@label@OnlyB};
    \draw (\@venn@Cx,\@venn@Cy) node[below] (labelOnlyC) {\@venn@label@OnlyC};
%    \end{macrocode}
% Region labels
%    \begin{macrocode}
    \draw (\@venn@vgap,\@venn@hgap) node (labelNotABC) {\@venn@label@NotABC};
    \draw (\@venn@Ax,\@venn@Ay+\@venn@radius)
      node[below] (labelA) {\@venn@label@A};
    \draw (\@venn@Bx,\@venn@By+\@venn@radius)
      node[below] (labelB) {\@venn@label@B};
    \draw (\@venn@Cx,\@venn@vgap) node[above] (labelC) {\@venn@label@C};
    \draw (\@venn@Cx,0.5*\@venn@h) node (labelABC) {\@venn@label@ABC};
    \draw (\@venn@Cx,\@venn@Ay) node[above] (labelOnlyAB) {\@venn@label@OnlyAB};
    \draw (\@venn@Ax,\@venn@Ay) ++(-60:\@venn@radius-0.5*\@venn@overlap)
      node[below left] (labelOnlyAC) {\@venn@label@OnlyAC};
    \draw (\@venn@Bx,\@venn@By) ++(-120:\@venn@radius-0.5*\@venn@overlap)
      node[below right] (labelOnlyBC) {\@venn@label@OnlyBC};
    \@postvennhook
  \end{tikzpicture}
}
%    \end{macrocode}
%\end{environment}
%
%\begin{macro}{\@postvennhook}
% Hook called just before the end of the \env{tikzpicture}
% environment.
%    \begin{macrocode}
\newcommand*{\@postvennhook}{}
%    \end{macrocode}
%\end{macro}
%\begin{macro}{\setpostvennhook}
% User interface to set the post hook.
%    \begin{macrocode}
\newcommand*{\setpostvennhook}[1]{\def\@postvennhook{#1}}
%    \end{macrocode}
%\end{macro}
%
%\begin{macro}{\@venn@fillOnlyA@threesets}
%\changes{1.1}{2016-03-16}{new}
%\cs{fillOnlyA} is set to this for the three set version.
% Reimplemented in v1.1 to use new circle intersect code.
%    \begin{macrocode}
\newcommand*\@venn@fillOnlyA@threesets{%
%    \end{macrocode}
% Get the intersection points between $A$ and $B$.
%    \begin{macrocode}
    \@venn@computecircleintersects{\@venn@Ax}{\@venn@Ay}{\@venn@Bx}{\@venn@By}%
%    \end{macrocode}
% If there aren't two points of intersection, then the circles don't
% overlap.
%    \begin{macrocode}
    \ifnum\@venn@intersect@n=2\relax
%    \end{macrocode}
% We need the point that's furthest from $C$.
%    \begin{macrocode}
      \pgfmathsetlength{\dimen@i}{\@venn@Cx-\@venn@intersect@i@x}%
      \pgfmathsetlength{\dimen@ii}{\@venn@Cy-\@venn@intersect@i@y}%
      \pgfmathsetlength{\dimen@}{\dimen@i*\dimen@i+\dimen@ii*\dimen@ii}%
      \pgfmathsetlength{\dimen@i}{\@venn@Cx-\@venn@intersect@ii@x}%
      \pgfmathsetlength{\dimen@ii}{\@venn@Cy-\@venn@intersect@ii@y}%
      \pgfmathsetlength{\dimen@i}{\dimen@i*\dimen@i+\dimen@ii*\dimen@ii}%
      \ifdim\dimen@>\dimen@i
        \pgfmathsetlength{\dimen@i}{\@venn@intersect@i@x-\@venn@Ax}%
        \pgfmathsetlength{\dimen@ii}{\@venn@intersect@i@y-\@venn@Ay}%
        \edef\@venn@start@pt@x{\the\@venn@intersect@i@x}%
        \edef\@venn@start@pt@y{\the\@venn@intersect@i@y}%
      \else
        \pgfmathsetlength{\dimen@i}{\@venn@intersect@ii@x-\@venn@Ax}%
        \pgfmathsetlength{\dimen@ii}{\@venn@intersect@ii@y-\@venn@Ay}%
        \edef\@venn@start@pt@x{\the\@venn@intersect@ii@x}%
        \edef\@venn@start@pt@y{\the\@venn@intersect@ii@y}%
      \fi
%    \end{macrocode}
% Compute the initial angle of the first arc.
%    \begin{macrocode}
      \ifvennoldpgf
        \pgfmathatantwo{\dimen@i}{\dimen@ii}%
      \else
        \pgfmathatantwo{\dimen@ii}{\dimen@i}%
      \fi
      \let\@venn@start@i@angle\pgfmathresult
%    \end{macrocode}
% Compute the intersection between $A$ and $C$.
%    \begin{macrocode}
      \@venn@computecircleintersects{\@venn@Ax}{\@venn@Ay}{\@venn@Cx}{\@venn@Cy}%
%    \end{macrocode}
% Need the point furthest from $B$.
%    \begin{macrocode}
      \pgfmathsetlength{\dimen@i}{\@venn@Bx-\@venn@intersect@i@x}%
      \pgfmathsetlength{\dimen@ii}{\@venn@By-\@venn@intersect@i@y}%
      \pgfmathsetlength{\dimen@}{\dimen@i*\dimen@i+\dimen@ii*\dimen@ii}%
      \pgfmathsetlength{\dimen@i}{\@venn@Bx-\@venn@intersect@ii@x}%
      \pgfmathsetlength{\dimen@ii}{\@venn@By-\@venn@intersect@ii@y}%
      \pgfmathsetlength{\dimen@i}{\dimen@i*\dimen@i+\dimen@ii*\dimen@ii}%
      \ifdim\dimen@>\dimen@i
        \pgfmathsetlength{\dimen@i}{\@venn@intersect@i@x-\@venn@Ax}%
        \pgfmathsetlength{\dimen@ii}{\@venn@intersect@i@y-\@venn@Ay}%
        \edef\@venn@end@pt@i@x{\the\@venn@intersect@i@x}%
        \edef\@venn@end@pt@i@y{\the\@venn@intersect@i@y}%
      \else
        \pgfmathsetlength{\dimen@i}{\@venn@intersect@ii@x-\@venn@Ax}%
        \pgfmathsetlength{\dimen@ii}{\@venn@intersect@ii@y-\@venn@Ay}%
        \edef\@venn@end@pt@i@x{\the\@venn@intersect@ii@x}%
        \edef\@venn@end@pt@i@y{\the\@venn@intersect@ii@y}%
      \fi
%    \end{macrocode}
% Compute the end angle of the first arc.
%    \begin{macrocode}
      \ifvennoldpgf
        \pgfmathatantwo{\dimen@i}{\dimen@ii}%
      \else
        \pgfmathatantwo{\dimen@ii}{\dimen@i}%
      \fi
      \let\@venn@end@i@angle\pgfmathresult
      \pgfmathadd{\@venn@end@i@angle}{360}%
      \let\@venn@end@i@angle\pgfmathresult
%    \end{macrocode}
% Compute the start angle of the second arc.
%    \begin{macrocode}
      \pgfmathsetlength{\dimen@i}{\@venn@end@pt@i@x-\@venn@Cx}%
      \pgfmathsetlength{\dimen@ii}{\@venn@end@pt@i@y-\@venn@Cy}%
      \ifvennoldpgf
        \pgfmathatantwo{\dimen@i}{\dimen@ii}%
      \else
        \pgfmathatantwo{\dimen@ii}{\dimen@i}%
      \fi
      \let\@venn@start@ii@angle\pgfmathresult
%    \end{macrocode}
% Compute the intersection between $B$ and $C$.
%    \begin{macrocode}
      \@venn@computecircleintersects{\@venn@Bx}{\@venn@By}{\@venn@Cx}{\@venn@Cy}%
%    \end{macrocode}
% Need the point closest to $A$.
%    \begin{macrocode}
      \pgfmathsetlength{\dimen@i}{\@venn@Ax-\@venn@intersect@i@x}%
      \pgfmathsetlength{\dimen@ii}{\@venn@Ay-\@venn@intersect@i@y}%
      \pgfmathsetlength{\dimen@}{\dimen@i*\dimen@i+\dimen@ii*\dimen@ii}%
      \pgfmathsetlength{\dimen@i}{\@venn@Ax-\@venn@intersect@ii@x}%
      \pgfmathsetlength{\dimen@ii}{\@venn@Ay-\@venn@intersect@ii@y}%
      \pgfmathsetlength{\dimen@i}{\dimen@i*\dimen@i+\dimen@ii*\dimen@ii}%
      \ifdim\dimen@<\dimen@i
        \pgfmathsetlength{\dimen@i}{\@venn@intersect@i@x-\@venn@Cx}%
        \pgfmathsetlength{\dimen@ii}{\@venn@intersect@i@y-\@venn@Cy}%
        \edef\@venn@end@pt@ii@x{\the\@venn@intersect@i@x}%
        \edef\@venn@end@pt@ii@y{\the\@venn@intersect@i@y}%
      \else
        \pgfmathsetlength{\dimen@i}{\@venn@intersect@ii@x-\@venn@Cx}%
        \pgfmathsetlength{\dimen@ii}{\@venn@intersect@ii@y-\@venn@Cy}%
        \edef\@venn@end@pt@ii@x{\the\@venn@intersect@ii@x}%
        \edef\@venn@end@pt@ii@y{\the\@venn@intersect@ii@y}%
      \fi
%    \end{macrocode}
% Compute the end angle of the second arc.
%    \begin{macrocode}
      \ifvennoldpgf
        \pgfmathatantwo{\dimen@i}{\dimen@ii}%
      \else
        \pgfmathatantwo{\dimen@ii}{\dimen@i}%
      \fi
      \let\@venn@end@ii@angle\pgfmathresult
%    \end{macrocode}
% Compute the start angle of the third arc.
%    \begin{macrocode}
      \pgfmathsetlength{\dimen@i}{\@venn@end@pt@ii@x-\@venn@Bx}%
      \pgfmathsetlength{\dimen@ii}{\@venn@end@pt@ii@y-\@venn@By}%
      \ifvennoldpgf
        \pgfmathatantwo{\dimen@i}{\dimen@ii}%
      \else
        \pgfmathatantwo{\dimen@ii}{\dimen@i}%
      \fi
      \let\@venn@start@iii@angle\pgfmathresult
%    \end{macrocode}
% Compute the end angle of the third arc.
%    \begin{macrocode}
      \pgfmathsetlength{\dimen@i}{\@venn@start@pt@x-\@venn@Bx}%
      \pgfmathsetlength{\dimen@ii}{\@venn@start@pt@y-\@venn@By}%
      \ifvennoldpgf
        \pgfmathatantwo{\dimen@i}{\dimen@ii}%
      \else
        \pgfmathatantwo{\dimen@ii}{\dimen@i}%
      \fi
      \let\@venn@end@iii@angle\pgfmathresult
      \pgfmathsubtract{\@venn@end@iii@angle}{360}%
      \let\@venn@end@iii@angle\pgfmathresult
%    \end{macrocode}
% Fill path
%    \begin{macrocode}
      \path[fill=\@venn@shade] (\@venn@start@pt@x,\@venn@start@pt@y) 
      arc[radius=\@venn@radius,start angle=\@venn@start@i@angle,
      end angle=\@venn@end@i@angle]
      arc[radius=\@venn@radius,start angle=\@venn@start@ii@angle,
      end angle=\@venn@end@ii@angle]
      arc[radius=\@venn@radius,start angle=\@venn@start@iii@angle,
      end angle=\@venn@end@iii@angle] -- cycle;
    \else
       \fillOnlyA
    \fi
}%
%    \end{macrocode}
%\end{macro}
%
%\begin{macro}{\@venn@fillOnlyB@threesets}
%\changes{1.1}{2016-03-16}{new}
%\cs{fillOnlyB} is set to this for the three set version.
% Reimplemented in v1.1 to use new circle intersect code.
%    \begin{macrocode}
\newcommand*\@venn@fillOnlyB@threesets{%
%    \end{macrocode}
% Get the intersection points between $B$ and $A$.
%    \begin{macrocode}
    \@venn@computecircleintersects{\@venn@Bx}{\@venn@By}{\@venn@Ax}{\@venn@Ay}%
%    \end{macrocode}
% If there aren't two points of intersection, then the circles don't
% overlap.
%    \begin{macrocode}
    \ifnum\@venn@intersect@n=2\relax
%    \end{macrocode}
% We need the point that's furthest from $C$.
%    \begin{macrocode}
      \pgfmathsetlength{\dimen@i}{\@venn@Cx-\@venn@intersect@i@x}%
      \pgfmathsetlength{\dimen@ii}{\@venn@Cy-\@venn@intersect@i@y}%
      \pgfmathsetlength{\dimen@}{\dimen@i*\dimen@i+\dimen@ii*\dimen@ii}%
      \pgfmathsetlength{\dimen@i}{\@venn@Cx-\@venn@intersect@ii@x}%
      \pgfmathsetlength{\dimen@ii}{\@venn@Cy-\@venn@intersect@ii@y}%
      \pgfmathsetlength{\dimen@i}{\dimen@i*\dimen@i+\dimen@ii*\dimen@ii}%
      \ifdim\dimen@>\dimen@i
        \pgfmathsetlength{\dimen@i}{\@venn@intersect@i@x-\@venn@Bx}%
        \pgfmathsetlength{\dimen@ii}{\@venn@intersect@i@y-\@venn@By}%
        \edef\@venn@start@pt@x{\the\@venn@intersect@i@x}%
        \edef\@venn@start@pt@y{\the\@venn@intersect@i@y}%
      \else
        \pgfmathsetlength{\dimen@i}{\@venn@intersect@ii@x-\@venn@Bx}%
        \pgfmathsetlength{\dimen@ii}{\@venn@intersect@ii@y-\@venn@By}%
        \edef\@venn@start@pt@x{\the\@venn@intersect@ii@x}%
        \edef\@venn@start@pt@y{\the\@venn@intersect@ii@y}%
      \fi
%    \end{macrocode}
% Compute the initial angle of the first arc.
%    \begin{macrocode}
      \ifvennoldpgf
        \pgfmathatantwo{\dimen@i}{\dimen@ii}%
      \else
        \pgfmathatantwo{\dimen@ii}{\dimen@i}%
      \fi
      \let\@venn@start@i@angle\pgfmathresult
%    \end{macrocode}
% Compute the intersection between $B$ and $C$.
%    \begin{macrocode}
      \@venn@computecircleintersects{\@venn@Bx}{\@venn@By}{\@venn@Cx}{\@venn@Cy}%
%    \end{macrocode}
% Need the point furthest from $A$.
%    \begin{macrocode}
      \pgfmathsetlength{\dimen@i}{\@venn@Ax-\@venn@intersect@i@x}%
      \pgfmathsetlength{\dimen@ii}{\@venn@Ay-\@venn@intersect@i@y}%
      \pgfmathsetlength{\dimen@}{\dimen@i*\dimen@i+\dimen@ii*\dimen@ii}%
      \pgfmathsetlength{\dimen@i}{\@venn@Ax-\@venn@intersect@ii@x}%
      \pgfmathsetlength{\dimen@ii}{\@venn@Ay-\@venn@intersect@ii@y}%
      \pgfmathsetlength{\dimen@i}{\dimen@i*\dimen@i+\dimen@ii*\dimen@ii}%
      \ifdim\dimen@>\dimen@i
        \pgfmathsetlength{\dimen@i}{\@venn@intersect@i@x-\@venn@Bx}%
        \pgfmathsetlength{\dimen@ii}{\@venn@intersect@i@y-\@venn@By}%
        \edef\@venn@end@pt@i@x{\the\@venn@intersect@i@x}%
        \edef\@venn@end@pt@i@y{\the\@venn@intersect@i@y}%
      \else
        \pgfmathsetlength{\dimen@i}{\@venn@intersect@ii@x-\@venn@Bx}%
        \pgfmathsetlength{\dimen@ii}{\@venn@intersect@ii@y-\@venn@By}%
        \edef\@venn@end@pt@i@x{\the\@venn@intersect@ii@x}%
        \edef\@venn@end@pt@i@y{\the\@venn@intersect@ii@y}%
      \fi
%    \end{macrocode}
% Compute the end angle of the first arc.
%    \begin{macrocode}
      \ifvennoldpgf
        \pgfmathatantwo{\dimen@i}{\dimen@ii}%
      \else
        \pgfmathatantwo{\dimen@ii}{\dimen@i}%
      \fi
      \let\@venn@end@i@angle\pgfmathresult
%    \end{macrocode}
% Compute the start angle of the second arc.
%    \begin{macrocode}
      \pgfmathsetlength{\dimen@i}{\@venn@end@pt@i@x-\@venn@Cx}%
      \pgfmathsetlength{\dimen@ii}{\@venn@end@pt@i@y-\@venn@Cy}%
      \ifvennoldpgf
        \pgfmathatantwo{\dimen@i}{\dimen@ii}%
      \else
        \pgfmathatantwo{\dimen@ii}{\dimen@i}%
      \fi
      \let\@venn@start@ii@angle\pgfmathresult
%    \end{macrocode}
% Compute the intersection between $A$ and $C$.
%    \begin{macrocode}
      \@venn@computecircleintersects{\@venn@Ax}{\@venn@Ay}{\@venn@Cx}{\@venn@Cy}%
%    \end{macrocode}
% Need the point closest to $A$.
%    \begin{macrocode}
      \pgfmathsetlength{\dimen@i}{\@venn@Bx-\@venn@intersect@i@x}%
      \pgfmathsetlength{\dimen@ii}{\@venn@By-\@venn@intersect@i@y}%
      \pgfmathsetlength{\dimen@}{\dimen@i*\dimen@i+\dimen@ii*\dimen@ii}%
      \pgfmathsetlength{\dimen@i}{\@venn@Bx-\@venn@intersect@ii@x}%
      \pgfmathsetlength{\dimen@ii}{\@venn@By-\@venn@intersect@ii@y}%
      \pgfmathsetlength{\dimen@i}{\dimen@i*\dimen@i+\dimen@ii*\dimen@ii}%
      \ifdim\dimen@<\dimen@i
        \pgfmathsetlength{\dimen@i}{\@venn@intersect@i@x-\@venn@Cx}%
        \pgfmathsetlength{\dimen@ii}{\@venn@intersect@i@y-\@venn@Cy}%
        \edef\@venn@end@pt@ii@x{\the\@venn@intersect@i@x}%
        \edef\@venn@end@pt@ii@y{\the\@venn@intersect@i@y}%
      \else
        \pgfmathsetlength{\dimen@i}{\@venn@intersect@ii@x-\@venn@Cx}%
        \pgfmathsetlength{\dimen@ii}{\@venn@intersect@ii@y-\@venn@Cy}%
        \edef\@venn@end@pt@ii@x{\the\@venn@intersect@ii@x}%
        \edef\@venn@end@pt@ii@y{\the\@venn@intersect@ii@y}%
      \fi
%    \end{macrocode}
% Compute the end angle of the second arc.
%    \begin{macrocode}
      \ifvennoldpgf
        \pgfmathatantwo{\dimen@i}{\dimen@ii}%
      \else
        \pgfmathatantwo{\dimen@ii}{\dimen@i}%
      \fi
      \let\@venn@end@ii@angle\pgfmathresult
%    \end{macrocode}
% Compute the start angle of the third arc.
%    \begin{macrocode}
      \pgfmathsetlength{\dimen@i}{\@venn@end@pt@ii@x-\@venn@Ax}%
      \pgfmathsetlength{\dimen@ii}{\@venn@end@pt@ii@y-\@venn@Ay}%
      \ifvennoldpgf
        \pgfmathatantwo{\dimen@i}{\dimen@ii}%
      \else
        \pgfmathatantwo{\dimen@ii}{\dimen@i}%
      \fi
      \let\@venn@start@iii@angle\pgfmathresult
%    \end{macrocode}
% Compute the end angle of the third arc.
%    \begin{macrocode}
      \pgfmathsetlength{\dimen@i}{\@venn@start@pt@x-\@venn@Ax}%
      \pgfmathsetlength{\dimen@ii}{\@venn@start@pt@y-\@venn@Ay}%
      \ifvennoldpgf
        \pgfmathatantwo{\dimen@i}{\dimen@ii}%
      \else
        \pgfmathatantwo{\dimen@ii}{\dimen@i}%
      \fi
      \let\@venn@end@iii@angle\pgfmathresult
%    \end{macrocode}
% Fill path
%    \begin{macrocode}
      \path[fill=\@venn@shade] 
      (\@venn@start@pt@x,\@venn@start@pt@y) 
      arc[radius=\@venn@radius,start angle=\@venn@start@i@angle,
      end angle=\@venn@end@i@angle]
      arc[radius=\@venn@radius,start angle=\@venn@start@ii@angle,
      end angle=\@venn@end@ii@angle]
      arc[radius=\@venn@radius,start angle=\@venn@start@iii@angle,
      end angle=\@venn@end@iii@angle] -- cycle;
    \else
       \fillOnlyB
    \fi
}%
%    \end{macrocode}
%\end{macro}
%
%\begin{macro}{\@venn@fillOnlyC@threesets}
%\changes{1.1}{2016-03-16}{new}
%\cs{fillOnlyC} is set to this for the three set version.
% Reimplemented in v1.1 to use new circle intersect code.
%    \begin{macrocode}
\newcommand*\@venn@fillOnlyC@threesets{%
%    \end{macrocode}
% Get the intersection points between $C$ and $B$.
%    \begin{macrocode}
    \@venn@computecircleintersects{\@venn@Cx}{\@venn@Cy}{\@venn@Bx}{\@venn@By}%
%    \end{macrocode}
% If there aren't two points of intersection, then the circles don't
% overlap.
%    \begin{macrocode}
    \ifnum\@venn@intersect@n=2\relax
%    \end{macrocode}
% We need the point that's furthest from $A$.
%    \begin{macrocode}
      \pgfmathsetlength{\dimen@i}{\@venn@Ax-\@venn@intersect@i@x}%
      \pgfmathsetlength{\dimen@ii}{\@venn@Ay-\@venn@intersect@i@y}%
      \pgfmathsetlength{\dimen@}{\dimen@i*\dimen@i+\dimen@ii*\dimen@ii}%
      \pgfmathsetlength{\dimen@i}{\@venn@Ax-\@venn@intersect@ii@x}%
      \pgfmathsetlength{\dimen@ii}{\@venn@Ay-\@venn@intersect@ii@y}%
      \pgfmathsetlength{\dimen@i}{\dimen@i*\dimen@i+\dimen@ii*\dimen@ii}%
      \ifdim\dimen@>\dimen@i
        \pgfmathsetlength{\dimen@i}{\@venn@intersect@i@x-\@venn@Cx}%
        \pgfmathsetlength{\dimen@ii}{\@venn@intersect@i@y-\@venn@Cy}%
        \edef\@venn@start@pt@x{\the\@venn@intersect@i@x}%
        \edef\@venn@start@pt@y{\the\@venn@intersect@i@y}%
      \else
        \pgfmathsetlength{\dimen@i}{\@venn@intersect@ii@x-\@venn@Cx}%
        \pgfmathsetlength{\dimen@ii}{\@venn@intersect@ii@y-\@venn@Cy}%
        \edef\@venn@start@pt@x{\the\@venn@intersect@ii@x}%
        \edef\@venn@start@pt@y{\the\@venn@intersect@ii@y}%
      \fi
%    \end{macrocode}
% Compute the initial angle of the first arc.
%    \begin{macrocode}
      \ifvennoldpgf
        \pgfmathatantwo{\dimen@i}{\dimen@ii}%
      \else
        \pgfmathatantwo{\dimen@ii}{\dimen@i}%
      \fi
      \let\@venn@start@i@angle\pgfmathresult
%    \end{macrocode}
% Compute the intersection between $C$ and $A$.
%    \begin{macrocode}
      \@venn@computecircleintersects{\@venn@Cx}{\@venn@Cy}{\@venn@Ax}{\@venn@Ay}%
%    \end{macrocode}
% Need the point furthest from $B$.
%    \begin{macrocode}
      \pgfmathsetlength{\dimen@i}{\@venn@Bx-\@venn@intersect@i@x}%
      \pgfmathsetlength{\dimen@ii}{\@venn@By-\@venn@intersect@i@y}%
      \pgfmathsetlength{\dimen@}{\dimen@i*\dimen@i+\dimen@ii*\dimen@ii}%
      \pgfmathsetlength{\dimen@i}{\@venn@Bx-\@venn@intersect@ii@x}%
      \pgfmathsetlength{\dimen@ii}{\@venn@By-\@venn@intersect@ii@y}%
      \pgfmathsetlength{\dimen@i}{\dimen@i*\dimen@i+\dimen@ii*\dimen@ii}%
      \ifdim\dimen@>\dimen@i
        \pgfmathsetlength{\dimen@i}{\@venn@intersect@i@x-\@venn@Cx}%
        \pgfmathsetlength{\dimen@ii}{\@venn@intersect@i@y-\@venn@Cy}%
        \edef\@venn@end@pt@i@x{\the\@venn@intersect@i@x}%
        \edef\@venn@end@pt@i@y{\the\@venn@intersect@i@y}%
      \else
        \pgfmathsetlength{\dimen@i}{\@venn@intersect@ii@x-\@venn@Cx}%
        \pgfmathsetlength{\dimen@ii}{\@venn@intersect@ii@y-\@venn@Cy}%
        \edef\@venn@end@pt@i@x{\the\@venn@intersect@ii@x}%
        \edef\@venn@end@pt@i@y{\the\@venn@intersect@ii@y}%
      \fi
%    \end{macrocode}
% Compute the end angle of the first arc.
%    \begin{macrocode}
      \ifvennoldpgf
        \pgfmathatantwo{\dimen@i}{\dimen@ii}%
      \else
        \pgfmathatantwo{\dimen@ii}{\dimen@i}%
      \fi
      \let\@venn@end@i@angle\pgfmathresult
      \pgfmathsubtract{\@venn@end@i@angle}{360}%
      \let\@venn@end@i@angle\pgfmathresult
%    \end{macrocode}
% Compute the start angle of the second arc.
%    \begin{macrocode}
      \pgfmathsetlength{\dimen@i}{\@venn@end@pt@i@x-\@venn@Ax}%
      \pgfmathsetlength{\dimen@ii}{\@venn@end@pt@i@y-\@venn@Ay}%
      \ifvennoldpgf
        \pgfmathatantwo{\dimen@i}{\dimen@ii}%
      \else
        \pgfmathatantwo{\dimen@ii}{\dimen@i}%
      \fi
      \let\@venn@start@ii@angle\pgfmathresult
%    \end{macrocode}
% Compute the intersection between $B$ and $A$.
%    \begin{macrocode}
      \@venn@computecircleintersects{\@venn@Bx}{\@venn@By}{\@venn@Ax}{\@venn@Ay}%
%    \end{macrocode}
% Need the point closest to $C$.
%    \begin{macrocode}
      \pgfmathsetlength{\dimen@i}{\@venn@Cx-\@venn@intersect@i@x}%
      \pgfmathsetlength{\dimen@ii}{\@venn@Cy-\@venn@intersect@i@y}%
      \pgfmathsetlength{\dimen@}{\dimen@i*\dimen@i+\dimen@ii*\dimen@ii}%
      \pgfmathsetlength{\dimen@i}{\@venn@Cx-\@venn@intersect@ii@x}%
      \pgfmathsetlength{\dimen@ii}{\@venn@Cy-\@venn@intersect@ii@y}%
      \pgfmathsetlength{\dimen@i}{\dimen@i*\dimen@i+\dimen@ii*\dimen@ii}%
      \ifdim\dimen@<\dimen@i
        \pgfmathsetlength{\dimen@i}{\@venn@intersect@i@x-\@venn@Ax}%
        \pgfmathsetlength{\dimen@ii}{\@venn@intersect@i@y-\@venn@Ay}%
        \edef\@venn@end@pt@ii@x{\the\@venn@intersect@i@x}%
        \edef\@venn@end@pt@ii@y{\the\@venn@intersect@i@y}%
      \else
        \pgfmathsetlength{\dimen@i}{\@venn@intersect@ii@x-\@venn@Ax}%
        \pgfmathsetlength{\dimen@ii}{\@venn@intersect@ii@y-\@venn@Ay}%
        \edef\@venn@end@pt@ii@x{\the\@venn@intersect@ii@x}%
        \edef\@venn@end@pt@ii@y{\the\@venn@intersect@ii@y}%
      \fi
%    \end{macrocode}
% Compute the end angle of the second arc.
%    \begin{macrocode}
      \ifvennoldpgf
        \pgfmathatantwo{\dimen@i}{\dimen@ii}%
      \else
        \pgfmathatantwo{\dimen@ii}{\dimen@i}%
      \fi
      \let\@venn@end@ii@angle\pgfmathresult
%    \end{macrocode}
% Compute the start angle of the third arc.
%    \begin{macrocode}
      \pgfmathsetlength{\dimen@i}{\@venn@end@pt@ii@x-\@venn@Bx}%
      \pgfmathsetlength{\dimen@ii}{\@venn@end@pt@ii@y-\@venn@By}%
      \ifvennoldpgf
        \pgfmathatantwo{\dimen@i}{\dimen@ii}%
      \else
        \pgfmathatantwo{\dimen@ii}{\dimen@i}%
      \fi
      \let\@venn@start@iii@angle\pgfmathresult
%    \end{macrocode}
% Compute the end angle of the third arc.
%    \begin{macrocode}
      \pgfmathsetlength{\dimen@i}{\@venn@start@pt@x-\@venn@Bx}%
      \pgfmathsetlength{\dimen@ii}{\@venn@start@pt@y-\@venn@By}%
      \ifvennoldpgf
        \pgfmathatantwo{\dimen@i}{\dimen@ii}%
      \else
        \pgfmathatantwo{\dimen@ii}{\dimen@i}%
      \fi
      \let\@venn@end@iii@angle\pgfmathresult
%    \end{macrocode}
% Fill path
%    \begin{macrocode}
      \path[fill=\@venn@shade] (\@venn@start@pt@x,\@venn@start@pt@y) 
      arc[radius=\@venn@radius,start angle=\@venn@start@i@angle,
      end angle=\@venn@end@i@angle]
      arc[radius=\@venn@radius,start angle=\@venn@start@ii@angle,
      end angle=\@venn@end@ii@angle]
      arc[radius=\@venn@radius,start angle=\@venn@start@iii@angle,
      end angle=\@venn@end@iii@angle] -- cycle;
    \else
       \fillOnlyC
    \fi
}%
%    \end{macrocode}
%\end{macro}
%
%\begin{macro}{\@venn@fillNotABC@threesets}
%\changes{1.1}{2016-03-16}{new}
%\cs{fillNotABC} is set to this for the three set version.
% Reimplemented in v1.1 to use new circle intersect code.
%    \begin{macrocode}
\newcommand*\@venn@fillNotABC@threesets{%
%    \end{macrocode}
% Get the intersection points between $A$ and $B$.
%    \begin{macrocode}
    \@venn@computecircleintersects{\@venn@Ax}{\@venn@Ay}{\@venn@Bx}{\@venn@By}%
%    \end{macrocode}
% If there aren't two points of intersection, then the circles don't
% overlap.
%    \begin{macrocode}
    \ifnum\@venn@intersect@n=2\relax
%    \end{macrocode}
% We need the point that's furthest from $C$.
%    \begin{macrocode}
      \pgfmathsetlength{\dimen@i}{\@venn@Cx-\@venn@intersect@i@x}%
      \pgfmathsetlength{\dimen@ii}{\@venn@Cy-\@venn@intersect@i@y}%
      \pgfmathsetlength{\dimen@}{\dimen@i*\dimen@i+\dimen@ii*\dimen@ii}%
      \pgfmathsetlength{\dimen@i}{\@venn@Cx-\@venn@intersect@ii@x}%
      \pgfmathsetlength{\dimen@ii}{\@venn@Cy-\@venn@intersect@ii@y}%
      \pgfmathsetlength{\dimen@i}{\dimen@i*\dimen@i+\dimen@ii*\dimen@ii}%
      \ifdim\dimen@>\dimen@i
        \pgfmathsetlength{\dimen@i}{\@venn@intersect@i@x-\@venn@Ax}%
        \pgfmathsetlength{\dimen@ii}{\@venn@intersect@i@y-\@venn@Ay}%
        \edef\@venn@start@pt@x{\the\@venn@intersect@i@x}%
        \edef\@venn@start@pt@y{\the\@venn@intersect@i@y}%
      \else
        \pgfmathsetlength{\dimen@i}{\@venn@intersect@ii@x-\@venn@Ax}%
        \pgfmathsetlength{\dimen@ii}{\@venn@intersect@ii@y-\@venn@Ay}%
        \edef\@venn@start@pt@x{\the\@venn@intersect@ii@x}%
        \edef\@venn@start@pt@y{\the\@venn@intersect@ii@y}%
      \fi
%    \end{macrocode}
% Compute the initial angle of the first arc.
%    \begin{macrocode}
      \ifvennoldpgf
        \pgfmathatantwo{\dimen@i}{\dimen@ii}%
      \else
        \pgfmathatantwo{\dimen@ii}{\dimen@i}%
      \fi
      \let\@venn@start@i@angle\pgfmathresult
%    \end{macrocode}
% Compute the intersection between $A$ and $C$.
%    \begin{macrocode}
      \@venn@computecircleintersects{\@venn@Ax}{\@venn@Ay}{\@venn@Cx}{\@venn@Cy}%
%    \end{macrocode}
% We need the point that's furthest from $B$.
%    \begin{macrocode}
      \pgfmathsetlength{\dimen@i}{\@venn@Bx-\@venn@intersect@i@x}%
      \pgfmathsetlength{\dimen@ii}{\@venn@By-\@venn@intersect@i@y}%
      \pgfmathsetlength{\dimen@}{\dimen@i*\dimen@i+\dimen@ii*\dimen@ii}%
      \pgfmathsetlength{\dimen@i}{\@venn@Bx-\@venn@intersect@ii@x}%
      \pgfmathsetlength{\dimen@ii}{\@venn@By-\@venn@intersect@ii@y}%
      \pgfmathsetlength{\dimen@i}{\dimen@i*\dimen@i+\dimen@ii*\dimen@ii}%
      \ifdim\dimen@>\dimen@i
        \pgfmathsetlength{\dimen@i}{\@venn@intersect@i@x-\@venn@Ax}%
        \pgfmathsetlength{\dimen@ii}{\@venn@intersect@i@y-\@venn@Ay}%
        \edef\@venn@end@pt@i@x{\the\@venn@intersect@i@x}%
        \edef\@venn@end@pt@i@y{\the\@venn@intersect@i@y}%
      \else
        \pgfmathsetlength{\dimen@i}{\@venn@intersect@ii@x-\@venn@Ax}%
        \pgfmathsetlength{\dimen@ii}{\@venn@intersect@ii@y-\@venn@Ay}%
        \edef\@venn@end@pt@i@x{\the\@venn@intersect@ii@x}%
        \edef\@venn@end@pt@i@y{\the\@venn@intersect@ii@y}%
      \fi
%    \end{macrocode}
% Compute the end angle of the first arc.
%    \begin{macrocode}
      \ifvennoldpgf
        \pgfmathatantwo{\dimen@i}{\dimen@ii}%
      \else
        \pgfmathatantwo{\dimen@ii}{\dimen@i}%
      \fi
      \let\@venn@end@i@angle\pgfmathresult
      \pgfmathadd{\@venn@end@i@angle}{360}%
      \let\@venn@end@i@angle\pgfmathresult
%    \end{macrocode}
% Compute the start angle of the second arc.
%    \begin{macrocode}
      \pgfmathsetlength{\dimen@i}{\@venn@end@pt@i@x-\@venn@Cx}%
      \pgfmathsetlength{\dimen@ii}{\@venn@end@pt@i@y-\@venn@Cy}%
      \ifvennoldpgf
        \pgfmathatantwo{\dimen@i}{\dimen@ii}%
      \else
        \pgfmathatantwo{\dimen@ii}{\dimen@i}%
      \fi
      \let\@venn@start@ii@angle\pgfmathresult
%    \end{macrocode}
% Compute the intersection between $C$ and $B$.
%    \begin{macrocode}
      \@venn@computecircleintersects{\@venn@Cx}{\@venn@Cy}{\@venn@Bx}{\@venn@By}%
%    \end{macrocode}
% We need the point that's furthest from $A$.
%    \begin{macrocode}
      \pgfmathsetlength{\dimen@i}{\@venn@Ax-\@venn@intersect@i@x}%
      \pgfmathsetlength{\dimen@ii}{\@venn@Ay-\@venn@intersect@i@y}%
      \pgfmathsetlength{\dimen@}{\dimen@i*\dimen@i+\dimen@ii*\dimen@ii}%
      \pgfmathsetlength{\dimen@i}{\@venn@Ax-\@venn@intersect@ii@x}%
      \pgfmathsetlength{\dimen@ii}{\@venn@Ay-\@venn@intersect@ii@y}%
      \pgfmathsetlength{\dimen@i}{\dimen@i*\dimen@i+\dimen@ii*\dimen@ii}%
      \ifdim\dimen@>\dimen@i
        \pgfmathsetlength{\dimen@i}{\@venn@intersect@i@x-\@venn@Cx}%
        \pgfmathsetlength{\dimen@ii}{\@venn@intersect@i@y-\@venn@Cy}%
        \edef\@venn@end@pt@ii@x{\the\@venn@intersect@i@x}%
        \edef\@venn@end@pt@ii@y{\the\@venn@intersect@i@y}%
      \else
        \pgfmathsetlength{\dimen@i}{\@venn@intersect@ii@x-\@venn@Cx}%
        \pgfmathsetlength{\dimen@ii}{\@venn@intersect@ii@y-\@venn@Cy}%
        \edef\@venn@end@pt@ii@x{\the\@venn@intersect@ii@x}%
        \edef\@venn@end@pt@ii@y{\the\@venn@intersect@ii@y}%
      \fi
%    \end{macrocode}
% Compute the end angle of the second arc.
%    \begin{macrocode}
      \ifvennoldpgf
        \pgfmathatantwo{\dimen@i}{\dimen@ii}%
      \else
        \pgfmathatantwo{\dimen@ii}{\dimen@i}%
      \fi
      \let\@venn@end@ii@angle\pgfmathresult
      \pgfmathadd{\@venn@end@ii@angle}{360}%
      \let\@venn@end@ii@angle\pgfmathresult
%    \end{macrocode}
% Compute the start angle of the third arc.
%    \begin{macrocode}
      \pgfmathsetlength{\dimen@i}{\@venn@end@pt@ii@x-\@venn@Bx}%
      \pgfmathsetlength{\dimen@ii}{\@venn@end@pt@ii@y-\@venn@By}%
      \ifvennoldpgf
        \pgfmathatantwo{\dimen@i}{\dimen@ii}%
      \else
        \pgfmathatantwo{\dimen@ii}{\dimen@i}%
      \fi
      \let\@venn@start@iii@angle\pgfmathresult
%    \end{macrocode}
% Compute the end angle of the third arc.
%    \begin{macrocode}
      \pgfmathsetlength{\dimen@i}{\@venn@start@pt@x-\@venn@Bx}%
      \pgfmathsetlength{\dimen@ii}{\@venn@start@pt@y-\@venn@By}%
      \ifvennoldpgf
        \pgfmathatantwo{\dimen@i}{\dimen@ii}%
      \else
        \pgfmathatantwo{\dimen@ii}{\dimen@i}%
      \fi
      \let\@venn@end@iii@angle\pgfmathresult
%    \end{macrocode}
% Fill path
%    \begin{macrocode}
      \path[fill=\@venn@shade,even odd rule]
        (0,0) rectangle (\@venn@w,\@venn@h)
       (\@venn@start@pt@x,\@venn@start@pt@y) 
      arc[radius=\@venn@radius,start angle=\@venn@start@i@angle,
      end angle=\@venn@end@i@angle]
      arc[radius=\@venn@radius,start angle=\@venn@start@ii@angle,
      end angle=\@venn@end@ii@angle]
      arc[radius=\@venn@radius,start angle=\@venn@start@iii@angle,
      end angle=\@venn@end@iii@angle]
      -- cycle;
    \else
      \path[fill=\@venn@shade,even odd rule]
        (0,0) rectangle (\@venn@w,\@venn@h)
       (\@venn@Ax,\@venn@Ay) circle (\@venn@radius)
       (\@venn@Bx,\@venn@By) circle (\@venn@radius)
       (\@venn@Cx,\@venn@Cy) circle (\@venn@radius);%
    \fi
}%
%    \end{macrocode}
%\end{macro}
%
%\begin{environment}{venndiagram2sets}
%    \begin{macrocode}
\newenvironment{venndiagram2sets}[1][]%
{%
%    \end{macrocode}
% Disable the keys that aren't applicable.
%    \begin{macrocode}
  \disable@keys{venn}{labelABC,labelOnlyC,labelOnlyAC,labelOnlyBC,%
    labelNotABC,labelC,labelOnlyAB}%
%    \end{macrocode}
% Set the key values given in the optional argument.
%    \begin{macrocode}
  \setkeys{venn}{#1}%
%    \end{macrocode}
  % Calculate centre of A
%    \begin{macrocode}
  \pgfmathsetlength{\@venn@Ax}{\@venn@hgap+\@venn@radius}%
  \pgfmathsetlength{\@venn@Ay}{\@venn@vgap+\@venn@radius}%
%    \end{macrocode}
% Calculate centre of B
%    \begin{macrocode}
  \pgfmathsetlength{\@venn@Bx}{\@venn@hgap+3*\@venn@radius
    -\@venn@overlap}%
  \setlength{\@venn@By}{\@venn@Ay}%
%    \end{macrocode}
% Compute dimensions of entire diagram
%    \begin{macrocode}
  \pgfmathsetlength{\@venn@w}{2*\@venn@hgap+4*\@venn@radius
     -\@venn@overlap}%
  \pgfmathsetlength{\@venn@h}{2*\@venn@vgap+2*\@venn@radius}%
%    \end{macrocode}
% Define filling commands
%    \begin{macrocode}
  \def\fillA{\path[fill=\@venn@shade] (\@venn@Ax,\@venn@Ay)
    circle (\@venn@radius);}%
  \def\fillB{\path[fill=\@venn@shade] (\@venn@Bx,\@venn@By)
    circle (\@venn@radius);}%
  \def\fillAll{\path[fill=\@venn@shade] (0,0)
    rectangle (\@venn@w,\@venn@h);}%
%    \end{macrocode}
% Fill only set $A$
%    \begin{macrocode}
  \let\fillOnlyA\@venn@fillOnlyA@twosets
%    \end{macrocode}
% Fill only set $B$
%    \begin{macrocode}
  \let\fillOnlyB\@venn@fillOnlyB@twosets
%    \end{macrocode}
% Fill everything except $A$
%    \begin{macrocode}
  \def\fillNotA{\path[fill=\@venn@shade,even odd rule]
    (0,0) rectangle (\@venn@w,\@venn@h)
    (\@venn@Ax,\@venn@Ay) circle (\@venn@radius);}%
%    \end{macrocode}
% Fill everything except $B$
%    \begin{macrocode}
  \def\fillNotB{\path[fill=\@venn@shade,even odd rule]
    (0,0) rectangle (\@venn@w,\@venn@h)
    (\@venn@Bx,\@venn@By) circle (\@venn@radius);}%
%    \end{macrocode}
% Fill everything except $A$ or $B$ ($(A\cup B)^c$)
%    \begin{macrocode}
  \def\fillNotAorB{%
    \begin{scope}
    \path[clip]
    (0,0) rectangle (\@venn@w,\@venn@h)
    (\@venn@Bx,\@venn@By) circle (\@venn@radius)
     ;
    \path[fill=\@venn@shade,even odd rule]
    (0,0) rectangle (\@venn@w,\@venn@h)
    (\@venn@Ax,\@venn@Ay) circle (\@venn@radius)
    ;
    \end{scope}
   }%
%    \end{macrocode}
% Fill not $A$ or not $B$ ($(A\cap B)^c$)
%    \begin{macrocode}
  \def\fillNotAorNotB{%
    \path[fill=\@venn@shade,nonzero rule]
    (0,0) rectangle (\@venn@w,\@venn@h)
    (\@venn@Ax,\@venn@Ay) circle (\@venn@radius)
    (0,0) rectangle (\@venn@w,\@venn@h)
    (\@venn@Bx,\@venn@By) circle (\@venn@radius)
    ;
   }%
%    \end{macrocode}
% Fill $A$ but not $B$ (same as only $A$ for two sets).
%\changes{1.1}{2016-03-16}{made \cs{fillANotB} a synonym for
%\cs{fillOnlyA}}
%    \begin{macrocode}
  \let\fillANotB\fillOnlyA
%    \end{macrocode}
% Fill $B$ but not $A$ (same as only $B$ for two sets).
%\changes{1.1}{2016-03-16}{made \cs{fillBNotA} a synonym for
%\cs{fillOnlyB}}
%    \begin{macrocode}
  \let\fillBNotA\fillOnlyB
%    \end{macrocode}
% Fill $A$ intersect $B$
%    \begin{macrocode}
  \def\fillACapB{%
    \begin{scope}
      \clip (\@venn@Ax,\@venn@Ay) circle (\@venn@radius);
      \path[fill=\@venn@shade] 
        (\@venn@Bx,\@venn@By) circle (\@venn@radius);
    \end{scope}
  }%
%    \end{macrocode}
% Define synonym:
%    \begin{macrocode}
  \let\fillBCapA\fillACapB
%    \end{macrocode}
% Start the \env{tikzpicture} environment.
%    \begin{macrocode}
  \ifdefempty{\@venn@tikzoptions}%
  {%
    \def\@venn@dobegin{\begin{tikzpicture}}%
  }%
  {%
    \edef\@venn@dobegin{\noexpand\begin{tikzpicture}%
      [\expandonce\@venn@tikzoptions]}%
  }%
  \@venn@dobegin
%    \end{macrocode}
% coordinates of the Venn diagram corners
%    \begin{macrocode}
  \path (0,0) coordinate (venn bottom left)
        (0,\@venn@h) coordinate (venn top left)
        (\@venn@w,\@venn@h) coordinate (venn top right)
        (\@venn@w,0) coordinate (venn bottom right);
}%
%    \end{macrocode}
% End environment code
%    \begin{macrocode}
{%
%    \end{macrocode}
% Draw outlines
%    \begin{macrocode}
    \draw (venn bottom left) rectangle (\@venn@w,\@venn@h);
    \draw (\@venn@Ax,\@venn@Ay) circle (\@venn@radius);
    \draw (\@venn@Bx,\@venn@By) circle (\@venn@radius);
%    \end{macrocode}
% Draw labels
%    \begin{macrocode}
    \draw (\@venn@Ax,\@venn@Ay) node[above,left] (labelOnlyA)
       {\@venn@label@OnlyA};
    \draw (\@venn@Bx,\@venn@By) node[above,right] (labelOnlyB)
       {\@venn@label@OnlyB};
%    \end{macrocode}
% Region labels
%    \begin{macrocode}
    \draw (\@venn@vgap,\@venn@hgap) node (labelNotAB) {\@venn@label@NotAB};
    \draw (\@venn@Ax,\@venn@Ay+\@venn@radius)
      node[below] (labelA) {\@venn@label@A};
    \draw (\@venn@Bx,\@venn@By+\@venn@radius)
      node[below] (labelB) {\@venn@label@B};
    \draw (0.5*\@venn@w,0.5*\@venn@h) node (labelAB) {\@venn@label@AB};
    \@postvennhook
  \end{tikzpicture}
}
%    \end{macrocode}
%\end{environment}
%
%\begin{macro}{\@venn@fillOnlyA@twosets}
%\changes{1.1}{2016-03-16}{new}
%\cs{fillOnlyA} is set to this for the two set version.
% Reimplemented in v1.1 to use new circle intersect code.
%    \begin{macrocode}
\newcommand*\@venn@fillOnlyA@twosets{%
%    \end{macrocode}
% Get the intersection points between $A$ and $B$.
%    \begin{macrocode}
    \@venn@computecircleintersects{\@venn@Ax}{\@venn@Ay}{\@venn@Bx}{\@venn@By}%
%    \end{macrocode}
% If there aren't two points of intersection, then the circles don't
% overlap.
%    \begin{macrocode}
    \ifnum\@venn@intersect@n=2\relax
%    \end{macrocode}
% Compute the start angle for the first arc.
%    \begin{macrocode}
      \pgfmathsetlength{\dimen@i}{\@venn@intersect@i@x-\@venn@Ax}%
      \pgfmathsetlength{\dimen@ii}{\@venn@intersect@i@y-\@venn@Ay}%
      \ifvennoldpgf
        \pgfmathatantwo{\dimen@i}{\dimen@ii}%
      \else
        \pgfmathatantwo{\dimen@ii}{\dimen@i}%
      \fi
      \let\@venn@start@i@angle\pgfmathresult
%    \end{macrocode}
% Compute the end angle for the first arc.
%    \begin{macrocode}
      \pgfmathsetlength{\dimen@i}{\@venn@intersect@ii@x-\@venn@Ax}%
      \pgfmathsetlength{\dimen@ii}{\@venn@intersect@ii@y-\@venn@Ay}%
      \ifvennoldpgf
        \pgfmathatantwo{\dimen@i}{\dimen@ii}%
      \else
        \pgfmathatantwo{\dimen@ii}{\dimen@i}%
      \fi
      \let\@venn@end@i@angle\pgfmathresult
      \pgfmathadd{\@venn@end@i@angle}{360}%
      \let\@venn@end@i@angle\pgfmathresult
%    \end{macrocode}
% Compute the start angle for the second arc.
%    \begin{macrocode}
      \pgfmathsetlength{\dimen@i}{\@venn@intersect@ii@x-\@venn@Bx}%
      \pgfmathsetlength{\dimen@ii}{\@venn@intersect@ii@y-\@venn@By}%
      \ifvennoldpgf
        \pgfmathatantwo{\dimen@i}{\dimen@ii}%
      \else
        \pgfmathatantwo{\dimen@ii}{\dimen@i}%
      \fi
      \let\@venn@start@ii@angle\pgfmathresult
%    \end{macrocode}
% Compute the end angle for the second arc.
%    \begin{macrocode}
      \pgfmathsetlength{\dimen@i}{\@venn@intersect@i@x-\@venn@Bx}%
      \pgfmathsetlength{\dimen@ii}{\@venn@intersect@i@y-\@venn@By}%
      \ifvennoldpgf
        \pgfmathatantwo{\dimen@i}{\dimen@ii}%
      \else
        \pgfmathatantwo{\dimen@ii}{\dimen@i}%
      \fi
      \let\@venn@end@ii@angle\pgfmathresult
      \pgfmathadd{\@venn@start@ii@angle}{360}%
      \let\@venn@start@ii@angle\pgfmathresult
%    \end{macrocode}
% Fill the path
%    \begin{macrocode}
      \path[fill=\@venn@shade]
       (\@venn@intersect@i@x,\@venn@intersect@i@y)
      arc[radius=\@venn@radius,start angle=\@venn@start@i@angle,
      end angle=\@venn@end@i@angle]
      arc[radius=\@venn@radius,start angle=\@venn@start@ii@angle,
      end angle=\@venn@end@ii@angle]
       -- cycle;
    \else
       \fillOnlyA
    \fi
}%
%    \end{macrocode}
%\end{macro}
%
%\begin{macro}{\@venn@fillOnlyB@twosets}
%\changes{1.1}{2016-03-16}{new}
%\cs{fillOnlyB} is set to this for the two set version.
% Reimplemented in v1.1 to use new circle intersect code.
%    \begin{macrocode}
\newcommand*\@venn@fillOnlyB@twosets{%
%    \end{macrocode}
% Get the intersection points between $A$ and $B$.
%    \begin{macrocode}
    \@venn@computecircleintersects{\@venn@Ax}{\@venn@Ay}{\@venn@Bx}{\@venn@By}%
%    \end{macrocode}
% If there aren't two points of intersection, then the circles don't
% overlap.
%    \begin{macrocode}
    \ifnum\@venn@intersect@n=2\relax
%    \end{macrocode}
% Compute the start angle for the first arc.
%    \begin{macrocode}
      \pgfmathsetlength{\dimen@i}{\@venn@intersect@i@x-\@venn@Bx}%
      \pgfmathsetlength{\dimen@ii}{\@venn@intersect@i@y-\@venn@By}%
      \ifvennoldpgf
        \pgfmathatantwo{\dimen@i}{\dimen@ii}%
      \else
        \pgfmathatantwo{\dimen@ii}{\dimen@i}%
      \fi
      \let\@venn@start@i@angle\pgfmathresult
%    \end{macrocode}
% Compute the end angle for the first arc.
%    \begin{macrocode}
      \pgfmathsetlength{\dimen@i}{\@venn@intersect@ii@x-\@venn@Bx}%
      \pgfmathsetlength{\dimen@ii}{\@venn@intersect@ii@y-\@venn@By}%
      \ifvennoldpgf
        \pgfmathatantwo{\dimen@i}{\dimen@ii}%
      \else
        \pgfmathatantwo{\dimen@ii}{\dimen@i}%
      \fi
      \let\@venn@end@i@angle\pgfmathresult
%      \pgfmathadd{\@venn@end@i@angle}{360}%
%      \let\@venn@end@i@angle\pgfmathresult
%    \end{macrocode}
% Compute the start angle for the second arc.
%    \begin{macrocode}
      \pgfmathsetlength{\dimen@i}{\@venn@intersect@ii@x-\@venn@Ax}%
      \pgfmathsetlength{\dimen@ii}{\@venn@intersect@ii@y-\@venn@Ay}%
      \ifvennoldpgf
        \pgfmathatantwo{\dimen@i}{\dimen@ii}%
      \else
        \pgfmathatantwo{\dimen@ii}{\dimen@i}%
      \fi
      \let\@venn@start@ii@angle\pgfmathresult
%    \end{macrocode}
% Compute the end angle for the second arc.
%    \begin{macrocode}
      \pgfmathsetlength{\dimen@i}{\@venn@intersect@i@x-\@venn@Ax}%
      \pgfmathsetlength{\dimen@ii}{\@venn@intersect@i@y-\@venn@Ay}%
      \ifvennoldpgf
        \pgfmathatantwo{\dimen@i}{\dimen@ii}%
      \else
        \pgfmathatantwo{\dimen@ii}{\dimen@i}%
      \fi
      \let\@venn@end@ii@angle\pgfmathresult
%      \pgfmathadd{\@venn@start@ii@angle}{360}%
%      \let\@venn@start@ii@angle\pgfmathresult
%    \end{macrocode}
% Fill the path
%    \begin{macrocode}
      \path[fill=\@venn@shade]
       (\@venn@intersect@i@x,\@venn@intersect@i@y)
      arc[radius=\@venn@radius,start angle=\@venn@start@i@angle,
      end angle=\@venn@end@i@angle]
      arc[radius=\@venn@radius,start angle=\@venn@start@ii@angle,
      end angle=\@venn@end@ii@angle]
       -- cycle;
    \else
       \fillOnlyB
    \fi
}%
%    \end{macrocode}
%\end{macro}
%\iffalse
%    \begin{macrocode}
%</venndiagram.sty>
%    \end{macrocode}
%\fi
%\Finale
\endinput
