\ifx\documentclass\undefined
\documentstyle[fr,programs,keywords]{article}
\else
\documentclass{article}
\usepackage{CheckForDef}
\usepackage{fr}
\usepackage{programs}
\usepackage{keywords}
\fi

\def\p#1{{\bf{}#1}}

\def\wideoutput{%
\setlength{\topmargin}{-1.54cm}%
\setlength{\headsep}{0cm}%
\setlength{\textwidth}{17.5cm}%
\setlength{\textheight}{25.5cm}%
\setlength{\oddsidemargin}{-1cm}%
\setlength{\evensidemargin}{-1cm}}
\wideoutput

\InBodyLeftNumberLine

\ProgKeywords

\begin{document}

Dans le programme suivant, quel est le type de \p{A}?
celui de~\p{B}?
\p{A} et~\p{B} ont-ils m\^eme type?
Les instructions des
lignes~\ref{progFORTTYPAGEinstruction1}--\ref{progFORTTYPAGEinstruction3}
sont-elles correctes? 
Pourquoi?

\begin{programs}[0.5cm]
\PROCEDURE FORT\_TYPAGE \IS
  \TYPE COULEURS \IS
           (ROUGE, ORANGE, JAUNE, VERT, BLEU, INDIGO, VIOLET);
  A : \ARRAY (1..5) \OF COULEURS
                        := (BLEU, ROUGE, VERT, JAUNE, VIOLET);
  B : \ARRAY (1..5) \OF COULEURS;
\BEGIN
  B(1) := A(1); \label{progFORTTYPAGEinstruction1}
  B(1..5) := A(1..5); \label{progFORTTYPAGEinstruction2}
  B := A; \label{progFORTTYPAGEinstruction3}
\END FORT\_TYPAGE;
\end{programs}

La variable \p{A} est d'un type anonyme tableau \`a une
dimension de \p{COULEURS}.
Il en est de m\^eme pour \p{B}.
Bien que la description des deux types soit identique, ils sont
diff\'erents.

L'instruction de la ligne~\ref{progFORTTYPAGEinstruction1} est
correcte (affectation d'une valeur de type \p{COULEURS} dans une
variable du m\^eme type). 
Les instructions des lignes~\ref{progFORTTYPAGEinstruction2}
et~\ref{progFORTTYPAGEinstruction3} sont ill\'egales (les types
sont diff\'erents).
Par contre, si l'on avait \'ecrit le programme comme suit

\NewKeyword{\BEGIN}{this is the beginning}[ceci est le d\'ebut]
\FUAlgoKeywords

\begin{programs}[0.5cm]
\PROCEDURE FORT\_TYPAGE \IS
  \TYPE COULEURS \IS
           (ROUGE, ORANGE, JAUNE, VERT, BLEU, INDIGO, VIOLET);
  \TYPE T \IS \ARRAY (INTEGER \RANGE <>) \OF COULEURS;
  A : T(1..5) := (BLEU, ROUGE, VERT, JAUNE, VIOLET);
  B : T(1..5);
\BEGIN
  B(1) := A(1);
  B(1..5) := A(1..5);
  B := A;
\END FORT\_TYPAGE;
\end{programs}

\noindent%
il n'y aurait pas eu d'erreur.
Dans ce cas, en effet, \p{A} et \p{B} appartiennent au m\^eme
sous-type initial, et un contr\^ole sera mis en place pour
v\'erifier les bornes des intervalles \`a l'ex\'ecution.

\bigskip

Un exercice de num\'erotation:

\NewKeyword{\END}{this IS the real end}
\AlgoKeywords

\begin{programs}*
    \WITH TEXT\_IO;  \USE TEXT\_IO;
    \PROCEDURE NUMEROTATION
              (FICHIER\_ENTREE : \IN STRING;
               FICHIER\_SORTIE : \IN STRING := "") IS
      \SUBTYPE LONGUEUR\_LIGNE \IS INTEGER
                             \RANGE 1..255;
      FICHIER\_IN  : FILE\_TYPE;
      FICHIER\_OUT : FILE\_TYPE;
      COMPTEUR : NATURAL := 0;
      LIGNE : STRING (LONGUEUR\_LIGNE);
      FIN\_DE\_LIGNE : NATURAL;
    \BEGIN
      OPEN (FILE => FICHIER\_IN,
            MODE => IN\_FILE,
            NAME => FICHIER\_ENTREE);
      \IF FICHIER\_SORTIE = "" \THEN
        CREATE (FILE => FICHIER\_OUT,
                MODE => OUT\_FILE,
                NAME => FICHIER\_ENTREE \& ".num");
      \ELSE
        CREATE (FILE => FICHIER\_OUT,
                MODE => OUT\_FILE,
                NAME => FICHIER\_SORTIE);
      \END \IF;
      RESET (FICHIER\_IN);
      \WHILE \NOT END\_OF\_FILE (FICHIER\_IN) \LOOP
        GET\_LINE (FILE => FICHIER\_IN,
                  ITEM => LIGNE,
                  LAST => FIN\_DE\_LIGNE);
        COMPTEUR := COMPTEUR + 1;
        PUT\_LINE (FILE => FICHIER\_OUT,
                  ITEM => NATURAL'IMAGE (COMPTEUR) \&
                          " " \& LIGNE (1..FIN\_DE\_LIGNE));
      \END \LOOP;
      CLOSE (FILE => FICHIER\_IN);
      CLOSE (FILE => FICHIER\_OUT);
    \END NUMEROTATION;

    \WITH TEXT\_IO;  \USE TEXT\_IO;
    \WITH NUMEROTATION;
    \PROCEDURE TEST\_NUMEROTATION \IS
      CARACTERES  : NATURAL;
      NOM\_FICHIER : STRING (1..50);
    \BEGIN
      PUT ("quel fichier voulez-vous ");
      PUT ("numeroter ? ");
      GET\_LINE (NOM\_FICHIER,CARACTERES);
      NUMEROTATION (NOM\_FICHIER (1..CARACTERES));
      PUT\_LINE ("C'est fini");
    \END TEST\_NUMEROTATION;
\end{programs}

% Local Variables: 
% mode: latex
% TeX-master: t
% End: 
    

\end{document}
