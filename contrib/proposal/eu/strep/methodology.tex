\section{Scientific/Technical Methodology and Work Plan}\label{sec:methodology}
\begin{todo}{from the proposal template}
  A detailed work plan should be presented, broken down into work packages\footnote{A work
    package is a major sub-division of the proposed project with a verifiable end-point –
    normally a deliverable or an important milestone in the overall project.} (WPs) which
  should follow the logical phases of the implementation of the project, and include
  consortium management and assessment of progress and results. (Note that your overall
  approach to management will be described later, in Section 2).

Notes: The number of work packages used must be appropriate to the complexity of the work
and the overall value of the proposed project. The planning should be sufficiently
detailed to justify the proposed effort and allow progress monitoring by the Commission.

Any significant risks should be identified, and contingency plans described
\end{todo}
\newpage\section{Objectives and Work Programme}\label{sec:workplan}

\subsection{Anticipated Total Duration of the Project}\label{sec:duration}

\begin{todo}{from the proposal template}
Please state
\begin{itemize}
 \item the project's intended duration 1 and how long DFG funds will be necessary,
 \item for ongoing projects: since when the project has been active.
\end{itemize}
\end{todo}

\subsection{Objectives}\label{sec:objectives}

\begin{objective}[id=firstobj,title=Supporting Authors]
  This is the first objective, after all we have to write proposals all the time, and we
  would rather spend time on research. 
\end{objective}

\begin{objective}[id=secondobj,title=Supporting Reviewers]
  They are only human too, so let's have a heart for them as well. 
\end{objective}


\subsection{Work Programme Including Proposed Research Methods}\label{sec:wawp}

%%%%%%%%%%%%%%%%%%%%%%%%%%%%%%%%%%%%%%%%%%%%%%%%%%%%%%%%%%%%%%%%%%%%%%%%%%%%%%%%%
\LaTeX is the best document markup language, it can even be used for literate
programming~\cite{DK:LP,Lamport:ladps94,Knuth:ttb84}

\begin{todo}{from the proposal template}
 review the state of the art in the and your own contribution to it; probably you want to
  divide this into subsubsections. 
\end{todo}

\begin{todo}{from the proposal template}
For each applicant

Please give a detailed account of the steps planned during the proposed funding pe-
riod. (For experimental projects, a schedule detailing all planned experiments should
be provided.)

The quality of the work programme is critical to the success of a funding proposal. The
work programme should clearly state how much funding will be requested, why the
funds are needed, and how they will be used, providing details on individual items
where applicable.

Please provide a detailed description of the methods that you plan to use in the project:
What methods are already available? What methods need to be developed? What as-
sistance is needed from outside your own group/institute?
Please list all cited publications pertaining to the description of your work programme
in your bibliography under section 3.
\end{todo}

The project is organized around \pdatacount{all}{wa} large-scale work areas which correspond
to the objectives formulated above. These are subdivided into \pdatacount{all}{wp} work
packages, which we summarize in Figure~\ref{fig:wplist}. Work area
\WAref{mansubsus} will run over the whole project\ednote{come up with a better
  example, this is still oriented towards an EU project} duration of {\pn}. All
{\pdatacount{systems}{wp}} work packages in {\WAref{systems}} will and have to be
covered simultaneously in order to benefit from design-implementation-application feedback
loops.

\wpfig

\begin{workplan}
\begin{workarea}[id=mansubsus,title={Management, Support \& Sustainability}, short=Management]
  This work-group corresponds to Objective \OBJref{firstobj} and has two work packages:
  one for management proper ({\WPref{management}}), and one each for
  dissemination ({\WPref{dissem}})
   
  This work group ensures the dissemination and creation of the periodic integrative
  reports containing the periodic Project Management Report, the Project Management
  Handbook, an Knowledge Dissemination Plan ({\WPref{management}}), the Proceedings of the
  Annual {\pn} Summer School as well as non-public Dissemination and Exploitation plans
  ({\WPref{dissem}}), as well as a report of the {\pn} project milestones.
   
\begin{workpackage}[id=management,lead=jacu,
  title=Project Management,
 jacuRM=2,jacuRAM=8,pcgRM=2]
  Based on the ``Bewilligungsbescheid'' of the DFG, and based on the financial and
  administrative data agreed, the project manager will carry out the overall project
  management, including administrative management.  A project quality handbook will be
  defined, and a {\pn} help-desk for answering questions about the format (first
  project-internal, and after month 12 public) will be established. The project management
  will consist of the following tasks
\begin{tasklist} 
\begin{task}[id=foo,wphases=0-3,requires=\taskin{t1}{dissem}]
  To perform the administrative, scientific/technical, and financial management of the
  project 
\end{task}
\begin{task}[wphases=13-17!.5]
  To co-ordinate the contacts with the DFG and other funding bodies, building on the
  results in \taskref{management}{foo}
\end{task}
\begin{task}
  To control quality and timing of project results and to resolve conflicts
\end{task}
\begin{task}
  To set up inter-project communication rules and mechanisms
\end{task}
\end{tasklist}

\end{workpackage}
 
\begin{workpackage}[id=dissem,lead=pcg,
 title=Dissemination and Exploitation,
pcgRM=8,jacuRAM=2] 
Much of the activity of a project involves small groups of nodes in joint work. This work
 package is set up to ensure their best wide-scale integration, communication, and
 synergetic presentation of the results. Clearly identified means of dissemination of
 work-in-progress as well as final results will serve the effectiveness of work within the
 project and steadily improve the visibility and usage of the emerging semantic services.


 The work package members set up events for dissemination of the research and
 work-in-progress results for researchers (workshops and summer schools), and for industry
 (trade fairs). An in-depth evaluation will be undertaken of the response of test-users.
 
 \begin{tasklist}
  \begin{task}[id=t1,wphases=6-7]
    sdfkj
  \end{task}
  \begin{task}[wphases=12-13]
    sdflkjsdf
  \end{task}
  \begin{task}[wphases=18-19]
    sdflkjsdf
  \end{task}
 \begin{task}[wphases=22-24] 
 \end{task}
\end{tasklist}

Within two months of the start of the project, a project website will go live. This
website will have two areas: a members' area and a public area.\ldots
\end{workpackage}
\end{workarea}
 

\begin{workarea}[id=systems,title={System Development}]
  This workarea does not correspond to \OBJtref{secondobj}, but it has two work packages:
  one for the development of the {\LaTeX} class ({\WPref{class}}), and for the
  proposal template ({\WPref{temple}})

  This work group coordinates the system development.

\begin{workpackage}[id=class,lead=jacu,
                    title=A LaTeX class for EU Proposals,short=Class,
                   jacuRM=12,jacuRAM=8,pcgRM=12,pcgRAM=2]
We plan to develop a {\LaTeX} class for marking up EU Proposals

We will follow strict software design principles, first comes a
requirements analys, then \ldots
\begin{tasklist}
  \begin{task}[wphases=0-2]
    sdfsdf
  \end{task}
  \begin{task}[wphases=4-8]
    sdfsdf
  \end{task}
  \begin{task}[id=t3,wphases=10-14]
    sdfsdf
  \end{task}
  \begin{task}[wphases=20-24]
    sdfsdfd
  \end{task}
\end{tasklist}
\end{workpackage} 

\begin{workpackage}[id=temple,lead=pcg,
  title= Proposal Template,short=Template,jacuRM=12]

We plan to develop a template file for {\pn} proposals

We abstract an example from existing proposals
\begin{tasklist}
  \begin{task}[wphases=6-12]
    sdfdsf 
  \end{task}
  \begin{task}[id=temple2,wphases=18-24,requires=\taskin{t3}{class}]
    sdfsdf
  \end{task} 
\end{tasklist}
\end{workpackage}

\begin{workpackage}[id=workphase,title=A work package without tasks,
  wphases=0-4!.5]
  
  And finally, a work package without tasks, so we can see the effect on the gantt chart
  in fig~\ref{fig:gantt}.
\end{workpackage}
\end{workarea}
\end{workplan} 

\ganttchart[draft,xscale=.45] 

\subsection{Data Handling}\label{sec:data}

The \pn project will not systematically produce researchdata. All project results will be
published for at least $x$ years at our archive at \url{http://example.org}.

\subsection{-- 2.7 (Other Information / Explanations on the Proposed Investigations / Information on Scientific and Financial Involvement of International
  Cooperation Partners) \qquad \sf n/a}


%%% Local Variables: 
%%% mode: LaTeX
%%% TeX-master: "proposal"
%%% End: 

% LocalWords:  workplan.tex wplist dfgcount wa mansubsus duratio ipower wpfig
% LocalWords:  ganttchart xscale workplan workarea pdataref dissem workpackage foo
% LocalWords:  tasklist taskin taskref sdfkj sdflkjsdf sdfsdf sdfsdfd sdfdsf pn
% LocalWords:  firstobj secondobj pdatacount WAref ednote OBJref pcgRM pcg
% LocalWords:  ldots OBJtref workphase


\newpage
\subsection{Work Package List}\label{sec:wplist}

\begin{todo}{from the proposal template}
Please indicate one activity per work package:
RTD = Research and technological development; DEM = Demonstration; MGT = Management of the consortium
\end{todo}

%\makeatletter\wp@total@RM{management}\makeatother
\wpfigstyle{\footnotesize}
\wpfig[pages,type,start,end]

\newpage\subsection{List of Deliverables}\label{sec:deliverables}

\begin{todo}{from the proposal template}
\begin{compactenum}
\item Deliverable numbers in order of delivery dates. Please use the numbering convention <WP number>.<number of deliverable within
that WP>. For example, deliverable 4.2 would be the second deliverable from work package 4.
\item Please indicate the nature of the deliverable using one of the following codes:
R = Report, P = Prototype, D = Demonstrator, O = Other
\item Please indicate the dissemination level using one of the following codes:
PU = Public
PP = Restricted to other programme participants (including the Commission Services).
RE = Restricted to a group specified by the consortium (including the Commission Services).
CO = Confidential, only for members of the consortium (including the Commission Services).
\end{compactenum}
\end{todo}
We will now give an overview over the deliverables and milestones of the work
packages. Note that the times of deliverables after month 24 are estimates and may change
as the work packages progress.

In the table below, {\emph{integrating work deliverables}} (see top of
section~\ref{sec:wplist}) are printed in boldface to mark them. They integrate
contributions from multiple work packages. \ednote{CL: the rest of this paragraph does not
  comply with the EU guide for applicants, needs to be rewritten}These can have the
dissemination level ``partial'', which indicates that it contains parts of level
``project'' that are to be disseminated to the project and evaluators only. In such
reports, two versions are prepared, and disseminated accordingly.

{\footnotesize\inputdelivs{8cm}}


%%% Local Variables: 
%%% mode: latex
%%% TeX-master: "propB"
%%% End: 

\newpage\subsection{List of Milestones}\label{sec:milestones}

\begin{todo}{from the proposal template}
  Milestones are control points where decisions are needed with regard to the next stage
  of the project. For example, a milestone may occur when a major result has been
  achieved, if its successful attainment is a requirement for the next phase of
  work. Another example would be a point when the consortium must decide which of several
  technologies to adopt for further development.

  Means of verification: Show how you will confirm that the milestone has been
  attained. Refer to indicators if appropriate. For examples: a laboratory prototype
  completed and running flawlessly, software released and validated by a user group, field
  survey complete and data quality validated.
\end{todo}


The work in the {\pn} project is structured by seven milestones, which coincide with the
project meetings in summer and fall.  Since the meetings are the main face-to-face
interaction points in the project, it is suitable to schedule the milestones for these
events, where they can be discussed in detail. We envision that this setup will give the
project the vital coherence in spite of the broad mix of disciplinary backgrounds of the
participants.\ednote{maybe automate the milestones}

\begin{milestones}
  \milestone[id=kickoff,verif=Inspection,month=1]
    {Initial Infrastructure}
    {Set up the organizational infrastructure, in particular: Web Presence, project TRAC,\ldots}
  \milestone[id=consensus,verif=Inspection,month=24]{Consensus} {Reach Consensus on the
    way the project goes}
  \milestone[id=exploitation,verif=Inspection,month=36]{Exploitation}{The exploitation
    plan should be clear so that we can start on this in the last year.}
  \milestone[id=final,verif=Inspection,month=48]{Final Results}{all is done}
\end{milestones}

%%% Local Variables: 
%%% mode: latex
%%% TeX-master: "propB"
%%% End: 

% LocalWords:  pn ednote verif ldots


\subsection{Work Package Descriptions}\label{sec:workpackages}
\begin{workplan}
%%%%%%%%%%%%%%%%%%%%%%%%%%%%%%
%  Work Package Description  %
%%%%%%%%%%%%%%%%%%%%%%%%%%%%%%

\begin{workpackage}{MANAGEMENT  WORK PACKAGE}
  \label{wp:management} %change and use appropriate description

  %%%%%%%%%%%%%%%%%% TOP TABLE %%%%%%%%%%%%%%%%%%%%%%%%%%%%%
  % Data for the top table
  \wpstart{1} %Starting Month
  \wpend{36} %End Month
  \wptype{Activity type} %RTD, DEM, MGT, or OTHER

  % Person Months per participant (required, max 7, * for leader)  
  % syntax: \personmonths{Participant number}{value}    (not wp leader)
  %     or  \personmonths{Participant short name}{value} (not wp leader)
  %         \personmonths*{Participant number}{value}    (wp leader)
  % for example:
  \personmonths*{UoC}{12}
  \personmonths{UoP2}{3}
  \personmonths{UoP3}{2}
  % etc.

  \makewptable % Work package summary table
    
  % Work Package Objectives
  \begin{wpobjectives}
    This work package has the following objectives:
    \begin{enumerate}
    \item To develop ....
    \item To apply this ....
    \item etc.
    \end{enumerate}
  \end{wpobjectives}
  
  % Work Package Description
  \begin{wpdescription}
    % Divide work package into multiple tasks.
    % Use \wptask command
    % syntax: \wptask{leader}{contributors}{start-m}{end-m}{title}{description}   
 
    Description of work carried out in WP, broken down into tasks, and
    with role of partners list. Use the \texttt{\textbackslash wptask} command.

    \wptask{UoC}{UoC}{1}{12}{Test}{
      \label{task:wp1test}
      Here we will test the WP Task code. 
    }
    \wptask{UoC}{UoC}{6}{9}{Integrate}{
      \label{task:wp1integrate}
      In this task UZH will integrate the work done in ~\ref{task:wp1test}.
    }    
    \wptask{UoP3}{All other}{9}{12}{Apply}{
      Here all the WP participants will apply the results to...
    }
    
    \paragraph{Role of partners}
    \begin{description}
    \item[Participant short name] will lead Task~\ref{task:wp1integrate}.
    \item[UoC] will..
    \end{description}
  \end{wpdescription}
  
  % Work Package Deliverable
  \begin{wpdeliverables}
    % Data for the deliverables and milestones  tables
    % syntax: \deliverable[delivery date]{nature}{dissemination
    % level}{description} 
    %
    % nature: R = Report, P = Prototype, D = Demonstrator, O = Other
    % dissemination level: PU = Public, PP = Restricted to other
    % programme participants (including the Commission Services), RE =
    % Restricted to a group specified by the consortium (including the
    % Commission Services), CO = Confidential, only for members of the
    % consortium (including the Commission Services).
    % 
    % \wpdeliverable[date]{R}{PU}{A report on \ldots}

    \wpdeliverable[36]{UoC}{R}{PU}{Report on the definition of the model
      specifications.}\label{dev:wp1specs}
    
    \wpdeliverable[12]{UoP3}{R}{PU}{Report on Feasibility study for the model
      implementation.}\label{dev:wp1implementation}

    \wpdeliverable[24]{UoP2}{R}{PU}{Prototype of model
      implementation.}\label{dev:wp1prototype}

  \end{wpdeliverables}

\end{workpackage}


%%% Local Variables:
%%% mode: latex
%%% TeX-master: "proposal-main"
%%% End:
\newpage
\begin{workpackage}%
[id=dissem,type=RTD,lead=efo,
 wphases=10-24!1,
 title=Dissemination and Exploitation,short=Dissemination,
 efoRM=8,jacuRM=2,barRM=2,bazRM=2]
We can state the state of the art and similar things before the summary in the boxes
here. 
\wpheadertable

\begin{wpobjectives}
  Much of the activity of a project involves small groups of nodes in joint work. This
  work package is set up to ensure their best wide-scale integration, communication, and
  synergetic presentation of the results. Clearly identified means of dissemination of
  work-in-progress as well as final results will serve the effectiveness of work within
  the project and steadily improve the visibility and usage of the emerging semantic
  services.
\end{wpobjectives}

\begin{wpdescription}
  The work package members set up events for dissemination of the research and
  work-in-progress results for researchers (workshops and summer schools), and for
  industry (trade fairs). An in-depth evaluation will be undertaken of the response of
  test-users.

  Within two months of the start of the project, a project website will go live. This
  website will have two areas: a members' area and a public area.\ldots
\end{wpdescription}

\begin{wpdelivs}
  \begin{wpdeliv}[due=2,id=website,nature=O,dissem=PU,miles=kickoff]
     {Set-up of the Project web server}
   \end{wpdeliv}
   \begin{wpdeliv}[due=8,id=ws1proc,nature=R,dissem=PU,miles={kickoff}]
     {Proceedings of the first {\pn} Summer School.}
   \end{wpdeliv}
   \begin{wpdeliv}[due=9,id=dissem,nature=R,dissem=PP]
     {Dissemination Plan}
   \end{wpdeliv}
   \begin{wpdeliv}[due=9,id=exploitplan,nature=R,dissem=PP,miles=exploitation]
     {Scientific and Commercial Exploitation Plan}
   \end{wpdeliv}
   \begin{wpdeliv}[due=20,id=ws2proc,nature=R,dissem=PU,miles={exploitation}]
     {Proceedings of the second {\pn} Summer School.}
   \end{wpdeliv}
   \begin{wpdeliv}[due=32,id=ss1proc,nature=R,dissem=PU,miles={exploitation}]
     {Proceedings of the third {\pn} Summer School.}
   \end{wpdeliv}
   \begin{wpdeliv}[due=44,id=ws3proc,nature=R,dissem=PU,miles=exploitation]
     {Proceedings of the fourth {\pn} Summer School.}
   \end{wpdeliv}
 \end{wpdelivs}
\end{workpackage}

%%% Local Variables: 
%%% mode: LaTeX
%%% TeX-master: "propB"
%%% End: 

% LocalWords:  wp-dissem.tex workpackage dissem efo fromto bazRM wpheadertable
% LocalWords:  wpobjectives wpdescription ldots wpdelivs wpdeliv ws1proc pn
% LocalWords:  exploitplan ws2proc ss1proc ws3proc pdataRef deliv
% LocalWords:  mansubsusintReport
\newpage
\begin{workpackage}[id=class,type=RTD,lead=jacu,
                    wphases=3-9!1,
                    title=A {\LaTeX} class for EU Proposals,short=Class,
                    jacuRM=12,barRM=12]
We can state the state of the art and similar things before the summary in the boxes
here. 
\wpheadertable
\begin{wpobjectives}
\LaTeX is the best document markup language, it can even be used for literate
programming~\cite{DK:LP,Lamport:ladps94,Knuth:ttb84}

  To develop a {\LaTeX} class for marking up EU Proposals
\end{wpobjectives}

\begin{wpdescription}
  We will follow strict software design principles, first comes a requirements analys,
  then \ldots
\end{wpdescription}

\begin{wpdelivs}
  \begin{wpdeliv}[due=6,id=req,nature=R,dissem=PP,miles=kickoff]
     {Requirements analysis}
   \end{wpdeliv}
   \begin{wpdeliv}[due=12,id=spec,nature=R,dissem=PU,miles=consensus]
     {{\pn} Specification }
   \end{wpdeliv}
   \begin{wpdeliv}[due=18,id=demonstrator,nature=P,dissem=PU,miles={consensus,final}]
     {First demonstrator ({\tt{article.cls}} really)}
   \end{wpdeliv}
   \begin{wpdeliv}[due=24,id=proto,nature=P,dissem=PU,miles=final]
     {First prototype}
   \end{wpdeliv}
    \begin{wpdeliv}[due=36,id=release,nature=P,dissem=PU,miles=final]
      {Final {\LaTeX} class, ready for release}
    \end{wpdeliv}
  \end{wpdelivs}
Furthermore, this work package contributes to {\pdataRef{deliv}{managementreport2}{label}} and
{\pdataRef{deliv}{managementreport7}{label}}.
\end{workpackage}

%%% Local Variables: 
%%% mode: LaTeX
%%% TeX-master: "propB"
%%% End: 
\newpage
\begin{workpackage}[id=temple,type=DEM,lead=bar,
  wphases=6-12!1,
  title={\pn} Proposal Template,short=Template,barRM=6,bazRM=6]
We can state the state of the art and similar things before the summary in the boxes
here. 
\wpheadertable

\begin{wpobjectives}
  To develop a template file for {\pn} proposals
\end{wpobjectives}

\begin{wpdescription}
  We abstract an example from existing proposals
\end{wpdescription}

\begin{wpdelivs}
  \begin{wpdeliv}[due=6,id=req,nature=R,dissem=PP,miles=kickoff]
    {Requirements analysis}
  \end{wpdeliv}
  \begin{wpdeliv}[due=12,id=spec,nature=R,dissem=PU,miles=consensus]
    {{\pn} Specification }
  \end{wpdeliv}
  \begin{wpdeliv}[due=18,id=demonstrator,nature=D,dissem=PU,miles={consensus,final}]
    {First demonstrator ({\tt{article.cls}} really)}
  \end{wpdeliv}
  \begin{wpdeliv}[due=24,id=proto,nature=P,dissem=PU,miles=final]
    {First prototype}
  \end{wpdeliv}
  \begin{wpdeliv}[due=36,id=release,nature=P,dissem=PU,miles=final]
    {Final Template, ready for release}
  \end{wpdeliv}
\end{wpdelivs}
Furthermore, this work package contributes to {\pdataRef{deliv}{managementreport2}{label}} and
{\pdataRef{deliv}{managementreport7}{label}}.
\end{workpackage}

%%% Local Variables: 
%%% mode: LaTeX
%%% TeX-master: "propB"
%%% End: 

% LocalWords:  wp-temple.tex workpackage fromto pn bazRM wpheadertable wpdelivs
% LocalWords:  wpobjectives wpdescription wpdeliv req dissem tt article.cls
% LocalWords:  pdataRef deliv systemsintReport
\newpage
\end{workplan}
\newpage\subsection{Significant Risks and Associated Contingency Plans}\label{sec:risks}
\begin{todo}{from the proposal template}
  Describe any significant risks, and associated contingency plans
\end{todo}
\begin{oldpart}{need to integrate this somewhere. CL: I will check other proposals to see how they did it; the Guide does not really prescribe anything.}
\paragraph{Global Risk Management}
The crucial problem of \pn (and similar endeavors that offer a new basis for communication
and interaction) is that of community uptake: Unless we can convince scientists and
knowledge workers industry to use the new tools and interactions, we will
never be able to assemble the large repositories of flexiformal mathematical knowledge we
envision. We will consider uptake to be the main ongoing evaluation criterion for the network.
\end{oldpart}

%%% Local Variables: 
%%% mode: latex
%%% TeX-master: "propB"
%%% End: 



%%% Local Variables: 
%%% mode: latex
%%% TeX-master: "propB"
%%% End: 

% LocalWords:  workplan newpage wplist makeatletter makeatother wpfig
% LocalWords:  workpackages wp-dissem wp-class wp-temple
