\def\filename{eiad.dtx}
\def\fileversion{1.0}
\def\filedate{1996/11/13}
\let\docversion=\fileversion
\let\docdate=\filedate
% \iffalse meta-comment
%
% Copyright 1996 by Uwe Muench
% 
% For further copyright information, and conditions for modification
% and distribution, see the file legal.txt of the LaTeX2e
% distribution, and any other copyright notices in this file.
% 
%   This system is distributed in the hope that it will be useful,
%   but WITHOUT ANY WARRANTY; without even the implied warranty of
%   MERCHANTABILITY or FITNESS FOR A PARTICULAR PURPOSE.
% 
%   For error reports concerning UNCHANGED versions of this file, please
%   contact me (email: muench@ph-cip.uni-koeln.de)
% 
%   Please do not request updates from me directly.  Primary
%   distribution is through the CTAN archives.
% 
% 
% Permission is granted to copy this file to another file with a
% clearly different name and to customize the declarations in that
% copy to serve the needs of your installation, provided that you
% comply with the conditions in the file legal.txt of the LaTeX2e
% distribution. 
% 
% However, NO PERMISSION is granted to produce or to distribute a
% modified version of this file under its original name.
%  
% You are NOT ALLOWED to change this file.
% 
% 
% 
% \fi
% \iffalse
%%% File: eiad.dtx
%% Copyright (C) 1996 Uwe Muench
%% all rights reserved.
%<package>\NeedsTeXFormat{LaTeX2e}
%<package>\ProvidesPackage{eiad}[1996/11/13 v1.0 LaTeX package eiad]
%<*driver>
\documentclass{ltxdoc}
\usepackage{eiad}
\GetFileInfo{eiad.sty}
\EnableCrossrefs
\RecordChanges
\begin{document}
\title{The \texttt{eiad} package\thanks
       {This file has version number \fileversion, dated \filedate.}\\
      for use with \LaTeX2e}
\author{Uwe M\"unch\\Schmittgasse~92\\51143 K\"oln\\
  \texttt{muench@ph-cip.uni-koeln.de}}
\date{\docdate}
\maketitle
\DocInput{eiad.dtx}
\PrintChanges
\end{document}
%</driver>
% \fi
%
% \CheckSum{15}
%% \CharacterTable
%%  {Upper-case    \A\B\C\D\E\F\G\H\I\J\K\L\M\N\O\P\Q\R\S\T\U\V\W\X\Y\Z
%%   Lower-case    \a\b\c\d\e\f\g\h\i\j\k\l\m\n\o\p\q\r\s\t\u\v\w\x\y\z
%%   Digits        \0\1\2\3\4\5\6\7\8\9
%%   Exclamation   \!     Double quote  \"     Hash (number) \#
%%   Dollar        \$     Percent       \%     Ampersand     \&
%%   Acute accent  \'     Left paren    \(     Right paren   \)
%%   Asterisk      \*     Plus          \+     Comma         \,
%%   Minus         \-     Point         \.     Solidus       \/
%%   Colon         \:     Semicolon     \;     Less than     \<
%%   Equals        \=     Greater than  \>     Question mark \?
%%   Commercial at \@     Left bracket  \[     Backslash     \\
%%   Right bracket \]     Circumflex    \^     Underscore    \_
%%   Grave accent  \`     Left brace    \{     Vertical bar  \|
%%   Right brace   \}     Tilde         \~}
%
%    \changes{v1.0}{1996/03/03}{First release}
%
%    \section{Introduction}
%
%    The font \textsf{eiad} provides a roman and a bold version of
%    Irish fonts for typesetting in gaelic. They are based on the
%    Computer Modern fonts. This font family was written by
%    Ivan~A.~Derzhanski and can be found on the CTAN in the directory 
%    \texttt{tex-archive/fonts/eiad}.
%    This package provides means to use this font under \LaTeXe.
%
%
%    \section{Usage}
%
%    This file can be used as a package by placing its name
%    in the argument of |\usepackage|. Afterwards the font family
%    \textsf{eiad} 
%    is defined. This could also have been done by providing a
%    font definition file.
%
%    \DescribeMacro{\eiadfamily}
%    The command |\eiadfamily| changes the current font family to
%    \texttt{eiad} and the encoding to \texttt{U} (unknown). The
%    \texttt{U} encoding is used 
%    since the Irish long vowels are positioned at places where 
%    e.\,g. the ligatures of the \texttt{OT1}-encoding (like ffl) can be found
%    normally (the new additional characters have to be somewhere
%    obviously\dots). 
%    Usually this command should be used in a \TeX{} group only.
%
%    \DescribeMacro{\texteiad}
%    The command |\texteiad| typesets its argument in the
%    \textsf{eiad} font.
%
%    \DescribeMacro{\eiad}
%    The command |\eiad| provides the old font changing command (like
%    |\rm| compared to |\textrm|). A use of |\eiad| in math mode is
%    not possible.
%
%    To access the new long vowels, the aspirated consonants and the
%    ligature `agus', I cite Ivan Derzhanski:
%    \begin{quote}
%    Within \TeX{} the accented Irish letters are accessible as ligatures, in
%    a manner consistent with the usage on the mailing-list
%    \texttt{gaelic-l}.  A vowel followed by a slash yields a long
%    vowel; a lowercase consonant followed by `h', or 
%    an uppercase consonant followed by either `H' or `h', yields an
%    aspirated consonant (the latter only works with those consonants
%    which actually undergo aspiration; for such things as `n' with a dot,
%    as found in some old texts, the standard \TeX nique (in this case |\.n|)
%    must be used.)  The ligature `agus' is accessed as  s`  (`s'-backquote).
%    \end{quote}
%    So we get the following tables, first the vowels:
%    \begin{quote}
%    \begin{tabbing}
%    Input:\hspace{1cm}\=  |a/| \=  |e/| \=  |i/| \=  |o/|  \=  |u/| \\
%    \textsf{eiad} font \>  \texteiad{a/} \> \texteiad{e/}\>
%    \texteiad{i/}\> \texteiad{o/} \> \texteiad{u/} 
%    \end{tabbing}
%    \end{quote}
%    And now the aspirated consonants and the agus ligature:
%    \begin{quote}
%    \begin{tabbing}
%    Input:\hspace{1cm}\=  
%    |s`| (agus)  \=  |bh|  \=  |ch| \=  |dh|  \=  |fh|  \=  |gh|
%    \=  |mh| \= |ph|  \=  |sh|  \=  |th| \\ 
%    \textsf{eiad} font \>  \texteiad{s`} \>
%    \texteiad{bh} \> \texteiad{ch} \> \texteiad{dh} \> \texteiad{fh} \>
%    \texteiad{gh} \> \texteiad{mh} \> \texteiad{ph} \> \texteiad{sh} \>
%    \texteiad{th}  
%    \end{tabbing}
%    \end{quote}
%    One could write commands which access these special characters only
%    during typesetting in the \textsf{eiad} font, otherwise creating a normal
%    vowel or consonant (instead of e.\,g.~a/ in the output). But I
%    don't think this is reasonable since the above transcription
%    seems to be common use. So it is possible to use your input text as
%    transcription also; a conversion to normal vowels or consonants
%    would be only an annoyance. Please contact me, if there is a
%    different transcription method, one could implement through such
%    macros. 
%
%    \StopEventually{}
%
%
%    \section{Implementation}
%
%    First we declare a new font family for the \textsf{eiad} font.
%    \begin{macrocode}
\DeclareFontFamily{OT1}{eiad}{}
%    \end{macrocode}
%
%    This font is only available in the normal and the bold
%    series at 10 point, so we always scale silently to the desired
%    size. 
% 
%    \begin{macrocode}
\DeclareFontShape{OT1}{eiad}{m}{n}{<->s* eiad10}{}
\DeclareFontShape{OT1}{eiad}{m}{bx}{<->s* eiadbf10}{}
%    \end{macrocode}
%
%    Now we define the font changing commands.
% 
%    \begin{macro}{\eiadfamily}
%    The macro |\eiadfamily| selects the \textsf{eiad} family. 
%    \begin{macrocode}
\DeclareRobustCommand\eiadfamily{%
  \fontfamily{eiad}%
  \fontencoding{OT1}%
  \selectfont}
%    \end{macrocode}
%    \end{macro}
%
%    \begin{macro}{\texteiad}
%    The macro |\texteiad| typesets its argument in the \textsf{eiad} font.
%    \begin{macrocode}
\DeclareTextFontCommand\texteiad{\eiadfamily}
%    \end{macrocode}
%    \end{macro}
%
%    \begin{macro}{\eiad}
%    The macro |\eiad| is the old |\texteiad| version (compare |\rm|
%    to |\textrm|). A use of the \textsf{eiad} font in math mode is forbidden:
%    In math mode the switch |\eiad| is defined as |\relax|. 
%    \begin{macrocode}
\DeclareOldFontCommand{\eiad}{\eiadfamily}{%
        \relax}% Switch for math mode
%    \end{macrocode}
%    \end{macro}
%
% \Finale
%
\endinput
