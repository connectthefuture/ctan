%%
%% This is file `example.tex',
%% generated with the docstrip utility.
%%
%% The original source files were:
%%
%% fbithesis.dtx  (with options: `example')
%% 
%% This is `example.tex', an example file for the fbithesis package.
%% Copyright (C) 2002-2011 Andre Dierker
%% 
%% This file is part of the fbithesis package.
%% -------------------------------------------
%% 
%% It may be distributed and/or modified under the conditions of the
%% LaTeX Project Public License, either version 1.3 of this license or
%% (at your option) any later version.
%% 
%% The latest version of this license is in
%%   http://www.latex-project.org/lppl.txt
%% and version 1.3 or later is part of all distributions of LaTeX
%% version 2005/12/01 or later.
%% 
%% This file may not be distributed without the original source file
%% `fbithesis.dtx'.
%% 
%% The list of all files belonging to the fbithesis package is given
%% in the file `README'.
%% 
%% For more details, LaTeX the source `fbithesis.dtx'.
%% 
%% \CharacterTable
%%   {Upper-case    \A\B\C\D\E\F\G\H\I\J\K\L\M\N\O\P\Q\R\S\T\U\V\W\X\Y\Z
%%   Lower-case    \a\b\c\d\e\f\g\h\i\j\k\l\m\n\o\p\q\r\s\t\u\v\w\x\y\z
%%   Digits        \0\1\2\3\4\5\6\7\8\9
%%   Exclamation   \!     Double quote  \"     Hash (number) \#
%%   Dollar        \$     Percent       \%     Ampersand     \&
%%   Acute accent  \'     Left paren    \(     Right paren   \)
%%   Asterisk      \*     Plus          \+     Comma         \,
%%   Minus         \-     Point         \.     Solidus       \/
%%   Colon         \:     Semicolon     \;     Less than     \<
%%   Equals        \=     Greater than  \>     Question mark \?
%%   Commercial at \@     Left bracket  \[     Backslash     \\
%%   Right bracket \]     Circumflex    \^     Underscore    \_
%%   Grave accent  \`     Left brace    \{     Vertical bar  \|
%%   Right brace   \}     Tilde         \~}
\NeedsTeXFormat{LaTeX2e}[1994/12/01]%
\ProvidesFile{example.tex}
\def\DescribesFile#1 [#2 #3 #4 (#5)]
  {\def\filedate{#2}\def\fileversion{#3}}
\DescribesFile{fbithesis.cls}
  [2011/02/06 v1.2m
 Example
 for fbithesis (AD)]
 %
 %
 %
 % At first we include an auxiliary file that contains the logos. You
 % may ignore this line. It is only necessary for this example.
\include{exampleaux}
 %
 % fbithesis supports three classes as `baseclass'. To use `book' or
 % `amsbook' you have to use one of the following lines. To use
 % `scrbook' from KOMA-Script as baseclass you have to do nothing since
 % this is the default. If you don't know what I'm talking about just
 % leave these lines as they are. Almost everyone uses `scrbook' as
 % baseclass. It is a wise decision.
 %
 %           \def\baseclass{book}
 %           \def\baseclass{amsbook}
 %
 % Of course we choose fbithesis as document class.  Additionally we
 % want to look at the draft mode and test the option forwarding of
 % a4paper to the baseclass.  Since the example is in English, we also
 % choose english.
\documentclass[a4paper, english]{fbithesis}
\begin{document}
  \frontmatter
 %
 % As in the standard LaTeX classes we use the title command.  Normally
 % one can trust TeX's ability to compute a satisfactory line breaking.
 % However TeX's algorithm is not optimized for titles but for
 % continuous text.  To make it more difficult the cardboard window is
 % quite small.  So if you prefer a different make up, help yourself
 % with an appropriate placed `\\', as you can see in this example.
  \title{Example file for the\\ \texttt{fbithesis} package%
 %
 % The `\thanks' command may be used to provide further
 % information. But as you can see the result of the `\thanks'
 % mechanism does not look well. Therefore I do not recommend the
 % usage. It is better to write a preface instead.
    \thanks{The \texttt{fbithesis}~package has version number
      \fileversion.  It was last revised on \filedate.}%
  }
 %
 % The usage of the \author command: In general master/phd-theses will
 % have only one author, but just in case the \and-command is also
 % provided.
  \author{Andre Dierker%
 %
 % Again the use of the command `\thanks' is possible but not
 % recommmended. Please refer to the documentation.
 %  \thanks{\texttt{software@kand.de}}%
  \and Nobody Else%
  %  \thanks{\texttt{no@body.el.se}}%
  }
 %
 % Normally the subject would be something like `Diplomarbeit' or
 % `Dissertation' ...
  \subject{Example}
 %
 % You may give the beginning and the deadline of your thesis here.
  \date[Created April 3, 2002]{Printed \today}
 %
 % Providing the supervisors of the thesis.
  \supervisors{First Tutor}{Second Tutor}
 %
 % You may use the following commands if you want to place a logos on
 % the title page.  Conforming to Carlisle: Packages in the `graphics'
 % bundle we skip the extensions of the filenames. Due to this TeX is
 % able to include the correct version of the file (EPS or PDF)
  \unidologo{tulogo}
  \chairlogo{ls9logo}
  \thesislogo{thesislogo}
 %
 % The chair you are writing your thesis at.
  \chair{Chair IX (Virtual Research)\\
    Department of Computer Science\\
    TU Dortmund}
 %
 % The data provided by the above macros is now used to set the title page.
 % This is done with the macro \maketitle.
  \maketitle
 %
 % So after the title page is set your thesis may begin:
  \mainmatter
  Now here comes your text.
\end{document}
\endinput
%%
%% End of file `example.tex'.
