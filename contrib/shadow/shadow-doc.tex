\documentclass[pagesize=auto, fontsize=12pt, DIV=10, parskip=half]{scrartcl}

\usepackage{fixltx2e}
\usepackage{etex}
\usepackage{lmodern}
\usepackage[T1]{fontenc}
\usepackage{textcomp}
\usepackage{microtype}
\usepackage{hyperref}

\newcommand*{\mail}[1]{\href{mailto:#1}{\texttt{#1}}}
\newcommand*{\pkg}[1]{\textsf{#1}}
\newcommand*{\cs}[1]{\texttt{\textbackslash#1}}
\makeatletter
\newcommand*{\cmd}[1]{\cs{\expandafter\@gobble\string#1}}
\makeatother
\newcommand*{\meta}[1]{\textlangle\textsl{#1}\textrangle}
\newcommand*{\marg}[1]{\texttt{\{}\meta{#1}\texttt{\}}}

\addtokomafont{title}{\rmfamily}

\title{The \pkg{shadow} package\thanks{This manual corresponds to \pkg{shadow.sty}~v1.3, dated~19 February 2003.}}
\author{Mauro Orlandini\thanks{\mail{orlandini@bo.iasf.cnr.it}}}
\date{19 February 2003}


\begin{document}

\maketitle

\begin{abstract}
  \noindent
  The command \cmd{\shabox} has the same meaning of the
  \LaTeX\ command \cmd{\fbox} except for the fact that a
  ``shadow'' is added to the bottom and the right side
  of the box. It computes the right dimension of the
  box, even if the text spans over more than one
  line; in this case a warning messagge is given.
\end{abstract}

There are three parameters governing:
%
\begin{enumerate}
\item the width of the lines delimiting the box:
  \cmd{\sboxrule}
\item the separation between the edge of the box and
  its contents: \cmd{\sboxsep}
\item the dimension of the shadow: \cmd{\sdim}
\end{enumerate}


\minisec{Sintax:}

\cmd{\shabox}\marg{text}\\
where \meta{text} is the text to be put in the
framed box. It can be an entire paragraph.

Adapted from the file \texttt{dropshadow.tex} by
\mail{drstrip@cd.sandia.gov}.
%
\begin{labeling}{V1.1}
\item[V1.1] Works in a double column environment.
\item[V1.2] When there is an online shadow box, it
  will be centered on the line (in V1.1 the
  box was aligned with the baseline).
  (Courtesy by Mike Piff)''
\item[V1.3] Added a number of missing \verb|%| signs\\
  no other cleanup done (FMi)
\end{labeling}

\end{document}
