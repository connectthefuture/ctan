\documentclass[a4paper]{ltxdoc}
\usepackage[latin1]{inputenc}
\usepackage{longtable}
\usepackage[commabeforerest,titleformat=commasep]{jurabib}
\interfootnotelinepenalty=10000 \raggedbottom
\newif\ifHtml
\newcommand{\mymarginpar}[1]{\marginpar{\raggedleft\textsf{#1}}}
\newcommand{\NEW}[1]{\marginpar{\raggedleft\textsf{#1~{\large NEW\,!}}}}
\newcommand{\CH} [1]{\marginpar{\raggedleft\textsf{#1~{\large CHANGED\,!}}}}
\newcommand{\REM}[1]{\marginpar{\raggedleft\textsf{#1~{\large REMOVED\,!}}}}
\makeatletter
  \@ifundefined{pdfoutput}{%
    \let\pdfoutput\@undefined
    \ExecuteOptions{dvips}%
  }{%
    \ifcase\pdfoutput
        \let\pdfoutput\@undefined
        \ExecuteOptions{dvips}%
    \else
        \usepackage[pdftex,colorlinks=true,plainpages=false,pdfpagelabels,linktocpage]{hyperref}%
        \hypersetup{%
          pdftitle={Documentation for jurabib package},%
          pdfauthor={Jens Berger},%
    }%
    \fi
  }%
\makeatother \pagestyle{headings}
\newcommand\bibentry[1]{\par\medskip{\renewcommand\jbauthorfont{\textbf}\renewcommand\jbauthorfnfont{\textbf}\noindent\qquad\fullcite{#1}}\par\medskip}
\AtEndDocument{\bibliography{jbtest,jbtesthu}}
\renewcommand{\contentsname}{Contents}
\providecommand{\url}[1]{\texttt{#1}}
\providecommand\href[2]{#2}
\providecommand\texorpdfstring[2]{#1}
\newenvironment{bibexample}{%
       \begin{description}%
            \setlength{\itemindent}{-2.5em}
            \setlength{\leftmargin}{2.5em}
       }{\end{description}}
\newenvironment{Options}%
 {\begin{list}{}{%
    \renewcommand{\makelabel}[1]{\texttt{##1}\hfill}%
    \setlength{\itemsep}{-.5\parsep}
    \settowidth{\labelwidth}{\texttt{xxx\space}}%
    \setlength{\leftmargin}{\labelwidth}%
    \addtolength{\leftmargin}{\labelsep}}%
  } {\end{list}}
\def\jbfileversion{v0.6}
\def\jbfiledate{\today}
\title{The \textsf{jurabib} Package}
\author{{\Large Jens Berger}\thanks{Translated into English by Maarten Wisse.}\\
        \href{mailto:jb@jurabib.org}{\texttt{jb@jurabib.org}}\\[1.8ex]
        {\small Stefan Ulrich}}
\date{\jbfiledate \qquad \textsf{\jbfileversion}\\[1ex]
      {\footnotesize CTAN:
       \href{ftp://ftp.dante.de/tex-archive/macros/latex/contrib/supported/jurabib/}{\texttt{macros/latex/contrib/supported/jurabib/}}}%
}%
\begin{document}
   \ifHtml
      \renewcommand{\href}[2]{\HCode{<a href=#1>}#2\HCode{</a>}}
      \makeatletter
      \let\DescribeMacro\@gobble
      \let\NEW\@gobble
      \let\CH\@gobble
      \let\REM\@gobble
      \let\mymarginpar\@gobble
      \makeatother
   \fi

\maketitle

   \begin{abstract}
   \noindent This package enables automated citation with \BibTeX{} for legal
   studies and the humanities. In addition, the package provides commands for
   specifying annotators in a commentary in a convenient way. Simplified
   formatting of the citation as well as the bibliography entry is
   also provided. It is possible to display the (short) title of a
   work only if an authors is cited with multiple works. Giving a full
   citation in the text, conforming to the bibliography entry, is
   supported. Several options are provided which might be of special
   interest for those outside legal studies---for instance,
   displaying multiple full citations. In addition, the format of last
   names and first names of authors may be changed easily. Cross
   references to other footnotes are possible. Language dependent
   handling of bibliography entries is possible by the special
   \texttt{language} field.
   \end{abstract}%

\tableofcontents

\section{Introduction}\setcounter{page}{1}

Up till now, support for citation according to the rules used in legal studies
was only available by the environment \texttt{jurabibliography} from
\textsf{jura.cls}. However, this environment does not work with \BibTeX.
\textsf{jurabib.sty} should solve this problem. The package primarily counts as
an extension to \textsf{jura.cls}, but it is built in such a way that it works
well together with all \LaTeX\ standard classes and also with the
\textsf{KOMA-Script}-classes \texttt{scrbook}, \texttt{scrreprt} and
\texttt{scrartcl}. Hence, it is possible to use the package together with
\textsf{alphanum} and the \texttt{book}-class, for writing a doctoral
dissertation.

\textsf{jurabib.sty} (re)defines \cs{[foot]cite} in such a way that it now
contains two optional arguments, so that possible annotators of a publication may
be specified in the second optional argument. Apart from this, the
\BibTeX-styles (\texttt{jurabib.bst} and \texttt{jureco.bst} which are
delivered with the package enable it to
\begin{enumerate}
\item automatically recognize the lastnames of the authors and cite
  these accordingly in short citations.

\item automatically generate short title information from article and
  periodical citations.

\item nevertheless explicitly specify a short form of the author's
  name or title.

\item (default) give the short title only if multiple works of the author are cited.
\end{enumerate}
The functionality described above is exclusively carried out by the
\BibTeX-styles \texttt{jurabib.bst}, \texttt{jurunsrt.bst}  and \texttt{jureco.bst}.

\section{Prequisities}
After processing \texttt{jurabib.ins} with \LaTeX{}, the following files should
be created at least:

\begin{itemize}
 \item \texttt{jurabib.sty}
 \item \texttt{jurabib.bst}
 \item \texttt{jureco.bst}
 \item \texttt{jurunsrt.bst}
 \item \texttt{jox.bst}
 \item \texttt{jbtesthu.bib}
 \item \texttt{jbtest.bib}
 \item \texttt{jurabib.cfg}
 \item some \texttt{.ldf} files
 \item some \texttt{jbtest*.tex} files
\end{itemize}
In a TDS-conforming \TeX-System the \texttt{.sty}-files and the
\texttt{.ldf}-files should be in \texttt{\slash [local]texmf\slash tex\slash
latex\slash jurabib}, the \texttt{.bst}-files in \texttt{\slash
[local]texmf\slash bibtex\slash bst\slash jurabib} and the \texttt{.bib}-files in
\texttt{\slash [local]texmf\slash bibtex\slash bib\slash jurabib}.
May be it's necessary to update a ``Filename Database''. If you are using
teTeX, run \texttt{texhash}, if you are using MiKTeX, run \texttt{initexmf -u}
on a DOS prompt or the appropriate graphical frontends. All mentioned files
could also be placed in the directory in which \LaTeX\ is executed. The files
\texttt{jbtest.tex} and \texttt{jbtest.bib} demonstrate the use of the
\textsf{jurabib}-package.

\texttt{jbtestbt.tex} serves as a demonstration of \textsf{jurabib.sty} in
cooperation with \textsf{bibtopic.sty} for generating bibliographies by topic.

\texttt{jbtestmb.tex} serves as a demonstration of \textsf{jurabib.sty} in
cooperation with \textsf{multibib.sty} for generating multiple bibliographies.

\texttt{jbtestcb.tex} serves as a demonstration of \textsf{jurabib.sty} in
cooperation with \textsf{chapterbib.sty} for generating bibliographies for each chapter.

\texttt{jbtestbu.tex} serves as a demonstration of \textsf{jurabib.sty} in
cooperation with \textsf{bibunits.sty} for generating multiple bibliographies too.

One enables the package in the following way:

\begin{quote}
 \cs{usepackage}\oarg{Options}\texttt{\{jurabib\}}

 \noindent{\footnotesize You can use \cs{jurabibsetup} in the preamble or in the
 configurations file:
 \begin{verbatim}
   \jurabibsetup{
     authorformat=smallcaps,
     commabeforerest,
     titleformat=colonsep,
     bibformat=tabular
   }
 \end{verbatim}

 You can place several values of some keys into a pair of braces:
 \begin{verbatim}
   \jurabibsetup{bibformat={tabular,ibidem,numbered}}
 \end{verbatim}}
 \end{quote}

At the point where the bibliography should be placed, the following should be
specified:

\begin{quote}
  \cs{bibliography}\marg{{\upshape\BibTeX}-database}
\end{quote}
followed by
\begin{quote}
  \cs{bibliographystyle}\marg{{\upshape\BibTeX}-stylefile}
\end{quote}

Now, the package is ready to work. When you are new to \BibTeX{} please
remember that for \BibTeX{} working properly, a sequence of one \LaTeX{} run, one
\BibTeX{} run, and two subsequent \LaTeX{} runs are required:

\begin{verbatim}
    latex data
   bibtex data
    latex data
    latex data
\end{verbatim}

\subsection{If you are working with a Windows-Distribution}
You have to (!) use a large version of \BibTeX{} (\texttt{bibtex8}) with a huge
amount of memory enabled. \marginpar{\raggedleft\textsf{\Huge !!!}} You can achieve that by using a command-line
parameter like |--huge| or |--wolfgang|, otherwise you will get error
messages like that:

\begin{verbatim}
   The style file: jurabib.bst
   5017: Sorry---you've exceeded BibTeX's wizard-defined function space 3000
   (That was a fatal error)
\end{verbatim}

This is problem can be solved by using one of the command line parameters of |bibtex8|:

\begin{verbatim}
    bibtex8 --wolfgang file
\end{verbatim}

\subsection{If you are working with a Unix/Linux-Distribution}
As a user of the te\TeX- or \TeX Live-Distribution you do not need |bibtex8|, 
because those Bib\TeX{} executables were compiled with enough memory enabled.
But it is not senseless to use |bibtex8|\,! If you are using author names or titles with umlauts, 
|bibtex8| is able to handle this by default, the normal Bib\TeX{} requires 
translation into |{\"u}| etc. At the moment only \TeX live comes with |bibtex8|, it is missed in te\TeX.
You would need to compile it yourself, you will find the sources on CTAN.

\section[An Example]{An Example\footnotemark[1]}

\footnotetext[1]{For all examples---if not explicitly specified
differently---the options \texttt{titleformat=commasep} and
\texttt{commabeforerest} as well as the command \cs{cite} have been used.}

Suppose one wants to cite a juridical work. Then, the first thing to do is to
add the work to the bibliography database:

\begin{verbatim}
  @BOOK{kkstrr,
    author      = {Kurt Kodal and Joachim Kr{\"a}mer},
    title       = {Stra{\ss}enrecht},
    shortauthor = {Kodal/Kr{\"a}mer},
    shorttitle  = {StrR},
    year        = 1995,
    address     = {M{\"u}nchen},
    edition     = {5},
    pages       = {30--34, \S~24}
  }
\end{verbatim}

Look at the fields \texttt{shortauthor} and \texttt{shorttitle}. These are
provided by the \BibTeX-styles belonging to the package.
\texttt{shortauthor} and \texttt{shorttitle} are the equivalent fields
available. There, the required short forms of the author and title
are provided. (on this, see section~\ref{auto} on page~\pageref{auto}).

\section{The Tools}
\subsection{The \texorpdfstring{\cs{cite}}{cite} command}

\DescribeMacro{\cite}
A citation is specified as usual:
\begin{verbatim}
   \cite[\S~12]{kkstrr}
\end{verbatim}
Instead of the standard layout:

\medskip

 [1, \S~12]

\medskip\noindent
the citation now looks like:

\cite[\S~12]{kkstrr}

The short form of the title (StrR) would only be cited when Kodal and
Kr{\"a}mer were cited with another work or the option \texttt{titleformat=all}
is given. Then, the  citation would look as follows:

\citetitle[\S~12]{kkstrr}

Because the command \cs{cite} is redefined by \textsf{jurabib.sty}, it has now
two optional arguments instead of one:

\medskip

\noindent\CH{0.6}
\fbox{\begin{minipage}{\textwidth}
{\Large\itshape Attention: since v0.6 the order of optional arguments has changed\,!}
\begin{quote}
\cs{cite}\oarg{page range}\marg{key}\\
\cs{cite}\oarg{annotator}{\ttfamily[]}\marg{key}\\
\cs{cite}\oarg{annotator}\oarg{page range}\marg{key}\\
 with \texttt{see}\\
\cs{cite}{\ttfamily[Vgl.]}\oarg{page range}\marg{key}
\end{quote}
The compatibility option \texttt{jurabiborder} lets you compile old documents:
\begin{quote}
\cs{cite}\oarg{page range}\marg{key}\\
\cs{cite}{\ttfamily[]}\oarg{annotator}\marg{key}\\
\cs{cite}\oarg{page range}\oarg{annotator}\marg{key}\\
with \texttt{see}\\
\cs{cite}\oarg{page range}{\ttfamily[see]}\marg{key}
\end{quote}
\end{minipage}}

\medskip

Here's a short table with the new syntax:

\medskip

{\small
\begin{tabular}{ll}
\hline
Source & Output \\
\hline
|\cite{broxbgb}| & \cite{broxbgb}\\
|\cite[p.~12]{broxbgb}| & \cite[p.~12]{broxbgb}\\
|\cite[Bassenge][]{broxbgb}| & \cite[Bassenge][]{broxbgb}\\
|\cite[Bassenge][p.~12]{broxbgb}| & \cite[Bassenge][p.~12]{broxbgb}\\
\hline
\end{tabular}}

\bigskip

So, a possible annotator will be specified as follows:

\begin{verbatim}
     \cite[Bassenge][\S~12]{kkstrr}
\end{verbatim}
 The following citation would emerge from this cite
command---provided that multiple works by Kodal and Kr{\"a}mer are cited:

\medskip

\cite[Bassenge][\S~12]{kkstrr} % Kodal/Kr�mer/Bassenge, StrR, \S~12.

\medskip

\noindent The bibliography entry belonging to this citation would then be: \par

\bibentry{kkstrr}

If it might happen that one does not want to give pageranges, margin numbers or
the like, and \emph{nevertheless} wants to specify an annotator, then, the first
optional argument should be left empty:

\begin{verbatim}
   \cite[Bassenge][]{kkstrr}% before 0.6:  \cite[][Bassenge]{kkstrr}
\end{verbatim}

\DescribeMacro{\citetitle} The command \cs{citetitle} basically behaves like
\cs{cite}, but may be used for explicitly citing by short title, independent
from the author's being cited with multiple works. Otherwise, the same goes for
\cs{citetitle} as for \cs{cite}.

\DescribeMacro{\cite*} This asterisk-from of the  \cs{cite} command cites
\emph{always} without title. In this case, it doesn't matter whether the
\texttt{shorttitle}-field has been provided in the database. The options
\texttt{citefull=all} too,  are deactivated in this case. (see
section~\ref{opt}). Note, however, that hereby, ambiguous\mymarginpar{!}
citations might occur, since the automatical specification of short title/full
title is deactivated for this citation. Therefore, this command should be used
with care.

\DescribeMacro{\citetitlefortype} This command can be used to declare a list of
publication types, for those the titles should appear always:

\begin{verbatim}
    \citetitlefortype{article,book, ... }
\end{verbatim}

\DescribeMacro{\citenotitlefortype} For use together with the option
\texttt{titleformat=all} to declare a list of publication types, for those the
titles shouldn't appear:

\begin{verbatim}
    \citenotitlefortype{article,book, ... }
\end{verbatim}

\subsection{The \texorpdfstring{\cs{footcite}}{footcite} commands}

\DescribeMacro{\footcite} \DescribeMacro{\footcite*}
\DescribeMacro{\footcitetitle}
  These commands are different from the \cs{cite} commands, in that they
  automatically generate a footnote and place a period at the end of
  it. A space before the \cs{footcite} commands is ignored:
\begin{verbatim}
  ...  Annahme. \footcite[Rn.~357]{medicus}
  ...  Annahme.\footcite[Rn.~357]{medicus}
\end{verbatim}

shows in both cases:

\medskip
\dots{}~Annahme.\footnote{Medicus, Rn.~357.}

\noindent However, a combination of multiple citations should be provided as
usual:
\begin{verbatim}
    \footnote{\cite[S.~13--34]{brox:bgb}; \cite[S.~24]{canaris}.}
\end{verbatim}

This results in:
\par\medskip
\dots{}~Annahme.\footnote{Brox, BGB, S.~13--34; Canaris, S.~24.} Here, people
should remember the period themselves, regrettably.

 If you are using some |\footcite| or |\footnote| commands \NEW{0.5f} subsequently,
 \textsf{jurabib} is able to put a comma between the superscripted footnote marks.
 This is the same behavior as known from the |multiple| option from the \textsf{footmisc} package. An example:

 \begin{verbatim}
    ... \footcite{brox:bgb}\footcite{brox:schr}\footcite{brox:ja}
 \end{verbatim}
 \vspace{-1em}
 produces:

 \medskip
 \dots{}\textsuperscript{1,2,3}

\subsection{The \texorpdfstring{\cs{fullcite}}{fullcite} commands}

\DescribeMacro{\fullcite} \DescribeMacro{\footfullcite} These commands generate
a full citation, i.e., the complete entry from the bibliography is inserted
here. A possibly present annotator will be placed before the citation and
separated from the author by ``in''.\footnote{This reflects the automatic
activation of the option \texttt{annotatorfirstsep=in} for  \emph{this} citation.} The
page range will be added at the end.

\subsection{The \texorpdfstring{\cs{nextcite}}{nextcite} commands}

\DescribeMacro{\nextciteshort} \DescribeMacro{\nextcitefull}
 With these commands you can determine with a comma-separated (without whitespaces\,!) list of keys, how a cited work should appear
 from now on:
\begin{verbatim}
    \nextciteshort{brox:bgb,canaris, ... }
\end{verbatim}
in a short or
\begin{verbatim}
    \nextcitefull{brox:bgb,canaris, ... }
\end{verbatim}
in a long kind.
\begin{verbatim}
    \nextcitenotitle{brox:bgb,canaris, ... }
\end{verbatim}
With that command all given works appear from now on without title. \emph{All
three commands are overriding the \cs{fullcite}- and \cs{cite*} commands\,!}

\smallskip
\noindent With \DescribeMacro{\nextcitereset}
\begin{verbatim}
    \nextcitereset{brox:bgb,canaris, ... }
\end{verbatim}
you can switch back to the normal citation kind (specified by the chosen
options).

\DescribeMacro{\citeswithoutentry} The new command \cs{citeswithoutentry} is
very similar to the \cs{nextcite} commands. You are able to specify a list of
works, which should not appear in the bibliography, but you can use all of the
\cs{cite} commands normally. Note, that the \cs{nextcitereset} command will not work here.

\subsection{The \texorpdfstring{\cs{citefield}}{citefield} command}
With that command you have access to the contents of the fields
\texttt{author}, \texttt{shortauthor}, \texttt{title}, \texttt{shorttitle},
\texttt{url}, \texttt{apy} (Address-Publisher-Year) and
\texttt{year}. You have to give the field name as the first mandatory argument,
the key as the second mandatory argument. Furthermore you can give a
page(range) with the optional argument:
\begin{quote}
   {\small\cs{citefield}\oarg{pagerange}\marg{field}\marg{key}}
\end{quote}
With the following entry
\begin{verbatim}
   @BOOK{brox:bgb,
    author      = {Hans Brox},
    title       = {Allgemeiner Teil des B{\"u}rgerlichen Gesetzbuches},
    shorttitle  = {BGB~AT},
    year        = 1996,
    language    = {german},
    address     = {K{\"o}ln, Berlin, Bonn, M{\"u}nchen},
    edition     = 20
   }
\end{verbatim}
we get with a \verb+\citefield{title}{brox:bgb}+: Allgemeiner Teil des
B{\"u}rgerlichen Gesetzbuches. \verb+\citefield[\S~23]{shorttitle}{brox:bgb}+
gives: BGB~AT,~\S~23. This command is especially for non-lawyers. It is
independent of all \textsf{jurabib}-specific automatisms, e.g. it is ignored by
the \texttt{ibidem}-options. If you are using the \textsf{hyperref}-package, a
link to the bibliography entry will be created.

\textsf{jurabib} is able to emulate some basic citation commands of the \textsf{natbib} package:

\medskip

\begin{tabular}{@{}lll@{}}\hline\\[-2ex]
|\[foot]citep{Kraft74}|       & $\rightarrow$ & (Kraft et al., 1937) \\
|\[foot]citet{Kraft74}|       & $\rightarrow$ & Kraft et al. (1937)\\
|\[foot]citealt{Kraft74}|     & $\rightarrow$ & Kraft et al. 1937\\
|\[foot]citealp{Kraft74}|     & $\rightarrow$ & Kraft et al., 1937\\
|\[foot]citeauthor{Kraft74}|  & $\rightarrow$ & Kraft et al.\\
\hline
\end{tabular}

\section{Automations}\label{auto}

\subsection{Empty \texttt{shortauthor}-field}

The package is now capable of automatically figuring out the lastname of the
author, in case of empty or failing \texttt{shortauthor}-field. Up till three
authors are typed out, separated by a slash. In case of more than three
authors, the name of the first author is typed out, along with ``et~al.''. So,
for instance, we leave the  \texttt{shortauthor}-field out in our example entry
and add a third author:
\begin{verbatim}
  @BOOK{kkstrr,
    author      = {Kurt Kodal and Joachim Kr{\"a}mer
                   and Hans Mustermann},
    title       = {Stra{\ss}enrecht},
    shorttitle  = {StrR},
    year        = 1995,
    address     = {M{\"u}nchen},
    edition     = {5.},
    pages       = {S.\,30--34, \S~24}
  }
\end{verbatim}
Then, the citation looks as follows:
\par\medskip
\qquad Kodal/Kr{\"a}mer/Mustermann, StrR, \S~12.
\par\medskip

\noindent Separation marks may be modified in the following way.
\begin{verbatim}
   \renewcommand{\jbbtasep}{ and } % bta  = between two authors sep
   \renewcommand{\jbbfsasep}{, }   % bfsa = between first and second author sep
   \renewcommand{\jbbstasep}{ and }% bsta = between second and third author sep
\end{verbatim}

shows, for instance, instead of the example shown above:

\par\medskip
\qquad Kodal, Kr{\"a}mer and Mustermann, StrR, \S~12.
\par\medskip

\noindent Separation marks in the bibliography remain untouched by this
modification. Those can be modified by the following commands:
\begin{verbatim}
   \renewcommand{\bibbtasep}{ and } % bta  = between two authors sep
   \renewcommand{\bibbfsasep}{, }   % bfsa = between first and second author sep
   \renewcommand{\bibbstasep}{ and }% bsta = between second and third author sep
and for the editors:
   \renewcommand{\bibbtesep}{ and } % bte  = between two editors sep
   \renewcommand{\bibbfsesep}{, }   % bfse = between first and second editor sep
   \renewcommand{\bibbstesep}{ and }% bste = between second and third editor sep
\end{verbatim}

If we add a fourth author (we save work by skipping the code), then the
citation looks like:
\par\medskip
\qquad Kodal et~al., StrR, \S~12.

\subsection{Empty \texttt{shorttitle}-field}

If the \texttt{shorttitle}-field is empty or nonexistent, the following happens:
The full title specified in the \texttt{title}-field will be
used\footnote{Exceptions are  \texttt{@ARTICLE}- and \texttt{@PERIODICAL}
entries.}, if (locally) the command \cs{[foot]citetitle} or (global) the option
\texttt{citefull=all} is being used \emph{or multiple works of the author have
been cited}.

In case one wants author's name only for a particular work---regardless what
kind of option have been activated---then, the command \cs{[foot]cite*} should
be used. This may lead to ambiguous citations, of which you will get a warning
in the  \texttt{.log}-file.

\begin{verbatim}
  @BOOK{kkstrr,
    author      = {Kurt Kodal and Joachim Kr{\"a}mer
                   and Hans Mustermann},
    title       = {Stra{\ss}enrecht},
    year        = 1995,
    address     = {M{\"u}nchen},
    edition     = {5.},
    pages       = {S.\,30--34, \S~24}
  }
\end{verbatim}
Result:
\par\medskip
\qquad Kodal/Kr{\"a}mer/Mustermann, Stra\ss{}enrecht, \S~12.
\par\medskip

An alleviation should also be the fact that, in case of Article and Periodical
citations, \textsf{jurabib} now automatically cites  (\meta{journal}
\meta{year}). Therefore, in these cases, specification of \texttt{shorttitle} is
no longer required---if no special requirements are present. In the following
example, neither
 \texttt{shortauthor} nor
\texttt{shorttitle} has been specified:
\begin{verbatim}
   @ARTICLE{brox:ja,
     author      = {Hans Brox},
     title       = {Die Anfechtung bei der Stellvertretung},
     journal     = {JA},
     year        = 1980,
     pages       = {S.\,449ff},
     edition     = {20.},
     address     = {M{\"u}nchen}
   }
\end{verbatim}

Nevertheless, \cs{citetitle}\verb+{brox:ja}+ results in:
\par\medskip
\qquad Brox, JA 1980.
\par\medskip
\noindent If you want something different, you can use the \texttt{short*}-fields:
\begin{verbatim}
   @ARTICLE{brox:ja,
     author      = {Hans Brox},
     title       = {Die Anfechtung bei der Stellvertretung},
     journal     = {JA},
     shortauthor = {Hans Brox},
     shorttitle  = {JA},
     year        = 1980,
     pages       = {S.\,449ff},
     edition     = {20.},
     address     = {M{\"u}nchen}
   }
\end{verbatim}

and you get:
\par\medskip
\qquad Hans Brox, JA.
\par\medskip

\DescribeMacro{author = } You should notice that the dynamic generations of the
juridical short titles is possible only, when the contents of the
\texttt{author}-fields in the \texttt{.bib}-file are \emph{absolutely
identical}\,! Suppose \textsc{Kodal} and \textsc{Kr{\"a}mer} have published a
second work, then, the entries in the database should look as follows:
\begin{verbatim}
   @BOOK{kkstrr,
      author     = {Kurt Kodal and Joachim Kr{\"a}mer},
      title      = {Stra{\ss}enrecht}
   }
\end{verbatim}

\begin{verbatim}
  @BOOK{kkanything,
      author     = {Kurt Kodal and Joachim Kr{\"a}mer},
      title      = {anything}
   }
\end{verbatim}

\section{Options}\label{opt}
The following default behavior of the package has been implemented: The author
and editor in the citation are shown in normal font, in the bibliography,
however, in bold font. The editor is put after the author, separated by a
slash. This default behavior can be modified by the options of the package.

\subsection{Tweaking the Format of the Citation in the Text}

\subsubsection{Format of the Author and Editor}

\begin{Options}
\item[authorformat=smallcaps] Puts author and annotator in small caps:
  \textsc{Kodal\slash Kr{\"a}mer\slash Bassenge}, StrR, \S~12

\item[authorformat=italic] Puts author and annotator in italic font:
  \textit{Kodal}\slash\textit{Kr{\"a}mer}\slash\textit{Bassenge}, StrR,
  \S~12

\item[authorformat=dynamic] The font of the author depends on whether
  an annotator is present or not.  If not, the authors retain their font
   as specified in the command \cs{jbactualauthorfont} as \cs{textit}
  (default): \textit{Kodal\slash Kr{\"a}mer}, StrR,
  \S~12.  If an annotator has been specified, the annotator is formatted
  according to the setting of \cs{jbactualauthorfont}, and the authors
  are formatted according to \cs{jbauthorfontifcoauthor} (
  \cs{normalfont}): Kodal\slash Kr{\"a}mer\slash\textit{Bassenge}, StrR,
  \S~12.

\item[authorformat=citationreversed] Using this option together with options
  \texttt{citefull=first}, \texttt{ibidem=name} and
  \texttt{ibidem=name\&title} firstnames appear before last names: Hans
  \textsc{Brox}: BGB~AT, S.~23. instead of \textsc{Brox}, Hans: BGB~AT, S.~23.

\item[authorformat=allreversed] By this option, the behavior mentioned
  in the previous option works also in the bibliography.

\item[authorformat=firstnotreversed] Sometimes, strange requirements
  are set on authors. Therefore, this option exists, which puts all
  authors in \meta{Firstname} \meta{Lastname}, \emph{apart from the
    first author}: \textsc{Kodal}, Kurt\slash Joachim \textsc{Kr{\"a}mer}\slash Hans \textsc{Mustermann}.

\item[authorformat=reducedifibidem] If this option is active together
  with \texttt{ibidem=name} only the last name of the author is shown
  in recurring citations.

\item[authorformat=and] Instead of the default slashes
  the authors will be separated by ``,'' and ``, and''.

\item[authorformat=year] The year will appear after the author name. \cs{jbyearaftertitle} puts the year after the title.
  Formatting is possible by using \cs{jbcitationyearformat}:
 \begin{verbatim}
  \renewcommand{\jbcitationyearformat}[1]{(#1)}
 \end{verbatim}

\item[authorformat=indexed] All authors (independently) are indexed.
  You have to use the \textsf{makeidx}-package correctly:
  \begin{verbatim}
    \usepackage{jurabib} % load before makeidx.sty!
    \usepackage{makeidx}
    \makeindex
    \begin{document}
     ...
    \printindex
    \end{document}
  \end{verbatim}
  This options works for |\nobibliography| too (since 0.51).

  If you want to have not cited authors (e.g. by using \cs{nocite}) indexed, you can use
 \cs{jbindexbib} in your preamble.

 To emphasize the authors in the \NEW{0.52} index, you can determine the font of the indexed authors via |\jbauthorindexfont|:
  \begin{verbatim}
    \renewcommand{\jbauthorindexfont}{\textit}% or \textsf,
                                              %    \textsc,
                                              %    \textbf
  \end{verbatim}
 If you are using a non-standard |.ist| file (|makeindex| style file), it is possible
 that you have to use |\jbmakeindexactual| to use the correct `actual' operator. The default is |@|.
 If your |.ist| file contains
  \begin{verbatim}
     actual '='
  \end{verbatim}
 you should use:
  \begin{verbatim}
     \renewcommand{\jbmakeindexactual}{\=}%
  \end{verbatim}
% I don't know why the backslash is necessary.

An often requested feature was \NEW{0.6} to provide some macros to tune the indexing of authors and editors.
Here it is, these commands should be used in the preamble as usual:

{\small\begin{tabular}{ll}
\hline
|\jbdonotindexeditors|          & Do not index editors\\
|\jbdonotindexauthors|          & Do not index authors\\
|\jbdonotindexorganizations|    & Do not index organizations\\
|\jbindexolyfirsteditors|       & Do index editors only for first citation\\
|\jbindexonlyfirstauthors|      & Do index authors only for first citation\\
|\jbindexolyfirstorganizations| & Do index organizations only for first citation\\
\hline
\end{tabular}}

\medskip

\item[authorformat=abbrv] Since version 0.5 the \BibTeX-styles
    \texttt{jurplain.bst} and \texttt{jurabbrv.bst}
    are obsolete, the abbreviated form can be determined by using this option.

\item[annotatorformat=italic (formerly known as: coauthorformat=italic)] Puts the annotator in italics: Kodal\slash Kr{\"a}mer\slash \textit{Bassenge}.

\item[annotatorformat=normal (formerly known as: coauthorformat=normal)] Puts the annotator upright:
  \textit{Kodal}\slash \textit{Kr{\"a}mer}\slash Bassenge.

\item[round] Round brackets around (non-footnote-) citation:
  (\textsc{Kodal}\slash \textsc{Kr{\"a}mer}).

\item[square] Square brackets around (non-footnote-) citation:
  [\textsc{Kodal}\slash \textsc{Kr{\"a}mer}].

\item[superscriptedition=all] Places a superscripted edition number in the citation.
 Some examples:
 \par
 Kodal/Kr{\"a}mer$^{3}$, \S~12.\\
 Kodal/Kr{\"a}mer, Stra"senrecht$^{3}$, \S~12.\\
 Kodal/Kr{\"a}mer$^{3}$--Bassenge, \S~12.\\
 Kodal/Kr{\"a}mer--Bassenge, Stra\ss{}enrecht$^{3}$, \S~12.\\
 Bassenge in Kodal/Kr{\"a}mer$^{3}$, \S~12.
 \par\medskip

\item[superscriptedition=commented] Same like above, but only for \texttt{@COMMENTED}.

\item[superscriptedition=switch] With the new field \texttt{ssedition} you can specify
  the appearence of the superscripted edition number explicitly:
   \begin{verbatim}
   @COMMENTED{soergel,
      author      = {Hans Theodor Soergel},
      title       = {Kommentar zum B{\"u}rgerlichen Gesetzbuch},
      address     = {Stuttgart, Berlin, K{\"o}ln, Mainz},
      year        = 1987,
      edition     = 12,
      ssedition   = 1, <=== !
   }
   \end{verbatim}

\item[superscriptedition=kerning] This option\NEW{0.51h} replaces the command |\jbsseditionkerned| and does some kerning:

Kodal/Kr{\"a}mer$^{3}$\kern-1ex, \S~12.\\
Kodal/Kr{\"a}mer, Stra{\ss}enrecht$^{3}$\kern-1ex, \S~12.\\
Kodal/Kr{\"a}mer$^{3}$--Bassenge, \S~12.\\
Kodal/Kr{\"a}mer--Bassenge, Stra{\ss}enrecht$^{3}$\kern-1ex, \S~12.\\
Bassenge in Kodal/Kr{\"a}mer$^{3}$\kern-1ex, \S~12.
\par\medskip
\item[superscriptedition=bib/address] superscripted edition numbers \CH{0.51j} in the bibliography (before address).
\item[superscriptedition=year] superscripted edition numbers \NEW{0.51j} in the bibliography (before year).
\item[superscriptedition=multiple] This option enables \NEW{0.51h} the output of a superscripted edition number
for the case that different editions of the same work were cited. The output of the shorttitle is suppressed.
\end{Options}

If a coauthor was given via the second optional argument, the superscripted edition number
appears after the coauthor, as long as you are using the default or one of the |annotatorlastsep| options:

\medskip

Kodal/Kr{\"a}mer--Bassenge$^{3}$, \S~12.

\medskip

If you like to have the superscripted number immediately after the author, please use the
following command in the preamble of your document:

\begin{verbatim}
    \jbsuperscripteditionafterauthor
\end{verbatim}

Should result in:

\medskip

Kodal/Kr{\"a}mer$^{3}$--Bassenge, \S~12.

\begin{Options}
\item[biblikecite] Bibliography will be automatically formatted like the citations (as far as possible\,!).
\item[edby] (Only for \texttt{@INCOLLECTION}!) The sequence ``Lipcoll, David~J. (ed.)'' will be changed
 to ``edited by Lipcoll, David~J.''\par
For redefinitions please use:
   \begin{verbatim}
   \AddTo\bibsgerman{%
       \def\edbyname{ed. by}%
    }
   \end{verbatim}
\item[endnote] The \texttt{note} field appears at the end of the bibliographic entry, and if you are using
|dotafter=bibentry|, \emph{after} the closing period.
\end{Options}

\DescribeMacro{\jbauthorfont}
\DescribeMacro{\jbcoauthorfont}
 If these options do not generate the desired result, one can realize
 many things by redefining several commands.

The formats of the authors and annotators are directed by the commands
 \cs{jbauthorfont} and \cs{jbcoauthorfont} and may be modified in the
 following way (these examples presuppose the default values):
\begin{verbatim}
   \renewcommand{\jbauthorfont}{\textit}
   \renewcommand{\jbcoauthorfont}{\textsl}
\end{verbatim}

\DescribeMacro{\jbactualauthorfont} \DescribeMacro{\jbauthorfontifcoauthor}
Analogically, the following commands exist \emph{only} if option
\texttt{authorformat=dynamic} is active:
\begin{verbatim}
  \renewcommand{\jbactualauthorfont}{\textsc}
  \renewcommand{\jbauthorfontifcoauthor}{\textsl}
\end{verbatim}

These redefinitions should be placed in the preamble of the \TeX-file, i.e.
before \cs{begin}\verb+{document}+. One should notice that  \cs{text} commands
should be used, for example, \cs{textit}, \cs{textbf} etc.), and \emph{not}
those starting with \texttt{series}, \texttt{family} or ending with
\texttt{shape}  (declaration form, for instance, \cs{bfseries}, \cs{slshape},
\cs{sffamily})!

\subsubsection{Title Format}

\begin{Options}
\item[titleformat=italic] Puts title in italics: Kodal\slash
  Kr{\"a}mer\slash Bassenge, \textit{StrR}, \S~12.

\item[titleformat=all] Provides short titles in \emph{all} cases,
  regardless whether an author has been cited with multiple works.

\item[titleformat=colonsep] Separates author and title by a colon
  (only if a title is displayed):
  Kodal\slash Kr{\"a}mer\slash Bassenge: StrR, \S~12.

\item[titleformat=commasep] Separates author and title by a comma:
  Kodal\slash Kr{\"a}mer\slash Bassenge, StrR, \S~12.

\item[titleformat=noreplace] You can disable globally the
   default replacement of the missing \texttt{shorttitle} by \texttt{title}.
   See also the \cs{cite*}- and \cs{nextcitenotitle} commands.
\end{Options}

\DescribeMacro{\jbtitlefont} For changing the format of the short title, the
command \cs{jbtitlefont} is available, which may be tweaked analogically:

\begin{verbatim}
   \renewcommand{\jbtitlefont}{\textit}
\end{verbatim}

\DescribeMacro{\jbhowsepbeforetitle} For options \texttt{titleformat=commasep}
and \texttt{titleformat=colonsep} is specific formatting of the separation
marks possible by:
\begin{verbatim}
   \renewcommand{\jbhowsepbeforetitle}{; } .
\end{verbatim}

One of both options mentioned above should be active in this case.

\subsubsection{Separation of the annotator}

As explained above, the format of the annotator after the author(s) and the
separation by a slash have been set by default. These may be modified by
options.

\begin{Options}
\item[annotatorlastsep=divis (formerly known as: colastsep=divis)] This option changes the slash as separation
  mark to an (en)-dash: Kodal\slash Kr{\"a}mer--Bassenge StrR, \S~12.

\item[annotatorfirstsep=in (formerly known as: cofirstsep=in)] The annotator appears first in the citation,
  followed by `` in ''\,: Bassenge in Kodal\slash Kr{\"a}mer
  StrR, \S~12.

\item[annotatorfirstsep=comma (formerly known as: cofirstsep=comma)] This option works like \texttt{annotatorfirstsep=in},
  but  `` in '' is now replaced by a comma: Bassenge,
  Kodal\slash Kr{\"a}mer StrR, \S~12.
\end{Options}

\subsubsection{Behaviour in Recurring Citations}

 \begin{Options}
 \item[ibidem or ibidem=strict] If an author is cited several times
   after each other, it may be useful to replace the short citation by
   the shorthand ``ibid.''  \emph{The use of this shorthand is not
     accepted by everyone because it does not improve readability of
     the text.} In this respect, \textsf{jurabib} has been configured
   in such a way that `ibid' will be displayed only if the recurrence
   immediately follows the previous citation, and the recurring
   citation is not the first on the actual page (see the examples).
   By contrast, in the humanities the use of ``ibid.'' is often compulsory:
   it can be an error to repeat a citation in full. Because of this fact,
   \texttt{ibidem=strict} is the default for the humanities.

\item[ibidem=nostrict] If one wants to suppress the settings mentioned
  above, one could use this option, which allows the shorthand be
  placed as the first on a page. One should be careful with this
  option. It is useful only if one uses basically one author in the
  document, so that it is clear to whom one refers.

\item[ibidem=strictdoublepage] Granted, this option is of an academic
  nature, because it quite carefully sorts out whether an first
  citation or short citation occurs on the facing page of the
  recurring citation.(Many thanks to \textsc{Stefan \mbox{Ulrich}}) If
  so, ibidem is allowed as first citation on the page. If not, short
  citation is used for recurring citations. Of course, this is useful
  only when two-side printing is concerned, i.e. if the
  \texttt{book}-class is used or the class option
  \texttt{twoside}. Compare the following survey:
\end{Options}%

\begin{center}
\noindent{\small
\begin{tabular}{@{}llll@{}}\hline\\[-2ex]
{\footnotesize\cs{footcite}|..|}& {\footnotesize|ibidem|/|ibidem=strict|}&
{\footnotesize|ibidem=strictdoublepage|} &
{\footnotesize|ibidem=nostrict|}\\\hline\\[-2ex]
|..[\S~12]{erm}| & $^{1}$\textsc{Erman}, \S~12. & $^{1}$\textsc{Erman}, \S~12.              & $^{1}$\textsc{Erman}, \S~12.\\
|..[\S~12]{erm}| & $^{2}$ibid.                  & $^{2}$ibid.                               & $^{2}$ibid.\\
|..[\S~20]{erm}| & $^{3}$ibid., \S~20.          & $^{3}$ibid., \S~20.                       & $^{3}$ibid., \S~20.\\
|..{mueko}|      & $^{4}$\textsc{M{\"u}Ko}.     & $^{4}$\textsc{M{\"u}Ko}.                  & $^{4}$\textsc{M{\"u}Ko}.\\
|..[\S~12]{erm}| & $^{5}$\textsc{Erman}, \S~12. & $^{5}$\textsc{Erman}, \S~12.              & $^{5}$\textsc{Erman}, \S~12.\\
|..[\S~12]{erm}| & $^{6}$ibid.                  & $^{6}$ibid.                               & $^{6}$ibid.\\[.8ex]
\multicolumn{4}{@{}c@{}}{\emph{Pagebreak from odd (right) to even (left)}}\\[1.6ex]
|..[\S~12]{erm}| & $^{7}$\textsc{Erman}, \S~12. & $^{7}$\textsc{Erman}, \S~12. $\longleftarrow$ \textsf{!!!}  & $^{7}$ibid. $\longleftarrow$ \textsf{!!!}\\
|..[\S~12]{erm}| & $^{8}$ibid.                  & $^{8}$ibid.                               & $^{8}$ibid.\\[.8ex]
\multicolumn{4}{@{}c@{}}{\emph{Pagebreak from even (left) to odd (right)}}\\[1.6ex]
|..[\S~12]{erm}| & $^{9}$\textsc{Erman}, \S~12. & $^{9}$ibid. $\longleftarrow$ \textsf{!!!} & $^{9}$ibid. $\longleftarrow$ \textsf{!!!}\\
|..[\S~12]{erm}| & $^{10}$ibid.                 & $^{10}$ibid.                              & $^{10}$ibid.\\
\hline
\end{tabular}}
\end{center}

\DescribeMacro{\noibidem} Disables the \texttt{ibidem}-mechanism for the next
(\emph{and only for the next\,!}) citation.

\subsubsection{Remaining Options}

\begin{Options}
\item[commabeforerest] If active, a comma will be placed before page
  ranges, margin numbers or the like: Kodal\slash
  Kr{\"a}mer\slash Bassenge: StrR, \S~12.
\item[silent (formerly known as: \cs{jbsilent})] Suppresses all \textsf{jurabib} warnings.
\end{Options}

\subsection{Tweaking the Format of the Bibliography}

\subsubsection{Font Commands}

\DescribeMacro{\biblnfont}
\DescribeMacro{\bibelnfont}
\DescribeMacro{\bibfnfont}
\DescribeMacro{\bibefnfont}
\DescribeMacro{\bibtfont}
\DescribeMacro{\bibbtfont}
\DescribeMacro{\bibjtfont}
\DescribeMacro{\bibapifont}
\DescribeMacro{\bibsnfont}
   The possibilities for formatting are limited to the
   modification of font formats of particular parts of entries in the
   bibliography. For this purpose, the following commands are
   available. \cs{biblnfont}, for formatting the last name of the
   author and \cs{bibelnfont} for the last name of the editor. \cs{bibfnfont}, for formatting the first name of
   author and \cs{bibefnfont} for the first name of editor. \cs{bibtfont}, for modification of the title of
   books etc. \cs{bibbtfont}, for formatting the title of collections
   of essays.  \cs{bibjtfont}, for formatting the title of the journal
   in article entries. With \cs{bibsnfont} you can determine the appearence of the series name.

Additionally available are \cs{bibapifont}, for formatting the title of an
article or essay in a collection. This command is active for the entry types
\texttt{@ARTICLE}, \texttt{@PERIODICAL} and \texttt{@INCOLLECTION}. The default
format reflects the following definitions:

\begin{verbatim}
   \renewcommand{\biblnfont}{\bfseries}
   \renewcommand{\bibfnfont}{\bfseries}
   \renewcommand{\bibtfont}{}
   \renewcommand{\bibbtfont}{}
   \renewcommand{\bibjtfont}{}
   \renewcommand{\bibapifont}{}
\end{verbatim}
Modification is possible analogically to the examples shown above. You should
notice here, too, that commands should be used which start with \cs{text} (Font
switch commands with arguments, for instance, \cs{textit}, \cs{textbf} etc.),
and \emph{not} those starting with \texttt{series}, \texttt{family} or ending
with \texttt{shape}  (declaration form, for instance, \cs{bfseries},
\cs{slshape}, \cs{sffamily})!

\subsubsection{Options for the Bibliography}

 \begin{Options}
  \item[bibformat=nohang] Hereby, the default indent of the second and
    following lines in a bibliography entry will be suppressed.
\noindent If one wants to set the indent to a certain length, one should put
the following in the preamble of one's document:
\begin{verbatim}
  \setlength{\jbbibhang}{1.5em}
\end{verbatim}

An indent of  2.5\,em is the default.
 \item[bibformat=tabular] In this case, the bibliography will be
   displayed in two-column tabular form. The authors appear in the
   left column, and the remainder of the entry in the right column. The
   width of the columns may be customized by the following commands
   (the values indicated are the defaults):
\begin{verbatim}
   \renewcommand{\bibleftcolumn}{\textwidth/3}
   \renewcommand{\bibrightcolumn}{\textwidth-\bibleftcolumn-1cm}
\end{verbatim}
Modification of the alignment within the columns is also possible by redefining
the following (defaults are displayed):
\begin{verbatim}
   \renewcommand{\bibleftcolumnadjust}{\raggedright}
   \renewcommand{\bibrightcolumnadjust}{\raggedright}
\end{verbatim}
 For better hyphenation, use of the package \mbox{\textsf{ragged2e.sty}} is \emph{highly} recommended:
 \begin{verbatim}
  \usepackage{ragged2e}
 \end{verbatim}
 Loading the package \textsf{ragged2e} is sufficient. Redefinition of the necessary commands is
 handled automatically.
\item[bibformat=numbered] This option results in a numbered bibliography. The format of the number can be determined by redefining |\bibnumberformat|:
\begin{verbatim}
  \renewcommand{\bibnumberformat}[1]{(#1)}
\end{verbatim}
\item[bibformat=ibidem] Replaces recurring authors name(s) by a dash (or whatever you want),
 if multiple works of the author appearing in the bibliography. \textsf{jurabib} is built in such a way,
 that the replacement is suppressed, when a recurring entry lies on top of a page. Note that
 it may be necessary to do several (up to four or more\,!) \LaTeX-runs to make this mechanism work.

 Modifications can be done by using (only an example\,!):
\begin{verbatim}
  \renewcommand{\bibauthormultiple}{The same}
\end{verbatim}
\item[lookforgender] Uses the |gender|-field given in the |.bib| file.With these field you are able to determine gender-specific abbreviations while using |bibformat=ibidem|. Following abbreviations are defined:
\end{Options}
\noindent{\footnotesize
   \begin{tabular}{@{}llllll@{}}
    \hline\\[-2ex]
    Abbrv. & Meaning      & Citation & Defined by: & Bibliography & Defined by:\\
    \hline\\[-2ex]
    |sf| & single female  & Idem/idem     & |\idem[S,s]fname| & Idem/idem   & |\bibidem[S,s]fname|\\
    |sm| & single male    & Idem/idem     & |\idem[S,s]mname| & Idem/idem   & |\bibidem[S,s]mname|\\
    |pf| & plural female  & Idem/idem     & |\idem[P,p]fname| & Idem/idem   & |\bibidem[P,p]fname|\\
    |pm| & plural male    & Idem/idem     & |\idem[P,p]mname| & Idem/idem   & |\bibidem[P,p]mname|\\
    |sn| & single neutrum & Idem/idem     & |\idem[S,s]nname| & Idem/idem   & |\bibidem[S,s]nname|\\
    |pn| & plural neutrum & Idem/idem     & |\idem[P,p]nname| & Idem/idem   & |\bibidem[P,p]nname|\\
    \hline\\[-2ex]
   \end{tabular}}

\medskip

\noindent If you want to redefine the idem replacement for a single women author:
\begin{verbatim}
     \AddTo\bibsenglish{%
           \renewcommand\idemSfname{Eadem}%
           \renewcommand\idemsfname{eadem}%
           \renewcommand\bibidemSfname{Eadem}%
           \renewcommand\bibidemsfname{eadem}%
     }
\end{verbatim}


\begin{Options}
\item[bibformat=ibidemalt] An alternative format of the bibliography, especially for German law students.
\item[bibformat=compress] The bibliography will be printed more compact, i.\,e. the vertical space between the items will be reduced.
\item[bibformat=raggedright] The bibliography will be printed with right ragged margin. The use is recommended especially
   when using |bibformat=tabular| too or when you are using a small textwidth.
\item[annote] The content of the |annote| field will be printed (only for the bibliography\,!).
    It is possible -- similar to \textsf{natbib}~-- to leave the |annote| field empty
    and let \textsf{jurabib} \NEW{0.51s} use an annote file instead. This file will be used
    if it is named like the database entry key with |.tex| extension.
    Nothing will be printed out, if |annote| field is empty and no annote file exists.

\DescribeMacro{\bibAnnotePath}
    With |\bibAnnotePath| you can specify a path to annote files.
    The syntax is the same as for |\graphicspath|: |\bibAnnotePath{{annotes/}}|
    uses the annote files from subdirectory |annotes| of the current directory.
\item[super] will convert all\NEW{0.6} |\cite| commands into |\footcite|'s,
\item[config=\meta{file}] you are able to use \NEW{0.6} several |.cfg| files. This option will load the named file. Please do not add the extension |.cfg|\,!
\item[dotafter=bibentry (formerly known as: \cs{jbdotafterbibentry})] places a dot at the end of each entry in the bibliography.
\item[dotafter=endnote (formerly known as: \cs{jbdotafterendnote})]  places a dot at the end of each endnote (if you are using \textsf{endnotes.sty}).
\end{Options}

\subsubsection{Further Possibilities for Customisation}

\DescribeMacro{\bibbtsep} \DescribeMacro{\bibjtsep} Because of usually
different opinions about what citations should look like, the commands
\cs{bibbtsep} and \cs{bibjtsep} are available. They function as
``\textbf{b}ook\textbf{t}itle \textbf{sep}aration'' and
``\textbf{j}ournal\textbf{t}itle \textbf{sep}aration''.
\begin{bibexample}
\item \textbf{Brinkmann, Franz~Josef:} Der Zugang der
  Willenserkl{\"a}rungen, M{\"u}nsterische Beitr{\"a}ge zur Rechtswissenschaft,
  Bd.~3 Berlin, 1984
\end{bibexample}
If one, for instance, wants  ``in~'' preceding book titles or journal titles,
then, the following redefenitions are required:
\begin{verbatim}
  \renewcommand{\bibbtsep}{in }
  \renewcommand{\bibjtsep}{in }
\end{verbatim}

After redefinition, the following comes out:
\begin{bibexample}
\item \textbf{Brinkmann, Franz~Josef:} Der Zugang der Willenserkl{\"a}rungen, in M{\"u}nsterische
Beitr{\"a}ge zur Rechtswissenschaft, Bd.~3 Berlin, 1984
\end{bibexample}

\DescribeMacro{\bibansep} \DescribeMacro{\bibatsep} \DescribeMacro{\bibbdsep}
The separation marks between authors, titles and between address and
month\slash\ year are a matter of discussion. This problem is countered by the
commands \cs{bibansep} (\textbf{a}fter \textbf{n}ame \textbf{sep}aration),
\cs{bibatsep} (\textbf{a}fter \textbf{t}itle \textbf{sep}aration) and
\cs{bibbdsep} (\textbf{b}efore \textbf{d}ate \textbf{sep}aration).

If one, for instance, wants no colon after the author's name, a period after
the title and no comma between address and year, one could accomplish this by
the following redefinitions:
\begin{verbatim}
  \renewcommand{\bibansep}{}
  \renewcommand{\bibatsep}{.}
  \renewcommand{\bibbdsep}{}
\end{verbatim}

The result is as follows:
\begin{bibexample}
\item \textbf{Brinkmann, Franz~Josef} Der Zugang der
  Willenserkl{\"a}rungen. M{\"u}nsterische Beitr{\"a}ge zur Rechtswissenschaft,
  Bd.~3 Berlin 1984
\end{bibexample}

\subsubsection{Cited as \ldots}

\DescribeMacro{howcited=normal} \DescribeMacro{howcited=multiple}
\DescribeMacro{howcited=compare} \DescribeMacro{howcited=all} The
\texttt{howcited}-options put (under certain conditions) a commentary behind
selected entries, which indicates how the work has been cited in the text. The
commentary changes dynamically, just as we know that from the citation itself,
i.e. the form in the bibliography always reflects the \emph{final} form of the
citation in the text. In case of article and periodical citations, the default
is that no indication of the way of citation is given---this is the default for
all \texttt{howcited}-options---except of \texttt{howcited=all}---because the
way of citation is in these cases always author's name and journal
title.\footnote{Special thanks to \textsc{Christian Meyn} for this suggestion.}

If you want to have the howcited-remark for articles and periodicals too:
\begin{verbatim}
      \makeatletter
      \jb@allow@howcited@art@periodtrue
      \makeatother
\end{verbatim}
These options can be enabled in the following way:

\begin{Options}
\item[howcited=normal]\DescribeMacro{howcited=}
    The option \texttt{howcited=normal} displays the remark
    ``(cited: \meta{author})'', if in the \texttt{.bib}-file the
    field \texttt{howcited} has been specified.\footnote{By this option,
    one can avoid unwanted automation of howcited remarks which may occur
    when using the option \texttt{howcited=compare}. Thus, it is possible
    to specify for each work whether it should have a howcited remark and
    if it must have, what it should look like.}

    Two possibilities exist for utilizing this field. The field functions as a switch when putting
    \texttt{1} in the field. This results in displaying the original citation from the text in the
    bibliography. An example:
    \begin{verbatim}
    @BOOK{enne:nipp,
      author      = {Ludwig Enneccerus and Hans Carl Nipperdey},
      title       = {Allgemeiner Teil des B{\"u}rgerlichen Rechts},
      year        = 1960,
      volume      = 1,
      address     = {T{\"u}bingen},
      edition     = 15,
      howcited    = 1
    }
    \end{verbatim}
    Displays the following (use of  \cs{[foot]cite} presupposed):
    \begin{bibexample}
    \item \textbf{Enneccerus, Ludwig\slash Nipperdey, Hans~Carl:}
       Allgemeiner Teil des B{\"u}rgerlichen Rechts. Bd.~1, 15.~Auf\/lage,
       T{\"u}bingen 1960 (cited: Eneccerus\slash Nipperdey)
    \end{bibexample}
    In case you want to let the howcited remark differ from the original citation, then, you should
    simply put in the field what you want to be displayed. An example:
    \begin{verbatim}
    @BOOK{enne:nipp,
      author      = {Ludwig Enneccerus and Hans Carl Nipperdey},
      title       = {Allgemeiner Teil des B{\"u}rgerlichen Rechts},
      year        = 1960,
      volume      = 1,
      volumetitle = {zweiter Halbband},
      address     = {T{\"u}bingen},
      edition     = {15.},
      howcited    = {Enneccerus/Nipperdey, B{\"u}rgerliches Recht}
    }
    \end{verbatim}
    Shows:
    \begin{bibexample}
    \item \textbf{Enneccerus, Ludwig\slash Nipperdey, Hans~Carl:}
       Allgemeiner Teil des B{\"u}rgerlichen Rechts. Bd.~1, zweiter Halbband, 15.~Auf\/lage,
       T{\"u}bingen 1960 (cited: Enneccerus\slash Nipperdey, B{\"u}rgerliches Recht)
    \end{bibexample}
    In order to ensure consequent layout when using options which affect font format of author's
    names, one could insert fontcommands in the field.

\item[howcited=compare]\DescribeMacro{howcited=compare}
    Displays the additional ``(cited: \meta{author})'' only,
    \emph{if an entry contains the field \texttt{shorttitle}}, \emph{and} the information in this field
    differs from that specified in \texttt{title}. The field  \texttt{howcited} in the
    \texttt{.bib}-file will now be ignored. Decisive for displaying the remark is now only the
    difference between \texttt{shorttitle} and \texttt{title}\,! That is the case in the following
    example, cited with \cs{[foot]citetitle}:
    \begin{verbatim}
    @BOOK{kkstrr,
      author      = {Kurt Kodal and Joachim Kr{\"a}mer},
      title       = {Stra{\ss}enrecht},
      shorttitle  = {StrR},
      year        = 1995,
      address     = {M{\"u}nchen},
      edition     = {5.},
      pages       = {S.\,30--34, \S~24}
    }
    \end{verbatim}
    \begin{bibexample}
    \item \textbf{Kodal, K.\slash Kr{\"a}mer, J.:} Stra\ss{}enrecht, 5.~Auf\/lage M{\"u}nchen, 1995
          (cited: Kodal\slash Kr{\"a}mer, StrR)
    \end{bibexample}

\item[howcited=multiple]\DescribeMacro{howcited=multiple}
    This option places ``(cited: \meta{author})'',
    if more than one work of an author is cited. There is an exception for commentaries by
    default (the remark will be displayed always), but this could be changed
    (\cs{jb@@arg}$=$\texttt{1}, if more than one work of the same author is cited):
    \begin{verbatim}
    \makeatletter
    \renewcommand{\jb@make@howcited@multiple}{%
       \jb@suppress@dot@for@howcitedtrue
       \ifthenelse{\equal{\jb@@arg}{1}}{%
         \jb@make@howcited
         \jb@make@comment@howcited
         \jb@make@artperiod@howcited
       }{%
         \let\bibhowcited\@empty
         \let\bibcommenthowcited\@empty
         \let\bibartperiodhowcited\@empty
       }%
    }%
    \makeatother
    \end{verbatim}
 \item[howcited=all]\DescribeMacro{howcited=all}
   The \texttt{howcited}-remark appears for all entries.
\end{Options}

The default for the remark is ``(cited: \meta{author})''. Customization is
possible by the following commands:
\begin{verbatim}
   \newcommand*{\bibhowcitedprefix}{-- as }
   \newcommand*{\bibhowcitedsuffix}{ cited.}
\end{verbatim}

Notice the spaces used. This results in:
\begin{bibexample}
\item \textbf{Kodal, K.\slash Kr{\"a}mer, J.:} Stra\ss{}enrecht, 5.~Auf\/lage M{\"u}nchen, 1995
---as Kodal\slash Kr{\"a}mer, StrR cited.
\end{bibexample}
If you want to put something in the bibliography, but it has not been cited in
the text, you can use command \cs{nocite}:
\begin{verbatim}
   \nocite{kkstrr}
\end{verbatim}

Or, in order to put \emph{all} works which are not cited in the bibliography:
\begin{verbatim}
   \nocite{*}
\end{verbatim}

Then, the commentary contains the actual meaning of the command
\cs{bibnotcited}. This command is empty by default. It might easily be modified
by:
\begin{verbatim}
   \renewcommand{\bibnotcited}{(not cited)}
\end{verbatim}

Shows the following:
\begin{bibexample}
\item \textbf{Kodal, K.\slash Kr{\"a}mer, J.:} Stra\ss{}enrecht, 5.~Auf\/lage M{\"u}nchen, 1995 (not cited)
\end{bibexample}

Of course, for this to happen, one of the two option \texttt{howcited=normal}
or \texttt{howcited=compare} should be active.

Now \NEW{0.51} all useful combinations of |howcited| options are possible.

Remark: using these options without using the option \texttt{citefull=all} or
for some citations the command \cs{[foot]citetitle}, seems not useful.


\subsubsection{More Entry Fields and Types}

\DescribeMacro{url =} Although citing the World Wide Web is not widely
practised in law studies, I nevertheless added an extra field \texttt{url}.

\DescribeMacro{urldate =} A field \texttt{urldate} is available, which enables one to specify the date on which
one visited the link which has been specified in \texttt{url}. The default for
this command is  ``visited on '' and may be customized by redefining:
\begin{verbatim}
   \AddTo\bibsenglish{\renewcommand*{\urldatecomment}{accessed on }} .
\end{verbatim}
The separation between URL and |\urldatecomment| is configurable and represented by |\bibbudcsep|.
It is defined as "` -- "' by default.

\DescribeMacro{\biburlprefix} \DescribeMacro{\biburlsuffix}
\DescribeMacro{\biburlfont} One can customize the format of the
\texttt{url}-field in two ways. On
   the one hand, by modifying \cs{biburlprefix}, which inserts the
   prefix ``URL:'' before the link. On the other hand, by modifying
   the command \cs{biburlfont}, which specifies the font format of the
   link. Defaults are:

\begin{verbatim}
   \renewcommand*{\biburlprefix}{\jblangle{}URL:}
   \renewcommand*{\biburlsuffix}{\jbrangle{}}
\end{verbatim}

Customisation is analog to the other commands by using \cs{renewcommand}. In
order to properly break URLs and properly display characters like \verb+~+ and
\verb+_+, \textsf{jurabib} is loading the \textsf{url}-package.
The |\biburlfont| command was changed in version \CH{0.51}0.51. You can modify the url font
with the following syntax (only these four values are possible\,!)
\begin{verbatim}
   \biburlfont{tt}   % typewriter
   \biburlfont{rm}   % roman
   \biburlfont{sf}   % sans serif
   \biburlfont{same} % same as text
\end{verbatim}

\DescribeMacro{@WWW}
New entry type for URL's. Required is only
\texttt{url}, optional are \texttt{urldate}, \texttt{author}, and
\texttt{title}.

\DescribeMacro{@PERIODICAL} After a suggestion by  \textsc{Andreas Stefanski},
I've added a new entry type \texttt{@PERIODICAL} for periodicals which are not
cited by year, but by volume number. This entry type satisfies the requirement
to put the year between square brackets. Additionally, the specification of the
volume is possible:
\begin{verbatim}
  @PERIODICAL{oellers,
    author      = {Bernd Oellers},
    title       = {Doppelwirkung im Recht?},
    journal     = {AcP},
    year        = 1969,
    volume      = 169,
    pages       = {S.\,67ff}
  }
\end{verbatim}

This shows the following:
\begin{bibexample}
\item \textbf{Oellers, Bernd:} Doppelwirkung im Recht? AcP 169 [1969], S.\,67ff
\end{bibexample}

\DescribeMacro{\bibpldelim} \DescribeMacro{\bibprdelim} Changing the format of
the brackets is possible by redefining the commands \cs{bibpldelim}
(\textbf{p}eriodical \textbf{l}eft \textbf{delim}iter) and \cs{bibprdelim}
(\textbf{p}eriodical \textbf{r}ight \textbf{delim}iter):
\begin{verbatim}
   \renewcommand{\bibpldelim}{(}
   \renewcommand{\bibprdelim}{)}
\end{verbatim}

\begin{bibexample}
\item \textbf{Oellers, Bernd:} Doppelwirkung im Recht? AcP 169 (1969), S.\,67ff
\end{bibexample}

\DescribeMacro{@COMMENTED} By definition of the entry type \texttt{@COMMENTED}
it is possible to cite commentaries as such. In connection with the option
\texttt{howcited=normal}  (cited as \meta{author}\slash annotator) or (cited as
annotator in \meta{author}) appears at the end of the bibliography entry.

\begin{bibexample}
\item \textbf{M{\"u}nchener Kommentar:} Kommentar zum B{\"u}rgerlichen Gesetzbuch, Bd.~2,
\mbox{--~Schuldrecht~--} Allgemeiner Teil, 3.~Auf\/lage, M{\"u}nchen, 1994,
\S\kern-.8pt\S~241--432 (cited: M{\"u}Ko\slash annotator)
\end{bibexample}

\DescribeMacro{updated =} This new field \NEW{0.51e} will be recognized while using |@COMMENTED|
to give the date of last update.
\begin{bibexample}
\item \textbf{M"unchener Kommentar:} Kommentar zum B"urgerlichen Gesetzbuch, Bd.~2,
\mbox{--~Schuldrecht~--} Allgemeiner Teil, 3rd edition, M"unchen, last update: May~1994
\end{bibexample}
|updated| does not overwrite the |year| field\,! The separation from the
|address|\slash|publisher|/|year| block is done by the |\updatesep| macro, which is
defined as comma by default.
In front of the content of the |updated| field appears ``last update''. This is configurable via |\updatename|.
\begin{verbatim}
   \AddTo\bibsenglish{%
        \def\updatesep{.}
        \def\updatename{updated:}
   }
\end{verbatim}

If one doesn't use this entry type, although the entry is a commentary, (cited
as \meta{author} appears, which is incorrect, because the actual citation looks
different. Those who do not use the option \texttt{howcited=normal} cannot
avoid using the entry type \texttt{@COMMENTED}.

\DescribeMacro{volumetitle} By the field  \texttt{volumetitle} it is possible
to specify a volume title which appears after the volume number. This field is
available for the entry types \texttt{@COMMENTED}, \texttt{@BOOK},
\texttt{@INBOOK} and \texttt{@INCOLLECTION}.

\DescribeMacro{titleaddon =}
  This field can be used to place a commentary, a note, some remarks about translators, coauthors etc. after the title.
\begin{verbatim}
  @COMMENTED{mueko,
    [...]
    title       = {Kommentar zum B{\"u}rgerlichen Gesetzbuch},
    titleaddon  = {Unter Mitarbeit von Hans Mustermann},
    [...]
  }
\end{verbatim}
shows us:
\begin{bibexample}
  \item \textbf{M{\"u}nchener Kommentar:} Kommentar zum B{\"u}rgerlichen Gesetzbuch, Unter Mitarbeit von Hans Mustermann, Bd.~2,
       \mbox{--~Schuldrecht~--} Allgemeiner Teil, 3.~Auflage, M{\"u}nchen, 1994, \S\kern-.8pt\S~241--432
\end{bibexample}

\DescribeMacro{booktitleaddon =}
  The same as |titleaddon| for booktitles in |@INCOLLECTION|'s.

\DescribeMacro{editortype =}
  If you want to place something other than ``(eds.)'' after a person, which isn't
  really an editor, you can use the field |editortype|:
\begin{verbatim}
   @COMMENTED{palandt,
     editor      = {Otto Palandt},
     editortype  = {Begr.},
     title       = {B{\"u}rgerliches Gesetzbuch mit Einf{\"u}hrungsgesetz [...]},
     [...]
   }
\end{verbatim}
\begin{bibexample}
 \item \textbf{Palandt, Otto (Begr.):} B{\"u}rgerliches Gesetzbuch mit Einf{\"u}hrungsgesetz [...],
      59.~Auflage, M{\"u}nchen, 2000
\end{bibexample}
  This works for |@INCOLLECTION| too.
\par\medskip\par
\DescribeMacro{sortkey =}
  It seems to be required sometimes to determine the sorting of some works different from the normal sorting algorithm.
  This problem can be solved with the |sortkey| field, which can be used to sort the work with highest priority -- contrary to
  the standard |key| field, which is mostly a fallback if |author| and |editor| are missing.

\DescribeMacro{annote =}
 Some people wrote me they would need the |annote| field to give a short abstract or something similar at the end of the bibliographic entry.
 This is working now. You are able to switch this feature on and off by using the |annote| option in the preamble or in your local  |jurabib.cfg| file.
\begin{verbatim}
   @COMMENTED{palandt,
     editor      = {Otto Palandt},
     title       = {B{\"u}rgerliches Gesetzbuch mit Einf{\"u}hrungsgesetz [...]},
     annote      = {Some people wrote me they ...},
     [...]
   }
\end{verbatim}
\begin{bibexample}
 \item \textbf{Palandt, Otto:} B{\"u}rgerliches Gesetzbuch mit Einf{\"u}hrungsgesetz [...],
  59.~Auflage, M{\"u}nchen, 2000 \par {\small Some people wrote me they would need the |annote| field to
  give a short abstract or something similar at the end of the bibliographic entry. This is working now.
  You are able to switch this feature on and off by using the |annote| option in the preamble or in your
  local |jurabib.cfg| file.}
\end{bibexample}
 The content of the field is printed out in |\small| by default.
 If you want change that:
\begin{verbatim}
   \renewcommand*{\jbannoteformat}[1]{{\footnotesize\begin{quote}#1\end{quote}}}
\end{verbatim}
\begin{bibexample}
\item \textbf{Palandt, Otto:} B{\"u}rgerliches Gesetzbuch mit Einf{\"u}hrungsgesetz [...],
 59.~Auflage, M{\"u}nchen, 2000 \par {\footnotesize\begin{quote} Some people wrote me they would need the |annote|
 field to give a short abstract or something similar at the end of the bibliographic entry.
 This is working now. You are able to switch this feature on and off by using the |annote| option
 in the preamble or in your local  |jurabib.cfg| file.\end{quote}}
\end{bibexample}

\DescribeMacro{textedition =}
 In v0.51e \CH{v0.51e} this field was removed again, because enclosing the |edition| in
 curly brackets will have the same result.

\subsubsection{Citing Juridical Dissertations and the Like}

\DescribeMacro{dissyear =} Of course one can cite juridical dissertations as
normal dissertations, but a juridical dissertation may have been published as a
book as well. In the first case, the entry type \texttt{@JURTHESIS} (or
\texttt{@PHDTHESIS}/\texttt{@MASTERSTHESIS}) should be used. In the second
case, using the entry type \texttt{@BOOK} should be preferred. For this reason,
a new field \texttt{dissyear} has been created, which enables one to specify
the year in which a book appeared as doctoral dissertation. Furthermore, if
\texttt{dissyear} is present, fields  \texttt{school} and \texttt{type} are
available for \texttt{@BOOK} as well. If \texttt{dissyear} is not present,
\texttt{type} and  \texttt{school} are ignored. For example:
\begin{verbatim}
  @BOOK{alexy,
    author      = {Alexy, Robert},
    title       = {Theorie der Grundrechte},
    year        = 1985,
    address     = {Baden-Baden},
    type        = {Habil.},
    school      = {G{\"o}ttingen},
    dissyear    = 1984
  }
\end{verbatim}

shows the following:
\begin{bibexample}
\item \textbf{Alexy, Robert:} Theorie der Grundrechte, Baden-Baden 1985 (also Habil. G{\"o}ttingen 1984)
\end{bibexample}

Suppose this work had not been published as a book, the following entry would
have been appropriate::
\begin{verbatim}
  @PHDTHESIS{alexy,
    author      = {Alexy, Robert},
    title       = {Theorie der Grundrechte},
    year        = 1984,
    type        = {Habil.},
    school      = {G{\"o}ttingen}
  }
\end{verbatim}

and would have shown the following:
\begin{bibexample}
 \item \textbf{Alexy, Robert:} Theorie der Grundrechte, Habil. G{\"o}ttingen 1984
\end{bibexample}
The default for this type is ``Jur. Diss.'':
\begin{verbatim}
  @PHDTHESIS{alexy,
    author      = {Alexy, Robert},
    title       = {Theorie der Grundrechte},
    year        = 1984,
    school      = {G{\"o}ttingen}
  }
\end{verbatim}
\begin{bibexample}
\item \textbf{Alexy, Robert:} Theorie der Grundrechte, Jur. Diss. G{\"o}ttingen 1984
\end{bibexample}

\DescribeMacro{\SSS}
The command \cs{SS} is no longer redefined by \textsf{jurabib}\,!
The new command \cs{SSS} defines two section marks with reduced space in
between the two. Compare \cs{S}\cs{S}: \S\S\ and \cs{SSS}: \S\kern-.8pt\S

\section{Remaining things~\dots}

Some helpful commands, which were introduced into the package without any
announcement.

\begin{Options}
 \item[pages=format] (Primarily for the humanities) You can switch on
    preformatting of the pages given by the \texttt{pages}-field.
    You are able to write \texttt{pages = \{22-34\}} instead of
    \texttt{pages = \{pp.\~{}22-34\}}. Not enough, you are able to do
    the same with the pages given by the optional argument of your
    |\cite| command. If you want to give something else as page(ranges)s
    or if you want to add something after a page(range),
    please use the |\nopage| and |\pageadd| command (see below for an example).
    \textsf{jurabib} will insert the appropriate and---if you are using babel---it
    will use the chosen main document language. \textsf{jurabib} makes a
    difference between a page and a pagerange. If you like to change the defaults:
   \begin{verbatim}
     \AddTo\bibsenglish{%
        \def\jbpagename{page}%
        \def\jbpagesname{pages}%
     }
   \end{verbatim}

With version 0.51g \NEW{0.51g} you can use separate  macros for the bibliography. The macros are named
|\bibpagename| and |\bibpagesname|. Please note that the definitions of |\bibpage[s]name|
are the same as for |\jbpage[s]name|. This means, if you are redefining |\jbpage[s]name|, this
will have effect on the meaning of |\bibpage[s]name| unless you are redefining
|\bibpage[s]name| separately.

 Let's take a look what \textsf{jurabib} can do to minimize your work\footnote{This rewritten option was
    inspired by \textsf{pageranges.sty}, available at CTAN, and completely contributed by Stefan Ulrich. Thanks a lot.}:
 \begin{center}
 {\small
 \begin{tabular}{@{}ll@{}}
 \hline\\[-2ex]
 |\cite[45]{<key>}                           |& \dots{}, p.~45\\
 |\cite[45--47]{<key>}                       |& \dots{}, pp.~45--47\\
 |\cite[45, 47 and 49]{<key>}                |& \dots{}, pp.~45, 47 and 49\\
 |\cite[45f]{<key>}                          |& \dots{}, pp.~45f.\\
 |\cite[45ff]{<key>}                         |& \dots{}, pp.~45ff.\\
 |\cite[\nopage{I, III and IV}]{<key>}       |& \dots{}, I, III and IV\\
 |\cite[13,\pageadd{something text}]{<key>}  |& \dots{}, p.~13, something text\\
 |        ^ No whitespace here!!!            |& \\
 \hline\\[-2ex]
 \end{tabular}}
 \end{center}

\item[pages=test] By default, page(range)s, which are given via the \texttt{pages}-field in
  the \texttt{.bib}-file, are suppressed in the citation.  With \texttt{pages=test} it will be tested,
  if a page(range) is given by the optional argument of the \cs{cite} command. If so, that one will be used.
  If there's no optional page(range) given, the one from the \texttt{.bib}-file will be used.
\item[pages=always] The page(range) given by the \texttt{.bib}-file are printed always.

\item[hypercite=false] Disables the automated \NEW{0.51c} conversion of citations into hyperlinks when using the
  \textsf{hyperref} package.

\end{Options}

\begin{description}
%\item[\cs{jbdotafterbibentry}] places a dot at the end of each entry in the bibliography.
%\item[\cs{jbdotafterendnote}] places a dot at the end of each endnote (if you are using \textsf{endnotes.sty}).
\item[\cs{jbedseplikecite}] Separation of the editors in the bibliography will be the same like in the citation.
\item[\cs{jbdisablecitationcrossref}] Disables crossrefs, which aren't \texttt{@INCOLLECTION}s.
%\item[\cs{jbsilent}] Suppresses all \textsf{jurabib} warnings.
\item[\cs{formatpages} \textit{formerly known as} \cs{formatarticlepages}] This command allows
  you to determine the appearance of page(range)s of citations of all types.
  \textsf{jurabib} is able to extract the starting page from a given pagerange (in the database).
  Since v0.5f this command can be used with all publication types.
  You can specify a list of publication types by using the first mandatory argument.
  This command takes \textit{two\,!} \CH{0.51o} optional and three mandatory arguments:
\begin{center}
 \small\cs{formatpages}\oarg{after start page separator}\oarg{before start page separator}%
 \marg{typelist}\marg{left delim}\marg{right delim}
\end{center}
 If you type |\formatpages[: ]{article,periodical}{(}{)}|, \textsf{jurabib} will format the citation |\cite[48]|\marg{key}
 with the follwowing database entry:
\begin{verbatim}
    @ARTICLE/PERIDOCAL{broxja,
      author      = {Hans Brox},
      title       = {Die Anfechtung bei der Stellvertretung},
      journal     = {JA},
      language    = {german},
      year        = 1980,
      pages       = {45--60},
      address     = {M{\"u}nchen}
    }
\end{verbatim}
 like that:
 \par\medskip
 \qquad Brox, JA 1980, 45: (48).
 \par\medskip
 If you want format only the pages given by the optional argument of |\cite|, you can leave the optional arguments
 empty:
 \par\medskip
 |\formatarticlepages{article}{[}{]}|
 \par\medskip
 \qquad Brox, JA 1980, [48].
 \par\medskip

If you are using both |\formatpages| \NEW{0.52b} and |pages=format|, it is default now to suppress
formatting of pages given by the optional argument of |\cite|, e.g. you have written
|\formatpages[, ]{article}{}{}| into your preamble and you are using |pages=format| too,
then the ouput will look like:

\par\medskip
\qquad Brox, JA 1980 p.~45, 48.
\par\medskip

If you want to have a formatted second page number,  you can use
|\jbnoformatafterstartpagefalse| in the preamble:

\par\medskip
\qquad Brox, JA 1980 p.~45, p.~48.


\item[\cs{jbfirstcitepageranges}] If you have given \NEW{0.52h} a pagerange via the |pages|-field for
  |@ARTICLE| or |@PERIODICAL| type, this pagerange will be printed out for first (using |citefull=first|)
  and full citations done by using |\[foot]fullcite|. This works independently from the |pages| options\,!
  If you add a page using the optional argument of the |\[foot]cite| command, this page will be added
  after the pagerange, separated by ``here:'', which is represented by the macro |\herename|:

\par\medskip
\qquad [\dots], p.~45, here: p.~48.
\par\medskip

Redefinition as usual:

  \begin{verbatim}
    \AddTo\bibsenglish{%
        \def\herename{there:}%
    }
  \end{verbatim}


\end{description}

\section{The Configuration file \texttt{jurabib.cfg}}
 That file can be used to save redefinitions and options. The name has to be \texttt{jurabib.cfg} and
 it has to live in the working directory or in the same directory where \textsf{jurabib} lives.
 \begin{verbatim}
   \jurabibsetup{%
     authorformat=smallcaps,
     commabeforerest,
     titleformat=colonsep,
     bibformat={tabular,ibidem,numbered}
   }
 \end{verbatim}

\section{Options for Other Academic Disciplines}\label{sec:nichtjur}
The following options are not intended primarily for juridical work, but
satisfy the needs of, among others, historians, philosophers, etc. They are the
(provisional) answers to the most different questions which were proposed to
me. I hope them to be useful. Suggestions for improvement are welcome.

\begin{Options}
\item[ibidem=name] By this option, more extensive data will be
  inserted then by using \texttt{ibidem/ibidem=strict}. If this option
  is active, the complete name of the author will be given (If not the
  option \texttt{authorformat=reducedifibidem} is active, because then, only
  the last name will be given).  This option is intended for use
  together with \texttt{citefull=first}, and therefor, that option
  will be activated automatically.

If an author is cited with multiple works, it may happen that \textsf{jurabib}
automatically switches to the next option, for guaranteeing the unambiguity of
the citation. You will find a hint to this in the  \texttt{.log}-file.

\item[ibidem=name\&title] Just like \texttt{ibidem=name}, but in this
  case, the title will be given as well. Also here, the option
  \texttt{citefull=first} will be active.
\end{Options}
This is a survey to the options explained above:
\begin{center}
{\small
\begin{tabular}{@{}lr@{}p{4.55cm}r@{}p{4.55cm}@{}}\hline\\[-2ex]
{\footnotesize\cs{footcite}|..|} & & {\footnotesize|ibidem=name|}  & &{\footnotesize|ibidem=name&title|}\\\hline\\[-2ex]
|..{brox:bgb}| & $^{1}$ &\textsc{Brox}, Hans: \textit{Allgemeiner Teil des B{\"u}rgerlichen Gesetzbuches.} 20.\,Auf\/lage, K{\"o}ln, Berlin, Bonn, M{\"u}nchen 1996.
               & $^{1}$ &\textsc{Brox}, Hans: \textit{Allgemeiner Teil des B{\"u}rgerlichen Gesetzbuches.} 20.\,Auf\/lage, K{\"o}ln, Berlin, Bonn, M{\"u}nchen 1996.\\[.3ex]
|..{brox:bgb}| & $^{2}$ &\textsc{Brox}, Hans, ibid. & $^{2}$&\textsc{Brox}, Hans: BGB AT, ibid.        \\[.3ex]
|..{brox:bgb}| & $^{3}$ &\textsc{Brox}, Hans, ibid. & $^{3}$&\textsc{Brox}, Hans: BGB AT, ibid.        \\[.3ex]
|..{oellers}|  & $^{4}$ &\textsc{Oellers}, Bernd:\,\textit{Doppelwirkung im Recht\,?} AcP 169 [1969].
               & $^{4}$ &\textsc{Oellers}, Bernd:\,\textit{Doppelwirkung im Recht\,?} AcP 169 [1969].  \\[.5ex]
|..{brox:bgb}| & $^{5}$ &\textsc{Brox}, Hans, ibid. & $^{5}$&\textsc{Brox}, Hans: BGB AT, ibid.        \\[.3ex]
|..{brox:bgb}| & $^{6}$ &\textsc{Brox}, Hans, ibid. & $^{6}$&\textsc{Brox}, Hans: BGB AT, ibid.        \\[.8ex]
                                                            \multicolumn{5}{@{}c@{}}{\emph{Pagebreak}}\\[1.6ex]
|..{brox:bgb}| & $^{7}$ &\textsc{Brox}, Hans, ibid. & $^{7}$&\textsc{Brox}, Hans: BGB AT, ibid.        \\[.3ex]
|..{brox:bgb}| & $^{8}$ &\textsc{Brox}, Hans, ibid. & $^{8}$&\textsc{Brox}, Hans: BGB AT, ibid.        \\[.8ex]
                             \multicolumn{5}{@{}l@{}}{\emph{now with |authorformat=citationreversed|:}}\\[.8ex]
|..{brox:bgb}| & $^{9}$ & Hans \textsc{Brox}, ibid. & $^{9}$&Hans \textsc{Brox}: BGB AT, ibid.         \\[.8ex]
                               \multicolumn{5}{@{}l@{}}{\emph{or with |authorformat=reducedifibidem|:}}\\[.8ex]
|..{brox:bgb}| & $^{10}$ &\textsc{Brox}, ibid.      & $^{10}$&\textsc{Brox}: BGB AT, ibid.             \\[.3ex]
\hline
\end{tabular}}
\end{center}

 \begin{Options}
 \item[\texttt{ibidem=name\&title\&auto}]
    This option \NEW{0.5f} can be useful for often repeated citations of different work of the same author.
    For the first citation the full entry is printed (|citefull=first| is automatically used).
    As long as the same work from an author is cited, only the name of the author will be used
    (this is equal to |ibidem=name|).

    If the work is cited again some footnotes later,
    name and title will be printed out (|ibidem=name&title|).
    This will avoid ambiguity if an author is cited with more than one work.
    If the immediately following citations are from the same author, only the name will be printed out.

    If you are using |ibidem=name| only, \textsf{jurabib} will check if the citations
    seems to be ambiguous and then, |ibidem=name&title&auto| will be used automatically. You will find a
    remark in the |.log| file.
 \end{Options}
 Here a survey:
 \begin{center}
 {\small
 \begin{tabular}{@{}lr@{}p{9.5cm}@{}}\hline\\[-2ex]
 {\footnotesize\cs{footcite}|..|} & & {\footnotesize|ibidem=name&title&auto|}\\\hline\\[-2ex]
 |..{brox:bgb}|  & $^{1}$&\textsc{Brox}: \textit{Allgemeiner Teil des B{\"u}rgerlichen Gesetzbuches.} 20.\,Auflage, K{\"o}ln, Berlin, Bonn, M{\"u}nchen 1996.\\[.3ex]
 |..{brox:schr}| & $^{2}$&\textsc{Brox}: \textit{Besonderes Schuldrecht.} 20.\,Auf\/lage, M{\"u}nchen 1995.\\[.3ex]
 |..{brox:bgb}|  & $^{5}$&\textsc{Brox}, ibid.\\[.8ex]
 \multicolumn{3}{@{}l@{}}{\hfill\emph{now another work is cited\,!}}\\[.8ex]
 |..{brox:schr}| & $^{6}$&\textsc{Brox}: SchR~BT, ibid.\\[.8ex]
 |..{brox:schr}| & $^{7}$&\textsc{Brox}, ibid.\\[.8ex]
 \multicolumn{3}{@{}l@{}}{\hfill\emph{\dots{} and we are switching back again\,!}}\\[.8ex]
 |..{brox:bgb}|  & $^{8}$&\textsc{Brox}: BGB AT, ibid.\\[.8ex]
 |..{brox:bgb}|  & $^{8}$&\textsc{Brox}, ibid.\\[.8ex]
\hline
 \end{tabular}}
 \end{center}

\noindent Changing the sequence of ``ibid.'' is possible by redefining the
commands \cs{ibidemname} and \cs{ibidemmidname}:
    \begin{verbatim}
      \AddTo\bibsenglish{%
         \renewcommand{\ibidemname}{Ibid.}
         \renewcommand{\ibidemmidname}{ibid.}
      }
    \end{verbatim}
\cs{ibidemname} is used by the options \texttt{ibidem=strict},
\texttt{ibidem=strictdoublepage} and \texttt{ibidem=nostrict} and appears at
the beginning of the---suppressed---citation (It is therefore possible to
capitalize it). \cs{ibidemmidname} however, appears together with the options
\texttt{ibidem=name} and \texttt{ibidem=name\&title} and can be written in
lowercase---which depends on your preference.

Now it's possible to make a difference \NEW{0.51} between a subsequent citation
with same page(s) and a subsequent citation with different page(s).
For the first case, the macros |\samepageibidemname| or |\samepageibidemmidname|
will be used internally. Its definiton is the same as for |\ibidemname| or
|\ibidemmidname| by default.
For the second case, the macros |\diffpageibidemname| or |\diffpageibidemmidname|
are used. If you like to make a difference between these two cases, you can
redefine the macros to your needs. I don't know  useful abbreviations for the
English language, so I will demonstrate it with a dummy:
\begin{verbatim}
     \renewcommand*{\samepageibidemname}{[same pages]}
\end{verbatim}
A short table will explain more:
\begin{center}
{\small\begin{tabular}{@{}lcl@{}}\hline\\[-2ex]
|\footcite[45]{broxbgb}|     & $\longrightarrow$ &\textsuperscript{1}Brox, 45.\\
|\footcite[45--47]{broxbgb}| & $\longrightarrow$ &\textsuperscript{2}Ibid., 45--47.\\
|\footcite[45--47]{broxbgb}| & $\longrightarrow$ &\textsuperscript{3}[same pages]\\
|\footcite[45f]{broxbgb}|    & $\longrightarrow$ &\textsuperscript{4}Ibid., 45\,f.\\
\hline\\[-2ex]
\end{tabular}}
\end{center}

\begin{Options}
\item[citefull=first] By aid of this option, it is possible to show
  the full bibliography entry in the first citation. For all
  subsequent citations, a short form will be used. To force a full
  citation in later citations, use the commands \cs{fullcite} and
  \cs{footfullcite}. With this option, annotators appear before author's
  names, separated by `in''. In order to get uniform citation and to
  avoid confusion of the reader, this option automatically activates
  the option  \texttt{annotatorfirstsep=in} and thus activates  options
  which are presupposed by that option. Apart from that, the
  \texttt{howcited}-options are deactivated.
\item[citefull=chapter] switches on \texttt{citefull=first} automatically and resets each chapter.
\item[citefull=section] switches on \texttt{citefull=first} automatically and resets each section.
\item[citefull=all] This option shows all citations as full citations.
  It switches also all separators to \texttt{annotatorfirstsep=in}. The
  \texttt{howcited}-options are deactivated. However, combination
  with \texttt{ibidem} is possible.
\item[see] Because those outside law studies don't need the second
  optional argument of the \cs{cite*} commands, by this option, one
  can add phrases like ``See'' or ``Compare'' before the citation.
  This option works globally.
\item[natoptargorder] Hereby, the sequence of the optional parameters
  is reversed, for instance because of making the document compatible
  with \texttt{natbib.sty} (before writing it).
\item[crossref=dynamic] You can use the other \texttt{crossref}-options
   together with that option to achieve crossrefs, which are different
   in their length---longer, if a work is cited the first time---shorter
   if it is cited again.
   Please compare the lonely use of \texttt{crossref=dynamic} (for better
   understanding you may find the crossrefs inside square brackets):
 \begin{quote}
 \small $^1$\,Lincoll, Daniel D.: Semigroups of Recurrences. In
              [Lipcoll/Lawrie/Sameh: High Speed Computer and Algorithm Organization].\\
        $^2$\,Lincoll, Daniel D.: Semigroups of Recurrences. In [Lipcoll/Lawrie/Sameh].
 \end{quote}
 with the combination of \texttt{crossref=dynamic} with \texttt{crossref=long}:
 \begin{quote}
 \small $^1$\,Lincoll, Daniel D.: Semigroups of Recurrences. In [Lipcoll, David~J./Lawrie,
             D.~H./Sameh, A.~H. (eds.):
             High Speed Computer and Algorithm Organization.
             3rd edition, New York: Academic Press, September 1977 (Fast Computers 23)].\\
        $^2$\,Lincoll, Daniel D.: Semigroups of Recurrences. In [Lipcoll/Lawrie/Sameh:
             High Speed Computer and Algorithm Organization].
 \end{quote}

\item[crossref=normal] (Default) Hereby, cross references specified by
  the special field \texttt{crossref} in the
  bibliography are displayed with author
  (\texttt{shortauthor} prevailing over  \texttt{author}) and title
  (\texttt{shorttitle} if available, else
  \texttt{title}).

\item[crossref=short] If no ambiguities result, title is left out with
  this option. In case a title is needed, \texttt{shorttitle} prevails
  over \texttt{title}.

\item[crossref=long] With this option, the cross reference will be
  displayed as full citation.

\item[lookat] Hereby, references to footnotes are enabled which
  contain the full citation to which is referred. This is possible
  \emph{only} when using the \cs{footcite} command in connection with
  the option \texttt{citefull=first}. This may be useful for
  articles which do not contain bibliographies. For this purpose, the command
  \par\medskip
  \qquad \cs{nobibliography}\marg{bibfile}\DescribeMacro{\nobibliography}
  \par\medskip
  is available, which suppresses bibliography generation.\footnote{This command is primarily for
  use with \texttt{lookat}, but it doesn't require \texttt{lookat}. But one of the \texttt{citefull}-options
  has to be enabled. Thanks to \textsc{Stefan Ulrich}.}
  In later citations, short forms of citation are used, accompanied by a hint to the footnote in
  which the full citation can be found.\footnote{Example: \textsc{Brox}: BGB AT (wie Anm.
  \textit{$\langle$Nr.$\rangle$}), Rn.~168.}

  Please notice that for correct parsing of the references with
  \texttt{lookat}, after running \BibTeX, \emph{three} \LaTeX-runs are
  necessary!

  In case you want to use the package \textsf{varioref} or
  \textsf{fancyref}, then instead of  \cs{ref} the command
  \cs{vref} will be used, resulting in different references, in
  particular if the full citation is one or two pages away.
  In case you want to use the package \textsf{varioref} or
  \textsf{fancyref} in your document, without consequences for your
  citations, you can put the command\DescribeMacro{\jbignorevarioref}
  \cs{jbignorevarioref} in the preamble of your document.

  It is possible to customize the behavior of the \texttt{lookat} option by the following commands:
  \begin{verbatim}
   \renewcommand{\lookatprefix}{\space(see footnote~}
   \renewcommand{\lookatsuffix}{)}
  \end{verbatim}

  \texttt{lookat} \emph{may} be used in connection with the
  {ibidem}-options, but such is not recommended.\par When using
  \cs{cite} commands in the main text (outside footnotes!) \emph{and} \cs{footcite}
  commands \mymarginpar{CAVE\,!}
  (or \cs{cite} commands inside footnotes)
  errors may occur from  \textsf{alphanum}
  or---if used---\textsf{varioref}.
  \textsf{alphanum} may complain in the following way:
  \begin{verbatim}
    ! Package alphanum Error: Self-reference detected!.
       [...]
    ?
  \end{verbatim}

  In this case, you have cited a work in the main text for the first time, to which one refers in the
  same section by a \cs{footcite} command. Because the first citation is not inside a footnote, the
  section number will be taken as reference label. Because by using the option \texttt{lookat},
  following citations refer to the first citation, and this citation is in the same section,
  \textsf{alphanum} complains as mentioned above.\par Another error report which is caused by the
  same problem  might look as follows:
  \begin{verbatim}
    ! Extra }, or forgotten \endgroup.
    \J@refP ...nta #1\,\J@INumberRoot {#1}{#2}
       [...]
    ?
  \end{verbatim}

  The solution is to put at least the first citation inside a footnote (whether one uses
  \cs{footcite} or \cs{cite} inside a \cs{footnote} makes no difference.).

  \emph{The \texttt{lookat}-option cannot be used with documents based
  on the \texttt{book}- or \texttt{report}-class or their
  derivatives.} Use of this option with the
  \textsf{footnpag.sty}-package is equally impossible.

  Using  \texttt{lookat} in connection with  \texttt{book}- or \texttt{report}-classes is possible
  by the  \textsf{remreset}-package. This package disables resetting the footnote counter at the
  start of a new chapter, thus enabling unambiguous references. In order to do so, you should insert
  the following in your preamble:
  \begin{verbatim}
    \usepackage{remreset}
    \makeatletter
    \@removefromreset{footnote}{chapter}
    \makeatother
  \end{verbatim}

  In oder to gain consequent results, you should also properly set the counters of images and tables:
  \begin{verbatim}
    \usepackage{remreset}
    \makeatletter
    \@removefromreset{footnote}{chapter}
    \@removefromreset{figure}{chapter}
    \renewcommand{\thefigure}{\@arabic\c@figure}
    \@removefromreset{table}{chapter}
    \renewcommand{\thetable}{\@arabic\c@table}
    \makeatother
  \end{verbatim}
\end{Options}

  \begin{Options}
  \item[\texttt{idem}]
  This new option is very similar to the |ibidem| option, but there
  is not the whole citation replaced by an abbreviation. If the author is cited
  again, his name will be replaced by ``Idem'' or ``idem''. |idem| is working
  together with all |ibidem| options. The following values are possible: |idem|
  (same as |idem=strict|), |idem=strictdoublepage| and |idem=nostrict|.
  The behavior of |idem| at page breaks is the same as for the |ibidem| options, therefore I did not make any survey.

  Redefinitions as usual:
     \begin{verbatim}
       \AddTo\bibsenglish{%
          \renewcommand*{\idemname}{Eadem}
          \renewcommand*{\idemmidname}{eadem}
       }
     \end{verbatim}
 \end{Options}

 \DescribeMacro{\noidem} |\noidem| is working analogous to |\noibidem| and disables the |idem| mechanism for the following citation.

  \begin{Options}
  \item[\texttt{opcit}]
  This option is at experimental stage and places the abbreviation ``\textit{op.\,cit.}''
  (opere citato: already cited) in the citation.
  An example:\\[1.5ex]
  {\footnotesize\textsuperscript{1}\,Aamport, \textit{Gnats and Gnus} (1986), p.\,25.}\\
  {\footnotesize [\ldots]}\\
  {\footnotesize\textsuperscript{5}\,Aamport, \textit{op.\,cit.}, p.\,37.}

 If you like to modify:

    \begin{verbatim}
      \renewcommand*{\opcit}{\textit{op.\,cit.}}
    \end{verbatim}

  \DescribeMacro{opcit=chapter}
  \DescribeMacro{opcit=section}
  |opcit| can be resetted at the begin of each chapter/section with the values
  |chapter|/|section|. This is analogous to |citefull=chapter| or |citefull=section|.
 \end{Options}

\section{Linguistic Stuff}

Because of increasing use of the  \textsf{jurabib} package by people outside
law studies, I have reworked the \BibTeX-styles in such a way that it is
possible to switch between different languages.

\DescribeMacro{language =} It is now possible, to specify the language of a
certain \BibTeX-entry by providing the field \texttt{language} with the
appropriate language. At the moment, only English, German, French, Dutch, Spanish and Italian are
implemented:
  \begin{verbatim}
      @INCOLLECTION{incollection-crossref,
          author      = {Daniel D. Lincoll},
          title       = {Semigroups of Recurrences},
          pages       = {179--183},
          language    = {english}
      }
  \end{verbatim}
If \texttt{language = \marg{other language}} is specified here, then the
\emph{hyphenation patterns} for that entry will be modified, not the keywords
like ``editor'' and so on\,! For the correct translation of the keywords
\textsf{jurabib} will detect the used main language (e.g. if you are using the
\textsf{babel}-package or one of the following packages: \textsf{german},
\textsf{french}, \textsf{frenchle}, \textsf{pmfrench}, \textsf{mlp}).
\begin{quote}
\emph{\textsf{jurabib} doesn't switch the keywords, but the hyphenation
patterns for each bibliographic entry, for which a language is given\,!}
\end{quote}
You should notice, that (for German users: contrary to the
\texttt{bibgerm}-package) only the relevant entries (which differs from the
main language) should be provided with \texttt{language}-fields. You can modify
the defaults by using the \cs{AddTo} functionality:
  \begin{verbatim}
    \AddTo\bibsgerman{\def\editorname{ed.}}
  \end{verbatim}\vspace{-1em}
The macro for English is \cs{bibsenglish}, for French \cs{bibsfrench}, for
Dutch \cs{bibsdutch}, for Spanish \cs{bibsspanish}.

If I've forgotten something which has to do with language support, please
e-mail me. The same goes for supporting more languages.

\section{Across the bounderies}

\subsection{\textsf{jura.cls}}

As I said already, use of \textsf{\mbox{jura.cls}} is possible.

\subsection{\textsf{bibtopic.sty}}

\textsf{jurabib.sty} is already for a long time---under certain
conditions---compatible with \textsf{bibtopic.sty}: All short titles should be
specified, because otherwise, ambiguous citations may occur. At this moment, it
is not possible to use automatic generation of short titles when separated
bibliography files are in use. Therefore, the option \texttt{titleformat=all}
is activated automatically, if \textsf{bibtopic.sty} has been loaded. Note: You
should use at least version \texttt{1.0j}.

\subsection{\textsf{hyperref.sty}}

\textsf{jurabib} is compatible with
\textsf{hyperref}.\footnote{ \textsc{Stefan Ulrich} was behind this
  feature as well.} However, incompatibility still exists with
\textsf{alphanum}, of which the result is that the \cs{ref} mechanism of
\textsf{alphanum} functions only to a limited extent. That means primarily that
the command \cs{ref*} provides relative references only. If you want to use
\textsf{alphanum} and \textsf{hyperref} without \textsf{jurabib}, you should
put the following in your preamble:
 \begin{verbatim}
   \makeatletter
     \let\J@SetCurrent\relax
     \def\toclevel@lvla{0}\def\toclevel@lvlb{1}
     \def\toclevel@lvlc{2}\def\toclevel@lvld{3}
     \def\toclevel@lvle{4}\def\toclevel@lvlf{5}
     \def\toclevel@lvlg{6}\def\toclevel@lvlh{7}
     \def\toclevel@lvli{8}\def\toclevel@lvlj{9}
     \def\toclevel@lvlj{10}\def\toclevel@lvll{11}
     \newcommand*{\theHlvla}{\J@Number}\newcommand*{\theHlvlb}{\J@Number}
     \newcommand*{\theHlvlc}{\J@Number}\newcommand*{\theHlvld}{\J@Number}
     \newcommand*{\theHlvle}{\J@Number}\newcommand*{\theHlvlf}{\J@Number}
     \newcommand*{\theHlvlg}{\J@Number}\newcommand*{\theHlvlh}{\J@Number}
     \newcommand*{\theHlvli}{\J@Number}\newcommand*{\theHlvlj}{\J@Number}
     \newcommand*{\theHlvlk}{\J@Number}\newcommand*{\theHlvll}{\J@Number}
     \renewcommand{\J@LongToc}[2][]{
       \@startsection{lvl\alph{tiefe}}{\number\value{tiefe}}{0pt}
       {\ifnum\value{tiefe}=1 -4ex plus-1,5ex minus-0,ex\else
       -2,7ex plus-0,8ex minus-0,2ex\fi}{\ifnum\value{tiefe}>7
       -1em plus-0,5em\relax\else 0,6ex plus0,3ex minus0,1ex\fi}
       {\sectfont\csname lvl\alph{tiefe}style\endcsname}[#1]{#2}
     }
   \makeatother
 \end{verbatim}

\subsection{\textsf{babel.sty}}

\textsf{jurabib} is compatible with  \textsf{babel}. Please make sure that \textsf{jurabib} is loaded  after \textsf{babel}\,!

\subsection{\textsf{chapterbib.sty}}

\textsf{jurabib} is compatible with  \textsf{chapterbib}.

\subsection{\textsf{bibunits.sty}}

\textsf{jurabib} is fully compatible with \textsf{bibunits}, you should use
v2.1n or higher.

 \subsection{\textsf{index.sty}}

 If you are using the \textsf{french}-, \textsf{pmfrench}- or the \textsf{frenchle}-packge, you are not
 able to use this feature at the moment.

 With the option \texttt{authorformat=indexed} you can index all cited authors.
 If you want to generate a separate author index, it's possible to use the \textsf{index} package by
 \textsc{David M. Jones}, which is part of the \textsf{camel} bundle.
 \begin{verbatim}
   [...]
  \usepackage{index}
  \newindex{default}{idx}{ind}{Index}    % for the normal Index
  \newindex{aut}{adx}{and}{Authorindex}  % for the new author Index
  \renewcommand{\jbindextype}{aut}       %
   [...]
  \begin{document}
   [...]
  \printindex                            % for the normal Index
  \printindex[aut]                       % for the new author Index
  \end{document}
 \end{verbatim}
 Please note that \cs{jbindextype} has to contain the same value as the first argument of \cs{newindex}.

 To generate the index, run:
 \begin{verbatim}
   makeindex -o datei.and datei.adx
 \end{verbatim}
 Then you have to run \LaTeX{} again. For further explanations please take a look at the
 \textsf{index} package documentation.

\subsection{\textsf{endnotes.sty}}
\textsf{jurabib} \NEW{0.51} is now compatible to \textsf{endnotes}.
With a simple |\usepackage{endnotes}| in your preamble all your |\foot[full]cite| citations
are converted into endnotes, which are appearing where you typed |\theendnotes|.
Citations, which are enclosed in |\footnote| commands or normal |\cite| commands,
were not converted by default\,! If you need this feature, please use the option
`|citetoend=true|' or consult the documentation of the \textsf{endnotes} package.

Endnotes does not have a closing dot by default. If you would have a closing period, please use
|dotafter=endnote| in the preamble of your document.

\section{Cooperators}
This package would have never been come into existence without the powerful
support of \textsc{Stefan Ulrich}, \textsc{Andreas Stefanski} and \textsc{Oren
Patashnik}. It is especially due to \textsc{Stefan Ulrich} that the package now
exists in its present form and not stumbled in its provisional stage of
development. To him, my special thanks. \textsc{Heiko Oberdiek} provided some
valuable hints. \textsc{Andreas Stefanski} was my unexhaustible tester and
advisor in juridical formalities. Not to omit \textsc{Oren Patashnik}, who
provided important parts of the \BibTeX-styles which enable the dynamic
generation of the juridical shorttitles. \textsc{Bernard Gaulle} has done a lot
for compatiblity with the french packages and gave many other helpful hints,
especially about the linguistic stuff. I want to thank \textsc{Maarten Wisse}
for translating the documentation, for his patience and for a lot of helpful
hints during implementation of the basic humanities features. \textsc{Peter
Flynn} and \textsc{P\'{a}draig de~Br\'{u}n} also suggested a lot of new
features for the humanities. And there are lots of people who worked as beta
testers and reported bugs: \textsc{Alexander Wisspeintner}, \textsc{Andreas K.~Foerster},
\textsc{Arne Engels}, \textsc{Axel Sodtalbers}, \textsc{Bastian Kruse},
\textsc{Christian Folini}, \textsc{Christian Meyn}, \textsc{David Feest},
\textsc{Daniel M.~Grisworld}, \textsc{H\'{e}l\`{e}ne Fernandez}, \textsc{Henning Eiden},
\textsc{Holger Pollmann}, \textsc{Hubert Selhofer}, \textsc{Ivan Blatter},
\textsc{Jean-Pierre Drucbert}, \textsc{Joachim Trinkwitz}, \textsc{Max Dornseif},
\textsc{Moritz Moeller-Herrmann}, \textsc{Nikolai Warneke}, \textsc{Olaf Meltzer},
\textsc{Oliver Schilling}, \textsc{Peter Wuesten}, \textsc{Ralph Sinkus},
\textsc{Rebekka Rieger}, \textsc{Robert Goulding},
\textsc{Thorsten Manegold} and \textsc{Tilman Finke}.

\section{Response requested \dots}
Questions, recommendations and critique or the like can be sent to:
\texttt{jb <at> jurabib <dot> org}

\newpage

\bibliographystyle{jurabib}

\end{document}
