\documentclass[12pt]{article}

\usepackage{cm-lingmacros, pst-tree-dvips-emu, avm, pst-coil}

\def\pstricks{PSTricks}

 \def\bck{\char'134}

\def\makespace#1{\makebox[#1]{}}

\begin{document}
\begin{center}{\large Partial Emulation of {\sf tree-dvips.sty} in {\pstricks}
({\sf pst-node} module)}\\
Avery D Andrews\\
Jan 2006
\end{center}
The {\pstricks} package {\sf pst-node} is far more powerful than Emma Pease's
{\sf tree-dvips}, but the latter strikes me as being frequently more convenient
for tasks that arise in linguistics.  So this is a partial emulation of
{\sf tree-dvips} built on the {\sf pst-node} module of \pstricks.
Most of the original {\sf tree-dvips} commands work unaltered
(\verb&\(a)nodecurve& being the most important exception), although some
have been extended to take advantage of the facilities of \pstricks.

So in this emulation, some {\sf tree-dvips} facilities are extended to support
additional possibilities afforded by {\sf pst-node} (and its parent package
{\pstricks}), while others will require (modest) changes in the source text
to run.  And there are a few completely new facilities which seem to me to
be useful and in the spirit of {\sf tree-dvips}, and easy to provide with
{\sf pst-node} available as the basis.

One of the advantages of {\sf pst-node} is that many {\tt .dvi} viewers
understand its Postscript specials, so that lines will tend to be drawn.
However not always quite correctly: in Yap, for example (the Postscript
viewer that comes with MiKTeX), lines aren't drawn when they cross over nodes,
resulting in wrong-looking displays of arrows leading into f-structures, for
example.  And some of the positioning commands don't work either.
So if things look wrong, check the postscript output before attempting
serious debugging of your line-drawing code.

It might also be worth mentioning that the {\pstricks} error-messages aren't
always very illuminating, and the {\sf tree-dvips} emulation layer does
nothing to improve this situation.

The treatment is divided into basics, which can be used without serious knowledge
of \pstricks, and advanced.  It assumes previous knowledge of {\sf tree-dvips}.

\section{Basic Commands}
\subsection{\tt \bck node(point)}
The \verb&\node& command looks just like it does in {\sf tree-dvips}:
\begin{example}
\begin{verbatim}
\node{label}{stuff}
\end{verbatim}
\end{example}
{\tt stuff} can be anything, even a regular {\pstricks} node.

\verb&\nodepoint& is slightly extended with an option for the final argument
to be an angle, in parentheses:
\begin{example}
\begin{verbatim}
\nodepoint{label}[horizontal displacement][vertical displacement]
\nodepoint{label}[displacement](angle}
\end{verbatim}
\end{example}
where the displacements are dimensions, and the angle is a number, 0 by
default, so sole displacement is always horizontal to the right.

\subsection{\tt \bck treelinewidth}
This is a dimension, setting the default width of the lines produced
by the node connection commands.  Its default is .4pt, rather thinner
than default {\pstricks} lines.  So if we want the thicken up the
node connection lines a bit, this will do the trick:
\begin{example}
\begin{verbatim}
\treelinewidth=.6pt
\end{verbatim}
\end{example}
The {\tt linewidth} graphics parameter in individual line-drawing commands
overrides this.

\subsection{\tt \bck(a)nodeconnect}
\verb&\nodeconnect& takes the same arguments as in {\sf tree-dvips}, with a final optional
\verb&[ ]& argument taking {\pstricks} graphics parameters:
\begin{example}
\begin{verbatim}
\nodeconnect[from loc]{from node}[to loc]{to node}[gr. params]
\end{verbatim}
\end{example}
The locations are the usual edges and corners,
\verb&t&, \verb&tr&, \verb&r&, \verb&br&, \verb&b&, \verb&bl&,
\verb&l&, and \verb&tl&, with the obvious interpretations. 

So here's an example:
\begin{example}
\node{a}{\begin{tabular}{c}Some fairly\\silly stuff\end{tabular}}\hskip 3em
\node{b}{explained}
\nodeconnect[r]a[l]b[linewidth=1pt, nodesepB=.5ex]
\end{example}
which is produced by this input:
\begin{example}
\begin{verbatim}
\node{a}{\begin{tabular}{c}Some fairly\\silly stuff\end{tabular}}\hskip 3em
\node{b}{explained}
\nodeconnect[r]a[l]b[linewidth=1pt, nodesepB=.5ex]
\end{verbatim}
\end{example}
The final (new) {\tt [ ]} argument allows many further properties of the
line to be set; for example we get this:
\begin{example}
\node{a}{\begin{tabular}{c}Some fairly\\silly stuff\end{tabular}}\hskip 3em
\node{b}{explained}
\nodeconnect[r]a[l]b[linewidth=1pt, nodesepB=.5ex, arrows=->>]
\end{example}
from this:
\begin{example}
\begin{verbatim}
\node{a}{\begin{tabular}{c}Some fairly\\silly stuff\end{tabular}}\hskip 3em
\node{b}{explained}
\nodeconnect[r]a[l]b[linewidth=1pt, nodesepB=.5ex, arrows=->>]
\end{verbatim}
\end{example}
For the full assortment of graphics parameters, see the {\pstricks}
documentation, but here are some that strike me as useful in the
present context:
\begin{examples}
\item {\tt nodesep=Xdim}, {\tt nodesepA=Xdim}, {\tt nodesepB=Xdim}: the distance between
the edge of the node and the endpoints of the connection line; {\tt nodesep}
applies to beginning and the end, {\tt nodesepA} only to the beginning,
and {\tt nodesepB} only to the end.
\item {\tt arrows=ARR}: the style of arrow head, stated iconically (value of
{\tt -} gives no arrowhead)
\item {\tt linewidth=Xdim}
\item {\tt dashed}: (no value)
\item {\tt dash=Xdim Ydim}: {\tt Xdim} is the length of the blank portion,
{\tt Ydim} of the dashes
\item {\tt dotted}: (no value)
\end{examples}

\verb&\anodeconnect& takes all the same arguments as \verb&\nodeconnect&,
but draws an arrowhead at the end of the connection.  So from this:
\begin{example}
\begin{verbatim}
\node{a}{\begin{tabular}{c}Some fairly\\silly stuff\end{tabular}}\hskip 3em
\node{b}{explained}
\anodeconnect[r]a[l]b[nodesepB=.5ex]
\end{verbatim}
\end{example}
We get this:
\begin{example}
\node{a}{\begin{tabular}{c}Some fairly\\silly stuff\end{tabular}}\hskip 3em
\node{b}{explained}
\anodeconnect[r]a[l]b[nodesepB=.5ex]
\end{example}
You can or course specify an {\tt arrows} parameter with {\tt anodeconnect},
but if you do, plain old {\tt nodeconnect} would do the job as well.

\subsection{\tt \bck arrow(width|length)}
These aren't implemented, since the {\pstricks} scheme is so different.
Use:
\begin{example}
\begin{verbatim}
\psset{arrowsize=Xdim Y, arrowlength=Z. arrowinset=W}
\end{verbatim}
\end{example}
{\tt pst-tree-dvips-emu.sty} sets defaults which look decent to me.


\subsection{\tt \bck (a)nodecurve}
Not a full implementation, because the way in which {\pstricks} handles
curvature of links between nodes is quite different from the way in which
{\sf tree-dvips} does (and, I'd say, is better).  {\sf tree-dvips} requires
a final argument which specifies the curvature in terms of a dimension,
with no sensible default, while {\pstricks} uses a number that does have
a sensible default.  The current \verb&\anodecurve& command simply doesn't
read this final argument, so it will be dumped into the text, so that an
the input:
\begin{example}
\begin{verbatim}
\node{a}{the dog}\hskip 5em\node{b}{barked}
\nodecurve[tr]a[tl]b{2ex}
\end{verbatim}
\end{example}
produces this wrong result
\begin{example}
\node{a}{the dog}\hskip 5em\node{b}{barked}
\nodecurve[tr]a[tl]b{2ex}
\end{example}
What we have instead to specify curvature is an optional argument in
parentheses, containing one or two numbers, the bigger the number, the bigger
the `pull' of the curve away from its beginning or final endpoint.  So to
get this:
\begin{example}\vskip 4ex
\node{a}{the dog}\hskip 5em\node{b}{barked}
\nodecurve[tr]a[tl]b(1.3)
\end{example}
we write this:
\begin{example}
\begin{verbatim}
\node{a}{the dog}\hskip 5em\node{b}{barked}
\nodecurve[tr]a[tl]b(1.3)
\end{verbatim}
\end{example}
And this:
\begin{example}
\begin{verbatim}
\node{a}{the dog}\hskip 5em\node{b}{barked}
\nodecurve[tr]a[tl]b(2,.4)
\end{verbatim}
\end{example}
yields this:
\begin{example}
\node{a}{the dog}\hskip 5em\node{b}{barked}
\nodecurve[tr]a[tl]b(2,.4)
\end{example}
And finally we get all the usual graphics parameterss in square brackets at
the end.  Of these, there are two that are specifically important for curves,
{\tt angleA} and {\tt angleB}.  These specify the angle at which the line
goes from its source and two its target node.  If nothing is specified,
this is determined by the position just as in {\sf tree-dvips}.  So we
can `flatten' the curve above by writing for example:
\begin{example}
%begin{verbatim}
\node{a}{the dog}\hskip 5em\node{b}{barked}
\nodecurve[tr]a[tl]b[angleA=20, angleB=160]
%\end{verbatim}
\end{example}
An angle of 0 degrees points to the right edge of the page.

\subsection{\tt \bck (a)barnodeconnect}
As in {\sf tree-dvips.sty}, except with final graphics parameters.
So we get this:
\begin{example}
\node{a}{connect} some words with a
  \node{b}{\tt barnodeconnect} command
\barnodeconnect {a}{b}
\end{example}
from this:
\begin{example}
\begin{verbatim}
\node{a}{connect} some words with a
  \node{b}{\tt barnodeconnect} command
\barnodeconnect {a}{b}
\end{verbatim}
\end{example}
A useful set of options is {\tt offset(A|B)}, which shifts the connection
points to the left or right, so that:
\begin{example}
\begin{verbatim}
\node{a}{connect} some words with the
  \node{b}{barnodeconnect} command
\barnodeconnect[2ex]{a}{b}[offsetB=3em]
\end{verbatim}
\end{example}
produces:
\begin{example}
\node{a}{connect} some words with the
  \node{b}{barnodeconnect} command
\barnodeconnect[2ex]{a}{b}[offsetB=3em]
\end{example}

And if we want an arrow at the end, we can use \verb&\abarnodeconnect&,
which takes all the same options.


\subsection{\tt \bck nodetriangle}
\verb&\nodetriangle{a}{b}[gr. params]& draws a triangle with apex
at bottom of {\tt a}, and base the top side of {\tt b} (for `triangles of
laziness') in syntax trees).  

\subsection{\tt \bck node\{circle|oval|box\}}
Circle and oval require dimensions to be specified, but since I have hopes of eventually
overcoming this limitations, the specifications are in parentheses even
though they're obligatory.  For the circle there is one dimension, the radius,
for the oval, two: \verb&(height,width)&. Here's some sample input:
\begin{example}
\begin{verbatim}
\makespace{2em}\node{a}{A}
\nodecircle{a}(1em)[linewidth=1pt, linestyle=dotted]\\[4ex]
\makespace{2em}\node{a}{A}
\nodeoval{a}(2em, 1em)\\[4ex]
\end{verbatim}
\end{example}
These commands differ from the {\pstricks} ovalnode, etc, in that they
don't create nodes (ovalnodes containing text are bigger than a plain node
containing the same text), but just draw lines around a pre-existing
node.

The \verb&\nodebox{node}& box works as in {\sf tree-dvips}, drawing
the box at a distance of \verb&\nodemargin& around its content.  Put
as usual, there's an optional square-bracketted graphics parameter argument,
which can be used to specify the corner-rounding {\tt framearc} parameter:
\begin{examples}
\item
\node{a}{Some Stuff$\/$}
\nodebox{a}[framearc=.3]
\item
\begin{verbatim}
\node{a}{Some Stuff$\/$}
\nodebox{a}[framearc=.3]
\end{verbatim}
\end{examples}


\section{More advanced}
These commands depend on a bit more of a knowledge of {\pstricks} than
the previous ones.
\subsection{\tt \bck putpoint, \bck putstuff}
\verb&\putpoint& one puts a {\pstricks} point node at a position relative to a
{\sf tree-dvips} node center or corner.  The usage is:
\begin{example}
\begin{verbatim}
\putpoint{new node name}{distance}(angle)[corner]{old node name}
\end{verbatim}
\end{example}
Where {\tt distance} can be either a single dimension (distance), or a
 pair of dimensions
 \verb&{horizontal, vertical}&, in which case the
angle is ignored.
For examples, see the file {\tt treetest.txp}, where it is used in
conjunction with \verb&\psccurve& to put loops around contiguous
sets of phrase-structure nodes.

\verb&\putstuff& takes the same arguments, except that the first one
is any
regular text or {\LaTeX} code, rather than a label for a point node.

\subsection{\tt <command>(options) arguments to \bck nodeconnect}
In addition to the regular arguments, \verb&\nodeconnect& can take an
initial argument in angle-brackets, to specify the actual node connection
command to be used, followed by graphics parameters in parentheses.  This makes
it easy to define variants of the node connection commands, such as
\verb%\coilconnect%
below:
\begin{examples}
\item
\begin{verbatim}
\def\coilconnect{\nodeconnect<\nccoil>(coilwidth=1ex,coilarmA=1ex,
   coilarmB=2ex, arrows=->)}
\node{a}{source}\hskip 5em\node{b}{target}
\coilconnect[r]a[l]b
\end{verbatim}
\item
\def\coilconnect{\nodeconnect<\pccoil>(coilwidth=1ex,coilarmA=1ex,
   coilarmB=2ex, arrows=->)}
\node{a}{source}\hskip 5em\node{b}{target}
\coilconnect[r]a[l]b
\end{examples}
Only \verb$\pc...$ commands will work here, their \verb$\nc...$ commands
will produce enigmatic errors, such as \verb&! Argument of \nodeloc has an extra }&.

\subsection{\tt \bck nodeloc}
This command delivers locations of nodes, taking the corner/edge as the first,
argument, the node name as the second.  These locations can only be used when
\verb$\SpecialCoor$ is active, and must be enclosed in parentheses to work.
For examples, look at some of the definitions in {\tt pst-tree-dvips-emu.sty}.

\end{document}