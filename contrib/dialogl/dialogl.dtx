%%% ====================================================================
%%%  @LaTeX-style-file{
%%%     filename        = "dialogl.dtx",
%%%     version         = "1.99a",
%%%     date            = "2013/01/24",
%%%     author          = "Michael Downes",
%%%     copyright       = "This file is part of the dialogl package, released
%%%                        under the LPPL; see dialogl.ins for details."
%%%     keywords        = "TeX, dialog",
%%%     supported       = "no",
%%%     abstract        = "This file provides macros for writing
%%%       messages and menus on screen, and reading user responses. It
%%%       can be used with LaTeX as a documentstyle option, or in
%%%       other forms of TeX by a standard \input statement.",
%%%  }
%%% ====================================================================
%
% \iffalse
%<*driver>
% Copyright 1994 Michael John Downes
% Copyright 2013 TeX Users Group
% This file is part of the dialogl package, released under the LPPL;
% see dialogl.ins for details.
%
%    Originally published in TUGboat with ltugboat documentstyle.
%    Switched to generic article/twocolumn afterward.
%    Does not work with either LaTeX2e or LaTeX 2.09 compatibility mode :(.
\documentstyle[doc,dialogl-doc]{article}

\title{Interaction tools: {\tt dialog.sty} and {\tt menus.sty}}
\author{Michael Downes}
\date{November 3, 1994}

% one latex2e command gets used.
\def\ensuremath#1{\ifmmode #1\else $#1$\fi\relax}

\hyphenation{pro-duces}

\renewcommand{\thepart}{\arabic{part}}
\renewcommand{\thesection}{\thepart.\arabic{section}}

%    Some customizations of doc.sty parameters.
\def\PrintMacroName#1{}
%
%\renewcommand{\PrintDescribeMacro}[1]{}
%\renewcommand{\PrintDescribeEnv}[1]{}

\setlength{\MacroIndent}{0pt}
\def\MacroFont{\verbatimfont}% doc.sty uses \MacroFont

%    No modules in the files to be processed; so don't bother checking.
\DontCheckModules

\multicoldefaultlayout
\begin{document}
\jobswitch % print only selected parts if \jobname so indicates; see
           % dialog.sty
\maketitle
\thispagestyle{empty}

\begin{multicols}{2}
\section*{Introduction}

\ifall
This article describes
\fn{dialog.sty} and \fn{menus.sty}, which
provide functions for printing messages or menus on screen and reading
users' responses. The file \fn{dialog.sty} contains basic message and
input-reading functions; \fn{menus.sty} takes \fn{dialog.sty} for its
base and uses some of its functions in defining more complex menu
construction functions. These two files are set up in the form of
\LaTeX{} documentstyle option files, but in writing them I spent some
extra effort to try to make them usable with \plaintex/ or other
common macro packages that include \plaintex/ in their base, such as
\AmSTeX{} or \eplain/.

The appendix describes \fn{grabhedr.sty}, required by \fn{dialog.sty},
which provides two useful file-handling features: (1)~a command
\cw{inputfwh} that when substituted for \cw{input} makes it possible to
grab information such as file name, version, and date from standardized
file headers in the style promoted by Nelson Beebe\Dash and to grab it
in the process of first inputting the file, as opposed to inputting the
file twice, or \cw{read}ing the information separately (unreliable due
to system-dependent differences in the equivalence of \tex/'s \cw{input}
search path and \cw{openin} search path). And (2)~functions
\cw{localcatcodes} and \cw{restorecatcodes} that make it possible for
\fn{dialog.sty} (or any file) to manage internal catcode changes
properly regardless of the surrounding context.

These files and a few others are combined in
\else
This package is part of
\fi
a suite of files that goes
by the name of {\bf dialogl}, available on the Internet by anonymous ftp
from CTAN (Comprehensive \tex/ Archive Network), e.g., \fn{ftp.shsu.edu}
(USA), or \fn{ftp.uni-stuttgart.de} (Europe).
%
\ifall
The file \fn{listout.tex} is a utility for verbatim printing of plain
text files, with reasonably good handling of overlong lines, tab
characters, other nonprinting characters, etc. It uses \fn{menus.sty} to
present an elaborate menu system for changing options (like font size,
line spacing, or how many spaces should be printed for a tab character).
\else
It includes the packages \pkg{dialog}, \pkg{menus}, and \pkg{grabhedr}
and a utility file \fn{listout.tex} for verbatim printing of plain text
files.
\fi

\ifmenus
Here's an example from the menu system of \fn{listout.tex} to
demonstrate the use of some features from \fn{dialog.sty} and
\fn{menus.sty}. First, the menu that you would see if you wanted to
change the font or line spacing:
%%%%%%%%%%%%%%%%%%%%%%%%%%%%%%%%%%%%%%%%%%%%%%%%%%%%%%%%%%%%%%%%%%%%%%
%    This funny way of calling \samepage is (I think) the most
%    convenient way to avoid turning off page breaks for the preceding
%    paragraph.
%%%%%%%%%%%%%%%%%%%%%%%%%%%%%%%%%%%%%%%%%%%%%%%%%%%%%%%%%%%%%%%%%%%%%%
\def\next{\begin{verbatim}\samepage}\next
===============================================
  F  Change font
  S  Change font size
  L  Change line spacing

Current settings: typewriter 8 / 10.0pt.

    Q  Quit         X  Exit        ?  Help
===============================================

Your choice?
\end{verbatim}
Suppose you wanted to change line spacing to 9 points, so you entered
\qc{\l} (lowercase \qc{\L}) and then \verb'9pt', except that on your
first attempt you accidentally mistyped \verb'9pe' instead of
\verb'9pt'. Here's what you would see on screen:
\begin{verbatim}
Your choice? l
Desired line spacing [TeX units] ? 9pe
?---I don't understand "9pe".
Desired line spacing [TeX units] ? 9pt

* New line spacing: 9.0pt
\end{verbatim}
Both lowercase \verb'l' and capital \verb'L' are acceptable responses,
and the value given for line spacing is checked to make sure it's a
valid \tex/ dimension. Before continuing, the internalized version of
the user's value is echoed on screen to confirm that the entered value
was read correctly.

Now here's how the above menu is programmed in \fn{listout.tex}.
A function \cw{menuF} is constructed using \cw{fxmenu}:
\begin{verbatim}
\fxmenu\menuF{}{
F  Change font
S  Change font size
L  Change line spacing
}{
Current settings: &\mainfont &\mainfontsize / %
&\the&\normalbaselineskip.
}
%
\def\moptionF{\lettermenu F}
\end{verbatim}
In the definition of \cw{moptionF}, \cw{lettermenu} is a high-level
function from \fn{menus.sty} that prints \cw{menuF} on screen (given
the argument \verb'F'), reads a line of input from the user, extracts
the first character and forces it to uppercase, then branches to
the next menu as determined by that character. The response of
\verb'l' causes a branch to the function \cw{moptionFL}:
\begin{verbatim}
\def\moptionFL{%
  \promptmesj{%
    Desired line spacing [TeX units] ? }%
  \readline{Q}\reply
\end{verbatim}
If \verb'Q', \verb'X', or \verb'?' was entered, the test
\cw{xoptiontest} will return `true'; then we should skip the dimension
checking and go directly to \cw{optionexec}, which knows what to do
with those responses:
\begin{verbatim}
  \if\xoptiontest\reply
  \else
\end{verbatim}
Otherwise we check the given dimension to make sure it's usable. If so,
echo the new value as confirmation.
\begin{verbatim}
    \checkdimen\reply\dimen@
    \ifdim\dimen@>\z@
      \normalbaselineskip\dimen@\relax
      \normalbaselines
      \confirm{New line spacing:
        \the\normalbaselineskip}%
      \def\reply{Q}%
    \fi
\end{verbatim}
If \cw{reply} was changed to \verb'Q' during the above step,
\cw{optionexec} will pop back up to the previous menu level (normal
continuation); otherwise \cw{reply} retains its prior definition\Dash
e.g., \verb'9pe'\Dash to which \cw{optionexec} will simply say ``I
don't understand that'' and repeat the current prompt.
\begin{verbatim}
  \fi
  \optionexec\reply
}
\end{verbatim}

For maximum portability, \fn{listout.tex} uses in its menus only
lowest-common-denominator ordinary printable ASCII characters in the
range 32\dash 126. Fancier menus can be obtained at a cost of forgoing
system independence, for instance by using em\tex/'s \verb'/o' option
to output the box-drawing characters in the standard PC DOS character
set.
\fi

\subsection*{Notation}

Double-hat notation such as \verb'^^J' is used herein for
control characters, as in \TB{}, although occasionally the alternate
form `{\sc control}-J' is used when the emphasis is away from the
character's tokenized state inside \tex/. A couple of abbreviations
from \fn{grabhedr.sty} are used frequently in the macro code: \cw{xp@}
= \cw{expandafter}, and \cw{nx@} = \cw{noexpand}. Standard
abbreviations from \fn{plain.tex} such as \cw{z@} or \cw{toks@} are
used without special comment.

\ifall
\part{Basic dialog functions: \fn{dialog.sty}}
\fi
\ifdialog\hDocInput{dialogl.dtx}\fi

\ifall
\part{Menu functions: \fn{menus.sty}}
\setcounter{section}{0}
\fi
\ifmenus\hDocInput{menus.dtx}\fi

\ifall
%    It's hard to make this part title come out right without assuming
%    \partname is defined.
\part*{Appendix\penalty-10000
Miscellaneous support functions:
\fn{grabhedr.sty}}
\setcounter{section}{0}
\renewcommand{\thepart}{A} % for use in \thesection
\fi
\ifgrabhedr\hDocInput{grabhedr.dtx}\fi

\end{multicols}

\ifdim\textwidth>\textheight
\typeout{^^JWarning: remember to print using LANDSCAPE orientation^^J}
\fi
\end{document}

%</driver>
% \fi
%
% \section{History}
%    This file, \fn{dialog.sty}, was born out of a utility called
%    \fn{listout.tex} that I wrote for my personal use. The purpose of
%    \fn{listout.tex} was to facilitate printing out plain text
%    files\Dash electronic mail, program source files in various
%    programming languages, and, foremost, \tex/ macro files and log
%    files. An important part of my \tex/ programming practice is to
%    print out a macro file on paper and read it through, marking
%    corrections along the way, then use the marked copy as a script
%    for editing the file. (For one thing, this allows me to analyze
%    and mark corrections while riding the bus to work, or
%    sitting out in the back yard to supervise the kids.) The output I
%    normally desired was two `pages' per sheet of U.S. letter-size
%    paper printed landscape, in order to conserve paper.
%
%    Once created, \fn{listout.tex} quickly became my favorite means
%    of printing out plain text files, not to mention an indispensable
%    tool in my debugging toolbox: I turn on \cw{tracingmacros} and
%    \cw{tracingcommands}, then print out the resulting log file so
%    that I can see several hundred lines of the log at once (by
%    spreading out two or three pages on my desk with 100+ lines per
%    page); then I trace through, cross things out, label other
%    things, draw arrows, and so forth.
%
%    I soon added a filename prompting loop to make it convenient to
%    print multiple files in a single run. In the process of
%    perfecting this simple prompting routine\Dash over two or three
%    years\Dash and adding the ability to optionally specify things like
%    number of columns at run time, eventually I wrote so much
%    dialog-related macro code that it became clear this code should
%    be moved out of \fn{listout.tex} into its own module. The result
%    was \fn{dialog.sty}.
%
%    Before getting into the macro definitions and technical
%    commentary, here are descriptions from the user's
%    perspective of the functions defined in this file.
%
% %%%%%%%%%%%%%%%%%%%%%%%%%%%%%%%%%%%%%%%%%%%%%%%%%%%%%%%%%%%%%%%%%%%%
% \section{Message-sending functions}
%
%\begin{usage}
%\mesj{<text>}
%\end{usage}
%    Sends the message verbatim: category 12 for all special
%    characters except braces, tab characters, and carriage returns:
%
%^^V { } ^^I ^^M
%
%    Naturally, the catcode changes are effective only if \cw{mesj} is
%    used directly, not inside a macro argument or definition
%    replacement text.
%
%    Multiple spaces in the argument of \cw{mesj} print as multiple
%    spaces on screen. A tab character produces always eight spaces;
%    `smart' handling of tabs is more complicated than I would care to
%    attempt.
%
%    Line breaks in the argument of \cw{mesj} will produce line breaks
%    on screen. That is, you don't need to enter a special sequence
%    such as \ctrl{J}\qc{\%} to get line breaks. See the technical
%    commentary for \cw{mesjsetup} for details. Even though curly
%    braces are left with their normal catcodes, they can be printed
%    in a message without any problem, if they occur in balanced
%    pairs. If not, the message should be sent using \cw{xmesj}
%    instead of \cw{mesj}.
%
%    Because of its careful handling of the message text, \cw{mesj} is
%    extremely easy to use. The only thing you have to worry about is
%    having properly matched braces. Beyond that, you simply type
%    everything exactly as you want it to appear on screen.
%
%\begin{usage}
%\xmesj{<text>}
%\end{usage}
%    This is like \cw{mesj} but expands embedded control sequences
%    instead of printing them verbatim. All special characters have
%    category 12 except backslash, percent, braces, tab, return, and
%    ampersand:
%
%^^V \ % { } ^^I ^^M &
%
%    The first four have normal \tex/ catcodes to make it possible to
%    use most normal \tex/ commands, and comments, in the message
%    text. \ctrl{I} and \ctrl{M} are catcode 13 and behave as
%    described for \cw{mesj}. The \qc{\&} is a special convenience, an
%    abbreviation for \cw{noexpand}, to use for controlling expansion
%    inside the message text.
%
%    Doubled backslash \cs{\\} in the argument will produce a single
%    category 12 backslash character\Dash thus, \verb'\\xxx' can be used
%    instead of \cw{string}\cw{xxx} or \cw{noexpand}\cw{xxx} (notice
%    that this works even for outer things like \cw{bye} or
%    \cw{newif}). Similarly \cs{\%}, \cs{\{}, \cs{\}} and \cs{\&}
%    produce the corresponding single characters.
%
%    Category 12 space means that you cannot write something like
%
%^^V \ifvmode h\else v\fi rule
%
%    in the argument of \cw{xmesj} without getting a space after the
%    \cw{ifvmode}, \cw{else}, and \cw{fi}.%^^A
%^^A
%\footnote{Well, actually, you could replace each space by
%\qc{\%}\meta{newline} to get rid of it. But that makes the message
%text harder to read for the programmer.}
%^^A
%    Since occasionally this may be troublesome, \cs{\.} is defined
%    inside the argument of \cw{xmesj} to be a `control word
%    terminator': If the expansion of \cw{foo} is \verb'abc', then
%    \verb'\foo\.xyz' produces \verb'abcxyz' on screen (as opposed to
%    \verb'\foo xyz' which would produce \verb'abc xyz'). Thus the
%    above conditional could be written as
%
%^^V \ifvmode\.h\else\.v\fi\.rule
%
%    Even though the catcode changes done by \cw{xmesj} setup have no
%    effect if \cw{xmesj} is used inside an argument or definition
%    replacement text, I find it convenient occasionally to use
%    \cw{xmesj} in those contexts, in order to get other aspects of the
%    \cw{xmesj} setup. For instance, if you need to embed a message
%    that contains a percent sign inside a definition, you can write
%
%^^V \def\foo{...
%^^V   \xmesj{... this is a percent
%^^V     sign: \% (sans backslash) ...}
%^^V ...}
%
%    To further support such uses of \cw{xmesj}, the following changes
%    are also done by \cw{xmesj} setup: the backslash-space control
%    symbol {\tt\bslash\char32} is made equivalent to \cw{space};
%    \cs{\^^J} and \cs{\^^M} are defined to produce a \cw{newlinechar};
%    and active tilde \qc{\~} will produce a category-12 tilde.
%
%    Among other things, this setup makes it easier to obtain
%    newlines and multiple spaces in an embedded message. For example,
%    in the following definition the message will have a line break on
%    screen for each backslash at the end of a line, and the third
%    line will be indented four spaces.
%
%^^V \def\bar{...
%^^V   \xmesj{First line\
%^^V     Second line\
%^^V     \ \ \ \ Indented line\
%^^V     Last line}%
%^^V ...}
%
%    The alternative of defining a separate message function
%    \cw{barfoo} with \cw{f[x]mesj} and calling \cw{barfoo} inside of
%    \cw{bar} would allow more natural entry of the newlines and the
%    multiple spaces, but would be slightly more expensive in string
%    pool and hash table usage.
%
%\begin{usage}
%\promptmesj{<text>}
%\promptxmesj{<text>}
%\end{usage}
%    These are like \cw{mesj}, \cw{xmesj} but use \cw{message} rather
%    than \cw{immediate}\cw{write}\verb'16' internally, thus if the
%    following operation is a \cw{read}, the user will see the cursor
%    on screen at the end of the last line, as may be desired when
%    prompting for a short reply, rather than at the beginning of the
%    next line. The character \qc{\!} is preempted internally for
%    newlinechar, for these two functions only,
%    which means that it cannot be actually printed in
%    the message text. Use of a visible character such as \qc{\!},
%    rather than the normal \cw{newlinechar} \ctrl{J}, is necessary
%    for robustness because of the fact that the \cw{message}
%    primitive was unable to use an `invisible' character (outside the
%    range 32--126) for newlines up until \tex/ version 3.1415, which
%    some users do not yet have (at the time of this writing\Dash July
%    1994).
%
%\begin{usage}
%\storemesj\foo{<text>}
%\storexmesj\foo{<text>}
%\end{usage}
%    These functions are similar to \cw{mesj}, \cw{xmesj} but store the
%    given text in the control sequence \cw{foo} instead of immediately
%    sending the message. Standard \tex/ parameter syntax can be used
%    to make \cw{foo} a function with arguments, e.g. after
%
%^^V \storemesj\foo#1{...#1...}
%
%    then you can later write
%
%^^V \message{\foo{\the\hsize}}
%
%    and get the current value of \cw{hsize} into the middle of the
%    message text. Consequently also in the x-version \cw{storexmesj} a
%    category-12 \qc{\#} character can be obtained with \cs{\#}.
%
%\begin{usage}
%\fmesj\foobar#1#2...{...#1...#2...}
%\end{usage}
%    Defines \cw{foobar} as a function that will take the given
%    arguments, sow them into the message text \verb"{...}", and send
%    the message. In the message text all special characters are
%    category 12 except for braces, \qc{\#}, tab, and carriage return.
%
%    If an unmatched brace or a \qc{\#} must be printed in the message
%    text \cw{fxmesj} must be used instead. (\arg{#} could be used to
%    insert a single category-6 \qc{\#} token into the message text, and
%    \tex/ would print it without an error, but both \cw{message} and
%    \cw{write} would print it as two \arg{#} characters, even though
%    it's only a single token internally.)
%
%\begin{usage}
%\fxmesj\foobar#1#2...{...#1...#2...}
%\end{usage}
%    Combination of \cw{xmesj} and \cw{fmesj}. Defines \cw{foobar} like
%    \cw{fmesj}, but with full expansion of the replacement text and
%    with normal category codes for backslash, percent, braces, and hash
%    \qc{\#}. The control symbols \cs{\\} \cs{\%} \cs{\{} \cs{\}}
%    \cs{\&} and \cs{\.} can be used as in \cw{xmesj}, with also \cs{\#}
%    for printing a \qc{\#} character of category 12.
%
% %%%%%%%%%%%%%%%%%%%%%%%%%%%%%%%%%%%%%%%%%%%%%%%%%%%%%%%%%%%%%%%%%%%%
% \section{Reading functions}
%
%\begin{usage}
%\readline{<default>}\answer
%\end{usage}
%    This reads a line of input from the user into the macro
%    \cw{answer}. (The macro name can be anything chosen by the
%    programmer, not just \cw{answer}.) Before reading, all special
%    characters are deactivated, so that the primitive \cw{read} will
%    not choke if the user happens to enter something like \cw{newif}
%    or {\sc control-l} or \qc{\}}. Depending on the operating
%    system, certain characters\Dash e.g., {\sc control-c}, {\sc
%    control-z}, {\sc control-d}, {\sc control-h}\Dash might have
%    special effects instead of being entered into the replacement
%    text of \cw{answer}, regardless of the catcode changes. To
%    take the most obvious example, under most
%    operating systems, typing {\sc control-h} (the Rubout or
%    Backward-Delete key) will delete the previous character from the
%    user's response, instead of entering an \ascii/ character 8
%    into \cw{answer}.
%
%    There is one significant exception from the catcode changes that
%    are done for \cw{readline}: spaces and tabs retain their normal
%    catcode of 10, so that multiple spaces in an answer will be
%    reduced to a single space, and macros with normal space-delimited
%    arguments will work when applied to the answer. (I can't think of
%    any likely scenario where category 12 for spaces would be
%    useful.) Also, the catcode of \ctrl{M} is set to 9 (ignore) so
%    that an empty line\Dash meaning that the user just pressed the
%    carriage return/enter key\Dash will result in an empty \cw{answer}.
%    If the answer is empty, the given default string will be
%    substituted. The default string can be empty.
%
%\begin{usage}
%\xreadline{<default>}\answer
%\end{usage}
%    Like \cw{readline} but the answer is read as executable tokens;
%    the usual catcodes of the \tex/ special characters remain in
%    effect while reading the answer. A few common outer things
%    (\cw{bye}, \cs{\+}, \cw{newif}, \ctrl{L}, among others) are
%    neutralized before the \cw{read} is done, but the user can still
%    cause problems by entering some other outer control sequence or
%    unbalanced braces. I doubt there's any bulletproof solution, if
%    the tokens are to remain executable, short of the usual last
%    resort: reading the answer using \cw{readline}, writing it to a
%    file, then inputting the file.
%
%\begin{usage}
%\readchar{<default>}\answer
%\end{usage}
%    This is like \cw{readline} but it reduces the answer to its first
%    character. \meta{default} is either a single character or empty.
%
%\begin{usage}
%\readChar{<default>}\answer
%\end{usage}
%    This is like \cw{readchar} and also uppercases the answer.
%
%\begin{usage}
%\changecase\uppercase\answer
%\end{usage}
%    The function \cw{changecase} redefines its second argument, which
%    must be a macro, to contain the same text as before, but
%    uppercased or lowercased according to the first argument. Thus
%    \cw{readChar}\verb'{Q}'\cw{answer} is equivalent to
%
%^^V \readchar{q}\answer
%^^V \changecase\uppercase\answer
%
%    It might sometimes be desirable to force lower case before using
%    a file name given by the user, for example.
%
% %%%%%%%%%%%%%%%%%%%%%%%%%%%%%%%%%%%%%%%%%%%%%%%%%%%%%%%%%%%%%%%%%%%%
% \section{Checking functions}
%
%\begin{usage}
%\checkinteger\reply\tempcount
%\end{usage}
%    To read in and check an answer that is supposed to be an integer,
%    use \cw{readline}{}\cw{reply} and then apply \cw{checkinteger} to
%    the \cw{reply}, supplying a count register, not necessarily named
%    \cw{tempcount}, wherein \cw{checkinteger} will leave the validated
%    integer. If \cw{reply} does not contain a valid integer the
%    returned value will be \verb'-'\cw{maxdimen}.
%
%    At the present time only decimal digits are handled;
%    some valid \tex/ numbers such as \verb'"AB', \verb'`\@',
%    \cw{number}\cw{prevgraf}, or a count register name, will not be
%    recognized by \cw{checkinteger}. There seems to be no bulletproof
%    way to allow these possibilities.
%
%    Tests that hide \cw{checkinteger} under the hood, such as a
%    \cw{nonnegativeinteger} test, are not provided because as often as
%    not the number being prompted for will have to be tested to see if
%    it falls inside a more specific range, such as 0\dash 255 for an
%    8-bit number or 1\dash 31 for a date, and it seems common sense to
%    omit overhead if it would usually be redundant. It's easy enough to
%    define such a test for yourself, if you want one.
%
%\begin{usage}
%\checkdimen\reply\tempdim
%\end{usage}
%    Analog of \cw{checkinteger} for dimension values. If \cw{reply}
%    does not contain a valid dimension the value returned in
%    \cw{tempdim} will be \verb'-'\cw{maxdimen}.
%
%    Only explicit dimensions with decimal digits, optional decimal
%    point and more decimal digits, followed by explicit units
%    \verb'pt' \verb'cm' \verb'in' or whatever are checked for; some
%    valid \tex/ dimensions such as \cw{parindent},
%    \verb'.3'\cw{baselineskip}, or \cw{fontdimen}\verb'5'\cw{font}
%    will not be recognized by \cw{checkdimen}.
%
% \subsection*{What good is all this?}
%    What good is all this stuff, practically speaking?\Dash you may
%    ask. Well, a typical application might be something like: At the
%    beginning of a document, prompt interactively to find out if the
%    user wants to print on A4 or US letter-size paper, or change the
%    top or left margin. Such a query could be done like this:
%
%^^V \promptxmesj{
%^^V Do you want to print on A4 or US letter paper?
%^^V Enter u or U for US letter, anything else for A4: }
%^^V \readChar{A}\reply % default = A4
%^^V \if U\reply \textheight=11in \textwidth=8.5in
%^^V \else \textheight=297mm \textwidth=210mm \fi
%^^V %    Subtract space for 1-inch margins
%^^V \addtolength{\textheight}{-2in}
%^^V \addtolength{\textwidth}{-2in}
%^^V
%^^V \promptxmesj{
%^^V Left margin setting? [Return = keep current value,
%^^V \the\oddsidemargin]: }
%^^V \readline{\the\oddsidemargin}\reply
%^^V \checkdimen\reply{\dimen0}
%^^V \ifdim\dimen0>-\maxdimen
%^^V   \setlength\oddsidemargin{\dimen0}%
%^^V   \xmesj{OK, using new left margin of %
%^^V \the\oddsidemargin.}
%^^V \else
%^^V   \xmesj{Sorry, I don't understand %
%^^V that reply: `\reply'.\
%^^V Using default value: \the\oddsidemargin.}
%^^V \fi
%
%    Although \latex/'s \cw{typeout} and \cw{typein} functions can
%    be used for this same task, they are rather more awkward, and
%    checking the margin value for validity would be quite difficult.
%
% \StopEventually{}
%
% \section{Implementation}
%    Standard package identification:
%    \begin{macrocode}
%<*2e>
\NeedsTeXFormat{LaTeX2e}
\ProvidesPackage{dialog}[1994/11/08 v0.9y]
%</2e>
%    \end{macrocode}
%%%%%%%%%%%%%%%%%%%%%%%%%%%%%%%%%%%%%%%%%%%%%%%%%%%%%%%%%%%%%%%%%%%%%%
% \section{Preliminaries}
%
%    \begin{macrocode}
%<*2e>
\RequirePackage{grabhedr}
%</2e>
%    \end{macrocode}
%
%    If \fn{grabhedr.sty} is not already loaded, load it now. The
%    \cw{trap.input} function is explained in \fn{grabhedr.doc}.
%    \begin{macrocode}
%<*209>
\csname trap.input\endcsname
\input grabhedr.sty \relax
\fileversiondate{dialog.sty}{0.9y}{1994/11/08}%
%</209>
%    \end{macrocode}
%
%    The functions \cw{localcatcodes} and \cw{restore\-catcodes} are
%    defined in \fn{grabhedr.sty}. We use them to save and restore
%    catcodes of any special characters needed in this file whose
%    current catcodes might not be what we want them to be. Saving and
%    restoring catcode of at-sign \qc{\@} makes this file work equally
%    well as a \latex/ documentstyle option or as a simple input file in
%    other contexts. The double quote character \qc{\"} might be active
%    for German and other languages. Saving and restoring tilde \qc{\~},
%    hash \qc{\#}, caret \qc{\^}, and left quote \qc{\`} catcodes is
%    normally redundant but reduces the number of assumptions we must
%    rely on. (The following catcodes are assumed: \qc{\\} 0, \qc{\{} 1,
%    \qc{\}} 2, \qc{\%} 14, {\tt a}\dash {\tt z} {\tt A}\dash {\tt Z}
%    11, {\tt 0}\dash {\tt 9} \qc{\.} \qc{\-} 12. Also note that
%    \cw{endlinechar} is assumed to have a non-null value.)
%    \begin{macrocode}
%% The line break is significant here:
\localcatcodes{\@{11}\ {10}\
{5}\~{13}\"{12}\#{6}\^{7}\`{12}}
%    \end{macrocode}
%
%%%%%%%%%%%%%%%%%%%%%%%%%%%%%%%%%%%%%%%%%%%%%%%%%%%%%%%%%%%%%%%%%%%%%%
% \section{Definitions}
%
% \begin{macro}{\otherchars}
%
%    For deactivating characters with special catcodes during
%    \cw{readline} we use, instead of \cw{dospecials}, a more
%    bulletproof (albeit slower) combination of \cw{otherchars},
%    \cw{controlchars}, and \cw{highchars} that covers all characters in
%    the range 0\dash 255 except letters and digits. Handling the
%    characters above 127 triples the overhead done for each read
%    operation or message definition but seems mandatory for maximum
%    robustness.^^A
%^^A%%%%%%%%%%%%%%%%%%%%%%%%%%%%%%%%%%%%%%%%%%%%%%%%%%%%%%%%%%%%%%%%%%
% \footnote{If you are using \fn{dialog.sty} functions on a slow
% computer, you might want to try setting \cw{highchars} = empty to
% see if that helps the speed.}
%^^A%%%%%%%%%%%%%%%%%%%%%%%%%%%%%%%%%%%%%%%%%%%%%%%%%%%%%%%%%%%%%%%%%%
%
%    \cw{otherchars} includes the thirty-three nonalphanumeric visible
%    characters (counting space as visible). It is intended as an
%    executable list like \cw{dospecials} but, as an exercise in
%    memory conservation, it is constructed without the \cw{do}s. For
%    the usual application of changing catcodes, the list can still be
%    executed nicely as shown below. Also, if we arrange to
%    make sure that each character token gets category 12, it's not
%    necessary to use control symbols such as \cs{\%} in place of
%    those few special characters that would otherwise be difficult to
%    place inside of a definition. This avoids a problem that would
%    otherwise arise if we included \cs{\+} in the list and tried to
%    process the list with a typical definition of \cs{do}: \cs{\+} is
%    `outer' in plain \tex/ and would cause an error message when
%    \cw{do} attempted to read it as an argument. (As a matter of fact
%    the catcode changes below show a different way around that
%    problem, but a list of category-12 character tokens is a fun
%    thing to have around anyway.)
%
%    \begin{macrocode}
\begingroup
%    \end{macrocode}
%    First we start a group to localize \cw{catcode} changes. Then we
%    change all relevant catcodes to 12 except for backslash, open
%    brace, and close brace, which can be handled by judicious
%    application of \cw{escapechar}, \cw{string}, \cw{edef}, and
%    \cw{xdef}. By defining \cw{do} in a slightly backward way, so that
%    it doesn't take an argument, we don't need to worry about the
%    presence of \cs{\+} in the list of control symbols. Notice the
%    absence of \cs{\`} from the list of control symbols; it was already
%    catcoded to 12 in the \cw{localcatcodes} declaration at the
%    beginning of this file\Dash otherwise it would be troublesome to
%    make the definition of \cw{do} bulletproof (consider the
%    possibilities that \qc{\`} might have catcode 0, 5, 9, or 14).
%
%    \begin{macrocode}
\def\do{12 \catcode`}
\catcode`\~\do\!\do\@\do\#\do\$\do\^\do\&
\do\*\do\(\do\)\do\-\do\_\do\=\do\[\do\]
\do\;\do\:\do\'\do\"\do\<\do\>\do\,\do\.
\do\/\do\?\do\|12\relax
%    \end{macrocode}
%    To handle backslash and braces, we define \cs{\\},
%    \cs{\{}, and \cs{\}} to produce the corresponding category-12
%    character tokens. Setting \cw{escapechar} to $-1$ means that
%    \cw{string} will omit the leading backslash that it would
%    otherwise produce when applied to a control sequence.
%    \begin{macrocode}
\escapechar -1
\edef\\{\string\\}
\edef\{{\string\{}\edef\}{\string\}}
%    \end{macrocode}
%    Space and percent are done last. Then, with almost all the
%    special characters now category 12, it's rather easy to define
%    \cw{otherchars}.
%    \begin{macrocode}
\catcode`\ =12\catcode`\%=12
\xdef\otherchars
{ !"#$%&'()*+,-./:;<=>?[\\]^_`\{|\}~}
\endgroup %              ^     ^  ^
%    \end{macrocode}
% \end{macro}
%
% \begin{macro}{\controlchars}
%
%    \cw{controlchars} is another list for the control characters
%    \ascii/ 0\dash 31 and 127. The construction of this list is similar
%    to the construction of \cw{otherchars}. We need to turn off
%    \cw{endlinechar} because the catcode of \ctrl{M} is going to be
%    changed. The \ctrl{L} inside the \cw{gdef} is not a problem (as it
%    might have been, due to the usual outerness of \ctrl{L}) because
%    the catcode is changed from 13 to 12 before that point.
%
%    \begin{macrocode}
\begingroup
\endlinechar = -1
\def\do{12 \catcode`}
\catcode`\^^@\do\^^A\do\^^B\do\^^C
\do\^^D\do\^^E\do\^^F\do\^^G\do\^^H\do\^^I
\do\^^J\do\^^K\do\^^L\do\^^M\do\^^N\do\^^O
\do\^^P\do\^^Q\do\^^R\do\^^S\do\^^T\do\^^U
\do\^^V\do\^^W\do\^^X\do\^^Y\do\^^Z\do\^^[
\do\^^\\do\^^]\do\^^^\do\^^_\do\^^? 12\relax
%
\gdef\controlchars{^^@^^A^^B^^C^^D^^E^^F^^G
  ^^H^^I^^J^^K^^L^^M^^N^^O^^P^^Q^^R^^S^^T
  ^^U^^V^^W^^X^^Y^^Z^^[^^\^^]^^^^^_^^?}
\endgroup
%    \end{macrocode}
% \end{macro}
%
% \begin{macro}{\highchars}
%    And finally, the list \cw{highchars} contains characters
%    128\dash 255, the ones that have the eighth bit set.
%
%    \begin{macrocode}
\begingroup
\def\do{12 \catcode`}
\catcode`\^^80\do\^^81\do\^^82\do\^^83\do\^^84
\do\^^85\do\^^86\do\^^87\do\^^88\do\^^89\do\^^8a
\do\^^8b\do\^^8c\do\^^8d\do\^^8e\do\^^8f
\do\^^90\do\^^91\do\^^92\do\^^93\do\^^94\do\^^95
%    \end{macrocode}
%\verbdots\iffalse
\do\^^96\do\^^97\do\^^98\do\^^99\do\^^9a\do\^^9b
\do\^^9c\do\^^9d\do\^^9e\do\^^9f
\do\^^a0\do\^^a1\do\^^a2\do\^^a3\do\^^a4\do\^^a5
\do\^^a6\do\^^a7\do\^^a8\do\^^a9\do\^^aa\do\^^ab
\do\^^ac\do\^^ad\do\^^ae\do\^^af
\do\^^b0\do\^^b1\do\^^b2\do\^^b3\do\^^b4\do\^^b5
\do\^^b6\do\^^b7\do\^^b8\do\^^b9\do\^^ba\do\^^bb
\do\^^bc\do\^^bd\do\^^be\do\^^bf
\do\^^c0\do\^^c1\do\^^c2\do\^^c3\do\^^c4\do\^^c5
\do\^^c6\do\^^c7\do\^^c8\do\^^c9\do\^^ca\do\^^cb
\do\^^cc\do\^^cd\do\^^ce\do\^^cf
\do\^^d0\do\^^d1\do\^^d2\do\^^d3\do\^^d4\do\^^d5
\do\^^d6\do\^^d7\do\^^d8\do\^^d9\do\^^da\do\^^db
\do\^^dc\do\^^dd\do\^^de\do\^^df
\do\^^e0\do\^^e1\do\^^e2\do\^^e3\do\^^e4\do\^^e5
\do\^^e6\do\^^e7\do\^^e8\do\^^e9\do\^^ea\do\^^eb
\do\^^ec\do\^^ed\do\^^ee\do\^^ef
\do\^^f0\do\^^f1\do\^^f2\do\^^f3\do\^^f4\do\^^f5
\do\^^f6\do\^^f7\do\^^f8\do\^^f9\do\^^fa\do\^^fb
%\fi
%    \begin{macrocode}
\do\^^fc\do\^^fd\do\^^fe\do\^^ff 12\relax
%
\gdef\highchars{%
^^80^^81^^82^^83^^84^^85^^86^^87^^88%
^^89^^8a^^8b^^8c^^8d^^8e^^8f%
^^90^^91^^92^^93^^94^^95^^96^^97^^98%
%    \end{macrocode}
%\verbdots\iffalse
^^99^^9a^^9b^^9c^^9d^^9e^^9f%
^^a0^^a1^^a2^^a3^^a4^^a5^^a6^^a7^^a8%
^^a9^^aa^^ab^^ac^^ad^^ae^^af%
^^b0^^b1^^b2^^b3^^b4^^b5^^b6^^b7^^b8%
^^b9^^ba^^bb^^bc^^bd^^be^^bf%
^^c0^^c1^^c2^^c3^^c4^^c5^^c6^^c7^^c8%
^^c9^^ca^^cb^^cc^^cd^^ce^^cf%
^^d0^^d1^^d2^^d3^^d4^^d5^^d6^^d7^^d8%
^^d9^^da^^db^^dc^^dd^^de^^df%
^^e0^^e1^^e2^^e3^^e4^^e5^^e6^^e7^^e8%
^^e9^^ea^^eb^^ec^^ed^^ee^^ef%
^^f0^^f1^^f2^^f3^^f4^^f5^^f6^^f7^^f8%
%\fi
%    \begin{macrocode}
^^f9^^fa^^fb^^fc^^fd^^fe^^ff}
\endgroup
%    \end{macrocode}
% \end{macro}
%
% \begin{macro}{\actively}
%
%    The function \cw{actively} makes a given character active and
%    carries out the assignment given as the first argument. The
%    assignment can be embedded in the replacement text of a macro
%    definition without requiring any special setup to produce an active
%    character in the replacement text. The argument should be a control
%    symbol, e.g. \cw{@} or \cs{\#} or \cs{\^^M}, rather than a single
%    character. (Except that \qc{\+} is safer than \cs{\+} in
%    \plaintex/.) If the assignment is a definition (\cw{def},
%    \cw{edef}, \cw{gdef}, \cw{xdef}) it is allowed to take arguments in
%    the normal \tex/ way. Prefixes such as \cw{global}, \cw{long}, or
%    \cw{outer} must go inside the first argument rather than before
%    \cw{actively}.
%
%    Usage:
%\begin{usage}
%\actively\def\?{<replacement text>}
%\actively\def\%#1#2{<replacement text>}
%\actively{\global\let}\^^@=\space
%\end{usage}
%    One place where this function can be put to good use is in
%    making \ctrl{M} active in order to get special action at the end
%    of each line of input. The usual way of going about this would be
%    to write
%
%^^V \def\par{something}\obeylines
%
%    \noindent which is a puzzling construction to the \tex/ novice
%    who doesn't know what \cw{obeylines} does with \cw{par}. The same
%    effect could be gotten a little more transparently with
%
%^^V \actively\def\^^M{something}
%
%    In the definition of \cw{actively} we use the unique properties
%    of \cw{lowercase} to create an active character with the right
%    character code, overlapping with a \cw{begingroup} \cw{endgroup}
%    structure that localizes the necessary lc-code change.
%    \begin{macrocode}
\def\actively#1#2{\catcode`#2\active
  \begingroup \lccode`\~=`#2\relax
  \lowercase{\endgroup#1~}}
%    \end{macrocode}
% \end{macro}
%
%%%%%%%%%%%%%%%%%%%%%%%%%%%%%%%%%%%%%%%%%%%%%%%%%%%%%%%%%%%%%%%%%%%%%%
%
% \begin{macro}{\mesjsetup}
%
%    The \cw{mesjsetup} function starts a group to localize catcode
%    changes. The group will be closed eventually by a separate
%    function that does the actual sending or stores the message text
%    for later retrieval.
%
%    We want to change the catcode of each character in the three
%    lists \cw{otherchars}, \cw{controlchars}, and \cw{highchars} to
%    12. After giving \cw{do} a recursive definition, we apply it to
%    each of the three lists, adding a suitable element at the end of
%    the list to make the recursion stop there. This allows
%    leaving out the \cw{do} tokens from the character lists, without
%    incurring the cost of an if test at each recursion step.
%    \begin{macrocode}
\def\mesjsetup{\begingroup \count@=12
  \def\do##1{\catcode`##1\count@ \do}%
%    \end{macrocode}
%    The abbreviation \cw{xp@} = \cw{expandafter} is from
%    \fn{grabhedr.sty}.
%    \begin{macrocode}
  \xp@\do\otherchars{a11 \@gobbletwo}%
  \xp@\do\controlchars{a11 \@gobbletwo}%
  \xp@\do\highchars{a11 \@gobbletwo}%
%    \end{macrocode}
%    Make the tab character produce eight spaces:
%    \begin{macrocode}
  \actively\edef\^^I{ \space\space\space
    \space\space\space\space}%
%    \end{macrocode}
%    The convenient treatment of newlines in the argument of \cw{mesj}
%    (every line break produces a line break on screen) is achieved by
%    making the \ctrl{M} character active and defining it to produce a
%    category-12 \ctrl{J} character. Although for \cw{mesj} it would
%    have sufficed to make \ctrl{M} category 12 and locally set
%    \cw{newlinechar} = \ctrl{M} while sending the message, it turns out
%    to be useful for other functions to have the \ctrl{M} character
%    active, so that it can be remapped to an arbitrary function for
%    handling new lines (e.g., perhaps adding extra spaces at the
%    beginning of each line). And if \cw{mesj} treats \ctrl{M} the same,
%    we can arrange for it to share the setup routines needed for the
%    other functions.
%    \begin{macrocode}
  \endlinechar=`\^^M\actively\let\^^M=\relax
  \catcode`\{=1 \catcode`\}=2 }
%    \end{macrocode}
% \end{macro}
%
% \begin{macro}{\sendmesj}
%
%    In \cw{sendmesj} we go to a little extra trouble to make sure
%    \ctrl{M} produces a newline character, no matter what the value of
%    \cw{newlinechar} might be in the surrounding environment. The
%    impending \cw{endgroup} will restore \cw{newlinechar} to its
%    previous value. One reason for using \ctrl{J} (instead of, say,
%    \ctrl{M} directly) is to allow e.g. \cw{mesj}\verb'{xxx^^Jxxxx}' to
%    be written inside a definition, as is sometimes convenient. This
%    would be difficult with \ctrl{M} instead of \ctrl{J} because of
%    catcodes.
%
%    \begin{macrocode}
\def\sendmesj{\newlinechar`\^^J%
  \actively\def\^^M{^^J}%
  \immediate\write\sixt@@n{\mesjtext}\endgroup}
%    \end{macrocode}
% \end{macro}
%
% \begin{macro}{\mesj}
%
%    Given the support functions defined above, the definition of
%    \cw{mesj} is easy: Use \cw{mesjsetup} to clear all special
%    catcodes, then set up \cw{sendmesj} to be triggered by the next
%    assignment, then read the following balanced-braces group into
%    \cw{mesjtext}. As soon as the definition is completed, \tex/ will
%    execute \cw{sendmesj}, which will send the text and close the
%    group that was started in \cw{mesjsetup} to localize the catcode
%    changes.
%
%    \begin{macrocode}
\def\mesj{\mesjsetup \afterassignment\sendmesj
  \def\mesjtext}
%    \end{macrocode}
% \end{macro}
%
% \begin{macro}{\sendprompt}
%
%    The \cw{sendprompt} function is just like \cw{sendmesj} except
%    that it uses \cw{message} instead of \cw{write}, as might be
%    desired when prompting for user input, so that the on-screen
%    cursor stays on the same line as the prompt instead of hopping
%    down to the beginning of the next line. In order for newlines to
%    work with \cw{message} we must use a visible character instead of
%    \ctrl{J}. When everyone has \tex/ version 3.1415 or later this
%    will no longer be true. The choice of \qc{\!} might be construed
%    (if you wish) as editorial comment that \qc{\!} should not be
%    shouted at the user in a prompt.
%
%    \begin{macrocode}
\def\sendprompt{%
  \newlinechar`\!\relax \actively\def\^^M{!}%
  \message{\mesjtext}\endgroup}
%    \end{macrocode}
% \end{macro}
%
% \begin{macro}{\promptmesj}
%
%    This function is like \cw{mesj} but uses \cw{sendprompt} instead of
%    \cw{sendmesj}.
%
%    \begin{macrocode}
\def\promptmesj{\mesjsetup
  \afterassignment\sendprompt \def\mesjtext}
%    \end{macrocode}
% \end{macro}
%
% \begin{macro}{\storemesj}
%
%    Arg \arg{1} of \cw{storemesj} is the control sequence under which
%    the message text is to be stored.
%    \begin{macrocode}
\def\storemesj#1{\mesjsetup
  \catcode`\#=6 % to allow arguments if needed
  \afterassignment\endgroup
  \long\gdef#1}
%    \end{macrocode}
% \end{macro}
%
% \begin{macro}{\fmesj}
%
%    While \cw{storemesj}\cw{foo}\verb"{...}" is more or less the
%    same as \cw{def}\cw{foo}\verb"{...}" with special catcode
%    changes, \cw{fmesj}\cw{foo}\verb"{...}" corresponds to
%    \cw{def}\cw{foo}\qc{\{}\cw{mesj}\verb"{...}}", that is, after
%    \cw{fmesj}\cw{foo} the function \cw{foo} can be executed directly
%    to send the message. Thus \cw{storemesj} is typically used for
%    storing pieces of messages, while \cw{fmesj} is used for storing
%    entire messages.
%
%    To read the parameter text \arg{2}, we use the peculiar
%    \verb'#{' feature of \tex/ to read everything up to the opening
%    brace.
%    \begin{macrocode}
\def\fmesj#1#2#{\mesjsetup
  \catcode`\#=6 % restore to normal
%    \end{macrocode}
%    The parameter text \arg{2} must be stored in a token register
%    rather than a macro to avoid problems with \qc{\#} characters.
%    The \cw{long} prefix is just to admit the (unlikely) possibility
%    of using \cw{fmesj} to define something such as an error message
%    saying `You can't use \arg{1} here' where one of the
%    possibilities for \arg{1} is \cw{string}\cw{par}.
%    \begin{macrocode}
  \toks@{\long\gdef#1#2}%
%    \end{macrocode}
%    Define \cw{@tempa} to put together the first two arguments and
%    [pseudo]argument \arg{3} and make the definition of \arg{1}.
%    \begin{macrocode}
  \def\@tempa{%
    \edef\@tempa{%
      \the\toks@{%
%    \end{macrocode}
%    The abbreviation \cw{nx@} = \cw{noexpand} is from
%    \fn{grabhedr.sty}.
%    \begin{macrocode}
      \begingroup\def\nx@\mesjtext{\the\toks2 }%
        \nx@\sendmesj}%
    }%
    \@tempa
    \endgroup % Turn off the \mesjsetup catcodes
  }%
  \afterassignment\@tempa
  \toks2=}
%    \end{macrocode}
% \end{macro}
%
% \begin{macro}{\xmesjsetup}
%
%    \cw{xmesjsetup} is like \cw{mesjsetup} except it prepares to allow
%    control sequence tokens and normal comments in the message text.
%    For \tex/nicians' convenience certain other features are
%    thrown in.
%
%    Here, unlike the setup for \cw{xreadline}, I don't bother to
%    remove the outerness of \cw{bye}, \cw{newif}, etc., because I
%    presume the arguments of \cw{xmesj}, \cw{fxmesj},
%    \cw{storexmesj}, \cw{fxmenu}, etc.\ are more likely to be written
%    by a \tex/nician than by an average end user, whereas
%    \cw{xreadline} is designed to handle arbitrary input from
%    arbitrary users.
%
%    \begin{macrocode}
\def\xmesjsetup{\mesjsetup
%    \end{macrocode}
%    Throw in pseudo braces just in case we are inside an \cw{halign}
%    with \cs{\\} let equal to \cw{cr} at the time when \cw{xmesjsetup}
%    is called. (As might happen in \amstex/.)
%    \begin{macrocode}
  \iffalse{\fi
  \catcode`\\=0 \catcode`\%=14
%    \end{macrocode}
%    Define {\def~{\spacefactor1500 \space}%
%    \cs{\%}~\cs{\\}~\cs{\{}~\cs{\}}~\cs{\&}} to produce the
%    corresponding single characters, category 12. The \cw{lowercase}
%    trick here allows these definitions to be nonglobal.
%    \begin{macrocode}
  \begingroup \lccode`\0=`\\\lccode`\1=`\{%
  \lccode`\2=`\}\lccode`\3=`\%%
  \lowercase{\endgroup \def\\{0}\def\{{1}%
    \def\}{2}\def\%{3}}%
  \iffalse}\fi
  \edef\&{\string &}%
%    \end{macrocode}
%    Let \qc{\&} \qc{\=} \cw{noexpand} for expansion control inside the
%    argument text; let active \ctrl{M} \qc{\=} \cw{relax} so that
%    newlines will remain inert during the expansion.
%    \begin{macrocode}
  \actively\let\&=\noexpand
  \actively\let\^^M=\relax
%    \end{macrocode}
%    Define \cs{\.} to be a noop, for terminating a control word when
%    it is followed by letters and no space is wanted.
%    \begin{macrocode}
  \def\.{}%
%    \end{macrocode}
%    Support for use of \cw{xmesj} inside a definition replacement
%    text or macro argument: control-space
%    \verb*"\ " = \cw{space}, tilde \qc{\~} prints as itself,
%    \cs{\^^M} (i.e., a lone backslash at the end of a line) will
%    produce a newline, also \cs{\^^J}, while finally \cw{par} \qc{\=}
%    blank line translates to two newlines.
%    \begin{macrocode}
  \def\ { }\edef~{\string ~}%
%    \end{macrocode}
%    Define \cs{\^^M} to produce an active \ctrl{M} character, which (we
%    hope) will be suitably defined to produce a newline or whatever.
%    \begin{macrocode}
  \begingroup \lccode`\~=`\^^M%
    \lowercase{\endgroup \def\^^M{~}}%
  \let\^^J\^^M \def\par{\^^M\^^M}%
}
%    \end{macrocode}
% \end{macro}
%
% \begin{macro}{\xmesj}
%
%    \cw{xmesj} uses \cw{xmesjsetup} and \cw{edef}.
%    \begin{macrocode}
\def\xmesj{\xmesjsetup \afterassignment\sendmesj
  \edef\mesjtext}
%    \end{macrocode}
% \end{macro}
%
% \begin{macro}{\promptxmesj}
%
%    \cw{promptxmesj} is analogous to \cw{promptmesj}, but with
%    expansion.
%    \begin{macrocode}
\def\promptxmesj{\xmesjsetup
  \afterassignment\sendprompt \edef\mesjtext}
%    \end{macrocode}
% \end{macro}
%
% \begin{macro}{\storexmesj}
%
%    And \cw{storexmesj} is like \cw{storemesj}, with expansion. Since
%    we allow arguments for the function being defined, we also must
%    define \cs{\#} to produce a single category-12 \qc{\#} character so
%    that there will be a way to print \qc{\#} in the message text.
%    \begin{macrocode}
\def\storexmesj#1#2#{\xmesjsetup
  \catcode`\#=6 % to allow arguments if needed
  \edef\#{\string##}%
  \afterassignment\endgroup
  \long\xdef#1#2}
%    \end{macrocode}
% \end{macro}
%
% \begin{macro}{\fxmesj}
%
%    And \cw{fxmesj} is the expansive analog of \cw{fmesj}.
%    \begin{macrocode}
\def\fxmesj#1#2#{\xmesjsetup
  \catcode`\#=6 % restore to normal
  \edef\#{\string##}%
  \toks@{\long\xdef#1#2}%
  \def\@tempa{%
    \edef\@tempa{%
      \the\toks@{\begingroup
      \def\nx@\nx@\nx@\mesjtext{\the\toks\tw@}%
      \nx@\nx@\nx@\sendmesj}}%
    \@tempa % execute the constructed xdef
    \endgroup % restore normal catcodes
  }%
  \afterassignment\@tempa
  \toks\tw@=}
%    \end{macrocode}
% \end{macro}
%
%%%%%%%%%%%%%%%%%%%%%%%%%%%%%%%%%%%%%%%%%%%%%%%%%%%%%%%%%%%%%%%%%%%%%%
% \section{Reading functions}
%
% \begin{macro}{\readline}
%    The \cw{readline} function gets one line of input from the user.
%    Arguments are: \arg{1} default to be used if the user response is
%    empty (i.e., if the user just pressed the return/enter key),
%    \arg{2} macro to receive the input.
%
%    \begin{macrocode}
\def\readline#1#2{%
  \begingroup \count@ 12 %
  \def\do##1{\catcode`##1\count@ \do}%
  \xp@\do\otherchars{a11 \@gobbletwo}%
  \xp@\do\controlchars{a11 \@gobbletwo}%
  \xp@\do\highchars{a11 \@gobbletwo}%
%    \end{macrocode}
%    Make spaces and tabs normal instead of category 12.
%    \begin{macrocode}
  \catcode`\ =10 \catcode`\^^I=10 %
  \catcode`\^^M=9 % ignore
%    \end{macrocode}
%    Reset end-of-line char to normal, just in case.
%    \begin{macrocode}
  \endlinechar`\^^M
%    \end{macrocode}
%    We go to a little trouble to avoid \cw{gdef}-ing \arg{2}, in order
%    to prevent save stack buildup if the user of \cw{readline} carries
%    on unaware doing local redefinitions of \arg{2} after the initial
%    read.
%    \begin{macrocode}
  \read\m@ne to#2%
  \edef#2{\def\nx@#2{#2}}%
  \xp@\endgroup #2%
  \ifx\@empty#2\def#2{#1}\fi
}
%    \end{macrocode}
% \end{macro}
%
% \begin{macro}{\xreadline}
%
%    \cw{xreadline} is like \cw{readline} except that it leaves almost
%    all catcodes unchanged so that the return value is executable
%    tokens instead of strictly character tokens of category 11 or 12.
%    \begin{macrocode}
\def\xreadline#1#2{%
  \begingroup
%    \end{macrocode}
%    Render some outer control sequences innocuous.
%    \begin{macrocode}
    \xp@\let\csname bye\endcsname\relax
    \xp@\let\csname newif\endcsname\relax
    \xp@\let\csname newcount\endcsname\relax
    \xp@\let\csname newdimen\endcsname\relax
    \xp@\let\csname newskip\endcsname\relax
    \xp@\let\csname newmuskip\endcsname\relax
    \xp@\let\csname newtoks\endcsname\relax
    \xp@\let\csname newbox\endcsname\relax
    \xp@\let\csname newinsert\endcsname\relax
    \xp@\let\csname +\endcsname\relax
    \actively\let\^^L\relax
  \catcode`\^^M=9 % ignore
  \endlinechar`\^^M% reset to normal
  \read\m@ne to#2%
  \toks@\xp@{#2}%
  \edef\@tempa{\def\nx@#2{\the\toks@}}%
  \xp@\endgroup \@tempa
  \ifx\@empty#2\def#2{#1}\fi
}
%    \end{macrocode}
% \end{macro}
%
% \begin{macro}{\readchar}
%
%    \cw{readchar} reduces the user response to a single character.
%    \begin{macrocode}
\def\readchar#1#2{%
  \readline{#1}#2%
%    \end{macrocode}
%    If the user's response and the default response are both empty,
%    we need something after \arg{1} to keep \cw{@car} from running
%    away, so we add an empty pair of braces.
%    \begin{macrocode}
  \edef#2{\xp@\@car#2#1{}\@nil}%
}
%    \end{macrocode}
% \end{macro}
%
% \begin{macro}{\readChar}
%
%    \cw{readChar} reduces the user response to a single uppercase
%    character. (This is useful to simplify testing the response
%    later with \cw{if}.)
%
%    \begin{macrocode}
\def\readChar#1#2{%
  \readline{#1}#2%
  \changecase\uppercase#2%
%    \end{macrocode}
%    Reduce \arg{2} to its first character, or the first character of
%    \arg{1}, if \arg{2} is empty. The extra braces \verb'{}' are to
%    prevent a runaway argument error from \cw{@car} if \arg{2} and
%    \arg{1} are both empty.
%    \begin{macrocode}
  \edef#2{\xp@\@car #2#1{}\@nil}%
}
%    \end{macrocode}
% \end{macro}
%
% \begin{macro}{\changecase}
%    The function \cw{changecase} uppercases or lowercases the
%    replacement text of its second argument, which
%    must be a macro. The first argument should be \cw{uppercase} or
%    \cw{lowercase}.
%    \begin{macrocode}
\def\changecase#1#2{\@casetoks\xp@{#2}%
  \edef#2{#1{\def\nx@#2{\the\@casetoks}}}#2}
%    \end{macrocode}
% \end{macro}
%
% \begin{macro}{\@casetoks}
%    We allocate a token register just for the use of \cw{changecase}
%    because it might be used at a low level internally where we don't
%    want to interfere with other uses of the scratch token registers
%    0\dash 9.
%    \begin{macrocode}
\newtoks\@casetoks
%    \end{macrocode}
% \end{macro}
%
%    A common task in reading user input is to verify, when an answer
%    of a certain kind was requested, that the response has indeed the
%    desired form\Dash for example, if a nonnegative integer is required
%    for subsequent processing, it behooves us to verify that we have
%    a nonnegative integer in hand before doing anything that might
%    lead to inconvenient error messages. However, it's not easy to
%    decide how best to handle such verification. One possibility
%    might be to have a function
%
%^^V \readnonnegativeinteger\foo
%
%    to do all the work of going out and fetching a number from the
%    user and leaving it in the macro \cw{foo}. Another possibility
%    would be to read the response using \cw{readline} and then apply
%    a separate function that can be used in combination with \cw{if},
%    for example
%
%^^V   \readline{}\reply
%^^V   \if\validnumber\reply ... \else ... \fi
%
%    For maximum flexibility, a slightly lower-level approach is
%    chosen here. The target syntax is
%
%^^V   \readline{}\reply
%^^V   \checkinteger\reply\tempcount
%
%    where \cw{tempcount} will be set to \verb'-'\cw{maxdimen} if
%    \cw{reply} does {\em not\/} contain a valid integer. (Negative
%    integers are allowed, as long as they are greater than
%    \verb'-'\cw{maxdimen}.)  Then the function that calls
%    \cw{checkinteger} is free to make additional checks on the range
%    of the reply and give error messages tailored to the
%    circumstances. And the handling of an empty \cw{reply} can be
%    arbitrarily customized, something that would tend to be
%    inconvenient for the first method mentioned.
%
%    The first and second approaches can be built on top of the third
%    if desired, e.g.\ (for the second approach)
%
%^^V \def\validnumber#1{TT\fi
%^^V   \checkinteger#1\tempcount%
%^^V   \ifnum\tempcount>-\maxdimen }
%
%    The curious \verb'TT\fi'\verbdots\cw{ifnum} construction is from
%    \TeXhax{} 1989, no.~20 and no.~38 (a suggestion of D. E. Knuth in
%    reply to a query by S. von Bechtolsheim).
%
% \begin{macro}{\checkinteger}
%
%    The arguments of \cw{checkinteger}'s are: \arg{2}, a count register
%    to hold the result; \arg{1}, a macro holding zero or more arbitrary
%    characters of category 11 or 12.
%
%    \begin{macrocode}
\def\checkinteger#1#2{\let\scansign@\@empty
  \def\scanresult@{#2}%
  \xp@\scanint#1x\endscan}
%    \end{macrocode}
% \end{macro}
%
% \begin{macro}{\scanint}
%
%    To validate a number, the function \cw{scanint} must first scan
%    away leading \qc{\+} or \qc{\-} signs (keeping track in
%    \cw{scansign@}), then look at the first token after that: if it's
%    a digit, fine, scan that digit and any succeeding digits into the
%    given count register (\cw{scanresult@}), ending with \cw{endscan}
%    to get rid of any following garbage tokens that might just
%    possibly show up.
%
%    Typical usage of \cw{scanint} includes initializing \cw{scansign@}
%    to empty, as in the definition of \cw{checkinteger}.
%\begin{usage}
%\let\scansign@\@empty
%\def\scanresult@{\tempcount}%
%\xp@\scanint\reply x\endscan
%\end{usage}
%    Assumption: \cw{reply} is either empty or contains only category
%    11 or 12 characters (which it will if you used \cw{readline}!). If
%    a separate check is done earlier to trap the case where \cw{reply}
%    is empty\Dash for example, by using a nonempty default for
%    \cw{readline}\Dash then the \verb'x' before \cw{endscan} is
%    superfluous.
%
%    Arg \arg{1} = next character from the string being tested.
%    The test whether \arg{1} is a decimal digit is similar in spirit
%    to the test \verb'\if!#1!' to see if an argument is empty
%    (\TB, Appendix~D, p.~376).
%    \begin{macrocode}
\def\scanint#1{%
  \ifodd 0#11 %
%    \end{macrocode}
%    Is \arg{1} a decimal digit? If so read all digits into
%    \cw{scanresult@} with the sign prefix.
%    \begin{macrocode}
    \def\@tempa{\afterassignment\endscan
      \scanresult@=\scansign@#1}%
  \else
    \if -#1\relax
%    \end{macrocode}
%    Here we flipflop the sign; watch closely.
%    \begin{macrocode}
      \edef\scansign@{%
        \ifx\@empty\scansign@ -\fi}%
      \def\@tempa{\scanint}%
    \else
%    \end{macrocode}
%    A plus sign can just be ignored.
%    \begin{macrocode}
      \if +#1\relax
        \def\@tempa{\scanint}%
      \else % not a valid number
        \def\@tempa{%
          \scanresult@=-\maxdimen\endscan}%
  \fi\fi\fi
  \@tempa
}
%    \end{macrocode}
% \end{macro}
%
% \begin{macro}{\endscan}
%    The \cw{endscan} function just gobbles any remaining garbage. It
%    uses its own name as the argument delimiter.
%    \begin{macrocode}
\def\endscan#1\endscan{}
%    \end{macrocode}
% \end{macro}
%
% \begin{macro}{\dimenfirstpart}
% \begin{macro}{\dimentoks}
%    \cw{dimenfirstpart}, a count register, receives the digits, if
%    any, preceding the decimal point. \cw{dimentoks}, a token
%    register, receives any digits after the decimal point.
%    \begin{macrocode}
\newcount\dimenfirstpart
\newtoks\dimentoks
%    \end{macrocode}
% \end{macro}
% \end{macro}
%
% \begin{macro}{\scandimen}
%
%    \cw{scandimen} is similar to \cw{scanint} but has to call some
%    auxiliary functions to scan the various subcomponents of a
%    dimension (leading digits, decimal point, fractional part, and
%    units, with optional {\tt true}, in addition to the sign). The
%    minimum requirements of \tex/'s syntax for dimensions are a digit
%    or decimal point \qc{\+} the units; all the other components are
%    optional (\TB, Exercise~10.3, p.~58).
%
%    When scanning for the digits of a fractional part, we can't throw
%    away leading zeros; therefore we don't read the fractional part
%    into a count register as we did for the digits before the decimal
%    point; instead we read the digits one by one and store them in
%    \cw{dimentoks}.
%
%    The function that calls \cw{scandimen} should initialize
%    \cw{scansign@} to \cw{@empty}, \cw{dimenfirstpart} to \cw{z@},
%    \cw{dimentoks} to empty \verb'{}', and \cw{dimentrue@} to
%    \cw{@empty}.
%
%    Test values: {\tt 0pt}, {\tt 1.1in}, {\tt -2cm}, {\tt .3mm}, {\tt
%    0.4dd}, {\tt 5.cc}, {\tt.10000000009pc}, \cw{hsize}, {\tt em}.
%
%    \begin{macrocode}
\def\scandimen#1{%
  \ifodd 0#11
    \def\@tempa{\def\@tempa{\scandimenb}%
      \afterassignment\@tempa
      \dimenfirstpart#1}%
  \else
%    \end{macrocode}
%
%    The following test resolves to true if \arg{1} is either a period
%    or a comma (both recognized by \tex/ as decimal point
%    characters).
%
%    \begin{macrocode}
    \if \if,#1.\else#1\fi.%
      \def\@tempa{\scandimenc}%
    \else
      \if -#1% then flipflop the sign
        \edef\scansign@{%
          \ifx\@empty\scansign@ -\fi}%
        \def\@tempa{\scandimen}%
      \else
        \if +#1% then ignore it
          \def\@tempa{\scandimen}%
        \else % not a valid dimen
          \def\@tempa{%
            \scanresult@=-\maxdimen\endscan}%
  \fi\fi\fi\fi
  \@tempa
}
%    \end{macrocode}
% \end{macro}
%
% \begin{macro}{\scandimenb}
%
%    Scan for an optional decimal point.
%
%    \begin{macrocode}
\def\scandimenb#1{%
  \if \if,#1.\else#1\fi.%
    \def\@tempa{\scandimenc}%
  \else
%    \end{macrocode}
%    If the decimal point is absent, we need to put back \arg{2} and
%    rescan it to see if it is the first letter of the units.
%    \begin{macrocode}
    \def\@tempa{\scanunitsa#1}%
  \fi
  \@tempa
}
%    \end{macrocode}
% \end{macro}
%
% \begin{macro}{\scandimenc}
%
%    Scan for the fractional part: digits after the decimal point.
%
%    \begin{macrocode}
\def\scandimenc#1{%
%    \end{macrocode}
%    If \arg{1} is a digit, add it to \cw{dimentoks}.
%    \begin{macrocode}
  \ifodd 0#11 \dimentoks\xp@{%
      \the\dimentoks#1}%
    \def\@tempa{\scandimenc}%
  \else
%    \end{macrocode}
%    Otherwise rescan \arg{1}, presumably the first letter of the
%    units.
%    \begin{macrocode}
  \def\@tempa{\scanunitsa#1}%
  \fi
  \@tempa
}
%    \end{macrocode}
% \end{macro}
%
% \begin{macro}{\scanunitsa}
%
%    \begin{macrocode}
\def\scanunitsa#1\endscan{%
%    \end{macrocode}
%    Check for \verb'true' qualifier.
%    \begin{macrocode}
  \def\@tempa##1true##2##3\@tempa{##2}%
%    \end{macrocode}
%
%    The peculiar nature of \cw{lowercase} is evident here as we are
%    able to apply it to only the test part of the conditional without
%    running into brace-matching problems. (Compare the braces in this
%    example to something like \cw{message}\qc{\{}\cw{iffalse}
%    \verb"A}"\cw{else} \verb"B}"\cw{fi}.)
%
%    \begin{macrocode}
  \lowercase{%
    \xp@\ifx\xp@\end
    \@tempa#1true\end\@tempa
  }%
%    \end{macrocode}
%    No \verb'true' was found:
%    \begin{macrocode}
    \let\dimentrue@\@empty
    \def\@tempa{\scanunitsb#1\endscan}%
  \else
    \def\dimentrue@{true}%
    \def\@tempa##1true##2\@tempa{%
      \def\@tempa{##1}%
      \ifx\@tempa\@empty
        \def\@tempa{\scanunitsb##2\endscan}%
      \else
        \def\@tempa{\scanunitsb xx\endscan}%
      \fi}%
    \@tempa#1\@tempa
  \fi
  \@tempa
}
%    \end{macrocode}
% \end{macro}
%
% \begin{macro}{\scanunitsb}
%
%    Scan for the name of the units and complete the assignment of the
%    scanned value to \cw{scanresult@}. Notice that, because of the way
%    \cw{scanunitsb} picks up \arg{1} and \arg{2} as macro arguments,
%    \verb"p t" is allowed as a variation of \verb"pt". Eliminating
%    this permissiveness doesn't seem worth the speed penalty
%    that would be incurred in \cw{scanunitsb}.
%
%    The method for detecting a valid units string is to define the
%    scratch function \cw{@tempa} to apply \tex/'s parameter-matching
%    abilities to a special string that will yield a boolean value
%    of true if and only if the given string is a valid \tex/ unit.
%    \begin{macrocode}
\def\scanunitsb#1#2{%
  \def\@tempa##1#1#2##2##3\@nil{##2}%
  \def\@tempb##1{T\@tempa
    pcTptTcmTccTemTexTinTmmTddTspT##1F\@nil}%
%    \end{macrocode}
%    Force lowercase just in case the units were entered with
%    uppercase letters (accepted by \tex/, so we had better accept
%    uppercase also).
%    \begin{macrocode}
  \lowercase{%
  \if\@tempb{#1#2}%
  }%
   \scanresult@=\scansign@
     \number\dimenfirstpart.\the\dimentoks
     \dimentrue@#1#2\relax
  \else
    \scanresult@=-\maxdimen
  \fi
%    \end{macrocode}
%    Call \cw{endscan} to gobble garbage tokens, if any.
%    \begin{macrocode}
  \endscan
}
%    \end{macrocode}
% \end{macro}
%
% \begin{macro}{\checkdimen}
%
%    Argument \arg{2} must be a dimen register; \arg{1} is expected to
%    be a macro holding zero or more arbitrary characters of category 11
%    or 12.
%    \begin{macrocode}
\def\checkdimen#1#2{%
  \let\scansign@\@empty \def\scanresult@{#2}%
  \let\dimentrue@\@empty
  \dimenfirstpart\z@ \dimentoks{}%
  \xp@\scandimen#1xx\endscan
}
%    \end{macrocode}
% \end{macro}
%
%    Finish up.
%    \begin{macrocode}
\restorecatcodes
\endinput
%    \end{macrocode}
%
% \CheckSum{858}
% \Finale
