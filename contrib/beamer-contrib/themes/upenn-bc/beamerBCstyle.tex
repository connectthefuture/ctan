%% beamerBCstyle.tex
%% Copyright 2008-2013 Georg Strasser < georg.strasser@bc.edu >
%
% This work may be distributed and/or modified under the
% conditions of the LaTeX Project Public License, version 1.3
% or later. The latest version of this license is in
% http://www.latex-project.org/lppl.txt
% and version 1.3 or later is part of all distributions of LaTeX
% version 2005/12/01 or later.
%
% This work has the LPPL maintenance status `maintained'.
%
% The Current Maintainer of this work is Georg Strasser.
%
% This work consists of the files beamercolorthemegoeagles.sty,
% beamercolorthemepenn.sty, and beamerBCstyle.tex.

% This is an example file for beamercolorthemegoeagles.sty and 
% beamercolorthemepenn.sty.

\documentclass[dvips]{beamer}
\usetheme{Warsaw}
\usecolortheme{goeagles}
%\usecolortheme{penn}
\usepackage[english]{babel}
\usepackage{xspace}
\usepackage{lmodern}
\usepackage[T1]{fontenc} 

\title[Color Themes]{Color Themes\\for the \LaTeX \xspace beamer class}
\author[Georg Strasser]{Georg Strasser}
\institute{Boston College}
\date{March 2013}


\begin{document}
\frame{\titlepage}


\section[Package]{The Package}
\frame{\sectionpage}

\subsection{Introduction}

\begin{frame}
\frametitle{Color Themes}
This is a $\mathrm{L\!\!^{{}_{\scriptstyle A}} \!\!\!\!\!\;\; T\!_{\displaystyle E} \! X}$ beamer-theme in the colors of the
\begin{enumerate}
\item University of Pennsylvania, USA, and 
\item Boston College, USA.
\end{enumerate}
Both were tested for the presentation theme ``Warsaw''.

\vspace{0.3cm}

Please note that these color themes are neither official nor exact! The colors are approximated from the universities' style guidelines and websites, and slightly modified where necessary to generate an appealing look. The universities neither endorse, nor provide any support for, these color themes. 
\end{frame}

\subsection{Installation and Usage}

\begin{frame}
\frametitle{Installation}
\begin{itemize}
\item Move the files beamercolorthemepenn.sty and beamercolorthemegoeagles.sty into a global or local directory which is searched by tex, ideally in the beamer tree.
For example, move the folder ``color'' to 

C:\textbackslash TeX\textbackslash MiKTeX 2.9\textbackslash tex\textbackslash latex\textbackslash beamer\textbackslash base\textbackslash themes\textbackslash 

or to a new folder like

C:\textbackslash TeX\textbackslash MiKTeX 2.9\textbackslash tex\textbackslash latex\textbackslash beamer\textbackslash themes\textbackslash.
\item Then run the MiKTeX settings application and update the formats and file name database (FNDB).
\end{itemize}
\end{frame}

\begin{frame}
\frametitle{Usage}

Add \textbackslash usepackage\{goeagles\} or \textbackslash usepackage\{penn\} \emph{after} loading the beamer package and the Warsaw theme to your TeX file.

\end{frame}




\section{Examples}
\frame{\sectionpage}

\subsection{Environments}

\frame{
\begin{definition}
A weird number is a natural number that is abundant but not semiperfect.
\end{definition}
}

\frame{
\begin{theorem}
The following proof is true.
\end{theorem}

\begin{proof}
The previous theorem is false.
\end{proof}
}

\frame{
\begin{exampleblock}{Example}
``... the [government] expenditure of money is better justified when it is made for walls, docks, harbors, aqueducts, and all those works which are of service to the community. There is, to be sure, more of present satisfaction in what is handed out, like cash down; nevertheless public improvements win us greater gratitude with posterity.'' (Cicero, De officiis, Book II: Expediency, XVII. (60))
\end{exampleblock}
}

\frame{
\begin{alertblock}{An Old Alert}
``We must, therefore, take measures that there shall be no indebtedness of a nature to endanger the public safety. It is a menace that can be averted in many ways; but should a serious debt be incurred, we are not to allow the rich to lose their property, while the debtors profit by what is their neighbor's. For there is nothing that upholds a government more powerfully than its credit; and it can have no credit, unless the payment of debts is enforced by law.'' (Cicero, De officiis, Book II: Expediency, XXIV. (84))
\end{alertblock}
}

\subsection{Enumeration}

\frame{\frametitle{Enumeration}
\begin{itemize}[<+-| alert@+>]
\item One million dollars
\item Two billion dollars
\item Three trillion dollars
\item Four quadrillion dollars
\begin{itemize}
\item There is no upper bound to nominal quantities.
\end{itemize}

\end{itemize}
\begin{enumerate}
\item One ocean
\item Two oceans
\end{enumerate}
}

\subsection{Computational Neuroscience}
\frame{...}

\end{document}