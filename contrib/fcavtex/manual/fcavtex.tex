%% fcavtex.tex, v-1.1 waltermaldonado
%% Copyright 2015 by Walter Maldonado Jr 
%%
%% This work may be distributed and/or modified under the
%% conditions of the LaTeX Project Public License, either version 1.3
%% of this license or (at your option) any later version.
%% The latest version of this license is in
%%   http://www.latex-project.org/lppl.txt
%% and version 1.3 or later is part of all distributions of LaTeX
%% version 2005/12/01 or later.
%%
%% This work has the LPPL maintenance status `maintained'.
%%
%% The Current Maintainer of this work is Walter Maldonado Jr
%%
%% Creator and original mantainer: Walter Maldonado Jr <walter@agroestat.com.br>

\documentclass[a4paper]{ltxdoc}
\usepackage{lmodern}			% Usa a fonte Latin Modern			
\usepackage[T1]{fontenc}		% seleção de códigos de fonte.
\usepackage[utf8]{inputenc}		% determina a codificação utiizada (conversão automática dos acentos)
\usepackage{hyperref}  			% controla a formação do índice
\usepackage{parskip}			% espaçamento entre os parágrafos
\usepackage{microtype} 			% para melhorias de justificação
\usepackage{morefloats}			% permite mais floats


% Babel e ajustes
\usepackage[brazil]{babel}		% idiomas
\addto\captionsbrazil{
    %% ajusta nomes padroes do babel
    \renewcommand{\bibname}{Refer\^encias}
    \renewcommand{\indexname}{\'Indice}
    \renewcommand{\listfigurename}{Lista de ilustra\c{c}\~{o}es}
    \renewcommand{\listtablename}{Lista de tabelas}
    %% ajusta nomes usados com a macro \autoref
    \renewcommand{\pageautorefname}{p\'agina}
    \renewcommand{\sectionautorefname}{se{\c c}\~ao}
    \renewcommand{\subsectionautorefname}{subse{\c c}\~ao}
    \renewcommand{\paragraphautorefname}{par\'agrafo}
    \renewcommand{\subsubsectionautorefname}{subse{\c c}\~ao}
    \renewcommand{\paragraphautorefname}{subse{\c c}\~ao}
}  

\title{\textbf{A classe \textsf{fcavTeX}} \\ \Large{Teses e dissertações \\ Faculdade de Ciências Agrárias e Veterinárias de Jaboticabal}}

%   \thanks{Este documento
%   se referete ao \textsf{abntex2} versão \fileversion,
%   de \filedate.}
  
\author{Walter Maldonado Jr\\walter@agroestat.com.br} 

\date{\today, v-1.1}

\hypersetup{
	pdftitle={A classe fcavTeX},
	pdfauthor={Walter Maldonado Jr},
	pdfsubject={Teses e dissertações da Faculdade de Ciências Agrárias e Veterinárias de Jaboticabal}, 
	pdfkeywords={FCAV}{UNESP}{trabalho acadêmico}{dissertação}{tese}{Jaboticabal}, 
	pdfproducer={Walter Maldonado Jr -- walter@agroestat.com.br}, 	% producer of the document
    pdfcreator={LaTeX with fcavTeX},
	colorlinks=true,
	linkcolor=blue,
	citecolor=blue,
	urlcolor=blue
}

\EnableCrossrefs
\CodelineIndex
\RecordChanges

\changes{v1.0}{2015/06/27}{Versão inicial}

\usepackage{xcolor}
\usepackage{listings}

\lstset
{
    language=[LaTeX]TeX,
    breaklines=true,
    basicstyle=\tt\scriptsize,
    keywordstyle=\color{blue},
    identifierstyle=\color{black},
    extendedchars=true,
    literate={á}{{\'a}}1 {ã}{{\~a}}1 {é}{{\'e}}1 {ó}{{\'o}}1 {ç}{{\c{c}}}1 {í}{{\'i}}1 {Ç}{{\c{C}}}1 {Ã}{{\~A}}1 {Â}{{\^A}}1 {ô}{{\^o}}1 {õ}{{\~o}}1,
}

\begin{document}


\maketitle

\begin{abstract}
A formatação de um trabalho
acadêmico é sempre uma tarefa árdua, mecânica e cansativa. Quando
nos deparamos com um trabalho extenso e com diversas referências bibliográficas,
semanas de trabalho se vão. A classe fcavTeX poupa o usuário de horas perdidas
em frente a um editor de texto limitado que não irá garantir a padronização do 
seu documento. Com um conhecimento básico de \LaTeX \ é possível escrever toda a
sua tese ou dissertação sem precisar se preocupar com tamanho de fonte, espaçamento, etc.
Quantos espaços após o título? Será que esse título é em negrito? São perguntas
que não merecem desperdício de tempo excessivo de um pesquisador.
\end{abstract}

\tableofcontents

\listoftables
 
% ------
\section{Introdução}
% ------

O progresso da tecnologa é fascinante. A cada dia que passa aumentam as possiblilidades
e os recursos que ela nos oferece. Se compararmos algumas tarefas que executamos hoje com a maneira
com que eram desenvolvidas dez anos atrás veremos que há um progresso impressionante. Tomem como exemplo os \emph{smartphones} 
e como ficou fácil gerenciar a nossa agenda de compromissos, que além de tudo nos alerta quando esquecemos de algo,
ao nosso alcance 24h por dia.

E nós ainda estamos escrevendo nossos trabalhos acadêmicos como escrevíamos há 25 anos. Escrevemos um título, selecionamos
e mudamos os atributos de fonte. Quando muito, temos estilos pré-definidos para essa tarefa, o que ajuda um pouco, mas
não é suficiente. Digo que não é suficiente pois, se estamos fazendo um trabalho acadêmico, com certeza estamos seguindo 
diversas regras que nos são impostas. Regras essas que \emph{todos} os alunos deverão seguir.

Pois bem, quando estava decidindo como escreveria minha tese de doutorado, tais pensamentos me vieram à mente. Depois
de um tempo de reflexão cheguei à seguinte conclusão: ``Irei escrever minha tese utilizando o \LaTeX!!''. Dessa maneira,
seria possível, além de garantir um nível de qualidade superior ao meu trabalho, transformar os padrões que
desenvolveria em uma classe para que todos os alunos a pudessem utilizar e economizar muito tempo.

E aqui está a fcavTeX. Espero que seja útil e me coloco a disposição para questionamentos e para que possamos melhorá-la.
Tenho certeza de que, com um conhecimento básico de \LaTeX, muito tempo poderá ser economizado e a qualidade estética
dos trabalhos de nossa universidade será melhorada consideravelmente.

\section{Exemplo de utlilização} % (fold)
\label{sec:exemplo_de_utliliza_o}
O uso da classe é extremamente intuitivo. Os comandos são auto-decritivos e basta trocar o conteúdo entre as chaves dentro
do seu arquivo .tex. Atualmente, somente a estrutura de tese em capítulos é suportada, mas o modelo convencional também será
incluído. Todos os recursos apresentados pela classe estão em conformidade com as normas da universidade, que podem ser conferidas 
\href{http://www.fcav.unesp.br/Home/posgraduacao/normas_disss_tese.pdf}{aqui}. Segue o código do exemplo.

\begin{lstlisting}
    \documentclass{fcavtex}

    \begin{document}

    \titulo{ESTIMATIVA DA PRODUÇÃO DE CITROS USANDO IMAGENS DIGITAIS}
    \tituloingles{Citrus yield estimation using digital images}
    \autor{Walter Maldonado Jr}
    \orientador{Prof. Dr. José Carlos Barbosa}
    \qualificacaoautor{Engenheiro agrônomo}
    \instituicao{UNIVERSIDADE ESTADUAL PAULISTA - UNESP\par CÂMPUS DE JABOTICABAL}
    \tipodoc{Tese}
    \titulopretendido{Doutor}
    \programa{Agronomia (Produção Vegetal)}
    \ano{2015}

    \capa
    \folhaderosto
    \fichacatalografica
    \certificadodeaprovacao
    \dadoscurriculares{Dados curriculares aqui!}
    \epigrafe{Epígrafe}
    \dedicatoria{Dedicatória}
    \agradecimentos{Agradecimentos aqui!}
    \sumario
    \include{resumo}
    \abstract{}{}
    \listadetabelas
    \listadefiguras

    \corpodotextoemcapitulos

    \include{cap1}

    \end{document}
\end{lstlisting}

O comando \emph{include} chama o arquivo \emph{cap1.tex}, que deverá ter o seguinte formato para que todas as citações feitas com 
os comandos \emph{\textbackslash cite} possam gerar a lista de referências da maneira correta.

\begin{lstlisting}
    \chapter{Considerações gerais}
    \section{Introdução}

    Texto texto texto texto.

    \section{Revisão de Literatura}

    Primeiro parágrado da seção Revisão de Literatura. Texto texto texto.

    Segundo parágrafo.

    \bibliography{meuarquivobibtex.bib}

\end{lstlisting}

% section exemplo_de_utliliza_o (end)

\section{Overlead e shareLaTeX} % (fold)
\label{sec:sharelatex}
O Overlead e o shareLaTeX são serviços gratuitos (apresentando algumas versões aprimoradas pagas) que podem ser utilizados para gerar a sua dissertação ou tese. A principal vantagem desse serviço é que ele dispensa o download e instalação das pesadas distribuições LaTeX na sua máquina local. Sendo assim, basta você acessar o website \href{http://www.overleaf.com}{http://www.overleaf.com} ou \href{https://pt.sharelatex.com/}{https://pt.sharelatex.com/}, criar uma conta e gerar os seus documentos \LaTeX\ na hora!

Foram disponibilizados projetos modelo para que seja possível checar como ocorre o processo de edição e compilação dos arquivos nos dois serviços citados. Eles podem ser acessados através dos links \href{https://www.overleaf.com/latex/templates/univeresidade-estadual-paulista-unesp-thesis-template/tgggvvxccvqv}{Overleaf} e \href{https://pt.sharelatex.com/project/558eb0eed8509a876d9e2e51}{shareLaTeX}. 

No momento da compilação, os utilitários do \LaTeX\ buscam pelos arquivos de classe e de estilos bibliográficos na mesma pasta do arquivo principal do seu projeto (\emph{.tex}). Portanto, basta adicionar os arquivos \emph{fcavtex.bst} e \emph{fcavtex.cls} na mesma pasta do seu projeto. Tal procedimento vale para uma pasta em seu computador local, caso tenha instalado uma distribuição \LaTeX\ ou para um novo projeto no shareLaTeX, onde os arquivos podem ser visualizados na lista à esquerda na tela. Tais arquivos podem ser obtidos no site de desenvolvimento da classe \href{https://github.com/waltermaldonado/fcavTeX}{https://github.com/waltermaldonado/fcavTeX} e submetidos ao seu projeto pela ferramenta apropriada.
% section sharelatex (end)

\section{A classe \emph{memoir}} % (fold)
\label{sec:a_classe_memoir}
A classe base para o desenvolvimento do projeto foi a \emph{memoir}. É uma classe muito conhecida, com recursos que facilitam a implementação das regras e que é vastamente documentada. Ao escrever o seu documento \LaTeX, você pode utililzar todos os recursos da classe, de acordo com a documentação disponível no link \href{http://texdoc.net/texmf-dist/doc/latex/memoir/memman.pdf}{http://texdoc.net/texmf-dist/doc/latex/memoir/memman.pdf}.
% section a_classe_memoir (end)

\section{Modificações da classe abnTeX2} % (fold)
\label{sec:modifica_es_da_classe_abntex2}
Não foi possível utilizar a classe abnTeX2 em sua forma original para o cumprimento das normas da FCAV/UNESP.\@ O único módulo utilizado foi o de estilos bibliográficos para citações do tipo autor, ano, listadas em ordem alfabética. Na seção 1.3.2.1 das normas são expostas as regras da lista de referências e algumas delas não estavam em concordância com o disponível pela classe. As modificações realizadas foram:
\begin{description}
    \item[Nomes iniciais dos autores não estavam sendo abreviados] Foi necessário corrigir esse comportamento para que os nomes inicias fossem abreviados.
    \item[Citação de trabalho em evento, impresso, eletrônico e CD-ROM] Foi adicionado um recurso para a inclusão dos termos `Anais\ldots' e `Anais eletrônicos\ldots', além do ajuste da ordem dos elementos volume, ano, local de publicação e editor. Tais citações são originadas do tipo BibTeX \emph{inproceedings}, listado no Zotero como `Conferência'.
    \item[Citação de capítulo em livro] O nome do livro nas citações do tipo \emph{inbook} ou \emph{incollection} foi alterado para todas as letras e maiúscula e o título do livro colocado em ênfase (negrito no caso, exigido pelas normas).
    \item[Data de acesso do documento eletrônico] No caso dos documentos que possuem url e que a data de acesso está presente (\emph{urldate}), a classe abnTeX2 coloca como `Acesso em: 2015-06-27', por exemplo, apenas copiando os valores presentes no campo do registro. O registro deve ser preenchido com a data no formato do exemplo e o estilo foi alterado para corrigir essa ordem e colocar o nome abreviado do mês.
\end{description}

% section modifica_es_da_classe_abntex2 (end)

\section{Colabore com o fcavTeX} % (fold)
\label{sec:colabore_com_o_fcavtex}
Você pode colaborar com o projeto através do site no GitHub. Há uma seção específica para o apontamento de bugs e sugestões (\emph{Issues}) e, se desejar, você pode clonar o nosso projeto em sua máquina, fazer as modificações que julgar necessárias e submete-las novamente. O git é um sistema de controle de versão, que permite a diversos colaboradores a modificação dos originais de um sistema e a incorporação dessas modificações mediante aprovação do responsável pelo projeto. Também é possível que um repositório git seja utilizado para criar um novo projeto.

Na prática, o repositório git é uma pasta, na qual os arquivos do projeto residem. A cada vez que você modifica um arquivo nessa pasta, o git detecta essa ação e, quando você executa um \emph{commit}, essas alterações geram uma nova versão do sistema, passível de ser recuperada caso seja necessário após novos \emph{commits}. Essas alterações, por fim, são sincronizadas com um servidor (no caso o GitHub).

Você pode também endereçar os seus comentários e sugestões para o e-mail disponível no início deste documento.
% section colabore_com_o_fcavtex (end)


\end{document}
