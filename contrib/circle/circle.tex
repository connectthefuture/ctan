\documentclass[DIV=9, parskip=half, pagesize=auto]{scrartcl}

\usepackage{fixltx2e}
\usepackage{etex}
\usepackage{xspace}
\usepackage{lmodern}
\usepackage[T1]{fontenc}
\usepackage{textcomp}
\usepackage{microtype}
\usepackage[unicode=true]{hyperref}

\newcommand*{\mail}[1]{\href{mailto:#1}{\texttt{#1}}}
\newcommand*{\pkg}[1]{\textsf{#1}}
\newcommand*{\cs}[1]{\texttt{\textbackslash#1}}
\makeatletter
\newcommand*{\cmd}[1]{\cs{\expandafter\@gobble\string#1}}
\makeatother

\addtokomafont{title}{\rmfamily}

\title{The \pkg{circle} package}
\author{Klaus G. Barthelmann, \mail{barthel@informatik.uni-mainz.de}}
\date{1998/07/15}


\begin{document}

\maketitle

As the name says, \cmd{\Circle} gives a circle in math mode. Its size lies between
that of the binary operator \cmd{\circ} and that of the unary operator \cmd{\bigcirc}.
It can be used as the nextstep operator of temporal logic in conjunction with
\cmd{\Box} and \cmd{\Diamond} (\pkg{latexsym}) or \cmd{\square} and \cmd{\lozenge} (\pkg{amssymb}). \verb+\Circle[f]+
gives a filled circle.

As you probably know, L.~Lamport discouraged the use of the nextstep
operator for program verification. This could be the reason that he did not
provide a symbol for it in \LaTeX.

The circles are taken from the \pkg{lcircle10} font. We try to choose the
appropriate size. If you need a high quality output, this solution will not
suit you.

Bugs: circles have size $n$ at fontsize $2n-1$ and $2n$; they do not scale linearly
depending on the fontsize

\end{document}
