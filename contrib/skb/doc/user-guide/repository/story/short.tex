\opt{text}{\skbheading{The Short Story}}

I have written papers, done a lot of presentations, provided some book chapters,
still working on a book, participated in many research proposals and projects, and
created tons of notes and figures. As of early 2009, most of that information was
distributed over the repositories of different projects and organisations I worked
for, in some document management systems, on several websites, databases, my preferred
email client (which changed twice), different computers and later even different
external hard drives and \skbacft{USB} sticks. Looking for specific text or a particular figure
could easily end in a days work. Tools like desktop search engines can help to find 
'stuff'. I used them, but if they found anything it was hard to maintain the context
it was written in and some formats or sources were out of reach for them. Even worse
with figures and the many versions some of them evolved in over time. After multiple
jobs and several years, all I had is kind of a very messy base of knowledge, well-hidden
somewhere, thus very difficult to locate and impossible to maintain.

So I started early 2009 to re-organise my 'stuff'. At the same time, I did realise that
moving away from \LaTeX~was part of the problem (and I thought using the other text 
processor would help, it actually didn't, long-term). So \LaTeX~became, again, the 
text processor of choice, and with it the ability for a complete different approach
to organise my 'stuff'. This was the moment the \skbacft{A3DS:SKB} was created. \skbacft{A3DS:SKB} stands for Sven's
Knowledge Base. The \LaTeX~package \skbem[code]{skb}, described in this article, forms part of a larger
software system that uses \skbacft{ISO:SQL}ite databases, a small \skbacft{PHP} framework, Apache for \skbacft{W3C:HTML} access
and recently also a Java port.

My document repository uses the \skbem[code]{skb} package, so most of my documents are eventually
\LaTeX~documents. I am saying eventually because I still use other tools (like Microsoft's Powerpoint),
but integrate their output in my repository. I do all my figures these days using Inkscape, so the source
is \skbacft{W3C:SVG} and the output for \LaTeX~documents \skbacft{ISO:PDF}. For editing the text files I do flip between UE Studio
and LeD. Parts of the content (such as acronyms and bibliographic information) are maintained in \skbacft{ISO:SQL}ite 
databases and exported to \LaTeX. This package now shows how I build my documents.