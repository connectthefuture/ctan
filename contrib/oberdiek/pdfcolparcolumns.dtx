% \iffalse meta-comment
%
% File: pdfcolparcolumns.dtx
% Version: 2016/05/16 v1.4
% Info: Color stacks for parcolumns
%
% Copyright (C) 2007, 2008, 2010 by
%    Heiko Oberdiek <heiko.oberdiek at googlemail.com>
%    2016
%    https://github.com/ho-tex/oberdiek/issues
%
% This work may be distributed and/or modified under the
% conditions of the LaTeX Project Public License, either
% version 1.3c of this license or (at your option) any later
% version. This version of this license is in
%    http://www.latex-project.org/lppl/lppl-1-3c.txt
% and the latest version of this license is in
%    http://www.latex-project.org/lppl.txt
% and version 1.3 or later is part of all distributions of
% LaTeX version 2005/12/01 or later.
%
% This work has the LPPL maintenance status "maintained".
%
% This Current Maintainer of this work is Heiko Oberdiek.
%
% This work consists of the main source file pdfcolparcolumns.dtx
% and the derived files
%    pdfcolparcolumns.sty, pdfcolparcolumns.pdf, pdfcolparcolumns.ins,
%    pdfcolparcolumns.drv, pdfcolparcolumns-test1.tex.
%
% Distribution:
%    CTAN:macros/latex/contrib/oberdiek/pdfcolparcolumns.dtx
%    CTAN:macros/latex/contrib/oberdiek/pdfcolparcolumns.pdf
%
% Unpacking:
%    (a) If pdfcolparcolumns.ins is present:
%           tex pdfcolparcolumns.ins
%    (b) Without pdfcolparcolumns.ins:
%           tex pdfcolparcolumns.dtx
%    (c) If you insist on using LaTeX
%           latex \let\install=y% \iffalse meta-comment
%
% File: pdfcolparcolumns.dtx
% Version: 2016/05/16 v1.4
% Info: Color stacks for parcolumns
%
% Copyright (C) 2007, 2008, 2010 by
%    Heiko Oberdiek <heiko.oberdiek at googlemail.com>
%    2016
%    https://github.com/ho-tex/oberdiek/issues
%
% This work may be distributed and/or modified under the
% conditions of the LaTeX Project Public License, either
% version 1.3c of this license or (at your option) any later
% version. This version of this license is in
%    http://www.latex-project.org/lppl/lppl-1-3c.txt
% and the latest version of this license is in
%    http://www.latex-project.org/lppl.txt
% and version 1.3 or later is part of all distributions of
% LaTeX version 2005/12/01 or later.
%
% This work has the LPPL maintenance status "maintained".
%
% This Current Maintainer of this work is Heiko Oberdiek.
%
% This work consists of the main source file pdfcolparcolumns.dtx
% and the derived files
%    pdfcolparcolumns.sty, pdfcolparcolumns.pdf, pdfcolparcolumns.ins,
%    pdfcolparcolumns.drv, pdfcolparcolumns-test1.tex.
%
% Distribution:
%    CTAN:macros/latex/contrib/oberdiek/pdfcolparcolumns.dtx
%    CTAN:macros/latex/contrib/oberdiek/pdfcolparcolumns.pdf
%
% Unpacking:
%    (a) If pdfcolparcolumns.ins is present:
%           tex pdfcolparcolumns.ins
%    (b) Without pdfcolparcolumns.ins:
%           tex pdfcolparcolumns.dtx
%    (c) If you insist on using LaTeX
%           latex \let\install=y% \iffalse meta-comment
%
% File: pdfcolparcolumns.dtx
% Version: 2016/05/16 v1.4
% Info: Color stacks for parcolumns
%
% Copyright (C) 2007, 2008, 2010 by
%    Heiko Oberdiek <heiko.oberdiek at googlemail.com>
%    2016
%    https://github.com/ho-tex/oberdiek/issues
%
% This work may be distributed and/or modified under the
% conditions of the LaTeX Project Public License, either
% version 1.3c of this license or (at your option) any later
% version. This version of this license is in
%    http://www.latex-project.org/lppl/lppl-1-3c.txt
% and the latest version of this license is in
%    http://www.latex-project.org/lppl.txt
% and version 1.3 or later is part of all distributions of
% LaTeX version 2005/12/01 or later.
%
% This work has the LPPL maintenance status "maintained".
%
% This Current Maintainer of this work is Heiko Oberdiek.
%
% This work consists of the main source file pdfcolparcolumns.dtx
% and the derived files
%    pdfcolparcolumns.sty, pdfcolparcolumns.pdf, pdfcolparcolumns.ins,
%    pdfcolparcolumns.drv, pdfcolparcolumns-test1.tex.
%
% Distribution:
%    CTAN:macros/latex/contrib/oberdiek/pdfcolparcolumns.dtx
%    CTAN:macros/latex/contrib/oberdiek/pdfcolparcolumns.pdf
%
% Unpacking:
%    (a) If pdfcolparcolumns.ins is present:
%           tex pdfcolparcolumns.ins
%    (b) Without pdfcolparcolumns.ins:
%           tex pdfcolparcolumns.dtx
%    (c) If you insist on using LaTeX
%           latex \let\install=y% \iffalse meta-comment
%
% File: pdfcolparcolumns.dtx
% Version: 2016/05/16 v1.4
% Info: Color stacks for parcolumns
%
% Copyright (C) 2007, 2008, 2010 by
%    Heiko Oberdiek <heiko.oberdiek at googlemail.com>
%    2016
%    https://github.com/ho-tex/oberdiek/issues
%
% This work may be distributed and/or modified under the
% conditions of the LaTeX Project Public License, either
% version 1.3c of this license or (at your option) any later
% version. This version of this license is in
%    http://www.latex-project.org/lppl/lppl-1-3c.txt
% and the latest version of this license is in
%    http://www.latex-project.org/lppl.txt
% and version 1.3 or later is part of all distributions of
% LaTeX version 2005/12/01 or later.
%
% This work has the LPPL maintenance status "maintained".
%
% This Current Maintainer of this work is Heiko Oberdiek.
%
% This work consists of the main source file pdfcolparcolumns.dtx
% and the derived files
%    pdfcolparcolumns.sty, pdfcolparcolumns.pdf, pdfcolparcolumns.ins,
%    pdfcolparcolumns.drv, pdfcolparcolumns-test1.tex.
%
% Distribution:
%    CTAN:macros/latex/contrib/oberdiek/pdfcolparcolumns.dtx
%    CTAN:macros/latex/contrib/oberdiek/pdfcolparcolumns.pdf
%
% Unpacking:
%    (a) If pdfcolparcolumns.ins is present:
%           tex pdfcolparcolumns.ins
%    (b) Without pdfcolparcolumns.ins:
%           tex pdfcolparcolumns.dtx
%    (c) If you insist on using LaTeX
%           latex \let\install=y\input{pdfcolparcolumns.dtx}
%        (quote the arguments according to the demands of your shell)
%
% Documentation:
%    (a) If pdfcolparcolumns.drv is present:
%           latex pdfcolparcolumns.drv
%    (b) Without pdfcolparcolumns.drv:
%           latex pdfcolparcolumns.dtx; ...
%    The class ltxdoc loads the configuration file ltxdoc.cfg
%    if available. Here you can specify further options, e.g.
%    use A4 as paper format:
%       \PassOptionsToClass{a4paper}{article}
%
%    Programm calls to get the documentation (example):
%       pdflatex pdfcolparcolumns.dtx
%       makeindex -s gind.ist pdfcolparcolumns.idx
%       pdflatex pdfcolparcolumns.dtx
%       makeindex -s gind.ist pdfcolparcolumns.idx
%       pdflatex pdfcolparcolumns.dtx
%
% Installation:
%    TDS:tex/latex/oberdiek/pdfcolparcolumns.sty
%    TDS:doc/latex/oberdiek/pdfcolparcolumns.pdf
%    TDS:doc/latex/oberdiek/test/pdfcolparcolumns-test1.tex
%    TDS:source/latex/oberdiek/pdfcolparcolumns.dtx
%
%<*ignore>
\begingroup
  \catcode123=1 %
  \catcode125=2 %
  \def\x{LaTeX2e}%
\expandafter\endgroup
\ifcase 0\ifx\install y1\fi\expandafter
         \ifx\csname processbatchFile\endcsname\relax\else1\fi
         \ifx\fmtname\x\else 1\fi\relax
\else\csname fi\endcsname
%</ignore>
%<*install>
\input docstrip.tex
\Msg{************************************************************************}
\Msg{* Installation}
\Msg{* Package: pdfcolparcolumns 2016/05/16 v1.4 Color stacks for parcolumns (HO)}
\Msg{************************************************************************}

\keepsilent
\askforoverwritefalse

\let\MetaPrefix\relax
\preamble

This is a generated file.

Project: pdfcolparcolumns
Version: 2016/05/16 v1.4

Copyright (C) 2007, 2008, 2010 by
   Heiko Oberdiek <heiko.oberdiek at googlemail.com>

This work may be distributed and/or modified under the
conditions of the LaTeX Project Public License, either
version 1.3c of this license or (at your option) any later
version. This version of this license is in
   http://www.latex-project.org/lppl/lppl-1-3c.txt
and the latest version of this license is in
   http://www.latex-project.org/lppl.txt
and version 1.3 or later is part of all distributions of
LaTeX version 2005/12/01 or later.

This work has the LPPL maintenance status "maintained".

This Current Maintainer of this work is Heiko Oberdiek.

This work consists of the main source file pdfcolparcolumns.dtx
and the derived files
   pdfcolparcolumns.sty, pdfcolparcolumns.pdf, pdfcolparcolumns.ins,
   pdfcolparcolumns.drv, pdfcolparcolumns-test1.tex.

\endpreamble
\let\MetaPrefix\DoubleperCent

\generate{%
  \file{pdfcolparcolumns.ins}{\from{pdfcolparcolumns.dtx}{install}}%
  \file{pdfcolparcolumns.drv}{\from{pdfcolparcolumns.dtx}{driver}}%
  \usedir{tex/latex/oberdiek}%
  \file{pdfcolparcolumns.sty}{\from{pdfcolparcolumns.dtx}{package}}%
  \usedir{doc/latex/oberdiek/test}%
  \file{pdfcolparcolumns-test1.tex}{\from{pdfcolparcolumns.dtx}{test1}}%
  \nopreamble
  \nopostamble
  \usedir{source/latex/oberdiek/catalogue}%
  \file{pdfcolparcolumns.xml}{\from{pdfcolparcolumns.dtx}{catalogue}}%
}

\catcode32=13\relax% active space
\let =\space%
\Msg{************************************************************************}
\Msg{*}
\Msg{* To finish the installation you have to move the following}
\Msg{* file into a directory searched by TeX:}
\Msg{*}
\Msg{*     pdfcolparcolumns.sty}
\Msg{*}
\Msg{* To produce the documentation run the file `pdfcolparcolumns.drv'}
\Msg{* through LaTeX.}
\Msg{*}
\Msg{* Happy TeXing!}
\Msg{*}
\Msg{************************************************************************}

\endbatchfile
%</install>
%<*ignore>
\fi
%</ignore>
%<*driver>
\NeedsTeXFormat{LaTeX2e}
\ProvidesFile{pdfcolparcolumns.drv}%
  [2016/05/16 v1.4 Color stacks for parcolumns (HO)]%
\documentclass{ltxdoc}
\usepackage{holtxdoc}[2011/11/22]
\begin{document}
  \DocInput{pdfcolparcolumns.dtx}%
\end{document}
%</driver>
% \fi
%
%
% \CharacterTable
%  {Upper-case    \A\B\C\D\E\F\G\H\I\J\K\L\M\N\O\P\Q\R\S\T\U\V\W\X\Y\Z
%   Lower-case    \a\b\c\d\e\f\g\h\i\j\k\l\m\n\o\p\q\r\s\t\u\v\w\x\y\z
%   Digits        \0\1\2\3\4\5\6\7\8\9
%   Exclamation   \!     Double quote  \"     Hash (number) \#
%   Dollar        \$     Percent       \%     Ampersand     \&
%   Acute accent  \'     Left paren    \(     Right paren   \)
%   Asterisk      \*     Plus          \+     Comma         \,
%   Minus         \-     Point         \.     Solidus       \/
%   Colon         \:     Semicolon     \;     Less than     \<
%   Equals        \=     Greater than  \>     Question mark \?
%   Commercial at \@     Left bracket  \[     Backslash     \\
%   Right bracket \]     Circumflex    \^     Underscore    \_
%   Grave accent  \`     Left brace    \{     Vertical bar  \|
%   Right brace   \}     Tilde         \~}
%
% \GetFileInfo{pdfcolparcolumns.drv}
%
% \title{The \xpackage{pdfcolparcolumns} package}
% \date{2016/05/16 v1.4}
% \author{Heiko Oberdiek\thanks
% {Please report any issues at https://github.com/ho-tex/oberdiek/issues}\\
% \xemail{heiko.oberdiek at googlemail.com}}
%
% \maketitle
%
% \begin{abstract}
% Since version 1.40 \pdfTeX\ supports several color stacks.
% This package uses them to fix color problems in
% package \xpackage{parcolumns}.
% \end{abstract}
%
% \tableofcontents
%
% \section{Usage}
%
% \begin{quote}
% |\usepackage{pdfcolparcolumns}|
% \end{quote}
% The package \xpackage{pdfcolparcolumns} loads package \xpackage{parcolums}
% \cite{parcolumns}. If color stacks are available then the
% macros of \xpackage{parcolumns} are patched to add support
% for color stacks.
%
% \subsection{Option \xoption{rulebetweencolor}}
%
% Package \xpackage{pdfcolparcolumns} also fixes the color for the
% rule between columns (if \xoption{rulebetween} is set).
% Default color is \cs{normalcolor}. But this can be changed by using
% option \xoption{rulebetweencolor}. It takes a color specification
% as value. If the value is empty, then the default (\cs{normalcolor})
% is used.
% Examples:
% \begin{quote}
%   |rulebetweencolor=blue|,\\
%   |rulebetweencolor={red}|,\\
%   |rulebetweencolor={}|, \textit{\% \cs{normalcolor} is used}\\
%   |rulebetweencolor=[rgb]{1,0,.5}| \textit{\% see below}
% \end{quote}
% If used inside the optional argument of environment \textsf{parcolumns}
% and the value contains an optional argument, then whole value
% must be put in curly braces:
%\begin{quote}
%\begin{verbatim}
%\begin{parcolumns}[
%  rulebetween,
%  rulebetweencolor={[rgb]{1,0,.5}},
%]{2}
%  ...
%\end{parcolumns}
%\end{verbatim}
%\end{quote}
% This option \xoption{rulebetweencolor} can also be set using
% \cs{setkeys}:
%\begin{quote}
%\begin{verbatim}
%\setkeys{parcolumns}{rulebetweencolor=green}
%\end{verbatim}
%\end{quote}
%
% \subsection{Future}
%
% Currently package \xpackage{parcolumns} does not seem to be
% maintained. Nevertheless if there will be a new version that
% adds support for color stacks, then this package may become
% obsolete.
%
% \StopEventually{
% }
%
% \section{Implementation}
%
% \subsection{Identification}
%
%    \begin{macrocode}
%<*package>
\NeedsTeXFormat{LaTeX2e}
\ProvidesPackage{pdfcolparcolumns}%
  [2016/05/16 v1.4 Color stacks for parcolumns (HO)]%
%    \end{macrocode}
%
% \subsection{Load packages}
%
% \subsubsection{Package \xpackage{parcolumns}}
%
%    Currently package \xpackage{parcolumns} does not define options.
%    Thus it is just a precaution that the options of
%    package \xpackage{pdfcolparcolumns} are passed to
%    package \xpackage{parcolumns}.
%    \begin{macrocode}
\DeclareOption*{%
  \PassoptionsToPackage{\CurrentOption}{parcolumns}%
}
\ProcessOptions\relax
\RequirePackage{parcolumns}[2004/11/25]
%    \end{macrocode}
%
% \subsubsection{Package \xpackage{pdfcol}}
%
%    \begin{macrocode}
\RequirePackage{pdfcol}[2007/09/09]
\ifpdfcolAvailable
\else
  \PackageInfo{pdfcolparcolumns}{%
    Loading aborted, because color stacks are not available%
  }%
  \expandafter\endinput
\fi
%    \end{macrocode}
%
% \subsubsection{Package \xpackage{infwarerr}}
%
%    \begin{macrocode}
\RequirePackage{infwarerr}[2007/09/09]
%    \end{macrocode}
%
% \subsection{Color stack macros}
%
%    \begin{macro}{\pcpc@MaxStack}
%    Macro \cs{pcpc@MaxStack} holds the highest number of
%    allocated stacks.
%    \begin{macrocode}
\global\chardef\pcpc@MaxStack=\z@
%    \end{macrocode}
%    \end{macro}
%    \begin{macro}{\pcpc@InitStacks}
%    Macro \cs{pcpc@InitStacks} takes the number of columns
%    as argument and ensures that there are enough color
%    stacks for all columns.
%    \begin{macrocode}
\def\pcpc@InitStacks#1{%
  \ifnum#1>\pcpc@MaxStack
    \begingroup
      \count@\pcpc@MaxStack
      \loop
        \advance\count@\@ne
        \pdfcolInitStack{pcpc@\the\count@}%
      \ifnum#1>\count@
      \repeat
      \global\chardef\pcpc@MaxStack=\count@
    \endgroup
  \fi
}
%    \end{macrocode}
%    \end{macro}
%
%    \begin{macro}{\pcpc@SwitchStack}
%    \begin{macrocode}
\def\pcpc@SwitchStack#1{%
  \pdfcolSwitchStack{pcpc@\number#1}%
}
%    \end{macrocode}
%    \end{macro}
%
%    \begin{macro}{\pcpc@SetCurrent}
%    \begin{macrocode}
\def\pcpc@SetCurrent#1{%
  \pdfcolSetCurrent{pcpc@\number#1}%
}
%    \end{macrocode}
%    \end{macro}
%
% \subsection{Patches}
%
%     Now the color stack macros are patched into the macros
%     of package \xpackage{parcolumns}.
%
% \subsubsection{Init stacks}
%
%    \cs{pcpc@InitStacks} should go into the definition of
%    environment |parcolumns|. \cs{pc@alloccolumns} is executed
%    there and nowhere else, thus we hook into it.
%    \begin{macrocode}
\g@addto@macro\pc@alloccolumns{%
  \pcpc@InitStacks\pc@columncount
}
%    \end{macrocode}
%
% \subsubsection{Switch stack}
%
%    \cs{pcpc@SwitchStack} should be called by marco \cs{colchunk@}.
%    However it is easier to patch \cs{pc@setcolumnwidth} that
%    is executed in \cs{colchunk@} only.
%    \begin{macrocode}
\g@addto@macro\pc@setcolumnwidth{%
  \pcpc@SwitchStack\pc@columnctr
}
%    \end{macrocode}
%
% \subsubsection{Set current stack color}
%
%    \cs{pcpc@SetCurrent} is set at the begin of each line.
%    It must be inserted into \cs{pc@placeboxes}. Unhappily
%    there is no easy way. Therefore we check and
%    redefine \cs{pc@placeboxes}.
%    \begin{macrocode}
\begingroup
  \def\x{%
    \global\let\@tempa\relax
    \count@\z@
    \hb@xt@\linewidth{%
      \vfuzz30ex %
      \vbadness\@M
      \splittopskip\z@skip
      \loop
      \ifnum\count@<\pc@columncount
        \advance\count@\@ne
        \expandafter\ifvoid\csname pc@column@\number\count@\endcsname
          \hskip\csname pc@column@width@\number\count@\endcsname
        \else
          \expandafter\setbox\expandafter\@tempboxa\expandafter
          \vsplit\csname pc@column@\number\count@\endcsname
              to \dp\strutbox
          \vbox{%
            \unvbox\@tempboxa
          }%
        \fi
        \expandafter\ifvoid\csname pc@column@\number\count@\endcsname
        \else
          \global\let\@tempa\pc@placeboxes
        \fi
        \ifnum\count@<\pc@columncount
          \strut
          \hfill
          \ifpc@rulebetween
            \vrule
            \hfill
          \fi
        \fi
      \repeat
    }%
    \@tempa
  }%
  \ifx\x\pc@placeboxes
  \else
    \@PackageWarningNoLine{pdfcolparcolumns}{%
      Command \string\pc@placeboxes\space has changed.\MessageBreak
      Supported versions of package `parcolumns':\MessageBreak
      \space\space 2004/08/05.\MessageBreak
      The redefinition of \string\pc@placeboxes\space may not%
      \MessageBreak
      behave correctly depending on the changes%
    }%
  \fi
\endgroup
%    \end{macrocode}
%    \begin{macro}{\pc@placeboxes}
%    \begin{macrocode}
\renewcommand*{\pc@placeboxes}{%
  \global\let\@tempa\relax
  \count@\z@
  \hb@xt@\linewidth{%
    \vfuzz30ex %
    \vbadness\@M
    \splittopskip\z@skip
    \loop
    \ifnum\count@<\pc@columncount
      \advance\count@\@ne
      \expandafter\ifvoid\csname pc@column@\number\count@\endcsname
        \hskip\csname pc@column@width@\number\count@\endcsname
      \else
        \expandafter\setbox\expandafter\@tempboxa\expandafter
        \vsplit\csname pc@column@\number\count@\endcsname
            to \dp\strutbox
        \vbox{%
          \pcpc@SetCurrent\count@
          \unvbox\@tempboxa
        }%
      \fi
      \expandafter\ifvoid\csname pc@column@\number\count@\endcsname
      \else
        \global\let\@tempa\pc@placeboxes
      \fi
      \ifnum\count@<\pc@columncount
        \strut
        \hfill
        \ifpc@rulebetween
          \begingroup
            \pcpc@RuleBetweenColor
            \vrule
          \endgroup
          \hfill
        \fi
      \fi
    \repeat
  }%
  \@tempa
}
%    \end{macrocode}
%    \end{macro}
%    \begin{macro}{\pcpc@RuleBetweenColorDefault}
%    \begin{macrocode}
\def\pcpc@RuleBetweenColorDefault{%
  \normalcolor
}
%    \end{macrocode}
%    \end{macro}
%    \begin{macro}{\pcpc@RuleBetweenColor}
%    \begin{macrocode}
\let\pcpc@RuleBetweenColor\pcpc@RuleBetweenColorDefault
%    \end{macrocode}
%    \end{macro}
%    \begin{macrocode}
\define@key{parcolumns}{rulebetweencolor}{%
  \edef\pcpc@temp{#1}%
  \ifx\pcpc@temp\@empty
    \let\pcpc@RuleBetweenColor\pcpc@RuleBetweenColorDefault
  \else
    \edef\pcpc@temp{%
      \noexpand\@ifnextchar[{%
        \def\noexpand\pcpc@RuleBetweenColor{%
          \noexpand\color\pcpc@temp
        }%
        \noexpand\pcpc@GobbleNil
      }{%
        \def\noexpand\pcpc@RuleBetweenColor{%
          \noexpand\color{\pcpc@temp}%
        }%
        \noexpand\pcpc@GobbleNil
      }%
      \pcpc@temp\noexpand\@nil
    }%
    \pcpc@temp
  \fi
}
%    \end{macrocode}
%    \begin{macro}{\pcpc@GobbleNil}
%    \begin{macrocode}
\long\def\pcpc@GobbleNil#1\@nil{}
%    \end{macrocode}
%    \end{macro}
%
%    \begin{macrocode}
%</package>
%    \end{macrocode}
%
% \section{Test}
%
%    The test file is a modified version of the file that
%    Donald Goodman has posted in \xnewsgroup{comp.text.tex}: ^^A
%    \URL{``\link{Re: \xpackage{xcolor} glitches}''}^^A
%    {http://groups.google.com/group/comp.text.tex/msg/8eda74ed292012bb}
%    \begin{macrocode}
%<*test1>
\NeedsTeXFormat{LaTeX2e}
\AtEndDocument{%
  \typeout{}%
  \typeout{**************************************}%
  \typeout{*** \space Check the PDF file manually! \space ***}%
  \typeout{**************************************}%
  \typeout{}%
}
\documentclass{article}
\usepackage{xcolor}
\usepackage{pdfcolparcolumns}

\newcommand{\instruct}[1]{%
  \noindent
  \footnotesize
  \textcolor{red}{#1}%
}

\begin{document}
  \begin{parcolumns}[colwidths={1=2.3in,2=2.3in},sloppy]{2}%
    \colchunk[1]{%
      \instruct{Et non dicitur versus} %
      Fidelium anim\ae\ %
      \instruct{%
        sed immediate subiungitur antiphona finalis %
        beat\ae\ Mari\ae\ Virginis%
      } %
      100.%
    }%
    \colchunk[2]{%
      \instruct{%
        And the verse %
        \textcolor{black}{May the souls of the faithful} %
        is not said, but the final antiphon of the %
        Blessed Virgin Mary, %
        \textcolor{black}{100,} %
        is immediately joined.%
      }%
    }%
  \end{parcolumns}%
\end{document}
%</test1>
%    \end{macrocode}
%
% \section{Installation}
%
% \subsection{Download}
%
% \paragraph{Package.} This package is available on
% CTAN\footnote{\url{http://ctan.org/pkg/pdfcolparcolumns}}:
% \begin{description}
% \item[\CTAN{macros/latex/contrib/oberdiek/pdfcolparcolumns.dtx}] The source file.
% \item[\CTAN{macros/latex/contrib/oberdiek/pdfcolparcolumns.pdf}] Documentation.
% \end{description}
%
%
% \paragraph{Bundle.} All the packages of the bundle `oberdiek'
% are also available in a TDS compliant ZIP archive. There
% the packages are already unpacked and the documentation files
% are generated. The files and directories obey the TDS standard.
% \begin{description}
% \item[\CTAN{install/macros/latex/contrib/oberdiek.tds.zip}]
% \end{description}
% \emph{TDS} refers to the standard ``A Directory Structure
% for \TeX\ Files'' (\CTAN{tds/tds.pdf}). Directories
% with \xfile{texmf} in their name are usually organized this way.
%
% \subsection{Bundle installation}
%
% \paragraph{Unpacking.} Unpack the \xfile{oberdiek.tds.zip} in the
% TDS tree (also known as \xfile{texmf} tree) of your choice.
% Example (linux):
% \begin{quote}
%   |unzip oberdiek.tds.zip -d ~/texmf|
% \end{quote}
%
% \paragraph{Script installation.}
% Check the directory \xfile{TDS:scripts/oberdiek/} for
% scripts that need further installation steps.
% Package \xpackage{attachfile2} comes with the Perl script
% \xfile{pdfatfi.pl} that should be installed in such a way
% that it can be called as \texttt{pdfatfi}.
% Example (linux):
% \begin{quote}
%   |chmod +x scripts/oberdiek/pdfatfi.pl|\\
%   |cp scripts/oberdiek/pdfatfi.pl /usr/local/bin/|
% \end{quote}
%
% \subsection{Package installation}
%
% \paragraph{Unpacking.} The \xfile{.dtx} file is a self-extracting
% \docstrip\ archive. The files are extracted by running the
% \xfile{.dtx} through \plainTeX:
% \begin{quote}
%   \verb|tex pdfcolparcolumns.dtx|
% \end{quote}
%
% \paragraph{TDS.} Now the different files must be moved into
% the different directories in your installation TDS tree
% (also known as \xfile{texmf} tree):
% \begin{quote}
% \def\t{^^A
% \begin{tabular}{@{}>{\ttfamily}l@{ $\rightarrow$ }>{\ttfamily}l@{}}
%   pdfcolparcolumns.sty & tex/latex/oberdiek/pdfcolparcolumns.sty\\
%   pdfcolparcolumns.pdf & doc/latex/oberdiek/pdfcolparcolumns.pdf\\
%   test/pdfcolparcolumns-test1.tex & doc/latex/oberdiek/test/pdfcolparcolumns-test1.tex\\
%   pdfcolparcolumns.dtx & source/latex/oberdiek/pdfcolparcolumns.dtx\\
% \end{tabular}^^A
% }^^A
% \sbox0{\t}^^A
% \ifdim\wd0>\linewidth
%   \begingroup
%     \advance\linewidth by\leftmargin
%     \advance\linewidth by\rightmargin
%   \edef\x{\endgroup
%     \def\noexpand\lw{\the\linewidth}^^A
%   }\x
%   \def\lwbox{^^A
%     \leavevmode
%     \hbox to \linewidth{^^A
%       \kern-\leftmargin\relax
%       \hss
%       \usebox0
%       \hss
%       \kern-\rightmargin\relax
%     }^^A
%   }^^A
%   \ifdim\wd0>\lw
%     \sbox0{\small\t}^^A
%     \ifdim\wd0>\linewidth
%       \ifdim\wd0>\lw
%         \sbox0{\footnotesize\t}^^A
%         \ifdim\wd0>\linewidth
%           \ifdim\wd0>\lw
%             \sbox0{\scriptsize\t}^^A
%             \ifdim\wd0>\linewidth
%               \ifdim\wd0>\lw
%                 \sbox0{\tiny\t}^^A
%                 \ifdim\wd0>\linewidth
%                   \lwbox
%                 \else
%                   \usebox0
%                 \fi
%               \else
%                 \lwbox
%               \fi
%             \else
%               \usebox0
%             \fi
%           \else
%             \lwbox
%           \fi
%         \else
%           \usebox0
%         \fi
%       \else
%         \lwbox
%       \fi
%     \else
%       \usebox0
%     \fi
%   \else
%     \lwbox
%   \fi
% \else
%   \usebox0
% \fi
% \end{quote}
% If you have a \xfile{docstrip.cfg} that configures and enables \docstrip's
% TDS installing feature, then some files can already be in the right
% place, see the documentation of \docstrip.
%
% \subsection{Refresh file name databases}
%
% If your \TeX~distribution
% (\teTeX, \mikTeX, \dots) relies on file name databases, you must refresh
% these. For example, \teTeX\ users run \verb|texhash| or
% \verb|mktexlsr|.
%
% \subsection{Some details for the interested}
%
% \paragraph{Attached source.}
%
% The PDF documentation on CTAN also includes the
% \xfile{.dtx} source file. It can be extracted by
% AcrobatReader 6 or higher. Another option is \textsf{pdftk},
% e.g. unpack the file into the current directory:
% \begin{quote}
%   \verb|pdftk pdfcolparcolumns.pdf unpack_files output .|
% \end{quote}
%
% \paragraph{Unpacking with \LaTeX.}
% The \xfile{.dtx} chooses its action depending on the format:
% \begin{description}
% \item[\plainTeX:] Run \docstrip\ and extract the files.
% \item[\LaTeX:] Generate the documentation.
% \end{description}
% If you insist on using \LaTeX\ for \docstrip\ (really,
% \docstrip\ does not need \LaTeX), then inform the autodetect routine
% about your intention:
% \begin{quote}
%   \verb|latex \let\install=y\input{pdfcolparcolumns.dtx}|
% \end{quote}
% Do not forget to quote the argument according to the demands
% of your shell.
%
% \paragraph{Generating the documentation.}
% You can use both the \xfile{.dtx} or the \xfile{.drv} to generate
% the documentation. The process can be configured by the
% configuration file \xfile{ltxdoc.cfg}. For instance, put this
% line into this file, if you want to have A4 as paper format:
% \begin{quote}
%   \verb|\PassOptionsToClass{a4paper}{article}|
% \end{quote}
% An example follows how to generate the
% documentation with pdf\LaTeX:
% \begin{quote}
%\begin{verbatim}
%pdflatex pdfcolparcolumns.dtx
%makeindex -s gind.ist pdfcolparcolumns.idx
%pdflatex pdfcolparcolumns.dtx
%makeindex -s gind.ist pdfcolparcolumns.idx
%pdflatex pdfcolparcolumns.dtx
%\end{verbatim}
% \end{quote}
%
% \section{Catalogue}
%
% The following XML file can be used as source for the
% \href{http://mirror.ctan.org/help/Catalogue/catalogue.html}{\TeX\ Catalogue}.
% The elements \texttt{caption} and \texttt{description} are imported
% from the original XML file from the Catalogue.
% The name of the XML file in the Catalogue is \xfile{pdfcolparcolumns.xml}.
%    \begin{macrocode}
%<*catalogue>
<?xml version='1.0' encoding='us-ascii'?>
<!DOCTYPE entry SYSTEM 'catalogue.dtd'>
<entry datestamp='$Date$' modifier='$Author$' id='pdfcolparcolumns'>
  <name>pdfcolparcolumns</name>
  <caption>Fix colour problems in package 'parcolumns'.</caption>
  <authorref id='auth:oberdiek'/>
  <copyright owner='Heiko Oberdiek' year='2007,2008,2010'/>
  <license type='lppl1.3'/>
  <version number='1.4'/>
  <description>
    Since version 1.40 pdfTeX supports colour stacks.
    This package uses them to fix colour problems in
    package <xref refid='parcolumns'>parcolumns</xref>.
    <p/>
    The package is part of the <xref refid='oberdiek'>oberdiek</xref>
    bundle.
  </description>
  <documentation details='Package documentation'
      href='ctan:/macros/latex/contrib/oberdiek/pdfcolparcolumns.pdf'/>
  <ctan file='true' path='/macros/latex/contrib/oberdiek/pdfcolparcolumns.dtx'/>
  <miktex location='oberdiek'/>
  <texlive location='oberdiek'/>
  <install path='/macros/latex/contrib/oberdiek/oberdiek.tds.zip'/>
</entry>
%</catalogue>
%    \end{macrocode}
%
% \begin{thebibliography}{9}
%
% \bibitem{parcolumns}
%   Jonathan Sauer: \textit{The \xpackage{parcolumns} package};
%   2004/11/25;\\
%   \CTAN{macros/latex/contrib/sauerj/parcolumns.pdf}.
%
% \bibitem{pdfcol}
%   Heiko Oberdiek: \textit{The \xpackage{pdfcol} package};
%   2007/09/09;\\
%   \CTAN{macros/latex/contrib/oberdiek/pdfcol.pdf}.
%
% \end{thebibliography}
%
% \begin{History}
%   \begin{Version}{2007/07/26 v1.0}
%   \item
%     First version, published in the newsgroup \xnewsgroup{comp.text.tex}
%     with the name \xpackage{parcolumns-colorstacks}: ^^A no line break
%     \URL{``\link{Re: \xpackage{xcolor} glitches}''}^^A
%     {http://groups.google.com/group/comp.text.tex/msg/56bd897b11bca414}
%   \end{Version}
%   \begin{Version}{2007/09/09 v1.1}
%   \item
%     CTAN version, package name renamed to \xpackage{pdfcolparcolumns}.
%   \item
%     Uses package \xpackage{pdfcol}.
%   \item
%     Documentation added.
%   \item
%     Test file added.
%   \end{Version}
%   \begin{Version}{2008/08/11 v1.2}
%   \item
%     Code is not changed.
%   \item
%     URLs updated.
%   \end{Version}
%   \begin{Version}{2010/01/11 v1.3}
%   \item
%     Fix for rule color.
%   \item
%     New option \xoption{rulebetweencolor} for environment |parcolumns|.
%   \end{Version}
%   \begin{Version}{2016/05/16 v1.4}
%   \item
%     Documentation updates.
%   \end{Version}
% \end{History}
%
% \PrintIndex
%
% \Finale
\endinput

%        (quote the arguments according to the demands of your shell)
%
% Documentation:
%    (a) If pdfcolparcolumns.drv is present:
%           latex pdfcolparcolumns.drv
%    (b) Without pdfcolparcolumns.drv:
%           latex pdfcolparcolumns.dtx; ...
%    The class ltxdoc loads the configuration file ltxdoc.cfg
%    if available. Here you can specify further options, e.g.
%    use A4 as paper format:
%       \PassOptionsToClass{a4paper}{article}
%
%    Programm calls to get the documentation (example):
%       pdflatex pdfcolparcolumns.dtx
%       makeindex -s gind.ist pdfcolparcolumns.idx
%       pdflatex pdfcolparcolumns.dtx
%       makeindex -s gind.ist pdfcolparcolumns.idx
%       pdflatex pdfcolparcolumns.dtx
%
% Installation:
%    TDS:tex/latex/oberdiek/pdfcolparcolumns.sty
%    TDS:doc/latex/oberdiek/pdfcolparcolumns.pdf
%    TDS:doc/latex/oberdiek/test/pdfcolparcolumns-test1.tex
%    TDS:source/latex/oberdiek/pdfcolparcolumns.dtx
%
%<*ignore>
\begingroup
  \catcode123=1 %
  \catcode125=2 %
  \def\x{LaTeX2e}%
\expandafter\endgroup
\ifcase 0\ifx\install y1\fi\expandafter
         \ifx\csname processbatchFile\endcsname\relax\else1\fi
         \ifx\fmtname\x\else 1\fi\relax
\else\csname fi\endcsname
%</ignore>
%<*install>
\input docstrip.tex
\Msg{************************************************************************}
\Msg{* Installation}
\Msg{* Package: pdfcolparcolumns 2016/05/16 v1.4 Color stacks for parcolumns (HO)}
\Msg{************************************************************************}

\keepsilent
\askforoverwritefalse

\let\MetaPrefix\relax
\preamble

This is a generated file.

Project: pdfcolparcolumns
Version: 2016/05/16 v1.4

Copyright (C) 2007, 2008, 2010 by
   Heiko Oberdiek <heiko.oberdiek at googlemail.com>

This work may be distributed and/or modified under the
conditions of the LaTeX Project Public License, either
version 1.3c of this license or (at your option) any later
version. This version of this license is in
   http://www.latex-project.org/lppl/lppl-1-3c.txt
and the latest version of this license is in
   http://www.latex-project.org/lppl.txt
and version 1.3 or later is part of all distributions of
LaTeX version 2005/12/01 or later.

This work has the LPPL maintenance status "maintained".

This Current Maintainer of this work is Heiko Oberdiek.

This work consists of the main source file pdfcolparcolumns.dtx
and the derived files
   pdfcolparcolumns.sty, pdfcolparcolumns.pdf, pdfcolparcolumns.ins,
   pdfcolparcolumns.drv, pdfcolparcolumns-test1.tex.

\endpreamble
\let\MetaPrefix\DoubleperCent

\generate{%
  \file{pdfcolparcolumns.ins}{\from{pdfcolparcolumns.dtx}{install}}%
  \file{pdfcolparcolumns.drv}{\from{pdfcolparcolumns.dtx}{driver}}%
  \usedir{tex/latex/oberdiek}%
  \file{pdfcolparcolumns.sty}{\from{pdfcolparcolumns.dtx}{package}}%
  \usedir{doc/latex/oberdiek/test}%
  \file{pdfcolparcolumns-test1.tex}{\from{pdfcolparcolumns.dtx}{test1}}%
  \nopreamble
  \nopostamble
  \usedir{source/latex/oberdiek/catalogue}%
  \file{pdfcolparcolumns.xml}{\from{pdfcolparcolumns.dtx}{catalogue}}%
}

\catcode32=13\relax% active space
\let =\space%
\Msg{************************************************************************}
\Msg{*}
\Msg{* To finish the installation you have to move the following}
\Msg{* file into a directory searched by TeX:}
\Msg{*}
\Msg{*     pdfcolparcolumns.sty}
\Msg{*}
\Msg{* To produce the documentation run the file `pdfcolparcolumns.drv'}
\Msg{* through LaTeX.}
\Msg{*}
\Msg{* Happy TeXing!}
\Msg{*}
\Msg{************************************************************************}

\endbatchfile
%</install>
%<*ignore>
\fi
%</ignore>
%<*driver>
\NeedsTeXFormat{LaTeX2e}
\ProvidesFile{pdfcolparcolumns.drv}%
  [2016/05/16 v1.4 Color stacks for parcolumns (HO)]%
\documentclass{ltxdoc}
\usepackage{holtxdoc}[2011/11/22]
\begin{document}
  \DocInput{pdfcolparcolumns.dtx}%
\end{document}
%</driver>
% \fi
%
%
% \CharacterTable
%  {Upper-case    \A\B\C\D\E\F\G\H\I\J\K\L\M\N\O\P\Q\R\S\T\U\V\W\X\Y\Z
%   Lower-case    \a\b\c\d\e\f\g\h\i\j\k\l\m\n\o\p\q\r\s\t\u\v\w\x\y\z
%   Digits        \0\1\2\3\4\5\6\7\8\9
%   Exclamation   \!     Double quote  \"     Hash (number) \#
%   Dollar        \$     Percent       \%     Ampersand     \&
%   Acute accent  \'     Left paren    \(     Right paren   \)
%   Asterisk      \*     Plus          \+     Comma         \,
%   Minus         \-     Point         \.     Solidus       \/
%   Colon         \:     Semicolon     \;     Less than     \<
%   Equals        \=     Greater than  \>     Question mark \?
%   Commercial at \@     Left bracket  \[     Backslash     \\
%   Right bracket \]     Circumflex    \^     Underscore    \_
%   Grave accent  \`     Left brace    \{     Vertical bar  \|
%   Right brace   \}     Tilde         \~}
%
% \GetFileInfo{pdfcolparcolumns.drv}
%
% \title{The \xpackage{pdfcolparcolumns} package}
% \date{2016/05/16 v1.4}
% \author{Heiko Oberdiek\thanks
% {Please report any issues at https://github.com/ho-tex/oberdiek/issues}\\
% \xemail{heiko.oberdiek at googlemail.com}}
%
% \maketitle
%
% \begin{abstract}
% Since version 1.40 \pdfTeX\ supports several color stacks.
% This package uses them to fix color problems in
% package \xpackage{parcolumns}.
% \end{abstract}
%
% \tableofcontents
%
% \section{Usage}
%
% \begin{quote}
% |\usepackage{pdfcolparcolumns}|
% \end{quote}
% The package \xpackage{pdfcolparcolumns} loads package \xpackage{parcolums}
% \cite{parcolumns}. If color stacks are available then the
% macros of \xpackage{parcolumns} are patched to add support
% for color stacks.
%
% \subsection{Option \xoption{rulebetweencolor}}
%
% Package \xpackage{pdfcolparcolumns} also fixes the color for the
% rule between columns (if \xoption{rulebetween} is set).
% Default color is \cs{normalcolor}. But this can be changed by using
% option \xoption{rulebetweencolor}. It takes a color specification
% as value. If the value is empty, then the default (\cs{normalcolor})
% is used.
% Examples:
% \begin{quote}
%   |rulebetweencolor=blue|,\\
%   |rulebetweencolor={red}|,\\
%   |rulebetweencolor={}|, \textit{\% \cs{normalcolor} is used}\\
%   |rulebetweencolor=[rgb]{1,0,.5}| \textit{\% see below}
% \end{quote}
% If used inside the optional argument of environment \textsf{parcolumns}
% and the value contains an optional argument, then whole value
% must be put in curly braces:
%\begin{quote}
%\begin{verbatim}
%\begin{parcolumns}[
%  rulebetween,
%  rulebetweencolor={[rgb]{1,0,.5}},
%]{2}
%  ...
%\end{parcolumns}
%\end{verbatim}
%\end{quote}
% This option \xoption{rulebetweencolor} can also be set using
% \cs{setkeys}:
%\begin{quote}
%\begin{verbatim}
%\setkeys{parcolumns}{rulebetweencolor=green}
%\end{verbatim}
%\end{quote}
%
% \subsection{Future}
%
% Currently package \xpackage{parcolumns} does not seem to be
% maintained. Nevertheless if there will be a new version that
% adds support for color stacks, then this package may become
% obsolete.
%
% \StopEventually{
% }
%
% \section{Implementation}
%
% \subsection{Identification}
%
%    \begin{macrocode}
%<*package>
\NeedsTeXFormat{LaTeX2e}
\ProvidesPackage{pdfcolparcolumns}%
  [2016/05/16 v1.4 Color stacks for parcolumns (HO)]%
%    \end{macrocode}
%
% \subsection{Load packages}
%
% \subsubsection{Package \xpackage{parcolumns}}
%
%    Currently package \xpackage{parcolumns} does not define options.
%    Thus it is just a precaution that the options of
%    package \xpackage{pdfcolparcolumns} are passed to
%    package \xpackage{parcolumns}.
%    \begin{macrocode}
\DeclareOption*{%
  \PassoptionsToPackage{\CurrentOption}{parcolumns}%
}
\ProcessOptions\relax
\RequirePackage{parcolumns}[2004/11/25]
%    \end{macrocode}
%
% \subsubsection{Package \xpackage{pdfcol}}
%
%    \begin{macrocode}
\RequirePackage{pdfcol}[2007/09/09]
\ifpdfcolAvailable
\else
  \PackageInfo{pdfcolparcolumns}{%
    Loading aborted, because color stacks are not available%
  }%
  \expandafter\endinput
\fi
%    \end{macrocode}
%
% \subsubsection{Package \xpackage{infwarerr}}
%
%    \begin{macrocode}
\RequirePackage{infwarerr}[2007/09/09]
%    \end{macrocode}
%
% \subsection{Color stack macros}
%
%    \begin{macro}{\pcpc@MaxStack}
%    Macro \cs{pcpc@MaxStack} holds the highest number of
%    allocated stacks.
%    \begin{macrocode}
\global\chardef\pcpc@MaxStack=\z@
%    \end{macrocode}
%    \end{macro}
%    \begin{macro}{\pcpc@InitStacks}
%    Macro \cs{pcpc@InitStacks} takes the number of columns
%    as argument and ensures that there are enough color
%    stacks for all columns.
%    \begin{macrocode}
\def\pcpc@InitStacks#1{%
  \ifnum#1>\pcpc@MaxStack
    \begingroup
      \count@\pcpc@MaxStack
      \loop
        \advance\count@\@ne
        \pdfcolInitStack{pcpc@\the\count@}%
      \ifnum#1>\count@
      \repeat
      \global\chardef\pcpc@MaxStack=\count@
    \endgroup
  \fi
}
%    \end{macrocode}
%    \end{macro}
%
%    \begin{macro}{\pcpc@SwitchStack}
%    \begin{macrocode}
\def\pcpc@SwitchStack#1{%
  \pdfcolSwitchStack{pcpc@\number#1}%
}
%    \end{macrocode}
%    \end{macro}
%
%    \begin{macro}{\pcpc@SetCurrent}
%    \begin{macrocode}
\def\pcpc@SetCurrent#1{%
  \pdfcolSetCurrent{pcpc@\number#1}%
}
%    \end{macrocode}
%    \end{macro}
%
% \subsection{Patches}
%
%     Now the color stack macros are patched into the macros
%     of package \xpackage{parcolumns}.
%
% \subsubsection{Init stacks}
%
%    \cs{pcpc@InitStacks} should go into the definition of
%    environment |parcolumns|. \cs{pc@alloccolumns} is executed
%    there and nowhere else, thus we hook into it.
%    \begin{macrocode}
\g@addto@macro\pc@alloccolumns{%
  \pcpc@InitStacks\pc@columncount
}
%    \end{macrocode}
%
% \subsubsection{Switch stack}
%
%    \cs{pcpc@SwitchStack} should be called by marco \cs{colchunk@}.
%    However it is easier to patch \cs{pc@setcolumnwidth} that
%    is executed in \cs{colchunk@} only.
%    \begin{macrocode}
\g@addto@macro\pc@setcolumnwidth{%
  \pcpc@SwitchStack\pc@columnctr
}
%    \end{macrocode}
%
% \subsubsection{Set current stack color}
%
%    \cs{pcpc@SetCurrent} is set at the begin of each line.
%    It must be inserted into \cs{pc@placeboxes}. Unhappily
%    there is no easy way. Therefore we check and
%    redefine \cs{pc@placeboxes}.
%    \begin{macrocode}
\begingroup
  \def\x{%
    \global\let\@tempa\relax
    \count@\z@
    \hb@xt@\linewidth{%
      \vfuzz30ex %
      \vbadness\@M
      \splittopskip\z@skip
      \loop
      \ifnum\count@<\pc@columncount
        \advance\count@\@ne
        \expandafter\ifvoid\csname pc@column@\number\count@\endcsname
          \hskip\csname pc@column@width@\number\count@\endcsname
        \else
          \expandafter\setbox\expandafter\@tempboxa\expandafter
          \vsplit\csname pc@column@\number\count@\endcsname
              to \dp\strutbox
          \vbox{%
            \unvbox\@tempboxa
          }%
        \fi
        \expandafter\ifvoid\csname pc@column@\number\count@\endcsname
        \else
          \global\let\@tempa\pc@placeboxes
        \fi
        \ifnum\count@<\pc@columncount
          \strut
          \hfill
          \ifpc@rulebetween
            \vrule
            \hfill
          \fi
        \fi
      \repeat
    }%
    \@tempa
  }%
  \ifx\x\pc@placeboxes
  \else
    \@PackageWarningNoLine{pdfcolparcolumns}{%
      Command \string\pc@placeboxes\space has changed.\MessageBreak
      Supported versions of package `parcolumns':\MessageBreak
      \space\space 2004/08/05.\MessageBreak
      The redefinition of \string\pc@placeboxes\space may not%
      \MessageBreak
      behave correctly depending on the changes%
    }%
  \fi
\endgroup
%    \end{macrocode}
%    \begin{macro}{\pc@placeboxes}
%    \begin{macrocode}
\renewcommand*{\pc@placeboxes}{%
  \global\let\@tempa\relax
  \count@\z@
  \hb@xt@\linewidth{%
    \vfuzz30ex %
    \vbadness\@M
    \splittopskip\z@skip
    \loop
    \ifnum\count@<\pc@columncount
      \advance\count@\@ne
      \expandafter\ifvoid\csname pc@column@\number\count@\endcsname
        \hskip\csname pc@column@width@\number\count@\endcsname
      \else
        \expandafter\setbox\expandafter\@tempboxa\expandafter
        \vsplit\csname pc@column@\number\count@\endcsname
            to \dp\strutbox
        \vbox{%
          \pcpc@SetCurrent\count@
          \unvbox\@tempboxa
        }%
      \fi
      \expandafter\ifvoid\csname pc@column@\number\count@\endcsname
      \else
        \global\let\@tempa\pc@placeboxes
      \fi
      \ifnum\count@<\pc@columncount
        \strut
        \hfill
        \ifpc@rulebetween
          \begingroup
            \pcpc@RuleBetweenColor
            \vrule
          \endgroup
          \hfill
        \fi
      \fi
    \repeat
  }%
  \@tempa
}
%    \end{macrocode}
%    \end{macro}
%    \begin{macro}{\pcpc@RuleBetweenColorDefault}
%    \begin{macrocode}
\def\pcpc@RuleBetweenColorDefault{%
  \normalcolor
}
%    \end{macrocode}
%    \end{macro}
%    \begin{macro}{\pcpc@RuleBetweenColor}
%    \begin{macrocode}
\let\pcpc@RuleBetweenColor\pcpc@RuleBetweenColorDefault
%    \end{macrocode}
%    \end{macro}
%    \begin{macrocode}
\define@key{parcolumns}{rulebetweencolor}{%
  \edef\pcpc@temp{#1}%
  \ifx\pcpc@temp\@empty
    \let\pcpc@RuleBetweenColor\pcpc@RuleBetweenColorDefault
  \else
    \edef\pcpc@temp{%
      \noexpand\@ifnextchar[{%
        \def\noexpand\pcpc@RuleBetweenColor{%
          \noexpand\color\pcpc@temp
        }%
        \noexpand\pcpc@GobbleNil
      }{%
        \def\noexpand\pcpc@RuleBetweenColor{%
          \noexpand\color{\pcpc@temp}%
        }%
        \noexpand\pcpc@GobbleNil
      }%
      \pcpc@temp\noexpand\@nil
    }%
    \pcpc@temp
  \fi
}
%    \end{macrocode}
%    \begin{macro}{\pcpc@GobbleNil}
%    \begin{macrocode}
\long\def\pcpc@GobbleNil#1\@nil{}
%    \end{macrocode}
%    \end{macro}
%
%    \begin{macrocode}
%</package>
%    \end{macrocode}
%
% \section{Test}
%
%    The test file is a modified version of the file that
%    Donald Goodman has posted in \xnewsgroup{comp.text.tex}: ^^A
%    \URL{``\link{Re: \xpackage{xcolor} glitches}''}^^A
%    {http://groups.google.com/group/comp.text.tex/msg/8eda74ed292012bb}
%    \begin{macrocode}
%<*test1>
\NeedsTeXFormat{LaTeX2e}
\AtEndDocument{%
  \typeout{}%
  \typeout{**************************************}%
  \typeout{*** \space Check the PDF file manually! \space ***}%
  \typeout{**************************************}%
  \typeout{}%
}
\documentclass{article}
\usepackage{xcolor}
\usepackage{pdfcolparcolumns}

\newcommand{\instruct}[1]{%
  \noindent
  \footnotesize
  \textcolor{red}{#1}%
}

\begin{document}
  \begin{parcolumns}[colwidths={1=2.3in,2=2.3in},sloppy]{2}%
    \colchunk[1]{%
      \instruct{Et non dicitur versus} %
      Fidelium anim\ae\ %
      \instruct{%
        sed immediate subiungitur antiphona finalis %
        beat\ae\ Mari\ae\ Virginis%
      } %
      100.%
    }%
    \colchunk[2]{%
      \instruct{%
        And the verse %
        \textcolor{black}{May the souls of the faithful} %
        is not said, but the final antiphon of the %
        Blessed Virgin Mary, %
        \textcolor{black}{100,} %
        is immediately joined.%
      }%
    }%
  \end{parcolumns}%
\end{document}
%</test1>
%    \end{macrocode}
%
% \section{Installation}
%
% \subsection{Download}
%
% \paragraph{Package.} This package is available on
% CTAN\footnote{\url{http://ctan.org/pkg/pdfcolparcolumns}}:
% \begin{description}
% \item[\CTAN{macros/latex/contrib/oberdiek/pdfcolparcolumns.dtx}] The source file.
% \item[\CTAN{macros/latex/contrib/oberdiek/pdfcolparcolumns.pdf}] Documentation.
% \end{description}
%
%
% \paragraph{Bundle.} All the packages of the bundle `oberdiek'
% are also available in a TDS compliant ZIP archive. There
% the packages are already unpacked and the documentation files
% are generated. The files and directories obey the TDS standard.
% \begin{description}
% \item[\CTAN{install/macros/latex/contrib/oberdiek.tds.zip}]
% \end{description}
% \emph{TDS} refers to the standard ``A Directory Structure
% for \TeX\ Files'' (\CTAN{tds/tds.pdf}). Directories
% with \xfile{texmf} in their name are usually organized this way.
%
% \subsection{Bundle installation}
%
% \paragraph{Unpacking.} Unpack the \xfile{oberdiek.tds.zip} in the
% TDS tree (also known as \xfile{texmf} tree) of your choice.
% Example (linux):
% \begin{quote}
%   |unzip oberdiek.tds.zip -d ~/texmf|
% \end{quote}
%
% \paragraph{Script installation.}
% Check the directory \xfile{TDS:scripts/oberdiek/} for
% scripts that need further installation steps.
% Package \xpackage{attachfile2} comes with the Perl script
% \xfile{pdfatfi.pl} that should be installed in such a way
% that it can be called as \texttt{pdfatfi}.
% Example (linux):
% \begin{quote}
%   |chmod +x scripts/oberdiek/pdfatfi.pl|\\
%   |cp scripts/oberdiek/pdfatfi.pl /usr/local/bin/|
% \end{quote}
%
% \subsection{Package installation}
%
% \paragraph{Unpacking.} The \xfile{.dtx} file is a self-extracting
% \docstrip\ archive. The files are extracted by running the
% \xfile{.dtx} through \plainTeX:
% \begin{quote}
%   \verb|tex pdfcolparcolumns.dtx|
% \end{quote}
%
% \paragraph{TDS.} Now the different files must be moved into
% the different directories in your installation TDS tree
% (also known as \xfile{texmf} tree):
% \begin{quote}
% \def\t{^^A
% \begin{tabular}{@{}>{\ttfamily}l@{ $\rightarrow$ }>{\ttfamily}l@{}}
%   pdfcolparcolumns.sty & tex/latex/oberdiek/pdfcolparcolumns.sty\\
%   pdfcolparcolumns.pdf & doc/latex/oberdiek/pdfcolparcolumns.pdf\\
%   test/pdfcolparcolumns-test1.tex & doc/latex/oberdiek/test/pdfcolparcolumns-test1.tex\\
%   pdfcolparcolumns.dtx & source/latex/oberdiek/pdfcolparcolumns.dtx\\
% \end{tabular}^^A
% }^^A
% \sbox0{\t}^^A
% \ifdim\wd0>\linewidth
%   \begingroup
%     \advance\linewidth by\leftmargin
%     \advance\linewidth by\rightmargin
%   \edef\x{\endgroup
%     \def\noexpand\lw{\the\linewidth}^^A
%   }\x
%   \def\lwbox{^^A
%     \leavevmode
%     \hbox to \linewidth{^^A
%       \kern-\leftmargin\relax
%       \hss
%       \usebox0
%       \hss
%       \kern-\rightmargin\relax
%     }^^A
%   }^^A
%   \ifdim\wd0>\lw
%     \sbox0{\small\t}^^A
%     \ifdim\wd0>\linewidth
%       \ifdim\wd0>\lw
%         \sbox0{\footnotesize\t}^^A
%         \ifdim\wd0>\linewidth
%           \ifdim\wd0>\lw
%             \sbox0{\scriptsize\t}^^A
%             \ifdim\wd0>\linewidth
%               \ifdim\wd0>\lw
%                 \sbox0{\tiny\t}^^A
%                 \ifdim\wd0>\linewidth
%                   \lwbox
%                 \else
%                   \usebox0
%                 \fi
%               \else
%                 \lwbox
%               \fi
%             \else
%               \usebox0
%             \fi
%           \else
%             \lwbox
%           \fi
%         \else
%           \usebox0
%         \fi
%       \else
%         \lwbox
%       \fi
%     \else
%       \usebox0
%     \fi
%   \else
%     \lwbox
%   \fi
% \else
%   \usebox0
% \fi
% \end{quote}
% If you have a \xfile{docstrip.cfg} that configures and enables \docstrip's
% TDS installing feature, then some files can already be in the right
% place, see the documentation of \docstrip.
%
% \subsection{Refresh file name databases}
%
% If your \TeX~distribution
% (\teTeX, \mikTeX, \dots) relies on file name databases, you must refresh
% these. For example, \teTeX\ users run \verb|texhash| or
% \verb|mktexlsr|.
%
% \subsection{Some details for the interested}
%
% \paragraph{Attached source.}
%
% The PDF documentation on CTAN also includes the
% \xfile{.dtx} source file. It can be extracted by
% AcrobatReader 6 or higher. Another option is \textsf{pdftk},
% e.g. unpack the file into the current directory:
% \begin{quote}
%   \verb|pdftk pdfcolparcolumns.pdf unpack_files output .|
% \end{quote}
%
% \paragraph{Unpacking with \LaTeX.}
% The \xfile{.dtx} chooses its action depending on the format:
% \begin{description}
% \item[\plainTeX:] Run \docstrip\ and extract the files.
% \item[\LaTeX:] Generate the documentation.
% \end{description}
% If you insist on using \LaTeX\ for \docstrip\ (really,
% \docstrip\ does not need \LaTeX), then inform the autodetect routine
% about your intention:
% \begin{quote}
%   \verb|latex \let\install=y% \iffalse meta-comment
%
% File: pdfcolparcolumns.dtx
% Version: 2016/05/16 v1.4
% Info: Color stacks for parcolumns
%
% Copyright (C) 2007, 2008, 2010 by
%    Heiko Oberdiek <heiko.oberdiek at googlemail.com>
%    2016
%    https://github.com/ho-tex/oberdiek/issues
%
% This work may be distributed and/or modified under the
% conditions of the LaTeX Project Public License, either
% version 1.3c of this license or (at your option) any later
% version. This version of this license is in
%    http://www.latex-project.org/lppl/lppl-1-3c.txt
% and the latest version of this license is in
%    http://www.latex-project.org/lppl.txt
% and version 1.3 or later is part of all distributions of
% LaTeX version 2005/12/01 or later.
%
% This work has the LPPL maintenance status "maintained".
%
% This Current Maintainer of this work is Heiko Oberdiek.
%
% This work consists of the main source file pdfcolparcolumns.dtx
% and the derived files
%    pdfcolparcolumns.sty, pdfcolparcolumns.pdf, pdfcolparcolumns.ins,
%    pdfcolparcolumns.drv, pdfcolparcolumns-test1.tex.
%
% Distribution:
%    CTAN:macros/latex/contrib/oberdiek/pdfcolparcolumns.dtx
%    CTAN:macros/latex/contrib/oberdiek/pdfcolparcolumns.pdf
%
% Unpacking:
%    (a) If pdfcolparcolumns.ins is present:
%           tex pdfcolparcolumns.ins
%    (b) Without pdfcolparcolumns.ins:
%           tex pdfcolparcolumns.dtx
%    (c) If you insist on using LaTeX
%           latex \let\install=y\input{pdfcolparcolumns.dtx}
%        (quote the arguments according to the demands of your shell)
%
% Documentation:
%    (a) If pdfcolparcolumns.drv is present:
%           latex pdfcolparcolumns.drv
%    (b) Without pdfcolparcolumns.drv:
%           latex pdfcolparcolumns.dtx; ...
%    The class ltxdoc loads the configuration file ltxdoc.cfg
%    if available. Here you can specify further options, e.g.
%    use A4 as paper format:
%       \PassOptionsToClass{a4paper}{article}
%
%    Programm calls to get the documentation (example):
%       pdflatex pdfcolparcolumns.dtx
%       makeindex -s gind.ist pdfcolparcolumns.idx
%       pdflatex pdfcolparcolumns.dtx
%       makeindex -s gind.ist pdfcolparcolumns.idx
%       pdflatex pdfcolparcolumns.dtx
%
% Installation:
%    TDS:tex/latex/oberdiek/pdfcolparcolumns.sty
%    TDS:doc/latex/oberdiek/pdfcolparcolumns.pdf
%    TDS:doc/latex/oberdiek/test/pdfcolparcolumns-test1.tex
%    TDS:source/latex/oberdiek/pdfcolparcolumns.dtx
%
%<*ignore>
\begingroup
  \catcode123=1 %
  \catcode125=2 %
  \def\x{LaTeX2e}%
\expandafter\endgroup
\ifcase 0\ifx\install y1\fi\expandafter
         \ifx\csname processbatchFile\endcsname\relax\else1\fi
         \ifx\fmtname\x\else 1\fi\relax
\else\csname fi\endcsname
%</ignore>
%<*install>
\input docstrip.tex
\Msg{************************************************************************}
\Msg{* Installation}
\Msg{* Package: pdfcolparcolumns 2016/05/16 v1.4 Color stacks for parcolumns (HO)}
\Msg{************************************************************************}

\keepsilent
\askforoverwritefalse

\let\MetaPrefix\relax
\preamble

This is a generated file.

Project: pdfcolparcolumns
Version: 2016/05/16 v1.4

Copyright (C) 2007, 2008, 2010 by
   Heiko Oberdiek <heiko.oberdiek at googlemail.com>

This work may be distributed and/or modified under the
conditions of the LaTeX Project Public License, either
version 1.3c of this license or (at your option) any later
version. This version of this license is in
   http://www.latex-project.org/lppl/lppl-1-3c.txt
and the latest version of this license is in
   http://www.latex-project.org/lppl.txt
and version 1.3 or later is part of all distributions of
LaTeX version 2005/12/01 or later.

This work has the LPPL maintenance status "maintained".

This Current Maintainer of this work is Heiko Oberdiek.

This work consists of the main source file pdfcolparcolumns.dtx
and the derived files
   pdfcolparcolumns.sty, pdfcolparcolumns.pdf, pdfcolparcolumns.ins,
   pdfcolparcolumns.drv, pdfcolparcolumns-test1.tex.

\endpreamble
\let\MetaPrefix\DoubleperCent

\generate{%
  \file{pdfcolparcolumns.ins}{\from{pdfcolparcolumns.dtx}{install}}%
  \file{pdfcolparcolumns.drv}{\from{pdfcolparcolumns.dtx}{driver}}%
  \usedir{tex/latex/oberdiek}%
  \file{pdfcolparcolumns.sty}{\from{pdfcolparcolumns.dtx}{package}}%
  \usedir{doc/latex/oberdiek/test}%
  \file{pdfcolparcolumns-test1.tex}{\from{pdfcolparcolumns.dtx}{test1}}%
  \nopreamble
  \nopostamble
  \usedir{source/latex/oberdiek/catalogue}%
  \file{pdfcolparcolumns.xml}{\from{pdfcolparcolumns.dtx}{catalogue}}%
}

\catcode32=13\relax% active space
\let =\space%
\Msg{************************************************************************}
\Msg{*}
\Msg{* To finish the installation you have to move the following}
\Msg{* file into a directory searched by TeX:}
\Msg{*}
\Msg{*     pdfcolparcolumns.sty}
\Msg{*}
\Msg{* To produce the documentation run the file `pdfcolparcolumns.drv'}
\Msg{* through LaTeX.}
\Msg{*}
\Msg{* Happy TeXing!}
\Msg{*}
\Msg{************************************************************************}

\endbatchfile
%</install>
%<*ignore>
\fi
%</ignore>
%<*driver>
\NeedsTeXFormat{LaTeX2e}
\ProvidesFile{pdfcolparcolumns.drv}%
  [2016/05/16 v1.4 Color stacks for parcolumns (HO)]%
\documentclass{ltxdoc}
\usepackage{holtxdoc}[2011/11/22]
\begin{document}
  \DocInput{pdfcolparcolumns.dtx}%
\end{document}
%</driver>
% \fi
%
%
% \CharacterTable
%  {Upper-case    \A\B\C\D\E\F\G\H\I\J\K\L\M\N\O\P\Q\R\S\T\U\V\W\X\Y\Z
%   Lower-case    \a\b\c\d\e\f\g\h\i\j\k\l\m\n\o\p\q\r\s\t\u\v\w\x\y\z
%   Digits        \0\1\2\3\4\5\6\7\8\9
%   Exclamation   \!     Double quote  \"     Hash (number) \#
%   Dollar        \$     Percent       \%     Ampersand     \&
%   Acute accent  \'     Left paren    \(     Right paren   \)
%   Asterisk      \*     Plus          \+     Comma         \,
%   Minus         \-     Point         \.     Solidus       \/
%   Colon         \:     Semicolon     \;     Less than     \<
%   Equals        \=     Greater than  \>     Question mark \?
%   Commercial at \@     Left bracket  \[     Backslash     \\
%   Right bracket \]     Circumflex    \^     Underscore    \_
%   Grave accent  \`     Left brace    \{     Vertical bar  \|
%   Right brace   \}     Tilde         \~}
%
% \GetFileInfo{pdfcolparcolumns.drv}
%
% \title{The \xpackage{pdfcolparcolumns} package}
% \date{2016/05/16 v1.4}
% \author{Heiko Oberdiek\thanks
% {Please report any issues at https://github.com/ho-tex/oberdiek/issues}\\
% \xemail{heiko.oberdiek at googlemail.com}}
%
% \maketitle
%
% \begin{abstract}
% Since version 1.40 \pdfTeX\ supports several color stacks.
% This package uses them to fix color problems in
% package \xpackage{parcolumns}.
% \end{abstract}
%
% \tableofcontents
%
% \section{Usage}
%
% \begin{quote}
% |\usepackage{pdfcolparcolumns}|
% \end{quote}
% The package \xpackage{pdfcolparcolumns} loads package \xpackage{parcolums}
% \cite{parcolumns}. If color stacks are available then the
% macros of \xpackage{parcolumns} are patched to add support
% for color stacks.
%
% \subsection{Option \xoption{rulebetweencolor}}
%
% Package \xpackage{pdfcolparcolumns} also fixes the color for the
% rule between columns (if \xoption{rulebetween} is set).
% Default color is \cs{normalcolor}. But this can be changed by using
% option \xoption{rulebetweencolor}. It takes a color specification
% as value. If the value is empty, then the default (\cs{normalcolor})
% is used.
% Examples:
% \begin{quote}
%   |rulebetweencolor=blue|,\\
%   |rulebetweencolor={red}|,\\
%   |rulebetweencolor={}|, \textit{\% \cs{normalcolor} is used}\\
%   |rulebetweencolor=[rgb]{1,0,.5}| \textit{\% see below}
% \end{quote}
% If used inside the optional argument of environment \textsf{parcolumns}
% and the value contains an optional argument, then whole value
% must be put in curly braces:
%\begin{quote}
%\begin{verbatim}
%\begin{parcolumns}[
%  rulebetween,
%  rulebetweencolor={[rgb]{1,0,.5}},
%]{2}
%  ...
%\end{parcolumns}
%\end{verbatim}
%\end{quote}
% This option \xoption{rulebetweencolor} can also be set using
% \cs{setkeys}:
%\begin{quote}
%\begin{verbatim}
%\setkeys{parcolumns}{rulebetweencolor=green}
%\end{verbatim}
%\end{quote}
%
% \subsection{Future}
%
% Currently package \xpackage{parcolumns} does not seem to be
% maintained. Nevertheless if there will be a new version that
% adds support for color stacks, then this package may become
% obsolete.
%
% \StopEventually{
% }
%
% \section{Implementation}
%
% \subsection{Identification}
%
%    \begin{macrocode}
%<*package>
\NeedsTeXFormat{LaTeX2e}
\ProvidesPackage{pdfcolparcolumns}%
  [2016/05/16 v1.4 Color stacks for parcolumns (HO)]%
%    \end{macrocode}
%
% \subsection{Load packages}
%
% \subsubsection{Package \xpackage{parcolumns}}
%
%    Currently package \xpackage{parcolumns} does not define options.
%    Thus it is just a precaution that the options of
%    package \xpackage{pdfcolparcolumns} are passed to
%    package \xpackage{parcolumns}.
%    \begin{macrocode}
\DeclareOption*{%
  \PassoptionsToPackage{\CurrentOption}{parcolumns}%
}
\ProcessOptions\relax
\RequirePackage{parcolumns}[2004/11/25]
%    \end{macrocode}
%
% \subsubsection{Package \xpackage{pdfcol}}
%
%    \begin{macrocode}
\RequirePackage{pdfcol}[2007/09/09]
\ifpdfcolAvailable
\else
  \PackageInfo{pdfcolparcolumns}{%
    Loading aborted, because color stacks are not available%
  }%
  \expandafter\endinput
\fi
%    \end{macrocode}
%
% \subsubsection{Package \xpackage{infwarerr}}
%
%    \begin{macrocode}
\RequirePackage{infwarerr}[2007/09/09]
%    \end{macrocode}
%
% \subsection{Color stack macros}
%
%    \begin{macro}{\pcpc@MaxStack}
%    Macro \cs{pcpc@MaxStack} holds the highest number of
%    allocated stacks.
%    \begin{macrocode}
\global\chardef\pcpc@MaxStack=\z@
%    \end{macrocode}
%    \end{macro}
%    \begin{macro}{\pcpc@InitStacks}
%    Macro \cs{pcpc@InitStacks} takes the number of columns
%    as argument and ensures that there are enough color
%    stacks for all columns.
%    \begin{macrocode}
\def\pcpc@InitStacks#1{%
  \ifnum#1>\pcpc@MaxStack
    \begingroup
      \count@\pcpc@MaxStack
      \loop
        \advance\count@\@ne
        \pdfcolInitStack{pcpc@\the\count@}%
      \ifnum#1>\count@
      \repeat
      \global\chardef\pcpc@MaxStack=\count@
    \endgroup
  \fi
}
%    \end{macrocode}
%    \end{macro}
%
%    \begin{macro}{\pcpc@SwitchStack}
%    \begin{macrocode}
\def\pcpc@SwitchStack#1{%
  \pdfcolSwitchStack{pcpc@\number#1}%
}
%    \end{macrocode}
%    \end{macro}
%
%    \begin{macro}{\pcpc@SetCurrent}
%    \begin{macrocode}
\def\pcpc@SetCurrent#1{%
  \pdfcolSetCurrent{pcpc@\number#1}%
}
%    \end{macrocode}
%    \end{macro}
%
% \subsection{Patches}
%
%     Now the color stack macros are patched into the macros
%     of package \xpackage{parcolumns}.
%
% \subsubsection{Init stacks}
%
%    \cs{pcpc@InitStacks} should go into the definition of
%    environment |parcolumns|. \cs{pc@alloccolumns} is executed
%    there and nowhere else, thus we hook into it.
%    \begin{macrocode}
\g@addto@macro\pc@alloccolumns{%
  \pcpc@InitStacks\pc@columncount
}
%    \end{macrocode}
%
% \subsubsection{Switch stack}
%
%    \cs{pcpc@SwitchStack} should be called by marco \cs{colchunk@}.
%    However it is easier to patch \cs{pc@setcolumnwidth} that
%    is executed in \cs{colchunk@} only.
%    \begin{macrocode}
\g@addto@macro\pc@setcolumnwidth{%
  \pcpc@SwitchStack\pc@columnctr
}
%    \end{macrocode}
%
% \subsubsection{Set current stack color}
%
%    \cs{pcpc@SetCurrent} is set at the begin of each line.
%    It must be inserted into \cs{pc@placeboxes}. Unhappily
%    there is no easy way. Therefore we check and
%    redefine \cs{pc@placeboxes}.
%    \begin{macrocode}
\begingroup
  \def\x{%
    \global\let\@tempa\relax
    \count@\z@
    \hb@xt@\linewidth{%
      \vfuzz30ex %
      \vbadness\@M
      \splittopskip\z@skip
      \loop
      \ifnum\count@<\pc@columncount
        \advance\count@\@ne
        \expandafter\ifvoid\csname pc@column@\number\count@\endcsname
          \hskip\csname pc@column@width@\number\count@\endcsname
        \else
          \expandafter\setbox\expandafter\@tempboxa\expandafter
          \vsplit\csname pc@column@\number\count@\endcsname
              to \dp\strutbox
          \vbox{%
            \unvbox\@tempboxa
          }%
        \fi
        \expandafter\ifvoid\csname pc@column@\number\count@\endcsname
        \else
          \global\let\@tempa\pc@placeboxes
        \fi
        \ifnum\count@<\pc@columncount
          \strut
          \hfill
          \ifpc@rulebetween
            \vrule
            \hfill
          \fi
        \fi
      \repeat
    }%
    \@tempa
  }%
  \ifx\x\pc@placeboxes
  \else
    \@PackageWarningNoLine{pdfcolparcolumns}{%
      Command \string\pc@placeboxes\space has changed.\MessageBreak
      Supported versions of package `parcolumns':\MessageBreak
      \space\space 2004/08/05.\MessageBreak
      The redefinition of \string\pc@placeboxes\space may not%
      \MessageBreak
      behave correctly depending on the changes%
    }%
  \fi
\endgroup
%    \end{macrocode}
%    \begin{macro}{\pc@placeboxes}
%    \begin{macrocode}
\renewcommand*{\pc@placeboxes}{%
  \global\let\@tempa\relax
  \count@\z@
  \hb@xt@\linewidth{%
    \vfuzz30ex %
    \vbadness\@M
    \splittopskip\z@skip
    \loop
    \ifnum\count@<\pc@columncount
      \advance\count@\@ne
      \expandafter\ifvoid\csname pc@column@\number\count@\endcsname
        \hskip\csname pc@column@width@\number\count@\endcsname
      \else
        \expandafter\setbox\expandafter\@tempboxa\expandafter
        \vsplit\csname pc@column@\number\count@\endcsname
            to \dp\strutbox
        \vbox{%
          \pcpc@SetCurrent\count@
          \unvbox\@tempboxa
        }%
      \fi
      \expandafter\ifvoid\csname pc@column@\number\count@\endcsname
      \else
        \global\let\@tempa\pc@placeboxes
      \fi
      \ifnum\count@<\pc@columncount
        \strut
        \hfill
        \ifpc@rulebetween
          \begingroup
            \pcpc@RuleBetweenColor
            \vrule
          \endgroup
          \hfill
        \fi
      \fi
    \repeat
  }%
  \@tempa
}
%    \end{macrocode}
%    \end{macro}
%    \begin{macro}{\pcpc@RuleBetweenColorDefault}
%    \begin{macrocode}
\def\pcpc@RuleBetweenColorDefault{%
  \normalcolor
}
%    \end{macrocode}
%    \end{macro}
%    \begin{macro}{\pcpc@RuleBetweenColor}
%    \begin{macrocode}
\let\pcpc@RuleBetweenColor\pcpc@RuleBetweenColorDefault
%    \end{macrocode}
%    \end{macro}
%    \begin{macrocode}
\define@key{parcolumns}{rulebetweencolor}{%
  \edef\pcpc@temp{#1}%
  \ifx\pcpc@temp\@empty
    \let\pcpc@RuleBetweenColor\pcpc@RuleBetweenColorDefault
  \else
    \edef\pcpc@temp{%
      \noexpand\@ifnextchar[{%
        \def\noexpand\pcpc@RuleBetweenColor{%
          \noexpand\color\pcpc@temp
        }%
        \noexpand\pcpc@GobbleNil
      }{%
        \def\noexpand\pcpc@RuleBetweenColor{%
          \noexpand\color{\pcpc@temp}%
        }%
        \noexpand\pcpc@GobbleNil
      }%
      \pcpc@temp\noexpand\@nil
    }%
    \pcpc@temp
  \fi
}
%    \end{macrocode}
%    \begin{macro}{\pcpc@GobbleNil}
%    \begin{macrocode}
\long\def\pcpc@GobbleNil#1\@nil{}
%    \end{macrocode}
%    \end{macro}
%
%    \begin{macrocode}
%</package>
%    \end{macrocode}
%
% \section{Test}
%
%    The test file is a modified version of the file that
%    Donald Goodman has posted in \xnewsgroup{comp.text.tex}: ^^A
%    \URL{``\link{Re: \xpackage{xcolor} glitches}''}^^A
%    {http://groups.google.com/group/comp.text.tex/msg/8eda74ed292012bb}
%    \begin{macrocode}
%<*test1>
\NeedsTeXFormat{LaTeX2e}
\AtEndDocument{%
  \typeout{}%
  \typeout{**************************************}%
  \typeout{*** \space Check the PDF file manually! \space ***}%
  \typeout{**************************************}%
  \typeout{}%
}
\documentclass{article}
\usepackage{xcolor}
\usepackage{pdfcolparcolumns}

\newcommand{\instruct}[1]{%
  \noindent
  \footnotesize
  \textcolor{red}{#1}%
}

\begin{document}
  \begin{parcolumns}[colwidths={1=2.3in,2=2.3in},sloppy]{2}%
    \colchunk[1]{%
      \instruct{Et non dicitur versus} %
      Fidelium anim\ae\ %
      \instruct{%
        sed immediate subiungitur antiphona finalis %
        beat\ae\ Mari\ae\ Virginis%
      } %
      100.%
    }%
    \colchunk[2]{%
      \instruct{%
        And the verse %
        \textcolor{black}{May the souls of the faithful} %
        is not said, but the final antiphon of the %
        Blessed Virgin Mary, %
        \textcolor{black}{100,} %
        is immediately joined.%
      }%
    }%
  \end{parcolumns}%
\end{document}
%</test1>
%    \end{macrocode}
%
% \section{Installation}
%
% \subsection{Download}
%
% \paragraph{Package.} This package is available on
% CTAN\footnote{\url{http://ctan.org/pkg/pdfcolparcolumns}}:
% \begin{description}
% \item[\CTAN{macros/latex/contrib/oberdiek/pdfcolparcolumns.dtx}] The source file.
% \item[\CTAN{macros/latex/contrib/oberdiek/pdfcolparcolumns.pdf}] Documentation.
% \end{description}
%
%
% \paragraph{Bundle.} All the packages of the bundle `oberdiek'
% are also available in a TDS compliant ZIP archive. There
% the packages are already unpacked and the documentation files
% are generated. The files and directories obey the TDS standard.
% \begin{description}
% \item[\CTAN{install/macros/latex/contrib/oberdiek.tds.zip}]
% \end{description}
% \emph{TDS} refers to the standard ``A Directory Structure
% for \TeX\ Files'' (\CTAN{tds/tds.pdf}). Directories
% with \xfile{texmf} in their name are usually organized this way.
%
% \subsection{Bundle installation}
%
% \paragraph{Unpacking.} Unpack the \xfile{oberdiek.tds.zip} in the
% TDS tree (also known as \xfile{texmf} tree) of your choice.
% Example (linux):
% \begin{quote}
%   |unzip oberdiek.tds.zip -d ~/texmf|
% \end{quote}
%
% \paragraph{Script installation.}
% Check the directory \xfile{TDS:scripts/oberdiek/} for
% scripts that need further installation steps.
% Package \xpackage{attachfile2} comes with the Perl script
% \xfile{pdfatfi.pl} that should be installed in such a way
% that it can be called as \texttt{pdfatfi}.
% Example (linux):
% \begin{quote}
%   |chmod +x scripts/oberdiek/pdfatfi.pl|\\
%   |cp scripts/oberdiek/pdfatfi.pl /usr/local/bin/|
% \end{quote}
%
% \subsection{Package installation}
%
% \paragraph{Unpacking.} The \xfile{.dtx} file is a self-extracting
% \docstrip\ archive. The files are extracted by running the
% \xfile{.dtx} through \plainTeX:
% \begin{quote}
%   \verb|tex pdfcolparcolumns.dtx|
% \end{quote}
%
% \paragraph{TDS.} Now the different files must be moved into
% the different directories in your installation TDS tree
% (also known as \xfile{texmf} tree):
% \begin{quote}
% \def\t{^^A
% \begin{tabular}{@{}>{\ttfamily}l@{ $\rightarrow$ }>{\ttfamily}l@{}}
%   pdfcolparcolumns.sty & tex/latex/oberdiek/pdfcolparcolumns.sty\\
%   pdfcolparcolumns.pdf & doc/latex/oberdiek/pdfcolparcolumns.pdf\\
%   test/pdfcolparcolumns-test1.tex & doc/latex/oberdiek/test/pdfcolparcolumns-test1.tex\\
%   pdfcolparcolumns.dtx & source/latex/oberdiek/pdfcolparcolumns.dtx\\
% \end{tabular}^^A
% }^^A
% \sbox0{\t}^^A
% \ifdim\wd0>\linewidth
%   \begingroup
%     \advance\linewidth by\leftmargin
%     \advance\linewidth by\rightmargin
%   \edef\x{\endgroup
%     \def\noexpand\lw{\the\linewidth}^^A
%   }\x
%   \def\lwbox{^^A
%     \leavevmode
%     \hbox to \linewidth{^^A
%       \kern-\leftmargin\relax
%       \hss
%       \usebox0
%       \hss
%       \kern-\rightmargin\relax
%     }^^A
%   }^^A
%   \ifdim\wd0>\lw
%     \sbox0{\small\t}^^A
%     \ifdim\wd0>\linewidth
%       \ifdim\wd0>\lw
%         \sbox0{\footnotesize\t}^^A
%         \ifdim\wd0>\linewidth
%           \ifdim\wd0>\lw
%             \sbox0{\scriptsize\t}^^A
%             \ifdim\wd0>\linewidth
%               \ifdim\wd0>\lw
%                 \sbox0{\tiny\t}^^A
%                 \ifdim\wd0>\linewidth
%                   \lwbox
%                 \else
%                   \usebox0
%                 \fi
%               \else
%                 \lwbox
%               \fi
%             \else
%               \usebox0
%             \fi
%           \else
%             \lwbox
%           \fi
%         \else
%           \usebox0
%         \fi
%       \else
%         \lwbox
%       \fi
%     \else
%       \usebox0
%     \fi
%   \else
%     \lwbox
%   \fi
% \else
%   \usebox0
% \fi
% \end{quote}
% If you have a \xfile{docstrip.cfg} that configures and enables \docstrip's
% TDS installing feature, then some files can already be in the right
% place, see the documentation of \docstrip.
%
% \subsection{Refresh file name databases}
%
% If your \TeX~distribution
% (\teTeX, \mikTeX, \dots) relies on file name databases, you must refresh
% these. For example, \teTeX\ users run \verb|texhash| or
% \verb|mktexlsr|.
%
% \subsection{Some details for the interested}
%
% \paragraph{Attached source.}
%
% The PDF documentation on CTAN also includes the
% \xfile{.dtx} source file. It can be extracted by
% AcrobatReader 6 or higher. Another option is \textsf{pdftk},
% e.g. unpack the file into the current directory:
% \begin{quote}
%   \verb|pdftk pdfcolparcolumns.pdf unpack_files output .|
% \end{quote}
%
% \paragraph{Unpacking with \LaTeX.}
% The \xfile{.dtx} chooses its action depending on the format:
% \begin{description}
% \item[\plainTeX:] Run \docstrip\ and extract the files.
% \item[\LaTeX:] Generate the documentation.
% \end{description}
% If you insist on using \LaTeX\ for \docstrip\ (really,
% \docstrip\ does not need \LaTeX), then inform the autodetect routine
% about your intention:
% \begin{quote}
%   \verb|latex \let\install=y\input{pdfcolparcolumns.dtx}|
% \end{quote}
% Do not forget to quote the argument according to the demands
% of your shell.
%
% \paragraph{Generating the documentation.}
% You can use both the \xfile{.dtx} or the \xfile{.drv} to generate
% the documentation. The process can be configured by the
% configuration file \xfile{ltxdoc.cfg}. For instance, put this
% line into this file, if you want to have A4 as paper format:
% \begin{quote}
%   \verb|\PassOptionsToClass{a4paper}{article}|
% \end{quote}
% An example follows how to generate the
% documentation with pdf\LaTeX:
% \begin{quote}
%\begin{verbatim}
%pdflatex pdfcolparcolumns.dtx
%makeindex -s gind.ist pdfcolparcolumns.idx
%pdflatex pdfcolparcolumns.dtx
%makeindex -s gind.ist pdfcolparcolumns.idx
%pdflatex pdfcolparcolumns.dtx
%\end{verbatim}
% \end{quote}
%
% \section{Catalogue}
%
% The following XML file can be used as source for the
% \href{http://mirror.ctan.org/help/Catalogue/catalogue.html}{\TeX\ Catalogue}.
% The elements \texttt{caption} and \texttt{description} are imported
% from the original XML file from the Catalogue.
% The name of the XML file in the Catalogue is \xfile{pdfcolparcolumns.xml}.
%    \begin{macrocode}
%<*catalogue>
<?xml version='1.0' encoding='us-ascii'?>
<!DOCTYPE entry SYSTEM 'catalogue.dtd'>
<entry datestamp='$Date$' modifier='$Author$' id='pdfcolparcolumns'>
  <name>pdfcolparcolumns</name>
  <caption>Fix colour problems in package 'parcolumns'.</caption>
  <authorref id='auth:oberdiek'/>
  <copyright owner='Heiko Oberdiek' year='2007,2008,2010'/>
  <license type='lppl1.3'/>
  <version number='1.4'/>
  <description>
    Since version 1.40 pdfTeX supports colour stacks.
    This package uses them to fix colour problems in
    package <xref refid='parcolumns'>parcolumns</xref>.
    <p/>
    The package is part of the <xref refid='oberdiek'>oberdiek</xref>
    bundle.
  </description>
  <documentation details='Package documentation'
      href='ctan:/macros/latex/contrib/oberdiek/pdfcolparcolumns.pdf'/>
  <ctan file='true' path='/macros/latex/contrib/oberdiek/pdfcolparcolumns.dtx'/>
  <miktex location='oberdiek'/>
  <texlive location='oberdiek'/>
  <install path='/macros/latex/contrib/oberdiek/oberdiek.tds.zip'/>
</entry>
%</catalogue>
%    \end{macrocode}
%
% \begin{thebibliography}{9}
%
% \bibitem{parcolumns}
%   Jonathan Sauer: \textit{The \xpackage{parcolumns} package};
%   2004/11/25;\\
%   \CTAN{macros/latex/contrib/sauerj/parcolumns.pdf}.
%
% \bibitem{pdfcol}
%   Heiko Oberdiek: \textit{The \xpackage{pdfcol} package};
%   2007/09/09;\\
%   \CTAN{macros/latex/contrib/oberdiek/pdfcol.pdf}.
%
% \end{thebibliography}
%
% \begin{History}
%   \begin{Version}{2007/07/26 v1.0}
%   \item
%     First version, published in the newsgroup \xnewsgroup{comp.text.tex}
%     with the name \xpackage{parcolumns-colorstacks}: ^^A no line break
%     \URL{``\link{Re: \xpackage{xcolor} glitches}''}^^A
%     {http://groups.google.com/group/comp.text.tex/msg/56bd897b11bca414}
%   \end{Version}
%   \begin{Version}{2007/09/09 v1.1}
%   \item
%     CTAN version, package name renamed to \xpackage{pdfcolparcolumns}.
%   \item
%     Uses package \xpackage{pdfcol}.
%   \item
%     Documentation added.
%   \item
%     Test file added.
%   \end{Version}
%   \begin{Version}{2008/08/11 v1.2}
%   \item
%     Code is not changed.
%   \item
%     URLs updated.
%   \end{Version}
%   \begin{Version}{2010/01/11 v1.3}
%   \item
%     Fix for rule color.
%   \item
%     New option \xoption{rulebetweencolor} for environment |parcolumns|.
%   \end{Version}
%   \begin{Version}{2016/05/16 v1.4}
%   \item
%     Documentation updates.
%   \end{Version}
% \end{History}
%
% \PrintIndex
%
% \Finale
\endinput
|
% \end{quote}
% Do not forget to quote the argument according to the demands
% of your shell.
%
% \paragraph{Generating the documentation.}
% You can use both the \xfile{.dtx} or the \xfile{.drv} to generate
% the documentation. The process can be configured by the
% configuration file \xfile{ltxdoc.cfg}. For instance, put this
% line into this file, if you want to have A4 as paper format:
% \begin{quote}
%   \verb|\PassOptionsToClass{a4paper}{article}|
% \end{quote}
% An example follows how to generate the
% documentation with pdf\LaTeX:
% \begin{quote}
%\begin{verbatim}
%pdflatex pdfcolparcolumns.dtx
%makeindex -s gind.ist pdfcolparcolumns.idx
%pdflatex pdfcolparcolumns.dtx
%makeindex -s gind.ist pdfcolparcolumns.idx
%pdflatex pdfcolparcolumns.dtx
%\end{verbatim}
% \end{quote}
%
% \section{Catalogue}
%
% The following XML file can be used as source for the
% \href{http://mirror.ctan.org/help/Catalogue/catalogue.html}{\TeX\ Catalogue}.
% The elements \texttt{caption} and \texttt{description} are imported
% from the original XML file from the Catalogue.
% The name of the XML file in the Catalogue is \xfile{pdfcolparcolumns.xml}.
%    \begin{macrocode}
%<*catalogue>
<?xml version='1.0' encoding='us-ascii'?>
<!DOCTYPE entry SYSTEM 'catalogue.dtd'>
<entry datestamp='$Date$' modifier='$Author$' id='pdfcolparcolumns'>
  <name>pdfcolparcolumns</name>
  <caption>Fix colour problems in package 'parcolumns'.</caption>
  <authorref id='auth:oberdiek'/>
  <copyright owner='Heiko Oberdiek' year='2007,2008,2010'/>
  <license type='lppl1.3'/>
  <version number='1.4'/>
  <description>
    Since version 1.40 pdfTeX supports colour stacks.
    This package uses them to fix colour problems in
    package <xref refid='parcolumns'>parcolumns</xref>.
    <p/>
    The package is part of the <xref refid='oberdiek'>oberdiek</xref>
    bundle.
  </description>
  <documentation details='Package documentation'
      href='ctan:/macros/latex/contrib/oberdiek/pdfcolparcolumns.pdf'/>
  <ctan file='true' path='/macros/latex/contrib/oberdiek/pdfcolparcolumns.dtx'/>
  <miktex location='oberdiek'/>
  <texlive location='oberdiek'/>
  <install path='/macros/latex/contrib/oberdiek/oberdiek.tds.zip'/>
</entry>
%</catalogue>
%    \end{macrocode}
%
% \begin{thebibliography}{9}
%
% \bibitem{parcolumns}
%   Jonathan Sauer: \textit{The \xpackage{parcolumns} package};
%   2004/11/25;\\
%   \CTAN{macros/latex/contrib/sauerj/parcolumns.pdf}.
%
% \bibitem{pdfcol}
%   Heiko Oberdiek: \textit{The \xpackage{pdfcol} package};
%   2007/09/09;\\
%   \CTAN{macros/latex/contrib/oberdiek/pdfcol.pdf}.
%
% \end{thebibliography}
%
% \begin{History}
%   \begin{Version}{2007/07/26 v1.0}
%   \item
%     First version, published in the newsgroup \xnewsgroup{comp.text.tex}
%     with the name \xpackage{parcolumns-colorstacks}: ^^A no line break
%     \URL{``\link{Re: \xpackage{xcolor} glitches}''}^^A
%     {http://groups.google.com/group/comp.text.tex/msg/56bd897b11bca414}
%   \end{Version}
%   \begin{Version}{2007/09/09 v1.1}
%   \item
%     CTAN version, package name renamed to \xpackage{pdfcolparcolumns}.
%   \item
%     Uses package \xpackage{pdfcol}.
%   \item
%     Documentation added.
%   \item
%     Test file added.
%   \end{Version}
%   \begin{Version}{2008/08/11 v1.2}
%   \item
%     Code is not changed.
%   \item
%     URLs updated.
%   \end{Version}
%   \begin{Version}{2010/01/11 v1.3}
%   \item
%     Fix for rule color.
%   \item
%     New option \xoption{rulebetweencolor} for environment |parcolumns|.
%   \end{Version}
%   \begin{Version}{2016/05/16 v1.4}
%   \item
%     Documentation updates.
%   \end{Version}
% \end{History}
%
% \PrintIndex
%
% \Finale
\endinput

%        (quote the arguments according to the demands of your shell)
%
% Documentation:
%    (a) If pdfcolparcolumns.drv is present:
%           latex pdfcolparcolumns.drv
%    (b) Without pdfcolparcolumns.drv:
%           latex pdfcolparcolumns.dtx; ...
%    The class ltxdoc loads the configuration file ltxdoc.cfg
%    if available. Here you can specify further options, e.g.
%    use A4 as paper format:
%       \PassOptionsToClass{a4paper}{article}
%
%    Programm calls to get the documentation (example):
%       pdflatex pdfcolparcolumns.dtx
%       makeindex -s gind.ist pdfcolparcolumns.idx
%       pdflatex pdfcolparcolumns.dtx
%       makeindex -s gind.ist pdfcolparcolumns.idx
%       pdflatex pdfcolparcolumns.dtx
%
% Installation:
%    TDS:tex/latex/oberdiek/pdfcolparcolumns.sty
%    TDS:doc/latex/oberdiek/pdfcolparcolumns.pdf
%    TDS:doc/latex/oberdiek/test/pdfcolparcolumns-test1.tex
%    TDS:source/latex/oberdiek/pdfcolparcolumns.dtx
%
%<*ignore>
\begingroup
  \catcode123=1 %
  \catcode125=2 %
  \def\x{LaTeX2e}%
\expandafter\endgroup
\ifcase 0\ifx\install y1\fi\expandafter
         \ifx\csname processbatchFile\endcsname\relax\else1\fi
         \ifx\fmtname\x\else 1\fi\relax
\else\csname fi\endcsname
%</ignore>
%<*install>
\input docstrip.tex
\Msg{************************************************************************}
\Msg{* Installation}
\Msg{* Package: pdfcolparcolumns 2016/05/16 v1.4 Color stacks for parcolumns (HO)}
\Msg{************************************************************************}

\keepsilent
\askforoverwritefalse

\let\MetaPrefix\relax
\preamble

This is a generated file.

Project: pdfcolparcolumns
Version: 2016/05/16 v1.4

Copyright (C) 2007, 2008, 2010 by
   Heiko Oberdiek <heiko.oberdiek at googlemail.com>

This work may be distributed and/or modified under the
conditions of the LaTeX Project Public License, either
version 1.3c of this license or (at your option) any later
version. This version of this license is in
   http://www.latex-project.org/lppl/lppl-1-3c.txt
and the latest version of this license is in
   http://www.latex-project.org/lppl.txt
and version 1.3 or later is part of all distributions of
LaTeX version 2005/12/01 or later.

This work has the LPPL maintenance status "maintained".

This Current Maintainer of this work is Heiko Oberdiek.

This work consists of the main source file pdfcolparcolumns.dtx
and the derived files
   pdfcolparcolumns.sty, pdfcolparcolumns.pdf, pdfcolparcolumns.ins,
   pdfcolparcolumns.drv, pdfcolparcolumns-test1.tex.

\endpreamble
\let\MetaPrefix\DoubleperCent

\generate{%
  \file{pdfcolparcolumns.ins}{\from{pdfcolparcolumns.dtx}{install}}%
  \file{pdfcolparcolumns.drv}{\from{pdfcolparcolumns.dtx}{driver}}%
  \usedir{tex/latex/oberdiek}%
  \file{pdfcolparcolumns.sty}{\from{pdfcolparcolumns.dtx}{package}}%
  \usedir{doc/latex/oberdiek/test}%
  \file{pdfcolparcolumns-test1.tex}{\from{pdfcolparcolumns.dtx}{test1}}%
  \nopreamble
  \nopostamble
  \usedir{source/latex/oberdiek/catalogue}%
  \file{pdfcolparcolumns.xml}{\from{pdfcolparcolumns.dtx}{catalogue}}%
}

\catcode32=13\relax% active space
\let =\space%
\Msg{************************************************************************}
\Msg{*}
\Msg{* To finish the installation you have to move the following}
\Msg{* file into a directory searched by TeX:}
\Msg{*}
\Msg{*     pdfcolparcolumns.sty}
\Msg{*}
\Msg{* To produce the documentation run the file `pdfcolparcolumns.drv'}
\Msg{* through LaTeX.}
\Msg{*}
\Msg{* Happy TeXing!}
\Msg{*}
\Msg{************************************************************************}

\endbatchfile
%</install>
%<*ignore>
\fi
%</ignore>
%<*driver>
\NeedsTeXFormat{LaTeX2e}
\ProvidesFile{pdfcolparcolumns.drv}%
  [2016/05/16 v1.4 Color stacks for parcolumns (HO)]%
\documentclass{ltxdoc}
\usepackage{holtxdoc}[2011/11/22]
\begin{document}
  \DocInput{pdfcolparcolumns.dtx}%
\end{document}
%</driver>
% \fi
%
%
% \CharacterTable
%  {Upper-case    \A\B\C\D\E\F\G\H\I\J\K\L\M\N\O\P\Q\R\S\T\U\V\W\X\Y\Z
%   Lower-case    \a\b\c\d\e\f\g\h\i\j\k\l\m\n\o\p\q\r\s\t\u\v\w\x\y\z
%   Digits        \0\1\2\3\4\5\6\7\8\9
%   Exclamation   \!     Double quote  \"     Hash (number) \#
%   Dollar        \$     Percent       \%     Ampersand     \&
%   Acute accent  \'     Left paren    \(     Right paren   \)
%   Asterisk      \*     Plus          \+     Comma         \,
%   Minus         \-     Point         \.     Solidus       \/
%   Colon         \:     Semicolon     \;     Less than     \<
%   Equals        \=     Greater than  \>     Question mark \?
%   Commercial at \@     Left bracket  \[     Backslash     \\
%   Right bracket \]     Circumflex    \^     Underscore    \_
%   Grave accent  \`     Left brace    \{     Vertical bar  \|
%   Right brace   \}     Tilde         \~}
%
% \GetFileInfo{pdfcolparcolumns.drv}
%
% \title{The \xpackage{pdfcolparcolumns} package}
% \date{2016/05/16 v1.4}
% \author{Heiko Oberdiek\thanks
% {Please report any issues at https://github.com/ho-tex/oberdiek/issues}\\
% \xemail{heiko.oberdiek at googlemail.com}}
%
% \maketitle
%
% \begin{abstract}
% Since version 1.40 \pdfTeX\ supports several color stacks.
% This package uses them to fix color problems in
% package \xpackage{parcolumns}.
% \end{abstract}
%
% \tableofcontents
%
% \section{Usage}
%
% \begin{quote}
% |\usepackage{pdfcolparcolumns}|
% \end{quote}
% The package \xpackage{pdfcolparcolumns} loads package \xpackage{parcolums}
% \cite{parcolumns}. If color stacks are available then the
% macros of \xpackage{parcolumns} are patched to add support
% for color stacks.
%
% \subsection{Option \xoption{rulebetweencolor}}
%
% Package \xpackage{pdfcolparcolumns} also fixes the color for the
% rule between columns (if \xoption{rulebetween} is set).
% Default color is \cs{normalcolor}. But this can be changed by using
% option \xoption{rulebetweencolor}. It takes a color specification
% as value. If the value is empty, then the default (\cs{normalcolor})
% is used.
% Examples:
% \begin{quote}
%   |rulebetweencolor=blue|,\\
%   |rulebetweencolor={red}|,\\
%   |rulebetweencolor={}|, \textit{\% \cs{normalcolor} is used}\\
%   |rulebetweencolor=[rgb]{1,0,.5}| \textit{\% see below}
% \end{quote}
% If used inside the optional argument of environment \textsf{parcolumns}
% and the value contains an optional argument, then whole value
% must be put in curly braces:
%\begin{quote}
%\begin{verbatim}
%\begin{parcolumns}[
%  rulebetween,
%  rulebetweencolor={[rgb]{1,0,.5}},
%]{2}
%  ...
%\end{parcolumns}
%\end{verbatim}
%\end{quote}
% This option \xoption{rulebetweencolor} can also be set using
% \cs{setkeys}:
%\begin{quote}
%\begin{verbatim}
%\setkeys{parcolumns}{rulebetweencolor=green}
%\end{verbatim}
%\end{quote}
%
% \subsection{Future}
%
% Currently package \xpackage{parcolumns} does not seem to be
% maintained. Nevertheless if there will be a new version that
% adds support for color stacks, then this package may become
% obsolete.
%
% \StopEventually{
% }
%
% \section{Implementation}
%
% \subsection{Identification}
%
%    \begin{macrocode}
%<*package>
\NeedsTeXFormat{LaTeX2e}
\ProvidesPackage{pdfcolparcolumns}%
  [2016/05/16 v1.4 Color stacks for parcolumns (HO)]%
%    \end{macrocode}
%
% \subsection{Load packages}
%
% \subsubsection{Package \xpackage{parcolumns}}
%
%    Currently package \xpackage{parcolumns} does not define options.
%    Thus it is just a precaution that the options of
%    package \xpackage{pdfcolparcolumns} are passed to
%    package \xpackage{parcolumns}.
%    \begin{macrocode}
\DeclareOption*{%
  \PassoptionsToPackage{\CurrentOption}{parcolumns}%
}
\ProcessOptions\relax
\RequirePackage{parcolumns}[2004/11/25]
%    \end{macrocode}
%
% \subsubsection{Package \xpackage{pdfcol}}
%
%    \begin{macrocode}
\RequirePackage{pdfcol}[2007/09/09]
\ifpdfcolAvailable
\else
  \PackageInfo{pdfcolparcolumns}{%
    Loading aborted, because color stacks are not available%
  }%
  \expandafter\endinput
\fi
%    \end{macrocode}
%
% \subsubsection{Package \xpackage{infwarerr}}
%
%    \begin{macrocode}
\RequirePackage{infwarerr}[2007/09/09]
%    \end{macrocode}
%
% \subsection{Color stack macros}
%
%    \begin{macro}{\pcpc@MaxStack}
%    Macro \cs{pcpc@MaxStack} holds the highest number of
%    allocated stacks.
%    \begin{macrocode}
\global\chardef\pcpc@MaxStack=\z@
%    \end{macrocode}
%    \end{macro}
%    \begin{macro}{\pcpc@InitStacks}
%    Macro \cs{pcpc@InitStacks} takes the number of columns
%    as argument and ensures that there are enough color
%    stacks for all columns.
%    \begin{macrocode}
\def\pcpc@InitStacks#1{%
  \ifnum#1>\pcpc@MaxStack
    \begingroup
      \count@\pcpc@MaxStack
      \loop
        \advance\count@\@ne
        \pdfcolInitStack{pcpc@\the\count@}%
      \ifnum#1>\count@
      \repeat
      \global\chardef\pcpc@MaxStack=\count@
    \endgroup
  \fi
}
%    \end{macrocode}
%    \end{macro}
%
%    \begin{macro}{\pcpc@SwitchStack}
%    \begin{macrocode}
\def\pcpc@SwitchStack#1{%
  \pdfcolSwitchStack{pcpc@\number#1}%
}
%    \end{macrocode}
%    \end{macro}
%
%    \begin{macro}{\pcpc@SetCurrent}
%    \begin{macrocode}
\def\pcpc@SetCurrent#1{%
  \pdfcolSetCurrent{pcpc@\number#1}%
}
%    \end{macrocode}
%    \end{macro}
%
% \subsection{Patches}
%
%     Now the color stack macros are patched into the macros
%     of package \xpackage{parcolumns}.
%
% \subsubsection{Init stacks}
%
%    \cs{pcpc@InitStacks} should go into the definition of
%    environment |parcolumns|. \cs{pc@alloccolumns} is executed
%    there and nowhere else, thus we hook into it.
%    \begin{macrocode}
\g@addto@macro\pc@alloccolumns{%
  \pcpc@InitStacks\pc@columncount
}
%    \end{macrocode}
%
% \subsubsection{Switch stack}
%
%    \cs{pcpc@SwitchStack} should be called by marco \cs{colchunk@}.
%    However it is easier to patch \cs{pc@setcolumnwidth} that
%    is executed in \cs{colchunk@} only.
%    \begin{macrocode}
\g@addto@macro\pc@setcolumnwidth{%
  \pcpc@SwitchStack\pc@columnctr
}
%    \end{macrocode}
%
% \subsubsection{Set current stack color}
%
%    \cs{pcpc@SetCurrent} is set at the begin of each line.
%    It must be inserted into \cs{pc@placeboxes}. Unhappily
%    there is no easy way. Therefore we check and
%    redefine \cs{pc@placeboxes}.
%    \begin{macrocode}
\begingroup
  \def\x{%
    \global\let\@tempa\relax
    \count@\z@
    \hb@xt@\linewidth{%
      \vfuzz30ex %
      \vbadness\@M
      \splittopskip\z@skip
      \loop
      \ifnum\count@<\pc@columncount
        \advance\count@\@ne
        \expandafter\ifvoid\csname pc@column@\number\count@\endcsname
          \hskip\csname pc@column@width@\number\count@\endcsname
        \else
          \expandafter\setbox\expandafter\@tempboxa\expandafter
          \vsplit\csname pc@column@\number\count@\endcsname
              to \dp\strutbox
          \vbox{%
            \unvbox\@tempboxa
          }%
        \fi
        \expandafter\ifvoid\csname pc@column@\number\count@\endcsname
        \else
          \global\let\@tempa\pc@placeboxes
        \fi
        \ifnum\count@<\pc@columncount
          \strut
          \hfill
          \ifpc@rulebetween
            \vrule
            \hfill
          \fi
        \fi
      \repeat
    }%
    \@tempa
  }%
  \ifx\x\pc@placeboxes
  \else
    \@PackageWarningNoLine{pdfcolparcolumns}{%
      Command \string\pc@placeboxes\space has changed.\MessageBreak
      Supported versions of package `parcolumns':\MessageBreak
      \space\space 2004/08/05.\MessageBreak
      The redefinition of \string\pc@placeboxes\space may not%
      \MessageBreak
      behave correctly depending on the changes%
    }%
  \fi
\endgroup
%    \end{macrocode}
%    \begin{macro}{\pc@placeboxes}
%    \begin{macrocode}
\renewcommand*{\pc@placeboxes}{%
  \global\let\@tempa\relax
  \count@\z@
  \hb@xt@\linewidth{%
    \vfuzz30ex %
    \vbadness\@M
    \splittopskip\z@skip
    \loop
    \ifnum\count@<\pc@columncount
      \advance\count@\@ne
      \expandafter\ifvoid\csname pc@column@\number\count@\endcsname
        \hskip\csname pc@column@width@\number\count@\endcsname
      \else
        \expandafter\setbox\expandafter\@tempboxa\expandafter
        \vsplit\csname pc@column@\number\count@\endcsname
            to \dp\strutbox
        \vbox{%
          \pcpc@SetCurrent\count@
          \unvbox\@tempboxa
        }%
      \fi
      \expandafter\ifvoid\csname pc@column@\number\count@\endcsname
      \else
        \global\let\@tempa\pc@placeboxes
      \fi
      \ifnum\count@<\pc@columncount
        \strut
        \hfill
        \ifpc@rulebetween
          \begingroup
            \pcpc@RuleBetweenColor
            \vrule
          \endgroup
          \hfill
        \fi
      \fi
    \repeat
  }%
  \@tempa
}
%    \end{macrocode}
%    \end{macro}
%    \begin{macro}{\pcpc@RuleBetweenColorDefault}
%    \begin{macrocode}
\def\pcpc@RuleBetweenColorDefault{%
  \normalcolor
}
%    \end{macrocode}
%    \end{macro}
%    \begin{macro}{\pcpc@RuleBetweenColor}
%    \begin{macrocode}
\let\pcpc@RuleBetweenColor\pcpc@RuleBetweenColorDefault
%    \end{macrocode}
%    \end{macro}
%    \begin{macrocode}
\define@key{parcolumns}{rulebetweencolor}{%
  \edef\pcpc@temp{#1}%
  \ifx\pcpc@temp\@empty
    \let\pcpc@RuleBetweenColor\pcpc@RuleBetweenColorDefault
  \else
    \edef\pcpc@temp{%
      \noexpand\@ifnextchar[{%
        \def\noexpand\pcpc@RuleBetweenColor{%
          \noexpand\color\pcpc@temp
        }%
        \noexpand\pcpc@GobbleNil
      }{%
        \def\noexpand\pcpc@RuleBetweenColor{%
          \noexpand\color{\pcpc@temp}%
        }%
        \noexpand\pcpc@GobbleNil
      }%
      \pcpc@temp\noexpand\@nil
    }%
    \pcpc@temp
  \fi
}
%    \end{macrocode}
%    \begin{macro}{\pcpc@GobbleNil}
%    \begin{macrocode}
\long\def\pcpc@GobbleNil#1\@nil{}
%    \end{macrocode}
%    \end{macro}
%
%    \begin{macrocode}
%</package>
%    \end{macrocode}
%
% \section{Test}
%
%    The test file is a modified version of the file that
%    Donald Goodman has posted in \xnewsgroup{comp.text.tex}: ^^A
%    \URL{``\link{Re: \xpackage{xcolor} glitches}''}^^A
%    {http://groups.google.com/group/comp.text.tex/msg/8eda74ed292012bb}
%    \begin{macrocode}
%<*test1>
\NeedsTeXFormat{LaTeX2e}
\AtEndDocument{%
  \typeout{}%
  \typeout{**************************************}%
  \typeout{*** \space Check the PDF file manually! \space ***}%
  \typeout{**************************************}%
  \typeout{}%
}
\documentclass{article}
\usepackage{xcolor}
\usepackage{pdfcolparcolumns}

\newcommand{\instruct}[1]{%
  \noindent
  \footnotesize
  \textcolor{red}{#1}%
}

\begin{document}
  \begin{parcolumns}[colwidths={1=2.3in,2=2.3in},sloppy]{2}%
    \colchunk[1]{%
      \instruct{Et non dicitur versus} %
      Fidelium anim\ae\ %
      \instruct{%
        sed immediate subiungitur antiphona finalis %
        beat\ae\ Mari\ae\ Virginis%
      } %
      100.%
    }%
    \colchunk[2]{%
      \instruct{%
        And the verse %
        \textcolor{black}{May the souls of the faithful} %
        is not said, but the final antiphon of the %
        Blessed Virgin Mary, %
        \textcolor{black}{100,} %
        is immediately joined.%
      }%
    }%
  \end{parcolumns}%
\end{document}
%</test1>
%    \end{macrocode}
%
% \section{Installation}
%
% \subsection{Download}
%
% \paragraph{Package.} This package is available on
% CTAN\footnote{\url{http://ctan.org/pkg/pdfcolparcolumns}}:
% \begin{description}
% \item[\CTAN{macros/latex/contrib/oberdiek/pdfcolparcolumns.dtx}] The source file.
% \item[\CTAN{macros/latex/contrib/oberdiek/pdfcolparcolumns.pdf}] Documentation.
% \end{description}
%
%
% \paragraph{Bundle.} All the packages of the bundle `oberdiek'
% are also available in a TDS compliant ZIP archive. There
% the packages are already unpacked and the documentation files
% are generated. The files and directories obey the TDS standard.
% \begin{description}
% \item[\CTAN{install/macros/latex/contrib/oberdiek.tds.zip}]
% \end{description}
% \emph{TDS} refers to the standard ``A Directory Structure
% for \TeX\ Files'' (\CTAN{tds/tds.pdf}). Directories
% with \xfile{texmf} in their name are usually organized this way.
%
% \subsection{Bundle installation}
%
% \paragraph{Unpacking.} Unpack the \xfile{oberdiek.tds.zip} in the
% TDS tree (also known as \xfile{texmf} tree) of your choice.
% Example (linux):
% \begin{quote}
%   |unzip oberdiek.tds.zip -d ~/texmf|
% \end{quote}
%
% \paragraph{Script installation.}
% Check the directory \xfile{TDS:scripts/oberdiek/} for
% scripts that need further installation steps.
% Package \xpackage{attachfile2} comes with the Perl script
% \xfile{pdfatfi.pl} that should be installed in such a way
% that it can be called as \texttt{pdfatfi}.
% Example (linux):
% \begin{quote}
%   |chmod +x scripts/oberdiek/pdfatfi.pl|\\
%   |cp scripts/oberdiek/pdfatfi.pl /usr/local/bin/|
% \end{quote}
%
% \subsection{Package installation}
%
% \paragraph{Unpacking.} The \xfile{.dtx} file is a self-extracting
% \docstrip\ archive. The files are extracted by running the
% \xfile{.dtx} through \plainTeX:
% \begin{quote}
%   \verb|tex pdfcolparcolumns.dtx|
% \end{quote}
%
% \paragraph{TDS.} Now the different files must be moved into
% the different directories in your installation TDS tree
% (also known as \xfile{texmf} tree):
% \begin{quote}
% \def\t{^^A
% \begin{tabular}{@{}>{\ttfamily}l@{ $\rightarrow$ }>{\ttfamily}l@{}}
%   pdfcolparcolumns.sty & tex/latex/oberdiek/pdfcolparcolumns.sty\\
%   pdfcolparcolumns.pdf & doc/latex/oberdiek/pdfcolparcolumns.pdf\\
%   test/pdfcolparcolumns-test1.tex & doc/latex/oberdiek/test/pdfcolparcolumns-test1.tex\\
%   pdfcolparcolumns.dtx & source/latex/oberdiek/pdfcolparcolumns.dtx\\
% \end{tabular}^^A
% }^^A
% \sbox0{\t}^^A
% \ifdim\wd0>\linewidth
%   \begingroup
%     \advance\linewidth by\leftmargin
%     \advance\linewidth by\rightmargin
%   \edef\x{\endgroup
%     \def\noexpand\lw{\the\linewidth}^^A
%   }\x
%   \def\lwbox{^^A
%     \leavevmode
%     \hbox to \linewidth{^^A
%       \kern-\leftmargin\relax
%       \hss
%       \usebox0
%       \hss
%       \kern-\rightmargin\relax
%     }^^A
%   }^^A
%   \ifdim\wd0>\lw
%     \sbox0{\small\t}^^A
%     \ifdim\wd0>\linewidth
%       \ifdim\wd0>\lw
%         \sbox0{\footnotesize\t}^^A
%         \ifdim\wd0>\linewidth
%           \ifdim\wd0>\lw
%             \sbox0{\scriptsize\t}^^A
%             \ifdim\wd0>\linewidth
%               \ifdim\wd0>\lw
%                 \sbox0{\tiny\t}^^A
%                 \ifdim\wd0>\linewidth
%                   \lwbox
%                 \else
%                   \usebox0
%                 \fi
%               \else
%                 \lwbox
%               \fi
%             \else
%               \usebox0
%             \fi
%           \else
%             \lwbox
%           \fi
%         \else
%           \usebox0
%         \fi
%       \else
%         \lwbox
%       \fi
%     \else
%       \usebox0
%     \fi
%   \else
%     \lwbox
%   \fi
% \else
%   \usebox0
% \fi
% \end{quote}
% If you have a \xfile{docstrip.cfg} that configures and enables \docstrip's
% TDS installing feature, then some files can already be in the right
% place, see the documentation of \docstrip.
%
% \subsection{Refresh file name databases}
%
% If your \TeX~distribution
% (\teTeX, \mikTeX, \dots) relies on file name databases, you must refresh
% these. For example, \teTeX\ users run \verb|texhash| or
% \verb|mktexlsr|.
%
% \subsection{Some details for the interested}
%
% \paragraph{Attached source.}
%
% The PDF documentation on CTAN also includes the
% \xfile{.dtx} source file. It can be extracted by
% AcrobatReader 6 or higher. Another option is \textsf{pdftk},
% e.g. unpack the file into the current directory:
% \begin{quote}
%   \verb|pdftk pdfcolparcolumns.pdf unpack_files output .|
% \end{quote}
%
% \paragraph{Unpacking with \LaTeX.}
% The \xfile{.dtx} chooses its action depending on the format:
% \begin{description}
% \item[\plainTeX:] Run \docstrip\ and extract the files.
% \item[\LaTeX:] Generate the documentation.
% \end{description}
% If you insist on using \LaTeX\ for \docstrip\ (really,
% \docstrip\ does not need \LaTeX), then inform the autodetect routine
% about your intention:
% \begin{quote}
%   \verb|latex \let\install=y% \iffalse meta-comment
%
% File: pdfcolparcolumns.dtx
% Version: 2016/05/16 v1.4
% Info: Color stacks for parcolumns
%
% Copyright (C) 2007, 2008, 2010 by
%    Heiko Oberdiek <heiko.oberdiek at googlemail.com>
%    2016
%    https://github.com/ho-tex/oberdiek/issues
%
% This work may be distributed and/or modified under the
% conditions of the LaTeX Project Public License, either
% version 1.3c of this license or (at your option) any later
% version. This version of this license is in
%    http://www.latex-project.org/lppl/lppl-1-3c.txt
% and the latest version of this license is in
%    http://www.latex-project.org/lppl.txt
% and version 1.3 or later is part of all distributions of
% LaTeX version 2005/12/01 or later.
%
% This work has the LPPL maintenance status "maintained".
%
% This Current Maintainer of this work is Heiko Oberdiek.
%
% This work consists of the main source file pdfcolparcolumns.dtx
% and the derived files
%    pdfcolparcolumns.sty, pdfcolparcolumns.pdf, pdfcolparcolumns.ins,
%    pdfcolparcolumns.drv, pdfcolparcolumns-test1.tex.
%
% Distribution:
%    CTAN:macros/latex/contrib/oberdiek/pdfcolparcolumns.dtx
%    CTAN:macros/latex/contrib/oberdiek/pdfcolparcolumns.pdf
%
% Unpacking:
%    (a) If pdfcolparcolumns.ins is present:
%           tex pdfcolparcolumns.ins
%    (b) Without pdfcolparcolumns.ins:
%           tex pdfcolparcolumns.dtx
%    (c) If you insist on using LaTeX
%           latex \let\install=y% \iffalse meta-comment
%
% File: pdfcolparcolumns.dtx
% Version: 2016/05/16 v1.4
% Info: Color stacks for parcolumns
%
% Copyright (C) 2007, 2008, 2010 by
%    Heiko Oberdiek <heiko.oberdiek at googlemail.com>
%    2016
%    https://github.com/ho-tex/oberdiek/issues
%
% This work may be distributed and/or modified under the
% conditions of the LaTeX Project Public License, either
% version 1.3c of this license or (at your option) any later
% version. This version of this license is in
%    http://www.latex-project.org/lppl/lppl-1-3c.txt
% and the latest version of this license is in
%    http://www.latex-project.org/lppl.txt
% and version 1.3 or later is part of all distributions of
% LaTeX version 2005/12/01 or later.
%
% This work has the LPPL maintenance status "maintained".
%
% This Current Maintainer of this work is Heiko Oberdiek.
%
% This work consists of the main source file pdfcolparcolumns.dtx
% and the derived files
%    pdfcolparcolumns.sty, pdfcolparcolumns.pdf, pdfcolparcolumns.ins,
%    pdfcolparcolumns.drv, pdfcolparcolumns-test1.tex.
%
% Distribution:
%    CTAN:macros/latex/contrib/oberdiek/pdfcolparcolumns.dtx
%    CTAN:macros/latex/contrib/oberdiek/pdfcolparcolumns.pdf
%
% Unpacking:
%    (a) If pdfcolparcolumns.ins is present:
%           tex pdfcolparcolumns.ins
%    (b) Without pdfcolparcolumns.ins:
%           tex pdfcolparcolumns.dtx
%    (c) If you insist on using LaTeX
%           latex \let\install=y\input{pdfcolparcolumns.dtx}
%        (quote the arguments according to the demands of your shell)
%
% Documentation:
%    (a) If pdfcolparcolumns.drv is present:
%           latex pdfcolparcolumns.drv
%    (b) Without pdfcolparcolumns.drv:
%           latex pdfcolparcolumns.dtx; ...
%    The class ltxdoc loads the configuration file ltxdoc.cfg
%    if available. Here you can specify further options, e.g.
%    use A4 as paper format:
%       \PassOptionsToClass{a4paper}{article}
%
%    Programm calls to get the documentation (example):
%       pdflatex pdfcolparcolumns.dtx
%       makeindex -s gind.ist pdfcolparcolumns.idx
%       pdflatex pdfcolparcolumns.dtx
%       makeindex -s gind.ist pdfcolparcolumns.idx
%       pdflatex pdfcolparcolumns.dtx
%
% Installation:
%    TDS:tex/latex/oberdiek/pdfcolparcolumns.sty
%    TDS:doc/latex/oberdiek/pdfcolparcolumns.pdf
%    TDS:doc/latex/oberdiek/test/pdfcolparcolumns-test1.tex
%    TDS:source/latex/oberdiek/pdfcolparcolumns.dtx
%
%<*ignore>
\begingroup
  \catcode123=1 %
  \catcode125=2 %
  \def\x{LaTeX2e}%
\expandafter\endgroup
\ifcase 0\ifx\install y1\fi\expandafter
         \ifx\csname processbatchFile\endcsname\relax\else1\fi
         \ifx\fmtname\x\else 1\fi\relax
\else\csname fi\endcsname
%</ignore>
%<*install>
\input docstrip.tex
\Msg{************************************************************************}
\Msg{* Installation}
\Msg{* Package: pdfcolparcolumns 2016/05/16 v1.4 Color stacks for parcolumns (HO)}
\Msg{************************************************************************}

\keepsilent
\askforoverwritefalse

\let\MetaPrefix\relax
\preamble

This is a generated file.

Project: pdfcolparcolumns
Version: 2016/05/16 v1.4

Copyright (C) 2007, 2008, 2010 by
   Heiko Oberdiek <heiko.oberdiek at googlemail.com>

This work may be distributed and/or modified under the
conditions of the LaTeX Project Public License, either
version 1.3c of this license or (at your option) any later
version. This version of this license is in
   http://www.latex-project.org/lppl/lppl-1-3c.txt
and the latest version of this license is in
   http://www.latex-project.org/lppl.txt
and version 1.3 or later is part of all distributions of
LaTeX version 2005/12/01 or later.

This work has the LPPL maintenance status "maintained".

This Current Maintainer of this work is Heiko Oberdiek.

This work consists of the main source file pdfcolparcolumns.dtx
and the derived files
   pdfcolparcolumns.sty, pdfcolparcolumns.pdf, pdfcolparcolumns.ins,
   pdfcolparcolumns.drv, pdfcolparcolumns-test1.tex.

\endpreamble
\let\MetaPrefix\DoubleperCent

\generate{%
  \file{pdfcolparcolumns.ins}{\from{pdfcolparcolumns.dtx}{install}}%
  \file{pdfcolparcolumns.drv}{\from{pdfcolparcolumns.dtx}{driver}}%
  \usedir{tex/latex/oberdiek}%
  \file{pdfcolparcolumns.sty}{\from{pdfcolparcolumns.dtx}{package}}%
  \usedir{doc/latex/oberdiek/test}%
  \file{pdfcolparcolumns-test1.tex}{\from{pdfcolparcolumns.dtx}{test1}}%
  \nopreamble
  \nopostamble
  \usedir{source/latex/oberdiek/catalogue}%
  \file{pdfcolparcolumns.xml}{\from{pdfcolparcolumns.dtx}{catalogue}}%
}

\catcode32=13\relax% active space
\let =\space%
\Msg{************************************************************************}
\Msg{*}
\Msg{* To finish the installation you have to move the following}
\Msg{* file into a directory searched by TeX:}
\Msg{*}
\Msg{*     pdfcolparcolumns.sty}
\Msg{*}
\Msg{* To produce the documentation run the file `pdfcolparcolumns.drv'}
\Msg{* through LaTeX.}
\Msg{*}
\Msg{* Happy TeXing!}
\Msg{*}
\Msg{************************************************************************}

\endbatchfile
%</install>
%<*ignore>
\fi
%</ignore>
%<*driver>
\NeedsTeXFormat{LaTeX2e}
\ProvidesFile{pdfcolparcolumns.drv}%
  [2016/05/16 v1.4 Color stacks for parcolumns (HO)]%
\documentclass{ltxdoc}
\usepackage{holtxdoc}[2011/11/22]
\begin{document}
  \DocInput{pdfcolparcolumns.dtx}%
\end{document}
%</driver>
% \fi
%
%
% \CharacterTable
%  {Upper-case    \A\B\C\D\E\F\G\H\I\J\K\L\M\N\O\P\Q\R\S\T\U\V\W\X\Y\Z
%   Lower-case    \a\b\c\d\e\f\g\h\i\j\k\l\m\n\o\p\q\r\s\t\u\v\w\x\y\z
%   Digits        \0\1\2\3\4\5\6\7\8\9
%   Exclamation   \!     Double quote  \"     Hash (number) \#
%   Dollar        \$     Percent       \%     Ampersand     \&
%   Acute accent  \'     Left paren    \(     Right paren   \)
%   Asterisk      \*     Plus          \+     Comma         \,
%   Minus         \-     Point         \.     Solidus       \/
%   Colon         \:     Semicolon     \;     Less than     \<
%   Equals        \=     Greater than  \>     Question mark \?
%   Commercial at \@     Left bracket  \[     Backslash     \\
%   Right bracket \]     Circumflex    \^     Underscore    \_
%   Grave accent  \`     Left brace    \{     Vertical bar  \|
%   Right brace   \}     Tilde         \~}
%
% \GetFileInfo{pdfcolparcolumns.drv}
%
% \title{The \xpackage{pdfcolparcolumns} package}
% \date{2016/05/16 v1.4}
% \author{Heiko Oberdiek\thanks
% {Please report any issues at https://github.com/ho-tex/oberdiek/issues}\\
% \xemail{heiko.oberdiek at googlemail.com}}
%
% \maketitle
%
% \begin{abstract}
% Since version 1.40 \pdfTeX\ supports several color stacks.
% This package uses them to fix color problems in
% package \xpackage{parcolumns}.
% \end{abstract}
%
% \tableofcontents
%
% \section{Usage}
%
% \begin{quote}
% |\usepackage{pdfcolparcolumns}|
% \end{quote}
% The package \xpackage{pdfcolparcolumns} loads package \xpackage{parcolums}
% \cite{parcolumns}. If color stacks are available then the
% macros of \xpackage{parcolumns} are patched to add support
% for color stacks.
%
% \subsection{Option \xoption{rulebetweencolor}}
%
% Package \xpackage{pdfcolparcolumns} also fixes the color for the
% rule between columns (if \xoption{rulebetween} is set).
% Default color is \cs{normalcolor}. But this can be changed by using
% option \xoption{rulebetweencolor}. It takes a color specification
% as value. If the value is empty, then the default (\cs{normalcolor})
% is used.
% Examples:
% \begin{quote}
%   |rulebetweencolor=blue|,\\
%   |rulebetweencolor={red}|,\\
%   |rulebetweencolor={}|, \textit{\% \cs{normalcolor} is used}\\
%   |rulebetweencolor=[rgb]{1,0,.5}| \textit{\% see below}
% \end{quote}
% If used inside the optional argument of environment \textsf{parcolumns}
% and the value contains an optional argument, then whole value
% must be put in curly braces:
%\begin{quote}
%\begin{verbatim}
%\begin{parcolumns}[
%  rulebetween,
%  rulebetweencolor={[rgb]{1,0,.5}},
%]{2}
%  ...
%\end{parcolumns}
%\end{verbatim}
%\end{quote}
% This option \xoption{rulebetweencolor} can also be set using
% \cs{setkeys}:
%\begin{quote}
%\begin{verbatim}
%\setkeys{parcolumns}{rulebetweencolor=green}
%\end{verbatim}
%\end{quote}
%
% \subsection{Future}
%
% Currently package \xpackage{parcolumns} does not seem to be
% maintained. Nevertheless if there will be a new version that
% adds support for color stacks, then this package may become
% obsolete.
%
% \StopEventually{
% }
%
% \section{Implementation}
%
% \subsection{Identification}
%
%    \begin{macrocode}
%<*package>
\NeedsTeXFormat{LaTeX2e}
\ProvidesPackage{pdfcolparcolumns}%
  [2016/05/16 v1.4 Color stacks for parcolumns (HO)]%
%    \end{macrocode}
%
% \subsection{Load packages}
%
% \subsubsection{Package \xpackage{parcolumns}}
%
%    Currently package \xpackage{parcolumns} does not define options.
%    Thus it is just a precaution that the options of
%    package \xpackage{pdfcolparcolumns} are passed to
%    package \xpackage{parcolumns}.
%    \begin{macrocode}
\DeclareOption*{%
  \PassoptionsToPackage{\CurrentOption}{parcolumns}%
}
\ProcessOptions\relax
\RequirePackage{parcolumns}[2004/11/25]
%    \end{macrocode}
%
% \subsubsection{Package \xpackage{pdfcol}}
%
%    \begin{macrocode}
\RequirePackage{pdfcol}[2007/09/09]
\ifpdfcolAvailable
\else
  \PackageInfo{pdfcolparcolumns}{%
    Loading aborted, because color stacks are not available%
  }%
  \expandafter\endinput
\fi
%    \end{macrocode}
%
% \subsubsection{Package \xpackage{infwarerr}}
%
%    \begin{macrocode}
\RequirePackage{infwarerr}[2007/09/09]
%    \end{macrocode}
%
% \subsection{Color stack macros}
%
%    \begin{macro}{\pcpc@MaxStack}
%    Macro \cs{pcpc@MaxStack} holds the highest number of
%    allocated stacks.
%    \begin{macrocode}
\global\chardef\pcpc@MaxStack=\z@
%    \end{macrocode}
%    \end{macro}
%    \begin{macro}{\pcpc@InitStacks}
%    Macro \cs{pcpc@InitStacks} takes the number of columns
%    as argument and ensures that there are enough color
%    stacks for all columns.
%    \begin{macrocode}
\def\pcpc@InitStacks#1{%
  \ifnum#1>\pcpc@MaxStack
    \begingroup
      \count@\pcpc@MaxStack
      \loop
        \advance\count@\@ne
        \pdfcolInitStack{pcpc@\the\count@}%
      \ifnum#1>\count@
      \repeat
      \global\chardef\pcpc@MaxStack=\count@
    \endgroup
  \fi
}
%    \end{macrocode}
%    \end{macro}
%
%    \begin{macro}{\pcpc@SwitchStack}
%    \begin{macrocode}
\def\pcpc@SwitchStack#1{%
  \pdfcolSwitchStack{pcpc@\number#1}%
}
%    \end{macrocode}
%    \end{macro}
%
%    \begin{macro}{\pcpc@SetCurrent}
%    \begin{macrocode}
\def\pcpc@SetCurrent#1{%
  \pdfcolSetCurrent{pcpc@\number#1}%
}
%    \end{macrocode}
%    \end{macro}
%
% \subsection{Patches}
%
%     Now the color stack macros are patched into the macros
%     of package \xpackage{parcolumns}.
%
% \subsubsection{Init stacks}
%
%    \cs{pcpc@InitStacks} should go into the definition of
%    environment |parcolumns|. \cs{pc@alloccolumns} is executed
%    there and nowhere else, thus we hook into it.
%    \begin{macrocode}
\g@addto@macro\pc@alloccolumns{%
  \pcpc@InitStacks\pc@columncount
}
%    \end{macrocode}
%
% \subsubsection{Switch stack}
%
%    \cs{pcpc@SwitchStack} should be called by marco \cs{colchunk@}.
%    However it is easier to patch \cs{pc@setcolumnwidth} that
%    is executed in \cs{colchunk@} only.
%    \begin{macrocode}
\g@addto@macro\pc@setcolumnwidth{%
  \pcpc@SwitchStack\pc@columnctr
}
%    \end{macrocode}
%
% \subsubsection{Set current stack color}
%
%    \cs{pcpc@SetCurrent} is set at the begin of each line.
%    It must be inserted into \cs{pc@placeboxes}. Unhappily
%    there is no easy way. Therefore we check and
%    redefine \cs{pc@placeboxes}.
%    \begin{macrocode}
\begingroup
  \def\x{%
    \global\let\@tempa\relax
    \count@\z@
    \hb@xt@\linewidth{%
      \vfuzz30ex %
      \vbadness\@M
      \splittopskip\z@skip
      \loop
      \ifnum\count@<\pc@columncount
        \advance\count@\@ne
        \expandafter\ifvoid\csname pc@column@\number\count@\endcsname
          \hskip\csname pc@column@width@\number\count@\endcsname
        \else
          \expandafter\setbox\expandafter\@tempboxa\expandafter
          \vsplit\csname pc@column@\number\count@\endcsname
              to \dp\strutbox
          \vbox{%
            \unvbox\@tempboxa
          }%
        \fi
        \expandafter\ifvoid\csname pc@column@\number\count@\endcsname
        \else
          \global\let\@tempa\pc@placeboxes
        \fi
        \ifnum\count@<\pc@columncount
          \strut
          \hfill
          \ifpc@rulebetween
            \vrule
            \hfill
          \fi
        \fi
      \repeat
    }%
    \@tempa
  }%
  \ifx\x\pc@placeboxes
  \else
    \@PackageWarningNoLine{pdfcolparcolumns}{%
      Command \string\pc@placeboxes\space has changed.\MessageBreak
      Supported versions of package `parcolumns':\MessageBreak
      \space\space 2004/08/05.\MessageBreak
      The redefinition of \string\pc@placeboxes\space may not%
      \MessageBreak
      behave correctly depending on the changes%
    }%
  \fi
\endgroup
%    \end{macrocode}
%    \begin{macro}{\pc@placeboxes}
%    \begin{macrocode}
\renewcommand*{\pc@placeboxes}{%
  \global\let\@tempa\relax
  \count@\z@
  \hb@xt@\linewidth{%
    \vfuzz30ex %
    \vbadness\@M
    \splittopskip\z@skip
    \loop
    \ifnum\count@<\pc@columncount
      \advance\count@\@ne
      \expandafter\ifvoid\csname pc@column@\number\count@\endcsname
        \hskip\csname pc@column@width@\number\count@\endcsname
      \else
        \expandafter\setbox\expandafter\@tempboxa\expandafter
        \vsplit\csname pc@column@\number\count@\endcsname
            to \dp\strutbox
        \vbox{%
          \pcpc@SetCurrent\count@
          \unvbox\@tempboxa
        }%
      \fi
      \expandafter\ifvoid\csname pc@column@\number\count@\endcsname
      \else
        \global\let\@tempa\pc@placeboxes
      \fi
      \ifnum\count@<\pc@columncount
        \strut
        \hfill
        \ifpc@rulebetween
          \begingroup
            \pcpc@RuleBetweenColor
            \vrule
          \endgroup
          \hfill
        \fi
      \fi
    \repeat
  }%
  \@tempa
}
%    \end{macrocode}
%    \end{macro}
%    \begin{macro}{\pcpc@RuleBetweenColorDefault}
%    \begin{macrocode}
\def\pcpc@RuleBetweenColorDefault{%
  \normalcolor
}
%    \end{macrocode}
%    \end{macro}
%    \begin{macro}{\pcpc@RuleBetweenColor}
%    \begin{macrocode}
\let\pcpc@RuleBetweenColor\pcpc@RuleBetweenColorDefault
%    \end{macrocode}
%    \end{macro}
%    \begin{macrocode}
\define@key{parcolumns}{rulebetweencolor}{%
  \edef\pcpc@temp{#1}%
  \ifx\pcpc@temp\@empty
    \let\pcpc@RuleBetweenColor\pcpc@RuleBetweenColorDefault
  \else
    \edef\pcpc@temp{%
      \noexpand\@ifnextchar[{%
        \def\noexpand\pcpc@RuleBetweenColor{%
          \noexpand\color\pcpc@temp
        }%
        \noexpand\pcpc@GobbleNil
      }{%
        \def\noexpand\pcpc@RuleBetweenColor{%
          \noexpand\color{\pcpc@temp}%
        }%
        \noexpand\pcpc@GobbleNil
      }%
      \pcpc@temp\noexpand\@nil
    }%
    \pcpc@temp
  \fi
}
%    \end{macrocode}
%    \begin{macro}{\pcpc@GobbleNil}
%    \begin{macrocode}
\long\def\pcpc@GobbleNil#1\@nil{}
%    \end{macrocode}
%    \end{macro}
%
%    \begin{macrocode}
%</package>
%    \end{macrocode}
%
% \section{Test}
%
%    The test file is a modified version of the file that
%    Donald Goodman has posted in \xnewsgroup{comp.text.tex}: ^^A
%    \URL{``\link{Re: \xpackage{xcolor} glitches}''}^^A
%    {http://groups.google.com/group/comp.text.tex/msg/8eda74ed292012bb}
%    \begin{macrocode}
%<*test1>
\NeedsTeXFormat{LaTeX2e}
\AtEndDocument{%
  \typeout{}%
  \typeout{**************************************}%
  \typeout{*** \space Check the PDF file manually! \space ***}%
  \typeout{**************************************}%
  \typeout{}%
}
\documentclass{article}
\usepackage{xcolor}
\usepackage{pdfcolparcolumns}

\newcommand{\instruct}[1]{%
  \noindent
  \footnotesize
  \textcolor{red}{#1}%
}

\begin{document}
  \begin{parcolumns}[colwidths={1=2.3in,2=2.3in},sloppy]{2}%
    \colchunk[1]{%
      \instruct{Et non dicitur versus} %
      Fidelium anim\ae\ %
      \instruct{%
        sed immediate subiungitur antiphona finalis %
        beat\ae\ Mari\ae\ Virginis%
      } %
      100.%
    }%
    \colchunk[2]{%
      \instruct{%
        And the verse %
        \textcolor{black}{May the souls of the faithful} %
        is not said, but the final antiphon of the %
        Blessed Virgin Mary, %
        \textcolor{black}{100,} %
        is immediately joined.%
      }%
    }%
  \end{parcolumns}%
\end{document}
%</test1>
%    \end{macrocode}
%
% \section{Installation}
%
% \subsection{Download}
%
% \paragraph{Package.} This package is available on
% CTAN\footnote{\url{http://ctan.org/pkg/pdfcolparcolumns}}:
% \begin{description}
% \item[\CTAN{macros/latex/contrib/oberdiek/pdfcolparcolumns.dtx}] The source file.
% \item[\CTAN{macros/latex/contrib/oberdiek/pdfcolparcolumns.pdf}] Documentation.
% \end{description}
%
%
% \paragraph{Bundle.} All the packages of the bundle `oberdiek'
% are also available in a TDS compliant ZIP archive. There
% the packages are already unpacked and the documentation files
% are generated. The files and directories obey the TDS standard.
% \begin{description}
% \item[\CTAN{install/macros/latex/contrib/oberdiek.tds.zip}]
% \end{description}
% \emph{TDS} refers to the standard ``A Directory Structure
% for \TeX\ Files'' (\CTAN{tds/tds.pdf}). Directories
% with \xfile{texmf} in their name are usually organized this way.
%
% \subsection{Bundle installation}
%
% \paragraph{Unpacking.} Unpack the \xfile{oberdiek.tds.zip} in the
% TDS tree (also known as \xfile{texmf} tree) of your choice.
% Example (linux):
% \begin{quote}
%   |unzip oberdiek.tds.zip -d ~/texmf|
% \end{quote}
%
% \paragraph{Script installation.}
% Check the directory \xfile{TDS:scripts/oberdiek/} for
% scripts that need further installation steps.
% Package \xpackage{attachfile2} comes with the Perl script
% \xfile{pdfatfi.pl} that should be installed in such a way
% that it can be called as \texttt{pdfatfi}.
% Example (linux):
% \begin{quote}
%   |chmod +x scripts/oberdiek/pdfatfi.pl|\\
%   |cp scripts/oberdiek/pdfatfi.pl /usr/local/bin/|
% \end{quote}
%
% \subsection{Package installation}
%
% \paragraph{Unpacking.} The \xfile{.dtx} file is a self-extracting
% \docstrip\ archive. The files are extracted by running the
% \xfile{.dtx} through \plainTeX:
% \begin{quote}
%   \verb|tex pdfcolparcolumns.dtx|
% \end{quote}
%
% \paragraph{TDS.} Now the different files must be moved into
% the different directories in your installation TDS tree
% (also known as \xfile{texmf} tree):
% \begin{quote}
% \def\t{^^A
% \begin{tabular}{@{}>{\ttfamily}l@{ $\rightarrow$ }>{\ttfamily}l@{}}
%   pdfcolparcolumns.sty & tex/latex/oberdiek/pdfcolparcolumns.sty\\
%   pdfcolparcolumns.pdf & doc/latex/oberdiek/pdfcolparcolumns.pdf\\
%   test/pdfcolparcolumns-test1.tex & doc/latex/oberdiek/test/pdfcolparcolumns-test1.tex\\
%   pdfcolparcolumns.dtx & source/latex/oberdiek/pdfcolparcolumns.dtx\\
% \end{tabular}^^A
% }^^A
% \sbox0{\t}^^A
% \ifdim\wd0>\linewidth
%   \begingroup
%     \advance\linewidth by\leftmargin
%     \advance\linewidth by\rightmargin
%   \edef\x{\endgroup
%     \def\noexpand\lw{\the\linewidth}^^A
%   }\x
%   \def\lwbox{^^A
%     \leavevmode
%     \hbox to \linewidth{^^A
%       \kern-\leftmargin\relax
%       \hss
%       \usebox0
%       \hss
%       \kern-\rightmargin\relax
%     }^^A
%   }^^A
%   \ifdim\wd0>\lw
%     \sbox0{\small\t}^^A
%     \ifdim\wd0>\linewidth
%       \ifdim\wd0>\lw
%         \sbox0{\footnotesize\t}^^A
%         \ifdim\wd0>\linewidth
%           \ifdim\wd0>\lw
%             \sbox0{\scriptsize\t}^^A
%             \ifdim\wd0>\linewidth
%               \ifdim\wd0>\lw
%                 \sbox0{\tiny\t}^^A
%                 \ifdim\wd0>\linewidth
%                   \lwbox
%                 \else
%                   \usebox0
%                 \fi
%               \else
%                 \lwbox
%               \fi
%             \else
%               \usebox0
%             \fi
%           \else
%             \lwbox
%           \fi
%         \else
%           \usebox0
%         \fi
%       \else
%         \lwbox
%       \fi
%     \else
%       \usebox0
%     \fi
%   \else
%     \lwbox
%   \fi
% \else
%   \usebox0
% \fi
% \end{quote}
% If you have a \xfile{docstrip.cfg} that configures and enables \docstrip's
% TDS installing feature, then some files can already be in the right
% place, see the documentation of \docstrip.
%
% \subsection{Refresh file name databases}
%
% If your \TeX~distribution
% (\teTeX, \mikTeX, \dots) relies on file name databases, you must refresh
% these. For example, \teTeX\ users run \verb|texhash| or
% \verb|mktexlsr|.
%
% \subsection{Some details for the interested}
%
% \paragraph{Attached source.}
%
% The PDF documentation on CTAN also includes the
% \xfile{.dtx} source file. It can be extracted by
% AcrobatReader 6 or higher. Another option is \textsf{pdftk},
% e.g. unpack the file into the current directory:
% \begin{quote}
%   \verb|pdftk pdfcolparcolumns.pdf unpack_files output .|
% \end{quote}
%
% \paragraph{Unpacking with \LaTeX.}
% The \xfile{.dtx} chooses its action depending on the format:
% \begin{description}
% \item[\plainTeX:] Run \docstrip\ and extract the files.
% \item[\LaTeX:] Generate the documentation.
% \end{description}
% If you insist on using \LaTeX\ for \docstrip\ (really,
% \docstrip\ does not need \LaTeX), then inform the autodetect routine
% about your intention:
% \begin{quote}
%   \verb|latex \let\install=y\input{pdfcolparcolumns.dtx}|
% \end{quote}
% Do not forget to quote the argument according to the demands
% of your shell.
%
% \paragraph{Generating the documentation.}
% You can use both the \xfile{.dtx} or the \xfile{.drv} to generate
% the documentation. The process can be configured by the
% configuration file \xfile{ltxdoc.cfg}. For instance, put this
% line into this file, if you want to have A4 as paper format:
% \begin{quote}
%   \verb|\PassOptionsToClass{a4paper}{article}|
% \end{quote}
% An example follows how to generate the
% documentation with pdf\LaTeX:
% \begin{quote}
%\begin{verbatim}
%pdflatex pdfcolparcolumns.dtx
%makeindex -s gind.ist pdfcolparcolumns.idx
%pdflatex pdfcolparcolumns.dtx
%makeindex -s gind.ist pdfcolparcolumns.idx
%pdflatex pdfcolparcolumns.dtx
%\end{verbatim}
% \end{quote}
%
% \section{Catalogue}
%
% The following XML file can be used as source for the
% \href{http://mirror.ctan.org/help/Catalogue/catalogue.html}{\TeX\ Catalogue}.
% The elements \texttt{caption} and \texttt{description} are imported
% from the original XML file from the Catalogue.
% The name of the XML file in the Catalogue is \xfile{pdfcolparcolumns.xml}.
%    \begin{macrocode}
%<*catalogue>
<?xml version='1.0' encoding='us-ascii'?>
<!DOCTYPE entry SYSTEM 'catalogue.dtd'>
<entry datestamp='$Date$' modifier='$Author$' id='pdfcolparcolumns'>
  <name>pdfcolparcolumns</name>
  <caption>Fix colour problems in package 'parcolumns'.</caption>
  <authorref id='auth:oberdiek'/>
  <copyright owner='Heiko Oberdiek' year='2007,2008,2010'/>
  <license type='lppl1.3'/>
  <version number='1.4'/>
  <description>
    Since version 1.40 pdfTeX supports colour stacks.
    This package uses them to fix colour problems in
    package <xref refid='parcolumns'>parcolumns</xref>.
    <p/>
    The package is part of the <xref refid='oberdiek'>oberdiek</xref>
    bundle.
  </description>
  <documentation details='Package documentation'
      href='ctan:/macros/latex/contrib/oberdiek/pdfcolparcolumns.pdf'/>
  <ctan file='true' path='/macros/latex/contrib/oberdiek/pdfcolparcolumns.dtx'/>
  <miktex location='oberdiek'/>
  <texlive location='oberdiek'/>
  <install path='/macros/latex/contrib/oberdiek/oberdiek.tds.zip'/>
</entry>
%</catalogue>
%    \end{macrocode}
%
% \begin{thebibliography}{9}
%
% \bibitem{parcolumns}
%   Jonathan Sauer: \textit{The \xpackage{parcolumns} package};
%   2004/11/25;\\
%   \CTAN{macros/latex/contrib/sauerj/parcolumns.pdf}.
%
% \bibitem{pdfcol}
%   Heiko Oberdiek: \textit{The \xpackage{pdfcol} package};
%   2007/09/09;\\
%   \CTAN{macros/latex/contrib/oberdiek/pdfcol.pdf}.
%
% \end{thebibliography}
%
% \begin{History}
%   \begin{Version}{2007/07/26 v1.0}
%   \item
%     First version, published in the newsgroup \xnewsgroup{comp.text.tex}
%     with the name \xpackage{parcolumns-colorstacks}: ^^A no line break
%     \URL{``\link{Re: \xpackage{xcolor} glitches}''}^^A
%     {http://groups.google.com/group/comp.text.tex/msg/56bd897b11bca414}
%   \end{Version}
%   \begin{Version}{2007/09/09 v1.1}
%   \item
%     CTAN version, package name renamed to \xpackage{pdfcolparcolumns}.
%   \item
%     Uses package \xpackage{pdfcol}.
%   \item
%     Documentation added.
%   \item
%     Test file added.
%   \end{Version}
%   \begin{Version}{2008/08/11 v1.2}
%   \item
%     Code is not changed.
%   \item
%     URLs updated.
%   \end{Version}
%   \begin{Version}{2010/01/11 v1.3}
%   \item
%     Fix for rule color.
%   \item
%     New option \xoption{rulebetweencolor} for environment |parcolumns|.
%   \end{Version}
%   \begin{Version}{2016/05/16 v1.4}
%   \item
%     Documentation updates.
%   \end{Version}
% \end{History}
%
% \PrintIndex
%
% \Finale
\endinput

%        (quote the arguments according to the demands of your shell)
%
% Documentation:
%    (a) If pdfcolparcolumns.drv is present:
%           latex pdfcolparcolumns.drv
%    (b) Without pdfcolparcolumns.drv:
%           latex pdfcolparcolumns.dtx; ...
%    The class ltxdoc loads the configuration file ltxdoc.cfg
%    if available. Here you can specify further options, e.g.
%    use A4 as paper format:
%       \PassOptionsToClass{a4paper}{article}
%
%    Programm calls to get the documentation (example):
%       pdflatex pdfcolparcolumns.dtx
%       makeindex -s gind.ist pdfcolparcolumns.idx
%       pdflatex pdfcolparcolumns.dtx
%       makeindex -s gind.ist pdfcolparcolumns.idx
%       pdflatex pdfcolparcolumns.dtx
%
% Installation:
%    TDS:tex/latex/oberdiek/pdfcolparcolumns.sty
%    TDS:doc/latex/oberdiek/pdfcolparcolumns.pdf
%    TDS:doc/latex/oberdiek/test/pdfcolparcolumns-test1.tex
%    TDS:source/latex/oberdiek/pdfcolparcolumns.dtx
%
%<*ignore>
\begingroup
  \catcode123=1 %
  \catcode125=2 %
  \def\x{LaTeX2e}%
\expandafter\endgroup
\ifcase 0\ifx\install y1\fi\expandafter
         \ifx\csname processbatchFile\endcsname\relax\else1\fi
         \ifx\fmtname\x\else 1\fi\relax
\else\csname fi\endcsname
%</ignore>
%<*install>
\input docstrip.tex
\Msg{************************************************************************}
\Msg{* Installation}
\Msg{* Package: pdfcolparcolumns 2016/05/16 v1.4 Color stacks for parcolumns (HO)}
\Msg{************************************************************************}

\keepsilent
\askforoverwritefalse

\let\MetaPrefix\relax
\preamble

This is a generated file.

Project: pdfcolparcolumns
Version: 2016/05/16 v1.4

Copyright (C) 2007, 2008, 2010 by
   Heiko Oberdiek <heiko.oberdiek at googlemail.com>

This work may be distributed and/or modified under the
conditions of the LaTeX Project Public License, either
version 1.3c of this license or (at your option) any later
version. This version of this license is in
   http://www.latex-project.org/lppl/lppl-1-3c.txt
and the latest version of this license is in
   http://www.latex-project.org/lppl.txt
and version 1.3 or later is part of all distributions of
LaTeX version 2005/12/01 or later.

This work has the LPPL maintenance status "maintained".

This Current Maintainer of this work is Heiko Oberdiek.

This work consists of the main source file pdfcolparcolumns.dtx
and the derived files
   pdfcolparcolumns.sty, pdfcolparcolumns.pdf, pdfcolparcolumns.ins,
   pdfcolparcolumns.drv, pdfcolparcolumns-test1.tex.

\endpreamble
\let\MetaPrefix\DoubleperCent

\generate{%
  \file{pdfcolparcolumns.ins}{\from{pdfcolparcolumns.dtx}{install}}%
  \file{pdfcolparcolumns.drv}{\from{pdfcolparcolumns.dtx}{driver}}%
  \usedir{tex/latex/oberdiek}%
  \file{pdfcolparcolumns.sty}{\from{pdfcolparcolumns.dtx}{package}}%
  \usedir{doc/latex/oberdiek/test}%
  \file{pdfcolparcolumns-test1.tex}{\from{pdfcolparcolumns.dtx}{test1}}%
  \nopreamble
  \nopostamble
  \usedir{source/latex/oberdiek/catalogue}%
  \file{pdfcolparcolumns.xml}{\from{pdfcolparcolumns.dtx}{catalogue}}%
}

\catcode32=13\relax% active space
\let =\space%
\Msg{************************************************************************}
\Msg{*}
\Msg{* To finish the installation you have to move the following}
\Msg{* file into a directory searched by TeX:}
\Msg{*}
\Msg{*     pdfcolparcolumns.sty}
\Msg{*}
\Msg{* To produce the documentation run the file `pdfcolparcolumns.drv'}
\Msg{* through LaTeX.}
\Msg{*}
\Msg{* Happy TeXing!}
\Msg{*}
\Msg{************************************************************************}

\endbatchfile
%</install>
%<*ignore>
\fi
%</ignore>
%<*driver>
\NeedsTeXFormat{LaTeX2e}
\ProvidesFile{pdfcolparcolumns.drv}%
  [2016/05/16 v1.4 Color stacks for parcolumns (HO)]%
\documentclass{ltxdoc}
\usepackage{holtxdoc}[2011/11/22]
\begin{document}
  \DocInput{pdfcolparcolumns.dtx}%
\end{document}
%</driver>
% \fi
%
%
% \CharacterTable
%  {Upper-case    \A\B\C\D\E\F\G\H\I\J\K\L\M\N\O\P\Q\R\S\T\U\V\W\X\Y\Z
%   Lower-case    \a\b\c\d\e\f\g\h\i\j\k\l\m\n\o\p\q\r\s\t\u\v\w\x\y\z
%   Digits        \0\1\2\3\4\5\6\7\8\9
%   Exclamation   \!     Double quote  \"     Hash (number) \#
%   Dollar        \$     Percent       \%     Ampersand     \&
%   Acute accent  \'     Left paren    \(     Right paren   \)
%   Asterisk      \*     Plus          \+     Comma         \,
%   Minus         \-     Point         \.     Solidus       \/
%   Colon         \:     Semicolon     \;     Less than     \<
%   Equals        \=     Greater than  \>     Question mark \?
%   Commercial at \@     Left bracket  \[     Backslash     \\
%   Right bracket \]     Circumflex    \^     Underscore    \_
%   Grave accent  \`     Left brace    \{     Vertical bar  \|
%   Right brace   \}     Tilde         \~}
%
% \GetFileInfo{pdfcolparcolumns.drv}
%
% \title{The \xpackage{pdfcolparcolumns} package}
% \date{2016/05/16 v1.4}
% \author{Heiko Oberdiek\thanks
% {Please report any issues at https://github.com/ho-tex/oberdiek/issues}\\
% \xemail{heiko.oberdiek at googlemail.com}}
%
% \maketitle
%
% \begin{abstract}
% Since version 1.40 \pdfTeX\ supports several color stacks.
% This package uses them to fix color problems in
% package \xpackage{parcolumns}.
% \end{abstract}
%
% \tableofcontents
%
% \section{Usage}
%
% \begin{quote}
% |\usepackage{pdfcolparcolumns}|
% \end{quote}
% The package \xpackage{pdfcolparcolumns} loads package \xpackage{parcolums}
% \cite{parcolumns}. If color stacks are available then the
% macros of \xpackage{parcolumns} are patched to add support
% for color stacks.
%
% \subsection{Option \xoption{rulebetweencolor}}
%
% Package \xpackage{pdfcolparcolumns} also fixes the color for the
% rule between columns (if \xoption{rulebetween} is set).
% Default color is \cs{normalcolor}. But this can be changed by using
% option \xoption{rulebetweencolor}. It takes a color specification
% as value. If the value is empty, then the default (\cs{normalcolor})
% is used.
% Examples:
% \begin{quote}
%   |rulebetweencolor=blue|,\\
%   |rulebetweencolor={red}|,\\
%   |rulebetweencolor={}|, \textit{\% \cs{normalcolor} is used}\\
%   |rulebetweencolor=[rgb]{1,0,.5}| \textit{\% see below}
% \end{quote}
% If used inside the optional argument of environment \textsf{parcolumns}
% and the value contains an optional argument, then whole value
% must be put in curly braces:
%\begin{quote}
%\begin{verbatim}
%\begin{parcolumns}[
%  rulebetween,
%  rulebetweencolor={[rgb]{1,0,.5}},
%]{2}
%  ...
%\end{parcolumns}
%\end{verbatim}
%\end{quote}
% This option \xoption{rulebetweencolor} can also be set using
% \cs{setkeys}:
%\begin{quote}
%\begin{verbatim}
%\setkeys{parcolumns}{rulebetweencolor=green}
%\end{verbatim}
%\end{quote}
%
% \subsection{Future}
%
% Currently package \xpackage{parcolumns} does not seem to be
% maintained. Nevertheless if there will be a new version that
% adds support for color stacks, then this package may become
% obsolete.
%
% \StopEventually{
% }
%
% \section{Implementation}
%
% \subsection{Identification}
%
%    \begin{macrocode}
%<*package>
\NeedsTeXFormat{LaTeX2e}
\ProvidesPackage{pdfcolparcolumns}%
  [2016/05/16 v1.4 Color stacks for parcolumns (HO)]%
%    \end{macrocode}
%
% \subsection{Load packages}
%
% \subsubsection{Package \xpackage{parcolumns}}
%
%    Currently package \xpackage{parcolumns} does not define options.
%    Thus it is just a precaution that the options of
%    package \xpackage{pdfcolparcolumns} are passed to
%    package \xpackage{parcolumns}.
%    \begin{macrocode}
\DeclareOption*{%
  \PassoptionsToPackage{\CurrentOption}{parcolumns}%
}
\ProcessOptions\relax
\RequirePackage{parcolumns}[2004/11/25]
%    \end{macrocode}
%
% \subsubsection{Package \xpackage{pdfcol}}
%
%    \begin{macrocode}
\RequirePackage{pdfcol}[2007/09/09]
\ifpdfcolAvailable
\else
  \PackageInfo{pdfcolparcolumns}{%
    Loading aborted, because color stacks are not available%
  }%
  \expandafter\endinput
\fi
%    \end{macrocode}
%
% \subsubsection{Package \xpackage{infwarerr}}
%
%    \begin{macrocode}
\RequirePackage{infwarerr}[2007/09/09]
%    \end{macrocode}
%
% \subsection{Color stack macros}
%
%    \begin{macro}{\pcpc@MaxStack}
%    Macro \cs{pcpc@MaxStack} holds the highest number of
%    allocated stacks.
%    \begin{macrocode}
\global\chardef\pcpc@MaxStack=\z@
%    \end{macrocode}
%    \end{macro}
%    \begin{macro}{\pcpc@InitStacks}
%    Macro \cs{pcpc@InitStacks} takes the number of columns
%    as argument and ensures that there are enough color
%    stacks for all columns.
%    \begin{macrocode}
\def\pcpc@InitStacks#1{%
  \ifnum#1>\pcpc@MaxStack
    \begingroup
      \count@\pcpc@MaxStack
      \loop
        \advance\count@\@ne
        \pdfcolInitStack{pcpc@\the\count@}%
      \ifnum#1>\count@
      \repeat
      \global\chardef\pcpc@MaxStack=\count@
    \endgroup
  \fi
}
%    \end{macrocode}
%    \end{macro}
%
%    \begin{macro}{\pcpc@SwitchStack}
%    \begin{macrocode}
\def\pcpc@SwitchStack#1{%
  \pdfcolSwitchStack{pcpc@\number#1}%
}
%    \end{macrocode}
%    \end{macro}
%
%    \begin{macro}{\pcpc@SetCurrent}
%    \begin{macrocode}
\def\pcpc@SetCurrent#1{%
  \pdfcolSetCurrent{pcpc@\number#1}%
}
%    \end{macrocode}
%    \end{macro}
%
% \subsection{Patches}
%
%     Now the color stack macros are patched into the macros
%     of package \xpackage{parcolumns}.
%
% \subsubsection{Init stacks}
%
%    \cs{pcpc@InitStacks} should go into the definition of
%    environment |parcolumns|. \cs{pc@alloccolumns} is executed
%    there and nowhere else, thus we hook into it.
%    \begin{macrocode}
\g@addto@macro\pc@alloccolumns{%
  \pcpc@InitStacks\pc@columncount
}
%    \end{macrocode}
%
% \subsubsection{Switch stack}
%
%    \cs{pcpc@SwitchStack} should be called by marco \cs{colchunk@}.
%    However it is easier to patch \cs{pc@setcolumnwidth} that
%    is executed in \cs{colchunk@} only.
%    \begin{macrocode}
\g@addto@macro\pc@setcolumnwidth{%
  \pcpc@SwitchStack\pc@columnctr
}
%    \end{macrocode}
%
% \subsubsection{Set current stack color}
%
%    \cs{pcpc@SetCurrent} is set at the begin of each line.
%    It must be inserted into \cs{pc@placeboxes}. Unhappily
%    there is no easy way. Therefore we check and
%    redefine \cs{pc@placeboxes}.
%    \begin{macrocode}
\begingroup
  \def\x{%
    \global\let\@tempa\relax
    \count@\z@
    \hb@xt@\linewidth{%
      \vfuzz30ex %
      \vbadness\@M
      \splittopskip\z@skip
      \loop
      \ifnum\count@<\pc@columncount
        \advance\count@\@ne
        \expandafter\ifvoid\csname pc@column@\number\count@\endcsname
          \hskip\csname pc@column@width@\number\count@\endcsname
        \else
          \expandafter\setbox\expandafter\@tempboxa\expandafter
          \vsplit\csname pc@column@\number\count@\endcsname
              to \dp\strutbox
          \vbox{%
            \unvbox\@tempboxa
          }%
        \fi
        \expandafter\ifvoid\csname pc@column@\number\count@\endcsname
        \else
          \global\let\@tempa\pc@placeboxes
        \fi
        \ifnum\count@<\pc@columncount
          \strut
          \hfill
          \ifpc@rulebetween
            \vrule
            \hfill
          \fi
        \fi
      \repeat
    }%
    \@tempa
  }%
  \ifx\x\pc@placeboxes
  \else
    \@PackageWarningNoLine{pdfcolparcolumns}{%
      Command \string\pc@placeboxes\space has changed.\MessageBreak
      Supported versions of package `parcolumns':\MessageBreak
      \space\space 2004/08/05.\MessageBreak
      The redefinition of \string\pc@placeboxes\space may not%
      \MessageBreak
      behave correctly depending on the changes%
    }%
  \fi
\endgroup
%    \end{macrocode}
%    \begin{macro}{\pc@placeboxes}
%    \begin{macrocode}
\renewcommand*{\pc@placeboxes}{%
  \global\let\@tempa\relax
  \count@\z@
  \hb@xt@\linewidth{%
    \vfuzz30ex %
    \vbadness\@M
    \splittopskip\z@skip
    \loop
    \ifnum\count@<\pc@columncount
      \advance\count@\@ne
      \expandafter\ifvoid\csname pc@column@\number\count@\endcsname
        \hskip\csname pc@column@width@\number\count@\endcsname
      \else
        \expandafter\setbox\expandafter\@tempboxa\expandafter
        \vsplit\csname pc@column@\number\count@\endcsname
            to \dp\strutbox
        \vbox{%
          \pcpc@SetCurrent\count@
          \unvbox\@tempboxa
        }%
      \fi
      \expandafter\ifvoid\csname pc@column@\number\count@\endcsname
      \else
        \global\let\@tempa\pc@placeboxes
      \fi
      \ifnum\count@<\pc@columncount
        \strut
        \hfill
        \ifpc@rulebetween
          \begingroup
            \pcpc@RuleBetweenColor
            \vrule
          \endgroup
          \hfill
        \fi
      \fi
    \repeat
  }%
  \@tempa
}
%    \end{macrocode}
%    \end{macro}
%    \begin{macro}{\pcpc@RuleBetweenColorDefault}
%    \begin{macrocode}
\def\pcpc@RuleBetweenColorDefault{%
  \normalcolor
}
%    \end{macrocode}
%    \end{macro}
%    \begin{macro}{\pcpc@RuleBetweenColor}
%    \begin{macrocode}
\let\pcpc@RuleBetweenColor\pcpc@RuleBetweenColorDefault
%    \end{macrocode}
%    \end{macro}
%    \begin{macrocode}
\define@key{parcolumns}{rulebetweencolor}{%
  \edef\pcpc@temp{#1}%
  \ifx\pcpc@temp\@empty
    \let\pcpc@RuleBetweenColor\pcpc@RuleBetweenColorDefault
  \else
    \edef\pcpc@temp{%
      \noexpand\@ifnextchar[{%
        \def\noexpand\pcpc@RuleBetweenColor{%
          \noexpand\color\pcpc@temp
        }%
        \noexpand\pcpc@GobbleNil
      }{%
        \def\noexpand\pcpc@RuleBetweenColor{%
          \noexpand\color{\pcpc@temp}%
        }%
        \noexpand\pcpc@GobbleNil
      }%
      \pcpc@temp\noexpand\@nil
    }%
    \pcpc@temp
  \fi
}
%    \end{macrocode}
%    \begin{macro}{\pcpc@GobbleNil}
%    \begin{macrocode}
\long\def\pcpc@GobbleNil#1\@nil{}
%    \end{macrocode}
%    \end{macro}
%
%    \begin{macrocode}
%</package>
%    \end{macrocode}
%
% \section{Test}
%
%    The test file is a modified version of the file that
%    Donald Goodman has posted in \xnewsgroup{comp.text.tex}: ^^A
%    \URL{``\link{Re: \xpackage{xcolor} glitches}''}^^A
%    {http://groups.google.com/group/comp.text.tex/msg/8eda74ed292012bb}
%    \begin{macrocode}
%<*test1>
\NeedsTeXFormat{LaTeX2e}
\AtEndDocument{%
  \typeout{}%
  \typeout{**************************************}%
  \typeout{*** \space Check the PDF file manually! \space ***}%
  \typeout{**************************************}%
  \typeout{}%
}
\documentclass{article}
\usepackage{xcolor}
\usepackage{pdfcolparcolumns}

\newcommand{\instruct}[1]{%
  \noindent
  \footnotesize
  \textcolor{red}{#1}%
}

\begin{document}
  \begin{parcolumns}[colwidths={1=2.3in,2=2.3in},sloppy]{2}%
    \colchunk[1]{%
      \instruct{Et non dicitur versus} %
      Fidelium anim\ae\ %
      \instruct{%
        sed immediate subiungitur antiphona finalis %
        beat\ae\ Mari\ae\ Virginis%
      } %
      100.%
    }%
    \colchunk[2]{%
      \instruct{%
        And the verse %
        \textcolor{black}{May the souls of the faithful} %
        is not said, but the final antiphon of the %
        Blessed Virgin Mary, %
        \textcolor{black}{100,} %
        is immediately joined.%
      }%
    }%
  \end{parcolumns}%
\end{document}
%</test1>
%    \end{macrocode}
%
% \section{Installation}
%
% \subsection{Download}
%
% \paragraph{Package.} This package is available on
% CTAN\footnote{\url{http://ctan.org/pkg/pdfcolparcolumns}}:
% \begin{description}
% \item[\CTAN{macros/latex/contrib/oberdiek/pdfcolparcolumns.dtx}] The source file.
% \item[\CTAN{macros/latex/contrib/oberdiek/pdfcolparcolumns.pdf}] Documentation.
% \end{description}
%
%
% \paragraph{Bundle.} All the packages of the bundle `oberdiek'
% are also available in a TDS compliant ZIP archive. There
% the packages are already unpacked and the documentation files
% are generated. The files and directories obey the TDS standard.
% \begin{description}
% \item[\CTAN{install/macros/latex/contrib/oberdiek.tds.zip}]
% \end{description}
% \emph{TDS} refers to the standard ``A Directory Structure
% for \TeX\ Files'' (\CTAN{tds/tds.pdf}). Directories
% with \xfile{texmf} in their name are usually organized this way.
%
% \subsection{Bundle installation}
%
% \paragraph{Unpacking.} Unpack the \xfile{oberdiek.tds.zip} in the
% TDS tree (also known as \xfile{texmf} tree) of your choice.
% Example (linux):
% \begin{quote}
%   |unzip oberdiek.tds.zip -d ~/texmf|
% \end{quote}
%
% \paragraph{Script installation.}
% Check the directory \xfile{TDS:scripts/oberdiek/} for
% scripts that need further installation steps.
% Package \xpackage{attachfile2} comes with the Perl script
% \xfile{pdfatfi.pl} that should be installed in such a way
% that it can be called as \texttt{pdfatfi}.
% Example (linux):
% \begin{quote}
%   |chmod +x scripts/oberdiek/pdfatfi.pl|\\
%   |cp scripts/oberdiek/pdfatfi.pl /usr/local/bin/|
% \end{quote}
%
% \subsection{Package installation}
%
% \paragraph{Unpacking.} The \xfile{.dtx} file is a self-extracting
% \docstrip\ archive. The files are extracted by running the
% \xfile{.dtx} through \plainTeX:
% \begin{quote}
%   \verb|tex pdfcolparcolumns.dtx|
% \end{quote}
%
% \paragraph{TDS.} Now the different files must be moved into
% the different directories in your installation TDS tree
% (also known as \xfile{texmf} tree):
% \begin{quote}
% \def\t{^^A
% \begin{tabular}{@{}>{\ttfamily}l@{ $\rightarrow$ }>{\ttfamily}l@{}}
%   pdfcolparcolumns.sty & tex/latex/oberdiek/pdfcolparcolumns.sty\\
%   pdfcolparcolumns.pdf & doc/latex/oberdiek/pdfcolparcolumns.pdf\\
%   test/pdfcolparcolumns-test1.tex & doc/latex/oberdiek/test/pdfcolparcolumns-test1.tex\\
%   pdfcolparcolumns.dtx & source/latex/oberdiek/pdfcolparcolumns.dtx\\
% \end{tabular}^^A
% }^^A
% \sbox0{\t}^^A
% \ifdim\wd0>\linewidth
%   \begingroup
%     \advance\linewidth by\leftmargin
%     \advance\linewidth by\rightmargin
%   \edef\x{\endgroup
%     \def\noexpand\lw{\the\linewidth}^^A
%   }\x
%   \def\lwbox{^^A
%     \leavevmode
%     \hbox to \linewidth{^^A
%       \kern-\leftmargin\relax
%       \hss
%       \usebox0
%       \hss
%       \kern-\rightmargin\relax
%     }^^A
%   }^^A
%   \ifdim\wd0>\lw
%     \sbox0{\small\t}^^A
%     \ifdim\wd0>\linewidth
%       \ifdim\wd0>\lw
%         \sbox0{\footnotesize\t}^^A
%         \ifdim\wd0>\linewidth
%           \ifdim\wd0>\lw
%             \sbox0{\scriptsize\t}^^A
%             \ifdim\wd0>\linewidth
%               \ifdim\wd0>\lw
%                 \sbox0{\tiny\t}^^A
%                 \ifdim\wd0>\linewidth
%                   \lwbox
%                 \else
%                   \usebox0
%                 \fi
%               \else
%                 \lwbox
%               \fi
%             \else
%               \usebox0
%             \fi
%           \else
%             \lwbox
%           \fi
%         \else
%           \usebox0
%         \fi
%       \else
%         \lwbox
%       \fi
%     \else
%       \usebox0
%     \fi
%   \else
%     \lwbox
%   \fi
% \else
%   \usebox0
% \fi
% \end{quote}
% If you have a \xfile{docstrip.cfg} that configures and enables \docstrip's
% TDS installing feature, then some files can already be in the right
% place, see the documentation of \docstrip.
%
% \subsection{Refresh file name databases}
%
% If your \TeX~distribution
% (\teTeX, \mikTeX, \dots) relies on file name databases, you must refresh
% these. For example, \teTeX\ users run \verb|texhash| or
% \verb|mktexlsr|.
%
% \subsection{Some details for the interested}
%
% \paragraph{Attached source.}
%
% The PDF documentation on CTAN also includes the
% \xfile{.dtx} source file. It can be extracted by
% AcrobatReader 6 or higher. Another option is \textsf{pdftk},
% e.g. unpack the file into the current directory:
% \begin{quote}
%   \verb|pdftk pdfcolparcolumns.pdf unpack_files output .|
% \end{quote}
%
% \paragraph{Unpacking with \LaTeX.}
% The \xfile{.dtx} chooses its action depending on the format:
% \begin{description}
% \item[\plainTeX:] Run \docstrip\ and extract the files.
% \item[\LaTeX:] Generate the documentation.
% \end{description}
% If you insist on using \LaTeX\ for \docstrip\ (really,
% \docstrip\ does not need \LaTeX), then inform the autodetect routine
% about your intention:
% \begin{quote}
%   \verb|latex \let\install=y% \iffalse meta-comment
%
% File: pdfcolparcolumns.dtx
% Version: 2016/05/16 v1.4
% Info: Color stacks for parcolumns
%
% Copyright (C) 2007, 2008, 2010 by
%    Heiko Oberdiek <heiko.oberdiek at googlemail.com>
%    2016
%    https://github.com/ho-tex/oberdiek/issues
%
% This work may be distributed and/or modified under the
% conditions of the LaTeX Project Public License, either
% version 1.3c of this license or (at your option) any later
% version. This version of this license is in
%    http://www.latex-project.org/lppl/lppl-1-3c.txt
% and the latest version of this license is in
%    http://www.latex-project.org/lppl.txt
% and version 1.3 or later is part of all distributions of
% LaTeX version 2005/12/01 or later.
%
% This work has the LPPL maintenance status "maintained".
%
% This Current Maintainer of this work is Heiko Oberdiek.
%
% This work consists of the main source file pdfcolparcolumns.dtx
% and the derived files
%    pdfcolparcolumns.sty, pdfcolparcolumns.pdf, pdfcolparcolumns.ins,
%    pdfcolparcolumns.drv, pdfcolparcolumns-test1.tex.
%
% Distribution:
%    CTAN:macros/latex/contrib/oberdiek/pdfcolparcolumns.dtx
%    CTAN:macros/latex/contrib/oberdiek/pdfcolparcolumns.pdf
%
% Unpacking:
%    (a) If pdfcolparcolumns.ins is present:
%           tex pdfcolparcolumns.ins
%    (b) Without pdfcolparcolumns.ins:
%           tex pdfcolparcolumns.dtx
%    (c) If you insist on using LaTeX
%           latex \let\install=y\input{pdfcolparcolumns.dtx}
%        (quote the arguments according to the demands of your shell)
%
% Documentation:
%    (a) If pdfcolparcolumns.drv is present:
%           latex pdfcolparcolumns.drv
%    (b) Without pdfcolparcolumns.drv:
%           latex pdfcolparcolumns.dtx; ...
%    The class ltxdoc loads the configuration file ltxdoc.cfg
%    if available. Here you can specify further options, e.g.
%    use A4 as paper format:
%       \PassOptionsToClass{a4paper}{article}
%
%    Programm calls to get the documentation (example):
%       pdflatex pdfcolparcolumns.dtx
%       makeindex -s gind.ist pdfcolparcolumns.idx
%       pdflatex pdfcolparcolumns.dtx
%       makeindex -s gind.ist pdfcolparcolumns.idx
%       pdflatex pdfcolparcolumns.dtx
%
% Installation:
%    TDS:tex/latex/oberdiek/pdfcolparcolumns.sty
%    TDS:doc/latex/oberdiek/pdfcolparcolumns.pdf
%    TDS:doc/latex/oberdiek/test/pdfcolparcolumns-test1.tex
%    TDS:source/latex/oberdiek/pdfcolparcolumns.dtx
%
%<*ignore>
\begingroup
  \catcode123=1 %
  \catcode125=2 %
  \def\x{LaTeX2e}%
\expandafter\endgroup
\ifcase 0\ifx\install y1\fi\expandafter
         \ifx\csname processbatchFile\endcsname\relax\else1\fi
         \ifx\fmtname\x\else 1\fi\relax
\else\csname fi\endcsname
%</ignore>
%<*install>
\input docstrip.tex
\Msg{************************************************************************}
\Msg{* Installation}
\Msg{* Package: pdfcolparcolumns 2016/05/16 v1.4 Color stacks for parcolumns (HO)}
\Msg{************************************************************************}

\keepsilent
\askforoverwritefalse

\let\MetaPrefix\relax
\preamble

This is a generated file.

Project: pdfcolparcolumns
Version: 2016/05/16 v1.4

Copyright (C) 2007, 2008, 2010 by
   Heiko Oberdiek <heiko.oberdiek at googlemail.com>

This work may be distributed and/or modified under the
conditions of the LaTeX Project Public License, either
version 1.3c of this license or (at your option) any later
version. This version of this license is in
   http://www.latex-project.org/lppl/lppl-1-3c.txt
and the latest version of this license is in
   http://www.latex-project.org/lppl.txt
and version 1.3 or later is part of all distributions of
LaTeX version 2005/12/01 or later.

This work has the LPPL maintenance status "maintained".

This Current Maintainer of this work is Heiko Oberdiek.

This work consists of the main source file pdfcolparcolumns.dtx
and the derived files
   pdfcolparcolumns.sty, pdfcolparcolumns.pdf, pdfcolparcolumns.ins,
   pdfcolparcolumns.drv, pdfcolparcolumns-test1.tex.

\endpreamble
\let\MetaPrefix\DoubleperCent

\generate{%
  \file{pdfcolparcolumns.ins}{\from{pdfcolparcolumns.dtx}{install}}%
  \file{pdfcolparcolumns.drv}{\from{pdfcolparcolumns.dtx}{driver}}%
  \usedir{tex/latex/oberdiek}%
  \file{pdfcolparcolumns.sty}{\from{pdfcolparcolumns.dtx}{package}}%
  \usedir{doc/latex/oberdiek/test}%
  \file{pdfcolparcolumns-test1.tex}{\from{pdfcolparcolumns.dtx}{test1}}%
  \nopreamble
  \nopostamble
  \usedir{source/latex/oberdiek/catalogue}%
  \file{pdfcolparcolumns.xml}{\from{pdfcolparcolumns.dtx}{catalogue}}%
}

\catcode32=13\relax% active space
\let =\space%
\Msg{************************************************************************}
\Msg{*}
\Msg{* To finish the installation you have to move the following}
\Msg{* file into a directory searched by TeX:}
\Msg{*}
\Msg{*     pdfcolparcolumns.sty}
\Msg{*}
\Msg{* To produce the documentation run the file `pdfcolparcolumns.drv'}
\Msg{* through LaTeX.}
\Msg{*}
\Msg{* Happy TeXing!}
\Msg{*}
\Msg{************************************************************************}

\endbatchfile
%</install>
%<*ignore>
\fi
%</ignore>
%<*driver>
\NeedsTeXFormat{LaTeX2e}
\ProvidesFile{pdfcolparcolumns.drv}%
  [2016/05/16 v1.4 Color stacks for parcolumns (HO)]%
\documentclass{ltxdoc}
\usepackage{holtxdoc}[2011/11/22]
\begin{document}
  \DocInput{pdfcolparcolumns.dtx}%
\end{document}
%</driver>
% \fi
%
%
% \CharacterTable
%  {Upper-case    \A\B\C\D\E\F\G\H\I\J\K\L\M\N\O\P\Q\R\S\T\U\V\W\X\Y\Z
%   Lower-case    \a\b\c\d\e\f\g\h\i\j\k\l\m\n\o\p\q\r\s\t\u\v\w\x\y\z
%   Digits        \0\1\2\3\4\5\6\7\8\9
%   Exclamation   \!     Double quote  \"     Hash (number) \#
%   Dollar        \$     Percent       \%     Ampersand     \&
%   Acute accent  \'     Left paren    \(     Right paren   \)
%   Asterisk      \*     Plus          \+     Comma         \,
%   Minus         \-     Point         \.     Solidus       \/
%   Colon         \:     Semicolon     \;     Less than     \<
%   Equals        \=     Greater than  \>     Question mark \?
%   Commercial at \@     Left bracket  \[     Backslash     \\
%   Right bracket \]     Circumflex    \^     Underscore    \_
%   Grave accent  \`     Left brace    \{     Vertical bar  \|
%   Right brace   \}     Tilde         \~}
%
% \GetFileInfo{pdfcolparcolumns.drv}
%
% \title{The \xpackage{pdfcolparcolumns} package}
% \date{2016/05/16 v1.4}
% \author{Heiko Oberdiek\thanks
% {Please report any issues at https://github.com/ho-tex/oberdiek/issues}\\
% \xemail{heiko.oberdiek at googlemail.com}}
%
% \maketitle
%
% \begin{abstract}
% Since version 1.40 \pdfTeX\ supports several color stacks.
% This package uses them to fix color problems in
% package \xpackage{parcolumns}.
% \end{abstract}
%
% \tableofcontents
%
% \section{Usage}
%
% \begin{quote}
% |\usepackage{pdfcolparcolumns}|
% \end{quote}
% The package \xpackage{pdfcolparcolumns} loads package \xpackage{parcolums}
% \cite{parcolumns}. If color stacks are available then the
% macros of \xpackage{parcolumns} are patched to add support
% for color stacks.
%
% \subsection{Option \xoption{rulebetweencolor}}
%
% Package \xpackage{pdfcolparcolumns} also fixes the color for the
% rule between columns (if \xoption{rulebetween} is set).
% Default color is \cs{normalcolor}. But this can be changed by using
% option \xoption{rulebetweencolor}. It takes a color specification
% as value. If the value is empty, then the default (\cs{normalcolor})
% is used.
% Examples:
% \begin{quote}
%   |rulebetweencolor=blue|,\\
%   |rulebetweencolor={red}|,\\
%   |rulebetweencolor={}|, \textit{\% \cs{normalcolor} is used}\\
%   |rulebetweencolor=[rgb]{1,0,.5}| \textit{\% see below}
% \end{quote}
% If used inside the optional argument of environment \textsf{parcolumns}
% and the value contains an optional argument, then whole value
% must be put in curly braces:
%\begin{quote}
%\begin{verbatim}
%\begin{parcolumns}[
%  rulebetween,
%  rulebetweencolor={[rgb]{1,0,.5}},
%]{2}
%  ...
%\end{parcolumns}
%\end{verbatim}
%\end{quote}
% This option \xoption{rulebetweencolor} can also be set using
% \cs{setkeys}:
%\begin{quote}
%\begin{verbatim}
%\setkeys{parcolumns}{rulebetweencolor=green}
%\end{verbatim}
%\end{quote}
%
% \subsection{Future}
%
% Currently package \xpackage{parcolumns} does not seem to be
% maintained. Nevertheless if there will be a new version that
% adds support for color stacks, then this package may become
% obsolete.
%
% \StopEventually{
% }
%
% \section{Implementation}
%
% \subsection{Identification}
%
%    \begin{macrocode}
%<*package>
\NeedsTeXFormat{LaTeX2e}
\ProvidesPackage{pdfcolparcolumns}%
  [2016/05/16 v1.4 Color stacks for parcolumns (HO)]%
%    \end{macrocode}
%
% \subsection{Load packages}
%
% \subsubsection{Package \xpackage{parcolumns}}
%
%    Currently package \xpackage{parcolumns} does not define options.
%    Thus it is just a precaution that the options of
%    package \xpackage{pdfcolparcolumns} are passed to
%    package \xpackage{parcolumns}.
%    \begin{macrocode}
\DeclareOption*{%
  \PassoptionsToPackage{\CurrentOption}{parcolumns}%
}
\ProcessOptions\relax
\RequirePackage{parcolumns}[2004/11/25]
%    \end{macrocode}
%
% \subsubsection{Package \xpackage{pdfcol}}
%
%    \begin{macrocode}
\RequirePackage{pdfcol}[2007/09/09]
\ifpdfcolAvailable
\else
  \PackageInfo{pdfcolparcolumns}{%
    Loading aborted, because color stacks are not available%
  }%
  \expandafter\endinput
\fi
%    \end{macrocode}
%
% \subsubsection{Package \xpackage{infwarerr}}
%
%    \begin{macrocode}
\RequirePackage{infwarerr}[2007/09/09]
%    \end{macrocode}
%
% \subsection{Color stack macros}
%
%    \begin{macro}{\pcpc@MaxStack}
%    Macro \cs{pcpc@MaxStack} holds the highest number of
%    allocated stacks.
%    \begin{macrocode}
\global\chardef\pcpc@MaxStack=\z@
%    \end{macrocode}
%    \end{macro}
%    \begin{macro}{\pcpc@InitStacks}
%    Macro \cs{pcpc@InitStacks} takes the number of columns
%    as argument and ensures that there are enough color
%    stacks for all columns.
%    \begin{macrocode}
\def\pcpc@InitStacks#1{%
  \ifnum#1>\pcpc@MaxStack
    \begingroup
      \count@\pcpc@MaxStack
      \loop
        \advance\count@\@ne
        \pdfcolInitStack{pcpc@\the\count@}%
      \ifnum#1>\count@
      \repeat
      \global\chardef\pcpc@MaxStack=\count@
    \endgroup
  \fi
}
%    \end{macrocode}
%    \end{macro}
%
%    \begin{macro}{\pcpc@SwitchStack}
%    \begin{macrocode}
\def\pcpc@SwitchStack#1{%
  \pdfcolSwitchStack{pcpc@\number#1}%
}
%    \end{macrocode}
%    \end{macro}
%
%    \begin{macro}{\pcpc@SetCurrent}
%    \begin{macrocode}
\def\pcpc@SetCurrent#1{%
  \pdfcolSetCurrent{pcpc@\number#1}%
}
%    \end{macrocode}
%    \end{macro}
%
% \subsection{Patches}
%
%     Now the color stack macros are patched into the macros
%     of package \xpackage{parcolumns}.
%
% \subsubsection{Init stacks}
%
%    \cs{pcpc@InitStacks} should go into the definition of
%    environment |parcolumns|. \cs{pc@alloccolumns} is executed
%    there and nowhere else, thus we hook into it.
%    \begin{macrocode}
\g@addto@macro\pc@alloccolumns{%
  \pcpc@InitStacks\pc@columncount
}
%    \end{macrocode}
%
% \subsubsection{Switch stack}
%
%    \cs{pcpc@SwitchStack} should be called by marco \cs{colchunk@}.
%    However it is easier to patch \cs{pc@setcolumnwidth} that
%    is executed in \cs{colchunk@} only.
%    \begin{macrocode}
\g@addto@macro\pc@setcolumnwidth{%
  \pcpc@SwitchStack\pc@columnctr
}
%    \end{macrocode}
%
% \subsubsection{Set current stack color}
%
%    \cs{pcpc@SetCurrent} is set at the begin of each line.
%    It must be inserted into \cs{pc@placeboxes}. Unhappily
%    there is no easy way. Therefore we check and
%    redefine \cs{pc@placeboxes}.
%    \begin{macrocode}
\begingroup
  \def\x{%
    \global\let\@tempa\relax
    \count@\z@
    \hb@xt@\linewidth{%
      \vfuzz30ex %
      \vbadness\@M
      \splittopskip\z@skip
      \loop
      \ifnum\count@<\pc@columncount
        \advance\count@\@ne
        \expandafter\ifvoid\csname pc@column@\number\count@\endcsname
          \hskip\csname pc@column@width@\number\count@\endcsname
        \else
          \expandafter\setbox\expandafter\@tempboxa\expandafter
          \vsplit\csname pc@column@\number\count@\endcsname
              to \dp\strutbox
          \vbox{%
            \unvbox\@tempboxa
          }%
        \fi
        \expandafter\ifvoid\csname pc@column@\number\count@\endcsname
        \else
          \global\let\@tempa\pc@placeboxes
        \fi
        \ifnum\count@<\pc@columncount
          \strut
          \hfill
          \ifpc@rulebetween
            \vrule
            \hfill
          \fi
        \fi
      \repeat
    }%
    \@tempa
  }%
  \ifx\x\pc@placeboxes
  \else
    \@PackageWarningNoLine{pdfcolparcolumns}{%
      Command \string\pc@placeboxes\space has changed.\MessageBreak
      Supported versions of package `parcolumns':\MessageBreak
      \space\space 2004/08/05.\MessageBreak
      The redefinition of \string\pc@placeboxes\space may not%
      \MessageBreak
      behave correctly depending on the changes%
    }%
  \fi
\endgroup
%    \end{macrocode}
%    \begin{macro}{\pc@placeboxes}
%    \begin{macrocode}
\renewcommand*{\pc@placeboxes}{%
  \global\let\@tempa\relax
  \count@\z@
  \hb@xt@\linewidth{%
    \vfuzz30ex %
    \vbadness\@M
    \splittopskip\z@skip
    \loop
    \ifnum\count@<\pc@columncount
      \advance\count@\@ne
      \expandafter\ifvoid\csname pc@column@\number\count@\endcsname
        \hskip\csname pc@column@width@\number\count@\endcsname
      \else
        \expandafter\setbox\expandafter\@tempboxa\expandafter
        \vsplit\csname pc@column@\number\count@\endcsname
            to \dp\strutbox
        \vbox{%
          \pcpc@SetCurrent\count@
          \unvbox\@tempboxa
        }%
      \fi
      \expandafter\ifvoid\csname pc@column@\number\count@\endcsname
      \else
        \global\let\@tempa\pc@placeboxes
      \fi
      \ifnum\count@<\pc@columncount
        \strut
        \hfill
        \ifpc@rulebetween
          \begingroup
            \pcpc@RuleBetweenColor
            \vrule
          \endgroup
          \hfill
        \fi
      \fi
    \repeat
  }%
  \@tempa
}
%    \end{macrocode}
%    \end{macro}
%    \begin{macro}{\pcpc@RuleBetweenColorDefault}
%    \begin{macrocode}
\def\pcpc@RuleBetweenColorDefault{%
  \normalcolor
}
%    \end{macrocode}
%    \end{macro}
%    \begin{macro}{\pcpc@RuleBetweenColor}
%    \begin{macrocode}
\let\pcpc@RuleBetweenColor\pcpc@RuleBetweenColorDefault
%    \end{macrocode}
%    \end{macro}
%    \begin{macrocode}
\define@key{parcolumns}{rulebetweencolor}{%
  \edef\pcpc@temp{#1}%
  \ifx\pcpc@temp\@empty
    \let\pcpc@RuleBetweenColor\pcpc@RuleBetweenColorDefault
  \else
    \edef\pcpc@temp{%
      \noexpand\@ifnextchar[{%
        \def\noexpand\pcpc@RuleBetweenColor{%
          \noexpand\color\pcpc@temp
        }%
        \noexpand\pcpc@GobbleNil
      }{%
        \def\noexpand\pcpc@RuleBetweenColor{%
          \noexpand\color{\pcpc@temp}%
        }%
        \noexpand\pcpc@GobbleNil
      }%
      \pcpc@temp\noexpand\@nil
    }%
    \pcpc@temp
  \fi
}
%    \end{macrocode}
%    \begin{macro}{\pcpc@GobbleNil}
%    \begin{macrocode}
\long\def\pcpc@GobbleNil#1\@nil{}
%    \end{macrocode}
%    \end{macro}
%
%    \begin{macrocode}
%</package>
%    \end{macrocode}
%
% \section{Test}
%
%    The test file is a modified version of the file that
%    Donald Goodman has posted in \xnewsgroup{comp.text.tex}: ^^A
%    \URL{``\link{Re: \xpackage{xcolor} glitches}''}^^A
%    {http://groups.google.com/group/comp.text.tex/msg/8eda74ed292012bb}
%    \begin{macrocode}
%<*test1>
\NeedsTeXFormat{LaTeX2e}
\AtEndDocument{%
  \typeout{}%
  \typeout{**************************************}%
  \typeout{*** \space Check the PDF file manually! \space ***}%
  \typeout{**************************************}%
  \typeout{}%
}
\documentclass{article}
\usepackage{xcolor}
\usepackage{pdfcolparcolumns}

\newcommand{\instruct}[1]{%
  \noindent
  \footnotesize
  \textcolor{red}{#1}%
}

\begin{document}
  \begin{parcolumns}[colwidths={1=2.3in,2=2.3in},sloppy]{2}%
    \colchunk[1]{%
      \instruct{Et non dicitur versus} %
      Fidelium anim\ae\ %
      \instruct{%
        sed immediate subiungitur antiphona finalis %
        beat\ae\ Mari\ae\ Virginis%
      } %
      100.%
    }%
    \colchunk[2]{%
      \instruct{%
        And the verse %
        \textcolor{black}{May the souls of the faithful} %
        is not said, but the final antiphon of the %
        Blessed Virgin Mary, %
        \textcolor{black}{100,} %
        is immediately joined.%
      }%
    }%
  \end{parcolumns}%
\end{document}
%</test1>
%    \end{macrocode}
%
% \section{Installation}
%
% \subsection{Download}
%
% \paragraph{Package.} This package is available on
% CTAN\footnote{\url{http://ctan.org/pkg/pdfcolparcolumns}}:
% \begin{description}
% \item[\CTAN{macros/latex/contrib/oberdiek/pdfcolparcolumns.dtx}] The source file.
% \item[\CTAN{macros/latex/contrib/oberdiek/pdfcolparcolumns.pdf}] Documentation.
% \end{description}
%
%
% \paragraph{Bundle.} All the packages of the bundle `oberdiek'
% are also available in a TDS compliant ZIP archive. There
% the packages are already unpacked and the documentation files
% are generated. The files and directories obey the TDS standard.
% \begin{description}
% \item[\CTAN{install/macros/latex/contrib/oberdiek.tds.zip}]
% \end{description}
% \emph{TDS} refers to the standard ``A Directory Structure
% for \TeX\ Files'' (\CTAN{tds/tds.pdf}). Directories
% with \xfile{texmf} in their name are usually organized this way.
%
% \subsection{Bundle installation}
%
% \paragraph{Unpacking.} Unpack the \xfile{oberdiek.tds.zip} in the
% TDS tree (also known as \xfile{texmf} tree) of your choice.
% Example (linux):
% \begin{quote}
%   |unzip oberdiek.tds.zip -d ~/texmf|
% \end{quote}
%
% \paragraph{Script installation.}
% Check the directory \xfile{TDS:scripts/oberdiek/} for
% scripts that need further installation steps.
% Package \xpackage{attachfile2} comes with the Perl script
% \xfile{pdfatfi.pl} that should be installed in such a way
% that it can be called as \texttt{pdfatfi}.
% Example (linux):
% \begin{quote}
%   |chmod +x scripts/oberdiek/pdfatfi.pl|\\
%   |cp scripts/oberdiek/pdfatfi.pl /usr/local/bin/|
% \end{quote}
%
% \subsection{Package installation}
%
% \paragraph{Unpacking.} The \xfile{.dtx} file is a self-extracting
% \docstrip\ archive. The files are extracted by running the
% \xfile{.dtx} through \plainTeX:
% \begin{quote}
%   \verb|tex pdfcolparcolumns.dtx|
% \end{quote}
%
% \paragraph{TDS.} Now the different files must be moved into
% the different directories in your installation TDS tree
% (also known as \xfile{texmf} tree):
% \begin{quote}
% \def\t{^^A
% \begin{tabular}{@{}>{\ttfamily}l@{ $\rightarrow$ }>{\ttfamily}l@{}}
%   pdfcolparcolumns.sty & tex/latex/oberdiek/pdfcolparcolumns.sty\\
%   pdfcolparcolumns.pdf & doc/latex/oberdiek/pdfcolparcolumns.pdf\\
%   test/pdfcolparcolumns-test1.tex & doc/latex/oberdiek/test/pdfcolparcolumns-test1.tex\\
%   pdfcolparcolumns.dtx & source/latex/oberdiek/pdfcolparcolumns.dtx\\
% \end{tabular}^^A
% }^^A
% \sbox0{\t}^^A
% \ifdim\wd0>\linewidth
%   \begingroup
%     \advance\linewidth by\leftmargin
%     \advance\linewidth by\rightmargin
%   \edef\x{\endgroup
%     \def\noexpand\lw{\the\linewidth}^^A
%   }\x
%   \def\lwbox{^^A
%     \leavevmode
%     \hbox to \linewidth{^^A
%       \kern-\leftmargin\relax
%       \hss
%       \usebox0
%       \hss
%       \kern-\rightmargin\relax
%     }^^A
%   }^^A
%   \ifdim\wd0>\lw
%     \sbox0{\small\t}^^A
%     \ifdim\wd0>\linewidth
%       \ifdim\wd0>\lw
%         \sbox0{\footnotesize\t}^^A
%         \ifdim\wd0>\linewidth
%           \ifdim\wd0>\lw
%             \sbox0{\scriptsize\t}^^A
%             \ifdim\wd0>\linewidth
%               \ifdim\wd0>\lw
%                 \sbox0{\tiny\t}^^A
%                 \ifdim\wd0>\linewidth
%                   \lwbox
%                 \else
%                   \usebox0
%                 \fi
%               \else
%                 \lwbox
%               \fi
%             \else
%               \usebox0
%             \fi
%           \else
%             \lwbox
%           \fi
%         \else
%           \usebox0
%         \fi
%       \else
%         \lwbox
%       \fi
%     \else
%       \usebox0
%     \fi
%   \else
%     \lwbox
%   \fi
% \else
%   \usebox0
% \fi
% \end{quote}
% If you have a \xfile{docstrip.cfg} that configures and enables \docstrip's
% TDS installing feature, then some files can already be in the right
% place, see the documentation of \docstrip.
%
% \subsection{Refresh file name databases}
%
% If your \TeX~distribution
% (\teTeX, \mikTeX, \dots) relies on file name databases, you must refresh
% these. For example, \teTeX\ users run \verb|texhash| or
% \verb|mktexlsr|.
%
% \subsection{Some details for the interested}
%
% \paragraph{Attached source.}
%
% The PDF documentation on CTAN also includes the
% \xfile{.dtx} source file. It can be extracted by
% AcrobatReader 6 or higher. Another option is \textsf{pdftk},
% e.g. unpack the file into the current directory:
% \begin{quote}
%   \verb|pdftk pdfcolparcolumns.pdf unpack_files output .|
% \end{quote}
%
% \paragraph{Unpacking with \LaTeX.}
% The \xfile{.dtx} chooses its action depending on the format:
% \begin{description}
% \item[\plainTeX:] Run \docstrip\ and extract the files.
% \item[\LaTeX:] Generate the documentation.
% \end{description}
% If you insist on using \LaTeX\ for \docstrip\ (really,
% \docstrip\ does not need \LaTeX), then inform the autodetect routine
% about your intention:
% \begin{quote}
%   \verb|latex \let\install=y\input{pdfcolparcolumns.dtx}|
% \end{quote}
% Do not forget to quote the argument according to the demands
% of your shell.
%
% \paragraph{Generating the documentation.}
% You can use both the \xfile{.dtx} or the \xfile{.drv} to generate
% the documentation. The process can be configured by the
% configuration file \xfile{ltxdoc.cfg}. For instance, put this
% line into this file, if you want to have A4 as paper format:
% \begin{quote}
%   \verb|\PassOptionsToClass{a4paper}{article}|
% \end{quote}
% An example follows how to generate the
% documentation with pdf\LaTeX:
% \begin{quote}
%\begin{verbatim}
%pdflatex pdfcolparcolumns.dtx
%makeindex -s gind.ist pdfcolparcolumns.idx
%pdflatex pdfcolparcolumns.dtx
%makeindex -s gind.ist pdfcolparcolumns.idx
%pdflatex pdfcolparcolumns.dtx
%\end{verbatim}
% \end{quote}
%
% \section{Catalogue}
%
% The following XML file can be used as source for the
% \href{http://mirror.ctan.org/help/Catalogue/catalogue.html}{\TeX\ Catalogue}.
% The elements \texttt{caption} and \texttt{description} are imported
% from the original XML file from the Catalogue.
% The name of the XML file in the Catalogue is \xfile{pdfcolparcolumns.xml}.
%    \begin{macrocode}
%<*catalogue>
<?xml version='1.0' encoding='us-ascii'?>
<!DOCTYPE entry SYSTEM 'catalogue.dtd'>
<entry datestamp='$Date$' modifier='$Author$' id='pdfcolparcolumns'>
  <name>pdfcolparcolumns</name>
  <caption>Fix colour problems in package 'parcolumns'.</caption>
  <authorref id='auth:oberdiek'/>
  <copyright owner='Heiko Oberdiek' year='2007,2008,2010'/>
  <license type='lppl1.3'/>
  <version number='1.4'/>
  <description>
    Since version 1.40 pdfTeX supports colour stacks.
    This package uses them to fix colour problems in
    package <xref refid='parcolumns'>parcolumns</xref>.
    <p/>
    The package is part of the <xref refid='oberdiek'>oberdiek</xref>
    bundle.
  </description>
  <documentation details='Package documentation'
      href='ctan:/macros/latex/contrib/oberdiek/pdfcolparcolumns.pdf'/>
  <ctan file='true' path='/macros/latex/contrib/oberdiek/pdfcolparcolumns.dtx'/>
  <miktex location='oberdiek'/>
  <texlive location='oberdiek'/>
  <install path='/macros/latex/contrib/oberdiek/oberdiek.tds.zip'/>
</entry>
%</catalogue>
%    \end{macrocode}
%
% \begin{thebibliography}{9}
%
% \bibitem{parcolumns}
%   Jonathan Sauer: \textit{The \xpackage{parcolumns} package};
%   2004/11/25;\\
%   \CTAN{macros/latex/contrib/sauerj/parcolumns.pdf}.
%
% \bibitem{pdfcol}
%   Heiko Oberdiek: \textit{The \xpackage{pdfcol} package};
%   2007/09/09;\\
%   \CTAN{macros/latex/contrib/oberdiek/pdfcol.pdf}.
%
% \end{thebibliography}
%
% \begin{History}
%   \begin{Version}{2007/07/26 v1.0}
%   \item
%     First version, published in the newsgroup \xnewsgroup{comp.text.tex}
%     with the name \xpackage{parcolumns-colorstacks}: ^^A no line break
%     \URL{``\link{Re: \xpackage{xcolor} glitches}''}^^A
%     {http://groups.google.com/group/comp.text.tex/msg/56bd897b11bca414}
%   \end{Version}
%   \begin{Version}{2007/09/09 v1.1}
%   \item
%     CTAN version, package name renamed to \xpackage{pdfcolparcolumns}.
%   \item
%     Uses package \xpackage{pdfcol}.
%   \item
%     Documentation added.
%   \item
%     Test file added.
%   \end{Version}
%   \begin{Version}{2008/08/11 v1.2}
%   \item
%     Code is not changed.
%   \item
%     URLs updated.
%   \end{Version}
%   \begin{Version}{2010/01/11 v1.3}
%   \item
%     Fix for rule color.
%   \item
%     New option \xoption{rulebetweencolor} for environment |parcolumns|.
%   \end{Version}
%   \begin{Version}{2016/05/16 v1.4}
%   \item
%     Documentation updates.
%   \end{Version}
% \end{History}
%
% \PrintIndex
%
% \Finale
\endinput
|
% \end{quote}
% Do not forget to quote the argument according to the demands
% of your shell.
%
% \paragraph{Generating the documentation.}
% You can use both the \xfile{.dtx} or the \xfile{.drv} to generate
% the documentation. The process can be configured by the
% configuration file \xfile{ltxdoc.cfg}. For instance, put this
% line into this file, if you want to have A4 as paper format:
% \begin{quote}
%   \verb|\PassOptionsToClass{a4paper}{article}|
% \end{quote}
% An example follows how to generate the
% documentation with pdf\LaTeX:
% \begin{quote}
%\begin{verbatim}
%pdflatex pdfcolparcolumns.dtx
%makeindex -s gind.ist pdfcolparcolumns.idx
%pdflatex pdfcolparcolumns.dtx
%makeindex -s gind.ist pdfcolparcolumns.idx
%pdflatex pdfcolparcolumns.dtx
%\end{verbatim}
% \end{quote}
%
% \section{Catalogue}
%
% The following XML file can be used as source for the
% \href{http://mirror.ctan.org/help/Catalogue/catalogue.html}{\TeX\ Catalogue}.
% The elements \texttt{caption} and \texttt{description} are imported
% from the original XML file from the Catalogue.
% The name of the XML file in the Catalogue is \xfile{pdfcolparcolumns.xml}.
%    \begin{macrocode}
%<*catalogue>
<?xml version='1.0' encoding='us-ascii'?>
<!DOCTYPE entry SYSTEM 'catalogue.dtd'>
<entry datestamp='$Date$' modifier='$Author$' id='pdfcolparcolumns'>
  <name>pdfcolparcolumns</name>
  <caption>Fix colour problems in package 'parcolumns'.</caption>
  <authorref id='auth:oberdiek'/>
  <copyright owner='Heiko Oberdiek' year='2007,2008,2010'/>
  <license type='lppl1.3'/>
  <version number='1.4'/>
  <description>
    Since version 1.40 pdfTeX supports colour stacks.
    This package uses them to fix colour problems in
    package <xref refid='parcolumns'>parcolumns</xref>.
    <p/>
    The package is part of the <xref refid='oberdiek'>oberdiek</xref>
    bundle.
  </description>
  <documentation details='Package documentation'
      href='ctan:/macros/latex/contrib/oberdiek/pdfcolparcolumns.pdf'/>
  <ctan file='true' path='/macros/latex/contrib/oberdiek/pdfcolparcolumns.dtx'/>
  <miktex location='oberdiek'/>
  <texlive location='oberdiek'/>
  <install path='/macros/latex/contrib/oberdiek/oberdiek.tds.zip'/>
</entry>
%</catalogue>
%    \end{macrocode}
%
% \begin{thebibliography}{9}
%
% \bibitem{parcolumns}
%   Jonathan Sauer: \textit{The \xpackage{parcolumns} package};
%   2004/11/25;\\
%   \CTAN{macros/latex/contrib/sauerj/parcolumns.pdf}.
%
% \bibitem{pdfcol}
%   Heiko Oberdiek: \textit{The \xpackage{pdfcol} package};
%   2007/09/09;\\
%   \CTAN{macros/latex/contrib/oberdiek/pdfcol.pdf}.
%
% \end{thebibliography}
%
% \begin{History}
%   \begin{Version}{2007/07/26 v1.0}
%   \item
%     First version, published in the newsgroup \xnewsgroup{comp.text.tex}
%     with the name \xpackage{parcolumns-colorstacks}: ^^A no line break
%     \URL{``\link{Re: \xpackage{xcolor} glitches}''}^^A
%     {http://groups.google.com/group/comp.text.tex/msg/56bd897b11bca414}
%   \end{Version}
%   \begin{Version}{2007/09/09 v1.1}
%   \item
%     CTAN version, package name renamed to \xpackage{pdfcolparcolumns}.
%   \item
%     Uses package \xpackage{pdfcol}.
%   \item
%     Documentation added.
%   \item
%     Test file added.
%   \end{Version}
%   \begin{Version}{2008/08/11 v1.2}
%   \item
%     Code is not changed.
%   \item
%     URLs updated.
%   \end{Version}
%   \begin{Version}{2010/01/11 v1.3}
%   \item
%     Fix for rule color.
%   \item
%     New option \xoption{rulebetweencolor} for environment |parcolumns|.
%   \end{Version}
%   \begin{Version}{2016/05/16 v1.4}
%   \item
%     Documentation updates.
%   \end{Version}
% \end{History}
%
% \PrintIndex
%
% \Finale
\endinput
|
% \end{quote}
% Do not forget to quote the argument according to the demands
% of your shell.
%
% \paragraph{Generating the documentation.}
% You can use both the \xfile{.dtx} or the \xfile{.drv} to generate
% the documentation. The process can be configured by the
% configuration file \xfile{ltxdoc.cfg}. For instance, put this
% line into this file, if you want to have A4 as paper format:
% \begin{quote}
%   \verb|\PassOptionsToClass{a4paper}{article}|
% \end{quote}
% An example follows how to generate the
% documentation with pdf\LaTeX:
% \begin{quote}
%\begin{verbatim}
%pdflatex pdfcolparcolumns.dtx
%makeindex -s gind.ist pdfcolparcolumns.idx
%pdflatex pdfcolparcolumns.dtx
%makeindex -s gind.ist pdfcolparcolumns.idx
%pdflatex pdfcolparcolumns.dtx
%\end{verbatim}
% \end{quote}
%
% \section{Catalogue}
%
% The following XML file can be used as source for the
% \href{http://mirror.ctan.org/help/Catalogue/catalogue.html}{\TeX\ Catalogue}.
% The elements \texttt{caption} and \texttt{description} are imported
% from the original XML file from the Catalogue.
% The name of the XML file in the Catalogue is \xfile{pdfcolparcolumns.xml}.
%    \begin{macrocode}
%<*catalogue>
<?xml version='1.0' encoding='us-ascii'?>
<!DOCTYPE entry SYSTEM 'catalogue.dtd'>
<entry datestamp='$Date$' modifier='$Author$' id='pdfcolparcolumns'>
  <name>pdfcolparcolumns</name>
  <caption>Fix colour problems in package 'parcolumns'.</caption>
  <authorref id='auth:oberdiek'/>
  <copyright owner='Heiko Oberdiek' year='2007,2008,2010'/>
  <license type='lppl1.3'/>
  <version number='1.4'/>
  <description>
    Since version 1.40 pdfTeX supports colour stacks.
    This package uses them to fix colour problems in
    package <xref refid='parcolumns'>parcolumns</xref>.
    <p/>
    The package is part of the <xref refid='oberdiek'>oberdiek</xref>
    bundle.
  </description>
  <documentation details='Package documentation'
      href='ctan:/macros/latex/contrib/oberdiek/pdfcolparcolumns.pdf'/>
  <ctan file='true' path='/macros/latex/contrib/oberdiek/pdfcolparcolumns.dtx'/>
  <miktex location='oberdiek'/>
  <texlive location='oberdiek'/>
  <install path='/macros/latex/contrib/oberdiek/oberdiek.tds.zip'/>
</entry>
%</catalogue>
%    \end{macrocode}
%
% \begin{thebibliography}{9}
%
% \bibitem{parcolumns}
%   Jonathan Sauer: \textit{The \xpackage{parcolumns} package};
%   2004/11/25;\\
%   \CTAN{macros/latex/contrib/sauerj/parcolumns.pdf}.
%
% \bibitem{pdfcol}
%   Heiko Oberdiek: \textit{The \xpackage{pdfcol} package};
%   2007/09/09;\\
%   \CTAN{macros/latex/contrib/oberdiek/pdfcol.pdf}.
%
% \end{thebibliography}
%
% \begin{History}
%   \begin{Version}{2007/07/26 v1.0}
%   \item
%     First version, published in the newsgroup \xnewsgroup{comp.text.tex}
%     with the name \xpackage{parcolumns-colorstacks}: ^^A no line break
%     \URL{``\link{Re: \xpackage{xcolor} glitches}''}^^A
%     {http://groups.google.com/group/comp.text.tex/msg/56bd897b11bca414}
%   \end{Version}
%   \begin{Version}{2007/09/09 v1.1}
%   \item
%     CTAN version, package name renamed to \xpackage{pdfcolparcolumns}.
%   \item
%     Uses package \xpackage{pdfcol}.
%   \item
%     Documentation added.
%   \item
%     Test file added.
%   \end{Version}
%   \begin{Version}{2008/08/11 v1.2}
%   \item
%     Code is not changed.
%   \item
%     URLs updated.
%   \end{Version}
%   \begin{Version}{2010/01/11 v1.3}
%   \item
%     Fix for rule color.
%   \item
%     New option \xoption{rulebetweencolor} for environment |parcolumns|.
%   \end{Version}
%   \begin{Version}{2016/05/16 v1.4}
%   \item
%     Documentation updates.
%   \end{Version}
% \end{History}
%
% \PrintIndex
%
% \Finale
\endinput

%        (quote the arguments according to the demands of your shell)
%
% Documentation:
%    (a) If pdfcolparcolumns.drv is present:
%           latex pdfcolparcolumns.drv
%    (b) Without pdfcolparcolumns.drv:
%           latex pdfcolparcolumns.dtx; ...
%    The class ltxdoc loads the configuration file ltxdoc.cfg
%    if available. Here you can specify further options, e.g.
%    use A4 as paper format:
%       \PassOptionsToClass{a4paper}{article}
%
%    Programm calls to get the documentation (example):
%       pdflatex pdfcolparcolumns.dtx
%       makeindex -s gind.ist pdfcolparcolumns.idx
%       pdflatex pdfcolparcolumns.dtx
%       makeindex -s gind.ist pdfcolparcolumns.idx
%       pdflatex pdfcolparcolumns.dtx
%
% Installation:
%    TDS:tex/latex/oberdiek/pdfcolparcolumns.sty
%    TDS:doc/latex/oberdiek/pdfcolparcolumns.pdf
%    TDS:doc/latex/oberdiek/test/pdfcolparcolumns-test1.tex
%    TDS:source/latex/oberdiek/pdfcolparcolumns.dtx
%
%<*ignore>
\begingroup
  \catcode123=1 %
  \catcode125=2 %
  \def\x{LaTeX2e}%
\expandafter\endgroup
\ifcase 0\ifx\install y1\fi\expandafter
         \ifx\csname processbatchFile\endcsname\relax\else1\fi
         \ifx\fmtname\x\else 1\fi\relax
\else\csname fi\endcsname
%</ignore>
%<*install>
\input docstrip.tex
\Msg{************************************************************************}
\Msg{* Installation}
\Msg{* Package: pdfcolparcolumns 2016/05/16 v1.4 Color stacks for parcolumns (HO)}
\Msg{************************************************************************}

\keepsilent
\askforoverwritefalse

\let\MetaPrefix\relax
\preamble

This is a generated file.

Project: pdfcolparcolumns
Version: 2016/05/16 v1.4

Copyright (C) 2007, 2008, 2010 by
   Heiko Oberdiek <heiko.oberdiek at googlemail.com>

This work may be distributed and/or modified under the
conditions of the LaTeX Project Public License, either
version 1.3c of this license or (at your option) any later
version. This version of this license is in
   http://www.latex-project.org/lppl/lppl-1-3c.txt
and the latest version of this license is in
   http://www.latex-project.org/lppl.txt
and version 1.3 or later is part of all distributions of
LaTeX version 2005/12/01 or later.

This work has the LPPL maintenance status "maintained".

This Current Maintainer of this work is Heiko Oberdiek.

This work consists of the main source file pdfcolparcolumns.dtx
and the derived files
   pdfcolparcolumns.sty, pdfcolparcolumns.pdf, pdfcolparcolumns.ins,
   pdfcolparcolumns.drv, pdfcolparcolumns-test1.tex.

\endpreamble
\let\MetaPrefix\DoubleperCent

\generate{%
  \file{pdfcolparcolumns.ins}{\from{pdfcolparcolumns.dtx}{install}}%
  \file{pdfcolparcolumns.drv}{\from{pdfcolparcolumns.dtx}{driver}}%
  \usedir{tex/latex/oberdiek}%
  \file{pdfcolparcolumns.sty}{\from{pdfcolparcolumns.dtx}{package}}%
  \usedir{doc/latex/oberdiek/test}%
  \file{pdfcolparcolumns-test1.tex}{\from{pdfcolparcolumns.dtx}{test1}}%
  \nopreamble
  \nopostamble
  \usedir{source/latex/oberdiek/catalogue}%
  \file{pdfcolparcolumns.xml}{\from{pdfcolparcolumns.dtx}{catalogue}}%
}

\catcode32=13\relax% active space
\let =\space%
\Msg{************************************************************************}
\Msg{*}
\Msg{* To finish the installation you have to move the following}
\Msg{* file into a directory searched by TeX:}
\Msg{*}
\Msg{*     pdfcolparcolumns.sty}
\Msg{*}
\Msg{* To produce the documentation run the file `pdfcolparcolumns.drv'}
\Msg{* through LaTeX.}
\Msg{*}
\Msg{* Happy TeXing!}
\Msg{*}
\Msg{************************************************************************}

\endbatchfile
%</install>
%<*ignore>
\fi
%</ignore>
%<*driver>
\NeedsTeXFormat{LaTeX2e}
\ProvidesFile{pdfcolparcolumns.drv}%
  [2016/05/16 v1.4 Color stacks for parcolumns (HO)]%
\documentclass{ltxdoc}
\usepackage{holtxdoc}[2011/11/22]
\begin{document}
  \DocInput{pdfcolparcolumns.dtx}%
\end{document}
%</driver>
% \fi
%
%
% \CharacterTable
%  {Upper-case    \A\B\C\D\E\F\G\H\I\J\K\L\M\N\O\P\Q\R\S\T\U\V\W\X\Y\Z
%   Lower-case    \a\b\c\d\e\f\g\h\i\j\k\l\m\n\o\p\q\r\s\t\u\v\w\x\y\z
%   Digits        \0\1\2\3\4\5\6\7\8\9
%   Exclamation   \!     Double quote  \"     Hash (number) \#
%   Dollar        \$     Percent       \%     Ampersand     \&
%   Acute accent  \'     Left paren    \(     Right paren   \)
%   Asterisk      \*     Plus          \+     Comma         \,
%   Minus         \-     Point         \.     Solidus       \/
%   Colon         \:     Semicolon     \;     Less than     \<
%   Equals        \=     Greater than  \>     Question mark \?
%   Commercial at \@     Left bracket  \[     Backslash     \\
%   Right bracket \]     Circumflex    \^     Underscore    \_
%   Grave accent  \`     Left brace    \{     Vertical bar  \|
%   Right brace   \}     Tilde         \~}
%
% \GetFileInfo{pdfcolparcolumns.drv}
%
% \title{The \xpackage{pdfcolparcolumns} package}
% \date{2016/05/16 v1.4}
% \author{Heiko Oberdiek\thanks
% {Please report any issues at https://github.com/ho-tex/oberdiek/issues}\\
% \xemail{heiko.oberdiek at googlemail.com}}
%
% \maketitle
%
% \begin{abstract}
% Since version 1.40 \pdfTeX\ supports several color stacks.
% This package uses them to fix color problems in
% package \xpackage{parcolumns}.
% \end{abstract}
%
% \tableofcontents
%
% \section{Usage}
%
% \begin{quote}
% |\usepackage{pdfcolparcolumns}|
% \end{quote}
% The package \xpackage{pdfcolparcolumns} loads package \xpackage{parcolums}
% \cite{parcolumns}. If color stacks are available then the
% macros of \xpackage{parcolumns} are patched to add support
% for color stacks.
%
% \subsection{Option \xoption{rulebetweencolor}}
%
% Package \xpackage{pdfcolparcolumns} also fixes the color for the
% rule between columns (if \xoption{rulebetween} is set).
% Default color is \cs{normalcolor}. But this can be changed by using
% option \xoption{rulebetweencolor}. It takes a color specification
% as value. If the value is empty, then the default (\cs{normalcolor})
% is used.
% Examples:
% \begin{quote}
%   |rulebetweencolor=blue|,\\
%   |rulebetweencolor={red}|,\\
%   |rulebetweencolor={}|, \textit{\% \cs{normalcolor} is used}\\
%   |rulebetweencolor=[rgb]{1,0,.5}| \textit{\% see below}
% \end{quote}
% If used inside the optional argument of environment \textsf{parcolumns}
% and the value contains an optional argument, then whole value
% must be put in curly braces:
%\begin{quote}
%\begin{verbatim}
%\begin{parcolumns}[
%  rulebetween,
%  rulebetweencolor={[rgb]{1,0,.5}},
%]{2}
%  ...
%\end{parcolumns}
%\end{verbatim}
%\end{quote}
% This option \xoption{rulebetweencolor} can also be set using
% \cs{setkeys}:
%\begin{quote}
%\begin{verbatim}
%\setkeys{parcolumns}{rulebetweencolor=green}
%\end{verbatim}
%\end{quote}
%
% \subsection{Future}
%
% Currently package \xpackage{parcolumns} does not seem to be
% maintained. Nevertheless if there will be a new version that
% adds support for color stacks, then this package may become
% obsolete.
%
% \StopEventually{
% }
%
% \section{Implementation}
%
% \subsection{Identification}
%
%    \begin{macrocode}
%<*package>
\NeedsTeXFormat{LaTeX2e}
\ProvidesPackage{pdfcolparcolumns}%
  [2016/05/16 v1.4 Color stacks for parcolumns (HO)]%
%    \end{macrocode}
%
% \subsection{Load packages}
%
% \subsubsection{Package \xpackage{parcolumns}}
%
%    Currently package \xpackage{parcolumns} does not define options.
%    Thus it is just a precaution that the options of
%    package \xpackage{pdfcolparcolumns} are passed to
%    package \xpackage{parcolumns}.
%    \begin{macrocode}
\DeclareOption*{%
  \PassoptionsToPackage{\CurrentOption}{parcolumns}%
}
\ProcessOptions\relax
\RequirePackage{parcolumns}[2004/11/25]
%    \end{macrocode}
%
% \subsubsection{Package \xpackage{pdfcol}}
%
%    \begin{macrocode}
\RequirePackage{pdfcol}[2007/09/09]
\ifpdfcolAvailable
\else
  \PackageInfo{pdfcolparcolumns}{%
    Loading aborted, because color stacks are not available%
  }%
  \expandafter\endinput
\fi
%    \end{macrocode}
%
% \subsubsection{Package \xpackage{infwarerr}}
%
%    \begin{macrocode}
\RequirePackage{infwarerr}[2007/09/09]
%    \end{macrocode}
%
% \subsection{Color stack macros}
%
%    \begin{macro}{\pcpc@MaxStack}
%    Macro \cs{pcpc@MaxStack} holds the highest number of
%    allocated stacks.
%    \begin{macrocode}
\global\chardef\pcpc@MaxStack=\z@
%    \end{macrocode}
%    \end{macro}
%    \begin{macro}{\pcpc@InitStacks}
%    Macro \cs{pcpc@InitStacks} takes the number of columns
%    as argument and ensures that there are enough color
%    stacks for all columns.
%    \begin{macrocode}
\def\pcpc@InitStacks#1{%
  \ifnum#1>\pcpc@MaxStack
    \begingroup
      \count@\pcpc@MaxStack
      \loop
        \advance\count@\@ne
        \pdfcolInitStack{pcpc@\the\count@}%
      \ifnum#1>\count@
      \repeat
      \global\chardef\pcpc@MaxStack=\count@
    \endgroup
  \fi
}
%    \end{macrocode}
%    \end{macro}
%
%    \begin{macro}{\pcpc@SwitchStack}
%    \begin{macrocode}
\def\pcpc@SwitchStack#1{%
  \pdfcolSwitchStack{pcpc@\number#1}%
}
%    \end{macrocode}
%    \end{macro}
%
%    \begin{macro}{\pcpc@SetCurrent}
%    \begin{macrocode}
\def\pcpc@SetCurrent#1{%
  \pdfcolSetCurrent{pcpc@\number#1}%
}
%    \end{macrocode}
%    \end{macro}
%
% \subsection{Patches}
%
%     Now the color stack macros are patched into the macros
%     of package \xpackage{parcolumns}.
%
% \subsubsection{Init stacks}
%
%    \cs{pcpc@InitStacks} should go into the definition of
%    environment |parcolumns|. \cs{pc@alloccolumns} is executed
%    there and nowhere else, thus we hook into it.
%    \begin{macrocode}
\g@addto@macro\pc@alloccolumns{%
  \pcpc@InitStacks\pc@columncount
}
%    \end{macrocode}
%
% \subsubsection{Switch stack}
%
%    \cs{pcpc@SwitchStack} should be called by marco \cs{colchunk@}.
%    However it is easier to patch \cs{pc@setcolumnwidth} that
%    is executed in \cs{colchunk@} only.
%    \begin{macrocode}
\g@addto@macro\pc@setcolumnwidth{%
  \pcpc@SwitchStack\pc@columnctr
}
%    \end{macrocode}
%
% \subsubsection{Set current stack color}
%
%    \cs{pcpc@SetCurrent} is set at the begin of each line.
%    It must be inserted into \cs{pc@placeboxes}. Unhappily
%    there is no easy way. Therefore we check and
%    redefine \cs{pc@placeboxes}.
%    \begin{macrocode}
\begingroup
  \def\x{%
    \global\let\@tempa\relax
    \count@\z@
    \hb@xt@\linewidth{%
      \vfuzz30ex %
      \vbadness\@M
      \splittopskip\z@skip
      \loop
      \ifnum\count@<\pc@columncount
        \advance\count@\@ne
        \expandafter\ifvoid\csname pc@column@\number\count@\endcsname
          \hskip\csname pc@column@width@\number\count@\endcsname
        \else
          \expandafter\setbox\expandafter\@tempboxa\expandafter
          \vsplit\csname pc@column@\number\count@\endcsname
              to \dp\strutbox
          \vbox{%
            \unvbox\@tempboxa
          }%
        \fi
        \expandafter\ifvoid\csname pc@column@\number\count@\endcsname
        \else
          \global\let\@tempa\pc@placeboxes
        \fi
        \ifnum\count@<\pc@columncount
          \strut
          \hfill
          \ifpc@rulebetween
            \vrule
            \hfill
          \fi
        \fi
      \repeat
    }%
    \@tempa
  }%
  \ifx\x\pc@placeboxes
  \else
    \@PackageWarningNoLine{pdfcolparcolumns}{%
      Command \string\pc@placeboxes\space has changed.\MessageBreak
      Supported versions of package `parcolumns':\MessageBreak
      \space\space 2004/08/05.\MessageBreak
      The redefinition of \string\pc@placeboxes\space may not%
      \MessageBreak
      behave correctly depending on the changes%
    }%
  \fi
\endgroup
%    \end{macrocode}
%    \begin{macro}{\pc@placeboxes}
%    \begin{macrocode}
\renewcommand*{\pc@placeboxes}{%
  \global\let\@tempa\relax
  \count@\z@
  \hb@xt@\linewidth{%
    \vfuzz30ex %
    \vbadness\@M
    \splittopskip\z@skip
    \loop
    \ifnum\count@<\pc@columncount
      \advance\count@\@ne
      \expandafter\ifvoid\csname pc@column@\number\count@\endcsname
        \hskip\csname pc@column@width@\number\count@\endcsname
      \else
        \expandafter\setbox\expandafter\@tempboxa\expandafter
        \vsplit\csname pc@column@\number\count@\endcsname
            to \dp\strutbox
        \vbox{%
          \pcpc@SetCurrent\count@
          \unvbox\@tempboxa
        }%
      \fi
      \expandafter\ifvoid\csname pc@column@\number\count@\endcsname
      \else
        \global\let\@tempa\pc@placeboxes
      \fi
      \ifnum\count@<\pc@columncount
        \strut
        \hfill
        \ifpc@rulebetween
          \begingroup
            \pcpc@RuleBetweenColor
            \vrule
          \endgroup
          \hfill
        \fi
      \fi
    \repeat
  }%
  \@tempa
}
%    \end{macrocode}
%    \end{macro}
%    \begin{macro}{\pcpc@RuleBetweenColorDefault}
%    \begin{macrocode}
\def\pcpc@RuleBetweenColorDefault{%
  \normalcolor
}
%    \end{macrocode}
%    \end{macro}
%    \begin{macro}{\pcpc@RuleBetweenColor}
%    \begin{macrocode}
\let\pcpc@RuleBetweenColor\pcpc@RuleBetweenColorDefault
%    \end{macrocode}
%    \end{macro}
%    \begin{macrocode}
\define@key{parcolumns}{rulebetweencolor}{%
  \edef\pcpc@temp{#1}%
  \ifx\pcpc@temp\@empty
    \let\pcpc@RuleBetweenColor\pcpc@RuleBetweenColorDefault
  \else
    \edef\pcpc@temp{%
      \noexpand\@ifnextchar[{%
        \def\noexpand\pcpc@RuleBetweenColor{%
          \noexpand\color\pcpc@temp
        }%
        \noexpand\pcpc@GobbleNil
      }{%
        \def\noexpand\pcpc@RuleBetweenColor{%
          \noexpand\color{\pcpc@temp}%
        }%
        \noexpand\pcpc@GobbleNil
      }%
      \pcpc@temp\noexpand\@nil
    }%
    \pcpc@temp
  \fi
}
%    \end{macrocode}
%    \begin{macro}{\pcpc@GobbleNil}
%    \begin{macrocode}
\long\def\pcpc@GobbleNil#1\@nil{}
%    \end{macrocode}
%    \end{macro}
%
%    \begin{macrocode}
%</package>
%    \end{macrocode}
%
% \section{Test}
%
%    The test file is a modified version of the file that
%    Donald Goodman has posted in \xnewsgroup{comp.text.tex}: ^^A
%    \URL{``\link{Re: \xpackage{xcolor} glitches}''}^^A
%    {http://groups.google.com/group/comp.text.tex/msg/8eda74ed292012bb}
%    \begin{macrocode}
%<*test1>
\NeedsTeXFormat{LaTeX2e}
\AtEndDocument{%
  \typeout{}%
  \typeout{**************************************}%
  \typeout{*** \space Check the PDF file manually! \space ***}%
  \typeout{**************************************}%
  \typeout{}%
}
\documentclass{article}
\usepackage{xcolor}
\usepackage{pdfcolparcolumns}

\newcommand{\instruct}[1]{%
  \noindent
  \footnotesize
  \textcolor{red}{#1}%
}

\begin{document}
  \begin{parcolumns}[colwidths={1=2.3in,2=2.3in},sloppy]{2}%
    \colchunk[1]{%
      \instruct{Et non dicitur versus} %
      Fidelium anim\ae\ %
      \instruct{%
        sed immediate subiungitur antiphona finalis %
        beat\ae\ Mari\ae\ Virginis%
      } %
      100.%
    }%
    \colchunk[2]{%
      \instruct{%
        And the verse %
        \textcolor{black}{May the souls of the faithful} %
        is not said, but the final antiphon of the %
        Blessed Virgin Mary, %
        \textcolor{black}{100,} %
        is immediately joined.%
      }%
    }%
  \end{parcolumns}%
\end{document}
%</test1>
%    \end{macrocode}
%
% \section{Installation}
%
% \subsection{Download}
%
% \paragraph{Package.} This package is available on
% CTAN\footnote{\url{http://ctan.org/pkg/pdfcolparcolumns}}:
% \begin{description}
% \item[\CTAN{macros/latex/contrib/oberdiek/pdfcolparcolumns.dtx}] The source file.
% \item[\CTAN{macros/latex/contrib/oberdiek/pdfcolparcolumns.pdf}] Documentation.
% \end{description}
%
%
% \paragraph{Bundle.} All the packages of the bundle `oberdiek'
% are also available in a TDS compliant ZIP archive. There
% the packages are already unpacked and the documentation files
% are generated. The files and directories obey the TDS standard.
% \begin{description}
% \item[\CTAN{install/macros/latex/contrib/oberdiek.tds.zip}]
% \end{description}
% \emph{TDS} refers to the standard ``A Directory Structure
% for \TeX\ Files'' (\CTAN{tds/tds.pdf}). Directories
% with \xfile{texmf} in their name are usually organized this way.
%
% \subsection{Bundle installation}
%
% \paragraph{Unpacking.} Unpack the \xfile{oberdiek.tds.zip} in the
% TDS tree (also known as \xfile{texmf} tree) of your choice.
% Example (linux):
% \begin{quote}
%   |unzip oberdiek.tds.zip -d ~/texmf|
% \end{quote}
%
% \paragraph{Script installation.}
% Check the directory \xfile{TDS:scripts/oberdiek/} for
% scripts that need further installation steps.
% Package \xpackage{attachfile2} comes with the Perl script
% \xfile{pdfatfi.pl} that should be installed in such a way
% that it can be called as \texttt{pdfatfi}.
% Example (linux):
% \begin{quote}
%   |chmod +x scripts/oberdiek/pdfatfi.pl|\\
%   |cp scripts/oberdiek/pdfatfi.pl /usr/local/bin/|
% \end{quote}
%
% \subsection{Package installation}
%
% \paragraph{Unpacking.} The \xfile{.dtx} file is a self-extracting
% \docstrip\ archive. The files are extracted by running the
% \xfile{.dtx} through \plainTeX:
% \begin{quote}
%   \verb|tex pdfcolparcolumns.dtx|
% \end{quote}
%
% \paragraph{TDS.} Now the different files must be moved into
% the different directories in your installation TDS tree
% (also known as \xfile{texmf} tree):
% \begin{quote}
% \def\t{^^A
% \begin{tabular}{@{}>{\ttfamily}l@{ $\rightarrow$ }>{\ttfamily}l@{}}
%   pdfcolparcolumns.sty & tex/latex/oberdiek/pdfcolparcolumns.sty\\
%   pdfcolparcolumns.pdf & doc/latex/oberdiek/pdfcolparcolumns.pdf\\
%   test/pdfcolparcolumns-test1.tex & doc/latex/oberdiek/test/pdfcolparcolumns-test1.tex\\
%   pdfcolparcolumns.dtx & source/latex/oberdiek/pdfcolparcolumns.dtx\\
% \end{tabular}^^A
% }^^A
% \sbox0{\t}^^A
% \ifdim\wd0>\linewidth
%   \begingroup
%     \advance\linewidth by\leftmargin
%     \advance\linewidth by\rightmargin
%   \edef\x{\endgroup
%     \def\noexpand\lw{\the\linewidth}^^A
%   }\x
%   \def\lwbox{^^A
%     \leavevmode
%     \hbox to \linewidth{^^A
%       \kern-\leftmargin\relax
%       \hss
%       \usebox0
%       \hss
%       \kern-\rightmargin\relax
%     }^^A
%   }^^A
%   \ifdim\wd0>\lw
%     \sbox0{\small\t}^^A
%     \ifdim\wd0>\linewidth
%       \ifdim\wd0>\lw
%         \sbox0{\footnotesize\t}^^A
%         \ifdim\wd0>\linewidth
%           \ifdim\wd0>\lw
%             \sbox0{\scriptsize\t}^^A
%             \ifdim\wd0>\linewidth
%               \ifdim\wd0>\lw
%                 \sbox0{\tiny\t}^^A
%                 \ifdim\wd0>\linewidth
%                   \lwbox
%                 \else
%                   \usebox0
%                 \fi
%               \else
%                 \lwbox
%               \fi
%             \else
%               \usebox0
%             \fi
%           \else
%             \lwbox
%           \fi
%         \else
%           \usebox0
%         \fi
%       \else
%         \lwbox
%       \fi
%     \else
%       \usebox0
%     \fi
%   \else
%     \lwbox
%   \fi
% \else
%   \usebox0
% \fi
% \end{quote}
% If you have a \xfile{docstrip.cfg} that configures and enables \docstrip's
% TDS installing feature, then some files can already be in the right
% place, see the documentation of \docstrip.
%
% \subsection{Refresh file name databases}
%
% If your \TeX~distribution
% (\teTeX, \mikTeX, \dots) relies on file name databases, you must refresh
% these. For example, \teTeX\ users run \verb|texhash| or
% \verb|mktexlsr|.
%
% \subsection{Some details for the interested}
%
% \paragraph{Attached source.}
%
% The PDF documentation on CTAN also includes the
% \xfile{.dtx} source file. It can be extracted by
% AcrobatReader 6 or higher. Another option is \textsf{pdftk},
% e.g. unpack the file into the current directory:
% \begin{quote}
%   \verb|pdftk pdfcolparcolumns.pdf unpack_files output .|
% \end{quote}
%
% \paragraph{Unpacking with \LaTeX.}
% The \xfile{.dtx} chooses its action depending on the format:
% \begin{description}
% \item[\plainTeX:] Run \docstrip\ and extract the files.
% \item[\LaTeX:] Generate the documentation.
% \end{description}
% If you insist on using \LaTeX\ for \docstrip\ (really,
% \docstrip\ does not need \LaTeX), then inform the autodetect routine
% about your intention:
% \begin{quote}
%   \verb|latex \let\install=y% \iffalse meta-comment
%
% File: pdfcolparcolumns.dtx
% Version: 2016/05/16 v1.4
% Info: Color stacks for parcolumns
%
% Copyright (C) 2007, 2008, 2010 by
%    Heiko Oberdiek <heiko.oberdiek at googlemail.com>
%    2016
%    https://github.com/ho-tex/oberdiek/issues
%
% This work may be distributed and/or modified under the
% conditions of the LaTeX Project Public License, either
% version 1.3c of this license or (at your option) any later
% version. This version of this license is in
%    http://www.latex-project.org/lppl/lppl-1-3c.txt
% and the latest version of this license is in
%    http://www.latex-project.org/lppl.txt
% and version 1.3 or later is part of all distributions of
% LaTeX version 2005/12/01 or later.
%
% This work has the LPPL maintenance status "maintained".
%
% This Current Maintainer of this work is Heiko Oberdiek.
%
% This work consists of the main source file pdfcolparcolumns.dtx
% and the derived files
%    pdfcolparcolumns.sty, pdfcolparcolumns.pdf, pdfcolparcolumns.ins,
%    pdfcolparcolumns.drv, pdfcolparcolumns-test1.tex.
%
% Distribution:
%    CTAN:macros/latex/contrib/oberdiek/pdfcolparcolumns.dtx
%    CTAN:macros/latex/contrib/oberdiek/pdfcolparcolumns.pdf
%
% Unpacking:
%    (a) If pdfcolparcolumns.ins is present:
%           tex pdfcolparcolumns.ins
%    (b) Without pdfcolparcolumns.ins:
%           tex pdfcolparcolumns.dtx
%    (c) If you insist on using LaTeX
%           latex \let\install=y% \iffalse meta-comment
%
% File: pdfcolparcolumns.dtx
% Version: 2016/05/16 v1.4
% Info: Color stacks for parcolumns
%
% Copyright (C) 2007, 2008, 2010 by
%    Heiko Oberdiek <heiko.oberdiek at googlemail.com>
%    2016
%    https://github.com/ho-tex/oberdiek/issues
%
% This work may be distributed and/or modified under the
% conditions of the LaTeX Project Public License, either
% version 1.3c of this license or (at your option) any later
% version. This version of this license is in
%    http://www.latex-project.org/lppl/lppl-1-3c.txt
% and the latest version of this license is in
%    http://www.latex-project.org/lppl.txt
% and version 1.3 or later is part of all distributions of
% LaTeX version 2005/12/01 or later.
%
% This work has the LPPL maintenance status "maintained".
%
% This Current Maintainer of this work is Heiko Oberdiek.
%
% This work consists of the main source file pdfcolparcolumns.dtx
% and the derived files
%    pdfcolparcolumns.sty, pdfcolparcolumns.pdf, pdfcolparcolumns.ins,
%    pdfcolparcolumns.drv, pdfcolparcolumns-test1.tex.
%
% Distribution:
%    CTAN:macros/latex/contrib/oberdiek/pdfcolparcolumns.dtx
%    CTAN:macros/latex/contrib/oberdiek/pdfcolparcolumns.pdf
%
% Unpacking:
%    (a) If pdfcolparcolumns.ins is present:
%           tex pdfcolparcolumns.ins
%    (b) Without pdfcolparcolumns.ins:
%           tex pdfcolparcolumns.dtx
%    (c) If you insist on using LaTeX
%           latex \let\install=y% \iffalse meta-comment
%
% File: pdfcolparcolumns.dtx
% Version: 2016/05/16 v1.4
% Info: Color stacks for parcolumns
%
% Copyright (C) 2007, 2008, 2010 by
%    Heiko Oberdiek <heiko.oberdiek at googlemail.com>
%    2016
%    https://github.com/ho-tex/oberdiek/issues
%
% This work may be distributed and/or modified under the
% conditions of the LaTeX Project Public License, either
% version 1.3c of this license or (at your option) any later
% version. This version of this license is in
%    http://www.latex-project.org/lppl/lppl-1-3c.txt
% and the latest version of this license is in
%    http://www.latex-project.org/lppl.txt
% and version 1.3 or later is part of all distributions of
% LaTeX version 2005/12/01 or later.
%
% This work has the LPPL maintenance status "maintained".
%
% This Current Maintainer of this work is Heiko Oberdiek.
%
% This work consists of the main source file pdfcolparcolumns.dtx
% and the derived files
%    pdfcolparcolumns.sty, pdfcolparcolumns.pdf, pdfcolparcolumns.ins,
%    pdfcolparcolumns.drv, pdfcolparcolumns-test1.tex.
%
% Distribution:
%    CTAN:macros/latex/contrib/oberdiek/pdfcolparcolumns.dtx
%    CTAN:macros/latex/contrib/oberdiek/pdfcolparcolumns.pdf
%
% Unpacking:
%    (a) If pdfcolparcolumns.ins is present:
%           tex pdfcolparcolumns.ins
%    (b) Without pdfcolparcolumns.ins:
%           tex pdfcolparcolumns.dtx
%    (c) If you insist on using LaTeX
%           latex \let\install=y\input{pdfcolparcolumns.dtx}
%        (quote the arguments according to the demands of your shell)
%
% Documentation:
%    (a) If pdfcolparcolumns.drv is present:
%           latex pdfcolparcolumns.drv
%    (b) Without pdfcolparcolumns.drv:
%           latex pdfcolparcolumns.dtx; ...
%    The class ltxdoc loads the configuration file ltxdoc.cfg
%    if available. Here you can specify further options, e.g.
%    use A4 as paper format:
%       \PassOptionsToClass{a4paper}{article}
%
%    Programm calls to get the documentation (example):
%       pdflatex pdfcolparcolumns.dtx
%       makeindex -s gind.ist pdfcolparcolumns.idx
%       pdflatex pdfcolparcolumns.dtx
%       makeindex -s gind.ist pdfcolparcolumns.idx
%       pdflatex pdfcolparcolumns.dtx
%
% Installation:
%    TDS:tex/latex/oberdiek/pdfcolparcolumns.sty
%    TDS:doc/latex/oberdiek/pdfcolparcolumns.pdf
%    TDS:doc/latex/oberdiek/test/pdfcolparcolumns-test1.tex
%    TDS:source/latex/oberdiek/pdfcolparcolumns.dtx
%
%<*ignore>
\begingroup
  \catcode123=1 %
  \catcode125=2 %
  \def\x{LaTeX2e}%
\expandafter\endgroup
\ifcase 0\ifx\install y1\fi\expandafter
         \ifx\csname processbatchFile\endcsname\relax\else1\fi
         \ifx\fmtname\x\else 1\fi\relax
\else\csname fi\endcsname
%</ignore>
%<*install>
\input docstrip.tex
\Msg{************************************************************************}
\Msg{* Installation}
\Msg{* Package: pdfcolparcolumns 2016/05/16 v1.4 Color stacks for parcolumns (HO)}
\Msg{************************************************************************}

\keepsilent
\askforoverwritefalse

\let\MetaPrefix\relax
\preamble

This is a generated file.

Project: pdfcolparcolumns
Version: 2016/05/16 v1.4

Copyright (C) 2007, 2008, 2010 by
   Heiko Oberdiek <heiko.oberdiek at googlemail.com>

This work may be distributed and/or modified under the
conditions of the LaTeX Project Public License, either
version 1.3c of this license or (at your option) any later
version. This version of this license is in
   http://www.latex-project.org/lppl/lppl-1-3c.txt
and the latest version of this license is in
   http://www.latex-project.org/lppl.txt
and version 1.3 or later is part of all distributions of
LaTeX version 2005/12/01 or later.

This work has the LPPL maintenance status "maintained".

This Current Maintainer of this work is Heiko Oberdiek.

This work consists of the main source file pdfcolparcolumns.dtx
and the derived files
   pdfcolparcolumns.sty, pdfcolparcolumns.pdf, pdfcolparcolumns.ins,
   pdfcolparcolumns.drv, pdfcolparcolumns-test1.tex.

\endpreamble
\let\MetaPrefix\DoubleperCent

\generate{%
  \file{pdfcolparcolumns.ins}{\from{pdfcolparcolumns.dtx}{install}}%
  \file{pdfcolparcolumns.drv}{\from{pdfcolparcolumns.dtx}{driver}}%
  \usedir{tex/latex/oberdiek}%
  \file{pdfcolparcolumns.sty}{\from{pdfcolparcolumns.dtx}{package}}%
  \usedir{doc/latex/oberdiek/test}%
  \file{pdfcolparcolumns-test1.tex}{\from{pdfcolparcolumns.dtx}{test1}}%
  \nopreamble
  \nopostamble
  \usedir{source/latex/oberdiek/catalogue}%
  \file{pdfcolparcolumns.xml}{\from{pdfcolparcolumns.dtx}{catalogue}}%
}

\catcode32=13\relax% active space
\let =\space%
\Msg{************************************************************************}
\Msg{*}
\Msg{* To finish the installation you have to move the following}
\Msg{* file into a directory searched by TeX:}
\Msg{*}
\Msg{*     pdfcolparcolumns.sty}
\Msg{*}
\Msg{* To produce the documentation run the file `pdfcolparcolumns.drv'}
\Msg{* through LaTeX.}
\Msg{*}
\Msg{* Happy TeXing!}
\Msg{*}
\Msg{************************************************************************}

\endbatchfile
%</install>
%<*ignore>
\fi
%</ignore>
%<*driver>
\NeedsTeXFormat{LaTeX2e}
\ProvidesFile{pdfcolparcolumns.drv}%
  [2016/05/16 v1.4 Color stacks for parcolumns (HO)]%
\documentclass{ltxdoc}
\usepackage{holtxdoc}[2011/11/22]
\begin{document}
  \DocInput{pdfcolparcolumns.dtx}%
\end{document}
%</driver>
% \fi
%
%
% \CharacterTable
%  {Upper-case    \A\B\C\D\E\F\G\H\I\J\K\L\M\N\O\P\Q\R\S\T\U\V\W\X\Y\Z
%   Lower-case    \a\b\c\d\e\f\g\h\i\j\k\l\m\n\o\p\q\r\s\t\u\v\w\x\y\z
%   Digits        \0\1\2\3\4\5\6\7\8\9
%   Exclamation   \!     Double quote  \"     Hash (number) \#
%   Dollar        \$     Percent       \%     Ampersand     \&
%   Acute accent  \'     Left paren    \(     Right paren   \)
%   Asterisk      \*     Plus          \+     Comma         \,
%   Minus         \-     Point         \.     Solidus       \/
%   Colon         \:     Semicolon     \;     Less than     \<
%   Equals        \=     Greater than  \>     Question mark \?
%   Commercial at \@     Left bracket  \[     Backslash     \\
%   Right bracket \]     Circumflex    \^     Underscore    \_
%   Grave accent  \`     Left brace    \{     Vertical bar  \|
%   Right brace   \}     Tilde         \~}
%
% \GetFileInfo{pdfcolparcolumns.drv}
%
% \title{The \xpackage{pdfcolparcolumns} package}
% \date{2016/05/16 v1.4}
% \author{Heiko Oberdiek\thanks
% {Please report any issues at https://github.com/ho-tex/oberdiek/issues}\\
% \xemail{heiko.oberdiek at googlemail.com}}
%
% \maketitle
%
% \begin{abstract}
% Since version 1.40 \pdfTeX\ supports several color stacks.
% This package uses them to fix color problems in
% package \xpackage{parcolumns}.
% \end{abstract}
%
% \tableofcontents
%
% \section{Usage}
%
% \begin{quote}
% |\usepackage{pdfcolparcolumns}|
% \end{quote}
% The package \xpackage{pdfcolparcolumns} loads package \xpackage{parcolums}
% \cite{parcolumns}. If color stacks are available then the
% macros of \xpackage{parcolumns} are patched to add support
% for color stacks.
%
% \subsection{Option \xoption{rulebetweencolor}}
%
% Package \xpackage{pdfcolparcolumns} also fixes the color for the
% rule between columns (if \xoption{rulebetween} is set).
% Default color is \cs{normalcolor}. But this can be changed by using
% option \xoption{rulebetweencolor}. It takes a color specification
% as value. If the value is empty, then the default (\cs{normalcolor})
% is used.
% Examples:
% \begin{quote}
%   |rulebetweencolor=blue|,\\
%   |rulebetweencolor={red}|,\\
%   |rulebetweencolor={}|, \textit{\% \cs{normalcolor} is used}\\
%   |rulebetweencolor=[rgb]{1,0,.5}| \textit{\% see below}
% \end{quote}
% If used inside the optional argument of environment \textsf{parcolumns}
% and the value contains an optional argument, then whole value
% must be put in curly braces:
%\begin{quote}
%\begin{verbatim}
%\begin{parcolumns}[
%  rulebetween,
%  rulebetweencolor={[rgb]{1,0,.5}},
%]{2}
%  ...
%\end{parcolumns}
%\end{verbatim}
%\end{quote}
% This option \xoption{rulebetweencolor} can also be set using
% \cs{setkeys}:
%\begin{quote}
%\begin{verbatim}
%\setkeys{parcolumns}{rulebetweencolor=green}
%\end{verbatim}
%\end{quote}
%
% \subsection{Future}
%
% Currently package \xpackage{parcolumns} does not seem to be
% maintained. Nevertheless if there will be a new version that
% adds support for color stacks, then this package may become
% obsolete.
%
% \StopEventually{
% }
%
% \section{Implementation}
%
% \subsection{Identification}
%
%    \begin{macrocode}
%<*package>
\NeedsTeXFormat{LaTeX2e}
\ProvidesPackage{pdfcolparcolumns}%
  [2016/05/16 v1.4 Color stacks for parcolumns (HO)]%
%    \end{macrocode}
%
% \subsection{Load packages}
%
% \subsubsection{Package \xpackage{parcolumns}}
%
%    Currently package \xpackage{parcolumns} does not define options.
%    Thus it is just a precaution that the options of
%    package \xpackage{pdfcolparcolumns} are passed to
%    package \xpackage{parcolumns}.
%    \begin{macrocode}
\DeclareOption*{%
  \PassoptionsToPackage{\CurrentOption}{parcolumns}%
}
\ProcessOptions\relax
\RequirePackage{parcolumns}[2004/11/25]
%    \end{macrocode}
%
% \subsubsection{Package \xpackage{pdfcol}}
%
%    \begin{macrocode}
\RequirePackage{pdfcol}[2007/09/09]
\ifpdfcolAvailable
\else
  \PackageInfo{pdfcolparcolumns}{%
    Loading aborted, because color stacks are not available%
  }%
  \expandafter\endinput
\fi
%    \end{macrocode}
%
% \subsubsection{Package \xpackage{infwarerr}}
%
%    \begin{macrocode}
\RequirePackage{infwarerr}[2007/09/09]
%    \end{macrocode}
%
% \subsection{Color stack macros}
%
%    \begin{macro}{\pcpc@MaxStack}
%    Macro \cs{pcpc@MaxStack} holds the highest number of
%    allocated stacks.
%    \begin{macrocode}
\global\chardef\pcpc@MaxStack=\z@
%    \end{macrocode}
%    \end{macro}
%    \begin{macro}{\pcpc@InitStacks}
%    Macro \cs{pcpc@InitStacks} takes the number of columns
%    as argument and ensures that there are enough color
%    stacks for all columns.
%    \begin{macrocode}
\def\pcpc@InitStacks#1{%
  \ifnum#1>\pcpc@MaxStack
    \begingroup
      \count@\pcpc@MaxStack
      \loop
        \advance\count@\@ne
        \pdfcolInitStack{pcpc@\the\count@}%
      \ifnum#1>\count@
      \repeat
      \global\chardef\pcpc@MaxStack=\count@
    \endgroup
  \fi
}
%    \end{macrocode}
%    \end{macro}
%
%    \begin{macro}{\pcpc@SwitchStack}
%    \begin{macrocode}
\def\pcpc@SwitchStack#1{%
  \pdfcolSwitchStack{pcpc@\number#1}%
}
%    \end{macrocode}
%    \end{macro}
%
%    \begin{macro}{\pcpc@SetCurrent}
%    \begin{macrocode}
\def\pcpc@SetCurrent#1{%
  \pdfcolSetCurrent{pcpc@\number#1}%
}
%    \end{macrocode}
%    \end{macro}
%
% \subsection{Patches}
%
%     Now the color stack macros are patched into the macros
%     of package \xpackage{parcolumns}.
%
% \subsubsection{Init stacks}
%
%    \cs{pcpc@InitStacks} should go into the definition of
%    environment |parcolumns|. \cs{pc@alloccolumns} is executed
%    there and nowhere else, thus we hook into it.
%    \begin{macrocode}
\g@addto@macro\pc@alloccolumns{%
  \pcpc@InitStacks\pc@columncount
}
%    \end{macrocode}
%
% \subsubsection{Switch stack}
%
%    \cs{pcpc@SwitchStack} should be called by marco \cs{colchunk@}.
%    However it is easier to patch \cs{pc@setcolumnwidth} that
%    is executed in \cs{colchunk@} only.
%    \begin{macrocode}
\g@addto@macro\pc@setcolumnwidth{%
  \pcpc@SwitchStack\pc@columnctr
}
%    \end{macrocode}
%
% \subsubsection{Set current stack color}
%
%    \cs{pcpc@SetCurrent} is set at the begin of each line.
%    It must be inserted into \cs{pc@placeboxes}. Unhappily
%    there is no easy way. Therefore we check and
%    redefine \cs{pc@placeboxes}.
%    \begin{macrocode}
\begingroup
  \def\x{%
    \global\let\@tempa\relax
    \count@\z@
    \hb@xt@\linewidth{%
      \vfuzz30ex %
      \vbadness\@M
      \splittopskip\z@skip
      \loop
      \ifnum\count@<\pc@columncount
        \advance\count@\@ne
        \expandafter\ifvoid\csname pc@column@\number\count@\endcsname
          \hskip\csname pc@column@width@\number\count@\endcsname
        \else
          \expandafter\setbox\expandafter\@tempboxa\expandafter
          \vsplit\csname pc@column@\number\count@\endcsname
              to \dp\strutbox
          \vbox{%
            \unvbox\@tempboxa
          }%
        \fi
        \expandafter\ifvoid\csname pc@column@\number\count@\endcsname
        \else
          \global\let\@tempa\pc@placeboxes
        \fi
        \ifnum\count@<\pc@columncount
          \strut
          \hfill
          \ifpc@rulebetween
            \vrule
            \hfill
          \fi
        \fi
      \repeat
    }%
    \@tempa
  }%
  \ifx\x\pc@placeboxes
  \else
    \@PackageWarningNoLine{pdfcolparcolumns}{%
      Command \string\pc@placeboxes\space has changed.\MessageBreak
      Supported versions of package `parcolumns':\MessageBreak
      \space\space 2004/08/05.\MessageBreak
      The redefinition of \string\pc@placeboxes\space may not%
      \MessageBreak
      behave correctly depending on the changes%
    }%
  \fi
\endgroup
%    \end{macrocode}
%    \begin{macro}{\pc@placeboxes}
%    \begin{macrocode}
\renewcommand*{\pc@placeboxes}{%
  \global\let\@tempa\relax
  \count@\z@
  \hb@xt@\linewidth{%
    \vfuzz30ex %
    \vbadness\@M
    \splittopskip\z@skip
    \loop
    \ifnum\count@<\pc@columncount
      \advance\count@\@ne
      \expandafter\ifvoid\csname pc@column@\number\count@\endcsname
        \hskip\csname pc@column@width@\number\count@\endcsname
      \else
        \expandafter\setbox\expandafter\@tempboxa\expandafter
        \vsplit\csname pc@column@\number\count@\endcsname
            to \dp\strutbox
        \vbox{%
          \pcpc@SetCurrent\count@
          \unvbox\@tempboxa
        }%
      \fi
      \expandafter\ifvoid\csname pc@column@\number\count@\endcsname
      \else
        \global\let\@tempa\pc@placeboxes
      \fi
      \ifnum\count@<\pc@columncount
        \strut
        \hfill
        \ifpc@rulebetween
          \begingroup
            \pcpc@RuleBetweenColor
            \vrule
          \endgroup
          \hfill
        \fi
      \fi
    \repeat
  }%
  \@tempa
}
%    \end{macrocode}
%    \end{macro}
%    \begin{macro}{\pcpc@RuleBetweenColorDefault}
%    \begin{macrocode}
\def\pcpc@RuleBetweenColorDefault{%
  \normalcolor
}
%    \end{macrocode}
%    \end{macro}
%    \begin{macro}{\pcpc@RuleBetweenColor}
%    \begin{macrocode}
\let\pcpc@RuleBetweenColor\pcpc@RuleBetweenColorDefault
%    \end{macrocode}
%    \end{macro}
%    \begin{macrocode}
\define@key{parcolumns}{rulebetweencolor}{%
  \edef\pcpc@temp{#1}%
  \ifx\pcpc@temp\@empty
    \let\pcpc@RuleBetweenColor\pcpc@RuleBetweenColorDefault
  \else
    \edef\pcpc@temp{%
      \noexpand\@ifnextchar[{%
        \def\noexpand\pcpc@RuleBetweenColor{%
          \noexpand\color\pcpc@temp
        }%
        \noexpand\pcpc@GobbleNil
      }{%
        \def\noexpand\pcpc@RuleBetweenColor{%
          \noexpand\color{\pcpc@temp}%
        }%
        \noexpand\pcpc@GobbleNil
      }%
      \pcpc@temp\noexpand\@nil
    }%
    \pcpc@temp
  \fi
}
%    \end{macrocode}
%    \begin{macro}{\pcpc@GobbleNil}
%    \begin{macrocode}
\long\def\pcpc@GobbleNil#1\@nil{}
%    \end{macrocode}
%    \end{macro}
%
%    \begin{macrocode}
%</package>
%    \end{macrocode}
%
% \section{Test}
%
%    The test file is a modified version of the file that
%    Donald Goodman has posted in \xnewsgroup{comp.text.tex}: ^^A
%    \URL{``\link{Re: \xpackage{xcolor} glitches}''}^^A
%    {http://groups.google.com/group/comp.text.tex/msg/8eda74ed292012bb}
%    \begin{macrocode}
%<*test1>
\NeedsTeXFormat{LaTeX2e}
\AtEndDocument{%
  \typeout{}%
  \typeout{**************************************}%
  \typeout{*** \space Check the PDF file manually! \space ***}%
  \typeout{**************************************}%
  \typeout{}%
}
\documentclass{article}
\usepackage{xcolor}
\usepackage{pdfcolparcolumns}

\newcommand{\instruct}[1]{%
  \noindent
  \footnotesize
  \textcolor{red}{#1}%
}

\begin{document}
  \begin{parcolumns}[colwidths={1=2.3in,2=2.3in},sloppy]{2}%
    \colchunk[1]{%
      \instruct{Et non dicitur versus} %
      Fidelium anim\ae\ %
      \instruct{%
        sed immediate subiungitur antiphona finalis %
        beat\ae\ Mari\ae\ Virginis%
      } %
      100.%
    }%
    \colchunk[2]{%
      \instruct{%
        And the verse %
        \textcolor{black}{May the souls of the faithful} %
        is not said, but the final antiphon of the %
        Blessed Virgin Mary, %
        \textcolor{black}{100,} %
        is immediately joined.%
      }%
    }%
  \end{parcolumns}%
\end{document}
%</test1>
%    \end{macrocode}
%
% \section{Installation}
%
% \subsection{Download}
%
% \paragraph{Package.} This package is available on
% CTAN\footnote{\url{http://ctan.org/pkg/pdfcolparcolumns}}:
% \begin{description}
% \item[\CTAN{macros/latex/contrib/oberdiek/pdfcolparcolumns.dtx}] The source file.
% \item[\CTAN{macros/latex/contrib/oberdiek/pdfcolparcolumns.pdf}] Documentation.
% \end{description}
%
%
% \paragraph{Bundle.} All the packages of the bundle `oberdiek'
% are also available in a TDS compliant ZIP archive. There
% the packages are already unpacked and the documentation files
% are generated. The files and directories obey the TDS standard.
% \begin{description}
% \item[\CTAN{install/macros/latex/contrib/oberdiek.tds.zip}]
% \end{description}
% \emph{TDS} refers to the standard ``A Directory Structure
% for \TeX\ Files'' (\CTAN{tds/tds.pdf}). Directories
% with \xfile{texmf} in their name are usually organized this way.
%
% \subsection{Bundle installation}
%
% \paragraph{Unpacking.} Unpack the \xfile{oberdiek.tds.zip} in the
% TDS tree (also known as \xfile{texmf} tree) of your choice.
% Example (linux):
% \begin{quote}
%   |unzip oberdiek.tds.zip -d ~/texmf|
% \end{quote}
%
% \paragraph{Script installation.}
% Check the directory \xfile{TDS:scripts/oberdiek/} for
% scripts that need further installation steps.
% Package \xpackage{attachfile2} comes with the Perl script
% \xfile{pdfatfi.pl} that should be installed in such a way
% that it can be called as \texttt{pdfatfi}.
% Example (linux):
% \begin{quote}
%   |chmod +x scripts/oberdiek/pdfatfi.pl|\\
%   |cp scripts/oberdiek/pdfatfi.pl /usr/local/bin/|
% \end{quote}
%
% \subsection{Package installation}
%
% \paragraph{Unpacking.} The \xfile{.dtx} file is a self-extracting
% \docstrip\ archive. The files are extracted by running the
% \xfile{.dtx} through \plainTeX:
% \begin{quote}
%   \verb|tex pdfcolparcolumns.dtx|
% \end{quote}
%
% \paragraph{TDS.} Now the different files must be moved into
% the different directories in your installation TDS tree
% (also known as \xfile{texmf} tree):
% \begin{quote}
% \def\t{^^A
% \begin{tabular}{@{}>{\ttfamily}l@{ $\rightarrow$ }>{\ttfamily}l@{}}
%   pdfcolparcolumns.sty & tex/latex/oberdiek/pdfcolparcolumns.sty\\
%   pdfcolparcolumns.pdf & doc/latex/oberdiek/pdfcolparcolumns.pdf\\
%   test/pdfcolparcolumns-test1.tex & doc/latex/oberdiek/test/pdfcolparcolumns-test1.tex\\
%   pdfcolparcolumns.dtx & source/latex/oberdiek/pdfcolparcolumns.dtx\\
% \end{tabular}^^A
% }^^A
% \sbox0{\t}^^A
% \ifdim\wd0>\linewidth
%   \begingroup
%     \advance\linewidth by\leftmargin
%     \advance\linewidth by\rightmargin
%   \edef\x{\endgroup
%     \def\noexpand\lw{\the\linewidth}^^A
%   }\x
%   \def\lwbox{^^A
%     \leavevmode
%     \hbox to \linewidth{^^A
%       \kern-\leftmargin\relax
%       \hss
%       \usebox0
%       \hss
%       \kern-\rightmargin\relax
%     }^^A
%   }^^A
%   \ifdim\wd0>\lw
%     \sbox0{\small\t}^^A
%     \ifdim\wd0>\linewidth
%       \ifdim\wd0>\lw
%         \sbox0{\footnotesize\t}^^A
%         \ifdim\wd0>\linewidth
%           \ifdim\wd0>\lw
%             \sbox0{\scriptsize\t}^^A
%             \ifdim\wd0>\linewidth
%               \ifdim\wd0>\lw
%                 \sbox0{\tiny\t}^^A
%                 \ifdim\wd0>\linewidth
%                   \lwbox
%                 \else
%                   \usebox0
%                 \fi
%               \else
%                 \lwbox
%               \fi
%             \else
%               \usebox0
%             \fi
%           \else
%             \lwbox
%           \fi
%         \else
%           \usebox0
%         \fi
%       \else
%         \lwbox
%       \fi
%     \else
%       \usebox0
%     \fi
%   \else
%     \lwbox
%   \fi
% \else
%   \usebox0
% \fi
% \end{quote}
% If you have a \xfile{docstrip.cfg} that configures and enables \docstrip's
% TDS installing feature, then some files can already be in the right
% place, see the documentation of \docstrip.
%
% \subsection{Refresh file name databases}
%
% If your \TeX~distribution
% (\teTeX, \mikTeX, \dots) relies on file name databases, you must refresh
% these. For example, \teTeX\ users run \verb|texhash| or
% \verb|mktexlsr|.
%
% \subsection{Some details for the interested}
%
% \paragraph{Attached source.}
%
% The PDF documentation on CTAN also includes the
% \xfile{.dtx} source file. It can be extracted by
% AcrobatReader 6 or higher. Another option is \textsf{pdftk},
% e.g. unpack the file into the current directory:
% \begin{quote}
%   \verb|pdftk pdfcolparcolumns.pdf unpack_files output .|
% \end{quote}
%
% \paragraph{Unpacking with \LaTeX.}
% The \xfile{.dtx} chooses its action depending on the format:
% \begin{description}
% \item[\plainTeX:] Run \docstrip\ and extract the files.
% \item[\LaTeX:] Generate the documentation.
% \end{description}
% If you insist on using \LaTeX\ for \docstrip\ (really,
% \docstrip\ does not need \LaTeX), then inform the autodetect routine
% about your intention:
% \begin{quote}
%   \verb|latex \let\install=y\input{pdfcolparcolumns.dtx}|
% \end{quote}
% Do not forget to quote the argument according to the demands
% of your shell.
%
% \paragraph{Generating the documentation.}
% You can use both the \xfile{.dtx} or the \xfile{.drv} to generate
% the documentation. The process can be configured by the
% configuration file \xfile{ltxdoc.cfg}. For instance, put this
% line into this file, if you want to have A4 as paper format:
% \begin{quote}
%   \verb|\PassOptionsToClass{a4paper}{article}|
% \end{quote}
% An example follows how to generate the
% documentation with pdf\LaTeX:
% \begin{quote}
%\begin{verbatim}
%pdflatex pdfcolparcolumns.dtx
%makeindex -s gind.ist pdfcolparcolumns.idx
%pdflatex pdfcolparcolumns.dtx
%makeindex -s gind.ist pdfcolparcolumns.idx
%pdflatex pdfcolparcolumns.dtx
%\end{verbatim}
% \end{quote}
%
% \section{Catalogue}
%
% The following XML file can be used as source for the
% \href{http://mirror.ctan.org/help/Catalogue/catalogue.html}{\TeX\ Catalogue}.
% The elements \texttt{caption} and \texttt{description} are imported
% from the original XML file from the Catalogue.
% The name of the XML file in the Catalogue is \xfile{pdfcolparcolumns.xml}.
%    \begin{macrocode}
%<*catalogue>
<?xml version='1.0' encoding='us-ascii'?>
<!DOCTYPE entry SYSTEM 'catalogue.dtd'>
<entry datestamp='$Date$' modifier='$Author$' id='pdfcolparcolumns'>
  <name>pdfcolparcolumns</name>
  <caption>Fix colour problems in package 'parcolumns'.</caption>
  <authorref id='auth:oberdiek'/>
  <copyright owner='Heiko Oberdiek' year='2007,2008,2010'/>
  <license type='lppl1.3'/>
  <version number='1.4'/>
  <description>
    Since version 1.40 pdfTeX supports colour stacks.
    This package uses them to fix colour problems in
    package <xref refid='parcolumns'>parcolumns</xref>.
    <p/>
    The package is part of the <xref refid='oberdiek'>oberdiek</xref>
    bundle.
  </description>
  <documentation details='Package documentation'
      href='ctan:/macros/latex/contrib/oberdiek/pdfcolparcolumns.pdf'/>
  <ctan file='true' path='/macros/latex/contrib/oberdiek/pdfcolparcolumns.dtx'/>
  <miktex location='oberdiek'/>
  <texlive location='oberdiek'/>
  <install path='/macros/latex/contrib/oberdiek/oberdiek.tds.zip'/>
</entry>
%</catalogue>
%    \end{macrocode}
%
% \begin{thebibliography}{9}
%
% \bibitem{parcolumns}
%   Jonathan Sauer: \textit{The \xpackage{parcolumns} package};
%   2004/11/25;\\
%   \CTAN{macros/latex/contrib/sauerj/parcolumns.pdf}.
%
% \bibitem{pdfcol}
%   Heiko Oberdiek: \textit{The \xpackage{pdfcol} package};
%   2007/09/09;\\
%   \CTAN{macros/latex/contrib/oberdiek/pdfcol.pdf}.
%
% \end{thebibliography}
%
% \begin{History}
%   \begin{Version}{2007/07/26 v1.0}
%   \item
%     First version, published in the newsgroup \xnewsgroup{comp.text.tex}
%     with the name \xpackage{parcolumns-colorstacks}: ^^A no line break
%     \URL{``\link{Re: \xpackage{xcolor} glitches}''}^^A
%     {http://groups.google.com/group/comp.text.tex/msg/56bd897b11bca414}
%   \end{Version}
%   \begin{Version}{2007/09/09 v1.1}
%   \item
%     CTAN version, package name renamed to \xpackage{pdfcolparcolumns}.
%   \item
%     Uses package \xpackage{pdfcol}.
%   \item
%     Documentation added.
%   \item
%     Test file added.
%   \end{Version}
%   \begin{Version}{2008/08/11 v1.2}
%   \item
%     Code is not changed.
%   \item
%     URLs updated.
%   \end{Version}
%   \begin{Version}{2010/01/11 v1.3}
%   \item
%     Fix for rule color.
%   \item
%     New option \xoption{rulebetweencolor} for environment |parcolumns|.
%   \end{Version}
%   \begin{Version}{2016/05/16 v1.4}
%   \item
%     Documentation updates.
%   \end{Version}
% \end{History}
%
% \PrintIndex
%
% \Finale
\endinput

%        (quote the arguments according to the demands of your shell)
%
% Documentation:
%    (a) If pdfcolparcolumns.drv is present:
%           latex pdfcolparcolumns.drv
%    (b) Without pdfcolparcolumns.drv:
%           latex pdfcolparcolumns.dtx; ...
%    The class ltxdoc loads the configuration file ltxdoc.cfg
%    if available. Here you can specify further options, e.g.
%    use A4 as paper format:
%       \PassOptionsToClass{a4paper}{article}
%
%    Programm calls to get the documentation (example):
%       pdflatex pdfcolparcolumns.dtx
%       makeindex -s gind.ist pdfcolparcolumns.idx
%       pdflatex pdfcolparcolumns.dtx
%       makeindex -s gind.ist pdfcolparcolumns.idx
%       pdflatex pdfcolparcolumns.dtx
%
% Installation:
%    TDS:tex/latex/oberdiek/pdfcolparcolumns.sty
%    TDS:doc/latex/oberdiek/pdfcolparcolumns.pdf
%    TDS:doc/latex/oberdiek/test/pdfcolparcolumns-test1.tex
%    TDS:source/latex/oberdiek/pdfcolparcolumns.dtx
%
%<*ignore>
\begingroup
  \catcode123=1 %
  \catcode125=2 %
  \def\x{LaTeX2e}%
\expandafter\endgroup
\ifcase 0\ifx\install y1\fi\expandafter
         \ifx\csname processbatchFile\endcsname\relax\else1\fi
         \ifx\fmtname\x\else 1\fi\relax
\else\csname fi\endcsname
%</ignore>
%<*install>
\input docstrip.tex
\Msg{************************************************************************}
\Msg{* Installation}
\Msg{* Package: pdfcolparcolumns 2016/05/16 v1.4 Color stacks for parcolumns (HO)}
\Msg{************************************************************************}

\keepsilent
\askforoverwritefalse

\let\MetaPrefix\relax
\preamble

This is a generated file.

Project: pdfcolparcolumns
Version: 2016/05/16 v1.4

Copyright (C) 2007, 2008, 2010 by
   Heiko Oberdiek <heiko.oberdiek at googlemail.com>

This work may be distributed and/or modified under the
conditions of the LaTeX Project Public License, either
version 1.3c of this license or (at your option) any later
version. This version of this license is in
   http://www.latex-project.org/lppl/lppl-1-3c.txt
and the latest version of this license is in
   http://www.latex-project.org/lppl.txt
and version 1.3 or later is part of all distributions of
LaTeX version 2005/12/01 or later.

This work has the LPPL maintenance status "maintained".

This Current Maintainer of this work is Heiko Oberdiek.

This work consists of the main source file pdfcolparcolumns.dtx
and the derived files
   pdfcolparcolumns.sty, pdfcolparcolumns.pdf, pdfcolparcolumns.ins,
   pdfcolparcolumns.drv, pdfcolparcolumns-test1.tex.

\endpreamble
\let\MetaPrefix\DoubleperCent

\generate{%
  \file{pdfcolparcolumns.ins}{\from{pdfcolparcolumns.dtx}{install}}%
  \file{pdfcolparcolumns.drv}{\from{pdfcolparcolumns.dtx}{driver}}%
  \usedir{tex/latex/oberdiek}%
  \file{pdfcolparcolumns.sty}{\from{pdfcolparcolumns.dtx}{package}}%
  \usedir{doc/latex/oberdiek/test}%
  \file{pdfcolparcolumns-test1.tex}{\from{pdfcolparcolumns.dtx}{test1}}%
  \nopreamble
  \nopostamble
  \usedir{source/latex/oberdiek/catalogue}%
  \file{pdfcolparcolumns.xml}{\from{pdfcolparcolumns.dtx}{catalogue}}%
}

\catcode32=13\relax% active space
\let =\space%
\Msg{************************************************************************}
\Msg{*}
\Msg{* To finish the installation you have to move the following}
\Msg{* file into a directory searched by TeX:}
\Msg{*}
\Msg{*     pdfcolparcolumns.sty}
\Msg{*}
\Msg{* To produce the documentation run the file `pdfcolparcolumns.drv'}
\Msg{* through LaTeX.}
\Msg{*}
\Msg{* Happy TeXing!}
\Msg{*}
\Msg{************************************************************************}

\endbatchfile
%</install>
%<*ignore>
\fi
%</ignore>
%<*driver>
\NeedsTeXFormat{LaTeX2e}
\ProvidesFile{pdfcolparcolumns.drv}%
  [2016/05/16 v1.4 Color stacks for parcolumns (HO)]%
\documentclass{ltxdoc}
\usepackage{holtxdoc}[2011/11/22]
\begin{document}
  \DocInput{pdfcolparcolumns.dtx}%
\end{document}
%</driver>
% \fi
%
%
% \CharacterTable
%  {Upper-case    \A\B\C\D\E\F\G\H\I\J\K\L\M\N\O\P\Q\R\S\T\U\V\W\X\Y\Z
%   Lower-case    \a\b\c\d\e\f\g\h\i\j\k\l\m\n\o\p\q\r\s\t\u\v\w\x\y\z
%   Digits        \0\1\2\3\4\5\6\7\8\9
%   Exclamation   \!     Double quote  \"     Hash (number) \#
%   Dollar        \$     Percent       \%     Ampersand     \&
%   Acute accent  \'     Left paren    \(     Right paren   \)
%   Asterisk      \*     Plus          \+     Comma         \,
%   Minus         \-     Point         \.     Solidus       \/
%   Colon         \:     Semicolon     \;     Less than     \<
%   Equals        \=     Greater than  \>     Question mark \?
%   Commercial at \@     Left bracket  \[     Backslash     \\
%   Right bracket \]     Circumflex    \^     Underscore    \_
%   Grave accent  \`     Left brace    \{     Vertical bar  \|
%   Right brace   \}     Tilde         \~}
%
% \GetFileInfo{pdfcolparcolumns.drv}
%
% \title{The \xpackage{pdfcolparcolumns} package}
% \date{2016/05/16 v1.4}
% \author{Heiko Oberdiek\thanks
% {Please report any issues at https://github.com/ho-tex/oberdiek/issues}\\
% \xemail{heiko.oberdiek at googlemail.com}}
%
% \maketitle
%
% \begin{abstract}
% Since version 1.40 \pdfTeX\ supports several color stacks.
% This package uses them to fix color problems in
% package \xpackage{parcolumns}.
% \end{abstract}
%
% \tableofcontents
%
% \section{Usage}
%
% \begin{quote}
% |\usepackage{pdfcolparcolumns}|
% \end{quote}
% The package \xpackage{pdfcolparcolumns} loads package \xpackage{parcolums}
% \cite{parcolumns}. If color stacks are available then the
% macros of \xpackage{parcolumns} are patched to add support
% for color stacks.
%
% \subsection{Option \xoption{rulebetweencolor}}
%
% Package \xpackage{pdfcolparcolumns} also fixes the color for the
% rule between columns (if \xoption{rulebetween} is set).
% Default color is \cs{normalcolor}. But this can be changed by using
% option \xoption{rulebetweencolor}. It takes a color specification
% as value. If the value is empty, then the default (\cs{normalcolor})
% is used.
% Examples:
% \begin{quote}
%   |rulebetweencolor=blue|,\\
%   |rulebetweencolor={red}|,\\
%   |rulebetweencolor={}|, \textit{\% \cs{normalcolor} is used}\\
%   |rulebetweencolor=[rgb]{1,0,.5}| \textit{\% see below}
% \end{quote}
% If used inside the optional argument of environment \textsf{parcolumns}
% and the value contains an optional argument, then whole value
% must be put in curly braces:
%\begin{quote}
%\begin{verbatim}
%\begin{parcolumns}[
%  rulebetween,
%  rulebetweencolor={[rgb]{1,0,.5}},
%]{2}
%  ...
%\end{parcolumns}
%\end{verbatim}
%\end{quote}
% This option \xoption{rulebetweencolor} can also be set using
% \cs{setkeys}:
%\begin{quote}
%\begin{verbatim}
%\setkeys{parcolumns}{rulebetweencolor=green}
%\end{verbatim}
%\end{quote}
%
% \subsection{Future}
%
% Currently package \xpackage{parcolumns} does not seem to be
% maintained. Nevertheless if there will be a new version that
% adds support for color stacks, then this package may become
% obsolete.
%
% \StopEventually{
% }
%
% \section{Implementation}
%
% \subsection{Identification}
%
%    \begin{macrocode}
%<*package>
\NeedsTeXFormat{LaTeX2e}
\ProvidesPackage{pdfcolparcolumns}%
  [2016/05/16 v1.4 Color stacks for parcolumns (HO)]%
%    \end{macrocode}
%
% \subsection{Load packages}
%
% \subsubsection{Package \xpackage{parcolumns}}
%
%    Currently package \xpackage{parcolumns} does not define options.
%    Thus it is just a precaution that the options of
%    package \xpackage{pdfcolparcolumns} are passed to
%    package \xpackage{parcolumns}.
%    \begin{macrocode}
\DeclareOption*{%
  \PassoptionsToPackage{\CurrentOption}{parcolumns}%
}
\ProcessOptions\relax
\RequirePackage{parcolumns}[2004/11/25]
%    \end{macrocode}
%
% \subsubsection{Package \xpackage{pdfcol}}
%
%    \begin{macrocode}
\RequirePackage{pdfcol}[2007/09/09]
\ifpdfcolAvailable
\else
  \PackageInfo{pdfcolparcolumns}{%
    Loading aborted, because color stacks are not available%
  }%
  \expandafter\endinput
\fi
%    \end{macrocode}
%
% \subsubsection{Package \xpackage{infwarerr}}
%
%    \begin{macrocode}
\RequirePackage{infwarerr}[2007/09/09]
%    \end{macrocode}
%
% \subsection{Color stack macros}
%
%    \begin{macro}{\pcpc@MaxStack}
%    Macro \cs{pcpc@MaxStack} holds the highest number of
%    allocated stacks.
%    \begin{macrocode}
\global\chardef\pcpc@MaxStack=\z@
%    \end{macrocode}
%    \end{macro}
%    \begin{macro}{\pcpc@InitStacks}
%    Macro \cs{pcpc@InitStacks} takes the number of columns
%    as argument and ensures that there are enough color
%    stacks for all columns.
%    \begin{macrocode}
\def\pcpc@InitStacks#1{%
  \ifnum#1>\pcpc@MaxStack
    \begingroup
      \count@\pcpc@MaxStack
      \loop
        \advance\count@\@ne
        \pdfcolInitStack{pcpc@\the\count@}%
      \ifnum#1>\count@
      \repeat
      \global\chardef\pcpc@MaxStack=\count@
    \endgroup
  \fi
}
%    \end{macrocode}
%    \end{macro}
%
%    \begin{macro}{\pcpc@SwitchStack}
%    \begin{macrocode}
\def\pcpc@SwitchStack#1{%
  \pdfcolSwitchStack{pcpc@\number#1}%
}
%    \end{macrocode}
%    \end{macro}
%
%    \begin{macro}{\pcpc@SetCurrent}
%    \begin{macrocode}
\def\pcpc@SetCurrent#1{%
  \pdfcolSetCurrent{pcpc@\number#1}%
}
%    \end{macrocode}
%    \end{macro}
%
% \subsection{Patches}
%
%     Now the color stack macros are patched into the macros
%     of package \xpackage{parcolumns}.
%
% \subsubsection{Init stacks}
%
%    \cs{pcpc@InitStacks} should go into the definition of
%    environment |parcolumns|. \cs{pc@alloccolumns} is executed
%    there and nowhere else, thus we hook into it.
%    \begin{macrocode}
\g@addto@macro\pc@alloccolumns{%
  \pcpc@InitStacks\pc@columncount
}
%    \end{macrocode}
%
% \subsubsection{Switch stack}
%
%    \cs{pcpc@SwitchStack} should be called by marco \cs{colchunk@}.
%    However it is easier to patch \cs{pc@setcolumnwidth} that
%    is executed in \cs{colchunk@} only.
%    \begin{macrocode}
\g@addto@macro\pc@setcolumnwidth{%
  \pcpc@SwitchStack\pc@columnctr
}
%    \end{macrocode}
%
% \subsubsection{Set current stack color}
%
%    \cs{pcpc@SetCurrent} is set at the begin of each line.
%    It must be inserted into \cs{pc@placeboxes}. Unhappily
%    there is no easy way. Therefore we check and
%    redefine \cs{pc@placeboxes}.
%    \begin{macrocode}
\begingroup
  \def\x{%
    \global\let\@tempa\relax
    \count@\z@
    \hb@xt@\linewidth{%
      \vfuzz30ex %
      \vbadness\@M
      \splittopskip\z@skip
      \loop
      \ifnum\count@<\pc@columncount
        \advance\count@\@ne
        \expandafter\ifvoid\csname pc@column@\number\count@\endcsname
          \hskip\csname pc@column@width@\number\count@\endcsname
        \else
          \expandafter\setbox\expandafter\@tempboxa\expandafter
          \vsplit\csname pc@column@\number\count@\endcsname
              to \dp\strutbox
          \vbox{%
            \unvbox\@tempboxa
          }%
        \fi
        \expandafter\ifvoid\csname pc@column@\number\count@\endcsname
        \else
          \global\let\@tempa\pc@placeboxes
        \fi
        \ifnum\count@<\pc@columncount
          \strut
          \hfill
          \ifpc@rulebetween
            \vrule
            \hfill
          \fi
        \fi
      \repeat
    }%
    \@tempa
  }%
  \ifx\x\pc@placeboxes
  \else
    \@PackageWarningNoLine{pdfcolparcolumns}{%
      Command \string\pc@placeboxes\space has changed.\MessageBreak
      Supported versions of package `parcolumns':\MessageBreak
      \space\space 2004/08/05.\MessageBreak
      The redefinition of \string\pc@placeboxes\space may not%
      \MessageBreak
      behave correctly depending on the changes%
    }%
  \fi
\endgroup
%    \end{macrocode}
%    \begin{macro}{\pc@placeboxes}
%    \begin{macrocode}
\renewcommand*{\pc@placeboxes}{%
  \global\let\@tempa\relax
  \count@\z@
  \hb@xt@\linewidth{%
    \vfuzz30ex %
    \vbadness\@M
    \splittopskip\z@skip
    \loop
    \ifnum\count@<\pc@columncount
      \advance\count@\@ne
      \expandafter\ifvoid\csname pc@column@\number\count@\endcsname
        \hskip\csname pc@column@width@\number\count@\endcsname
      \else
        \expandafter\setbox\expandafter\@tempboxa\expandafter
        \vsplit\csname pc@column@\number\count@\endcsname
            to \dp\strutbox
        \vbox{%
          \pcpc@SetCurrent\count@
          \unvbox\@tempboxa
        }%
      \fi
      \expandafter\ifvoid\csname pc@column@\number\count@\endcsname
      \else
        \global\let\@tempa\pc@placeboxes
      \fi
      \ifnum\count@<\pc@columncount
        \strut
        \hfill
        \ifpc@rulebetween
          \begingroup
            \pcpc@RuleBetweenColor
            \vrule
          \endgroup
          \hfill
        \fi
      \fi
    \repeat
  }%
  \@tempa
}
%    \end{macrocode}
%    \end{macro}
%    \begin{macro}{\pcpc@RuleBetweenColorDefault}
%    \begin{macrocode}
\def\pcpc@RuleBetweenColorDefault{%
  \normalcolor
}
%    \end{macrocode}
%    \end{macro}
%    \begin{macro}{\pcpc@RuleBetweenColor}
%    \begin{macrocode}
\let\pcpc@RuleBetweenColor\pcpc@RuleBetweenColorDefault
%    \end{macrocode}
%    \end{macro}
%    \begin{macrocode}
\define@key{parcolumns}{rulebetweencolor}{%
  \edef\pcpc@temp{#1}%
  \ifx\pcpc@temp\@empty
    \let\pcpc@RuleBetweenColor\pcpc@RuleBetweenColorDefault
  \else
    \edef\pcpc@temp{%
      \noexpand\@ifnextchar[{%
        \def\noexpand\pcpc@RuleBetweenColor{%
          \noexpand\color\pcpc@temp
        }%
        \noexpand\pcpc@GobbleNil
      }{%
        \def\noexpand\pcpc@RuleBetweenColor{%
          \noexpand\color{\pcpc@temp}%
        }%
        \noexpand\pcpc@GobbleNil
      }%
      \pcpc@temp\noexpand\@nil
    }%
    \pcpc@temp
  \fi
}
%    \end{macrocode}
%    \begin{macro}{\pcpc@GobbleNil}
%    \begin{macrocode}
\long\def\pcpc@GobbleNil#1\@nil{}
%    \end{macrocode}
%    \end{macro}
%
%    \begin{macrocode}
%</package>
%    \end{macrocode}
%
% \section{Test}
%
%    The test file is a modified version of the file that
%    Donald Goodman has posted in \xnewsgroup{comp.text.tex}: ^^A
%    \URL{``\link{Re: \xpackage{xcolor} glitches}''}^^A
%    {http://groups.google.com/group/comp.text.tex/msg/8eda74ed292012bb}
%    \begin{macrocode}
%<*test1>
\NeedsTeXFormat{LaTeX2e}
\AtEndDocument{%
  \typeout{}%
  \typeout{**************************************}%
  \typeout{*** \space Check the PDF file manually! \space ***}%
  \typeout{**************************************}%
  \typeout{}%
}
\documentclass{article}
\usepackage{xcolor}
\usepackage{pdfcolparcolumns}

\newcommand{\instruct}[1]{%
  \noindent
  \footnotesize
  \textcolor{red}{#1}%
}

\begin{document}
  \begin{parcolumns}[colwidths={1=2.3in,2=2.3in},sloppy]{2}%
    \colchunk[1]{%
      \instruct{Et non dicitur versus} %
      Fidelium anim\ae\ %
      \instruct{%
        sed immediate subiungitur antiphona finalis %
        beat\ae\ Mari\ae\ Virginis%
      } %
      100.%
    }%
    \colchunk[2]{%
      \instruct{%
        And the verse %
        \textcolor{black}{May the souls of the faithful} %
        is not said, but the final antiphon of the %
        Blessed Virgin Mary, %
        \textcolor{black}{100,} %
        is immediately joined.%
      }%
    }%
  \end{parcolumns}%
\end{document}
%</test1>
%    \end{macrocode}
%
% \section{Installation}
%
% \subsection{Download}
%
% \paragraph{Package.} This package is available on
% CTAN\footnote{\url{http://ctan.org/pkg/pdfcolparcolumns}}:
% \begin{description}
% \item[\CTAN{macros/latex/contrib/oberdiek/pdfcolparcolumns.dtx}] The source file.
% \item[\CTAN{macros/latex/contrib/oberdiek/pdfcolparcolumns.pdf}] Documentation.
% \end{description}
%
%
% \paragraph{Bundle.} All the packages of the bundle `oberdiek'
% are also available in a TDS compliant ZIP archive. There
% the packages are already unpacked and the documentation files
% are generated. The files and directories obey the TDS standard.
% \begin{description}
% \item[\CTAN{install/macros/latex/contrib/oberdiek.tds.zip}]
% \end{description}
% \emph{TDS} refers to the standard ``A Directory Structure
% for \TeX\ Files'' (\CTAN{tds/tds.pdf}). Directories
% with \xfile{texmf} in their name are usually organized this way.
%
% \subsection{Bundle installation}
%
% \paragraph{Unpacking.} Unpack the \xfile{oberdiek.tds.zip} in the
% TDS tree (also known as \xfile{texmf} tree) of your choice.
% Example (linux):
% \begin{quote}
%   |unzip oberdiek.tds.zip -d ~/texmf|
% \end{quote}
%
% \paragraph{Script installation.}
% Check the directory \xfile{TDS:scripts/oberdiek/} for
% scripts that need further installation steps.
% Package \xpackage{attachfile2} comes with the Perl script
% \xfile{pdfatfi.pl} that should be installed in such a way
% that it can be called as \texttt{pdfatfi}.
% Example (linux):
% \begin{quote}
%   |chmod +x scripts/oberdiek/pdfatfi.pl|\\
%   |cp scripts/oberdiek/pdfatfi.pl /usr/local/bin/|
% \end{quote}
%
% \subsection{Package installation}
%
% \paragraph{Unpacking.} The \xfile{.dtx} file is a self-extracting
% \docstrip\ archive. The files are extracted by running the
% \xfile{.dtx} through \plainTeX:
% \begin{quote}
%   \verb|tex pdfcolparcolumns.dtx|
% \end{quote}
%
% \paragraph{TDS.} Now the different files must be moved into
% the different directories in your installation TDS tree
% (also known as \xfile{texmf} tree):
% \begin{quote}
% \def\t{^^A
% \begin{tabular}{@{}>{\ttfamily}l@{ $\rightarrow$ }>{\ttfamily}l@{}}
%   pdfcolparcolumns.sty & tex/latex/oberdiek/pdfcolparcolumns.sty\\
%   pdfcolparcolumns.pdf & doc/latex/oberdiek/pdfcolparcolumns.pdf\\
%   test/pdfcolparcolumns-test1.tex & doc/latex/oberdiek/test/pdfcolparcolumns-test1.tex\\
%   pdfcolparcolumns.dtx & source/latex/oberdiek/pdfcolparcolumns.dtx\\
% \end{tabular}^^A
% }^^A
% \sbox0{\t}^^A
% \ifdim\wd0>\linewidth
%   \begingroup
%     \advance\linewidth by\leftmargin
%     \advance\linewidth by\rightmargin
%   \edef\x{\endgroup
%     \def\noexpand\lw{\the\linewidth}^^A
%   }\x
%   \def\lwbox{^^A
%     \leavevmode
%     \hbox to \linewidth{^^A
%       \kern-\leftmargin\relax
%       \hss
%       \usebox0
%       \hss
%       \kern-\rightmargin\relax
%     }^^A
%   }^^A
%   \ifdim\wd0>\lw
%     \sbox0{\small\t}^^A
%     \ifdim\wd0>\linewidth
%       \ifdim\wd0>\lw
%         \sbox0{\footnotesize\t}^^A
%         \ifdim\wd0>\linewidth
%           \ifdim\wd0>\lw
%             \sbox0{\scriptsize\t}^^A
%             \ifdim\wd0>\linewidth
%               \ifdim\wd0>\lw
%                 \sbox0{\tiny\t}^^A
%                 \ifdim\wd0>\linewidth
%                   \lwbox
%                 \else
%                   \usebox0
%                 \fi
%               \else
%                 \lwbox
%               \fi
%             \else
%               \usebox0
%             \fi
%           \else
%             \lwbox
%           \fi
%         \else
%           \usebox0
%         \fi
%       \else
%         \lwbox
%       \fi
%     \else
%       \usebox0
%     \fi
%   \else
%     \lwbox
%   \fi
% \else
%   \usebox0
% \fi
% \end{quote}
% If you have a \xfile{docstrip.cfg} that configures and enables \docstrip's
% TDS installing feature, then some files can already be in the right
% place, see the documentation of \docstrip.
%
% \subsection{Refresh file name databases}
%
% If your \TeX~distribution
% (\teTeX, \mikTeX, \dots) relies on file name databases, you must refresh
% these. For example, \teTeX\ users run \verb|texhash| or
% \verb|mktexlsr|.
%
% \subsection{Some details for the interested}
%
% \paragraph{Attached source.}
%
% The PDF documentation on CTAN also includes the
% \xfile{.dtx} source file. It can be extracted by
% AcrobatReader 6 or higher. Another option is \textsf{pdftk},
% e.g. unpack the file into the current directory:
% \begin{quote}
%   \verb|pdftk pdfcolparcolumns.pdf unpack_files output .|
% \end{quote}
%
% \paragraph{Unpacking with \LaTeX.}
% The \xfile{.dtx} chooses its action depending on the format:
% \begin{description}
% \item[\plainTeX:] Run \docstrip\ and extract the files.
% \item[\LaTeX:] Generate the documentation.
% \end{description}
% If you insist on using \LaTeX\ for \docstrip\ (really,
% \docstrip\ does not need \LaTeX), then inform the autodetect routine
% about your intention:
% \begin{quote}
%   \verb|latex \let\install=y% \iffalse meta-comment
%
% File: pdfcolparcolumns.dtx
% Version: 2016/05/16 v1.4
% Info: Color stacks for parcolumns
%
% Copyright (C) 2007, 2008, 2010 by
%    Heiko Oberdiek <heiko.oberdiek at googlemail.com>
%    2016
%    https://github.com/ho-tex/oberdiek/issues
%
% This work may be distributed and/or modified under the
% conditions of the LaTeX Project Public License, either
% version 1.3c of this license or (at your option) any later
% version. This version of this license is in
%    http://www.latex-project.org/lppl/lppl-1-3c.txt
% and the latest version of this license is in
%    http://www.latex-project.org/lppl.txt
% and version 1.3 or later is part of all distributions of
% LaTeX version 2005/12/01 or later.
%
% This work has the LPPL maintenance status "maintained".
%
% This Current Maintainer of this work is Heiko Oberdiek.
%
% This work consists of the main source file pdfcolparcolumns.dtx
% and the derived files
%    pdfcolparcolumns.sty, pdfcolparcolumns.pdf, pdfcolparcolumns.ins,
%    pdfcolparcolumns.drv, pdfcolparcolumns-test1.tex.
%
% Distribution:
%    CTAN:macros/latex/contrib/oberdiek/pdfcolparcolumns.dtx
%    CTAN:macros/latex/contrib/oberdiek/pdfcolparcolumns.pdf
%
% Unpacking:
%    (a) If pdfcolparcolumns.ins is present:
%           tex pdfcolparcolumns.ins
%    (b) Without pdfcolparcolumns.ins:
%           tex pdfcolparcolumns.dtx
%    (c) If you insist on using LaTeX
%           latex \let\install=y\input{pdfcolparcolumns.dtx}
%        (quote the arguments according to the demands of your shell)
%
% Documentation:
%    (a) If pdfcolparcolumns.drv is present:
%           latex pdfcolparcolumns.drv
%    (b) Without pdfcolparcolumns.drv:
%           latex pdfcolparcolumns.dtx; ...
%    The class ltxdoc loads the configuration file ltxdoc.cfg
%    if available. Here you can specify further options, e.g.
%    use A4 as paper format:
%       \PassOptionsToClass{a4paper}{article}
%
%    Programm calls to get the documentation (example):
%       pdflatex pdfcolparcolumns.dtx
%       makeindex -s gind.ist pdfcolparcolumns.idx
%       pdflatex pdfcolparcolumns.dtx
%       makeindex -s gind.ist pdfcolparcolumns.idx
%       pdflatex pdfcolparcolumns.dtx
%
% Installation:
%    TDS:tex/latex/oberdiek/pdfcolparcolumns.sty
%    TDS:doc/latex/oberdiek/pdfcolparcolumns.pdf
%    TDS:doc/latex/oberdiek/test/pdfcolparcolumns-test1.tex
%    TDS:source/latex/oberdiek/pdfcolparcolumns.dtx
%
%<*ignore>
\begingroup
  \catcode123=1 %
  \catcode125=2 %
  \def\x{LaTeX2e}%
\expandafter\endgroup
\ifcase 0\ifx\install y1\fi\expandafter
         \ifx\csname processbatchFile\endcsname\relax\else1\fi
         \ifx\fmtname\x\else 1\fi\relax
\else\csname fi\endcsname
%</ignore>
%<*install>
\input docstrip.tex
\Msg{************************************************************************}
\Msg{* Installation}
\Msg{* Package: pdfcolparcolumns 2016/05/16 v1.4 Color stacks for parcolumns (HO)}
\Msg{************************************************************************}

\keepsilent
\askforoverwritefalse

\let\MetaPrefix\relax
\preamble

This is a generated file.

Project: pdfcolparcolumns
Version: 2016/05/16 v1.4

Copyright (C) 2007, 2008, 2010 by
   Heiko Oberdiek <heiko.oberdiek at googlemail.com>

This work may be distributed and/or modified under the
conditions of the LaTeX Project Public License, either
version 1.3c of this license or (at your option) any later
version. This version of this license is in
   http://www.latex-project.org/lppl/lppl-1-3c.txt
and the latest version of this license is in
   http://www.latex-project.org/lppl.txt
and version 1.3 or later is part of all distributions of
LaTeX version 2005/12/01 or later.

This work has the LPPL maintenance status "maintained".

This Current Maintainer of this work is Heiko Oberdiek.

This work consists of the main source file pdfcolparcolumns.dtx
and the derived files
   pdfcolparcolumns.sty, pdfcolparcolumns.pdf, pdfcolparcolumns.ins,
   pdfcolparcolumns.drv, pdfcolparcolumns-test1.tex.

\endpreamble
\let\MetaPrefix\DoubleperCent

\generate{%
  \file{pdfcolparcolumns.ins}{\from{pdfcolparcolumns.dtx}{install}}%
  \file{pdfcolparcolumns.drv}{\from{pdfcolparcolumns.dtx}{driver}}%
  \usedir{tex/latex/oberdiek}%
  \file{pdfcolparcolumns.sty}{\from{pdfcolparcolumns.dtx}{package}}%
  \usedir{doc/latex/oberdiek/test}%
  \file{pdfcolparcolumns-test1.tex}{\from{pdfcolparcolumns.dtx}{test1}}%
  \nopreamble
  \nopostamble
  \usedir{source/latex/oberdiek/catalogue}%
  \file{pdfcolparcolumns.xml}{\from{pdfcolparcolumns.dtx}{catalogue}}%
}

\catcode32=13\relax% active space
\let =\space%
\Msg{************************************************************************}
\Msg{*}
\Msg{* To finish the installation you have to move the following}
\Msg{* file into a directory searched by TeX:}
\Msg{*}
\Msg{*     pdfcolparcolumns.sty}
\Msg{*}
\Msg{* To produce the documentation run the file `pdfcolparcolumns.drv'}
\Msg{* through LaTeX.}
\Msg{*}
\Msg{* Happy TeXing!}
\Msg{*}
\Msg{************************************************************************}

\endbatchfile
%</install>
%<*ignore>
\fi
%</ignore>
%<*driver>
\NeedsTeXFormat{LaTeX2e}
\ProvidesFile{pdfcolparcolumns.drv}%
  [2016/05/16 v1.4 Color stacks for parcolumns (HO)]%
\documentclass{ltxdoc}
\usepackage{holtxdoc}[2011/11/22]
\begin{document}
  \DocInput{pdfcolparcolumns.dtx}%
\end{document}
%</driver>
% \fi
%
%
% \CharacterTable
%  {Upper-case    \A\B\C\D\E\F\G\H\I\J\K\L\M\N\O\P\Q\R\S\T\U\V\W\X\Y\Z
%   Lower-case    \a\b\c\d\e\f\g\h\i\j\k\l\m\n\o\p\q\r\s\t\u\v\w\x\y\z
%   Digits        \0\1\2\3\4\5\6\7\8\9
%   Exclamation   \!     Double quote  \"     Hash (number) \#
%   Dollar        \$     Percent       \%     Ampersand     \&
%   Acute accent  \'     Left paren    \(     Right paren   \)
%   Asterisk      \*     Plus          \+     Comma         \,
%   Minus         \-     Point         \.     Solidus       \/
%   Colon         \:     Semicolon     \;     Less than     \<
%   Equals        \=     Greater than  \>     Question mark \?
%   Commercial at \@     Left bracket  \[     Backslash     \\
%   Right bracket \]     Circumflex    \^     Underscore    \_
%   Grave accent  \`     Left brace    \{     Vertical bar  \|
%   Right brace   \}     Tilde         \~}
%
% \GetFileInfo{pdfcolparcolumns.drv}
%
% \title{The \xpackage{pdfcolparcolumns} package}
% \date{2016/05/16 v1.4}
% \author{Heiko Oberdiek\thanks
% {Please report any issues at https://github.com/ho-tex/oberdiek/issues}\\
% \xemail{heiko.oberdiek at googlemail.com}}
%
% \maketitle
%
% \begin{abstract}
% Since version 1.40 \pdfTeX\ supports several color stacks.
% This package uses them to fix color problems in
% package \xpackage{parcolumns}.
% \end{abstract}
%
% \tableofcontents
%
% \section{Usage}
%
% \begin{quote}
% |\usepackage{pdfcolparcolumns}|
% \end{quote}
% The package \xpackage{pdfcolparcolumns} loads package \xpackage{parcolums}
% \cite{parcolumns}. If color stacks are available then the
% macros of \xpackage{parcolumns} are patched to add support
% for color stacks.
%
% \subsection{Option \xoption{rulebetweencolor}}
%
% Package \xpackage{pdfcolparcolumns} also fixes the color for the
% rule between columns (if \xoption{rulebetween} is set).
% Default color is \cs{normalcolor}. But this can be changed by using
% option \xoption{rulebetweencolor}. It takes a color specification
% as value. If the value is empty, then the default (\cs{normalcolor})
% is used.
% Examples:
% \begin{quote}
%   |rulebetweencolor=blue|,\\
%   |rulebetweencolor={red}|,\\
%   |rulebetweencolor={}|, \textit{\% \cs{normalcolor} is used}\\
%   |rulebetweencolor=[rgb]{1,0,.5}| \textit{\% see below}
% \end{quote}
% If used inside the optional argument of environment \textsf{parcolumns}
% and the value contains an optional argument, then whole value
% must be put in curly braces:
%\begin{quote}
%\begin{verbatim}
%\begin{parcolumns}[
%  rulebetween,
%  rulebetweencolor={[rgb]{1,0,.5}},
%]{2}
%  ...
%\end{parcolumns}
%\end{verbatim}
%\end{quote}
% This option \xoption{rulebetweencolor} can also be set using
% \cs{setkeys}:
%\begin{quote}
%\begin{verbatim}
%\setkeys{parcolumns}{rulebetweencolor=green}
%\end{verbatim}
%\end{quote}
%
% \subsection{Future}
%
% Currently package \xpackage{parcolumns} does not seem to be
% maintained. Nevertheless if there will be a new version that
% adds support for color stacks, then this package may become
% obsolete.
%
% \StopEventually{
% }
%
% \section{Implementation}
%
% \subsection{Identification}
%
%    \begin{macrocode}
%<*package>
\NeedsTeXFormat{LaTeX2e}
\ProvidesPackage{pdfcolparcolumns}%
  [2016/05/16 v1.4 Color stacks for parcolumns (HO)]%
%    \end{macrocode}
%
% \subsection{Load packages}
%
% \subsubsection{Package \xpackage{parcolumns}}
%
%    Currently package \xpackage{parcolumns} does not define options.
%    Thus it is just a precaution that the options of
%    package \xpackage{pdfcolparcolumns} are passed to
%    package \xpackage{parcolumns}.
%    \begin{macrocode}
\DeclareOption*{%
  \PassoptionsToPackage{\CurrentOption}{parcolumns}%
}
\ProcessOptions\relax
\RequirePackage{parcolumns}[2004/11/25]
%    \end{macrocode}
%
% \subsubsection{Package \xpackage{pdfcol}}
%
%    \begin{macrocode}
\RequirePackage{pdfcol}[2007/09/09]
\ifpdfcolAvailable
\else
  \PackageInfo{pdfcolparcolumns}{%
    Loading aborted, because color stacks are not available%
  }%
  \expandafter\endinput
\fi
%    \end{macrocode}
%
% \subsubsection{Package \xpackage{infwarerr}}
%
%    \begin{macrocode}
\RequirePackage{infwarerr}[2007/09/09]
%    \end{macrocode}
%
% \subsection{Color stack macros}
%
%    \begin{macro}{\pcpc@MaxStack}
%    Macro \cs{pcpc@MaxStack} holds the highest number of
%    allocated stacks.
%    \begin{macrocode}
\global\chardef\pcpc@MaxStack=\z@
%    \end{macrocode}
%    \end{macro}
%    \begin{macro}{\pcpc@InitStacks}
%    Macro \cs{pcpc@InitStacks} takes the number of columns
%    as argument and ensures that there are enough color
%    stacks for all columns.
%    \begin{macrocode}
\def\pcpc@InitStacks#1{%
  \ifnum#1>\pcpc@MaxStack
    \begingroup
      \count@\pcpc@MaxStack
      \loop
        \advance\count@\@ne
        \pdfcolInitStack{pcpc@\the\count@}%
      \ifnum#1>\count@
      \repeat
      \global\chardef\pcpc@MaxStack=\count@
    \endgroup
  \fi
}
%    \end{macrocode}
%    \end{macro}
%
%    \begin{macro}{\pcpc@SwitchStack}
%    \begin{macrocode}
\def\pcpc@SwitchStack#1{%
  \pdfcolSwitchStack{pcpc@\number#1}%
}
%    \end{macrocode}
%    \end{macro}
%
%    \begin{macro}{\pcpc@SetCurrent}
%    \begin{macrocode}
\def\pcpc@SetCurrent#1{%
  \pdfcolSetCurrent{pcpc@\number#1}%
}
%    \end{macrocode}
%    \end{macro}
%
% \subsection{Patches}
%
%     Now the color stack macros are patched into the macros
%     of package \xpackage{parcolumns}.
%
% \subsubsection{Init stacks}
%
%    \cs{pcpc@InitStacks} should go into the definition of
%    environment |parcolumns|. \cs{pc@alloccolumns} is executed
%    there and nowhere else, thus we hook into it.
%    \begin{macrocode}
\g@addto@macro\pc@alloccolumns{%
  \pcpc@InitStacks\pc@columncount
}
%    \end{macrocode}
%
% \subsubsection{Switch stack}
%
%    \cs{pcpc@SwitchStack} should be called by marco \cs{colchunk@}.
%    However it is easier to patch \cs{pc@setcolumnwidth} that
%    is executed in \cs{colchunk@} only.
%    \begin{macrocode}
\g@addto@macro\pc@setcolumnwidth{%
  \pcpc@SwitchStack\pc@columnctr
}
%    \end{macrocode}
%
% \subsubsection{Set current stack color}
%
%    \cs{pcpc@SetCurrent} is set at the begin of each line.
%    It must be inserted into \cs{pc@placeboxes}. Unhappily
%    there is no easy way. Therefore we check and
%    redefine \cs{pc@placeboxes}.
%    \begin{macrocode}
\begingroup
  \def\x{%
    \global\let\@tempa\relax
    \count@\z@
    \hb@xt@\linewidth{%
      \vfuzz30ex %
      \vbadness\@M
      \splittopskip\z@skip
      \loop
      \ifnum\count@<\pc@columncount
        \advance\count@\@ne
        \expandafter\ifvoid\csname pc@column@\number\count@\endcsname
          \hskip\csname pc@column@width@\number\count@\endcsname
        \else
          \expandafter\setbox\expandafter\@tempboxa\expandafter
          \vsplit\csname pc@column@\number\count@\endcsname
              to \dp\strutbox
          \vbox{%
            \unvbox\@tempboxa
          }%
        \fi
        \expandafter\ifvoid\csname pc@column@\number\count@\endcsname
        \else
          \global\let\@tempa\pc@placeboxes
        \fi
        \ifnum\count@<\pc@columncount
          \strut
          \hfill
          \ifpc@rulebetween
            \vrule
            \hfill
          \fi
        \fi
      \repeat
    }%
    \@tempa
  }%
  \ifx\x\pc@placeboxes
  \else
    \@PackageWarningNoLine{pdfcolparcolumns}{%
      Command \string\pc@placeboxes\space has changed.\MessageBreak
      Supported versions of package `parcolumns':\MessageBreak
      \space\space 2004/08/05.\MessageBreak
      The redefinition of \string\pc@placeboxes\space may not%
      \MessageBreak
      behave correctly depending on the changes%
    }%
  \fi
\endgroup
%    \end{macrocode}
%    \begin{macro}{\pc@placeboxes}
%    \begin{macrocode}
\renewcommand*{\pc@placeboxes}{%
  \global\let\@tempa\relax
  \count@\z@
  \hb@xt@\linewidth{%
    \vfuzz30ex %
    \vbadness\@M
    \splittopskip\z@skip
    \loop
    \ifnum\count@<\pc@columncount
      \advance\count@\@ne
      \expandafter\ifvoid\csname pc@column@\number\count@\endcsname
        \hskip\csname pc@column@width@\number\count@\endcsname
      \else
        \expandafter\setbox\expandafter\@tempboxa\expandafter
        \vsplit\csname pc@column@\number\count@\endcsname
            to \dp\strutbox
        \vbox{%
          \pcpc@SetCurrent\count@
          \unvbox\@tempboxa
        }%
      \fi
      \expandafter\ifvoid\csname pc@column@\number\count@\endcsname
      \else
        \global\let\@tempa\pc@placeboxes
      \fi
      \ifnum\count@<\pc@columncount
        \strut
        \hfill
        \ifpc@rulebetween
          \begingroup
            \pcpc@RuleBetweenColor
            \vrule
          \endgroup
          \hfill
        \fi
      \fi
    \repeat
  }%
  \@tempa
}
%    \end{macrocode}
%    \end{macro}
%    \begin{macro}{\pcpc@RuleBetweenColorDefault}
%    \begin{macrocode}
\def\pcpc@RuleBetweenColorDefault{%
  \normalcolor
}
%    \end{macrocode}
%    \end{macro}
%    \begin{macro}{\pcpc@RuleBetweenColor}
%    \begin{macrocode}
\let\pcpc@RuleBetweenColor\pcpc@RuleBetweenColorDefault
%    \end{macrocode}
%    \end{macro}
%    \begin{macrocode}
\define@key{parcolumns}{rulebetweencolor}{%
  \edef\pcpc@temp{#1}%
  \ifx\pcpc@temp\@empty
    \let\pcpc@RuleBetweenColor\pcpc@RuleBetweenColorDefault
  \else
    \edef\pcpc@temp{%
      \noexpand\@ifnextchar[{%
        \def\noexpand\pcpc@RuleBetweenColor{%
          \noexpand\color\pcpc@temp
        }%
        \noexpand\pcpc@GobbleNil
      }{%
        \def\noexpand\pcpc@RuleBetweenColor{%
          \noexpand\color{\pcpc@temp}%
        }%
        \noexpand\pcpc@GobbleNil
      }%
      \pcpc@temp\noexpand\@nil
    }%
    \pcpc@temp
  \fi
}
%    \end{macrocode}
%    \begin{macro}{\pcpc@GobbleNil}
%    \begin{macrocode}
\long\def\pcpc@GobbleNil#1\@nil{}
%    \end{macrocode}
%    \end{macro}
%
%    \begin{macrocode}
%</package>
%    \end{macrocode}
%
% \section{Test}
%
%    The test file is a modified version of the file that
%    Donald Goodman has posted in \xnewsgroup{comp.text.tex}: ^^A
%    \URL{``\link{Re: \xpackage{xcolor} glitches}''}^^A
%    {http://groups.google.com/group/comp.text.tex/msg/8eda74ed292012bb}
%    \begin{macrocode}
%<*test1>
\NeedsTeXFormat{LaTeX2e}
\AtEndDocument{%
  \typeout{}%
  \typeout{**************************************}%
  \typeout{*** \space Check the PDF file manually! \space ***}%
  \typeout{**************************************}%
  \typeout{}%
}
\documentclass{article}
\usepackage{xcolor}
\usepackage{pdfcolparcolumns}

\newcommand{\instruct}[1]{%
  \noindent
  \footnotesize
  \textcolor{red}{#1}%
}

\begin{document}
  \begin{parcolumns}[colwidths={1=2.3in,2=2.3in},sloppy]{2}%
    \colchunk[1]{%
      \instruct{Et non dicitur versus} %
      Fidelium anim\ae\ %
      \instruct{%
        sed immediate subiungitur antiphona finalis %
        beat\ae\ Mari\ae\ Virginis%
      } %
      100.%
    }%
    \colchunk[2]{%
      \instruct{%
        And the verse %
        \textcolor{black}{May the souls of the faithful} %
        is not said, but the final antiphon of the %
        Blessed Virgin Mary, %
        \textcolor{black}{100,} %
        is immediately joined.%
      }%
    }%
  \end{parcolumns}%
\end{document}
%</test1>
%    \end{macrocode}
%
% \section{Installation}
%
% \subsection{Download}
%
% \paragraph{Package.} This package is available on
% CTAN\footnote{\url{http://ctan.org/pkg/pdfcolparcolumns}}:
% \begin{description}
% \item[\CTAN{macros/latex/contrib/oberdiek/pdfcolparcolumns.dtx}] The source file.
% \item[\CTAN{macros/latex/contrib/oberdiek/pdfcolparcolumns.pdf}] Documentation.
% \end{description}
%
%
% \paragraph{Bundle.} All the packages of the bundle `oberdiek'
% are also available in a TDS compliant ZIP archive. There
% the packages are already unpacked and the documentation files
% are generated. The files and directories obey the TDS standard.
% \begin{description}
% \item[\CTAN{install/macros/latex/contrib/oberdiek.tds.zip}]
% \end{description}
% \emph{TDS} refers to the standard ``A Directory Structure
% for \TeX\ Files'' (\CTAN{tds/tds.pdf}). Directories
% with \xfile{texmf} in their name are usually organized this way.
%
% \subsection{Bundle installation}
%
% \paragraph{Unpacking.} Unpack the \xfile{oberdiek.tds.zip} in the
% TDS tree (also known as \xfile{texmf} tree) of your choice.
% Example (linux):
% \begin{quote}
%   |unzip oberdiek.tds.zip -d ~/texmf|
% \end{quote}
%
% \paragraph{Script installation.}
% Check the directory \xfile{TDS:scripts/oberdiek/} for
% scripts that need further installation steps.
% Package \xpackage{attachfile2} comes with the Perl script
% \xfile{pdfatfi.pl} that should be installed in such a way
% that it can be called as \texttt{pdfatfi}.
% Example (linux):
% \begin{quote}
%   |chmod +x scripts/oberdiek/pdfatfi.pl|\\
%   |cp scripts/oberdiek/pdfatfi.pl /usr/local/bin/|
% \end{quote}
%
% \subsection{Package installation}
%
% \paragraph{Unpacking.} The \xfile{.dtx} file is a self-extracting
% \docstrip\ archive. The files are extracted by running the
% \xfile{.dtx} through \plainTeX:
% \begin{quote}
%   \verb|tex pdfcolparcolumns.dtx|
% \end{quote}
%
% \paragraph{TDS.} Now the different files must be moved into
% the different directories in your installation TDS tree
% (also known as \xfile{texmf} tree):
% \begin{quote}
% \def\t{^^A
% \begin{tabular}{@{}>{\ttfamily}l@{ $\rightarrow$ }>{\ttfamily}l@{}}
%   pdfcolparcolumns.sty & tex/latex/oberdiek/pdfcolparcolumns.sty\\
%   pdfcolparcolumns.pdf & doc/latex/oberdiek/pdfcolparcolumns.pdf\\
%   test/pdfcolparcolumns-test1.tex & doc/latex/oberdiek/test/pdfcolparcolumns-test1.tex\\
%   pdfcolparcolumns.dtx & source/latex/oberdiek/pdfcolparcolumns.dtx\\
% \end{tabular}^^A
% }^^A
% \sbox0{\t}^^A
% \ifdim\wd0>\linewidth
%   \begingroup
%     \advance\linewidth by\leftmargin
%     \advance\linewidth by\rightmargin
%   \edef\x{\endgroup
%     \def\noexpand\lw{\the\linewidth}^^A
%   }\x
%   \def\lwbox{^^A
%     \leavevmode
%     \hbox to \linewidth{^^A
%       \kern-\leftmargin\relax
%       \hss
%       \usebox0
%       \hss
%       \kern-\rightmargin\relax
%     }^^A
%   }^^A
%   \ifdim\wd0>\lw
%     \sbox0{\small\t}^^A
%     \ifdim\wd0>\linewidth
%       \ifdim\wd0>\lw
%         \sbox0{\footnotesize\t}^^A
%         \ifdim\wd0>\linewidth
%           \ifdim\wd0>\lw
%             \sbox0{\scriptsize\t}^^A
%             \ifdim\wd0>\linewidth
%               \ifdim\wd0>\lw
%                 \sbox0{\tiny\t}^^A
%                 \ifdim\wd0>\linewidth
%                   \lwbox
%                 \else
%                   \usebox0
%                 \fi
%               \else
%                 \lwbox
%               \fi
%             \else
%               \usebox0
%             \fi
%           \else
%             \lwbox
%           \fi
%         \else
%           \usebox0
%         \fi
%       \else
%         \lwbox
%       \fi
%     \else
%       \usebox0
%     \fi
%   \else
%     \lwbox
%   \fi
% \else
%   \usebox0
% \fi
% \end{quote}
% If you have a \xfile{docstrip.cfg} that configures and enables \docstrip's
% TDS installing feature, then some files can already be in the right
% place, see the documentation of \docstrip.
%
% \subsection{Refresh file name databases}
%
% If your \TeX~distribution
% (\teTeX, \mikTeX, \dots) relies on file name databases, you must refresh
% these. For example, \teTeX\ users run \verb|texhash| or
% \verb|mktexlsr|.
%
% \subsection{Some details for the interested}
%
% \paragraph{Attached source.}
%
% The PDF documentation on CTAN also includes the
% \xfile{.dtx} source file. It can be extracted by
% AcrobatReader 6 or higher. Another option is \textsf{pdftk},
% e.g. unpack the file into the current directory:
% \begin{quote}
%   \verb|pdftk pdfcolparcolumns.pdf unpack_files output .|
% \end{quote}
%
% \paragraph{Unpacking with \LaTeX.}
% The \xfile{.dtx} chooses its action depending on the format:
% \begin{description}
% \item[\plainTeX:] Run \docstrip\ and extract the files.
% \item[\LaTeX:] Generate the documentation.
% \end{description}
% If you insist on using \LaTeX\ for \docstrip\ (really,
% \docstrip\ does not need \LaTeX), then inform the autodetect routine
% about your intention:
% \begin{quote}
%   \verb|latex \let\install=y\input{pdfcolparcolumns.dtx}|
% \end{quote}
% Do not forget to quote the argument according to the demands
% of your shell.
%
% \paragraph{Generating the documentation.}
% You can use both the \xfile{.dtx} or the \xfile{.drv} to generate
% the documentation. The process can be configured by the
% configuration file \xfile{ltxdoc.cfg}. For instance, put this
% line into this file, if you want to have A4 as paper format:
% \begin{quote}
%   \verb|\PassOptionsToClass{a4paper}{article}|
% \end{quote}
% An example follows how to generate the
% documentation with pdf\LaTeX:
% \begin{quote}
%\begin{verbatim}
%pdflatex pdfcolparcolumns.dtx
%makeindex -s gind.ist pdfcolparcolumns.idx
%pdflatex pdfcolparcolumns.dtx
%makeindex -s gind.ist pdfcolparcolumns.idx
%pdflatex pdfcolparcolumns.dtx
%\end{verbatim}
% \end{quote}
%
% \section{Catalogue}
%
% The following XML file can be used as source for the
% \href{http://mirror.ctan.org/help/Catalogue/catalogue.html}{\TeX\ Catalogue}.
% The elements \texttt{caption} and \texttt{description} are imported
% from the original XML file from the Catalogue.
% The name of the XML file in the Catalogue is \xfile{pdfcolparcolumns.xml}.
%    \begin{macrocode}
%<*catalogue>
<?xml version='1.0' encoding='us-ascii'?>
<!DOCTYPE entry SYSTEM 'catalogue.dtd'>
<entry datestamp='$Date$' modifier='$Author$' id='pdfcolparcolumns'>
  <name>pdfcolparcolumns</name>
  <caption>Fix colour problems in package 'parcolumns'.</caption>
  <authorref id='auth:oberdiek'/>
  <copyright owner='Heiko Oberdiek' year='2007,2008,2010'/>
  <license type='lppl1.3'/>
  <version number='1.4'/>
  <description>
    Since version 1.40 pdfTeX supports colour stacks.
    This package uses them to fix colour problems in
    package <xref refid='parcolumns'>parcolumns</xref>.
    <p/>
    The package is part of the <xref refid='oberdiek'>oberdiek</xref>
    bundle.
  </description>
  <documentation details='Package documentation'
      href='ctan:/macros/latex/contrib/oberdiek/pdfcolparcolumns.pdf'/>
  <ctan file='true' path='/macros/latex/contrib/oberdiek/pdfcolparcolumns.dtx'/>
  <miktex location='oberdiek'/>
  <texlive location='oberdiek'/>
  <install path='/macros/latex/contrib/oberdiek/oberdiek.tds.zip'/>
</entry>
%</catalogue>
%    \end{macrocode}
%
% \begin{thebibliography}{9}
%
% \bibitem{parcolumns}
%   Jonathan Sauer: \textit{The \xpackage{parcolumns} package};
%   2004/11/25;\\
%   \CTAN{macros/latex/contrib/sauerj/parcolumns.pdf}.
%
% \bibitem{pdfcol}
%   Heiko Oberdiek: \textit{The \xpackage{pdfcol} package};
%   2007/09/09;\\
%   \CTAN{macros/latex/contrib/oberdiek/pdfcol.pdf}.
%
% \end{thebibliography}
%
% \begin{History}
%   \begin{Version}{2007/07/26 v1.0}
%   \item
%     First version, published in the newsgroup \xnewsgroup{comp.text.tex}
%     with the name \xpackage{parcolumns-colorstacks}: ^^A no line break
%     \URL{``\link{Re: \xpackage{xcolor} glitches}''}^^A
%     {http://groups.google.com/group/comp.text.tex/msg/56bd897b11bca414}
%   \end{Version}
%   \begin{Version}{2007/09/09 v1.1}
%   \item
%     CTAN version, package name renamed to \xpackage{pdfcolparcolumns}.
%   \item
%     Uses package \xpackage{pdfcol}.
%   \item
%     Documentation added.
%   \item
%     Test file added.
%   \end{Version}
%   \begin{Version}{2008/08/11 v1.2}
%   \item
%     Code is not changed.
%   \item
%     URLs updated.
%   \end{Version}
%   \begin{Version}{2010/01/11 v1.3}
%   \item
%     Fix for rule color.
%   \item
%     New option \xoption{rulebetweencolor} for environment |parcolumns|.
%   \end{Version}
%   \begin{Version}{2016/05/16 v1.4}
%   \item
%     Documentation updates.
%   \end{Version}
% \end{History}
%
% \PrintIndex
%
% \Finale
\endinput
|
% \end{quote}
% Do not forget to quote the argument according to the demands
% of your shell.
%
% \paragraph{Generating the documentation.}
% You can use both the \xfile{.dtx} or the \xfile{.drv} to generate
% the documentation. The process can be configured by the
% configuration file \xfile{ltxdoc.cfg}. For instance, put this
% line into this file, if you want to have A4 as paper format:
% \begin{quote}
%   \verb|\PassOptionsToClass{a4paper}{article}|
% \end{quote}
% An example follows how to generate the
% documentation with pdf\LaTeX:
% \begin{quote}
%\begin{verbatim}
%pdflatex pdfcolparcolumns.dtx
%makeindex -s gind.ist pdfcolparcolumns.idx
%pdflatex pdfcolparcolumns.dtx
%makeindex -s gind.ist pdfcolparcolumns.idx
%pdflatex pdfcolparcolumns.dtx
%\end{verbatim}
% \end{quote}
%
% \section{Catalogue}
%
% The following XML file can be used as source for the
% \href{http://mirror.ctan.org/help/Catalogue/catalogue.html}{\TeX\ Catalogue}.
% The elements \texttt{caption} and \texttt{description} are imported
% from the original XML file from the Catalogue.
% The name of the XML file in the Catalogue is \xfile{pdfcolparcolumns.xml}.
%    \begin{macrocode}
%<*catalogue>
<?xml version='1.0' encoding='us-ascii'?>
<!DOCTYPE entry SYSTEM 'catalogue.dtd'>
<entry datestamp='$Date$' modifier='$Author$' id='pdfcolparcolumns'>
  <name>pdfcolparcolumns</name>
  <caption>Fix colour problems in package 'parcolumns'.</caption>
  <authorref id='auth:oberdiek'/>
  <copyright owner='Heiko Oberdiek' year='2007,2008,2010'/>
  <license type='lppl1.3'/>
  <version number='1.4'/>
  <description>
    Since version 1.40 pdfTeX supports colour stacks.
    This package uses them to fix colour problems in
    package <xref refid='parcolumns'>parcolumns</xref>.
    <p/>
    The package is part of the <xref refid='oberdiek'>oberdiek</xref>
    bundle.
  </description>
  <documentation details='Package documentation'
      href='ctan:/macros/latex/contrib/oberdiek/pdfcolparcolumns.pdf'/>
  <ctan file='true' path='/macros/latex/contrib/oberdiek/pdfcolparcolumns.dtx'/>
  <miktex location='oberdiek'/>
  <texlive location='oberdiek'/>
  <install path='/macros/latex/contrib/oberdiek/oberdiek.tds.zip'/>
</entry>
%</catalogue>
%    \end{macrocode}
%
% \begin{thebibliography}{9}
%
% \bibitem{parcolumns}
%   Jonathan Sauer: \textit{The \xpackage{parcolumns} package};
%   2004/11/25;\\
%   \CTAN{macros/latex/contrib/sauerj/parcolumns.pdf}.
%
% \bibitem{pdfcol}
%   Heiko Oberdiek: \textit{The \xpackage{pdfcol} package};
%   2007/09/09;\\
%   \CTAN{macros/latex/contrib/oberdiek/pdfcol.pdf}.
%
% \end{thebibliography}
%
% \begin{History}
%   \begin{Version}{2007/07/26 v1.0}
%   \item
%     First version, published in the newsgroup \xnewsgroup{comp.text.tex}
%     with the name \xpackage{parcolumns-colorstacks}: ^^A no line break
%     \URL{``\link{Re: \xpackage{xcolor} glitches}''}^^A
%     {http://groups.google.com/group/comp.text.tex/msg/56bd897b11bca414}
%   \end{Version}
%   \begin{Version}{2007/09/09 v1.1}
%   \item
%     CTAN version, package name renamed to \xpackage{pdfcolparcolumns}.
%   \item
%     Uses package \xpackage{pdfcol}.
%   \item
%     Documentation added.
%   \item
%     Test file added.
%   \end{Version}
%   \begin{Version}{2008/08/11 v1.2}
%   \item
%     Code is not changed.
%   \item
%     URLs updated.
%   \end{Version}
%   \begin{Version}{2010/01/11 v1.3}
%   \item
%     Fix for rule color.
%   \item
%     New option \xoption{rulebetweencolor} for environment |parcolumns|.
%   \end{Version}
%   \begin{Version}{2016/05/16 v1.4}
%   \item
%     Documentation updates.
%   \end{Version}
% \end{History}
%
% \PrintIndex
%
% \Finale
\endinput

%        (quote the arguments according to the demands of your shell)
%
% Documentation:
%    (a) If pdfcolparcolumns.drv is present:
%           latex pdfcolparcolumns.drv
%    (b) Without pdfcolparcolumns.drv:
%           latex pdfcolparcolumns.dtx; ...
%    The class ltxdoc loads the configuration file ltxdoc.cfg
%    if available. Here you can specify further options, e.g.
%    use A4 as paper format:
%       \PassOptionsToClass{a4paper}{article}
%
%    Programm calls to get the documentation (example):
%       pdflatex pdfcolparcolumns.dtx
%       makeindex -s gind.ist pdfcolparcolumns.idx
%       pdflatex pdfcolparcolumns.dtx
%       makeindex -s gind.ist pdfcolparcolumns.idx
%       pdflatex pdfcolparcolumns.dtx
%
% Installation:
%    TDS:tex/latex/oberdiek/pdfcolparcolumns.sty
%    TDS:doc/latex/oberdiek/pdfcolparcolumns.pdf
%    TDS:doc/latex/oberdiek/test/pdfcolparcolumns-test1.tex
%    TDS:source/latex/oberdiek/pdfcolparcolumns.dtx
%
%<*ignore>
\begingroup
  \catcode123=1 %
  \catcode125=2 %
  \def\x{LaTeX2e}%
\expandafter\endgroup
\ifcase 0\ifx\install y1\fi\expandafter
         \ifx\csname processbatchFile\endcsname\relax\else1\fi
         \ifx\fmtname\x\else 1\fi\relax
\else\csname fi\endcsname
%</ignore>
%<*install>
\input docstrip.tex
\Msg{************************************************************************}
\Msg{* Installation}
\Msg{* Package: pdfcolparcolumns 2016/05/16 v1.4 Color stacks for parcolumns (HO)}
\Msg{************************************************************************}

\keepsilent
\askforoverwritefalse

\let\MetaPrefix\relax
\preamble

This is a generated file.

Project: pdfcolparcolumns
Version: 2016/05/16 v1.4

Copyright (C) 2007, 2008, 2010 by
   Heiko Oberdiek <heiko.oberdiek at googlemail.com>

This work may be distributed and/or modified under the
conditions of the LaTeX Project Public License, either
version 1.3c of this license or (at your option) any later
version. This version of this license is in
   http://www.latex-project.org/lppl/lppl-1-3c.txt
and the latest version of this license is in
   http://www.latex-project.org/lppl.txt
and version 1.3 or later is part of all distributions of
LaTeX version 2005/12/01 or later.

This work has the LPPL maintenance status "maintained".

This Current Maintainer of this work is Heiko Oberdiek.

This work consists of the main source file pdfcolparcolumns.dtx
and the derived files
   pdfcolparcolumns.sty, pdfcolparcolumns.pdf, pdfcolparcolumns.ins,
   pdfcolparcolumns.drv, pdfcolparcolumns-test1.tex.

\endpreamble
\let\MetaPrefix\DoubleperCent

\generate{%
  \file{pdfcolparcolumns.ins}{\from{pdfcolparcolumns.dtx}{install}}%
  \file{pdfcolparcolumns.drv}{\from{pdfcolparcolumns.dtx}{driver}}%
  \usedir{tex/latex/oberdiek}%
  \file{pdfcolparcolumns.sty}{\from{pdfcolparcolumns.dtx}{package}}%
  \usedir{doc/latex/oberdiek/test}%
  \file{pdfcolparcolumns-test1.tex}{\from{pdfcolparcolumns.dtx}{test1}}%
  \nopreamble
  \nopostamble
  \usedir{source/latex/oberdiek/catalogue}%
  \file{pdfcolparcolumns.xml}{\from{pdfcolparcolumns.dtx}{catalogue}}%
}

\catcode32=13\relax% active space
\let =\space%
\Msg{************************************************************************}
\Msg{*}
\Msg{* To finish the installation you have to move the following}
\Msg{* file into a directory searched by TeX:}
\Msg{*}
\Msg{*     pdfcolparcolumns.sty}
\Msg{*}
\Msg{* To produce the documentation run the file `pdfcolparcolumns.drv'}
\Msg{* through LaTeX.}
\Msg{*}
\Msg{* Happy TeXing!}
\Msg{*}
\Msg{************************************************************************}

\endbatchfile
%</install>
%<*ignore>
\fi
%</ignore>
%<*driver>
\NeedsTeXFormat{LaTeX2e}
\ProvidesFile{pdfcolparcolumns.drv}%
  [2016/05/16 v1.4 Color stacks for parcolumns (HO)]%
\documentclass{ltxdoc}
\usepackage{holtxdoc}[2011/11/22]
\begin{document}
  \DocInput{pdfcolparcolumns.dtx}%
\end{document}
%</driver>
% \fi
%
%
% \CharacterTable
%  {Upper-case    \A\B\C\D\E\F\G\H\I\J\K\L\M\N\O\P\Q\R\S\T\U\V\W\X\Y\Z
%   Lower-case    \a\b\c\d\e\f\g\h\i\j\k\l\m\n\o\p\q\r\s\t\u\v\w\x\y\z
%   Digits        \0\1\2\3\4\5\6\7\8\9
%   Exclamation   \!     Double quote  \"     Hash (number) \#
%   Dollar        \$     Percent       \%     Ampersand     \&
%   Acute accent  \'     Left paren    \(     Right paren   \)
%   Asterisk      \*     Plus          \+     Comma         \,
%   Minus         \-     Point         \.     Solidus       \/
%   Colon         \:     Semicolon     \;     Less than     \<
%   Equals        \=     Greater than  \>     Question mark \?
%   Commercial at \@     Left bracket  \[     Backslash     \\
%   Right bracket \]     Circumflex    \^     Underscore    \_
%   Grave accent  \`     Left brace    \{     Vertical bar  \|
%   Right brace   \}     Tilde         \~}
%
% \GetFileInfo{pdfcolparcolumns.drv}
%
% \title{The \xpackage{pdfcolparcolumns} package}
% \date{2016/05/16 v1.4}
% \author{Heiko Oberdiek\thanks
% {Please report any issues at https://github.com/ho-tex/oberdiek/issues}\\
% \xemail{heiko.oberdiek at googlemail.com}}
%
% \maketitle
%
% \begin{abstract}
% Since version 1.40 \pdfTeX\ supports several color stacks.
% This package uses them to fix color problems in
% package \xpackage{parcolumns}.
% \end{abstract}
%
% \tableofcontents
%
% \section{Usage}
%
% \begin{quote}
% |\usepackage{pdfcolparcolumns}|
% \end{quote}
% The package \xpackage{pdfcolparcolumns} loads package \xpackage{parcolums}
% \cite{parcolumns}. If color stacks are available then the
% macros of \xpackage{parcolumns} are patched to add support
% for color stacks.
%
% \subsection{Option \xoption{rulebetweencolor}}
%
% Package \xpackage{pdfcolparcolumns} also fixes the color for the
% rule between columns (if \xoption{rulebetween} is set).
% Default color is \cs{normalcolor}. But this can be changed by using
% option \xoption{rulebetweencolor}. It takes a color specification
% as value. If the value is empty, then the default (\cs{normalcolor})
% is used.
% Examples:
% \begin{quote}
%   |rulebetweencolor=blue|,\\
%   |rulebetweencolor={red}|,\\
%   |rulebetweencolor={}|, \textit{\% \cs{normalcolor} is used}\\
%   |rulebetweencolor=[rgb]{1,0,.5}| \textit{\% see below}
% \end{quote}
% If used inside the optional argument of environment \textsf{parcolumns}
% and the value contains an optional argument, then whole value
% must be put in curly braces:
%\begin{quote}
%\begin{verbatim}
%\begin{parcolumns}[
%  rulebetween,
%  rulebetweencolor={[rgb]{1,0,.5}},
%]{2}
%  ...
%\end{parcolumns}
%\end{verbatim}
%\end{quote}
% This option \xoption{rulebetweencolor} can also be set using
% \cs{setkeys}:
%\begin{quote}
%\begin{verbatim}
%\setkeys{parcolumns}{rulebetweencolor=green}
%\end{verbatim}
%\end{quote}
%
% \subsection{Future}
%
% Currently package \xpackage{parcolumns} does not seem to be
% maintained. Nevertheless if there will be a new version that
% adds support for color stacks, then this package may become
% obsolete.
%
% \StopEventually{
% }
%
% \section{Implementation}
%
% \subsection{Identification}
%
%    \begin{macrocode}
%<*package>
\NeedsTeXFormat{LaTeX2e}
\ProvidesPackage{pdfcolparcolumns}%
  [2016/05/16 v1.4 Color stacks for parcolumns (HO)]%
%    \end{macrocode}
%
% \subsection{Load packages}
%
% \subsubsection{Package \xpackage{parcolumns}}
%
%    Currently package \xpackage{parcolumns} does not define options.
%    Thus it is just a precaution that the options of
%    package \xpackage{pdfcolparcolumns} are passed to
%    package \xpackage{parcolumns}.
%    \begin{macrocode}
\DeclareOption*{%
  \PassoptionsToPackage{\CurrentOption}{parcolumns}%
}
\ProcessOptions\relax
\RequirePackage{parcolumns}[2004/11/25]
%    \end{macrocode}
%
% \subsubsection{Package \xpackage{pdfcol}}
%
%    \begin{macrocode}
\RequirePackage{pdfcol}[2007/09/09]
\ifpdfcolAvailable
\else
  \PackageInfo{pdfcolparcolumns}{%
    Loading aborted, because color stacks are not available%
  }%
  \expandafter\endinput
\fi
%    \end{macrocode}
%
% \subsubsection{Package \xpackage{infwarerr}}
%
%    \begin{macrocode}
\RequirePackage{infwarerr}[2007/09/09]
%    \end{macrocode}
%
% \subsection{Color stack macros}
%
%    \begin{macro}{\pcpc@MaxStack}
%    Macro \cs{pcpc@MaxStack} holds the highest number of
%    allocated stacks.
%    \begin{macrocode}
\global\chardef\pcpc@MaxStack=\z@
%    \end{macrocode}
%    \end{macro}
%    \begin{macro}{\pcpc@InitStacks}
%    Macro \cs{pcpc@InitStacks} takes the number of columns
%    as argument and ensures that there are enough color
%    stacks for all columns.
%    \begin{macrocode}
\def\pcpc@InitStacks#1{%
  \ifnum#1>\pcpc@MaxStack
    \begingroup
      \count@\pcpc@MaxStack
      \loop
        \advance\count@\@ne
        \pdfcolInitStack{pcpc@\the\count@}%
      \ifnum#1>\count@
      \repeat
      \global\chardef\pcpc@MaxStack=\count@
    \endgroup
  \fi
}
%    \end{macrocode}
%    \end{macro}
%
%    \begin{macro}{\pcpc@SwitchStack}
%    \begin{macrocode}
\def\pcpc@SwitchStack#1{%
  \pdfcolSwitchStack{pcpc@\number#1}%
}
%    \end{macrocode}
%    \end{macro}
%
%    \begin{macro}{\pcpc@SetCurrent}
%    \begin{macrocode}
\def\pcpc@SetCurrent#1{%
  \pdfcolSetCurrent{pcpc@\number#1}%
}
%    \end{macrocode}
%    \end{macro}
%
% \subsection{Patches}
%
%     Now the color stack macros are patched into the macros
%     of package \xpackage{parcolumns}.
%
% \subsubsection{Init stacks}
%
%    \cs{pcpc@InitStacks} should go into the definition of
%    environment |parcolumns|. \cs{pc@alloccolumns} is executed
%    there and nowhere else, thus we hook into it.
%    \begin{macrocode}
\g@addto@macro\pc@alloccolumns{%
  \pcpc@InitStacks\pc@columncount
}
%    \end{macrocode}
%
% \subsubsection{Switch stack}
%
%    \cs{pcpc@SwitchStack} should be called by marco \cs{colchunk@}.
%    However it is easier to patch \cs{pc@setcolumnwidth} that
%    is executed in \cs{colchunk@} only.
%    \begin{macrocode}
\g@addto@macro\pc@setcolumnwidth{%
  \pcpc@SwitchStack\pc@columnctr
}
%    \end{macrocode}
%
% \subsubsection{Set current stack color}
%
%    \cs{pcpc@SetCurrent} is set at the begin of each line.
%    It must be inserted into \cs{pc@placeboxes}. Unhappily
%    there is no easy way. Therefore we check and
%    redefine \cs{pc@placeboxes}.
%    \begin{macrocode}
\begingroup
  \def\x{%
    \global\let\@tempa\relax
    \count@\z@
    \hb@xt@\linewidth{%
      \vfuzz30ex %
      \vbadness\@M
      \splittopskip\z@skip
      \loop
      \ifnum\count@<\pc@columncount
        \advance\count@\@ne
        \expandafter\ifvoid\csname pc@column@\number\count@\endcsname
          \hskip\csname pc@column@width@\number\count@\endcsname
        \else
          \expandafter\setbox\expandafter\@tempboxa\expandafter
          \vsplit\csname pc@column@\number\count@\endcsname
              to \dp\strutbox
          \vbox{%
            \unvbox\@tempboxa
          }%
        \fi
        \expandafter\ifvoid\csname pc@column@\number\count@\endcsname
        \else
          \global\let\@tempa\pc@placeboxes
        \fi
        \ifnum\count@<\pc@columncount
          \strut
          \hfill
          \ifpc@rulebetween
            \vrule
            \hfill
          \fi
        \fi
      \repeat
    }%
    \@tempa
  }%
  \ifx\x\pc@placeboxes
  \else
    \@PackageWarningNoLine{pdfcolparcolumns}{%
      Command \string\pc@placeboxes\space has changed.\MessageBreak
      Supported versions of package `parcolumns':\MessageBreak
      \space\space 2004/08/05.\MessageBreak
      The redefinition of \string\pc@placeboxes\space may not%
      \MessageBreak
      behave correctly depending on the changes%
    }%
  \fi
\endgroup
%    \end{macrocode}
%    \begin{macro}{\pc@placeboxes}
%    \begin{macrocode}
\renewcommand*{\pc@placeboxes}{%
  \global\let\@tempa\relax
  \count@\z@
  \hb@xt@\linewidth{%
    \vfuzz30ex %
    \vbadness\@M
    \splittopskip\z@skip
    \loop
    \ifnum\count@<\pc@columncount
      \advance\count@\@ne
      \expandafter\ifvoid\csname pc@column@\number\count@\endcsname
        \hskip\csname pc@column@width@\number\count@\endcsname
      \else
        \expandafter\setbox\expandafter\@tempboxa\expandafter
        \vsplit\csname pc@column@\number\count@\endcsname
            to \dp\strutbox
        \vbox{%
          \pcpc@SetCurrent\count@
          \unvbox\@tempboxa
        }%
      \fi
      \expandafter\ifvoid\csname pc@column@\number\count@\endcsname
      \else
        \global\let\@tempa\pc@placeboxes
      \fi
      \ifnum\count@<\pc@columncount
        \strut
        \hfill
        \ifpc@rulebetween
          \begingroup
            \pcpc@RuleBetweenColor
            \vrule
          \endgroup
          \hfill
        \fi
      \fi
    \repeat
  }%
  \@tempa
}
%    \end{macrocode}
%    \end{macro}
%    \begin{macro}{\pcpc@RuleBetweenColorDefault}
%    \begin{macrocode}
\def\pcpc@RuleBetweenColorDefault{%
  \normalcolor
}
%    \end{macrocode}
%    \end{macro}
%    \begin{macro}{\pcpc@RuleBetweenColor}
%    \begin{macrocode}
\let\pcpc@RuleBetweenColor\pcpc@RuleBetweenColorDefault
%    \end{macrocode}
%    \end{macro}
%    \begin{macrocode}
\define@key{parcolumns}{rulebetweencolor}{%
  \edef\pcpc@temp{#1}%
  \ifx\pcpc@temp\@empty
    \let\pcpc@RuleBetweenColor\pcpc@RuleBetweenColorDefault
  \else
    \edef\pcpc@temp{%
      \noexpand\@ifnextchar[{%
        \def\noexpand\pcpc@RuleBetweenColor{%
          \noexpand\color\pcpc@temp
        }%
        \noexpand\pcpc@GobbleNil
      }{%
        \def\noexpand\pcpc@RuleBetweenColor{%
          \noexpand\color{\pcpc@temp}%
        }%
        \noexpand\pcpc@GobbleNil
      }%
      \pcpc@temp\noexpand\@nil
    }%
    \pcpc@temp
  \fi
}
%    \end{macrocode}
%    \begin{macro}{\pcpc@GobbleNil}
%    \begin{macrocode}
\long\def\pcpc@GobbleNil#1\@nil{}
%    \end{macrocode}
%    \end{macro}
%
%    \begin{macrocode}
%</package>
%    \end{macrocode}
%
% \section{Test}
%
%    The test file is a modified version of the file that
%    Donald Goodman has posted in \xnewsgroup{comp.text.tex}: ^^A
%    \URL{``\link{Re: \xpackage{xcolor} glitches}''}^^A
%    {http://groups.google.com/group/comp.text.tex/msg/8eda74ed292012bb}
%    \begin{macrocode}
%<*test1>
\NeedsTeXFormat{LaTeX2e}
\AtEndDocument{%
  \typeout{}%
  \typeout{**************************************}%
  \typeout{*** \space Check the PDF file manually! \space ***}%
  \typeout{**************************************}%
  \typeout{}%
}
\documentclass{article}
\usepackage{xcolor}
\usepackage{pdfcolparcolumns}

\newcommand{\instruct}[1]{%
  \noindent
  \footnotesize
  \textcolor{red}{#1}%
}

\begin{document}
  \begin{parcolumns}[colwidths={1=2.3in,2=2.3in},sloppy]{2}%
    \colchunk[1]{%
      \instruct{Et non dicitur versus} %
      Fidelium anim\ae\ %
      \instruct{%
        sed immediate subiungitur antiphona finalis %
        beat\ae\ Mari\ae\ Virginis%
      } %
      100.%
    }%
    \colchunk[2]{%
      \instruct{%
        And the verse %
        \textcolor{black}{May the souls of the faithful} %
        is not said, but the final antiphon of the %
        Blessed Virgin Mary, %
        \textcolor{black}{100,} %
        is immediately joined.%
      }%
    }%
  \end{parcolumns}%
\end{document}
%</test1>
%    \end{macrocode}
%
% \section{Installation}
%
% \subsection{Download}
%
% \paragraph{Package.} This package is available on
% CTAN\footnote{\url{http://ctan.org/pkg/pdfcolparcolumns}}:
% \begin{description}
% \item[\CTAN{macros/latex/contrib/oberdiek/pdfcolparcolumns.dtx}] The source file.
% \item[\CTAN{macros/latex/contrib/oberdiek/pdfcolparcolumns.pdf}] Documentation.
% \end{description}
%
%
% \paragraph{Bundle.} All the packages of the bundle `oberdiek'
% are also available in a TDS compliant ZIP archive. There
% the packages are already unpacked and the documentation files
% are generated. The files and directories obey the TDS standard.
% \begin{description}
% \item[\CTAN{install/macros/latex/contrib/oberdiek.tds.zip}]
% \end{description}
% \emph{TDS} refers to the standard ``A Directory Structure
% for \TeX\ Files'' (\CTAN{tds/tds.pdf}). Directories
% with \xfile{texmf} in their name are usually organized this way.
%
% \subsection{Bundle installation}
%
% \paragraph{Unpacking.} Unpack the \xfile{oberdiek.tds.zip} in the
% TDS tree (also known as \xfile{texmf} tree) of your choice.
% Example (linux):
% \begin{quote}
%   |unzip oberdiek.tds.zip -d ~/texmf|
% \end{quote}
%
% \paragraph{Script installation.}
% Check the directory \xfile{TDS:scripts/oberdiek/} for
% scripts that need further installation steps.
% Package \xpackage{attachfile2} comes with the Perl script
% \xfile{pdfatfi.pl} that should be installed in such a way
% that it can be called as \texttt{pdfatfi}.
% Example (linux):
% \begin{quote}
%   |chmod +x scripts/oberdiek/pdfatfi.pl|\\
%   |cp scripts/oberdiek/pdfatfi.pl /usr/local/bin/|
% \end{quote}
%
% \subsection{Package installation}
%
% \paragraph{Unpacking.} The \xfile{.dtx} file is a self-extracting
% \docstrip\ archive. The files are extracted by running the
% \xfile{.dtx} through \plainTeX:
% \begin{quote}
%   \verb|tex pdfcolparcolumns.dtx|
% \end{quote}
%
% \paragraph{TDS.} Now the different files must be moved into
% the different directories in your installation TDS tree
% (also known as \xfile{texmf} tree):
% \begin{quote}
% \def\t{^^A
% \begin{tabular}{@{}>{\ttfamily}l@{ $\rightarrow$ }>{\ttfamily}l@{}}
%   pdfcolparcolumns.sty & tex/latex/oberdiek/pdfcolparcolumns.sty\\
%   pdfcolparcolumns.pdf & doc/latex/oberdiek/pdfcolparcolumns.pdf\\
%   test/pdfcolparcolumns-test1.tex & doc/latex/oberdiek/test/pdfcolparcolumns-test1.tex\\
%   pdfcolparcolumns.dtx & source/latex/oberdiek/pdfcolparcolumns.dtx\\
% \end{tabular}^^A
% }^^A
% \sbox0{\t}^^A
% \ifdim\wd0>\linewidth
%   \begingroup
%     \advance\linewidth by\leftmargin
%     \advance\linewidth by\rightmargin
%   \edef\x{\endgroup
%     \def\noexpand\lw{\the\linewidth}^^A
%   }\x
%   \def\lwbox{^^A
%     \leavevmode
%     \hbox to \linewidth{^^A
%       \kern-\leftmargin\relax
%       \hss
%       \usebox0
%       \hss
%       \kern-\rightmargin\relax
%     }^^A
%   }^^A
%   \ifdim\wd0>\lw
%     \sbox0{\small\t}^^A
%     \ifdim\wd0>\linewidth
%       \ifdim\wd0>\lw
%         \sbox0{\footnotesize\t}^^A
%         \ifdim\wd0>\linewidth
%           \ifdim\wd0>\lw
%             \sbox0{\scriptsize\t}^^A
%             \ifdim\wd0>\linewidth
%               \ifdim\wd0>\lw
%                 \sbox0{\tiny\t}^^A
%                 \ifdim\wd0>\linewidth
%                   \lwbox
%                 \else
%                   \usebox0
%                 \fi
%               \else
%                 \lwbox
%               \fi
%             \else
%               \usebox0
%             \fi
%           \else
%             \lwbox
%           \fi
%         \else
%           \usebox0
%         \fi
%       \else
%         \lwbox
%       \fi
%     \else
%       \usebox0
%     \fi
%   \else
%     \lwbox
%   \fi
% \else
%   \usebox0
% \fi
% \end{quote}
% If you have a \xfile{docstrip.cfg} that configures and enables \docstrip's
% TDS installing feature, then some files can already be in the right
% place, see the documentation of \docstrip.
%
% \subsection{Refresh file name databases}
%
% If your \TeX~distribution
% (\teTeX, \mikTeX, \dots) relies on file name databases, you must refresh
% these. For example, \teTeX\ users run \verb|texhash| or
% \verb|mktexlsr|.
%
% \subsection{Some details for the interested}
%
% \paragraph{Attached source.}
%
% The PDF documentation on CTAN also includes the
% \xfile{.dtx} source file. It can be extracted by
% AcrobatReader 6 or higher. Another option is \textsf{pdftk},
% e.g. unpack the file into the current directory:
% \begin{quote}
%   \verb|pdftk pdfcolparcolumns.pdf unpack_files output .|
% \end{quote}
%
% \paragraph{Unpacking with \LaTeX.}
% The \xfile{.dtx} chooses its action depending on the format:
% \begin{description}
% \item[\plainTeX:] Run \docstrip\ and extract the files.
% \item[\LaTeX:] Generate the documentation.
% \end{description}
% If you insist on using \LaTeX\ for \docstrip\ (really,
% \docstrip\ does not need \LaTeX), then inform the autodetect routine
% about your intention:
% \begin{quote}
%   \verb|latex \let\install=y% \iffalse meta-comment
%
% File: pdfcolparcolumns.dtx
% Version: 2016/05/16 v1.4
% Info: Color stacks for parcolumns
%
% Copyright (C) 2007, 2008, 2010 by
%    Heiko Oberdiek <heiko.oberdiek at googlemail.com>
%    2016
%    https://github.com/ho-tex/oberdiek/issues
%
% This work may be distributed and/or modified under the
% conditions of the LaTeX Project Public License, either
% version 1.3c of this license or (at your option) any later
% version. This version of this license is in
%    http://www.latex-project.org/lppl/lppl-1-3c.txt
% and the latest version of this license is in
%    http://www.latex-project.org/lppl.txt
% and version 1.3 or later is part of all distributions of
% LaTeX version 2005/12/01 or later.
%
% This work has the LPPL maintenance status "maintained".
%
% This Current Maintainer of this work is Heiko Oberdiek.
%
% This work consists of the main source file pdfcolparcolumns.dtx
% and the derived files
%    pdfcolparcolumns.sty, pdfcolparcolumns.pdf, pdfcolparcolumns.ins,
%    pdfcolparcolumns.drv, pdfcolparcolumns-test1.tex.
%
% Distribution:
%    CTAN:macros/latex/contrib/oberdiek/pdfcolparcolumns.dtx
%    CTAN:macros/latex/contrib/oberdiek/pdfcolparcolumns.pdf
%
% Unpacking:
%    (a) If pdfcolparcolumns.ins is present:
%           tex pdfcolparcolumns.ins
%    (b) Without pdfcolparcolumns.ins:
%           tex pdfcolparcolumns.dtx
%    (c) If you insist on using LaTeX
%           latex \let\install=y% \iffalse meta-comment
%
% File: pdfcolparcolumns.dtx
% Version: 2016/05/16 v1.4
% Info: Color stacks for parcolumns
%
% Copyright (C) 2007, 2008, 2010 by
%    Heiko Oberdiek <heiko.oberdiek at googlemail.com>
%    2016
%    https://github.com/ho-tex/oberdiek/issues
%
% This work may be distributed and/or modified under the
% conditions of the LaTeX Project Public License, either
% version 1.3c of this license or (at your option) any later
% version. This version of this license is in
%    http://www.latex-project.org/lppl/lppl-1-3c.txt
% and the latest version of this license is in
%    http://www.latex-project.org/lppl.txt
% and version 1.3 or later is part of all distributions of
% LaTeX version 2005/12/01 or later.
%
% This work has the LPPL maintenance status "maintained".
%
% This Current Maintainer of this work is Heiko Oberdiek.
%
% This work consists of the main source file pdfcolparcolumns.dtx
% and the derived files
%    pdfcolparcolumns.sty, pdfcolparcolumns.pdf, pdfcolparcolumns.ins,
%    pdfcolparcolumns.drv, pdfcolparcolumns-test1.tex.
%
% Distribution:
%    CTAN:macros/latex/contrib/oberdiek/pdfcolparcolumns.dtx
%    CTAN:macros/latex/contrib/oberdiek/pdfcolparcolumns.pdf
%
% Unpacking:
%    (a) If pdfcolparcolumns.ins is present:
%           tex pdfcolparcolumns.ins
%    (b) Without pdfcolparcolumns.ins:
%           tex pdfcolparcolumns.dtx
%    (c) If you insist on using LaTeX
%           latex \let\install=y\input{pdfcolparcolumns.dtx}
%        (quote the arguments according to the demands of your shell)
%
% Documentation:
%    (a) If pdfcolparcolumns.drv is present:
%           latex pdfcolparcolumns.drv
%    (b) Without pdfcolparcolumns.drv:
%           latex pdfcolparcolumns.dtx; ...
%    The class ltxdoc loads the configuration file ltxdoc.cfg
%    if available. Here you can specify further options, e.g.
%    use A4 as paper format:
%       \PassOptionsToClass{a4paper}{article}
%
%    Programm calls to get the documentation (example):
%       pdflatex pdfcolparcolumns.dtx
%       makeindex -s gind.ist pdfcolparcolumns.idx
%       pdflatex pdfcolparcolumns.dtx
%       makeindex -s gind.ist pdfcolparcolumns.idx
%       pdflatex pdfcolparcolumns.dtx
%
% Installation:
%    TDS:tex/latex/oberdiek/pdfcolparcolumns.sty
%    TDS:doc/latex/oberdiek/pdfcolparcolumns.pdf
%    TDS:doc/latex/oberdiek/test/pdfcolparcolumns-test1.tex
%    TDS:source/latex/oberdiek/pdfcolparcolumns.dtx
%
%<*ignore>
\begingroup
  \catcode123=1 %
  \catcode125=2 %
  \def\x{LaTeX2e}%
\expandafter\endgroup
\ifcase 0\ifx\install y1\fi\expandafter
         \ifx\csname processbatchFile\endcsname\relax\else1\fi
         \ifx\fmtname\x\else 1\fi\relax
\else\csname fi\endcsname
%</ignore>
%<*install>
\input docstrip.tex
\Msg{************************************************************************}
\Msg{* Installation}
\Msg{* Package: pdfcolparcolumns 2016/05/16 v1.4 Color stacks for parcolumns (HO)}
\Msg{************************************************************************}

\keepsilent
\askforoverwritefalse

\let\MetaPrefix\relax
\preamble

This is a generated file.

Project: pdfcolparcolumns
Version: 2016/05/16 v1.4

Copyright (C) 2007, 2008, 2010 by
   Heiko Oberdiek <heiko.oberdiek at googlemail.com>

This work may be distributed and/or modified under the
conditions of the LaTeX Project Public License, either
version 1.3c of this license or (at your option) any later
version. This version of this license is in
   http://www.latex-project.org/lppl/lppl-1-3c.txt
and the latest version of this license is in
   http://www.latex-project.org/lppl.txt
and version 1.3 or later is part of all distributions of
LaTeX version 2005/12/01 or later.

This work has the LPPL maintenance status "maintained".

This Current Maintainer of this work is Heiko Oberdiek.

This work consists of the main source file pdfcolparcolumns.dtx
and the derived files
   pdfcolparcolumns.sty, pdfcolparcolumns.pdf, pdfcolparcolumns.ins,
   pdfcolparcolumns.drv, pdfcolparcolumns-test1.tex.

\endpreamble
\let\MetaPrefix\DoubleperCent

\generate{%
  \file{pdfcolparcolumns.ins}{\from{pdfcolparcolumns.dtx}{install}}%
  \file{pdfcolparcolumns.drv}{\from{pdfcolparcolumns.dtx}{driver}}%
  \usedir{tex/latex/oberdiek}%
  \file{pdfcolparcolumns.sty}{\from{pdfcolparcolumns.dtx}{package}}%
  \usedir{doc/latex/oberdiek/test}%
  \file{pdfcolparcolumns-test1.tex}{\from{pdfcolparcolumns.dtx}{test1}}%
  \nopreamble
  \nopostamble
  \usedir{source/latex/oberdiek/catalogue}%
  \file{pdfcolparcolumns.xml}{\from{pdfcolparcolumns.dtx}{catalogue}}%
}

\catcode32=13\relax% active space
\let =\space%
\Msg{************************************************************************}
\Msg{*}
\Msg{* To finish the installation you have to move the following}
\Msg{* file into a directory searched by TeX:}
\Msg{*}
\Msg{*     pdfcolparcolumns.sty}
\Msg{*}
\Msg{* To produce the documentation run the file `pdfcolparcolumns.drv'}
\Msg{* through LaTeX.}
\Msg{*}
\Msg{* Happy TeXing!}
\Msg{*}
\Msg{************************************************************************}

\endbatchfile
%</install>
%<*ignore>
\fi
%</ignore>
%<*driver>
\NeedsTeXFormat{LaTeX2e}
\ProvidesFile{pdfcolparcolumns.drv}%
  [2016/05/16 v1.4 Color stacks for parcolumns (HO)]%
\documentclass{ltxdoc}
\usepackage{holtxdoc}[2011/11/22]
\begin{document}
  \DocInput{pdfcolparcolumns.dtx}%
\end{document}
%</driver>
% \fi
%
%
% \CharacterTable
%  {Upper-case    \A\B\C\D\E\F\G\H\I\J\K\L\M\N\O\P\Q\R\S\T\U\V\W\X\Y\Z
%   Lower-case    \a\b\c\d\e\f\g\h\i\j\k\l\m\n\o\p\q\r\s\t\u\v\w\x\y\z
%   Digits        \0\1\2\3\4\5\6\7\8\9
%   Exclamation   \!     Double quote  \"     Hash (number) \#
%   Dollar        \$     Percent       \%     Ampersand     \&
%   Acute accent  \'     Left paren    \(     Right paren   \)
%   Asterisk      \*     Plus          \+     Comma         \,
%   Minus         \-     Point         \.     Solidus       \/
%   Colon         \:     Semicolon     \;     Less than     \<
%   Equals        \=     Greater than  \>     Question mark \?
%   Commercial at \@     Left bracket  \[     Backslash     \\
%   Right bracket \]     Circumflex    \^     Underscore    \_
%   Grave accent  \`     Left brace    \{     Vertical bar  \|
%   Right brace   \}     Tilde         \~}
%
% \GetFileInfo{pdfcolparcolumns.drv}
%
% \title{The \xpackage{pdfcolparcolumns} package}
% \date{2016/05/16 v1.4}
% \author{Heiko Oberdiek\thanks
% {Please report any issues at https://github.com/ho-tex/oberdiek/issues}\\
% \xemail{heiko.oberdiek at googlemail.com}}
%
% \maketitle
%
% \begin{abstract}
% Since version 1.40 \pdfTeX\ supports several color stacks.
% This package uses them to fix color problems in
% package \xpackage{parcolumns}.
% \end{abstract}
%
% \tableofcontents
%
% \section{Usage}
%
% \begin{quote}
% |\usepackage{pdfcolparcolumns}|
% \end{quote}
% The package \xpackage{pdfcolparcolumns} loads package \xpackage{parcolums}
% \cite{parcolumns}. If color stacks are available then the
% macros of \xpackage{parcolumns} are patched to add support
% for color stacks.
%
% \subsection{Option \xoption{rulebetweencolor}}
%
% Package \xpackage{pdfcolparcolumns} also fixes the color for the
% rule between columns (if \xoption{rulebetween} is set).
% Default color is \cs{normalcolor}. But this can be changed by using
% option \xoption{rulebetweencolor}. It takes a color specification
% as value. If the value is empty, then the default (\cs{normalcolor})
% is used.
% Examples:
% \begin{quote}
%   |rulebetweencolor=blue|,\\
%   |rulebetweencolor={red}|,\\
%   |rulebetweencolor={}|, \textit{\% \cs{normalcolor} is used}\\
%   |rulebetweencolor=[rgb]{1,0,.5}| \textit{\% see below}
% \end{quote}
% If used inside the optional argument of environment \textsf{parcolumns}
% and the value contains an optional argument, then whole value
% must be put in curly braces:
%\begin{quote}
%\begin{verbatim}
%\begin{parcolumns}[
%  rulebetween,
%  rulebetweencolor={[rgb]{1,0,.5}},
%]{2}
%  ...
%\end{parcolumns}
%\end{verbatim}
%\end{quote}
% This option \xoption{rulebetweencolor} can also be set using
% \cs{setkeys}:
%\begin{quote}
%\begin{verbatim}
%\setkeys{parcolumns}{rulebetweencolor=green}
%\end{verbatim}
%\end{quote}
%
% \subsection{Future}
%
% Currently package \xpackage{parcolumns} does not seem to be
% maintained. Nevertheless if there will be a new version that
% adds support for color stacks, then this package may become
% obsolete.
%
% \StopEventually{
% }
%
% \section{Implementation}
%
% \subsection{Identification}
%
%    \begin{macrocode}
%<*package>
\NeedsTeXFormat{LaTeX2e}
\ProvidesPackage{pdfcolparcolumns}%
  [2016/05/16 v1.4 Color stacks for parcolumns (HO)]%
%    \end{macrocode}
%
% \subsection{Load packages}
%
% \subsubsection{Package \xpackage{parcolumns}}
%
%    Currently package \xpackage{parcolumns} does not define options.
%    Thus it is just a precaution that the options of
%    package \xpackage{pdfcolparcolumns} are passed to
%    package \xpackage{parcolumns}.
%    \begin{macrocode}
\DeclareOption*{%
  \PassoptionsToPackage{\CurrentOption}{parcolumns}%
}
\ProcessOptions\relax
\RequirePackage{parcolumns}[2004/11/25]
%    \end{macrocode}
%
% \subsubsection{Package \xpackage{pdfcol}}
%
%    \begin{macrocode}
\RequirePackage{pdfcol}[2007/09/09]
\ifpdfcolAvailable
\else
  \PackageInfo{pdfcolparcolumns}{%
    Loading aborted, because color stacks are not available%
  }%
  \expandafter\endinput
\fi
%    \end{macrocode}
%
% \subsubsection{Package \xpackage{infwarerr}}
%
%    \begin{macrocode}
\RequirePackage{infwarerr}[2007/09/09]
%    \end{macrocode}
%
% \subsection{Color stack macros}
%
%    \begin{macro}{\pcpc@MaxStack}
%    Macro \cs{pcpc@MaxStack} holds the highest number of
%    allocated stacks.
%    \begin{macrocode}
\global\chardef\pcpc@MaxStack=\z@
%    \end{macrocode}
%    \end{macro}
%    \begin{macro}{\pcpc@InitStacks}
%    Macro \cs{pcpc@InitStacks} takes the number of columns
%    as argument and ensures that there are enough color
%    stacks for all columns.
%    \begin{macrocode}
\def\pcpc@InitStacks#1{%
  \ifnum#1>\pcpc@MaxStack
    \begingroup
      \count@\pcpc@MaxStack
      \loop
        \advance\count@\@ne
        \pdfcolInitStack{pcpc@\the\count@}%
      \ifnum#1>\count@
      \repeat
      \global\chardef\pcpc@MaxStack=\count@
    \endgroup
  \fi
}
%    \end{macrocode}
%    \end{macro}
%
%    \begin{macro}{\pcpc@SwitchStack}
%    \begin{macrocode}
\def\pcpc@SwitchStack#1{%
  \pdfcolSwitchStack{pcpc@\number#1}%
}
%    \end{macrocode}
%    \end{macro}
%
%    \begin{macro}{\pcpc@SetCurrent}
%    \begin{macrocode}
\def\pcpc@SetCurrent#1{%
  \pdfcolSetCurrent{pcpc@\number#1}%
}
%    \end{macrocode}
%    \end{macro}
%
% \subsection{Patches}
%
%     Now the color stack macros are patched into the macros
%     of package \xpackage{parcolumns}.
%
% \subsubsection{Init stacks}
%
%    \cs{pcpc@InitStacks} should go into the definition of
%    environment |parcolumns|. \cs{pc@alloccolumns} is executed
%    there and nowhere else, thus we hook into it.
%    \begin{macrocode}
\g@addto@macro\pc@alloccolumns{%
  \pcpc@InitStacks\pc@columncount
}
%    \end{macrocode}
%
% \subsubsection{Switch stack}
%
%    \cs{pcpc@SwitchStack} should be called by marco \cs{colchunk@}.
%    However it is easier to patch \cs{pc@setcolumnwidth} that
%    is executed in \cs{colchunk@} only.
%    \begin{macrocode}
\g@addto@macro\pc@setcolumnwidth{%
  \pcpc@SwitchStack\pc@columnctr
}
%    \end{macrocode}
%
% \subsubsection{Set current stack color}
%
%    \cs{pcpc@SetCurrent} is set at the begin of each line.
%    It must be inserted into \cs{pc@placeboxes}. Unhappily
%    there is no easy way. Therefore we check and
%    redefine \cs{pc@placeboxes}.
%    \begin{macrocode}
\begingroup
  \def\x{%
    \global\let\@tempa\relax
    \count@\z@
    \hb@xt@\linewidth{%
      \vfuzz30ex %
      \vbadness\@M
      \splittopskip\z@skip
      \loop
      \ifnum\count@<\pc@columncount
        \advance\count@\@ne
        \expandafter\ifvoid\csname pc@column@\number\count@\endcsname
          \hskip\csname pc@column@width@\number\count@\endcsname
        \else
          \expandafter\setbox\expandafter\@tempboxa\expandafter
          \vsplit\csname pc@column@\number\count@\endcsname
              to \dp\strutbox
          \vbox{%
            \unvbox\@tempboxa
          }%
        \fi
        \expandafter\ifvoid\csname pc@column@\number\count@\endcsname
        \else
          \global\let\@tempa\pc@placeboxes
        \fi
        \ifnum\count@<\pc@columncount
          \strut
          \hfill
          \ifpc@rulebetween
            \vrule
            \hfill
          \fi
        \fi
      \repeat
    }%
    \@tempa
  }%
  \ifx\x\pc@placeboxes
  \else
    \@PackageWarningNoLine{pdfcolparcolumns}{%
      Command \string\pc@placeboxes\space has changed.\MessageBreak
      Supported versions of package `parcolumns':\MessageBreak
      \space\space 2004/08/05.\MessageBreak
      The redefinition of \string\pc@placeboxes\space may not%
      \MessageBreak
      behave correctly depending on the changes%
    }%
  \fi
\endgroup
%    \end{macrocode}
%    \begin{macro}{\pc@placeboxes}
%    \begin{macrocode}
\renewcommand*{\pc@placeboxes}{%
  \global\let\@tempa\relax
  \count@\z@
  \hb@xt@\linewidth{%
    \vfuzz30ex %
    \vbadness\@M
    \splittopskip\z@skip
    \loop
    \ifnum\count@<\pc@columncount
      \advance\count@\@ne
      \expandafter\ifvoid\csname pc@column@\number\count@\endcsname
        \hskip\csname pc@column@width@\number\count@\endcsname
      \else
        \expandafter\setbox\expandafter\@tempboxa\expandafter
        \vsplit\csname pc@column@\number\count@\endcsname
            to \dp\strutbox
        \vbox{%
          \pcpc@SetCurrent\count@
          \unvbox\@tempboxa
        }%
      \fi
      \expandafter\ifvoid\csname pc@column@\number\count@\endcsname
      \else
        \global\let\@tempa\pc@placeboxes
      \fi
      \ifnum\count@<\pc@columncount
        \strut
        \hfill
        \ifpc@rulebetween
          \begingroup
            \pcpc@RuleBetweenColor
            \vrule
          \endgroup
          \hfill
        \fi
      \fi
    \repeat
  }%
  \@tempa
}
%    \end{macrocode}
%    \end{macro}
%    \begin{macro}{\pcpc@RuleBetweenColorDefault}
%    \begin{macrocode}
\def\pcpc@RuleBetweenColorDefault{%
  \normalcolor
}
%    \end{macrocode}
%    \end{macro}
%    \begin{macro}{\pcpc@RuleBetweenColor}
%    \begin{macrocode}
\let\pcpc@RuleBetweenColor\pcpc@RuleBetweenColorDefault
%    \end{macrocode}
%    \end{macro}
%    \begin{macrocode}
\define@key{parcolumns}{rulebetweencolor}{%
  \edef\pcpc@temp{#1}%
  \ifx\pcpc@temp\@empty
    \let\pcpc@RuleBetweenColor\pcpc@RuleBetweenColorDefault
  \else
    \edef\pcpc@temp{%
      \noexpand\@ifnextchar[{%
        \def\noexpand\pcpc@RuleBetweenColor{%
          \noexpand\color\pcpc@temp
        }%
        \noexpand\pcpc@GobbleNil
      }{%
        \def\noexpand\pcpc@RuleBetweenColor{%
          \noexpand\color{\pcpc@temp}%
        }%
        \noexpand\pcpc@GobbleNil
      }%
      \pcpc@temp\noexpand\@nil
    }%
    \pcpc@temp
  \fi
}
%    \end{macrocode}
%    \begin{macro}{\pcpc@GobbleNil}
%    \begin{macrocode}
\long\def\pcpc@GobbleNil#1\@nil{}
%    \end{macrocode}
%    \end{macro}
%
%    \begin{macrocode}
%</package>
%    \end{macrocode}
%
% \section{Test}
%
%    The test file is a modified version of the file that
%    Donald Goodman has posted in \xnewsgroup{comp.text.tex}: ^^A
%    \URL{``\link{Re: \xpackage{xcolor} glitches}''}^^A
%    {http://groups.google.com/group/comp.text.tex/msg/8eda74ed292012bb}
%    \begin{macrocode}
%<*test1>
\NeedsTeXFormat{LaTeX2e}
\AtEndDocument{%
  \typeout{}%
  \typeout{**************************************}%
  \typeout{*** \space Check the PDF file manually! \space ***}%
  \typeout{**************************************}%
  \typeout{}%
}
\documentclass{article}
\usepackage{xcolor}
\usepackage{pdfcolparcolumns}

\newcommand{\instruct}[1]{%
  \noindent
  \footnotesize
  \textcolor{red}{#1}%
}

\begin{document}
  \begin{parcolumns}[colwidths={1=2.3in,2=2.3in},sloppy]{2}%
    \colchunk[1]{%
      \instruct{Et non dicitur versus} %
      Fidelium anim\ae\ %
      \instruct{%
        sed immediate subiungitur antiphona finalis %
        beat\ae\ Mari\ae\ Virginis%
      } %
      100.%
    }%
    \colchunk[2]{%
      \instruct{%
        And the verse %
        \textcolor{black}{May the souls of the faithful} %
        is not said, but the final antiphon of the %
        Blessed Virgin Mary, %
        \textcolor{black}{100,} %
        is immediately joined.%
      }%
    }%
  \end{parcolumns}%
\end{document}
%</test1>
%    \end{macrocode}
%
% \section{Installation}
%
% \subsection{Download}
%
% \paragraph{Package.} This package is available on
% CTAN\footnote{\url{http://ctan.org/pkg/pdfcolparcolumns}}:
% \begin{description}
% \item[\CTAN{macros/latex/contrib/oberdiek/pdfcolparcolumns.dtx}] The source file.
% \item[\CTAN{macros/latex/contrib/oberdiek/pdfcolparcolumns.pdf}] Documentation.
% \end{description}
%
%
% \paragraph{Bundle.} All the packages of the bundle `oberdiek'
% are also available in a TDS compliant ZIP archive. There
% the packages are already unpacked and the documentation files
% are generated. The files and directories obey the TDS standard.
% \begin{description}
% \item[\CTAN{install/macros/latex/contrib/oberdiek.tds.zip}]
% \end{description}
% \emph{TDS} refers to the standard ``A Directory Structure
% for \TeX\ Files'' (\CTAN{tds/tds.pdf}). Directories
% with \xfile{texmf} in their name are usually organized this way.
%
% \subsection{Bundle installation}
%
% \paragraph{Unpacking.} Unpack the \xfile{oberdiek.tds.zip} in the
% TDS tree (also known as \xfile{texmf} tree) of your choice.
% Example (linux):
% \begin{quote}
%   |unzip oberdiek.tds.zip -d ~/texmf|
% \end{quote}
%
% \paragraph{Script installation.}
% Check the directory \xfile{TDS:scripts/oberdiek/} for
% scripts that need further installation steps.
% Package \xpackage{attachfile2} comes with the Perl script
% \xfile{pdfatfi.pl} that should be installed in such a way
% that it can be called as \texttt{pdfatfi}.
% Example (linux):
% \begin{quote}
%   |chmod +x scripts/oberdiek/pdfatfi.pl|\\
%   |cp scripts/oberdiek/pdfatfi.pl /usr/local/bin/|
% \end{quote}
%
% \subsection{Package installation}
%
% \paragraph{Unpacking.} The \xfile{.dtx} file is a self-extracting
% \docstrip\ archive. The files are extracted by running the
% \xfile{.dtx} through \plainTeX:
% \begin{quote}
%   \verb|tex pdfcolparcolumns.dtx|
% \end{quote}
%
% \paragraph{TDS.} Now the different files must be moved into
% the different directories in your installation TDS tree
% (also known as \xfile{texmf} tree):
% \begin{quote}
% \def\t{^^A
% \begin{tabular}{@{}>{\ttfamily}l@{ $\rightarrow$ }>{\ttfamily}l@{}}
%   pdfcolparcolumns.sty & tex/latex/oberdiek/pdfcolparcolumns.sty\\
%   pdfcolparcolumns.pdf & doc/latex/oberdiek/pdfcolparcolumns.pdf\\
%   test/pdfcolparcolumns-test1.tex & doc/latex/oberdiek/test/pdfcolparcolumns-test1.tex\\
%   pdfcolparcolumns.dtx & source/latex/oberdiek/pdfcolparcolumns.dtx\\
% \end{tabular}^^A
% }^^A
% \sbox0{\t}^^A
% \ifdim\wd0>\linewidth
%   \begingroup
%     \advance\linewidth by\leftmargin
%     \advance\linewidth by\rightmargin
%   \edef\x{\endgroup
%     \def\noexpand\lw{\the\linewidth}^^A
%   }\x
%   \def\lwbox{^^A
%     \leavevmode
%     \hbox to \linewidth{^^A
%       \kern-\leftmargin\relax
%       \hss
%       \usebox0
%       \hss
%       \kern-\rightmargin\relax
%     }^^A
%   }^^A
%   \ifdim\wd0>\lw
%     \sbox0{\small\t}^^A
%     \ifdim\wd0>\linewidth
%       \ifdim\wd0>\lw
%         \sbox0{\footnotesize\t}^^A
%         \ifdim\wd0>\linewidth
%           \ifdim\wd0>\lw
%             \sbox0{\scriptsize\t}^^A
%             \ifdim\wd0>\linewidth
%               \ifdim\wd0>\lw
%                 \sbox0{\tiny\t}^^A
%                 \ifdim\wd0>\linewidth
%                   \lwbox
%                 \else
%                   \usebox0
%                 \fi
%               \else
%                 \lwbox
%               \fi
%             \else
%               \usebox0
%             \fi
%           \else
%             \lwbox
%           \fi
%         \else
%           \usebox0
%         \fi
%       \else
%         \lwbox
%       \fi
%     \else
%       \usebox0
%     \fi
%   \else
%     \lwbox
%   \fi
% \else
%   \usebox0
% \fi
% \end{quote}
% If you have a \xfile{docstrip.cfg} that configures and enables \docstrip's
% TDS installing feature, then some files can already be in the right
% place, see the documentation of \docstrip.
%
% \subsection{Refresh file name databases}
%
% If your \TeX~distribution
% (\teTeX, \mikTeX, \dots) relies on file name databases, you must refresh
% these. For example, \teTeX\ users run \verb|texhash| or
% \verb|mktexlsr|.
%
% \subsection{Some details for the interested}
%
% \paragraph{Attached source.}
%
% The PDF documentation on CTAN also includes the
% \xfile{.dtx} source file. It can be extracted by
% AcrobatReader 6 or higher. Another option is \textsf{pdftk},
% e.g. unpack the file into the current directory:
% \begin{quote}
%   \verb|pdftk pdfcolparcolumns.pdf unpack_files output .|
% \end{quote}
%
% \paragraph{Unpacking with \LaTeX.}
% The \xfile{.dtx} chooses its action depending on the format:
% \begin{description}
% \item[\plainTeX:] Run \docstrip\ and extract the files.
% \item[\LaTeX:] Generate the documentation.
% \end{description}
% If you insist on using \LaTeX\ for \docstrip\ (really,
% \docstrip\ does not need \LaTeX), then inform the autodetect routine
% about your intention:
% \begin{quote}
%   \verb|latex \let\install=y\input{pdfcolparcolumns.dtx}|
% \end{quote}
% Do not forget to quote the argument according to the demands
% of your shell.
%
% \paragraph{Generating the documentation.}
% You can use both the \xfile{.dtx} or the \xfile{.drv} to generate
% the documentation. The process can be configured by the
% configuration file \xfile{ltxdoc.cfg}. For instance, put this
% line into this file, if you want to have A4 as paper format:
% \begin{quote}
%   \verb|\PassOptionsToClass{a4paper}{article}|
% \end{quote}
% An example follows how to generate the
% documentation with pdf\LaTeX:
% \begin{quote}
%\begin{verbatim}
%pdflatex pdfcolparcolumns.dtx
%makeindex -s gind.ist pdfcolparcolumns.idx
%pdflatex pdfcolparcolumns.dtx
%makeindex -s gind.ist pdfcolparcolumns.idx
%pdflatex pdfcolparcolumns.dtx
%\end{verbatim}
% \end{quote}
%
% \section{Catalogue}
%
% The following XML file can be used as source for the
% \href{http://mirror.ctan.org/help/Catalogue/catalogue.html}{\TeX\ Catalogue}.
% The elements \texttt{caption} and \texttt{description} are imported
% from the original XML file from the Catalogue.
% The name of the XML file in the Catalogue is \xfile{pdfcolparcolumns.xml}.
%    \begin{macrocode}
%<*catalogue>
<?xml version='1.0' encoding='us-ascii'?>
<!DOCTYPE entry SYSTEM 'catalogue.dtd'>
<entry datestamp='$Date$' modifier='$Author$' id='pdfcolparcolumns'>
  <name>pdfcolparcolumns</name>
  <caption>Fix colour problems in package 'parcolumns'.</caption>
  <authorref id='auth:oberdiek'/>
  <copyright owner='Heiko Oberdiek' year='2007,2008,2010'/>
  <license type='lppl1.3'/>
  <version number='1.4'/>
  <description>
    Since version 1.40 pdfTeX supports colour stacks.
    This package uses them to fix colour problems in
    package <xref refid='parcolumns'>parcolumns</xref>.
    <p/>
    The package is part of the <xref refid='oberdiek'>oberdiek</xref>
    bundle.
  </description>
  <documentation details='Package documentation'
      href='ctan:/macros/latex/contrib/oberdiek/pdfcolparcolumns.pdf'/>
  <ctan file='true' path='/macros/latex/contrib/oberdiek/pdfcolparcolumns.dtx'/>
  <miktex location='oberdiek'/>
  <texlive location='oberdiek'/>
  <install path='/macros/latex/contrib/oberdiek/oberdiek.tds.zip'/>
</entry>
%</catalogue>
%    \end{macrocode}
%
% \begin{thebibliography}{9}
%
% \bibitem{parcolumns}
%   Jonathan Sauer: \textit{The \xpackage{parcolumns} package};
%   2004/11/25;\\
%   \CTAN{macros/latex/contrib/sauerj/parcolumns.pdf}.
%
% \bibitem{pdfcol}
%   Heiko Oberdiek: \textit{The \xpackage{pdfcol} package};
%   2007/09/09;\\
%   \CTAN{macros/latex/contrib/oberdiek/pdfcol.pdf}.
%
% \end{thebibliography}
%
% \begin{History}
%   \begin{Version}{2007/07/26 v1.0}
%   \item
%     First version, published in the newsgroup \xnewsgroup{comp.text.tex}
%     with the name \xpackage{parcolumns-colorstacks}: ^^A no line break
%     \URL{``\link{Re: \xpackage{xcolor} glitches}''}^^A
%     {http://groups.google.com/group/comp.text.tex/msg/56bd897b11bca414}
%   \end{Version}
%   \begin{Version}{2007/09/09 v1.1}
%   \item
%     CTAN version, package name renamed to \xpackage{pdfcolparcolumns}.
%   \item
%     Uses package \xpackage{pdfcol}.
%   \item
%     Documentation added.
%   \item
%     Test file added.
%   \end{Version}
%   \begin{Version}{2008/08/11 v1.2}
%   \item
%     Code is not changed.
%   \item
%     URLs updated.
%   \end{Version}
%   \begin{Version}{2010/01/11 v1.3}
%   \item
%     Fix for rule color.
%   \item
%     New option \xoption{rulebetweencolor} for environment |parcolumns|.
%   \end{Version}
%   \begin{Version}{2016/05/16 v1.4}
%   \item
%     Documentation updates.
%   \end{Version}
% \end{History}
%
% \PrintIndex
%
% \Finale
\endinput

%        (quote the arguments according to the demands of your shell)
%
% Documentation:
%    (a) If pdfcolparcolumns.drv is present:
%           latex pdfcolparcolumns.drv
%    (b) Without pdfcolparcolumns.drv:
%           latex pdfcolparcolumns.dtx; ...
%    The class ltxdoc loads the configuration file ltxdoc.cfg
%    if available. Here you can specify further options, e.g.
%    use A4 as paper format:
%       \PassOptionsToClass{a4paper}{article}
%
%    Programm calls to get the documentation (example):
%       pdflatex pdfcolparcolumns.dtx
%       makeindex -s gind.ist pdfcolparcolumns.idx
%       pdflatex pdfcolparcolumns.dtx
%       makeindex -s gind.ist pdfcolparcolumns.idx
%       pdflatex pdfcolparcolumns.dtx
%
% Installation:
%    TDS:tex/latex/oberdiek/pdfcolparcolumns.sty
%    TDS:doc/latex/oberdiek/pdfcolparcolumns.pdf
%    TDS:doc/latex/oberdiek/test/pdfcolparcolumns-test1.tex
%    TDS:source/latex/oberdiek/pdfcolparcolumns.dtx
%
%<*ignore>
\begingroup
  \catcode123=1 %
  \catcode125=2 %
  \def\x{LaTeX2e}%
\expandafter\endgroup
\ifcase 0\ifx\install y1\fi\expandafter
         \ifx\csname processbatchFile\endcsname\relax\else1\fi
         \ifx\fmtname\x\else 1\fi\relax
\else\csname fi\endcsname
%</ignore>
%<*install>
\input docstrip.tex
\Msg{************************************************************************}
\Msg{* Installation}
\Msg{* Package: pdfcolparcolumns 2016/05/16 v1.4 Color stacks for parcolumns (HO)}
\Msg{************************************************************************}

\keepsilent
\askforoverwritefalse

\let\MetaPrefix\relax
\preamble

This is a generated file.

Project: pdfcolparcolumns
Version: 2016/05/16 v1.4

Copyright (C) 2007, 2008, 2010 by
   Heiko Oberdiek <heiko.oberdiek at googlemail.com>

This work may be distributed and/or modified under the
conditions of the LaTeX Project Public License, either
version 1.3c of this license or (at your option) any later
version. This version of this license is in
   http://www.latex-project.org/lppl/lppl-1-3c.txt
and the latest version of this license is in
   http://www.latex-project.org/lppl.txt
and version 1.3 or later is part of all distributions of
LaTeX version 2005/12/01 or later.

This work has the LPPL maintenance status "maintained".

This Current Maintainer of this work is Heiko Oberdiek.

This work consists of the main source file pdfcolparcolumns.dtx
and the derived files
   pdfcolparcolumns.sty, pdfcolparcolumns.pdf, pdfcolparcolumns.ins,
   pdfcolparcolumns.drv, pdfcolparcolumns-test1.tex.

\endpreamble
\let\MetaPrefix\DoubleperCent

\generate{%
  \file{pdfcolparcolumns.ins}{\from{pdfcolparcolumns.dtx}{install}}%
  \file{pdfcolparcolumns.drv}{\from{pdfcolparcolumns.dtx}{driver}}%
  \usedir{tex/latex/oberdiek}%
  \file{pdfcolparcolumns.sty}{\from{pdfcolparcolumns.dtx}{package}}%
  \usedir{doc/latex/oberdiek/test}%
  \file{pdfcolparcolumns-test1.tex}{\from{pdfcolparcolumns.dtx}{test1}}%
  \nopreamble
  \nopostamble
  \usedir{source/latex/oberdiek/catalogue}%
  \file{pdfcolparcolumns.xml}{\from{pdfcolparcolumns.dtx}{catalogue}}%
}

\catcode32=13\relax% active space
\let =\space%
\Msg{************************************************************************}
\Msg{*}
\Msg{* To finish the installation you have to move the following}
\Msg{* file into a directory searched by TeX:}
\Msg{*}
\Msg{*     pdfcolparcolumns.sty}
\Msg{*}
\Msg{* To produce the documentation run the file `pdfcolparcolumns.drv'}
\Msg{* through LaTeX.}
\Msg{*}
\Msg{* Happy TeXing!}
\Msg{*}
\Msg{************************************************************************}

\endbatchfile
%</install>
%<*ignore>
\fi
%</ignore>
%<*driver>
\NeedsTeXFormat{LaTeX2e}
\ProvidesFile{pdfcolparcolumns.drv}%
  [2016/05/16 v1.4 Color stacks for parcolumns (HO)]%
\documentclass{ltxdoc}
\usepackage{holtxdoc}[2011/11/22]
\begin{document}
  \DocInput{pdfcolparcolumns.dtx}%
\end{document}
%</driver>
% \fi
%
%
% \CharacterTable
%  {Upper-case    \A\B\C\D\E\F\G\H\I\J\K\L\M\N\O\P\Q\R\S\T\U\V\W\X\Y\Z
%   Lower-case    \a\b\c\d\e\f\g\h\i\j\k\l\m\n\o\p\q\r\s\t\u\v\w\x\y\z
%   Digits        \0\1\2\3\4\5\6\7\8\9
%   Exclamation   \!     Double quote  \"     Hash (number) \#
%   Dollar        \$     Percent       \%     Ampersand     \&
%   Acute accent  \'     Left paren    \(     Right paren   \)
%   Asterisk      \*     Plus          \+     Comma         \,
%   Minus         \-     Point         \.     Solidus       \/
%   Colon         \:     Semicolon     \;     Less than     \<
%   Equals        \=     Greater than  \>     Question mark \?
%   Commercial at \@     Left bracket  \[     Backslash     \\
%   Right bracket \]     Circumflex    \^     Underscore    \_
%   Grave accent  \`     Left brace    \{     Vertical bar  \|
%   Right brace   \}     Tilde         \~}
%
% \GetFileInfo{pdfcolparcolumns.drv}
%
% \title{The \xpackage{pdfcolparcolumns} package}
% \date{2016/05/16 v1.4}
% \author{Heiko Oberdiek\thanks
% {Please report any issues at https://github.com/ho-tex/oberdiek/issues}\\
% \xemail{heiko.oberdiek at googlemail.com}}
%
% \maketitle
%
% \begin{abstract}
% Since version 1.40 \pdfTeX\ supports several color stacks.
% This package uses them to fix color problems in
% package \xpackage{parcolumns}.
% \end{abstract}
%
% \tableofcontents
%
% \section{Usage}
%
% \begin{quote}
% |\usepackage{pdfcolparcolumns}|
% \end{quote}
% The package \xpackage{pdfcolparcolumns} loads package \xpackage{parcolums}
% \cite{parcolumns}. If color stacks are available then the
% macros of \xpackage{parcolumns} are patched to add support
% for color stacks.
%
% \subsection{Option \xoption{rulebetweencolor}}
%
% Package \xpackage{pdfcolparcolumns} also fixes the color for the
% rule between columns (if \xoption{rulebetween} is set).
% Default color is \cs{normalcolor}. But this can be changed by using
% option \xoption{rulebetweencolor}. It takes a color specification
% as value. If the value is empty, then the default (\cs{normalcolor})
% is used.
% Examples:
% \begin{quote}
%   |rulebetweencolor=blue|,\\
%   |rulebetweencolor={red}|,\\
%   |rulebetweencolor={}|, \textit{\% \cs{normalcolor} is used}\\
%   |rulebetweencolor=[rgb]{1,0,.5}| \textit{\% see below}
% \end{quote}
% If used inside the optional argument of environment \textsf{parcolumns}
% and the value contains an optional argument, then whole value
% must be put in curly braces:
%\begin{quote}
%\begin{verbatim}
%\begin{parcolumns}[
%  rulebetween,
%  rulebetweencolor={[rgb]{1,0,.5}},
%]{2}
%  ...
%\end{parcolumns}
%\end{verbatim}
%\end{quote}
% This option \xoption{rulebetweencolor} can also be set using
% \cs{setkeys}:
%\begin{quote}
%\begin{verbatim}
%\setkeys{parcolumns}{rulebetweencolor=green}
%\end{verbatim}
%\end{quote}
%
% \subsection{Future}
%
% Currently package \xpackage{parcolumns} does not seem to be
% maintained. Nevertheless if there will be a new version that
% adds support for color stacks, then this package may become
% obsolete.
%
% \StopEventually{
% }
%
% \section{Implementation}
%
% \subsection{Identification}
%
%    \begin{macrocode}
%<*package>
\NeedsTeXFormat{LaTeX2e}
\ProvidesPackage{pdfcolparcolumns}%
  [2016/05/16 v1.4 Color stacks for parcolumns (HO)]%
%    \end{macrocode}
%
% \subsection{Load packages}
%
% \subsubsection{Package \xpackage{parcolumns}}
%
%    Currently package \xpackage{parcolumns} does not define options.
%    Thus it is just a precaution that the options of
%    package \xpackage{pdfcolparcolumns} are passed to
%    package \xpackage{parcolumns}.
%    \begin{macrocode}
\DeclareOption*{%
  \PassoptionsToPackage{\CurrentOption}{parcolumns}%
}
\ProcessOptions\relax
\RequirePackage{parcolumns}[2004/11/25]
%    \end{macrocode}
%
% \subsubsection{Package \xpackage{pdfcol}}
%
%    \begin{macrocode}
\RequirePackage{pdfcol}[2007/09/09]
\ifpdfcolAvailable
\else
  \PackageInfo{pdfcolparcolumns}{%
    Loading aborted, because color stacks are not available%
  }%
  \expandafter\endinput
\fi
%    \end{macrocode}
%
% \subsubsection{Package \xpackage{infwarerr}}
%
%    \begin{macrocode}
\RequirePackage{infwarerr}[2007/09/09]
%    \end{macrocode}
%
% \subsection{Color stack macros}
%
%    \begin{macro}{\pcpc@MaxStack}
%    Macro \cs{pcpc@MaxStack} holds the highest number of
%    allocated stacks.
%    \begin{macrocode}
\global\chardef\pcpc@MaxStack=\z@
%    \end{macrocode}
%    \end{macro}
%    \begin{macro}{\pcpc@InitStacks}
%    Macro \cs{pcpc@InitStacks} takes the number of columns
%    as argument and ensures that there are enough color
%    stacks for all columns.
%    \begin{macrocode}
\def\pcpc@InitStacks#1{%
  \ifnum#1>\pcpc@MaxStack
    \begingroup
      \count@\pcpc@MaxStack
      \loop
        \advance\count@\@ne
        \pdfcolInitStack{pcpc@\the\count@}%
      \ifnum#1>\count@
      \repeat
      \global\chardef\pcpc@MaxStack=\count@
    \endgroup
  \fi
}
%    \end{macrocode}
%    \end{macro}
%
%    \begin{macro}{\pcpc@SwitchStack}
%    \begin{macrocode}
\def\pcpc@SwitchStack#1{%
  \pdfcolSwitchStack{pcpc@\number#1}%
}
%    \end{macrocode}
%    \end{macro}
%
%    \begin{macro}{\pcpc@SetCurrent}
%    \begin{macrocode}
\def\pcpc@SetCurrent#1{%
  \pdfcolSetCurrent{pcpc@\number#1}%
}
%    \end{macrocode}
%    \end{macro}
%
% \subsection{Patches}
%
%     Now the color stack macros are patched into the macros
%     of package \xpackage{parcolumns}.
%
% \subsubsection{Init stacks}
%
%    \cs{pcpc@InitStacks} should go into the definition of
%    environment |parcolumns|. \cs{pc@alloccolumns} is executed
%    there and nowhere else, thus we hook into it.
%    \begin{macrocode}
\g@addto@macro\pc@alloccolumns{%
  \pcpc@InitStacks\pc@columncount
}
%    \end{macrocode}
%
% \subsubsection{Switch stack}
%
%    \cs{pcpc@SwitchStack} should be called by marco \cs{colchunk@}.
%    However it is easier to patch \cs{pc@setcolumnwidth} that
%    is executed in \cs{colchunk@} only.
%    \begin{macrocode}
\g@addto@macro\pc@setcolumnwidth{%
  \pcpc@SwitchStack\pc@columnctr
}
%    \end{macrocode}
%
% \subsubsection{Set current stack color}
%
%    \cs{pcpc@SetCurrent} is set at the begin of each line.
%    It must be inserted into \cs{pc@placeboxes}. Unhappily
%    there is no easy way. Therefore we check and
%    redefine \cs{pc@placeboxes}.
%    \begin{macrocode}
\begingroup
  \def\x{%
    \global\let\@tempa\relax
    \count@\z@
    \hb@xt@\linewidth{%
      \vfuzz30ex %
      \vbadness\@M
      \splittopskip\z@skip
      \loop
      \ifnum\count@<\pc@columncount
        \advance\count@\@ne
        \expandafter\ifvoid\csname pc@column@\number\count@\endcsname
          \hskip\csname pc@column@width@\number\count@\endcsname
        \else
          \expandafter\setbox\expandafter\@tempboxa\expandafter
          \vsplit\csname pc@column@\number\count@\endcsname
              to \dp\strutbox
          \vbox{%
            \unvbox\@tempboxa
          }%
        \fi
        \expandafter\ifvoid\csname pc@column@\number\count@\endcsname
        \else
          \global\let\@tempa\pc@placeboxes
        \fi
        \ifnum\count@<\pc@columncount
          \strut
          \hfill
          \ifpc@rulebetween
            \vrule
            \hfill
          \fi
        \fi
      \repeat
    }%
    \@tempa
  }%
  \ifx\x\pc@placeboxes
  \else
    \@PackageWarningNoLine{pdfcolparcolumns}{%
      Command \string\pc@placeboxes\space has changed.\MessageBreak
      Supported versions of package `parcolumns':\MessageBreak
      \space\space 2004/08/05.\MessageBreak
      The redefinition of \string\pc@placeboxes\space may not%
      \MessageBreak
      behave correctly depending on the changes%
    }%
  \fi
\endgroup
%    \end{macrocode}
%    \begin{macro}{\pc@placeboxes}
%    \begin{macrocode}
\renewcommand*{\pc@placeboxes}{%
  \global\let\@tempa\relax
  \count@\z@
  \hb@xt@\linewidth{%
    \vfuzz30ex %
    \vbadness\@M
    \splittopskip\z@skip
    \loop
    \ifnum\count@<\pc@columncount
      \advance\count@\@ne
      \expandafter\ifvoid\csname pc@column@\number\count@\endcsname
        \hskip\csname pc@column@width@\number\count@\endcsname
      \else
        \expandafter\setbox\expandafter\@tempboxa\expandafter
        \vsplit\csname pc@column@\number\count@\endcsname
            to \dp\strutbox
        \vbox{%
          \pcpc@SetCurrent\count@
          \unvbox\@tempboxa
        }%
      \fi
      \expandafter\ifvoid\csname pc@column@\number\count@\endcsname
      \else
        \global\let\@tempa\pc@placeboxes
      \fi
      \ifnum\count@<\pc@columncount
        \strut
        \hfill
        \ifpc@rulebetween
          \begingroup
            \pcpc@RuleBetweenColor
            \vrule
          \endgroup
          \hfill
        \fi
      \fi
    \repeat
  }%
  \@tempa
}
%    \end{macrocode}
%    \end{macro}
%    \begin{macro}{\pcpc@RuleBetweenColorDefault}
%    \begin{macrocode}
\def\pcpc@RuleBetweenColorDefault{%
  \normalcolor
}
%    \end{macrocode}
%    \end{macro}
%    \begin{macro}{\pcpc@RuleBetweenColor}
%    \begin{macrocode}
\let\pcpc@RuleBetweenColor\pcpc@RuleBetweenColorDefault
%    \end{macrocode}
%    \end{macro}
%    \begin{macrocode}
\define@key{parcolumns}{rulebetweencolor}{%
  \edef\pcpc@temp{#1}%
  \ifx\pcpc@temp\@empty
    \let\pcpc@RuleBetweenColor\pcpc@RuleBetweenColorDefault
  \else
    \edef\pcpc@temp{%
      \noexpand\@ifnextchar[{%
        \def\noexpand\pcpc@RuleBetweenColor{%
          \noexpand\color\pcpc@temp
        }%
        \noexpand\pcpc@GobbleNil
      }{%
        \def\noexpand\pcpc@RuleBetweenColor{%
          \noexpand\color{\pcpc@temp}%
        }%
        \noexpand\pcpc@GobbleNil
      }%
      \pcpc@temp\noexpand\@nil
    }%
    \pcpc@temp
  \fi
}
%    \end{macrocode}
%    \begin{macro}{\pcpc@GobbleNil}
%    \begin{macrocode}
\long\def\pcpc@GobbleNil#1\@nil{}
%    \end{macrocode}
%    \end{macro}
%
%    \begin{macrocode}
%</package>
%    \end{macrocode}
%
% \section{Test}
%
%    The test file is a modified version of the file that
%    Donald Goodman has posted in \xnewsgroup{comp.text.tex}: ^^A
%    \URL{``\link{Re: \xpackage{xcolor} glitches}''}^^A
%    {http://groups.google.com/group/comp.text.tex/msg/8eda74ed292012bb}
%    \begin{macrocode}
%<*test1>
\NeedsTeXFormat{LaTeX2e}
\AtEndDocument{%
  \typeout{}%
  \typeout{**************************************}%
  \typeout{*** \space Check the PDF file manually! \space ***}%
  \typeout{**************************************}%
  \typeout{}%
}
\documentclass{article}
\usepackage{xcolor}
\usepackage{pdfcolparcolumns}

\newcommand{\instruct}[1]{%
  \noindent
  \footnotesize
  \textcolor{red}{#1}%
}

\begin{document}
  \begin{parcolumns}[colwidths={1=2.3in,2=2.3in},sloppy]{2}%
    \colchunk[1]{%
      \instruct{Et non dicitur versus} %
      Fidelium anim\ae\ %
      \instruct{%
        sed immediate subiungitur antiphona finalis %
        beat\ae\ Mari\ae\ Virginis%
      } %
      100.%
    }%
    \colchunk[2]{%
      \instruct{%
        And the verse %
        \textcolor{black}{May the souls of the faithful} %
        is not said, but the final antiphon of the %
        Blessed Virgin Mary, %
        \textcolor{black}{100,} %
        is immediately joined.%
      }%
    }%
  \end{parcolumns}%
\end{document}
%</test1>
%    \end{macrocode}
%
% \section{Installation}
%
% \subsection{Download}
%
% \paragraph{Package.} This package is available on
% CTAN\footnote{\url{http://ctan.org/pkg/pdfcolparcolumns}}:
% \begin{description}
% \item[\CTAN{macros/latex/contrib/oberdiek/pdfcolparcolumns.dtx}] The source file.
% \item[\CTAN{macros/latex/contrib/oberdiek/pdfcolparcolumns.pdf}] Documentation.
% \end{description}
%
%
% \paragraph{Bundle.} All the packages of the bundle `oberdiek'
% are also available in a TDS compliant ZIP archive. There
% the packages are already unpacked and the documentation files
% are generated. The files and directories obey the TDS standard.
% \begin{description}
% \item[\CTAN{install/macros/latex/contrib/oberdiek.tds.zip}]
% \end{description}
% \emph{TDS} refers to the standard ``A Directory Structure
% for \TeX\ Files'' (\CTAN{tds/tds.pdf}). Directories
% with \xfile{texmf} in their name are usually organized this way.
%
% \subsection{Bundle installation}
%
% \paragraph{Unpacking.} Unpack the \xfile{oberdiek.tds.zip} in the
% TDS tree (also known as \xfile{texmf} tree) of your choice.
% Example (linux):
% \begin{quote}
%   |unzip oberdiek.tds.zip -d ~/texmf|
% \end{quote}
%
% \paragraph{Script installation.}
% Check the directory \xfile{TDS:scripts/oberdiek/} for
% scripts that need further installation steps.
% Package \xpackage{attachfile2} comes with the Perl script
% \xfile{pdfatfi.pl} that should be installed in such a way
% that it can be called as \texttt{pdfatfi}.
% Example (linux):
% \begin{quote}
%   |chmod +x scripts/oberdiek/pdfatfi.pl|\\
%   |cp scripts/oberdiek/pdfatfi.pl /usr/local/bin/|
% \end{quote}
%
% \subsection{Package installation}
%
% \paragraph{Unpacking.} The \xfile{.dtx} file is a self-extracting
% \docstrip\ archive. The files are extracted by running the
% \xfile{.dtx} through \plainTeX:
% \begin{quote}
%   \verb|tex pdfcolparcolumns.dtx|
% \end{quote}
%
% \paragraph{TDS.} Now the different files must be moved into
% the different directories in your installation TDS tree
% (also known as \xfile{texmf} tree):
% \begin{quote}
% \def\t{^^A
% \begin{tabular}{@{}>{\ttfamily}l@{ $\rightarrow$ }>{\ttfamily}l@{}}
%   pdfcolparcolumns.sty & tex/latex/oberdiek/pdfcolparcolumns.sty\\
%   pdfcolparcolumns.pdf & doc/latex/oberdiek/pdfcolparcolumns.pdf\\
%   test/pdfcolparcolumns-test1.tex & doc/latex/oberdiek/test/pdfcolparcolumns-test1.tex\\
%   pdfcolparcolumns.dtx & source/latex/oberdiek/pdfcolparcolumns.dtx\\
% \end{tabular}^^A
% }^^A
% \sbox0{\t}^^A
% \ifdim\wd0>\linewidth
%   \begingroup
%     \advance\linewidth by\leftmargin
%     \advance\linewidth by\rightmargin
%   \edef\x{\endgroup
%     \def\noexpand\lw{\the\linewidth}^^A
%   }\x
%   \def\lwbox{^^A
%     \leavevmode
%     \hbox to \linewidth{^^A
%       \kern-\leftmargin\relax
%       \hss
%       \usebox0
%       \hss
%       \kern-\rightmargin\relax
%     }^^A
%   }^^A
%   \ifdim\wd0>\lw
%     \sbox0{\small\t}^^A
%     \ifdim\wd0>\linewidth
%       \ifdim\wd0>\lw
%         \sbox0{\footnotesize\t}^^A
%         \ifdim\wd0>\linewidth
%           \ifdim\wd0>\lw
%             \sbox0{\scriptsize\t}^^A
%             \ifdim\wd0>\linewidth
%               \ifdim\wd0>\lw
%                 \sbox0{\tiny\t}^^A
%                 \ifdim\wd0>\linewidth
%                   \lwbox
%                 \else
%                   \usebox0
%                 \fi
%               \else
%                 \lwbox
%               \fi
%             \else
%               \usebox0
%             \fi
%           \else
%             \lwbox
%           \fi
%         \else
%           \usebox0
%         \fi
%       \else
%         \lwbox
%       \fi
%     \else
%       \usebox0
%     \fi
%   \else
%     \lwbox
%   \fi
% \else
%   \usebox0
% \fi
% \end{quote}
% If you have a \xfile{docstrip.cfg} that configures and enables \docstrip's
% TDS installing feature, then some files can already be in the right
% place, see the documentation of \docstrip.
%
% \subsection{Refresh file name databases}
%
% If your \TeX~distribution
% (\teTeX, \mikTeX, \dots) relies on file name databases, you must refresh
% these. For example, \teTeX\ users run \verb|texhash| or
% \verb|mktexlsr|.
%
% \subsection{Some details for the interested}
%
% \paragraph{Attached source.}
%
% The PDF documentation on CTAN also includes the
% \xfile{.dtx} source file. It can be extracted by
% AcrobatReader 6 or higher. Another option is \textsf{pdftk},
% e.g. unpack the file into the current directory:
% \begin{quote}
%   \verb|pdftk pdfcolparcolumns.pdf unpack_files output .|
% \end{quote}
%
% \paragraph{Unpacking with \LaTeX.}
% The \xfile{.dtx} chooses its action depending on the format:
% \begin{description}
% \item[\plainTeX:] Run \docstrip\ and extract the files.
% \item[\LaTeX:] Generate the documentation.
% \end{description}
% If you insist on using \LaTeX\ for \docstrip\ (really,
% \docstrip\ does not need \LaTeX), then inform the autodetect routine
% about your intention:
% \begin{quote}
%   \verb|latex \let\install=y% \iffalse meta-comment
%
% File: pdfcolparcolumns.dtx
% Version: 2016/05/16 v1.4
% Info: Color stacks for parcolumns
%
% Copyright (C) 2007, 2008, 2010 by
%    Heiko Oberdiek <heiko.oberdiek at googlemail.com>
%    2016
%    https://github.com/ho-tex/oberdiek/issues
%
% This work may be distributed and/or modified under the
% conditions of the LaTeX Project Public License, either
% version 1.3c of this license or (at your option) any later
% version. This version of this license is in
%    http://www.latex-project.org/lppl/lppl-1-3c.txt
% and the latest version of this license is in
%    http://www.latex-project.org/lppl.txt
% and version 1.3 or later is part of all distributions of
% LaTeX version 2005/12/01 or later.
%
% This work has the LPPL maintenance status "maintained".
%
% This Current Maintainer of this work is Heiko Oberdiek.
%
% This work consists of the main source file pdfcolparcolumns.dtx
% and the derived files
%    pdfcolparcolumns.sty, pdfcolparcolumns.pdf, pdfcolparcolumns.ins,
%    pdfcolparcolumns.drv, pdfcolparcolumns-test1.tex.
%
% Distribution:
%    CTAN:macros/latex/contrib/oberdiek/pdfcolparcolumns.dtx
%    CTAN:macros/latex/contrib/oberdiek/pdfcolparcolumns.pdf
%
% Unpacking:
%    (a) If pdfcolparcolumns.ins is present:
%           tex pdfcolparcolumns.ins
%    (b) Without pdfcolparcolumns.ins:
%           tex pdfcolparcolumns.dtx
%    (c) If you insist on using LaTeX
%           latex \let\install=y\input{pdfcolparcolumns.dtx}
%        (quote the arguments according to the demands of your shell)
%
% Documentation:
%    (a) If pdfcolparcolumns.drv is present:
%           latex pdfcolparcolumns.drv
%    (b) Without pdfcolparcolumns.drv:
%           latex pdfcolparcolumns.dtx; ...
%    The class ltxdoc loads the configuration file ltxdoc.cfg
%    if available. Here you can specify further options, e.g.
%    use A4 as paper format:
%       \PassOptionsToClass{a4paper}{article}
%
%    Programm calls to get the documentation (example):
%       pdflatex pdfcolparcolumns.dtx
%       makeindex -s gind.ist pdfcolparcolumns.idx
%       pdflatex pdfcolparcolumns.dtx
%       makeindex -s gind.ist pdfcolparcolumns.idx
%       pdflatex pdfcolparcolumns.dtx
%
% Installation:
%    TDS:tex/latex/oberdiek/pdfcolparcolumns.sty
%    TDS:doc/latex/oberdiek/pdfcolparcolumns.pdf
%    TDS:doc/latex/oberdiek/test/pdfcolparcolumns-test1.tex
%    TDS:source/latex/oberdiek/pdfcolparcolumns.dtx
%
%<*ignore>
\begingroup
  \catcode123=1 %
  \catcode125=2 %
  \def\x{LaTeX2e}%
\expandafter\endgroup
\ifcase 0\ifx\install y1\fi\expandafter
         \ifx\csname processbatchFile\endcsname\relax\else1\fi
         \ifx\fmtname\x\else 1\fi\relax
\else\csname fi\endcsname
%</ignore>
%<*install>
\input docstrip.tex
\Msg{************************************************************************}
\Msg{* Installation}
\Msg{* Package: pdfcolparcolumns 2016/05/16 v1.4 Color stacks for parcolumns (HO)}
\Msg{************************************************************************}

\keepsilent
\askforoverwritefalse

\let\MetaPrefix\relax
\preamble

This is a generated file.

Project: pdfcolparcolumns
Version: 2016/05/16 v1.4

Copyright (C) 2007, 2008, 2010 by
   Heiko Oberdiek <heiko.oberdiek at googlemail.com>

This work may be distributed and/or modified under the
conditions of the LaTeX Project Public License, either
version 1.3c of this license or (at your option) any later
version. This version of this license is in
   http://www.latex-project.org/lppl/lppl-1-3c.txt
and the latest version of this license is in
   http://www.latex-project.org/lppl.txt
and version 1.3 or later is part of all distributions of
LaTeX version 2005/12/01 or later.

This work has the LPPL maintenance status "maintained".

This Current Maintainer of this work is Heiko Oberdiek.

This work consists of the main source file pdfcolparcolumns.dtx
and the derived files
   pdfcolparcolumns.sty, pdfcolparcolumns.pdf, pdfcolparcolumns.ins,
   pdfcolparcolumns.drv, pdfcolparcolumns-test1.tex.

\endpreamble
\let\MetaPrefix\DoubleperCent

\generate{%
  \file{pdfcolparcolumns.ins}{\from{pdfcolparcolumns.dtx}{install}}%
  \file{pdfcolparcolumns.drv}{\from{pdfcolparcolumns.dtx}{driver}}%
  \usedir{tex/latex/oberdiek}%
  \file{pdfcolparcolumns.sty}{\from{pdfcolparcolumns.dtx}{package}}%
  \usedir{doc/latex/oberdiek/test}%
  \file{pdfcolparcolumns-test1.tex}{\from{pdfcolparcolumns.dtx}{test1}}%
  \nopreamble
  \nopostamble
  \usedir{source/latex/oberdiek/catalogue}%
  \file{pdfcolparcolumns.xml}{\from{pdfcolparcolumns.dtx}{catalogue}}%
}

\catcode32=13\relax% active space
\let =\space%
\Msg{************************************************************************}
\Msg{*}
\Msg{* To finish the installation you have to move the following}
\Msg{* file into a directory searched by TeX:}
\Msg{*}
\Msg{*     pdfcolparcolumns.sty}
\Msg{*}
\Msg{* To produce the documentation run the file `pdfcolparcolumns.drv'}
\Msg{* through LaTeX.}
\Msg{*}
\Msg{* Happy TeXing!}
\Msg{*}
\Msg{************************************************************************}

\endbatchfile
%</install>
%<*ignore>
\fi
%</ignore>
%<*driver>
\NeedsTeXFormat{LaTeX2e}
\ProvidesFile{pdfcolparcolumns.drv}%
  [2016/05/16 v1.4 Color stacks for parcolumns (HO)]%
\documentclass{ltxdoc}
\usepackage{holtxdoc}[2011/11/22]
\begin{document}
  \DocInput{pdfcolparcolumns.dtx}%
\end{document}
%</driver>
% \fi
%
%
% \CharacterTable
%  {Upper-case    \A\B\C\D\E\F\G\H\I\J\K\L\M\N\O\P\Q\R\S\T\U\V\W\X\Y\Z
%   Lower-case    \a\b\c\d\e\f\g\h\i\j\k\l\m\n\o\p\q\r\s\t\u\v\w\x\y\z
%   Digits        \0\1\2\3\4\5\6\7\8\9
%   Exclamation   \!     Double quote  \"     Hash (number) \#
%   Dollar        \$     Percent       \%     Ampersand     \&
%   Acute accent  \'     Left paren    \(     Right paren   \)
%   Asterisk      \*     Plus          \+     Comma         \,
%   Minus         \-     Point         \.     Solidus       \/
%   Colon         \:     Semicolon     \;     Less than     \<
%   Equals        \=     Greater than  \>     Question mark \?
%   Commercial at \@     Left bracket  \[     Backslash     \\
%   Right bracket \]     Circumflex    \^     Underscore    \_
%   Grave accent  \`     Left brace    \{     Vertical bar  \|
%   Right brace   \}     Tilde         \~}
%
% \GetFileInfo{pdfcolparcolumns.drv}
%
% \title{The \xpackage{pdfcolparcolumns} package}
% \date{2016/05/16 v1.4}
% \author{Heiko Oberdiek\thanks
% {Please report any issues at https://github.com/ho-tex/oberdiek/issues}\\
% \xemail{heiko.oberdiek at googlemail.com}}
%
% \maketitle
%
% \begin{abstract}
% Since version 1.40 \pdfTeX\ supports several color stacks.
% This package uses them to fix color problems in
% package \xpackage{parcolumns}.
% \end{abstract}
%
% \tableofcontents
%
% \section{Usage}
%
% \begin{quote}
% |\usepackage{pdfcolparcolumns}|
% \end{quote}
% The package \xpackage{pdfcolparcolumns} loads package \xpackage{parcolums}
% \cite{parcolumns}. If color stacks are available then the
% macros of \xpackage{parcolumns} are patched to add support
% for color stacks.
%
% \subsection{Option \xoption{rulebetweencolor}}
%
% Package \xpackage{pdfcolparcolumns} also fixes the color for the
% rule between columns (if \xoption{rulebetween} is set).
% Default color is \cs{normalcolor}. But this can be changed by using
% option \xoption{rulebetweencolor}. It takes a color specification
% as value. If the value is empty, then the default (\cs{normalcolor})
% is used.
% Examples:
% \begin{quote}
%   |rulebetweencolor=blue|,\\
%   |rulebetweencolor={red}|,\\
%   |rulebetweencolor={}|, \textit{\% \cs{normalcolor} is used}\\
%   |rulebetweencolor=[rgb]{1,0,.5}| \textit{\% see below}
% \end{quote}
% If used inside the optional argument of environment \textsf{parcolumns}
% and the value contains an optional argument, then whole value
% must be put in curly braces:
%\begin{quote}
%\begin{verbatim}
%\begin{parcolumns}[
%  rulebetween,
%  rulebetweencolor={[rgb]{1,0,.5}},
%]{2}
%  ...
%\end{parcolumns}
%\end{verbatim}
%\end{quote}
% This option \xoption{rulebetweencolor} can also be set using
% \cs{setkeys}:
%\begin{quote}
%\begin{verbatim}
%\setkeys{parcolumns}{rulebetweencolor=green}
%\end{verbatim}
%\end{quote}
%
% \subsection{Future}
%
% Currently package \xpackage{parcolumns} does not seem to be
% maintained. Nevertheless if there will be a new version that
% adds support for color stacks, then this package may become
% obsolete.
%
% \StopEventually{
% }
%
% \section{Implementation}
%
% \subsection{Identification}
%
%    \begin{macrocode}
%<*package>
\NeedsTeXFormat{LaTeX2e}
\ProvidesPackage{pdfcolparcolumns}%
  [2016/05/16 v1.4 Color stacks for parcolumns (HO)]%
%    \end{macrocode}
%
% \subsection{Load packages}
%
% \subsubsection{Package \xpackage{parcolumns}}
%
%    Currently package \xpackage{parcolumns} does not define options.
%    Thus it is just a precaution that the options of
%    package \xpackage{pdfcolparcolumns} are passed to
%    package \xpackage{parcolumns}.
%    \begin{macrocode}
\DeclareOption*{%
  \PassoptionsToPackage{\CurrentOption}{parcolumns}%
}
\ProcessOptions\relax
\RequirePackage{parcolumns}[2004/11/25]
%    \end{macrocode}
%
% \subsubsection{Package \xpackage{pdfcol}}
%
%    \begin{macrocode}
\RequirePackage{pdfcol}[2007/09/09]
\ifpdfcolAvailable
\else
  \PackageInfo{pdfcolparcolumns}{%
    Loading aborted, because color stacks are not available%
  }%
  \expandafter\endinput
\fi
%    \end{macrocode}
%
% \subsubsection{Package \xpackage{infwarerr}}
%
%    \begin{macrocode}
\RequirePackage{infwarerr}[2007/09/09]
%    \end{macrocode}
%
% \subsection{Color stack macros}
%
%    \begin{macro}{\pcpc@MaxStack}
%    Macro \cs{pcpc@MaxStack} holds the highest number of
%    allocated stacks.
%    \begin{macrocode}
\global\chardef\pcpc@MaxStack=\z@
%    \end{macrocode}
%    \end{macro}
%    \begin{macro}{\pcpc@InitStacks}
%    Macro \cs{pcpc@InitStacks} takes the number of columns
%    as argument and ensures that there are enough color
%    stacks for all columns.
%    \begin{macrocode}
\def\pcpc@InitStacks#1{%
  \ifnum#1>\pcpc@MaxStack
    \begingroup
      \count@\pcpc@MaxStack
      \loop
        \advance\count@\@ne
        \pdfcolInitStack{pcpc@\the\count@}%
      \ifnum#1>\count@
      \repeat
      \global\chardef\pcpc@MaxStack=\count@
    \endgroup
  \fi
}
%    \end{macrocode}
%    \end{macro}
%
%    \begin{macro}{\pcpc@SwitchStack}
%    \begin{macrocode}
\def\pcpc@SwitchStack#1{%
  \pdfcolSwitchStack{pcpc@\number#1}%
}
%    \end{macrocode}
%    \end{macro}
%
%    \begin{macro}{\pcpc@SetCurrent}
%    \begin{macrocode}
\def\pcpc@SetCurrent#1{%
  \pdfcolSetCurrent{pcpc@\number#1}%
}
%    \end{macrocode}
%    \end{macro}
%
% \subsection{Patches}
%
%     Now the color stack macros are patched into the macros
%     of package \xpackage{parcolumns}.
%
% \subsubsection{Init stacks}
%
%    \cs{pcpc@InitStacks} should go into the definition of
%    environment |parcolumns|. \cs{pc@alloccolumns} is executed
%    there and nowhere else, thus we hook into it.
%    \begin{macrocode}
\g@addto@macro\pc@alloccolumns{%
  \pcpc@InitStacks\pc@columncount
}
%    \end{macrocode}
%
% \subsubsection{Switch stack}
%
%    \cs{pcpc@SwitchStack} should be called by marco \cs{colchunk@}.
%    However it is easier to patch \cs{pc@setcolumnwidth} that
%    is executed in \cs{colchunk@} only.
%    \begin{macrocode}
\g@addto@macro\pc@setcolumnwidth{%
  \pcpc@SwitchStack\pc@columnctr
}
%    \end{macrocode}
%
% \subsubsection{Set current stack color}
%
%    \cs{pcpc@SetCurrent} is set at the begin of each line.
%    It must be inserted into \cs{pc@placeboxes}. Unhappily
%    there is no easy way. Therefore we check and
%    redefine \cs{pc@placeboxes}.
%    \begin{macrocode}
\begingroup
  \def\x{%
    \global\let\@tempa\relax
    \count@\z@
    \hb@xt@\linewidth{%
      \vfuzz30ex %
      \vbadness\@M
      \splittopskip\z@skip
      \loop
      \ifnum\count@<\pc@columncount
        \advance\count@\@ne
        \expandafter\ifvoid\csname pc@column@\number\count@\endcsname
          \hskip\csname pc@column@width@\number\count@\endcsname
        \else
          \expandafter\setbox\expandafter\@tempboxa\expandafter
          \vsplit\csname pc@column@\number\count@\endcsname
              to \dp\strutbox
          \vbox{%
            \unvbox\@tempboxa
          }%
        \fi
        \expandafter\ifvoid\csname pc@column@\number\count@\endcsname
        \else
          \global\let\@tempa\pc@placeboxes
        \fi
        \ifnum\count@<\pc@columncount
          \strut
          \hfill
          \ifpc@rulebetween
            \vrule
            \hfill
          \fi
        \fi
      \repeat
    }%
    \@tempa
  }%
  \ifx\x\pc@placeboxes
  \else
    \@PackageWarningNoLine{pdfcolparcolumns}{%
      Command \string\pc@placeboxes\space has changed.\MessageBreak
      Supported versions of package `parcolumns':\MessageBreak
      \space\space 2004/08/05.\MessageBreak
      The redefinition of \string\pc@placeboxes\space may not%
      \MessageBreak
      behave correctly depending on the changes%
    }%
  \fi
\endgroup
%    \end{macrocode}
%    \begin{macro}{\pc@placeboxes}
%    \begin{macrocode}
\renewcommand*{\pc@placeboxes}{%
  \global\let\@tempa\relax
  \count@\z@
  \hb@xt@\linewidth{%
    \vfuzz30ex %
    \vbadness\@M
    \splittopskip\z@skip
    \loop
    \ifnum\count@<\pc@columncount
      \advance\count@\@ne
      \expandafter\ifvoid\csname pc@column@\number\count@\endcsname
        \hskip\csname pc@column@width@\number\count@\endcsname
      \else
        \expandafter\setbox\expandafter\@tempboxa\expandafter
        \vsplit\csname pc@column@\number\count@\endcsname
            to \dp\strutbox
        \vbox{%
          \pcpc@SetCurrent\count@
          \unvbox\@tempboxa
        }%
      \fi
      \expandafter\ifvoid\csname pc@column@\number\count@\endcsname
      \else
        \global\let\@tempa\pc@placeboxes
      \fi
      \ifnum\count@<\pc@columncount
        \strut
        \hfill
        \ifpc@rulebetween
          \begingroup
            \pcpc@RuleBetweenColor
            \vrule
          \endgroup
          \hfill
        \fi
      \fi
    \repeat
  }%
  \@tempa
}
%    \end{macrocode}
%    \end{macro}
%    \begin{macro}{\pcpc@RuleBetweenColorDefault}
%    \begin{macrocode}
\def\pcpc@RuleBetweenColorDefault{%
  \normalcolor
}
%    \end{macrocode}
%    \end{macro}
%    \begin{macro}{\pcpc@RuleBetweenColor}
%    \begin{macrocode}
\let\pcpc@RuleBetweenColor\pcpc@RuleBetweenColorDefault
%    \end{macrocode}
%    \end{macro}
%    \begin{macrocode}
\define@key{parcolumns}{rulebetweencolor}{%
  \edef\pcpc@temp{#1}%
  \ifx\pcpc@temp\@empty
    \let\pcpc@RuleBetweenColor\pcpc@RuleBetweenColorDefault
  \else
    \edef\pcpc@temp{%
      \noexpand\@ifnextchar[{%
        \def\noexpand\pcpc@RuleBetweenColor{%
          \noexpand\color\pcpc@temp
        }%
        \noexpand\pcpc@GobbleNil
      }{%
        \def\noexpand\pcpc@RuleBetweenColor{%
          \noexpand\color{\pcpc@temp}%
        }%
        \noexpand\pcpc@GobbleNil
      }%
      \pcpc@temp\noexpand\@nil
    }%
    \pcpc@temp
  \fi
}
%    \end{macrocode}
%    \begin{macro}{\pcpc@GobbleNil}
%    \begin{macrocode}
\long\def\pcpc@GobbleNil#1\@nil{}
%    \end{macrocode}
%    \end{macro}
%
%    \begin{macrocode}
%</package>
%    \end{macrocode}
%
% \section{Test}
%
%    The test file is a modified version of the file that
%    Donald Goodman has posted in \xnewsgroup{comp.text.tex}: ^^A
%    \URL{``\link{Re: \xpackage{xcolor} glitches}''}^^A
%    {http://groups.google.com/group/comp.text.tex/msg/8eda74ed292012bb}
%    \begin{macrocode}
%<*test1>
\NeedsTeXFormat{LaTeX2e}
\AtEndDocument{%
  \typeout{}%
  \typeout{**************************************}%
  \typeout{*** \space Check the PDF file manually! \space ***}%
  \typeout{**************************************}%
  \typeout{}%
}
\documentclass{article}
\usepackage{xcolor}
\usepackage{pdfcolparcolumns}

\newcommand{\instruct}[1]{%
  \noindent
  \footnotesize
  \textcolor{red}{#1}%
}

\begin{document}
  \begin{parcolumns}[colwidths={1=2.3in,2=2.3in},sloppy]{2}%
    \colchunk[1]{%
      \instruct{Et non dicitur versus} %
      Fidelium anim\ae\ %
      \instruct{%
        sed immediate subiungitur antiphona finalis %
        beat\ae\ Mari\ae\ Virginis%
      } %
      100.%
    }%
    \colchunk[2]{%
      \instruct{%
        And the verse %
        \textcolor{black}{May the souls of the faithful} %
        is not said, but the final antiphon of the %
        Blessed Virgin Mary, %
        \textcolor{black}{100,} %
        is immediately joined.%
      }%
    }%
  \end{parcolumns}%
\end{document}
%</test1>
%    \end{macrocode}
%
% \section{Installation}
%
% \subsection{Download}
%
% \paragraph{Package.} This package is available on
% CTAN\footnote{\url{http://ctan.org/pkg/pdfcolparcolumns}}:
% \begin{description}
% \item[\CTAN{macros/latex/contrib/oberdiek/pdfcolparcolumns.dtx}] The source file.
% \item[\CTAN{macros/latex/contrib/oberdiek/pdfcolparcolumns.pdf}] Documentation.
% \end{description}
%
%
% \paragraph{Bundle.} All the packages of the bundle `oberdiek'
% are also available in a TDS compliant ZIP archive. There
% the packages are already unpacked and the documentation files
% are generated. The files and directories obey the TDS standard.
% \begin{description}
% \item[\CTAN{install/macros/latex/contrib/oberdiek.tds.zip}]
% \end{description}
% \emph{TDS} refers to the standard ``A Directory Structure
% for \TeX\ Files'' (\CTAN{tds/tds.pdf}). Directories
% with \xfile{texmf} in their name are usually organized this way.
%
% \subsection{Bundle installation}
%
% \paragraph{Unpacking.} Unpack the \xfile{oberdiek.tds.zip} in the
% TDS tree (also known as \xfile{texmf} tree) of your choice.
% Example (linux):
% \begin{quote}
%   |unzip oberdiek.tds.zip -d ~/texmf|
% \end{quote}
%
% \paragraph{Script installation.}
% Check the directory \xfile{TDS:scripts/oberdiek/} for
% scripts that need further installation steps.
% Package \xpackage{attachfile2} comes with the Perl script
% \xfile{pdfatfi.pl} that should be installed in such a way
% that it can be called as \texttt{pdfatfi}.
% Example (linux):
% \begin{quote}
%   |chmod +x scripts/oberdiek/pdfatfi.pl|\\
%   |cp scripts/oberdiek/pdfatfi.pl /usr/local/bin/|
% \end{quote}
%
% \subsection{Package installation}
%
% \paragraph{Unpacking.} The \xfile{.dtx} file is a self-extracting
% \docstrip\ archive. The files are extracted by running the
% \xfile{.dtx} through \plainTeX:
% \begin{quote}
%   \verb|tex pdfcolparcolumns.dtx|
% \end{quote}
%
% \paragraph{TDS.} Now the different files must be moved into
% the different directories in your installation TDS tree
% (also known as \xfile{texmf} tree):
% \begin{quote}
% \def\t{^^A
% \begin{tabular}{@{}>{\ttfamily}l@{ $\rightarrow$ }>{\ttfamily}l@{}}
%   pdfcolparcolumns.sty & tex/latex/oberdiek/pdfcolparcolumns.sty\\
%   pdfcolparcolumns.pdf & doc/latex/oberdiek/pdfcolparcolumns.pdf\\
%   test/pdfcolparcolumns-test1.tex & doc/latex/oberdiek/test/pdfcolparcolumns-test1.tex\\
%   pdfcolparcolumns.dtx & source/latex/oberdiek/pdfcolparcolumns.dtx\\
% \end{tabular}^^A
% }^^A
% \sbox0{\t}^^A
% \ifdim\wd0>\linewidth
%   \begingroup
%     \advance\linewidth by\leftmargin
%     \advance\linewidth by\rightmargin
%   \edef\x{\endgroup
%     \def\noexpand\lw{\the\linewidth}^^A
%   }\x
%   \def\lwbox{^^A
%     \leavevmode
%     \hbox to \linewidth{^^A
%       \kern-\leftmargin\relax
%       \hss
%       \usebox0
%       \hss
%       \kern-\rightmargin\relax
%     }^^A
%   }^^A
%   \ifdim\wd0>\lw
%     \sbox0{\small\t}^^A
%     \ifdim\wd0>\linewidth
%       \ifdim\wd0>\lw
%         \sbox0{\footnotesize\t}^^A
%         \ifdim\wd0>\linewidth
%           \ifdim\wd0>\lw
%             \sbox0{\scriptsize\t}^^A
%             \ifdim\wd0>\linewidth
%               \ifdim\wd0>\lw
%                 \sbox0{\tiny\t}^^A
%                 \ifdim\wd0>\linewidth
%                   \lwbox
%                 \else
%                   \usebox0
%                 \fi
%               \else
%                 \lwbox
%               \fi
%             \else
%               \usebox0
%             \fi
%           \else
%             \lwbox
%           \fi
%         \else
%           \usebox0
%         \fi
%       \else
%         \lwbox
%       \fi
%     \else
%       \usebox0
%     \fi
%   \else
%     \lwbox
%   \fi
% \else
%   \usebox0
% \fi
% \end{quote}
% If you have a \xfile{docstrip.cfg} that configures and enables \docstrip's
% TDS installing feature, then some files can already be in the right
% place, see the documentation of \docstrip.
%
% \subsection{Refresh file name databases}
%
% If your \TeX~distribution
% (\teTeX, \mikTeX, \dots) relies on file name databases, you must refresh
% these. For example, \teTeX\ users run \verb|texhash| or
% \verb|mktexlsr|.
%
% \subsection{Some details for the interested}
%
% \paragraph{Attached source.}
%
% The PDF documentation on CTAN also includes the
% \xfile{.dtx} source file. It can be extracted by
% AcrobatReader 6 or higher. Another option is \textsf{pdftk},
% e.g. unpack the file into the current directory:
% \begin{quote}
%   \verb|pdftk pdfcolparcolumns.pdf unpack_files output .|
% \end{quote}
%
% \paragraph{Unpacking with \LaTeX.}
% The \xfile{.dtx} chooses its action depending on the format:
% \begin{description}
% \item[\plainTeX:] Run \docstrip\ and extract the files.
% \item[\LaTeX:] Generate the documentation.
% \end{description}
% If you insist on using \LaTeX\ for \docstrip\ (really,
% \docstrip\ does not need \LaTeX), then inform the autodetect routine
% about your intention:
% \begin{quote}
%   \verb|latex \let\install=y\input{pdfcolparcolumns.dtx}|
% \end{quote}
% Do not forget to quote the argument according to the demands
% of your shell.
%
% \paragraph{Generating the documentation.}
% You can use both the \xfile{.dtx} or the \xfile{.drv} to generate
% the documentation. The process can be configured by the
% configuration file \xfile{ltxdoc.cfg}. For instance, put this
% line into this file, if you want to have A4 as paper format:
% \begin{quote}
%   \verb|\PassOptionsToClass{a4paper}{article}|
% \end{quote}
% An example follows how to generate the
% documentation with pdf\LaTeX:
% \begin{quote}
%\begin{verbatim}
%pdflatex pdfcolparcolumns.dtx
%makeindex -s gind.ist pdfcolparcolumns.idx
%pdflatex pdfcolparcolumns.dtx
%makeindex -s gind.ist pdfcolparcolumns.idx
%pdflatex pdfcolparcolumns.dtx
%\end{verbatim}
% \end{quote}
%
% \section{Catalogue}
%
% The following XML file can be used as source for the
% \href{http://mirror.ctan.org/help/Catalogue/catalogue.html}{\TeX\ Catalogue}.
% The elements \texttt{caption} and \texttt{description} are imported
% from the original XML file from the Catalogue.
% The name of the XML file in the Catalogue is \xfile{pdfcolparcolumns.xml}.
%    \begin{macrocode}
%<*catalogue>
<?xml version='1.0' encoding='us-ascii'?>
<!DOCTYPE entry SYSTEM 'catalogue.dtd'>
<entry datestamp='$Date$' modifier='$Author$' id='pdfcolparcolumns'>
  <name>pdfcolparcolumns</name>
  <caption>Fix colour problems in package 'parcolumns'.</caption>
  <authorref id='auth:oberdiek'/>
  <copyright owner='Heiko Oberdiek' year='2007,2008,2010'/>
  <license type='lppl1.3'/>
  <version number='1.4'/>
  <description>
    Since version 1.40 pdfTeX supports colour stacks.
    This package uses them to fix colour problems in
    package <xref refid='parcolumns'>parcolumns</xref>.
    <p/>
    The package is part of the <xref refid='oberdiek'>oberdiek</xref>
    bundle.
  </description>
  <documentation details='Package documentation'
      href='ctan:/macros/latex/contrib/oberdiek/pdfcolparcolumns.pdf'/>
  <ctan file='true' path='/macros/latex/contrib/oberdiek/pdfcolparcolumns.dtx'/>
  <miktex location='oberdiek'/>
  <texlive location='oberdiek'/>
  <install path='/macros/latex/contrib/oberdiek/oberdiek.tds.zip'/>
</entry>
%</catalogue>
%    \end{macrocode}
%
% \begin{thebibliography}{9}
%
% \bibitem{parcolumns}
%   Jonathan Sauer: \textit{The \xpackage{parcolumns} package};
%   2004/11/25;\\
%   \CTAN{macros/latex/contrib/sauerj/parcolumns.pdf}.
%
% \bibitem{pdfcol}
%   Heiko Oberdiek: \textit{The \xpackage{pdfcol} package};
%   2007/09/09;\\
%   \CTAN{macros/latex/contrib/oberdiek/pdfcol.pdf}.
%
% \end{thebibliography}
%
% \begin{History}
%   \begin{Version}{2007/07/26 v1.0}
%   \item
%     First version, published in the newsgroup \xnewsgroup{comp.text.tex}
%     with the name \xpackage{parcolumns-colorstacks}: ^^A no line break
%     \URL{``\link{Re: \xpackage{xcolor} glitches}''}^^A
%     {http://groups.google.com/group/comp.text.tex/msg/56bd897b11bca414}
%   \end{Version}
%   \begin{Version}{2007/09/09 v1.1}
%   \item
%     CTAN version, package name renamed to \xpackage{pdfcolparcolumns}.
%   \item
%     Uses package \xpackage{pdfcol}.
%   \item
%     Documentation added.
%   \item
%     Test file added.
%   \end{Version}
%   \begin{Version}{2008/08/11 v1.2}
%   \item
%     Code is not changed.
%   \item
%     URLs updated.
%   \end{Version}
%   \begin{Version}{2010/01/11 v1.3}
%   \item
%     Fix for rule color.
%   \item
%     New option \xoption{rulebetweencolor} for environment |parcolumns|.
%   \end{Version}
%   \begin{Version}{2016/05/16 v1.4}
%   \item
%     Documentation updates.
%   \end{Version}
% \end{History}
%
% \PrintIndex
%
% \Finale
\endinput
|
% \end{quote}
% Do not forget to quote the argument according to the demands
% of your shell.
%
% \paragraph{Generating the documentation.}
% You can use both the \xfile{.dtx} or the \xfile{.drv} to generate
% the documentation. The process can be configured by the
% configuration file \xfile{ltxdoc.cfg}. For instance, put this
% line into this file, if you want to have A4 as paper format:
% \begin{quote}
%   \verb|\PassOptionsToClass{a4paper}{article}|
% \end{quote}
% An example follows how to generate the
% documentation with pdf\LaTeX:
% \begin{quote}
%\begin{verbatim}
%pdflatex pdfcolparcolumns.dtx
%makeindex -s gind.ist pdfcolparcolumns.idx
%pdflatex pdfcolparcolumns.dtx
%makeindex -s gind.ist pdfcolparcolumns.idx
%pdflatex pdfcolparcolumns.dtx
%\end{verbatim}
% \end{quote}
%
% \section{Catalogue}
%
% The following XML file can be used as source for the
% \href{http://mirror.ctan.org/help/Catalogue/catalogue.html}{\TeX\ Catalogue}.
% The elements \texttt{caption} and \texttt{description} are imported
% from the original XML file from the Catalogue.
% The name of the XML file in the Catalogue is \xfile{pdfcolparcolumns.xml}.
%    \begin{macrocode}
%<*catalogue>
<?xml version='1.0' encoding='us-ascii'?>
<!DOCTYPE entry SYSTEM 'catalogue.dtd'>
<entry datestamp='$Date$' modifier='$Author$' id='pdfcolparcolumns'>
  <name>pdfcolparcolumns</name>
  <caption>Fix colour problems in package 'parcolumns'.</caption>
  <authorref id='auth:oberdiek'/>
  <copyright owner='Heiko Oberdiek' year='2007,2008,2010'/>
  <license type='lppl1.3'/>
  <version number='1.4'/>
  <description>
    Since version 1.40 pdfTeX supports colour stacks.
    This package uses them to fix colour problems in
    package <xref refid='parcolumns'>parcolumns</xref>.
    <p/>
    The package is part of the <xref refid='oberdiek'>oberdiek</xref>
    bundle.
  </description>
  <documentation details='Package documentation'
      href='ctan:/macros/latex/contrib/oberdiek/pdfcolparcolumns.pdf'/>
  <ctan file='true' path='/macros/latex/contrib/oberdiek/pdfcolparcolumns.dtx'/>
  <miktex location='oberdiek'/>
  <texlive location='oberdiek'/>
  <install path='/macros/latex/contrib/oberdiek/oberdiek.tds.zip'/>
</entry>
%</catalogue>
%    \end{macrocode}
%
% \begin{thebibliography}{9}
%
% \bibitem{parcolumns}
%   Jonathan Sauer: \textit{The \xpackage{parcolumns} package};
%   2004/11/25;\\
%   \CTAN{macros/latex/contrib/sauerj/parcolumns.pdf}.
%
% \bibitem{pdfcol}
%   Heiko Oberdiek: \textit{The \xpackage{pdfcol} package};
%   2007/09/09;\\
%   \CTAN{macros/latex/contrib/oberdiek/pdfcol.pdf}.
%
% \end{thebibliography}
%
% \begin{History}
%   \begin{Version}{2007/07/26 v1.0}
%   \item
%     First version, published in the newsgroup \xnewsgroup{comp.text.tex}
%     with the name \xpackage{parcolumns-colorstacks}: ^^A no line break
%     \URL{``\link{Re: \xpackage{xcolor} glitches}''}^^A
%     {http://groups.google.com/group/comp.text.tex/msg/56bd897b11bca414}
%   \end{Version}
%   \begin{Version}{2007/09/09 v1.1}
%   \item
%     CTAN version, package name renamed to \xpackage{pdfcolparcolumns}.
%   \item
%     Uses package \xpackage{pdfcol}.
%   \item
%     Documentation added.
%   \item
%     Test file added.
%   \end{Version}
%   \begin{Version}{2008/08/11 v1.2}
%   \item
%     Code is not changed.
%   \item
%     URLs updated.
%   \end{Version}
%   \begin{Version}{2010/01/11 v1.3}
%   \item
%     Fix for rule color.
%   \item
%     New option \xoption{rulebetweencolor} for environment |parcolumns|.
%   \end{Version}
%   \begin{Version}{2016/05/16 v1.4}
%   \item
%     Documentation updates.
%   \end{Version}
% \end{History}
%
% \PrintIndex
%
% \Finale
\endinput
|
% \end{quote}
% Do not forget to quote the argument according to the demands
% of your shell.
%
% \paragraph{Generating the documentation.}
% You can use both the \xfile{.dtx} or the \xfile{.drv} to generate
% the documentation. The process can be configured by the
% configuration file \xfile{ltxdoc.cfg}. For instance, put this
% line into this file, if you want to have A4 as paper format:
% \begin{quote}
%   \verb|\PassOptionsToClass{a4paper}{article}|
% \end{quote}
% An example follows how to generate the
% documentation with pdf\LaTeX:
% \begin{quote}
%\begin{verbatim}
%pdflatex pdfcolparcolumns.dtx
%makeindex -s gind.ist pdfcolparcolumns.idx
%pdflatex pdfcolparcolumns.dtx
%makeindex -s gind.ist pdfcolparcolumns.idx
%pdflatex pdfcolparcolumns.dtx
%\end{verbatim}
% \end{quote}
%
% \section{Catalogue}
%
% The following XML file can be used as source for the
% \href{http://mirror.ctan.org/help/Catalogue/catalogue.html}{\TeX\ Catalogue}.
% The elements \texttt{caption} and \texttt{description} are imported
% from the original XML file from the Catalogue.
% The name of the XML file in the Catalogue is \xfile{pdfcolparcolumns.xml}.
%    \begin{macrocode}
%<*catalogue>
<?xml version='1.0' encoding='us-ascii'?>
<!DOCTYPE entry SYSTEM 'catalogue.dtd'>
<entry datestamp='$Date$' modifier='$Author$' id='pdfcolparcolumns'>
  <name>pdfcolparcolumns</name>
  <caption>Fix colour problems in package 'parcolumns'.</caption>
  <authorref id='auth:oberdiek'/>
  <copyright owner='Heiko Oberdiek' year='2007,2008,2010'/>
  <license type='lppl1.3'/>
  <version number='1.4'/>
  <description>
    Since version 1.40 pdfTeX supports colour stacks.
    This package uses them to fix colour problems in
    package <xref refid='parcolumns'>parcolumns</xref>.
    <p/>
    The package is part of the <xref refid='oberdiek'>oberdiek</xref>
    bundle.
  </description>
  <documentation details='Package documentation'
      href='ctan:/macros/latex/contrib/oberdiek/pdfcolparcolumns.pdf'/>
  <ctan file='true' path='/macros/latex/contrib/oberdiek/pdfcolparcolumns.dtx'/>
  <miktex location='oberdiek'/>
  <texlive location='oberdiek'/>
  <install path='/macros/latex/contrib/oberdiek/oberdiek.tds.zip'/>
</entry>
%</catalogue>
%    \end{macrocode}
%
% \begin{thebibliography}{9}
%
% \bibitem{parcolumns}
%   Jonathan Sauer: \textit{The \xpackage{parcolumns} package};
%   2004/11/25;\\
%   \CTAN{macros/latex/contrib/sauerj/parcolumns.pdf}.
%
% \bibitem{pdfcol}
%   Heiko Oberdiek: \textit{The \xpackage{pdfcol} package};
%   2007/09/09;\\
%   \CTAN{macros/latex/contrib/oberdiek/pdfcol.pdf}.
%
% \end{thebibliography}
%
% \begin{History}
%   \begin{Version}{2007/07/26 v1.0}
%   \item
%     First version, published in the newsgroup \xnewsgroup{comp.text.tex}
%     with the name \xpackage{parcolumns-colorstacks}: ^^A no line break
%     \URL{``\link{Re: \xpackage{xcolor} glitches}''}^^A
%     {http://groups.google.com/group/comp.text.tex/msg/56bd897b11bca414}
%   \end{Version}
%   \begin{Version}{2007/09/09 v1.1}
%   \item
%     CTAN version, package name renamed to \xpackage{pdfcolparcolumns}.
%   \item
%     Uses package \xpackage{pdfcol}.
%   \item
%     Documentation added.
%   \item
%     Test file added.
%   \end{Version}
%   \begin{Version}{2008/08/11 v1.2}
%   \item
%     Code is not changed.
%   \item
%     URLs updated.
%   \end{Version}
%   \begin{Version}{2010/01/11 v1.3}
%   \item
%     Fix for rule color.
%   \item
%     New option \xoption{rulebetweencolor} for environment |parcolumns|.
%   \end{Version}
%   \begin{Version}{2016/05/16 v1.4}
%   \item
%     Documentation updates.
%   \end{Version}
% \end{History}
%
% \PrintIndex
%
% \Finale
\endinput
|
% \end{quote}
% Do not forget to quote the argument according to the demands
% of your shell.
%
% \paragraph{Generating the documentation.}
% You can use both the \xfile{.dtx} or the \xfile{.drv} to generate
% the documentation. The process can be configured by the
% configuration file \xfile{ltxdoc.cfg}. For instance, put this
% line into this file, if you want to have A4 as paper format:
% \begin{quote}
%   \verb|\PassOptionsToClass{a4paper}{article}|
% \end{quote}
% An example follows how to generate the
% documentation with pdf\LaTeX:
% \begin{quote}
%\begin{verbatim}
%pdflatex pdfcolparcolumns.dtx
%makeindex -s gind.ist pdfcolparcolumns.idx
%pdflatex pdfcolparcolumns.dtx
%makeindex -s gind.ist pdfcolparcolumns.idx
%pdflatex pdfcolparcolumns.dtx
%\end{verbatim}
% \end{quote}
%
% \section{Catalogue}
%
% The following XML file can be used as source for the
% \href{http://mirror.ctan.org/help/Catalogue/catalogue.html}{\TeX\ Catalogue}.
% The elements \texttt{caption} and \texttt{description} are imported
% from the original XML file from the Catalogue.
% The name of the XML file in the Catalogue is \xfile{pdfcolparcolumns.xml}.
%    \begin{macrocode}
%<*catalogue>
<?xml version='1.0' encoding='us-ascii'?>
<!DOCTYPE entry SYSTEM 'catalogue.dtd'>
<entry datestamp='$Date$' modifier='$Author$' id='pdfcolparcolumns'>
  <name>pdfcolparcolumns</name>
  <caption>Fix colour problems in package 'parcolumns'.</caption>
  <authorref id='auth:oberdiek'/>
  <copyright owner='Heiko Oberdiek' year='2007,2008,2010'/>
  <license type='lppl1.3'/>
  <version number='1.4'/>
  <description>
    Since version 1.40 pdfTeX supports colour stacks.
    This package uses them to fix colour problems in
    package <xref refid='parcolumns'>parcolumns</xref>.
    <p/>
    The package is part of the <xref refid='oberdiek'>oberdiek</xref>
    bundle.
  </description>
  <documentation details='Package documentation'
      href='ctan:/macros/latex/contrib/oberdiek/pdfcolparcolumns.pdf'/>
  <ctan file='true' path='/macros/latex/contrib/oberdiek/pdfcolparcolumns.dtx'/>
  <miktex location='oberdiek'/>
  <texlive location='oberdiek'/>
  <install path='/macros/latex/contrib/oberdiek/oberdiek.tds.zip'/>
</entry>
%</catalogue>
%    \end{macrocode}
%
% \begin{thebibliography}{9}
%
% \bibitem{parcolumns}
%   Jonathan Sauer: \textit{The \xpackage{parcolumns} package};
%   2004/11/25;\\
%   \CTAN{macros/latex/contrib/sauerj/parcolumns.pdf}.
%
% \bibitem{pdfcol}
%   Heiko Oberdiek: \textit{The \xpackage{pdfcol} package};
%   2007/09/09;\\
%   \CTAN{macros/latex/contrib/oberdiek/pdfcol.pdf}.
%
% \end{thebibliography}
%
% \begin{History}
%   \begin{Version}{2007/07/26 v1.0}
%   \item
%     First version, published in the newsgroup \xnewsgroup{comp.text.tex}
%     with the name \xpackage{parcolumns-colorstacks}: ^^A no line break
%     \URL{``\link{Re: \xpackage{xcolor} glitches}''}^^A
%     {http://groups.google.com/group/comp.text.tex/msg/56bd897b11bca414}
%   \end{Version}
%   \begin{Version}{2007/09/09 v1.1}
%   \item
%     CTAN version, package name renamed to \xpackage{pdfcolparcolumns}.
%   \item
%     Uses package \xpackage{pdfcol}.
%   \item
%     Documentation added.
%   \item
%     Test file added.
%   \end{Version}
%   \begin{Version}{2008/08/11 v1.2}
%   \item
%     Code is not changed.
%   \item
%     URLs updated.
%   \end{Version}
%   \begin{Version}{2010/01/11 v1.3}
%   \item
%     Fix for rule color.
%   \item
%     New option \xoption{rulebetweencolor} for environment |parcolumns|.
%   \end{Version}
%   \begin{Version}{2016/05/16 v1.4}
%   \item
%     Documentation updates.
%   \end{Version}
% \end{History}
%
% \PrintIndex
%
% \Finale
\endinput
