% \iffalse meta-comment
%
% File: pdfcolparallel.dtx
% Version: 2016/05/16 v1.4
% Info: Color stacks support for parallel
%
% Copyright (C) 2007, 2008, 2010 by
%    Heiko Oberdiek <heiko.oberdiek at googlemail.com>
%    2016
%    https://github.com/ho-tex/oberdiek/issues
%
% This work may be distributed and/or modified under the
% conditions of the LaTeX Project Public License, either
% version 1.3c of this license or (at your option) any later
% version. This version of this license is in
%    http://www.latex-project.org/lppl/lppl-1-3c.txt
% and the latest version of this license is in
%    http://www.latex-project.org/lppl.txt
% and version 1.3 or later is part of all distributions of
% LaTeX version 2005/12/01 or later.
%
% This work has the LPPL maintenance status "maintained".
%
% This Current Maintainer of this work is Heiko Oberdiek.
%
% This work consists of the main source file pdfcolparallel.dtx
% and the derived files
%    pdfcolparallel.sty, pdfcolparallel.pdf, pdfcolparallel.ins,
%    pdfcolparallel.drv, pdfcolparallel-test1.tex.
%
% Distribution:
%    CTAN:macros/latex/contrib/oberdiek/pdfcolparallel.dtx
%    CTAN:macros/latex/contrib/oberdiek/pdfcolparallel.pdf
%
% Unpacking:
%    (a) If pdfcolparallel.ins is present:
%           tex pdfcolparallel.ins
%    (b) Without pdfcolparallel.ins:
%           tex pdfcolparallel.dtx
%    (c) If you insist on using LaTeX
%           latex \let\install=y% \iffalse meta-comment
%
% File: pdfcolparallel.dtx
% Version: 2016/05/16 v1.4
% Info: Color stacks support for parallel
%
% Copyright (C) 2007, 2008, 2010 by
%    Heiko Oberdiek <heiko.oberdiek at googlemail.com>
%    2016
%    https://github.com/ho-tex/oberdiek/issues
%
% This work may be distributed and/or modified under the
% conditions of the LaTeX Project Public License, either
% version 1.3c of this license or (at your option) any later
% version. This version of this license is in
%    http://www.latex-project.org/lppl/lppl-1-3c.txt
% and the latest version of this license is in
%    http://www.latex-project.org/lppl.txt
% and version 1.3 or later is part of all distributions of
% LaTeX version 2005/12/01 or later.
%
% This work has the LPPL maintenance status "maintained".
%
% This Current Maintainer of this work is Heiko Oberdiek.
%
% This work consists of the main source file pdfcolparallel.dtx
% and the derived files
%    pdfcolparallel.sty, pdfcolparallel.pdf, pdfcolparallel.ins,
%    pdfcolparallel.drv, pdfcolparallel-test1.tex.
%
% Distribution:
%    CTAN:macros/latex/contrib/oberdiek/pdfcolparallel.dtx
%    CTAN:macros/latex/contrib/oberdiek/pdfcolparallel.pdf
%
% Unpacking:
%    (a) If pdfcolparallel.ins is present:
%           tex pdfcolparallel.ins
%    (b) Without pdfcolparallel.ins:
%           tex pdfcolparallel.dtx
%    (c) If you insist on using LaTeX
%           latex \let\install=y% \iffalse meta-comment
%
% File: pdfcolparallel.dtx
% Version: 2016/05/16 v1.4
% Info: Color stacks support for parallel
%
% Copyright (C) 2007, 2008, 2010 by
%    Heiko Oberdiek <heiko.oberdiek at googlemail.com>
%    2016
%    https://github.com/ho-tex/oberdiek/issues
%
% This work may be distributed and/or modified under the
% conditions of the LaTeX Project Public License, either
% version 1.3c of this license or (at your option) any later
% version. This version of this license is in
%    http://www.latex-project.org/lppl/lppl-1-3c.txt
% and the latest version of this license is in
%    http://www.latex-project.org/lppl.txt
% and version 1.3 or later is part of all distributions of
% LaTeX version 2005/12/01 or later.
%
% This work has the LPPL maintenance status "maintained".
%
% This Current Maintainer of this work is Heiko Oberdiek.
%
% This work consists of the main source file pdfcolparallel.dtx
% and the derived files
%    pdfcolparallel.sty, pdfcolparallel.pdf, pdfcolparallel.ins,
%    pdfcolparallel.drv, pdfcolparallel-test1.tex.
%
% Distribution:
%    CTAN:macros/latex/contrib/oberdiek/pdfcolparallel.dtx
%    CTAN:macros/latex/contrib/oberdiek/pdfcolparallel.pdf
%
% Unpacking:
%    (a) If pdfcolparallel.ins is present:
%           tex pdfcolparallel.ins
%    (b) Without pdfcolparallel.ins:
%           tex pdfcolparallel.dtx
%    (c) If you insist on using LaTeX
%           latex \let\install=y% \iffalse meta-comment
%
% File: pdfcolparallel.dtx
% Version: 2016/05/16 v1.4
% Info: Color stacks support for parallel
%
% Copyright (C) 2007, 2008, 2010 by
%    Heiko Oberdiek <heiko.oberdiek at googlemail.com>
%    2016
%    https://github.com/ho-tex/oberdiek/issues
%
% This work may be distributed and/or modified under the
% conditions of the LaTeX Project Public License, either
% version 1.3c of this license or (at your option) any later
% version. This version of this license is in
%    http://www.latex-project.org/lppl/lppl-1-3c.txt
% and the latest version of this license is in
%    http://www.latex-project.org/lppl.txt
% and version 1.3 or later is part of all distributions of
% LaTeX version 2005/12/01 or later.
%
% This work has the LPPL maintenance status "maintained".
%
% This Current Maintainer of this work is Heiko Oberdiek.
%
% This work consists of the main source file pdfcolparallel.dtx
% and the derived files
%    pdfcolparallel.sty, pdfcolparallel.pdf, pdfcolparallel.ins,
%    pdfcolparallel.drv, pdfcolparallel-test1.tex.
%
% Distribution:
%    CTAN:macros/latex/contrib/oberdiek/pdfcolparallel.dtx
%    CTAN:macros/latex/contrib/oberdiek/pdfcolparallel.pdf
%
% Unpacking:
%    (a) If pdfcolparallel.ins is present:
%           tex pdfcolparallel.ins
%    (b) Without pdfcolparallel.ins:
%           tex pdfcolparallel.dtx
%    (c) If you insist on using LaTeX
%           latex \let\install=y\input{pdfcolparallel.dtx}
%        (quote the arguments according to the demands of your shell)
%
% Documentation:
%    (a) If pdfcolparallel.drv is present:
%           latex pdfcolparallel.drv
%    (b) Without pdfcolparallel.drv:
%           latex pdfcolparallel.dtx; ...
%    The class ltxdoc loads the configuration file ltxdoc.cfg
%    if available. Here you can specify further options, e.g.
%    use A4 as paper format:
%       \PassOptionsToClass{a4paper}{article}
%
%    Programm calls to get the documentation (example):
%       pdflatex pdfcolparallel.dtx
%       makeindex -s gind.ist pdfcolparallel.idx
%       pdflatex pdfcolparallel.dtx
%       makeindex -s gind.ist pdfcolparallel.idx
%       pdflatex pdfcolparallel.dtx
%
% Installation:
%    TDS:tex/latex/oberdiek/pdfcolparallel.sty
%    TDS:doc/latex/oberdiek/pdfcolparallel.pdf
%    TDS:doc/latex/oberdiek/test/pdfcolparallel-test1.tex
%    TDS:source/latex/oberdiek/pdfcolparallel.dtx
%
%<*ignore>
\begingroup
  \catcode123=1 %
  \catcode125=2 %
  \def\x{LaTeX2e}%
\expandafter\endgroup
\ifcase 0\ifx\install y1\fi\expandafter
         \ifx\csname processbatchFile\endcsname\relax\else1\fi
         \ifx\fmtname\x\else 1\fi\relax
\else\csname fi\endcsname
%</ignore>
%<*install>
\input docstrip.tex
\Msg{************************************************************************}
\Msg{* Installation}
\Msg{* Package: pdfcolparallel 2016/05/16 v1.4 Color stacks support for parallel (HO)}
\Msg{************************************************************************}

\keepsilent
\askforoverwritefalse

\let\MetaPrefix\relax
\preamble

This is a generated file.

Project: pdfcolparallel
Version: 2016/05/16 v1.4

Copyright (C) 2007, 2008, 2010 by
   Heiko Oberdiek <heiko.oberdiek at googlemail.com>

This work may be distributed and/or modified under the
conditions of the LaTeX Project Public License, either
version 1.3c of this license or (at your option) any later
version. This version of this license is in
   http://www.latex-project.org/lppl/lppl-1-3c.txt
and the latest version of this license is in
   http://www.latex-project.org/lppl.txt
and version 1.3 or later is part of all distributions of
LaTeX version 2005/12/01 or later.

This work has the LPPL maintenance status "maintained".

This Current Maintainer of this work is Heiko Oberdiek.

This work consists of the main source file pdfcolparallel.dtx
and the derived files
   pdfcolparallel.sty, pdfcolparallel.pdf, pdfcolparallel.ins,
   pdfcolparallel.drv, pdfcolparallel-test1.tex.

\endpreamble
\let\MetaPrefix\DoubleperCent

\generate{%
  \file{pdfcolparallel.ins}{\from{pdfcolparallel.dtx}{install}}%
  \file{pdfcolparallel.drv}{\from{pdfcolparallel.dtx}{driver}}%
  \usedir{tex/latex/oberdiek}%
  \file{pdfcolparallel.sty}{\from{pdfcolparallel.dtx}{package}}%
  \usedir{doc/latex/oberdiek/test}%
  \file{pdfcolparallel-test1.tex}{\from{pdfcolparallel.dtx}{test1}}%
  \nopreamble
  \nopostamble
  \usedir{source/latex/oberdiek/catalogue}%
  \file{pdfcolparallel.xml}{\from{pdfcolparallel.dtx}{catalogue}}%
}

\catcode32=13\relax% active space
\let =\space%
\Msg{************************************************************************}
\Msg{*}
\Msg{* To finish the installation you have to move the following}
\Msg{* file into a directory searched by TeX:}
\Msg{*}
\Msg{*     pdfcolparallel.sty}
\Msg{*}
\Msg{* To produce the documentation run the file `pdfcolparallel.drv'}
\Msg{* through LaTeX.}
\Msg{*}
\Msg{* Happy TeXing!}
\Msg{*}
\Msg{************************************************************************}

\endbatchfile
%</install>
%<*ignore>
\fi
%</ignore>
%<*driver>
\NeedsTeXFormat{LaTeX2e}
\ProvidesFile{pdfcolparallel.drv}%
  [2016/05/16 v1.4 Color stacks support for parallel (HO)]%
\documentclass{ltxdoc}
\usepackage{holtxdoc}[2011/11/22]
\begin{document}
  \DocInput{pdfcolparallel.dtx}%
\end{document}
%</driver>
% \fi
%
%
% \CharacterTable
%  {Upper-case    \A\B\C\D\E\F\G\H\I\J\K\L\M\N\O\P\Q\R\S\T\U\V\W\X\Y\Z
%   Lower-case    \a\b\c\d\e\f\g\h\i\j\k\l\m\n\o\p\q\r\s\t\u\v\w\x\y\z
%   Digits        \0\1\2\3\4\5\6\7\8\9
%   Exclamation   \!     Double quote  \"     Hash (number) \#
%   Dollar        \$     Percent       \%     Ampersand     \&
%   Acute accent  \'     Left paren    \(     Right paren   \)
%   Asterisk      \*     Plus          \+     Comma         \,
%   Minus         \-     Point         \.     Solidus       \/
%   Colon         \:     Semicolon     \;     Less than     \<
%   Equals        \=     Greater than  \>     Question mark \?
%   Commercial at \@     Left bracket  \[     Backslash     \\
%   Right bracket \]     Circumflex    \^     Underscore    \_
%   Grave accent  \`     Left brace    \{     Vertical bar  \|
%   Right brace   \}     Tilde         \~}
%
% \GetFileInfo{pdfcolparallel.drv}
%
% \title{The \xpackage{pdfcolparallel} package}
% \date{2016/05/16 v1.4}
% \author{Heiko Oberdiek\thanks
% {Please report any issues at https://github.com/ho-tex/oberdiek/issues}\\
% \xemail{heiko.oberdiek at googlemail.com}}
%
% \maketitle
%
% \begin{abstract}
% This packages fixes bugs in \xpackage{parallel} and
% improves color support by using several color stacks
% that are provided by \pdfTeX\ since version 1.40.
% \end{abstract}
%
% \tableofcontents
%
% \section{Usage}
%
% \begin{quote}
% |\usepackage{pdfcolparallel}|
% \end{quote}
% The package \xpackage{pdfcolparallel} loads package \xpackage{parallel}
% \cite{parallel} and redefines some macros to fix bugs.
%
% If color stacks are available then package \xpackage{parallel}
% is further patched to support them.
%
% \subsection{Option \xoption{rulebetweencolor}}
%
% Package \xpackage{pdfcolparallel} also fixes the color for the
% rule between columns.
% Default color is \cs{normalcolor}. But this can be changed by using
% option \xoption{rulebetweencolor} for |\setkeys{parallel}|
% (see package \xpackage{keyval}). The option takes a color specification
% as value. If the value is empty, then the default (\cs{normalcolor})
% is used.
% Examples:
% \begin{quote}
%   |\setkeys{parallel}{rulebetweencolor=blue}|,\\
%   |\setkeys{parallel}{rulebetweencolor={red}}|,\\
%   |\setkeys{parallel}{rulebetweencolor={}}|,
%     \textit{\% \cs{normalcolor} is used}\\
%   |\setkeys{parallel}{rulebetweencolor=[rgb]{1,0,.5}}|
% \end{quote}
%
% \subsection{Future}
%
% If there will be a new version of package \xpackage{parallel}
% that adds support for color stacks, then this package may become
% obsolete.
%
% \StopEventually{
% }
%
% \section{Implementation}
%
% \subsection{Identification}
%
%    \begin{macrocode}
%<*package>
\NeedsTeXFormat{LaTeX2e}
\ProvidesPackage{pdfcolparallel}%
  [2016/05/16 v1.4 Color stacks support for parallel (HO)]%
%    \end{macrocode}
%
% \subsection{Load and fix package \xpackage{parallel}}
%
%    Package \xpackage{parallel} is loaded. Before options of package
%    \xpackage{pdfcolparallel} are passed to package \xpackage{parallel}.
%    \begin{macrocode}
\DeclareOption*{%
  \PassoptionsToPackage{\CurrentOption}{parallel}%
}
\ProcessOptions\relax
\RequirePackage{parallel}[2003/04/13]
%    \end{macrocode}
%
%    \begin{macrocode}
\RequirePackage{infwarerr}[2007/09/09]
%    \end{macrocode}
%
%    \begin{macro}{\pcp@ColorPatch}
%    \begin{macrocode}
\begingroup\expandafter\expandafter\expandafter\endgroup
\expandafter\ifx\csname currentgrouplevel\endcsname\relax
  \def\pcp@ColorPatch{}%
\else
  \def\pcp@ColorPatch{%
    \@ifundefined{set@color}{%
      \gdef\pcp@ColorPatch{}%
    }{%
      \gdef\pcp@ColorPatch{%
        \gdef\pcp@ColorResets{}%
        \bgroup
        \aftergroup\pcp@ColorResets
        \aftergroup\egroup
        \let\pcp@OrgSetColor\set@color
        \let\set@color\pcp@SetColor
        \edef\pcp@GroupLevel{\the\currentgrouplevel}%
      }%
    }%
    \pcp@ColorPatch
  }%
%    \end{macrocode}
%    \end{macro}
%    \begin{macro}{\pcp@SetColor}
%    \begin{macrocode}
  \def\pcp@SetColor{%
    \ifnum\pcp@GroupLevel=\currentgrouplevel
      \let\pcp@OrgAfterGroup\aftergroup
      \def\aftergroup{%
        \g@addto@macro\pcp@ColorResets
      }%
      \pcp@OrgSetColor
      \let\aftergroup\pcp@OrgAfterGroup
    \else
      \pcp@OrgSetColor
    \fi
  }%
\fi
%    \end{macrocode}
%    \end{macro}
%
%    \begin{macro}{\pcp@CmdCheckRedef}
%    \begin{macrocode}
\def\pcp@CmdCheckRedef#1{%
  \begingroup
    \def\pcp@cmd{#1}%
    \afterassignment\pcp@CmdDo
    \long\def\reserved@a
}
\def\pcp@CmdDo{%
    \expandafter\ifx\pcp@cmd\reserved@a
    \else
      \edef\x*{\expandafter\string\pcp@cmd}%
      \@PackageWarningNoLine{pdfcolparallel}{%
        Command \x* has changed.\MessageBreak
        Supported versions of package `parallel':\MessageBreak
        \space\space 2003/04/13\MessageBreak
        The redefinition of \x* may\MessageBreak
        not behave correctly depending on the changes%
      }%
    \fi
  \expandafter\endgroup
  \expandafter\def\pcp@cmd
}
%    \end{macrocode}
%    \end{macro}
%
%    \begin{macrocode}
\def\pcp@SwitchStack#1#2{}
%    \end{macrocode}
%    \begin{macrocode}
\def\pcp@SetCurrent#1{}
%    \end{macrocode}
%
%    \begin{macro}{\ParallelLText}
%    \begin{macrocode}
\pcp@CmdCheckRedef\ParallelLText{%
  \everypar{}%
  \@restorepar
  \begingroup
    \hbadness=3000 %
    \let\footnote=\ParallelLFootnote
    \ParallelWhichBox=0 %
    \global\setbox\ParallelLBox=\vbox\bgroup
      \hsize=\ParallelLWidth
      \aftergroup\ParallelAfterText
      \begingroup
        \afterassignment\ParallelCheckOpenBrace
        \let\x=%
}{%
  \everypar{}%
  \@restorepar
  \@nobreakfalse
  \begingroup
    \hbadness=3000 %
    \let\footnote=\ParallelLFootnote
    \ParallelWhichBox=0 %
    \global\setbox\ParallelLBox=\vbox\bgroup
      \hsize=\ParallelLWidth
      \linewidth=\ParallelLWidth
      \pcp@SwitchStack{Left}\ParallelLBox
      \aftergroup\ParallelAfterText
      \pcp@ColorPatch
      \begingroup
        \afterassignment\ParallelCheckOpenBrace
        \let\x=%
}
%    \end{macrocode}
%    \end{macro}
%
%    \begin{macro}{\ParallelRText}
%    \begin{macrocode}
\pcp@CmdCheckRedef\ParallelRText{%
  \everypar{}%
  \@restorepar
  \begingroup
    \hbadness=3000 %
    \ifnum\ParallelFNMode=\@ne
      \let\footnote=\ParallelRFootnote
    \else
      \let\footnote=\ParallelLFootnote
    \fi
    \ParallelWhichBox=\@ne
    \global\setbox\ParallelRBox=\vbox\bgroup
      \hsize=\ParallelRWidth
      \aftergroup\ParallelAfterText
      \begingroup
        \afterassignment\ParallelCheckOpenBrace
        \let\x=%
}{%
  \everypar{}%
  \@restorepar
  \@nobreakfalse
  \begingroup
    \hbadness=3000 %
    \ifnum\ParallelFNMode=\@ne
      \let\footnote=\ParallelRFootnote
    \else
      \let\footnote=\ParallelLFootnote
    \fi
    \ParallelWhichBox=\@ne
    \global\setbox\ParallelRBox=\vbox\bgroup
      \hsize=\ParallelRWidth
      \linewidth=\ParallelRWidth
      \pcp@SwitchStack{Right}\ParallelRBox
      \aftergroup\ParallelAfterText
      \pcp@ColorPatch
      \begingroup
        \afterassignment\ParallelCheckOpenBrace
        \let\x=%
}
%    \end{macrocode}
%    \end{macro}
%
%    \begin{macro}{\ParallelParTwoPages}
%    \begin{macrocode}
\pcp@CmdCheckRedef\ParallelParTwoPages{%
  \ifnum\ParallelBoolVar=\@ne
    \par
    \begingroup
      \global\ParallelWhichBox=\@ne
      \newpage
      \vbadness=10000 %
      \vfuzz=3ex %
      \splittopskip=\z@skip
      \loop%
        \ifnum\ParallelBoolVar=\@ne%
          \ifnum\ParallelWhichBox=\@ne
            \ifvoid\ParallelLBox
              \mbox{} %
              \newpage
            \else
              \global\ParallelWhichBox=\z@
            \fi
          \else
            \ifvoid\ParallelRBox
              \mbox{} %
              \newpage
            \else
              \global\ParallelWhichBox=\@ne
            \fi
          \fi
          \ifnum\ParallelWhichBox=\z@
            \ifodd\thepage
              \mbox{} %
              \newpage
            \fi
            \hbox to\textwidth{%
              \vbox{\vsplit\ParallelLBox to.98\textheight}%
            }%
          \else
            \ifodd\thepage\relax
            \else
              \mbox{} %
              \newpage
            \fi
            \hbox to\textwidth{%
              \vbox{\vsplit\ParallelRBox to.98\textheight}%
            }%
          \fi
          \vspace*{\fill}%
          \newpage
        \fi
        \ifvoid\ParallelLBox
          \ifvoid\ParallelRBox
            \global\ParallelBoolVar=\z@
          \fi
        \fi
      \ifnum\ParallelBoolVar=\@ne
      \repeat
      \par
    \endgroup
  \fi
}{%
%    \end{macrocode}
%    Additional fixes:
%    \begin{itemize}
%    \item Unnecessary white space removed.
%    \item |\ifodd\thepage| changed to |\ifodd\value{page}|.
%    \end{itemize}
%    \begin{macrocode}
  \ifnum\ParallelBoolVar=\@ne
    \par
    \begingroup
      \global\ParallelWhichBox=\@ne
      \newpage
      \vbadness=10000 %
      \vfuzz=3ex %
      \splittopskip=\z@skip
      \loop%
        \ifnum\ParallelBoolVar=\@ne%
          \ifnum\ParallelWhichBox=\@ne
            \ifvoid\ParallelLBox
              \mbox{}%
              \newpage
            \else
              \global\ParallelWhichBox=\z@
            \fi
          \else
            \ifvoid\ParallelRBox
              \null
              \newpage
            \else
              \global\ParallelWhichBox=\@ne
            \fi
          \fi
          \ifnum\ParallelWhichBox=\z@
            \ifodd\value{page}%
              \null
              \newpage
            \fi
            \hbox to\textwidth{%
              \pcp@SetCurrent{Left}%
              \setbox\z@=\vsplit\ParallelLBox to.98\textheight
              \vbox to.98\textheight{%
                \@texttop
                \unvbox\z@
                \@textbottom
              }%
            }%
          \else
            \ifodd\value{page}%
            \else
              \mbox{}%
              \newpage
            \fi
            \hbox to\textwidth{%
              \pcp@SetCurrent{Right}%
              \setbox\z@=\vsplit\ParallelRBox to.98\textheight
              \vbox to.98\textheight{%
                \@texttop
                \unvbox\z@
                \@textbottom
              }%
            }%
          \fi
          \vspace*{\fill}%
          \newpage
        \fi
        \ifvoid\ParallelLBox
          \ifvoid\ParallelRBox
            \global\ParallelBoolVar=\z@
          \fi
        \fi
      \ifnum\ParallelBoolVar=\@ne
      \repeat
      \par
    \endgroup
    \pcp@SetCurrent{}%
  \fi
}
%    \end{macrocode}
%    \end{macro}
%
% \subsection{Color stack support}
%
%    \begin{macrocode}
\RequirePackage{pdfcol}[2007/12/12]
\ifpdfcolAvailable
\else
  \PackageInfo{pdfcolparallel}{%
    Loading aborted, because color stacks are not available%
  }%
  \expandafter\endinput
\fi
%    \end{macrocode}
%
%    \begin{macrocode}
\pdfcolInitStack{pcp@Left}
\pdfcolInitStack{pcp@Right}
%    \end{macrocode}
%    \begin{macro}{\pcp@Box}
%    \begin{macrocode}
\newbox\pcp@Box
%    \end{macrocode}
%    \end{macro}
%    \begin{macro}{\pcp@SwitchStack}
%    \begin{macrocode}
\def\pcp@SwitchStack#1#2{%
  \pdfcolSwitchStack{pcp@#1}%
  \global\setbox\pcp@Box=\vbox to 0pt{%
    \pdfcolSetCurrentColor
  }%
  \aftergroup\pcp@FixBox
  \aftergroup#2%
}
%    \end{macrocode}
%    \end{macro}
%    \begin{macro}{\pcp@FixBox}
%    \begin{macrocode}
\def\pcp@FixBox#1{%
  \global\setbox#1=\vbox{%
    \unvbox\pcp@Box
    \unvbox#1%
  }%
}
%    \end{macrocode}
%    \end{macro}
%    \begin{macro}{\pcp@SetCurrent}
%    \begin{macrocode}
\def\pcp@SetCurrent#1{%
  \ifx\\#1\\%
    \pdfcolSetCurrent{}%
  \else
    \pdfcolSetCurrent{pcp@#1}%
  \fi
}
%    \end{macrocode}
%    \end{macro}
%
% \subsection{Redefinitions}
%
%    \begin{macro}{\ParallelParOnePage}
%    \begin{macrocode}
\pcp@CmdCheckRedef\ParallelParOnePage{%
  \ifnum\ParallelBoolVar=\@ne
    \par
    \begingroup
      \leftmargin=\z@
      \rightmargin=\z@
      \parskip=\z@skip
      \parindent=\z@
      \vbadness=10000 %
      \vfuzz=3ex %
      \splittopskip=\z@skip
      \loop
        \ifnum\ParallelBoolVar=\@ne
          \noindent
          \hbox to\textwidth{%
            \hskip\ParallelLeftMargin
            \hbox to\ParallelTextWidth{%
              \ifvoid\ParallelLBox
                \hskip\ParallelLWidth
              \else
                \ParallelWhichBox=\z@
                \vbox{%
                  \setbox\ParallelBoxVar
                      =\vsplit\ParallelLBox to\dp\strutbox
                  \unvbox\ParallelBoxVar
                }%
              \fi
              \strut
              \ifnum\ParallelBoolMid=\@ne
                \hskip\ParallelMainMidSkip
                \vrule
              \else
                \hss
              \fi
              \hss
              \ifvoid\ParallelRBox
                \hskip\ParallelRWidth
              \else
                \ParallelWhichBox=\@ne
                \vbox{%
                  \setbox\ParallelBoxVar
                      =\vsplit\ParallelRBox to\dp\strutbox
                  \unvbox\ParallelBoxVar
                }%
              \fi
            }%
          }%
          \ifvoid\ParallelLBox
            \ifvoid\ParallelRBox
              \global\ParallelBoolVar=\z@
            \fi
          \fi%
        \fi%
      \ifnum\ParallelBoolVar=\@ne
        \penalty\interlinepenalty
      \repeat
      \par
    \endgroup
  \fi
}{%
  \ifnum\ParallelBoolVar=\@ne
    \par
    \begingroup
      \leftmargin=\z@
      \rightmargin=\z@
      \parskip=\z@skip
      \parindent=\z@
      \vbadness=10000 %
      \vfuzz=3ex %
      \splittopskip=\z@skip
      \loop
        \ifnum\ParallelBoolVar=\@ne
          \noindent
          \hbox to\textwidth{%
            \hskip\ParallelLeftMargin
            \hbox to\ParallelTextWidth{%
              \ifvoid\ParallelLBox
                \hskip\ParallelLWidth
              \else
                \pcp@SetCurrent{Left}%
                \ParallelWhichBox=\z@
                \vbox{%
                  \setbox\ParallelBoxVar
                      =\vsplit\ParallelLBox to\dp\strutbox
                  \unvbox\ParallelBoxVar
                }%
              \fi
              \strut
              \ifnum\ParallelBoolMid=\@ne
                \hskip\ParallelMainMidSkip
                \begingroup
                  \pcp@RuleBetweenColor
                  \vrule
                \endgroup
              \else
                \hss
              \fi
              \hss
              \ifvoid\ParallelRBox
                \hskip\ParallelRWidth
              \else
                \pcp@SetCurrent{Right}%
                \ParallelWhichBox=\@ne
                \vbox{%
                  \setbox\ParallelBoxVar
                      =\vsplit\ParallelRBox to\dp\strutbox
                  \unvbox\ParallelBoxVar
                }%
              \fi
            }%
          }%
          \ifvoid\ParallelLBox
            \ifvoid\ParallelRBox
              \global\ParallelBoolVar=\z@
            \fi
          \fi%
        \fi%
      \ifnum\ParallelBoolVar=\@ne
        \penalty\interlinepenalty
      \repeat
      \par
    \endgroup
    \pcp@SetCurrent{}%
  \fi
}
%    \end{macrocode}
%    \end{macro}
%    \begin{macro}{\pcp@RuleBetweenColorDefault}
%    \begin{macrocode}
\def\pcp@RuleBetweenColorDefault{%
  \normalcolor
}
%    \end{macrocode}
%    \end{macro}
%    \begin{macro}{\pcp@RuleBetweenColor}
%    \begin{macrocode}
\let\pcp@RuleBetweenColor\pcp@RuleBetweenColorDefault
%    \end{macrocode}
%    \end{macro}
%    \begin{macrocode}
\RequirePackage{keyval}
\define@key{parallel}{rulebetweencolor}{%
  \edef\pcp@temp{#1}%
  \ifx\pcp@temp\@empty
    \let\pcp@RuleBetweenColor\pcp@RuleBetweenColorDefault
  \else
    \edef\pcp@temp{%
      \noexpand\@ifnextchar[{%
        \def\noexpand\pcp@RuleBetweenColor{%
          \noexpand\color\pcp@temp
        }%
        \noexpand\pcp@GobbleNil
      }{%
        \def\noexpand\pcp@RuleBetweenColor{%
          \noexpand\color{\pcp@temp}%
        }%
        \noexpand\pcp@GobbleNil
      }%
      \pcp@temp\noexpand\@nil
    }%
    \pcp@temp
  \fi
}
%    \end{macrocode}
%    \begin{macro}{\pcp@GobbleNil}
%    \begin{macrocode}
\long\def\pcp@GobbleNil#1\@nil{}
%    \end{macrocode}
%    \end{macro}
%
%    \begin{macrocode}
%</package>
%    \end{macrocode}
%
% \section{Test}
%
%    The test file is a modified version of the file that
%    Alexander Hirsch has posted in \xnewsgroup{de.comp.text.tex}:
%    \URL{``\link{\texttt{parallel.sty} und farbiger Text}''}^^A
%    {http://groups.google.com/group/de.comp.text.tex/msg/6a759cf33bb071a5}
%    \begin{macrocode}
%<*test1>
\AtEndDocument{%
  \typeout{}%
  \typeout{**************************************}%
  \typeout{*** \space Check the PDF file manually! \space ***}%
  \typeout{**************************************}%
  \typeout{}%
}
\documentclass{article}
\usepackage{xcolor}
\usepackage{pdfcolparallel}[2016/05/16]

\begin{document}
  \color{green}%
  Green%
  \begin{Parallel}{0.47\textwidth}{0.47\textwidth}%
    \ParallelLText{%
      \textcolor{red}{%
        Ein Absatz, der sich ueber zwei Zeilen erstrecken soll. %
        Ein Absatz, der sich ueber zwei Zeilen erstrecken soll.%
      }%
    }%
    \ParallelRText{%
      \textcolor{blue}{%
        Ein Absatz, der sich ueber zwei Zeilen erstrecken soll. %
        Ein Absatz, der sich ueber zwei Zeilen erstrecken soll.%
      }%
    }%
    \ParallelPar
    \ParallelLText{%
      Default %
      \color{red}%
      Ein Absatz, der sich ueber zwei Zeilen erstrecken soll. %
      Ein Absatz, der sich ueber zwei Zeilen erstrecken soll.%
    }%
    \ParallelRText{%
      Default %
      \color{blue}%
      Ein Absatz, der sich ueber zwei Zeilen erstrecken soll. %
      Ein Absatz, der sich ueber zwei Zeilen erstrecken soll.%
    }%
    \ParallelPar
    \ParallelLText{%
      \begin{enumerate}%
      \item left text, left text, left text, left text, %
            left text, left text, left text, left text,%
      \item left text, left text, left text, left text, %
            left text, left text, left text, left text.%
      \end{enumerate}%
    }%
    \ParallelRText{%
      \begin{enumerate}%
      \item right text, right text, right text, right text, %
            right text, right text, right text, right text.%
      \item right text, right text, right text, right text, %
            right text, right text, right text, right text.%
      \end{enumerate}%
    }%
  \end{Parallel}%
  \begin{Parallel}[p]{\textwidth}{\textwidth}%
    \ParallelLText{%
      \textcolor{red}{%
        Ein Absatz, der sich ueber zwei Zeilen erstrecken soll. %
        Ein Absatz, der sich ueber zwei Zeilen erstrecken soll. %
        Foo bar bla bla bla.%
      }%
      \par
      Und noch ein Absatz.%
    }%
    \ParallelRText{%
      \textcolor{blue}{%
        Ein Absatz, der sich ueber zwei Zeilen erstrecken soll. %
        Ein Absatz, der sich ueber zwei Zeilen erstrecken soll. %
        Foo bar bla bla bla.%
      }%
    }%
  \end{Parallel}%
  \begin{Parallel}[p]{\textwidth}{\textwidth}%
    \ParallelLText{%
      \rule{1pt}{.98\textheight}\Huge g%
    }%
    \ParallelRText{%
      \rule{1pt}{.98\textheight}y%
    }%
  \end{Parallel}%
  Green%
\end{document}
%</test1>
%    \end{macrocode}
%
% \section{Installation}
%
% \subsection{Download}
%
% \paragraph{Package.} This package is available on
% CTAN\footnote{\url{http://ctan.org/pkg/pdfcolparallel}}:
% \begin{description}
% \item[\CTAN{macros/latex/contrib/oberdiek/pdfcolparallel.dtx}] The source file.
% \item[\CTAN{macros/latex/contrib/oberdiek/pdfcolparallel.pdf}] Documentation.
% \end{description}
%
%
% \paragraph{Bundle.} All the packages of the bundle `oberdiek'
% are also available in a TDS compliant ZIP archive. There
% the packages are already unpacked and the documentation files
% are generated. The files and directories obey the TDS standard.
% \begin{description}
% \item[\CTAN{install/macros/latex/contrib/oberdiek.tds.zip}]
% \end{description}
% \emph{TDS} refers to the standard ``A Directory Structure
% for \TeX\ Files'' (\CTAN{tds/tds.pdf}). Directories
% with \xfile{texmf} in their name are usually organized this way.
%
% \subsection{Bundle installation}
%
% \paragraph{Unpacking.} Unpack the \xfile{oberdiek.tds.zip} in the
% TDS tree (also known as \xfile{texmf} tree) of your choice.
% Example (linux):
% \begin{quote}
%   |unzip oberdiek.tds.zip -d ~/texmf|
% \end{quote}
%
% \paragraph{Script installation.}
% Check the directory \xfile{TDS:scripts/oberdiek/} for
% scripts that need further installation steps.
% Package \xpackage{attachfile2} comes with the Perl script
% \xfile{pdfatfi.pl} that should be installed in such a way
% that it can be called as \texttt{pdfatfi}.
% Example (linux):
% \begin{quote}
%   |chmod +x scripts/oberdiek/pdfatfi.pl|\\
%   |cp scripts/oberdiek/pdfatfi.pl /usr/local/bin/|
% \end{quote}
%
% \subsection{Package installation}
%
% \paragraph{Unpacking.} The \xfile{.dtx} file is a self-extracting
% \docstrip\ archive. The files are extracted by running the
% \xfile{.dtx} through \plainTeX:
% \begin{quote}
%   \verb|tex pdfcolparallel.dtx|
% \end{quote}
%
% \paragraph{TDS.} Now the different files must be moved into
% the different directories in your installation TDS tree
% (also known as \xfile{texmf} tree):
% \begin{quote}
% \def\t{^^A
% \begin{tabular}{@{}>{\ttfamily}l@{ $\rightarrow$ }>{\ttfamily}l@{}}
%   pdfcolparallel.sty & tex/latex/oberdiek/pdfcolparallel.sty\\
%   pdfcolparallel.pdf & doc/latex/oberdiek/pdfcolparallel.pdf\\
%   test/pdfcolparallel-test1.tex & doc/latex/oberdiek/test/pdfcolparallel-test1.tex\\
%   pdfcolparallel.dtx & source/latex/oberdiek/pdfcolparallel.dtx\\
% \end{tabular}^^A
% }^^A
% \sbox0{\t}^^A
% \ifdim\wd0>\linewidth
%   \begingroup
%     \advance\linewidth by\leftmargin
%     \advance\linewidth by\rightmargin
%   \edef\x{\endgroup
%     \def\noexpand\lw{\the\linewidth}^^A
%   }\x
%   \def\lwbox{^^A
%     \leavevmode
%     \hbox to \linewidth{^^A
%       \kern-\leftmargin\relax
%       \hss
%       \usebox0
%       \hss
%       \kern-\rightmargin\relax
%     }^^A
%   }^^A
%   \ifdim\wd0>\lw
%     \sbox0{\small\t}^^A
%     \ifdim\wd0>\linewidth
%       \ifdim\wd0>\lw
%         \sbox0{\footnotesize\t}^^A
%         \ifdim\wd0>\linewidth
%           \ifdim\wd0>\lw
%             \sbox0{\scriptsize\t}^^A
%             \ifdim\wd0>\linewidth
%               \ifdim\wd0>\lw
%                 \sbox0{\tiny\t}^^A
%                 \ifdim\wd0>\linewidth
%                   \lwbox
%                 \else
%                   \usebox0
%                 \fi
%               \else
%                 \lwbox
%               \fi
%             \else
%               \usebox0
%             \fi
%           \else
%             \lwbox
%           \fi
%         \else
%           \usebox0
%         \fi
%       \else
%         \lwbox
%       \fi
%     \else
%       \usebox0
%     \fi
%   \else
%     \lwbox
%   \fi
% \else
%   \usebox0
% \fi
% \end{quote}
% If you have a \xfile{docstrip.cfg} that configures and enables \docstrip's
% TDS installing feature, then some files can already be in the right
% place, see the documentation of \docstrip.
%
% \subsection{Refresh file name databases}
%
% If your \TeX~distribution
% (\teTeX, \mikTeX, \dots) relies on file name databases, you must refresh
% these. For example, \teTeX\ users run \verb|texhash| or
% \verb|mktexlsr|.
%
% \subsection{Some details for the interested}
%
% \paragraph{Attached source.}
%
% The PDF documentation on CTAN also includes the
% \xfile{.dtx} source file. It can be extracted by
% AcrobatReader 6 or higher. Another option is \textsf{pdftk},
% e.g. unpack the file into the current directory:
% \begin{quote}
%   \verb|pdftk pdfcolparallel.pdf unpack_files output .|
% \end{quote}
%
% \paragraph{Unpacking with \LaTeX.}
% The \xfile{.dtx} chooses its action depending on the format:
% \begin{description}
% \item[\plainTeX:] Run \docstrip\ and extract the files.
% \item[\LaTeX:] Generate the documentation.
% \end{description}
% If you insist on using \LaTeX\ for \docstrip\ (really,
% \docstrip\ does not need \LaTeX), then inform the autodetect routine
% about your intention:
% \begin{quote}
%   \verb|latex \let\install=y\input{pdfcolparallel.dtx}|
% \end{quote}
% Do not forget to quote the argument according to the demands
% of your shell.
%
% \paragraph{Generating the documentation.}
% You can use both the \xfile{.dtx} or the \xfile{.drv} to generate
% the documentation. The process can be configured by the
% configuration file \xfile{ltxdoc.cfg}. For instance, put this
% line into this file, if you want to have A4 as paper format:
% \begin{quote}
%   \verb|\PassOptionsToClass{a4paper}{article}|
% \end{quote}
% An example follows how to generate the
% documentation with pdf\LaTeX:
% \begin{quote}
%\begin{verbatim}
%pdflatex pdfcolparallel.dtx
%makeindex -s gind.ist pdfcolparallel.idx
%pdflatex pdfcolparallel.dtx
%makeindex -s gind.ist pdfcolparallel.idx
%pdflatex pdfcolparallel.dtx
%\end{verbatim}
% \end{quote}
%
% \section{Catalogue}
%
% The following XML file can be used as source for the
% \href{http://mirror.ctan.org/help/Catalogue/catalogue.html}{\TeX\ Catalogue}.
% The elements \texttt{caption} and \texttt{description} are imported
% from the original XML file from the Catalogue.
% The name of the XML file in the Catalogue is \xfile{pdfcolparallel.xml}.
%    \begin{macrocode}
%<*catalogue>
<?xml version='1.0' encoding='us-ascii'?>
<!DOCTYPE entry SYSTEM 'catalogue.dtd'>
<entry datestamp='$Date$' modifier='$Author$' id='pdfcolparallel'>
  <name>pdfcolparallel</name>
  <caption>Fix colour problems in package 'parallel'.</caption>
  <authorref id='auth:oberdiek'/>
  <copyright owner='Heiko Oberdiek' year='2007,2008,2010'/>
  <license type='lppl1.3'/>
  <version number='1.4'/>
  <description>
    Since version 1.40 pdfTeX supports colour stacks.
    This package uses them to fix colour problems in
    package <xref refid='parallel'>parallel</xref>.
    <p/>
    The package is part of the <xref refid='oberdiek'>oberdiek</xref>
    bundle.
  </description>
  <documentation details='Package documentation'
      href='ctan:/macros/latex/contrib/oberdiek/pdfcolparallel.pdf'/>
  <ctan file='true' path='/macros/latex/contrib/oberdiek/pdfcolparallel.dtx'/>
  <miktex location='oberdiek'/>
  <texlive location='oberdiek'/>
  <install path='/macros/latex/contrib/oberdiek/oberdiek.tds.zip'/>
</entry>
%</catalogue>
%    \end{macrocode}
%
% \begin{thebibliography}{9}
%
% \bibitem{parallel}
%   Matthias Eckermann: \textit{The \xpackage{parallel}-package};
%   2003/04/13;\\
%   \CTAN{macros/latex/contrib/parallel/}.
%
% \bibitem{pdfcol}
%   Heiko Oberdiek: \textit{The \xpackage{pdfcol} package};
%   2007/09/09;\\
%   \CTAN{macros/latex/contrib/oberdiek/pdfcol.pdf}.
%
% \end{thebibliography}
%
% \begin{History}
%   \begin{Version}{2007/09/09 v1.0}
%   \item
%     First version.
%   \end{Version}
%   \begin{Version}{2007/12/12 v1.1}
%   \item
%     Adds patch for setting \cs{linewidth} to fix bug
%     in package \xpackage{parallel}.
%   \item
%     Package \xpackage{parallel} is also fixed if color
%     stacks are not available.
%   \item
%     Bug fix, switched stacks now initialized with current color.
%   \item
%     Fix for package \xpackage{parallel}: \cs{raggedbottom} is respected.
%   \end{Version}
%   \begin{Version}{2008/08/11 v1.2}
%   \item
%     Code is not changed.
%   \item
%     URLs updated.
%   \end{Version}
%   \begin{Version}{2010/01/11 v1.3}
%   \item
%     Option `rulebetweencolor' added.
%   \end{Version}
%   \begin{Version}{2016/05/16 v1.4}
%   \item
%     Documentation updates.
%   \end{Version}
% \end{History}
%
% \PrintIndex
%
% \Finale
\endinput

%        (quote the arguments according to the demands of your shell)
%
% Documentation:
%    (a) If pdfcolparallel.drv is present:
%           latex pdfcolparallel.drv
%    (b) Without pdfcolparallel.drv:
%           latex pdfcolparallel.dtx; ...
%    The class ltxdoc loads the configuration file ltxdoc.cfg
%    if available. Here you can specify further options, e.g.
%    use A4 as paper format:
%       \PassOptionsToClass{a4paper}{article}
%
%    Programm calls to get the documentation (example):
%       pdflatex pdfcolparallel.dtx
%       makeindex -s gind.ist pdfcolparallel.idx
%       pdflatex pdfcolparallel.dtx
%       makeindex -s gind.ist pdfcolparallel.idx
%       pdflatex pdfcolparallel.dtx
%
% Installation:
%    TDS:tex/latex/oberdiek/pdfcolparallel.sty
%    TDS:doc/latex/oberdiek/pdfcolparallel.pdf
%    TDS:doc/latex/oberdiek/test/pdfcolparallel-test1.tex
%    TDS:source/latex/oberdiek/pdfcolparallel.dtx
%
%<*ignore>
\begingroup
  \catcode123=1 %
  \catcode125=2 %
  \def\x{LaTeX2e}%
\expandafter\endgroup
\ifcase 0\ifx\install y1\fi\expandafter
         \ifx\csname processbatchFile\endcsname\relax\else1\fi
         \ifx\fmtname\x\else 1\fi\relax
\else\csname fi\endcsname
%</ignore>
%<*install>
\input docstrip.tex
\Msg{************************************************************************}
\Msg{* Installation}
\Msg{* Package: pdfcolparallel 2016/05/16 v1.4 Color stacks support for parallel (HO)}
\Msg{************************************************************************}

\keepsilent
\askforoverwritefalse

\let\MetaPrefix\relax
\preamble

This is a generated file.

Project: pdfcolparallel
Version: 2016/05/16 v1.4

Copyright (C) 2007, 2008, 2010 by
   Heiko Oberdiek <heiko.oberdiek at googlemail.com>

This work may be distributed and/or modified under the
conditions of the LaTeX Project Public License, either
version 1.3c of this license or (at your option) any later
version. This version of this license is in
   http://www.latex-project.org/lppl/lppl-1-3c.txt
and the latest version of this license is in
   http://www.latex-project.org/lppl.txt
and version 1.3 or later is part of all distributions of
LaTeX version 2005/12/01 or later.

This work has the LPPL maintenance status "maintained".

This Current Maintainer of this work is Heiko Oberdiek.

This work consists of the main source file pdfcolparallel.dtx
and the derived files
   pdfcolparallel.sty, pdfcolparallel.pdf, pdfcolparallel.ins,
   pdfcolparallel.drv, pdfcolparallel-test1.tex.

\endpreamble
\let\MetaPrefix\DoubleperCent

\generate{%
  \file{pdfcolparallel.ins}{\from{pdfcolparallel.dtx}{install}}%
  \file{pdfcolparallel.drv}{\from{pdfcolparallel.dtx}{driver}}%
  \usedir{tex/latex/oberdiek}%
  \file{pdfcolparallel.sty}{\from{pdfcolparallel.dtx}{package}}%
  \usedir{doc/latex/oberdiek/test}%
  \file{pdfcolparallel-test1.tex}{\from{pdfcolparallel.dtx}{test1}}%
  \nopreamble
  \nopostamble
  \usedir{source/latex/oberdiek/catalogue}%
  \file{pdfcolparallel.xml}{\from{pdfcolparallel.dtx}{catalogue}}%
}

\catcode32=13\relax% active space
\let =\space%
\Msg{************************************************************************}
\Msg{*}
\Msg{* To finish the installation you have to move the following}
\Msg{* file into a directory searched by TeX:}
\Msg{*}
\Msg{*     pdfcolparallel.sty}
\Msg{*}
\Msg{* To produce the documentation run the file `pdfcolparallel.drv'}
\Msg{* through LaTeX.}
\Msg{*}
\Msg{* Happy TeXing!}
\Msg{*}
\Msg{************************************************************************}

\endbatchfile
%</install>
%<*ignore>
\fi
%</ignore>
%<*driver>
\NeedsTeXFormat{LaTeX2e}
\ProvidesFile{pdfcolparallel.drv}%
  [2016/05/16 v1.4 Color stacks support for parallel (HO)]%
\documentclass{ltxdoc}
\usepackage{holtxdoc}[2011/11/22]
\begin{document}
  \DocInput{pdfcolparallel.dtx}%
\end{document}
%</driver>
% \fi
%
%
% \CharacterTable
%  {Upper-case    \A\B\C\D\E\F\G\H\I\J\K\L\M\N\O\P\Q\R\S\T\U\V\W\X\Y\Z
%   Lower-case    \a\b\c\d\e\f\g\h\i\j\k\l\m\n\o\p\q\r\s\t\u\v\w\x\y\z
%   Digits        \0\1\2\3\4\5\6\7\8\9
%   Exclamation   \!     Double quote  \"     Hash (number) \#
%   Dollar        \$     Percent       \%     Ampersand     \&
%   Acute accent  \'     Left paren    \(     Right paren   \)
%   Asterisk      \*     Plus          \+     Comma         \,
%   Minus         \-     Point         \.     Solidus       \/
%   Colon         \:     Semicolon     \;     Less than     \<
%   Equals        \=     Greater than  \>     Question mark \?
%   Commercial at \@     Left bracket  \[     Backslash     \\
%   Right bracket \]     Circumflex    \^     Underscore    \_
%   Grave accent  \`     Left brace    \{     Vertical bar  \|
%   Right brace   \}     Tilde         \~}
%
% \GetFileInfo{pdfcolparallel.drv}
%
% \title{The \xpackage{pdfcolparallel} package}
% \date{2016/05/16 v1.4}
% \author{Heiko Oberdiek\thanks
% {Please report any issues at https://github.com/ho-tex/oberdiek/issues}\\
% \xemail{heiko.oberdiek at googlemail.com}}
%
% \maketitle
%
% \begin{abstract}
% This packages fixes bugs in \xpackage{parallel} and
% improves color support by using several color stacks
% that are provided by \pdfTeX\ since version 1.40.
% \end{abstract}
%
% \tableofcontents
%
% \section{Usage}
%
% \begin{quote}
% |\usepackage{pdfcolparallel}|
% \end{quote}
% The package \xpackage{pdfcolparallel} loads package \xpackage{parallel}
% \cite{parallel} and redefines some macros to fix bugs.
%
% If color stacks are available then package \xpackage{parallel}
% is further patched to support them.
%
% \subsection{Option \xoption{rulebetweencolor}}
%
% Package \xpackage{pdfcolparallel} also fixes the color for the
% rule between columns.
% Default color is \cs{normalcolor}. But this can be changed by using
% option \xoption{rulebetweencolor} for |\setkeys{parallel}|
% (see package \xpackage{keyval}). The option takes a color specification
% as value. If the value is empty, then the default (\cs{normalcolor})
% is used.
% Examples:
% \begin{quote}
%   |\setkeys{parallel}{rulebetweencolor=blue}|,\\
%   |\setkeys{parallel}{rulebetweencolor={red}}|,\\
%   |\setkeys{parallel}{rulebetweencolor={}}|,
%     \textit{\% \cs{normalcolor} is used}\\
%   |\setkeys{parallel}{rulebetweencolor=[rgb]{1,0,.5}}|
% \end{quote}
%
% \subsection{Future}
%
% If there will be a new version of package \xpackage{parallel}
% that adds support for color stacks, then this package may become
% obsolete.
%
% \StopEventually{
% }
%
% \section{Implementation}
%
% \subsection{Identification}
%
%    \begin{macrocode}
%<*package>
\NeedsTeXFormat{LaTeX2e}
\ProvidesPackage{pdfcolparallel}%
  [2016/05/16 v1.4 Color stacks support for parallel (HO)]%
%    \end{macrocode}
%
% \subsection{Load and fix package \xpackage{parallel}}
%
%    Package \xpackage{parallel} is loaded. Before options of package
%    \xpackage{pdfcolparallel} are passed to package \xpackage{parallel}.
%    \begin{macrocode}
\DeclareOption*{%
  \PassoptionsToPackage{\CurrentOption}{parallel}%
}
\ProcessOptions\relax
\RequirePackage{parallel}[2003/04/13]
%    \end{macrocode}
%
%    \begin{macrocode}
\RequirePackage{infwarerr}[2007/09/09]
%    \end{macrocode}
%
%    \begin{macro}{\pcp@ColorPatch}
%    \begin{macrocode}
\begingroup\expandafter\expandafter\expandafter\endgroup
\expandafter\ifx\csname currentgrouplevel\endcsname\relax
  \def\pcp@ColorPatch{}%
\else
  \def\pcp@ColorPatch{%
    \@ifundefined{set@color}{%
      \gdef\pcp@ColorPatch{}%
    }{%
      \gdef\pcp@ColorPatch{%
        \gdef\pcp@ColorResets{}%
        \bgroup
        \aftergroup\pcp@ColorResets
        \aftergroup\egroup
        \let\pcp@OrgSetColor\set@color
        \let\set@color\pcp@SetColor
        \edef\pcp@GroupLevel{\the\currentgrouplevel}%
      }%
    }%
    \pcp@ColorPatch
  }%
%    \end{macrocode}
%    \end{macro}
%    \begin{macro}{\pcp@SetColor}
%    \begin{macrocode}
  \def\pcp@SetColor{%
    \ifnum\pcp@GroupLevel=\currentgrouplevel
      \let\pcp@OrgAfterGroup\aftergroup
      \def\aftergroup{%
        \g@addto@macro\pcp@ColorResets
      }%
      \pcp@OrgSetColor
      \let\aftergroup\pcp@OrgAfterGroup
    \else
      \pcp@OrgSetColor
    \fi
  }%
\fi
%    \end{macrocode}
%    \end{macro}
%
%    \begin{macro}{\pcp@CmdCheckRedef}
%    \begin{macrocode}
\def\pcp@CmdCheckRedef#1{%
  \begingroup
    \def\pcp@cmd{#1}%
    \afterassignment\pcp@CmdDo
    \long\def\reserved@a
}
\def\pcp@CmdDo{%
    \expandafter\ifx\pcp@cmd\reserved@a
    \else
      \edef\x*{\expandafter\string\pcp@cmd}%
      \@PackageWarningNoLine{pdfcolparallel}{%
        Command \x* has changed.\MessageBreak
        Supported versions of package `parallel':\MessageBreak
        \space\space 2003/04/13\MessageBreak
        The redefinition of \x* may\MessageBreak
        not behave correctly depending on the changes%
      }%
    \fi
  \expandafter\endgroup
  \expandafter\def\pcp@cmd
}
%    \end{macrocode}
%    \end{macro}
%
%    \begin{macrocode}
\def\pcp@SwitchStack#1#2{}
%    \end{macrocode}
%    \begin{macrocode}
\def\pcp@SetCurrent#1{}
%    \end{macrocode}
%
%    \begin{macro}{\ParallelLText}
%    \begin{macrocode}
\pcp@CmdCheckRedef\ParallelLText{%
  \everypar{}%
  \@restorepar
  \begingroup
    \hbadness=3000 %
    \let\footnote=\ParallelLFootnote
    \ParallelWhichBox=0 %
    \global\setbox\ParallelLBox=\vbox\bgroup
      \hsize=\ParallelLWidth
      \aftergroup\ParallelAfterText
      \begingroup
        \afterassignment\ParallelCheckOpenBrace
        \let\x=%
}{%
  \everypar{}%
  \@restorepar
  \@nobreakfalse
  \begingroup
    \hbadness=3000 %
    \let\footnote=\ParallelLFootnote
    \ParallelWhichBox=0 %
    \global\setbox\ParallelLBox=\vbox\bgroup
      \hsize=\ParallelLWidth
      \linewidth=\ParallelLWidth
      \pcp@SwitchStack{Left}\ParallelLBox
      \aftergroup\ParallelAfterText
      \pcp@ColorPatch
      \begingroup
        \afterassignment\ParallelCheckOpenBrace
        \let\x=%
}
%    \end{macrocode}
%    \end{macro}
%
%    \begin{macro}{\ParallelRText}
%    \begin{macrocode}
\pcp@CmdCheckRedef\ParallelRText{%
  \everypar{}%
  \@restorepar
  \begingroup
    \hbadness=3000 %
    \ifnum\ParallelFNMode=\@ne
      \let\footnote=\ParallelRFootnote
    \else
      \let\footnote=\ParallelLFootnote
    \fi
    \ParallelWhichBox=\@ne
    \global\setbox\ParallelRBox=\vbox\bgroup
      \hsize=\ParallelRWidth
      \aftergroup\ParallelAfterText
      \begingroup
        \afterassignment\ParallelCheckOpenBrace
        \let\x=%
}{%
  \everypar{}%
  \@restorepar
  \@nobreakfalse
  \begingroup
    \hbadness=3000 %
    \ifnum\ParallelFNMode=\@ne
      \let\footnote=\ParallelRFootnote
    \else
      \let\footnote=\ParallelLFootnote
    \fi
    \ParallelWhichBox=\@ne
    \global\setbox\ParallelRBox=\vbox\bgroup
      \hsize=\ParallelRWidth
      \linewidth=\ParallelRWidth
      \pcp@SwitchStack{Right}\ParallelRBox
      \aftergroup\ParallelAfterText
      \pcp@ColorPatch
      \begingroup
        \afterassignment\ParallelCheckOpenBrace
        \let\x=%
}
%    \end{macrocode}
%    \end{macro}
%
%    \begin{macro}{\ParallelParTwoPages}
%    \begin{macrocode}
\pcp@CmdCheckRedef\ParallelParTwoPages{%
  \ifnum\ParallelBoolVar=\@ne
    \par
    \begingroup
      \global\ParallelWhichBox=\@ne
      \newpage
      \vbadness=10000 %
      \vfuzz=3ex %
      \splittopskip=\z@skip
      \loop%
        \ifnum\ParallelBoolVar=\@ne%
          \ifnum\ParallelWhichBox=\@ne
            \ifvoid\ParallelLBox
              \mbox{} %
              \newpage
            \else
              \global\ParallelWhichBox=\z@
            \fi
          \else
            \ifvoid\ParallelRBox
              \mbox{} %
              \newpage
            \else
              \global\ParallelWhichBox=\@ne
            \fi
          \fi
          \ifnum\ParallelWhichBox=\z@
            \ifodd\thepage
              \mbox{} %
              \newpage
            \fi
            \hbox to\textwidth{%
              \vbox{\vsplit\ParallelLBox to.98\textheight}%
            }%
          \else
            \ifodd\thepage\relax
            \else
              \mbox{} %
              \newpage
            \fi
            \hbox to\textwidth{%
              \vbox{\vsplit\ParallelRBox to.98\textheight}%
            }%
          \fi
          \vspace*{\fill}%
          \newpage
        \fi
        \ifvoid\ParallelLBox
          \ifvoid\ParallelRBox
            \global\ParallelBoolVar=\z@
          \fi
        \fi
      \ifnum\ParallelBoolVar=\@ne
      \repeat
      \par
    \endgroup
  \fi
}{%
%    \end{macrocode}
%    Additional fixes:
%    \begin{itemize}
%    \item Unnecessary white space removed.
%    \item |\ifodd\thepage| changed to |\ifodd\value{page}|.
%    \end{itemize}
%    \begin{macrocode}
  \ifnum\ParallelBoolVar=\@ne
    \par
    \begingroup
      \global\ParallelWhichBox=\@ne
      \newpage
      \vbadness=10000 %
      \vfuzz=3ex %
      \splittopskip=\z@skip
      \loop%
        \ifnum\ParallelBoolVar=\@ne%
          \ifnum\ParallelWhichBox=\@ne
            \ifvoid\ParallelLBox
              \mbox{}%
              \newpage
            \else
              \global\ParallelWhichBox=\z@
            \fi
          \else
            \ifvoid\ParallelRBox
              \null
              \newpage
            \else
              \global\ParallelWhichBox=\@ne
            \fi
          \fi
          \ifnum\ParallelWhichBox=\z@
            \ifodd\value{page}%
              \null
              \newpage
            \fi
            \hbox to\textwidth{%
              \pcp@SetCurrent{Left}%
              \setbox\z@=\vsplit\ParallelLBox to.98\textheight
              \vbox to.98\textheight{%
                \@texttop
                \unvbox\z@
                \@textbottom
              }%
            }%
          \else
            \ifodd\value{page}%
            \else
              \mbox{}%
              \newpage
            \fi
            \hbox to\textwidth{%
              \pcp@SetCurrent{Right}%
              \setbox\z@=\vsplit\ParallelRBox to.98\textheight
              \vbox to.98\textheight{%
                \@texttop
                \unvbox\z@
                \@textbottom
              }%
            }%
          \fi
          \vspace*{\fill}%
          \newpage
        \fi
        \ifvoid\ParallelLBox
          \ifvoid\ParallelRBox
            \global\ParallelBoolVar=\z@
          \fi
        \fi
      \ifnum\ParallelBoolVar=\@ne
      \repeat
      \par
    \endgroup
    \pcp@SetCurrent{}%
  \fi
}
%    \end{macrocode}
%    \end{macro}
%
% \subsection{Color stack support}
%
%    \begin{macrocode}
\RequirePackage{pdfcol}[2007/12/12]
\ifpdfcolAvailable
\else
  \PackageInfo{pdfcolparallel}{%
    Loading aborted, because color stacks are not available%
  }%
  \expandafter\endinput
\fi
%    \end{macrocode}
%
%    \begin{macrocode}
\pdfcolInitStack{pcp@Left}
\pdfcolInitStack{pcp@Right}
%    \end{macrocode}
%    \begin{macro}{\pcp@Box}
%    \begin{macrocode}
\newbox\pcp@Box
%    \end{macrocode}
%    \end{macro}
%    \begin{macro}{\pcp@SwitchStack}
%    \begin{macrocode}
\def\pcp@SwitchStack#1#2{%
  \pdfcolSwitchStack{pcp@#1}%
  \global\setbox\pcp@Box=\vbox to 0pt{%
    \pdfcolSetCurrentColor
  }%
  \aftergroup\pcp@FixBox
  \aftergroup#2%
}
%    \end{macrocode}
%    \end{macro}
%    \begin{macro}{\pcp@FixBox}
%    \begin{macrocode}
\def\pcp@FixBox#1{%
  \global\setbox#1=\vbox{%
    \unvbox\pcp@Box
    \unvbox#1%
  }%
}
%    \end{macrocode}
%    \end{macro}
%    \begin{macro}{\pcp@SetCurrent}
%    \begin{macrocode}
\def\pcp@SetCurrent#1{%
  \ifx\\#1\\%
    \pdfcolSetCurrent{}%
  \else
    \pdfcolSetCurrent{pcp@#1}%
  \fi
}
%    \end{macrocode}
%    \end{macro}
%
% \subsection{Redefinitions}
%
%    \begin{macro}{\ParallelParOnePage}
%    \begin{macrocode}
\pcp@CmdCheckRedef\ParallelParOnePage{%
  \ifnum\ParallelBoolVar=\@ne
    \par
    \begingroup
      \leftmargin=\z@
      \rightmargin=\z@
      \parskip=\z@skip
      \parindent=\z@
      \vbadness=10000 %
      \vfuzz=3ex %
      \splittopskip=\z@skip
      \loop
        \ifnum\ParallelBoolVar=\@ne
          \noindent
          \hbox to\textwidth{%
            \hskip\ParallelLeftMargin
            \hbox to\ParallelTextWidth{%
              \ifvoid\ParallelLBox
                \hskip\ParallelLWidth
              \else
                \ParallelWhichBox=\z@
                \vbox{%
                  \setbox\ParallelBoxVar
                      =\vsplit\ParallelLBox to\dp\strutbox
                  \unvbox\ParallelBoxVar
                }%
              \fi
              \strut
              \ifnum\ParallelBoolMid=\@ne
                \hskip\ParallelMainMidSkip
                \vrule
              \else
                \hss
              \fi
              \hss
              \ifvoid\ParallelRBox
                \hskip\ParallelRWidth
              \else
                \ParallelWhichBox=\@ne
                \vbox{%
                  \setbox\ParallelBoxVar
                      =\vsplit\ParallelRBox to\dp\strutbox
                  \unvbox\ParallelBoxVar
                }%
              \fi
            }%
          }%
          \ifvoid\ParallelLBox
            \ifvoid\ParallelRBox
              \global\ParallelBoolVar=\z@
            \fi
          \fi%
        \fi%
      \ifnum\ParallelBoolVar=\@ne
        \penalty\interlinepenalty
      \repeat
      \par
    \endgroup
  \fi
}{%
  \ifnum\ParallelBoolVar=\@ne
    \par
    \begingroup
      \leftmargin=\z@
      \rightmargin=\z@
      \parskip=\z@skip
      \parindent=\z@
      \vbadness=10000 %
      \vfuzz=3ex %
      \splittopskip=\z@skip
      \loop
        \ifnum\ParallelBoolVar=\@ne
          \noindent
          \hbox to\textwidth{%
            \hskip\ParallelLeftMargin
            \hbox to\ParallelTextWidth{%
              \ifvoid\ParallelLBox
                \hskip\ParallelLWidth
              \else
                \pcp@SetCurrent{Left}%
                \ParallelWhichBox=\z@
                \vbox{%
                  \setbox\ParallelBoxVar
                      =\vsplit\ParallelLBox to\dp\strutbox
                  \unvbox\ParallelBoxVar
                }%
              \fi
              \strut
              \ifnum\ParallelBoolMid=\@ne
                \hskip\ParallelMainMidSkip
                \begingroup
                  \pcp@RuleBetweenColor
                  \vrule
                \endgroup
              \else
                \hss
              \fi
              \hss
              \ifvoid\ParallelRBox
                \hskip\ParallelRWidth
              \else
                \pcp@SetCurrent{Right}%
                \ParallelWhichBox=\@ne
                \vbox{%
                  \setbox\ParallelBoxVar
                      =\vsplit\ParallelRBox to\dp\strutbox
                  \unvbox\ParallelBoxVar
                }%
              \fi
            }%
          }%
          \ifvoid\ParallelLBox
            \ifvoid\ParallelRBox
              \global\ParallelBoolVar=\z@
            \fi
          \fi%
        \fi%
      \ifnum\ParallelBoolVar=\@ne
        \penalty\interlinepenalty
      \repeat
      \par
    \endgroup
    \pcp@SetCurrent{}%
  \fi
}
%    \end{macrocode}
%    \end{macro}
%    \begin{macro}{\pcp@RuleBetweenColorDefault}
%    \begin{macrocode}
\def\pcp@RuleBetweenColorDefault{%
  \normalcolor
}
%    \end{macrocode}
%    \end{macro}
%    \begin{macro}{\pcp@RuleBetweenColor}
%    \begin{macrocode}
\let\pcp@RuleBetweenColor\pcp@RuleBetweenColorDefault
%    \end{macrocode}
%    \end{macro}
%    \begin{macrocode}
\RequirePackage{keyval}
\define@key{parallel}{rulebetweencolor}{%
  \edef\pcp@temp{#1}%
  \ifx\pcp@temp\@empty
    \let\pcp@RuleBetweenColor\pcp@RuleBetweenColorDefault
  \else
    \edef\pcp@temp{%
      \noexpand\@ifnextchar[{%
        \def\noexpand\pcp@RuleBetweenColor{%
          \noexpand\color\pcp@temp
        }%
        \noexpand\pcp@GobbleNil
      }{%
        \def\noexpand\pcp@RuleBetweenColor{%
          \noexpand\color{\pcp@temp}%
        }%
        \noexpand\pcp@GobbleNil
      }%
      \pcp@temp\noexpand\@nil
    }%
    \pcp@temp
  \fi
}
%    \end{macrocode}
%    \begin{macro}{\pcp@GobbleNil}
%    \begin{macrocode}
\long\def\pcp@GobbleNil#1\@nil{}
%    \end{macrocode}
%    \end{macro}
%
%    \begin{macrocode}
%</package>
%    \end{macrocode}
%
% \section{Test}
%
%    The test file is a modified version of the file that
%    Alexander Hirsch has posted in \xnewsgroup{de.comp.text.tex}:
%    \URL{``\link{\texttt{parallel.sty} und farbiger Text}''}^^A
%    {http://groups.google.com/group/de.comp.text.tex/msg/6a759cf33bb071a5}
%    \begin{macrocode}
%<*test1>
\AtEndDocument{%
  \typeout{}%
  \typeout{**************************************}%
  \typeout{*** \space Check the PDF file manually! \space ***}%
  \typeout{**************************************}%
  \typeout{}%
}
\documentclass{article}
\usepackage{xcolor}
\usepackage{pdfcolparallel}[2016/05/16]

\begin{document}
  \color{green}%
  Green%
  \begin{Parallel}{0.47\textwidth}{0.47\textwidth}%
    \ParallelLText{%
      \textcolor{red}{%
        Ein Absatz, der sich ueber zwei Zeilen erstrecken soll. %
        Ein Absatz, der sich ueber zwei Zeilen erstrecken soll.%
      }%
    }%
    \ParallelRText{%
      \textcolor{blue}{%
        Ein Absatz, der sich ueber zwei Zeilen erstrecken soll. %
        Ein Absatz, der sich ueber zwei Zeilen erstrecken soll.%
      }%
    }%
    \ParallelPar
    \ParallelLText{%
      Default %
      \color{red}%
      Ein Absatz, der sich ueber zwei Zeilen erstrecken soll. %
      Ein Absatz, der sich ueber zwei Zeilen erstrecken soll.%
    }%
    \ParallelRText{%
      Default %
      \color{blue}%
      Ein Absatz, der sich ueber zwei Zeilen erstrecken soll. %
      Ein Absatz, der sich ueber zwei Zeilen erstrecken soll.%
    }%
    \ParallelPar
    \ParallelLText{%
      \begin{enumerate}%
      \item left text, left text, left text, left text, %
            left text, left text, left text, left text,%
      \item left text, left text, left text, left text, %
            left text, left text, left text, left text.%
      \end{enumerate}%
    }%
    \ParallelRText{%
      \begin{enumerate}%
      \item right text, right text, right text, right text, %
            right text, right text, right text, right text.%
      \item right text, right text, right text, right text, %
            right text, right text, right text, right text.%
      \end{enumerate}%
    }%
  \end{Parallel}%
  \begin{Parallel}[p]{\textwidth}{\textwidth}%
    \ParallelLText{%
      \textcolor{red}{%
        Ein Absatz, der sich ueber zwei Zeilen erstrecken soll. %
        Ein Absatz, der sich ueber zwei Zeilen erstrecken soll. %
        Foo bar bla bla bla.%
      }%
      \par
      Und noch ein Absatz.%
    }%
    \ParallelRText{%
      \textcolor{blue}{%
        Ein Absatz, der sich ueber zwei Zeilen erstrecken soll. %
        Ein Absatz, der sich ueber zwei Zeilen erstrecken soll. %
        Foo bar bla bla bla.%
      }%
    }%
  \end{Parallel}%
  \begin{Parallel}[p]{\textwidth}{\textwidth}%
    \ParallelLText{%
      \rule{1pt}{.98\textheight}\Huge g%
    }%
    \ParallelRText{%
      \rule{1pt}{.98\textheight}y%
    }%
  \end{Parallel}%
  Green%
\end{document}
%</test1>
%    \end{macrocode}
%
% \section{Installation}
%
% \subsection{Download}
%
% \paragraph{Package.} This package is available on
% CTAN\footnote{\url{http://ctan.org/pkg/pdfcolparallel}}:
% \begin{description}
% \item[\CTAN{macros/latex/contrib/oberdiek/pdfcolparallel.dtx}] The source file.
% \item[\CTAN{macros/latex/contrib/oberdiek/pdfcolparallel.pdf}] Documentation.
% \end{description}
%
%
% \paragraph{Bundle.} All the packages of the bundle `oberdiek'
% are also available in a TDS compliant ZIP archive. There
% the packages are already unpacked and the documentation files
% are generated. The files and directories obey the TDS standard.
% \begin{description}
% \item[\CTAN{install/macros/latex/contrib/oberdiek.tds.zip}]
% \end{description}
% \emph{TDS} refers to the standard ``A Directory Structure
% for \TeX\ Files'' (\CTAN{tds/tds.pdf}). Directories
% with \xfile{texmf} in their name are usually organized this way.
%
% \subsection{Bundle installation}
%
% \paragraph{Unpacking.} Unpack the \xfile{oberdiek.tds.zip} in the
% TDS tree (also known as \xfile{texmf} tree) of your choice.
% Example (linux):
% \begin{quote}
%   |unzip oberdiek.tds.zip -d ~/texmf|
% \end{quote}
%
% \paragraph{Script installation.}
% Check the directory \xfile{TDS:scripts/oberdiek/} for
% scripts that need further installation steps.
% Package \xpackage{attachfile2} comes with the Perl script
% \xfile{pdfatfi.pl} that should be installed in such a way
% that it can be called as \texttt{pdfatfi}.
% Example (linux):
% \begin{quote}
%   |chmod +x scripts/oberdiek/pdfatfi.pl|\\
%   |cp scripts/oberdiek/pdfatfi.pl /usr/local/bin/|
% \end{quote}
%
% \subsection{Package installation}
%
% \paragraph{Unpacking.} The \xfile{.dtx} file is a self-extracting
% \docstrip\ archive. The files are extracted by running the
% \xfile{.dtx} through \plainTeX:
% \begin{quote}
%   \verb|tex pdfcolparallel.dtx|
% \end{quote}
%
% \paragraph{TDS.} Now the different files must be moved into
% the different directories in your installation TDS tree
% (also known as \xfile{texmf} tree):
% \begin{quote}
% \def\t{^^A
% \begin{tabular}{@{}>{\ttfamily}l@{ $\rightarrow$ }>{\ttfamily}l@{}}
%   pdfcolparallel.sty & tex/latex/oberdiek/pdfcolparallel.sty\\
%   pdfcolparallel.pdf & doc/latex/oberdiek/pdfcolparallel.pdf\\
%   test/pdfcolparallel-test1.tex & doc/latex/oberdiek/test/pdfcolparallel-test1.tex\\
%   pdfcolparallel.dtx & source/latex/oberdiek/pdfcolparallel.dtx\\
% \end{tabular}^^A
% }^^A
% \sbox0{\t}^^A
% \ifdim\wd0>\linewidth
%   \begingroup
%     \advance\linewidth by\leftmargin
%     \advance\linewidth by\rightmargin
%   \edef\x{\endgroup
%     \def\noexpand\lw{\the\linewidth}^^A
%   }\x
%   \def\lwbox{^^A
%     \leavevmode
%     \hbox to \linewidth{^^A
%       \kern-\leftmargin\relax
%       \hss
%       \usebox0
%       \hss
%       \kern-\rightmargin\relax
%     }^^A
%   }^^A
%   \ifdim\wd0>\lw
%     \sbox0{\small\t}^^A
%     \ifdim\wd0>\linewidth
%       \ifdim\wd0>\lw
%         \sbox0{\footnotesize\t}^^A
%         \ifdim\wd0>\linewidth
%           \ifdim\wd0>\lw
%             \sbox0{\scriptsize\t}^^A
%             \ifdim\wd0>\linewidth
%               \ifdim\wd0>\lw
%                 \sbox0{\tiny\t}^^A
%                 \ifdim\wd0>\linewidth
%                   \lwbox
%                 \else
%                   \usebox0
%                 \fi
%               \else
%                 \lwbox
%               \fi
%             \else
%               \usebox0
%             \fi
%           \else
%             \lwbox
%           \fi
%         \else
%           \usebox0
%         \fi
%       \else
%         \lwbox
%       \fi
%     \else
%       \usebox0
%     \fi
%   \else
%     \lwbox
%   \fi
% \else
%   \usebox0
% \fi
% \end{quote}
% If you have a \xfile{docstrip.cfg} that configures and enables \docstrip's
% TDS installing feature, then some files can already be in the right
% place, see the documentation of \docstrip.
%
% \subsection{Refresh file name databases}
%
% If your \TeX~distribution
% (\teTeX, \mikTeX, \dots) relies on file name databases, you must refresh
% these. For example, \teTeX\ users run \verb|texhash| or
% \verb|mktexlsr|.
%
% \subsection{Some details for the interested}
%
% \paragraph{Attached source.}
%
% The PDF documentation on CTAN also includes the
% \xfile{.dtx} source file. It can be extracted by
% AcrobatReader 6 or higher. Another option is \textsf{pdftk},
% e.g. unpack the file into the current directory:
% \begin{quote}
%   \verb|pdftk pdfcolparallel.pdf unpack_files output .|
% \end{quote}
%
% \paragraph{Unpacking with \LaTeX.}
% The \xfile{.dtx} chooses its action depending on the format:
% \begin{description}
% \item[\plainTeX:] Run \docstrip\ and extract the files.
% \item[\LaTeX:] Generate the documentation.
% \end{description}
% If you insist on using \LaTeX\ for \docstrip\ (really,
% \docstrip\ does not need \LaTeX), then inform the autodetect routine
% about your intention:
% \begin{quote}
%   \verb|latex \let\install=y% \iffalse meta-comment
%
% File: pdfcolparallel.dtx
% Version: 2016/05/16 v1.4
% Info: Color stacks support for parallel
%
% Copyright (C) 2007, 2008, 2010 by
%    Heiko Oberdiek <heiko.oberdiek at googlemail.com>
%    2016
%    https://github.com/ho-tex/oberdiek/issues
%
% This work may be distributed and/or modified under the
% conditions of the LaTeX Project Public License, either
% version 1.3c of this license or (at your option) any later
% version. This version of this license is in
%    http://www.latex-project.org/lppl/lppl-1-3c.txt
% and the latest version of this license is in
%    http://www.latex-project.org/lppl.txt
% and version 1.3 or later is part of all distributions of
% LaTeX version 2005/12/01 or later.
%
% This work has the LPPL maintenance status "maintained".
%
% This Current Maintainer of this work is Heiko Oberdiek.
%
% This work consists of the main source file pdfcolparallel.dtx
% and the derived files
%    pdfcolparallel.sty, pdfcolparallel.pdf, pdfcolparallel.ins,
%    pdfcolparallel.drv, pdfcolparallel-test1.tex.
%
% Distribution:
%    CTAN:macros/latex/contrib/oberdiek/pdfcolparallel.dtx
%    CTAN:macros/latex/contrib/oberdiek/pdfcolparallel.pdf
%
% Unpacking:
%    (a) If pdfcolparallel.ins is present:
%           tex pdfcolparallel.ins
%    (b) Without pdfcolparallel.ins:
%           tex pdfcolparallel.dtx
%    (c) If you insist on using LaTeX
%           latex \let\install=y\input{pdfcolparallel.dtx}
%        (quote the arguments according to the demands of your shell)
%
% Documentation:
%    (a) If pdfcolparallel.drv is present:
%           latex pdfcolparallel.drv
%    (b) Without pdfcolparallel.drv:
%           latex pdfcolparallel.dtx; ...
%    The class ltxdoc loads the configuration file ltxdoc.cfg
%    if available. Here you can specify further options, e.g.
%    use A4 as paper format:
%       \PassOptionsToClass{a4paper}{article}
%
%    Programm calls to get the documentation (example):
%       pdflatex pdfcolparallel.dtx
%       makeindex -s gind.ist pdfcolparallel.idx
%       pdflatex pdfcolparallel.dtx
%       makeindex -s gind.ist pdfcolparallel.idx
%       pdflatex pdfcolparallel.dtx
%
% Installation:
%    TDS:tex/latex/oberdiek/pdfcolparallel.sty
%    TDS:doc/latex/oberdiek/pdfcolparallel.pdf
%    TDS:doc/latex/oberdiek/test/pdfcolparallel-test1.tex
%    TDS:source/latex/oberdiek/pdfcolparallel.dtx
%
%<*ignore>
\begingroup
  \catcode123=1 %
  \catcode125=2 %
  \def\x{LaTeX2e}%
\expandafter\endgroup
\ifcase 0\ifx\install y1\fi\expandafter
         \ifx\csname processbatchFile\endcsname\relax\else1\fi
         \ifx\fmtname\x\else 1\fi\relax
\else\csname fi\endcsname
%</ignore>
%<*install>
\input docstrip.tex
\Msg{************************************************************************}
\Msg{* Installation}
\Msg{* Package: pdfcolparallel 2016/05/16 v1.4 Color stacks support for parallel (HO)}
\Msg{************************************************************************}

\keepsilent
\askforoverwritefalse

\let\MetaPrefix\relax
\preamble

This is a generated file.

Project: pdfcolparallel
Version: 2016/05/16 v1.4

Copyright (C) 2007, 2008, 2010 by
   Heiko Oberdiek <heiko.oberdiek at googlemail.com>

This work may be distributed and/or modified under the
conditions of the LaTeX Project Public License, either
version 1.3c of this license or (at your option) any later
version. This version of this license is in
   http://www.latex-project.org/lppl/lppl-1-3c.txt
and the latest version of this license is in
   http://www.latex-project.org/lppl.txt
and version 1.3 or later is part of all distributions of
LaTeX version 2005/12/01 or later.

This work has the LPPL maintenance status "maintained".

This Current Maintainer of this work is Heiko Oberdiek.

This work consists of the main source file pdfcolparallel.dtx
and the derived files
   pdfcolparallel.sty, pdfcolparallel.pdf, pdfcolparallel.ins,
   pdfcolparallel.drv, pdfcolparallel-test1.tex.

\endpreamble
\let\MetaPrefix\DoubleperCent

\generate{%
  \file{pdfcolparallel.ins}{\from{pdfcolparallel.dtx}{install}}%
  \file{pdfcolparallel.drv}{\from{pdfcolparallel.dtx}{driver}}%
  \usedir{tex/latex/oberdiek}%
  \file{pdfcolparallel.sty}{\from{pdfcolparallel.dtx}{package}}%
  \usedir{doc/latex/oberdiek/test}%
  \file{pdfcolparallel-test1.tex}{\from{pdfcolparallel.dtx}{test1}}%
  \nopreamble
  \nopostamble
  \usedir{source/latex/oberdiek/catalogue}%
  \file{pdfcolparallel.xml}{\from{pdfcolparallel.dtx}{catalogue}}%
}

\catcode32=13\relax% active space
\let =\space%
\Msg{************************************************************************}
\Msg{*}
\Msg{* To finish the installation you have to move the following}
\Msg{* file into a directory searched by TeX:}
\Msg{*}
\Msg{*     pdfcolparallel.sty}
\Msg{*}
\Msg{* To produce the documentation run the file `pdfcolparallel.drv'}
\Msg{* through LaTeX.}
\Msg{*}
\Msg{* Happy TeXing!}
\Msg{*}
\Msg{************************************************************************}

\endbatchfile
%</install>
%<*ignore>
\fi
%</ignore>
%<*driver>
\NeedsTeXFormat{LaTeX2e}
\ProvidesFile{pdfcolparallel.drv}%
  [2016/05/16 v1.4 Color stacks support for parallel (HO)]%
\documentclass{ltxdoc}
\usepackage{holtxdoc}[2011/11/22]
\begin{document}
  \DocInput{pdfcolparallel.dtx}%
\end{document}
%</driver>
% \fi
%
%
% \CharacterTable
%  {Upper-case    \A\B\C\D\E\F\G\H\I\J\K\L\M\N\O\P\Q\R\S\T\U\V\W\X\Y\Z
%   Lower-case    \a\b\c\d\e\f\g\h\i\j\k\l\m\n\o\p\q\r\s\t\u\v\w\x\y\z
%   Digits        \0\1\2\3\4\5\6\7\8\9
%   Exclamation   \!     Double quote  \"     Hash (number) \#
%   Dollar        \$     Percent       \%     Ampersand     \&
%   Acute accent  \'     Left paren    \(     Right paren   \)
%   Asterisk      \*     Plus          \+     Comma         \,
%   Minus         \-     Point         \.     Solidus       \/
%   Colon         \:     Semicolon     \;     Less than     \<
%   Equals        \=     Greater than  \>     Question mark \?
%   Commercial at \@     Left bracket  \[     Backslash     \\
%   Right bracket \]     Circumflex    \^     Underscore    \_
%   Grave accent  \`     Left brace    \{     Vertical bar  \|
%   Right brace   \}     Tilde         \~}
%
% \GetFileInfo{pdfcolparallel.drv}
%
% \title{The \xpackage{pdfcolparallel} package}
% \date{2016/05/16 v1.4}
% \author{Heiko Oberdiek\thanks
% {Please report any issues at https://github.com/ho-tex/oberdiek/issues}\\
% \xemail{heiko.oberdiek at googlemail.com}}
%
% \maketitle
%
% \begin{abstract}
% This packages fixes bugs in \xpackage{parallel} and
% improves color support by using several color stacks
% that are provided by \pdfTeX\ since version 1.40.
% \end{abstract}
%
% \tableofcontents
%
% \section{Usage}
%
% \begin{quote}
% |\usepackage{pdfcolparallel}|
% \end{quote}
% The package \xpackage{pdfcolparallel} loads package \xpackage{parallel}
% \cite{parallel} and redefines some macros to fix bugs.
%
% If color stacks are available then package \xpackage{parallel}
% is further patched to support them.
%
% \subsection{Option \xoption{rulebetweencolor}}
%
% Package \xpackage{pdfcolparallel} also fixes the color for the
% rule between columns.
% Default color is \cs{normalcolor}. But this can be changed by using
% option \xoption{rulebetweencolor} for |\setkeys{parallel}|
% (see package \xpackage{keyval}). The option takes a color specification
% as value. If the value is empty, then the default (\cs{normalcolor})
% is used.
% Examples:
% \begin{quote}
%   |\setkeys{parallel}{rulebetweencolor=blue}|,\\
%   |\setkeys{parallel}{rulebetweencolor={red}}|,\\
%   |\setkeys{parallel}{rulebetweencolor={}}|,
%     \textit{\% \cs{normalcolor} is used}\\
%   |\setkeys{parallel}{rulebetweencolor=[rgb]{1,0,.5}}|
% \end{quote}
%
% \subsection{Future}
%
% If there will be a new version of package \xpackage{parallel}
% that adds support for color stacks, then this package may become
% obsolete.
%
% \StopEventually{
% }
%
% \section{Implementation}
%
% \subsection{Identification}
%
%    \begin{macrocode}
%<*package>
\NeedsTeXFormat{LaTeX2e}
\ProvidesPackage{pdfcolparallel}%
  [2016/05/16 v1.4 Color stacks support for parallel (HO)]%
%    \end{macrocode}
%
% \subsection{Load and fix package \xpackage{parallel}}
%
%    Package \xpackage{parallel} is loaded. Before options of package
%    \xpackage{pdfcolparallel} are passed to package \xpackage{parallel}.
%    \begin{macrocode}
\DeclareOption*{%
  \PassoptionsToPackage{\CurrentOption}{parallel}%
}
\ProcessOptions\relax
\RequirePackage{parallel}[2003/04/13]
%    \end{macrocode}
%
%    \begin{macrocode}
\RequirePackage{infwarerr}[2007/09/09]
%    \end{macrocode}
%
%    \begin{macro}{\pcp@ColorPatch}
%    \begin{macrocode}
\begingroup\expandafter\expandafter\expandafter\endgroup
\expandafter\ifx\csname currentgrouplevel\endcsname\relax
  \def\pcp@ColorPatch{}%
\else
  \def\pcp@ColorPatch{%
    \@ifundefined{set@color}{%
      \gdef\pcp@ColorPatch{}%
    }{%
      \gdef\pcp@ColorPatch{%
        \gdef\pcp@ColorResets{}%
        \bgroup
        \aftergroup\pcp@ColorResets
        \aftergroup\egroup
        \let\pcp@OrgSetColor\set@color
        \let\set@color\pcp@SetColor
        \edef\pcp@GroupLevel{\the\currentgrouplevel}%
      }%
    }%
    \pcp@ColorPatch
  }%
%    \end{macrocode}
%    \end{macro}
%    \begin{macro}{\pcp@SetColor}
%    \begin{macrocode}
  \def\pcp@SetColor{%
    \ifnum\pcp@GroupLevel=\currentgrouplevel
      \let\pcp@OrgAfterGroup\aftergroup
      \def\aftergroup{%
        \g@addto@macro\pcp@ColorResets
      }%
      \pcp@OrgSetColor
      \let\aftergroup\pcp@OrgAfterGroup
    \else
      \pcp@OrgSetColor
    \fi
  }%
\fi
%    \end{macrocode}
%    \end{macro}
%
%    \begin{macro}{\pcp@CmdCheckRedef}
%    \begin{macrocode}
\def\pcp@CmdCheckRedef#1{%
  \begingroup
    \def\pcp@cmd{#1}%
    \afterassignment\pcp@CmdDo
    \long\def\reserved@a
}
\def\pcp@CmdDo{%
    \expandafter\ifx\pcp@cmd\reserved@a
    \else
      \edef\x*{\expandafter\string\pcp@cmd}%
      \@PackageWarningNoLine{pdfcolparallel}{%
        Command \x* has changed.\MessageBreak
        Supported versions of package `parallel':\MessageBreak
        \space\space 2003/04/13\MessageBreak
        The redefinition of \x* may\MessageBreak
        not behave correctly depending on the changes%
      }%
    \fi
  \expandafter\endgroup
  \expandafter\def\pcp@cmd
}
%    \end{macrocode}
%    \end{macro}
%
%    \begin{macrocode}
\def\pcp@SwitchStack#1#2{}
%    \end{macrocode}
%    \begin{macrocode}
\def\pcp@SetCurrent#1{}
%    \end{macrocode}
%
%    \begin{macro}{\ParallelLText}
%    \begin{macrocode}
\pcp@CmdCheckRedef\ParallelLText{%
  \everypar{}%
  \@restorepar
  \begingroup
    \hbadness=3000 %
    \let\footnote=\ParallelLFootnote
    \ParallelWhichBox=0 %
    \global\setbox\ParallelLBox=\vbox\bgroup
      \hsize=\ParallelLWidth
      \aftergroup\ParallelAfterText
      \begingroup
        \afterassignment\ParallelCheckOpenBrace
        \let\x=%
}{%
  \everypar{}%
  \@restorepar
  \@nobreakfalse
  \begingroup
    \hbadness=3000 %
    \let\footnote=\ParallelLFootnote
    \ParallelWhichBox=0 %
    \global\setbox\ParallelLBox=\vbox\bgroup
      \hsize=\ParallelLWidth
      \linewidth=\ParallelLWidth
      \pcp@SwitchStack{Left}\ParallelLBox
      \aftergroup\ParallelAfterText
      \pcp@ColorPatch
      \begingroup
        \afterassignment\ParallelCheckOpenBrace
        \let\x=%
}
%    \end{macrocode}
%    \end{macro}
%
%    \begin{macro}{\ParallelRText}
%    \begin{macrocode}
\pcp@CmdCheckRedef\ParallelRText{%
  \everypar{}%
  \@restorepar
  \begingroup
    \hbadness=3000 %
    \ifnum\ParallelFNMode=\@ne
      \let\footnote=\ParallelRFootnote
    \else
      \let\footnote=\ParallelLFootnote
    \fi
    \ParallelWhichBox=\@ne
    \global\setbox\ParallelRBox=\vbox\bgroup
      \hsize=\ParallelRWidth
      \aftergroup\ParallelAfterText
      \begingroup
        \afterassignment\ParallelCheckOpenBrace
        \let\x=%
}{%
  \everypar{}%
  \@restorepar
  \@nobreakfalse
  \begingroup
    \hbadness=3000 %
    \ifnum\ParallelFNMode=\@ne
      \let\footnote=\ParallelRFootnote
    \else
      \let\footnote=\ParallelLFootnote
    \fi
    \ParallelWhichBox=\@ne
    \global\setbox\ParallelRBox=\vbox\bgroup
      \hsize=\ParallelRWidth
      \linewidth=\ParallelRWidth
      \pcp@SwitchStack{Right}\ParallelRBox
      \aftergroup\ParallelAfterText
      \pcp@ColorPatch
      \begingroup
        \afterassignment\ParallelCheckOpenBrace
        \let\x=%
}
%    \end{macrocode}
%    \end{macro}
%
%    \begin{macro}{\ParallelParTwoPages}
%    \begin{macrocode}
\pcp@CmdCheckRedef\ParallelParTwoPages{%
  \ifnum\ParallelBoolVar=\@ne
    \par
    \begingroup
      \global\ParallelWhichBox=\@ne
      \newpage
      \vbadness=10000 %
      \vfuzz=3ex %
      \splittopskip=\z@skip
      \loop%
        \ifnum\ParallelBoolVar=\@ne%
          \ifnum\ParallelWhichBox=\@ne
            \ifvoid\ParallelLBox
              \mbox{} %
              \newpage
            \else
              \global\ParallelWhichBox=\z@
            \fi
          \else
            \ifvoid\ParallelRBox
              \mbox{} %
              \newpage
            \else
              \global\ParallelWhichBox=\@ne
            \fi
          \fi
          \ifnum\ParallelWhichBox=\z@
            \ifodd\thepage
              \mbox{} %
              \newpage
            \fi
            \hbox to\textwidth{%
              \vbox{\vsplit\ParallelLBox to.98\textheight}%
            }%
          \else
            \ifodd\thepage\relax
            \else
              \mbox{} %
              \newpage
            \fi
            \hbox to\textwidth{%
              \vbox{\vsplit\ParallelRBox to.98\textheight}%
            }%
          \fi
          \vspace*{\fill}%
          \newpage
        \fi
        \ifvoid\ParallelLBox
          \ifvoid\ParallelRBox
            \global\ParallelBoolVar=\z@
          \fi
        \fi
      \ifnum\ParallelBoolVar=\@ne
      \repeat
      \par
    \endgroup
  \fi
}{%
%    \end{macrocode}
%    Additional fixes:
%    \begin{itemize}
%    \item Unnecessary white space removed.
%    \item |\ifodd\thepage| changed to |\ifodd\value{page}|.
%    \end{itemize}
%    \begin{macrocode}
  \ifnum\ParallelBoolVar=\@ne
    \par
    \begingroup
      \global\ParallelWhichBox=\@ne
      \newpage
      \vbadness=10000 %
      \vfuzz=3ex %
      \splittopskip=\z@skip
      \loop%
        \ifnum\ParallelBoolVar=\@ne%
          \ifnum\ParallelWhichBox=\@ne
            \ifvoid\ParallelLBox
              \mbox{}%
              \newpage
            \else
              \global\ParallelWhichBox=\z@
            \fi
          \else
            \ifvoid\ParallelRBox
              \null
              \newpage
            \else
              \global\ParallelWhichBox=\@ne
            \fi
          \fi
          \ifnum\ParallelWhichBox=\z@
            \ifodd\value{page}%
              \null
              \newpage
            \fi
            \hbox to\textwidth{%
              \pcp@SetCurrent{Left}%
              \setbox\z@=\vsplit\ParallelLBox to.98\textheight
              \vbox to.98\textheight{%
                \@texttop
                \unvbox\z@
                \@textbottom
              }%
            }%
          \else
            \ifodd\value{page}%
            \else
              \mbox{}%
              \newpage
            \fi
            \hbox to\textwidth{%
              \pcp@SetCurrent{Right}%
              \setbox\z@=\vsplit\ParallelRBox to.98\textheight
              \vbox to.98\textheight{%
                \@texttop
                \unvbox\z@
                \@textbottom
              }%
            }%
          \fi
          \vspace*{\fill}%
          \newpage
        \fi
        \ifvoid\ParallelLBox
          \ifvoid\ParallelRBox
            \global\ParallelBoolVar=\z@
          \fi
        \fi
      \ifnum\ParallelBoolVar=\@ne
      \repeat
      \par
    \endgroup
    \pcp@SetCurrent{}%
  \fi
}
%    \end{macrocode}
%    \end{macro}
%
% \subsection{Color stack support}
%
%    \begin{macrocode}
\RequirePackage{pdfcol}[2007/12/12]
\ifpdfcolAvailable
\else
  \PackageInfo{pdfcolparallel}{%
    Loading aborted, because color stacks are not available%
  }%
  \expandafter\endinput
\fi
%    \end{macrocode}
%
%    \begin{macrocode}
\pdfcolInitStack{pcp@Left}
\pdfcolInitStack{pcp@Right}
%    \end{macrocode}
%    \begin{macro}{\pcp@Box}
%    \begin{macrocode}
\newbox\pcp@Box
%    \end{macrocode}
%    \end{macro}
%    \begin{macro}{\pcp@SwitchStack}
%    \begin{macrocode}
\def\pcp@SwitchStack#1#2{%
  \pdfcolSwitchStack{pcp@#1}%
  \global\setbox\pcp@Box=\vbox to 0pt{%
    \pdfcolSetCurrentColor
  }%
  \aftergroup\pcp@FixBox
  \aftergroup#2%
}
%    \end{macrocode}
%    \end{macro}
%    \begin{macro}{\pcp@FixBox}
%    \begin{macrocode}
\def\pcp@FixBox#1{%
  \global\setbox#1=\vbox{%
    \unvbox\pcp@Box
    \unvbox#1%
  }%
}
%    \end{macrocode}
%    \end{macro}
%    \begin{macro}{\pcp@SetCurrent}
%    \begin{macrocode}
\def\pcp@SetCurrent#1{%
  \ifx\\#1\\%
    \pdfcolSetCurrent{}%
  \else
    \pdfcolSetCurrent{pcp@#1}%
  \fi
}
%    \end{macrocode}
%    \end{macro}
%
% \subsection{Redefinitions}
%
%    \begin{macro}{\ParallelParOnePage}
%    \begin{macrocode}
\pcp@CmdCheckRedef\ParallelParOnePage{%
  \ifnum\ParallelBoolVar=\@ne
    \par
    \begingroup
      \leftmargin=\z@
      \rightmargin=\z@
      \parskip=\z@skip
      \parindent=\z@
      \vbadness=10000 %
      \vfuzz=3ex %
      \splittopskip=\z@skip
      \loop
        \ifnum\ParallelBoolVar=\@ne
          \noindent
          \hbox to\textwidth{%
            \hskip\ParallelLeftMargin
            \hbox to\ParallelTextWidth{%
              \ifvoid\ParallelLBox
                \hskip\ParallelLWidth
              \else
                \ParallelWhichBox=\z@
                \vbox{%
                  \setbox\ParallelBoxVar
                      =\vsplit\ParallelLBox to\dp\strutbox
                  \unvbox\ParallelBoxVar
                }%
              \fi
              \strut
              \ifnum\ParallelBoolMid=\@ne
                \hskip\ParallelMainMidSkip
                \vrule
              \else
                \hss
              \fi
              \hss
              \ifvoid\ParallelRBox
                \hskip\ParallelRWidth
              \else
                \ParallelWhichBox=\@ne
                \vbox{%
                  \setbox\ParallelBoxVar
                      =\vsplit\ParallelRBox to\dp\strutbox
                  \unvbox\ParallelBoxVar
                }%
              \fi
            }%
          }%
          \ifvoid\ParallelLBox
            \ifvoid\ParallelRBox
              \global\ParallelBoolVar=\z@
            \fi
          \fi%
        \fi%
      \ifnum\ParallelBoolVar=\@ne
        \penalty\interlinepenalty
      \repeat
      \par
    \endgroup
  \fi
}{%
  \ifnum\ParallelBoolVar=\@ne
    \par
    \begingroup
      \leftmargin=\z@
      \rightmargin=\z@
      \parskip=\z@skip
      \parindent=\z@
      \vbadness=10000 %
      \vfuzz=3ex %
      \splittopskip=\z@skip
      \loop
        \ifnum\ParallelBoolVar=\@ne
          \noindent
          \hbox to\textwidth{%
            \hskip\ParallelLeftMargin
            \hbox to\ParallelTextWidth{%
              \ifvoid\ParallelLBox
                \hskip\ParallelLWidth
              \else
                \pcp@SetCurrent{Left}%
                \ParallelWhichBox=\z@
                \vbox{%
                  \setbox\ParallelBoxVar
                      =\vsplit\ParallelLBox to\dp\strutbox
                  \unvbox\ParallelBoxVar
                }%
              \fi
              \strut
              \ifnum\ParallelBoolMid=\@ne
                \hskip\ParallelMainMidSkip
                \begingroup
                  \pcp@RuleBetweenColor
                  \vrule
                \endgroup
              \else
                \hss
              \fi
              \hss
              \ifvoid\ParallelRBox
                \hskip\ParallelRWidth
              \else
                \pcp@SetCurrent{Right}%
                \ParallelWhichBox=\@ne
                \vbox{%
                  \setbox\ParallelBoxVar
                      =\vsplit\ParallelRBox to\dp\strutbox
                  \unvbox\ParallelBoxVar
                }%
              \fi
            }%
          }%
          \ifvoid\ParallelLBox
            \ifvoid\ParallelRBox
              \global\ParallelBoolVar=\z@
            \fi
          \fi%
        \fi%
      \ifnum\ParallelBoolVar=\@ne
        \penalty\interlinepenalty
      \repeat
      \par
    \endgroup
    \pcp@SetCurrent{}%
  \fi
}
%    \end{macrocode}
%    \end{macro}
%    \begin{macro}{\pcp@RuleBetweenColorDefault}
%    \begin{macrocode}
\def\pcp@RuleBetweenColorDefault{%
  \normalcolor
}
%    \end{macrocode}
%    \end{macro}
%    \begin{macro}{\pcp@RuleBetweenColor}
%    \begin{macrocode}
\let\pcp@RuleBetweenColor\pcp@RuleBetweenColorDefault
%    \end{macrocode}
%    \end{macro}
%    \begin{macrocode}
\RequirePackage{keyval}
\define@key{parallel}{rulebetweencolor}{%
  \edef\pcp@temp{#1}%
  \ifx\pcp@temp\@empty
    \let\pcp@RuleBetweenColor\pcp@RuleBetweenColorDefault
  \else
    \edef\pcp@temp{%
      \noexpand\@ifnextchar[{%
        \def\noexpand\pcp@RuleBetweenColor{%
          \noexpand\color\pcp@temp
        }%
        \noexpand\pcp@GobbleNil
      }{%
        \def\noexpand\pcp@RuleBetweenColor{%
          \noexpand\color{\pcp@temp}%
        }%
        \noexpand\pcp@GobbleNil
      }%
      \pcp@temp\noexpand\@nil
    }%
    \pcp@temp
  \fi
}
%    \end{macrocode}
%    \begin{macro}{\pcp@GobbleNil}
%    \begin{macrocode}
\long\def\pcp@GobbleNil#1\@nil{}
%    \end{macrocode}
%    \end{macro}
%
%    \begin{macrocode}
%</package>
%    \end{macrocode}
%
% \section{Test}
%
%    The test file is a modified version of the file that
%    Alexander Hirsch has posted in \xnewsgroup{de.comp.text.tex}:
%    \URL{``\link{\texttt{parallel.sty} und farbiger Text}''}^^A
%    {http://groups.google.com/group/de.comp.text.tex/msg/6a759cf33bb071a5}
%    \begin{macrocode}
%<*test1>
\AtEndDocument{%
  \typeout{}%
  \typeout{**************************************}%
  \typeout{*** \space Check the PDF file manually! \space ***}%
  \typeout{**************************************}%
  \typeout{}%
}
\documentclass{article}
\usepackage{xcolor}
\usepackage{pdfcolparallel}[2016/05/16]

\begin{document}
  \color{green}%
  Green%
  \begin{Parallel}{0.47\textwidth}{0.47\textwidth}%
    \ParallelLText{%
      \textcolor{red}{%
        Ein Absatz, der sich ueber zwei Zeilen erstrecken soll. %
        Ein Absatz, der sich ueber zwei Zeilen erstrecken soll.%
      }%
    }%
    \ParallelRText{%
      \textcolor{blue}{%
        Ein Absatz, der sich ueber zwei Zeilen erstrecken soll. %
        Ein Absatz, der sich ueber zwei Zeilen erstrecken soll.%
      }%
    }%
    \ParallelPar
    \ParallelLText{%
      Default %
      \color{red}%
      Ein Absatz, der sich ueber zwei Zeilen erstrecken soll. %
      Ein Absatz, der sich ueber zwei Zeilen erstrecken soll.%
    }%
    \ParallelRText{%
      Default %
      \color{blue}%
      Ein Absatz, der sich ueber zwei Zeilen erstrecken soll. %
      Ein Absatz, der sich ueber zwei Zeilen erstrecken soll.%
    }%
    \ParallelPar
    \ParallelLText{%
      \begin{enumerate}%
      \item left text, left text, left text, left text, %
            left text, left text, left text, left text,%
      \item left text, left text, left text, left text, %
            left text, left text, left text, left text.%
      \end{enumerate}%
    }%
    \ParallelRText{%
      \begin{enumerate}%
      \item right text, right text, right text, right text, %
            right text, right text, right text, right text.%
      \item right text, right text, right text, right text, %
            right text, right text, right text, right text.%
      \end{enumerate}%
    }%
  \end{Parallel}%
  \begin{Parallel}[p]{\textwidth}{\textwidth}%
    \ParallelLText{%
      \textcolor{red}{%
        Ein Absatz, der sich ueber zwei Zeilen erstrecken soll. %
        Ein Absatz, der sich ueber zwei Zeilen erstrecken soll. %
        Foo bar bla bla bla.%
      }%
      \par
      Und noch ein Absatz.%
    }%
    \ParallelRText{%
      \textcolor{blue}{%
        Ein Absatz, der sich ueber zwei Zeilen erstrecken soll. %
        Ein Absatz, der sich ueber zwei Zeilen erstrecken soll. %
        Foo bar bla bla bla.%
      }%
    }%
  \end{Parallel}%
  \begin{Parallel}[p]{\textwidth}{\textwidth}%
    \ParallelLText{%
      \rule{1pt}{.98\textheight}\Huge g%
    }%
    \ParallelRText{%
      \rule{1pt}{.98\textheight}y%
    }%
  \end{Parallel}%
  Green%
\end{document}
%</test1>
%    \end{macrocode}
%
% \section{Installation}
%
% \subsection{Download}
%
% \paragraph{Package.} This package is available on
% CTAN\footnote{\url{http://ctan.org/pkg/pdfcolparallel}}:
% \begin{description}
% \item[\CTAN{macros/latex/contrib/oberdiek/pdfcolparallel.dtx}] The source file.
% \item[\CTAN{macros/latex/contrib/oberdiek/pdfcolparallel.pdf}] Documentation.
% \end{description}
%
%
% \paragraph{Bundle.} All the packages of the bundle `oberdiek'
% are also available in a TDS compliant ZIP archive. There
% the packages are already unpacked and the documentation files
% are generated. The files and directories obey the TDS standard.
% \begin{description}
% \item[\CTAN{install/macros/latex/contrib/oberdiek.tds.zip}]
% \end{description}
% \emph{TDS} refers to the standard ``A Directory Structure
% for \TeX\ Files'' (\CTAN{tds/tds.pdf}). Directories
% with \xfile{texmf} in their name are usually organized this way.
%
% \subsection{Bundle installation}
%
% \paragraph{Unpacking.} Unpack the \xfile{oberdiek.tds.zip} in the
% TDS tree (also known as \xfile{texmf} tree) of your choice.
% Example (linux):
% \begin{quote}
%   |unzip oberdiek.tds.zip -d ~/texmf|
% \end{quote}
%
% \paragraph{Script installation.}
% Check the directory \xfile{TDS:scripts/oberdiek/} for
% scripts that need further installation steps.
% Package \xpackage{attachfile2} comes with the Perl script
% \xfile{pdfatfi.pl} that should be installed in such a way
% that it can be called as \texttt{pdfatfi}.
% Example (linux):
% \begin{quote}
%   |chmod +x scripts/oberdiek/pdfatfi.pl|\\
%   |cp scripts/oberdiek/pdfatfi.pl /usr/local/bin/|
% \end{quote}
%
% \subsection{Package installation}
%
% \paragraph{Unpacking.} The \xfile{.dtx} file is a self-extracting
% \docstrip\ archive. The files are extracted by running the
% \xfile{.dtx} through \plainTeX:
% \begin{quote}
%   \verb|tex pdfcolparallel.dtx|
% \end{quote}
%
% \paragraph{TDS.} Now the different files must be moved into
% the different directories in your installation TDS tree
% (also known as \xfile{texmf} tree):
% \begin{quote}
% \def\t{^^A
% \begin{tabular}{@{}>{\ttfamily}l@{ $\rightarrow$ }>{\ttfamily}l@{}}
%   pdfcolparallel.sty & tex/latex/oberdiek/pdfcolparallel.sty\\
%   pdfcolparallel.pdf & doc/latex/oberdiek/pdfcolparallel.pdf\\
%   test/pdfcolparallel-test1.tex & doc/latex/oberdiek/test/pdfcolparallel-test1.tex\\
%   pdfcolparallel.dtx & source/latex/oberdiek/pdfcolparallel.dtx\\
% \end{tabular}^^A
% }^^A
% \sbox0{\t}^^A
% \ifdim\wd0>\linewidth
%   \begingroup
%     \advance\linewidth by\leftmargin
%     \advance\linewidth by\rightmargin
%   \edef\x{\endgroup
%     \def\noexpand\lw{\the\linewidth}^^A
%   }\x
%   \def\lwbox{^^A
%     \leavevmode
%     \hbox to \linewidth{^^A
%       \kern-\leftmargin\relax
%       \hss
%       \usebox0
%       \hss
%       \kern-\rightmargin\relax
%     }^^A
%   }^^A
%   \ifdim\wd0>\lw
%     \sbox0{\small\t}^^A
%     \ifdim\wd0>\linewidth
%       \ifdim\wd0>\lw
%         \sbox0{\footnotesize\t}^^A
%         \ifdim\wd0>\linewidth
%           \ifdim\wd0>\lw
%             \sbox0{\scriptsize\t}^^A
%             \ifdim\wd0>\linewidth
%               \ifdim\wd0>\lw
%                 \sbox0{\tiny\t}^^A
%                 \ifdim\wd0>\linewidth
%                   \lwbox
%                 \else
%                   \usebox0
%                 \fi
%               \else
%                 \lwbox
%               \fi
%             \else
%               \usebox0
%             \fi
%           \else
%             \lwbox
%           \fi
%         \else
%           \usebox0
%         \fi
%       \else
%         \lwbox
%       \fi
%     \else
%       \usebox0
%     \fi
%   \else
%     \lwbox
%   \fi
% \else
%   \usebox0
% \fi
% \end{quote}
% If you have a \xfile{docstrip.cfg} that configures and enables \docstrip's
% TDS installing feature, then some files can already be in the right
% place, see the documentation of \docstrip.
%
% \subsection{Refresh file name databases}
%
% If your \TeX~distribution
% (\teTeX, \mikTeX, \dots) relies on file name databases, you must refresh
% these. For example, \teTeX\ users run \verb|texhash| or
% \verb|mktexlsr|.
%
% \subsection{Some details for the interested}
%
% \paragraph{Attached source.}
%
% The PDF documentation on CTAN also includes the
% \xfile{.dtx} source file. It can be extracted by
% AcrobatReader 6 or higher. Another option is \textsf{pdftk},
% e.g. unpack the file into the current directory:
% \begin{quote}
%   \verb|pdftk pdfcolparallel.pdf unpack_files output .|
% \end{quote}
%
% \paragraph{Unpacking with \LaTeX.}
% The \xfile{.dtx} chooses its action depending on the format:
% \begin{description}
% \item[\plainTeX:] Run \docstrip\ and extract the files.
% \item[\LaTeX:] Generate the documentation.
% \end{description}
% If you insist on using \LaTeX\ for \docstrip\ (really,
% \docstrip\ does not need \LaTeX), then inform the autodetect routine
% about your intention:
% \begin{quote}
%   \verb|latex \let\install=y\input{pdfcolparallel.dtx}|
% \end{quote}
% Do not forget to quote the argument according to the demands
% of your shell.
%
% \paragraph{Generating the documentation.}
% You can use both the \xfile{.dtx} or the \xfile{.drv} to generate
% the documentation. The process can be configured by the
% configuration file \xfile{ltxdoc.cfg}. For instance, put this
% line into this file, if you want to have A4 as paper format:
% \begin{quote}
%   \verb|\PassOptionsToClass{a4paper}{article}|
% \end{quote}
% An example follows how to generate the
% documentation with pdf\LaTeX:
% \begin{quote}
%\begin{verbatim}
%pdflatex pdfcolparallel.dtx
%makeindex -s gind.ist pdfcolparallel.idx
%pdflatex pdfcolparallel.dtx
%makeindex -s gind.ist pdfcolparallel.idx
%pdflatex pdfcolparallel.dtx
%\end{verbatim}
% \end{quote}
%
% \section{Catalogue}
%
% The following XML file can be used as source for the
% \href{http://mirror.ctan.org/help/Catalogue/catalogue.html}{\TeX\ Catalogue}.
% The elements \texttt{caption} and \texttt{description} are imported
% from the original XML file from the Catalogue.
% The name of the XML file in the Catalogue is \xfile{pdfcolparallel.xml}.
%    \begin{macrocode}
%<*catalogue>
<?xml version='1.0' encoding='us-ascii'?>
<!DOCTYPE entry SYSTEM 'catalogue.dtd'>
<entry datestamp='$Date$' modifier='$Author$' id='pdfcolparallel'>
  <name>pdfcolparallel</name>
  <caption>Fix colour problems in package 'parallel'.</caption>
  <authorref id='auth:oberdiek'/>
  <copyright owner='Heiko Oberdiek' year='2007,2008,2010'/>
  <license type='lppl1.3'/>
  <version number='1.4'/>
  <description>
    Since version 1.40 pdfTeX supports colour stacks.
    This package uses them to fix colour problems in
    package <xref refid='parallel'>parallel</xref>.
    <p/>
    The package is part of the <xref refid='oberdiek'>oberdiek</xref>
    bundle.
  </description>
  <documentation details='Package documentation'
      href='ctan:/macros/latex/contrib/oberdiek/pdfcolparallel.pdf'/>
  <ctan file='true' path='/macros/latex/contrib/oberdiek/pdfcolparallel.dtx'/>
  <miktex location='oberdiek'/>
  <texlive location='oberdiek'/>
  <install path='/macros/latex/contrib/oberdiek/oberdiek.tds.zip'/>
</entry>
%</catalogue>
%    \end{macrocode}
%
% \begin{thebibliography}{9}
%
% \bibitem{parallel}
%   Matthias Eckermann: \textit{The \xpackage{parallel}-package};
%   2003/04/13;\\
%   \CTAN{macros/latex/contrib/parallel/}.
%
% \bibitem{pdfcol}
%   Heiko Oberdiek: \textit{The \xpackage{pdfcol} package};
%   2007/09/09;\\
%   \CTAN{macros/latex/contrib/oberdiek/pdfcol.pdf}.
%
% \end{thebibliography}
%
% \begin{History}
%   \begin{Version}{2007/09/09 v1.0}
%   \item
%     First version.
%   \end{Version}
%   \begin{Version}{2007/12/12 v1.1}
%   \item
%     Adds patch for setting \cs{linewidth} to fix bug
%     in package \xpackage{parallel}.
%   \item
%     Package \xpackage{parallel} is also fixed if color
%     stacks are not available.
%   \item
%     Bug fix, switched stacks now initialized with current color.
%   \item
%     Fix for package \xpackage{parallel}: \cs{raggedbottom} is respected.
%   \end{Version}
%   \begin{Version}{2008/08/11 v1.2}
%   \item
%     Code is not changed.
%   \item
%     URLs updated.
%   \end{Version}
%   \begin{Version}{2010/01/11 v1.3}
%   \item
%     Option `rulebetweencolor' added.
%   \end{Version}
%   \begin{Version}{2016/05/16 v1.4}
%   \item
%     Documentation updates.
%   \end{Version}
% \end{History}
%
% \PrintIndex
%
% \Finale
\endinput
|
% \end{quote}
% Do not forget to quote the argument according to the demands
% of your shell.
%
% \paragraph{Generating the documentation.}
% You can use both the \xfile{.dtx} or the \xfile{.drv} to generate
% the documentation. The process can be configured by the
% configuration file \xfile{ltxdoc.cfg}. For instance, put this
% line into this file, if you want to have A4 as paper format:
% \begin{quote}
%   \verb|\PassOptionsToClass{a4paper}{article}|
% \end{quote}
% An example follows how to generate the
% documentation with pdf\LaTeX:
% \begin{quote}
%\begin{verbatim}
%pdflatex pdfcolparallel.dtx
%makeindex -s gind.ist pdfcolparallel.idx
%pdflatex pdfcolparallel.dtx
%makeindex -s gind.ist pdfcolparallel.idx
%pdflatex pdfcolparallel.dtx
%\end{verbatim}
% \end{quote}
%
% \section{Catalogue}
%
% The following XML file can be used as source for the
% \href{http://mirror.ctan.org/help/Catalogue/catalogue.html}{\TeX\ Catalogue}.
% The elements \texttt{caption} and \texttt{description} are imported
% from the original XML file from the Catalogue.
% The name of the XML file in the Catalogue is \xfile{pdfcolparallel.xml}.
%    \begin{macrocode}
%<*catalogue>
<?xml version='1.0' encoding='us-ascii'?>
<!DOCTYPE entry SYSTEM 'catalogue.dtd'>
<entry datestamp='$Date$' modifier='$Author$' id='pdfcolparallel'>
  <name>pdfcolparallel</name>
  <caption>Fix colour problems in package 'parallel'.</caption>
  <authorref id='auth:oberdiek'/>
  <copyright owner='Heiko Oberdiek' year='2007,2008,2010'/>
  <license type='lppl1.3'/>
  <version number='1.4'/>
  <description>
    Since version 1.40 pdfTeX supports colour stacks.
    This package uses them to fix colour problems in
    package <xref refid='parallel'>parallel</xref>.
    <p/>
    The package is part of the <xref refid='oberdiek'>oberdiek</xref>
    bundle.
  </description>
  <documentation details='Package documentation'
      href='ctan:/macros/latex/contrib/oberdiek/pdfcolparallel.pdf'/>
  <ctan file='true' path='/macros/latex/contrib/oberdiek/pdfcolparallel.dtx'/>
  <miktex location='oberdiek'/>
  <texlive location='oberdiek'/>
  <install path='/macros/latex/contrib/oberdiek/oberdiek.tds.zip'/>
</entry>
%</catalogue>
%    \end{macrocode}
%
% \begin{thebibliography}{9}
%
% \bibitem{parallel}
%   Matthias Eckermann: \textit{The \xpackage{parallel}-package};
%   2003/04/13;\\
%   \CTAN{macros/latex/contrib/parallel/}.
%
% \bibitem{pdfcol}
%   Heiko Oberdiek: \textit{The \xpackage{pdfcol} package};
%   2007/09/09;\\
%   \CTAN{macros/latex/contrib/oberdiek/pdfcol.pdf}.
%
% \end{thebibliography}
%
% \begin{History}
%   \begin{Version}{2007/09/09 v1.0}
%   \item
%     First version.
%   \end{Version}
%   \begin{Version}{2007/12/12 v1.1}
%   \item
%     Adds patch for setting \cs{linewidth} to fix bug
%     in package \xpackage{parallel}.
%   \item
%     Package \xpackage{parallel} is also fixed if color
%     stacks are not available.
%   \item
%     Bug fix, switched stacks now initialized with current color.
%   \item
%     Fix for package \xpackage{parallel}: \cs{raggedbottom} is respected.
%   \end{Version}
%   \begin{Version}{2008/08/11 v1.2}
%   \item
%     Code is not changed.
%   \item
%     URLs updated.
%   \end{Version}
%   \begin{Version}{2010/01/11 v1.3}
%   \item
%     Option `rulebetweencolor' added.
%   \end{Version}
%   \begin{Version}{2016/05/16 v1.4}
%   \item
%     Documentation updates.
%   \end{Version}
% \end{History}
%
% \PrintIndex
%
% \Finale
\endinput

%        (quote the arguments according to the demands of your shell)
%
% Documentation:
%    (a) If pdfcolparallel.drv is present:
%           latex pdfcolparallel.drv
%    (b) Without pdfcolparallel.drv:
%           latex pdfcolparallel.dtx; ...
%    The class ltxdoc loads the configuration file ltxdoc.cfg
%    if available. Here you can specify further options, e.g.
%    use A4 as paper format:
%       \PassOptionsToClass{a4paper}{article}
%
%    Programm calls to get the documentation (example):
%       pdflatex pdfcolparallel.dtx
%       makeindex -s gind.ist pdfcolparallel.idx
%       pdflatex pdfcolparallel.dtx
%       makeindex -s gind.ist pdfcolparallel.idx
%       pdflatex pdfcolparallel.dtx
%
% Installation:
%    TDS:tex/latex/oberdiek/pdfcolparallel.sty
%    TDS:doc/latex/oberdiek/pdfcolparallel.pdf
%    TDS:doc/latex/oberdiek/test/pdfcolparallel-test1.tex
%    TDS:source/latex/oberdiek/pdfcolparallel.dtx
%
%<*ignore>
\begingroup
  \catcode123=1 %
  \catcode125=2 %
  \def\x{LaTeX2e}%
\expandafter\endgroup
\ifcase 0\ifx\install y1\fi\expandafter
         \ifx\csname processbatchFile\endcsname\relax\else1\fi
         \ifx\fmtname\x\else 1\fi\relax
\else\csname fi\endcsname
%</ignore>
%<*install>
\input docstrip.tex
\Msg{************************************************************************}
\Msg{* Installation}
\Msg{* Package: pdfcolparallel 2016/05/16 v1.4 Color stacks support for parallel (HO)}
\Msg{************************************************************************}

\keepsilent
\askforoverwritefalse

\let\MetaPrefix\relax
\preamble

This is a generated file.

Project: pdfcolparallel
Version: 2016/05/16 v1.4

Copyright (C) 2007, 2008, 2010 by
   Heiko Oberdiek <heiko.oberdiek at googlemail.com>

This work may be distributed and/or modified under the
conditions of the LaTeX Project Public License, either
version 1.3c of this license or (at your option) any later
version. This version of this license is in
   http://www.latex-project.org/lppl/lppl-1-3c.txt
and the latest version of this license is in
   http://www.latex-project.org/lppl.txt
and version 1.3 or later is part of all distributions of
LaTeX version 2005/12/01 or later.

This work has the LPPL maintenance status "maintained".

This Current Maintainer of this work is Heiko Oberdiek.

This work consists of the main source file pdfcolparallel.dtx
and the derived files
   pdfcolparallel.sty, pdfcolparallel.pdf, pdfcolparallel.ins,
   pdfcolparallel.drv, pdfcolparallel-test1.tex.

\endpreamble
\let\MetaPrefix\DoubleperCent

\generate{%
  \file{pdfcolparallel.ins}{\from{pdfcolparallel.dtx}{install}}%
  \file{pdfcolparallel.drv}{\from{pdfcolparallel.dtx}{driver}}%
  \usedir{tex/latex/oberdiek}%
  \file{pdfcolparallel.sty}{\from{pdfcolparallel.dtx}{package}}%
  \usedir{doc/latex/oberdiek/test}%
  \file{pdfcolparallel-test1.tex}{\from{pdfcolparallel.dtx}{test1}}%
  \nopreamble
  \nopostamble
  \usedir{source/latex/oberdiek/catalogue}%
  \file{pdfcolparallel.xml}{\from{pdfcolparallel.dtx}{catalogue}}%
}

\catcode32=13\relax% active space
\let =\space%
\Msg{************************************************************************}
\Msg{*}
\Msg{* To finish the installation you have to move the following}
\Msg{* file into a directory searched by TeX:}
\Msg{*}
\Msg{*     pdfcolparallel.sty}
\Msg{*}
\Msg{* To produce the documentation run the file `pdfcolparallel.drv'}
\Msg{* through LaTeX.}
\Msg{*}
\Msg{* Happy TeXing!}
\Msg{*}
\Msg{************************************************************************}

\endbatchfile
%</install>
%<*ignore>
\fi
%</ignore>
%<*driver>
\NeedsTeXFormat{LaTeX2e}
\ProvidesFile{pdfcolparallel.drv}%
  [2016/05/16 v1.4 Color stacks support for parallel (HO)]%
\documentclass{ltxdoc}
\usepackage{holtxdoc}[2011/11/22]
\begin{document}
  \DocInput{pdfcolparallel.dtx}%
\end{document}
%</driver>
% \fi
%
%
% \CharacterTable
%  {Upper-case    \A\B\C\D\E\F\G\H\I\J\K\L\M\N\O\P\Q\R\S\T\U\V\W\X\Y\Z
%   Lower-case    \a\b\c\d\e\f\g\h\i\j\k\l\m\n\o\p\q\r\s\t\u\v\w\x\y\z
%   Digits        \0\1\2\3\4\5\6\7\8\9
%   Exclamation   \!     Double quote  \"     Hash (number) \#
%   Dollar        \$     Percent       \%     Ampersand     \&
%   Acute accent  \'     Left paren    \(     Right paren   \)
%   Asterisk      \*     Plus          \+     Comma         \,
%   Minus         \-     Point         \.     Solidus       \/
%   Colon         \:     Semicolon     \;     Less than     \<
%   Equals        \=     Greater than  \>     Question mark \?
%   Commercial at \@     Left bracket  \[     Backslash     \\
%   Right bracket \]     Circumflex    \^     Underscore    \_
%   Grave accent  \`     Left brace    \{     Vertical bar  \|
%   Right brace   \}     Tilde         \~}
%
% \GetFileInfo{pdfcolparallel.drv}
%
% \title{The \xpackage{pdfcolparallel} package}
% \date{2016/05/16 v1.4}
% \author{Heiko Oberdiek\thanks
% {Please report any issues at https://github.com/ho-tex/oberdiek/issues}\\
% \xemail{heiko.oberdiek at googlemail.com}}
%
% \maketitle
%
% \begin{abstract}
% This packages fixes bugs in \xpackage{parallel} and
% improves color support by using several color stacks
% that are provided by \pdfTeX\ since version 1.40.
% \end{abstract}
%
% \tableofcontents
%
% \section{Usage}
%
% \begin{quote}
% |\usepackage{pdfcolparallel}|
% \end{quote}
% The package \xpackage{pdfcolparallel} loads package \xpackage{parallel}
% \cite{parallel} and redefines some macros to fix bugs.
%
% If color stacks are available then package \xpackage{parallel}
% is further patched to support them.
%
% \subsection{Option \xoption{rulebetweencolor}}
%
% Package \xpackage{pdfcolparallel} also fixes the color for the
% rule between columns.
% Default color is \cs{normalcolor}. But this can be changed by using
% option \xoption{rulebetweencolor} for |\setkeys{parallel}|
% (see package \xpackage{keyval}). The option takes a color specification
% as value. If the value is empty, then the default (\cs{normalcolor})
% is used.
% Examples:
% \begin{quote}
%   |\setkeys{parallel}{rulebetweencolor=blue}|,\\
%   |\setkeys{parallel}{rulebetweencolor={red}}|,\\
%   |\setkeys{parallel}{rulebetweencolor={}}|,
%     \textit{\% \cs{normalcolor} is used}\\
%   |\setkeys{parallel}{rulebetweencolor=[rgb]{1,0,.5}}|
% \end{quote}
%
% \subsection{Future}
%
% If there will be a new version of package \xpackage{parallel}
% that adds support for color stacks, then this package may become
% obsolete.
%
% \StopEventually{
% }
%
% \section{Implementation}
%
% \subsection{Identification}
%
%    \begin{macrocode}
%<*package>
\NeedsTeXFormat{LaTeX2e}
\ProvidesPackage{pdfcolparallel}%
  [2016/05/16 v1.4 Color stacks support for parallel (HO)]%
%    \end{macrocode}
%
% \subsection{Load and fix package \xpackage{parallel}}
%
%    Package \xpackage{parallel} is loaded. Before options of package
%    \xpackage{pdfcolparallel} are passed to package \xpackage{parallel}.
%    \begin{macrocode}
\DeclareOption*{%
  \PassoptionsToPackage{\CurrentOption}{parallel}%
}
\ProcessOptions\relax
\RequirePackage{parallel}[2003/04/13]
%    \end{macrocode}
%
%    \begin{macrocode}
\RequirePackage{infwarerr}[2007/09/09]
%    \end{macrocode}
%
%    \begin{macro}{\pcp@ColorPatch}
%    \begin{macrocode}
\begingroup\expandafter\expandafter\expandafter\endgroup
\expandafter\ifx\csname currentgrouplevel\endcsname\relax
  \def\pcp@ColorPatch{}%
\else
  \def\pcp@ColorPatch{%
    \@ifundefined{set@color}{%
      \gdef\pcp@ColorPatch{}%
    }{%
      \gdef\pcp@ColorPatch{%
        \gdef\pcp@ColorResets{}%
        \bgroup
        \aftergroup\pcp@ColorResets
        \aftergroup\egroup
        \let\pcp@OrgSetColor\set@color
        \let\set@color\pcp@SetColor
        \edef\pcp@GroupLevel{\the\currentgrouplevel}%
      }%
    }%
    \pcp@ColorPatch
  }%
%    \end{macrocode}
%    \end{macro}
%    \begin{macro}{\pcp@SetColor}
%    \begin{macrocode}
  \def\pcp@SetColor{%
    \ifnum\pcp@GroupLevel=\currentgrouplevel
      \let\pcp@OrgAfterGroup\aftergroup
      \def\aftergroup{%
        \g@addto@macro\pcp@ColorResets
      }%
      \pcp@OrgSetColor
      \let\aftergroup\pcp@OrgAfterGroup
    \else
      \pcp@OrgSetColor
    \fi
  }%
\fi
%    \end{macrocode}
%    \end{macro}
%
%    \begin{macro}{\pcp@CmdCheckRedef}
%    \begin{macrocode}
\def\pcp@CmdCheckRedef#1{%
  \begingroup
    \def\pcp@cmd{#1}%
    \afterassignment\pcp@CmdDo
    \long\def\reserved@a
}
\def\pcp@CmdDo{%
    \expandafter\ifx\pcp@cmd\reserved@a
    \else
      \edef\x*{\expandafter\string\pcp@cmd}%
      \@PackageWarningNoLine{pdfcolparallel}{%
        Command \x* has changed.\MessageBreak
        Supported versions of package `parallel':\MessageBreak
        \space\space 2003/04/13\MessageBreak
        The redefinition of \x* may\MessageBreak
        not behave correctly depending on the changes%
      }%
    \fi
  \expandafter\endgroup
  \expandafter\def\pcp@cmd
}
%    \end{macrocode}
%    \end{macro}
%
%    \begin{macrocode}
\def\pcp@SwitchStack#1#2{}
%    \end{macrocode}
%    \begin{macrocode}
\def\pcp@SetCurrent#1{}
%    \end{macrocode}
%
%    \begin{macro}{\ParallelLText}
%    \begin{macrocode}
\pcp@CmdCheckRedef\ParallelLText{%
  \everypar{}%
  \@restorepar
  \begingroup
    \hbadness=3000 %
    \let\footnote=\ParallelLFootnote
    \ParallelWhichBox=0 %
    \global\setbox\ParallelLBox=\vbox\bgroup
      \hsize=\ParallelLWidth
      \aftergroup\ParallelAfterText
      \begingroup
        \afterassignment\ParallelCheckOpenBrace
        \let\x=%
}{%
  \everypar{}%
  \@restorepar
  \@nobreakfalse
  \begingroup
    \hbadness=3000 %
    \let\footnote=\ParallelLFootnote
    \ParallelWhichBox=0 %
    \global\setbox\ParallelLBox=\vbox\bgroup
      \hsize=\ParallelLWidth
      \linewidth=\ParallelLWidth
      \pcp@SwitchStack{Left}\ParallelLBox
      \aftergroup\ParallelAfterText
      \pcp@ColorPatch
      \begingroup
        \afterassignment\ParallelCheckOpenBrace
        \let\x=%
}
%    \end{macrocode}
%    \end{macro}
%
%    \begin{macro}{\ParallelRText}
%    \begin{macrocode}
\pcp@CmdCheckRedef\ParallelRText{%
  \everypar{}%
  \@restorepar
  \begingroup
    \hbadness=3000 %
    \ifnum\ParallelFNMode=\@ne
      \let\footnote=\ParallelRFootnote
    \else
      \let\footnote=\ParallelLFootnote
    \fi
    \ParallelWhichBox=\@ne
    \global\setbox\ParallelRBox=\vbox\bgroup
      \hsize=\ParallelRWidth
      \aftergroup\ParallelAfterText
      \begingroup
        \afterassignment\ParallelCheckOpenBrace
        \let\x=%
}{%
  \everypar{}%
  \@restorepar
  \@nobreakfalse
  \begingroup
    \hbadness=3000 %
    \ifnum\ParallelFNMode=\@ne
      \let\footnote=\ParallelRFootnote
    \else
      \let\footnote=\ParallelLFootnote
    \fi
    \ParallelWhichBox=\@ne
    \global\setbox\ParallelRBox=\vbox\bgroup
      \hsize=\ParallelRWidth
      \linewidth=\ParallelRWidth
      \pcp@SwitchStack{Right}\ParallelRBox
      \aftergroup\ParallelAfterText
      \pcp@ColorPatch
      \begingroup
        \afterassignment\ParallelCheckOpenBrace
        \let\x=%
}
%    \end{macrocode}
%    \end{macro}
%
%    \begin{macro}{\ParallelParTwoPages}
%    \begin{macrocode}
\pcp@CmdCheckRedef\ParallelParTwoPages{%
  \ifnum\ParallelBoolVar=\@ne
    \par
    \begingroup
      \global\ParallelWhichBox=\@ne
      \newpage
      \vbadness=10000 %
      \vfuzz=3ex %
      \splittopskip=\z@skip
      \loop%
        \ifnum\ParallelBoolVar=\@ne%
          \ifnum\ParallelWhichBox=\@ne
            \ifvoid\ParallelLBox
              \mbox{} %
              \newpage
            \else
              \global\ParallelWhichBox=\z@
            \fi
          \else
            \ifvoid\ParallelRBox
              \mbox{} %
              \newpage
            \else
              \global\ParallelWhichBox=\@ne
            \fi
          \fi
          \ifnum\ParallelWhichBox=\z@
            \ifodd\thepage
              \mbox{} %
              \newpage
            \fi
            \hbox to\textwidth{%
              \vbox{\vsplit\ParallelLBox to.98\textheight}%
            }%
          \else
            \ifodd\thepage\relax
            \else
              \mbox{} %
              \newpage
            \fi
            \hbox to\textwidth{%
              \vbox{\vsplit\ParallelRBox to.98\textheight}%
            }%
          \fi
          \vspace*{\fill}%
          \newpage
        \fi
        \ifvoid\ParallelLBox
          \ifvoid\ParallelRBox
            \global\ParallelBoolVar=\z@
          \fi
        \fi
      \ifnum\ParallelBoolVar=\@ne
      \repeat
      \par
    \endgroup
  \fi
}{%
%    \end{macrocode}
%    Additional fixes:
%    \begin{itemize}
%    \item Unnecessary white space removed.
%    \item |\ifodd\thepage| changed to |\ifodd\value{page}|.
%    \end{itemize}
%    \begin{macrocode}
  \ifnum\ParallelBoolVar=\@ne
    \par
    \begingroup
      \global\ParallelWhichBox=\@ne
      \newpage
      \vbadness=10000 %
      \vfuzz=3ex %
      \splittopskip=\z@skip
      \loop%
        \ifnum\ParallelBoolVar=\@ne%
          \ifnum\ParallelWhichBox=\@ne
            \ifvoid\ParallelLBox
              \mbox{}%
              \newpage
            \else
              \global\ParallelWhichBox=\z@
            \fi
          \else
            \ifvoid\ParallelRBox
              \null
              \newpage
            \else
              \global\ParallelWhichBox=\@ne
            \fi
          \fi
          \ifnum\ParallelWhichBox=\z@
            \ifodd\value{page}%
              \null
              \newpage
            \fi
            \hbox to\textwidth{%
              \pcp@SetCurrent{Left}%
              \setbox\z@=\vsplit\ParallelLBox to.98\textheight
              \vbox to.98\textheight{%
                \@texttop
                \unvbox\z@
                \@textbottom
              }%
            }%
          \else
            \ifodd\value{page}%
            \else
              \mbox{}%
              \newpage
            \fi
            \hbox to\textwidth{%
              \pcp@SetCurrent{Right}%
              \setbox\z@=\vsplit\ParallelRBox to.98\textheight
              \vbox to.98\textheight{%
                \@texttop
                \unvbox\z@
                \@textbottom
              }%
            }%
          \fi
          \vspace*{\fill}%
          \newpage
        \fi
        \ifvoid\ParallelLBox
          \ifvoid\ParallelRBox
            \global\ParallelBoolVar=\z@
          \fi
        \fi
      \ifnum\ParallelBoolVar=\@ne
      \repeat
      \par
    \endgroup
    \pcp@SetCurrent{}%
  \fi
}
%    \end{macrocode}
%    \end{macro}
%
% \subsection{Color stack support}
%
%    \begin{macrocode}
\RequirePackage{pdfcol}[2007/12/12]
\ifpdfcolAvailable
\else
  \PackageInfo{pdfcolparallel}{%
    Loading aborted, because color stacks are not available%
  }%
  \expandafter\endinput
\fi
%    \end{macrocode}
%
%    \begin{macrocode}
\pdfcolInitStack{pcp@Left}
\pdfcolInitStack{pcp@Right}
%    \end{macrocode}
%    \begin{macro}{\pcp@Box}
%    \begin{macrocode}
\newbox\pcp@Box
%    \end{macrocode}
%    \end{macro}
%    \begin{macro}{\pcp@SwitchStack}
%    \begin{macrocode}
\def\pcp@SwitchStack#1#2{%
  \pdfcolSwitchStack{pcp@#1}%
  \global\setbox\pcp@Box=\vbox to 0pt{%
    \pdfcolSetCurrentColor
  }%
  \aftergroup\pcp@FixBox
  \aftergroup#2%
}
%    \end{macrocode}
%    \end{macro}
%    \begin{macro}{\pcp@FixBox}
%    \begin{macrocode}
\def\pcp@FixBox#1{%
  \global\setbox#1=\vbox{%
    \unvbox\pcp@Box
    \unvbox#1%
  }%
}
%    \end{macrocode}
%    \end{macro}
%    \begin{macro}{\pcp@SetCurrent}
%    \begin{macrocode}
\def\pcp@SetCurrent#1{%
  \ifx\\#1\\%
    \pdfcolSetCurrent{}%
  \else
    \pdfcolSetCurrent{pcp@#1}%
  \fi
}
%    \end{macrocode}
%    \end{macro}
%
% \subsection{Redefinitions}
%
%    \begin{macro}{\ParallelParOnePage}
%    \begin{macrocode}
\pcp@CmdCheckRedef\ParallelParOnePage{%
  \ifnum\ParallelBoolVar=\@ne
    \par
    \begingroup
      \leftmargin=\z@
      \rightmargin=\z@
      \parskip=\z@skip
      \parindent=\z@
      \vbadness=10000 %
      \vfuzz=3ex %
      \splittopskip=\z@skip
      \loop
        \ifnum\ParallelBoolVar=\@ne
          \noindent
          \hbox to\textwidth{%
            \hskip\ParallelLeftMargin
            \hbox to\ParallelTextWidth{%
              \ifvoid\ParallelLBox
                \hskip\ParallelLWidth
              \else
                \ParallelWhichBox=\z@
                \vbox{%
                  \setbox\ParallelBoxVar
                      =\vsplit\ParallelLBox to\dp\strutbox
                  \unvbox\ParallelBoxVar
                }%
              \fi
              \strut
              \ifnum\ParallelBoolMid=\@ne
                \hskip\ParallelMainMidSkip
                \vrule
              \else
                \hss
              \fi
              \hss
              \ifvoid\ParallelRBox
                \hskip\ParallelRWidth
              \else
                \ParallelWhichBox=\@ne
                \vbox{%
                  \setbox\ParallelBoxVar
                      =\vsplit\ParallelRBox to\dp\strutbox
                  \unvbox\ParallelBoxVar
                }%
              \fi
            }%
          }%
          \ifvoid\ParallelLBox
            \ifvoid\ParallelRBox
              \global\ParallelBoolVar=\z@
            \fi
          \fi%
        \fi%
      \ifnum\ParallelBoolVar=\@ne
        \penalty\interlinepenalty
      \repeat
      \par
    \endgroup
  \fi
}{%
  \ifnum\ParallelBoolVar=\@ne
    \par
    \begingroup
      \leftmargin=\z@
      \rightmargin=\z@
      \parskip=\z@skip
      \parindent=\z@
      \vbadness=10000 %
      \vfuzz=3ex %
      \splittopskip=\z@skip
      \loop
        \ifnum\ParallelBoolVar=\@ne
          \noindent
          \hbox to\textwidth{%
            \hskip\ParallelLeftMargin
            \hbox to\ParallelTextWidth{%
              \ifvoid\ParallelLBox
                \hskip\ParallelLWidth
              \else
                \pcp@SetCurrent{Left}%
                \ParallelWhichBox=\z@
                \vbox{%
                  \setbox\ParallelBoxVar
                      =\vsplit\ParallelLBox to\dp\strutbox
                  \unvbox\ParallelBoxVar
                }%
              \fi
              \strut
              \ifnum\ParallelBoolMid=\@ne
                \hskip\ParallelMainMidSkip
                \begingroup
                  \pcp@RuleBetweenColor
                  \vrule
                \endgroup
              \else
                \hss
              \fi
              \hss
              \ifvoid\ParallelRBox
                \hskip\ParallelRWidth
              \else
                \pcp@SetCurrent{Right}%
                \ParallelWhichBox=\@ne
                \vbox{%
                  \setbox\ParallelBoxVar
                      =\vsplit\ParallelRBox to\dp\strutbox
                  \unvbox\ParallelBoxVar
                }%
              \fi
            }%
          }%
          \ifvoid\ParallelLBox
            \ifvoid\ParallelRBox
              \global\ParallelBoolVar=\z@
            \fi
          \fi%
        \fi%
      \ifnum\ParallelBoolVar=\@ne
        \penalty\interlinepenalty
      \repeat
      \par
    \endgroup
    \pcp@SetCurrent{}%
  \fi
}
%    \end{macrocode}
%    \end{macro}
%    \begin{macro}{\pcp@RuleBetweenColorDefault}
%    \begin{macrocode}
\def\pcp@RuleBetweenColorDefault{%
  \normalcolor
}
%    \end{macrocode}
%    \end{macro}
%    \begin{macro}{\pcp@RuleBetweenColor}
%    \begin{macrocode}
\let\pcp@RuleBetweenColor\pcp@RuleBetweenColorDefault
%    \end{macrocode}
%    \end{macro}
%    \begin{macrocode}
\RequirePackage{keyval}
\define@key{parallel}{rulebetweencolor}{%
  \edef\pcp@temp{#1}%
  \ifx\pcp@temp\@empty
    \let\pcp@RuleBetweenColor\pcp@RuleBetweenColorDefault
  \else
    \edef\pcp@temp{%
      \noexpand\@ifnextchar[{%
        \def\noexpand\pcp@RuleBetweenColor{%
          \noexpand\color\pcp@temp
        }%
        \noexpand\pcp@GobbleNil
      }{%
        \def\noexpand\pcp@RuleBetweenColor{%
          \noexpand\color{\pcp@temp}%
        }%
        \noexpand\pcp@GobbleNil
      }%
      \pcp@temp\noexpand\@nil
    }%
    \pcp@temp
  \fi
}
%    \end{macrocode}
%    \begin{macro}{\pcp@GobbleNil}
%    \begin{macrocode}
\long\def\pcp@GobbleNil#1\@nil{}
%    \end{macrocode}
%    \end{macro}
%
%    \begin{macrocode}
%</package>
%    \end{macrocode}
%
% \section{Test}
%
%    The test file is a modified version of the file that
%    Alexander Hirsch has posted in \xnewsgroup{de.comp.text.tex}:
%    \URL{``\link{\texttt{parallel.sty} und farbiger Text}''}^^A
%    {http://groups.google.com/group/de.comp.text.tex/msg/6a759cf33bb071a5}
%    \begin{macrocode}
%<*test1>
\AtEndDocument{%
  \typeout{}%
  \typeout{**************************************}%
  \typeout{*** \space Check the PDF file manually! \space ***}%
  \typeout{**************************************}%
  \typeout{}%
}
\documentclass{article}
\usepackage{xcolor}
\usepackage{pdfcolparallel}[2016/05/16]

\begin{document}
  \color{green}%
  Green%
  \begin{Parallel}{0.47\textwidth}{0.47\textwidth}%
    \ParallelLText{%
      \textcolor{red}{%
        Ein Absatz, der sich ueber zwei Zeilen erstrecken soll. %
        Ein Absatz, der sich ueber zwei Zeilen erstrecken soll.%
      }%
    }%
    \ParallelRText{%
      \textcolor{blue}{%
        Ein Absatz, der sich ueber zwei Zeilen erstrecken soll. %
        Ein Absatz, der sich ueber zwei Zeilen erstrecken soll.%
      }%
    }%
    \ParallelPar
    \ParallelLText{%
      Default %
      \color{red}%
      Ein Absatz, der sich ueber zwei Zeilen erstrecken soll. %
      Ein Absatz, der sich ueber zwei Zeilen erstrecken soll.%
    }%
    \ParallelRText{%
      Default %
      \color{blue}%
      Ein Absatz, der sich ueber zwei Zeilen erstrecken soll. %
      Ein Absatz, der sich ueber zwei Zeilen erstrecken soll.%
    }%
    \ParallelPar
    \ParallelLText{%
      \begin{enumerate}%
      \item left text, left text, left text, left text, %
            left text, left text, left text, left text,%
      \item left text, left text, left text, left text, %
            left text, left text, left text, left text.%
      \end{enumerate}%
    }%
    \ParallelRText{%
      \begin{enumerate}%
      \item right text, right text, right text, right text, %
            right text, right text, right text, right text.%
      \item right text, right text, right text, right text, %
            right text, right text, right text, right text.%
      \end{enumerate}%
    }%
  \end{Parallel}%
  \begin{Parallel}[p]{\textwidth}{\textwidth}%
    \ParallelLText{%
      \textcolor{red}{%
        Ein Absatz, der sich ueber zwei Zeilen erstrecken soll. %
        Ein Absatz, der sich ueber zwei Zeilen erstrecken soll. %
        Foo bar bla bla bla.%
      }%
      \par
      Und noch ein Absatz.%
    }%
    \ParallelRText{%
      \textcolor{blue}{%
        Ein Absatz, der sich ueber zwei Zeilen erstrecken soll. %
        Ein Absatz, der sich ueber zwei Zeilen erstrecken soll. %
        Foo bar bla bla bla.%
      }%
    }%
  \end{Parallel}%
  \begin{Parallel}[p]{\textwidth}{\textwidth}%
    \ParallelLText{%
      \rule{1pt}{.98\textheight}\Huge g%
    }%
    \ParallelRText{%
      \rule{1pt}{.98\textheight}y%
    }%
  \end{Parallel}%
  Green%
\end{document}
%</test1>
%    \end{macrocode}
%
% \section{Installation}
%
% \subsection{Download}
%
% \paragraph{Package.} This package is available on
% CTAN\footnote{\url{http://ctan.org/pkg/pdfcolparallel}}:
% \begin{description}
% \item[\CTAN{macros/latex/contrib/oberdiek/pdfcolparallel.dtx}] The source file.
% \item[\CTAN{macros/latex/contrib/oberdiek/pdfcolparallel.pdf}] Documentation.
% \end{description}
%
%
% \paragraph{Bundle.} All the packages of the bundle `oberdiek'
% are also available in a TDS compliant ZIP archive. There
% the packages are already unpacked and the documentation files
% are generated. The files and directories obey the TDS standard.
% \begin{description}
% \item[\CTAN{install/macros/latex/contrib/oberdiek.tds.zip}]
% \end{description}
% \emph{TDS} refers to the standard ``A Directory Structure
% for \TeX\ Files'' (\CTAN{tds/tds.pdf}). Directories
% with \xfile{texmf} in their name are usually organized this way.
%
% \subsection{Bundle installation}
%
% \paragraph{Unpacking.} Unpack the \xfile{oberdiek.tds.zip} in the
% TDS tree (also known as \xfile{texmf} tree) of your choice.
% Example (linux):
% \begin{quote}
%   |unzip oberdiek.tds.zip -d ~/texmf|
% \end{quote}
%
% \paragraph{Script installation.}
% Check the directory \xfile{TDS:scripts/oberdiek/} for
% scripts that need further installation steps.
% Package \xpackage{attachfile2} comes with the Perl script
% \xfile{pdfatfi.pl} that should be installed in such a way
% that it can be called as \texttt{pdfatfi}.
% Example (linux):
% \begin{quote}
%   |chmod +x scripts/oberdiek/pdfatfi.pl|\\
%   |cp scripts/oberdiek/pdfatfi.pl /usr/local/bin/|
% \end{quote}
%
% \subsection{Package installation}
%
% \paragraph{Unpacking.} The \xfile{.dtx} file is a self-extracting
% \docstrip\ archive. The files are extracted by running the
% \xfile{.dtx} through \plainTeX:
% \begin{quote}
%   \verb|tex pdfcolparallel.dtx|
% \end{quote}
%
% \paragraph{TDS.} Now the different files must be moved into
% the different directories in your installation TDS tree
% (also known as \xfile{texmf} tree):
% \begin{quote}
% \def\t{^^A
% \begin{tabular}{@{}>{\ttfamily}l@{ $\rightarrow$ }>{\ttfamily}l@{}}
%   pdfcolparallel.sty & tex/latex/oberdiek/pdfcolparallel.sty\\
%   pdfcolparallel.pdf & doc/latex/oberdiek/pdfcolparallel.pdf\\
%   test/pdfcolparallel-test1.tex & doc/latex/oberdiek/test/pdfcolparallel-test1.tex\\
%   pdfcolparallel.dtx & source/latex/oberdiek/pdfcolparallel.dtx\\
% \end{tabular}^^A
% }^^A
% \sbox0{\t}^^A
% \ifdim\wd0>\linewidth
%   \begingroup
%     \advance\linewidth by\leftmargin
%     \advance\linewidth by\rightmargin
%   \edef\x{\endgroup
%     \def\noexpand\lw{\the\linewidth}^^A
%   }\x
%   \def\lwbox{^^A
%     \leavevmode
%     \hbox to \linewidth{^^A
%       \kern-\leftmargin\relax
%       \hss
%       \usebox0
%       \hss
%       \kern-\rightmargin\relax
%     }^^A
%   }^^A
%   \ifdim\wd0>\lw
%     \sbox0{\small\t}^^A
%     \ifdim\wd0>\linewidth
%       \ifdim\wd0>\lw
%         \sbox0{\footnotesize\t}^^A
%         \ifdim\wd0>\linewidth
%           \ifdim\wd0>\lw
%             \sbox0{\scriptsize\t}^^A
%             \ifdim\wd0>\linewidth
%               \ifdim\wd0>\lw
%                 \sbox0{\tiny\t}^^A
%                 \ifdim\wd0>\linewidth
%                   \lwbox
%                 \else
%                   \usebox0
%                 \fi
%               \else
%                 \lwbox
%               \fi
%             \else
%               \usebox0
%             \fi
%           \else
%             \lwbox
%           \fi
%         \else
%           \usebox0
%         \fi
%       \else
%         \lwbox
%       \fi
%     \else
%       \usebox0
%     \fi
%   \else
%     \lwbox
%   \fi
% \else
%   \usebox0
% \fi
% \end{quote}
% If you have a \xfile{docstrip.cfg} that configures and enables \docstrip's
% TDS installing feature, then some files can already be in the right
% place, see the documentation of \docstrip.
%
% \subsection{Refresh file name databases}
%
% If your \TeX~distribution
% (\teTeX, \mikTeX, \dots) relies on file name databases, you must refresh
% these. For example, \teTeX\ users run \verb|texhash| or
% \verb|mktexlsr|.
%
% \subsection{Some details for the interested}
%
% \paragraph{Attached source.}
%
% The PDF documentation on CTAN also includes the
% \xfile{.dtx} source file. It can be extracted by
% AcrobatReader 6 or higher. Another option is \textsf{pdftk},
% e.g. unpack the file into the current directory:
% \begin{quote}
%   \verb|pdftk pdfcolparallel.pdf unpack_files output .|
% \end{quote}
%
% \paragraph{Unpacking with \LaTeX.}
% The \xfile{.dtx} chooses its action depending on the format:
% \begin{description}
% \item[\plainTeX:] Run \docstrip\ and extract the files.
% \item[\LaTeX:] Generate the documentation.
% \end{description}
% If you insist on using \LaTeX\ for \docstrip\ (really,
% \docstrip\ does not need \LaTeX), then inform the autodetect routine
% about your intention:
% \begin{quote}
%   \verb|latex \let\install=y% \iffalse meta-comment
%
% File: pdfcolparallel.dtx
% Version: 2016/05/16 v1.4
% Info: Color stacks support for parallel
%
% Copyright (C) 2007, 2008, 2010 by
%    Heiko Oberdiek <heiko.oberdiek at googlemail.com>
%    2016
%    https://github.com/ho-tex/oberdiek/issues
%
% This work may be distributed and/or modified under the
% conditions of the LaTeX Project Public License, either
% version 1.3c of this license or (at your option) any later
% version. This version of this license is in
%    http://www.latex-project.org/lppl/lppl-1-3c.txt
% and the latest version of this license is in
%    http://www.latex-project.org/lppl.txt
% and version 1.3 or later is part of all distributions of
% LaTeX version 2005/12/01 or later.
%
% This work has the LPPL maintenance status "maintained".
%
% This Current Maintainer of this work is Heiko Oberdiek.
%
% This work consists of the main source file pdfcolparallel.dtx
% and the derived files
%    pdfcolparallel.sty, pdfcolparallel.pdf, pdfcolparallel.ins,
%    pdfcolparallel.drv, pdfcolparallel-test1.tex.
%
% Distribution:
%    CTAN:macros/latex/contrib/oberdiek/pdfcolparallel.dtx
%    CTAN:macros/latex/contrib/oberdiek/pdfcolparallel.pdf
%
% Unpacking:
%    (a) If pdfcolparallel.ins is present:
%           tex pdfcolparallel.ins
%    (b) Without pdfcolparallel.ins:
%           tex pdfcolparallel.dtx
%    (c) If you insist on using LaTeX
%           latex \let\install=y% \iffalse meta-comment
%
% File: pdfcolparallel.dtx
% Version: 2016/05/16 v1.4
% Info: Color stacks support for parallel
%
% Copyright (C) 2007, 2008, 2010 by
%    Heiko Oberdiek <heiko.oberdiek at googlemail.com>
%    2016
%    https://github.com/ho-tex/oberdiek/issues
%
% This work may be distributed and/or modified under the
% conditions of the LaTeX Project Public License, either
% version 1.3c of this license or (at your option) any later
% version. This version of this license is in
%    http://www.latex-project.org/lppl/lppl-1-3c.txt
% and the latest version of this license is in
%    http://www.latex-project.org/lppl.txt
% and version 1.3 or later is part of all distributions of
% LaTeX version 2005/12/01 or later.
%
% This work has the LPPL maintenance status "maintained".
%
% This Current Maintainer of this work is Heiko Oberdiek.
%
% This work consists of the main source file pdfcolparallel.dtx
% and the derived files
%    pdfcolparallel.sty, pdfcolparallel.pdf, pdfcolparallel.ins,
%    pdfcolparallel.drv, pdfcolparallel-test1.tex.
%
% Distribution:
%    CTAN:macros/latex/contrib/oberdiek/pdfcolparallel.dtx
%    CTAN:macros/latex/contrib/oberdiek/pdfcolparallel.pdf
%
% Unpacking:
%    (a) If pdfcolparallel.ins is present:
%           tex pdfcolparallel.ins
%    (b) Without pdfcolparallel.ins:
%           tex pdfcolparallel.dtx
%    (c) If you insist on using LaTeX
%           latex \let\install=y\input{pdfcolparallel.dtx}
%        (quote the arguments according to the demands of your shell)
%
% Documentation:
%    (a) If pdfcolparallel.drv is present:
%           latex pdfcolparallel.drv
%    (b) Without pdfcolparallel.drv:
%           latex pdfcolparallel.dtx; ...
%    The class ltxdoc loads the configuration file ltxdoc.cfg
%    if available. Here you can specify further options, e.g.
%    use A4 as paper format:
%       \PassOptionsToClass{a4paper}{article}
%
%    Programm calls to get the documentation (example):
%       pdflatex pdfcolparallel.dtx
%       makeindex -s gind.ist pdfcolparallel.idx
%       pdflatex pdfcolparallel.dtx
%       makeindex -s gind.ist pdfcolparallel.idx
%       pdflatex pdfcolparallel.dtx
%
% Installation:
%    TDS:tex/latex/oberdiek/pdfcolparallel.sty
%    TDS:doc/latex/oberdiek/pdfcolparallel.pdf
%    TDS:doc/latex/oberdiek/test/pdfcolparallel-test1.tex
%    TDS:source/latex/oberdiek/pdfcolparallel.dtx
%
%<*ignore>
\begingroup
  \catcode123=1 %
  \catcode125=2 %
  \def\x{LaTeX2e}%
\expandafter\endgroup
\ifcase 0\ifx\install y1\fi\expandafter
         \ifx\csname processbatchFile\endcsname\relax\else1\fi
         \ifx\fmtname\x\else 1\fi\relax
\else\csname fi\endcsname
%</ignore>
%<*install>
\input docstrip.tex
\Msg{************************************************************************}
\Msg{* Installation}
\Msg{* Package: pdfcolparallel 2016/05/16 v1.4 Color stacks support for parallel (HO)}
\Msg{************************************************************************}

\keepsilent
\askforoverwritefalse

\let\MetaPrefix\relax
\preamble

This is a generated file.

Project: pdfcolparallel
Version: 2016/05/16 v1.4

Copyright (C) 2007, 2008, 2010 by
   Heiko Oberdiek <heiko.oberdiek at googlemail.com>

This work may be distributed and/or modified under the
conditions of the LaTeX Project Public License, either
version 1.3c of this license or (at your option) any later
version. This version of this license is in
   http://www.latex-project.org/lppl/lppl-1-3c.txt
and the latest version of this license is in
   http://www.latex-project.org/lppl.txt
and version 1.3 or later is part of all distributions of
LaTeX version 2005/12/01 or later.

This work has the LPPL maintenance status "maintained".

This Current Maintainer of this work is Heiko Oberdiek.

This work consists of the main source file pdfcolparallel.dtx
and the derived files
   pdfcolparallel.sty, pdfcolparallel.pdf, pdfcolparallel.ins,
   pdfcolparallel.drv, pdfcolparallel-test1.tex.

\endpreamble
\let\MetaPrefix\DoubleperCent

\generate{%
  \file{pdfcolparallel.ins}{\from{pdfcolparallel.dtx}{install}}%
  \file{pdfcolparallel.drv}{\from{pdfcolparallel.dtx}{driver}}%
  \usedir{tex/latex/oberdiek}%
  \file{pdfcolparallel.sty}{\from{pdfcolparallel.dtx}{package}}%
  \usedir{doc/latex/oberdiek/test}%
  \file{pdfcolparallel-test1.tex}{\from{pdfcolparallel.dtx}{test1}}%
  \nopreamble
  \nopostamble
  \usedir{source/latex/oberdiek/catalogue}%
  \file{pdfcolparallel.xml}{\from{pdfcolparallel.dtx}{catalogue}}%
}

\catcode32=13\relax% active space
\let =\space%
\Msg{************************************************************************}
\Msg{*}
\Msg{* To finish the installation you have to move the following}
\Msg{* file into a directory searched by TeX:}
\Msg{*}
\Msg{*     pdfcolparallel.sty}
\Msg{*}
\Msg{* To produce the documentation run the file `pdfcolparallel.drv'}
\Msg{* through LaTeX.}
\Msg{*}
\Msg{* Happy TeXing!}
\Msg{*}
\Msg{************************************************************************}

\endbatchfile
%</install>
%<*ignore>
\fi
%</ignore>
%<*driver>
\NeedsTeXFormat{LaTeX2e}
\ProvidesFile{pdfcolparallel.drv}%
  [2016/05/16 v1.4 Color stacks support for parallel (HO)]%
\documentclass{ltxdoc}
\usepackage{holtxdoc}[2011/11/22]
\begin{document}
  \DocInput{pdfcolparallel.dtx}%
\end{document}
%</driver>
% \fi
%
%
% \CharacterTable
%  {Upper-case    \A\B\C\D\E\F\G\H\I\J\K\L\M\N\O\P\Q\R\S\T\U\V\W\X\Y\Z
%   Lower-case    \a\b\c\d\e\f\g\h\i\j\k\l\m\n\o\p\q\r\s\t\u\v\w\x\y\z
%   Digits        \0\1\2\3\4\5\6\7\8\9
%   Exclamation   \!     Double quote  \"     Hash (number) \#
%   Dollar        \$     Percent       \%     Ampersand     \&
%   Acute accent  \'     Left paren    \(     Right paren   \)
%   Asterisk      \*     Plus          \+     Comma         \,
%   Minus         \-     Point         \.     Solidus       \/
%   Colon         \:     Semicolon     \;     Less than     \<
%   Equals        \=     Greater than  \>     Question mark \?
%   Commercial at \@     Left bracket  \[     Backslash     \\
%   Right bracket \]     Circumflex    \^     Underscore    \_
%   Grave accent  \`     Left brace    \{     Vertical bar  \|
%   Right brace   \}     Tilde         \~}
%
% \GetFileInfo{pdfcolparallel.drv}
%
% \title{The \xpackage{pdfcolparallel} package}
% \date{2016/05/16 v1.4}
% \author{Heiko Oberdiek\thanks
% {Please report any issues at https://github.com/ho-tex/oberdiek/issues}\\
% \xemail{heiko.oberdiek at googlemail.com}}
%
% \maketitle
%
% \begin{abstract}
% This packages fixes bugs in \xpackage{parallel} and
% improves color support by using several color stacks
% that are provided by \pdfTeX\ since version 1.40.
% \end{abstract}
%
% \tableofcontents
%
% \section{Usage}
%
% \begin{quote}
% |\usepackage{pdfcolparallel}|
% \end{quote}
% The package \xpackage{pdfcolparallel} loads package \xpackage{parallel}
% \cite{parallel} and redefines some macros to fix bugs.
%
% If color stacks are available then package \xpackage{parallel}
% is further patched to support them.
%
% \subsection{Option \xoption{rulebetweencolor}}
%
% Package \xpackage{pdfcolparallel} also fixes the color for the
% rule between columns.
% Default color is \cs{normalcolor}. But this can be changed by using
% option \xoption{rulebetweencolor} for |\setkeys{parallel}|
% (see package \xpackage{keyval}). The option takes a color specification
% as value. If the value is empty, then the default (\cs{normalcolor})
% is used.
% Examples:
% \begin{quote}
%   |\setkeys{parallel}{rulebetweencolor=blue}|,\\
%   |\setkeys{parallel}{rulebetweencolor={red}}|,\\
%   |\setkeys{parallel}{rulebetweencolor={}}|,
%     \textit{\% \cs{normalcolor} is used}\\
%   |\setkeys{parallel}{rulebetweencolor=[rgb]{1,0,.5}}|
% \end{quote}
%
% \subsection{Future}
%
% If there will be a new version of package \xpackage{parallel}
% that adds support for color stacks, then this package may become
% obsolete.
%
% \StopEventually{
% }
%
% \section{Implementation}
%
% \subsection{Identification}
%
%    \begin{macrocode}
%<*package>
\NeedsTeXFormat{LaTeX2e}
\ProvidesPackage{pdfcolparallel}%
  [2016/05/16 v1.4 Color stacks support for parallel (HO)]%
%    \end{macrocode}
%
% \subsection{Load and fix package \xpackage{parallel}}
%
%    Package \xpackage{parallel} is loaded. Before options of package
%    \xpackage{pdfcolparallel} are passed to package \xpackage{parallel}.
%    \begin{macrocode}
\DeclareOption*{%
  \PassoptionsToPackage{\CurrentOption}{parallel}%
}
\ProcessOptions\relax
\RequirePackage{parallel}[2003/04/13]
%    \end{macrocode}
%
%    \begin{macrocode}
\RequirePackage{infwarerr}[2007/09/09]
%    \end{macrocode}
%
%    \begin{macro}{\pcp@ColorPatch}
%    \begin{macrocode}
\begingroup\expandafter\expandafter\expandafter\endgroup
\expandafter\ifx\csname currentgrouplevel\endcsname\relax
  \def\pcp@ColorPatch{}%
\else
  \def\pcp@ColorPatch{%
    \@ifundefined{set@color}{%
      \gdef\pcp@ColorPatch{}%
    }{%
      \gdef\pcp@ColorPatch{%
        \gdef\pcp@ColorResets{}%
        \bgroup
        \aftergroup\pcp@ColorResets
        \aftergroup\egroup
        \let\pcp@OrgSetColor\set@color
        \let\set@color\pcp@SetColor
        \edef\pcp@GroupLevel{\the\currentgrouplevel}%
      }%
    }%
    \pcp@ColorPatch
  }%
%    \end{macrocode}
%    \end{macro}
%    \begin{macro}{\pcp@SetColor}
%    \begin{macrocode}
  \def\pcp@SetColor{%
    \ifnum\pcp@GroupLevel=\currentgrouplevel
      \let\pcp@OrgAfterGroup\aftergroup
      \def\aftergroup{%
        \g@addto@macro\pcp@ColorResets
      }%
      \pcp@OrgSetColor
      \let\aftergroup\pcp@OrgAfterGroup
    \else
      \pcp@OrgSetColor
    \fi
  }%
\fi
%    \end{macrocode}
%    \end{macro}
%
%    \begin{macro}{\pcp@CmdCheckRedef}
%    \begin{macrocode}
\def\pcp@CmdCheckRedef#1{%
  \begingroup
    \def\pcp@cmd{#1}%
    \afterassignment\pcp@CmdDo
    \long\def\reserved@a
}
\def\pcp@CmdDo{%
    \expandafter\ifx\pcp@cmd\reserved@a
    \else
      \edef\x*{\expandafter\string\pcp@cmd}%
      \@PackageWarningNoLine{pdfcolparallel}{%
        Command \x* has changed.\MessageBreak
        Supported versions of package `parallel':\MessageBreak
        \space\space 2003/04/13\MessageBreak
        The redefinition of \x* may\MessageBreak
        not behave correctly depending on the changes%
      }%
    \fi
  \expandafter\endgroup
  \expandafter\def\pcp@cmd
}
%    \end{macrocode}
%    \end{macro}
%
%    \begin{macrocode}
\def\pcp@SwitchStack#1#2{}
%    \end{macrocode}
%    \begin{macrocode}
\def\pcp@SetCurrent#1{}
%    \end{macrocode}
%
%    \begin{macro}{\ParallelLText}
%    \begin{macrocode}
\pcp@CmdCheckRedef\ParallelLText{%
  \everypar{}%
  \@restorepar
  \begingroup
    \hbadness=3000 %
    \let\footnote=\ParallelLFootnote
    \ParallelWhichBox=0 %
    \global\setbox\ParallelLBox=\vbox\bgroup
      \hsize=\ParallelLWidth
      \aftergroup\ParallelAfterText
      \begingroup
        \afterassignment\ParallelCheckOpenBrace
        \let\x=%
}{%
  \everypar{}%
  \@restorepar
  \@nobreakfalse
  \begingroup
    \hbadness=3000 %
    \let\footnote=\ParallelLFootnote
    \ParallelWhichBox=0 %
    \global\setbox\ParallelLBox=\vbox\bgroup
      \hsize=\ParallelLWidth
      \linewidth=\ParallelLWidth
      \pcp@SwitchStack{Left}\ParallelLBox
      \aftergroup\ParallelAfterText
      \pcp@ColorPatch
      \begingroup
        \afterassignment\ParallelCheckOpenBrace
        \let\x=%
}
%    \end{macrocode}
%    \end{macro}
%
%    \begin{macro}{\ParallelRText}
%    \begin{macrocode}
\pcp@CmdCheckRedef\ParallelRText{%
  \everypar{}%
  \@restorepar
  \begingroup
    \hbadness=3000 %
    \ifnum\ParallelFNMode=\@ne
      \let\footnote=\ParallelRFootnote
    \else
      \let\footnote=\ParallelLFootnote
    \fi
    \ParallelWhichBox=\@ne
    \global\setbox\ParallelRBox=\vbox\bgroup
      \hsize=\ParallelRWidth
      \aftergroup\ParallelAfterText
      \begingroup
        \afterassignment\ParallelCheckOpenBrace
        \let\x=%
}{%
  \everypar{}%
  \@restorepar
  \@nobreakfalse
  \begingroup
    \hbadness=3000 %
    \ifnum\ParallelFNMode=\@ne
      \let\footnote=\ParallelRFootnote
    \else
      \let\footnote=\ParallelLFootnote
    \fi
    \ParallelWhichBox=\@ne
    \global\setbox\ParallelRBox=\vbox\bgroup
      \hsize=\ParallelRWidth
      \linewidth=\ParallelRWidth
      \pcp@SwitchStack{Right}\ParallelRBox
      \aftergroup\ParallelAfterText
      \pcp@ColorPatch
      \begingroup
        \afterassignment\ParallelCheckOpenBrace
        \let\x=%
}
%    \end{macrocode}
%    \end{macro}
%
%    \begin{macro}{\ParallelParTwoPages}
%    \begin{macrocode}
\pcp@CmdCheckRedef\ParallelParTwoPages{%
  \ifnum\ParallelBoolVar=\@ne
    \par
    \begingroup
      \global\ParallelWhichBox=\@ne
      \newpage
      \vbadness=10000 %
      \vfuzz=3ex %
      \splittopskip=\z@skip
      \loop%
        \ifnum\ParallelBoolVar=\@ne%
          \ifnum\ParallelWhichBox=\@ne
            \ifvoid\ParallelLBox
              \mbox{} %
              \newpage
            \else
              \global\ParallelWhichBox=\z@
            \fi
          \else
            \ifvoid\ParallelRBox
              \mbox{} %
              \newpage
            \else
              \global\ParallelWhichBox=\@ne
            \fi
          \fi
          \ifnum\ParallelWhichBox=\z@
            \ifodd\thepage
              \mbox{} %
              \newpage
            \fi
            \hbox to\textwidth{%
              \vbox{\vsplit\ParallelLBox to.98\textheight}%
            }%
          \else
            \ifodd\thepage\relax
            \else
              \mbox{} %
              \newpage
            \fi
            \hbox to\textwidth{%
              \vbox{\vsplit\ParallelRBox to.98\textheight}%
            }%
          \fi
          \vspace*{\fill}%
          \newpage
        \fi
        \ifvoid\ParallelLBox
          \ifvoid\ParallelRBox
            \global\ParallelBoolVar=\z@
          \fi
        \fi
      \ifnum\ParallelBoolVar=\@ne
      \repeat
      \par
    \endgroup
  \fi
}{%
%    \end{macrocode}
%    Additional fixes:
%    \begin{itemize}
%    \item Unnecessary white space removed.
%    \item |\ifodd\thepage| changed to |\ifodd\value{page}|.
%    \end{itemize}
%    \begin{macrocode}
  \ifnum\ParallelBoolVar=\@ne
    \par
    \begingroup
      \global\ParallelWhichBox=\@ne
      \newpage
      \vbadness=10000 %
      \vfuzz=3ex %
      \splittopskip=\z@skip
      \loop%
        \ifnum\ParallelBoolVar=\@ne%
          \ifnum\ParallelWhichBox=\@ne
            \ifvoid\ParallelLBox
              \mbox{}%
              \newpage
            \else
              \global\ParallelWhichBox=\z@
            \fi
          \else
            \ifvoid\ParallelRBox
              \null
              \newpage
            \else
              \global\ParallelWhichBox=\@ne
            \fi
          \fi
          \ifnum\ParallelWhichBox=\z@
            \ifodd\value{page}%
              \null
              \newpage
            \fi
            \hbox to\textwidth{%
              \pcp@SetCurrent{Left}%
              \setbox\z@=\vsplit\ParallelLBox to.98\textheight
              \vbox to.98\textheight{%
                \@texttop
                \unvbox\z@
                \@textbottom
              }%
            }%
          \else
            \ifodd\value{page}%
            \else
              \mbox{}%
              \newpage
            \fi
            \hbox to\textwidth{%
              \pcp@SetCurrent{Right}%
              \setbox\z@=\vsplit\ParallelRBox to.98\textheight
              \vbox to.98\textheight{%
                \@texttop
                \unvbox\z@
                \@textbottom
              }%
            }%
          \fi
          \vspace*{\fill}%
          \newpage
        \fi
        \ifvoid\ParallelLBox
          \ifvoid\ParallelRBox
            \global\ParallelBoolVar=\z@
          \fi
        \fi
      \ifnum\ParallelBoolVar=\@ne
      \repeat
      \par
    \endgroup
    \pcp@SetCurrent{}%
  \fi
}
%    \end{macrocode}
%    \end{macro}
%
% \subsection{Color stack support}
%
%    \begin{macrocode}
\RequirePackage{pdfcol}[2007/12/12]
\ifpdfcolAvailable
\else
  \PackageInfo{pdfcolparallel}{%
    Loading aborted, because color stacks are not available%
  }%
  \expandafter\endinput
\fi
%    \end{macrocode}
%
%    \begin{macrocode}
\pdfcolInitStack{pcp@Left}
\pdfcolInitStack{pcp@Right}
%    \end{macrocode}
%    \begin{macro}{\pcp@Box}
%    \begin{macrocode}
\newbox\pcp@Box
%    \end{macrocode}
%    \end{macro}
%    \begin{macro}{\pcp@SwitchStack}
%    \begin{macrocode}
\def\pcp@SwitchStack#1#2{%
  \pdfcolSwitchStack{pcp@#1}%
  \global\setbox\pcp@Box=\vbox to 0pt{%
    \pdfcolSetCurrentColor
  }%
  \aftergroup\pcp@FixBox
  \aftergroup#2%
}
%    \end{macrocode}
%    \end{macro}
%    \begin{macro}{\pcp@FixBox}
%    \begin{macrocode}
\def\pcp@FixBox#1{%
  \global\setbox#1=\vbox{%
    \unvbox\pcp@Box
    \unvbox#1%
  }%
}
%    \end{macrocode}
%    \end{macro}
%    \begin{macro}{\pcp@SetCurrent}
%    \begin{macrocode}
\def\pcp@SetCurrent#1{%
  \ifx\\#1\\%
    \pdfcolSetCurrent{}%
  \else
    \pdfcolSetCurrent{pcp@#1}%
  \fi
}
%    \end{macrocode}
%    \end{macro}
%
% \subsection{Redefinitions}
%
%    \begin{macro}{\ParallelParOnePage}
%    \begin{macrocode}
\pcp@CmdCheckRedef\ParallelParOnePage{%
  \ifnum\ParallelBoolVar=\@ne
    \par
    \begingroup
      \leftmargin=\z@
      \rightmargin=\z@
      \parskip=\z@skip
      \parindent=\z@
      \vbadness=10000 %
      \vfuzz=3ex %
      \splittopskip=\z@skip
      \loop
        \ifnum\ParallelBoolVar=\@ne
          \noindent
          \hbox to\textwidth{%
            \hskip\ParallelLeftMargin
            \hbox to\ParallelTextWidth{%
              \ifvoid\ParallelLBox
                \hskip\ParallelLWidth
              \else
                \ParallelWhichBox=\z@
                \vbox{%
                  \setbox\ParallelBoxVar
                      =\vsplit\ParallelLBox to\dp\strutbox
                  \unvbox\ParallelBoxVar
                }%
              \fi
              \strut
              \ifnum\ParallelBoolMid=\@ne
                \hskip\ParallelMainMidSkip
                \vrule
              \else
                \hss
              \fi
              \hss
              \ifvoid\ParallelRBox
                \hskip\ParallelRWidth
              \else
                \ParallelWhichBox=\@ne
                \vbox{%
                  \setbox\ParallelBoxVar
                      =\vsplit\ParallelRBox to\dp\strutbox
                  \unvbox\ParallelBoxVar
                }%
              \fi
            }%
          }%
          \ifvoid\ParallelLBox
            \ifvoid\ParallelRBox
              \global\ParallelBoolVar=\z@
            \fi
          \fi%
        \fi%
      \ifnum\ParallelBoolVar=\@ne
        \penalty\interlinepenalty
      \repeat
      \par
    \endgroup
  \fi
}{%
  \ifnum\ParallelBoolVar=\@ne
    \par
    \begingroup
      \leftmargin=\z@
      \rightmargin=\z@
      \parskip=\z@skip
      \parindent=\z@
      \vbadness=10000 %
      \vfuzz=3ex %
      \splittopskip=\z@skip
      \loop
        \ifnum\ParallelBoolVar=\@ne
          \noindent
          \hbox to\textwidth{%
            \hskip\ParallelLeftMargin
            \hbox to\ParallelTextWidth{%
              \ifvoid\ParallelLBox
                \hskip\ParallelLWidth
              \else
                \pcp@SetCurrent{Left}%
                \ParallelWhichBox=\z@
                \vbox{%
                  \setbox\ParallelBoxVar
                      =\vsplit\ParallelLBox to\dp\strutbox
                  \unvbox\ParallelBoxVar
                }%
              \fi
              \strut
              \ifnum\ParallelBoolMid=\@ne
                \hskip\ParallelMainMidSkip
                \begingroup
                  \pcp@RuleBetweenColor
                  \vrule
                \endgroup
              \else
                \hss
              \fi
              \hss
              \ifvoid\ParallelRBox
                \hskip\ParallelRWidth
              \else
                \pcp@SetCurrent{Right}%
                \ParallelWhichBox=\@ne
                \vbox{%
                  \setbox\ParallelBoxVar
                      =\vsplit\ParallelRBox to\dp\strutbox
                  \unvbox\ParallelBoxVar
                }%
              \fi
            }%
          }%
          \ifvoid\ParallelLBox
            \ifvoid\ParallelRBox
              \global\ParallelBoolVar=\z@
            \fi
          \fi%
        \fi%
      \ifnum\ParallelBoolVar=\@ne
        \penalty\interlinepenalty
      \repeat
      \par
    \endgroup
    \pcp@SetCurrent{}%
  \fi
}
%    \end{macrocode}
%    \end{macro}
%    \begin{macro}{\pcp@RuleBetweenColorDefault}
%    \begin{macrocode}
\def\pcp@RuleBetweenColorDefault{%
  \normalcolor
}
%    \end{macrocode}
%    \end{macro}
%    \begin{macro}{\pcp@RuleBetweenColor}
%    \begin{macrocode}
\let\pcp@RuleBetweenColor\pcp@RuleBetweenColorDefault
%    \end{macrocode}
%    \end{macro}
%    \begin{macrocode}
\RequirePackage{keyval}
\define@key{parallel}{rulebetweencolor}{%
  \edef\pcp@temp{#1}%
  \ifx\pcp@temp\@empty
    \let\pcp@RuleBetweenColor\pcp@RuleBetweenColorDefault
  \else
    \edef\pcp@temp{%
      \noexpand\@ifnextchar[{%
        \def\noexpand\pcp@RuleBetweenColor{%
          \noexpand\color\pcp@temp
        }%
        \noexpand\pcp@GobbleNil
      }{%
        \def\noexpand\pcp@RuleBetweenColor{%
          \noexpand\color{\pcp@temp}%
        }%
        \noexpand\pcp@GobbleNil
      }%
      \pcp@temp\noexpand\@nil
    }%
    \pcp@temp
  \fi
}
%    \end{macrocode}
%    \begin{macro}{\pcp@GobbleNil}
%    \begin{macrocode}
\long\def\pcp@GobbleNil#1\@nil{}
%    \end{macrocode}
%    \end{macro}
%
%    \begin{macrocode}
%</package>
%    \end{macrocode}
%
% \section{Test}
%
%    The test file is a modified version of the file that
%    Alexander Hirsch has posted in \xnewsgroup{de.comp.text.tex}:
%    \URL{``\link{\texttt{parallel.sty} und farbiger Text}''}^^A
%    {http://groups.google.com/group/de.comp.text.tex/msg/6a759cf33bb071a5}
%    \begin{macrocode}
%<*test1>
\AtEndDocument{%
  \typeout{}%
  \typeout{**************************************}%
  \typeout{*** \space Check the PDF file manually! \space ***}%
  \typeout{**************************************}%
  \typeout{}%
}
\documentclass{article}
\usepackage{xcolor}
\usepackage{pdfcolparallel}[2016/05/16]

\begin{document}
  \color{green}%
  Green%
  \begin{Parallel}{0.47\textwidth}{0.47\textwidth}%
    \ParallelLText{%
      \textcolor{red}{%
        Ein Absatz, der sich ueber zwei Zeilen erstrecken soll. %
        Ein Absatz, der sich ueber zwei Zeilen erstrecken soll.%
      }%
    }%
    \ParallelRText{%
      \textcolor{blue}{%
        Ein Absatz, der sich ueber zwei Zeilen erstrecken soll. %
        Ein Absatz, der sich ueber zwei Zeilen erstrecken soll.%
      }%
    }%
    \ParallelPar
    \ParallelLText{%
      Default %
      \color{red}%
      Ein Absatz, der sich ueber zwei Zeilen erstrecken soll. %
      Ein Absatz, der sich ueber zwei Zeilen erstrecken soll.%
    }%
    \ParallelRText{%
      Default %
      \color{blue}%
      Ein Absatz, der sich ueber zwei Zeilen erstrecken soll. %
      Ein Absatz, der sich ueber zwei Zeilen erstrecken soll.%
    }%
    \ParallelPar
    \ParallelLText{%
      \begin{enumerate}%
      \item left text, left text, left text, left text, %
            left text, left text, left text, left text,%
      \item left text, left text, left text, left text, %
            left text, left text, left text, left text.%
      \end{enumerate}%
    }%
    \ParallelRText{%
      \begin{enumerate}%
      \item right text, right text, right text, right text, %
            right text, right text, right text, right text.%
      \item right text, right text, right text, right text, %
            right text, right text, right text, right text.%
      \end{enumerate}%
    }%
  \end{Parallel}%
  \begin{Parallel}[p]{\textwidth}{\textwidth}%
    \ParallelLText{%
      \textcolor{red}{%
        Ein Absatz, der sich ueber zwei Zeilen erstrecken soll. %
        Ein Absatz, der sich ueber zwei Zeilen erstrecken soll. %
        Foo bar bla bla bla.%
      }%
      \par
      Und noch ein Absatz.%
    }%
    \ParallelRText{%
      \textcolor{blue}{%
        Ein Absatz, der sich ueber zwei Zeilen erstrecken soll. %
        Ein Absatz, der sich ueber zwei Zeilen erstrecken soll. %
        Foo bar bla bla bla.%
      }%
    }%
  \end{Parallel}%
  \begin{Parallel}[p]{\textwidth}{\textwidth}%
    \ParallelLText{%
      \rule{1pt}{.98\textheight}\Huge g%
    }%
    \ParallelRText{%
      \rule{1pt}{.98\textheight}y%
    }%
  \end{Parallel}%
  Green%
\end{document}
%</test1>
%    \end{macrocode}
%
% \section{Installation}
%
% \subsection{Download}
%
% \paragraph{Package.} This package is available on
% CTAN\footnote{\url{http://ctan.org/pkg/pdfcolparallel}}:
% \begin{description}
% \item[\CTAN{macros/latex/contrib/oberdiek/pdfcolparallel.dtx}] The source file.
% \item[\CTAN{macros/latex/contrib/oberdiek/pdfcolparallel.pdf}] Documentation.
% \end{description}
%
%
% \paragraph{Bundle.} All the packages of the bundle `oberdiek'
% are also available in a TDS compliant ZIP archive. There
% the packages are already unpacked and the documentation files
% are generated. The files and directories obey the TDS standard.
% \begin{description}
% \item[\CTAN{install/macros/latex/contrib/oberdiek.tds.zip}]
% \end{description}
% \emph{TDS} refers to the standard ``A Directory Structure
% for \TeX\ Files'' (\CTAN{tds/tds.pdf}). Directories
% with \xfile{texmf} in their name are usually organized this way.
%
% \subsection{Bundle installation}
%
% \paragraph{Unpacking.} Unpack the \xfile{oberdiek.tds.zip} in the
% TDS tree (also known as \xfile{texmf} tree) of your choice.
% Example (linux):
% \begin{quote}
%   |unzip oberdiek.tds.zip -d ~/texmf|
% \end{quote}
%
% \paragraph{Script installation.}
% Check the directory \xfile{TDS:scripts/oberdiek/} for
% scripts that need further installation steps.
% Package \xpackage{attachfile2} comes with the Perl script
% \xfile{pdfatfi.pl} that should be installed in such a way
% that it can be called as \texttt{pdfatfi}.
% Example (linux):
% \begin{quote}
%   |chmod +x scripts/oberdiek/pdfatfi.pl|\\
%   |cp scripts/oberdiek/pdfatfi.pl /usr/local/bin/|
% \end{quote}
%
% \subsection{Package installation}
%
% \paragraph{Unpacking.} The \xfile{.dtx} file is a self-extracting
% \docstrip\ archive. The files are extracted by running the
% \xfile{.dtx} through \plainTeX:
% \begin{quote}
%   \verb|tex pdfcolparallel.dtx|
% \end{quote}
%
% \paragraph{TDS.} Now the different files must be moved into
% the different directories in your installation TDS tree
% (also known as \xfile{texmf} tree):
% \begin{quote}
% \def\t{^^A
% \begin{tabular}{@{}>{\ttfamily}l@{ $\rightarrow$ }>{\ttfamily}l@{}}
%   pdfcolparallel.sty & tex/latex/oberdiek/pdfcolparallel.sty\\
%   pdfcolparallel.pdf & doc/latex/oberdiek/pdfcolparallel.pdf\\
%   test/pdfcolparallel-test1.tex & doc/latex/oberdiek/test/pdfcolparallel-test1.tex\\
%   pdfcolparallel.dtx & source/latex/oberdiek/pdfcolparallel.dtx\\
% \end{tabular}^^A
% }^^A
% \sbox0{\t}^^A
% \ifdim\wd0>\linewidth
%   \begingroup
%     \advance\linewidth by\leftmargin
%     \advance\linewidth by\rightmargin
%   \edef\x{\endgroup
%     \def\noexpand\lw{\the\linewidth}^^A
%   }\x
%   \def\lwbox{^^A
%     \leavevmode
%     \hbox to \linewidth{^^A
%       \kern-\leftmargin\relax
%       \hss
%       \usebox0
%       \hss
%       \kern-\rightmargin\relax
%     }^^A
%   }^^A
%   \ifdim\wd0>\lw
%     \sbox0{\small\t}^^A
%     \ifdim\wd0>\linewidth
%       \ifdim\wd0>\lw
%         \sbox0{\footnotesize\t}^^A
%         \ifdim\wd0>\linewidth
%           \ifdim\wd0>\lw
%             \sbox0{\scriptsize\t}^^A
%             \ifdim\wd0>\linewidth
%               \ifdim\wd0>\lw
%                 \sbox0{\tiny\t}^^A
%                 \ifdim\wd0>\linewidth
%                   \lwbox
%                 \else
%                   \usebox0
%                 \fi
%               \else
%                 \lwbox
%               \fi
%             \else
%               \usebox0
%             \fi
%           \else
%             \lwbox
%           \fi
%         \else
%           \usebox0
%         \fi
%       \else
%         \lwbox
%       \fi
%     \else
%       \usebox0
%     \fi
%   \else
%     \lwbox
%   \fi
% \else
%   \usebox0
% \fi
% \end{quote}
% If you have a \xfile{docstrip.cfg} that configures and enables \docstrip's
% TDS installing feature, then some files can already be in the right
% place, see the documentation of \docstrip.
%
% \subsection{Refresh file name databases}
%
% If your \TeX~distribution
% (\teTeX, \mikTeX, \dots) relies on file name databases, you must refresh
% these. For example, \teTeX\ users run \verb|texhash| or
% \verb|mktexlsr|.
%
% \subsection{Some details for the interested}
%
% \paragraph{Attached source.}
%
% The PDF documentation on CTAN also includes the
% \xfile{.dtx} source file. It can be extracted by
% AcrobatReader 6 or higher. Another option is \textsf{pdftk},
% e.g. unpack the file into the current directory:
% \begin{quote}
%   \verb|pdftk pdfcolparallel.pdf unpack_files output .|
% \end{quote}
%
% \paragraph{Unpacking with \LaTeX.}
% The \xfile{.dtx} chooses its action depending on the format:
% \begin{description}
% \item[\plainTeX:] Run \docstrip\ and extract the files.
% \item[\LaTeX:] Generate the documentation.
% \end{description}
% If you insist on using \LaTeX\ for \docstrip\ (really,
% \docstrip\ does not need \LaTeX), then inform the autodetect routine
% about your intention:
% \begin{quote}
%   \verb|latex \let\install=y\input{pdfcolparallel.dtx}|
% \end{quote}
% Do not forget to quote the argument according to the demands
% of your shell.
%
% \paragraph{Generating the documentation.}
% You can use both the \xfile{.dtx} or the \xfile{.drv} to generate
% the documentation. The process can be configured by the
% configuration file \xfile{ltxdoc.cfg}. For instance, put this
% line into this file, if you want to have A4 as paper format:
% \begin{quote}
%   \verb|\PassOptionsToClass{a4paper}{article}|
% \end{quote}
% An example follows how to generate the
% documentation with pdf\LaTeX:
% \begin{quote}
%\begin{verbatim}
%pdflatex pdfcolparallel.dtx
%makeindex -s gind.ist pdfcolparallel.idx
%pdflatex pdfcolparallel.dtx
%makeindex -s gind.ist pdfcolparallel.idx
%pdflatex pdfcolparallel.dtx
%\end{verbatim}
% \end{quote}
%
% \section{Catalogue}
%
% The following XML file can be used as source for the
% \href{http://mirror.ctan.org/help/Catalogue/catalogue.html}{\TeX\ Catalogue}.
% The elements \texttt{caption} and \texttt{description} are imported
% from the original XML file from the Catalogue.
% The name of the XML file in the Catalogue is \xfile{pdfcolparallel.xml}.
%    \begin{macrocode}
%<*catalogue>
<?xml version='1.0' encoding='us-ascii'?>
<!DOCTYPE entry SYSTEM 'catalogue.dtd'>
<entry datestamp='$Date$' modifier='$Author$' id='pdfcolparallel'>
  <name>pdfcolparallel</name>
  <caption>Fix colour problems in package 'parallel'.</caption>
  <authorref id='auth:oberdiek'/>
  <copyright owner='Heiko Oberdiek' year='2007,2008,2010'/>
  <license type='lppl1.3'/>
  <version number='1.4'/>
  <description>
    Since version 1.40 pdfTeX supports colour stacks.
    This package uses them to fix colour problems in
    package <xref refid='parallel'>parallel</xref>.
    <p/>
    The package is part of the <xref refid='oberdiek'>oberdiek</xref>
    bundle.
  </description>
  <documentation details='Package documentation'
      href='ctan:/macros/latex/contrib/oberdiek/pdfcolparallel.pdf'/>
  <ctan file='true' path='/macros/latex/contrib/oberdiek/pdfcolparallel.dtx'/>
  <miktex location='oberdiek'/>
  <texlive location='oberdiek'/>
  <install path='/macros/latex/contrib/oberdiek/oberdiek.tds.zip'/>
</entry>
%</catalogue>
%    \end{macrocode}
%
% \begin{thebibliography}{9}
%
% \bibitem{parallel}
%   Matthias Eckermann: \textit{The \xpackage{parallel}-package};
%   2003/04/13;\\
%   \CTAN{macros/latex/contrib/parallel/}.
%
% \bibitem{pdfcol}
%   Heiko Oberdiek: \textit{The \xpackage{pdfcol} package};
%   2007/09/09;\\
%   \CTAN{macros/latex/contrib/oberdiek/pdfcol.pdf}.
%
% \end{thebibliography}
%
% \begin{History}
%   \begin{Version}{2007/09/09 v1.0}
%   \item
%     First version.
%   \end{Version}
%   \begin{Version}{2007/12/12 v1.1}
%   \item
%     Adds patch for setting \cs{linewidth} to fix bug
%     in package \xpackage{parallel}.
%   \item
%     Package \xpackage{parallel} is also fixed if color
%     stacks are not available.
%   \item
%     Bug fix, switched stacks now initialized with current color.
%   \item
%     Fix for package \xpackage{parallel}: \cs{raggedbottom} is respected.
%   \end{Version}
%   \begin{Version}{2008/08/11 v1.2}
%   \item
%     Code is not changed.
%   \item
%     URLs updated.
%   \end{Version}
%   \begin{Version}{2010/01/11 v1.3}
%   \item
%     Option `rulebetweencolor' added.
%   \end{Version}
%   \begin{Version}{2016/05/16 v1.4}
%   \item
%     Documentation updates.
%   \end{Version}
% \end{History}
%
% \PrintIndex
%
% \Finale
\endinput

%        (quote the arguments according to the demands of your shell)
%
% Documentation:
%    (a) If pdfcolparallel.drv is present:
%           latex pdfcolparallel.drv
%    (b) Without pdfcolparallel.drv:
%           latex pdfcolparallel.dtx; ...
%    The class ltxdoc loads the configuration file ltxdoc.cfg
%    if available. Here you can specify further options, e.g.
%    use A4 as paper format:
%       \PassOptionsToClass{a4paper}{article}
%
%    Programm calls to get the documentation (example):
%       pdflatex pdfcolparallel.dtx
%       makeindex -s gind.ist pdfcolparallel.idx
%       pdflatex pdfcolparallel.dtx
%       makeindex -s gind.ist pdfcolparallel.idx
%       pdflatex pdfcolparallel.dtx
%
% Installation:
%    TDS:tex/latex/oberdiek/pdfcolparallel.sty
%    TDS:doc/latex/oberdiek/pdfcolparallel.pdf
%    TDS:doc/latex/oberdiek/test/pdfcolparallel-test1.tex
%    TDS:source/latex/oberdiek/pdfcolparallel.dtx
%
%<*ignore>
\begingroup
  \catcode123=1 %
  \catcode125=2 %
  \def\x{LaTeX2e}%
\expandafter\endgroup
\ifcase 0\ifx\install y1\fi\expandafter
         \ifx\csname processbatchFile\endcsname\relax\else1\fi
         \ifx\fmtname\x\else 1\fi\relax
\else\csname fi\endcsname
%</ignore>
%<*install>
\input docstrip.tex
\Msg{************************************************************************}
\Msg{* Installation}
\Msg{* Package: pdfcolparallel 2016/05/16 v1.4 Color stacks support for parallel (HO)}
\Msg{************************************************************************}

\keepsilent
\askforoverwritefalse

\let\MetaPrefix\relax
\preamble

This is a generated file.

Project: pdfcolparallel
Version: 2016/05/16 v1.4

Copyright (C) 2007, 2008, 2010 by
   Heiko Oberdiek <heiko.oberdiek at googlemail.com>

This work may be distributed and/or modified under the
conditions of the LaTeX Project Public License, either
version 1.3c of this license or (at your option) any later
version. This version of this license is in
   http://www.latex-project.org/lppl/lppl-1-3c.txt
and the latest version of this license is in
   http://www.latex-project.org/lppl.txt
and version 1.3 or later is part of all distributions of
LaTeX version 2005/12/01 or later.

This work has the LPPL maintenance status "maintained".

This Current Maintainer of this work is Heiko Oberdiek.

This work consists of the main source file pdfcolparallel.dtx
and the derived files
   pdfcolparallel.sty, pdfcolparallel.pdf, pdfcolparallel.ins,
   pdfcolparallel.drv, pdfcolparallel-test1.tex.

\endpreamble
\let\MetaPrefix\DoubleperCent

\generate{%
  \file{pdfcolparallel.ins}{\from{pdfcolparallel.dtx}{install}}%
  \file{pdfcolparallel.drv}{\from{pdfcolparallel.dtx}{driver}}%
  \usedir{tex/latex/oberdiek}%
  \file{pdfcolparallel.sty}{\from{pdfcolparallel.dtx}{package}}%
  \usedir{doc/latex/oberdiek/test}%
  \file{pdfcolparallel-test1.tex}{\from{pdfcolparallel.dtx}{test1}}%
  \nopreamble
  \nopostamble
  \usedir{source/latex/oberdiek/catalogue}%
  \file{pdfcolparallel.xml}{\from{pdfcolparallel.dtx}{catalogue}}%
}

\catcode32=13\relax% active space
\let =\space%
\Msg{************************************************************************}
\Msg{*}
\Msg{* To finish the installation you have to move the following}
\Msg{* file into a directory searched by TeX:}
\Msg{*}
\Msg{*     pdfcolparallel.sty}
\Msg{*}
\Msg{* To produce the documentation run the file `pdfcolparallel.drv'}
\Msg{* through LaTeX.}
\Msg{*}
\Msg{* Happy TeXing!}
\Msg{*}
\Msg{************************************************************************}

\endbatchfile
%</install>
%<*ignore>
\fi
%</ignore>
%<*driver>
\NeedsTeXFormat{LaTeX2e}
\ProvidesFile{pdfcolparallel.drv}%
  [2016/05/16 v1.4 Color stacks support for parallel (HO)]%
\documentclass{ltxdoc}
\usepackage{holtxdoc}[2011/11/22]
\begin{document}
  \DocInput{pdfcolparallel.dtx}%
\end{document}
%</driver>
% \fi
%
%
% \CharacterTable
%  {Upper-case    \A\B\C\D\E\F\G\H\I\J\K\L\M\N\O\P\Q\R\S\T\U\V\W\X\Y\Z
%   Lower-case    \a\b\c\d\e\f\g\h\i\j\k\l\m\n\o\p\q\r\s\t\u\v\w\x\y\z
%   Digits        \0\1\2\3\4\5\6\7\8\9
%   Exclamation   \!     Double quote  \"     Hash (number) \#
%   Dollar        \$     Percent       \%     Ampersand     \&
%   Acute accent  \'     Left paren    \(     Right paren   \)
%   Asterisk      \*     Plus          \+     Comma         \,
%   Minus         \-     Point         \.     Solidus       \/
%   Colon         \:     Semicolon     \;     Less than     \<
%   Equals        \=     Greater than  \>     Question mark \?
%   Commercial at \@     Left bracket  \[     Backslash     \\
%   Right bracket \]     Circumflex    \^     Underscore    \_
%   Grave accent  \`     Left brace    \{     Vertical bar  \|
%   Right brace   \}     Tilde         \~}
%
% \GetFileInfo{pdfcolparallel.drv}
%
% \title{The \xpackage{pdfcolparallel} package}
% \date{2016/05/16 v1.4}
% \author{Heiko Oberdiek\thanks
% {Please report any issues at https://github.com/ho-tex/oberdiek/issues}\\
% \xemail{heiko.oberdiek at googlemail.com}}
%
% \maketitle
%
% \begin{abstract}
% This packages fixes bugs in \xpackage{parallel} and
% improves color support by using several color stacks
% that are provided by \pdfTeX\ since version 1.40.
% \end{abstract}
%
% \tableofcontents
%
% \section{Usage}
%
% \begin{quote}
% |\usepackage{pdfcolparallel}|
% \end{quote}
% The package \xpackage{pdfcolparallel} loads package \xpackage{parallel}
% \cite{parallel} and redefines some macros to fix bugs.
%
% If color stacks are available then package \xpackage{parallel}
% is further patched to support them.
%
% \subsection{Option \xoption{rulebetweencolor}}
%
% Package \xpackage{pdfcolparallel} also fixes the color for the
% rule between columns.
% Default color is \cs{normalcolor}. But this can be changed by using
% option \xoption{rulebetweencolor} for |\setkeys{parallel}|
% (see package \xpackage{keyval}). The option takes a color specification
% as value. If the value is empty, then the default (\cs{normalcolor})
% is used.
% Examples:
% \begin{quote}
%   |\setkeys{parallel}{rulebetweencolor=blue}|,\\
%   |\setkeys{parallel}{rulebetweencolor={red}}|,\\
%   |\setkeys{parallel}{rulebetweencolor={}}|,
%     \textit{\% \cs{normalcolor} is used}\\
%   |\setkeys{parallel}{rulebetweencolor=[rgb]{1,0,.5}}|
% \end{quote}
%
% \subsection{Future}
%
% If there will be a new version of package \xpackage{parallel}
% that adds support for color stacks, then this package may become
% obsolete.
%
% \StopEventually{
% }
%
% \section{Implementation}
%
% \subsection{Identification}
%
%    \begin{macrocode}
%<*package>
\NeedsTeXFormat{LaTeX2e}
\ProvidesPackage{pdfcolparallel}%
  [2016/05/16 v1.4 Color stacks support for parallel (HO)]%
%    \end{macrocode}
%
% \subsection{Load and fix package \xpackage{parallel}}
%
%    Package \xpackage{parallel} is loaded. Before options of package
%    \xpackage{pdfcolparallel} are passed to package \xpackage{parallel}.
%    \begin{macrocode}
\DeclareOption*{%
  \PassoptionsToPackage{\CurrentOption}{parallel}%
}
\ProcessOptions\relax
\RequirePackage{parallel}[2003/04/13]
%    \end{macrocode}
%
%    \begin{macrocode}
\RequirePackage{infwarerr}[2007/09/09]
%    \end{macrocode}
%
%    \begin{macro}{\pcp@ColorPatch}
%    \begin{macrocode}
\begingroup\expandafter\expandafter\expandafter\endgroup
\expandafter\ifx\csname currentgrouplevel\endcsname\relax
  \def\pcp@ColorPatch{}%
\else
  \def\pcp@ColorPatch{%
    \@ifundefined{set@color}{%
      \gdef\pcp@ColorPatch{}%
    }{%
      \gdef\pcp@ColorPatch{%
        \gdef\pcp@ColorResets{}%
        \bgroup
        \aftergroup\pcp@ColorResets
        \aftergroup\egroup
        \let\pcp@OrgSetColor\set@color
        \let\set@color\pcp@SetColor
        \edef\pcp@GroupLevel{\the\currentgrouplevel}%
      }%
    }%
    \pcp@ColorPatch
  }%
%    \end{macrocode}
%    \end{macro}
%    \begin{macro}{\pcp@SetColor}
%    \begin{macrocode}
  \def\pcp@SetColor{%
    \ifnum\pcp@GroupLevel=\currentgrouplevel
      \let\pcp@OrgAfterGroup\aftergroup
      \def\aftergroup{%
        \g@addto@macro\pcp@ColorResets
      }%
      \pcp@OrgSetColor
      \let\aftergroup\pcp@OrgAfterGroup
    \else
      \pcp@OrgSetColor
    \fi
  }%
\fi
%    \end{macrocode}
%    \end{macro}
%
%    \begin{macro}{\pcp@CmdCheckRedef}
%    \begin{macrocode}
\def\pcp@CmdCheckRedef#1{%
  \begingroup
    \def\pcp@cmd{#1}%
    \afterassignment\pcp@CmdDo
    \long\def\reserved@a
}
\def\pcp@CmdDo{%
    \expandafter\ifx\pcp@cmd\reserved@a
    \else
      \edef\x*{\expandafter\string\pcp@cmd}%
      \@PackageWarningNoLine{pdfcolparallel}{%
        Command \x* has changed.\MessageBreak
        Supported versions of package `parallel':\MessageBreak
        \space\space 2003/04/13\MessageBreak
        The redefinition of \x* may\MessageBreak
        not behave correctly depending on the changes%
      }%
    \fi
  \expandafter\endgroup
  \expandafter\def\pcp@cmd
}
%    \end{macrocode}
%    \end{macro}
%
%    \begin{macrocode}
\def\pcp@SwitchStack#1#2{}
%    \end{macrocode}
%    \begin{macrocode}
\def\pcp@SetCurrent#1{}
%    \end{macrocode}
%
%    \begin{macro}{\ParallelLText}
%    \begin{macrocode}
\pcp@CmdCheckRedef\ParallelLText{%
  \everypar{}%
  \@restorepar
  \begingroup
    \hbadness=3000 %
    \let\footnote=\ParallelLFootnote
    \ParallelWhichBox=0 %
    \global\setbox\ParallelLBox=\vbox\bgroup
      \hsize=\ParallelLWidth
      \aftergroup\ParallelAfterText
      \begingroup
        \afterassignment\ParallelCheckOpenBrace
        \let\x=%
}{%
  \everypar{}%
  \@restorepar
  \@nobreakfalse
  \begingroup
    \hbadness=3000 %
    \let\footnote=\ParallelLFootnote
    \ParallelWhichBox=0 %
    \global\setbox\ParallelLBox=\vbox\bgroup
      \hsize=\ParallelLWidth
      \linewidth=\ParallelLWidth
      \pcp@SwitchStack{Left}\ParallelLBox
      \aftergroup\ParallelAfterText
      \pcp@ColorPatch
      \begingroup
        \afterassignment\ParallelCheckOpenBrace
        \let\x=%
}
%    \end{macrocode}
%    \end{macro}
%
%    \begin{macro}{\ParallelRText}
%    \begin{macrocode}
\pcp@CmdCheckRedef\ParallelRText{%
  \everypar{}%
  \@restorepar
  \begingroup
    \hbadness=3000 %
    \ifnum\ParallelFNMode=\@ne
      \let\footnote=\ParallelRFootnote
    \else
      \let\footnote=\ParallelLFootnote
    \fi
    \ParallelWhichBox=\@ne
    \global\setbox\ParallelRBox=\vbox\bgroup
      \hsize=\ParallelRWidth
      \aftergroup\ParallelAfterText
      \begingroup
        \afterassignment\ParallelCheckOpenBrace
        \let\x=%
}{%
  \everypar{}%
  \@restorepar
  \@nobreakfalse
  \begingroup
    \hbadness=3000 %
    \ifnum\ParallelFNMode=\@ne
      \let\footnote=\ParallelRFootnote
    \else
      \let\footnote=\ParallelLFootnote
    \fi
    \ParallelWhichBox=\@ne
    \global\setbox\ParallelRBox=\vbox\bgroup
      \hsize=\ParallelRWidth
      \linewidth=\ParallelRWidth
      \pcp@SwitchStack{Right}\ParallelRBox
      \aftergroup\ParallelAfterText
      \pcp@ColorPatch
      \begingroup
        \afterassignment\ParallelCheckOpenBrace
        \let\x=%
}
%    \end{macrocode}
%    \end{macro}
%
%    \begin{macro}{\ParallelParTwoPages}
%    \begin{macrocode}
\pcp@CmdCheckRedef\ParallelParTwoPages{%
  \ifnum\ParallelBoolVar=\@ne
    \par
    \begingroup
      \global\ParallelWhichBox=\@ne
      \newpage
      \vbadness=10000 %
      \vfuzz=3ex %
      \splittopskip=\z@skip
      \loop%
        \ifnum\ParallelBoolVar=\@ne%
          \ifnum\ParallelWhichBox=\@ne
            \ifvoid\ParallelLBox
              \mbox{} %
              \newpage
            \else
              \global\ParallelWhichBox=\z@
            \fi
          \else
            \ifvoid\ParallelRBox
              \mbox{} %
              \newpage
            \else
              \global\ParallelWhichBox=\@ne
            \fi
          \fi
          \ifnum\ParallelWhichBox=\z@
            \ifodd\thepage
              \mbox{} %
              \newpage
            \fi
            \hbox to\textwidth{%
              \vbox{\vsplit\ParallelLBox to.98\textheight}%
            }%
          \else
            \ifodd\thepage\relax
            \else
              \mbox{} %
              \newpage
            \fi
            \hbox to\textwidth{%
              \vbox{\vsplit\ParallelRBox to.98\textheight}%
            }%
          \fi
          \vspace*{\fill}%
          \newpage
        \fi
        \ifvoid\ParallelLBox
          \ifvoid\ParallelRBox
            \global\ParallelBoolVar=\z@
          \fi
        \fi
      \ifnum\ParallelBoolVar=\@ne
      \repeat
      \par
    \endgroup
  \fi
}{%
%    \end{macrocode}
%    Additional fixes:
%    \begin{itemize}
%    \item Unnecessary white space removed.
%    \item |\ifodd\thepage| changed to |\ifodd\value{page}|.
%    \end{itemize}
%    \begin{macrocode}
  \ifnum\ParallelBoolVar=\@ne
    \par
    \begingroup
      \global\ParallelWhichBox=\@ne
      \newpage
      \vbadness=10000 %
      \vfuzz=3ex %
      \splittopskip=\z@skip
      \loop%
        \ifnum\ParallelBoolVar=\@ne%
          \ifnum\ParallelWhichBox=\@ne
            \ifvoid\ParallelLBox
              \mbox{}%
              \newpage
            \else
              \global\ParallelWhichBox=\z@
            \fi
          \else
            \ifvoid\ParallelRBox
              \null
              \newpage
            \else
              \global\ParallelWhichBox=\@ne
            \fi
          \fi
          \ifnum\ParallelWhichBox=\z@
            \ifodd\value{page}%
              \null
              \newpage
            \fi
            \hbox to\textwidth{%
              \pcp@SetCurrent{Left}%
              \setbox\z@=\vsplit\ParallelLBox to.98\textheight
              \vbox to.98\textheight{%
                \@texttop
                \unvbox\z@
                \@textbottom
              }%
            }%
          \else
            \ifodd\value{page}%
            \else
              \mbox{}%
              \newpage
            \fi
            \hbox to\textwidth{%
              \pcp@SetCurrent{Right}%
              \setbox\z@=\vsplit\ParallelRBox to.98\textheight
              \vbox to.98\textheight{%
                \@texttop
                \unvbox\z@
                \@textbottom
              }%
            }%
          \fi
          \vspace*{\fill}%
          \newpage
        \fi
        \ifvoid\ParallelLBox
          \ifvoid\ParallelRBox
            \global\ParallelBoolVar=\z@
          \fi
        \fi
      \ifnum\ParallelBoolVar=\@ne
      \repeat
      \par
    \endgroup
    \pcp@SetCurrent{}%
  \fi
}
%    \end{macrocode}
%    \end{macro}
%
% \subsection{Color stack support}
%
%    \begin{macrocode}
\RequirePackage{pdfcol}[2007/12/12]
\ifpdfcolAvailable
\else
  \PackageInfo{pdfcolparallel}{%
    Loading aborted, because color stacks are not available%
  }%
  \expandafter\endinput
\fi
%    \end{macrocode}
%
%    \begin{macrocode}
\pdfcolInitStack{pcp@Left}
\pdfcolInitStack{pcp@Right}
%    \end{macrocode}
%    \begin{macro}{\pcp@Box}
%    \begin{macrocode}
\newbox\pcp@Box
%    \end{macrocode}
%    \end{macro}
%    \begin{macro}{\pcp@SwitchStack}
%    \begin{macrocode}
\def\pcp@SwitchStack#1#2{%
  \pdfcolSwitchStack{pcp@#1}%
  \global\setbox\pcp@Box=\vbox to 0pt{%
    \pdfcolSetCurrentColor
  }%
  \aftergroup\pcp@FixBox
  \aftergroup#2%
}
%    \end{macrocode}
%    \end{macro}
%    \begin{macro}{\pcp@FixBox}
%    \begin{macrocode}
\def\pcp@FixBox#1{%
  \global\setbox#1=\vbox{%
    \unvbox\pcp@Box
    \unvbox#1%
  }%
}
%    \end{macrocode}
%    \end{macro}
%    \begin{macro}{\pcp@SetCurrent}
%    \begin{macrocode}
\def\pcp@SetCurrent#1{%
  \ifx\\#1\\%
    \pdfcolSetCurrent{}%
  \else
    \pdfcolSetCurrent{pcp@#1}%
  \fi
}
%    \end{macrocode}
%    \end{macro}
%
% \subsection{Redefinitions}
%
%    \begin{macro}{\ParallelParOnePage}
%    \begin{macrocode}
\pcp@CmdCheckRedef\ParallelParOnePage{%
  \ifnum\ParallelBoolVar=\@ne
    \par
    \begingroup
      \leftmargin=\z@
      \rightmargin=\z@
      \parskip=\z@skip
      \parindent=\z@
      \vbadness=10000 %
      \vfuzz=3ex %
      \splittopskip=\z@skip
      \loop
        \ifnum\ParallelBoolVar=\@ne
          \noindent
          \hbox to\textwidth{%
            \hskip\ParallelLeftMargin
            \hbox to\ParallelTextWidth{%
              \ifvoid\ParallelLBox
                \hskip\ParallelLWidth
              \else
                \ParallelWhichBox=\z@
                \vbox{%
                  \setbox\ParallelBoxVar
                      =\vsplit\ParallelLBox to\dp\strutbox
                  \unvbox\ParallelBoxVar
                }%
              \fi
              \strut
              \ifnum\ParallelBoolMid=\@ne
                \hskip\ParallelMainMidSkip
                \vrule
              \else
                \hss
              \fi
              \hss
              \ifvoid\ParallelRBox
                \hskip\ParallelRWidth
              \else
                \ParallelWhichBox=\@ne
                \vbox{%
                  \setbox\ParallelBoxVar
                      =\vsplit\ParallelRBox to\dp\strutbox
                  \unvbox\ParallelBoxVar
                }%
              \fi
            }%
          }%
          \ifvoid\ParallelLBox
            \ifvoid\ParallelRBox
              \global\ParallelBoolVar=\z@
            \fi
          \fi%
        \fi%
      \ifnum\ParallelBoolVar=\@ne
        \penalty\interlinepenalty
      \repeat
      \par
    \endgroup
  \fi
}{%
  \ifnum\ParallelBoolVar=\@ne
    \par
    \begingroup
      \leftmargin=\z@
      \rightmargin=\z@
      \parskip=\z@skip
      \parindent=\z@
      \vbadness=10000 %
      \vfuzz=3ex %
      \splittopskip=\z@skip
      \loop
        \ifnum\ParallelBoolVar=\@ne
          \noindent
          \hbox to\textwidth{%
            \hskip\ParallelLeftMargin
            \hbox to\ParallelTextWidth{%
              \ifvoid\ParallelLBox
                \hskip\ParallelLWidth
              \else
                \pcp@SetCurrent{Left}%
                \ParallelWhichBox=\z@
                \vbox{%
                  \setbox\ParallelBoxVar
                      =\vsplit\ParallelLBox to\dp\strutbox
                  \unvbox\ParallelBoxVar
                }%
              \fi
              \strut
              \ifnum\ParallelBoolMid=\@ne
                \hskip\ParallelMainMidSkip
                \begingroup
                  \pcp@RuleBetweenColor
                  \vrule
                \endgroup
              \else
                \hss
              \fi
              \hss
              \ifvoid\ParallelRBox
                \hskip\ParallelRWidth
              \else
                \pcp@SetCurrent{Right}%
                \ParallelWhichBox=\@ne
                \vbox{%
                  \setbox\ParallelBoxVar
                      =\vsplit\ParallelRBox to\dp\strutbox
                  \unvbox\ParallelBoxVar
                }%
              \fi
            }%
          }%
          \ifvoid\ParallelLBox
            \ifvoid\ParallelRBox
              \global\ParallelBoolVar=\z@
            \fi
          \fi%
        \fi%
      \ifnum\ParallelBoolVar=\@ne
        \penalty\interlinepenalty
      \repeat
      \par
    \endgroup
    \pcp@SetCurrent{}%
  \fi
}
%    \end{macrocode}
%    \end{macro}
%    \begin{macro}{\pcp@RuleBetweenColorDefault}
%    \begin{macrocode}
\def\pcp@RuleBetweenColorDefault{%
  \normalcolor
}
%    \end{macrocode}
%    \end{macro}
%    \begin{macro}{\pcp@RuleBetweenColor}
%    \begin{macrocode}
\let\pcp@RuleBetweenColor\pcp@RuleBetweenColorDefault
%    \end{macrocode}
%    \end{macro}
%    \begin{macrocode}
\RequirePackage{keyval}
\define@key{parallel}{rulebetweencolor}{%
  \edef\pcp@temp{#1}%
  \ifx\pcp@temp\@empty
    \let\pcp@RuleBetweenColor\pcp@RuleBetweenColorDefault
  \else
    \edef\pcp@temp{%
      \noexpand\@ifnextchar[{%
        \def\noexpand\pcp@RuleBetweenColor{%
          \noexpand\color\pcp@temp
        }%
        \noexpand\pcp@GobbleNil
      }{%
        \def\noexpand\pcp@RuleBetweenColor{%
          \noexpand\color{\pcp@temp}%
        }%
        \noexpand\pcp@GobbleNil
      }%
      \pcp@temp\noexpand\@nil
    }%
    \pcp@temp
  \fi
}
%    \end{macrocode}
%    \begin{macro}{\pcp@GobbleNil}
%    \begin{macrocode}
\long\def\pcp@GobbleNil#1\@nil{}
%    \end{macrocode}
%    \end{macro}
%
%    \begin{macrocode}
%</package>
%    \end{macrocode}
%
% \section{Test}
%
%    The test file is a modified version of the file that
%    Alexander Hirsch has posted in \xnewsgroup{de.comp.text.tex}:
%    \URL{``\link{\texttt{parallel.sty} und farbiger Text}''}^^A
%    {http://groups.google.com/group/de.comp.text.tex/msg/6a759cf33bb071a5}
%    \begin{macrocode}
%<*test1>
\AtEndDocument{%
  \typeout{}%
  \typeout{**************************************}%
  \typeout{*** \space Check the PDF file manually! \space ***}%
  \typeout{**************************************}%
  \typeout{}%
}
\documentclass{article}
\usepackage{xcolor}
\usepackage{pdfcolparallel}[2016/05/16]

\begin{document}
  \color{green}%
  Green%
  \begin{Parallel}{0.47\textwidth}{0.47\textwidth}%
    \ParallelLText{%
      \textcolor{red}{%
        Ein Absatz, der sich ueber zwei Zeilen erstrecken soll. %
        Ein Absatz, der sich ueber zwei Zeilen erstrecken soll.%
      }%
    }%
    \ParallelRText{%
      \textcolor{blue}{%
        Ein Absatz, der sich ueber zwei Zeilen erstrecken soll. %
        Ein Absatz, der sich ueber zwei Zeilen erstrecken soll.%
      }%
    }%
    \ParallelPar
    \ParallelLText{%
      Default %
      \color{red}%
      Ein Absatz, der sich ueber zwei Zeilen erstrecken soll. %
      Ein Absatz, der sich ueber zwei Zeilen erstrecken soll.%
    }%
    \ParallelRText{%
      Default %
      \color{blue}%
      Ein Absatz, der sich ueber zwei Zeilen erstrecken soll. %
      Ein Absatz, der sich ueber zwei Zeilen erstrecken soll.%
    }%
    \ParallelPar
    \ParallelLText{%
      \begin{enumerate}%
      \item left text, left text, left text, left text, %
            left text, left text, left text, left text,%
      \item left text, left text, left text, left text, %
            left text, left text, left text, left text.%
      \end{enumerate}%
    }%
    \ParallelRText{%
      \begin{enumerate}%
      \item right text, right text, right text, right text, %
            right text, right text, right text, right text.%
      \item right text, right text, right text, right text, %
            right text, right text, right text, right text.%
      \end{enumerate}%
    }%
  \end{Parallel}%
  \begin{Parallel}[p]{\textwidth}{\textwidth}%
    \ParallelLText{%
      \textcolor{red}{%
        Ein Absatz, der sich ueber zwei Zeilen erstrecken soll. %
        Ein Absatz, der sich ueber zwei Zeilen erstrecken soll. %
        Foo bar bla bla bla.%
      }%
      \par
      Und noch ein Absatz.%
    }%
    \ParallelRText{%
      \textcolor{blue}{%
        Ein Absatz, der sich ueber zwei Zeilen erstrecken soll. %
        Ein Absatz, der sich ueber zwei Zeilen erstrecken soll. %
        Foo bar bla bla bla.%
      }%
    }%
  \end{Parallel}%
  \begin{Parallel}[p]{\textwidth}{\textwidth}%
    \ParallelLText{%
      \rule{1pt}{.98\textheight}\Huge g%
    }%
    \ParallelRText{%
      \rule{1pt}{.98\textheight}y%
    }%
  \end{Parallel}%
  Green%
\end{document}
%</test1>
%    \end{macrocode}
%
% \section{Installation}
%
% \subsection{Download}
%
% \paragraph{Package.} This package is available on
% CTAN\footnote{\url{http://ctan.org/pkg/pdfcolparallel}}:
% \begin{description}
% \item[\CTAN{macros/latex/contrib/oberdiek/pdfcolparallel.dtx}] The source file.
% \item[\CTAN{macros/latex/contrib/oberdiek/pdfcolparallel.pdf}] Documentation.
% \end{description}
%
%
% \paragraph{Bundle.} All the packages of the bundle `oberdiek'
% are also available in a TDS compliant ZIP archive. There
% the packages are already unpacked and the documentation files
% are generated. The files and directories obey the TDS standard.
% \begin{description}
% \item[\CTAN{install/macros/latex/contrib/oberdiek.tds.zip}]
% \end{description}
% \emph{TDS} refers to the standard ``A Directory Structure
% for \TeX\ Files'' (\CTAN{tds/tds.pdf}). Directories
% with \xfile{texmf} in their name are usually organized this way.
%
% \subsection{Bundle installation}
%
% \paragraph{Unpacking.} Unpack the \xfile{oberdiek.tds.zip} in the
% TDS tree (also known as \xfile{texmf} tree) of your choice.
% Example (linux):
% \begin{quote}
%   |unzip oberdiek.tds.zip -d ~/texmf|
% \end{quote}
%
% \paragraph{Script installation.}
% Check the directory \xfile{TDS:scripts/oberdiek/} for
% scripts that need further installation steps.
% Package \xpackage{attachfile2} comes with the Perl script
% \xfile{pdfatfi.pl} that should be installed in such a way
% that it can be called as \texttt{pdfatfi}.
% Example (linux):
% \begin{quote}
%   |chmod +x scripts/oberdiek/pdfatfi.pl|\\
%   |cp scripts/oberdiek/pdfatfi.pl /usr/local/bin/|
% \end{quote}
%
% \subsection{Package installation}
%
% \paragraph{Unpacking.} The \xfile{.dtx} file is a self-extracting
% \docstrip\ archive. The files are extracted by running the
% \xfile{.dtx} through \plainTeX:
% \begin{quote}
%   \verb|tex pdfcolparallel.dtx|
% \end{quote}
%
% \paragraph{TDS.} Now the different files must be moved into
% the different directories in your installation TDS tree
% (also known as \xfile{texmf} tree):
% \begin{quote}
% \def\t{^^A
% \begin{tabular}{@{}>{\ttfamily}l@{ $\rightarrow$ }>{\ttfamily}l@{}}
%   pdfcolparallel.sty & tex/latex/oberdiek/pdfcolparallel.sty\\
%   pdfcolparallel.pdf & doc/latex/oberdiek/pdfcolparallel.pdf\\
%   test/pdfcolparallel-test1.tex & doc/latex/oberdiek/test/pdfcolparallel-test1.tex\\
%   pdfcolparallel.dtx & source/latex/oberdiek/pdfcolparallel.dtx\\
% \end{tabular}^^A
% }^^A
% \sbox0{\t}^^A
% \ifdim\wd0>\linewidth
%   \begingroup
%     \advance\linewidth by\leftmargin
%     \advance\linewidth by\rightmargin
%   \edef\x{\endgroup
%     \def\noexpand\lw{\the\linewidth}^^A
%   }\x
%   \def\lwbox{^^A
%     \leavevmode
%     \hbox to \linewidth{^^A
%       \kern-\leftmargin\relax
%       \hss
%       \usebox0
%       \hss
%       \kern-\rightmargin\relax
%     }^^A
%   }^^A
%   \ifdim\wd0>\lw
%     \sbox0{\small\t}^^A
%     \ifdim\wd0>\linewidth
%       \ifdim\wd0>\lw
%         \sbox0{\footnotesize\t}^^A
%         \ifdim\wd0>\linewidth
%           \ifdim\wd0>\lw
%             \sbox0{\scriptsize\t}^^A
%             \ifdim\wd0>\linewidth
%               \ifdim\wd0>\lw
%                 \sbox0{\tiny\t}^^A
%                 \ifdim\wd0>\linewidth
%                   \lwbox
%                 \else
%                   \usebox0
%                 \fi
%               \else
%                 \lwbox
%               \fi
%             \else
%               \usebox0
%             \fi
%           \else
%             \lwbox
%           \fi
%         \else
%           \usebox0
%         \fi
%       \else
%         \lwbox
%       \fi
%     \else
%       \usebox0
%     \fi
%   \else
%     \lwbox
%   \fi
% \else
%   \usebox0
% \fi
% \end{quote}
% If you have a \xfile{docstrip.cfg} that configures and enables \docstrip's
% TDS installing feature, then some files can already be in the right
% place, see the documentation of \docstrip.
%
% \subsection{Refresh file name databases}
%
% If your \TeX~distribution
% (\teTeX, \mikTeX, \dots) relies on file name databases, you must refresh
% these. For example, \teTeX\ users run \verb|texhash| or
% \verb|mktexlsr|.
%
% \subsection{Some details for the interested}
%
% \paragraph{Attached source.}
%
% The PDF documentation on CTAN also includes the
% \xfile{.dtx} source file. It can be extracted by
% AcrobatReader 6 or higher. Another option is \textsf{pdftk},
% e.g. unpack the file into the current directory:
% \begin{quote}
%   \verb|pdftk pdfcolparallel.pdf unpack_files output .|
% \end{quote}
%
% \paragraph{Unpacking with \LaTeX.}
% The \xfile{.dtx} chooses its action depending on the format:
% \begin{description}
% \item[\plainTeX:] Run \docstrip\ and extract the files.
% \item[\LaTeX:] Generate the documentation.
% \end{description}
% If you insist on using \LaTeX\ for \docstrip\ (really,
% \docstrip\ does not need \LaTeX), then inform the autodetect routine
% about your intention:
% \begin{quote}
%   \verb|latex \let\install=y% \iffalse meta-comment
%
% File: pdfcolparallel.dtx
% Version: 2016/05/16 v1.4
% Info: Color stacks support for parallel
%
% Copyright (C) 2007, 2008, 2010 by
%    Heiko Oberdiek <heiko.oberdiek at googlemail.com>
%    2016
%    https://github.com/ho-tex/oberdiek/issues
%
% This work may be distributed and/or modified under the
% conditions of the LaTeX Project Public License, either
% version 1.3c of this license or (at your option) any later
% version. This version of this license is in
%    http://www.latex-project.org/lppl/lppl-1-3c.txt
% and the latest version of this license is in
%    http://www.latex-project.org/lppl.txt
% and version 1.3 or later is part of all distributions of
% LaTeX version 2005/12/01 or later.
%
% This work has the LPPL maintenance status "maintained".
%
% This Current Maintainer of this work is Heiko Oberdiek.
%
% This work consists of the main source file pdfcolparallel.dtx
% and the derived files
%    pdfcolparallel.sty, pdfcolparallel.pdf, pdfcolparallel.ins,
%    pdfcolparallel.drv, pdfcolparallel-test1.tex.
%
% Distribution:
%    CTAN:macros/latex/contrib/oberdiek/pdfcolparallel.dtx
%    CTAN:macros/latex/contrib/oberdiek/pdfcolparallel.pdf
%
% Unpacking:
%    (a) If pdfcolparallel.ins is present:
%           tex pdfcolparallel.ins
%    (b) Without pdfcolparallel.ins:
%           tex pdfcolparallel.dtx
%    (c) If you insist on using LaTeX
%           latex \let\install=y\input{pdfcolparallel.dtx}
%        (quote the arguments according to the demands of your shell)
%
% Documentation:
%    (a) If pdfcolparallel.drv is present:
%           latex pdfcolparallel.drv
%    (b) Without pdfcolparallel.drv:
%           latex pdfcolparallel.dtx; ...
%    The class ltxdoc loads the configuration file ltxdoc.cfg
%    if available. Here you can specify further options, e.g.
%    use A4 as paper format:
%       \PassOptionsToClass{a4paper}{article}
%
%    Programm calls to get the documentation (example):
%       pdflatex pdfcolparallel.dtx
%       makeindex -s gind.ist pdfcolparallel.idx
%       pdflatex pdfcolparallel.dtx
%       makeindex -s gind.ist pdfcolparallel.idx
%       pdflatex pdfcolparallel.dtx
%
% Installation:
%    TDS:tex/latex/oberdiek/pdfcolparallel.sty
%    TDS:doc/latex/oberdiek/pdfcolparallel.pdf
%    TDS:doc/latex/oberdiek/test/pdfcolparallel-test1.tex
%    TDS:source/latex/oberdiek/pdfcolparallel.dtx
%
%<*ignore>
\begingroup
  \catcode123=1 %
  \catcode125=2 %
  \def\x{LaTeX2e}%
\expandafter\endgroup
\ifcase 0\ifx\install y1\fi\expandafter
         \ifx\csname processbatchFile\endcsname\relax\else1\fi
         \ifx\fmtname\x\else 1\fi\relax
\else\csname fi\endcsname
%</ignore>
%<*install>
\input docstrip.tex
\Msg{************************************************************************}
\Msg{* Installation}
\Msg{* Package: pdfcolparallel 2016/05/16 v1.4 Color stacks support for parallel (HO)}
\Msg{************************************************************************}

\keepsilent
\askforoverwritefalse

\let\MetaPrefix\relax
\preamble

This is a generated file.

Project: pdfcolparallel
Version: 2016/05/16 v1.4

Copyright (C) 2007, 2008, 2010 by
   Heiko Oberdiek <heiko.oberdiek at googlemail.com>

This work may be distributed and/or modified under the
conditions of the LaTeX Project Public License, either
version 1.3c of this license or (at your option) any later
version. This version of this license is in
   http://www.latex-project.org/lppl/lppl-1-3c.txt
and the latest version of this license is in
   http://www.latex-project.org/lppl.txt
and version 1.3 or later is part of all distributions of
LaTeX version 2005/12/01 or later.

This work has the LPPL maintenance status "maintained".

This Current Maintainer of this work is Heiko Oberdiek.

This work consists of the main source file pdfcolparallel.dtx
and the derived files
   pdfcolparallel.sty, pdfcolparallel.pdf, pdfcolparallel.ins,
   pdfcolparallel.drv, pdfcolparallel-test1.tex.

\endpreamble
\let\MetaPrefix\DoubleperCent

\generate{%
  \file{pdfcolparallel.ins}{\from{pdfcolparallel.dtx}{install}}%
  \file{pdfcolparallel.drv}{\from{pdfcolparallel.dtx}{driver}}%
  \usedir{tex/latex/oberdiek}%
  \file{pdfcolparallel.sty}{\from{pdfcolparallel.dtx}{package}}%
  \usedir{doc/latex/oberdiek/test}%
  \file{pdfcolparallel-test1.tex}{\from{pdfcolparallel.dtx}{test1}}%
  \nopreamble
  \nopostamble
  \usedir{source/latex/oberdiek/catalogue}%
  \file{pdfcolparallel.xml}{\from{pdfcolparallel.dtx}{catalogue}}%
}

\catcode32=13\relax% active space
\let =\space%
\Msg{************************************************************************}
\Msg{*}
\Msg{* To finish the installation you have to move the following}
\Msg{* file into a directory searched by TeX:}
\Msg{*}
\Msg{*     pdfcolparallel.sty}
\Msg{*}
\Msg{* To produce the documentation run the file `pdfcolparallel.drv'}
\Msg{* through LaTeX.}
\Msg{*}
\Msg{* Happy TeXing!}
\Msg{*}
\Msg{************************************************************************}

\endbatchfile
%</install>
%<*ignore>
\fi
%</ignore>
%<*driver>
\NeedsTeXFormat{LaTeX2e}
\ProvidesFile{pdfcolparallel.drv}%
  [2016/05/16 v1.4 Color stacks support for parallel (HO)]%
\documentclass{ltxdoc}
\usepackage{holtxdoc}[2011/11/22]
\begin{document}
  \DocInput{pdfcolparallel.dtx}%
\end{document}
%</driver>
% \fi
%
%
% \CharacterTable
%  {Upper-case    \A\B\C\D\E\F\G\H\I\J\K\L\M\N\O\P\Q\R\S\T\U\V\W\X\Y\Z
%   Lower-case    \a\b\c\d\e\f\g\h\i\j\k\l\m\n\o\p\q\r\s\t\u\v\w\x\y\z
%   Digits        \0\1\2\3\4\5\6\7\8\9
%   Exclamation   \!     Double quote  \"     Hash (number) \#
%   Dollar        \$     Percent       \%     Ampersand     \&
%   Acute accent  \'     Left paren    \(     Right paren   \)
%   Asterisk      \*     Plus          \+     Comma         \,
%   Minus         \-     Point         \.     Solidus       \/
%   Colon         \:     Semicolon     \;     Less than     \<
%   Equals        \=     Greater than  \>     Question mark \?
%   Commercial at \@     Left bracket  \[     Backslash     \\
%   Right bracket \]     Circumflex    \^     Underscore    \_
%   Grave accent  \`     Left brace    \{     Vertical bar  \|
%   Right brace   \}     Tilde         \~}
%
% \GetFileInfo{pdfcolparallel.drv}
%
% \title{The \xpackage{pdfcolparallel} package}
% \date{2016/05/16 v1.4}
% \author{Heiko Oberdiek\thanks
% {Please report any issues at https://github.com/ho-tex/oberdiek/issues}\\
% \xemail{heiko.oberdiek at googlemail.com}}
%
% \maketitle
%
% \begin{abstract}
% This packages fixes bugs in \xpackage{parallel} and
% improves color support by using several color stacks
% that are provided by \pdfTeX\ since version 1.40.
% \end{abstract}
%
% \tableofcontents
%
% \section{Usage}
%
% \begin{quote}
% |\usepackage{pdfcolparallel}|
% \end{quote}
% The package \xpackage{pdfcolparallel} loads package \xpackage{parallel}
% \cite{parallel} and redefines some macros to fix bugs.
%
% If color stacks are available then package \xpackage{parallel}
% is further patched to support them.
%
% \subsection{Option \xoption{rulebetweencolor}}
%
% Package \xpackage{pdfcolparallel} also fixes the color for the
% rule between columns.
% Default color is \cs{normalcolor}. But this can be changed by using
% option \xoption{rulebetweencolor} for |\setkeys{parallel}|
% (see package \xpackage{keyval}). The option takes a color specification
% as value. If the value is empty, then the default (\cs{normalcolor})
% is used.
% Examples:
% \begin{quote}
%   |\setkeys{parallel}{rulebetweencolor=blue}|,\\
%   |\setkeys{parallel}{rulebetweencolor={red}}|,\\
%   |\setkeys{parallel}{rulebetweencolor={}}|,
%     \textit{\% \cs{normalcolor} is used}\\
%   |\setkeys{parallel}{rulebetweencolor=[rgb]{1,0,.5}}|
% \end{quote}
%
% \subsection{Future}
%
% If there will be a new version of package \xpackage{parallel}
% that adds support for color stacks, then this package may become
% obsolete.
%
% \StopEventually{
% }
%
% \section{Implementation}
%
% \subsection{Identification}
%
%    \begin{macrocode}
%<*package>
\NeedsTeXFormat{LaTeX2e}
\ProvidesPackage{pdfcolparallel}%
  [2016/05/16 v1.4 Color stacks support for parallel (HO)]%
%    \end{macrocode}
%
% \subsection{Load and fix package \xpackage{parallel}}
%
%    Package \xpackage{parallel} is loaded. Before options of package
%    \xpackage{pdfcolparallel} are passed to package \xpackage{parallel}.
%    \begin{macrocode}
\DeclareOption*{%
  \PassoptionsToPackage{\CurrentOption}{parallel}%
}
\ProcessOptions\relax
\RequirePackage{parallel}[2003/04/13]
%    \end{macrocode}
%
%    \begin{macrocode}
\RequirePackage{infwarerr}[2007/09/09]
%    \end{macrocode}
%
%    \begin{macro}{\pcp@ColorPatch}
%    \begin{macrocode}
\begingroup\expandafter\expandafter\expandafter\endgroup
\expandafter\ifx\csname currentgrouplevel\endcsname\relax
  \def\pcp@ColorPatch{}%
\else
  \def\pcp@ColorPatch{%
    \@ifundefined{set@color}{%
      \gdef\pcp@ColorPatch{}%
    }{%
      \gdef\pcp@ColorPatch{%
        \gdef\pcp@ColorResets{}%
        \bgroup
        \aftergroup\pcp@ColorResets
        \aftergroup\egroup
        \let\pcp@OrgSetColor\set@color
        \let\set@color\pcp@SetColor
        \edef\pcp@GroupLevel{\the\currentgrouplevel}%
      }%
    }%
    \pcp@ColorPatch
  }%
%    \end{macrocode}
%    \end{macro}
%    \begin{macro}{\pcp@SetColor}
%    \begin{macrocode}
  \def\pcp@SetColor{%
    \ifnum\pcp@GroupLevel=\currentgrouplevel
      \let\pcp@OrgAfterGroup\aftergroup
      \def\aftergroup{%
        \g@addto@macro\pcp@ColorResets
      }%
      \pcp@OrgSetColor
      \let\aftergroup\pcp@OrgAfterGroup
    \else
      \pcp@OrgSetColor
    \fi
  }%
\fi
%    \end{macrocode}
%    \end{macro}
%
%    \begin{macro}{\pcp@CmdCheckRedef}
%    \begin{macrocode}
\def\pcp@CmdCheckRedef#1{%
  \begingroup
    \def\pcp@cmd{#1}%
    \afterassignment\pcp@CmdDo
    \long\def\reserved@a
}
\def\pcp@CmdDo{%
    \expandafter\ifx\pcp@cmd\reserved@a
    \else
      \edef\x*{\expandafter\string\pcp@cmd}%
      \@PackageWarningNoLine{pdfcolparallel}{%
        Command \x* has changed.\MessageBreak
        Supported versions of package `parallel':\MessageBreak
        \space\space 2003/04/13\MessageBreak
        The redefinition of \x* may\MessageBreak
        not behave correctly depending on the changes%
      }%
    \fi
  \expandafter\endgroup
  \expandafter\def\pcp@cmd
}
%    \end{macrocode}
%    \end{macro}
%
%    \begin{macrocode}
\def\pcp@SwitchStack#1#2{}
%    \end{macrocode}
%    \begin{macrocode}
\def\pcp@SetCurrent#1{}
%    \end{macrocode}
%
%    \begin{macro}{\ParallelLText}
%    \begin{macrocode}
\pcp@CmdCheckRedef\ParallelLText{%
  \everypar{}%
  \@restorepar
  \begingroup
    \hbadness=3000 %
    \let\footnote=\ParallelLFootnote
    \ParallelWhichBox=0 %
    \global\setbox\ParallelLBox=\vbox\bgroup
      \hsize=\ParallelLWidth
      \aftergroup\ParallelAfterText
      \begingroup
        \afterassignment\ParallelCheckOpenBrace
        \let\x=%
}{%
  \everypar{}%
  \@restorepar
  \@nobreakfalse
  \begingroup
    \hbadness=3000 %
    \let\footnote=\ParallelLFootnote
    \ParallelWhichBox=0 %
    \global\setbox\ParallelLBox=\vbox\bgroup
      \hsize=\ParallelLWidth
      \linewidth=\ParallelLWidth
      \pcp@SwitchStack{Left}\ParallelLBox
      \aftergroup\ParallelAfterText
      \pcp@ColorPatch
      \begingroup
        \afterassignment\ParallelCheckOpenBrace
        \let\x=%
}
%    \end{macrocode}
%    \end{macro}
%
%    \begin{macro}{\ParallelRText}
%    \begin{macrocode}
\pcp@CmdCheckRedef\ParallelRText{%
  \everypar{}%
  \@restorepar
  \begingroup
    \hbadness=3000 %
    \ifnum\ParallelFNMode=\@ne
      \let\footnote=\ParallelRFootnote
    \else
      \let\footnote=\ParallelLFootnote
    \fi
    \ParallelWhichBox=\@ne
    \global\setbox\ParallelRBox=\vbox\bgroup
      \hsize=\ParallelRWidth
      \aftergroup\ParallelAfterText
      \begingroup
        \afterassignment\ParallelCheckOpenBrace
        \let\x=%
}{%
  \everypar{}%
  \@restorepar
  \@nobreakfalse
  \begingroup
    \hbadness=3000 %
    \ifnum\ParallelFNMode=\@ne
      \let\footnote=\ParallelRFootnote
    \else
      \let\footnote=\ParallelLFootnote
    \fi
    \ParallelWhichBox=\@ne
    \global\setbox\ParallelRBox=\vbox\bgroup
      \hsize=\ParallelRWidth
      \linewidth=\ParallelRWidth
      \pcp@SwitchStack{Right}\ParallelRBox
      \aftergroup\ParallelAfterText
      \pcp@ColorPatch
      \begingroup
        \afterassignment\ParallelCheckOpenBrace
        \let\x=%
}
%    \end{macrocode}
%    \end{macro}
%
%    \begin{macro}{\ParallelParTwoPages}
%    \begin{macrocode}
\pcp@CmdCheckRedef\ParallelParTwoPages{%
  \ifnum\ParallelBoolVar=\@ne
    \par
    \begingroup
      \global\ParallelWhichBox=\@ne
      \newpage
      \vbadness=10000 %
      \vfuzz=3ex %
      \splittopskip=\z@skip
      \loop%
        \ifnum\ParallelBoolVar=\@ne%
          \ifnum\ParallelWhichBox=\@ne
            \ifvoid\ParallelLBox
              \mbox{} %
              \newpage
            \else
              \global\ParallelWhichBox=\z@
            \fi
          \else
            \ifvoid\ParallelRBox
              \mbox{} %
              \newpage
            \else
              \global\ParallelWhichBox=\@ne
            \fi
          \fi
          \ifnum\ParallelWhichBox=\z@
            \ifodd\thepage
              \mbox{} %
              \newpage
            \fi
            \hbox to\textwidth{%
              \vbox{\vsplit\ParallelLBox to.98\textheight}%
            }%
          \else
            \ifodd\thepage\relax
            \else
              \mbox{} %
              \newpage
            \fi
            \hbox to\textwidth{%
              \vbox{\vsplit\ParallelRBox to.98\textheight}%
            }%
          \fi
          \vspace*{\fill}%
          \newpage
        \fi
        \ifvoid\ParallelLBox
          \ifvoid\ParallelRBox
            \global\ParallelBoolVar=\z@
          \fi
        \fi
      \ifnum\ParallelBoolVar=\@ne
      \repeat
      \par
    \endgroup
  \fi
}{%
%    \end{macrocode}
%    Additional fixes:
%    \begin{itemize}
%    \item Unnecessary white space removed.
%    \item |\ifodd\thepage| changed to |\ifodd\value{page}|.
%    \end{itemize}
%    \begin{macrocode}
  \ifnum\ParallelBoolVar=\@ne
    \par
    \begingroup
      \global\ParallelWhichBox=\@ne
      \newpage
      \vbadness=10000 %
      \vfuzz=3ex %
      \splittopskip=\z@skip
      \loop%
        \ifnum\ParallelBoolVar=\@ne%
          \ifnum\ParallelWhichBox=\@ne
            \ifvoid\ParallelLBox
              \mbox{}%
              \newpage
            \else
              \global\ParallelWhichBox=\z@
            \fi
          \else
            \ifvoid\ParallelRBox
              \null
              \newpage
            \else
              \global\ParallelWhichBox=\@ne
            \fi
          \fi
          \ifnum\ParallelWhichBox=\z@
            \ifodd\value{page}%
              \null
              \newpage
            \fi
            \hbox to\textwidth{%
              \pcp@SetCurrent{Left}%
              \setbox\z@=\vsplit\ParallelLBox to.98\textheight
              \vbox to.98\textheight{%
                \@texttop
                \unvbox\z@
                \@textbottom
              }%
            }%
          \else
            \ifodd\value{page}%
            \else
              \mbox{}%
              \newpage
            \fi
            \hbox to\textwidth{%
              \pcp@SetCurrent{Right}%
              \setbox\z@=\vsplit\ParallelRBox to.98\textheight
              \vbox to.98\textheight{%
                \@texttop
                \unvbox\z@
                \@textbottom
              }%
            }%
          \fi
          \vspace*{\fill}%
          \newpage
        \fi
        \ifvoid\ParallelLBox
          \ifvoid\ParallelRBox
            \global\ParallelBoolVar=\z@
          \fi
        \fi
      \ifnum\ParallelBoolVar=\@ne
      \repeat
      \par
    \endgroup
    \pcp@SetCurrent{}%
  \fi
}
%    \end{macrocode}
%    \end{macro}
%
% \subsection{Color stack support}
%
%    \begin{macrocode}
\RequirePackage{pdfcol}[2007/12/12]
\ifpdfcolAvailable
\else
  \PackageInfo{pdfcolparallel}{%
    Loading aborted, because color stacks are not available%
  }%
  \expandafter\endinput
\fi
%    \end{macrocode}
%
%    \begin{macrocode}
\pdfcolInitStack{pcp@Left}
\pdfcolInitStack{pcp@Right}
%    \end{macrocode}
%    \begin{macro}{\pcp@Box}
%    \begin{macrocode}
\newbox\pcp@Box
%    \end{macrocode}
%    \end{macro}
%    \begin{macro}{\pcp@SwitchStack}
%    \begin{macrocode}
\def\pcp@SwitchStack#1#2{%
  \pdfcolSwitchStack{pcp@#1}%
  \global\setbox\pcp@Box=\vbox to 0pt{%
    \pdfcolSetCurrentColor
  }%
  \aftergroup\pcp@FixBox
  \aftergroup#2%
}
%    \end{macrocode}
%    \end{macro}
%    \begin{macro}{\pcp@FixBox}
%    \begin{macrocode}
\def\pcp@FixBox#1{%
  \global\setbox#1=\vbox{%
    \unvbox\pcp@Box
    \unvbox#1%
  }%
}
%    \end{macrocode}
%    \end{macro}
%    \begin{macro}{\pcp@SetCurrent}
%    \begin{macrocode}
\def\pcp@SetCurrent#1{%
  \ifx\\#1\\%
    \pdfcolSetCurrent{}%
  \else
    \pdfcolSetCurrent{pcp@#1}%
  \fi
}
%    \end{macrocode}
%    \end{macro}
%
% \subsection{Redefinitions}
%
%    \begin{macro}{\ParallelParOnePage}
%    \begin{macrocode}
\pcp@CmdCheckRedef\ParallelParOnePage{%
  \ifnum\ParallelBoolVar=\@ne
    \par
    \begingroup
      \leftmargin=\z@
      \rightmargin=\z@
      \parskip=\z@skip
      \parindent=\z@
      \vbadness=10000 %
      \vfuzz=3ex %
      \splittopskip=\z@skip
      \loop
        \ifnum\ParallelBoolVar=\@ne
          \noindent
          \hbox to\textwidth{%
            \hskip\ParallelLeftMargin
            \hbox to\ParallelTextWidth{%
              \ifvoid\ParallelLBox
                \hskip\ParallelLWidth
              \else
                \ParallelWhichBox=\z@
                \vbox{%
                  \setbox\ParallelBoxVar
                      =\vsplit\ParallelLBox to\dp\strutbox
                  \unvbox\ParallelBoxVar
                }%
              \fi
              \strut
              \ifnum\ParallelBoolMid=\@ne
                \hskip\ParallelMainMidSkip
                \vrule
              \else
                \hss
              \fi
              \hss
              \ifvoid\ParallelRBox
                \hskip\ParallelRWidth
              \else
                \ParallelWhichBox=\@ne
                \vbox{%
                  \setbox\ParallelBoxVar
                      =\vsplit\ParallelRBox to\dp\strutbox
                  \unvbox\ParallelBoxVar
                }%
              \fi
            }%
          }%
          \ifvoid\ParallelLBox
            \ifvoid\ParallelRBox
              \global\ParallelBoolVar=\z@
            \fi
          \fi%
        \fi%
      \ifnum\ParallelBoolVar=\@ne
        \penalty\interlinepenalty
      \repeat
      \par
    \endgroup
  \fi
}{%
  \ifnum\ParallelBoolVar=\@ne
    \par
    \begingroup
      \leftmargin=\z@
      \rightmargin=\z@
      \parskip=\z@skip
      \parindent=\z@
      \vbadness=10000 %
      \vfuzz=3ex %
      \splittopskip=\z@skip
      \loop
        \ifnum\ParallelBoolVar=\@ne
          \noindent
          \hbox to\textwidth{%
            \hskip\ParallelLeftMargin
            \hbox to\ParallelTextWidth{%
              \ifvoid\ParallelLBox
                \hskip\ParallelLWidth
              \else
                \pcp@SetCurrent{Left}%
                \ParallelWhichBox=\z@
                \vbox{%
                  \setbox\ParallelBoxVar
                      =\vsplit\ParallelLBox to\dp\strutbox
                  \unvbox\ParallelBoxVar
                }%
              \fi
              \strut
              \ifnum\ParallelBoolMid=\@ne
                \hskip\ParallelMainMidSkip
                \begingroup
                  \pcp@RuleBetweenColor
                  \vrule
                \endgroup
              \else
                \hss
              \fi
              \hss
              \ifvoid\ParallelRBox
                \hskip\ParallelRWidth
              \else
                \pcp@SetCurrent{Right}%
                \ParallelWhichBox=\@ne
                \vbox{%
                  \setbox\ParallelBoxVar
                      =\vsplit\ParallelRBox to\dp\strutbox
                  \unvbox\ParallelBoxVar
                }%
              \fi
            }%
          }%
          \ifvoid\ParallelLBox
            \ifvoid\ParallelRBox
              \global\ParallelBoolVar=\z@
            \fi
          \fi%
        \fi%
      \ifnum\ParallelBoolVar=\@ne
        \penalty\interlinepenalty
      \repeat
      \par
    \endgroup
    \pcp@SetCurrent{}%
  \fi
}
%    \end{macrocode}
%    \end{macro}
%    \begin{macro}{\pcp@RuleBetweenColorDefault}
%    \begin{macrocode}
\def\pcp@RuleBetweenColorDefault{%
  \normalcolor
}
%    \end{macrocode}
%    \end{macro}
%    \begin{macro}{\pcp@RuleBetweenColor}
%    \begin{macrocode}
\let\pcp@RuleBetweenColor\pcp@RuleBetweenColorDefault
%    \end{macrocode}
%    \end{macro}
%    \begin{macrocode}
\RequirePackage{keyval}
\define@key{parallel}{rulebetweencolor}{%
  \edef\pcp@temp{#1}%
  \ifx\pcp@temp\@empty
    \let\pcp@RuleBetweenColor\pcp@RuleBetweenColorDefault
  \else
    \edef\pcp@temp{%
      \noexpand\@ifnextchar[{%
        \def\noexpand\pcp@RuleBetweenColor{%
          \noexpand\color\pcp@temp
        }%
        \noexpand\pcp@GobbleNil
      }{%
        \def\noexpand\pcp@RuleBetweenColor{%
          \noexpand\color{\pcp@temp}%
        }%
        \noexpand\pcp@GobbleNil
      }%
      \pcp@temp\noexpand\@nil
    }%
    \pcp@temp
  \fi
}
%    \end{macrocode}
%    \begin{macro}{\pcp@GobbleNil}
%    \begin{macrocode}
\long\def\pcp@GobbleNil#1\@nil{}
%    \end{macrocode}
%    \end{macro}
%
%    \begin{macrocode}
%</package>
%    \end{macrocode}
%
% \section{Test}
%
%    The test file is a modified version of the file that
%    Alexander Hirsch has posted in \xnewsgroup{de.comp.text.tex}:
%    \URL{``\link{\texttt{parallel.sty} und farbiger Text}''}^^A
%    {http://groups.google.com/group/de.comp.text.tex/msg/6a759cf33bb071a5}
%    \begin{macrocode}
%<*test1>
\AtEndDocument{%
  \typeout{}%
  \typeout{**************************************}%
  \typeout{*** \space Check the PDF file manually! \space ***}%
  \typeout{**************************************}%
  \typeout{}%
}
\documentclass{article}
\usepackage{xcolor}
\usepackage{pdfcolparallel}[2016/05/16]

\begin{document}
  \color{green}%
  Green%
  \begin{Parallel}{0.47\textwidth}{0.47\textwidth}%
    \ParallelLText{%
      \textcolor{red}{%
        Ein Absatz, der sich ueber zwei Zeilen erstrecken soll. %
        Ein Absatz, der sich ueber zwei Zeilen erstrecken soll.%
      }%
    }%
    \ParallelRText{%
      \textcolor{blue}{%
        Ein Absatz, der sich ueber zwei Zeilen erstrecken soll. %
        Ein Absatz, der sich ueber zwei Zeilen erstrecken soll.%
      }%
    }%
    \ParallelPar
    \ParallelLText{%
      Default %
      \color{red}%
      Ein Absatz, der sich ueber zwei Zeilen erstrecken soll. %
      Ein Absatz, der sich ueber zwei Zeilen erstrecken soll.%
    }%
    \ParallelRText{%
      Default %
      \color{blue}%
      Ein Absatz, der sich ueber zwei Zeilen erstrecken soll. %
      Ein Absatz, der sich ueber zwei Zeilen erstrecken soll.%
    }%
    \ParallelPar
    \ParallelLText{%
      \begin{enumerate}%
      \item left text, left text, left text, left text, %
            left text, left text, left text, left text,%
      \item left text, left text, left text, left text, %
            left text, left text, left text, left text.%
      \end{enumerate}%
    }%
    \ParallelRText{%
      \begin{enumerate}%
      \item right text, right text, right text, right text, %
            right text, right text, right text, right text.%
      \item right text, right text, right text, right text, %
            right text, right text, right text, right text.%
      \end{enumerate}%
    }%
  \end{Parallel}%
  \begin{Parallel}[p]{\textwidth}{\textwidth}%
    \ParallelLText{%
      \textcolor{red}{%
        Ein Absatz, der sich ueber zwei Zeilen erstrecken soll. %
        Ein Absatz, der sich ueber zwei Zeilen erstrecken soll. %
        Foo bar bla bla bla.%
      }%
      \par
      Und noch ein Absatz.%
    }%
    \ParallelRText{%
      \textcolor{blue}{%
        Ein Absatz, der sich ueber zwei Zeilen erstrecken soll. %
        Ein Absatz, der sich ueber zwei Zeilen erstrecken soll. %
        Foo bar bla bla bla.%
      }%
    }%
  \end{Parallel}%
  \begin{Parallel}[p]{\textwidth}{\textwidth}%
    \ParallelLText{%
      \rule{1pt}{.98\textheight}\Huge g%
    }%
    \ParallelRText{%
      \rule{1pt}{.98\textheight}y%
    }%
  \end{Parallel}%
  Green%
\end{document}
%</test1>
%    \end{macrocode}
%
% \section{Installation}
%
% \subsection{Download}
%
% \paragraph{Package.} This package is available on
% CTAN\footnote{\url{http://ctan.org/pkg/pdfcolparallel}}:
% \begin{description}
% \item[\CTAN{macros/latex/contrib/oberdiek/pdfcolparallel.dtx}] The source file.
% \item[\CTAN{macros/latex/contrib/oberdiek/pdfcolparallel.pdf}] Documentation.
% \end{description}
%
%
% \paragraph{Bundle.} All the packages of the bundle `oberdiek'
% are also available in a TDS compliant ZIP archive. There
% the packages are already unpacked and the documentation files
% are generated. The files and directories obey the TDS standard.
% \begin{description}
% \item[\CTAN{install/macros/latex/contrib/oberdiek.tds.zip}]
% \end{description}
% \emph{TDS} refers to the standard ``A Directory Structure
% for \TeX\ Files'' (\CTAN{tds/tds.pdf}). Directories
% with \xfile{texmf} in their name are usually organized this way.
%
% \subsection{Bundle installation}
%
% \paragraph{Unpacking.} Unpack the \xfile{oberdiek.tds.zip} in the
% TDS tree (also known as \xfile{texmf} tree) of your choice.
% Example (linux):
% \begin{quote}
%   |unzip oberdiek.tds.zip -d ~/texmf|
% \end{quote}
%
% \paragraph{Script installation.}
% Check the directory \xfile{TDS:scripts/oberdiek/} for
% scripts that need further installation steps.
% Package \xpackage{attachfile2} comes with the Perl script
% \xfile{pdfatfi.pl} that should be installed in such a way
% that it can be called as \texttt{pdfatfi}.
% Example (linux):
% \begin{quote}
%   |chmod +x scripts/oberdiek/pdfatfi.pl|\\
%   |cp scripts/oberdiek/pdfatfi.pl /usr/local/bin/|
% \end{quote}
%
% \subsection{Package installation}
%
% \paragraph{Unpacking.} The \xfile{.dtx} file is a self-extracting
% \docstrip\ archive. The files are extracted by running the
% \xfile{.dtx} through \plainTeX:
% \begin{quote}
%   \verb|tex pdfcolparallel.dtx|
% \end{quote}
%
% \paragraph{TDS.} Now the different files must be moved into
% the different directories in your installation TDS tree
% (also known as \xfile{texmf} tree):
% \begin{quote}
% \def\t{^^A
% \begin{tabular}{@{}>{\ttfamily}l@{ $\rightarrow$ }>{\ttfamily}l@{}}
%   pdfcolparallel.sty & tex/latex/oberdiek/pdfcolparallel.sty\\
%   pdfcolparallel.pdf & doc/latex/oberdiek/pdfcolparallel.pdf\\
%   test/pdfcolparallel-test1.tex & doc/latex/oberdiek/test/pdfcolparallel-test1.tex\\
%   pdfcolparallel.dtx & source/latex/oberdiek/pdfcolparallel.dtx\\
% \end{tabular}^^A
% }^^A
% \sbox0{\t}^^A
% \ifdim\wd0>\linewidth
%   \begingroup
%     \advance\linewidth by\leftmargin
%     \advance\linewidth by\rightmargin
%   \edef\x{\endgroup
%     \def\noexpand\lw{\the\linewidth}^^A
%   }\x
%   \def\lwbox{^^A
%     \leavevmode
%     \hbox to \linewidth{^^A
%       \kern-\leftmargin\relax
%       \hss
%       \usebox0
%       \hss
%       \kern-\rightmargin\relax
%     }^^A
%   }^^A
%   \ifdim\wd0>\lw
%     \sbox0{\small\t}^^A
%     \ifdim\wd0>\linewidth
%       \ifdim\wd0>\lw
%         \sbox0{\footnotesize\t}^^A
%         \ifdim\wd0>\linewidth
%           \ifdim\wd0>\lw
%             \sbox0{\scriptsize\t}^^A
%             \ifdim\wd0>\linewidth
%               \ifdim\wd0>\lw
%                 \sbox0{\tiny\t}^^A
%                 \ifdim\wd0>\linewidth
%                   \lwbox
%                 \else
%                   \usebox0
%                 \fi
%               \else
%                 \lwbox
%               \fi
%             \else
%               \usebox0
%             \fi
%           \else
%             \lwbox
%           \fi
%         \else
%           \usebox0
%         \fi
%       \else
%         \lwbox
%       \fi
%     \else
%       \usebox0
%     \fi
%   \else
%     \lwbox
%   \fi
% \else
%   \usebox0
% \fi
% \end{quote}
% If you have a \xfile{docstrip.cfg} that configures and enables \docstrip's
% TDS installing feature, then some files can already be in the right
% place, see the documentation of \docstrip.
%
% \subsection{Refresh file name databases}
%
% If your \TeX~distribution
% (\teTeX, \mikTeX, \dots) relies on file name databases, you must refresh
% these. For example, \teTeX\ users run \verb|texhash| or
% \verb|mktexlsr|.
%
% \subsection{Some details for the interested}
%
% \paragraph{Attached source.}
%
% The PDF documentation on CTAN also includes the
% \xfile{.dtx} source file. It can be extracted by
% AcrobatReader 6 or higher. Another option is \textsf{pdftk},
% e.g. unpack the file into the current directory:
% \begin{quote}
%   \verb|pdftk pdfcolparallel.pdf unpack_files output .|
% \end{quote}
%
% \paragraph{Unpacking with \LaTeX.}
% The \xfile{.dtx} chooses its action depending on the format:
% \begin{description}
% \item[\plainTeX:] Run \docstrip\ and extract the files.
% \item[\LaTeX:] Generate the documentation.
% \end{description}
% If you insist on using \LaTeX\ for \docstrip\ (really,
% \docstrip\ does not need \LaTeX), then inform the autodetect routine
% about your intention:
% \begin{quote}
%   \verb|latex \let\install=y\input{pdfcolparallel.dtx}|
% \end{quote}
% Do not forget to quote the argument according to the demands
% of your shell.
%
% \paragraph{Generating the documentation.}
% You can use both the \xfile{.dtx} or the \xfile{.drv} to generate
% the documentation. The process can be configured by the
% configuration file \xfile{ltxdoc.cfg}. For instance, put this
% line into this file, if you want to have A4 as paper format:
% \begin{quote}
%   \verb|\PassOptionsToClass{a4paper}{article}|
% \end{quote}
% An example follows how to generate the
% documentation with pdf\LaTeX:
% \begin{quote}
%\begin{verbatim}
%pdflatex pdfcolparallel.dtx
%makeindex -s gind.ist pdfcolparallel.idx
%pdflatex pdfcolparallel.dtx
%makeindex -s gind.ist pdfcolparallel.idx
%pdflatex pdfcolparallel.dtx
%\end{verbatim}
% \end{quote}
%
% \section{Catalogue}
%
% The following XML file can be used as source for the
% \href{http://mirror.ctan.org/help/Catalogue/catalogue.html}{\TeX\ Catalogue}.
% The elements \texttt{caption} and \texttt{description} are imported
% from the original XML file from the Catalogue.
% The name of the XML file in the Catalogue is \xfile{pdfcolparallel.xml}.
%    \begin{macrocode}
%<*catalogue>
<?xml version='1.0' encoding='us-ascii'?>
<!DOCTYPE entry SYSTEM 'catalogue.dtd'>
<entry datestamp='$Date$' modifier='$Author$' id='pdfcolparallel'>
  <name>pdfcolparallel</name>
  <caption>Fix colour problems in package 'parallel'.</caption>
  <authorref id='auth:oberdiek'/>
  <copyright owner='Heiko Oberdiek' year='2007,2008,2010'/>
  <license type='lppl1.3'/>
  <version number='1.4'/>
  <description>
    Since version 1.40 pdfTeX supports colour stacks.
    This package uses them to fix colour problems in
    package <xref refid='parallel'>parallel</xref>.
    <p/>
    The package is part of the <xref refid='oberdiek'>oberdiek</xref>
    bundle.
  </description>
  <documentation details='Package documentation'
      href='ctan:/macros/latex/contrib/oberdiek/pdfcolparallel.pdf'/>
  <ctan file='true' path='/macros/latex/contrib/oberdiek/pdfcolparallel.dtx'/>
  <miktex location='oberdiek'/>
  <texlive location='oberdiek'/>
  <install path='/macros/latex/contrib/oberdiek/oberdiek.tds.zip'/>
</entry>
%</catalogue>
%    \end{macrocode}
%
% \begin{thebibliography}{9}
%
% \bibitem{parallel}
%   Matthias Eckermann: \textit{The \xpackage{parallel}-package};
%   2003/04/13;\\
%   \CTAN{macros/latex/contrib/parallel/}.
%
% \bibitem{pdfcol}
%   Heiko Oberdiek: \textit{The \xpackage{pdfcol} package};
%   2007/09/09;\\
%   \CTAN{macros/latex/contrib/oberdiek/pdfcol.pdf}.
%
% \end{thebibliography}
%
% \begin{History}
%   \begin{Version}{2007/09/09 v1.0}
%   \item
%     First version.
%   \end{Version}
%   \begin{Version}{2007/12/12 v1.1}
%   \item
%     Adds patch for setting \cs{linewidth} to fix bug
%     in package \xpackage{parallel}.
%   \item
%     Package \xpackage{parallel} is also fixed if color
%     stacks are not available.
%   \item
%     Bug fix, switched stacks now initialized with current color.
%   \item
%     Fix for package \xpackage{parallel}: \cs{raggedbottom} is respected.
%   \end{Version}
%   \begin{Version}{2008/08/11 v1.2}
%   \item
%     Code is not changed.
%   \item
%     URLs updated.
%   \end{Version}
%   \begin{Version}{2010/01/11 v1.3}
%   \item
%     Option `rulebetweencolor' added.
%   \end{Version}
%   \begin{Version}{2016/05/16 v1.4}
%   \item
%     Documentation updates.
%   \end{Version}
% \end{History}
%
% \PrintIndex
%
% \Finale
\endinput
|
% \end{quote}
% Do not forget to quote the argument according to the demands
% of your shell.
%
% \paragraph{Generating the documentation.}
% You can use both the \xfile{.dtx} or the \xfile{.drv} to generate
% the documentation. The process can be configured by the
% configuration file \xfile{ltxdoc.cfg}. For instance, put this
% line into this file, if you want to have A4 as paper format:
% \begin{quote}
%   \verb|\PassOptionsToClass{a4paper}{article}|
% \end{quote}
% An example follows how to generate the
% documentation with pdf\LaTeX:
% \begin{quote}
%\begin{verbatim}
%pdflatex pdfcolparallel.dtx
%makeindex -s gind.ist pdfcolparallel.idx
%pdflatex pdfcolparallel.dtx
%makeindex -s gind.ist pdfcolparallel.idx
%pdflatex pdfcolparallel.dtx
%\end{verbatim}
% \end{quote}
%
% \section{Catalogue}
%
% The following XML file can be used as source for the
% \href{http://mirror.ctan.org/help/Catalogue/catalogue.html}{\TeX\ Catalogue}.
% The elements \texttt{caption} and \texttt{description} are imported
% from the original XML file from the Catalogue.
% The name of the XML file in the Catalogue is \xfile{pdfcolparallel.xml}.
%    \begin{macrocode}
%<*catalogue>
<?xml version='1.0' encoding='us-ascii'?>
<!DOCTYPE entry SYSTEM 'catalogue.dtd'>
<entry datestamp='$Date$' modifier='$Author$' id='pdfcolparallel'>
  <name>pdfcolparallel</name>
  <caption>Fix colour problems in package 'parallel'.</caption>
  <authorref id='auth:oberdiek'/>
  <copyright owner='Heiko Oberdiek' year='2007,2008,2010'/>
  <license type='lppl1.3'/>
  <version number='1.4'/>
  <description>
    Since version 1.40 pdfTeX supports colour stacks.
    This package uses them to fix colour problems in
    package <xref refid='parallel'>parallel</xref>.
    <p/>
    The package is part of the <xref refid='oberdiek'>oberdiek</xref>
    bundle.
  </description>
  <documentation details='Package documentation'
      href='ctan:/macros/latex/contrib/oberdiek/pdfcolparallel.pdf'/>
  <ctan file='true' path='/macros/latex/contrib/oberdiek/pdfcolparallel.dtx'/>
  <miktex location='oberdiek'/>
  <texlive location='oberdiek'/>
  <install path='/macros/latex/contrib/oberdiek/oberdiek.tds.zip'/>
</entry>
%</catalogue>
%    \end{macrocode}
%
% \begin{thebibliography}{9}
%
% \bibitem{parallel}
%   Matthias Eckermann: \textit{The \xpackage{parallel}-package};
%   2003/04/13;\\
%   \CTAN{macros/latex/contrib/parallel/}.
%
% \bibitem{pdfcol}
%   Heiko Oberdiek: \textit{The \xpackage{pdfcol} package};
%   2007/09/09;\\
%   \CTAN{macros/latex/contrib/oberdiek/pdfcol.pdf}.
%
% \end{thebibliography}
%
% \begin{History}
%   \begin{Version}{2007/09/09 v1.0}
%   \item
%     First version.
%   \end{Version}
%   \begin{Version}{2007/12/12 v1.1}
%   \item
%     Adds patch for setting \cs{linewidth} to fix bug
%     in package \xpackage{parallel}.
%   \item
%     Package \xpackage{parallel} is also fixed if color
%     stacks are not available.
%   \item
%     Bug fix, switched stacks now initialized with current color.
%   \item
%     Fix for package \xpackage{parallel}: \cs{raggedbottom} is respected.
%   \end{Version}
%   \begin{Version}{2008/08/11 v1.2}
%   \item
%     Code is not changed.
%   \item
%     URLs updated.
%   \end{Version}
%   \begin{Version}{2010/01/11 v1.3}
%   \item
%     Option `rulebetweencolor' added.
%   \end{Version}
%   \begin{Version}{2016/05/16 v1.4}
%   \item
%     Documentation updates.
%   \end{Version}
% \end{History}
%
% \PrintIndex
%
% \Finale
\endinput
|
% \end{quote}
% Do not forget to quote the argument according to the demands
% of your shell.
%
% \paragraph{Generating the documentation.}
% You can use both the \xfile{.dtx} or the \xfile{.drv} to generate
% the documentation. The process can be configured by the
% configuration file \xfile{ltxdoc.cfg}. For instance, put this
% line into this file, if you want to have A4 as paper format:
% \begin{quote}
%   \verb|\PassOptionsToClass{a4paper}{article}|
% \end{quote}
% An example follows how to generate the
% documentation with pdf\LaTeX:
% \begin{quote}
%\begin{verbatim}
%pdflatex pdfcolparallel.dtx
%makeindex -s gind.ist pdfcolparallel.idx
%pdflatex pdfcolparallel.dtx
%makeindex -s gind.ist pdfcolparallel.idx
%pdflatex pdfcolparallel.dtx
%\end{verbatim}
% \end{quote}
%
% \section{Catalogue}
%
% The following XML file can be used as source for the
% \href{http://mirror.ctan.org/help/Catalogue/catalogue.html}{\TeX\ Catalogue}.
% The elements \texttt{caption} and \texttt{description} are imported
% from the original XML file from the Catalogue.
% The name of the XML file in the Catalogue is \xfile{pdfcolparallel.xml}.
%    \begin{macrocode}
%<*catalogue>
<?xml version='1.0' encoding='us-ascii'?>
<!DOCTYPE entry SYSTEM 'catalogue.dtd'>
<entry datestamp='$Date$' modifier='$Author$' id='pdfcolparallel'>
  <name>pdfcolparallel</name>
  <caption>Fix colour problems in package 'parallel'.</caption>
  <authorref id='auth:oberdiek'/>
  <copyright owner='Heiko Oberdiek' year='2007,2008,2010'/>
  <license type='lppl1.3'/>
  <version number='1.4'/>
  <description>
    Since version 1.40 pdfTeX supports colour stacks.
    This package uses them to fix colour problems in
    package <xref refid='parallel'>parallel</xref>.
    <p/>
    The package is part of the <xref refid='oberdiek'>oberdiek</xref>
    bundle.
  </description>
  <documentation details='Package documentation'
      href='ctan:/macros/latex/contrib/oberdiek/pdfcolparallel.pdf'/>
  <ctan file='true' path='/macros/latex/contrib/oberdiek/pdfcolparallel.dtx'/>
  <miktex location='oberdiek'/>
  <texlive location='oberdiek'/>
  <install path='/macros/latex/contrib/oberdiek/oberdiek.tds.zip'/>
</entry>
%</catalogue>
%    \end{macrocode}
%
% \begin{thebibliography}{9}
%
% \bibitem{parallel}
%   Matthias Eckermann: \textit{The \xpackage{parallel}-package};
%   2003/04/13;\\
%   \CTAN{macros/latex/contrib/parallel/}.
%
% \bibitem{pdfcol}
%   Heiko Oberdiek: \textit{The \xpackage{pdfcol} package};
%   2007/09/09;\\
%   \CTAN{macros/latex/contrib/oberdiek/pdfcol.pdf}.
%
% \end{thebibliography}
%
% \begin{History}
%   \begin{Version}{2007/09/09 v1.0}
%   \item
%     First version.
%   \end{Version}
%   \begin{Version}{2007/12/12 v1.1}
%   \item
%     Adds patch for setting \cs{linewidth} to fix bug
%     in package \xpackage{parallel}.
%   \item
%     Package \xpackage{parallel} is also fixed if color
%     stacks are not available.
%   \item
%     Bug fix, switched stacks now initialized with current color.
%   \item
%     Fix for package \xpackage{parallel}: \cs{raggedbottom} is respected.
%   \end{Version}
%   \begin{Version}{2008/08/11 v1.2}
%   \item
%     Code is not changed.
%   \item
%     URLs updated.
%   \end{Version}
%   \begin{Version}{2010/01/11 v1.3}
%   \item
%     Option `rulebetweencolor' added.
%   \end{Version}
%   \begin{Version}{2016/05/16 v1.4}
%   \item
%     Documentation updates.
%   \end{Version}
% \end{History}
%
% \PrintIndex
%
% \Finale
\endinput

%        (quote the arguments according to the demands of your shell)
%
% Documentation:
%    (a) If pdfcolparallel.drv is present:
%           latex pdfcolparallel.drv
%    (b) Without pdfcolparallel.drv:
%           latex pdfcolparallel.dtx; ...
%    The class ltxdoc loads the configuration file ltxdoc.cfg
%    if available. Here you can specify further options, e.g.
%    use A4 as paper format:
%       \PassOptionsToClass{a4paper}{article}
%
%    Programm calls to get the documentation (example):
%       pdflatex pdfcolparallel.dtx
%       makeindex -s gind.ist pdfcolparallel.idx
%       pdflatex pdfcolparallel.dtx
%       makeindex -s gind.ist pdfcolparallel.idx
%       pdflatex pdfcolparallel.dtx
%
% Installation:
%    TDS:tex/latex/oberdiek/pdfcolparallel.sty
%    TDS:doc/latex/oberdiek/pdfcolparallel.pdf
%    TDS:doc/latex/oberdiek/test/pdfcolparallel-test1.tex
%    TDS:source/latex/oberdiek/pdfcolparallel.dtx
%
%<*ignore>
\begingroup
  \catcode123=1 %
  \catcode125=2 %
  \def\x{LaTeX2e}%
\expandafter\endgroup
\ifcase 0\ifx\install y1\fi\expandafter
         \ifx\csname processbatchFile\endcsname\relax\else1\fi
         \ifx\fmtname\x\else 1\fi\relax
\else\csname fi\endcsname
%</ignore>
%<*install>
\input docstrip.tex
\Msg{************************************************************************}
\Msg{* Installation}
\Msg{* Package: pdfcolparallel 2016/05/16 v1.4 Color stacks support for parallel (HO)}
\Msg{************************************************************************}

\keepsilent
\askforoverwritefalse

\let\MetaPrefix\relax
\preamble

This is a generated file.

Project: pdfcolparallel
Version: 2016/05/16 v1.4

Copyright (C) 2007, 2008, 2010 by
   Heiko Oberdiek <heiko.oberdiek at googlemail.com>

This work may be distributed and/or modified under the
conditions of the LaTeX Project Public License, either
version 1.3c of this license or (at your option) any later
version. This version of this license is in
   http://www.latex-project.org/lppl/lppl-1-3c.txt
and the latest version of this license is in
   http://www.latex-project.org/lppl.txt
and version 1.3 or later is part of all distributions of
LaTeX version 2005/12/01 or later.

This work has the LPPL maintenance status "maintained".

This Current Maintainer of this work is Heiko Oberdiek.

This work consists of the main source file pdfcolparallel.dtx
and the derived files
   pdfcolparallel.sty, pdfcolparallel.pdf, pdfcolparallel.ins,
   pdfcolparallel.drv, pdfcolparallel-test1.tex.

\endpreamble
\let\MetaPrefix\DoubleperCent

\generate{%
  \file{pdfcolparallel.ins}{\from{pdfcolparallel.dtx}{install}}%
  \file{pdfcolparallel.drv}{\from{pdfcolparallel.dtx}{driver}}%
  \usedir{tex/latex/oberdiek}%
  \file{pdfcolparallel.sty}{\from{pdfcolparallel.dtx}{package}}%
  \usedir{doc/latex/oberdiek/test}%
  \file{pdfcolparallel-test1.tex}{\from{pdfcolparallel.dtx}{test1}}%
  \nopreamble
  \nopostamble
  \usedir{source/latex/oberdiek/catalogue}%
  \file{pdfcolparallel.xml}{\from{pdfcolparallel.dtx}{catalogue}}%
}

\catcode32=13\relax% active space
\let =\space%
\Msg{************************************************************************}
\Msg{*}
\Msg{* To finish the installation you have to move the following}
\Msg{* file into a directory searched by TeX:}
\Msg{*}
\Msg{*     pdfcolparallel.sty}
\Msg{*}
\Msg{* To produce the documentation run the file `pdfcolparallel.drv'}
\Msg{* through LaTeX.}
\Msg{*}
\Msg{* Happy TeXing!}
\Msg{*}
\Msg{************************************************************************}

\endbatchfile
%</install>
%<*ignore>
\fi
%</ignore>
%<*driver>
\NeedsTeXFormat{LaTeX2e}
\ProvidesFile{pdfcolparallel.drv}%
  [2016/05/16 v1.4 Color stacks support for parallel (HO)]%
\documentclass{ltxdoc}
\usepackage{holtxdoc}[2011/11/22]
\begin{document}
  \DocInput{pdfcolparallel.dtx}%
\end{document}
%</driver>
% \fi
%
%
% \CharacterTable
%  {Upper-case    \A\B\C\D\E\F\G\H\I\J\K\L\M\N\O\P\Q\R\S\T\U\V\W\X\Y\Z
%   Lower-case    \a\b\c\d\e\f\g\h\i\j\k\l\m\n\o\p\q\r\s\t\u\v\w\x\y\z
%   Digits        \0\1\2\3\4\5\6\7\8\9
%   Exclamation   \!     Double quote  \"     Hash (number) \#
%   Dollar        \$     Percent       \%     Ampersand     \&
%   Acute accent  \'     Left paren    \(     Right paren   \)
%   Asterisk      \*     Plus          \+     Comma         \,
%   Minus         \-     Point         \.     Solidus       \/
%   Colon         \:     Semicolon     \;     Less than     \<
%   Equals        \=     Greater than  \>     Question mark \?
%   Commercial at \@     Left bracket  \[     Backslash     \\
%   Right bracket \]     Circumflex    \^     Underscore    \_
%   Grave accent  \`     Left brace    \{     Vertical bar  \|
%   Right brace   \}     Tilde         \~}
%
% \GetFileInfo{pdfcolparallel.drv}
%
% \title{The \xpackage{pdfcolparallel} package}
% \date{2016/05/16 v1.4}
% \author{Heiko Oberdiek\thanks
% {Please report any issues at https://github.com/ho-tex/oberdiek/issues}\\
% \xemail{heiko.oberdiek at googlemail.com}}
%
% \maketitle
%
% \begin{abstract}
% This packages fixes bugs in \xpackage{parallel} and
% improves color support by using several color stacks
% that are provided by \pdfTeX\ since version 1.40.
% \end{abstract}
%
% \tableofcontents
%
% \section{Usage}
%
% \begin{quote}
% |\usepackage{pdfcolparallel}|
% \end{quote}
% The package \xpackage{pdfcolparallel} loads package \xpackage{parallel}
% \cite{parallel} and redefines some macros to fix bugs.
%
% If color stacks are available then package \xpackage{parallel}
% is further patched to support them.
%
% \subsection{Option \xoption{rulebetweencolor}}
%
% Package \xpackage{pdfcolparallel} also fixes the color for the
% rule between columns.
% Default color is \cs{normalcolor}. But this can be changed by using
% option \xoption{rulebetweencolor} for |\setkeys{parallel}|
% (see package \xpackage{keyval}). The option takes a color specification
% as value. If the value is empty, then the default (\cs{normalcolor})
% is used.
% Examples:
% \begin{quote}
%   |\setkeys{parallel}{rulebetweencolor=blue}|,\\
%   |\setkeys{parallel}{rulebetweencolor={red}}|,\\
%   |\setkeys{parallel}{rulebetweencolor={}}|,
%     \textit{\% \cs{normalcolor} is used}\\
%   |\setkeys{parallel}{rulebetweencolor=[rgb]{1,0,.5}}|
% \end{quote}
%
% \subsection{Future}
%
% If there will be a new version of package \xpackage{parallel}
% that adds support for color stacks, then this package may become
% obsolete.
%
% \StopEventually{
% }
%
% \section{Implementation}
%
% \subsection{Identification}
%
%    \begin{macrocode}
%<*package>
\NeedsTeXFormat{LaTeX2e}
\ProvidesPackage{pdfcolparallel}%
  [2016/05/16 v1.4 Color stacks support for parallel (HO)]%
%    \end{macrocode}
%
% \subsection{Load and fix package \xpackage{parallel}}
%
%    Package \xpackage{parallel} is loaded. Before options of package
%    \xpackage{pdfcolparallel} are passed to package \xpackage{parallel}.
%    \begin{macrocode}
\DeclareOption*{%
  \PassoptionsToPackage{\CurrentOption}{parallel}%
}
\ProcessOptions\relax
\RequirePackage{parallel}[2003/04/13]
%    \end{macrocode}
%
%    \begin{macrocode}
\RequirePackage{infwarerr}[2007/09/09]
%    \end{macrocode}
%
%    \begin{macro}{\pcp@ColorPatch}
%    \begin{macrocode}
\begingroup\expandafter\expandafter\expandafter\endgroup
\expandafter\ifx\csname currentgrouplevel\endcsname\relax
  \def\pcp@ColorPatch{}%
\else
  \def\pcp@ColorPatch{%
    \@ifundefined{set@color}{%
      \gdef\pcp@ColorPatch{}%
    }{%
      \gdef\pcp@ColorPatch{%
        \gdef\pcp@ColorResets{}%
        \bgroup
        \aftergroup\pcp@ColorResets
        \aftergroup\egroup
        \let\pcp@OrgSetColor\set@color
        \let\set@color\pcp@SetColor
        \edef\pcp@GroupLevel{\the\currentgrouplevel}%
      }%
    }%
    \pcp@ColorPatch
  }%
%    \end{macrocode}
%    \end{macro}
%    \begin{macro}{\pcp@SetColor}
%    \begin{macrocode}
  \def\pcp@SetColor{%
    \ifnum\pcp@GroupLevel=\currentgrouplevel
      \let\pcp@OrgAfterGroup\aftergroup
      \def\aftergroup{%
        \g@addto@macro\pcp@ColorResets
      }%
      \pcp@OrgSetColor
      \let\aftergroup\pcp@OrgAfterGroup
    \else
      \pcp@OrgSetColor
    \fi
  }%
\fi
%    \end{macrocode}
%    \end{macro}
%
%    \begin{macro}{\pcp@CmdCheckRedef}
%    \begin{macrocode}
\def\pcp@CmdCheckRedef#1{%
  \begingroup
    \def\pcp@cmd{#1}%
    \afterassignment\pcp@CmdDo
    \long\def\reserved@a
}
\def\pcp@CmdDo{%
    \expandafter\ifx\pcp@cmd\reserved@a
    \else
      \edef\x*{\expandafter\string\pcp@cmd}%
      \@PackageWarningNoLine{pdfcolparallel}{%
        Command \x* has changed.\MessageBreak
        Supported versions of package `parallel':\MessageBreak
        \space\space 2003/04/13\MessageBreak
        The redefinition of \x* may\MessageBreak
        not behave correctly depending on the changes%
      }%
    \fi
  \expandafter\endgroup
  \expandafter\def\pcp@cmd
}
%    \end{macrocode}
%    \end{macro}
%
%    \begin{macrocode}
\def\pcp@SwitchStack#1#2{}
%    \end{macrocode}
%    \begin{macrocode}
\def\pcp@SetCurrent#1{}
%    \end{macrocode}
%
%    \begin{macro}{\ParallelLText}
%    \begin{macrocode}
\pcp@CmdCheckRedef\ParallelLText{%
  \everypar{}%
  \@restorepar
  \begingroup
    \hbadness=3000 %
    \let\footnote=\ParallelLFootnote
    \ParallelWhichBox=0 %
    \global\setbox\ParallelLBox=\vbox\bgroup
      \hsize=\ParallelLWidth
      \aftergroup\ParallelAfterText
      \begingroup
        \afterassignment\ParallelCheckOpenBrace
        \let\x=%
}{%
  \everypar{}%
  \@restorepar
  \@nobreakfalse
  \begingroup
    \hbadness=3000 %
    \let\footnote=\ParallelLFootnote
    \ParallelWhichBox=0 %
    \global\setbox\ParallelLBox=\vbox\bgroup
      \hsize=\ParallelLWidth
      \linewidth=\ParallelLWidth
      \pcp@SwitchStack{Left}\ParallelLBox
      \aftergroup\ParallelAfterText
      \pcp@ColorPatch
      \begingroup
        \afterassignment\ParallelCheckOpenBrace
        \let\x=%
}
%    \end{macrocode}
%    \end{macro}
%
%    \begin{macro}{\ParallelRText}
%    \begin{macrocode}
\pcp@CmdCheckRedef\ParallelRText{%
  \everypar{}%
  \@restorepar
  \begingroup
    \hbadness=3000 %
    \ifnum\ParallelFNMode=\@ne
      \let\footnote=\ParallelRFootnote
    \else
      \let\footnote=\ParallelLFootnote
    \fi
    \ParallelWhichBox=\@ne
    \global\setbox\ParallelRBox=\vbox\bgroup
      \hsize=\ParallelRWidth
      \aftergroup\ParallelAfterText
      \begingroup
        \afterassignment\ParallelCheckOpenBrace
        \let\x=%
}{%
  \everypar{}%
  \@restorepar
  \@nobreakfalse
  \begingroup
    \hbadness=3000 %
    \ifnum\ParallelFNMode=\@ne
      \let\footnote=\ParallelRFootnote
    \else
      \let\footnote=\ParallelLFootnote
    \fi
    \ParallelWhichBox=\@ne
    \global\setbox\ParallelRBox=\vbox\bgroup
      \hsize=\ParallelRWidth
      \linewidth=\ParallelRWidth
      \pcp@SwitchStack{Right}\ParallelRBox
      \aftergroup\ParallelAfterText
      \pcp@ColorPatch
      \begingroup
        \afterassignment\ParallelCheckOpenBrace
        \let\x=%
}
%    \end{macrocode}
%    \end{macro}
%
%    \begin{macro}{\ParallelParTwoPages}
%    \begin{macrocode}
\pcp@CmdCheckRedef\ParallelParTwoPages{%
  \ifnum\ParallelBoolVar=\@ne
    \par
    \begingroup
      \global\ParallelWhichBox=\@ne
      \newpage
      \vbadness=10000 %
      \vfuzz=3ex %
      \splittopskip=\z@skip
      \loop%
        \ifnum\ParallelBoolVar=\@ne%
          \ifnum\ParallelWhichBox=\@ne
            \ifvoid\ParallelLBox
              \mbox{} %
              \newpage
            \else
              \global\ParallelWhichBox=\z@
            \fi
          \else
            \ifvoid\ParallelRBox
              \mbox{} %
              \newpage
            \else
              \global\ParallelWhichBox=\@ne
            \fi
          \fi
          \ifnum\ParallelWhichBox=\z@
            \ifodd\thepage
              \mbox{} %
              \newpage
            \fi
            \hbox to\textwidth{%
              \vbox{\vsplit\ParallelLBox to.98\textheight}%
            }%
          \else
            \ifodd\thepage\relax
            \else
              \mbox{} %
              \newpage
            \fi
            \hbox to\textwidth{%
              \vbox{\vsplit\ParallelRBox to.98\textheight}%
            }%
          \fi
          \vspace*{\fill}%
          \newpage
        \fi
        \ifvoid\ParallelLBox
          \ifvoid\ParallelRBox
            \global\ParallelBoolVar=\z@
          \fi
        \fi
      \ifnum\ParallelBoolVar=\@ne
      \repeat
      \par
    \endgroup
  \fi
}{%
%    \end{macrocode}
%    Additional fixes:
%    \begin{itemize}
%    \item Unnecessary white space removed.
%    \item |\ifodd\thepage| changed to |\ifodd\value{page}|.
%    \end{itemize}
%    \begin{macrocode}
  \ifnum\ParallelBoolVar=\@ne
    \par
    \begingroup
      \global\ParallelWhichBox=\@ne
      \newpage
      \vbadness=10000 %
      \vfuzz=3ex %
      \splittopskip=\z@skip
      \loop%
        \ifnum\ParallelBoolVar=\@ne%
          \ifnum\ParallelWhichBox=\@ne
            \ifvoid\ParallelLBox
              \mbox{}%
              \newpage
            \else
              \global\ParallelWhichBox=\z@
            \fi
          \else
            \ifvoid\ParallelRBox
              \null
              \newpage
            \else
              \global\ParallelWhichBox=\@ne
            \fi
          \fi
          \ifnum\ParallelWhichBox=\z@
            \ifodd\value{page}%
              \null
              \newpage
            \fi
            \hbox to\textwidth{%
              \pcp@SetCurrent{Left}%
              \setbox\z@=\vsplit\ParallelLBox to.98\textheight
              \vbox to.98\textheight{%
                \@texttop
                \unvbox\z@
                \@textbottom
              }%
            }%
          \else
            \ifodd\value{page}%
            \else
              \mbox{}%
              \newpage
            \fi
            \hbox to\textwidth{%
              \pcp@SetCurrent{Right}%
              \setbox\z@=\vsplit\ParallelRBox to.98\textheight
              \vbox to.98\textheight{%
                \@texttop
                \unvbox\z@
                \@textbottom
              }%
            }%
          \fi
          \vspace*{\fill}%
          \newpage
        \fi
        \ifvoid\ParallelLBox
          \ifvoid\ParallelRBox
            \global\ParallelBoolVar=\z@
          \fi
        \fi
      \ifnum\ParallelBoolVar=\@ne
      \repeat
      \par
    \endgroup
    \pcp@SetCurrent{}%
  \fi
}
%    \end{macrocode}
%    \end{macro}
%
% \subsection{Color stack support}
%
%    \begin{macrocode}
\RequirePackage{pdfcol}[2007/12/12]
\ifpdfcolAvailable
\else
  \PackageInfo{pdfcolparallel}{%
    Loading aborted, because color stacks are not available%
  }%
  \expandafter\endinput
\fi
%    \end{macrocode}
%
%    \begin{macrocode}
\pdfcolInitStack{pcp@Left}
\pdfcolInitStack{pcp@Right}
%    \end{macrocode}
%    \begin{macro}{\pcp@Box}
%    \begin{macrocode}
\newbox\pcp@Box
%    \end{macrocode}
%    \end{macro}
%    \begin{macro}{\pcp@SwitchStack}
%    \begin{macrocode}
\def\pcp@SwitchStack#1#2{%
  \pdfcolSwitchStack{pcp@#1}%
  \global\setbox\pcp@Box=\vbox to 0pt{%
    \pdfcolSetCurrentColor
  }%
  \aftergroup\pcp@FixBox
  \aftergroup#2%
}
%    \end{macrocode}
%    \end{macro}
%    \begin{macro}{\pcp@FixBox}
%    \begin{macrocode}
\def\pcp@FixBox#1{%
  \global\setbox#1=\vbox{%
    \unvbox\pcp@Box
    \unvbox#1%
  }%
}
%    \end{macrocode}
%    \end{macro}
%    \begin{macro}{\pcp@SetCurrent}
%    \begin{macrocode}
\def\pcp@SetCurrent#1{%
  \ifx\\#1\\%
    \pdfcolSetCurrent{}%
  \else
    \pdfcolSetCurrent{pcp@#1}%
  \fi
}
%    \end{macrocode}
%    \end{macro}
%
% \subsection{Redefinitions}
%
%    \begin{macro}{\ParallelParOnePage}
%    \begin{macrocode}
\pcp@CmdCheckRedef\ParallelParOnePage{%
  \ifnum\ParallelBoolVar=\@ne
    \par
    \begingroup
      \leftmargin=\z@
      \rightmargin=\z@
      \parskip=\z@skip
      \parindent=\z@
      \vbadness=10000 %
      \vfuzz=3ex %
      \splittopskip=\z@skip
      \loop
        \ifnum\ParallelBoolVar=\@ne
          \noindent
          \hbox to\textwidth{%
            \hskip\ParallelLeftMargin
            \hbox to\ParallelTextWidth{%
              \ifvoid\ParallelLBox
                \hskip\ParallelLWidth
              \else
                \ParallelWhichBox=\z@
                \vbox{%
                  \setbox\ParallelBoxVar
                      =\vsplit\ParallelLBox to\dp\strutbox
                  \unvbox\ParallelBoxVar
                }%
              \fi
              \strut
              \ifnum\ParallelBoolMid=\@ne
                \hskip\ParallelMainMidSkip
                \vrule
              \else
                \hss
              \fi
              \hss
              \ifvoid\ParallelRBox
                \hskip\ParallelRWidth
              \else
                \ParallelWhichBox=\@ne
                \vbox{%
                  \setbox\ParallelBoxVar
                      =\vsplit\ParallelRBox to\dp\strutbox
                  \unvbox\ParallelBoxVar
                }%
              \fi
            }%
          }%
          \ifvoid\ParallelLBox
            \ifvoid\ParallelRBox
              \global\ParallelBoolVar=\z@
            \fi
          \fi%
        \fi%
      \ifnum\ParallelBoolVar=\@ne
        \penalty\interlinepenalty
      \repeat
      \par
    \endgroup
  \fi
}{%
  \ifnum\ParallelBoolVar=\@ne
    \par
    \begingroup
      \leftmargin=\z@
      \rightmargin=\z@
      \parskip=\z@skip
      \parindent=\z@
      \vbadness=10000 %
      \vfuzz=3ex %
      \splittopskip=\z@skip
      \loop
        \ifnum\ParallelBoolVar=\@ne
          \noindent
          \hbox to\textwidth{%
            \hskip\ParallelLeftMargin
            \hbox to\ParallelTextWidth{%
              \ifvoid\ParallelLBox
                \hskip\ParallelLWidth
              \else
                \pcp@SetCurrent{Left}%
                \ParallelWhichBox=\z@
                \vbox{%
                  \setbox\ParallelBoxVar
                      =\vsplit\ParallelLBox to\dp\strutbox
                  \unvbox\ParallelBoxVar
                }%
              \fi
              \strut
              \ifnum\ParallelBoolMid=\@ne
                \hskip\ParallelMainMidSkip
                \begingroup
                  \pcp@RuleBetweenColor
                  \vrule
                \endgroup
              \else
                \hss
              \fi
              \hss
              \ifvoid\ParallelRBox
                \hskip\ParallelRWidth
              \else
                \pcp@SetCurrent{Right}%
                \ParallelWhichBox=\@ne
                \vbox{%
                  \setbox\ParallelBoxVar
                      =\vsplit\ParallelRBox to\dp\strutbox
                  \unvbox\ParallelBoxVar
                }%
              \fi
            }%
          }%
          \ifvoid\ParallelLBox
            \ifvoid\ParallelRBox
              \global\ParallelBoolVar=\z@
            \fi
          \fi%
        \fi%
      \ifnum\ParallelBoolVar=\@ne
        \penalty\interlinepenalty
      \repeat
      \par
    \endgroup
    \pcp@SetCurrent{}%
  \fi
}
%    \end{macrocode}
%    \end{macro}
%    \begin{macro}{\pcp@RuleBetweenColorDefault}
%    \begin{macrocode}
\def\pcp@RuleBetweenColorDefault{%
  \normalcolor
}
%    \end{macrocode}
%    \end{macro}
%    \begin{macro}{\pcp@RuleBetweenColor}
%    \begin{macrocode}
\let\pcp@RuleBetweenColor\pcp@RuleBetweenColorDefault
%    \end{macrocode}
%    \end{macro}
%    \begin{macrocode}
\RequirePackage{keyval}
\define@key{parallel}{rulebetweencolor}{%
  \edef\pcp@temp{#1}%
  \ifx\pcp@temp\@empty
    \let\pcp@RuleBetweenColor\pcp@RuleBetweenColorDefault
  \else
    \edef\pcp@temp{%
      \noexpand\@ifnextchar[{%
        \def\noexpand\pcp@RuleBetweenColor{%
          \noexpand\color\pcp@temp
        }%
        \noexpand\pcp@GobbleNil
      }{%
        \def\noexpand\pcp@RuleBetweenColor{%
          \noexpand\color{\pcp@temp}%
        }%
        \noexpand\pcp@GobbleNil
      }%
      \pcp@temp\noexpand\@nil
    }%
    \pcp@temp
  \fi
}
%    \end{macrocode}
%    \begin{macro}{\pcp@GobbleNil}
%    \begin{macrocode}
\long\def\pcp@GobbleNil#1\@nil{}
%    \end{macrocode}
%    \end{macro}
%
%    \begin{macrocode}
%</package>
%    \end{macrocode}
%
% \section{Test}
%
%    The test file is a modified version of the file that
%    Alexander Hirsch has posted in \xnewsgroup{de.comp.text.tex}:
%    \URL{``\link{\texttt{parallel.sty} und farbiger Text}''}^^A
%    {http://groups.google.com/group/de.comp.text.tex/msg/6a759cf33bb071a5}
%    \begin{macrocode}
%<*test1>
\AtEndDocument{%
  \typeout{}%
  \typeout{**************************************}%
  \typeout{*** \space Check the PDF file manually! \space ***}%
  \typeout{**************************************}%
  \typeout{}%
}
\documentclass{article}
\usepackage{xcolor}
\usepackage{pdfcolparallel}[2016/05/16]

\begin{document}
  \color{green}%
  Green%
  \begin{Parallel}{0.47\textwidth}{0.47\textwidth}%
    \ParallelLText{%
      \textcolor{red}{%
        Ein Absatz, der sich ueber zwei Zeilen erstrecken soll. %
        Ein Absatz, der sich ueber zwei Zeilen erstrecken soll.%
      }%
    }%
    \ParallelRText{%
      \textcolor{blue}{%
        Ein Absatz, der sich ueber zwei Zeilen erstrecken soll. %
        Ein Absatz, der sich ueber zwei Zeilen erstrecken soll.%
      }%
    }%
    \ParallelPar
    \ParallelLText{%
      Default %
      \color{red}%
      Ein Absatz, der sich ueber zwei Zeilen erstrecken soll. %
      Ein Absatz, der sich ueber zwei Zeilen erstrecken soll.%
    }%
    \ParallelRText{%
      Default %
      \color{blue}%
      Ein Absatz, der sich ueber zwei Zeilen erstrecken soll. %
      Ein Absatz, der sich ueber zwei Zeilen erstrecken soll.%
    }%
    \ParallelPar
    \ParallelLText{%
      \begin{enumerate}%
      \item left text, left text, left text, left text, %
            left text, left text, left text, left text,%
      \item left text, left text, left text, left text, %
            left text, left text, left text, left text.%
      \end{enumerate}%
    }%
    \ParallelRText{%
      \begin{enumerate}%
      \item right text, right text, right text, right text, %
            right text, right text, right text, right text.%
      \item right text, right text, right text, right text, %
            right text, right text, right text, right text.%
      \end{enumerate}%
    }%
  \end{Parallel}%
  \begin{Parallel}[p]{\textwidth}{\textwidth}%
    \ParallelLText{%
      \textcolor{red}{%
        Ein Absatz, der sich ueber zwei Zeilen erstrecken soll. %
        Ein Absatz, der sich ueber zwei Zeilen erstrecken soll. %
        Foo bar bla bla bla.%
      }%
      \par
      Und noch ein Absatz.%
    }%
    \ParallelRText{%
      \textcolor{blue}{%
        Ein Absatz, der sich ueber zwei Zeilen erstrecken soll. %
        Ein Absatz, der sich ueber zwei Zeilen erstrecken soll. %
        Foo bar bla bla bla.%
      }%
    }%
  \end{Parallel}%
  \begin{Parallel}[p]{\textwidth}{\textwidth}%
    \ParallelLText{%
      \rule{1pt}{.98\textheight}\Huge g%
    }%
    \ParallelRText{%
      \rule{1pt}{.98\textheight}y%
    }%
  \end{Parallel}%
  Green%
\end{document}
%</test1>
%    \end{macrocode}
%
% \section{Installation}
%
% \subsection{Download}
%
% \paragraph{Package.} This package is available on
% CTAN\footnote{\url{http://ctan.org/pkg/pdfcolparallel}}:
% \begin{description}
% \item[\CTAN{macros/latex/contrib/oberdiek/pdfcolparallel.dtx}] The source file.
% \item[\CTAN{macros/latex/contrib/oberdiek/pdfcolparallel.pdf}] Documentation.
% \end{description}
%
%
% \paragraph{Bundle.} All the packages of the bundle `oberdiek'
% are also available in a TDS compliant ZIP archive. There
% the packages are already unpacked and the documentation files
% are generated. The files and directories obey the TDS standard.
% \begin{description}
% \item[\CTAN{install/macros/latex/contrib/oberdiek.tds.zip}]
% \end{description}
% \emph{TDS} refers to the standard ``A Directory Structure
% for \TeX\ Files'' (\CTAN{tds/tds.pdf}). Directories
% with \xfile{texmf} in their name are usually organized this way.
%
% \subsection{Bundle installation}
%
% \paragraph{Unpacking.} Unpack the \xfile{oberdiek.tds.zip} in the
% TDS tree (also known as \xfile{texmf} tree) of your choice.
% Example (linux):
% \begin{quote}
%   |unzip oberdiek.tds.zip -d ~/texmf|
% \end{quote}
%
% \paragraph{Script installation.}
% Check the directory \xfile{TDS:scripts/oberdiek/} for
% scripts that need further installation steps.
% Package \xpackage{attachfile2} comes with the Perl script
% \xfile{pdfatfi.pl} that should be installed in such a way
% that it can be called as \texttt{pdfatfi}.
% Example (linux):
% \begin{quote}
%   |chmod +x scripts/oberdiek/pdfatfi.pl|\\
%   |cp scripts/oberdiek/pdfatfi.pl /usr/local/bin/|
% \end{quote}
%
% \subsection{Package installation}
%
% \paragraph{Unpacking.} The \xfile{.dtx} file is a self-extracting
% \docstrip\ archive. The files are extracted by running the
% \xfile{.dtx} through \plainTeX:
% \begin{quote}
%   \verb|tex pdfcolparallel.dtx|
% \end{quote}
%
% \paragraph{TDS.} Now the different files must be moved into
% the different directories in your installation TDS tree
% (also known as \xfile{texmf} tree):
% \begin{quote}
% \def\t{^^A
% \begin{tabular}{@{}>{\ttfamily}l@{ $\rightarrow$ }>{\ttfamily}l@{}}
%   pdfcolparallel.sty & tex/latex/oberdiek/pdfcolparallel.sty\\
%   pdfcolparallel.pdf & doc/latex/oberdiek/pdfcolparallel.pdf\\
%   test/pdfcolparallel-test1.tex & doc/latex/oberdiek/test/pdfcolparallel-test1.tex\\
%   pdfcolparallel.dtx & source/latex/oberdiek/pdfcolparallel.dtx\\
% \end{tabular}^^A
% }^^A
% \sbox0{\t}^^A
% \ifdim\wd0>\linewidth
%   \begingroup
%     \advance\linewidth by\leftmargin
%     \advance\linewidth by\rightmargin
%   \edef\x{\endgroup
%     \def\noexpand\lw{\the\linewidth}^^A
%   }\x
%   \def\lwbox{^^A
%     \leavevmode
%     \hbox to \linewidth{^^A
%       \kern-\leftmargin\relax
%       \hss
%       \usebox0
%       \hss
%       \kern-\rightmargin\relax
%     }^^A
%   }^^A
%   \ifdim\wd0>\lw
%     \sbox0{\small\t}^^A
%     \ifdim\wd0>\linewidth
%       \ifdim\wd0>\lw
%         \sbox0{\footnotesize\t}^^A
%         \ifdim\wd0>\linewidth
%           \ifdim\wd0>\lw
%             \sbox0{\scriptsize\t}^^A
%             \ifdim\wd0>\linewidth
%               \ifdim\wd0>\lw
%                 \sbox0{\tiny\t}^^A
%                 \ifdim\wd0>\linewidth
%                   \lwbox
%                 \else
%                   \usebox0
%                 \fi
%               \else
%                 \lwbox
%               \fi
%             \else
%               \usebox0
%             \fi
%           \else
%             \lwbox
%           \fi
%         \else
%           \usebox0
%         \fi
%       \else
%         \lwbox
%       \fi
%     \else
%       \usebox0
%     \fi
%   \else
%     \lwbox
%   \fi
% \else
%   \usebox0
% \fi
% \end{quote}
% If you have a \xfile{docstrip.cfg} that configures and enables \docstrip's
% TDS installing feature, then some files can already be in the right
% place, see the documentation of \docstrip.
%
% \subsection{Refresh file name databases}
%
% If your \TeX~distribution
% (\teTeX, \mikTeX, \dots) relies on file name databases, you must refresh
% these. For example, \teTeX\ users run \verb|texhash| or
% \verb|mktexlsr|.
%
% \subsection{Some details for the interested}
%
% \paragraph{Attached source.}
%
% The PDF documentation on CTAN also includes the
% \xfile{.dtx} source file. It can be extracted by
% AcrobatReader 6 or higher. Another option is \textsf{pdftk},
% e.g. unpack the file into the current directory:
% \begin{quote}
%   \verb|pdftk pdfcolparallel.pdf unpack_files output .|
% \end{quote}
%
% \paragraph{Unpacking with \LaTeX.}
% The \xfile{.dtx} chooses its action depending on the format:
% \begin{description}
% \item[\plainTeX:] Run \docstrip\ and extract the files.
% \item[\LaTeX:] Generate the documentation.
% \end{description}
% If you insist on using \LaTeX\ for \docstrip\ (really,
% \docstrip\ does not need \LaTeX), then inform the autodetect routine
% about your intention:
% \begin{quote}
%   \verb|latex \let\install=y% \iffalse meta-comment
%
% File: pdfcolparallel.dtx
% Version: 2016/05/16 v1.4
% Info: Color stacks support for parallel
%
% Copyright (C) 2007, 2008, 2010 by
%    Heiko Oberdiek <heiko.oberdiek at googlemail.com>
%    2016
%    https://github.com/ho-tex/oberdiek/issues
%
% This work may be distributed and/or modified under the
% conditions of the LaTeX Project Public License, either
% version 1.3c of this license or (at your option) any later
% version. This version of this license is in
%    http://www.latex-project.org/lppl/lppl-1-3c.txt
% and the latest version of this license is in
%    http://www.latex-project.org/lppl.txt
% and version 1.3 or later is part of all distributions of
% LaTeX version 2005/12/01 or later.
%
% This work has the LPPL maintenance status "maintained".
%
% This Current Maintainer of this work is Heiko Oberdiek.
%
% This work consists of the main source file pdfcolparallel.dtx
% and the derived files
%    pdfcolparallel.sty, pdfcolparallel.pdf, pdfcolparallel.ins,
%    pdfcolparallel.drv, pdfcolparallel-test1.tex.
%
% Distribution:
%    CTAN:macros/latex/contrib/oberdiek/pdfcolparallel.dtx
%    CTAN:macros/latex/contrib/oberdiek/pdfcolparallel.pdf
%
% Unpacking:
%    (a) If pdfcolparallel.ins is present:
%           tex pdfcolparallel.ins
%    (b) Without pdfcolparallel.ins:
%           tex pdfcolparallel.dtx
%    (c) If you insist on using LaTeX
%           latex \let\install=y% \iffalse meta-comment
%
% File: pdfcolparallel.dtx
% Version: 2016/05/16 v1.4
% Info: Color stacks support for parallel
%
% Copyright (C) 2007, 2008, 2010 by
%    Heiko Oberdiek <heiko.oberdiek at googlemail.com>
%    2016
%    https://github.com/ho-tex/oberdiek/issues
%
% This work may be distributed and/or modified under the
% conditions of the LaTeX Project Public License, either
% version 1.3c of this license or (at your option) any later
% version. This version of this license is in
%    http://www.latex-project.org/lppl/lppl-1-3c.txt
% and the latest version of this license is in
%    http://www.latex-project.org/lppl.txt
% and version 1.3 or later is part of all distributions of
% LaTeX version 2005/12/01 or later.
%
% This work has the LPPL maintenance status "maintained".
%
% This Current Maintainer of this work is Heiko Oberdiek.
%
% This work consists of the main source file pdfcolparallel.dtx
% and the derived files
%    pdfcolparallel.sty, pdfcolparallel.pdf, pdfcolparallel.ins,
%    pdfcolparallel.drv, pdfcolparallel-test1.tex.
%
% Distribution:
%    CTAN:macros/latex/contrib/oberdiek/pdfcolparallel.dtx
%    CTAN:macros/latex/contrib/oberdiek/pdfcolparallel.pdf
%
% Unpacking:
%    (a) If pdfcolparallel.ins is present:
%           tex pdfcolparallel.ins
%    (b) Without pdfcolparallel.ins:
%           tex pdfcolparallel.dtx
%    (c) If you insist on using LaTeX
%           latex \let\install=y% \iffalse meta-comment
%
% File: pdfcolparallel.dtx
% Version: 2016/05/16 v1.4
% Info: Color stacks support for parallel
%
% Copyright (C) 2007, 2008, 2010 by
%    Heiko Oberdiek <heiko.oberdiek at googlemail.com>
%    2016
%    https://github.com/ho-tex/oberdiek/issues
%
% This work may be distributed and/or modified under the
% conditions of the LaTeX Project Public License, either
% version 1.3c of this license or (at your option) any later
% version. This version of this license is in
%    http://www.latex-project.org/lppl/lppl-1-3c.txt
% and the latest version of this license is in
%    http://www.latex-project.org/lppl.txt
% and version 1.3 or later is part of all distributions of
% LaTeX version 2005/12/01 or later.
%
% This work has the LPPL maintenance status "maintained".
%
% This Current Maintainer of this work is Heiko Oberdiek.
%
% This work consists of the main source file pdfcolparallel.dtx
% and the derived files
%    pdfcolparallel.sty, pdfcolparallel.pdf, pdfcolparallel.ins,
%    pdfcolparallel.drv, pdfcolparallel-test1.tex.
%
% Distribution:
%    CTAN:macros/latex/contrib/oberdiek/pdfcolparallel.dtx
%    CTAN:macros/latex/contrib/oberdiek/pdfcolparallel.pdf
%
% Unpacking:
%    (a) If pdfcolparallel.ins is present:
%           tex pdfcolparallel.ins
%    (b) Without pdfcolparallel.ins:
%           tex pdfcolparallel.dtx
%    (c) If you insist on using LaTeX
%           latex \let\install=y\input{pdfcolparallel.dtx}
%        (quote the arguments according to the demands of your shell)
%
% Documentation:
%    (a) If pdfcolparallel.drv is present:
%           latex pdfcolparallel.drv
%    (b) Without pdfcolparallel.drv:
%           latex pdfcolparallel.dtx; ...
%    The class ltxdoc loads the configuration file ltxdoc.cfg
%    if available. Here you can specify further options, e.g.
%    use A4 as paper format:
%       \PassOptionsToClass{a4paper}{article}
%
%    Programm calls to get the documentation (example):
%       pdflatex pdfcolparallel.dtx
%       makeindex -s gind.ist pdfcolparallel.idx
%       pdflatex pdfcolparallel.dtx
%       makeindex -s gind.ist pdfcolparallel.idx
%       pdflatex pdfcolparallel.dtx
%
% Installation:
%    TDS:tex/latex/oberdiek/pdfcolparallel.sty
%    TDS:doc/latex/oberdiek/pdfcolparallel.pdf
%    TDS:doc/latex/oberdiek/test/pdfcolparallel-test1.tex
%    TDS:source/latex/oberdiek/pdfcolparallel.dtx
%
%<*ignore>
\begingroup
  \catcode123=1 %
  \catcode125=2 %
  \def\x{LaTeX2e}%
\expandafter\endgroup
\ifcase 0\ifx\install y1\fi\expandafter
         \ifx\csname processbatchFile\endcsname\relax\else1\fi
         \ifx\fmtname\x\else 1\fi\relax
\else\csname fi\endcsname
%</ignore>
%<*install>
\input docstrip.tex
\Msg{************************************************************************}
\Msg{* Installation}
\Msg{* Package: pdfcolparallel 2016/05/16 v1.4 Color stacks support for parallel (HO)}
\Msg{************************************************************************}

\keepsilent
\askforoverwritefalse

\let\MetaPrefix\relax
\preamble

This is a generated file.

Project: pdfcolparallel
Version: 2016/05/16 v1.4

Copyright (C) 2007, 2008, 2010 by
   Heiko Oberdiek <heiko.oberdiek at googlemail.com>

This work may be distributed and/or modified under the
conditions of the LaTeX Project Public License, either
version 1.3c of this license or (at your option) any later
version. This version of this license is in
   http://www.latex-project.org/lppl/lppl-1-3c.txt
and the latest version of this license is in
   http://www.latex-project.org/lppl.txt
and version 1.3 or later is part of all distributions of
LaTeX version 2005/12/01 or later.

This work has the LPPL maintenance status "maintained".

This Current Maintainer of this work is Heiko Oberdiek.

This work consists of the main source file pdfcolparallel.dtx
and the derived files
   pdfcolparallel.sty, pdfcolparallel.pdf, pdfcolparallel.ins,
   pdfcolparallel.drv, pdfcolparallel-test1.tex.

\endpreamble
\let\MetaPrefix\DoubleperCent

\generate{%
  \file{pdfcolparallel.ins}{\from{pdfcolparallel.dtx}{install}}%
  \file{pdfcolparallel.drv}{\from{pdfcolparallel.dtx}{driver}}%
  \usedir{tex/latex/oberdiek}%
  \file{pdfcolparallel.sty}{\from{pdfcolparallel.dtx}{package}}%
  \usedir{doc/latex/oberdiek/test}%
  \file{pdfcolparallel-test1.tex}{\from{pdfcolparallel.dtx}{test1}}%
  \nopreamble
  \nopostamble
  \usedir{source/latex/oberdiek/catalogue}%
  \file{pdfcolparallel.xml}{\from{pdfcolparallel.dtx}{catalogue}}%
}

\catcode32=13\relax% active space
\let =\space%
\Msg{************************************************************************}
\Msg{*}
\Msg{* To finish the installation you have to move the following}
\Msg{* file into a directory searched by TeX:}
\Msg{*}
\Msg{*     pdfcolparallel.sty}
\Msg{*}
\Msg{* To produce the documentation run the file `pdfcolparallel.drv'}
\Msg{* through LaTeX.}
\Msg{*}
\Msg{* Happy TeXing!}
\Msg{*}
\Msg{************************************************************************}

\endbatchfile
%</install>
%<*ignore>
\fi
%</ignore>
%<*driver>
\NeedsTeXFormat{LaTeX2e}
\ProvidesFile{pdfcolparallel.drv}%
  [2016/05/16 v1.4 Color stacks support for parallel (HO)]%
\documentclass{ltxdoc}
\usepackage{holtxdoc}[2011/11/22]
\begin{document}
  \DocInput{pdfcolparallel.dtx}%
\end{document}
%</driver>
% \fi
%
%
% \CharacterTable
%  {Upper-case    \A\B\C\D\E\F\G\H\I\J\K\L\M\N\O\P\Q\R\S\T\U\V\W\X\Y\Z
%   Lower-case    \a\b\c\d\e\f\g\h\i\j\k\l\m\n\o\p\q\r\s\t\u\v\w\x\y\z
%   Digits        \0\1\2\3\4\5\6\7\8\9
%   Exclamation   \!     Double quote  \"     Hash (number) \#
%   Dollar        \$     Percent       \%     Ampersand     \&
%   Acute accent  \'     Left paren    \(     Right paren   \)
%   Asterisk      \*     Plus          \+     Comma         \,
%   Minus         \-     Point         \.     Solidus       \/
%   Colon         \:     Semicolon     \;     Less than     \<
%   Equals        \=     Greater than  \>     Question mark \?
%   Commercial at \@     Left bracket  \[     Backslash     \\
%   Right bracket \]     Circumflex    \^     Underscore    \_
%   Grave accent  \`     Left brace    \{     Vertical bar  \|
%   Right brace   \}     Tilde         \~}
%
% \GetFileInfo{pdfcolparallel.drv}
%
% \title{The \xpackage{pdfcolparallel} package}
% \date{2016/05/16 v1.4}
% \author{Heiko Oberdiek\thanks
% {Please report any issues at https://github.com/ho-tex/oberdiek/issues}\\
% \xemail{heiko.oberdiek at googlemail.com}}
%
% \maketitle
%
% \begin{abstract}
% This packages fixes bugs in \xpackage{parallel} and
% improves color support by using several color stacks
% that are provided by \pdfTeX\ since version 1.40.
% \end{abstract}
%
% \tableofcontents
%
% \section{Usage}
%
% \begin{quote}
% |\usepackage{pdfcolparallel}|
% \end{quote}
% The package \xpackage{pdfcolparallel} loads package \xpackage{parallel}
% \cite{parallel} and redefines some macros to fix bugs.
%
% If color stacks are available then package \xpackage{parallel}
% is further patched to support them.
%
% \subsection{Option \xoption{rulebetweencolor}}
%
% Package \xpackage{pdfcolparallel} also fixes the color for the
% rule between columns.
% Default color is \cs{normalcolor}. But this can be changed by using
% option \xoption{rulebetweencolor} for |\setkeys{parallel}|
% (see package \xpackage{keyval}). The option takes a color specification
% as value. If the value is empty, then the default (\cs{normalcolor})
% is used.
% Examples:
% \begin{quote}
%   |\setkeys{parallel}{rulebetweencolor=blue}|,\\
%   |\setkeys{parallel}{rulebetweencolor={red}}|,\\
%   |\setkeys{parallel}{rulebetweencolor={}}|,
%     \textit{\% \cs{normalcolor} is used}\\
%   |\setkeys{parallel}{rulebetweencolor=[rgb]{1,0,.5}}|
% \end{quote}
%
% \subsection{Future}
%
% If there will be a new version of package \xpackage{parallel}
% that adds support for color stacks, then this package may become
% obsolete.
%
% \StopEventually{
% }
%
% \section{Implementation}
%
% \subsection{Identification}
%
%    \begin{macrocode}
%<*package>
\NeedsTeXFormat{LaTeX2e}
\ProvidesPackage{pdfcolparallel}%
  [2016/05/16 v1.4 Color stacks support for parallel (HO)]%
%    \end{macrocode}
%
% \subsection{Load and fix package \xpackage{parallel}}
%
%    Package \xpackage{parallel} is loaded. Before options of package
%    \xpackage{pdfcolparallel} are passed to package \xpackage{parallel}.
%    \begin{macrocode}
\DeclareOption*{%
  \PassoptionsToPackage{\CurrentOption}{parallel}%
}
\ProcessOptions\relax
\RequirePackage{parallel}[2003/04/13]
%    \end{macrocode}
%
%    \begin{macrocode}
\RequirePackage{infwarerr}[2007/09/09]
%    \end{macrocode}
%
%    \begin{macro}{\pcp@ColorPatch}
%    \begin{macrocode}
\begingroup\expandafter\expandafter\expandafter\endgroup
\expandafter\ifx\csname currentgrouplevel\endcsname\relax
  \def\pcp@ColorPatch{}%
\else
  \def\pcp@ColorPatch{%
    \@ifundefined{set@color}{%
      \gdef\pcp@ColorPatch{}%
    }{%
      \gdef\pcp@ColorPatch{%
        \gdef\pcp@ColorResets{}%
        \bgroup
        \aftergroup\pcp@ColorResets
        \aftergroup\egroup
        \let\pcp@OrgSetColor\set@color
        \let\set@color\pcp@SetColor
        \edef\pcp@GroupLevel{\the\currentgrouplevel}%
      }%
    }%
    \pcp@ColorPatch
  }%
%    \end{macrocode}
%    \end{macro}
%    \begin{macro}{\pcp@SetColor}
%    \begin{macrocode}
  \def\pcp@SetColor{%
    \ifnum\pcp@GroupLevel=\currentgrouplevel
      \let\pcp@OrgAfterGroup\aftergroup
      \def\aftergroup{%
        \g@addto@macro\pcp@ColorResets
      }%
      \pcp@OrgSetColor
      \let\aftergroup\pcp@OrgAfterGroup
    \else
      \pcp@OrgSetColor
    \fi
  }%
\fi
%    \end{macrocode}
%    \end{macro}
%
%    \begin{macro}{\pcp@CmdCheckRedef}
%    \begin{macrocode}
\def\pcp@CmdCheckRedef#1{%
  \begingroup
    \def\pcp@cmd{#1}%
    \afterassignment\pcp@CmdDo
    \long\def\reserved@a
}
\def\pcp@CmdDo{%
    \expandafter\ifx\pcp@cmd\reserved@a
    \else
      \edef\x*{\expandafter\string\pcp@cmd}%
      \@PackageWarningNoLine{pdfcolparallel}{%
        Command \x* has changed.\MessageBreak
        Supported versions of package `parallel':\MessageBreak
        \space\space 2003/04/13\MessageBreak
        The redefinition of \x* may\MessageBreak
        not behave correctly depending on the changes%
      }%
    \fi
  \expandafter\endgroup
  \expandafter\def\pcp@cmd
}
%    \end{macrocode}
%    \end{macro}
%
%    \begin{macrocode}
\def\pcp@SwitchStack#1#2{}
%    \end{macrocode}
%    \begin{macrocode}
\def\pcp@SetCurrent#1{}
%    \end{macrocode}
%
%    \begin{macro}{\ParallelLText}
%    \begin{macrocode}
\pcp@CmdCheckRedef\ParallelLText{%
  \everypar{}%
  \@restorepar
  \begingroup
    \hbadness=3000 %
    \let\footnote=\ParallelLFootnote
    \ParallelWhichBox=0 %
    \global\setbox\ParallelLBox=\vbox\bgroup
      \hsize=\ParallelLWidth
      \aftergroup\ParallelAfterText
      \begingroup
        \afterassignment\ParallelCheckOpenBrace
        \let\x=%
}{%
  \everypar{}%
  \@restorepar
  \@nobreakfalse
  \begingroup
    \hbadness=3000 %
    \let\footnote=\ParallelLFootnote
    \ParallelWhichBox=0 %
    \global\setbox\ParallelLBox=\vbox\bgroup
      \hsize=\ParallelLWidth
      \linewidth=\ParallelLWidth
      \pcp@SwitchStack{Left}\ParallelLBox
      \aftergroup\ParallelAfterText
      \pcp@ColorPatch
      \begingroup
        \afterassignment\ParallelCheckOpenBrace
        \let\x=%
}
%    \end{macrocode}
%    \end{macro}
%
%    \begin{macro}{\ParallelRText}
%    \begin{macrocode}
\pcp@CmdCheckRedef\ParallelRText{%
  \everypar{}%
  \@restorepar
  \begingroup
    \hbadness=3000 %
    \ifnum\ParallelFNMode=\@ne
      \let\footnote=\ParallelRFootnote
    \else
      \let\footnote=\ParallelLFootnote
    \fi
    \ParallelWhichBox=\@ne
    \global\setbox\ParallelRBox=\vbox\bgroup
      \hsize=\ParallelRWidth
      \aftergroup\ParallelAfterText
      \begingroup
        \afterassignment\ParallelCheckOpenBrace
        \let\x=%
}{%
  \everypar{}%
  \@restorepar
  \@nobreakfalse
  \begingroup
    \hbadness=3000 %
    \ifnum\ParallelFNMode=\@ne
      \let\footnote=\ParallelRFootnote
    \else
      \let\footnote=\ParallelLFootnote
    \fi
    \ParallelWhichBox=\@ne
    \global\setbox\ParallelRBox=\vbox\bgroup
      \hsize=\ParallelRWidth
      \linewidth=\ParallelRWidth
      \pcp@SwitchStack{Right}\ParallelRBox
      \aftergroup\ParallelAfterText
      \pcp@ColorPatch
      \begingroup
        \afterassignment\ParallelCheckOpenBrace
        \let\x=%
}
%    \end{macrocode}
%    \end{macro}
%
%    \begin{macro}{\ParallelParTwoPages}
%    \begin{macrocode}
\pcp@CmdCheckRedef\ParallelParTwoPages{%
  \ifnum\ParallelBoolVar=\@ne
    \par
    \begingroup
      \global\ParallelWhichBox=\@ne
      \newpage
      \vbadness=10000 %
      \vfuzz=3ex %
      \splittopskip=\z@skip
      \loop%
        \ifnum\ParallelBoolVar=\@ne%
          \ifnum\ParallelWhichBox=\@ne
            \ifvoid\ParallelLBox
              \mbox{} %
              \newpage
            \else
              \global\ParallelWhichBox=\z@
            \fi
          \else
            \ifvoid\ParallelRBox
              \mbox{} %
              \newpage
            \else
              \global\ParallelWhichBox=\@ne
            \fi
          \fi
          \ifnum\ParallelWhichBox=\z@
            \ifodd\thepage
              \mbox{} %
              \newpage
            \fi
            \hbox to\textwidth{%
              \vbox{\vsplit\ParallelLBox to.98\textheight}%
            }%
          \else
            \ifodd\thepage\relax
            \else
              \mbox{} %
              \newpage
            \fi
            \hbox to\textwidth{%
              \vbox{\vsplit\ParallelRBox to.98\textheight}%
            }%
          \fi
          \vspace*{\fill}%
          \newpage
        \fi
        \ifvoid\ParallelLBox
          \ifvoid\ParallelRBox
            \global\ParallelBoolVar=\z@
          \fi
        \fi
      \ifnum\ParallelBoolVar=\@ne
      \repeat
      \par
    \endgroup
  \fi
}{%
%    \end{macrocode}
%    Additional fixes:
%    \begin{itemize}
%    \item Unnecessary white space removed.
%    \item |\ifodd\thepage| changed to |\ifodd\value{page}|.
%    \end{itemize}
%    \begin{macrocode}
  \ifnum\ParallelBoolVar=\@ne
    \par
    \begingroup
      \global\ParallelWhichBox=\@ne
      \newpage
      \vbadness=10000 %
      \vfuzz=3ex %
      \splittopskip=\z@skip
      \loop%
        \ifnum\ParallelBoolVar=\@ne%
          \ifnum\ParallelWhichBox=\@ne
            \ifvoid\ParallelLBox
              \mbox{}%
              \newpage
            \else
              \global\ParallelWhichBox=\z@
            \fi
          \else
            \ifvoid\ParallelRBox
              \null
              \newpage
            \else
              \global\ParallelWhichBox=\@ne
            \fi
          \fi
          \ifnum\ParallelWhichBox=\z@
            \ifodd\value{page}%
              \null
              \newpage
            \fi
            \hbox to\textwidth{%
              \pcp@SetCurrent{Left}%
              \setbox\z@=\vsplit\ParallelLBox to.98\textheight
              \vbox to.98\textheight{%
                \@texttop
                \unvbox\z@
                \@textbottom
              }%
            }%
          \else
            \ifodd\value{page}%
            \else
              \mbox{}%
              \newpage
            \fi
            \hbox to\textwidth{%
              \pcp@SetCurrent{Right}%
              \setbox\z@=\vsplit\ParallelRBox to.98\textheight
              \vbox to.98\textheight{%
                \@texttop
                \unvbox\z@
                \@textbottom
              }%
            }%
          \fi
          \vspace*{\fill}%
          \newpage
        \fi
        \ifvoid\ParallelLBox
          \ifvoid\ParallelRBox
            \global\ParallelBoolVar=\z@
          \fi
        \fi
      \ifnum\ParallelBoolVar=\@ne
      \repeat
      \par
    \endgroup
    \pcp@SetCurrent{}%
  \fi
}
%    \end{macrocode}
%    \end{macro}
%
% \subsection{Color stack support}
%
%    \begin{macrocode}
\RequirePackage{pdfcol}[2007/12/12]
\ifpdfcolAvailable
\else
  \PackageInfo{pdfcolparallel}{%
    Loading aborted, because color stacks are not available%
  }%
  \expandafter\endinput
\fi
%    \end{macrocode}
%
%    \begin{macrocode}
\pdfcolInitStack{pcp@Left}
\pdfcolInitStack{pcp@Right}
%    \end{macrocode}
%    \begin{macro}{\pcp@Box}
%    \begin{macrocode}
\newbox\pcp@Box
%    \end{macrocode}
%    \end{macro}
%    \begin{macro}{\pcp@SwitchStack}
%    \begin{macrocode}
\def\pcp@SwitchStack#1#2{%
  \pdfcolSwitchStack{pcp@#1}%
  \global\setbox\pcp@Box=\vbox to 0pt{%
    \pdfcolSetCurrentColor
  }%
  \aftergroup\pcp@FixBox
  \aftergroup#2%
}
%    \end{macrocode}
%    \end{macro}
%    \begin{macro}{\pcp@FixBox}
%    \begin{macrocode}
\def\pcp@FixBox#1{%
  \global\setbox#1=\vbox{%
    \unvbox\pcp@Box
    \unvbox#1%
  }%
}
%    \end{macrocode}
%    \end{macro}
%    \begin{macro}{\pcp@SetCurrent}
%    \begin{macrocode}
\def\pcp@SetCurrent#1{%
  \ifx\\#1\\%
    \pdfcolSetCurrent{}%
  \else
    \pdfcolSetCurrent{pcp@#1}%
  \fi
}
%    \end{macrocode}
%    \end{macro}
%
% \subsection{Redefinitions}
%
%    \begin{macro}{\ParallelParOnePage}
%    \begin{macrocode}
\pcp@CmdCheckRedef\ParallelParOnePage{%
  \ifnum\ParallelBoolVar=\@ne
    \par
    \begingroup
      \leftmargin=\z@
      \rightmargin=\z@
      \parskip=\z@skip
      \parindent=\z@
      \vbadness=10000 %
      \vfuzz=3ex %
      \splittopskip=\z@skip
      \loop
        \ifnum\ParallelBoolVar=\@ne
          \noindent
          \hbox to\textwidth{%
            \hskip\ParallelLeftMargin
            \hbox to\ParallelTextWidth{%
              \ifvoid\ParallelLBox
                \hskip\ParallelLWidth
              \else
                \ParallelWhichBox=\z@
                \vbox{%
                  \setbox\ParallelBoxVar
                      =\vsplit\ParallelLBox to\dp\strutbox
                  \unvbox\ParallelBoxVar
                }%
              \fi
              \strut
              \ifnum\ParallelBoolMid=\@ne
                \hskip\ParallelMainMidSkip
                \vrule
              \else
                \hss
              \fi
              \hss
              \ifvoid\ParallelRBox
                \hskip\ParallelRWidth
              \else
                \ParallelWhichBox=\@ne
                \vbox{%
                  \setbox\ParallelBoxVar
                      =\vsplit\ParallelRBox to\dp\strutbox
                  \unvbox\ParallelBoxVar
                }%
              \fi
            }%
          }%
          \ifvoid\ParallelLBox
            \ifvoid\ParallelRBox
              \global\ParallelBoolVar=\z@
            \fi
          \fi%
        \fi%
      \ifnum\ParallelBoolVar=\@ne
        \penalty\interlinepenalty
      \repeat
      \par
    \endgroup
  \fi
}{%
  \ifnum\ParallelBoolVar=\@ne
    \par
    \begingroup
      \leftmargin=\z@
      \rightmargin=\z@
      \parskip=\z@skip
      \parindent=\z@
      \vbadness=10000 %
      \vfuzz=3ex %
      \splittopskip=\z@skip
      \loop
        \ifnum\ParallelBoolVar=\@ne
          \noindent
          \hbox to\textwidth{%
            \hskip\ParallelLeftMargin
            \hbox to\ParallelTextWidth{%
              \ifvoid\ParallelLBox
                \hskip\ParallelLWidth
              \else
                \pcp@SetCurrent{Left}%
                \ParallelWhichBox=\z@
                \vbox{%
                  \setbox\ParallelBoxVar
                      =\vsplit\ParallelLBox to\dp\strutbox
                  \unvbox\ParallelBoxVar
                }%
              \fi
              \strut
              \ifnum\ParallelBoolMid=\@ne
                \hskip\ParallelMainMidSkip
                \begingroup
                  \pcp@RuleBetweenColor
                  \vrule
                \endgroup
              \else
                \hss
              \fi
              \hss
              \ifvoid\ParallelRBox
                \hskip\ParallelRWidth
              \else
                \pcp@SetCurrent{Right}%
                \ParallelWhichBox=\@ne
                \vbox{%
                  \setbox\ParallelBoxVar
                      =\vsplit\ParallelRBox to\dp\strutbox
                  \unvbox\ParallelBoxVar
                }%
              \fi
            }%
          }%
          \ifvoid\ParallelLBox
            \ifvoid\ParallelRBox
              \global\ParallelBoolVar=\z@
            \fi
          \fi%
        \fi%
      \ifnum\ParallelBoolVar=\@ne
        \penalty\interlinepenalty
      \repeat
      \par
    \endgroup
    \pcp@SetCurrent{}%
  \fi
}
%    \end{macrocode}
%    \end{macro}
%    \begin{macro}{\pcp@RuleBetweenColorDefault}
%    \begin{macrocode}
\def\pcp@RuleBetweenColorDefault{%
  \normalcolor
}
%    \end{macrocode}
%    \end{macro}
%    \begin{macro}{\pcp@RuleBetweenColor}
%    \begin{macrocode}
\let\pcp@RuleBetweenColor\pcp@RuleBetweenColorDefault
%    \end{macrocode}
%    \end{macro}
%    \begin{macrocode}
\RequirePackage{keyval}
\define@key{parallel}{rulebetweencolor}{%
  \edef\pcp@temp{#1}%
  \ifx\pcp@temp\@empty
    \let\pcp@RuleBetweenColor\pcp@RuleBetweenColorDefault
  \else
    \edef\pcp@temp{%
      \noexpand\@ifnextchar[{%
        \def\noexpand\pcp@RuleBetweenColor{%
          \noexpand\color\pcp@temp
        }%
        \noexpand\pcp@GobbleNil
      }{%
        \def\noexpand\pcp@RuleBetweenColor{%
          \noexpand\color{\pcp@temp}%
        }%
        \noexpand\pcp@GobbleNil
      }%
      \pcp@temp\noexpand\@nil
    }%
    \pcp@temp
  \fi
}
%    \end{macrocode}
%    \begin{macro}{\pcp@GobbleNil}
%    \begin{macrocode}
\long\def\pcp@GobbleNil#1\@nil{}
%    \end{macrocode}
%    \end{macro}
%
%    \begin{macrocode}
%</package>
%    \end{macrocode}
%
% \section{Test}
%
%    The test file is a modified version of the file that
%    Alexander Hirsch has posted in \xnewsgroup{de.comp.text.tex}:
%    \URL{``\link{\texttt{parallel.sty} und farbiger Text}''}^^A
%    {http://groups.google.com/group/de.comp.text.tex/msg/6a759cf33bb071a5}
%    \begin{macrocode}
%<*test1>
\AtEndDocument{%
  \typeout{}%
  \typeout{**************************************}%
  \typeout{*** \space Check the PDF file manually! \space ***}%
  \typeout{**************************************}%
  \typeout{}%
}
\documentclass{article}
\usepackage{xcolor}
\usepackage{pdfcolparallel}[2016/05/16]

\begin{document}
  \color{green}%
  Green%
  \begin{Parallel}{0.47\textwidth}{0.47\textwidth}%
    \ParallelLText{%
      \textcolor{red}{%
        Ein Absatz, der sich ueber zwei Zeilen erstrecken soll. %
        Ein Absatz, der sich ueber zwei Zeilen erstrecken soll.%
      }%
    }%
    \ParallelRText{%
      \textcolor{blue}{%
        Ein Absatz, der sich ueber zwei Zeilen erstrecken soll. %
        Ein Absatz, der sich ueber zwei Zeilen erstrecken soll.%
      }%
    }%
    \ParallelPar
    \ParallelLText{%
      Default %
      \color{red}%
      Ein Absatz, der sich ueber zwei Zeilen erstrecken soll. %
      Ein Absatz, der sich ueber zwei Zeilen erstrecken soll.%
    }%
    \ParallelRText{%
      Default %
      \color{blue}%
      Ein Absatz, der sich ueber zwei Zeilen erstrecken soll. %
      Ein Absatz, der sich ueber zwei Zeilen erstrecken soll.%
    }%
    \ParallelPar
    \ParallelLText{%
      \begin{enumerate}%
      \item left text, left text, left text, left text, %
            left text, left text, left text, left text,%
      \item left text, left text, left text, left text, %
            left text, left text, left text, left text.%
      \end{enumerate}%
    }%
    \ParallelRText{%
      \begin{enumerate}%
      \item right text, right text, right text, right text, %
            right text, right text, right text, right text.%
      \item right text, right text, right text, right text, %
            right text, right text, right text, right text.%
      \end{enumerate}%
    }%
  \end{Parallel}%
  \begin{Parallel}[p]{\textwidth}{\textwidth}%
    \ParallelLText{%
      \textcolor{red}{%
        Ein Absatz, der sich ueber zwei Zeilen erstrecken soll. %
        Ein Absatz, der sich ueber zwei Zeilen erstrecken soll. %
        Foo bar bla bla bla.%
      }%
      \par
      Und noch ein Absatz.%
    }%
    \ParallelRText{%
      \textcolor{blue}{%
        Ein Absatz, der sich ueber zwei Zeilen erstrecken soll. %
        Ein Absatz, der sich ueber zwei Zeilen erstrecken soll. %
        Foo bar bla bla bla.%
      }%
    }%
  \end{Parallel}%
  \begin{Parallel}[p]{\textwidth}{\textwidth}%
    \ParallelLText{%
      \rule{1pt}{.98\textheight}\Huge g%
    }%
    \ParallelRText{%
      \rule{1pt}{.98\textheight}y%
    }%
  \end{Parallel}%
  Green%
\end{document}
%</test1>
%    \end{macrocode}
%
% \section{Installation}
%
% \subsection{Download}
%
% \paragraph{Package.} This package is available on
% CTAN\footnote{\url{http://ctan.org/pkg/pdfcolparallel}}:
% \begin{description}
% \item[\CTAN{macros/latex/contrib/oberdiek/pdfcolparallel.dtx}] The source file.
% \item[\CTAN{macros/latex/contrib/oberdiek/pdfcolparallel.pdf}] Documentation.
% \end{description}
%
%
% \paragraph{Bundle.} All the packages of the bundle `oberdiek'
% are also available in a TDS compliant ZIP archive. There
% the packages are already unpacked and the documentation files
% are generated. The files and directories obey the TDS standard.
% \begin{description}
% \item[\CTAN{install/macros/latex/contrib/oberdiek.tds.zip}]
% \end{description}
% \emph{TDS} refers to the standard ``A Directory Structure
% for \TeX\ Files'' (\CTAN{tds/tds.pdf}). Directories
% with \xfile{texmf} in their name are usually organized this way.
%
% \subsection{Bundle installation}
%
% \paragraph{Unpacking.} Unpack the \xfile{oberdiek.tds.zip} in the
% TDS tree (also known as \xfile{texmf} tree) of your choice.
% Example (linux):
% \begin{quote}
%   |unzip oberdiek.tds.zip -d ~/texmf|
% \end{quote}
%
% \paragraph{Script installation.}
% Check the directory \xfile{TDS:scripts/oberdiek/} for
% scripts that need further installation steps.
% Package \xpackage{attachfile2} comes with the Perl script
% \xfile{pdfatfi.pl} that should be installed in such a way
% that it can be called as \texttt{pdfatfi}.
% Example (linux):
% \begin{quote}
%   |chmod +x scripts/oberdiek/pdfatfi.pl|\\
%   |cp scripts/oberdiek/pdfatfi.pl /usr/local/bin/|
% \end{quote}
%
% \subsection{Package installation}
%
% \paragraph{Unpacking.} The \xfile{.dtx} file is a self-extracting
% \docstrip\ archive. The files are extracted by running the
% \xfile{.dtx} through \plainTeX:
% \begin{quote}
%   \verb|tex pdfcolparallel.dtx|
% \end{quote}
%
% \paragraph{TDS.} Now the different files must be moved into
% the different directories in your installation TDS tree
% (also known as \xfile{texmf} tree):
% \begin{quote}
% \def\t{^^A
% \begin{tabular}{@{}>{\ttfamily}l@{ $\rightarrow$ }>{\ttfamily}l@{}}
%   pdfcolparallel.sty & tex/latex/oberdiek/pdfcolparallel.sty\\
%   pdfcolparallel.pdf & doc/latex/oberdiek/pdfcolparallel.pdf\\
%   test/pdfcolparallel-test1.tex & doc/latex/oberdiek/test/pdfcolparallel-test1.tex\\
%   pdfcolparallel.dtx & source/latex/oberdiek/pdfcolparallel.dtx\\
% \end{tabular}^^A
% }^^A
% \sbox0{\t}^^A
% \ifdim\wd0>\linewidth
%   \begingroup
%     \advance\linewidth by\leftmargin
%     \advance\linewidth by\rightmargin
%   \edef\x{\endgroup
%     \def\noexpand\lw{\the\linewidth}^^A
%   }\x
%   \def\lwbox{^^A
%     \leavevmode
%     \hbox to \linewidth{^^A
%       \kern-\leftmargin\relax
%       \hss
%       \usebox0
%       \hss
%       \kern-\rightmargin\relax
%     }^^A
%   }^^A
%   \ifdim\wd0>\lw
%     \sbox0{\small\t}^^A
%     \ifdim\wd0>\linewidth
%       \ifdim\wd0>\lw
%         \sbox0{\footnotesize\t}^^A
%         \ifdim\wd0>\linewidth
%           \ifdim\wd0>\lw
%             \sbox0{\scriptsize\t}^^A
%             \ifdim\wd0>\linewidth
%               \ifdim\wd0>\lw
%                 \sbox0{\tiny\t}^^A
%                 \ifdim\wd0>\linewidth
%                   \lwbox
%                 \else
%                   \usebox0
%                 \fi
%               \else
%                 \lwbox
%               \fi
%             \else
%               \usebox0
%             \fi
%           \else
%             \lwbox
%           \fi
%         \else
%           \usebox0
%         \fi
%       \else
%         \lwbox
%       \fi
%     \else
%       \usebox0
%     \fi
%   \else
%     \lwbox
%   \fi
% \else
%   \usebox0
% \fi
% \end{quote}
% If you have a \xfile{docstrip.cfg} that configures and enables \docstrip's
% TDS installing feature, then some files can already be in the right
% place, see the documentation of \docstrip.
%
% \subsection{Refresh file name databases}
%
% If your \TeX~distribution
% (\teTeX, \mikTeX, \dots) relies on file name databases, you must refresh
% these. For example, \teTeX\ users run \verb|texhash| or
% \verb|mktexlsr|.
%
% \subsection{Some details for the interested}
%
% \paragraph{Attached source.}
%
% The PDF documentation on CTAN also includes the
% \xfile{.dtx} source file. It can be extracted by
% AcrobatReader 6 or higher. Another option is \textsf{pdftk},
% e.g. unpack the file into the current directory:
% \begin{quote}
%   \verb|pdftk pdfcolparallel.pdf unpack_files output .|
% \end{quote}
%
% \paragraph{Unpacking with \LaTeX.}
% The \xfile{.dtx} chooses its action depending on the format:
% \begin{description}
% \item[\plainTeX:] Run \docstrip\ and extract the files.
% \item[\LaTeX:] Generate the documentation.
% \end{description}
% If you insist on using \LaTeX\ for \docstrip\ (really,
% \docstrip\ does not need \LaTeX), then inform the autodetect routine
% about your intention:
% \begin{quote}
%   \verb|latex \let\install=y\input{pdfcolparallel.dtx}|
% \end{quote}
% Do not forget to quote the argument according to the demands
% of your shell.
%
% \paragraph{Generating the documentation.}
% You can use both the \xfile{.dtx} or the \xfile{.drv} to generate
% the documentation. The process can be configured by the
% configuration file \xfile{ltxdoc.cfg}. For instance, put this
% line into this file, if you want to have A4 as paper format:
% \begin{quote}
%   \verb|\PassOptionsToClass{a4paper}{article}|
% \end{quote}
% An example follows how to generate the
% documentation with pdf\LaTeX:
% \begin{quote}
%\begin{verbatim}
%pdflatex pdfcolparallel.dtx
%makeindex -s gind.ist pdfcolparallel.idx
%pdflatex pdfcolparallel.dtx
%makeindex -s gind.ist pdfcolparallel.idx
%pdflatex pdfcolparallel.dtx
%\end{verbatim}
% \end{quote}
%
% \section{Catalogue}
%
% The following XML file can be used as source for the
% \href{http://mirror.ctan.org/help/Catalogue/catalogue.html}{\TeX\ Catalogue}.
% The elements \texttt{caption} and \texttt{description} are imported
% from the original XML file from the Catalogue.
% The name of the XML file in the Catalogue is \xfile{pdfcolparallel.xml}.
%    \begin{macrocode}
%<*catalogue>
<?xml version='1.0' encoding='us-ascii'?>
<!DOCTYPE entry SYSTEM 'catalogue.dtd'>
<entry datestamp='$Date$' modifier='$Author$' id='pdfcolparallel'>
  <name>pdfcolparallel</name>
  <caption>Fix colour problems in package 'parallel'.</caption>
  <authorref id='auth:oberdiek'/>
  <copyright owner='Heiko Oberdiek' year='2007,2008,2010'/>
  <license type='lppl1.3'/>
  <version number='1.4'/>
  <description>
    Since version 1.40 pdfTeX supports colour stacks.
    This package uses them to fix colour problems in
    package <xref refid='parallel'>parallel</xref>.
    <p/>
    The package is part of the <xref refid='oberdiek'>oberdiek</xref>
    bundle.
  </description>
  <documentation details='Package documentation'
      href='ctan:/macros/latex/contrib/oberdiek/pdfcolparallel.pdf'/>
  <ctan file='true' path='/macros/latex/contrib/oberdiek/pdfcolparallel.dtx'/>
  <miktex location='oberdiek'/>
  <texlive location='oberdiek'/>
  <install path='/macros/latex/contrib/oberdiek/oberdiek.tds.zip'/>
</entry>
%</catalogue>
%    \end{macrocode}
%
% \begin{thebibliography}{9}
%
% \bibitem{parallel}
%   Matthias Eckermann: \textit{The \xpackage{parallel}-package};
%   2003/04/13;\\
%   \CTAN{macros/latex/contrib/parallel/}.
%
% \bibitem{pdfcol}
%   Heiko Oberdiek: \textit{The \xpackage{pdfcol} package};
%   2007/09/09;\\
%   \CTAN{macros/latex/contrib/oberdiek/pdfcol.pdf}.
%
% \end{thebibliography}
%
% \begin{History}
%   \begin{Version}{2007/09/09 v1.0}
%   \item
%     First version.
%   \end{Version}
%   \begin{Version}{2007/12/12 v1.1}
%   \item
%     Adds patch for setting \cs{linewidth} to fix bug
%     in package \xpackage{parallel}.
%   \item
%     Package \xpackage{parallel} is also fixed if color
%     stacks are not available.
%   \item
%     Bug fix, switched stacks now initialized with current color.
%   \item
%     Fix for package \xpackage{parallel}: \cs{raggedbottom} is respected.
%   \end{Version}
%   \begin{Version}{2008/08/11 v1.2}
%   \item
%     Code is not changed.
%   \item
%     URLs updated.
%   \end{Version}
%   \begin{Version}{2010/01/11 v1.3}
%   \item
%     Option `rulebetweencolor' added.
%   \end{Version}
%   \begin{Version}{2016/05/16 v1.4}
%   \item
%     Documentation updates.
%   \end{Version}
% \end{History}
%
% \PrintIndex
%
% \Finale
\endinput

%        (quote the arguments according to the demands of your shell)
%
% Documentation:
%    (a) If pdfcolparallel.drv is present:
%           latex pdfcolparallel.drv
%    (b) Without pdfcolparallel.drv:
%           latex pdfcolparallel.dtx; ...
%    The class ltxdoc loads the configuration file ltxdoc.cfg
%    if available. Here you can specify further options, e.g.
%    use A4 as paper format:
%       \PassOptionsToClass{a4paper}{article}
%
%    Programm calls to get the documentation (example):
%       pdflatex pdfcolparallel.dtx
%       makeindex -s gind.ist pdfcolparallel.idx
%       pdflatex pdfcolparallel.dtx
%       makeindex -s gind.ist pdfcolparallel.idx
%       pdflatex pdfcolparallel.dtx
%
% Installation:
%    TDS:tex/latex/oberdiek/pdfcolparallel.sty
%    TDS:doc/latex/oberdiek/pdfcolparallel.pdf
%    TDS:doc/latex/oberdiek/test/pdfcolparallel-test1.tex
%    TDS:source/latex/oberdiek/pdfcolparallel.dtx
%
%<*ignore>
\begingroup
  \catcode123=1 %
  \catcode125=2 %
  \def\x{LaTeX2e}%
\expandafter\endgroup
\ifcase 0\ifx\install y1\fi\expandafter
         \ifx\csname processbatchFile\endcsname\relax\else1\fi
         \ifx\fmtname\x\else 1\fi\relax
\else\csname fi\endcsname
%</ignore>
%<*install>
\input docstrip.tex
\Msg{************************************************************************}
\Msg{* Installation}
\Msg{* Package: pdfcolparallel 2016/05/16 v1.4 Color stacks support for parallel (HO)}
\Msg{************************************************************************}

\keepsilent
\askforoverwritefalse

\let\MetaPrefix\relax
\preamble

This is a generated file.

Project: pdfcolparallel
Version: 2016/05/16 v1.4

Copyright (C) 2007, 2008, 2010 by
   Heiko Oberdiek <heiko.oberdiek at googlemail.com>

This work may be distributed and/or modified under the
conditions of the LaTeX Project Public License, either
version 1.3c of this license or (at your option) any later
version. This version of this license is in
   http://www.latex-project.org/lppl/lppl-1-3c.txt
and the latest version of this license is in
   http://www.latex-project.org/lppl.txt
and version 1.3 or later is part of all distributions of
LaTeX version 2005/12/01 or later.

This work has the LPPL maintenance status "maintained".

This Current Maintainer of this work is Heiko Oberdiek.

This work consists of the main source file pdfcolparallel.dtx
and the derived files
   pdfcolparallel.sty, pdfcolparallel.pdf, pdfcolparallel.ins,
   pdfcolparallel.drv, pdfcolparallel-test1.tex.

\endpreamble
\let\MetaPrefix\DoubleperCent

\generate{%
  \file{pdfcolparallel.ins}{\from{pdfcolparallel.dtx}{install}}%
  \file{pdfcolparallel.drv}{\from{pdfcolparallel.dtx}{driver}}%
  \usedir{tex/latex/oberdiek}%
  \file{pdfcolparallel.sty}{\from{pdfcolparallel.dtx}{package}}%
  \usedir{doc/latex/oberdiek/test}%
  \file{pdfcolparallel-test1.tex}{\from{pdfcolparallel.dtx}{test1}}%
  \nopreamble
  \nopostamble
  \usedir{source/latex/oberdiek/catalogue}%
  \file{pdfcolparallel.xml}{\from{pdfcolparallel.dtx}{catalogue}}%
}

\catcode32=13\relax% active space
\let =\space%
\Msg{************************************************************************}
\Msg{*}
\Msg{* To finish the installation you have to move the following}
\Msg{* file into a directory searched by TeX:}
\Msg{*}
\Msg{*     pdfcolparallel.sty}
\Msg{*}
\Msg{* To produce the documentation run the file `pdfcolparallel.drv'}
\Msg{* through LaTeX.}
\Msg{*}
\Msg{* Happy TeXing!}
\Msg{*}
\Msg{************************************************************************}

\endbatchfile
%</install>
%<*ignore>
\fi
%</ignore>
%<*driver>
\NeedsTeXFormat{LaTeX2e}
\ProvidesFile{pdfcolparallel.drv}%
  [2016/05/16 v1.4 Color stacks support for parallel (HO)]%
\documentclass{ltxdoc}
\usepackage{holtxdoc}[2011/11/22]
\begin{document}
  \DocInput{pdfcolparallel.dtx}%
\end{document}
%</driver>
% \fi
%
%
% \CharacterTable
%  {Upper-case    \A\B\C\D\E\F\G\H\I\J\K\L\M\N\O\P\Q\R\S\T\U\V\W\X\Y\Z
%   Lower-case    \a\b\c\d\e\f\g\h\i\j\k\l\m\n\o\p\q\r\s\t\u\v\w\x\y\z
%   Digits        \0\1\2\3\4\5\6\7\8\9
%   Exclamation   \!     Double quote  \"     Hash (number) \#
%   Dollar        \$     Percent       \%     Ampersand     \&
%   Acute accent  \'     Left paren    \(     Right paren   \)
%   Asterisk      \*     Plus          \+     Comma         \,
%   Minus         \-     Point         \.     Solidus       \/
%   Colon         \:     Semicolon     \;     Less than     \<
%   Equals        \=     Greater than  \>     Question mark \?
%   Commercial at \@     Left bracket  \[     Backslash     \\
%   Right bracket \]     Circumflex    \^     Underscore    \_
%   Grave accent  \`     Left brace    \{     Vertical bar  \|
%   Right brace   \}     Tilde         \~}
%
% \GetFileInfo{pdfcolparallel.drv}
%
% \title{The \xpackage{pdfcolparallel} package}
% \date{2016/05/16 v1.4}
% \author{Heiko Oberdiek\thanks
% {Please report any issues at https://github.com/ho-tex/oberdiek/issues}\\
% \xemail{heiko.oberdiek at googlemail.com}}
%
% \maketitle
%
% \begin{abstract}
% This packages fixes bugs in \xpackage{parallel} and
% improves color support by using several color stacks
% that are provided by \pdfTeX\ since version 1.40.
% \end{abstract}
%
% \tableofcontents
%
% \section{Usage}
%
% \begin{quote}
% |\usepackage{pdfcolparallel}|
% \end{quote}
% The package \xpackage{pdfcolparallel} loads package \xpackage{parallel}
% \cite{parallel} and redefines some macros to fix bugs.
%
% If color stacks are available then package \xpackage{parallel}
% is further patched to support them.
%
% \subsection{Option \xoption{rulebetweencolor}}
%
% Package \xpackage{pdfcolparallel} also fixes the color for the
% rule between columns.
% Default color is \cs{normalcolor}. But this can be changed by using
% option \xoption{rulebetweencolor} for |\setkeys{parallel}|
% (see package \xpackage{keyval}). The option takes a color specification
% as value. If the value is empty, then the default (\cs{normalcolor})
% is used.
% Examples:
% \begin{quote}
%   |\setkeys{parallel}{rulebetweencolor=blue}|,\\
%   |\setkeys{parallel}{rulebetweencolor={red}}|,\\
%   |\setkeys{parallel}{rulebetweencolor={}}|,
%     \textit{\% \cs{normalcolor} is used}\\
%   |\setkeys{parallel}{rulebetweencolor=[rgb]{1,0,.5}}|
% \end{quote}
%
% \subsection{Future}
%
% If there will be a new version of package \xpackage{parallel}
% that adds support for color stacks, then this package may become
% obsolete.
%
% \StopEventually{
% }
%
% \section{Implementation}
%
% \subsection{Identification}
%
%    \begin{macrocode}
%<*package>
\NeedsTeXFormat{LaTeX2e}
\ProvidesPackage{pdfcolparallel}%
  [2016/05/16 v1.4 Color stacks support for parallel (HO)]%
%    \end{macrocode}
%
% \subsection{Load and fix package \xpackage{parallel}}
%
%    Package \xpackage{parallel} is loaded. Before options of package
%    \xpackage{pdfcolparallel} are passed to package \xpackage{parallel}.
%    \begin{macrocode}
\DeclareOption*{%
  \PassoptionsToPackage{\CurrentOption}{parallel}%
}
\ProcessOptions\relax
\RequirePackage{parallel}[2003/04/13]
%    \end{macrocode}
%
%    \begin{macrocode}
\RequirePackage{infwarerr}[2007/09/09]
%    \end{macrocode}
%
%    \begin{macro}{\pcp@ColorPatch}
%    \begin{macrocode}
\begingroup\expandafter\expandafter\expandafter\endgroup
\expandafter\ifx\csname currentgrouplevel\endcsname\relax
  \def\pcp@ColorPatch{}%
\else
  \def\pcp@ColorPatch{%
    \@ifundefined{set@color}{%
      \gdef\pcp@ColorPatch{}%
    }{%
      \gdef\pcp@ColorPatch{%
        \gdef\pcp@ColorResets{}%
        \bgroup
        \aftergroup\pcp@ColorResets
        \aftergroup\egroup
        \let\pcp@OrgSetColor\set@color
        \let\set@color\pcp@SetColor
        \edef\pcp@GroupLevel{\the\currentgrouplevel}%
      }%
    }%
    \pcp@ColorPatch
  }%
%    \end{macrocode}
%    \end{macro}
%    \begin{macro}{\pcp@SetColor}
%    \begin{macrocode}
  \def\pcp@SetColor{%
    \ifnum\pcp@GroupLevel=\currentgrouplevel
      \let\pcp@OrgAfterGroup\aftergroup
      \def\aftergroup{%
        \g@addto@macro\pcp@ColorResets
      }%
      \pcp@OrgSetColor
      \let\aftergroup\pcp@OrgAfterGroup
    \else
      \pcp@OrgSetColor
    \fi
  }%
\fi
%    \end{macrocode}
%    \end{macro}
%
%    \begin{macro}{\pcp@CmdCheckRedef}
%    \begin{macrocode}
\def\pcp@CmdCheckRedef#1{%
  \begingroup
    \def\pcp@cmd{#1}%
    \afterassignment\pcp@CmdDo
    \long\def\reserved@a
}
\def\pcp@CmdDo{%
    \expandafter\ifx\pcp@cmd\reserved@a
    \else
      \edef\x*{\expandafter\string\pcp@cmd}%
      \@PackageWarningNoLine{pdfcolparallel}{%
        Command \x* has changed.\MessageBreak
        Supported versions of package `parallel':\MessageBreak
        \space\space 2003/04/13\MessageBreak
        The redefinition of \x* may\MessageBreak
        not behave correctly depending on the changes%
      }%
    \fi
  \expandafter\endgroup
  \expandafter\def\pcp@cmd
}
%    \end{macrocode}
%    \end{macro}
%
%    \begin{macrocode}
\def\pcp@SwitchStack#1#2{}
%    \end{macrocode}
%    \begin{macrocode}
\def\pcp@SetCurrent#1{}
%    \end{macrocode}
%
%    \begin{macro}{\ParallelLText}
%    \begin{macrocode}
\pcp@CmdCheckRedef\ParallelLText{%
  \everypar{}%
  \@restorepar
  \begingroup
    \hbadness=3000 %
    \let\footnote=\ParallelLFootnote
    \ParallelWhichBox=0 %
    \global\setbox\ParallelLBox=\vbox\bgroup
      \hsize=\ParallelLWidth
      \aftergroup\ParallelAfterText
      \begingroup
        \afterassignment\ParallelCheckOpenBrace
        \let\x=%
}{%
  \everypar{}%
  \@restorepar
  \@nobreakfalse
  \begingroup
    \hbadness=3000 %
    \let\footnote=\ParallelLFootnote
    \ParallelWhichBox=0 %
    \global\setbox\ParallelLBox=\vbox\bgroup
      \hsize=\ParallelLWidth
      \linewidth=\ParallelLWidth
      \pcp@SwitchStack{Left}\ParallelLBox
      \aftergroup\ParallelAfterText
      \pcp@ColorPatch
      \begingroup
        \afterassignment\ParallelCheckOpenBrace
        \let\x=%
}
%    \end{macrocode}
%    \end{macro}
%
%    \begin{macro}{\ParallelRText}
%    \begin{macrocode}
\pcp@CmdCheckRedef\ParallelRText{%
  \everypar{}%
  \@restorepar
  \begingroup
    \hbadness=3000 %
    \ifnum\ParallelFNMode=\@ne
      \let\footnote=\ParallelRFootnote
    \else
      \let\footnote=\ParallelLFootnote
    \fi
    \ParallelWhichBox=\@ne
    \global\setbox\ParallelRBox=\vbox\bgroup
      \hsize=\ParallelRWidth
      \aftergroup\ParallelAfterText
      \begingroup
        \afterassignment\ParallelCheckOpenBrace
        \let\x=%
}{%
  \everypar{}%
  \@restorepar
  \@nobreakfalse
  \begingroup
    \hbadness=3000 %
    \ifnum\ParallelFNMode=\@ne
      \let\footnote=\ParallelRFootnote
    \else
      \let\footnote=\ParallelLFootnote
    \fi
    \ParallelWhichBox=\@ne
    \global\setbox\ParallelRBox=\vbox\bgroup
      \hsize=\ParallelRWidth
      \linewidth=\ParallelRWidth
      \pcp@SwitchStack{Right}\ParallelRBox
      \aftergroup\ParallelAfterText
      \pcp@ColorPatch
      \begingroup
        \afterassignment\ParallelCheckOpenBrace
        \let\x=%
}
%    \end{macrocode}
%    \end{macro}
%
%    \begin{macro}{\ParallelParTwoPages}
%    \begin{macrocode}
\pcp@CmdCheckRedef\ParallelParTwoPages{%
  \ifnum\ParallelBoolVar=\@ne
    \par
    \begingroup
      \global\ParallelWhichBox=\@ne
      \newpage
      \vbadness=10000 %
      \vfuzz=3ex %
      \splittopskip=\z@skip
      \loop%
        \ifnum\ParallelBoolVar=\@ne%
          \ifnum\ParallelWhichBox=\@ne
            \ifvoid\ParallelLBox
              \mbox{} %
              \newpage
            \else
              \global\ParallelWhichBox=\z@
            \fi
          \else
            \ifvoid\ParallelRBox
              \mbox{} %
              \newpage
            \else
              \global\ParallelWhichBox=\@ne
            \fi
          \fi
          \ifnum\ParallelWhichBox=\z@
            \ifodd\thepage
              \mbox{} %
              \newpage
            \fi
            \hbox to\textwidth{%
              \vbox{\vsplit\ParallelLBox to.98\textheight}%
            }%
          \else
            \ifodd\thepage\relax
            \else
              \mbox{} %
              \newpage
            \fi
            \hbox to\textwidth{%
              \vbox{\vsplit\ParallelRBox to.98\textheight}%
            }%
          \fi
          \vspace*{\fill}%
          \newpage
        \fi
        \ifvoid\ParallelLBox
          \ifvoid\ParallelRBox
            \global\ParallelBoolVar=\z@
          \fi
        \fi
      \ifnum\ParallelBoolVar=\@ne
      \repeat
      \par
    \endgroup
  \fi
}{%
%    \end{macrocode}
%    Additional fixes:
%    \begin{itemize}
%    \item Unnecessary white space removed.
%    \item |\ifodd\thepage| changed to |\ifodd\value{page}|.
%    \end{itemize}
%    \begin{macrocode}
  \ifnum\ParallelBoolVar=\@ne
    \par
    \begingroup
      \global\ParallelWhichBox=\@ne
      \newpage
      \vbadness=10000 %
      \vfuzz=3ex %
      \splittopskip=\z@skip
      \loop%
        \ifnum\ParallelBoolVar=\@ne%
          \ifnum\ParallelWhichBox=\@ne
            \ifvoid\ParallelLBox
              \mbox{}%
              \newpage
            \else
              \global\ParallelWhichBox=\z@
            \fi
          \else
            \ifvoid\ParallelRBox
              \null
              \newpage
            \else
              \global\ParallelWhichBox=\@ne
            \fi
          \fi
          \ifnum\ParallelWhichBox=\z@
            \ifodd\value{page}%
              \null
              \newpage
            \fi
            \hbox to\textwidth{%
              \pcp@SetCurrent{Left}%
              \setbox\z@=\vsplit\ParallelLBox to.98\textheight
              \vbox to.98\textheight{%
                \@texttop
                \unvbox\z@
                \@textbottom
              }%
            }%
          \else
            \ifodd\value{page}%
            \else
              \mbox{}%
              \newpage
            \fi
            \hbox to\textwidth{%
              \pcp@SetCurrent{Right}%
              \setbox\z@=\vsplit\ParallelRBox to.98\textheight
              \vbox to.98\textheight{%
                \@texttop
                \unvbox\z@
                \@textbottom
              }%
            }%
          \fi
          \vspace*{\fill}%
          \newpage
        \fi
        \ifvoid\ParallelLBox
          \ifvoid\ParallelRBox
            \global\ParallelBoolVar=\z@
          \fi
        \fi
      \ifnum\ParallelBoolVar=\@ne
      \repeat
      \par
    \endgroup
    \pcp@SetCurrent{}%
  \fi
}
%    \end{macrocode}
%    \end{macro}
%
% \subsection{Color stack support}
%
%    \begin{macrocode}
\RequirePackage{pdfcol}[2007/12/12]
\ifpdfcolAvailable
\else
  \PackageInfo{pdfcolparallel}{%
    Loading aborted, because color stacks are not available%
  }%
  \expandafter\endinput
\fi
%    \end{macrocode}
%
%    \begin{macrocode}
\pdfcolInitStack{pcp@Left}
\pdfcolInitStack{pcp@Right}
%    \end{macrocode}
%    \begin{macro}{\pcp@Box}
%    \begin{macrocode}
\newbox\pcp@Box
%    \end{macrocode}
%    \end{macro}
%    \begin{macro}{\pcp@SwitchStack}
%    \begin{macrocode}
\def\pcp@SwitchStack#1#2{%
  \pdfcolSwitchStack{pcp@#1}%
  \global\setbox\pcp@Box=\vbox to 0pt{%
    \pdfcolSetCurrentColor
  }%
  \aftergroup\pcp@FixBox
  \aftergroup#2%
}
%    \end{macrocode}
%    \end{macro}
%    \begin{macro}{\pcp@FixBox}
%    \begin{macrocode}
\def\pcp@FixBox#1{%
  \global\setbox#1=\vbox{%
    \unvbox\pcp@Box
    \unvbox#1%
  }%
}
%    \end{macrocode}
%    \end{macro}
%    \begin{macro}{\pcp@SetCurrent}
%    \begin{macrocode}
\def\pcp@SetCurrent#1{%
  \ifx\\#1\\%
    \pdfcolSetCurrent{}%
  \else
    \pdfcolSetCurrent{pcp@#1}%
  \fi
}
%    \end{macrocode}
%    \end{macro}
%
% \subsection{Redefinitions}
%
%    \begin{macro}{\ParallelParOnePage}
%    \begin{macrocode}
\pcp@CmdCheckRedef\ParallelParOnePage{%
  \ifnum\ParallelBoolVar=\@ne
    \par
    \begingroup
      \leftmargin=\z@
      \rightmargin=\z@
      \parskip=\z@skip
      \parindent=\z@
      \vbadness=10000 %
      \vfuzz=3ex %
      \splittopskip=\z@skip
      \loop
        \ifnum\ParallelBoolVar=\@ne
          \noindent
          \hbox to\textwidth{%
            \hskip\ParallelLeftMargin
            \hbox to\ParallelTextWidth{%
              \ifvoid\ParallelLBox
                \hskip\ParallelLWidth
              \else
                \ParallelWhichBox=\z@
                \vbox{%
                  \setbox\ParallelBoxVar
                      =\vsplit\ParallelLBox to\dp\strutbox
                  \unvbox\ParallelBoxVar
                }%
              \fi
              \strut
              \ifnum\ParallelBoolMid=\@ne
                \hskip\ParallelMainMidSkip
                \vrule
              \else
                \hss
              \fi
              \hss
              \ifvoid\ParallelRBox
                \hskip\ParallelRWidth
              \else
                \ParallelWhichBox=\@ne
                \vbox{%
                  \setbox\ParallelBoxVar
                      =\vsplit\ParallelRBox to\dp\strutbox
                  \unvbox\ParallelBoxVar
                }%
              \fi
            }%
          }%
          \ifvoid\ParallelLBox
            \ifvoid\ParallelRBox
              \global\ParallelBoolVar=\z@
            \fi
          \fi%
        \fi%
      \ifnum\ParallelBoolVar=\@ne
        \penalty\interlinepenalty
      \repeat
      \par
    \endgroup
  \fi
}{%
  \ifnum\ParallelBoolVar=\@ne
    \par
    \begingroup
      \leftmargin=\z@
      \rightmargin=\z@
      \parskip=\z@skip
      \parindent=\z@
      \vbadness=10000 %
      \vfuzz=3ex %
      \splittopskip=\z@skip
      \loop
        \ifnum\ParallelBoolVar=\@ne
          \noindent
          \hbox to\textwidth{%
            \hskip\ParallelLeftMargin
            \hbox to\ParallelTextWidth{%
              \ifvoid\ParallelLBox
                \hskip\ParallelLWidth
              \else
                \pcp@SetCurrent{Left}%
                \ParallelWhichBox=\z@
                \vbox{%
                  \setbox\ParallelBoxVar
                      =\vsplit\ParallelLBox to\dp\strutbox
                  \unvbox\ParallelBoxVar
                }%
              \fi
              \strut
              \ifnum\ParallelBoolMid=\@ne
                \hskip\ParallelMainMidSkip
                \begingroup
                  \pcp@RuleBetweenColor
                  \vrule
                \endgroup
              \else
                \hss
              \fi
              \hss
              \ifvoid\ParallelRBox
                \hskip\ParallelRWidth
              \else
                \pcp@SetCurrent{Right}%
                \ParallelWhichBox=\@ne
                \vbox{%
                  \setbox\ParallelBoxVar
                      =\vsplit\ParallelRBox to\dp\strutbox
                  \unvbox\ParallelBoxVar
                }%
              \fi
            }%
          }%
          \ifvoid\ParallelLBox
            \ifvoid\ParallelRBox
              \global\ParallelBoolVar=\z@
            \fi
          \fi%
        \fi%
      \ifnum\ParallelBoolVar=\@ne
        \penalty\interlinepenalty
      \repeat
      \par
    \endgroup
    \pcp@SetCurrent{}%
  \fi
}
%    \end{macrocode}
%    \end{macro}
%    \begin{macro}{\pcp@RuleBetweenColorDefault}
%    \begin{macrocode}
\def\pcp@RuleBetweenColorDefault{%
  \normalcolor
}
%    \end{macrocode}
%    \end{macro}
%    \begin{macro}{\pcp@RuleBetweenColor}
%    \begin{macrocode}
\let\pcp@RuleBetweenColor\pcp@RuleBetweenColorDefault
%    \end{macrocode}
%    \end{macro}
%    \begin{macrocode}
\RequirePackage{keyval}
\define@key{parallel}{rulebetweencolor}{%
  \edef\pcp@temp{#1}%
  \ifx\pcp@temp\@empty
    \let\pcp@RuleBetweenColor\pcp@RuleBetweenColorDefault
  \else
    \edef\pcp@temp{%
      \noexpand\@ifnextchar[{%
        \def\noexpand\pcp@RuleBetweenColor{%
          \noexpand\color\pcp@temp
        }%
        \noexpand\pcp@GobbleNil
      }{%
        \def\noexpand\pcp@RuleBetweenColor{%
          \noexpand\color{\pcp@temp}%
        }%
        \noexpand\pcp@GobbleNil
      }%
      \pcp@temp\noexpand\@nil
    }%
    \pcp@temp
  \fi
}
%    \end{macrocode}
%    \begin{macro}{\pcp@GobbleNil}
%    \begin{macrocode}
\long\def\pcp@GobbleNil#1\@nil{}
%    \end{macrocode}
%    \end{macro}
%
%    \begin{macrocode}
%</package>
%    \end{macrocode}
%
% \section{Test}
%
%    The test file is a modified version of the file that
%    Alexander Hirsch has posted in \xnewsgroup{de.comp.text.tex}:
%    \URL{``\link{\texttt{parallel.sty} und farbiger Text}''}^^A
%    {http://groups.google.com/group/de.comp.text.tex/msg/6a759cf33bb071a5}
%    \begin{macrocode}
%<*test1>
\AtEndDocument{%
  \typeout{}%
  \typeout{**************************************}%
  \typeout{*** \space Check the PDF file manually! \space ***}%
  \typeout{**************************************}%
  \typeout{}%
}
\documentclass{article}
\usepackage{xcolor}
\usepackage{pdfcolparallel}[2016/05/16]

\begin{document}
  \color{green}%
  Green%
  \begin{Parallel}{0.47\textwidth}{0.47\textwidth}%
    \ParallelLText{%
      \textcolor{red}{%
        Ein Absatz, der sich ueber zwei Zeilen erstrecken soll. %
        Ein Absatz, der sich ueber zwei Zeilen erstrecken soll.%
      }%
    }%
    \ParallelRText{%
      \textcolor{blue}{%
        Ein Absatz, der sich ueber zwei Zeilen erstrecken soll. %
        Ein Absatz, der sich ueber zwei Zeilen erstrecken soll.%
      }%
    }%
    \ParallelPar
    \ParallelLText{%
      Default %
      \color{red}%
      Ein Absatz, der sich ueber zwei Zeilen erstrecken soll. %
      Ein Absatz, der sich ueber zwei Zeilen erstrecken soll.%
    }%
    \ParallelRText{%
      Default %
      \color{blue}%
      Ein Absatz, der sich ueber zwei Zeilen erstrecken soll. %
      Ein Absatz, der sich ueber zwei Zeilen erstrecken soll.%
    }%
    \ParallelPar
    \ParallelLText{%
      \begin{enumerate}%
      \item left text, left text, left text, left text, %
            left text, left text, left text, left text,%
      \item left text, left text, left text, left text, %
            left text, left text, left text, left text.%
      \end{enumerate}%
    }%
    \ParallelRText{%
      \begin{enumerate}%
      \item right text, right text, right text, right text, %
            right text, right text, right text, right text.%
      \item right text, right text, right text, right text, %
            right text, right text, right text, right text.%
      \end{enumerate}%
    }%
  \end{Parallel}%
  \begin{Parallel}[p]{\textwidth}{\textwidth}%
    \ParallelLText{%
      \textcolor{red}{%
        Ein Absatz, der sich ueber zwei Zeilen erstrecken soll. %
        Ein Absatz, der sich ueber zwei Zeilen erstrecken soll. %
        Foo bar bla bla bla.%
      }%
      \par
      Und noch ein Absatz.%
    }%
    \ParallelRText{%
      \textcolor{blue}{%
        Ein Absatz, der sich ueber zwei Zeilen erstrecken soll. %
        Ein Absatz, der sich ueber zwei Zeilen erstrecken soll. %
        Foo bar bla bla bla.%
      }%
    }%
  \end{Parallel}%
  \begin{Parallel}[p]{\textwidth}{\textwidth}%
    \ParallelLText{%
      \rule{1pt}{.98\textheight}\Huge g%
    }%
    \ParallelRText{%
      \rule{1pt}{.98\textheight}y%
    }%
  \end{Parallel}%
  Green%
\end{document}
%</test1>
%    \end{macrocode}
%
% \section{Installation}
%
% \subsection{Download}
%
% \paragraph{Package.} This package is available on
% CTAN\footnote{\url{http://ctan.org/pkg/pdfcolparallel}}:
% \begin{description}
% \item[\CTAN{macros/latex/contrib/oberdiek/pdfcolparallel.dtx}] The source file.
% \item[\CTAN{macros/latex/contrib/oberdiek/pdfcolparallel.pdf}] Documentation.
% \end{description}
%
%
% \paragraph{Bundle.} All the packages of the bundle `oberdiek'
% are also available in a TDS compliant ZIP archive. There
% the packages are already unpacked and the documentation files
% are generated. The files and directories obey the TDS standard.
% \begin{description}
% \item[\CTAN{install/macros/latex/contrib/oberdiek.tds.zip}]
% \end{description}
% \emph{TDS} refers to the standard ``A Directory Structure
% for \TeX\ Files'' (\CTAN{tds/tds.pdf}). Directories
% with \xfile{texmf} in their name are usually organized this way.
%
% \subsection{Bundle installation}
%
% \paragraph{Unpacking.} Unpack the \xfile{oberdiek.tds.zip} in the
% TDS tree (also known as \xfile{texmf} tree) of your choice.
% Example (linux):
% \begin{quote}
%   |unzip oberdiek.tds.zip -d ~/texmf|
% \end{quote}
%
% \paragraph{Script installation.}
% Check the directory \xfile{TDS:scripts/oberdiek/} for
% scripts that need further installation steps.
% Package \xpackage{attachfile2} comes with the Perl script
% \xfile{pdfatfi.pl} that should be installed in such a way
% that it can be called as \texttt{pdfatfi}.
% Example (linux):
% \begin{quote}
%   |chmod +x scripts/oberdiek/pdfatfi.pl|\\
%   |cp scripts/oberdiek/pdfatfi.pl /usr/local/bin/|
% \end{quote}
%
% \subsection{Package installation}
%
% \paragraph{Unpacking.} The \xfile{.dtx} file is a self-extracting
% \docstrip\ archive. The files are extracted by running the
% \xfile{.dtx} through \plainTeX:
% \begin{quote}
%   \verb|tex pdfcolparallel.dtx|
% \end{quote}
%
% \paragraph{TDS.} Now the different files must be moved into
% the different directories in your installation TDS tree
% (also known as \xfile{texmf} tree):
% \begin{quote}
% \def\t{^^A
% \begin{tabular}{@{}>{\ttfamily}l@{ $\rightarrow$ }>{\ttfamily}l@{}}
%   pdfcolparallel.sty & tex/latex/oberdiek/pdfcolparallel.sty\\
%   pdfcolparallel.pdf & doc/latex/oberdiek/pdfcolparallel.pdf\\
%   test/pdfcolparallel-test1.tex & doc/latex/oberdiek/test/pdfcolparallel-test1.tex\\
%   pdfcolparallel.dtx & source/latex/oberdiek/pdfcolparallel.dtx\\
% \end{tabular}^^A
% }^^A
% \sbox0{\t}^^A
% \ifdim\wd0>\linewidth
%   \begingroup
%     \advance\linewidth by\leftmargin
%     \advance\linewidth by\rightmargin
%   \edef\x{\endgroup
%     \def\noexpand\lw{\the\linewidth}^^A
%   }\x
%   \def\lwbox{^^A
%     \leavevmode
%     \hbox to \linewidth{^^A
%       \kern-\leftmargin\relax
%       \hss
%       \usebox0
%       \hss
%       \kern-\rightmargin\relax
%     }^^A
%   }^^A
%   \ifdim\wd0>\lw
%     \sbox0{\small\t}^^A
%     \ifdim\wd0>\linewidth
%       \ifdim\wd0>\lw
%         \sbox0{\footnotesize\t}^^A
%         \ifdim\wd0>\linewidth
%           \ifdim\wd0>\lw
%             \sbox0{\scriptsize\t}^^A
%             \ifdim\wd0>\linewidth
%               \ifdim\wd0>\lw
%                 \sbox0{\tiny\t}^^A
%                 \ifdim\wd0>\linewidth
%                   \lwbox
%                 \else
%                   \usebox0
%                 \fi
%               \else
%                 \lwbox
%               \fi
%             \else
%               \usebox0
%             \fi
%           \else
%             \lwbox
%           \fi
%         \else
%           \usebox0
%         \fi
%       \else
%         \lwbox
%       \fi
%     \else
%       \usebox0
%     \fi
%   \else
%     \lwbox
%   \fi
% \else
%   \usebox0
% \fi
% \end{quote}
% If you have a \xfile{docstrip.cfg} that configures and enables \docstrip's
% TDS installing feature, then some files can already be in the right
% place, see the documentation of \docstrip.
%
% \subsection{Refresh file name databases}
%
% If your \TeX~distribution
% (\teTeX, \mikTeX, \dots) relies on file name databases, you must refresh
% these. For example, \teTeX\ users run \verb|texhash| or
% \verb|mktexlsr|.
%
% \subsection{Some details for the interested}
%
% \paragraph{Attached source.}
%
% The PDF documentation on CTAN also includes the
% \xfile{.dtx} source file. It can be extracted by
% AcrobatReader 6 or higher. Another option is \textsf{pdftk},
% e.g. unpack the file into the current directory:
% \begin{quote}
%   \verb|pdftk pdfcolparallel.pdf unpack_files output .|
% \end{quote}
%
% \paragraph{Unpacking with \LaTeX.}
% The \xfile{.dtx} chooses its action depending on the format:
% \begin{description}
% \item[\plainTeX:] Run \docstrip\ and extract the files.
% \item[\LaTeX:] Generate the documentation.
% \end{description}
% If you insist on using \LaTeX\ for \docstrip\ (really,
% \docstrip\ does not need \LaTeX), then inform the autodetect routine
% about your intention:
% \begin{quote}
%   \verb|latex \let\install=y% \iffalse meta-comment
%
% File: pdfcolparallel.dtx
% Version: 2016/05/16 v1.4
% Info: Color stacks support for parallel
%
% Copyright (C) 2007, 2008, 2010 by
%    Heiko Oberdiek <heiko.oberdiek at googlemail.com>
%    2016
%    https://github.com/ho-tex/oberdiek/issues
%
% This work may be distributed and/or modified under the
% conditions of the LaTeX Project Public License, either
% version 1.3c of this license or (at your option) any later
% version. This version of this license is in
%    http://www.latex-project.org/lppl/lppl-1-3c.txt
% and the latest version of this license is in
%    http://www.latex-project.org/lppl.txt
% and version 1.3 or later is part of all distributions of
% LaTeX version 2005/12/01 or later.
%
% This work has the LPPL maintenance status "maintained".
%
% This Current Maintainer of this work is Heiko Oberdiek.
%
% This work consists of the main source file pdfcolparallel.dtx
% and the derived files
%    pdfcolparallel.sty, pdfcolparallel.pdf, pdfcolparallel.ins,
%    pdfcolparallel.drv, pdfcolparallel-test1.tex.
%
% Distribution:
%    CTAN:macros/latex/contrib/oberdiek/pdfcolparallel.dtx
%    CTAN:macros/latex/contrib/oberdiek/pdfcolparallel.pdf
%
% Unpacking:
%    (a) If pdfcolparallel.ins is present:
%           tex pdfcolparallel.ins
%    (b) Without pdfcolparallel.ins:
%           tex pdfcolparallel.dtx
%    (c) If you insist on using LaTeX
%           latex \let\install=y\input{pdfcolparallel.dtx}
%        (quote the arguments according to the demands of your shell)
%
% Documentation:
%    (a) If pdfcolparallel.drv is present:
%           latex pdfcolparallel.drv
%    (b) Without pdfcolparallel.drv:
%           latex pdfcolparallel.dtx; ...
%    The class ltxdoc loads the configuration file ltxdoc.cfg
%    if available. Here you can specify further options, e.g.
%    use A4 as paper format:
%       \PassOptionsToClass{a4paper}{article}
%
%    Programm calls to get the documentation (example):
%       pdflatex pdfcolparallel.dtx
%       makeindex -s gind.ist pdfcolparallel.idx
%       pdflatex pdfcolparallel.dtx
%       makeindex -s gind.ist pdfcolparallel.idx
%       pdflatex pdfcolparallel.dtx
%
% Installation:
%    TDS:tex/latex/oberdiek/pdfcolparallel.sty
%    TDS:doc/latex/oberdiek/pdfcolparallel.pdf
%    TDS:doc/latex/oberdiek/test/pdfcolparallel-test1.tex
%    TDS:source/latex/oberdiek/pdfcolparallel.dtx
%
%<*ignore>
\begingroup
  \catcode123=1 %
  \catcode125=2 %
  \def\x{LaTeX2e}%
\expandafter\endgroup
\ifcase 0\ifx\install y1\fi\expandafter
         \ifx\csname processbatchFile\endcsname\relax\else1\fi
         \ifx\fmtname\x\else 1\fi\relax
\else\csname fi\endcsname
%</ignore>
%<*install>
\input docstrip.tex
\Msg{************************************************************************}
\Msg{* Installation}
\Msg{* Package: pdfcolparallel 2016/05/16 v1.4 Color stacks support for parallel (HO)}
\Msg{************************************************************************}

\keepsilent
\askforoverwritefalse

\let\MetaPrefix\relax
\preamble

This is a generated file.

Project: pdfcolparallel
Version: 2016/05/16 v1.4

Copyright (C) 2007, 2008, 2010 by
   Heiko Oberdiek <heiko.oberdiek at googlemail.com>

This work may be distributed and/or modified under the
conditions of the LaTeX Project Public License, either
version 1.3c of this license or (at your option) any later
version. This version of this license is in
   http://www.latex-project.org/lppl/lppl-1-3c.txt
and the latest version of this license is in
   http://www.latex-project.org/lppl.txt
and version 1.3 or later is part of all distributions of
LaTeX version 2005/12/01 or later.

This work has the LPPL maintenance status "maintained".

This Current Maintainer of this work is Heiko Oberdiek.

This work consists of the main source file pdfcolparallel.dtx
and the derived files
   pdfcolparallel.sty, pdfcolparallel.pdf, pdfcolparallel.ins,
   pdfcolparallel.drv, pdfcolparallel-test1.tex.

\endpreamble
\let\MetaPrefix\DoubleperCent

\generate{%
  \file{pdfcolparallel.ins}{\from{pdfcolparallel.dtx}{install}}%
  \file{pdfcolparallel.drv}{\from{pdfcolparallel.dtx}{driver}}%
  \usedir{tex/latex/oberdiek}%
  \file{pdfcolparallel.sty}{\from{pdfcolparallel.dtx}{package}}%
  \usedir{doc/latex/oberdiek/test}%
  \file{pdfcolparallel-test1.tex}{\from{pdfcolparallel.dtx}{test1}}%
  \nopreamble
  \nopostamble
  \usedir{source/latex/oberdiek/catalogue}%
  \file{pdfcolparallel.xml}{\from{pdfcolparallel.dtx}{catalogue}}%
}

\catcode32=13\relax% active space
\let =\space%
\Msg{************************************************************************}
\Msg{*}
\Msg{* To finish the installation you have to move the following}
\Msg{* file into a directory searched by TeX:}
\Msg{*}
\Msg{*     pdfcolparallel.sty}
\Msg{*}
\Msg{* To produce the documentation run the file `pdfcolparallel.drv'}
\Msg{* through LaTeX.}
\Msg{*}
\Msg{* Happy TeXing!}
\Msg{*}
\Msg{************************************************************************}

\endbatchfile
%</install>
%<*ignore>
\fi
%</ignore>
%<*driver>
\NeedsTeXFormat{LaTeX2e}
\ProvidesFile{pdfcolparallel.drv}%
  [2016/05/16 v1.4 Color stacks support for parallel (HO)]%
\documentclass{ltxdoc}
\usepackage{holtxdoc}[2011/11/22]
\begin{document}
  \DocInput{pdfcolparallel.dtx}%
\end{document}
%</driver>
% \fi
%
%
% \CharacterTable
%  {Upper-case    \A\B\C\D\E\F\G\H\I\J\K\L\M\N\O\P\Q\R\S\T\U\V\W\X\Y\Z
%   Lower-case    \a\b\c\d\e\f\g\h\i\j\k\l\m\n\o\p\q\r\s\t\u\v\w\x\y\z
%   Digits        \0\1\2\3\4\5\6\7\8\9
%   Exclamation   \!     Double quote  \"     Hash (number) \#
%   Dollar        \$     Percent       \%     Ampersand     \&
%   Acute accent  \'     Left paren    \(     Right paren   \)
%   Asterisk      \*     Plus          \+     Comma         \,
%   Minus         \-     Point         \.     Solidus       \/
%   Colon         \:     Semicolon     \;     Less than     \<
%   Equals        \=     Greater than  \>     Question mark \?
%   Commercial at \@     Left bracket  \[     Backslash     \\
%   Right bracket \]     Circumflex    \^     Underscore    \_
%   Grave accent  \`     Left brace    \{     Vertical bar  \|
%   Right brace   \}     Tilde         \~}
%
% \GetFileInfo{pdfcolparallel.drv}
%
% \title{The \xpackage{pdfcolparallel} package}
% \date{2016/05/16 v1.4}
% \author{Heiko Oberdiek\thanks
% {Please report any issues at https://github.com/ho-tex/oberdiek/issues}\\
% \xemail{heiko.oberdiek at googlemail.com}}
%
% \maketitle
%
% \begin{abstract}
% This packages fixes bugs in \xpackage{parallel} and
% improves color support by using several color stacks
% that are provided by \pdfTeX\ since version 1.40.
% \end{abstract}
%
% \tableofcontents
%
% \section{Usage}
%
% \begin{quote}
% |\usepackage{pdfcolparallel}|
% \end{quote}
% The package \xpackage{pdfcolparallel} loads package \xpackage{parallel}
% \cite{parallel} and redefines some macros to fix bugs.
%
% If color stacks are available then package \xpackage{parallel}
% is further patched to support them.
%
% \subsection{Option \xoption{rulebetweencolor}}
%
% Package \xpackage{pdfcolparallel} also fixes the color for the
% rule between columns.
% Default color is \cs{normalcolor}. But this can be changed by using
% option \xoption{rulebetweencolor} for |\setkeys{parallel}|
% (see package \xpackage{keyval}). The option takes a color specification
% as value. If the value is empty, then the default (\cs{normalcolor})
% is used.
% Examples:
% \begin{quote}
%   |\setkeys{parallel}{rulebetweencolor=blue}|,\\
%   |\setkeys{parallel}{rulebetweencolor={red}}|,\\
%   |\setkeys{parallel}{rulebetweencolor={}}|,
%     \textit{\% \cs{normalcolor} is used}\\
%   |\setkeys{parallel}{rulebetweencolor=[rgb]{1,0,.5}}|
% \end{quote}
%
% \subsection{Future}
%
% If there will be a new version of package \xpackage{parallel}
% that adds support for color stacks, then this package may become
% obsolete.
%
% \StopEventually{
% }
%
% \section{Implementation}
%
% \subsection{Identification}
%
%    \begin{macrocode}
%<*package>
\NeedsTeXFormat{LaTeX2e}
\ProvidesPackage{pdfcolparallel}%
  [2016/05/16 v1.4 Color stacks support for parallel (HO)]%
%    \end{macrocode}
%
% \subsection{Load and fix package \xpackage{parallel}}
%
%    Package \xpackage{parallel} is loaded. Before options of package
%    \xpackage{pdfcolparallel} are passed to package \xpackage{parallel}.
%    \begin{macrocode}
\DeclareOption*{%
  \PassoptionsToPackage{\CurrentOption}{parallel}%
}
\ProcessOptions\relax
\RequirePackage{parallel}[2003/04/13]
%    \end{macrocode}
%
%    \begin{macrocode}
\RequirePackage{infwarerr}[2007/09/09]
%    \end{macrocode}
%
%    \begin{macro}{\pcp@ColorPatch}
%    \begin{macrocode}
\begingroup\expandafter\expandafter\expandafter\endgroup
\expandafter\ifx\csname currentgrouplevel\endcsname\relax
  \def\pcp@ColorPatch{}%
\else
  \def\pcp@ColorPatch{%
    \@ifundefined{set@color}{%
      \gdef\pcp@ColorPatch{}%
    }{%
      \gdef\pcp@ColorPatch{%
        \gdef\pcp@ColorResets{}%
        \bgroup
        \aftergroup\pcp@ColorResets
        \aftergroup\egroup
        \let\pcp@OrgSetColor\set@color
        \let\set@color\pcp@SetColor
        \edef\pcp@GroupLevel{\the\currentgrouplevel}%
      }%
    }%
    \pcp@ColorPatch
  }%
%    \end{macrocode}
%    \end{macro}
%    \begin{macro}{\pcp@SetColor}
%    \begin{macrocode}
  \def\pcp@SetColor{%
    \ifnum\pcp@GroupLevel=\currentgrouplevel
      \let\pcp@OrgAfterGroup\aftergroup
      \def\aftergroup{%
        \g@addto@macro\pcp@ColorResets
      }%
      \pcp@OrgSetColor
      \let\aftergroup\pcp@OrgAfterGroup
    \else
      \pcp@OrgSetColor
    \fi
  }%
\fi
%    \end{macrocode}
%    \end{macro}
%
%    \begin{macro}{\pcp@CmdCheckRedef}
%    \begin{macrocode}
\def\pcp@CmdCheckRedef#1{%
  \begingroup
    \def\pcp@cmd{#1}%
    \afterassignment\pcp@CmdDo
    \long\def\reserved@a
}
\def\pcp@CmdDo{%
    \expandafter\ifx\pcp@cmd\reserved@a
    \else
      \edef\x*{\expandafter\string\pcp@cmd}%
      \@PackageWarningNoLine{pdfcolparallel}{%
        Command \x* has changed.\MessageBreak
        Supported versions of package `parallel':\MessageBreak
        \space\space 2003/04/13\MessageBreak
        The redefinition of \x* may\MessageBreak
        not behave correctly depending on the changes%
      }%
    \fi
  \expandafter\endgroup
  \expandafter\def\pcp@cmd
}
%    \end{macrocode}
%    \end{macro}
%
%    \begin{macrocode}
\def\pcp@SwitchStack#1#2{}
%    \end{macrocode}
%    \begin{macrocode}
\def\pcp@SetCurrent#1{}
%    \end{macrocode}
%
%    \begin{macro}{\ParallelLText}
%    \begin{macrocode}
\pcp@CmdCheckRedef\ParallelLText{%
  \everypar{}%
  \@restorepar
  \begingroup
    \hbadness=3000 %
    \let\footnote=\ParallelLFootnote
    \ParallelWhichBox=0 %
    \global\setbox\ParallelLBox=\vbox\bgroup
      \hsize=\ParallelLWidth
      \aftergroup\ParallelAfterText
      \begingroup
        \afterassignment\ParallelCheckOpenBrace
        \let\x=%
}{%
  \everypar{}%
  \@restorepar
  \@nobreakfalse
  \begingroup
    \hbadness=3000 %
    \let\footnote=\ParallelLFootnote
    \ParallelWhichBox=0 %
    \global\setbox\ParallelLBox=\vbox\bgroup
      \hsize=\ParallelLWidth
      \linewidth=\ParallelLWidth
      \pcp@SwitchStack{Left}\ParallelLBox
      \aftergroup\ParallelAfterText
      \pcp@ColorPatch
      \begingroup
        \afterassignment\ParallelCheckOpenBrace
        \let\x=%
}
%    \end{macrocode}
%    \end{macro}
%
%    \begin{macro}{\ParallelRText}
%    \begin{macrocode}
\pcp@CmdCheckRedef\ParallelRText{%
  \everypar{}%
  \@restorepar
  \begingroup
    \hbadness=3000 %
    \ifnum\ParallelFNMode=\@ne
      \let\footnote=\ParallelRFootnote
    \else
      \let\footnote=\ParallelLFootnote
    \fi
    \ParallelWhichBox=\@ne
    \global\setbox\ParallelRBox=\vbox\bgroup
      \hsize=\ParallelRWidth
      \aftergroup\ParallelAfterText
      \begingroup
        \afterassignment\ParallelCheckOpenBrace
        \let\x=%
}{%
  \everypar{}%
  \@restorepar
  \@nobreakfalse
  \begingroup
    \hbadness=3000 %
    \ifnum\ParallelFNMode=\@ne
      \let\footnote=\ParallelRFootnote
    \else
      \let\footnote=\ParallelLFootnote
    \fi
    \ParallelWhichBox=\@ne
    \global\setbox\ParallelRBox=\vbox\bgroup
      \hsize=\ParallelRWidth
      \linewidth=\ParallelRWidth
      \pcp@SwitchStack{Right}\ParallelRBox
      \aftergroup\ParallelAfterText
      \pcp@ColorPatch
      \begingroup
        \afterassignment\ParallelCheckOpenBrace
        \let\x=%
}
%    \end{macrocode}
%    \end{macro}
%
%    \begin{macro}{\ParallelParTwoPages}
%    \begin{macrocode}
\pcp@CmdCheckRedef\ParallelParTwoPages{%
  \ifnum\ParallelBoolVar=\@ne
    \par
    \begingroup
      \global\ParallelWhichBox=\@ne
      \newpage
      \vbadness=10000 %
      \vfuzz=3ex %
      \splittopskip=\z@skip
      \loop%
        \ifnum\ParallelBoolVar=\@ne%
          \ifnum\ParallelWhichBox=\@ne
            \ifvoid\ParallelLBox
              \mbox{} %
              \newpage
            \else
              \global\ParallelWhichBox=\z@
            \fi
          \else
            \ifvoid\ParallelRBox
              \mbox{} %
              \newpage
            \else
              \global\ParallelWhichBox=\@ne
            \fi
          \fi
          \ifnum\ParallelWhichBox=\z@
            \ifodd\thepage
              \mbox{} %
              \newpage
            \fi
            \hbox to\textwidth{%
              \vbox{\vsplit\ParallelLBox to.98\textheight}%
            }%
          \else
            \ifodd\thepage\relax
            \else
              \mbox{} %
              \newpage
            \fi
            \hbox to\textwidth{%
              \vbox{\vsplit\ParallelRBox to.98\textheight}%
            }%
          \fi
          \vspace*{\fill}%
          \newpage
        \fi
        \ifvoid\ParallelLBox
          \ifvoid\ParallelRBox
            \global\ParallelBoolVar=\z@
          \fi
        \fi
      \ifnum\ParallelBoolVar=\@ne
      \repeat
      \par
    \endgroup
  \fi
}{%
%    \end{macrocode}
%    Additional fixes:
%    \begin{itemize}
%    \item Unnecessary white space removed.
%    \item |\ifodd\thepage| changed to |\ifodd\value{page}|.
%    \end{itemize}
%    \begin{macrocode}
  \ifnum\ParallelBoolVar=\@ne
    \par
    \begingroup
      \global\ParallelWhichBox=\@ne
      \newpage
      \vbadness=10000 %
      \vfuzz=3ex %
      \splittopskip=\z@skip
      \loop%
        \ifnum\ParallelBoolVar=\@ne%
          \ifnum\ParallelWhichBox=\@ne
            \ifvoid\ParallelLBox
              \mbox{}%
              \newpage
            \else
              \global\ParallelWhichBox=\z@
            \fi
          \else
            \ifvoid\ParallelRBox
              \null
              \newpage
            \else
              \global\ParallelWhichBox=\@ne
            \fi
          \fi
          \ifnum\ParallelWhichBox=\z@
            \ifodd\value{page}%
              \null
              \newpage
            \fi
            \hbox to\textwidth{%
              \pcp@SetCurrent{Left}%
              \setbox\z@=\vsplit\ParallelLBox to.98\textheight
              \vbox to.98\textheight{%
                \@texttop
                \unvbox\z@
                \@textbottom
              }%
            }%
          \else
            \ifodd\value{page}%
            \else
              \mbox{}%
              \newpage
            \fi
            \hbox to\textwidth{%
              \pcp@SetCurrent{Right}%
              \setbox\z@=\vsplit\ParallelRBox to.98\textheight
              \vbox to.98\textheight{%
                \@texttop
                \unvbox\z@
                \@textbottom
              }%
            }%
          \fi
          \vspace*{\fill}%
          \newpage
        \fi
        \ifvoid\ParallelLBox
          \ifvoid\ParallelRBox
            \global\ParallelBoolVar=\z@
          \fi
        \fi
      \ifnum\ParallelBoolVar=\@ne
      \repeat
      \par
    \endgroup
    \pcp@SetCurrent{}%
  \fi
}
%    \end{macrocode}
%    \end{macro}
%
% \subsection{Color stack support}
%
%    \begin{macrocode}
\RequirePackage{pdfcol}[2007/12/12]
\ifpdfcolAvailable
\else
  \PackageInfo{pdfcolparallel}{%
    Loading aborted, because color stacks are not available%
  }%
  \expandafter\endinput
\fi
%    \end{macrocode}
%
%    \begin{macrocode}
\pdfcolInitStack{pcp@Left}
\pdfcolInitStack{pcp@Right}
%    \end{macrocode}
%    \begin{macro}{\pcp@Box}
%    \begin{macrocode}
\newbox\pcp@Box
%    \end{macrocode}
%    \end{macro}
%    \begin{macro}{\pcp@SwitchStack}
%    \begin{macrocode}
\def\pcp@SwitchStack#1#2{%
  \pdfcolSwitchStack{pcp@#1}%
  \global\setbox\pcp@Box=\vbox to 0pt{%
    \pdfcolSetCurrentColor
  }%
  \aftergroup\pcp@FixBox
  \aftergroup#2%
}
%    \end{macrocode}
%    \end{macro}
%    \begin{macro}{\pcp@FixBox}
%    \begin{macrocode}
\def\pcp@FixBox#1{%
  \global\setbox#1=\vbox{%
    \unvbox\pcp@Box
    \unvbox#1%
  }%
}
%    \end{macrocode}
%    \end{macro}
%    \begin{macro}{\pcp@SetCurrent}
%    \begin{macrocode}
\def\pcp@SetCurrent#1{%
  \ifx\\#1\\%
    \pdfcolSetCurrent{}%
  \else
    \pdfcolSetCurrent{pcp@#1}%
  \fi
}
%    \end{macrocode}
%    \end{macro}
%
% \subsection{Redefinitions}
%
%    \begin{macro}{\ParallelParOnePage}
%    \begin{macrocode}
\pcp@CmdCheckRedef\ParallelParOnePage{%
  \ifnum\ParallelBoolVar=\@ne
    \par
    \begingroup
      \leftmargin=\z@
      \rightmargin=\z@
      \parskip=\z@skip
      \parindent=\z@
      \vbadness=10000 %
      \vfuzz=3ex %
      \splittopskip=\z@skip
      \loop
        \ifnum\ParallelBoolVar=\@ne
          \noindent
          \hbox to\textwidth{%
            \hskip\ParallelLeftMargin
            \hbox to\ParallelTextWidth{%
              \ifvoid\ParallelLBox
                \hskip\ParallelLWidth
              \else
                \ParallelWhichBox=\z@
                \vbox{%
                  \setbox\ParallelBoxVar
                      =\vsplit\ParallelLBox to\dp\strutbox
                  \unvbox\ParallelBoxVar
                }%
              \fi
              \strut
              \ifnum\ParallelBoolMid=\@ne
                \hskip\ParallelMainMidSkip
                \vrule
              \else
                \hss
              \fi
              \hss
              \ifvoid\ParallelRBox
                \hskip\ParallelRWidth
              \else
                \ParallelWhichBox=\@ne
                \vbox{%
                  \setbox\ParallelBoxVar
                      =\vsplit\ParallelRBox to\dp\strutbox
                  \unvbox\ParallelBoxVar
                }%
              \fi
            }%
          }%
          \ifvoid\ParallelLBox
            \ifvoid\ParallelRBox
              \global\ParallelBoolVar=\z@
            \fi
          \fi%
        \fi%
      \ifnum\ParallelBoolVar=\@ne
        \penalty\interlinepenalty
      \repeat
      \par
    \endgroup
  \fi
}{%
  \ifnum\ParallelBoolVar=\@ne
    \par
    \begingroup
      \leftmargin=\z@
      \rightmargin=\z@
      \parskip=\z@skip
      \parindent=\z@
      \vbadness=10000 %
      \vfuzz=3ex %
      \splittopskip=\z@skip
      \loop
        \ifnum\ParallelBoolVar=\@ne
          \noindent
          \hbox to\textwidth{%
            \hskip\ParallelLeftMargin
            \hbox to\ParallelTextWidth{%
              \ifvoid\ParallelLBox
                \hskip\ParallelLWidth
              \else
                \pcp@SetCurrent{Left}%
                \ParallelWhichBox=\z@
                \vbox{%
                  \setbox\ParallelBoxVar
                      =\vsplit\ParallelLBox to\dp\strutbox
                  \unvbox\ParallelBoxVar
                }%
              \fi
              \strut
              \ifnum\ParallelBoolMid=\@ne
                \hskip\ParallelMainMidSkip
                \begingroup
                  \pcp@RuleBetweenColor
                  \vrule
                \endgroup
              \else
                \hss
              \fi
              \hss
              \ifvoid\ParallelRBox
                \hskip\ParallelRWidth
              \else
                \pcp@SetCurrent{Right}%
                \ParallelWhichBox=\@ne
                \vbox{%
                  \setbox\ParallelBoxVar
                      =\vsplit\ParallelRBox to\dp\strutbox
                  \unvbox\ParallelBoxVar
                }%
              \fi
            }%
          }%
          \ifvoid\ParallelLBox
            \ifvoid\ParallelRBox
              \global\ParallelBoolVar=\z@
            \fi
          \fi%
        \fi%
      \ifnum\ParallelBoolVar=\@ne
        \penalty\interlinepenalty
      \repeat
      \par
    \endgroup
    \pcp@SetCurrent{}%
  \fi
}
%    \end{macrocode}
%    \end{macro}
%    \begin{macro}{\pcp@RuleBetweenColorDefault}
%    \begin{macrocode}
\def\pcp@RuleBetweenColorDefault{%
  \normalcolor
}
%    \end{macrocode}
%    \end{macro}
%    \begin{macro}{\pcp@RuleBetweenColor}
%    \begin{macrocode}
\let\pcp@RuleBetweenColor\pcp@RuleBetweenColorDefault
%    \end{macrocode}
%    \end{macro}
%    \begin{macrocode}
\RequirePackage{keyval}
\define@key{parallel}{rulebetweencolor}{%
  \edef\pcp@temp{#1}%
  \ifx\pcp@temp\@empty
    \let\pcp@RuleBetweenColor\pcp@RuleBetweenColorDefault
  \else
    \edef\pcp@temp{%
      \noexpand\@ifnextchar[{%
        \def\noexpand\pcp@RuleBetweenColor{%
          \noexpand\color\pcp@temp
        }%
        \noexpand\pcp@GobbleNil
      }{%
        \def\noexpand\pcp@RuleBetweenColor{%
          \noexpand\color{\pcp@temp}%
        }%
        \noexpand\pcp@GobbleNil
      }%
      \pcp@temp\noexpand\@nil
    }%
    \pcp@temp
  \fi
}
%    \end{macrocode}
%    \begin{macro}{\pcp@GobbleNil}
%    \begin{macrocode}
\long\def\pcp@GobbleNil#1\@nil{}
%    \end{macrocode}
%    \end{macro}
%
%    \begin{macrocode}
%</package>
%    \end{macrocode}
%
% \section{Test}
%
%    The test file is a modified version of the file that
%    Alexander Hirsch has posted in \xnewsgroup{de.comp.text.tex}:
%    \URL{``\link{\texttt{parallel.sty} und farbiger Text}''}^^A
%    {http://groups.google.com/group/de.comp.text.tex/msg/6a759cf33bb071a5}
%    \begin{macrocode}
%<*test1>
\AtEndDocument{%
  \typeout{}%
  \typeout{**************************************}%
  \typeout{*** \space Check the PDF file manually! \space ***}%
  \typeout{**************************************}%
  \typeout{}%
}
\documentclass{article}
\usepackage{xcolor}
\usepackage{pdfcolparallel}[2016/05/16]

\begin{document}
  \color{green}%
  Green%
  \begin{Parallel}{0.47\textwidth}{0.47\textwidth}%
    \ParallelLText{%
      \textcolor{red}{%
        Ein Absatz, der sich ueber zwei Zeilen erstrecken soll. %
        Ein Absatz, der sich ueber zwei Zeilen erstrecken soll.%
      }%
    }%
    \ParallelRText{%
      \textcolor{blue}{%
        Ein Absatz, der sich ueber zwei Zeilen erstrecken soll. %
        Ein Absatz, der sich ueber zwei Zeilen erstrecken soll.%
      }%
    }%
    \ParallelPar
    \ParallelLText{%
      Default %
      \color{red}%
      Ein Absatz, der sich ueber zwei Zeilen erstrecken soll. %
      Ein Absatz, der sich ueber zwei Zeilen erstrecken soll.%
    }%
    \ParallelRText{%
      Default %
      \color{blue}%
      Ein Absatz, der sich ueber zwei Zeilen erstrecken soll. %
      Ein Absatz, der sich ueber zwei Zeilen erstrecken soll.%
    }%
    \ParallelPar
    \ParallelLText{%
      \begin{enumerate}%
      \item left text, left text, left text, left text, %
            left text, left text, left text, left text,%
      \item left text, left text, left text, left text, %
            left text, left text, left text, left text.%
      \end{enumerate}%
    }%
    \ParallelRText{%
      \begin{enumerate}%
      \item right text, right text, right text, right text, %
            right text, right text, right text, right text.%
      \item right text, right text, right text, right text, %
            right text, right text, right text, right text.%
      \end{enumerate}%
    }%
  \end{Parallel}%
  \begin{Parallel}[p]{\textwidth}{\textwidth}%
    \ParallelLText{%
      \textcolor{red}{%
        Ein Absatz, der sich ueber zwei Zeilen erstrecken soll. %
        Ein Absatz, der sich ueber zwei Zeilen erstrecken soll. %
        Foo bar bla bla bla.%
      }%
      \par
      Und noch ein Absatz.%
    }%
    \ParallelRText{%
      \textcolor{blue}{%
        Ein Absatz, der sich ueber zwei Zeilen erstrecken soll. %
        Ein Absatz, der sich ueber zwei Zeilen erstrecken soll. %
        Foo bar bla bla bla.%
      }%
    }%
  \end{Parallel}%
  \begin{Parallel}[p]{\textwidth}{\textwidth}%
    \ParallelLText{%
      \rule{1pt}{.98\textheight}\Huge g%
    }%
    \ParallelRText{%
      \rule{1pt}{.98\textheight}y%
    }%
  \end{Parallel}%
  Green%
\end{document}
%</test1>
%    \end{macrocode}
%
% \section{Installation}
%
% \subsection{Download}
%
% \paragraph{Package.} This package is available on
% CTAN\footnote{\url{http://ctan.org/pkg/pdfcolparallel}}:
% \begin{description}
% \item[\CTAN{macros/latex/contrib/oberdiek/pdfcolparallel.dtx}] The source file.
% \item[\CTAN{macros/latex/contrib/oberdiek/pdfcolparallel.pdf}] Documentation.
% \end{description}
%
%
% \paragraph{Bundle.} All the packages of the bundle `oberdiek'
% are also available in a TDS compliant ZIP archive. There
% the packages are already unpacked and the documentation files
% are generated. The files and directories obey the TDS standard.
% \begin{description}
% \item[\CTAN{install/macros/latex/contrib/oberdiek.tds.zip}]
% \end{description}
% \emph{TDS} refers to the standard ``A Directory Structure
% for \TeX\ Files'' (\CTAN{tds/tds.pdf}). Directories
% with \xfile{texmf} in their name are usually organized this way.
%
% \subsection{Bundle installation}
%
% \paragraph{Unpacking.} Unpack the \xfile{oberdiek.tds.zip} in the
% TDS tree (also known as \xfile{texmf} tree) of your choice.
% Example (linux):
% \begin{quote}
%   |unzip oberdiek.tds.zip -d ~/texmf|
% \end{quote}
%
% \paragraph{Script installation.}
% Check the directory \xfile{TDS:scripts/oberdiek/} for
% scripts that need further installation steps.
% Package \xpackage{attachfile2} comes with the Perl script
% \xfile{pdfatfi.pl} that should be installed in such a way
% that it can be called as \texttt{pdfatfi}.
% Example (linux):
% \begin{quote}
%   |chmod +x scripts/oberdiek/pdfatfi.pl|\\
%   |cp scripts/oberdiek/pdfatfi.pl /usr/local/bin/|
% \end{quote}
%
% \subsection{Package installation}
%
% \paragraph{Unpacking.} The \xfile{.dtx} file is a self-extracting
% \docstrip\ archive. The files are extracted by running the
% \xfile{.dtx} through \plainTeX:
% \begin{quote}
%   \verb|tex pdfcolparallel.dtx|
% \end{quote}
%
% \paragraph{TDS.} Now the different files must be moved into
% the different directories in your installation TDS tree
% (also known as \xfile{texmf} tree):
% \begin{quote}
% \def\t{^^A
% \begin{tabular}{@{}>{\ttfamily}l@{ $\rightarrow$ }>{\ttfamily}l@{}}
%   pdfcolparallel.sty & tex/latex/oberdiek/pdfcolparallel.sty\\
%   pdfcolparallel.pdf & doc/latex/oberdiek/pdfcolparallel.pdf\\
%   test/pdfcolparallel-test1.tex & doc/latex/oberdiek/test/pdfcolparallel-test1.tex\\
%   pdfcolparallel.dtx & source/latex/oberdiek/pdfcolparallel.dtx\\
% \end{tabular}^^A
% }^^A
% \sbox0{\t}^^A
% \ifdim\wd0>\linewidth
%   \begingroup
%     \advance\linewidth by\leftmargin
%     \advance\linewidth by\rightmargin
%   \edef\x{\endgroup
%     \def\noexpand\lw{\the\linewidth}^^A
%   }\x
%   \def\lwbox{^^A
%     \leavevmode
%     \hbox to \linewidth{^^A
%       \kern-\leftmargin\relax
%       \hss
%       \usebox0
%       \hss
%       \kern-\rightmargin\relax
%     }^^A
%   }^^A
%   \ifdim\wd0>\lw
%     \sbox0{\small\t}^^A
%     \ifdim\wd0>\linewidth
%       \ifdim\wd0>\lw
%         \sbox0{\footnotesize\t}^^A
%         \ifdim\wd0>\linewidth
%           \ifdim\wd0>\lw
%             \sbox0{\scriptsize\t}^^A
%             \ifdim\wd0>\linewidth
%               \ifdim\wd0>\lw
%                 \sbox0{\tiny\t}^^A
%                 \ifdim\wd0>\linewidth
%                   \lwbox
%                 \else
%                   \usebox0
%                 \fi
%               \else
%                 \lwbox
%               \fi
%             \else
%               \usebox0
%             \fi
%           \else
%             \lwbox
%           \fi
%         \else
%           \usebox0
%         \fi
%       \else
%         \lwbox
%       \fi
%     \else
%       \usebox0
%     \fi
%   \else
%     \lwbox
%   \fi
% \else
%   \usebox0
% \fi
% \end{quote}
% If you have a \xfile{docstrip.cfg} that configures and enables \docstrip's
% TDS installing feature, then some files can already be in the right
% place, see the documentation of \docstrip.
%
% \subsection{Refresh file name databases}
%
% If your \TeX~distribution
% (\teTeX, \mikTeX, \dots) relies on file name databases, you must refresh
% these. For example, \teTeX\ users run \verb|texhash| or
% \verb|mktexlsr|.
%
% \subsection{Some details for the interested}
%
% \paragraph{Attached source.}
%
% The PDF documentation on CTAN also includes the
% \xfile{.dtx} source file. It can be extracted by
% AcrobatReader 6 or higher. Another option is \textsf{pdftk},
% e.g. unpack the file into the current directory:
% \begin{quote}
%   \verb|pdftk pdfcolparallel.pdf unpack_files output .|
% \end{quote}
%
% \paragraph{Unpacking with \LaTeX.}
% The \xfile{.dtx} chooses its action depending on the format:
% \begin{description}
% \item[\plainTeX:] Run \docstrip\ and extract the files.
% \item[\LaTeX:] Generate the documentation.
% \end{description}
% If you insist on using \LaTeX\ for \docstrip\ (really,
% \docstrip\ does not need \LaTeX), then inform the autodetect routine
% about your intention:
% \begin{quote}
%   \verb|latex \let\install=y\input{pdfcolparallel.dtx}|
% \end{quote}
% Do not forget to quote the argument according to the demands
% of your shell.
%
% \paragraph{Generating the documentation.}
% You can use both the \xfile{.dtx} or the \xfile{.drv} to generate
% the documentation. The process can be configured by the
% configuration file \xfile{ltxdoc.cfg}. For instance, put this
% line into this file, if you want to have A4 as paper format:
% \begin{quote}
%   \verb|\PassOptionsToClass{a4paper}{article}|
% \end{quote}
% An example follows how to generate the
% documentation with pdf\LaTeX:
% \begin{quote}
%\begin{verbatim}
%pdflatex pdfcolparallel.dtx
%makeindex -s gind.ist pdfcolparallel.idx
%pdflatex pdfcolparallel.dtx
%makeindex -s gind.ist pdfcolparallel.idx
%pdflatex pdfcolparallel.dtx
%\end{verbatim}
% \end{quote}
%
% \section{Catalogue}
%
% The following XML file can be used as source for the
% \href{http://mirror.ctan.org/help/Catalogue/catalogue.html}{\TeX\ Catalogue}.
% The elements \texttt{caption} and \texttt{description} are imported
% from the original XML file from the Catalogue.
% The name of the XML file in the Catalogue is \xfile{pdfcolparallel.xml}.
%    \begin{macrocode}
%<*catalogue>
<?xml version='1.0' encoding='us-ascii'?>
<!DOCTYPE entry SYSTEM 'catalogue.dtd'>
<entry datestamp='$Date$' modifier='$Author$' id='pdfcolparallel'>
  <name>pdfcolparallel</name>
  <caption>Fix colour problems in package 'parallel'.</caption>
  <authorref id='auth:oberdiek'/>
  <copyright owner='Heiko Oberdiek' year='2007,2008,2010'/>
  <license type='lppl1.3'/>
  <version number='1.4'/>
  <description>
    Since version 1.40 pdfTeX supports colour stacks.
    This package uses them to fix colour problems in
    package <xref refid='parallel'>parallel</xref>.
    <p/>
    The package is part of the <xref refid='oberdiek'>oberdiek</xref>
    bundle.
  </description>
  <documentation details='Package documentation'
      href='ctan:/macros/latex/contrib/oberdiek/pdfcolparallel.pdf'/>
  <ctan file='true' path='/macros/latex/contrib/oberdiek/pdfcolparallel.dtx'/>
  <miktex location='oberdiek'/>
  <texlive location='oberdiek'/>
  <install path='/macros/latex/contrib/oberdiek/oberdiek.tds.zip'/>
</entry>
%</catalogue>
%    \end{macrocode}
%
% \begin{thebibliography}{9}
%
% \bibitem{parallel}
%   Matthias Eckermann: \textit{The \xpackage{parallel}-package};
%   2003/04/13;\\
%   \CTAN{macros/latex/contrib/parallel/}.
%
% \bibitem{pdfcol}
%   Heiko Oberdiek: \textit{The \xpackage{pdfcol} package};
%   2007/09/09;\\
%   \CTAN{macros/latex/contrib/oberdiek/pdfcol.pdf}.
%
% \end{thebibliography}
%
% \begin{History}
%   \begin{Version}{2007/09/09 v1.0}
%   \item
%     First version.
%   \end{Version}
%   \begin{Version}{2007/12/12 v1.1}
%   \item
%     Adds patch for setting \cs{linewidth} to fix bug
%     in package \xpackage{parallel}.
%   \item
%     Package \xpackage{parallel} is also fixed if color
%     stacks are not available.
%   \item
%     Bug fix, switched stacks now initialized with current color.
%   \item
%     Fix for package \xpackage{parallel}: \cs{raggedbottom} is respected.
%   \end{Version}
%   \begin{Version}{2008/08/11 v1.2}
%   \item
%     Code is not changed.
%   \item
%     URLs updated.
%   \end{Version}
%   \begin{Version}{2010/01/11 v1.3}
%   \item
%     Option `rulebetweencolor' added.
%   \end{Version}
%   \begin{Version}{2016/05/16 v1.4}
%   \item
%     Documentation updates.
%   \end{Version}
% \end{History}
%
% \PrintIndex
%
% \Finale
\endinput
|
% \end{quote}
% Do not forget to quote the argument according to the demands
% of your shell.
%
% \paragraph{Generating the documentation.}
% You can use both the \xfile{.dtx} or the \xfile{.drv} to generate
% the documentation. The process can be configured by the
% configuration file \xfile{ltxdoc.cfg}. For instance, put this
% line into this file, if you want to have A4 as paper format:
% \begin{quote}
%   \verb|\PassOptionsToClass{a4paper}{article}|
% \end{quote}
% An example follows how to generate the
% documentation with pdf\LaTeX:
% \begin{quote}
%\begin{verbatim}
%pdflatex pdfcolparallel.dtx
%makeindex -s gind.ist pdfcolparallel.idx
%pdflatex pdfcolparallel.dtx
%makeindex -s gind.ist pdfcolparallel.idx
%pdflatex pdfcolparallel.dtx
%\end{verbatim}
% \end{quote}
%
% \section{Catalogue}
%
% The following XML file can be used as source for the
% \href{http://mirror.ctan.org/help/Catalogue/catalogue.html}{\TeX\ Catalogue}.
% The elements \texttt{caption} and \texttt{description} are imported
% from the original XML file from the Catalogue.
% The name of the XML file in the Catalogue is \xfile{pdfcolparallel.xml}.
%    \begin{macrocode}
%<*catalogue>
<?xml version='1.0' encoding='us-ascii'?>
<!DOCTYPE entry SYSTEM 'catalogue.dtd'>
<entry datestamp='$Date$' modifier='$Author$' id='pdfcolparallel'>
  <name>pdfcolparallel</name>
  <caption>Fix colour problems in package 'parallel'.</caption>
  <authorref id='auth:oberdiek'/>
  <copyright owner='Heiko Oberdiek' year='2007,2008,2010'/>
  <license type='lppl1.3'/>
  <version number='1.4'/>
  <description>
    Since version 1.40 pdfTeX supports colour stacks.
    This package uses them to fix colour problems in
    package <xref refid='parallel'>parallel</xref>.
    <p/>
    The package is part of the <xref refid='oberdiek'>oberdiek</xref>
    bundle.
  </description>
  <documentation details='Package documentation'
      href='ctan:/macros/latex/contrib/oberdiek/pdfcolparallel.pdf'/>
  <ctan file='true' path='/macros/latex/contrib/oberdiek/pdfcolparallel.dtx'/>
  <miktex location='oberdiek'/>
  <texlive location='oberdiek'/>
  <install path='/macros/latex/contrib/oberdiek/oberdiek.tds.zip'/>
</entry>
%</catalogue>
%    \end{macrocode}
%
% \begin{thebibliography}{9}
%
% \bibitem{parallel}
%   Matthias Eckermann: \textit{The \xpackage{parallel}-package};
%   2003/04/13;\\
%   \CTAN{macros/latex/contrib/parallel/}.
%
% \bibitem{pdfcol}
%   Heiko Oberdiek: \textit{The \xpackage{pdfcol} package};
%   2007/09/09;\\
%   \CTAN{macros/latex/contrib/oberdiek/pdfcol.pdf}.
%
% \end{thebibliography}
%
% \begin{History}
%   \begin{Version}{2007/09/09 v1.0}
%   \item
%     First version.
%   \end{Version}
%   \begin{Version}{2007/12/12 v1.1}
%   \item
%     Adds patch for setting \cs{linewidth} to fix bug
%     in package \xpackage{parallel}.
%   \item
%     Package \xpackage{parallel} is also fixed if color
%     stacks are not available.
%   \item
%     Bug fix, switched stacks now initialized with current color.
%   \item
%     Fix for package \xpackage{parallel}: \cs{raggedbottom} is respected.
%   \end{Version}
%   \begin{Version}{2008/08/11 v1.2}
%   \item
%     Code is not changed.
%   \item
%     URLs updated.
%   \end{Version}
%   \begin{Version}{2010/01/11 v1.3}
%   \item
%     Option `rulebetweencolor' added.
%   \end{Version}
%   \begin{Version}{2016/05/16 v1.4}
%   \item
%     Documentation updates.
%   \end{Version}
% \end{History}
%
% \PrintIndex
%
% \Finale
\endinput

%        (quote the arguments according to the demands of your shell)
%
% Documentation:
%    (a) If pdfcolparallel.drv is present:
%           latex pdfcolparallel.drv
%    (b) Without pdfcolparallel.drv:
%           latex pdfcolparallel.dtx; ...
%    The class ltxdoc loads the configuration file ltxdoc.cfg
%    if available. Here you can specify further options, e.g.
%    use A4 as paper format:
%       \PassOptionsToClass{a4paper}{article}
%
%    Programm calls to get the documentation (example):
%       pdflatex pdfcolparallel.dtx
%       makeindex -s gind.ist pdfcolparallel.idx
%       pdflatex pdfcolparallel.dtx
%       makeindex -s gind.ist pdfcolparallel.idx
%       pdflatex pdfcolparallel.dtx
%
% Installation:
%    TDS:tex/latex/oberdiek/pdfcolparallel.sty
%    TDS:doc/latex/oberdiek/pdfcolparallel.pdf
%    TDS:doc/latex/oberdiek/test/pdfcolparallel-test1.tex
%    TDS:source/latex/oberdiek/pdfcolparallel.dtx
%
%<*ignore>
\begingroup
  \catcode123=1 %
  \catcode125=2 %
  \def\x{LaTeX2e}%
\expandafter\endgroup
\ifcase 0\ifx\install y1\fi\expandafter
         \ifx\csname processbatchFile\endcsname\relax\else1\fi
         \ifx\fmtname\x\else 1\fi\relax
\else\csname fi\endcsname
%</ignore>
%<*install>
\input docstrip.tex
\Msg{************************************************************************}
\Msg{* Installation}
\Msg{* Package: pdfcolparallel 2016/05/16 v1.4 Color stacks support for parallel (HO)}
\Msg{************************************************************************}

\keepsilent
\askforoverwritefalse

\let\MetaPrefix\relax
\preamble

This is a generated file.

Project: pdfcolparallel
Version: 2016/05/16 v1.4

Copyright (C) 2007, 2008, 2010 by
   Heiko Oberdiek <heiko.oberdiek at googlemail.com>

This work may be distributed and/or modified under the
conditions of the LaTeX Project Public License, either
version 1.3c of this license or (at your option) any later
version. This version of this license is in
   http://www.latex-project.org/lppl/lppl-1-3c.txt
and the latest version of this license is in
   http://www.latex-project.org/lppl.txt
and version 1.3 or later is part of all distributions of
LaTeX version 2005/12/01 or later.

This work has the LPPL maintenance status "maintained".

This Current Maintainer of this work is Heiko Oberdiek.

This work consists of the main source file pdfcolparallel.dtx
and the derived files
   pdfcolparallel.sty, pdfcolparallel.pdf, pdfcolparallel.ins,
   pdfcolparallel.drv, pdfcolparallel-test1.tex.

\endpreamble
\let\MetaPrefix\DoubleperCent

\generate{%
  \file{pdfcolparallel.ins}{\from{pdfcolparallel.dtx}{install}}%
  \file{pdfcolparallel.drv}{\from{pdfcolparallel.dtx}{driver}}%
  \usedir{tex/latex/oberdiek}%
  \file{pdfcolparallel.sty}{\from{pdfcolparallel.dtx}{package}}%
  \usedir{doc/latex/oberdiek/test}%
  \file{pdfcolparallel-test1.tex}{\from{pdfcolparallel.dtx}{test1}}%
  \nopreamble
  \nopostamble
  \usedir{source/latex/oberdiek/catalogue}%
  \file{pdfcolparallel.xml}{\from{pdfcolparallel.dtx}{catalogue}}%
}

\catcode32=13\relax% active space
\let =\space%
\Msg{************************************************************************}
\Msg{*}
\Msg{* To finish the installation you have to move the following}
\Msg{* file into a directory searched by TeX:}
\Msg{*}
\Msg{*     pdfcolparallel.sty}
\Msg{*}
\Msg{* To produce the documentation run the file `pdfcolparallel.drv'}
\Msg{* through LaTeX.}
\Msg{*}
\Msg{* Happy TeXing!}
\Msg{*}
\Msg{************************************************************************}

\endbatchfile
%</install>
%<*ignore>
\fi
%</ignore>
%<*driver>
\NeedsTeXFormat{LaTeX2e}
\ProvidesFile{pdfcolparallel.drv}%
  [2016/05/16 v1.4 Color stacks support for parallel (HO)]%
\documentclass{ltxdoc}
\usepackage{holtxdoc}[2011/11/22]
\begin{document}
  \DocInput{pdfcolparallel.dtx}%
\end{document}
%</driver>
% \fi
%
%
% \CharacterTable
%  {Upper-case    \A\B\C\D\E\F\G\H\I\J\K\L\M\N\O\P\Q\R\S\T\U\V\W\X\Y\Z
%   Lower-case    \a\b\c\d\e\f\g\h\i\j\k\l\m\n\o\p\q\r\s\t\u\v\w\x\y\z
%   Digits        \0\1\2\3\4\5\6\7\8\9
%   Exclamation   \!     Double quote  \"     Hash (number) \#
%   Dollar        \$     Percent       \%     Ampersand     \&
%   Acute accent  \'     Left paren    \(     Right paren   \)
%   Asterisk      \*     Plus          \+     Comma         \,
%   Minus         \-     Point         \.     Solidus       \/
%   Colon         \:     Semicolon     \;     Less than     \<
%   Equals        \=     Greater than  \>     Question mark \?
%   Commercial at \@     Left bracket  \[     Backslash     \\
%   Right bracket \]     Circumflex    \^     Underscore    \_
%   Grave accent  \`     Left brace    \{     Vertical bar  \|
%   Right brace   \}     Tilde         \~}
%
% \GetFileInfo{pdfcolparallel.drv}
%
% \title{The \xpackage{pdfcolparallel} package}
% \date{2016/05/16 v1.4}
% \author{Heiko Oberdiek\thanks
% {Please report any issues at https://github.com/ho-tex/oberdiek/issues}\\
% \xemail{heiko.oberdiek at googlemail.com}}
%
% \maketitle
%
% \begin{abstract}
% This packages fixes bugs in \xpackage{parallel} and
% improves color support by using several color stacks
% that are provided by \pdfTeX\ since version 1.40.
% \end{abstract}
%
% \tableofcontents
%
% \section{Usage}
%
% \begin{quote}
% |\usepackage{pdfcolparallel}|
% \end{quote}
% The package \xpackage{pdfcolparallel} loads package \xpackage{parallel}
% \cite{parallel} and redefines some macros to fix bugs.
%
% If color stacks are available then package \xpackage{parallel}
% is further patched to support them.
%
% \subsection{Option \xoption{rulebetweencolor}}
%
% Package \xpackage{pdfcolparallel} also fixes the color for the
% rule between columns.
% Default color is \cs{normalcolor}. But this can be changed by using
% option \xoption{rulebetweencolor} for |\setkeys{parallel}|
% (see package \xpackage{keyval}). The option takes a color specification
% as value. If the value is empty, then the default (\cs{normalcolor})
% is used.
% Examples:
% \begin{quote}
%   |\setkeys{parallel}{rulebetweencolor=blue}|,\\
%   |\setkeys{parallel}{rulebetweencolor={red}}|,\\
%   |\setkeys{parallel}{rulebetweencolor={}}|,
%     \textit{\% \cs{normalcolor} is used}\\
%   |\setkeys{parallel}{rulebetweencolor=[rgb]{1,0,.5}}|
% \end{quote}
%
% \subsection{Future}
%
% If there will be a new version of package \xpackage{parallel}
% that adds support for color stacks, then this package may become
% obsolete.
%
% \StopEventually{
% }
%
% \section{Implementation}
%
% \subsection{Identification}
%
%    \begin{macrocode}
%<*package>
\NeedsTeXFormat{LaTeX2e}
\ProvidesPackage{pdfcolparallel}%
  [2016/05/16 v1.4 Color stacks support for parallel (HO)]%
%    \end{macrocode}
%
% \subsection{Load and fix package \xpackage{parallel}}
%
%    Package \xpackage{parallel} is loaded. Before options of package
%    \xpackage{pdfcolparallel} are passed to package \xpackage{parallel}.
%    \begin{macrocode}
\DeclareOption*{%
  \PassoptionsToPackage{\CurrentOption}{parallel}%
}
\ProcessOptions\relax
\RequirePackage{parallel}[2003/04/13]
%    \end{macrocode}
%
%    \begin{macrocode}
\RequirePackage{infwarerr}[2007/09/09]
%    \end{macrocode}
%
%    \begin{macro}{\pcp@ColorPatch}
%    \begin{macrocode}
\begingroup\expandafter\expandafter\expandafter\endgroup
\expandafter\ifx\csname currentgrouplevel\endcsname\relax
  \def\pcp@ColorPatch{}%
\else
  \def\pcp@ColorPatch{%
    \@ifundefined{set@color}{%
      \gdef\pcp@ColorPatch{}%
    }{%
      \gdef\pcp@ColorPatch{%
        \gdef\pcp@ColorResets{}%
        \bgroup
        \aftergroup\pcp@ColorResets
        \aftergroup\egroup
        \let\pcp@OrgSetColor\set@color
        \let\set@color\pcp@SetColor
        \edef\pcp@GroupLevel{\the\currentgrouplevel}%
      }%
    }%
    \pcp@ColorPatch
  }%
%    \end{macrocode}
%    \end{macro}
%    \begin{macro}{\pcp@SetColor}
%    \begin{macrocode}
  \def\pcp@SetColor{%
    \ifnum\pcp@GroupLevel=\currentgrouplevel
      \let\pcp@OrgAfterGroup\aftergroup
      \def\aftergroup{%
        \g@addto@macro\pcp@ColorResets
      }%
      \pcp@OrgSetColor
      \let\aftergroup\pcp@OrgAfterGroup
    \else
      \pcp@OrgSetColor
    \fi
  }%
\fi
%    \end{macrocode}
%    \end{macro}
%
%    \begin{macro}{\pcp@CmdCheckRedef}
%    \begin{macrocode}
\def\pcp@CmdCheckRedef#1{%
  \begingroup
    \def\pcp@cmd{#1}%
    \afterassignment\pcp@CmdDo
    \long\def\reserved@a
}
\def\pcp@CmdDo{%
    \expandafter\ifx\pcp@cmd\reserved@a
    \else
      \edef\x*{\expandafter\string\pcp@cmd}%
      \@PackageWarningNoLine{pdfcolparallel}{%
        Command \x* has changed.\MessageBreak
        Supported versions of package `parallel':\MessageBreak
        \space\space 2003/04/13\MessageBreak
        The redefinition of \x* may\MessageBreak
        not behave correctly depending on the changes%
      }%
    \fi
  \expandafter\endgroup
  \expandafter\def\pcp@cmd
}
%    \end{macrocode}
%    \end{macro}
%
%    \begin{macrocode}
\def\pcp@SwitchStack#1#2{}
%    \end{macrocode}
%    \begin{macrocode}
\def\pcp@SetCurrent#1{}
%    \end{macrocode}
%
%    \begin{macro}{\ParallelLText}
%    \begin{macrocode}
\pcp@CmdCheckRedef\ParallelLText{%
  \everypar{}%
  \@restorepar
  \begingroup
    \hbadness=3000 %
    \let\footnote=\ParallelLFootnote
    \ParallelWhichBox=0 %
    \global\setbox\ParallelLBox=\vbox\bgroup
      \hsize=\ParallelLWidth
      \aftergroup\ParallelAfterText
      \begingroup
        \afterassignment\ParallelCheckOpenBrace
        \let\x=%
}{%
  \everypar{}%
  \@restorepar
  \@nobreakfalse
  \begingroup
    \hbadness=3000 %
    \let\footnote=\ParallelLFootnote
    \ParallelWhichBox=0 %
    \global\setbox\ParallelLBox=\vbox\bgroup
      \hsize=\ParallelLWidth
      \linewidth=\ParallelLWidth
      \pcp@SwitchStack{Left}\ParallelLBox
      \aftergroup\ParallelAfterText
      \pcp@ColorPatch
      \begingroup
        \afterassignment\ParallelCheckOpenBrace
        \let\x=%
}
%    \end{macrocode}
%    \end{macro}
%
%    \begin{macro}{\ParallelRText}
%    \begin{macrocode}
\pcp@CmdCheckRedef\ParallelRText{%
  \everypar{}%
  \@restorepar
  \begingroup
    \hbadness=3000 %
    \ifnum\ParallelFNMode=\@ne
      \let\footnote=\ParallelRFootnote
    \else
      \let\footnote=\ParallelLFootnote
    \fi
    \ParallelWhichBox=\@ne
    \global\setbox\ParallelRBox=\vbox\bgroup
      \hsize=\ParallelRWidth
      \aftergroup\ParallelAfterText
      \begingroup
        \afterassignment\ParallelCheckOpenBrace
        \let\x=%
}{%
  \everypar{}%
  \@restorepar
  \@nobreakfalse
  \begingroup
    \hbadness=3000 %
    \ifnum\ParallelFNMode=\@ne
      \let\footnote=\ParallelRFootnote
    \else
      \let\footnote=\ParallelLFootnote
    \fi
    \ParallelWhichBox=\@ne
    \global\setbox\ParallelRBox=\vbox\bgroup
      \hsize=\ParallelRWidth
      \linewidth=\ParallelRWidth
      \pcp@SwitchStack{Right}\ParallelRBox
      \aftergroup\ParallelAfterText
      \pcp@ColorPatch
      \begingroup
        \afterassignment\ParallelCheckOpenBrace
        \let\x=%
}
%    \end{macrocode}
%    \end{macro}
%
%    \begin{macro}{\ParallelParTwoPages}
%    \begin{macrocode}
\pcp@CmdCheckRedef\ParallelParTwoPages{%
  \ifnum\ParallelBoolVar=\@ne
    \par
    \begingroup
      \global\ParallelWhichBox=\@ne
      \newpage
      \vbadness=10000 %
      \vfuzz=3ex %
      \splittopskip=\z@skip
      \loop%
        \ifnum\ParallelBoolVar=\@ne%
          \ifnum\ParallelWhichBox=\@ne
            \ifvoid\ParallelLBox
              \mbox{} %
              \newpage
            \else
              \global\ParallelWhichBox=\z@
            \fi
          \else
            \ifvoid\ParallelRBox
              \mbox{} %
              \newpage
            \else
              \global\ParallelWhichBox=\@ne
            \fi
          \fi
          \ifnum\ParallelWhichBox=\z@
            \ifodd\thepage
              \mbox{} %
              \newpage
            \fi
            \hbox to\textwidth{%
              \vbox{\vsplit\ParallelLBox to.98\textheight}%
            }%
          \else
            \ifodd\thepage\relax
            \else
              \mbox{} %
              \newpage
            \fi
            \hbox to\textwidth{%
              \vbox{\vsplit\ParallelRBox to.98\textheight}%
            }%
          \fi
          \vspace*{\fill}%
          \newpage
        \fi
        \ifvoid\ParallelLBox
          \ifvoid\ParallelRBox
            \global\ParallelBoolVar=\z@
          \fi
        \fi
      \ifnum\ParallelBoolVar=\@ne
      \repeat
      \par
    \endgroup
  \fi
}{%
%    \end{macrocode}
%    Additional fixes:
%    \begin{itemize}
%    \item Unnecessary white space removed.
%    \item |\ifodd\thepage| changed to |\ifodd\value{page}|.
%    \end{itemize}
%    \begin{macrocode}
  \ifnum\ParallelBoolVar=\@ne
    \par
    \begingroup
      \global\ParallelWhichBox=\@ne
      \newpage
      \vbadness=10000 %
      \vfuzz=3ex %
      \splittopskip=\z@skip
      \loop%
        \ifnum\ParallelBoolVar=\@ne%
          \ifnum\ParallelWhichBox=\@ne
            \ifvoid\ParallelLBox
              \mbox{}%
              \newpage
            \else
              \global\ParallelWhichBox=\z@
            \fi
          \else
            \ifvoid\ParallelRBox
              \null
              \newpage
            \else
              \global\ParallelWhichBox=\@ne
            \fi
          \fi
          \ifnum\ParallelWhichBox=\z@
            \ifodd\value{page}%
              \null
              \newpage
            \fi
            \hbox to\textwidth{%
              \pcp@SetCurrent{Left}%
              \setbox\z@=\vsplit\ParallelLBox to.98\textheight
              \vbox to.98\textheight{%
                \@texttop
                \unvbox\z@
                \@textbottom
              }%
            }%
          \else
            \ifodd\value{page}%
            \else
              \mbox{}%
              \newpage
            \fi
            \hbox to\textwidth{%
              \pcp@SetCurrent{Right}%
              \setbox\z@=\vsplit\ParallelRBox to.98\textheight
              \vbox to.98\textheight{%
                \@texttop
                \unvbox\z@
                \@textbottom
              }%
            }%
          \fi
          \vspace*{\fill}%
          \newpage
        \fi
        \ifvoid\ParallelLBox
          \ifvoid\ParallelRBox
            \global\ParallelBoolVar=\z@
          \fi
        \fi
      \ifnum\ParallelBoolVar=\@ne
      \repeat
      \par
    \endgroup
    \pcp@SetCurrent{}%
  \fi
}
%    \end{macrocode}
%    \end{macro}
%
% \subsection{Color stack support}
%
%    \begin{macrocode}
\RequirePackage{pdfcol}[2007/12/12]
\ifpdfcolAvailable
\else
  \PackageInfo{pdfcolparallel}{%
    Loading aborted, because color stacks are not available%
  }%
  \expandafter\endinput
\fi
%    \end{macrocode}
%
%    \begin{macrocode}
\pdfcolInitStack{pcp@Left}
\pdfcolInitStack{pcp@Right}
%    \end{macrocode}
%    \begin{macro}{\pcp@Box}
%    \begin{macrocode}
\newbox\pcp@Box
%    \end{macrocode}
%    \end{macro}
%    \begin{macro}{\pcp@SwitchStack}
%    \begin{macrocode}
\def\pcp@SwitchStack#1#2{%
  \pdfcolSwitchStack{pcp@#1}%
  \global\setbox\pcp@Box=\vbox to 0pt{%
    \pdfcolSetCurrentColor
  }%
  \aftergroup\pcp@FixBox
  \aftergroup#2%
}
%    \end{macrocode}
%    \end{macro}
%    \begin{macro}{\pcp@FixBox}
%    \begin{macrocode}
\def\pcp@FixBox#1{%
  \global\setbox#1=\vbox{%
    \unvbox\pcp@Box
    \unvbox#1%
  }%
}
%    \end{macrocode}
%    \end{macro}
%    \begin{macro}{\pcp@SetCurrent}
%    \begin{macrocode}
\def\pcp@SetCurrent#1{%
  \ifx\\#1\\%
    \pdfcolSetCurrent{}%
  \else
    \pdfcolSetCurrent{pcp@#1}%
  \fi
}
%    \end{macrocode}
%    \end{macro}
%
% \subsection{Redefinitions}
%
%    \begin{macro}{\ParallelParOnePage}
%    \begin{macrocode}
\pcp@CmdCheckRedef\ParallelParOnePage{%
  \ifnum\ParallelBoolVar=\@ne
    \par
    \begingroup
      \leftmargin=\z@
      \rightmargin=\z@
      \parskip=\z@skip
      \parindent=\z@
      \vbadness=10000 %
      \vfuzz=3ex %
      \splittopskip=\z@skip
      \loop
        \ifnum\ParallelBoolVar=\@ne
          \noindent
          \hbox to\textwidth{%
            \hskip\ParallelLeftMargin
            \hbox to\ParallelTextWidth{%
              \ifvoid\ParallelLBox
                \hskip\ParallelLWidth
              \else
                \ParallelWhichBox=\z@
                \vbox{%
                  \setbox\ParallelBoxVar
                      =\vsplit\ParallelLBox to\dp\strutbox
                  \unvbox\ParallelBoxVar
                }%
              \fi
              \strut
              \ifnum\ParallelBoolMid=\@ne
                \hskip\ParallelMainMidSkip
                \vrule
              \else
                \hss
              \fi
              \hss
              \ifvoid\ParallelRBox
                \hskip\ParallelRWidth
              \else
                \ParallelWhichBox=\@ne
                \vbox{%
                  \setbox\ParallelBoxVar
                      =\vsplit\ParallelRBox to\dp\strutbox
                  \unvbox\ParallelBoxVar
                }%
              \fi
            }%
          }%
          \ifvoid\ParallelLBox
            \ifvoid\ParallelRBox
              \global\ParallelBoolVar=\z@
            \fi
          \fi%
        \fi%
      \ifnum\ParallelBoolVar=\@ne
        \penalty\interlinepenalty
      \repeat
      \par
    \endgroup
  \fi
}{%
  \ifnum\ParallelBoolVar=\@ne
    \par
    \begingroup
      \leftmargin=\z@
      \rightmargin=\z@
      \parskip=\z@skip
      \parindent=\z@
      \vbadness=10000 %
      \vfuzz=3ex %
      \splittopskip=\z@skip
      \loop
        \ifnum\ParallelBoolVar=\@ne
          \noindent
          \hbox to\textwidth{%
            \hskip\ParallelLeftMargin
            \hbox to\ParallelTextWidth{%
              \ifvoid\ParallelLBox
                \hskip\ParallelLWidth
              \else
                \pcp@SetCurrent{Left}%
                \ParallelWhichBox=\z@
                \vbox{%
                  \setbox\ParallelBoxVar
                      =\vsplit\ParallelLBox to\dp\strutbox
                  \unvbox\ParallelBoxVar
                }%
              \fi
              \strut
              \ifnum\ParallelBoolMid=\@ne
                \hskip\ParallelMainMidSkip
                \begingroup
                  \pcp@RuleBetweenColor
                  \vrule
                \endgroup
              \else
                \hss
              \fi
              \hss
              \ifvoid\ParallelRBox
                \hskip\ParallelRWidth
              \else
                \pcp@SetCurrent{Right}%
                \ParallelWhichBox=\@ne
                \vbox{%
                  \setbox\ParallelBoxVar
                      =\vsplit\ParallelRBox to\dp\strutbox
                  \unvbox\ParallelBoxVar
                }%
              \fi
            }%
          }%
          \ifvoid\ParallelLBox
            \ifvoid\ParallelRBox
              \global\ParallelBoolVar=\z@
            \fi
          \fi%
        \fi%
      \ifnum\ParallelBoolVar=\@ne
        \penalty\interlinepenalty
      \repeat
      \par
    \endgroup
    \pcp@SetCurrent{}%
  \fi
}
%    \end{macrocode}
%    \end{macro}
%    \begin{macro}{\pcp@RuleBetweenColorDefault}
%    \begin{macrocode}
\def\pcp@RuleBetweenColorDefault{%
  \normalcolor
}
%    \end{macrocode}
%    \end{macro}
%    \begin{macro}{\pcp@RuleBetweenColor}
%    \begin{macrocode}
\let\pcp@RuleBetweenColor\pcp@RuleBetweenColorDefault
%    \end{macrocode}
%    \end{macro}
%    \begin{macrocode}
\RequirePackage{keyval}
\define@key{parallel}{rulebetweencolor}{%
  \edef\pcp@temp{#1}%
  \ifx\pcp@temp\@empty
    \let\pcp@RuleBetweenColor\pcp@RuleBetweenColorDefault
  \else
    \edef\pcp@temp{%
      \noexpand\@ifnextchar[{%
        \def\noexpand\pcp@RuleBetweenColor{%
          \noexpand\color\pcp@temp
        }%
        \noexpand\pcp@GobbleNil
      }{%
        \def\noexpand\pcp@RuleBetweenColor{%
          \noexpand\color{\pcp@temp}%
        }%
        \noexpand\pcp@GobbleNil
      }%
      \pcp@temp\noexpand\@nil
    }%
    \pcp@temp
  \fi
}
%    \end{macrocode}
%    \begin{macro}{\pcp@GobbleNil}
%    \begin{macrocode}
\long\def\pcp@GobbleNil#1\@nil{}
%    \end{macrocode}
%    \end{macro}
%
%    \begin{macrocode}
%</package>
%    \end{macrocode}
%
% \section{Test}
%
%    The test file is a modified version of the file that
%    Alexander Hirsch has posted in \xnewsgroup{de.comp.text.tex}:
%    \URL{``\link{\texttt{parallel.sty} und farbiger Text}''}^^A
%    {http://groups.google.com/group/de.comp.text.tex/msg/6a759cf33bb071a5}
%    \begin{macrocode}
%<*test1>
\AtEndDocument{%
  \typeout{}%
  \typeout{**************************************}%
  \typeout{*** \space Check the PDF file manually! \space ***}%
  \typeout{**************************************}%
  \typeout{}%
}
\documentclass{article}
\usepackage{xcolor}
\usepackage{pdfcolparallel}[2016/05/16]

\begin{document}
  \color{green}%
  Green%
  \begin{Parallel}{0.47\textwidth}{0.47\textwidth}%
    \ParallelLText{%
      \textcolor{red}{%
        Ein Absatz, der sich ueber zwei Zeilen erstrecken soll. %
        Ein Absatz, der sich ueber zwei Zeilen erstrecken soll.%
      }%
    }%
    \ParallelRText{%
      \textcolor{blue}{%
        Ein Absatz, der sich ueber zwei Zeilen erstrecken soll. %
        Ein Absatz, der sich ueber zwei Zeilen erstrecken soll.%
      }%
    }%
    \ParallelPar
    \ParallelLText{%
      Default %
      \color{red}%
      Ein Absatz, der sich ueber zwei Zeilen erstrecken soll. %
      Ein Absatz, der sich ueber zwei Zeilen erstrecken soll.%
    }%
    \ParallelRText{%
      Default %
      \color{blue}%
      Ein Absatz, der sich ueber zwei Zeilen erstrecken soll. %
      Ein Absatz, der sich ueber zwei Zeilen erstrecken soll.%
    }%
    \ParallelPar
    \ParallelLText{%
      \begin{enumerate}%
      \item left text, left text, left text, left text, %
            left text, left text, left text, left text,%
      \item left text, left text, left text, left text, %
            left text, left text, left text, left text.%
      \end{enumerate}%
    }%
    \ParallelRText{%
      \begin{enumerate}%
      \item right text, right text, right text, right text, %
            right text, right text, right text, right text.%
      \item right text, right text, right text, right text, %
            right text, right text, right text, right text.%
      \end{enumerate}%
    }%
  \end{Parallel}%
  \begin{Parallel}[p]{\textwidth}{\textwidth}%
    \ParallelLText{%
      \textcolor{red}{%
        Ein Absatz, der sich ueber zwei Zeilen erstrecken soll. %
        Ein Absatz, der sich ueber zwei Zeilen erstrecken soll. %
        Foo bar bla bla bla.%
      }%
      \par
      Und noch ein Absatz.%
    }%
    \ParallelRText{%
      \textcolor{blue}{%
        Ein Absatz, der sich ueber zwei Zeilen erstrecken soll. %
        Ein Absatz, der sich ueber zwei Zeilen erstrecken soll. %
        Foo bar bla bla bla.%
      }%
    }%
  \end{Parallel}%
  \begin{Parallel}[p]{\textwidth}{\textwidth}%
    \ParallelLText{%
      \rule{1pt}{.98\textheight}\Huge g%
    }%
    \ParallelRText{%
      \rule{1pt}{.98\textheight}y%
    }%
  \end{Parallel}%
  Green%
\end{document}
%</test1>
%    \end{macrocode}
%
% \section{Installation}
%
% \subsection{Download}
%
% \paragraph{Package.} This package is available on
% CTAN\footnote{\url{http://ctan.org/pkg/pdfcolparallel}}:
% \begin{description}
% \item[\CTAN{macros/latex/contrib/oberdiek/pdfcolparallel.dtx}] The source file.
% \item[\CTAN{macros/latex/contrib/oberdiek/pdfcolparallel.pdf}] Documentation.
% \end{description}
%
%
% \paragraph{Bundle.} All the packages of the bundle `oberdiek'
% are also available in a TDS compliant ZIP archive. There
% the packages are already unpacked and the documentation files
% are generated. The files and directories obey the TDS standard.
% \begin{description}
% \item[\CTAN{install/macros/latex/contrib/oberdiek.tds.zip}]
% \end{description}
% \emph{TDS} refers to the standard ``A Directory Structure
% for \TeX\ Files'' (\CTAN{tds/tds.pdf}). Directories
% with \xfile{texmf} in their name are usually organized this way.
%
% \subsection{Bundle installation}
%
% \paragraph{Unpacking.} Unpack the \xfile{oberdiek.tds.zip} in the
% TDS tree (also known as \xfile{texmf} tree) of your choice.
% Example (linux):
% \begin{quote}
%   |unzip oberdiek.tds.zip -d ~/texmf|
% \end{quote}
%
% \paragraph{Script installation.}
% Check the directory \xfile{TDS:scripts/oberdiek/} for
% scripts that need further installation steps.
% Package \xpackage{attachfile2} comes with the Perl script
% \xfile{pdfatfi.pl} that should be installed in such a way
% that it can be called as \texttt{pdfatfi}.
% Example (linux):
% \begin{quote}
%   |chmod +x scripts/oberdiek/pdfatfi.pl|\\
%   |cp scripts/oberdiek/pdfatfi.pl /usr/local/bin/|
% \end{quote}
%
% \subsection{Package installation}
%
% \paragraph{Unpacking.} The \xfile{.dtx} file is a self-extracting
% \docstrip\ archive. The files are extracted by running the
% \xfile{.dtx} through \plainTeX:
% \begin{quote}
%   \verb|tex pdfcolparallel.dtx|
% \end{quote}
%
% \paragraph{TDS.} Now the different files must be moved into
% the different directories in your installation TDS tree
% (also known as \xfile{texmf} tree):
% \begin{quote}
% \def\t{^^A
% \begin{tabular}{@{}>{\ttfamily}l@{ $\rightarrow$ }>{\ttfamily}l@{}}
%   pdfcolparallel.sty & tex/latex/oberdiek/pdfcolparallel.sty\\
%   pdfcolparallel.pdf & doc/latex/oberdiek/pdfcolparallel.pdf\\
%   test/pdfcolparallel-test1.tex & doc/latex/oberdiek/test/pdfcolparallel-test1.tex\\
%   pdfcolparallel.dtx & source/latex/oberdiek/pdfcolparallel.dtx\\
% \end{tabular}^^A
% }^^A
% \sbox0{\t}^^A
% \ifdim\wd0>\linewidth
%   \begingroup
%     \advance\linewidth by\leftmargin
%     \advance\linewidth by\rightmargin
%   \edef\x{\endgroup
%     \def\noexpand\lw{\the\linewidth}^^A
%   }\x
%   \def\lwbox{^^A
%     \leavevmode
%     \hbox to \linewidth{^^A
%       \kern-\leftmargin\relax
%       \hss
%       \usebox0
%       \hss
%       \kern-\rightmargin\relax
%     }^^A
%   }^^A
%   \ifdim\wd0>\lw
%     \sbox0{\small\t}^^A
%     \ifdim\wd0>\linewidth
%       \ifdim\wd0>\lw
%         \sbox0{\footnotesize\t}^^A
%         \ifdim\wd0>\linewidth
%           \ifdim\wd0>\lw
%             \sbox0{\scriptsize\t}^^A
%             \ifdim\wd0>\linewidth
%               \ifdim\wd0>\lw
%                 \sbox0{\tiny\t}^^A
%                 \ifdim\wd0>\linewidth
%                   \lwbox
%                 \else
%                   \usebox0
%                 \fi
%               \else
%                 \lwbox
%               \fi
%             \else
%               \usebox0
%             \fi
%           \else
%             \lwbox
%           \fi
%         \else
%           \usebox0
%         \fi
%       \else
%         \lwbox
%       \fi
%     \else
%       \usebox0
%     \fi
%   \else
%     \lwbox
%   \fi
% \else
%   \usebox0
% \fi
% \end{quote}
% If you have a \xfile{docstrip.cfg} that configures and enables \docstrip's
% TDS installing feature, then some files can already be in the right
% place, see the documentation of \docstrip.
%
% \subsection{Refresh file name databases}
%
% If your \TeX~distribution
% (\teTeX, \mikTeX, \dots) relies on file name databases, you must refresh
% these. For example, \teTeX\ users run \verb|texhash| or
% \verb|mktexlsr|.
%
% \subsection{Some details for the interested}
%
% \paragraph{Attached source.}
%
% The PDF documentation on CTAN also includes the
% \xfile{.dtx} source file. It can be extracted by
% AcrobatReader 6 or higher. Another option is \textsf{pdftk},
% e.g. unpack the file into the current directory:
% \begin{quote}
%   \verb|pdftk pdfcolparallel.pdf unpack_files output .|
% \end{quote}
%
% \paragraph{Unpacking with \LaTeX.}
% The \xfile{.dtx} chooses its action depending on the format:
% \begin{description}
% \item[\plainTeX:] Run \docstrip\ and extract the files.
% \item[\LaTeX:] Generate the documentation.
% \end{description}
% If you insist on using \LaTeX\ for \docstrip\ (really,
% \docstrip\ does not need \LaTeX), then inform the autodetect routine
% about your intention:
% \begin{quote}
%   \verb|latex \let\install=y% \iffalse meta-comment
%
% File: pdfcolparallel.dtx
% Version: 2016/05/16 v1.4
% Info: Color stacks support for parallel
%
% Copyright (C) 2007, 2008, 2010 by
%    Heiko Oberdiek <heiko.oberdiek at googlemail.com>
%    2016
%    https://github.com/ho-tex/oberdiek/issues
%
% This work may be distributed and/or modified under the
% conditions of the LaTeX Project Public License, either
% version 1.3c of this license or (at your option) any later
% version. This version of this license is in
%    http://www.latex-project.org/lppl/lppl-1-3c.txt
% and the latest version of this license is in
%    http://www.latex-project.org/lppl.txt
% and version 1.3 or later is part of all distributions of
% LaTeX version 2005/12/01 or later.
%
% This work has the LPPL maintenance status "maintained".
%
% This Current Maintainer of this work is Heiko Oberdiek.
%
% This work consists of the main source file pdfcolparallel.dtx
% and the derived files
%    pdfcolparallel.sty, pdfcolparallel.pdf, pdfcolparallel.ins,
%    pdfcolparallel.drv, pdfcolparallel-test1.tex.
%
% Distribution:
%    CTAN:macros/latex/contrib/oberdiek/pdfcolparallel.dtx
%    CTAN:macros/latex/contrib/oberdiek/pdfcolparallel.pdf
%
% Unpacking:
%    (a) If pdfcolparallel.ins is present:
%           tex pdfcolparallel.ins
%    (b) Without pdfcolparallel.ins:
%           tex pdfcolparallel.dtx
%    (c) If you insist on using LaTeX
%           latex \let\install=y% \iffalse meta-comment
%
% File: pdfcolparallel.dtx
% Version: 2016/05/16 v1.4
% Info: Color stacks support for parallel
%
% Copyright (C) 2007, 2008, 2010 by
%    Heiko Oberdiek <heiko.oberdiek at googlemail.com>
%    2016
%    https://github.com/ho-tex/oberdiek/issues
%
% This work may be distributed and/or modified under the
% conditions of the LaTeX Project Public License, either
% version 1.3c of this license or (at your option) any later
% version. This version of this license is in
%    http://www.latex-project.org/lppl/lppl-1-3c.txt
% and the latest version of this license is in
%    http://www.latex-project.org/lppl.txt
% and version 1.3 or later is part of all distributions of
% LaTeX version 2005/12/01 or later.
%
% This work has the LPPL maintenance status "maintained".
%
% This Current Maintainer of this work is Heiko Oberdiek.
%
% This work consists of the main source file pdfcolparallel.dtx
% and the derived files
%    pdfcolparallel.sty, pdfcolparallel.pdf, pdfcolparallel.ins,
%    pdfcolparallel.drv, pdfcolparallel-test1.tex.
%
% Distribution:
%    CTAN:macros/latex/contrib/oberdiek/pdfcolparallel.dtx
%    CTAN:macros/latex/contrib/oberdiek/pdfcolparallel.pdf
%
% Unpacking:
%    (a) If pdfcolparallel.ins is present:
%           tex pdfcolparallel.ins
%    (b) Without pdfcolparallel.ins:
%           tex pdfcolparallel.dtx
%    (c) If you insist on using LaTeX
%           latex \let\install=y\input{pdfcolparallel.dtx}
%        (quote the arguments according to the demands of your shell)
%
% Documentation:
%    (a) If pdfcolparallel.drv is present:
%           latex pdfcolparallel.drv
%    (b) Without pdfcolparallel.drv:
%           latex pdfcolparallel.dtx; ...
%    The class ltxdoc loads the configuration file ltxdoc.cfg
%    if available. Here you can specify further options, e.g.
%    use A4 as paper format:
%       \PassOptionsToClass{a4paper}{article}
%
%    Programm calls to get the documentation (example):
%       pdflatex pdfcolparallel.dtx
%       makeindex -s gind.ist pdfcolparallel.idx
%       pdflatex pdfcolparallel.dtx
%       makeindex -s gind.ist pdfcolparallel.idx
%       pdflatex pdfcolparallel.dtx
%
% Installation:
%    TDS:tex/latex/oberdiek/pdfcolparallel.sty
%    TDS:doc/latex/oberdiek/pdfcolparallel.pdf
%    TDS:doc/latex/oberdiek/test/pdfcolparallel-test1.tex
%    TDS:source/latex/oberdiek/pdfcolparallel.dtx
%
%<*ignore>
\begingroup
  \catcode123=1 %
  \catcode125=2 %
  \def\x{LaTeX2e}%
\expandafter\endgroup
\ifcase 0\ifx\install y1\fi\expandafter
         \ifx\csname processbatchFile\endcsname\relax\else1\fi
         \ifx\fmtname\x\else 1\fi\relax
\else\csname fi\endcsname
%</ignore>
%<*install>
\input docstrip.tex
\Msg{************************************************************************}
\Msg{* Installation}
\Msg{* Package: pdfcolparallel 2016/05/16 v1.4 Color stacks support for parallel (HO)}
\Msg{************************************************************************}

\keepsilent
\askforoverwritefalse

\let\MetaPrefix\relax
\preamble

This is a generated file.

Project: pdfcolparallel
Version: 2016/05/16 v1.4

Copyright (C) 2007, 2008, 2010 by
   Heiko Oberdiek <heiko.oberdiek at googlemail.com>

This work may be distributed and/or modified under the
conditions of the LaTeX Project Public License, either
version 1.3c of this license or (at your option) any later
version. This version of this license is in
   http://www.latex-project.org/lppl/lppl-1-3c.txt
and the latest version of this license is in
   http://www.latex-project.org/lppl.txt
and version 1.3 or later is part of all distributions of
LaTeX version 2005/12/01 or later.

This work has the LPPL maintenance status "maintained".

This Current Maintainer of this work is Heiko Oberdiek.

This work consists of the main source file pdfcolparallel.dtx
and the derived files
   pdfcolparallel.sty, pdfcolparallel.pdf, pdfcolparallel.ins,
   pdfcolparallel.drv, pdfcolparallel-test1.tex.

\endpreamble
\let\MetaPrefix\DoubleperCent

\generate{%
  \file{pdfcolparallel.ins}{\from{pdfcolparallel.dtx}{install}}%
  \file{pdfcolparallel.drv}{\from{pdfcolparallel.dtx}{driver}}%
  \usedir{tex/latex/oberdiek}%
  \file{pdfcolparallel.sty}{\from{pdfcolparallel.dtx}{package}}%
  \usedir{doc/latex/oberdiek/test}%
  \file{pdfcolparallel-test1.tex}{\from{pdfcolparallel.dtx}{test1}}%
  \nopreamble
  \nopostamble
  \usedir{source/latex/oberdiek/catalogue}%
  \file{pdfcolparallel.xml}{\from{pdfcolparallel.dtx}{catalogue}}%
}

\catcode32=13\relax% active space
\let =\space%
\Msg{************************************************************************}
\Msg{*}
\Msg{* To finish the installation you have to move the following}
\Msg{* file into a directory searched by TeX:}
\Msg{*}
\Msg{*     pdfcolparallel.sty}
\Msg{*}
\Msg{* To produce the documentation run the file `pdfcolparallel.drv'}
\Msg{* through LaTeX.}
\Msg{*}
\Msg{* Happy TeXing!}
\Msg{*}
\Msg{************************************************************************}

\endbatchfile
%</install>
%<*ignore>
\fi
%</ignore>
%<*driver>
\NeedsTeXFormat{LaTeX2e}
\ProvidesFile{pdfcolparallel.drv}%
  [2016/05/16 v1.4 Color stacks support for parallel (HO)]%
\documentclass{ltxdoc}
\usepackage{holtxdoc}[2011/11/22]
\begin{document}
  \DocInput{pdfcolparallel.dtx}%
\end{document}
%</driver>
% \fi
%
%
% \CharacterTable
%  {Upper-case    \A\B\C\D\E\F\G\H\I\J\K\L\M\N\O\P\Q\R\S\T\U\V\W\X\Y\Z
%   Lower-case    \a\b\c\d\e\f\g\h\i\j\k\l\m\n\o\p\q\r\s\t\u\v\w\x\y\z
%   Digits        \0\1\2\3\4\5\6\7\8\9
%   Exclamation   \!     Double quote  \"     Hash (number) \#
%   Dollar        \$     Percent       \%     Ampersand     \&
%   Acute accent  \'     Left paren    \(     Right paren   \)
%   Asterisk      \*     Plus          \+     Comma         \,
%   Minus         \-     Point         \.     Solidus       \/
%   Colon         \:     Semicolon     \;     Less than     \<
%   Equals        \=     Greater than  \>     Question mark \?
%   Commercial at \@     Left bracket  \[     Backslash     \\
%   Right bracket \]     Circumflex    \^     Underscore    \_
%   Grave accent  \`     Left brace    \{     Vertical bar  \|
%   Right brace   \}     Tilde         \~}
%
% \GetFileInfo{pdfcolparallel.drv}
%
% \title{The \xpackage{pdfcolparallel} package}
% \date{2016/05/16 v1.4}
% \author{Heiko Oberdiek\thanks
% {Please report any issues at https://github.com/ho-tex/oberdiek/issues}\\
% \xemail{heiko.oberdiek at googlemail.com}}
%
% \maketitle
%
% \begin{abstract}
% This packages fixes bugs in \xpackage{parallel} and
% improves color support by using several color stacks
% that are provided by \pdfTeX\ since version 1.40.
% \end{abstract}
%
% \tableofcontents
%
% \section{Usage}
%
% \begin{quote}
% |\usepackage{pdfcolparallel}|
% \end{quote}
% The package \xpackage{pdfcolparallel} loads package \xpackage{parallel}
% \cite{parallel} and redefines some macros to fix bugs.
%
% If color stacks are available then package \xpackage{parallel}
% is further patched to support them.
%
% \subsection{Option \xoption{rulebetweencolor}}
%
% Package \xpackage{pdfcolparallel} also fixes the color for the
% rule between columns.
% Default color is \cs{normalcolor}. But this can be changed by using
% option \xoption{rulebetweencolor} for |\setkeys{parallel}|
% (see package \xpackage{keyval}). The option takes a color specification
% as value. If the value is empty, then the default (\cs{normalcolor})
% is used.
% Examples:
% \begin{quote}
%   |\setkeys{parallel}{rulebetweencolor=blue}|,\\
%   |\setkeys{parallel}{rulebetweencolor={red}}|,\\
%   |\setkeys{parallel}{rulebetweencolor={}}|,
%     \textit{\% \cs{normalcolor} is used}\\
%   |\setkeys{parallel}{rulebetweencolor=[rgb]{1,0,.5}}|
% \end{quote}
%
% \subsection{Future}
%
% If there will be a new version of package \xpackage{parallel}
% that adds support for color stacks, then this package may become
% obsolete.
%
% \StopEventually{
% }
%
% \section{Implementation}
%
% \subsection{Identification}
%
%    \begin{macrocode}
%<*package>
\NeedsTeXFormat{LaTeX2e}
\ProvidesPackage{pdfcolparallel}%
  [2016/05/16 v1.4 Color stacks support for parallel (HO)]%
%    \end{macrocode}
%
% \subsection{Load and fix package \xpackage{parallel}}
%
%    Package \xpackage{parallel} is loaded. Before options of package
%    \xpackage{pdfcolparallel} are passed to package \xpackage{parallel}.
%    \begin{macrocode}
\DeclareOption*{%
  \PassoptionsToPackage{\CurrentOption}{parallel}%
}
\ProcessOptions\relax
\RequirePackage{parallel}[2003/04/13]
%    \end{macrocode}
%
%    \begin{macrocode}
\RequirePackage{infwarerr}[2007/09/09]
%    \end{macrocode}
%
%    \begin{macro}{\pcp@ColorPatch}
%    \begin{macrocode}
\begingroup\expandafter\expandafter\expandafter\endgroup
\expandafter\ifx\csname currentgrouplevel\endcsname\relax
  \def\pcp@ColorPatch{}%
\else
  \def\pcp@ColorPatch{%
    \@ifundefined{set@color}{%
      \gdef\pcp@ColorPatch{}%
    }{%
      \gdef\pcp@ColorPatch{%
        \gdef\pcp@ColorResets{}%
        \bgroup
        \aftergroup\pcp@ColorResets
        \aftergroup\egroup
        \let\pcp@OrgSetColor\set@color
        \let\set@color\pcp@SetColor
        \edef\pcp@GroupLevel{\the\currentgrouplevel}%
      }%
    }%
    \pcp@ColorPatch
  }%
%    \end{macrocode}
%    \end{macro}
%    \begin{macro}{\pcp@SetColor}
%    \begin{macrocode}
  \def\pcp@SetColor{%
    \ifnum\pcp@GroupLevel=\currentgrouplevel
      \let\pcp@OrgAfterGroup\aftergroup
      \def\aftergroup{%
        \g@addto@macro\pcp@ColorResets
      }%
      \pcp@OrgSetColor
      \let\aftergroup\pcp@OrgAfterGroup
    \else
      \pcp@OrgSetColor
    \fi
  }%
\fi
%    \end{macrocode}
%    \end{macro}
%
%    \begin{macro}{\pcp@CmdCheckRedef}
%    \begin{macrocode}
\def\pcp@CmdCheckRedef#1{%
  \begingroup
    \def\pcp@cmd{#1}%
    \afterassignment\pcp@CmdDo
    \long\def\reserved@a
}
\def\pcp@CmdDo{%
    \expandafter\ifx\pcp@cmd\reserved@a
    \else
      \edef\x*{\expandafter\string\pcp@cmd}%
      \@PackageWarningNoLine{pdfcolparallel}{%
        Command \x* has changed.\MessageBreak
        Supported versions of package `parallel':\MessageBreak
        \space\space 2003/04/13\MessageBreak
        The redefinition of \x* may\MessageBreak
        not behave correctly depending on the changes%
      }%
    \fi
  \expandafter\endgroup
  \expandafter\def\pcp@cmd
}
%    \end{macrocode}
%    \end{macro}
%
%    \begin{macrocode}
\def\pcp@SwitchStack#1#2{}
%    \end{macrocode}
%    \begin{macrocode}
\def\pcp@SetCurrent#1{}
%    \end{macrocode}
%
%    \begin{macro}{\ParallelLText}
%    \begin{macrocode}
\pcp@CmdCheckRedef\ParallelLText{%
  \everypar{}%
  \@restorepar
  \begingroup
    \hbadness=3000 %
    \let\footnote=\ParallelLFootnote
    \ParallelWhichBox=0 %
    \global\setbox\ParallelLBox=\vbox\bgroup
      \hsize=\ParallelLWidth
      \aftergroup\ParallelAfterText
      \begingroup
        \afterassignment\ParallelCheckOpenBrace
        \let\x=%
}{%
  \everypar{}%
  \@restorepar
  \@nobreakfalse
  \begingroup
    \hbadness=3000 %
    \let\footnote=\ParallelLFootnote
    \ParallelWhichBox=0 %
    \global\setbox\ParallelLBox=\vbox\bgroup
      \hsize=\ParallelLWidth
      \linewidth=\ParallelLWidth
      \pcp@SwitchStack{Left}\ParallelLBox
      \aftergroup\ParallelAfterText
      \pcp@ColorPatch
      \begingroup
        \afterassignment\ParallelCheckOpenBrace
        \let\x=%
}
%    \end{macrocode}
%    \end{macro}
%
%    \begin{macro}{\ParallelRText}
%    \begin{macrocode}
\pcp@CmdCheckRedef\ParallelRText{%
  \everypar{}%
  \@restorepar
  \begingroup
    \hbadness=3000 %
    \ifnum\ParallelFNMode=\@ne
      \let\footnote=\ParallelRFootnote
    \else
      \let\footnote=\ParallelLFootnote
    \fi
    \ParallelWhichBox=\@ne
    \global\setbox\ParallelRBox=\vbox\bgroup
      \hsize=\ParallelRWidth
      \aftergroup\ParallelAfterText
      \begingroup
        \afterassignment\ParallelCheckOpenBrace
        \let\x=%
}{%
  \everypar{}%
  \@restorepar
  \@nobreakfalse
  \begingroup
    \hbadness=3000 %
    \ifnum\ParallelFNMode=\@ne
      \let\footnote=\ParallelRFootnote
    \else
      \let\footnote=\ParallelLFootnote
    \fi
    \ParallelWhichBox=\@ne
    \global\setbox\ParallelRBox=\vbox\bgroup
      \hsize=\ParallelRWidth
      \linewidth=\ParallelRWidth
      \pcp@SwitchStack{Right}\ParallelRBox
      \aftergroup\ParallelAfterText
      \pcp@ColorPatch
      \begingroup
        \afterassignment\ParallelCheckOpenBrace
        \let\x=%
}
%    \end{macrocode}
%    \end{macro}
%
%    \begin{macro}{\ParallelParTwoPages}
%    \begin{macrocode}
\pcp@CmdCheckRedef\ParallelParTwoPages{%
  \ifnum\ParallelBoolVar=\@ne
    \par
    \begingroup
      \global\ParallelWhichBox=\@ne
      \newpage
      \vbadness=10000 %
      \vfuzz=3ex %
      \splittopskip=\z@skip
      \loop%
        \ifnum\ParallelBoolVar=\@ne%
          \ifnum\ParallelWhichBox=\@ne
            \ifvoid\ParallelLBox
              \mbox{} %
              \newpage
            \else
              \global\ParallelWhichBox=\z@
            \fi
          \else
            \ifvoid\ParallelRBox
              \mbox{} %
              \newpage
            \else
              \global\ParallelWhichBox=\@ne
            \fi
          \fi
          \ifnum\ParallelWhichBox=\z@
            \ifodd\thepage
              \mbox{} %
              \newpage
            \fi
            \hbox to\textwidth{%
              \vbox{\vsplit\ParallelLBox to.98\textheight}%
            }%
          \else
            \ifodd\thepage\relax
            \else
              \mbox{} %
              \newpage
            \fi
            \hbox to\textwidth{%
              \vbox{\vsplit\ParallelRBox to.98\textheight}%
            }%
          \fi
          \vspace*{\fill}%
          \newpage
        \fi
        \ifvoid\ParallelLBox
          \ifvoid\ParallelRBox
            \global\ParallelBoolVar=\z@
          \fi
        \fi
      \ifnum\ParallelBoolVar=\@ne
      \repeat
      \par
    \endgroup
  \fi
}{%
%    \end{macrocode}
%    Additional fixes:
%    \begin{itemize}
%    \item Unnecessary white space removed.
%    \item |\ifodd\thepage| changed to |\ifodd\value{page}|.
%    \end{itemize}
%    \begin{macrocode}
  \ifnum\ParallelBoolVar=\@ne
    \par
    \begingroup
      \global\ParallelWhichBox=\@ne
      \newpage
      \vbadness=10000 %
      \vfuzz=3ex %
      \splittopskip=\z@skip
      \loop%
        \ifnum\ParallelBoolVar=\@ne%
          \ifnum\ParallelWhichBox=\@ne
            \ifvoid\ParallelLBox
              \mbox{}%
              \newpage
            \else
              \global\ParallelWhichBox=\z@
            \fi
          \else
            \ifvoid\ParallelRBox
              \null
              \newpage
            \else
              \global\ParallelWhichBox=\@ne
            \fi
          \fi
          \ifnum\ParallelWhichBox=\z@
            \ifodd\value{page}%
              \null
              \newpage
            \fi
            \hbox to\textwidth{%
              \pcp@SetCurrent{Left}%
              \setbox\z@=\vsplit\ParallelLBox to.98\textheight
              \vbox to.98\textheight{%
                \@texttop
                \unvbox\z@
                \@textbottom
              }%
            }%
          \else
            \ifodd\value{page}%
            \else
              \mbox{}%
              \newpage
            \fi
            \hbox to\textwidth{%
              \pcp@SetCurrent{Right}%
              \setbox\z@=\vsplit\ParallelRBox to.98\textheight
              \vbox to.98\textheight{%
                \@texttop
                \unvbox\z@
                \@textbottom
              }%
            }%
          \fi
          \vspace*{\fill}%
          \newpage
        \fi
        \ifvoid\ParallelLBox
          \ifvoid\ParallelRBox
            \global\ParallelBoolVar=\z@
          \fi
        \fi
      \ifnum\ParallelBoolVar=\@ne
      \repeat
      \par
    \endgroup
    \pcp@SetCurrent{}%
  \fi
}
%    \end{macrocode}
%    \end{macro}
%
% \subsection{Color stack support}
%
%    \begin{macrocode}
\RequirePackage{pdfcol}[2007/12/12]
\ifpdfcolAvailable
\else
  \PackageInfo{pdfcolparallel}{%
    Loading aborted, because color stacks are not available%
  }%
  \expandafter\endinput
\fi
%    \end{macrocode}
%
%    \begin{macrocode}
\pdfcolInitStack{pcp@Left}
\pdfcolInitStack{pcp@Right}
%    \end{macrocode}
%    \begin{macro}{\pcp@Box}
%    \begin{macrocode}
\newbox\pcp@Box
%    \end{macrocode}
%    \end{macro}
%    \begin{macro}{\pcp@SwitchStack}
%    \begin{macrocode}
\def\pcp@SwitchStack#1#2{%
  \pdfcolSwitchStack{pcp@#1}%
  \global\setbox\pcp@Box=\vbox to 0pt{%
    \pdfcolSetCurrentColor
  }%
  \aftergroup\pcp@FixBox
  \aftergroup#2%
}
%    \end{macrocode}
%    \end{macro}
%    \begin{macro}{\pcp@FixBox}
%    \begin{macrocode}
\def\pcp@FixBox#1{%
  \global\setbox#1=\vbox{%
    \unvbox\pcp@Box
    \unvbox#1%
  }%
}
%    \end{macrocode}
%    \end{macro}
%    \begin{macro}{\pcp@SetCurrent}
%    \begin{macrocode}
\def\pcp@SetCurrent#1{%
  \ifx\\#1\\%
    \pdfcolSetCurrent{}%
  \else
    \pdfcolSetCurrent{pcp@#1}%
  \fi
}
%    \end{macrocode}
%    \end{macro}
%
% \subsection{Redefinitions}
%
%    \begin{macro}{\ParallelParOnePage}
%    \begin{macrocode}
\pcp@CmdCheckRedef\ParallelParOnePage{%
  \ifnum\ParallelBoolVar=\@ne
    \par
    \begingroup
      \leftmargin=\z@
      \rightmargin=\z@
      \parskip=\z@skip
      \parindent=\z@
      \vbadness=10000 %
      \vfuzz=3ex %
      \splittopskip=\z@skip
      \loop
        \ifnum\ParallelBoolVar=\@ne
          \noindent
          \hbox to\textwidth{%
            \hskip\ParallelLeftMargin
            \hbox to\ParallelTextWidth{%
              \ifvoid\ParallelLBox
                \hskip\ParallelLWidth
              \else
                \ParallelWhichBox=\z@
                \vbox{%
                  \setbox\ParallelBoxVar
                      =\vsplit\ParallelLBox to\dp\strutbox
                  \unvbox\ParallelBoxVar
                }%
              \fi
              \strut
              \ifnum\ParallelBoolMid=\@ne
                \hskip\ParallelMainMidSkip
                \vrule
              \else
                \hss
              \fi
              \hss
              \ifvoid\ParallelRBox
                \hskip\ParallelRWidth
              \else
                \ParallelWhichBox=\@ne
                \vbox{%
                  \setbox\ParallelBoxVar
                      =\vsplit\ParallelRBox to\dp\strutbox
                  \unvbox\ParallelBoxVar
                }%
              \fi
            }%
          }%
          \ifvoid\ParallelLBox
            \ifvoid\ParallelRBox
              \global\ParallelBoolVar=\z@
            \fi
          \fi%
        \fi%
      \ifnum\ParallelBoolVar=\@ne
        \penalty\interlinepenalty
      \repeat
      \par
    \endgroup
  \fi
}{%
  \ifnum\ParallelBoolVar=\@ne
    \par
    \begingroup
      \leftmargin=\z@
      \rightmargin=\z@
      \parskip=\z@skip
      \parindent=\z@
      \vbadness=10000 %
      \vfuzz=3ex %
      \splittopskip=\z@skip
      \loop
        \ifnum\ParallelBoolVar=\@ne
          \noindent
          \hbox to\textwidth{%
            \hskip\ParallelLeftMargin
            \hbox to\ParallelTextWidth{%
              \ifvoid\ParallelLBox
                \hskip\ParallelLWidth
              \else
                \pcp@SetCurrent{Left}%
                \ParallelWhichBox=\z@
                \vbox{%
                  \setbox\ParallelBoxVar
                      =\vsplit\ParallelLBox to\dp\strutbox
                  \unvbox\ParallelBoxVar
                }%
              \fi
              \strut
              \ifnum\ParallelBoolMid=\@ne
                \hskip\ParallelMainMidSkip
                \begingroup
                  \pcp@RuleBetweenColor
                  \vrule
                \endgroup
              \else
                \hss
              \fi
              \hss
              \ifvoid\ParallelRBox
                \hskip\ParallelRWidth
              \else
                \pcp@SetCurrent{Right}%
                \ParallelWhichBox=\@ne
                \vbox{%
                  \setbox\ParallelBoxVar
                      =\vsplit\ParallelRBox to\dp\strutbox
                  \unvbox\ParallelBoxVar
                }%
              \fi
            }%
          }%
          \ifvoid\ParallelLBox
            \ifvoid\ParallelRBox
              \global\ParallelBoolVar=\z@
            \fi
          \fi%
        \fi%
      \ifnum\ParallelBoolVar=\@ne
        \penalty\interlinepenalty
      \repeat
      \par
    \endgroup
    \pcp@SetCurrent{}%
  \fi
}
%    \end{macrocode}
%    \end{macro}
%    \begin{macro}{\pcp@RuleBetweenColorDefault}
%    \begin{macrocode}
\def\pcp@RuleBetweenColorDefault{%
  \normalcolor
}
%    \end{macrocode}
%    \end{macro}
%    \begin{macro}{\pcp@RuleBetweenColor}
%    \begin{macrocode}
\let\pcp@RuleBetweenColor\pcp@RuleBetweenColorDefault
%    \end{macrocode}
%    \end{macro}
%    \begin{macrocode}
\RequirePackage{keyval}
\define@key{parallel}{rulebetweencolor}{%
  \edef\pcp@temp{#1}%
  \ifx\pcp@temp\@empty
    \let\pcp@RuleBetweenColor\pcp@RuleBetweenColorDefault
  \else
    \edef\pcp@temp{%
      \noexpand\@ifnextchar[{%
        \def\noexpand\pcp@RuleBetweenColor{%
          \noexpand\color\pcp@temp
        }%
        \noexpand\pcp@GobbleNil
      }{%
        \def\noexpand\pcp@RuleBetweenColor{%
          \noexpand\color{\pcp@temp}%
        }%
        \noexpand\pcp@GobbleNil
      }%
      \pcp@temp\noexpand\@nil
    }%
    \pcp@temp
  \fi
}
%    \end{macrocode}
%    \begin{macro}{\pcp@GobbleNil}
%    \begin{macrocode}
\long\def\pcp@GobbleNil#1\@nil{}
%    \end{macrocode}
%    \end{macro}
%
%    \begin{macrocode}
%</package>
%    \end{macrocode}
%
% \section{Test}
%
%    The test file is a modified version of the file that
%    Alexander Hirsch has posted in \xnewsgroup{de.comp.text.tex}:
%    \URL{``\link{\texttt{parallel.sty} und farbiger Text}''}^^A
%    {http://groups.google.com/group/de.comp.text.tex/msg/6a759cf33bb071a5}
%    \begin{macrocode}
%<*test1>
\AtEndDocument{%
  \typeout{}%
  \typeout{**************************************}%
  \typeout{*** \space Check the PDF file manually! \space ***}%
  \typeout{**************************************}%
  \typeout{}%
}
\documentclass{article}
\usepackage{xcolor}
\usepackage{pdfcolparallel}[2016/05/16]

\begin{document}
  \color{green}%
  Green%
  \begin{Parallel}{0.47\textwidth}{0.47\textwidth}%
    \ParallelLText{%
      \textcolor{red}{%
        Ein Absatz, der sich ueber zwei Zeilen erstrecken soll. %
        Ein Absatz, der sich ueber zwei Zeilen erstrecken soll.%
      }%
    }%
    \ParallelRText{%
      \textcolor{blue}{%
        Ein Absatz, der sich ueber zwei Zeilen erstrecken soll. %
        Ein Absatz, der sich ueber zwei Zeilen erstrecken soll.%
      }%
    }%
    \ParallelPar
    \ParallelLText{%
      Default %
      \color{red}%
      Ein Absatz, der sich ueber zwei Zeilen erstrecken soll. %
      Ein Absatz, der sich ueber zwei Zeilen erstrecken soll.%
    }%
    \ParallelRText{%
      Default %
      \color{blue}%
      Ein Absatz, der sich ueber zwei Zeilen erstrecken soll. %
      Ein Absatz, der sich ueber zwei Zeilen erstrecken soll.%
    }%
    \ParallelPar
    \ParallelLText{%
      \begin{enumerate}%
      \item left text, left text, left text, left text, %
            left text, left text, left text, left text,%
      \item left text, left text, left text, left text, %
            left text, left text, left text, left text.%
      \end{enumerate}%
    }%
    \ParallelRText{%
      \begin{enumerate}%
      \item right text, right text, right text, right text, %
            right text, right text, right text, right text.%
      \item right text, right text, right text, right text, %
            right text, right text, right text, right text.%
      \end{enumerate}%
    }%
  \end{Parallel}%
  \begin{Parallel}[p]{\textwidth}{\textwidth}%
    \ParallelLText{%
      \textcolor{red}{%
        Ein Absatz, der sich ueber zwei Zeilen erstrecken soll. %
        Ein Absatz, der sich ueber zwei Zeilen erstrecken soll. %
        Foo bar bla bla bla.%
      }%
      \par
      Und noch ein Absatz.%
    }%
    \ParallelRText{%
      \textcolor{blue}{%
        Ein Absatz, der sich ueber zwei Zeilen erstrecken soll. %
        Ein Absatz, der sich ueber zwei Zeilen erstrecken soll. %
        Foo bar bla bla bla.%
      }%
    }%
  \end{Parallel}%
  \begin{Parallel}[p]{\textwidth}{\textwidth}%
    \ParallelLText{%
      \rule{1pt}{.98\textheight}\Huge g%
    }%
    \ParallelRText{%
      \rule{1pt}{.98\textheight}y%
    }%
  \end{Parallel}%
  Green%
\end{document}
%</test1>
%    \end{macrocode}
%
% \section{Installation}
%
% \subsection{Download}
%
% \paragraph{Package.} This package is available on
% CTAN\footnote{\url{http://ctan.org/pkg/pdfcolparallel}}:
% \begin{description}
% \item[\CTAN{macros/latex/contrib/oberdiek/pdfcolparallel.dtx}] The source file.
% \item[\CTAN{macros/latex/contrib/oberdiek/pdfcolparallel.pdf}] Documentation.
% \end{description}
%
%
% \paragraph{Bundle.} All the packages of the bundle `oberdiek'
% are also available in a TDS compliant ZIP archive. There
% the packages are already unpacked and the documentation files
% are generated. The files and directories obey the TDS standard.
% \begin{description}
% \item[\CTAN{install/macros/latex/contrib/oberdiek.tds.zip}]
% \end{description}
% \emph{TDS} refers to the standard ``A Directory Structure
% for \TeX\ Files'' (\CTAN{tds/tds.pdf}). Directories
% with \xfile{texmf} in their name are usually organized this way.
%
% \subsection{Bundle installation}
%
% \paragraph{Unpacking.} Unpack the \xfile{oberdiek.tds.zip} in the
% TDS tree (also known as \xfile{texmf} tree) of your choice.
% Example (linux):
% \begin{quote}
%   |unzip oberdiek.tds.zip -d ~/texmf|
% \end{quote}
%
% \paragraph{Script installation.}
% Check the directory \xfile{TDS:scripts/oberdiek/} for
% scripts that need further installation steps.
% Package \xpackage{attachfile2} comes with the Perl script
% \xfile{pdfatfi.pl} that should be installed in such a way
% that it can be called as \texttt{pdfatfi}.
% Example (linux):
% \begin{quote}
%   |chmod +x scripts/oberdiek/pdfatfi.pl|\\
%   |cp scripts/oberdiek/pdfatfi.pl /usr/local/bin/|
% \end{quote}
%
% \subsection{Package installation}
%
% \paragraph{Unpacking.} The \xfile{.dtx} file is a self-extracting
% \docstrip\ archive. The files are extracted by running the
% \xfile{.dtx} through \plainTeX:
% \begin{quote}
%   \verb|tex pdfcolparallel.dtx|
% \end{quote}
%
% \paragraph{TDS.} Now the different files must be moved into
% the different directories in your installation TDS tree
% (also known as \xfile{texmf} tree):
% \begin{quote}
% \def\t{^^A
% \begin{tabular}{@{}>{\ttfamily}l@{ $\rightarrow$ }>{\ttfamily}l@{}}
%   pdfcolparallel.sty & tex/latex/oberdiek/pdfcolparallel.sty\\
%   pdfcolparallel.pdf & doc/latex/oberdiek/pdfcolparallel.pdf\\
%   test/pdfcolparallel-test1.tex & doc/latex/oberdiek/test/pdfcolparallel-test1.tex\\
%   pdfcolparallel.dtx & source/latex/oberdiek/pdfcolparallel.dtx\\
% \end{tabular}^^A
% }^^A
% \sbox0{\t}^^A
% \ifdim\wd0>\linewidth
%   \begingroup
%     \advance\linewidth by\leftmargin
%     \advance\linewidth by\rightmargin
%   \edef\x{\endgroup
%     \def\noexpand\lw{\the\linewidth}^^A
%   }\x
%   \def\lwbox{^^A
%     \leavevmode
%     \hbox to \linewidth{^^A
%       \kern-\leftmargin\relax
%       \hss
%       \usebox0
%       \hss
%       \kern-\rightmargin\relax
%     }^^A
%   }^^A
%   \ifdim\wd0>\lw
%     \sbox0{\small\t}^^A
%     \ifdim\wd0>\linewidth
%       \ifdim\wd0>\lw
%         \sbox0{\footnotesize\t}^^A
%         \ifdim\wd0>\linewidth
%           \ifdim\wd0>\lw
%             \sbox0{\scriptsize\t}^^A
%             \ifdim\wd0>\linewidth
%               \ifdim\wd0>\lw
%                 \sbox0{\tiny\t}^^A
%                 \ifdim\wd0>\linewidth
%                   \lwbox
%                 \else
%                   \usebox0
%                 \fi
%               \else
%                 \lwbox
%               \fi
%             \else
%               \usebox0
%             \fi
%           \else
%             \lwbox
%           \fi
%         \else
%           \usebox0
%         \fi
%       \else
%         \lwbox
%       \fi
%     \else
%       \usebox0
%     \fi
%   \else
%     \lwbox
%   \fi
% \else
%   \usebox0
% \fi
% \end{quote}
% If you have a \xfile{docstrip.cfg} that configures and enables \docstrip's
% TDS installing feature, then some files can already be in the right
% place, see the documentation of \docstrip.
%
% \subsection{Refresh file name databases}
%
% If your \TeX~distribution
% (\teTeX, \mikTeX, \dots) relies on file name databases, you must refresh
% these. For example, \teTeX\ users run \verb|texhash| or
% \verb|mktexlsr|.
%
% \subsection{Some details for the interested}
%
% \paragraph{Attached source.}
%
% The PDF documentation on CTAN also includes the
% \xfile{.dtx} source file. It can be extracted by
% AcrobatReader 6 or higher. Another option is \textsf{pdftk},
% e.g. unpack the file into the current directory:
% \begin{quote}
%   \verb|pdftk pdfcolparallel.pdf unpack_files output .|
% \end{quote}
%
% \paragraph{Unpacking with \LaTeX.}
% The \xfile{.dtx} chooses its action depending on the format:
% \begin{description}
% \item[\plainTeX:] Run \docstrip\ and extract the files.
% \item[\LaTeX:] Generate the documentation.
% \end{description}
% If you insist on using \LaTeX\ for \docstrip\ (really,
% \docstrip\ does not need \LaTeX), then inform the autodetect routine
% about your intention:
% \begin{quote}
%   \verb|latex \let\install=y\input{pdfcolparallel.dtx}|
% \end{quote}
% Do not forget to quote the argument according to the demands
% of your shell.
%
% \paragraph{Generating the documentation.}
% You can use both the \xfile{.dtx} or the \xfile{.drv} to generate
% the documentation. The process can be configured by the
% configuration file \xfile{ltxdoc.cfg}. For instance, put this
% line into this file, if you want to have A4 as paper format:
% \begin{quote}
%   \verb|\PassOptionsToClass{a4paper}{article}|
% \end{quote}
% An example follows how to generate the
% documentation with pdf\LaTeX:
% \begin{quote}
%\begin{verbatim}
%pdflatex pdfcolparallel.dtx
%makeindex -s gind.ist pdfcolparallel.idx
%pdflatex pdfcolparallel.dtx
%makeindex -s gind.ist pdfcolparallel.idx
%pdflatex pdfcolparallel.dtx
%\end{verbatim}
% \end{quote}
%
% \section{Catalogue}
%
% The following XML file can be used as source for the
% \href{http://mirror.ctan.org/help/Catalogue/catalogue.html}{\TeX\ Catalogue}.
% The elements \texttt{caption} and \texttt{description} are imported
% from the original XML file from the Catalogue.
% The name of the XML file in the Catalogue is \xfile{pdfcolparallel.xml}.
%    \begin{macrocode}
%<*catalogue>
<?xml version='1.0' encoding='us-ascii'?>
<!DOCTYPE entry SYSTEM 'catalogue.dtd'>
<entry datestamp='$Date$' modifier='$Author$' id='pdfcolparallel'>
  <name>pdfcolparallel</name>
  <caption>Fix colour problems in package 'parallel'.</caption>
  <authorref id='auth:oberdiek'/>
  <copyright owner='Heiko Oberdiek' year='2007,2008,2010'/>
  <license type='lppl1.3'/>
  <version number='1.4'/>
  <description>
    Since version 1.40 pdfTeX supports colour stacks.
    This package uses them to fix colour problems in
    package <xref refid='parallel'>parallel</xref>.
    <p/>
    The package is part of the <xref refid='oberdiek'>oberdiek</xref>
    bundle.
  </description>
  <documentation details='Package documentation'
      href='ctan:/macros/latex/contrib/oberdiek/pdfcolparallel.pdf'/>
  <ctan file='true' path='/macros/latex/contrib/oberdiek/pdfcolparallel.dtx'/>
  <miktex location='oberdiek'/>
  <texlive location='oberdiek'/>
  <install path='/macros/latex/contrib/oberdiek/oberdiek.tds.zip'/>
</entry>
%</catalogue>
%    \end{macrocode}
%
% \begin{thebibliography}{9}
%
% \bibitem{parallel}
%   Matthias Eckermann: \textit{The \xpackage{parallel}-package};
%   2003/04/13;\\
%   \CTAN{macros/latex/contrib/parallel/}.
%
% \bibitem{pdfcol}
%   Heiko Oberdiek: \textit{The \xpackage{pdfcol} package};
%   2007/09/09;\\
%   \CTAN{macros/latex/contrib/oberdiek/pdfcol.pdf}.
%
% \end{thebibliography}
%
% \begin{History}
%   \begin{Version}{2007/09/09 v1.0}
%   \item
%     First version.
%   \end{Version}
%   \begin{Version}{2007/12/12 v1.1}
%   \item
%     Adds patch for setting \cs{linewidth} to fix bug
%     in package \xpackage{parallel}.
%   \item
%     Package \xpackage{parallel} is also fixed if color
%     stacks are not available.
%   \item
%     Bug fix, switched stacks now initialized with current color.
%   \item
%     Fix for package \xpackage{parallel}: \cs{raggedbottom} is respected.
%   \end{Version}
%   \begin{Version}{2008/08/11 v1.2}
%   \item
%     Code is not changed.
%   \item
%     URLs updated.
%   \end{Version}
%   \begin{Version}{2010/01/11 v1.3}
%   \item
%     Option `rulebetweencolor' added.
%   \end{Version}
%   \begin{Version}{2016/05/16 v1.4}
%   \item
%     Documentation updates.
%   \end{Version}
% \end{History}
%
% \PrintIndex
%
% \Finale
\endinput

%        (quote the arguments according to the demands of your shell)
%
% Documentation:
%    (a) If pdfcolparallel.drv is present:
%           latex pdfcolparallel.drv
%    (b) Without pdfcolparallel.drv:
%           latex pdfcolparallel.dtx; ...
%    The class ltxdoc loads the configuration file ltxdoc.cfg
%    if available. Here you can specify further options, e.g.
%    use A4 as paper format:
%       \PassOptionsToClass{a4paper}{article}
%
%    Programm calls to get the documentation (example):
%       pdflatex pdfcolparallel.dtx
%       makeindex -s gind.ist pdfcolparallel.idx
%       pdflatex pdfcolparallel.dtx
%       makeindex -s gind.ist pdfcolparallel.idx
%       pdflatex pdfcolparallel.dtx
%
% Installation:
%    TDS:tex/latex/oberdiek/pdfcolparallel.sty
%    TDS:doc/latex/oberdiek/pdfcolparallel.pdf
%    TDS:doc/latex/oberdiek/test/pdfcolparallel-test1.tex
%    TDS:source/latex/oberdiek/pdfcolparallel.dtx
%
%<*ignore>
\begingroup
  \catcode123=1 %
  \catcode125=2 %
  \def\x{LaTeX2e}%
\expandafter\endgroup
\ifcase 0\ifx\install y1\fi\expandafter
         \ifx\csname processbatchFile\endcsname\relax\else1\fi
         \ifx\fmtname\x\else 1\fi\relax
\else\csname fi\endcsname
%</ignore>
%<*install>
\input docstrip.tex
\Msg{************************************************************************}
\Msg{* Installation}
\Msg{* Package: pdfcolparallel 2016/05/16 v1.4 Color stacks support for parallel (HO)}
\Msg{************************************************************************}

\keepsilent
\askforoverwritefalse

\let\MetaPrefix\relax
\preamble

This is a generated file.

Project: pdfcolparallel
Version: 2016/05/16 v1.4

Copyright (C) 2007, 2008, 2010 by
   Heiko Oberdiek <heiko.oberdiek at googlemail.com>

This work may be distributed and/or modified under the
conditions of the LaTeX Project Public License, either
version 1.3c of this license or (at your option) any later
version. This version of this license is in
   http://www.latex-project.org/lppl/lppl-1-3c.txt
and the latest version of this license is in
   http://www.latex-project.org/lppl.txt
and version 1.3 or later is part of all distributions of
LaTeX version 2005/12/01 or later.

This work has the LPPL maintenance status "maintained".

This Current Maintainer of this work is Heiko Oberdiek.

This work consists of the main source file pdfcolparallel.dtx
and the derived files
   pdfcolparallel.sty, pdfcolparallel.pdf, pdfcolparallel.ins,
   pdfcolparallel.drv, pdfcolparallel-test1.tex.

\endpreamble
\let\MetaPrefix\DoubleperCent

\generate{%
  \file{pdfcolparallel.ins}{\from{pdfcolparallel.dtx}{install}}%
  \file{pdfcolparallel.drv}{\from{pdfcolparallel.dtx}{driver}}%
  \usedir{tex/latex/oberdiek}%
  \file{pdfcolparallel.sty}{\from{pdfcolparallel.dtx}{package}}%
  \usedir{doc/latex/oberdiek/test}%
  \file{pdfcolparallel-test1.tex}{\from{pdfcolparallel.dtx}{test1}}%
  \nopreamble
  \nopostamble
  \usedir{source/latex/oberdiek/catalogue}%
  \file{pdfcolparallel.xml}{\from{pdfcolparallel.dtx}{catalogue}}%
}

\catcode32=13\relax% active space
\let =\space%
\Msg{************************************************************************}
\Msg{*}
\Msg{* To finish the installation you have to move the following}
\Msg{* file into a directory searched by TeX:}
\Msg{*}
\Msg{*     pdfcolparallel.sty}
\Msg{*}
\Msg{* To produce the documentation run the file `pdfcolparallel.drv'}
\Msg{* through LaTeX.}
\Msg{*}
\Msg{* Happy TeXing!}
\Msg{*}
\Msg{************************************************************************}

\endbatchfile
%</install>
%<*ignore>
\fi
%</ignore>
%<*driver>
\NeedsTeXFormat{LaTeX2e}
\ProvidesFile{pdfcolparallel.drv}%
  [2016/05/16 v1.4 Color stacks support for parallel (HO)]%
\documentclass{ltxdoc}
\usepackage{holtxdoc}[2011/11/22]
\begin{document}
  \DocInput{pdfcolparallel.dtx}%
\end{document}
%</driver>
% \fi
%
%
% \CharacterTable
%  {Upper-case    \A\B\C\D\E\F\G\H\I\J\K\L\M\N\O\P\Q\R\S\T\U\V\W\X\Y\Z
%   Lower-case    \a\b\c\d\e\f\g\h\i\j\k\l\m\n\o\p\q\r\s\t\u\v\w\x\y\z
%   Digits        \0\1\2\3\4\5\6\7\8\9
%   Exclamation   \!     Double quote  \"     Hash (number) \#
%   Dollar        \$     Percent       \%     Ampersand     \&
%   Acute accent  \'     Left paren    \(     Right paren   \)
%   Asterisk      \*     Plus          \+     Comma         \,
%   Minus         \-     Point         \.     Solidus       \/
%   Colon         \:     Semicolon     \;     Less than     \<
%   Equals        \=     Greater than  \>     Question mark \?
%   Commercial at \@     Left bracket  \[     Backslash     \\
%   Right bracket \]     Circumflex    \^     Underscore    \_
%   Grave accent  \`     Left brace    \{     Vertical bar  \|
%   Right brace   \}     Tilde         \~}
%
% \GetFileInfo{pdfcolparallel.drv}
%
% \title{The \xpackage{pdfcolparallel} package}
% \date{2016/05/16 v1.4}
% \author{Heiko Oberdiek\thanks
% {Please report any issues at https://github.com/ho-tex/oberdiek/issues}\\
% \xemail{heiko.oberdiek at googlemail.com}}
%
% \maketitle
%
% \begin{abstract}
% This packages fixes bugs in \xpackage{parallel} and
% improves color support by using several color stacks
% that are provided by \pdfTeX\ since version 1.40.
% \end{abstract}
%
% \tableofcontents
%
% \section{Usage}
%
% \begin{quote}
% |\usepackage{pdfcolparallel}|
% \end{quote}
% The package \xpackage{pdfcolparallel} loads package \xpackage{parallel}
% \cite{parallel} and redefines some macros to fix bugs.
%
% If color stacks are available then package \xpackage{parallel}
% is further patched to support them.
%
% \subsection{Option \xoption{rulebetweencolor}}
%
% Package \xpackage{pdfcolparallel} also fixes the color for the
% rule between columns.
% Default color is \cs{normalcolor}. But this can be changed by using
% option \xoption{rulebetweencolor} for |\setkeys{parallel}|
% (see package \xpackage{keyval}). The option takes a color specification
% as value. If the value is empty, then the default (\cs{normalcolor})
% is used.
% Examples:
% \begin{quote}
%   |\setkeys{parallel}{rulebetweencolor=blue}|,\\
%   |\setkeys{parallel}{rulebetweencolor={red}}|,\\
%   |\setkeys{parallel}{rulebetweencolor={}}|,
%     \textit{\% \cs{normalcolor} is used}\\
%   |\setkeys{parallel}{rulebetweencolor=[rgb]{1,0,.5}}|
% \end{quote}
%
% \subsection{Future}
%
% If there will be a new version of package \xpackage{parallel}
% that adds support for color stacks, then this package may become
% obsolete.
%
% \StopEventually{
% }
%
% \section{Implementation}
%
% \subsection{Identification}
%
%    \begin{macrocode}
%<*package>
\NeedsTeXFormat{LaTeX2e}
\ProvidesPackage{pdfcolparallel}%
  [2016/05/16 v1.4 Color stacks support for parallel (HO)]%
%    \end{macrocode}
%
% \subsection{Load and fix package \xpackage{parallel}}
%
%    Package \xpackage{parallel} is loaded. Before options of package
%    \xpackage{pdfcolparallel} are passed to package \xpackage{parallel}.
%    \begin{macrocode}
\DeclareOption*{%
  \PassoptionsToPackage{\CurrentOption}{parallel}%
}
\ProcessOptions\relax
\RequirePackage{parallel}[2003/04/13]
%    \end{macrocode}
%
%    \begin{macrocode}
\RequirePackage{infwarerr}[2007/09/09]
%    \end{macrocode}
%
%    \begin{macro}{\pcp@ColorPatch}
%    \begin{macrocode}
\begingroup\expandafter\expandafter\expandafter\endgroup
\expandafter\ifx\csname currentgrouplevel\endcsname\relax
  \def\pcp@ColorPatch{}%
\else
  \def\pcp@ColorPatch{%
    \@ifundefined{set@color}{%
      \gdef\pcp@ColorPatch{}%
    }{%
      \gdef\pcp@ColorPatch{%
        \gdef\pcp@ColorResets{}%
        \bgroup
        \aftergroup\pcp@ColorResets
        \aftergroup\egroup
        \let\pcp@OrgSetColor\set@color
        \let\set@color\pcp@SetColor
        \edef\pcp@GroupLevel{\the\currentgrouplevel}%
      }%
    }%
    \pcp@ColorPatch
  }%
%    \end{macrocode}
%    \end{macro}
%    \begin{macro}{\pcp@SetColor}
%    \begin{macrocode}
  \def\pcp@SetColor{%
    \ifnum\pcp@GroupLevel=\currentgrouplevel
      \let\pcp@OrgAfterGroup\aftergroup
      \def\aftergroup{%
        \g@addto@macro\pcp@ColorResets
      }%
      \pcp@OrgSetColor
      \let\aftergroup\pcp@OrgAfterGroup
    \else
      \pcp@OrgSetColor
    \fi
  }%
\fi
%    \end{macrocode}
%    \end{macro}
%
%    \begin{macro}{\pcp@CmdCheckRedef}
%    \begin{macrocode}
\def\pcp@CmdCheckRedef#1{%
  \begingroup
    \def\pcp@cmd{#1}%
    \afterassignment\pcp@CmdDo
    \long\def\reserved@a
}
\def\pcp@CmdDo{%
    \expandafter\ifx\pcp@cmd\reserved@a
    \else
      \edef\x*{\expandafter\string\pcp@cmd}%
      \@PackageWarningNoLine{pdfcolparallel}{%
        Command \x* has changed.\MessageBreak
        Supported versions of package `parallel':\MessageBreak
        \space\space 2003/04/13\MessageBreak
        The redefinition of \x* may\MessageBreak
        not behave correctly depending on the changes%
      }%
    \fi
  \expandafter\endgroup
  \expandafter\def\pcp@cmd
}
%    \end{macrocode}
%    \end{macro}
%
%    \begin{macrocode}
\def\pcp@SwitchStack#1#2{}
%    \end{macrocode}
%    \begin{macrocode}
\def\pcp@SetCurrent#1{}
%    \end{macrocode}
%
%    \begin{macro}{\ParallelLText}
%    \begin{macrocode}
\pcp@CmdCheckRedef\ParallelLText{%
  \everypar{}%
  \@restorepar
  \begingroup
    \hbadness=3000 %
    \let\footnote=\ParallelLFootnote
    \ParallelWhichBox=0 %
    \global\setbox\ParallelLBox=\vbox\bgroup
      \hsize=\ParallelLWidth
      \aftergroup\ParallelAfterText
      \begingroup
        \afterassignment\ParallelCheckOpenBrace
        \let\x=%
}{%
  \everypar{}%
  \@restorepar
  \@nobreakfalse
  \begingroup
    \hbadness=3000 %
    \let\footnote=\ParallelLFootnote
    \ParallelWhichBox=0 %
    \global\setbox\ParallelLBox=\vbox\bgroup
      \hsize=\ParallelLWidth
      \linewidth=\ParallelLWidth
      \pcp@SwitchStack{Left}\ParallelLBox
      \aftergroup\ParallelAfterText
      \pcp@ColorPatch
      \begingroup
        \afterassignment\ParallelCheckOpenBrace
        \let\x=%
}
%    \end{macrocode}
%    \end{macro}
%
%    \begin{macro}{\ParallelRText}
%    \begin{macrocode}
\pcp@CmdCheckRedef\ParallelRText{%
  \everypar{}%
  \@restorepar
  \begingroup
    \hbadness=3000 %
    \ifnum\ParallelFNMode=\@ne
      \let\footnote=\ParallelRFootnote
    \else
      \let\footnote=\ParallelLFootnote
    \fi
    \ParallelWhichBox=\@ne
    \global\setbox\ParallelRBox=\vbox\bgroup
      \hsize=\ParallelRWidth
      \aftergroup\ParallelAfterText
      \begingroup
        \afterassignment\ParallelCheckOpenBrace
        \let\x=%
}{%
  \everypar{}%
  \@restorepar
  \@nobreakfalse
  \begingroup
    \hbadness=3000 %
    \ifnum\ParallelFNMode=\@ne
      \let\footnote=\ParallelRFootnote
    \else
      \let\footnote=\ParallelLFootnote
    \fi
    \ParallelWhichBox=\@ne
    \global\setbox\ParallelRBox=\vbox\bgroup
      \hsize=\ParallelRWidth
      \linewidth=\ParallelRWidth
      \pcp@SwitchStack{Right}\ParallelRBox
      \aftergroup\ParallelAfterText
      \pcp@ColorPatch
      \begingroup
        \afterassignment\ParallelCheckOpenBrace
        \let\x=%
}
%    \end{macrocode}
%    \end{macro}
%
%    \begin{macro}{\ParallelParTwoPages}
%    \begin{macrocode}
\pcp@CmdCheckRedef\ParallelParTwoPages{%
  \ifnum\ParallelBoolVar=\@ne
    \par
    \begingroup
      \global\ParallelWhichBox=\@ne
      \newpage
      \vbadness=10000 %
      \vfuzz=3ex %
      \splittopskip=\z@skip
      \loop%
        \ifnum\ParallelBoolVar=\@ne%
          \ifnum\ParallelWhichBox=\@ne
            \ifvoid\ParallelLBox
              \mbox{} %
              \newpage
            \else
              \global\ParallelWhichBox=\z@
            \fi
          \else
            \ifvoid\ParallelRBox
              \mbox{} %
              \newpage
            \else
              \global\ParallelWhichBox=\@ne
            \fi
          \fi
          \ifnum\ParallelWhichBox=\z@
            \ifodd\thepage
              \mbox{} %
              \newpage
            \fi
            \hbox to\textwidth{%
              \vbox{\vsplit\ParallelLBox to.98\textheight}%
            }%
          \else
            \ifodd\thepage\relax
            \else
              \mbox{} %
              \newpage
            \fi
            \hbox to\textwidth{%
              \vbox{\vsplit\ParallelRBox to.98\textheight}%
            }%
          \fi
          \vspace*{\fill}%
          \newpage
        \fi
        \ifvoid\ParallelLBox
          \ifvoid\ParallelRBox
            \global\ParallelBoolVar=\z@
          \fi
        \fi
      \ifnum\ParallelBoolVar=\@ne
      \repeat
      \par
    \endgroup
  \fi
}{%
%    \end{macrocode}
%    Additional fixes:
%    \begin{itemize}
%    \item Unnecessary white space removed.
%    \item |\ifodd\thepage| changed to |\ifodd\value{page}|.
%    \end{itemize}
%    \begin{macrocode}
  \ifnum\ParallelBoolVar=\@ne
    \par
    \begingroup
      \global\ParallelWhichBox=\@ne
      \newpage
      \vbadness=10000 %
      \vfuzz=3ex %
      \splittopskip=\z@skip
      \loop%
        \ifnum\ParallelBoolVar=\@ne%
          \ifnum\ParallelWhichBox=\@ne
            \ifvoid\ParallelLBox
              \mbox{}%
              \newpage
            \else
              \global\ParallelWhichBox=\z@
            \fi
          \else
            \ifvoid\ParallelRBox
              \null
              \newpage
            \else
              \global\ParallelWhichBox=\@ne
            \fi
          \fi
          \ifnum\ParallelWhichBox=\z@
            \ifodd\value{page}%
              \null
              \newpage
            \fi
            \hbox to\textwidth{%
              \pcp@SetCurrent{Left}%
              \setbox\z@=\vsplit\ParallelLBox to.98\textheight
              \vbox to.98\textheight{%
                \@texttop
                \unvbox\z@
                \@textbottom
              }%
            }%
          \else
            \ifodd\value{page}%
            \else
              \mbox{}%
              \newpage
            \fi
            \hbox to\textwidth{%
              \pcp@SetCurrent{Right}%
              \setbox\z@=\vsplit\ParallelRBox to.98\textheight
              \vbox to.98\textheight{%
                \@texttop
                \unvbox\z@
                \@textbottom
              }%
            }%
          \fi
          \vspace*{\fill}%
          \newpage
        \fi
        \ifvoid\ParallelLBox
          \ifvoid\ParallelRBox
            \global\ParallelBoolVar=\z@
          \fi
        \fi
      \ifnum\ParallelBoolVar=\@ne
      \repeat
      \par
    \endgroup
    \pcp@SetCurrent{}%
  \fi
}
%    \end{macrocode}
%    \end{macro}
%
% \subsection{Color stack support}
%
%    \begin{macrocode}
\RequirePackage{pdfcol}[2007/12/12]
\ifpdfcolAvailable
\else
  \PackageInfo{pdfcolparallel}{%
    Loading aborted, because color stacks are not available%
  }%
  \expandafter\endinput
\fi
%    \end{macrocode}
%
%    \begin{macrocode}
\pdfcolInitStack{pcp@Left}
\pdfcolInitStack{pcp@Right}
%    \end{macrocode}
%    \begin{macro}{\pcp@Box}
%    \begin{macrocode}
\newbox\pcp@Box
%    \end{macrocode}
%    \end{macro}
%    \begin{macro}{\pcp@SwitchStack}
%    \begin{macrocode}
\def\pcp@SwitchStack#1#2{%
  \pdfcolSwitchStack{pcp@#1}%
  \global\setbox\pcp@Box=\vbox to 0pt{%
    \pdfcolSetCurrentColor
  }%
  \aftergroup\pcp@FixBox
  \aftergroup#2%
}
%    \end{macrocode}
%    \end{macro}
%    \begin{macro}{\pcp@FixBox}
%    \begin{macrocode}
\def\pcp@FixBox#1{%
  \global\setbox#1=\vbox{%
    \unvbox\pcp@Box
    \unvbox#1%
  }%
}
%    \end{macrocode}
%    \end{macro}
%    \begin{macro}{\pcp@SetCurrent}
%    \begin{macrocode}
\def\pcp@SetCurrent#1{%
  \ifx\\#1\\%
    \pdfcolSetCurrent{}%
  \else
    \pdfcolSetCurrent{pcp@#1}%
  \fi
}
%    \end{macrocode}
%    \end{macro}
%
% \subsection{Redefinitions}
%
%    \begin{macro}{\ParallelParOnePage}
%    \begin{macrocode}
\pcp@CmdCheckRedef\ParallelParOnePage{%
  \ifnum\ParallelBoolVar=\@ne
    \par
    \begingroup
      \leftmargin=\z@
      \rightmargin=\z@
      \parskip=\z@skip
      \parindent=\z@
      \vbadness=10000 %
      \vfuzz=3ex %
      \splittopskip=\z@skip
      \loop
        \ifnum\ParallelBoolVar=\@ne
          \noindent
          \hbox to\textwidth{%
            \hskip\ParallelLeftMargin
            \hbox to\ParallelTextWidth{%
              \ifvoid\ParallelLBox
                \hskip\ParallelLWidth
              \else
                \ParallelWhichBox=\z@
                \vbox{%
                  \setbox\ParallelBoxVar
                      =\vsplit\ParallelLBox to\dp\strutbox
                  \unvbox\ParallelBoxVar
                }%
              \fi
              \strut
              \ifnum\ParallelBoolMid=\@ne
                \hskip\ParallelMainMidSkip
                \vrule
              \else
                \hss
              \fi
              \hss
              \ifvoid\ParallelRBox
                \hskip\ParallelRWidth
              \else
                \ParallelWhichBox=\@ne
                \vbox{%
                  \setbox\ParallelBoxVar
                      =\vsplit\ParallelRBox to\dp\strutbox
                  \unvbox\ParallelBoxVar
                }%
              \fi
            }%
          }%
          \ifvoid\ParallelLBox
            \ifvoid\ParallelRBox
              \global\ParallelBoolVar=\z@
            \fi
          \fi%
        \fi%
      \ifnum\ParallelBoolVar=\@ne
        \penalty\interlinepenalty
      \repeat
      \par
    \endgroup
  \fi
}{%
  \ifnum\ParallelBoolVar=\@ne
    \par
    \begingroup
      \leftmargin=\z@
      \rightmargin=\z@
      \parskip=\z@skip
      \parindent=\z@
      \vbadness=10000 %
      \vfuzz=3ex %
      \splittopskip=\z@skip
      \loop
        \ifnum\ParallelBoolVar=\@ne
          \noindent
          \hbox to\textwidth{%
            \hskip\ParallelLeftMargin
            \hbox to\ParallelTextWidth{%
              \ifvoid\ParallelLBox
                \hskip\ParallelLWidth
              \else
                \pcp@SetCurrent{Left}%
                \ParallelWhichBox=\z@
                \vbox{%
                  \setbox\ParallelBoxVar
                      =\vsplit\ParallelLBox to\dp\strutbox
                  \unvbox\ParallelBoxVar
                }%
              \fi
              \strut
              \ifnum\ParallelBoolMid=\@ne
                \hskip\ParallelMainMidSkip
                \begingroup
                  \pcp@RuleBetweenColor
                  \vrule
                \endgroup
              \else
                \hss
              \fi
              \hss
              \ifvoid\ParallelRBox
                \hskip\ParallelRWidth
              \else
                \pcp@SetCurrent{Right}%
                \ParallelWhichBox=\@ne
                \vbox{%
                  \setbox\ParallelBoxVar
                      =\vsplit\ParallelRBox to\dp\strutbox
                  \unvbox\ParallelBoxVar
                }%
              \fi
            }%
          }%
          \ifvoid\ParallelLBox
            \ifvoid\ParallelRBox
              \global\ParallelBoolVar=\z@
            \fi
          \fi%
        \fi%
      \ifnum\ParallelBoolVar=\@ne
        \penalty\interlinepenalty
      \repeat
      \par
    \endgroup
    \pcp@SetCurrent{}%
  \fi
}
%    \end{macrocode}
%    \end{macro}
%    \begin{macro}{\pcp@RuleBetweenColorDefault}
%    \begin{macrocode}
\def\pcp@RuleBetweenColorDefault{%
  \normalcolor
}
%    \end{macrocode}
%    \end{macro}
%    \begin{macro}{\pcp@RuleBetweenColor}
%    \begin{macrocode}
\let\pcp@RuleBetweenColor\pcp@RuleBetweenColorDefault
%    \end{macrocode}
%    \end{macro}
%    \begin{macrocode}
\RequirePackage{keyval}
\define@key{parallel}{rulebetweencolor}{%
  \edef\pcp@temp{#1}%
  \ifx\pcp@temp\@empty
    \let\pcp@RuleBetweenColor\pcp@RuleBetweenColorDefault
  \else
    \edef\pcp@temp{%
      \noexpand\@ifnextchar[{%
        \def\noexpand\pcp@RuleBetweenColor{%
          \noexpand\color\pcp@temp
        }%
        \noexpand\pcp@GobbleNil
      }{%
        \def\noexpand\pcp@RuleBetweenColor{%
          \noexpand\color{\pcp@temp}%
        }%
        \noexpand\pcp@GobbleNil
      }%
      \pcp@temp\noexpand\@nil
    }%
    \pcp@temp
  \fi
}
%    \end{macrocode}
%    \begin{macro}{\pcp@GobbleNil}
%    \begin{macrocode}
\long\def\pcp@GobbleNil#1\@nil{}
%    \end{macrocode}
%    \end{macro}
%
%    \begin{macrocode}
%</package>
%    \end{macrocode}
%
% \section{Test}
%
%    The test file is a modified version of the file that
%    Alexander Hirsch has posted in \xnewsgroup{de.comp.text.tex}:
%    \URL{``\link{\texttt{parallel.sty} und farbiger Text}''}^^A
%    {http://groups.google.com/group/de.comp.text.tex/msg/6a759cf33bb071a5}
%    \begin{macrocode}
%<*test1>
\AtEndDocument{%
  \typeout{}%
  \typeout{**************************************}%
  \typeout{*** \space Check the PDF file manually! \space ***}%
  \typeout{**************************************}%
  \typeout{}%
}
\documentclass{article}
\usepackage{xcolor}
\usepackage{pdfcolparallel}[2016/05/16]

\begin{document}
  \color{green}%
  Green%
  \begin{Parallel}{0.47\textwidth}{0.47\textwidth}%
    \ParallelLText{%
      \textcolor{red}{%
        Ein Absatz, der sich ueber zwei Zeilen erstrecken soll. %
        Ein Absatz, der sich ueber zwei Zeilen erstrecken soll.%
      }%
    }%
    \ParallelRText{%
      \textcolor{blue}{%
        Ein Absatz, der sich ueber zwei Zeilen erstrecken soll. %
        Ein Absatz, der sich ueber zwei Zeilen erstrecken soll.%
      }%
    }%
    \ParallelPar
    \ParallelLText{%
      Default %
      \color{red}%
      Ein Absatz, der sich ueber zwei Zeilen erstrecken soll. %
      Ein Absatz, der sich ueber zwei Zeilen erstrecken soll.%
    }%
    \ParallelRText{%
      Default %
      \color{blue}%
      Ein Absatz, der sich ueber zwei Zeilen erstrecken soll. %
      Ein Absatz, der sich ueber zwei Zeilen erstrecken soll.%
    }%
    \ParallelPar
    \ParallelLText{%
      \begin{enumerate}%
      \item left text, left text, left text, left text, %
            left text, left text, left text, left text,%
      \item left text, left text, left text, left text, %
            left text, left text, left text, left text.%
      \end{enumerate}%
    }%
    \ParallelRText{%
      \begin{enumerate}%
      \item right text, right text, right text, right text, %
            right text, right text, right text, right text.%
      \item right text, right text, right text, right text, %
            right text, right text, right text, right text.%
      \end{enumerate}%
    }%
  \end{Parallel}%
  \begin{Parallel}[p]{\textwidth}{\textwidth}%
    \ParallelLText{%
      \textcolor{red}{%
        Ein Absatz, der sich ueber zwei Zeilen erstrecken soll. %
        Ein Absatz, der sich ueber zwei Zeilen erstrecken soll. %
        Foo bar bla bla bla.%
      }%
      \par
      Und noch ein Absatz.%
    }%
    \ParallelRText{%
      \textcolor{blue}{%
        Ein Absatz, der sich ueber zwei Zeilen erstrecken soll. %
        Ein Absatz, der sich ueber zwei Zeilen erstrecken soll. %
        Foo bar bla bla bla.%
      }%
    }%
  \end{Parallel}%
  \begin{Parallel}[p]{\textwidth}{\textwidth}%
    \ParallelLText{%
      \rule{1pt}{.98\textheight}\Huge g%
    }%
    \ParallelRText{%
      \rule{1pt}{.98\textheight}y%
    }%
  \end{Parallel}%
  Green%
\end{document}
%</test1>
%    \end{macrocode}
%
% \section{Installation}
%
% \subsection{Download}
%
% \paragraph{Package.} This package is available on
% CTAN\footnote{\url{http://ctan.org/pkg/pdfcolparallel}}:
% \begin{description}
% \item[\CTAN{macros/latex/contrib/oberdiek/pdfcolparallel.dtx}] The source file.
% \item[\CTAN{macros/latex/contrib/oberdiek/pdfcolparallel.pdf}] Documentation.
% \end{description}
%
%
% \paragraph{Bundle.} All the packages of the bundle `oberdiek'
% are also available in a TDS compliant ZIP archive. There
% the packages are already unpacked and the documentation files
% are generated. The files and directories obey the TDS standard.
% \begin{description}
% \item[\CTAN{install/macros/latex/contrib/oberdiek.tds.zip}]
% \end{description}
% \emph{TDS} refers to the standard ``A Directory Structure
% for \TeX\ Files'' (\CTAN{tds/tds.pdf}). Directories
% with \xfile{texmf} in their name are usually organized this way.
%
% \subsection{Bundle installation}
%
% \paragraph{Unpacking.} Unpack the \xfile{oberdiek.tds.zip} in the
% TDS tree (also known as \xfile{texmf} tree) of your choice.
% Example (linux):
% \begin{quote}
%   |unzip oberdiek.tds.zip -d ~/texmf|
% \end{quote}
%
% \paragraph{Script installation.}
% Check the directory \xfile{TDS:scripts/oberdiek/} for
% scripts that need further installation steps.
% Package \xpackage{attachfile2} comes with the Perl script
% \xfile{pdfatfi.pl} that should be installed in such a way
% that it can be called as \texttt{pdfatfi}.
% Example (linux):
% \begin{quote}
%   |chmod +x scripts/oberdiek/pdfatfi.pl|\\
%   |cp scripts/oberdiek/pdfatfi.pl /usr/local/bin/|
% \end{quote}
%
% \subsection{Package installation}
%
% \paragraph{Unpacking.} The \xfile{.dtx} file is a self-extracting
% \docstrip\ archive. The files are extracted by running the
% \xfile{.dtx} through \plainTeX:
% \begin{quote}
%   \verb|tex pdfcolparallel.dtx|
% \end{quote}
%
% \paragraph{TDS.} Now the different files must be moved into
% the different directories in your installation TDS tree
% (also known as \xfile{texmf} tree):
% \begin{quote}
% \def\t{^^A
% \begin{tabular}{@{}>{\ttfamily}l@{ $\rightarrow$ }>{\ttfamily}l@{}}
%   pdfcolparallel.sty & tex/latex/oberdiek/pdfcolparallel.sty\\
%   pdfcolparallel.pdf & doc/latex/oberdiek/pdfcolparallel.pdf\\
%   test/pdfcolparallel-test1.tex & doc/latex/oberdiek/test/pdfcolparallel-test1.tex\\
%   pdfcolparallel.dtx & source/latex/oberdiek/pdfcolparallel.dtx\\
% \end{tabular}^^A
% }^^A
% \sbox0{\t}^^A
% \ifdim\wd0>\linewidth
%   \begingroup
%     \advance\linewidth by\leftmargin
%     \advance\linewidth by\rightmargin
%   \edef\x{\endgroup
%     \def\noexpand\lw{\the\linewidth}^^A
%   }\x
%   \def\lwbox{^^A
%     \leavevmode
%     \hbox to \linewidth{^^A
%       \kern-\leftmargin\relax
%       \hss
%       \usebox0
%       \hss
%       \kern-\rightmargin\relax
%     }^^A
%   }^^A
%   \ifdim\wd0>\lw
%     \sbox0{\small\t}^^A
%     \ifdim\wd0>\linewidth
%       \ifdim\wd0>\lw
%         \sbox0{\footnotesize\t}^^A
%         \ifdim\wd0>\linewidth
%           \ifdim\wd0>\lw
%             \sbox0{\scriptsize\t}^^A
%             \ifdim\wd0>\linewidth
%               \ifdim\wd0>\lw
%                 \sbox0{\tiny\t}^^A
%                 \ifdim\wd0>\linewidth
%                   \lwbox
%                 \else
%                   \usebox0
%                 \fi
%               \else
%                 \lwbox
%               \fi
%             \else
%               \usebox0
%             \fi
%           \else
%             \lwbox
%           \fi
%         \else
%           \usebox0
%         \fi
%       \else
%         \lwbox
%       \fi
%     \else
%       \usebox0
%     \fi
%   \else
%     \lwbox
%   \fi
% \else
%   \usebox0
% \fi
% \end{quote}
% If you have a \xfile{docstrip.cfg} that configures and enables \docstrip's
% TDS installing feature, then some files can already be in the right
% place, see the documentation of \docstrip.
%
% \subsection{Refresh file name databases}
%
% If your \TeX~distribution
% (\teTeX, \mikTeX, \dots) relies on file name databases, you must refresh
% these. For example, \teTeX\ users run \verb|texhash| or
% \verb|mktexlsr|.
%
% \subsection{Some details for the interested}
%
% \paragraph{Attached source.}
%
% The PDF documentation on CTAN also includes the
% \xfile{.dtx} source file. It can be extracted by
% AcrobatReader 6 or higher. Another option is \textsf{pdftk},
% e.g. unpack the file into the current directory:
% \begin{quote}
%   \verb|pdftk pdfcolparallel.pdf unpack_files output .|
% \end{quote}
%
% \paragraph{Unpacking with \LaTeX.}
% The \xfile{.dtx} chooses its action depending on the format:
% \begin{description}
% \item[\plainTeX:] Run \docstrip\ and extract the files.
% \item[\LaTeX:] Generate the documentation.
% \end{description}
% If you insist on using \LaTeX\ for \docstrip\ (really,
% \docstrip\ does not need \LaTeX), then inform the autodetect routine
% about your intention:
% \begin{quote}
%   \verb|latex \let\install=y% \iffalse meta-comment
%
% File: pdfcolparallel.dtx
% Version: 2016/05/16 v1.4
% Info: Color stacks support for parallel
%
% Copyright (C) 2007, 2008, 2010 by
%    Heiko Oberdiek <heiko.oberdiek at googlemail.com>
%    2016
%    https://github.com/ho-tex/oberdiek/issues
%
% This work may be distributed and/or modified under the
% conditions of the LaTeX Project Public License, either
% version 1.3c of this license or (at your option) any later
% version. This version of this license is in
%    http://www.latex-project.org/lppl/lppl-1-3c.txt
% and the latest version of this license is in
%    http://www.latex-project.org/lppl.txt
% and version 1.3 or later is part of all distributions of
% LaTeX version 2005/12/01 or later.
%
% This work has the LPPL maintenance status "maintained".
%
% This Current Maintainer of this work is Heiko Oberdiek.
%
% This work consists of the main source file pdfcolparallel.dtx
% and the derived files
%    pdfcolparallel.sty, pdfcolparallel.pdf, pdfcolparallel.ins,
%    pdfcolparallel.drv, pdfcolparallel-test1.tex.
%
% Distribution:
%    CTAN:macros/latex/contrib/oberdiek/pdfcolparallel.dtx
%    CTAN:macros/latex/contrib/oberdiek/pdfcolparallel.pdf
%
% Unpacking:
%    (a) If pdfcolparallel.ins is present:
%           tex pdfcolparallel.ins
%    (b) Without pdfcolparallel.ins:
%           tex pdfcolparallel.dtx
%    (c) If you insist on using LaTeX
%           latex \let\install=y\input{pdfcolparallel.dtx}
%        (quote the arguments according to the demands of your shell)
%
% Documentation:
%    (a) If pdfcolparallel.drv is present:
%           latex pdfcolparallel.drv
%    (b) Without pdfcolparallel.drv:
%           latex pdfcolparallel.dtx; ...
%    The class ltxdoc loads the configuration file ltxdoc.cfg
%    if available. Here you can specify further options, e.g.
%    use A4 as paper format:
%       \PassOptionsToClass{a4paper}{article}
%
%    Programm calls to get the documentation (example):
%       pdflatex pdfcolparallel.dtx
%       makeindex -s gind.ist pdfcolparallel.idx
%       pdflatex pdfcolparallel.dtx
%       makeindex -s gind.ist pdfcolparallel.idx
%       pdflatex pdfcolparallel.dtx
%
% Installation:
%    TDS:tex/latex/oberdiek/pdfcolparallel.sty
%    TDS:doc/latex/oberdiek/pdfcolparallel.pdf
%    TDS:doc/latex/oberdiek/test/pdfcolparallel-test1.tex
%    TDS:source/latex/oberdiek/pdfcolparallel.dtx
%
%<*ignore>
\begingroup
  \catcode123=1 %
  \catcode125=2 %
  \def\x{LaTeX2e}%
\expandafter\endgroup
\ifcase 0\ifx\install y1\fi\expandafter
         \ifx\csname processbatchFile\endcsname\relax\else1\fi
         \ifx\fmtname\x\else 1\fi\relax
\else\csname fi\endcsname
%</ignore>
%<*install>
\input docstrip.tex
\Msg{************************************************************************}
\Msg{* Installation}
\Msg{* Package: pdfcolparallel 2016/05/16 v1.4 Color stacks support for parallel (HO)}
\Msg{************************************************************************}

\keepsilent
\askforoverwritefalse

\let\MetaPrefix\relax
\preamble

This is a generated file.

Project: pdfcolparallel
Version: 2016/05/16 v1.4

Copyright (C) 2007, 2008, 2010 by
   Heiko Oberdiek <heiko.oberdiek at googlemail.com>

This work may be distributed and/or modified under the
conditions of the LaTeX Project Public License, either
version 1.3c of this license or (at your option) any later
version. This version of this license is in
   http://www.latex-project.org/lppl/lppl-1-3c.txt
and the latest version of this license is in
   http://www.latex-project.org/lppl.txt
and version 1.3 or later is part of all distributions of
LaTeX version 2005/12/01 or later.

This work has the LPPL maintenance status "maintained".

This Current Maintainer of this work is Heiko Oberdiek.

This work consists of the main source file pdfcolparallel.dtx
and the derived files
   pdfcolparallel.sty, pdfcolparallel.pdf, pdfcolparallel.ins,
   pdfcolparallel.drv, pdfcolparallel-test1.tex.

\endpreamble
\let\MetaPrefix\DoubleperCent

\generate{%
  \file{pdfcolparallel.ins}{\from{pdfcolparallel.dtx}{install}}%
  \file{pdfcolparallel.drv}{\from{pdfcolparallel.dtx}{driver}}%
  \usedir{tex/latex/oberdiek}%
  \file{pdfcolparallel.sty}{\from{pdfcolparallel.dtx}{package}}%
  \usedir{doc/latex/oberdiek/test}%
  \file{pdfcolparallel-test1.tex}{\from{pdfcolparallel.dtx}{test1}}%
  \nopreamble
  \nopostamble
  \usedir{source/latex/oberdiek/catalogue}%
  \file{pdfcolparallel.xml}{\from{pdfcolparallel.dtx}{catalogue}}%
}

\catcode32=13\relax% active space
\let =\space%
\Msg{************************************************************************}
\Msg{*}
\Msg{* To finish the installation you have to move the following}
\Msg{* file into a directory searched by TeX:}
\Msg{*}
\Msg{*     pdfcolparallel.sty}
\Msg{*}
\Msg{* To produce the documentation run the file `pdfcolparallel.drv'}
\Msg{* through LaTeX.}
\Msg{*}
\Msg{* Happy TeXing!}
\Msg{*}
\Msg{************************************************************************}

\endbatchfile
%</install>
%<*ignore>
\fi
%</ignore>
%<*driver>
\NeedsTeXFormat{LaTeX2e}
\ProvidesFile{pdfcolparallel.drv}%
  [2016/05/16 v1.4 Color stacks support for parallel (HO)]%
\documentclass{ltxdoc}
\usepackage{holtxdoc}[2011/11/22]
\begin{document}
  \DocInput{pdfcolparallel.dtx}%
\end{document}
%</driver>
% \fi
%
%
% \CharacterTable
%  {Upper-case    \A\B\C\D\E\F\G\H\I\J\K\L\M\N\O\P\Q\R\S\T\U\V\W\X\Y\Z
%   Lower-case    \a\b\c\d\e\f\g\h\i\j\k\l\m\n\o\p\q\r\s\t\u\v\w\x\y\z
%   Digits        \0\1\2\3\4\5\6\7\8\9
%   Exclamation   \!     Double quote  \"     Hash (number) \#
%   Dollar        \$     Percent       \%     Ampersand     \&
%   Acute accent  \'     Left paren    \(     Right paren   \)
%   Asterisk      \*     Plus          \+     Comma         \,
%   Minus         \-     Point         \.     Solidus       \/
%   Colon         \:     Semicolon     \;     Less than     \<
%   Equals        \=     Greater than  \>     Question mark \?
%   Commercial at \@     Left bracket  \[     Backslash     \\
%   Right bracket \]     Circumflex    \^     Underscore    \_
%   Grave accent  \`     Left brace    \{     Vertical bar  \|
%   Right brace   \}     Tilde         \~}
%
% \GetFileInfo{pdfcolparallel.drv}
%
% \title{The \xpackage{pdfcolparallel} package}
% \date{2016/05/16 v1.4}
% \author{Heiko Oberdiek\thanks
% {Please report any issues at https://github.com/ho-tex/oberdiek/issues}\\
% \xemail{heiko.oberdiek at googlemail.com}}
%
% \maketitle
%
% \begin{abstract}
% This packages fixes bugs in \xpackage{parallel} and
% improves color support by using several color stacks
% that are provided by \pdfTeX\ since version 1.40.
% \end{abstract}
%
% \tableofcontents
%
% \section{Usage}
%
% \begin{quote}
% |\usepackage{pdfcolparallel}|
% \end{quote}
% The package \xpackage{pdfcolparallel} loads package \xpackage{parallel}
% \cite{parallel} and redefines some macros to fix bugs.
%
% If color stacks are available then package \xpackage{parallel}
% is further patched to support them.
%
% \subsection{Option \xoption{rulebetweencolor}}
%
% Package \xpackage{pdfcolparallel} also fixes the color for the
% rule between columns.
% Default color is \cs{normalcolor}. But this can be changed by using
% option \xoption{rulebetweencolor} for |\setkeys{parallel}|
% (see package \xpackage{keyval}). The option takes a color specification
% as value. If the value is empty, then the default (\cs{normalcolor})
% is used.
% Examples:
% \begin{quote}
%   |\setkeys{parallel}{rulebetweencolor=blue}|,\\
%   |\setkeys{parallel}{rulebetweencolor={red}}|,\\
%   |\setkeys{parallel}{rulebetweencolor={}}|,
%     \textit{\% \cs{normalcolor} is used}\\
%   |\setkeys{parallel}{rulebetweencolor=[rgb]{1,0,.5}}|
% \end{quote}
%
% \subsection{Future}
%
% If there will be a new version of package \xpackage{parallel}
% that adds support for color stacks, then this package may become
% obsolete.
%
% \StopEventually{
% }
%
% \section{Implementation}
%
% \subsection{Identification}
%
%    \begin{macrocode}
%<*package>
\NeedsTeXFormat{LaTeX2e}
\ProvidesPackage{pdfcolparallel}%
  [2016/05/16 v1.4 Color stacks support for parallel (HO)]%
%    \end{macrocode}
%
% \subsection{Load and fix package \xpackage{parallel}}
%
%    Package \xpackage{parallel} is loaded. Before options of package
%    \xpackage{pdfcolparallel} are passed to package \xpackage{parallel}.
%    \begin{macrocode}
\DeclareOption*{%
  \PassoptionsToPackage{\CurrentOption}{parallel}%
}
\ProcessOptions\relax
\RequirePackage{parallel}[2003/04/13]
%    \end{macrocode}
%
%    \begin{macrocode}
\RequirePackage{infwarerr}[2007/09/09]
%    \end{macrocode}
%
%    \begin{macro}{\pcp@ColorPatch}
%    \begin{macrocode}
\begingroup\expandafter\expandafter\expandafter\endgroup
\expandafter\ifx\csname currentgrouplevel\endcsname\relax
  \def\pcp@ColorPatch{}%
\else
  \def\pcp@ColorPatch{%
    \@ifundefined{set@color}{%
      \gdef\pcp@ColorPatch{}%
    }{%
      \gdef\pcp@ColorPatch{%
        \gdef\pcp@ColorResets{}%
        \bgroup
        \aftergroup\pcp@ColorResets
        \aftergroup\egroup
        \let\pcp@OrgSetColor\set@color
        \let\set@color\pcp@SetColor
        \edef\pcp@GroupLevel{\the\currentgrouplevel}%
      }%
    }%
    \pcp@ColorPatch
  }%
%    \end{macrocode}
%    \end{macro}
%    \begin{macro}{\pcp@SetColor}
%    \begin{macrocode}
  \def\pcp@SetColor{%
    \ifnum\pcp@GroupLevel=\currentgrouplevel
      \let\pcp@OrgAfterGroup\aftergroup
      \def\aftergroup{%
        \g@addto@macro\pcp@ColorResets
      }%
      \pcp@OrgSetColor
      \let\aftergroup\pcp@OrgAfterGroup
    \else
      \pcp@OrgSetColor
    \fi
  }%
\fi
%    \end{macrocode}
%    \end{macro}
%
%    \begin{macro}{\pcp@CmdCheckRedef}
%    \begin{macrocode}
\def\pcp@CmdCheckRedef#1{%
  \begingroup
    \def\pcp@cmd{#1}%
    \afterassignment\pcp@CmdDo
    \long\def\reserved@a
}
\def\pcp@CmdDo{%
    \expandafter\ifx\pcp@cmd\reserved@a
    \else
      \edef\x*{\expandafter\string\pcp@cmd}%
      \@PackageWarningNoLine{pdfcolparallel}{%
        Command \x* has changed.\MessageBreak
        Supported versions of package `parallel':\MessageBreak
        \space\space 2003/04/13\MessageBreak
        The redefinition of \x* may\MessageBreak
        not behave correctly depending on the changes%
      }%
    \fi
  \expandafter\endgroup
  \expandafter\def\pcp@cmd
}
%    \end{macrocode}
%    \end{macro}
%
%    \begin{macrocode}
\def\pcp@SwitchStack#1#2{}
%    \end{macrocode}
%    \begin{macrocode}
\def\pcp@SetCurrent#1{}
%    \end{macrocode}
%
%    \begin{macro}{\ParallelLText}
%    \begin{macrocode}
\pcp@CmdCheckRedef\ParallelLText{%
  \everypar{}%
  \@restorepar
  \begingroup
    \hbadness=3000 %
    \let\footnote=\ParallelLFootnote
    \ParallelWhichBox=0 %
    \global\setbox\ParallelLBox=\vbox\bgroup
      \hsize=\ParallelLWidth
      \aftergroup\ParallelAfterText
      \begingroup
        \afterassignment\ParallelCheckOpenBrace
        \let\x=%
}{%
  \everypar{}%
  \@restorepar
  \@nobreakfalse
  \begingroup
    \hbadness=3000 %
    \let\footnote=\ParallelLFootnote
    \ParallelWhichBox=0 %
    \global\setbox\ParallelLBox=\vbox\bgroup
      \hsize=\ParallelLWidth
      \linewidth=\ParallelLWidth
      \pcp@SwitchStack{Left}\ParallelLBox
      \aftergroup\ParallelAfterText
      \pcp@ColorPatch
      \begingroup
        \afterassignment\ParallelCheckOpenBrace
        \let\x=%
}
%    \end{macrocode}
%    \end{macro}
%
%    \begin{macro}{\ParallelRText}
%    \begin{macrocode}
\pcp@CmdCheckRedef\ParallelRText{%
  \everypar{}%
  \@restorepar
  \begingroup
    \hbadness=3000 %
    \ifnum\ParallelFNMode=\@ne
      \let\footnote=\ParallelRFootnote
    \else
      \let\footnote=\ParallelLFootnote
    \fi
    \ParallelWhichBox=\@ne
    \global\setbox\ParallelRBox=\vbox\bgroup
      \hsize=\ParallelRWidth
      \aftergroup\ParallelAfterText
      \begingroup
        \afterassignment\ParallelCheckOpenBrace
        \let\x=%
}{%
  \everypar{}%
  \@restorepar
  \@nobreakfalse
  \begingroup
    \hbadness=3000 %
    \ifnum\ParallelFNMode=\@ne
      \let\footnote=\ParallelRFootnote
    \else
      \let\footnote=\ParallelLFootnote
    \fi
    \ParallelWhichBox=\@ne
    \global\setbox\ParallelRBox=\vbox\bgroup
      \hsize=\ParallelRWidth
      \linewidth=\ParallelRWidth
      \pcp@SwitchStack{Right}\ParallelRBox
      \aftergroup\ParallelAfterText
      \pcp@ColorPatch
      \begingroup
        \afterassignment\ParallelCheckOpenBrace
        \let\x=%
}
%    \end{macrocode}
%    \end{macro}
%
%    \begin{macro}{\ParallelParTwoPages}
%    \begin{macrocode}
\pcp@CmdCheckRedef\ParallelParTwoPages{%
  \ifnum\ParallelBoolVar=\@ne
    \par
    \begingroup
      \global\ParallelWhichBox=\@ne
      \newpage
      \vbadness=10000 %
      \vfuzz=3ex %
      \splittopskip=\z@skip
      \loop%
        \ifnum\ParallelBoolVar=\@ne%
          \ifnum\ParallelWhichBox=\@ne
            \ifvoid\ParallelLBox
              \mbox{} %
              \newpage
            \else
              \global\ParallelWhichBox=\z@
            \fi
          \else
            \ifvoid\ParallelRBox
              \mbox{} %
              \newpage
            \else
              \global\ParallelWhichBox=\@ne
            \fi
          \fi
          \ifnum\ParallelWhichBox=\z@
            \ifodd\thepage
              \mbox{} %
              \newpage
            \fi
            \hbox to\textwidth{%
              \vbox{\vsplit\ParallelLBox to.98\textheight}%
            }%
          \else
            \ifodd\thepage\relax
            \else
              \mbox{} %
              \newpage
            \fi
            \hbox to\textwidth{%
              \vbox{\vsplit\ParallelRBox to.98\textheight}%
            }%
          \fi
          \vspace*{\fill}%
          \newpage
        \fi
        \ifvoid\ParallelLBox
          \ifvoid\ParallelRBox
            \global\ParallelBoolVar=\z@
          \fi
        \fi
      \ifnum\ParallelBoolVar=\@ne
      \repeat
      \par
    \endgroup
  \fi
}{%
%    \end{macrocode}
%    Additional fixes:
%    \begin{itemize}
%    \item Unnecessary white space removed.
%    \item |\ifodd\thepage| changed to |\ifodd\value{page}|.
%    \end{itemize}
%    \begin{macrocode}
  \ifnum\ParallelBoolVar=\@ne
    \par
    \begingroup
      \global\ParallelWhichBox=\@ne
      \newpage
      \vbadness=10000 %
      \vfuzz=3ex %
      \splittopskip=\z@skip
      \loop%
        \ifnum\ParallelBoolVar=\@ne%
          \ifnum\ParallelWhichBox=\@ne
            \ifvoid\ParallelLBox
              \mbox{}%
              \newpage
            \else
              \global\ParallelWhichBox=\z@
            \fi
          \else
            \ifvoid\ParallelRBox
              \null
              \newpage
            \else
              \global\ParallelWhichBox=\@ne
            \fi
          \fi
          \ifnum\ParallelWhichBox=\z@
            \ifodd\value{page}%
              \null
              \newpage
            \fi
            \hbox to\textwidth{%
              \pcp@SetCurrent{Left}%
              \setbox\z@=\vsplit\ParallelLBox to.98\textheight
              \vbox to.98\textheight{%
                \@texttop
                \unvbox\z@
                \@textbottom
              }%
            }%
          \else
            \ifodd\value{page}%
            \else
              \mbox{}%
              \newpage
            \fi
            \hbox to\textwidth{%
              \pcp@SetCurrent{Right}%
              \setbox\z@=\vsplit\ParallelRBox to.98\textheight
              \vbox to.98\textheight{%
                \@texttop
                \unvbox\z@
                \@textbottom
              }%
            }%
          \fi
          \vspace*{\fill}%
          \newpage
        \fi
        \ifvoid\ParallelLBox
          \ifvoid\ParallelRBox
            \global\ParallelBoolVar=\z@
          \fi
        \fi
      \ifnum\ParallelBoolVar=\@ne
      \repeat
      \par
    \endgroup
    \pcp@SetCurrent{}%
  \fi
}
%    \end{macrocode}
%    \end{macro}
%
% \subsection{Color stack support}
%
%    \begin{macrocode}
\RequirePackage{pdfcol}[2007/12/12]
\ifpdfcolAvailable
\else
  \PackageInfo{pdfcolparallel}{%
    Loading aborted, because color stacks are not available%
  }%
  \expandafter\endinput
\fi
%    \end{macrocode}
%
%    \begin{macrocode}
\pdfcolInitStack{pcp@Left}
\pdfcolInitStack{pcp@Right}
%    \end{macrocode}
%    \begin{macro}{\pcp@Box}
%    \begin{macrocode}
\newbox\pcp@Box
%    \end{macrocode}
%    \end{macro}
%    \begin{macro}{\pcp@SwitchStack}
%    \begin{macrocode}
\def\pcp@SwitchStack#1#2{%
  \pdfcolSwitchStack{pcp@#1}%
  \global\setbox\pcp@Box=\vbox to 0pt{%
    \pdfcolSetCurrentColor
  }%
  \aftergroup\pcp@FixBox
  \aftergroup#2%
}
%    \end{macrocode}
%    \end{macro}
%    \begin{macro}{\pcp@FixBox}
%    \begin{macrocode}
\def\pcp@FixBox#1{%
  \global\setbox#1=\vbox{%
    \unvbox\pcp@Box
    \unvbox#1%
  }%
}
%    \end{macrocode}
%    \end{macro}
%    \begin{macro}{\pcp@SetCurrent}
%    \begin{macrocode}
\def\pcp@SetCurrent#1{%
  \ifx\\#1\\%
    \pdfcolSetCurrent{}%
  \else
    \pdfcolSetCurrent{pcp@#1}%
  \fi
}
%    \end{macrocode}
%    \end{macro}
%
% \subsection{Redefinitions}
%
%    \begin{macro}{\ParallelParOnePage}
%    \begin{macrocode}
\pcp@CmdCheckRedef\ParallelParOnePage{%
  \ifnum\ParallelBoolVar=\@ne
    \par
    \begingroup
      \leftmargin=\z@
      \rightmargin=\z@
      \parskip=\z@skip
      \parindent=\z@
      \vbadness=10000 %
      \vfuzz=3ex %
      \splittopskip=\z@skip
      \loop
        \ifnum\ParallelBoolVar=\@ne
          \noindent
          \hbox to\textwidth{%
            \hskip\ParallelLeftMargin
            \hbox to\ParallelTextWidth{%
              \ifvoid\ParallelLBox
                \hskip\ParallelLWidth
              \else
                \ParallelWhichBox=\z@
                \vbox{%
                  \setbox\ParallelBoxVar
                      =\vsplit\ParallelLBox to\dp\strutbox
                  \unvbox\ParallelBoxVar
                }%
              \fi
              \strut
              \ifnum\ParallelBoolMid=\@ne
                \hskip\ParallelMainMidSkip
                \vrule
              \else
                \hss
              \fi
              \hss
              \ifvoid\ParallelRBox
                \hskip\ParallelRWidth
              \else
                \ParallelWhichBox=\@ne
                \vbox{%
                  \setbox\ParallelBoxVar
                      =\vsplit\ParallelRBox to\dp\strutbox
                  \unvbox\ParallelBoxVar
                }%
              \fi
            }%
          }%
          \ifvoid\ParallelLBox
            \ifvoid\ParallelRBox
              \global\ParallelBoolVar=\z@
            \fi
          \fi%
        \fi%
      \ifnum\ParallelBoolVar=\@ne
        \penalty\interlinepenalty
      \repeat
      \par
    \endgroup
  \fi
}{%
  \ifnum\ParallelBoolVar=\@ne
    \par
    \begingroup
      \leftmargin=\z@
      \rightmargin=\z@
      \parskip=\z@skip
      \parindent=\z@
      \vbadness=10000 %
      \vfuzz=3ex %
      \splittopskip=\z@skip
      \loop
        \ifnum\ParallelBoolVar=\@ne
          \noindent
          \hbox to\textwidth{%
            \hskip\ParallelLeftMargin
            \hbox to\ParallelTextWidth{%
              \ifvoid\ParallelLBox
                \hskip\ParallelLWidth
              \else
                \pcp@SetCurrent{Left}%
                \ParallelWhichBox=\z@
                \vbox{%
                  \setbox\ParallelBoxVar
                      =\vsplit\ParallelLBox to\dp\strutbox
                  \unvbox\ParallelBoxVar
                }%
              \fi
              \strut
              \ifnum\ParallelBoolMid=\@ne
                \hskip\ParallelMainMidSkip
                \begingroup
                  \pcp@RuleBetweenColor
                  \vrule
                \endgroup
              \else
                \hss
              \fi
              \hss
              \ifvoid\ParallelRBox
                \hskip\ParallelRWidth
              \else
                \pcp@SetCurrent{Right}%
                \ParallelWhichBox=\@ne
                \vbox{%
                  \setbox\ParallelBoxVar
                      =\vsplit\ParallelRBox to\dp\strutbox
                  \unvbox\ParallelBoxVar
                }%
              \fi
            }%
          }%
          \ifvoid\ParallelLBox
            \ifvoid\ParallelRBox
              \global\ParallelBoolVar=\z@
            \fi
          \fi%
        \fi%
      \ifnum\ParallelBoolVar=\@ne
        \penalty\interlinepenalty
      \repeat
      \par
    \endgroup
    \pcp@SetCurrent{}%
  \fi
}
%    \end{macrocode}
%    \end{macro}
%    \begin{macro}{\pcp@RuleBetweenColorDefault}
%    \begin{macrocode}
\def\pcp@RuleBetweenColorDefault{%
  \normalcolor
}
%    \end{macrocode}
%    \end{macro}
%    \begin{macro}{\pcp@RuleBetweenColor}
%    \begin{macrocode}
\let\pcp@RuleBetweenColor\pcp@RuleBetweenColorDefault
%    \end{macrocode}
%    \end{macro}
%    \begin{macrocode}
\RequirePackage{keyval}
\define@key{parallel}{rulebetweencolor}{%
  \edef\pcp@temp{#1}%
  \ifx\pcp@temp\@empty
    \let\pcp@RuleBetweenColor\pcp@RuleBetweenColorDefault
  \else
    \edef\pcp@temp{%
      \noexpand\@ifnextchar[{%
        \def\noexpand\pcp@RuleBetweenColor{%
          \noexpand\color\pcp@temp
        }%
        \noexpand\pcp@GobbleNil
      }{%
        \def\noexpand\pcp@RuleBetweenColor{%
          \noexpand\color{\pcp@temp}%
        }%
        \noexpand\pcp@GobbleNil
      }%
      \pcp@temp\noexpand\@nil
    }%
    \pcp@temp
  \fi
}
%    \end{macrocode}
%    \begin{macro}{\pcp@GobbleNil}
%    \begin{macrocode}
\long\def\pcp@GobbleNil#1\@nil{}
%    \end{macrocode}
%    \end{macro}
%
%    \begin{macrocode}
%</package>
%    \end{macrocode}
%
% \section{Test}
%
%    The test file is a modified version of the file that
%    Alexander Hirsch has posted in \xnewsgroup{de.comp.text.tex}:
%    \URL{``\link{\texttt{parallel.sty} und farbiger Text}''}^^A
%    {http://groups.google.com/group/de.comp.text.tex/msg/6a759cf33bb071a5}
%    \begin{macrocode}
%<*test1>
\AtEndDocument{%
  \typeout{}%
  \typeout{**************************************}%
  \typeout{*** \space Check the PDF file manually! \space ***}%
  \typeout{**************************************}%
  \typeout{}%
}
\documentclass{article}
\usepackage{xcolor}
\usepackage{pdfcolparallel}[2016/05/16]

\begin{document}
  \color{green}%
  Green%
  \begin{Parallel}{0.47\textwidth}{0.47\textwidth}%
    \ParallelLText{%
      \textcolor{red}{%
        Ein Absatz, der sich ueber zwei Zeilen erstrecken soll. %
        Ein Absatz, der sich ueber zwei Zeilen erstrecken soll.%
      }%
    }%
    \ParallelRText{%
      \textcolor{blue}{%
        Ein Absatz, der sich ueber zwei Zeilen erstrecken soll. %
        Ein Absatz, der sich ueber zwei Zeilen erstrecken soll.%
      }%
    }%
    \ParallelPar
    \ParallelLText{%
      Default %
      \color{red}%
      Ein Absatz, der sich ueber zwei Zeilen erstrecken soll. %
      Ein Absatz, der sich ueber zwei Zeilen erstrecken soll.%
    }%
    \ParallelRText{%
      Default %
      \color{blue}%
      Ein Absatz, der sich ueber zwei Zeilen erstrecken soll. %
      Ein Absatz, der sich ueber zwei Zeilen erstrecken soll.%
    }%
    \ParallelPar
    \ParallelLText{%
      \begin{enumerate}%
      \item left text, left text, left text, left text, %
            left text, left text, left text, left text,%
      \item left text, left text, left text, left text, %
            left text, left text, left text, left text.%
      \end{enumerate}%
    }%
    \ParallelRText{%
      \begin{enumerate}%
      \item right text, right text, right text, right text, %
            right text, right text, right text, right text.%
      \item right text, right text, right text, right text, %
            right text, right text, right text, right text.%
      \end{enumerate}%
    }%
  \end{Parallel}%
  \begin{Parallel}[p]{\textwidth}{\textwidth}%
    \ParallelLText{%
      \textcolor{red}{%
        Ein Absatz, der sich ueber zwei Zeilen erstrecken soll. %
        Ein Absatz, der sich ueber zwei Zeilen erstrecken soll. %
        Foo bar bla bla bla.%
      }%
      \par
      Und noch ein Absatz.%
    }%
    \ParallelRText{%
      \textcolor{blue}{%
        Ein Absatz, der sich ueber zwei Zeilen erstrecken soll. %
        Ein Absatz, der sich ueber zwei Zeilen erstrecken soll. %
        Foo bar bla bla bla.%
      }%
    }%
  \end{Parallel}%
  \begin{Parallel}[p]{\textwidth}{\textwidth}%
    \ParallelLText{%
      \rule{1pt}{.98\textheight}\Huge g%
    }%
    \ParallelRText{%
      \rule{1pt}{.98\textheight}y%
    }%
  \end{Parallel}%
  Green%
\end{document}
%</test1>
%    \end{macrocode}
%
% \section{Installation}
%
% \subsection{Download}
%
% \paragraph{Package.} This package is available on
% CTAN\footnote{\url{http://ctan.org/pkg/pdfcolparallel}}:
% \begin{description}
% \item[\CTAN{macros/latex/contrib/oberdiek/pdfcolparallel.dtx}] The source file.
% \item[\CTAN{macros/latex/contrib/oberdiek/pdfcolparallel.pdf}] Documentation.
% \end{description}
%
%
% \paragraph{Bundle.} All the packages of the bundle `oberdiek'
% are also available in a TDS compliant ZIP archive. There
% the packages are already unpacked and the documentation files
% are generated. The files and directories obey the TDS standard.
% \begin{description}
% \item[\CTAN{install/macros/latex/contrib/oberdiek.tds.zip}]
% \end{description}
% \emph{TDS} refers to the standard ``A Directory Structure
% for \TeX\ Files'' (\CTAN{tds/tds.pdf}). Directories
% with \xfile{texmf} in their name are usually organized this way.
%
% \subsection{Bundle installation}
%
% \paragraph{Unpacking.} Unpack the \xfile{oberdiek.tds.zip} in the
% TDS tree (also known as \xfile{texmf} tree) of your choice.
% Example (linux):
% \begin{quote}
%   |unzip oberdiek.tds.zip -d ~/texmf|
% \end{quote}
%
% \paragraph{Script installation.}
% Check the directory \xfile{TDS:scripts/oberdiek/} for
% scripts that need further installation steps.
% Package \xpackage{attachfile2} comes with the Perl script
% \xfile{pdfatfi.pl} that should be installed in such a way
% that it can be called as \texttt{pdfatfi}.
% Example (linux):
% \begin{quote}
%   |chmod +x scripts/oberdiek/pdfatfi.pl|\\
%   |cp scripts/oberdiek/pdfatfi.pl /usr/local/bin/|
% \end{quote}
%
% \subsection{Package installation}
%
% \paragraph{Unpacking.} The \xfile{.dtx} file is a self-extracting
% \docstrip\ archive. The files are extracted by running the
% \xfile{.dtx} through \plainTeX:
% \begin{quote}
%   \verb|tex pdfcolparallel.dtx|
% \end{quote}
%
% \paragraph{TDS.} Now the different files must be moved into
% the different directories in your installation TDS tree
% (also known as \xfile{texmf} tree):
% \begin{quote}
% \def\t{^^A
% \begin{tabular}{@{}>{\ttfamily}l@{ $\rightarrow$ }>{\ttfamily}l@{}}
%   pdfcolparallel.sty & tex/latex/oberdiek/pdfcolparallel.sty\\
%   pdfcolparallel.pdf & doc/latex/oberdiek/pdfcolparallel.pdf\\
%   test/pdfcolparallel-test1.tex & doc/latex/oberdiek/test/pdfcolparallel-test1.tex\\
%   pdfcolparallel.dtx & source/latex/oberdiek/pdfcolparallel.dtx\\
% \end{tabular}^^A
% }^^A
% \sbox0{\t}^^A
% \ifdim\wd0>\linewidth
%   \begingroup
%     \advance\linewidth by\leftmargin
%     \advance\linewidth by\rightmargin
%   \edef\x{\endgroup
%     \def\noexpand\lw{\the\linewidth}^^A
%   }\x
%   \def\lwbox{^^A
%     \leavevmode
%     \hbox to \linewidth{^^A
%       \kern-\leftmargin\relax
%       \hss
%       \usebox0
%       \hss
%       \kern-\rightmargin\relax
%     }^^A
%   }^^A
%   \ifdim\wd0>\lw
%     \sbox0{\small\t}^^A
%     \ifdim\wd0>\linewidth
%       \ifdim\wd0>\lw
%         \sbox0{\footnotesize\t}^^A
%         \ifdim\wd0>\linewidth
%           \ifdim\wd0>\lw
%             \sbox0{\scriptsize\t}^^A
%             \ifdim\wd0>\linewidth
%               \ifdim\wd0>\lw
%                 \sbox0{\tiny\t}^^A
%                 \ifdim\wd0>\linewidth
%                   \lwbox
%                 \else
%                   \usebox0
%                 \fi
%               \else
%                 \lwbox
%               \fi
%             \else
%               \usebox0
%             \fi
%           \else
%             \lwbox
%           \fi
%         \else
%           \usebox0
%         \fi
%       \else
%         \lwbox
%       \fi
%     \else
%       \usebox0
%     \fi
%   \else
%     \lwbox
%   \fi
% \else
%   \usebox0
% \fi
% \end{quote}
% If you have a \xfile{docstrip.cfg} that configures and enables \docstrip's
% TDS installing feature, then some files can already be in the right
% place, see the documentation of \docstrip.
%
% \subsection{Refresh file name databases}
%
% If your \TeX~distribution
% (\teTeX, \mikTeX, \dots) relies on file name databases, you must refresh
% these. For example, \teTeX\ users run \verb|texhash| or
% \verb|mktexlsr|.
%
% \subsection{Some details for the interested}
%
% \paragraph{Attached source.}
%
% The PDF documentation on CTAN also includes the
% \xfile{.dtx} source file. It can be extracted by
% AcrobatReader 6 or higher. Another option is \textsf{pdftk},
% e.g. unpack the file into the current directory:
% \begin{quote}
%   \verb|pdftk pdfcolparallel.pdf unpack_files output .|
% \end{quote}
%
% \paragraph{Unpacking with \LaTeX.}
% The \xfile{.dtx} chooses its action depending on the format:
% \begin{description}
% \item[\plainTeX:] Run \docstrip\ and extract the files.
% \item[\LaTeX:] Generate the documentation.
% \end{description}
% If you insist on using \LaTeX\ for \docstrip\ (really,
% \docstrip\ does not need \LaTeX), then inform the autodetect routine
% about your intention:
% \begin{quote}
%   \verb|latex \let\install=y\input{pdfcolparallel.dtx}|
% \end{quote}
% Do not forget to quote the argument according to the demands
% of your shell.
%
% \paragraph{Generating the documentation.}
% You can use both the \xfile{.dtx} or the \xfile{.drv} to generate
% the documentation. The process can be configured by the
% configuration file \xfile{ltxdoc.cfg}. For instance, put this
% line into this file, if you want to have A4 as paper format:
% \begin{quote}
%   \verb|\PassOptionsToClass{a4paper}{article}|
% \end{quote}
% An example follows how to generate the
% documentation with pdf\LaTeX:
% \begin{quote}
%\begin{verbatim}
%pdflatex pdfcolparallel.dtx
%makeindex -s gind.ist pdfcolparallel.idx
%pdflatex pdfcolparallel.dtx
%makeindex -s gind.ist pdfcolparallel.idx
%pdflatex pdfcolparallel.dtx
%\end{verbatim}
% \end{quote}
%
% \section{Catalogue}
%
% The following XML file can be used as source for the
% \href{http://mirror.ctan.org/help/Catalogue/catalogue.html}{\TeX\ Catalogue}.
% The elements \texttt{caption} and \texttt{description} are imported
% from the original XML file from the Catalogue.
% The name of the XML file in the Catalogue is \xfile{pdfcolparallel.xml}.
%    \begin{macrocode}
%<*catalogue>
<?xml version='1.0' encoding='us-ascii'?>
<!DOCTYPE entry SYSTEM 'catalogue.dtd'>
<entry datestamp='$Date$' modifier='$Author$' id='pdfcolparallel'>
  <name>pdfcolparallel</name>
  <caption>Fix colour problems in package 'parallel'.</caption>
  <authorref id='auth:oberdiek'/>
  <copyright owner='Heiko Oberdiek' year='2007,2008,2010'/>
  <license type='lppl1.3'/>
  <version number='1.4'/>
  <description>
    Since version 1.40 pdfTeX supports colour stacks.
    This package uses them to fix colour problems in
    package <xref refid='parallel'>parallel</xref>.
    <p/>
    The package is part of the <xref refid='oberdiek'>oberdiek</xref>
    bundle.
  </description>
  <documentation details='Package documentation'
      href='ctan:/macros/latex/contrib/oberdiek/pdfcolparallel.pdf'/>
  <ctan file='true' path='/macros/latex/contrib/oberdiek/pdfcolparallel.dtx'/>
  <miktex location='oberdiek'/>
  <texlive location='oberdiek'/>
  <install path='/macros/latex/contrib/oberdiek/oberdiek.tds.zip'/>
</entry>
%</catalogue>
%    \end{macrocode}
%
% \begin{thebibliography}{9}
%
% \bibitem{parallel}
%   Matthias Eckermann: \textit{The \xpackage{parallel}-package};
%   2003/04/13;\\
%   \CTAN{macros/latex/contrib/parallel/}.
%
% \bibitem{pdfcol}
%   Heiko Oberdiek: \textit{The \xpackage{pdfcol} package};
%   2007/09/09;\\
%   \CTAN{macros/latex/contrib/oberdiek/pdfcol.pdf}.
%
% \end{thebibliography}
%
% \begin{History}
%   \begin{Version}{2007/09/09 v1.0}
%   \item
%     First version.
%   \end{Version}
%   \begin{Version}{2007/12/12 v1.1}
%   \item
%     Adds patch for setting \cs{linewidth} to fix bug
%     in package \xpackage{parallel}.
%   \item
%     Package \xpackage{parallel} is also fixed if color
%     stacks are not available.
%   \item
%     Bug fix, switched stacks now initialized with current color.
%   \item
%     Fix for package \xpackage{parallel}: \cs{raggedbottom} is respected.
%   \end{Version}
%   \begin{Version}{2008/08/11 v1.2}
%   \item
%     Code is not changed.
%   \item
%     URLs updated.
%   \end{Version}
%   \begin{Version}{2010/01/11 v1.3}
%   \item
%     Option `rulebetweencolor' added.
%   \end{Version}
%   \begin{Version}{2016/05/16 v1.4}
%   \item
%     Documentation updates.
%   \end{Version}
% \end{History}
%
% \PrintIndex
%
% \Finale
\endinput
|
% \end{quote}
% Do not forget to quote the argument according to the demands
% of your shell.
%
% \paragraph{Generating the documentation.}
% You can use both the \xfile{.dtx} or the \xfile{.drv} to generate
% the documentation. The process can be configured by the
% configuration file \xfile{ltxdoc.cfg}. For instance, put this
% line into this file, if you want to have A4 as paper format:
% \begin{quote}
%   \verb|\PassOptionsToClass{a4paper}{article}|
% \end{quote}
% An example follows how to generate the
% documentation with pdf\LaTeX:
% \begin{quote}
%\begin{verbatim}
%pdflatex pdfcolparallel.dtx
%makeindex -s gind.ist pdfcolparallel.idx
%pdflatex pdfcolparallel.dtx
%makeindex -s gind.ist pdfcolparallel.idx
%pdflatex pdfcolparallel.dtx
%\end{verbatim}
% \end{quote}
%
% \section{Catalogue}
%
% The following XML file can be used as source for the
% \href{http://mirror.ctan.org/help/Catalogue/catalogue.html}{\TeX\ Catalogue}.
% The elements \texttt{caption} and \texttt{description} are imported
% from the original XML file from the Catalogue.
% The name of the XML file in the Catalogue is \xfile{pdfcolparallel.xml}.
%    \begin{macrocode}
%<*catalogue>
<?xml version='1.0' encoding='us-ascii'?>
<!DOCTYPE entry SYSTEM 'catalogue.dtd'>
<entry datestamp='$Date$' modifier='$Author$' id='pdfcolparallel'>
  <name>pdfcolparallel</name>
  <caption>Fix colour problems in package 'parallel'.</caption>
  <authorref id='auth:oberdiek'/>
  <copyright owner='Heiko Oberdiek' year='2007,2008,2010'/>
  <license type='lppl1.3'/>
  <version number='1.4'/>
  <description>
    Since version 1.40 pdfTeX supports colour stacks.
    This package uses them to fix colour problems in
    package <xref refid='parallel'>parallel</xref>.
    <p/>
    The package is part of the <xref refid='oberdiek'>oberdiek</xref>
    bundle.
  </description>
  <documentation details='Package documentation'
      href='ctan:/macros/latex/contrib/oberdiek/pdfcolparallel.pdf'/>
  <ctan file='true' path='/macros/latex/contrib/oberdiek/pdfcolparallel.dtx'/>
  <miktex location='oberdiek'/>
  <texlive location='oberdiek'/>
  <install path='/macros/latex/contrib/oberdiek/oberdiek.tds.zip'/>
</entry>
%</catalogue>
%    \end{macrocode}
%
% \begin{thebibliography}{9}
%
% \bibitem{parallel}
%   Matthias Eckermann: \textit{The \xpackage{parallel}-package};
%   2003/04/13;\\
%   \CTAN{macros/latex/contrib/parallel/}.
%
% \bibitem{pdfcol}
%   Heiko Oberdiek: \textit{The \xpackage{pdfcol} package};
%   2007/09/09;\\
%   \CTAN{macros/latex/contrib/oberdiek/pdfcol.pdf}.
%
% \end{thebibliography}
%
% \begin{History}
%   \begin{Version}{2007/09/09 v1.0}
%   \item
%     First version.
%   \end{Version}
%   \begin{Version}{2007/12/12 v1.1}
%   \item
%     Adds patch for setting \cs{linewidth} to fix bug
%     in package \xpackage{parallel}.
%   \item
%     Package \xpackage{parallel} is also fixed if color
%     stacks are not available.
%   \item
%     Bug fix, switched stacks now initialized with current color.
%   \item
%     Fix for package \xpackage{parallel}: \cs{raggedbottom} is respected.
%   \end{Version}
%   \begin{Version}{2008/08/11 v1.2}
%   \item
%     Code is not changed.
%   \item
%     URLs updated.
%   \end{Version}
%   \begin{Version}{2010/01/11 v1.3}
%   \item
%     Option `rulebetweencolor' added.
%   \end{Version}
%   \begin{Version}{2016/05/16 v1.4}
%   \item
%     Documentation updates.
%   \end{Version}
% \end{History}
%
% \PrintIndex
%
% \Finale
\endinput
|
% \end{quote}
% Do not forget to quote the argument according to the demands
% of your shell.
%
% \paragraph{Generating the documentation.}
% You can use both the \xfile{.dtx} or the \xfile{.drv} to generate
% the documentation. The process can be configured by the
% configuration file \xfile{ltxdoc.cfg}. For instance, put this
% line into this file, if you want to have A4 as paper format:
% \begin{quote}
%   \verb|\PassOptionsToClass{a4paper}{article}|
% \end{quote}
% An example follows how to generate the
% documentation with pdf\LaTeX:
% \begin{quote}
%\begin{verbatim}
%pdflatex pdfcolparallel.dtx
%makeindex -s gind.ist pdfcolparallel.idx
%pdflatex pdfcolparallel.dtx
%makeindex -s gind.ist pdfcolparallel.idx
%pdflatex pdfcolparallel.dtx
%\end{verbatim}
% \end{quote}
%
% \section{Catalogue}
%
% The following XML file can be used as source for the
% \href{http://mirror.ctan.org/help/Catalogue/catalogue.html}{\TeX\ Catalogue}.
% The elements \texttt{caption} and \texttt{description} are imported
% from the original XML file from the Catalogue.
% The name of the XML file in the Catalogue is \xfile{pdfcolparallel.xml}.
%    \begin{macrocode}
%<*catalogue>
<?xml version='1.0' encoding='us-ascii'?>
<!DOCTYPE entry SYSTEM 'catalogue.dtd'>
<entry datestamp='$Date$' modifier='$Author$' id='pdfcolparallel'>
  <name>pdfcolparallel</name>
  <caption>Fix colour problems in package 'parallel'.</caption>
  <authorref id='auth:oberdiek'/>
  <copyright owner='Heiko Oberdiek' year='2007,2008,2010'/>
  <license type='lppl1.3'/>
  <version number='1.4'/>
  <description>
    Since version 1.40 pdfTeX supports colour stacks.
    This package uses them to fix colour problems in
    package <xref refid='parallel'>parallel</xref>.
    <p/>
    The package is part of the <xref refid='oberdiek'>oberdiek</xref>
    bundle.
  </description>
  <documentation details='Package documentation'
      href='ctan:/macros/latex/contrib/oberdiek/pdfcolparallel.pdf'/>
  <ctan file='true' path='/macros/latex/contrib/oberdiek/pdfcolparallel.dtx'/>
  <miktex location='oberdiek'/>
  <texlive location='oberdiek'/>
  <install path='/macros/latex/contrib/oberdiek/oberdiek.tds.zip'/>
</entry>
%</catalogue>
%    \end{macrocode}
%
% \begin{thebibliography}{9}
%
% \bibitem{parallel}
%   Matthias Eckermann: \textit{The \xpackage{parallel}-package};
%   2003/04/13;\\
%   \CTAN{macros/latex/contrib/parallel/}.
%
% \bibitem{pdfcol}
%   Heiko Oberdiek: \textit{The \xpackage{pdfcol} package};
%   2007/09/09;\\
%   \CTAN{macros/latex/contrib/oberdiek/pdfcol.pdf}.
%
% \end{thebibliography}
%
% \begin{History}
%   \begin{Version}{2007/09/09 v1.0}
%   \item
%     First version.
%   \end{Version}
%   \begin{Version}{2007/12/12 v1.1}
%   \item
%     Adds patch for setting \cs{linewidth} to fix bug
%     in package \xpackage{parallel}.
%   \item
%     Package \xpackage{parallel} is also fixed if color
%     stacks are not available.
%   \item
%     Bug fix, switched stacks now initialized with current color.
%   \item
%     Fix for package \xpackage{parallel}: \cs{raggedbottom} is respected.
%   \end{Version}
%   \begin{Version}{2008/08/11 v1.2}
%   \item
%     Code is not changed.
%   \item
%     URLs updated.
%   \end{Version}
%   \begin{Version}{2010/01/11 v1.3}
%   \item
%     Option `rulebetweencolor' added.
%   \end{Version}
%   \begin{Version}{2016/05/16 v1.4}
%   \item
%     Documentation updates.
%   \end{Version}
% \end{History}
%
% \PrintIndex
%
% \Finale
\endinput
|
% \end{quote}
% Do not forget to quote the argument according to the demands
% of your shell.
%
% \paragraph{Generating the documentation.}
% You can use both the \xfile{.dtx} or the \xfile{.drv} to generate
% the documentation. The process can be configured by the
% configuration file \xfile{ltxdoc.cfg}. For instance, put this
% line into this file, if you want to have A4 as paper format:
% \begin{quote}
%   \verb|\PassOptionsToClass{a4paper}{article}|
% \end{quote}
% An example follows how to generate the
% documentation with pdf\LaTeX:
% \begin{quote}
%\begin{verbatim}
%pdflatex pdfcolparallel.dtx
%makeindex -s gind.ist pdfcolparallel.idx
%pdflatex pdfcolparallel.dtx
%makeindex -s gind.ist pdfcolparallel.idx
%pdflatex pdfcolparallel.dtx
%\end{verbatim}
% \end{quote}
%
% \section{Catalogue}
%
% The following XML file can be used as source for the
% \href{http://mirror.ctan.org/help/Catalogue/catalogue.html}{\TeX\ Catalogue}.
% The elements \texttt{caption} and \texttt{description} are imported
% from the original XML file from the Catalogue.
% The name of the XML file in the Catalogue is \xfile{pdfcolparallel.xml}.
%    \begin{macrocode}
%<*catalogue>
<?xml version='1.0' encoding='us-ascii'?>
<!DOCTYPE entry SYSTEM 'catalogue.dtd'>
<entry datestamp='$Date$' modifier='$Author$' id='pdfcolparallel'>
  <name>pdfcolparallel</name>
  <caption>Fix colour problems in package 'parallel'.</caption>
  <authorref id='auth:oberdiek'/>
  <copyright owner='Heiko Oberdiek' year='2007,2008,2010'/>
  <license type='lppl1.3'/>
  <version number='1.4'/>
  <description>
    Since version 1.40 pdfTeX supports colour stacks.
    This package uses them to fix colour problems in
    package <xref refid='parallel'>parallel</xref>.
    <p/>
    The package is part of the <xref refid='oberdiek'>oberdiek</xref>
    bundle.
  </description>
  <documentation details='Package documentation'
      href='ctan:/macros/latex/contrib/oberdiek/pdfcolparallel.pdf'/>
  <ctan file='true' path='/macros/latex/contrib/oberdiek/pdfcolparallel.dtx'/>
  <miktex location='oberdiek'/>
  <texlive location='oberdiek'/>
  <install path='/macros/latex/contrib/oberdiek/oberdiek.tds.zip'/>
</entry>
%</catalogue>
%    \end{macrocode}
%
% \begin{thebibliography}{9}
%
% \bibitem{parallel}
%   Matthias Eckermann: \textit{The \xpackage{parallel}-package};
%   2003/04/13;\\
%   \CTAN{macros/latex/contrib/parallel/}.
%
% \bibitem{pdfcol}
%   Heiko Oberdiek: \textit{The \xpackage{pdfcol} package};
%   2007/09/09;\\
%   \CTAN{macros/latex/contrib/oberdiek/pdfcol.pdf}.
%
% \end{thebibliography}
%
% \begin{History}
%   \begin{Version}{2007/09/09 v1.0}
%   \item
%     First version.
%   \end{Version}
%   \begin{Version}{2007/12/12 v1.1}
%   \item
%     Adds patch for setting \cs{linewidth} to fix bug
%     in package \xpackage{parallel}.
%   \item
%     Package \xpackage{parallel} is also fixed if color
%     stacks are not available.
%   \item
%     Bug fix, switched stacks now initialized with current color.
%   \item
%     Fix for package \xpackage{parallel}: \cs{raggedbottom} is respected.
%   \end{Version}
%   \begin{Version}{2008/08/11 v1.2}
%   \item
%     Code is not changed.
%   \item
%     URLs updated.
%   \end{Version}
%   \begin{Version}{2010/01/11 v1.3}
%   \item
%     Option `rulebetweencolor' added.
%   \end{Version}
%   \begin{Version}{2016/05/16 v1.4}
%   \item
%     Documentation updates.
%   \end{Version}
% \end{History}
%
% \PrintIndex
%
% \Finale
\endinput
