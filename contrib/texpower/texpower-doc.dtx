% \iffalse meta-comment
% --------------------------------------------------------------
% Part of the TeXPower bundle
% Copyright (C) 1999-2004 Stephan Lehmke
% Copyright (C) 2003-2005 Hans Fredrik Nordhaug
%
% This program is free software; you can redistribute it and/or
% modify it under the terms of the GNU General Public License
% as published by the Free Software Foundation; either version 2
% of the License, or (at your option) any later version.
%
% This program is distributed in the hope that it will be useful,
% but WITHOUT ANY WARRANTY; without even the implied warranty of
% MERCHANTABILITY or FITNESS FOR A PARTICULAR PURPOSE.  See the
% GNU General Public License for more details.
% --------------------------------------------------------------
%
% texpower-doc.dtx,v 1.13 2005/04/09 23:12:36 hansfn Exp
%
% \fi
%
% \iffalse
%
%<*driver>
\documentclass{article}
\begin{document}
\typeout{***********************************************************}
\typeout{*}
\typeout{*\space\space\space\space\space\space\space Do NOT compile this dtx file!}
\typeout{*}
\typeout{***********************************************************}
\end{document}
%</driver>
%
% \fi
%
%<faq-disp>\ProvidesFile{FAQ-display.tex}%
%<faq-print>\ProvidesFile{FAQ-printout.tex}%
%<faq-disp|faq-print>      [2005/04/07 TeXPower FAQ]
%
%<manual>\ProvidesFile{manual.tex}%
%<manual>      [2005/04/07 TeXPower manual]
%<fulldemo>\ProvidesFile{fulldemo.tex}%
%<fulldemo>      [2005/04/07 TeXPower full demo and manual]
%
%<bckwrdexample>\ProvidesFile{bckwrdexample.tex}%
%<bgndexample>\ProvidesFile{bgndexample.tex}%
%<divexample>\ProvidesFile{divexample.tex}%
%<fancyexample>\ProvidesFile{fancyexample.tex}%
%<foilsdemo>\ProvidesFile{foilsdemo.tex}%
%<hilitexample>\ProvidesFile{hilitexample.tex}%
%<ifmslidemo>\ProvidesFile{ifmslidemo.tex}%
%<mathexample>\ProvidesFile{mathexample.tex}%
%<panelexample>\ProvidesFile{panelexample.tex}%
%<parexample>\ProvidesFile{parexample.tex}%
%<pdfscrdemo>\ProvidesFile{pdfscrdemo.tex}%
%<pdfslidemo>\ProvidesFile{pdfslidemo.tex}%
%<picexample>\ProvidesFile{picexample.tex}%
%<pp4sldemo>\ProvidesFile{pp4sldemo.tex}%
%<prosperdemo>\ProvidesFile{prosperdemo.tex}%
%<seminardemo>\ProvidesFile{seminardemo.tex}%
%<simpledemo>\ProvidesFile{simpledemo.tex}%
%<slidesdemo>\ProvidesFile{slidesdemo.tex}%
%<spanelexample>\ProvidesFile{spanelexample.tex}%
%<tabexample>\ProvidesFile{tabexample.tex}%
%<verbexample>\ProvidesFile{verbexample.tex}%
%<bckwrdexample|bgndexample|divexample|fancyexample|foilsdemo|hilitexample|ifmslidemo>      [2005/04/07 TeXPower example file]
%<mathexample|panelexample|parexample|pdfscrdemo|pdfslidemo|picexample|pp4sldemo>      [2005/04/07 TeXPower example file]
%<prosperdemo|seminardemo|simpledemo|slidesdemo|spanelexample|tabexample|verbexample>      [2005/04/07 TeXPower example file]
%
%<*version>
%<<KEEPCOMMENTS

% Version info used in titles
\def\tpversion{v0.2 of April 8, 2005}
%KEEPCOMMENTS
%</version>
%
% \CheckSum{0}
%
% \CharacterTable
%  {Upper-case    \A\B\C\D\E\F\G\H\I\J\K\L\M\N\O\P\Q\R\S\T\U\V\W\X\Y\Z
%   Lower-case    \a\b\c\d\e\f\g\h\i\j\k\l\m\n\o\p\q\r\s\t\u\v\w\x\y\z
%   Digits        \0\1\2\3\4\5\6\7\8\9
%   Exclamation   \!     Double quote  \"     Hash (number) \#
%   Dollar        \$     Percent       \%     Ampersand     \&
%   Acute accent  \'     Left paren    \(     Right paren   \)
%   Asterisk      \*     Plus          \+     Comma         \,
%   Minus         \-     Point         \.     Solidus       \/
%   Colon         \:     Semicolon     \;     Less than     \<
%   Equals        \=     Greater than  \>     Question mark \?
%   Commercial at \@     Left bracket  \[     Backslash     \\
%   Right bracket \]     Circumflex    \^     Underscore    \_
%   Grave accent  \`     Left brace    \{     Vertical bar  \|
%   Right brace   \}     Tilde         \~}
%
%<*config>
%<<KEEPCOMMENTS
%-----------------------------------------------------------------------------------------------------------------
%
% Code for user-specific configuration of TeXPower documentation files.
%
% This file is input by others. Don't compile it separately.
%
\hypersetup{baseurl={http://texpower.sourceforge.net/doc/}}
\hypersetup{pdfsubject={Documentation and Examples for the texpower package}}
\hypersetup{pdfauthor={Stephan Lehmke}}
%KEEPCOMMENTS
%</config>
%=================================================================================================================
%<*preamble>
%<<KEEPCOMMENTS
%
\documentclass
[%
%-----------------------------------------------------------------------------------------------------------------
% Document class options:
% -----------------------
%
% Landscape slides formatted for letter paper fit most screen resolutions (more or less).
%
  letterpaper,%
  landscape,%
% 
% The KOMA option makes powersem load scrartcl.cls instead of article.cls.
%
  KOMA,%
% KOMA document class options are accepted.
  smallheadings,%
%
% The calcdimensions option makes powersem calculate the slide dimensions automatically from paper size and margins.
  calcdimensions,%
%
% The display option sets everything up for producing slides to be displayed interactively.
% This option is also recognized by the texpower package.
%
  display%
%-----------------------------------------------------------------------------------------------------------------
]
%-----------------------------------------------------------------------------------------------------------------
% Document class powersem, based on seminar.cls for simulating ppower via latex+distiller (instead of pdflatex).
%
{powersem}

%-----------------------------------------------------------------------------------------------------------------
% Set slide margins rather small for maximum use of space. This is a demo, remember.
%
\renewcommand{\slidetopmargin}{5mm}
\renewcommand{\slidebottommargin}{5mm}

\renewcommand{\slideleftmargin}{5mm}
\renewcommand{\sliderightmargin}{5mm}


%-----------------------------------------------------------------------------------------------------------------
% Some setup for more reasonable spacing.
%

\makeatletter

\renewcommand\section{\@startsection{section}{1}{\z@}%
  {-1.5ex\@plus -1ex \@minus -.5ex}%
  {.5ex \@plus .2ex}%
  {\raggedsection\normalfont\size@section\sectfont}}

\renewcommand\subsection{\@startsection{subsection}{2}{\z@}%
  {-1.25ex\@plus -1ex \@minus -.2ex}%
  {.5ex \@plus .2ex}%
  {\raggedsection\normalfont\size@subsection\sectfont}}

\renewcommand\subsubsection{\@startsection{subsubsection}{3}{\z@}%
  {-1.25ex\@plus -1ex \@minus -.2ex}%
  {.5ex \@plus .2ex}%
  {\raggedsection\normalfont\size@subsubsection\sectfont}}

\renewcommand\paragraph{\@startsection{paragraph}{4}{\z@}%
  {1.25ex \@plus1ex \@minus.2ex}%
  {-1em}%
  {\raggedsection\normalfont\size@paragraph\sectfont}}

\def\slideitemsep{.5ex plus .3ex minus .2ex}

\makeatother

%-----------------------------------------------------------------------------------------------------------------
% We need some more packages...
%

\usepackage{url}

\usepackage[latin1]{inputenc}

% One more Text emphasis command...

\let\name=\textsc

%-----------------------------------------------------------------------------------------------------------------
% We load hyperref and fixseminar which fixes some problems with seminar.
%
\usepackage[ps2pdf,plainpages=false,bookmarksopen,colorlinks,urlcolor=red,pdfpagemode=FullScreen]{hyperref}
\usepackage{fixseminar}

%-----------------------------------------------------------------------------------------------------------------
% Finally, the texpower package is loaded. 
%
\usepackage{texpower}

%% The configuration file allows user-specific settings.

\input{__TPcfg}

%-----------------------------------------------------------------------------------------------------------------
% Some more parameters...
%
\slidesmag{5}
\slideframe{none}
\pagestyle{empty}
\setcounter{tocdepth}{2}
\renewcommand{\currentpagevalue}{\value{slide}}

%-----------------------------------------------------------------------------------------------------------------
% The following command produces a title page for every example and documentation file.

\newcommand{\makeslidetitle}[1]
{%
  \title{The \TeX Power bundle\\[2ex]{\normalfont #1}}
  \author
  {%
    Stephan Lehmke\\
    \mdseries
    University of Dortmund\\
    \mdseries
    Department of Computer Science I\\
    \url{mailto:Stephan.Lehmke@udo.edu}%
  }
  {\centerslidestrue
  \maketitle
  \newslide}
  \setcounter{firststep}{1}% This way, the first step of all examples is displayed.
}
%KEEPCOMMENTS
%</preamble>
%=================================================================================================================
%<*indexing>
%<<KEEPCOMMENTS
%
% Setting up indexing for the TeXPower fulldemo and manual.
% 
%-----------------------------------------------------------------------------------------------------------------

\usepackage{makeidx}
\makeindex
\newcommand{\indexcode}[1]{\index{#1@\code{#1}}}
\newcommand{\indexmacro}[1]{\index{#1@\macroname{#1}}}
\newcommand{\indexmacroopt}[2]{%
  \index{#1 macro options@\macroname{#1} macro options!#2@\code{#2}}%
  \index{#2@\code{#2}|see{\macroname{#1} macro options}}}
\newcommand{\indexfile}[2]{\index{#1@\code{#1} #2}}
\newcommand{\indexpckopt}[2]{%
  \index{#1 package options@\code{#1} package options!#2@\code{#2}}%
  \index{#2@\code{#2}|see{\code{#1} package options}}}
\newcommand{\indexpckswitch}[2]{%
  \index{#1 package switches@\code{#1} package switches!#2@\code{#2}}%
  \index{#2@\code{#2}|see{\code{#1} package switches}}}
\newcommand{\indexstepwise}[2]{%
  \index{stepwise@\macroname{stepwise}!#1@\code{#1} (#2)}%
  \index{#1@\code{#1}|see{\macroname{stepwise}}}}

%KEEPCOMMENTS
%</indexing>
%=================================================================================================================
%<*faq-disp>
%<<KEEPCOMMENTS

% Enable all color emphasis and highlighting options; use a light background and slifonts.

\PassOptionsToPackage{coloremph,colormath,colorhighlight,lightbackground}{texpower}
\RequirePackage{tpslifonts}

% Input the generic preamble.

\input{__TPpreamble}
\hypersetup{pdftitle={TeXPower Frequently Asked Questions list}}

\slidesmag{4}

% The package soul is needed for \highlighttext to work.

\usepackage{soul}


% The following package makes code look a little nicer, but it may not be present on all systems.

\IfFileExists{cmtt.sty}{\usepackage[override]{cmtt}}{}

\usepackage{fancyvrb}

\usepackage{array}

\usepackage{longtable}

%-----------------------------------------------------------------------------------------------------------------
% Finally, everything is set up. Here we go...
%
\begin{document}
\begin{slide}
%KEEPCOMMENTS
%</faq-disp>
%=================================================================================================================
%<*faq-print>
%<<KEEPCOMMENTS

\documentclass[12pt]{scrartcl}

%-----------------------------------------------------------------------------------------------------------------
% We need some more packages...
%

\usepackage{url}

\usepackage[latin1]{inputenc}

% The following package makes code look a little nicer, but it may not be present on all systems.

\IfFileExists{cmtt.sty}{\usepackage[override]{cmtt}}{}

% Loading the soul package enables the \highlighttext command.

\usepackage{soul}

% One more Text emphasis command...

\let\name=\textsc


\usepackage{fancyvrb}

\usepackage{array}

\usepackage{longtable}


%-----------------------------------------------------------------------------------------------------------------
% Load hyperref.
%
\usepackage[bookmarksopen,colorlinks]{hyperref}

%-----------------------------------------------------------------------------------------------------------------
% Finally, the texpower package is loaded. 
%
\usepackage{texpower}

%-----------------------------------------------------------------------------------------------------------------
% The configuration file allows user-specific settings.

\input{__TPcfg}

%-----------------------------------------------------------------------------------------------------------------
% The code in the file __TPFAQ is meant for the document class seminar. Thus, it contains some seminar-specific commands
% which are replaced by dummies here.

\let\newslide=\relax

%-----------------------------------------------------------------------------------------------------------------
% The following command produces the title for this document. 

\newcommand{\makeslidetitle}[1]
{%
  \hypersetup{pdftitle={#1}}
  \title{The \TeX Power bundle\\{\normalfont #1}}
  \author{Stephan Lehmke\\\url{mailto:Stephan.Lehmke@cs.uni-dortmund.de}}
  \maketitle
}

  \tolerance 1414
  \hbadness 1414
  \emergencystretch 1.5em
  \hfuzz 0.3pt
  \vfuzz \hfuzz
  \relax

%-----------------------------------------------------------------------------------------------------------------
% Finally, everything is set up. Here we go...
%

\begin{document}
%KEEPCOMMENTS
%</faq-print>
%=================================================================================================================
%<*faq>
%<<KEEPCOMMENTS

\providecommand{\vanishcolor}{}

\DefineVerbatimEnvironment{LaTeXCode}{Verbatim}
{gobble=2,formatcom=\codeswitch,frame=single,numbers=left,xleftmargin=1em,numbersep=.5em}

%-----------------------------------------------------------------------------------------------------------------
%
\makeslidetitle
{%
  Frequently asked questions list%
  \thanks{FAQ for \TeX Power \tpversion .}%
  }%

\setcounter{firststep}{0}

\tableofcontents

\newslide

%-----------------------------------------------------------------------------------------------------------------
%
\section{General}
\subsection{Where can I get the newest version of the \TeX Power FAQ\,?}
You can download the latest version of the \TeX Power FAQ from the following URLs:

\begin{center}
Screen version:\\
\url{http://texpower.sourceforge.net/doc/FAQ-display.pdf}

Printout version:\\
\url{http://texpower.sourceforge.net/doc/FAQ-printout.pdf}
\end{center}


\newslide

\subsection{What is \TeX Power\,?}
The \TeX Power bundle contains style and class files for creating dynamic online presentations with \LaTeX. 

The heart of the bundle is the package \code{texpower.sty} which implements some commands for presentation effects. This
includes setting page transitions, color highlighting and displaying pages incrementally.

The document class \code{powersem.cls} is a wrapper for seminar which sets up everything for dynamic presentations. 


\newslide

\subsection{Where can I obtain \TeX Power\,?}
The complete bundle, together with its documentation, can be found under the URL
\begin{center}
  \url{http://texpower.sourceforge.net/}
\end{center}

\newslide

\subsection{Where can I discuss \TeX Power or ask for help\,?}
Bug and problem reports should go to 
\href{http://sourceforge.net/tracker/?group_id=60743&atid=495145}{the bug tracker}.

Discussions about \TeX Power should take place on 
\href{http://lists.sourceforge.net/lists/listinfo/texpower-users/}{the mailing list}
or in 
\href{http://sourceforge.net/forum/forum.php?forum_id=204738}{the discussion forum}.


\newslide

\subsection{What alternatives are there to using \TeX Power\,?}
The most prominent alternative to \TeX Power is the \concept{Pdf Presentation Post Processor} PPower4, the homepage of
which is
\begin{quote}
  \url{http://www-sp.iti.informatik.tu-darmstadt.de/software/ppower4/}
\end{quote}

Another alternative is the \concept{Utopia PDF Presentations Bundle}, which provides a complete presentation design
environment. Its home page is
\begin{quote}
  \url{http://www.utopiatype.com.au/products/ubundle.html}
\end{quote}

Comparisons of different presentation packages can be found on the home page of \name{Prof.\,D.\,P.\,Story}:
\begin{quote}
  \url{http://www.math.uakron.edu/~dpstory/pdf_demos.html}
\end{quote}
and in the talk held by \name{Ross Moore} at the \name{California Institute of Technology} on 8th May 2000:
\begin{quote}
  \url{http://www.cds.caltech.edu/caltex/2000/}
\end{quote}
 




\newslide

%-----------------------------------------------------------------------------------------------------------------
%
\section{Usage}

\subsection{How do I design a presentation with \TeX Power\,?}
It should be stressed that \TeX Power is \underl{not} (currently) a complete presentation package. It just adds dynamic
presentation effects (and some other gimmicks specifically interesting for dynamic presentations) and should always be
combined with a document class dedicated to designing presentations (or a package like
\href{ftp://ftp.dante.de/tex-archive/help/Catalogue/entries/pdfslide.html}{\code{pdfslide}}).

There are demos in the \href{http://texpower.sourceforge.net/doc/}{\code{doc}} directory for most
popular presentation-making document classes and packages.


\newslide

\subsection{I find \TeX Power very complicated. How can I learn how to realize dynamic effects\,?}

As always with \TeX, you should first make up your mind what kind of effect you desire, and what \LaTeX{} structures
will be involved.

Then you should check the examples in the \href{http://texpower.sourceforge.net/doc/}{\code{doc}}
directory for anything similar to what you want. If you find anything suitable, read the corresponding code. There are
some inline comments to explain what's going on. Print out the
\href{http://texpower.sourceforge.net/doc/manual.pdf}{\code{manual}} for documentation of the \TeX
Power commands.

Further `recepies' can be found in section \ref{Sec:HowTo}.

If you don't find anything suitable you can modify to your needs, and can't figure out from the documentation how to
achieve your aims, please ask on
\href{http://lists.sourceforge.net/lists/listinfo/texpower-users/}{the mailing list}. If
you've found an application for \TeX Power not covered by the examples, a new example should be created.


\newslide

\subsection{Can I combine \TeX Power with PPower4\,?}
There is no problem postprocessing documents in which \TeX Power is used. This can be useful, for instance, for
realising structured backgrounds with the \code{background} package from the
\href{http://www-sp.iti.informatik.tu-darmstadt.de/software/ppower4/}{PPower4 bundle}.

If there are presentation effects for which you'd like to use PPower4's implementation of the \macroname{pause} command,
then just load PPower4's \code{pause} package. PPower4's definition of \macroname{pause} will override
\code{texpower}'s. Then you can combine PPower4's \macroname{pause} functionality with \TeX Power's \macroname{stepwise}
functionality, for maximum expressive power.

\newslide

\subsection{I'm missing some of the classes and packages used in the demo and example files.}

First of all, it has to be said that \TeX Power makes use of some `modern' features which have been introduced into the
\TeX{} System quite recently and are evolving swiftly. The core of the \code{texpower} package, namely the commands
\macroname{pause} and \macroname{stepwise} is implemented in `pure' \LaTeX{} and should be largely independent of any
fancy extensions, but to get most out of \TeX Power's presentation features and process the more advanced examples, it
is recommended to have a moderately new \TeX{} distribution installed (rule of thumb: not older than one year).

But even if your distribution is quite new, it might not contain some of the classes and packages used by the demos and
examples. Here's a list of (hopefully all of) the packages and classes used (which are not part of core \LaTeX) and
their availability:


\newlength{\twidth}%
{%
  \makeatletter
  \ifthenelse{\boolean{display}}{\def\@outputpage{\setbox\@cclv\box\@outputbox\slide@output}}{}%
  \makeatother
  \setlength{\twidth}{\linewidth-\widthof{\code{pdfscreen}}-2cm-6\tabcolsep}%
\begin{longtable}{lp{2cm}>{\raggedright\footnotesize}p{\twidth}}
  Package&used in&\normalsize available from\tabularnewline\hline\endhead
  \href{ftp://ftp.dante.de/tex-archive/help/Catalogue/entries/hyperref.html}{\code{hyperref}}&most
  &\href{http://www.ctan.org/}{CTAN}, e.\,g.\
  \url{ftp://ftp.dante.de/tex-archive/macros/latex/contrib/supported/hyperref}
  \tabularnewline
  \href{ftp://ftp.dante.de/tex-archive/help/Catalogue/entries/url.html}{\code{url}}&most
  &\href{http://www.ctan.org/}{CTAN}, e.\,g.\ \url{ftp://ftp.dante.de/tex-archive/macros/latex/contrib/other/misc}
  \tabularnewline
  \href{ftp://ftp.dante.de/tex-archive/help/Catalogue/entries/soul.html}{\code{soul}}&many
  &\href{http://www.ctan.org/}{CTAN}, e.\,g.\ \url{ftp://ftp.dante.de/tex-archive/macros/latex/contrib/supported/soul}
  \tabularnewline
  \href{ftp://ftp.dante.de/tex-archive/help/Catalogue/entries/pstricks.html}{\code{pstricks}}
  &\small\code{fulldemo}, \code{picexample}
  &\href{http://www.ctan.org/}{CTAN}, e.\,g.\ \url{ftp://ftp.dante.de/tex-archive/graphics/pstricks}
  \tabularnewline
  \href{ftp://ftp.dante.de/tex-archive/help/Catalogue/entries/xr.html}{\code{xr-hyper}}
  &\small\code{manual}
  &\href{http://www.ctan.org/}{CTAN}, e.\,g.\
  \url{ftp://ftp.dante.de/tex-archive/macros/latex/contrib/supported/hyperref} 
  \tabularnewline
  \href{ftp://ftp.dante.de/tex-archive/help/Catalogue/entries/fancyvrb.html}{\code{fancyvrb}}
  &\small\code{FAQ}
  &\href{http://www.ctan.org/}{CTAN}, e.\,g.\
  \url{ftp://ftp.dante.de/tex-archive/macros/latex/contrib/supported/fancyvrb} 
  \tabularnewline
  \href{ftp://ftp.dante.de/tex-archive/help/Catalogue/entries/pdfscreen.html}{\code{pdfscreen}}
  &\small\code{pdfscrdemo}
  &\href{http://www.ctan.org/}{CTAN}, e.\,g.\
  \url{ftp://ftp.dante.de/tex-archive/macros/latex/contrib/supported/pdfscreen} 
  \tabularnewline
  \href{ftp://ftp.dante.de/tex-archive/help/Catalogue/entries/pdfslide.html}{\code{pdfslide}}
  &\small\code{pdfslidemo}
  &\href{http://www.ctan.org/}{CTAN}, e.\,g.\
  \url{ftp://ftp.dante.de/tex-archive/macros/latex/contrib/supported/pdfslide} 
  \tabularnewline
  \href{ftp://ftp.dante.de/tex-archive/help/Catalogue/entries/ppower4.html}{\code{pp4slide}}
  &\small\code{pp4sldemo}
  &\href{http://www.ctan.org/}{CTAN}, e.\,g.\
  \url{ftp://ftp.dante.de/tex-archive/support/ppower4/pp4sty.zip} 
  \tabularnewline
  \href{ftp://ftp.dante.de/tex-archive/help/Catalogue/entries/ifmslide.html}{\code{ifmslide}}
  &\small\code{ifmslidemo}
  &\href{http://www.ctan.org/}{CTAN}, e.\,g.\
  \url{ftp://ftp.dante.de/tex-archive/macros/latex/contrib/supported/ifmslide} 
  \tabularnewline
\end{longtable}
}
  
{%
  \makeatletter
  \ifthenelse{\boolean{display}}{\def\@outputpage{\setbox\@cclv\box\@outputbox\slide@output}}{}%
  \makeatother
  \setlength{\twidth}{\linewidth-\widthof{\code{scrartcl}}-2cm-6\tabcolsep}%
\begin{longtable}{lp{2cm}>{\raggedright\footnotesize}p{\twidth}}
  Class&used in&\normalsize available from\tabularnewline\hline\endhead
  \href{ftp://ftp.dante.de/tex-archive/help/Catalogue/entries/seminar.html}{\code{seminar}}&most
  &\href{http://www.ctan.org/}{CTAN}, e.\,g.\
  \url{ftp://ftp.dante.de/tex-archive/macros/latex/contrib/other/seminar}
  \tabularnewline
  \href{ftp://ftp.dante.de/tex-archive/help/Catalogue/entries/koma-script.html}{\code{scrartcl}}&most
  &\href{http://www.ctan.org/}{CTAN}, e.\,g.\
  \url{ftp://ftp.dante.de/tex-archive/macros/latex/contrib/supported/koma-script}
  \tabularnewline
  \href{ftp://ftp.dante.de/tex-archive/help/Catalogue/entries/slides.html}{\code{slides}}
  &\small\code{slidesdemo}
  &\href{http://www.ctan.org/}{CTAN}, e.\,g.\
  \url{ftp://ftp.dante.de/tex-archive/macros/latex/base/}
  \tabularnewline
  \href{ftp://ftp.dante.de/tex-archive/help/Catalogue/entries/foiltex.html}{\code{foils}}
  &\small\code{foilsdemo}, \code{pp4sldemo}
  &\href{http://www.ctan.org/}{CTAN}, e.\,g.\
  \url{ftp://ftp.dante.de/tex-archive/nonfree/macros/latex/contrib/supported/foiltex}
  \tabularnewline
  \href{ftp://ftp.dante.de/tex-archive/help/Catalogue/entries/prosper.html}{\code{prosper}}
  &\small\code{prosperdemo}
  &\url{http://prosper.sourceforge.net/}
  \tabularnewline
\end{longtable}
}

\newslide

%-----------------------------------------------------------------------------------------------------------------
%
\section{How do I\dots}\label{Sec:HowTo}

\subsection{How can I incrementally display a paragraph of text\,?}\label{Q:Par}

The easiest solution is to use \macroname{parstepwise}, but if the arguments of \macroname{step} are long, you'll get
problems with line breaks, as \macroname{parstepwise} forces \macroname{step} to put its argument in a box.

You can use \macroname{hidetext} like this:

\begin{LaTeXCode}
  \stepwise[\let\hidestepcontents=\hidetext]
  {\step{Line breaks} \step{work in here.}}
\end{LaTeXCode}


\stepwise[\let\hidestepcontents=\hidetext]
{%
  \ifthenelse{\boolean{display}}
  {%
    yields
    \present
    {%
      \begin{minipage}{7em}
        \step{Line breaks} \step{work in here.}
      \end{minipage}%
      }
    }
  {}
  
  But note that \macroname{hidetext}, being implemented using the \code{soul} package, is quite fragile (compare
  \ref{Sec:hidetext}).

  }

\newslide

If you're not using structured backgrounds, \macroname{hidevanish} is another alternative which can be used exactly like
\macroname{hidetext}, but is much more robust (note that this will fail whenever your text should appear in front of
different background colors, for any reason).

In the argument of \macroname{hidevanish}, which uses \macroname{textcolor}, paragraph breaks are not allowed. Using
\macroname{vstep} is a little less restrictive:


\begin{LaTeXCode}
  \stepwise
  {%
    {\vstep Line and paragraph breaks 
    \vstep work in here.\par Yeah!}%
    }
\end{LaTeXCode}

\stepwise[\renewcommand{\vanishcolor}{presentcolor}]
{%
  \ifthenelse{\boolean{display}}
  {%
    yields
    \present
    {%
      \begin{minipage}{13em}
        \vstep Line and paragraph breaks \vstep work in here.\par Yeah!
      \end{minipage}%
      }
    }
  {}
}

\newslide

To facilitate the decision, here's a side-by-side comparison of the pros and cons:

\macroname{parstepwise}:
\begin{itemize}
\item[\origmath{+}] robust
\item[\origmath{+}] works with structured backgrounds
\item[\origmath{-}] no automatic line breaks in \macroname{step}'s argument 
\item[\origmath{-}] no paragraph breaks in \macroname{step}'s argument 
\end{itemize}

\macroname{hidetext}:
\begin{itemize}
\item[\origmath{-}] very fragile
\item[\origmath{+}] works with structured backgrounds
\item[\origmath{+}] allows automatic line breaks in \macroname{step}'s argument 
\item[\origmath{-}] no paragraph breaks in \macroname{step}'s argument 
\end{itemize}

\newslide

\macroname{hidevanish}:
\begin{itemize}
\item[\origmath{+}] robust
\item[\origmath{-}] fails with structured backgrounds
\item[\origmath{+}] allows automatic line breaks in \macroname{step}'s argument 
\item[\origmath{-}] no paragraph breaks in \macroname{step}'s argument 
\end{itemize}

\macroname{vstep}:
\begin{itemize}
\item[\origmath{+}] very robust
\item[\origmath{-}] fails with structured backgrounds
\item[\origmath{+}] allows automatic line breaks
\item[\origmath{+}] allows paragraph breaks
\end{itemize}

\newslide


\subsection{Instead of making text appear `out of nowhere', I'd rather just change colors from `dimmed' to normal.}

There are some analogies between this item and \ref{Q:Par}.

If you're using \code{texpower}'s standard colors, probably \macroname{hidedimmed} does what you want:

\begin{LaTeXCode}
  \stepwise[\let\hidestepcontents=\hidedimmed]
  {%
    \step{This works with} \step{\emph{all}} 
    \step{\highlighttext{highlighting} commands.}%
    }
\end{LaTeXCode}

\stepwise[\let\hidestepcontents=\hidedimmed]
{%
  \ifthenelse{\boolean{display}}
  {%
    yields
    \present
    {%
      \begin{minipage}{10em}
        \step{This works with} \step{\emph{all}} \step{\highlighttext{highlighting} commands.}
      \end{minipage}%
      }
    }
  {}
  }

\newslide

In the argument of \macroname{hidedimmed}, which uses \macroname{textcolor}, paragraph breaks are not allowed. Using
\macroname{dstep} is a little less restrictive. The following achieves the same result as above:

\begin{LaTeXCode}
  \stepwise
  {%
    \dstep This works with \dstep \emph{all}
    \dstep \highlighttext{highlighting} commands.%
    }
\end{LaTeXCode}

\newslide

If the dimmed colors look too fancy to you, you can also use \macroname{vstep} for this purpose, setting
\macroname{vanishcolor} to some `dimmed' color:

\begin{LaTeXCode}
  \stepwise[\renewcommand{\vanishcolor}{inactivecolor}]
  {%
    \vstep This works with \vstep \emph{all}
    \vstep \highlighttext{highlighting} commands.%
    }
\end{LaTeXCode}

\stepwise[\renewcommand{\vanishcolor}{inactivecolor}]
{%
  \ifthenelse{\boolean{display}}
  {%
    yields
    \present
    {%
      \begin{minipage}{10em}
        \vstep This works with \vstep \emph{all} \vstep \highlighttext{highlighting} commands.
      \end{minipage}%
      }
    }
  {}
  }

Achieving the same with \macroname{hidevanish} is left as an exercise to the reader.

\newslide

\subsection{\macroname{dstep} and \macroname{hidedimmed} work only with \code{texpower}'s standard colors. How can I dim
  my own colors\,?} 

\code{texpower} maintains a list of colors which will be affected by \macroname{dimcolors} (which is behind
\macroname{dstep} and \macroname{hidedimmed}). 

You can add your own colors to this list by issuing \commandapp{addTPcolor}{mycolor}. Then you only have to define
another color \code{dmycolor} which will be replaced for \code{mycolor} automatically when \macroname{dimcolors} is
executed.

\newslide

For instance:
{\small
\begin{LaTeXCode}
  \definecolor{mycolor}{rgb}{1,0.5,0}%
  \definecolor{dmycolor}{rgb}{0.9,0.8,0.6}%
  \addTPcolor{mycolor}
  \stepwise
  {\dstep My \emph{own} \dstep \textcolor{mycolor}{color}.}
\end{LaTeXCode}
}%

\ifthenelse{\boolean{display}}
{%
  \definecolor{mycolor}{rgb}{1,0.5,0}%
  \definecolor{dmycolor}{rgb}{0.9,0.8,0.6}%
  \addTPcolor{mycolor}%
  \liststepwise
  {%
    yields
    \present
    {%
      \dstep My \emph{own} \dstep \textcolor{mycolor}{color}.
      }
    }
  }
{}

Note that if you ever wish to use \macroname{enhancecolors} or \macroname{highlightenhanced}, you'll also need an
\emph{enhanced} version of your new color named \code{emycolor}.

If you wish to use one of the commands \macroname{whitebackground}, \macroname{lightbackground},
\macroname{darkbackground}, or \macroname{blackbackground}, you'll need even more variants of your new color. In this
case, you'll better define it in the file \code{tpsettings.cfg} (which contains an example).


\newslide

%-----------------------------------------------------------------------------------------------------------------
%
\section{Problems}

\subsection{I'm loading the \code{texpower} package, but dynamic features don't seem to work.}

Remember that you have to turn on dynamic features explicitly by giving the \code{display} option either to
\code{texpower} or as a global option. Otherwise, a printout version of your document is produced.

\newslide

\subsection{When I use the \code{colormath} option, my displayed formulae are not colored.}

Don't use the \TeX{} environment \code{\$\$}\dots\code{\$\$} for displayed formulae if you want to profit from math
coloring.

\code{texpower} supports \LaTeX's environments \macroname{[}\dots\macroname{]}, \code{displaymath}, \code{equation},
\code{eqnarray}, and \code{eqnarray*}. It also works with the diverse displayed math environments from the
\href{ftp://ftp.dante.de/tex-archive/help/Catalogue/entries/amsmath.html}{\code{amsmath}} package.

Replacing \code{\$\$}\dots\code{\$\$} everywhere by \macroname{[}\dots\macroname{]} should solve this problem.

\newslide

\subsection{It seems I can't use \macroname{vfill} in combination with \macroname{pause}.}

This is a problem indeed, as \LaTeX{} never gets to see anything after \macroname{pause} when the first part of the
sequence is produced. You can use \macroname{vfill} with \macroname{stepwise} if you
\begin{enumerate}
\item use a configuration where \macroname{step} leaves blank space (to ensure proper vertical spacing);
\item put \emph{all} \macroname{vfill}s into the argument of \macroname{stepwise}, \emph{outside} the argument of any
  \macroname{step}.
\end{enumerate}

For instance:

\begin{LaTeXCode}
  \parstepwise
  {\step{One.}\vfill\step{Two.}\vfill\step{Three.}}
\end{LaTeXCode}

\parstepwise
{%
  \ifthenelse{\boolean{display}}
  {%
    yields
    \present
    {%
      \begin{minipage}[c][10ex][s]{10em}
        \step{One.}\vfill\step{Two.}\vfill\step{Three.}
      \end{minipage}
      }
    }
  {}
  }


\newslide

\subsection{When I use \LaTeX+\code{dvips}+\code{distiller}, the result looks strange and `hyper' features don't work.}

Check the \code{log} file of your document. If it contains the line
{\codeswitch%
\begin{verbatim}
*hyperref using default driver hypertex*
\end{verbatim}%
}%
then the default hyperref driver for your system is not suited for processing by \code{dvips}+\code{distiller}.

Either you set another default driver (for instance, in the file \code{hyperref.cfg}), or you use the option
\code{dvips} in your document as a global option or an option to \commandapp{usepackage}{hyperref}. See the
documentation of the \href{ftp://ftp.dante.de/tex-archive/help/Catalogue/entries/hyperref.html}{\code{hyperref}} package
for details.

\newslide

\subsection{When using \macroname{highlighttext} or \macroname{hidetext}, I'm getting strange error messages.}
\label{Sec:hidetext}

Note that both these commands are implemented using the
\href{ftp://ftp.dante.de/tex-archive/help/Catalogue/entries/soul.html}{\code{soul}} package. \code{soul} has some rather
severe restrictions concerning what is allowed to appear in the argument of commands using it. Consult the documentation
of \href{ftp://ftp.dante.de/tex-archive/help/Catalogue/entries/soul.html}{\code{soul}} for a detailed description of
these restrictions.

The most prominent one is that almost no \LaTeX{} command is allowed in the argument of a command implemented using
\code{soul}. For instance, to use an emphasis or highlighting command like \macroname{emph}, you have to use a sequence
of \macroname{highlighttext} commands, putting \macroname{emph} `outside'. Expect glitches in display quality though.

\code{\small\commandapp{highlighttext}{This~}\discretionary{\%}{}{}\commandapp{emph}{\commandapp{highlighttext}{annoying~}}\discretionary{\%}{}{}\commandapp{highlighttext}{behaviour}}
yields \highlighttext{This }\emph{\highlighttext{annoying }}\highlighttext{behaviour}.

\newslide

Another restriction is that accents are separated from the characters they belong to and break. You have to enclose the
complete accented character with braces or use an appropriate input encoding, typing accented characters `as one'.
\begin{center}
  \begin{tabular}{l@{ yields }l}
    \commandapp{highlighttext}{S\{\macroname{"}u\}\macroname{ss} es}&\highlighttext{S{\"u}\ss es}\\
    \commandapp{highlighttext}{S��es}&\highlighttext{S��es}\\
  \end{tabular}
\end{center}

\newslide

\subsection{Inside the argument of \macroname{stepwise}, all counters seem to be freezed on all pages of the sequence
  generated. How can I use a self-defined counter which does not freeze\,?}

Freezing counters is a desirable behaviour in general, for instance to stop equation numbers from going astray.

But \code{texpower} maintains a list of counters which are \emph{not} freezed, containing for instance the counter
\code{step}. 

If you need a counter for special effects while the incremental sequence is generated (for instance: generating a
sequence of MetaPost figures with the \code{emp} and \code{feynmp} packages), use 
\begin{LaTeXCode}
  \releasecounter{mycounter}
\end{LaTeXCode}
to release the counter \code{mycounter}.
%KEEPCOMMENTS
%</faq>
%=================================================================================================================
%<*manual>
%<<KEEPCOMMENTS

\documentclass[12pt]{scrartcl}

%-----------------------------------------------------------------------------------------------------------------
% We need some more packages...
%
\usepackage[nottoc]{tocbibind}

\usepackage{textcomp}% Just for \textregistered. Comment out if you like.

\usepackage{url}

% The following package makes code look a little nicer, but it may not be present on all systems.

\IfFileExists{cmtt.sty}{\usepackage[override]{cmtt}}{}

% Loading the soul package enables the \highlighttext command.

\IfFileExists{soul.sty}{\usepackage{soul}}

% One more Text emphasis command...

\let\name=\textsc

% We input the xr package for external references.

\usepackage{xr-hyper}

%-----------------------------------------------------------------------------------------------------------------
% Load hyperref.
%
\usepackage[bookmarksopen,colorlinks]{hyperref}

%-----------------------------------------------------------------------------------------------------------------
% Finally, the texpower package is loaded. 
%
\usepackage{texpower}

%-----------------------------------------------------------------------------------------------------------------
% The configuration file allows user-specific settings.

\input{__TPcfg}

%-----------------------------------------------------------------------------------------------------------------
% Setting up indexing and custom indexing commands
\input{__TPindexing}

%-----------------------------------------------------------------------------------------------------------------
% We use references from the full demo file.

%\externaldocument{fulldemo}[http://texpower.sourceforge.net/doc/fulldemo.pdf]
\externaldocument{fulldemo}


%-----------------------------------------------------------------------------------------------------------------
% The code in the manual (doc tag) is meant for the document class seminar. Thus, it contains some 
% seminar-specific commands which are replaced by dummies here.

\let\newslide=\relax

%-----------------------------------------------------------------------------------------------------------------
% The following command produces the title for this document. 

\newcommand{\makeslidetitle}[1]
{%
  \hypersetup{pdftitle={#1}}
  \title{The \TeX Power bundle\\{\normalfont #1}}
  \author
  {%
    Stephan Lehmke\\\url{mailto:Stephan.Lehmke@udo.edu}
    \and
    Hans Fr.\ Nordhaug\\\url{mailto:hansfn@users.sourceforge.net}%
  }
  \maketitle

  \tableofcontents

  \subsubsection*{}
}


%-----------------------------------------------------------------------------------------------------------------
% As the documentation text contains a lot of stuff in boxes, we use \sloppy to avoid stuff projecting into the margin. 

\sloppy

%-----------------------------------------------------------------------------------------------------------------
% Finally, everything is set up. Here we go...
%

\begin{document}
%KEEPCOMMENTS
%</manual>
%=================================================================================================================
%<*fulldemo>
%<<KEEPCOMMENTS

% Enable all color emphasis and highlighting options; use a light background and slifonts.

\PassOptionsToPackage{coloremph,colormath,colorhighlight,lightbackground}{texpower}
\RequirePackage{tpslifonts}

% Input the generic preamble.

\input{__TPpreamble}
\hypersetup{pdftitle={texpower full demo and documentation}}

% To get correct links in the index (and remove annoying warnings)
\pausesafecounter{slide}

% Setting up indexing and custom indexing commands
\input{__TPindexing}

\backgroundstyle{vgradient}

\setlength{\unitlength}{5mm}

\ifthenelse{\boolean{psspecialsallowed}}% Can we use PSTricks?
{% Yes.
  % PsTricks is used for creating the picture example.

  \usepackage[noxcolor]{pstricks}
  \usepackage{pstcol}
  \usepackage{pst-node}
  
  \psset{unit=\unitlength}
  }
{% No. We'll make do without.
  }


% We also include an mps (metapost postscript) image...
\usepackage{graphicx}
% The mps extension isn't supported out-of-the-box for latex+dvips
\ifthenelse{\boolean{psspecialsallowed}}{%
\DeclareGraphicsExtensions{.mps}}{}

% ... and write some aligned equations.
\usepackage{amsmath}
% Make nested braces grow.
\delimitershortfall-1sp


% The package soul is needed for \highlighttext to work.

\usepackage{soul}


% The following package makes code look a little nicer, but it may not be present on all systems.

\IfFileExists{cmtt.sty}{\usepackage[override]{cmtt}}{}

% Hack to override seminar's definition of twocolumn needed by theindex
\makeatletter
\let\slidebox@restore@orig=\slidebox@restore%
\def\slidebox@restore{%
  \slidebox@restore@orig%
  \long\def\twocolumn[##1]{\section{Index}}%
}
\makeatother


%-----------------------------------------------------------------------------------------------------------------
% Finally, everything is set up. Here we go...
%
\begin{document}
\begin{slide}
%
% First, some words of explanation.
%
  \title
  {%
    The \TeX Power bundle\\
    {%
      \normalfont 
      Creating dynamic online presentations with \LaTeX\\
      Full demo and documentation for \TeX Power \tpversion%
      }%
    }

  \author{Stephan Lehmke\\\url{mailto:Stephan.Lehmke@cs.uni-dortmund.de}}

  \maketitle

  \newslide
  
  This document is a demonstration and manual for the \TeX Power bundle which allows to create dynamic
  presentations in a very flexible way.
  
  The heart of the bundle is the package \code{texpower} which implements some commands for presentation effects. This
  includes setting page transitions, color highlighting and displaying pages incrementally.
  
  All features of \code{texpower} are implemented entirely using \TeX{} and \LaTeX; they are meant for `online'
  presentation with \concept{Adobe Acrobat\textsuperscript{\textregistered} Reader} and work with all ways of \code{pdf}
  creation. The combination of \LaTeX{} + \code{dvips} + Acrobat Distiller / 
  \code{ps2pdf} is possible as well as pdf\LaTeX{} and other \code{pdf} creation 
  methods.

  \newslide

  \minisec{Disclaimer}
  This is still work inprogress.

  During the subsequent error correction and extension of the functionality, the syntax and implementation of the macros
  described here are liable to change. 

  Even though we are using dtx-files, these are still not fully documented dtx-files.

  \newslide

  \minisec{Credits}%
  I am indepted to \href{mailto:guntermann@iti.informatik.tu-darmstadt.de}{\name{Klaus Guntermann}}. His package
  \href{http://www-sp.iti.informatik.tu-darmstadt.de/software/ppower4/pp4sty.zip}{\code{texpause}} from the \concept{Pdf
    Presentation Post Processor} \href{http://www-sp.iti.informatik.tu-darmstadt.de/software/ppower4/}{PPower4} bundle
  is the basis for the code of \code{texpower}.
  
  Further thanks go to \href{mailto:dongen@cs.ucc.ie}{\name{Marc van Dongen}} for allowing me to include his code for
  page transitions and to \href{mailto:Martin.Schroeder@ACM.org}{\name{Martin Schr\"oder}} for permission to use his
  \concept{everyshi} code.
  
  Useful hints for error corrections and improvements of code have been provided by \name{Marc van Dongen},
  \name{Friedrich Eisenbrand}, \name{Thomas Emmel}, \name{Ross Moore}, \name{Heiko Oberdiek}, \name{Heiner Richter}, and
  \name{Robert J.\ Vanderbei}.

  \newslide

  \tableofcontents
\end{slide}
\begin{slide}
  %-----------------------------------------------------------------------------------------------------------------
  %
  \section{Examples}\label{Sec:Ex}
  First, two simple examples for the \macroname{pause} command.
  
  All other examples are meant to illustrate the expressive power of the \macroname{stepwise} command.
  
  Looking at the code for the examples will probably be the best way of understanding how certain effects can be
  achieved.

  \newslide

  %-----------------------------------------------------------------------------------------------------------------
  %
  \subsection{Some examples for \macroname{pause}}
  % It doesn't hurt to put \pause inside a paragraph, but a new paragraph is forced at each occurrence of \pause.
  \begin{center}
    a\pause b\pause c
  \end{center}

  \pause

  % We set Page Transitions to Dissolve for the rest of this slide.
  \pageTransitionDissolve
  \begin{itemize}
  \item foo\pause
  \item bar\pause
  \item baz
  \end{itemize}

  \newslide
  % Page Transitions are set back to Replace.
  \pageTransitionReplace

%-----------------------------------------------------------------------------------------------------------------
% The `titles' for the examples are simply subsection headings. 

\renewcommand{\makeslidetitle}[1]
{%
  \subsection{#1}%
}

%KEEPCOMMENTS
%</fulldemo>
%<*fulldemo-middle>
%<<KEEPCOMMENTS
\end{slide}

\begin{slide}\relax

%-----------------------------------------------------------------------------------------------------------------
% The `title' for the documentation is a section heading. As sectioning in the manual (doc-tag) file starts with 
% \section, we have to `downgrade' sectioning commands.  

\let\origsection=\section

\renewcommand{\makeslidetitle}[1]
{%
  \let\section\subsection
  \let\subsection\subsubsection
  \def\thanks##1{}%
  \origsection{#1}
}

  \addtocontents{toc}{\protect\clearpage}
  
%KEEPCOMMENTS
%</fulldemo-middle>
%=================================================================================================================
%<*docu>
%<<KEEPCOMMENTS

\makeslidetitle
{%
  Documentation%
  \thanks{Documentation for \TeX Power \tpversion .}%
  }%

The \TeX Power bundle contains style and class files for creating dynamic online presentations with \LaTeX. 

The heart of the bundle is the package \code{texpower.sty} which implements some commands for presentation effects. This
includes setting page transitions, color highlighting and displaying pages incrementally.

For finding out how to achieve special effects (as shown in the \nameref{Sec:Ex}), 
please look at the comments inside the files ending with \code{example.tex} and \code{demo.tex}
and read this manual to find out what's going on.

\newslide

For your own first steps with \TeX Power, the simple demo file \code{simpledemo.tex} is the best starting
place. There, some basic applications of the dynamic features provided by the \code{texpower} package are
demonstrated. You can make your own dynamic presentations by modifying that demo to your convenience.

\code{simpledemo.tex} uses the \code{article} document class for maximum
compatibility. There are other simple demos named
\code{slidesdemo.tex}, \code{foilsdemo.tex}, \code{seminardemo.tex}, 
\code{pp4sldemo.tex}, \code{pdfslidemo.tex}, \code{pdfscrdemo.tex}, 
\code{prosperdemo.tex}, and \code{ifmslidemo.tex} 
which demonstrate how to combine \TeX Power with the 
most popular presentation-making document classes and packages.

\newslide

The other, more sophisticated examples demonstrate the expressive power of the
\code{texpower} package. Look at the commented code of these examples to find out how to achieve special effects and
create your own presentation effects with \TeX Power.
  
\newslide

%-----------------------------------------------------------------------------------------------------------------
%
\section{Usage and general options}
The \code{texpower} package is loaded by putting
\begin{center}
  \present{\commandapp{usepackage}{texpower}}
\end{center}
into the preamble of a document.
  
There are no specific restrictions as to which document classes can be used. 

It should be stressed that \TeX Power is \underl{not} (currently) a complete presentation package. It just adds dynamic
presentation effects (and some other gimmicks specifically interesting for dynamic presentations) and should always be
combined with a document class dedicated to designing presentations (or a package like
\href{ftp://ftp.dante.de/tex-archive/help/Catalogue/entries/pdfslide.html}{\code{pdfslide}}).
  
Some of the presentation effects created by \code{texpower} require special capabilities of the viewer which is used for
presenting the resulting document. The target for the development of \code{texpower} has so far been 
\href{http://www.adobe.com/products/acrobat/readermain.html}%
{\concept{Adobe Acrobat\textsuperscript{\textregistered} Reader}}, which means the
document should (finally) be produced in \code{pdf} format. The produced
\code{pdf} documents should display well in 
\href{http://www.cs.wisc.edu/~ghost/gsview/}{\concept{GSview}} also, but that 
viewer doesn't support page transitions and duration.
  
There are no specific restrictions as to which way the \code{pdf} format is produced. All demos and examples
have been tested with pdf\LaTeX{} and standard \LaTeX, using 
\code{dvips} and \href{http://www.adobe.com/products/acrobat/}%
{\concept{Adobe Acrobat\textsuperscript{\textregistered} Distiller}}
or \code{dvips} and \code{ps2pdf} (from the \href{http://www.ghostscript.com/}%
{\concept{Ghostscript suite}}) for generating \code{pdf}.
  
\newslide
  
\subsection{General options}\label{Sec:GenOpt}
\begin{description}
\item[\present{option: \code{display}}.]\indexpckopt{texpower}{display} Enable `dynamic' features. If not set, it is assumed that the document is to be
  printed, and all commands for dynamic presentations, like \macroname{pause} or \macroname{stepwise} have no effect.

\item[\present{option: \code{printout} (default)}.]\indexpckopt{texpower}{printout} Disable `dynamic' features. As this is the default behaviour,
  setting this option explicitly is useful only if the option \code{display} is set by default for instance in the
  \code{tpoptions.cfg} file (see section \ref{Sec:Config}).

\item[\present{option: \code{verbose}}.]\indexpckopt{texpower}{verbose} Output some administrative info.
\end{description}
Some font options are listed in section \ref{Sec:BaseFont}.

\newslide

\subsection{Side effects of page contents duplication}\label{Sec:Dupl}
In the implementation of the \macroname{pause} and \macroname{stepwise} commands, it is neccessary to duplicate some
material on the page. 

This way, not only `visible' page contents will be duplicated, but also some `invisible' control code stored in
\concept{whatsits} (see the \TeX book for an explanation of this concept). Duplicating whatsits can lead to undesirable
side effects. 

For instance, a \macroname{section} command creates a whatsit for writing the table of contents entry. Duplicating this
whatsit will also duplicate the toc entry. So, whatsit items effecting file access are inhibited when duplicating page 
material.

\newslide

The current version of \code{texpower} is a little smarter when handling whatsits. Some commands (related to writing to 
files and hyperlinks) are made stepwise-aware. This means that links can point to the actual subpage where the
anchor is and not to the last (sub)page of an incremental page. However, if you want the old behaviour just use
\begin{description}
\item[\present{option: \code{oldfiltering}}]\indexpckopt{texpower}{oldfiltering} switches on the old
  (pre 0.2) very aggressive/robust filtering of whatsits.
\end{description}
The \code{oldfiltering} can be turned on and off inside the document using \macroname{oldfilteringon/off}. This command is
useful if \code{texpower} isn't smart enough...

\newslide

A second type of whatsits is created by \TeX's \macroname{special} command which is used for instance for color
management. Some drivers, like \code{dvips} and \code{textures}, use a color stack which is controlled by
\macroname{special} items included in the dvi file. When page contents are duplicated, then these \macroname{special}s
are also duplicated, which can seriously mess up the color stack.

\newslide

\code{texpower} implements a `color stack correction' method by maintaining a stack of color corrections, which should
counteract this effect. Owing to potential performance problems, this method is turned off by default. 
\begin{description}
\item[\present{option: \code{fixcolorstack}}]\indexpckopt{texpower}{fixcolorstack} switches on color stack correction. Use it if you experience strange color
  switches in your document.
\end{description}

\newslide

\subsection{Setting the base font}\label{Sec:BaseFont}
\code{texpower} offers no options for setting the base font of the document.
Use the \code{tpslifonts} package in stead. Read more in section \ref{Sec:TPslifonts}. 

Further, there are packages like \code{cmbright} or \code{beton} which change the whole set of fonts to something less
fragile than cmr.

\newslide

\subsection{Switches}
There are some boolean registers provided and set automatically by \code{texpower}.

\begin{description}
\item[\present{boolean: \code{psspecialsallowed}}]\indexpckswitch{texpower}{psspecialsallowed} True if PostScript\textsuperscript{\textregistered} specials may be
  used.
  
  \code{texpower} tries to find out whether or not PostScript\textsuperscript{\textregistered} specials may be used in
  the current document. For instance, pdf\LaTeX{} can't interpret arbitrary specials. This switch is set automatically
  and can be used inside a document to enable/disable parts which need PostScript\textsuperscript{\textregistered}
  specials.

\newslide

\item[\present{boolean: \code{display}}]\indexpckswitch{texpower}{display} True if \code{display} option was given.

  This switch indicates whether `dynamic' features of \code{texpower} are enabled. Use it inside your document
  to distinguish between the `presented' and the printed version of your document.
  
\item[\present{boolean: \code{TPcolor}}]\indexpckswitch{texpower}{TPcolor} True if any of the color highlighting options (see section
  \ref{Sec:ColorEmphasis}) were given.
  
  This switch indicates whether `color' features of \code{texpower} are enabled (compare section
  \ref{Sec:ColorEmphasis}). You can use it inside your document to distinguish between a `colored' and a `monochrome'
  version of your document.
\end{description}

\newslide

\subsection{Configuration files}\label{Sec:Config}
\code{texpower} loads three configuration files (if present):
\begin{description}
\item[\present{file: \code{tpoptions.cfg}}]\indexcode{tpoptions.cfg} 
  is loaded before options are processed. Can be used to set default options
  in a system-specific way. See the comments inside the file 
  \code{tpoptions.cfg} which is part of the \TeX Power bundle
  for instructions.
  
\item[\present{file: \code{tpsettings.cfg}}]\indexcode{tpsettings.cfg} 
  is loaded at the end of \code{texpower}. Here, you can do some
  system-specific settings. See the comments inside the
  file \code{tpsettings.cfg} which is part of the 
  \TeX Power bundle for instructions.

\item[\present{file: \code{tpcolors.cfg}}]\indexcode{tpcolors.cfg}
  is loaded if \code{TPcolor} is true. The file defines the standard 
  colors/colorsets (see section \ref{Sec:ColorEmphasis}). See the 
  comments inside the file \code{tpcolors.cfg} which is part of the 
  \TeX Power bundle for instructions.
\end{description}

\newslide

\subsection{Miscellaneous commands}\label{Sec:MiscCmd}
Some important commands that don't fit in the latter sections:
\begin{description}
\item[\present{\macroname{oldfilteringon}}]\indexmacro{oldfilteringon} 
  reverts to the old (pre v0.2) aggressive/robust filtering of whatsits.
\item[\present{\macroname{oldfilteringoff}}]\indexmacro{oldfilteringoff} 
  turns on the new better treatment of whatsits.
\item[\present{\commandapp{currentpagevalue}{\carg{value}}}]\indexmacro{currentpagevalue} 
  sets how to find the number of the current page, \commandapp{value}{page} is default.
  Used to name the hyper target on the first subpage of every page. Also used in the
  TeXPower navigation buttons.
\item[\present{\commandapp{pausesafecounter}{\carg{counter}}}]\indexmacro{pausesafecounter} 
  is used to add counters that are to be restored to their original value after \macroname{pause}. The \code{page}
  counter is always restored. In addition the \code{slide} counter is restored if the \code{seminar} class is used. If
  you need more counters to be restored after \macroname{pause}, use \macroname{pausesafecounter}.
\end{description}

\newslide

\subsection{Page Anchors}\label{Sec:PageAnch}

For each physical page \TeX Power (when in display mode) makes a number of subpages - this is
the dynamics. For convenience \TeX Power defines an anchor to the first subpage of physical page n,
\code{firstpage.n}\indexcode{firstpage.n}. The standard page anchor for physical page n,
\code{page.n}\indexcode{page.n}, points to the last subpage of physical page n. If you want to
link to any other subpage just insert a \macroname{hyperlink} in the standard way assuming you haven't
turned on the old filtering (\ref{Sec:Dupl}).

\newslide

\subsection{Dependencies on other packages}
\code{textpower} always loads the packages \code{ifthen} and \code{calc}, as the extended command syntax provided by
these is indispensable for the macros to work. They are in the \code{base} and \code{tools} area of the \LaTeX{}
distribution, respectively, so I hope they are available on all systems.

Furthermore, \code{texpower} loads the package \code{color} if any color-specific options are set (see section
\ref{Sec:ColorEmphasis}).

Further packages are \emph{not} loaded automatically by \code{texpower} to avoid incompatibilities, although some
features of \code{texpower} are enabled \emph{only} if a certain package is loaded. If you wish to use these features,
you are responsible for loading the respective package yourself. 

If some necessary package is \emph{not} loaded, \code{texpower} will issue a warning and disable the respective
features.

The following packages are neccessary for certain features of \code{texpower}:
\begin{description}
\item[\present{package: \code{hyperref}}]\indexfile{hyperref}{package} 
  is neccessary for page transition effects to work (see section 
  \ref{Sec:PageTrans}). 
  
  In particular, the \macroname{pageDuration} (see section \ref{Sec:PageDuration}) command only works if the version of
  hyperref loaded is at least v6.70a (where the \code{pdfpageduration} key was introduced).
  
  Commands which work only when \code{hyperref} is loaded are marked with \textbf{\textsf{h}} in the description.
  
\newslide

\item[\present{package: \code{soul}}]\indexfile{soul}{package} 
  is neccessary for the implementation of the commands \macroname{hidetext} and
  \macroname{highlighttext} (see section \ref{Sec:displaycustom}).
  
  Commands which work only when \code{soul} is loaded are marked with \textbf{\textsf{s}} in the description.
\end{description}

\newslide

\subsection{What else is part of the \TeX Power bundle?}
Besides the package \code{texpower} (which is described here), there are four
more packages, \code{tpslifonts}, \code{fixseminar}, \code{automata} and 
\code{tplists}, and one document class, \code{powersem}, in the \TeX Power bundle. 
Except for \code{tpslifonts} and \code{tplists} these files have no documentation 
of their own. They will be described in this section until they are turned 
into \code{dtx} files producing their own documentation.

See the file \code{00readme.txt} which is part of the \TeX Power bundle for a short description of all files.

\newslide

\minisec{The document class \code{powersem}}\indexfile{powersem}{class}
This is planned to provide a more `modern' version of \code{seminar} which can be used for creating dynamic
presentations. 

Currently, this document class doesn't do much more than load \code{seminar} and apply some fixes, but it is planned to
add some presentation-specific features (like navigation panels).

\newslide

There are three new options which are specific for \code{powersem}, all other options are passed to \code{seminar}:
\begin{description}
\item[\present{option: \code{display}}]\indexpckopt{powersem}{display} 
  Turns off all features of \code{seminar} (notes, vertical centering of slides)
  which can disturb dynamic presentations.

\item[\present{option: \code{calcdimensions}}]\indexpckopt{powersem}{calcdimensions}
  \code{seminar} automatically calculates the slide dimensions
  \macroname{slidewidth} and \macroname{slideheight} only for the default \code{letter} and for its own option
  \code{a4}. For all the other paper sizes which are possible with the \code{KOMA} option, the slide dimensions are not
  calculated automatically. 
  
  The \code{calcdimensions} option makes \code{powersem} calculate the slide dimensions automatically from paper size
  and margins.

\newslide

\item[\present{option: \code{truepagenumbers}}]\indexpckopt{powersem}{truepagenumbers}
  The truepagenumbers option makes powersem count pages with the counter page, independently of the counter slide. This
  enables proper working of TeXPowers navigation buttons (some of which calculate relative page numbers) even when the
  counter slide is reset frequently (for slide numberings of the type \verb|<l>.<n>.<m>|).
  
\item[\present{option: \code{KOMA}}]\indexpckopt{powersem}{KOMA} Makes 
  \code{seminar} load \code{scrartcl} (from the KOMA-Script bundle) instead of
  \code{article} as its base class. All new features of \code{scrartcl} are then available also for slides.

\item[\present{option: \code{UseBaseClass}}]\indexpckopt{powersem}{UseBaseClass} 
  Makes \code{seminar} load the class \macroname{baseclass} (initially \code{article}) instead of
  \code{article} as its base class. 

\item[\present{option: \code{reportclass}}]\indexpckopt{powersem}{reportclass}
  Makes \code{seminar} load the class \macroname{baseclass} (initially \code{report}) instead of
  \code{article}.

\item[\present{option: \code{bookclass}}]\indexpckopt{powersem}{bookclass}
  Makes \code{seminar} load the class \macroname{baseclass} (initially \code{book}) instead of
  \code{article}.
 
\end{description}

There is one change in \code{powersem} which will lead to incompatibilities with \code{seminar}. \code{seminar} has the
unfortunate custom of \emph{not} exchanging \macroname{paperwidth} and \macroname{paperheight} when making landscape
slides, as for instance \code{typearea} and \code{geometry} do.

This leads to problems with setting the paper size for \code{pdf} files, as done for instance by the \code{hyperref}
package. 

\code{powersem} effectively turns off \code{seminar}'s papersize management and leaves this to the base class (with the
pleasant side effect that you can use e.\,g.\ \commandapp[KOMA,a0paper]{documentclass}{powersem} for making posters). 

In consequence, the \code{portrait} option of \code{seminar} is turned on by \code{powersem} to avoid confusing
\code{seminar}. You have to explicitly use the \code{landscape} option (and a base class or package which understands
this option) to get landscape slides with powersem.

\newslide

\minisec{The package \code{fixseminar}}\indexfile{fixseminar}{package}
Unfortunately, there are some fixes to seminar which can \emph{not} be applied in \code{powersem} because they have to
be applied after \code{hyperref} is loaded (if this package should be loaded).

The package \code{fixseminar} applies these fixes, so this package should be loaded after \code{hyperref} (if
\code{hyperref} is loaded at all, otherwise \code{fixseminar} can be loaded anywhere in the preamble).

\newslide

It applies two fixes:
\begin{itemize}
\item In case \code{pdflatex} is being run, the lengths \macroname{pdfpageheight} and \macroname{pdfpagewidth} have to
  be set in a `magnification-sensitive' way.

\item \code{hyperref} introduces some code at the beginning of every page which can produce spurious vertical space,
  which in turn disturbs building dynamic pages. This code is `fixed' so it cannot produce vertical space.
\end{itemize}

\newslide

\minisec{The package \code{tpslifonts}}
\indexfile{tpslifonts}{package}\label{Sec:TPslifonts}
Presentations to be displayed `online' with a video beamer have special needs concerning font configuration owing to low
`screen' resolution and bad contrast caused by possibly bad light conditions combined with color highlighting.

This package tries to cater to these needs by offering a holistic configuration of all document fonts, including text,
typewriter, and math fonts. Special features are `smooth scaling' of Type1 fonts and careful design size selection for
optimal readability.

\newslide

For more information on package options and used fonts (and on implementation) read the documentation coming with the
package - check the \code{tpslifonts} directory.

\minisec{The package \code{automata}}\indexfile{automata}{package}
Experimental package for drawing automata in the sense of theoretical computer 
science (using PSTricks) and animating them with TeXPower.  Only DFA and Mealy 
automata are supported so far.

\newslide

\minisec{The package \code{tplists}}\indexfile{tplists}{package}
Experimental package providing easy dynamic lists. Currently there are stepped, flipped and dimmed versions of itemize
and enumerate (and corresponding lists from the \code{eqlist} and \code{paralist} package).  For more information
and an example, compile (and then read) the file \code{tplists.dtx}.

\newslide

%-----------------------------------------------------------------------------------------------------------------
%
\section{The \macroname{pause} command}\label{Sec:pause}
\present{\macroname{pause}}\indexmacro{pause} is derived from the \macroname{pause} command from the package
\href{http://www-sp.iti.informatik.tu-darmstadt.de/software/ppower4/pp4sty.zip}{\code{texpause}} which is part of the
\href{http://www-sp.iti.informatik.tu-darmstadt.de/software/ppower4/}{PPower4 suite} by
\href{mailto:guntermann@iti.informatik.tu-darmstadt.de}{Klaus Guntermann}.
  
  It will ship out the current page, start a new page and copy whatever was on the current page onto the new page, where
  typesetting is resumed.

  \ifthenelse{\boolean{display}}{Here, let's do a little demo\pause}{}

  This will create the effect of a \concept{pause} in the presentation, i.\,e.\ the presentation stops because the
  current page ends at the point where the \macroname{pause} command occurred and is resumed at this point when the
  presenter switches to the next page. 

  \newslide

  \minisec{Things to pay attention to}
  \begin{enumerate}
  \item \macroname{pause} should appear in \concept{vertical mode} only, i.\,e.\ between paragraphs or at places where
    ending the current paragraph doesn't hurt.
    
  \item This means \macroname{pause} is forbidden in all \concept{boxed} material (including \code{tabular}),
    \concept{headers/footers}, and \concept{floats}.

  \item \macroname{pause} shouldn't appear either in environments which have to be \emph{closed} to work properly, like
    \code{picture}, \code{tabbing}, and (unfortunately) environments for \concept{aligned math formulas}.

  \item \macroname{pause} does work in all environments which mainly influence paragraph formatting, like \code{center},
    \code{quote} or all \concept{list} environments.

    \newslide

  \item \macroname{pause} doesn't really have problems with automatic page breaking, but beware of \emph{overfull}
    pages/slides. In this case, it may occur that only the last page(s)/slide(s) of a sequence are overfull, which
    changes vertical spacing, making lines `wobble' when switching to the last page/slide of a sequence.
    
    \newslide
    
  \item The duplication of page material done by \macroname{pause} can lead to unwanted side effects. See section
    \ref{Sec:Dupl} for further explanations. In particular, if you should experience strange color switches when using
    \macroname{pause} (and you are \underl{not} using \code{pdftex}), turn on color stack correction with the option
    \code{fixcolorstack}. In addition you should be aware of \macroname{pausesafecounter}, see section
    \ref{Sec:MiscCmd}.

  \end{enumerate}
  
  A lot of the restrictions for the use of pause can be avoided by using \macroname{stepwise} (see next section).
  
  \newslide

  %-----------------------------------------------------------------------------------------------------------------
  %
  \section{The \macroname{stepwise} command}
  \present{\commandapp{stepwise}{\carg{contents}}}\indexmacro{stepwise} is a command for displaying some part of a \LaTeX{} document (which
  is contained in \carg{contents}) `step by step'. As of itself, \macroname{stepwise} doesn't do very much. If
  \carg{contents} contains one or more constructs of the form \present{\commandapp{step}{\carg{stepcontents}}}\indexmacro{step}, the
  following happens:
  \begin{enumerate}
  \item The current contents of the page are saved (as with \macroname{pause}).

  \item As many pages as there are \macroname{step} commands in \carg{contents} are produced.

    Every page starts with what was on the current page when \macroname{stepwise} started.

    \newslide
    
    The first page also contains everything in \carg{contents} which is \emph{not} in \carg{stepcontents} for any
    \macroname{step} command.

    The second page additionally contains the \carg{stepcontents} for the \emph{first} \macroname{step} command, and so
    on, until all \carg{stepcontents} are displayed.

  \item When all \carg{stepcontents} are displayed, \macroname{stepwise} ends and typesetting is resumed (still on the
    current page).
  \end{enumerate}
  
  \parstepwise{This will create the effect that the \macroname{step} commands are executed `\step{step} \step{by}
    \step{step}'.}

  \newslide

  \minisec{Things to pay attention to}
  \begin{enumerate}
  \item \macroname{stepwise} should appear in \concept{vertical mode} only, i.\,e.\ between paragraphs, just like
    \macroname{pause}.

  \item Don't put \macroname{pause} or nested occurrences of \macroname{stepwise} into \carg{contents}.
    
  \item Structures where \macroname{pause} does not work (like \code{tabular} or aligned equations) can go
    \emph{completely} into \carg{contents}, where \macroname{step} can be used freely (see \nameref{Sec:Ex}).
    
  \item As \carg{contents} is read as a macro argument, constructs involving \concept{catcode} changes (like
    \macroname{verb} or language switches) won't work in \carg{contents} \textbf{unless} you use the
    \code{fragilesteps} environment (\ref{Sec:fragilesteps}).

\newslide

  \item Several instances of \macroname{stepwise} may occur on one page, also combined with \macroname{pause} (outside
    of \carg{contents}).
    
    But beware of page breaks in \carg{contents}. This will really mess things up. 
    
    Overfull pages/slides are also a problem, just like with \macroname{pause}. See the description of \macroname{pause}
    (section \ref{Sec:pause}) concerning this and also concerning side effects of duplicating page material.

  \item \macroname{step} can go in \carg{stepcontents}. The order of execution of \macroname{step} commands is just the
    order in which they appear in \carg{contents}, independent of nesting within each other.
    
    \newslide
    
  \item As \carg{contents} is executed several times, \LaTeX{} constructs changing \concept{global counters}, accessing
    \concept{files} etc.\ are problematic. This concerns sections, numbered equations, labels, hyperlinks and the like.
    
    Counters are taken care of explicitly by \macroname{stepwise}, so equation numbers are no problem. 

    Commands accessing toc files and such (like \macroname{section}) are taken care of by the whatsit suppression
    mechanism (compare section \ref{Sec:Dupl}).
  \end{enumerate}

  \newslide
    
  \subsection{\code{fragilesteps} environment}\label{Sec:fragilesteps}%

  The \code{fragilesteps}\indexcode{fragilesteps} environment is a wrapper around \macroname{stepwise}
  that makes it possible to use verbatim. The code for this environment is based on similar code from beamer - an
  excellent presentation class written by Till Tantau - thanks! Using the \code{fragilesteps} environment
  enables the use of the \code{listings} package to display code line by line. There are some examples in
  \code{verbexample.tex}.

  \newslide

  \subsection{\macroname{boxedsteps} and \macroname{nonboxedsteps}}\label{Sec:boxedsteps}%
  By default, \carg{stepcontents} belonging to a \macroname{step} which is not yet `active' are ignored altogether. This
  makes it possible to include e.\,g.\ tabulators \code{\&} or line breaks into \carg{stepcontents} without breaking
  anything.
  
  Sometimes, however, this behaviour is undesirable, for instance when stepping through an equation `from outer to
  inner', or when filling in blanks in a paragraph. Then, the desired behaviour of a \macroname{step} which is not yet
  `active' is to create an appropriate amount of \emph{blank space} where \carg{stepcontents} can go as soon as it is
  activated.

  \newslide
  
  The simplest and most robust way of doing this is to create an empty box (aka \macroname{phantom}) with the same
  dimensions as the text to be hidden.

  This behaviour is toggled by the following commands. See section \ref{Sec:displaycustom} for more sophisticated
  (albeit more fragile) variants.
  \begin{description}
  \item[\present{\macroname{boxedsteps}}]\indexmacro{boxedsteps} makes \macroname{step} create a blank box the size of \carg{stepcontents} when
    inactive and put \carg{stepcontents} into a box when active.
  \item[\present{\macroname{nonboxedsteps}}]\indexmacro{nonboxedsteps} makes \macroname{step} ignore \carg{stepcontents} when inactive and leave
    \carg{stepcontents} alone when active (default).
  \end{description}

  \newslide

  \minisec{Things to pay attention to}
  \begin{enumerate}
  \item The settings effected by \macroname{boxedsteps} and \macroname{nonboxedsteps} are \emph{local}, i.\,e.\ whenever
    a group closes, the setting is restored to its previous value.

  \item Putting stuff into boxes can break things like tabulators (\code{\&}). It can also mess up math spacing, which
    then has to be corrected manually. Compare the following examples:

    \parstepwise{%
      \begin{displaymath}
        \left(\frac{a+b}{c}\right)\qquad\left(\frac{a\step{+b}}{c}\right)\qquad\left(\frac{a\restep{{}+b}}{c}\right)
      \end{displaymath}%
      }%
  \end{enumerate}

  \newslide

  \subsection{Custom versions of \macroname{stepwise}}%
  Sometimes, it might happen that vertical spacing is different on every page of a sequence generated by
  \macroname{stepwise}, making lines `wobble'. This is usually fixed if you use \macroname{liststepwise} or
  \macroname{parstepwise} (described below) in stead of \macroname{stepwise}.

  \newslide

  There are two custom versions of \macroname{stepwise} which should produce better vertical spacing.
  \begin{description}
  \item[\present{\commandapp{liststepwise}{\carg{contents}}}]
    \indexmacro{liststepwise} works exactly like \macroname{stepwise}, but adds 
    an `invisible rule' before \carg{contents}. Use for list environments and 
    aligned equations.
  \item[\present{\commandapp{parstepwise}{\carg{contents}}}]
    \indexmacro{parstepwise} works like \macroname{liststepwise}, but
    \macroname{boxedsteps} is turned on by default. Use for texts where 
    \macroname{step}s are to be filled into blank spaces.
  \end{description}

  \newslide

  \subsection{Starred versions of \macroname{stepwise} commands}\label{Sec:StarredStepwise}%
  Usually, the first page of a sequence produced contains \emph{only} material which is \emph{not} part of any
  \carg{stepcontents}. The first \carg{stepcontents} are displayed on the second page of the sequence.

  For special effects (see example \ref{Sec:Exhl}), it might be desirable to have the first \carg{stepcontents} active
  even on the first page of the sequence.

  All variants of  \macroname{stepwise} have a starred version (e.\,g.\ \macroname{stepwise*}) which does exactly that.

  \newslide

  \subsection{The optional argument of \macroname{stepwise}}%
  Every variant of \macroname{stepwise} takes an optional argument, like this
  \begin{center}
    \present{\commandapp[\carg{settings}]{stepwise}{\carg{contents}}}.
  \end{center}
  \carg{settings} will be placed right before the internal loop which produces the sequence of pages.  It can 
  contain settings of parameters which modify the behaviour of \macroname{stepwise} or \macroname{step}. \carg{settings}
  is placed inside a group so all changes are local to this call of \macroname{stepwise}.

  Some internal macros and counters which can be adjusted are explained in the following.

  \newslide

  \subsection{Customizing the way \carg{stepcontents} is diplayed}\label{Sec:displaycustom}%
  Internally, there are three macros (taking one argument each) which control how \carg{stepcontents} is displayed:
  \macroname{displaystepcontents}\indexmacro{displaystepcontents}, 
  \macroname{hidestepcontents}\indexmacro{hidestepcontents}, and 
  \macroname{activatestep}\indexmacro{activatestep}. Virtually, every
  \commandapp{step}{\carg{stepcontents}} is replaced by
  \begin{description}
  \item[\present{\commandapp{hidestepcontents}{\carg{stepcontents}}}]\mbox{}\\ when this step is not yet active.
  \item[\present{\commandapp{displaystepcontents}{\commandapp{activatestep}{\carg{stepcontents}}}}] when this step is
    activated \emph{for the first time}.
  \item[\present{\commandapp{displaystepcontents}{\carg{stepcontents}}}]\mbox{}\\
    when this step has been activated before.
  \end{description}
  
  By redefining these macros, the behaviour of \macroname{step} is changed accordingly. You can redefine them inside
  \carg{contents} to provide a change affecting one \macroname{step} only, or in the optional argument of
  \macroname{stepwise} to provide a change for all \macroname{step}s inside \carg{contents}.
  
  In the \nameref{Sec:Ex}, it is demonstrated how special effects can be achieved by redefining these macros.

  \macroname{activatestep} is set to \macroname{displayidentical} by default, the default settings of
  \macroname{hidestepcontents} and \macroname{displaystepcontents} depend on whether \macroname{boxedsteps} or
  \macroname{nonboxedsteps} (default) is used.
  
  \newslide

  \code{texpower} offers nine standard definitions.

  For interpreting \macroname{displaystepcontents}:
  \begin{description}
  \item[\present{\macroname{displayidentical}}]\indexmacro{displayidentical} 
    Simply expands to its argument. The same as \LaTeX s
    \macroname{@ident}. Used by \macroname{nonboxedsteps} (default).

  \item[\present{\macroname{displayboxed}}]\indexmacro{displayboxed}
    Expands to an \macroname{mbox} containing its argument. Used by
    \macroname{boxedsteps}.
  \end{description}

  \newslide

  For interpreting \macroname{hidestepcontents}:
  \begin{description}
  \item[\present{\macroname{hideignore}}]\indexmacro{hideignore}
    Expands to nothing. The same as \LaTeX s \macroname{@gobble}. Used by
    \macroname{nonboxedsteps} (default).

  \item[\present{\macroname{hidephantom}}]\indexmacro{hidephantom}
    Expands to a \macroname{phantom} containing its argument. Used by
    \macroname{boxedsteps}.
    
  \item[\present{\macroname{hidevanish}}]\indexmacro{hidevanish}
    In a colored document, makes its argument `vanish' by setting all colors to
    \macroname{vanishcolor} (defaults to \code{pagecolor}; compare section 
    \ref{Sec:Colorcommands}). Note that this will give weird results with 
    structures backgrounds.

    For monochrome documents, there is no useful interpretation for this command, so it is disabled.

    \newslide
    
  \item[{\present[s]{\macroname{hidetext}}}]\indexmacro{hidetext}
    Produces blank space of the same dimensions as the space that would be
    occupied if its argument would be typeset in the current paragraph. Respects automatic hyphenation and line breaks.
    
    This command needs the \href{ftp://ftp.dante.de/tex-archive/help/Catalogue/entries/soul.html}{\code{soul}} package
    to work, which is not loaded by \code{texpower} itself. Consult the documentation of
    \href{ftp://ftp.dante.de/tex-archive/help/Catalogue/entries/soul.html}{\code{soul}} concerning restrictions on
    commands implemented using \code{soul}. If you don't load the \code{soul} package yourself, there is no useful
    definition for this command, so it defaults to \macroname{hidephantom}.

    \newslide
    
  \item[\present{\macroname{hidedimmed}}]\indexmacro{hidedimmed}
    In a colored document, displays its argument with dimmed colors (compare
    section \ref{Sec:Colors}). Note that this doesn't make the argument completely invisible.

    For monochrome documents, there is no useful interpretation for this command, so it is disabled.
  \end{description}

  \newslide

  For interpreting \macroname{activatestep}:
  \begin{description}
  \item[\present{\macroname{highlightboxed}}]\indexmacro{highlightboxed}
    If the \code{colorhighlight} option (see section \ref{Sec:ColorEmphasis})
    is set, expands to a \highlightboxed{box with colored background} containing its argument. Otherwise, expands to an
    \macroname{fbox} containing its argument. It is made sure that the resulting box has the same dimensions as the
    argument (the outer frame may overlap surrounding text).
    
    There is a new length register \present{\macroname{highlightboxsep}}
    \indexmacro{highlightboxsep} which acts like \macroname{fboxsep} for the
    resulting box and defaults to \code{0.5\macroname{fboxsep}}.
    
    \newslide
    
  \item[{\present[s]{\macroname{highlighttext}}}]\indexmacro{highlighttext}
    If the \code{colorhighlight} option (see section \ref{Sec:ColorEmphasis}) 
    is set, puts its argument \highlighttext{on colored background}. Otherwise, underlines its
    argument. It is made sure that the resulting text has the same dimensions as the argument (the outer frame may
    overlap surrounding text).
    
    \present{\macroname{highlightboxsep}} is used to determine the extent of the coloured box(es) used as background. 
    
    This command needs the \href{ftp://ftp.dante.de/tex-archive/help/Catalogue/entries/soul.html}{\code{soul}} package
    to work (compare the description of \macroname{hidetext}). If you don't load the \code{soul} package yourself, there
    is no useful definition for this command, so it is disabled.
    
    \newslide
    
  \item[\present{\macroname{highlightenhanced}}]\indexmacro{highlightenhanced}
    In a colored document, displays its argument \highlightenhanced{with
    enhanced colors} (compare section \ref{Sec:Colors}).

    For monochrome documents, there is no useful interpretation for this command, so it is disabled.
  \end{description}

  \newslide

  \subsection{Variants of \macroname{step}}
  There are a couple of custom versions of \macroname{step} which make it easier to achieve special effects needed
  frequently.
  \begin{description}
  \item[\present{\macroname{bstep}}]\indexmacro{bstep}
    Like \macroname{step}, but is \emph{always} boxed (see section
    \ref{Sec:boxedsteps}). \commandapp{bstep}{\carg{stepcontents}} is 
    implemented in principle as
    \code{\{\macroname{boxedsteps}\commandapp{step}{\carg{stepcontents}}\}}.
    
    In aligned equations where \macroname{stepwise} is used for being able to put tabulators into \carg{stepcontents},
    but where nested occurrences of \macroname{step} should be boxed to assure correct sizes of growing braces or such,
    this variant of \macroname{step} is more convenient than using \macroname{boxedsteps} for every nested occurrence
    of \macroname{step}.
    
  \item[\present{\commandapp{switch}{\carg{ifinactive}\}\{\carg{ifactive}}}]
    \indexmacro{switch} is a variant of \macroname{step} which,
    instead of making its argument appear, switches between \carg{ifinactive} 
    and \carg{ifactive} when activated.

    In fact, \commandapp{step}{\carg{stepcontents}} is in principle implemented by
    \begin{tabbing}
      \macroname{switch}\=\code{\{\commandapp{hidestepcontents}{\carg{stepcontents}}\}}\\
      \>\code{\{\commandapp{displaystepcontents}{\carg{stepcontents}}\}}
    \end{tabbing}
    
    This command can be used, for instance, to add an \macroname{underbrace} to a formula, which is difficult using
    \macroname{step}. 
    
    Beware of problems when \carg{ifinactive} and \carg{ifactive} have different dimensions.

    \newslide
    
  \item[\present{\macroname{dstep}}]\indexmacro{dstep}
    A variant of \macroname{step} which takes \underl{no} argument, but simply 
    switches colors to `dimmed' (compare section \ref{Sec:Colors}) if not 
    active. Not that the scope of this color change will
    last until the next outer group closes. This command does nothing in a monochrome document.
    
  \item[\present{\macroname{vstep}}]\indexmacro{vstep}
    A variant of \macroname{step} which takes \underl{no} argument, but simply 
    switches all colors to \macroname{vanishcolor} (defaults to 
    \code{pagecolor}; compare section \ref{Sec:Colorcommands}) if not
    active. Not that the scope of this color change will last until the next outer group closes. This command does
    nothing in a monochrome document.
    
  \item[\present{\macroname{steponce}}]\indexmacro{steponce}
    % \steponce[<activatefirst>]{<stepcontents>}               
    Like \macroname{step}, but goes inactive again in the subsequent step.
    
  \item[\present{\macroname{multistep}}]\indexmacro{multistep}
   is a shorthand macro for executing several steps successively. In
   fact, it would better be called \macroname{multiswitch}, because it's 
   functionality is based on \macroname{switch}, it only acts like a 
   (simplified) \macroname{step} command which is executed `several times'.
   The syntax is
   \begin{center}
   \commandapp[\carg{activatefirst}]{multistep}{\carg{n}\}\{\carg{stepcontents}}
   \end{center}
   where \carg{n} is the number of steps. Only one instance of 
   \carg{stepcontents} is displayed at a time. Inside \carg{stepcontents}, a 
   counter \code{substep} can be evaluated which tells the number of the 
   current instance. In the starred form the last instance of 
   \carg{stepcontents} stays visible.
    
  \item[\present{\macroname{movie}}]\indexmacro{movie}
    works like \macroname{multistep}, but between \macroname{steps}, pages are
    advanced automatically every \carg{dur} seconds. The syntax is
   \begin{center}
   \commandapp{movie}{\carg{n}\}\{\carg{dur}\}[\carg{stop}]\{\carg{stepcontents}}
   \end{center}
   where \carg{n} is the number of steps. The additional optional argument 
   \carg{stop} gives the code (default: \macroname{stopAdvancing})
   \indexmacro{stopAdvancing} which stops the animation.
   (\macroname{movie} accepts the same first optional argument as 
   \macroname{multistep} but it was left out above.)
    
  
  \item[\present{\macroname{overlays}}]\indexmacro{overlays}
    is another shorthand macro for executing several steps successively. In 
    contrast to \macroname{multistep}, it doesn't print things \emph{after} 
    each other, but \emph{over} each other. The syntax is 
    \begin{center}
    \commandapp[\carg{activatefirst}]{overlays}{\carg{n}\}\{\carg{stepcontents}}
    \end{center}
    where \carg{n} is the number of steps. Inside \carg{stepcontents}, a 
    counter \code{substep} can be evaluated which tells the number of the 
    current instance.
    \newslide
   
  \item[\small%
    \present{\macroname{restep}}\indexmacro{restep}, 
    \present{\macroname{rebstep}}\indexmacro{rebstep}, 
    \present{\macroname{reswitch}}\indexmacro{reswitch},
    \present{\macroname{redstep}}\indexmacro{redstep}, 
    \present{\macroname{revstep}}\indexmacro{revstep}.]\mbox{}\\
    Frequently, it is desirable for two or more steps to appear at the same 
    time, for instance to fill in arguments at
    several places in a formula at once (see example \ref{Sec:ExEq}).
    
    \present{\commandapp{restep}{\carg{stepcontents}}} is identical with \commandapp{step}{\carg{stepcontents}}, but is
    activated at the same time as the previous occurrence of \macroname{step}.
    
    \present{\macroname{rebstep}}, \present{\macroname{reswitch}}, \present{\macroname{redstep}}, and
    \present{\macroname{revstep}} do the same for \macroname{bstep}, \macroname{switch}, \macroname{dstep}, and
    \macroname{vstep}.
  \end{description}

  \newslide

  \subsection{Optional arguments of \macroname{step}}%
  Sometimes, letting two \macroname{step}s appear at the same time (with \macroname{restep}) is not the only desirable
  modification of the order in which \macroname{step}s appear. \macroname{step}, \macroname{bstep} and
  \macroname{switch} take two optional arguments for influencing the mode of activation, like this:
  \begin{center}
    \present{\commandapp[{\carg{activatefirst}][\carg{whenactive}}]{step}{\carg{stepcontents}}}.
  \end{center}
  Both \carg{activatefirst} and \carg{whenactive} should be conditions in the syntax of the \macroname{ifthenelse}
  command (see the documentation of the
  \href{ftp://ftp.dante.de/tex-archive/help/Catalogue/entries/ifthen.html}{\code{ifthen}} package for details).
  
  \newslide

  \present{\carg{activatefirst}} checks whether this \macroname{step} is to be activated \emph{for the first time}. The
  default value is \present{\commandapp{value}{step}=\commandapp{value}{stepcommand}} (see section \ref{Sec:Internals}
  for a list of internal values). By using \commandapp{value}{step}=\carg{$n$}, this \macroname{step} can be forced to
  appear as the $n$th one. See example \ref{Sec:ExPar} for a demonstration of how this can be used to make
  \macroname{step}s appear in arbitrary order.
  
  \present{\carg{whenactive}} checks whether this \macroname{step} is to be considered active \emph{at all}. The default
  behaviour is to check whether this \macroname{step} has been activated before (this is saved internally for every
  step). See example \ref{Sec:ExFooling} for a demonstration of how this can be used to make \macroname{step}s appear
  and disappear after a defined fashion.
  
  \minisec{If you know what you're doing\dots}
  Both optional arguments allow two syntctical forms: 
  \begin{enumerate}
  \item enclosed in square brackets \code{[]} like explained above.
  \item enclosed in braces \code{()}. In this case, \carg{activatefirst} and \carg{whenactive} are \emph{not} treated as
    conditions in the sense of \macroname{ifthenelse}, but as conditionals like those used internally by \LaTeX. That
    means, \carg{activatefirst} (when enclosed in braces) can contain arbitrary \TeX{} code which then takes two
    arguments and expands to one of them, depending on whether the condition is fulfilled or not fulfilled. For
    instance, \commandapp[{\carg{activatefirst}}]{step}{\carg{stepcontents}} could be replaced by
    \macroname{step}\code{(\commandapp{ifthenelse}{\carg{activatefirst}})\{\carg{stepcontents}\}}.

    See example \ref{Sec:ExBackwards} for a simple application of this syntax.
  \end{enumerate}

  Internally, the default for the treatment of \carg{whenactive} is \code{(\macroname{if@first@TP@true})}, where
  \macroname{if@first@TP@true} is an internal condition checking whether this \macroname{step} has been activated before.

  \newslide

  \subsection{Finding out what's going on}\label{Sec:Internals}%
  Inside \carg{settings} and \carg{contents}, you can refer to the following internal state variables which provide
  information about the current state of the process executed by \macroname{stepwise}:
  \begin{description}
  \item[\present{counter: \code{firststep}}]
    \indexstepwise{firststep}{counter}
    The number from which to start counting steps (see counter \code{step}
    below). Is $0$ by default and $1$ for starred versions (section \ref{Sec:StarredStepwise}) of \macroname{stepwise}.
    You can set this in \carg{settings} for special effects (see example \ref{Sec:ExBackwards}).

  \item[\present{counter: \code{totalsteps}}]
    \indexstepwise{totalsteps}{counter}
    The total number of \macroname{step} commands occurring in \carg{contents}.
    
    \newslide

  \item[\present{counter: \code{step}}]
    \indexstepwise{step}{counter}
    The number of the current iteration, i.\,e.\ the number of the current page in
    the sequence of pages produced by \macroname{stepwise}. Runs from \commandapp{value}{firststep} to
    \commandapp{value}{totalsteps}.

  \item[\present{counter: \code{stepcommand}}]
    \indexstepwise{stepcommand}{counter}
    The number of the \macroname{step} command currently being executed.

  \item[\present{boolean: \code{firstactivation}}]
    \indexstepwise{firstactivation}{boolean}
    \code{true} if this \macroname{step} is active for the first time,
    \code{false} otherwise.

  \item[\present{boolean: \code{active}}]\indexstepwise{active}{boolean}
    \code{true} if this \macroname{step} is currently active, \code{false}
    otherwise.
  \end{description}
  \code{stepcommand}, \code{firstactivation}, and \code{active} are useful only inside \carg{stepcontents}.


  \newslide

  \subsection{\macroname{afterstep}}\indexmacro{afterstep}%
  It might be neccessary to set some parameters which affect the appearance of the \emph{page} (like page transitions)
  inside \carg{stepcontents}. However, the \macroname{step} commands are usually placed deeply inside some structure, so
  that all \emph{local} settings are likely to be undone by groups closing before the page is completed.

  \present{\commandapp{afterstep}{\carg{settings}}} puts \carg{settings} right before the end of the page, after the
  current step is performed.

  \newslide

  \minisec{Things to pay attention to}
  \begin{enumerate}
  \item There can be only one effective value for \carg{settings}. Every occurrence of \macroname{afterstep} overwrites
    this value globally.
    
  \item \macroname{afterstep} will \emph{not} be executed in \carg{stepcontents} if the corresponding \macroname{step}
    is not active, even if \carg{stepcontents} is displayed owing to a redefinition of \macroname{hidestepcontents},
    like in example \ref{Sec:Exhl}.
    
  \item As \carg{settings} is put immediately before the page break, there is no means of restoring the original value
    of whatever has been set. So if you set something via \macroname{afterstep} and want it to be reset in some later
    step, you have to reset it explicitly with another call of \macroname{afterstep}.
  \end{enumerate}

  \newslide
  
  %-----------------------------------------------------------------------------------------------------------------
  %
  \section{Page transitions and automatic advancing}\label{Sec:PageTrans}
  \subsection{Page transitions}
  I am indepted to \href{mailto:dongen@cs.ucc.ie}{\name{Marc van Dongen}} for allowing me to include a suite of commands
  written by him and posted to the \href{http://www-sp.iti.informatik.tu-darmstadt.de/software/ppower4/}{PPower4}
  mailing list which set page transitions (using
  \href{ftp://ftp.dante.de/tex-archive/help/Catalogue/entries/hyperref.html}{\code{hyperref}s} \macroname{hypersetup}).

  These commands work only if the \code{hyperref} package is loaded.

    \newslide

  The following page transition commands are defined:\pause
  \begin{description}
  \item[{\present[h]{\macroname{pageTransitionSplitHO}}}]
    \indexmacro{pageTransitionSplitHO}
    Split Horizontally to the outside. \pageTransitionSplitHO\pause

  \item[{\present[h]{\macroname{pageTransitionSplitHI}}}]
    \indexmacro{pageTransitionSplitHI}
    Split Horizontally to the inside. \pageTransitionSplitHI\pause

  \item[{\present[h]{\macroname{pageTransitionSplitVO}}}]
    \indexmacro{pageTransitionSplitVO}
    Split Vertically to the outside. \pageTransitionSplitVO\pause

  \item[{\present[h]{\macroname{pageTransitionSplitVI}}}]
    \indexmacro{pageTransitionSplitVI}
    Split Vertically to the inside. \pageTransitionSplitVI\pause

  \item[{\present[h]{\macroname{pageTransitionBlindsH}}}]
    \indexmacro{pageTransitionBlindsH}
    Horizontal Blinds. \pageTransitionBlindsH\pause

  \item[{\present[h]{\macroname{pageTransitionBlindsV}}}]
    \indexmacro{pageTransitionBlindsV}
    Vertical Blinds. \pageTransitionBlindsV

    \newslide

  \item[{\present[h]{\macroname{pageTransitionBoxO}}}]
    \indexmacro{pageTransitionBoxO}
    Growing Box. \pageTransitionBoxO\pause

  \item[{\present[h]{\macroname{pageTransitionBoxI}}}]
    \indexmacro{pageTransitionBoxI}
    Shrinking Box. \pageTransitionBoxI\pause
    
  \item[{\present[h]{\commandapp{pageTransitionWipe}{\carg{angle}}}}]
    \indexmacro{pageTransitionWipe}\mbox{}\\
    Wipe from one edge of the page to the facing edge.
    
    \stepwise
    {%
      \carg{angle} is a number between $0$ and $360$ which specifies the direction (in degrees) in which to wipe.
    
      Apparently, only the values \afterstep{\pageTransitionWipe{0}}$0$, \step{\afterstep{\pageTransitionWipe{90}}$90$,}
      \step{\afterstep{\pageTransitionWipe{180}}$180$,} \step{$270$ are supported.}%
      }%
    \pageTransitionWipe{270}\pause

  \item[{\present[h]{\macroname{pageTransitionDissolve}}}]
    \indexmacro{pageTransitionDissolve}
    Dissolve. \pageTransitionDissolve

    \newslide
    
  \item[{\present[h]{\commandapp{pageTransitionGlitter}{\carg{angle}}}}]
    \indexmacro{pageTransitionGlitter}\mbox{}\\
    Glitter from one edge of the page to the facing edge.
    
    \stepwise
    {%
      \carg{angle} is a number between $0$ and $360$ which specifies the direction (in degrees) in which to glitter.
    
      Apparently, only the values \afterstep{\pageTransitionGlitter{0}}$0$,
      \step{\afterstep{\pageTransitionGlitter{270}}$270$,} \step{$315$ are supported.}%
      }%
    \pageTransitionGlitter{315}\pause

  \item[{\present[h]{\macroname{pageTransitionReplace}}}]
    \indexmacro{pageTransitionReplace}
    Simple Replace (the default). 
  \end{description}

  \pageTransitionReplace

  \newslide

  \minisec{Things to pay attention to}
  \begin{enumerate}
  \item The setting of the page transition is a property of the \emph{page}, i.~e.\ whatever page transition is in
    effect when a page break occurs, will be assigned to the corresponding pdf page.
    
  \item The setting of the page transition is undone when a group ends.
    
    Make sure no \LaTeX{} environment is ended between a \macroname{pageTransition} setting and the next page break. In
    particular, in \carg{stepcontents}, \macroname{afterstep} should be used (see example \ref{Sec:ExPic}).

  \newslide

  \item Setting page transitions works well with \macroname{pause}. Here, \macroname{pause} acts as a page break,
    i.\,e.\ a different page transition can be set before every occurrence of \macroname{pause}.
  \end{enumerate}
  
  \newslide

  \subsection{Automatic advancing of pages}\label{Sec:PageDuration}
  If you have loaded a sufficiently new version of the
  \href{ftp://ftp.dante.de/tex-archive/help/Catalogue/entries/hyperref.html}{\code{hyperref}} package (which allows to
  set \code{pdfpageduration}), then the following command is defined which enables automatic advancing of \code{pdf}
  pages.

  \present[h]{\commandapp{pageDuration}{\carg{dur}}}\indexmacro{pageDuration}
  causes pages to be advanced automatically every \carg{dur}
  seconds. \carg{dur} should be a non-negative fixed-point number.

  \pageDuration{2}\pause

  Depending on the \code{pdf} viewer, this will happen only in full-screen mode.

  See example \ref{Sec:ExFooling} for a demonstration of this effect.

  \stopAdvancing
  
  \newslide
  
  The same restrictions as for \concept{page transitions} apply. In particular, the page duration setting is undone by
  the end of a group, i.\,e.\ it is useless to set the page duration if a \LaTeX{} environment ends before the next page
  break.
  
  There is no `neutral' value for \carg{dur} ($0$ means advance as fast as possible). You can make automatic advancing
  stop by calling \commandapp{pageDuration}{}. \code{texpower} offers the custom command
  \begin{center}
    \present[h]{\macroname{stopAdvancing}}\indexmacro{stopAdvancing}
  \end{center}
  to do this.

  \newslide

  %-----------------------------------------------------------------------------------------------------------------
  %
  \section{Color management, color emphasis and highlighting}\label{Sec:ColorEmphasis}
  \TeX Power tries to find out whether you are making a colored document. This is assumed if
  \begin{itemize}
  \item the \href{ftp://ftp.dante.de/tex-archive/help/Catalogue/entries/color.html}{\code{color}} package has been
    loaded before the \code{texpower} package or
  \item a color-related option (see sections \ref{Sec:ColBgdOpt} and \ref{Sec:ColorOptions}) is given to the
    \code{texpower} package (in this case, the
    \href{ftp://ftp.dante.de/tex-archive/help/Catalogue/entries/color.html}{\code{color}} package is loaded
    automatically).
  \end{itemize}
  If this is the case, \TeX Power installs an extensive color management scheme on top of the kernel of the
  \href{ftp://ftp.dante.de/tex-archive/help/Catalogue/entries/color.html}{\code{color}} package.
  
  In the following, some new concepts established by this management scheme are explained. Sections \ref{Sec:ColBgdOpt}
  and \ref{Sec:ColorOptions} list options for color activation, section \ref{Sec:Colorcommands} lists some new
  highlighting commands, and section \ref{Sec:Colors} gives the names and meaning of \TeX Power's predefined colors.
  
  Note that parts of the kernel of the
  \href{ftp://ftp.dante.de/tex-archive/help/Catalogue/entries/color.html}{\code{color}} package are overloaded for
  special purposes (getting driver-independent representations of defined colors to be used by \macroname{colorbetween}
  (\ref{Sec:MiscColorCommands}), for instance), so it is recommended to execute color definition commands like
  \macroname{definecolor} \emph{after} the \code{texpower} package has been loaded (see also the next section on
  \macroname{defineTPcolor}).

  \newslide

  \subsection{Standard colors}\label{Sec:StdCols}
  \TeX Power maintains a list of \concept{standard colors} which are recognized and handled by \TeX Power's color
  management. Some commands like \macroname{dimcolors} (see section \ref{Sec:ColorVariants}) affect \emph{all} standard
  colors. There are some predefined colors which are in this list from the outset (see section \ref{Sec:Colors}).

  If colors defined by the user are to be recognized by \TeX Power, they have to be included in this list. The easiest
  way is to use the following command for defining them.

  \newslide

  \present{\commandapp{defineTPcolor}{\carg{name}\}\{\carg{model}\}\{\carg{def}}}
  \indexmacro{defineTPcolor} acts like \macroname{definecolor}
  \indexmacro{defineTPcolor} from the \code{color} package, but the color 
  \carg{name} is also added to the list of standard colors.
  
  If you want to make a color a standard color which is defined elsewhere (by a document class, say), you can simply add
  it to the list of standard colors with the command 
  \present{\commandapp{addTPcolor}{\carg{name}}}\indexmacro{addTPcolor}.

  \newslide

  \subsection{Color sets}
  Every standard color may be defined in one or several \concept{color sets}. There are two fundamentally different
  types of color set:
  \begin{description}
  \item[The current color set.] This contains the current definition of every standard color which is actually used at
    the moment. Every standard color should be defined at least in the current color set. The current color set is not
    distinguished by a special name.

    \newslide
    
  \item[Named color sets.] These are `containers' for a full set of color definitions (for the standard colors) which
    can be activated by respective commands (see below). The color sets are distinguished by their names. Color
    definitions in a named color set are not currently available, they have to be made available by activating the named
    color set.

    There are four predefined color sets named \code{whitebg}\indexcode{whitebg},
    \code{lightbg}\indexcode{lightbg}, \code{darkbg}\indexcode{darkbg}, 
    \code{blackbg}\indexcode{blackbg}, each of which contains a full set of (predefined)
    standard colors customized for a white, light, dark, black background
    color, respectively.
  \end{description}

  \newslide

  There are the following commands for manipulating color sets:
  \begin{description}
  \item[\present{\commandapp{usecolorset}{\carg{name}}}]\indexmacro{usecolorset}
    Make the color set named \carg{name} the current color set.
    \emph{All standard colors in the current color set which are also in color set \carg{name} are overwritten.}

    The standard color \code{textcolor} is set automatically after activating color set \carg{name}.
    
  \item[\present{\commandapp{dumpcolorset}{\carg{name}}}]\indexmacro{dumpcolorset}
    Copy the definitions of \emph{all} standard colors in the
    current color set into color set named \carg{name}. All standard colors in color set \carg{name} will be
    overwritten.
  \end{description}

  \newslide
  
  Using \commandapp{defineTPcolor}{\carg{name}} or \commandapp{definecolor}{\carg{name}} will define the color
  \carg{name} in the \emph{current} color set. To define a color in color set \carg{cset}, use
  \present{\commandapp[\carg{cset}]{defineTPcolor}{\carg{name}}}.
  
    \newslide

  \minisec{Things to pay attention to}
  \begin{enumerate}
  \item Color sets are not really `\TeX{} objects', but are distinguished by color name suffixes. This means, a color
    named \code{foo} is automatically in the current color set. Executing \commandapp[\carg{cset}]{defineTPcolor}{foo}
    means executing \macroname{definecolor} for a specific color the name of which is a combination of \code{foo} and
    \carg{cset}.

    Consequently, \macroname{usecolorset} and \macroname{dumpcolorset} do not copy color sets as composite objects, but
    simply all colors the names of which are generated from the list of standard colors.

    \newslide

  \item The command \commandapp{usecolorset}{\carg{name}} overwrites only those colors which have been defined in color
    set \carg{name}. If a standard color is defined in the current color set, but not in color set \carg{name}, it is
    preserved (but if \commandapp{dumpcolorset}{\carg{name}} is executed later, then it will also be copied back into
    the color set \carg{name}).
  \end{enumerate}

  \newslide

  \subsection{Color Background Options}\label{Sec:ColBgdOpt}
  For activating the predefined color sets, there are shorthands 
  \macroname{whitebackground}\indexmacro{whitebackground},
  \macroname{lightbackground}\indexmacro{lightbackground}, 
  \macroname{darkbackground}\indexmacro{darkbackground}, 
  \macroname{blackbackground}\indexmacro{blackbackground} which execute
  \macroname{usecolorset} and additionally set the background color to its current value.
  
  \newslide

  When one of the following options is given, the respective command is executed automatically at the beginning of the
  document.
  \begin{description}
  \item[\present{option: \code{whitebackground} (default)}]
    \indexpckopt{texpower}{whitebackground}
    Set standard colors to match a white background color.

  \item[\present{option: \code{lightbackground}}]
    \indexpckopt{texpower}{lightbackground}
    Set standard colors to match a light (but not white) background color.

  \item[\present{option: \code{darkbackground}}]
    \indexpckopt{texpower}{darkbackground}
    Set standard colors to match a dark (but not black) background color.

  \item[\present{option: \code{blackbackground}}]
    \indexpckopt{texpower}{blackbackground}
    Set standard colors to match a black background color.
  \end{description}
  
  \newslide

  \subsection{Color variants}\label{Sec:ColorVariants}
  In addition to color sets, \TeX Power implements a concept of \concept{color variant}. Currently, every color has three
  variants: \concept{normal}, \concept{dimmed}, and \concept{enhanced}. The normal variant is what is usually seen, text
  written in the dimmed variant appears ``faded into the background'' and text written in the enhanced variant appears
  to ``stick out''.

  \newslide

  When switching variants, for every color one of two cases can occur:
  \begin{enumerate}
  \item A \concept{designated color} for this variant has been defined.
    
    For color \carg{color} the designated name of the \concept{dimmed}
    \indexcode{dimmed color variant} variant is \code{d\carg{color}}, the designated
    name of the \concept{enhanced}\indexcode{enhanced color variant} variant is 
    \code{e\carg{color}}.

    If a color by that name exists at the time the variant is switched to, then variant switching is executed by
    replacing color \carg{color} with the designated color.

    \newslide

  \item A \concept{designated color} for this variant has not been defined.
    
    If a color by the designated name does not exist at the time a color variant is switched to, then variant switching
    is executed by \concept{automatically} calculating the color variant from the original color.

    The method for calculation depends on the variant:

    \newslide
    
    \begin{description}
    \item[dimmed.] The dimmed variant is calculated by \concept{interpolating} between \code{pagecolor} and the color to
      be dimmed, using the \macroname{colorbetween} command (see \ref{Sec:MiscColorCommands}). 
      
      There is a command \present{\macroname{dimlevel}}\indexmacro{dimlevel} 
      which contains the parameter \carg{weight} given to \macroname{colorbetween}
      (default: \code{0.7}).  This default can be overridden by either redefining 
      \macroname{dimlevel} or giving an alternative \carg{weight} as an optional
      argument to the color dimming command (see below).

      \newslide

    \item[enhanced.] The enhanced variant is calculated by \concept{extrapolating} the color to be enhanced (relative to
      \code{pagecolor}). 
      
      There is a command \present{\macroname{enhancelevel}}\indexmacro{enhancelevel} 
      which gives the \concept{extent} of the extrapolation (default: \code{0.5}). 
      The same holds for overriding this default as for \macroname{dimlevel}.
    \end{description}
  \end{enumerate}

  \newslide

  The following commands switch color variants:
  \begin{description}
  \item[\present{\commandapp[\carg{level}]{dimcolor}{\carg{color}}}]
    \indexmacro{dimcolor} switches color \carg{color} to the \concept{dimmed}
    variant. If given, \carg{level} replaces the value of \macroname{dimlevel} in automatic calculation of the dimmed
    variant (see above).
    
  \item[\present{\code{\macroname{dimcolors}[\carg{level}]}}]
    \indexmacro{dimcolors} switches \emph{all} standard colors to the \concept{dimmed}
    variant. The optional argument \carg{level} acts as for \macroname{dimcolor}.
    
    \newslide

  \item[\present{\commandapp[\carg{level}]{enhancecolor}{\carg{color}}}]
    \indexmacro{enhancecolor} switches color \carg{color} to the
    \concept{enhanced} variant. If given, \carg{level} replaces the value of \macroname{enhancelevel} in automatic
    calculation of the enhanced variant (see above).
    
  \item[\present{\code{\macroname{enhancecolors}[\carg{level}]}}]
    \indexmacro{enhancecolors} switches \emph{all} standard colors to the
    \concept{enhanceed} variant. The optional argument \carg{level} acts as for \macroname{enhancecolor}.
  \end{description}

  \newslide

  \minisec{Things to pay attention to}
  \begin{enumerate}
  \item While automatic calculation of a \concept{dimmed} color will almost always yield the desired result
    (interpolating between colors by calculating a weighted average is trivial), automatic calculation of an
    \concept{enhanced} color by `extrapolating' is tricky at best and will often lead to unsatisfactory results. This is
    because the idea of making a color `stronger' is very hard to formulate numerically.
    
    \newslide

    The following effects of the current algorithm should be kept in mind:
    \begin{itemize}
    \item if the background color is light, enhancing a color will make it darker;
    \item if the background color is dark, enhancing a color will make it lighter;
    \item sometimes, the numerical values describing an enhanced color have to be \concept{bounded} to avoid exceeding
      the allowed range, diminishing the enhancing effect. For instance, if the background color is black and the color
      to be enhanced is a `full-powered' yellow, there is no way of enhancing it by simple numeric calculation.
    \end{itemize}

    \newslide

    As a conclusion, for best results it is recommended to provide custom \code{e} variants of colors to be enhanced. By
    default, \TeX Power does not provide dedicated enhanced colors, but the file \code{tpsettings.cfg} contains complete
    sets of enhanced variants for the standard colors in the different color sets, which you can uncomment and
    experiment with as convenient.

    \newslide

  \item Currently, switching to a different color variant is done by simply overwriting the current definitions of all
    standard colors. This means
    \begin{itemize}
    \item there is no way of `undimming' a color once it has been dimmed,
    \item a dimmed color can not be enhanced and vice versa.
    \end{itemize}
    Maybe this will be solved in a slightly more clever way in subsequent releases of \TeX Power.

    \newslide

    Hence, it is recommended to 
    \begin{itemize}
    \item restrict the \concept{scope} of a global variant switching command like \macroname{dimcolors},
      \macroname{enhancecolors} or \macroname{dstep} by enlcosing it into a \LaTeX{} group (like \code{\{\dots\}}) or
    \item use \macroname{dumpcolorset} before the command to save the current definitions of all colors, to be restored
      with \macroname{usecolorset}.
      
      At the very beginning of a \macroname{stepwise} command, \TeX Power executes \commandapp{dumpcolorset}{stwcolors},
      so you can restore the colors anywhere in the argument of \macroname{stepwise} by saying
      \commandapp{usecolorset}{stwcolors}.
    \end{itemize}

    \newslide

  \item Some rudimentary attempts are made to keep track of which color is in what variant, to the effect that
    \begin{itemize}
    \item a color which is not in the normal variant will neither be dimmed nor enhanced;
    \item when \macroname{usecolorset} overwrites a color with its normal variant, this is registered.
    \end{itemize}
    Still, it is easy to get in trouble by mixing variant changes with color set changes (say, if not all standard
    colors are defined in a color set, or if a color set is dumped when not all colors are in normal variant), so it is
    recommended not to use or dump color sets when outside the normal variant (unless for special applications like
    undoing a variant change by \commandapp{usecolorset}{stwcolors}).
  \end{enumerate}
  
  
    \newslide

  \subsection{Miscellaneous color management commands}\label{Sec:MiscColorCommands}
  \begin{description}
  \item[\present{\commandapp[\carg{tset}]{replacecolor}{\carg{tcolor}\}[\carg{sset}]\{\carg{scolor}}}]
    \indexmacro{} makes
    \carg{tcolor} have the same definition as \carg{scolor} (if \carg{scolor} is defined at all), where \carg{tcolor}
    and \carg{scolor} are color names as given in the first argument of \macroname{definecolor}.  If (one of)
    \carg{tset} and \carg{sset} are given, the respective color is taken from the respective color set, otherwise from
    the current color set.

    If \carg{scolor} is not defined (in color set \carg{sset}), \carg{tcolor} is left alone.

    \newslide

  \item[\present{\commandapp[\carg{weight}]{colorbetween}{\carg{src1}\}\{\carg{src2}\}\{\carg{target}}}]
    \indexmacro{colorbetween} calculates a
    `weighted average' between two colors. \carg{src1} and \carg{src2} are the names of the two colors. \carg{weight}
    (default: $0.5$) is a fixed-point number between $0$ and $1$ giving the `weight' for the interpolation between
    \carg{src1} and \carg{src2}. \carg{target} is the name to be given to the resulting mixed color.

    If \carg{weight} is $1$, then \carg{target} will be identical to \carg{src1} (up to color model conversions, see
    below), if \carg{weight} is $0$, then \carg{target} will be identical to \carg{src2}, if \carg{weight} is $0.5$
    (default), then \carg{target} will be exactly in the middle between \carg{src1} and \carg{src2}.
    
    \macroname{colorbetween} supports the following color models: \code{rgb}, \code{RGB}, \code{gray}, \code{cmyk},
    \code{hsb}. If both colors are of the same model, the resulting color is also of the respective model. If
    \carg{src1} and \carg{src2} are from \emph{different} models, then \carg{target} will \emph{always} be an \code{rgb}
    color. The only exception is the \code{hsb} color model: As I don't know how to convert \code{hsb} to \code{rgb},
    mixing \code{hsb} with another color model will always raise an error.
    
    \newslide

  \item[\present{\commandapp{mkfactor}{\carg{expr}\}\{\carg{macroname}}}]
    \indexmacro{mkfactor} is a helper command for automatically
    generating the fixed point numbers between $0$ and $1$ which are employed by the color calculation commands.
    \carg{expr} can be any expression which can stand behind \code{*} in expressions allowed by the
    \href{ftp://ftp.dante.de/tex-archive/help/Catalogue/entries/calc.html}{\code{calc}} package (for instance:
    \code{\commandapp{value}{counter}/\commandapp{value}{maxcounter}} or \macroname{ratio} or whatever).
    \carg{macroname} should be a valid macro name. \carg{expr} is converted into a fixed-point representation which is
    then assigned to \carg{macroname}.
    
    \newslide

  \item[\present{\code{\macroname{vanishcolors}[\carg{color}]}}]
    \indexmacro{vanishcolors} is similar to the color variant command
    \macroname{dimcolors}, but instead of dimming colors, all standard colors are replaced by a single color given by
    the new command \present{\macroname{vanishcolor}} (default: \code{pagecolor}). Hence, the result of calling
    \macroname{vanishcolors} should be that all text vanishes, as it is written in the background color (this doesn't
    work with structured backgrounds, of course).
    
    For getting a color different from the default \code{pagecolor}, you can either redefinine \macroname{vanishcolor}
    or give an alternative \carg{color} as an optional argument to \macroname{vanishcolors}.
    
    There is no dedicated command for making a single color vanish. To achieve this, use
    \commandapp{replacecolor}{\carg{color}\}\{\macroname{vanishcolor}}.
  \end{description}
  

  \subsection{Color Emphasis and Highlighting}\label{Sec:ColorOptions}
  \code{texpower} offers some support for text emphasis and highlighting with colors (instead of, say, font
  changes). These features are enabled by the following options:
  \begin{description}
  \item[\present{option: \code{coloremph}}]\indexpckopt{texpower}{coloremph}
    Make \macroname{em} and \macroname{emph} switch colors instead of fonts.

  \item[\present{option: \code{colormath}}]\indexpckopt{texpower}{colormath}
    Color all mathematical formulae.
    
  \item[\present{option: \code{colorhighlight}}]\indexpckopt{texpower}{colorhighlight}
    Make new highlighting and emphasis commands defined by \code{texpower}
    use colors.
  \end{description}
  
  \minisec{Things to pay attention to}
  \begin{enumerate}
  \item You need the \code{color} package to use any of the color features.

  \item To implement the options \code{coloremph} and \code{colormath}, it is neccessary to redefine some \LaTeX{}
    internals. This can lead to problems and incompatibilities with other packages. Use with caution.
    
  \item If the \code{colorhighlight} option is \emph{not} given, new highlighting and emphasis commands defined by
    \code{texpower} are realized otherwise. Sometimes, however, there is no good alternative to colors, so different
    emphasis commands can become disabled or indistinguishable.

    \newslide

  \item Because of font changes, emphasized or highlighted text can have different dimensions whether or not the options
    \code{coloremph}, \code{colormath}, and \code{colorhighlight} are set. Prepare for different line and page breaks
    when changing one of these options.
    
  \item Color emphasis and highlighting makes use of the predefined standard colors described in section
    \ref{Sec:Colors}. See sections \ref{Sec:StdCols} to \ref{Sec:ColBgdOpt} for further information on standard colors,
    color sets, and customization.
  \end{enumerate}

  \newslide

  \subsection{New commands for emphasis and highlighting elements}\label{Sec:Colorcommands}
  Some things like setting the page or text color, making emphasised text or math colored are done automatically when
  the respective options are set. There are some additional new commands for creating emphasis and highlighting
  elements.

  \minisec{Concerning math:}
  \begin{description}
  \item[\present{\macroname{origmath}}]\indexmacro{origmath}
    When the \code{colormath} option is given, \emph{everything} which appears in
    math mode is colored accordingly. Sometimes, however, math mode is used for something besides mathematical formulae.
    Some \LaTeX{} commands which internally use math mode (like \code{tabular} or \macroname{textsuperscript}) are
    redefined accordingly when the \code{colormath} option is given (this is a potential source of trouble; beware of
    problems\dots).
    
    If you need to use math mode for something which is not to be colored (like a symbol for \code{itemize}), you can
    use the \macroname{origmath} command which works exactly like \macroname{ensuremath} but doesn't color its argument.
    If a nested use of math mode should occur in the argument of \macroname{origmath}, it will again be colored.
  \end{description}
    
  \newslide

  \minisec{Documenting \TeX{} code:}
  \begin{description}
  \item[\present{\macroname{code}}]\indexmacro{code}
    Simple command for typesetting \code{code} (like shell commands).
    
  \item[\present{\macroname{macroname}}]\indexmacro{macroname}
    For \macroname{macro names}. Like \macroname{code}, but with a \macroname{} in
    front.
    
  \item[\present{\commandapp[\carg{opt arg}]{commandapp}{\carg{command}\}\{\carg{arg}}}]
    \indexmacro{commandapp} For \TeX{} commands. \carg{arg}
    stands for the command argument, \carg{opt arg} for an optional argument.
    
  \item[\present{\macroname{carg}}]\indexmacro{carg}
    For \carg{macro arguments}. 
  \end{description}
    
  \newslide

  \minisec{Additional emphasis commands:}
  \begin{description}
  \item[\present{\macroname{underl}}]\indexmacro{underl}
    Additional \underl{emphasis} command. Can be used like \macroname{emph}. Defaults
    to \textbf{bold face} if the \code{colorhighlight} option is not given.
    
  \item[\present{\macroname{concept}}]\indexmacro{concept}
    Additional \concept{emphasis} command, especially for new concepts. Can be
    augmented by things like automatic index entry creation. Also defaults to \textbf{bold face} if the
    \code{colorhighlight} option is not given.
    
  \item[\present{\macroname{inactive}}]\indexmacro{inactive}
    Additional \inactive{emphasis} command, this time for `de-emphasising'. There is
    no sensible default if the \code{colorhighlight} option is not given, as base \LaTeX{} doesn't offer an appropriate
    font. In this case, \macroname{inactive} defaults to \macroname{monochromeinactive}, which does nothing. 
    
    You can (re-)define \macroname{monochromeinactive} to provide some sensible behaviour in the absence of colors, for
    instance striking out if you're using the
    \href{ftp://ftp.dante.de/tex-archive/help/Catalogue/entries/soul.html}{\code{soul}} package.
  \end{description}


  \minisec{Color Highlighting:}
  \begin{description}
  \item[\present{\macroname{present}}]\indexmacro{present}
    Highlighting command which puts its argument into a \present{box with colored
    background}. Defaults to an \fbox{\macroname{fbox}} if the \code{colorhighlight} option is not given.

    See section \ref{Sec:displaycustom} for some further highlighting commands.
  \end{description}

  \newslide

  \subsection{Predefined standard colors}\label{Sec:Colors}
  In previous subsections, it has been mentioned that \TeX Power predefines some standard colors which have appropriate
  values in the predefined color sets \code{whitebg}, \code{lightbg}, \code{darkbg}, and \code{blackbg} (see sections
  \ref{Sec:StdCols} to \ref{Sec:ColBgdOpt} for further information on standard colors, color sets, and customization).
  \begin{description}
  \item[\present{color: \code{pagecolor}}]\indexcode{pagecolor}
    Background color of the page. Is set automatically at the beginning of the
    document if color management is active.

  \item[\present{color: \code{textcolor}}]\indexcode{textcolor}
    Color of normal text. Is set automatically at the beginning of the
    document if color management is active.

  \item[\present{color: \code{emcolor}}]\indexcode{emcolor}
    Color used for \emph{emphasis} if the \code{coloremph} option is set.
    
  \item[\present{color: \code{altemcolor}}]\indexcode{altemcolor}
    Color used \emph{for \emph{double} emphasis} if the \code{coloremph} option
    is set.
    
  \item[\present{color: \code{mathcolor}}]\indexcode{mathcolor}
    Color used for math $a^2+b^2=c^2$ if the \code{colormath} option is set.

  \item[\present{color: \code{codecolor}}]\indexcode{codecolor}
    Color used by the \macroname{code} command if the \code{colorhighlight}
    option is set.
    
  \item[\present{color: \code{underlcolor}}]\indexcode{underlcolor}
    Color used by the \underl{\macroname{underl} command} if the
    \code{colorhighlight} option is set.
    
  \item[\present{color: \code{conceptcolor}}]\indexcode{conceptcolor}
    Color used by the \concept{\macroname{concept} command} if the
    \code{colorhighlight} option is set.

    \newslide
    
  \item[\present{color: \code{inactivecolor}}]\indexcode{inactivecolor}
    Color used by the \inactive{\macroname{inactive} command} if the
    \code{colorhighlight} option is set.
    
  \item[\present{color: \code{presentcolor}}]\indexcode{presentcolor}
    Color used as background color by the \present{\macroname{present}}
    command if the \code{colorhighlight} option is set.
    
  \item[\present{color: \code{highlightcolor}}]\indexcode{highlightcolor}
    Color used as background color by the
    \highlightboxed{\macroname{highlightboxed}} and \macroname{highlighttext} commands (see section
    \ref{Sec:displaycustom}) if the \code{colorhighlight} option is set.
  \end{description}

\ifthenelse{\boolean{TPcolor}}
{%
  \newslide

  \whitebackground

  \minisec{Color tables}

  \newlength{\widthfirstcol}
  \settowidth{\widthfirstcol}{\textbf{\footnotesize white background}}

  \vspace*{\fill}
  \begin{center}
    \begin{tabular}{|p{\widthfirstcol}*{3}{|p{2cm}}|}
      \cline{2-4}
      \multicolumn{1}{l|}{\textbf{\footnotesize white background}}&standard&dimmed&enhanced\\\hline
      \code{textcolor}&\textcolor{textcolor}{\rule{2cm}{\heightof{l}}}&\dimcolor{textcolor}\textcolor{textcolor}{\rule{2cm}{\heightof{l}}}&\enhancecolor{textcolor}\textcolor{textcolor}{\rule{2cm}{\heightof{l}}}\\\hline
      \code{emcolor}&\textcolor{emcolor}{\rule{2cm}{\heightof{l}}}&\dimcolor{emcolor}\textcolor{emcolor}{\rule{2cm}{\heightof{l}}}&\enhancecolor{emcolor}\textcolor{emcolor}{\rule{2cm}{\heightof{l}}}\\\hline
      \code{altemcolor}&\textcolor{altemcolor}{\rule{2cm}{\heightof{l}}}&\dimcolor{altemcolor}\textcolor{altemcolor}{\rule{2cm}{\heightof{l}}}&\enhancecolor{altemcolor}\textcolor{altemcolor}{\rule{2cm}{\heightof{l}}}\\\hline
      \code{mathcolor}&\textcolor{mathcolor}{\rule{2cm}{\heightof{l}}}&\dimcolor{mathcolor}\textcolor{mathcolor}{\rule{2cm}{\heightof{l}}}&\enhancecolor{mathcolor}\textcolor{mathcolor}{\rule{2cm}{\heightof{l}}}\\\hline
      \code{codecolor}&\textcolor{codecolor}{\rule{2cm}{\heightof{l}}}&\dimcolor{codecolor}\textcolor{codecolor}{\rule{2cm}{\heightof{l}}}&\enhancecolor{codecolor}\textcolor{codecolor}{\rule{2cm}{\heightof{l}}}\\\hline
      \code{underlcolor}&\textcolor{underlcolor}{\rule{2cm}{\heightof{l}}}&\dimcolor{underlcolor}\textcolor{underlcolor}{\rule{2cm}{\heightof{l}}}&\enhancecolor{underlcolor}\textcolor{underlcolor}{\rule{2cm}{\heightof{l}}}\\\hline
      \code{conceptcolor}&\textcolor{conceptcolor}{\rule{2cm}{\heightof{l}}}&\dimcolor{conceptcolor}\textcolor{conceptcolor}{\rule{2cm}{\heightof{l}}}&\enhancecolor{conceptcolor}\textcolor{conceptcolor}{\rule{2cm}{\heightof{l}}}\\\hline
      \code{inactivecolor}&\textcolor{inactivecolor}{\rule{2cm}{\heightof{l}}}&\dimcolor{inactivecolor}\textcolor{inactivecolor}{\rule{2cm}{\heightof{l}}}&\enhancecolor{inactivecolor}\textcolor{inactivecolor}{\rule{2cm}{\heightof{l}}}\\\hline
      \code{presentcolor}&\textcolor{presentcolor}{\rule{2cm}{\heightof{l}}}&\dimcolor{presentcolor}\textcolor{presentcolor}{\rule{2cm}{\heightof{l}}}&\enhancecolor{presentcolor}\textcolor{presentcolor}{\rule{2cm}{\heightof{l}}}\\\hline
      \code{highlightcolor}&\textcolor{highlightcolor}{\rule{2cm}{\heightof{l}}}&\dimcolor{highlightcolor}\textcolor{highlightcolor}{\rule{2cm}{\heightof{l}}}&\enhancecolor{highlightcolor}\textcolor{highlightcolor}{\rule{2cm}{\heightof{l}}}\\\hline
    \end{tabular}
  \end{center}

  \newslide

  \lightbackground

  \vspace*{\fill}
  \begin{center}
    \begin{tabular}{|p{\widthfirstcol}*{3}{|p{2cm}}|}
      \cline{2-4}
      \multicolumn{1}{l|}{\textbf{\footnotesize light background}}&standard&dimmed&enhanced\\\hline
      \code{textcolor}&\textcolor{textcolor}{\rule{2cm}{\heightof{l}}}&\dimcolor{textcolor}\textcolor{textcolor}{\rule{2cm}{\heightof{l}}}&\enhancecolor{textcolor}\textcolor{textcolor}{\rule{2cm}{\heightof{l}}}\\\hline
      \code{emcolor}&\textcolor{emcolor}{\rule{2cm}{\heightof{l}}}&\dimcolor{emcolor}\textcolor{emcolor}{\rule{2cm}{\heightof{l}}}&\enhancecolor{emcolor}\textcolor{emcolor}{\rule{2cm}{\heightof{l}}}\\\hline
      \code{altemcolor}&\textcolor{altemcolor}{\rule{2cm}{\heightof{l}}}&\dimcolor{altemcolor}\textcolor{altemcolor}{\rule{2cm}{\heightof{l}}}&\enhancecolor{altemcolor}\textcolor{altemcolor}{\rule{2cm}{\heightof{l}}}\\\hline
      \code{mathcolor}&\textcolor{mathcolor}{\rule{2cm}{\heightof{l}}}&\dimcolor{mathcolor}\textcolor{mathcolor}{\rule{2cm}{\heightof{l}}}&\enhancecolor{mathcolor}\textcolor{mathcolor}{\rule{2cm}{\heightof{l}}}\\\hline
      \code{codecolor}&\textcolor{codecolor}{\rule{2cm}{\heightof{l}}}&\dimcolor{codecolor}\textcolor{codecolor}{\rule{2cm}{\heightof{l}}}&\enhancecolor{codecolor}\textcolor{codecolor}{\rule{2cm}{\heightof{l}}}\\\hline
      \code{underlcolor}&\textcolor{underlcolor}{\rule{2cm}{\heightof{l}}}&\dimcolor{underlcolor}\textcolor{underlcolor}{\rule{2cm}{\heightof{l}}}&\enhancecolor{underlcolor}\textcolor{underlcolor}{\rule{2cm}{\heightof{l}}}\\\hline
      \code{conceptcolor}&\textcolor{conceptcolor}{\rule{2cm}{\heightof{l}}}&\dimcolor{conceptcolor}\textcolor{conceptcolor}{\rule{2cm}{\heightof{l}}}&\enhancecolor{conceptcolor}\textcolor{conceptcolor}{\rule{2cm}{\heightof{l}}}\\\hline
      \code{inactivecolor}&\textcolor{inactivecolor}{\rule{2cm}{\heightof{l}}}&\dimcolor{inactivecolor}\textcolor{inactivecolor}{\rule{2cm}{\heightof{l}}}&\enhancecolor{inactivecolor}\textcolor{inactivecolor}{\rule{2cm}{\heightof{l}}}\\\hline
      \code{presentcolor}&\textcolor{presentcolor}{\rule{2cm}{\heightof{l}}}&\dimcolor{presentcolor}\textcolor{presentcolor}{\rule{2cm}{\heightof{l}}}&\enhancecolor{presentcolor}\textcolor{presentcolor}{\rule{2cm}{\heightof{l}}}\\\hline
      \code{highlightcolor}&\textcolor{highlightcolor}{\rule{2cm}{\heightof{l}}}&\dimcolor{highlightcolor}\textcolor{highlightcolor}{\rule{2cm}{\heightof{l}}}&\enhancecolor{highlightcolor}\textcolor{highlightcolor}{\rule{2cm}{\heightof{l}}}\\\hline
    \end{tabular}
  \end{center}
  
  \newslide

  \darkbackground

  \vspace*{\fill}
  \begin{center}
    \begin{tabular}{|p{\widthfirstcol}*{3}{|p{2cm}}|}
      \cline{2-4}
      \multicolumn{1}{l|}{\textbf{\footnotesize dark background}}&standard&dimmed&enhanced\\\hline
      \code{textcolor}&\textcolor{textcolor}{\rule{2cm}{\heightof{l}}}&\dimcolor{textcolor}\textcolor{textcolor}{\rule{2cm}{\heightof{l}}}&\enhancecolor{textcolor}\textcolor{textcolor}{\rule{2cm}{\heightof{l}}}\\\hline
      \code{emcolor}&\textcolor{emcolor}{\rule{2cm}{\heightof{l}}}&\dimcolor{emcolor}\textcolor{emcolor}{\rule{2cm}{\heightof{l}}}&\enhancecolor{emcolor}\textcolor{emcolor}{\rule{2cm}{\heightof{l}}}\\\hline
      \code{altemcolor}&\textcolor{altemcolor}{\rule{2cm}{\heightof{l}}}&\dimcolor{altemcolor}\textcolor{altemcolor}{\rule{2cm}{\heightof{l}}}&\enhancecolor{altemcolor}\textcolor{altemcolor}{\rule{2cm}{\heightof{l}}}\\\hline
      \code{mathcolor}&\textcolor{mathcolor}{\rule{2cm}{\heightof{l}}}&\dimcolor{mathcolor}\textcolor{mathcolor}{\rule{2cm}{\heightof{l}}}&\enhancecolor{mathcolor}\textcolor{mathcolor}{\rule{2cm}{\heightof{l}}}\\\hline
      \code{codecolor}&\textcolor{codecolor}{\rule{2cm}{\heightof{l}}}&\dimcolor{codecolor}\textcolor{codecolor}{\rule{2cm}{\heightof{l}}}&\enhancecolor{codecolor}\textcolor{codecolor}{\rule{2cm}{\heightof{l}}}\\\hline
      \code{underlcolor}&\textcolor{underlcolor}{\rule{2cm}{\heightof{l}}}&\dimcolor{underlcolor}\textcolor{underlcolor}{\rule{2cm}{\heightof{l}}}&\enhancecolor{underlcolor}\textcolor{underlcolor}{\rule{2cm}{\heightof{l}}}\\\hline
      \code{conceptcolor}&\textcolor{conceptcolor}{\rule{2cm}{\heightof{l}}}&\dimcolor{conceptcolor}\textcolor{conceptcolor}{\rule{2cm}{\heightof{l}}}&\enhancecolor{conceptcolor}\textcolor{conceptcolor}{\rule{2cm}{\heightof{l}}}\\\hline
      \code{inactivecolor}&\textcolor{inactivecolor}{\rule{2cm}{\heightof{l}}}&\dimcolor{inactivecolor}\textcolor{inactivecolor}{\rule{2cm}{\heightof{l}}}&\enhancecolor{inactivecolor}\textcolor{inactivecolor}{\rule{2cm}{\heightof{l}}}\\\hline
      \code{presentcolor}&\textcolor{presentcolor}{\rule{2cm}{\heightof{l}}}&\dimcolor{presentcolor}\textcolor{presentcolor}{\rule{2cm}{\heightof{l}}}&\enhancecolor{presentcolor}\textcolor{presentcolor}{\rule{2cm}{\heightof{l}}}\\\hline
      \code{highlightcolor}&\textcolor{highlightcolor}{\rule{2cm}{\heightof{l}}}&\dimcolor{highlightcolor}\textcolor{highlightcolor}{\rule{2cm}{\heightof{l}}}&\enhancecolor{highlightcolor}\textcolor{highlightcolor}{\rule{2cm}{\heightof{l}}}\\\hline
    \end{tabular}
  \end{center}
  
  \newslide

  \blackbackground

  \vspace*{\fill}
  \begin{center}
    \begin{tabular}{|p{\widthfirstcol}*{3}{|p{2cm}}|}
      \cline{2-4}
      \multicolumn{1}{l|}{\textbf{\footnotesize black background}}&standard&dimmed&enhanced\\\hline
      \code{textcolor}&\textcolor{textcolor}{\rule{2cm}{\heightof{l}}}&\dimcolor{textcolor}\textcolor{textcolor}{\rule{2cm}{\heightof{l}}}&\enhancecolor{textcolor}\textcolor{textcolor}{\rule{2cm}{\heightof{l}}}\\\hline
      \code{emcolor}&\textcolor{emcolor}{\rule{2cm}{\heightof{l}}}&\dimcolor{emcolor}\textcolor{emcolor}{\rule{2cm}{\heightof{l}}}&\enhancecolor{emcolor}\textcolor{emcolor}{\rule{2cm}{\heightof{l}}}\\\hline
      \code{altemcolor}&\textcolor{altemcolor}{\rule{2cm}{\heightof{l}}}&\dimcolor{altemcolor}\textcolor{altemcolor}{\rule{2cm}{\heightof{l}}}&\enhancecolor{altemcolor}\textcolor{altemcolor}{\rule{2cm}{\heightof{l}}}\\\hline
      \code{mathcolor}&\textcolor{mathcolor}{\rule{2cm}{\heightof{l}}}&\dimcolor{mathcolor}\textcolor{mathcolor}{\rule{2cm}{\heightof{l}}}&\enhancecolor{mathcolor}\textcolor{mathcolor}{\rule{2cm}{\heightof{l}}}\\\hline
      \code{codecolor}&\textcolor{codecolor}{\rule{2cm}{\heightof{l}}}&\dimcolor{codecolor}\textcolor{codecolor}{\rule{2cm}{\heightof{l}}}&\enhancecolor{codecolor}\textcolor{codecolor}{\rule{2cm}{\heightof{l}}}\\\hline
      \code{underlcolor}&\textcolor{underlcolor}{\rule{2cm}{\heightof{l}}}&\dimcolor{underlcolor}\textcolor{underlcolor}{\rule{2cm}{\heightof{l}}}&\enhancecolor{underlcolor}\textcolor{underlcolor}{\rule{2cm}{\heightof{l}}}\\\hline
      \code{conceptcolor}&\textcolor{conceptcolor}{\rule{2cm}{\heightof{l}}}&\dimcolor{conceptcolor}\textcolor{conceptcolor}{\rule{2cm}{\heightof{l}}}&\enhancecolor{conceptcolor}\textcolor{conceptcolor}{\rule{2cm}{\heightof{l}}}\\\hline
      \code{inactivecolor}&\textcolor{inactivecolor}{\rule{2cm}{\heightof{l}}}&\dimcolor{inactivecolor}\textcolor{inactivecolor}{\rule{2cm}{\heightof{l}}}&\enhancecolor{inactivecolor}\textcolor{inactivecolor}{\rule{2cm}{\heightof{l}}}\\\hline
      \code{presentcolor}&\textcolor{presentcolor}{\rule{2cm}{\heightof{l}}}&\dimcolor{presentcolor}\textcolor{presentcolor}{\rule{2cm}{\heightof{l}}}&\enhancecolor{presentcolor}\textcolor{presentcolor}{\rule{2cm}{\heightof{l}}}\\\hline
      \code{highlightcolor}&\textcolor{highlightcolor}{\rule{2cm}{\heightof{l}}}&\dimcolor{highlightcolor}\textcolor{highlightcolor}{\rule{2cm}{\heightof{l}}}&\enhancecolor{highlightcolor}\textcolor{highlightcolor}{\rule{2cm}{\heightof{l}}}\\\hline
    \end{tabular}
  \end{center}
}{}  

  \newslide

  \ifthenelse{\boolean{TPcolor}}{\lightbackground}{}
  
  %-----------------------------------------------------------------------------
  %
  \section{Structured page backgrounds and panels}\label{Sec:PageBackPanel}
  \subsection{Structured page backgrounds}

  \present{\commandapp[\carg{options}]{backgroundstyle}{\carg{style}}}
  \indexmacro{backgroundstyle}
  is the central command for structured page backgrounds. It works like 
  \macroname{pagestyle} and other commands of this type. This means 
  \carg{style} is a symbolic name specifying the general method by which the 
  page background is constructed.
  
  The detailed construction is influenced by parameters which can be set in 
  \carg{options}. If given, the optional parameter \carg{options} should 
  contain a list of settings in ``keyval'' manner. The keyval method 
  % (which is used by the \includegraphics command from the graphicx package, 
  %for instance) 
  is based on associating a symbolic name with every parameter. \carg{options}
  is then a comma-separated list of parameter settings of the form 
  \carg{name}=\carg{value}, where \carg{name} is the symbolic name of the 
  parameter to be set and \carg{value} is the value it is to be set to.
  
  Not every \carg{style} evaluates every parameter. In the following, a 
  description of all styles, together with lists of the parameters employed, is
  given. It is followed by a list of all parameters. Note that some parameter 
  names internally access the same parameter. For instance, parameters 
  \code{startcolor} and \code{startcolordef} both set the start color of a color
  gradient. In case of conflict, the last setting in the list \carg{options} 
  will prevail. It is noted in the list of parameters which other parameters 
  are overwritten. 

  \newslide

  \carg{style} may have one of the following values:
  \begin{description}
  \item[\present{Style: \code{none}}]
    \indexmacroopt{backgroundstyle}{none}% 
    No background. This means the page background is whatever it would be if 
    \macroname{backgroundstyle} wasn't used at all (for instance, a plain area 
    of color pagecolor if one of the color options has been given).
    
    Parameters used: none.
    
  \item[\present{Style: \code{plain}}]
    \indexmacroopt{backgroundstyle}{plain}% 
    Plain background. This means the page background is whatever it would be if
    \macroname{backgroundstyle} wasn't used at all (as for no background). In 
    addition to background style \code{none}, the background style \code{plain}
    does produce panel backgrounds. The colors and dimensions of a \code{top 
    panel}, \code{bottom panel}, \code{left panel}, and \code{right panel} can 
    be specified. 

    \begin{flushleft} 
    Parameters used: \code{hpanels}, \code{autopanels}, \code{toppanelcolor}, 
    \code{bottompanelcolor}, \code{leftpanelcolor}, \code{rightpanelcolor}, 
    \code{toppanelcolordef}, \code{bottompanelcolordef}, \code{leftpanelcolordef}, 
    \code{rightpanelcolordef}, \code{toppanelheight}, \code{bottompanelheight}, 
    \code{leftpanelwidth}, \code{rightpanelwidth}.
    \end{flushleft} 
    
  \item[\present{Style: \code{vgradient}}]
    \indexmacroopt{backgroundstyle}{vgradient}% 
    Vertical gradient. The page background is constructed using the 
    \macroname{vgradrule}\indexmacro{vgradrule} command. In addition to the 
    usual parameters of gradient rules, the vgradient background style allows
    to leave space for headers, footers, or panels. The colors and dimensions
    of a \code{top panel}, \code{bottom panel}, \code{left panel}, and 
    \code{right panel} can be specified. The gradient rule fills the 
    rectangular space left between the specified panels.

    \begin{flushleft} 
    Parameters used: \code{stripes}, \code{firstgradprogression}, 
    \code{startcolor}, \code{startcolordef}, \code{endcolor}, \code{endcolordef} 
    in addition to the parameters used for style \code{plain}.
    \end{flushleft} 
    
  \item[\present{Style: \code{hgradient}}]
    \indexmacroopt{backgroundstyle}{hgradient}% 
    Horizontal gradient. The page background is constructed using the 
    \macroname{hgradrule}\indexmacro{hgradrule} command.  See the description of 
    \macroname{vgradient} concerning panels.

    Parameters used: See list for style \code{vgradient}.
    
  \item[\present{Style: \code{doublevgradient}}]
    \indexmacroopt{backgroundstyle}{doublevgradient}% 
    Double vertical gradient. The page background is constructed using the 
    \macroname{dblvgradrule}\indexmacro{dblvgradrule} command. See the 
    description of \macroname{vgradient} concerning panels.

    \begin{flushleft} 
    Parameters used: \code{gradmidpoint}, \code{secondgradprogression}, 
    \code{midcolor}, \code{midcolordef} in addition to the parameters used for 
    style \code{vgradient} (and \code{plain}).
    \end{flushleft} 

  \item[\present{Style: \code{doublehgradient}}]
    \indexmacroopt{backgroundstyle}{doublehgradient}% 
    Double horizontal gradient. The page background is constructed using the 
    \macroname{dblhgradrule}\indexmacro{dblhgradrule} command. See the description of 
    \macroname{vgradient} concerning panels.

    Parameters used: See list for \code{doublevgradient}.

  \end{description}
  Now, a list of all parameters and their meaning. In the following,
  \begin{itemize}\setlength{\itemsep}{0cm}
  \item[\carg{n}]   denotes a (calc expression for a) nonnegative integer
  \item[\carg{i}]   denotes a (calc expression for an) integer
  \item[\carg{r}]   denotes a fixed-point number
  \item[\carg{l}]   denotes a (calc expression for a) length
  \item[\carg{c}]   denotes the name of a defined color
  \item[\carg{cm}]  denotes a valid color model name (in the sense of the color 
    package)
  \item[\carg{cd}]  denotes a valid color definition (in the sense of the color 
    package) wrt a given \carg{cm} parameter
  \item[\carg{t}]   denotes a `truth value' in the sense of the ifthen package: 
    either true or false. As usual for keyval, if =\carg{t} is omitted, the 
    default true is assumed.
  \end{itemize}
  \begin{description}
  \item[\present{Option: \code{stripes=}\carg{n}}] Set the \carg{stripes} 
    parameter of gradient rules to \carg{n}.\\  
    Default: \macroname{bgndstripes}. \\
    Used by: \code{vgradient}, \code{hgradient}, \code{doublevgradient}, 
    \code{doublehgradient}.

  \item[\present{Option: \code{gradmidpoint=}\carg{r}}] Set the \carg{midpoint} 
    parameter of double gradient rules to \carg{r}.\\
    Default: \macroname{bgndgradmidpoint}\\
    Used by: doublevgradient, doublehgradient

  \item[\present{Option: \code{firstgradprogression=}\carg{i}}] 
    Set the first gradient progression of gradient rules to \carg{i}.\\
    Default: \macroname{bgndfirstgradprogression}\\
    Used by: vgradient, hgradient, doublevgradient, doublehgradient

  \item[\present{Option: \code{secondgradprogression=}\carg{i}}] 
    Set the second gradient progression of double gradient rules to \carg{i}.\\
    Default: \macroname{bgndsecondgradprogression}\\
    Used by: doublevgradient, doublehgradient

  \item[\present{Option: \code{startcolor=}\carg{c}}] 
    Set the \carg{startcolor} parameter of gradient rules to \carg{c}.\\
    Default: If neither startcolor nor startcolordef is given, the color 
      bgndstartcolor is used as startcolor.\\
    Used by: vgradient, hgradient, doublevgradient, doublehgradient\\
    Overwrites: startcolordef

  \item[\present{Option: \code{startcolordef=\{\carg{cm}\}\{\carg{cd}\}}}] 
    Set the \carg{startcolor} parameter of gradient rules to color foo, which 
    is obtained by \macroname{definecolor\{foo\}\{\carg{cm}\}\{\carg{cd}\}}. 
    Note that the two pairs of curly braces are mandatory.\\
    Default: If neither startcolor nor startcolordef is given, the color 
      bgndstartcolor is used as startcolor.\\
    Used by: vgradient, hgradient, doublevgradient, doublehgradient\\
    Overwrites: startcolor

  \item[\present{Option: \code{endcolor=}\carg{c}}] 
    Set the \carg{endcolor} parameter of gradient rules to \carg{c}.\\
    Default: If neither endcolor nor endcolordef is given, the color 
      bgndendcolor is used as endcolor.\\
    Used by: vgradient, hgradient, doublevgradient, doublehgradient\\
    Overwrites: endcolordef

 \item[\present{Option: \code{endcolordef=\{\carg{cm}\}\{\carg{cd}\}}}] 
    Set the \carg{endcolor} parameter of gradient rules to color foo, which 
    is obtained by \macroname{definecolor\{foo\}\{\carg{cm}\}\{\carg{cd}\}}. 
    Note that the two pairs of curly braces are mandatory.\\
    Default: If neither endcolor nor endcolordef is given, the color 
      bgndendcolor is used as endcolor.\\
    Used by: vgradient, hgradient, doublevgradient, doublehgradient\\
    Overwrites: endcolor

  \item[\present{Option: \code{midcolor=}\carg{c}}] 
    Set the \carg{midcolor} parameter of double gradient rules to \carg{c}.\\
    Default: If neither midcolor nor midcolordef is given, the color 
      bgndmidcolor is used as midcolor.\\
    Used by: doublevgradient, doublehgradient\\
    Overwrites: midcolordef

  \item[\present{Option: \code{midcolordef=\{\carg{cm}\}\{\carg{cd}\}}}] 
    Set the \carg{midcolor} parameter of double gradient rules to color foo, 
    which is obtained by \macroname{definecolor\{foo\}\{\carg{cm}\}\{\carg{cd}\}}. 
    Note that the two pairs of curly braces are mandatory.\\
    Default: If neither midcolor nor midcolordef is given, the color 
      bgndmidcolor is used as midcolor.\\
    Used by: doublevgradient, doublehgradient\\
    Overwrites: midcolor

 \item[\present{Option: \code{hpanels=}\carg{t}}] 
    Specifies the `direction' of panels produced. hpanels=true means the top 
    and bottom panel span the full width of the screen. In the space left in
    the middle, the left panel, the background itself, and the right panel are
    displayed. hpanels=false means the left and right panel span the full 
    height of the screen. In the space left in the middle, the top panel, 
    the background itself, and the bottom panel are
    displayed.\\
    Default: hpanels=true is the default for plain, hgradient and
      doublehgradient. hpanels=false is the default for vgradient and
      doublevgradient.\\
    Used by: plain, vgradient, hgradient, doublevgradient, doublehgradient
 
 \item[\present{Option: \code{autopanels=}\carg{t}}] 
    Specifies whether the default values of the parameters toppanelheight,
    bottompanelheight, leftpanelwidth, rightpanelwidth should be calculated 
    automatically from the contents of declared panels. The automatism used 
    is analogous to that of \macroname{DeclarePanel*}. Note that for panel 
    arrangement, both the width and the height of all declared panels are 
    overwritten. If you don't want this, calculate the panel parameters 
    yourself and set autopanels=false. In this case, the current panel 
    dimensions of declared panels are used as defaults for toppanelheight, 
    bottompanelheight, leftpanelwidth, rightpanelwidth.\\
    Default: true.\\
    Used by: plain, vgradient, hgradient, doublevgradient, doublehgradient
 
 \item[\present{Option: \code{\carg{pos}panelheight=}\carg{l}}] 
    Set the height/width of the space left for the top / bottom / left / right
    panel to \carg{l}. Note that the remaining dimensions of panels, for 
    instance the width of the top panel, are always calculated automatically, 
    depending on the setting of the hpanels parameter.\\
    Default: If a respective panel has been defined using 
      \macroname{DeclarePanel}, the default used depends on the setting of the 
      autopanels parameter. If autopanels=true, the correct dimension is 
      calculated from the contents of the panel. The respective one of 
      \macroname{toppanelheight}, \macroname{bottompanelheight},
      \macroname{leftpanelwidth}, \macroname{rightpanelwidth} is overwritten
      with the result. If autopanels=false, then the respective setting of 
      \macroname{toppanelheight}, \macroname{bottompanelheight}, 
      \macroname{leftpanelwidth}, \macroname{rightpanelwidth} is taken as the 
      default. If a panel has not been declared, the appropriate one of 
      \macroname{bgndtoppanelheight}, \macroname{bgndbottompanelheight}, 
      \macroname{bgndleftpanelwidth}, \macroname{bgndrightpanelwidth} is used 
      as default. \\
    Used by: plain, vgradient, hgradient, doublevgradient, doublehgradient

 \item[\present{Option: \code{\carg{pos}panelcolor=}\carg{c}}] 
    Set the color of the space left for the top / bottom / left / right panel 
    to \carg{c}.\\
    Default: The standard colors toppanelcolor, bottompanelcolor, leftpanelcolor,
      rightpanelcolor are used as defaults.\\
    Used by: plain, vgradient, hgradient, doublevgradient, doublehgradient\\
    Overwrites: toppanelcolordef bottompanelcolordef leftpanelcolordef 
    rightpanelcolordef 

 \item[\present{Option: \code{\carg{pos}panelcolordef=\{\carg{cm}\}\{\carg{cd}\}}}] 
    Set the color of the space left for the top / bottom / left / right panel 
    to color foo, which is obtained by 
    \macroname{definecolor\{foo\}\{\carg{cm}\}\{\carg{cd}\}}. 
    Note that the two pairs of curly braces are mandatory. \\ 
    Default: See the description of top/bottom/left/rightpanelcolor.\\
    Used by: plain, vgradient, hgradient, doublevgradient, doublehgradient\\
    Overwrites: toppanelcolor bottompanelcolor leftpanelcolor rightpanelcolor 
 
  \end{description}

  \newslide

  \subsection{Panel-specific user level commands}
  If you're using a package that has it own panel (as 
  \href{ftp://ftp.dante.de/tex-archive/help/Catalogue/entries/pdfscreen.html}%
  {\code{pdfscreen}}) don't even consider using the following. 

  \present{\commandapp[\carg{name}]{DeclarePanel}{\carg{pos}\}\{\carg{contents}}}
  \indexmacro{DeclarePanel}
  declares the contents \carg{contents} of the panel at position \carg{pos}.
  Afterwards, on every page the panel contents are set in a parbox of
  dimensions and position specified by \carg{pos}panelwidth,
  \carg{pos}panelheight, \macroname{panelmargin} and \carg{pos}panelshift for
  top and bottom panels and \carg{pos}panelraise for left and right panels. The
  parbox is constructed anew on every page, so all changes influencing panel
  contents or parameters (like a \macroname{thepage} in the panel contents) are
  respected.

  The panel contents are set in color \carg{pos}paneltextcolor. There is
  another standard color \carg{pos}panelcolor, which is however not activated by
  \macroname{DeclarePanel} but by selecting an appropriate background style.

  Note that \macroname{backgroundstyle} must be called after the panel
  declaration.

  \newslide
 
  Pages are constructed as follows: first the page background, then
  the panels, and then the page contents. \emph{Hence, panels overwrite the background
  and the page contents overwrite the panels.} The user is supposed to make sure
  themselves that there is enough space left on the page for the panels
  (document class specific settings).  The panel declaration is global. A panel
  can be `undeclared' by using \macroname{DeclarePanel\{\carg{pos}\}\{\}}.

  If the optional argument \carg{name} is given, the panel contents and
  (calculated) size will also be stored under the given name, to be restored
  later with \macroname{restorepanels}. This is nice for switching between
  different sets of panels.

  \newslide

  For an example look at the files \code{simplepanel.tex} and \code{panelexample.tex}.
  A very simple example follows:

  \begin{verbatim}
\DeclarePanel{left}{%
  \textsf{Your Name}

  \vfill

  \button{\Acrobatmenu{PrevPage}}{Back}
  \button{\Acrobatmenu{NextPage}}{Next} } 
\end{verbatim}
   
  \newslide

  There is a starred version which will (try to) automatically calculate the
  \indexmacro{DeclarePanel*} `flexible' dimension of each panel. For top and 
  bottom panels this is the height, for left and right panels this is the width.
  Make sure the panel contents are `valid' at the time \macroname{DeclarePanel*} 
  is called so the calculation can be carried out in a meaningful way.  

  While the automatic calculation of the height of top and bottom panels is 
  trivial (using \macroname{settoheight}), there is a sophisticated procedure 
  for calculating a `good' width for the parbox containing the panel. Owing to 
  limitations set by TeX, there are certain limits to the sophistication of the
  procedure.

  \newslide

  For instance, any `whatsits' (specials (like color changes), file
  accesses (like \macroname{label}), or hyper anchors)
  or rules which are inserted directly in the vertical list of the parbox
  `block' the analysis, so the procedure can't `see' past them (starting at the
  bottom of the box) when analysing the contents of the parbox.

  The user should make sure such items are set in horizontal mode (by using
  \macroname{leavevmode} or enclosing stuff in boxes). Furthermore, only
  overfull and underfull hboxes which occur while setting the parbox are
  considered when judging which width is `best'. This will reliably make the
  width large enough to contain `wide' objects like tabulars, logos and buttons,
  but might not give optimal results for justified text.  vboxes occurring
  directly in the vbox are ignored.  

  \newslide

  Note further that hboxes with fixed width
  (made by \macroname{hbox} to...) which occur directly in the vbox may disturb
  the procedure, because the fixed width cannot be recovered. These hboxes will
  be reformatted with the width of the vbox, generating an extremely large
  badness, unsettling the calculation of maximum badness. To avoid this such
  hboxes should be either contained in a vbox or set in horizontal mode with
  appropriate glue at the end.

  \newslide

  \subsection{Navigation buttons}
  The following provides only the very basics for navigation buttons. If you're
  using a package that has it's own naviagtion buttons (as
  \href{ftp://ftp.dante.de/tex-archive/help/Catalogue/entries/pdfscreen.html}%
  {\code{pdfscreen}}) don't even consider using the following. 

  \present{\macroname{button\{\carg{navcommand}\}\{\carg{text}\}}}
  \indexmacro{button}
  creates a button labelled \carg{text} which executes \carg{navcommand} when
  pressed. The command takes four optional arguments (left out above): 
  \carg{width}, \carg{height}, \carg{depth} and \carg{alignment} in that order.
  \carg{navcommand} can be for instance \commandapp{Acrobatmenu}{\carg{command}} or
  \commandapp{hyperlink}{\carg{target}} (note that \carg{navcommand} should take
  one (more) argument specifying the sensitive area which is provided by
  \macroname{button}).  If given, the optional parameters \carg{width}, \carg{height}, and
  \carg{depth} give the width, height and depth, respectively, of the framed
  area comprising the button (excluding the shadow, but including the
  frame). Default are the `real' width, height and depth, respectively, of
  \carg{text}, plus allowance for the frame.  If given, the optional parameter
  \carg{alignment} (one of l,c,r) gives the alignment of \carg{text} inside the
  button box (makes sense only if \carg{width} is given).

  The button appearence is defined by some configurable button parameters:
  \begin{description}
  \item[\present{\macroname{buttonsep}}]
    \indexmacro{buttonsep}
    Space between button label and border. (Default: \macroname{fboxsep})

  \item[\present{\macroname{buttonrule}}]
    \indexmacro{buttonrule}
    Width of button frame. (Default: \code{0pt})

  \item[\present{\macroname{buttonshadowhshift}}]
    \indexmacro{buttonshadowhshift}
    Horizontal displacement of button shadow. (Default: 0.3\macroname{fboxsep})

  \item[\present{\macroname{buttonshadowvshift}}]
    \indexmacro{buttonshadowvshift}
    Vertical displacement of button shadow. (Default: 0.3\macroname{fboxsep})
  \end{description}  

  A list of predefined buttons follows:
  \begin{description}

  \item[\present{\macroname{backpagebutton[\carg{width}]}}]
    \indexmacro{backpagebutton}
    Last subpage of previous page.

  \item[\present{\macroname{backstepbutton[\carg{width}]}}]
    \indexmacro{backstepbutton}
    Previous step.

  \item[\present{\macroname{gobackbutton[\carg{width}]}}]
    \indexmacro{gobackbutton}
    `Undo action' (go back to whatever was before last action).

  \item[\present{\macroname{nextstepbutton[\carg{width}]}}]
    \indexmacro{nextstepbutton}
    Next step.

  \item[\present{\macroname{nextpagebutton[\carg{width}]}}]
    \indexmacro{nextpagebutton}
    First subpage of next page.

  \item[\present{\macroname{nextfullpagebutton[\carg{width}]}}]
    \indexmacro{nextfullpagebutton}
    Last subpage of next page.

  \item[\present{\macroname{fullscreenbutton[\carg{width}]}}]
    \indexmacro{fullscreenbutton}
    Toggle fullscreen mode.

  \end{description}  

\clearpage
\printindex
%KEEPCOMMENTS
%</docu>
%=================================================================================================================
%<*bckwrdexample>
%<<KEEPCOMMENTS
%-----------------------------------------------------------------------------------------------------------------
%
% Backwards writing example for the package texpower.sty.
% 
%-----------------------------------------------------------------------------------------------------------------
% Set background color to black and use slifonts. 

\PassOptionsToPackage{blackbackground}{texpower}
\RequirePackage{tpslifonts}

% Input the generic preamble.

\input{__TPpreamble}
\hypersetup{pdftitle={texpower backwards writing example}}

% Define \includegraphics for including an mps (metapost postscript) image.
\usepackage{graphicx}
% The mps extension isn't supported out-of-the-box for latex+dvips
\ifthenelse{\boolean{psspecialsallowed}}{%
\DeclareGraphicsExtensions{.mps}}{}

%-----------------------------------------------------------------------------------------------------------------
% Finally, everything is set up. Here we go...
%
\begin{document}
\begin{slide}
%KEEPCOMMENTS
%</bckwrdexample>
%<*bckwrdexample-src>
%<<KEEPCOMMENTS

  %-----------------------------------------------------------------------------------------------------------------
  %
  \makeslidetitle{\macroname{stepwise} Example: Writing Backwards}\label{Sec:ExBackwards}
  %
  % The following example doesn't really demonstrate a useful application. Its purpose is twofold:
  % a) Show how the functionality of \stepwise can be extended by the user by referring to the macros provided by the
  %    package.
  % b) Show that \steps can appear in any order, and can be made to appear simultaneously in several places (and mention
  %    the problems this raises). 

  % We define a new macro \backstep which will call \step, but the steps will be executed in _reverse_ order.
  % This is achieved as follows:
  % * We refer to the counter totalsteps, which gives the total number of steps occurring in this argument of
  %   \stepwise.
  % * From this, we subtract the value of the counter stepcommand, which gives the number of this \step command (in the
  %   order of appearance). 
  % * The result is compared with the counter step, which gives the number of the current step.
  % (the default for `triggering' a \step is the condition \value{step}=\value{stepcommand})
  %
  % Note that if the ifthen Package would support this syntax, we could use the bracketed optional argument of \step (as
  % in the previous example), defining \backstep like this:
  % 
  % \newcommand{\backstep}{\step[\value{totalsteps}-\value{stepcommand}+1=\value{step}]}
  %
  % As \ifthenelse doesn't offer this syntax, we have to calculate \value{totalsteps}-\value{stepcommand}+1 in a separate
  % counter. This doesn't fit the intended use of \step's optional argument. But hey, there is another syntax ;-)
  % If the optional argument is specified with braces (...), then it can contain an arbitrary test which takes two
  % arguments and selects the first if it succeeds and the second if it fails. We use this variant here.
  %
  \newcounter{reversestepno}%
  \newcommand{\backstep}{\step(\setcounter{reversestepno}{\value{totalsteps}-\value{stepcommand}+1}\ifthenelse{\value{step}=\value{reversestepno}})}%
  % 
  % We use the custom command \parstepwise which not only wraps the whole argument of \stepwise into a minipage (because
  % otherwise vertical spacing goes haywire, don't ask me why), but also gives substance to steps.
  %
  % If the following \stepwise command would only contain the calls to \backstep, everything would be fine.
  % But we _had_ to add something else....
  % In the second part of this application of \stepwise, several steps are executed simultaneously with those executed
  % backwards in the first part. This means the value of the counter totalsteps is 14, i.e. the calls to \backstep
  % correspond to steps 8...14. To remedy this, we decree that the first step performed shall be number 7, by setting
  % the counter firststep accordingly in the optional argument of \stepwise.
  % 
  \parstepwise[\setcounter{firststep}{\value{totalsteps}/2+\value{firststep}}]
  {%
    \begin{quote}
      \Huge 
      \backstep{Is} \backstep{now} \backstep{it}
      \backstep{backwards} \backstep{write} \backstep{to}
      \backstep{possible\,!}
      
      \bigskip
        
      % By determining explicitly the times at which the following steps are executed, we make them appear
      % simultaneously with the preceding flock of \backsteps. As we have set the counter firststep to 7, we start
      % counting with 8.
      %
      \step[\value{step}=8]{\includegraphics[width=2cm]{fig-1}}
      \step[\value{step}=9]{\includegraphics[width=2cm]{fig-1}}
      \step[\value{step}=10]{\includegraphics[width=2cm]{fig-1}}
      \step[\value{step}=11]{\includegraphics[width=2cm]{fig-1}}
      \step[\value{step}=12]{\includegraphics[width=2cm]{fig-1}}
      \step[\value{step}=13]{\includegraphics[width=2cm]{fig-1}}
      \step[\value{step}=14]{\includegraphics[width=2cm]{fig-1}}%
    \end{quote}
    }%
  \newslide
%KEEPCOMMENTS
%</bckwrdexample-src>
%=================================================================================================================
%<*bgndexample>
%<<KEEPCOMMENTS
%-----------------------------------------------------------------------------------------------------------------
%
% Background style example for the package texpower.sty.
% 
%-----------------------------------------------------------------------------------------------------------------
% Use slifonts and a dark background. 

\PassOptionsToPackage{darkbackground,colorhighlight,verbose}{texpower}

\RequirePackage{tpslifonts}

% Input the generic preamble.

\input{__TPpreamble}
\hypersetup{pdftitle={texpower background style example}}

\newcommand{\skipTo}[1]{\hyperlink{#1}{\present{\textsf{\textbf{Skip animation}}}}}

\makeatletter
\newcommand{\histogram}[1]
{{%
    \renewcommand{\vstripe@TP}[4]
    {\rule{##2-2pt}{(##3)*\real{##1}}\hspace*{2pt}##4}%
    #1%
    }}
\makeatother

\newcommand{\totalbarwidth}{2cm}

\newcommand{\mkbar}[2][100]
{%
  \ifthenelse{#1<#2}{\def\percentval{#1}}{\def\percentval{#2}}%
  \mkfactor{\intensity}{\percentval/100}%
  \colorbetween[\intensity]{ecolor}{green}{red}%
  \hgradrule[\percentval]{red}{ecolor}{\totalbarwidth*\real{\intensity}}{1ex}
  \textbf{\boldmath$\mathsf{\ifthenelse{\percentval<10}{\phantom{0}}{}\percentval\%}$}%
  }

\renewcommand{\bgndstripes}{100}

\setlength{\fboxrule}{1pt}

%-----------------------------------------------------------------------------------------------------------------
% Finally, everything is set up. Here we go...
%
\begin{document}
\begin{slide}
%KEEPCOMMENTS
%</bgndexample>
%<*bgndexample-src>
%<<KEEPCOMMENTS

\centerslidestrue
\title{The \TeX Power bundle\\[2ex]{\normalfont Structured
    rules, box and page backgrounds}}
\author{Stephan Lehmke\\\url{mailto:Stephan.Lehmke@cs.uni-dortmund.de}}
\maketitle

\begin{small}
  This example demonstrates \TeX Power's support for structured rules,
  box and page backgrounds. The usage and parameterization of the
  corresponding commands is documented in the manual. Here, we only
  demonstrate the effects achievable with the parameters.
\end{small}

\pageDuration{0.01}

\parstepwise*%
{%
  \multistep
  {50}
  {%
    \mkfactor{\intensity}{(\value{substep}-1)/49}%
    \colorbetween[\intensity]{stcolor}{pagecolor}{white}%
    \backgroundstyle[startcolor=stcolor,endcolor=white]{vgradient}%
  }%
  \multistep
  {3}
  {%
    \afterstep{\pageDuration{1}}%
    \backgroundstyle[startcolor=pagecolor,endcolor=white,firstgradprogression=\value{substep}]{vgradient}%
  }%
  \multistep
  {10}
  {%
    \afterstep{\pageDuration{0.01}}%
    \mkfactor{\intensity}{(\value{substep}-1)/40}%
    \colorbetween[\intensity]{ecolor}{pagecolor}{white}%
    \backgroundstyle[startcolor=pagecolor,endcolor=ecolor,firstgradprogression=3]{vgradient}%
  }%
  \skipTo{eotitle}
}

\hypertarget{eotitle}{}

\stopAdvancing

\newslide

\renewcommand{\rulestripes}{100}

\newcounter{mstep}

\section{Color Gradients}
\liststepwise[\let\hidestepcontents=\hidesmartignore]
{%
  \concept{Horizontal} \step{or \concept{vertical}; \concept{single}} \step{or \concept{double}.}
  
  \step
  {%
    Parameters:
    \begin{itemize}
    \item Gradient \concept{start} {\bstep[\value{step}=13]{(and
          \concept{middle})}} and \concept{end} color.

      \step[\value{step}=23]{\item Number of \concept{stripes}.}

      \step[\value{step}=33]{\item \concept{Midpoint} of a double
        gradient.}  
      
      \step[\value{step}=43]{\item Gradient \concept{Progression}}
      \step[\value{step}=56]{\par(independent for double gradients).}
    \end{itemize}
  }
  \vfill
  \steponce[\value{step}=0]{\hgradrule{red}{green}{\linewidth}{5ex}}%
  \steponce[\value{step}=1]{\vgradrule{red}{green}{\linewidth}{5ex}}%
  \steponce[\value{step}=2]{\dblvgradrule{red}{yellow}{green}{\linewidth}{5ex}}%
  \steponce[\value{step}>2\and\value{step}<13]
  {%
    \skipTo{eostartend}\\
    \multistep(\setcounter{mstep}{\value{substep}+2}\ifthenelse{\value{step}=\value{mstep}}){10}
    {%
      \mkfactor{\intensity}{(\value{substep}-1)/9}%
      \colorbetween[\intensity]{scolor}{blue}{red}%
      \colorbetween[\intensity]{ecolor}{yellow}{green}%
      \hgradrule{scolor}{ecolor}{\linewidth}{5ex}%
      \ifthenelse{\value{substep}=10}
      {\hypertarget{eostartend}{}\afterstep{\stopAdvancing}}
      {\afterstep{\pageDuration{0.01}}}%
    }%
  }%
  \steponce[\value{step}>12\and\value{step}<23]
  {%
    \skipTo{eomidcolor}\\
    \multistep(\setcounter{mstep}{\value{substep}+12}\ifthenelse{\value{step}=\value{mstep}}){10}
    {%
      \colorbetween{scolor}{blue}{yellow}%
      \mkfactor{\intensity}{(\value{substep}-1)/9}%
      \colorbetween[\intensity]{ecolor}{red}{scolor}%
      \dblhgradrule{blue}{ecolor}{yellow}{\linewidth}{5ex}%
      \ifthenelse{\value{substep}=10}
      {\hypertarget{eomidcolor}{}\afterstep{\stopAdvancing}}
      {\afterstep{\pageDuration{0.01}}}%
    }%
  }%
  \steponce[\value{step}>22\and\value{step}<33]
  {%
    \skipTo{eostripes}\\
    \multistep(\setcounter{mstep}{\value{substep}+22}\ifthenelse{\value{step}=\value{mstep}}){10}
    {%
      \histogram{\dblhgradrule[][10*\value{substep}]{red}{yellow}{green}{\linewidth}{2.5ex}}\\%
      \dblhgradrule[][10*\value{substep}]{red}{yellow}{green}{\linewidth}{2.5ex}%
      \ifthenelse{\value{substep}=10}
      {\hypertarget{eostripes}{}\afterstep{\stopAdvancing}}
      {\afterstep{\pageDuration{0.5}}}%
    }%
  }%
  \steponce[\value{step}>32\and\value{step}<43]
  {%
    \skipTo{eomidpoint}\\
    \multistep(\setcounter{mstep}{\value{substep}+32}\ifthenelse{\value{step}=\value{mstep}}){10}
    {%
      \mkfactor{\midpoint}{(\value{substep}-1)/9}%
      \histogram{\dblhgradrule[\midpoint]{blue}{red}{yellow}{\linewidth}{2.5ex}}\\%
      \dblhgradrule[\midpoint]{red}{yellow}{green}{\linewidth}{2.5ex}%
      \ifthenelse{\value{substep}=10}
      {\hypertarget{eomidpoint}{}\afterstep{\stopAdvancing}}
      {\afterstep{\pageDuration{0.5}}}%
    }%
  }%
  \steponce[\value{step}>42\and\value{step}<56]
  {%
    \skipTo{eofirstprog}\\
    \multistep(\setcounter{mstep}{\value{substep}+42}\ifthenelse{\value{step}=\value{mstep}}){13}
    {%
      \renewcommand{\rulefirstgradprogression}{\value{substep}-7}%
      \histogram{\hgradrule{red}{green}{\linewidth}{2.5ex}}\\%
      \hgradrule{red}{green}{\linewidth}{2.5ex}%
      \ifthenelse{\value{substep}=13}
      {\hypertarget{eofirstprog}{}\afterstep{\stopAdvancing}}
      {\afterstep{\pageDuration{0.5}}}%
    }%
  }%
  \steponce[\value{step}>55\and\value{step}<69]
  {%
    \skipTo{eosecondprog}\\
    \multistep(\setcounter{mstep}{\value{substep}+55}\ifthenelse{\value{step}=\value{mstep}}){13}
    {%
      \renewcommand{\rulefirstgradprogression}{\value{substep}-7}%
      \renewcommand{\rulesecondgradprogression}{7-\value{substep}}%
      \histogram{\dblhgradrule{red}{green}{red}{\linewidth}{2.5ex}}\\%
      \dblhgradrule{red}{green}{red}{\linewidth}{2.5ex}%
      \ifthenelse{\value{substep}=13}
      {\hypertarget{eosecondprog}{}\afterstep{\stopAdvancing}}
      {\afterstep{\pageDuration{0.5}}}%
    }%
  }%
}

\newslide

Applications of gradients:
\centerslidesfalse
\liststepwise*
{%
  \begin{itemize}
  \item 
    \begin{tabular}[t]{@{}l@{}}
      As rules:\\
      \skipTo{eoruledemo}
    \end{tabular}
    \multistep{72}
    {%
      \present
      {%
        \small\renewcommand{\arraystretch}{.9}%
        \begin{tabular}{rp{\totalbarwidth+2em}}
          \multicolumn{2}{c}{\textbf{Compression rates}}\\[2ex]
          \code{compress} & \mkbar[\thesubstep]{51}\\
          \code{gzip -1} & \mkbar[\thesubstep]{62}\\
          \code{gzip -9} & \mkbar[\thesubstep]{66}\\
          \code{bzip2 -1} & \mkbar[\thesubstep]{65}\\
          \code{bzip2 -9} & \mkbar[\thesubstep]{73}
        \end{tabular}%
        }%
      \afterstep{\pageDuration{0.01}}%
      }%
    \step
    {%
      \present
      {%
        \small\renewcommand{\arraystretch}{.9}%
        \begin{tabular}{rp{\totalbarwidth+2em}}
          \multicolumn{2}{c}{\textbf{Compression rates}}\\[2ex]
          \code{compress} & \mkbar{51}\\
          \code{gzip -1} & \mkbar{62}\\
          \code{gzip -9} & \mkbar{66}\\
          \code{bzip2 -1} & \mkbar{65}\\
          \code{bzip2 -9} & \mkbar{73}
        \end{tabular}%
        }%
      \ifthenelse{\boolean{firstactivation}}{\AtShipout{\hypertarget{eoruledemo}{}}\afterstep{\stopAdvancing}}{}%
      }%

  \step
  {%
  \item 
    \begin{tabular}[t]{@{}l@{}}
      As box backgrounds:\\
      \skipTo{eoboxdemo}
    \end{tabular}
    \afterstep{\pageDuration{0.01}}%
    \multistep{10}
    {%
      \mkfactor{\intensity}{(\value{substep}-1)/9}%
      \colorbetween[\intensity]{ecolor}{green}{blue}%
      \colorbetween[\intensity]{scolor}{yellow}{green}%
      \colorbetween{mcolor}{scolor}{ecolor}%
      \complementcolor{tcolor}{mcolor}%
      \raisebox{-.5\height}{\hgradbox{scolor}{ecolor}{\Huge\textsf{\textbf{\mbox{{\textcolor{tcolor}{Groovy!}}}}}}}%
      }%
    \multistep{10}
    {%
      \mkfactor{\intensity}{(\value{substep}-1)/9}%
      \colorbetween[\intensity]{ecolor}{yellow}{green}%
      \colorbetween[\intensity]{scolor}{red}{yellow}%
      \colorbetween{mcolor}{scolor}{ecolor}%
      \complementcolor{tcolor}{mcolor}%
      \raisebox{-.5\height}{\hgradbox{scolor}{ecolor}{\Huge\textsf{\textbf{\textcolor{tcolor}{Groovy!}}}}}%
      }%
    \multistep{10}
    {%
      \mkfactor{\intensity}{(\value{substep}-1)/9}%
      \colorbetween[\intensity]{ecolor}{red}{yellow}%
      \colorbetween[\intensity]{scolor}{blue}{red}%
      \colorbetween{mcolor}{scolor}{ecolor}%
      \complementcolor{tcolor}{mcolor}%
      \raisebox{-.5\height}{\hgradbox{scolor}{ecolor}{\Huge\textsf{\textbf{\textcolor{tcolor}{Groovy!}}}}}%
      }%
    \multistep{10}
    {%
      \mkfactor{\intensity}{(\value{substep}-1)/9}%
      \colorbetween[\intensity]{ecolor}{blue}{red}%
      \colorbetween[\intensity]{scolor}{green}{blue}%
      \colorbetween{mcolor}{scolor}{ecolor}%
      \complementcolor{tcolor}{mcolor}%
      \raisebox{-.5\height}{\hgradbox{scolor}{ecolor}{\Huge\textsf{\textbf{\textcolor{tcolor}{Groovy!}}}}}%
      }%
    \step
    {%
      \colorbetween{mcolor}{green}{blue}%
      \complementcolor{tcolor}{mcolor}%
      \raisebox{-.5\height}{\hgradbox{green}{blue}{\Huge\textsf{\textbf{\textcolor{tcolor}{Groovy!}}}}}%
      \ifthenelse{\boolean{firstactivation}}{\AtShipout{\hypertarget{eoboxdemo}{}}\afterstep{\stopAdvancing}}{}%
      }%
    }

  \step
  {%
  \item As page backgrounds.  \skipTo{eobgnddemo}
    \colorbetween[.22]{ecolor}{pagecolor}{white}%
    \afterstep{\pageDuration{0.01}}%
    \multistep{20}
    {%
      \backgroundstyle
      [%
        startcolor=pagecolor,endcolor=ecolor,firstgradprogression=3,
        rightpanelwidth=\TPpagewidth*\real{.025}*\value{substep},rightpanelcolor=pagecolor,
        leftpanelwidth=\TPpagewidth*\real{.025}*\value{substep},leftpanelcolor=pagecolor,
        toppanelheight=\TPpageheight*\real{.025}*\value{substep},toppanelcolor=pagecolor,
        bottompanelheight=\TPpageheight*\real{.025}*\value{substep},bottompanelcolor=pagecolor%
      ]{vgradient}%
      }%
    \multistep{20}
    {%
      \backgroundstyle
      [%
        startcolordef={rgb}{0.4,0,0},endcolordef={rgb}{0,0.4,0},firstgradprogression=3,
        rightpanelwidth=\TPpagewidth*\real{.025}*(20-\value{substep}),rightpanelcolor=pagecolor,
        leftpanelwidth=\TPpagewidth*\real{.025}*(20-\value{substep}),leftpanelcolor=pagecolor,
        toppanelheight=\TPpageheight*\real{.025}*(20-\value{substep}),toppanelcolor=pagecolor,
        bottompanelheight=\TPpageheight*\real{.025}*(20-\value{substep}),bottompanelcolor=pagecolor%
      ]{vgradient}%
      }%
    }
  \end{itemize}
  }
\hypertarget{eobgnddemo}{}
\end{slide}



\begin{slide}[\slidewidth-40mm,\slideheight-40mm]
\renewcommand{\sliderightmargin}{45mm}%
\renewcommand{\slidetopmargin}{25mm}%
\renewcommand{\slidebottommargin}{25mm}%
\colorbetween[.22]{ecolor}{pagecolor}{white}%
\backgroundstyle[startcolor=pagecolor,endcolor=ecolor,firstgradprogression=3]{vgradient}%

\liststepwise
{%
  Special parameters for page backgrounds:
  \begin{itemize}
  \item Leave space for panels, headers and footers.\\
    \skipTo{eopaneldemo}
    \afterstep{\pageDuration{0.01}}%
    \multistep{20}
    {%
      \backgroundstyle
      [%
      startcolor=pagecolor,endcolor=ecolor,firstgradprogression=3,
      toppanelheight=.1\semcm*\value{substep},toppanelcolor=black,
      bottompanelheight=.1\semcm*\value{substep}%
      ]{vgradient}%
      }%
    \multistep{20}
    {%
      \backgroundstyle
      [%
      startcolor=pagecolor,endcolor=ecolor,firstgradprogression=3,
      rightpanelwidth=.2\semcm*\value{substep},rightpanelcolordef={rgb}{0,0.4,0.6},
      toppanelheight=2\semcm,toppanelcolor=black,
      bottompanelheight=2\semcm%
      ]{vgradient}%
      }%
  \end{itemize}
}

\hypertarget{eopaneldemo}{}
%KEEPCOMMENTS
%</bgndexample-src>
%=================================================================================================================
%<*divexample>
%<<KEEPCOMMENTS
%-----------------------------------------------------------------------------------------------------------------
%
% Divisibility example (demonstrating \step's optional arguments) for the package texpower.sty.
% 
%-----------------------------------------------------------------------------------------------------------------
% We input the generic preamble.

\input{__TPpreamble}
\hypersetup{pdftitle={texpower divisibility example}}


%-----------------------------------------------------------------------------------------------------------------
% Finally, everything is set up. Here we go...
%
\begin{document}
\begin{slide}
  % As the sectioning in the example files starts with \subsection, we `grade down' the sectioning commands. 
  \let\subsubsection\subsection
  \let\subsection\section
%KEEPCOMMENTS
%</divexample>
%<*divexample-src>
%<<KEEPCOMMENTS
%-----------------------------------------------------------------------------------------------------------------
%
\makeslidetitle{\macroname{stepwise} Example: Fooling Around}\label{Sec:ExFooling}
% The following example is just to show that a lot of fancy things are possible by appropriately defining the diverse
% `hooks' offererd by \stepwise.
`Tweaking' the hooks a little allows some truly bizarre applications\dots

% In the following \stepwise command, \activatestep will be set to \textbf. This means the `first' occurrence of a number
% is wider (because bf is an extended font) than all other occurrences of it. To avoid glitches in line breaks, all
% other occurrences of numbers (visible and invisible) need to be of the same width. Hence we define new versions of
% \displaystepcontents and \hidestepcontents.
\def\mydisplay#1{\makebox[\widthof{\textbf{#1}}]{#1}}
\def\myhide#1{\phantom{\makebox[\widthof{\textbf{#1}}]{#1}}}

\newcounter{modulo}
\newcounter{i}

% Finally, we define a custom \step command which sets the optional arguments of \step. 
% We already have introduced the first optional argument, which determines when a \step is activated for the first
% time. Here, this one is left at its default value (\value{step}=\value{stepcommand}). 
% Here, we also use the second optional argument, which has the same syntax as the first one (i.e. it can be surrounded
% either by square brackets or braces), but determines when a \step is active _at all_ (i.e. whether
% \displaystepcontents or \hidestepcontents is used). The default is the internal check \if@first@TP@true, which determines
% whether step number \value{stepcommand} has already been activated for the first time (this is saved internally for
% every step). Now, we redefine this condition to be fulfilled whenever the value of the counter step is divisible by
% the value of the counter stepcommand. 
\def\mystep{\step[](\setcounter{modulo}{\value{step}/\value{stepcommand}*\value{stepcommand}}\ifthenelse{\value{step}=\value{modulo}})}%


% By setting the page duration to 0.5 seconds, we make the following sequence of slides appear automatically one by one,
% one every half second.

\pageDuration{0.5}

% 
% We use the custom command \parstepwise which not only wraps the whole argument of \stepwise into a minipage (because
% otherwise vertical spacing goes haywire, don't ask me why), but also gives substance to steps. We use the starred
% version of this command so that the number 1 appears on the first slide generated by \parstepwise.
%
% \activatestep is set to \textbf to emphasize the number the divisors of which are displayed.
\parstepwise*[\let\activatestep=\textbf\let\displaystepcontents=\mydisplay\let\hidestepcontents=\myhide]
{%
  \huge
  \setcounter{i}{0}%
    Divisibility demo:
    
    % We just have to generate the appropriate number of \mystep commands which display `their' numbers.
    \whiledo{\value{i}<40}{\stepcounter{i}\mystep{\arabic{i}} }%

  }

% Stop automatic advancing of pages.

\stopAdvancing
%KEEPCOMMENTS
%</divexample-src>
%=================================================================================================================
%<*fancyexample>
%<<KEEPCOMMENTS

\documentclass[KOMA,letterpaper,landscape,display,calcdimensions]{powersem}

\usepackage{graphicx}

\usepackage{soul}

\usepackage{palatino}

\usepackage[ps2pdf,pdfpagemode={FullScreen}]{hyperref}
\usepackage{fixseminar}
\usepackage[whitebackground]{texpower}

\slidesmag{5}

\slideframe{none}
\pagestyle{empty}

\renewcommand{\slideleftmargin}{2cm}
\renewcommand{\sliderightmargin}{2cm}
\renewcommand{\slidetopmargin}{2cm}
\renewcommand{\slidebottommargin}{2cm}

\newcounter{nosteps}
\setcounter{nosteps}{10} % Controls the "resolution" - a value of 100 gives a *very* long compilation time.
\newcounter{mycount}

\makeatletter
\DeclareRobustCommand*\appearI[1]
{%
  \SOUL@setup
  \def\SOUL@everytoken{\makebox[\widthof{\the\SOUL@token\SOUL@setkern\SOUL@charkern}]{\scalebox{#1}{\the\SOUL@token\SOUL@setkern\SOUL@charkern}}}%
  \SOUL@%
  }%
\DeclareRobustCommand*\appearII[1]
{%
  \SOUL@setup
  \def\SOUL@everytoken{\makebox[\widthof{\the\SOUL@token\SOUL@setkern\SOUL@charkern}*\real{#1}]{\the\SOUL@token\SOUL@setkern\SOUL@charkern}}%
  \SOUL@%
  }%
\DeclareRobustCommand*\appearIII[1]
{%
  \SOUL@setup
  \def\SOUL@everyspace{\rule{.3em}{\fboxrule}}%
  \def\SOUL@everysyllable{\the\SOUL@syllable\SOUL@setkern\SOUL@charkern\rule{#1}{\fboxrule}}%
  \SOUL@%
  }%
\DeclareRobustCommand*\appearIV[1]
{%
  \SOUL@setup
  \def\SOUL@everytoken{\makebox[\widthof{\the\SOUL@token\SOUL@setkern\SOUL@charkern}+2em-2em*\real{#1}]{\scalebox{#1}{\setcounter{mycount}{\value{nosteps}*4-\value{nosteps}*4*\real{#1}}\rotatebox[origin=c]{\themycount}{\the\SOUL@token\SOUL@setkern\SOUL@charkern}}}}%
  \SOUL@%
  }%
\makeatother

\begin{document}
\begin{slide}
  \pageDuration{1}%
  \stepwise
  {%
    \begin{center}
      \movie*{\value{nosteps}}{0.01}[\pageDuration{1}]
      {%
        \makebox[0pt]
        {\rule{2\textwidth-2\textwidth/\value{nosteps}*\value{substep}}{0pt}`Twas brillig, and the slithy toves}%
        }

      \movie*{\value{nosteps}}{0.01}[\pageDuration{1}]
      {%
        \mkfactor{\mag}{\value{substep}/\value{nosteps}}%
        \scalebox{\mag}{Did gyre and gimble in the wabe:}%
        }

      \movie*{\value{nosteps}}{0.01}[\pageDuration{1}]
      {%
        \mkfactor{\mag}{5*(\value{nosteps}-\value{substep})/\value{nosteps}+1pt}%
        \makebox[0pt]{\scalebox{\mag}[1]{All mimsy were the borogoves,}}%
        }

      \movie*{\value{nosteps}}{0.01}[\pageDuration{1}]
      {%
        \mkfactor{\mag}{\value{substep}/\value{nosteps}}%
        \colorbetween[\mag]{mycolor}{textcolor}{pagecolor}
        \textcolor{mycolor}{And the mome raths outgrabe.}%
        }

      \medskip

      \movie*{\value{nosteps}}{0.01}[\pageDuration{1}]
      {%
        \mkfactor{\mag}{\value{substep}/\value{nosteps}}%
        \appearI{\mag}{"Beware the Jabberwock, my son!}%
        }

      \movie*{\value{nosteps}}{0.01}[\pageDuration{1}]
      {%
        \mkfactor{\mag}{\value{substep}/\value{nosteps}}%
        \appearII{\mag}{The jaws that bite, the claws that catch!}%
        }

      \movie{\value{nosteps}}{0.01}[\pageDuration{0.01}]
      {%
        \makebox[0pt]{\appearIII{(\thenosteps pt-\thesubstep pt)*\real{1.5}}{Beware the Jubjub bird, and shun}}%
        }%
      %
      \step{\afterstep{\pageDuration{1}}Beware the Jubjub bird, and shun}

      \movie*{\value{nosteps}}{0.01}[\pageDuration{1}]
      {%
        \mkfactor{\mag}{\value{substep}/\value{nosteps}}%
        \setcounter{mycount}{2*(\value{nosteps}-\value{substep})}%
        \strut\rotatebox[origin=c]{\themycount}{\makebox[0pt]{\smash{\scalebox{\mag}{The frumious Bandersnatch!"}}}}%
        }

      \medskip

      \movie*{\value{nosteps}}{0.01}[\pageDuration{1}]
      {%
        \mkfactor{\mag}{\value{substep}/\value{nosteps}}%
        \hspace*{\fill}%
        \emph
        {%
          \rlap
          {%
            \makebox[\widthof{Lewis Carroll}]
            {\rule{5cm-5cm/\value{nosteps}*\value{substep}}{0pt}\strut\raisebox{1cm-1cm/\value{nosteps}*\value{substep}}[0pt][0pt]{\appearIV{\mag}{Lewis Carroll}}}%
            }%
          \rlap
          {%
            \makebox[\widthof{Lewis Carroll}]
            {\strut\raisebox{1.5cm-1.5cm/\value{nosteps}*\value{substep}}[0pt][0pt]{\appearIV{\mag}{Lewis Carroll}}\rule{4cm-4cm/\value{nosteps}*\value{substep}}{0pt}}%
            }%
          \rlap
          {%
            \makebox[\widthof{Lewis Carroll}]
            {\rule{3cm-3cm/\value{nosteps}*\value{substep}}{0pt}\strut\raisebox{-.5cm+.5cm/\value{nosteps}*\value{substep}}[0pt][0pt]{\appearIV{\mag}{Lewis Carroll}}}%
            }%
          \makebox[\widthof{Lewis Carroll}]
          {\strut\raisebox{-1cm+1cm/\value{nosteps}*\value{substep}}[0pt][0pt]{\appearIV{\mag}{Lewis Carroll}}\rule{6cm-6cm/\value{nosteps}*\value{substep}}{0pt}}%
          }%
        }
    \end{center}
    }
\end{slide}
\end{document}
%KEEPCOMMENTS
%</fancyexample>
%=================================================================================================================
%<*foilsdemo>
%<<KEEPCOMMENTS
%-----------------------------------------------------------------------------------------------------------------
%
% Simple examples the for combining the foils class with the dynamic features provided by the package texpower.sty. 
% 
%-----------------------------------------------------------------------------------------------------------------

\documentclass[landscape]{foils}

% foils understands the landscape option, but to get it through to pdflatex, we need to set \pdfpageheight etc. The
% package fixseminar does this.

\usepackage{fixseminar}

%-----------------------------------------------------------------------------------------------------------------
% The texpower package is loaded. 
% We give the display option so dynamic features are enabled.
%
\usepackage[display]{texpower}


\begin{document}

\title{The \code{texpower} Package\\{\normalfont \texttt{foils} Demo}}
\author{Stephan Lehmke\\\code{mailto:Stephan.Lehmke@cs.uni-dortmund.de}}
\maketitle


\foilhead{A list environment}
  
% The \pause command `splits' the current page at the place it appears, producing two pages, one with everything which
% came before the \pause command, one containing this and additionally the stuff coming after \pause. When these pages
% are presented with acrobar reader in full screen mode (or any other viewer with this capability), the presentation
% will appear to `stop' at the point the \pause command was issued and `resume' in the moment the presenter switches to
% the next page.

\pause

% As \pause forces a paragraph break, it can not be used to separate a description label from the associated text. For
% this, we use the (very flexible) \stepwise command. Inside the argument of \stepwise, an arbitrary number of \step
% commands may occur. \stepwise will produce as many pages as there are \step commands, making the arguments of the
% \step commands appear ``one by one''.

\stepwise
{%
  \begin{description}
  \item[foo.] \step{bar.}
  \step{\item[baz.]} \step{qux.}
  \end{description}
  }



\foilhead{An aligned equation}

\pause

% Normally for \stepwise, if a \step is not yet active, its argument is ignored completely. This would disturb
% alignments, because the width changes with every new activated \step.
% \parstepwise is a variant of \stepwise where the argument of an inactive \step is put into a \phantom, leaving the
% proper amount of white space.

\parstepwise
{%
  % Using eqnarray with equation numbers here means all equation numbers will be visible from the outset, because only
  % the contents of the lines are `filled in'. See the full demo for an example of aligned equations where equation
  % numbers `appear'.
  \begin{eqnarray}
    %
    % When the argument of \step is put into a box (as it happens with \parstepwise), tabulators can not go in there. As
    % we want the equals sign to appear at the same time as the right side of the equation, we use \restep for the
    % latter. \restep is like \step, but it appears at the same time as the previous \step command.
    % 
    \sum_{i=1}^{n} i & \step{=} & \restep{1 + 2 + \cdots + (n-1) + n}\\
    %
                     & \step{=} & \restep{1 + n + 2 + (n-1) + \cdots}\\
    %
                     & \step{=} & \restep
                                  {% We can nest \step commands inside each other. The order of execution is just the
                                   % order of appearance, independent of nesting.
                                   % \switch is a variant of \step which takes two arguments and toggles between them on
                                   % activation. This way, we can make the \underbrace `appear'.
                                   % We insert a \vphantom in the first argument so that the equation numbers will be
                                   % placed correctly whether or not the underbrace is didplayed.
                                    \switch
                                    {%
                                      \vphantom{\underbrace{(1 + n) + \cdots + (1 + n)}_{\times\frac{n}{2}}}%
                                      (1 + n) + \cdots + (1 + n)%
                                      }
                                    {\underbrace{(1 + n) + \cdots + (1 + n)}_{\times\frac{n}{2}}}%
                                    }
                                  \\
    %
    % This is another nested application of \step. Note that the spacing of \cdot has to be corrected manually by
    % inserting {} left of it, because otherwise it would behave like a prefix operator.
    %
                     & \step{=} & \restep{\frac{(1 + n)\step{{}\cdot n}}{\restep{2}}}
  \end{eqnarray}
}




\foilhead{An array}

\stepwise
{% With arrays, beware of problems with automatic calculation of cell widths.
 % 
 % If you want all widths to be calculated automatically, you need to use \parstepwise, with the consequence that
 %   a) tabulators or newlines can not go into the argument of \step,
 %   b) the array `structure' (rules) will be completely visible right from the beginning.
 %   
 % If you want to use \stepwise for being able to build the `structure' (like \hilne's) dynamically (as done in the
 % following), you have to make sure that the cell widths are correct from the very first line, because otherwise the
 % array will expand horizontally, destroying the dynamic effect. This can be assured by
 %   a) using only p cells,
 %   b) making sure all the cells in the first line are at least as wide as the widest cell which will appear later. If
 %      you are using the calc package, this is easiest by putting \makebox[\widthof{widest entry}]{first entry} into
 %      the first cell. Otherwise, you can use \settowidth.
 %      
  \begin{displaymath}
    \begin{array}{rrrrr}
      \step
      {%
            n &        \log n        &        n\log n       & \lefteqn{n^2}\phantom{25} & \lefteqn{2^n}\phantom{32} \\
        \hline%
        }%
      \step{0 &} \step{\textrm{---}  &} \step{\textrm{---}  &} \step{0                  &} \step{1                  \\}%
      \step{1 &} \step{0\phantom{.6} &} \step{0\phantom{.8} &} \step{1                  &} \step{2                  \\}%
      \step{2 &} \step{1\phantom{.6} &} \step{2\phantom{.8} &} \step{4                  &} \step{4                  \\}%
      \step{3 &} \step{1.6           &} \step{4.8           &} \step{9                  &} \step{8                  \\}%
      \step{4 &} \step{2\phantom{.6} &} \step{8\phantom{.8} &} \step{16                 &} \step{16                 \\}%
      \step{5 &} \step{2.3           &} \step{11.6          &} \step{25                 &} \step{32                   }%
    \end{array}
  \end{displaymath}
}




\foilhead{A picture}

\pause

\begin{center}%
  \stepwise
  {%
    \setlength{\unitlength}{1.6cm}%
    \delimitershortfall-1sp% Just for the nested braces
    \begin{picture}(14,2)
      \put(0,1){\vector(1,0){1}}
      \put(0.5,0.5){\makebox(0,0){\small $x(t)$}}
      \put(13,1){\vector(1,0){1}}
      \put(13.5,0.5){\makebox(0,0){\small $y(t)$}}
      \step
      {
        \put(1,1){\line(3,2){1.5}}
        \put(1,1){\line(3,-2){1.5}}
        \put(2.5,0){\line(0,1){2}}
        \put(2,1){\makebox(0,0){\large $\varphi$}}
        }
      \step
      {
        \put(2.5,1){\vector(1,0){3.5}}
        \put(4.25,0.5){\makebox(0,0){\small $F_t = \varphi\left(x(t)\right)$}}
        }
      \step
      {
        \put(6,0){\framebox(2,2){\large $\Phi$}}
        }
      \step
      {
        \put(8,1){\vector(1,0){3.5}}
        %
        % Here, we find another nested use of \step inside \step.
        % \bstep is a variant of \step which _always_ puts its argument into a box for leaving the correct amount of
        % white space. We cannot use \parstepwise here because \put can't go into a box. Hence, just using \step for
        % building the nested formula on the next line would give the wrong size for the nested braces.
        % 
        \put(9.75,0.5){\makebox(0,0){\small $G_t = \Phi\left(\bstep{\varphi\left(\bstep{x(t)}\right)}\right)$}}
        }
      \step
      {
        \put(13,1){\line(-3,2){1.5}}
        \put(13,1){\line(-3,-2){1.5}}
        \put(11.5,0){\line(0,1){2}}
        \put(12,1){\makebox(0,0){\large $\delta$}}
        }
    \end{picture}%
    }%
\end{center}%
\end{document}
%KEEPCOMMENTS
%</foilsdemo>
%=================================================================================================================
%<*hilitexample>
%<<KEEPCOMMENTS
%-----------------------------------------------------------------------------------------------------------------
%
% Highlighting example for the package texpower.sty.
% 
%-----------------------------------------------------------------------------------------------------------------
% Use slifonts. 

\RequirePackage{tpslifonts}

% Input the generic preamble.

\input{__TPpreamble}
\hypersetup{pdftitle={texpower highlighting example}}

% The package soul is needed for \highlighttext to work.

\usepackage{soul}

%-----------------------------------------------------------------------------------------------------------------
% Finally, everything is set up. Here we go...
%
\begin{document}
\begin{slide}
%KEEPCOMMENTS
%</hilitexample>
%<*hilitexample-src>
%<<KEEPCOMMENTS
%-----------------------------------------------------------------------------------------------------------------
%
\makeslidetitle{\macroname{stepwise} Example: Highlighting Text}\label{Sec:Exhl}
% The default for \step's which are not yet `active' is to be `invisible'. In preceding examples, we have redefined
% internal macros like \hidestepcontents or \activatestep to achieve special `highlighting' effects.
% Here, we demonstrate how make text `stand out' from the background incrementally without having it appear from thin
% air. 

% The first example demonstrates how to make \step its argument `stand out' instead of making it appear `out of
% nowhere'. 
  
\makeatletter
\ifthenelse{\boolean{TPcolor}}% Can we use colors?
{% Yes. In this case \step should change colors from `dimmed' to `active'.

  % The command \hidedimmed switches colors to `dimmed', the command \highlightenhanced switches colors to `enhanced'.
  \liststepwise[\let\hidestepcontents=\hidedimmed\let\activatestep=\highlightenhanced]
  {%
    \vspace*{\fill}
    \begin{quote}
      \Large%
      \step{Instead of making things appear out of `thin air',} \step{we can also make them appear `out of the
        background'} \step{by incrementally changing color from \concept{inactive}} \step{to \concept{active}.}
      \step{This also works with \emph{color emphasis}} \step{and \concept{math coloring}: $a=b^2$.}
    \end{quote}
    \vspace*{\fill}\vspace*{\fill}
    }%
  }
{% No. In this case we have to `handwave' using the soul package.
  % Instead of hiding its argument, the new command \myhide just changes the font size (maintaining the overall text
  % length though).
  \def\myhide
  {%
    \leavevmode%
    \SOUL@setup
    \def\SOUL@everytoken{\makebox[\widthof{\the\SOUL@token}][s]{\rule[\depthof{\the\SOUL@token}*-1]{0pt}{\depthof{\the\SOUL@token}+\heightof{\the\SOUL@token}}\hrulefill\tiny\the\SOUL@token\hrulefill}\SOUL@setkern\SOUL@charkern}%
    \SOUL@%
    }
  
  \liststepwise[\let\hidestepcontents=\myhide\setcounter{firststep}{0}]
  {%
    \vspace*{\fill}
    \begin{quote}
      \LARGE%
      \step{Instead of making things appear out of `thin air',} \step{we can also make them appear `out of the
        background'} \step{incrementally.}
    \end{quote}
    \vspace*{\fill}\vspace*{\fill}
    }%
  }
\makeatother

\newslide

% Next, it is demonstrated how we can `flip through' an itemize environment by just highlighting the items in turn
% instead of making them appear one by one. 
%
% Define a macro \mystep which implements the highlighting effect.
\ifthenelse{\boolean{TPcolor}}% Can we use colors?
{% Yes. In this case highlighting is implemented by switching color.
  \def\mystep% Note that \mystep takes no argument. It just changes the way the next item is displayed.
  {% 
    \usecolorset{stwcolors}%             Restore the undimmed colors valid at the beginning of \stepwise.
    \dstep[][\boolean{firstactivation}]% \dstep switches colors. The optional argument makes it appear only once.
    }%
}
{% No. In this case highlighting is implemented by redefining \labelitemi.
  \def\mystep
  {%
    \switch[][\boolean{firstactivation}]% The optional arguments make \switch act only once.
    {\def\labelitemi{\origmath{\gg}}}{\def\labelitemi{\origmath{\cdot}}}%
    }%
  }

% We define a custom itemize environment which automatically adds calls to \mystep:
\newenvironment{stepitemize}
{%
  \huge
  \begin{itemize}
    \let\origitem=\item
    % Here, the \mystep command is hidden inside \item
    \renewcommand{\item}{\mystep\origitem}%
    }
  {%
  \end{itemize}
  }
    
Instead of displaying incrementally, we can just `flip through' some items by highlighting them:
  
% Note that we use the starred version of \liststepwise so that the first item is highlighted on the first slide
% produced by \liststepwise.
%
\liststepwise*
{%
  \begin{stepitemize}
  \item Item 1
  \item Item 2
  \item Item 3
  \end{stepitemize}
  }

\pause

% The following example demonstrates highlighting inside a paragraph using \highlighttext. By redefining \activatestep
% to \highlighttext, the argument of every \step will be highlighted when the \step is activated for the first
% time. Note that highlighting doesn't influence line breaks because \highlighttext is implemented using the soul
% package.
%
% Again, we define \hidestepcontents to display its argument, so that the complete text is visible from the outset. 

\stepwise[\let\activatestep=\highlighttext\let\hidestepcontents=\displayidentical]
{%
  \vspace*{\fill}
  \begin{minipage}{\linewidth}
    \LARGE
    \step{Inside} a paragraph, we can \step{highlight} text \step{without influencing} \step{line breaks}.
  \end{minipage}
  \vspace*{\fill}
  }
\newslide
%KEEPCOMMENTS
%</hilitexample-src>
%=================================================================================================================
%<*ifmslidemo>
%<<KEEPCOMMENTS
%-----------------------------------------------------------------------------------------------------------------
%
% Simple examples the for combining the ifmslide package with the dynamic features provided by the package texpower.sty. 
% 
%-----------------------------------------------------------------------------------------------------------------

\documentclass[KOMA,landscape,display]{powersem}

\usepackage[stmo,button]{ifmslide}

\slidepagestyle{sidebar}
\centerslidesfalse

\hypersetup{linkcolor=red}

\pagestyle{plain}

\panellogo{fig-2}

\renewcommand{\currentpagevalue}{\value{slide}}

\begin{document}


\begin{slide}
\title{The \code{texpower} Package\\{\normalfont \texttt{ifmslide} Demo}}
\author{Stephan Lehmke\\\code{mailto:Stephan.Lehmke@cs.uni-dortmund.de}}
\maketitle

\tableofcontents
\end{slide}

\begin{slide}
\section{A list environment}
  
% The \pause command `splits' the current page at the place it appears, producing two pages, one with everything which
% came before the \pause command, one containing this and additionally the stuff coming after \pause. When these pages
% are presented with acrobar reader in full screen mode (or any other viewer with this capability), the presentation
% will appear to `stop' at the point the \pause command was issued and `resume' in the moment the presenter switches to
% the next page.

\pause

% As \pause forces a paragraph break, it can not be used to separate a description label from the associated text. For
% this, we use the (very flexible) \stepwise command. Inside the argument of \stepwise, an arbitrary number of \step
% commands may occur. \stepwise will produce as many pages as there are \step commands, making the arguments of the
% \step commands appear ``one by one''.

\stepwise
{%
  \begin{description}
  \item[foo.] \step{bar.}
  \step{\item[baz.]} \step{qux.}
  \end{description}
  }



\section{An aligned equation}

\pause

% Normally for \stepwise, if a \step is not yet active, its argument is ignored completely. This would disturb
% alignments, because the width changes with every new activated \step.
% \parstepwise is a variant of \stepwise where the argument of an inactive \step is put into a \phantom, leaving the
% proper amount of white space.

\parstepwise
{%
  % Using eqnarray with equation numbers here means all equation numbers will be visible from the outset, because only
  % the contents of the lines are `filled in'. See the full demo for an example of aligned equations where equation
  % numbers `appear'.
  \begin{eqnarray}
    %
    % When the argument of \step is put into a box (as it happens with \parstepwise), tabulators can not go in there. As
    % we want the equals sign to appear at the same time as the right side of the equation, we use \restep for the
    % latter. \restep is like \step, but it appears at the same time as the previous \step command.
    % 
    \sum_{i=1}^{n} i & \step{=} & \restep{1 + 2 + \cdots + (n-1) + n}\\
    %
                     & \step{=} & \restep{1 + n + 2 + (n-1) + \cdots}\\
    %
                     & \step{=} & \restep
                                  {% We can nest \step commands inside each other. The order of execution is just the
                                   % order of appearance, independent of nesting.
                                   % \switch is a variant of \step which takes two arguments and toggles between them on
                                   % activation. This way, we can make the \underbrace `appear'.
                                   % We insert a \vphantom in the first argument so that the equation numbers will be
                                   % placed correctly whether or not the underbrace is didplayed.
                                    \switch
                                    {%
                                      \vphantom{\underbrace{(1 + n) + \cdots + (1 + n)}_{\times\frac{n}{2}}}%
                                      (1 + n) + \cdots + (1 + n)%
                                      }
                                    {\underbrace{(1 + n) + \cdots + (1 + n)}_{\times\frac{n}{2}}}%
                                    }
                                  \\
    %
    % This is another nested application of \step. Note that the spacing of \cdot has to be corrected manually by
    % inserting {} left of it, because otherwise it would behave like a prefix operator.
    %
                     & \step{=} & \restep{\frac{(1 + n)\step{{}\cdot n}}{\restep{2}}}
  \end{eqnarray}
}




\section{An array}

\stepwise
{% With arrays, beware of problems with automatic calculation of cell widths.
 % 
 % If you want all widths to be calculated automatically, you need to use \parstepwise, with the consequence that
 %   a) tabulators or newlines can not go into the argument of \step,
 %   b) the array `structure' (rules) will be completely visible right from the beginning.
 %   
 % If you want to use \stepwise for being able to build the `structure' (like \hilne's) dynamically (as done in the
 % following), you have to make sure that the cell widths are correct from the very first line, because otherwise the
 % array will expand horizontally, destroying the dynamic effect. This can be assured by
 %   a) using only p cells,
 %   b) making sure all the cells in the first line are at least as wide as the widest cell which will appear later. If
 %      you are using the calc package, this is easiest by putting \makebox[\widthof{widest entry}]{first entry} into
 %      the first cell. Otherwise, you can use \settowidth.
 %      
  \begin{displaymath}
    \begin{array}{rrrrr}
      \step
      {%
            n &        \log n        &        n\log n       & \lefteqn{n^2}\phantom{25} & \lefteqn{2^n}\phantom{32} \\
        \hline%
        }%
      \step{0 &} \step{\textrm{---}  &} \step{\textrm{---}  &} \step{0                  &} \step{1                  \\}%
      \step{1 &} \step{0\phantom{.6} &} \step{0\phantom{.8} &} \step{1                  &} \step{2                  \\}%
      \step{2 &} \step{1\phantom{.6} &} \step{2\phantom{.8} &} \step{4                  &} \step{4                  \\}%
      \step{3 &} \step{1.6           &} \step{4.8           &} \step{9                  &} \step{8                  \\}%
      \step{4 &} \step{2\phantom{.6} &} \step{8\phantom{.8} &} \step{16                 &} \step{16                 \\}%
      \step{5 &} \step{2.3           &} \step{11.6          &} \step{25                 &} \step{32                   }%
    \end{array}
  \end{displaymath}
}




\section{A picture}

\pause

\begin{center}%
  \stepwise
  {%
    \setlength{\unitlength}{1.5\semcm}%
    \delimitershortfall-1sp% Just for the nested braces
    \begin{picture}(14,2)
      \put(0,1){\vector(1,0){1}}
      \put(0.5,0.5){\makebox(0,0){\small $x(t)$}}
      \put(13,1){\vector(1,0){1}}
      \put(13.5,0.5){\makebox(0,0){\small $y(t)$}}
      \step
      {
        \put(1,1){\line(3,2){1.5}}
        \put(1,1){\line(3,-2){1.5}}
        \put(2.5,0){\line(0,1){2}}
        \put(2,1){\makebox(0,0){\large $\varphi$}}
        }
      \step
      {
        \put(2.5,1){\vector(1,0){3.5}}
        \put(4.25,0.5){\makebox(0,0){\small $F_t = \varphi\left(x(t)\right)$}}
        }
      \step
      {
        \put(6,0){\framebox(2,2){\large $\Phi$}}
        }
      \step
      {
        \put(8,1){\vector(1,0){3.5}}
        %
        % Here, we find another nested use of \step inside \step.
        % \bstep is a variant of \step which _always_ puts its argument into a box for leaving the correct amount of
        % white space. We cannot use \parstepwise here because \put can't go into a box. Hence, just using \step for
        % building the nested formula on the next line would give the wrong size for the nested braces.
        % 
        \put(9.75,0.5){\makebox(0,0){\small $G_t = \Phi\left(\bstep{\varphi\left(\bstep{x(t)}\right)}\right)$}}
        }
      \step
      {
        \put(13,1){\line(-3,2){1.5}}
        \put(13,1){\line(-3,-2){1.5}}
        \put(11.5,0){\line(0,1){2}}
        \put(12,1){\makebox(0,0){\large $\delta$}}
        }
    \end{picture}%
    }%
\end{center}%
\end{slide}
\end{document}
%KEEPCOMMENTS
%</ifmslidemo>
%=================================================================================================================
%<*mathexample>
%<<KEEPCOMMENTS
%-----------------------------------------------------------------------------------------------------------------
%
% Math example for the package texpower.sty.
% 
%-----------------------------------------------------------------------------------------------------------------
% Enable all color emphasis and highlighting options. Use white background and slifonts.

\PassOptionsToPackage{coloremph,colormath,colorhighlight,whitebackground}{texpower}

% Input the generic preamble.

\input{__TPpreamble}

\usepackage{tpslifonts}

\hypersetup{pdftitle={texpower math example}}


%-----------------------------------------------------------------------------------------------------------------
% Packages and Preamble settings individual for this example.

% We write some aligned equations.

\usepackage{amsmath}
% Make nested braces grow.
\delimitershortfall-1sp

%-----------------------------------------------------------------------------------------------------------------
% Finally, everything is set up. Here we go...
%
\begin{document}
\begin{slide}
%KEEPCOMMENTS
%</mathexample>
%<*mathexample-src>
%<<KEEPCOMMENTS
%-----------------------------------------------------------------------------------------------------------------
%
\makeslidetitle{\macroname{stepwise} Example: An Aligned Equation}\label{Sec:ExEq}
% In the following, an aligned system of equations is built incrementally. We use the custom command \liststepwise to
% avoid glitches in vertical spacing.
%
\liststepwise%
{%
  %
  % This is just for compressing the equations so they can be squeezed on one slide.
  %
  \fontsize{7.8pt}{9pt}\selectfont
  \renewcommand{\arraystretch}{0}%
  \setlength{\arraycolsep}{0pt}%
  \setlength{\abovedisplayskip}{0pt}%
  \setlength{\belowdisplayskip}{0pt}%
  %
  % \highlightboxed will be used for underlaying some formulas with color. To minimize overlap, the width of the outer
  % frame is reduced.
  \setlength{\highlightboxsep}{1pt}%
  %
  \begin{align*}
    \lefteqn
    {%
      \min
      \left(
        % The nested braces are filled `from outer to inner'. This means nesting a lot of steps inside each other...
        % The outermost brace is displayed from the outset.
        % The first step (which follows right here) displays the next inner brace (the first argument of \min), filled
        % with an almost `empty' array (apart from one comma and some dots).
        % \bstep is used to get appropriate white space when the step is not yet active.
        \bstep
        {\max
          \left(
            \begin{array}{l}
              % The next two steps fill in the lines of the array.
              \bstep{\min\left(F'(x),\min\left(F_1(x),G_1(y)\right)\right)},\\[-2ex]
              \vdots\\
              \bstep{\min\left(F'(x),\min\left(F_n(x),G_n(y)\right)\right)}
            \end{array}
          \right)
          },
        % After the first brace is filled, the next step provides the second argument of \min.
        \bstep{\min\left(G_i(y),H_i(z)\right)}
      \right)
      }
    &
    % The next couple of steps will create the remaining lines of the aligned equations. These need to be
    % insubstantial (as is the default for \liststepwise), because & can't go in a box.
    % As a consequence, the horizontal alignment cannot kick in until the last step is performed. This would make the
    % alignment `flicker' sidewise.
    % So we have to bite the bullet and duplicate the widest entry here (invisibly), so that the horizontal alignment
    % is constant during all steps. *sigh*
    \phantom
    {%
      {}=
      \min
      \left(
        F'(x),
        \min
        \left(
          \max
          \left(
            \begin{array}{l}
              \min\left(F_1(x),\min\left(G_1(y),G_i(y)\right)\right),\\[-1.5ex]
              \vdots\\[-.5ex]
              \min\left(F_n(x),\min\left(G_n(y),G_i(y)\right)\right)
            \end{array}
          \right),
          H_i(z)
        \right)
      \right)
      }
    % The next step displays two lines at a time, but incompletely, i.e. some parts are missing (which are inside
    % nested calls of \bstep).
    % This way, it is demonstrated how the arguments of the nested \min's are reordered.
    \step
    {%
      \\
      &=
      \max
      \left(
        % The macro \activatestep is used by \stepwise to `wrap' the argument of a \bstep command at the _first_ time
        % it appears.
        % Usually, it does nothing. Now, we redefine it to highlight its background, so it is easier to spot the
        % places where the additional arguments were inserted.
        \let\activatestep\highlightboxed
        \begin{array}{l}
          \min
          \left(
            % The inner \bstep's display the missing arguments, which are completely identical in both lines.
            % It is intended that all the missing arguments appear at the same time, so \rebstep is used for the
            % remaining arguments which have been left out.
            \min\left(\bstep{F'(x)},\min\left(\rebstep{F_1(x),G_1(y)}\right)\right),\min\left(G_i(y),H_i(z)\right)
          \right),\\[-2ex]
          \vdots\\[-1ex]
          \min
          \left(
            \min\left(\rebstep{F'(x)},\min\left(\rebstep{F_n(x),G_n(y)}\right)\right),\min\left(G_i(y),H_i(z)\right)
          \right)
        \end{array}
      \right)
      \\
      &=
      \max
      \left(
        \let\activatestep\highlightboxed
        \begin{array}{l}
          \min
          \left(
            \min\left(
              % Here are the remaining arguments of \min which are all to be displayed in one step (together with
              % those from the previous line). 
              \rebstep{F'(x)},\min\left(\rebstep{F_1(x)},\min\left(\rebstep{G_1(y)},G_i(y)\right)\right)
            \right),
            H_i(z)
          \right),\\[-2.5ex]
          \vdots\\[-1.5ex]
          \min
          \left(
            \min\left(
              \rebstep{F'(x)},\min\left(\rebstep{F_n(x)},\min\left(\rebstep{G_n(y)},G_i(y)\right)\right)
            \right),
            H_i(z)
          \right)
        \end{array}
      \right)
      }
    \step
    {%
      \\
      &=
      \min
      \left(
        F'(x),
        \min
        \left(
          \max
          \left(
            \begin{array}{l}
              \min\left(F_1(x),\min\left(G_1(y),G_i(y)\right)\right),\\[-1.5ex]
              \vdots\\[-.5ex]
              \min\left(F_n(x),\min\left(G_n(y),G_i(y)\right)\right)
            \end{array}
          \right),
          H_i(z)
        \right)
      \right)
      }
  \end{align*}
  }%
  \newslide
%KEEPCOMMENTS
%</mathexample-src>
%=================================================================================================================
%<*panelexample>
%<<KEEPCOMMENTS
%------------------------------------------------------------------------------
%
% Example for the panel facilities of TeXPower.
% 
%------------------------------------------------------------------------------
% Enable all color emphasis and highlighting options. Use white
% background and slifonts. 

\PassOptionsToPackage{coloremph,colormath,colorhighlight,whitebackground}
{texpower}

% Input the generic preamble.

\input{__TPpreamble}

\usepackage{tpslifonts}

\hypersetup{pdftitle={texpower panel example}}

%------------------------------------------------------------------------------
% Settings individual for this example.

\renewcommand{\buttonsep}{2pt}
\renewcommand{\buttonshadowhshift}{1pt}
\renewcommand{\buttonshadowvshift}{-1pt}

\usepackage{graphicx}
% The mps extension isn't supported out-of-the-box for latex+dvips
\ifthenelse{\boolean{psspecialsallowed}}{%
\DeclareGraphicsExtensions{.mps}}{}

\newcounter{typed}

\makeatletter
\newcommand{\storehead}[3]
{%
  \global\let\@typed=\empty
  \setcounter{typed}{0}%
  \def\@typeto{#2}%
  \let\type@next=\@typeit
  \type@next#3\@nil
  \global\let#1=\@typed
}

\newcommand{\@typeit}[1]
{%
  \ifx\@nil#1
   \else
    \stepcounter{typed}%
    \ifthenelse{\value{typed}>\@typeto}
    {\let\type@next=\@gobbletail}
    {\g@addto@macro\@typed{#1}}%
    \expandafter\type@next
  \fi
}

\long\def\@gobbletail#1\@nil{}

\newcounter{i}

\newsavebox\logobox
\newsavebox\hookbox

\makeatother

\mklength{\slidetopmargin}{\ht\logobox*\ratio{1cm}{\semcm}}

\newlength{\buttonwidth}

\newcommand{\mybutton}[2]
{\raisebox{\depth}{\makebox[\buttonwidth][l]{\button[\buttonwidth]{#1}{#2}}}}

\slidesonlyfalse\notestrue\noxcomment

\begin{document}

% Because the Context Support Macros are loaded at BeginDocument
\savebox\hookbox{\includegraphics[width=1cm]{fig-2}}
\makeatletter
\savebox\logobox{\includegraphics[width=\strip@pt\paperwidth truept]{fig-3}}
\makeatother
\savebox\logobox{\raisebox{0cm}[\height-2ex][0pt]{\rlap{\usebox{\logobox}}}}

\pageDuration{0.01}

\begingroup
\loop
 \ifnum\value{i}<76
  \stepcounter{i}%
  \storehead{\partialtext}{\value{i}}
  {%
    Panels\space automatically\space adapt\space to\space the\space
    size\space of\space their\space contents.\par
    They\space can\space be\space placed
  }%
  \DeclarePanel{left}
  {%
    \leavevmode\scriptsize
    \mbox{\usebox{\logobox}}
  
    {\bfseries Stephan Lehmke
      
      Lehr\-stuhl Informatik~I}
    
    \vfill
  
    \nointerlineskip
    \rule{\linewidth}{\fboxrule}
    
    \nointerlineskip\kern1ex
    \partialtext
    
    \nointerlineskip\kern1ex
    \rule{\linewidth}{\fboxrule}
    
    \vfill
  
    \ifthenelse{\lengthtest{\linewidth>2cm}}
    {\setlength{\buttonwidth}{.5\linewidth-.5ex}}
    {\setlength{\buttonwidth}{\linewidth}}%
    %
    \lineskip1ex\relax
    %
    \mybutton{\Acrobatmenu{FirstPage}}{Start}\hfill
    \mybutton{\Acrobatmenu{LastPage}}{End}\hfill
    \mybutton{\Acrobatmenu{PrevPage}}{Prev}\hfill
    \mybutton{\Acrobatmenu{NextPage}}{Next}

    \mbox{\usebox{\hookbox}}
  }%

  \backgroundstyle{plain}

  \mklength{\slideleftmargin}{\leftpanelwidth*\ratio{1cm}{\semcm}+.5cm}
  \setlength{\slidewidth}{\paperwidth-\slideleftmargin-\sliderightmargin}

  \begin{slide}
    \makeslidetitle{\TeX Power Example: panels}
  \end{slide}

\repeat
\endgroup

\DeclarePanel*{left}{}
\setcounter{i}{0}

\setlength{\buttonwidth}{1.5cm}

\DeclarePanel*{bottom}
{%
  \leavevmode\scriptsize\smash{\usebox{\hookbox}}\hfill%
  \mybutton{\Acrobatmenu{FirstPage}}{Start}~%
  \mybutton{\Acrobatmenu{LastPage}}{End}~%
  \mybutton{\Acrobatmenu{PrevPage}}{Prev}~%
  \mybutton{\Acrobatmenu{NextPage}}{Next}%
}

\mklength{\slidebottommargin}{\bottompanelheight*\ratio{1cm}{\semcm}+.5cm}

\begingroup
\loop
 \ifnum\value{i}<24
  \stepcounter{i}%
  \storehead{\partialtext}{\value{i}}
  {%
    on\space any\space side\space of\space the\space page.
  }%
  \DeclarePanel{top}
  {%
    \leavevmode\scriptsize\parskip0pt\relax
    \mbox{\usebox{\logobox}}
    
    \medskip
  
    \parbox[t]{\widthof{\bfseries Lehrstuhl Informatik~I}}
    {\bfseries Stephan Lehmke\\ Lehrstuhl Informatik~I}\quad
    \parbox[t]{\linewidth-\widthof{\bfseries Lehrstuhl Informatik~I}-3cm}
    {%
      Panels\space automatically\space adapt\space to\space the\space
      size\space of\space their\space contents.\par
      They\space can\space be\space placed
      \partialtext
    }
  }%

  \backgroundstyle{plain}

  \mklength{\slidetopmargin}{\toppanelheight*\ratio{1cm}{\semcm}+.5cm}
  \setlength{\slideheight}{\paperheight-\slidetopmargin-\slidebottommargin}

  \begin{slide}
    \small\makeslidetitle{\TeX Power Example: panels}
  \end{slide}

\repeat
\endgroup

\DeclarePanel*{top}{}
\setlength{\slideheight}{\paperheight-\slidetopmargin-\slidebottommargin}

\setcounter{i}{0}

\makeatletter
\savebox\logobox
{%
  \raisebox{0cm}[\height-1.5ex][0pt]
  {\llap{\includegraphics[width=\strip@pt\paperwidth truept-1cm]{fig-3}}}%
}
\makeatother

\mklength{\slidetopmargin}{\ht\logobox*\ratio{1cm}{\semcm}}

\begingroup
\loop
 \ifnum\value{i}<146
  \stepcounter{i}%
  \storehead{\partialtext}{\value{i}}
  {%
    Normally,\space automatic\space line\space breaks\space are\space
    not\space needed\space inside\space panels,\space but\space
    it\space is\space nice\space that\space the\space size\space
    automatically\space adapts\space to\space the\space size\space
    of\space logos,\space buttons\space etc. 
  }%
  \DeclarePanel{right}
  {%
    \leavevmode\scriptsize
    \hspace*{\fill}\mbox{\usebox{\logobox}}\hspace*{-1.5ex}
    
    \medskip
  
    {\bfseries Stephan Lehmke
      
      Lehr\-stuhl Informatik~I}
    
    \vfill
  
    \nointerlineskip
    \rule{\linewidth}{\fboxrule}
    
    \nointerlineskip\kern1ex
    Panels\space automatically\space adapt\space to\space the\space
    size\space of\space their\space contents.\par
    They\space can\space be\space placed on\space any\space side\space
    of\space the\space page. 

    \partialtext
  }%

  \backgroundstyle[hpanels=false]{plain}

  \mklength{\sliderightmargin}{\rightpanelwidth*\ratio{1cm}{\semcm}+.5cm}
  \setlength{\slidewidth}{\paperwidth-\slideleftmargin-\sliderightmargin}

  \begin{slide}
    \makeslidetitle{\TeX Power Example: panels}
  \end{slide}

\repeat
\endgroup

\hypersetup{pdfpageduration={}}


\end{document}
%KEEPCOMMENTS
%</panelexample>
%=================================================================================================================
%<*parexample>
%<<KEEPCOMMENTS
%-----------------------------------------------------------------------------------------------------------------
%
% Paragraph example for the package texpower.sty.
% 
%-----------------------------------------------------------------------------------------------------------------
% Use slifonts. 

\RequirePackage{tpslifonts}

% Input the generic preamble.

\input{__TPpreamble}
\hypersetup{pdftitle={texpower paragraph example}}

\usepackage{soul}


%-----------------------------------------------------------------------------------------------------------------
% Finally, everything is set up. Here we go...
%
\begin{document}
\begin{slide}
%KEEPCOMMENTS
%</parexample>
%<*parexample-src>
%<<KEEPCOMMENTS
%
%-----------------------------------------------------------------------------------------------------------------
%
\makeslidetitle{\macroname{stepwise} Example: Inside A Paragraph}\label{Sec:ExPar}

%
% We show that \stepwise can be used for highlighting words within a paragraph.
% Furthermore, it is demonstrated how the order in which \step's are executed can be changed.
%
% We define a `placeholder' which will mark the place of missing words (instead of \displayphantom).
% 
\newcommand{\placeholder}[1]{\leavevmode\phantom{#1}\llap{\rule{\widthof{\phantom{#1}}}{\fboxrule}}}%
  %
  % We use the custom command \parstepwise which not only wraps the whole argument of \stepwise into a minipage (because
  % otherwise vertical spacing goes haywire, don't ask me why), but also gives substance to steps.
  % 
  % All variants of \stepwise take an optional argument the contents of which are executed inside a group before the
  % inner loop of starts. It can be used to set parameters locally.
  % Here, we redefine \activatestep (which has been explained in the equation example) to highlight the first
  % appearance of any word.
  % \hidestepcontents is used as a `wrapper' for those arguments of \step which should not appear yet. It either
  % displays nothing (this is the default for \stepwise and \liststepwise) or puts its argument into a \phantom
  % (the default for \parstepwise); this behaviour is also toggled by \boxedsteps and \nonboxedsteps.
  % Here, we redefine it to use our selfdefined \placeholder to mark `missing' words.
  % 
  \parstepwise[\let\hidestepcontents=\placeholder\let\activatestep=\highlightboxed]%
  {%
    \begin{quote}
      \Huge We can \step{create} a \step{fill-in-the-blanks}
      %
      % \step takes an optional argument with which it can be specified _when_ its argument is to appear. This is
      % expressed in \ifthenelse syntax (see the documentation of the ifthen package).
      % Here, we refer to the counter step which is advanced by \stepwise and contains the number of the current step.
      % This way, steps can be made to appear in any order.
      \step[\value{step}=5]{text} which is then
      \step[\value{step}=4]{filled} in in
      \step[\value{step}=3]{\textbf{any}} order!
    \end{quote}
    }%

  \newslide

  % The \hidetext command hides its argument while respecting automatic line breaks and such. The command needs the soul
  % package to work. Read the documentation of soul for restrictions as to what can go in the argument of \hidetext. 

  \stepwise[\let\hidestepcontents=\hidetext\let\activatestep=\highlighttext]
  {%
    \vspace*{\fill}
    \begin{minipage}{\linewidth}
      \begin{quote}
        \Huge We can step through a \step{paragraph} of \step{free text}
        \step{without disturbing} line breaks!
      \end{quote}
    \end{minipage}
    \vspace*{\fill}\vspace*{\fill}
    }
  \newslide
%KEEPCOMMENTS
%</parexample-src>
%=================================================================================================================
%<*pdfscrdemo>
%<<KEEPCOMMENTS
%-----------------------------------------------------------------------------------------------------------------
%
% Simple examples the for combining the pdfscreen package with the dynamic features provided by the package
% texpower.sty.   
% 
%-----------------------------------------------------------------------------------------------------------------

\documentclass[12pt]{article}

\usepackage[screen,panelright,sectionbreak]{pdfscreen}

\margins{2ex}{2ex}{2ex}{2ex}
\screensize{6.25in}{8in}

\emblema{fig-2}
\urlid{http://texpower.sourceforge.net/}

%-----------------------------------------------------------------------------------------------------------------
% The texpower package is loaded. 
% We give the display option so dynamic features are enabled.
% We use the tpslifonts package for better readability.
%
\usepackage[display]{texpower}
\usepackage{tpslifonts}


\begin{document}

\title{The \code{texpower} Package\\{\normalfont \texttt{pdfscreen} Demo}}
\author{Stephan Lehmke\\\code{mailto:Stephan.Lehmke@cs.uni-dortmund.de}}
\maketitle

\tableofcontents

\section{A list environment}
  
% The \pause command `splits' the current page at the place it appears, producing two pages, one with everything which
% came before the \pause command, one containing this and additionally the stuff coming after \pause. When these pages
% are presented with acrobar reader in full screen mode (or any other viewer with this capability), the presentation
% will appear to `stop' at the point the \pause command was issued and `resume' in the moment the presenter switches to
% the next page.

\pause

% As \pause forces a paragraph break, it can not be used to separate a description label from the associated text. For
% this, we use the (very flexible) \stepwise command. Inside the argument of \stepwise, an arbitrary number of \step
% commands may occur. \stepwise will produce as many pages as there are \step commands, making the arguments of the
% \step commands appear ``one by one''.

\stepwise
{%
  \begin{description}
  \item[foo.] \step{bar.}
  \step{\item[baz.]} \step{qux.}
  \end{description}
  }



\section{An aligned equation}

\pause

% Normally for \stepwise, if a \step is not yet active, its argument is ignored completely. This would disturb
% alignments, because the width changes with every new activated \step.
% \parstepwise is a variant of \stepwise where the argument of an inactive \step is put into a \phantom, leaving the
% proper amount of white space.

\parstepwise
{%
  % Using eqnarray with equation numbers here means all equation numbers will be visible from the outset, because only
  % the contents of the lines are `filled in'. See the full demo for an example of aligned equations where equation
  % numbers `appear'.
  \begin{eqnarray}
    %
    % When the argument of \step is put into a box (as it happens with \parstepwise), tabulators can not go in there. As
    % we want the equals sign to appear at the same time as the right side of the equation, we use \restep for the
    % latter. \restep is like \step, but it appears at the same time as the previous \step command.
    % 
    \sum_{i=1}^{n} i & \step{=} & \restep{1 + 2 + \cdots + (n-1) + n}\\
    %
                     & \step{=} & \restep{1 + n + 2 + (n-1) + \cdots}\\
    %
                     & \step{=} & \restep
                                  {% We can nest \step commands inside each other. The order of execution is just the
                                   % order of appearance, independent of nesting.
                                   % \switch is a variant of \step which takes two arguments and toggles between them on
                                   % activation. This way, we can make the \underbrace `appear'.
                                   % We insert a \vphantom in the first argument so that the equation numbers will be
                                   % placed correctly whether or not the underbrace is didplayed.
                                    \switch
                                    {%
                                      \vphantom{\underbrace{(1 + n) + \cdots + (1 + n)}_{\times\frac{n}{2}}}%
                                      (1 + n) + \cdots + (1 + n)%
                                      }
                                    {\underbrace{(1 + n) + \cdots + (1 + n)}_{\times\frac{n}{2}}}%
                                    }
                                  \\
    %
    % This is another nested application of \step. Note that the spacing of \cdot has to be corrected manually by
    % inserting {} left of it, because otherwise it would behave like a prefix operator.
    %
                     & \step{=} & \restep{\frac{(1 + n)\step{{}\cdot n}}{\restep{2}}}
  \end{eqnarray}
}




\section{An array}

\stepwise
{% With arrays, beware of problems with automatic calculation of cell widths.
 % 
 % If you want all widths to be calculated automatically, you need to use \parstepwise, with the consequence that
 %   a) tabulators or newlines can not go into the argument of \step,
 %   b) the array `structure' (rules) will be completely visible right from the beginning.
 %   
 % If you want to use \stepwise for being able to build the `structure' (like \hilne's) dynamically (as done in the
 % following), you have to make sure that the cell widths are correct from the very first line, because otherwise the
 % array will expand horizontally, destroying the dynamic effect. This can be assured by
 %   a) using only p cells,
 %   b) making sure all the cells in the first line are at least as wide as the widest cell which will appear later. If
 %      you are using the calc package, this is easiest by putting \makebox[\widthof{widest entry}]{first entry} into
 %      the first cell. Otherwise, you can use \settowidth.
 %      
  \begin{displaymath}
    \begin{array}{rrrrr}
      \step
      {%
            n &        \log n        &        n\log n       & \lefteqn{n^2}\phantom{25} & \lefteqn{2^n}\phantom{32} \\
        \hline%
        }%
      \step{0 &} \step{\textrm{---}  &} \step{\textrm{---}  &} \step{0                  &} \step{1                  \\}%
      \step{1 &} \step{0\phantom{.6} &} \step{0\phantom{.8} &} \step{1                  &} \step{2                  \\}%
      \step{2 &} \step{1\phantom{.6} &} \step{2\phantom{.8} &} \step{4                  &} \step{4                  \\}%
      \step{3 &} \step{1.6           &} \step{4.8           &} \step{9                  &} \step{8                  \\}%
      \step{4 &} \step{2\phantom{.6} &} \step{8\phantom{.8} &} \step{16                 &} \step{16                 \\}%
      \step{5 &} \step{2.3           &} \step{11.6          &} \step{25                 &} \step{32                   }%
    \end{array}
  \end{displaymath}
}




\section{A picture}

\pause

\begin{center}%
  \stepwise
  {%
    \setlength{\unitlength}{1.1cm}%
    \delimitershortfall-1sp% Just for the nested braces
    \begin{picture}(14,2)
      \put(0,1){\vector(1,0){1}}
      \put(0.5,0.5){\makebox(0,0){\small $x(t)$}}
      \put(13,1){\vector(1,0){1}}
      \put(13.5,0.5){\makebox(0,0){\small $y(t)$}}
      \step
      {
        \put(1,1){\line(3,2){1.5}}
        \put(1,1){\line(3,-2){1.5}}
        \put(2.5,0){\line(0,1){2}}
        \put(2,1){\makebox(0,0){\large $\varphi$}}
        }
      \step
      {
        \put(2.5,1){\vector(1,0){3.5}}
        \put(4.25,0.5){\makebox(0,0){\small $F_t = \varphi\left(x(t)\right)$}}
        }
      \step
      {
        \put(6,0){\framebox(2,2){\large $\Phi$}}
        }
      \step
      {
        \put(8,1){\vector(1,0){3.5}}
        %
        % Here, we find another nested use of \step inside \step.
        % \bstep is a variant of \step which _always_ puts its argument into a box for leaving the correct amount of
        % white space. We cannot use \parstepwise here because \put can't go into a box. Hence, just using \step for
        % building the nested formula on the next line would give the wrong size for the nested braces.
        % 
        \put(9.75,0.5){\makebox(0,0){\small $G_t = \Phi\left(\bstep{\varphi\left(\bstep{x(t)}\right)}\right)$}}
        }
      \step
      {
        \put(13,1){\line(-3,2){1.5}}
        \put(13,1){\line(-3,-2){1.5}}
        \put(11.5,0){\line(0,1){2}}
        \put(12,1){\makebox(0,0){\large $\delta$}}
        }
    \end{picture}%
    }%
\end{center}%
\end{document}
%KEEPCOMMENTS
%</pdfscrdemo>
%=================================================================================================================
%<*pdfslidemo>
%<<KEEPCOMMENTS
%-----------------------------------------------------------------------------------------------------------------
%
% Simple examples the for combining the pdfslide package with the dynamic features provided by the package texpower.sty. 
% 
%-----------------------------------------------------------------------------------------------------------------

\documentclass{article}

\usepackage{pdfslide}

\hypersetup{linkcolor=red}

\overlay{bg.jpg}

\setcounter{secnumdepth}{2}

%-----------------------------------------------------------------------------------------------------------------
% The texpower package is loaded. 
% We give the display option so dynamic features are enabled.
% The overlay forces a black background, so we select the blackbackground option of texpower.
%
\usepackage[display,blackbackground]{texpower}


\begin{document}

\title{The \code{texpower} Package\\{\normalfont \texttt{pdfslide} Demo}}
\author{Stephan Lehmke\\\code{mailto:Stephan.Lehmke@cs.uni-dortmund.de}}
\maketitle

\tableofcontents

\section{A list environment}
  
% The \pause command `splits' the current page at the place it appears, producing two pages, one with everything which
% came before the \pause command, one containing this and additionally the stuff coming after \pause. When these pages
% are presented with acrobar reader in full screen mode (or any other viewer with this capability), the presentation
% will appear to `stop' at the point the \pause command was issued and `resume' in the moment the presenter switches to
% the next page.

\pause

% As \pause forces a paragraph break, it can not be used to separate a description label from the associated text. For
% this, we use the (very flexible) \stepwise command. Inside the argument of \stepwise, an arbitrary number of \step
% commands may occur. \stepwise will produce as many pages as there are \step commands, making the arguments of the
% \step commands appear ``one by one''.

\stepwise
{%
  \begin{description}
  \item[foo.] \step{bar.}
  \step{\item[baz.]} \step{qux.}
  \end{description}
  }



\section{An aligned equation}

\pause

% Normally for \stepwise, if a \step is not yet active, its argument is ignored completely. This would disturb
% alignments, because the width changes with every new activated \step.
% \parstepwise is a variant of \stepwise where the argument of an inactive \step is put into a \phantom, leaving the
% proper amount of white space.

\parstepwise
{%
  % Using eqnarray with equation numbers here means all equation numbers will be visible from the outset, because only
  % the contents of the lines are `filled in'. See the full demo for an example of aligned equations where equation
  % numbers `appear'.
  \begin{eqnarray}
    %
    % When the argument of \step is put into a box (as it happens with \parstepwise), tabulators can not go in there. As
    % we want the equals sign to appear at the same time as the right side of the equation, we use \restep for the
    % latter. \restep is like \step, but it appears at the same time as the previous \step command.
    % 
    \sum_{i=1}^{n} i & \step{=} & \restep{1 + 2 + \cdots + (n-1) + n}\\
    %
                     & \step{=} & \restep{1 + n + 2 + (n-1) + \cdots}\\
    %
                     & \step{=} & \restep
                                  {% We can nest \step commands inside each other. The order of execution is just the
                                   % order of appearance, independent of nesting.
                                   % \switch is a variant of \step which takes two arguments and toggles between them on
                                   % activation. This way, we can make the \underbrace `appear'.
                                   % We insert a \vphantom in the first argument so that the equation numbers will be
                                   % placed correctly whether or not the underbrace is didplayed.
                                    \switch
                                    {%
                                      \vphantom{\underbrace{(1 + n) + \cdots + (1 + n)}_{\times\frac{n}{2}}}%
                                      (1 + n) + \cdots + (1 + n)%
                                      }
                                    {\underbrace{(1 + n) + \cdots + (1 + n)}_{\times\frac{n}{2}}}%
                                    }
                                  \\
    %
    % This is another nested application of \step. Note that the spacing of \cdot has to be corrected manually by
    % inserting {} left of it, because otherwise it would behave like a prefix operator.
    %
                     & \step{=} & \restep{\frac{(1 + n)\step{{}\cdot n}}{\restep{2}}}
  \end{eqnarray}
}




\section{An array}

\stepwise
{% With arrays, beware of problems with automatic calculation of cell widths.
 % 
 % If you want all widths to be calculated automatically, you need to use \parstepwise, with the consequence that
 %   a) tabulators or newlines can not go into the argument of \step,
 %   b) the array `structure' (rules) will be completely visible right from the beginning.
 %   
 % If you want to use \stepwise for being able to build the `structure' (like \hilne's) dynamically (as done in the
 % following), you have to make sure that the cell widths are correct from the very first line, because otherwise the
 % array will expand horizontally, destroying the dynamic effect. This can be assured by
 %   a) using only p cells,
 %   b) making sure all the cells in the first line are at least as wide as the widest cell which will appear later. If
 %      you are using the calc package, this is easiest by putting \makebox[\widthof{widest entry}]{first entry} into
 %      the first cell. Otherwise, you can use \settowidth.
 %      
  \begin{displaymath}
    \begin{array}{rrrrr}
      \step
      {%
            n &        \log n        &        n\log n       & \lefteqn{n^2}\phantom{25} & \lefteqn{2^n}\phantom{32} \\
        \hline%
        }%
      \step{0 &} \step{\textrm{---}  &} \step{\textrm{---}  &} \step{0                  &} \step{1                  \\}%
      \step{1 &} \step{0\phantom{.6} &} \step{0\phantom{.8} &} \step{1                  &} \step{2                  \\}%
      \step{2 &} \step{1\phantom{.6} &} \step{2\phantom{.8} &} \step{4                  &} \step{4                  \\}%
      \step{3 &} \step{1.6           &} \step{4.8           &} \step{9                  &} \step{8                  \\}%
      \step{4 &} \step{2\phantom{.6} &} \step{8\phantom{.8} &} \step{16                 &} \step{16                 \\}%
      \step{5 &} \step{2.3           &} \step{11.6          &} \step{25                 &} \step{32                   }%
    \end{array}
  \end{displaymath}
}




\section{A picture}

\pause

\begin{center}%
  \stepwise
  {%
    \setlength{\unitlength}{1.2cm}%
    \delimitershortfall-1sp% Just for the nested braces
    \begin{picture}(14,2)
      \put(0,1){\vector(1,0){1}}
      \put(0.5,0.5){\makebox(0,0){\small $x(t)$}}
      \put(13,1){\vector(1,0){1}}
      \put(13.5,0.5){\makebox(0,0){\small $y(t)$}}
      \step
      {
        \put(1,1){\line(3,2){1.5}}
        \put(1,1){\line(3,-2){1.5}}
        \put(2.5,0){\line(0,1){2}}
        \put(2,1){\makebox(0,0){\large $\varphi$}}
        }
      \step
      {
        \put(2.5,1){\vector(1,0){3.5}}
        \put(4.25,0.5){\makebox(0,0){\small $F_t = \varphi\left(x(t)\right)$}}
        }
      \step
      {
        \put(6,0){\framebox(2,2){\large $\Phi$}}
        }
      \step
      {
        \put(8,1){\vector(1,0){3.5}}
        %
        % Here, we find another nested use of \step inside \step.
        % \bstep is a variant of \step which _always_ puts its argument into a box for leaving the correct amount of
        % white space. We cannot use \parstepwise here because \put can't go into a box. Hence, just using \step for
        % building the nested formula on the next line would give the wrong size for the nested braces.
        % 
        \put(9.75,0.5){\makebox(0,0){\small $G_t = \Phi\left(\bstep{\varphi\left(\bstep{x(t)}\right)}\right)$}}
        }
      \step
      {
        \put(13,1){\line(-3,2){1.5}}
        \put(13,1){\line(-3,-2){1.5}}
        \put(11.5,0){\line(0,1){2}}
        \put(12,1){\makebox(0,0){\large $\delta$}}
        }
    \end{picture}%
    }%
\end{center}%
\end{document}
%KEEPCOMMENTS
%</pdfslidemo>
%=================================================================================================================
%<*picexample>
%<<KEEPCOMMENTS
%-----------------------------------------------------------------------------------------------------------------
%
% Picture example for the package texpower.sty.
% 
%-----------------------------------------------------------------------------------------------------------------
% Use slifonts and a dark background. 

\PassOptionsToPackage{colormath,colorhighlight,darkbackground}{texpower}
\RequirePackage{tpslifonts}

% Input the generic preamble.

\input{__TPpreamble}
\hypersetup{pdftitle={texpower picture example}}


\ifthenelse{\boolean{psspecialsallowed}}% Can we use PSTricks?
{% Yes.
  % PsTricks (sic) is used for creating the picture example.

  \usepackage[noxcolor]{pstricks}
  \usepackage{pstcol}
  \usepackage{pst-node}
  
  \psset{unit=\unitlength}
  }
{% No. We'll make do without.
  }


%-----------------------------------------------------------------------------------------------------------------
% Finally, everything is set up. Here we go...
%
\begin{document}
\begin{slide}
%KEEPCOMMENTS
%</picexample>
%<*picexample-src>
%<<KEEPCOMMENTS

%-----------------------------------------------------------------------------------------------------------------
%
\makeslidetitle{\macroname{stepwise} Example: A Picture}\label{Sec:ExPic}

\ifthenelse{\boolean{psspecialsallowed}}% Can we use PSTricks?
{\input{picpsexample}}{\input{picltxexample}}
\newslide
\pageTransitionReplace
%KEEPCOMMENTS
%</picexample-src>
%=================================================================================================================
%<*picltxexample>
%<<KEEPCOMMENTS
%
% Code for the LaTeX picture example for the package texpower.sty.
% 
% This file is input by others. Don't compile it separately.
%
%-----------------------------------------------------------------------------------------------------------------

  %
  % This has nothing to do with \stepwise, just setting up the picture...
  %
  \newcommand{\leftm}{\left\{\!\!\left\{}
  \newcommand{\rightm}{\right\}\!\!\right\}}
  \newcommand{\cls}[3]{\left[#1,#2,#3\right]}
  \newcommand{\WT}[2]{\fbox{${#1}#2$}}
  \newcommand{\GEQ}[1]{\WT{\geq}{#1}}

  %
  % In the following picture, picture items are built incrementally.
  %
  % \parstepwise generates a sequence of slides, all alike. The only difference ist that on every slide, one more of the
  % \step commands occurring in the argument of \stepwise are `activated'. This way the stuff inside the argument of
  % \parstepwise is gone through `step by step'. (\parstepwise is a special case of \stepwise.)
  %
  {%
    \setlength{\unitlength}{1.35\semcm}%
    \footnotesize%
    \setlength{\fboxsep}{1.5pt}%
    \parstepwise
    {%
      \begin{center}
        \begin{picture}(12,13)(-7,-16)
          \put(-3,-3.5){\makebox(0,0){$\cls{\leftm p_1, p_2\rightm}{1}{\GEQ{0.2}}$}}
          \put(3,-3.5){\makebox(0,0){$\cls{\leftm \neg p_2,p_1\rightm}{1}{\GEQ{0.1}}$}}
          %
          % In the following, we see the very first occurrence of the \step command. Its effect is to `hide' its
          % argument on the first slide of the sequence generated by \stepwise and display it on all other slides of
          % this sequence. The second \step command starts displaying on the third slide and so on...
          %
          \step{%
          % We use \afterstep{\pageTransitionDissolve} to make this first step of the diagram appear with a fancy page
          % transition. We have to use \afterstep because the page transition setting would be undone by the group
          % closing contained in \end{picture}.
            \afterstep{\pageTransitionDissolve}%
            \put(-3,-4){\line(3,-1){3}}
            \put(3,-4){\line(-3,-1){3}}
            \put(0,-4.5){\makebox(0,0){(ass.)}}
            \put(0,-5.5){%
              \makebox(0,0){%
              %
              % Here, one use of \step is nested inside the other. 
              %
              % Usually `hiding' means ignoring altogehter. For the following application however, the `hidden' objects
              % are nested inside a brace. This means that instead of ignoring its argument, it's better if
              % \step displays an appropriate amount of blank space. This behaviour (create blank space) is exhibited by
              % the custom command \bstep (for `boxed' step).
              %
              % The command \rebstep allows both boxes to appear simultaneously.
              %
              % \afterstep is used again to reset the page transition to Replace for the building of the formula.
              % 
              % Setting \activatestep to \highlightboxed makes the additional formulae `stick out' when they are filled
              % in. 
              % 
                \let\activatestep=\highlightboxed
                $
                \cls
                {\leftm p_1, \bstep{\afterstep{\pageTransitionReplace}p_1}, p_2,\rebstep{\neg p_2}\rightm}
                {2.2}
                {\rebstep{\GEQ{0.1}}}%
                $%
                }%
              }
            }
          \step{%
            \afterstep{\pageTransitionDissolve}%
            \put(0,-6){\line(0,-1){1}}
            \put(0.2,-6.5){\makebox(0,0)[l]{(removing)}}
            \put(0,-7.5){\makebox(0,0){$\cls{\leftm p_1, p_1\rightm}{1.2}{\GEQ{0.1}}$}}
            }
          \step{%
            \put(-8,-7.5){\makebox(0,0){$\left[\neg p_1,\GEQ{0.4}\right]$}}
            \put(-8,-8){\line(5,-1){5}}
            \put(0,-8){\line(-3,-1){3}}
            \put(-3,-8.5){\makebox(0,0){(ass.)}}
            \put(-3,-9.5){\makebox(0,0){$\cls{\leftm p_1, p_1,\neg p_1\rightm}{1.6}{\GEQ{0.1}}$}}
            \put(-3,-10){\line(0,-1){1}}
            \put(-2.8,-10.5){\makebox(0,0)[l]{(removing)}}
            \put(-3,-11.5){\makebox(0,0){$\cls{\leftm p_1\rightm}{0.6}{\GEQ{0.1}}$}}
            }
          \step{%
            \put(-8,-8){\line(0,-1){4}}
            \put(-8,-12){\line(5,-1){5}}
            \put(-3,-12){\line(0,-1){1}}
            \put(-2.8,-12.5){\makebox(0,0)[l]{(assembling)}}
            \put(-3,-13.5){\makebox(0,0){$\cls{\leftm p_1,\neg p_1\rightm}{1.0}{\GEQ{0.1}}$}}
            \put(-3,-14){\line(0,-1){1}}
            \put(-2.8,-14.5){\makebox(0,0)[l]{(removing)}}
            \put(-3,-15.5){\makebox(0,0){$\left[\fbox{\phantom{x}},\GEQ{0.1}\right]$}}
            }
        \end{picture}%
      \end{center}
      }%
    }

  % The whole execution of \stepwise is encapsuled in a group to make all changes local. This affects also the setting
  % of the page transition to dissolve in the last but one step. We set it again explicitly to make the last step appear
  % with this page transition as well.

  \pageTransitionDissolve
%KEEPCOMMENTS
%</picltxexample>
%=================================================================================================================
%<*picpsexample>
%<<KEEPCOMMENTS
%
% Code for the PSTricks picture example for the package texpower.sty.
% 
% This file is input by others. Don't compile it separately.
% 
%-----------------------------------------------------------------------------------------------------------------
%
% This has nothing to do with \stepwise, just setting up the picture...
%
\newcommand{\Block}[1]
{%
  \begin{pspicture}(-4,-2)(4,2)
    \pspolygon(-4,0)(-2,2)(2,2)(4,0)(2,-2)(-2,-2)
    \rput(0,0){#1}
  \end{pspicture}%
  }%
\ifthenelse{\boolean{TPcolor}}
{\psset{unit=3mm,fillstyle=solid,fillcolor=highlightcolor,linecolor=textcolor}}%
{\psset{unit=3mm}}%
%
% In the following picture, picture items are built incrementally.
%
% \parstepwise generates a sequence of slides, all alike. The only difference ist that on every slide, one more of the
% \step commands occurring in the argument of \stepwise are `activated'. This way the stuff inside the argument of
% \parstepwise is gone through `step by step'. (\parstepwise is a special case of \stepwise.)
%
\parstepwise
{%
  \begin{center}
    \large
    \begin{pspicture}(-1,-13)(33,9)
      \rput(0,0){\rnode{x}{\fboxrule0pt\fbox{$x[t]$}}}
      \rput(32,0){\rnode{y}{\fboxrule0pt\fbox{$y(t)$}}}
      {%
        % The effect of the \step command is to `hide' its argument on the first slide of the sequence generated by
        % \stepwise and display it on all other slides of this sequence. The second \step command starts displaying on
        % the third slide and so on...   
        %
        % Usually `hiding' means ignoring altogehter. For the following application however, the box displayed by
        % \step is used as an node in the picture. This means that instead of ignoring its argument, it's better if
        % \step displays an appropriate amount of blank space.
        % This behaviour (create blank space) is exhibited by the command \bstep.
        \rput(4,0){\rnode{V}{\bstep{\psframebox{\Large$V$}}}}
        \ncline{->}{x}{V}
        %
        % The command \rebstep allows both boxes to appear simultaneously.
        %
        \rput(28,0){\rnode{plus}{\rebstep{\psframebox{\Large$+$}}}}
        }%
      \ncline{->}{plus}{y}
      % By using \restep again, the two boxes defining the operators appear at the same time as the first block of the
      % diagram, which is produced by the following commands.
      \restep
      {%
        % We use \afterstep{\pageTransitionDissolve} to make this first step of the diagram appear with a fancy page
        % transition. We have to use \afterstep because the page transition setting would be undone by the group
        % closing contained in \end{pspicture}.
        \afterstep{\pageTransitionDissolve}%
        \rput(16,0){\rnode{IBlk}{\Block{I-Block}}}
        \ncline{->}{V}{IBlk}
        \Aput{$x[t]$}
        \ncline{->}{IBlk}{plus}
        \Aput
        {%
          %
          % Here, one use of \step is nested inside the other. 
          $%
          % Note how the math spacing is corrected manually by adding {} after \cdot. Otherwise, \cdot wouldn't be
          % aware that something is following and act as a postfix (instead of infix) operator.
          % \afterstep is used again to reset the page transition to Replace for the building of the formula.
          \bstep{\afterstep{\pageTransitionReplace}{}b\cdot{}}%
          \bstep{\int\limits^t_0 x(\tau)\,d\tau}%
          $%
          }%
        }%
      \step
      {%
        \afterstep{\pageTransitionDissolve}%
        \rput(16,6){\rnode{PBlk}{\Block{P-Block}}}
        \ncangle[angleA=90,angleB=180,fillstyle=none]{->}{V}{PBlk}
        \Aput{$x(t)$}
        \ncangle[angleB=90,fillstyle=none]{->}{PBlk}{plus}
        \aput(.5){$a\cdot x(t)$}
        }%
      \step
      {%
        \rput(16,-6){\rnode{DBlk}{\Block{D-Block}}}
        \ncangle[angleA=-90,angleB=180,fillstyle=none]{->}{V}{DBlk}
        \Aput{$x(t)$}
        \ncangle[angleB=-90,fillstyle=none]{->}{DBlk}{plus}
        \aput(.5){$\displaystyle c\cdot \left(\frac{dx}{d\tau}\right)(t)$}
        }
    \end{pspicture}%
  \end{center}
  }%

% The whole execution of \stepwise is encapsuled in a group to make all changes local. This affects also the setting
% of the page transition to dissolve in the last but one step. We set it again explicitly to make the last step appear
% with this page transition as well.

\pageTransitionDissolve

%KEEPCOMMENTS
%</picpsexample>
%<*pp4sldemo>
%<<KEEPCOMMENTS
%-----------------------------------------------------------------------------------------------------------------
%
% Simple examples the for combining the foils class with the pp4slide package and the dynamic features provided by the
% package texpower.sty.  
% 
%-----------------------------------------------------------------------------------------------------------------

\documentclass[landscape]{foils}

\usepackage{pp4slide}

%-----------------------------------------------------------------------------------------------------------------
% The texpower package is loaded. 
% We give the display option so dynamic features are enabled.
% As pp4slide sets some colors designed for dark backgrounds, we give the darkbackground option, and colormath to make
% formulae colored.
%
\usepackage[display,darkbackground,colormath]{texpower}


\begin{document}

\title{The \code{texpower} Package\\{\normalfont \texttt{pp4slide} Demo}}
\author{Stephan Lehmke\\\code{mailto:Stephan.Lehmke@cs.uni-dortmund.de}}
\maketitle


\foilhead{A list environment}
  
% The \pause command `splits' the current page at the place it appears, producing two pages, one with everything which
% came before the \pause command, one containing this and additionally the stuff coming after \pause. When these pages
% are presented with acrobar reader in full screen mode (or any other viewer with this capability), the presentation
% will appear to `stop' at the point the \pause command was issued and `resume' in the moment the presenter switches to
% the next page.

\pause

% As \pause forces a paragraph break, it can not be used to separate a description label from the associated text. For
% this, we use the (very flexible) \stepwise command. Inside the argument of \stepwise, an arbitrary number of \step
% commands may occur. \stepwise will produce as many pages as there are \step commands, making the arguments of the
% \step commands appear ``one by one''.

\stepwise
{%
  \begin{description}
  \item[foo.] \step{bar.}
  \step{\item[baz.]} \step{qux.}
  \end{description}
  }



\foilhead{An aligned equation}

\pause

% Normally for \stepwise, if a \step is not yet active, its argument is ignored completely. This would disturb
% alignments, because the width changes with every new activated \step.
% \parstepwise is a variant of \stepwise where the argument of an inactive \step is put into a \phantom, leaving the
% proper amount of white space.

\parstepwise
{%
  % Using eqnarray with equation numbers here means all equation numbers will be visible from the outset, because only
  % the contents of the lines are `filled in'. See the full demo for an example of aligned equations where equation
  % numbers `appear'.
  \begin{eqnarray}
    %
    % When the argument of \step is put into a box (as it happens with \parstepwise), tabulators can not go in there. As
    % we want the equals sign to appear at the same time as the right side of the equation, we use \restep for the
    % latter. \restep is like \step, but it appears at the same time as the previous \step command.
    % 
    \sum_{i=1}^{n} i & \step{=} & \restep{1 + 2 + \cdots + (n-1) + n}\\
    %
                     & \step{=} & \restep{1 + n + 2 + (n-1) + \cdots}\\
    %
                     & \step{=} & \restep
                                  {% We can nest \step commands inside each other. The order of execution is just the
                                   % order of appearance, independent of nesting.
                                   % \switch is a variant of \step which takes two arguments and toggles between them on
                                   % activation. This way, we can make the \underbrace `appear'.
                                   % We insert a \vphantom in the first argument so that the equation numbers will be
                                   % placed correctly whether or not the underbrace is didplayed.
                                    \switch
                                    {%
                                      \vphantom{\underbrace{(1 + n) + \cdots + (1 + n)}_{\times\frac{n}{2}}}%
                                      (1 + n) + \cdots + (1 + n)%
                                      }
                                    {\underbrace{(1 + n) + \cdots + (1 + n)}_{\times\frac{n}{2}}}%
                                    }
                                  \\
    %
    % This is another nested application of \step. Note that the spacing of \cdot has to be corrected manually by
    % inserting {} left of it, because otherwise it would behave like a prefix operator.
    %
                     & \step{=} & \restep{\frac{(1 + n)\step{{}\cdot n}}{\restep{2}}}
  \end{eqnarray}
}




\foilhead{An array}

\stepwise
{% With arrays, beware of problems with automatic calculation of cell widths.
 % 
 % If you want all widths to be calculated automatically, you need to use \parstepwise, with the consequence that
 %   a) tabulators or newlines can not go into the argument of \step,
 %   b) the array `structure' (rules) will be completely visible right from the beginning.
 %   
 % If you want to use \stepwise for being able to build the `structure' (like \hilne's) dynamically (as done in the
 % following), you have to make sure that the cell widths are correct from the very first line, because otherwise the
 % array will expand horizontally, destroying the dynamic effect. This can be assured by
 %   a) using only p cells,
 %   b) making sure all the cells in the first line are at least as wide as the widest cell which will appear later. If
 %      you are using the calc package, this is easiest by putting \makebox[\widthof{widest entry}]{first entry} into
 %      the first cell. Otherwise, you can use \settowidth.
 %      
  \begin{displaymath}
    \begin{array}{rrrrr}
      \step
      {%
            n &        \log n        &        n\log n       & \lefteqn{n^2}\phantom{25} & \lefteqn{2^n}\phantom{32} \\
        \hline%
        }%
      \step{0 &} \step{\textrm{---}  &} \step{\textrm{---}  &} \step{0                  &} \step{1                  \\}%
      \step{1 &} \step{0\phantom{.6} &} \step{0\phantom{.8} &} \step{1                  &} \step{2                  \\}%
      \step{2 &} \step{1\phantom{.6} &} \step{2\phantom{.8} &} \step{4                  &} \step{4                  \\}%
      \step{3 &} \step{1.6           &} \step{4.8           &} \step{9                  &} \step{8                  \\}%
      \step{4 &} \step{2\phantom{.6} &} \step{8\phantom{.8} &} \step{16                 &} \step{16                 \\}%
      \step{5 &} \step{2.3           &} \step{11.6          &} \step{25                 &} \step{32                   }%
    \end{array}
  \end{displaymath}
}




\foilhead{A picture}

\pause

\begin{center}%
  \stepwise
  {%
    \setlength{\unitlength}{1.6cm}%
    \delimitershortfall-1sp% Just for the nested braces
    \begin{picture}(14,2)
      \put(0,1){\vector(1,0){1}}
      \put(0.5,0.5){\makebox(0,0){\small $x(t)$}}
      \put(13,1){\vector(1,0){1}}
      \put(13.5,0.5){\makebox(0,0){\small $y(t)$}}
      \step
      {
        \put(1,1){\line(3,2){1.5}}
        \put(1,1){\line(3,-2){1.5}}
        \put(2.5,0){\line(0,1){2}}
        \put(2,1){\makebox(0,0){\large $\varphi$}}
        }
      \step
      {
        \put(2.5,1){\vector(1,0){3.5}}
        \put(4.25,0.5){\makebox(0,0){\small $F_t = \varphi\left(x(t)\right)$}}
        }
      \step
      {
        \put(6,0){\framebox(2,2){\large $\Phi$}}
        }
      \step
      {
        \put(8,1){\vector(1,0){3.5}}
        %
        % Here, we find another nested use of \step inside \step.
        % \bstep is a variant of \step which _always_ puts its argument into a box for leaving the correct amount of
        % white space. We cannot use \parstepwise here because \put can't go into a box. Hence, just using \step for
        % building the nested formula on the next line would give the wrong size for the nested braces.
        % 
        \put(9.75,0.5){\makebox(0,0){\small $G_t = \Phi\left(\bstep{\varphi\left(\bstep{x(t)}\right)}\right)$}}
        }
      \step
      {
        \put(13,1){\line(-3,2){1.5}}
        \put(13,1){\line(-3,-2){1.5}}
        \put(11.5,0){\line(0,1){2}}
        \put(12,1){\makebox(0,0){\large $\delta$}}
        }
    \end{picture}%
    }%
\end{center}%
\end{document}
%KEEPCOMMENTS
%</pp4sldemo>
%=================================================================================================================
%<*prosperdemo>
%<<KEEPCOMMENTS
%  
\documentclass[pdf,colorBG,slideColor,whitecross]{prosper}
\usepackage[display,oldfiltering]{texpower}
\begin{document}
\ifpdf
  No way to compile a prosper document with pdftex!
 \else
\stepwise[\boxedsteps]
{%
\begin{slide}{An aligned equation}
  \begin{eqnarray}
    \step{\sum_{i=1}^{n} i} & \step{=} & \restep{1 + 2 + \cdots + (n-1) + n}\\
    %
                     & \step{=} & \restep{1 + n + 2 + (n-1) + \cdots}\\
    %
                     & \step{=} & \restep
                                  {% 
                                    \switch
                                    {%
                                      \vphantom{\underbrace{(1 + n) + \cdots + (1 + n)}_{\times\frac{n}{2}}}%
                                      (1 + n) + \cdots + (1 + n)%
                                      }
                                    {\underbrace{(1 + n) + \cdots + (1 + n)}_{\times\frac{n}{2}}}%
                                    }
                                  \\
    %
                     & \step{=} & \restep{\frac{(1 + n)\step{{}\cdot n}}{\restep{2}}}
  \end{eqnarray}
\end{slide}
}
\fi


\end{document}
%KEEPCOMMENTS
%</prosperdemo>
%=================================================================================================================
%<*seminardemo>
%<<KEEPCOMMENTS
%-----------------------------------------------------------------------------------------------------------------
%
% Simple examples the for combining the seminar class with the dynamic features provided by the package texpower.sty. 
% 
%-----------------------------------------------------------------------------------------------------------------

\documentclass[portrait,semrot]{seminar}

% We need fixseminar for setting the page size correctly.

\usepackage{fixseminar}


%-----------------------------------------------------------------------------------------------------------------
% The texpower package is loaded. 
% We give the display option so dynamic features are enabled.
%
\usepackage[display]{texpower}

\rotateheaderstrue

\begin{document}
\begin{slide}

\title{The \code{texpower} Package\\{\normalfont \texttt{seminar} Demo}}
\author{Stephan Lehmke\\\code{mailto:Stephan.Lehmke@cs.uni-dortmund.de}}
\maketitle

\newslide

\tableofcontents
\end{slide}

\begin{slide}
\centerslidesfalse
\section{A list environment}
  
% The \pause command `splits' the current page at the place it appears, producing two pages, one with everything which
% came before the \pause command, one containing this and additionally the stuff coming after \pause. When these pages
% are presented with acrobar reader in full screen mode (or any other viewer with this capability), the presentation
% will appear to `stop' at the point the \pause command was issued and `resume' in the moment the presenter switches to
% the next page.

\pause

% As \pause forces a paragraph break, it can not be used to separate a description label from the associated text. For
% this, we use the (very flexible) \stepwise command. Inside the argument of \stepwise, an arbitrary number of \step
% commands may occur. \stepwise will produce as many pages as there are \step commands, making the arguments of the
% \step commands appear ``one by one''.

\stepwise
{%
  \begin{description}
  \item[foo.] \step{bar.}
  \step{\item[baz.]} \step{qux.}
  \end{description}
  }



\end{slide}

\begin{slide}
\centerslidesfalse
\section{An aligned equation}

\pause

% Normally for \stepwise, if a \step is not yet active, its argument is ignored completely. This would disturb
% alignments, because the width changes with every new activated \step.
% \parstepwise is a variant of \stepwise where the argument of an inactive \step is put into a \phantom, leaving the
% proper amount of white space.

\parstepwise
{%
  % Using eqnarray with equation numbers here means all equation numbers will be visible from the outset, because only
  % the contents of the lines are `filled in'. See the full demo for an example of aligned equations where equation
  % numbers `appear'.
  \begin{eqnarray}
    %
    % When the argument of \step is put into a box (as it happens with \parstepwise), tabulators can not go in there. As
    % we want the equals sign to appear at the same time as the right side of the equation, we use \restep for the
    % latter. \restep is like \step, but it appears at the same time as the previous \step command.
    % 
    \sum_{i=1}^{n} i & \step{=} & \restep{1 + 2 + \cdots + (n-1) + n}\\
    %
                     & \step{=} & \restep{1 + n + 2 + (n-1) + \cdots}\\
    %
                     & \step{=} & \restep
                                  {% We can nest \step commands inside each other. The order of execution is just the
                                   % order of appearance, independent of nesting.
                                   % \switch is a variant of \step which takes two arguments and toggles between them on
                                   % activation. This way, we can make the \underbrace `appear'.
                                   % We insert a \vphantom in the first argument so that the equation numbers will be
                                   % placed correctly whether or not the underbrace is didplayed.
                                    \switch
                                    {%
                                      \vphantom{\underbrace{(1 + n) + \cdots + (1 + n)}_{\times\frac{n}{2}}}%
                                      (1 + n) + \cdots + (1 + n)%
                                      }
                                    {\underbrace{(1 + n) + \cdots + (1 + n)}_{\times\frac{n}{2}}}%
                                    }
                                  \\
    %
    % This is another nested application of \step. Note that the spacing of \cdot has to be corrected manually by
    % inserting {} left of it, because otherwise it would behave like a prefix operator.
    %
                     & \step{=} & \restep{\frac{(1 + n)\step{{}\cdot n}}{\restep{2}}}
  \end{eqnarray}
}




\end{slide}

\begin{slide}
\centerslidesfalse
\section{An array}

\stepwise
{% With arrays, beware of problems with automatic calculation of cell widths.
 % 
 % If you want all widths to be calculated automatically, you need to use \parstepwise, with the consequence that
 %   a) tabulators or newlines can not go into the argument of \step,
 %   b) the array `structure' (rules) will be completely visible right from the beginning.
 %   
 % If you want to use \stepwise for being able to build the `structure' (like \hilne's) dynamically (as done in the
 % following), you have to make sure that the cell widths are correct from the very first line, because otherwise the
 % array will expand horizontally, destroying the dynamic effect. This can be assured by
 %   a) using only p cells,
 %   b) making sure all the cells in the first line are at least as wide as the widest cell which will appear later. If
 %      you are using the calc package, this is easiest by putting \makebox[\widthof{widest entry}]{first entry} into
 %      the first cell. Otherwise, you can use \settowidth.
 %      
  \begin{displaymath}
    \begin{array}{rrrrr}
      \step
      {%
            n &        \log n        &        n\log n       & \lefteqn{n^2}\phantom{25} & \lefteqn{2^n}\phantom{32} \\
        \hline%
        }%
      \step{0 &} \step{\textrm{---}  &} \step{\textrm{---}  &} \step{0                  &} \step{1                  \\}%
      \step{1 &} \step{0\phantom{.6} &} \step{0\phantom{.8} &} \step{1                  &} \step{2                  \\}%
      \step{2 &} \step{1\phantom{.6} &} \step{2\phantom{.8} &} \step{4                  &} \step{4                  \\}%
      \step{3 &} \step{1.6           &} \step{4.8           &} \step{9                  &} \step{8                  \\}%
      \step{4 &} \step{2\phantom{.6} &} \step{8\phantom{.8} &} \step{16                 &} \step{16                 \\}%
      \step{5 &} \step{2.3           &} \step{11.6          &} \step{25                 &} \step{32                   }%
    \end{array}
  \end{displaymath}
}




\end{slide}

\begin{slide}
\centerslidesfalse
\section{A picture}

\pause

\begin{center}%
  \stepwise
  {%
    \setlength{\unitlength}{1.5\semcm}%
    \delimitershortfall-1sp% Just for the nested braces
    \begin{picture}(14,2)
      \put(0,1){\vector(1,0){1}}
      \put(0.5,0.5){\makebox(0,0){\small $x(t)$}}
      \put(13,1){\vector(1,0){1}}
      \put(13.5,0.5){\makebox(0,0){\small $y(t)$}}
      \step
      {
        \put(1,1){\line(3,2){1.5}}
        \put(1,1){\line(3,-2){1.5}}
        \put(2.5,0){\line(0,1){2}}
        \put(2,1){\makebox(0,0){\large $\varphi$}}
        }
      \step
      {
        \put(2.5,1){\vector(1,0){3.5}}
        \put(4.25,0.5){\makebox(0,0){\small $F_t = \varphi\left(x(t)\right)$}}
        }
      \step
      {
        \put(6,0){\framebox(2,2){\large $\Phi$}}
        }
      \step
      {
        \put(8,1){\vector(1,0){3.5}}
        %
        % Here, we find another nested use of \step inside \step.
        % \bstep is a variant of \step which _always_ puts its argument into a box for leaving the correct amount of
        % white space. We cannot use \parstepwise here because \put can't go into a box. Hence, just using \step for
        % building the nested formula on the next line would give the wrong size for the nested braces.
        % 
        \put(9.75,0.5){\makebox(0,0){\footnotesize $G_t = \Phi\left(\bstep{\varphi\left(\bstep{x(t)}\right)}\right)$}}
        }
      \step
      {
        \put(13,1){\line(-3,2){1.5}}
        \put(13,1){\line(-3,-2){1.5}}
        \put(11.5,0){\line(0,1){2}}
        \put(12,1){\makebox(0,0){\large $\delta$}}
        }
    \end{picture}%
    }%
\end{center}%
\end{slide}
\end{document}
%KEEPCOMMENTS
%</seminardemo>
%=================================================================================================================
%<*simpledemo>
%<<KEEPCOMMENTS
%-----------------------------------------------------------------------------------------------------------------
%
% Simple examples the for the dynamic features provided by the package texpower.sty.
% 
%-----------------------------------------------------------------------------------------------------------------

\documentclass[12pt]{article}

%-----------------------------------------------------------------------------------------------------------------
% The texpower package is loaded. 
% We give the display option so dynamic features are enabled.
%
\usepackage[display]{texpower}

\begin{document}

\title{The \code{texpower} Package\\{\normalfont Simple Demo}}
\author{Stephan Lehmke\\\code{mailto:Stephan.Lehmke@cs.uni-dortmund.de}}
\maketitle

% The \pause command `splits' the current page at the place it appears, producing two pages, one with everything which
% came before the \pause command, one containing this and additionally the stuff coming after \pause. When these pages
% are presented with acrobar reader in full screen mode (or any other viewer with this capability), the presentation
% will appear to `stop' at the point the \pause command was issued and `resume' in the moment the presenter switches to
% the next page.

\pause

\tableofcontents

\pause

\section{A list environment}

\pause

% As \pause forces a paragraph break, it can not be used to separate a description label from the associated text. For
% this, we use the (very flexible) \stepwise command. Inside the argument of \stepwise, an arbitrary number of \step
% commands may occur. \stepwise will produce as many pages as there are \step commands, making the arguments of the
% \step commands appear ``one by one''.

\stepwise
{%
  \begin{description}
  \item[foo.] \step{bar.}
  \step{\item[baz.]} \step{qux.}
  \end{description}
  }

\newpage

\section{An aligned equation}

\pause

% Normally for \stepwise, if a \step is not yet active, its argument is ignored completely. This would disturb
% alignments, because the width changes with every new activated \step.
% \parstepwise is a variant of \stepwise where the argument of an inactive \step is put into a \phantom, leaving the
% proper amount of white space.

\parstepwise
{%
  % Using eqnarray with equation numbers here means all equation numbers will be visible from the outset, because only
  % the contents of the lines are `filled in'. See the full demo for an example of aligned equations where equation
  % numbers `appear'.
  \begin{eqnarray}
    %
    % When the argument of \step is put into a box (as it happens with \parstepwise), tabulators can not go in there. As
    % we want the equals sign to appear at the same time as the right side of the equation, we use \restep for the
    % latter. \restep is like \step, but it appears at the same time as the previous \step command.
    % 
    \sum_{i=1}^{n} i & \step{=} & \restep{1 + 2 + \cdots + (n-1) + n}\\
    %
                     & \step{=} & \restep{1 + n + 2 + (n-1) + \cdots}\\
    %
                     & \step{=} & \restep
                                  {% We can nest \step commands inside each other. The order of execution is just the
                                   % order of appearance, independent of nesting.
                                   % \switch is a variant of \step which takes two arguments and toggles between them on
                                   % activation. This way, we can make the \underbrace `appear'.
                                   % We insert a \vphantom in the first argument so that the equation numbers will be
                                   % placed correctly whether or not the underbrace is displayed.
                                    \switch
                                    {%
                                      \vphantom{\underbrace{(1 + n) + \cdots + (1 + n)}_{\times\frac{n}{2}}}%
                                      (1 + n) + \cdots + (1 + n)%
                                      }
                                    {\underbrace{(1 + n) + \cdots + (1 + n)}_{\times\frac{n}{2}}}%
                                    }
                                  \\
    %
    % This is another nested application of \step. Note that the spacing of \cdot has to be corrected manually by
    % inserting {} left of it, because otherwise it would behave like a prefix operator.
    %
                     & \step{=} & \restep{\frac{(1 + n)\step{{}\cdot n}}{\restep{2}}}
  \end{eqnarray}
}

\pause

\section{An array}

\stepwise
{% With arrays, beware of problems with automatic calculation of cell widths.
 % 
 % If you want all widths to be calculated automatically, you need to use \parstepwise, with the consequence that
 %   a) tabulators or newlines can not go into the argument of \step,
 %   b) the array `structure' (rules) will be completely visible right from the beginning.
 %   
 % If you want to use \stepwise for being able to build the `structure' (like \hilne's) dynamically (as done in the
 % following), you have to make sure that the cell widths are correct from the very first line, because otherwise the
 % array will expand horizontally, destroying the dynamic effect. This can be assured by
 %   a) using only p cells,
 %   b) making sure all the cells in the first line are at least as wide as the widest cell which will appear later. If
 %      you are using the calc package, this is easiest by putting \makebox[\widthof{widest entry}]{first entry} into
 %      the first cell. Otherwise, you can use \settowidth.
 %      
  \begin{displaymath}
    \begin{array}{rrrrr}
      \step{ n & \log n               & n\log n              & n^2       & 2^n      \\\hline}%
      \step{0 &} \step{\textrm{---}  &} \step{\textrm{---}  &} \step{0  &} \step{1  \\}%
      \step{1 &} \step{0\phantom{.6} &} \step{0\phantom{.8} &} \step{1  &} \step{2  \\}%
      \step{2 &} \step{1\phantom{.6} &} \step{2\phantom{.8} &} \step{4  &} \step{4  \\}%
      \step{3 &} \step{1.6           &} \step{4.8           &} \step{9  &} \step{8  \\}%
      \step{4 &} \step{2\phantom{.6} &} \step{8\phantom{.8} &} \step{16 &} \step{16 \\}%
      \step{5 &} \step{2.3           &} \step{11.6          &} \step{25 &} \step{32   }%
    \end{array}
  \end{displaymath}
}

\pause

\section{A picture}

\pause

\begin{center}%
  \stepwise
  {%
    \setlength{\unitlength}{.95cm}%
    \delimitershortfall-1sp% Just for the nested braces
    \begin{picture}(14,2)
      \put(0,1){\vector(1,0){1}}
      \put(0.5,0.5){\makebox(0,0){\small $x(t)$}}
      \put(13,1){\vector(1,0){1}}
      \put(13.5,0.5){\makebox(0,0){\small $y(t)$}}
      \step
      {
        \put(1,1){\line(3,2){1.5}}
        \put(1,1){\line(3,-2){1.5}}
        \put(2.5,0){\line(0,1){2}}
        \put(2,1){\makebox(0,0){\large $\varphi$}}
        }
      \step
      {
        \put(2.5,1){\vector(1,0){3.5}}
        \put(4.25,0.5){\makebox(0,0){\small $F_t = \varphi\left(x(t)\right)$}}
        }
      \step
      {
        \put(6,0){\framebox(2,2){\large $\Phi$}}
        }
      \step
      {
        \put(8,1){\vector(1,0){3.5}}
        %
        % Here, we find another nested use of \step inside \step.
        % \bstep is a variant of \step which _always_ puts its argument into a box for leaving the correct amount of
        % white space. We cannot use \parstepwise here because \put can't go into a box. Hence, just using \step for
        % building the nested formula on the next line would give the wrong size for the nested braces.
        % 
        \put(9.75,0.5){\makebox(0,0){\small $G_t = \Phi\left(\bstep{\varphi\left(\bstep{x(t)}\right)}\right)$}}
        }
      \step
      {
        \put(13,1){\line(-3,2){1.5}}
        \put(13,1){\line(-3,-2){1.5}}
        \put(11.5,0){\line(0,1){2}}
        \put(12,1){\makebox(0,0){\large $\delta$}}
        }
    \end{picture}%
    }%
\end{center}%
\end{document}
%KEEPCOMMENTS
%</simpledemo>
%=================================================================================================================
%<*slidesdemo>
%<<KEEPCOMMENTS
%-----------------------------------------------------------------------------------------------------------------
%
% Simple examples the for combining the slides class with the dynamic features provided by the package texpower.sty. 
% 
%-----------------------------------------------------------------------------------------------------------------

\documentclass[landscape]{slides}

% slides understands the landscape option, but to get it through to pdflatex, we need to set \pdfpageheight etc. The
% package fixseminar does this.

\usepackage{fixseminar}

% We need to set this page style globally so all dynamically built slides have the right headers.

\pagestyle{slide}


%-----------------------------------------------------------------------------------------------------------------
% The texpower package is loaded. 
% We give the display option so dynamic features are enabled.
%
\usepackage[display]{texpower}


\begin{document}

\begin{slide}
  \title{The \code{texpower} Package\\{\normalfont \texttt{slides} Demo}}
  \author{Stephan Lehmke\\\code{mailto:Stephan.Lehmke@cs.uni-dortmund.de}}
  \maketitle
\end{slide}

\begin{slide}
  \begin{center}
    \textbf{A list environment}
  \end{center}
  
% The \pause command `splits' the current page at the place it appears, producing two pages, one with everything which
% came before the \pause command, one containing this and additionally the stuff coming after \pause. When these pages
% are presented with acrobar reader in full screen mode (or any other viewer with this capability), the presentation
% will appear to `stop' at the point the \pause command was issued and `resume' in the moment the presenter switches to
% the next page.

\pause

% As \pause forces a paragraph break, it can not be used to separate a description label from the associated text. For
% this, we use the (very flexible) \stepwise command. Inside the argument of \stepwise, an arbitrary number of \step
% commands may occur. \stepwise will produce as many pages as there are \step commands, making the arguments of the
% \step commands appear ``one by one''.

\stepwise
{%
  \begin{description}
  \item[foo.] \step{bar.}
  \step{\item[baz.]} \step{qux.}
  \end{description}
  }

\end{slide}


\begin{slide}
  \begin{center}
    \textbf{An aligned equation}
  \end{center}

\pause

% Normally for \stepwise, if a \step is not yet active, its argument is ignored completely. This would disturb
% alignments, because the width changes with every new activated \step.
% \parstepwise is a variant of \stepwise where the argument of an inactive \step is put into a \phantom, leaving the
% proper amount of white space.

\parstepwise
{%
  % Using eqnarray with equation numbers here means all equation numbers will be visible from the outset, because only
  % the contents of the lines are `filled in'. See the full demo for an example of aligned equations where equation
  % numbers `appear'.
  \begin{eqnarray}
    %
    % When the argument of \step is put into a box (as it happens with \parstepwise), tabulators can not go in there. As
    % we want the equals sign to appear at the same time as the right side of the equation, we use \restep for the
    % latter. \restep is like \step, but it appears at the same time as the previous \step command.
    % 
    \sum_{i=1}^{n} i & \step{=} & \restep{1 + 2 + \cdots + (n-1) + n}\\
    %
                     & \step{=} & \restep{1 + n + 2 + (n-1) + \cdots}\\
    %
                     & \step{=} & \restep
                                  {% We can nest \step commands inside each other. The order of execution is just the
                                   % order of appearance, independent of nesting.
                                   % \switch is a variant of \step which takes two arguments and toggles between them on
                                   % activation. This way, we can make the \underbrace `appear'.
                                   % We insert a \vphantom in the first argument so that the equation numbers will be
                                   % placed correctly whether or not the underbrace is didplayed.
                                    \switch
                                    {%
                                      \vphantom{\underbrace{(1 + n) + \cdots + (1 + n)}_{\times\frac{n}{2}}}%
                                      (1 + n) + \cdots + (1 + n)%
                                      }
                                    {\underbrace{(1 + n) + \cdots + (1 + n)}_{\times\frac{n}{2}}}%
                                    }
                                  \\
    %
    % This is another nested application of \step. Note that the spacing of \cdot has to be corrected manually by
    % inserting {} left of it, because otherwise it would behave like a prefix operator.
    %
                     & \step{=} & \restep{\frac{(1 + n)\step{{}\cdot n}}{\restep{2}}}
  \end{eqnarray}
}
\end{slide}

\begin{slide}
  \begin{center}
    \textbf{An array}
  \end{center}

\stepwise
{% With arrays, beware of problems with automatic calculation of cell widths.
 % 
 % If you want all widths to be calculated automatically, you need to use \parstepwise, with the consequence that
 %   a) tabulators or newlines can not go into the argument of \step,
 %   b) the array `structure' (rules) will be completely visible right from the beginning.
 %   
 % If you want to use \stepwise for being able to build the `structure' (like \hilne's) dynamically (as done in the
 % following), you have to make sure that the cell widths are correct from the very first line, because otherwise the
 % array will expand horizontally, destroying the dynamic effect. This can be assured by
 %   a) using only p cells,
 %   b) making sure all the cells in the first line are at least as wide as the widest cell which will appear later. If
 %      you are using the calc package, this is easiest by putting \makebox[\widthof{widest entry}]{first entry} into
 %      the first cell. Otherwise, you can use \settowidth.
 %      
  \begin{displaymath}
    \begin{array}{rrrrr}
      \step
      {%
            n &        \log n        &        n\log n       & \lefteqn{n^2}\phantom{25} & \lefteqn{2^n}\phantom{32} \\
        \hline%
        }%
      \step{0 &} \step{\textrm{---}  &} \step{\textrm{---}  &} \step{0                  &} \step{1                  \\}%
      \step{1 &} \step{0\phantom{.6} &} \step{0\phantom{.8} &} \step{1                  &} \step{2                  \\}%
      \step{2 &} \step{1\phantom{.6} &} \step{2\phantom{.8} &} \step{4                  &} \step{4                  \\}%
      \step{3 &} \step{1.6           &} \step{4.8           &} \step{9                  &} \step{8                  \\}%
      \step{4 &} \step{2\phantom{.6} &} \step{8\phantom{.8} &} \step{16                 &} \step{16                 \\}%
      \step{5 &} \step{2.3           &} \step{11.6          &} \step{25                 &} \step{32                   }%
    \end{array}
  \end{displaymath}
}
\end{slide}

\begin{slide}
  \begin{center}
    \textbf{A picture}
  \end{center}

\pause

\begin{center}%
  \stepwise
  {%
    \setlength{\unitlength}{1.6cm}%
    \delimitershortfall-1sp% Just for the nested braces
    \begin{picture}(14,2)
      \put(0,1){\vector(1,0){1}}
      \put(0.5,0.5){\makebox(0,0){\small $x(t)$}}
      \put(13,1){\vector(1,0){1}}
      \put(13.5,0.5){\makebox(0,0){\small $y(t)$}}
      \step
      {
        \put(1,1){\line(3,2){1.5}}
        \put(1,1){\line(3,-2){1.5}}
        \put(2.5,0){\line(0,1){2}}
        \put(2,1){\makebox(0,0){\large $\varphi$}}
        }
      \step
      {
        \put(2.5,1){\vector(1,0){3.5}}
        \put(4.25,0.5){\makebox(0,0){\small $F_t = \varphi\left(x(t)\right)$}}
        }
      \step
      {
        \put(6,0){\framebox(2,2){\large $\Phi$}}
        }
      \step
      {
        \put(8,1){\vector(1,0){3.5}}
        %
        % Here, we find another nested use of \step inside \step.
        % \bstep is a variant of \step which _always_ puts its argument into a box for leaving the correct amount of
        % white space. We cannot use \parstepwise here because \put can't go into a box. Hence, just using \step for
        % building the nested formula on the next line would give the wrong size for the nested braces.
        % 
        \put(9.75,0.5){\makebox(0,0){\small $G_t = \Phi\left(\bstep{\varphi\left(\bstep{x(t)}\right)}\right)$}}
        }
      \step
      {
        \put(13,1){\line(-3,2){1.5}}
        \put(13,1){\line(-3,-2){1.5}}
        \put(11.5,0){\line(0,1){2}}
        \put(12,1){\makebox(0,0){\large $\delta$}}
        }
    \end{picture}%
    }%
\end{center}%
\end{slide}
\end{document}
%KEEPCOMMENTS
%</slidesdemo>
%=================================================================================================================
%<*spanelexample>
%<<KEEPCOMMENTS
%-----------------------------------------------------------------------------------------------------------------
%  
% Very simple panel example - compare with panelexample.tex

\documentclass[calcdimensions,landscape,letterpaper,KOMA]{powersem}

\usepackage[ps2pdf,colorlinks,pdfpagemode=FullScreen,plainpages=false]{hyperref}

\usepackage[lightbackground,display]{texpower}
\renewcommand{\currentpagevalue}{\value{slide}}
\usepackage{fixseminar}
\usepackage{tpslifonts}

\newlength{\buttonwidth}
\renewcommand{\buttonsep}{0.2\fboxsep}
\renewcommand{\buttonshadowhshift}{0.1\fboxsep}
\renewcommand{\buttonshadowvshift}{-0.1\fboxsep}

\DeclarePanel{left}
{%
  \textsf{\textbf{\footnotesize
      \begin{tabular}[b]{@{}l@{}}
        Your Logo\\
        Your Name\\
        Your Company\\
      \end{tabular}%
      }}%
  \vfill

  \setlength{\buttonwidth}{.5\linewidth-.5ex}
  \button[\buttonwidth]{\Acrobatmenu{FirstPage}}{Start}
  \button[\buttonwidth]{\Acrobatmenu{LastPage}}{End}
  \button[\buttonwidth]{\Acrobatmenu{PrevPage}}{Back}
  \button[\buttonwidth]{\Acrobatmenu{NextPage}}{Next}
}%

\newcommand{\MyHeader}{}

\newcommand{\myheader}[1]{\renewcommand{\MyHeader}{#1}}

\DeclarePanel{top}{\Large\strut\MyHeader}
  
\backgroundstyle{vgradient}

\slideframe{none}
\centerslidesfalse
\pagestyle{empty}

\mklength{\slideleftmargin}{\leftpanelwidth*\ratio{1cm}{\semcm}+.5cm}
\renewcommand{\sliderightmargin}{.5cm}
\mklength{\slidetopmargin}{\toppanelheight*\ratio{1cm}{\semcm}+.5cm}
\renewcommand{\slidebottommargin}{.5cm}

\begin{document}

\begin{slide}

\myheader{Just some info about panels}

If you're using a package that has it's own panel and/or naviagtion buttons (as
\href{ftp://ftp.dante.de/tex-archive/help/Catalogue/entries/pdfscreen.html}%
{\code{pdfscreen}}) don't even consider using the preliminary support for this
in \concept{\TeX Power}. 

Using it for \code{seminar} / \code{powersem} as we are doing here is OK \dots

Things to remember:
\begin{itemize}
\item If you want to declare your own panels, you have to call
\macroname{backgroundstyle} after the panel declaration.      
\item The font color used in the panels are controlled by \code{\carg{pos}paneltextcolor}. 
\end{itemize}

\newslide

\myheader{Some examples for \macroname{pause}}

\pageTransitionDissolve
With dissolve page transition
\begin{itemize}
\item foo\pause
\item bar\pause
\end{itemize}
\pageTransitionReplace
and with plain replace page transition
\begin{itemize}
\item foo\pause
\item bar\pause
\end{itemize}

\end{slide}

\end{document}
%KEEPCOMMENTS
%</spanelexample>
%=================================================================================================================
%<*tabexample>
%<<KEEPCOMMENTS
%-----------------------------------------------------------------------------------------------------------------
%
% Tabular example for the package texpower.sty.
% 
%-----------------------------------------------------------------------------------------------------------------
% Use slifonts. 

\RequirePackage{tpslifonts}

% Input the generic preamble.

\input{__TPpreamble}
\hypersetup{pdftitle={texpower tabular example}}


%-----------------------------------------------------------------------------------------------------------------
% Finally, everything is set up. Here we go...
%
\begin{document}
\begin{slide}
%KEEPCOMMENTS
%</tabexample>
%<*tabexample-src>
%<<KEEPCOMMENTS


%-----------------------------------------------------------------------------------------------------------------
%
\makeslidetitle{\macroname{stepwise} Example: A Tabular}

% In the following, a tabular object is built incrementally. 
% Observe how & and \\ are placed inside the following step to avoid `opening' empty cells.
% The macro \tabend is redefined to `replace' the final \hline by \cline's when a line is not yet complete.
%
  \newcommand{\tabend}{\\\hline}%

  \liststepwise{%
    \begin{center}
      \step{%
        \begin{tabular}{|l|l|l|}
          \hline
          They can & be built & line by line%
          \step{\\\hline or cell\renewcommand{\tabend}{\\\cline{1-1}}}%
          \step{& by\renewcommand{\tabend}{\\\cline{1-2}}}%
          \step{& cell\renewcommand{\tabend}{\\\hline}}%
          \step
          {%
            \\\hline
                                %
                                % Again, \step's are nested inside each other...
                                %
            \step{or}&\step{like}&\step{this.}%
            }%
          \step{\\\hline But\renewcommand{\tabend}{\\\cline{1-1}}}%
          \step{& beware\renewcommand{\tabend}{\\\cline{1-2}}}%
          \step{& of cells growing horizontally!\renewcommand{\tabend}{\\\hline}}%
          \tabend
        \end{tabular}%
        }%
    \end{center}%
    }%
  \newslide
%KEEPCOMMENTS
%</tabexample-src>
%=================================================================================================================
%<*verbexample>
%<<KEEPCOMMENTS
%-----------------------------------------------------------------------------------------------------------------
%  
% Example showing the use of verbatim/fragile steps (using the fragilesteps environment).
%
%-----------------------------------------------------------------------------------------------------------------

\documentclass[12pt,a4paper]{article}
\usepackage[display]{texpower}
\usepackage{ fancyvrb,listings,alltt}
% Command to easily include some lines of code from a file
\def\listcodefromfile#1#2#3{%
\lstinputlisting[firstline=#1,lastline=#2,aboveskip=0pt,belowskip=0pt]{#3}}
\begin{document}

\begin{center}
{\textbf{Example showing the use of verbatim/fragile steps with the fragilesteps environment}}
\end{center}

A simple example (with alltt) showing steps with verbatim text.
Multiple lines in one step is only supported by alltt (and only for the
standard step command, not bstep).
\pause
\begin{fragilesteps}
\begin{alltt}
    One\step{
    Two
    Three}
    Four
\end{alltt}
\end{fragilesteps}

\pause

An easy example (with listings) for including small parts of code in each step.
If you want to include code line by line look at the next example. Notice that in this
example you really don't need to use the fragilesteps environment, because
there are no verbatim code - the next example however...
\lstset{language=Java}
\begin{fragilesteps}
\step{\listcodefromfile{1}{3}{dummy.java}}
\step{\listcodefromfile{4}{7}{dummy.java}}
\end{fragilesteps}

\pause

A (fancy) example using the fancyvrb interface of the listings package.
The re(b)step command is used to make some lines of code appear at the same time.
(You can not include multiple lines in each (b)step command.)
\pause
\lstset{fancyvrb=true}
\fvset{commandchars=\\\[\]}
\begin{fragilesteps}
\begin{Verbatim}[fontfamily=cmr]
    public long recur(int n) {
\bstep[        if (n<1) {]
\rebstep[            return 0;]
\bstep[        } else if (n == 1) {]
\rebstep[            return 1;]
\bstep[        } else {]
\rebstep[            return \bstep[recur(n-1)+recur(n-2);]]
\rebstep[        }]
    }
\end{Verbatim}
\end{fragilesteps}

\end{document}
%KEEPCOMMENTS
%</verbexample>
%=================================================================================================================
%<*end>
\end{slide}
\end{document}
%</end>
%<*enddoc>
\end{document}
%</enddoc>
%=================================================================================================================
%<*figure-mpsrc>
prologues:=0;

beginfig(1);
path p;
p = (-1cm,0)..(0,-1cm)..(1cm,0);
fill p{up}..(0,0){-1,-2}..{up}cycle withcolor red;
draw p..(0,1cm)..cycle withcolor red;
currentpicture := currentpicture shifted -llcorner currentpicture;
endfig;

beginfig(2);
color myblue;
myblue = (0,0.1,1);
path p, q, r;
z1 = (10cm,0cm);
z2 = z1 + 1cm * dir -60;
z3 = z2 + 1cm * dir 0;
z4 = z3 + 1cm * dir 120;
z5 = (1.2cm,0);
z6 = -3 * z5 + 1.2cm * dir -60;
p = z1--z2--z3--z4;
fill p--cycle withcolor myblue;
p := p shifted z5;
fill p--cycle withcolor red;
p := p shifted z5;
fill p--cycle withcolor myblue;
p := p shifted z5;
fill p--cycle withcolor myblue;
for j=1 upto 3:
  p := p shifted z6;
  for i=1 upto 3: 
    fill p--cycle withcolor myblue;
    p := p shifted z5;
  endfor
  fill p--cycle withcolor myblue;
endfor

z7 = z1 + 4*z5 - (0.2cm,0) + 0.2cm * dir 60;
z8 = z7 + 3.5cm*dir 60;
z9 = z8 + (0.35cm,0cm); 
z7 = z10 + whatever*dir 120;
z9 = z10 + whatever*dir 60;
q = z8--z7;
r = z10--z9;
z11 = 1.2cm * dir -60;
z12 = z10-z7;
p := q--r;
fill p--cycle withcolor (0,0.7,1);
q := q shifted z11; 
r := r shifted (z11 + 0.5*z12); 
z13 = urcorner q + whatever*dir 60;
y13 = y8;
z14 = urcorner r + whatever*dir 60;
y14 = y8;
p := z13--q--r--z14;
fill p--cycle withcolor (0,0.5,1);
q := q shifted z11; 
r := r shifted (z11 + 0.5*z12); 
z15 = urcorner q + whatever*dir 60;
y15 = y8;
z16 = urcorner r + whatever*dir 60;
y16 = y8;
p := z15--q--r--z16;
fill p--cycle withcolor (0,0.3,1);
q := q shifted z11; 
r := r shifted (z11 + 0.5*z12); 
z17 = urcorner q + whatever*dir 60;
y17 = y8;
z18 = urcorner r + whatever*dir 60;
y18 = y8;
z19 = llcorner q + 1.0cm * dir -60;
p := z17--q--z19--z18;
fill p--cycle withcolor myblue;

currentpicture := currentpicture shifted -llcorner currentpicture;
endfig;

beginfig(3);
color myblue;
myblue = (0.4,0.7,1);
path p, q, r;
z1 = (0cm,0cm);
z2 = (10cm,0cm);
z3 = z2 + 1cm * dir 65;
z4 = z1 + 1cm * dir 65;
z5 = (1.2cm,0);
z6 = -3 * z5 + 1.2cm * dir -60;
p = z1--z2--z3--z4;
fill p--cycle withcolor myblue;
defaultfont:="rphvr";
label.rt(btex \font\sans=lcmssb8 at 0.3cm \sans UNIVERSIT\"AT DORTMUND etex,0.5*(z1+z4)+(0.5cm,0)) withcolor white;

currentpicture := currentpicture shifted -llcorner currentpicture;
endfig;

end
%</figure-mpsrc>
%<*fig-1>
%<<KEEPCOMMENTS
%!PS
%%BoundingBox: 0 0 58 58 
%%Creator: MetaPost
%%CreationDate: 2004.03.23:0921
%%Pages: 1
%%EndProlog
%%Page: 1 1
 1 0 0 setrgbcolor
newpath 0.25 28.59645 moveto
0.25 12.94115 12.94115 0.25 28.59645 0.25 curveto
44.25175 0.25 56.9429 12.94115 56.9429 28.59645 curveto
56.9429 40.7281 36.86218 45.12794 28.59645 28.59645 curveto
20.33072 12.06496 0.25 16.4648 0.25 28.59645 curveto closepath fill
 0 0.5 dtransform truncate idtransform setlinewidth pop [] 0 setdash
 1 setlinejoin 10 setmiterlimit
newpath 0.25 28.59645 moveto
0.25 12.94115 12.94115 0.25 28.59645 0.25 curveto
44.25175 0.25 56.9429 12.94115 56.9429 28.59645 curveto
56.9429 44.25175 44.25175 56.9429 28.59645 56.9429 curveto
12.94115 56.9429 0.25 44.25175 0.25 28.59645 curveto closepath stroke
showpage
%%EOF
%KEEPCOMMENTS
%</fig-1>
%<*fig-2>
%<<KEEPCOMMENTS
%!PS
%%BoundingBox: 0 0 310 204 
%%Creator: MetaPost
%%CreationDate: 2004.03.24:1259
%%Pages: 1
%%EndProlog
%%Page: 1 1
 0 0.1 1 setrgbcolor
newpath 0 112.92435 moveto
14.17323 88.37553 lineto
42.51968 88.37553 lineto
28.34645 112.92435 lineto
 closepath fill
 1 0 0 setrgbcolor
newpath 34.01566 112.92435 moveto
48.18889 88.37553 lineto
76.53534 88.37553 lineto
62.3621 112.92435 lineto
 closepath fill
 0 0.1 1 setrgbcolor
newpath 68.03131 112.92435 moveto
82.20454 88.37553 lineto
110.551 88.37553 lineto
96.37776 112.92435 lineto
 closepath fill
newpath 102.04697 112.92435 moveto
116.2202 88.37553 lineto
144.56665 88.37553 lineto
130.39342 112.92435 lineto
 closepath fill
newpath 17.00783 83.46584 moveto
31.18106 58.91702 lineto
59.52751 58.91702 lineto
45.35428 83.46584 lineto
 closepath fill
newpath 51.02348 83.46584 moveto
65.19672 58.91702 lineto
93.54317 58.91702 lineto
79.36993 83.46584 lineto
 closepath fill
newpath 85.03914 83.46584 moveto
99.21237 58.91702 lineto
127.55882 58.91702 lineto
113.38559 83.46584 lineto
 closepath fill
newpath 119.0548 83.46584 moveto
133.22803 58.91702 lineto
161.57448 58.91702 lineto
147.40125 83.46584 lineto
 closepath fill
newpath 34.01566 54.00732 moveto
48.18889 29.45851 lineto
76.53534 29.45851 lineto
62.3621 54.00732 lineto
 closepath fill
newpath 68.03131 54.00732 moveto
82.20454 29.45851 lineto
110.551 29.45851 lineto
96.37776 54.00732 lineto
 closepath fill
newpath 102.04697 54.00732 moveto
116.2202 29.45851 lineto
144.56665 29.45851 lineto
130.39342 54.00732 lineto
 closepath fill
newpath 136.06262 54.00732 moveto
150.23586 29.45851 lineto
178.5823 29.45851 lineto
164.40907 54.00732 lineto
 closepath fill
newpath 51.02348 24.54881 moveto
65.19672 0 lineto
93.54317 0 lineto
79.36993 24.54881 lineto
 closepath fill
newpath 85.03914 24.54881 moveto
99.21237 0 lineto
127.55882 0 lineto
113.38559 24.54881 lineto
 closepath fill
newpath 119.0548 24.54881 moveto
133.22803 0 lineto
161.57448 0 lineto
147.40125 24.54881 lineto
 closepath fill
newpath 153.07045 24.54881 moveto
167.24368 0 lineto
195.59013 0 lineto
181.4169 24.54881 lineto
 closepath fill
 0 0.7 1 setrgbcolor
newpath 182.83432 203.75488 moveto
133.22803 117.83403 lineto
138.18874 109.24179 lineto
192.75575 203.75488 lineto
 closepath fill
 0 0.5 1 setrgbcolor
newpath 216.84998 203.75488 moveto
199.84215 174.29637 lineto
150.23586 88.37552 lineto
157.67693 75.48715 lineto
212.24394 170.00024 lineto
231.73213 203.75488 lineto
 closepath fill
 0 0.3 1 setrgbcolor
newpath 250.86565 203.75488 moveto
216.84998 144.83786 lineto
167.24368 58.917 lineto
177.16512 41.73251 lineto
231.73213 136.2456 lineto
270.70853 203.75488 lineto
 closepath fill
 0 0.1 1 setrgbcolor
newpath 284.8813 203.75488 moveto
233.8578 115.37935 lineto
184.25151 29.4585 lineto
198.42474 4.90968 lineto
309.6849 203.75488 lineto
 closepath fill
showpage
%%EOF
%KEEPCOMMENTS
%</fig-2>
%<*fig-3>
%<<KEEPCOMMENTS
%!PS
%%BoundingBox: 0 0 296 26 
%%Creator: MetaPost
%%CreationDate: 2004.03.24:1531
%%Pages: 1
%*Font: lcmssb8 8.504 7.97011 41:988e7c0000000002
%%EndProlog
%%Page: 1 1
 0.4 0.7 1 setrgbcolor
newpath 0 0 moveto
283.46451 0 lineto
295.44437 25.6907 lineto
11.97986 25.6907 lineto
 closepath fill
 1 setgray
23.16316 9.27249 moveto
(UNIVERSIT) lcmssb8 8.504 fshow
82.72096 10.5127 moveto
(\177) lcmssb8 8.504 fshow
81.86467 9.27249 moveto
(AT) lcmssb8 8.504 fshow
101.67796 9.27249 moveto
(DORTMUND) lcmssb8 8.504 fshow
showpage
%%EOF
%KEEPCOMMENTS
%</fig-3>
%<*dummy-java>
%<<KEEPCOMMENTS
    public int dummy(int n) {
        if (n<1) {
            return 0;
        } else {
            return 1;
        }
    }
%KEEPCOMMENTS
%</dummy-java>
%=================================================================================================================
%
% Local Variables: 
% fill-column: 120
% TeX-master: t
% End:
