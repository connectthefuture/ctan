\section{Problèmes connus et FAQ}

\subsection{Incompatibilité avec \tkzname{colortbl.sty}}

 Le problème provient du fait que \tkzname{colortbl.sty} est parfois incompatible avec la commande \tkzname{multicolumn}. Le texte utilisé dans la commande \tkzname{multicolumn} ne doit contenir qu'un seul paragraphe. 
 Il faut simplement ne pas utiliser la commande \tkzname{AQmessage}. Une solution est d'interrompre le QCM pour afficher ce que l'on souhaite puis reprendre le tableau.
 
 \subsection{FAQ}
  \subsubsection{Traduction des commandes}
  Certaines commandes peuvent être traduites ou modifiées comme par exemple : \tkzcname{aq@pre}  et \tkzcname{aq@preVF}, il suffit pour cela d'utiliser \tkzcname{renewcommand} 
  
\begin{tkzltxexample}[]
 \renewcommand{\aq@pre}{Pour chacune des questions ci-dessous, une seule des
  r\'eponses propos\'ees est exacte. Vous devez  cocher la r\'eponse exacte
   sans justification.
 Une bonne r\'eponse rapporte \textbf{\cmdAQ@global@bonus\ point}. Une
  mauvaise r\'eponse enl\`eve \textbf{\cmdAQ@global@malus\ point}.  L'absence
  de r\'eponse ne rapporte ni n'enl\`eve aucun point. Si le total des points
  est n\'egatif, la note globale attribu\'ee \`a l'exercice est \textbf{0}.}% 
\end{tkzltxexample}

  
\endinput