\documentclass{article}
\usepackage[fleqn]{amsmath}
%
% Try these various options.
%
\usepackage[pointsonleft,nototals,forpaper,useforms,
% Select exactly one of the next three options
    nosolutions,
%    answerkey,
%    vspacewithsolns,
    obeylocalversions, % try this in combination with
]{eqexam}

\university
{%
      NORTHWEST FLORIDA STATE COLLEGE\\
          Department of Mathematics
}
\email{storyd@nwfsc.edu}

\examNum{1}\numVersions{5}\forVersion{e}
\subject[CA]{College Algebra}
\longTitleText
    {Test~{\nExam} A}
    {Test~{\nExam} B}
    {Test~{\nExam} C}
    {Test~{\nExam} D}
    {Test~{\nExam} E}
\endlongTitleText
\shortTitleText
    {T\nExam-A}
    {T\nExam-B}
    {T\nExam-C}
    {T\nExam-D}
    {T\nExam-E}
\endshortTitleText
\title[\sExam]{\Exam}
\author{Dr.\ D. P. Story}
\date{\thisterm, \the\year}
\duedate{09/30/09}
\keywords{MAC 1105, Exam \nExam, {\thisterm} semester, \theduedate, at NWFSC}

\newcommand{\cs}[1]{\texttt{\char`\\#1}}
\solAtEndFormatting{\eqequesitemsep{3pt}}
\turnContAnnotOn


\begin{document}

\maketitle

\begin{exam}{T1}

\begin{eqComments}[Instructions:]
This file was used to test the revised definitions of
\cs{longTitleText} \cs{shortTitleText}, and \cs{selectVersion}. This
file has 5 versions, \verb!\numVersions{5}!; however, not every
problem has this number of versions, some have 2, other have 3 or 4.
When you specify a value of \cs{forVersion}, and each problem has a
\cs{selectVersion}, \textsf{eqexam} will perform modular arithmetic
on the number of available versions of a problem, in this way each
problem will be properly posed; consequently, when we say
\verb!\forVersion{e}!, we get version B for the first problem,
version A for 2(a), and so on. Try compiling with other values for
\cs{forVersion} (a--e).

Try compiling the document with the with the \texttt{vspacewithsolns} option
\begin{verbatim}
    \usepackage[vspacewithsolns.pointsonleft,nototals,forpaper,useforms]{eqexam}
\end{verbatim}
The solutions appear at the end of the document, note the solutions
match to the version selected for the question.  This required a
little trick with the solutions file, and requires the exerquiz
package dated 2009/10/05 or later, if one of the pdf options is
used (\texttt{pdf}, \texttt{links}, \texttt{online},
\texttt{email}).

After you have exhausted yourself, try using the \texttt{obeylocalversions} option,
for this to work correctly, you need to comment out the \cs{forVersion} specification
in the preamble.  Go through the file and select one of the versions, by specifying
the first argument of the \cs{selectVersion} command.

Another feature, apropos to the \texttt{obeylocalversions} option is the optional
argument of \cs{longTitleText} and \cs{shortTitleText}. If you say
\cs{longTitleText[b]}, the second title is selected for display in the document;
similarly, for \cs{shortTitleText}. The optional argument can also be used
when \texttt{obeylocalversions}, and the version is set by \cs{forVersion}.
\end{eqComments}

\selectVersion{}{3}
\begin{problem}[5]
\verb!\selectVersion{}{3}! This problem is version \vA{A}\vB{B}\vC{C} of 3.

\begin{solution}
The first problem, version \vA{A}\vB{B}\vC{C} of 3.
\end{solution}
\end{problem}

\begin{problem*}[2ea]
Multi-part question.
    \begin{parts}
\selectVersion{}{4}
    \item \verb!\selectVersion{}{4}! This is problem, version \vA{A}\vB{B}\vC{C}\vD{D} of 4.
\begin{solution}
This is version \vA{A}\vB{B}\vC{C}\vD{D}

The answer is:
\begin{verA}
This is version A
\end{verA}
\begin{verB}
This is version B
\end{verB}
\begin{verC}
This is version C
\end{verC}
\begin{verD}
This is version D
\end{verD}
\end{solution}

\selectVersion{}{3}
    \item \verb!\selectVersion{}{3}! This is a problem, version \vA{A}\vB{B}\vC{C} of 3.
\begin{verA}
This is A
\end{verA}
\begin{verB}
This is version B
\end{verB}
\begin{verE}
This is version E
\end{verE}

\begin{solution}
This is version \vA{A}\vB{B}\vC{C}\vD{D}

The answer is:
\begin{verA}
This is version A
\end{verA}
\begin{verB}
This is version B
\end{verB}
\begin{verC}
This is version C
\end{verC}
\begin{verD}
This is version D
\end{verD}
\end{solution}

\pushProblem
\begin{eqComments}
We insert a new page command so we can see the shortened titles on the next page
to verify that the new system of title management is working correctly.
\end{eqComments}
\emitMessageNearBottom*[.5\textheight]{%
    \vfill\hfill\textbf{Problem~{\eqeCurrProb} continues on next page}}
\popProblem



\selectVersion{}{5}
    \item \verb!\selectVersion{}{5}! This is a problem, version \vA{A}\vB{B}\vC{C}\vD{D}\vE{E} of 5.
\begin{verB}
This is version B
\end{verB}
\begin{verE}
This is version E
\end{verE}
\begin{solution}
This is version \vA{A}\vB{B}\vC{C}\vD{D}\vE{E}

The answer is:
\begin{verB}
This is version B
\end{verB}
\begin{verE}
This is version E
\end{verE}
\end{solution}

\selectVersion{}{4}
    \item \verb!\selectVersion{}{4}! This is a problem, version \vA{A}\vB{B}\vC{C}\vD{D} of 4.
\begin{verA}
This is A
\end{verA}
\begin{verB}
This is version B
\end{verB}
\begin{verE}
This is version E
\end{verE}
\begin{solution}
This is version \vA{A}\vB{B}\vC{C}\vD{D}\vE{E}.

The answer is:
\begin{verA}
This is version A
\end{verA}
\begin{verB}
This is version B
\end{verB}
\begin{verC}
This is version C
\end{verC}
\begin{verD}
This is version D
\end{verD}
\begin{verE}
This is version E
\end{verE}
\end{solution}
\end{parts}
\end{problem*}


\end{exam}

\end{document}
