\documentclass{book}
\usepackage[pagestyles,outermarks,clearempty]{titlesec}[2005/01/22 v2.6]
\usepackage{titletoc}[2005/01/22 v1.5]
\usepackage[repeat]{drama}
\usepackage{example}

\TextHeight {6in}
\TextWidth  {4.5in}

\nonfrenchspacing
\hfuzz 1pt

\newpagestyle {MainMatterPage} {
  \sethead   [\arabic{page}]
             [\small \textsc{oscar wilde}]
             []
             {}
             {\small \textsc{the importance of being earnest}}
             {\arabic{page}}
}

\Forma \personae  {\hangafter 1 \hangindent 1em}
\Facies           {\textsc{\MakeLowercase{#1}}{#2}:%
                   \hskip .5em plus .25em minus .125em %
                   \\ 
                   \textsc{\MakeLowercase{#1}}}
\SpatiumSupra     {.3ex plus .1ex}

\Facies \[        {[\textit{#1\/}]}
\Forma            {\parindent 1em} 
\SpatiumAnte      {.5em plus .25em minus .125em}
\SpatiumPost      {.5em plus .25em minus .125em}
\SpatiumSupra     {1ex }
\SpatiumInfra     {.67ex}

\Facies \(        {\itshape}
\SpatiumSupra     {2ex}
\SpatiumInfra     {1ex}

\Novus \titulus   \Act
\Facies           {\newpage\thispagestyle{empty}%
                   \Rule [parallel, Height = 2pt]%
                   \vskip 8ex %
                   \Nact*{=+1} \LetterSpace{ACT}}
\SpatiumInfra     {5ex}

\Novus \numerus \Nact
\Facies         {\ordinal {#1}\LETTERspace{\theordinal}}

\Novus \titulus \Scene
\Facies         {\RelSize{1}\textsc{\LetterSpace{scene}}}
\SpatiumInfra   {.5\leading}

\begin{document}

\ExampleTitle {OSCAR WILDE}{The Importance of\\[.5ex] Being Earnest}
              {Plays\\[1ex]Penguin Books, 1954}

\begingroup

\SpatiumSupra \titulus {6\leading}
\Facies                {\scshape #1}
\SpatiumInfra \\{1ex}

\titulus {to\\ robert baldwin ross\\ in appreciation\\ and\\  affection\\*}

\cleardoublepage
\thispagestyle{empty}

\titulus {\textit{The Persons of the Play}\\[4ex]
john worthing, j.p.\\
algernon montcrieff \\
rev. canon chasuble, d.d.\\
merriman, \textit{Butler}\\
lane, \textit{Manservant}\\
lady bracknell\\
hon. gwendolen fairfax\\
cecily cardew\\
miss prism, \textit{Governess}
}

\endgroup

\newpage
\thispagestyle{empty}

\Drama

\persona*[1]{jack}
\persona*[2]{algernon}
\persona*[3]{lady bracknell}
\persona*[4]{gwendolen}
\persona*[5]{cecily}
\persona*[6]{miss prism}
\persona*[7]{chasuble\\canon chasuble}
\persona*[8]{lane}

\pagestyle {MainMatterPage} 

\Act \Scene


\(Morning-room in Algernon's flat in Half-Moon Street. 
  The room is luxuriously and artistically furnished. The sound
  of a piano is heard in the adjoining room.\)

\[\8 is arranging afternoon tea on the table, and after the music
  has ceased, \2 enters\]

\2 Did you hear what I was playing, Lane?
\8 I didn't think it polite to listen, sir.
\2  I'm sorry for that, for your sake.  I don't play
accurately \textemdash any one can play accurately \textemdash but I play with
wonderful expression.
As far as the piano is concerned, sentiment
is my forte.  I keep science for Life.
\8  Yes, sir.
\2  And, speaking of the science of Life, have you got the
cucumber sandwiches cut for Lady Bracknell?
\8  Yes, sir. \[Hands them on a salver\]
\2  \[Inspects them, takes two, and sits down on the sofa\]
Oh! \ldots{} by the way, Lane, I see from your book that on Thursday
night, when Lord Shoreman and Mr. Worthing were dining with me,
eight bottles of champagne are entered as having been consumed.
\8  Yes, sir; eight bottles and a pint.

\2  Why is it that at a bachelor's establishment the
servants invariably drink the champagne?  I ask merely for
information.

\8  I attribute it to the superior quality of the wine, sir.  I
have often observed that in married households the champagne is
rarely of a first-rate brand.
\2  Good heavens!  Is marriage so demoralising as that?

\8  I believe it \textit{is} a very pleasant state, sir.  I have had very
little experience of it myself up to the present.  I have only been
married once.  That was in consequence of a misunderstanding
between myself and a young person.

\2  \[languidly\]  I don't know that I am much interested in
your family life, Lane.

\8  No, sir; it is not a very interesting subject.  I never
think of it myself.

\2  Very natural, I am sure.  That will do, Lane, thank you.

\8  Thank you, sir. 

\[c]\8 goes out\]

\2  Lanes views on marriage seem somewhat lax.  Really, if
the lower orders don't set us a good example, what on earth is the
use of them?  They seem, as a class, to have absolutely no sense of
moral responsibility.

\[c]Enter \8\]

\8  Mr. Ernest Worthing.

\[c]Enter \1. \8 goes out\]

\2  How are you, my dear Ernest?  What brings you up to
town?

\1  Oh, pleasure, pleasure!  What else should bring one
anywhere?  Eating as usual, I see, Algy!

\2  \[stiffly\]  I believe it is customary in good society to
take some slight refreshment at five o'clock.  Where have you been
since last Thursday?

\1  \[sitting down on the sofa\]  In the country.

\2  What on earth do you do there?

\1 \[pulling off his gloves\] When one is in town one amuses oneself.
When one is in the country one amuses other people. It is excessively
boring.

\2  And who are the people you amuse?

\1  \[airily\]  Oh, neighbours, neighbours.

\2  Got nice neighbours in your part of Shropshire?

\1  Perfectly horrid!  Never speak to one of them.

\2  How immensely you must amuse them  \[Goes over and takes
sandwich\]  By the way, Shropshire is your county, is it
not?

\1  Eh?  Shropshire?  Yes, of course.  Hallo!  Why all these
cups?  Why cucumber sandwiches?  Why such reckless extravagance in
one so young?  Who is coming to tea?

\2  Oh! merely Aunt Augusta and Gwendolen.

\1  How perfectly delightful!

\2  Yes, that is all very well; but I am afraid Aunt Augusta
won't quite approve of your being here.

\1  May I ask why?

\2  My dear fellow, the way you flirt with Gwendolen is
perfectly disgraceful.  It is almost as bad as the way Gwendolen
flirts with you.

\1  I am in love with Gwendolen.  I have come up to town
expressly to propose to her.

\2  I thought you had come up for pleasure? \ldots{} I call
that business.

\1  How utterly unromantic you are!

\2  I really don't see anything romantic in proposing.  It
is very romantic to be in love.  But there is nothing romantic
about a definite proposal.  Why, one may be accepted.  One usually
is, I believe.  Then the excitement is all over.  The very essence
of romance is uncertainty.  If ever I get married, I'll certainly
try to forget the fact.

\1  I have no doubt about that, dear Algy.  The Divorce Court
was specially invented for people whose memories are so curiously
constituted.

\2  Oh! there is no use speculating on that subject.
Divorces are made in Heaven \textendash\,\textendash  \[\1 puts out his hand to take a
sandwich.  \2 at once interferes\]  Please don't touch the
cucumber sandwiches.  They are ordered specially for Aunt Augusta.
\[Takes one and eats it\]

\1  Well, you have been eating them all the time.

\2  That is quite a different matter.  She is my aunt.
\[Takes plate from below\] Have some bread and butter.  The bread
and butter is for Gwendolen.  Gwendolen is devoted to bread and
butter.

\1  \[advancing to table and helping himself\]  And very good
bread and butter it is too.

\2  Well, my dear fellow, you need not eat as if you were
going to eat it all.  You behave as if you were married to her
already.  You are not married to her already, and I don't think you
ever will be.

\1  Why on earth do you say that?

\2  Well, in the first place girls never marry the men they
flirt with.  Girls don't think it right.

\1  Oh, that is nonsense!

\2  It isn't.  It is a great truth.  It accounts for the
extraordinary number of bachelors that one sees all over the place.
In the second place, I don't give my consent.

\1  Your consent!

\2  My dear fellow, Gwendolen is my first cousin.  And
before I allow you to marry her, you will have to clear up the
whole question of Cecily.  \[r]Rings bell\]

\1  Cecily!  What on earth do you mean?  What do you mean, Algy,
by Cecily!  I don't know any one of the name of Cecily.

\[c]Enter \8\]

\2  Bring me that cigarette case Mr. Worthing left in the
smoking-room the last time he dined here.

\8  Yes, sir.

\[c]\8 goes out\]

\1  Do you mean to say you have had my cigarette case all this
time?  I wish to goodness you had let me know.  I have been writing
frantic letters to Scotland Yard about it.  I was very nearly
offering a large reward.

\2  Well, I wish you would offer one.  I happen to be more
than usually hard up.

\1  There is no good offering a large reward now that the thing
is found.

\[c]Enter \8 with the cigarette case on a salver. \\\2 takes it
at once.  \8 goes out\]

\2  I think that is rather mean of you, Ernest, I must say.
\[Opens case and examines it\]  However, it makes no matter, for,
now that I look at the inscription inside, I find that the thing
isn't yours after all.

\1  Of course it's mine.  \[Moving to him\]  You have seen me
with it a hundred times, and you have no right whatsoever to read
what is written inside.  It is a very ungentlemanly thing to read a
private cigarette case.

\2  Oh! it is absurd to have a hard and fast rule about what
one should read and what one shouldn't.  More than half of modern
culture depends on what one shouldn't read.

\1  I am quite aware of the fact, and I don't propose to discuss
modern culture.  It isn't the sort of thing one should talk of in
private.  I simply want my cigarette case back.

\2  Yes; but this isn't your cigarette case.  This cigarette
case is a present from some one of the name of Cecily, and you said
you didn't know any one of that name.

\1  Well, if you want to know, Cecily happens to be my aunt.

\2  Your aunt!

\1  Yes.  Charming old lady she is, too.  Lives at Tunbridge
Wells.  Just give it back to me, Algy.

\2  \[retreating to back of sofa\]  But why does she call
herself little Cecily if she is your aunt and lives at Tunbridge
Wells?  \[Reading\]  `From little Cecily with her fondest love.'

\1  \[moving to sofa and kneeling upon it\]  My dear fellow, what
on earth is there in that?  Some aunts are tall, some aunts are not
tall.  That is a matter that surely an aunt may be allowed to
decide for herself.  You seem to think that every aunt should be
exactly like your aunt!  That is absurd!  For Heaven's sake give me
back my cigarette case.  \[Follows \2 round the room\]

\2  Yes.  But why does your aunt call you her uncle?  `From
little Cecily, with her fondest love to her dear Uncle Jack.'
There is no objection, I admit, to an aunt being a small aunt, but
why an aunt, no matter what her size may be, should call her own
nephew her uncle, I can't quite make out.  Besides, your name isn't
Jack at all; it is Ernest.

\1  It isn't Ernest; it's Jack.

\2  You have always told me it was Ernest.  I have
introduced you to every one as Ernest.  You answer to the name of
Ernest.  You look as if your name was Ernest.  You are the most
earnest-looking person I ever saw in my life.  It is perfectly
absurd your saying that your name isn't Ernest.  It's on your
cards.  Here is one of them.  \[Taking it from case\]  'Mr. Ernest
Worthing, B. 4, The Albany.'  I'll keep this as a proof that your
name is Ernest if ever you attempt to deny it to me, or to
Gwendolen, or to any one else.  \[Puts the card in his pocket\]

\1  Well, my name is Ernest in town and Jack in the country, and
the cigarette case was given to me in the country.

\2  Yes, but that does not account for the fact that your
small Aunt Cecily, who lives at Tunbridge Wells, calls you her dear
uncle.  Come, old boy, you had much better have the thing out at
once.

\1  My dear Algy, you talk exactly as if you were a dentist.  It
is very vulgar to talk like a dentist when one isn't a dentist.  It
produces a false impression,

\2  Well, that is exactly what dentists always do.  Now, go
on!  Tell me the whole thing.  I may mention that I have always
suspected you of being a confirmed and secret Bunburyist; and I am
quite sure of it now.

\1  Bunburyist? What on earth do you mean by a Bunburyist?

\2  I'll reveal to you the meaning of that incomparable
expression as soon as you are kind enough to inform me why you are
Ernest in town and Jack in the country.

\1  Well, produce my cigarette case first.

\2  Here it is.  \[Hands cigarette case\]  Now produce your
explanation, and pray make it improbable.  \[Sits on sofa\]

\1  My dear fellow, there is nothing improbable about my
explanation at all.  In fact it's perfectly ordinary.  Old Mr.
Thomas Cardew, who adopted me when I was a little boy, made me in
his will guardian to his grand-daughter, Miss Cecily Cardew.
Cecily, who addresses me as her uncle from motives of respect that
you could not possibly appreciate, lives at my place in the country
under the charge of her admirable governess, Miss Prism.

\2  Where in that place in the country, by the way?

\1  That is nothing to you, dear boy.  You are not going to be
invited \ldots{} I may tell you candidly that the place is not in
Shropshire.

\2  I suspected that, my dear fellow!  I have Bunburyed all
over Shropshire on two separate occasions.  Now, go on.  Why are
you Ernest in town and Jack in the country?

\1  My dear Algy, I don't know whether you will be able to
understand my real motives.  You are hardly serious enough.  When
one is placed in the position of guardian, one has to adopt a very
high moral tone on all subjects.  It's one's duty to do so.  And as
a high moral tone can hardly be said to conduce very much to either
one's health or one's happiness, in order to get up to town I have
always pretended to have a younger brother of the name of Ernest,
who lives in the Albany, and gets into the most dreadful scrapes.
That, my dear Algy, is the whole truth pure and simple.

\2  The truth is rarely pure and never simple.  Modern life
would be very tedious if it were either, and modern literature a
complete impossibility!

\1  That wouldn't be at all a bad thing.

\2  Literary criticism is not your forte, my dear fellow.
Don't try it.  You should leave that to people who haven't been at
a University.  They do it so well in the daily papers.  What you
really are is a Bunburyist.  I was quite right in saying you were a
Bunburyist.  You are one of the most advanced Bunburyists I know.

\1  What on earth do you mean?

\2  You have invented a very useful younger brother called
Ernest, in order that you may be able to come up to town as often
as you like.  I have invented an invaluable permanent invalid
called Bunbury, in order that I may be able to go down into the
country whenever I choose.  Bunbury is perfectly invaluable.  If it
wasn't for Bunbury's extraordinary bad health, for instance, I
wouldn't be able to dine with you at Willis's to-night, for I have
been really engaged to Aunt Augusta for more than a week.

\1  I haven't asked you to dine with me anywhere to-night.

\2  I know.  You are absurdly careless about sending out
invitations.  It is very foolish of you.  Nothing annoys people so
much as not receiving invitations.

\1  You had much better dine with your Aunt Augusta.

\2  I haven't the smallest intention of doing anything of
the kind.  To begin with, I dined there on Monday, and once a week
is quite enough to dine with one's own relations.  In the second
place, whenever I do dine there I am always treated as a member of
the family, and sent down with either no woman at all, or two.  In
the third place, I know perfectly well whom she will place me next
to, to-night.  She will place me next Mary Farquhar, who always
flirts with her own husband across the dinner-table.  That is not
very pleasant.  Indeed, it is not even decent \ldots{} and that sort
of thing is enormously on the increase.  The amount of women in
London who flirt with their own husbands is perfectly scandalous.
It looks so bad.  It in simply washing one's clean linen in public.
Besides, now that I know you to be a confirmed Bunburyist I
naturally want to talk to you about Bunburying.  I want to tell you
the rules.

\1  I'm not a Bunburyist at all.  If Gwendolen accepts me, I am
going to kill my brother, indeed I think I'll kill him in any case.
Cecily is a little too much interested in him.  It is rather a
bore.  So I am going to get rid of Ernest.  And I strongly advise
you to do the same with Mr \ldots{} with your invalid friend who has
the absurd name.

\2  Nothing will induce me to part with Bunbury, and if you
ever get married, which seems to me extremely problematic, you will
be very glad to know Bunbury.  A man who marries without knowing
Bunbury has a very tedious time of it.

\1  That is nonsense.  If I marry a charming girl like
Gwendolen, and she is the only girl I ever saw in my life that I
would marry, I certainly won't want to know Bunbury.

\2  Then your wife will.  You don't seem to realise, that in
married life three is company and two is none.

\1  \[sententiously\]  That, my dear young friend, is the theory
that the corrupt French Drama has been propounding for the last
fifty years.

\2  Yes; and that the happy English home has proved in half
the time.

\1  For heaven's sake, don't try to be cynical.  It's perfectly
easy to be cynical.

\2  My dear fellow, it isn't easy to be anything nowadays.
There's such a lot of beastly competition about.  \[The sound of an
electric bell is heard\]  Ah! that must be Aunt Augusta.  Only
relatives, or creditors, ever ring in that Wagnerian manner.  Now,
if I get her out of the way for ten minutes, so that you can have
an opportunity for proposing to Gwendolen, may I dine with you
to-night at Willis's?

\1  I suppose so, if you want to.

\2  Yes, but you must be serious about it.  I hate people
who are not serious about meals.  It is so shallow of them.

\[c]Enter \8\]

\8 Lady Bracknell and Miss Fairfax.
\newpage
\[\2 goes forward to meet them.  Enter \3 and
\4\]

\3  Good afternoon, dear Algernon, I hope you are
behaving very well.

\2  I'm feeling very well, Aunt Augusta.

\3  That's not quite the same thing.  In fact the two
things rarely go together.  \[Sees \1 and bows to him with icy
coldness\]

\2  \[to \4\]  Dear me, you are smart!

\4  I am always smart!  Am I not, Mr. Worthing?

\1  You're quite perfect, Miss Fairfax.

\4  Oh! I hope I am not that.  It would leave no room for
developments, and I intend to develop in many directions.
\[\4 and \1 sit down together in the corner\]

\3  I'm sorry if we are a little late, Algernon, but I
was obliged to call on dear Lady Harbury.  I hadn't been there
since her poor husband's death.  I never saw a woman so altered;
she looks quite twenty years younger.  And now I'll have a cup of
tea, and one of those nice cucumber sandwiches you promised me.

\2  Certainly, Aunt Augusta.  \[Goes over to tea-table\]

\3  Won't you come and sit here, Gwendolen?

\4  Thanks, mamma, I'm quite comfortable where I am.

\2  \[picking up empty plate in horror\]  Good heavens!
Lane!  Why are there no cucumber sandwiches?  I ordered them
specially.

\8  \[gravely\]  There were no cucumbers in the market this
morning, sir.  I went down twice.

\2  No cucumbers!

\8  No, sir.  Not even for ready money.

\2  That will do, Lane, thank you.

\8  Thank you, sir.  \[Goes out\]

\2  I am greatly distressed, Aunt Augusta, about there being
no cucumbers, not even for ready money.

\3  It really makes no matter, Algernon.  I had some
crumpets with Lady Harbury, who seems to me to be living entirely
for pleasure now.

\2  I hear her hair has turned quite gold from grief.

\3  It certainly has changed its colour.  From what
cause I, of course, cannot say.\[\2 crosses and hands tea\]%
Thank you.  I've quite a treat for you to-night, Algernon.  I am
going to send you down with Mary Farquhar.  She is such a nice
woman, and so attentive to her husband.  It's delightful to watch
them.

\2  I am afraid, Aunt Augusta, I shall have to give up the
pleasure of dining with you to-night after all.

\3  \[frowning\]I hope not, Algernon.  It would put
my table completely out.  Your uncle would have to dine upstairs.
Fortunately he is accustomed to that.

\2  It is a great bore, and, I need hardly say, a terrible
disappointment to me, but the fact is I have just had a telegram to
say that my poor friend Bunbury is very ill again.     \[Exchanges
glances with \1\]      They seem to think I should be with him.

\3  It is very strange.  This Mr. Bunbury seems to
suffer from curiously bad health.

\2  Yes; poor Bunbury is a dreadful invalid.

\3  Well, I must say, Algernon, that I think it is
high time that Mr. Bunbury made up his mind whether he was going to
live or to die.  This shilly-shallying with the question is absurd.
Nor do I in any way approve of the modern sympathy with invalids.
I consider it morbid.  Illness of any kind is hardly a thing to be
encouraged in others.  Health is the primary duty of life.  I am
always telling that to your poor uncle, but he never seems to take
much notice \ldots{} as far as any improvement in his ailment goes.  I
should be much obliged if you would ask Mr. Bunbury, from me, to be
kind enough not to have a relapse on Saturday, for I rely on you to
arrange my music for me.  It is my last reception, and one wants
something that will encourage conversation, particularly at the end
of the season when every one has practically said whatever they had
to say, which, in most cases, was probably not much.

\2  I'll speak to Bunbury, Aunt Augusta, if he is still
conscious, and I think I can promise you he'll be all right by
Saturday.  Of course the music is a great difficulty.  You see, if
one plays good music, people don't listen, and if one plays bad
music people don't talk.  But I'll ran over the programme I've
drawn out, if you will kindly come into the next room for a moment.

\3  Thank you, Algernon.  It is very thoughtful of
you.  \[Rising, and following \2\]  I'm sure the programme
will be delightful, after a few expurgations.  French songs I
cannot possibly allow.  People always seem to think that they are
improper, and either look shocked, which is vulgar, or laugh, which
is worse.  But German sounds a thoroughly respectable language, and
indeed, I believe is so.  Gwendolen, you will accompany me.

\4  Certainly, mamma.

\[c]\3 and \2 go into the music-room,\\ \4 remains behind\]

\1  Charming day it has been, Miss Fairfax.

\4  Pray don't talk to me about the weather, Mr. Worthing.
Whenever people talk to me about the weather, I always feel quite
certain that they mean something else.  And that makes me so
nervous.

\1  I do mean something else.

\4  I thought so.  In fact, I am never wrong.

\1  And I would like to be allowed to take advantage of Lady
Bracknell's temporary absence \ldots{}

\4  I would certainly advise you to do so.  Mamma has a way
of coming back suddenly into a room that I have often had to speak
to her about.

\1  \[nervously\]  Miss Fairfax, ever since I met you I have
admired you more than any girl \ldots{} I have ever met since \ldots{} I
met you.

\4  Yes, I am quite well aware of the fact.  And I often
wish that in public, at any rate, you had been more demonstrative.
For me you have always had an irresistible fascination.  Even
before I met you I was far from indifferent to you.  \[\1 looks at
her in amazement\]  We live, as I hope you know, Mr Worthing, in an
age of ideals.  The fact is constantly mentioned in the more
expensive monthly magazines, and has reached the provincial
pulpits, I am told; and my ideal has always been to love some one
of the name of Ernest.  There is something in that name that
inspires absolute confidence.  The moment Algernon first mentioned
to me that he had a friend called Ernest, I knew I was destined to
love you.

\1  You really love me, Gwendolen?

\4  Passionately!

\1  Darling!  You don't know how happy you've made me.

\4  My own Ernest!

\1  But you don't really mean to say that you couldn't love me
if my name wasn't Ernest?

\4  But your name is Ernest.

\1  Yes, I know it is.  But supposing it was something else?  Do
you mean to say you couldn't love me then?

\4  \[glibly\]  Ah! that is clearly a metaphysical
speculation, and like most metaphysical speculations has very
little reference at all to the actual facts of real life, as we
know them.

\1  Personally, darling, to speak quite candidly, I don't much
care about the name of Ernest \ldots{} I don't think the name suits me
at all.

\4  It suits you perfectly.  It is a divine name.  It has a
music of its own.  It produces vibrations.

\1  Well, really, Gwendolen, I must say that I think there are
lots of other much nicer names.  I think Jack, for instance, a
charming name.

\4  Jack? \ldots{} No, there is very little music in the name
Jack, if any at all, indeed.  It does not thrill.  It produces
absolutely no vibrations \ldots{} I have known several Jacks, and they
all, without exception, were more than usually plain.  Besides,
Jack is a notorious domesticity for John!  And I pity any woman who
is married to a man called John.  She would probably never be
allowed to know the entrancing pleasure of a single moment's
solitude.  The only really safe name is Ernest

\1  Gwendolen, I must get christened at once \textendash\,\textendash I mean we must
get married at once.  There is no time to be lost.

\4  Married, Mr. Worthing?

\1  \[astounded\]  Well \ldots{} surely.  You know that I love you,
and you led me to believe, Miss Fairfax, that you were not
absolutely indifferent to me.

\4  I adore you.  But you haven't proposed to me yet.
Nothing has been said at all about marriage.  The subject has not
even been touched on.

\1  Well \ldots{} may I propose to you now?

\4  I think it would be an admirable opportunity.  And to
spare you any possible disappointment, Mr. Worthing, I think it
only fair to tell you quite frankly before-hand that I am fully
determined to accept you.

\1  Gwendolen!

\4  Yes, Mr. Worthing, what have you got to say to me?

\1  You know what I have got to say to you.

\4  Yes, but you don't say it.

\1  Gwendolen, will you marry me?  \[Goes on his knees\]

\4  Of course I will, darling.  How long you have been
about it!  I am afraid you have had very little experience in how
to propose.

\1  My own one, I have never loved any one in the world but you.

\4  Yes, but men often propose for practice.  I know my
brother Gerald does.  All my girl-friends tell me so.  What
wonderfully blue eyes you have, Ernest!  They are quite, quite,
blue.  I hope you will always look at me just like that, especially
when there are other people present.

\[c]Enter \3\]

\3  Mr. Worthing!  Rise, sir, from this semi-re\-cum\-bent
posture.  It is most indecorous.

\4  Mamma!  \[He tries to rise; she restrains him\]  I must
beg you to retire.  This is no place for you.  Besides, Mr.
Worthing has not quite finished yet.

\3  Finished what, may I ask?

\4  I am engaged to Mr. Worthing, mamma.  \[They rise
together\]

\3  Pardon me, you are not engaged to any one.  When
you do become engaged to some one, I, or your father, should his
health permit him, will inform you of the fact.  An engagement
should come on a young girl as a surprise, pleasant or unpleasant,
as the case may be.  It is hardly a matter that she could be
allowed to arrange for herself \ldots{} And now I have a few questions
to put to you, Mr. Worthing.  While I am making these inquiries,
you, Gwendolen, will wait for me below in the carriage.

\4  \[reproachfully\]  Mamma!

\3  In the carriage, Gwendolen!\[\4 goes to
the door.  She and \1 blow kisses to each other behind 
\3's back.  \3 looks vaguely about as if she
could not understand what the noise was.  Finally turns round\]
Gwendolen, the carriage!

\4  Yes, mamma.  \[Goes out, looking back at \1\]

\3  \[sitting down\]  You can take a seat, Mr.
Worthing.

\[c]Looks in her pocket for note-book and pencil\]

\1  Thank you, Lady Bracknell, I prefer standing.

\3  \[Pencil and note-book in hand\]  I feel bound to
tell you that you are not down on my list of eligible young men,
although I have the same list as the dear Duchess of Bolton has.
We work together, in fact.  However, I am quite ready to enter your
name, should your answers be what a really affectionate mother
requires.  Do you smoke?

\1  Well, yes, I must admit I smoke.

\3  I am glad to hear it.  A man should always have an
occupation of some kind.  There are far too many idle men in London
as it is.  How old are you?

\1  Twenty-nine.

\3  A very good age to be married at.  I have always
been of opinion that a man who desires to get married should know
either everything or nothing.  Which do you know?

\1  \[after some hesitation\]  I know nothing, Lady Bracknell.

\3  I am pleased to hear it.  I do not approve of
anything that tampers with natural ignorance.  Ignorance is like a
delicate exotic fruit; touch it and the bloom is gone.  The whole
theory of modern education is radically unsound.  Fortunately in
England, at any rate, education produces no effect whatsoever.  If
it did, it would prove a serious danger to the upper classes, and
probably lead to acts of violence in Grosvenor Square.  What is
your income?

\1  Between seven and eight thousand a year.

\3  \[makes a note in her book\]  In land, or in
investments?

\1  In investments, chiefly.

\3  That is satisfactory.  What between the duties
expected of one during one's lifetime, and the duties exacted from
one after one's death, land has ceased to be either a profit or a
pleasure.  It gives one position, and prevents one from keeping it
up.  That's all that can be said about land.

\1  I have a country house with some land, of course, attached
to it, about fifteen hundred acres, I believe; but I don't depend
on that for my real income.  In fact, as far as I can make out, the
poachers are the only people who make anything out of it.

\3  A country house!  How many bedrooms?  Well, that
point can be cleared up afterwards.  You have a town house, I hope?
A girl with a simple, unspoiled nature, like Gwendolen, could
hardly be expected to reside in the country.

\1  Well, I own a house in Belgrave Square, but it is let by the
year to Lady Bloxham.  Of course, I can get it back whenever I
like, at six months' notice.

\3  Lady Bloxham?  I don't know her.

\1  Oh, she goes about very little.  She is a lady considerably
advanced in years.

\3  Ah, nowadays that is no guarantee of
respectability of character.  What number in Belgrave Square?

\1  \oldstylenums{149}.

\3  \[shaking her head\]  The unfashionable side.  I
thought there was something.  However, that could easily be
altered.

\1  Do you mean the fashion, or the side?

\3  \[sternly\]  Both, if necessary, I presume.  What
are your polities?

\1  Well, I am afraid I really have none.  I am a Liberal
Unionist.

\3  Oh, they count as Tories.  They dine with us.  Or
come in the evening, at any rate.  Now to minor matters.  Are your
parents living?

\1  I have lost both my parents.

\3  To lose one parent, Mr. Worthing, may be regarded
as a misfortune; to lose both looks like carelessness.  Who was
your father?  He was evidently a man of some wealth.  Was he born
in what the Radical papers call the purple of commerce, or did he
rise from the ranks of the aristocracy?

\1  I am afraid I really don't know.  The fact is, Lady
Bracknell, I said I had lost my parents.  It would be nearer the
truth to say that my parents seem to have lost me \ldots{} I don't
actually know who I am by birth.  I was \ldots{} well, I was found.

\3  Found!

\1  The late Mr. Thomas Cardew, an old gentleman of a very
charitable and kindly disposition, found me, and gave me the name
of Worthing, because he happened to have a first-class ticket for
Worthing in his pocket at the time.  Worthing is a place in Sussex.
It is a seaside resort.

\3  Where did the charitable gentleman who had a
first-class ticket for this seaside resort find you?

\1  \[gravely\]  In a hand-bag.

\3  A hand-bag?

\1  \[very seriously\]  Yes, Lady Bracknell.  I was in a hand-bag%
\textemdash a somewhat large, black leather hand-bag, with handles to
it\textemdash an ordinary hand-bag in fact.

\3  In what locality did this Mr. James, or Thomas,
Cardew come across this ordinary hand-bag?

\1  In the cloak-room at Victoria Station.  It was given to him
in mistake for his own.

\3  The cloak-room at Victoria Station?

\1  Yes.  The Brighton line.

\3  The line is immaterial.  Mr. Worthing, I confess I
feel somewhat bewildered by what you have just told me.  To be
born, or at any rate bred, in a hand-bag, whether it had handles or
not, seems to me to display a contempt for the ordinary decencies
of family life that reminds one of the worst excesses of the French
Revolution.  And I presume you know what that unfortunate movement
led to?  As for the particular locality in which the hand-bag was
found, a cloak-room at a railway station might serve to conceal a
social indiscretion\textemdash has probably, indeed, been used for that
purpose before now\textemdash but it could hardly be regarded as an assured
basis for a recognised position in good society.

\1  May I ask you then what you would advise me to do?  I need
hardly say I would do anything in the world to ensure Gwendolen's
happiness.

\3  I would strongly advise you, Mr. Worthing, to try
and acquire some relations as soon as possible, and to make a
definite effort to produce at any rate one parent, of either sex,
before the season is quite over.

\1  Well, I don't see how I could possibly manage to do that.  I
can produce the hand-bag at any moment.  It is in my dressing-room
at home.  I really think that should satisfy you, Lady Bracknell.

\3  Me, sir!  What has it to do with me?  You can
hardly imagine that I and Lord Bracknell would dream of allowing
our only daughter\textemdash a girl brought up with the utmost care%
\textemdash to
marry into a cloak-room, and form an alliance with a parcel?  Good
morning, Mr. Worthing!

\[c]\3 sweeps out in majestic indignation\]

\1  Good morning!  \[\2, from the other room, strikes up
the Wedding March.  Jack looks perfectly furious, and goes to the
door\]  For goodness' sake don't play that ghastly tune, Algy.  How
idiotic you are!

\[c]The music stops and \2 enters cheerily\]

\2  Didn't it go off all right, old boy?  You don't mean to
say Gwendolen refused you?  I know it is a way she has.  She is
always refusing people.  I think it is most ill-natured of her.

\1  Oh, Gwendolen is as right as a trivet.  As far as she is
concerned, we are engaged.  Her mother is perfectly unbearable.
Never met such a Gorgon \ldots{} I don't really know what a Gorgon is
like, but I am quite sure that Lady Bracknell is one.  In any case,
she is a monster, without being a myth, which is rather unfair . .
. I beg your pardon, Algy, I suppose I shouldn't talk about your
own aunt in that way before you.

\2  My dear boy, I love hearing my relations abused.  It is
the only thing that makes me put up with them at all.  Relations
are simply a tedious pack of people, who haven't got the remotest
knowledge of how to live, nor the smallest instinct about when to
die.

\1  Oh, that is nonsense!

\2  It isn't!

\1  Well, I won't argue about the matter.  You always want to
argue about things.

\2  That is exactly what things were originally made for.

\1  Upon my word, if I thought that, I'd shoot myself \ldots{} \[A
pause\]  You don't think there is any chance of Gwendolen becoming
like her mother in about a hundred and fifty years, do you, Algy?

\2  All women become like their mothers.  That is their
tragedy.  No man does.  That's his.

\1  Is that clever?

\2  It is perfectly phrased! and quite as true as any
observation in civilised life should be.

\1  I am sick to death of cleverness.  Everybody is clever
nowadays.  You can't go anywhere without meeting clever people.
The thing has become an absolute public nuisance.  I wish to
goodness we had a few fools left.

\2  We have.

\1  I should extremely like to meet them.  What do they talk
about?

\2  The fools?  Oh! about the clever people, of course.

\1  What fools!

\2  By the way, did you tell Gwendolen the truth about your
being Ernest in town, and Jack in the country?

\1  \[in a very patronising manner\]  My dear fellow, the truth
isn't quite the sort of thing one tells to a nice, sweet, refined
girl.  What extraordinary ideas you have about the way to behave to
a woman!

\2  The only way to behave to a woman is to make love to
her, if she is pretty, and to some one else, if she is plain.

\1  Oh, that is nonsense.

\2  What about your brother?  What about the profligate
Ernest?

\1  Oh, before the end of the week I shall have got rid of him.
I'll say he died in Paris of apoplexy.  Lots of people die of
apoplexy, quite suddenly, don't they?

\2  Yes, but it's hereditary, my dear fellow.  It's a sort
of thing that runs in families.  You had much better say a severe
chill.

\1  You are sure a severe chill isn't hereditary, or anything of
that kind?

\2  Of course it isn't!

\1  Very well, then.  My poor brother Ernest to carried off
suddenly, in Paris, by a severe chill.  That gets rid of him.

\2  But I thought you said that \ldots{} Miss Cardew was a
little too much interested in your poor brother Ernest?  Won't she
feel his loss a good deal?

\1  Oh, that is all right.  Cecily is not a silly romantic girl,
I am glad to say.  She has got a capital appetite, goes long walks,
and pays no attention at all to her lessons.

\2  I would rather like to see Cecily.

\1  I will take very good care you never do.  She is excessively
pretty, and she is only just eighteen.

\2  Have you told Gwendolen yet that you have an excessively
pretty ward who is only just eighteen?

\1  Oh! one doesn't blurt these things out to people.  Cecily
and Gwendolen are perfectly certain to be extremely great friends.
I'll bet you anything you like that half an hour after they have
met, they will be calling each other sister.

\2  Women only do that when they have called each other a
lot of other things first.  Now, my dear boy, if we want to get a
good table at Willis's, we really must go and dress.  Do you know
it is nearly seven?

\1  \[irritably\]  Oh!  It always is nearly seven.

\2  Well, I'm hungry.

\1  I never knew you when you weren't \ldots{}

\2  What shall we do after dinner?  Go to a theatre?

\1  Oh no!  I loathe listening.

\2  Well, let us go to the Club?

\1  Oh, no!  I hate talking.

\2  Well, we might trot round to the Empire at ten?

\1  Oh, no!  I can't bear looking at things.  It is so silly.

\2  Well, what shall we do?

\1  Nothing!

\2  It is awfully hard work doing nothing.  However, I don't
mind hard work where there is no definite object of any kind.

\[c]Enter \8\]

\8  Miss Fairfax.

\[c]Enter \4  \8 goes out\]

\2  Gwendolen, upon my word!

\4 Algy, kindly turn your back.  I have something very
particular to say to Mr. Worthing.

\2  Really, Gwendolen, I don't think I can allow this at
all.

\4  Algy, you always adopt a strictly immoral attitude
towards life.  You are not quite old enough to do that.  \[\2
retires to the fireplace\]

\1  My own darling!

\4  Ernest, we may never be married.  From the expression
on mamma's face I fear we never shall.  Few parents nowadays pay
any regard to what their children say to them.  The old-fashioned
respect for the young is fast dying out.  Whatever influence I ever
had over mamma, I lost at the age of three.  But although she may
prevent us from becoming man and wife, and I may marry some one
else, and marry often, nothing that she can possibly do can alter
my eternal devotion to you.

\1  Dear Gwendolen!

\4  The story of your romantic origin, as related to me by
mamma, with unpleasing comments, has naturally stirred the deeper
fibres of my nature.  Your Christian name has an irresistible
fascination.  The simplicity of your character makes you
exquisitely incomprehensible to me.  Your town address at the
Albany I have.  What is your address in the country?

\1  The Manor House, Woolton, Hertfordshire.

\[\2, who has been carefully listening, smiles to himself, and
writes the address on his shirt-cuff.  Then picks up the Railway
Guide\]

\4  There is a good postal service, I suppose?  It may be
necessary to do something desperate.  That of course will require
serious consideration.  I will communicate with you daily.

\1  My own one!

\4  How long do you remain in town?

\1  Till Monday.

\4  Good!  Algy, you may turn round now.

\2  Thanks, I've turned round already.

\4  You may also ring the bell.

\1  You will let me see you to your carriage, my own darling?

\4  Certainly.

\1  \[to \8, who now enters\]  I will see Miss Fairfax out.

\8  Yes, sir.  \[\1 and \4 go off\]

\[\8 presents several letters on a salver to \2  It is to
be surmised that they are bills, as \2, after looking at the
envelopes, tears them up\]

\2  A glass of sherry, Lane.

\8  Yes, sir.

\2  To-morrow, Lane, I'm going Bunburying.

\8  Yes, sir.

\2  I shall probably not be back till Monday.  You can put
up my dress clothes, my smoking jacket, and all the Bunbury suits .
. .

\8  Yes, sir.  \[Handing sherry\]

\2  I hope to-morrow will be a fine day, Lane.

\8  It never is, sir.

\2  Lane, you're a perfect pessimist.

\8  I do my best to give satisfaction, sir.

\[c]Enter \1.  \8 goes off\]

\1  There's a sensible, intellectual girl! the only girl I ever
cared for in my life.  \[\2 is laughing immoderately\]  What
on earth are you so amused at?

\2  Oh, I'm a little anxious about poor Bunbury, that in
all.

\1  If you don't take care, your friend Bunbury will get you
into a serious scrape some day.

\2  I love scrapes.  They are the only things that are never
serious.

\1  Oh, that's nonsense, Algy.  You never talk anything but
nonsense.

\2  Nobody ever does.

\[\1 looks indignantly at him, and leaves the room.  \2
lights a cigarette, reads his shirt-cuff, and smiles\]

\Act \Scene

\persona*[8]{Merriman}

\(Garden at the Manor House.  A flight of grey stone steps leads up
to the house.  The garden, an old-fashioned one, full of roses.
Time of year, July. Basket chairs, and a table covered with books,
are set under a large yew-tree.\)


\[\6 discovered seated at the table. \5 is at the back, watering
flowers.\]

\6\[calling\] Cecily, Cecily!  Surely such a utilitarian
occupation as the watering of flowers is rather Moulton's duty than
yours?  Especially at a moment when intellectual pleasures await
you.  Your German grammar is on the table.  Pray open it at page
fifteen.  We will repeat yesterday's lesson.

\5\[coming over very slowly\]  But I don't like German.  It
isn't at all a becoming language.  I know perfectly well that I
look quite plain after my German lesson.

\6 Child, you know how anxious your guardian is that you
should improve yourself in every way.  He laid particular stress on
your German, as he was leaving for town yesterday.  Indeed, he
always lays stress on your German when he is leaving for town.

\5  Dear Uncle Jack is so very serious!  Sometimes he is so
serious that I think he cannot be quite well.

\6  \[drawing herself up\]  Your guardian enjoys the best
of health, and his gravity of demeanour is especially to be
commanded in one so comparatively young as he is.
I know no one who has a higher sense of duty and responsibility.

\5  I suppose that is why he often looks a little bored when
we three are together.

\6  Cecily!  I am surprised at you.  Mr. Worthing has many
troubles in his life.  Idle merriment and triviality would be out
of place in his conversation.  You must remember his constant
anxiety about that unfortunate young man his brother.

\5  I wish Uncle Jack would allow that unfortunate young man,
his brother, to come down here sometimes.  We might have a good
influence over him, Miss Prism.  I am sure you certainly would.
You know German, and geology, and things of that kind influence a
man very much.\[\5 begins to write in her diary\]

\6  \[shaking her head\]  I do not think that even I could
produce any effect on a character that according to his own
brother's admission is irretrievably weak and vacillating.  Indeed
I am not sure that I would desire to reclaim him.  I am not in
favour of this modern mania for turning bad people into good people
at a moment's notice.  As a man sows so let him reap.  You must put
away your diary, Cecily.  I really don't see why you should keep a
diary at all.

\5  I keep a diary in order to enter the wonderful secrets of
my life.  If I didn't write them down, I should probably forget all
about them.

\6  Memory, my dear Cecily, is the diary that we all carry
about with~us.

\5  Yes, but it usually chronicles the things that have never
happened, and couldn't possibly have happened.  I believe that
Memory is responsible for nearly all the three-volume novels that
Mudie sends us.

\6  Do not speak slightingly of the three-volume novel,
Cecily.  I wrote one myself in earlier days.

\5  Did you really, Miss Prism?  How wonderfully clever you
are!  I hope it did not end happily?  I don't like novels that end
happily.  They depress me so much.

\6  The good ended happily, and the bad unhappily.  That
is what Fiction means.

\5  I suppose so.  But it seems very unfair.  And was your
novel ever published?

\6  Alas! no.  The manuscript unfortunately was abandoned. \hfill\break
\[\5 starts\]
I use the word in the sense of lost or mislaid.
To your work, child, these speculations are profitless.

\5  \[smiling\]  But I see dear Dr. Chasuble coming up through
the garden.


\6  \[rising and advancing\]  Dr. Chasuble!  This is indeed
a pleasure.

\[c]Enter \7.\]

\7 And how are we this morning?  Miss Prism, you are, I
trust, well?

\5  Miss Prism has just been complaining of a slight headache.
I think it would do her so much good to have a short stroll with
you in the Park, Dr.~Chasuble.

\6  Cecily, I have not mentioned anything about a
headache.

\5  No, dear Miss Prism, I know that, but I felt instinctively
that you had a headache.  Indeed I was thinking about that, and not
about my German lesson, when the Rector came in.

\7  I hope, Cecily, you are not inattentive.

\5  Oh, I am afraid I am.

{\tolerance=10000
\7  That is strange.  Were I fortunate enough to be Miss\break
Prism's pupil, I would hang upon her lips. \[\6 glares\]
I spoke metaphorically. \textemdash My metaphor was drawn from bees.  Ahem!
Mr. Worthing, I suppose, has not returned from town yet?
}

\6  We do not expect him till Monday afternoon.

\7  Ah yes, he usually likes to spend his Sunday in London.
He is not one of those whose sole aim is enjoyment, as, by all
accounts, that unfortunate young man his brother seems to be.  But
I must not disturb Egeria and her pupil any longer.

\6  Egeria?  My name is Laetitia, Doctor.

\7  \[bowing\]  A classical allusion merely, drawn from the
Pagan authors.  I shall see you both no doubt at Evensong?

\6  I think, dear Doctor, I will have a stroll with you.
I find I have a headache after all, and a walk might do it good.

\7  With pleasure, Miss Prism, with pleasure.  We might go
as far as the schools and back.

\6  That would be delightful.  Cecily, you will read your
Political Economy in my absence.  The chapter on the Fall of the
Rupee you may omit.  It is somewhat too sensational.  Even these
metallic problems have their melodramatic side.

\[c]Goes down the garden with \7.\]

\5  \[picks up books and throws them back on table\]  Horrid
Political Economy!  Horrid Geography!  Horrid, horrid German!

\[c]Enter \8 with a card on a salver.\]

\8 Mr. Ernest Worthing has just driven over from the
station.  He has brought his luggage with him.

\5  \[takes the card and reads it\]  'Mr. Ernest Worthing, B.
4, The Albany, W.'  Uncle Jack's brother!  Did you tell him Mr.
Worthing was in town?

\8 Yes, Miss.  He seemed very much disappointed.  I
mentioned that you and Miss Prism were in the garden.  He said he
was anxious to speak to you privately for a moment.

\5  Ask Mr. Ernest Worthing to come here.  I suppose you had
better talk to the housekeeper about a room for him.

\8  Yes, Miss. \[\8 goes off.\]


\5  I have never met any really wicked person before.  I feel
rather frightened.  I am so afraid he will look just like every one
else.

\[c]Enter \2, very gay and debonnair\] 
He does!

\2 \[raising his hat\]  You are my little cousin Cecily, I'm
sure.

\5  You are under some strange mistake.  I am not little.  In
fact, I believe I am more than usually tall for my age.  \[\2
is rather taken aback\]  But I am your cousin Cecily.  You, I see
from your card, are Uncle Jack's brother, my cousin Ernest, my
wicked cousin Ernest.

\2  Oh! I am not really wicked at all, cousin Cecily.  You
mustn't think that I am wicked.

\5  If you are not, then you have certainly been deceiving us
all in a very inexcusable manner.  I hope you have not been leading
a double life, pretending to be wicked and being really good all
the time.  That would be hypocrisy.

\2  \[looks at her in amazement\]  Oh!  Of course I have been
rather reckless.

\5  I am glad to hear it.

\2  In fact, now you mention the subject, I have been very
bad in my own small way.

\5  I don't think you should be so proud of that, though I am
sure it must have been very pleasant.

\2  It is much pleasanter being here with you.

\5  I can't understand how you are here at all.  Uncle Jack
won't be back till Monday afternoon.

\2  That is a great disappointment.  I am obliged to go up
by the first train on Monday morning.  I have a business
appointment that I am anxious . . . to miss?

\5  Couldn't you miss it anywhere but in London?

\2  No: the appointment is in London.

\5  Well, I know, of course, how important it is not to keep a
business engagement, if one wants to retain any sense of the beauty
of life, but still I think you had better wait till Uncle Jack
arrives.  I know he wants to speak to you about your emigrating.

\2  About my what?

\5  Your emigrating.  He has gone up to buy your outfit.

\2  I certainly wouldn't let Jack buy my outfit.  He has no
taste in neckties at all.

\5  I don't think you will require neckties.  Uncle Jack is
sending you to Australia.

\2  Australia!  I'd sooner die.

\5  Well, he said at dinner on Wednesday night, that you would
have to choose between this world, the next world, and Australia.

\2  Oh, well!  The accounts I have received of Australia and
the next world, are not particularly encouraging.  This world is
good enough for me, cousin Cecily.

\5  Yes, but are you good enough for it?

\2  I'm afraid I'm not that.  That is why I want you to
reform me.  You might make that your mission, if you don't mind,
cousin Cecily.

\5  I'm afraid I've no time, this afternoon.

\2  Well, would you mind my reforming myself this afternoon?

\5  It is rather Quixotic of you.  But I think you should try.

\2  I will.  I feel better already.

\5  You are looking a little worse.

\2  That is because I am hungry.

\5  How thoughtless of me.  I should have remembered that when
one is going to lead an entirely new life, one requires regular and
wholesome meals.  Won't you come in?

\2  Thank you.  Might I have a buttonhole first?  I never
have any appetite unless I have a buttonhole first.

\5  A Marechal Niel?  \[Picks up scissors\]

\2  No, I'd sooner have a pink rose.

\5  Why?  \[Cuts a flower\]

\2  Because you are like a pink rose, Cousin Cecily.

\5  I don't think it can be right for you to talk to me like
that.  Miss Prism never says such things to me.

\2  Then Miss Prism is a short-sighted old lady.  \[\5
puts the rose in his buttonhole\]  You are the prettiest girl I
ever saw.

\5  Miss Prism says that all good looks are a snare.

\2  They are a snare that every sensible man would like to
be caught~in.

\5  Oh, I don't think I would care to catch a sensible man.  I
shouldn't know what to talk to him about.

\[c]They pass into the house.  \6 and \7 return.\]

\6  You are too much alone, dear Dr. Chasuble.  You should
get married.  A misanthrope I can understand \textemdash a womanthrope,
never!

\7  \[with a scholar's shudder\]  Believe me, I do not
deserve so neologistic a phrase.  The precept as well as the
practice of the Primitive Church was distinctly against matrimony.

\6  \[sententiously\]  That is obviously the reason why the
Primitive Church has not lasted up to the present day.  And you do
not seem to realise, dear Doctor, that by persistently remaining
single, a man converts himself into a permanent public temptation.
Men should be more careful; this very celibacy leads weaker vessels
astray.

\7  But is a man not equally attractive when married?

\6  No married man is ever attractive except to his wife.

\7  And often, I've been told, not even to her.

\6  That depends on the intellectual sympathies of the
woman.  Maturity can always be depended on.  Ripeness can be
trusted.  Young women are green.  \[\7 starts\]       I spoke
horticulturally.  My metaphor was drawn from fruits.  But where is
Cecily?

\7  Perhaps she followed us to the schools.


\[Enter \1 slowly from the back of the garden.  He is dressed in
the deepest mourning, with crape hatband and black gloves\]

\6  Mr. Worthing!

\7  Mr. Worthing?

\6  This is indeed a surprise.  We did not look for you
till Monday afternoon.

\1\[shakes \6's hand in a tragic manner\]  I have
returned sooner than I expected.  Dr. Chasuble, I hope you are well?

\7  Dear Mr. Worthing, I trust this garb of woe does not
betoken some terrible calamity?

\1  My brother.

\6  More shameful debts and extravagance?

\7  Still leading his life of pleasure?

\1  \[shaking his head\]                      Dead!

\7  Your brother Ernest dead?

\1  Quite dead.

\6  What a lesson for him!  I trust he will profit by it.

\7  Mr. Worthing, I offer you my sincere condolence.  You
have at least the consolation of knowing that you were always the
most generous and forgiving of brothers.

\1  Poor Ernest!  He had many faults, but it is a sad, sad blow.

\7  Very sad indeed.  Were you with him at the end?

\1  No.  He died abroad; in Paris, in fact.  I had a telegram
last night from the manager of the Grand Hotel.

\7  Was the cause of death mentioned?

\1  A severe chill, it seems.

\6  As a man sows, so shall he reap.

\7  \[raising his hand\]  Charity, dear Miss Prism, charity!
None of us are perfect.  I myself am peculiarly susceptible to
draughts.  Will the interment take place here?

\1  No.  He seems to have expressed a desire to be buried in
Paris.

\7  In Paris! \[Shakes his head\] I fear that hardly points
to any very serious state of mind at the last.  You would no doubt
wish me to make some slight allusion to this tragic domestic
affliction next Sunday. \[\1 presses his hand convulsively\]
My sermon on the meaning of the manna in the wilderness can be adapted
to almost any occasion, joyful, or, as in the present case,
distressing. \[All sigh\] I have preached it at harvest
celebrations, christenings, confirmations, on days of humiliation
and festal days.  The last time I delivered it was in the
Cathedral, as a charity sermon on behalf of the Society for the
Prevention of Discontent among the Upper Orders.  The Bishop, who
was present, was much struck by some of the analogies I drew.

\1  Ah! that reminds me, you mentioned christenings I think, Dr.
Chasuble?  I suppose you know how to christen all right?  \[\7
looks astounded\]  I mean, of course, you are continually
christening, aren't you?

\6  It is, I regret to say, one of the Rector's most
constant duties in this parish.  I have often spoken to the poorer
classes on the subject.  But they don't seem to know what thrift
is.

\7  But is there any particular infant in whom you are
interested, Mr. Worthing?  Your brother was, I believe, unmarried,
was he not?

\1  Oh yes.

\6  \[bitterly\]  People who live entirely for pleasure
usually are.

\1  But it is not for any child, dear Doctor.  I am very fond of
children.  No! the fact is, I would like to be christened myself,
this afternoon, if you have nothing better to do.

\7  But surely, Mr. Worthing, you have been christened
already?

\1  I don't remember anything about it.

\7  But have you any grave doubts on the subject?

\1  I certainly intend to have.  Of course I don't know if the
thing would bother you in any way, or if you think I am a little
too old now.

\7  Not at all.  The sprinkling, and, indeed, the immersion
of adults is a perfectly canonical practice.

\1  Immersion!

\7  You need have no apprehensions.  Sprinkling is all that
is necessary, or indeed I think advisable.  Our weather is so
changeable.  At what hour would you wish the ceremony performed?

\1  Oh, I might trot round about five if that would suit you.

\7  Perfectly, perfectly!  In fact I have two similar
ceremonies to perform at that time.  A case of twins that occurred
recently in one of the outlying cottages on your own estate.  Poor
Jenkins the carter, a most hard-working man.

\1  Oh!  I don't see much fun in being christened along with
other babies.  It would be childish.  Would half-past five do?

\7  Admirably!  Admirably!  \[Takes out watch\]  And now,
dear Mr. Worthing, I will not intrude any longer into a house of
sorrow.  I would merely beg you not to be too much bowed down by
grief.  What seem to us bitter trials are often blessings in
disguise.

\6  This seems to me a blessing of an extremely obvious
kind.

\[c]Enter \5 from the house.\]

\5  Uncle Jack!  Oh, I am pleased to see you back.  But what
horrid clothes you have got on!  Do go and change them.

\6  Cecily!

\7  My child! my child!  \[\5 goes towards \1; he
kisses her brow in a melancholy manner\]

\5  What is the matter, Uncle Jack?  Do look happy!  You look
as if you had toothache, and I have got such a surprise for you.
Who do you think is in the dining-room?  Your brother!

\1  Who?

\5  Your brother Ernest.  He arrived about half an hour ago.

\1  What nonsense!  I haven't got a brother.

\5  Oh, don't say that.  However badly he may have behaved to
you in the past he is still your brother.  You couldn't be so
heartless as to disown him.  I'll tell him to come out.  And you
will shake hands with him, won't you, Uncle Jack?  \[Runs back into
the house\]

\7  These are very joyful tidings.

\6  After we had all been resigned to his loss, his sudden
return seems to me peculiarly distressing.

\1  My brother is in the dining-room?  I don't know what it all
means.  I think it is perfectly absurd.

\[c]Enter \2 and \5 hand in hand.\\  They come slowly up to
\1.\]

\1  Good heavens!  \[Motions \2 away\]

\2  Brother John, I have come down from town to tell you
that I am very sorry for all the trouble I have given you, and that
I intend to lead a better life in the future.  \[\1 glares at him
and does not take his hand\]

\5  Uncle Jack, you are not going to refuse your own brother's
hand?

\1  Nothing will induce me to take his hand.  I think his coming
down here disgraceful.  He knows perfectly well why.

\5  Uncle Jack, do be nice.  There is some good in every one.
Ernest has just been telling me about his poor invalid friend Mr.
Bunbury whom he goes to visit so often.  And surely there must be
much good in one who is kind to an invalid, and leaves the
pleasures of London to sit by a bed of pain.

\1  Oh! he has been talking about Bunbury, has he?

\5  Yes, he has told me all about poor Mr. Bunbury, and his
terrible state of health.

\1  Bunbury!  Well, I won't have him talk to you about Bunbury
or about anything else.  It is enough to drive one perfectly
frantic.

\2  Of course I admit that the faults were all on my side.
But I must say that I think that Brother John's coldness to me is
peculiarly painful.  I expected a more enthusiastic welcome,
especially considering it is the first time I have come here.

\5  Uncle Jack, if you don't shake hands with Ernest I will
never forgive you.

\1  Never forgive me?

\5  Never, never, never!

\1  Well, this is the last time I shall ever do it.  \[Shakes
with \2 and glares\]

\7  It's pleasant, is it not, to see so perfect a
reconciliation?  I think we might leave the two brothers together.

\6  Cecily, you will come with us.

\5  Certainly, Miss Prism.  My little task of reconciliation
is over.

\7  You have done a beautiful action to-day, dear child.

\6  We must not be premature in our judgments.

\5  I feel very happy.  \[They all go off except \1 and \2\]

\1  You young scoundrel, Algy, you must get out of this place as
soon as possible.  I don't allow any Bunburying here.

\[c]Enter \8.\]

\8  I have put Mr. Ernest's things in the room next to
yours, sir.  I suppose that is all right?

\1  What?

\8  Mr. Ernest's luggage, sir.  I have unpacked it and put
it in the room next to your own.

\1  His luggage?

\8  Yes, sir.  Three portmanteaus, a dressing-case, two hat-boxes,
and a large luncheon-basket.

\2  I am afraid I can't stay more than a week this time.

\1  Merriman, order the dog-cart at once.  Mr. Ernest has been
suddenly called back to town.

\8  Yes, sir.  \[Goes back into the house\]

\2  What a fearful liar you are, Jack.  I have not been
called back to town at all.

\1  Yes, you have.

\2  I haven't heard any one call me.

\1  Your duty as a gentleman calls you back.

\2  My duty as a gentleman has never interfered with my
pleasures in the smallest degree.

\1  I can quite understand that.

\2  Well, Cecily is a darling.

\1  You are not to talk of Miss Cardew like that.  I don't like
it.

\2  Well, I don't like your clothes.  You look perfectly
ridiculous in them.  Why on earth don't you go up and change?  It
is perfectly childish to be in deep mourning for a man who is
actually staying for a whole week with you in your house as a
guest.  I call it grotesque.

\1  You are certainly not staying with me for a whole week as a
guest or anything else.  You have got to leave . . . by the four-five
train.

\2  I certainly won't leave you so long as you are in
mourning.  It would be most unfriendly.  If I were in mourning you
would stay with me, I suppose.  I should think it very unkind if
you didn't.

\1  Well, will you go if I change my clothes?

\2  Yes, if you are not too long.  I never saw anybody take
so long to dress, and with such little result.

\1  Well, at any rate, that is better than being always over-dressed
as you are.

\2  If I am occasionally a little over-dressed, I make up
for it by being always immensely over-educated.

\1  Your vanity is ridiculous, your conduct an outrage, and your
presence in my garden utterly absurd.  However, you have got to
catch the four-five, and I hope you will have a pleasant journey
back to town.  This Bunburying, as you call it, has not been a
great success for you.

\[c]Goes into the house\]

\2  I think it has been a great success.  I'm in love with
Cecily, and that is everything.

\[Enter \5 at the back of the garden.  She picks up the can and
begins to water the flowers\]

\00But I must see her before I go, and
make arrangements for another Bunbury.  Ah, there she is.

\5  Oh, I merely came back to water the roses.  I thought you
were with Uncle Jack.

\2  He's gone to order the dog-cart for me.

\5  Oh, is he going to take you for a nice drive?

\2  He's going to send me away.

\5  Then have we got to part?

\2  I am afraid so.  It's a very painful parting.

\5  It is always painful to part from people whom one has
known for a very brief space of time.  The absence of old friends
one can endure with equanimity.  But even a momentary separation
from anyone to whom one has just been introduced is almost
unbearable.

\2  Thank you.

\[c]Enter \8.\]

\8  The dog-cart is at the door, sir.

\[c]\2 looks appealingly at \5\]

\5  It can wait, Merriman for . . . five minutes.

\8  Yes, Miss.

\[c]Exit \8\]

\2  I hope, Cecily, I shall not offend you if I state quite
frankly and openly that you seem to me to be in every way the
visible personification of absolute perfection.

\5  I think your frankness does you great credit, Ernest.  If
you will allow me, I will copy your remarks into my diary.  \[Goes
over to table and begins writing in diary\]

\2  Do you really keep a diary?  I'd give anything to look
at it.  May~I?

\5  Oh no.  \[Puts her hand over it\]  You see, it is simply a
very young girl's record of her own thoughts and impressions, and
consequently meant for publication.  When it appears in volume form
I hope you will order a copy.  But pray, Ernest, don't stop.  I
delight in taking down from dictation.  I have reached 'absolute
perfection'.  You can go on.  I am quite ready for more.

\2  \[somewhat taken aback\]  Ahem!  Ahem!

\5  Oh, don't cough, Ernest.  When one is dictating one should
speak fluently and not cough.  Besides, I don't know how to spell a
cough.  \[Writes as \2 speaks\]

\2  \[speaking very rapidly\]  Cecily, ever since I first
looked upon your wonderful and incomparable beauty, I have dared to
love you wildly, passionately, devotedly, hopelessly.

\5  I don't think that you should tell me that you love me
wildly, passionately, devotedly, hopelessly.  Hopelessly doesn't
seem to make much sense, does it?

\2  Cecily!

\[c]Enter \8.\]

\8  The dog-cart is waiting, sir.

\2  Tell it to come round next week, at the same hour.

\8  \[Looks at \5, who makes no sign\]  Yes, sir.

\[c]\8 retires.\]

\5  Uncle Jack would be very much annoyed if he knew you were
staying on till next week, at the same hour.

\2  Oh, I don't care about Jack.  I don't care for anybody
in the whole world but you.  I love you, Cecily.  You will marry
me, won't you?

\5  You silly boy!  Of course.  Why, we have been engaged for
the last three months.

\2  For the last three months?

\5  Yes, it will be exactly three months on Thursday.

\2  But how did we become engaged?

\5  Well, ever since dear Uncle Jack first confessed to us
that he had a younger brother who was very wicked and bad, you of
course have formed the chief topic of conversation between myself
and Miss Prism.  And of course a man who is much talked about is
always very attractive.  One feels there must be something in him,
after all.  I daresay it was foolish of me, but I fell in love with
you, Ernest.

\2  Darling!  And when was the engagement actually settled?

\5  On the 14th of February last.  Worn out by your entire
ignorance of my existence, I determined to end the matter one way
or the other, and after a long struggle with myself I accepted you
under this dear old tree here.  The next day I bought this little
ring in your name, and this is the little bangle with the true
lover's knot I promised you always to wear.

\2  Did I give you this?  It's very pretty, isn't it?

\5  Yes, you've wonderfully good taste, Ernest.  It's the
excuse I've always given for your leading such a bad life.  And
this is the box in which I keep all your dear letters.  \[Kneels at
table, opens box, and produces letters tied up with blue ribbon\]

\2  My letters!  But, my own sweet Cecily, I have never
written you any letters.

\5  You need hardly remind me of that, Ernest.  I remember
only too well that I was forced to write your letters for you.  I
wrote always three times a week, and sometimes oftener.

\2  Oh, do let me read them, Cecily?


\5  Oh, I couldn't possibly.  They would make you far too
conceited. \[Replaces box\]  The three you wrote me after I had
broken of the engagement are so beautiful, and so badly spelled,
that even now I can hardly read them without crying a little.

\2  But was our engagement ever broken off?

\5  Of course it was.  On the 22nd of last March.  You can see
the entry if you like. \[Shows diary\]  `To-day I broke off my
engagement with Ernest.  I feel it is better to do so.  The weather
still continues charming.'

\2  But why on earth did you break it of?  What had I done?
I had done nothing at all.  Cecily, I am very much hurt indeed to
hear you broke it off.  Particularly when the weather was so
charming.

\5  It would hardly have been a really serious engagement if
it hadn't been broken off at least once.  But I forgave you before
the week was out.

\2  \[crossing to her, and kneeling\]  What a perfect angel
you are, Cecily.

\5  You dear romantic boy.  \[He kisses her, she puts her
fingers through his hair\]  I hope your hair curls naturally, does
it?

\2  Yes, darling, with a little help from others.

\5  I am so glad.

\2  You'll never break of our engagement again, Cecily?

\5  I don't think I could break it off now that I have
actually met you.  Besides, of course, there is the question of
your name.

\2  Yes, of course.  \[Nervously\]

\5  You must not laugh at me, darling, but it had always been
a girlish dream of mine to love some one whose name was Ernest.
\[\2 rises, \5 also\]  There is something in that name
that seems to inspire absolute confidence.  I pity any poor married
woman whose husband is not called Ernest.

\2  But, my dear child, do you mean to say you could not
love me if I had some other name?

\5  But what name?

\2  Oh, any name you like \textemdash Algernon \textemdash for instance . . .

\5  But I don't like the name of Algernon.

\2  Well, my own dear, sweet, loving little darling, I
really can't see why you should object to the name of Algernon.  It
is not at all a bad name.  In fact, it is rather an aristocratic
name.  Half of the chaps who get into the Bankruptcy Court are
called Algernon.  But seriously, Cecily . . . \[Moving to her\] . . .
if my name was Algy, couldn't you love me?

\5  \[rising\]  I might respect you, Ernest, I might admire
your character, but I fear that I should not be able to give you my
undivided attention.

\2  Ahem!  Cecily!  \[Picking up hat\]  Your Rector here is,
I suppose, thoroughly experienced in the practice of all the rites
and ceremonials of the Church?

\5  Oh, yes.  Dr. Chasuble is a most learned man.  He has
never written a single book, so you can imagine how much he knows.

\2  I must see him at once on a most important christening
\textemdash I mean on most important business.

\5  Oh!

\2  I shan't be away more than half an hour.

\5  Considering that we have been engaged since February the
14th, and that I only met you to-day for the first time, I think it
is rather hard that you should leave me for so long a period as
half an hour.  Couldn't you make it twenty minutes?

\2  I'll be back in no time. \[Kisses her and rushes down the garden.\]

\5  What an impetuous boy he is!  I like his hair so much.  I
must enter his proposal in my diary.

\[c]Enter \8.\]

\8  A Miss Fairfax has just called to see Mr. Worthing.  On
very important business, Miss Fairfax states.

\5  Isn't Mr. Worthing in his library?

\8  Mr. Worthing went over in the direction of the Rectory
some time ago.

\5  Pray ask the lady to come out here; Mr. Worthing is sure
to be back soon.  And you can bring tea.

\8  Yes, Miss.  

\[c]Goes out\]

\5  Miss Fairfax!  I suppose one of the many good elderly
women who are associated with Uncle Jack in some of his
philanthropic work in London.  I don't quite like women who are
interested in philanthropic work.  I think it is so forward of
them.

\[c]Enter \8.\]

\8  Miss Fairfax.

\[c]Enter \4. Exit \8.\]

\5  \[advancing to meet her\]  Pray let me introduce myself to
you.  My name is Cecily Cardew.

\7 Cecily Cardew?  \[Moving to her and shaking hands\]
What a very sweet name!  Something tells me that we are going to be
great friends.  I like you already more than I can say.  My first
impressions of people are never wrong.

\5  How nice of you to like me so much after we have known
each other such a comparatively short time.  Pray sit down.

\4  \[still standing up\]  I may call you Cecily, may I not?

\5  With pleasure!

\4  And you will always call me Gwendolen, won't you?

\5  If you wish.

\4  Then that is all quite settled, is it not?

\5  I hope so.  \[A pause.  They both sit down together\]

\4  Perhaps this might be a favourable opportunity for my
mentioning who I am.  My father is Lord Bracknell.  You have never
heard of papa, I suppose?

\5  I don't think so.

\4  Outside the family circle, papa, I am glad to say, is
entirely unknown.  I think that is quite as it should be.  The home
seems to me to be the proper sphere for the man.  And certainly
once a man begins to neglect his domestic duties he becomes
painfully effeminate, does he not?  And I don't like that.  It
makes men so very attractive.  Cecily, mamma, whose views on
education are remarkably strict, has brought me up to be extremely
short-sighted; it is part of her system; so do you mind my looking
at you through my glasses?

\5  Oh! not at all, Gwendolen.  I am very fond of being looked
at.

\4  \[After examining \5 carefully through a lorgnette\]\break
You are here on a short visit, I suppose.

\5  Oh no!  I live here.

\4  \[severely\]  Really?  Your mother, no doubt, or some
female relative of advanced years, resides here also?

\5  Oh no!  I have no mother, nor, in fact, any relations.

\4  Indeed?

\5  My dear guardian, with the assistance of Miss Prism, has
the arduous task of looking after me.

\4  Your guardian?

\5  Yes, I am Mr. Worthing's ward.

\4  Oh!  It is strange he never mentioned to me that he had
a ward.  How secretive of him!  He grows more interesting hourly.
I am not sure, however, that the news inspires me with feelings of
unmixed delight.  \[Rising and going to her\]  I am very fond of
you, Cecily; I have liked you ever since I met you!  But I am bound
to state that now that I know that you are Mr. Worthing's ward, I
cannot help expressing a wish you were \textemdash well, just a little older
than you seem to be \textemdash and not quite so very alluring in appearance.
In fact, if I may speak candidly \textemdash

\5  Pray do!  I think that whenever one has anything
unpleasant to say, one should always be quite candid.

\4  Well, to speak with perfect candour, Cecily, I wish
that you were fully forty-two, and more than usually plain for your
age.  Ernest has a strong upright nature.  He is the very soul of
truth and honour.  Disloyalty would be as impossible to him as
deception.  But even men of the noblest possible moral character
are extremely susceptible to the influence of the physical charms
of others.  Modern, no less than Ancient History, supplies us with
many most painful examples of what I refer to.  If it were not so,
indeed, History would be quite unreadable.

\5  I beg your pardon, Gwendolen, did you say Ernest?

\4  Yes.

\5  Oh, but it is not Mr. Ernest Worthing who is my guardian.
It is his brother \textemdash his elder brother.

\4  \[sitting down again\]  Ernest never mentioned to me
that he had a brother.

\5  I am sorry to say they have not been on good terms for a
long time.

\4  Ah! that accounts for it.  And now that I think of it I
have never heard any man mention his brother.  The subject seems
distasteful to most men.  Cecily, you have lifted a load from my
mind.  I was growing almost anxious.  It would have been terrible
if any cloud had come across a friendship like ours, would it not?
Of course you are quite, quite sure that it is not Mr. Ernest
Worthing who is your guardian?

\5  Quite sure.  \[A pause\]  In fact, I am going to be his.

\4  \[inquiringly\]  I beg your pardon?

\5  \[rather shy and confidingly\]  Dearest Gwendolen, there is
no reason why I should make a secret of it to you.  Our little
county newspaper is sure to chronicle the fact next week.  Mr.
Ernest Worthing and I are engaged to be married.

\4  \[quite politely, rising\]  My darling Cecily, I think
there must be some slight error.  Mr. Ernest Worthing is engaged to
me.  The announcement will appear in the \textit{Morning Post} on Saturday
at the latest.

\5  \[very politely, rising\]  I am afraid you must be under
some misconception.  Ernest proposed to me exactly ten minutes ago.
%\[Shows diary\]

\4  \[examines diary through her lorgnettte carefully\]  It
is certainly very curious, for he asked me to be his wife yesterday
afternoon at 5.30.  If you would care to verify the incident, pray
do so.  \[Produces diary of her own\]  I never travel without my
diary.  One should always have something sensational to read in the
train.  I am so sorry, dear Cecily, if it is any disappointment to
you, but I am afraid I have the prior claim.

\5  It would distress me more than I can tell you, dear
Gwendolen, if it caused you any mental or physical anguish, but I
feel bound to point out that since Ernest proposed to you he
clearly has changed his mind.

\4  \[meditatively\]  If the poor fellow has been entrapped
into any foolish promise I shall consider it my duty to rescue him
at once, and with a firm hand.

\5  \[thoughtfully and sadly\]  Whatever unfortunate
entanglement my dear boy may have got into, I will never reproach
him with it after we are married.

\4  Do you allude to me, Miss Cardew, as an entanglement?
You are presumptuous.  On an occasion of this kind it becomes more
than a moral duty to speak one's mind.  It becomes a pleasure.

\5  Do you suggest, Miss Fairfax, that I entrapped Ernest into
an engagement?  How dare you?  This is no time for wearing the
shallow mask of manners.  When I see a spade I call it a spade.

\4  \[satirically\]  I am glad to say that I have never seen
a spade.  It is obvious that our social spheres have been widely
different.

\[Enter \8, followed by the footman.  He carries a salver,
table cloth, and plate stand.  \5 is about to retort.  The
presence of the servants exercises a restraining influence, under
which both girls chafe.\]

\8  Shall I lay tea here as usual, Miss?

\5  \[sternly, in a calm voice\]  Yes, as usual.

\[\8 begins to clear table and lay cloth.  A long pause.  \5 and
\4 glare at each other\]


\4  \DriveOut* Are there many interesting walks in the vicinity, Miss
Cardew? 

\5  Oh! yes! a great many.  From the top of one of the hills
quite close one can see five counties.

\4  Five counties!  I don't think I should like that; I
hate crowds.

\5  \[sweetly\]  I suppose that is why you live in town?
\[\4 bites her lip, and beats her foot nervously with her
parasol\]

\4  \[looking round\]  Quite a well-kept garden this is,
Miss Cardew.

\5  So glad you like it, Miss Fairfax.

\4  I had no idea there were any flowers in the country.

\5  Oh, flowers are as common here, Miss Fairfax, as people
are in London.

\4  Personally I cannot understand how anybody manages to
exist in the country, if anybody who is anybody does.  The country
always bores me to death.

\5  Ah!  This is what the newspapers call agricultural
depression, is it not?  I believe the aristocracy are suffering
very much from it just at present.  It is almost an epidemic
amongst them, I have been told.  May I offer you some tea, Miss
Fairfax?

\4\[with elaborate politeness\] Thank you. \[Aside\] Detestable girl!
But I require tea!

\5  \[sweetly\]  Sugar?

\4  \[superciliously\]  No, thank you.  Sugar is not
fashionable any more. \[\5 looks angrily at her, takes up the
tongs and puts four lumps of sugar into the cup\]

\5  \[severely\]  Cake or bread and butter?

\4  \[in a bored manner\]  Bread and butter, please.  Cake
is rarely seen at the best houses nowadays.

\5  \[cuts a very large slice of cake, and puts it on the
tray\]  Hand that to Miss Fairfax.

\[\8 does so, and goes out with footman.  \4 drinks the
tea and makes a grimace. Puts down cup at once, reaches out her
hand to the bread and butter, looks at it, and finds it is cake.
Rises in indignation.\]

\4  You have filled my tea with lumps of sugar, and though
I asked most distinctly for bread and butter, you have given me
cake.  I am known for the gentleness of my disposition, and the
extraordinary sweetness of my nature, but I warn you, Miss Cardew,
you may go too far.

\5  \[rising\]  To save my poor, innocent, trusting boy from
the machinations of any other girl there are no lengths to which I
would not go.

\4  From the moment I saw you I distrusted you.  I felt
that you were false and deceitful.  I am never deceived in such
matters.  My first impressions of people are invariably right.

\5  It seems to me, Miss Fairfax, that I am trespassing on
your valuable time.  No doubt you have many other calls of a
similar character to make in the neighbourhood.

\[c]Enter \1.\]

\4  \[catching sight of him\]  Ernest!  My own Ernest!

\1  Gwendolen!  Darling!  \[Offers to kiss her\]

\4  \[draws back\]  A moment!  May I ask if you are engaged
to be married to this young lady? \[Points to \5\]

\1  \[laughing\]  To dear little Cecily!  Of course not!  What
could have put such an idea into your pretty little head?

\4  Thank you.  You may!  \[Offers her cheek\]

\5  \[very sweetly\]  I knew there must be some
misunderstanding, Miss Fairfax.  The gentleman whose arm is at
present round your waist is my guardian, Mr. John Worthing.

\4  I beg your pardon?

\5  This is Uncle Jack.

\4  \[receding\] Jack!  Oh!

\[c]Enter \2.\]

\5  Here is Ernest.

\2  \[goes straight over to \5 without noticing any one
else\] 

My own love!  \[Offers to kiss her\]

\5  \[drawing back\]  A moment, Ernest!  May I ask you \textemdash are
you engaged to be married to this young lady?

\2  \[looking round\]  To what young lady?  Good heavens!\break
Gwendolen!

\5  Yes! to good heavens, Gwendolen, I mean to Gwendolen.

\2  \[laughing\]  Of course not!  What could have put such an
idea into your pretty little head?

\5  Thank you.  \[Presenting her cheek to be kissed\]  You may.\hfil\break
\[\2 kisses her\]

\4  I felt there was some slight error, Miss Cardew.  The
gentleman who is now embracing you is my cousin, Mr. Algernon
Moncrieff.

\5  \[breaking away from \2\]  Algernon Moncrieff!  Oh!
\[The two girls move towards each other and put their arms round
each other's waists as if for protection\]

\5  Are you called Algernon?

\2  I cannot deny it.

\5  Oh!

\4  Is your name really John?

\1  \[standing rather proudly\]  I could deny it if I liked.  I
could deny anything if I liked.  But my name certainly is John.  It
has been John for years.

\5  \[to \4\]  A gross deception has been practised on
both of us.

\4  My poor wounded Cecily!

\5  My sweet wronged Gwendolen!

\4  \[slowly and seriously\]  You will call me sister, will
you not? \[They embrace.  \1 and \2 groan and walk up and down\]

\5  \[rather brightly\]  There is just one question I would
like to be allowed to ask my guardian.

\4  An admirable idea!  Mr. Worthing, there is just one
question I would like to be permitted to put to you.  Where is your
brother Ernest?  We are both engaged to be married to your brother
Ernest, so it is a matter of some importance to us to know where
your brother Ernest is at present.

\1  \[slowly and hesitatingly\]  Gwendolen \textemdash Cecily
\textemdash it is very
painful for me to be forced to speak the truth.  It is the first
time in my life that I have ever been reduced to such a painful
position, and I am really quite inexperienced in doing anything of
the kind.  However, I will tell you quite frankly that I have no
brother Ernest.  I have no brother at all.  I never had a brother
in my life, and I certainly have not the smallest intention of ever
having one in the future.

\5  \[surprised\]  No brother at all?

\1  \[cheerily\]  None!

\4  \[severely\]  Had you never a brother of any kind?

\1  \[pleasantly\]  Never.  Not even of an kind.

\4  I am afraid it is quite clear, Cecily, that neither of
us is engaged to be married to any one.

\5  It is not a very pleasant position for a young girl
suddenly to find herself in.  Is it?

\4  Let us go into the house.  They will hardly venture to
come after us there.

\5  No, men are so cowardly, aren't they?

\[c]They retire into the house with scornful looks.\]

\1  This ghastly state of things is what you call Bunburying, I
suppose?

\2  Yes, and a perfectly wonderful Bunbury it is.  The most
wonderful Bunbury I have ever had in my life.

\1  Well, you've no right whatsoever to Bunbury here.

\2  That is absurd.  One has a right to Bunbury anywhere one
chooses.  Every serious Bunburyist knows that.

\1  Serious Bunburyist!  Good heavens!

\2  Well, one must be serious about something, if one wants
to have any amusement in life.  I happen to be serious about
Bunburying.  What on earth you are serious about I haven't got the
remotest idea.  About everything, I should fancy.  You have such an
absolutely trivial nature.

\1  Well, the only small satisfaction I have in the whole of
this wretched business is that your friend Bunbury is quite
exploded.  You won't be able to run down to the country quite so
often as you used to do, dear Algy.  And a very good thing too.

\2  Your brother is a little off colour, isn't he, dear
Jack?  You won't be able to disappear to London quite so frequently
as your wicked custom was.  And not a bad thing either.

\1  As for your conduct towards Miss Cardew, I must say that
your taking in a sweet, simple, innocent girl like that is quite
inexcusable.  To say nothing of the fact that she is my ward.

\2  I can see no possible defence at all for your deceiving
a brilliant, clever, thoroughly experienced young lady like Miss
Fairfax.  To say nothing of the fact that she is my cousin.

\1  I wanted to be engaged to Gwendolen, that is all.  I love
her.

\2  Well, I simply wanted to be engaged to Cecily.  I adore
her.

\1  There is certainly no chance of your marrying Miss Cardew.

\2  I don't think there is much likelihood, Jack, of you and
Miss Fairfax being united.

\1  Well, that is no business of yours.

\2  If it was my business, I wouldn't talk about it.
\[Begins to eat muffins\]  It is very vulgar to talk about one's
business.  Only people like stock-brokers do that, and then merely
at dinner parties.

\1  How can you sit there, calmly eating muffins when we are in
this horrible trouble, I can't make out.  You seem to me to be
perfectly heartless.

\2  Well, I can't eat muffins in an agitated manner.  The
butter would probably get on my cuffs.  One should always eat
muffins quite calmly.  It is the only way to eat them.

\1  I say it's perfectly heartless your eating muffins at all,
under the circumstances.

\2  When I am in trouble, eating is the only thing that
consoles me.  Indeed, when I am in really great trouble, as any one
who knows me intimately will tell you, I refuse everything except
food and drink.  At the present moment I am eating muffins because
I am unhappy.  Besides, I am particularly fond of muffins.
\[r]Rising\]

\1  \[rising\]  Well, that is no reason why you should eat them
all in that greedy way. \[Takes muffins from \2\]

\2  \[offering tea-cake\]  I wish you would have tea-cake
instead.  I don't like tea-cake.

\1  Good heavens!  I suppose a man may eat his own muffins in
his own garden.

\2  But you have just said it was perfectly heartless to eat
muffins.

\1  I said it was perfectly heartless of you, under the
circumstances.  That is a very different thing.

\2  That may be.  But the muffins are the same.  \[He seizes
the muffin-dish from \1\]

\1  Algy, I wish to goodness you would go.

\2  You can't possibly ask me to go without having some
dinner.  It's absurd.  I never go without my dinner.  No one ever
does, except vegetarians and people like that.  Besides I have just
made arrangements with Dr. Chasuble to be christened at a quarter
to six under the name of Ernest.

\1  My dear fellow, the sooner you give up that nonsense the
better.  I made arrangements this morning with Dr. Chasuble to be
christened myself at 5.30, and I naturally will take the name of
Ernest.  Gwendolen would wish it.  We can't both be christened
Ernest.  It's absurd.  Besides, I have a perfect right to be
christened if I like.  There is no evidence at all that I have ever
been christened by anybody.  I should think it extremely probable I
never was, and so does Dr. Chasuble.  It is entirely different in
your case.  You have been christened already.

\2  Yes, but I have not been christened for years.

\1  Yes, but you have been christened.  That is the important
thing.

\2  Quite so.  So I know my constitution can stand it.  If
you are not quite sure about your ever having been christened, I
must say I think it rather dangerous your venturing on it now.  It
might make you very unwell.  You can hardly have forgotten that
some one very closely connected with you was very nearly carried
off this week in Paris by a severe chill.

\1  Yes, but you said yourself that a severe chill was not
hereditary.

\2  It usen't to be, I know \textemdash but I daresay it is now.
Science is always making wonderful improvements in things.

\1  \[picking up the muffin-dish\]  Oh, that is nonsense; you are
always talking nonsense.

\2  Jack, you are at the muffins again!  I wish you
wouldn't.  There are only two left.  \[Takes them\]  I told you I
was particularly fond of muffins.

\1  But I hate tea-cake.

\2  Why on earth then do you allow tea-cake to be served up
for your guests?  What ideas you have of hospitality!

\1  Algernon!  I have already told you to go.  I don't want you
here.  Why don't you go!

\2  I haven't quite finished my tea yet! and there is still
one muffin left.
\[\1 groans, and sinks into a chair. \2 still continues eating.\]

\Act \Scene

\(c)Morning-room at the Manor House.\)

\[\4 and \5 are at the window, looking out into the
garden\]

\4  The fact that they did not follow us at once into the
house, as any one else would have done, seems to me to show that
they have some sense of shame left.

\5  They have been eating muffins.  That looks like
repentance.

\4  \[after a pause\]  They don't seem to notice us at all.
Couldn't you cough?

\5  But I haven't got a cough.

\4  They're looking at us.  What effrontery!

\5  They're approaching.  That's very forward of them.

\4  Let us preserve a dignified silence.

\5  Certainly.  It's the only thing to do now.

\[c]Enter \1 followed by \2  They whistle some dreadful\\ popular air from
a British Opera\]

\4  This dignified silence seems to produce an unpleasant
effect.

\5  A most distasteful one.

\4  But we will not be the first to speak.

\5  Certainly not.

\4  Mr. Worthing, I have something very particular to ask
you.  Much depends on your reply.

\5  Gwendolen, your common sense is invaluable.  Mr.
Moncrieff, kindly answer me the following question.  Why did you
pretend to be my guardian's brother?

\2  In order that I might have an opportunity of meeting
you.

\5  \[to \4\]  That certainly seems a satisfactory
explanation, does it not?

\4  Yes, dear, if you can believe him.

\5  I don't.  But that does not affect the wonderful beauty of
his answer.

\4  True.  In matters of grave importance, style, not
sincerity is the vital thing.  Mr. Worthing, what explanation can
you offer to me for pretending to have a brother?  Was it in order
that you might have an opportunity of coming up to town to see me
as often as possible?

\1  Can you doubt it, Miss Fairfax?

\4  I have the gravest doubts upon the subject.  But I
intend to crush them.  This is not the moment for German
scepticism.  \[Moving to \5\]  Their explanations appear to be
quite satisfactory, especially Mr. Worthing's.  That seems to me to
have the stamp of truth upon it.

\5  I am more than content with what Mr. Moncrieff said.  His
voice alone inspires one with absolute credulity.

\4  Then you think we should forgive them?

\5  Yes.  I mean no.

\4  True!  I had forgotten.  There are principles at stake
that one cannot surrender.  Which of us should tell them?  The task
is not a pleasant one.

\5  Could we not both speak at the same time?

\4  An excellent idea!  I nearly always speak at the same
time as other people.  Will you take the time from me?

\5  Certainly.  \[\4 beats time with uplifted finger\]
\personae {Gwendolen \textit{and} Cecily} \[Speaking together\]  Your Christian names are
still an insuperable barrier.  That is all!

\personae {Jack \textit{and} Algernon}\[Speaking together\]  Our Christian names!  Is
that all?  But we are going to be christened this afternoon.

\4  \[to \1\]  For my sake you are prepared to do this
terrible thing?

\1  I am.

\5  \[to \2\]  To please me you are ready to face this
fearful ordeal?

\2  I am!

\4  How absurd to talk of the equality of the sexes!  Where
questions of self-sacrifice are concerned, men are infinitely
beyond us.

\1  We are.  \[Clasps hands with \2\]

\5  They have moments of physical courage of which we women
know absolutely nothing.

\4  \[to \1\]  Darling!

\2  \[to \5\]  Darling!  \[They fall into each other's
arms\]

\[c]Enter \8  When he enters he coughs loudly,\\ seeing the
situation\]

\8  Ahem!  Ahem!  Lady Bracknell!

\1  Good heavens!

\[c]Enter \3  The couples separate in alarm.\\ Exit \8\]

\3  Gwendolen!  What does this mean?

\4  Merely that I am engaged to be married to Mr. Worthing,
mamma.

\3  Come here.  Sit down.  Sit down immediately.
Hesitation of any kind is a sign of mental decay in the young, of
physical weakness in the old.  \[Turns to \1\]  Apprised, sir, of
my daughter's sudden flight by her trusty maid, whose confidence I
purchased by means of a small coin, I followed her at once by a
luggage train.  Her unhappy father is, I am glad to say, under the
impression that she is attending a more than usually lengthy
lecture by the University Extension Scheme on the Influence of a
permanent income on Thought.  I do not propose to undeceive him.
Indeed I have never undeceived him on any question.  I would
consider it wrong.  But of course, you will clearly understand that
all communication between yourself and my daughter must cease
immediately from this moment.  On this point, as indeed on all
points, I am firm.

\1  I am engaged to be married to Gwendolen, Lady Bracknell!

\3  You are nothing of the kind, sir.  And now, as
regards Algernon! . . . Algernon!

\2  Yes, Aunt Augusta.

\3  May I ask if it is in this house that your invalid
friend Mr. Bunbury resides?

\2  \[stammering\]  Oh!  No!  Bunbury doesn't live here.
Bunbury is somewhere else at present.  In fact, Bunbury is dead,

\3  Dead!  When did Mr. Bunbury die?  His death must
have been extremely sudden.

\2  \[airily\]  Oh!  I killed Bunbury this afternoon.  I mean
poor Bunbury died this afternoon.

\3  What did he die of?

\2  Bunbury?  Oh, he was quite exploded.

\3  Exploded!  Was he the victim of a revolutionary
outrage?  I was not aware that Mr. Bunbury was interested in social
legislation.  If so, he is well punished for his morbidity.

\2  My dear Aunt Augusta, I mean he was found out!  The
doctors found out that Bunbury could not live, that is what I mean
\textemdash so Bunbury died.

\3  He seems to have had great confidence in the
opinion of his physicians.  I am glad, however, that he made up his
mind at the last to some definite course of action, and acted under
proper medical advice.  And now that we have finally got rid of
this Mr. Bunbury, may I ask, Mr. Worthing, who is that young person
whose hand my nephew Algernon is now holding in what seems to me a
peculiarly unnecessary manner?

\1  That lady is Miss Cecily Cardew, my ward.  \[\3
bows coldly to \5\]

\2  I am engaged to be married to Cecily, Aunt Augusta.

\3  I beg your pardon?

\5  Mr. Moncrieff and I are engaged to be married, Lady
Bracknell.

\3  \[with a shiver, crossing to the sofa and sitting
down\]  I do not know whether there is anything peculiarly exciting
in the air of this particular part of Hertfordshire, but the number
of engagements that go on seems to me considerably above the proper
average that statistics have laid down for our guidance.  I think
some preliminary inquiry on my part would not be out of place.  Mr.
Worthing, is Miss Cardew at all connected with any of the larger
railway stations in London?  I merely desire information.  Until
yesterday I had no idea that there were any families or persons
whose origin was a Terminus.  \[\1 looks perfectly furious, but
restrains himself\]

\1  \[in a clear, cold voice\]  Miss Cardew is the grand-daughter
of the late Mr. Thomas Cardew of 149 Belgrave Square, S.W.; Gervase
Park, Dorking, Surrey; and the Sporran, Fifeshire, N.B.

\3  That sounds not unsatisfactory.  Three addresses
always inspire confidence, even in tradesmen.  But what proof have
I of their authenticity?

\1  I have carefully preserved the Court Guides of the period.
They are open to your inspection, Lady Bracknell.

\3  \[grimly\]  I have known strange errors in that
publication.

\1  Miss Cardew's family solicitors are Messrs. Markby, Markby,
and Markby.

\3  Markby, Markby, and Markby?  A firm of the very
highest position in their profession.  Indeed I am told that one of
the Mr. Markby's is occasionally to be seen at dinner parties.  So
far I am satisfied.

\1  \[very irritably\]  How extremely kind of you, Lady
Bracknell!  I have also in my possession, you will be pleased to
hear, certificates of Miss Cardew's birth, baptism, whooping cough,
registration, vaccination, confirmation, and the measles; both the
German and the English variety.

\3  Ah! A life crowded with incident, I see; though
perhaps somewhat too exciting for a young girl.  I am not myself in
favour of premature experiences.  \[Rises, looks at her watch\]
Gwendolen! the time approaches for our departure.  We have not a
moment to lose.  As a matter of form, Mr. Worthing, I had better
ask you if Miss Cardew has any little fortune?

\1  Oh! about a hundred and thirty thousand pounds in the Funds.
That is all.  Goodbye, Lady Bracknell.  So pleased to have seen
you.

\3  \[sitting down again\]  A moment, Mr. Worthing.  A
hundred and thirty thousand pounds!  And in the Funds!  Miss Cardew
seems to me a most attractive young lady, now that I look at her.
Few girls of the present day have any really solid qualities, any
of the qualities that last, and improve with time.  We live, I
regret to say, in an age of surfaces.  \[To \5\]  Come over
here, dear.  \[\5 goes across\]  Pretty child! your dress is
sadly simple, and your hair seems almost as Nature might have left
it.  But we can soon alter all that.  A thoroughly experienced
French maid produces a really marvellous result in a very brief
space of time.  I remember recommending one to young Lady Lancing,
and after three months her own husband did not know her.

\1  And after six months nobody knew her.

\3  \[glares at \1 for a few moments.  Then bends,
with a practised smile, to \5\]  Kindly turn round, sweet
child.  \[\5 turns completely round\]  No, the side view is what
I want.  \[\5 presents her profile\]  Yes, quite as I expected.
There are distinct social possibilities in your profile.  The two
weak points in our age are its want of principle and its want of
profile.  The chin a little higher, dear.  Style largely depends on
the way the chin is worn.  They are worn very high, just at
present.  Algernon!

\2  Yes, Aunt Augusta!

\3  There are distinct social possibilities in Miss
Cardew's profile.

\2  Cecily is the sweetest, dearest, prettiest girl in the
whole world.  And I don't care twopence about social possibilities.

\3  Never speak disrespectfully of Society, Algernon.
Only people who can't get into it do that.  \[To \5\]  Dear
child, of course you know that Algernon has nothing but his debts
to depend upon.  But I do not approve of mercenary marriages.  When
I married Lord Bracknell I had no fortune of any kind.  But I never
dreamed for a moment of allowing that to stand in my way.  Well, I
suppose I must give my consent.

\2  Thank you, Aunt Augusta.

\3  Cecily, you may kiss me!

\5  \[kisses her\]  Thank you, Lady Bracknell.

\3  You may also address me as Aunt Augusta for the
future.

\5  Thank you, Aunt Augusta.

\3  The marriage, I think, had better take place quite
soon.

\2  Thank you, Aunt Augusta.

\5  Thank you, Aunt Augusta.

\3  To speak frankly, I am not in favour of long
engagements.  They give people the opportunity of finding out each
other's character before marriage, which I think is never
advisable.

\1  I beg your pardon for interrupting you, Lady Bracknell, but
this engagement is quite out of the question.  I am Miss Cardew's
guardian, and she cannot marry without my consent until she comes
of age.  That consent I absolutely decline to give.

\3  Upon what grounds may I ask?  Algernon is an
extremely, I may almost say an ostentatiously, eligible young man.
He has nothing, but he looks everything.  What more can one desire?

\1  It pains me very much to have to speak frankly to you, Lady
Bracknell, about your nephew, but the fact is that I do not approve
at all of his moral character.  I suspect him of being untruthful.
\[\2 and \5 look at him in indignant amazement\]

\3  Untruthful!  My nephew Algernon?  Impossible!  He
is an Oxonian.

\1  I fear there can be no possible doubt about the matter.
This afternoon during my temporary absence in London on an
important question of romance, he obtained admission to my house by
means of the false pretence of being my brother.  Under an assumed
name he drank, I've just been informed by my butler, an entire pint
bottle of my Perrier-Jouet, Brut, '89; wine I was specially
reserving for myself.  Continuing his disgraceful deception, he
succeeded in the course of the afternoon in alienating the
affections of my only ward.  He subsequently stayed to tea, and
devoured every single muffin.  And what makes his conduct all the
more heartless is, that he was perfectly well aware from the first
that I have no brother, that I never had a brother, and that I
don't intend to have a brother, not even of any kind.  I distinctly
told him so myself yesterday afternoon.

\3  Ahem!  Mr. Worthing, after careful consideration I
have decided entirely to overlook my nephew's conduct to you.

\1  That is very generous of you, Lady Bracknell.  My own
decision, however, is unalterable.  I decline to give my consent.

\3  \[to \5\]  Come here, sweet child.  \[\5
goes over\]  How old are you, dear?

\5  Well, I am really only eighteen, but I always admit to
twenty when I go to evening parties.

\3  You are perfectly right in making some slight
alteration.  Indeed, no woman should ever be quite accurate about
her age.  It looks so calculating . . . \[In a meditative manner\]
Eighteen, but admitting to twenty at evening parties.  Well, it
will not be very long before you are of age and free from the
restraints of tutelage.  So I don't think your guardian's consent
is, after all, a matter of any importance.

\1  Pray excuse me, Lady Bracknell, for interrupting you again,
but it is only fair to tell you that according to the terms of her
grandfather's will Miss Cardew does not come legally of age till
she is thirty-five.

\3  That does not seem to me to be a grave objection.
Thirty-five is a very attractive age.  London society is full of
women of the very highest birth who have, of their own free choice,
remained thirty-five for years.  Lady Dumbleton is an instance in
point.  To my own knowledge she has been thirty-five ever since she
arrived at the age of forty, which was many years ago now.  I see
no reason why our dear Cecily should not be even still more
attractive at the age you mention than she is at present.  There
will be a large accumulation of property.

\5  Algy, could you wait for me till I was thirty-five?

\2  Of course I could, Cecily.  You know I could.

\5  Yes, I felt it instinctively, but I couldn't wait all that
time.  I hate waiting even five minutes for anybody.  It always
makes me rather cross.  I am not punctual myself, I know, but I do
like punctuality in others, and waiting, even to be married, is
quite out of the question.

\2  Then what is to be done, Cecily?

\5  I don't know, Mr. Moncrieff.

\3  My dear Mr. Worthing, as Miss Cardew states
positively that she cannot wait till she is thirty-five \textemdash a remark
which I am bound to say seems to me to show a somewhat impatient
nature \textemdash I would beg of you to reconsider your decision.

\1  But my dear Lady Bracknell, the matter is entirely in your
own hands.  The moment you consent to my marriage with Gwendolen, I
will most gladly allow your nephew to form an alliance with my
ward.

\3  \[rising and drawing herself up\]  You must be
quite aware that what you propose is out of the question.

\1  Then a passionate celibacy is all that any of us can look
forward to.

\3  That is not the destiny I propose for Gwendolen.
Algernon, of course, can choose for himself.  \[Pulls out her
watch\]  Come, dear, \[\4 rises\] we have already missed five,
if not six, trains.  To miss any more might expose us to comment on
the platform.

\[c]Enter \7\]

\7  Everything is quite ready for the christenings.

\3  The christenings, sir!  Is not that somewhat
premature?

\7  \[looking rather puzzled, and pointing to \1 and
\2\]  Both these gentlemen have expressed a desire for
immediate baptism.

\3  At their age?  The idea is grotesque and
irreligious!  Algernon, I forbid you to be baptized.  I will not
hear of such excesses.  Lord Bracknell would be highly displeased
if he learned that that was the way in which you wasted your time
and money.

\7  Am I to understand then that there are to he no
christenings at all this afternoon?

\1  I don't think that, as things are now, it would be of much
practical value to either of us, Dr.~Chasuble.

\7  I am grieved to hear such sentiments from you, Mr.
Worthing.  They savour of the heretical views of the Anabaptists,
views that I have completely refuted in four of my unpublished
sermons.  However, as your present mood seems to be one peculiarly
secular, I will return to the church at once.  Indeed, I have just
been informed by the pew-opener that for the last hour and a half
Miss Prism has been waiting for me in the vestry.

\3  \[starting\]  Miss Prism!  Did I bear you mention a
Miss Prism?

\7  Yes, Lady Bracknell.  I am on my way to join her.

\3  Pray allow me to detain you for a moment.  This
matter may prove to be one of vital importance to Lord Bracknell
and myself.  Is this Miss Prism a female of repellent aspect,
remotely connected with education?

\7  \[somewhat indignantly\]  She is the most cultivated of
ladies, and the very picture of respectability.

\3  It is obviously the same person.  May I ask what
position she holds in your household?

\7  \[severely\]  I am a celibate, madam.

\1  \[interposing\]  Miss Prism, Lady Bracknell, has been for the
last three years Miss Cardew's esteemed governess and valued
companion.

\3  In spite of what I hear of her, I must see her at
once.  Let her be sent for.

\7  \[looking off\]  She approaches; she is nigh.

\[c]Enter \6 hurriedly\]

\6  I was told you expected me in the vestry, dear Canon.
I have been waiting for you there for an hour and three-quarters.
\[Catches sight of \3, who has fixed her with a stony
glare.  \6 grows pale and quails.  She looks anxiously
round as if desirous to escape\]

\3  \[in a severe, judicial voice\]  Prism!  \[\6
bows her head in shame\]  Come here, Prism!  \[\6
approaches in a humble manner\]  Prism!  Where is that baby?
\[General consternation.  The Canon starts back in horror.  \2
and \1 pretend to be anxious to shield \5 and \4 from
hearing the details of a terrible public scandal\] Twenty-eight
years ago, Prism, you left Lord Bracknell's house, Number 104,
Upper Grosvenor Street, in charge of a perambulator that contained
a baby of the male sex.  You never returned.  A few weeks later,
through the elaborate investigations of the Metropolitan police,
the perambulator was discovered at midnight, standing by itself in
a remote corner of Bayswater.  It contained the manuscript of a
three-volume novel of more than usually revolting sentimentality.
\[\6 starts in involuntary indignation\]  But the baby was
not there!  \[Every one looks at \6\]  Prism!  Where is that
baby?  \[A pause\]

\6  Lady Bracknell, I admit with shame that I do not know.
I only wish I did.  The plain facts of the case are these.  On the
morning of the day you mention, a day that is for ever branded on
my memory, I prepared as usual to take the baby out in its
perambulator.  I had also with me a somewhat old, but capacious
hand-bag in which I had intended to place the manuscript of a work
of fiction that I had written during my few unoccupied hours.  In a
moment of mental abstraction, for which I never can forgive myself,
I deposited the manuscript in the basinette, and placed the baby in
the hand-bag.

\1  \[who has been listening attentively\]  But where did you
deposit the hand-bag?

\6  Do not ask me, Mr. Worthing.

\1  Miss Prism, this is a matter of no small importance to me.
I insist on knowing where you deposited the hand-bag that contained
that infant.

\6  I left it in the cloak-room of one of the larger
railway stations in London.

\1  What railway station?

\6  \[quite crushed\]  Victoria.  The Brighton line. \[Sinks into a chair\]

\1  I must retire to my room for a moment.  Gwendolen, wait here
for me.

\4  If you are not too long, I will wait here for you all
my life. 

\[c]Exit \1 in great excitement\]

\7  What do you think this means, Lady Bracknell?

\3  I dare not even suspect, Dr.~Chasuble.  I need
hardly tell you that in families of high position strange
coincidences are not supposed to occur.  They are hardly considered
the thing.

\[c]Noises heard overhead as if some one was throwing trunks about.\\
Every one looks up\]

\5  Uncle Jack seems strangely agitated.

\7  Your guardian has a very emotional nature.

\3  This noise is extremely unpleasant.  It sounds as
if he was having an argument.  I dislike arguments of any kind.
They are always vulgar, and often convincing.

\7  \[looking up\]  It has stopped now.  \[The noise is
redoubled\]

\3  I wish he would arrive at some conclusion.

\4  This suspense is terrible.  I hope it will last.

\[c]Enter \1 with a hand-bag of black leather in his hand\]

\1  \[rushing over to \6\]  Is this the handbag, Miss
Prism?  Examine it carefully before you speak.  The happiness of
more than one life depends on your answer.

\6  \[calmly\]  It seems to be mine.  Yes, here is the
injury it received through the upsetting of a Gower Street omnibus
in younger and happier days.  Here is the stain on the lining
caused by the explosion of a temperance beverage, an incident that
occurred at Leamington.  And here, on the lock, are my initials.  I
had forgotten that in an extravagant mood I had had them placed
there.  The bag is undoubtedly mine.  I am delighted to have it so
unexpectedly restored to me.  It has been a great inconvenience
being without it all these years.

\1  \[in a pathetic voice\]  Miss Prism, more is restored to you
than this hand-bag.  I was the baby you placed in it.

\6  \[amazed\]  You?

\1  \[embracing her\]  Yes . . . mother!

\6  \[recoiling in indignant astonishment\]  Mr. Worthing!
I am unmarried

\1  Unmarried!  I do not deny that is a serious blow.  But after
all, who has the right to cast a stone against one who has
suffered?  Cannot repentance wipe out an act of folly?  Why should
there be one law for men, and another for women?  Mother, I forgive
you.  \[Tries to embrace her again\]

\6  \[still more indignant\]  Mr. Worthing, there is some
error.  \[Pointing to \3\]  There is the lady who can
tell you who you really are.

\1  \[after a pause\]  Lady Bracknell, I hate to seem
inquisitive, but would you kindly inform me who I am?

\3  I am afraid that the news I have to give you will
not altogether please you.  You are the son of my poor sister, Mrs.
Moncrieff, and consequently Algernon's elder brother.

\1  Algy's elder brother!  Then I have a brother after all.  I
knew I had a brother!  I always said I had a brother!  Cecily,
\textemdash
how could you have ever doubted that I had a brother?  \[Seizes hold
of \2\]  Dr.~Chasuble, my unfortunate brother.  Miss Prism,
my unfortunate brother.  Gwendolen, my unfortunate brother.  Algy,
you young scoundrel, you will have to treat me with more respect in
the future.  You have never behaved to me like a brother in all
your life.

\2  Well, not till to-day, old boy, I admit.  I did my best,
however, though I was out of practice.

\[c]Shakes hands\]

\4  \[to \1\]  My own!  But what own are you?  What is
your Christian name, now that you have become some one else?

\1  Good heavens! . . . I had quite forgotten that point.  Your
decision on the subject of my name is irrevocable, I suppose?

\4  I never change, except in my affections.

\5  What a noble nature you have, Gwendolen!

\1  Then the question had better be cleared up at once.  Aunt
Augusta, a moment.  At the time when Miss Prism left me in the
hand-bag, had I been christened already?

\3 {\hbadness=1500 
 Every luxury that money could buy, including \break
christening, had been lavished on you by your fond and doting
parents.}

\1  Then I was christened!  That is settled.  Now, what name was
I given?  Let me know the worst.

\3  Being the eldest son you were naturally christened
after your father.

\1  \[irritably\]  Yes, but what was my father's Christian name?

\3  \[meditatively\]  I cannot at the present moment
recall what the General's Christian name was.  But I have no doubt
he had one.  He was eccentric, I admit.  But only in later years.
And that was the result of the Indian climate, and marriage, and
indigestion, and other things of that kind.

\1  Algy!  Can't you recollect what our father's Christian name
was?

\2  My dear boy, we were never even on speaking terms.  He
died before I was a year old.

\1  His name would appear in the Army Lists of the period, I
suppose, Aunt Augusta?

\3  The General was essentially a man of peace, except
in his domestic life.  But I have no doubt his name would appear in
any military directory.

\1  The Army Lists of the last forty years are here.  These
delightful records should have been my constant study.  \[Rushes to
bookcase and tears the books out\]  M. Generals . . . Mallam,
Maxbohm, Magley, what ghastly names they have \textemdash Markby, Migsby,
Mobbs, Moncrieff!  Lieutenant 1840, Captain, Lieutenant-Colonel,
Colonel, General 1869, Christian names, Ernest John.  \[Puts book
very quietly down and speaks quite calmly\]  I always told you,
Gwendolen, my name was Ernest, didn't I?  Well, it is Ernest after
all.  I mean it naturally is Ernest.

\3  Yes, I remember now that the General was called
Ernest, I knew I had some particular reason for disliking the name.

\4  Ernest!  My own Ernest!  I felt from the first that you
could have no other name!

\1  Gwendolen, it is a terrible thing for a man to find out
suddenly that all his life he has been speaking nothing but the
truth.  Can you forgive me?

\4  I can.  For I feel that you are sure to change.

\1  My own one!

\7  \[to \6\]  Laetitia!  \[Embraces her\]

\6  \[enthusiastically\]  Frederick!  At last!

\2  Cecily!  \[Embraces her\]  At last!

\1  Gwendolen!  \[Embraces her\]  At last!

\3  My nephew, you seem to be displaying signs of
triviality.

\1  On the contrary, Aunt Augusta, I've now realised for the
first time in my life the vital Importance of Being Earnest.

\spatium {2\leading}
\titulus {\scshape tableau\\[1\leading] curtain}
\endDrama

\end{document}
