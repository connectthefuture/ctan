\RequirePackage{snapshot}
\documentclass[a4paper,KOMA,landscape]{powersem}
%\usepackage{ifmslide}
\usepackage[display,stmo,button]{ifmslide}
%% user definitions
\newcommand{\ifmslide}{{\code{ifmslide.sty}}}
\newcommand{\tp}{{\code{texpower{}} \cite{texpower} }}
\newcommand{\hf}{{\code{hyperref{}} \cite{hyperref} }}

\hypersetup{pdfauthor={Thomas Emmel}}
\hypersetup{pdftitle={ifmslide manual}}
\hypersetup{pdfsubject={examples and commands}}

\IfFileExists{cmtt.sty}{\usepackage[override]{cmtt}%
                        \newcommand{\bs}{{\mtt\\}}}{%
                        \newcommand{\bs}{{$\setminus$}}}%

%%%%%%%%%%%%%%%%%%%%%%%%%%%%%%%%%%%%%%%%%%%%%%%%%%%%%%%%%%%%%%%%%%%%
%\usepackage{thumbpdf}
%%%%%%%%%%%%%%%%%%%%%%%%%%%%%%%%%%%%%%%%%%%%%%%%%%%%%%%%%%%%%%%%%%%%
\begin{document}
\pageTransitionDissolve
\sffamily

\background{ifmlogoc}

\orgname{TU Darmstadt -- Institute of Mechanics -- AG4\footnote{
This is long ago and I am no member of the institute anymore, see further notes
...}}

\title{\begin{minipage}[t]{0.98\textwidth}\begin{center}
      {\mdseries \ifmslide{} v0.47, 2010}\\[1ex]
      FINAL RELEASE!\\[1ex]
      Enhanced presentations with (PDF)\LaTeX{}\\
      combining the texpower, hyperref and seminar-packages\\
    \end{center}\end{minipage}}

\author{\scalebox{1}[1.3]{Thomas Emmel}}

\address{\href{mailto:thomas@family-emmel.de}%
  {thomas@family-emmel.de}}
\orgurl{http://www.mechanik.tu-darmstadt.de/}
\slidepagestyle{panel}
\paneltile[on](0,160,200,200){aqua_ravines}
%%%%%%%%%%%%%%%%%%%%%%%%%%%%%%%%%%%%%%%%%%%%%%%%%%%%%%%%%%%%%%%%%%%%
\begin{slide}
  \maketitle
\end{slide}
%%%%%%%%%%%%%%%%%%%%%%%%%%%%%%%%%%%%%%%%%%%%%%%%%%%%%%%%%%%%%%%%%%%%
\paneltile[off](,,,){}
\hidebackground
\pageTransitionReplace
\centerslidesfalse
\begin{slide}
  \section{Note: This is the end!}
  Yes, it (\ifmslide{}) has ended and it was - hopefully - a very 
  helpful package for all of you. I decided to stop this since the 
  \code{beamer}-class is now much better than it was when I started this
  package. Indeed, the beamer-class hasn't really existed at this time.
  Today, it provides more features, more extensions, although it is a bit
  more complex to learn.\\[2ex]

  Time has gone and even the insitute doesn't exist anymore! Yes, that's sad 
  but true and has to do a lot with politics, although you can still study
  mechanics in Darmstadt, have a look at the web-page.\\[3ex]


  
  Thank you all for the feedback, I will still maintain \code{ticket.sty}... 
\end{slide}
\begin{slide}
  \section{About ifmslide.sty}
% testing margins
%\begin{minipage}[t]{.9\textwidth}
  Early in May 2000 ...\\[1ex]
%\end{minipage}
%\begin{minipage}[t]{.1\textwidth}
%note
%\end{minipage}
  OK! Cancel this annoying story about my problems generating
  printouts for my presentations.\\
  {\color{section1} What can \ifmslide{} do for you?}
  \begin{center}
    You like to do nice presentations of your business with beamers or on a 
    screen?\\
    You are not sure that all these things work when you need them?\\
    You feel better with a stack of printed slides in your bag?
  \end{center}
  \ifmslide{} provides both: getting a presentation with PDF\LaTeX{} and
  printouts with \LaTeX{} and - as a side effect - simple production of
  your slides using your favorite DVI-viewer.\footnote{\tiny This is
    {\color{red} not} \code{pdfslide} \cite{pdfslide} as it seems to be in a
    first view.  It was indeed developed from that style but now
    completely rewritten with many new features implemented and
    functionally extended. \code{pdfslide} cannot produce printouts without
    generating a PDF-file and rescaling the output with all
    difficulties.}  
  You don't need \code{PPower4} to get all these nice effects
  with page-transitions and stepwise building of the pages. All you need
  is \tp and \hf for the links and buttons etc.  \ifmslide{} makes use of
  the special features of the classes \code{seminar} and \code{powersem} 
  (part of \TeX{}Power).
\end{slide}
%\setInsideMargin{16mm}{26mm}{14mm}{16mm}
%%%%%%%%%%%%%%%%%%%%%%%%%%%%%%%%%%%%%%%%%%%%%%%%%%%%%%%%%%%%%%%%%%%%
\definecolor{background}{gray}{1.}
\begin{slide}
  \section{Features}
  \stepwise{
    \begin{itemize}
    \item[...]  produces DVI (\LaTeX{}) for printouts and PDF
      (PDF\LaTeX{}) for direct presentation.  
      \step{\item[...]
        DVI-Version with extra margins for the printer.}
      \step{\item[...] draft-mode with simple frames instead of
        colored boxes (easier to display and debug your slides with
        your DVI-viewer).}  
      \step{\item[...] direct use of bookmarks
        to navigate in the PDF-version.}  
      \step{\item[...] local or
        global configuration file { ifmslide.cfg} for colors and many
        features.}  
      \step{\item[...] panel position is free: right,
        left, bottom, top and outside of the slide! You
        can change it in the document as you like.}  
      \step{\item[...]
        position of the buttons and the logo is completely free!}
      \step{\item[...] changing the size of your slide...}
      \step{\item[...] free choice of button-design, background for
        the panel and the frame... }  
      \step{\item[...] you can use
        most of the features of the seminar-package: magnification
        etc.}
    \end{itemize}
    }
\end{slide}
%%%%%%%%%%%%%%%%%%%%%%%%%%%%%%%%%%%%%%%%%%%%%%%%%%%%%%%%%%%%%%%%%%%%
\begin{slide}
  \section{Options}
  The following options are provided:
  \stepwise{
    \step{{\color{red}\bfseries draft\\}%
      shows colored boxes as white boxes with frames, efficient for
      debugging and writing the slides. The { pause}-command is
      displayed as a small orange box.\\[2ex]}
    \step{{\color{red}\bfseries display\\}%
      \TeX{}Power-option: all effects are turned off if not set!\\[2ex]}
    \step{{\color{red}\bfseries ams, cnav, cnavo, stmo\\}%
      turn the navigation buttons on and sets the style. Just try it...\\[2ex]}
    \step{{\color{red}\bfseries button\\}%
      use definable buttons together with the last option instead of
      just putting a box around the signs.}  }  \vfill
  \footnotetext{\tiny {\bfseries colorinfo} from older versions will
    be ignored, {\bfseries contnav} and {\bfseries amsnav} are changed
    to {\bfseries cnav} and {\bfseries ams}}
\end{slide}
%%%%%%%%%%%%%%%%%%%%%%%%%%%%%%%%%%%%%%%%%%%%%%%%%%%%%%%%%%%%%%%%%%%%
\slidepagestyle{sidebar}
\begin{slide}
  \section{Pagestyles}
  {\color{red}\bfseries sidebar, panel\\}%
  Switch the navigation panel on.\\[1ex]
  {\bfseries plain, myheadings, headings} are redefined to it.\\[2ex]
  You can put the panel at all positions with
  \hyperlink{panelpos}{\code{\bs{}panelposition}} \vfill
  \footnotetext{\tiny pagestyle {\bfseries title} is changed to
    {\bfseries sidebar} and {\bfseries background} is deleted, because
    of the \code{\bs{}background}-command}
\end{slide}
%%%%%%%%%%%%%%%%%%%%%%%%%%%%%%%%%%%%%%%%%%%%%%%%%%%%%%%%%%%%%%%%%%%%
\slidepagestyle{empty}
\begin{slide}
  \section*{Pagestyles}
  {\color{red}\bfseries empty\\}%
  No panel -- only a frame and background.\\[2ex]
  But you can put a panel outside of the frame with
  \hyperlink{panelpos}{\bs{}panelposition} which is not present in
  \LaTeX-mode and outside of the normal paper-size in PDF\LaTeX-mode!
\end{slide}
%%%%%%%%%%%%%%%%%%%%%%%%%%%%%%%%%%%%%%%%%%%%%%%%%%%%%%%%%%%%%%%%%%%%
\slidepagestyle{sidebar}
\begin{slide}
  \section*{Pagestyles}
  \subsection{panelposition}\hypertarget{panelpos}{}
  \code{\bs{}panelposition\{left/right/top/bottom\}} changes the position of the panel to the given value. The slide is recalculated for every  position...\\[2ex]
  \code{\bs{}panelposition\{outsidebottom\}} set a panel outside of the paper!\\[2ex]
  \code{\bs{}panelposition\{empty\}} set no panel which is in practice useful with\\
  \code{\bs{}pagestyle\{empty\}} only.
\end{slide}
%%%%%%%%%%%%%%%%%%%%%%%%%%%%%%%%%%%%%%%%%%%%%%%%%%%%%%%%%%%%%%%%%%%%
\centerslidestrue
\panelposition{top}\begin{slide}Panel on top\end{slide}
%%%%%%%%%%%%%%%%%%%%%%%%%%%%%%%%%%%%%%%%%%%%%%%%%%%%%%%%%%%%%%%%%%%%
\panelposition{left}\begin{slide}Panel left\end{slide}
%%%%%%%%%%%%%%%%%%%%%%%%%%%%%%%%%%%%%%%%%%%%%%%%%%%%%%%%%%%%%%%%%%%%
\panelposition{bottom}\begin{slide}Panel on bottom\end{slide}
%%%%%%%%%%%%%%%%%%%%%%%%%%%%%%%%%%%%%%%%%%%%%%%%%%%%%%%%%%%%%%%%%%%%
\panelposition{right}\begin{slide}Panel right\end{slide}
%%%%%%%%%%%%%%%%%%%%%%%%%%%%%%%%%%%%%%%%%%%%%%%%%%%%%%%%%%%%%%%%%%%%
\slidepagestyle{empty}
\panelposition{outsidebottom}
\begin{slide}Panel outsidebottom (you cannot see it on the printed slides)!
\end{slide}
%%%%%%%%%%%%%%%%%%%%%%%%%%%%%%%%%%%%%%%%%%%%%%%%%%%%%%%%%%%%%%%%%%%%
\centerslidesfalse \slidepagestyle{sidebar} \releaselogo
\releasebutton \buttonsize(16mm,7mm)(13,13) \freebutton(236,167)[h]
\freelogo(207,2)[2cm]
\begin{slide}
  \section{Buttons and logo}
  If you don't like to put the buttons and the logo into the panel,
  you can release them with \code{\bs{}releaselogo} and 
  \code{\bs{}releasebutton}.
  \code{\bs{}catchbutton} catches them ;-).\\
  \code{\bs{}buttonsize(width$_{max}$,height$_{max}$)(distance$_h$,distance$_v$)]}
  gives the maximal size of the buttons (aspect-ratio is kept) and the
  distance between the buttons without length. Keep in mind that the
  frame and the panel is placed into a picture-environment with
  \code{\bs{}unitlength=1$mm$} and all positions given in $mm$.\\
  \code{\bs{}freebutton(pos$_x$,pos$_y$)[h/v]} is the position of the buttons 
  and the direction [h] for horizontal and [v] for vertical buttons. 
  For horizontal buttons the anchor is the lower right point of the 
  right button and for vertical buttons the point between the lowest buttons.\\
  Use \code{\bs{}nobuttons} to disable buttons all over (e.g. for slides). \\
  \code{\bs{}freelogo(pos$_x$,pos$_y$)[width]} the anchor for the position is the 
  lower left point of the logo, the width should be clear.\\
  If you think that the page-counter is too lonely in PDF-mode switch
  him off with \code{\bs{}pagecounter[off/on]}.
\end{slide}
%%%%%%%%%%%%%%%%%%%%%%%%%%%%%%%%%%%%%%%%%%%%%%%%%%%%%%%%%%%%%%%%%%%%
\SlideHeightOverAll{270mm}
\SlideWidthOverAll{187mm}
\paperwidth=210mm
\paperheight=297mm
\slidepagestyle{empty}
\freebutton(180,263)[h]
\freelogo(2,2)[2cm]
\buttonsize(16mm,6mm)(12,12)
\begin{slide}
  \vspace*{10truemm}
  useful application for this feature....
\end{slide}
%%%%%%%%%%%%%%%%%%%%%%%%%%%%%%%%%%%%%%%%%%%%%%%%%%%%%%%%%%%%%%%%%%%%
\SlideHeightOverAll{187mm}
\SlideWidthOverAll{270mm}
\paperwidth=297mm
\paperheight=210mm
\slidepagestyle{sidebar}
\catchlogo
\catchbutton
\paneltile[on](0,160,200,200){liquid_helium}
\begin{slide}
  \section{Background}
  Older versions of \ifmslide{} provide \code{\bs{}pagestyle\{background\}}
  which makes objectively no sense, due to the fact that one needs a
  background but no panel or vice versa.
  The \code{\bs{}background\{filename\}} command substitute it together with the 
  switches \code{\bs{}showbackground} and \code{\bs{}hidebackground}.\\[2ex]
  A new feature for the panel-background is:\\
  \code{\bs{}paneltile[on/off](cut$_{llx}$,cut$_{lly}$,cut$_{urx}$,cut$_{ury}$)\{filename\}}\\
  which is now fully experimental which means that it can change it's
  definition and possibilities in further versions. Presently it works
  only for vertical panels. It takes a picture\footnote{\tiny JPEG
    etc. in PDF-mode and it's EPS-version for DVI-mode (produced with
    some tool like {\sl xv})} and fill the panel with tiles of it. It
  takes one tile in x-direction and so many as needed to fill in
  y-direction. There is still a rest which is filled by a
  part of the picture defined by the four cut-values...\\
  That did not work in all cases and in PDF-mode the rest of the tile
  is printed outside the visible area, but it is pre-alpha so what.
  \vfill
\end{slide}
%%%%%%%%%%%%%%%%%%%%%%%%%%%%%%%%%%%%%%%%%%%%%%%%%%%%%%%%%%%%%%%%%%%%
\begin{slide}
  \section{Useful commands}
  \stepwise{
    \code{\bs{}setInsideMargin\{left\}\{right\}\{top\}\{bottom\}} set the extra
    margins inside the frame. This is the easiest way to improve your output.\\[2ex]
    \step{Take \code{\bs{}headskip=length} to set the 
      {\bs{}section}-title to the correct vertical position.\\[2ex]}
    \step{Make own buttons and use them with 
      \code{\bs{}OnButton\{file1\}\{file2\}},\\
      \code{\bs{}OffButton\{file1\}\{file2\}} and
      \code{\bs{}DraftButton\{file1\}\{file2\}}.\\
      That is, using these commands you can replace the background buttons on
      which the control symbols are printed in the button panel. Above, file1
      and file2 are image files for the wide and small buttons, respectively\footnote{\tiny text by David Cyganski, thank you David}.
       There is a tiny perl-script called \code{genbutton} coming with \ifmslide{} which can be used to change
       the color of the standard buttons (\code{button1..}) and write new buttons.}
      \vfill
    }
\end{slide}
%%%%%%%%%%%%%%%%%%%%%%%%%%%%%%%%%%%%%%%%%%%%%%%%%%%%%%%%%%%%%%%%%%%%
\begin{slide}
  \headskip=0pt
  \section{Hints}
  \stepwise{
      The file \code{ifmslide.cfg} can be changed to set 
      up global or local values such as colors, the logo, backgrounds, sizes,
      the \code{baseurl} of your company, buttons, margins ... without 
      changing \ifmslide{} itself.\\[2ex]
    \step{Make use of \code{\bs{}setslidelength\{somelength\}\{somesize\}}, 
      \code{\bs{}semcm} and \code{\bs{}semin} as described in 
      \code{seminar.cls}.\\[2ex]}
    \step{Change font sizes with \code{\bs{}slidesmag\{mag\}} and 
      \code{\bs{}ptsize\{size\}} from \code{seminar.cls}.\\[2ex]}
    \step{Produce an EPS and a PDF-version\footnotemark{} of your 
      pictures  
      and include them with\\ 
      \code{\bs{}includegraphics[height=5\bs{}semcm,width=4\bs{}semcm]\{file\}\footnotemark}\\[2ex]}
    \step{Further information about \code{\bs{}step}, \code{\bs{}pause}, 
      \code{\bs{}href}, slides and other stuff
      can be found in the manuals for the packages 
      \tp, \hf and \code{seminar}.}
    \vfill}
  \addtocounter{footnote}{-1}
  \footnotetext{\tiny\code{epstopdf} is a good tool to convert eps and the 
    \code{graphicx} package knows many formats...}
  \stepcounter{footnote}
  \footnotetext{\tiny file is the filename without suffix}
 \end{slide}
%%%%%%%%%%%%%%%%%%%%%%%%%%%%%%%%%%%%%%%%%%%%%%%%%%%%%%%%%%%%%%%%%%%%
\begin{slide}
\section{"Bugs"/Limitations}
  \stepwise{
     \step{\code{\bs{}marginpar} is disabled by \code{seminar.cls}, there is currently no good work-around or replacement.\\[2ex]}
     \step{\code{\bs{}markboth, \bs{}markright, \bs{}leftmark} and \code{\bs{}rightmark} are currently without any use for all page-styles, since the whole layout is connected with the page headers. I will try to provide some of these in new versions. In addition some free objects are planed, placeable everywhere on the slide ...}
  }
\end{slide}
%%%%%%%%%%%%%%%%%%%%%%%%%%%%%%%%%%%%%%%%%%%%%%%%%%%%%%%%%%%%%%%%%%%%
\begin{slide}
\section{Required...}
There were lots of email concerning questions about required packages and where to find them. In addition to \code{texpower}, \code{hyperref} and \code{seminar} you need (not complete):\\[2ex]
\code{koma-script}\\ I haven't proved it, but this very useful package seems not to be included in the base-miktex distribution by default.\\[1ex]
\code{url}\\ used by hyperref...\\[1ex]
\code{graphicx}\\ someone complained about it :)\\[2ex]
All packages can be found in the \href{http://www.dante.de}{CTAN-archive}.
\end{slide}
%%%%%%%%%%%%%%%%%%%%%%%%%%%%%%%%%%%%%%%%%%%%%%%%%%%%%%%%%%%%%%%%%%%%
\slidepagestyle{empty}
\begin{slide}
  \begin{thebibliography}{99}
  \bibitem{texpower} {\ttfamily texpower}-Package: {\sl Stephan Lehmke}, 
    (Stephan.Lehmke@cs.uni-dortmund.de), 
    \href{http://ls1-www.cs.uni-dortmund.de/~lehmke/texpower}{University of Dortmund} or CTAN.
  \bibitem{hyperref} {\ttfamily hyperref}-Package: {\sl Sebastian Rahtz}, 
    \href{http://www.tug.org/applications/hyperref}{www.tug.org}
  \bibitem{pdfslide} {\ttfamily pdfslide.sty}: {\sl C. V. Radhakrishnan},
    (cvr@river-valley.com), \href{http://www.dante.de}{CTAN-archive}. 
  \end{thebibliography}
\end{slide}
%%%%%%%%%%%%%%%%%%%%%%%%%%%%%%%%%%%%%%%%%%%%%%%%%%%%%%%%%%%%%%%%%%%%
\end{document}
%%%%%%%%%%%%%%%%%%%%%%%%%%%%%%%%%%%%%%%%%%%%%%%%%%%%%%%%%%%%%%%%%%%%
%%%%%%%%%%%%%%%%%%%%%%%%%%%%%%%%%%%%%%%%%%%%%%%%%%%%%%%%%%%%%%%%%%%%


% LocalWords:  PDF ifmslide
