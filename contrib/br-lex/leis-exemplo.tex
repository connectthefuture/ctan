% ======================================================================
% leis-exemplo.tex
% Copyright (c) Youssef Cherem <ycherem(at)gmail.com>, 2016
%
% This file is part of the br-lex LaTeX2e class.

% This work may be distributed and/or modified under the conditions of
% the LaTeX Project Public License, version 1.3c of the license.
% The latest version of this license is in
%   http://www.latex-project.org/lppl.txt
% and version 1.3c or later is part of all distributions of LaTeX
% version 2005/12/01 and of this work.
%
% This work has the LPPL maintenance status "author-maintained".
% ======================================================================

\documentclass[a5paper,capitulo,titlepage=false]{br-lex}
\setmainfont{Old Standard}
%\usepackage[sfdefault]{FiraSans}


\begin{document}

\begin{center}
		Presidência da República\\
	Casa Civil\\
	Subchefia para Assuntos Jurídicos	
\end{center}

\titulo{Lei Complementar Nº 95,\\ 
	de 26 de fevereiro de 1998}

\descricao{Dispõe sobre a elaboração, a redação, a alteração e a consolidação das leis, conforme determina o parágrafo único do art. 59 da Constituição Federal, e estabelece normas para a consolidação dos atos normativos que menciona}


O PRESIDENTE DA REPÚBLICA Faço  saber  que   o    Congresso  Nacional decreta e eu sanciono  a  seguinte Lei Complementar:
	
\chapter{Definições Preliminares}
\label{chap:disposicoes}


\artigo A elaboração, a redação, a alteração e a consolidação das leis obedecerão ao disposto nesta Lei Complementar.

Parágrafo único. As disposições desta Lei Complementar aplicam-se, ainda, às medidas provisórias e demais atos normativos referidos no art. 59 da Constituição Federal, bem como, no que couber, aos decretos e aos demais atos de regulamentação expedidos por órgãos do Poder Executivo.

\artigo (VETADO)

\begin{paragrafos}
\paragrafo (VETADO)

\paragrafo Na numeração das leis serão observados, ainda, os seguintes critérios:
\end{paragrafos}


\begin{easylist}

#  as emendas à Constituição Federal terão sua numeração iniciada a partir da promulgação da Constituição;

##  as leis complementares, as leis ordinárias e as leis delegadas terão numeração sequencial em continuidade às séries iniciadas em 1946.

\end{easylist}


\chapter[Das técnicas de elaboração, redação e alteração das leis]{DAS TÉCNICAS DE ELABORAÇÃO,\\ REDAÇÃO E ALTERAÇÃO DAS LEIS}
\label{chap:tecnicas}
\section{Da Estruturação das Leis}

\artigo A lei será estruturada em três partes básicas:

\begin{easylist}
# parte preliminar, compreendendo a epígrafe, a ementa, o preâmbulo, o enunciado do objeto e a indicação do âmbito de aplicação das disposições normativas;

#  parte normativa, compreendendo o texto das normas de conteúdo substantivo relacionadas com a matéria regulada;

#  parte final, compreendendo as disposições pertinentes às medidas necessárias à implementação das normas de conteúdo substantivo, às disposições transitórias, se for o caso, a cláusula de vigência e a cláusula de revogação, quando couber.
\end{easylist}

\artigo A epígrafe, grafada em caracteres maiúsculos, propiciará identificação numérica singular à lei e será formada pelo título designativo da espécie normativa, pelo número respectivo e pelo ano de promulgação.

\artigo A ementa será grafada por meio de caracteres que a realcem e explicitará, de modo conciso e sob a forma de título, o objeto da lei.

\artigo O preâmbulo indicará o órgão ou instituição competente para a prática do ato e sua base legal.

\artigo O primeiro artigo do texto indicará o objeto da lei e o respectivo âmbito de aplicação, observados os seguintes princípios:

\begin{easylist}
#  excetuadas as codificações, cada lei tratará de um único objeto;

#  a lei não conterá matéria estranha a seu objeto ou a este não vinculada por afinidade, pertinência ou conexão;

#  o âmbito de aplicação da lei será estabelecido de forma tão específica quanto o possibilite o conhecimento técnico ou científico da área respectiva;

#  o mesmo assunto não poderá ser disciplinado por mais de uma lei, exceto quando a subsequente se destine a complementar lei considerada básica, vinculando-se a esta por remissão expressa.
\end{easylist}

\artigo A vigência da lei será indicada de forma expressa e de modo a contemplar prazo razoável para que dela se tenha amplo conhecimento, reservada a cláusula "entra em vigor na data de sua publicação" para as leis de pequena repercussão.

\begin{paragrafos}

\paragrafo A contagem do prazo para entrada em vigor das leis que estabeleçam período de vacância far-se-á com a inclusão da data da publicação e do último dia do prazo, entrando em vigor no dia subseqüente à sua consumação integral.    (Incluído pela Lei Complementar nº 107, de 26.4.2001)

\paragrafo As leis que estabeleçam período de vacância deverão utilizar a cláusula ‘esta lei entra em vigor após decorridos (o número de) dias de sua publicação oficial’ .    (Incluído pela Lei Complementar nº 107, de 26.4.2001)

\end{paragrafos}

\cortado{Art. 9º Quando necessária a cláusula de revogação, esta deverá indicar expressamente as leis ou disposições legais revogadas.}

\artigo A cláusula de revogação deverá enumerar, expressamente, as leis ou disposições legais revogadas.    (Redação dada pela Lei Complementar nº 107, de 26.4.2001)

Parágrafo único. (VETADO)     (Incluído pela Lei Complementar nº 107, de 26.4.2001)

\section[Da Articulação e da Redação das Leis]{Da Articulação e da Redação das Leis}

\artigo Os textos legais serão articulados com observância dos seguintes princípios:

\begin{easylist}
# a unidade básica de articulação será o artigo, indicado pela abreviatura "Art.", seguida de numeração ordinal até o nono e cardinal a partir deste;

# os artigos desdobrar-se-ão em parágrafos ou em incisos; os parágrafos em incisos, os incisos em alíneas e as alíneas em itens;

# os parágrafos serão representados pelo sinal gráfico "§", seguido de numeração ordinal até o nono e cardinal a partir deste, utilizando-se, quando existente apenas um, a expressão "parágrafo único" por extenso;

# os incisos serão representados por algarismos romanos, as alíneas por letras minúsculas e os itens por algarismos arábicos;

# o agrupamento de artigos poderá constituir Subseções; o de Subseções, a Seção; o de Seções, o Capítulo; o de Capítulos, o Título; o de Títulos, o Livro e o de Livros, a Parte;

# os Capítulos, Títulos, Livros e Partes serão grafados em letras maiúsculas e identificados por algarismos romanos, podendo estas últimas desdobrar-se em Parte Geral e Parte Especial ou ser subdivididas em partes expressas em numeral ordinal, por extenso;

# as Subseções e Seções serão identificadas em algarismos romanos, grafadas em letras minúsculas e postas em negrito ou caracteres que as coloquem em realce;

# a composição prevista no inciso V poderá também compreender agrupamentos em Disposições Preliminares, Gerais, Finais ou Transitórias, conforme necessário.	
\end{easylist}

\artigo As disposições normativas serão redigidas com clareza, precisão e ordem lógica, observadas, para esse propósito, as seguintes normas: 

\begin{easylist}
	# para a obtenção de clareza:
	## usar as palavras e as expressões em seu sentido comum, salvo quando a norma versar sobre assunto técnico, hipótese em que se empregará a nomenclatura própria da área em que se esteja legislando;
	##  usar frases curtas e concisas;
	## construir as orações na ordem direta, evitando preciosismo, neologismo e adjetivações dispensáveis;
	
	## buscar a uniformidade do tempo verbal em todo o texto das normas legais, dando preferência ao tempo presente ou ao futuro simples do presente;
	
	## usar os recursos de pontuação de forma judiciosa, evitando os abusos de caráter estilístico;
	
	# para a obtenção de precisão:
	
	## articular a linguagem, técnica ou comum, de modo a ensejar perfeita compreensão do objetivo da lei e a permitir que seu texto evidencie com clareza o conteúdo e o alcance que o legislador pretende dar à norma;
	
	## expressar a idéia, quando repetida no texto, por meio das mesmas palavras, evitando o emprego de sinonímia com propósito meramente estilístico;
	
	## evitar o emprego de expressão ou palavra que confira duplo sentido ao texto;
	
	## escolher termos que tenham o mesmo sentido e significado na maior parte do território nacional, evitando o uso de expressões locais ou regionais;
	
	## usar apenas siglas consagradas pelo uso, observado o princípio de que a primeira referência no texto seja acompanhada de explicitação de seu significado;
	
	\cortado{f) grafar por extenso quaisquer referências feitas, no texto, a números e percentuais;}
	
	## grafar por extenso quaisquer referências a números e percentuais, exceto data, número de lei e nos casos em que houver prejuízo para a compreensão do texto;     (Redação dada pela Lei Complementar nº 107, de 26.4.2001)
	
	## indicar, expressamente o dispositivo objeto de remissão, em vez de usar as expressões ‘anterior’, ‘seguinte’ ou equivalentes;     (Incluída pela Lei Complementar nº 107, de 26.4.2001)
	
	# para a obtenção de ordem lógica:
	
## reunir sob as categorias de agregação - subseção, seção, capítulo, título e livro - apenas as disposições relacionadas com o objeto da lei;
	
	## restringir o conteúdo de cada artigo da lei a um único assunto ou princípio;
	
	## expressar por meio dos parágrafos os aspectos complementares à norma enunciada no caput do artigo e as exceções à regra por este estabelecida;
	
	## promover as discriminações e enumerações por meio dos incisos, alíneas e itens.
	\end{easylist}
	\section{Da Alteração das Leis}
	
	\artigo A alteração da lei será feita:
	
	\begin{easylist}
		#  mediante reprodução integral em novo texto, quando se tratar de alteração considerável;
		
		\cortado{II - na hipótese de revogação;}
		#  mediante revogação parcial;    (Redação dada pela Lei Complementar nº 107, de 26.4.2001)
		# nos demais casos, por meio de substituição, no próprio texto, do dispositivo alterado, ou acréscimo de dispositivo novo, observadas as seguintes regras:
		
		\cortado{a) não poderá ser modificada a numeração dos dispositivos alterados;}
		
		## revogado; (Redação dada pela Lei Complementar nº 107, de 26.4.2001)
		
		\cortado{b) no acréscimo de dispositivos novos entre preceitos legais em vigor, é vedada, mesmo quando recomendável, qualquer renumeração, devendo ser utilizado o mesmo número do dispositivo imediatamente anterior, seguido de letras maiúsculas, em ordem alfabética, tantas quantas forem suficientes para identificar os acréscimos;}
		
		## no acréscimo de dispositivos novos entre preceitos legais em vigor, é vedada, mesmo quando recomendável, qualquer renumeração, devendo ser utilizado o mesmo número do dispositivo imediatamente anterior, seguido de letras maiúsculas, em ordem alfabética, tantas quantas forem suficientes para identificar os acréscimos;
			\end{easylist}

\begin{easylist}
\ListProperties(Start1=37)
# aos autores
# são assegurados
## a proteção ## o direito
#
#
#
#
#
#
#
#
##
##
###
###
##
#

\end{easylist}

\addtocounter{artigo}{100}

\artigo

\begin{easylist}
	#
	#
	#
	## ## ## 
	### ### ###
	#
	#
	#

\end{easylist}

\end{document}


