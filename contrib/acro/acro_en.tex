% !arara: pdflatex
% !arara: biber
% !arara: pdflatex
% arara: pdflatex
% arara: pdflatex
% --------------------------------------------------------------------------
% the ACRO package
% 
%   Typeset Acronyms
% 
% --------------------------------------------------------------------------
% Clemens Niederberger
% Web:    http://www.mychemsitry.eu/forums/forum/acro/
% E-Mail: contact@mychemistry.eu
% --------------------------------------------------------------------------
% Copyright 2011-2017 Clemens Niederberger
% 
% This work may be distributed and/or modified under the
% conditions of the LaTeX Project Public License, either version 1.3
% of this license or (at your option) any later version.
% The latest version of this license is in
%   http://www.latex-project.org/lppl.txt
% and version 1.3 or later is part of all distributions of LaTeX
% version 2005/12/01 or later.
% 
% This work has the LPPL maintenance status `maintained'.
% 
% The Current Maintainer of this work is Clemens Niederberger.
% --------------------------------------------------------------------------
% The acro package consists of the files
%  - acro.sty, acro_en.tex, acro_en.pdf, README
% --------------------------------------------------------------------------
% If you have any ideas, questions, suggestions or bugs to report, please
% feel free to contact me.
% --------------------------------------------------------------------------
\documentclass[load-preamble+]{cnltx-doc}

\usepackage[utf8]{inputenc}
\usepackage[single,macros,accsupp,index]{acro}
\setcnltx{
  package  = {acro} ,
  info     = {Typeset Acronyms and other Abbreviations} ,
  authors  = Clemens Niederberger ,
  email    = contact@mychemistry.eu ,
  url      = https://bitbucket.org/cgnieder/acro/ ,
  abstract = {%
    \acro\ allows you to define and use abbreviations in a simple way.
    Abbreviations can be divided into different classes of abbreviations.
    Lists of abbreviations can be created (also of separate classes of
    abbreviations) and printed wherever you want the lists to appear.\par
    \acro\ provides an option \option{single} which ignores abbreviations that
    are used only once in the whole document.\par
    As an experimental feature \acro\ also offers the option \option{sort} which
    automatically sorts the list created by \cs{printacronyms}.\par
    \acro\ also has the feature of creating \emph{local} lists
  } ,
  add-cmds = {
    ac, Ac, aca, Aca, acap, Acap, acbarrier, acdot, acf, Acf, acflike, acg,
      acfp, Acfp, acfplike, aciftrailing, acl, Acl, aclp, Aclp, acp, Acp,
      AcroRegisterTrailing, acs, Acs, acsingle, Acsingle, acsp, Acsp, acspace,
      acreset, acresetall, acsetup,
    DeclareAcroCommand, DeclareAcronym, DeclareAcroExtraStyle,
      DeclareAcroFirstStyle, DeclareAcroListHeading, 
      DeclareAcroListStyle, DeclareAcroPageStyle, DeclarePseudoAcroCommand,
    iac, Iac, iaca, Iaca, iacs, Iacs, iacl, Iacl, iacf, Iacf, iacflike,
      Iacflike,
    NewAcroCommand, NewPseudoAcroCommand,
    printacronyms,
    ProvideAcroCommand, ProvideAcroEnding, ProvidePseudoAcroCommand,
    RenewAcroCommand, RenewPseudoAcroCommand
  } ,
  add-silent-cmds = {
    addcolon, DeclareInstance, babelhyphen, ExplSyntaxOff, ExplSyntaxOn, nato,
    NewDocumentCommand, newlist, ny, setlist
  } ,
  index-setup = { level = \section , headers={\indexname}{\indexname} }
}

\acsetup{
  use-barriers = true ,
  hyperref     = true ,
  log
}

\usepackage{varioref}

\defbibheading{bibliography}{\section{References}}

\usepackage{csquotes}

\usepackage[biblatex]{embrac}[2012/06/29]
\ChangeEmph{[}[,.02em]{]}[.055em,-.08em]
\ChangeEmph{(}[-.01em,.04em]{)}[.04em,-.05em]

\usepackage{filecontents}

\addbibresource{\jobname.bib}
\begin{filecontents}{\jobname.bib}
@online{wikipedia,
  author   = {Wikipedia},
  title    = {Acronym and initialism},
  urldate  = {2012-06-21},
  url      = {http://en.wikipedia.org/wiki/Acronyms},
  year     = {2012}
}
@online{NewYork,
  author   = {Wikipedia},
  title    = {New York City},
  urldate  = {2012-09-27},
  url      = {http://en.wikipedia.org/wiki/New_York_City},
  year     = {2012}
}
@manual{interface3,
  author    = {{The \LaTeX3 Project Team}} ,
  shorthand = {L3P} ,
  sortname  = {LaTeX3 Project Team} ,
  title     = {The \LaTeX3 Interfaces} ,
  date      = {2015-09-06} ,
  url       = {http://mirrors.ctan.org/macros/latex/contrib/l3kernel/interface3.pdf}
}
\end{filecontents}

% additional packages:
\usepackage{longtable,array,booktabs,enumitem,amssymb}

\newcommand*\wikipedia{%\libertineGlyph{W.alt}\kern-.055em
\textsc{Wikipedia}}
\newcommand*\h[1]{\textcolor{cnltx}{\textbf{#1}}}

\ProvideAcroEnding {possessive} {'s} {'s}

% declare acronyms
\DeclareAcronym{cd}
  {
    short        = cd ,
    long         = Compact Disc ,
    short-format = \scshape
  }
\let\ctan\relax
\DeclareAcronym{ctan}
  {
    short     = ctan ,
    long      = Comprehensive \TeX\ Archive Network ,
    format    = \scshape ,
    pdfstring = CTAN ,
    accsupp   = CTAN
  }
\def\ctan{\acs{ctan}}
\DeclareAcronym{ecu}
  {
    short   = ECU ,
    long    = Steuerger\"at ,
    foreign = Electronic Control Unit ,
    foreign-lang = english
  }
\DeclareAcronym{id}
  {
    short        = id ,
    long         = identification string ,
    short-format = \scshape
  }
\DeclareAcronym{jpg}
  {
    short = JPEG ,
    sort  = jpeg ,
    alt   = JPG ,
    long  = Joint Photographic Experts Group
  }
\DeclareAcronym{la}
  {
    short        = LA ,
    short-plural = ,
    long         = Los Angeles,
    long-plural  = ,
    class        = city
  }
\let\lppl\relax
\DeclareAcronym{lppl}
  {
    short     = lppl ,
    long      = \LaTeX\ Project Public License ,
    format    = \scshape ,
    pdfstring = LPPL ,
    accsupp   = LPPL ,
    index-cmd = \csname @gobble\endcsname
  }
\def\lppl{\acs{lppl}}
\DeclareAcronym{MP}
  {
    short = MP ,
    long  = Member of Parliament ,
    long-plural-form = Members of Parliament
  }
\DeclareAcronym{nato}
  {
    short        = nato ,
    long         = North Atlantic Treaty Organization ,
    extra        = \emph{deutsch}: Organisation des Nordatlantikvertrags ,
    short-format = \scshape
  }
\DeclareAcronym{ny}
  {
    short        = NY ,
    short-plural = ,
    long         = New York ,
    long-plural  = ,
    class        = city ,
    cite         = NewYork
  }
\DeclareAcronym{ot}
  {
    short        = ot ,
    long         = Other Test ,
    short-format = \scshape
  }
\DeclareAcronym{pdf}
  {
    short     = pdf ,
    long      = Portable Document Format ,
    format    = \scshape ,
    pdfstring = PDF ,
    accsupp   = PDF
  }
\DeclareAcronym{sw}
  {
    short       = SW ,
    long        = Sammelwerk ,
    long-plural = e
  }
\DeclareAcronym{test}
  {
    short = ST ,
    long  = Some Test
  }
\DeclareAcronym{tex.sx}
  {
    short = \TeX.sx ,
    sort  = TeX.sx ,
    long  = \TeX{} StackExchange
  }
\DeclareAcronym{ufo}{
   short           = UFO ,
   long            = unidentified flying object ,
   long-indefinite = an
}


\makeatletter
\protected\def\@versionstar{\raisebox{-.25em}{*}}
\newcommand\versionstar{\texorpdfstring{\@versionstar}{*}}
\makeatother

\newcommand*\latin{\textit}

\makeatletter
\newcommand*\TF{\textcolor{red}{\uline{\code{\textcolor{cs}{\textit{TF}}}}}}
\renewenvironment{commands}
  {%
    \cnltx@set@catcode_{12}%
    \let\command\cnltx@command
    \cnltxlist
  }
  {\endcnltxlist}
\makeatother

\newcommand*\mailto[1]{\texttt{\href{mailto:#1}{#1}}}

\begin{document}
\section{Licence and Requirements}
\license

\acro\ loads and needs the following packages:
\pkg{expl3}\footnote{\CTANurl{l3kernel}}, \pkg{xparse}, \pkg{xtemplate},
\pkg{l3keys2e}\footnote{\CTANurl{l3packages}},
\pkg{zref-abspage}\footnote{\CTANurl{oberdiek}} and
\needpackage{translations}~\cite{pkg:translations}.

\section{Basics}
\subsection{Creating New Acronyms}
Acronyms are created with the command \cs{DeclareAcronym}.
\begin{commands}
  \command{DeclareAcronym}[\marg{id}\marg{list of keys}]
    The basic command for declaring an acronym.
\end{commands}
This command understands a number of keys which are listed below.  Some of
them are not described immediately but at appropriate places in the
documentation.
\begin{options}
  %% short
  \keyval{short}{text}\Default!
    The short form of the acronym.  This option is required: an acronym must
    have a short form.  If this is set it \emph{must} be set as first option!
    If another option is set first and notices the \option{short} option
    missing it assumes that the \acs{id} should be used as short version and
    sets it accordingly.  A warning will be written to the log then.
  %% long
  \keyval{long}{text}\Default!
    The long form of the acronym.  This option is required: an acronym must
    have a description.
  %% short-plural
  \keyval{short-plural}{text}\Default{s}
    The plural ending appended to the short form.
  %% short-plural-form
  \keyval{short-plural-form}{text}
    The\sinceversion{2.0} plural short form of the acronym; replaces the short
    form when used instead of appending the plural ending.
  %% long-plural
  \keyval{long-plural}{text}\Default{s}
    The plural ending appended to the long form.
  %% long-plural-form
  \keyval{long-plural-form}{text}
    Plural long form of the acronym; replaces the long form when used
    instead of appending the plural ending.
   %% alt-plural
  \keyval{alt-plural}{text}\Default{s}
    The\sinceversion{2.0} plural ending appended to the alternative form.
  %% alt-plural-form
  \keyval{alt-plural-form}{text}
    The\sinceversion{2.0} plural alternative form of the acronym; replaces the
    alternative form when used instead of appending the plural ending.
  %% list
  \keyval{list}{text}
    If specified this will be written in the list as description instead of
    the long form.
  %% short-indefinite
  \keyval{short-indefinite}{text}\Default{a}
    Indefinite article for the short form.
  %% long-indefinite
  \keyval{long-indefinite}{text}\Default{a}
    Indefinite article for the long form.
  %% long-pre
  \keyval{long-pre}{text}
    \meta{text} is prepended to the long form in the text but not in the list
    of acronyms.
  %% long-post
  \keyval{long-post}{text}
    \meta{text} is appended to the long form in the text but not in the list
    of acronyms.
  %% alt
  \keyval{alt}{text}
    Alternative short form.
  %% alt-indefinite
  \keyval{alt-indefinite}{text}\Default{a}
    Indefinite article for the alternative form.
  %% extra
  \keyval{extra}{text}
    Extra information to be added in the list of acronyms.
  %% foreign
  \keyval{foreign}{original long form}
    Can be useful when dealing with acronyms in foreign languages, see
    section~\vref{ssec:foreign} for details.
  %% foreign-lang
  \keyval{foreign-lang}{language}
    \sinceversion{2.3}The \pkg{babel}~\cite{pkg:babel} or
    \pkg{polyglossia}~\cite{pkg:polyglossia} language of the foreign form.
    This language is used to wrap the entry with
    \cs*{foreignlanguage}\marg{language} if either \pkg{babel} or
    \pkg{polyglossia} is loaded.  You'll need to take care that the
    corresponding language is loaded by \pkg{babel} or \pkg{polyglossia}.
  %% single
  \keyval{single}{text}
    \sinceversion{2.3}If provided \meta{text} will be used instead of the long
    form if the acronym is only used a single time \emph{and} the option
    \keyis{single}{true} is active.
  %% sort
  \keyval{sort}{text}
    If used the acronym will be sorted according to this key instead of its
    \acs{id}.
  %% class
  \keyval{class}{csv list}
    The\changedversion{2.4} class(es) the acronym belongs to.
  %% cite
  \keylit{cite}{\oarg{prenote}\oarg{postnote}\marg{citation keys}}
    A citation that is printed to the acronym according to an option explained
    later.
  %% short-format
  \keyval{short-format}{\TeX{} code}
    The format used for the short form of the acronym.
  %% long-format
  \keyval{long-format}{\TeX{} code}
    The format used for the long form of the acronym.
  %% first-long-format
  \keyval{first-long-format}{\TeX{} code}
    The format used for the first long form of the acronym as set with \cs{ac},
    \cs{acf} or \cs{acflike} and their uppercase, plural and indefinite forms.
  %% single-format
  \keyval{single-format}{\TeX{} code}
    \sinceversion{2.3}The format used for the acronym if the acronym is only
    used a single time.
  %% first-style
  \keychoice{first-style}{default,empty,square,short,long,reversed,footnote,sidenote,%
    footnote-reversed,sidenote-reversed}
    \sinceversion{2.3}The style of the first appearance of the acronym, see
    also section~\ref{sec:opti-regard-acronyms}.
  %% pdfstring
  \keylit{pdfstring}{\meta{text}/\meta{plural ending}}
    \changedversion{2.4b}Used as \acs{pdf} string replacement in bookmarks
    when used together with the \pkg{hyperref} package.  The appended plural
    ending is optional.  If you leave it (\emph{and} the \code{/}) the default
    ending is used.  \meta{text} is expanded before it is saved.
  %% accsupp
  \keyval{accsupp}{text}
    Sets the \code{ActualText} key as presented by the \pkg{accsupp} package
    for the acronym.
  %% tooltip
  \keyval{tooltip}{text}
    \sinceversion{2.1}Sets the tooltip description for an acronym.  For
    actually getting tooltips you also need an appropriate setting of the
    \option{tooltip-cmd} option or to set the package option
    \option{tooltip}.
  %% index-sort
  \keyval{index-sort}{text}
    If you use the package option \option{index} every occurrence of an
    acronym is recorded to the index and sorted by its \acs{id} or (if set) by
    the value of the \option{sort} key.  This key allows to set an individual
    sorting option for the index.  See section~\vref{ssec:index} for details.
  %% index
  \keyval{index}{text}
    This key allows to overwrite the automatic index entry with an arbitrary
    one.  See section~\vref{ssec:index} for details.
 %% index-cmd
  \keyval{index-cmd}{control sequence}
    This key let's you set an individual index creating command for this
    acronym.  It should be a command that takes one mandatory argument.  See
    section~\vref{ssec:index} for details. 
\end{options}

In its simplest form an acronym needs a short and a long form.  Please note
that both keys \emph{must} be set and that the \option{short} key \emph{must}
always be the \emph{first} key that is set.
\begin{sourcecode}
  % preamble:
  \DeclareAcronym{test}{
    short = ST ,
    long  = Some Test
  }
\end{sourcecode}
This creates the acronym ``\acs{test}'' with the \acs{id} ``test'' and the
long form ``\acl{test}.''

The \option{format} key allows you to choose a specific format for the short
form of an acronym:
\begin{sourcecode}
  % preamble:
  \DeclareAcronym{ot}{
    short        = ot ,
    long         = Other Test ,
    short-format = \scshape
  }
\end{sourcecode}
The short form now looks like this: \acs{ot}.

The \option{cite} key needs a bit explaining.  It expects arguments like the
standard \cs*{cite} command, \latin{i.e.}, two optional arguments setting the
\meta{prenote} and \meta{postnote} and one mandatory argument setting the
citation key.
\begin{sourcecode}
  % preamble:
  \DeclareAcronym{ny}{
    short        = NY ,
    short-plural = ,
    long         = New York ,
    long-plural  = ,
    cite         = {NewYork} 
  }
\end{sourcecode}

\begin{sourcecode}[sourcecode-options={style=cnltx-bibtex}]
  % bib file for use with biber/biblatex:
  @online{NewYork,
    author  = {Wikipedia},
    title   = {New York City},
    urldate = {2012-09-27},
    url     = {http://en.wikipedia.org/wiki/New_York_City},
    year    = {2012}
  }
\end{sourcecode}
The first appearance now looks as follows\footnote{The appearance of the
  citation of course depends on the citation style you're using.}: \acf{ny}.

\subsection{Logging of Acronyms}
When you activate \acro's option\sinceversion{2.5} \option{log} \acro' writes
information about the acronyms it defines to the log file.
\begin{options}
  \keychoice{log}{\default{true},false,silent,verbose}\Default{false}
    When set to \code{true}/\code{silent} \acro{} writes the main properties
    of an acronym to the log file.  When set to \code{verbose} \acro' writes
    \emph{all}  properties of an acronym to the log file.
\end{options}

This is an example of the logging info with \keyis{log}{true} or
\keyis{log}{silent}.
\begin{sourcecode}
  =================================================
  | acro info -- defining new acronym:
  |   ID = {jpg}
  |   short = {JPEG}
  |   long = {Joint Photographic Experts Group}
  |   alt = {JPG}
  |   sort = {jpeg}
  |   class = {}
  |   list = {}
  |   extra = {}
  |   foreign = {}
  |   pdfstring = {}
  |   cite = {}
  =================================================
\end{sourcecode}

\subsection{Using the Acronyms -- the Commands}\label{sec:using-acronyms-comm}
Acronyms are used with one of the following commands:
\begin{commands}
  \command{ac}[\sarg\marg{id}]
    basic command; the first output is different from subsequent ones.
  \command{Ac}[\sarg\marg{id}]
    same as \cs{ac} but capitalizes the first letter of the long form.
  \command{acs}[\sarg\marg{id}]
    \h{s}hort form; the actual acronym.
  \command{acl}[\sarg\marg{id}]
    \h{l}ong form; the meaning of the acronym.
  \command{Acl}[\sarg\marg{id}] 
    same as \cs{acl} but capitalizes first letter.
  \command{aca}[\sarg\marg{id}]
    \h{a}lternative short form as specified in the \option{alt} key of
    \cs{DeclareAcronym}; if it hasn't been specified this is identical to
    \cs{acs}.
  \command{acf}[\sarg\marg{id}]
    first form; output like the first time \cs{ac} is output.
  \command{Acf}[\sarg\marg{id}]
    same as \cs{acf} but capitalizes first letter of the long form.
  \command{acp}[\sarg\marg{id}]
    \h{p}lural form of \cs{ac};
  \command{Acp}[\sarg\marg{id}]
    same as \cs{acp} but capitalizes first letter of the long form.
  \command{acsp}[\sarg\marg{id}]
    plural form of \cs{acs};
  \command{aclp}[\sarg\marg{id}]
    plural form of \cs{acl};
  \command{Aclp}[\sarg\marg{id}]
    same as \cs{aclp} but capitalizes first letter.
  \command{acap}[\sarg\marg{id}]
    plural form of \cs{aca};
  \command{acfp}[\sarg\marg{id}]
    plural form of \cs{acf};
  \command{Acfp}[\sarg\marg{id}]
    same as \cs{acfp} but capitalizes first letter of the long form.
\end{commands}
If an acronym is used the first time with \cs{ac} its output is different from
subsequent uses.  To be clear on this: the first time!  If the acronym has
been used with \emph{any} of the output commands before it is \emph{not} the
first time any more.

If you use the starred variant an acronym will not be marked as used.  This
proves useful if an acronym is typeset in a section title, for example, since
then the appearance in the table of contents won't mark it as used.

\begin{example}[side-by-side]
  % preamble:
  % \DeclareAcronym{cd}{
  %   short        = cd ,
  %   long         = Compact Disc ,
  %   short-format = \scshape
  % }
  first time: \ac{cd} \\
  second time: \ac{cd} \\
  short: \acs{cd} \\
  alternative: \aca{cd} \\
  first again: \acf{cd} \\
  long: \acl{cd} \\
  short plural: \acsp{cd} \\
  long plural: \aclp{cd}
\end{example}

\subsection{Plural Forms}
If an acronym is defined in the standard way \acro\ uses an `s' that's appended
to both the short and the long form when one of the plural commands is used.
However, that is not always the best solution.  For one thing not all acronyms
may have a plural form.  Second, the plural form especially of the long forms
may be formed differently.  And third, other languages can have other plural
endings.

For these reasons \cs{DeclareAcronym} can get the following keys:
\begin{options}
  \keyval{short-plural}{text}\Default{s}
    The plural ending of the short form.
  \keyval{long-plural}{text}\Default{s}
    The plural ending of the long form.
  \keyval{long-plural-form}{text}
    An alternative plural form for the long form.
\end{options}
These keys are optional.  If they're not used, the default setting is
\code{s}.  If you use \option{long-plural-form} the long form will be replaced
by the specified plural form when necessary.

Suppose we define the following acronyms:
\begin{sourcecode}
  \DeclareAcronym{cd}{
    short        = cd ,
    long         = Compact Disc ,
    short-format = \scshape
  }
  \DeclareAcronym{ny}{
    short        = NY ,
    short-plural = ,
    long         = New York ,
    long-plural  =
  }
  \DeclareAcronym{sw}{
    short       = SW ,
    long        = Sammelwerk ,
    long-plural = e
  }
  \DeclareAcronym{MP}{
    short            = MP ,
    long             = Member of Parliament ,
    long-plural-form = Members of Parliament
  }
\end{sourcecode}
These acronyms now have the following plural appearances:
\begin{example}[side-by-side]
  \acsp{cd}, \aclp{cd} \\
  \acsp{ny}, \aclp{ny} \\
  \acsp{sw}, \aclp{sw} \\
  \acsp{MP}, \aclp{MP}
\end{example}

\subsection{Alternative Short Forms}
For some acronyms it might be useful to have alternative forms.  For this
\cs{DeclareAcronym} has another key:
\begin{options}
 \keyval{alt}{text}
   Alternative short form.
\end{options}
\begin{example}
  % preamble:
  % \DeclareAcronym{jpg}{
  %   short = JPEG ,
  %   sort  = jpeg ,
  %   alt   = JPG ,
  %   long  = Joint Photographic Experts Group
  % }
  default: \acs{jpg} \\
  alt.: \aca{jpg}
\end{example}
The alternative form uses the same plural ending as the default short form and
is formatted in the same way.

\subsection{Extra Information for the List Entry}
Of course you can print a list of acronyms where their meaning is explained.
Sometimes it can be useful to add additional information there.  This is done
with another key to \cs{DeclareAcronym}:
\begin{options}
  \keyval{extra}{text}
    Additional information for the list of acronyms.
\end{options}
These information will only be displayed in the list.  See
section~\vref{sec:print_lists} for the impact of the following example.

\begin{example}
  % preamble:
  % \DeclareAcronym{nato}{
  %   short        = nato ,
  %   long         = North Atlantic Treaty Organization ,
  %   extra        = \textit{deutsch}: Organisation des Nordatlantikvertrags ,
  %   short-format = \scshape
  % }
  The \ac{nato} is an intergovernmental military alliance based on the
  North Atlantic Treaty which was signed on 4~April 1949. \ac{nato}
  headquarters are in Brussels, Belgium, one of the 28 member states
  across North America and Europe, the newest of which, Albania and
  Croatia, joined in April 2009.
\end{example}

\subsection{Foreign Language Acronyms}\label{ssec:foreign}
I repeatedly read the wish for being able to add translations to acronyms when
the acronyms stem from another language than the document language,
\latin{i.e.}, something like the following in a German document:
\begin{example}[side-by-side]
  \ac{ecu}\\
  \ac{ecu}
\end{example}
That's why I decided to add the following properties:
\begin{options}
  \keyval{foreign}{original long form}
    A description for an acronym originating in another language than the
    document language.
  \keyval{foreign-lang}{language}
    \sinceversion{2.3}The \pkg{babel}~\cite{pkg:babel} or
    \pkg{polyglossia}~\cite{pkg:polyglossia} language of the foreign form.
    This language is used to wrap the entry with
    \cs*{foreignlanguage}\marg{language}.
\end{options}

Here is the definition of the above mentioned \ac{ecu} acronym:
\begin{sourcecode}
  \DeclareAcronym{ecu}{
    short   = ECU ,
    long    = Steuerger\"at ,
    foreign = Electronic Control Unit ,
    foreign-lang = english
  }
\end{sourcecode}
As you have seen this adds the \option{foreign} entry to the first appearance
of an acronym.  It is also added in parentheses to the list of acronyms after
the \option{long} entry.  Actually the entry there is the argument to the
following command:
\begin{commands}
  \command{acroenparen}[\marg{argument}]
    Places \meta{argument} in parentheses: \cs{acroenparen}\Marg{example}:
    \acroenparen{example}.  See page~\pageref{key:list-foreign-format} for a
    way to customize this other than redefining it.
\end{commands}

\section{Additional Commands and Possibilities}
\subsection{Indefinite Forms}

Unlike many other languages\footnote{Let's better say: unlike the other
  languages where I know at least the basics.} in English the indefinite
article is not determined by the grammatical case, gender or number but by the
pronounciation of the following word.  This means that the short and the long
form of an acronym can have different indefinite articles.  For these cases
\acro\ offers the keys \option{short-indefinite}, \option{alt-indefinite} and
\option{long-indefinite} whose default is \code{a}.  For every lowercase
singular command two alternatives exist, preceded by \code{i} and \code{I},
respectively, which output the lowercase and uppercase version of the
corresponding indefinite article.

\begin{example}
  % preamble:
  % \DeclareAcronym{ufo}{
  %   short           = UFO ,
  %   long            = unidentified flying object ,
  %   long-indefinite = an
  % }
  \Iac{ufo}; \iacs{ufo}; \iacl{ufo}
\end{example}

\subsection{Uppercasing}
\begin{commands}
  \command{acfirstupper}[\marg{token list}]
     This command uppercases the first token in \meta{token list}.  The
     command is less powerful than \cs{makefirstuc} that is provided by the
     \pkg{mfirstuc} package~\cite{pkg:mfirstuc} but it is expandable.  Obvious
     downsides are for example that it does not uppercase accented letters.
\end{commands}

\subsection{Simulating the First Appearance}\label{sec:simul-first-appe}
Users told me\footnote{Well -- one, to be precise ;)} that there are cases
when it might be useful to have the the acronym typeset according to the
\option{first-style} but with another text than the long form.  For such cases
\acro\ offers the following commands.
\begin{commands}
  \command{acflike}[\sarg\marg{id}\marg{instead of long form}]
    Write some alternative long form for acronym with \acs{id} \meta{id} as if
    it were the first time the acronym was used.
  \command{acfplike}[\sarg\marg{id}\marg{instead of long form}]
    Plural form of \cs{acflike}.
\end{commands}

\begin{example}[side-by-side]
  \acsetup{first-style=footnote}
  \acflike{ny}{the big apple}
\end{example}

The plural ending in \cs{acfplike} is only appended to the short form.  It
makes no sense to append it to the text that is inserted manually anyway.
Note that whatever text you're inserting might be gobbled depending on the
\option{first-style} you're using.

\subsection{Fetching the Single Appearance}
There\sinceversion{2.3} are macros that fetch the \emph{single} appearance of
an acronym even if it has been used more than once and the \option{single}
option is active.
\begin{commands}
  \command{acsingle}[\sarg\marg{id}]
    Write acronym as if it were used only a single time.
  \command{Acsingle}[\sarg\marg{id}]
    Uppercase form of \cs{acsingle}.
\end{commands}

\begin{example}[side-by-side]
  \acsingle{ny}
\end{example}

\subsection{Using Classes}
The acronyms of \acro\ can be divided into different classes.  This doesn't
change the output but allows different acronym lists, see
section~\vref{sec:print_lists}.  For this \cs{DeclareAcronym} has an additional
key:
\begin{options}
  \keyval{class}{csv list}
    Associated\changedversion{2.4} class(es) for an acronym.
\end{options}

This might be useful if you can and want to divide your acronyms into
different types, technical and grammatical ones, say, that shall be listed in
different lists.  Since every acronym can get a list of associated classes
those classes can effectively be used like tags for filtering acronyms.

\begin{example}[side-by-side]
  % preamble:
  % \DeclareAcronym{la}{
  %   short        = LA ,
  %   short-plural = ,
  %   long         = Los Angeles ,
  %   long-plural  = ,
  %   class        = city
  % }
  % \DeclareAcronym{ny}{
  %   short        = NY ,
  %   short-plural = ,
  %   long         = New York ,
  %   long-plural  = ,
  %   class        = city ,
  %   cite         = NewYork
  % }
  \acl{la} (\acs{la}) \\
  \acl{ny} (\acs{ny})
\end{example}

\subsection{Reset or Mark as Used, Test if Acronym Has Been Used}

If you want for some reason to fool \acro\ into thinking that an acronym is
used for the first time you can call one of these commands:
\begin{commands}
  \command{acreset}[\marg{comma separated list of ids}]
    This will reset a used acronym such that the next use of \cs{ac} will
    again print it as if it were used the first time.  This will \emph{not}
    remove an acronym from being printed in the list if it actually \emph{has}
    been used before.
  \command{acresetall}
    Reset all acronyms.
  \command{acifused}[\marg{id}\marg{true}\marg{false}]
    This command tests if the acronym with \ac{id} \meta{id} has already been
    used and either puts \code{true} or \code{false} in the input stream.
\end{commands}
\begin{example}[side-by-side]
  \acreset{ny}\ac{ny}
\end{example}
Beware that both commands act \emph{globally}!  There are also commands that
effectively do the opposite of \cs{acreset}, \latin{i.e.}, mark acronyms as
used:
\begin{commands}
  \command{acuse}[\marg{comma separated list of ids}]
    This has the same effect as if an acronym had been used twice, that is,
    further uses of \cs{ac} will print the short form and the acronym will in
    any case be printed in the list (as long as its class is not excluded).
  \command{acuseall}
    Mark all acronyms as used.
\end{commands}
Then there are two further commands related to using acronyms:
\begin{commands}
  \command{acswitchoff}
    This\sinceversion{2.6} command is for patching in certain situations.  For
    example some table environments like \env*{tabularx} or \env*{tabu} pass
    their content two or more times for determining the width of the table
    columns.  Those can be patched to add \cs{acswitchoff} to their trial
    phase.
  \command{acswitchon}
    Effectively\sinceversion{2.6} the opposite of \cs{acswitchoff} -- this
    command should probably never be needed.
\end{commands}

\subsection{\cs*{ac} and Friends in \acs*{pdf} Bookmarks, Accessibility
  Support, Tooltips}
\subsubsection{\acs*{pdf} Bookmarks}
\acro's commands usually are not expandable which means they'd leave unallowed
tokens in \acs{pdf} bookmarks.  \pkg{hyperref} offers \cs*{texorpdfstring} to
circumvent that issue manually but that isn't really a nice solution.  What's
the point of having macros to get output for you if you have to specify it
manually after all?

That is why \acro\ offers a preliminary solution for this.  In a bookmark
every \cs{ac} like command falls back to a simple text string typesetting what
\cs{acs} would do (or \cs{acsp} for plural forms).  These text strings both
can accessed manually and can be modified to an output reserved for \acs{pdf}
bookmarks.

\begin{commands}
  \command{acpdfstring}[\marg{id}]
    Access the text string used in \acs{pdf} bookmarks.
  \command{acpdfstringplural}[\marg{id}]
    Access the plural form of the text string used in \acs{pdf} bookmarks.
\end{commands}
\begin{options}
  \keylit{pdfstring}{\Marg{\meta{pdfstring}/\meta{plural ending}}}
    Key for \cs{DeclareAcronym} to declare a custom text string for \acs{pdf}
    bookmarks.  The plural ending can be set optionally.
\end{options}

For example the \acs{pdf} acronym used in the title for this section is defined
as follows:
\begin{sourcecode}
  \DeclareAcronym{pdf}{
    short     = pdf ,
    long      = Portable Document Format ,
    format    = \scshape ,
    pdfstring = PDF ,
    accsupp   = PDF
  }
\end{sourcecode}

\subsubsection{Accessibility Support}

The last example also demonstrates the \option{accsupp} key.  The idea is to
have something different visible in the \acs{pdf} file compared to what you
get when you select and copy the corresponding string.  In the example visible
string is a lowercase \code{pdf} in small caps while the string copied is an
uppercase \code{PDF}.

For this to work you need to use the \emph{package option} \option{accsupp},
too, which will load the package \pkg{accsupp} if it isn't loaded by the user
already.  Then the key \option{accsupp} will set the \code{ActualText}
property of \cs*{BeginAccSupp}.  Please refer to \pkg{accsupp}'s documentation
for details.  To see its effect copy \ac{pdf} and paste it into a text file.
You should get uppercase letters instead of lowercase ones.

\begin{options}
  \keyval{accsupp}{text}
    Key for \cs{DeclareAcronym} to set the \code{ActualText} property of
    \cs*{BeginAccSupp} (see \pkg{accsupp}'s documentation for details) to be
    used for an acronym.  It only has an effect when the package option
    \option{accsupp} is used, too.
\end{options}

\subsubsection{Tooltips}

The idea of a tooltip is to have some text shown when you hover with the
mouse over the short form of an acronym.  This is only available in some
\acs{pdf} viewers, though.  On possibility for such tooltips is loading the
\pkg{pdfcomment} package~\cite{pkg:pdfcomment} and using its \cs*{pdftooltip}
macro.

\begin{options}
  \keybool{tooltip}\Default{false}
    This\sinceversion{2.1} options loads the \pkg{pdfcomment} package and sets
    the command for creating tooltips to \cs*{pdftooltip}.
  \keyval{tooltip-cmd}{control sequence}\Default{\cs*{@firstoftwo}}
    This\sinceversion{2.1} allows users using another macro for tooltips --
    maybe one provided by another package or some own macro.  It needs to be a
    macro with two mandatory arguments, the first being the string typeset in
    the \acs{pdf}, the second being the tooltip description text.
\end{options}

For using this with acronyms they have a property \option{tooltip} which can
be used inside \cs{DeclareAcronym} for specifying the description text of the
tooltip.  If the \option{tooltip} package option is used but the property is
not set for an acronym then the \option{long} property is used instead.

If an acronym is used inside of another acronym then the tooltips of the
``inner'' acronyms are disabled.

\subsection{Adding Acronyms to the Index}\label{ssec:index}
\acro\ has the package option \option{index}.  If it is used an index entry
will be recorded every time an \emph{unstarred} acronym command is used.  The
index entry will be \code{\meta{id}@\meta{short}},
\code{\meta{sort}@\meta{short}} if the \option{sort} key has been set,
\code{\meta{index-sort}@\meta{short}} if the \option{index-sort} has been set,
or \meta{index} if the key \option{index} has been set for the specific
acronym.  The short versions appearing there are formatted according to the
chosen format of the corresponding acronym, of course.

This document demonstrates the feature.  You can find every acronym that has
been declared in the index.  In order to allow flexibility the indexing
command can be chosen both globally via package option and individually for
every acronym.  This would allow to add acronyms to a specific index if more
than one index is used, for example with help of the \pkg*{imakeidx} package.

I'm not yet convinced this is a feature many people if anyone needs and if
they do if it is flexible enough.  If you have any thoughts on this I'd
appreciate an email.

\section{Printing the List}\label{sec:print_lists}
Printing the whole list of acronyms is easy: just place \cs{printacronyms}
where ever you want the list to be.
\begin{commands}
  \command{printacronyms}[\oarg{options}]
    Print the list of acronyms.
\end{commands}
The commands takes a few options, namely the following ones:
\begin{options}
  \keyval{include-classes}{list of classes}
    Takes a comma-separated list of the classes of acronyms that should be in
    the list.
  \keyval{exclude-classes}{list of classes}
    Takes a comma-separated list of the classes of acronyms that should
    \emph{not} be in the list.  \emph{Note that this list overwrites any
      entries in \option{include-classes}!} If a class is both included and
    excluded then the corresponding acronyms will not be added to the list.
  \keyval{name}{name of the list}
    sets the name for the list.
  \keyval{heading}{sectioning command without leading backslash}%
    \Default{section*}
    Sets the sectioning command for the heading of the list.  A special value
    is \code{none} which suppresses the heading. 
  \keybool{sort}\Default{true}
    Set sorting for this list only.
  \keybool{local-to-barriers}\Default{false}
    This\sinceversion{2.4} option can be used to create a list of only the
    acronyms of the current \enquote{barrier group}, see
    section~\ref{sec:divid-your-docum}.
\end{options}
\begin{example}
  \acsetup{extra-style=comma}
  \printacronyms[exclude-classes=city]
 
  \printacronyms[include-classes=city,name={City Acronyms}]
\end{example}

You can see that the default layout is a \code{description} list with a
\cs*{section}\sarg\ title.  Both can be changed, see
section~\vref{sec:customization}.

The command \cs{printacronyms} needs two \LaTeX{} runs.  This is a precaution
to avoid error messages with a possibly empty list.  But since almost all
documents need at least two runs and often are compiled much more often than
that, this fact shouldn't cause too much inconvenience.

\section{Options and Customization}\label{sec:customization}
\subsection{General Options}
There are a few options which change the general behaviour of \acro.
\default{Underlined} values are used if no value is given.
\begin{options}
  %%
  % \keychoice{version}{0,1}\Default{1}
  %   Provide backwards compatibility for documents set with \acro\ in a version
  %   prior to v1.0.
  %%
  \keychoice{messages}{silent,loud}\Default{loud}
    Setting \keyis{messages}{silent} will turn all of \acro's error messages
    into warnings and all of \acro's warnings into info messages.  Be sure to
    check the log file carefully if you decide to set this option.
  \keybool{single}\Default{false}
    If set to \code{true} an acronym that's used only once (with \cs{ac}) in a
    document will only print the acronym in a specified form and will not be
    printed in the list.
  %%
  \keychoice{single-form}{long,short,alt,first}\Default{long}
    \sinceversion{2.0}Determines how a single appearance of an acronym is
    printed if \keyis{single}{true} has been chosen.
  %%
  \keybool{hyperref}\Default{false}
    If set to \code{true} the short forms of the acronyms will be linked to
    their list entry.
  %%
  \keybool{label}\Default{false}
    If set to \code{true} this option will place
    \cs*{label}\Marg{\meta{prefix}\meta{id}} the first time the acronym with
    \ac{id} \meta{id} is used. 
  %%
  \keyval{label-prefix}{text}\Default{ac:}
    The prefix for the \cs*{label} that is placed when option
    \keyis{label}{true} is used.
  %%
  \keybool{only-used}\Default{true}
    This option is \code{true} as default.  It means that only acronyms that
    are actually used in the document are printed in the list.  If
    \code{false}, all acronyms defined with \cs{DeclareAcronym} will be
    written to the list.
  %%
  \keychoice{mark-as-used}{first,any}\Default{any}
    This option determines wether an acronym is mark as used when the
    \emph{first} form is used the first time (with \cs{ac}, \cs{acf} or
    \cs{acflike} and their uppercase, plural and indefinite forms) or when any
    of the \cs{ac}-like commands is used.   
  %%
  \keybool{macros}\Default{false}
    If set to \code{true} this option will create a macro \cs*{\meta{id}} for
    each acronym as a shortcut for \cs{ac}\marg{id}.  Already existing macros
    will \emph{not} be overwritten.
  %%
  \keybool{xspace}\Default{false}
    If set to \code{true} this option will append \cs*{xspace} from the
    \pkg*{xspace} package to the commands created with the \option{macros}
    option.
  %%
  \keybool{strict}\Default{false}
    If set to \code{true} and the option \keyis{macros}{true} is in effect
    then already existing macros will be overwritten.
  %%
  \keybool{sort}\Default{true}
    If set to \code{true} the acronym list will be sorted automatically.  The
    entries are sorted by their \acs{id} ignoring upper and lower case.  This
    option needs the experimental package \pkg{l3sort} (from the
    \pkg{l3experimental} bundle) and can only be set in the preamble.
  %%
  \keychoice{cite}{\default{all},first,none}\Default{first}
    \changedversion{2.4b}This option decides whether citations that are added
    via \option{cite} are added to each first, every or no appearance of an
    acronym.  If \code{first} is chosen, the option \keyis{single}{true} is
    active and an acronym appears only once it still will get the citation.
  %%
  \keyval{cite-cmd}{control sequence}\Default{\cs*{cite}}
    This option determines which command is used for the citation.  Each
    citation command that takes the cite key as argument is valid, for example
    \pkg*{biblatex}'s \cs*{footcite}.
  %%
  \keyval{cite-connect}{code}\Default{\cs*{nobreakspace}}
    Depending on the citation command in use a space should be inserted before
    the citation or maybe not (e.g.\ \cs*{footcite}\ldots).  This option
    allows you to set this.  Actually it can be used to place arbitrary code
    right before the citation.
  %%
  \keybool{group-citation}\Default{false}
    \sinceversion{2.0}If set to true the short form (or the long form) and the
    citation of an acronym will be printed together in parentheses when an
    acronym is used the first time.
  %%
  \keyval{group-cite-cmd}{control sequence}\Default{\cs*{cite}}
    \sinceversion{2.0}This option determines which command is used for the
    citation when an acronym is used the first time \emph{and}
    \keyis{group-citation}{true}.  Each citation command that takes the cite
    key as argument is valid, for example \pkg*{biblatex}'s \cs*{footcite}.
  %%
  \keybool{index}\Default{false}
    If set to \code{true} an index entry will be recorded every time an
    \emph{unstarred} acronym command is used for the corresponding acronym.
  %%
  \keyval{index-cmd}{control sequence}\Default{\cs*{index}}
    Chooses the index command that is used when option \option{index} has been
    set to \code{true}.
  %%
  \keybool{accsupp}\Default{false}
    Activates the access support as provided by the \pkg{accsupp} package.
  %%
  \keybool{tooltip}\Default{false}
    \sinceversion{2.1}Activates tooltip support for \acro\ using the
    \pkg{pdfcomment} package.
  %%
  \keyval{tooltip-cmd}{control sequence}\Default{\cs*{@firstoftwo}}
    \sinceversion{2.1}A macro taking two mandatory arguments, the first being
    the short form of the acronym and the second being some tooltip
    description.
  %%
  \keyval{uc-cmd}{control sequence}\Default{\cs{acfirstupper}}
    The command that is used to capitalize the first word in the \cs{Ac} and
    the like commands.  You can change it to another one like for example
    \cs*{makefirstuc}\footnote{from the \pkg{mfirstuc} package} or
    \cs*{MakeTextUppercase}\footnote{from the \pkg*{textcase} package}.
\end{options}
 
All options of this and the following sections can be set up either as package
options or via the setup command:
\begin{commands}
  \command{acsetup}[\marg{options}]
   Set up \acro\ anywhere in the document.  Or separate package loading from
   setup.
\end{commands}

\begin{example}
  % with \acsetup{macros}
  we could have used these before: \nato, \ny
\end{example}

\subsection{Options Regarding Acronyms}\label{sec:opti-regard-acronyms}
The options described in this section all influence the layout of one of the
possible output forms of the acronyms.
\begin{options}
  %%
  \keyval{short-format}{format}\Default
    Sets a format for all short forms. For example
    \keyis{short-format}{\cs*{scshape}} would print all short forms in small
    caps.
  %%
  \keyval{long-format}{format}\Default
    The same for the long forms.
  %%
  \keyval{foreign-format}{format}\Default
    The format for the \option{foreign} entry when it appears as part of the
    first appearance of an acronym.
  %%
  \keyval{single-format}{format}\Default
    \sinceversion{2.3}%
    The format for the acronym when it is used only once.  If not specified
    the formatting according to \option{single-form} is used.
  %%
  \keyval{first-long-format}{format}\Default
    The format for the long form on first usage (with \cs{ac}, \cs{acf} or
    \cs{acflike} and their uppercase, plural and indefinite forms).
  %%
  \keyval{list-short-format}{format}\Default
    An extra format for the short entries in the list.  If not used this is
    the same as \option{short-format}.  Please be aware that a call of
    \option{short-format} after this one will overwrite it again.
  %%
  \keyval{list-short-width}{dim}\Default{3em}
    \sinceversion{2.1}This option controls the width reserved for the short
    forms of the acronyms in the \code{lof} list style.
  %%
  \keyval{list-long-format}{format}\Default
    An extra format for the long entries in the list.  If not used this is the
    same as \option{long-format}.  Please be aware that a call of
    \option{long-format} after this one will overwrite it again.
  %%
  \keyval{list-foreign-format}{format}\Default{\cs{acroenparen}}
    \label{key:list-foreign-format}The format for the \option{foreign} entry
    as it appears in the list.  This may be code that ends with a macro that
    takes a mandatory argument.
  %%
  \keyval{extra-format}{format}\Default
    The same for the extra information.
  %%
  \keychoice{first-style}{default,empty,square,short,long,reversed,footnote,sidenote,%
    footnote-reversed,sidenote-reversed}\Default{default}
    The basic style of the first appearance of an acronym.  The value
    \code{sidenote} needs the command \cs*{sidenote} to be defined (for
    example by the \pkg*{sidenotes} package).
  %%
  \keychoice{extra-style}{default,plain,comma,paren,bracket}\Default{default}
    Defines the way the extra information is printed in the list.
  %%
  \keylit{plural-ending}{\meta{short}/\meta{long}}\Default{s/s}
    \changedversion{2.4b}With this option the default plural ending can be
    set.  The appended \meta{long} ending is optional.  If you leave it
    (\emph{and} the \code{/}) the \meta{short} ending is used for both short
    and long versions.
\end{options}
 
\begin{example}[side-by-side]
  % (Keep in mind that we're in
  % a minipage here!)
  \acsetup{first-style=empty}
  empty: \acf{ny} \\
  \acsetup{first-style=footnote}
  footnote: \acf{ny} \\
  \acsetup{first-style=square}
  square: \acf{ny} \\
  \acsetup{first-style=short}
  short: \acf{ny} \\
  \acsetup{first-style=long}
  long: \acf{ny} \\
  \acsetup{first-style=reversed}
  reversed: \acf{ny} \\
  \acsetup{
    first-style = footnote-reversed
  }
  footnote-reversed: \acf{ny}
\end{example}

\subsection{Options Regarding the List}
\begin{options}
  %%
  \keychoice{page-style}{none,plain,comma,paren}\Default{none}
    If this option is set to a value other than \code{none} the page numbers
    of the an acronym appeared on are printed in the list.  Please note that
    this is an experimental feature and might fail in quite a number of cases.
    If you notice anything please send me an email!
  %%
  \keychoice{pages}{all,first}\Default{all}
    If the option \option{page-style} has any value other than \code{none}
    this option determines wether all usages of the acronyms are listed or
    only the first time.  Implicitly sets \keyis{label}{true}.
  %%
  \keyval{page-name}{page name}\Default{p.\cs*{@}\cs*{,}}
    The ``name'' of the page label.  This is automatically translated to the
    active language. However for the time being there are many translations
    missing, yet.  Please notify me if you find your language missing.
  %%
  \keyval{pages-name}{page name plural}\Default{pp.\cs*{@}\cs*{,}}
    The ``name'' of the page label when there are more than one page.  This is
    automatically translated to the active language.  However for the time
    being there are many translations missing, yet.  Please notify me if you
    find your language missing. 
  %%
  \keybool{following-page}\Default{false}
    If set to \code{true} a page range in the list of acronyms that consists
    of two pages will be written by the first page and an appended
    \code{f}. This depends on the option \option{next-page}. 
  %%
  \keybool{following-pages}\Default{false}
    If set to \code{true} a page range in the list of acronyms that set
    consists of more than two pages will be written by the first page and an
    appended \code{ff}. This depends on the option \option{next-pages}.
  %%
  \keybool{following-pages*}\Default{false}
    \sinceversion{2.5}If set to \code{true} this sets both options
    \keyis{following-page}{true} and \keyis{following-pages}{true}.
    \code{false} sets \keyis{following-page}{false} and
    \keyis{following-pages}{false}.
  %%
  \keyval{next-page}{text}\Default{\cs*{,}f.\cs*{@}}
    Appended to a page number when \option{following-page} is set to
    \code{true} and the range is only 2 pages long.  This is automatically
    translated to the active language.  However, for the time being there are
    many translations missing, yet.  Please notify me if you find your
    language missing.
  %%
  \keyval{next-pages}{text}\Default{\cs*{,}ff.\cs*{@}}
    Appended to a page number when \option{following-pages} is set to
    \code{true} and the range is more than 2 pages long.  This is
    automatically translated to the active language.  However, for the time 
    being there are many translations missing, yet.  Please notify me if you
    find your language missing.
  %%
  \keychoice{list-style}{description,lof,longtable,extra-longtable,%
    extra-longtable-rev,extra-tabular,extra-tabular-rev,tabular,toc}%
    \Default{description}
    \changedversion{2.2}Choose with which style the list of acronyms should be
    typeset.  If you choose \meta{longtable}, \code{extra-longtable} or
    \code{extra-longtable-rev} you have to load the
    \pkg{longtable}~\cite{pkg:longtable} package in your preamble.  The values
    \code{extra-\meta{something}} put the extra information in a column of it
    own.  \emph{Be aware that per default \emph{all} \code{extra-table} styles
      only use \code{l} columns.  Since acronym descriptions can easily get
      longer that a line you should probably define your own style if you want
      to use them.}  See section~\vref{sec:lists} for details.
  %%
  \keychoice{list-heading}{chapter,chapter*,section,section*,subsection,%
    subsection*,subsubsection,subsubsection*,addchap,addsec,none}%
  \Default{section*}
    \changedversion{2.0}The heading type of the list. The last two only work
    with a \KOMAScript{} class that also defines the appropriate command.  A
    special value is \code{none} which suppresses the heading.
  %%
  \keyval{list-name}{list name}\Default{Acronyms}
    The name of the list.  This is what's written in the list-heading.  This
    is automatically translated to the active language.  However, for the time
    being there are many translations missing, yet.  Please notify me if you
    find your language missing.
  %%
  \keybool{list-caps}\Default{false}
    Print the first letters of the long form capitalized.
\end{options}

\section{Trailing Tokens and Special Action}

\acro\ has the possibility\sinceversion{2.0} to look ahead for certain tokens
and switch a boolean if it finds them.  Per default \acro\ knows about three
tokens: the \enquote{\code{dot}} (\code{.}), the \enquote{\code{dash}}
(\code{-}) and the \enquote{\code{babel-hyphen}} (\cs*{babelhyphen}).

A token is made known to \acro\ with the following macro:
\begin{commands}
  \command{AcroRegisterTrailing}[\meta{token}\marg{name}]
    This registers the token \meta{token} so \acro\ looks if it follows
    directly after an acronym macro.  \meta{name} is the internal name for
    this token.
\end{commands}
The \acro\ package already registers the above mentioned tokens:
\begin{sourcecode}
  \AcroRegisterTrailing . {dot}
  \AcroRegisterTrailing - {dash}
  \AcroRegisterTrailing \babelhyphen {babel-hyphen}
\end{sourcecode}

If a token is registered it doesn't mean that \acro\ looks for it.  The token
must first be activated for this:
\begin{options}
  \keyval{activate-trailing-tokens}{csv list of token names}
    Tell \acro\ to look for trailing tokens.  This is done by giving a csv
    list of the internal \emph{names} of the tokens.  Per default only
    \code{dot} is activated.
  \keyval{deactivate-trailing-tokens}{csv list of token names}
    Tell \acro\ not to look for trailing tokens.  This is done by giving a csv
    list of the internal \emph{names} of the tokens.
\end{options}

All of the above on its own does nothing visible. However: inside of an
acronym, \ie, for example inside the long or the short form it can be tested
for those trailing tokens:
\begin{commands}
  \command{aciftrailing}[\marg{csv list of token
    names}\marg{true}\marg{false}]
    Check if one of the tokens listed in \meta{csv list of token names} is
    following and either place \meta{true} or \meta{false} in the input
    stream.
\end{commands}
\acro\ uses this to define to further macros:
\begin{commands}
  \command{acdot} Inserts a \code{.} if no \code{dot} follows.
  \command{acspace} Inserts a \cs*{space} if no \code{dash} or
    \code{babel-hyphen} follows.
\end{commands}
The definitions are equivalent\footnote{Not \emph{quite}: \acro's definitions
  are engine protected.} to the following code:
\begin{sourcecode}
  \newcommand*\acdot{\aciftrailing{dot}{}{.\@}}
  \newcommand*\acspace{\aciftrailing{dash,babel-hyphen}{}{\space}}
\end{sourcecode}

This could be used to define an acronym as follows:
\begin{sourcecode}
  \DeclareAcronym{etc}{
    short = {\textit{etc}\acdot} ,
    long  = {\textit{et cetera}} ,
    short-plural = , long-plural =
  }
\end{sourcecode}
If now you somewhere use
\begin{sourcecode}
  \ac{etc}.
\end{sourcecode}
there won't be two dots printed.

The command \cs{acspace} is used already in the definition of the first
appearance of a macro.  Let's say you're a German chemist and you have
\begin{sourcecode}
  \DeclareAcronym{PU}{
    long = Polyurethan ,
    long-plural = e
  }
\end{sourcecode}
and you use it the first time like this:
\begin{sourcecode}
  \ac{PU}-Hartschaum
\end{sourcecode}
then according to German orthography and typesetting rules this should be
printed as
\begin{center}
  \enquote{Polyurethan(PU)-Hartschaum}
\end{center}
\ie, with \emph{no} space between long and short form.  This is exactly what
happens it you say
\begin{sourcecode}
  \acsetup{activate-trailing-tokens = {dash,babel-hyphen}}
\end{sourcecode}
in the preamble.

\section{About Page Ranges}
If you enable the \option{page-style} option \acro\ adds page numbers to the list
of acronyms.  In version~0.\versionstar{} it would add a page reference for an
acronym in the list of acronyms that used \cs*{pageref} to refer to the first
appearance of an acronym.  This is retained using \keyis{pages}{first}.
Version~1.0 uses a different approach that doesn't use a label but instead
will list \emph{all} pages an acronym appeared on.  With \pkg{hyperref} the
pages are referenced using \cs*{hyperpage}.

There are some options that control how this list will be typeset, e.g.,
\option{following-page}, \option{next-pages} or the option \option{page-style}
itself.  It is important to mention that the page list will always take at
least two compilation runs until changes in the options or the actual page
numbers affect it.  This is due to the fact that the updated sequence is first
written to the \code{aux} file and only read in during the next run.

\section{Dividing Your Document Into Pieces -- Creating Local
  Lists}\label{sec:divid-your-docum}

\acro\sinceversion{2.4} introduces the concept of \emph{barriers} which can
divide the document into parts. It is possible to create lists of only those
acronyms used between two such barriers.
\begin{commands}
  \command{acbarrier}
    Sets a barrier at the point of use in the document.  The begin and the end
    of the document mark implicit barriers.
\end{commands}
\begin{options}
  \keybool{use-barriers}\Default{false}
    \sinceversion{2.5}If you want to use barriers and local lists you have to
    activate the feature first.  This should be set in the preamble in order
    to work reliably.  Make sure to watch out for log file messages asking you
    to rerun.
  \keybool{reset-at-barriers}\Default{false}
    If this option is set to \code{true} \cs{acbarrier} implicitly calls
    \cs{acresetall}.
  \keybool{local-to-barriers}\Default{false}
    This option can \emph{only} be used as option to the \cs{printacronyms}
    command. It then prints a list of only the acronyms of the current
    \enquote{barrier group}.
\end{options}

\begin{example}
  \acbarrier
  \printacronyms[local-to-barriers]
  \ac{ctan} and \ac{lppl}
  \acbarrier
\end{example}

\section{Language Support}
\acro\ detects if packages \pkg{babel}~\cite{pkg:babel} or
\pkg{polyglossia}~\cite{pkg:babel} are being loaded and tries to adapt certain
strings to match the chosen language.  However, due to my limited language
knowledge only a few translations are provided.  I'll show how the English
translations are defined so you can add the translations to your preamble if
needed.  Even better would be you'd send me a short email to
\mailto{contact@mychemistry.eu} with the appropriate translations for your
language and I'll add them to \acro.

\begin{sourcecode}
  \DeclareTranslation{English}{acronym-list-name}{Acronyms}
  \DeclareTranslation{English}{acronym-page-name}{p.}
  \DeclareTranslation{English}{acronym-pages-name}{pp.}
  \DeclareTranslation{English}{acronym-next-page}{f.}
  \DeclareTranslation{English}{acronym-next-pages}{ff.}
\end{sourcecode}

\section{hyperref Support}
The option \keyis{hyperref}{true} adds internal links from all short (or
alternative) forms to their respective list entries.  Of course this only
works if you have loaded the \pkg{hyperref} package in your preamble.  You
should use this option with care: if you don't use \cs{printacronyms} anywhere
this option will result in loads of \pkg{hyperref} warnings.  Also printing
several lists can result in warnings if don't clearly separate the lists into
different classes.  If an acronym appears in more than one list there will
also be more than one hypertarget for this acronym.

Using \pkg{hyperref} will also add \cs*{hyperpage} to the page numbers in the
list (provided they are displayed in the style chosen).  Like with an index
the references will thus not point to the acronyms directly but to the page
they're on.

\section{Defining Own Acronym Macros}\label{sec:defining-own-acronym}

The commands\sinceversion{2.0} explained in
section~\vref{sec:using-acronyms-comm} have all been defined with a dedicated
command -- there is a family of dedicated commands, actually:
\begin{commands}
  \command{NewAcroCommand}[\marg{cs}\marg{code}]
    Defines a new \acro\ acronym command \meta{cs}.  This sets up the
    necessary framework needed by acronym commands and defines \meta{cs} with
    an optional star argument and a mandatory argument for the acronym id
    using \pkg{xparse}'s \cs*{NewDocumentCommand}. Inside \meta{code} one can
    refer to the \acs{id} \meta{id} with \code{\#1}.
  \command{RenewAcroCommand}[\marg{cs}\marg{code}]
    Defines a new \acro\ acronym command \meta{cs}.  This sets up the
    necessary framework needed by acronym commands and defines \meta{cs} with
    an optional star argument and a mandatory argument for the acronym id
    using \pkg{xparse}'s \cs*{RenewDocumentCommand}. Inside \meta{code} one
    can refer to the \acs{id} \meta{id} with \code{\#1}.
  \command{DeclareAcroCommand}[\marg{cs}\marg{code}]
    Defines a new \acro\ acronym command \meta{cs}.  This sets up the
    necessary framework needed by acronym commands and defines \meta{cs} with
    an optional star argument and a mandatory argument for the acronym id
    using \pkg{xparse}'s \cs*{DeclareDocumentCommand}. Inside \meta{code} one
    can refer to the \acs{id} \meta{id} with \code{\#1}.
  \command{ProvideAcroCommand}[\marg{cs}\marg{code}]
    Defines a new \acro\ acronym command \meta{cs}.  This sets up the
    necessary framework needed by acronym commands and defines \meta{cs} with
    an optional star argument and a mandatory argument for the acronym id
    using \pkg{xparse}'s \cs*{ProvideDocumentCommand}. Inside \meta{code} one
    can refer to the \acs{id} \meta{id} with \code{\#1}.
\end{commands}

Inside these macros one can use a number of low-level expl3
commands\footnote{Which is why you need to use them inside an expl3
  programming environment.  This means in the preamble surround the
  definitions with \cs*{ExplSyntaxOn} and \cs*{ExplSyntaxOff}.}.

\paragraph{Acronym fetching commands}
\begin{commands}
  \command*{acro_use:n}[ \marg{id}]
    Fetches the acronym using either the first or the short form depending on
    earlier uses.
  \command*{acro_short:n}[ \marg{id}]
    Fetches the short form of the acronym.
  \command*{acro_long:n}[ \marg{id}]
    Fetches the long form of the acronym.
  \command*{acro_alt:n}[ \marg{id}]
    Fetches the alternative short form of the acronym.
  \command*{acro_foreign:n}[ \marg{id}]
    Fetches the foreign property of the acronym if available.
  \command*{acro_extra:n}[ \marg{id}]
    Fetches the extra property of the acronym if available.
\end{commands}

\paragraph{Acronym setup commands}
\begin{commands}
  \command*{acro_first_upper:}
    \acro\ setup command which tells the macros above that we want to
    uppercase the first letter of the long version.  Should be used
    \emph{before} one of the acronym fetching commands.
  \command*{acro_plural:}
    \acro\ setup command which tells the macros above that we want to use
    plural forms. Should be used \emph{before} one of the acronym fetching
    commands.
  \command*{acro_indefinite:}
    \acro\ setup command which tells the macros above that we want to add the
    indefinite article.  Should be used \emph{before} one of the acronym
    fetching commands.
  \command*{acro_cite:}
    \acro\ setup command which tells the macros above that we want to add the
    citation in any case independent of the option \option{cite}.  Should be
    used \emph{before} one of the acronym fetching commands.
  \command*{acro_no_cite:}
    \acro\ setup command which tells the macros above that we want to have no
    citation independent of the option \option{cite}.  Should be used
    \emph{before} one of the acronym fetching commands.
  \command*{acro_index:}
    \acro\ setup command which tells the macros above that we want to add an
    index entry in any case independent of the option \option{index}.  Should
    be used \emph{before} one of the acronym fetching commands.
  \command*{acro_reset_specials:}
    This\sinceversion{2.0b} macro is called implicitly by \cs{NewAcroCommand}
    and \cs{NewPseudoAcroCommand}.  If you plan to define an \acro\ command by
    yourself using \cs*{NewDocumentCommand} this should be the first macro
    after \verbcode+\acro_begin:+.  It ensures that in nested acronyms the
    inner acronyms don't inherit indefinite articles, uppercasing,
    endings\ldots
\end{commands}

\paragraph{Additional macros for further uses}
\begin{commands}
  \command*{acro_begin:}
    When an acronym macro is defined \enquote{by hand}, \ie, \emph{not using
    \cs{NewAcroCommand}} then this must be the first macro in the code.
    \emph{Must have a matching \cs*{acro_end:}}.
  \command*{acro_end:}
    When an acronym macro is defined \enquote{by hand}, \ie, \emph{not using 
    \cs{NewAcroCommand}} then this must be the last macro in the code.
    \emph{Must have a matching \cs*{acro_begin:}}.
  \command*{acro_check_and_mark_if:nn}[ \marg{boolean expression} \marg{id}]
    Checks if the acronym with the \acs{id}  \meta{id} exists and marks it as
    used when \meta{boolean} expression evaluates to \code{true}.  This macro
    is used inside \cs{NewAcroCommand} and friends implicitly.
  \command*{acro_check_acronym:nn}[ \marg{id} \Marg{true|false}]
    Checks if the acronym with the \acs{id} \meta{id} exists and marks it as
    used if \code{true} or doesn't.  This macro is used inside
    \cs*{acro_check_and_mark_if:nn}.
  \command*{acro_use_acronym:n}[ \Marg{true|false}]
    Tell \cs*{acro_use:n} and similar commands wether to mark the acronym as
    used or not.  This macro is used inside \cs*{acro_check_acronym:nn}.  If
    this macro is used explicitly it should be used before \cs*{acro_use:n}
    (or a similar command) otherwise it has no effect.  An acronym marked as
    used cannot be unmarked.
  \command*{acro_mark_as_used:n}[ \marg{id}]
    Explicitly use the acronym with the \acs{id} \meta{id}.  This is the expl3
    macro applied to all entries in \cs{acuse}.
  \expandable\command*{acro_if_acronym_used:n}[\TF\ \marg{id} \marg{true}
  \marg{false}]
    The code-level version of \cs{acifused}.  This macro is expandable.
  \command*{acro_for_all_acronyms_do:n}[ \marg{code}]
    Loops over all acronyms known when the macro is used.  Inside of
    \meta{code} you can refer to the \acs{id} \meta{id} of an acronym with
    \verbcode+#1+.
  \command*{acro_barrier:}
    The code-level version of \cs{acbarrier}.
  \command*{acro_switch_off:}
    The\sinceversion{2.6} expl3 version of \cs{acswitchoff}.
  \command*{acro_switch_on:}
    The\sinceversion{2.6} expl3 version of \cs{acswitchon}.
  \command*{acro_add_action:n}[ \marg{code}]
    Adds\sinceversion{2.7} code to \cs*{acro_get:n}.  Inside of \meta{code}
    you can refer to the \ac{id} of the acronym with \verbcode+#1+.
  \command*{acro_get_property:nn}[\TF\ \marg{id} \marg{property} \marg{true}
    \marg{false}]
    Fetches\sinceversion{2.7} the property \meta{property} of the acronym
    \meta{id} and stores it in a tokenlist variable
    \cs*{l__acro_\meta{property}_tl} where all dashes in the property names
    are replaced with underscores.  \meta{true} is placed in the input stream
    if the property had been set, \meta{false} otherwise.
  \command*{acro_get_property:nn}[\marg{id} \marg{property}]
    Like\sinceversion{2.7} \cs*{acro_get_property:nn}\TF, but without the
    \meta{true} and \meta{false} arguments.
  \command*{acro_if_property:nn}[\TF\ \marg{id} \marg{property} \marg{true}
    \marg{false}]
    Checks\sinceversion{2.7} if the property \meta{property} of the acronym
    \meta{id} is set and places \meta{true} is in the input stream if yes and
    \meta{false} otherwise.
\end{commands}

\paragraph{Examples}
The usage of above macros is best explained with a few examples.  The
following definition is done by \acro:
\begin{sourcecode}
  \NewAcroCommand \ac { \acro_use:n {#1} }
\end{sourcecode}
An equivalent definition for \cs{ac} would be
\begin{sourcecode}
  \NewDocumentCommand \ac {sm}
    {
      \acro_begin:
        \acro_reset_specials:
        \acro_check_and_mark_if:nn {#1} {#2}
        \acro_use:n {#2}
      \acro_end:
    }
\end{sourcecode}
which should explain what the actual framework is which \cs{NewAcroCommand}
adds.

Other definitions by \acro\ are for example the following ones:
\begin{sourcecode}
  \NewAcroCommand \Ac
    {
      \acro_first_upper:
      \acro_use:n {#1}
    }
  \NewAcroCommand \iac
    {
      \acro_indefinite:
      \acro_use:n {#1}
    }
  \NewAcroCommand \acp
    {
      \acro_plural:
      \acro_use:n {#1}
    }
  \NewAcroCommand \Acp
    {
      \acro_plural:
      \acro_first_upper:
      \acro_use:n {#1}
    }
  \NewAcroCommand \Aclp
    {
      \acro_plural:
      \acro_first_upper:
      \acro_long:n {#1}
    }
\end{sourcecode}

\section{About Plural Forms, Possessive Forms and Similar Constructs -- the
  Concept of Endings}

\acro\ has a concept of \emph{endings}.  All of \acro's plural options are
defined by saying
\begin{sourcecode}
  \ProvideAcroEnding {plural} {s} {s}
\end{sourcecode}
The command's syntax and what it does is as follows:
\begin{commands}
  \command{ProvideAcroEnding}[\marg{name}\marg{short default}\marg{long default}]
    This macro defines the options
    \begin{itemize}
      \item \option*{\meta{name}-ending},
      \item \option*{short-\meta{name}-ending},
      \item \option*{alt-\meta{name}-ending} and
      \item \option*{long-\meta{name}-ending}.
    \end{itemize}
    It also defines the acronym properties
    \begin{itemize}
      \item \option*{short-\meta{name}},
      \item \option*{short-\meta{name}-form},
      \item \option*{alt-\meta{name}},
      \item \option*{alt-\meta{name}-form},
      \item \option*{long-\meta{name}} and
      \item \option*{long-\meta{name}-form}.
    \end{itemize}
    Additionally it defines a setup macro as
    described in section~\vref{sec:defining-own-acronym},
    \cs*{acro_\meta{name}:}.  If \meta{name} contains a \code{-} (dash) it is
    replaced by \code{\_} before \cs*{acro_\meta{name}:} is built.  So if you
    choose \code{my-name} the corresponding macro is named
    \cs*{acro_my_name:}.  If you use any other non-letters you are on your
    own. \changedversion{2.4b}If you use the command with the same \meta{name}
    a second time the command only resets the defaults.

    Note that you \emph{must use \cs{ProvideAcroEnding} before any acronym
      definition}!
\end{commands}

This could be used together with the macros described in
section~\vref{sec:defining-own-acronym} for adding support for possessive
forms:
\begin{example}
  \ExplSyntaxOn
  % this now only works because I've used the same already in the preamble so
  % it does nothing here:
  \ProvideAcroEnding {possessive} {'s} {'s}

  \ProvideAcroCommand \acg
    {
      \acro_possessive:
      \acro_use:n {#1}
    }
  \ExplSyntaxOff
  The \acg{cd} booklet says\ldots
\end{example}

Please note that different endings are cumulative which you probably want to
avoid! Imagine a macro
\begin{sourcecode}
  \NewAcroCommand \acgp
    {
      \acro_possessive:
      \acro_plural:
      \acro_use:n {#1}
    }
\end{sourcecode}
This would give \enquote{\ac{cd}s's} instead of \enquote{\ac{cd}s'}.  To solve
this you might want to consider
\begin{sourcecode}
    \ProvideAcroEnding {possessive-singular} {'s} {'s}
    \ProvideAcroEnding {possessive-plural}   {s'} {s'}
\end{sourcecode}

\section{More on Customization}\label{sec:more-custom}
\subsection{Background Information}
Several of \acro's objects are customized using templates.  For each of these
objects it is possible to define own templates\footnote{This requires some
knowledge of \pkg{xtemplate} and expl3.  Plans are to provide a documented
interface for users of \acro{} in the future.}.  Possibly more interesting: it
is easily possible to define further instances of an object using a certain
template.  How this works is explained in the following sections. However, the
basics are always the same.  There is a command
\begin{commands}
  \command*{DeclareAcro\meta{object
      type}Style}[\marg{name}\marg{template}\marg{options}]
    which allows to define a new style (\ie, instance) for the object
    \meta{object type} using the template \marg{template}.
\end{commands}

\subsection{Lists}\label{sec:lists}
\subsubsection{Own List Style}

The different existing list styles are all built from four different
templates, \code{list}, \code{list-of}, \code{table} and \code{extra-table}.
Those templates are defined with the help of the \pkg{xtemplate}
package (from~\cite{bnd:l3packages}).  Each of these templates has a few
options which are described in table~\vref{tab:list-templates}.  New list
styles now are defined via the following macro:

\begin{commands}
  \command{DeclareAcroListStyle}[\marg{name}\marg{template}\marg{options}]
    Declares a new \acro\ list style \meta{name}.  \meta{name} will be the
    value which can be chosen in the option \option{list-style}.
    \meta{template} is the name of the template to be used by the style.
    Available templates are listed in table~\ref{tab:list-templates}.
    \meta{options} are the option settings for the corresponding template.
\end{commands}

\begin{table}[hbp]
  \centering
  \caption{Available List Templates and Their Options}\label{tab:list-templates}
  \begin{tabular}{*{4}{>{\ttfamily}l}}
    \toprule
      \normalfont\bfseries Template & \normalfont\bfseries Option &
      \normalfont\bfseries Option Type & \normalfont\bfseries Default \\
    \midrule
      list        & list        & tokenlist & description \\
                  & foreign-sep & tokenlist & \cs*{space} \\
                  & reverse     & boolean   & false \\
                  & before      & tokenlist \\
                  & after       & tokenlist \\
    \midrule
      list-of     & style       & tokenlist & toc \\
                  & foreign-sep & tokenlist & \cs*{space} \\
                  & reverse     & boolean   & false \\
                  & before      & tokenlist \\
                  & after       & tokenlist \\
    \midrule
      table       & table       & tokenlist & tabular \\
                  & table-spec  & tokenlist & lp\{.7\cs*{linewidth}\} \\
                  & foreign-sep & tokenlist & \cs*{space} \\
                  & reverse     & boolean   & false \\
                  & before      & tokenlist \\
                  & after       & tokenlist \\
    \midrule
      extra-table & table       & tokenlist & tabular \\
                  & table-spec  & tokenlist & llll \\
                  & foreign-sep & tokenlist & \cs*{space} \\
                  & reverse     & boolean   & false \\
                  & before      & tokenlist \\
                  & after       & tokenlist \\
    \bottomrule
  \end{tabular}
\end{table}

For defining new styles you need some information on what the different
templates and options do:
\begin{itemize}
  \item The option \code{list} of the \code{list} template sets the list
    environment.  This must be a classic \LaTeX\ list where items are listed
    with \cs*{item}.  In those lists short entries will always be fed as
    optional argument to \cs*{item}: \\
    \code{\cs*{item}[\meta{short}]\meta{long}\meta{extra}\meta{page}}
  \item The template \code{list-of} simulates a table of contents or a list of
    figures.  This can be chosen by setting the option \code{style} to either
    \code{toc} or \code{lof}.
  \item The template \code{table} typesets the list in a table with two
    columns: \\
    \code{\meta{short} \& \meta{long}\meta{extra}\meta{page}
      \cs*{tabularnewline}}
  \item The template \code{extra-table} typesets the list in a table with four
    columns: \\
    \code{\meta{short} \& \meta{long} \& \meta{extra} \& \meta{page}
      \cs*{tabularnewline}}
  \item The option \code{foreign-sep} is the code inserted between long form
    and foreign entry (if a foreign entry is present).
  \item The options \code{before} and \code{after} are inserted directly
    before and after the complete list.
  \item The option \code{reverse} switches the place of \meta{long} with
    \meta{extra}.
  \item The option \code{table-spec} sets the column types for the table
    templates.  It must correspond to the number of columns the corresponding
    template uses.
\end{itemize}

As an example let's define a style \code{longtabu} which uses the corresponding
table environment from the package \pkg{tabu}~\cite{pkg:tabu}:

\begin{sourcecode}
  \usepackage{tabu,longtable}
  \DeclareAcroListStyle{longtabu}{table}{
    table = longtabu ,
    table-spec = @{}>{\bfseries}lX@{}
  }
  \acsetup{list-style=longtabu}
\end{sourcecode}

As another example let's define a new list with the help of the \pkg{enumitem}
package~\cite{pkg:enumitem}:

\begin{sourcecode}
  % preamble:
  % \usepackage{enumitem}
  \newlist{acronyms}{description}{1}
  \newcommand*\addcolon[1]{#1:}
  \setlist[acronyms]{
    labelwidth = 3em,
    leftmargin = 3.5em,
    noitemsep,
    itemindent = 0pt,
    font=\addcolon}
  \DeclareAcroListStyle{mystyle}{list}{ list = acronyms }
  \acsetup{ list-style = mystyle }
\end{sourcecode}
This would look as follows:
\newlist{acronyms}{description}{1}
\newcommand*\addcolon[1]{#1:}
\setlist[acronyms]{
  labelwidth=3em,
  leftmargin=3.5em,
  noitemsep,
  itemindent=0pt,
  font=\addcolon}
\DeclareAcroListStyle{mystyle}{list}{ list = acronyms }
\acsetup{list-style=mystyle,hyperref=false}
\printacronyms

\subsubsection{Own List Heading Command}

With the option \option{list-heading} you can choose which command prints the
heading of the list.  If you need a different choice than what's already
provided you can use the following command to define a new option:
\begin{commands}
  \command{DeclareAcroListHeading}[\marg{name}\marg{control sequence}]
    Defines a new value \meta{name} for the option \option{list-heading}.
    \meta{control sequence} must be a control sequence which takes one
    mandatory argument.
\end{commands}
As an example here is how the value \code{section} is defined:
\begin{sourcecode}
  \DeclareAcroListHeading{section}{\section}
\end{sourcecode}

\subsection{First Styles}
The first styles define how an acronym is typeset when it is used for the
first time.  It is set with the option \option{first-style}.  Legal values for
this option are defined with the following command:
\begin{commands}
  \command{DeclareAcroFirstStyle}[\marg{name}\marg{template}\marg{options}]
    Declares a new \acro\ first style \meta{name}.  \meta{name} will be the
    value which can be chosen in the option \option{first-style}.
    \meta{template} is the name of the template to be used by the style.
    Available templates are listed in table~\ref{tab:first-templates}.
    \meta{options} are the option settings for the corresponding template.
\end{commands}

Here are two examples of the already available styles and how they are
defined:
\begin{sourcecode}
  \DeclareAcroFirstStyle{short}{inline}{
    only-short = true ,
    brackets   = false
  }
  \DeclareAcroFirstStyle{sidenote-reversed}{note}{
    note-command = \sidenote{#1} ,
    reversed     = true
  }
\end{sourcecode}

\begin{table}[hbp]
  \centering
  \caption{Available First Style Templates and Their Options}\label{tab:first-templates}
  \begin{tabular}{*{4}{>{\ttfamily}l}}
    \toprule
      \normalfont\bfseries Template & \normalfont\bfseries Option &
      \normalfont\bfseries Option Type & \normalfont\bfseries Default \\
    \midrule
      inline      & brackets      & boolean   & true \\
                  & brackets-type & tokenlist & () \\
                  & only-short    & boolean   & false \\
                  & only-long     & boolean   & false \\
                  & reversed      & boolean   & false \\
                  & between       & tokenlist \\
                  & foreign-sep   & tokenlist & ,\textasciitilde \\
    \midrule
      note        & use-note     & boolean   & true \\
                  & note-command & function  & \cs*{footnote}\Marg{\#1} \\
                  & reversed     & boolean   & false \\
                  & foreign-sep  & tokenlist & ,\textasciitilde \\
    \bottomrule
  \end{tabular}
\end{table}

\subsection{Extra Styles}
The extra styles define how the extra information of an acronym is typeset in
the list.  It is set with the option \option{extra-style}.  Legal values for
this option are defined with the following command:
\begin{commands}
  \command{DeclareAcroExtraStyle}[\marg{name}\marg{template}\marg{options}]
    Declares a new \acro\ extra style \meta{name}.  \meta{name} will be the
    value which can be chosen in the option \option{extra-style}.
    \meta{template} is the name of the template to be used by the style.
    Available templates are listed in table~\ref{tab:extra-templates}.
    \meta{options} are the option settings for the corresponding template.
\end{commands}

Here are two examples of the already available styles and how they are
defined:
\begin{sourcecode}
  \DeclareAcroExtraStyle{default}{inline}{
    brackets     = false ,
    punct        = true ,
    punct-symbol = .
  }
  \DeclareAcroExtraStyle{paren}{inline}{
    brackets     = true ,
    punct        = true ,
    punct-symbol =
  }
\end{sourcecode}

\begin{table}
  \centering
  \caption{Available Extra Style Templates and Their Options}\label{tab:extra-templates}
  \begin{tabular}{*{4}{>{\ttfamily}l}}
    \toprule
      \normalfont\bfseries Template & \normalfont\bfseries Option &
      \normalfont\bfseries Option Type & \normalfont\bfseries Default \\
    \midrule
      inline      & punct         & boolean   & true \\
                  & punct-symbol  & tokenlist & , \\
                  & brackets      & boolean   & true \\
                  & brackets-type & tokenlist & () \\
    \bottomrule
  \end{tabular}
\end{table}

\subsection{Page Number Styles}
The page number styles define how the page numbers where acronyms have been
used are typeset in the list.  It is set with the option \option{page-style}.
Legal values for this option are defined with the following command:
\begin{commands}
  \command{DeclareAcroPageStyle}[\marg{name}\marg{template}\marg{options}]
    Declares a new \acro\ extra style \meta{name}.  \meta{name} will be the
    value which can be chosen in the option \option{page-style}.
    \meta{template} is the name of the template to be used by the style.
    Available templates are listed in table~\ref{tab:page-templates}.
    \meta{options} are the option settings for the corresponding template.
\end{commands}

Here are two examples of the already available styles and how they are
defined:
\begin{sourcecode}
  \DeclareAcroPageStyle{default}{inline}{
    punct = true ,
    punct-symbol = .
  }
  \DeclareAcroPageStyle{paren}{inline}{
    brackets=true ,
    punct-symbol = ~
  }
\end{sourcecode}

\begin{table}[hbp]
  \centering
  \caption{Available Page Number Style Templates and Their
    Options}\label{tab:page-templates}
  \begin{tabular}{*{3}{>{\ttfamily}l}>{\ttfamily}p{.25\linewidth}}
    \toprule
      \normalfont\bfseries Template & \normalfont\bfseries Option &
      \normalfont\bfseries Option Type & \normalfont\bfseries Default \\
    \midrule
      inline & display       & boolean   & true \\
             & punct         & boolean   & false \\
             & punct-symbol  & tokenlist & , \\
             & brackets      & boolean   & false \\
             & brackets-type & tokenlist & () \\
             & space         & skip      & .333333em plus .166666em minus
                                           .111111em \\
    \bottomrule
  \end{tabular}
\end{table}

\subsection{Configuration Files}
If\sinceversion{2.2} you repeatedly have the same setup and definitions for
\acro\ in your preamble\footnote{For example defining new endings, \acro{}
  commands, list styles, \ldots} you might want to place those in a
configuration file.  If \acro\ finds a file named \code{acro.cfg} present it
inputs it at the end of the package.  The only thing to be aware of is that
this file is input like a package which means that \code{@} is treated as a
letter (category code~11).

\appendix

\acsetup{
  list-style       = description ,
  list-heading     = section ,
  list-name        = All Acronyms Used in this Documentation ,
  page-style       = comma ,
  following-pages* = true ,
  extra-style      = comma
}

\printacronyms

\end{document}

