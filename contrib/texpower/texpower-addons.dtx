% \iffalse meta-comment
% --------------------------------------------------------------
% Part of the TeXPower bundle
% Copyright (C) 1999-2004 Stephan Lehmke
%
% This program is free software; you can redistribute it and/or
% modify it under the terms of the GNU General Public License
% as published by the Free Software Foundation; either version 2
% of the License, or (at your option) any later version.
%
% This program is distributed in the hope that it will be useful,
% but WITHOUT ANY WARRANTY; without even the implied warranty of
% MERCHANTABILITY or FITNESS FOR A PARTICULAR PURPOSE.  See the
% GNU General Public License for more details.
% --------------------------------------------------------------
%
% texpower-addons.dtx,v 1.3 2005/03/28 22:36:05 hansfn Exp
%
% \fi
%
% \iffalse
%
%<*driver>
\documentclass{article}
\begin{document}
\typeout{***********************************************************}
\typeout{*}
\typeout{*\space\space\space\space\space\space\space Do NOT compile this dtx file!}
\typeout{*}
\typeout{***********************************************************}
\end{document}
%</driver>
%
% \fi
%
%<*automata>
%<<KEEPCOMMENTS
%=======================================================================================================================
% File: automata.sty
%
% Draw and animate `automata' as used in theoretical computer science, using PSTricks and TeXPower.
%
% This package is an addition to the TeXPower bundle, to be found at
% 
% http://texpower.sourceforge.net/
%
%=======================================================================================================================
% Author: Stephan Lehmke <Stephan.Lehmke@cs.uni-dortmund.de>
%
% v0.2  Oct 05, 2001 First version for the pre-alpha release of texpower.
%                    Only deterministic FA and Mealy Automata yet.
%

\NeedsTeXFormat{LaTeX2e}

\ProvidesPackage%
{automata}%
[2001/10/05 v0.2 Draw and animate automata.]


%=======================================================================================================================

\RequirePackage{ifthen}

\RequirePackage{relsize}

\RequirePackage{texpower}

\ifthenelse{\boolean{psspecialsallowed}}{\RequirePackage{tppstcol}}{}


%=======================================================================================================================
% Implementation of the functionality.

%-----------------------------------------------------------------------------------------------------------------------
% Some definitions of general interest.

\defineTPcolor{automatoncolor}{gray}{0.8}
\defineTPcolor{autoplatecolor}{gray}{0.95}
\defineTPcolor{currstatecolor}{rgb}{0.8,1,0.8}
\colorbetween{markstatecolor@TPATM}{pagecolor}{currstatecolor}
\replacecolor{cellcolor}{white}



\newlength{\tapepos@TPATM}

\newcounter{tapecells@TPATM}
\newcounter{inputlength@TPATM}
\newcounter{states@TPATM}
\newcounter{currstate@TPATM}
\newcounter{nextstate@TPATM}
\newcounter{currinputsymbol@TPATM}
\newcounter{currtapepos@TPATM}

\newtoks\tabrows@TPATM
\newtoks\tabdecl@TPATM

\newsavebox{\tabbox@TPATM}

\newboolean{stategraph@TPATM}

\newif\iffliplabel@TPATM

\newcommand{\formatsymbol}[1]{\texttt{#1}}


%-----------------------------------------------------------------------------------------------------------------------
% Helper macros.

% Make a fixed point number have exactly five decimal places, either by truncating or by filling up with trailing
% zeroes.
\newcommand{\filltofive@TPATM}[1]{\expandafter\@filltofive@TPATM#1000000\@nil#1}

\def\@filltofive@TPATM#1.#2#3#4#5#6#7\@nil#8{\def#8{#1.#2#3#4#5#6}}


% Count the number of items in a comma-separated list.
\newcommand{\countsymbols@TPATM}[2]{\setcounter{#2}{1}\@countsymbols@TPATM#1,\@nil{#2}}

\def\@countsymbols@TPATM#1,#2\@nil#3{\ifthenelse{\equal{#2}{}}{}{\stepcounter{#3}\@countsymbols@TPATM#2\@nil{#3}}}



% Get a symbol from a (macro containing a) comma-separated list.
\newcommand{\getsymbol@TPATM}[2]
{%
  \expandafter\@getsymbol@TPATM#1\@nil{#1}{#2}%
}

\def\@getsymbol@TPATM#1,#2\@nil#3#4{\gdef#3{#2}\@@getsymbol@TPATM#1\@nil{#4}}

\def\@@getsymbol@TPATM{\@ifnextchar({\@@@getsymbol@TPATM}{\@@@getsymbol@TPATM()}}

\def\@@@getsymbol@TPATM(#1)#2\@nil#3{\def\statepos@TPATM{#1}\def#3{#2}}


% Get a state from a (macro containing a) comma-separated list of states and check whether it's accepting.

\newif\ifacceptingstate@TPATM

\newcommand{\getstate@TPATM}[4]{\expandafter\@getstate@TPATM#1\@nil{#1}{#2}{#3}{#4}}

\def\@getstate@TPATM#1,#2\@nil#3#4#5#6{\gdef#3{#2}\@@getstate@TPATM#1\@nil{#4}{#5}{#6}}

\def\@@getstate@TPATM{\@ifnextchar({\@@@getstate@TPATM}{\@@@getstate@TPATM()}}

\def\@@@getstate@TPATM(#1)#2#3\@nil#4#5#6%
{%
  \def\statepos@TPATM{#1}%
  \def#6{#2#3}%
  \ifx\accept#2%
    \acceptingstate@TPATMtrue%
    \def#5{#4}%
   \else
    \acceptingstate@TPATMfalse%
    \let#5\empty
  \fi
  }

% Get a state transition from a (macro containing a) comma-separated list of states.
\newcommand{\getdeltapair@TPATM}[3]
{%
  \expandafter\@getdeltapair@TPATM#1\@nil{#1}{#2}{#3}%
}

\def\@getdeltapair@TPATM{\@ifnextchar[{\@@getdeltapair@TPATM}{\@@getdeltapair@TPATM[\@nil]}}

\def\@@getdeltapair@TPATM[#1]%
{%
  \def\optarg@TPATM{#1}%
  \@ifstar{\let\fliplabel@TPATM=\fliplabel@TPATMtrue\@@@getdeltapair@TPATM}
  {\let\fliplabel@TPATM=\fliplabel@TPATMfalse\@@@getdeltapair@TPATM}%
  }

\def\@@@getdeltapair@TPATM{\@ifnextchar({\@@@@getdeltapair@TPATM}{\@@@@getdeltapair@TPATM(.5)}}

\def\@@@@getdeltapair@TPATM(#1)%
{%
  \def\labelpos@TPATM{#1}%
  \@ifnextchar[{\@@@@@getdeltapair@TPATM}{\@@@@@getdeltapair@TPATM[\@nil]}%
  }

\def\@@@@@getdeltapair@TPATM[#1]#2->#3,#4\@nil#5#6#7{\def\labelopt@TPATM{#1}\protected@edef#6{#2}\protected@edef#7{#3}\gdef#5{#4}}

% Decompose a comma-separated pair (without parentheses) into its components.
\def\decomposestpos@TPATM#1,#2\@nil#3#4{\def#3{#1}\def#4{#2}}

% Remove trailing comma from a comma-separated list.
\def\gobblelastcomma@TPATM#1,\@nil#2{\gdef#2{#1}}

% A first step towards internationalization - german only at the moment.
\newcommand{\automatonname}{Automatenmodell}
\newcommand{\inputname}{Eingabe}
\newcommand{\inputtapename}{Eingabeband}
\newcommand{\outputname}{Ausgabe}
\newcommand{\outputtapename}{Ausgabeband}
\newcommand{\acceptname}{Akzeptieren}
\newcommand{\statename}{Zustand}
\newcommand{\statesname}{Zust"ande}
\newcommand{\statetransname}{Zustands"uberg"ange}
\newcommand{\nextstatename}{Folgezust.}

\newcommand{\statetransitiontablepreamble}{}

\newcommand{\transitionarrayparameters}
{%
  \renewcommand{\arraystretch}{.3}%
  \setlength{\extrarowheight}{16pt}%
  \setlength{\tabcolsep}{.5pt}%
}

\let\accept\empty
\let\comma\empty


%-----------------------------------------------------------------------------------------------------------------------
% Finite Automata.

% Generic macros for drawing deterministic finite automata. The difference between simple dfa and mealy automata is made
% by setting parameters.

\def\finalisetabrows@TPATM#1\\\@nil#2{#2={#1\end{tabular}}}%

\newcommand{\draw@outputtape@TPATM}[1]{}%
\newcommand{\getoutput@TPATM}[2]{}
\newcommand{\inittabrows@TPATM}[1]{}
\newcommand{\addtabrow@TPATM}[2]{}
\newcommand{\label@output@TPATM}{}%
\newcommand{\fix@nextstate@TPATM}{}
\newcommand{\mk@new@outputtape@TPATM}[1]{}%



\newcommand{\genericdfa@TPATM}[8]
{%
  \begin{pspicture}(1,1)
    % Draw automaton body.
    \psframe[fillstyle=solid,fillcolor=automatoncolor](0.05,0)(0.95,1)
    \rput[b]{90}(0.05,0.5){\psframebox[linestyle=none,fillstyle=solid,fillcolor=autoplatecolor]{\automatonname}}%
    % Draw input tape.
    \rput[l](0.07,0.93){\psframebox[linestyle=none,fillstyle=solid,fillcolor=autoplatecolor]{\inputtapename}}%
    \setlength{\tapepos@TPATM}{0.55pt-0.1pt*\real{#6}}%
    \setcounter{currinputsymbol@TPATM}{1*\real{#6}}%
    \mkfactor{\startx@TPATM}{\ratio{\tapepos@TPATM}{1pt}}%
    \setcounter{tapecells@TPATM}{6+\value{currinputsymbol@TPATM}}%
    \countsymbols@TPATM{#5}{inputlength@TPATM}%
    \ifthenelse{\value{tapecells@TPATM}>\value{inputlength@TPATM}}
    {\setcounter{tapecells@TPATM}{\value{inputlength@TPATM}}}
    {}%
    \multips(\startx@TPATM,0.88)(0.1,0){\value{tapecells@TPATM}}{\psframe[fillstyle=solid,fillcolor=cellcolor](0.1,0.1)}%
    \mkfactor{\startx@TPATM}{\ratio{\tapepos@TPATM+0.05pt}{1pt}}%
    \filltofive@TPATM{\startx@TPATM}%
    \protected@edef\tapecontents@TPATM{#5,}%
    \setcounter{currtapepos@TPATM}{0}%
    \let\inputtapesymbol@TPATM\relax
    \let\nextinputtapesymbol@TPATM\relax
    \multido{\n=\startx@TPATM+0.10000}{\value{tapecells@TPATM}}
    {%
      \getsymbol@TPATM\tapecontents@TPATM\inputtapesymbol@TPATM
      \ifnum\value{currtapepos@TPATM}=\value{currinputsymbol@TPATM}\relax
        \global\let\nextinputtapesymbol@TPATM\inputtapesymbol@TPATM
      \fi
      \stepcounter{currtapepos@TPATM}%
      \ifnum\value{currtapepos@TPATM}=\value{currinputsymbol@TPATM}\relax
        \global\let\inputtapesymbol@TPATM\inputtapesymbol@TPATM
      \fi
      \rput(\n,0.93){\formatsymbol\inputtapesymbol@TPATM}%
      }%
    \psframe[linewidth=2pt,linecolor=conceptcolor](0.45,0.88)(0.55,0.98)
    % Draw state transitions
    \countsymbols@TPATM{#1}{states@TPATM}%
    \mkfactor{\tapeadvancement@TPATM}{\real{#6}-\value{currinputsymbol@TPATM}pt}%
    \def\defdelta@TPATM{#3,}%
    \whiledo{\not\equal{\defdelta@TPATM}{}}
    {%
      \getdeltapair@TPATM\defdelta@TPATM\domainvalue@TPATM\rangevalue@TPATM
      \expandafter\def\csname delta(\domainvalue@TPATM)@TPATM\expandafter\endcsname\expandafter{\rangevalue@TPATM}%
      \expandafter\let\csname deltaopt(\domainvalue@TPATM)@TPATM\endcsname=\optarg@TPATM
      \expandafter\let\csname deltalabelpos(\domainvalue@TPATM)@TPATM\endcsname=\labelpos@TPATM
      \expandafter\let\csname deltalabelopt(\domainvalue@TPATM)@TPATM\endcsname=\labelopt@TPATM
      \expandafter\let\csname deltaflag(\domainvalue@TPATM)@TPATM\endcsname=\empty
      \expandafter\let\csname deltafliplabel(\domainvalue@TPATM)@TPATM\endcsname=\fliplabel@TPATM
      }%
    \ifthenelse{\boolean{stategraph@TPATM}}% Draw state transition diagram?
    {% Yes.
      \ifthenelse{\lengthtest{\tapeadvancement@TPATM pt=0pt}}
      {%
        \let\nextinputtapesymbol@TPATM\inputtapesymbol@TPATM
        \ifthenelse{\value{nextstate@TPATM}>-1}{\def\tapeadvancement@TPATM{1}}{}%
        }{}%
      \colorbetween[\tapeadvancement@TPATM]{activetranscolor@TPATM}{currstatecolor}{pagecolor}%
      \rput(0.5,0.5)
      {%
        \begingroup
          % Make `subpicture' containing the state transition diagram.
          \psset{xunit=.43\psxunit,yunit=.36\psyunit,labelsep=1pt}%
          \begin{pspicture}(0,0)
            \psframe[fillstyle=solid,fillcolor=cellcolor,dimen=inner](-1,-1)(1,1)
            \rput[lt](-1,1){\psframebox[linestyle=none,fillstyle=solid,fillcolor=autoplatecolor]{\statetransname}}%
            % First, the nodes representing states.
            \mkfactor{\statedistance@TPATM}{2/\value{states@TPATM}}%
            \mkfactor\startx@TPATM{-1+\statedistance@TPATM pt/2}%
            \def\states@TPATM{#1,}%
            \setcounter{currstate@TPATM}{0}%
            \multido{\n=\startx@TPATM+\statedistance@TPATM}{\value{states@TPATM}}
            {%
              \stepcounter{currstate@TPATM}%
              \getstate@TPATM{\states@TPATM}{,linewidth=2pt,linecolor=red}{\statetype@TPATM}{\state@TPATM}%
              \ifnum\value{currstate@TPATM}=#7\relax
                \ifacceptingstate@TPATM\global\acceptingstate@TPATMtrue\else\global\acceptingstate@TPATMfalse\fi
              \fi
              \edef\next
              {%
                \noexpand
                \cnode
                [%
                  fillstyle=solid,%
                  fillcolor=\ifnum\value{currstate@TPATM}=#7currstatecolor\else cellcolor\fi\statetype@TPATM
                  ]%
                (\ifx\statepos@TPATM\empty\n,0\else\statepos@TPATM\fi)%
                }%
              \next{.4}{{state\state@TPATM}}%
              \rput(\ifx\statepos@TPATM\empty\n,0\else\statepos@TPATM\fi){\let\comma,\state@TPATM}%
              \ifthenelse{\value{currstate@TPATM}=#4}
              {%
                \ifx\statepos@TPATM\empty
                  \mkfactor{\stxpos@TPATM}{\real{\n}-.2pt}%
                  \def\stypos@TPATM{0}%
                 \else
                  \expandafter\decomposestpos@TPATM\statepos@TPATM\@nil{\stxpos@TPATM}{\stypos@TPATM}%
                  \mkfactor{\stxpos@TPATM}{\real{\stxpos@TPATM}-.2pt}%
                \fi
                \pnode(\stxpos@TPATM,\stypos@TPATM){start}%
                \ncline{->}{start}{state\state@TPATM}%
                }{}%
              }%
            % Next, the node connections.
            \setcounter{currstate@TPATM}{0}%
            \def\states@TPATM{#1,}%
            \whiledo{\not\equal{\states@TPATM}{}}
            {%
              \stepcounter{currstate@TPATM}%
              \getsymbol@TPATM\states@TPATM\state@TPATM
              \protected@edef\state@TPATM{\state@TPATM}%
              \def\inputalphabet@TPATM{#2,}%
              \whiledo{\not\equal{\inputalphabet@TPATM}{}}
              {%
                \getsymbol@TPATM\inputalphabet@TPATM\inputsymbol@TPATM
                \getoutput@TPATM\state@TPATM\inputsymbol@TPATM
                \ifthenelse{\equal{\csname deltaflag(\state@TPATM;\inputsymbol@TPATM)@TPATM\endcsname}{}}
                {%
                  \let\currentnstate@TPATM=\nstate@TPATM
                  \let\currentia@TPATM=\inputalphabet@TPATM
                  \inittabrows@TPATM\tabrows@TPATM
                  \def\arrowparams@TPATM{border=1pt}%
                  \begingroup
                    \def\inputalphabet@TPATM{#2,}%
                    \whiledo{\not\equal{\inputalphabet@TPATM}{}}
                    {%
                      \getsymbol@TPATM\inputalphabet@TPATM\inputsymbol@TPATM
                      \getoutput@TPATM\state@TPATM\inputsymbol@TPATM
                      \ifx\currentnstate@TPATM\nstate@TPATM
                        \expandafter\global\expandafter\let
                        \csname deltaflag(\state@TPATM;\inputsymbol@TPATM)@TPATM\endcsname=\relax
                        \ifthenelse
                        {%
                          \value{currstate@TPATM}=#7%
                          \and\equal\inputsymbol@TPATM\nextinputtapesymbol@TPATM
                          \and\lengthtest{\tapeadvancement@TPATM pt>0pt}%
                          }
                        {%
                          \xdef\arrowparams@TPATM
                          {%
                            doubleline=true,doublecolor=activetranscolor@TPATM,doublesep=\tapeadvancement@TPATM pt,%
                            arrowsize=1pt 1.5,border=1pt%
                            }%
                          \def\@hlinput@TPATM{\noexpand\rowcolor{activetranscolor@TPATM}}%
                          }
                        {\let\@hlinput@TPATM=\empty}%
                        \addtabrow@TPATM\tabrows@TPATM{\inputsymbol@TPATM}%
                        \ifnum\value{currstate@TPATM}=#7\relax
                          \ifx\inputsymbol@TPATM\inputtapesymbol@TPATM
                            \fix@nextstate@TPATM%
                          \fi
                        \fi
                      \fi
                      }%
                    \expandafter\finalisetabrows@TPATM\the\tabrows@TPATM\@nil\tabrows@TPATM
                    \global\tabrows@TPATM=\tabrows@TPATM
                  \endgroup
                  \expandafter\gdef\expandafter\fullarrowparams@TPATM\expandafter{\expandafter[\arrowparams@TPATM]}%
                  \expandafter\gdef\expandafter\addarrowparams@TPATM\expandafter{\expandafter,\arrowparams@TPATM}%
                  \global\let\inputalphabet@TPATM=\currentia@TPATM
                  \transitionarrayparameters
                  \csname deltafliplabel(\state@TPATM;\inputsymbol@TPATM)@TPATM\endcsname
                  \edef\mkcirclabel@TPATM
                  {%
                    \iffliplabel@TPATM\noexpand\aput\else\noexpand\bput\fi
                    \expandafter\ifx\csname deltalabelopt(\state@TPATM;\inputsymbol@TPATM)@TPATM\endcsname\@nnil
                     \else
                      [\csname deltalabelopt(\state@TPATM;\inputsymbol@TPATM)@TPATM\endcsname]%
                    \fi
                    (\csname deltalabelpos(\state@TPATM;\inputsymbol@TPATM)@TPATM\endcsname)%
                    }%
                  \edef\mklinelabel@TPATM
                  {%
                    \iffliplabel@TPATM\noexpand\bput\else\noexpand\aput\fi
                    \expandafter\ifx\csname deltalabelopt(\state@TPATM;\inputsymbol@TPATM)@TPATM\endcsname\@nnil
                     \else
                      [\csname deltalabelopt(\state@TPATM;\inputsymbol@TPATM)@TPATM\endcsname]%
                    \fi
                    (\csname deltalabelpos(\state@TPATM;\inputsymbol@TPATM)@TPATM\endcsname)%
                    }%
                  \expandafter
                  \ifx\csname deltaopt(\state@TPATM;\inputsymbol@TPATM)@TPATM\endcsname\@nnil
                    \ifx\state@TPATM\nstate@TPATM
                      \expandafter\nccircle\fullarrowparams@TPATM{->}{state\state@TPATM}{.35}%
                      \mkcirclabel@TPATM{\the\tabrows@TPATM}%
                     \else
                      \expandafter\ncline\fullarrowparams@TPATM{->}{state\state@TPATM}{state\nstate@TPATM}%
                      \mklinelabel@TPATM{\the\tabrows@TPATM}%
                    \fi
                   \else
                    \ifx\state@TPATM\nstate@TPATM
                      \edef\next
                      {%
                        \noexpand\nccircle
                        \expandafter
                        \ifx\csname deltaopt(\state@TPATM;\inputsymbol@TPATM)@TPATM\endcsname\empty
                          \fullarrowparams@TPATM
                         \else
                          [\csname deltaopt(\state@TPATM;\inputsymbol@TPATM)@TPATM\endcsname\addarrowparams@TPATM]%
                        \fi
                        }%
                      \next{->}{state\state@TPATM}{.35}%
                      \mkcirclabel@TPATM{\the\tabrows@TPATM}%
                     \else
                      \edef\next
                      {%
                        \noexpand\ncarc
                        \expandafter
                        \ifx\csname deltaopt(\state@TPATM;\inputsymbol@TPATM)@TPATM\endcsname\empty
                          \fullarrowparams@TPATM
                         \else
                          [\csname deltaopt(\state@TPATM;\inputsymbol@TPATM)@TPATM\endcsname\addarrowparams@TPATM]%
                        \fi
                        }%
                      \next{->}{state\state@TPATM}{state\nstate@TPATM}%
                      \mklinelabel@TPATM{\the\tabrows@TPATM}%
                    \fi
                  \fi
                  }%
                {}%
                }%
              }%
          \end{pspicture}%
        \endgroup
        }%
      }
    {% No; draw state transition table.
      % First, the list of states.
      \rput[l](0.07,0.81){\psframebox[linestyle=none,fillstyle=solid,fillcolor=autoplatecolor]{\statesname}}%
      \mkfactor\startx@TPATM
      {\ratio{\widthof{\psframebox[linestyle=none,fillstyle=solid,fillcolor=autoplatecolor]{\statesname}}}{\psyunit}}%
      \mkfactor{\statewidth@TPATM}{0+(0.88pt-\startx@TPATM pt)/\value{states@TPATM}}%
      \ifthenelse{\lengthtest{\statewidth@TPATM pt>.1pt}}{\def\statewidth@TPATM{.1}}{}%
      \filltofive@TPATM{\startx@TPATM}%
      \filltofive@TPATM{\statewidth@TPATM}%
      \def\states@TPATM{#1,}%
      \setcounter{currstate@TPATM}{0}%
      \multido{\n=\startx@TPATM+\statewidth@TPATM}{\value{states@TPATM}}
      {%
        \stepcounter{currstate@TPATM}%
        \mkfactor{\endx@TPATM}{\real{\n}+\statewidth@TPATM pt}%
        \getstate@TPATM{\states@TPATM}{,linewidth=2pt,linecolor=red}{\statetype@TPATM}{\state@TPATM}%
        \ifnum\value{currstate@TPATM}=#7
          \ifacceptingstate@TPATM\global\acceptingstate@TPATMtrue\else\global\acceptingstate@TPATMfalse\fi
        \fi
        \edef\next
        {%
          \noexpand
          \psframe
          [%
            fillstyle=solid,%
            fillcolor=\ifnum\value{currstate@TPATM}=#7currstatecolor\else cellcolor\fi\statetype@TPATM
          ]%
          }%
        \next(\n,0.76)(\endx@TPATM,0.86)%
        \mkfactor{\charpos@TPATM}{\real{\n}+\statewidth@TPATM pt/2}%
        \rput(\charpos@TPATM,0.81){\state@TPATM}%
        \ifthenelse{\value{currstate@TPATM}=#4}{\psline{->}(\charpos@TPATM,0.74)(\charpos@TPATM,0.76)}{}%
        }%
      % Next, the state transition table itself.
      \psframe[fillstyle=solid,fillcolor=cellcolor,dimen=inner](0.07,0.14)(0.93,0.74)
      \rput[lt](0.07,0.74){\psframebox[linestyle=none,fillstyle=solid,fillcolor=autoplatecolor]{\smaller\statetransname}}%
      % Construct tabular declaration and first row.
      \def\inputalphabet@TPATM{#2,}%
      \tabrows@TPATM={}%
      \tabdecl@TPATM={}%
      \ifthenelse{\equal{\inputtapesymbol@TPATM}{\nextinputtapesymbol@TPATM}}
      {%
        \replacecolor{oldinputcolor@TPATM}{markstatecolor@TPATM}%
        \replacecolor{newinputcolor@TPATM}{markstatecolor@TPATM}%
        \replacecolor{oldoutputcolor@TPATM}{currstatecolor}%
        \replacecolor{newoutputcolor@TPATM}{currstatecolor}%
        }
      {%
        \colorbetween[\tapeadvancement@TPATM]{oldinputcolor@TPATM}{pagecolor}{markstatecolor@TPATM}%
        \colorbetween[\tapeadvancement@TPATM]{newinputcolor@TPATM}{markstatecolor@TPATM}{pagecolor}%
        \ifthenelse{\lengthtest{\tapeadvancement@TPATM pt>0.5pt}}
        {%
          \colorbetween[\tapeadvancement@TPATM]{newoutputcolor@TPATM}{currstatecolor}{pagecolor}%
          \replacecolor{oldoutputcolor@TPATM}{markstatecolor@TPATM}%
          }
        {%
          \colorbetween[\tapeadvancement@TPATM]{oldoutputcolor@TPATM}{pagecolor}{currstatecolor}%
          \replacecolor{newoutputcolor@TPATM}{markstatecolor@TPATM}%
          }%
        }%
      \whiledo{\not\equal{\inputalphabet@TPATM}{}}
      {%
        \getsymbol@TPATM\inputalphabet@TPATM\inputsymbol@TPATM
        \protected@edef\newtabrows@TPATM{\the\tabrows@TPATM&\noexpand\formatsymbol{\inputsymbol@TPATM}}%
        \expandafter\tabrows@TPATM\expandafter{\newtabrows@TPATM}%
        \edef\newtabdecl@TPATM
        {%
          \the\tabdecl@TPATM 
          \ifx\inputsymbol@TPATM\nextinputtapesymbol@TPATM>{\noexpand\columncolor{newinputcolor@TPATM}}\fi
          \ifx\inputsymbol@TPATM\inputtapesymbol@TPATM>{\noexpand\columncolor{oldinputcolor@TPATM}}\fi
          c%
          }%
        \expandafter\tabdecl@TPATM\expandafter{\newtabdecl@TPATM}%
        }%
      % Construct remaining rows.
      \setcounter{currstate@TPATM}{0}%
      \def\states@TPATM{#1,}%
      \whiledo{\not\equal{\states@TPATM}{}}
      {%
        \stepcounter{currstate@TPATM}%
        \getsymbol@TPATM\states@TPATM\state@TPATM
        \protected@edef\newtabrows@TPATM
        {%
          \the\tabrows@TPATM\noexpand\\%
          \ifnum\value{currstate@TPATM}=#7 \noexpand\rowcolor{markstatecolor@TPATM}\fi\state@TPATM
          }%
        \expandafter\tabrows@TPATM\expandafter{\newtabrows@TPATM}%
        \def\inputalphabet@TPATM{#2,}%
        \whiledo{\not\equal{\inputalphabet@TPATM}{}}
        {%
          \getsymbol@TPATM\inputalphabet@TPATM\inputsymbol@TPATM
          \protected@edef\newtabrows@TPATM
          {%
            \the\tabrows@TPATM&%
            \ifnum\value{currstate@TPATM}=#7
              \ifx\inputsymbol@TPATM\inputtapesymbol@TPATM
                \noexpand
                \multicolumn{1}{>{\noexpand\columncolor{oldoutputcolor@TPATM}}c}
                {\formatdfaoutput@TPATM\state@TPATM\inputsymbol@TPATM}%
               \else
                \ifx\inputsymbol@TPATM\nextinputtapesymbol@TPATM
                  \noexpand
                  \multicolumn{1}{>{\noexpand\columncolor{newoutputcolor@TPATM}}c}
                  {\formatdfaoutput@TPATM\state@TPATM\inputsymbol@TPATM}%
                 \else
                  \formatdfaoutput@TPATM\state@TPATM\inputsymbol@TPATM
                \fi
              \fi
             \else
              \formatdfaoutput@TPATM\state@TPATM\inputsymbol@TPATM
            \fi
            }%
          \expandafter\tabrows@TPATM\expandafter{\newtabrows@TPATM}%
          % Set succeeding state (and output symbol).
          \ifnum\value{currstate@TPATM}=#7\relax
            \ifx\inputsymbol@TPATM\inputtapesymbol@TPATM
              \getoutput@TPATM{\state@TPATM}{\inputsymbol@TPATM}%
              \fix@nextstate@TPATM
            \fi
          \fi
          }%
        }%
      % Save tabular into a box.
      \savebox{\tabbox@TPATM}
      {%
        \raisebox{\depth}
        {%
          \setlength{\fboxsep}{0.4pt}%
          \fbox
          {%
            \clipbox
            {%
              \edef\next{\noexpand\begin{tabular}[b]{c\the\tabdecl@TPATM}}%
                \statetransitiontablepreamble
                \next
                \the\tabrows@TPATM
              \end{tabular}%
              }%
            }%
          }%
        }%
      % Print state transition table, with some decoration.
      \rput(0.5,0.44)
      {%
        \raisebox{.5\tabcolsep}
        {\rotatebox{90}{\psframebox[linestyle=none,fillstyle=solid,fillcolor=autoplatecolor]{\smaller \statename}}}%
        \parbox[b]{\wd\tabbox@TPATM}
        {%
          \offinterlineskip\raggedleft
          \psframebox[linestyle=none,fillstyle=solid,fillcolor=autoplatecolor]{\smaller \inputname}\kern\tabcolsep\\%
          \usebox{\tabbox@TPATM}%
          }%
        \raisebox{.5\tabcolsep}
        {%
          \rotatebox{90}
          {\psframebox[linestyle=none,fillstyle=solid,fillcolor=autoplatecolor]{\smaller[2]\nextstatename}}%
          }%
        \label@output@TPATM
        }%
      }%
    % Draw output tape, if mealy automaton.
    \draw@outputtape@TPATM{#8}%
    % Calculate number of succeeding state.
    \setcounter{currstate@TPATM}{0}%
    \def\states@TPATM{#1,}%
    \whiledo{\not\equal{\states@TPATM}{}}
    {%
      \stepcounter{currstate@TPATM}%
      \getsymbol@TPATM\states@TPATM\state@TPATM
      \edef\state@TPATM{\state@TPATM}%
      \ifx\state@TPATM\nextstate@TPATM
        \setcounter{nextstate@TPATM}{\value{currstate@TPATM}}%
      \fi
      }%
    % Calculate new output tape.
    \mk@new@outputtape@TPATM{#8}%
  \end{pspicture}%
  \endgroup
  }


% \dfa{<width>}{<height>}{<states>}{<inputalphabet>}{<statetransitions>}
% {<initial state>}{<input tape>}{<tape position>}{<current state>}
% draws a deterministic finite automaton.
%    
% The parameters have the following meanings:
% <width>             Width of the automaton to be drawn.
% <height>            Height of the automaton to be drawn.
% <states>            Comma-separated list (symbolically) specifying the set of states.
% <inputalphabet>     Comma-separated list (symbolically) specifying the set of input symbols.
% <statetransitions>  Comma-separated list (symbolically) specifying the state transition function (also giving the
%                     output symbols associated with state transitions).
% <initial state>     Number of the initial state.
% <input tape>        Comma-separated list (symbolically) specifying the contents of the input tape.
% <tape position>     Fixed-point number (greater or equal 1) specifying the current position of the head reading input
%                     and output tape. The symbol currently read corresponds to the integer part of <tape position>.
% <current state>     Number of the current state.
%
% The standard version of \dfa displays a state transition table, i.e. for every combination of state and
% input symbol, the successor state is displayed in a tabular. The current state, current input
% symbol and next input symbol (if <tape position> is not integer) are indicated by colored bars.
%
% There is a starred version \mealyautomaton* which displays a state transition graph where states are displayed as
% circles and state transitions as arrows between circles. The `active' transition (leading from the current to the next
% state) is highlighted.
% 
% The <states> parameter looks like <state1>,<state2>,...,<staten>, where every single <state> has the following form:
% {(<x>,<y>)\accept <symbol>}
% where (<x>,<y>) and the special keyword \accept are optional and <symbol> is the symbolic name of the state.
% The optional parameter (<x>,<y>) is used only in the starred version \mealyautomaton*. If given, it specifies the
% position of the circle representing the state in the state transition graph, in PSTricks notation. The origin of the
% state transition graph is in the middle of the area reserved for it, stretching one unit in all directions, so that
% the corners of the area are at coordinates (-1,1), (1,1), (1,-1), (-1,-1). If not given in the starred version
% \mealyautomaton*, the position is calculated automatically in a `bead' style.
% The special keyword \accept flags this state as an accepting one.
% If the optional coordinates argument is not given, the outer curly braces may be left out, but otherwise leaving the
% curly braces out would disturb the parsing of the comma-separated list because of the comma between <x> and <y>.
%
% The <statetransitions> parameter looks like <trans1>,<trans2>,...,<transm>, where every single <trans> has the
% following form: 
% [<opt>]<state>;<input>-><state'>
% where [<opt>] is optional. <state> and <input> are the arguments of the state transition function, and <state'> is the
% successor state. 
% The optional parameter [<opt>] is used only in the starred version \mealyautomaton*. If given, it specifies additional
% optional parameters for the node connection command (\nccircle or \ncarc) drawing the arrow for this
% transition. \ncline is used for transitions from a state to itself. For transitions between states, \ncarc is used
% whenever the optional argument is given, otherwise \ncline is used. This is a seldom instance where giving an empty
% optional argument is different from giving no optional argument at all: Giving the optional argument [] switches
% between \ncline and \ncarc (useful if two transitions go `back and forth' between two states). 
%    

\newcommand{\initializedfa@TPATM}
{%
  \renewcommand{\draw@outputtape@TPATM}[1]
  {%
    \colorbetween{outercirclecolor@TPATM}{black}{automatoncolor}%
    \pscircle[fillstyle=solid,fillcolor=outercirclecolor@TPATM,linewidth=.5\pslinewidth](0.5,0.07){0.25}%
    \ifacceptingstate@TPATM    
      \replacecolor{acceptlampcolor@TPATM}{red}%
     \else
      \colorbetween[.1]{acceptlampcolor@TPATM}{red}{automatoncolor}%
    \fi
    \pscircle[fillstyle=solid,fillcolor=acceptlampcolor@TPATM,linewidth=.5\pslinewidth](0.5,0.07){0.15}%
    \rput[l](0.53,0.07){\psframebox[linestyle=none,fillstyle=solid,fillcolor=autoplatecolor]{\acceptname}}%
    }%
  \renewcommand{\getoutput@TPATM}[2]
  {%
    \expandafter\expandafter\expandafter\def\expandafter\expandafter\expandafter\nstate@TPATM
    \expandafter\expandafter\expandafter{\csname delta(##1;##2)@TPATM\endcsname}%
    }%
  \renewcommand{\inittabrows@TPATM}[1]{##1={\smaller\begin{tabular}{@{}r@{}}}}%
  \renewcommand{\addtabrow@TPATM}[2]
  {%
    \protected@edef\newtabrows@TPATM{\the##1\@hlinput@TPATM\formatsymbol{##2}\noexpand\\}%
    \expandafter##1\expandafter{\newtabrows@TPATM}%
    }%
  \def\formatdfaoutput@TPATM##1##2{\csname delta(##1;##2)@TPATM\endcsname}%
  \renewcommand{\label@output@TPATM}{}%
  \renewcommand{\fix@nextstate@TPATM}
  {%
    \global\let\nextstate@TPATM=\nstate@TPATM
    }%
  \renewcommand{\mk@new@outputtape@TPATM}[1]{}%
  }%

\newcommand{\dfa}
{%
  \initializedfa@TPATM
  \dfa@TPATM
  }

\newcommand{\dfa@TPATM}
{%
  \@ifstar
  {\setboolean{stategraph@TPATM}{true}\@dfa@TPATM}
  {\setboolean{stategraph@TPATM}{false}\@dfa@TPATM}%
  }

\newcommand{\@dfa@TPATM}[2]
{%
  \ifthenelse{\boolean{psspecialsallowed}}
  {%
    \begingroup
      \setlength{\@tempdima}{#1}%
      \setlength{\@tempdimb}{#2}%
      \psset{xunit=\@tempdima,yunit=\@tempdimb,arrowinset=0}%
      \let\next=\@@dfa@TPATM
    }
  {%
    \rule{#1}{#2}%
    \let\next=\@@nodfa@TPATM
    }%
  \next
}

\newcommand{\@@nodfa@TPATM}[7]{}

\newcommand{\@@dfa@TPATM}[7]{\genericdfa@TPATM{#1}{#2}{#3}{#4}{#5}{#6}{#7}{}}


% \simulatedfa{<width>}{<height>}{<states>}{<inputalphabet>}{<statetransitions>}
% {<initial state>}{<input tape>}{<number of steps>}{<tape move resolution>}
% simulates a deterministic finite automaton. 
%    
% The parameters <width>, <height>, <states>, <inputalphabet>, <statetransitions>, <initial state>, and <input tape>
% have the same meaning as for \dfa. 
% <number of steps> gives the number of state transitions to perform (in fact, the first state transition occurs after
% the first step, so <number of steps>=1 means no state transition.
% The simulation starts with both tapes at the beginning, in the initial state. After this, the simulation proceeds in
% the following steps: 
% 1) Change state.
% 2) Advance input tape and reiterate.
% The steps are realised with TeXPower's \step mechanism. This means the whole simulation will lead to the corresponding
% number of `subpages' each displaying the automaton in changing states.
% If <tape move resolution>=1, then step number 3 will just switch over the tape position. Otherwise, advancing the tape
% will take <tape move resolution> intermediate steps, which are automatically animated using hyperref's pdfpageduration
% mechanism, i.e. when displayed in acroread's full screen mode (with automatic advancing activated), a `smooth' tape
% movement will be seen.
% \simulatedfa comes in normal and starred versions, with the same semantics as for \dfa, i.e. the normal version
% displays a state transition table, the starred version displays a state transition graph. 

\newcommand{\simulatedfa}
{%
  \initializedfa@TPATM
  \@ifstar
  {\setboolean{stategraph@TPATM}{true}\@simulatedfa@TPATM}
  {\setboolean{stategraph@TPATM}{false}\@simulatedfa@TPATM}%
  }

\newcommand{\@simulatedfa@TPATM}[2]
{%
  \ifthenelse{\boolean{psspecialsallowed}}
  {%
    \begingroup
      \setlength{\@tempdima}{#1}%
      \setlength{\@tempdimb}{#2}%
      \psset{xunit=\@tempdima,yunit=\@tempdimb,arrowinset=0}%
      \let\next=\@@simulatedfa@TPATM
    }
  {%
    \rule{#1}{#2}%
    \let\next=\@@nosimulatedfa@TPATM
    }%
  \next
}

\newcommand{\@@nosimulatedfa@TPATM}[7]{}

\newcounter{currentdfastate@TPATM}
\newcounter{currentsupstep@TPATM}
\newcounter{currentsubstep@TPATM}
\newboolean{beginautosim@TPATM}
\newcommand{\autosimduration}{0.04}
\releasecounter{currentdfastate@TPATM}\releasecounter{nextstate@TPATM}

\newcommand{\@@simulatedfa@TPATM}[7]
{%
  % Setup initial state.
  \setcounter{currentsupstep@TPATM}{0}%
  \setcounter{currentdfastate@TPATM}{#4}%
  \setcounter{nextstate@TPATM}{0}%
  \stepwise[\setcounter{currentdfastate@TPATM}{#4}\setcounter{nextstate@TPATM}{0}]
  {%
  \setboolean{beginautosim@TPATM}{true}%
  % Main loop for state transitions.
  \whiledo{\value{currentsupstep@TPATM}<#6}
  {%
    % At the beginning, the situation right before the state transition is displayed. This is effected by only
    % displaying the last step of the tape movement.
    \ifthenelse{\boolean{beginautosim@TPATM}}
    {\setboolean{beginautosim@TPATM}{false}\setcounter{currentsubstep@TPATM}{#7-1}}
    {\setcounter{currentsubstep@TPATM}{0}}%
    % Inner loop for tape movement.
    \whiledo{\value{currentsubstep@TPATM}<#7}
    {%
      \stepcounter{currentsubstep@TPATM}%
      % Display sequence of steps while the tape gradually moves.
      \mkfactor{\tapeposition@TPATM}{\value{currentsupstep@TPATM}+\arabic{currentsubstep@TPATM}pt/#7}%
      \ifthenelse{\value{currentsubstep@TPATM}=#7}
      {%
        \setcounter{tmpcnta@TP}{\value{currentsupstep@TPATM}+1}%
        \ifthenelse{\value{tmpcnta@TP}=#6}{\switch}{\switch[][\boolean{firstactivation}]}{}
        {%
          \begingroup\genericdfa@TPATM{#1}{#2}{#3}{#4}{#5}{\tapeposition@TPATM}{\value{currentdfastate@TPATM}}{}%
          \afterstep{\stopAdvancing}%
          \setcounter{currentdfastate@TPATM}{\value{nextstate@TPATM}}\setcounter{nextstate@TPATM}{-1}%
          }%
        }
      {%
        \switch[][\boolean{firstactivation}]{}
        {%
          \begingroup\genericdfa@TPATM{#1}{#2}{#3}{#4}{#5}{\tapeposition@TPATM}{\value{currentdfastate@TPATM}}{}%
          \afterstep{\pageDuration{\autosimduration}}%
          }%
        }%
      }%
    \stepcounter{currentsupstep@TPATM}%
    % After the tape is in the next (or initial) position, switch state and reiterate.
    \ifthenelse{\value{currentsupstep@TPATM}=#6}{}
    {%
      \switch[][\boolean{firstactivation}]{}%
      {%
        \begingroup\genericdfa@TPATM{#1}{#2}{#3}{#4}{#5}{\tapeposition@TPATM}{\value{currentdfastate@TPATM}}{}%
        }%
      }%
    }%
  }%
\endgroup
}


%-----------------------------------------------------------------------------------------------------------------------
% Finite Automata with output.

% \mealyautomaton{<width>}{<height>}{<states>}{<inputalphabet>}{<statetransitions>}
% {<initial state>}{<input tape>}{<tape position>}{<current state>}{<output tape>}
% draws a mealy automaton (deterministic finite automaton with output tape, output is triggered by state transitions).
%    
% The parameters have the same meanings as for \dfa, with one addition:
% <output tape>       Comma-separated list (symbolically) specifying the contents of the output tape.
%
% The standard version of \mealyautomaton displays a state transition table, i.e. for every combination of state and
% input symbol, the successor state and output symbol are displayed in a tabular. The current state, current input
% symbol and next input symbol (if <tape position> is not integer) are indicated by colored bars.
%
% There is a starred version \mealyautomaton* which displays a state transition graph where states are displayed as
% circles and state transitions as arrows between circles. The `active' transition (leading from the current to the next
% state) is highlighted.
% 
% The <states> parameter has the same structure as for \dfa.
%
% The <statetransitions> parameter looks like <trans1>,<trans2>,...,<transm>, where every single <trans> has the
% following form: 
% [<opt>]<state>;<input>-><state'>;<output>
% where [<opt>] is optional. <state> and <input> are the arguments of the state transition function, and <state'> and
% <output> are the successor state and output symbol, respectively.
% The optional parameter [<opt>] is used only in the starred version \mealyautomaton*. If given, it specifies additional
% optional parameters for the node connection command (\nccircle or \ncarc) drawing the arrow for this
% transition. \ncline is used for transitions from a state to itself. For transitions between states, \ncarc is used
% whenever the optional argument is given, otherwise \ncline is used. This is a seldom instance where giving an empty
% optional argument is different from giving no optional argument at all: Giving the optional argument [] switches
% between \ncline and \ncarc (useful if two transitions go `back and forth' between two states). 
%    

\newcommand{\initializema@TPATM}
{%
  \renewcommand{\draw@outputtape@TPATM}[1]
  {%
    \rput[l](0.07,0.07){\psframebox[linestyle=none,fillstyle=solid,fillcolor=autoplatecolor]{\outputtapename}}%
    \mkfactor{\startx@TPATM}{\ratio{\tapepos@TPATM}{1pt}}%
    \setcounter{tapecells@TPATM}{6+\value{currinputsymbol@TPATM}}%
    \multips(\startx@TPATM,0.02)(0.1,0){\value{tapecells@TPATM}}{\psframe[fillstyle=solid,fillcolor=cellcolor](0.1,0.1)}%
    \mkfactor{\startx@TPATM}{\ratio{\tapepos@TPATM+0.05pt}{1pt}}%
    \filltofive@TPATM{\startx@TPATM}%
    \protected@edef\tapecontents@TPATM{##1,}%
    \setcounter{currtapepos@TPATM}{0}%
    \multido{\n=\startx@TPATM+0.10000}{\value{tapecells@TPATM}}
    {%
      \ifx\tapecontents@TPATM\empty
       \else
        \stepcounter{currtapepos@TPATM}%
        \getsymbol@TPATM\tapecontents@TPATM\outputtapesymbol@TPATM
        \ifnum\value{currtapepos@TPATM}=\value{currinputsymbol@TPATM}\relax
          \global\let\outputtapesymbol@TPATM\outputtapesymbol@TPATM
        \fi
        \rput(\n,0.07){\formatsymbol\outputtapesymbol@TPATM}%
      \fi
      }%
    \psframe[linewidth=2pt,linecolor=mathcolor](0.45,0.02)(0.55,0.12)
    }%
  \renewcommand{\getoutput@TPATM}[2]
  {\expandafter\expandafter\expandafter\@getoutput@TPATM\csname delta(##1;##2)@TPATM\endcsname\@nil}%
  \def\@getoutput@TPATM##1;##2\@nil{\def\nstate@TPATM{##1}\def\output@TPATM{##2}}%
  \renewcommand{\inittabrows@TPATM}[1]{##1={\smaller\begin{tabular}{rcl}}}%
  \renewcommand{\addtabrow@TPATM}[2]
  {%
    \protected@edef\newtabrows@TPATM
    {%
      \the##1%
      \@hlinput@TPATM\formatsymbol{##2}&\origmath{\vartriangleright}&\noexpand\formatsymbol{\output@TPATM}\noexpand\\%
      }%
    \expandafter##1\expandafter{\newtabrows@TPATM}%
    }%
  \def\formatdfaoutput@TPATM##1##2%
  {\expandafter\expandafter\expandafter\@formatdfaoutput@TPATM\csname delta(##1;##2)@TPATM\endcsname\@nil}%
  \def\@formatdfaoutput@TPATM##1;##2\@nil{##1;\formatsymbol{##2}}%
  \renewcommand{\label@output@TPATM}
  {%
    \raisebox{.5\tabcolsep}
    {\rotatebox{90}{\psframebox[linestyle=none,fillstyle=solid,fillcolor=autoplatecolor]{\smaller \outputname}}}%
    }%
  \renewcommand{\fix@nextstate@TPATM}
  {%
    \global\let\newoutputsymbol@TPATM=\output@TPATM
    \global\let\nextstate@TPATM=\nstate@TPATM
    }%
  \renewcommand{\mk@new@outputtape@TPATM}[1]%
  {%
    \setcounter{currtapepos@TPATM}{0}%
    \protected@edef\tapecontents@TPATM{##1,}%
    \let\newoutputtape@TPATM=\empty
    \whiledo{\not\equal{\tapecontents@TPATM}{}}
    {%
      \stepcounter{currtapepos@TPATM}%
      \getsymbol@TPATM\tapecontents@TPATM\outputsymbol@TPATM
      \ifnum\value{currtapepos@TPATM}=\value{currinputsymbol@TPATM}%
        \let\outputsymbol@TPATM=\newoutputsymbol@TPATM%
      \fi
      \protected@edef\newoutputtape@TPATM{\newoutputtape@TPATM\outputsymbol@TPATM,}%
      }%
    \expandafter\gobblelastcomma@TPATM\newoutputtape@TPATM\@nil\newoutputtape@TPATM
    }%
  }

\newcommand{\mealyautomaton}
{%
  \initializema@TPATM
  \mealyautomaton@TPATM
  }

\newcommand{\mealyautomaton@TPATM}
{%
  \@ifstar
  {\setboolean{stategraph@TPATM}{true}\@mealyautomaton@TPATM}
  {\setboolean{stategraph@TPATM}{false}\@mealyautomaton@TPATM}%
  }

\newcommand{\@mealyautomaton@TPATM}[2]
{%
  \ifthenelse{\boolean{psspecialsallowed}}
  {%
    \begingroup
      \setlength{\@tempdima}{#1}%
      \setlength{\@tempdimb}{#2}%
      \psset{xunit=\@tempdima,yunit=\@tempdimb,arrowinset=0}%
      \let\next=\@@genericdfa@TPATM
    }
  {%
    \rule{#1}{#2}%
    \let\next=\@@nomealyautomaton@TPATM
    }%
  \next
}

\newcommand{\@@nomealyautomaton@TPATM}[8]{}


% \simulatema{<width>}{<height>}{<states>}{<inputalphabet>}{<statetransitions>}
% {<initial state>}{<input tape>}{<number of steps>}{<tape move resolution>}{<output tape>}
% simulates a mealy automaton (deterministic finite automaton with output tape, output is triggered by state
% transitions). 
%    
% The parameters <width>, <height>, <states>, <inputalphabet>, <statetransitions>, <initial state>, <input tape>,
% <output tape> have the same meaning as for \mealyautomaton.
% The parameters <number of steps> and <tape move resolution> have the same meaning as for \simulatedfa.
% The simulation starts with both tapes at the beginning, in the initial state and with the output tape in the state
% given by <output tape>. After this, the simulation proceeds in the following steps:
% 1) Write output symbol of the `active' transition to the output tape.
% 2) Change state.
% 3) Advance both tapes and reiterate.
% \simulatema comes in normal and starred versions, with the same semantics as for \mealyautomaton, i.e. the normal
% version displays a state transition table, the starred version displays a state transition graph.

\newcommand{\simulatema}
{%
  \initializema@TPATM
  \@ifstar
  {\setboolean{stategraph@TPATM}{true}\@simulatema@TPATM}
  {\setboolean{stategraph@TPATM}{false}\@simulatema@TPATM}%
  }

\newcommand{\@simulatema@TPATM}[2]
{%
  \ifthenelse{\boolean{psspecialsallowed}}
  {%
    \begingroup
      \setlength{\@tempdima}{#1}%
      \setlength{\@tempdimb}{#2}%
      \psset{xunit=\@tempdima,yunit=\@tempdimb,arrowinset=0}%
      \let\next=\@@simulatema@TPATM
    }
  {%
    \rule{#1}{#2}%
    \let\next=\@@nosimulatema@TPATM
    }%
  \next
}

\newcommand{\@@nosimulatema@TPATM}[8]{}

\newcommand{\@@simulatema@TPATM}[8]
{%
  % Setup initial state.
  \setcounter{currentsupstep@TPATM}{0}%
  \setcounter{currentdfastate@TPATM}{#4}\gdef\currentoutputtape@TPATM{#8}%
  \stepwise[\setcounter{currentdfastate@TPATM}{#4}\gdef\currentoutputtape@TPATM{#8}]
  {%
  \setboolean{beginautosim@TPATM}{true}%
  % Main loop for state transitions.
  \whiledo{\value{currentsupstep@TPATM}<#6}
  {%
    % At the beginning, the situation right before the state transition is displayed. This is effected by only
    % displaying the last step of the tape movement.
    \ifthenelse{\boolean{beginautosim@TPATM}}
    {\setboolean{beginautosim@TPATM}{false}\setcounter{currentsubstep@TPATM}{#7-1}}
    {\setcounter{currentsubstep@TPATM}{0}}%
    % Inner loop for tape movement.
    \whiledo{\value{currentsubstep@TPATM}<#7}
    {%
      \stepcounter{currentsubstep@TPATM}%
      % Display sequence of steps while the tape gradually moves.
      \mkfactor{\tapeposition@TPATM}{\value{currentsupstep@TPATM}+\arabic{currentsubstep@TPATM}pt/#7}%
      \switch[][\boolean{firstactivation}]{}
      {%
        \begingroup\genericdfa@TPATM{#1}{#2}{#3}{#4}{#5}{\tapeposition@TPATM}{\value{currentdfastate@TPATM}}{\currentoutputtape@TPATM}%
        \ifthenelse{\value{currentsubstep@TPATM}=#7}
        {\afterstep{\stopAdvancing}\global\let\currentoutputtape@TPATM\newoutputtape@TPATM}%
        {\afterstep{\pageDuration{\autosimduration}}}%
        }%
      }%
    \stepcounter{currentsupstep@TPATM}%
    % After the tape is in the next (or initial) position, display new output symbol.
    \switch[][\boolean{firstactivation}]{}
    {%
      \begingroup\genericdfa@TPATM{#1}{#2}{#3}{#4}{#5}{\tapeposition@TPATM}{\value{currentdfastate@TPATM}}{\currentoutputtape@TPATM}%
      \setcounter{currentdfastate@TPATM}{\value{nextstate@TPATM}}\setcounter{nextstate@TPATM}{-1}%
      }%
    % Next, switch state and reiterate.
    \ifthenelse{\value{currentsupstep@TPATM}=#6}{\switch}{\switch[][\boolean{firstactivation}]}{}%
    {%
      \begingroup\genericdfa@TPATM{#1}{#2}{#3}{#4}{#5}{\tapeposition@TPATM}{\value{currentdfastate@TPATM}}{\currentoutputtape@TPATM}%
      }%
    }%
  }%
\endgroup
}

%KEEPCOMMENTS
%</automata>
%=======================================================================================================================
%<*fixseminar>
%<<KEEPCOMMENTS
%=======================================================================================================================
% File: fixseminar.sty
%
% Fix some problems in conjunction with seminar.
%
% This package is part of the TeXPower bundle, to be found at
% 
% http://texpower.sourceforge.net/
%
% If using hyperref, load this package _after_ hyperref.
%
%=======================================================================================================================
% Author: Stephan Lehmke <Stephan.Lehmke@cs.uni-dortmund.de>
%
% v0.0.1  Mar 28, 2000 First version for the pre-alpha release of texpower and hyperref v6.70a.
%
% v0.0.1a May 18, 2000 Added a \vss to a \vbox to get rid of an `overfull vbox' message.
%
% v0.0.2  Jun 02, 2000 Now handle seminar as well as powersem.
%
% v0.0.3  Sep 07, 2000 hyperref v6.70u and later don't have the \@hyperfixhead bug any more.
%
% v0.0.4  Nov 22, 2002 Now using ifpdf package if available.
%                      As hyperref v6.70u exists for some time now, the \@hyperfixhead hack is removed.
%                      Fixed a problem with \twoup (bug found and fix proposed by Laurent Mugnier).
%                      Added a preliminary fix for rotating slides with pdftex.
%
% v0.0.5  May 26, 2003 A small change to make fixseminar work better when hyperref is not loaded.


\NeedsTeXFormat{LaTeX2e}

\ProvidesPackage%
{fixseminar}%
[2003/05/26 v0.0.5 Fix some problems of seminar.]


%=======================================================================================================================

\RequirePackage{ifthen}


%=======================================================================================================================
% Options and general setup.

\ProcessOptions


% The switch \ifpdf is to determine whether pdfLaTeX is being run and outputting pdf, using Heiko Oberdiek's faultproof
% pdf detector: 

\IfFileExists{ifpdf.sty}
{%
  \RequirePackage{ifpdf}%
  }
{%
  \@ifundefined{pdftrue}
  {%
  \expandafter\newif\csname ifpdf\endcsname
  \ifx\pdfoutput\undefined
   \else
    \ifx\pdfoutput\relax
     \else
      \ifcase\pdfoutput
       \else
        \pdftrue
       \fi
     \fi
   \fi
   }{}%
   }


%=======================================================================================================================
% Here come the fixes.

% We need to make sure that magnification, as used by seminar, is respected when setting \pdfpageheight and
% \pdfpagewidth. We strip the pt and add truept which respects magnification.


\ifthenelse{\boolean{pdf}}%          Are we outputting PDF?
{% Yes.
  \@ifclassloaded{seminar}%         seminar document class?
  {%
    \RequirePackage{graphics}%
    \def\leftsliderotation#1{\rotatebox{90}{#1}}%
    \def\rightsliderotation#1{\rotatebox{270}{#1}}%
    \sliderotation{left}%
  }{}%
  \AtBeginDocument%                   Then we need to set \pdfpagewidth and \pdfpageheight in a _true_ unit to make sure
  {%                                  the pdf page has the right dimensions even when magnifying (as seminar does).
    \setlength{\pdfhorigin}{1truein}% \pdfhorigin and \pdfvorigin also need to be set to true dimensions.
    \setlength{\pdfvorigin}{1truein}%
    \@ifclassloaded{seminar}%         seminar document class?
    {%
      \ifportrait%                    Portrait slides?
        \setlength{\pdfpageheight}{\strip@pt\paperheight truept}%
        \setlength{\pdfpagewidth}{\strip@pt\paperwidth truept}%
       \else%                         Landscape. Need to exchange height and width. 
        \renewcommand{\printlandscape}
        {%
          \setlength{\pdfpagewidth}{\strip@pt\paperheight truept}%
          \setlength{\pdfpageheight}{\strip@pt\paperwidth truept}%
        }%
      \fi
      }%
    {%
      \setlength{\pdfpageheight}{\strip@pt\paperheight truept}%
      \setlength{\pdfpagewidth}{\strip@pt\paperwidth truept}%
    }%
  }%
}
{% No. In this case we define \special@paper which will put an appropriate \special into the dvi file.
  \@ifclassloaded{seminar}
  {%
    \ifportrait
      \edef\special@paper{\the\paperwidth,\the\paperheight}%
     \else%                                                     Need to exchange height and width. 
      \renewcommand{\printlandscape}{\edef\special@paper{\the\paperheight,\the\paperwidth}}
    \fi
    \AtBeginDocument{\special{papersize=\special@paper}}%
    }%
  {%
    }%
  }%


% There is a bug in old hyperref versions to the effect that apparently, when using seminar there is a glitch in
% vertical spacing. Issue a warning if hyperref is too old.

\@ifpackageloaded{hyperref}%               Looking for \@hyperfixhead bug.
{%
  \@ifpackagelater{hyperref}{2000/09/07}%  Is the \@hyperfixhead bug present in hyperref?
  {}%                                      No. 
  {%                                       Yes. Issue a warning.
    \PackageWarning{fixseminar}
    {Package hyperref too old (older than 6.70u).\MessageBreak Expect problems with seminar}%
    }%
  }{}
%KEEPCOMMENTS
%</fixseminar>
%=======================================================================================================================
%<*tppstcol>
%<<KEEPCOMMENTS
%=======================================================================================================================
% File: tppstcol.sty
%
% Replacement for ``pstcol.sty'' with some quirks corrected.
% There is a small problem with pstcol.sty which makes it impossible
% to set something like "fillcolor=\linecolor" in pstricks (see LaTeX
% Bug graphics/3366). As this is absolutely indispensible for
% automata.sty, I had to make a modified version to be used until the
% bug is mended.
% The following code is identical with the original pstcol.sty, with
% some modifications to correct the bug. 
%
% This package is an addition to the TeXPower bundle, to be found at
% 
% http://texpower.sourceforge.net/
%
% Beware! As soon as the original bug in pstcol is corrected, this
% package will be deleted.
%
%=======================================================================================================================

\NeedsTeXFormat{LaTeX2e}
\ProvidesPackage{tppstcol}
          [2001/08/28 v0.2  PSTricks color compatibility (DPC; revised
          by STL)]
%%%%%%%%%%%%%%%%%%
\let\saved@gray\color@gray
\RequirePackage{pstricks}
\let\color@gray\saved@gray
%%%%%%%%%%%%%%%%
\DeclareOption*{\PassOptionsToPackage\CurrentOption{color}}
\ProcessOptions
\RequirePackage{color}
%%%%%%%%%%%%%%%%%%%%%%%%%%%%%%%%%%%%%%%%%%%%%%%%%%%%%
\def\pst@getcolor#1#2{%
\@ifundefined{\string\color@#1}%
{\@pstrickserr{Color `#1' not defined}\@eha}%
 {\edef#2{#1}}}
\def\pst@usecolor#1%
{%
  \expandafter\expandafter\expandafter
  \c@lor@to@ps\csname\string\color @#1\endcsname\@@\space%
  }
%%%%%%%%%%%%%%%%%%%%%%%%%%%%%%%%%%%%%%%%%%%%%%%%%%%%%%%%%%%%%%%%%%%%%
\definecolor{darkgray}{gray}{.25}
\definecolor{gray}{gray}{.5}
\definecolor{lightgray}{gray}{.75}
\definecolor{gradbegin}{rgb}{0,1,1}
\definecolor{gradend}{rgb}{0,.1,.95}
%%%%%%%%%%%%%%%%%%%%%%%%%%%%%%%%%%%%%%%%%%%%%%%%%%%%%%%%
\psset@bordercolor{white}
\psset@doublecolor{white}
\psset@linecolor{black}
\psset@fillcolor{white}
\psset@hatchcolor{black}
\psset@gridcolor{black}
\psset@gridlabelcolor{black}
\psset@shadowcolor{darkgray}
\psset@subgridcolor{gray}
%%%%%%%%%%%%%%%%%%%%%%%%%%%%%%%%%%%%%%%%%%%%%%%%%%%%%%%%%%%%%%%%%%%%
\def\newgray#1{%
  \definecolor{#1}{gray}}
\def\newrgbcolor#1#2{%
  \newrgbcolorx{#1}#2\@@}
\def\newrgbcolorx#1#2 #3 #4\@@{%
  \definecolor{#1}{rgb}{#2,#3,#4}}
\def\newcmykcolor#1#2{%
  \newcmykcolorx{#1}#2\@@}
\def\newcmykcolorx#1#2 #3 #4 #5\@@{%
  \definecolor{#1}{cmyk}{#2,#3,#4,#5}}
%KEEPCOMMENTS
%</tppstcol>
%=======================================================================================================================
%<*tpsem-a4>
%<<KEEPCOMMENTS
% Fixed for LaTeX2e (\paperwidth and \paperheight are dimensions) by
% Stephan Lehmke.
\setlength\paperwidth{210mm}
\setlength\paperheight{297mm}
\input sem-page.sty
\slidewidth 222mm
\slideheight 152mm
%KEEPCOMMENTS
%</tpsem-a4>
\endinput
