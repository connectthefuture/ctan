\documentclass{article}
\usepackage[T1]{fontenc}
\usepackage[utf8]{inputenc}
\usepackage{listings,tabularx}
\title{The \texttt{ocr} package\medskip\\
  \large\LaTeX\ support for the OCR fonts (Optical Chatacter
  Recognition fonts)}
\author{version 1.0 \today}
\lstset{language=[latex]tex,breaklines=true}
\date{Palle J\o rgensen}
\newcommand*\ocrfont[1]{{\fontencoding{OT1}\fontfamily{#1}\selectfont 123}}
\newcommand*\sourcefile[1]{\subsection{#1}
  \lstinputlisting{#1}}

\begin{document}
\maketitle
\section{Introduction}
\label{sec:introduction}
The \texttt{ocr} package provides support for the OCR fonts.
The OCR fonts are already installed on many systems, this is only
support for using the OCR fonts with \LaTeX.

The license of the ocr pcakage and the related files is GNU General
Public License.

\clearpage
\section{Using the ocr package}
\label{sec:using-OCR-fonts}

\subsection{Package options}
\label{sec:package-options}

The \texttt{ocr} package has the following options\bigskip

\noindent\begin{tabularx}{\linewidth}{>\ttfamily lXlX}
  \hline
  \rmfamily Option:              & Effect:                                      & Example        & Comments: \\\hline
  ocr-a                & Selects the OCR font to be OCR-A             & \ocrfont{ocra}             \\
  ocr-b                & Selects the OCR font to be OCR-B             & \ocrfont{ocrb} & Default   \\
  ocr-b-outline        & Selects the OCR font to be an outline of
  OCR-B                & \ocrfont{ocrbo}                                                           \\
  ocr-b-negative       & Selects the OCR font to be a negative  OCR-B & \ocrfont{ocrbn}            \\
  ocr-b-sharp          & Selects the OCR font to be a ``sharp''  OCR-B
                       & \ocrfont{ocrbs}                              & The sharp fonts are not constructed properly. Use
  with care                                                                                        \\
  ocr-b-sharp-negative & Selects the OCR font to be a negative
  ``sharp''  OCR-B     & \ocrfont{ocrbns}                             & The sharp fonts are not constructed properly. Use
  with care                                                                                        \\\hline
\end{tabularx}

\subsection{Commands}
\label{sec:commands}

If you want some text typeset with the selected OCR font for a short
text you can use the command \verb+\ocr+; ie.\ \verb+\ocr{cheese}+.

It is possible to use the command \verb+\ocrfamily+ but this command
also changes the current fontencoding; use with caution\dots

\subsubsection{Negative fonts}
\label{sec:negative-fonts}

If you have selectec the option \verb+ocr-b+ or \verb+ocr-b-sharp+ it
is possible to use the command \verb+\ocrneg+ which typsesets the text
with the ``negative'' of the selected OCR font.

\clearpage
\section{Source of the files in the OCR bundle}
\label{sec:source}

\sourcefile{ocr.sty}
\sourcefile{ot1oca.fd}
\sourcefile{ot1ocra.fd}
\sourcefile{ot1ocrb.fd}
\sourcefile{ot1ocrbn.fd}
\sourcefile{ot1ocrbns.fd}
\clearpage
\sourcefile{ot1ocrbo.fd}
\sourcefile{ot1ocrbs.fd}


\end{document}

%%% Local Variables: 
%%% mode: latex
%%% TeX-master: t
%%% End: 
