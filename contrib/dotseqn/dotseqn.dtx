% \iffalse
%
% D O T S E Q N . D T X    ver 1.1
%
% Copyright (C) 1995 by Donald Arseneau
% This file may be freely transmitted and reproduced, but it may
% not be changed unless the name is changed also (except that you
% may freely change the paper-size option for \documentclass).
% This notice must be left intact.
%
%<package>\NeedsTeXFormat{LaTeX2e}
%<*dtx>
\ProvidesFile{dotseqn.dtx}
%</dtx>
%<package>\ProvidesPackage{dotseqn}
%<driver>\ProvidesFile{dotseqn.drv}
  [1995/03/22  v1.1 \space Dots leading to equation numbers]
%<*driver>
\makeatletter
\typein[\answer]{Do you want to generate dotseqn.sty? (Yes/No)}
\edef\answer{`\expandafter\@car\answer N\@nil}
\ifodd\answer
\begin{filecontents}{dotseqn.ins}
% dotseqn.ins:   This is a temporary file.  You can delete it.
\keepsilent
\preamble

D O T S E Q N . S T Y   is  Copyright (c) 1995 by Donald Arseneau

This file may be freely transmitted and reproduced, but it may not be
changed unless the name is changed also.  This notice must remain intact.

\endpreamble
\generateFile{dotseqn.sty}{t}{\from{dotseqn.dtx}{package}}
\Msg{ }
\Msg{You should now copy dotseqn.sty to a directory where LaTeX checks for}
\Msg{input files. \space Run LaTeX again to generate documentation.}
\Msg{ }\endinput
\end{filecontents}
\makeatletter \closein\@inputcheck % because a bug in LaTeX
\def\batchfile{dotseqn.ins}
\csname fi\endcsname\input docstrip.tex % avoid unbalanced if
\fi
%%%%%%%%%%%%%%%%%%%%%%%%%%%%%%%%%%%%%%%%%%%%%%%%%%%%%%%%%%%%
%
 \documentclass{ltxdoc}
%
% \documentclass[a4paper]{ltxdoc}
%
%%%%%%%%%%%%%%%%%%%%%%%%%%%%%%%%%%%%%%%%%%%%%%%%%%%%%%%%%%%%
\typeout{Generating user documentation for the dotseqn package.}
\typein[\answer]{Do you also want a commented source listing? (Yes/No)}
\edef\answer{`\expandafter\@car\answer N\@nil}
\ifodd\answer \else \OnlyDescription \fi
\DisableCrossrefs
\providecommand{\Lenv}[1]{\textsf{#1}}
\providecommand{\Lopt}[1]{\textsf{#1}}
\providecommand{\pkg}[1]{\texttt{#1}}
\setlength\hfuzz{2pt}
\begin{document}
\DocInput{dotseqn.dtx}
\end{document}
%</driver>
% \fi
%  \CheckSum{211}
%  \GetFileInfo{dotseqn.dtx}
%  \renewcommand\baselinestretch{1.13}
%  \title{DOTSEQN: Flush-left equations with dotted leaders to the
%    equation number}
%  \author{Donald Arseneau}
%  \date{version \fileversion, revised \filedate}
%  \maketitle
%
%  \section{Introduction}
%
%    The \Lopt{dotseqn} package provides a different format for typesetting
%    equations, one reportedly used in `old style Britsh books'~-- equations
%    aligned on the left, with dots on the right leading to the equation
%    number.  Like this
%    \begin{itemize}
%    \item[]$\displaystyle x = \frac{-b\pm\sqrt{b^2-4ac}}{2a}$\leaders
%      \hbox{\kern4pt.\kern4pt}\hfill(1)
%    \end{itemize}
%    \iffalse
%              !     Of course you should type equations normally!
%              !     I cheated here so that this documentation could
%              !     be printed *without* installing dotseqn.sty.
%    \fi
%    Equations without numbers have no leaders.
%
%    This package behaves a lot like \Lopt{fleqn} (so that option is
%    superfluous).  In particular, the equations are not centered, but
%    indented a constant distance from the left margin, controlled by
%    \cs{mathindent}.  The default indentation is the same as for lists.
%
%  \section{Affected environments}
%
%    Math environments with equation numbers, \Lenv{equation} and
%    \Lenv{eqnarray}, are changed to produce left-justified equations,
%    and to draw dotted leaders between the equation and the equation
%    number.  If there is no number, as specified by \cs{nonumber},
%    then no leaders are drawn.
%
%    Other math environments~-- \Lenv{displaymath}, \Lenv{eqnarray$*$},
%    and |\[|~\dots\,|\]| -- do not produce equation numbers and therefore
%    behave the same as they do under the \Lopt{fleqn} document option.
%
%  \section{Options}
%
%    This package is very similar to the \Lopt{fleqn} document class option,
%    so specifying \Lopt{fleqn} is superfluous.  The \Lopt{leqno} class
%    option is forbidden because it is incompatible with the dotted style
%    of equation labeling. (Where would the leaders go?)
%    As is the case for the ordinary \Lopt{fleqn} option, the equation
%    indentation is controlled by the length \cs{mathindent}, with a default
%    indentation equal to the indentation of lists.  It can be changed in
%    the document preamble (with \cs{setlength}).  Reasonable alternatives
%    are zero (|0pt|) or the paragraph indentation (\cs{parindent}).
%
%    The \pkg{dotseqn} package has two options of its own which can be
%    specified with the \cs{usepackage} command.  The \Lopt{leftjust} option
%    causes the first column in \Lenv{eqnarray} and \Lenv{eqnarray$*$}
%    to be flush left.  Ordinarily, these pieces are flush right against
%    the middle column.
%
%    Another option is \Lopt{nocolsep}, which gets rid of that (ridiculous)
%    extra space in the second column of \Lenv{eqnarray} equal to  twice
%    \cs{arraycolsep}.  I would prefer to make \Lopt{nocolsep} the default,
%    but it seems best that the normal behavior agrees with the usual
%    documented behavior of \LaTeX.
%
%    This typeset documentation was generated by running \LaTeX\ on
%    |dotseqn.dtx|, and |dotseqn.sty| is generated the same way.  This
%    is my experiment with single-file |dtx| distribution.
%
%  \StopEventually{}
%
%  \section{The implementation}
%
%  \iffalse
%<*package>
%  \fi
%
% \begin{macro}{\EqnDots}
%    The definitions here are based closely on the class option \Lopt{fleqn}.
%    The principal difference is that dotted leaders are used to replace
%    some horizontal filling commands.  The leaders are given by the fairly
%    typical \cs{leaders} command:
%    \begin{macrocode}
\newcommand\EqnDots{\leaders\hbox{\kern4\p@ .\kern4\p@}\hfill}
%    \end{macrocode}
%    which can be changed by the intrepid document designer.
% \end{macro}
%
% \begin{macro}{\mathindent}
%    Imitating the definitions for \pkg{fleqn}, we define a math indentation,
%    but only if it is not defined already, and set the default indentation
%    equal to the outer-level list indentation.
%    \begin{macrocode}
\@ifundefined{mathindent}{\newdimen\mathindent \mathindent\leftmargini}{}
%    \end{macrocode}
%    The implementation in \pkg{fleqn.clo} is complicated by the need
%    to delay assigning the default value to \cs{mathindent}, but that is
%    not necessary here because packages like this are processed after the
%    document class is fully declared.
% \end{macro}
% \begin{environment}{displaymath}
% \begin{macro}{\[}
%    The environment \Lenv{displaymath}, alias |\[|~\dots\,|\]|, is
%    identical with its definition in \pkg{fleqn.clo}.  First the beginning
%    \begin{macrocode}
\renewcommand{\[}{\relax \ifmmode\@badmath \else
 \begin{trivlist}%
  \@beginparpenalty\predisplaypenalty \@endparpenalty\postdisplaypenalty
  \item[]\leavevmode \hbox to\linewidth\bgroup $\m@th\displaystyle %$
  \hskip\mathindent\bgroup
 \fi}
%    \end{macrocode}
% \end{macro}
% \begin{macro}{\]}
% \dots and then the |\end{displaymath}| or |\]|
%    \begin{macrocode}
\renewcommand{\]}{\relax\ifmmode \egroup $\hfil% $
    \egroup \end{trivlist}%
  \else \@badmath \fi}
%    \end{macrocode}
% \end{macro}
% \end{environment}
%
% \begin{environment}{equation}
%    The \Lenv{equation} environment begins exactly the same was as in
%    \pkg{fleqn}, but it ends using \cs{EqnDots} in place of \cs{hfil}.
%    \begin{macrocode}
\renewenvironment{equation}%
 {\@beginparpenalty\predisplaypenalty \@endparpenalty\postdisplaypenalty
     \refstepcounter{equation}\trivlist \item[]\leavevmode
       \hbox to\linewidth\bgroup $\m@th% $
         \displaystyle \hskip\mathindent}%
 {$\EqnDots % $   Replace `\hfil' with dotted leaders `\EqnDots'.
     \displaywidth\linewidth\hbox{\@eqnnum}\egroup \endtrivlist}
%    \end{macrocode}
%    The \Lenv{equation$*$} environment need not be redefined because it
%    is defined in terms of \Lenv{equation}, and it has no equation number
%    so it needs no \cs{EqnDots}.
% \end{environment}
%
% \begin{environment}{eqnarray}
%    The \Lenv{eqnarray} environment has the most changes.  In order to make
%    the leaders fill the width, the space between the equation and the
%    equation number must be part of a column entry; it cannot be filled by
%    tabskip glue.  Thus, this redefinition has one fewer column~-- there is
%    no separate column for the equation number!  This also necessitates
%    a changed definition for |\\|, so see \cs{@@eqncr} below. As yet,
%    \Lenv{eqnarray} is not described in detail in \pkg{classes.dtx}, but
%    there are some useful comments in \pkg{ltmath.dtx}. They are reproduced
%    here:
%
%    \begin{quotation}
%    To get a proper \cs{@currentlabel} we have to redefine it for the
%    whole display. Note that we can't use \cs{refstepcounter} as this
%    results in \cs{@currentlabel} getting [frozen at the beginning] and
%    thus always writing the first label to the \texttt{.aux} file.
%    Default is for left-hand side of equations to be flushright.
%    To make them flushleft, |\let\@eqnsel = \hfil|
%    [this is just what the \Lopt{leftjust} option does].
%    \end{quotation}
%
%    \noindent In further explanation, I should add that the counter
%    \cs{@eqcnt} is globally set in each of the columns so that |\\|
%    will know how many column separators (|&|) to insert before the
%    equation number. This redefined version has only 3 alignment columns,
%    so this insertion is modified (in \cs{@@eqncr}).  In the third and
%    final column, after the math material, there is a confusing
%    |\hskip\@centering|.  In fact, this has no effect when an
%    equation number is present, because the \cs{EqnDots} are infinitely
%    more stretchable, but the \cs{hskip} is left in place for the case
%    of \cs{nonumber} which needs some filling.
%
%    So, in the absence of tabskip glue, how is the equation forced out
%    to the full line width?  Using \ |\halign to\linewidth|' \ will not
%    work because it only stretches \cs{tabskip} glue. The original
%    \ |\halign to\linewidth| \ is retained only to give overfull box
%    warnings when an alignment is too wide for the page.  To stretch
%    the third coulumn out to the full line width, a blank row is added
%    at the end of the alignment by |\end{eqnarray}|, with a single entry
%    spanning the full line width.  This makes use of an often annoying
%    behavior of \TeX: when multiple columns need to be enlarged to match
%    a spanning entry, all the extra space is added to the final column.
%    The extra `phantom' row uses a negative \cs{vskip} and a strut with
%    a depth given by \cs{prevdepth} to simulate the depth of the expected
%    last line in the equation array.  A temporary macro is used to preserve
%    the value of \cs{prevdepth} and use it after an end-of-group.
%
%    A special command `\cs{DEQ@acs}' surrounds the entry for the second
%    column.  Ordinarily this gives |\hskip 2\arraycolsep|, but with the
%    \Lopt{nocolsep} option it gives |\null| (like~|\mbox{}|), which produces
%    the natural math-mode spacing around the entry.
%    \begin{macrocode}
\renewenvironment{eqnarray}{%
   \stepcounter{equation}%
   \def\@currentlabel{\p@equation\theequation}%
   \global\@eqnswtrue \m@th \global\@eqcnt\z@  \tabskip\mathindent
   \let\\\@eqncr \setlength\abovedisplayskip\topsep
   \ifvmode \addtolength\abovedisplayskip\partopsep \fi
   \addtolength\abovedisplayskip\parskip
   \setlength\belowdisplayskip\abovedisplayskip
   \setlength\belowdisplayshortskip\abovedisplayskip
   \setlength\abovedisplayshortskip\abovedisplayskip
   $$\everycr{}\halign to\linewidth% $$
   \bgroup
     \hskip\@centering
     $\displaystyle\tabskip\z@skip{##}$\@eqnsel&%
     \global\@eqcnt\@ne \hfil${\DEQ@acs##\DEQ@acs}$\hfil&%
     \global\@eqcnt\tw@ $\displaystyle{##}$\hskip\@centering\cr%
 }% end of "\begin" part
 {\@@eqncr
   \noalign{% vertical skip up to overlay phantom line
      \penalty\@M \vskip-\prevdepth
      \edef\@tempa{\omit\span\omit\span\omit   % span three columns
        \vrule\@depth\the\prevdepth \@width\z@ % strut of proper depth
        \kern-\mathindent \kern\linewidth}%    % full line width
      \nointerlineskip \expandafter % use saved |\@tempa| outside group
   }\@tempa\cr
   \egroup
   \global\advance\c@equation\m@ne$$% $$
   \global\@ignoretrue
 }
%    \end{macrocode}
% \end{environment}
%
% \begin{macro}{\@@eqncr}
%    Now the \cs{@@eqncr} macro (alias |\\|) needs redefining.  The
%    number of inserted |&| separators is reduced, the warning is
%    removed because it is quite proper to already be in the last column,
%    and the placement of the equation number is changed: instead
%    of letting the \cs{halign} template do the spacing, \cs{@@eqncr}
%    leaves math mode, inserts the leaders, typesets the equation number
%    (using \cs{@eqnnum}), then begins math mode again so things will
%    balance when the template tries to end math.  Note: \cs{reserved@a}
%    is set to something innocuous before \cs{ifcase} just in case it
%    was let equal to something like \cs{fi}.
%    \begin{macrocode}
\def\@@eqncr{\let\reserved@a\@empty
   \ifcase\@eqcnt \def\reserved@a{& &}\or \def\reserved@a{&}\fi
   \reserved@a
   \if@eqnsw \egroup $\EqnDots \@eqnnum $\bgroup \stepcounter{equation}%
   \fi \global\@eqnswtrue\global\@eqcnt\z@\cr}
%    \end{macrocode}
% \end{macro}
%
%    Finally, we handle the package options \Lopt{leftjust} and
%    \Lopt{nocolsep}. Declare the default behavior; declare the optional
%    behavior, as described above; and process any options specified.
%    \begin{macrocode}
\def\DEQ@acs{\hskip\tw@\arraycolsep}
\DeclareOption{leftjust}{\let\@eqnsel\hfil}
\DeclareOption{nocolsep}{\let\DEQ@acs\null}
\ProcessOptions
%    \end{macrocode}
%
%  \iffalse
%</package>
%  \fi
%
% \Finale
\endinput
