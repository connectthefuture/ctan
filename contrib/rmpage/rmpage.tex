% copyright 1996 and 1997 R. McDonnell
% email rebecca@astrid.u-net.com

\documentclass[11pt,loose,twoside,touchwider,longish,
                      noheaders,a4paper,notstdmargins]{report}

\usepackage{rmpage,% rmstuffnew,
shortvrb,tabularx,layout,xspace}

% Extracted from rmstuffnew.sty
\makeatletter
% Might as well
%
\newcommand*{\OzTeX}{O\kern-.03em z\kern-.15em\TeX}
\newcommand*{\OzTools}{O\kern-.03em z\kern-.10em Tools}
% This command is horrible and doesn't work right with LaTeX2e.  
% Please forgive it.
%
% Could probably re-implement \unit with 2e's requiremathmode stuff.  RTFM.
% 
\newcommand*{\unit}[1]{\ifmmode\@unit{\hbox{#1}}\else$\@unit{\hbox{#1}}$\fi}%

\let\units=\unit

\def\@unit#1{{\setbox0=\hbox{#1}\ifdim\wd0<1.125em\,\else\>\fi\box0}}           


% at char for tt family
\newcommand*{\at}{\char'100}
\makeatother
% End extracted from rmstuffnew.sty

%
% Note: Search and replace %% \DeclareOption -> \asleepoption
%                          \DeclareOption -> \awakeoption
%
\def\awakefount{\mdseries\ttfamily}
\def\asleepfount{\mdseries\ttfamily\slshape}
\def\awakeoption#1{\item[{\awakefount#1}]}
\def\asleepoption#1{\item[{\asleepfount#1}]}

\newenvironment{ikkystuff}{\begin{quotation}\small}{\end{quotation}}

\newcommand*{\keypress}[1]{{\ttfamily #1}}
\newcommand*{\filename}[1]{{\ttfamily #1}}
\newcommand*{\packname}[1]{{\sffamily #1}}
\DeclareRobustCommand*{\comname}[1]{{\ttfamily\makeatletter\bs #1\makeatother}}
\newcommand*{\envname}[1]{{\ttfamily #1}}
\newcommand*{\classname}[1]{{\ttfamily #1}}
\newcommand*{\optname}[1]{{\ttfamily #1}}

\newcommand*{\rmpage}{\classname{rmpage}\xspace}

\MakeShortVerb{\|}

\author{Rowland McDonnell\\
\texttt{rebecca@astrid.u-net.com}}

\title{The \packname{rmpage} package\\Alpha documentation---trust
nothing!}

\date{Some time ago}

\edef\RMsaveparskip{\the\parskip}
\edef\RMsaveparindent{\the\parindent}

\newcommand{\shorty}{\setlength\parskip{\RMsaveparskip}%
\setlength\parindent{\RMsaveparindent}}

\newcommand{\squish}{\setlength\itemsep{0pt}}

\begin{document}

\maketitle

\begin{abstract}

The \rmpage package lets you change page layout parameters in small
steps over a range of values using options.  It can set |\textwidth|
appropriately for the main fount, and ensure that the text fits inside
the printable area of a printer.

An \rmpage-formatted document can be typeset identically without
\rmpage after a single cut and paste operation.  Local configuration
can set defaults: for all documents; or by class, by printer, and by
paper size.

The geometry package is better if you want to set layout parameters to
particular measurements.

\end{abstract}


\tableofcontents

\chapter{Introduction}

This documentation needs re-writing and tidying up, and \rmpage needs 
finishing, but the package is 99\% finished even if the
documentation's a mess -- I'm not happy with it but it's probably 
usable, so here it is.

The \packname{rmpage} package has five files: 
\filename{rmpage.tex} (documentation), \filename{rmpage.sty} (the 
package file itself), \filename{rmplocal.gfc} (a configuration file), 
and \filename{rmpgen.cfg} (another configuration file), and 
\filename{readme}.

Only \filename{rmpage.sty} and \filename{rmpgen.cfg} \emph{need} to be 
on your \TeX\ search path -- read on to find out more about 
configuring \packname{rmpage}.

I seem to have ground to a halt on this package -- it's worked fairly 
well for some time and I don't seem to have been able to get things 
together to sort out the documentation and stuff.  I've come across 
one bug only in the last year, so I thought the best idea would be to 
upload the package to CTAN so someone else could use it, and perhaps 
any feedback I get might persuade me to pull my finger out and tidy it 
all up.

The \filename{dtx} files aren't ready to be typeset yet---this is the
only \LaTeX able documentation.

What I do with \rmpage in the future depends mainly on what you tell
me: if you've used \rmpage (or decided not to), I would consider it a
great kindness if you told me why.  I'm also interested in what you
like and dislike, any suggestions you have, and anything else about
this package and its documentation---an email message just saying
`\rmpage is rubbish, \packname{geometry}'s much better for\ldots'
would be useful if that's what you think.

The chapters on how things work and all the options still need a lot
of work, and the chapter on configuration isn't how I'd like it to be.
This document will eventually be finished and included in a proper
\filename{.dtx} file.  I thought releasing this package now was best,
because I've suddenly become employed, and this final polishing will
take quite a long time.

\rmpage sets \LaTeX\ page layout parameters to user-controlled
values, without the user having to deal with particular measurements,
check whether the result will fit inside the printing area of the
selected printer, and so on.  This is done with options like:
\optname{wider}, \optname{noheaders}, \optname{lower}, and
\optname{morecolsep}; all changable layout parameters can be varied
in small steps over a range of values.

\rmpage only changes parameters like \comname{textwidth} and
\comname{columnsep}: those lengths that affect where the text goes on
the page and how much of the page it occupies.  It doesn't change
layout parameters that affect the internal appear of your text, such
as paragraph indents, spacing around section headings and the like.
One of the design aims was to make changing page layout similar to a
wysiwyg word processor, where you can use the mouse to fiddle with the
page layout, making things a bit bigger and smaller until it's just
right.

There's a configuration file for you to play with and hooks galore:
\rmpage is meant to be configured the way you want it---if you
normally use A4 paper and produce most documents without headers, you
can configure \rmpage to give you that by default, over-ridable by
passing options from your document.  If you follow the instructions,
local configuration won't stop you producing documents with a modified
layout that typeset identically on different systems---you can even
copy the modified layout data into your document, and it'll typeset
identically on a \LaTeX\ installation without \rmpage.

The package is meant to be used directly in \LaTeX\
documents, and for creating local classes.  I have used it, for
example, to create a class for producing theses according to the
regulations.  I can play about with the layout as much as I like,
because \rmpage will ensure that the final document is within
specification---some of the code that does this can be seen in the
configuration file.


Aside: this documentation was hell to write---I hadn't realized what a 
monster this package was until I came to document it.  With a bit of 
luck, you'll be able to use \rmpage as your flexible friend like I do.  
I never write a \LaTeX\ document without it.  And strangely, even 
though \rmpage has been over a year in the making, and has been used 
constantly in that time, I spotted lots of improvements that needed 
doing and bugs that needed removing while I was writing this 
documentation, which I didn't start seriously until I thought I'd 
finished the first release version of the package.  I have a suspicion 
that writing detailed documentation of software is a very, very useful 
part of getting it right, especially when the software's moderately 
complicated.  Personally, I'm now suspicious of anything that isn't 
documented thoroughly; I've looked at the various packages I've got 
from CTAN and use myself, and I find that those with thorough, clear 
documentation of what they do \emph{and} how they do it are the ones 
that seem most usesful and flexible.  The poorer-documented packages 
appear to be less well thought out and less able to do things my way, 
rather than the author's.  By the way, when I refer to an explanation 
of how something works, I'm not talking about the annotated code with 
added jargon that the \LaTeX\ core appears to be turning into.  This 
is fairly useful and probably inevitable, but people like me who 
aren't expert hackers can't understand it and can't find out how to 
understand it, which is worse.


\section{\LaTeX's standard classes}

\label{gen:standardclassproblems}

\LaTeX's standard classes have all been designed well, and the
\LaTeXe\ versions have a cunning way of calculating
\comname{textwidth} and \comname{textheight} that adapts the page
layout to the size of paper, which is useful for people like me who
never print on US letter paper.

But the standard \LaTeX\ page layout was intended for paper sizes near
US letter, and assumes you'll be using headers and footers.  If
you're not, things begin to look a bit off; your printed
pages have a larger white space at the top than at the bottom, which
looks `bottom heavy'---standard typographical design has the larger
gap at the bottom rather than the top, which looks better to the eye,
even if only because we're used to it.  Personally, I rarely use
headers, which means the standard \LaTeX\ classes produce an
unpleasant output most of the time.  And for some reason, the standard
classes don't allow enough space in the header box for type sizes
bigger than 10pt.  This is one of life's inexplicable mysteries.

If you have different sized margins, the standard classes always make
the outside margin the larger one, and the margins are in fixed
proportions to each other.  This is fine for conventional typeography,
but not so good if you're formatting things to go in a ring-binder,
for example.  Some flexibility in this matter would be good.

And there's the matter of the fixed text height and width of the
standard classes.  Now then, this is a good idea in one respect,
because the user doesn't get to typeset lines that are too long or too
short without having to do a bit of work.  On the other hand, I for
one have often \LaTeX ed a letter which has \emph{just} run on to two
pages; being able to extend the page a small amount would be good
under these circumstances.  Admittedly, \LaTeXe\ introduced the
\comname{enlargethispage} command which can help out, but if you make
a page more than one or two lines longer this way, the results look a
bit iffy---this command only extends \comname{textheight}, which
reduces the size of the bottom margin, and can make the page look
bottom-heavy.

A more subtle problem with the standard fixed widths is that one
factor affecting the ease of reading is the line width in characters,
not in inches.  So if you're using a fount with a different number of
characters per inch, you can end up with a line that is noticably too
long (Zapf Chancery), or too short (Lucida Casual).

\section{How does \rmpage help?}

\rmpage has options (the details come later) to format a page for
typesetting with or without headers or footers, and always leaves
enough space for a \comname{normalsize} line of text in the header
box.  You can specify which margin you want to be the larger one, and
adjust the relative proportions of the inside and outside margins.
And there are options to change: the width and height of the text
taking into account the size of the main text fount; the position of
the text on the page (up and down, and left and right); space between
columns, above footers, below headers; size and position of marginal
paragraphs; and lots, lots, more, so hurry!  Buy now while the sale's
still on!

\rmpage isn't like the \packname{geometry} package: you rarely work
with measurements\@.  \rmpage's options are of the form:
\optname{wider}/\optname{narrower} or
\optname{moreheadsep}/\optname{lessheadsep}, and they scale sizes up
and down over a large range in fairly small steps if you want.  And
unlike the \packname{Koma-Script} bundle of packages, you're not
limited to a fixed aspect ratio printing area.  Not that there's
anything wrong with \packname{Koma-Script} or \packname{geometry}:
they do a different job.

\rmpage can take into account different founts, number of columns, and
all sorts of things, including the physical printing area of your
printer.

Options exist to change all (I think I got them all) of \LaTeX's basic
page layout parameters, and some new parameters that I created for:
setting the size of marginal paragraphs, and changing the position of
the text area in relation to the paper area.  There are some  \LaTeX\
page layout parameters that you don't get to affect
directly; this is because of the way I looked at page layout when I
wrote \rmpage.  These include \comname{evesidemargin},
\comname{topmargin}, and some others, which are fine for a computer
assembling a page, but not so good for me, trying to describe a
layout I want in terms I like.

I have paramaterized everything that didn't run away, so you can, for
example, write a thesis class that limits the text area as
specified in the regulations, but still allows the user some
flexibility if it's needed.

And \rmpage knows about a lot more paper sizes, including envelopes
and ISO long sizes, which it attempts to handle in an intelligent
fashion (it knows one's likely to want to print out 1/3 A3 on A4
paper, for example).

\section{Installing \rmpage}

You can install \rmpage by running \LaTeX\ on the file
\filename{rmpage.ins}.  The \filename{dtx} files might be useful if
you want to dig around inside \rmpage, otherwise throw them away with
the \filename{ins} files.  Put the \filename{sty}, \filename{pko},
\filename{cfg}, and \filename{gfc} files somewhere in your \TeX\
search path.

The file \filename{rmpage.sty} is the package you call from your
\LaTeX\ document, and the \filename{cfg} files contains most of the
options and locally-configurable things.  Please don't change any of
these files---make a copy called \filename{rmplocal.cfg} of either:
\filename{rmpgen.cfg} or \filename{rmplocal.gfc}, and change that
instead.  The file \filename{rmplocal.gfc} is the same as
\filename{rmpgen.cfg} with some options commented out to make it
faster.

\rmpage is meant to be configured to suit you; I suggest that everyone
changes the default options in \filename{rmplocal.cfg}.  Don't do this
just yet unless you really want to---have a read of
chapter~\ref{chap:use} and section~\ref{optcfg:settingup} first.


\section{Compatibility}
\label{intro:compatibility}

\rmpage has been tested with the June 1996 release of \LaTeX. It
appears to work well with the \classname{article}, \classname{report},
\classname{letter}, \classname{book}, and \classname{slides} classes,
and a number of local classes based on these.  \rmpage doesn't work
well with the \classname{ltxdoc} class, because of what
\classname{ltxdoc} does with marginal paragraphs.  Mind you,
\classname{ltxdoc} doesn't seem to have much success with marginal
paragraphs anyway.  \rmpage seems happy with the \classname{ltxguide}
class, but I've not tested it thoroughly.

Because \rmpage only changes \LaTeX\ parameters at the beginning of a
\LaTeX\ run, it doesn't in general have trouble working with other
packages and classes as long as \rmpage is loaded last.  \rmpage
includes code to help it work with the \packname{PSNFSS} packages, the
\packname{beton} package, and the \packname{foils} class.  The
\packname{beton} package needed a little extra work because it changes
\comname{baselineskip} after all packages have been loaded;
\packname{foils} uses four (rather than three) extra-large base point
sizes, which again took a little extra code.  The \packname{PSNFSS}
package support is just for convenience, so you can load a fount with
one option rather than a \comname{usepackage} command and an option
to \rmpage.

In general, if \rmpage is loaded after the document's normal size
fount has been selected, and after the document class has finished
setting the various text layout parameters, there should be no
problems---.  If you are combining \rmpage with a package that also
changes page layout parameters, you will have to find out how both
packages work to ensure you get what you want.  Loading \rmpage last
is usually enough to ensure everything works right.  For example,
\rmpage must be loaded after \packname{setspace} has been used to set
the line spacing for a document, so that \rmpage can set
\comname{textheight} to a valid value.

\begin{ikkystuff}
Two things to watch for are changes to the main document
fount, and changes to \comname{baselineskip}.  \rmpage calculates
\comname{textheight} as an integer multiple of \comname{baselineskip}
plus \comname{topskip}.  If these are changed after \rmpage has been
called, you'll probably have lots of bad page breaks.
\comname{textwidth} is normally taken to be a certain number of
average-sized characters; if \rmpage has a false idea about the
typeface and size you are using, \comname{textwidth} will probably
not be set appropriately.

If you ensure that normal size in the normal body fount with the
normal \comname{baselineskip} has been selected before loading
\rmpage, everything should always be fine.  The \packname{beton}
package sets \comname{baselineskip} \comname{AtBeginDocument}; other
packages and classes which do this kind of thing will almost certainly
need attention to get \rmpage to work right.
\end{ikkystuff}

\section{What made me write \rmpage}

I started using \LaTeX~2.09 about eight or nine years ago when I was
an undergraduate ('ere, Colin, 'ave you got a word processor on that
bloody great workstation of yours?  No, but I've got something
better\ldots) and quickly realised that the standard formats looked
daft on A4 paper.  No worries, there was this \packname{a4l} style
file that sort of sorted things out.  After a bit, I wanted slightly
wider columns, sometimes centred on the page and sometimes not.  I
found out how to write style files and eventually I ended up with a
suite of simple style files that let you fiddle with the printing area
by selecting one file for the kind of text area you wanted: centred,
not centred, wide, very wide, long, standard \LaTeX\ length, my
standard length, all hard-coded to A4 paper with no headers.

I was chewing over the idea of introducing the idea of paper sizes
myself, so I could write style files that weren't hard-coded to any
particular paper size, when \LaTeXe\ came out with the job already
done, and packages you could pass options to and all that good stuff.

So what I did was use my original \LaTeX\ 2.09 packages to form the
basis of the original \rmpage, which let you do everything the
original styles did, but all in one file instead of over a dozen, and
let you format your text on any size paper with or without headers or
footers.  I decided what I really wanted was a package that gave me
wysiwyg-style flexibility (you know, the way you can extend line
length just a bit with the mouse, without worrying about the actual
numbers) without producing poor layouts, and taking advantage of
\LaTeX's `extensive macro capability'\footnote{Whoever first used this
phrase should be shot, after the politicians but before the lawyers}
to build in a bit of intelligence.

I looked at the \LaTeXe\ way of calculating the text region, and used
those ideas in my package, and started adding options to change more
aspects of the printing area.  And more, and more, and more.  The
result was a mess that could change almost anything, and supported any
size paper, different printers, and so.

The transmogrification from the original large mess to the current
large mess was in two main steps: I started out by tidying up bits of
code piecemeal, rationalized command names and the like, and tried to
work out how everything worked together.  I'd done as much of this as
I could, then started to document \rmpage systematically for myself,
making notes on what I intended to change when I'd finished
documenting the package.

Eventually I got fed up, and decided that a good spring cleaning was
in order.  I read some typeography books, some British Standards
(which are mainly ISO standards too, so I'm not being parochial)
looked at the \packname{koma-script}, \packname{vmargin}, and
\packname{geometry} packages, and cleaned up the code good and proper,
none of this pussy-footing around with careful plans.  I parameterized
some things which had escaped the first time round, changed the
numbers to give a rational, \ae sthetically pleasing, and functional
spread of values for everything\footnote{That's my story, and I'm
sticking to it.}, and got the whole thing more-or-less sorted, with a
few extra bits thrown in where they were missing.  The result seems
much more useful that my original careful plan would have produced, so
I'm happy.  Writing the documentation has smoothed out several things
I wasn't very happy with to begin with, improved some features, added
some others, and unearthed more bugs than was reasonable given that
I'd tested the bloody thing, honest.

\section{Future plans}

What I do with \rmpage depends mainly on what you tell me; if you've
used \rmpage (or decided not to), I would consider it a great kindness
if you let me know what you think about it---two words or two pages:
whatever you might tell me would be useful and appreciated.  I'm
interested in what you like and dislike, any suggestions you have, and
anything else about this package and its documentation, gonzo
ontology, the books of Harlan Ellison and Robert Anton Wilson, good
beer, fast bikes, and prawn crackers, but don't worry about most of
that.

I intend to make the \rmpage code more elegant, and make \rmpage more
\emph{useful}---if you think of any useful changes or additions,
please let me know.  Of course I'll try to fix any bugs and
misfeatures that you report.

I'm working on reducing the restrictions on the use of options---I
hope to arrange things so that more options can be used in an
\comname{ExecuteOptions} statement, and things like that.

A project that I'll finish eventually is a version of \rmpage that can
be processed with \packname{doc} and \packname{docstrip} in the
conventional way, but don't hold your breath---it's taken me perhaps
two years to get this far.  But I will fix bugs quicker than that.

\chapter{Using \rmpage}
\label{chap:use}

Section~\ref{use:changingpagelayout} on
page~\pageref{use:changingpagelayout} of this chapter describes how to
use \rmpage to change the page layout.  The sections before that are
background information intended to explain the jargon and conventions
I use in this document, and a little about the philosophy behind
\rmpage---all of which should make the rest of this an easier read.

\rmpage is a \LaTeX{} package written to change the page layout.  You
can control it with options, and by editing a configuration file.
You don't need to edit the configuration file for \rmpage to be
useful, but doing so can save you time and effort.  I suggest you read
this chapter and play with \rmpage a little, then read
chapter~\ref{chap:config} on page~\pageref{chap:config} to find out
about local configuration.

\rmpage is a normal \LaTeX\ package, so put:
\begin{verbatim}
\usepackage{rmpage}
\end{verbatim}
in the preamble of your document.

You can pass options directly to \rmpage in the optional argument of
the \comname{usepackage} command, but I almost always put all my
options in the optional argument of the \comname{documentclass}
command.  This is because \rmpage uses several standard options; if I
made a habit of passing options to \rmpage directly, I might forget
that the document class needs to know about (for example) the
\optname{twocolumn} option.

\rmpage was designed to change \LaTeX's page layout parameters, but it
doesn't change all of them in a direct way---read on to see what I
mean.  If you aren't familiar with \LaTeX's basic page layout
parameters, have a look at a copy of the \LaTeX\ manual, ask a
convenient guru, or use the \packname{layout} package to show you what
they are---I think that what they are is obvious from their names, but
I'm not a Finnish \LaTeX\ novice, so my opinion is clearly suspect.

Figure~\ref{fig:layout} on page~\pageref{fig:layout} shows the output
from the \packname{layout} package's \comname{layout} command.  I've
fooled the \comname{layout} command into thinking this is a one-sided
document, and this document is formatted without headers, so you can't
see clearly that there is a box of height \comname{headheight} a
distance \comname{headsep} above the main text body.  This box
contains the header, if one exists.

\section{\rmpage's view of a page}

The user interface to \rmpage takes the view that a page consists of
a physical paper size, with a non-printable border around its inside
edge.  \rmpage will not produce a layout that attempts to put ink
beyond the printable region thus defined.

\rmpage considers the \emph{main text block} to consist of the header,
body text, and footer.  This is the region that \rmpage moves up and
down with \optname{altitude} options, and left to right with
\optname{offset} options.  The body text is part of the main text
block, and is the matter that fits inside the area defined by
\comname{textwidth} and \comname{textheight}.

Marginal paragraphs stick on the side of the main text block.  They
begin a certain distance from the side of the body text, and extend to
within a certain distance of the edge of the page, or to a certain
maximum width.

The space between the main text block and the edge of the paper: top,
bottom, left, and right; is considered as four different margins,
measured from the edges of the paper.  \LaTeX's
\comname{evensidemargin}, \comname{oddsidemargin}, and
\comname{topmargin} parameters measure margins differently, from a
point one inch in from the top left hand corner of the paper.

Asking for \optname{noheaders} or \optname{nofooters} reduces the size
of the space left for the approriate element to
0\units{pt}---respectively \comname{headheight} \emph{and}
\comname{headsep}, or \comname{footskip}, so the main text block might
consist of body text only.

\section{Parameter naming conventions}

I refer to various parameters in this document, so you might find it
useful to know what the names are supposed to mean.

In general, parameter names ending in |clearance|, |clear|, or
|margin| refer to a distance from the edge of the paper.  Parameter
names containing |sep| refer to a distance between text elements on
the page.  |mpar| means marginal paragraph; |width| refers to
horizontal dimensions; |height| or |length| refer to vertical
dimensions.

Parameter names ending in |option| contain numbers that control what
value the parameter referred to is set to.  Parameter names containing
|min| or |max| are minimum or maximum limits for the parameter
referred to.

All \rmpage's parameters and commands begin with \comname{RM}; those
beginning with \comname{RM@} are not meant to be set outside a class,
package, or configuration file.  The one exception to this is the
\comname{sloppiness} command.


\section{Don't read me}

Without any options specified by your document or the configuration
file, on paper about A4 or US letter size, \rmpage will produce a
slightly different format to the standard classes: \comname{textwidth}
and the position of the text body on the page will be a fraction of a
point different; everything else should be the same (if not, you've
found a fault---please let me know).  Smaller paper sizes, around A5
or half US letter, will have a noticeable wider \comname{textwidth}.
You can force \rmpage to make \comname{textwidth} and
\comname{textheight} identical to the standard values (so line and
page breaks are not changed) with the \optname{stdwidth} and
\optname{stdlength} options; main text block positioning is never
identical to standard.

\rmpage with the standard configuration file follows \LaTeX's defaults
and produces layouts very close to standard \LaTeX. You can change
this by editing the configuration file---see chapter~\ref{chap:config}
for the details.  I wrote \rmpage expecting that everyone would edit
the configuration file to match their preferences; for example, if you
usually don't use headers, or if you usually don't print on US letter
paper.

If you are using typefaces other than the standard Computer Modern
Roman, or a package that changes \comname{baselineskip}, have a look
at section~\ref{intro:compatibility} on
page~\pageref{intro:compatibility} and
section~\ref{use:changingpagelayout} on
page~\pageref{use:changingpagelayout}---there are things that need
doing to avoid a poor layout.

The observant will notice that I prefer spelling things according to
the Oxford English Dictionary, rather than Webster's.  Fear not: I
realize that \LaTeX\ follows US English convention, so \rmpage
includes options spelt both ways where there's a difference.

\section{Option naming conventions}

\rmpage's options are largely of the form: \optname{narrowest},
\optname{narrower}, \optname{narrowish}, \optname{normalwidth}; or
\optname{mostheadsep}, \optname{moreheadsep},
\optname{moreishheadsep}, \optname{normalheadsep}.  These two examples
are each part of an option set (they continue with \optname{widish}
and \optname{lessishheadsep}).  Any option without \optname{touch} or
\optname{t@uch} in its name is considered a main option.

You should use only one main option from each set at a time---if you
do use more than one, \rmpage will apply the settings of the option
that is declared last in the package file.

The \optname{touch} options all step up or down one third of the way
(usually in a geometrical sequence) to the next main option.  The
\optname{t@uch} options are identical, but can only be used in class
and package files.  The following two examples produce an identical
\comname{textwidth}:
\begin{verbatim}
\usepackage[wider,t@uchwider,touchwider]{rmpage}
\usepackage[widest,touchnarrower]{rmpage}
\end{verbatim}

The following three examples produce a smoothly increasing
\comname{textwidth}:
\begin{verbatim}
\usepackage[touchwider]{rmpage}
\usepackage[widish,touchnarrower]{rmpage}
\usepackage[widish]{rmpage}
\end{verbatim}

It's complete okay, but slightly silly because it has no effect, to
combine, say, \optname{touchwider} and \optname{touchnarrower}.

\section{Changing the page layout}
\label{use:changingpagelayout}

\rmpage is controlled by options passed to it in the conventional way,
and by various things you can do to the configuration file.  This
section explains the basic use of most of the options.
Chapter~\ref{chap:alltheoptions} on page~\pageref{chap:alltheoptions}
lists all the options and what they do.  Chapter~\ref{chap:config} on
page~\pageref{chap:config} deals with the configuration file.


\subsection{Width of the main text block}

\label{use:width}

According to the text books, the optimum width of a block of text is
about 1.5--2.5 alphabets in the main fount.  This is about 45--75
characters (including spaces and punctuation) or ordinary English
prose.  When you set the width of the main text block, \rmpage
measures the width of one column, and warns you if it exceeds these
limits.  Note that the standard width is at the upper limit for
optimum readability; any increase will produce a warning.  I very
strongly suggest you use multiple columns if you find yourself using a
width wider than \optname{widish}.

\rmpage has options for producing multiple column layouts: see
section~\ref{use:multiplecolumns} on
page~\pageref{use:multiplecolumns} for more details.  If you are
producing displayed material (a single large table on a page, for
example), read the section below, called `On other width setting
controls'.

\rmpage sets the width of the main text block (the header, body text,
and footer) with these options, which are referred to as the width
option set:
\begin{tabbing}
\optname{narrowest}, \= \optname{narrower}, \= \optname{narrow}, \=
\optname{narrowish}.\kill
\optname{widest}, \> \optname{wider}, \> \optname{wide}, \>
\optname{widish}, \\
\optname{normalwidth},\\
\optname{narrowest}, \> \optname{narrower}, \> \optname{narrow}, \>
\optname{narrowish}.
\end{tabbing}

Make sure you only use one of the above options at a time.  The
\optname{touch} options can be used with any of the main options; they
are often exactly what's needed what used alone: \optname{touchwider}
and \optname{touchlonger} have often reduced my document's
page count to what I wanted.

The options:
\begin{tabbing}
\optname{touchwider} \= \optname{touchnarrower} \\
\optname{t@uchwider} \> \optname{t@uchnarrower}
\end{tabbing}
give a width one third of the way towards the next main option.  The
\optname{t@uch} options can only be used in a class or package file.

There are four main width options that pay no attention to the
\optname{touch} options.  They are:
\begin{tabbing}
\optname{halfinchmargins} \= \kill
\optname{stdwidth} \> Exactly the width calculated by the standard classes\\
\optname{fullwidth} \> The full width of the printable region\\
\optname{oneinchmargins} \>
The left and right margins sum to 2\units{in}---average 1\units{in}\\
\optname{halfinchmargins} \>
The left and right margins sum to 1\units{in}---average 0.5\units{in}
\end{tabbing}
The \optname{fullwidth} option fills the width of the printable region
as well as it can, ensuring the specified relationship between the
inside and outside margins.  You will usually get a larger
\optname{textwidth} if you also ask for \optname{centre} or
\optname{center}; see section~\ref{use:horizontalpos} on
page~\pageref{use:horizontalpos}.

The \optname{one-} and \optname{halfinchmargins} options give inside
and outside margins of that measurement only if you are printing
centred; otherwise, the average margin size is as specified (e.g.,
inside 1.2\units{in}, outside 0.8\units{in}; $(1.2 + 0.8) / 2 = 1$)

\subsubsection{Other width setting controls}

The initial \comname{textwidth} is normally calculated as the smaller
of two different widths: one, a certain number of characters; the
other, a certain fraction of the \comname{paperwidth}.  The precise
figures depend on the width options you've used.

This is not always appropriate---for example, if you are producing
a weekly timetable on A4 landscape paper, I can't see why
\rmpage should pay attention to the character-based width.  So I
created these options:
\begin{tabbing}
\optname{characterwidthset} \= Choose the character-based width
regardless \\
\optname{paperwidthset} \> Choose the paper-based width regardless \\
\optname{bothwidthset} \> Default: choose the smaller of the two
widths\\
\end{tabbing}

Note that \rmpage never ignores its paper-based limits: saying
\optname{characterwidthset} will produce a printable layout that takes
notice of all the restrictions documented elsewhere.

I've provided a \optname{ringbinding} option which sets the minimum
allowed inside margin to at least 15\units{mm} if you are printing in
portrait orientation, and does nothing but warn you if you are using
landscape orientation.  It's probably not a good idea to use this with
long paper sizes, but no check is made.

\subsection{Length of the main text block}
\label{use:length}

\rmpage sets the height  of the main text block (the header, body text,
and footer) with these options, which are referred to as the length
option set:
\begin{tabbing}
\optname{shortest}, \= \optname{shorter}, \= \optname{short}, \=
\optname{shortish}.\kill
\optname{longest}, \> \optname{longer}, \> \optname{long}, \>
\optname{longish}, \\
\optname{normallength},\\
\optname{shortest}, \> \optname{shorter}, \> \optname{short}, \>
\optname{shortish}.
\end{tabbing}
Make sure you only use one of the above options at a time.  The
\optname{touch} options can be used with any of the main options; they
are often exactly what's needed what used alone: \optname{touchwider}
and \optname{touchlonger} have often reduced my document's
page count to what I wanted.

The options:
\begin{tabbing}
\optname{touchlonger} \= \optname{touchshorter} \\
\optname{t@uchlonger} \> \optname{t@uchshorter}
\end{tabbing}
give a width one third of the way towards the next main option.  The
\optname{t@uch} options can only be used in a class or package file.


The height you get is (where $Z$ is an integer; the body text is $Z+1$
lines long):
\begin{displaymath}
Z \times \mbox{\comname{baselineskip}} + \mbox{\comname{topskip}} +
\mbox{\comname{headheight}} + \mbox{\comname{headsep}} +
\mbox{\comname{footskip}}
\end{displaymath}

The length of the main text block is set to be a certain fraction of
\comname{paperheight}, so \comname{textheight} will increase  if you
turn headers or footers off.

\subsection{Headers and footers}
\label{use:headfoot}

\rmpage doesn't select a page style to use or not use headers or
footers---you've got to arrange for that to be done separately with a
\comname{pagestyle} command.  It does calculate a page layout that does
or does not allow space for a header or a footer.  If you turn
footers off and forget to choose a footer-free page style, the result
is mildly comical.

You can allow space (or not) for headers and footers using these
options:
\begin{tabbing}
\optname{headers} \= \optname{noheaders} \\
\optname{footers} \> \optname{nofooters}
\end{tabbing}
Telling \rmpage not to allow space for either headers or footers will increase
\comname{textheight}, and vice-versa.  See section~\ref{use:length}
for more about this.

\LaTeX's standard classes allow a box 12\units{pt} high for headers.
This is too small for point sizes greater than 12\units{pt}, so
\rmpage changes the size of the box containing the header to be
\comname{baselineskip}.  If you want to use a header which is a
different height to that, define the command \comname{RMheadheight} to
be whatever the height is before calling \rmpage.  For example, if
your header is to be 32\units{pt} high, do this:
\begin{verbatim}
\providecommand{\RMheadheight}{32pt}
\usepackage{rmpage}
\end{verbatim}

If you don't usually use headers, I suggest that you edit the
configuration file so, by default, \rmpage calculates a page layout
that doesn't allow space for them.  See chapter~\ref{chap:config} on
page~\pageref{chap:config} for how to do this.

These options let you change the space between the header and the body
text---use only one of these at a time:
\begin{tabbing}
\optname{lessishheadsep} \=   \optname{moreheadsep} \=
\optname{moreishheadsep} \kill
\optname{mostheadsep} \>   \optname{moreheadsep} \>
\optname{moreishheadsep} \\
\optname{normalheadsep} \\
\optname{lessishheadsep} \> \optname{lessheadsep} \> \optname{leastheadsep}
\end{tabbing}

These options let you change the space between the footer and the body
text---use only one of these at a time:
\begin{tabbing}
\optname{lessishfootskip} \=   \optname{morefootskip} \=
\optname{moreishfootskip} \kill
\optname{mostfootskip} \>   \optname{morefootskip} \>
\optname{moreishfootskip} \\
\optname{normalfootskip} \\
\optname{lessishfootskip} \> \optname{lessfootskip} \> \optname{leastfootskip}
\end{tabbing}
The \optname{footskip} options scale the gap between the \emph{top} of the
footer and the bottom of the body text---the calculation assumes that
the footer is one line high.  The standard \LaTeX\ parameter
\comname{footskip} is the distance from the bottom of the body text
to the bottom of the footer.

Both the option sets above have corresponding \optname{touch}
options---these can be used with any of the main options above:
\begin{tabbing}
\optname{touchmorefootskip} \= \optname{touchlessfootskip} \\
\optname{t@uchmorefootskip} \> \optname{t@uchlessfootskip}\\
\optname{touchmoreheadsep}  \> \optname{touchlessheadsep}\\
\optname{t@uchmoreheadsep}  \> \optname{t@uchlessheadsep}
\end{tabbing}
These options increase or decrease the corresponding parameter one
third of the way towards the value given by the next main option.


\subsection{Position of the main text block}

\subsubsection{Vertical position}
\label{use:verticalpos}

You can raise and lower the position of the main text block on the page
using the altitude set of options:
\begin{tabbing}
\optname{highest} \= \optname{higher} \= \optname{lower} \= \optname{highish} \kill
\optname{highest} \> \optname{higher} \>  \optname{high} \> \optname{highish}\\
\optname{normalaltitude}\\
\optname{lowish}  \> \optname{low} \>  \optname{lower} \>  \optname{lowest}
\end{tabbing}
Be sure you only use one of the main options above at a time.  The
\optname{touch} options below can be used with any of the main options.

The options:
\begin{tabbing}
\optname{touchhigher} \= \optname{touchlower} \\
\optname{t@uchhigher} \> \optname{t@uchlower}
\end{tabbing}
give a width one third of the way towards the next main option.  The
\optname{t@uch} options can only be used in a class or package file.

The text books say that the white space at the bottom of a page should
be larger than the white space at the top.  \LaTeX's and \rmpage's
standard setting splits the space evenly between top and bottom; this
results in an apparently larger space at the bottom---the printing
appears to finish at the bottom of the body text, because the footer
is usually just a page number.

The altitude options consider the top margin to be the space above the
top of the header box, and the bottom margin to be the space below the
footer baseline.  They work by changing the ratio between these two
spaces; the sum of the top and bottom margins is not changed.

But if you ask for a page layout which would result in text exceeding
the various vertical limits, \rmpage will increase the top or bottom
margin as appropriate without attempting to retain a fixed ratio
between them.  For example, if the layout would extend 2\units{mm} off
the top of the printable area, the top margin would be increased by
2\units{mm} and the bottom margin would remain the same.

This is different to the horizontal position options, which do ensure
a fixed ratio between the inside and outside margins; if the inside
margin is reduced, the outside margin is reduced to retain the
requested proportions.

These two ways of doing things were deliberate design decisions; if
anyone thinks I've got it wrong, please email me and try to persuade
me that you're right.

\subsubsection{Horizontal position}
\label{use:horizontalpos}

The horizontal positioning of the main text block is controlled by
three types of options which: vary the ratio between the larger and
smaller margins, switch the larger margin from the inside to the
outside, and force both margins to be the same size or not.

The options:
\begin{tabbing}
\optname{notcentre} or \optname{notcenter} \= \kill
\optname{centre} or \optname{center} \> Equal inside and outside
margins  \\
\optname{notcentre} or \optname{notcenter} \>  Usually unequal inside
and outside margins.
\end{tabbing}
control whether or not the text will be centred horizontally.  Using
the \optname{centre} option forces the inside and outside margins to be
the same; the \optname{notcentre} option means they can be different.
Be sure you only use one of these options at a time.

The options:
\begin{tabbing}
\optname{notstdmargins} \= \kill
\optname{stdmargins} \> Outside margin larger \\
\optname{notstdmargins} \> Inside margin larger
\end{tabbing}
control which of the two margins on a page will be the larger when you
have requested a \optname{notcentre}d layout.  \LaTeX's convention,
and standard typographical convention, has the larger of the two
margins on the outside.  This is given by \optname{stdmargins} (an
abbreviation for standard margins).  The \optname{notstdmargins}
option gives the opposite effect---it is not standard practice to have
the  inside margin larger than the outside margin, although it is
useful when you're producing material for a ring binder.
Be sure you only use one of these options at a time.



You can shift the main text block from right to left using the
\optname{offset} set of options:
\begin{tabbing}
\optname{lessishoffset} \=  \optname{moreoffset} \= \optname{moreishoffset}  \kill
\optname{mostoffset} \>   \optname{moreoffset} \> \optname{moreishoffset}\\
\optname{normaloffset}\\
\optname{lessishoffset} \>  \optname{lessoffset} \> \optname{leastoffset}\\
\end{tabbing}
These options change the difference between the inside and outside
margin: \optname{leastoffset} gives you centred printing, with equal
inside and outside margins.  \optname{mostoffset} produces inside and
outside margins in the proportions $87\% : 13\%$.

These options produce a particular ratio between the inside and
outside margins.  If you have asked for a very wide
\comname{textwidth} which is limited by the non-printing margin of
your printer, \comname{textwidth} might be reduced to produce margins
in the requested proportions.  If you have symmetrical left and right
non-printing margins on your printer, you can only completely fill the
available width if you request centred printing with the
\optname{centre} or \optname{leastoffset} options.

Be sure you only use one of the main options above at a time.  The
\optname{touch} options below can be used with any of the main options.

The options:
\begin{tabbing}
\optname{touchmoreoffset} \= \optname{touchlessoffset} \\
\optname{t@uchmoreoffset} \> \optname{t@uchlessoffset}
\end{tabbing}
give an offset one third of the way towards the next main option.  The
\optname{t@uch} options can only be used in a class or package file.

An oddity to watch out for is this: \optname{leastoffset} produces the
least offset between the left and right margins, namely none.  If you
use the \optname{leastoffset} and the \optname{touchlessoffset}
options together, you get what I call a negative offset.  That is, if
you've asked for \optname{stdmargins}, you'll end up with a larger
inside margin than outside, and if you've asked for
\optname{notstdmargins}, you get a larger outside margin than inside.
\rmpage will draw your attention to this is it happens.

\subsection{Marginal paragraphs}
\label{use:marginalparapgraphs}

Marginal paragraphs begin a distance \comname{marginparsep} away from
the side of the main text block, and extend to a distance
\comname{RM@mparclearance} from the edge of the paper.  The maximum
width of a marginal paragraph is given by \comname{RM@maxmparwidth}.

With a conventional layout on paper similar to A4, marginal paragraphs
usually fill the space from \comname{marginparsep} (about 4\units{mm})
away from the main text block, to \comname{RM@mparclearance} (about
10\units{mm}) in from the edge of the paper.
\comname{RM@maxmparwidth} (about 50\units{mm}) is not usually a limit.

All of these parameters can be controlled by options.
\comname{marginparsep} is a standard \LaTeX\ length which is initially
set by the class file; the other two are \rmpage commands, and are
initially set to a fraction of \comname{paperwidth}.

The options:
\begin{tabbing}
\optname{leastmparsep} \= \optname{moremparsep} \=
\optname{moreishmparsep} \kill
\optname{mostmparsep} \> \optname{moremparsep} \>
\optname{moreishmparsep} \\
\optname{normalmparsep} \\
\optname{leastmparsep} \> \optname{lessmparsep} \>
\optname{lessishmparsep}
\end{tabbing}
control the size of the gap between the marginal paragraph and the
main text block.

Be sure you only use one of the main options above at a time.  The
\optname{touch} options below can be used with any of the main options.

The options:
\begin{tabbing}
\optname{touchmoremparsep} \= \optname{touchlessmparsep} \\
\optname{t@uchmoremparsep} \> \optname{t@uchlessmparsep}
\end{tabbing}
give an offset one third of the way towards the next main option.  The
\optname{t@uch} options can only be used in a class or package file.

The options:
\begin{tabbing}
\optname{lessmparclearance} \= \optname{lessishmparclearance} \=
\optname{moreishmparclearance}\kill
\optname{mostmparclearance} \> \optname{moremparclearance} \>
\optname{moreishmparclearance}\\
\optname{normalmparclearance} \\
\optname{lessmparclearance} \> \optname{lessishmparclearance} \>
\optname{leastmparclearance}
\end{tabbing}
control the size of the gap between the marginal paragraph and the
edge of the paper.

Be sure you only use one of the main options above at a time.  The
\optname{touch} options below can be used with any of the main options.

The options:
\begin{tabbing}
\optname{touchmoremparclearance} \= \optname{touchlessmparclearance} \\
\optname{t@uchmoremparclearance} \> \optname{t@uchlessmparclearance}
\end{tabbing}
give an offset one third of the way towards the next main option.  The
\optname{t@uch} options can only be used in a class or package file.

The options:
\begin{tabbing}
\optname{lessmaxmparwidth} \= \optname{lessishmaxmparwidth} \=
\optname{moreishmaxmparwidth}\kill
\optname{mostmaxmparwidth} \> \optname{moremaxmparwidth} \>
\optname{moreishmaxmparwidth}\\
\optname{normalmaxmparwidth} \\
\optname{lessmaxmparwidth} \> \optname{lessishmaxmparwidth} \>
\optname{leastmaxmparwidth}
\end{tabbing}
control the maximum size of marginal paragraphs; marginal paragraphs
usually stop \comname{RM@mparclearance} away from the edge of the
paper, but there is the additional limit that marginal paragraphs
cannot be larger than {RM@maxmparwidth}.

Be sure you only use one of the main options above at a time.  The
\optname{touch} options below can be used with any of the main options.

The options:
\begin{tabbing}
\optname{touchmoremaxmparwidth} \= \optname{touchlessmaxmparwidth} \\
\optname{t@uchmoremaxmparwidth} \> \optname{t@uchlessmaxmparwidth}
\end{tabbing}
give an offset one third of the way towards the next main option.  The
\optname{t@uch} options can only be used in a class or package file.

If you want an unusual layout with a very much larger than usual
maximum marginal paragraph width, or a very much larger than usual gap
between the marginal paragraph and the edge of the paper, the
\optname{largebase} options will double the initial size of these
parameters.  This means that all the sizes produced by the
corresponding options above are doubled.  The \optname{normalbase}
options give the default size.

The additional options:

\noindent\begin{tabularx}{\textwidth}{lX}
&\\
\optname{normalbasemaxmparwidth} & Normal maximum size marginal
paragraphs \\
\optname{largebasemaxmparwidth} & Double sized maximum size marginal
paragraphs \\
\optname{normalbasemparclear} & Normal gap between the edge of the paper
and the end of marginal paragraphs. \\
\optname{largebasemparclear}& Double sized gap between the edge of the paper
and the end of marginal paragraphs. \\
&\\
\end{tabularx}
are intended to be used only when the range of sizes
given by the conventional options aren't enough; \optname{more} and
\optname{touchmore} combine to double the size of any of the marginal
paragraph parameters (the precise factor is 2.0394).

\subsection{Multiple columns}
\label{use:multiplecolumns}

There are two things to consider with a multiple column layout: the
text columns themselves, and the gap in between.

\subsubsection{The width of text columns}

If you are producing a layout with more than one column, \rmpage
usually needs to know how many columns, because it takes into account
the width of each text column, measured against the width of the
average character.

If you're using \LaTeX's standard \optname{onecolumn} or
\optname{twocolumn} options, \rmpage takes note:
\begin{tabbing}
\optname{onecolumn} \= Default. Produce a layout assuming one text column \\
\optname{twocolumn} \>  Produce a layout assuming two text columns
\end{tabbing}
You must pass these options to the class file by placing them in the
optional argument to the \comname{documentclass} command, or your text
will not be set in the number of columns you expect.

You might be producing a multiple column layout using the
\packname{multicol} package.  If so, you should use different options
passed to \rmpage to tell it how many text columns your document will
be set in.   The following options---\optname{onecolumnwidth} is the
default---tell \rmpage to calculate a layout
assuming the text body will be set in the named number of text columns:
\par
\vspace{\baselineskip}
\begin{raggedright}
\optname{tencolumnwidth}
\optname{ninecolumnwidth}
\optname{eightcolumnwidth}
\optname{sevencolumnwidth}
\optname{sixcolumnwidth}
\optname{fivecolumnwidth}
\optname{fourcolumnwidth}
\optname{threecolumnwidth}
\optname{twocolumnwidth}
\optname{onecolumnwidth}
\end{raggedright}
\vspace{\baselineskip}

% \begin{tabbing}
% \optname{sevencolumnwidth} \= \kill
% \optname{tencolumnwidth} \> Produce a layout assuming ten text columns\\
% \optname{ninecolumnwidth} \> Produce a layout assuming nine text
% columns\\
% \optname{eightcolumnwidth} \> Produce a layout assuming eight text
% columns\\
% \optname{sevencolumnwidth} \> Produce a layout assuming seven text
% columns\\
% \optname{sixcolumnwidth} \> Produce a layout assuming six text columns\\
% \optname{fivecolumnwidth} \> Produce a layout assuming five text columns\\
% \optname{fourcolumnwidth} \> Produce a layout assuming four text
% columns\\
% \optname{threecolumnwidth} \>Produce a layout assuming three text columns \\
% \optname{twocolumnwidth} \> Produce a layout assuming two text columns\\
% \optname{onecolumnwidth} \> Default. Produce a layout assuming one text column
% \end{tabbing}
Make sure you only use one of the twelve options above at a time.

If you are producing a document with different numbers of columns in
different places, try starting out by telling \rmpage that you are
using the smallest number of columns in your document.  For example,
if your document has three columns in some places and four columns in
others, pass the \optname{threecolumnwidth} option to \rmpage.  If the
width needs changing after that, begin by trying the \optname{width}
options in section~\ref{use:width} on page~\pageref{use:width}.

\subsubsection{The space between columns}

The main options:
\begin{tabbing}
\optname{lessishcolsep} \=   \optname{morecolsep} \=
\optname{moreishcolsep} \kill
\optname{mostcolsep} \>   \optname{morecolsep} \>
\optname{moreishcolsep} \\
\optname{normalcolsep} \\
\optname{lessishcolsep} \> \optname{lesscolsep} \> \optname{leastcolsep}
\end{tabbing}
increase or decrease the separation between the columns---they scale
the standard \LaTeX\ \comname{columnsep} parameter.  The default
\optname{normalcolsep} option does nothing---you get \LaTeX's standard
column separation.  Make sure you only use one of these main options
above at a time.

The \optname{touch} options below are meant to be used with any of the
main options, and increase or decrease the gap between the columns one
third of the way towards the next main option.
\begin{tabbing}
\optname{touchmorecolsep} \= \optname{touchlesscolsep} \\
\optname{t@uchmorecolsep} \> \optname{t@uchlesscolsep}\\
\end{tabbing}

If you don't like \LaTeX's standard \comname{columnsep}, the options
below calculate a different default value:

\noindent
\begin{tabularx}{\textwidth}{lX}
&\\
\optname{adaptivecolumnsep}   & Calculates a normal \comname{columnsep}
which is 2.3 times the average character width of the selected
fount---a 0.1\units{pt} increase for 10\units{pt} Computer Modern
Roman; quite a bit different for other founts.\\
\optname{noadaptivecolumnsep} & Default: gives you the standard
\comname{columnsep}, which can be changed by any of the
\optname{colsep} options above\\
&\\
\end{tabularx}

The \comname{columnsep} produced by the \optname{adaptivecolumnsep}
option can be scaled by any of the \optname{colsep} options.


\subsection{Paper size}
\label{use:papersize}

\rmpage knows about three types of paper size options: main size, long
size, and orientation.

\subsubsection{Orientation---landscape or portrait}
\label{use:orientation}

There are two options to select the paper orientation:
\begin{tabbing}
\optname{landscape} \= Ensures that the longest side is horizontal\\
\optname{portrait} \> Default. Ensures that the shortest side is
horizontal
\end{tabbing}
Make sure you only use one of these at a time.

\subsubsection{Long sizes}
\label{use:longpapersizes}

A long paper size is based on a larger paper size; it formed by
cutting off the specified fraction of a parent paper size, divided
along the longer edge.  For example, the common long size 2/3~A4 is
$210\units{mm} \times 2/3 297\units{mm} = 210\units{mm} \times
198\units{mm}$.

These sizes are only formally defined for ISO A and B sizes.  \rmpage
will make any main paper size into a long size with one of these
options:
\begin{tabbing}
notlongpaper \= Default. Does nothing.\\
7/8longpaper \> Multiply the parent paper size length by 7/8 \\
3/4longpaper \> Multiply the parent paper size length by 3/4 \\
2/3longpaper \> Multiply the parent paper size length by 2/3 \\
5/8longpaper \> Multiply the parent paper size length by 5/8 \\
1/2longpaper \> Multiply the parent paper size length by 1/2 \\
3/8longpaper \> Multiply the parent paper size length by 3/8 \\
1/3longpaper \> Multiply the parent paper size length by 1/3 \\
1/4longpaper \>  Multiply the parent paper size length by 1/4 \\
1/8longpaper \> Multiply the parent paper size length by 1/8
\end{tabbing}
Make sure you only use one of these options at a time.  They must be
used with a main paper size option; the \optname{letterpaper} main
paper size option is used by default.

The resulting paper size is made \optname{portrait} by default, or
\optname{landscape} if you've used that option, and printing limits
are calculated based on the assumption that you will be printing on
the parent paper size.  That is, \rmpage assumes that if you've asked
for 2/3 long A4, you'll be printing on the top 2/3 of a sheet of A4,
not a cut sheet of 2/3~A4.  Or that if you've asked for 1/4 long A4,
you'll be printing 4 pages on one sheet of A4.

See the chapters~\ref{chap:htw} and~\ref{chap:config} to find out how
and why this is done, and how to change these assumptions.

\subsubsection{Main sizes}
\label{use:mainpapersizes}

There's a lot more paper sizes available now; have a look at
sections~\ref{optrmp:paper-sizes} and~\ref{optcfg:papersizes} for the
full list.  Any of the main paper sizes can be turned into a long paper
size (see the section on long paper sizes above), and any paper size
can be made landscape or portrait.

The available paper sizes include:
\begin{tabbing}
\optname{dlpaper} and \optname{c7/6paper} \= \kill
\optname{letterpaper}, \optname{executivepaper}, \\
and \optname{legalpaper}. \> US paper sizes \\
\optname{a0paper} to \optname{a10paper} \> ISO stationery sizes \\
\optname{b0paper} to \optname{b10paper} \> ISO poster sizes \\
\optname{c0paper} to \optname{c7paper} \> ISO envelopes \\
\optname{dlpaper} and \optname{c7/6paper} \> ISO envelopes \\
\optname{no10envelopepaper} \> US envelopes \\
\optname{foolscapefoliopaper} \> obsolete stationery \\
\end{tabbing}
There's over 70 sizes in all---to find out how paper sizes are
declared to \rmpage, and how to add new one, see chapters~\ref{chap:htw}
and~\ref{chap:config}.

\subsection{Founts}
\label{use:founts}

The standard \LaTeX\ classes calculate a \comname{textwidth} on the
assumption that you will be using  Computer Modern Roman  as the main
body text fount.  But the legibility of a line of text depends in
part on the the width of a line measured in characters, and
different founts have a different average character widths, so it's
sensible to calculate a different \comname{textwidth} if you're using
a different main body text fount.

\rmpage will calculate an appropriate \comname{textwidth} if you use
one of the options below to tell it what you are using as your main
body text fount.
\begin{tabbing}
\optname{lucasualwidth}  \= \kill
\optname{avantwidth}     \> \packname{PSFNSS} Adobe Avant Garde. \\
\optname{bookmanwidth}   \> \packname{PSFNSS} Adobe Bookman. \\
\optname{chancerywidth}  \> \packname{PSFNSS} Adobe Zapf Chancery.\\
\optname{cmrwidth}       \> Default.  Computer Modern Roman. \\
\optname{concretewidth}  \> Donald Knuth's Concrete Roman. \\
\optname{courierwidth}   \> \packname{PSFNSS} Adobe Courier.  \\
\optname{helvetwidth}    \> \packname{PSFNSS} Adobe Helvetica.  \\
\optname{lucasualwidth}  \> \packname{bh} Lucida casual.  \\
\optname{newcentwidth}   \> \packname{PSFNSS} Adobe New Century Schoolbook. \\
\optname{palatinowidth}  \> \packname{PSFNSS} Adobe Palatino.  \\
\optname{timeswidth}     \> \packname{PSFNSS} Adobe Times.  \\
\optname{utopiawidth}    \> \packname{PSFNSS} Adobe Utopia.  \\
\optname{thisfountwidth} \> Bases \comname{textwidth} on the
currently selected fount.
\end{tabbing}
The \optname{thisfountwidth} is useful if you are using a fount not
covered by the standard options: it works by measuring the average
character width of the fount that was selected when \rmpage was
loaded.  For this to work properly, you must ensure that the main body
text fount has been selected before loading \rmpage.  For example, if
you are loading Adobe Baskerville in your preamble, you could ask
\rmpage to set an appropriate \comname{textwidth} like this:
\begin{verbatim}
\documentclass[thisfountwidth]{article}
\renewcommand{\rmdefault}{pgm}
\rmfamily
\usepackage{rmpage}
\begin{document}
...
\end{verbatim}
If you haven't told \rmpage to shut up with the \optname{yorkshire}
option, it will tell you which fount it's using as the basis for
\comname{textwidth}---this is useful for people like me who get
horribly confused by the details of fount selection.

If you want to use one of the \packname{PSNFSS} packages to load a
fount as well as set a \comname{textwidth} based on this fount, you
can use one of these options:
\begin{tabbing}
\optname{loadchancery}  \= \kill
\optname{loadavant}    \> Requires the \packname{avant} package. \\
\optname{loadbookman}  \> Requires the \packname{bookman} package.\\
\optname{loadchancery} \> Requires the \packname{chancery} package.\\
\optname{loadhelvet}   \> Requires the \packname{helvet} package.\\
\optname{loadnewcent}  \> Requires the \packname{newcent} package.\\
\optname{loadpalatino} \> Requires the \packname{palatino} package.\\
\optname{loadtimes}    \> Requires the \packname{times} package.\\
\optname{loadutopia}   \> Requires the \packname{utopia} package.
\end{tabbing}
Each of these \optname{loadfount} options does three things:
\begin{enumerate}
\item Loads the named package
\item Calculates a \comname{textwidth} based on the named fount
\item Sets the typesetting parameters to looser values.
\end{enumerate}

The typesetting parameters are only loosened a little.  The change
does not affect the \LaTeX{} commands \comname{fussy} and
\comname{sloppy}, so using the \comname{onecolumn} and
\comname{twocolumn} commands will over-ride this change.  You can
duplicate the effect of this loosening with the \comname{sloppiness}
command---see section~\ref{use:tightness} on
page~\pageref{use:tightness} for more details.

If you've asked for a \optname{twocolumn} layout, you get typesetting
parameters close to the standard \LaTeX\ sloppy values, unless you
over-ride this looseness.  Because the \packname{multicol} package
makes its own arrangements, you don't get the sloppy values if you
asked for \optname{twocolumnwidth} to \optname{tencolumnwidth};
\rmpage sets the typesetting parameters as if you were using a one
column layout.  Please email me if you have any thoughts on this
matter.

The three \optname{loadfount} options below are a little different to
the \packname{PSNFSS} fount loading options above:

\noindent
\begin{tabularx}{\textwidth}{lX}
%
&\\
%
\optname{loadconcrete} & Requires the \packname{beton} package;
calculates \comname{textheight} based on \packname{beton}'s modified
\comname{baselineskip}; sets \comname{textwidth} for Concrete Roman;
doesn't loosen typesetting.  See section~\ref{use:beton} on
page~\pageref{use:beton} for more on \packname{beton}.\\
%
\optname{loadcourier} & Makes the default roman fount Courier, sets an
appropriate \comname{textwidth}, and asks for \optname{loose}
typesetting.  I think this is ugly and crude: you might be better off
using the \packname{times} package and \comname{ttfamily}\\
%
\optname{loadlucasual} &  Requires the \packname{lucasual} package;
sets loose typesetting and a \comname{textwidth} to match Lucida
Casual.  Needs the \packname{lucasual} files;
available from CTAN at \filename{fonts/psfonts/bh/lucasual/}. \\
%
&\\
\end{tabularx}


\subsection{Printers}
\label{use:printers}

It can be useful to let \rmpage know about your intended output
device, because it can ensure that the layout it produces will fit
inside the printable region of the paper.  At the moment, there
aren't very many printer options that match real printers.  As I get
more information, I shall add more real printer options.

You can add an option for your own printer, or change the way your
\rmpage installation set non-printing margins for your printer---the
details are described in section~\ref{cfg:printertype} on
page~\pageref{cfg:printertype}.

The available printer options that I'l admit to here are:
\begin{tabbing}
\optname{pessimisticprinter} \= \kill
\optname{fullbleedprinter} \> Prints right to the edge of the paper\\
\optname{generalprinter} \> This should be fine for anyone\\
\optname{optimisticprinter} \> \\
\optname{pessimisticprinter} \> Uses the largest non-printing margins
I've found\\
\optname{dw500printer} \> Any HP 500 series inkjet\\
\optname{dw600printer} \>  Any HP 600 series inkjet\\
\end{tabbing}

\subsection{Date format}
\label{use:dateformat}

You can change the way the \comname{today} command prints the date
with the options below:
\begin{tabbing}
\optname{ukdate} \= \optname{othernicedate} \= \kill
\optname{ukdate} \> \optname{nicedate} \> 5th November 1693\\
\optname{usdate} \> \optname{othernicedate} \> Default: July 4, 1776
\end{tabbing}
Only the \optname{ukdate} and \optname{nicedate} options make any
changes: the \optname{usdate} and \optname{othernicedate} options do
nothing.

The options names happened like this: I once wrote a \LaTeX\ style
file called \packname{nicedate}, which produced the same effect as
\rmpage's \optname{nicedate} option---I like dates printed like that,
you see.  When I included the \packname{nicedate} code in \rmpage, it
made sense to add a complementary option; hence \optname{othernicedate}.

\optname{othernicedate} seems preferable to \optname{nastydate}, but
it's not terribly memorable, so I created the synonyms
\optname{ukdate} and \optname{usdate}.  I'm not keen on these option
names, but I can at least remember them.  If you can think of
something different, please let me know.

\subsection{The \packname{beton} package}
\label{use:beton}

The easy way of using Frank Jensen's \packname{beton} package---to use
Donald Knuth's Concrete Roman founts---with \rmpage is to pass the
\optname{loadconcrete} option to \rmpage.  This will load the
\packname{beton} package, and set vertical and horizontal layout
parameters for the Concrete Roman founts:
\begin{verbatim}
\documentclass[loadconcrete,concrete-math]{article}
\usepackage{rmpage}
\end{verbatim}
The example above tells \rmpage to load the concrete founts using the
\packname{beton} package.  All global options are passed to all
packages, so \packname{beton} and \rmpage are passed
\optname{loadconcrete} and \optname{concrete-math}.  \rmpage ignores
\optname{concrete-math} and acts on \optname{loadconcrete}; while
\packname{beton} ignores \optname{loadconcrete} and acts on
\optname{concrete-math}.

Because the \packname{beton} package changes \comname{baselineskip},
but the changes don't take effect until the \verb|begin{document}|
command has been executed, and \rmpage needs to know about the value
of \comname{baselineskip} when it's setting \comname{textheight}, the
\packname{beton} package needs special support in \rmpage.
Everything's taken care of if you use the \optname{loadconcrete}
option.  If you want to load \packname{beton} with a
\comname{usepackage} command, you should do this:
\begin{itemize}
\item Load \packname{beton} before \rmpage
\item Pass the \optname{beton} option to \rmpage
\item Pass the \optname{concretewidth} option to \rmpage---see
section~\ref{use:founts} on page~\pageref{use:founts}.
\end{itemize}
Like this, for example:
\begin{verbatim}
\documentclass[beton,concretewidth]{report}
\usepackage{beton}
\usepackage{rmpage}
\begin{document}
...
\end{verbatim}
There's no need to pass the \optname{beton} option to \rmpage if
you're also passing the \optname{stdbaselineskip} option to
\packname{beton}, but it will do no harm.

\rmpage needs code that is in \packname{beton}~v1.3, 5th March 1995,
to get things right.  This version of \packname{beton} was current in
August 1996.  I've made \rmpage check the definition of the
\packname{beton} command it uses, but if you have any doubts that
\rmpage is doing its job properly, you can try this:
\begin{verbatim}
\documentclass[beton,chatty]{article}
\usepackage{beton}
\usepackage{rmpage}
\begin{document}
\typeout{\the\baselineskip\space according to beton}
\end{document}
\end{verbatim}
Look through the console output for the lines that look like this
(the numbers will vary depending on paper size etc):
\begin{verbatim}
\textheight is:
   48 x 13.0pt + 10.0pt = 634.0pt
\end{verbatim}
This is a report of the number of lines in the text body ($48+1=49$ in
this case), and how the final \comname{textheight} is arrived at.  The
second number---13\units{pt} in this case---is \comname{baselineskip}
as seen by \rmpage.  If this number is the same as the
\comname{baselineskip} according to \packname{beton}---as reported on
the console---everything's probably okay.  (The last term in the sum
above is \comname{topskip}).

\subsection{Typesetting parameters}
\label{use:tightness}

\LaTeX's \comname{sloppy} command tells \TeX\ to be less fussy about
linebreaking.  It's sometimes useful to be able to tell \TeX\ to be
less fussy than normal, but not as sloppy as \comname{sloppy}.  A good
example is typesetting with founts installed by Alan Jeffries's
\packname{fontInst} package----for example, the founts in the
\packname{PSNFSS} bundle of packages.  Because these founts have a
tighter inter-word space to Computer Modern Roman, Alan Jeffries
recommends slightly looser typesetting parameters to usual, but not as
loose as \comname{sloppy}---the values were reported by Sebastian
Rahtz in his `Notes on setup of PostScript fonts for \LaTeX2', 14th
August 1994.  Oh dear: but Rahtz reports in `\filename{PSNFSS2e.tex}'
(05/11/96) that \packname{fontinst} now produces founts with looser
inter-word spacing, and suggests that the extra looseness is no longer
required.  I reckon a little extra looseness helps (the founts do seem
to be tighter than the computer modern family), so the
\optname{loadPSfount} options have been changed to use the
\optname{looseish} settings normally, or the \optname{sloppyish}
settings if you've asked for two columns.

(What \packname{fontInst} mainly does is make \filename{vf},
\filename{tfm}, \filename{fd}, and \filename{sty} files from
\filename{afm} files, so you can use with \LaTeX\ founts which haven't
been created with {\sffamily\scshape Metafont}.  It's available from
CTAN.)

\rmpage has an option that selects similar typesetting parameters, and
variants: two fussier and two sloppier.  Each option can in effect be
selected at any point in your document, using the \comname{sloppiness}
command as shown.
\begin{tabbing}
\optname{looseish} \= \kill
\optname{tight} \> Default.  Standard \LaTeX\ \comname{fussy} settings. |\sloppiness{0}|\\
\optname{looseish} \> |\sloppiness{1}| \\
\optname{loose} \> Similar to Jeffries's suggestion |\sloppiness{2}| \\
\optname{looser} \> |\sloppiness{3}|\\
\optname{loosest} \> |\sloppiness{4}|\\
\optname{sloppyish} \> For two columns |\sloppiness{5}|\\
\end{tabbing}
The default changes to \optname{loose} if you have use a
\optname{loadfount} option which loads a \packname{FontInst} fount.
See section~\ref{use:founts} on page~\pageref{use:founts} for more
information.

Note that \sloppiness{0} is \emph{not} equivalent to \fussy; because
standard \LaTeX's \comname{sloppy} and \comname{fussy} commands
change fewer parameters to \rmpage's \comname{sloppiness} commands,
the \comname{fussy} command won't give you fussy typsetting if used
after one of \rmpage's typesetting commands or options.


\chapter{Old Using \rmpage}





\section{Some options}

\rmpage is more useful when you start using options: there's quite a
lot of them, grouped in fairly consistently named sets.  For example,
you can select \rmpage's normal \comname{textwidth} by saying
\optname{normalwidth}.  If you want a slightly larger
\comname{textwidth}, use the \optname{widish} option instead.  The
\optname{wide}, \optname{wider}, and \optname{widest} options will
give you a progressively wider \comname{textwidth}.  On the other
hand, you can use the \optname{narrowish} option to get a slightly
narrower \comname{textwidth} to standard, and the
\optname{narrow},\optname{narrower}, and \optname{narrowest} options
do exactly what you might expect.

When you are dealing with a set of options like the width options
above (all 13 of them; there's four odd ones I've not mentioned), be
sure you only use one at a time.  \rmpage doesn't check, and you can
get unexpected results if there's more than one option used from each
set.

\begin{description}\shorty

\item[Text width]
%
\optname{narrowest}, \optname{narrower},
\optname{narrow}, \optname{narrowish}, \optname{normalwidth},
\optname{wideish}, \optname{wide}, \optname{wider}, and
\optname{widest}
%
 select a progressively larger \comname{textwidth}.
%
\optname{fullwidth} gives you the widest \comname{textwidth} that'll
fit inside the printing region; \optname{stdwidth} selects the same
width as you would get with the standard classes;
\optname{oneinchmargins} selects a \comname{textwidth} that gives you
an average margin size of one inch---the inside and outside margins
add up to two inches; \optname{halfinchmargins} is similar, but the
inside and outside margins add up to one inch rather than two.

\rmpage looks out for the \optname{onecolumn} and \optname{twocolumn}
options, and calculates a possibly larger \comname{textwidth} if you
say \optname{twocolumn}. The class file is responsible for telling
\TeX\ to set text in two columns, so these options shouldn't be passed
to \rmpage only.

If you're using the \packname{multicol} package, you can pass to
\rmpage the options: \optname{twocolumnwidth},
\optname{threecolumnwidth}, \optname{fourcolumnwidth}, and so on up to
\optname{tencolumnwidth}.  This will give you a \comname{textwidth}
based on that number of columns.

\item[Text height]
%
\optname{shortest}, \optname{shorter}, \optname{short},
\optname{shortish}, \optname{normallength}, \optname{longish},
\optname{long}, \optname{longer}, \optname{longest} select a
progressively longer \comname{textheight}.
%
\optname{fulllength} gives you the longest \comname{textheight} that
will fit inside the printing region; \optname{stdlength} gives you
the \comname{textheight} you'd get with the standard class.

\item[Headers and footers]
%
The \optname{headers} and \optname{footers} option leave space for
headers and footers respectively; \optname{noheaders} and
\optname{nofooters} give you page layout designed for no headers or
no footers, respectively.

\optname{leastheadsep}, \optname{lessheadsep},
\optname{lessishheadsep}, \optname{normalheadsep},
\optname{moreishheadsep}, \optname{moreheadsep}, and
\optname{mostheadsep} enlarge and shrink the gap between the top of
the text body and the bottom of the header; they work by scaling the
standard value---\optname{normalheadsep} does nothing.

\optname{leastfootskip}, \optname{lessfootskip},
\optname{lessishfootskip}, \optname{normalfootskip},
\optname{moreishfootskip}, \optname{morefootskip}, and
\optname{mostfootskip} enlarge and shrink the gap between the bottom of
the text body and the bottom of the footer; they work by scaling the
standard value---\optname{normalfootskip} does nothing.

\item[Positioning the text body horizontally]
%
\rmpage looks out for the standard \LaTeX\ class options:
\optname{twoside} and \optname{oneside}.  \rmpage will produce a
layout either giving you a text body intended for printing on one side
of the paper, or alternating with odd pages on a right-hand page, and
even pages on a left-hand page.

The \optname{centre} (or \optname{center}) option places the text body
on the page with equal margins to the left and right.  The
\optname{notcentre} (or \optname{notcenter}) option places the text
body with possibly unequal margins to the left and right.

If you've asked for \optname{notcentre}, you will find that the larger
margin is on the inside (intended for ring-binding).  This effect is
produced with the \optname{notstdmargins} option.  The
\optname{stdmargins} options reverses this, so the larger margin is on
the outside (just like the standard classes).

\optname{leastoffset}, \optname{lessoffset},
\optname{lessishoffset}, \optname{normaloffset},
\optname{moreishoffset}, \optname{moreoffset}, and
\optname{mostoffset} enlarge and shrink the difference between the
larger and smaller margin.  \optname{leastoffset} gives you
\optname{centre}d printing; there are subtle differences between
\optname{leastoffset} and \optname{centre}.

\item[Positioning the text body vertically]
%
\optname{lowest}, \optname{lower}, \optname{low}, \optname{lowish},
\optname{normalaltitude}, \optname{highish}, \optname{high},
\optname{higher}, and \optname{highest}, shift the text body up and
down the page---they are referred to as the altitude option set in
this document (I know, but do you have any better ideas?)

\item[Paper and printers]
%
\optname{landscape} and \optname{portrait} force that orientation,
whatever the size paper.

There's lots of new paper sizes: see section~\ref{gen:papersizes} on
page~\pageref{gen:papersizes}.  You can ask for
\optname{7/8longpaper}, \optname{3/4longpaper},
\optname{5/8longpaper}, and so on down to \optname{1/8longpaper}.
This calculates what's called a long paper size, based on the main
paper size selected, by dividing the main paper size into the
specified fraction along the long edge. For example, A4 is
$210\units{mm} \times 297\units{mm}$; 1/3 long A4 is $210\units{mm}
\times 99\units{mm}$.

\optname{lj4printer} tells \rmpage you're using a Hewlett-Packard
LaserJet 4 printer, and it'll calculate your page layout using
appropriate non-printing margins.  There are similar options for
several other printers; you can add your own printer if it's not
already here, and set any to be your default.  For more details, see
section~\ref{optcfg:printers} on page~\pageref{optcfg:printers}.

\end{description}

\section{Some other options}


Several option sets have corresponding \optname{touch} options.  These
options increase or decrease whatever the parameter is by an amount in
between the current main option and the next one.  Generally, the
sequence of values is a smooth geometrical one.

For example, passing \optname{wide,} \optname{touchwider} to \rmpage
gives you a \comname{textwidth} a third of the way up from \optname{wide}
to \optname{wider}.  \optname{wider,} \optname{touchnarrower} gives
you a \comname{textwidth} a third of the way down from
\optname{wider} to \optname{wide}.

A touch option can be used with a main option (any option which doesn't
have \optname{touch} or \optname{t@uch} in the name is a main option;
not all main options have \optname{touch} options).  The
\optname{t@uch} options should only be used in class and package
files.  They have exactly the same effect as a \optname{touch}
option, so saying \optname{wider,} \optname{touchnarrrower,}
\optname{t@uchnarrower} gives you the same width as \optname{wider,}
\optname{touchwider}.

\begin{description}\shorty

\item[Text width setting]
%
\optname{paperwidthset} makes \rmpage set the \comname{textwidth} going by
paper-based \comname{textwidth} only; \optname{characterwidthset} goes
by character-based \comname{textwith} only, but still pays attention
to the printing limits specified by the paper size and printer
selected.  \optname{bothwidthset} is the default setting.

\optname{paperwidthset} was created so I could produce a layout for a
time-table, fitting on A4 landscape paper.  Paying attention to the
number of characters in a line is inappropriate for that job, hence
the option.  The other two options are the natural complements.

\end{description}


The \optname{touchlonger}, \optname{touchshorter},
\optname{touchwider}, and \optname{touchnarrower} options increase or
decrease \comname{textheight} or \comname{textwidth} one third of the
way (in a geometrical sequence) towards the next main increment.  That
is, if you've said \optname{wide} and \optname{touchwider}, the width
you get will be one third of the way towards the width given by
\optname{wider}.


\section{Layout details}
\label{gen:layout}

The standard \LaTeX\ package, \packname{layout}, displays all the page
layout parameters and their values.  This document loads the package
and uses the \comname{layout} command to display the values.  You can
see the results of the \comname{layout} command in figure~\ref{fig:layout}.

\begin{figure}[h!t]
\noindent
% dirty trick to fool \layout into producing a diagram for one page
% only, that can float where it needs to.
\begin{parbox}{\textwidth}\makeatletter
 {
  \relax
  \makeatletter
  \if@twoside\relax
   \@twosidefalse
   \typeout{twoside fudge}
   \layout
   \makeatletter
   \@twosidetrue
   \makeatother
  \else
   \typeout{No twoside fudge}
   \layout
  \fi
 }\makeatother
\end{parbox}
{\footnotesize
\vspace*{8\baselineskip}}
\caption{The output of the \comname{layout} command \label{fig:layout}}
\thispagestyle{plain}
\end{figure}

Note that the layout for this document was calculated by \rmpage,
which was told to leave no space for a header with the
\optname{noheaders} option; that is why both \comname{headheight} and
\comname{headsep} have zero size.


\section{Paper sizes}
\label{gen:papersizes}

\rmpage knows about lots of paper sizes, including ISO long sizes.
The ISO standard which defines the A and B series of sizes also
defines long sizes.  For example, A4 paper is defined as $210 \times
297\units{mm}$.  The long size $1/3\units{A4}$ is $210 \times
99\units{mm}$---the long size is the base size divided into the
specifed number of parts along the long edge.  Because you can apply
this division to any piece of paper you like, \rmpage is happy to make
non-ISO sizes into ISO-style long sizes.  This means that Americans
who want to print small booklets with two pages fitting on one sheet
of letter paper can produce pages very easily with \rmpage, by asking
for \optname{letterpaper} and \optname{1/2longpaper}.  You can do the
same thing with A4 paper, but why not just ask for \optname{a5paper}?
One commonly-used long size is 2/3 A4, which is often used for company
invoices and commercial letters; it's often used in landscape
orientation ($210\units{mm} \times 198\units{mm}$) so it fits neatly
into a world designed for A4-wide paper.

\begin{ikkystuff}
If you are going to print two pages on one sheet of US letter paper,
as I suggested, \rmpage standard way of deciding what to set the
non-printing margins to might be inappropriate.  If so, look at
the printer paper setting code in the configuration file, and specify
clearances for the appropriate printer and paper combinations.
Section~\ref{cfg:printerpaper} on page~\pageref{cfg:printerpaper} has
more details on this.

A good reason for not asking for \optname{a4paper} and then
\optname{1/2longpaper}, is that if the configuration file has
particular settings for A5 paper, it won't apply them to this long
paper size, even though it's the same physical size.  A good reason
for doing so is that you might want that.
\end{ikkystuff}

\rmpage accepts these paper options (and more):

\begin{tabbing}
\optname{a0paper} \= to \=  \optname{a10paper}, \\
\optname{b0paper} \> to \>  \optname{b10paper}, \\
\optname{c0paper} \> \optname{c7paper},\= \optname{c7/6paper},\= \optname{dlpaper},
\optname{no10envelopepaper},\\
etc\\
\end{tabbing}

Options not listed here include some non-ISO envelope sizes, old
British book sizes and so on: the odder sizes are kept in the
configuration file.  See section~\ref{optcfg:papersizes} on
page~\pageref{optcfg:papersizes} for more details.

You can get a long size by passing one of these options along with a
paper size option: \optname{notlongpaper}, \optname{7/8longpaper},
\optname{3/4longpaper}, \optname{2/3longpaper},
\optname{5/8longpaper}, \optname{1/2longpaper},
\optname{3/8longpaper}, \optname{1/3longpaper},
\optname{1/4longpaper}, or \optname{1/8longpaper}.

\rmpage makes an honest attempt to work out how you'll be printing
these long sizes out (see section \ref{htw:paperprinter}), but its
decision might apply limits that aren't what you want.  If so, you'll
have to add some code to the configuration file to over-ride its
guess.  Section~\ref{cfg:printerpaper} on page~\pageref{cfg:printerpaper} has
more on how to do this.


By the way, because A3 printers aren't unheard of, and because you can
print (say) 1/3 long A2 on A3 paper, it makes perfect sense to include
an option to define A2 paper.  It seemed churlish not to go all the
way up to A0.  2/3 A4 is apparently a size commonly used for printing
business invoices and the like. If anyone really does use \rmpage for
producing A3 pages, please let me know---it's something I've been
wondering about.

If anyone would like to let me know about more US paper sizes, I'd be
happy to include them in future versions of \rmpage.


\section{Setting the body text size}
\label{gen:bodytextsize}

There's more detail on how this works in
sections~\ref{htw:textheightcalculation}
and~\ref{htw:textwidthcalculation}.

The range of values available for \comname{textwidth} and
\comname{textheight} is a compromise.  I wanted to keep the number of
options down, produce a wide spread of values, with a small minimum
step size.

The way things have turned out, the step size from one length option
to the next is, in general, different to the step size from one width
option to the next.  So you can't maintain a particular balance of
top:outside and bottom:inside margins (or whatever) by moving up to
the next width and length option.  There's no easy way round this---I
wrote \rmpage to set \comname{textheight} and \comname{textwidth}
independently, because that seemed most sensible at the time.  To
ensure a fixed aspect ratio would mean I would have to re-write it to
include code to scale up and down through a range of aspect ratios,
which might have to be related to the aspect ratio of the paper, and
allow you to scale the overall size of the text body up and down.
This is a big job.  I'd like to be able to set the text size that way
as well as the way \rmpage does it at the moment, but it'll have to
wait---I suspect I'll have to re-structure the entire package to be
able to support both ways of doing things without getting completely
mixed up.

I decided to set both the width and height of the text body by scaling
up and down geometric series.  The \comname{textheight} series was
chosen by deciding that the \optname{longest} length should fill an
ordinary page of A4, assuming a 6\units{mm} non-printing border.  The
\optname{normallength} length was set to the standard value (this is
only strictly true if you're using headers and footers), which defined
the geometric series.  The \optname{shortest} length on A4 is about
163\units{mm}, just over half the page.  It turns out that the series
produces a central minimum step (\optname{normallength} plus
\optname{touchlonger}) that is about one line (13\units{pt}, near
enough), so I think it's appropriate.

The \comname{textwidth} paper-based width series was derived
similarly; the final values give a spread based on a \optname{widest}
value that fills an A4 page to within about 7\units{mm} of the edge,
and down to just over half the page.  The \comname{textwidth}
character-based width series has been more of a headache; the current
(\filename{rmpwnorm.pko v 0.52}) version gives a spread from about 75
characters in the middle, to about 98 and 57 characters at the
extremes.  The central minimum step (\optname{normalwidth} plus
\optname{touchlonger}) is just under 2~characters, which seems
suitably small.  The problem with the character-based width setting is
that the standard \comname{textwidth} is at the upper limit for easy
reading; the widest available width, being wider than this, is far,
far, far too wide for easy reading; but the minimum
\comname{textwidth} doesn't get close to the minimum
\comname{textwidth} for easy reading.  But that probably won't matter
too much; the minimum width is less than half an A4 page if you're
using 11\units{pt} Computer Modern Roman in one column, which seems
small enough for most things I can think of (you might want a layout
which places figures and extensive side notes in the margin; if most
of the marginal note space were to be filled, having a very large
margin and small \comname{textwidth} makes sense.)


\section{Founts-}
\label{gen:founts}


\section{Marginal paragraphs-}
\label{gen:mpars}

\rmpage looks at the everything including the sunspot cycle and the
phase of the moon when it calculates the width of marginal paragraphs.
Section~\ref{optcfg:mpars} on page~\pageref{optcfg:mpars} lists all the
options that control marginal paragraph size, and has some more
information.

Note this: the default margin for marginal notes is the outside
margin for twosided printing, and the right-hand margin for one sided
printing.  The standard \LaTeX\ command \comname{reversemarginpar}
will reverse the marginal note placement; \comname{normalmarginpar}
will put it back to normal.  If you are printing twocolumn, the
marginal notes end up in the nearest margin.

\rmpage pays attention to all of this, and makes the marginal notes as
large as possible given the margin they will appear in.  The size is
constrained by various parameters, controlled by the
\optname{maxmparwidth}, \optname{mparsep}, and \optname{mparclearance}
sets of options.

The |clearance| parameter---the length \comname{RM@mparclearance}---is
the minimum gap between the marginal note and the edge of the paper,
subject to the additional restraint of the available printing area.
It's hardwired to 0.4\units{in} in standard \LaTeXe; \rmpage sets it
to be a fraction of \comname{paperwidth} which works out to be
0.4\units{in} if you're using US letter paper.  The |sep|
parameter---the length \comname{marginparsep}---is the gap between the
marginal note and the body text---this is a standard \LaTeX\
parameter.  The |maxwidth| parameter---the length
\comname{RM@maxmparwidth}---is the maximum allowed width: 2\units{in}
in standard \LaTeX, a fraction of \comname{paperwidth} which works out
to 2\units{in} if you're using US letter paper with \rmpage.  There
are also the \optname{small} and \optname{large}
\optname{basemparclear} and \optname{basemaxmparwidth} options.  The
large versions of these options set the normal parameter size to twice
the usual default value.

So if you're going to switch to \comname{reversemarginpar} in your
document, do so \emph{before} you load \rmpage.  If you're going to
switch from one to the other, before you load \rmpage select the one
which allows the least space for marginal paragraphs.

If you're deeply interested in the workings of marginal paragraph size
and placement, have a look at the \filename{dtx} files: I think the
basic idea is obvious.  Note that the parameters
\verb|\RM@mparclearance| \verb|\RM@maxmparwidth| can be set to any
positive value by one of the hooks (\verb|\RM@PrinterPaperSettings|
might be a good one to use) in your configuration file: negative
values are used by way of flags to set initial values.


\section{The configuration files}

\rmpage uses a configuration file to do lots of stuff.  The way it
works is this: if you're \LaTeX ing with \rmpage a file called
\filename{ermintrude.tex}, and a file \filename{ermintrude.rmp} exists
on the \TeX\ search path, then \filename{ermintrude.rmp} is used by
\rmpage as the configuration file for that run.  If not, then \rmpage
looks for a file given by the command \comname{\RMconfigfile}.  If
that file exists (if the command hasn't been defined before \rmpage is
loaded, it's set to \filename{rmplocal}; the file is searched for with
and without the \filename{cfg} extension added).  If that file can't
be found, \rmpage looks for \filename{rmpgen.cfg}, which is a
standard configuration file with everything enabled.

The file \filename{rmplocal.gfc} is meant to be renamed
\filename{rmplocal.cfg} and changed locally.  Have a read of the
comments in the file first, and change the optional argument to the
\comname{ProvidesFile} command to say that the file's been changed by
you.

The idea is that you don't change \filename{rmpgen.cfg}; it exists so
everyone has a standard configuration file that will produce identical
results.  You can tell \rmpage to use it in any given document by
saying:
\begin{verbatim}
|\newcommand{\RMconfigfile}{rmpgen.cfg}|
...
\usepackage{rmpage}
\end{verbatim}
in the preamble of the document, \emph{before} loading \rmpage.  But please
 do change \filename{rmplocal.cfg}, if you like.

The configuration file exists for these things:
\begin{itemize}

\item Declare additional standard options

\item Declare user options

\item Set local defaults

\item Place code in various hooks to be executed inside \rmpage, for
supporting local classes, particular combinations of printer and
paper, and so on.

\end{itemize}

Every installation of \rmpage should have a modified configuration
file for at least one reason: option processing is slow, and the fewer
options you have, the faster it works.  Folk with twin 225MHz
PPC604e processors, or something huge and humming with the word `Sun'
on the front probably don't care about \LaTeX ing speed, but I do.

The idea is this: edit \filename{rmpgen.cfg} and comment out all
the options you think you won't often use.  You can always uncomment
them later.  There's quite a few of these standard options that are
commented out to begin with; you might want to uncomment some of them
(if, for example, you're in the habit of printing on crown folio
paper).  By the way, don't add or delete anything anywhere above the
local option declaration section; if you do, upgrading to a new
version of \rmpage might turn out to be more awkward than you'd like,
and I'll come round and pester you, steal all your milk and put my
feet up on the sofa.

Setting local defaults isn't hard either: there's an
\comname{ExecuteOptions} statement with my local defaults in it.
Take them out and replace them with what you want.  If you use a
LaserJet 4 printer and US letter paper, and most of your output goes
into ring binders, say:
\begin{verbatim}
\ExecuteOptions{lj4printer,letterpaper,usdate,notstdmargins}
\end{verbatim}
Note that I've assumed someone printing on US letter paper want dates
formatted US style.  If you don't want \optname{notstdmargins}, say
\optname{stdmargins}.

There are some options that you can't specify in any
\comname{ExecuteOptions} statement seen by \rmpage: the \optname{touch}
options aren't allowed---they must be executed after their
corresponding `ordinary' options, which can't be arranged if they're
bunged in an \comname{ExecuteOptions} statement.  Try it if you
like: \rmpage will whinge at you.

If you want to declare a new printer type, for example, just copy one
of the existing option declarations to the section marked
`\textsc{begin local option declaration}', change the name and the
parameters, and there you go.

You might want to set up particular printer/paper settings; there's a
place for that and an example of how to do it in
\filename{rmplocal.cfg}.  Have a look and a fiddle: if I could do it,
it shouldn't be hard for someone as clever as you.

Adding code to the hooks in a useful fashion is best left to those who
like fiddling around: have a look at the guts of \filename{rmpage.dtx}
and so on.  Hopefully you'll get the hang of how I did things, and
you'll be able to add code to suit your installation.  I intend to
document the whole thing properly eventually, but I only have 24 hours
in each day.  Try to stick to the conventions I've established: it'll
make it more likely that your code will work properly with future
versions of \rmpage.  My intention is to keep things more-or-less as
they are, but this is the first version on public release, and I don't
know what changes will be needed yet.

\section{Sending \rmpage-formatted documents elsewhere}

There's a few potential problems with sending to other people
documents that you've formatted with the help of \rmpage.  One is that
they might not have \rmpage, and another is that even if they do have
\rmpage, it is possible that you have specified options that produce
layout parameters that are dependent on your particular printer, which
your recipient might not have, and even if they do have that
particular printer, they might have local configuration
code---particularly printer/paper specific settings---that produces a
slightly different result.

There's several different things you might do, depending on the
circumstances---the thing to do is think about where your document's
going, what's going to happen to it, and how you want it formatted.

In the usual run of things, a document that's been formatted with
\LaTeX\ and intended for general release is likely to be printed on A4
paper and US letter paper, so you have installation-dependent
differences\footnote{A4 paper is $210\units{mm} \times 297\units{mm}$
or $8.27\units{inches} \times 11.69\units{inches}$; US letter paper
is $8.5\units{inches} \times 11\units{inches}$ or $215.9\units{mm}
\times 279.4\units{mm}$} even without \rmpage---it's always worth
checking that this difference won't cause a problem (so US authors:
check that your document works on A4 paper, and everyone else in the
world make an effort for those isolated Americans.  Look, just go
metric, will you?  I read maps in miles and drink beer in
pints\footnote{real pints, not your sawn-off US version}, but really,
metric units make life so much easier.  If us \emph{and} the Italians
have made the switch, so can you.  Italians?  Yes, Italians: just
which empire do you think is referred to in the phrase `Imperial
units' Yes, that's right, the Roman Empire.  How about advancing into
the 19th century before the 21st begins, eh?  Sorry, a minor rant, but
one that I think should be made occasionally.)

Ignoring paper sizes for the moment, if your recipent has a copy of
\rmpage, format your document with the \optname{pessimisticprinter}
option, ensure that the result is acceptable, and send the thing.

If your local configuration file has non-printer-specific settings
that affect the output, you might send a copy of it (included in the
preamble of your file before the |\usepackage{rmpage}| command,
using the \envname{filecontents} environment) under the name
\filename{jobname.rmp}.  That is, if the document is in a file called
\filename{canes-venaciti.tex}, call the configuration file
\filename{canes-venaciti.rmp} and it will be used as the
configuration file for that document only.

Or you could run \LaTeX\ with the \optname{chatty} option specified to
\rmpage, and copy all the page layout parameters from the console
window and paste them into the preamble of your document.  If your
version of \LaTeX\ doesn't allow you to copy text from the console
window (\OzTeX~2.0 for the Macintosh does; I don't know about other
versions), copy the text from the \filename{log} file.  Then comment
out the call to \rmpage.  The problem with this is that it fixes
everything, including paper sizes, so a recipient who uses A4 paper
won't be impressed if they\footnote{A useful way of avoiding he/she
that has an ancient history} get a document hard-formatted for US
letter paper, and vice-versa.  Of course, you could provide two sets
of page layout parameters, with a note to the recipient to choose one
or the other.  This is messy, but it might produce the most reliable and
legible results.

A possible way round the paper problem is to assume that no-one will
be printing on anything other than US letter or A4 paper, and the
people receiving the document won't mind if the layout isn't very
good.  That way, you might use the \optname{letter4paper} paper size.
This paper size is $210\units{mm} \times 8.5\units{inches}$, and
documents formatted with it will fit on A4 and US letter paper without
formatting changes.  Mind you, the results won't be very nice, and
worse on US letter paper than on A4---\optname{letter4paper} pages
printed on A4 will be a little too short, which enlarges the gap at
the bottom.  Very often, this isn't too noticeable.  But
\optname{letter4paper} pages printed on US letter paper will have the
text body too far to the left, which tends to look very awkward.

\section{Speed and what to do about it}

\rmpage chews up a lot of processor time when it is being processed at
the start of a \LaTeX\ run, but adds nothing to subsequent processing
time---all it does is change page layout parameters.  Narrow columns
or short pages give \TeX\ a harder time in line and page breaking,
which might increase processing time, but I wouldn't worry about that
if I were you---the underfull \comname{hbox}es and \comname{vbox}es
will cause you more problems that the extra 0.2s time.

One verion of \rmpage (v0.65) added 18.5\units{s} to a \LaTeX\ run on
my Mac.  The same version of \rmpage added 19.1s to the processing
time when I used it with the comments left in (before being processed
by \packname{docstrip}).  This is an increase of 0.6s or 3\%.
Version~0.86 added 16\units{s} with comments, or 15\units{s} without,
an increase of 7\%.  In this case, file size was reduced from
200\units{K} to 85\units{K}.  It seems that, on my computer, the main
benefit of using \packname{docstrip} to remove comments from an input
file is reduced file size, not reduced processing time.  A
not-terribly-formal test concluded that each character in the first
argument to each \comname{DeclareOption} command added about
10\units{ms} to processing time---I saved 1.6\units{s} with
version~0.72 by commenting-out eight options containing 176
characters.

My computer is a Macintosh Performa 475 12/160 with a 25MHz 68LC040
microprocessor; I am told this processor is roughly equivalent to a
40MHz 80486SX. For the tests noted above I used OzTeX version 2.0.1
and 2.1 under system B1 7.1 P5 SU3.

You can speed up \rmpage quite a lot.  A large amount of the time
taken by \rmpage is in processing options---\LaTeXe's option
processing mechanism is very slow.  Commenting out options in the
config file is very effective at reducing processing time (it's the
length of the option names, not the amount of code, that increases
this processing time).

For example, \rmpage version~0.66 increased the time to process a
document by 21.2s; with 73 options commented out, \rmpage only added
12.1s.  This is an improvement of nearly 60\%.  This is why there are
two config files supplied with \rmpage: a slow one with all the
options enabled, and a faster one with some options commented out.

You can make \rmpage work faster by commenting out (don't delete
them---you never know when you might need them) all the options you
think you won't use very often.  If you do want to use one of these
commented-out options, uncomment it (and all the options in the same
group---you'll see what this means when you look at the file) and
leave it uncommented.  The Alpha text editor for Macintoshes can
comment out a group of lines if you select the lines and press
\keypress{cmd-D}; it can reverse the process if you press
\keypress{cmd-opt-D}.  I expect other text editors can do the same job
somehow: it might be worth finding out how to do this with your text
editor if you don't already know

Some versions of \TeX\ under some operating systems take quite a time
to find files to be input.  If this is the case on your computer, you
might be able to reduce processing time quite a lot by placing
frequently-used files in a place that is searched early on.  I have a
list of folders in my \filename{TeX-inputs} folder that looks a bit
like this:

\begin{verbatim}
aa LaTeX
ab Rat2e <--- rmpage is in here
ab tools
ac mfnfss
ac other 2e <--- other people's packages
ac other fd
ac PSNFSS
ba .cfg files etc
ba other 2.09
ba Rat old
bb chicken <--- Liverpool John Moore's University logo
bb new fds
bb other LaTeX
bb some AMS-LaTeX
za afm
za fontinst
za fontinst examples
za fontinst inputs
zz Graphics
zz Plain
zz Rat ex + tmp
\end{verbatim}

It's a compromise between speed and ease of management: I can change
parts of my \LaTeX\ system without getting a headache, and it's not
too slow for me.  The commonly-used folders are prefixed
\filename{aa}, \filename{ab}, end so on; the rarely used ones are
prefixed \filename{zz}.  My version of \LaTeX\ is set up to search the
folders early on in the alphabet first: this is not necessarily the
case with any other version of \LaTeX, even versions of \OzTeX\ with
the \filename{tex-inputs} list specified differently.

Some \TeX\ implementations take a lot of time to search multiple
directories, and are faster with fewer directories to search.  If this
is the case with your implementation, putting your input files into
fewer directories might help.  The only way I know to find if this is
the case is to sit down and do lots of tests: it might be quicker in
the long run not to bother.


\chapter{A brief lecture on typography}

I'm not a professional typographer, and this won't turn you into one.
Caveat emptor\footnote{buyer beware}, and remember how much this cost
you.  If you want to learn about typography, do the sensible thing and
get some books out of the library.  But given that I'm giving you
typographical controls, I'd better give you some idea how to use them
intelligently.  If you have any expertise in typography and think you
can do better than this, please do!  You've got my email address\ldots

The notes below are what I think are the main points of ordinary
typography the average \LaTeX\ user needs to know about, dealing with
the areas of typography that \rmpage gives you new control over; this
had no pretensions to being a course in typography---visit that
library!

\section{Introduction}

The usual reason for writing something is so that people can read and
digest the content---the reader is usually not interested in the form
of written work, just the words themselves.  It follows that the form
in which your words are presented should usually be un-noticed by the
reader: easy to read, conventional, and pleasing to the eye.  The
eye-catching typography used by advertisers does have its
place, but this short piece concentrates on mundane typography:
setting chunks of text, one word after another, line after line, to
form a document which is unobtrusive to read.  After all, if you were
battling with the mathematics of elastohydrodynamics, you wouldn't
appreciate a layout that was as hard to decipher as the content.

First thing is this: it's easy to make a right mess of things when you
can control the layout of a page, but many of the mistakes one can
make are in setting the paragraph indentation, vertical space around
displayed material, headings, and the like.  \rmpage doesn't touch
these, so you can forget about fouling up that area of design for
now.  As for the things you can control: don't stray too far from the
normal settings, and the results should be fine.

The easiest mistake to make is in setting the width of the body
text---\comname{textwidth} in \LaTeX\ terms.  Typography texts warn
that the most common fault is making the body text too narrow (less
than about 45 average characters); but my experience is that the
average non-typographer is more likely to make the body text too wide
(more than about 75 average characters).

The best advice to begin with is follow convention---most books on
typography make the point that your design has succeeded if no-one
notices it, so stick with the conventional.  Of course, you've got to
find out what conventional is.  Look at professionally designed
typography: at how the text body sits on the page of a book, and in
magazines and newspapers---is the text high, low, or in the middle?
closer to the binding edge or the outside edge?  How big, in visual
relationship to each other and the text body, are the margins top,
bottom, inside and outside?    What about columns of text in magazines and
newspapers---how wide are they, what's the gap between columns like
in relationship to the column width, the size of the average word
space, and the page margins?  How does the column width affect
readability---think about the effect of column width on line-breaking
and the flow of reading.

You'll notice that conventionally, lines of text longer than about 75
characters are avoided, as are lines less than about 45 characters;
and the outside margin is usually larger than the inside margin, for
the very straighforward reason that the outside margin is the one
that's most delicate and most handled.  If it's big, damage to the
paper is less likely to deface the text.  The bottom margin is usually
bigger than the top margin for the same reason---and even though this
reason is a sound practical one, if you produce a page with a smaller
bottom margin than top margin, it looks very odd to most people.  It's
also fairly conventional for the aspect ratio of the text to match the
aspect ratio of the paper---if the text is half the width of the
paper, it'll be half the height as well.

The standard \LaTeX\ classes produce a page layout that's generally in
line with those conventions, so if you don't pass extreme options to
\rmpage, you can't go far wrong.  Of course, if you're aiming for a
particular effect for a good reason, such as filling a page with a
time-table, you might want the longest and widest text area possible.
If you find that you \emph{are} using an extreme option (\optname{longest},
for example), think very carefully about why you are doing it, what
effect this option produces, and whether this is desirable.  If you're
just trying to cram that much text on one page, two columns is
probably better.

Because typography is to some extent an artistic endeavour, there is
no such thing as an ideal layout, although some are clearly
inappropriate.  But after a few hours at the keyboard, it's easy to
lose the ability to judge the passing of time, let alone the effect a
design has on a new reader.  I find that it often pays to trust
initial reactions, especially when comparing a set of alternatives.
So, if you're in doubt about which of your final two choices to use,
ask someone else what they think of them.

\section{Positioning the text body}

Convention has it that the outside margin is larger than the inside
margin, and that the bottom margin is larger than the top margin.
It's not unusual to match the sizes of the outside margin and the top
margin.  The \packname{Koma-script} documentation reports that Jan
Tschichold recommends that the bottom margin should be twice the size
of the top margin, and the outside margin should be twice the size of
the inside margin.

The reason \rmpage defaults to having the inside margin larger than
the outside margin is this: most of what I typeset ends up in
ring-binders, where you need a large inside margin to avoid punching
holes in the text.  You can change this default setting by editing an
\comname{ExecuteOptions} statement in the configuration file---see
chapter~\ref{chap:config}.  You can change this default in the
\filename{rmplocal.cfg} file by putting the \optname{stdmargins}
option in the default \comname{ExecuteOptions} statement.  If you
don't like the amount by which the inside margin is bigger than the
outside margin---bearing in mind that one convention has the outside
margin matching the top margin---you can change it with the
\optname{offset} option set.

If the top margin looks bigger than the bottom margin, the page looks
`bottom heavy' and rather odd.  It is possible that the only reason
for this is that everyone lays out pages with a smaller top margin,
but follow this convention to begin with.  \rmpage actually makes the
top and bottom margins the same size, just as the standard classes do;
the bottom margin looks larger because the bottom margin
\emph{appears} to begin at the bottom of the text body; \LaTeX\
measures it from the bottom of the footer.

i couldn't figure out a reliable way of getting \rmpage to calculate
\emph{apparent} ratios of top and bottom margins, so you'll have to
balance this by eye.  The \optname{altitude}, \optname{long}, and
\optname{short} option sets can help.

\section{Size of the text body}

Conventionally, one text column is 1.5 to 2.5 alphabets wide; the
standard \LaTeX\ widths give you about 2.5 alphabets wide (which is
about the average width of 75 characters of normal prose, hence
Lamport's comment in the \LaTeX\ book.  1.5 alphabets is about 44
average characters, but \rmpage warns when you get below 39 characters
wide---\TeX's a better line breaker than a human typesetter.  \rmpage
warns you if you exceed the 75---39 character limits; too wide and too
narrow are both awkward to read, and too narrow makes for rotten line
breaks.\footnote{I'm lying: \rmpage warns when  you exceed the
standard number of characters per line, which is 78.5 characters for
10pt, 74.8 for 11pt, and 75.5 characters for 12pt.}  \rmpage does
report the final \comname{textwidth} in terms of what it thinks are
average characters, which should help to give you an idea of what's
going on.

If you must fill a very wide space, try using multiple columns.  If
you must use narrow columns, ragged right setting sometimes looks
better, although a multiple-column narrow-columned layout might need
to have \comname{columnsep} adjusted with \rmpage's \optname{colsep}
options---look at a newspaper to get the idea.

The problem with long lines is this: when you get to the end of one
line, you need to find the start of the next one.  If the lines are
too long, this job is made harder.  The problem with short lines is
twofold: the job of finding the start of a line is a bother, so the
less you have to do it the better; and short lines make for rotten
line breaks, which makes it harder to follow the text.

Tschichold recommends that the aspect ratio of the text body be made
to match the aspect ratio of the paper.  This refers to the
\emph{apparent} aspect ratios---apparent paper size depends on the
binding used, and apparent text body size depends on the nature of the
headers and footers.  I couldn't come up with a reliable way of
calculating the comparative apparent aspect ratios, so if you want
these aspect ratios matched, you'll have to do it by eye.  The
\optname{altitude}, \optname{offset}, \optname{long}, \optname{short},
\optname{wide}, \optname{narrow} option sets can help.  Note that to
my eye at least, matching aspect ratios often matters less than
matching top and outside margins, which does appear often to make a
big difference to my comfort with a layout on small (A5 or so) pages.
And in any case, the match need not be spot on---but quite what near
enough is can't be specified exactly.  Sometimes an
almost-but-not-quite match is fine, sometimes it looks awful and
you're better off with a deliberate `I'm clearly not trying to match
these dimensions' layout.

\section{Typefaces}

The main thing the average \LaTeX\ user is concerned with is
legibility, and the computer modern typefaces score highly here.  I
have just seen a volume of conference proceedings (Proceedings of the
Applied Optics Divisional Conference of The Insitute of Physics, held
at Reading 16--19 September 1996) in which most papers were produced
with \LaTeX, using the publisher's style file.  To my eye, the papers
typeset in Times look most legible at a glance; the papers typeset in
Computer Modern feel easier to read when you get down to it.  I
mention this to make the point that just because something looks
right, doesn't mean that it's easier to read---not that Times is
anything but a famously legible typeface.

As a general rule, serif typefaces are easier to read in blocks of
text, and sans serif typefaces are easier to read as isolated words
or phrases.  This explains the choice of typeface on road signs and in
newspapers.  Typefaces with a heavy emphasis on vertical strokes---such
as the Bodoni beloved of US newspaper headlines---disrupt the
left-right flow of reading, whereas typefaces with a more horizontal
emphasis, such as Times or Baskerville, ease the left-right flow of
reading.

Choosing founts is a tricky thing---read those typography books!  But
there are some rules which can help you produce pleasant, legible
pages.  Before a real expert shoots me down in flames, I don't claim
that the list below is `rules for the correct use of typefaces'---it's
just, erm, `received wisdom', sort of.

\begin{itemize}

\item Always use a serif typeface for your body text.

\item Keep the number of type faces and type families in a design to a
minimum---use bold, italic, different sized, and maybe slanted
type faces from the one family where appropriate.

\item Always follow convention---don't invent a new convention
unless you really need to.

\item Don't mix similar but different typefaces---Times for the body
copy and New Century Schoolbook for captions looks awful.

\item But do mix very different typefaces---Times for the body copy
and Helvetica (which I personally loath with a deep loathing) for
captions or headings, for example, can be very effective.

\item Use a body text size from 9\units{pt} to 12\units{pt}; some
books suggest no more than 11\units{pt} for body text, and I reckon
10\units{pt} is a bit too small at 300\unit{dpi}, but fine at
600\unit{dpi}.

\item Never, ever, underline unless you absolutely have to on pain of
severe dandruff.

\item And George Orwell said in his rules for good English,
break any of these rules rather than do anything outright barbarous.

\end{itemize}


\chapter{All the options (rmpage v0.69 and rmplocal.cfg v0.11)}

\label{chap:alltheoptions}

Most of the options in \rmpage work by setting an internal paramters,
which is later used to decide what value to set something to as part
of a more involved calculation.  Sometimes more than one parameter is
used in this decision.  The description of each option tells you what
this parameter is set to, and what effect the option has.




\section{Options in \rmpage}

This is a list of the options contained in the file \rmpage; there's
lots more in the configuration file.  None of the options in \rmpage
are commented out, nor are any of the options in the configuration
file \filename{rmpgen.cfg}.  Some of the options in the configuration file
\filename{rmplocal.gfc} have been commented out for speed's sake; this file
will be used by \rmpage if you rename it \filename{rmplocal.cfg}

\subsection{Reporting dimensions and tracing calculations}


These options control the amount of stuff that \rmpage litters your
console window with.  They do this by setting the \verb|\RM@chatlevel|
parameter, which is looked at by a bunch of reporting commands---look
at \rmpage for the details about these.  The default is
\optname{taciturn}; you get a handful of dimensions and warnings get
printed on the console.  I expect that most people won't find
\optname{garrulous} useful.  The \optname{yorkshire} causes \rmpage to
print nought: this is a British regional joke that I'm allowed to make
because I live in Lancashire, which is another British regional joke.
(C.f.  Ian MacMillan on `4th Column', BBC R4 6/10/96: ``'Ey oop.  All
right.  That should be enough for a column.  Y'see, I'm from South
Yorkshire, and we don't talk a lot.')

\begin{description}\shorty\squish
\awakeoption{garrulous}  \rmpage reports everything I thought might be
useful someday.  It used to be worse.  Sets \verb|\RM@chatlevel| to 0

\awakeoption{chatty}  \rmpage reports all \LaTeX\ layout parameters that
it changes, plenty of \rmpage's own parameters, and some information
about what's going on as it caluculates them.  Sets \verb|\RM@chatlevel| to 1

\awakeoption{taciturn}  Default. \rmpage reports the height and width of the
text and paper, the width of the text in characters, and a few other
things.  Warnings are also printed.  Sets \verb|\RM@chatlevel| to 2.

\awakeoption{yorkshire} Allows \rmpage to print errors only, although
warnings are put in the log file with a few other bits.  Sets
\verb|\RM@chatlevel| to 3
\end{description}

\subsection{Paper sizes}
\label{optrmp:paper-sizes}

More paper types are defined in the configuration file, as are the
long paper sizes.  Each paper size is given a code number, because
it's easier and faster to check for a number than a name.  If you want
to define your own paper sizes, I suggest you use code numbers above
1000 so that future versions of \rmpage don't have standard sizes than
conflict with your sizes.  The paper size number 0 doesn't have much of a
role in life yet, but I'm working on it.

The \optname{landscape} and \optname{portrait} options force the paper
size to be in that orientation; the standard class swap
\comname{textheight} and \comname{textwidth} when you ask for
landscape.  Landscape orientation is defined as the long side being
horizontal; portrait orientation is defined as the short side
horizontal.  2/3 A4 is usually used in landscape orientation ($210
\times 198\units{mm}$, as are DL envelopes.  Some printer drivers,
when dealing with DL envelopes, call `landscape', `portrait', and
vice-versa.

\rmpage knows about lots of sizes; some of them are large, obsolete, or
untrimmed paper sizes which I don't expect will be directly useful to
anyone.  It just seemed inelegant to leave them out.  Perhaps
I have a warped sense of \ae sthetics.  If you have a Macintosh with
QuickDraw~GX, you can tell your printer driver about any paper size
that your printer is physically able to deal with, so these odd sizes
might be more useful than I think.

If any Americans would like to send me the dimensions of some more US
paper sizes (including envelopes), I'll include them in future
versions.  The only places I could find US paper size data were
\LaTeX\ classes and my printer's manual.

According to BS4000:
\begin{quotation}
 The ISO A series is based on A0, with a surface
area of $1\units{m}^{2}$.  Each ISO B paper size is a geometric mean between
adjacent A sizes, with sides in the same proportions.

Each size shall be acheived by dividing the size immediately above it
into two equal parts, the division being parallel to the shorter
side.  Consequently, the areas of two successive sizes shall be in the
ratio $2:1$.

All the size in each series shall be geometrically similar to one
another.
\end{quotation}

What this means is that the ratio of the sides must be $1:\sqrt{2}$,
so that A0 is $841\units{mm} \times 1189\units{mm}$.

Tolerances are specified thus:

\begin{tabbing}
$150\mathrm{mm}<\mathrm{sizes}\leq 600\mathrm{mm}$ \= $\pm2 \mathrm{mm}$\kill
$\mathrm{sizes}\leq 150 \mathrm{mm}$ \> $\pm1.5 \mathrm{mm}$ \\
$150\mathrm{mm}<\mathrm{sizes}\leq 600\mathrm{mm}$ \> $\pm2 \mathrm{mm}$\\
$600\mathrm{mm}<\mathrm{sizes}\mathrm{mm}$ \> $\pm3 \mathrm{mm}$\\
\end{tabbing}

Long ISO sizes are created by dividing ordinary ISO sizes into slices,
cutting parallel to the short edge, e.g.,

\begin{tabbing}
1/3 A4 \= $99 \times 210$ \\
1/4 A4 \> $74 \times 210$ \\
1/8 A8 \> $13 \times 74 $ \\
\end{tabbing}

2/3 A4 is apparently a common size commercially, used for invoices and
the like.  It is defined as $198 \times 210 \mathrm{mm}$.  Note that
the standard defines sizes to the nearest millimetre, but \rmpage does
\emph{not} round the calculated long sizes, nor does it ensure that
only ISO sizes are processed by the `long' options.

\rmpage does not limit you to making long sizes out of ISO paper only
because you can chop up any bit of paper you like.  The reason \rmpage
does not round long sizes to the nearest millimetre is that if you are
printing on a piece of ready-cut long paper, \rmpage's maximum
deviation from the specified size, 0.5\unit{mm}, is within tolerance;
and if you are making your own long paper by cutting up a straight
size, rounding to the nearest millimetre on a 1/8 size could result in
a 4\unit{mm} error by the time you cut the strip furthest from your
datum edge, which is outside the specified tolerance for the size of
the sheet you are cutting.  The fact that not rounding is easier to
code is, of course, entirely co-incidental and played no part in the
design decision.

Data source for the old British book sizes: Pears Cyclopedia, 68th
edition, 1959-1960.  Pelham Books Ltd., 1959.  (General Compendium,
page N13).  A, B, and other untrimmed sizes taken from BS4000.  C3,
C4, C5, C6, DL, and non-ISO envelope sizes taken from BS4264.  C0, C1,
C2, C7, and C7/6 taken from The Cambridge Factfinder, Cambridge
University Press, 1993.


Each paper type is given a number:

\begin{tabbing}
0=undefined, 1=letter, 2=legal, 3=executive,\\
9=letter4paper\\
10=a0, \= \ldots , \= 20=a10 (a4=14, a5=15)\\
30=b0, \> \ldots , \> 40=b10\\
50=c0, \> \ldots , \> 57=c7, 58=dl, 59=c7/6,\\
64=bslargelegalenvelope, \= 65=bscalendarenvelope\kill
60=bspopseedenvelope,     \> 61=bspopnonisoenvelope, \\
62=bsbrochureenvelope,   \> 63=bslegalenvelope,\\
64=bslargelegalenvelope, \> 65=bscalendarenvelope\\
66=no10envelopepaper\\
85=medium quarto, \= 80=imperial quarto, \= 81=imperial octavo\kill
70=foolscap folio, \> 71=foolscap quarto, \> 72=foolscap octavo\\
73=crown folio, \> 74=crown quarto, \> 75=crown octavo\\
76=royal folio, \> 77=royal quarto, \> 78=royal octavo\\
79=imperial folio, \> 80=imperial quarto, \> 81=imperial octavo\\
82=large crown octavo\\
83=demy quarto, \> 84=demy octavo\\
85=medium quarto, \> 86=medium octavo\\
93=sra0, \= 94=sra1, \= 95=sra2\kill
90=ra0, \> 91=ra1, \> 92=ra2,\\
93=sra0, \> 94=sra1, \> 95=sra2\\
98=metric large quad crown paper, \= 99=metric quad demy paper\kill
96=metric double crown paper, \> 97=metric quad crown paper\\
98=metric large quad crown paper, \> 99=metric quad demy paper\\
100=metric small quad royal paper\\
\end{tabbing}

The paper types are divided up like this:
\begin{tabbing}
10 \= C sizes plus dlpaper defined by 10 options: for envelopes, etc.
\= (50-59)\kill
 3 \> US sizes recognised by 3 options \>  (1-3)\\
 1 \>  bodge size recognised by 1 option  \> (9)\\
11  \> A sizes defined by 11 options: for writing paper, books, etc. \>
(10-20)\\
11  \> B sizes defined by 11 options: for posters, etc.  \> (30-40)\\
10  \> C sizes plus dlpaper defined by 10 options: for envelopes, etc.
 \> (50-59)\\
 6 \>  BS4264 envelope sizes defined by 6 options:  \> (60-65)\\
 1  \> US envelope size defined by 1 option:  \> (66)\\
17  \> old British sizes defined by 18 options. \>  (70-86)\\
11  \> BS4000 untrimmed sizes defined by 11 options  \> (90-100)\\
\\
71 \>  different paper sizes defined by 72 options.
\end{tabbing}



\begin{description}\shorty\squish

\awakeoption{letterpaper} Sets |\RM@papertype| to 1
-- 11in by 8.5in

\awakeoption{legalpaper} Sets |\RM@papertype| to 2
-- 14in by 8.5in

\awakeoption{executivepaper} Sets |\RM@papertype| to 3
-- 10.5in by 7.25in

\awakeoption{a4paper} Sets |\RM@papertype| to 14
-- 297mm by 210mm

\awakeoption{a5paper} Sets |\RM@papertype| to 15
-- 210mm by 148mm

\awakeoption{b5paper} Sets |\RM@papertype| to 35
-- 250mm by 176mm

\awakeoption{c6paper} Sets |\RM@papertype| to 56
-- 162mm by 114mm

\awakeoption{dlpaper} Sets |\RM@papertype| to 58
220mm by 110mm.  Ordinary envelopes.

\awakeoption{no10envelopepaper} Sets |\RM@papertype| to 66
9.5in by 4.12in.  Ordinary US envelopes.

\end{description}

% long paper type numbers:
%
% 0 = not long (or 1)
% 1 = 7/8  2 = 3/4  3 = 2/3  4 = 5/8  5 = 1/2
% 6 = 3/8  7 = 1/3  8 = 1/4  9 = 1/8
%
% long paper options in rmplocal.cfg

\subsection{Typesetting tightness}
\label{optrmp:tightness}

These options only exists because \rmpage can load the
\packname{PSNFSS} founts, and Karl Berry says that these founts
(produced with \packname{FontInst}) have too little slack in the
inter-word space for \TeX\ to be able to form paragraphs well with the
standard typesetting parameters.

These options must be executed before the \optname{load<fount>}
options.  That's inevitable if I have these option declarations before
the \optname{load<fount>} option declarations, and use
\comname{ProcessOptions} rather than \comname{ProcessOptions*}.  This
is so that these options can over-ride the default typesetting
tightness (\optname{looseish}) requested by the \optname{load<fount>}
options.

\begin{description}\shorty\squish
\awakeoption{tight} Default. Leaves the typesetting parameters alone.  Defines
\verb|\RM@looseoption| to be 0.

\awakeoption{looseish} Changes the typesetting parameters to be
something in between the adjacent options.  Defines
\verb|\RM@looseoption| to be 1

\awakeoption{loose} Changes the typesetting parameters to something
close to Alan Jeffries's recommendations.  Defines \verb|\RM@looseoption|
to be 2

\awakeoption{looser} Changes the typesetting parameters to be
something in between the adjacent options.  Defines
\verb|\RM@looseoption| to be 3

\awakeoption{loosest} Changes the typesetting parameters to something
close to twice as sloppy as Karl Berry's recommendations.  Defines
\verb|\RM@looseoption| to be 4

\awakeoption{sloppyish} Changes the typesetting parameters to be even
looser than \verb|\sloppy|, for two column typesetting with
\packname{PSFNSS} founts.  Defines \verb|\RM@looseoption| to be 5.

\end{description}


\subsection{Textheight setting}
\label{optrmp:textheight}

These request a \comname{textheight} shorter or longer than normal.
If you allow space for headers and footers with the \optname{headers}
and \optname{footers} options, \optname{normallength} gives you the
same \comname{textheight} as you'd get with the standard classes.  The
other lengths are scaled up and down from the normal value in a
geometrical sequence.  (Actually, the total space above and below the
text body including headers and footers is the dimension that's scaled
in a geometrical sequence, but that shouldn't bother you too much.)

The \optname{touchlength} options add or subtract one from the value
set here.

There are also \optname{stdlength} and \optname{fulllength} options in
the configuration file; see section~\ref{optcfg:odd-lengths} on
page~\pageref{optcfg:odd-lengths}.

\begin{description}\shorty\squish
\awakeoption{shortest} Sets |\RM@lengthoption=3|; this
number gives you: $|\RM@totalheadfootclearance|=0.5200|\paperheight|$
\awakeoption{shorter} Sets |\RM@lengthoption=6|
\awakeoption{short} Sets |\RM@lengthoption=9|
\awakeoption{shortish} Sets |\RM@lengthoption=12|
\awakeoption{normallength} Default. Sets |\RM@lengthoption=15|; this
number gives you: $|\RM@totalheadfootclearance|=0.2130|\paperheight|$
\awakeoption{longish} Sets |\RM@lengthoption=18|
\awakeoption{long} Sets |\RM@lengthoption=21|
\awakeoption{longer} Sets |\RM@lengthoption=24|
\awakeoption{longest} Sets |\RM@lengthoption=27|; this
number gives you: $|\RM@totalheadfootclearance|=0.0872|\paperheight|$
\end{description}
%
%
\subsection{Headers and footers}

These options only deal with the gap between text body and the header
(or footer).  See section~\ref{optrmp:hfonoff} on
page~\pageref{optrmp:hfonoff} for options to turn the headers and footers
on and off.

The way these options work is by setting a parameter which is passed
to the |\RM@scalebyoption| command, to scale the required page layout
parameter by a value in a geometric sequence.  Have a look at the
command in \rmpage for the details.

\comname{headsep} is just multiplied by the requested value.
\comname{footskip} is the distance from the bottom of the text body to
the bottom of the footer.  To approximate a scaling of the distance
between the top of the footer and the bottom of the text, \rmpage
assumes that the footer is a single line that is \comname{baselineskip}
high, and subtracts \comname{baselineskip} from \comname{footskip}
before scaling, and adds it back afterwards.  This seems to work well
enough.

The value set by the options below can be modified by the
\optname{touchheadsep} and \optname{touchfootskip} options, which add
or subtract one from the appropriate parameter.

% Headsep is scaled by option
%
\begin{description}\shorty\squish
\awakeoption{leastheadsep} Sets |\RM@headsepoption=3|
\awakeoption{lessheadsep} Sets |\RM@headsepoption=6|
\awakeoption{lessishheadsep} Sets |\RM@headsepoption=9|
\awakeoption{normalheadsep} Default. Sets |\RM@headsepoption=12|
\awakeoption{moreishheadsep} Sets |\RM@headsepoption=15|
\awakeoption{moreheadsep} Sets |\RM@headsepoption=18|
\awakeoption{mostheadsep} Sets |\RM@headsepoption=21|
%
% Footskip is scaled by option
%
\awakeoption{leastfootskip} Sets |\RM@footskipoption=3|
\awakeoption{lessfootskip} Sets |\RM@footskipoption=6|
\awakeoption{lessishfootskip} Sets |\RM@footskipoption=9|
\awakeoption{normalfootskip} Default. Sets |\RM@footskipoption=12|
\awakeoption{moreishfootskip} Sets |\RM@footskipoption=15|
\awakeoption{morefootskip} Sets |\RM@footskipoption=18|
\awakeoption{mostfootskip} Sets |\RM@footskipoption=21|
\end{description}
%
% The marginpar setting options are now in rmplocal.cfg

\subsection{Columnsep}
\label{optrmp:columnsep}

\comname{columnsep} is a standard \LaTeX\ parameter: it is the space
in between columns of text on a multiple column page.  The
\optname{colsep} options scale \comname{columnsep} using the internal
\comname{RM@scalebyoption} command: see
section~\ref{htw:headfootmpar} on page~\pageref{htw:headfootmpar}
for details.  Briefly, \optname{normalcolsep} does nothing;
\optname{mostcolsep} multiplies \comname{colsep} by 2.5;
\optname{leastcolsep} divides \comname{columnsep} by 2.5; and
intermediate options use a factor in between along a geometrical
sequence.

There are corresponding \optname{touchcolsep} options.  They must be
executed after these main options, which is easily arranged---see
section~\ref{optcfg:touchoptions} on page~\pageref{optcfg:touchoptions} for
the details.

\begin{description}\shorty\squish
\awakeoption{leastcolsep}  Sets \comname{RM@columnsepoption} to 3
\awakeoption{lesscolsep}  Sets \comname{RM@columnsepoption} to 6
\awakeoption{lessishcolsep}  Sets \comname{RM@columnsepoption} to 9
\awakeoption{normalcolsep}  Default.  Sets \comname{RM@columnsepoption} to 12
\awakeoption{moreishcolsep}  Sets \comname{RM@columnsepoption} to 15
\awakeoption{morecolsep}  Sets \comname{RM@columnsepoption} to 18
\awakeoption{mostcolsep}  Sets \comname{RM@columnsepoption} to 21
\end{description}


If \comname{RM@adaptivecolseptrue}, then \comname{columnsep} is set to
be a fraction of the number of points per character.  This isn't
always appropriate, and the flag is set to false by default.

\begin{description}\shorty\squish
\awakeoption{adaptivecolsep}  \comname{columnsep} is set to be 2.3
times the width of one average character, according to \rmpage's
reckoning.  This new value can be scaled by the
\optname{mostcolsep} to \optname{leastcolsep} options.

The standard \comname{columnsep} is 2.03 times the width of one
average 10\units{pt} Computer Modern Roman character.  This is a
useful option if you are creating a style based on a fount size larger
than 12\units{pt}.  Otherwise, it seems to be a good idea on Mondays,
Wednesdays, and Fridays; not so good on Tuesdays, Thursdays, and
Saturdays; and on Sundays, I write Ogham on tree bark.

\awakeoption{noadaptivecolsep}  Default.  \comname{columnsep} is not
changed from its default value, although it might be scaled by the
\optname{mostcolsep} to \optname{leastcolsep} options.

\end{description}
%
% stdmargins: large margin outside, small margin inside.
% notdstdmargins: large margin inside, small margin outside.
%
%
% Note that the width option setting options need to be executed in
% the order they are declared here: so that you can increment or
% decrement narrowest---widest; but stdwidth, half and one inchmargins,
% and fullwidth are all fixed options.
%
% One easy way of ensuring this is to keep these options declared in
% this order and use \ProcessOptions rather than \ProcessOptions*
%
\subsection{Width of the text body}
\label{optrmp:textwidth}

The width options let you ask for a larger or smaller
\comname{textwidth}.  Following the basic idea of the standard
classes, \rmpage calculates two different \comname{textwidths}: one
is based on the number of characters in a line; the other is based on
the size of the paper.  The smaller of these two guesses is used as
the basis for the final \comname{textwidth}---\comname{textwidth} is
also constrained by \comname{RM@mintextwidth},
\comname{RM@maxtextwidth}, \comname{RM@mininsidemargin},
\comname{RM@minoutsidemargin}, \comname{RM@minleftclearance}, and
\comname{RM@minrightclearance}.

\rmpage is inclined to print out warnings if it has to change its
preferred \comname{textwidth} because of one of the above
restrictions---the \optname{yorkshire} option will silence these
warnings if you find them irritating.

The \optname{normalwidth} option gives a \comname{textwidth} close to
the standard \LaTeX\ width on US letter or A4 paper, where the
character-based width is usually used (wide founts like Lucida Casual
are the exception to this).  \comname{textwidth} is larger than usual
if you print on smaller paper, where the paper-based
\comname{textwidth} is used.  The options ranging out to
\optname{widest} and \optname{narrowest} request a
\comname{textwidth} varying in a smooth geometrical sequence, but
remember that the smaller of the two widths (character-based and
paper-based) is used, and there are several other restrictions on
\comname{textwidth}, so this smooth progression may not be apparent
as you step up or down through the options.  The gory details are in
the file \filename{rmpnorm}.

You can control which of the two widths---character-based or
paper-based---is used as the final \comname{textwidth}.  See the next
section (section~\ref{gen:widthsettingcontrol}) for details.

The \optname{stdwidth} option forces \rmpage to calculate the
\comname{textwidth} in the same way as the standard classes, except
that the final value is still subject to the restrictions listed
above.  So it is possible to ask for \optname{stdwidth} and get a
\comname{textwidth} that is not what you'd've got with the a standard
class.  \rmpage will warn you if this happens.

\begin{description}\shorty\squish

\awakeoption{widest}  Paper-based textwidth is set to 1.3096 times the
\optname{normalwidth} value; character-based textwidth is set to 1.9761
times the \optname{normalwidth} value.  Sets \comname{RM@widthoption} to 26

\awakeoption{wider}  Sets \comname{RM@widthoption} to 23

\awakeoption{wide}  Sets \comname{RM@widthoption} to 20

\awakeoption{wideish}  Sets \comname{RM@widthoption} to 17

\awakeoption{normalwidth} Default.  Paper-based \comname{textwidth} is
set to |0.7138\paperwidth|; character-based \comname{textwidth} is set
to 78.5 characters (10\units{pt}), 74.8 characters (11\units{pt}), or
75.5 characters (12\units{pt})---this produces a character-based
\comname{textwidth} very close to the standard width setting code.
Sets \comname{RM@widthoption} to 14

\awakeoption{narrowish} Sets \comname{RM@widthoption} to 11

\awakeoption{narrow}  Sets \comname{RM@widthoption} to 8

\awakeoption{narrower}  Sets \comname{RM@widthoption} to 5

\awakeoption{narrowest}   Paper-based textwidth is set to 0.7636 times the
\optname{normalwidth} value; character-based textwidth is set to 0.5061
times the \optname{normalwidth} value.  Sets \comname{RM@widthoption} to 2

%
% The touchwidth options are in rmplocal.cfg now.
%
\awakeoption{stdwidth} Attempts to produce a page with the same
\comname{textwidth} as the standard classes would give.  Sets
\comname{RM@widthoption} to 32

\awakeoption{halfinchmargins} Attempts to produce a page with a total
horizontal margin space of one inch.  If you have asked for
\optname{centre}d printing, \rmpage will try to produce half inch
margins either side.
Sets \comname{RM@widthoption} to 31

\awakeoption{oneinchmargins} Attempts to produce a page with a total
horizontal margin space of two inches.  If you have asked for
\optname{centre}d printing, \rmpage will try to produce one inch
margins either side.
Sets \comname{RM@widthoption} to 30

\awakeoption{fullwidth} Produces the widest possible
\comname{textwidth} given all other restrictions.  Sets
\comname{RM@widthoption} to 29
%
\end{description}

\subsection{Width setting control}
\label{gen:widthsettingcontrol}

These control which dimensions \rmpage takes notice of when setting
\comname{textwidth}---\rmpage can look at the size of the paper and
the number of characters when it's setting \comname{textwidth}.
Normally it looks at both, and picks the one that results in the
smallest \comname{textwidth}.  The code that does this is in the width
setting \filename{pko} file; it works out what to do based on the
value of the |\RM@setwidthby| command.

If you ask for one of these widths: \optname{oneinchmargin},
\optname{halfinchmargin}, and \optname{fullwidth}, you shouldn't also
ask for \optname{characterwidthset}, because the width options ask
for widths that are inherently based on the size of the paper.
\rmpage will point out this mistake if you make it, and carry on as if
you'd not said \optname{characterwidthset}.

I'm not that keen on these option names, especially
\optname{bothwidthset} which is formed in a regular sequence with the
other two, and ends up both ugly and not very descriptive; if you can
come up with something better, please let me know.

%
% 0 = set width by character and paper-based widths (default)
% 1 = set width by character-based width only
% 2 = set width by paper-based width only
%
% The bothwidth option is redundant, which is a good thing because I
% really don't like the name

\begin{description}\shorty\squish
\awakeoption{bothwidthset} Default.  Use both paper and character
based \comname{textwidth} requests to set \comname{textwidth}.
Defines|\RM@setwidthby| to be 0

\awakeoption{characterwidthset} Use the character based \comname{textwidth}
request only to set \comname{textwidth}.  Defines |\RM@setwidthby| to be 1

\awakeoption{paperwidthset}  Use the paper based \comname{textwidth}
request only to set \comname{textwidth}.  Defines |\RM@setwidthby| to be 2
\end{description}
%


\subsection{Margins}
\label{optrmp:margins}

The options in this section control the horizontal position of the
text body.  The \optname{twoside} and \optname{oneside} options are
standard options that \rmpage understand; the rest of them are new in
\rmpage.

It's probably best to read about all of these options, not just some
of them, because they all interact to some extent.


\begin{description}\shorty\squish

\awakeoption{twoside} Places the text body for printing on both sides
of the paper, taking into account the requested offest and which
margin you want to be the larger one (inside or out).

Sets \verb|\RM@twosidetrue| and \verb|\@mparswitchtrue|; the latter
step is performed by the standard classes.  I'm not certain this is
the right thing to do; it means you get the standard classes' effect
if you pass this option to \rmpage only.  But you might want to avoid
the standard classes' effect\ldots{} But if you're that clever, you
can reset the switch yourself.


\awakeoption{oneside} Default.  Places the text body for printing on
both sides of the paper, taking into account the requested offest and
which margin you want to be the larger one (inside or out).  Sets
\verb|\RM@twosidefalse|

\awakeoption{centre} Forces the left and right margins to be the same
size.  Sets \verb|\RM@centretrue|.

\awakeoption{notcentre} Default.  Allows the left and right margins
to be different sizes.  Sets \verb|\RM@centrefalse|

\end{description}


The options in the list below control which margin is the larger one.
Conventional book typesetting and \LaTeX\ makes the outside margin the
larger one; \rmpage makes the inside margin the larger one.  This is
because most of what I produce is bound in loose-leaf ring-binders,
where having a small inside margin often results in a holes in the
text.

You can change this default setting by changing the
\optname{notstdmargins} option to \optname{stdmargins} in the default
\comname{ExecuteOptions} statement in your local configuration file.
You will find this statement just below the line in the configuration
file that reads: \textsc{change this line to match your local
preferences}.

\begin{description}\shorty\squish

\awakeoption{stdmargins}  Default.  Outside margin is the larger one.  Sets
\comname{RM@stdmarginstrue}

\awakeoption{notstdmargins}  Inside margin in the larger
one.  Sets \comname{RM@stdmarginsfalse}

\end{description}


You can control the relative proportions of the inside and outside
margins with the \optname{offset} options.
These offset options don't do anything if the centre option has been
specified.

The default offset is 60\% of the total horizontal margin space in the
larger margin, 40\% in the smaller.  This the the standard \LaTeXe\
offset.  \optname{leastoffset} gives you equal margins;
\optname{touchlessoffset} and \optname{leastoffset} together makes the
nominally larger margin into the smaller one, by a small amount.
\rmpage will warn you if this happens.  \optname{mostoffset} puts
87\% of the total horizontal margin space into the larger margin; this
is about as far over to one side as your printer is likely to be able
to print.

There are \optname{touchoffset} options in the standard configuration
files---see section~\ref{optcfg:touchoptions} on page~\pageref{optcfg:touchoptions}

Have a look at \filename{rmpnorm} for more details if you need them.


\begin{description}\shorty\squish
\awakeoption{leastoffset} 50\% larger margin. Sets \comname{RM@offsetoption} to 2
\awakeoption{lessoffset}  53\% larger margin. Sets \comname{RM@offsetoption} to 5
\awakeoption{lessishoffset}  56\% larger margin. Sets \comname{RM@offsetoption} to 8
\awakeoption{normaloffset}  Default.  60\% larger margin. Sets \comname{RM@offsetoption} to 11
\awakeoption{moreishoffset} 68\%  larger margin. Sets \comname{RM@offsetoption} to 14
\awakeoption{moreoffset}  77\% larger margin. Sets \comname{RM@offsetoption} to 17
\awakeoption{mostoffset}  87\% larger margin. Sets \comname{RM@offsetoption} to 20
\end{description}

The touchoffset options must be executed after the offset options.
This is easy to arrange: just declare the options with the
\optname{touch} options after the main options, and use
\comname{ProcessOptions} rather than \comname{ProcessOptions*} (that
is, these options must be processed in the order of declaration,
rather than the order given in the calling commands).

\subsection{Number of columns}

Note that the standard classes set the \comname{@twocolumn} flag true
or false, depending.  \rmpage doesn't, and works quite happily without
it.

The config file has \optname{onecolumnwidth} to
\optname{tencolumnwidth} options, which change \comname{textwidth} but
don't change the number of columns that \LaTeX\ typesets text in.  You
can use the \packname{multicol} package to do that.

\begin{description}\shorty\squish

\awakeoption{onecolumn} Default.  This standard option is recognised
by \rmpage.  This option makes the standard classes typeset one column
to a page; \rmpage\ calculates a character-based \comname{textwidth}
based on this.  Defines \comname{RM@textcols} to be 1.

\awakeoption{twocolumn} This standard option is recognised by \rmpage.
This option makes the standard classes typeset two columns to a page;
\rmpage\ calculates a character-based \comname{textwidth} based on
this.  Defines \comname{RM@textcols} to be 2.

\end{description}

\subsection{Paper orientation}

These options really do force the appropriate orientation; the
standard classes just swap \comname{textheight} and
\comname{textwidth} when asked for \optname{landscape}.  Remember that
you'll most likely want to print your envelopes landscape, even if
your printer driver thinks you mean portrait (Hewlett Packard's
DeskWriter series 6.0 printer driver gets this wrong.  Oops.)  And
2/3~A4 is usually used in landscape orientation, even though you'll
probably think it's portrait---I know I did.

\begin{description}\shorty\squish

\awakeoption{portrait} Default.  Forces \comname{textwidth} to be less
than \comname{textheight}.  Sets |\RM@portraittrue|

\awakeoption{landscape} Forces \comname{textwidth} to be more than
\comname{textheight}.  Sets |\RM@portraitfalse|

\end{description}

\subsection{Headers and footers}
\label{optrmp:hfonoff}

Allows space for headers and footers, or not, as you wish.  These
options \emph{do not} affect the contents of headers and footers in
any way: if you want to change the \LaTeX\ \comname{pagestyle}, you
must do that separately.

Turning headers and footers on and off changes \comname{textheight}:
this is because of the way \rmpage calculates \comname{textheight}.

\rmpage first calculates the sum of the blank space above and below
all the text on the page; this is a constant fraction of
\comname{paperheight} for any given length option.  What is left over
after space has been allowed for headers and footers is
\comname{textheight}

\begin{description}\shorty\squish
\awakeoption{noheaders} Produce a layout for pages without headers.
Sets |\RM@headersfalse|; this results in \comname{headheight} and
\comname{headsep} being set to 0\units{pt}.

\awakeoption{headers} Default.  Produce a layout for pages with
headers.  Sets |\RM@headerstrue|; this results in \comname{headheight}
being set to \comname{baselineskip}.

\awakeoption{nofooters} Produce a layout for pages without footers.
Sets |\RM@footersfalse|; this results in \comname{footskip} being set
to 0\units{pt}

\awakeoption{footers} Default.  Produce a layout for pages with
footers.  Sets |\RM@footerstrue|.

\end{description}

\subsection{Positioning the text body vertically}
\label{optrmp:altitude}

These options affect the ratio between the gap below all the text and
the gap above all the text.  The \optname{touchaltitude}
options change this ratio in increments of $1/24$.

\begin{description}\shorty\squish

\awakeoption{highest} Top:bottom space = 0:8.  Sets
|\RM@headfootbalance=0|

\awakeoption{higher} Top:bottom space = 1:8.  Sets
|\RM@headfootbalance=3|

\awakeoption{high} Top:bottom space = 2:8.  Sets
|\RM@headfootbalance=6|

\awakeoption{highish} Top:bottom space = 3:8.  Sets
|\RM@headfootbalance=9|

\awakeoption{normalaltitude} Default.  Top:bottom space = 4:8.  Sets
|\RM@headfootbalance=12|

\awakeoption{lowish} Top:bottom space = 5:8.  Sets
|\RM@headfootbalance=15|

\awakeoption{low} Top:bottom space = 6:8.  Sets
|\RM@headfootbalance=18|

\awakeoption{lower} Top:bottom space = 7:8.  Sets
|\RM@headfootbalance=21|

\awakeoption{lowest} Top:bottom space = 8:8.  Sets
|\RM@headfootbalance=24|

\end{description}

\subsection{Changing the date format}


\begin{description}\shorty\squish

\awakeoption{usdate} Default.  Sets |\RM@nicedatefalse|, which causes
nothing to happen; the \comname{today} command is unmolested.

\awakeoption{ukdate} Sets |\RM@nicedatetrue|, which causes the
\comname{today} command to be re-defined to produce a date of the form
`4th April 1984'.  This is the setting I use as a default; I do this
by putting the \optname{ukdate} option in the local settings
\comname{ExecuteOptions} statement in my local configuration file.

\end{description}
%
% The option to load founts and set textwidth by fount size are now
% in rmplocal.cfg
%
% Printer types were:
% 0=fullbleed, 1=general, 2=dw520, 3=dw600, 4=pessimistic
%
% all printer options are now in rmplocal.cfg
%
% Printer types are:
% 0=fullbleed, 1=general, 2=pessimistic, 3=optimistic
% 10=dw300     11=dw500  12=dw600  (HP deskwriter inkjet series)
% 20=lj2       21=lj3    22=lj4    (HP laserjet laser printer series)
% 30=canonbjx bubblejet something  (Canon bubblejet inkjet series)
% 40+ whatever else comes up
%
% The figures for all these printers are guesses, except for the DW500
% and DW600: any data on printing margins for the printers above or
% other commonly-used printers would be gratefully received.  I need to
% know about printing limits at the top, bottom, left, and right for
% portrait and landscape modes, and whether the data is what the book
% says or what you measured (preferrably both, but anything'll help).
% If anyone really uses LaTeX with an A3 printer, do tell: it's
% something I've been wondering about.
%
%
\subsection{Dealing with the \packname{beton} package}
\label{optrmp:beton}

The code to let \rmpage work with \packname{beton} felt rather
complicated to write.  The thing about the \packname{beton} package
is that it changes \comname{baselineskip} to something non-standard.
\rmpage needs to know what \comname{baselineskip} is so that it can
set \comname{textheight}, but \packname{beton}'s changes aren't made
until the \comname{AtBeginDocument} hook is executed by \LaTeX, which
is after \rmpage has been loaded.  I had to steal code from
\packname{beton}~v1.3 to deal with this, which might cause problems if you
try to use \rmpage with other versions of \packname{beton}.


The result is that if you are using the \packname{beton} package
without passing it the \optname{standard-baselineskips} option, you should
specify either the \optname{beton} or the \optname{nobeton} option to
\rmpage: the first option uses \packname{beton}'s modified
\comname{baselineskip} to set \comname{textheight}; the second option
uses the standard \comname{baselineskip}, and can be ommitted if you
have specified the \optname{standard-baselineskips} option to
\packname{beton}.


The problem with the \optname{beton} option is that if you specify it,
\rmpage uses code stolen from the guts of \packname{beton}~version~1.3
to set the appropriate \comname{baselineskip}.  There is no guarantee
that this code will work with other versions of \packname{beton}.

\begin{description}\shorty\squish
\awakeoption{beton}  calculate a \comname{textheight}
based on the \packname{beton} package's \comname{baselineskip}

\awakeoption{nobeton} calculate a \comname{textheight}
based on the standard \comname{baselineskip}

Both these options set the |\RM@ifbeton| command to a number.  It's
played about with before and after here.  The final value of the
|\RM@ifbeton| command is given these meanings within \rmpage:
\end{description}

\noindent
\begin{tabularx}{\textwidth}{lX}

0&\packname{beton} package loaded and the \optname{beton} option
specified\\

1&The \packname{beton} package has been loaded with neither the
\optname{beton} nor the \optname{nobeton} option specified\\

2&\packname{beton} package loaded and the \optname{beton} option not specified\\

3& The \packname{beton} package not loaded with neither the
\optname{beton} nor the \optname{nobeton} option specified\\

4& The \packname{beton} package not loaded and the \optname{nobeton}
option specified.

\end{tabularx}




\section{From the configuration file}

The following options are all from the configuration file.  There's
nothing magical about this: they could all just as easily be in
\filename{rmpage.sty} at the point where the configuration file is
loaded.  But the idea is that you can change the configuration file,
but not \rmpage, and \rmpage works faster with fewer option.  So
comment out any of these options that you don't use very often (please
don't delete them: you never know when you might need them).

The distributed configuration file \filename{rmpgen.cfg} has no
options commented out; this means it's quite slow.  The distributed
configuration file \filename{rmplocal.gfc} does have some options
commented out---\rmpage works faster using this file.  \rmpage will
not use \filename{rmplocal.gfc} as a configuration file unless you
tell it to.  The most straightforward way to get \rmpage to use this faster
configuration file is to rename it \filename{rmplocal.cfg}.

The idea is that you don't change \filename{rmpgen.cfg} at all: it's
intended to be a standard configuration file that any document can use
to produce identical output on any system by saying
|\newcommand{\RMconfigfile}{rmpgen.cfg}| in the preamble before the
|\usepackage{rmpage}| command.

If you are short of disc space, you could delete
\filename{rmpgen.cfg}, but you might find you have to re-install it
one day to process a file that requires it.

\filename{rmplocal.cfg} is intended to be changed by anyone.  Please
read the comments in the file first, add a comment at the start of the
file to identify it as yours, and a note to the same effect in the
optional argument of the \comname{ProvidesFile}: life can get very
confused otherwise.  Don't add or delete anything except comment
characters between the \comname{ProvidesFile} command and the line
\texttt{LOCAL CODE BELOW HERE PLEASE}.  Make sensible changes below
the line \texttt{LOCAL CODE BELOW HERE PLEASE}; read the comments in
the configuration file and \filename{rmpage.dtx}, and use the commands
I use for doing things.  If you do this, your code should work
perfectly with future versions of \rmpage.

In the list below, options that look like this: {\asleepfount
obscurefunction}, are commented out in \filename{rmplocal.gfc}, whilst
options that look like this: {\awakefount usefulfunction}, are not
commented out.


\subsection{Other paper sizes}
\label{optcfg:papersizes}

There are some notes on paper sizes in section~\ref{optrmp:paper-sizes}
on page~\pageref{optrmp:paper-sizes}.  I suspect that the larger sizes
and untrimmed sizes will be useless, but it seemed churlish to leave
them out.


\begin{description}\shorty\squish
\asleepoption{undefinedpaper} Sets |\RM@papertype| to 0 -- does
nothing to \comname{paperheight} or \comname{paperwidth}.  This paper
type has no purpose in life, yet.

\asleepoption{letter4paper} Sets |\RM@papertype| to 9; this paper size
is an unholy bodge with the width of A4 and the height of US
letter.  Documents typeset with this paper size will fit on A4 and US
letter paper, and look terrible on both.  Size is 210mm by 8.5in.

\asleepoption{a0paper} Sets |\RM@papertype| to 10
-- 1189mm by 841mm

\asleepoption{a1paper} Sets |\RM@papertype| to 11
-- 841mm by 594mm

\asleepoption{a2paper} Sets |\RM@papertype| to 12
-- 594mm by 420mm

\awakeoption{a3paper} Sets |\RM@papertype| to 13
-- 420mm by 297mm


\awakeoption{a6paper} Sets |\RM@papertype| to 16
-- 148mm by 105mm

\asleepoption{a7paper} Sets |\RM@papertype| to 17
-- 105mm by 74mm

\asleepoption{a8paper} Sets |\RM@papertype| to 18
-- 74mm by 52mm

\asleepoption{a9paper} Sets |\RM@papertype| to 19
-- 52mm by 37mm

\asleepoption{a10paper} Sets |\RM@papertype| to 20
-- 37mm by 26mm

\asleepoption{b0paper} Sets |\RM@papertype| to 30
-- 1414mm by 1000mm

\asleepoption{b1paper} Sets |\RM@papertype| to 31
-- 1000mm by 707mm

\asleepoption{b2paper} Sets |\RM@papertype| to 32
-- 707mm by 500mm

\asleepoption{b3paper} Sets |\RM@papertype| to 33
-- 500mm by 353mm

\awakeoption{b4paper} Sets |\RM@papertype| to 34
-- 353mm by 250mm


\awakeoption{b6paper} Sets |\RM@papertype| to 36
-- 176mm by 125mm

\awakeoption{b7paper} Sets |\RM@papertype| to 37
-- 125mm by 88mm

\asleepoption{b8paper} Sets |\RM@papertype| to 38
-- 88mm by 62mm

\asleepoption{b9paper} Sets |\RM@papertype| to 39
-- 62mm by 44mm

\asleepoption{b10paper} Sets |\RM@papertype| to 40
-- 44mm by 31mm

\asleepoption{c0paper} Sets |\RM@papertype| to 50
-- 1297mm by 917mm

\asleepoption{c1paper} Sets |\RM@papertype| to 51
-- 917mm by 648mm

\asleepoption{c2paper} Sets |\RM@papertype| to 52
-- 648mm by 458mm

\asleepoption{c3paper} Sets |\RM@papertype| to 53
-- 458mm by 324mm

\asleepoption{c4paper} Sets |\RM@papertype| to 54
-- 324mm by 229mm

\awakeoption{c5paper} Sets |\RM@papertype| to 55
-- 229mm by 162mm

\asleepoption{c7paper} Sets |\RM@papertype| to 57
-- 114mm by 81mm

\asleepoption{c7/6paper} Sets |\RM@papertype| to 59
-- 162mm by 81mm

\asleepoption{bspopseedenvelopepaper} Sets |\RM@papertype| to 60
-- 152mm by 102mm. BS4264 UK post office preferred envelope: seed packets,
wage slips, general packaging.  The name is one I invented.

\asleepoption{bspopnonisoenvelopepaper} Sets |\RM@papertype| to 61 --
229mm by 102mm.  BS4264 UK post office preferred envelope: gen
commercial, non iso sizes.  The name is one I invented.

\asleepoption{bsbrochureenvelopepaper} Sets |\RM@papertype| to 62 --
254mm by 178mm.  BS4264 envelope; bulky A5, catalogues, brochures.
  The name is one I invented.

\asleepoption{bslegalenvelopepaper} Sets |\RM@papertype| to 63 --
270mm by 216mm.\newline%Horrid bodge
  BS4264 envelope; legal docs, catalogues, photos.  The
name is one I invented.

\asleepoption{bslargelegalenvelopepaper} Sets |\RM@papertype| to 64 --
305mm by 127mm.  BS4264 envelope; insurance policies, legal docs.  The
name is one I invented.

\asleepoption{bscalendarenvelopepaper} Sets |\RM@papertype| to 65 --
381mm by 254mm.  BS4264 envelope; bulky docs, calendars.  The name is
one I invented.

\awakeoption{foolscapfoliopaper} Sets |\RM@papertype| to 70
-- 13.5in by 8.5in

\asleepoption{foolscappaper} Sets |\RM@papertype| to 70
-- 13.5in by 8.5in

\asleepoption{foolscapquartopaper} Sets |\RM@papertype| to 71
-- 8.5in by 6.75in

\asleepoption{foolscapoctavopaper} Sets |\RM@papertype| to 72
-- 6.75in by 4.25in

\asleepoption{crownfoliopaper} Sets |\RM@papertype| to 73
-- 15in by 10in

\asleepoption{crownquartopaper} Sets |\RM@papertype| to 74
-- 10in by 7.5in

\asleepoption{crownoctavopaper} Sets |\RM@papertype| to 75
-- 7.5in by 5in

\asleepoption{royalfoliopaper} Sets |\RM@papertype| to 76
-- 20in by 12.5in

\asleepoption{royalquartopaper} Sets |\RM@papertype| to 77
-- 12.5in by 10in

\asleepoption{royaloctavopaper} Sets |\RM@papertype| to 78
-- 10in by 6.25in

\asleepoption{imperialfoliopaper} Sets |\RM@papertype| to 79
-- 22in by 15.5in

\asleepoption{imperialquartopaper} Sets |\RM@papertype| to 80
-- 15in by 11in

\asleepoption{imperialoctavopaper} Sets |\RM@papertype| to 81
-- 11in by 7.5in

\asleepoption{largecrownoctavopaper} Sets |\RM@papertype| to 82
-- 8in by 5.25in

\asleepoption{demyoquartopaper} Sets |\RM@papertype| to 83
-- 11.25in by 8.75in

\asleepoption{demyoctavopaper} Sets |\RM@papertype| to 84
-- 8.75in by 5.625in

\asleepoption{mediumquartopaper} Sets |\RM@papertype| to 85
-- 12in by 9.5in

\asleepoption{mediumoctavopaper} Sets |\RM@papertype| to 86
-- 9.5in by 6in

\asleepoption{ra0paper} Sets |\RM@papertype| to 90
-- 1270mm by 960mm

\asleepoption{ra1paper} Sets |\RM@papertype| to 91
-- 1270mm by 960mm

\asleepoption{ra2paper} Sets |\RM@papertype| to 92
-- 1270mm by 960mm

\asleepoption{sra0paper} Sets |\RM@papertype| to 93
-- 1280mm by 900mm

\asleepoption{sra1paper} Sets |\RM@papertype| to 94
-- 900mm by 840mm

\asleepoption{sra2paper} Sets |\RM@papertype| to 95
-- 640mm by 450mm

\asleepoption{metricdoublecrownpaper} Sets |\RM@papertype| to 96
-- 770mm by 505mm

\asleepoption{metricquadcrownpaper} Sets |\RM@papertype| to 97
-- 1010mm by 770mm

\asleepoption{metriclargequadcrownpaper} Sets |\RM@papertype| to 98
-- 1060mm by 820mm

\asleepoption{metricquaddemypaper} Sets |\RM@papertype| to 99
-- 1030mm by 890mm

\asleepoption{metricsmallquadroyalpaper} Sets |\RM@papertype| to 100
--\newline%Horrid bodge
1270mm by 960mm

\end{description}

The long paper sizes are described in detail in
section~\ref{optrmp:paper-sizes} on page~\pageref{optrmp:paper-sizes}.

\begin{description}\shorty\squish

\asleepoption{notlongpaper} Sets  |\RM@longpapertypelong| to 0; not
long---the default.

\asleepoption{7/8longpaper} Sets  |\RM@longpapertypelong| to 1; 7/8
long.  The selected paper size has its longest dimension multiplied by
7/8.

\awakeoption{3/4longpaper}Sets  |\RM@longpapertypelong| to 2; 3/4
long.  The selected paper size has its longest dimension multiplied by
3/4.

\awakeoption{2/3longpaper}Sets  |\RM@longpapertypelong| to 3; 2/3
long.  The selected paper size has its longest dimension multiplied by
2/3.

\asleepoption{5/8longpaper}Sets  |\RM@longpapertypelong| to 4; 5/8
long.  The selected paper size has its longest dimension multiplied by
2/3.

\asleepoption{1/2longpaper}Sets |\RM@longpapertypelong| to 5; 1/2 long.
The selected paper size has its longest dimension multiplied by 1/2.
This is slightly different for asking for the next size down in an ISO
series; these long sizes are not rounded to the nearest millimetre, as
are standard ISO paper sizes, and code which sets things up for
particular printer/paper combinations does not recognize 1/2 long A3
as A4 (for example).

\asleepoption{3/8longpaper}Sets  |\RM@longpapertypelong| to 6; 3/8
long.  The selected paper size has its longest dimension multiplied by
3/8.

\awakeoption{1/3longpaper}Sets  |\RM@longpapertypelong| to 7; 1/3
long.  The selected paper size has its longest dimension multiplied by
1/3.

\awakeoption{1/4longpaper}Sets  |\RM@longpapertypelong| to 8; 1/4
long.  The selected paper size has its longest dimension multiplied by
1/4.

\asleepoption{1/8longpaper}Sets  |\RM@longpapertypelong| to 9; 1/8
long.  The selected paper size has its longest dimension multiplied by
1/8.

\end{description}
%%
%%
\subsection{Marginal paragraph options}
\label{optcfg:mpars}

The width of a marginal paragraph is set to the space left in the
appropriate margin, taking into account all the limits.  \rmpage
thinks the appropriate margin is this: in the case of multi-column
printing, the smallest margin; in the case of one sided printing,
normal marginal paragraph placement, in the outside margin; in the
case of one sided printing, reverse marginal paragraph placement, in
the inside margin; in the case of two sided printing, normal marginal
paragraph placement, in the outside margin; and in the case of two
sided printing, reverse marginal pragraph placement, in the inside
margin.

The size is calculated on this basis: the standard \LaTeX\ length
\comname{margin\-par\-sep} gives the space between the text body and the
marginal paragraph.  The new length \comname{RM@mparclearance} gives
the minimum space between the outside edge of the marginal paragraph
and the edge of the paper (subject to the additional restrictions of
\comname{RM@minrightclearance} and \comname{RM@minleftclearance} (but
not \comname{RM@mininsidemargin} or \comname{RM@minoutsidemargin};
these apply to the text body only).  Within these limits,
\comname{marginparwidth} cannot be set to greater than the length
\comname{RM@maxmparwidth}.

This way of setting marginal paragraphs is derived from the standard
\LaTeXe\ method, which uses 2\units{in} as the largest allowed size,
and 0.4\units{in} as the minimum gap to the edge of the paper.
\rmpage's equivalent parameters

You can scale the size of \comname{marginparsep},
\comname{RM@mparclearance}, and \comname{RM@max\-mpar\-width} using the
\optname{mparsep}, \optname{mparclearance}, and
\optname{maxmparwidth} option sets.  Look at section~\ref{gen:mpars} on
page~\pageref{gen:mpars} for more on marginal paragraphs.

If you think that the base value of any of these lengths is too small,
you can do something about it.  With \comname{marginparsep}, you could
use the \comname{setlength} command to set it to a different value
before loading \rmpage.  For example,
\begin{verbatim}
\setlength{\marginparsep}{2\marginparsep}
\usepackage{rmpage}
\end{verbatim}

Because the other two parameters are given their initial values in
\rmpage, this technique won't work.  The initial values of
\comname{RM@mparclearance} and \comname{RM@max\-mpar\-width} are
calculated as a certain fraction of \comname{paperwidth}; the initial
value of the appropriate parameter is doubled if you specify the
\optname{largebasemparclear} or \optname{largebasemaxmparwidth}
options.  You can set either of these parameters in the configuration
file---the values -666\units{pt} and -667\units{pt} are reserved by
\rmpage as flag values; any positive length that's not too long is
okay.  Read the source and consider setting these parameters on a
class-by-class basis if you do need to change them.


%%
%% marginparsep is scaled by option
The \comname{mparsep} options scale the \comname{marginparsep} length
using \comname{RM@scale\-by\-option}.  See section~\ref{htw:headfootmpar}
on page~\pageref{htw:headfootmpar} for the details.

\begin{description}\shorty\squish
\awakeoption{leastmparsep} Sets \comname{RM@mparsepoption} to 3
\awakeoption{lessmparsep} Sets \comname{RM@mparsepoption} to 6
\awakeoption{lessishmparsep} Sets \comname{RM@mparsepoption} to 9
\awakeoption{normalmparsep} Default.  Sets \comname{RM@mparsepoption} to 12
\awakeoption{moreishmparsep} Sets \comname{RM@mparsepoption} to 15
\awakeoption{moremparsep} Sets \comname{RM@mparsepoption} to 18
\awakeoption{mostmparsep} Sets \comname{RM@mparsepoption} to 21
\end{description}
%%

The \optname{...basemparclear} options need to be executed after
\comname{paperwidth} has been set.  Easily done with
\comname{ProcessOptions} rather than \comname{ProcessOptions*}, and
the papersize setting options declared above rather than below.  The
\comname{small\-base\-mpar\-clear} value is set after option processing if
no other value has been set.  If \comname{RM@mpar\-clear\-ance} is
-666\unit{pt}, the normalbasemparclear value is set; if it's
-667\unit{pt}, the largebaselinemparclear value is set.  The larger
value is double the smaller value.  This setting is done  just
after the \comname{RM@PrinterPaperSettings} hook is executed, which is
well after \comname{paperwidth} is set.  This value is scaled by
option (using the \comname{RM@mparclearoption} passed to the
\comname{RM@scalebyoption} command) just before it's used, so one can
use the \comname{RM@BeforeWidthSetting} hook to change things.

\begin{description}\shorty\squish
\asleepoption{normalbasemparclear} Default.  Sets \comname{RM@mparclearance}
to -666pt.
\asleepoption{largebasemparclear}  Sets \comname{RM@mparclearance} to -667pt
%%
\asleepoption{normalbasemaxmparwidth}  Default.   Sets \comname{RM@maxmparwidth} to -666pt
\asleepoption{largebasemaxmparwidth}  Sets \comname{RM@maxmparwidth} to -667pt
\end{description}

The gap between the edge of the paper and the edge of a marginal
paragraph is 0.4\units{in} (10.16\units{mm}) with \LaTeX's standard
classes.  \rmpage changes this by introducing a new parameter,
\comname{RM@mparclearance}, which is calculated as a fraction of
\comname{paperwidth}.  Normal \comname{RM@mparclearance} with A4
portrait paper is 9.88\units{mm}; 0.4\unit{in} with US letter paper).

Touch options for \comname{RM@mparclearance} and
\comname{RM@maxmparwidth} have been added now.

\begin{description}\shorty\squish
\awakeoption{leastmparclearance}  Sets \comname{RM@mparclearoption} to 3
\awakeoption{lessmparclearance}  Sets \comname{RM@mparclearoption} to 6
\awakeoption{lessishmparclearance}  Sets \comname{RM@mparclearoption} to 9
\awakeoption{normalmparclearance}  Default.  Sets \comname{RM@mparclearoption} to 12
\awakeoption{moreishmparclearance}  Sets \comname{RM@mparclearoption} to 15
\awakeoption{moremparclearance}  Sets \comname{RM@mparclearoption} to 18
\awakeoption{mostmparclearance}  Sets \comname{RM@mparclearoption} to 21
\end{description}

\comname{RM@maxmparwidth} is set as a fraction of
\comname{paperwidth}, such that with portrait US letter paper, you get
2\units{in} as standard, just like the standard classes.  It's scaled
by option (see section~\ref{htw:headfootmpar} on
page~\pageref{htw:headfootmpar} for the details).  If you want to
change the default base value of \comname{RM@maxmparwidth}, the
\comname{RM@BeforeWidthSetting} hook is an ideal place to do it.

\begin{description}\shorty\squish
\awakeoption{leastmaxmparwidth}  Sets \comname{RM@maxmparwidthoption} to 3
\awakeoption{lessmaxmparwidth}  Sets \comname{RM@maxmparwidthoption} to 6
\awakeoption{lessishmaxmparwidth}  Sets \comname{RM@maxmparwidthoption} to 9
\awakeoption{normalmaxmparwidth} Default.  Sets \comname{RM@maxmparwidthoption} to 12
\awakeoption{moreishmaxmparwidth}  Sets \comname{RM@maxmparwidthoption} to 15
\awakeoption{moremaxmparwidth}  Sets \comname{RM@maxmparwidthoption} to 18
\awakeoption{mostmaxmparwidth}  Sets \comname{RM@maxmparwidthoption} to 21
\end{description}
%%
%%
\subsection{Touch options}
\label{optcfg:touchoptions}

All the \optname{touch} options add or subtract one from a counter
that is used to control the size of a page layout parameter.  The
effect of this is to give the layout parameter a size in between the
`main' sizes.  That is, if you ask for \optname{wide} and
\optname{touchwider}, \comname{textwidth} is set to a value 1/3 of
the way (in a geometrical sequence) from \optname{wide} to
\optname{wider}.  This results in even step sizes: \optname{wide},
\optname{touchwider}, and \optname{t@uchwider} give a width the same
as \optname{wider} and \optname{touchnarrower}.

The \optname{touch} options are intended to be used in documents; the
\optname{t@uch} options are intended to be used in class files.  The
reason is this: a class designer can develop a suitable layout by
passing options to \rmpage.  When this is done, the options passed to
\rmpage can be passed using the \comname{PassOptionsToPackage}
command in a class file.  Any \optname{touch} options should be
turned into \optname{t@uch} options, so that the controlled parameter
can be incremented or decremented by a \optname{touch} option in a
document.  Whether this is a good thing is unclear, but it's
certainly deliberate.

Note that all the \optname{touch} options need to be executed after
their corresponding `straight' options.  To ensure this, none of the
\optname{touch} can be allowed in an \comname{ExecuteOptions}
statement.  The \comname{RM@notinexecuteoptions} is used in each of
these option declarations: it produces an error message if used before
the \comname{RM@donewithoptions} flag is set true, which is
immediately before the \comname{ProcessOptions} statement.

You can find out more about the affected layout parameters by looking
at the documentation for the main options corresponding to the
\optname{touch} options listed here.

\begin{description}\shorty\squish
\asleepoption{t@uchlonger}  Adds 1  to \comname{RM@lengthoption}
\asleepoption{t@uchshorter}  Adds -1  to \comname{RM@lengthoption}
\awakeoption{touchlonger}  Adds  1 to \comname{RM@lengthoption}
\awakeoption{touchshorter}  Adds   -1 to \comname{RM@lengthoption}.
See section~\ref{optrmp:textheight}.
%%
\awakeoption{touchmorecolsep}  Adds  1  to \comname{RM@columnsepoption}
\awakeoption{touchlesscolsep}  Adds  -1 to \comname{RM@columnsepoption}
\asleepoption{t@uchmorecolsep}  Adds  1 to \comname{RM@columnsepoption}
\asleepoption{t@uchlesscolsep}  Adds  -1 to \comname{RM@columnsepoption}.
See section~\ref{optrmp:columnsep}.
%%
\awakeoption{touchmoremparsep}  Adds   1 to \comname{RM@mparsepoption}
\awakeoption{touchlessmparsep}  Adds  -1 to \comname{RM@mparsepoption}
\asleepoption{t@uchmoremparsep}  Adds  1 to \comname{RM@mparsepoption}
\asleepoption{t@uchlessmparsep}  Adds  -1 to \comname{RM@mparsepoption}.
See section~\ref{optcfg:mpars}.
%%
\awakeoption{touchmorefootskip}  Adds  1 to \comname{RM@footskipoption}
\awakeoption{touchlessfootskip}  Adds  -1 to \comname{RM@footskipoption}
\asleepoption{t@uchmorefootskip}  Adds 1  to \comname{RM@footskipoption}
\asleepoption{t@uchlessfootskip}  Adds  -1 to \comname{RM@footskipoption}.
See section~\ref{optrmp:hfonoff}.
%%
\awakeoption{touchmoreheadsep}  Adds  1 to \comname{RM@headsepoption}
\awakeoption{touchlessheadsep}  Adds  -1 to \comname{RM@headsepoption}
\asleepoption{t@uchmoreheadsep}  Adds  1 to \comname{RM@headsepoption}
\asleepoption{t@uchlessheadsep}  Adds  -1 to \comname{RM@headsepoption}.
See section~\ref{optrmp:hfonoff}.
%%
\awakeoption{t@uchwider}  Adds  1 to \comname{RM@widthoption}
\asleepoption{t@uchnarrower}  Adds  -1 to \comname{RM@widthoption}
\awakeoption{touchwider}  Adds  1 to \comname{RM@widthoption}
\awakeoption{touchnarrower}  Adds  -1 to \comname{RM@widthoption}.
See section~\ref{optrmp:textwidth}.
%%
\asleepoption{t@uchmoreoffset}  Adds  1 to \comname{RM@offsetoption}
\asleepoption{t@uchlessoffset}  Adds  -1 to \comname{RM@offsetoption}
\awakeoption{touchmoreoffset}  Adds  1 to \comname{RM@offsetoption}
\awakeoption{touchlessoffset}  Adds  -1 to \comname{RM@offsetoption}.
See section~\ref{optrmp:margins}.
%%
\asleepoption{t@uchhigher}  Adds -1  to \comname{RM@headfootbalance}
\asleepoption{t@uchlower}  Adds  1 to \comname{RM@headfootbalance}
\awakeoption{touchhigher}  Adds  -1 to \comname{RM@headfootbalance}
\awakeoption{touchlower}  Adds  1 to \comname{RM@headfootbalance}.
See section~\ref{optrmp:altitude}.
%%
\asleepoption{t@uchlessmparclearance}  Adds -1  to \comname{RM@mparclearoption}
\asleepoption{t@uchmoremparclearance}  Adds 1   to \comname{RM@mparclearoption}
\asleepoption{touchlessmparclearance}  Adds -1  to \comname{RM@mparclearoption}
\asleepoption{touchmoremparclearance}  Adds 1   to \comname{RM@mparclearoption}.
\newline% Horrid bodge
See section~\ref{optcfg:mpars}.
%%
\asleepoption{t@uchlessmaxmparwidth}  Adds -1  to \comname{RM@maxmparwidthoption}
\asleepoption{t@uchmoremaxmparwidth}  Adds  1  to \comname{RM@maxmparwidthoption}
\asleepoption{touchlessmaxmparwidth}  Adds -1  to \comname{RM@maxmparwidthoption}
\asleepoption{touchmoremaxmparwidth}  Adds  1  to \comname{RM@maxmparwidthoption}.
\newline% Horrid bodge
See section~\ref{optcfg:mpars}.
%%
\end{description}

\subsection{More length options}
\label{optcfg:odd-lengths}

These options need to be executed after the \optname{touchlength}
options; otherwise, they'd be in \filename{rmpage.sty}.


Most of the \comname{textheight} setting options are in \rmpage
proper; see \ref{optrmp:textheight} on page~\pageref{optrmp:textheight}.

\begin{description}\shorty\squish
\awakeoption{fulllength} Sets |\RM@lengthoption=30|.  Makes
\comname{textheight} as long as possible, taking into account the
various restrictions and the need to keep \comname{textheight} to an
integer number times \comname{baselineskip} plus \comname{topskip}.

\awakeoption{stdlength} Sets |\RM@lengthoption=0|.  Makes
\comname{textheight} the size it would be if you were using the
standard \LaTeX\ classes.  I'm not sure I've checked everything that
needs to be checked to ensure that this option always does what you'd
expect, but I think I have.  Note that \rmpage still takes notice of
footers and headers when you use this option, so you can get the
standard \comname{textheight} with different vertical positioning of
the text body.

\end{description}

\subsection{Number of columns}

If you are typesetting text in more than one column, \rmpage needs to
know so that it can set \comname{textwidth} appropriately.

The standard \LaTeX\ options \optname{onecolumn} and \optname{twocolumn}
are recognized by \rmpage.

The \optname{...columnwidth} options tell \rmpage that you will be
typesetting your text in that number of columns, but the only effect
of the options is to change the \comname{textwidth} calculation.  If
you want to change the number of columns that your text is set in, you
can use a package like \packname{multicol}.

\begin{ikkystuff}
These options affect only the character-based \comname{textwidth}
calculation---for any given width option (\optname{normalwidth},
\optname{narrower}, or whatever) one text column is allowed to be a
certain number of characters wide.  Use the \optname{...columnwidth}
options to tell \rmpage how many columns wide your text is, so it can
calculate \comname{textwidth} appropriately.

\rmpage makes the character-based \comname{textwidth} guess equal to a
certain number of characters times the number of columns, plus
\comname{columnsep} times one less than the number of columns.
\end{ikkystuff}

If your text is something like a single large table (for example, a
timetable), it might be more appropriate to use the
\optname{paperwidthset} option to set \comname{textwidth} using the
paper-based \comname{textwidth} only.

\begin{description}\shorty\squish

\awakeoption{onecolumnwidth} Default.
 Defines \comname{RM@textcols} to 1
 
\awakeoption{twocolumnwidth}  Defines \comname{RM@textcols} to 2

\awakeoption{threecolumnwidth}  Defines \comname{RM@textcols} to 3

\asleepoption{fourcolumnwidth}  Defines \comname{RM@textcols} to 4

\asleepoption{fivecolumnwidth}  Defines \comname{RM@textcols} to 5

\asleepoption{sixcolumnwidth}  Defines \comname{RM@textcols} to 6

\asleepoption{sevencolumnwidth}  Defines \comname{RM@textcols} to 7

\asleepoption{eightcolumnwidth}  Defines \comname{RM@textcols} to 8

\asleepoption{ninecolumnwidth}  Defines \comname{RM@textcols} to 9

\asleepoption{tencolumnwidth}  \packname{multicol}'s limit.
Defines \comname{RM@textcols} to 10

\end{description}
%%
%%
\subsection{Loading founts}

All the options below change how \comname{textwidth} is set, and the
options with \optname{load} in their name also call one of the
standard \packname{PSNFSS} packages to load the fount (aside from
Lucida Casual and Concrete, which aren't \packname{PSFNSS} founts,
and Courier, which is handled anomalously).

\begin{ikkystuff}
The only Lucida typeface I have is Lucida Casual, so that's the only
Lucida typeface that \rmpage deals with explicitly.  If you do use
others, you can have an appropriate \comname{textwidth} set using the
\optname{thisfountwidth} option.

Given that the \packname{PSFNSS} distribution has support for all the
Lucida founts, I could be persuaded to include explicit support for
them in \rmpage if anyone's interested.
\end{ikkystuff}

Remember that \comname{textwidth} is usually set to be a certain
number of character wide.  Well, not all founts have the same number
of characters per inch.  If you say \optname{timeswidth} to \rmpage,
it will calculate a \comname{textwidth} based on the measured average
width of one character in  Times of the specified size.

This means that while the standard \LaTeX\ classes would give you a
\comname{text\-width} that is far too wide for the Times fount (which is
generally narrower than Computer Modern Roman), \rmpage will give you
a \comname{textwidth} that is pretty much the same number of
characters across, which means you retain good legibility (as well as
similar line and page breaks).

If you say \optname{loadtimes}, \rmpage changes its \comname{textwidth}
calculation, loads the \packname{PSNFSS} package that loads the Times
fount family, and (very important, this is) changes the typesetting
parameters to similar values to the ones suggested by Karl Berry, the
chap who wrote the \packname{fontinst} program that generated the
virtual founts used to typeset the \packname{PSNFSS} founts.

If you specify a looseness option yourself---see
section~\ref{optrmp:tightness} on page~\pageref{optrmp:tightness}---it will
over-ride the standard looseness set by a \optname{loadfount} option.
The \optname{load\-concrete} option requests standard tight typesetting,
and is anyway not recommended: if you want to use the concrete founts,
try the \packname{beton} package, which does a very good job of
setting up \LaTeX\ to use these founts.  \rmpage can work happily with
\packname{beton}: see section~\ref{optrmp:beton} on
page~\pageref{optrmp:beton}.

You can tell \rmpage to set the \comname{textwidth} based on the width
of a fount it doesn't know about with the \optname{thisfountwidth}
option.  If you use this option, \rmpage will calculate a
\comname{textwidth} based on the size of the fount that is current
when \rmpage is loaded.  So if you want a \comname{textwidth} based on, say,
Grunge Update (family name fgr on my computer), you could say:
\begin{verbatim}
\documentclass[thisfountwidth,12pt]{article}
\renewcommand{\rmdefault}{fgr}
\rmfamily
\usepackage{rmpage}

\begin{document}

...

\end{verbatim}

\rmpage will tell you which fount it is working with, and the results
of its calculations. If you get confused by \LaTeX's fount selection
scheme, read the manual; it confuses me too.

The fount options work like this: each option sets the
\comname{RM@fountfamily} command to a particular value.  Any option
which sets the \comname{RM@loadfount} flag true forces code later on
in \rmpage to load the appropriate fount, most of them using one of
the standard \packname{PSNFSS} packages.  The fount loading code is
written specially for each fount; there's no easy way to add more
founts to the list that's already dealt with.  But you could add code to
the \comname{RM@AfterProcessOptions} hook if you want to do this; I
suggest that this code loads the fount, selects it, and defines
\comname{RM@fountfamily} to be 12, to make the width setting code
measure its width.  There are no hooks in \filename{rmpwnorm.pko} to
add this sort of thing.

\begin{ikkystuff}
The thing about using the \optname{loadfount} options is that the
standard \packname{PSNFSS} packages don't always set the default main
document typeface (the one you get when you ask for
\comname{rmfamily}) to be the fount you've asked for.  So if you say
\optname{loadhelvet}, you'll get a \comname{textwidth} based on the
width of the Helvetica typeface, but Helvetica is the fount you get
when you ask for \comname{sffamily}, which might not be what you want.

The thing to do is be sure which typeface will be the main document
face, and ask \rmpage to set the \comname{textwidth} accordingly.  You
might use the appropriate \optname{loadfount} option for this, or load
the founts you want with separate calls to the appropriate packages in
your document's preamble.
\end{ikkystuff}

If you don't have the file needed to load the requested fount
family, \rmpage complains.

Fount families are set like this:
\begin{verbatim}
0=cmr
1=avant garde 2=bookman 3=zapf chancery 4=helvetica
5=new century schoolbook 6=palatino 7=times 8=utopia
9=lucida casual 10=courier 11=concrete 12=this fount width
13=lucida casual dirty trick
\end{verbatim}

The dirty trick works like this: if you ask for
\optname{loadlucidacasual}, rather than \optname{loadluccasua}
(\comname{RM@fountfamily} 13), the fount loading code later on does a
|\RequirePackage{lucida-casual}|, and then sets
\comname{RM@fountfamily} to 9.  This lets Rowland use his own
\filename{.fd} files for Lucida Casual, and allows access to the
standard \filename{.fd} files.  The option to do this is in Rowland's
curious option section of the configuration file.


\begin{description}\shorty\squish

\asleepoption{cmrwidth} Computer modern roman; redundant
options.\newline% horrid bodge
Defines \comname{RM@fountfamily} to 0

\asleepoption{loadcmr}  Loads nothing; does whinge a little.

\awakeoption{avantwidth}  Avant Garde.  Defines \comname{RM@fountfamily} to 1

\awakeoption{loadavant}  Requires the \packname{avant} package.

\awakeoption{bookmanwidth} Bookman.  Defines \comname{RM@fountfamily} to 2

\awakeoption{loadbookman}  Requires the \packname{bookman} package.

\awakeoption{chancerywidth} Zapf Chancery.  Defines \comname{RM@fountfamily} to 3

\awakeoption{loadchancery}  Requires the \packname{chancery} package.

\awakeoption{helvetwidth} Helvetica.  Defines \comname{RM@fountfamily} to 4

\awakeoption{loadhelvet}  Requires the \packname{helvet} package.

\awakeoption{newcentwidth} New Century Schoolbook.  Defines \comname{RM@fountfamily} to 5

\awakeoption{loadnewcent}  Requires the \packname{newcent} package.

\awakeoption{palatinowidth} Palatino.  Defines \comname{RM@fountfamily} to 6

\awakeoption{loadpalatino}  Requires the \packname{palatino} package.

\awakeoption{timeswidth} Times.  Defines \comname{RM@fountfamily} to 7

\awakeoption{loadtimes}  Requires the \packname{times} package.

\awakeoption{utopiawidth} Utopia.  Defines \comname{RM@fountfamily} to 8

\awakeoption{loadutopia}  Requires the \packname{utopia} package.

\awakeoption{lucasualwidth} Lucida casual.  Defines \comname{RM@fountfamily} to 9

\awakeoption{loadlucasual}  Requires the \packname{lucasual} package.
%% lucida-casual option is now in rmplocal.cfg
%%

\awakeoption{courierwidth} Courier.  Defines \comname{RM@fountfamily} to 10

\asleepoption{loadcourier} This option makes the default roman fount
Courier.  I think this is ugly and crude: you might be better off
using the times package and \comname{ttfamily}

\awakeoption{concretewidth} Concrete.  Defines \comname{RM@fountfamily} to 11

\asleepoption{loadconcrete} This option loads the \packname{beton}
package and sets \comname{textwidth} for the Concrete Roman founts.

\awakeoption{thisfountwidth} Bases \comname{textwidth} on the current
fount.\newline% horrid bodge
Defines \comname{RM@fountfamily} to 12

%%
\end{description}

\subsection{Stuff for beton support}

\begin{description}\shorty\squish
\awakeoption{standard-baselineskips}  Passes this option to
\packname{beton} so \rmpage can detect whether this option's been
passed to \packname{beton}.
\asleepoption{oldstyle-equation-numbers}  Passes this option to
\packname{beton}.
\asleepoption{concrete-math}  Passes this option to
\packname{beton}.
\end{description}
%%
%%
\subsection{Other synonyms for some options}

Why oh why oh why do I have to make my package speak US English?
(mutter mumble grumble).  Yes, all right, it's how the convention's
worked out.  And I've put the original options to change the
\comname{today} command here too.  (It's like this: I once had a
\LaTeX~2.09 style file called \packname{nicedate} that changed the
\comname{today} command into something I liked.  Then I wrote \rmpage,
and included the \packname{nicedate} code, activated by the
\optname{nicedate} option.  Obviously, the \optname{nicedate} option
needed a complementary option, and the obvious name for this option
was \optname{nastydate}, which eventually got turned into
\optname{othernicedate}.  After I started turning \rmpage into
something for the rest of the world to look at (which it wasn't
originally), I added the \optname{ukdate} and \optname{usdate}
options.  But I still like \optname{nicedate}, so I've kept it.  So
there.)

\begin{description}\shorty\squish

\awakeoption{othernicedate}  The same as \optname{usdate}; does nothing

\awakeoption{nicedate}  The same as \optname{ukdate}; changes date
format.

\asleepoption{verbose} Synonym for chatty
\asleepoption{silent} Synonym for yorkshire
\asleepoption{errorshow} Synonym for  yorkshire
\asleepoption{warningshow} Synonym for   taciturn
\asleepoption{infoshow} Synonym for  chatty
\asleepoption{debugshow} Synonym for   garrulous


%%
%% And I object to this, I really do.  Why can't people learn to
%% spell properly?
%%

\asleepoption{center} The same as \optname{centre}

\asleepoption{notcenter} The same as \optname{notcentre}

\end{description}
%%
\subsection{Margin options}
\label{optcfg:margins}

\begin{description}\shorty\squish

\awakeoption{ringbinding} This option sets the minimum allowed inside
margin to be at least 15\units{mm} if you are printing in portrait
orientation.  It's in the config file because it must be executed
after the \optname{landscape} and \optname{portrait} options.  It does
nothing but warn you if you use it in landscape orientation.

Beware that this option takes no notice of long paper sizes at all,
and is likely to give iffy results if you combine it with them.  If
you have any thoughts about this option, please email me---I'm not
terribly happy with it.

\end{description}

\subsection{Printer options}
\label{optcfg:printers}

Each printer option must set these ten parameters:

\begin{description}\shorty\squish

\item[\comname{RM@printertype}]  A code number, defined below.  This
number is used by \rmpage to keep track of the printer used; you can
tell \rmpage to do things for certain printers and not for others.

\item[\comname{RM@ptrrportclear}] The non-printing margin on the
right-hand side in portrait orientation

\item[\comname{RM@ptrlportclear}] The non-printing margin  on the
left-hand side in portrait orientation

\item[\comname{RM@ptrtportclear}] The non-printing margin  at the
top in portrait orientation

\item[\comname{RM@ptrbportclear}] The non-printing margin  at the
bottom in portrait orientation

\item[\comname{RM@ptrrlandclear}] The non-printing margin  on the
right-hand side in landscape orientation

\item[\comname{RM@ptrllandclear}] The non-printing margin  on the
left-hand side in landscape orientation

\item[\comname{RM@ptrtlandclear}] The non-printing margin  at the
top in landscape orientation

\item[\comname{RM@ptrblandclear}] The non-printing margin  at the
bottom in landscape orientation

\item[\comname{RM@ptrpostol}] Nominally, the amount you expect the
position of the paper to vary.  The value of this command is added to
each of the \comname{ptr...clear} parameters before they are used.

\end{description}

Printer types are:
\par
\noindent
\begin{tabbing}
0=fullbleed \= 1=general \= 2=pessimistic \= 3=optimistic\\
10=dw300    \> 11=dw500  \> 12=dw600 \>  (HP deskwriter inkjet series)\\
20=lj2      \> 21=lj3    \> 22=lj4   \>  (HP laserjet laser printer series)\\
30=canonbjx  (Canon bubblejet flurble)  \\
40+ others   (whatever  comes up)\\
1000+   local printers    to avoid clashes\\
\end{tabbing}

The figures for all these printers are guesses, except for the DW500
and DW600: any data on printing margins for the printers above or
other commonly-used printers would be gratefully received.  I need to
know about printing limits at the top, bottom, left, and right for
portrait and landscape modes, and whether the data is what the book
says or what you measured (preferrably both, but anything'll help).
If anyone really uses \LaTeX\ with an A3 printer, do tell: it's
something I've been wondering about.

When I've got a better idea of what's going on, I'll define more
printer options.

\comname{RM@ptrpostol} generally set to 1\unit{mm} (paper sizes are to
$\pm2$\unit{mm}), except for our DW520 which I keep a careful eye on.

\begin{description}\shorty\squish

\awakeoption{fullbleedprinter}  Lets you print all the way to the
edge of the paper.
\begin{verbatim}
\def\RM@printertype{0}
\def\RM@ptrrportclear{0mm}  \def\RM@ptrrlandclear{0mm}
\def\RM@ptrlportclear{0mm}  \def\RM@ptrllandclear{0mm}
\def\RM@ptrtportclear{0mm}  \def\RM@ptrtlandclear{0mm}
\def\RM@ptrbportclear{0mm}  \def\RM@ptrblandclear{0mm}
\def\RM@ptrpostol{0mm}
\end{verbatim}

\awakeoption{generalprinter} Arbitrary settings that probably ensure
a layout inside the printing area on most A4 printers.
\begin{verbatim}
\def\RM@printertype{1}
\def\RM@ptrrportclear{8mm}    \def\RM@ptrrlandclear{8mm}
\def\RM@ptrlportclear{8mm}    \def\RM@ptrllandclear{8mm}
\def\RM@ptrtportclear{8mm}    \def\RM@ptrtlandclear{8mm}
\def\RM@ptrbportclear{15mm}   \def\RM@ptrblandclear{15mm}
\def\RM@ptrpostol{1mm}
\end{verbatim}

\asleepoption{pessimisticprinter} This uses the worst limits I can remember
meeting, so it force documents inside the printing area on any printer.
\begin{verbatim}
\def\RM@printertype{2}
\def\RM@ptrrportclear{10mm}  \def\RM@ptrrlandclear{19mm}
\def\RM@ptrlportclear{10mm}  \def\RM@ptrllandclear{10mm}
\def\RM@ptrtportclear{10mm}  \def\RM@ptrtlandclear{10mm}
\def\RM@ptrbportclear{19mm}  \def\RM@ptrblandclear{10mm}
\def\RM@ptrpostol{1mm}
\end{verbatim}

\asleepoption{optimisticprinter} This uses the best limits I'd expect
from a real printer.
\begin{verbatim}
\def\RM@printertype{3}
\def\RM@ptrrportclear{3mm}  \def\RM@ptrrlandclear{3mm}
\def\RM@ptrlportclear{3mm}  \def\RM@ptrllandclear{3mm}
\def\RM@ptrtportclear{3mm}  \def\RM@ptrtlandclear{3mm}
\def\RM@ptrbportclear{3mm}  \def\RM@ptrblandclear{3mm}
\def\RM@ptrpostol{0.5mm}
\end{verbatim}

\asleepoption{dw300printer}  A guess.
\begin{verbatim}
\def\RM@printertype{11}
\def\RM@ptrrportclear{6mm}  \def\RM@ptrrlandclear{15mm}
\def\RM@ptrlportclear{6mm}  \def\RM@ptrllandclear{7mm}
\def\RM@ptrtportclear{7mm}  \def\RM@ptrtlandclear{6mm}
\def\RM@ptrbportclear{15mm} \def\RM@ptrblandclear{6mm}
\def\RM@ptrpostol{1mm}
\end{verbatim}

\asleepoption{dw500printer} Hewlett-Packard's specification for its
DeskWriter and DeskJet 500/510/520/540 printers.
\begin{verbatim}
\def\RM@printertype{11}
\def\RM@ptrrportclear{6mm}  \def\RM@ptrrlandclear{15mm}
\def\RM@ptrlportclear{6mm}  \def\RM@ptrllandclear{7mm}
\def\RM@ptrtportclear{7mm}  \def\RM@ptrtlandclear{6mm}
\def\RM@ptrbportclear{15mm} \def\RM@ptrblandclear{6mm}
\def\RM@ptrpostol{1mm}
\end{verbatim}

\awakeoption{dw600printer}  Measured from a particular HP 600 series
inkjet printer, with a bit added.
\begin{verbatim}
\def\RM@printertype{12}
\def\RM@ptrrportclear{5mm}  \def\RM@ptrrlandclear{15mm}
\def\RM@ptrlportclear{5mm}  \def\RM@ptrllandclear{2mm}
\def\RM@ptrtportclear{2mm}  \def\RM@ptrtlandclear{5mm}
\def\RM@ptrbportclear{15mm} \def\RM@ptrblandclear{5mm}
\def\RM@ptrpostol{1mm}
\end{verbatim}
 
\asleepoption{lj2printer} an arbitrary guess
\begin{verbatim}
\def\RM@printertype{20}
\def\RM@ptrrportclear{7mm}  \def\RM@ptrrlandclear{7mm}
\def\RM@ptrlportclear{7mm}  \def\RM@ptrllandclear{7mm}
\def\RM@ptrtportclear{7mm}  \def\RM@ptrtlandclear{7mm}
\def\RM@ptrbportclear{7mm}  \def\RM@ptrblandclear{7mm}
\def\RM@ptrpostol{1mm}
\end{verbatim}

\asleepoption{lj3printer} an arbitrary guess
\begin{verbatim}
\def\RM@printertype{21}
\def\RM@ptrrportclear{6mm}  \def\RM@ptrrlandclear{6mm}
\def\RM@ptrlportclear{6mm}  \def\RM@ptrllandclear{6mm}
\def\RM@ptrtportclear{6mm}  \def\RM@ptrtlandclear{6mm}
\def\RM@ptrbportclear{6mm}  \def\RM@ptrblandclear{6mm}
\def\RM@ptrpostol{1mm}
\end{verbatim}

\asleepoption{lj4printer} an arbitrary guess
\begin{verbatim}
\def\RM@printertype{22}
\def\RM@ptrrportclear{5mm}  \def\RM@ptrrlandclear{5mm}
\def\RM@ptrlportclear{5mm}  \def\RM@ptrllandclear{5mm}
\def\RM@ptrtportclear{5mm}  \def\RM@ptrtlandclear{5mm}
\def\RM@ptrbportclear{5mm}  \def\RM@ptrblandclear{5mm}
\def\RM@ptrpostol{1mm}
\end{verbatim}

\asleepoption{canonbjxprinter} an arbitrary guess
\begin{verbatim}
\def\RM@printertype{22}
\def\RM@ptrrportclear{7mm}  \def\RM@ptrrlandclear{12mm}
\def\RM@ptrlportclear{7mm}  \def\RM@ptrllandclear{7mm}
\def\RM@ptrtportclear{7mm}  \def\RM@ptrtlandclear{7mm}
\def\RM@ptrbportclear{12mm} \def\RM@ptrblandclear{7mm}
\def\RM@ptrpostol{1mm}
\end{verbatim}

\end{description}

\subsection{Rowland's curious options}


These are curious options, defined by me (RJMM) to perform dark and
eldritch deeds.  These aren't intended for hoi polloi, mainly 'cos
they're a bit iffy in places, but I like them and they might give
you some ideas.


Our DW520 isn't quite to spec.

\begin{description}\shorty\squish

\awakeoption{R+R-dw520printer}
\begin{verbatim}
\def\RM@printertype{2}
\def\RM@ptrrportclear{7mm}  \def\RM@ptrrlandclear{15mm}
\def\RM@ptrlportclear{6mm}  \def\RM@ptrllandclear{7mm}
\def\RM@ptrtportclear{7mm}  \def\RM@ptrtlandclear{7mm}
\def\RM@ptrbportclear{15mm} \def\RM@ptrblandclear{6mm}
\def\RM@ptrpostol{0.5mm}
\end{verbatim}

\awakeoption{lucidacasualwidth}   Lucida casual
Defines \comname{RM@fountfamily} to 9

\awakeoption{loadlucidacasual} A dirty trick to load my .fd version of
lucida casual rather than the PSNFSS version.  If you
\optname{loadlucidacasual}, \comname{RM@fountfamily} is set to 9 after
the \packname{lucida-casual} package has been
\comname{RequirePackage}d.  That's done by code further on in rmpage,
specially written for this dirty trick.
\end{description}

I have written packages that do the same job as the standard
\filename{size10.clo} etc., files, but for larger sizes.  Because the
standard \LaTeX\ \comname{@ptsize} parameter is intended to be a
single digit, and I want to use several different sizes, I have
defined a new parameter that holds the point size of the main body
type, the command \comname{RM@ptsize}.  This parameter is only
defined for my larger sizes.

Because my size packages can be loaded before or after \rmpage, and
because both need to know about the point size, \rmpage says
\begin{verbatim}
\providecommand{\RM@ptsize}{666}
\end{verbatim}
before the extra size options are executed.  The options set
\comname{RM@ptsize} to the appropriate value if this hasn't already
been done.  All my other packages that recognize the larger point size
options do something similar.

The other thing that \rmpage does with these larger point size
options is set \comname{@ptsize} to the 12\unit{pt} value; this is to
fool sections of \rmpage into thinking that it's dealing with a
12\units{pt} fount.

\begin{description}\shorty\squish
\awakeoption{14pt} Sets \comname{RM@ptsize} to |14| if needed and
sets \comname{@ptsize} to 2 (meaning 12\units{12}, to fool \rmpage)
\awakeoption{24pt} Sets \comname{RM@ptsize} to |24| if needed and
sets \comname{@ptsize} to 2 (meaning 12\units{12}, to fool \rmpage)

\awakeoption{36pt} Sets \comname{RM@ptsize} to |36| if needed and
sets \comname{@ptsize} to 2 (meaning 12\units{12}, to fool \rmpage)

\end{description}






\chapter{How things work}
\label{chap:htw}

\section{\comname{textheight} calculation}
\label{htw:textheightcalculation}

The way \rmpage decides on the value of \comname{textheight} is this:
the length options set a length called
\comname{RM@totalheadfootclearance} to be certain fraction of the
\comname{paperheight}.  The value of this command will be the sum of
the blank space above the header and below footer, after it has been
checked against several restrictions.

\begin{ikkystuff}
If the \optname{noheaders} option has used, \comname{headheight} and
\comname{headsep} are both set to 0\units{pt}; if the
\optname{nofooters} option has been used, \comname{footskip} is set
to 0\units{pt}.  So a given length option will fill the page to the
same extent whether or not headers or footers are used; turning
headers and footers off will increase \comname{textheight}.
\end{ikkystuff}

The first check that's made is that the \comname{textheight} produced
by this value of \comname{RM@totalheadfootclearance} will not exceed
the bounds set by \comname{RM@mintextheight} and
\comname{RM@maxtextheight}.  These two commands are intended to be set
by local code on a class-by-class basis in the configuration file,
using the \comname{RM@OnClassType} command in the
\comname{RM@AfterProcessOptions} hook, and define the allowed range of
\comname{textheight}.
%
This check defines the commands:
\begin{verbatim}
\RM@maxpractextheight and \RM@mintotalheadfootclearance
\RM@minpractextheight and \RM@maxtotalheadfootclearance
\end{verbatim}
They are calculated from the user supplied limits; they are based on the
largest and smallest values that \comname{textheight} is allowed to
have, given the limits of the discrete values it is allowed to take.
The \comname{practextheight} parameters are unused at the moment;
they might come in handy one day.

The total space above and below the footer is divided into two: a
certain fraction of this space to the gap at the top, the rest to the
gap at the bottom.  These two lengths are saved in
\comname{RM@totalheadclearance} and \comname{RM@totalfootclearance}.
The \optname{altitude} options control this division of space---see
section~\ref{optrmp:altitude} on page~\pageref{optrmp:altitude}.

Checks are then made to ensure that nothing will be printed outside
the allowed printing region along the vertical axis.  This is defined
by the lengths \comname{RM@minheadclearance} and
\comname{RM@minfootclearance}, which are initially set to the
non-printing margin top and bottom.  \rmpage ensures that these two
lengths are at least as large as either \comname{RM@mintopmargin} or
\comname{RM@minbottommargin}, as appropriate---these two parameters
are intended to be set class-by-class, so you can ensure that your
layout meets regulations, for example.

\rmpage issues warnings if it decides to make either the top or bottom
gap larger to fit the text inside the printing region.  The warnings
are issued because changing the top or bottom gap changes the
\comname{textheight} and vertical position of the text on the page;
you might have been expecting a smooth increment from the last
\optname{altitude} or \comname{length} option, or a particular balance
of space above and below the text given by a particular
\optname{altitude} option, and it might be useful to know that you've
not got what you might expect.


Finally, \comname{textheight} is set to a value such that:
\begin{displaymath}
|\textheight| = \mathrm{integer} \times |\baselineskip| + |\topskip|
\end{displaymath}

The values of the three \comname{RM@total...clearance} parameters are
increased to match the reduction in the size of \comname{textheight},
and \comname{topmargin} is set to whatever it needs to be.

The apparently special case of \optname{stdlength} is handled by
setting \comname{RM@totalheadfootclearance} to a value that will
yield the same \comname{textheight} as the standard classes; the
\classname{book}, \classname{article}, \classname{report}, and
\classname{letter} classes use one value; the \classname{slides}
class another.  Have a look at \filename{rmpage.dtx} and the standard
class files to see how this is done.  Note that \rmpage gives you the
same \comname{textheight} as the standard classes whether or not you
are using headers or footers---the number of text body lines on the
page is always the same (at least, it always has been in testing).
There are exceptions to this: it is possible to ask \rmpage to
position a \optname{stdlength} page on the paper in such a fashion
that the text would end up outside the printable region.  In this
case, \rmpage will issue a warning and reduce \comname{textheight} to
fit inside the allowed area.

\begin{ikkystuff}
The parameters that define the available printing region along the
vertical axis are \comname{RM@minheadclearance} and
\comname{RM@minfootclearance}.  The values of these
\comname{RM@...clearance} parameters are set printer by printer, and
possibly paper-size by paper-size.  \rmpage ensure that they are at
least as large as either {RM@mintopmargin} or
\comname{RM@minbottommargin}, as appropriate.

If the particular combination of printer and paper has set the flag
\comname{RM@jackup} to be true, \rmpage will lift the printing region
to clear an over-large non-printing margin at the bottom of the page.
This is useful for people with Hewlett-Packard inkjet printers.  There
is no similar facility for automatically lowering the printing region
or shifting it sideways.  This is because a larger than expected
space at the bottom of the page is rarely a problem, but lower than
normal or shifted sidedways is usually a problem.  These effects may
be achieved, but you have to do it by using options to have the
specific effect you want.

Each printer option must define the commands:
\begin{center}
\begin{tabular}{lll}
\comname{RM@ptrrportclear}&\comname{RM@ptrlportclear}&\comname{RM@ptrtportclear}\\
\comname{RM@ptrbportclear}&\comname{RM@ptrrlandclear}&\comname{RM@ptrllandclear}\\
\comname{RM@ptrtlandclear}&\comname{RM@ptrblandclear}&\comname{RM@ptrpostol}\\
\end{tabular}
\end{center}

They define the non-printing clearances in
landscape and portrait orientation, and the assumed maximum positional
error.  Code can be added in the \comname{RM@PrinterPaperSettings}
hook to set particular clearances for particular combinations of
printer and paper (to cope with, for example, the 19\units{mm}
non-printing margin at the bottom of an envelope fed into an HP
DeskWriter 520, which is much larger than the 15\units{mm}
non-printing margin at the bottom of a normal bit of paper).


The \packname{beton} package changes \comname{baselineskip}, but does
so at the start of the document, using the \comname{AtBeginDocument}
hook provided by standard \LaTeX. \rmpage needs to know what the
\comname{baselineskip} of the main document fount will be.  To get
round this problem, \rmpage steals some code from \packname{beton} so
it can set \comname{baselineskip} to the value it will have after
\comname{AtBeginDocument}.  When the vertical page parameters have
been calculated, \rmpage puts \comname{baselineskip} back to its
initial value.

None of this happens unless you load the \packname{beton} package, and
specify the \packname{beton} option to \rmpage.  If you don't want
\rmpage to use \packname{beton}'s \comname{baselineskip}, specify the
\optname{nobeton} option to \rmpage.  Note that you must load the
\packname{beton} package before you load \rmpage.  Sorry.
\end{ikkystuff}

\section{\comname{textwidth} calculation}
\label{htw:textwidthcalculation}

The way \comname{textwidth} is worked out is this: two different
\comname{textwidth} initial guesses are calculated---one based on the
width of a certain number of characters allowed in each column, the
other based on a certain fraction of \comname{paperwidth}---and the
smaller one is used as the basis for the final \comname{textwidth}.
The particular number of characters and fraction of \comname{paperwidth}
is set by the \optname{width} option specified.  The default
\optname{normalwidth} option gives a character-based width very close
to standard, but the paper-based width is quite a bit larger than
standard.  This is only significant when \rmpage uses the paper-based
width, as it usually does when you are printing on A5 paper.

\begin{ikkystuff}
The technique of choosing the smaller of two \comname{textwidth}s:
one based on the number of characters, and the other based on the
size of the paper, was derived from the standard classes' way of doing
things---a fixed width (different for each size) is compared to
$\mbox{\comname{paperwidth}}-2\units{in}$, and the smaller is used.  The
fixed width is different for each size; the conventional classes use
hardwired sizes, and the \classname{slides} class uses $65/2 \times
\mbox{width of (|\hbox{\rmfamily im}|)}$.

\rmpage asks for a normal text-based width based on the same number of
characters as the standard widths, and compares this to a paper-based
width that is calculated as $0.7138 \times \mbox{\comname{paperwidth}}$.  The
fraction used gives a larger \comname{textwidth} than the standard
classes when typesetting on A5 paper, but a smaller \comname{textwidth}
when typesetting large founts on A4 paper.

The normal character-based \comname{textwidth} calculated by \rmpage
is different to that calculated by the standard classes because the
standard classes round calculated dimensions down to the next lowest
integer number of points.  I think this is a mistake, because it
introduces an unnecessary error in margin sizes, so \rmpage doesn't do
it (unless you're asking for \optname{stdwidth}, which does truncates
the calculated value of \comname{textwidth} only).

The special widths: \optname{oneinchmargins},
\optname{halfinchmargins}, and \optname{fullwidth}, all set the
paper-based textwidth to a fixed value to leave the specified amount
of space either side, assuming that you have asked for
\optname{centre}d printing (\optname{fullwidth} leaves no space).If
you haven't, the total space either side will add up to what you'd
expect, but you can get 1.5\units{in} on one side, and 0.5\units{in}
on the other, for example, if you're using \optname{oneinchmargins}.

\optname{stdwidth} sets both the paper and character-based initial
guesses to the same value as the standard classes.  If the selected
initial \comname{textwidth} value isn't reduced, the result will be
the same as the standard classes' calculations (insert standard
disclaimer here---not because I'm afraid of being sued, but because I
think you should check that the value of \comname{textwidth} is what
it should be if it's really important.  This is because I think this
is a complicated piece of software because \TeX's piggin' awful for
doing maths, and I've not verified the algorithm to my own
satisfaction.  I have tested it, and it appears to work the way I
want, so that'll have to do for now.)

\end{ikkystuff}

The first check ensures that this first guess is within the bounds of
\comname{RM@mintextwidth} or \comname{RM@maxtextwidth}.  If it's not,
it's made big or small enough.

\begin{ikkystuff}
If you've asked for \optname{characterwidthset}, then the paper-based
\comname{textwidth} is set to a large value; similarly, if you've
asked for \optname{paperwidthset}, the character-based
\comname{textwidth} is set to a large value.  If you've asked for an
inherently paper-based width like \optname{oneinchmargins}, both the
paper and character-based \comname{textwidth}s are set to the
appropriate value.  So if you also ask for
\optname{characterwidthset}, \rmpage will give you
\optname{oneinchmargins} anyway, and a complaint.
\end{ikkystuff}

Now the smallest of the two \comname{textwidth}s is selected as the
one to use for real, and \rmpage calculates \comname{evensidemargin}
based on the requested offset---the proportions in which the available
horizontal space is divided between the larger and smaller margins
(see section~\ref{optrmp:margins} on page~\pageref{optrmp:margins} for the
details).

\begin{ikkystuff}
When \rmpage checks \comname{textwidth}, it takes into account
whether you're printing two sided or not.  The checking code looks a
bit complicated, because there are two sets of limits that apply to
the horizontal extent of the text: \comname{RM@minrightclearance} and
\comname{RM@minleftclearance}; and \comname{RM@minoutsidemargin}  and
\comname{RM@mininsidemargin}.  \rmpage looks at the appropriate limits.

The \comname{left} and \comname{right} \comname{minclearance}
parameters define the possible printing region, as set for the
requested printer and paper combination; all text must fit inside
these limits.  The \comname{inside} and \comname{outside}
\comname{minmargin} parameters define the permitted extent of the main
body text, excluding marginal paragraphs.  The \comname{minmargin}
parameters are defined to be |0pt| by default; they were introduced so
that I could write a thesis class which had to ensure particular
minimum margins to meet the regulations.  You can see how I used them
in the \comname{RM@AfterProcessOptions} hook definition in the
configuration file.
\end{ikkystuff}

If \comname{evensidemargin} is too small to allow the text to print
on the page, it is increased, and \comname{textwidth} decreased to
maintain the requested offset proportions.

Then the right-hand edge of \comname{textwidth} is checked to ensure
that it is within the allowed printing region.  If it's not,
\comname{textwidth} is reduced.  If so, \comname{evensidemargin} must
be increased to maintain the requested offset proportions, and
\comname{textwidth} reduced by the same amount to keep the right-hand
margin the same size.


\begin{ikkystuff}
If you've asked for \comname{fullwidth}, \rmpage won't attempt to
retain the offset proportions, nor will it issue as many warnings
about decreasing things to fit.
\end{ikkystuff}

When that's done, \comname{oddsidemargin} is set to the appropriate
value: equal to \comname{evensidemargin} if you've asked for one sided
or centred printing;  or equal to the right-hand margin on an
even-numbered page if you're printing two sided and not centred.

And finally, \rmpage checks that the final value of
\comname{textwidth} is still larger than \comname{RM@mintextwidth}.
If it's not, \rmpage can't do anything about it, so just issues an
error message.


\section{Hooks}
\label{htw:hooks}

There's five hooks:

\begin{tabbing}
\comname{RM@BeforeProcessOptions}, \= \comname{RM@AfterProcessOptions},\\
\comname{RM@PrinterPaperSettings}, \\
\comname{RM@BeforeWidthSetting}, \> \comname{RM@AfterWidthSetting}.\\
\end{tabbing}

\begin{description}\shorty

\item[\comname{RM@BeforeProcessOptions}] This hook is executed just
before \comname{ProcessOptions}, and before the \comname{RM@donewithoptions}
flag has been set to true, so options which can only be
specified in an \comname{ExecuteOptions} statement can be requested.

\item[\comname{RM@AfterProcessOptions}] This hook is executed well after
\comname{ProcessOptions}.  It is executed after most of the fiddling
about prior to working out page parameters has been done, just after
the current class has been identified, but before class-specific code
is executed.  This is the hook to use if you want to add a new class:
you should preferably set \comname{RM@classtype} with a new option,
declared either directly in the config file, or by using the
\comname{RM@BeforeProcessOptions} hook.  Then put class-specific code in
the \comname{RM@AfterProcessOptions} hook; use the \comname{RM@OnClassType}
command.

\item[\comname{RM@PrinterPaperSettings}] This hook is executed after the
standard printer/paper specific code has been executed.  Use the
\comname{RM@OnPrinterType}, \comname{RM@OnPortraitPaperType},
\comname{RM@OnLandscapePaperType}, and \comname{RM@On\-Paper\-Type} commands
here.

\item[\comname{RM@BeforeWidthSetting}]  If you want to use a different
file for width setting, define \comname{RM@widthsetter} to be the name of the
file in this hook.  The \comname{RM@OnClassType} command can be used to
select which class this file should be used for.  This hook is
executed after the \comname{RM@OnClassType} command has been set by the
standard code, and just before the width setting file is loaded.

\item[\comname{RM@AfterWidthSetting}]  This hook is executed on returning
from the width setting file.  It's here for \ae sthetic reasons.

\end{description}

\section{Marginal paragraphs}
\label{htw:mpars}

There is more on setting the size of marginal paragraphs in
section~\ref{htw:headfootmpar} on page~\pageref{htw:headfootmpar}.

If you are going to use marginal notes in your document, ensure that
you specify \comname{normalmarginpar} or \comname{reversemarginpar}
\emph{before} loading \rmpage.  This is because \rmpage calculates the
size of marginal paragraphs based on the space available, and if
\rmpage thinks you're going to put marginal notes in the margin which
is largest, and you really put them in the margin that is smallest,
it'll get the calculation wrong and you'll have marginal notes that
don't fit on the page.

If you're going to switch between \comname{normalmarginpar} and
\comname{reversemarginpar} in your document, select whichever one puts
the marginal notes in the smallest margin before you load \rmpage.
Messy, I know---sorry.

The standard \LaTeX\ classes, report and article, create marginal
paragraphs that are a fixed distance away from the text, with a
minimum clearance from the edge of the paper of 0.8in (one sided
printing), or 0.4in (two sided printing), and a maximum
width of 2in.

\rmpage says yah boo sucks to all this.

\section{Dealing with different classes}

\rmpage gives each class a number; the number of the current class is
stored in the command \comname{RM@classtype}.  Classes are detected in
\rmpage, and in the config file.  More than one class can have been
loaded; the idea is that the first loaded class is defined as the
current class.

\section{Different paper types and printers}
\label{htw:paperprinter}

\rmpage knows about non-printing margins, and about different paper
sizes and orientations.  Each printer has its own defined
non-printing margins, which are used to calculate the non-printing
margins for each paper size.

The non-printing margins for the selected paper size are calculated and
stored in the commands:

\begin{tabbing}
\comname{RM@minrightclearance} \= \comname{RM@mintopclearance} \\
\comname{RM@minleftclearance} \> \comname{RM@minbottomclearance} \\
\end{tabbing}

Each printer option must define nine commands so that \rmpage can
calculate the non-printing margins for each paper size.  These
commands to define the non-printing margins for each printer are:

\begin{tabbing}
Portrait orientation non-printing margins:\\
\comname{RM@ptrrportclear} \=  right-hand side\\
\comname{RM@ptrlportclear} \>  left-hand side\\
\comname{RM@ptrbportclear} \>  bottom edge \\
\comname{RM@ptrtportclear} \>  top edge \\
Landscape orientation non-printing margins:
\comname{RM@ptrrlandclear} \>  right-hand side\\
\comname{RM@ptrllandclear} \>  left-hand side\\
\comname{RM@ptrblandclear} \>  bottom edge \\
\comname{RM@ptrtlandclear} \>  top edge \\
Assumed maximum positional error:
\comname{RM@ptrpostol} \\
\end{tabbing}

The \comname{RM@ptrpostol} command holds a length which is added to
the \comname{RM@minclearance} values right at the end of the
non-printing margin calculations, \emph{after} the
\comname{RM@PrinterPaperSettings} hook has been executed.  Properly
speaking, there should be four of these: one for left-right error and
one for up-down error in both portrait and landscape orientation, but
I think that one is probably adequate.  The standard printer types set
\comname{RM@ptrpostol} to 1\units{mm}.

\rmpage works out which printer clearance parameters to use as the
non-printing margins like this;

\begin{enumerate}

\item The flag \comname{RM@portrait} is set true if you are printing
in portrait orientation, false if you are printing landscape.

\item The flag \comname{RM@portlandinvert} is set true if you are
using a long paper size which reduces the length of the parent paper
size to less than the width of the parent paper size.

\item
\begin{tabbing}
if \= (|\RM@portrait| and not |\RM@portlandinvert|) \\
\>    or (not |\RM@portrait| and |\RM@portlandinvert|)\\
then use the |portclear| parameters \\
\end{tabbing}

\item
\begin{tabbing}
if \= (|\RM@portrait| and |\RM@portlandinvert|)  \\
\>    or (not |\RM@portrait| and not |\RM@portlandinvert|)\\
then use the |landclear| parameters\\
\end{tabbing}

\item If you are using a long paper size that is greater than half the
parent size, set to 0\unit{pt} the non-printing margin at the edge
where you cut the parent size to create the long size (assumed to be
either the right-hand or bottom edge).

\item Execute the \comname{RM@PrinterPaperSettings} hook

\item Add the value of the command \comname{RM@ptrpostol} to the
non-printing margins

\end{enumerate}


\section{Headers, footers, and marginal paragraphs}
\label{htw:headfootmpar}

This section looks at how to control the size of the gap between the
main body text and: headers, footers, and marginal paragraphs, as well
as the gap between marginal paragraphs and the edge of the paper, the
maximum width of marginal paragraphs, and the gap between columns of
text in a multiple-column layout.

The details of how the size of marginal paragraphs is calculated are
in section~\ref{htw:mpars} on page~\pageref{htw:mpars}.

Some lengths used as page layout parameters are set to their final
values by the command:
\begin{verbatim}
\RM@scalebyoption{<length to be scaled>}{<option number>}
\end{verbatim}
The |<length to be scaled>| is multiplied by a factor controlled the
the |<option number>|.  The option number is set to 12 by
default (which means multiply by one), and is allowed to range from 1
to 23.  The multipliers are in a geometrical sequence from 0.3263 to
3.0646.  The option |least...| sets the option number to 3, and gives
a multiplier of 0.4.  The option |most...| set the option number to
21, and gives a multiplier of 2.5.  The |touchmore...| and
|touchless...| options add or subtract one from the option number, as
do the |t@uchmore...| and |t@uchless...| options (which are reserved
for use by class files).

These following lengths are scaled by option; the name of the counter
storing the controlling option number is given in each case.

\begin{enumerate}
\item \comname{headsep}---the gap between the top of the text and the
box containing the head.  Controlled by \comname{RM@headsepoption}.

\item \comname{footskip}---the gap between the bottom of the text and
the baseline of the foot.  \rmpage scales the length
($\mbox{\comname{footskip}}-\mbox{\comname{baselineskip}}$), to scale the
apparent gap between the (assumed one line) footer and the bottom of
the text body.  Controlled by \comname{RM@footskipoption}.

\item \comname{columnsep}---the gap between columns on a multi-column
page (nothing to do with tables).  Controlled by
\comname{RM@columnsepoption}.

\item \comname{marginparsep}---the gap between the text body and
marginal paragraphs.  Controlled by \comname{RM@mparsepoption}.

\item \comname{RM@mparclearance}---the minimum gap between the edge of
the paper and marginal paragraphs.  Controlled by \comname{RM@mparclearoption}.

\item \comname{RM@maxmparwidth}---the maximum width of marginal paragraphs.
  Controlled by \comname{RM@maxmparwidthoption}.

\end{enumerate}

The first four are standard \LaTeX\ lengths, and the only change \rmpage
makes to them is with \comname{RM@scalebyoption}.  The fifth and sixth
lengths are new parameters set by \rmpage.

You can play about with all of these parameters by passing options to
\rmpage.

If the spread of values given by the standard options isn't enough,
you can say:
\begin{verbatim}
\setlength\columnsep{3\columnsep}
\end{verbatim}
or some such in your preamble, \textbf{before} loading \rmpage.  This
only works for the standard \LaTeX\ parameters.

If you want to do something like this for \comname{RM@mparclearance}
or \comname{RM@maxmparwidth}, you can use the
\optname{largebasemparclear} or \optname{largebasemaxmparwidth}
options.  These multiply the corresponding parameter by 2 if it has
not been set by a configuration file.

\chapter{Configuring \rmpage}
\label{chap:config}

The configuration file exists so you can tailor your installation of
\rmpage to your preferences.  Some obvious things to set are the
default printer type, the default paper type, the default date style,
whether you want the inside or the outside margin larger.

\section{Setting up a new installation}
\label{optcfg:settingup}

\rmpage will work entirely happily without local configuration, but
you might want to customize its behaviour, for example to speed it up.
This chapter explains how.

For the sake of compatibility with everyone else and future versions,
please keep the file \filename{rmpgen.cfg} unchanged in your \TeX{}
search path, and make changes only to a copy of one of the standard
configuration files that came with \rmpage.  I suggest you make a copy
called \filename{rmplocal.cfg}, from either \filename{rmpgen.cfg}
(everything active) or \filename{rmplocal.gfc} (fastest).

When you create your local configuration file, begin by doing two
things: add a comment on the top line identifying this file as yours;
and change the \comname{ProvidesFile} command to match the new name
and identify the file as yours---don't forget to change the date and
version number:
\begin{verbatim}
\ProvidesFile{rmplocal.cfg}
 [1381/04/01 v0.1 Wat Tylers's
  local configuration file for the rmpage package.]
\end{verbatim}

It's okay to comment out and uncomment options, but don't make any
other changes above the line in the configuration file that says:
{\ttfamily\scshape local code below here please}.  I won't complain
if you do, but you'll find it harder to upgrade to new versions of
\rmpage.

\section{Configuration basics}

I assume that you have made a copy called \filename{rmplocal.cfg} of
either \filename{rmplocal.gfc} or \filename{rmpgen.cfg}, and that
you've changed the \comname{ProvidesFile} command.  If not, read the
start of this chapter again.  I will refer to \filename{rmplocal.cfg}
as the config file in this chapter; there are other names a local
configuation file might have.

Please don't change your config file above the line that says:
{\ttfamily\scshape local code below here please}.  I won't complain if
you do, but you'll find it harder to upgrade to new versions of
\rmpage.

\subsection{Unknown option error}

If you are using a copy of \filename{rmplocal.gfc} and \LaTeX{}
complains about an unknown option, edit your config file and
uncomment the option you need.  The more options that are
uncommented, the slower \rmpage will work.

\subsection{Default options}

Look through your config file for the line \textsc{change the command
below to match your local preferences}.  There's an
\comname{ExecuteOptions} statement just below it.  This statement sets
your default options---my default settings are below.  I have a
special printer type, because my printer is old, tired, and
out-of-spec; I usually print on A4 paper with a large inside margin so
I can put my printouts in a ring binder, and I like my dates like
this: 5th November 1996.  You can change the argument of this
statement to match your preferences: if you're an American with a
LaserJet 4, replace the printer option with \optname{lj4printer},
change the paper type to \optname{letterpaper}, and the date to
\optname{usdate}.  If most of your output goes in ring-binders, keep
the \optname{notstdmargins} option; otherwise, change it to
\optname{stdmargins} to give a conventional large outside margin.

\begin{verbatim}
%%     CHANGE THE COMMAND BELOW TO MATCH YOUR LOCAL PREFERENCES
%%     --------------------------------------------------------
%%
\ExecuteOptions{R+R-dw520printer,a4paper,notstdmargins,nicedate}
%%
%%     --------------------------------------------------------
\end{verbatim}

You can add almost any option you like to this statement: it sets the
defaults for everything you typeset with \rmpage.  Some of the options
in \filename{rmplocal} have been commented out to speed things up, so
you might need to uncomment them to allow them to work.  You can tell
whether an option needs uncommenting from the list of all the options
in chapter~\ref{chap:alltheoptions} on
page~\pageref{chap:alltheoptions}: options that look like this:
{\asleepfount obscureoption} are commented out; options that look like
this: {\awakefount usefuloption} are not commented out.  None of the
options in \rmpage are commented out---you only need to edit the
configuration file.

You can have additional default options for particular classes.
Because the options for particular classes are executed after the
general defaults, they can over-ride the general defaults.  So it's
quite all right to say \optname{a4paper} in the
\optname{ExecuteOptions} statement above if you normally print out
slides on B5~paper, because the \optname{b5paper} option executed
later on will over-ride the original \optname{a4paper} option.

If you want different default options for different classes, or if you
intend to use the slides class, read the next section which explains
all.

\section{Configuring \rmpage for particular classes}
\label{cfg:classes}

Just below the \comname{ExecuteOptions} statement which sets the
global defaults, there's a section headed \textsc{default options for
particular classes}.  This is the place intended for your
class-specific default settings.  \rmpage provides a command to do
this:
\begin{verbatim}
\RM@OnClassExecuteOptions{<class name>}{<comma separated options>}
\end{verbatim}
It's just an \comname{ExecuteOptions} command that is only executed
for the named class.  For example, my configuration file has the
statement:
\begin{verbatim}
\RM@OnClassExecuteOptions{slides}
  {centre,ukdate,R+R-dw520printer}
\end{verbatim}
Which executes the given options only when I'm using the
\classname{slides} class.  You might like to change this statement to
match your preferences.

You can have as many of these \comname{RM@OnClassExecuteOptions}
statements as you like, although one statement for each class is
probably best.  If you are creating a class of your own, say a thesis
class based on \classname{report}, it's probably best to make sure
that the default options for your thesis class are executed after the
default options for the \classname{report} class.  That way, the
\classname{report} class's defaults don't over-ride your thesis
class's defaults.

If you want to create a new class with the help of \rmpage, read
section~\ref{cfg:newclass} on page~\pageref{cfg:newclass} for more
details.

\section{Defining a new printer type}
\label{cfg:printertype}

\section{Dealing with particular combinations of printer
and paper}
\label{cfg:printerpaper}



\section{Telling \rmpage about a new class}
\label{cfg:newclass}

If you are going to define a new class type, there are two obvious
ways of doing it: either declare a new option which sets the
\comname{RM@classtype} command to your new class number (above 100,
please), or put a line in the \comname{RM@DefineNewClasses} hook:
%
\begin{verbatim}
\DeclareOption{nuthesisclass}{\def\RM@classtype{101}}
\end{verbatim}
or
\begin{verbatim}
\newcommand*{\RM@DefineNewClasses}{
\RM@SetClassType{nuthesis}{101}
}% endRM@DefineNewClasses
\end{verbatim}

The advantage of not using an option is greater speed.  The advantage
of using an option is that you can pass the option to \rmpage, and be
sure that your particular settings are acted upon, even if you change
the name of your class.


You can set default options for a particular class in the config file:
the \textsc{default options for particular classes} section is for you
to add:
\begin{verbatim}
\RM@ClassExecuteOptions{<class name>}{<options list>}
\end{verbatim}
statements for each class you want to \comname{ExecuteOptions} for.
make sure that if you are building one class upon another (e.g.,
building nuthesis on report), that you execute the options for the
base class first (e.g., do report before nuthesis).


The \comname{RM@AfterProcessOptions} hook is the ideal place to use
the \comname{RM@OnClassType} command to set up things for particular
classes.  You can set things like minimum margins, maximum textwidth,
and the like there.  See the config file for some examples.

\subsection{Dealing with options}

If you want to build a new class by modifying a standard class with
the help of \rmpage, you need to think about what's going to happen
to options.

You can tell your new class, let's call it the \classname{nuthesis} class,
to pass options on to \rmpage quite happily, by including a
line:
\begin{verbatim}
\DeclareOption*{\PassOptionsToPackage{rmpage}
    {\CurrentOption}}
\end{verbatim}
in the option declaration section of your class file to pass all
options you don't deal with to \rmpage (not forgetting to say:
\begin{verbatim}
\ProcessOptions
...
\RequirePackage{rmpage}
\end{verbatim}
later on).

The problem with this is that any options you have set up explicitly
(for example, you might have pass the \optname{wide} option to rmpage
to get a particular \comname{textwidth}, and the user might have asked
for \optname{narrow}.  Do \emph{you} know which one takes precendent?)
might be over-ridden by the user.

The safest way to deal with this is to decide which of \rmpage's options
you will allow the user to use.  You might stick with the
\begin{verbatim}
\DeclareOption*{\PassOptionsToPackage{rmpage}
    {\CurrentOption}}
\end{verbatim}
statement in your class file and prepare a special \rmpage
config file for your new class, which has all other options commented
out.  If you call this config file \filename{rmpage-nuthesis.cfg},
include the line
\begin{verbatim}
\newcommand*{\RMconfigfile}{rmpage-nuthesis.cfg}
\end{verbatim}
in your class
file before loading \rmpage.  Or you might use:
\begin{verbatim}
\DeclareOption{<option name>}
    {\PassOptionsToPackage{rmpage}{<option name>}}
\end{verbatim}
to pass each allowed option on to \rmpage, and ensure that you're not
passing options to \rmpage with \comname{DeclareOption*}.

If you have used \rmpage to help you get printing dimensions right for
a fixed format and you don't want the user to change anything, you
might find it best to set the various parameters directly in your
class file, and forget about using \rmpage entirely.  You can find out
what they were set to by looking in the \filename{log} file; by the
time I've released \rmpage, everything that \rmpage changes damned
well ought to be noted there, and if not you can curse me for a
careless fool, and specify the \optname{garrulous} option to \rmpage
which will print everything and then some in your \TeX\ console
window.  Don't try this if you're going to keep using \rmpage to build
your class---I have no idea what'll happen.



\subsection{Things you can do with your new class number}

\subsubsection{Changing textheight setting}

You can define the command \comname{RM@textheightgroup} to any number
you like.  It's probably best to do this in the
\comname{RM@AfterProcessOptions} hook.

Currently, the initial value of textheight is set by one of two blocks
of code: one is executed if \comname{RM@textheightgroup} is 0
(default); the other is executed if \comname{RM@textheightgroup} is 1
(slides only).

If you set \comname{RM@textheightgroup} to anything other than 0 or 1
for any of your classes, you will need to add some code to set
textheight!  Put your new code in the
\comname{RM@BeforeTextheightSetting} hook; have a look at \rmpage to
see how I did it.

\subsubsection{Changing textwidth setting}

ALL THIS IS WRONG NOW!

You can also define the command \comname{RM@widthsetter} to be any
filename you like, using any of the hooks executed before the width
setting code is used.  If the command is not defined just before the
\comname{RM@BeforeWidthSetting} hook is executed, it is defined to be
rmpwnorm.pko (the extension stands for package option).  If the class
type is 5 (\classname{slides}), the command is then defined to be
rmpwslid.pko.  This filename is the file loaded to set the various
horizontal parameters.  By default, \filename{rmpwnorm.pko} is loaded
for all classes expect slides, which uses \filename{rmpwslid}.  An
example of this sort of thing is this fragment of config file code:

\noindent
{\footnotesize
\begin{verbatim}
\newcommand*{\RM@AfterProcessOptions}{
    \RM@OnClassType{101}{% class 101 = nightmare university thesis class
        % Use different textheight setting code to everything else.
        \def\RM@textheightgroup{2}
        % Set minimum margins as specified in the regulations.
        % Everything else is done by the class file.  These commands
        % are defined to be 1742pt at the start of rmpage, so they
        % can't be set in the class file.
        \def\RM@minoutsidemargin{15mm}
        \def\RM@mininsidemargin{40mm}
        \def\RM@mintopmargin{15mm}
        \def\RM@minbottommargin{15mm}
        % set minimum and maximum textwidth, as defined by the regs
        \def\RM@mintextwidth{130mm}
        \def\RM@maxtextwidth{160mm}
        % load custom width setting code in file vulture-widths.jkl
        \def\RM@widthsetter{vulture-widths.jkl}
    }{}
}
%
%
%
\newcommand*{\RM@DefineNewClasses}{
    \RM@SetClassType{rmcv}{20}
    \RM@SetClassType{rmletter}{21}
    \RM@SetClassType{bithesis}{22}
    \RM@SetClassType{ljmueepexam}{23}
    \RM@SetClassType{nuthesis}{101}
}% endRM@DefineNewClasses
%
%
% nightmare u. thesis uses fixed total text area height of 7in
\newcommand*{\RM@BeforeTextheightSetting}{
    \RM@OnTextheightGroup{2}{%
        \setlength\RM@totalheadfootclearance{\paperheight}
        \addtolength\RM@totalheadfootclearance{-7in}
    }
}% endRM@BeforeTextheightSetting
%
\end{verbatim}
}

The code fragment above defines a new class type,
\classname{nuthesis}, number 101, which is a class for preparing
theses for Nightmare University.  This class uses textheight setting
code that asks for a total text body height of as near to 7\units{in}
as possible no matter what textheight options are specified, and
horizontal text parameters are set by the file called
\filename{vulture-widths.jkl}.  The other limits on the printing
region specified by the university's regulations are placed in the
\comname{RM@AfterProcessOptions} hook.  What I have typed above is in
addition to any code which might be in those hooks anyway; don't
remove anything unless you've got a good reason---who knows what might
go wrong?




\end{document}




