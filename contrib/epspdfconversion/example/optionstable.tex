% !TEX root = epspdfconversion_docu.tex
%START CODE FOR THE LONG TABLE
\small
\begin{longtable}{
>{\hsize=.9\hsize}X
>{\hsize=1.1\hsize}X  
}
%Die Summe muss/sollte 2 sein. Man kontrolliert damit die Breite der Spalten, wobei Zeilenumbruch in der Spalte funktioniert!
%
% Definition of Headers and caption, these are shown where the table starts
\caption{Options for the package {\pack}\label{optiontable}}
\\ 
\toprule
option & explanation  \\ \midrule
\endfirsthead 
% % Definition of Headers and caption, these are shown after a page break
\caption[]{Options for the package {\pack} -- continued}
\\ 
\toprule
option & explanation  \\ \midrule
\endhead 
% before a page break you will see:
\midrule 
\multicolumn{2}{r}{... continued on the next page ....}    \\  
\endfoot 
% when the tabloe ends, you will see:
\bottomrule
\endlastfoot 
%%%%%
%%%%% the content of the table comes after here
%
\multicolumn{2}{l}{\textbf{Options related to epspdf}}  \\
%
\verb"help" & You will be shown the help of the epspdf command in your logfile. This option does not overrule all the others as previously.\\  & \\
%
\verb"simple" & the epspdf-conversion will be done with no option at all. Don't use it together with any of the options below.\\  & \\
%
\verb"gray | grey | GRAY | GREY" & gray | grey : convert eps-figures to grayscale (success not guaranteed); GRAY | GREY: Try harder to convert to grayscale (success still not guaranteed) \\  
%
\verb"nogray | nogrey" & nogray | nogrey : do not try to convert eps-figures to grayscale \\  & \\
%
\verb"default | printer |" \verb"prepress | screen | ebook |" \verb"target=default |" \verb"target=printer | " \verb"target=prepess |" \verb"target=screen | " \verb"target=ebook"  & target use of pdf \\  & \\
%
\verb"pdfversion=default |" \verb"pdfversion=1.2 |" \verb"pdfversion=1.3 |" \verb"pdfversion=1.4"  & Pdf version to be generated. Setting another version than those on the left will result in an error. `default' means
whatever Ghostscript's default is. \\  & \\
%
\verb"bbox |" \verb"bbox=true" | \verb"bbox=false" & If true: Compute tight boundingbox\\  & \\
%
\verb"nopdftops" & Ignore pdftops even if available (default: use if available)\\  & \\
%
\verb"pdftops" & use pdftops if available\\  & \\
%
\verb"hires" & compute hires-Boundingbox\\  & \\
%
\verb"no-hires" & don't compute hires-Boundingbox\\  & \\
%
\verb"custom={-dPDFX}" & This option allows you to pass a string to the ghostscript-commandline. On the left it would be  Here you can set custom options for conversion to pdf, 
view \href{http://pages.cs.wisc.edu/~ghost/doc/cvs/Use.htm}{Use.htm} and \href{http://pages.cs.wisc.edu/~ghost/doc/cvs/Ps2pdf.htm}{ps2pdf.htm} from 
the Ghostscript documentation set. The example on the left adds \verb"-dPDFX" to the ghostscript-call by epstopdf \\  & \\
%
\verb"psoptions={-level2}" & This sets the options for pdftops; the default is -level2, don't include -eps or page number options; these will be generated by epstopdf itself \\  & \\
%
\verb"pagenumber={1}" & Page (in the source-file) to be converted\\  & \\
%
\multicolumn{2}{l}{\textbf{Options related to epstopdf.sty (the package)}}  \\
%
 &  These options are options that are passed over to epstopdf.sty that works in the background. You could also use 
\verb"\epstopdfsetup{OPTIONSforEPSTOPDF}", 
but you can also control the behaviour of epstopdf.sty by means of setting options of {\pack}. The explanation is borrowed from the \href{http://www.ctan.org/tex-archive/macros/latex/contrib/oberdiek/epstopdf.pdf}{documentation of epstopdf}.\\  & \\
%
\verb"prepend" | \verb"prepend=true" | \verb"prepend=false"  &  Determines whether .eps is appended (if false) or prepended (if true) to the Graphics extension search list. (default: false). (Note that there is no option append. Use \verb"prepend=false" instead.)\\  & \\
%
\verb"update" | \verb"update=true" | \verb"update=false" & The conversion program is only called, if the target file does not exist or is older than the source image file. If false, the conversion is done with every run of pdflatex. \verb"update=false" makes sense when you are not yet sure which settings for the conversion to pdf you are going to use.\\  & \\
%
\verb"verbose" | \verb"verbose=true" | \verb"verbose=false" & prints some information about the image in the .log file (default: true). 
\\  & \\
%
\verb"suffix={-mysuffix}" & defines a string that is put between the file name base and the extension of the output file. This avoids that existing pdf-files are overwritten. See the  \href{http://www.ctan.org/tex-archive/macros/latex/contrib/oberdiek/epstopdf.pdf}{documentation of epstopdf} for a more detailed explanation. (default: suffix=-epspdf-to)\\  & \\
%
\verb"prefersuffix" | \verb"prefersuffix=true" | \verb"prefersuffix=false" & If a suffix is set by option suffix, then there can be two image file names that could be taken into account for inclusion: A image file name with the suffix string inside its name and a image file name without; e.g. for 
\verb"foo.eps" the names could be: 
\verb"foo-suffix.pdf", \verb"foo.pdf" 
If option prefersuffix is turned on, the file \verb"foo-suffix.pdf" and its generation 
is preferred over using \verb"foo.pdf". Otherwise \verb"foo.pdf" is included without generating \verb"foo-suffix.pdf". (default: true)\\  & \\
%
\verb"outdir=./" & The converted file may put in another output directory. The value of outdir must include the directory separator. Example for the current directory: \newline
\verb"\epstopdfsetup{outdir=./}"\newline
For other directories ensure that they can be found. See \verb"\graphicspath" or \verb"TEXINPUTS". You might need to set \verb"suffix=" to use another directory than the current. (default: outdir not set, converted images are saved in the same directory as the source-files.)  \\
%
\verb"pdftopdf" | \verb"pdftopdf=true" | \verb"pdftopdf=false" &  Enable conversion also for .pdf-files. Detects an .pdf-file, converts it to an pdf, applying the other options that are set (grayscaling, ... ) and used the converted .pdf-file. (default: \verb"pdftopdf =false")  \\
%
\verb"pstopdf" | \verb"pstopdf=true" | \verb"pstopdf=false" &  Enable conversion also for .ps-files. Detects an .ps-file, converts it to an pdf and used this. (default: \verb"pstopdf=false")  \\ \end{longtable}
%END CODE FOR THE LONG TABLE




