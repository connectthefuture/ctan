\ProvidesFile{blogexec.tex}[2012/12/20 documenting blogexec.sty]
\title{{\huge\pkg{blogexec.sty}}\\---\\Overcoming 
       \pkg{blog.sty}'s Pure Expansion\thanks{This
       document describes version
       \textcolor{blue}{\UseVersionOf{\jobname.sty}}
       of \textsf{\jobname.sty} as of \UseDateOf{\jobname.sty}.}}
% \listfiles
{ \RequirePackage{makedoc} 
  \ProcessLineMessage{} %%% {\message{.}} 
  \MakeJobDoc{17} 
  {\SectionLevelThreeParseInput}  }
\documentclass[fleqn]{article}%% TODO paper dimensions!?
\ProvidesFile{makedoc.cfg}[2011/06/27 documentation settings] 

\author{Uwe L\"uck\thanks{\url{http://contact-ednotes.sty.de.vu}}}
% \author{Uwe L\"uck---{\tt http://contact-ednotes.sty.de.vu}}

%% hyperref:
\RequirePackage{ifpdf}
\usepackage[%
  \ifpdf
%     bookmarks=false,          %% 2010/12/22
%     bookmarksnumbered,
    bookmarksopen,              %% 2011/01/24!?
    bookmarksopenlevel=2,       %% 2011/01/23
%     pdfpagemode=UseNone,
%     pdfstartpage=10,
%     pdfstartview=FitH,
    citebordercolor={ .6 1    .6},
    filebordercolor={1    .6 1},
    linkbordercolor={1    .9  .7},
     urlbordercolor={ .7 1   1},   %% playing 2011/01/24
  \else
    draft
  \fi
]{hyperref}

\RequirePackage{niceverb}[2011/01/24] 
\RequirePackage{readprov}               %% 2010/12/08
\RequirePackage{hypertoc}               %% 2011/01/23
\RequirePackage{texlinks}               %% 2011/01/24
\makeatletter
  \@ifundefined{strong} 
               {\let\strong\textbf}     %% 2011/01/24
               {} 
  \@ifundefined{file} 
               {\let\file\texttt}       %% 2011/05/23
               {} 
\makeatother

\errorcontextlines=4
\pagestyle{headings}

\endinput

 %% shared formatting settings
\usepackage{filesdo} \MDfinaldatechecks     %% 2012/12/20
\providecommand*{\HTML}{\acro{HTML}}
\ReadPackageInfos{blogexec}
\sloppy
\begin{document}
\maketitle
\begin{abstract}\noindent
'blog.sty' before v0.7 generated \HTML\ by pure macro expansion 
and could use \LaTeX\ macros (redefined) only to a very limited 
extent. On adding 'blogexec.sty', some macros (configurable) are 
``intercepted" before expansion in a 'blog' run for 
``running" some code, basically `\begin', `\end', 
and a new general `\EXECUTE'. 
A table environment with active characters inside only is 
provided---perhaps ``nicer than \LaTeX."
\end{abstract}
\tableofcontents

%   \newpage
\section{Features and Usage}

% \subsection{Installing and Calling}
The file 'blogexec.sty' is provided ready, installation only requires
putting it somewhere where \TeX\ finds it
(which may need updating the filename data
 base).\urlfoot{ukfaqref}{inst-wlcf}           %% corr. 2011/02/08

'blogexec.sty' may be loaded by
\begin{verbatim}
  \RequirePackage{blogexec}
\end{verbatim}
in a driver file for 'blog.sty'. \strong{Alternatively}, 
the following commands in a 'blog' driver file 
(in a certain way even in a \emph{source} file)
load 'blogexec.sty' and then are carried out according to their 
definitions in 'blogexec':
\begin{description}
  \cmdboxitem|\BlogInterceptExecute| \ intercepts |\EXECUTE| only. 
  \cmdboxitem|\BlogInterceptEnvironments| \ intercepts |\EXECUTE|,
    |\begin|, and |\end| only; the latter two then work much 
    as with \LaTeX. They expand to \HTML\ code as with 'blog'; 
    `\begin{<env>}' additionally executes commands according 
    to an (optional) 
    \[|\MakeBlogBeginRun{<env>}<arguments>{<begin-code>}|.\]
  \cmdboxitem|\BlogInterceptExtra| \ intercepts all the commands 
    in certain lists (using the \ctanpkgref{dowith} package), 
    including `\EXECUTE', `\begin', `\end'. E.g., 
    \[|\MakeBlogOneArgInterception{<cmd>}{<run>}{<write>}|\]
    adds <cmd> to such a list and tells that <run> should be 
    carried out and that <cmd><one-argument> should be replaced 
    by <write> in a line containing <cmd> 
    (not hidden in braces, and there better should not be much more 
     in the line).
  \cmdboxitem|\BlogInterceptHash| \ does \emph{not} choose an 
    \strong{``interception level"} as the previous commands do, 
    but may be necessary for allowing parameters in macro 
    definitions to be run in the course of an interception.
    It is automatically (``implicitly") envoked by the star forms 
    of the above commands, i.e., by either of
    \begin{itemize}
      \item |\BlogInterceptExecute*|
      \item |\BlogInterceptEnvironments*|
      \item |\BlogInterceptExtra*|
    \end{itemize}
\end{description}
The reader may find additional details in the following sections 
near the code implementing the commands.

'blogexec.sty' also modifies 'blog.sty''s (v0.7) 
|{stdallrulestable}| environment as follows:
\begin{itemize}
  \item[\InlineCmdBox{\texttt{&|}}] 
    The vertical stroke becomes an active character 
    that closes a table cell and opens another one 
    (being an alias for 'blog.sty''s v0.7 `\endcell')---just
    as |&| does it with \TeX/\LaTeX. 
  \cmdboxitem|&| The ampersand becomes an active character 
    that---\strong{differently to \TeX\slash\LaTeX}---as an alias for 
    'blog.sty''s |\figurespace| produces the Unicode figure space 
    for alignment of figures. 
\end{itemize}
\emph{Outside} the `{stdallrules}' environment, both characters 
have their ``usual" meaning, i.e., |&| may be used for accessing 
\HTML\ entities (as 'blog.sty' allows it). |\cr| (and |\endline|, 
provided by 'blog.sty' v0.7) ends a table row and starts a new one. 
|\\| is not touched---a \strong{difference to \LaTeX} and may still be 
used for breaking a line within a table cell. 

% \section{Example}

%   \pagebreak
% \section{Implementation}
\section{Package File Header (Legalize)}
\NeedsTeXFormat{LaTeX2e}[1994/12/01] %% \newcommand* etc. 
\ProvidesPackage{blogexec}[2012/12/20 v0.21
                           assignments with blog.sty (UL)] 
%% copyright (C) 2011 Uwe Lueck, 
%% http://www.contact-ednotes.sty.de.vu 
%% -- author-maintained in the sense of LPPL below.
%%
%% This file can be redistributed and/or modified under 
%% the terms of the LaTeX Project Public License; either 
%% version 1.3c of the License, or any later version.
%% The latest version of this license is in
%%     http://www.latex-project.org/lppl.txt
%% We did our best to help you, but there is NO WARRANTY. 
%%
%% Please report bugs, problems, and suggestions via 
%% 
%%   http://www.contact-ednotes.sty.de.vu 
%%
%% == Requirements == 
%% The \ctanpkgref{dowith} package is needed for managing 
%% and running lists of macros to be intercepted:
\RequirePackage{dowith}
%% Admittedly, `\do' and `\@elt' lists (as discussed in `dowith.pdf') 
%% would be faster than the 'dowith' method, which might be relevant 
%% here (TODO: how much?). I may abandon 'dowith' later, I just cannot 
%% afford removing it now (2011/11/05, TODO).
%% 
%% == Processing Source Files ==
%% With |\BlogInterceptExtra|, 'blog.sty'
%% deals with \emph{empty} input lines just like 
%% \[`\BlogCopyFile[<changes>]{<src-file>}'\] 
%% does; \emph{otherwise} the content of `\fdInputLine' is copied to 
%% |\fdOutputCode|. Before the latter is writen to the output file 
%% <output> (as determined by a recent `\ResultFile{<output>}'), 
%% |\BlogInterceptions| is run, its purpose is to extract assignment and other 
%% ``execution" commands and to turn `\fdOutputCode' into an 
%% expandable macro. We use `\def' because 'blog.sty' may have 
%% provided a preliminary definition earlier:
\def\blog@icl@xtra{%
    \let\BlogProcessLine\BlogAllowIntercepting
    \let\BlogInterceptions\AllBlogInterceptions}
\def\BlogInterceptExtra{\@ifstar@intercept@hash\blog@icl@xtra}
\def\@ifstar@intercept@hash#1{\@ifstar{#1\BlogInterceptHash}#1}
%% And this is the default setting (TODO!?):
\BlogInterceptExtra
%% Below, there are commands for restricted (faster---TODO: relevant? 
%% or less complex, to reduce danger) 
%% interception functionality. (Maybe the file should be restructured.) 
%% |\AllBlogInterceptions| first is nothing:
\InitializeListMacro\AllBlogInterceptions
%% ---and should become more below.
%% 
%% |\BlogAllowIntercepting| stores the difference to 'blog.sty': 
\newcommand*{\BlogAllowIntercepting}{%
    \let\fdOutputCode\fdInputLine
    \BlogInterceptions
%% When, after removing the intercepted command, the line is 
%% empty, it is \emph{not} written into output:
    \ifx\fdOutputCode\@empty \else 
        \WriteResult{%
            \ProcessExpandedWith\fdOutputCode\BlogOutputJob}%
%% ... enabling ``ligatures" with 'blog.sty' v0.7.  %% 2011/11/21
    \fi}
%% ... TODO: in 'fifinddo' with something like `\fdInterceptions'?
%%
%% Especially for storing file-specific macro definitions 
%% with `\EXECUTE' (below), a parameter character 
%% (usually hash mark) is needed. 
%% %% mod. 2011/11/20:
%% 'fifinddo.sty' (so far---2011/11/20) 
%% does not include it with `\BasicNormalCatCodes', 
%% and 'blog.sty' 
%% does not include it with `\BlogCodes'
%% ---the following |\BlogInterceptHash| does.
%% Moreover, |\MakeHashParameter| enables such definitions
%% when placed in a source file within the argument of a 
%% separate(!) `\EXECUTE'.
\providecommand*{\MakeHashParameter}{\catcode`\#6 }
\def\BlogInterceptHash{% 
    \ToListMacroAdd\BlogCodes\MakeHashParameter
    \MakeHashParameter}
%% TODO: default? 0-arg interception?
%%
%% == Intercepting Single-Parameter Commands ==
%% === The General Method              ===
%% Macros to be intercepted that have a single argument will be 
%% collected in |\blogOneArgInterceptions|: 
\InitializeListMacro\blogOneArgInterceptions
\ToListMacroAdd\AllBlogInterceptions{%
    \DoWithAllIn\blogTryOneArgCmd
                \blogOneArgInterceptions}
%% Here |\blogTryOneArgCmd{<cmd>}| creates a ``sandbox" for parsing 
%% in a similar way as 'fifinddo' does it, searching for <cmd>. 
%% The method there was made thinking of reading files with 
%% ``plain text" category codes, not aware of 'blog.sty'. 
%% Maybe this was a mistake, and I will reconsider it. 
%% There I also introduce a separate sandbox macro 
%% for each search pattern, thinking of different types of sandboxes. 
%% This is not done/needed here (strangely, TODO).---The sandbox 
%% starts with the parsing macro. The latter's name derives from 
%% <cmd> by prefixing something to its name.
%% |\StripEsc| is a little helper for removing the backslash 
%% from a macro name. 
\providecommand*{\StripEsc}{\expandafter\@gobble\string}
%% Name spaces:                                         %% 2012/08/29
\newcommand*{\blog@x}{\StripEsc\blogx}
\newcommand*{\blogTryOneArgCmd}[1]{%
    \csname \blog@x:\StripEsc#1\expandafter\endcsname
        \fdOutputCode \@gobble#1\@empty\@nil}
%% Here, `\@empty' is the dummy argument for <cmd>---this is what 
%% must be modified for <cmd> with more than one parameter.
%% At present (2011/11/05), that tail starting with `\@gobble' 
%% may stay at the end of `\fdOutputCode' for each interception per 
%% `\fdInputLine', until it expands to nothing in the `\write'.
%%
%% |\MakeBlogOneArgInterception{<cmd>}{<run>}{<write>}| \ \
%% says that when <cmd> is found in `\fdOutputCode', 
%% <run> should be executed, and `<cmd><arg>' should be replaced 
%% by <write> in `\fdOutputCode' where <arg> is the argument for 
%% <cmd> found in `\fdOutputCode'. Let <arc> be <arg> without 
%% delimiting braces if `<arg>' is `{<arc>}' 
%% (otherwise `<arc>' is the same as `<arg>'). 
%% Then use `#2' for referring to <arc> inside <run> and <write>. 
%% (Sorry, I cannot afford replacing `#2' by a more natural 
%%  placeholder right now.)
\begingroup 
  \catcode`\|\z@ |MakeOther|\%                  %% \z@ 2011/11/22
  |@ifdefinable|MakeBlogOneArgInterception{%
    |gdef|MakeBlogOneArgInterception#1#2#3{%
%% First we add <cmd> to `\blogOneArgInterceptions', 
%% unless it is already there:                      %% 2011/11/07
      |TestListMacroForToken|blogOneArgInterceptions#1%
      |ifin@ 
        |PackageWarning{blogexec}{Redeclaring |string#1.}%
      |else
        |ToListMacroAdd|blogOneArgInterceptions#1%
      |fi
%% Now the parsing macro is defined, together with the actions 
%% depending on the result: 
%%                          %% TODO csnames/namemod.sty 2011/11/22
      |@namedef{|blog@x:|StripEsc#1}##1#1##2##3|@nil{%
%% #3 will be empty if and only if <cmd> does \emph{not} occur 
%% in `\fdOutputCode'. A backslash made ``other" will not occur 
%% in `\fdOutputCode', therefore the following 
%% `\ifx' becomes true if and only if #3 is empty, i.e., 
%% <cmd> does \emph{not} occur in `\fdOutputCode':
          |ifx\##3\%
%% In this case we just do nothing.
          |else
%% Otherwise, we apply <run> and <write>:
              #2%
              |def|fdOutputCode{##1#3##3}%
          |fi}%
    }%
  }%
|endgroup
%%                                  %% was single `%' 2011/11/20
%% === &\EXECUTE                       ===
%% |\EXECUTE{<run>}| runs <run> and is removed from the 
%% output line:
\MakeBlogOneArgInterception\EXECUTE{#2}{}
%% You can \strong{store settings} <set> for processing a source 
%% file in this file by `\EXECUTE{<set>}' 
%% (e.g., shorthand macros only useful in this single file). 
%% You even can switch off the interception functionality 
%% after running the other settings <set> 
%% by `\EXECUTE{<set>\BlogCopyLines}'.
%%
%% `\EXECUTE{<run>}' may be a great relief thinking of pure 
%% expansion with 'blog.sty'. You may be happy enough with it 
%% and \emph{restrict} the interception functionality to `\EXECUTE'
%% by |\BlogInterceptExecute|. Its definition may be a 
%% redefinition of the preliminary macro in 'blog.sty'.
%% (TODO: option for stopping here, avoid 'dowith'.)
\def\blog@icl@exec{%
    \let\BlogProcessLine\BlogAllowIntercepting
    \def\BlogInterceptions{\blogTryOneArgCmd\EXECUTE}}
\def\BlogInterceptExecute{\@ifstar@intercept@hash\blog@icl@exec}
%%
%% === &\begin\ and &\end              === 
%% At present (2011/11/06), only |\begin{<env>}| will run   %% box 2012/08/28
%% settings. Macros `\<env>' and `\end<env>' will expand 
%% in the `.html' as with 'blog.sty' alone, not touched here. 
%% Settings to be \emph{run} must be stored in a macro 
%% `\blogx.b:<env>'. If this has not been done, only `\relax' 
%% (from `\csname') will be ``run."
\MakeBlogOneArgInterception\begin{%
%% Indeed, we have a ``modified selection" from \LaTeX's original `\begin':
    \@ifundefined{#2}%
      {\def\@tempa{\@latex@error{Environment #2 undefined}\@eha}}%
      {\def\@tempa{\def\@currenvir{#2}%
%        \edef\@currenvline{\on@line}%              %% not in source
       \csname \blog@x.b:#2\endcsname}}%  %% \StripEsc->: 2012/08/28
    \begingroup \@tempa}{%
    \csname #2\endcsname} 
%% % %% TODO single token!?                     %% rm. 2011/12/15
%% \[|\MakeBlogBeginRun{<env>}<args>{<begin-code>}|\] resembles 
%%  \[`\newenvironment*{<env>}<args>{<begin-code>}{<end-code>}'\] except that 
%% it does not have `{<end-code>}':
\newcommand*{\MakeBlogBeginRun}{\@makeblogbeginrun\newcommand}
%% v0.2 allows redefinition by 
%% \[|\ChangeBlogBeginRun{<env>}<args>{<begin-code>}|\]
\newcommand*{\@makeblogbeginrun}[2]{%
    \expandafter #1\expandafter *%
        \csname \blog@x.b:#2\endcsname}   %%  \StripEsc->: 2012/08/28
\newcommand*{\ChangeBlogBeginRun}{\@makeblogbeginrun\renewcommand}
%% Moreover, v0.2 allows copying that action by 
%% \[|\CopyBlogBeginRunTo{<env>}{<enw>}|\]
\newcommand*{\CopyBlogBeginRunTo}[2]{%
    \withcsname \let \blog@x.b:#2\expandafter\endcsname
                    \csname \blog@x.b:#1\endcsname}
%% |\end{<env>}|: 
\MakeBlogOneArgInterception\end{\@checkend{#2}\endgroup}{\end{#2}}
%    \expandafter\show\csname blogx:end\endcsname
%% \[|\BlogInterceptEnvironments|\] restricts interception functionality 
%% to `\EXECUTE', `\begin', and `\end':
\def\blog@icl@envs{%
    \BlogInterceptExecute
    \ToListMacroAdd\BlogInterceptions{%
        \blogTryOneArgCmd\begin\blogTryOneArgCmd\end}}
\def\BlogInterceptEnvironments{\@ifstar@intercept@hash\blog@icl@envs}
%% TODO: \ 1.~imitate \LaTeX's toggling with `\emph' 
%%       (redefine it in italic environments)
%%       \ 2.~code indenting (cf.~\ctanpkgref{inputtrc})
%% 
%% === Skipping Source Code            ===
%% The |{noblog}| environment ``suppresses" \TeX~source 
%% code in the sense that it does not produce \HTML~code---while
%% 'blog.sty''s `{commentlines}' produces an \HTML~comment.
\newenvironment*{noblog}{}{}    %% 2012/03/04 from ...
\MakeBlogBeginRun{noblog}{%
    \BlogInterceptEnvironments  %% 2012/06/22
    \let\WriteResult\@gobble} 
%% 
%% === A Comfortable Table Environment ===
%% As an application of |\MakeBlogBeginRun| for 'blog.sty''s
%% |{stdallrulestable}|, we provide \lq`|'\rq\ as an active character 
%% invoking 'blog.sty''s  `\endcell' (move to next cell) %% was \cr 2011/11/10
%% and an active character \lq`&'\rq\ for `\figurespace', 
%% i.e., a Unicode symbol for aligning figures. 
%% Indeed, we are \emph{not} going back to \LaTeX\ and Plain \TeX\ 
%% by using `&' for moving to the next cell, I consider the present 
%% choice more intuitive. 
\MakeBlogBeginRun{stdallrulestable}{%
    \MakeActiveDef\|{\endcell}\MakeActiveDef\&{\figurespace}}
%% I hope nobody will confuse `&' and `8'. A little drawback may be 
%% that you now can't use `&' for inserting \acro{HTML} entities. 
%% However, recall that these settings are restricted to the 
%% `{stdrulestable}' environment, and that you can use 
%% `\MakeBlogBeginRun{stdallrulestable}' again for your own choice 
%% of shorthands.
%% (TODO: `\MakeActiveLet')
%% 
%%
%% == Intercepting Two-Parameter Macros ==
%% Here especially I have a macro 
%% |\labelsection{<label>}{<title>}| \emph{in mind} 
%% (TODO).
%% It could be handled by the one-argument approach 
%% by storing the first argument and inserting another macro 
%% that reads the second argument. Therefore I am not sure ...
%% (2011/11/04)
%%
%% == Leaving and HISTORY ==
\endinput
%% \enlargethispage{2\baselineskip}                     %% 2012/08/28
      VERSION HISTORY 

v0.1    2011/11/04  started; arrived at \EXECUTE
        2011/11/05  rm. \blogx@dummy, corrected loop, 
                    \BlogInterceptExtra, \BlogInterceptExecute
        2011/11/06  \BlogAllowIntercepting, emptiness test 
                    with "other" backslash, \begin/\end
        2011/11/07  debugging (\catcode... in \@ifdefinable);
                    warning on reusing interceptor, 
                    \BlogInterceptEnvironments;
                    doc.: raise interception level in \EXECUTE
        2011/11/08  \BlogInterceptHash (understanding needed hours)
        2011/11/10  `v0.1' in \Provides..., doc. fix, 
                    removing experimental code, doc. all 1-arg 
                    interceptions in one section
        2011/11/20  \BlogInterceptHash improved
        2011/11/20  doc. `%' doubled
        2011/11/21  \BlogOutputJob
        2011/11/22  TODO + \z@ for \MakeBlogOne...
        2011/12/15  rm. TODO
v0.2    2012/08/28  \begin/\end revised (\StripEsc wrong)
        2012/08/29  \ChangeBlogBeginRun, \CopyBlogBeginRun, 
                    \blog@x
v0.21   2012/12/20  {noblog}

\end{document}

VERSION HISTORY

2011/11/04  for v0.1    very first
2011/11/09f.            something more
2011/11/20              star forms
2012/12/20  for v0.21   date checks


