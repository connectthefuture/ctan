% \iffalse meta-comment
% mylatexformat : 2011/02/12 v3.4 - Use mylatexformat.ltx to make a format based on the preamble of any LaTeX file]
%
% This work may be distributed and/or modified under the
% conditions of the LaTeX Project Public License, either
% version 1.3 of this license or (at your option) any later
% version. The latest version of this license is in
%    http://www.latex-project.org/lppl.txt
%
% This work consists of the main source file mylatexformat.dtx
% and the derived files
%    mylatexformat.ltx, mylatexformat.pdf, mylatexformat.ins
%
% Unpacking:
%    (a) If mylatexformat.ins is present:
%           etex mylatexformat.ins
%    (b) Without mylatexformat.ins:
%           etex mylatexformat.dtx
%    (c) If you insist on using LaTeX
%           latex \let\install=y% \iffalse meta-comment
% mylatexformat : 2011/02/12 v3.4 - Use mylatexformat.ltx to make a format based on the preamble of any LaTeX file]
%
% This work may be distributed and/or modified under the
% conditions of the LaTeX Project Public License, either
% version 1.3 of this license or (at your option) any later
% version. The latest version of this license is in
%    http://www.latex-project.org/lppl.txt
%
% This work consists of the main source file mylatexformat.dtx
% and the derived files
%    mylatexformat.ltx, mylatexformat.pdf, mylatexformat.ins
%
% Unpacking:
%    (a) If mylatexformat.ins is present:
%           etex mylatexformat.ins
%    (b) Without mylatexformat.ins:
%           etex mylatexformat.dtx
%    (c) If you insist on using LaTeX
%           latex \let\install=y% \iffalse meta-comment
% mylatexformat : 2011/02/12 v3.4 - Use mylatexformat.ltx to make a format based on the preamble of any LaTeX file]
%
% This work may be distributed and/or modified under the
% conditions of the LaTeX Project Public License, either
% version 1.3 of this license or (at your option) any later
% version. The latest version of this license is in
%    http://www.latex-project.org/lppl.txt
%
% This work consists of the main source file mylatexformat.dtx
% and the derived files
%    mylatexformat.ltx, mylatexformat.pdf, mylatexformat.ins
%
% Unpacking:
%    (a) If mylatexformat.ins is present:
%           etex mylatexformat.ins
%    (b) Without mylatexformat.ins:
%           etex mylatexformat.dtx
%    (c) If you insist on using LaTeX
%           latex \let\install=y% \iffalse meta-comment
% mylatexformat : 2011/02/12 v3.4 - Use mylatexformat.ltx to make a format based on the preamble of any LaTeX file]
%
% This work may be distributed and/or modified under the
% conditions of the LaTeX Project Public License, either
% version 1.3 of this license or (at your option) any later
% version. The latest version of this license is in
%    http://www.latex-project.org/lppl.txt
%
% This work consists of the main source file mylatexformat.dtx
% and the derived files
%    mylatexformat.ltx, mylatexformat.pdf, mylatexformat.ins
%
% Unpacking:
%    (a) If mylatexformat.ins is present:
%           etex mylatexformat.ins
%    (b) Without mylatexformat.ins:
%           etex mylatexformat.dtx
%    (c) If you insist on using LaTeX
%           latex \let\install=y\input{mylatexformat.dtx}
%        (quote the arguments according to the demands of your shell)
%
% Documentation:
%           (pdf)latex mylatexformat.dtx
% Copyright (C) 2010-2011 by Florent Chervet <florent.chervet@free.fr>
%<*ignore>
\begingroup
  \def\x{LaTeX2e}%
\expandafter\endgroup
\ifcase 0\ifx\install y1\fi\expandafter
         \ifx\csname processbatchFile\endcsname\relax\else1\fi
         \ifx\fmtname\x\else 1\fi\relax
\else\csname fi\endcsname
%</ignore>
%<*install>
\input docstrip.tex
\Msg{************************************************************************}
\Msg{* Installation}
\Msg{* Package 2011/02/12 v3.4 - Use mylatexformat.ltx to make a format based on the preamble of any LaTeX file*}
\Msg{************************************************************************}

\keepsilent
\askforoverwritefalse

\let\MetaPrefix\relax
\preamble

This is a generated file.

mylatexformat : 2011/02/12 v3.4 - Use mylatexformat.ltx to make a format based on the preamble of any LaTeX file

This work may be distributed and/or modified under the
conditions of the LaTeX Project Public License, either
version 1.3 of this license or (at your option) any later
version. The latest version of this license is in
   http://www.latex-project.org/lppl.txt

This work consists of the main source file mylatexformat.dtx
and the derived files
   mylatexformat.ltx, mylatexformat.pdf, mylatexformat.ins

mylatexformat - Use mylatexformat.ltx to make a format based on the preamble of any LaTeX file
Copyright (C) 2010-2011 by Florent Chervet <florent.chervet@free.fr>

\endpreamble
\let\MetaPrefix\DoubleperCent

\generate{%
   \file{mylatexformat.ins}{\from{mylatexformat.dtx}{install}}%
   \file{mylatexformat.ltx}{\from{mylatexformat.dtx}{package}}%
}

\askforoverwritefalse
\generate{%
   \file{mylatexformat.drv}{\from{mylatexformat.dtx}{driver}}%
}

\obeyspaces
\Msg{************************************************************************}
\Msg{*}
\Msg{* To finish the installation you have to move the following}
\Msg{* file into a directory searched by TeX:}
\Msg{*}
\Msg{*     mylatexformat.ltx}
\Msg{*}
\Msg{* To produce the documentation run the file `mylatexformat.dtx'}
\Msg{* through LaTeX.}
\Msg{*}
\Msg{* Happy TeXing!}
\Msg{*}
\Msg{************************************************************************}

\endbatchfile
%</install>
%<*ignore>
\fi
%</ignore>
%<*driver>
\edef\thisfile{\jobname}
\def\thisinfo{Make a format based on the preamble of any \texorpdfstring{\LaTeX{}}{LaTeX} file}
\def\thisdate{2011/02/12}
\def\thisversion{3.4}
\let\loadclass\LoadClass
\def\LoadClass#1{\loadclass[abstracton]{scrartcl}\let\scrmaketitle\maketitle\AtEndOfClass{\let\maketitle\scrmaketitle}}
{\makeatletter{\endlinechar`\^^J\obeyspaces
 \gdef\ErrorUpdate#1=#2,{\@ifpackagelater{#1}{#2}{}{\let\CheckDate\errmessage\toks@\expandafter{\the\toks@
        \thisfile-documentation: updates required !
              package #1 must be later than #2
              to compile this documentation.}}}}%
 \gdef\CheckDate#1{{\let\CheckDate\relax\toks@{}\@for\x:=#1\do{\expandafter\ErrorUpdate\x,}\CheckDate\expandafter{\the\toks@}}}}
\AtBeginDocument{\DeleteShortVerb{\|}\CheckDate{interfaces=2011/01/26,tabu=2011/01/26}}
\PassOptionsToPackage{svgnames}{xcolor}
\PassOptionsToPackage{hyperfootnotes}{hyperref}
\documentclass[a4paper,oneside]{ltxdoc}
\usepackage[latin1]{inputenc}
\usepackage[american]{babel}
\usepackage[T1]{fontenc}
\usepackage{etex,etoolbox,geometry,lastpage,xspace,xcolor,bbding,txfonts,atveryend,moresize,relsize}
\usepackage{tocloft,titlesec,graphicx,fancyhdr,framed,multirow,makecell,tabu,enumitem,needspace}
\usepackage{holtxdoc,bookmark,embedfile,tabu,keycommand}
\usepackage{microtype}
\usepackage{interfaces}
\usetikz{basic}
\csname endofdump\endcsname
\CodelineNumbered
\usepackage{fancyvrb}\fvset{gobble=1,listparameters={\topsep=0pt}}
\lastlinefit999
\geometry{top=0pt,includeheadfoot,headheight=7mm,headsep=.6cm,bottom=.6cm,footskip=.5cm,left=4cm,right=14mm}
\hypersetup{%
  pdftitle={The mylatexformat package},
  pdfsubject={Use mylatexformat.ltx to make a format based on the preamble of any LaTeX file},
  pdfauthor={Florent CHERVET},
  colorlinks,linkcolor=reflink,
  pdfstartview={FitH},
  pdfkeywords={tex, e-tex, latex, package, format, formats, pdfTeX},
  bookmarksopen=true,bookmarksopenlevel=2}
\embedfile{\thisfile.dtx}
\begin{document}
   \DocInput{\thisfile.dtx}
\end{document}
%</driver>
% \fi
%
% \CheckSum{317}
%
% \CharacterTable
%  {Upper-case    \A\B\C\D\E\F\G\H\I\J\K\L\M\N\O\P\Q\R\S\T\U\V\W\X\Y\Z
%   Lower-case    \a\b\c\d\e\f\g\h\i\j\k\l\m\n\o\p\q\r\s\t\u\v\w\x\y\z
%   Digits        \0\1\2\3\4\5\6\7\8\9
%   Exclamation   \!     Double quote  \"     Hash (number) \#
%   Dollar        \$     Percent       \%     Ampersand     \&
%   Acute accent  \'     Left paren    \(     Right paren   \)
%   Asterisk      \*     Plus          \+     Comma         \,
%   Minus         \-     Point         \.     Solidus       \/
%   Colon         \:     Semicolon     \;     Less than     \<
%   Equals        \=     Greater than  \>     Question mark \?
%   Commercial at \@     Left bracket  \[     Backslash     \\
%   Right bracket \]     Circumflex    \^     Underscore    \_
%   Grave accent  \`     Left brace    \{     Vertical bar  \|
%   Right brace   \}     Tilde         \~}
%
% \DoNotIndex{\begin,\CodelineIndex,\CodelineNumbered,\def,\DisableCrossrefs,\~,\@ifpackagelater}
% \DoNotIndex{\ttfamily,\tiny,\sffamily,\small,\normalfont,\newindex,\makeindex,\makeglossary,\LARGE,\Large,\large,\itshape,\immediate}
% \DoNotIndex{\&,\/,\=,\@dofilelist,\@spaces,\@undefined,\^,\active,\bfseries,\boldmath,\etex@primitive@openout,\footnotesize,\g@addto@macro,\Huge}
% \DoNotIndex{\DocInput,\documentclass,\EnableCrossrefs,\end,\GetFileInfo}
% \DoNotIndex{\NeedsTeXFormat,\OnlyDescription,\RecordChanges,\usepackage}
% \DoNotIndex{\ProvidesClass,\ProvidesPackage,\ProvidesFile,\RequirePackage}
% \DoNotIndex{\filename,\fileversion,\filedate,\let}
% \DoNotIndex{\@listctr,\@nameuse,\csname,\else,\endcsname,\expandafter}
% \DoNotIndex{\gdef,\global,\if,\item,\newcommand,\nobibliography}
% \DoNotIndex{\par,\providecommand,\relax,\renewcommand,\renewenvironment}
% \DoNotIndex{\stepcounter,\usecounter,\nocite,\fi}
% \DoNotIndex{\@fileswfalse,\@gobble,\@ifstar,\@unexpandable@protect}
% \DoNotIndex{\AtBeginDocument,\AtEndDocument,\begingroup,\endgroup}
% \DoNotIndex{\frenchspacing,\MessageBreak,\newif,\PackageWarningNoLine}
% \DoNotIndex{\protect,\string,\xdef,\ifx,\texttt,\@biblabel,\bibitem}
% \DoNotIndex{\z@,\wd,\wheremsg,\vrule,\voidb@x,\verb,\bibitem}
% \DoNotIndex{\FrameCommand,\MakeFramed,\FrameRestore,\hskip,\hfil,\hfill,\hsize,\hspace,\hss,\hbox,\hb@xt@,\endMakeFramed,\escapechar}
% \DoNotIndex{\do,\date,\if@tempswa,\@tempdima,\@tempboxa,\@tempswatrue,\@tempswafalse,\ifdefined,\ifhmode,\ifmmode,\cr}
% \DoNotIndex{\box,\author,\advance,\multiply,\Command,\outer,\next,\leavevmode,\kern,\title,\toks@,\trcg@where,\tt}
% \DoNotIndex{\the,\width,\star,\space,\section,\subsection,\textasteriskcentered,\textwidth}
% \DoNotIndex{\",\:,\@empty,\@for,\@gtempa,\@latex@error,\@namedef,\@nameuse,\@tempa,\@testopt,\@width,\\,\m@ne,\makeatletter,\makeatother}
% \DoNotIndex{\maketitle,\parindent,\setbox,\x,\kernel@ifnextchar}
% \DoNotIndex{\KVS@CommaComma,\KVS@CommaSpace,\KVS@EqualsSpace,\KVS@Equals,\KVS@Global,\KVS@SpaceEquals,\KVS@SpaceComma,\KVS@Comma}
% \DoNotIndex{\DefineShortVerb,\DeleteShortVerb,\UndefineShortVerb,\MakeShortVerb,\endinput}
%
% \let\ClearPage\clearpage
% \makeatletter
% \catcode`\� \active   \def�{\@ifnextchar �{\par\nobreak\vskip-2\parskip}{\par\nobreak\vskip-\parskip}}
% \newrobustcmd\Xspace{\ifnum\catcode`\ =\active\else\expandafter\xspace\fi}
% \newrobustcmd*\stform{\ifincsname\else\expandafter\@stform\fi}
% \newrobustcmd*\@stform{\@ifnextchar*{\@@stform[]\textasteriskcentered\@gobble}\@@stform}
% \newrobustcmd*\@@stform[2][\string]{\textttbf{#1#2}\Xspace}
%
% \newrobustcmd*\thispackage{\@ifstar
%  {\xpackage{{\thisfile}}\Xspace}
%  {\xpackage{{\pkgcolor\thisfile}}\Xspace}}
% \def\ThisPackage{\Xpackage{\thisfile}\xspace}
% \def\CTANpackage#1{\href{http://www.tex.ac.uk/tex-archive/help/Catalogue/entries/#1.html}{\xpackage{#1}}}
% \def\Xpackage{\@dblarg\X@package}
% \def\X@package[#1]#2{\@testopt{\X@@package{#1}{#2}}{}}
% \def\X@@package#1#2[#3]{%
%     \xpackage{#2\footnote{\noindent\xpackage{#2}: \href{http://www.ctan.org/tex-archive/macros/latex/contrib/#1}{\nolinkurl{CTAN:macros/latex/contrib/#1}}#3}}}
% \def\Underbrace#1_#2{$\underbrace{\vtop to2ex{}\hbox{#1}}_{\footnotesize\hbox{#2}}$}
% \newrobustcmd*\FC[1][\color{copper}]{{#1\usefont{T1}{fts}xn FC}}
%
% \parindent\z@\parskip.4\baselineskip\topsep\parskip\partopsep\z@
% \g@addto@macro\macro@font{\macrocodecolor\let\AltMacroFont\macro@font}
% \g@addto@macro\@list@extra{\parsep\parskip\topsep\z@\itemsep\z@}
% \def\smex{\leavevmode\hb@xt@2em{\hfil$\longrightarrow$\hfil}}
% \newrobustcmd\verbfont{\usefont{T1}{\ttdefault}{\f@series}{n}}    \let\vb\verbfont
% \renewrobustcmd\#[1]{{\usefont{T1}{pcr}{bx}{n}\char`\##1}}
% \newrobustcmd\csred[1]{\textcolor{red}{\cs{#1}}}
% \renewrobustcmd\cs[2][]{\mbox{\vb#1\expandafter\@gobble\string\\#2}}
% \newrobustcmd\CSbf[1]{\textbf{\CS{#1}}}
% \newrobustcmd\csbf[2][]{\textbf{\cs[{#1}]{#2}}}
% \newrobustcmd\textttbf[1]{\textbf{\texttt{#1}}}
% \renewrobustcmd*\bf{\bfseries}\newcommand\nnn{\normalfont\mdseries\upshape}\newcommand\nbf{\normalfont\bfseries\upshape}
% \newrobustcmd*\blue{\color{blue}}\newcommand*\red{\color{dr}}\newcommand*\green{\color{green}}\newcommand\rred{\color{red}}
% \newrobustcmd\rrbf{\color{red}\bfseries}
% \newcommand\textt[1]{\texorpdfstring{\texttt{#1}}{#1}}
% \newrobustcmd\texttbf[1]{\textbf{\texttt{#1}}}
% \colorlet{pkgcolor}{teal}\def\pkgcolor{\color{pkgcolor}}
% \definecolor{copper}{rgb}{0.67,0.33,0.00}  \newcommand\copper{\color{copper}}
% \definecolor{dg}{rgb}{0.16,0.33,0.00}      \newcommand\dg{\color{dg}}
% \definecolor{db}{rgb}{0,0,0.502}           \newcommand\db{\color{db}}
% \definecolor{dr}{rgb}{0.49,0.00,0.00}      \let\dr\red
% \definecolor{lk}{rgb}{0.25,0.25,0.25}      \newrobustcmd\lk{\color{lk}}
% \newrobustcmd\bk{\color{black}}\newcommand\md{\mdseries}
% \def\M{\@ifstar{\M@i\@firstofone}{\M@i\meta}}
% \def\M@i#1{\@ifnextchar[^^A]
%   {\M@square#1}
%   {\@ifnextchar(^^A)
%     {\M@paren#1}
%     {\M@brace#1}}^^A
% }
% \def\M@square#1[#2]{\M@Bracket[{#1{#2}}]}
% \def\M@paren#1(#2){\M@Bracket({#1{#2}})}
% \def\M@brace#1#2{\M@Bracket\{{#1{#2}}\}}
% \def\M@Bracket#1#2#3{{\ttfamily#1#2#3}}
% \newrobustcmd*\thisyear{\begingroup
%    \def\thisyear##1/##2\@nil{\endgroup
%       \oldstylenums{\ifnum##1=2010\else 2010\,\textendash\,\fi ##1}^^A
%    }\expandafter\thisyear\thisdate\@nil
% }
%
% \def\declmargin{\hspace*\declmarginwidth }
% \def\declmarginwidth{\dimexpr -\leftmargini +\arrayrulewidth +\tabcolsep\relax}
% \renewrobustcmd\declcs[2][]{^^A
%   \if@nobreak \par\nobreak
%   \else \par\addvspace\parskip
%         \Needspace{.08\textheight}\fi
%   \changefont{size+=2.5pt,spread=1,fam=\ttdefault}^^A
%   \newcommand*\csanchor[2][]{\raisedhyperdef[14pt]{declcs}{##2}{\cs[{##1}]{##2}}}^^A
%   \def\*{\unskip\,\texttt{*}}\noindent
%   \hskip-\leftmargini
%   \begin{tabu}{|l|}\hline
%     \expandafter\SpecialUsageIndex\csname #2\endcsname
%     \csanchor[{#1}]{#2}}
% \renewcommand\enddeclcs{%
%     \crcr \hline \end{tabu}\nobreak
%     \par  \nobreak \noindent
%     \ignorespacesafterend
%  }
% \newrobustcmd*\csref[2][]{\hyperref{}{declcs}{#2}{\cs[{#1}]{#2}}}
%
% \let\plainllap\llap
% \newrobustcmd\macro@llap[1]{{\global\let\llap\plainllap
%  \setbox0=\hbox\bgroup \raisedhyperdef{macro}{\saved@macroname}{#1}\egroup
%  \ifdim\wd0>2.4cm
%     \hbox to\z@ \bgroup\hss \hbox to2.4cm{\unhcopy0\hss}\egroup
%     \edef\@tempa{\hskip\dimexpr\the\wd0-2.4cm}\global\everypar\expandafter{\the\expandafter\everypar
%                                                                            \@tempa \global\everypar{}}^^A
%  \else \llap{\unhbox0}\fi}}
% \pretocmd\macro{\if@nobreak\else\Needspace{2\baselineskip}\fi
%           \MacrocodeTopsep\z@skip \MacroTopsep\z@skip \parsep\z@ \topsep\z@ \itemsep\z@ \partopsep\z@
%           \let\llap\macro@llap}{}{}
% \apptocmd\endmacro{\goodbreak\vskip.3\parskip}{}{}
% \newrobustcmd\macroref[2][]{\hyperref{}{macro}{#2}{\cs[{#1}]{#2}}}
%
% \DefineVerbatimEnvironment{Verb}{Verbatim}{gobble=1}
% \DefineVerbatimEnvironment{VerbLines}{Verbatim}{gobble=1,frame=lines,framesep=6pt,fontfamily=txtt,fontseries=b}
% \newkeyenvironment+[\|]{dblruled}[first=.4pt,second=.4pt,sep=1pt,left=\z@]%
%   {|\def\FrameCommand|{%
%     \vrule\@width\commandkey{first}
%     \hskip\commandkey{sep}
%     \vrule\@width\commandkey{second}
%     |\hspace|{\commandkey{left}}}
%   |\MakeFramed| {\advance\hsize-|\width \FrameRestore|}%
%  }{%
%   |\endMakeFramed|%
%  }
%
% \colorlet{HeadBand}{BlanchedAlmond}
% \newcommand*\topheader{\tikz[remember picture,overlay]{
%   \coordinate (A) at (current page.north west);
%   \coordinate (B) at ($(current page.north east)+(0cm,-\headheight)$);
%   \fill [color=HeadBand,fill opacity=.6] (A) rectangle (B);
%   }%
% }
%
% \pagesetup[plain]{
%     norules,
%     foot/right=\oldstylenums{\thepage} / \oldstylenums{\pageref{LastPage}},
%     foot/font=\lk\scriptsize,
%     foot/left/offset=3cm,
%     foot/right/offset=1cm,
%     foot/left=\vbox to\baselineskip{\vss{{\rotatebox[origin=l]{90}{\thispackage\,[rev.\thisversion]\,\copyright\,\thisyear\,\lower.4ex\hbox{\pkgcolor\NibRight}\,\FC\quad \xemail{florent.chervet at free.fr}}}}},
% }
% \pagesetup{
%     foot/right={\oldstylenums{\thepage} / \oldstylenums{\pageref{LastPage}}},
%     foot/font=\lk\scriptsize,
%     foot/left/offset=3cm,
%     foot/right/offset=1cm,
%     foot/left=\vbox to\baselineskip{\vss{{\rotatebox[origin=l]{90}{\thispackage\,[rev.\thisversion]\,\copyright\,\thisyear\,\lower.4ex\hbox{\pkgcolor\NibRight}\,\FC}}}},
%     head/font=\small\sffamily,
%     head/right/offset=1cm,
%     head/right=\noindent\raise.4ex\hbox{\thispackage : \thisinfo},
%     head/left=\topheader,
% }
% \sectionformat\section{
%     top=\medskipamount,
%     bottom=.5\smallskipamount,
%     bookmark={color=MidnightBlue},
% }
% \sectionformat\subsubsection{
%     top=\smallskipamount,bottom=0pt,
%     bookmark={color=lk},
% }
% \bookmarksetup{openlevel=3}
%
% \newcommand\macrocodecolor{\color{macrocode}}\definecolor{macrocode}{rgb}{0.05,0.02,0.07}
% \newcommand\reflinkcolor{\color{reflink}}\definecolor{reflink}{rgb}{0.49,0.00,0.00}
% \newrobustcmd*\IMPLEMENTATION{%
%     \bookmarksetup{openlevel=1}
%     ^^A\geometry{top=0pt,includeheadfoot,headheight=7mm,headsep=.6cm,bottom=.6cm,footskip=.5cm,left=4cm,right=1.5cm}
%     \newgeometry{top=0pt,includeheadfoot,headheight=7mm,headsep=.6cm,bottom=.6cm,footskip=.5cm,left=3cm,right=5mm}
%     \pagesetup*{left/offset-=1cm,right/offset-=1.5cm-2mm}
%     \section{Implementation} \label{sec:implementation}
% }
%
% \makeatother
%
% \deffootnote{1em}{0pt}{\rlap{\textsuperscript{\thefootnotemark}}\kern1em}
%
% \title{\vspace*{-28pt}\href{http://www.tex.ac.uk/tex-archive/help/Catalogue/entries/mylatexformat.html}{\HUGE\bfseries\sffamily\pkgcolor mylatexformat}\vspace*{6pt}}
% \date{}
% \subtitle{\LARGE\begin{tabu}{X[c]}
%           Use \thispackage* to make a format \\
%           based on the preamble of any LaTeX file \\[-1ex]
%           \small\FC
%           \end{tabu}
%           \vspace*{-12pt}}
% \author{\smaller\thisdate~--~\hyperref[\thisversion]{version \thisversion}}
%
% \maketitle
%
% {\makeatletter\let\@thefnmark\@empty\let\@makefntext\@firstofone
% \footnotetext{\noindent\parskip=0pt\scriptsize
% This documentation is produced with the \textt{DocStrip} utility.\par
% \begin{tabu}{X[-3]X[-1]X}
% \smex To get the package,                   &run:          &\texttt{etex \thisfile.dtx}                  \\
% \smex To get the documentation              &run (thrice): &\textt{pdflatex \thisfile.dtx}               \\
% \leavevmode\hphantom\smex To get the index, &run:          &\texttt{makeindex -s gind.ist \thisfile.idx}
% \end{tabu}�
% The \xext{dtx} file is embedded into this pdf file thank to \Xpackage{embedfile} by H. Oberdiek.}
% }
%
% \deffootnote{1em}{0pt}{\rlap{\thefootnotemark.}\kern1em}
%
% {\let\quotation\relax\let\endquotation\relax\vspace*{-1cm}%
% \begin{abstract}\parskip\smallskipamount\parindent0pt\lastlinefit0\leftskip.7cm\rightskip\leftskip
%
% \thispackage can be used to make a format from the preamble of any \LaTeX{} file.
% The use of formats helps to speed up compilations: packages which have been
% dumped in the format are loaded at very high speed. A format (also known as \emph{dump}) is therefore very suitable:
% \begin{itemize}[label=\textendash,leftmargin=3em]
% \item if you have many documents that share the same preamble,
% \item if you have a document which loads a large amount of packages (like \xpackage{TikZ} graphics libraries).
% \end{itemize}
%
% \medskip
%
% \thispackage will dump all definitions until \cs{begin}\M*{document} or \csref{endofdump}, more precisely:
%
% \csdef{tabu*}{\begingroup\tabcolsep0pt\begin{tabu}}
% \csdef{endtabu*}{\end{tabu}\endgroup}
% \definecolor{ly}{rgb}{1,1,.35}
%
% \hskip-2em
% \begin{tabu}to \linewidth{XXX[1.2]}
%
% \fcolorbox{black}{ly}{%
% \begin{tabu*}{X}
% \cs{RequirePackage}\M*{...} \\
% \cs{documentclass}\M*{...}  \\
% \cs{usepackage}\M*{...}     \\
% \noindent\,\,...       \\
% \noindent\,\,...       \\
% \csbf{begin}{\bfseries\M*{document}}
% \end{tabu*}}
% &
% \begin{tabu*}{X}
% \fcolorbox{black}{ly}{%
% \begin{tabu}{X}
% \cs{RequirePackage}\M*{...} \\
% \cs{documentclass}\M*{...}  \\
% \cs{usepackage}\M*{...}     \\
% \csbf{endofdump}       \\
% \end{tabu}} \\
% \noindent\,\, ... \\
% \noindent\,\,\cs{begin}\M*{document}
% \end{tabu*}
% &
% \begin{tabu*}{X}
% \fcolorbox{black}{ly}{%
% \begin{tabu*}{X}
% \cs{RequirePackage}\M*{...} \\
% \cs{documentclass}\M*{...}  \\
% \cs{usepackage}\M*{...}     \\
% \csbf{csname} \texttbf{endofdump}\csbf{endcsname}       \\
% \end{tabu*}} \\
% \noindent\,\, ... \\
% \noindent\,\,\cs{begin}\M*{document}
% \end{tabu*}
% \end{tabu}
% \medskip
%
% The colored parts correspond to the format. The \xext{log} file will report:
%
% \begin{tabu}{@{}X|[pkgcolor]X}
% \begin{tabu}{>{\ttfamily}X[l]}
% start reading document "my document" \\
% on input line *** (\cs{begin}\M*{document})
% \end{tabu}
% &
% \begin{tabu}{>{\ttfamily}X[l]}
% start reading document "my document" \\
% on input line *** (\cs{endofdump})
% \end{tabu}
% \end{tabu}
% \medskip
%
% \thispackage was primarily written from \CTANpackage{mylatex}
% by David Carlisle, but it is different in many points (see \ref{mylatex}). In particular, \thispackage allows to put almost
% any package in the format, with one exception: the package \CTANpackage{minitoc}
% cannot be included in the format (you must use \csref{endofdump} before \cs{usepackage}\M*{minitoc}).
%
% \end{abstract}
% }
%
% \tocsetup{
%     section/skip=4pt plus2pt minus2pt,
%     subsection/skip=4pt plus2pt minus2pt,
%     title=Contents,
%     title/skip=2pt,
%     title/after={\pkgcolor\quad\leaders\vrule height3.4pt depth-3pt\hfill\null},
%     multicols/afterend=\leavevmode{\pkgcolor\hrule},
%     twocolumns,
%     bookmark={text=Contents,bold},
% }
% \tableofcontents
% \clearpage
%
%
% \MakeShortVerb{\+}
%
% \section{Creating a format}
%
% \subsection{Using MikTeX}
%
% \subsubsection{With preloaded pdflatex format (pdf output)}
%
% Use the following command line:
%
% \quad+etex -initialize [opts] "&pdflatex" mylatexformat.ltx """abc.tex"""+
%
% You need 3 quotes around your \xext{tex} file if its name contains space
% (only one quote if it does not contain any space, but 3 quotes is more general...)
% \medskip
%
% Full example:��
% \begin{VerbLines}
%  etex -initialize -save-size=20000 -stack-size=20000
%                   -jobname="The-Name-Of-The-.fmt-File(without .fmt extension)"
%                   "&pdflatex" mylatexformat.ltx """Mon Fichier TeX.tex"""
% \end{VerbLines}
%
%
% \subsubsection{With preloaded latex format (dvi output)}
%
% Use the following command line:
%
% \quad+etex -initialize [opts] "&latex" mylatexformat.ltx """abc.tex"""+
%
%
% \subsubsection{With preloaded format xelatex (pdf output)}
%
% Use the following command line:
%
% \quad+xetex -initialize [opts] "&xelatex" mylatexformat.ltx """abc.tex"""+
%
%
% \subsection{Using other distributions}
%
% As I pointed out in the \texttt{abstract}, I have not been able to test the file on other distributions.
%
% Some users told me they typed the following command line:
%
% +latex -ini -jobname="fmt-file-name" "&latex" mylatexformat.ltx  """abc.tex"""+
%
% \section{Using a format}
%
% \subsection{Using MikTeX}
%
% Add the format to the very first line of your LaTeX file :
%
% \quad+%&"Mon Beau Format"+
%
% You may use the $\varepsilon$-\TeX{} option :    +-dont-parse-first-line+ \quad to avoid loading the format.
%
% \subsection{Using other distributions}
%
% As I pointed out in the \texttt{abstract}, I have not been able to test the file on other distributions.
%
% \clearpage
% \section[How does it work ? (and why it is different from mylatex.ltx)]{How does it work ? (and why it is different from \xpackage{mylatex.ltx})}
% \label{mylatex}
%
% \xpackage{mylatex.ltx} has three important limitations:
% \begin{enumerate}[label=\arabic*)~]
% \item The end of the format is either \verb+\begin{document}+ or the comment \verb+%mylatex+\\
%       Therefore, after the format has been loaded, \TeX{} checks the commentaries in the preamble
%       in order to skip everything before \verb+%mylatex+. \\
%       But this way, braces \textbf{must match inside commentaries !}
% \item \xpackage{mylatex.ltx} temporarily redefines the command \cs{document} in order to stop
%      the creation of the format (execution of \cs{dump}) at begin document. \\
%      However, the \textbf{initial definition of the macro \csbf{document}} is saved
%      and reloaded after the format has been loaded. \\
%      Therefore, packages that modifies this macro (such as \Xpackage[oberdiek/auxhook]{auxhook}[ H. Oberdiek] or
%      \Xpackage{etoolbox}[ P. Lehman] cannot be included in the format!
% \item If your document starts with :\\
%              \verb+\begin{document}% commentary+ or with :\\
%              \verb+\begin{document}\makeatletter+ \\
%      the format is not stopped at \verb+\begin{document}+ and an error occurs.
% \end{enumerate}
%
% \begin{dblruled}[first=2pt,sep=1pt,second=.6pt,left=4pt]
% With \xpackage{mylatexformat.ltx}, those limitations are lifted.
%
% \xpackage{mylatexformat.ltx}:
% \begin{enumerate}[label=\arabic*)~]
% \item Temporarily redefines \cs{begin} in order to check if its argument is \meta{document}. \\
%       Therefore, packages like \xpackage{auxhook} or \xpackage{etoolbox} can be included in the format.
% \item Stops the format when it encounters \verb+\begin{document}+ or \cs{endofdump}.
% \item Does not read inside commentaries
% \item After the format has been loaded, each line is checked to see if it \textbf{contains} either:
%        \begin{description}[align=left,labelsep=1cm,font=\bfseries,labelindent=1cm]
%        \item[] {\dr\verb+\begin{document}+}
%        \item[or: ] {\dr\verb+\endofdump+}
%        \item[or: ] {\dr\verb+\csname endofdump\endcsname+}\textsuperscript{\bfseries\textasteriskcentered}
%        \end{description}
%      Report is written in the \xext{log} file, with the job name, the name of the preloaded format,
%      the list of preloaded files and the first input line read after loading of format.
% \end{enumerate}
%
% {\footnotesize\textsuperscript{\bfseries\textasteriskcentered} this way, the command \verb+\csname endofdump\endcsname+ is silently ignored
% (equiv. to \cs{relax}) in case the format is not used.}
% \end{dblruled}
%
%
%
%
% \StopEventually{
% }
%
% \IMPLEMENTATION\DeleteShortVerb{\+}
%
%    \begin{macrocode}
%<*package>
%%% _____________________________________________________________________
%%%                           mylatexformat.ltx
%%%                           ^^^^^^^^^^^^^^^^^
%%% From an original idea by David Carlisle - mylatex.ltx 1994/09/27
%%% � lppl 2010-2011 - F Chervet - 2011/02/12 v3.4
%%% _____________________________________________________________________
%%%
%%% Use this file to make a format based on the preamble of any LaTeX
%%% file.
%%%
%%% The format is made from the very beginning of the file up to either:
%%%   i) the command : \endofdump
%%%                    placed in the preamble
%%%  ii) the command : \csname endofdump\endcsname
%%%                    placed in the preamble
%%% iii) \begin{document}
%%%                    if \endofdump nor \csname endofdump\endcsname
%%%                    haven't been found in the preamble.
%%%
%%% A format helps to speed up compilations, packages which have been
%%% dumped in the format are loaded at very high speed.
%%% --------------------------------------------------------------------
%%% There are no restrictions on the distribution or modification of
%%% this file, except that other people should not attempt to alter
%%% the master copy on the ctan archives.
%%%
%%% Making the format --------------------------------------------------
%%% ^^^^^^^^^^^^^^^^^                                       USING MikTeX
%%% *  With preloaded pdflatex format (pdf output) :
%%%    etex -initialize [opts] "&pdflatex" mylatexformat.ltx """abc.tex"""
%%%    => compile your document using :
%%%            pdflatex
%%%         or texify --tex-option=-output-format=pdf
%%%
%%% *  With preloaded latex format (dvi output) :
%%%    etex -initialize [opts] "&latex" mylatexformat.ltx """abc.tex"""
%%%    => compile your document using
%%%            latex
%%%         or texify --tex-option=-output-format=dvi
%%%
%%% * With preloaded format xelatex (pdf output) :
%%%    xetex -initialize [opts] "&xelatex" mylatexformat.ltx """abc.tex"""
%%%    => compile your document using xelatex
%%%
%%% NOTA BENE:    Please, notice the 3 quotes !
%%% ^^^^^^^^^^
%%%
%%% In fact, 3 quotes are necessary only if you "file name.tex" contains a space !
%%%
%%% -=-=-=-=-=-=-=-=-=-=-=-=-=-=-=-=-=-=-=-=-=-=-=-=-=-=-=-=-=-=-=-=-=-=
%%% Exemple of a command line for making format using pdfTeX :
%%%
%%%   etex -initialize -interaction=nonstopmode
%%%                    -save-size=20000 -stack-size=20000
%%%                    -jobname="Mon Beau Format Perso"
%%%                    "&pdflatex" mylatexformat.ltx "Mon Fichier TeX.tex"
%%%
%%%   "Mon Beau Format Perso.fmt" (en Fran�ais dans le texte) will be created
%%%   (or overwritten) in the current working directory.
%%%
%%% Using the format ---------------------------------------------------
%%% ^^^^^^^^^^^^^^^^^                                       USING MikTeX
%%% Add the format to the very first line of your LaTeX file :
%%% --> first line :       %&"format file name"
%%% without the .fmt extension.
%%% [You may use the option :    -dont-parse-first-line option
%%%  to avoid loading the format]
%%% --------------------------------------------------------------------
%% ____________________________________________________________________
%% The following informations come from mylatex.ltx from David Carlisle:
%% Given a LaTeX file `abc.tex', use initex as follows:
%%
%% initex &latex  mylatexformat.ltx abc      (Generic TeX)
%% initex \&latex mylatexformat.ltx abc      (unix and other TeX's)
%% tex /i &latex  mylatexformat.ltx abc      (emtex)
%%
%% If you are on a Mac or using some shell that makes it inconvenient
%% to use a command line such as the above examples then you may
%% make a file `mylatex.tex' with the single line
%% \input mylatex.ltx abc
%% and then pass the file mylatex.tex to your (ini)tex shell to produce
%% the format, ie something equivalent to initex &latex mylatex.tex.
%%
%% If you are using OzTeX, see the separate instructions below.
%%
%% This should make a format file mylatex.fmt which you can then use
%% as follows
%%
%% Using the new format
%% ^^^^^^^^^^^^^^^^^^^^
%%
%% tex  &mylatex abc          (generic TeX)
%% virtex \&mylatex abc       (Unix TeX)
%%
%% This will process your document, abc.tex, just as LaTeX does, but
%% quicker as the contents of the preamble will be stored in the
%% format file and will not need to be run each time.
%%
%% If (vir)tex fails to find your mylatex.fmt then it is not searching
%% in the current directory, either modify your TEXFORMATS path (or
%% equivalent) to search `.' or (on unix/dos) use   ./  as in:
%% virtex \&./mylatex abc
%%
%% Making and using the format with OzTeX
%% ^^^^^^^^^^^^^^^^^^^^^^^^^^^^^^^^^^^^^^
%%
%% Given a LaTeX file `abc.tex', do this: select `TeX...' from OzTeX's
%% TeX menu, and go to the folder containing the file `abc.tex' as if you
%% were selecting the file `abc.tex'. Then hit the Cancel button - this
%% procedure sets OzTeX's working folder to the one containing abc.tex.
%% Next, select iniTeX from the TeX menu, and type:
%%
%% &latex mylatex.ltx abc
%%
%% This should make a format file mylatex.fmt which you can save in the
%% same folder as the file you're working on.
%%
%% To use the new format, put this at the very start of the very first
%% line of your document:
%%
%% %&mylatex
%%
%% Further Notes
%%
%% The file abc.tex must contain a line *just* with
%% \begin{document}
%% Everything up to (but not including) the \begin{document} will
%% be saved in the format and not executed in subsequent runs.
%%
%% If you are modifying the document (or working on a similar document)
%% you may wish to add new commands to your document preamble.
%% The `mylatex' format normally skips the whole preamble (believing
%% it to be pre-loaded) and so such new commands do not take effect.
%% You could re-make the format, preloading the new preamble, but that
%% might be inconvenient to do every time, and so an alternative scheme
%% has been introduced.
%% If the preamble contains a comment   mylatex  (ie a line just
%% containing a % white space and the word mylatex) then the mylatex
%% format will start reading the preable at that point so any new
%% commands can be placed after such a comment and they will be
%% executed.
%% --------------------------------------------------------------------
%%
%%\RequirePackage{etex}% RequirePackage may be used before the class
%%
%% In principle \openout stream= filename need not be space terminated,
%% and need not be immediate, but this covers \makeindex \makeglossary
%% and index package's \newindex which are all the cases of \openout
%% that occur before \begin{document} that I could see.
%% Thanks to Ross Moore for pointing out \AtBeginDocument is too late
%% eg changebar package *closes* the stream in \AtBeginDocument, so need
%% to make sure it is opened before that. Make a special purpose hook.
\makeatletter
%    \end{macrocode}
%
%
%    \begin{macro}{\begin}
%
% \cs{begin} is overloaded in order to stop \cs{dump} at begin document
% if the macro \cs{endofdump} is not given somewhere in the preamble.
% Rem: it is NOT POSSIBLE to redefine \cs{document}, for the macro
%      \cs{document} is changed or patched by some classes or packages
%      (\xpackage{auxhook}, \xpackage{etoolbox}  etc.)
%
% With \Xpackage{etoolbox} version \textt{2.1} and earlier, \cs{begin} is patched,
% in order to insert the hooks \cs{AtBeginEnvironment} and \cs{AtEndEnvironment}.
%
% Therefore, it's not possible to restore, at end of dump, the orginal definition
% of \cs{begin} in the \LaTeX{} kernel. Instead, \thispackage inserts a reversible
% patch of the \cs{begin} command: the patched is reversed at end of dump by
% \macroref{MYLATEX@cleanup}.
%
%    \begin{macrocode}
%% \begin is overloaded in order to stop \dump at begin document
%% if the macro \endofdump is not given somewhere in the preamble.
%% Rem: it is NOT POSSIBLE to redefine \document, for the macro
%%      \document is changed or patched by some classes or packages
%%      (auxhook, etoolbox etc.)
\begingroup \def\x #1{\endgroup
   \gdef\begin ##1{\MYLATEX@StopAtdocument{##1}#1}
}\expandafter\x\expandafter{\begin{#1}}
\def\MYLATEX@StopAtdocument #1{\expandafter
   \ifx\csname #1\endcsname\document \expandafter\endofdump \fi
}% \MYLATEX@StopAtdocument
%    \end{macrocode}
%    \end{macro}
%
%    \begin{macro}{\openout}
%
%    Files cannot be opened out during the building of the format.
%    Instead, \cs{openout} is delayed until the end of the dump.
%
%    \begin{macrocode}
\let\MYLATEX@openout \openout
\def\openout#1 {\g@addto@macro\MYLATEX@opens{\immediate\openout#1 }}
\def\MYLATEX@opens {\let\MYLATEX@opens \@undefined}
%    \end{macrocode}
%    \end{macro}
%
%    \begin{macro}{\MYLATEX@cleanup}
%
%    Final clean up done at \cs{endofdump}: reverse the path for \cs{begin}
%    and restores \cs{openout} primitive.
%
%    \begin{macrocode}
\def\MYLATEX@cleanup {\let\MYLATEX@cleanup \@undefined
   \def\MYLATEX@StopAtdocument ##1\MYLATEX@StopAtdocument ##2##3\@nil{%
      \gdef\begin ####1{##1##3}%
   }\expandafter\MYLATEX@StopAtdocument \begin{##1}\@nil
   \let\openout                \MYLATEX@openout
   \let\MYLATEX@openout        \@undefined
   \let\MYLATEX@StopAtdocument \@undefined
}% \MYLATEX@cleanup
%    \end{macrocode}
%    \end{macro}
%
%
%    \begin{macro}{\MYLATEX@fontpreloading}
%
%    Some font can be preloaded: this gives an advantage to the format.
%
%    A hook for \CTANpackage{cmap} is necessary.
%
%    \begin{macrocode}
\def\MYLATEX@fontpreloading{\let\MYLATEX@fontpreloading \@undefined
   \begingroup \setbox0=\hbox {%
      $$% math (not bold, some setups don't have \boldmath)
      \def\x {\bfseries\itshape}{\itshape}\ttfamily\sffamily
      \normalfont \x
      \ifdefined\large \large \x \fi
      \ifdefined\Large \Large \x \fi
      \ifdefined\LARGE \LARGE \x \fi
      \ifdefined\Huge  \Huge  \x \fi
      \ifdefined\small \small \x \fi
      \ifdefined\footnotesize \footnotesize \x \fi
      \ifdefined\tiny  \tiny \x \fi}%
      \@for\x:={lae,lfe,lgr,ot1,ot1tt,ot6,t1,t2a,t2b,t2c,t5}\do{%
         \ifcsname cmap@set@\x\endcsname  \global\expandafter
               \let\csname cmap@set@\x\endcsname \@undefined   \fi
         \uppercase\expandafter{\expandafter\def\expandafter\x\expandafter{\x}}%
         \ifcsname cmap@set@\x\endcsname  \global\expandafter
               \let\csname cmap@set@\x\endcsname \@undefined   \fi}%
   \endgroup
}% \MYLATEX@fontpreloading
%    \end{macrocode}
%    \end{macro}
%
%
%    \begin{macro}{\endofdump}
%
%
%    \begin{macrocode}
\let\MYLATEX@listfiles \listfiles
\def\endofdump{\let\endofdump \relax
   \MYLATEX@cleanup
   \MYLATEX@fontpreloading
   \makeatother
   \everyjob\expandafter{\the\everyjob
      \let\MYLATEX@temp \listfiles
      {%
         \MYLATEX@listfiles
         \global\let\MYLATEX@listfiles \@undefined
         \expandafter\MYLATEX@banner \@dofilelist
      }%
      \let\listfiles \MYLATEX@temp     % to be able to used \listfiles once after the format
      \let\MYLATEX@temp \@undefined
      \MYLATEX@scanpreamble}%
   \dump
}% \endofdump
%    \end{macrocode}
%    \end{macro}
%
%
%
%    \begin{macro}{\MYLATEX@banner}
%    \begin{macro}{\MYLATEX@start}
%
%    Set to be expanded at \cs{everyjob}: displays a banner in the log file.
%
%    \begin{macrocode}
%% Banner for \everyjob.
{\@makeother\"  % just in case
\xdef\MYLATEX@banner #1#2#3\typeout #4{%
   \global\let\noexpand\MYLATEX@banner \noexpand\@undefined
   \newlinechar`\noexpand\^^J\message{%
   ==============================================================================^^J%
   JOB NAME\@spaces\@spaces\space: "\noexpand\jobname"^^J%
   CUSTOMISED FORMAT: "\jobname"^^J%
   PRELOADED FILES:^^J}%
   #3%
   \message{%
   ==============================================================================}%
}% \MYLATEX@banner
\long\gdef\MYLATEX@start #1#2{\global\let\MYLATEX@start \@undefined
   \message{%
   (mylatexformat)Info: start reading document "\jobname"^^J%
   (mylatexformat)\@spaces\space\on@line.
            (\ifcase#1 \string\endofdump\else\string\begin{document}\fi)^^J%
   ==============================================================================}%
   \endgroup
   \MYLATEX@opens\relax #2%
}% \MYLATEX@start
}% \catcode group

%    \end{macrocode}
%    \end{macro}
%    \end{macro}
%
%    \begin{macro}{\MYLATEX@scanpreamble}
%
%    While the preamble is being skipped (inside a group)
%    the EOL is active and defined to grab each line and
%    inspect it looking for :
%    \begin{Verbatim}
%            \endofdump
%     or     \csname endofdump\endcsname
%     or     \begin{document}
%    \end{Verbatim}
%
%    \begin{macrocode}
%% While the preamble is being skipped (inside a group)
%% the EOL is active and defined to grab each line and
%% inspect it looking for :
%%        \endofdump
%% or     \csname endofdump\endcsname
%% or     \begin{document}.
%%
%% The special catcodes required are not enabled until after the
%% first TeX command in the file, so as to avoid problems with
%% the special processing that TeX does on the first line, choosing
%% the format, or the file name etc.
{% catcode group
\catcode`\^^M=13 \catcode`\&  =8 %
\long\gdef\MYLATEX@scanpreamble {%
 \begingroup \catcode`\^^M=13%
 \long\def\MYLATEX@endofdump ##1\endofdump##2##3&##4##5##6\MYLATEX {##5}%
 \long\def\MYLATEX@csendofdump ##1\csname endofdump\endcsname##2##3&##4##5##6\MYLATEX {##5}%
 \long\def\MYLATEX@document ##1document##2##3&##4##5##6\MYLATEX {##5}%
 \long\def\MYLATEX@begindocument ##1\begin##2##3\MYLATEX##4{%
    \MYLATEX@document ##2document&&{\MYLATEX@start1{##4}}{^^M}\MYLATEX}%
 \long\def^^M##1^^M{%
    \MYLATEX@endofdump##1\endofdump&&%
       {\MYLATEX@start 0{##1}}%
       {\MYLATEX@csendofdump ##1\csname endofdump\endcsname&&%
          {\MYLATEX@start 0{##1}}%
          {\MYLATEX@begindocument ##1\begin\relax\MYLATEX{##1}}%
          \MYLATEX}%
       \MYLATEX}%
 ^^M%
}% \MYLATEX@scanpreamble
}% catcode group
%    \end{macrocode}
%    \end{macro}
%
%    \begin{macrocode}
%% Trick lookahead to allow mylatex.ltx and the document filename to be
%% given on the same command line. (initex &latex mylatex.ltx {abc.tex})
\expandafter\input\endinput%
%</package>
%    \end{macrocode}
%
% \restoregeometry\pagesetup*{left/offset+=1cm,right/offset+=1.5cm-2mm}
% \begin{History}
%   \sectionformat\subsection{font+=\smaller,bottom=\smallskipamount,top=\medskipamount}
%
%   \begin{Version}{2011/02/12 v3.4}\HistLabel{3.4}
%   \item Reorganisation of the code + optimisation.
%   \end{Version}
%
%   \begin{Version}{2011/01/19 v3.3}\HistLabel{3.3}
%   \item Modification of the code for \cs{begin} in order to keep the patching introduced by
%         \CTANpackage{etoolbox} version \textt{2.1} for the hooks \cs{AtBeginEnvironment}...\cs{AtEndEnvironement}.
%   \end{Version}
%
%   \begin{Version}{2011/01/14 v3.1}\HistLabel{3.1}
%   \item \cs{@for} loop introduced for compatibility with the \xpackage{cmap} package had the bad side effect to define a control sequence \cs{x}. Fixed.
%   \end{Version}
%
%   \begin{Version}{2011/01/12 v3.0}\HistLabel{3.0}
%   \item Fix a problem with package \CTANpackage{cmap} which could not be
%         included in the format.
%   \item Documentation revisited after Users' feedback (and the \CTANpackage{interfaces} package).
%   \end{Version}
%
%   \begin{Version}{2010/08/09 v2.9}\HistLabel{2.9}
%   \item Modification of the macros that scan the preamble when the \xext{tex} file is compiled with its format.\\
%     It is now possible to start a file with \cs{begin}\texttt{\{document\}} just after the format specification (\texttt{\&myformat}).
%   \item[To do:] Command line to make the format on Linux (\texttt{-ini} instead of \texttt{-initialize} ???).
%   \end{Version}
%
%   \begin{Version}{2010/06/30 v2.1}\HistLabel{2.1}
%   \item Added: \cs{listfiles} is now available in the preamble, after \cs{endofdump}, when the document is compiled with its format.
%   \item Documentation update.
%   \end{Version}
%
%   \begin{Version}{2010/05/20 v1.2}\HistLabel{1.2}
%   \item Fixed a bug in \cs{MYLATEXopens} \\
%         (There was problems when including an index in the format for example...)
%   \end{Version}
%
%   \begin{Version}{2010/04/30 v1.0}\HistLabel{1.0}
%   \item First version.
%   \end{Version}
%
% \end{History}
%
% ^^A\restoregeometry\pagesetup*{left/offset+=1cm,right/offset+=1.5cm-2mm}
% \clearpage
% \PrintIndex
%
% \Finale
%        (quote the arguments according to the demands of your shell)
%
% Documentation:
%           (pdf)latex mylatexformat.dtx
% Copyright (C) 2010-2011 by Florent Chervet <florent.chervet@free.fr>
%<*ignore>
\begingroup
  \def\x{LaTeX2e}%
\expandafter\endgroup
\ifcase 0\ifx\install y1\fi\expandafter
         \ifx\csname processbatchFile\endcsname\relax\else1\fi
         \ifx\fmtname\x\else 1\fi\relax
\else\csname fi\endcsname
%</ignore>
%<*install>
\input docstrip.tex
\Msg{************************************************************************}
\Msg{* Installation}
\Msg{* Package 2011/02/12 v3.4 - Use mylatexformat.ltx to make a format based on the preamble of any LaTeX file*}
\Msg{************************************************************************}

\keepsilent
\askforoverwritefalse

\let\MetaPrefix\relax
\preamble

This is a generated file.

mylatexformat : 2011/02/12 v3.4 - Use mylatexformat.ltx to make a format based on the preamble of any LaTeX file

This work may be distributed and/or modified under the
conditions of the LaTeX Project Public License, either
version 1.3 of this license or (at your option) any later
version. The latest version of this license is in
   http://www.latex-project.org/lppl.txt

This work consists of the main source file mylatexformat.dtx
and the derived files
   mylatexformat.ltx, mylatexformat.pdf, mylatexformat.ins

mylatexformat - Use mylatexformat.ltx to make a format based on the preamble of any LaTeX file
Copyright (C) 2010-2011 by Florent Chervet <florent.chervet@free.fr>

\endpreamble
\let\MetaPrefix\DoubleperCent

\generate{%
   \file{mylatexformat.ins}{\from{mylatexformat.dtx}{install}}%
   \file{mylatexformat.ltx}{\from{mylatexformat.dtx}{package}}%
}

\askforoverwritefalse
\generate{%
   \file{mylatexformat.drv}{\from{mylatexformat.dtx}{driver}}%
}

\obeyspaces
\Msg{************************************************************************}
\Msg{*}
\Msg{* To finish the installation you have to move the following}
\Msg{* file into a directory searched by TeX:}
\Msg{*}
\Msg{*     mylatexformat.ltx}
\Msg{*}
\Msg{* To produce the documentation run the file `mylatexformat.dtx'}
\Msg{* through LaTeX.}
\Msg{*}
\Msg{* Happy TeXing!}
\Msg{*}
\Msg{************************************************************************}

\endbatchfile
%</install>
%<*ignore>
\fi
%</ignore>
%<*driver>
\edef\thisfile{\jobname}
\def\thisinfo{Make a format based on the preamble of any \texorpdfstring{\LaTeX{}}{LaTeX} file}
\def\thisdate{2011/02/12}
\def\thisversion{3.4}
\let\loadclass\LoadClass
\def\LoadClass#1{\loadclass[abstracton]{scrartcl}\let\scrmaketitle\maketitle\AtEndOfClass{\let\maketitle\scrmaketitle}}
{\makeatletter{\endlinechar`\^^J\obeyspaces
 \gdef\ErrorUpdate#1=#2,{\@ifpackagelater{#1}{#2}{}{\let\CheckDate\errmessage\toks@\expandafter{\the\toks@
        \thisfile-documentation: updates required !
              package #1 must be later than #2
              to compile this documentation.}}}}%
 \gdef\CheckDate#1{{\let\CheckDate\relax\toks@{}\@for\x:=#1\do{\expandafter\ErrorUpdate\x,}\CheckDate\expandafter{\the\toks@}}}}
\AtBeginDocument{\DeleteShortVerb{\|}\CheckDate{interfaces=2011/01/26,tabu=2011/01/26}}
\PassOptionsToPackage{svgnames}{xcolor}
\PassOptionsToPackage{hyperfootnotes}{hyperref}
\documentclass[a4paper,oneside]{ltxdoc}
\usepackage[latin1]{inputenc}
\usepackage[american]{babel}
\usepackage[T1]{fontenc}
\usepackage{etex,etoolbox,geometry,lastpage,xspace,xcolor,bbding,txfonts,atveryend,moresize,relsize}
\usepackage{tocloft,titlesec,graphicx,fancyhdr,framed,multirow,makecell,tabu,enumitem,needspace}
\usepackage{holtxdoc,bookmark,embedfile,tabu,keycommand}
\usepackage{microtype}
\usepackage{interfaces}
\usetikz{basic}
\csname endofdump\endcsname
\CodelineNumbered
\usepackage{fancyvrb}\fvset{gobble=1,listparameters={\topsep=0pt}}
\lastlinefit999
\geometry{top=0pt,includeheadfoot,headheight=7mm,headsep=.6cm,bottom=.6cm,footskip=.5cm,left=4cm,right=14mm}
\hypersetup{%
  pdftitle={The mylatexformat package},
  pdfsubject={Use mylatexformat.ltx to make a format based on the preamble of any LaTeX file},
  pdfauthor={Florent CHERVET},
  colorlinks,linkcolor=reflink,
  pdfstartview={FitH},
  pdfkeywords={tex, e-tex, latex, package, format, formats, pdfTeX},
  bookmarksopen=true,bookmarksopenlevel=2}
\embedfile{\thisfile.dtx}
\begin{document}
   \DocInput{\thisfile.dtx}
\end{document}
%</driver>
% \fi
%
% \CheckSum{317}
%
% \CharacterTable
%  {Upper-case    \A\B\C\D\E\F\G\H\I\J\K\L\M\N\O\P\Q\R\S\T\U\V\W\X\Y\Z
%   Lower-case    \a\b\c\d\e\f\g\h\i\j\k\l\m\n\o\p\q\r\s\t\u\v\w\x\y\z
%   Digits        \0\1\2\3\4\5\6\7\8\9
%   Exclamation   \!     Double quote  \"     Hash (number) \#
%   Dollar        \$     Percent       \%     Ampersand     \&
%   Acute accent  \'     Left paren    \(     Right paren   \)
%   Asterisk      \*     Plus          \+     Comma         \,
%   Minus         \-     Point         \.     Solidus       \/
%   Colon         \:     Semicolon     \;     Less than     \<
%   Equals        \=     Greater than  \>     Question mark \?
%   Commercial at \@     Left bracket  \[     Backslash     \\
%   Right bracket \]     Circumflex    \^     Underscore    \_
%   Grave accent  \`     Left brace    \{     Vertical bar  \|
%   Right brace   \}     Tilde         \~}
%
% \DoNotIndex{\begin,\CodelineIndex,\CodelineNumbered,\def,\DisableCrossrefs,\~,\@ifpackagelater}
% \DoNotIndex{\ttfamily,\tiny,\sffamily,\small,\normalfont,\newindex,\makeindex,\makeglossary,\LARGE,\Large,\large,\itshape,\immediate}
% \DoNotIndex{\&,\/,\=,\@dofilelist,\@spaces,\@undefined,\^,\active,\bfseries,\boldmath,\etex@primitive@openout,\footnotesize,\g@addto@macro,\Huge}
% \DoNotIndex{\DocInput,\documentclass,\EnableCrossrefs,\end,\GetFileInfo}
% \DoNotIndex{\NeedsTeXFormat,\OnlyDescription,\RecordChanges,\usepackage}
% \DoNotIndex{\ProvidesClass,\ProvidesPackage,\ProvidesFile,\RequirePackage}
% \DoNotIndex{\filename,\fileversion,\filedate,\let}
% \DoNotIndex{\@listctr,\@nameuse,\csname,\else,\endcsname,\expandafter}
% \DoNotIndex{\gdef,\global,\if,\item,\newcommand,\nobibliography}
% \DoNotIndex{\par,\providecommand,\relax,\renewcommand,\renewenvironment}
% \DoNotIndex{\stepcounter,\usecounter,\nocite,\fi}
% \DoNotIndex{\@fileswfalse,\@gobble,\@ifstar,\@unexpandable@protect}
% \DoNotIndex{\AtBeginDocument,\AtEndDocument,\begingroup,\endgroup}
% \DoNotIndex{\frenchspacing,\MessageBreak,\newif,\PackageWarningNoLine}
% \DoNotIndex{\protect,\string,\xdef,\ifx,\texttt,\@biblabel,\bibitem}
% \DoNotIndex{\z@,\wd,\wheremsg,\vrule,\voidb@x,\verb,\bibitem}
% \DoNotIndex{\FrameCommand,\MakeFramed,\FrameRestore,\hskip,\hfil,\hfill,\hsize,\hspace,\hss,\hbox,\hb@xt@,\endMakeFramed,\escapechar}
% \DoNotIndex{\do,\date,\if@tempswa,\@tempdima,\@tempboxa,\@tempswatrue,\@tempswafalse,\ifdefined,\ifhmode,\ifmmode,\cr}
% \DoNotIndex{\box,\author,\advance,\multiply,\Command,\outer,\next,\leavevmode,\kern,\title,\toks@,\trcg@where,\tt}
% \DoNotIndex{\the,\width,\star,\space,\section,\subsection,\textasteriskcentered,\textwidth}
% \DoNotIndex{\",\:,\@empty,\@for,\@gtempa,\@latex@error,\@namedef,\@nameuse,\@tempa,\@testopt,\@width,\\,\m@ne,\makeatletter,\makeatother}
% \DoNotIndex{\maketitle,\parindent,\setbox,\x,\kernel@ifnextchar}
% \DoNotIndex{\KVS@CommaComma,\KVS@CommaSpace,\KVS@EqualsSpace,\KVS@Equals,\KVS@Global,\KVS@SpaceEquals,\KVS@SpaceComma,\KVS@Comma}
% \DoNotIndex{\DefineShortVerb,\DeleteShortVerb,\UndefineShortVerb,\MakeShortVerb,\endinput}
%
% \let\ClearPage\clearpage
% \makeatletter
% \catcode`\� \active   \def�{\@ifnextchar �{\par\nobreak\vskip-2\parskip}{\par\nobreak\vskip-\parskip}}
% \newrobustcmd\Xspace{\ifnum\catcode`\ =\active\else\expandafter\xspace\fi}
% \newrobustcmd*\stform{\ifincsname\else\expandafter\@stform\fi}
% \newrobustcmd*\@stform{\@ifnextchar*{\@@stform[]\textasteriskcentered\@gobble}\@@stform}
% \newrobustcmd*\@@stform[2][\string]{\textttbf{#1#2}\Xspace}
%
% \newrobustcmd*\thispackage{\@ifstar
%  {\xpackage{{\thisfile}}\Xspace}
%  {\xpackage{{\pkgcolor\thisfile}}\Xspace}}
% \def\ThisPackage{\Xpackage{\thisfile}\xspace}
% \def\CTANpackage#1{\href{http://www.tex.ac.uk/tex-archive/help/Catalogue/entries/#1.html}{\xpackage{#1}}}
% \def\Xpackage{\@dblarg\X@package}
% \def\X@package[#1]#2{\@testopt{\X@@package{#1}{#2}}{}}
% \def\X@@package#1#2[#3]{%
%     \xpackage{#2\footnote{\noindent\xpackage{#2}: \href{http://www.ctan.org/tex-archive/macros/latex/contrib/#1}{\nolinkurl{CTAN:macros/latex/contrib/#1}}#3}}}
% \def\Underbrace#1_#2{$\underbrace{\vtop to2ex{}\hbox{#1}}_{\footnotesize\hbox{#2}}$}
% \newrobustcmd*\FC[1][\color{copper}]{{#1\usefont{T1}{fts}xn FC}}
%
% \parindent\z@\parskip.4\baselineskip\topsep\parskip\partopsep\z@
% \g@addto@macro\macro@font{\macrocodecolor\let\AltMacroFont\macro@font}
% \g@addto@macro\@list@extra{\parsep\parskip\topsep\z@\itemsep\z@}
% \def\smex{\leavevmode\hb@xt@2em{\hfil$\longrightarrow$\hfil}}
% \newrobustcmd\verbfont{\usefont{T1}{\ttdefault}{\f@series}{n}}    \let\vb\verbfont
% \renewrobustcmd\#[1]{{\usefont{T1}{pcr}{bx}{n}\char`\##1}}
% \newrobustcmd\csred[1]{\textcolor{red}{\cs{#1}}}
% \renewrobustcmd\cs[2][]{\mbox{\vb#1\expandafter\@gobble\string\\#2}}
% \newrobustcmd\CSbf[1]{\textbf{\CS{#1}}}
% \newrobustcmd\csbf[2][]{\textbf{\cs[{#1}]{#2}}}
% \newrobustcmd\textttbf[1]{\textbf{\texttt{#1}}}
% \renewrobustcmd*\bf{\bfseries}\newcommand\nnn{\normalfont\mdseries\upshape}\newcommand\nbf{\normalfont\bfseries\upshape}
% \newrobustcmd*\blue{\color{blue}}\newcommand*\red{\color{dr}}\newcommand*\green{\color{green}}\newcommand\rred{\color{red}}
% \newrobustcmd\rrbf{\color{red}\bfseries}
% \newcommand\textt[1]{\texorpdfstring{\texttt{#1}}{#1}}
% \newrobustcmd\texttbf[1]{\textbf{\texttt{#1}}}
% \colorlet{pkgcolor}{teal}\def\pkgcolor{\color{pkgcolor}}
% \definecolor{copper}{rgb}{0.67,0.33,0.00}  \newcommand\copper{\color{copper}}
% \definecolor{dg}{rgb}{0.16,0.33,0.00}      \newcommand\dg{\color{dg}}
% \definecolor{db}{rgb}{0,0,0.502}           \newcommand\db{\color{db}}
% \definecolor{dr}{rgb}{0.49,0.00,0.00}      \let\dr\red
% \definecolor{lk}{rgb}{0.25,0.25,0.25}      \newrobustcmd\lk{\color{lk}}
% \newrobustcmd\bk{\color{black}}\newcommand\md{\mdseries}
% \def\M{\@ifstar{\M@i\@firstofone}{\M@i\meta}}
% \def\M@i#1{\@ifnextchar[^^A]
%   {\M@square#1}
%   {\@ifnextchar(^^A)
%     {\M@paren#1}
%     {\M@brace#1}}^^A
% }
% \def\M@square#1[#2]{\M@Bracket[{#1{#2}}]}
% \def\M@paren#1(#2){\M@Bracket({#1{#2}})}
% \def\M@brace#1#2{\M@Bracket\{{#1{#2}}\}}
% \def\M@Bracket#1#2#3{{\ttfamily#1#2#3}}
% \newrobustcmd*\thisyear{\begingroup
%    \def\thisyear##1/##2\@nil{\endgroup
%       \oldstylenums{\ifnum##1=2010\else 2010\,\textendash\,\fi ##1}^^A
%    }\expandafter\thisyear\thisdate\@nil
% }
%
% \def\declmargin{\hspace*\declmarginwidth }
% \def\declmarginwidth{\dimexpr -\leftmargini +\arrayrulewidth +\tabcolsep\relax}
% \renewrobustcmd\declcs[2][]{^^A
%   \if@nobreak \par\nobreak
%   \else \par\addvspace\parskip
%         \Needspace{.08\textheight}\fi
%   \changefont{size+=2.5pt,spread=1,fam=\ttdefault}^^A
%   \newcommand*\csanchor[2][]{\raisedhyperdef[14pt]{declcs}{##2}{\cs[{##1}]{##2}}}^^A
%   \def\*{\unskip\,\texttt{*}}\noindent
%   \hskip-\leftmargini
%   \begin{tabu}{|l|}\hline
%     \expandafter\SpecialUsageIndex\csname #2\endcsname
%     \csanchor[{#1}]{#2}}
% \renewcommand\enddeclcs{%
%     \crcr \hline \end{tabu}\nobreak
%     \par  \nobreak \noindent
%     \ignorespacesafterend
%  }
% \newrobustcmd*\csref[2][]{\hyperref{}{declcs}{#2}{\cs[{#1}]{#2}}}
%
% \let\plainllap\llap
% \newrobustcmd\macro@llap[1]{{\global\let\llap\plainllap
%  \setbox0=\hbox\bgroup \raisedhyperdef{macro}{\saved@macroname}{#1}\egroup
%  \ifdim\wd0>2.4cm
%     \hbox to\z@ \bgroup\hss \hbox to2.4cm{\unhcopy0\hss}\egroup
%     \edef\@tempa{\hskip\dimexpr\the\wd0-2.4cm}\global\everypar\expandafter{\the\expandafter\everypar
%                                                                            \@tempa \global\everypar{}}^^A
%  \else \llap{\unhbox0}\fi}}
% \pretocmd\macro{\if@nobreak\else\Needspace{2\baselineskip}\fi
%           \MacrocodeTopsep\z@skip \MacroTopsep\z@skip \parsep\z@ \topsep\z@ \itemsep\z@ \partopsep\z@
%           \let\llap\macro@llap}{}{}
% \apptocmd\endmacro{\goodbreak\vskip.3\parskip}{}{}
% \newrobustcmd\macroref[2][]{\hyperref{}{macro}{#2}{\cs[{#1}]{#2}}}
%
% \DefineVerbatimEnvironment{Verb}{Verbatim}{gobble=1}
% \DefineVerbatimEnvironment{VerbLines}{Verbatim}{gobble=1,frame=lines,framesep=6pt,fontfamily=txtt,fontseries=b}
% \newkeyenvironment+[\|]{dblruled}[first=.4pt,second=.4pt,sep=1pt,left=\z@]%
%   {|\def\FrameCommand|{%
%     \vrule\@width\commandkey{first}
%     \hskip\commandkey{sep}
%     \vrule\@width\commandkey{second}
%     |\hspace|{\commandkey{left}}}
%   |\MakeFramed| {\advance\hsize-|\width \FrameRestore|}%
%  }{%
%   |\endMakeFramed|%
%  }
%
% \colorlet{HeadBand}{BlanchedAlmond}
% \newcommand*\topheader{\tikz[remember picture,overlay]{
%   \coordinate (A) at (current page.north west);
%   \coordinate (B) at ($(current page.north east)+(0cm,-\headheight)$);
%   \fill [color=HeadBand,fill opacity=.6] (A) rectangle (B);
%   }%
% }
%
% \pagesetup[plain]{
%     norules,
%     foot/right=\oldstylenums{\thepage} / \oldstylenums{\pageref{LastPage}},
%     foot/font=\lk\scriptsize,
%     foot/left/offset=3cm,
%     foot/right/offset=1cm,
%     foot/left=\vbox to\baselineskip{\vss{{\rotatebox[origin=l]{90}{\thispackage\,[rev.\thisversion]\,\copyright\,\thisyear\,\lower.4ex\hbox{\pkgcolor\NibRight}\,\FC\quad \xemail{florent.chervet at free.fr}}}}},
% }
% \pagesetup{
%     foot/right={\oldstylenums{\thepage} / \oldstylenums{\pageref{LastPage}}},
%     foot/font=\lk\scriptsize,
%     foot/left/offset=3cm,
%     foot/right/offset=1cm,
%     foot/left=\vbox to\baselineskip{\vss{{\rotatebox[origin=l]{90}{\thispackage\,[rev.\thisversion]\,\copyright\,\thisyear\,\lower.4ex\hbox{\pkgcolor\NibRight}\,\FC}}}},
%     head/font=\small\sffamily,
%     head/right/offset=1cm,
%     head/right=\noindent\raise.4ex\hbox{\thispackage : \thisinfo},
%     head/left=\topheader,
% }
% \sectionformat\section{
%     top=\medskipamount,
%     bottom=.5\smallskipamount,
%     bookmark={color=MidnightBlue},
% }
% \sectionformat\subsubsection{
%     top=\smallskipamount,bottom=0pt,
%     bookmark={color=lk},
% }
% \bookmarksetup{openlevel=3}
%
% \newcommand\macrocodecolor{\color{macrocode}}\definecolor{macrocode}{rgb}{0.05,0.02,0.07}
% \newcommand\reflinkcolor{\color{reflink}}\definecolor{reflink}{rgb}{0.49,0.00,0.00}
% \newrobustcmd*\IMPLEMENTATION{%
%     \bookmarksetup{openlevel=1}
%     ^^A\geometry{top=0pt,includeheadfoot,headheight=7mm,headsep=.6cm,bottom=.6cm,footskip=.5cm,left=4cm,right=1.5cm}
%     \newgeometry{top=0pt,includeheadfoot,headheight=7mm,headsep=.6cm,bottom=.6cm,footskip=.5cm,left=3cm,right=5mm}
%     \pagesetup*{left/offset-=1cm,right/offset-=1.5cm-2mm}
%     \section{Implementation} \label{sec:implementation}
% }
%
% \makeatother
%
% \deffootnote{1em}{0pt}{\rlap{\textsuperscript{\thefootnotemark}}\kern1em}
%
% \title{\vspace*{-28pt}\href{http://www.tex.ac.uk/tex-archive/help/Catalogue/entries/mylatexformat.html}{\HUGE\bfseries\sffamily\pkgcolor mylatexformat}\vspace*{6pt}}
% \date{}
% \subtitle{\LARGE\begin{tabu}{X[c]}
%           Use \thispackage* to make a format \\
%           based on the preamble of any LaTeX file \\[-1ex]
%           \small\FC
%           \end{tabu}
%           \vspace*{-12pt}}
% \author{\smaller\thisdate~--~\hyperref[\thisversion]{version \thisversion}}
%
% \maketitle
%
% {\makeatletter\let\@thefnmark\@empty\let\@makefntext\@firstofone
% \footnotetext{\noindent\parskip=0pt\scriptsize
% This documentation is produced with the \textt{DocStrip} utility.\par
% \begin{tabu}{X[-3]X[-1]X}
% \smex To get the package,                   &run:          &\texttt{etex \thisfile.dtx}                  \\
% \smex To get the documentation              &run (thrice): &\textt{pdflatex \thisfile.dtx}               \\
% \leavevmode\hphantom\smex To get the index, &run:          &\texttt{makeindex -s gind.ist \thisfile.idx}
% \end{tabu}�
% The \xext{dtx} file is embedded into this pdf file thank to \Xpackage{embedfile} by H. Oberdiek.}
% }
%
% \deffootnote{1em}{0pt}{\rlap{\thefootnotemark.}\kern1em}
%
% {\let\quotation\relax\let\endquotation\relax\vspace*{-1cm}%
% \begin{abstract}\parskip\smallskipamount\parindent0pt\lastlinefit0\leftskip.7cm\rightskip\leftskip
%
% \thispackage can be used to make a format from the preamble of any \LaTeX{} file.
% The use of formats helps to speed up compilations: packages which have been
% dumped in the format are loaded at very high speed. A format (also known as \emph{dump}) is therefore very suitable:
% \begin{itemize}[label=\textendash,leftmargin=3em]
% \item if you have many documents that share the same preamble,
% \item if you have a document which loads a large amount of packages (like \xpackage{TikZ} graphics libraries).
% \end{itemize}
%
% \medskip
%
% \thispackage will dump all definitions until \cs{begin}\M*{document} or \csref{endofdump}, more precisely:
%
% \csdef{tabu*}{\begingroup\tabcolsep0pt\begin{tabu}}
% \csdef{endtabu*}{\end{tabu}\endgroup}
% \definecolor{ly}{rgb}{1,1,.35}
%
% \hskip-2em
% \begin{tabu}to \linewidth{XXX[1.2]}
%
% \fcolorbox{black}{ly}{%
% \begin{tabu*}{X}
% \cs{RequirePackage}\M*{...} \\
% \cs{documentclass}\M*{...}  \\
% \cs{usepackage}\M*{...}     \\
% \noindent\,\,...       \\
% \noindent\,\,...       \\
% \csbf{begin}{\bfseries\M*{document}}
% \end{tabu*}}
% &
% \begin{tabu*}{X}
% \fcolorbox{black}{ly}{%
% \begin{tabu}{X}
% \cs{RequirePackage}\M*{...} \\
% \cs{documentclass}\M*{...}  \\
% \cs{usepackage}\M*{...}     \\
% \csbf{endofdump}       \\
% \end{tabu}} \\
% \noindent\,\, ... \\
% \noindent\,\,\cs{begin}\M*{document}
% \end{tabu*}
% &
% \begin{tabu*}{X}
% \fcolorbox{black}{ly}{%
% \begin{tabu*}{X}
% \cs{RequirePackage}\M*{...} \\
% \cs{documentclass}\M*{...}  \\
% \cs{usepackage}\M*{...}     \\
% \csbf{csname} \texttbf{endofdump}\csbf{endcsname}       \\
% \end{tabu*}} \\
% \noindent\,\, ... \\
% \noindent\,\,\cs{begin}\M*{document}
% \end{tabu*}
% \end{tabu}
% \medskip
%
% The colored parts correspond to the format. The \xext{log} file will report:
%
% \begin{tabu}{@{}X|[pkgcolor]X}
% \begin{tabu}{>{\ttfamily}X[l]}
% start reading document "my document" \\
% on input line *** (\cs{begin}\M*{document})
% \end{tabu}
% &
% \begin{tabu}{>{\ttfamily}X[l]}
% start reading document "my document" \\
% on input line *** (\cs{endofdump})
% \end{tabu}
% \end{tabu}
% \medskip
%
% \thispackage was primarily written from \CTANpackage{mylatex}
% by David Carlisle, but it is different in many points (see \ref{mylatex}). In particular, \thispackage allows to put almost
% any package in the format, with one exception: the package \CTANpackage{minitoc}
% cannot be included in the format (you must use \csref{endofdump} before \cs{usepackage}\M*{minitoc}).
%
% \end{abstract}
% }
%
% \tocsetup{
%     section/skip=4pt plus2pt minus2pt,
%     subsection/skip=4pt plus2pt minus2pt,
%     title=Contents,
%     title/skip=2pt,
%     title/after={\pkgcolor\quad\leaders\vrule height3.4pt depth-3pt\hfill\null},
%     multicols/afterend=\leavevmode{\pkgcolor\hrule},
%     twocolumns,
%     bookmark={text=Contents,bold},
% }
% \tableofcontents
% \clearpage
%
%
% \MakeShortVerb{\+}
%
% \section{Creating a format}
%
% \subsection{Using MikTeX}
%
% \subsubsection{With preloaded pdflatex format (pdf output)}
%
% Use the following command line:
%
% \quad+etex -initialize [opts] "&pdflatex" mylatexformat.ltx """abc.tex"""+
%
% You need 3 quotes around your \xext{tex} file if its name contains space
% (only one quote if it does not contain any space, but 3 quotes is more general...)
% \medskip
%
% Full example:��
% \begin{VerbLines}
%  etex -initialize -save-size=20000 -stack-size=20000
%                   -jobname="The-Name-Of-The-.fmt-File(without .fmt extension)"
%                   "&pdflatex" mylatexformat.ltx """Mon Fichier TeX.tex"""
% \end{VerbLines}
%
%
% \subsubsection{With preloaded latex format (dvi output)}
%
% Use the following command line:
%
% \quad+etex -initialize [opts] "&latex" mylatexformat.ltx """abc.tex"""+
%
%
% \subsubsection{With preloaded format xelatex (pdf output)}
%
% Use the following command line:
%
% \quad+xetex -initialize [opts] "&xelatex" mylatexformat.ltx """abc.tex"""+
%
%
% \subsection{Using other distributions}
%
% As I pointed out in the \texttt{abstract}, I have not been able to test the file on other distributions.
%
% Some users told me they typed the following command line:
%
% +latex -ini -jobname="fmt-file-name" "&latex" mylatexformat.ltx  """abc.tex"""+
%
% \section{Using a format}
%
% \subsection{Using MikTeX}
%
% Add the format to the very first line of your LaTeX file :
%
% \quad+%&"Mon Beau Format"+
%
% You may use the $\varepsilon$-\TeX{} option :    +-dont-parse-first-line+ \quad to avoid loading the format.
%
% \subsection{Using other distributions}
%
% As I pointed out in the \texttt{abstract}, I have not been able to test the file on other distributions.
%
% \clearpage
% \section[How does it work ? (and why it is different from mylatex.ltx)]{How does it work ? (and why it is different from \xpackage{mylatex.ltx})}
% \label{mylatex}
%
% \xpackage{mylatex.ltx} has three important limitations:
% \begin{enumerate}[label=\arabic*)~]
% \item The end of the format is either \verb+\begin{document}+ or the comment \verb+%mylatex+\\
%       Therefore, after the format has been loaded, \TeX{} checks the commentaries in the preamble
%       in order to skip everything before \verb+%mylatex+. \\
%       But this way, braces \textbf{must match inside commentaries !}
% \item \xpackage{mylatex.ltx} temporarily redefines the command \cs{document} in order to stop
%      the creation of the format (execution of \cs{dump}) at begin document. \\
%      However, the \textbf{initial definition of the macro \csbf{document}} is saved
%      and reloaded after the format has been loaded. \\
%      Therefore, packages that modifies this macro (such as \Xpackage[oberdiek/auxhook]{auxhook}[ H. Oberdiek] or
%      \Xpackage{etoolbox}[ P. Lehman] cannot be included in the format!
% \item If your document starts with :\\
%              \verb+\begin{document}% commentary+ or with :\\
%              \verb+\begin{document}\makeatletter+ \\
%      the format is not stopped at \verb+\begin{document}+ and an error occurs.
% \end{enumerate}
%
% \begin{dblruled}[first=2pt,sep=1pt,second=.6pt,left=4pt]
% With \xpackage{mylatexformat.ltx}, those limitations are lifted.
%
% \xpackage{mylatexformat.ltx}:
% \begin{enumerate}[label=\arabic*)~]
% \item Temporarily redefines \cs{begin} in order to check if its argument is \meta{document}. \\
%       Therefore, packages like \xpackage{auxhook} or \xpackage{etoolbox} can be included in the format.
% \item Stops the format when it encounters \verb+\begin{document}+ or \cs{endofdump}.
% \item Does not read inside commentaries
% \item After the format has been loaded, each line is checked to see if it \textbf{contains} either:
%        \begin{description}[align=left,labelsep=1cm,font=\bfseries,labelindent=1cm]
%        \item[] {\dr\verb+\begin{document}+}
%        \item[or: ] {\dr\verb+\endofdump+}
%        \item[or: ] {\dr\verb+\csname endofdump\endcsname+}\textsuperscript{\bfseries\textasteriskcentered}
%        \end{description}
%      Report is written in the \xext{log} file, with the job name, the name of the preloaded format,
%      the list of preloaded files and the first input line read after loading of format.
% \end{enumerate}
%
% {\footnotesize\textsuperscript{\bfseries\textasteriskcentered} this way, the command \verb+\csname endofdump\endcsname+ is silently ignored
% (equiv. to \cs{relax}) in case the format is not used.}
% \end{dblruled}
%
%
%
%
% \StopEventually{
% }
%
% \IMPLEMENTATION\DeleteShortVerb{\+}
%
%    \begin{macrocode}
%<*package>
%%% _____________________________________________________________________
%%%                           mylatexformat.ltx
%%%                           ^^^^^^^^^^^^^^^^^
%%% From an original idea by David Carlisle - mylatex.ltx 1994/09/27
%%% � lppl 2010-2011 - F Chervet - 2011/02/12 v3.4
%%% _____________________________________________________________________
%%%
%%% Use this file to make a format based on the preamble of any LaTeX
%%% file.
%%%
%%% The format is made from the very beginning of the file up to either:
%%%   i) the command : \endofdump
%%%                    placed in the preamble
%%%  ii) the command : \csname endofdump\endcsname
%%%                    placed in the preamble
%%% iii) \begin{document}
%%%                    if \endofdump nor \csname endofdump\endcsname
%%%                    haven't been found in the preamble.
%%%
%%% A format helps to speed up compilations, packages which have been
%%% dumped in the format are loaded at very high speed.
%%% --------------------------------------------------------------------
%%% There are no restrictions on the distribution or modification of
%%% this file, except that other people should not attempt to alter
%%% the master copy on the ctan archives.
%%%
%%% Making the format --------------------------------------------------
%%% ^^^^^^^^^^^^^^^^^                                       USING MikTeX
%%% *  With preloaded pdflatex format (pdf output) :
%%%    etex -initialize [opts] "&pdflatex" mylatexformat.ltx """abc.tex"""
%%%    => compile your document using :
%%%            pdflatex
%%%         or texify --tex-option=-output-format=pdf
%%%
%%% *  With preloaded latex format (dvi output) :
%%%    etex -initialize [opts] "&latex" mylatexformat.ltx """abc.tex"""
%%%    => compile your document using
%%%            latex
%%%         or texify --tex-option=-output-format=dvi
%%%
%%% * With preloaded format xelatex (pdf output) :
%%%    xetex -initialize [opts] "&xelatex" mylatexformat.ltx """abc.tex"""
%%%    => compile your document using xelatex
%%%
%%% NOTA BENE:    Please, notice the 3 quotes !
%%% ^^^^^^^^^^
%%%
%%% In fact, 3 quotes are necessary only if you "file name.tex" contains a space !
%%%
%%% -=-=-=-=-=-=-=-=-=-=-=-=-=-=-=-=-=-=-=-=-=-=-=-=-=-=-=-=-=-=-=-=-=-=
%%% Exemple of a command line for making format using pdfTeX :
%%%
%%%   etex -initialize -interaction=nonstopmode
%%%                    -save-size=20000 -stack-size=20000
%%%                    -jobname="Mon Beau Format Perso"
%%%                    "&pdflatex" mylatexformat.ltx "Mon Fichier TeX.tex"
%%%
%%%   "Mon Beau Format Perso.fmt" (en Fran�ais dans le texte) will be created
%%%   (or overwritten) in the current working directory.
%%%
%%% Using the format ---------------------------------------------------
%%% ^^^^^^^^^^^^^^^^^                                       USING MikTeX
%%% Add the format to the very first line of your LaTeX file :
%%% --> first line :       %&"format file name"
%%% without the .fmt extension.
%%% [You may use the option :    -dont-parse-first-line option
%%%  to avoid loading the format]
%%% --------------------------------------------------------------------
%% ____________________________________________________________________
%% The following informations come from mylatex.ltx from David Carlisle:
%% Given a LaTeX file `abc.tex', use initex as follows:
%%
%% initex &latex  mylatexformat.ltx abc      (Generic TeX)
%% initex \&latex mylatexformat.ltx abc      (unix and other TeX's)
%% tex /i &latex  mylatexformat.ltx abc      (emtex)
%%
%% If you are on a Mac or using some shell that makes it inconvenient
%% to use a command line such as the above examples then you may
%% make a file `mylatex.tex' with the single line
%% \input mylatex.ltx abc
%% and then pass the file mylatex.tex to your (ini)tex shell to produce
%% the format, ie something equivalent to initex &latex mylatex.tex.
%%
%% If you are using OzTeX, see the separate instructions below.
%%
%% This should make a format file mylatex.fmt which you can then use
%% as follows
%%
%% Using the new format
%% ^^^^^^^^^^^^^^^^^^^^
%%
%% tex  &mylatex abc          (generic TeX)
%% virtex \&mylatex abc       (Unix TeX)
%%
%% This will process your document, abc.tex, just as LaTeX does, but
%% quicker as the contents of the preamble will be stored in the
%% format file and will not need to be run each time.
%%
%% If (vir)tex fails to find your mylatex.fmt then it is not searching
%% in the current directory, either modify your TEXFORMATS path (or
%% equivalent) to search `.' or (on unix/dos) use   ./  as in:
%% virtex \&./mylatex abc
%%
%% Making and using the format with OzTeX
%% ^^^^^^^^^^^^^^^^^^^^^^^^^^^^^^^^^^^^^^
%%
%% Given a LaTeX file `abc.tex', do this: select `TeX...' from OzTeX's
%% TeX menu, and go to the folder containing the file `abc.tex' as if you
%% were selecting the file `abc.tex'. Then hit the Cancel button - this
%% procedure sets OzTeX's working folder to the one containing abc.tex.
%% Next, select iniTeX from the TeX menu, and type:
%%
%% &latex mylatex.ltx abc
%%
%% This should make a format file mylatex.fmt which you can save in the
%% same folder as the file you're working on.
%%
%% To use the new format, put this at the very start of the very first
%% line of your document:
%%
%% %&mylatex
%%
%% Further Notes
%%
%% The file abc.tex must contain a line *just* with
%% \begin{document}
%% Everything up to (but not including) the \begin{document} will
%% be saved in the format and not executed in subsequent runs.
%%
%% If you are modifying the document (or working on a similar document)
%% you may wish to add new commands to your document preamble.
%% The `mylatex' format normally skips the whole preamble (believing
%% it to be pre-loaded) and so such new commands do not take effect.
%% You could re-make the format, preloading the new preamble, but that
%% might be inconvenient to do every time, and so an alternative scheme
%% has been introduced.
%% If the preamble contains a comment   mylatex  (ie a line just
%% containing a % white space and the word mylatex) then the mylatex
%% format will start reading the preable at that point so any new
%% commands can be placed after such a comment and they will be
%% executed.
%% --------------------------------------------------------------------
%%
%%\RequirePackage{etex}% RequirePackage may be used before the class
%%
%% In principle \openout stream= filename need not be space terminated,
%% and need not be immediate, but this covers \makeindex \makeglossary
%% and index package's \newindex which are all the cases of \openout
%% that occur before \begin{document} that I could see.
%% Thanks to Ross Moore for pointing out \AtBeginDocument is too late
%% eg changebar package *closes* the stream in \AtBeginDocument, so need
%% to make sure it is opened before that. Make a special purpose hook.
\makeatletter
%    \end{macrocode}
%
%
%    \begin{macro}{\begin}
%
% \cs{begin} is overloaded in order to stop \cs{dump} at begin document
% if the macro \cs{endofdump} is not given somewhere in the preamble.
% Rem: it is NOT POSSIBLE to redefine \cs{document}, for the macro
%      \cs{document} is changed or patched by some classes or packages
%      (\xpackage{auxhook}, \xpackage{etoolbox}  etc.)
%
% With \Xpackage{etoolbox} version \textt{2.1} and earlier, \cs{begin} is patched,
% in order to insert the hooks \cs{AtBeginEnvironment} and \cs{AtEndEnvironment}.
%
% Therefore, it's not possible to restore, at end of dump, the orginal definition
% of \cs{begin} in the \LaTeX{} kernel. Instead, \thispackage inserts a reversible
% patch of the \cs{begin} command: the patched is reversed at end of dump by
% \macroref{MYLATEX@cleanup}.
%
%    \begin{macrocode}
%% \begin is overloaded in order to stop \dump at begin document
%% if the macro \endofdump is not given somewhere in the preamble.
%% Rem: it is NOT POSSIBLE to redefine \document, for the macro
%%      \document is changed or patched by some classes or packages
%%      (auxhook, etoolbox etc.)
\begingroup \def\x #1{\endgroup
   \gdef\begin ##1{\MYLATEX@StopAtdocument{##1}#1}
}\expandafter\x\expandafter{\begin{#1}}
\def\MYLATEX@StopAtdocument #1{\expandafter
   \ifx\csname #1\endcsname\document \expandafter\endofdump \fi
}% \MYLATEX@StopAtdocument
%    \end{macrocode}
%    \end{macro}
%
%    \begin{macro}{\openout}
%
%    Files cannot be opened out during the building of the format.
%    Instead, \cs{openout} is delayed until the end of the dump.
%
%    \begin{macrocode}
\let\MYLATEX@openout \openout
\def\openout#1 {\g@addto@macro\MYLATEX@opens{\immediate\openout#1 }}
\def\MYLATEX@opens {\let\MYLATEX@opens \@undefined}
%    \end{macrocode}
%    \end{macro}
%
%    \begin{macro}{\MYLATEX@cleanup}
%
%    Final clean up done at \cs{endofdump}: reverse the path for \cs{begin}
%    and restores \cs{openout} primitive.
%
%    \begin{macrocode}
\def\MYLATEX@cleanup {\let\MYLATEX@cleanup \@undefined
   \def\MYLATEX@StopAtdocument ##1\MYLATEX@StopAtdocument ##2##3\@nil{%
      \gdef\begin ####1{##1##3}%
   }\expandafter\MYLATEX@StopAtdocument \begin{##1}\@nil
   \let\openout                \MYLATEX@openout
   \let\MYLATEX@openout        \@undefined
   \let\MYLATEX@StopAtdocument \@undefined
}% \MYLATEX@cleanup
%    \end{macrocode}
%    \end{macro}
%
%
%    \begin{macro}{\MYLATEX@fontpreloading}
%
%    Some font can be preloaded: this gives an advantage to the format.
%
%    A hook for \CTANpackage{cmap} is necessary.
%
%    \begin{macrocode}
\def\MYLATEX@fontpreloading{\let\MYLATEX@fontpreloading \@undefined
   \begingroup \setbox0=\hbox {%
      $$% math (not bold, some setups don't have \boldmath)
      \def\x {\bfseries\itshape}{\itshape}\ttfamily\sffamily
      \normalfont \x
      \ifdefined\large \large \x \fi
      \ifdefined\Large \Large \x \fi
      \ifdefined\LARGE \LARGE \x \fi
      \ifdefined\Huge  \Huge  \x \fi
      \ifdefined\small \small \x \fi
      \ifdefined\footnotesize \footnotesize \x \fi
      \ifdefined\tiny  \tiny \x \fi}%
      \@for\x:={lae,lfe,lgr,ot1,ot1tt,ot6,t1,t2a,t2b,t2c,t5}\do{%
         \ifcsname cmap@set@\x\endcsname  \global\expandafter
               \let\csname cmap@set@\x\endcsname \@undefined   \fi
         \uppercase\expandafter{\expandafter\def\expandafter\x\expandafter{\x}}%
         \ifcsname cmap@set@\x\endcsname  \global\expandafter
               \let\csname cmap@set@\x\endcsname \@undefined   \fi}%
   \endgroup
}% \MYLATEX@fontpreloading
%    \end{macrocode}
%    \end{macro}
%
%
%    \begin{macro}{\endofdump}
%
%
%    \begin{macrocode}
\let\MYLATEX@listfiles \listfiles
\def\endofdump{\let\endofdump \relax
   \MYLATEX@cleanup
   \MYLATEX@fontpreloading
   \makeatother
   \everyjob\expandafter{\the\everyjob
      \let\MYLATEX@temp \listfiles
      {%
         \MYLATEX@listfiles
         \global\let\MYLATEX@listfiles \@undefined
         \expandafter\MYLATEX@banner \@dofilelist
      }%
      \let\listfiles \MYLATEX@temp     % to be able to used \listfiles once after the format
      \let\MYLATEX@temp \@undefined
      \MYLATEX@scanpreamble}%
   \dump
}% \endofdump
%    \end{macrocode}
%    \end{macro}
%
%
%
%    \begin{macro}{\MYLATEX@banner}
%    \begin{macro}{\MYLATEX@start}
%
%    Set to be expanded at \cs{everyjob}: displays a banner in the log file.
%
%    \begin{macrocode}
%% Banner for \everyjob.
{\@makeother\"  % just in case
\xdef\MYLATEX@banner #1#2#3\typeout #4{%
   \global\let\noexpand\MYLATEX@banner \noexpand\@undefined
   \newlinechar`\noexpand\^^J\message{%
   ==============================================================================^^J%
   JOB NAME\@spaces\@spaces\space: "\noexpand\jobname"^^J%
   CUSTOMISED FORMAT: "\jobname"^^J%
   PRELOADED FILES:^^J}%
   #3%
   \message{%
   ==============================================================================}%
}% \MYLATEX@banner
\long\gdef\MYLATEX@start #1#2{\global\let\MYLATEX@start \@undefined
   \message{%
   (mylatexformat)Info: start reading document "\jobname"^^J%
   (mylatexformat)\@spaces\space\on@line.
            (\ifcase#1 \string\endofdump\else\string\begin{document}\fi)^^J%
   ==============================================================================}%
   \endgroup
   \MYLATEX@opens\relax #2%
}% \MYLATEX@start
}% \catcode group

%    \end{macrocode}
%    \end{macro}
%    \end{macro}
%
%    \begin{macro}{\MYLATEX@scanpreamble}
%
%    While the preamble is being skipped (inside a group)
%    the EOL is active and defined to grab each line and
%    inspect it looking for :
%    \begin{Verbatim}
%            \endofdump
%     or     \csname endofdump\endcsname
%     or     \begin{document}
%    \end{Verbatim}
%
%    \begin{macrocode}
%% While the preamble is being skipped (inside a group)
%% the EOL is active and defined to grab each line and
%% inspect it looking for :
%%        \endofdump
%% or     \csname endofdump\endcsname
%% or     \begin{document}.
%%
%% The special catcodes required are not enabled until after the
%% first TeX command in the file, so as to avoid problems with
%% the special processing that TeX does on the first line, choosing
%% the format, or the file name etc.
{% catcode group
\catcode`\^^M=13 \catcode`\&  =8 %
\long\gdef\MYLATEX@scanpreamble {%
 \begingroup \catcode`\^^M=13%
 \long\def\MYLATEX@endofdump ##1\endofdump##2##3&##4##5##6\MYLATEX {##5}%
 \long\def\MYLATEX@csendofdump ##1\csname endofdump\endcsname##2##3&##4##5##6\MYLATEX {##5}%
 \long\def\MYLATEX@document ##1document##2##3&##4##5##6\MYLATEX {##5}%
 \long\def\MYLATEX@begindocument ##1\begin##2##3\MYLATEX##4{%
    \MYLATEX@document ##2document&&{\MYLATEX@start1{##4}}{^^M}\MYLATEX}%
 \long\def^^M##1^^M{%
    \MYLATEX@endofdump##1\endofdump&&%
       {\MYLATEX@start 0{##1}}%
       {\MYLATEX@csendofdump ##1\csname endofdump\endcsname&&%
          {\MYLATEX@start 0{##1}}%
          {\MYLATEX@begindocument ##1\begin\relax\MYLATEX{##1}}%
          \MYLATEX}%
       \MYLATEX}%
 ^^M%
}% \MYLATEX@scanpreamble
}% catcode group
%    \end{macrocode}
%    \end{macro}
%
%    \begin{macrocode}
%% Trick lookahead to allow mylatex.ltx and the document filename to be
%% given on the same command line. (initex &latex mylatex.ltx {abc.tex})
\expandafter\input\endinput%
%</package>
%    \end{macrocode}
%
% \restoregeometry\pagesetup*{left/offset+=1cm,right/offset+=1.5cm-2mm}
% \begin{History}
%   \sectionformat\subsection{font+=\smaller,bottom=\smallskipamount,top=\medskipamount}
%
%   \begin{Version}{2011/02/12 v3.4}\HistLabel{3.4}
%   \item Reorganisation of the code + optimisation.
%   \end{Version}
%
%   \begin{Version}{2011/01/19 v3.3}\HistLabel{3.3}
%   \item Modification of the code for \cs{begin} in order to keep the patching introduced by
%         \CTANpackage{etoolbox} version \textt{2.1} for the hooks \cs{AtBeginEnvironment}...\cs{AtEndEnvironement}.
%   \end{Version}
%
%   \begin{Version}{2011/01/14 v3.1}\HistLabel{3.1}
%   \item \cs{@for} loop introduced for compatibility with the \xpackage{cmap} package had the bad side effect to define a control sequence \cs{x}. Fixed.
%   \end{Version}
%
%   \begin{Version}{2011/01/12 v3.0}\HistLabel{3.0}
%   \item Fix a problem with package \CTANpackage{cmap} which could not be
%         included in the format.
%   \item Documentation revisited after Users' feedback (and the \CTANpackage{interfaces} package).
%   \end{Version}
%
%   \begin{Version}{2010/08/09 v2.9}\HistLabel{2.9}
%   \item Modification of the macros that scan the preamble when the \xext{tex} file is compiled with its format.\\
%     It is now possible to start a file with \cs{begin}\texttt{\{document\}} just after the format specification (\texttt{\&myformat}).
%   \item[To do:] Command line to make the format on Linux (\texttt{-ini} instead of \texttt{-initialize} ???).
%   \end{Version}
%
%   \begin{Version}{2010/06/30 v2.1}\HistLabel{2.1}
%   \item Added: \cs{listfiles} is now available in the preamble, after \cs{endofdump}, when the document is compiled with its format.
%   \item Documentation update.
%   \end{Version}
%
%   \begin{Version}{2010/05/20 v1.2}\HistLabel{1.2}
%   \item Fixed a bug in \cs{MYLATEXopens} \\
%         (There was problems when including an index in the format for example...)
%   \end{Version}
%
%   \begin{Version}{2010/04/30 v1.0}\HistLabel{1.0}
%   \item First version.
%   \end{Version}
%
% \end{History}
%
% ^^A\restoregeometry\pagesetup*{left/offset+=1cm,right/offset+=1.5cm-2mm}
% \clearpage
% \PrintIndex
%
% \Finale
%        (quote the arguments according to the demands of your shell)
%
% Documentation:
%           (pdf)latex mylatexformat.dtx
% Copyright (C) 2010-2011 by Florent Chervet <florent.chervet@free.fr>
%<*ignore>
\begingroup
  \def\x{LaTeX2e}%
\expandafter\endgroup
\ifcase 0\ifx\install y1\fi\expandafter
         \ifx\csname processbatchFile\endcsname\relax\else1\fi
         \ifx\fmtname\x\else 1\fi\relax
\else\csname fi\endcsname
%</ignore>
%<*install>
\input docstrip.tex
\Msg{************************************************************************}
\Msg{* Installation}
\Msg{* Package 2011/02/12 v3.4 - Use mylatexformat.ltx to make a format based on the preamble of any LaTeX file*}
\Msg{************************************************************************}

\keepsilent
\askforoverwritefalse

\let\MetaPrefix\relax
\preamble

This is a generated file.

mylatexformat : 2011/02/12 v3.4 - Use mylatexformat.ltx to make a format based on the preamble of any LaTeX file

This work may be distributed and/or modified under the
conditions of the LaTeX Project Public License, either
version 1.3 of this license or (at your option) any later
version. The latest version of this license is in
   http://www.latex-project.org/lppl.txt

This work consists of the main source file mylatexformat.dtx
and the derived files
   mylatexformat.ltx, mylatexformat.pdf, mylatexformat.ins

mylatexformat - Use mylatexformat.ltx to make a format based on the preamble of any LaTeX file
Copyright (C) 2010-2011 by Florent Chervet <florent.chervet@free.fr>

\endpreamble
\let\MetaPrefix\DoubleperCent

\generate{%
   \file{mylatexformat.ins}{\from{mylatexformat.dtx}{install}}%
   \file{mylatexformat.ltx}{\from{mylatexformat.dtx}{package}}%
}

\askforoverwritefalse
\generate{%
   \file{mylatexformat.drv}{\from{mylatexformat.dtx}{driver}}%
}

\obeyspaces
\Msg{************************************************************************}
\Msg{*}
\Msg{* To finish the installation you have to move the following}
\Msg{* file into a directory searched by TeX:}
\Msg{*}
\Msg{*     mylatexformat.ltx}
\Msg{*}
\Msg{* To produce the documentation run the file `mylatexformat.dtx'}
\Msg{* through LaTeX.}
\Msg{*}
\Msg{* Happy TeXing!}
\Msg{*}
\Msg{************************************************************************}

\endbatchfile
%</install>
%<*ignore>
\fi
%</ignore>
%<*driver>
\edef\thisfile{\jobname}
\def\thisinfo{Make a format based on the preamble of any \texorpdfstring{\LaTeX{}}{LaTeX} file}
\def\thisdate{2011/02/12}
\def\thisversion{3.4}
\let\loadclass\LoadClass
\def\LoadClass#1{\loadclass[abstracton]{scrartcl}\let\scrmaketitle\maketitle\AtEndOfClass{\let\maketitle\scrmaketitle}}
{\makeatletter{\endlinechar`\^^J\obeyspaces
 \gdef\ErrorUpdate#1=#2,{\@ifpackagelater{#1}{#2}{}{\let\CheckDate\errmessage\toks@\expandafter{\the\toks@
        \thisfile-documentation: updates required !
              package #1 must be later than #2
              to compile this documentation.}}}}%
 \gdef\CheckDate#1{{\let\CheckDate\relax\toks@{}\@for\x:=#1\do{\expandafter\ErrorUpdate\x,}\CheckDate\expandafter{\the\toks@}}}}
\AtBeginDocument{\DeleteShortVerb{\|}\CheckDate{interfaces=2011/01/26,tabu=2011/01/26}}
\PassOptionsToPackage{svgnames}{xcolor}
\PassOptionsToPackage{hyperfootnotes}{hyperref}
\documentclass[a4paper,oneside]{ltxdoc}
\usepackage[latin1]{inputenc}
\usepackage[american]{babel}
\usepackage[T1]{fontenc}
\usepackage{etex,etoolbox,geometry,lastpage,xspace,xcolor,bbding,txfonts,atveryend,moresize,relsize}
\usepackage{tocloft,titlesec,graphicx,fancyhdr,framed,multirow,makecell,tabu,enumitem,needspace}
\usepackage{holtxdoc,bookmark,embedfile,tabu,keycommand}
\usepackage{microtype}
\usepackage{interfaces}
\usetikz{basic}
\csname endofdump\endcsname
\CodelineNumbered
\usepackage{fancyvrb}\fvset{gobble=1,listparameters={\topsep=0pt}}
\lastlinefit999
\geometry{top=0pt,includeheadfoot,headheight=7mm,headsep=.6cm,bottom=.6cm,footskip=.5cm,left=4cm,right=14mm}
\hypersetup{%
  pdftitle={The mylatexformat package},
  pdfsubject={Use mylatexformat.ltx to make a format based on the preamble of any LaTeX file},
  pdfauthor={Florent CHERVET},
  colorlinks,linkcolor=reflink,
  pdfstartview={FitH},
  pdfkeywords={tex, e-tex, latex, package, format, formats, pdfTeX},
  bookmarksopen=true,bookmarksopenlevel=2}
\embedfile{\thisfile.dtx}
\begin{document}
   \DocInput{\thisfile.dtx}
\end{document}
%</driver>
% \fi
%
% \CheckSum{317}
%
% \CharacterTable
%  {Upper-case    \A\B\C\D\E\F\G\H\I\J\K\L\M\N\O\P\Q\R\S\T\U\V\W\X\Y\Z
%   Lower-case    \a\b\c\d\e\f\g\h\i\j\k\l\m\n\o\p\q\r\s\t\u\v\w\x\y\z
%   Digits        \0\1\2\3\4\5\6\7\8\9
%   Exclamation   \!     Double quote  \"     Hash (number) \#
%   Dollar        \$     Percent       \%     Ampersand     \&
%   Acute accent  \'     Left paren    \(     Right paren   \)
%   Asterisk      \*     Plus          \+     Comma         \,
%   Minus         \-     Point         \.     Solidus       \/
%   Colon         \:     Semicolon     \;     Less than     \<
%   Equals        \=     Greater than  \>     Question mark \?
%   Commercial at \@     Left bracket  \[     Backslash     \\
%   Right bracket \]     Circumflex    \^     Underscore    \_
%   Grave accent  \`     Left brace    \{     Vertical bar  \|
%   Right brace   \}     Tilde         \~}
%
% \DoNotIndex{\begin,\CodelineIndex,\CodelineNumbered,\def,\DisableCrossrefs,\~,\@ifpackagelater}
% \DoNotIndex{\ttfamily,\tiny,\sffamily,\small,\normalfont,\newindex,\makeindex,\makeglossary,\LARGE,\Large,\large,\itshape,\immediate}
% \DoNotIndex{\&,\/,\=,\@dofilelist,\@spaces,\@undefined,\^,\active,\bfseries,\boldmath,\etex@primitive@openout,\footnotesize,\g@addto@macro,\Huge}
% \DoNotIndex{\DocInput,\documentclass,\EnableCrossrefs,\end,\GetFileInfo}
% \DoNotIndex{\NeedsTeXFormat,\OnlyDescription,\RecordChanges,\usepackage}
% \DoNotIndex{\ProvidesClass,\ProvidesPackage,\ProvidesFile,\RequirePackage}
% \DoNotIndex{\filename,\fileversion,\filedate,\let}
% \DoNotIndex{\@listctr,\@nameuse,\csname,\else,\endcsname,\expandafter}
% \DoNotIndex{\gdef,\global,\if,\item,\newcommand,\nobibliography}
% \DoNotIndex{\par,\providecommand,\relax,\renewcommand,\renewenvironment}
% \DoNotIndex{\stepcounter,\usecounter,\nocite,\fi}
% \DoNotIndex{\@fileswfalse,\@gobble,\@ifstar,\@unexpandable@protect}
% \DoNotIndex{\AtBeginDocument,\AtEndDocument,\begingroup,\endgroup}
% \DoNotIndex{\frenchspacing,\MessageBreak,\newif,\PackageWarningNoLine}
% \DoNotIndex{\protect,\string,\xdef,\ifx,\texttt,\@biblabel,\bibitem}
% \DoNotIndex{\z@,\wd,\wheremsg,\vrule,\voidb@x,\verb,\bibitem}
% \DoNotIndex{\FrameCommand,\MakeFramed,\FrameRestore,\hskip,\hfil,\hfill,\hsize,\hspace,\hss,\hbox,\hb@xt@,\endMakeFramed,\escapechar}
% \DoNotIndex{\do,\date,\if@tempswa,\@tempdima,\@tempboxa,\@tempswatrue,\@tempswafalse,\ifdefined,\ifhmode,\ifmmode,\cr}
% \DoNotIndex{\box,\author,\advance,\multiply,\Command,\outer,\next,\leavevmode,\kern,\title,\toks@,\trcg@where,\tt}
% \DoNotIndex{\the,\width,\star,\space,\section,\subsection,\textasteriskcentered,\textwidth}
% \DoNotIndex{\",\:,\@empty,\@for,\@gtempa,\@latex@error,\@namedef,\@nameuse,\@tempa,\@testopt,\@width,\\,\m@ne,\makeatletter,\makeatother}
% \DoNotIndex{\maketitle,\parindent,\setbox,\x,\kernel@ifnextchar}
% \DoNotIndex{\KVS@CommaComma,\KVS@CommaSpace,\KVS@EqualsSpace,\KVS@Equals,\KVS@Global,\KVS@SpaceEquals,\KVS@SpaceComma,\KVS@Comma}
% \DoNotIndex{\DefineShortVerb,\DeleteShortVerb,\UndefineShortVerb,\MakeShortVerb,\endinput}
%
% \let\ClearPage\clearpage
% \makeatletter
% \catcode`\� \active   \def�{\@ifnextchar �{\par\nobreak\vskip-2\parskip}{\par\nobreak\vskip-\parskip}}
% \newrobustcmd\Xspace{\ifnum\catcode`\ =\active\else\expandafter\xspace\fi}
% \newrobustcmd*\stform{\ifincsname\else\expandafter\@stform\fi}
% \newrobustcmd*\@stform{\@ifnextchar*{\@@stform[]\textasteriskcentered\@gobble}\@@stform}
% \newrobustcmd*\@@stform[2][\string]{\textttbf{#1#2}\Xspace}
%
% \newrobustcmd*\thispackage{\@ifstar
%  {\xpackage{{\thisfile}}\Xspace}
%  {\xpackage{{\pkgcolor\thisfile}}\Xspace}}
% \def\ThisPackage{\Xpackage{\thisfile}\xspace}
% \def\CTANpackage#1{\href{http://www.tex.ac.uk/tex-archive/help/Catalogue/entries/#1.html}{\xpackage{#1}}}
% \def\Xpackage{\@dblarg\X@package}
% \def\X@package[#1]#2{\@testopt{\X@@package{#1}{#2}}{}}
% \def\X@@package#1#2[#3]{%
%     \xpackage{#2\footnote{\noindent\xpackage{#2}: \href{http://www.ctan.org/tex-archive/macros/latex/contrib/#1}{\nolinkurl{CTAN:macros/latex/contrib/#1}}#3}}}
% \def\Underbrace#1_#2{$\underbrace{\vtop to2ex{}\hbox{#1}}_{\footnotesize\hbox{#2}}$}
% \newrobustcmd*\FC[1][\color{copper}]{{#1\usefont{T1}{fts}xn FC}}
%
% \parindent\z@\parskip.4\baselineskip\topsep\parskip\partopsep\z@
% \g@addto@macro\macro@font{\macrocodecolor\let\AltMacroFont\macro@font}
% \g@addto@macro\@list@extra{\parsep\parskip\topsep\z@\itemsep\z@}
% \def\smex{\leavevmode\hb@xt@2em{\hfil$\longrightarrow$\hfil}}
% \newrobustcmd\verbfont{\usefont{T1}{\ttdefault}{\f@series}{n}}    \let\vb\verbfont
% \renewrobustcmd\#[1]{{\usefont{T1}{pcr}{bx}{n}\char`\##1}}
% \newrobustcmd\csred[1]{\textcolor{red}{\cs{#1}}}
% \renewrobustcmd\cs[2][]{\mbox{\vb#1\expandafter\@gobble\string\\#2}}
% \newrobustcmd\CSbf[1]{\textbf{\CS{#1}}}
% \newrobustcmd\csbf[2][]{\textbf{\cs[{#1}]{#2}}}
% \newrobustcmd\textttbf[1]{\textbf{\texttt{#1}}}
% \renewrobustcmd*\bf{\bfseries}\newcommand\nnn{\normalfont\mdseries\upshape}\newcommand\nbf{\normalfont\bfseries\upshape}
% \newrobustcmd*\blue{\color{blue}}\newcommand*\red{\color{dr}}\newcommand*\green{\color{green}}\newcommand\rred{\color{red}}
% \newrobustcmd\rrbf{\color{red}\bfseries}
% \newcommand\textt[1]{\texorpdfstring{\texttt{#1}}{#1}}
% \newrobustcmd\texttbf[1]{\textbf{\texttt{#1}}}
% \colorlet{pkgcolor}{teal}\def\pkgcolor{\color{pkgcolor}}
% \definecolor{copper}{rgb}{0.67,0.33,0.00}  \newcommand\copper{\color{copper}}
% \definecolor{dg}{rgb}{0.16,0.33,0.00}      \newcommand\dg{\color{dg}}
% \definecolor{db}{rgb}{0,0,0.502}           \newcommand\db{\color{db}}
% \definecolor{dr}{rgb}{0.49,0.00,0.00}      \let\dr\red
% \definecolor{lk}{rgb}{0.25,0.25,0.25}      \newrobustcmd\lk{\color{lk}}
% \newrobustcmd\bk{\color{black}}\newcommand\md{\mdseries}
% \def\M{\@ifstar{\M@i\@firstofone}{\M@i\meta}}
% \def\M@i#1{\@ifnextchar[^^A]
%   {\M@square#1}
%   {\@ifnextchar(^^A)
%     {\M@paren#1}
%     {\M@brace#1}}^^A
% }
% \def\M@square#1[#2]{\M@Bracket[{#1{#2}}]}
% \def\M@paren#1(#2){\M@Bracket({#1{#2}})}
% \def\M@brace#1#2{\M@Bracket\{{#1{#2}}\}}
% \def\M@Bracket#1#2#3{{\ttfamily#1#2#3}}
% \newrobustcmd*\thisyear{\begingroup
%    \def\thisyear##1/##2\@nil{\endgroup
%       \oldstylenums{\ifnum##1=2010\else 2010\,\textendash\,\fi ##1}^^A
%    }\expandafter\thisyear\thisdate\@nil
% }
%
% \def\declmargin{\hspace*\declmarginwidth }
% \def\declmarginwidth{\dimexpr -\leftmargini +\arrayrulewidth +\tabcolsep\relax}
% \renewrobustcmd\declcs[2][]{^^A
%   \if@nobreak \par\nobreak
%   \else \par\addvspace\parskip
%         \Needspace{.08\textheight}\fi
%   \changefont{size+=2.5pt,spread=1,fam=\ttdefault}^^A
%   \newcommand*\csanchor[2][]{\raisedhyperdef[14pt]{declcs}{##2}{\cs[{##1}]{##2}}}^^A
%   \def\*{\unskip\,\texttt{*}}\noindent
%   \hskip-\leftmargini
%   \begin{tabu}{|l|}\hline
%     \expandafter\SpecialUsageIndex\csname #2\endcsname
%     \csanchor[{#1}]{#2}}
% \renewcommand\enddeclcs{%
%     \crcr \hline \end{tabu}\nobreak
%     \par  \nobreak \noindent
%     \ignorespacesafterend
%  }
% \newrobustcmd*\csref[2][]{\hyperref{}{declcs}{#2}{\cs[{#1}]{#2}}}
%
% \let\plainllap\llap
% \newrobustcmd\macro@llap[1]{{\global\let\llap\plainllap
%  \setbox0=\hbox\bgroup \raisedhyperdef{macro}{\saved@macroname}{#1}\egroup
%  \ifdim\wd0>2.4cm
%     \hbox to\z@ \bgroup\hss \hbox to2.4cm{\unhcopy0\hss}\egroup
%     \edef\@tempa{\hskip\dimexpr\the\wd0-2.4cm}\global\everypar\expandafter{\the\expandafter\everypar
%                                                                            \@tempa \global\everypar{}}^^A
%  \else \llap{\unhbox0}\fi}}
% \pretocmd\macro{\if@nobreak\else\Needspace{2\baselineskip}\fi
%           \MacrocodeTopsep\z@skip \MacroTopsep\z@skip \parsep\z@ \topsep\z@ \itemsep\z@ \partopsep\z@
%           \let\llap\macro@llap}{}{}
% \apptocmd\endmacro{\goodbreak\vskip.3\parskip}{}{}
% \newrobustcmd\macroref[2][]{\hyperref{}{macro}{#2}{\cs[{#1}]{#2}}}
%
% \DefineVerbatimEnvironment{Verb}{Verbatim}{gobble=1}
% \DefineVerbatimEnvironment{VerbLines}{Verbatim}{gobble=1,frame=lines,framesep=6pt,fontfamily=txtt,fontseries=b}
% \newkeyenvironment+[\|]{dblruled}[first=.4pt,second=.4pt,sep=1pt,left=\z@]%
%   {|\def\FrameCommand|{%
%     \vrule\@width\commandkey{first}
%     \hskip\commandkey{sep}
%     \vrule\@width\commandkey{second}
%     |\hspace|{\commandkey{left}}}
%   |\MakeFramed| {\advance\hsize-|\width \FrameRestore|}%
%  }{%
%   |\endMakeFramed|%
%  }
%
% \colorlet{HeadBand}{BlanchedAlmond}
% \newcommand*\topheader{\tikz[remember picture,overlay]{
%   \coordinate (A) at (current page.north west);
%   \coordinate (B) at ($(current page.north east)+(0cm,-\headheight)$);
%   \fill [color=HeadBand,fill opacity=.6] (A) rectangle (B);
%   }%
% }
%
% \pagesetup[plain]{
%     norules,
%     foot/right=\oldstylenums{\thepage} / \oldstylenums{\pageref{LastPage}},
%     foot/font=\lk\scriptsize,
%     foot/left/offset=3cm,
%     foot/right/offset=1cm,
%     foot/left=\vbox to\baselineskip{\vss{{\rotatebox[origin=l]{90}{\thispackage\,[rev.\thisversion]\,\copyright\,\thisyear\,\lower.4ex\hbox{\pkgcolor\NibRight}\,\FC\quad \xemail{florent.chervet at free.fr}}}}},
% }
% \pagesetup{
%     foot/right={\oldstylenums{\thepage} / \oldstylenums{\pageref{LastPage}}},
%     foot/font=\lk\scriptsize,
%     foot/left/offset=3cm,
%     foot/right/offset=1cm,
%     foot/left=\vbox to\baselineskip{\vss{{\rotatebox[origin=l]{90}{\thispackage\,[rev.\thisversion]\,\copyright\,\thisyear\,\lower.4ex\hbox{\pkgcolor\NibRight}\,\FC}}}},
%     head/font=\small\sffamily,
%     head/right/offset=1cm,
%     head/right=\noindent\raise.4ex\hbox{\thispackage : \thisinfo},
%     head/left=\topheader,
% }
% \sectionformat\section{
%     top=\medskipamount,
%     bottom=.5\smallskipamount,
%     bookmark={color=MidnightBlue},
% }
% \sectionformat\subsubsection{
%     top=\smallskipamount,bottom=0pt,
%     bookmark={color=lk},
% }
% \bookmarksetup{openlevel=3}
%
% \newcommand\macrocodecolor{\color{macrocode}}\definecolor{macrocode}{rgb}{0.05,0.02,0.07}
% \newcommand\reflinkcolor{\color{reflink}}\definecolor{reflink}{rgb}{0.49,0.00,0.00}
% \newrobustcmd*\IMPLEMENTATION{%
%     \bookmarksetup{openlevel=1}
%     ^^A\geometry{top=0pt,includeheadfoot,headheight=7mm,headsep=.6cm,bottom=.6cm,footskip=.5cm,left=4cm,right=1.5cm}
%     \newgeometry{top=0pt,includeheadfoot,headheight=7mm,headsep=.6cm,bottom=.6cm,footskip=.5cm,left=3cm,right=5mm}
%     \pagesetup*{left/offset-=1cm,right/offset-=1.5cm-2mm}
%     \section{Implementation} \label{sec:implementation}
% }
%
% \makeatother
%
% \deffootnote{1em}{0pt}{\rlap{\textsuperscript{\thefootnotemark}}\kern1em}
%
% \title{\vspace*{-28pt}\href{http://www.tex.ac.uk/tex-archive/help/Catalogue/entries/mylatexformat.html}{\HUGE\bfseries\sffamily\pkgcolor mylatexformat}\vspace*{6pt}}
% \date{}
% \subtitle{\LARGE\begin{tabu}{X[c]}
%           Use \thispackage* to make a format \\
%           based on the preamble of any LaTeX file \\[-1ex]
%           \small\FC
%           \end{tabu}
%           \vspace*{-12pt}}
% \author{\smaller\thisdate~--~\hyperref[\thisversion]{version \thisversion}}
%
% \maketitle
%
% {\makeatletter\let\@thefnmark\@empty\let\@makefntext\@firstofone
% \footnotetext{\noindent\parskip=0pt\scriptsize
% This documentation is produced with the \textt{DocStrip} utility.\par
% \begin{tabu}{X[-3]X[-1]X}
% \smex To get the package,                   &run:          &\texttt{etex \thisfile.dtx}                  \\
% \smex To get the documentation              &run (thrice): &\textt{pdflatex \thisfile.dtx}               \\
% \leavevmode\hphantom\smex To get the index, &run:          &\texttt{makeindex -s gind.ist \thisfile.idx}
% \end{tabu}�
% The \xext{dtx} file is embedded into this pdf file thank to \Xpackage{embedfile} by H. Oberdiek.}
% }
%
% \deffootnote{1em}{0pt}{\rlap{\thefootnotemark.}\kern1em}
%
% {\let\quotation\relax\let\endquotation\relax\vspace*{-1cm}%
% \begin{abstract}\parskip\smallskipamount\parindent0pt\lastlinefit0\leftskip.7cm\rightskip\leftskip
%
% \thispackage can be used to make a format from the preamble of any \LaTeX{} file.
% The use of formats helps to speed up compilations: packages which have been
% dumped in the format are loaded at very high speed. A format (also known as \emph{dump}) is therefore very suitable:
% \begin{itemize}[label=\textendash,leftmargin=3em]
% \item if you have many documents that share the same preamble,
% \item if you have a document which loads a large amount of packages (like \xpackage{TikZ} graphics libraries).
% \end{itemize}
%
% \medskip
%
% \thispackage will dump all definitions until \cs{begin}\M*{document} or \csref{endofdump}, more precisely:
%
% \csdef{tabu*}{\begingroup\tabcolsep0pt\begin{tabu}}
% \csdef{endtabu*}{\end{tabu}\endgroup}
% \definecolor{ly}{rgb}{1,1,.35}
%
% \hskip-2em
% \begin{tabu}to \linewidth{XXX[1.2]}
%
% \fcolorbox{black}{ly}{%
% \begin{tabu*}{X}
% \cs{RequirePackage}\M*{...} \\
% \cs{documentclass}\M*{...}  \\
% \cs{usepackage}\M*{...}     \\
% \noindent\,\,...       \\
% \noindent\,\,...       \\
% \csbf{begin}{\bfseries\M*{document}}
% \end{tabu*}}
% &
% \begin{tabu*}{X}
% \fcolorbox{black}{ly}{%
% \begin{tabu}{X}
% \cs{RequirePackage}\M*{...} \\
% \cs{documentclass}\M*{...}  \\
% \cs{usepackage}\M*{...}     \\
% \csbf{endofdump}       \\
% \end{tabu}} \\
% \noindent\,\, ... \\
% \noindent\,\,\cs{begin}\M*{document}
% \end{tabu*}
% &
% \begin{tabu*}{X}
% \fcolorbox{black}{ly}{%
% \begin{tabu*}{X}
% \cs{RequirePackage}\M*{...} \\
% \cs{documentclass}\M*{...}  \\
% \cs{usepackage}\M*{...}     \\
% \csbf{csname} \texttbf{endofdump}\csbf{endcsname}       \\
% \end{tabu*}} \\
% \noindent\,\, ... \\
% \noindent\,\,\cs{begin}\M*{document}
% \end{tabu*}
% \end{tabu}
% \medskip
%
% The colored parts correspond to the format. The \xext{log} file will report:
%
% \begin{tabu}{@{}X|[pkgcolor]X}
% \begin{tabu}{>{\ttfamily}X[l]}
% start reading document "my document" \\
% on input line *** (\cs{begin}\M*{document})
% \end{tabu}
% &
% \begin{tabu}{>{\ttfamily}X[l]}
% start reading document "my document" \\
% on input line *** (\cs{endofdump})
% \end{tabu}
% \end{tabu}
% \medskip
%
% \thispackage was primarily written from \CTANpackage{mylatex}
% by David Carlisle, but it is different in many points (see \ref{mylatex}). In particular, \thispackage allows to put almost
% any package in the format, with one exception: the package \CTANpackage{minitoc}
% cannot be included in the format (you must use \csref{endofdump} before \cs{usepackage}\M*{minitoc}).
%
% \end{abstract}
% }
%
% \tocsetup{
%     section/skip=4pt plus2pt minus2pt,
%     subsection/skip=4pt plus2pt minus2pt,
%     title=Contents,
%     title/skip=2pt,
%     title/after={\pkgcolor\quad\leaders\vrule height3.4pt depth-3pt\hfill\null},
%     multicols/afterend=\leavevmode{\pkgcolor\hrule},
%     twocolumns,
%     bookmark={text=Contents,bold},
% }
% \tableofcontents
% \clearpage
%
%
% \MakeShortVerb{\+}
%
% \section{Creating a format}
%
% \subsection{Using MikTeX}
%
% \subsubsection{With preloaded pdflatex format (pdf output)}
%
% Use the following command line:
%
% \quad+etex -initialize [opts] "&pdflatex" mylatexformat.ltx """abc.tex"""+
%
% You need 3 quotes around your \xext{tex} file if its name contains space
% (only one quote if it does not contain any space, but 3 quotes is more general...)
% \medskip
%
% Full example:��
% \begin{VerbLines}
%  etex -initialize -save-size=20000 -stack-size=20000
%                   -jobname="The-Name-Of-The-.fmt-File(without .fmt extension)"
%                   "&pdflatex" mylatexformat.ltx """Mon Fichier TeX.tex"""
% \end{VerbLines}
%
%
% \subsubsection{With preloaded latex format (dvi output)}
%
% Use the following command line:
%
% \quad+etex -initialize [opts] "&latex" mylatexformat.ltx """abc.tex"""+
%
%
% \subsubsection{With preloaded format xelatex (pdf output)}
%
% Use the following command line:
%
% \quad+xetex -initialize [opts] "&xelatex" mylatexformat.ltx """abc.tex"""+
%
%
% \subsection{Using other distributions}
%
% As I pointed out in the \texttt{abstract}, I have not been able to test the file on other distributions.
%
% Some users told me they typed the following command line:
%
% +latex -ini -jobname="fmt-file-name" "&latex" mylatexformat.ltx  """abc.tex"""+
%
% \section{Using a format}
%
% \subsection{Using MikTeX}
%
% Add the format to the very first line of your LaTeX file :
%
% \quad+%&"Mon Beau Format"+
%
% You may use the $\varepsilon$-\TeX{} option :    +-dont-parse-first-line+ \quad to avoid loading the format.
%
% \subsection{Using other distributions}
%
% As I pointed out in the \texttt{abstract}, I have not been able to test the file on other distributions.
%
% \clearpage
% \section[How does it work ? (and why it is different from mylatex.ltx)]{How does it work ? (and why it is different from \xpackage{mylatex.ltx})}
% \label{mylatex}
%
% \xpackage{mylatex.ltx} has three important limitations:
% \begin{enumerate}[label=\arabic*)~]
% \item The end of the format is either \verb+\begin{document}+ or the comment \verb+%mylatex+\\
%       Therefore, after the format has been loaded, \TeX{} checks the commentaries in the preamble
%       in order to skip everything before \verb+%mylatex+. \\
%       But this way, braces \textbf{must match inside commentaries !}
% \item \xpackage{mylatex.ltx} temporarily redefines the command \cs{document} in order to stop
%      the creation of the format (execution of \cs{dump}) at begin document. \\
%      However, the \textbf{initial definition of the macro \csbf{document}} is saved
%      and reloaded after the format has been loaded. \\
%      Therefore, packages that modifies this macro (such as \Xpackage[oberdiek/auxhook]{auxhook}[ H. Oberdiek] or
%      \Xpackage{etoolbox}[ P. Lehman] cannot be included in the format!
% \item If your document starts with :\\
%              \verb+\begin{document}% commentary+ or with :\\
%              \verb+\begin{document}\makeatletter+ \\
%      the format is not stopped at \verb+\begin{document}+ and an error occurs.
% \end{enumerate}
%
% \begin{dblruled}[first=2pt,sep=1pt,second=.6pt,left=4pt]
% With \xpackage{mylatexformat.ltx}, those limitations are lifted.
%
% \xpackage{mylatexformat.ltx}:
% \begin{enumerate}[label=\arabic*)~]
% \item Temporarily redefines \cs{begin} in order to check if its argument is \meta{document}. \\
%       Therefore, packages like \xpackage{auxhook} or \xpackage{etoolbox} can be included in the format.
% \item Stops the format when it encounters \verb+\begin{document}+ or \cs{endofdump}.
% \item Does not read inside commentaries
% \item After the format has been loaded, each line is checked to see if it \textbf{contains} either:
%        \begin{description}[align=left,labelsep=1cm,font=\bfseries,labelindent=1cm]
%        \item[] {\dr\verb+\begin{document}+}
%        \item[or: ] {\dr\verb+\endofdump+}
%        \item[or: ] {\dr\verb+\csname endofdump\endcsname+}\textsuperscript{\bfseries\textasteriskcentered}
%        \end{description}
%      Report is written in the \xext{log} file, with the job name, the name of the preloaded format,
%      the list of preloaded files and the first input line read after loading of format.
% \end{enumerate}
%
% {\footnotesize\textsuperscript{\bfseries\textasteriskcentered} this way, the command \verb+\csname endofdump\endcsname+ is silently ignored
% (equiv. to \cs{relax}) in case the format is not used.}
% \end{dblruled}
%
%
%
%
% \StopEventually{
% }
%
% \IMPLEMENTATION\DeleteShortVerb{\+}
%
%    \begin{macrocode}
%<*package>
%%% _____________________________________________________________________
%%%                           mylatexformat.ltx
%%%                           ^^^^^^^^^^^^^^^^^
%%% From an original idea by David Carlisle - mylatex.ltx 1994/09/27
%%% � lppl 2010-2011 - F Chervet - 2011/02/12 v3.4
%%% _____________________________________________________________________
%%%
%%% Use this file to make a format based on the preamble of any LaTeX
%%% file.
%%%
%%% The format is made from the very beginning of the file up to either:
%%%   i) the command : \endofdump
%%%                    placed in the preamble
%%%  ii) the command : \csname endofdump\endcsname
%%%                    placed in the preamble
%%% iii) \begin{document}
%%%                    if \endofdump nor \csname endofdump\endcsname
%%%                    haven't been found in the preamble.
%%%
%%% A format helps to speed up compilations, packages which have been
%%% dumped in the format are loaded at very high speed.
%%% --------------------------------------------------------------------
%%% There are no restrictions on the distribution or modification of
%%% this file, except that other people should not attempt to alter
%%% the master copy on the ctan archives.
%%%
%%% Making the format --------------------------------------------------
%%% ^^^^^^^^^^^^^^^^^                                       USING MikTeX
%%% *  With preloaded pdflatex format (pdf output) :
%%%    etex -initialize [opts] "&pdflatex" mylatexformat.ltx """abc.tex"""
%%%    => compile your document using :
%%%            pdflatex
%%%         or texify --tex-option=-output-format=pdf
%%%
%%% *  With preloaded latex format (dvi output) :
%%%    etex -initialize [opts] "&latex" mylatexformat.ltx """abc.tex"""
%%%    => compile your document using
%%%            latex
%%%         or texify --tex-option=-output-format=dvi
%%%
%%% * With preloaded format xelatex (pdf output) :
%%%    xetex -initialize [opts] "&xelatex" mylatexformat.ltx """abc.tex"""
%%%    => compile your document using xelatex
%%%
%%% NOTA BENE:    Please, notice the 3 quotes !
%%% ^^^^^^^^^^
%%%
%%% In fact, 3 quotes are necessary only if you "file name.tex" contains a space !
%%%
%%% -=-=-=-=-=-=-=-=-=-=-=-=-=-=-=-=-=-=-=-=-=-=-=-=-=-=-=-=-=-=-=-=-=-=
%%% Exemple of a command line for making format using pdfTeX :
%%%
%%%   etex -initialize -interaction=nonstopmode
%%%                    -save-size=20000 -stack-size=20000
%%%                    -jobname="Mon Beau Format Perso"
%%%                    "&pdflatex" mylatexformat.ltx "Mon Fichier TeX.tex"
%%%
%%%   "Mon Beau Format Perso.fmt" (en Fran�ais dans le texte) will be created
%%%   (or overwritten) in the current working directory.
%%%
%%% Using the format ---------------------------------------------------
%%% ^^^^^^^^^^^^^^^^^                                       USING MikTeX
%%% Add the format to the very first line of your LaTeX file :
%%% --> first line :       %&"format file name"
%%% without the .fmt extension.
%%% [You may use the option :    -dont-parse-first-line option
%%%  to avoid loading the format]
%%% --------------------------------------------------------------------
%% ____________________________________________________________________
%% The following informations come from mylatex.ltx from David Carlisle:
%% Given a LaTeX file `abc.tex', use initex as follows:
%%
%% initex &latex  mylatexformat.ltx abc      (Generic TeX)
%% initex \&latex mylatexformat.ltx abc      (unix and other TeX's)
%% tex /i &latex  mylatexformat.ltx abc      (emtex)
%%
%% If you are on a Mac or using some shell that makes it inconvenient
%% to use a command line such as the above examples then you may
%% make a file `mylatex.tex' with the single line
%% \input mylatex.ltx abc
%% and then pass the file mylatex.tex to your (ini)tex shell to produce
%% the format, ie something equivalent to initex &latex mylatex.tex.
%%
%% If you are using OzTeX, see the separate instructions below.
%%
%% This should make a format file mylatex.fmt which you can then use
%% as follows
%%
%% Using the new format
%% ^^^^^^^^^^^^^^^^^^^^
%%
%% tex  &mylatex abc          (generic TeX)
%% virtex \&mylatex abc       (Unix TeX)
%%
%% This will process your document, abc.tex, just as LaTeX does, but
%% quicker as the contents of the preamble will be stored in the
%% format file and will not need to be run each time.
%%
%% If (vir)tex fails to find your mylatex.fmt then it is not searching
%% in the current directory, either modify your TEXFORMATS path (or
%% equivalent) to search `.' or (on unix/dos) use   ./  as in:
%% virtex \&./mylatex abc
%%
%% Making and using the format with OzTeX
%% ^^^^^^^^^^^^^^^^^^^^^^^^^^^^^^^^^^^^^^
%%
%% Given a LaTeX file `abc.tex', do this: select `TeX...' from OzTeX's
%% TeX menu, and go to the folder containing the file `abc.tex' as if you
%% were selecting the file `abc.tex'. Then hit the Cancel button - this
%% procedure sets OzTeX's working folder to the one containing abc.tex.
%% Next, select iniTeX from the TeX menu, and type:
%%
%% &latex mylatex.ltx abc
%%
%% This should make a format file mylatex.fmt which you can save in the
%% same folder as the file you're working on.
%%
%% To use the new format, put this at the very start of the very first
%% line of your document:
%%
%% %&mylatex
%%
%% Further Notes
%%
%% The file abc.tex must contain a line *just* with
%% \begin{document}
%% Everything up to (but not including) the \begin{document} will
%% be saved in the format and not executed in subsequent runs.
%%
%% If you are modifying the document (or working on a similar document)
%% you may wish to add new commands to your document preamble.
%% The `mylatex' format normally skips the whole preamble (believing
%% it to be pre-loaded) and so such new commands do not take effect.
%% You could re-make the format, preloading the new preamble, but that
%% might be inconvenient to do every time, and so an alternative scheme
%% has been introduced.
%% If the preamble contains a comment   mylatex  (ie a line just
%% containing a % white space and the word mylatex) then the mylatex
%% format will start reading the preable at that point so any new
%% commands can be placed after such a comment and they will be
%% executed.
%% --------------------------------------------------------------------
%%
%%\RequirePackage{etex}% RequirePackage may be used before the class
%%
%% In principle \openout stream= filename need not be space terminated,
%% and need not be immediate, but this covers \makeindex \makeglossary
%% and index package's \newindex which are all the cases of \openout
%% that occur before \begin{document} that I could see.
%% Thanks to Ross Moore for pointing out \AtBeginDocument is too late
%% eg changebar package *closes* the stream in \AtBeginDocument, so need
%% to make sure it is opened before that. Make a special purpose hook.
\makeatletter
%    \end{macrocode}
%
%
%    \begin{macro}{\begin}
%
% \cs{begin} is overloaded in order to stop \cs{dump} at begin document
% if the macro \cs{endofdump} is not given somewhere in the preamble.
% Rem: it is NOT POSSIBLE to redefine \cs{document}, for the macro
%      \cs{document} is changed or patched by some classes or packages
%      (\xpackage{auxhook}, \xpackage{etoolbox}  etc.)
%
% With \Xpackage{etoolbox} version \textt{2.1} and earlier, \cs{begin} is patched,
% in order to insert the hooks \cs{AtBeginEnvironment} and \cs{AtEndEnvironment}.
%
% Therefore, it's not possible to restore, at end of dump, the orginal definition
% of \cs{begin} in the \LaTeX{} kernel. Instead, \thispackage inserts a reversible
% patch of the \cs{begin} command: the patched is reversed at end of dump by
% \macroref{MYLATEX@cleanup}.
%
%    \begin{macrocode}
%% \begin is overloaded in order to stop \dump at begin document
%% if the macro \endofdump is not given somewhere in the preamble.
%% Rem: it is NOT POSSIBLE to redefine \document, for the macro
%%      \document is changed or patched by some classes or packages
%%      (auxhook, etoolbox etc.)
\begingroup \def\x #1{\endgroup
   \gdef\begin ##1{\MYLATEX@StopAtdocument{##1}#1}
}\expandafter\x\expandafter{\begin{#1}}
\def\MYLATEX@StopAtdocument #1{\expandafter
   \ifx\csname #1\endcsname\document \expandafter\endofdump \fi
}% \MYLATEX@StopAtdocument
%    \end{macrocode}
%    \end{macro}
%
%    \begin{macro}{\openout}
%
%    Files cannot be opened out during the building of the format.
%    Instead, \cs{openout} is delayed until the end of the dump.
%
%    \begin{macrocode}
\let\MYLATEX@openout \openout
\def\openout#1 {\g@addto@macro\MYLATEX@opens{\immediate\openout#1 }}
\def\MYLATEX@opens {\let\MYLATEX@opens \@undefined}
%    \end{macrocode}
%    \end{macro}
%
%    \begin{macro}{\MYLATEX@cleanup}
%
%    Final clean up done at \cs{endofdump}: reverse the path for \cs{begin}
%    and restores \cs{openout} primitive.
%
%    \begin{macrocode}
\def\MYLATEX@cleanup {\let\MYLATEX@cleanup \@undefined
   \def\MYLATEX@StopAtdocument ##1\MYLATEX@StopAtdocument ##2##3\@nil{%
      \gdef\begin ####1{##1##3}%
   }\expandafter\MYLATEX@StopAtdocument \begin{##1}\@nil
   \let\openout                \MYLATEX@openout
   \let\MYLATEX@openout        \@undefined
   \let\MYLATEX@StopAtdocument \@undefined
}% \MYLATEX@cleanup
%    \end{macrocode}
%    \end{macro}
%
%
%    \begin{macro}{\MYLATEX@fontpreloading}
%
%    Some font can be preloaded: this gives an advantage to the format.
%
%    A hook for \CTANpackage{cmap} is necessary.
%
%    \begin{macrocode}
\def\MYLATEX@fontpreloading{\let\MYLATEX@fontpreloading \@undefined
   \begingroup \setbox0=\hbox {%
      $$% math (not bold, some setups don't have \boldmath)
      \def\x {\bfseries\itshape}{\itshape}\ttfamily\sffamily
      \normalfont \x
      \ifdefined\large \large \x \fi
      \ifdefined\Large \Large \x \fi
      \ifdefined\LARGE \LARGE \x \fi
      \ifdefined\Huge  \Huge  \x \fi
      \ifdefined\small \small \x \fi
      \ifdefined\footnotesize \footnotesize \x \fi
      \ifdefined\tiny  \tiny \x \fi}%
      \@for\x:={lae,lfe,lgr,ot1,ot1tt,ot6,t1,t2a,t2b,t2c,t5}\do{%
         \ifcsname cmap@set@\x\endcsname  \global\expandafter
               \let\csname cmap@set@\x\endcsname \@undefined   \fi
         \uppercase\expandafter{\expandafter\def\expandafter\x\expandafter{\x}}%
         \ifcsname cmap@set@\x\endcsname  \global\expandafter
               \let\csname cmap@set@\x\endcsname \@undefined   \fi}%
   \endgroup
}% \MYLATEX@fontpreloading
%    \end{macrocode}
%    \end{macro}
%
%
%    \begin{macro}{\endofdump}
%
%
%    \begin{macrocode}
\let\MYLATEX@listfiles \listfiles
\def\endofdump{\let\endofdump \relax
   \MYLATEX@cleanup
   \MYLATEX@fontpreloading
   \makeatother
   \everyjob\expandafter{\the\everyjob
      \let\MYLATEX@temp \listfiles
      {%
         \MYLATEX@listfiles
         \global\let\MYLATEX@listfiles \@undefined
         \expandafter\MYLATEX@banner \@dofilelist
      }%
      \let\listfiles \MYLATEX@temp     % to be able to used \listfiles once after the format
      \let\MYLATEX@temp \@undefined
      \MYLATEX@scanpreamble}%
   \dump
}% \endofdump
%    \end{macrocode}
%    \end{macro}
%
%
%
%    \begin{macro}{\MYLATEX@banner}
%    \begin{macro}{\MYLATEX@start}
%
%    Set to be expanded at \cs{everyjob}: displays a banner in the log file.
%
%    \begin{macrocode}
%% Banner for \everyjob.
{\@makeother\"  % just in case
\xdef\MYLATEX@banner #1#2#3\typeout #4{%
   \global\let\noexpand\MYLATEX@banner \noexpand\@undefined
   \newlinechar`\noexpand\^^J\message{%
   ==============================================================================^^J%
   JOB NAME\@spaces\@spaces\space: "\noexpand\jobname"^^J%
   CUSTOMISED FORMAT: "\jobname"^^J%
   PRELOADED FILES:^^J}%
   #3%
   \message{%
   ==============================================================================}%
}% \MYLATEX@banner
\long\gdef\MYLATEX@start #1#2{\global\let\MYLATEX@start \@undefined
   \message{%
   (mylatexformat)Info: start reading document "\jobname"^^J%
   (mylatexformat)\@spaces\space\on@line.
            (\ifcase#1 \string\endofdump\else\string\begin{document}\fi)^^J%
   ==============================================================================}%
   \endgroup
   \MYLATEX@opens\relax #2%
}% \MYLATEX@start
}% \catcode group

%    \end{macrocode}
%    \end{macro}
%    \end{macro}
%
%    \begin{macro}{\MYLATEX@scanpreamble}
%
%    While the preamble is being skipped (inside a group)
%    the EOL is active and defined to grab each line and
%    inspect it looking for :
%    \begin{Verbatim}
%            \endofdump
%     or     \csname endofdump\endcsname
%     or     \begin{document}
%    \end{Verbatim}
%
%    \begin{macrocode}
%% While the preamble is being skipped (inside a group)
%% the EOL is active and defined to grab each line and
%% inspect it looking for :
%%        \endofdump
%% or     \csname endofdump\endcsname
%% or     \begin{document}.
%%
%% The special catcodes required are not enabled until after the
%% first TeX command in the file, so as to avoid problems with
%% the special processing that TeX does on the first line, choosing
%% the format, or the file name etc.
{% catcode group
\catcode`\^^M=13 \catcode`\&  =8 %
\long\gdef\MYLATEX@scanpreamble {%
 \begingroup \catcode`\^^M=13%
 \long\def\MYLATEX@endofdump ##1\endofdump##2##3&##4##5##6\MYLATEX {##5}%
 \long\def\MYLATEX@csendofdump ##1\csname endofdump\endcsname##2##3&##4##5##6\MYLATEX {##5}%
 \long\def\MYLATEX@document ##1document##2##3&##4##5##6\MYLATEX {##5}%
 \long\def\MYLATEX@begindocument ##1\begin##2##3\MYLATEX##4{%
    \MYLATEX@document ##2document&&{\MYLATEX@start1{##4}}{^^M}\MYLATEX}%
 \long\def^^M##1^^M{%
    \MYLATEX@endofdump##1\endofdump&&%
       {\MYLATEX@start 0{##1}}%
       {\MYLATEX@csendofdump ##1\csname endofdump\endcsname&&%
          {\MYLATEX@start 0{##1}}%
          {\MYLATEX@begindocument ##1\begin\relax\MYLATEX{##1}}%
          \MYLATEX}%
       \MYLATEX}%
 ^^M%
}% \MYLATEX@scanpreamble
}% catcode group
%    \end{macrocode}
%    \end{macro}
%
%    \begin{macrocode}
%% Trick lookahead to allow mylatex.ltx and the document filename to be
%% given on the same command line. (initex &latex mylatex.ltx {abc.tex})
\expandafter\input\endinput%
%</package>
%    \end{macrocode}
%
% \restoregeometry\pagesetup*{left/offset+=1cm,right/offset+=1.5cm-2mm}
% \begin{History}
%   \sectionformat\subsection{font+=\smaller,bottom=\smallskipamount,top=\medskipamount}
%
%   \begin{Version}{2011/02/12 v3.4}\HistLabel{3.4}
%   \item Reorganisation of the code + optimisation.
%   \end{Version}
%
%   \begin{Version}{2011/01/19 v3.3}\HistLabel{3.3}
%   \item Modification of the code for \cs{begin} in order to keep the patching introduced by
%         \CTANpackage{etoolbox} version \textt{2.1} for the hooks \cs{AtBeginEnvironment}...\cs{AtEndEnvironement}.
%   \end{Version}
%
%   \begin{Version}{2011/01/14 v3.1}\HistLabel{3.1}
%   \item \cs{@for} loop introduced for compatibility with the \xpackage{cmap} package had the bad side effect to define a control sequence \cs{x}. Fixed.
%   \end{Version}
%
%   \begin{Version}{2011/01/12 v3.0}\HistLabel{3.0}
%   \item Fix a problem with package \CTANpackage{cmap} which could not be
%         included in the format.
%   \item Documentation revisited after Users' feedback (and the \CTANpackage{interfaces} package).
%   \end{Version}
%
%   \begin{Version}{2010/08/09 v2.9}\HistLabel{2.9}
%   \item Modification of the macros that scan the preamble when the \xext{tex} file is compiled with its format.\\
%     It is now possible to start a file with \cs{begin}\texttt{\{document\}} just after the format specification (\texttt{\&myformat}).
%   \item[To do:] Command line to make the format on Linux (\texttt{-ini} instead of \texttt{-initialize} ???).
%   \end{Version}
%
%   \begin{Version}{2010/06/30 v2.1}\HistLabel{2.1}
%   \item Added: \cs{listfiles} is now available in the preamble, after \cs{endofdump}, when the document is compiled with its format.
%   \item Documentation update.
%   \end{Version}
%
%   \begin{Version}{2010/05/20 v1.2}\HistLabel{1.2}
%   \item Fixed a bug in \cs{MYLATEXopens} \\
%         (There was problems when including an index in the format for example...)
%   \end{Version}
%
%   \begin{Version}{2010/04/30 v1.0}\HistLabel{1.0}
%   \item First version.
%   \end{Version}
%
% \end{History}
%
% ^^A\restoregeometry\pagesetup*{left/offset+=1cm,right/offset+=1.5cm-2mm}
% \clearpage
% \PrintIndex
%
% \Finale
%        (quote the arguments according to the demands of your shell)
%
% Documentation:
%           (pdf)latex mylatexformat.dtx
% Copyright (C) 2010-2011 by Florent Chervet <florent.chervet@free.fr>
%<*ignore>
\begingroup
  \def\x{LaTeX2e}%
\expandafter\endgroup
\ifcase 0\ifx\install y1\fi\expandafter
         \ifx\csname processbatchFile\endcsname\relax\else1\fi
         \ifx\fmtname\x\else 1\fi\relax
\else\csname fi\endcsname
%</ignore>
%<*install>
\input docstrip.tex
\Msg{************************************************************************}
\Msg{* Installation}
\Msg{* Package 2011/02/12 v3.4 - Use mylatexformat.ltx to make a format based on the preamble of any LaTeX file*}
\Msg{************************************************************************}

\keepsilent
\askforoverwritefalse

\let\MetaPrefix\relax
\preamble

This is a generated file.

mylatexformat : 2011/02/12 v3.4 - Use mylatexformat.ltx to make a format based on the preamble of any LaTeX file

This work may be distributed and/or modified under the
conditions of the LaTeX Project Public License, either
version 1.3 of this license or (at your option) any later
version. The latest version of this license is in
   http://www.latex-project.org/lppl.txt

This work consists of the main source file mylatexformat.dtx
and the derived files
   mylatexformat.ltx, mylatexformat.pdf, mylatexformat.ins

mylatexformat - Use mylatexformat.ltx to make a format based on the preamble of any LaTeX file
Copyright (C) 2010-2011 by Florent Chervet <florent.chervet@free.fr>

\endpreamble
\let\MetaPrefix\DoubleperCent

\generate{%
   \file{mylatexformat.ins}{\from{mylatexformat.dtx}{install}}%
   \file{mylatexformat.ltx}{\from{mylatexformat.dtx}{package}}%
}

\askforoverwritefalse
\generate{%
   \file{mylatexformat.drv}{\from{mylatexformat.dtx}{driver}}%
}

\obeyspaces
\Msg{************************************************************************}
\Msg{*}
\Msg{* To finish the installation you have to move the following}
\Msg{* file into a directory searched by TeX:}
\Msg{*}
\Msg{*     mylatexformat.ltx}
\Msg{*}
\Msg{* To produce the documentation run the file `mylatexformat.dtx'}
\Msg{* through LaTeX.}
\Msg{*}
\Msg{* Happy TeXing!}
\Msg{*}
\Msg{************************************************************************}

\endbatchfile
%</install>
%<*ignore>
\fi
%</ignore>
%<*driver>
\edef\thisfile{\jobname}
\def\thisinfo{Make a format based on the preamble of any \texorpdfstring{\LaTeX{}}{LaTeX} file}
\def\thisdate{2011/02/12}
\def\thisversion{3.4}
\let\loadclass\LoadClass
\def\LoadClass#1{\loadclass[abstracton]{scrartcl}\let\scrmaketitle\maketitle\AtEndOfClass{\let\maketitle\scrmaketitle}}
{\makeatletter{\endlinechar`\^^J\obeyspaces
 \gdef\ErrorUpdate#1=#2,{\@ifpackagelater{#1}{#2}{}{\let\CheckDate\errmessage\toks@\expandafter{\the\toks@
        \thisfile-documentation: updates required !
              package #1 must be later than #2
              to compile this documentation.}}}}%
 \gdef\CheckDate#1{{\let\CheckDate\relax\toks@{}\@for\x:=#1\do{\expandafter\ErrorUpdate\x,}\CheckDate\expandafter{\the\toks@}}}}
\AtBeginDocument{\DeleteShortVerb{\|}\CheckDate{interfaces=2011/01/26,tabu=2011/01/26}}
\PassOptionsToPackage{svgnames}{xcolor}
\PassOptionsToPackage{hyperfootnotes}{hyperref}
\documentclass[a4paper,oneside]{ltxdoc}
\usepackage[latin1]{inputenc}
\usepackage[american]{babel}
\usepackage[T1]{fontenc}
\usepackage{etex,etoolbox,geometry,lastpage,xspace,xcolor,bbding,txfonts,atveryend,moresize,relsize}
\usepackage{tocloft,titlesec,graphicx,fancyhdr,framed,multirow,makecell,tabu,enumitem,needspace}
\usepackage{holtxdoc,bookmark,embedfile,tabu,keycommand}
\usepackage{microtype}
\usepackage{interfaces}
\usetikz{basic}
\csname endofdump\endcsname
\CodelineNumbered
\usepackage{fancyvrb}\fvset{gobble=1,listparameters={\topsep=0pt}}
\lastlinefit999
\geometry{top=0pt,includeheadfoot,headheight=7mm,headsep=.6cm,bottom=.6cm,footskip=.5cm,left=4cm,right=14mm}
\hypersetup{%
  pdftitle={The mylatexformat package},
  pdfsubject={Use mylatexformat.ltx to make a format based on the preamble of any LaTeX file},
  pdfauthor={Florent CHERVET},
  colorlinks,linkcolor=reflink,
  pdfstartview={FitH},
  pdfkeywords={tex, e-tex, latex, package, format, formats, pdfTeX},
  bookmarksopen=true,bookmarksopenlevel=2}
\embedfile{\thisfile.dtx}
\begin{document}
   \DocInput{\thisfile.dtx}
\end{document}
%</driver>
% \fi
%
% \CheckSum{317}
%
% \CharacterTable
%  {Upper-case    \A\B\C\D\E\F\G\H\I\J\K\L\M\N\O\P\Q\R\S\T\U\V\W\X\Y\Z
%   Lower-case    \a\b\c\d\e\f\g\h\i\j\k\l\m\n\o\p\q\r\s\t\u\v\w\x\y\z
%   Digits        \0\1\2\3\4\5\6\7\8\9
%   Exclamation   \!     Double quote  \"     Hash (number) \#
%   Dollar        \$     Percent       \%     Ampersand     \&
%   Acute accent  \'     Left paren    \(     Right paren   \)
%   Asterisk      \*     Plus          \+     Comma         \,
%   Minus         \-     Point         \.     Solidus       \/
%   Colon         \:     Semicolon     \;     Less than     \<
%   Equals        \=     Greater than  \>     Question mark \?
%   Commercial at \@     Left bracket  \[     Backslash     \\
%   Right bracket \]     Circumflex    \^     Underscore    \_
%   Grave accent  \`     Left brace    \{     Vertical bar  \|
%   Right brace   \}     Tilde         \~}
%
% \DoNotIndex{\begin,\CodelineIndex,\CodelineNumbered,\def,\DisableCrossrefs,\~,\@ifpackagelater}
% \DoNotIndex{\ttfamily,\tiny,\sffamily,\small,\normalfont,\newindex,\makeindex,\makeglossary,\LARGE,\Large,\large,\itshape,\immediate}
% \DoNotIndex{\&,\/,\=,\@dofilelist,\@spaces,\@undefined,\^,\active,\bfseries,\boldmath,\etex@primitive@openout,\footnotesize,\g@addto@macro,\Huge}
% \DoNotIndex{\DocInput,\documentclass,\EnableCrossrefs,\end,\GetFileInfo}
% \DoNotIndex{\NeedsTeXFormat,\OnlyDescription,\RecordChanges,\usepackage}
% \DoNotIndex{\ProvidesClass,\ProvidesPackage,\ProvidesFile,\RequirePackage}
% \DoNotIndex{\filename,\fileversion,\filedate,\let}
% \DoNotIndex{\@listctr,\@nameuse,\csname,\else,\endcsname,\expandafter}
% \DoNotIndex{\gdef,\global,\if,\item,\newcommand,\nobibliography}
% \DoNotIndex{\par,\providecommand,\relax,\renewcommand,\renewenvironment}
% \DoNotIndex{\stepcounter,\usecounter,\nocite,\fi}
% \DoNotIndex{\@fileswfalse,\@gobble,\@ifstar,\@unexpandable@protect}
% \DoNotIndex{\AtBeginDocument,\AtEndDocument,\begingroup,\endgroup}
% \DoNotIndex{\frenchspacing,\MessageBreak,\newif,\PackageWarningNoLine}
% \DoNotIndex{\protect,\string,\xdef,\ifx,\texttt,\@biblabel,\bibitem}
% \DoNotIndex{\z@,\wd,\wheremsg,\vrule,\voidb@x,\verb,\bibitem}
% \DoNotIndex{\FrameCommand,\MakeFramed,\FrameRestore,\hskip,\hfil,\hfill,\hsize,\hspace,\hss,\hbox,\hb@xt@,\endMakeFramed,\escapechar}
% \DoNotIndex{\do,\date,\if@tempswa,\@tempdima,\@tempboxa,\@tempswatrue,\@tempswafalse,\ifdefined,\ifhmode,\ifmmode,\cr}
% \DoNotIndex{\box,\author,\advance,\multiply,\Command,\outer,\next,\leavevmode,\kern,\title,\toks@,\trcg@where,\tt}
% \DoNotIndex{\the,\width,\star,\space,\section,\subsection,\textasteriskcentered,\textwidth}
% \DoNotIndex{\",\:,\@empty,\@for,\@gtempa,\@latex@error,\@namedef,\@nameuse,\@tempa,\@testopt,\@width,\\,\m@ne,\makeatletter,\makeatother}
% \DoNotIndex{\maketitle,\parindent,\setbox,\x,\kernel@ifnextchar}
% \DoNotIndex{\KVS@CommaComma,\KVS@CommaSpace,\KVS@EqualsSpace,\KVS@Equals,\KVS@Global,\KVS@SpaceEquals,\KVS@SpaceComma,\KVS@Comma}
% \DoNotIndex{\DefineShortVerb,\DeleteShortVerb,\UndefineShortVerb,\MakeShortVerb,\endinput}
%
% \let\ClearPage\clearpage
% \makeatletter
% \catcode`\� \active   \def�{\@ifnextchar �{\par\nobreak\vskip-2\parskip}{\par\nobreak\vskip-\parskip}}
% \newrobustcmd\Xspace{\ifnum\catcode`\ =\active\else\expandafter\xspace\fi}
% \newrobustcmd*\stform{\ifincsname\else\expandafter\@stform\fi}
% \newrobustcmd*\@stform{\@ifnextchar*{\@@stform[]\textasteriskcentered\@gobble}\@@stform}
% \newrobustcmd*\@@stform[2][\string]{\textttbf{#1#2}\Xspace}
%
% \newrobustcmd*\thispackage{\@ifstar
%  {\xpackage{{\thisfile}}\Xspace}
%  {\xpackage{{\pkgcolor\thisfile}}\Xspace}}
% \def\ThisPackage{\Xpackage{\thisfile}\xspace}
% \def\CTANpackage#1{\href{http://www.tex.ac.uk/tex-archive/help/Catalogue/entries/#1.html}{\xpackage{#1}}}
% \def\Xpackage{\@dblarg\X@package}
% \def\X@package[#1]#2{\@testopt{\X@@package{#1}{#2}}{}}
% \def\X@@package#1#2[#3]{%
%     \xpackage{#2\footnote{\noindent\xpackage{#2}: \href{http://www.ctan.org/tex-archive/macros/latex/contrib/#1}{\nolinkurl{CTAN:macros/latex/contrib/#1}}#3}}}
% \def\Underbrace#1_#2{$\underbrace{\vtop to2ex{}\hbox{#1}}_{\footnotesize\hbox{#2}}$}
% \newrobustcmd*\FC[1][\color{copper}]{{#1\usefont{T1}{fts}xn FC}}
%
% \parindent\z@\parskip.4\baselineskip\topsep\parskip\partopsep\z@
% \g@addto@macro\macro@font{\macrocodecolor\let\AltMacroFont\macro@font}
% \g@addto@macro\@list@extra{\parsep\parskip\topsep\z@\itemsep\z@}
% \def\smex{\leavevmode\hb@xt@2em{\hfil$\longrightarrow$\hfil}}
% \newrobustcmd\verbfont{\usefont{T1}{\ttdefault}{\f@series}{n}}    \let\vb\verbfont
% \renewrobustcmd\#[1]{{\usefont{T1}{pcr}{bx}{n}\char`\##1}}
% \newrobustcmd\csred[1]{\textcolor{red}{\cs{#1}}}
% \renewrobustcmd\cs[2][]{\mbox{\vb#1\expandafter\@gobble\string\\#2}}
% \newrobustcmd\CSbf[1]{\textbf{\CS{#1}}}
% \newrobustcmd\csbf[2][]{\textbf{\cs[{#1}]{#2}}}
% \newrobustcmd\textttbf[1]{\textbf{\texttt{#1}}}
% \renewrobustcmd*\bf{\bfseries}\newcommand\nnn{\normalfont\mdseries\upshape}\newcommand\nbf{\normalfont\bfseries\upshape}
% \newrobustcmd*\blue{\color{blue}}\newcommand*\red{\color{dr}}\newcommand*\green{\color{green}}\newcommand\rred{\color{red}}
% \newrobustcmd\rrbf{\color{red}\bfseries}
% \newcommand\textt[1]{\texorpdfstring{\texttt{#1}}{#1}}
% \newrobustcmd\texttbf[1]{\textbf{\texttt{#1}}}
% \colorlet{pkgcolor}{teal}\def\pkgcolor{\color{pkgcolor}}
% \definecolor{copper}{rgb}{0.67,0.33,0.00}  \newcommand\copper{\color{copper}}
% \definecolor{dg}{rgb}{0.16,0.33,0.00}      \newcommand\dg{\color{dg}}
% \definecolor{db}{rgb}{0,0,0.502}           \newcommand\db{\color{db}}
% \definecolor{dr}{rgb}{0.49,0.00,0.00}      \let\dr\red
% \definecolor{lk}{rgb}{0.25,0.25,0.25}      \newrobustcmd\lk{\color{lk}}
% \newrobustcmd\bk{\color{black}}\newcommand\md{\mdseries}
% \def\M{\@ifstar{\M@i\@firstofone}{\M@i\meta}}
% \def\M@i#1{\@ifnextchar[^^A]
%   {\M@square#1}
%   {\@ifnextchar(^^A)
%     {\M@paren#1}
%     {\M@brace#1}}^^A
% }
% \def\M@square#1[#2]{\M@Bracket[{#1{#2}}]}
% \def\M@paren#1(#2){\M@Bracket({#1{#2}})}
% \def\M@brace#1#2{\M@Bracket\{{#1{#2}}\}}
% \def\M@Bracket#1#2#3{{\ttfamily#1#2#3}}
% \newrobustcmd*\thisyear{\begingroup
%    \def\thisyear##1/##2\@nil{\endgroup
%       \oldstylenums{\ifnum##1=2010\else 2010\,\textendash\,\fi ##1}^^A
%    }\expandafter\thisyear\thisdate\@nil
% }
%
% \def\declmargin{\hspace*\declmarginwidth }
% \def\declmarginwidth{\dimexpr -\leftmargini +\arrayrulewidth +\tabcolsep\relax}
% \renewrobustcmd\declcs[2][]{^^A
%   \if@nobreak \par\nobreak
%   \else \par\addvspace\parskip
%         \Needspace{.08\textheight}\fi
%   \changefont{size+=2.5pt,spread=1,fam=\ttdefault}^^A
%   \newcommand*\csanchor[2][]{\raisedhyperdef[14pt]{declcs}{##2}{\cs[{##1}]{##2}}}^^A
%   \def\*{\unskip\,\texttt{*}}\noindent
%   \hskip-\leftmargini
%   \begin{tabu}{|l|}\hline
%     \expandafter\SpecialUsageIndex\csname #2\endcsname
%     \csanchor[{#1}]{#2}}
% \renewcommand\enddeclcs{%
%     \crcr \hline \end{tabu}\nobreak
%     \par  \nobreak \noindent
%     \ignorespacesafterend
%  }
% \newrobustcmd*\csref[2][]{\hyperref{}{declcs}{#2}{\cs[{#1}]{#2}}}
%
% \let\plainllap\llap
% \newrobustcmd\macro@llap[1]{{\global\let\llap\plainllap
%  \setbox0=\hbox\bgroup \raisedhyperdef{macro}{\saved@macroname}{#1}\egroup
%  \ifdim\wd0>2.4cm
%     \hbox to\z@ \bgroup\hss \hbox to2.4cm{\unhcopy0\hss}\egroup
%     \edef\@tempa{\hskip\dimexpr\the\wd0-2.4cm}\global\everypar\expandafter{\the\expandafter\everypar
%                                                                            \@tempa \global\everypar{}}^^A
%  \else \llap{\unhbox0}\fi}}
% \pretocmd\macro{\if@nobreak\else\Needspace{2\baselineskip}\fi
%           \MacrocodeTopsep\z@skip \MacroTopsep\z@skip \parsep\z@ \topsep\z@ \itemsep\z@ \partopsep\z@
%           \let\llap\macro@llap}{}{}
% \apptocmd\endmacro{\goodbreak\vskip.3\parskip}{}{}
% \newrobustcmd\macroref[2][]{\hyperref{}{macro}{#2}{\cs[{#1}]{#2}}}
%
% \DefineVerbatimEnvironment{Verb}{Verbatim}{gobble=1}
% \DefineVerbatimEnvironment{VerbLines}{Verbatim}{gobble=1,frame=lines,framesep=6pt,fontfamily=txtt,fontseries=b}
% \newkeyenvironment+[\|]{dblruled}[first=.4pt,second=.4pt,sep=1pt,left=\z@]%
%   {|\def\FrameCommand|{%
%     \vrule\@width\commandkey{first}
%     \hskip\commandkey{sep}
%     \vrule\@width\commandkey{second}
%     |\hspace|{\commandkey{left}}}
%   |\MakeFramed| {\advance\hsize-|\width \FrameRestore|}%
%  }{%
%   |\endMakeFramed|%
%  }
%
% \colorlet{HeadBand}{BlanchedAlmond}
% \newcommand*\topheader{\tikz[remember picture,overlay]{
%   \coordinate (A) at (current page.north west);
%   \coordinate (B) at ($(current page.north east)+(0cm,-\headheight)$);
%   \fill [color=HeadBand,fill opacity=.6] (A) rectangle (B);
%   }%
% }
%
% \pagesetup[plain]{
%     norules,
%     foot/right=\oldstylenums{\thepage} / \oldstylenums{\pageref{LastPage}},
%     foot/font=\lk\scriptsize,
%     foot/left/offset=3cm,
%     foot/right/offset=1cm,
%     foot/left=\vbox to\baselineskip{\vss{{\rotatebox[origin=l]{90}{\thispackage\,[rev.\thisversion]\,\copyright\,\thisyear\,\lower.4ex\hbox{\pkgcolor\NibRight}\,\FC\quad \xemail{florent.chervet at free.fr}}}}},
% }
% \pagesetup{
%     foot/right={\oldstylenums{\thepage} / \oldstylenums{\pageref{LastPage}}},
%     foot/font=\lk\scriptsize,
%     foot/left/offset=3cm,
%     foot/right/offset=1cm,
%     foot/left=\vbox to\baselineskip{\vss{{\rotatebox[origin=l]{90}{\thispackage\,[rev.\thisversion]\,\copyright\,\thisyear\,\lower.4ex\hbox{\pkgcolor\NibRight}\,\FC}}}},
%     head/font=\small\sffamily,
%     head/right/offset=1cm,
%     head/right=\noindent\raise.4ex\hbox{\thispackage : \thisinfo},
%     head/left=\topheader,
% }
% \sectionformat\section{
%     top=\medskipamount,
%     bottom=.5\smallskipamount,
%     bookmark={color=MidnightBlue},
% }
% \sectionformat\subsubsection{
%     top=\smallskipamount,bottom=0pt,
%     bookmark={color=lk},
% }
% \bookmarksetup{openlevel=3}
%
% \newcommand\macrocodecolor{\color{macrocode}}\definecolor{macrocode}{rgb}{0.05,0.02,0.07}
% \newcommand\reflinkcolor{\color{reflink}}\definecolor{reflink}{rgb}{0.49,0.00,0.00}
% \newrobustcmd*\IMPLEMENTATION{%
%     \bookmarksetup{openlevel=1}
%     ^^A\geometry{top=0pt,includeheadfoot,headheight=7mm,headsep=.6cm,bottom=.6cm,footskip=.5cm,left=4cm,right=1.5cm}
%     \newgeometry{top=0pt,includeheadfoot,headheight=7mm,headsep=.6cm,bottom=.6cm,footskip=.5cm,left=3cm,right=5mm}
%     \pagesetup*{left/offset-=1cm,right/offset-=1.5cm-2mm}
%     \section{Implementation} \label{sec:implementation}
% }
%
% \makeatother
%
% \deffootnote{1em}{0pt}{\rlap{\textsuperscript{\thefootnotemark}}\kern1em}
%
% \title{\vspace*{-28pt}\href{http://www.tex.ac.uk/tex-archive/help/Catalogue/entries/mylatexformat.html}{\HUGE\bfseries\sffamily\pkgcolor mylatexformat}\vspace*{6pt}}
% \date{}
% \subtitle{\LARGE\begin{tabu}{X[c]}
%           Use \thispackage* to make a format \\
%           based on the preamble of any LaTeX file \\[-1ex]
%           \small\FC
%           \end{tabu}
%           \vspace*{-12pt}}
% \author{\smaller\thisdate~--~\hyperref[\thisversion]{version \thisversion}}
%
% \maketitle
%
% {\makeatletter\let\@thefnmark\@empty\let\@makefntext\@firstofone
% \footnotetext{\noindent\parskip=0pt\scriptsize
% This documentation is produced with the \textt{DocStrip} utility.\par
% \begin{tabu}{X[-3]X[-1]X}
% \smex To get the package,                   &run:          &\texttt{etex \thisfile.dtx}                  \\
% \smex To get the documentation              &run (thrice): &\textt{pdflatex \thisfile.dtx}               \\
% \leavevmode\hphantom\smex To get the index, &run:          &\texttt{makeindex -s gind.ist \thisfile.idx}
% \end{tabu}�
% The \xext{dtx} file is embedded into this pdf file thank to \Xpackage{embedfile} by H. Oberdiek.}
% }
%
% \deffootnote{1em}{0pt}{\rlap{\thefootnotemark.}\kern1em}
%
% {\let\quotation\relax\let\endquotation\relax\vspace*{-1cm}%
% \begin{abstract}\parskip\smallskipamount\parindent0pt\lastlinefit0\leftskip.7cm\rightskip\leftskip
%
% \thispackage can be used to make a format from the preamble of any \LaTeX{} file.
% The use of formats helps to speed up compilations: packages which have been
% dumped in the format are loaded at very high speed. A format (also known as \emph{dump}) is therefore very suitable:
% \begin{itemize}[label=\textendash,leftmargin=3em]
% \item if you have many documents that share the same preamble,
% \item if you have a document which loads a large amount of packages (like \xpackage{TikZ} graphics libraries).
% \end{itemize}
%
% \medskip
%
% \thispackage will dump all definitions until \cs{begin}\M*{document} or \csref{endofdump}, more precisely:
%
% \csdef{tabu*}{\begingroup\tabcolsep0pt\begin{tabu}}
% \csdef{endtabu*}{\end{tabu}\endgroup}
% \definecolor{ly}{rgb}{1,1,.35}
%
% \hskip-2em
% \begin{tabu}to \linewidth{XXX[1.2]}
%
% \fcolorbox{black}{ly}{%
% \begin{tabu*}{X}
% \cs{RequirePackage}\M*{...} \\
% \cs{documentclass}\M*{...}  \\
% \cs{usepackage}\M*{...}     \\
% \noindent\,\,...       \\
% \noindent\,\,...       \\
% \csbf{begin}{\bfseries\M*{document}}
% \end{tabu*}}
% &
% \begin{tabu*}{X}
% \fcolorbox{black}{ly}{%
% \begin{tabu}{X}
% \cs{RequirePackage}\M*{...} \\
% \cs{documentclass}\M*{...}  \\
% \cs{usepackage}\M*{...}     \\
% \csbf{endofdump}       \\
% \end{tabu}} \\
% \noindent\,\, ... \\
% \noindent\,\,\cs{begin}\M*{document}
% \end{tabu*}
% &
% \begin{tabu*}{X}
% \fcolorbox{black}{ly}{%
% \begin{tabu*}{X}
% \cs{RequirePackage}\M*{...} \\
% \cs{documentclass}\M*{...}  \\
% \cs{usepackage}\M*{...}     \\
% \csbf{csname} \texttbf{endofdump}\csbf{endcsname}       \\
% \end{tabu*}} \\
% \noindent\,\, ... \\
% \noindent\,\,\cs{begin}\M*{document}
% \end{tabu*}
% \end{tabu}
% \medskip
%
% The colored parts correspond to the format. The \xext{log} file will report:
%
% \begin{tabu}{@{}X|[pkgcolor]X}
% \begin{tabu}{>{\ttfamily}X[l]}
% start reading document "my document" \\
% on input line *** (\cs{begin}\M*{document})
% \end{tabu}
% &
% \begin{tabu}{>{\ttfamily}X[l]}
% start reading document "my document" \\
% on input line *** (\cs{endofdump})
% \end{tabu}
% \end{tabu}
% \medskip
%
% \thispackage was primarily written from \CTANpackage{mylatex}
% by David Carlisle, but it is different in many points (see \ref{mylatex}). In particular, \thispackage allows to put almost
% any package in the format, with one exception: the package \CTANpackage{minitoc}
% cannot be included in the format (you must use \csref{endofdump} before \cs{usepackage}\M*{minitoc}).
%
% \end{abstract}
% }
%
% \tocsetup{
%     section/skip=4pt plus2pt minus2pt,
%     subsection/skip=4pt plus2pt minus2pt,
%     title=Contents,
%     title/skip=2pt,
%     title/after={\pkgcolor\quad\leaders\vrule height3.4pt depth-3pt\hfill\null},
%     multicols/afterend=\leavevmode{\pkgcolor\hrule},
%     twocolumns,
%     bookmark={text=Contents,bold},
% }
% \tableofcontents
% \clearpage
%
%
% \MakeShortVerb{\+}
%
% \section{Creating a format}
%
% \subsection{Using MikTeX}
%
% \subsubsection{With preloaded pdflatex format (pdf output)}
%
% Use the following command line:
%
% \quad+etex -initialize [opts] "&pdflatex" mylatexformat.ltx """abc.tex"""+
%
% You need 3 quotes around your \xext{tex} file if its name contains space
% (only one quote if it does not contain any space, but 3 quotes is more general...)
% \medskip
%
% Full example:��
% \begin{VerbLines}
%  etex -initialize -save-size=20000 -stack-size=20000
%                   -jobname="The-Name-Of-The-.fmt-File(without .fmt extension)"
%                   "&pdflatex" mylatexformat.ltx """Mon Fichier TeX.tex"""
% \end{VerbLines}
%
%
% \subsubsection{With preloaded latex format (dvi output)}
%
% Use the following command line:
%
% \quad+etex -initialize [opts] "&latex" mylatexformat.ltx """abc.tex"""+
%
%
% \subsubsection{With preloaded format xelatex (pdf output)}
%
% Use the following command line:
%
% \quad+xetex -initialize [opts] "&xelatex" mylatexformat.ltx """abc.tex"""+
%
%
% \subsection{Using other distributions}
%
% As I pointed out in the \texttt{abstract}, I have not been able to test the file on other distributions.
%
% Some users told me they typed the following command line:
%
% +latex -ini -jobname="fmt-file-name" "&latex" mylatexformat.ltx  """abc.tex"""+
%
% \section{Using a format}
%
% \subsection{Using MikTeX}
%
% Add the format to the very first line of your LaTeX file :
%
% \quad+%&"Mon Beau Format"+
%
% You may use the $\varepsilon$-\TeX{} option :    +-dont-parse-first-line+ \quad to avoid loading the format.
%
% \subsection{Using other distributions}
%
% As I pointed out in the \texttt{abstract}, I have not been able to test the file on other distributions.
%
% \clearpage
% \section[How does it work ? (and why it is different from mylatex.ltx)]{How does it work ? (and why it is different from \xpackage{mylatex.ltx})}
% \label{mylatex}
%
% \xpackage{mylatex.ltx} has three important limitations:
% \begin{enumerate}[label=\arabic*)~]
% \item The end of the format is either \verb+\begin{document}+ or the comment \verb+%mylatex+\\
%       Therefore, after the format has been loaded, \TeX{} checks the commentaries in the preamble
%       in order to skip everything before \verb+%mylatex+. \\
%       But this way, braces \textbf{must match inside commentaries !}
% \item \xpackage{mylatex.ltx} temporarily redefines the command \cs{document} in order to stop
%      the creation of the format (execution of \cs{dump}) at begin document. \\
%      However, the \textbf{initial definition of the macro \csbf{document}} is saved
%      and reloaded after the format has been loaded. \\
%      Therefore, packages that modifies this macro (such as \Xpackage[oberdiek/auxhook]{auxhook}[ H. Oberdiek] or
%      \Xpackage{etoolbox}[ P. Lehman] cannot be included in the format!
% \item If your document starts with :\\
%              \verb+\begin{document}% commentary+ or with :\\
%              \verb+\begin{document}\makeatletter+ \\
%      the format is not stopped at \verb+\begin{document}+ and an error occurs.
% \end{enumerate}
%
% \begin{dblruled}[first=2pt,sep=1pt,second=.6pt,left=4pt]
% With \xpackage{mylatexformat.ltx}, those limitations are lifted.
%
% \xpackage{mylatexformat.ltx}:
% \begin{enumerate}[label=\arabic*)~]
% \item Temporarily redefines \cs{begin} in order to check if its argument is \meta{document}. \\
%       Therefore, packages like \xpackage{auxhook} or \xpackage{etoolbox} can be included in the format.
% \item Stops the format when it encounters \verb+\begin{document}+ or \cs{endofdump}.
% \item Does not read inside commentaries
% \item After the format has been loaded, each line is checked to see if it \textbf{contains} either:
%        \begin{description}[align=left,labelsep=1cm,font=\bfseries,labelindent=1cm]
%        \item[] {\dr\verb+\begin{document}+}
%        \item[or: ] {\dr\verb+\endofdump+}
%        \item[or: ] {\dr\verb+\csname endofdump\endcsname+}\textsuperscript{\bfseries\textasteriskcentered}
%        \end{description}
%      Report is written in the \xext{log} file, with the job name, the name of the preloaded format,
%      the list of preloaded files and the first input line read after loading of format.
% \end{enumerate}
%
% {\footnotesize\textsuperscript{\bfseries\textasteriskcentered} this way, the command \verb+\csname endofdump\endcsname+ is silently ignored
% (equiv. to \cs{relax}) in case the format is not used.}
% \end{dblruled}
%
%
%
%
% \StopEventually{
% }
%
% \IMPLEMENTATION\DeleteShortVerb{\+}
%
%    \begin{macrocode}
%<*package>
%%% _____________________________________________________________________
%%%                           mylatexformat.ltx
%%%                           ^^^^^^^^^^^^^^^^^
%%% From an original idea by David Carlisle - mylatex.ltx 1994/09/27
%%% � lppl 2010-2011 - F Chervet - 2011/02/12 v3.4
%%% _____________________________________________________________________
%%%
%%% Use this file to make a format based on the preamble of any LaTeX
%%% file.
%%%
%%% The format is made from the very beginning of the file up to either:
%%%   i) the command : \endofdump
%%%                    placed in the preamble
%%%  ii) the command : \csname endofdump\endcsname
%%%                    placed in the preamble
%%% iii) \begin{document}
%%%                    if \endofdump nor \csname endofdump\endcsname
%%%                    haven't been found in the preamble.
%%%
%%% A format helps to speed up compilations, packages which have been
%%% dumped in the format are loaded at very high speed.
%%% --------------------------------------------------------------------
%%% There are no restrictions on the distribution or modification of
%%% this file, except that other people should not attempt to alter
%%% the master copy on the ctan archives.
%%%
%%% Making the format --------------------------------------------------
%%% ^^^^^^^^^^^^^^^^^                                       USING MikTeX
%%% *  With preloaded pdflatex format (pdf output) :
%%%    etex -initialize [opts] "&pdflatex" mylatexformat.ltx """abc.tex"""
%%%    => compile your document using :
%%%            pdflatex
%%%         or texify --tex-option=-output-format=pdf
%%%
%%% *  With preloaded latex format (dvi output) :
%%%    etex -initialize [opts] "&latex" mylatexformat.ltx """abc.tex"""
%%%    => compile your document using
%%%            latex
%%%         or texify --tex-option=-output-format=dvi
%%%
%%% * With preloaded format xelatex (pdf output) :
%%%    xetex -initialize [opts] "&xelatex" mylatexformat.ltx """abc.tex"""
%%%    => compile your document using xelatex
%%%
%%% NOTA BENE:    Please, notice the 3 quotes !
%%% ^^^^^^^^^^
%%%
%%% In fact, 3 quotes are necessary only if you "file name.tex" contains a space !
%%%
%%% -=-=-=-=-=-=-=-=-=-=-=-=-=-=-=-=-=-=-=-=-=-=-=-=-=-=-=-=-=-=-=-=-=-=
%%% Exemple of a command line for making format using pdfTeX :
%%%
%%%   etex -initialize -interaction=nonstopmode
%%%                    -save-size=20000 -stack-size=20000
%%%                    -jobname="Mon Beau Format Perso"
%%%                    "&pdflatex" mylatexformat.ltx "Mon Fichier TeX.tex"
%%%
%%%   "Mon Beau Format Perso.fmt" (en Fran�ais dans le texte) will be created
%%%   (or overwritten) in the current working directory.
%%%
%%% Using the format ---------------------------------------------------
%%% ^^^^^^^^^^^^^^^^^                                       USING MikTeX
%%% Add the format to the very first line of your LaTeX file :
%%% --> first line :       %&"format file name"
%%% without the .fmt extension.
%%% [You may use the option :    -dont-parse-first-line option
%%%  to avoid loading the format]
%%% --------------------------------------------------------------------
%% ____________________________________________________________________
%% The following informations come from mylatex.ltx from David Carlisle:
%% Given a LaTeX file `abc.tex', use initex as follows:
%%
%% initex &latex  mylatexformat.ltx abc      (Generic TeX)
%% initex \&latex mylatexformat.ltx abc      (unix and other TeX's)
%% tex /i &latex  mylatexformat.ltx abc      (emtex)
%%
%% If you are on a Mac or using some shell that makes it inconvenient
%% to use a command line such as the above examples then you may
%% make a file `mylatex.tex' with the single line
%% \input mylatex.ltx abc
%% and then pass the file mylatex.tex to your (ini)tex shell to produce
%% the format, ie something equivalent to initex &latex mylatex.tex.
%%
%% If you are using OzTeX, see the separate instructions below.
%%
%% This should make a format file mylatex.fmt which you can then use
%% as follows
%%
%% Using the new format
%% ^^^^^^^^^^^^^^^^^^^^
%%
%% tex  &mylatex abc          (generic TeX)
%% virtex \&mylatex abc       (Unix TeX)
%%
%% This will process your document, abc.tex, just as LaTeX does, but
%% quicker as the contents of the preamble will be stored in the
%% format file and will not need to be run each time.
%%
%% If (vir)tex fails to find your mylatex.fmt then it is not searching
%% in the current directory, either modify your TEXFORMATS path (or
%% equivalent) to search `.' or (on unix/dos) use   ./  as in:
%% virtex \&./mylatex abc
%%
%% Making and using the format with OzTeX
%% ^^^^^^^^^^^^^^^^^^^^^^^^^^^^^^^^^^^^^^
%%
%% Given a LaTeX file `abc.tex', do this: select `TeX...' from OzTeX's
%% TeX menu, and go to the folder containing the file `abc.tex' as if you
%% were selecting the file `abc.tex'. Then hit the Cancel button - this
%% procedure sets OzTeX's working folder to the one containing abc.tex.
%% Next, select iniTeX from the TeX menu, and type:
%%
%% &latex mylatex.ltx abc
%%
%% This should make a format file mylatex.fmt which you can save in the
%% same folder as the file you're working on.
%%
%% To use the new format, put this at the very start of the very first
%% line of your document:
%%
%% %&mylatex
%%
%% Further Notes
%%
%% The file abc.tex must contain a line *just* with
%% \begin{document}
%% Everything up to (but not including) the \begin{document} will
%% be saved in the format and not executed in subsequent runs.
%%
%% If you are modifying the document (or working on a similar document)
%% you may wish to add new commands to your document preamble.
%% The `mylatex' format normally skips the whole preamble (believing
%% it to be pre-loaded) and so such new commands do not take effect.
%% You could re-make the format, preloading the new preamble, but that
%% might be inconvenient to do every time, and so an alternative scheme
%% has been introduced.
%% If the preamble contains a comment   mylatex  (ie a line just
%% containing a % white space and the word mylatex) then the mylatex
%% format will start reading the preable at that point so any new
%% commands can be placed after such a comment and they will be
%% executed.
%% --------------------------------------------------------------------
%%
%%\RequirePackage{etex}% RequirePackage may be used before the class
%%
%% In principle \openout stream= filename need not be space terminated,
%% and need not be immediate, but this covers \makeindex \makeglossary
%% and index package's \newindex which are all the cases of \openout
%% that occur before \begin{document} that I could see.
%% Thanks to Ross Moore for pointing out \AtBeginDocument is too late
%% eg changebar package *closes* the stream in \AtBeginDocument, so need
%% to make sure it is opened before that. Make a special purpose hook.
\makeatletter
%    \end{macrocode}
%
%
%    \begin{macro}{\begin}
%
% \cs{begin} is overloaded in order to stop \cs{dump} at begin document
% if the macro \cs{endofdump} is not given somewhere in the preamble.
% Rem: it is NOT POSSIBLE to redefine \cs{document}, for the macro
%      \cs{document} is changed or patched by some classes or packages
%      (\xpackage{auxhook}, \xpackage{etoolbox}  etc.)
%
% With \Xpackage{etoolbox} version \textt{2.1} and earlier, \cs{begin} is patched,
% in order to insert the hooks \cs{AtBeginEnvironment} and \cs{AtEndEnvironment}.
%
% Therefore, it's not possible to restore, at end of dump, the orginal definition
% of \cs{begin} in the \LaTeX{} kernel. Instead, \thispackage inserts a reversible
% patch of the \cs{begin} command: the patched is reversed at end of dump by
% \macroref{MYLATEX@cleanup}.
%
%    \begin{macrocode}
%% \begin is overloaded in order to stop \dump at begin document
%% if the macro \endofdump is not given somewhere in the preamble.
%% Rem: it is NOT POSSIBLE to redefine \document, for the macro
%%      \document is changed or patched by some classes or packages
%%      (auxhook, etoolbox etc.)
\begingroup \def\x #1{\endgroup
   \gdef\begin ##1{\MYLATEX@StopAtdocument{##1}#1}
}\expandafter\x\expandafter{\begin{#1}}
\def\MYLATEX@StopAtdocument #1{\expandafter
   \ifx\csname #1\endcsname\document \expandafter\endofdump \fi
}% \MYLATEX@StopAtdocument
%    \end{macrocode}
%    \end{macro}
%
%    \begin{macro}{\openout}
%
%    Files cannot be opened out during the building of the format.
%    Instead, \cs{openout} is delayed until the end of the dump.
%
%    \begin{macrocode}
\let\MYLATEX@openout \openout
\def\openout#1 {\g@addto@macro\MYLATEX@opens{\immediate\openout#1 }}
\def\MYLATEX@opens {\let\MYLATEX@opens \@undefined}
%    \end{macrocode}
%    \end{macro}
%
%    \begin{macro}{\MYLATEX@cleanup}
%
%    Final clean up done at \cs{endofdump}: reverse the path for \cs{begin}
%    and restores \cs{openout} primitive.
%
%    \begin{macrocode}
\def\MYLATEX@cleanup {\let\MYLATEX@cleanup \@undefined
   \def\MYLATEX@StopAtdocument ##1\MYLATEX@StopAtdocument ##2##3\@nil{%
      \gdef\begin ####1{##1##3}%
   }\expandafter\MYLATEX@StopAtdocument \begin{##1}\@nil
   \let\openout                \MYLATEX@openout
   \let\MYLATEX@openout        \@undefined
   \let\MYLATEX@StopAtdocument \@undefined
}% \MYLATEX@cleanup
%    \end{macrocode}
%    \end{macro}
%
%
%    \begin{macro}{\MYLATEX@fontpreloading}
%
%    Some font can be preloaded: this gives an advantage to the format.
%
%    A hook for \CTANpackage{cmap} is necessary.
%
%    \begin{macrocode}
\def\MYLATEX@fontpreloading{\let\MYLATEX@fontpreloading \@undefined
   \begingroup \setbox0=\hbox {%
      $$% math (not bold, some setups don't have \boldmath)
      \def\x {\bfseries\itshape}{\itshape}\ttfamily\sffamily
      \normalfont \x
      \ifdefined\large \large \x \fi
      \ifdefined\Large \Large \x \fi
      \ifdefined\LARGE \LARGE \x \fi
      \ifdefined\Huge  \Huge  \x \fi
      \ifdefined\small \small \x \fi
      \ifdefined\footnotesize \footnotesize \x \fi
      \ifdefined\tiny  \tiny \x \fi}%
      \@for\x:={lae,lfe,lgr,ot1,ot1tt,ot6,t1,t2a,t2b,t2c,t5}\do{%
         \ifcsname cmap@set@\x\endcsname  \global\expandafter
               \let\csname cmap@set@\x\endcsname \@undefined   \fi
         \uppercase\expandafter{\expandafter\def\expandafter\x\expandafter{\x}}%
         \ifcsname cmap@set@\x\endcsname  \global\expandafter
               \let\csname cmap@set@\x\endcsname \@undefined   \fi}%
   \endgroup
}% \MYLATEX@fontpreloading
%    \end{macrocode}
%    \end{macro}
%
%
%    \begin{macro}{\endofdump}
%
%
%    \begin{macrocode}
\let\MYLATEX@listfiles \listfiles
\def\endofdump{\let\endofdump \relax
   \MYLATEX@cleanup
   \MYLATEX@fontpreloading
   \makeatother
   \everyjob\expandafter{\the\everyjob
      \let\MYLATEX@temp \listfiles
      {%
         \MYLATEX@listfiles
         \global\let\MYLATEX@listfiles \@undefined
         \expandafter\MYLATEX@banner \@dofilelist
      }%
      \let\listfiles \MYLATEX@temp     % to be able to used \listfiles once after the format
      \let\MYLATEX@temp \@undefined
      \MYLATEX@scanpreamble}%
   \dump
}% \endofdump
%    \end{macrocode}
%    \end{macro}
%
%
%
%    \begin{macro}{\MYLATEX@banner}
%    \begin{macro}{\MYLATEX@start}
%
%    Set to be expanded at \cs{everyjob}: displays a banner in the log file.
%
%    \begin{macrocode}
%% Banner for \everyjob.
{\@makeother\"  % just in case
\xdef\MYLATEX@banner #1#2#3\typeout #4{%
   \global\let\noexpand\MYLATEX@banner \noexpand\@undefined
   \newlinechar`\noexpand\^^J\message{%
   ==============================================================================^^J%
   JOB NAME\@spaces\@spaces\space: "\noexpand\jobname"^^J%
   CUSTOMISED FORMAT: "\jobname"^^J%
   PRELOADED FILES:^^J}%
   #3%
   \message{%
   ==============================================================================}%
}% \MYLATEX@banner
\long\gdef\MYLATEX@start #1#2{\global\let\MYLATEX@start \@undefined
   \message{%
   (mylatexformat)Info: start reading document "\jobname"^^J%
   (mylatexformat)\@spaces\space\on@line.
            (\ifcase#1 \string\endofdump\else\string\begin{document}\fi)^^J%
   ==============================================================================}%
   \endgroup
   \MYLATEX@opens\relax #2%
}% \MYLATEX@start
}% \catcode group

%    \end{macrocode}
%    \end{macro}
%    \end{macro}
%
%    \begin{macro}{\MYLATEX@scanpreamble}
%
%    While the preamble is being skipped (inside a group)
%    the EOL is active and defined to grab each line and
%    inspect it looking for :
%    \begin{Verbatim}
%            \endofdump
%     or     \csname endofdump\endcsname
%     or     \begin{document}
%    \end{Verbatim}
%
%    \begin{macrocode}
%% While the preamble is being skipped (inside a group)
%% the EOL is active and defined to grab each line and
%% inspect it looking for :
%%        \endofdump
%% or     \csname endofdump\endcsname
%% or     \begin{document}.
%%
%% The special catcodes required are not enabled until after the
%% first TeX command in the file, so as to avoid problems with
%% the special processing that TeX does on the first line, choosing
%% the format, or the file name etc.
{% catcode group
\catcode`\^^M=13 \catcode`\&  =8 %
\long\gdef\MYLATEX@scanpreamble {%
 \begingroup \catcode`\^^M=13%
 \long\def\MYLATEX@endofdump ##1\endofdump##2##3&##4##5##6\MYLATEX {##5}%
 \long\def\MYLATEX@csendofdump ##1\csname endofdump\endcsname##2##3&##4##5##6\MYLATEX {##5}%
 \long\def\MYLATEX@document ##1document##2##3&##4##5##6\MYLATEX {##5}%
 \long\def\MYLATEX@begindocument ##1\begin##2##3\MYLATEX##4{%
    \MYLATEX@document ##2document&&{\MYLATEX@start1{##4}}{^^M}\MYLATEX}%
 \long\def^^M##1^^M{%
    \MYLATEX@endofdump##1\endofdump&&%
       {\MYLATEX@start 0{##1}}%
       {\MYLATEX@csendofdump ##1\csname endofdump\endcsname&&%
          {\MYLATEX@start 0{##1}}%
          {\MYLATEX@begindocument ##1\begin\relax\MYLATEX{##1}}%
          \MYLATEX}%
       \MYLATEX}%
 ^^M%
}% \MYLATEX@scanpreamble
}% catcode group
%    \end{macrocode}
%    \end{macro}
%
%    \begin{macrocode}
%% Trick lookahead to allow mylatex.ltx and the document filename to be
%% given on the same command line. (initex &latex mylatex.ltx {abc.tex})
\expandafter\input\endinput%
%</package>
%    \end{macrocode}
%
% \restoregeometry\pagesetup*{left/offset+=1cm,right/offset+=1.5cm-2mm}
% \begin{History}
%   \sectionformat\subsection{font+=\smaller,bottom=\smallskipamount,top=\medskipamount}
%
%   \begin{Version}{2011/02/12 v3.4}\HistLabel{3.4}
%   \item Reorganisation of the code + optimisation.
%   \end{Version}
%
%   \begin{Version}{2011/01/19 v3.3}\HistLabel{3.3}
%   \item Modification of the code for \cs{begin} in order to keep the patching introduced by
%         \CTANpackage{etoolbox} version \textt{2.1} for the hooks \cs{AtBeginEnvironment}...\cs{AtEndEnvironement}.
%   \end{Version}
%
%   \begin{Version}{2011/01/14 v3.1}\HistLabel{3.1}
%   \item \cs{@for} loop introduced for compatibility with the \xpackage{cmap} package had the bad side effect to define a control sequence \cs{x}. Fixed.
%   \end{Version}
%
%   \begin{Version}{2011/01/12 v3.0}\HistLabel{3.0}
%   \item Fix a problem with package \CTANpackage{cmap} which could not be
%         included in the format.
%   \item Documentation revisited after Users' feedback (and the \CTANpackage{interfaces} package).
%   \end{Version}
%
%   \begin{Version}{2010/08/09 v2.9}\HistLabel{2.9}
%   \item Modification of the macros that scan the preamble when the \xext{tex} file is compiled with its format.\\
%     It is now possible to start a file with \cs{begin}\texttt{\{document\}} just after the format specification (\texttt{\&myformat}).
%   \item[To do:] Command line to make the format on Linux (\texttt{-ini} instead of \texttt{-initialize} ???).
%   \end{Version}
%
%   \begin{Version}{2010/06/30 v2.1}\HistLabel{2.1}
%   \item Added: \cs{listfiles} is now available in the preamble, after \cs{endofdump}, when the document is compiled with its format.
%   \item Documentation update.
%   \end{Version}
%
%   \begin{Version}{2010/05/20 v1.2}\HistLabel{1.2}
%   \item Fixed a bug in \cs{MYLATEXopens} \\
%         (There was problems when including an index in the format for example...)
%   \end{Version}
%
%   \begin{Version}{2010/04/30 v1.0}\HistLabel{1.0}
%   \item First version.
%   \end{Version}
%
% \end{History}
%
% ^^A\restoregeometry\pagesetup*{left/offset+=1cm,right/offset+=1.5cm-2mm}
% \clearpage
% \PrintIndex
%
% \Finale