\documentclass{article}
\usepackage[
    web={pro,usetemplates},
    attachsource={tex},
    aebxmp
]{aeb_pro}
\usepackage[%
    donotmail,
    envelope=aebMyEnv.pdf,
%
% The path that follows needs to be edited to point to
% the location of the aebMyEnv.pdf on your system.
%
    path2folder=/c/acrotex/acrotex/aebpro/aebenvelope/custom,
]{aeb_envelope}

\setEnvDimensions{7.5in}{3.875in}

\begin{comment}
    These comments continue the description of how to create and use a custom
    eEnvelope. See the source file aebMyEnv.tex for the instructions on how
    the eEnvelope aebMyEnv.pdf was created. Now, we learn how to use the
    eEnvelope.
        1.  Place the file aebMyEnv.pdf anywhere on your hard drive and make
            note of its location. It is best to place it in the envelopes folder
            of the AeB eEnvelopes distribution, but there are some out there who
            want to use their own personal folder. You'll not above that we specify
            the envelope and the path2folder options. Since we are not using one
            of the standard eEnvelopes, we specify the file name of the eEnvelope,
            aebMyEnv.pdf, in this case. We also specify the path to the folder as
            /c/acrotex/acrotex/aebpro/acroenvelope.
                \usepackage[%
                    donotmail,
                    envelope=aebMyEnv.pdf,
                    path2folder=/c/acrotex/acrotex/aebpro/acroenvelope,
                ]{aeb_envelope}
            If you had put this file in the envelopes folder, then it would
            not be necessary to specify the path2folder option.
        2.  Since you created the eEnvelope, you know its dimensions. Specify these
            dimensions using the \setEnvDimensions command of aeb_envelope.
                \setEnvDimensions{7.5in}{3.875in}
            See the file aebMyEnv.tex for these dimensions. We need the papersize
            to match exactly the eEnvelope dimensions.
        3.  Done past the \DeclareDocInfo (optional) and the \mailTo (required) you'll
            see
                \setAddressEnv
                {%
                    \put(50,250){%
                    \begin{minipage}[t]{2in}\parindent0pt\raggedright\sffamily\bfseries
                        \displayAddr{From}\mailtoFrom
                    \end{minipage}}%
                    \put(200,130){%
                    \begin{minipage}[t]{2in}\parindent0pt\raggedright\sffamily\bfseries
                        \displayAddr{To}\mailtoName\\[1ex]
                        \displayAddr{Message}\mailtoMessageEnvelope
                    \end{minipage}}%
                }
            \setAddressEnv is a command whose argument includes some picture commands, \put
            in this case. This \put commands place the various elements entered through
            the \mailTo command on the envelope and in the mail dialog box.

            Where did the numbers in the \put commands come from? Well, that's where
            \template{aebMyEnv} comes in. Down below, you'll see that this command is
            commented out. I had this command uncommented, I latexed this file, and brought
            the dvi file in my dvi previewer. Using my previewer's measuring device, I can
            get rough numbers for the location of the address elements. That is what I did.
            When I was happy with the placement, I commented it command out again.
        4.  I believe that's about it. This is such a nice design, I should make it a
            standard eEnvelope. What do you think?
        5.  If you create an attractive eEnvelope, send it to me and, if it is worthy,
            I'll incorporate it into this distribution.
        6.  dpstory@acrotex.net
\end{comment}

\DeclareDocInfo
{
    university={\AcroTeX.Net},
    title={The AeB Pro eEnvelope System (APES)},
    author={D. P. Story, J. Gilg, S. Singer},
    email={dpstory@acrotex.net},
    subject={Demo APES: Using a custom eEnvelope},
    talksite={\url{www.acrotex.net}},
    version={1.0},
    keywords={Adobe Acrobat, JavaScript, eEnvelope, AcroTeX},
    copyrightStatus=True,
    copyrightNotice={Copyright (C) \the\year, D. P. Story},
    copyrightInfoURL={http://www.acrotex.net}
}

\mailTo
{
    UI=true,
    ToName=J\"{u}rgen Gilg\\Member of the ASDT,
    From={D. P. Story\\\href{http://www.acrotex.net}{AcroTeX.Net}\\\url{www.acrotex.net}},
    To=gilg@acrotex.net,
    CC=dpstory@acrotex.net,
%    BCC=gilg@acrotex.net,
    Subject=Testing the AeB eEnvelope Delivery System,
    MessageEnvelope={This is a custom envelope and is integrated into the AeB eEnvelope System.},
    MessageBody={%
        This document and its attachments test the AeB Envelope
        system for a custom design.  Seems to work. Will write
        down the steps used to create the new templates.\n\n
        Open the PDF attachment, the documents you ordered are attached to it.\n\n
        dps, dpstory@acrotex.net
    },
}

\setAddressEnv
{%
    \put(50,250){%
    \begin{minipage}[t]{2in}\parindent0pt\raggedright\sffamily\bfseries
         \displayAddr{From}\mailtoFrom
    \end{minipage}}%
    \put(200,130){%
    \begin{minipage}[t]{2in}\parindent0pt\raggedright\sffamily\bfseries
         \displayAddr{To}\mailtoName\\[1ex]
         \displayAddr{Message}\mailtoMessageEnvelope
    \end{minipage}}%
}

%\template{aebMyEnv}

\assembleEnvelope
