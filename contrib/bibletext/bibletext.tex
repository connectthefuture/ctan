\documentclass[a4paper]{ltxguide}

\usepackage[hidelinks]{hyperref}

\title{Inserting Bible texts with the \textsf{bible} package}
\author{Camil Staps\footnote{info@camilstaps.nl}}
\date{Version 0.1\\2016-02-18}

\usepackage{bibletext}

\usepackage{framed}
\usepackage{minted}
\newenvironment{example}
  {\VerbatimOut{\jobname.tmp}}
  {\endVerbatimOut
    \begin{framed}
      \setlength{\parskip}{-10pt}
      \inputminted[xleftmargin=-8pt]{latex}{\jobname.tmp}
      \medskip
      \input{\jobname.tmp}
    \end{framed}}

\begin{document}

\maketitle

\section{Introduction}
This package allows you to add Bible texts by specifying references, rather than copying the text into the document.

Under the hood, this uses the \texttt{bible} command, for which you need to install \texttt{bible-kjv}. This means that by default you will have the KJV translation. Because of the need for this package and other implementation details, this package won't work on Windows.

\section{Examples}
\begin{example}
  \bibletext{Jn 1:1-2}
\end{example}

\begin{example}
  \bibletext[ref=true]{Jn 1:1-2}
\end{example}

\begin{example}
  \bibletext[ref=true,omitrepeatedchapter=true]{Jn 1:1-2}
\end{example}

\begin{example}
  \bibletext[ref=true,omitchapter=true]{Jn 1:1-2}
\end{example}

\section{Future work}
\begin{itemize}
  \item When \verb$omitrepeatedchapter=true$, the chapter should again be outputted when a new chapter is begun.
  \item The verse indications should be parsed and a \verb$\versefont$ command should be defined.
  \item Implement the \verb$translation$ option to allow the writer to select a specific translation.
\end{itemize}

\section{Incompatibilities}
The \textsf{bible} package needs to pipe shell commands, for which it needs to \verb$\DeleteShortVerb{\|}$. This means that you cannot use \verb$|foo|$ as shorthand for \verb$\verb|foo|$\footnote{Credits go to egreg on TeX.SE, \url{http://tex.stackexchange.com/q/294732/23992}.}.

\end{document}

