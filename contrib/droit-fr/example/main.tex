\documentclass[a4paper,12pt,french,twoside,footnotereset=true,versetitle=true]{droit-fr}

\usepackage{ifluatex}
\ifluatex % compilation via Lua(La)TeX
	\usepackage{fontspec}
	\setmainfont{FreeSerif} % police proche de Times New Roman. A modifier le cas échéant.
	\setsansfont{FreeSans}
	\setmonofont{FreeMono}
\else % compilation via pdf(La)TeX
	\usepackage[utf8]{inputenc}
	\usepackage{times} % police proche de Times New Roman. A modifier le cas échéant.
	\usepackage[T1]{fontenc}
\fi
\usepackage{microtype} % amélioration du gris typographique
\usepackage{hyperref} % hyperliens PDF
\usepackage{bookmark} % signets PDF
\usepackage[nonumberlist,toc,noredefwarn]{glossaries} % glossaire
\usepackage{glossary-mcols} % glossaire sur deux colonnes
\usepackage[style=droit-fr,backend=biber,indexing=cite]{biblatex} % paramètres de bibliographie
\usepackage{lipsum} % génération de texte automatique

% paramètres des hyperliens PDF
\hypersetup{%
  pdftitle={Mon titre de thèse},
  pdfauthor={Prénom Nom}
}

% paramètres des signets PDF
\bookmarksetup{numbered=true,depth=4}

% création du glossaire et des indexs
\makeglossaries % glossaire, fichier généré: .gls
\makeindexv % index de base par versets, fichié généré: index.idx
\makeindexa % index d'auteurs par versets, fichié généré: auteurs.idx

% index des auteurs séparé de l'index de base
\DeclareIndexNameFormat{default}{%
  \usebibmacro{index:name}{\indexa}{#1}{#3}{#5}{#7}}

% chargement des fichiers de sources bibliographiques
\addbibresource{journaux.bib}
\addbibresource{bibliographie.bib}

% marges
\settrimmedsize{297mm}{210mm}{*}
\setlength{\trimtop}{0pt}
\setlength{\trimedge}{\stockwidth}
\addtolength{\trimedge}{-\paperwidth}
\settypeblocksize{634pt}{448.13pt}{*}
\setulmargins{4cm}{*}{*}
\setlrmargins{*}{*}{1.5}
\setmarginnotes{17pt}{51pt}{\onelineskip}
\setheadfoot{\onelineskip}{2\onelineskip}
\setheaderspaces{*}{2\onelineskip}{*}
\checkandfixthelayout

\listfiles

\renewcommand*{\glsgroupskip}{}


\begin{document}

\frontmatter % pages en chiffres romains, sections non numérotées
\pagestyle{plain} % en-tetes vides
\author{Prénom}{Nom}
\title{Mon titre de thèse}

\university{Nom de l'université}
\school{Droit international, droit européen, relations internationales et droit comparé}
\speciality{Droit privé}
\approvaldate{2 Janvier 2012}

\director{Monsieur}{Amstram}{Gram}{Professeur à l'Université Rébenthine}
\reportera{Monsieur}{Bidule}{Trucmuche}{Professeur à l'Université Lévision}
\reporterb{Monsieur}{Blabla}{Chose}{Professeur à l'Université Tanisé}
\membera{Madame}{Machin}{Chouette}{Professeur à l'Université Lorisation}
\memberb{Monsieur}{Pif}{Pouf}{Professeur à l'Université Alamenthe}

\maketitlepage

 % page de titre
\section{Avertissement}

La Faculté n’entend donner aucune approbation ni improbation aux opinions émises dans cette thèse; ces opinions doivent être considérées comme propres à leur auteur.

\cleardoublepage

\section{Remerciements}


\cleardoublepage

 % avertissement, remerciements, résumé
% sommaire
\renewcommand*{\contentsname}{Sommaire}
\let\oldchangetocdepth\changetocdepth
\renewcommand*{\changetocdepth}[1]{}
\let\oldcftchapterfillnum\cftchapterfillnum
\setcounter{tocdepth}{0}% Parties / Titres / Chapitres seulement

%\cftpagenumbersoff{book}
%\cftpagenumbersoff{part}
%\cftpagenumbersoff{chapter}

\renewcommand{\tocheadstart}{}

\renewcommand{\cftbeforebookskip}{1em}
%\renewcommand{\cftbookfont}{}
\renewcommand{\cftbookindent}{0em}
%\renewcommand{\cftbooknumwidth}{}
\renewcommand{\cftbookpagefont}{\normalfont\bfseries\large}
%\renewcommand{\cftafterbookskip}{}
%\renewcommand{\cftbookleader}{\cftdotfill{\cftdotsep}}

\renewcommand{\cftbeforepartskip}{0.5em}
\renewcommand{\cftpartfont}{\normalfont}
\renewcommand{\cftpartindent}{0.5em}
%\renewcommand{\cftpartnumwidth}{}
\renewcommand{\cftpartpagefont}{\normalfont\scshape}
%\renewcommand{\cftpartleader}{\cftdotfill{\cftdotsep}}

\renewcommand{\cftbeforechapterskip}{0em}
\renewcommand{\cftchapterfont}{\normalfont}
\renewcommand{\cftchaptername}{Chapitre\space}
\renewcommand{\cftchapterindent}{1em}
%\renewcommand{\cftchapternumwidth}{}
\renewcommand{\cftchapterpagefont}{\normalfont}
%\renewcommand{\cftchapterleader}{\cftdotfill{\cftdotsep}}

\clearpage
\tableofcontents

\newacronym{ibid}{ibid.}{ibidem}
\newacronym{dalloz}{D.}{Recueil Dalloz}
\newacronym{cedh}{CEDH}{Cour Européenne des Droits de l'Homme}
\newacronym{rtdciv}{RTD civ.}{Revue Trimestrielle de droit civil}
\newacronym{jcpg}{JCP G.}{Jurisclasseur Périodique, édition Générale}
\newacronym{defrenois}{Defrénois}{Répertoire Defrénois}
\newacronym{alinea}{al.}{alinéa}
\newacronym{assplen}{Ass. plén.}{Assemblée plénière}
\newacronym{bullciv}{Bull. civ.}{Bulletin civil des arrêts de la Cour de cassation}
\newacronym{appel}{CA}{Cour d'appel}
\newacronym{cassass}{Cass.}{Cour de cassation}
\newacronym{cciv}{C. civ.}{Code civil}
\newacronym{chreunies}{Ch. réun.}{Chambres réunies}
\newacronym{dretpatr}{Dr. et patr.}{Droit et Patrimoine}
\newacronym{fasc}{Fasc.}{Fascicule}
\newacronym{gazpal}{Gaz. Pal.}{Gazette du Palais}
\newacronym{jcpn}{JCP N.}{Jurisclasseur Périodique, édition Notariale}
\newacronym{jo}{JO}{Journal Officiel}
\newacronym{prec}{préc.}{précité}
\newacronym{puf}{PUF}{Presse Universitaire de France}
\newacronym{ridc}{RIDC}{Revue internationale de droit comparé}
\newacronym{rjpf}{RJPF}{Revue juridique personnes et famille}
\newacronym{supra}{supra}{ci-dessus}
\newacronym{sq}{sq.}{et suivants}
\newacronym{t}{t.}{tome}
\newacronym{infra}{infra}{ci-dessous}
\newacronym{IR}{IR}{Informations Rapides du recueil Dalloz}
\newacronym{ajfam}{AJ Famille}{Actualité Juridique Famille}
\newacronym{an}{AN}{Assemblée Nationale}
\newacronym{bullcrim}{Bull. crim.}{Bulletin des arrêts de la Cour de cassation (chambre criminelle)}
\newacronym{cah}{Cah.}{Cahiers}
\newacronym{ce}{CE}{Conseil d'État}
\newacronym{chap}{chap.}{chapitre}
\newacronym{v}{v.}{voir}
\newacronym{circmin}{Circ. Min.}{Circulaire Ministérielle}
\newacronym{constit}{Cons. Const.}{Conseil Constitutionnel}
\newacronym{etal}{et al.}{et autres}
\newacronym{gajciv}{GAJ civ.}{Grands Arrêts Jurisprudence civile}
\newacronym{jaf}{JAF}{Juge aux affaires familiales}
\newacronym{loi}{L.}{Loi}
\newacronym{lpa}{LPA}{Les Petites Affiches}
\newacronym{cpc}{CPC}{Code de procédure civile}
\newacronym{obs}{obs.}{observations}
\newacronym{comm}{comm.}{commentaire}
\newacronym{opcit}{op. cit.}{opere citato}
\newacronym{prop}{prop.}{proposition}
\newacronym{rapp}{rapp.}{rapporteur}
\newacronym{somm}{somm.}{sommaire}
\newacronym{vol}{vol.}{volume}
\newacronym{concl}{concl.}{conclusion}

\glsaddall
\printglossary[style=mcolindex,title=Glossaire]



\mainmatter % pages en chiffres arabes, sections numérotées
\pagestyle{corpus} % en-tetes/pied-de-pages en style "corpus"
% ======================================================================
% introduction.tex
% Copyright (c) Markus Kohm, 2001-2017
%
% This file is part of the LaTeX2e KOMA-Script bundle.
%
% This work may be distributed and/or modified under the conditions of
% the LaTeX Project Public License, version 1.3c of the license.
% The latest version of this license is in
%   http://www.latex-project.org/lppl.txt
% and version 1.3c or later is part of all distributions of LaTeX 
% version 2005/12/01 or later and of this work.
%
% This work has the LPPL maintenance status "author-maintained".
%
% The Current Maintainer and author of this work is Markus Kohm.
%
% This work consists of all files listed in manifest.txt.
% ----------------------------------------------------------------------
% introduction.tex
% Copyright (c) Markus Kohm, 2001-2017
%
% Dieses Werk darf nach den Bedingungen der LaTeX Project Public Lizenz,
% Version 1.3c, verteilt und/oder veraendert werden.
% Die neuste Version dieser Lizenz ist
%   http://www.latex-project.org/lppl.txt
% und Version 1.3c ist Teil aller Verteilungen von LaTeX
% Version 2005/12/01 oder spaeter und dieses Werks.
%
% Dieses Werk hat den LPPL-Verwaltungs-Status "author-maintained"
% (allein durch den Autor verwaltet).
%
% Der Aktuelle Verwalter und Autor dieses Werkes ist Markus Kohm.
% 
% Dieses Werk besteht aus den in manifest.txt aufgefuehrten Dateien.
% ======================================================================
%
% Introduction of the KOMA-Script guide
% Maintained by Markus Kohm
%
% ----------------------------------------------------------------------
%
% Einleitung der KOMA-Script-Anleitung
% Verwaltet von Markus Kohm
%
% ======================================================================

\KOMAProvidesFile{introduction.tex}
                 [$Date: 2017-01-02 13:30:07 +0100 (Mon, 02 Jan 2017) $
                  KOMA-Script guide introduction]

\chapter{Einleitung}
\labelbase{introduction}

Dieses Kapitel enth�lt \iffree{unter anderem }{}wichtige Informationen �ber
den Aufbau \iffree{der Anleitung}{des Buches} und die Geschichte von
\KOMAScript, die Jahre vor der ersten Version beginnt. Dar�ber hinaus finden
Sie Informationen f�r den Fall, dass Sie %
\iffalse % Umbruchkorreturauslassung
 \KOMAScript{} noch nicht installiert haben, oder %
\fi%
auf Fehler sto�en.

\iffree{%
\section{Vorwort}
\seclabel{preface}

{\KOMAScript} ist ein sehr komplexes Paket (engl. \emph{bundle}). Dies
ist schon allein darin begr�ndet, dass es nicht nur aus einer einzigen
Klasse (engl. \emph{class}) oder einem einzigen Paket (engl.
\emph{package}), sondern einer Vielzahl derer besteht. Zwar sind die
Klassen als Gegenst�cke zu den Standardklassen konzipiert (siehe
\autoref{cha:maincls}), das hei�t jedoch insbesondere nicht,
dass sie nur �ber die Befehle, Umgebungen und Einstellm�glichkeiten
der Standardklassen verf�gen oder deren Aussehen als
Standardeinstellung �bernehmen.
Die F�higkeiten von {\KOMAScript} reichen
teilweise weit �ber die F�higkeiten der Standardklassen hinaus.
Manche davon sind auch als Erg�nzung zu den Grundf�higkeiten des
\LaTeX-Kerns zu betrachten.

Allein aus dem Vorgenannten ergibt sich schon zwangsl�ufig, dass die
Dokumentation zu {\KOMAScript} sehr umfangreich ausf�llt. Hinzu kommt,
dass {\KOMAScript} in der Regel nicht gelehrt wird. Das hei�t, es gibt
keinen Lehrer, der seine Sch�ler kennt und damit den Unterricht und
das Unterrichtsmaterial entsprechend w�hlen und anpassen kann. Es w�re
ein Leichtes, die Dokumentation f�r irgendeine Zielgruppe zu
verfassen.  Die Schwierigkeit, der sich der Autor gegen�ber sieht,
besteht jedoch darin, dass eine Anleitung f�r alle m�glichen
Zielgruppen ben�tigt wird. Ich habe mich bem�ht, eine Anleitung zu
erstellen, die f�r den Informatiker gleicherma�en geeignet ist wie f�r
die Sekret�rin des Fischh�ndlers. Ich habe mich bem�ht, obwohl es sich
dabei eigentlich um ein unm�gliches Unterfangen handelt. Ergebnis sind
zahlreiche Kompromisse. Ich bitte jedoch, die Problematik bei
eventuellen Beschwerden zu ber�cksichtigen und bei der Verbesserung
der derzeitigen L�sung zu helfen.

Trotz des Umfangs der Anleitung bitte ich au�erdem darum, im Falle von
Problemen zun�chst die Dokumentation zu konsultieren. Als erste Anlaufstelle
sei auf den mehrteiligen Index am Ende des \iffree{Dokuments}{Buches}
hingewiesen. \iffree{Zur Dokumentation geh�ren neben dieser Anleitung auch
  alle Text-Dokumente, die Bestandteil des Pakets sind. Sie sind in
  \File{manifest.txt} vollst�ndig aufgef�hrt}{}
}{}

\section{Dokumentaufbau}
\seclabel{structure}

Diese Anleitung ist in mehrere Teile untergliedert. Es gibt einen Teil f�r
Anwender, einen f�r fortgeschrittene Anwender und Experten und einen Anhang
mit weiterf�hrenden Informationen und Beispielen f�r diejenigen, die es ganz
genau wissen wollen.

\autoref{part:forAuthors} richtet sich dabei an alle \KOMAScript-Anwender. Das
bedeutet, dass hier auch einige Informationen f�r \LaTeX-Neulinge zu finden
sind. Insbesondere ist dieser Teil mit vielen Beispielen angereichert, die dem
reinen Anwender zur Verdeutlichung der Erkl�rungen dienen sollen. Scheuen Sie
sich nicht, diese Beispiele selbst auszuprobieren und durch Abwandlung
herauszufinden, wie \KOMAScript{} funktioniert. Trotz allem ist diese
Anleitung jedoch keine Einf�hrung in \LaTeX. \LaTeX-Neulingen seien daher
Dokumente wie \cite{l2kurz} nahe gelegt. Wiedereinsteiger aus der Zeit von
\LaTeX~2.09 sei zumindest \cite{latex:usrguide} empfohlen. Auch das Studium des
einen oder anderen Buches zu \LaTeX{} wird empfohlen.  Literaturempfehlungen
finden sich beispielsweise in \cite{DANTE:FAQ}. Der Umfang von
\cite{DANTE:FAQ} ist ebenfalls erheblich. Dennoch wird darum gebeten, das
Dokument nicht nur irgendwo vorliegen zu haben, sondern es mindestens einmal
zu lesen und bei Problemen zu konsultieren.

\autoref{part:forExperts} richtet sich an fortgeschrittene
\KOMAScript-Anwender. Das sind all diejenigen, die sich bereits mit \LaTeX{}
auskennen oder schon einige Zeit mit \KOMAScript{} gearbeitet haben und jetzt
etwas besser verstehen wollen, wie \KOMAScript{} funktioniert, wie es mit
anderen Paketen interagiert und wie man speziellere Aufgaben mit \KOMAScript{}
l�sen kann. Dazu werden die Klassenbeschreibungen aus
\autoref{part:forAuthors} in einigen Aspekten nochmals aufgegriffen und n�her
erl�utert. Dazu kommt die Dokumentation von Anweisungen, die speziell f�r
fortgeschrittene Anwender und Experten vorgesehen sind. Erg�nzt wird dies
durch die Dokumentation von Paketen, die f�r den Anwender normalerweise insofern
verborgen sind, als sie unter der Oberfl�che der Klassen und Anwenderpakete
ihre Arbeit verrichten. Diese Pakete sind ausdr�cklich auch f�r die Verwendung
durch andere Klassen- und Paketautoren vorgesehen.

Der Anhang\iffree{, der nur in der Buchfassung zu finden ist,}{} richtet sich
an diejenigen, denen all diese Informationen nicht gen�gen. Es gibt dort zum
einen Hintergrundwissen zu Fragen der Typografie, mit denen dem
fortgeschrittenen Anwender eine Grundlage f�r fundierte eigene Entscheidungen
vermittelt werden soll. Dar�ber hinaus sind dort Beispiele f�r angehende
Paketautoren zu finden. Diese Beispiele sind weniger dazu gedacht, einfach
�bernommen zu werden. Vielmehr sollen sie Wissen um Planung und Durchf�hrung
von \LaTeX-Projekten sowie einiger grundlegender \LaTeX-Anweisungen f�r
Paketautoren vermitteln.

Die Kapitel-Einteilung der Anleitung soll ebenfalls dabei
helfen, nur die Teile lesen zu m�ssen, die tats�chlich von Interesse sind. Um
dies zu erreichen, sind die Informationen zu den einzelnen Klassen und Paketen
nicht �ber das gesamte Dokument verteilt, sondern jeweils in einem Kapitel
konzentriert. Querverweise in ein anderes Kapitel sind damit in der Regel auch
Verweise auf einen anderen Teil des Gesamtpakets. Da die drei Hauptklassen in
weiten Teilen �bereinstimmen, sind sie in einem gemeinsamen Kapitel
zusammengefasst. Die Unterschiede werden deutlich hervorgehoben, soweit
sinnvoll auch durch eine entsprechende Randnotiz. Dies geschieht
beispielsweise wie hier, wenn etwas nur die Klasse
\Class{scrartcl}\OnlyAt{\Class{scrartcl}} betrifft. Nachteil dieses Vorgehens
ist, dass diejenigen, die \KOMAScript{} insgesamt kennenlernen wollen, in
einigen Kapiteln auf bereits Bekanntes sto�en werden. Vielleicht nutzen Sie
die Gelegenheit, um Ihr Wissen zu vertiefen.

Unterschiedliche Schriftarten werden auch zur Hervorhebung unterschiedlicher
Dinge verwendet. So werden die Namen von \Package{Paketen} und \Class{Klassen}
anders angezeigt als \File{Dateinamen}. \Option{Optionen}, \Macro{Anweisungen},
\Environment{Umgebungen}, \Variable{Variablen} und
\PLength{Pseudol�ngen} werden einheitlich hervorgehoben. Der
\PName{Platzhalter} f�r einen Parameter wird jedoch anders dargestellt als ein
konkreter \PValue{Wert} eines Parameters. So zeigt etwa
\Macro{begin}\Parameter{Umgebung}, wie eine Umgebung ganz allgemein
eingeleitet wird, wohingegen \Macro{begin}\PParameter{document} angibt, wie die
konkrete Umgebung \Environment{document} beginnt. Dabei ist dann
\PValue{document} ein konkreter Wert f�r den Parameter \PName{Umgebung}
der Anweisung \Macro{begin}.

\iffalse% Umbruchkorrekturtext
Damit sollten Sie nun alles wissen, um diese Anleitung lesen und verstehen zu
k�nnen. Trotzdem k�nnte es sich lohnen, den Rest dieses Kapitels gelegentlich
auch zu lesen.
\fi


\section{Die Geschichte von \KOMAScript}
\seclabel{history}

%\begin{Explain}
  Anfang der 90er~Jahre wurde Frank Neukam damit beauf"|tragt, ein
  Vorlesungsskript zu setzen. Damals war noch \LaTeX~2.09 aktuell und
  es gab keine Unterscheidung nach Klassen und Paketen, sondern alles
  waren Stile (engl. \emph{styles}). Die Standarddokumentstile
  erschienen ihm f�r ein Vorlesungsskript nicht optimal und boten auch
  nicht alle Befehle und Umgebungen, die er ben�tigte.
  
  Zur selben Zeit besch�ftigte sich Frank auch mit Fragen der
  Typografie, insbesondere mit \cite{JTsch87}. Damit stand f�r ihn
  fest, nicht nur irgendeinen Dokumentstil f�r Skripten zu erstellen,
  sondern allgemein eine Stilfamilie, die den Regeln der europ�ischen
  Typografie folgt. {\Script} war geboren.
  
  Der \KOMAScript-Autor traf auf {\Script} ungef�hr zum Jahreswechsel
  1992/""1993. Im Gegensatz zu Frank Neukam hatte er h�ufig mit Dokumenten im
  A5-Format zu tun. Zu jenem Zeitpunkt wurde A5 weder von den Standardstilen
  noch von {\Script} unterst�tzt. Daher dauerte es nicht lange, bis er erste
  Ver�nderungen an {\Script} vornahm. Diese fanden sich auch in {\ScriptII}
  wieder, das im Dezember~1993 von Frank ver�ffentlicht wurde.
  
  Mitte 1994 erschien dann \LaTeXe. Die damit einhergehenden �nderungen waren
  tiefgreifend. Daher blieb dem Anwender von {\ScriptII} nur die Entscheidung,
  sich entweder auf den Kompatibilit�tsmodus von \LaTeX{} zu beschr�nken, oder
  auf {\Script} zu verzichten. Wie viele andere wollte ich beides nicht. Also
  machte der \KOMAScript-Autor sich daran, einen \Script-Nachfolger f�r
  {\LaTeXe} zu entwickeln, der am 7.~Juli~1994 unter dem Namen {\KOMAScript}
  erschienen ist. Ich will hier nicht n�her auf die Wirren eingehen, die es um
  die offizielle Nachfolge von {\Script} gab und warum dieser neue Name
  gew�hlt wurde. Tatsache ist, dass auch aus Franks Sicht {\KOMAScript} der
  Nachfolger von {\ScriptII} ist. Zu erw�hnen ist noch, dass {\KOMAScript}
  urspr�nglich ohne Briefklasse erschienen war. Diese wurde im Dezember~1994
  von Axel Kielhorn beigesteuert. Noch etwas sp�ter erstellte Axel Sommerfeldt
  den ersten richtigen scrguide zu {\KOMAScript}.
  
  Seither ist einige Zeit vergangen. {\LaTeX} hat sich kaum ver�ndert, die
  \LaTeX-Landschaft erheblich. {\KOMAScript} wurde weiterentwickelt. Es findet
  nicht mehr allein im deutschsprachigen Raum Anwender, sondern in ganz
  Europa, Nordamerika, Australien und Asien.  Diese Anwender suchen bei
  {\KOMAScript} nicht allein nach einem typografisch ansprechenden
  Ergebnis. Zu beobachten ist vielmehr, dass bei {\KOMAScript} ein neuer
  Schwerpunkt entstanden ist: Flexibilisierung durch Variabilisierung. Unter
  diesem Schlagwort verstehe ich die M�glichkeit, in das Erscheinungsbild an
  vielen Stellen eingreifen zu k�nnen. Dies f�hrte zu vielen neuen Makros, die
  mehr schlecht als recht in die existierende Dokumentation integriert wurden.
  Irgendwann wurde es damit auch Zeit f�r eine komplett �berarbeitete
  Anleitung.
%\end{Explain}


\iffree{%
\section{Danksagung}
\seclabel{thanks}

Mein pers�nlicher Dank gilt Frank Neukam, ohne dessen \Script-Familie es
vermutlich {\KOMAScript} nie gegeben h�tte. Mein Dank gilt denjenigen, die an
der Entstehung von {\KOMAScript} und den Anleitungen mitgewirkt
haben. Dieses Mal danke ich Elke, Jana, Ben und Edoardo stellvertretend f�r
Beta-Test und Kritik. Ich hoffe, ihr macht damit noch weiter.

Ganz besonderen Dank bin ich den Gr�ndern und den Mitgliedern von DANTE,
Deutschsprachige Anwendervereinigung \TeX~e.V\kern-.18em., schuldig, durch die
letztlich die Verbreitung von \TeX{} und \LaTeX{} und allen Paketen
einschlie�lich {\KOMAScript} an einer zentralen Stelle �berhaupt erm�glicht
wird. In gleicher Weise bedanke ich mich bei den aktiven Helfern auf der
Mailingliste \texttt{\TeX-D-L} (siehe \cite{DANTE:FAQ})m in der Usenet-Gruppe
\texttt{de.comp.text.tex} und den vielen \LaTeX-Foren im Internet, die mir so
manche Antwort auf Fragen zu {\KOMAScript} abnehmen.

Mein Dank gilt aber auch allen, die mich immer wieder aufgemuntert haben,
weiter zu machen und dieses oder jenes noch besser, weniger fehlerhaft oder
schlicht zus�tzlich zu implementieren. Ganz besonders bedanke ich mich noch
einmal bei dem �u�erst gro�z�gigen Spender, der mich mit dem bisher und
vermutlich f�r alle Zeiten gr��ten Einzelbetrag f�r die bisher geleistete
Arbeit an \KOMAScript{} bedacht hat.


\section{Rechtliches}
\seclabel{legal}

{\KOMAScript} steht unter der {\LaTeX} Project Public Licence. Eine nicht
offizielle deutsche �bersetzung ist Bestandteil des \KOMAScript-Pakets. In
allen Zweifelsf�llen gilt im deutschsprachigen Raum der Text
\File{lppl-de.txt}, w�hrend in allen anderen L�ndern der Text \File{lppl.txt}
anzuwenden ist.

\iffree{F�r die Korrektheit der Anleitung, Teile der Anleitung oder einer
  anderen in diesem Paket enthaltenen Dokumentation wird keine Gew�hr
  �bernommen.}%
{Diese gedruckte Ausgabe der Anleitung ist davon und von den in den Dateien
  \File{lppl.txt} und \File{lppl-de.txt} des \KOMAScript-Pakets
  festgeschriebenden rechtlichen Bedingungen ausdr�cklich ausgenommen.}
}{}

\section{Installation}
\seclabel{installation}

Die drei wichtigsten \TeX-Distributionen, Mac\TeX, MiK\TeX{} und \TeX~Live,
stellen \KOMAScript{} �ber Ihre jeweiligen Paketverwaltungen bereit. Es wird
empfohlen, die Installation und Aktualisierung von \KOMAScript{} dar�ber
vorzunehmen. Die manuelle Installation von {\KOMAScript} ohne Verwendung der
jeweiligen Paketverwaltung wird in der Datei \File{INSTALLD.txt}, die
Bestandteil jeder \KOMAScript-Verteilung ist, beschrieben.  Beachten Sie dazu
auch die jeweilige Dokumentation zur installierten \TeX-Distribution.

Daneben gibt es auf \cite{homepage} seit einiger Zeit Installationspakete von
Zwischenversionen von \KOMAScript{} f�r die wichtigsten Distributionen. F�r
deren Installation ist die dortige Anleitung zu beachten.


\section{Fehlermeldungen, Fragen, Probleme}
\seclabel{errors}

\iffree{%
  Sollten Sie der Meinung sein, dass Sie einen Fehler in der Anleitung, einer
  der \KOMAScript-Klassen, einem der \KOMAScript-Pakete oder einem anderen
  Bestandteil von \KOMAScript{} gefunden haben, so sollten Sie wie folgt
  vorgehen. Pr�fen Sie zun�chst, ob inzwischen eine neue Version von
  \KOMAScript{} erschienen ist. Installieren Sie diese neue Version und
  kontrollieren Sie, ob der Fehler oder das Problem auch dann noch vorhanden
  ist.

  Wenn es sich nicht um einen Fehler in der Dokumentation handelt und der
  Fehler oder das Problem nach einem Update noch immer auf"|tritt, erstellen
  Sie bitte wie in \cite{DANTE:FAQ} angegeben ein minimales Beispiel. Gehen
  Sie dazu wie unten beschrieben vor.
  % Ein solches Beispiel sollte nur einen minimalen Text und nur die Pakete
  % und Definitionen enthalten, die f�r die Verdeutlichung des Fehlers
  % unbedingt notwendig sind. Auf exotische Pakete sollte m�glichst ganz
  % verzichtet werden.
  Oft l�sst sich ein Problem durch ein minimales Beispiel so weit eingrenzen,
  dass bereits vom Anwender selbst festgestellt werden kann, ob es sich um
  einen Anwendungsfehler handelt oder nicht. Auch ist so sehr h�ufig zu
  erkennen, welche Pakete oder Klassen konkret das Problem verursachen und ob
  es sich �berhaupt um ein \KOMAScript-Problem handelt. Dies k�nnen Sie
  gegebenenfalls zus�tzlich �berpr�fen, indem Sie statt einer
  \KOMAScript-Klasse einen Test mit der entsprechenden Standardklasse
  vornehmen. Danach ist dann auch klar, ob der Fehlerbericht an den Autor von
  \KOMAScript{} oder an den Autor eines anderen Pakets zu richten ist. Sie
  sollten sp�testens jetzt noch einmal gr�ndlich die Anleitungen der
  entsprechenden Paket, Klassen und \KOMAScript-Bestandteile studieren sowie
  \cite{DANTE:FAQ} konsultieren. M�glicherweise existiert ja bereits eine
  L�sung f�r Ihr Problem, so dass sich eine Fehlermeldung er�brigt.

  Wenn Sie denken, dass Sie einen noch unbekannten Fehler gefunden haben, oder
  es aus anderem Grund f�r sinnvoll oder notwendig erachten, den
  \KOMAScript-Autor zu kontaktieren, so sollten Sie dabei folgende Angaben
  keinesfalls vergessen:
  \begin{itemize}
  \item Tritt das Problem auch auf, wenn statt einer \KOMAScript-Klasse eine
    Standardklasse verwendet wird? In dem Fall liegt der Fehler h�chst
    wahrscheinlich nicht bei \KOMAScript. Es ist dann sinnvoller, die Frage in
    einem �ffentlichen Forum, einer Mailingliste oder im Usenet zu stellen.
  \item Welche \KOMAScript-Version verwenden Sie? Entsprechende Informationen
    finden Sie in der \File{log}-Datei des \LaTeX-Laufs jedes Dokuments, das
    eine \KOMAScript-Klasse verwendet.
  \item Welches Betriebssystem und welche \TeX-Distribution wird verwendet?
    Diese Angaben erscheinen bei einem bestriebssystemunabh�ngigen Paket wie
    \KOMAScript{} oder \LaTeX{} eher �berfl�ssig. Es zeigt sich aber immer
    wieder, dass sie durchaus eine Rolle spielen k�nnen.
  \item Was genau ist das Problem oder der Fehler? Beschreiben Sie das Problem
    oder den Fehler lieber zu ausf�hrlich als zu knapp. Oftmals ist es
    sinnvoll auch die Hintergr�nde zu erl�utern.
  \item Wie sieht ein vollst�ndiges Minimalbeispiel aus? Ein solches
    vollst�ndiges Minimalbeispiel kann jeder leicht selbst erstellen, indem
    Schritt f�r Schritt Inhalte und Pakete aus dem Problemdokument
    auskommentiert werden. Am Ende bleibt ein Dokument, das nur die Pakete
    l�dt und nur die Teile enth�lt, die f�r das Problem notwendig
    sind. Au�erdem sollten alle geladenen Abbildungen durch
    \Macro{rule}-Anweisungen entsprechender Gr��e ersetzt werden. Vor dem
    Verschicken entfernt man nun die auskommentierten Teile, f�gt als erste
    Zeile die Anweisung \Macro{listfiles} ein und f�hrt einen weiteren
    \LaTeX-Lauf durch. Man erh�lt dann am Ende der \File{log}-Datei eine
    �bersicht �ber die verwendeten Pakete. Das vollst�ndige Minimalbeispiel
    und die \File{log}-Datei f�gen Sie ihrer Beschreibung hinzu.
  \end{itemize}
  Schicken Sie keine Pakete, PDF- oder PS- oder DVI-Dateien mit.  Falls die
  gesamte Problem- oder Fehlerbeschreibung einschlie�lich Minimalbeispiel und
  \File{log}-Datei gr��er als ein paar Dutzend KByte ist, haben Sie mit
  gr��ter Wahrscheinlichkeit etwas falsch gemacht. Anderenfalls schicken Sie
  Ihre Mitteilung an \href{mailto:komascript@gmx.info}{komascript@gmx.info}.

  H�ufig werden Sie eine Frage zu \KOMAScript{} oder im Zusammenhang mit
  \KOMAScript{} lieber �ffentlich, beispielsweise in \texttt{de.comp.text.tex}
  oder dem Forum auf \cite{homepage}, stellen wollen. Auch in diesem Fall
  sollten Sie obige Punkte beachten, in der Regel jedoch auf die
  \File{log}-Datei verzichten. F�gen Sie stattdessen nur die Liste der Pakete
  und Paketversionen aus der \File{log}-Datei an. Im Falle einer Fehlermeldung
  zitieren Sie diese ebenfalls aus der \File{log}-Datei.%
}{%
  Der Autor hat sich gro�e M�he gegeben, Fehler in diesem Buch zu
  vermeiden. Die Beispiele, die in diesem Buch abgedruckt sind, wurden
  gr��tenteils w�hrend ihrer Entstehung getestet. Leider sind trotzdem weder
  orthografische noch inhaltliche Fehler komplett auszuschlie�en. Sollten Sie
  Fehler in diesem Buch finden, so melden Sie diese bitte �ber die
  \KOMAScript-Support-Adresse, \mbox{komascript@gmx.info}, oder �ber
  \cite{homepage} an den Autor.

  Bei Fehlern an \KOMAScript{} selbst beachten Sie bitte die Prozedur, die in
  der Einleitung der freien \KOMAScript-Anleitung erkl�rt ist. Nur so ist
  sichergestellt, dass das Problem auch reproduziert werden kann. Dies ist
  f�r die Beseitigung eventueller Fehler eine Grundvoraussetzung.%
}

Bitte beachten Sie, dass typografisch nicht optimale Voreinstellungen keine
Fehler darstellen. Aus Gr�nden der Kompatibilit�t werden Voreinstellungen nach
M�glichkeit auch in neuen \KOMAScript-Versionen beibehalten. Dar�ber hinaus
ist Typografie auch eine Frage der Sprache und Kultur. Die Voreinstellungen
von \KOMAScript{} stellen daher zwangsl�ufig einen Kompromiss dar.

\iffree{%
\section{Weitere Informationen}
\seclabel{moreinfos}

Sobald Sie im Umgang mit \KOMAScript{} ge�bt sind, werden Sie sich
m�glicherweise Beispiele zu schwierigeren Aufgaben w�nschen. Solche Beispiele
gehen �ber die Vermittlung von Grundwissen hinaus und sind daher\iffree{}{
  au�er im Angang} nicht Bestandteil dieser Anleitung. Auf den Internetseiten
des \KOMAScript{} Documentation Projects \cite{homepage} finden Sie jedoch
weiterf�hrende Beispiele. Diese sind f�r fortgeschrittene \LaTeX-Anwender
konzipiert. F�r Anf�nger sind sie wenig oder nicht geeignet.
}{}

\endinput
%%% Local Variables: 
%%% mode: latex
%%% mode: flyspell
%%% coding: iso-latin-1
%%% ispell-local-dictionary: "de_DE"
%%% TeX-master: "../guide.tex"
%%% End: 

%  LocalWords:  Installationspakete Zwischenversionen

\partie{Un nom de partie}
\label{panom}

\verset{}
\lipsum % texte de remplissage
\footnote{Note de remplissage.}

\titre{Un nom de titre}
\label{tinom}

\verset{}
\lipsum % texte de remplissage
\footnote{Note de remplissage.}

\chapitre{Un nom de chapitre}
\label{chnom}

\verset{}
\lipsum % texte de remplissage
\footnote{Note de remplissage.}

\section{Un nom de section}
\label{secnom}

\verset{}
\lipsum % texte de remplissage
\footnote{Note de remplissage.}

\paragraphe{Un \emph{nom} de paragraphe}
\label{pgnom}

\verset{}
\lipsum % texte de remplissage
\footnote{Note de remplissage.}

\souspara{Un nom de sous-paragraphe}
\label{spgnom}

\verset{}
\lipsum % texte de remplissage
\footnote{Note de remplissage.}

\alinea{Un nom de alinéa}
\label{alnom}

\verset{}
\lipsum % texte de remplissage
\footnote{Note de remplissage.}

\sousalinea{Un nom de sous-alinéa}
\label{salnom}

\verset{}
\lipsum % texte de remplissage
\footnote{Note de remplissage.}

\point{Un nom de point}
\label{ptnom}

\verset{}
\lipsum % texte de remplissage
\footnote{Note de remplissage.}

\souspoint{Un nom de sous-point}
\label{sptnom}

\verset{}
\lipsum % texte de remplissage
\footnote{Note de remplissage.}
 % première partie
\partie{Un autre nom de partie}

\verset{}
\lipsum % texte de remplissage

\titre{Un autre nom de titre}

\verset{}
\lipsum % texte de remplissage

\chapitre{Un autre nom de chapitre}

\verset{}
\lipsum % texte de remplissage

\section{Un autre nom de section}

\verset{}
\lipsum % texte de remplissage
 % deuxième partie

\backmatter % pages en chiffres arabes, sections non numérotées
\bookmarksetup{startatroot} % RAZ du niveau des signets PDF
\section{Conclusion and future work}
\label{sec:conclusion}

Current rule-based query optimizers do not provide a very intuitive and
conceptually streamlined framework to define rules and actions.  Our
experiences with the Volcano optimizer generator suggest that its model
of rules and the expression of these rules is much more complicated and
too low-level than it needs to be.  As a consequence, rule sets in
Volcano are fragile, hard to write, and debug.  Similar problems may
exist in other contemporary rule-based query optimizers.

We believe that rule-based query optimizers will be standard tools
of future database systems.  The pragmatic difficulties of using
existing rule-based optimizers led us to develop Prairie, an
extensible and structured algebraic framework for specifying rules.
Prairie is similar to existing optimizers in that it supports both
transformation rules and implementation rules.  However, Prairie
makes several improvements:
\begin{enumerate}
\item it offers a conceptually more streamlined model for rule specification;
\item rules are encapsulated, there are no ``hidden'' operators or
      ``hidden'' algorithms;
\item implementation hints (\eg enforcers) are deduced automatically;
\item and it has efficient implementations.
\end{enumerate}

We have explained how the first three points are important for
simplifying rule specifications and making rule sets less brittle for
extensibility.  A consequence is that Prairie rules are simpler and
more robust than rules of existing optimizers (\eg Volcano).  We
addressed the fourth point by building a P2V pre-processor which uses
sophisticated algorithms to compose and compact a Prairie rule set into
a Volcano rule set.  To demonstrate the scalability of our approach, we
rewrote the TI Open OODB rule set as a Prairie rule set, generated its
Volcano counterpart, and showed that the performance of the synthesized
Volcano rule set closely matches the hand-crafted Volcano rule set.

Our future work will concentrate on developing higher-level
abstractions using Prairie, including automatically generating Prairie
rule sets, and combining multiple Prairie rule sets to automatically
generate efficient optimizers.

\section*{Acknowledgments}
\label{sec:acknowledgments}

We wish to thank Texas Instruments, Inc.\ for making the Open OODB
source code available to us.  Comments by Jos\'e Blakeley, Anne Ngu,
Vivek Singhal, Thomas Woo and the anonymous referees greatly improved
the quality of the paper.

\chapitre{Annexes}

\verset{}
\lipsum % texte de remplissage


%%% paramètres de la bibliographie %%%

% forcer l'impression de toutes les références non citées dans le texte
\nocite{*}

% titres des sections de la bibliographie
\defbibheading{france}{\section{Droit français}}
\defbibheading{europe}{\section{Droit européen}}

% titres de types de sources
\defbibheading{lois}{\paragraphe{Lois}}
\defbibheading{rapports}{\paragraphe{Rapports officiels}}
\defbibheading{jurisprudence}{\paragraphe{Jurisprudence}}
\defbibheading{generaux}{\paragraphe{Ouvrages généraux}}
\defbibheading{speciaux}{\paragraphe{Ouvrages speciaux}}
\defbibheading{theses}{\paragraphe{Thèses}}
\defbibheading{speciaux}{\paragraphe{Ouvrages spéciaux}}
\defbibheading{collectifs}{\paragraphe{Ouvrages collectifs}}
\defbibheading{articles}{\paragraphe{Articles}}

% titres pour la jurisprudence
\defbibheading{juris_ccel}{\souspara{Conseil constitutionnel}}
\defbibheading{juris_ce}{\souspara{Conseil d'État}}
\defbibheading{juris_cass}{\souspara{Cour de cassation}}
	\defbibheading{juris_cass_ass}{\alinea{Assemblée plénière}}
	\defbibheading{juris_cass_1civ}{\alinea{1\iere{} chambre civile}}
	\defbibheading{juris_cass_2civ}{\alinea{2\ieme{} chambre civile}}
	\defbibheading{juris_cass_3civ}{\alinea{3\ieme{} chambre civile}}
	\defbibheading{juris_cass_com}{\alinea{Chambre commerciale}}
	\defbibheading{juris_cass_soc}{\alinea{Chambre sociale}}
	\defbibheading{juris_cass_crim}{\alinea{Chambre criminelle}}
\defbibheading{juris_ca}{\souspara{Cours d'appel}}
\defbibheading{juris_tgi}{\souspara{Tribunaux de grande instance}}
\defbibheading{juris_ti}{\souspara{Tribunaux d'instance}}

% filtres de sélection
\defbibfilter{gen}{\(\type{book} \or \type{inbook}\) \and \not \keyword{special}}
\defbibfilter{spec}{\(\type{book} \or \type{inbook}\) \and \keyword{special}}
\defbibfilter{col}{\type{collection} \or \type{proceedings}}
\defbibfilter{art}{\type{incollection} \or \type{inproceedings} \or \type{article}}

%%% impression des différentes bibliographies %%%

\chapitre{Bibliographie}

\printbibheading[heading=france]

\printbibliography[heading=lois,type=legislation,keyword=french]
\printbibliography[heading=rapports,type=report,keyword=french]
\printbibheading[heading=jurisprudence]
	\newrefcontext[sorting=iymd]
	\printbibliography[heading=juris_ccel, type=jurisdiction, keyword=ccel]
	\printbibliography[heading=juris_ce, type=jurisdiction, keyword=ce]
	\printbibheading[heading=juris_cass]
		\printbibliography[heading=juris_cass_ass, type=jurisdiction, keyword=cassass]
		\printbibliography[heading=juris_cass_1civ, type=jurisdiction, keyword=cass1civ]
		\printbibliography[heading=juris_cass_2civ, type=jurisdiction, keyword=cass2civ]
		\printbibliography[heading=juris_cass_3civ, type=jurisdiction, keyword=cass3civ]
		\printbibliography[heading=juris_cass_com, type=jurisdiction, keyword=casscom]
		\printbibliography[heading=juris_cass_soc, type=jurisdiction, keyword=casssoc]
		\printbibliography[heading=juris_cass_crim, type=jurisdiction, keyword=casscrim]
	\printbibliography[heading=juris_ca, type=jurisdiction, keyword=ca]
	\printbibliography[heading=juris_tgi, type=jurisdiction, keyword=tgi]
	\printbibliography[heading=juris_ti, type=jurisdiction, keyword=ti]
	\endrefcontext % sorting=iymd
\printbibliography[heading=generaux, filter=gen, keyword=french]
\printbibliography[heading=theses, type=thesis, keyword=french]
\printbibliography[heading=speciaux, filter=spec, keyword=french]
\printbibliography[heading=collectifs, type=collection, keyword=french]
\printbibliography[heading=articles, filter=art, keyword=french]

\printbibheading[heading=europe]

\printbibliography[heading=lois, type=legislation, keyword=ue]
\printbibliography[heading=rapports, type=report, keyword=ue]
\newrefcontext[sorting=tymdi]
\printbibliography[heading=jurisprudence, type=jurisdiction, keyword=ue]
\endrefcontext % sorting=tymdi
\printbibliography[heading=generaux, filter=gen, keyword=ue]
\printbibliography[heading=speciaux, filter=spec, keyword=ue]
\printbibliography[heading=collectifs, filter=col, keyword=ue]
\printbibliography[heading=theses, type=thesis, keyword=ue]
\printbibliography[heading=articles, filter=art, keyword=ue]


\pagestyle{plain} % en-tetes vides
% \iffalse
% ====================================================================
%  @LaTeX-style-file{
%     filename        = "index.dtx",
%     version         = "4.02beta",
%     date            = "20 January 2004",
%     time            = "21:15:11 EST",
%     author          = "David M. Jones",
%     address         = "MIT Laboratory for Computer Science
%                        Room NE43-316
%                        545 Technology Square
%                        Cambridge, MA 02139
%                        USA",
%     telephone       = "(617) 253-5936",
%     FAX             = "(617) 253-3480",
%     checksum        = "39539 1398 6018 50774",
%     email           = "dmjones@theory.lcs.mit.edu",
%     codetable       = "ISO/ASCII",
%     keywords        = "LaTeX, index",
%     supported       = "yes",
%     docstring       = "This is a reimplementation of LaTeX's
%                        indexing macros to provide better support
%                        for indexing in LaTeX.  For example, it
%                        supports multiple indexes in a single
%                        document and provides a more robust \index
%                        command.
%
%                        The checksum field above contains a CRC-16
%                        checksum as the first value, followed by
%                        the equivalent of the standard UNIX wc
%                        (word count) utility output of lines,
%                        words, and characters.  This is produced
%                        by Robert Solovay's checksum utility.",
%
%  }
% ====================================================================
%
% This file may be distributed and/or modified under the
% conditions of the LaTeX Project Public License, either version 1.2
% of this license or (at your option) any later version.
% The latest version of this license is in
%    http://www.latex-project.org/lppl.txt
% and version 1.2 or later is part of all distributions of LaTeX
% version 1999/12/01 or later.
%
%    CAUTION: Use only as directed.  Do not take internally.  May cause
%    rash if applied directly to skin.  Federal law prohibits distributing
%    without a proscription.
%
% \fi
%
%% \CheckSum{755}
%% \CharacterTable
%%  {Upper-case    \A\B\C\D\E\F\G\H\I\J\K\L\M\N\O\P\Q\R\S\T\U\V\W\X\Y\Z
%%   Lower-case    \a\b\c\d\e\f\g\h\i\j\k\l\m\n\o\p\q\r\s\t\u\v\w\x\y\z
%%   Digits        \0\1\2\3\4\5\6\7\8\9
%%   Exclamation   \!     Double quote  \"     Hash (number) \#
%%   Dollar        \$     Percent       \%     Ampersand     \&
%%   Acute accent  \'     Left paren    \(     Right paren   \)
%%   Asterisk      \*     Plus          \+     Comma         \,
%%   Minus         \-     Point         \.     Solidus       \/
%%   Colon         \:     Semicolon     \;     Less than     \<
%%   Equals        \=     Greater than  \>     Question mark \?
%%   Commercial at \@     Left bracket  \[     Backslash     \\
%%   Right bracket \]     Circumflex    \^     Underscore    \_
%%   Grave accent  \`     Left brace    \{     Vertical bar  \|
%%   Right brace   \}     Tilde         \~}
%
%    \iffalse
%    \begin{macrocode}
%<*driver>
\ProvidesFile{index.dtx}[1995/09/28 v4.1beta Improved index support (dmj)]
%</driver>
%    \end{macrocode}
%
%    \begin{macrocode}
%<*driver>
\documentclass{ltxdoc}
\def\docdate {7 March 1994}
%    \end{macrocode}
%    We don't want everything to appear in the index
%    \begin{macrocode}
\DoNotIndex{\!,\/,\?,\@,\^,\_}
\DoNotIndex{\@@par,\@M,\@auxout,\@bsphack,\@esphack,\@depth,\@ehc}
\DoNotIndex{\@for,\@flushglue,\@gobble,\@gobbletwo,\@height,\@idxitem}
\DoNotIndex{\@ifnextchar,\@ifstar,\@ifundefined,\@input,\@latexerr}
\DoNotIndex{\@makeschapterhead,\@namedef,\@nameuse,\@nil}
\DoNotIndex{\@nobreakfalse,\@restonecolfalse,\@restonecoltrue}
\DoNotIndex{\@tempa,\@tempf,\@temptokena,\@themark,\@width}
\DoNotIndex{\active,\aindex,\baselineskip,\begin,\begingroup,\box}
\DoNotIndex{\c@page,\catcode,\chapter,\char,\chardef,\closeout}
\DoNotIndex{\CodelineIndex,\sp,\sb,\label,\leavevmode,\mark}
\DoNotIndex{\mark,\newinsert,\newwrite,\newtoks,\xdef}
\DoNotIndex{\columnsep,\columnseprule,\columnwidth,\csname,\def}
\DoNotIndex{\dimen,\do,\DocInput,\documentstyle,\edef,\em}
\DoNotIndex{\EnableCrossrefs,\end,\endcsname,\endgroup,\endinput}
\DoNotIndex{\everypar,\expandafter,\filedate,\fileversion}
\DoNotIndex{\footnotesize,\gdef,\global,\glossary,\hangindent}
\DoNotIndex{\if@filesw,\else,\fi}
\DoNotIndex{\if@nobreak,\if@twocolumn,\if@twoside,\fi,\fi,\fi}
\DoNotIndex{\hsize,\hskip}
\DoNotIndex{\ifhmode,\ifmmode,\ifodd,\ifvmode,\ifx,\fi,\fi,\fi,\fi,\fi}
\DoNotIndex{\immediate,\insert,\item,\jobname,\long}
\DoNotIndex{\let,\lineskip,\marginparsep,\marginparwidth,\maxdimen}
\DoNotIndex{\makeatletter,\noexpand,\openout,\protect,\rlap}
\DoNotIndex{\min,\newpage,\nobreak,\normalbaselineskip}
\DoNotIndex{\normallineskip,\p@,\par,\parfillskip,\parindent,\parskip}
\DoNotIndex{\penalty,\relax,\section,\sin,\sloppy,\space,\string}
\DoNotIndex{\tableofcontents,\the,\thepage,\thispagestyle,\toks,\tt}
\DoNotIndex{\twocolumn,\uppercase,\vbox,\vrule,\vskip,\vss}
\DoNotIndex{\write,\z@,\z@skip}
%    \end{macrocode}
%  Some useful macros and parameter settings:
%    \begin{macrocode}

\setcounter{StandardModuleDepth}{1}

\GetFileInfo{index.dtx}

\newcommand*{\email}[1]{$\langle$\texttt{#1}$\rangle$}
\newcommand*{\vdate}[1]{$\langle$#1$\rangle$}
\newcommand*{\bundle}[1]{\texttt{#1}}
\newcommand*{\program}[1]{\textsf{#1}}
\newcommand*{\Ltag}[1]{\texttt{\bslash#1}}
\newcommand*{\Lopt}[1]{\textsf {#1}}
\newcommand*{\Lenv}[1]{\texttt {#1}}
\newcommand*{\cls}[1]{\texttt {#1}}
\newcommand*{\pck}[1]{\texttt {#1}}
\newcommand*{\file}[1]{\texttt {#1}}
\CodelineIndex
%    \end{macrocode}
%    And the document itself:
%    \begin{macrocode}
\begin{document}
\DocInput{index.dtx}
\PrintIndex
% ^^A\PrintChanges
\end{document}
%</driver>
%    \end{macrocode}
%
%    \fi
%
%    \title{A new implementation of \LaTeX's indexing
%        commands\thanks{This file has version number \fileversion,
%        last revised \filedate, documentation dated \docdate.  The
%        definitive version of this file is at
%        \file{ftp://theory.lcs.mit.edu/pub/tex/index/}.}}
%
%    \author{David M. Jones}
%
%    \date{\filedate}
%
%    \maketitle
%
%    \section{Introduction}
%
%    This style file reimplements \LaTeX's indexing macros to provide
%    better and more robust support for indexes.  In particular, it
%    provides the following features:\footnote{Earlier versions of
%    this package provided a ``shortindexing'' feature (see below for
%    description).  This feature is now deprecated and will be removed
%    in a future release of this package.}
%    \begin{enumerate}
%
%    \item
%    Support for multiple indexes.
%
%    \item
%    Indexing of items by counters other than the page number.
%
%    \item
%    A $*$-variant of the \cs{index} command that, in addition to
%    putting it's argument in the index, also typesets it in the
%    running text.
%
%    \item
%    The \bundle{showidx} style option has been merged into this file.
%    The command \cs{proofmodetrue} can be used to enable the printing
%    of index entries in the margin of pages.  The size and style of
%    font can be controlled with the \cs{indexproofstyle} command.
%
%    \item
%    A two-stage process, similar to that used to create tables of
%    contents, for creating the raw index files.  This means that when
%    processing a portion of a document using the \cs{includeonly}
%    command, the index entries from the rest of the document are not
%    lost.
%
%    \item
%    A more robust \cs{index} command.  In particular, it no longer
%    depends on \cs{catcode} changes to work properly, so the new
%    \cs{index} command can be used in places that the original
%    couldn't, such as inside the arguments of other macros.
%
%    \end{enumerate}
%
%
%    \section{Creating an index with \LaTeX}
%
%    Conceptually, there are four stages to creating an index.  First,
%    \LaTeX\ must be informed of your intention to include an index in
%    your document.  Second, you must add appropriate markup commands
%    to your document to tell \LaTeX\ what to put in the index.
%    Third, after \LaTeX\ has been run on your document, the raw index
%    information must be processed and turned into a form that \LaTeX\
%    can process to typeset the index.  Finally, the finished index
%    must be inserted at the appropriate point in your document.
%
%    In \LaTeX, these steps are accomplished with the commands
%    \cs{makeindex}, \cs{index}, \cs{printindex}, and (typically) with
%    the auxiliary program \program{MakeIndex}.  For example, assuming
%    that your main file is called \file{foo.tex}, \cs{makeindex}
%    opens the file \file{foo.idx} and initializes it for holding the
%    raw index entries, and \cs{index} is used to add raw index
%    entries into \file{foo.idx}.  Then the raw index file is
%    processed by \program{MakeIndex}, which puts the finished index
%    in \file{foo.ind}.  Finally, the \cs{printindex} command is used
%    in your \LaTeX\ document to indicate where the file
%    \file{foo.idx} should be inserted, i.e., where the index should
%    appear in your document.
%
%    The \bundle{index} package modifies the \cs{makeindex},
%    \cs{index}, and \cs{printindex} commands, as described below.
%
%
%    \section{The user interface}
%
%    There are four pieces of information associated with each index:
%    \begin{enumerate}
%
%    \item
%    A short, unique tag that identifies the index.
%
%    \item
%    The extension of the output file where the raw index information
%    will be put by \LaTeX.
%
%    \item
%    The extension of the input file where the processed information
%    created by \program{MakeIndex} will be stored to be read in later
%    by \LaTeX.
%
%    \item
%    The title of the index.
%
%    \end{enumerate}
%
%    \DescribeMacro{\newindex}
%    Correspondingly, the \cs{newindex} command has four required
%    arguments.  For example, to declare an author index, you might
%    use the following:
%    \begin{verbatim}
%    \newindex{aut}{adx}{and}{Name Index}\end{verbatim}
%    Here, \texttt{aut} is the tag used to identify the author index,
%    and ``Name Index'' is the title of the index.  If the name of
%    your main file is \file{root.tex}, then \LaTeX\ will write the
%    raw index entries to the file \file{root.adx}, and you will
%    execute the following \program{MakeIndex} command to process the
%    author index:
%    \begin{verbatim}
%    makeindex -o root.and root.adx\end{verbatim}
%
%    By default, the \cs{index} tags its argument with the page number
%    (i.e., the value of \cs{thepage}), but occasionaly you may want
%    to index items according to a different counter.  For example,
%    you may want an index that contains figure numbers instead of
%    page numbers.  To accomodate, this, the \cs{newindex} command
%    takes an optional argument, which is the name of the command that
%    generates the number that should be included in the index.  For
%    instance, to include the number of a figure, you might say
%    \begin{verbatim}
%    \newindex[thefigure]{fig}{fdx}{fnd}{Figures}\end{verbatim}
%
%    However, this introduces a new technicality: When creating an
%    index with page numbers, the choice of which page number is to be
%    written to the \texttt{aux} file should be deferred until the
%    page containing the entry is shipped out to the \texttt{dvi}
%    file, otherwise the wrong number will sometimes be chosen.
%    However, when using counters other than the page counter, one
%    normally wants the opposite behaviour: the number written to the
%    \texttt{aux} file should be chosen immediately, otherwise every
%    item on a given page will be tagged with the number of the last
%    item on that page.  So, when a counter is specified using the
%    optional argument of \cs{newindex}, it is assumed that the
%    counter should be evaluated immediately.  If for some reason you
%    need the choice to be deferred until the page is written to the
%    \texttt{dvi} file, you can force this behaviour by putting a $*$
%    {\em after\/} the optional argument:
%    \begin{verbatim}
%    \newindex[thefigure]*{fig}{fdx}{fnd}{Figures}\end{verbatim}
%    (One consequence of this scheme is that if, for some reason, you
%    need the choice of page number to be made immediately instead of
%    being deferred until a page is shipped out to the \texttt{dvi}
%    file, you can acomplish this by beginning your index declaration
%    with
%    \begin{verbatim}
%    \newindex[thepage]*\end{verbatim}
%
%
%    \DescribeMacro{\renewindex}
%    The \cs{renewindex} command takes the same arguments as the
%    \cs{newindex} command and can be used to redefine indexes that
%    have been previously declared.
%
%
%    \DescribeMacro{\makeindex}
%    For backwards compatibility, the \cs{makeindex} command is
%    redefined to use \cs{newindex}.  It is essentially equivalent to
%    \begin{verbatim}
%    \newindex{default}{idx}{ind}{Index}\end{verbatim}
%    The index labeled \texttt{default} is special: it is the one that
%    will be used by \cs{index} and \cs{printindex} unless another
%    index is specified (see below).
%
%
%    \DescribeMacro{\printindex}
%    The \cs{printindex} command is modified by the addition of an
%    optional argument, which is the tag of the index that should be
%    printed.
%
%
%    \DescribeMacro{\index}
%    The \cs{index} command is modified in two ways.  First, there is
%    a $*$-variant of the command that, in addition to putting its
%    argument into an index, also typesets it on the page.  Second,
%    \cs{index} now takes an optional argument to indicate which index
%    the new entry should be added to.  If given, the optional
%    argument should be the identifying tag of a previously-defined
%    index.  If no such tag is supplied, the \texttt{default} index
%    (such as that opened by \cs{makeindex} above) is used.
%
%
%    \DescribeMacro{\shortindexingon}
%    \DescribeMacro{\shortindexingoff}
%    Perhaps the most dubious feature of \bundle{index.sty} is that it
%    allows you to define the characters |^| and |_| to be
%    abbreviations for \cs{index*} and \cs{index} outside of math
%    mode.  These abbreviations are enabled by the
%    \cs{shortindexingon} command and disabled by the
%    \cs{shortindexingoff} command.  The scope of both of these latter
%    commands is local to the current group.  (This might be useful,
%    for example, if you wanted the abbreviations turned on throughout
%    most of the documentation, but turned off in one particular
%    environment.)  In addition,
%    \DescribeEnv{shortindexingon}\Lenv{shortindexingon} can be used
%    as an environment if that seems appropriate.  \textbf{Warning:
%    This feature is deprecated and will disappear in a future release
%    of this package.}
%
%
%    \DescribeMacro{\proofmodetrue}
%    \DescribeMacro{\proofmodefalse}
%    \DescribeMacro{\indexproofstyle}
%    As mentioned above, the \bundle{showidx} document-style option
%    has been merged into \bundle{index.sty}.  It can be turned on
%    with \cs{proofmodetrue} and turned off with \cs{proofmodefalse}.
%    When it is turned on, all index entries\footnote{Well, most, at
%    least.  There are some circumstances under which the index
%    entries won't show up in the proofs, although they will show up
%    in the index.} will be put in the margin of the page where they
%    appear.  By default, they appear in the typewriter font at
%    \cs{footnotesize}, but the user can override this with the
%    \cs{indexproofstyle} command; for example,
%    \begin{verbatim}
%    \indexproofstyle{\footnotesize\it}\end{verbatim}
%    will cause them to be put in italics instead.
%
%
%    \DescribeMacro{\disableindex}
%    There are some circumstances where it might be helpful to
%    suppress the writing of a particular index.  The
%    \cs{disableindex} command is provided for this purpose.  It takes
%    one argument, a comma-separated list of tags of the indexes that
%    should be disabled.  This command should come {\em before\/} the
%    declarations for the indexes that are being
%    disabled\footnote{This limits its usefulness somewhat, but since
%    the output file for an index is opened when the index is
%    declared, the damage has already been done.  We could close the
%    file, but we can't prevent a new output stream from being
%    allocated and we can't keep the old file from being truncated.}.
%    One situation where the \cs{disableindex} command might be useful
%    is if there are so many indexes that you are exhausting \TeX's
%    supply of output streams\footnote{\TeX\ only has 16 output
%    streams, which are allocated with the {\tt\string\newwrite}
%    command.  The standard \LaTeX\ styles use from 3 to 7 of these,
%    which should leave room for up to 9 indexes.  Of course, if you
%    have extra output files, then there will be fewer output streams
%    left for indexes.}.  For example, suppose you have 10 indexes,
%    but only 5 output streams available for indexes.  Then you could
%    add a \cs{disableindex} command to the top of your file to
%    suppress the writing of all but 5 of the indexes.  (Note that the
%    index entries would still get written to the \texttt{aux} file;
%    they just wouldn't get copied to the individual raw index files
%    at the end of the run.)  At the end of the run, you could then
%    re-run your main file a couple of times with different indexes
%    disabled until you had created all of the raw index files.  This
%    is somewhat clumsy, but safer than any alternative I've come up
%    with\footnote{A less clumsy (for the user, at least) solution
%    would be to read the \texttt{aux} file multiple times at the end
%    of the run, each time writing just one of the raw index files.
%    The main disadvantage of this scheme at present is that it would
%    require a modification of {\tt\string\enddocument}.}.
%
%
%    \section{Caveats}
%
%    In order to implement this style file, it's been necessary to
%    modify a number of \LaTeX\ commands seemingly unrelated to
%    indexing, namely, \cs{@starttoc}, \cs{raggedbottom},
%    \cs{flushbottom}, \cs{addcontents}, \cs{markboth}, and
%    \cs{markright}.  Naturally, this could cause incompatibilities
%    between \bundle{index.sty} and any style files that either
%    redefine these same commands or make specific assumptions about
%    how they operate.  See Section~\ref{sec:thecode} for explanations
%    of why these various commands needed modification.
%
%    The redefinition of \cs{@starttoc} is particularly bad, since it
%    introduces an incompatibility with the AMS document classes.
%    This will be addressed soon.
%
%    Unfortunately, it's also been necessary to modify the
%    \Lenv{theindex} environment, so if you don't like the default
%    \LaTeX\ definition, you'll need copy the definition of
%    \Lenv{theindex} from this file and modify it appropriately.
%
%    In the current implementation, \bundle{index.sty} uses one output
%    stream for each index.  Since there are a limited number of
%    output indexes, this means that there is a limit on the number of
%    indexes you can have in a document.  See the description of
%    \cs{disableindex} for a fuller discussion of this problem and one
%    way around it.
%
%
%    \section{To do's}
%
%    It might be nice if the \cs{index*} command parsed its argument
%    so that, for example, instead of writing
%    `|\index{sin@$\sin$}$\sin$|', one could write
%    `|index*{sin@$\sin$}|'.  However, this is fraught with numerous
%    dangers, and I'm both too lazy and too cowardly to undertake it
%    now.
%
%    It would be reasonable to add support for \cs{makeglossary} and
%    similar things, if they were well-defined enough to decide what
%    the general syntax for defining them should be.
%
%    The documentation should be carefully read, edited, and finished,
%    especially since it's still based on the 2.09 version, even
%    though a few substantial changes have been made for the \LaTeXe\
%    version.
%
%    For some truly outlandish ideas, see the file \file{TODO} in the
%    distribution.
%
%    \StopEventually{}
%
%    \section{The code}
%    \label{sec:thecode}
%
%    As is customary, identify this as a \LaTeXe\ package.
%    \begin{macrocode}
%<*style>
\NeedsTeXFormat{LaTeX2e}[1995/06/01]

\ProvidesPackage{index}[2004/01/20 v4.2beta Improved index support (dmj)]
%    \end{macrocode}
%
%    \begin{macro}{\disableindex}
%    The \cs{disableindex} should come before the declarations of the
%    indexes it refers to.  (Question: If an index has been disabled,
%    should it show up in index proofs?  Maybe there should be a
%    separate command to disable index proofs on and index-by-index
%    basis.)
%    \begin{macrocode}
\def\disableindex#1{%
    \@for\@tempa:=#1\do{%
        \@namedef{disable@\@tempa}{}%
        \@ifundefined{tf@\@tempa}{}{%
            \PackageWarningNoLine{index}{It's too late to disable
                the `\@tempa' index;\MessageBreak
                \jobname.\@tempa\space has already
                been opened for output. You \MessageBreak
                should put the \string\disableindex\space command
                before\MessageBreak
                the declaration of the `\@tempa' index}%
        }%
    }%
}
%    \end{macrocode}
%    \end{macro}
%
%    \begin{macro}{\if@newindex}
%    \begin{macro}{\newindex}
%    \begin{macro}{\renewindex}
%    The \cs{newindex} and \cs{renewindex} commands are defined on
%    analogy with the \cs{[re]newcommand} macros.  Each index is
%    identified by a unique tag, which is specified in the first
%    required argument of \cs{newindex}.  Much of the information
%    about the index labeled \meta{tag} is kept in the macro
%    \cs{idx@}\meta{tag}, so we can check to see if a particular index
%    has already been defined by checking whether \cs{idx@}\meta{tag}
%    is defined.  \cs{newindex} and \cs{renewindex} both check to see
%    if their first argument is already associated with an index and
%    then either issue an appropriate error message or call
%    \cs{def@index}.
%
%    The \cs{if@newindex} flag will be used to keep \cs{renewindex}
%    from re-allocating \cs{write} and \cs{toks} registers later.  The
%    \cs{if@tempswa} switch will be used to determine whether the
%    \cs{write}s for this index should be done \cs{immediate}ly or
%    not.
%    \begin{macrocode}
\newif\if@newindex

\def\newindex{%
    \@tempswafalse
    \@ifnextchar[{\@tempswatrue\x@newindex}{\x@newindex[thepage]}%
}

\def\x@newindex[#1]{%
    \@ifstar {\@tempswafalse\y@newindex{#1}}
             {\y@newindex{#1}}%
}

\def\y@newindex#1#2{%
    \@ifundefined{idx@#2}%
        {\@newindextrue\def@index{#1}{#2}}%
        {%
            \@latexerr{Index type `\string#2' already defined}\@ehc
            \expandafter\@gobble\@gobbletwo
        }%
}

\def\renewindex{%
    \@tempswafalse
    \@ifnextchar[{\@tempswatrue\x@renewindex}{\x@renewindex[thepage]}%
}

\def\x@renewindex[#1]{%
    \@ifstar {\@tempswafalse\y@renewindex{#1}}
             {\y@renewindex{#1}}%
}

\def\y@renewindex#1#2{%
    \@ifundefined{idx@#2}%
        {%
            \@newindextrue
            \@latexerr{Index type `\string#2' not defined}\@ehc
        }%
        {\@newindexfalse}%
    \def@index{#1}{#2}%
}
%    \end{macrocode}
%    \end{macro}
%    \end{macro}
%    \end{macro}
%
%    \begin{macro}{\@preamblecmds}
%    Neither \cs{newindex}, \cs{renewindex}, nor \cs{disableindex}
%    should be used anywhere except inside style files or in the
%    preamble of a document, so we add them to the \cs{@preamblecmds}
%    list.
%    \begin{macrocode}
\@onlypreamble\newindex
\@onlypreamble\renewindex
\@onlypreamble\disableindex
%    \end{macrocode}
%    \end{macro}
%
%    \begin{macro}{\def@index}
%    \cs{def@index} does most of the work.  First, it picks up the
%    first three arguments of the \cs{[re]newindex} command and stores
%    the second two in an appropriate \cs{idx@} macro.  The title of
%    the index is treated differently, however, since it is
%    potentially fragile in a particularly odd way.  To prevent
%    mishaps, it is stored in a token register.  In addition to
%    stashing away the information about the index, \cs{def@index}
%    also opens an appropriate output file if we are writing auxiliary
%    files (i.e., unless \cs{nofiles} is in effect).
%
%    \begin{macrocode}
\def\def@index#1#2#3#4{%
    \@namedef{idx@#2}{#3:#4:#1}%
    \expandafter\let\csname if@immediate@#2\endcsname\if@tempswa
    \if@filesw
        \if@newindex
            \expandafter\newtoks\csname idxtitle@#2\endcsname
        \fi
        \@ifundefined{disable@#2}{%
            \if@newindex
                \expandafter\newwrite\csname tf@#2\endcsname
            \else
                \immediate\closeout\@nameuse{tf@#2}%
            \fi
            \immediate\openout\@nameuse{tf@#2}\jobname.#3 %
            \PackageInfo{index}{Writing index file \jobname.#3}%
        }
        {\PackageInfo{index}{Index `#2' disabled -- not opening
                      \jobname.#3}}%
    \fi
    \expandafter\csname idxtitle@#2\endcsname
}
%    \end{macrocode}
%    \end{macro}
%
%    \begin{macro}{\@second}
%    \begin{macro}{\@third}
%    These are useful macros for retrieving the second and third field
%    of an index specification.
%    \begin{macrocode}
\def\@second#1:#2:#3\@nil{#2}

\def\@third#1:#2:#3\@nil{#3}
%    \end{macrocode}
%    \end{macro}
%    \end{macro}
%
%    \begin{macro}{\@nearverbatim}
%    |\@nearverbatim\foo| is much like |\meaning\foo|,
%    except that it suppresses the ``\texttt{macro ->}'' string
%    produced when \cs{meaning} expands a macro.  It is used by
%    \cs{@wrindex} to produce an ``almost verbatim'' copy of their
%    arguments.  This method replaces the use of \cs{@sanitize} from
%    latex.tex and allows indexing macros to be used in places (such
%    as inside macro arguments) where the original \cs{index} command
%    could not.  Thanks to Donald Arseneau
%    \email{asnd@erich.triumf.ca} for pointing out this trick to me.
%    (For more information on this trick, see Dirty Trick \#3 of the
%    \TeX book, page 382).
%
%    As defined, \@nearverbatim only works on macros.  It would be
%    nice if it could work with other tokens, but it's more important
%    that it work only by expansion, which means we can't put in tests
%    to see what the next token is.
%    \begin{macrocode}
\def\@nearverbatim{\expandafter\strip@prefix\meaning}
%    \end{macrocode}
%    \end{macro}
%
%    Now we define the \cs{index} macro itself.  The following
%    definitions are adapted from \bundle{latex.tex} v2.09 \vdate{25
%    March 1992}.
%
%    \begin{macro}{\makeindex}
%    First we redefine \cs{makeindex} to define the default index
%    using \cs{newindex}.  We use \cs{edef} to make sure that
%    \cs{indexname} gets expanded here.  Otherwise we'll get into an
%    infinite loop later on when we try to redefine \cs{indexname}
%    inside the \cs{theindex} environment.
%
%    Unfortunately, this means that if the user changes \cs{indexname}
%    in the preamble, the index will come out with the wrong heading.
%    \begin{macrocode}
\edef\makeindex{%
    \noexpand\newindex{default}{idx}{ind}{\indexname}%
}
%    \end{macrocode}
%    \end{macro}
%
%    \begin{macro}{\if@silentindex}
%    \begin{macro}{\if@addtoindex}
%    \begin{macro}{\if@proofmode}
%    We need three new flags.  The first, \cs{if@silentindex},
%    indicates whether the entry should be typeset in running text, as
%    well as written out to the index; this is used to implement the
%    \cs{index*} command.  The second, \cs{if@addtoindex}, indicates
%    whether entries should be written to the index; this is used to
%    disable the \cs{index} command inside of page headings and tables
%    of contents.  The third, \cs{ifproofmode}, indicates whether
%    index entries should be put in the margin of the page for
%    proofing purposes.
%    \begin{macrocode}
\newif\if@silentindex\@silentindextrue

\newif\if@addtoindex\@addtoindextrue

\newif\ifproofmode\proofmodefalse
%    \end{macrocode}
%    \end{macro}
%    \end{macro}
%    \end{macro}
%
%    \begin{macro}{\index}
%    \begin{macro}{\p@index}
%    \begin{macro}{\x@index}
%    \cs{index} will be made self-protecting (a la \cs{em}, etc.) so
%    it can be used inside, for example, sectioning commands.
%    Unfortunately, to really make \cs{index} robust, we have to
%    redefine some of \LaTeX's commands for dealing with tables of
%    contents and page headings.  (See below.) $*$sigh$*$
%    \begin{macrocode}
\def\index{\protect\p@index}

\def\p@index{%
    \if@silentindex\@bsphack\fi
    \@ifstar{\@silentindexfalse\@xindex}{\@silentindextrue\@xindex}%
}

\def\@xindex{\@ifnextchar[{\@index}{\@index[default]}}
%    \end{macrocode}
%    \end{macro}
%    \end{macro}
%    \end{macro}
%
%    \begin{macro}{\@index}
%    \begin{macro}{\@@index}
%    \begin{macro}{\@wrindex}
%    The following is much more complicated than it should have to be.
%    First, note the check to see if \cs{index} is equal to
%    \cs{@gobble}.  This is so I don't have to redefine
%    \cs{@outputpage}, which temporarily disables \cs{label},
%    \cs{index}, and \cs{glossary} by \cs{let}'ing them equal to
%    \cs{@gobble}.  (For this reason, we have to be very careful to
%    make sure that \cs{index} has expanded to \cs{p@index} before it
%    gets to \cs{@outputpage}.)  Second, note that if
%    \cs{if@addtoindex} is false, we don't complain about undefined
%    index types.  This is because if your page headings, for example,
%    are being typeset in all uppercase, you might end up with
%    something like \cs{index[AUT]{...}} instead of
%    \cs{index[aut]{...}}.
%    \begin{macrocode}
\def\@index[#1]{%
    \ifx\index\@gobble
        \@addtoindexfalse
    \fi
    \def\@tempf{%
        \begingroup
            \@sanitize
            \@@index{#1}%
    }%
    \if@addtoindex
        \@ifundefined{idx@#1}%
            {%
              \def\@tempf{%
                  \@latexerr{Index type `\string#1' undefined}%
                  \@ehc
                  \@silentindextrue
                  \@gobble
              }%
            }%
            {}%
    \fi
    \@tempf
}

\def\@@index#1#2{%
    \endgroup
    \if@addtoindex
        \if@filesw\@wrindex{#1}{#2}\fi
        \ifproofmode\@showidx{#2}\fi
    \fi
    \if@silentindex
        \expandafter\@esphack
    \else
        \@silentindextrue#2%
    \fi
}

\def\@wrindex#1#2{%
    \begingroup
        \def\@tempa{#2}%
        \edef\@tempb{\@nameuse{idx@#1}}%
        \edef\@tempb{\expandafter\@third\@tempb\@nil}%
        \csname if@immediate@#1\endcsname \else
            \expandafter\let\csname\@tempb\endcsname\relax
        \fi
        \edef\@tempa{%
           \write\@auxout{%
              \string\@writefile{#1}{%
                  \string\indexentry{\@nearverbatim\@tempa}%
                                    {\@nameuse{\@tempb}}%
              }%
           }%
        }%
    \expandafter\endgroup\@tempa
    \if@nobreak\ifvmode\nobreak\fi\fi
}
%    \end{macrocode}
%    \end{macro}
%    \end{macro}
%    \end{macro}
%
%    \begin{macro}{\seename}
%    \begin{macro}{\see}
%    \begin{macro}{\printindex}
%    \begin{macro}{\@printindex}
%    The following are adapted from \bundle{makeidx.sty}, v2.09
%    \vdate{21 Oct 91}.  \cs{index@prologue} adapted from
%    \bundle{doc.dtx}.  \Lenv{theindex} based on version from
%    \bundle{classes.dtx}, v1.3g, 26 June 1995.
%    \begin{macrocode}
\providecommand{\seename}{see}

\providecommand*{\see}[2]{\emph{\seename} #1}

\@ifclassloaded{article}{%

    \renewenvironment{theindex}{%
        \edef\indexname{\the\@nameuse{idxtitle@\@indextype}}%
        \if@twocolumn
            \@restonecolfalse
        \else
            \@restonecoltrue
        \fi
        \columnseprule \z@
        \columnsep 35\p@
        \twocolumn[%
            \section*{\indexname}%
            \ifx\index@prologue\@empty\else
                \index@prologue
                \bigskip
            \fi
        ]%
        \@mkboth{\MakeUppercase\indexname}%
                {\MakeUppercase\indexname}%
        \thispagestyle{plain}%
        \parindent\z@
        \parskip\z@ \@plus .3\p@\relax
        \let\item\@idxitem
    }{%
        \if@restonecol
            \onecolumn
        \else
            \clearpage
        \fi
    }
}{%
    \renewenvironment{theindex}{%
        \edef\indexname{\the\@nameuse{idxtitle@\@indextype}}%
        \if@twocolumn
            \@restonecolfalse
        \else
            \@restonecoltrue
        \fi
        \columnseprule \z@
        \columnsep 35\p@
        \twocolumn[%
            \@makeschapterhead{\indexname}%
            \ifx\index@prologue\@empty\else
                \index@prologue
                \bigskip
            \fi
        ]%
        \@mkboth{\MakeUppercase\indexname}%
                {\MakeUppercase\indexname}%
        \thispagestyle{plain}%
        \parindent\z@
        \parskip\z@ \@plus .3\p@\relax
        \let\item\@idxitem
    }{%
        \if@restonecol
            \onecolumn
        \else
            \clearpage
        \fi
    }
}

\def\printindex{\@ifnextchar[{\@printindex}{\@printindex[default]}}

\def\@printindex[#1]{%
    \@ifnextchar[{\@print@index[#1]}{\@print@index[#1][]}%
}

\long\def\@print@index[#1][#2]{%
    \def\@indextype{#1}%
    \long\def\index@prologue{#2}%
    \@ifundefined{idx@#1}%
        {\@latexerr{Index type `\string#1' undefined}\@ehc}%
        {%
            \edef\@tempa{\@nameuse{idx@#1}}%
            \edef\@tempa{%
                \noexpand\@input@{\jobname.\expandafter\@second\@tempa\@nil}%
            }%
            \@tempa
        }%
}
%    \end{macrocode}
%    \end{macro}
%    \end{macro}
%    \end{macro}
%    \end{macro}
%
%    \begin{macro}{\@indexstar@}
%    Now we set things up for
%    \cs{shortindexing}.\footnote{\textbf{Warning:} This feature is
%    deprecated and will be removed entirely in a future release of
%    this package.} First, we define a one-token shorthand for
%    \cs{index*}.  This will be needed in the definition of
%    \cs{idx@activehat}.
%    \begin{macrocode}
\def\@indexstar@{\index*}
%    \end{macrocode}
%    \end{macro}
%    \begin{macro}{\idx@activehat}
%    \begin{macro}{\idx@activebar}
%    Next, we define the values that |^| and |_| will have when
%    shortindexing is turned on.
%    \begin{macrocode}
\def\idx@activehat{%
    \relax
    \ifmmode\expandafter\sp\else\expandafter\@indexstar@\fi
}

\def\idx@activebar{%
    \relax
    \ifmmode\expandafter\sb\else\expandafter\index\fi
}
%    \end{macrocode}
%    \end{macro}
%    \end{macro}
%
%    \begin{macro}{\shortindexingon}
%    \begin{macro}{\shortindexingoff}
%    Now we define the \cs{shortindexingon} and \cs{shortindexinoff}
%    commands to turn shortindexing on and off (surprise!).
%    \cs{shortindexingon} saves the old definitions and \cs{catcode}'s
%    of |^| and |_| so they can later be restored by
%    \cs{shortindexingoff}.  Both of these make their changes local to
%    any enclosing group, so they can be used as declarations to
%    disable or enable shortindexing temporarily.  In addition,
%    \Lenv{shortindexingon} can also be used as an environment.
%
%    This is potentially very confusing.  My basic rationale (if it
%    can be described as such) was that under normal circumstances,
%    one would put \cs{shortindexingon} in the preamble of one's
%    document, and never want to turn it off.  \cs{shortindexingoff}
%    is an attempt to make allowance for the contingency that someone
%    might want to turn shortindexing off, either permanently or
%    temporarily.
%    \begin{macrocode}
\newif\if@shortindexing

\begingroup

    \catcode`\^\active
    \catcode`\_\active

    \gdef\shortindexingon{%
        \@shortindexingtrue
        \chardef\old@idxhatcode\catcode`\^\relax
        \chardef\old@idxbarcode\catcode`\_\relax
        \catcode`\^\active
        \catcode`\_\active
        \let\old@idxhat ^%
        \let\old@idxbar _%
        \let^\idx@activehat
        \let_\idx@activebar
    }

    \gdef\shortindexingoff{%
        \if@shortindexing
            \@shortindexingfalse
            \let^\old@idxhat
            \let_\old@idxbar
            \catcode`\^\old@idxhatcode
            \catcode`\_\old@idxbarcode
        \fi
    }

\endgroup
%    \end{macrocode}
%    \end{macro}
%    \end{macro}
%
%    Now we take some code from \bundle{showidx.sty} and merge it into
%    our new system.  There are four reasons for redefining the
%    commands here rather than just inputting \bundle{showidx.sty} (or
%    requiring the user to do so).  First, \bundle{showidx.sty} ends
%    with a call to \cs{flushbottom}, which I want to avoid.  Second,
%    the instructions for successfully using \bundle{showidx.sty}
%    along with \bundle{index.sty} would be somewhat tricky.  This
%    way, I can just tell users not to use \bundle{showidx.sty} at
%    all.  Third, I need to make some alterations to \cs{@showidx}
%    anyway.  In particular, (a) I need to add the \cs{@sanitizeat}
%    command so this works correctly with AMS-\LaTeX\ and (b) I want
%    to add the \cs{indexproofstyle} command so the user can customize
%    the size and font used for the index proofs.  Finally,
%    \bundle{showidx.sty} has at least two annoying bugs in it.  See
%    the edit-history for version 2.01 for a description.
%
%    \begin{macro}{\@indexbox}
%    This code is adapted from \bundle{showidx.sty}, v2.09 \vdate{16
%    Jun 1991}.
%    \begin{macrocode}
\newinsert\@indexbox

\dimen\@indexbox\maxdimen
%    \end{macrocode}
%    \end{macro}
%
%    \begin{macro}{\@sanitizeat}
%    The definition of \cs{@sanitizeat} is slightly tricky, since we
%    need |@| to be active when this macro is defined, but we also
%    need it to be part of the control sequence name.
%    \begin{macrocode}
\begingroup
    \catcode`\@\active
    \expandafter\gdef\csname\string @sanitizeat\endcsname
        {\def @{\char`\@}}
\endgroup
%    \end{macrocode}
%    \end{macro}
%
%    \begin{macro}{\indexproofstyle}
%    \begin{macro}{\@showidx}
%    \begin{macro}{\@leftidx}
%    \begin{macro}{\@rightidx}
%    \begin{macro}{\@mkidx}
%    \begin{macro}{\raggedbottom}
%    \begin{macro}{\flushbottom}
%    \begin{macro}{\@texttop}
%    \begin{macrocode}
\newtoks\indexproofstyle

\indexproofstyle{\footnotesize\reset@font\ttfamily}

\def\@showidx#1{%
    \insert\@indexbox{%
        \@sanitizeat
        \the\indexproofstyle
        \hsize\marginparwidth
        \hangindent\marginparsep \parindent\z@
        \everypar{}\let\par\@@par \parfillskip\@flushglue
        \lineskip\normallineskip
        \baselineskip .8\normalbaselineskip\sloppy
        \raggedright \leavevmode
        \vrule \@height .7\normalbaselineskip \@width \z@\relax#1\relax
        \vrule \@height\z@ \@depth.3\normalbaselineskip \@width\z@\relax
    }%
    \ifhmode\penalty\@M \hskip\z@skip\fi
}

\def\@leftidx{\hskip-\marginparsep \hskip-\marginparwidth}

\def\@rightidx{\hskip\columnwidth \hskip\marginparsep}

\def\@mkidx{%
    \vbox to \z@{%
        \rlap{%
            \if@twocolumn
                \if@firstcolumn \@leftidx \else \@rightidx \fi
            \else
                \if@twoside
                    \ifodd\c@page \@rightidx \else \@leftidx \fi
                \else
                    \@rightidx
                \fi
            \fi
            \box\@indexbox
        }%
        \vss
    }%
}

\def\raggedbottom{%
    \def\@textbottom{\vskip\z@ plus.0001fil}%
    \let\@texttop\@mkidx
}

\def\flushbottom{\let\@textbottom\relax \let\@texttop\@mkidx}

\let\@texttop\@mkidx
%    \end{macrocode}
%    \end{macro}
%    \end{macro}
%    \end{macro}
%    \end{macro}
%    \end{macro}
%    \end{macro}
%    \end{macro}
%    \end{macro}
%
%    Now, this next bit really gets up my nose.  The only way to make
%    sure that the \cs{index} command gets handled correctly when used
%    inside of sectioning commands is to redefine a bunch of \LaTeX's
%    table of contents and running-heads macros. $*$blech$*$ Fragility
%    rears its ugly head again.
%
%    These are based on \bundle{latex.tex} 2.09 \vdate{25 March 1992}.
%
%    \begin{macro}{\addcontentsline}
%    We need to redefine \cs{addcontentsline} to keep it from
%    expanding \cs{index} commands too far.  In particular, we have
%    removed \cs{index} from the list of macros that are set equal to
%    \cs{@gobble} and we substitute \cs{@vwritefile} for
%    \cs{@writefile}.  This latter change also means that we can
%    simplify the definition of \cs{protect} somewhat.
%    \begin{macrocode}
\CheckCommand\addtocontents[2]{%
  \protected@write\@auxout
      {\let\label\@gobble \let\index\@gobble \let\glossary\@gobble}%
      {\string\@writefile{#1}{#2}}%
}

\renewcommand{\addtocontents}[2]{%
    \protected@write\@auxout
      {\let\label\@gobble \let\glossary\@gobble}%
      {\string\@writefile{#1}{#2}}%
}
%    \end{macrocode}
%    \end{macro}
%
%    \begin{macro}{\@starttoc}
%    We need to redefine \cs{@starttoc} to \cs{@addtoindexfalse} so
%    that items don't get written to the index from within tables of
%    contents.  The only change here is the addition of
%    \cs{@addtoindexfalse}.
%
%    Unfortunately, this will break pretty badly with the AMS document
%    classes, since they redefine \cs{@starttoc} to take two arguments
%    rather than one.  This must be addressed.
%
%    \begin{macrocode}
\let\old@starttoc\@starttoc

\renewcommand{\@starttoc}[1]{%
    \begingroup
        \@addtoindexfalse
        \old@starttoc{#1}%
    \endgroup
}
%    \end{macrocode}
%    \end{macro}
%
%    \begin{macro}{\markboth}
%    \begin{macro}{\markright}
%    Finally, we have to redefine \cs{markboth} and \cs{markright} to
%    keep them from disabling the expansion of \cs{index} while
%    putting section heads into the \cs{mark}.  Otherwise, we'd end up
%    with ``\cs{index}'' in the mark, which would cause problems when
%    \cs{@outputpage} redefines \cs{index} to be equal to
%    \cs{@gobble}.  Instead, we want \cs{index} to expand to
%    \cs{p@index} in the \cs{mark}, so we retain control over what
%    happens in \cs{@outputpage}.
%
%    This time, the only change is to remove \cs{index} from the list
%    of macros that are \cs{let} equal to \cs{relax}.
%    \begin{macrocode}
\CheckCommand*{\markboth}[2]{%
  \begingroup
    \let\label\relax \let\index\relax \let\glossary\relax
    \unrestored@protected@xdef\@themark {{#1}{#2}}%
    \@temptokena \expandafter{\@themark}%
    \mark{\the\@temptokena}%
  \endgroup
  \if@nobreak\ifvmode\nobreak\fi\fi}
\CheckCommand*{\markright}[1]{%
  \begingroup
    \let\label\relax \let\index\relax \let\glossary\relax
    \expandafter\@markright\@themark {#1}%
    \@temptokena \expandafter{\@themark}%
    \mark{\the\@temptokena}%
  \endgroup
  \if@nobreak\ifvmode\nobreak\fi\fi}

\renewcommand{\markboth}[2]{%
  \begingroup
    \let\label\relax \let\glossary\relax
    \unrestored@protected@xdef\@themark {{#1}{#2}}%
    \@temptokena \expandafter{\@themark}%
    \mark{\the\@temptokena}%
  \endgroup
  \if@nobreak\ifvmode\nobreak\fi\fi}

\renewcommand{\markright}[1]{%
  \begingroup
    \let\label\relax \let\glossary\relax
    \expandafter\@markright\@themark {#1}%
    \@temptokena \expandafter{\@themark}%
    \mark{\the\@temptokena}%
  \endgroup
  \if@nobreak\ifvmode\nobreak\fi\fi}
%</style>
%    \end{macrocode}
%    \end{macro}
%    \end{macro}
%
%    \section{Edit history}
%
%    \begin{description}
%
%    \item[v1.00 (4 Mar 1993)]
%    initial version, posted to comp.text.tex.
%
%    \item[v1.01 (4 Mar 1993)]
%    added \cs{renewindex} command and checking to make sure index is
%    (or is not) defined in \cs{newindex}, \cs{index} and
%    \cs{printindex}.  Also tightened up the code in various places
%    and added check to make sure file is only loaded once.
%
%    \item[v2.00 (24 Mar 1993)]
%    added support for \cs{index*}, proofmode, \cs{shortindexingon}
%    and \cs{shortindexingoff}.
%
%    \item[v2.01 (24 Jun 1993)]
%    Fixed 3 bugs.  (1) If proofmode was turned on, then something
%    like ``\cs{index{WORD}WORD}'' would suppress the hyphenation of
%    WORD.  This was fixed by adding ``|\penalty\@M\hskip\z@skip|'' to
%    the end of \cs{@showidx}.  (This is just the definition of
%    \cs{allowhyphens} borrowed from \bundle{german.sty}, v2 \vdate{4
%    Nov 1988}).  (2) The \cs{hbox} in \cs{@mkidx} was being set at
%    its natural width, which had a tendency to interfere with the
%    width of the page.  The \cs{hbox} is now replaced by \cs{rlap}.
%    (3) If the title of an index (i.e., the fourth argument of
%    \cs{newindex}) contained a particularly fragile command
%    like~\cs{d}, havoc would ensue when \cs{theindex} tried to
%    extract the title.  Titles are now kept in token registers to
%    prevent such unpleasantness.  Bugs (2) and (3) were reported by
%    Dominik Wujastyk \email{D.Wujastyk@ucl.ac.uk} on 24 June 1993.
%    Note that bugs (1) and (2) are actually bugs in showidx.sty,
%    v2.09 \vdate{16 Jun 1991}.
%
%    \item[v2.02 (25 Jun 1993)]
%    Rewrote the code that implements the short indexing commands (|^|
%    and |_|) to make index.sty compatible with other style files that
%    need to make |^| and |^| active in some contexts.  See the code
%    for more details.
%
%    \item[v2.03 (30 Jun 1993)]
%    Once again rewrote the code that implements the short indexing
%    commands.  Dumped the shortindexing environment and rewrote the
%    \cs{shortindexingon} and \cs{shortindxingoff} commands to save
%    and restore the \cs{catcode}'s and meanings of |^| and |^| in the
%    safest possible (I hope) order.  Also added the
%    \cs{if@shortindexing} flag to keep \cs{shortindexingoff} from
%    doing anything if it is called outside of the scope of a
%    \cs{shortindexingon} command.  (Question: Should
%    \cs{shortindexingon} check that flag before doing anything?)
%
%    \item[v2.04 (beta) (14 Jul 1993)]
%    Added \cs{disableindex} command.  Added \cs{newindex} and
%    \cs{renewindex} to \cs{@preamblecmds}.  Added \cs{if@newindex}
%    flag to \cs{@newindex} to prevent \cs{renewindex} from
%    re-allocating new \cs{write} and \cs{toks} registers.  Rewrote
%    using \bundle{doc.sty} and \program{DocStrip}.  Also cleaned up
%    the code somewhat.
%
%    \item[v3.00 (15 Jul 1993)]
%    Made further minor tweaks to code and internal documentation.
%    Booted version number up to 3.00 and released on the world.
%
%    \item[v3.01 (19 Jul 1993)]
%    Fixed \program{DocStrip} CheckSum.
%
%    \item[v3.02 (15 Sep 1993)]
%    Corrected spelling of \cs{@shortindexingfalse} in definition of
%    \cs{shortindexingoff}.  Thanks to Hendrik G. Seliger
%    \email{hank@Blimp.automat.uni-essen.de} for this bug report.
%    Also added redefinitions of \cs{@leftmark} and \cs{@rightmark} to
%    fix a bug reported by Dominik Wujastyk
%    \email{D.Wujastyk@ucl.ac.uk}.
%
%    \item[v3.03 (beta) (20 Feb 1994)]
%    Added \cs{long} to the definition of \cs{@ifundefined} to cover
%    the unlikely contingency that someone wanted to use, for example,
%    |\string\par| in the middle of a control sequence name.  Added an
%    optional argument to \cs{newindex} to specify which counter to
%    use in place of \cs{thepage}.  The first change was suggested by
%    Martin Schr\"oder \email{l15d@zfn.uni-bremen.de}; the second was
%    suggested independently by Schr\"oder and Stefan Heinrich
%    H\"oning \email{hoening@pool.informatik.rwth-aachen.de}.  The
%    \cs{@newindex} command was renamed \cs{def@index}.  Also fixed
%    the \cs{disableindex} command.
%
%    \item[v3.04 (7 Mar 1994)]
%    Rewrote the user documentation (Sections 1--5) and released on
%    the world.  Also deleted some extraneous spaces that had crept
%    into some macros.
%
%    \item[v4.00beta, (20 Feb 1995)]
%    Preliminary conversion to a native \LaTeXe\ package.  Fixed
%    \cs{@printindex} to work under \LaTeXe\ (bug reported by Carsten
%    Folkertsma \email{cai@butler.fee.uva.nl}).  Removed much code
%    that had been put in to work around various ancient versions of
%    \LaTeX~2.09.  Added \cs{index@prologue} support (modelled on
%    \bundle{doc.sty}) at suggestion of Nick Higham
%    \email{higham@ma.man.ac.uk}.
%
%    \item[v4.01beta (28 Sep 1995)]
%    Rewrote as a \LaTeXe\ package (finally!).  Changes too numerous
%    to list, but in general deleted some now-superfluous code,
%    replaced some tricks by tricks from the \LaTeXe\ kernel, and
%    added some bullet-proofing.  Much still remains to be done, but
%    this should be good enough for testing.
%
%    Changed definition of \cs{protect} in \cs{markright} and
%    \cs{markboth} to fix bug reported by Dominik Wujastyk.
%
%    \item[??? (5 Jan 2004)]
%
%
% \end{description}
%
% \DisableCrossrefs
%
% \section{The sample file}
%
%    \begin{macrocode}
%<*sample>
%% latex sample.tex
%% makeindex sample
%% makeindex -o sample.and sample.adx
%% makeindex -o sample.nnd sample.ndx
%% makeindex -o sample.lnd sample.ldx
%% latex sample.tex

\documentclass{book}
\usepackage{index}

\listfiles

\makeindex
\newindex{aut}{adx}{and}{Name Index}
\newindex{not}{ndx}{nnd}{List of Notation}

\newindex[theenumi]{list}{ldx}{lnd}{Items}

\shortindexingon

\proofmodetrue

\def\aindex{\index*[aut]}

\begin{document}

\tableofcontents

\newpage

\chapter{Here is a ^[aut]{chapter} title}

\section{Section header\index[aut]{section}}

Here is some text.\index{subject}

Here is \index[not]{notation}some more \index[not]{sin@$\sin$}
text.

\newpage

Here is some ^{more} _[not]{notation} text.

Here is yet more \aindex{text}.

\section{Another Section header _[aut]{section2}}

And here is some math: $x^1_b$.

Here is an ^[aut]{index} entry \fbox{inside an
\index[not]{min@$\min$}fbox}

\fbox{Here is an ^[aut]{entry} in a box.}

\section{An indexed list environment}

\begin{enumerate}

\item
First item

\item
Second item\index[list]{second item}

\item
Third item

\newpage

\item
Fourth item

\item
Fifth item\index[list]{fifth item}

\item
Sixth item

\end{enumerate}

\printindex[not]

\printindex[aut][Here is a prologue for the author index.
Note that it is set in a single column at the top of the
first page of the index.]

\printindex[list]

\printindex

\end{document}
%</sample>
%    \end{macrocode}
%
% \Finale
%
\endinput

% paramètres de la table des matières
\renewcommand*{\contentsname}{Table des matières}
\let\changetocdepth\oldchangetocdepth
\setcounter{tocdepth}{7}

% \cftpagenumberson{book}
% \cftpagenumberson{part}
% \cftpagenumberson{chapter}

\renewcommand{\tocheadstart}{\chapterheadstart}
\makeatletter
\renewcommand{\cftbookbreak}{\addpenalty{-\@highpenalty}\clearpage}
\makeatother
\renewcommand{\cftbeforebookskip}{2em}
\renewcommand{\cftbookfont}{\bfseries\huge}
\renewcommand{\cftbookindent}{0em}
\renewcommand{\cftbooknumwidth}{2em}
%\renewcommand{\cftafterbookskip}{}
\renewcommand{\cftbookleader}{\cftdotfill{\cftdotsep}}

\renewcommand{\cftbeforepartskip}{1.5em}
\renewcommand{\cftpartfont}{\huge}
\renewcommand{\cftpartindent}{0em}
%\renewcommand{\cftpartnumwidth}{}
\renewcommand{\cftpartleader}{\cftdotfill{\cftdotsep}}

\renewcommand{\cftbeforechapterskip}{1em}
\renewcommand{\cftchapterfont}{\LARGE}
\renewcommand{\cftchaptername}{Chapitre\space}
\renewcommand{\cftchapterindent}{0em}
%\renewcommand{\cftchapternumwidth}{}
\renewcommand{\cftchapterleader}{\cftdotfill{\cftdotsep}}

\renewcommand{\cftbeforesectionskip}{0.5em}
\renewcommand{\cftsectionfont}{\Large}
%\renewcommand{\cftsectionname}{Section\space}
\renewcommand{\cftsectionindent}{0em}
\renewcommand{\cftsectionnumwidth}{5.5em}
%\renewcommand{\cftsectionleader}{\cftdotfill{\cftdotsep}}

\renewcommand{\cftbeforesubsectionskip}{0.5em}
\renewcommand{\cftsubsectionfont}{\large}
%\renewcommand{\cftsubsectionname}{}
\renewcommand{\cftsubsectionindent}{0em}
\renewcommand{\cftsubsectionnumwidth}{2em}
%\renewcommand{\cftsubsectionleader}{\cftdotfill{\cftdotsep}}

%\renewcommand{\cftbeforesubsubsectionskip}{}
\renewcommand{\cftsubsubsectionfont}{\normalsize}
%\renewcommand{\cftsubsubsectionname}{}
\renewcommand{\cftsubsubsectionindent}{1em}
\renewcommand{\cftsubsubsectionnumwidth}{2em}
%\renewcommand{\cftsubsubsectionleader}{\cftdotfill{\cftdotsep}}

%\renewcommand{\cftbeforeparagraphskip}{}
\renewcommand{\cftparagraphfont}{\normalsize}
%\renewcommand{\cftparagraphname}{}
\renewcommand{\cftparagraphindent}{2em}
\renewcommand{\cftparagraphnumwidth}{2em}
%\renewcommand{\cftparagraphleader}{\cftdotfill{\cftdotsep}}

%\renewcommand{\cftbeforesubparagraphskip}{}
\renewcommand{\cftsubparagraphfont}{\normalsize}
%\renewcommand{\cftsubparagraphname}{}
\renewcommand{\cftsubparagraphindent}{3em}
\renewcommand{\cftsubparagraphnumwidth}{2em}
%\renewcommand{\cftsubparagraphleader}{\cftdotfill{\cftdotsep}}

\renewcommand{\cftsubsubparagraphfont}{\small}

\renewcommand{\cftsubsubsubparagraphfont}{\footnotesize}


\clearpage
\tableofcontents
 % table des matières
\bookmarksetup{startatroot} % RAZ du niveau des signets PDF
\begin{resume}

  \resumeitem{个人简历}

  xxxx 年 xx 月 xx 日出生于 xx 省 xx 县。

  xxxx 年 9 月考入 xx 大学 xx 系 xx 专业,xxxx 年 7 月本科毕业并获得 xx 学士学位。

  xxxx 年 9 月免试进入 xx 大学 xx 系攻读 xx 学位至今。

  \researchitem{发表的学术论文} % 发表的和录用的合在一起

  % 1. 已经刊载的学术论文(本人是第一作者,或者导师为第一作者本人是第二作者)
  \begin{publications}
    \item Yang Y, Ren T L, Zhang L T, et al. Miniature microphone with silicon-
      based ferroelectric thin films. Integrated Ferroelectrics, 2003,
      52:229-235. (SCI 收录, 检索号:758FZ.)
    \item 杨轶, 张宁欣, 任天令, 等. 硅基铁电微声学器件中薄膜残余应力的研究. 中国机
      械工程, 2005, 16(14):1289-1291. (EI 收录, 检索号:0534931 2907.)
    \item 杨轶, 张宁欣, 任天令, 等. 集成铁电器件中的关键工艺研究. 仪器仪表学报,
      2003, 24(S4):192-193. (EI 源刊.)
  \end{publications}

  % 2. 尚未刊载,但已经接到正式录用函的学术论文(本人为第一作者,或者
  %    导师为第一作者本人是第二作者)。
  \begin{publications}[before=\publicationskip,after=\publicationskip]
    \item Yang Y, Ren T L, Zhu Y P, et al. PMUTs for handwriting recognition. In
      press. (已被 Integrated Ferroelectrics 录用. SCI 源刊.)
  \end{publications}

  % 3. 其他学术论文。可列出除上述两种情况以外的其他学术论文,但必须是
  %    已经刊载或者收到正式录用函的论文。
  \begin{publications}
    \item Wu X M, Yang Y, Cai J, et al. Measurements of ferroelectric MEMS
      microphones. Integrated Ferroelectrics, 2005, 69:417-429. (SCI 收录, 检索号
      :896KM)
    \item 贾泽, 杨轶, 陈兢, 等. 用于压电和电容微麦克风的体硅腐蚀相关研究. 压电与声
      光, 2006, 28(1):117-119. (EI 收录, 检索号:06129773469)
    \item 伍晓明, 杨轶, 张宁欣, 等. 基于MEMS技术的集成铁电硅微麦克风. 中国集成电路,
      2003, 53:59-61.
  \end{publications}

  \researchitem{研究成果} % 有就写,没有就删除
  \begin{achievements}
    \item 任天令, 杨轶, 朱一平, 等. 硅基铁电微声学传感器畴极化区域控制和电极连接的
      方法: 中国, CN1602118A. (中国专利公开号)
    \item Ren T L, Yang Y, Zhu Y P, et al. Piezoelectric micro acoustic sensor
      based on ferroelectric materials: USA, No.11/215, 102. (美国发明专利申请号)
  \end{achievements}

\end{resume}


\end{document}
