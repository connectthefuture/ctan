%--------------------------------------------------------------------
%--------------------------------------------------------------------
% examdoc.tex
%
% This is the user's guide for the exam documentclass, 
% by Philip Hirschhorn.
%
% The exam documentclass itself is in the file exam.cls.


%%% Copyright (c) 1997, 2000, 2004, 2008, 2011, 2015 Philip S. Hirschhorn
%
% This work may be distributed and/or modified under the
% conditions of the LaTeX Project Public License, either version 1.3
% of this license or (at your option) any later version.
% The latest version of this license is in
%   http://www.latex-project.org/lppl.txt
% and version 1.3 or later is part of all distributions of LaTeX 
% version 2003/12/01 or later.
%
% This work consists of the files exam.cls and examdoc.tex

% The user's guide for exam.cls is in the file examdoc.tex.


%%% Philip Hirschhorn
%%% Department of Mathematics
%%% Wellesley College
%%% Wellesley, MA 02481
%%% psh@math.mit.edu

% The newest version of this documentclass should always be available
% from my web page: http://www-math.mit.edu/~psh/


%--------------------------------------------------------------------
%--------------------------------------------------------------------

\documentclass[12pt]{exam}
\usepackage{amssymb}

\usepackage{makeidx}

\usepackage[colorlinks]{hyperref}

% In case we're not using hyperref.sty:
\providecommand{\texorpdfstring}[2]{#1}
% The following can be used in \section commands
% without generating pdf warnings:
\newcommand{\bs}{\texorpdfstring{\char`\\}{}}

\newcommand{\docversion}{2.5}
\newcommand{\docdate}{May 7, 2015}
%\newcommand{\docdate}{\today}

%--------------------------------------------------------------------
%
% Changes since version 2.4 are described in the comments
% near the beginning of the file exam.cls.
%
%--------------------------------------------------------------------

\makeindex

\newcommand{\indc}[1]{\index{#1@\texttt{\char`\\#1}}}
\newcommand{\indcsub}[2]{\index{#1@\texttt{\char`\\#1}!#2}}
\newcommand{\indcstart}[1]{\index{#1@\texttt{\char`\\#1}|(}}
\newcommand{\indcstop}[1]{\index{#1@\texttt{\char`\\#1}|)}}

\newcommand{\indt}[1]{\index{#1@\texttt{#1}}}
\newcommand{\indtsub}[2]{\index{#1@\texttt{#1}!#2}}
\newcommand{\indtstart}[1]{\index{#1@\texttt{#1}|(}}
\newcommand{\indtstop}[1]{\index{#1@\texttt{#1}|)}}

%---------------------------------------------------------------------
\newenvironment{example}%
   {\bigskip\filbreak
    \subsubsection{Example:}
   }%
   {}

\def\samplehead#1#2#3#4{%
  \begin{trivlist}
    \item[]
    \leavevmode
    \hbox to \textwidth{%
      \rlap{\parbox[b]{\textwidth}{\raggedright#1\strut}}%
      \hfil\parbox[b]{\textwidth}{\centering#2\strut}\hfil
      \llap{\parbox[b]{\textwidth}{\raggedleft#3\strut}}%
    }% hbox
    #4
  \end{trivlist}
}

\def\samplefoot#1#2#3#4{%
  \begin{trivlist}
    \item[]
    \leavevmode
    #1
    \vskip 3pt

    \hbox to \textwidth{%
      \rlap{\parbox[t]{\textwidth}{\raggedright#2}}%
      \hfil\parbox[t]{\textwidth}{\centering#3}\hfil
      \llap{\parbox[t]{\textwidth}{\raggedleft#4}}%
    }% hbox
  \end{trivlist}
}

% \makeatletter
%   \@ifundefined{AmS}{\def\AmS{{\protect\the\textfont\tw@
%     A\kern-.1667em\lower.5ex\hbox{M}\kern-.125emS}}}
%     {}
% \makeatother


%---------------------------------------------------------------------
%---------------------------------------------------------------------
%---------------------------------------------------------------------
%---------------------------------------------------------------------

\begin{document}

\title{Using the exam document class}

\author{Philip Hirschhorn\\
  Department of Mathematics\\
  Wellesley College\\
  Wellesley, MA 02481\\
  psh@math.mit.edu\\[\bigskipamount]
  Copyright \copyright~1994, 1997, 2000, 2004, 2008, 2011, 2015 Philip
  Hirschhorn\\
  All rights reserved}

\date{\docdate}

\maketitle

\begin{center}
  \small
  This is the user's guide for version~\docversion{} of the
  \verb"exam" document class. 
\end{center}

\tableofcontents

%--------------------------------------------------------------------
\section{Introduction}

The file \verb"exam.cls" provides the \verb"exam" document class,
which attempts to make it easy for even a \LaTeX{} novice to prepare
exams.  Specifically, \verb"exam.cls" sets the page layout so that
there are one inch margins\index{margins} all around (no matter what
size paper you're using) and provides commands that make it easy to
format questions, create flexible headers and footers, change the
margins, and create grading tables.  In more detail:
\begin{itemize}
\item The class will automatically format and number the questions,
  parts of questions, subparts of parts, and subsubparts of subparts
  (see sections \ref{sec:Questions} and~\ref{sec:parts}).
\item You can include the point value\index{points} of each question
  (or part, or subpart, or subsubpart), with your choice of having the
  point values printed at the beginning of the text of the question,
  opposite that in the left margin, opposite that in the right margin,
  or in the right margin opposite the end of the question (see
  section~\ref{sec:points}).
\item The class will add up the total points for each question (and
  all of its parts, subparts, and subsubparts) and the total
  points\index{points!total} on each page, and make those totals
  available in macros (see sections \ref{sec:addup},
  \ref{sec:pointsofq} and~\ref{sec:pointsonp}).
\item You can have the class print a grading table, indexed either by
  question number or by page number (see
  section~\ref{sec:GradeTables}).
\item You specify the header in three parts: One part to be left
  justified, one part to be centered, and one part to be right
  justified, and one or all of these can be omitted (see
  section~\ref{sec:header}).
\item The footer is also specified in three parts: Left justified,
  centered, and right justified (see section~\ref{sec:footer}).
\item The header and footer for the first page can be different from
  the ones used on other pages (see sections \ref{sec:runningheader},
  \ref{sec:lhead}, \ref{sec:runningfooter}, and \ref{sec:lfoot}).
\item Both headers and footers can contain more than one line.  To
  accommodate headers and footers with several lines, simple commands
  are provided to enlarge the part of the page devoted to the header
  and/or footer, and these commands can give one amount of space on
  the first page and a different amount of space on all other pages
  (see section~\ref{sec:extra-room}).
\item Macros are defined to enable you to state the total number of
  pages in the exam (see section~\ref{sec:numpages}) and to change the
  header and/or footer that appears on the last page of the exam (see
  section~\ref{sec:lastpage}).
\item Macros are defined so that the headers and footers can vary
  depending on whether the current page begins a new question or
  continues a question that started on an earlier page (and, if one
  continues onto the current page, to say what the number of that
  question is).  Macros are also defined so that the headers and
  footers can vary depending on whether a question is complete on the
  current page or continues on to the next page (and, if one
  continues, to say what the number of that question is) (see
  section~\ref{sec:QuesSpan}).
\item You can have a horizontal rule at the base of the header and/or
  at the top of the footer (see section~\ref{sec:rules}).
\item The exam can begin with one or more cover pages, which are
  numbered separately from the main pages of the exam and which can
  have headers and footers different from the ones in the main pages
  of the exam (see section~\ref{sec:coverpages}).
\item You can include solutions in your \LaTeX{} file and have these
  solutions either printed or ignored (or replaced automatically by
  space in which the students can write their answers) depending on a
  single command (see section~\ref{sec:solutions},
  section~\ref{sec:CorrectChoice}, and section~\ref{sec:AnsShortAns}).
\end{itemize}

The latest version of \verb"exam.cls" (possibly a
%
beta\index{beta test} test version) should always be available at
\verb"http://www-math.mit.edu/~psh/"\index{web page}.

%--------------------------------------------------------------------
\subsection{License}

\begin{itemize}
\item This work may be distributed and/or modified under the
  %
  \index{Latex Project Public License@\LaTeX{} Project Public License}
  %
  conditions of the \LaTeX{} Project Public
  License\index{license},\index{LPPL} either version~1.3 of this
  license or (at your option) any later version.  The latest version
  of this license is in \verb"http://www.latex-project.org/lppl.txt"
  and version 1.3 or later is part of all distributions of \LaTeX{}
  version 2003/12/01 or later.
\item This work has the LPPL maintenance status ``author-maintained''.
\item This work consists of the files \verb"exam.cls" and
  \verb"examdoc.tex".
\end{itemize}

%--------------------------------------------------------------------
%--------------------------------------------------------------------
\section{Simple instructions for a basic exam}
\label{sec:Basic}

Producing a basic exam with the \texttt{exam} document class is fairly
easy.  After you learn to do that, you can read the parts of this
user's guide that describe how to customize whatever you want to
change.
%--------------------------------------------------------------------
\subsection{The \texttt{\bs documentclass} command}
\label{sec:BasicDocumentclass}

To use the \texttt{exam} document class, your \verb"\documentclass"
command should be\indc{documentclass}
\begin{center}
  \verb"\documentclass[addpoints]{exam}"
\end{center}
or, if you want to use 12~point type,
\begin{center}
  \verb"\documentclass[addpoints,12pt]{exam}"
\end{center}
There are other options; for the full story, see
section~\ref{sec:Documentclass}.

%--------------------------------------------------------------------
\subsection{Asking for the student's name}
\label{sec:BasicName}

If you'll be leaving space for the answers on the question pages (see
sections~\ref{sec:BasicSpace} and \ref{sec:LeaveSpace}), then you'll
probably also want to leave space for the student's name%
\index{name}\index{student's name}.  If you type
\begin{verbatim}
\begin{center}
  \fbox{\fbox{\parbox{5.5in}{\centering
        Answer the questions in the spaces provided on the
        question sheets.  If you run out of room for an answer,
        continue on the back of the page.}}}
\end{center}

\vspace{0.1in}

\makebox[\textwidth]{Name and section:\enspace\hrulefill}

\vspace{0.2in}

\makebox[\textwidth]{Instructor's name:\enspace\hrulefill}
\end{verbatim}
after the \verb"\begin{document}" command and before the
\verb"\begin{questions}" command (see section~\ref{sec:Questions}),
then you'll get\\
\noindent
\parbox{\textwidth}{
\begin{center}
  \fbox{\fbox{\parbox{5.5in}{\centering
        Answer the questions in the spaces provided on the
        question sheets.  If you run out of room for an answer,
        continue on the back of the page.}}}
\end{center}

\vspace{0.1in}

\makebox[\textwidth]{Name and section:\enspace\hrulefill}

\vspace{0.2in}

\makebox[\textwidth]{Instructor's name:\enspace\hrulefill}
}

%--------------------------------------------------------------------
\subsection{Asking questions}
\label{sec:BasicQuestions}

Once you've typed \verb"\begin{document}" and (if the students will be
writing their answers on the question pages) asked for the student's
name (see section~\ref{sec:BasicName}), you can list the questions
on the exam in a \texttt{questions} environment.  You begin each
question with a \verb"\question"\indc{question} command, which is
optionally followed by a number of points inside of square brackets.
For example, if you type 
%
\index{questions environment}\index{environment!questions}
\indt{questions}
\begin{verbatim}
\begin{questions}

\question[10]
Why is there air?

\question[15]

How much wood would a woodchuck chuck if a woodchuck could chuck
wood?

\question[10] Compute $\displaystyle\int_0^1 x^2 \, dx$.
\end{questions}
\end{verbatim}
then you'll get
\begin{questions}

\question[10]
Why is there air?

\question[15]

How much wood would a woodchuck chuck if a woodchuck could chuck
wood?

\question[10] Compute $\displaystyle\int_0^1 x^2 \, dx$.
\end{questions}

\bigskip

As the above example illustrates, you can leave blank lines
%
\index{blank lines} between the \verb"\question" command and the
actual beginning of the question, or before the first \verb"\question"
command in the environment, and they will be ignored.  The point
values are entirely optional; if you omit the point values, you should
also omit the square brackets containing them.

If you give the command \verb"\marksnotpoints"\indc{marksnotpoints},
then the words ``point'' and ``points'' will be replaced by the words
``mark'' and ``marks''.  There are many other customizations possible
for the points, including using a different language, changing the way
the points are printed, and by having them printed in one of the
margins.  For the full story, see section~\ref{sec:points}.


%--------------------------------------------------------------------
\subsection{Questions with parts, and subparts, and subsubparts}
\label{sec:BasicParts}

If you want to create a question with parts, you use a
\texttt{parts}\indt{parts} environment and begin each part with the
command \verb"\part"\indc{part}.  The \texttt{parts} environment can
be at the beginning of a question or after some text that begins the
question.  (There are also \texttt{subparts} and \texttt{subsubparts}
environments; for the full details, see section~\ref{sec:parts}.)  For
example, if you type
%
\index{parts environment}\index{environment!parts}
\begin{verbatim}
\begin{questions}
\question[10]
Why is there air?

\question
What if there were no air?

\begin{parts}
\part[5]
Describe the effect on the balloon industry.

\part[5]
Describe the effect on the aircraft industry.
\end{parts}

\question[20]

\begin{parts}
\part
Define the universe.  Give three examples.

\part
If the universe were to end, how would you know?

\end{parts}


\end{questions}
\end{verbatim}
then you'll get
\begin{questions}
\question[10]
Why is there air?

\question
What if there were no air?

\begin{parts}
\part[5]
Describe the effect on the balloon industry.

\part[5]
Describe the effect on the aircraft industry.
\end{parts}

\question[20]

\begin{parts}
\part
Define the universe.  Give three examples.

\part
If the universe were to end, how would you know?

\end{parts}


\end{questions}

\bigskip

The above example illustrates several things:
\begin{itemize}
\item Parts of a question should be put into a
  \verb"parts"\indt{parts}\index{parts environment}%
  \index{environment!parts} environment.
\item If a question begins with a \verb"parts" environment, then the
  first part will appear on the same line with the question number.
\item You can leave blank
  %
  lines\index{blank lines} before and after the \verb"\part" command,
  and they will be ignored.
\item You can assign points either to the question or to the parts of
  a question.
\end{itemize}

%--------------------------------------------------------------------
\subsection{Leaving space for the answers}
\label{sec:BasicSpace}
\index{space!for answers}
\index{answer space}

To leave blank space after a question you use the \verb"\vspace"
command, and you end a page with the \verb"\newpage" command.  For
example, if you wanted to leave 3~inches of space after the first
question and 2~inches of space after each of the parts of the second
question, your \texttt{questions} environment would contain the
following:
%
\index{blank space}\index{space!blank}
\begin{verbatim}
\question[10]
Why is there air?

\vspace{3in}

\question
What if there were no air?

\begin{parts}
\part[5]
Describe the effect on the balloon industry.
\vspace{2in}

\part[5]
Describe the effect on the aircraft industry.
\vspace{2in}
\end{parts}
\newpage
\end{verbatim}
If you want to equally distribute the space on a page among several
questions (or parts, or subparts, or subsubparts), you should
type\indc{stretch}\index{space!allocating}
%
\index{allocating space}
\begin{center}
  \verb"\vspace{\stretch{1}}"
\end{center}
after each of the questions (or parts, etc.\@) and end the page with
\verb"\newpage".  If you'd like one of those questions to receive
twice as much space as each of the others, then follow that one
question with \verb"\vspace{\stretch{2}}" and follow each of the
others with \verb"\vspace{\stretch{1}}".  (You can use any decimal
number for the \verb"\stretch" value.  For example,
\verb"\vspace{\stretch{1.5}}" will allocate 1.5 times the amount of
space allocated by \verb"\vspace{\stretch{1}}".)

In addition to leaving blank space, it's also possible to leave lined
space, dotted lined space, or an empty box.  For the full story, see
section~\ref{sec:LeaveSpace}.

%--------------------------------------------------------------------
\subsection{Headers and footers}
\label{sec:BasicHeadFoot}
\index{header|(}
\index{footer|(}

To produce headers and footers, you give the command\indc{pagestyle}
\begin{center}
  \verb"\pagestyle{headandfoot}"
\end{center}
in the preamble (i.e., after the \verb"\documentclass" command and
before the \verb"\begin{document}" command), followed by the commands
that say what will be in the headers and footers.  Headers and
footers each have three parts: One part on the left, one centered
part, and one part on the right, and any of these can be empty.

The headers and footers are determined by the commands%
\indc{firstpageheader}\indc{firstpagefooter}\indc{firstpageheadrule}%
\indc{firstpagefootrule}\indc{runningheader}\indc{runningfooter}%
\indc{runningheadrule}\indc{runningfootrule}
\begin{center}
  \begin{tabular}{l@{\qquad}l}
    \verb"\firstpageheader"& for the header on the first page,\\
    \verb"\firstpagefooter"& for the footer on the first page,\\
    \verb"\firstpageheadrule"& to put a horizontal rule in the head of
    the first page,\\
    \verb"\firstpagefootrule"& to put a horizontal rule in the foot of
    the first page,\\[0.5ex]
    \verb"\runningheader"& for the header after the first page,\\
    \verb"\runningfooter"& for the footer after the first page\\
    \verb"\runningheadrule"& to put a horizontal rule in the head
    after the first page, and\\
    \verb"\runningfootrule"& to put a horizontal rule in the foot
    after the first page.\\
  \end{tabular}
\end{center}

For example, to have the header
%
\samplehead{Math 115}{First Exam}{July 4, 1776}{}
%
on the first page, the header
%
\samplehead{Math 115}{First Exam, Page 2 of 5}{July 4, 1776}{\hrule}
%
on all pages after the first, and no footer on any page, give the
commands
\begin{verbatim}
\pagestyle{headandfoot}
\runningheadrule
\firstpageheader{Math 115}{First Exam}{July 4, 1776}
\runningheader{Math 115}
              {First Exam, Page \thepage\ of \numpages}
              {July 4, 1776}
\firstpagefooter{}{}{}
\runningfooter{}{}{}
\end{verbatim}

For another example, to have the header
%
\samplehead{Math 115}{First Exam}{July 4, 1776}{}
%
on the first page, no header on the pages after the first, no footer
on the first page, and the footer
%
\samplefoot{\hrule}{Math 115}{First Exam}{Page 2 of 5}
%
on all pages after the first, give the commands
\begin{verbatim}
\pagestyle{headandfoot}
\firstpageheader{Math 115}{First Exam}{July 4, 1776}
\runningheader{}{}{}
\firstpagefooter{}{}{}
\runningfooter{Math 115}{First Exam}{Page \thepage\ of \numpages}
\runningfootrule
\end{verbatim}

Sections~\ref{sec:beginexamples} through \ref{sec:endexamples} contain
many other examples of headers and footers, and the full explanation
of the commands for headers and footers is in
sections~\ref{sec:pagestyle}--\ref{sec:QuesSpan}.

\index{header|)}
\index{footer|)}

%---------------------------------------------------------------------
%--------------------------------------------------------------------
\section{The \texttt{\bs documentclass} command}
\label{sec:Documentclass}

To use the \verb"exam" document class, you should specify \verb"exam"
as the required argument to your
\verb"\documentclass"\indc{documentclass}
command, as in
\begin{center}
  \verb"\documentclass{exam}"
\end{center}
or, e.g., if you want to use 12~point type,
\begin{center}
  \verb"\documentclass[12pt]{exam}"
\end{center}
There are two possible optional arguments that are specific to the
\verb"exam" document class:
\verb"addpoints"\indcsub{documentclass}{options} and \verb"solutions".
Most documents that use the \verb"exam" document class should use the
option \verb"addpoints" (see section~\ref{sec:addpoints}); the option
\verb"answers" is used when you want to print solutions to the
questions (see section~\ref{sec:answers}).

%--------------------------------------------------------------------
\subsection{The option \texttt{addpoints}}
\label{sec:addpoints}

The \verb"\documentclass" option
\verb"addpoints"\indt{addpoints}\indcsub{documentclass}{options}, used
as in
\begin{center}
  \verb"\documentclass[addpoints]{exam}"
\end{center}
or
\begin{center}
  \verb"\documentclass[12pt,addpoints]{exam}"
\end{center}
enables the commands that add up the points (see sections
\ref{sec:addup}, \ref{sec:Count}, \ref{sec:pointsofq},
and~\ref{sec:pointsonp}) and that produce grading tables and point
tables (see sections \ref{sec:GradeTables} and~\ref{sec:PointTables}).
This is an option, rather than the default, mainly for the sake of
maintaining backward compatibility; most documents that use the
\verb"exam" document class should use this option.  The only reason
\emph{not} to use the \verb"addpoints" option is that this option
creates an error if you include any words in an argument that assigns
points to a question (see section~\ref{sec:points}).  If
\verb"addpoints" is not in effect you can put whatever you want in the
argument for the points, but if \verb"addpoints" \emph{is} in effect
then the argument that contains the points must contain only digits
and the command \verb"\half" (see section~\ref{sec:Half}).

You can also turn \verb"addpoints" on and off wherever you like in the
exam by using the commands\indc{addpoints}\indc{noaddpoints}
\begin{center}
  \begin{tabular}{l}
    \verb"\addpoints"\\
    \verb"\noaddpoints"
  \end{tabular}
\end{center}
Using the document class option \verb"addpoints" is equivalent to
giving the command \verb"\addpoints" at the beginning of the exam.

%--------------------------------------------------------------------
\subsection{The option \texttt{answers}}
\label{sec:answers}

If you will be typing solutions into the exam (see sections
\ref{sec:solutions}, \ref{sec:CorrectChoice},
and~\ref{sec:AnsShortAns}) and you'd like the solutions to be printed
on the next run of \LaTeX, you should include the document class
option \verb"answers"\indt{answers}\indcsub{documentclass}{options},
as in
\begin{center}
  \verb"\documentclass[answers]{exam}"
\end{center}
or
\begin{center}
  \verb"\documentclass[answers,12pt]{exam}"
\end{center}
or
\begin{center}
  \verb"\documentclass[12pt,addpoints,answers]{exam}"
\end{center}
Using the document class option \verb"answers" is equivalent to giving
the command \verb"\printanswers"\indc{printanswers} at the beginning
of the exam (see section~\ref{sec:solutions}).  The effect of this is
that
\begin{itemize}
\item the contents of the
  environments\indt{solution}\indt{solutionorbox}%
  \indt{solutionorlines}\indt{solutionordottedlines}
  \begin{center}
    \begin{tabular}{l}
      \verb"solution",\\
      \verb"solutionorbox",\\
      \verb"solutionorlines", and\\
      \verb"solutionordottedlines"
    \end{tabular}
  \end{center}
  (see section~\ref{sec:solutions}) will be printed on the next run of
  \LaTeX,
\item any choices of a \texttt{choices}\indt{choices},
  \texttt{oneparchoices}\indt{oneparchoices},
  \texttt{checkboxes}\indt{checkboxes}, or
  \texttt{oneparcheckboxes}\indt{oneparcheckboxes} environment (see
  section~\ref{sec:MulChc}) that are created using a
  \verb"\CorrectChoice"\indc{CorrectChoice} command (see
  section~\ref{sec:CorrectChoice}) instead of a
  \verb"\choice"\indc{choice} command will be printed with emphasis as
  determined by the argument of the most recent
  \verb"\CorrectChoiceEmphasis"\indc{CorrectChoiceEmphasis} command
  (which by default is \verb"\bfseries"), and
\item any \verb"\answerline"\indc{answerline} commands (see
  section~\ref{sec:ShortAns})\indcsub{answerline}{optional~argument}
  that include an optional argument containing the answer (see
  section~\ref{sec:AnsShortAns}) will have the answer printed on the
  answer line.
\end{itemize}


%---------------------------------------------------------------------
%--------------------------------------------------------------------
\section{Questions and points}
\label{sec:QuestionsPoints}

To create questions, you use a \verb"questions"\indt{questions}
%
environment\index{questions environment}\index{environment!questions}
(see section~\ref{sec:Questions}); each question is begun with a
\verb"\question"\indc{question} command, which takes an optional
argument to assign some number of points\index{points} to the question
(see section~\ref{sec:points}).  The questions are numbered
automatically.  Point values can include half points (see
section~\ref{sec:Half}).

If you want a question to have several parts, you use a
%
\verb"parts"\indt{parts}%
\index{parts environment}\index{environment!parts} environment (see
section~\ref{sec:parts}); each part is begun with a
\verb"\part"\indc{part} command, which takes an optional argument to
assign some number of points to the part (see
section~\ref{sec:points}).  The parts are numbered automatically.
Similarly, parts can have subparts, and subparts can have subsubparts
(see section~\ref{sec:parts}).

The points for a question (or part, or subpart, or subsubpart) appear
by default at the beginning of the question (or part, etc.), but there
are commands to have the points appear instead in either the left
margin or in the right margin (see section~\ref{sec:whereprint} and
section~\ref{sec:DrpPts}).  The formatting of the points can also be
customized (see sections \ref{sec:Boxed}, \ref{sec:pointname},
\ref{sec:margins}, and~\ref{sec:pointformat}).  It's also possible to
have questions (or parts, etc.) that are worth ``bonus points'', which
are added up separately from the non-bonus points (see
section~\ref{sec:Bonus}).


%--------------------------------------------------------------------
\subsection{Questions}
\label{sec:Questions}

To type the questions on the exam you use the
%
\verb"questions"\indt{questions}\index{questions environment}%
\index{environment!questions} environment.  Each question is then
begun with the command \verb"\question"\indc{question}, and the
questions are numbered automatically.

For example, if you type
\begin{verbatim}
\begin{questions}

\question
Why is there air?

\question

How much wood would a woodchuck chuck if a woodchuck could chuck
wood?

\question Compute $\displaystyle\int_0^1 x^2 \, dx$.
\end{questions}
\end{verbatim}
then you'll get
\begin{questions}

\question
Why is there air?

\question

How much wood would a woodchuck chuck if a woodchuck could chuck
wood?

\question Compute $\displaystyle\int_0^1 x^2 \, dx$.
\end{questions}

\bigskip

As the above example illustrates, you can leave blank lines
%
\index{blank lines} between the \verb"\question" command and the
actual beginning of the question, or before the first \verb"\question"
command in the environment, and they will be ignored.


%---------------------------------------------------------------------
\subsection{Questions with parts and subparts (and subsubparts)}
\label{sec:parts}

If you want a question to have several parts, then you use the
%
\verb"parts"\indt{parts}\index{parts environment}%
\index{environment!parts} environment.  For example, if you type
\begin{verbatim}
\begin{questions}
\question
Why is there air?

\question
What if there were no air?

\begin{parts}
\part
Describe the effect on the balloon industry.

\part
Describe the effect on the aircraft industry.
\end{parts}

\question

\begin{parts}
\part
Define the universe.  Give three examples.

\part
If the universe were to end, how would you know?

\end{parts}


\end{questions}
\end{verbatim}
then you'll get
\begin{questions}
\question
Why is there air?

\question
What if there were no air?

\begin{parts}
\part
Describe the effect on the balloon industry.

\part
Describe the effect on the aircraft industry.
\end{parts}

\question

\begin{parts}
\part
Define the universe.  Give three examples.

\part
If the universe were to end, how would you know?

\end{parts}


\end{questions}

\bigskip

The above example illustrates several things:
\begin{itemize}
\item Parts of a question should be put into a
  \verb"parts"\indt{parts}\index{parts environment}%
  \index{environment!parts} environment.
\item If a question begins with a \verb"parts" environment, then the
  first part will appear on the same line with the question number.
\item You can leave blank
  %
  lines\index{blank lines} before and after the \verb"\part" command,
  and they will be ignored.
\end{itemize}

\bigskip

There is also a \verb"subparts"\indt{subparts}%
\index{subparts environment}\index{environment!subparts} environment,
and even a \verb"subsubparts"\indt{subsubparts}%
\index{subsubparts environment}\index{environment!subsubparts}
environment, and they work just as you would expect.  For example, if
you type\indc{subpart}\indc{subsubpart}
\begin{verbatim}
\begin{questions}
\question
\begin{parts}
\part
What do you do with a drunken sailor?

\part
Is your answer different if it is before noon?
  
\end{parts}

\question
This is the second question.

\begin{parts}
\part
This is a part.

\part
This is also a part.

\begin{subparts}
\subpart
This is a subpart.

\subpart
This is a periscope.

\subpart
This is a pair of diving planes.

\subpart
\begin{subsubparts}
\subsubpart
This is a subsubpart.

\subsubpart
The lower surface of a diving plane?

\subsubpart
The ocean floor, perhaps?

\end{subsubparts}
\end{subparts}

\part
It's sad to be apart.

\end{parts}

\question
\begin{parts}
\part
\begin{subparts}
\subpart
This is a subpart.

\subpart
This is another subpart.
\end{subparts}
\part
This is another part.

\end{parts}


\end{questions}
\end{verbatim}
then you'll get
\begin{questions}
\question
\begin{parts}
\part
What do you do with a drunken sailor?

\part
Is your answer different if it is before noon?
  
\end{parts}

\question
This is the second question.

\begin{parts}
\part
This is a part.

\part
This is also a part.

\begin{subparts}
\subpart
This is a subpart.

\subpart
This is a periscope.

\subpart
This is a pair of diving planes.

\subpart
\begin{subsubparts}
\subsubpart
This is a subsubpart.

\subsubpart
The lower surface of a diving plane?

\subsubpart
The ocean floor, perhaps?

\end{subsubparts}
\end{subparts}

\part
It's sad to be apart.

\end{parts}

\question
\begin{parts}
\part
\begin{subparts}
\subpart
This is a subpart.

\subpart
This is another subpart.
\end{subparts}
\part
This is another part.

\end{parts}


\end{questions}

The numbers and letters used to number questions, parts, subparts, and
subsubparts can be customized; see section~\ref{sec:CustNumb}.

%---------------------------------------------------------------------
\subsection{Point values for the questions}
\label{sec:points}
\indcsub{question}{optional argument}
\indcsub{part}{optional argument}
\indcsub{subpart}{optional argument}
\indcsub{subsubpart}{optional argument}
\index{points|(}

Each of the commands \verb"\question", \verb"\part", \verb"\subpart",
and \verb"\subsubpart" takes an optional argument, which is the number
of points for that question, part, subpart, or subsubpart.  The
default setup is that
\begin{itemize}
\item the point value is printed at the beginning of the question (or
  part, or subpart, or subsubpart) (this can be changed to printing in
  either the left or right margins on the first line of the question
  or in the right margin at the end of the question; see
  section~\ref{sec:whereprint} and section~\ref{sec:DrpPts}), and
\item the point value is enclosed in parentheses (this can be changed
  to either square brackets or a box; see section~\ref{sec:Boxed}).
\end{itemize}
Both of these things can be further customized; see
section~\ref{sec:qformat} and section~\ref{sec:pointformat}.

%--------------------------------------------------------------------
\subsubsection{Where the points will be printed}
\label{sec:whereprint}
\index{points!where printed|(}

The default is that the point value will be inserted at the beginning
of the question (or part, or subpart, or subsubpart) in parentheses,
but
\begin{itemize}
\item the command \verb"\pointsinmargin"\indc{pointsinmargin} will
  cause the point values to be set in the left margin,
\item the command
  \verb"\pointsinrightmargin"\indc{pointsinrightmargin} will cause the
  point values to be set in the right margin, and
\item the commands \verb"\nopointsinmargin"\indc{nopointsinmargin} and
  \verb"\nopointsinrightmargin"\indc{nopointsinrightmargin} are
  equivalent, and either of them will revert to the default situation.
\end{itemize}
All of these print the point values on the first line of the question
(or part, or subpart, or subsubpart).  There is also a way to print
the point values on the \emph{last} line of the question (or part,
etc.); for this, see section~\ref{sec:DrpPts}.

\medbreak

For example, if you type
\begin{verbatim}
\begin{questions}
\question[20]
Why is there air?


\question
What if there were no air?

\begin{parts}
\part[10]
Describe the effect on the balloon industry.

\part[10]
Describe the effect on the aircraft industry.
\end{parts}

\end{questions}
\end{verbatim}
then, with the default setup, you'll get
\nopointsinmargin
\begin{questions}
\question[20]
Why is there air?


\question
What if there were no air?

\begin{parts}
\part[10]
Describe the effect on the balloon industry.

\part[10]
Describe the effect on the aircraft industry.
\end{parts}

\end{questions}
(To change the word ``points'', see the commands described in
section~\ref{sec:pointname}.)

\bigskip If you give the command
\verb"\pointsinmargin"\indc{pointsinmargin}, then the above input will
produce instead \pointsinmargin
\begin{questions}
\question[20]
Why is there air?


\question
What if there were no air?

\begin{parts}
\part[10]
Describe the effect on the balloon industry.

\part[10]
Describe the effect on the aircraft industry.
\end{parts}

\end{questions}
(To have a word (e.g., ``points'', or ``\%'') inside the parentheses
with the point values, see the
\verb"\marginpointname"\indc{marginpointname} command in
section~\ref{sec:pointname}.)


\bigskip 

If you give the command
\verb"\pointsinrightmargin"\indc{pointsinrightmargin}, then the above
input will produce instead
%
\pointsinrightmargin
\begin{questions}
\question[20]
Why is there air?


\question
What if there were no air?

\begin{parts}
\part[10]
Describe the effect on the balloon industry.

\part[10]
Describe the effect on the aircraft industry.
\end{parts}

\end{questions}

%--------------------------------------------------------------------
\subsubsection{Printing the points at the end of the question}
\label{sec:DrpPts}
\index{points!at the end of the question|(}
\indcstart{pointsdroppedatright}
\indcstart{droppoints}


Instead of having the point value of a question (or part, or subpart,
or subsubpart) printed on the first line of a question, you can have
it printed in the right margin opposite the \emph{last} line, or even
opposite a blank line following the paragraph.  This is done with the
\begin{center}
  \verb"\pointsdroppedatright"
  \quad and \quad
  \verb"\droppoints"
\end{center}
commands.

If you give the command
\begin{center}
  \verb"\pointsdroppedatright"
\end{center}
then point values are not printed until you give the command
\verb"\droppoints" (and they're not printed at all if you don't give
the command \verb"\droppoints").  The only exception to this rule is
that if you've given a \verb"\qformat"\indc{qformat} command (see
section~\ref{sec:qformat}), then question number lines are printed as
specified by the argument to the \verb"\qformat" command even if you
have given the command
\verb"\pointsdroppedatright"

The command \verb"\droppoints" should be given only at the end of a
paragraph or between paragraphs; if you give it within a paragraph, it
causes the paragraph to end.  \verb"\droppoints" prints the point
value in the right margin, formatted as it is when you give the
command \verb"\pointsinrightmargin"\indc{pointsinrightmargin}, except
that the points appear opposite the last line of the paragraph (or, if
the command \verb"\droppoints" is given between paragraphs, then
additional vertical space is left between the paragraphs and the
points are printed opposite the blank space).  The formatting can be
changed by giving the commands
\verb"\bracketedpoints"\indc{bracketedpoints},
\verb"\boxedpoints"\indc{boxedpoints}, or
\verb"\marginpointname"\indc{marginpointname} (see
section~\ref{sec:pointname}) just as if \verb"\pointsinrightmargin"
were in effect.  The command \verb"\droppoints" actually works this
way even if one of the commands
\begin{center}
  \verb"\nopointsinmargin",\quad \verb"\pointsinmargin",\quad or\quad
  \verb"\pointsinrightmargin"
\end{center}
is in effect, but if you use it that way the points will appear twice
on the page, which is probably not what you want.


For example, if you give the command \verb"\pointsdroppedatright" and
then type
\begin{verbatim}
\begin{questions}
\question[10]
Describe the implications of the Michelson-Morley ink drop experiment
for the use of punctuation marks such as colons and semicolons that
require multiple drops of ink.
\droppoints

\question[10]
Prove that the second dual of a finite dimensional real vector space
is naturally isomorphic to the given vector space, except when the
weapon chosen is a single shot pistol.\droppoints

\end{questions}
\end{verbatim}
then you'll get
\pointsdroppedatright
\begin{questions}
\question[10]
Describe the implications of the Michelson-Morley ink drop experiment
for the use of punctuation marks such as colons and semicolons that
require multiple drops of ink.
\droppoints

\question[10]
Prove that the second dual of a finite dimensional real vector space
is naturally isomorphic to the given vector space, except when the
weapon chosen is a single shot pistol.\droppoints

\end{questions}
and if you type
\begin{verbatim}
\begin{questions}
\question[10]
Describe the implications of the Michelson-Morley ink drop experiment
for the use of punctuation marks such as colons and semicolons that
require multiple drops of ink.

\droppoints

\question[10]
Prove that the second dual of a finite dimensional real vector space
is naturally isomorphic to the given vector space, except when the
weapon chosen is a single shot pistol.

\droppoints
\end{questions}
\end{verbatim}
then you'll get
\begin{questions}
\question[10]
Describe the implications of the Michelson-Morley ink drop experiment
for the use of punctuation marks such as colons and semicolons that
require multiple drops of ink.

\droppoints

\question[10]
Prove that the second dual of a finite dimensional real vector space
is naturally isomorphic to the given vector space, except when the
weapon chosen is a single shot pistol.

\droppoints
\end{questions}
\nopointsinmargin

\indcstop{droppoints}
\indcstop{pointsdroppedatright}
\index{points!at the end of the question|)}
\index{points!where printed|)}

%--------------------------------------------------------------------
\subsubsection{Half points}
\label{sec:Half}
\index{points!half|(}
\index{half points|(}

The point values of questions, parts, subparts, and subsubparts can
include half points.  To specify half points, you either
type\indc{half}
\begin{center}
  \verb"\half"
\end{center}
immediately following the integer part of the point value or just type
\verb"\half" as the entire point value. That is, the valid point
values and their (default) printed appearances are:\indc{half}
\begin{center}
  \begin{tabular}{@{\extracolsep{0.1in}}rccccccc}
    Typed:& \verb"0"& \verb"\half"& \verb"1"& \verb"1\half"& \verb"2"&
      \verb"2\half"& $\cdots$\\[1ex]
    Printed:& 0& \half& 1& 1\half& 2& 2\half& $\cdots$
  \end{tabular}
\end{center}
If you prefer, you can cause the fraction \half{} to be printed as
\makeatletter
\horiz@half.
\makeatother
To do this, you give the
command\indc{usehorizontalhalf}%
\index{customization!\texttt{\char`\\half}}%
\indcsub{half}{customization}
\begin{center}
  \verb"\usehorizontalhalf"
\end{center}
after which the printed appearance of point values will be
\usehorizontalhalf
\begin{center}
  \begin{tabular}{@{\extracolsep{0.1in}}rccccccc}
    Typed:& \verb"0"& \verb"\half"& \verb"1"& \verb"1\half"& \verb"2"&
      \verb"2\half"& $\cdots$\\[1ex]
    Printed:& 0& \half& 1& 1\half& 2& 2\half& $\cdots$
  \end{tabular}
\end{center}
If you want to return to using the default appearance, you can do that
by giving the command\indc{useslantedhalf}
\begin{center}
  \verb"\useslantedhalf"
\end{center}
\useslantedhalf

\index{half points|)}
\index{points!half|)}


%--------------------------------------------------------------------
\subsubsection{Surrounding the points: Parentheses, brackets, or a box}
\label{sec:Boxed}
\index{customization!points|(}
\index{points!customization|(}

If you prefer having the points enclosed in brackets instead of in
parentheses, give the command\indc{bracketedpoints}
\begin{center}
  \verb"\bracketedpoints"
\end{center}
For example, if you give the command \verb"\bracketedpoints", then the
questions typed above will produce
\nopointsinmargin
\bracketedpoints
\begin{questions}
\question[20]
Why is there air?


\question
What if there were no air?

\begin{parts}
\part[10]
Describe the effect on the balloon industry.

\part[10]
Describe the effect on the aircraft industry.
\end{parts}

\end{questions}


If you prefer having the points enclosed in a box instead of in
parentheses, give the command\indc{boxedpoints}
\begin{center}
  \verb"\boxedpoints"
\end{center}
For example, if you give the command \verb"\boxedpoints", then the
questions typed above will produce
\nopointsinmargin
\boxedpoints
\begin{questions}
\question[20]
Why is there air?


\question
What if there were no air?

\begin{parts}
\part[10]
Describe the effect on the balloon industry.

\part[10]
Describe the effect on the aircraft industry.
\end{parts}

\end{questions}
If you give the commands \verb"\boxedpoints"\indc{boxedpoints} and
\verb"\pointsinmargin"\indc{pointsinmargin}, then the above questions
will produce
%
\boxedpoints \pointsinmargin
\begin{questions}
\question[20]
Why is there air?


\question
What if there were no air?

\begin{parts}
\part[10]
Describe the effect on the balloon industry.

\part[10]
Describe the effect on the aircraft industry.
\end{parts}

\end{questions}
If you give the commands \verb"\bracketedpoints"\indc{bracketedpoints}
and \verb"\pointsinrightmargin"\indc{pointsinrightmargin}, then the
above questions will produce
%
\bracketedpoints
\pointsinrightmargin
\begin{questions}
\question[20]
Why is there air?


\question
What if there were no air?

\begin{parts}
\part[10]
Describe the effect on the balloon industry.

\part[10]
Describe the effect on the aircraft industry.
\end{parts}

\end{questions}
Other combinations of these commands will produce similar effects.

If you want to switch back and forth between formats during the exam,
you can do so by giving one of the
commands\indc{boxedpoints}\indc{bracketedpoints}\indc{noboxedpoints}
\indc{nobracketedpoints}
\begin{center}
  \begin{tabular}{l}
    \verb"\boxedpoints"\\
    \verb"\bracketedpoints"\\
    \verb"\noboxedpoints"\\
    \verb"\nobracketedpoints"
  \end{tabular}
\end{center}
whenever you want to switch.  (The commands
\verb"\nobracketedpoints" and \verb"\noboxedpoints" are equivalent;
they both return to the default of putting parentheses around the
points.)

If you want some text (e.g., `` points'', or ``\%'') automatically
inserted along with the point values when using either
\verb"\pointsinmargin" or \verb"\pointsinrightmargin", see the command
\verb"\marginpointname" in section~\ref{sec:pointname}.



%---------------------------------------------------------------------
\subsubsection{Using a substitute for the word ``points''}
\label{sec:pointname}
\index{points!substitute words|(}
\index{points!language|(}

With the default setup, the number of points is inserted before the
text of the question followed by a single space and the word ``point''
if the number of points is either \half{} or 1 and by the word
``points'' otherwise.  That is, the default is
\verb"\nopointsinmargin"\indc{nopointsinmargin} and
\verb"\noboxedpoints"\indc{noboxedpoints} (see sections
\ref{sec:whereprint} and \ref{sec:Boxed}), and if you type
%
\nopointsinmargin
\noboxedpoints
\begin{verbatim}
\begin{questions}
  \question[1]
  This is a short question.

  \question[5]
  This is a longer question, requiring more thought.
\end{questions}
\end{verbatim}
then you'll get
\begin{questions}
  \question[1]
  This is a short question.

  \question[5]
  This is a longer question, requiring more thought.
\end{questions}
The way this is achieved is:
\begin{itemize}
\item There is a command\indc{points}
  \begin{center}
    \verb"\points"
  \end{center}
  whose default definition expands to the word `point' if the number
  of points is either \half{} or 1 and to `points' otherwise.  (This
  behavior can be modified by the command
  \verb"\pointpoints"\indc{pointpoints}; see below.)
\item There is a command\indc{pointname}
  \begin{center}
    \verb"\pointname{Text}"
  \end{center}
  that causes `Text' to be inserted following the number of points of
  a question, part, subpart, or subsubpart.  The default setup is the
  result of the command\indc{pointname}\indc{points}
  \begin{center}
    \verb"\pointname{ \points}"
  \end{center}
  (Note the single space before the command \verb"\points".)
\end{itemize}

There is also a command\indc{pointpoints}
\begin{center}
  \verb"\pointpoints{SingularText}{PluralText}"
\end{center}
that changes the behavior of the \verb"\points" command so that it
expands to `SingularText' if the number of points is \half{} or 1 and
to `PluralText' otherwise.  (The default is the result of the command
\verb"\pointpoints{point}{points}"\indc{pointpoints}.)  Thus, if you
give the command
\begin{center}
  \verb"\pointpoints{Punkt}{Punkte}"
\end{center}
then typing the questions above would result in
\pointpoints{Punkt}{Punkte}
\begin{questions}
  \question[1]
  This is a short question.

  \question[5]
  This is a longer question, requiring more thought.
\end{questions}
There is also a command\indc{marksnotpoints}\index{marks}
\begin{center}
  \verb"\marksnotpoints"
\end{center}
that is equivalent to the two
commands\indc{pointpoints}\indc{bonuspointpoints}\index{marks}
\begin{center}
  \begin{tabular}{l}
    \verb"\pointpoints{mark}{marks}"\\
    \verb"\bonuspointpoints{mark (bonus)}{marks (bonus)}"
  \end{tabular}
\end{center}
(the command \verb"\bonuspointpoints" is described in
section~\ref{sec:Bonus}).  That command would cause the above
questions to be printed as
%
\pointpoints{mark}{marks}
\begin{questions}
  \question[1]
  This is a short question.

  \question[5]
  This is a longer question, requiring more thought.
\end{questions}

\bigskip

For another example, if you give the command\indc{pointname}
\begin{center}
\verb"\pointname{\%}"
\end{center}
then the above questions would be printed as
%
\pointname{\%}
\nopointsinmargin
\noboxedpoints
\begin{questions}
  \question[1]
  This is a short question.

  \question[5]
  This is a longer question, requiring more thought.
\end{questions}
\pointname{ \points}


\subsubsection*{Using \texttt{\char`\\marginpointname}}

Similarly, the command \verb"\marginpointname"\indc{marginpointname}
can be used to affect the text set with the number of points when
\verb"\pointsinmargin"\indc{pointsinmargin} or
\verb"\pointsinrightmargin"\indc{pointsinrightmargin} is in effect.
For example, if you give the commands
\begin{center}
  \begin{tabular}{l}
    \verb"\pointsinmargin"\\
    \verb"\marginpointname{\%}"
  \end{tabular}
\end{center}
and then type
\begin{verbatim}
\question[25]
Where, oh where, has my little dog gone?
\end{verbatim}
then you'll get
\pointsinmargin
\noboxedpoints
\marginpointname{\%}
\begin{questions}
\question[25]
Where, oh where, has my little dog gone?
\end{questions}
unless, e.g., you've also given the command
\verb"\boxedpoints"\indc{boxedpoints} (see section~\ref{sec:Boxed}),
in which case you'll get
%
\pointsinmargin
\boxedpoints
\marginpointname{\%}
\begin{questions}
\question[25]
Where, oh where, has my little dog gone?
\end{questions}
If you give the commands
\begin{center}
  \begin{tabular}{l}
    \verb"\marksnotpoints"\\
    \verb"\marginpointname{ \points}"
  \end{tabular}
\end{center}
then the above will print
\noboxedpoints
\marginpointname{ \points}
\marksnotpoints
\begin{questions}
\question[25]
Where, oh where, has my little dog gone?
\end{questions}
\nopointsinmargin
\marginpointname{}
\pointpoints{point}{points}
\bonuspointpoints{point (bonus)}{points (bonus)}

Using the default is equivalent to giving the
command\indc{marginpointname}
\begin{center}
  \verb"\marginpointname{}"
\end{center}


%--------------------------------------------------------------------
\subsubsection{Using \texttt{\bs marginpointname} and enlarging the
  margins}
\label{sec:margins}

The default margins are not very large, and so if you use
\verb"\marginpointname"\indc{marginpointname} (see
section~\ref{sec:pointname}) to place any words along with the point
values in the margin, you may run out of room.  The solution to this
problem is to enlarge the margins\index{margins!enlarge} by using the
\verb"\extrawidth"\indc{extrawidth} command with a negative argument
to \emph{decrease} the width of the text area, and thus
\emph{increase} the width of the margins.  For example, the commands
\begin{center}
  \begin{tabular}{ll}
    \verb"\extrawidth{-1in}"\\
    \verb"\marginpointname{ \points}"
  \end{tabular}
\end{center}
will increase each of the left and right margins by one half inch,
which will easily leave room for the word ``points''.  (See
section~\ref{sec:pagesize} for a discussion of the \verb"\extrawidth"
command.)

%--------------------------------------------------------------------
\subsubsection{Completely customizing the points}
\label{sec:pointformat}

If the customizations described in section~\ref{sec:Boxed} and
section~\ref{sec:pointname} (using \verb"\bracketedpoints",
\verb"\boxedpoints", \verb"\pointpoints", \verb"\pointname", and
\verb"\marginpointname") aren't sufficient, and you don't want to use
the \verb"\qformat" command to create a customized question number
line (see section~\ref{sec:qformat}), you can completely customize the
way the points are printed by using the
commands\indc{pointformat}\indc{bonuspointformat}
\begin{center}
  \begin{tabular}{l}
    \verb"\pointformat"\\
    \verb"\bonuspointformat"
\end{tabular}
\end{center}
The \verb"\pointformat" command controls the printing of ``regular''
points (i.e., non-bonus points) and the \verb"\bonuspointformat"
command controls the printing of bonus points (see
section~\ref{sec:Bonus}).

Each of the commands \verb"\pointformat" and \verb"\bonuspointformat"
takes one argument, and that argument is what actually appears on the
page whether the points are being printed at the beginning of the
question (the default), in the left margin (because of
\verb"\pointsinmargin"), in the right margin (because of
\verb"\pointsinrightmargin"), or dropped in the right margin at the
end of the question (because of \verb"\pointsdroppedatright").  The
arguments to \verb"\pointformat" and \verb"\bonuspointformat" should
contain either the command\indc{thepoints}
\begin{center}
  \verb"\thepoints"
\end{center}
(which for non-bonus points prints the number of points followed by
the argument to the last \verb"\pointname"\indc{pointname} command and
for bonus points prints the number of bonus points followed by the
argument to the last \verb"\bonuspointname"\indc{bonuspointname}
command; see section~\ref{sec:pointname}) or the
command\indc{themarginpoints}
\begin{center}
  \verb"\themarginpoints"
\end{center}
(which for non-bonus points prints the number of points followed by
the argument to the last \verb"\marginpointname"\indc{marginpointname}
command and for bonus points prints the number of bonus points
followed by the argument to the last
\verb"\marginbonuspointname"\indc{marginbonuspointname} command; see
section~\ref{sec:pointname}).  If the argument doesn't contain either
\verb"\thepoints" or \verb"\themarginpoints", then the number of
points will not appear.

For example, the command \verb"\bracketedpoints" is equivalent to
either\indc{bracketedpoints}\indc{pointformat}%
\indc{bonuspointformat}\indc{thepoints}
\begin{center}
  \begin{tabular}{l}
    \verb"\pointformat{[\thepoints]}"\\
    \verb"\bonuspointformat{[\thepoints]}"
  \end{tabular}
\end{center}
if the points are being printed at the beginning of the question, or
\indc{pointformat}\indc{bonuspointformat}\indc{themarginpoints}
\begin{center}
  \begin{tabular}{l}
    \verb"\pointformat{[\themarginpoints]}"\\
    \verb"\bonuspointformat{[\themarginpoints]}"
  \end{tabular}
\end{center}
if the points are being printed in one of the margins.

For another example, the command \verb"\boxedpoints" is equivalent to
either\indc{pointformat}\indc{bonuspointformat}\indc{thepoints}
\begin{center}
  \begin{tabular}{l}
    \verb"\pointformat{\fbox{\thepoints}}"\\
    \verb"\bonuspointformat{\fbox{\thepoints}}"
  \end{tabular}
\end{center}
if the points are being printed at the beginning of the question, or
\indc{pointformat}\indc{bonuspointformat}\indc{themarginpoints}
\begin{center}
  \begin{tabular}{l}
    \verb"\pointformat{\fbox{\themarginpoints}}"\\
    \verb"\bonuspointformat{\fbox{\themarginpoints}}"
  \end{tabular}
\end{center}
if the points are being printed in one of the margins.

For another example, the
commands\indc{marginpointname}\indc{pointsinmargin}\indc{pointformat}%
\indc{pointsinmargin}
\begin{verbatim}
\begin{questions}
  \pointsinmargin
  \marginpointname{\%}
  \pointformat{\slshape (\themarginpoints)}
  \question[3] What is the point?  
\end{questions}
\end{verbatim}
will produce
\begin{questions}
  \pointsinmargin
  \marginpointname{\%}
  \pointformat{\slshape (\themarginpoints)}
  \question[3] What is the point?  
\end{questions}

For another example, the
commands\indc{marginpointname}\indc{pointsinmargin}\indc{pointformat}%
\indc{pointsinrightmargin}
\begin{verbatim}
\begin{questions}
  \pointsinrightmargin
  \marginpointname{\%}
  \pointformat{\fbox{\bfseries\boldmath\themarginpoints}}
  \question[3\half] What is the point?  
\end{questions}
\end{verbatim}
will produce
\begin{questions}
  \pointsinrightmargin
  \marginpointname{\%}
  \pointformat{\fbox{\bfseries\boldmath\themarginpoints}}
  \question[3\half] What is the point?  
\end{questions}
\nopointsinmargin (Note: The commands \verb"\bfseries" and
\verb"\boldmath"\indc{boldmath} are both needed here because the
\half{} printed when you use half points is printed in math mode while
the rest of the points are printed in text mode.)

For another example, the commands\indc{pointformat}\indc{boldmath}
\begin{verbatim}
\begin{questions}
  \pointname{}
  \pointformat{\bfseries\boldmath(Points for this problem: \thepoints)}
  \question[3\half] What is the point?  
\end{questions}
\end{verbatim}
will produce
\begin{questions}
  \pointname{}
  \pointformat{\bfseries\boldmath(Points for this problem: \thepoints)}
  \question[3\half] What is the point?  
\end{questions}
% Put things back to the default situation, so we don't mess up
% later examples:
\noboxedpoints
\nopointsinmargin
\pointname{ \points}

\index{points!language|)}
\index{points!substitute words|)}
\index{points!customization|)}
\index{customization!points|)}

%---------------------------------------------------------------------
\subsubsection{Questions that begin with a parts environment}
\index{questions!that begin with a parts environment}
\index{points!for both question and part}

If a question begins with a parts environment, then the question
number and the number of the first part in the parts environment will
be printed on the same line (see section~\ref{sec:parts}).  Thus, if
points are assigned to both the question and to the first part, both
of those point values will be printed on the same line.  (Note: This
was not true in versions of the \verb"exam" document class earlier
than 2.3.  In those versions, only the later of the two point values
would actually be printed.)  If \verb"\pointsinmargin" or
\verb"\pointsinrightmargin" are in effect (see
section~\ref{sec:whereprint}), then the two point values will be
printed on top of each other, which is probably not what you want.
The same problem arises if a part begins with a subparts environment
or if a subpart begins with a subsubparts environment.

If you want to assign points to both a question and to each part of a
parts environment, and if the first part will begin on the same line
as the question, the solution is to either use the default location
(i.e., \verb"\nopointsinmargin") or to put some of the points in the
margin and others not.  For example, if you type
\begin{verbatim}
\begin{questions}
  \pointsinmargin
  \noaddpoints
  \question[10]
  \begin{parts}
    \nopointsinmargin
    \addpoints
    \part[5] Who put the ``bop'' in the ``bop, sh-bop, sh-bop''?
    \part[5] Who put the ``ram'' in the ``rama, rama ding-dong''?
  \end{parts}
\end{questions}
\end{verbatim}
then you'll get
\begin{questions}
  \pointsinmargin
  \noaddpoints
  \question[10]
  \begin{parts}
    \nopointsinmargin
    \addpoints
    \part[5] Who put the ``bop'' in the ``bop, sh-bop, sh-bop''?
    \part[5] Who put the ``ram'' in the ``rama, rama ding-dong''?
  \end{parts}
\end{questions}




%--------------------------------------------------------------------
\subsubsection{Adding up the points for a question}
\label{sec:addup}
\indt{addpoints}
\indc{addpoints}

If you use the document class option \verb"addpoints" (as in
\verb"\documentclass[addpoints]{exam}") or give the command
\verb"\addpoints" (see section~\ref{sec:addpoints}), then you can use
the command\indc{droptotalpoints}
\begin{center}
  \verb"\droptotalpoints"
\end{center}
to put into the right margin the total number of points for the
current question (including the points for all parts, subparts, and
subsubparts).  (Section~\ref{sec:adding} describes a different way to
print the total number of points for a question.)  The command
\verb"\droptotalpoints" should be used only at the end of a paragraph
or between paragraphs; if you use it within a paragraph, it causes the
paragraph to end.  \verb"\droptotalpoints" prints the total points for
the current question in the right margin, formatted by default as:
%
\par\nobreak\medskip
\makeatletter
\leavevmode\nobreak\hfill
\rlap{%\hskip\rightmargin  % Defined by the list environment
  \hskip\@rightmargin % Defined by exam.cls
  \hskip-\rightpointsmargin
  \llap{Total for Question 2: 25}%
}% rlap
\makeatother
\par\medskip\noindent
%
(Note: The number of points is followed by the argument to the most
recently executed \verb"\marginpointname"\indc{marginpointname}
command (see section~\ref{sec:pointname}), which by default is empty.)

If the current question is worth any bonus points (see
section~\ref{sec:Bonus}), then those bonus points will not be counted
by the \verb"\droptotalpoints" command, but they will be counted by
the\indc{droptotalbonuspoints}
\begin{center}
  \verb"\droptotalbonuspoints"
\end{center}
command, which puts into the right margin the total number of bonus
points for the current question, formatted by default as:
\par\nobreak\medskip
\makeatletter
\leavevmode\nobreak\hfill
\rlap{%\hskip\rightmargin  % Defined by the list environment
  \hskip\@rightmargin % Defined by exam.cls
  \hskip-\rightpointsmargin
  \llap{Total for Question 2: 10 (bonus)}%
}% rlap
\makeatother
\par\medskip\noindent
(Note: The number of bonus points is followed by the argument to the
most recently executed
\verb"\marginbonuspointname"\indc{marginbonuspointname} command (see
section~\ref{sec:Bonus}), which by default is `` (bonus)''.)

You can change the format used by the
\verb"\droptotalpoints"\indc{droptotalpoints} command with
the\indc{totalformat}
\begin{center}
  \verb"\totalformat"
\end{center}
command.  It takes one argument, and that argument becomes the command
to print the total points, right justified a distance of
\verb"\rightpointsmargin"\indc{rightpointsmargin} from the right edge
of the paper.  The argument can contain the
command\indc{totalpoints}\indc{thequestion}
\begin{center}
  \begin{tabular}{ll}
    \verb"\totalpoints"& to print the number of points, and\\
    \verb"\thequestion"& to print the question number.
  \end{tabular}
\end{center}
(\verb"\totalpoints" expands to
\verb"\pointsofquestion{\arabic{question}}"\indc{pointsofquestion};
see section~\ref{sec:pointsofq} for a discussion of the
\verb"\pointsofquestion" command.)  For example, if you give the
command\indc{totalformat}\indc{thequestion}\indc{totalpoints}
\begin{center}
  \verb"\totalformat{Question \thequestion: \totalpoints}"
\end{center}
then the total number of points will be printed in the format
%
\par\nobreak\medskip
\makeatletter
\leavevmode\nobreak\hfill
\rlap{%\hskip\rightmargin  % Defined by the list environment
  \hskip\@rightmargin % Defined by exam.cls
  \hskip-\rightpointsmargin
  \llap{Question 2: 25}%
}% rlap
\makeatother
\par\medskip\noindent
%
and if you give the command\indc{totalformat}\indc{totalpoints}
\begin{center}
\verb"\totalformat{\fbox{Total: \totalpoints}}"
\end{center}
then the total number of points will be printed as
%
\par\nobreak\medskip
\makeatletter
\leavevmode\nobreak\hfill
\rlap{%\hskip\rightmargin  % Defined by the list environment
  \hskip\@rightmargin % Defined by exam.cls
  \hskip-\rightpointsmargin
  \llap{\fbox{Total: 25}}%
}% rlap
\makeatother
\par\medskip\noindent

Thus, if you've given the
commands\indc{addpoints}\indc{bracketedpoints}\indc{pointsdroppedatright}
\begin{center}
  \begin{tabular}{l}
    \verb"\addpoints"\\
    \verb"\bracketedpoints"\\
    \verb"\pointsdroppedatright"
  \end{tabular}
\end{center}
and you type
\begin{verbatim}
\question
\begin{parts}
\part[10]
In no more than one paragraph, explain why the earth is round.
\droppoints
\part[10]
What changes to the van Allen radiation belt are needed to
make the earth into a regular icosahedron?
\droppoints
\end{parts}
\droptotalpoints
\end{verbatim}
then you'll get
\addpoints
\pointsdroppedatright
\bracketedpoints
\begin{questions}
\question
\begin{parts}
\part[10]
In no more than one paragraph, explain why the earth is round.
\droppoints
\part[10]
What changes to the van Allen radiation belt are needed to
make the earth into a regular icosahedron?
\droppoints
\end{parts}
  \leavevmode\unskip\nobreak\hfill
\makeatletter
  \rlap{\hskip\rightmargin  % Defined by the list environment
        \hskip\@rightmargin % Defined by exam.cls
        \hskip-\rightpointsmargin
        \llap{Total for Question 1: 20}%
  }% rlap
\makeatother
  \par
%\droptotalpoints
\end{questions}
\noaddpoints \nopointsinmargin \nobracketedpoints If you've also given
the command\indc{totalformat}\indc{totalpoints}\indc{thequestion}
\begin{center}
  \verb"\totalformat{Total for Question \thequestion: [\totalpoints]}"
\end{center}
then you'll get
\addpoints
\pointsdroppedatright
\bracketedpoints
\begin{questions}
\question
\begin{parts}
\part[10]
In no more than one paragraph, explain why the earth is round.
\droppoints
\part[10]
What changes to the van Allen radiation belt are needed to
make the earth into a regular icosahedron?
\droppoints
\end{parts}
%\droptotalpoints
  \leavevmode\unskip\nobreak\hfill
\makeatletter
  \rlap{\hskip\rightmargin  % Defined by the list environment
        \hskip\@rightmargin % Defined by exam.cls
        \hskip-\rightpointsmargin
        \llap{Total for Question 1: [20]}%
  }% rlap
\makeatother
  \par
\end{questions}
\noaddpoints
\nopointsinmargin
\nobracketedpoints

To change the format used by the
\verb"\droptotalbonuspoints"\indc{droptotalbonuspoints} command, you
use the command\indc{bonustotalformat}
\begin{center}
  \verb"\bonustotalformat"
\end{center}
which is analogous to the \verb"\totalformat" command.  The argument
to \verb"\bonustotalformat" can contain the
command\indc{totalbonuspoints}\indc{thequestion}
\begin{center}
  \begin{tabular}{ll}
    \verb"\totalbonuspoints"& to print the number of bonus points, and\\
    \verb"\thequestion"& to print the question number.
  \end{tabular}
\end{center}
(\verb"\totalbonuspoints" expands to
\verb"\bonuspointsofquestion{\arabic{question}}"; see
section~\ref{sec:BonusTables} for a discussion of the
\verb"\bonuspointsofquestion"\indc{bonuspointsofquestion} command.)

\index{points|)}

%--------------------------------------------------------------------
\subsubsection{Margin sizes when using \texttt{\bs pointsinmargin} and
  \texttt{\bs pointsinrightmargin}}
\index{margins!enlarge|(}

\begin{itemize}
\item If you give the command
  \verb"\pointsinmargin"\indc{pointsinmargin}, then the points are
  printed right justified in the left margin, with the right edge a
  distance of \verb"\marginpointssep"\indc{marginpointssep} from the
  left edge of the text area.  The default value of
  \verb"\marginpointssep" is the result of the command
  \begin{center}
    \verb"\setlength{\marginpointssep}{5pt}"
  \end{center}
  and you can change it by giving a new \verb"\setlength" command.
  
\item If you give the command
  \verb"\pointsinrightmargin"\indc{pointsinrightmargin}, then the
  points are printed right justified in the right margin, with the
  right edge a distance of
  \verb"\rightpointsmargin"\indc{rightpointsmargin} from the right
  edge of the paper.  The default value of \verb"\rightpointsmargin"
  is the result of the command
  \begin{center}
    \verb"\setlength{\rightpointsmargin}{1cm}"
  \end{center}
  and you can change it by giving a new \verb"\setlength" command.
\end{itemize}

You can change the width of the text area (and, thus, change the width
of the margins) with the \verb"\extrawidth"\indc{extrawidth} command
(see section~\ref{sec:margins}).

\index{margins!enlarge|)}

%--------------------------------------------------------------------
\subsection{Custom question number lines: The \texttt{\bs qformat}
  command}
\label{sec:qformat}
\indc{qformat}
\index{questions!customization|(}

The default setup is for the question number to appear in the left
margin and for the text of the question to begin on that line.  It's
possible to change this so that the text of the question starts on the
line following the question number, and the format of the line
containing the question number is determined by the user.  This is
done using the \verb"\qformat" command.  There is also a
\verb"\noqformat"\indc{noqformat} command, which reverts to the
default setup.

To use the \verb"\qformat"\indc{qformat} command, you give the command
\begin{center}
  \verb"\qformat{Format specification}"
\end{center}
where \texttt{Format specification} must contain some stretchability
(e.g., at least one \verb"\hfill" or \verb"\dotfill" or
\verb"\hrulefill" or \ldots) and can contain the commands
\begin{itemize}
\item \verb"\thequestion"\indc{thequestion}, which expands to the
  question number,
\item \verb"\thequestiontitle"\indc{thequestiontitle} (see
  section~\ref{sec:questiontitle}), which expands to
  \begin{itemize}
  \item the title of the question, if this question was defined using
    a \verb"\titledquestion"\indc{titledquestion} command (see
    section~\ref{sec:TitledQues}) instead of a \verb"\question"
    command, or
  \item the number of the question, if this question was defined using
    a \verb"\question" command,
  \end{itemize}
\item \verb"\thepoints"\indc{thepoints}, which expands to
  \begin{itemize}
  \item the number of points followed by the argument to the last
    \verb"\pointname"\indc{pointname} command, if there were points
    specified for the question, or
  \item nothing, if no points were specified for the question
  \end{itemize}
  (see section~\ref{sec:pointname} for a discussion of the
  \verb"\pointname" command), and
\item \verb"\totalpoints"\indc{totalpoints}, which expands to the
  total number of points specified for the question and all of its
  parts, subparts, and subsubparts (see section~\ref{sec:addup}).  The
  command \verb"\totalpoints" expands to
  \verb"\pointsofquestion{\arabic{question}}"\indc{pointsofquestion}
  (see section~\ref{sec:pointsofq}).
\end{itemize}
For example, if you give the
commands\indc{pointpoints}\indc{points}\indc{thepoints}\indc{thequestion}
%
\pointname{ \points}
\pointpoints{point}{points}
\begin{verbatim}
\qformat{Question \thequestion \dotfill \thepoints}
\begin{questions}
\question[10]
What did Billy Joe MacAllister throw off the Tallahatchie Bridge?
\end{questions}
\end{verbatim}
then you'll get
\qformat{Question \thequestion \dotfill \thepoints}
\begin{questions}
\question[10]
What did Billy Joe MacAllister throw off the Tallahatchie Bridge?
\end{questions}
%
\medskip
If you instead use the command
\begin{verbatim}
\qformat{\textbf{Question \thequestion}\quad (\thepoints)\hfill}
\end{verbatim}
then you'll get
\begin{questions}
\qformat{\textbf{Question \thequestion}\quad (\thepoints)\hfill}
\question[10]
What did Billy Joe MacAllister throw off the Tallahatchie Bridge?
\end{questions}
%
\medskip If you instead use the
command\indc{qformat}\indc{thequestion}\indc{thepoints}
\begin{verbatim}
\qformat{\hfill Question \thequestion\hfill (\thepoints)}
\end{verbatim}
then you'll get
\begin{questions}
\qformat{\hfill Question \thequestion\hfill (\thepoints)}
\question[10]
What did Billy Joe MacAllister throw off the Tallahatchie Bridge?
\end{questions}
\noqformat

If you use a \verb"\qformat"\indc{qformat} (and, if there are any
bonus questions, also a \verb"\bonusqformat"\indc{bonusqformat}; see
section~\ref{sec:Bonus}), so that there are no labels on any of the
questions, and you'd like to remove the
indentation\index{indentation!remove} of the questions, you should
give the command
\begin{verbatim}
\renewcommand{\questionshook}{%
  \setlength{\leftmargin}{0pt}%
  \setlength{\labelwidth}{-\labelsep}%
}
\end{verbatim}
(see section~\ref{sec:CustList}) before the \texttt{questions}
environment.

\index{questions!customization|)}

%--------------------------------------------------------------------
\subsubsection{Adding up the points for a question}
\label{sec:adding}
\index{points!adding}

You can combine \verb"\qformat"\indc{qformat} with the
\verb"\totalpoints"\indc{totalpoints} command (see
section~\ref{sec:addup}): If you assign points only to parts,
subparts, and subsubparts of questions, but none to the questions
themselves, and you give the
command\indc{thequestion}\indc{totalpoints}
\begin{verbatim}
\qformat{Question \thequestion\dotfill \emph{\totalpoints\ points}}
\end{verbatim}
(you must have used the document class option
\verb"addpoints"\indt{addpoints} or given the command
\verb"\addpoints"\indc{addpoints} in order to use the
\verb"\totalpoints" command; see section~\ref{sec:addpoints}), then
you can type
\begin{verbatim}
\begin{questions}
  \question
  \begin{parts}
    \part[10]
    In no more than one paragraph, explain why the earth is round.
    \part[10]
    What changes to the van Allen radiation belt are needed to make
    the earth into a regular icosahedron?
  \end{parts}
\end{questions}
\end{verbatim}
and you'll get
%
\nopointsinmargin
\qformat{Question \thequestion\dotfill \emph{20 points}}
\begin{questions}
  \question
  \begin{parts}
    \part[10]
    In no more than one paragraph, explain why the earth is round.
    \part[10]
    What changes to the van Allen radiation belt are needed to make
    the earth into a regular icosahedron?
  \end{parts}
\end{questions}
\noqformat

For another method of printing the total points for a question, see
the \verb"\droptotalpoints" command in section~\ref{sec:addup}.


%--------------------------------------------------------------------
\subsection{Titled questions}
\label{sec:Titled}

The \verb"\titledquestion"\indc{titledquestion} command (see
section~\ref{sec:TitledQues}) allows you to assign a \emph{title} to a
question, and to have the question title replace the question number
in a grading table or point table indexed by question number (see
sections \ref{sec:GradeTables} and~\ref{sec:PointTables}).  If you use
a \verb"\qformat"\indc{qformat} (see section~\ref{sec:qformat}) then
you can include the question title in the question by using the
\verb"\thequestiontitle"\indc{thequestiontitle} command (see the
example in section~\ref{sec:questiontitle}).


%--------------------------------------------------------------------
\subsubsection{\texttt{\bs titledquestion}}
\label{sec:TitledQues}

The command \verb"\titledquestion"\indc{titledquestion} is a
replacement for \verb"\question"\indc{question}; it takes one required
argument, the title, and one optional argument, the point value, as in
\begin{center}
  \verb"\titledquestion{Question \thequestion: Concepts}"
\end{center}
which sets the question title to ``Question 2: Concepts'' (if this is
the second question) and doesn't set any point values, or
\begin{center}
  \verb"\titledquestion{Design}[10]"
\end{center}
which sets the question title to ``Design'' and sets the point value
to ``10''.  Note that the question title doesn't include the number of
the question unless you put it there using the
\verb"\thequestion"\indc{thequestion} command.

Using \verb"\titledquestion"\indc{titledquestion} instead of
\verb"\question" doesn't change the appearance of the question (unless
you use \verb"\thequestiontitle"\indc{thequestiontitle} in the
argument of a \verb"\qformat"\indc{qformat} command; see
section~\ref{sec:questiontitle}).  The only direct effect of using
\verb"\titledquestion" instead of \verb"\question" is that if you
print a grading table or point table indexed by question number (see
section~\ref{sec:GradeTables}), then the title of the question will
replace the number of the question in the table.  To have the title of
the question appear in the question itself, see the example in
section~\ref{sec:questiontitle}.

If you create a grading table or point table indexed by question
number and you create a question using the
\verb"\titledquestion"\indc{titledquestion} command, then the number
of the
%
question\index{question number} won't appear in the table unless you
include \verb"\thequestion"\indc{thequestion} in the title of the
question.

%--------------------------------------------------------------------
\subsubsection{\texttt{\bs thequestiontitle}}
\label{sec:questiontitle}

The command \verb"\thequestiontitle"\indc{thequestiontitle} expands to
\begin{itemize}
\item the title of the question (if the question was defined using a
  \verb"\titledquestion"\indc{titledquestion} command) or
\item the number of the
%
  question\index{question number} (if the question was defined using a
  \verb"\question"\indc{question} command).
\end{itemize}
This is for use in the argument of a \verb"\qformat"\indc{qformat}
command (see section~\ref{sec:qformat}), so that you can have the
question title appear in the question itself, and not just in a
grading table or point table.

The contents of \verb"\thequestiontitle"\indc{thequestiontitle} don't
appear anywhere in the question itself unless you do something to make
them appear.  For example, if you give the command
\begin{center}
\verb"\qformat{Question \thequestion: \thequestiontitle\dotfill\thepoints}"
\end{center}
and then type\indc{titledquestion}
\begin{verbatim}
\begin{questions}
  \titledquestion{History of Empirical Aerodynamics}[10]
  What did Billy Joe MacAllister throw off the Tallahatchie Bridge?
\end{questions}
\end{verbatim}
then you'll get
\begin{questions}
  \qformat{Question \thequestion: \thequestiontitle\dotfill\thepoints}
  \titledquestion{History of Empirical Aerodynamics}[10]
  What did Billy Joe MacAllister throw off the Tallahatchie Bridge?
\end{questions}
\noqformat

If the command \verb"\thequestiontitle" is never used, then the only
effect of using the command \verb"\titledquestion" instead of the
command \verb"\question" is that, in a grading table or point table
indexed by question number, this question will be indexed by the title
of the question rather than by the question number, and the title of
the question will never appear except in the table.

%--------------------------------------------------------------------
\subsection{Bonus points}
\label{sec:Bonus}
\index{bonus points|(}
\index{points!bonus|(}

It's possible to create questions, parts, subparts, or subsubparts
that are worth \emph{bonus points}, instead of regular points.  To do
this, you use the
command\indc{bonusquestion}\indc{bonustitledquestion}%
\indc{bonuspart}\indc{bonussubpart}\indc{bonussubsubpart}
\begin{center}
  \begin{tabular}{l@{\qquad instead of\qquad}l}
    \verb"\bonusquestion"& \verb"\question",\\
    \verb"\bonustitledquestion"& \verb"\titledquestion",\\
    \verb"\bonuspart"& \verb"\part",\\
    \verb"\bonussubpart"& \verb"\subpart", and\\
    \verb"\bonussubsubpart"& \verb"\subsubpart".
  \end{tabular}
\end{center}
There is also a \verb"\bonusqformat"\indc{bonusqformat} command that
affects the \verb"\bonusquestion"\indc{bonusquestion} command in the
same way that the \verb"\qformat"\indc{qformat} command affects the
\verb"\question" command (see section~\ref{sec:qformat}), and a
\verb"\nobonusqformat"\indc{nobonusqformat} command that cancels the
effect of \verb"\bonusqformat" in the same way that
\verb"\noqformat"\indc{noqformat} cancels the effect of
\verb"\qformat".  Note that the command
\verb"\thepoints"\indc{thepoints} is defined so that it works equally
well in \verb"\qformat" and \verb"\bonusqformat"\indc{bonusqformat}
commands (see section~\ref{sec:qformat}).

If \indc{bonusquestion}\indc{bonustitledquestion}%
\indc{bonuspart}\indc{bonussubpart}\indc{bonussubsubpart}%
you don't assign any points to a \verb"\bonusquestion",
\verb"\bonuspart", \verb"\bonussubpart", or \verb"\bonussubsubpart"
then, aside from the fact that \verb"\bonusqformat" alters
\verb"\bonusquestion" while \verb"\qformat" alters \verb"\question",
then they're printed in the same way that they would be printed if you
had used the corresponding non-bonus command.  However, if you assign
points, then
\begin{itemize}
\item the points are typeset differently (see below),
\item {\sloppy the points are not counted by the
    \verb"\gradetable"\indc{gradetable}\indc{pointtable}%
    \indc{bonusgradetable}\indc{bonuspointtable}%
    \indc{combinedgradetable}\indc{combinedpointtable}%
    or \verb"\pointtable" commands (see sections \ref{sec:GradeTables}
    and~\ref{sec:PointTables}) or their partial variants (see
    section~\ref{sec:PartialTable}), though they \emph{are} counted by
    the \verb"\bonusgradetable", \verb"\bonuspointtable",
    \verb"\combinedgradetable", and \verb"\combinedpointtable"
    commands and their partial variants (see
    sections~\ref{sec:BonusTables} and \ref{sec:CombinedTables}),\par}
\item the points are not counted by the
  \verb"\droptotalpoints"\indc{droptotalpoints} command (see
  section~\ref{sec:addup}), though they are counted by the
  \verb"\droptotalbonuspoints"\indc{droptotalbonuspoints} command (see
  below),
\item the points are not counted by the
  \verb"\pointsofquestion"\indc{pointsofquestion} (see
  section~\ref{sec:pointsofq}) or
  \verb"\pointsonpage"\indc{pointsonpage} (see
  section~\ref{sec:pointsonp}) commands, though they \emph{are}
  counted by the
  \verb"\bonuspointsofquestion"\indc{bonuspointsofquestion} and
  \verb"\bonuspointsonpage"\indc{bonuspointsonpage} (see
  section~\ref{sec:BonusTables}) commands, and
\item the points are not counted by the
  \verb"\numpoints"\indc{numpoints} command (see
  section~\ref{sec:Count}), though they \emph{are} counted by the
  \verb"\numbonuspoints"\indc{numbonuspoints} command (see
  section~\ref{sec:Count}).
\end{itemize}

The default setup is that if you type
\begin{verbatim}
\begin{questions}
  \question[10] Who discovered America?
  \bonusquestion[1] On what day of the week did he do it?
  \bonusquestion[2] What color shirt was he wearing?
\end{questions}
\end{verbatim}
then you'll get
\begin{questions}
  \question[10] Who discovered America?
  \bonusquestion[1] On what day of the week did he do it?
  \bonusquestion[2] What color shirt was he wearing?
\end{questions}

There is also a command
\verb"\bonuspointpoints"\indc{bonuspointpoints} that is analogous to
\verb"\pointpoints" (see section~\ref{sec:pointname}); the default
setup was created by the command
\begin{center}
  \verb"\bonuspointpoints{point (bonus)}{points (bonus)}"
\end{center}
and if you give the command
\begin{center}
  \verb"\bonuspointpoints{bonus point}{bonus points}"
\end{center}
then typing the above would produce\indc{bonuspointpoints}
\begin{questions}
  \bonuspointpoints{bonus point}{bonus points}
  \question[10] Who discovered America?
  \bonusquestion[1] On what day of the week did he do it?
  \bonusquestion[2] What color shirt was he wearing?
\end{questions}
There are also
commands\indc{bonuspointname}\indc{marginbonuspointname}
\begin{center}
  \begin{tabular}{l}
    \verb"\bonuspointname" and\\
    \verb"\marginbonuspointname"
  \end{tabular}
  \qquad that are analogous to\qquad
  \begin{tabular}{l}
    \verb"\pointname" and\\
    \verb"\marginpointname"
  \end{tabular}
\end{center}
(see section~\ref{sec:pointname}), and
commands\indc{droptotalpoints}\indc{bonustotalformat}%
\indc{totalbonuspoints}
\begin{center}
  \begin{tabular}{l}
    \verb"\droptotalbonuspoints",\\
    \verb"\bonustotalformat", and\\
    \verb"\totalbonuspoints"
  \end{tabular}
  \qquad that are analogous to\qquad
  \begin{tabular}{l}
    \verb"\droptotalpoints",\\
    \verb"\totalformat", and\\
    \verb"\totalpoints"
  \end{tabular}
\end{center}
(see section~\ref{sec:addup}).

\index{points!bonus|)}
\index{bonus points|)}


%--------------------------------------------------------------------
\subsection{Counting the questions and adding up the points}
\label{sec:Count}
\index{questions!counting|(}


The \verb"exam" document class automatically counts the numbers of
questions, parts, and subparts, and makes these numbers available as
the macros\indc{numquestions}\indc{numparts}\indc{numsubparts}%
\indc{numsubsubparts}
\begin{center}
  \begin{tabular}{l}
    \verb"\numquestions"\\
    \verb"\numparts"\\
    \verb"\numsubparts"\\
    \verb"\numsubsubparts"
  \end{tabular}
\end{center}
These numbers are also printed on the screen when you run \LaTeX{} and
they are placed into the \verb".log" file as well.  If you have more
than one \verb"questions" environment (for example, if your exam has
several parts, with the questions in each part numbered beginning with
``1''), then \verb"\numquestions" will hold the total number of
questions on the exam.  If you do have more than one \verb"questions"
environment, though, you won't be able to use the
\verb"\gradetable"\indc{gradetable} or
\verb"\pointtable"\indc{pointtable} commands (see
section~\ref{sec:Tables}).

If you use the document class option \verb"addpoints"\indt{addpoints}
or give the command \verb"\addpoints"\indc{addpoints} (see
section~\ref{sec:addpoints}), then the class will add the total number
of points and the total number of bonus points (see
section~\ref{sec:Bonus}) that you've given to all of the questions,
parts, subparts, and subsubparts of the exam, and make those totals
available in the macros\indc{numpoints}\indc{numbonuspoints}
\begin{center}
  \begin{tabular}{l}
    \verb"\numpoints"\\
    \verb"\numbonuspoints"
  \end{tabular}
\end{center}
as well as displaying those totals on the screen when you run \LaTeX{}
and placing them into the \texttt{.log} file.  You can then type
\begin{verbatim}
\begin{center}
  This exam has \numquestions\ questions, for a total of \numpoints\ 
  points and \numbonuspoints\ bonus points.
\end{center}
\end{verbatim}
and you'll get
\begin{center}
  This exam has 8 questions, for a total of 120 points and 10 bonus
  points.
\end{center}

\emph{Warning}: If \verb"addpoints"\indt{addpoints} is in effect, your
point values for questions, parts, subparts, and subsubparts must not
contain anything other than digits and the command \verb"\half" (see
section~\ref{sec:Half}).  For example, if you \emph{don't} give the
command \verb"\addpoints", then you can
%
type\index{points!acceptable entries}
\begin{center}
  \verb"\question[10\%]"
\end{center}
with no problems, but this will cause errors if you've given the
command \verb"\addpoints"\indc{addpoints}.  The correct way to
accomplish what you want is to give the command
\verb"\marginpointname{\%}"\indc{marginpointname} or
\verb"\pointname{\%}"\indc{pointname} and then type
\verb"\question[10]" (see section~\ref{sec:pointname}).

If you want to temporarily turn off the adding of points (for example,
if you list both the total points for each question and the points for
each part, but you don't want to count the points twice), you can give
the command\indc{noaddpoints}
\begin{center}
  \verb"\noaddpoints"
\end{center}
to turn off the adding of points and the command \verb"\addpoints" to
turn it back on.

If \verb"addpoints"\indc{addpoints} is in effect, then you can create
grading tables (see section~\ref{sec:GradeTables}) and point tables
(see section~\ref{sec:PointTables}) and use the commands
\verb"\pointsofquestion" (see section~\ref{sec:pointsofq}) and
\verb"\pointsonpage" (see section~\ref{sec:pointsonp}).

\index{questions!counting|)}

%---------------------------------------------------------------------
\subsection{Referring to specific questions by number (cross
  references)}
\index{cross references|(}

You can use the standard \LaTeX{} commands \verb"\label" and
\verb"\ref" to refer to questions (or parts, or subparts, or
subsubparts) by number.  For example, if you type
\begin{verbatim}
The first question is question number~\ref{ques:first}.
Question number~\ref{ques:second} has both a good part
(part~\ref{part:good}) and a bad part (part~\ref{part:bad}).

\begin{questions}
\question
\label{ques:first}
This is the first question.

\question
\label{ques:second}
\begin{parts}
\part
\label{part:good}
This is the good part.

\part
\label{part:bad}
This is the \emph{bad} part.
\end{parts}

\question
Is there a question?
\end{questions}
\end{verbatim}
then you'll get:

\medskip

The first question is question number~\ref{ques:first}.
Question number~\ref{ques:second} has both a good part
(part~\ref{part:good}) and a bad part (part~\ref{part:bad}).

\begin{questions}
\question
\label{ques:first}
This is the first question.

\question
\label{ques:second}
\begin{parts}
\part
\label{part:good}
This is the good part.

\part
\label{part:bad}
This is the \emph{bad} part.
\end{parts}

\question
Is there a question?
\end{questions}



\bigskip

As with all other cross references in \LaTeX, you'll have to run your
file through \LaTeX{} \emph{twice} to be sure that all the cross
references are correct.

\index{cross references|)}

%--------------------------------------------------------------------
\subsection{Customizing the numbers}
\label{sec:CustNumb}
\index{customization!question numbers|(}


The default setup is that:
\begin{questions}
  \question Question numbers are arabic, and are followed by a period.
  \begin{parts}
    \part Part numbers are lower case letters, and are enclosed in
    parentheses.
    \begin{subparts}
      \subpart Subpart numbers are lower case roman, and are followed
      by a period.
      \begin{subsubparts}
        \subsubpart Subsubpart numbers are greek, and are followed by
        a right parenthesis.
        \begin{choices}
          \choice Choices are upper case letters, and are followed by
          a period.
        \end{choices}
      \end{subsubparts}
    \end{subparts}
  \end{parts}
\end{questions}
All of this can be changed.

\bigskip

To change the type of numbering, you would redefine the
commands\indc{thequestion}\indc{thepartno}\indc{thesubpart}%
\indc{thesubsubpart}\indc{thechoice}
\begin{center}
  \begin{tabular}{l}
    \verb"\thequestion"\\
    \verb"\thepartno"\\
    \verb"\thesubpart"\\
    \verb"\thesubsubpart"\\
    \verb"\thechoice"
  \end{tabular}
\end{center}
(Note: The second one listed there is \verb"\thepartno", \emph{not}
\verb"\thepart".  The command \verb"\thepart" refers to the counter
used in the article document class standard sectioning command
\verb"\part".)  The numbering commands available
are\indc{arabic}\indc{alph}\indc{Alph}\indc{roman}%
\indc{Roman}\indc{greeknum}
\begin{center}
  \begin{tabular}{ll}
    \verb"\arabic"& Regular arabic integers\\
    \verb"\alph"& Lower case letters\\
    \verb"\Alph"& Upper case letters\\
    \verb"\roman"& Lower case roman numbers\\
    \verb"\Roman"& Upper case roman numbers\\
    \verb"\greeknum"& Greek letters
  \end{tabular}
\end{center}
and any of these can be applied to the
counters\index{question counter}\index{part counter}%
\index{subpart counter}\index{subsubpart counter}%
\index{choice counter}%
\verb"question"\indt{question}, \verb"partno"\indt{partno},
\verb"subpart"\indt{subpart}, \verb"subsubpart"\indt{subsubpart}, and
\verb"choice"\indt{choice}.  (The \verb"\greeknum"\indc{greeknum}
command is defined by the exam document class, but all of the others
are standard \LaTeX{} commands.)

For example, to have questions numbered using upper case roman numbers
and parts numbered using upper case letters, you would give the
commands\indc{thequestion}\indc{thepartno}
\begin{center}
  \begin{tabular}{l}
    \verb"\renewcommand{\thequestion}{\Roman{question}}"\\
    \verb"\renewcommand{\thepartno}{\Alph{partno}}"
  \end{tabular}
\end{center}

\bigskip


The ``decorations'' around the numbers (i.e., the periods, or
parentheses, or \ldots) are determined by the
commands\indc{questionlabel}\indc{partlabel}\indc{subpartlabel}%
\indc{subsubpartlabel}\indc{choicelabel}
\begin{center}
  \begin{tabular}{l}
    \verb"\questionlabel"\\
    \verb"\partlabel"\\
    \verb"\subpartlabel"\\
    \verb"\subsubpartlabel"\\
    \verb"\choicelabel"
  \end{tabular}
\end{center}
the default definitions of which are:
\begin{center}
  \begin{tabular}{l}
    \verb"\newcommand{\questionlabel}{\thequestion.}"\\
    \verb"\newcommand{\partlabel}{(\thepartno)}"\\
    \verb"\newcommand{\subpartlabel}{\thesubpart.}"\\
    \verb"\newcommand{\subsubpartlabel}{\thesubsubpart)}"\\
    \verb"\newcommand{\choicelabel}{\thechoice.}"
  \end{tabular}
\end{center}
You can change any of these by giving \verb"\renewcommand" commands to
redefine them.  This can, of course, be used to change the
``decorations'' around the numbers, but there are other uses as well.
For example, if you use the Fourier project's Utopia
font\index{Utopia~font} by including the commands
\begin{center}
  \begin{tabular}{l}
    \verb"\usepackage[T1]{fontenc}"\\
    \verb"\usepackage{fourier}"
  \end{tabular}
\end{center}
after your \verb"\documentclass" command, then you'll see that the
kerning\index{kerning} makes ``\verb"F."" look so much like
``\verb"E"" that a student could confuse the two.  You can turn off
the kerning with the command
\verb"\renewcommand{\choicelabel}{\thechoice{}.}"\indc{choicelabel}.

\index{customization!question numbers|)}


%--------------------------------------------------------------------
\subsection{Customizing the list parameters}
\label{sec:CustList}
\index{list environment|(}
\index{environment!list|(}
\index{customization!list parameters|(}

The environments \texttt{questions}, \texttt{parts},
\texttt{subparts}, \texttt{subsubparts}, \texttt{choices}, and
\texttt{checkboxes} are all implemented as \LaTeX{} \texttt{list}
environments.  This section describes commands for advanced users who
want to customize the list parameters (\verb"\topsep"\indc{topsep},
\verb"\partopsep"\indc{partopsep}, \verb"\itemsep"\indc{itemsep},
\verb"\parsep"\indc{parsep}, etc.) for the lists that those
environments create.

The \verb"exam" document class includes the expansion
of\indc{questionshook}\indc{partshook}\indc{subpartshook}%
\indc{subsubpartshook}\indc{choiceshook}\indc{checkboxeshook}
\begin{center}
  \begin{tabular}{l@{\qquad when creating a\qquad}l@{\qquad}l}
    \verb"\questionshook"&\verb"questions"&environment,\\
    \verb"\partshook"&\verb"parts"&environment,\\
    \verb"\subpartshook"&\verb"subparts"&environment,\\
    \verb"\subsubpartshook"&\verb"subsubparts"&environment,\\
    \verb"\choiceshook"&\verb"choices"&environment, and\\
    \verb"\checkboxeshook"&\verb"checkboxes"&environment.
  \end{tabular}
\end{center}
The \verb"exam" document class defines these as follows:
\begin{center}
  \begin{tabular}{ll}
    \verb"\newcommand{\questionshook}{}"\\
    \verb"\newcommand{\partshook}{}"\\
    \verb"\newcommand{\subpartshook}{}"\\
    \verb"\newcommand{\subsubpartshook}{}"\\
    \verb"\newcommand{\choiceshook}{}"\\
    \verb"\newcommand{\checkboxeshook}{}"
  \end{tabular}
\end{center}
Thus, the default is that these commands have no effect, but you can
change any of them with a \verb"\renewcommand" command.  Thus, if you
redefine, e.g., \verb"\questionshook" to expand to commands that set
values for some of the list parameters, then those values will be used
in the \verb"questions" environment.

For example, if you give the command\indc{questionshook}
\begin{center}
  \verb"\renewcommand{\questionshook}{\setlength{\itemsep}{1in}}"
\end{center}
then there will be an additional one inch of blank space inserted
between questions (since the questions are the items in a
\verb"questions" environment).  For another example, if you give the
command\indc{partshook}
\begin{center}
  \verb"\renewcommand{\partshook}{\setlength{\topsep}{0.5in}}"
\end{center}
then each parts environment will have an additional one half inch of
space before and after the environment.

For another example, if you're using a \verb"\qformat"\indc{qformat}
(see section~\ref{sec:qformat}) (and, if there are any bonus
questions, also a \verb"\bonusqformat"\indc{bonusqformat}; see
section~\ref{sec:Bonus}), so that there are no labels on any of the
questions, and you'd like to remove the indentation of the questions,
give the
command\index{indentation!remove}\indc{leftmargin}\indc{labelsep}
\begin{verbatim}
\renewcommand{\questionshook}{%
  \setlength{\leftmargin}{0pt}%
  \setlength{\labelwidth}{-\labelsep}%
}
\end{verbatim}

\index{customization!list parameters|)}
\index{environment!list|)}
\index{list environment|)}

%--------------------------------------------------------------------
%--------------------------------------------------------------------
\section{Multiple choice and fill in the blank questions}
\label{sec:MulChc}
\index{multiple choice}

There are four environments for listing the possible answers to a
multiple choice question: two of them label the choices, and the other
two print checkboxes in front of the choices for the student to place
checks next to the chosen answers.
\begin{itemize}
\item The first two environments use upper case letters (i.e., ``A'',
  ``B'', ``C'', \ldots) to label the choices.  (To change the
  appearance of the labels, see section~\ref{sec:CustNumb}.)
  \begin{itemize}
  \item The \verb"choices"\indt{choices}\index{choices environment}%
    \index{environment!choices} environment creates a list environment
    with the choices as the items in the list (see
    section~\ref{sec:choices}).
  \item The \verb"oneparchoices"\indt{oneparchoices}%
    \index{oneparchoices environment}%
    \index{environment!oneparchoices} environment lists all of the
    choices in a single paragraph which is a continuation of the
    paragraph preceding the environment unless you leave a blank line
    before beginning the environment (see
    section~\ref{sec:oneparchoices}).
  \end{itemize}
  If you don't want the students to just circle the correct choices,
  you can use the \verb"\answerline"\indc{answerline} command to print
  a line on which the answer should be written (see
  section~\ref{sec:ShortAns}).  The
  \verb"\anwerline"\indcsub{answerline}{optional~argument}
  command allows you to include as an optional argument the answer to
  the question, and the answer will be printed on the line when
  solutions are bing printed (see section~\ref{sec:AnsShortAns}).
\item The last two environments print checkboxes in front of the
  choices for the students to use to place checks next to the chosen
  answers.
  \begin{itemize}
  \item The \verb"checkboxes"\indt{checkboxes}%
    \index{checkboxes environment}%
    \index{environment!checkboxes} environment creates a list
    environment with the choices as the items in the list (see
    section~\ref{sec:checkboxes}).
  \item The \verb"oneparcheckboxes"\indt{oneparcheckboxes}%
    \index{oneparcheckboxes environment}%
    \index{environment!oneparcheckboxes} environment lists all of the
    choices in a single paragraph which is a continuation of the
    paragraph preceding the environment unless you leave a blank line
    before beginning the environment (see
    section~\ref{sec:oneparcheckboxes}).
  \end{itemize}
\end{itemize}
All of those environments allow you to designate one or more choices
as correct choices and to have the correct choices printed with
emphasis when solutions are being printed (see
section~\ref{sec:CorrectChoice}).

%--------------------------------------------------------------------
\subsection{The \texttt{choices} environment}
\label{sec:choices}
\indt{choices}
\indc{choice}
\index{choices environment}
\index{environment!choices}

For example, if you type
\begin{verbatim}
\begin{questions}
\question
One of these things is not like the others; one of these things is not
the same.  Which one is different?
\begin{choices}
\choice John
\choice Paul

\choice

George

\choice Ringo

\choice Socrates

\end{choices}
    
\end{questions}
\end{verbatim}
Then you'll get
\begin{questions}
\question
One of these things is not like the others; one of these things is not
the same.  Which one is different?
\begin{choices}
\choice John
\choice Paul

\choice

George

\choice Ringo

\choice Socrates

\end{choices}
    
\end{questions}
As that example illustrates, you can leave blank lines between the
\verb"\choice" command and the text of the choice, or between the
various choices, and those blank lines will be ignored.

%--------------------------------------------------------------------
\subsection{The \texttt{oneparchoices} environment}
\label{sec:oneparchoices}
\indt{oneparchoices}
\indc{choice}
\index{oneparchoices environment}
\index{environment!oneparchoices}

If you use a \verb"oneparchoices" environment instead of the
\verb"choices" environment in the example of
section~\ref{sec:choices}, then you'll get
\begin{questions}
\question
One of these things is not like the others; one of these things is not
the same.  Which one is different?
\begin{oneparchoices}
\choice John
\choice Paul

\choice

George

\choice Ringo

\choice Socrates

\end{oneparchoices}
    
\end{questions}

If you insert a blank line before the \verb"\begin{oneparchoices}",
  then you'll get
\begin{questions}
\question
One of these things is not like the others; one of these things is not
the same.  Which one is different?

\begin{oneparchoices}
\choice John
\choice Paul

\choice

George

\choice Ringo

\choice Socrates

\end{oneparchoices}
    
\end{questions}


If you insert a blank line before the \verb"\begin{oneparchoices}" and
  \verb"\answerline" after the \verb"\end{oneparchoices}" (see
section~\ref{sec:ShortAns} for a description of the
\verb"\answerline"\indc{answerline} command), then you'll get
\begin{questions}
\question
One of these things is not like the others; one of these things is not
the same.  Which one is different?

\begin{oneparchoices}
\choice John
\choice Paul

\choice

George

\choice Ringo

\choice Socrates

\end{oneparchoices}
\answerline

\end{questions}

%--------------------------------------------------------------------
\subsection{The \texttt{checkboxes} environment}
\label{sec:checkboxes}
\indt{checkboxes}
\indc{choice}
\index{checkboxes environment}
\index{environment!checkboxes}

If you use a \texttt{checkboxes} environment instead of the
\texttt{choices} environment in the example of
section~\ref{sec:choices}, then you'll get
\begin{questions}
\question
One of these things is not like the others; one of these things is not
the same.  Which one is different?
\begin{checkboxes}
\choice John
\choice Paul

\choice

George

\choice Ringo

\choice Socrates

\end{checkboxes}
    
\end{questions}
The appearance of the checkbox can be customized (see
section~\ref{sec:MulChceCust}).


%--------------------------------------------------------------------
\subsection{The \texttt{oneparcheckboxes} environment}
\label{sec:oneparcheckboxes}
\indt{oneparcheckboxes}
\indc{choice}
\index{oneparcheckboxes environment}
\index{environment!oneparcheckboxes}

If you use a \verb"oneparcheckboxes" environment instead of the
\verb"choices" environment in the example of
section~\ref{sec:choices}, then you'll get
\begin{questions}
\question
One of these things is not like the others; one of these things is not
the same.  Which one is different?
\begin{oneparcheckboxes}
\choice John
\choice Paul

\choice

George

\choice Ringo

\choice Socrates

\end{oneparcheckboxes}
    
\end{questions}

If you insert a blank line before the \verb"\begin{oneparcheckboxes}",
  then you'll get
\begin{questions}
\question
One of these things is not like the others; one of these things is not
the same.  Which one is different?

\begin{oneparcheckboxes}
\choice John
\choice Paul

\choice

George

\choice Ringo

\choice Socrates

\end{oneparcheckboxes}
    
\end{questions}
The appearance of the checkbox can be customized (see
section~\ref{sec:MulChceCust}).



%--------------------------------------------------------------------
\subsection{Solutions to multiple choice questions}
\label{sec:CorrectChoice}
\index{solutions}
\index{choices environment!correct choice}
\index{environment!choices!correct choice}
\index{oneparchoices environment!correct choice}
\index{environment!oneparchoices!correct choice}
\index{checkboxes environment!correct choice}
\index{environment!checkboxes!correct choice}
\index{oneparcheckboxes environment!correct choice}
\index{environment!oneparcheckboxes!correct choice}


The \verb"exam" document class defines several environments for
solutions (see section~\ref{sec:solutions}), the contents of which
will be printed only if you use the document class option
\verb"answers"\indt{answers} (as in
\verb"\documentclass[answers]{exam}") or give the command
\verb"\printanswers"\indc{printanswers} (see
section~\ref{sec:answers}).  Each of the environments for multiple
choice questions (see
sections~\ref{sec:choices}--\ref{sec:oneparcheckboxes}) allows you to
designate one or more of the choices as \emph{correct choices}, and to
have the correct choices printed with emphasis when solutions are
being printed.

The command to define a correct choice is
\verb"\CorrectChoice"\indc{CorrectChoice}, which is used in place of
the command \verb"\choice"\indc{choice}.  (To avoid confusion,
\verb"\correctchoice"\indc{correctchoice} is defined to be a synonym
for \verb"\CorrectChoice".)  When solutions are not being printed,
\verb"\CorrectChoice" is equivalent to \verb"\choice".  When solutions
are being printed, though, any choices that were created with
\verb"\CorrectChoice" (rather than with \verb"\choice") will be
printed in bold.  (This can be changed; see
section~\ref{sec:MulChceCust}.)  In addition, in the
\texttt{checkboxes}\indt{checkboxes} and
\texttt{oneparcheckboxes}\indt{oneparcheckboxes} environments, each
correct choice will have its checkbox replaced by a check.

For example, if \verb"\printanswers"\indc{printanswers} is in effect
and you type\indc{CorrectChoice}
\begin{verbatim}
\begin{questions}
  \question One of these things is not like the others; one of these
  things is not the same.  Which one is different?
  \begin{oneparchoices}
    \choice John
    \choice Paul
    \choice George
    \choice Ringo
    \CorrectChoice Socrates
  \end{oneparchoices}
\end{questions}
\end{verbatim}
then you'll get
\printanswers
\begin{questions}
  \question One of these things is not like the others; one of these
  things is not the same.  Which one is different?
  \begin{oneparchoices}
    \choice John
    \choice Paul
    \choice George
    \choice Ringo
    \CorrectChoice Socrates
  \end{oneparchoices}
\end{questions}
\noprintanswers
%
If you use a \texttt{oneparcheckboxes}\indt{oneparcheckboxes}
environment instead of the \texttt{oneparchoices}\indt{oneparchoices}
environment in that example, then you'll get\indc{printanswers}
%
\printanswers
\begin{questions}
  \question One of these things is not like the others; one of these
  things is not the same.  Which one is different?
  \begin{oneparcheckboxes}
    \choice John
    \choice Paul
    \choice George
    \choice Ringo
    \CorrectChoice Socrates
  \end{oneparcheckboxes}
\end{questions}
\noprintanswers

%--------------------------------------------------------------------
\subsection{Fill in the blank questions}
\label{sec:Fillin}

\indcstart{fillin}
\index{fill in the blank|(}

There is a \verb"\fillin" command that prints underlined empty space
for creating ``fill in the blank'' questions.  The \verb"\fillin"
command can take two optional arguments:
\begin{itemize}
\item the first optional argument is the answer to the question, which
  will be printed above the line when answers are being printed, and
\item the second optional argument is the length of line to be
  printed, if you want some length other than the default.
\end{itemize}
If answers are being printed and you supply the answer as the first
(and possibly only) optional argument, then the answer is printed
subject to the declarations in the argument of the most recent
\verb"\CorrectChoiceEmphasis" command.  It is centered on the line
unless it is too long, in which case it extends to the right of the
line.  The default length of the line and the distance down from the
baseline of the text to the empty line can both be customized (see
section~\ref{sec:CustFillin}).

For example, if you type
\begin{verbatim}
\begin{questions}
  \question \fillin is the color of my true love's hair.
\end{questions}
\end{verbatim}
then you'll get
\begin{questions}
  \question \fillin is the color of my true love's hair.
\end{questions}
If, instead, you type
\begin{verbatim}
\begin{questions}
  \question \fillin[Black] is the color of my true love's hair.
\end{questions}
\end{verbatim}
then if answers are not being printed you'll get
\begin{questions}
  \question \fillin[Black] is the color of my true love's hair.
\end{questions}
and if answers are being printed you'll get
\printanswers
\begin{questions}
  \question \fillin[Black] is the color of my true love's hair.
\end{questions}
\noprintanswers

For another example, if answers are being printed and you type
\begin{verbatim}
\begin{questions}
  \question The magic word is \fillin[prestidigitator].
\end{questions}
\end{verbatim}
then you'll get
\printanswers
\begin{questions}
  \question The magic word is \fillin[prestidigitator].
\end{questions}
\noprintanswers
which shows what happens when the answer is too long for the fillin
line.  If, however, you type
\begin{verbatim}
\begin{questions}
  \question The magic word is \fillin[prestidigitator][2in].
\end{questions}
\end{verbatim}
then you'll get
\printanswers
\begin{questions}
  \question The magic word is \fillin[prestidigitator][2in].
\end{questions}
\noprintanswers

\verb"\fillin" eats (and ignores) space characters appearing before
the first optional argument.  It also eats (and ignores) space
characters appearing after the first optional argument and before the
second optional argument.  However, if exactly one optional argument
appears, and if there are one or more space characters following that
one optional argument, then those spaces are replaced by a single
space character, but not eaten.
%--------------------------------------------------------------------
\subsubsection{True/False questions}
\label{sec:truefalse}

\index{true/false questions|(}
The \verb"\fillin" command can also be used to create True/False
questions.  If you give the command
\begin{verbatim}
\newcommand{\tf}[1][{}]{%
  \fillin[#1][0.25in]%
}
\end{verbatim}
then you can type
\begin{verbatim}
\begin{questions}
  \question \tf[T] The world is all that is the case.
  \question \tf[F] My favorite color is blue.
\end{questions}
\end{verbatim}
\newcommand{\tf}[1][{}]{%
  \fillin[#1][0.25in]%
}
and if answers are not being printed you'll get
\begin{questions}
  \question \tf[T] The world is all that is the case.
  \question \tf[F] My favorite color is blue.
\end{questions}
but if answers are being printed then you'll get
\printanswers
\begin{questions}
  \question \tf[T] The world is all that is the case.
  \question \tf[F] My favorite color is blue.
\end{questions}
\noprintanswers

\index{true/false questions|)}

%--------------------------------------------------------------------
\subsubsection{Customizing the \texttt{\bs fillin} command}
\label{sec:CustFillin}
\index{customization!\texttt{\char`\\fillin}|(}

You can customize the \verb"\fillin" command by changing
\begin{itemize}
\item the default length of the underlined space,
\item the distance below the text baseline that the line is printed, and
\item the emphasis applied to the answer (when answers are being
  printed).
\end{itemize}
The default length of the underlined space is called
\verb"\fillinlinelength"\indc{fillinlinelength}, and its value was set
by the command
\begin{center}
  \verb"\setlength\fillinlinelength{1in}"
\end{center}
You can change this by giving a new \verb"\setlength" command.

The line is printed a distance of
\verb"\answerclearance"\indc{answerclearance} below the baseline, and
its value was set by the command
\begin{center}
  \verb"\setlength\answerclearance{0.2ex}"
\end{center}
You can change this by giving a new \verb"\setlength" command.  Note:
\verb"\answerclearance" is also used by the \verb"\answerline" command
(see section~\ref{sec:ShortAns}).

When answers are being printed and you include the optional argument
containing the answer, then the answer is printed subject to the
declarations in the argument of the most recent
\verb"\CorrectChoiceEmphasis"\indc{CorrectChoiceEmphasis} command.
The default situation was created by the command
\begin{center}
  \verb"\CorrectChoiceEmphasis{\bfseries}"
\end{center}
and you can change this by giving a new \verb"\CorrectChoiceEmphasis"
command (see section~\ref{sec:MulChceCust}).

\index{customization!\texttt{\char`\\fillin}|)}
\index{fill in the blank|)}
\indcstop{fillin}


%--------------------------------------------------------------------
\subsection{Customizing the multiple choice environments}
\label{sec:MulChceCust}
\index{customization!multiple choice questions|(}

When solutions are being printed, the correct choices in any of the
environments \texttt{choices}\indt{choices},
\texttt{oneparchoices}\indt{oneparchoices},
\texttt{checkboxes}\indt{checkboxes}, and
\texttt{oneparcheckboxes}\indt{oneparcheckboxes} will be printed
subject to the declarations in the argument of the most
recent\indc{CorrectChoiceEmphasis}
\begin{center}
  \verb"\CorrectChoiceEmphasis"
\end{center}
command.  The default situation is created by the command
\begin{center}
  \verb"\CorrectChoiceEmphasis{\bfseries}"
\end{center}
and so when solutions are being printed the correct choices are
printed in boldface by default.  You can change this, for example, by
giving the command
\begin{center}
  \verb"\CorrectChoiceEmphasis{\itshape}"
\end{center}
which would cause correct choices to be printed in italics when
solutions are being printed.

For another example, if you give the command
\verb"\usepackage{color}"\indt{color} in the preamble\index{preamble}
of your document (i.e., after the \verb"\documentclass" command and
before the \verb"\begin{document}"), and then give the
  command\indc{CorrectChoiceEmphasis}\indc{color}
\begin{center}
  \verb"\CorrectChoiceEmphasis{\color{red}}"
\end{center}
then when solutions are being printed the correct choices will be
printed in red (assuming your printer can produce that color).

For another example, if you give the command
\verb"\usepackage{color}"\indt{color} in the preamble of your document
(i.e., after the \verb"\documentclass" command and before the
\verb"\begin{document}" command) and then give the command\indc{color}
\begin{center}
  \verb"\CorrectChoiceEmphasis{\color{red}\bfseries}"
\end{center}
then the correct choices will be printed in boldface red (again
assuming that your printer can produce that color).

{\sloppy To avoid confusion,
  \verb"\correctchoiceemphasis"\indc{correctchoiceemphasis} is defined
  to be a synonym for
  \verb"\CorrectChoiceEmphasis"\indc{CorrectChoiceEmphasis}.\par}

\bigskip

The \texttt{checkboxes}\indt{checkboxes} and
\texttt{oneparcheckboxes}\indt{oneparcheckboxes} environments also
allow customization of the characters used for checkboxes and checks:
\begin{itemize}
\item The checkbox\index{checkbox} that is printed in the
  \texttt{checkboxes} and \texttt{oneparcheckboxes} environments is
  the argument of the most recent
  \verb"\checkboxchar"\indc{checkboxchar} command.  The default is
  created by the command
    \begin{center}
      \verb"\checkboxchar{$\bigcirc$}"
    \end{center}
    For example, if you give the command
    \verb"\usepackage{amssymb}"\indt{amssymb} in the
    preamble\index{preamble} of your document (i.e., after the
    \verb"\documentclass" command and before the
    \verb"\begin{document}"), then the command\indc{Box}
      \begin{center}
        \verb"\checkboxchar{$\Box$}"
      \end{center}
      will cause squares to be printed instead of circles.

    \item The check that is printed for correct choices in the
      \texttt{checkboxes}\indt{checkboxes} and
      \texttt{oneparcheckboxes}\indt{oneparcheckboxes} environments
      when solutions are being printed is the argument of the most
      recent \verb"\checkedchar"\indc{checkedchar} command.  The
      default is created by the command\indc{surd}
    \begin{center}
      \verb"\checkedchar{$\surd$}"
    \end{center}
\end{itemize}

For example, if you've given the command
\verb"\usepackage{amssymb}"\indt{amssymb} in the
preamble\index{preamble} of your document (i.e., after the
\verb"\documentclass" command and before the \verb"\begin{document}"),
  then the commands\indc{Box}\indc{blacksquare}
\begin{center}
  \begin{tabular}{l}
    \verb"\checkboxchar{$\Box$}"\\
    \verb"\checkedchar{$\blacksquare$}"
  \end{tabular}
\end{center}
will cause the example in section~\ref{sec:CorrectChoice} to appear as
%
\printanswers
\checkboxchar{$\Box$}
\checkedchar{$\blacksquare$}
\begin{questions}
  \question One of these things is not like the others; one of
  these things is not the same.  Which one is different?
  \begin{oneparcheckboxes}
    \choice John
    \choice Paul
    \choice George
    \choice Ringo
    \CorrectChoice Socrates
  \end{oneparcheckboxes}
\end{questions}
\noprintanswers

\index{customization!multiple choice questions|)}
%--------------------------------------------------------------------
%--------------------------------------------------------------------
\section{Escaping the indentation: \texttt{\bs uplevel} and \texttt{\bs fullwidth}}
\label{sec:uplevel}
\index{indentation|(}

There are two commands provided for printing text with a left margin
smaller than the one currently in effect:
\verb"\uplevel"\indc{uplevel} and \verb"\fullwidth"\indc{fullwidth}.
Each of these commands takes one argument and typesets the argument
with an altered left margin.

The \verb"\fullwidth" command prints its argument using the full width
of the printed region, temporarily canceling the indentation created
by the \texttt{questions} environment (and the \texttt{parts},
\texttt{subparts}, and \texttt{subsubparts} environments, if you're
inside any of those).  The \verb"\uplevel" command prints its argument
using the indentation in effect before you entered the current
innermost environment.

There are also two environments,
\texttt{EnvUplevel}\indt{EnvUplevel}\index{environment!EnvUplevel}%
\index{EnvUplevel environment} and
\texttt{EnvFullwidth}\indt{EnvFullwidth}\index{environment!EnvFullwidth}%
\index{EnvFullwidth environment}.  The \texttt{EnvUplevel} environment
is similar to the \verb"\uplevel" command and the
\texttt{EnvFullwidth} environment is similar to the \verb"\fullwidth"
command.  The environments have the advantage that you can have
verbatim material inside of the environment, while there cannot be any
verbatim material in the argument of \verb"\uplevel" or
\verb"\fullwidth".

The main uses for these commands are including instructions for a
group of questions (see section~\ref{sec:Instructions}) and naming the
parts of a long exam (see section~\ref{sec:Naming}), but you can use
them however you like.

%--------------------------------------------------------------------
\subsection{Including instructions for a group of questions}
\label{sec:Instructions}
\index{instructions|(}
\indcstart{uplevel}
\indtstart{EnvUplevel}
\index{EnvUplevel environment|(}
\index{environment!EnvUplevel|(}


If you are inside of a parts environment and you want to give
directions for the next few parts, then those directions should be
indented to the level of the question of which they are parts, i.e.,
up one level.  If you type
\begin{verbatim}
\begin{questions}
  \question
  Why did you come to Casablanca?

  \question
  \begin{parts}
    \part
    Why, Oh why, Oh why, Oh; why did I ever leave Ohio?

    \uplevel{The following two parts should be answered in classical
      Greek:}
    \part
    Why do birds sing?

    \part
    Why do fools fall in love?

  \end{parts}
\end{questions}
\end{verbatim}
then you'll get
\begin{questions}
  \question
  Why did you come to Casablanca?

  \question
  \begin{parts}
    \part
    Why, Oh why, Oh why, Oh; why did I ever leave Ohio?

    \uplevel{The following two parts should be answered in classical
      Greek:}
    \part
    Why do birds sing?

    \part
    Why do fools fall in love?

  \end{parts}
\end{questions}
You can accomplish exactly the same thing by replacing
\begin{verbatim}
    \uplevel{The following two parts should be answered in classical
      Greek:}
\end{verbatim}
by
\begin{verbatim}
    \begin{uplevel}
      The following two parts should be answered in classical Greek:
    \end{uplevel}
\end{verbatim}

If you want to give instructions for a group of questions, then the
indenting for those instructions should be to the outer left margin,
i.e., up one level.  For example, if you type
\begin{verbatim}
\begin{questions}
  \question
  Approximate $\displaystyle \int_0^1 \sin x^2 \, dx$ within $.001$ of
  its true value.

  \uplevel{Questions \ref{exact-start} through~\ref{exact-end} should
    be evaluated completely, not just approximated.}

  \question
  \label{exact-start}
  $\displaystyle \int_0^1 \frac{x^2 \, dx}{\sqrt{1-x^2}}$

  \question
  $\displaystyle \int_0^1 \frac{1}{1+x^2}\, dx$

  \question
  \label{exact-end}
  $\displaystyle \int_0^{\frac{\pi}{2}} \sin^3 x \cos x \, dx$
\end{questions}
\end{verbatim}
you'll get
\begin{questions}
  \question
  Approximate $\displaystyle \int_0^1 \sin x^2 \, dx$ within $.001$ of
  its true value.

  \uplevel{Questions \ref{exact-start} through~\ref{exact-end} should
    be evaluated completely, not just approximated.}

  \question
  \label{exact-start}
  $\displaystyle \int_0^1 \frac{x^2 \, dx}{\sqrt{1-x^2}}$

  \question
  $\displaystyle \int_0^1 \frac{1}{1+x^2}\, dx$

  \question
  \label{exact-end}
  $\displaystyle \int_0^{\frac{\pi}{2}} \sin^3 x \cos x \, dx$
\end{questions}
You can accomplish exactly the same thing by replacing
\begin{verbatim}
  \uplevel{Questions \ref{exact-start} through~\ref{exact-end} should
    be evaluated completely, not just approximated.}
\end{verbatim}
by
\begin{verbatim}
  \begin{EnvUplevel}
    Questions \ref{exact-start} through~\ref{exact-end} should be
    evaluated completely, not just approximated.
  \end{EnvUplevel}
\end{verbatim}
\indcstop{uplevel}
\indtstop{EnvUplevel}
\index{EnvUplevel environment|)}
\index{environment!EnvUplevel|)}

\indcstart{fullwidth}
\indtstart{EnvFullwidth}
\index{EnvFullwidth environment|(}
\index{environment!EnvFullwidth|(}
If you want to give instructions that use the full width of the page
(rather than just going up one level of indentation), then use the
\verb"\fullwidth" command.  For example, if you type
\begin{verbatim}
\begin{questions}
  \question
  This is the first question.

  \question
  \begin{parts}
    \part
    This is the first part.

    \part
    This is the second part.
    \begin{subparts}
      \subpart
      This is a subpart.
      
      \fullwidth{When you finish this exam, you should go back and
        reexamine your work, both on this exam and in your life up
        until the day of this exam, for any errors that you may have
        made.}

      \subpart
      This is another subpart.
    \end{subparts}
  \end{parts}
\end{questions}
\end{verbatim}
then you'll get
\begin{questions}
  \question
  This is the first question.

  \question
  \begin{parts}
    \part
    This is the first part.

    \part
    This is the second part.
    \begin{subparts}
      \subpart
      This is a subpart.
      
      \fullwidth{When you finish this exam, you should go back and
        reexamine your work, both on this exam and in your life up
        until the day of this exam, for any errors that you may have
        made.}

      \subpart
      This is another subpart.
    \end{subparts}
  \end{parts}
\end{questions}
You can accomplish exactly the same thing by replacing
\begin{verbatim}
      \fullwidth{When you finish this exam, you should go back and
        reexamine your work, both on this exam and in your life up
        until the day of this exam, for any errors that you may have
        made.}
\end{verbatim}
by
\begin{verbatim}
      \begin{EnvFullwidth}
        When you finish this exam, you should go back and
        reexamine your work, both on this exam and in your life up
        until the day of this exam, for any errors that you may have
        made.
      \end{EnvFullwidth}
\end{verbatim}
\indcstop{fullwidth}
\indtstop{EnvFullwidth}
\index{EnvFullwidth environment|)}
\index{environment!EnvFullwidth|)}

\index{instructions|)}

%---------------------------------------------------------------------
\subsection{Naming the parts of a long exam}
\label{sec:Naming}
\index{naming parts of the exam|(}

There are two ways of naming the parts of a long exam.  The first way
uses the \verb"\fullwidth"\indc{fullwidth} and
\verb"\uplevel"\indc{uplevel} commands (see
section~\ref{sec:Namefullwidth}), and the other way uses the standard
%
\verb"\part"\indcsub{part}{standard sectioning command}
%
and \verb"\section"\indc{section} commands (see
section~\ref{sec:NameSection}).

%--------------------------------------------------------------------
\subsubsection{Using \texttt{\bs uplevel} and  \texttt{\bs fullwidth}}
\label{sec:Namefullwidth}
\indcstart{uplevel}
\indcstart{fullwidth}
\indtstart{EnvFullwidth}
\index{EnvFullwidth environment|(}
\index{environment!EnvFullwidth|(}

To place a section name in the exam, just use a \verb"\fullwidth"
command (see section~\ref{sec:uplevel}) and include whatever font
changing commands that you want to use.  For example, if you type
\begin{verbatim}
\begin{questions}
  \question
  Is there, is there balm in Gilead?

  \fullwidth{\Large \textbf{Essay questions}}
  
  \question
  Explain how the cooling of matter in the centuries following the big
  bang has influenced the British parliamentary system of government.

  \fullwidth{\Large \textbf{Laboratory questions}}

  \question
  In the cabinet below your laboratory bench you will find a single
  edged razor blade, several C-clamps, and a bottle of whiskey.
  Remove your appendix.  Do not suture until your work has been
  inspected.
\end{questions}
\end{verbatim}
then you'll get
\begin{questions}
  \question
  Is there, is there balm in Gilead?

  \fullwidth{\Large \textbf{Essay questions}}
  
  \question
  Explain how the cooling of matter in the centuries following the big
  bang has influenced the British parliamentary system of government.

  \fullwidth{\Large \textbf{Laboratory questions}}

  \question
  In the cabinet below your laboratory bench you will find a single
  edged razor blade, several C-clamps, and a bottle of whiskey.
  Remove your appendix.  Do not suture until your work has been
  inspected.
\end{questions}
You can accomplish exactly the same thing by replacing
\begin{verbatim}
  \fullwidth{\Large \textbf{Essay questions}}
\end{verbatim}
by
\begin{verbatim}
  \begin{EnvFullwidth}
    \Large \textbf{Essay questions}
  \end{EnvFullwidth}
\end{verbatim}
and replacing
\begin{verbatim}
  \fullwidth{\Large \textbf{Laboratory questions}}
\end{verbatim}
by
\begin{verbatim}
  \begin{EnvFullwidth}
    \Large \textbf{Laboratory questions}
  \end{EnvFullwidth}
\end{verbatim}


\indcstop{fullwidth}
\indtstop{EnvFullwidth}
\index{EnvFullwidth environment|)}
\index{environment!EnvFullwidth|)}
\indcstop{uplevel}

%--------------------------------------------------------------------
\subsubsection{Using the standard sectioning commands}
\label{sec:NameSection}

The exam document class is built upon the standard article document
class, and so the sectioning commands used with the article document
class can be used here as well.  In particular, you can give the
commands \verb"\part",
%
\verb"\part"\indcsub{part}{standard sectioning command}
%
\verb"\part*", \verb"\section"\indc{section}, and \verb"\section*".
The definitions made in \verb"exam.cls" ensure that if a \verb"\part"
command appears \emph{outside of a parts environment} it will be
interpreted as a sectioning command, while if it appears \emph{inside
  of a parts environment} it will be interpreted as beginning a new
part of a question.

You can give these commands in the middle of a questions environment
so as not to interrupt the numbering of the questions, or you can end
a questions environment, give a sectioning command, and then start a
new questions environment (which would reset the question counter to
start again with number~1).  If you have more than one questions
environment, though, then you will be unable to use any of the
commands that create grading\indc{gradetable} tables or
point\indc{pointtable} tables (see section~\ref{sec:Tables}) or the
commands that change the headers and footers depending on whether or
not a question spans multiple pages (see
section~\ref{sec:QuesSpan}).\indc{ifincomplete}\indc{ifcontinuation}

If you give any of these commands while inside of a questions
environment, then the section titles will be indented to the same
extent that questions are indented unless they are given as the
argument of a \verb"\fullwidth"\indc{fullwidth} or
\verb"\uplevel"\indc{uplevel} command.  These commands have the
advantage, however, that the unstarred versions provide automatic
numbering of the parts or sections.


\index{naming parts of the exam|)}
\index{indentation|)}


%--------------------------------------------------------------------
%---------------------------------------------------------------------
\section{Leaving space for the answers}
\label{sec:LeaveSpace}
\index{space!for answers|(}
\index{answer space|(}

There are five different kinds of space that you can create:
\begin{itemize}
\item Blank space (see section~\ref{sec:BlankSpace}),
\item an empty box (see section~\ref{sec:EmptyBox}),
\item space filled with lines (see section~\ref{sec:FillLines}),
\item space filled with dotted lines (see
  section~\ref{sec:FillDotLines}), and
\item space filled with a grid (see section~\ref{sec:FillGrid}).
\end{itemize}
You can also create answer lines for short answer questions (see
section~\ref{sec:ShortAns}).
In section~\ref{sec:solutions} we describe five environments for
typing solutions to questions:
\begin{itemize}
\item \texttt{solution},
\item \texttt{solutionorbox},
\item \texttt{solutionorlines},
\item \texttt{solutionordottedlines}, and
\item \texttt{solutionorgrid}.
\end{itemize}
Each of those environments can take an optional argument specifying an
amount of space to be created when solutions are not being printed;
see section~\ref{sec:SolSpace}.


%--------------------------------------------------------------------
\subsection{Leaving blank space}
\label{sec:BlankSpace}
\index{blank space|(}
\index{space!blank|(}

To leave blank space on the page for the answer to a question, you
should use the \verb"\vspace*"\indc{vspace*} command.  For example,
the command \verb"\vspace*{1in}" inserts one inch of vertical space
after the line in which it appears.  (If it appears in between
paragraphs, then it inserts the space right there.)  You can also use
the \verb"\vspace"\indc{vspace} command, the difference being that
space inserted by \verb"\vspace" will be deleted if it occurs at the
top of a new page, whereas space inserted by \verb"\vspace*" will
never be deleted.

If you want to leave all the remaining space on the page blank, you
should give the commands\indc{stretch}
\begin{center}
  \begin{tabular}{l}
    \verb"\vspace*{\stretch{1}}"\\
    \verb"\newpage"
  \end{tabular}
\end{center}
If you want to equally distribute the blank space among several
questions on the page, then just put
\verb"\vspace*{\stretch{1}}"\index{space!allocating} after each of the
questions and use \verb"\newpage"\indc{newpage} to end the page.  You
can also distribute the available space in some other ratio.  For
example, to give one of the questions on the page twice as much space
as any of the others, put \verb"\vspace*{\stretch{2}}" after that
question and \verb"\vspace*{\stretch{1}}" after each of the others.

You can also leave blank space when solutions are not being printed by
using an optional argument with a \verb"solution" environment; see
section~\ref{sec:SolSpace}.

\index{space!blank|)}
\index{blank space|)}

%--------------------------------------------------------------------
\subsection{Printing an empty box}
\label{sec:EmptyBox}
\index{space!empty box|(}
\index{empty box|(}
\indcstart{makeemptybox}

You can print an empty box with the command
\begin{center}
  \verb"\makeemptybox{length}"
\end{center}
which prints an empty box of height \verb"length" whose width equals
that of the current line.  That is, the left edge is at the current
left margin (which depends on whether we're in a question, part,
subpart, or subsubpart) and the right edge is at the right margin.

\medskip

For example, if you type
\begin{verbatim}
\begin{questions}
  \question 
  In no more than one paragraph, explain why the earth is round.

  \makeemptybox{1in}
  
  \question
  \begin{parts}
    \part
    What changes to the van Allen radiation belt are needed to make
    the earth into a regular icosahedron?

    \makeemptybox{1in}

    \part
    Where should the field generator be constructed if you want one of
    the vertices to be located at the Royal Observatory at Greenwich?

    \makeemptybox{1in}

  \end{parts}
\end{questions}
\end{verbatim}
then you'll get
\begin{questions}
  \question 
  In no more than one paragraph, explain why the earth is round.

  \makeemptybox{1in}
  
  \question
  \begin{parts}
    \part
    What changes to the van Allen radiation belt are needed to make
    the earth into a regular icosahedron?

    \makeemptybox{1in}

    \part
    Where should the field generator be constructed if you want one of
    the vertices to be located at the Royal Observatory at Greenwich?

    \makeemptybox{1in}

  \end{parts}
\end{questions}

\medskip

If you want to fill the remaining space on the page with an empty box,
you should give the commands\indc{stretch}
\begin{center}
  \begin{tabular}{l}
    \verb"\makeemptybox{\stretch{1}}"\\
    \verb"\newpage"
  \end{tabular}
\end{center}
If you want to equally distribute the space for the answers among
several questions on the page, then just put
\verb"\makeemptybox{\stretch{1}}"\index{space!allocating} after each
of those questions and use \verb"\newpage"\indc{newpage} to end the
page.  You can also distribute the available space in some other
ratio.  For example, to give one of the questions on the page twice as
much space as any of the others, put \verb"\makeemptybox{\stretch{2}}"
after that question, \verb"\makeemptybox{\stretch{1}}" after each of
the others, and use \verb"\newpage" to end the page.

You can also insert an empty box when solutions are not being printed
by including an optional argument with the \verb"solutionoremptybox"
environment; see section~\ref{sec:SolSpace}.

\indcstop{makeemptybox}
\index{empty box|)}
\index{space!empty box|)}

%--------------------------------------------------------------------
\subsection{Printing lined space}
\label{sec:FillLines}
\index{space!lined|(}
\index{lined space|(}
\indcstart{fillwithlines}


You can fill space with lines with the command
\begin{center}
  \verb"\fillwithlines{length}"
\end{center}
which fills vertical space of height \verb"length" with horizontal
lines that run the length of the current line.  That is, they extend
from the current left margin (which depends on whether we're in a
question, part, subpart, or subsubpart) to the right margin.

\medskip

For example, if you type
\begin{verbatim}
\begin{questions}
  \question 
  In no more than one paragraph, explain why the earth is round.

  \fillwithlines{1in}
  
  \question
  \begin{parts}
    \part
    What changes to the van Allen radiation belt are needed to make
    the earth into a regular icosahedron?

    \fillwithlines{1in}

    \part
    Where should the field generator be constructed if you want one of
    the vertices to be located at the Royal Observatory at Greenwich?

    \fillwithlines{1in}

  \end{parts}
\end{questions}
\end{verbatim}
then you'll get
\begin{questions}
  \question 
  In no more than one paragraph, explain why the earth is round.

  \fillwithlines{1in}
  
  \question
  \begin{parts}
    \part
    What changes to the van Allen radiation belt are needed to make
    the earth into a regular icosahedron?

    \fillwithlines{1in}

    \part
    Where should the field generator be constructed if you want one of
    the vertices to be located at the Royal Observatory at Greenwich?

    \fillwithlines{1in}

  \end{parts}
\end{questions}

\medskip The distance between the lines is
\verb"\linefillheight"\indc{linefillheight}, the default value of
which is set with the
command\index{customization!\texttt{\char`\\fillwithlines}}
\begin{center}
  \verb"\setlength\linefillheight{.25in}"
\end{center}
That value can be changed by giving a new \verb"\setlength" command.
The thickness of the lines is
\verb"\linefillthickness"\indc{linefillthickness}, the default value
of which is set with the command
\begin{center}
  \verb"\setlength\linefillthickness{0.1pt}"
\end{center}
and that value can also be changed by giving a new \verb"\setlength"
command.

If you want to fill the remaining space on the page with lines, you
should give the commands\indc{stretch}
\begin{center}
  \begin{tabular}{l}
    \verb"\fillwithlines{\stretch{1}}"\\
    \verb"\newpage"
  \end{tabular}
\end{center}
If you want to equally distribute the space for the answers among
several questions on the page, then just put
\verb"\fillwithlines{\stretch{1}}"\index{space!allocating} after each
of those questions and use \verb"\newpage" to end the page.  You can
also distribute the available space in some other ratio.  For example,
to give one of the questions on the page twice as much space as any of
the others, put \verb"\fillwithlines{\stretch{2}}" after that
question, \verb"\fillwithlines{\stretch{1}}" after each of the others,
and use \verb"\newpage" to end the page.

You can also insert lined space when solutions are not being printed
by including an optional argument with the \verb"solutionorlines"
environment; see section~\ref{sec:SolSpace}.

\indcstop{fillwithlines}
\index{lined space|)}
\index{space!lined|)}

%--------------------------------------------------------------------
\subsection{Printing dotted lined space}
\label{sec:FillDotLines}
\index{space!dotted lined|(}
\index{dotted lined space|(}
\indcstart{fillwithdottedlines}


You can fill space with dotted lines with the command
\begin{center}
  \verb"\fillwithdottedlines{length}"
\end{center}
which fills vertical space of height \verb"length" with dotted
horizontal lines that run the length of the current line.  That is,
they extend from the current left margin (which depends on whether
we're in a question, part, subpart, or subsubpart) to the right
margin.  The command \verb"\fillwithdottedlines" is almost identical
to the \verb"\fillwithlines" command (see
section~\ref{sec:FillLines}), except that the lines created are dotted
and the distance between the lines is
\verb"\dottedlinefillheight"\indc{dottedlinefillheight}.  The default
value of \verb"\dottedlinefillheight" is set by the
command\index{customization!\texttt{\char`\\fillwithdottedlines}}
\begin{center}
  \verb"\setlength\dottedlinefillheight{.25in}"
\end{center}
and that value can be changed by giving a new \verb"\setlength"
command.

For example, if you type
\begin{verbatim}
\begin{questions}
  \question 
  In no more than one paragraph, explain why the earth is round.

  \fillwithdottedlines{1in}
  
  \question
  \begin{parts}
    \part
    What changes to the van Allen radiation belt are needed to make
    the earth into a regular icosahedron?

    \fillwithdottedlines{1in}

    \part
    Where should the field generator be constructed if you want one of
    the vertices to be located at the Royal Observatory at Greenwich?

    \fillwithdottedlines{1in}

  \end{parts}
\end{questions}
\end{verbatim}
then you'll get
\begin{questions}
  \question 
  In no more than one paragraph, explain why the earth is round.

  \fillwithdottedlines{1in}
  
  \question
  \begin{parts}
    \part
    What changes to the van Allen radiation belt are needed to make
    the earth into a regular icosahedron?

    \fillwithdottedlines{1in}

    \part
    Where should the field generator be constructed if you want one of
    the vertices to be located at the Royal Observatory at Greenwich?

    \fillwithdottedlines{1in}

  \end{parts}
\end{questions}

If you want to fill the remaining space on the page with dotted lines,
you should give the commands\indc{stretch}
\begin{center}
  \begin{tabular}{l}
    \verb"\fillwithdottedlines{\stretch{1}}"\\
    \verb"\newpage"
  \end{tabular}
\end{center}
If you want to equally distribute the space for the answers among
several questions on the page, then just put
\verb"\fillwithdottedlines{\stretch{1}}"\index{space!allocating} after
each question on the page and use \verb"\newpage" to end the page.
You can also distribute the available space in some other ratio.  For
example, to give one of the questions on the page twice as much space
as any of the others, put \verb"\fillwithdottedlines{\stretch{2}}"
after that question, \verb"\fillwithdottedlines{\stretch{1}}" after
each of the others, and use \verb"\newpage" to end the page.

You can also insert dotted lined space when solutions are not being
printed by including an optional argument with the
\verb"solutionordottedlines" environment; see
section~\ref{sec:SolSpace}.

\indcstop{fillwithdottedlines}
\index{dotted lined space|)}
\index{space!dotted lined|)}

%--------------------------------------------------------------------
\subsection{Printing space filled with a grid}
\label{sec:FillGrid}
\index{space!grid|(}
\index{grid space|(}
\indcstart{fillwithgrid}


You can fill space with a grid with the command
\begin{center}
  \verb"\fillwithgrid{length}"
\end{center}
which fills vertical space of height \texttt{length} with a grid that
runs the length of the current line.  That is, it extends from the
current left margin (which depends on whether we're in a question,
part, subpart, or subsubpart) to the right margin.  
The default grid size and grid line thickness were set by the
commands
\begin{center}
  \begin{tabular}{l}
    \verb"\setlength{\gridsize}{5mm}"\\
    \verb"\setlength{\gridlinewidth}{0.1pt}"
  \end{tabular}
\end{center}
\index{customization!\texttt{\char`\\fillwithgrid}} You can change
either or both of those by giving new \verb"\setlength" commands.  The
period of the grid is \verb"\gridsize" (both horizontally and
vertically).  That is, the horizontal distance from the left edge of
one vertical line to the left edge of the next vertical line is
\verb"\gridsize", as is the vertical distance from the top edge of one
horizontal line to the top edge of the next horizontal line.  Thus,
each square has outer side length equal to
\verb"\gridsize+\gridlinewidth".

By default, the created grids are in black.  However, if you give the
commands\indc{colorgrids}
\begin{center}
  \begin{tabular}{l}
    \verb"\usepackage{color}"\\
    \verb"\colorgrids"
  \end{tabular}
\end{center}
then the grids will be in color, by default a light gray.  That
default color was defined by the
command\indc{definecolor}\indt{GridColor}
\begin{center}
  \verb"\definecolor{GridColor}{gray}{0.8}"
\end{center}
\index{customization!\texttt{\char`\\fillwithgrid}} You can change the
color by giving a new \verb"\definecolor" command to redefine the
color \texttt{GridColor}.


For example, if you type
\begin{verbatim}
\begin{questions}
  \question 
  In no more than one paragraph, explain why the earth is round.

  \smallskip\fillwithgrid{1in}
  
  \question
  \begin{parts}
    \part
    What changes to the van Allen radiation belt are needed to make
    the earth into a regular icosahedron?

    \smallskip\fillwithgrid{1in}

    \part
    Where should the field generator be constructed if you want one of
    the vertices to be located at the Royal Observatory at Greenwich?

    \smallskip\fillwithgrid{1in}

  \end{parts}
\end{questions}
\end{verbatim}
then you'll get
\begin{questions}
  \question 
  In no more than one paragraph, explain why the earth is round.

  \smallskip\fillwithgrid{1in}
  
  \question
  \begin{parts}
    \part
    What changes to the van Allen radiation belt are needed to make
    the earth into a regular icosahedron?

    \smallskip\fillwithgrid{1in}

    \part
    Where should the field generator be constructed if you want one of
    the vertices to be located at the Royal Observatory at Greenwich?

    \smallskip\fillwithgrid{1in}

  \end{parts}
\end{questions}

If you want to fill the remaining space on the page with a grid, you
should give the commands\indc{stretch}
\begin{center}
  \begin{tabular}{l}
    \verb"\fillwithgrid{\stretch{1}}"\\
    \verb"\newpage"
  \end{tabular}
\end{center}
If you want to equally distribute the space for the answers among
several questions on the page, then just put
\verb"\fillwithgrid{\stretch{1}}"\index{space!allocating} after each
question on the page and use \verb"\newpage" to end the page.  You can
also distribute the available space in some other ratio.  For example,
to give one of the questions on the page twice as much space as any of
the others, put \verb"\fillwithgrid{\stretch{2}}" after that question,
\verb"\fillwithgrid{\stretch{1}}" after each of the others, and use
\verb"\newpage" to end the page.

You can also insert a grid only when solutions are not being
printed by including an optional argument with the
\verb"solutionorgrid" environment; see
section~\ref{sec:SolSpace}.

%--------------------------------------------------------------------
\subsubsection{Covering every page with a grid}
\label{sec:CoverWGrid}
\index{grid!covering every page}

If you'd like \emph{every} page to be completely covered by a grid,
with the exam printed right on top of that grid, you can do that by
using a \verb"\fillwithgrid" command in the footer.  For this, make
sure you're using either \verb"\pagestyle{headandfoot}" (the default)
or \verb"\pagestyle{foot}" (see section~\ref{sec:pagestyle}), and put
the following into your preamble (i.e., after the
\verb"\documentclass" command and before the \verb"\begin{document}"
  command):
\begin{verbatim}
\usepackage{color}
\colorgrids
\newlength{\scratchlength}
\newcommand{\gridfromfoot}{%
  \setlength{\scratchlength}{\textheight}%
  \addtolength{\scratchlength}{\baselineskip}%
  \leavevmode\rlap{\vbox to 0pt{%
      \vss
      \fillwithgrid{\scratchlength}%
      \vspace*{\footskip}%
    }% vbox
  }% rlap
}% gridfromfoot

\footer{\gridfromfoot Math 115}
       {}
       {Page \thepage\ of \numpages}
\end{verbatim}


%--------------------------------------------------------------------
\subsubsection{Printing graph paper}
\label{sec:GraphPaper}

You can also use the \verb"\fillwithgrid" command to print graph
paper:\index{graph paper}
\begin{verbatim}
\documentclass{exam}
\pagestyle{empty}
\begin{document}

\fillwithgrid{\stretch{1}}

\newpage

\fillwithgrid{\stretch{1}}

\end{document}
\end{verbatim}



\indcstop{fillwithgrid}
\index{grid space|)}
\index{space!grid|)}


%--------------------------------------------------------------------
\subsection{Short answer questions}
\label{sec:ShortAns}
\indcstart{answerline}
\index{short answers|(}

The exam class can print answer lines for short answer questions.  The
command
\begin{center}
  \verb"\answerline"
\end{center}
leaves blank vertical space of length
\verb"\answerskip"\indc{answerskip} and then inserts a horizontal line
of length \verb"\answerlinelength"\indc{answerlinelength}%
\index{customization!\texttt{\char`\\answerline}} at the right margin,
preceded by the number of the current question, part, subpart, or
subsubpart.  The default values of \verb"\answerskip" and
\verb"\answerlinelength" are set by the
commands\index{customization!\texttt{\char`\\answerline}}
\begin{center}
  \begin{tabular}{l}
    \verb"\setlength\answerskip{2ex}"\\
    \verb"\setlength\answerlinelength{1in}"
  \end{tabular}
\end{center}
and these can be changed by giving new \verb"\setlength" commands.
If you use the command \verb"\answerline" outside of a
\verb"questions" environment, then the number before the line will be
omitted.

For example, if you type
\begin{verbatim}
\begin{questions}
  \question
  Who's buried in Grant's tomb?
  \answerline

  \question
  What was the color of George Washington's white horse?
  \answerline

  \question
  Which is heavier: A pound of feathers, or a pound of lead?
  \answerline
\end{questions}
\end{verbatim}
then you'll get
\begin{questions}
  \question
  Who's buried in Grant's tomb?
  \answerline

  \question
  What was the color of George Washington's white horse?
  \answerline

  \question
  Which is heavier: A pound of feathers, or a pound of lead?
  \answerline
\end{questions}
and if you type
\begin{verbatim}
\begin{questions}
  \question
  Answer the following questions on the lines at the right:  
  \begin{parts}
    \part
    Who's buried in Grant's tomb? \answerline
  
    \part
    \begin{subparts}
      \subpart
      What light through yonder window breaks? \answerline

      \subpart
      To be, or not to be; that is the question? \answerline
    \end{subparts}
  
    \part
    What was the color of George Washington's white horse? \answerline
  \end{parts}
\end{questions}
\end{verbatim}
then you'll get
\begin{questions}
  \question
  Answer the following questions on the lines at the right:  
  \begin{parts}
    \part
    Who's buried in Grant's tomb? \answerline
  
    \part
    \begin{subparts}
      \subpart
      What light through yonder window breaks? \answerline

      \subpart
      To be, or not to be; that is the question? \answerline
    \end{subparts}
  
    \part
    What was the color of George Washington's white horse? \answerline
  \end{parts}
\end{questions}

%--------------------------------------------------------------------
\subsection{Printing the answers to short answer questions}
\label{sec:AnsShortAns}
\index{short answers!answers to|(}
\index{answers!to short answers|(}

The \verb"\answerline"\indcsub{answerline}{optional~argument} command
can take an optional argument so that you can have the answer to the
question printed on the answer line when answers are being printed
(see section~\ref{sec:answers}).  When answers are not being printed,
the optional argument is ignored.

When answers are being printed, the optional argument is printed
subject to the declarations in the argument of the last
\verb"\CorrectChoiceEmphasis"\indc{CorrectChoiceEmphasis} command
(which by default is
\verb"\bfseries"\index{customization!\texttt{\char`\\CorrectChoice}};
see section~\ref{sec:CorrectChoice}), and it is printed a distance of
\verb"\answerclearance"\indc{answerclearance}%
\index{customization!\texttt{\char`\\answerline}} above the line.  The
default value of \verb"\answerclearance" is set with the command
\begin{center}
  \verb"\setlength\answerclearance{0.2ex}"
\end{center}
and it can be changed with a \verb"\setlength" command.  The optional
argument is centered on the answer line unless it is longer than the
line, in which case it extends to the right of the line.

For example, if you type
\begin{verbatim}
\begin{questions}
  \question Who was the fifth Beatle? \answerline[Murray the K]
  \question What is the answer to the ultimate question of life, the
  universe, and everything? \answerline[42]
\end{questions}
\end{verbatim}
and if \verb"\noprintanswers" (the default) is in effect, you'll get
\begin{questions}
  \question Who was the fifth Beatle? \answerline[Murray the K]
  \question What is the answer to the ultimate question of life, the
  universe, and everything? \answerline[42]
\end{questions}
while if \verb"\printanswers" is in effect, you'll get \printanswers
\begin{questions}
  \question Who was the fifth Beatle? \answerline[Murray the K]
  \question What is the answer to the ultimate question of life, the
  universe, and everything? \answerline[42]
\end{questions}
\noprintanswers

\index{answers!to short answers|)}
\index{short answers!answers to|)}

\index{short answers|)}
\indcstop{answerline}

\index{answer space|)}
\index{space!for answers|)}

%--------------------------------------------------------------------
%--------------------------------------------------------------------
\section{Solutions}
\label{sec:solutions}

%--------------------------------------------------------------------
\subsection{Solution environments}
\label{sec:SolEnv}
\index{solutions|(}
\index{solution environment|(}
\index{answers|(}
\index{environment!solution|(}
\index{environment!solutionorbox|(}
\index{environment!solutionorlines|(}
\index{environment!solutionordottedlines|(}
\index{environment!solutionorgrid|(}
\index{solution environment|(}
\index{solutionorbox environment|(}
\index{solutionorlines environment|(}
\index{solutionordottedlines environment|(}


There are six environments for typing solutions to the problems.
These environments all allow you to either print the solutions or omit
them with a single command.  Whether or not solutions are
printed is controlled by the two
commands\indc{printanswers}\indc{noprintanswers}
\begin{center}
  \begin{tabular}{l}
    \verb"\printanswers"\\
    \verb"\noprintanswers"
  \end{tabular}
\end{center}
Using these commands to turn printing of solutions on and off, you can
print the solutions for the entire exam or for only certain parts of
it.  The default is that solutions are \emph{not} printed.  There is
also an \verb"answers"\indt{answers}\indcsub{documentclass}{options}
option to the \verb"exam" document class, used as in
\begin{center}
  \verb"\documentclass[answers]{exam}"
\end{center}
that is equivalent to giving the command
\verb"\printanswers"\indc{printanswers} at the beginning of the
document (see section~\ref{sec:answers}).

The six environments for solutions are divided into two types.
\begin{itemize}
\item The \texttt{solutionorbox} environment (see
  section~\ref{sec:solutionbox}) always prints a box, of whatever size
  you choose, and if solutions are being printed it prints the
  solution inside of the box.
\item Each of the remaining five (see section~\ref{sec:SolSpace}),
  \begin{center}
    \begin{tabular}{rll}
      the& \verb"solution"& environment,\\
      the& \verb"solutionorbox"& environment,\\
      the& \verb"solutionorlines"& environment,\\
      the& \verb"solutionordottedlines"& environment, and\\
      the& \verb"solutionorgrid"& environment,
    \end{tabular}
  \end{center}
  by default prints either the solution or nothing at all.  However,
  they all can take one optional argument, which is an amount of space
  to be left when solutions are not being printed.  If you use these
  environments without any optional arguments, then they are
  identical: They print either the solution or nothing at all.  They
  differ only in the type of space they leave when you do include the
  optional argument and solutions are not being printed.


  The differences between these five environments are that if the
  optional argument is used and solutions are not being printed,
  then\index{solution environment!optional argument}
  \begin{itemize}
  \item the \verb"solution" enviroment inserts that amount of blank
    space, as if you had given a \verb"\vspace*"\indc{vspace*} command
    (see section~\ref{sec:BlankSpace}),
  \item the \verb"solutionorbox" environment inserts an empty box of
    that height, as if you had given a
    \verb"\makeemptybox"\indc{makeemptybox} command (see
    section~\ref{sec:EmptyBox}),
  \item the \verb"solutionorlines" environment inserts that amount of
    space with ruled lines, as if you had given a
    \verb"\fillwithlines"\indc{fillwithlines} command (see
    section~\ref{sec:FillLines}),
  \item the \verb"solutionordottedlines" environment inserts that
    amount of space with dotted lines, as if you had given a
    \verb"\fillwithdottedlines"\indc{fillwithdottedlines} command (see
    section~\ref{sec:FillDotLines}), and
  \item the \verb"solutionorgrid" environment inserts that amount of
    space filled with a grid, as if you had given a
    \verb"\fillwithgrid"\indc{fillwithgrid} command (see
    section~\ref{sec:FillGrid}).
  \end{itemize}
\end{itemize}

You can use a parts, subparts, or subsubparts environment inside any
of the solution environments without having their points (if any)
counted as actual points on the exam.  If there are points assigned to
any of these commands inside any of the solution environments, those
points are not added to the points of the question or the points on
the page, and do not affect any gradetables or pointtables.


\index{solution environment|)}
\index{solutionorbox environment|)}
\index{solutionorlines environment|)}
\index{solutionordottedlines environment|)}
\index{environment!solution|)}
\index{environment!solutionorbox|)}
\index{environment!solutionorlines|)}
\index{environment!solutionordottedlines|)}
\index{environment!solutionorgrid|)}

%--------------------------------------------------------------------
\subsection{The appearance of the solution}
\label{sec:SolAppear}
\index{solution!appearance|(}

The solutions printed by the 
\begin{center}
  \begin{tabular}{l}
    \verb"solution",\\
    \verb"solutionorbox",\\
    \verb"solutionorlines",\\
    \verb"solutionordottedlines", and\\
    \verb"solutionorgrid"
  \end{tabular}
\end{center}
environments can be printed either
\begin{itemize}
\item in a box (i.e., an \verb"\fbox") (see section~\ref{sec:SolBox};
  this is the default),
\item on a shaded background (see section~\ref{sec:shaded}), or
\item with no framing or shading (see section~\ref{sec:SolPlain}).
\end{itemize}
In all of these cases, solutions can be broken across pages, and the
part on each page will be boxed or shaded as appropriate.


%--------------------------------------------------------------------
\subsubsection{Printing the solution in a box}
\label{sec:SolBox}
\index{solution!in a box|(}

By default, the solution is printed inside of a box (i.e., an
\verb"\fbox"\indc{fbox}), and if the solution is broken across pages,
then each piece is enclosed in a box.  There is also a
\verb"\shadedsolutions"\indc{shadedsolutions} command to instead have
the solution printed in a \verb"\colorbox"\indc{colorbox} (i.e.,
printed on a shaded background); for this, you must load the
%
\verb"color"\index{color package} package and your printer must be
capable of printing color, or at least grayscale (see
section~\ref{sec:shaded}).

\medskip

For example, if you type
\begin{verbatim}
\begin{questions}
  \question Did you mutter something?

  \begin{solution}
    Once upon a midnight dreary, while I pondered, weak and weary,
    Over many a quaint and curious volume of forgotten lore--- While I
    nodded, nearly napping, suddenly there came a tapping, As of some
    one gently rapping, rapping at my chamber door.  ``\,'Tis some
    visitor,'' I muttered, ``tapping at my chamber door--- Only this
    and nothing more.''
  \end{solution}
\end{questions}
\end{verbatim}
and you've given the command \verb"\printanswers"\indc{printanswers}
or used the document class option \verb"answers"\indt{answers} (see
section~\ref{sec:answers}), then you'll get \bigskip
\begin{questions}
  \question Did you mutter something?

  \begin{TheSolution}
    Once upon a midnight dreary, while I pondered, weak and weary,
    Over many a quaint and curious volume of forgotten lore--- While I
    nodded, nearly napping, suddenly there came a tapping, As of some
    one gently rapping, rapping at my chamber door.  ``\,'Tis some
    visitor,'' I muttered, ``tapping at my chamber door--- Only this
    and nothing more.''
  \end{TheSolution}
\end{questions}
The result would be exactly the same if that \texttt{solution}
environment were replaced by a \texttt{solutionorbox},
\texttt{solutionorlines}, \texttt{solutionordottedlines}, or
\texttt{solutionorgrid} environment.

By default, the solution is printed in a box whose width equals that
of the text of the current question (or part, or subpart, or
subsubpart).  That is, the indentation at the left of the solution
equals the current level of indentation.  You can change this by
enclosing the \texttt{solution}, \texttt{solutionorbox},
\texttt{solutionorlines}, \texttt{solutionordottedlines}, or
\texttt{solutionorgrid} environment in the argument of a
\verb"\fullwidth"\indc{fullwidth} or \verb"\uplevel"\indc{uplevel}
command, or inside of a \texttt{EnvFullwidth}\indt{EnvFullwidth}%
\index{EnvFullwidth environment}\index{environment!EnvFullwidth} or
%
\texttt{EnvUplevel}\indt{EnvUplevel}\index{EnvUplevel environment}%
\index{environment!EnvUplevel}%
environment (see section~\ref{sec:uplevel}).  For example, if you type
\begin{verbatim}
\begin{questions}
  \question Did you mutter something?

  \begin{EnvFullwidth}
    \begin{TheSolution}
      Once upon a midnight dreary, while I pondered, weak and weary,
      Over many a quaint and curious volume of forgotten lore--- While
      I nodded, nearly napping, suddenly there came a tapping, As of
      some one gently rapping, rapping at my chamber door.  ``\,'Tis
      some visitor,'' I muttered, ``tapping at my chamber door--- Only
      this and nothing more.''
    \end{TheSolution}
  \end{EnvFullwidth}
\end{questions}
\end{verbatim}
and you've given the command \verb"\printanswers"\indc{printanswers}
or used the document class option \verb"answers"\indt{answers}, then
you'll get \bigskip
\begin{questions}
  \question Did you mutter something?

  \begin{EnvFullwidth}
    \begin{TheSolution}
      Once upon a midnight dreary, while I pondered, weak and weary,
      Over many a quaint and curious volume of forgotten lore--- While
      I nodded, nearly napping, suddenly there came a tapping, As of
      some one gently rapping, rapping at my chamber door.  ``\,'Tis
      some visitor,'' I muttered, ``tapping at my chamber door--- Only
      this and nothing more.''
    \end{TheSolution}
  \end{EnvFullwidth}
\end{questions}

With all of these environments, if solutions are not being printed
then the contents of the environment are ignored.  If, however, you
include the optional argument (see section~\ref{sec:SolSpace}), then
the requested amount of space (blank, lined, or dotted lined) will be
inserted when the solution isn't printed.

\index{solution!in a box|)}

%--------------------------------------------------------------------
\subsubsection{Printing the solution on a shaded background}
\label{sec:shaded}
\index{solution!shaded|(}

If you load the \verb"color" package with the
command\index{color package}
\begin{center}
  \verb"\usepackage{color}"
\end{center}
in the preamble of your document (i.e., after the
\verb"\documentclass" command and before the \verb"\begin{document}"
  command), then you can give the command\indc{shadedsolutions}
\begin{center}
  \verb"\shadedsolutions"
\end{center}
This will cause solutions to be printed on a shaded background, which
by default is a light gray.  (Most laser printers can at least print
grayscale when printing with the correct driver.)  If you want to
change the color of the shading, you can do so by redefining the color
\verb"SolutionColor"\indc{SolutionColor}.  For example, if you give
the command
\begin{center}
  \verb"\definecolor{SolutionColor}{rgb}{0.8,0.9,1}"
\end{center}
then the solutions will be printed on a light blue background
(assuming that your printer is capable of producing that color).

If you want to switch back to printing solutions inside of an
\verb"\fbox"\indc{fbox}, you can do so by giving the
command\indc{framedsolutions}
\begin{center}
  \verb"\framedsolutions"
\end{center}
which returns you to the default situation.

\index{solution!shaded|)}

%--------------------------------------------------------------------
\subsubsection{Printing the solution with no framing or shading}
\label{sec:SolPlain}
\index{solution!no framing or shading|(}

If you give the command\indc{unframedsolutions}
\begin{center}
  \verb"\unframedsolutions"
\end{center}
then solutions will be printed as normal text, with no framing or
shading.  If you use this option, you'll almost certainly want to use
the command \verb"\SolutionEmphasis"\indc{SolutionEmphasis} (see
section~\ref{sec:SolEmph}) to distinguish the solution from the
surrounding text.

If you want to switch back to printing solutions inside of an
\verb"\fbox", you can do so by giving the
command\indc{framedsolutions}
\begin{center}
  \verb"\framedsolutions"
\end{center}
which returns you to the default situation.

\index{solution!no framing or shading|)}

%--------------------------------------------------------------------
\subsection{Customizing the solution}
\label{sec:SolCust}
\index{customization!solution|(}
\index{solution!customizing|(}

The solutions printed by any of the solution environments (see
section~\ref{sec:SolEnv}) can be customized:
\begin{itemize}
\item the title (by default, ``\textbf{Solution:}'') can be changed
  (see section~\ref{sec:SolTitle}),
\item emphasis can be applied to the solution to distinguish it from
  the surrounding text (see section~\ref{sec:SolEmph}), and
\item advanced users can entirely change the appearance of the
  solutions by defining a \LaTeX{} environment (see
  section~\ref{sec:AdvCust}).
\end{itemize}

%--------------------------------------------------------------------
\subsubsection{The title}
\label{sec:SolTitle}

The word ``\textbf{Solution:}'' that you see at the beginning of the
solutions printed by the \texttt{solution}, \texttt{solutionorbox},
\texttt{solutionorlines}, and \texttt{solutionordottedlines}
environments can be changed by redefining the command
\verb"\solutiontitle"\indc{solutiontitle}.  The default value was
created by the command
\begin{center}
\verb"\newcommand{\solutiontitle}{\noindent\textbf{Solution:}\enspace}"
\end{center}
If, for example, you give the command
\begin{center}
\verb"\renewcommand{\solutiontitle}{\noindent\textbf{Solution:}\par\noindent}"
\end{center}
then the \verb"solution" environment above would print
\renewcommand{\solutiontitle}{\noindent\textbf{Solution:}\par\noindent}
\par\bigskip
\begin{questions}
\question Did you mutter something?

\begin{TheSolution}
  Once upon a midnight dreary, while I pondered, weak and weary, Over
  many a quaint and curious volume of forgotten lore--- While I
  nodded, nearly napping, suddenly there came a tapping, As of some
  one gently rapping, rapping at my chamber door.  ``\,'Tis some
  visitor,'' I muttered, ``tapping at my chamber door--- Only this and
  nothing more.''
\end{TheSolution}
\end{questions}
\renewcommand{\solutiontitle}{\noindent\textbf{Solution:}\enspace}

%--------------------------------------------------------------------
\subsubsection{Solution emphasis}
\label{sec:SolEmph}

You can change the appearance of the text of the solution with the
\verb"\SolutionEmphasis"\indc{SolutionEmphasis} command.  The contents
of the solution are enclosed in a group, and that group is begun with
the argument of the most recent
\verb"\SolutionEmphasis"\indc{SolutionEmphasis} command.  The default
was created by the command
\begin{center}
  \verb"\SolutionEmphasis{}"
\end{center}
That is, the default is not to add any emphasis at all.  If, for
example, you give the command
\begin{center}
  \verb"\SolutionEmphasis{\itshape\small}"
\end{center}
then the solutions will be set in italics, in small type.  For another
example, if you've given the command \verb"\usepackage{color}" in the
preamble of your document (i.e., after the \verb"\documentclass"
command and before the \verb"\begin{document}") and you give the
command\indc{SolutionEmphasis}
\begin{center}
  \verb"\SolutionEmphasis{\color{red}}"
\end{center}
then the solutions will be printed in red (assuming your printer is
capable of producing that color).

\textbf{Note:} If you've given the command
\verb"\unframedsolutions"\indc{unframedsolutions} (see
section~\ref{sec:SolPlain}), so that solutions are printed as normal
text with no framing or shading, then you'll almost certainly want to
use the \verb"\SolutionEmphasis"\indc{SolutionEmphasis} command to
create some visible difference between the solutions and the
surrounding text.


%--------------------------------------------------------------------
\subsubsection{Advanced customizations}
\label{sec:AdvCust}
\index{TheSolution environment}
\index{environment!TheSolution}

The appearance of all of these solution environments can actually be
totally customized, if you're up to defining a \LaTeX{} environment.
The appearance of the solution typed into any of these environments is
defined by an environment called \verb"TheSolution"\indt{TheSolution},
and you can change the definition of \verb"TheSolution" with a
\verb"\renewenvironment" command.  That is, the \verb"solution"
environment decides whether the command \verb"\printanswers" has been
given and, if so, it feeds the contents of the \verb"solution"
environment into a \verb"TheSolution" environment.  (The behavior of a
\texttt{solutionorbox}, \texttt{solutionorlines} or
\texttt{solutionordottedlines} environment is similar.)  If you use
\verb"\renewenvironment" to change the definition of the
\verb"TheSolution" environment, you'll be changing the way the
solution is printed.  For the default definition of the
\verb"TheSolution" environment, see the file \verb"exam.cls".

\index{solution!customizing|)}
\index{customization!solution|)}

\index{solution!appearance|)}

%--------------------------------------------------------------------
\subsection{Leaving space for the answers when solutions aren't
  printed}
\label{sec:SolSpace}
\index{space!for answers|(}
\index{answer space|(}
\index{solution environment!optional argument|(}

Each of the \texttt{solution}, \texttt{solutionorbox},
\texttt{solutionorlines}, \texttt{solutionordottedlines}, and
\texttt{solutionorgrid} environments takes an optional argument:
\begin{itemize}
\item In a
  %
  \verb"solution"\indtsub{solution}{optional argument}
  %
  environment this is an amount of blank space to be left (just as if
  you had used a \verb"\vspace*"\indc{vspace*} command; see
  section~\ref{sec:BlankSpace}) when solutions are not being printed,
\item in a
  %
  \verb"solutionorbox"\indtsub{solutionorbox}{optional argument}
  %
  environment this is the height of an empty box to be drawn (just as
  if you had used a \verb"\makeemptybox"\indc{makeemptybox} command;
  see section~\ref{sec:EmptyBox}) when solutions are not being
  printed,
\item in a
  %
  \verb"solutionorlines"\indtsub{solutionorlines}{optional argument}
  %
  environment this is an amount of lined space to be left (just as if
  you had used a \verb"\fillwithlines"\indc{fillwithlines} command;
  see section~\ref{sec:FillLines}) when solutions are not being
  printed,
\item in a\indtsub{solutionordottedlines}{optional argument}
  \verb"solutionordottedlines" environment this is an amount of dotted
  lined space to be left (just as if you had used a
  \verb"\fillwithdottedlines"\indc{fillwithdottedlines} command; see
  section~\ref{sec:FillDotLines}) when solutions are not being
  printed.
\item in a\indtsub{solutionorgrid}{optional argument}
  \verb"solutionorgrid" environment this is an amount of space to be
  filled with a grid (just as if you had used a
  \verb"\fillwithgrid"\indc{fillwithgrid} command; see
  section~\ref{sec:FillGrid}) when solutions are not being
  printed.
\end{itemize}
For example, if you type
\begin{verbatim}
\begin{solution}[2in]
  Once upon a midnight dreary, while I pondered, weak and weary, Over
  many a quaint and curious volume of forgotten lore--- While I
  nodded, nearly napping, suddenly there came a tapping, As of some
  one gently rapping, rapping at my chamber door.  ``\,'Tis some
  visitor,'' I muttered, ``tapping at my chamber door--- Only this and
  nothing more.''
\end{solution}
\end{verbatim}
then there will be 2 inches of blank space inserted (as if you had
given the command \verb"\vspace*{2in}"; see
section~\ref{sec:BlankSpace}) when solutions are not being printed, if
you type
\begin{verbatim}
\begin{solutionorbox}[2in]
  Once upon a midnight dreary, while I pondered, weak and weary, Over
  many a quaint and curious volume of forgotten lore--- While I
  nodded, nearly napping, suddenly there came a tapping, As of some
  one gently rapping, rapping at my chamber door.  ``\,'Tis some
  visitor,'' I muttered, ``tapping at my chamber door--- Only this and
  nothing more.''
\end{solutionorbox}
\end{verbatim}
then there will be an empty box 2 inches tall inserted (as if you had
given the command \verb"\makeemptybox{2in}"; see
section~\ref{sec:EmptyBox}) when solutions are not being printed, if
you type
\begin{verbatim}
\begin{solutionorlines}[2in]
  Once upon a midnight dreary, while I pondered, weak and weary, Over
  many a quaint and curious volume of forgotten lore--- While I
  nodded, nearly napping, suddenly there came a tapping, As of some
  one gently rapping, rapping at my chamber door.  ``\,'Tis some
  visitor,'' I muttered, ``tapping at my chamber door--- Only this and
  nothing more.''
\end{solutionorlines}
\end{verbatim}
then there will be 2 inches of lined space inserted (as if you had
given the command \verb"\fillwithlines{2in}"; see
section~\ref{sec:FillLines}) when solutions are not being printed, if
you type
\begin{verbatim}
\begin{solutionordottedlines}[2in]
  Once upon a midnight dreary, while I pondered, weak and weary, Over
  many a quaint and curious volume of forgotten lore--- While I
  nodded, nearly napping, suddenly there came a tapping, As of some
  one gently rapping, rapping at my chamber door.  ``\,'Tis some
  visitor,'' I muttered, ``tapping at my chamber door--- Only this and
  nothing more.''
\end{solutionordottedlines}
\end{verbatim}
then there will be 2 inches of dotted lined space inserted (as if you
had given the command \verb"\fillwithdottedlines{2in}"; see
section~\ref{sec:FillDotLines}) when solutions are not being printed.
and if you type
\begin{verbatim}
\begin{solutionorgrid}[2in]
  Once upon a midnight dreary, while I pondered, weak and weary, Over
  many a quaint and curious volume of forgotten lore--- While I
  nodded, nearly napping, suddenly there came a tapping, As of some
  one gently rapping, rapping at my chamber door.  ``\,'Tis some
  visitor,'' I muttered, ``tapping at my chamber door--- Only this and
  nothing more.''
\end{solutionordottedlines}
\end{verbatim}
then there will be 2 inches of grid inserted (as if you had given the
command \verb"\fillwithgrid{2in}"; see section~\ref{sec:FillGrid})
when solutions are not being printed.


%--------------------------------------------------------------------
\subsubsection{Cancelling the space from the optional arguments}
\label{sec:cancelspace}
\indcstart{cancelspace}
\indcstart{nocancelspace}

If you've typed any of the environments \texttt{solution},
\texttt{solutionorbox}, \texttt{solutionorlines},
\texttt{solutionordottedlines}, or \texttt{solutionorgrid} (see
section~\ref{sec:SolEnv}), and you've
\begin{itemize}
\item included an optional argument to leave space for answers when
  solutions are not being printed, but
\item you would now like to print the exam \emph{without} printing the
  solutions and \emph{without} leaving space for the answers,
\end{itemize}
then you can cancel the space that would be created by the optional
arguments by using the
\texttt{cancelspace}\indt{cancelspace}\indcsub{documentclass}{options}
document class option.  That is, if you use the \verb"\documentclass"
command
\begin{center}
  \verb"\documentclass[cancelspace]{exam}"
\end{center}
then optional arguments to any of the four solution environments will
be ignored.  There are also commands \verb"\cancelspace" and
\verb"\nocancelspace"; using the document class option
\texttt{cancelspace} is equivalent to giving the command
\verb"\cancelspace" at the beginning of the file.  The command
\verb"\nocancelspace" reverts to the default situation (in which the
optional arguments are not ignored).

The document class option \texttt{cancelspace} has no effect at all if
solutions are being printed.

\emph{Note:} The document class option \texttt{cancelspace} and the
command \verb"\cancelspace" only affect the optional arguments of of
the five solution environments; they do \emph{not} affect any
\verb"\vspace" commands that you put into the file.

\indcstop{nocancelspace}
\indcstop{cancelspace}

\index{solution environment!optional argument|)}

%--------------------------------------------------------------------
\subsection{The solutionbox environment}
\label{sec:solutionbox}
\indt{solutionbox}
\index{solution!in a box}

The solutionbox environment is different from the other solution
environments (\texttt{solution}, \texttt{solutionorbox},
\texttt{solutionorlines}, \texttt{solutionordottedlines}, and
\texttt{solutionorgrid}), in that
\begin{enumerate}
\item The box is always printed, whether answers are being printed or
  not.
\item The argument giving the size of the box is a required argument,
  not an optional argument, and so it should be enclosed in braces,
  not in brackets.  It can be either a length or
  \verb"\stretch{number}".  Note: \emph{It is the user's
    responsibility to be sure that the box is large enough to hold the
    solution!}  If the solution takes up too much vertical space, then
  it will spill out of the bottom of the box, overwriting whatever
  follows the box.
\item The \texttt{solutionbox} environment makes no use of the
  \texttt{TheSolution} environment; the \texttt{solutionbox}
  environment is completely freestanding.
\end{enumerate}
If answers are not being printed then only the box is printed, with
nothing in it.  If answers are being printed, then the solution is
printed inside of the box.

For example, if you type
\begin{verbatim}
\begin{solutionbox}{2in}
  Once upon a midnight dreary, while I pondered, weak and weary, Over
  many a quaint and curious volume of forgotten lore--- While I
  nodded, nearly napping, suddenly there came a tapping, As of some
  one gently rapping, rapping at my chamber door.  ``\,'Tis some
  visitor,'' I muttered, ``tapping at my chamber door--- Only this and
  nothing more.''
\end{solutionbox}
\end{verbatim}
(note that the amount of space is required, and is typed as
``\verb"{2in}"'', rather than ``\verb"[2in]"'') and solutions are not
being printed, then you'll get
\begin{solutionbox}{2in}
  Once upon a midnight dreary, while I pondered, weak and weary, Over
  many a quaint and curious volume of forgotten lore--- While I
  nodded, nearly napping, suddenly there came a tapping, As of some
  one gently rapping, rapping at my chamber door.  ``\,'Tis some
  visitor,'' I muttered, ``tapping at my chamber door--- Only this and
  nothing more.''
\end{solutionbox}
If solutions \emph{are} being printed, then the exact same box would
be printed, but it would have the solution printed inside of it.



\index{answer space|)}
\index{space!for answers|)}


%--------------------------------------------------------------------
\subsection{Changes depending on whether or not solutions are
  being printed}
\label{sec:ifprintanswers}
\indcstart{ifprintanswers}

The command \verb"\ifprintanswers" is provided in case you want to
vary what appears on the exam in ways other than those provided by the
solution environments (see section~\ref{sec:solutions}), the
\verb"\CorrectChoice" command in multiple choice environments (see
section~\ref{sec:CorrectChoice}), and the optional argument to the
\verb"\answerline" command (see section~\ref{sec:AnsShortAns}).  You
use this command by typing
\begin{verbatim}
\ifprintanswers
  Stuff to appear only when answers are being printed.
\else
  Stuff to appear only when answers are not being printed.
\fi
\end{verbatim}
The \verb"\else" clause is optional; you can also type
\begin{verbatim}
\ifprintanswers
  Stuff to appear only when answers are being printed.
\fi  
\end{verbatim}
If you want something printed only when answers are not being printed,
you can type
\begin{verbatim}
\ifprintanswers
\else
  Stuff to appear only when answers are not being printed.
\fi
\end{verbatim}
Since the \verb"exam" document class automatically loads the
\verb"ifthen" package, you can accomplish the same things using the
\verb"\ifthenelse" command:
\begin{verbatim}
\ifthenelse{\boolean{printanswers}}
  {Stuff to appear only when answers are being printed.}
  {Stuff to appear only when answers are not being printed.}
\end{verbatim}
If you use \verb"\ifthenelse", then \verb"both" cases must be present,
even if you leave one of them empty.  Thus, e.g., if you want to print
something only when answers are not being printed, you'd type
\begin{verbatim}
\ifthenelse{\boolean{printanswers}}
  {}
  {Stuff to appear only when answers are not being printed.}
\end{verbatim}


%--------------------------------------------------------------------
\subsubsection{Printing only selected solutions}
\label{sec:selectedsolutions}
\index{selected solution}
\index{solution!selected}

The \verb"\ifprintanswers" command can be used to print only certain
of the solutions in an exam.  To do that, you'd put the lines
\begin{verbatim}
\newif\ifprintselected
\printselectedtrue
%\printselectedfalse

\newenvironment{select}
{\ifprintselected
   \printanswers
 \fi
}
{}
\end{verbatim}
into your preamble (i.e., after the \verb"\documentclass" command and
before the \verb"\begin{document}" command).  That defines a
  \texttt{select} environment, in which you can wrap any of the
  solution environments, as in
\begin{verbatim}
\begin{select}
  \begin{solutionorlines}[1in]
    Only when the number is two.
  \end{solutionorlines}
\end{select}
\end{verbatim}
As long as you leave the command \verb"\printselectedtrue"
uncommented, then everything wrapped in a \texttt{select} environment
will be printed.  If you comment out \verb"\printselectedtrue" and
uncomment \verb"\printselectedfalse", then the \texttt{select}
environment will have no effect.

\indcstop{ifprintanswers}

\index{answers|)}
\index{solution environment|)}
\index{solutions|)}


%--------------------------------------------------------------------
%--------------------------------------------------------------------
\section{Grading tables and point tables}
\label{sec:Tables}
\index{grading table|(}
\index{point table|(}
\indcstart{gradetable}
\indcstart{pointtable}

\newlength\cwidth

The \verb"exam" document class can print either
\begin{itemize}
\item a grading table, indexed either by question number or by page
  number, which lists the points possible and leaves space for you to
  fill in (by hand) the points earned (see
  section~\ref{sec:GradeTables}), or
\item a point table, indexed either by question number or by page
  number, which lists the points possible but does not leave space for
  you to fill in the points earned (see
  section~\ref{sec:PointTables}).
\end{itemize}
It's also possible to print either a grading table or point table to
cover only part of the exam (see section~\ref{sec:PartialTable}).

If the exam has any bonus points (see section~\ref{sec:Bonus}), then
the bonus points will be ignored by the tables described in sections
\ref{sec:GradeTables}, \ref{sec:PointTables},
and~\ref{sec:PartialTable}, but it's possible to print bonus grading
tables and bonus point tables that list the bonus points and ignore
the non-bonus points (see section~\ref{sec:BonusTables}) or combined
grading tables and combined point tables that list both regular points
and bonus points (see section~\ref{sec:CombinedTables}).

%--------------------------------------------------------------------
\subsection{Grading tables indexed by question number or by page
  number}
\label{sec:GradeTables}
\index{grading table!by question number or page number|(}

The \verb"exam" document class can print a grading table, indexed
either by question number or by page number.  That is, you can print
\begin{itemize}
\item a table listing the question numbers (or question titles for
  questions created using the \verb"\titledquestion" command instead
  of the \verb"\question" command; see section~\ref{sec:Titled}) and
  the total points possible for each question (including all of its
  parts, subparts, and subsubparts) and leaving space for you to fill
  in (by hand) the points earned on each question, or
\item a table listing each page that has at least one question, part,
  subpart, or subsubpart with points assigned to it, the total number
  of points possible on that page, and leaving space for you to fill
  in (by hand) the points earned on that page.
\end{itemize}
In order to use this feature you must either include the optional
argument
\verb"addpoints"\indt{addpoints}\indcsub{documentclass}{options} in
your \verb"\documentclass" command (as in
\verb"\documentclass[addpoints]{exam}") or give the command
\verb"\addpoints"\indc{addpoints} (see section~\ref{sec:addpoints}),
and there must be only one \verb"questions"
%
environment\index{questions environment!only one} in the entire exam.
In addition, you must run \LaTeX{} twice after making any changes to
the file in order to make sure that the point values are correct on a
grading table indexed by question number and at least three times for
a grading table indexed by page number.  (Since the table appears on
the third run of \LaTeX{}, and the space it occupies can change the
page on which each question falls, it may take a fourth run of
\LaTeX{} for the table to have the points per page correct.)

The command to create a grading table is 
\begin{center}
  \verb"\gradetable"
\end{center}
and it takes two optional
arguments:\indc{gradetable[v]}\indc{gradetable[h]}
\begin{enumerate}
\item The first optional argument can be either \verb"[v]" or
  \verb"[h]", to choose between a vertically oriented table and a
  horizontally oriented table, and
\item the second optional argument can be either \verb"[questions]" or
  \verb"[pages]", to choose between a table indexed by question number
  and a table indexed by page number.
\end{enumerate}
That is,\indc{gradetable[v][questions]}\indc{gradetable[v][pages]}%
\indc{gradetable[h][questions]}\indc{gradetable[h][pages]}
\begin{itemize}
\item \verb"\gradetable[v][questions]" prints a vertically oriented
  table indexed by question number,
\item \verb"\gradetable[h][questions]" prints a horizontally oriented
  table indexed by question number,
\item \verb"\gradetable[v][pages]" prints a vertically oriented table
  indexed by page number, and
\item \verb"\gradetable[h][pages]" prints a horizontally oriented
  table indexed by page number.
\end{itemize}
If you leave out the optional arguments (i.e., if you give the command
as ``\verb"\gradetable"'') you'll get a vertically oriented table
indexed by question number.

For example, if the exam has 8 questions, each worth a total of 15
points, and you type
\begin{verbatim}
\begin{center}
  \gradetable[v][questions]
\end{center}
\end{verbatim}
then you'll get
\begin{center}
  \cwidth=2em
  \renewcommand{\arraystretch}{1.5}
  \begin{tabular}{|c|c|c|}
    \hline
    Question& Points& Score\\
    \hline
    1& 15& \hbox to \cwidth{\hfill}\\
    \hline
    2& 15& \hbox to \cwidth{\hfill}\\
    \hline
    3& 15& \hbox to \cwidth{\hfill}\\
    \hline
    4& 15& \hbox to \cwidth{\hfill}\\
    \hline
    5& 15& \hbox to \cwidth{\hfill}\\
    \hline
    6& 15& \hbox to \cwidth{\hfill}\\
    \hline
    7& 15& \hbox to \cwidth{\hfill}\\
    \hline
    8& 15& \hbox to \cwidth{\hfill}\\
    \hline
    Total:& 120&\\
    \hline
  \end{tabular}
\end{center}
and if you type
\begin{verbatim}
\begin{center}
  \gradetable[h][questions]
\end{center}
\end{verbatim}
then you'll get
\begin{center}
  \cwidth=2em
  \renewcommand{\arraystretch}{1.5}
  \begin{tabular}{|l|c|c|c|c|c|c|c|c|c|}
    \hline
    Question:& 1& 2& 3& 4& 5& 6& 7& 8& Total\\
    \hline
    Points:&
    15&
    15&
    15&
    15&
    15&
    15&
    15&
    15&
    120\\
    \hline
    Score:&
    \hbox to \cwidth{\hfill}&
    \hbox to \cwidth{\hfill}&
    \hbox to \cwidth{\hfill}&
    \hbox to \cwidth{\hfill}&
    \hbox to \cwidth{\hfill}&
    \hbox to \cwidth{\hfill}&
    \hbox to \cwidth{\hfill}&
    \hbox to \cwidth{\hfill}&
    \\
    \hline
  \end{tabular}
\end{center}
The number of points listed for a question is the sum of the point
values for that question and all of its parts, subparts, and
subsubparts.

If those 8 questions are distributed two to a page on each of pages 2
through 5 and if you type
\begin{verbatim}
\begin{center}
  \gradetable[v][pages]
\end{center}
\end{verbatim}
then you'll get
\begin{center}
  \cwidth=2em
  \renewcommand{\arraystretch}{1.5}
  \begin{tabular}{|c|c|c|}
    \hline
    Page& Points& Score\\
    \hline
    2& 30& \hbox to \cwidth{\hfill}\\
    \hline
    3& 30& \hbox to \cwidth{\hfill}\\
    \hline
    4& 30& \hbox to \cwidth{\hfill}\\
    \hline
    5& 30& \hbox to \cwidth{\hfill}\\
    \hline
    Total:& 120&\\
    \hline
  \end{tabular}
\end{center}
and if you type
\begin{verbatim}
\begin{center}
  \gradetable[h][pages]
\end{center}
\end{verbatim}
then you'll get
\begin{center}
  \cwidth=2em
  \renewcommand{\arraystretch}{1.5}
  \begin{tabular}{|l|c|c|c|c|c|}
    \hline
    Page:& 2& 3& 4& 5& Total\\
    \hline
    Points:&
    30&
    30&
    30&
    30&
    120\\
    \hline
    Score:&
    \hbox to \cwidth{\hfill}&
    \hbox to \cwidth{\hfill}&
    \hbox to \cwidth{\hfill}&
    \hbox to \cwidth{\hfill}&
    \hbox to \cwidth{\hfill}
    \\
    \hline
  \end{tabular}
\end{center}

\emph{Warning:} If you have a large number of questions on
the exam, then these tables can easily become too large to fit on the
page.  If this becomes a problem, then you can either
\begin{itemize}
\item create grading ranges (see section~\ref{sec:PartialTable}) and
  print several partial grading tables, making each one small enough
  to fit on the page, or
\item use the \verb"\pointsofquestion" command (see
  section~\ref{sec:pointsofq}) or the \verb"\pointsonpage" command
  (see section~\ref{sec:pointsonp}) to create a custom \verb"tabular"
  environment that has more rows (or columns) than the tables produced
  by the \verb"gradingtable" command.
\end{itemize}
\index{grading table!by question number or page number|)}

%--------------------------------------------------------------------
\subsection{Point tables indexed by question number or by page
  number}
\label{sec:PointTables}
\index{point table!by question number or page number|(}

Point tables are similar to grading tables (see
section~\ref{sec:GradeTables}), except that there is no place in the
table to fill in the points earned.

The command to create a point table is 
\begin{center}
  \verb"\pointtable"
\end{center}
and it takes two optional
arguments:\indc{pointtable[v]}\indc{pointtable[h]}
\begin{enumerate}
\item The first optional argument can be either \verb"[v]" or
  \verb"[h]", to choose between a vertically oriented table and a
  horizontally oriented table, and
\item the second optional argument can be either \verb"[questions]" or
  \verb"[pages]", to choose between a table indexed by question number
  and a table indexed by page number.
\end{enumerate}
That is,\indc{pointtable[v][questions]}\indc{pointtable[v][pages]}%
\indc{pointtable[h][questions]}\indc{pointtable[h][pages]}
\begin{itemize}
\item \verb"\pointtable[v][questions]" prints a vertically oriented
  table indexed by question number,
\item \verb"\pointtable[h][questions]" prints a horizontally oriented
  table indexed by question number,
\item \verb"\pointtable[v][pages]" prints a vertically oriented table
  indexed by page number, and
\item \verb"\pointtable[h][pages]" prints a horizontally oriented
  table indexed by page number.
\end{itemize}
If you leave out the optional arguments (i.e., if you give the command
as ``\verb"\pointtable"'') you'll get a vertically oriented table
indexed by question number.

For example, if the exam has 8 questions, each worth a total of 15
points, and you type
\begin{verbatim}
\begin{center}
  \pointtable[v][questions]
\end{center}
\end{verbatim}
then you'll get
\begin{center}
  \cwidth=2em
  \renewcommand{\arraystretch}{1.5}
  \begin{tabular}{|c|c|}
    \hline
    Question& Points\\
    \hline
    1& 15\\
    \hline
    2& 15\\
    \hline
    3& 15\\
    \hline
    4& 15\\
    \hline
    5& 15\\
    \hline
    6& 15\\
    \hline
    7& 15\\
    \hline
    8& 15\\
    \hline
    Total:& 120\\
    \hline
  \end{tabular}
\end{center}
and if you type
\begin{verbatim}
\begin{center}
  \pointtable[h][questions]
\end{center}
\end{verbatim}
then you'll get
\begin{center}
  \cwidth=2em
  \renewcommand{\arraystretch}{1.5}
  \begin{tabular}{|l|c|c|c|c|c|c|c|c|c|}
    \hline
    Question:& 1& 2& 3& 4& 5& 6& 7& 8& Total\\
    \hline
    Points:&
    15&
    15&
    15&
    15&
    15&
    15&
    15&
    15&
    120\\
    \hline
  \end{tabular}
\end{center}
The number of points listed for a question is the sum of the point
values for that question and all of its parts, subparts, and
subsubparts.

If those 8 questions are distributed two to a page on each of pages 2
through 5 and if you type
\begin{verbatim}
\begin{center}
  \pointtable[v][pages]
\end{center}
\end{verbatim}
then you'll get
\begin{center}
  \cwidth=2em
  \renewcommand{\arraystretch}{1.5}
  \begin{tabular}{|c|c|}
    \hline
    Page& Points\\
    \hline
    2& 30\\
    \hline
    3& 30\\
    \hline
    4& 30\\
    \hline
    5& 30\\
    \hline
    Total:& 120\\
    \hline
  \end{tabular}
\end{center}
and if you type
\begin{verbatim}
\begin{center}
  \pointtable[h][pages]
\end{center}
\end{verbatim}
then you'll get
\begin{center}
  \cwidth=2em
  \renewcommand{\arraystretch}{1.5}
  \begin{tabular}{|l|c|c|c|c|c|}
    \hline
    Page:& 2& 3& 4& 5& Total\\
    \hline
    Points:&
    30&
    30&
    30&
    30&
    120\\
    \hline
  \end{tabular}
\end{center}

\emph{Warning:} If you have a large number of questions on
the exam, then these tables can easily become too large to fit on the
page.  If this becomes a problem, then you can either
\begin{itemize}
\item create grading ranges (see section~\ref{sec:PartialTable}) and
  print several partial point tables, making each one small enough
  to fit on the page, or
\item use the \verb"\pointsofquestion" command (see
  section~\ref{sec:pointsofq}) or the \verb"\pointsonpage" command
  (see section~\ref{sec:pointsonp}) to create a custom \verb"tabular"
  environment that has more rows (or columns) than the tables produced
  by the \verb"pointtable" command.
\end{itemize}
\index{point table!by question number or page number|)}

%--------------------------------------------------------------------
\subsection{Table entries as clickable links}
\label{sec:Clickable}
\index{grading table!clickable links}
\index{point table!clickable links}
\index{clickable links}

If you put the command\indc{usepackage\{hyperref\}}\indt{hyperref}
\begin{center}
  \verb"\usepackage{hyperref}"
\end{center}
into the preamble of your document (i.e., after the
\verb"\documentclass" command and before the \verb"\begin{document}"
command) and then produce a pdf file (either using dvipdf or 
directly by using pdflatex), then
\begin{itemize}
\item in a grading table or point table indexed by question number the
  question numbers in the table will be clickable links that take you
  to the corresponding question, and
\item in a grading table or point table indexed by page number the
  page numbers in the table will be clickable links that take you to
  the corresponding page.
\end{itemize}
If you create questions using the \verb"\titledquestion" command
instead of the \verb"\question" command (see
section~\ref{sec:Titled}), then a grading table or point table indexed
by question number will show the \emph{titles} of those questions
instead of the numbers of those questions, and those question titles
will be clickable links that take you to the corresponding question.

%--------------------------------------------------------------------
\subsection{Grading ranges, partial grading tables, and partial point tables}
\label{sec:PartialTable}
\index{grading range|(}
\indcstart{begingradingrange}
\indcstart{endgradingrange}
\index{range!grading|(}

It is possible to define one or more ``grading ranges'', and then to
count the points in a range and create grading tables and point tables
that cover only that range.

In more detail: You can define a grading range by choosing a name with
which to label the range, e.g., ``myrange'', and then putting the
command
\begin{itemize}
\item[] \verb"\begingradingrange{myrange}" at the start of the range
  and the command
\item[] \verb"\endgradingrange{myrange}" at the end of the range.
\end{itemize}
You can then use the commands
\begin{itemize}
\item \verb"\pointsinrange{myrange}"\indc{pointsinrange}, which will
  expand to the number of points in the range,
\item \verb"\firstqinrange{myrange}"\indc{firstqinrange}, which will
  expand to the number of the first question in the range,
\item \verb"\lastqinrange{myrange}"\indc{lastqinrange}, which will
  expand to the number of the last question in the range, and
\item \verb"\numqinrange{myrange}"\indc{numqinrange}, which will
  expand to the number of questions in the range.
\end{itemize}
You can define as many grading ranges as you like, wherever you like.
That is, grading ranges do \emph{not} need to be nested; you can begin
the range \texttt{otherrange} somewhere in the middle of the range
\texttt{myrange} and then end the range \texttt{otherrange} after the
end of \texttt{myrange}.

You can create a grading table or point table that covers only the
questions in a range by using the
\verb"\partialgradetable"\indc{partialgradetable} or
\verb"\partialpointtable"\indc{partialpointtable} command.  These are
similar to the \verb"\gradetable" and \verb"\pointtable" commands,
except that they each have one required argument (the name of the
grading range) before the two optional arguments (the first of which
must be either \verb"[v]" or \verb"[h]" and the second of which must
be either \verb"[questions]" or \verb"[pages]").  For example, the
command
\begin{center}
  \verb"\partialgradetable{myrange}[v][questions]"
\end{center}
would create a vertically grading oriented table indexed by the
questions in the range, and the command
\begin{center}
  \verb"\partialgradetable{myrange}[v][pages]"
\end{center}
would create a vertically oriented grading table indexed by all pages
with points starting with the page containing the command
\verb"\begingradingrange{myrange}" and ending with the page containing
the command \verb"\endgradingrange{myrange}"

If the exam has any bonus points (see section~\ref{sec:Bonus}), then
the bonus points will be ignored by the \verb"\pointsinrange",
\verb"\partialgradetable", and \verb"\partialpointtable" commands, but
they will be counted by the \verb"\bonuspointsinrange" command (see
section~\ref{sec:BonusTables}) and tabulated by the
\verb"\partialbonusgradetable", \verb"\partialbonuspointtable",
\verb"\partialcombinedgradetable", and
\verb"\partialcombinedponttable" commands (see
section~\ref{sec:BonusTables} and section~\ref{sec:CombinedTables}).

\index{range!grading|)}
\indcstop{endgradingrange}
\indcstop{begingradingrange}
\index{grading range|)}

%--------------------------------------------------------------------
\subsection{\texttt{\bs pointsofquestion}}
\label{sec:pointsofq}

If you use the document class option \verb"addpoints"\indt{addpoints}
(as in
\verb"\documentclass[addpoints]{exam}"\indcsub{documentclass}{options})
or give the command \verb"\addpoints"\indc{addpoints} (see
section~\ref{sec:addpoints}), then you can use
the\indc{pointsofquestion}
\begin{center}
  \verb"\pointsofquestion"
\end{center}
command.  This command takes one argument, which must be the number of
a question on the exam, and it prints the total number of points for
that question.  That is:
\begin{itemize}
\item \verb"\pointsofquestion{1}" prints the sum of the point values
  for question 1 and all of its parts, subparts, and subsubparts.
\item \verb"\pointsofquestion{2}" prints the sum of the point values
  for question 2 and all of its parts, subparts, and subsubparts.
\item Etc.
\end{itemize}
The \verb"\pointsofquestion" command is used by the \verb"\gradetable"
and \verb"\pointtable" commands (see section~\ref{sec:Tables}),
and it can be used to create a table using \LaTeX's \verb"tabular"
environment when the tables produced using the \verb"\gradetable" and
\verb"\pointtable" commands either don't fit on the page or are
unsuitable for some other reason.  It can also be used in a
\verb"\qformat" command to list the total number of points of all
parts, subparts, and subsubparts of a question on the line with the
question number (see section~\ref{sec:adding}).

The \verb"\pointsofquestion"\indc{pointsofquestion} command ignores
any bonus points that you may have created (see
section~\ref{sec:Bonus}).  The \verb"\bonuspointsofquestion" command,
however, counts the bonus points (see section~\ref{sec:BonusTables}).

%--------------------------------------------------------------------
\subsection{\texttt{\bs pointsonpage}}
\label{sec:pointsonp}

If you use the document class option \verb"addpoints"\indt{addpoints}
(as in
\verb"\documentclass[addpoints]{exam}"\indcsub{documentclass}{options})
or give the command \verb"\addpoints"\indc{addpoints} (see
section~\ref{sec:addpoints}), then you can use the
\verb"\pointsonpage"\indc{pointsonpage} command.  This command takes
one argument, which must be the number of a page of the exam, and it
prints the total number of points for all the questions, parts,
subparts, and subsubparts on that page.  That is:
\begin{itemize}
\item \verb"\pointsonpage{1}" prints the sum of the point values for
  all questions, etc., on page 1,
\item \verb"\pointsonpage{2}" prints the sum of the point values for
  all questions, etc., on page 2,
\item Etc.
\end{itemize}
This command can be used to create a table using \LaTeX's
\verb"tabular" environment when the tables produced using the
\verb"\gradetable" and \verb"\pointtable" commands either don't fit on
the page or are unsuitable for some other reason.  It can also be used
in headers and footers to print on each page the total number of
points available on that page; for example, the command
\begin{verbatim}
  \runningfooter{}
                {}
                {Points earned: \makebox[1in]{\hrulefill}
                 out of a possible \pointsonpage{\thepage} points}
\end{verbatim}
\indcsub{pointsonpage}{in footer} (see
section~\ref{sec:runningfooter}) will produce the footer
\samplefoot{}{}{}{Points earned: \makebox[1in]{\hrulefill} out of a
  possible 20 points} on all pages after the first.  (For another
example, see section~\ref{sec:PtsPgEx})

\textbf{Caution:} The command \verb"\pointsonpage{\thepage}" will work
reliably only in headers and footers, since the command
%
\verb"\thepage"\indcsub{thepage}{reliable only in headers and footers}
will work reliably only in headers and footers.

The \verb"\pointsonpage" command ignores any bonus points that you may
have created (see section~\ref{sec:Bonus}).  The
\verb"\bonuspointsonpage" command, however, counts the bonus points
(see section~\ref{sec:BonusTables}).

%--------------------------------------------------------------------
\subsection{Bonus grading tables and bonus point tables}
\label{sec:BonusTables}
\index{grading table!bonus|(}
\index{point table!bonus|(}

If your exam contains any bonus
%
points\index{bonus points}\index{points!bonus} (see
section~\ref{sec:Bonus}), then those bonus points will be ignored by
the grading table and point table commands described in sections
\ref{sec:GradeTables}, \ref{sec:PointTables},
and~\ref{sec:PartialTable} and by the
\verb"\pointsofquestion"\indc{pointsofquestion} (see
section~\ref{sec:pointsofq}) and
\verb"\pointsonpage"\indc{pointsonpage} (see
section~\ref{sec:pointsonp}) commands.  There are, however, analogous
commands that count only the bonus points and ignore the non-bonus
points.  (There are also commands that create tables listing both
non-bonus and bonus points; see section~\ref{sec:CombinedTables}.)
These commands, and the non-bonus commands to which they correspond,
are as follows:\indc{bonusgradetable}\indc{bonuspointtable}%
\indc{bonuspointsinrange}\indc{partialbonusgradetable}%
\indc{partialbonuspointtable}\indc{bonuspointsofquestion}%
\indc{bonuspointsonpage}
\begin{center}
  \begin{tabular}{l@{\qquad\qquad}l}
    Bonus points& Non-bonus points\\[\bigskipamount]
    \verb"\bonusgradetable"&        \verb"\gradetable"\\
    \verb"\bonuspointtable"& \verb"\pointtable"\\[\medskipamount]
    \verb"\bonuspointsinrange"& \verb"\pointsinrange"\\
    \verb"\partialbonusgradetable"& \verb"\partialgradetable"\\
    \verb"\partialbonuspointtable"&
                       \verb"\partialpointtable"\\[\medskipamount] 
    \verb"\bonuspointsofquestion"&  \verb"\pointsofquestion"\\
    \verb"\bonuspointsonpage"&      \verb"\pointsonpage"
  \end{tabular}
\end{center}
All of the the commands in the first column count bonus points and
ignore non-bonus points, while all of the commands in the second
column count non-bonus points and ignore bonus points.  The
\verb"\partialbonusgradetable" and \verb"\partialbonuspointtable"
commands use the same
\verb"\begingradingrange"\indc{begingradingrange} and
\verb"\endgradingrange"\indc{endgradingrange} commands that are used
by the corresponding non-bonus versions (see
section~\ref{sec:PartialTable}).  Just as is described in
section~\ref{sec:Clickable} for their non-bonus counterparts, if you
put the command\indc{usepackage\{hyperref\}}\indt{hyperref}
\begin{center}
  \verb"\usepackage{hyperref}"
\end{center}
into the preamble of your document (i.e., after the
\verb"\documentclass" command and before the \verb"\begin{document}"
command) and then produce a pdf file (either using dvipdf or 
directly by using pdflatex),
%
then\index{clickable links}
\begin{itemize}
\item in a bonus grading table or bonus point table indexed by
  question number, the question numbers (or question titles, if you
  used \verb"\bonustitledquestion"; see section~\ref{sec:Bonus}) in
  the table will be clickable links that take you to the corresponding
  question, and
\item in a bonus grading table or bonus point table indexed by page
  number the page numbers in the table will be clickable links that
  take you to the corresponding page.
\end{itemize}

For example, if the exam has 8 questions, with each odd numbered
question worth a total of 10 regular points and each even numbered
question worth a total of 5 bonus points, and you type
\begin{verbatim}
\begin{center}
  \bonusgradetable[h][questions]
\end{center}
\end{verbatim}
then you'll get
\begin{center}
  \cwidth=2em
  \renewcommand{\arraystretch}{1.5}
  \begin{tabular}{|l|c|c|c|c|c|c|c|c|c|}
    \hline
    Question:& 1& 2& 3& 4& 5& 6& 7& 8& Total\\
    \hline
    Bonus Points:&
    0&
    5&
    0&
    5&
    0&
    5&
    0&
    5&
    20\\
    \hline
    Score:&
    \hbox to \cwidth{\hfill}&
    \hbox to \cwidth{\hfill}&
    \hbox to \cwidth{\hfill}&
    \hbox to \cwidth{\hfill}&
    \hbox to \cwidth{\hfill}&
    \hbox to \cwidth{\hfill}&
    \hbox to \cwidth{\hfill}&
    \hbox to \cwidth{\hfill}&\\
    \hline
  \end{tabular}
\end{center}

\index{point table!bonus|)}
\index{grading table!bonus|)}

%--------------------------------------------------------------------
\subsection{Combined grading tables and combined point tables}
\label{sec:CombinedTables}
\index{grading table!combined|(}
\index{point table!combined|(}


There are commands for producing \emph{combined} grading and point
tables, both partial and full, that tabulate both regular points and
bonus points (see section~\ref{sec:Bonus}).  These commands, and the
non-bonus and bonus commands to which they correspond, are
\indc{combinedgradetable}\indc{combinedpointtable}%
\indc{partialcombinedgradetable}\indc{partialcombinedpointtable}%
\begin{center}
  \begin{tabular}{l@{\enskip}l@{\enskip}l}
    Non-Bonus points only& Bonus points only&
                           Combined\\[\medskipamount]
    \verb"\gradetable"& \verb"\bonusgradetable"&
                        \verb"\combinedgradetable"\\
    \verb"\pointtable"& \verb"\bonuspointtable"&
                        \verb"\combinedpointtable"\\
    \verb"\partialgradetable"& \verb"\partialbonusgradetable"&
                               \verb"\partialcombinedgradetable"\\
    \verb"\partialpointtable"& \verb"\partialbonuspointtable"&
                               \verb"\partialcombinedpointtable"
  \end{tabular}
\end{center}
The first two commands in each column take two optional arguments, the
first being either \verb"[v]" or \verb"[h]" and the second being
either \verb"[questions]" or \verb"[pages]".  The last two commands in
each column take one required argument, the name of the grading range
(defined using the \verb"\begingradingrange" and
\verb"\endgradingrange" commands; see section~\ref{sec:PartialTable}),
and then the same two optional arguments as the other commands (the
first being either \verb"[v]" or \verb"[h]" and the second being
either \verb"[questions]" or \verb"[pages]").

Just as is described in section~\ref{sec:Clickable} for their regular
points and section~\ref{sec:BonusTables} for bonus points, if you put
the
%
command\index{clickable links}\indc{usepackage\{hyperref\}}%
\indt{hyperref}
\begin{center}
  \verb"\usepackage{hyperref}"
\end{center}
into the preamble of your document (i.e., after the
\verb"\documentclass" command and before the \verb"\begin{document}"
command) and then produce a pdf file (either using dvipdf or 
directly by using pdflatex), then
\begin{itemize}
\item in a combined grading table or combined point table indexed by
  question number, the question numbers (or question titles, if you
  used either \verb"\titledquestion" (see section~\ref{sec:Titled}) or
  \verb"\bonustitledquestion" (see section~\ref{sec:Bonus})) in the
  table will be clickable links that take you to the corresponding
  question, and
\item in a combined grading table or combined point table indexed by
  page number the page numbers in the table will be clickable links
  that take you to the corresponding page.
\end{itemize}

For example, if the exam has 8 questions, with each odd numbered
question worth a total of 10 regular points and each even numbered
question worth a total of 5 bonus points, and you type
\begin{verbatim}
\begin{center}
  \combinedgradetable[h][questions]
\end{center}
\end{verbatim}
then you'll get
\begin{center}
  \cwidth=2em
  \renewcommand{\arraystretch}{1.5}
  \begin{tabular}{|l|c|c|c|c|c|c|c|c|c|}
    \hline
    Question:& 1& 2& 3& 4& 5& 6& 7& 8& Total\\
    \hline
    Points:&
    10&
    0&
    10&
    0&
    10&
    0&
    10&
    0&
    40\\
    \hline
    Bonus Points:&
    0&
    5&
    0&
    5&
    0&
    5&
    0&
    5&
    20\\
    \hline
    Score:&
    \hbox to \cwidth{\hfill}&
    \hbox to \cwidth{\hfill}&
    \hbox to \cwidth{\hfill}&
    \hbox to \cwidth{\hfill}&
    \hbox to \cwidth{\hfill}&
    \hbox to \cwidth{\hfill}&
    \hbox to \cwidth{\hfill}&
    \hbox to \cwidth{\hfill}&\\
    \hline
  \end{tabular}
\end{center}

\index{point table!combined|)}
\index{grading table!combined|)}

%--------------------------------------------------------------------
\subsection{Changing the total points in a table}
\label{sec:Totalpoints}
\index{grading table!changing total points|(}
\index{point table!changing total points|(}


There are commands for changing the total points and total bonus
points in grading and point tables, so that the stated point total
will not equal the sum of the point values in the table.  These
commands are intended for exams in which the students are required to
do some subset of the problems, but not all of them, and so the total
number of points possible on the exam is less than the sum of the
points for all of the questions.  The commands
are\indc{settabletotalpoints}\indc{settabletotalbonuspoints}
\begin{center}
  \begin{tabular}{l}
    \verb"\settabletotalpoints"\\
    \verb"\settabletotalbonuspoints"
  \end{tabular}
\end{center}
Each of those commands takes one argument.  After giving the command
\begin{center}
  \verb"\settabletotalpoints{number}"
\end{center}
any grading or point table (regular or combined, partial or full) will
list the total points as \texttt{number} instead of the actual total
number of points in the table.  After giving the command
\begin{center}
  \verb"\settabletotalbonuspoints{number}"
\end{center}
any grading or point table (bonus or combined, partial or full) will
list the total bonus points as \texttt{number} instead of the actual
total number of bonus points in the table.

To have these commands affect only a single table, confine the command
and the command to create the table inside of a group (possibly by
just placing those commands inside of a \texttt{center} environment).
For example, if an exam has 8 questions, each worth 10 points, and the
students are required to do any 6 of the 8, then the commands
\begin{verbatim}
\begin{center}
  \settabletotalpoints{60}
  \gradetable[h][questions]
\end{center}
\end{verbatim}
will produce
\begin{center}
  \cwidth=2em
  \renewcommand{\arraystretch}{1.5}
  \begin{tabular}{|l|c|c|c|c|c|c|c|c|c|}
    \hline
    Question:& 1& 2& 3& 4& 5& 6& 7& 8& Total\\
    \hline
    Points:&
    10&
    10&
    10&
    10&
    10&
    10&
    10&
    10&
    60\\
    \hline
    Score:&
    \hbox to \cwidth{\hfill}&
    \hbox to \cwidth{\hfill}&
    \hbox to \cwidth{\hfill}&
    \hbox to \cwidth{\hfill}&
    \hbox to \cwidth{\hfill}&
    \hbox to \cwidth{\hfill}&
    \hbox to \cwidth{\hfill}&
    \hbox to \cwidth{\hfill}&\\
    \hline
  \end{tabular}
\end{center}
and no other tables will be affected by that
\verb"\settabletotalpoints" command.

\index{point table!changing total points|)}
\index{grading table!changing total points|)}

%--------------------------------------------------------------------
\subsection{Customizing the tables}
\label{sec:CustTable}
\index{customization!grading table|(}
\index{customization!point table|(}
\index{grade table!customization|(}
\index{point table!customization|(}

There are three ways in which you can customize the appearance of
grading tables and point tables:
\begin{itemize}
\item You can change the words (and the fonts) that appear in the
  table.
\item You can change the minimum width of the blank cells into which
  you will write the scores.
\item You can change the value of
  \verb"\baselinestretch"\indc{baselinestretch} used for the table.
\end{itemize}
For all grading and point tables:\indc{gradetablestretch}\indc{cellwidth}
\begin{center}
  \begin{tabular}{l@{\qquad}l}
    \verb"\cellwidth{length}"& sets the minimum width of the blank cells to
    \verb"length"\\
    \verb"\gradetablestretch{number}"& uses \verb"number" as the
    \verb"\baselinestretch"
  \end{tabular}
\end{center}
For vertical non-bonus tables (i.e., tables created by
\verb"\gradetable[v]", \verb"\pointtable[v]",
\verb"\partialgradetable{rangename}[v]", or
\verb"\partialpointtable{rangename}[v]"):%
\indc{vqword}\indc{vpgword}\indc{vpword}\indc{vsword}\indc{vtword}
\begin{center}
  \begin{tabular}{l@{\qquad}l}
    Command& \multicolumn{1}{c}{Effect}\\[\medskipamount]
    \verb"\vqword{text}"& substitutes \verb"text" for ``Question''\\
    \verb"\vpgword{text}"& substitutes \verb"text" for ``Page''\\
    \verb"\vpword{text}"& substitutes \verb"text" for ``Points''\\
    \verb"\vsword{text}"& substitutes \verb"text" for ``Score''\\
    \verb"\vtword{text}"& substitutes \verb"text" for ``Total:''
  \end{tabular}
\end{center}
For horizontal non-bonus tables (i.e., tables created by
\verb"\gradetable[h]", \verb"\pointtable[h]",
\verb"\partialgradetable{rangename}[h]", or
\verb"\partialpointtable{rangename}[h]"):%
\indc{hqword}\indc{hpgword}\indc{hpword}\indc{hsword}\indc{htword}
\begin{center}
  \begin{tabular}{l@{\qquad}l}
    Command& \multicolumn{1}{c}{Effect}\\[\medskipamount]
    \verb"\hqword{text}"& substitutes \verb"text" for ``Question:''\\
    \verb"\hpgword{text}"& substitutes \verb"text" for ``Page:''\\
    \verb"\hpword{text}"& substitutes \verb"text" for ``Points:''\\
    \verb"\hsword{text}"& substitutes \verb"text" for ``Score:''\\
    \verb"\htword{text}"& substitutes \verb"text" for ``Total''
  \end{tabular}
\end{center}
For vertical bonus tables (i.e., tables created by
\verb"\bonusgradetable[v]", \verb"\bonuspointtable[v]",
\verb"\partialbonusgradetable{rangename}[v]", or
\verb"\partialbonuspointtable{rangename}[v]"):\indc{bvqword}%
\indc{bvpgword}\indc{bvpword}\indc{bvsword}\indc{bvtword}
\begin{center}
  \begin{tabular}{l@{\qquad}l}
    Command& \multicolumn{1}{c}{Effect}\\[\medskipamount]
    \verb"\bvqword{text}"& substitutes \verb"text" for ``Question''\\
    \verb"\bvpgword{text}"& substitutes \verb"text" for ``Page''\\
    \verb"\bvpword{text}"& substitutes \verb"text" for ``Bonus Points''\\
    \verb"\bvsword{text}"& substitutes \verb"text" for ``Score''\\
    \verb"\bvtword{text}"& substitutes \verb"text" for ``Total:''
  \end{tabular}
\end{center}
For horizontal bonus tables (i.e., tables created by
\verb"\bonusgradetable[h]", \\
\verb"\bonuspointtable[h]",
\verb"\partialbonusgradetable{rangename}[h]", or\\
\verb"\partialbonuspointtable{rangename}[h]"):\indc{bhqword}%
\indc{bhpgword}\indc{bhpword}\indc{bhsword}\indc{bhtword}
\begin{center}
  \begin{tabular}{l@{\qquad}l}
    Command& \multicolumn{1}{c}{Effect}\\[\medskipamount]
    \verb"\bhqword{text}"& substitutes \verb"text" for ``Question:''\\
    \verb"\bhpgword{text}"& substitutes \verb"text" for ``Page:''\\
    \verb"\bhpword{text}"& substitutes \verb"text" for ``Bonus Points:''\\
    \verb"\bhsword{text}"& substitutes \verb"text" for ``Score:''\\
    \verb"\bhtword{text}"& substitutes \verb"text" for ``Total''
  \end{tabular}
\end{center}
For vertical combined tables (i.e., tables created by
\verb"\combinedgradetable[v]",\\
\verb"\combinedpointtable[v]",
\verb"\partialcombinedgradetable{rangename}[v]", or\\
\verb"\partialcombinedpointtable{rangename}[v]"):\indc{cvqword}%
\indc{cvpgword}\indc{cvpword}\indc{cvbpword}\indc{cvsword}%
\indc{cvtword}
\begin{center}
  \begin{tabular}{l@{\qquad}l}
    Command& \multicolumn{1}{c}{Effect}\\[\medskipamount]
    \verb"\cvqword{text}"& substitutes \verb"text" for ``Question''\\
    \verb"\cvpgword{text}"& substitutes \verb"text" for ``Page''\\
    \verb"\cvpword{text}"& substitutes \verb"text" for ``Points''\\
    \verb"\cvbpword{text}"& substitutes \verb"text" for ``Bonus Points''\\
    \verb"\cvsword{text}"& substitutes \verb"text" for ``Score''\\
    \verb"\cvtword{text}"& substitutes \verb"text" for ``Total:''
  \end{tabular}
\end{center}
For horizontal combined tables (i.e., tables created by
\verb"\combinedgradetable[h]", \\
\verb"\combinedpointtable[h]",
\verb"\partialcombinedgradetable{rangename}[h]", or\\
\verb"\partialcombinedpointtable{rangename}[h]"):\indc{chqword}%
\indc{chpgword}\indc{chpword}\indc{chbpword}\indc{chsword}%
\indc{chtword}
\begin{center}
  \begin{tabular}{l@{\qquad}l}
    Command& \multicolumn{1}{c}{Effect}\\[\medskipamount]
    \verb"\chqword{text}"& substitutes \verb"text" for ``Question:''\\
    \verb"\chpgword{text}"& substitutes \verb"text" for ``Page:''\\
    \verb"\chpword{text}"& substitutes \verb"text" for ``Points:''\\
    \verb"\chbpword{text}"& substitutes \verb"text" for ``Bonus Points:''\\
    \verb"\chsword{text}"& substitutes \verb"text" for ``Score:''\\
    \verb"\chtword{text}"& substitutes \verb"text" for ``Total''
  \end{tabular}
\end{center}
If you don't use any of these commands then you get the default
values, which are defined by the commands in the following table:
\begin{center}
  \begin{tabular}{l@{\quad}l@{\quad}l}
    \multicolumn{3}{c}{\bfseries Defaults for grading and point
                                 tables}\\*[\bigskipamount]
    \verb"\hqword{Question:}"& \verb"\vqword{Question}"&
                               \verb"\cellwidth{2em}"\\
    \verb"\hpgword{Page:}"&    \verb"\vpgword{Page}"&
                               \verb"\gradetablestretch{1.5}"\\
    \verb"\hpword{Points:}"& \verb"\vpword{Points}"\\
    \verb"\hsword{Score:}"& \verb"\vsword{Score}"\\
    \verb"\htword{Total}"& \verb"\vtword{Total:}"\\[\medskipamount]
%
    \verb"\bhqword{Question:}"& \verb"\bvqword{Question}"\\
    \verb"\bhpgword{Page:}"&    \verb"\bvpgword{Page}"\\
    \verb"\bhpword{Bonus Points:}"& \verb"\bvpword{Bonus Points}"\\
    \verb"\bhsword{Score:}"& \verb"\bvsword{Score}"\\
    \verb"\bhtword{Total}"& \verb"\bvtword{Total:}"\\[\medskipamount]
%
    \verb"\chqword{Question:}"& \verb"\cvqword{Question}"\\
    \verb"\chpgword{Page:}"&    \verb"\cvpgword{Page}"\\
    \verb"\chpword{Points:}"& \verb"\cvpword{Points}"\\
    \verb"\chbpword{Bonus Points:}"& \verb"\cvbpword{Bonus Points}"\\
    \verb"\chsword{Score:}"& \verb"\cvsword{Score}"\\
    \verb"\chtword{Total}"& \verb"\cvtword{Total:}"
  \end{tabular}
\end{center}

\bigskip
For example, if you type
\begin{verbatim}
\begin{center}
  \hqword{Aufgabe Nr.:}
  \hpword{Punktzahl:}
  \htword{\textbf{Summe}}
  \hsword{Davon erreicht:}
  \cellwidth{2.2em}
  \gradetable[h][questions]
\end{center}
\end{verbatim}
then you'll get
\begin{center}
  \cwidth=2.2em
  \renewcommand{\arraystretch}{1.5}
  \begin{tabular}{|l|c|c|c|c|c|c|c|c|c|}
    \hline
    Aufgabe Nr.:& 1& 2& 3& 4& 5& 6& 7& 8& \textbf{Summe}\\
    \hline
    Punktzahl:&
    15&
    15&
    15&
    15&
    15&
    15&
    15&
    15&
    120\\
    \hline
    Davon erreicht:&
    \hbox to \cwidth{\hfill}&
    \hbox to \cwidth{\hfill}&
    \hbox to \cwidth{\hfill}&
    \hbox to \cwidth{\hfill}&
    \hbox to \cwidth{\hfill}&
    \hbox to \cwidth{\hfill}&
    \hbox to \cwidth{\hfill}&
    \hbox to \cwidth{\hfill}&
    \\
    \hline
  \end{tabular}
\end{center}

\index{point table!customization|)}
\index{grade table!customization|)}
\index{customization!point table|)}
\index{customization!grading table|)}


\indcstop{pointtable}
\indcstop{gradetable}
\index{point table|)}
\index{grading table|)}


%--------------------------------------------------------------------
%--------------------------------------------------------------------
\section{Changing the page margins}
\label{sec:pagesize}
\index{margins|(}
\index{customization!margins|(}

The exam document class arranges things so that you get one inch
margins at the top, bottom, and sides no matter what size paper you
use, as long as you use the corresponding document class option
(\verb"a4paper", \verb"a5paper", \verb"b5paper", \verb"letterpaper",
\verb"legalpaper", \verb"executivepaper", or \verb"landscape").  You
can change the sizes of these margins by using the commands that
change the size of the printed area.

To change the width of the printed area, you would use the
\verb"\extrawidth"\indc{extrawidth} command.  The \verb"\extrawidth"
command takes one argument, and enlarges the width of the printed area
by the amount of the argument.  It keeps the printed area centered as
it changes its width.  If the argument is negative, then the width of
the printed area is decreased.

For example, to enlarge the left and right margins by one half inch
each, you would use the command
\begin{center}
  \verb"\extrawidth{-1in}"
\end{center}
since the printed region must shrink by one inch to allow an
additional one half inch on both sides.  To decrease the left and
right margins to three quarters of an inch each, you would use the
command 
\begin{center}
  \verb"\extrawidth{.5in}"
\end{center}
since the printed region must grow by one half inch to decrease both
margins by one quarter of an inch.

To change the height of the printed area, you must choose whether the
top or the bottom of the printed area (or possibly both) should move.
The commands for this are principally intended to allow additional
room for large headers and footers, and so they are called
\verb"\extraheadheight"\indc{extraheadheight} and
\verb"\extrafootheight"\indc{extrafootheight}.  For a full description
of these commands, see section~\ref{sec:extra-room}.

To move the top of the printed region (and any header that's present)
downwards, you use the command \verb"\extraheadheight".  This command
takes one argument, and moves the top of the text and the header down
by this amount (keeping the distance between the header and the text
constant).  Thus, to increase the top margin by three quarters of an
inch, you would give the command
\begin{center}
  \verb"\extraheadheight{.75in}"
\end{center}
To decrease the top margin by one half inch, you would give the
command 
\begin{center}
  \verb"\extraheadheight{-.5in}"
\end{center}
The 
%
\verb"\extraheadheight"\indcsub{extraheadheight}{optional argument}%
\index{head height!first page different} command takes an optional
argument to provide a top margin on the first page that's different
from that on all other pages.  For an explanation of this, see
section~\ref{sec:extra-room}.

To move the bottom of the printed region (and any footer that's
present) upwards, you use the command
\verb"\extrafootheight"\indc{extrafootheight}.  This command takes one
argument, and moves the bottom of the text and the footer up by this
amount (keeping the distance between the footer and the text
constant).  Thus, to increase the bottom margin by three quarters of
an inch, you would give the command
\begin{center}
  \verb"\extrafootheight{.75in}"
\end{center}
To decrease the bottom margin by one half inch, you would give the
command
\begin{center}
  \verb"\extrafootheight{-.5in}"
\end{center}
The
%
\verb"\extrafootheight"\indcsub{extrafootheight}{optional argument}%
\index{foot height!first page different} command takes an optional
argument to provide a bottom margin on the first page that's different
from that on all other pages.  For an explanation of this, see
section~\ref{sec:extra-room}.

\index{customization!margins|)}
\index{margins|)}


%--------------------------------------------------------------------
%---------------------------------------------------------------------
\section{Headers and footers}
\label{sec:headfoot}
\index{header|(}
\index{footer|(}

The following sections explain all of the commands that deal with
headers and footers.  There are a number of things to explain here,
and so you may find it easier to skip these sections and instead look
at the examples in sections~\ref{sec:beginexamples}
through~\ref{sec:endexamples} (on pages~\pageref{sec:beginexamples}
through~\pageref{sec:endexamples}).  You can then refer back to these
sections for the full story on whatever isn't clear from the examples.
All of the commands described in these sections should be given in the
preamble, i.e., after the \verb"\documentclass" command and before the
\verb"\begin{document}" command.
  
It's also important to remember that if you use a \verb"coverpages"
environment (described in section~\ref{sec:coverpages}), then the
commands described here affect only the pages in the main section of
the exam, and not the pages of the \verb"coverpages" environment.
There are commands for headers and footers in the cover pages that are
analogous to the commands described in this section; for that, see
section~\ref{sec:Coverhdft}.

%---------------------------------------------------------------------
\subsection{Page styles: Headers and/or footers}
\label{sec:pagestyle}
\indcstart{pagestyle}
\index{header!on or off}
\index{footer!on or off}

It's the \verb"\pagestyle" command that determines whether the exam
will have headers, footers, both, or neither.  The contents of the
header and footer are specified using the commands described in
sections~\ref{sec:header} through~\ref{sec:rules}, but it's the
\verb"\pagestyle" command that determines whether the header and
footer that you construct will actually be placed onto the page.  The
\verb"\pagestyle" command should be given in the preamble, i.e., after
the \verb"\documentclass" command and before the
\verb"\begin{document}" command.

To have both a header and a footer, give the
command\indt{headandfoot}\indc{pagestyle\{headandfoot\}}
\begin{center}
  \verb"\pagestyle{headandfoot}"
\end{center}
If you want every page to have a header but no footer, give the
command\indt{head}\indc{pagestyle\{head\}}
\begin{center}
  \verb"\pagestyle{head}"
\end{center}
To give every page a footer but no header, give the
command\indt{foot}\indc{pagestyle\{foot\}}
\begin{center}
  \verb"\pagestyle{foot}"
\end{center}
Finally, to omit both the header and the footer from the page, give
the command\indt{empty}\indc{pagestyle\{empty\}}
\begin{center}
  \verb"\pagestyle{empty}"
\end{center}


As is true for all \LaTeX{} document classes, you can change the page
style used on a single page by giving the command\indc{thispagestyle}
\begin{center}
  \verb"\thispagestyle{somestyle}"
\end{center}
somewhere on that page (where \verb"somestyle" is the style that you
want to use on that page).  This is most often needed if you use the
\verb"\maketitle"\indc{maketitle}\indcsub{maketitle}{overriding}
command, since that command inserts a \verb"\thispagestyle{plain}"
immediately following the title.  If you use the \verb"\maketitle"
command and you want the entire document to use
\verb"\pagestyle{headandfoot}", then you'll need to put the command
\verb"\thispagestyle{headandfoot}"\indc{thispagestyle\{headandfoot\}}
immediately after the \verb"\maketitle" command to override the
\verb"\thispagestyle{plain}" that is inserted by \verb"\maketitle".

The default page style is
\verb"headandfoot"\indcsub{pagestyle}{default}.  That is, if you don't
give any \verb"\pagestyle" commands, then the header and footer will
appear on every page.  The default header is empty, and the default
footer consists of ``Page $n$'' centered in the footer (where $n$ is
the page number).

\indcstop{pagestyle}

%---------------------------------------------------------------------
\subsection{The three parts of the header} 
\label{sec:header}
\index{header!three parts|(}

The header is specified in three parts:
\begin{itemize}
\item One part to be left justified.
\item One part to be centered.
\item One part to be right justified.
\end{itemize}
There are two different ways in which you can specify the three parts
of the header.  The first uses the single command \verb"\header" to
specify all three parts of the header, or the commands
\verb"\firstpageheader" and \verb"\runningheader" to specify a
different header for the first page (see
section~\ref{sec:runningheader}). The second uses the commands
\verb"\lhead", \verb"\chead", and \verb"\rhead", each of which can
have an optional argument to specify a different header for the first
page (see section~\ref{sec:lhead}).  All of these commands should be
given in the preamble, i.e., after the \verb"\documentclass" command
and before the \verb"\begin{document}" command.

%--------------------------------------------------------------------
\subsubsection{Using \texttt{\bs header},  \texttt{\bs firstpageheader} and
  \texttt{\bs runningheader}}
\label{sec:runningheader}
\indcstart{header}
\indcstart{firstpageheader}
\indcstart{runningheader}

The command \verb"\header{Text 1}{Text 2}{Text 3}" puts ``Text~1''
into the left justified header, ``Text~2'' into the centered header,
and ``Text~3'' into the right justified header on every page.  If you
want the header on the first page to be different from that on the
other pages, then you should use the commands \verb"\firstpageheader"
and
%
\verb"\runningheader"\index{header!different on first page}, which
also take three arguments and affect either the first page or all
pages except the first.


For example, to put the header
\samplehead{Math 115}{Second Exam}{July 4, 1776}{}
on every page of the exam, you would give the command
\begin{verbatim}
\header{Math 115}{Second Exam}{July 4, 1776}
\end{verbatim}

If you want to use a header on the first page that's different from
the header used on all other pages, you would use the commands
\verb"\firstpageheader" and \verb"\runningheader" instead of
\verb"\header".  For example, if you want the above header for the
first page, but on all pages after the first you want to have the
header \samplehead{Math 115}{Second Exam (Continued)}{July 4, 1776}{}
then you would give the commands
\begin{verbatim}
\firstpageheader{Math 115}{Second Exam}{July 4, 1776}
\runningheader{Math 115}{Second Exam (Continued)}{July 4, 1776}
\end{verbatim}

You can leave one or more of the three parts empty.  To have the
header
\samplehead{Math 115}{Second Exam}{July 4, 1776}{}
on the first page, with the header
\samplehead{Math 115}{}{Second Exam (Continued)}{}
on all other pages, you would give the commands
\begin{verbatim}
\firstpageheader{Math 115}{Second Exam}{July 4, 1776}
\runningheader{Math 115}{}{Second Exam (Continued)}
\end{verbatim}

Any of the three parts of the header can have multiple lines.  To
specify where the line breaks should go, you just type \verb"\\". 
Thus, to have the header
\samplehead{\bfseries\large Math 115\\Professor Hilbert}{}
       {\bfseries\large First Exam\\July 4, 1776}{}
appear on every page, you would give the command
\begin{verbatim}
\header{\bfseries\large Math 115\\Professor Hilbert}
       {}
       {\bfseries\large First Exam\\July 4, 1776}
\end{verbatim}

\indcstop{header}
\indcstop{firstpageheader}
\indcstop{runningheader}

%--------------------------------------------------------------------
\subsubsection{Using \texttt{\bs lhead}, \texttt{\bs chead} and \texttt{\bs rhead}}
\label{sec:lhead}
\indcstart{lhead}
\indcstart{chead}
\indcstart{rhead}


The command \verb"\lhead{Text}" puts ``Text'' into the left justified
part of the header on every page.  The command
\index{header!different on first page}%
\indcsub{lhead}{optional argument}%
\indcsub{chead}{optional argument}%
\indcsub{rhead}{optional argument}%
\indcsub{lhead}{different on first page}%
\indcsub{chead}{different on first page}%
\indcsub{rhead}{different on first page}%
\verb"\lhead[Text 1]{Text 2}" puts ``Text~1'' into the left justified
header on the first page and ``Text~2'' into the left justified header
on all other pages.  The commands \verb"\chead" and \verb"\rhead" have
similar effects on the centered and right justified parts of the
header.


For example, to put the header 
\samplehead{Math 115}{Second Exam}{July 4, 1776}{}
on every page of the exam, you would give the commands
\begin{verbatim}
\lhead{Math 115}
\chead{Second Exam}
\rhead{July 4, 1776}
\end{verbatim}

If you want any of the three parts to have a special version to be
used only on the first page, then you just include that special
version as an optional argument (enclosed in square brackets) to the
command.  For example, if you want the above header for the first
page, but on all pages after the first you want to have the header
\samplehead{Math 115}{Second Exam (Continued)}{July 4, 1776}{}
then you would give the commands
\begin{verbatim}
\lhead{Math 115}
\chead[Second Exam]{Second Exam (Continued)}
\rhead{July 4, 1776}
\end{verbatim}

You can leave one or more of the three parts empty.  To have the
header
\samplehead{Math 115}{Second Exam}{July 4, 1776}{}
on the first page, with the header
\samplehead{Math 115}{}{Second Exam (Continued)}{}
on all other pages, you would give the commands
\begin{verbatim}
\lhead{Math 115}
\chead[Second Exam]{}
\rhead[July 4, 1776]{Second Exam Continued)}
\end{verbatim}

Any of the three parts of the header can have multiple lines.  To
specify where the line breaks should go, you just type \verb"\\". 
Thus, to have the
%
header\index{header!multiple lines}
\samplehead{\bfseries\large Math 115\\Professor Hilbert}{}
       {\bfseries\large First Exam\\July 4, 1776}{}
appear on every page, you would give the commands
\begin{verbatim}
\lhead{\bfseries\large Math 115\\Professor Hilbert}
\chead{}
\rhead{\bfseries\large First Exam\\July 4, 1776}
\end{verbatim}

\indcstop{lhead}
\indcstop{chead}
\indcstop{rhead}

\index{header!three parts|)}


%---------------------------------------------------------------------
\subsubsection*{Leaving extra room for multiple line headers}

See section~\ref{sec:extra-room} for a description of the
%
\verb"\extraheadheight"\index{header!multiple lines} command.

%---------------------------------------------------------------------
\subsection{The three parts of the footer}
\label{sec:footer}
\index{footer!three parts|(}

The footer is composed of three parts, the whole setup being similar
to that for the header (see section~\ref{sec:header}).  There are two
different ways in which you can specify the three parts of the footer.
The first uses the single command \verb"\footer" to specify all three
parts of the footer, or the commands \verb"\firstpagefooter" and
\verb"\runningfooter" to specify a different footer for the first page
(see section~\ref{sec:runningfooter}).  The second uses the commands
\verb"\lfoot", \verb"\cfoot", and \verb"\rfoot", each of which can
have an optional argument to specify a different footer for the first
page (see section~\ref{sec:lfoot}).  All of these commands should be
given in the preamble, i.e., after the \verb"\documentclass" command
and before the \verb"\begin{document}" command.

%--------------------------------------------------------------------
\subsubsection{Using \texttt{\bs footer}, \texttt{\bs firstpagefooter}
  and \texttt{\bs runningfooter}}
\label{sec:runningfooter}
\indcstart{footer}
\indcstart{firstpagefooter}
\indcstart{runningfooter}

The command \verb"\footer{Text 1}{Text 2}{Text 3}" puts ``Text~1''
into the left justified footer, ``Text~2'' into the centered footer
and ``Text~3'' into the right justified footer on every page.  If you
want the footer on the first page to be different from that on the
other pages, then you should use the commands \verb"\firstpagefooter"
and \verb"\runningfooter", which also take three arguments and affect
either the first page or all pages except the first.

For example, to have an empty footer on the first page and the footer
\samplefoot{}{}{Page 3 of 5}{}
on all pages after the first, you would give the commands
\begin{verbatim}
\firstpagefooter{}{}{}
\runningfooter{}{Page \thepage\ of \numpages}{}
\end{verbatim}
(For an explanation of the \verb"\numpages" command, see
section~\ref{sec:numpages}.)  

\indcstop{footer}
\indcstop{firstpagefooter}
\indcstop{runningfooter}

%--------------------------------------------------------------------
\subsubsection{Using \texttt{\bs lfoot}, \texttt{\bs cfoot}, and \texttt{\bs rfoot}}
\label{sec:lfoot}
\indcstart{lfoot}
\indcstart{cfoot}
\indcstart{rfoot}


The command \verb"\lfoot{Text}" puts ``Text'' into the left justified
part of the footer on every page.  The command
\verb"\lfoot[Text 1]{Text 2}" puts ``Text~1'' into the left justified
footer on the first page and ``Text~2'' into the left justified footer
on all other pages.  The commands \verb"\cfoot" and \verb"\rfoot" have
similar effects on the centered and right justified parts of the
footer. 

For example, to have an empty footer on the first page and the footer
\samplefoot{}{}{Page 3 of 5}{}
on all pages after the first, you would give the commands
\begin{verbatim}
\lfoot{}
\cfoot{}
\rfoot[]{Page \thepage\ of \numpages}
\end{verbatim}
(For an explanation of the \verb"\numpages" command, see
section~\ref{sec:numpages}.)  

\indcstop{lfoot}
\indcstop{cfoot}
\indcstop{rfoot}

\index{footer!three parts|)}

%---------------------------------------------------------------------
\subsubsection*{Leaving extra room for multiple line footers}

See section~\ref{sec:extra-room} for a description of the
\verb"\extrafootheight" command.

%---------------------------------------------------------------------
\subsection{Leaving extra room for multiple line headers and footers}
\label{sec:extra-room}

\subsubsection*{Headers}
\index{header!with multiple lines|(}

If you specify more than one or two lines for any part of the header,
then you may want to move the header down slightly so that it doesn't
run off of the top of the paper.  The command for this is
\verb"\extraheadheight"\indc{extraheadheight}.  You can also use the
\verb"\extraheadheight" command to adjust the size of the text area.
The \verb"\extraheadheight" command never changes the distance between
the header and the text.

For example, to move the header and the text an additional half inch
down from the top of the page, you would give the command
\begin{center}
  \verb"\extraheadheight{.5in}"
\end{center}
You can also specify a negative distance to \verb"\extraheadheight" to
move the header up closer to the top of the paper.  For example, the
command
\begin{center}
  \verb"\extraheadheight{-.25in}"
\end{center}
moves the header one quarter inch closer to the top of the paper.

If you want to have a different value for \verb"\extraheadheight" on
the first page from that on the pages after the first, then use the
same syntax as in the \verb"\lhead", \verb"\chead", and \verb"\rhead"
commands (see section~\ref{sec:lhead}): Include an optional argument
giving the extra head height for the first page, and the required
argument will apply only to those pages after the first.  For example,
the
%
command\indcsub{extraheadheight}{different on first page}
\begin{center}
  \verb"\extraheadheight[.5in]{.25in}"
\end{center}
gives a half inch of extra head height on the first page and a
quarter inch of extra head height on all pages after the first. 
If you say
\begin{center}
  \verb"\extraheadheight[.5in]{}"
\end{center}
then this will be interpreted as if it was
\begin{center}
  \verb"\extraheadheight[.5in]{0in}"
\end{center}
Note that the braces \emph{must} appear.

If you give an \verb"\extraheadheight"\indc{extraheadheight} command,
it should be in the preamble, i.e., after the \verb"\documentclass"
command and before the \verb"\begin{document}" command.  The
  \verb"\extraheadheight" command can also be used to change the size
  of the text region (see section~\ref{sec:pagesize}).

\index{header!with multiple lines|)}

%--------------------------------------------------------------------
\subsubsection*{Footers}
\index{footer!with multiple lines|(}


To leave extra room for multiple line footers, you use the command
\verb"\extrafootheight".  To move the footer one half inch higher up
on the paper, you would give the command\indc{extrafootheight}
\begin{center}
  \verb"\extrafootheight{.5in}"
\end{center}
If you wanted to move the footer an eighth of an inch lower down on
the paper, you would give the command
\begin{center}
  \verb"\extrafootheight{-.125in}"
\end{center}

If you want to have a different value for \verb"\extrafootheight" on
the first page from that on the pages after the first, then use the
same syntax as in the \verb"\lfoot", \verb"\cfoot", and \verb"\rfoot"
commands (see section~\ref{sec:lfoot}): Include an optional argument
giving the extra foot height for the first page, and the required
argument will apply only to those pages after the first.  For example,
the
%
command\indcsub{extrafootheight}{different on first page}
\begin{center}
  \verb"\extrafootheight[.5in]{.25in}"
\end{center}
gives a half inch of extra foot height on the first page and a
quarter inch of extra foot height on all pages after the first. 
If you say
\begin{center}
  \verb"\extrafootheight[.5in]{}"
\end{center}
then this will be interpreted as if it was
\begin{center}
  \verb"\extrafootheight[.5in]{0in}"
\end{center}
Note that the braces \emph{must} appear.

If you give an \verb"\extrafootheight"\indc{extrafootheight} command,
it should be in the preamble, i.e., after the \verb"\documentclass"
command and before the \verb"\begin{document}" command.  The
  \verb"\extrafootheight" command can also be used to change the size
  of the text region (see section~\ref{sec:pagesize}).

\index{footer!with multiple lines|)}


%---------------------------------------------------------------------
\subsection{Horizontal rules}
\label{sec:rules}
\index{rule!in header or footer|(}
\index{header!rule|(}
\index{footer!rule|(}

The \verb"exam" document class make it easy to put a horizontal rule
under the header and one above the footer.  It is also easy to do
this for the pages after the first page without affecting the first
page.

\begin{itemize}
\item The command \verb"\runningheadrule"\indc{runningheadrule} puts a
  horizontal rule below the header on all pages after the first.
\item The command \verb"\firstpageheadrule"\indc{firstpageheadrule}
  puts a rule under the header of only the first page.
\item The command \verb"\headrule"\indc{headrule} is equivalent to the
  two commands \verb"\firstpageheadrule" and \verb"\runningheadrule".
\item The command \verb"\runningfootrule"\indc{runningfootrule} puts a
  horizontal rule above the footer on all pages after the first.
\item The command \verb"\firstpagefootrule"\indc{firstpagefootrule}
  puts a rule above the footer of only the first page.
\item The command \verb"\footrule"\indc{footrule} is equivalent to the
  two commands \verb"\firstpagefootrule" and \verb"\runningfootrule".
\end{itemize}
For example, to have the header
\samplehead{Math 115}{First Exam}{July 4, 1776}{}
on the first page, with the header
\samplehead{Math 115}{First Exam}{July 4, 1776}{\hrule}
on all pages after the first, give either the commands
\begin{verbatim}
\runningheadrule
\header{Math 115}{First Exam}{July 4, 1776}
\end{verbatim}
or the commands
\begin{verbatim}
\runningheadrule
\lhead{Math 115}
\chead{First Exam}
\rhead{July 4, 1776}
\end{verbatim}
For a description of the command \verb"\header" see
section~\ref{sec:runningheader}; for a description of the commands
\verb"\lhead", \verb"\chead", and \verb"\rhead", see
section~\ref{sec:lhead}.

\bigskip

To have no footer on the first page, and the footer
\samplefoot{\hrule}{}{Page 3 of 5}{}
on all pages after the first, give the commands
\begin{verbatim}
\runningfootrule
\lfoot{}
\cfoot[]{Page \thepage\ of \numpages}
\rfoot{}
\end{verbatim}
(For an explanation of the \verb"\numpages" command, see
section~\ref{sec:numpages}.)  

\index{rule!in header or footer|)}
\index{header!rule|)}
\index{footer!rule|)}


%---------------------------------------------------------------------
\subsection{Listing the number of pages in the exam}
\label{sec:numpages}

The \verb"exam" document class defines the command
\verb"\numpages"\indc{numpages} so that it will expand to the number
of pages in the exam.  Thus, to have the footer
%
\samplefoot{}{}{Page \thepage\ of \numpages}{}
%
you should either give the command
\begin{verbatim}
\footer{}{Page \thepage\ of \numpages}{}
\end{verbatim}
or the commands
\begin{verbatim}
\lfoot{}
\cfoot{Page \thepage\ of \numpages}
\rfoot{}
\end{verbatim}
For a description of the command \verb"\footer" see
section~\ref{sec:runningfooter}; for a description of the commands
\verb"\lfoot", \verb"\cfoot", and \verb"\rfoot", see
section~\ref{sec:lfoot}.

Note that if your document has a \texttt{coverpages} environment (see
section~\ref{sec:coverpages}), then those pages will not be counted in
%
\verb"\numpages"\indcsub{numpages}{doesn't count cover pages}; the
\texttt{exam} document class defines the command
\verb"\numcoverpages"\indc{numcoverpages} so that it will expand to
the number of cover pages in the exam.  If there is no
\texttt{coverpages}\indt{coverpages}\index{environment!coverpages}
environment, then \verb"\numcoverpages" expands to 0.  The
\texttt{exam} class also defines the command
\verb"\totalnumpages"\indc{totalnumpages}, which expands to the sum of
\verb"\numpages" and \verb"\numcoverpages".  Note that if there are
pages preceding a \texttt{coverpages} environment, then those pages
will not be counted by \verb"\totalnumpages".

As with all other cross referencing commands in \LaTeX, you'll have to
run the file through \LaTeX{} \emph{twice} to be sure that
\verb"\numpages", \verb"\numcoverpages", and \verb"\totalnumpages" are
correct.



%---------------------------------------------------------------------
\subsection{Treating the last page differently}
\label{sec:lastpage}
\index{last page|(}

If you want to vary the text that appears in the header or footer on
the last page of the exam, you should use the command
\verb"\iflastpage"\indc{iflastpage}.  The command
\begin{center}
  \verb"\iflastpage{Text 1}{Text 2}"
\end{center}
expands to `Text~1' on the last page and to `Text~2' on all pages
before the last.  Thus, to have the footer
%
\samplefoot{}{}{Please go on to the next page\ldots}{}
%
on all pages before the last page and the footer
%
\samplefoot{}{}{End of exam}{}
%
on the last page, you should give the command
\begin{verbatim}
\footer{}
       {\iflastpage{End of exam}{Please go on to the next page\ldots}}
       {}
\end{verbatim}
or the commands
\begin{verbatim}
\lfoot{}
\cfoot{\iflastpage{End of exam}{Please go on to the next page\ldots}}
\rfoot{}
\end{verbatim}
For a description of the command \verb"\footer" see
section~\ref{sec:runningfooter}; for a description of the commands
\verb"\lfoot", \verb"\cfoot", and \verb"\rfoot", see
section~\ref{sec:lfoot}.

As with all other cross referencing commands in \LaTeX, you'll have to
run the file through \LaTeX{} \emph{twice} to be sure that
\verb"\iflastpage"\indc{iflastpage} correctly detects the last page.

\index{last page|)}

%--------------------------------------------------------------------
\subsection{Treating odd and even numbered pages differently}
\index{odd pages!different from even pages|(}
\index{even pages!different from odd pages|(}

If you'd like odd and even numbered pages to be treated differently
(perhaps because you'll be printing onto both sides of the paper), you
should use the \verb"\oddeven"\indc{oddeven} command.  The command
\begin{center}
  \verb"\oddeven{Text 1}{Text 2}"
\end{center}
expands to `Text~1' on odd numbered pages and to `Text~2' on even
numbered pages.

For example, to have the page number printed in the right head on odd
numbered pages and in the left head on even numbered pages, you should
give the command\indc{oddeven}
\begin{verbatim}
\header{\oddeven{}{\thepage}}%
       {}%
       {\oddeven{\thepage}{}}
\end{verbatim}
or the commands
\begin{center}
\begin{verbatim}
\lhead{\oddeven{}{\thepage}}
\rhead{\oddeven{\thepage}{}}
\chead{}
\end{verbatim}
\end{center}
If you want the footer of the even numbered pages to be empty and the
footer of the odd numbered pages to contain the centered message
``Please continue\dots'', except that the last page of the exam should
have an empty footer whether its page number is even or odd, then you
should give the command\indc{iflastpage}
\begin{verbatim}
\footer{}
       {\oddeven{\iflastpage{}{Please continue\ldots}}{}}
       {}
\end{verbatim}
or the commands
\begin{center}
\begin{verbatim}
\lfoot{}
\rfoot{}
\cfoot{\oddeven{\iflastpage{}{Please continue\dots}}{}}
\end{verbatim}
\end{center}
(see section~\ref{sec:lastpage} for an explanation of
\verb"\iflastpage"). 

The \verb"\oddeven"\indcsub{oddeven}{only in headers and footers}
command can be used reliably only in headers and footers.  This is
because \LaTeX{} generally processes more text than can fit on the
current page before chooses a page break it outputs a page.  When the
\verb"\oddeven" command is encountered it will act as though it will
appear on the current page whether it will actually be printed on that
page or on the following page; in headers and footers this never
creates a problem.

\index{odd pages!different from even pages|)}
\index{even pages!different from odd pages|)}

%--------------------------------------------------------------------
\subsection{Questions that span multiple pages}
\label{sec:QuesSpan}
\index{questions!that span multiple pages|(}

The exam document class provides the commands
\begin{itemize}
\item \verb"\ifcontinuation"\indc{ifcontinuation} and
  \verb"\ContinuedQuestion"\indc{ContinuedQuestion} to enable headers
  and footers that announce whether the current page is continuing a
  question begun on an earlier page and, if so, the number of that
  question (see section~\ref{sec:ifcontin}), and
\item \verb"\ifincomplete"\indc{ifincomplete} and
  \verb"\IncompleteQuestion"\indc{IncompleteQuestion} to enable
  headers and footers that announce whether the question at the bottom
  of the current page continues onto the next page and, if so, the
  number of that question (see section~\ref{sec:ifincomp}).
\end{itemize}
\emph{These commands are only guaranteed to work reliably in headers
  and footers.}%
\indcsub{ifcontinuation}{reliable only in headers and footers}%
\indcsub{ContinuedQuestion}{reliable only in headers and footers}%
\indcsub{ifincomplete}{reliable only in headers and footers}%
\indcsub{IncompleteQuestion}{reliable only in headers and footers} If
you use any of them elsewhere, they may be fooled by \LaTeX's practice
of typesetting a bit too much material to fit on a page before it
chooses a page break and ships out the page.  They also assume that
there is only one \verb"questions" environment in the entire exam.

%--------------------------------------------------------------------
\subsubsection{Questions continuing from the previous page}
\label{sec:ifcontin}

Headers and footers can announce whether the current page continues a
question begun on an earlier page.  The command\indc{ifcontinuation}
\begin{center}
  \verb"\ifcontinuation{Text 1}{Text 2}"
\end{center}
expands to `Text 1' if the current page begins with a continuation of
a question begun on an earlier page, and to `Text 2' otherwise.  If
the current page does continue a question begun on an earlier page,
then the command\indc{ContinuedQuestion}
\begin{center}
  \verb"\ContinuedQuestion"
\end{center}
expands to the number of that question.

For example, to have the header
\samplehead{}{First Exam}{Page 4}{}
%
on all pages that begin a new question and the header
\samplehead{Question 6 continues\ldots}{First Exam}{Page 4}{}
%
on all pages that continue a question begun on a previous page, you
would give the commands\indc{ContinuedQuestion}\indc{ifcontinuation}
\begin{verbatim}
\lhead{\ifcontinuation{Question \ContinuedQuestion\ continues\ldots}{}}
\chead{First Exam}
\rhead{Page \thepage}
\end{verbatim}
(See section~\ref{sec:lhead} for an explanation of \verb"\lhead",
\verb"\chead" and \verb"\rhead".)





The \verb"\ifcontinuation" command figures out which text to use by
checking whether there is a \verb"\question" command appearing on the
page before any of the commands \verb"\part", \verb"\subpart",
\verb"\subsubpart", \verb"\choice", or \verb"\CorrectChoice", and this
works well for most pages.  However, the \verb"\ifcontinuation"
command can make mistakes on pages on which none of those commands
appear.  The default is to assume that a page with none of those
commands continues a question begun on an earlier page, but there are
two commands you can use to tell the \verb"\ifcontinuation" command
that it should \emph{not} assume that:
\begin{itemize}
\item the \verb"\nomorequestions"\indc{nomorequestions} command can be
  used to mark the point at which the last question ends.  This can be
  useful if you want to include supplementary material (e.g., tables
  or figures) at the end of the exam but you don't want that material
  to be described as continuing the last question.  The
  \verb"\nomorequestions" command also tells the \verb"\ifincomplete"
  command (see section~\ref{sec:ifincomp}) that all questions are now
  complete.
\item The \verb"\noquestionsonthispage" command tells the
  \verb"\ifcontinuation" command (and the \verb"\ifincomplete"
  command; see section~\ref{sec:ifincomp}) that this page does not
  continue a question begun on an earlier page.  This is similar to
  the job done by the \verb"\nomorequestions" command for the pages
  that follow the end of all of the questions.  Note that if you're
  tempted to use this command on a page that follows the end of all of
  the questions, then you should probably use the command
  \verb"\nomorequestions" instead.
\end{itemize}

Thus, in more detail, the command
\verb"\ifcontinuation{Text 1}{Text 2}"
\begin{itemize}
\item expands to `Text 2' if
  \begin{itemize}
  \item a \verb"\question" appears on this page before any
    \verb"\part", \verb"\subpart", \verb"\subsubpart", \verb"\choice",
    or \verb"\CorrectChoice", or
  \item the command \verb"\noquestionsonthispage" has been given on
    this page, or
  \item the current page is before the page containing question number
    1, or
  \item the current page is later than a page with the
    \verb"\nomorequestions" command,
  \end{itemize}
\item and otherwise it expands to `Text 1'.
\end{itemize}


%--------------------------------------------------------------------
\subsubsection{Questions that continue onto a later page}
\label{sec:ifincomp}
\indcstart{ifincomplete}

Headers and footers can announce whether a question on the current
page continues onto a later page.  The command\indc{ifincomplete}
\begin{center}
  \verb"\ifincomplete{Text 1}{Text 2}"
\end{center}
expands to `Text 1' if the last question begun on or before the
current page continues onto a page after this one, and to `Text~2'
otherwise.  If the last question begun on or before the current page
does continue onto a later page, then the
command\indc{IncompleteQuestion}
\begin{center}
  \verb"\IncompleteQuestion"
\end{center}
expands to the number of that question.

For example, to have the footer
\samplefoot{\hrule}{}{Question 3 continues\ldots}{}
%
on all pages that end with a question that will be continued onto the
next page, the footer
\samplefoot{\hrule}{}{End of Exam}{}
%
on the last page, and the footer
\samplefoot{\hrule}{}{Exam continues\ldots}{}
%
on all other pages, you would give the
commands\indc{ifincomplete}\indc{IncompleteQuestion}\indc{iflastpage}
\begin{verbatim}
\footrule
\lfoot{}
\newcommand{\continues}{\ifincomplete
                     {Question \IncompleteQuestion\ continues\ldots}
                     {Exam continues\ldots}}
\cfoot{\iflastpage{End of Exam}{\continues}}
\rfoot{}
\end{verbatim}
(See section~\ref{sec:lfoot} for an explanation of \verb"\lfoot",
\verb"\cfoot" and \verb"\rfoot", and section~\ref{sec:lastpage} for an
explanation of \verb"\iflastpage").

There are two situations in which you may have to give an additional
command for the \verb"\ifincomplete" command to correctly decide if
the current question is incomplete:
\begin{enumerate}
\item The \verb"\ifincomplete" command expands to `Text~2' if there is
  a \verb"\part", \verb"\subpart", \verb"\subsubpart", \verb"\choice",
  or \verb"\CorrectChoice" command belonging to this question on the
  following page.  For those situations in which the current question
  flows onto the next page, but there is no \verb"\part",
  \verb"\subpart", \verb"\subsubpart", \verb"\choice", or
  \verb"\CorrectChoice" belonging to this question on that page, you
  should give the command\indc{addquestionobject}
  \begin{center}
    \verb"\addquestionobject"
  \end{center}
  at the end of the text of the question to inform the
  \verb"\ifincomplete" command that some portion of the question
  appears on that following page.

\item If there are pages on which none of those commands appear, the
  default is to assume that that page continues a question begun on an
  earlier page.  There are two commands that you can use to tell the
  \verb"\ifincomplete" command that it should \emph{not} assume that:
  \begin{itemize}
  \item the \verb"\nomorequestions"\indc{nomorequestions} command can
    be used to mark the point at which the last question ends.  This
    can be useful if you want to include supplementary material (e.g.,
    tables or figures) at the end of the exam but you don't want the
    last question on the exam to be described as incomplete.  The
    \verb"\nomorequestions" command also tells the
    \verb"\ifcontinuation" command (see section~\ref{sec:ifcontin})
    that all questions are now complete.
  \item The \verb"\noquestionsonthispage"\indc{noquestionsonthispage}
    command tells the \verb"\ifincomplete" command (and the
    \verb"\ifcontinuation" command; see section~\ref{sec:ifcontin})
    that this page does not contain any incomplete questions, and does
    not continue a question begun on an earlier page.  This is similar
    to the job done by the \verb"\nomorequestions" command for the
    pages that follow the end of all of the questions.  Note that if
    you're tempted to use this command on a page that follows the end
    of all of the questions, then you should probably use the command
    \verb"\nomorequestions" instead.
  \end{itemize}
\end{enumerate}
If you give the command
\verb"\noquestionsonthispage"\indc{noquestionsonthispage} on a page,
then
\begin{itemize}
\item an \verb"\ifcontinuation"\indc{ifcontinuation} command on that
  page will expand to its second argument,
\item an \verb"\ifincomplete" on that page will expand to its second
  argument, and
\item an \verb"\ifincomplete" command on an earlier page will not
  assume that a question from that earlier page continues onto this
  page.
\end{itemize}
The way that this command affects the \verb"\ifincomplete" command on
earlier pages is as follows: If there is a page with no
\verb"\question", \verb"\part", \verb"\subpart", \verb"\subsubparts",
\verb"\choice", \verb"\CorrectChoice", or
\verb"\addquestionobject"\indc{addquestionobject} commands, then the
last page before that with a \verb"\question" (or \verb"\part", etc.)
would normally be deemed to have an incomplete question; if, however,
the page with no \verb"\question" (or \verb"\part", etc.)  (along with
all adjacent pages with no \verb"\question" or \verb"\part" etc.) has
a \verb"\noquestionsonthispage" command, then that last page with a
\verb"\question" (or \verb"\part", etc.) will not be deemed
incomplete.


\indcstop{ifincomplete}

\index{questions!that span multiple pages|)}

%--------------------------------------------------------------------
\subsection{Examples}
\indcstart{lhead}
\indcstart{chead}
\indcstart{rhead}
\indcstart{lfoot}
\indcstart{cfoot}
\indcstart{rfoot}

\begin{example}
  \label{sec:beginexamples}
  To have the header
  \samplehead{Math 115}{First Exam}{July 4, 1776}{}
  on the first page, the header
  \samplehead{Math 115}{First Exam, Page 2 of 5}{July 4, 1776}{\hrule}
  on all pages after the first, and no footer on any page, give the
  commands\indc{firstpageheader}\indc{runningheader}%
  \indc{runningheadrule}
\begin{verbatim}
\pagestyle{head}
\runningheadrule
\firstpageheader{Math 115}{First Exam}{July 4, 1776}
\runningheader{Math 115}
              {First Exam, Page \thepage\ of \numpages}
              {July 4, 1776}
\end{verbatim}
  Alternatively, you could give the commands
\begin{verbatim}
\pagestyle{head}
\runningheadrule
\lhead{Math 115}
\chead[First Exam]{First Exam, Page \thepage\ of \numpages}
\rhead{July 4, 1776}
\end{verbatim}
\end{example}

%--------------------------------------------------------------------

\begin{example}
  To have the header
  \samplehead{Math 115}{First Exam}{July 4, 1776}{}
  on the first page, no header on the pages after the first, no footer
  on the first page, and the footer
  \samplefoot{\hrule}{Math 115}{First Exam}{Page 2 of 5}
  on all pages after the first, give the
  commands\indc{runningfootrule}\indc{firstpageheader}% 
  \indc{runningheader}\indc{firstpagefooter}\indc{runningfooter}
\begin{verbatim}
\pagestyle{headandfoot}
\runningfootrule
\firstpageheader{Math 115}{First Exam}{July 4, 1776}
\runningheader{}{}{}
\firstpagefooter{}{}{}
\runningfooter{Math 115}{First Exam}{Page \thepage\ of \numpages}
\end{verbatim}
  Alternatively, you could give the commands
\begin{verbatim}
\pagestyle{headandfoot}
\runningfootrule
\lhead[Math 115]{}
\chead[First Exam]{}
\rhead[July 4, 1776]{}
\lfoot[]{Math 115}
\cfoot[]{First Exam}
\rfoot[]{Page \thepage\ of \numpages}
\end{verbatim}
\end{example}

%--------------------------------------------------------------------

\begin{example}
  To have the header
  \samplehead{\large\bfseries Mathematics 115\\
              First Exam, July 4, 1776}%
    {}{\large\bfseries Name:\enspace\makebox[2in]{\hrulefill}}{}
  on the first page, the header
  \samplehead{\large\bfseries Mathematics 115\\
              First Exam, July 4, 1776}%
  {}{}{}
  on all pages after the first, an empty footer on the first page, and
  the footer
  \samplefoot{}{}{Page 2}{}
  on all pages after the first, give the
  commands\indc{firstpageheader}\indc{runningheader}%
  \indc{firstpagefooter}\indc{runningfooter}
\begin{verbatim}
\pagestyle{headandfoot}
\firstpageheader{\large\bfseries Mathematics 115\\
                  First Exam, July 4, 1776}
                {}
                {\large\bfseries Name:\enspace\makebox[2in]{\hrulefill}}
\runningheader{\large\bfseries Mathematics 115\\
               First Exam, July 4, 1776}{}{}
\firstpagefooter{}{}{}
\runningfooter{}{Page \thepage}{}
\end{verbatim}
  Alternatively, you could give the commands
\begin{verbatim}
\pagestyle{headandfoot}
\lhead{\large\bfseries Mathematics 115\\ First Exam, July 4, 1776}
\chead{}
\rhead[\large\bfseries Name:\enspace\makebox[2in]{\hrulefill}]{}
\lfoot{}
\cfoot[]{Page \thepage}
\rfoot{}
\end{verbatim}
\end{example}

%--------------------------------------------------------------------

\begin{example}
  To have the header
  \samplehead{}
             {Wellesley College\\
               Second Semester Final Examination, Spring 1993\\
               Mathematics 115}
             {}
             {}
  on the first page, the header
  \samplehead{}
             {Wellesley College\\
               Second Semester Final Examination, Spring 1993\\
               Mathematics 115 (Continued)}
             {}
             {}
  on all pages after the first, the footer
  \samplefoot{}{}{Page 3 of 10}{Please go on to the next page\ldots}
  on all pages \emph{except the last} page, and the footer
  \samplefoot{}{}{Page 10 of 10}{End of exam.}
  on the last page, give the
  commands\indc{extraheadheight}\indc{firstpageheader}%
  \indc{runningheader}\indc{footer}\indc{numpages}\indc{thepage}%
  \indc{iflastpage}
\begin{verbatim}
\pagestyle{headandfoot}
\extraheadheight{.25in}
\firstpageheader{}
                {Wellesley College\\
                  Second Semester Final Examination, Spring 1993\\
                  Mathematics 115}
                {}
\runningheader{}
              {Wellesley College\\
                Second Semester Final Examination, Spring 1993\\
                Mathematics 115 (Continued)}
              {}
\footer{}
       {Page \thepage\ of \numpages}
       {\iflastpage{End of exam.}{Please go on to the next page\ldots}}
\end{verbatim}
  Alternatively, you could give the commands
\begin{verbatim}
\pagestyle{headandfoot}
\extraheadheight{.25in}
\lhead{}
\chead[Wellesley College\\
        Second Semester Final Examination, Spring 1993\\
        Mathematics 115]
      {Wellesley College\\
        Second Semester Final Examination, Spring 1993\\
        Mathematics 115 (Continued)}
\rhead{}
\lfoot{}
\cfoot{Page \thepage\ of \numpages}
\rfoot{\iflastpage{End of exam.}{Please go on to the next page\ldots}}
\end{verbatim}
\end{example}

%--------------------------------------------------------------------

\begin{example}
  To have the header
  \samplehead{}
             {Wellesley College\\
               Second Semester Final Examination, Spring 1993\\
               Mathematics 115}
             {}
             {}
  on the first page, the header
  \samplehead{Mathematics 115 (Continued)}{}{Spring, 1993}{}
  on all pages after the first, the footer
  \samplefoot{}{}{Page 3 of 10}{Please go on to the next page\ldots}
  on all pages \emph{except the last} page, and the footer
  \samplefoot{}{}{Page 10 of 10}{End of exam.}
  on the last page, give the commands\indc{extraheadheight}%
  \indc{firstpageheader}\indc{runningheader}\indc{footer}
\begin{verbatim}
\pagestyle{headandfoot}
\extraheadheight[.25in]{}
\firstpageheader{}
                {Wellesley College\\
                  Second Semester Final Examination, Spring 1993\\
                  Mathematics 115}
                {}
\runningheader{Mathematics 115 (Continued)}{}{Spring, 1993}
\footer{}
       {Page \thepage\ of \numpages}
       {\iflastpage{End of exam.}{Please go on to the next page\ldots}}
\end{verbatim}
  Alternatively, you could give the commands
\begin{verbatim}
\pagestyle{headandfoot}
\extraheadheight[.25in]{}
\lhead[]{Mathematics 115 (Continued)}
\chead[Wellesley College\\
        Second Semester Final Examination, Spring 1993\\
        Mathematics 115]
      {}
\rhead[]{Spring, 1993}
\lfoot{}
\cfoot{Page \thepage\ of \numpages}
\rfoot{\iflastpage{End of exam.}{Please go on to the next page\ldots}}
\end{verbatim}
\end{example}

%--------------------------------------------------------------------

\begin{example}
  To have the header
%
  \samplehead{Math 115}{First Exam}{July 4, 1776}{}
  on the first page, the header
%
  \samplehead{Question 3 continues\ldots}{First Exam}{Page 3 of 10}{\hrule}
%
  on all pages after the first that continue a question begun on an
  earlier page, the header
%
  \samplehead{}{First Exam}{Page 3 of 10}{\hrule}
  on all pages after the first that don't continue a question begun on
  an earlier page, the footer
%
  \samplefoot{}{}{}{Question 6 continues on the next page\ldots}
%
  on all pages whose last question continues onto the following page,
  and no footer on pages that don't have a question that continues
  onto the following page, give the
  commands\indc{runningheadrule}\indc{ifcontinuation}%
  \indc{ContinuedQuestion}\indc{firstpageheader}\indc{runningheader}%
  \indc{footer}\indc{thepage}\indc{numpages}\indc{ifincomplete}%
  \indc{IncompleteQuestion}
\begin{verbatim}
\pagestyle{headandfoot}
\runningheadrule
\newcommand{\continuedmessage}{%
  \ifcontinuation{Question \ContinuedQuestion\ continues\ldots}{}%
}
\firstpageheader{Math 115}{First Exam}{July 4, 1776}
\runningheader{\continuedmessage}
              {First Exam}
              {Page \thepage\ of \numpages}
\footer{}
       {}
       {\ifincomplete{Question \IncompleteQuestion\ continues
         on the next page\ldots}{}}
\end{verbatim}
  Alternatively, you could give the commands
\begin{verbatim}
\pagestyle{headandfoot}
\runningheadrule
\newcommand{\continuedmessage}{%
  \ifcontinuation{Question \ContinuedQuestion\ continues\ldots}{}%
}
\lhead[Math 115]{\continuedmessage}
\chead[First Exam]{First Exam}
\rhead[July 4, 1776]{Page \thepage\ of \numpages}
\lfoot{}
\cfoot{}
\rfoot{\ifincomplete{Question \IncompleteQuestion\ continues
        on the next page\ldots}{}}
\end{verbatim}
\end{example}

%--------------------------------------------------------------------
\begin{example}
  \label{sec:PtsPgEx}
  To have the header
%
  \samplehead{Math 115}{Second Exam}{July 4, 1776}{}
%
  on the first page, the header
%
  \samplehead{Second Exam}{}{July 4, 1776}{}
%
  on all pages after the first, no footer on the first page, and the
  footer
%
  \samplefoot{}{Page 3 of 5}{}{Points earned: \makebox[.5in]{\hrulefill}\\
               out of a possible 20 points}
%
  on all pages after the first, give the commands\indc{firstpageheader}%
  \indc{runningheader}\indc{runningfooter}\indc{pointsonpage}%
  \indc{thepage}\indc{numpages}
\begin{verbatim}
\pagestyle{headandfoot}
\firstpageheader{Math 115}{Second Exam}{July 4, 1776}
\runningheader{Second Exam}{}{July 4, 1776}
\runningfooter{Page \thepage\ of \numpages}
              {}
              {Points earned: \makebox[.5in]{\hrulefill}\\
                out of a possible \pointsonpage{\thepage} points}
\end{verbatim}
  Alternatively, you could give the commands
\begin{verbatim}
\pagestyle{headandfoot}
\lhead[Math 115]{Second Exam}
\chead[Second Exam]{}
\rhead{July 4, 1776}
\lfoot[]{Page \thepage\ of \numpages}
\cfoot{}
\rfoot[]{Points earned: \makebox[.5in]{\hrulefill}\\
          out of a possible \pointsonpage{\thepage} points}
\end{verbatim}
\end{example}

%--------------------------------------------------------------------

\begin{example}
  \label{sec:endexamples}
  To have the header
%
  \samplehead{Math 115}{Second Exam}{July 4, 1776}{}
%
  on the first page, the header
%
  \samplehead{}{Second Exam}{July 4, 1776}{}
%
  on all odd numbered pages after the first, the header
%
  \samplehead{Question 3 continues\ldots}{Math 115}{}{}
%
  on all even numbered pages that continue a question begun on an
  earlier page, the header
%
  \samplehead{}{Math 115}{}{}
%
  on all even numbered pages that don't continue a question begun on
  an earlier page, the footer
%
  \samplefoot{}{}{}{Page 5 of 10\\Question 6 continues\ldots}
%
  on all odd numbered pages that have a question that continues onto
  the following page, the footer
%
  \samplefoot{}{}{}{Page 5 of 10}
%
  on all odd numbered pages that don't have a question that continues
  onto the following page, and the footer
%
  \samplefoot{}{Page 4 of 10}{}{}
%
  on all even numbered pages, give the commands\indc{ifcontinuation}%
  \indc{thepage}\indc{numpages}\indc{IncompleteQuestion}%
  \indc{firstpageheader}\indc{runningheader}\indc{oddeven}%
  \indc{footer}
\begin{verbatim}
\pagestyle{headandfoot}
\newcommand{\continuedmessage}{%
  \ifcontinuation{Question \ContinuedQuestion\ continues\ldots}{}%
}
\newcommand{\oddrightfoot}{%
  \ifincomplete{Page \thepage\ of \numpages\\
                 Question \IncompleteQuestion\ continues\ldots}%
               {Page \thepage\ of \numpages}
}
\firstpageheader{Math 115}{Second Exam}{July 4, 1776}
\runningheader{\oddeven{}{\continuedmessage}}
              {\oddeven{Second Exam}{Math 115}}
              {\oddeven{July 4, 1776}{}}
\footer{\oddeven{}{Page \thepage\ of \numpages}}
       {}
       {\oddeven{\oddrightfoot}{}}
\end{verbatim}
  Alternatively, you could give the commands
\begin{verbatim}
\pagestyle{headandfoot}
\newcommand{\continuedmessage}{%
  \ifcontinuation{Question \ContinuedQuestion\ continues\ldots}{}%
}
\newcommand{\oddrightfoot}{%
  \ifincomplete{Page \thepage\ of \numpages\\
                 Question \IncompleteQuestion\ continues\ldots}%
               {Page \thepage\ of \numpages}
}
\lhead[Math 115]{\oddeven{}{\continuedmessage}}
\chead{\oddeven{Second Exam}{Math 115}}
\rhead{\oddeven{July 4, 1776}{}}
\lfoot{\oddeven{}{Page \thepage\ of \numpages}}
\cfoot{}
\rfoot{\oddeven{\oddrightfoot}{}}
\end{verbatim}
\end{example}


\indcstop{lhead}
\indcstop{chead}
\indcstop{rhead}
\indcstop{lfoot}
\indcstop{cfoot}
\indcstop{rfoot}
\index{footer|)}
\index{header|)}

%--------------------------------------------------------------------
%--------------------------------------------------------------------
\section{Cover pages}
\label{sec:coverpages}
\index{cover pages|(}
\index{environment!coverpages|(}
\indtstart{coverpages}

There is a \verb"coverpages" environment that allows you to have one
or more pages before page~1 of the exam.  By default, there are no
headers or footers printed on the cover pages, but if you put the page
number into the headers and/or footers (using the commands described
in section~\ref{sec:Coverhdft}), the page numbers of cover pages are
printed in roman numerals.

The \verb"coverpages" environment must begin and end \emph{before} the
beginning of the \verb"questions" environment.  You begin the
environment with the command
\begin{center}
  \verb"\begin{coverpages}"
\end{center}
and you end it with the command
\begin{center}
  \verb"\end{coverpages}"
\end{center}
In between those two commands you can put whatever you want, except
that you are not allowed to begin the \verb"questions" environment
until you've ended the \verb"coverpages" environment.  The command
\verb"\begin{coverpages}" sets the page number to 1 and causes any
page numbers printed (using the commands described in
section~\ref{sec:Coverhdft}) to appear as roman numerals.  The command
\verb"\end{coverpages}" causes the current page to end, sets the page
number of the following page to 1, and changes page numbering back to
arabic numerals.

%--------------------------------------------------------------------
\subsection{Headers and footers}
\label{sec:Coverhdft}
\index{header!on cover pages|(}
\index{footer!on cover pages|(}

By default, the headers and footers on the cover pages are empty.  If
you'd like to have headers and/or footers on the cover pages, there
are commands to create them that are completely analogous to the
commands described in section~\ref{sec:headfoot} for headers and
footers in the main part of the document.  The commands for headers
and footers in cover pages and the corresponding commands for the main
exam pages are:\indc{coverheader}\indc{coverrunningheader}%
\indc{coverfirstpageheader}\indc{coverlhead}\indc{coverchead}%
\indc{coverrhead}\indc{coverfooter}\indc{coverrunningfooter}%
\indc{coverfirstpagefooter}\indc{coverlfoot}\indc{covercfoot}%
\indc{coverrfoot}\indc{coverextraheadheight}%
\indc{coverextrafootheight}
\begin{center}
  \begin{tabular}{l@{\qquad\qquad}l}
Cover Pages Commands& Main Pages Commands\\[\medskipamount]
\verb"\coverheader"&\verb"\header"\\
\verb"\coverrunningheader"&\verb"\runningheader"\\
\verb"\coverfirstpageheader"&\verb"\firstpageheader"\\[\medskipamount]

\verb"\coverlhead"&\verb"\lhead"\\
\verb"\coverchead"&\verb"\chead"\\
\verb"\coverrhead"&\verb"\rhead"\\[\medskipamount]

\verb"\coverfooter"&\verb"\footer"\\
\verb"\coverrunningfooter"&\verb"\runningheader"\\
\verb"\coverfirstpagefooter"&\verb"\firstpageheader"\\[\medskipamount]

\verb"\coverlfoot"&\verb"\lfoot"\\
\verb"\covercfoot"&\verb"\cfoot"\\
\verb"\coverrfoot"&\verb"\rfoot"\\[\medskipamount]

\verb"\coverextraheadheight"&\verb"\extraheadheight"\\
\verb"\coverextrafootheight"&\verb"\extrafootheight"
  \end{tabular}
\end{center}
The commands \verb"\coverlhead", \verb"\coverchead",
\verb"\coverrhead", \verb"\coverlfoot", \verb"\covercfoot",
\verb"\coverrfoot", \verb"\coverextraheadheight", and
\verb"\coverextrafootheight" all all take the same optional arguments
(for special treatment of page number 1) as the corresponding commands
for the main pages.  For an explanation of these commands, see
section~\ref{sec:headfoot}.

\index{header!on cover pages|)}
\index{footer!on cover pages|)}

%--------------------------------------------------------------------
\subsection{Listing the number of cover pages}
\label{sec:NumCvrpgs}
\index{cover pages!counting|(}
\indcstart{numcoverpages}
\indcstart{totalnumpages}

The \texttt{exam} document class defines the command
\verb"\numcoverpages" so that it will expand to the number of cover
pages in the exam.  If there is no \texttt{coverpages} environment,
then \verb"\numcoverpages" expands to 0.  The \texttt{exam} class also
defines the command \verb"\totalnumpages", which expands to the sum of
\verb"\numpages" and \verb"\numcoverpages".
\emph{Note:} if there are pages preceding a \texttt{coverpages}
environment, then those pages will not be counted by
\verb"\totalnumpages".

For an example, to have the footer
%
\samplefoot{}{}{Page ii of iv}{}
on all cover pages you should either give the
command\indc{romannumeral}
\begin{verbatim}
\coverfooter{}{Page \thepage\ of \romannumeral\numcoverpages}{}
\end{verbatim}
or the commands
\begin{verbatim}
\coverlfoot{}
\covercfoot{Page \thepage\ of \romannumeral\numcoverpages}
\coverrfoot{}
\end{verbatim}
As with all other cross referencing commands in \LaTeX, you'll have to
run the file through \LaTeX{} \emph{twice} to be sure that
\verb"\numpages", \verb"\numcoverpages", and \verb"\totalnumpages" are
correct.

\index{cover pages!counting|)}
\indcstop{numcoverpages}
\indcstop{totalnumpages}

\indtstop{coverpages}
\index{environment!coverpages|)}
\index{cover pages|)}

%--------------------------------------------------------------------
\clearpage
\addcontentsline{toc}{section}{Index}
\printindex

% The following is necessary to avoid warnings, since we have multiple
% questions environments in this documentation file:
\makeatletter
\@pointschangedfalse
\makeatother




\end{document}
%--------------------------------------------------------------------
%--------------------------------------------------------------------
%--------------------------------------------------------------------
