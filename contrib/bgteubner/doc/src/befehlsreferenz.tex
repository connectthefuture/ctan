%
% bgteubner class bundle
%
% befehlsreferenz.tex
% Copyright 2003--2012 Harald Harders
%
% This program may be distributed and/or modified under the
% conditions of the LaTeX Project Public License, either version 1.3
% of this license or (at your opinion) any later version.
% The latest version of this license is in
%    http://www.latex-project.org/lppl.txt
% and version 1.3 or later is part of all distributions of LaTeX
% version 1999/12/01 or later.
%
% This program consists of all files listed in manifest.txt.
% ===================================================================
\glossarycmd{advanced@\cs{advanced}}{einger�ckter Textblock f�r weitergehende Dinge (\ref{sec:tex:saetze})}%
\glossarycmd{answer@\cs{answer}}{nummerierte L�sung (\ref{sec:tex:aufgaben})}%
\glossarycmd{answer*@\cs{answer*}}{unnummerierte L�sung (\ref{sec:tex:aufgaben})}%
\glossarycmd{author@\cs{author}}{Autoren (\ref{sec:tex:aufbau})}%
\glossarycmd{backmatter@\cs{backmatter}}{Schluss des Buchs (wird ignoriert)}%
\glossarycmd{bigskip@\cs{bigskip}}{Abschnitt mit gro�em Abstand (\ref{sec:tex:teile})}%
\glossarycmd{cases*@\env{cases*}}{Fallunterscheidung mit schlie�ender Klammer (\ref{sec:tex:mathematik})}%
\glossarycmd{d@\cs{d}}{Mathematik: Differentialoperator $\d$, Text: Punkt unter dem folgenden Zeichen \d a (\ref{sec:tex:mathematik})}%
\glossarycmd{D@\cs{D}}{Differenzoperator $\D$ (\ref{sec:tex:mathematik})}%
\glossarycmd{dedication@\cs{dedication}}{Widmung (\ref{sec:tex:aufbau})}%
\glossarycmd{e@\cs{e}}{eulersche Zahl (\ref{sec:tex:mathematik})}%
\glossarycmd{edition@\cs{edition}}{Auf"|lage des Buchs (\ref{sec:tex:aufbau})}%
\glossarycmd{engl@\cs{engl}}{fremdsprachiger Begriff
  (\ref{sec:tex:auszeichnungen})}% 
\glossarycmd{equivalent@\cs{equivalent}}{Entspricht"=Zeichen:
  $\equivalent$ (\ref{sec:tex:mathematik})}%
\glossarycmd{exercise@\env{exercise}}{nummerierte Aufgabe (\ref{sec:tex:aufgaben})}%
\glossarycmd{exercise*@\env{exercise*}}{unnummerierte Aufgabe (\ref{sec:tex:aufgaben})}%
\glossarycmd{frontmatter@\cs{frontmatter}}{Titelei des Buchs (\ref{sec:tex:aufbau})}%
\glossarycmd{glossary@\cs{glossary\meta{Name}}}{definiert einen Eintrag f�r einen Glossar des Typs \meta{Name} (\ref{sec:tex:glossary})}%
\glossarycmd{glossaryname@\cs{glossary\meta{Name}name}}{definiert den �berschriftsnamen f�r einen Glossar des Typs \meta{Name} (\ref{sec:tex:glossary})}%
\glossarycmd{glossarypreamble@\cs{glossary\meta{Name}preamble}}{definiert die Pr�ambel eines Glossars des Typs \meta{Name} (\ref{sec:tex:glossary})}%
\glossarycmd{grad@\cs{grad}}{Gradient (\ref{sec:tex:mathematik})}%
\glossarycmd{important@\env{important}}{grau hinterlegte Box (mit Text
  beginnend) (\ref{sec:tex:important})}%
\glossarycmd{important*@\env{important*}}{grau hinterlegte Box (mit
  einer abgesetzten Formel beginnend) (\ref{sec:tex:important})}%
\glossarycmd{longimportant@\env{longimportant}}{lange grau hinterlegte
  Box (mit Text beginnend) (\ref{sec:tex:important})}%
\glossarycmd{longimportant*@\env{longimportant*}}{lange grau hinterlegte
  Box (\ref{sec:tex:important})}%
\glossarycmd{listofexamples@\cs{listofexamples}}{Beispielverzeichnis
  (\ref{sec:tex:beispielverzeichnis})}%
\glossarycmd{listofexercises@\cs{listofexercises}}{Aufgabenverzeichnis
  (\ref{sec:tex:beispielverzeichnis})}%
\glossarycmd{listofdefinitions@\cs{listofdefinitions}}{Verzeichnis der
  Definitionen (\ref{sec:tex:beispielverzeichnis})}%
\glossarycmd{listofproofs@\cs{listofproofs}}{Verzeichnis der
  Beweise (\ref{sec:tex:beispielverzeichnis})}%
\glossarycmd{listoftheorems@\cs{listoftheorems}}{Verzeichnis der Umgebungen einer Theoremart (\ref{sec:tex:saetze})}%
\glossarycmd{listoffigures@\cs{listoffigures}}{Abbildungsverzeichnis
  (\ref{sec:tex:abbildungsverzeichnis})}%
\glossarycmd{listoftables@\cs{listoftables}}{Tabellenverzeichnis
  (\ref{sec:tex:abbildungsverzeichnis})}%
\glossarycmd{mainmatter@\cs{mainmatter}}{Hauptteil des Buchs (\ref{sec:tex:aufbau})}%
\glossarycmd{makeglossary@\cs{makeglossary}}{definiert einen neuen Typ an glossar�hnlicher Liste wie z.\,B.\ dieser Befehlsreferenz (\ref{sec:tex:glossary})}%
\glossarycmd{matr@\cs{matr}}{Matrix (\ref{sec:tex:mathematik})}%
\glossarycmd{medskip@\cs{medskip}}{Abschnitt mit mittlerem Abstand (\ref{sec:tex:teile})}%
\glossarycmd{new@\cs{new}}{neu eingef�hrter Begriff
  (\ref{sec:tex:auszeichnungen})}%
\glossarycmd{newtheorem@\cs{newtheorem}}{Einrichten einer neuen theoremartigen Umgebung (\ref{sec:tex:saetze})}%
\glossarycmd{nomathindent@\env{nomathindent}}{Einzug von abgesetzten
  Gleichungen reduzieren (\ref{sec:tex:mathematik})}%
\glossarycmd{person@\cs{person}}{Personenname
  (\ref{sec:tex:auszeichnungen})}%
\glossarycmd{printglossary@\cs{printglossary\meta{Name}}}{Setzen eines Glossars des Typs \meta{Name} (\ref{sec:tex:glossary})}%
\glossarycmd{smallskip@\cs{smallskip}}{Abschnitt mit kleinem Abstand (\ref{sec:tex:teile})}%
\glossarycmd{subanswer@\env{subanswer}}{nummerierte L�sung (\ref{sec:tex:aufgaben})}%
\glossarycmd{subanswer*@\env{subanswer*}}{unnummerierte L�sung (\ref{sec:tex:aufgaben})}%
\glossarycmd{subexercise@\env{subexercise}}{Nummerierte Aufgabe (\ref{sec:tex:aufgaben})}%
\glossarycmd{subexercise*@\env{subexercise*}}{unnummerierte Aufgabe (\ref{sec:tex:aufgaben})}%
\glossarycmd{subtask@\env{subtask}}{Teilaufgaben (\ref{sec:tex:aufgaben})}%
\glossarycmd{subtaskref@\cs{subtaskref}}{Referenz auf eine Teilaufgabe (\ref{sec:tex:aufgaben})}%
\glossarycmd{subtitle@\cs{subtitle}}{Untertitel des Buchs (\ref{sec:tex:aufbau})}%
\glossarycmd{tensor@\cs{tensor}}{Tensor h�herer Stufe (\ref{sec:tex:mathematik})}%
\glossarycmd{theglossary@\env{theglossary}}{manuell eingegebener Glossar (\ref{sec:tex:glossary})}%
\glossarycmd{example@\env{example}}{nummieriertes, abgesetztes Beispiel
  (\ref{sec:tex:saetze})}%
\glossarycmd{example*@\env{example*}}{unnummeriertes, abgesetztes Beispiel (\ref{sec:tex:saetze})}%
\glossarycmd{definition@\env{definition}}{nummerierte, abgesetzte
  Definition (\ref{sec:tex:saetze})}% 
\glossarycmd{definition*@\env{definition*}}{unnummerierte, abgesetzte
  Definition (\ref{sec:tex:saetze})}% 
\glossarycmd{proof@\env{proof}}{nummerierte, abgesetzte
  Beweise (\ref{sec:tex:saetze})}% 
\glossarycmd{proof*@\env{proof*}}{unnummerierte, abgesetzte
  Beweise (\ref{sec:tex:saetze})}% 
\glossarycmd{qed@\cs{qed}}{rechtsb�ndiger schwarzer Kasten zum
  Abschluss eines Beweises (\ref{sec:tex:saetze})}% 
\glossarycmd{qedname@\cs{qedname}}{Text am Ende eines Beweises
  (\ref{sec:tex:saetze})}%
\glossarycmd{title@\cs{title}}{Titel des Buchs (\ref{sec:tex:aufbau})}%
\glossarycmd{tr@\cs{tr}}{Spur eines Tensors (\ref{sec:tex:mathematik})}%
\glossarycmd{signature@\cs{signature}}{Unterschrift der Autoren f�r ein Vorwort (\ref{sec:vorwort})}%
\glossarycmd{vec@\cs{vec}}{Vektor (\ref{sec:tex:mathematik})}%
\glossarycmd{acro@\cs{acro}}{druckt eine Abk�rzung aus
  Versalien etwas kleiner (\ref{sec:tex:acro})}%
\glossarycmd{version@\cs{version}}{pr�ft, ob die richtige Version von
  \texttt{bgteubner.cls} verwendet wird (\ref{sec:fortgeschrittene})}%
\glossarycmd{preface@\cs{preface}}{�berschrift f�r Vorworte (\ref{sec:vorwort})}%
\glossarycmd{f@\cs{f}}{im Index ein "`f"' an eine Seitenzahl h�ngen
  (\ref{sec:tex_index})}% 
\glossarycmd{ff@\cs{ff}}{im Index ein "`ff"' an eine Seitenzahl h�ngen
  (\ref{sec:tex_index})}% 
\glossarycmd{textbff@\cs{textbff}}{im Index ein "`f"' an eine fettgedruckte
  Seitenzahl h�ngen (\ref{sec:tex_index})}% 
\glossarycmd{textbfff@\cs{textbfff}}{im Index ein "`ff"' an eine fettgedruckte
  Seitenzahl h�ngen (\ref{sec:tex_index})}% 
\glossarycmd{subind@\cs{subind}}{Querverweis auf einen Unterpunkt im
  Index (\ref{sec:tex_index})}% 
\glossarycmd{index@\cs{index}}{Indexeintrag erzeugen
  (\ref{sec:tex_index})}%
\glossarycmd{theoremdelimiter@\env{theoremdelimiter}}{den Doppelpunkt
  nach der �berschrift einer theoremartigen Umgebung ersetzen
  (\ref{sec:tex:theorem})}%
\glossarycmd{settheoremmargin@\cs{settheoremmargin}}{Einr�ckung
  theoremartiger Umgebungen ver�ndern (\ref{sec:tex:theorem})}%
\glossarycmd{exercisedelimiter@\env{exercisedelimiter}}{den Doppelpunkt
  nach der �berschrift einer Aufgabe oder L�sung ersetzen
  (\ref{sec:tex:aufgaben})}%

% ===================================================================

%%% Local Variables: 
%%% mode: latex
%%% TeX-master: bgteubner.tex
%%% TeX-master: "bgteubner"
%%% End: 
