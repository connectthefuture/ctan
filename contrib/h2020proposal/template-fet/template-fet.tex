%% EU FET Open  Proposal LaTeX template
%% V1.0
%% Based on the h2020proposal.cls LaTeX class for writing EU H2020 RIA proposals.
%% 
%% Copyright (c) 2010, Giacomo Indiveri
%%
%%  This latex class is free software: you can redistribute it and/or modify
%%  it under the terms of the GNU General Public License as published by
%%  the Free Software Foundation, either version 3 of the License, or
%%  (at your option) any later version.
%%
%%  h2020proposal.cls is distributed in the hope that it will be useful,
%%  but WITHOUT ANY WARRANTY; without even the implied warranty of
%%  MERCHANTABILITY or FITNESS FOR A PARTICULAR PURPOSE.  See the
%%  GNU General Public License for more details.
%%
%%  You should have received a copy of the GNU General Public License
%%  along with h2020proposal.cls.  If not, see <http://www.gnu.org/licenses/>.
%%
%% Contributors: Elisabetta Chicca
%%
%% Disclaimer: The template is based on the document provided by the EU Participants Portal 
%% "Part B  Template FETOPEN sections 1-3 Final.doc"
%%
%% Use the original source and the http://ec.europa.eu/ documentation for reference. We make no
%% representations or warranties of any kind, express or implied, about the completeness, accuracy,
%% reliability, suitability or availability with respect to the original template.
%% In no event will we be liable for any loss or damage including without limitation, indirect or
%% consequential loss or damage, or any loss or damage whatsoever arising out of, or in connection
%% with, the use of this template and/or class.
%%
%% Makes use of the memoir class. Read the optimum memman documentation for
%% info on how to customize your proposal.


%\documentclass[]{h2020proposal}     % Remove 'draft' option for final version
\documentclass[draft]{h2020proposal} % Use 'draft' option to show comments and labels

% For in-line comments use:
% \marginpar{comment text}

%% Extra Packages
%% ========
%\usepackage{fontspec}% Latin Modern by default with xelatex

%% LaTeX Font encoding -- DO NOT CHANGE
\usepackage[OT1]{fontenc}

%% Input encoding 'utf8'. In some cases you might need 'utf8x' for
%% extra symbols. Not all editors, especially on Windows, are UTF-8
%% capable, so you may want to use 'latin1' instead.
%\usepackage[utf8,latin1]{inputenc}

%% Babel provides support for languages.  'english' uses British
%% English hyphenation and text snippets like "Figure" and
%% "Theorem". Use the option 'ngerman' if your document is in German.
%% Use 'american' for American English.  Note that if you change this,
%% the next LaTeX run may show spurious errors.  Simply run it again.
%% If they persist, remove the .aux file and try again.
\usepackage[english]{babel}

%% For underlined wrapped text.
\usepackage{soul}

%% This changes default fonts for both text and math mode to use Herman Zapfs
%% excellent Palatino font.  Do not change this.
\usepackage[sc]{mathpazo} % Not needed with xelatex

%% The AMS-LaTeX extensions for mathematical typesetting.  Do not
%% remove.
\usepackage{amsmath,amssymb,amsfonts,mathrsfs}

%% Gantt Charts in LaTeX
\usepackage{pgfgantt}

%% LaTeX' own graphics handling
\usepackage{graphicx}

%% Fancy character protrusion.  Must be loaded after all fonts.
\usepackage[activate]{pdfcprot}

%% Nicer tables.  Read the excellent documentation.
\usepackage{booktabs}

% Compressed itemized lists (with a * at the end)
\usepackage{mdwlist}

%% Nicer URLs.  
\usepackage{url}

%% Configure citation styles
\usepackage[numbers,sort&compress,square]{natbib}
\def\bibfont{\footnotesize}     %for smaller fonts in the biblio section

%% Hyper Ref package. In order to operate correctly, it must be the last package declared
\usepackage[colorlinks,pagebackref,breaklinks]{hyperref} 

%% Extra package options
\hypersetup{
  hypertexnames=true, linkcolor=blue, anchorcolor=black,
  citecolor=blue, urlcolor=blue  
}

\urlstyle{rm} %so it doesn't use a typewriter font for urls.
\DeclareGraphicsExtensions{.jpg,.pdf,.mps,.png} % for pdflatex
\graphicspath{{img/} {./}} %put all figures in these dirs

\newcommand{\alert}[1]{{\color{red}\textbf{#1}}}

%%%%%%%%%%%%%%%%%%%%%%%%%%%%%%%%%%%%%%%%%%%%%%%%%%%%%%%%%%%%%%%%%%%%%%

%% ========================================================
%% IMPORTANT store proposal information in global variables
%% ========================================================
\title{Full Title of the Proposal ($<$200 chars)}
\shortname{ACRONYM} 
\titlelogo{}{0.25} % file name and scale
\fundingscheme{Research and Innovation Action}
\topic{Work Programme topic addressed}
\coordinator{Name of coordinator}{email}{fax}
\participant{University of Coordinator}{UoC}{Country1} % First participant is the coordinator
\participant{University of partner 2}{UoP2}{Country2} % as example...
\participant{University of partner 3}{UoP3}{Country3} % as example...
% etc.

% Page Headers
%\makeoddhead{proposal}{\disptoken{@acronym}}{}{\rightmark}
%\makeevenhead{proposal}{\leftmark}{}{\disptoken{@acronym}}

%Page Footers
%\makeevenfoot{proposal}{ \thepage }{ \date{\today} }{ \disptoken{@acronym} }
%\makeoddfoot{proposal}{  }{ \date{\today} }{ \thepage }

%Page Style
\pagestyle{proposal} %use \pagestyle{showlocs} for debugging

%Heading style 
\makeheadstyles{default}{%
\renewcommand*{\chapnamefont}{\normalfont\bfseries}
\renewcommand*{\chapnumfont}{\normalfont\bfseries}
\renewcommand*{\chaptitlefont}{\normalfont\bfseries}
\renewcommand*{\secheadstyle}{\normalfont\bfseries}
}%
\headstyles{default}

%Chapter Style
\chapterstyle{section} %Avoid writing the word "Chapter" at the beginning of each proposal section
% other possible valid styles:
% article, bringhurst, crosshead, culver, dash, demo2, ell, southall, tandh, verville, wilsondob
\renewcommand*{\chaptitlefont}{\normalfont\Large\bfseries}
\renewcommand*{\chapnumfont}{\normalfont\Large\bfseries}



\begin{document}

\instructions{\centerline{\textbf{Proposal template}}}
\instructions{\centerline{\textbf{(Technical annex)}}}
\vskip0.5cm
\instructions{\centerline{\textit{\textbf{Research and Innovation actions}}}}
\vskip0.5cm
\instructions{\centerline{\textit{\textbf{Future and Emerging Technologies:}}}}
\instructions{\centerline{\textit{\textbf{Call FETPROACT adn FETOPEN}}}}
\vskip0.5cm
\instructions{Please follow the structure of this template when preparing your proposal. It has been designed to ensure that the important aspects of your planned work are presented in a way that will enable the experts to make an effective assessment against the evaluation criteria. Sections 1, 2 and 3 each correspond to an evaluation criterion.}
\vskip0.5cm
\instructions{\textbf{Page limit:}}
\vskip0.5cm
\instructions{\textbf{The part B (cover page and sections 1, 2 and 3) is strictly limited to 16 A4 pages and shall consist of:}
\begin{itemize}
\item \textbf{A single A4 title page with acronym, title and abstract of the proposal.}
\item \textbf{Maximum 15 A4 pages consisting of an S\&T section (section 1), an Impact section (section 2) and an Implementation section (section 3).}
\end{itemize}
}
\vskip0.5cm
\instructions{All tables in these sections must be included within this limit. The minimum font size allowed is 11 points.  The page size is A4, and all margins (top, bottom, left, right) should be at least 15 mm (not including any footers or headers).}
\vskip0.5cm
\instructions{\textbf{A proposal that does not comply with these page limits will be declared ineligible.}}
\vskip0.5cm
\instructions{\textit{Important remarks:}
\begin{itemize}
\item \textit{This strict page limitation does not apply to the other additional sections that contain information related to the description of the participating organisations and to the ethics self-assessment.}
\item \textit{The list of the participants’ main scientific publications relevant to the proposal is to be included in section 4. Any other list of scientific publications relevant to the proposal must be included in sections 1-3, that is within the strict page limit.}
\end{itemize}
}

%% TITLE
\maketitle
\instructions{
Maximum length for Sections 1,2,3: 16 pages including all tables.\\ 
First stage proposals have a limit of Cover page + 15 pages.\\
Use the same participant numbering as that used in the administrative proposal forms.
}


\vspace{-4em}
\renewcommand\contentsname{\normalsize Table of Contents \vspace{-4em}}
\setlength{\cftbeforechapterskip}{1.0em plus 0.3em minus 0.1em}
\renewcommand{\cftchapterbreak}{\addpenalty{-4000}}
%\makeparticipantstable          %for the ICT RIA proposals
%\tableofcontents*               %for the ICT RIA proposals
% !TEX root = main.tex
% !TEX encoding = Windows Latin 1
% !TEX TS-program = pdflatex
% 
% Archivo: abstract.tex (en ingles)


\chapter{Abstract} % No cambiar el titulo
\selectlanguage{english}
\noindent
Duis tristique sollicitudin leo nec consequat. Praesent et dui convallis velit tincidunt fermentum. Mauris cursus purus at sem viverra sed imperdiet sapien imperdiet. Aliquam mattis, elit eget rutrum vulputate, tortor sem pulvinar justo, sit amet mollis felis sem at nibh. Donec malesuada, neque id interdum eleifend, arcu augue porta elit, nec tristique libero metus at massa. Fusce fringilla laoreet rhoncus. Suspendisse potenti. Phasellus dignissim sodales mauris at pharetra. Donec gravida fringilla velit ac rutrum.

Curabitur ornare lectus id diam molestie eu imperdiet nulla tempus. Maecenas vestibulum enim et dui ornare blandit. Vivamus fermentum faucibus viverra. Maecenas at justo sapien. Aenean rhoncus augue mattis purus rhoncus venenatis. Suspendisse metus felis, porttitor in varius in, vulputate at tortor. Aliquam molestie, turpis et malesuada porta, tortor sapien pharetra sapien, ac rhoncus quam dolor a sapien. Pellentesque varius laoreet enim ut auctor. Nullam nec ultricies nisi. Nullam porta lectus et ante consectetur posuere.

Duis tristique sollicitudin leo nec consequat. Praesent et dui convallis velit tincidunt fermentum. Mauris cursus purus at sem viverra sed imperdiet sapien imperdiet. Aliquam mattis, elit eget rutrum vulputate, tortor sem pulvinar justo, sit amet mollis felis sem at nibh. Donec malesuada, neque id interdum eleifend, arcu augue porta elit, nec tristique libero metus at massa. Fusce fringilla laoreet rhoncus. Suspendisse potenti. Phasellus dignissim sodales mauris at pharetra. Donec gravida fringilla velit ac rutrum.

Duis tristique sollicitudin leo nec consequat. Praesent et dui convallis velit tincidunt fermentum. Mauris cursus purus at sem viverra sed imperdiet sapien imperdiet. Aliquam mattis, elit eget rutrum vulputate, tortor sem pulvinar justo, sit amet mollis felis sem at nibh. Donec malesuada, neque id interdum eleifend, arcu augue porta elit, nec tristique libero metus at massa. Fusce fringilla laoreet rhoncus. Suspendisse potenti. Phasellus dignissim sodales mauris at pharetra. Donec gravida fringilla velit ac rutrum.

Curabitur ornare lectus id diam molestie eu imperdiet nulla tempus. Maecenas vestibulum enim et dui ornare blandit. Vivamus fermentum faucibus viverra. Maecenas at justo sapien. Aenean rhoncus augue mattis purus rhoncus venenatis. Suspendisse metus felis, porttitor in varius in, vulputate at tortor. Aliquam molestie, turpis et malesuada porta, tortor sapien pharetra sapien, ac rhoncus quam dolor a sapien. Pellentesque varius laoreet enim ut auctor. Nullam nec ultricies nisi. Nullam porta lectus et ante consectetur posuere.

Duis tristique sollicitudin leo nec consequat. Praesent et dui convallis velit tincidunt fermentum. Mauris cursus purus at sem viverra sed imperdiet sapien imperdiet. Aliquam mattis, elit eget rutrum vulputate, tortor sem pulvinar justo, sit amet mollis felis sem at nibh. Donec malesuada, neque id interdum eleifend, arcu augue porta elit, nec tristique libero metus at massa. Fusce fringilla laoreet rhoncus. Suspendisse potenti. Phasellus dignissim sodales mauris at pharetra. Donec gravida fringilla velit ac rutrum.

\bigskip
\noindent
\textit{Key words:} first word; second word; third word.
% Separar palabras con punto-y-comas.

\checklanguage
% Fin archivo abstract.tex
\endinput 
\thispagestyle{empty}
\pagebreak



%% Main proposal

%% Fixed proposal structure - Do not change
%%% Important. To have correct table numberings
\renewcommand{\thetable}{\thesection\alph{table}}

\chapter[Excellence]{Excellence}
\label{cha:excellence}
\instructions{
Your proposal must address a work programme topic for this call for proposals. \\
\textit{This section of your proposal will be assessed only to the extent that it is relevant to that topic.}\\
}

\section{Objectives}
\label{sec:objectives}
\instructions{
\begin{itemize}
\item Describe the specific objectives for the project\footnote{The term ‘project’ used in this template equates to an ‘action’ in certain other Horizon 2020 documentation.}, which should be clear, measurable, realistic and achievable within the duration of the project. Objectives should be consistent with the expected exploitation and impact of the project (see section 2). 
\end{itemize}
}



\section{Relation to the work programme}
\label{sec:relation-to-work-programme}
\instructions{
\begin{itemize}
\item Indicate the work programme topic to which your proposal relates, and explain how your proposal addresses the specific challenge and scope of that topic, as set out in the work programme.
\end{itemize}
}


\section{Concept and approach}
\label{sec:concept}
\instructions{
\begin{itemize}
\item Describe and explain the overall concept underpinning the project. Describe the main ideas, models or assumptions involved. Identify any trans-disciplinary considerations;\\
\item Describe the positioning of the project e.g. where it is situated in the spectrum from idea to application, or from ``lab to market''. Refer to Technology Readiness Levels where relevant. (See General Annex G of the work programme);\\
\item Describe any national or international research and innovation activities which will be linked with the project, especially where the outputs from these will feed into the project;\\
\item Describe and explain the overall approach and methodology, distinguishing, as appropriate, activities indicated in the relevant section of the work programme, e.g. for research, demonstration, piloting, first market replication, etc;\\
\item Where relevant, describe how sex and/or gender analysis is taken into account in the project's content.
\end{itemize}
\emph{Sex and gender refer to biological characteristics and social/cultural factors respectively. For guidance
on methods of sex / gender analysis and the issues to be taken into account, please refer to
http://ec.europa.eu/research/science-society/gendered-innovations/index\_en.cfm}
}

\section{Ambition}
\label{sec:ambition}
\instructions{
\begin{itemize}
\item Describe the advance your proposal would provide beyond the state-of-the-art, and the extent the proposed work is ambitious. Your answer could refer to the ground-breaking nature of the objectives, concepts involved, issues and problems to be addressed, and approaches and methods to be used.\\
\item Describe the innovation potential which the proposal represents. Where relevant, refer to products and services already available on the market. Please refer to the results of any patent search carried out.\\
\end{itemize}
}
 % Section I
\chapter{Impact}
\label{cha:impact}


\section{Expected impact} 
\label{sec:expected-impact}
\instructions{
\textit{Please be specific, and provide only information that applies to the proposal and its objectives. Wherever possible, use quantified indicators and targets.}\\
\begin{itemize}
\item Describe how your project will contribute to the expected impacts set out in the work programme under the relevant topic. 
\item Describe the importance of the technological outcome with regards to its transformational impact on science, technology and/or society.
\item Describe the empowerment of new and high-potential actors towards future technological leadership.
\end{itemize}
}





\section{Measures to maximize impact} 
\label{sec:maximize-impact}

\subsection{Dissemination and exploitation of results}
\label{sec:dissemination-exploitation}
\instructions{
\begin{itemize}
\item Provide a plan for disseminating and exploiting the project results. The plan, which should be proportionate to the scale of the project, should contain measures to be implemented both during and after the project. 
\item Explain how the proposed measures will help to achieve the expected impact of the project. 
\item Where relevant, include information on how the participants will manage the research data generated and/or collected during the project, in particular addressing the following issues:\footnote{For further guidance on research data management, please refer to the H2020 Online Manual on the Participant Portal.}
\begin{itemize}
\item What types of data will the project generate/collect?
\item What standards will be used?
\item How will this data be exploited and/or shared/made accessible for verification and re-use? If data cannot be made available, explain why.
\item How will this data be curated and preserved?
\end{itemize}
\emph{You will need an appropriate consortium agreement to manage (amongst other things) the ownership and access to key knowledge (IPR, data etc.). Where relevant, these will allow you, collectively and individually, to pursue market opportunities arising from the project's results.} \\
\emph{The appropriate structure of the consortium to support exploitation is addressed in section 3.3.}\\
\begin{itemize}
\item Outline the strategy for knowledge management and protection. Include measures to provide open access (free on-line access, such as the ``green'' or ``gold'' model) to peer-reviewed scientific publications which might result from the project.\footnote{Open access must be granted to all scientific publications resulting from Horizon 2020 actions. Further guidance on open access is available in the H2020 Online Manual on the Participant Portal.}\\
\end{itemize} 
\emph{Open access publishing (also called 'gold' open access) means that an article is immediately provided in open access mode by the scientific publisher. The associated costs are usually shifted away from readers, and instead (for example) to the university or research institute to which the researcher is affiliated, or to the funding agency supporting the research.}\\
\emph{Self-archiving (also called 'green' open access) means that the published article or the final peer-reviewed manuscript is archived by the researcher - or a representative - in an online repository before, after or alongside its publication. Access to this article is often - but not necessarily - delayed (``embargo period''), as some scientific publishers may wish to recoup their investment by selling subscriptions and charging pay-per-download/view fees during an exclusivity period.}
\end{itemize}
}

\subsection{Communication activities}
\label{sec:communication}
\instructions{
\begin{itemize}
\item Describe the proposed communication measures for promoting the project and its findings during the period of the grant. Measures should be proportionate to the scale of the project, with clear objectives.  They should be tailored to the needs of various audiences, including groups beyond the project's own community. Where relevant, include measures for public/societal engagement on issues related to the project. 
\end{itemize}
} % Section II
% !Mode:: "TeX:UTF-8"

\chapter{示例:实现方案}
本章首先介绍了本系统实现的设计原则与目标,然后描述了系统的整体架构与总体设计方案,接着阐述了系统的各个模块实现时的解决方案,最后给出了系统在~Android~具体实现时的实现方法和使用说明。
\section{设计目标与原则}
	\subsection{设计目标}
	本系统能够检测已知恶意软件及其变种,并能通过模糊检测发现具有相似恶意行为的未知恶意软件,为~Android~平台这样的开放式移动平台提供安全保障,可广泛用于各种型号的~Android~设备。系统具体设计目标如下:
	\begin{enumerate}
	\item 适用于目前主流的~Android~平板及手机,至少可运行于3.0版本系统。
	\item 能够检测用户指定的程序是否为恶意程序。
	\item 能够自动检测设备上的所有程序,并可定时检测。
	\item 能够监控设备的程序安装行为,自动检测安装的程序是否为恶意程序。
	\item 能够保证本程序自身的特征库不被破坏,并能及时修复和更新特征库。
	\item 能够保证用户使用方便。
	\end{enumerate}
	
	\subsection{设计原则}
	 从安全产品的特点出发,本系统设计与实现将遵循下列一些设计原则:
	\begin{enumerate}
	\item 高效性
	
	 系统运行效率高,可快速实现对目标程序的特征提取与检测,并将结果用最清晰简洁的方式告知用户。
	\item 灵活性
	
	 为用户提供的各种功能具有可选性,用户可根据自己的需要选择其中的功能,而且用户可自行决定如何处理检测结果为恶意的程序。
	
	\item 实用性
	
	本系统应按照实用性原则进行设计,在保证对程序检测的同时,力求用户界面简洁友好。
	
	\item 可扩展性
	
	 目前本系统只检测程序的~Java~实现部分,但是还有极少数程序代码是用~C~语言编写的,在后续的开发过程中,可以在不改变程序结构的前提下,实现这一部分的检测功能。
	
	\item 健壮性 \par
	 系统应具有应对非法操作的能力,并且当针对于本系统的恶意攻击到来时,可以及时防御,防止自身特征库遭到损坏。
	\end{enumerate}
	
\section{系统方案}
	 本系统由两部分构成,第一部分是产品部分,即~Android~应用程序,采用~Java~作为编程语言。第二部分是检测算法模型构建部分,采用~Matlab~实现,其输出的模型数据供产品部分作为特征库使用。故本系统的特征库构建于~PC~端,而对目标程序的检测运行于~Android~端。从而将构建过程中包含的巨大计算量留在~PC~端。其具体结构如图~\ref{system}~所示。
\pic[htbp]{系统结构}{height=5.86cm}{system}

	 其中程序信息抽取模块是本系统检测恶意软件的基础,特征检测模块是系统的核心与实现难点。特征检测模块中又分为变种检测模块和模糊检测模块,变种检测模块检测待检软件是否是已知恶意软件的变种,而模糊检测模块实现的是本系统从人脸匹配中引入的新型检测手段,可对特征库中不存在的未知恶意软件进行检测。
	
\section{系统模块的实现}
	\subsection{程序分解模块}
		
		 Android~的程序文件为~APK~格式,APK~文件是~Android~最终的运行程序,是~Android~Package~的全称。类似于~Symbian~ 操作系统中~sis~文件,APK~文件其实是~Zip~文件格式,但后缀名被修改为~APK。通过解压,可以看到~Dex~ 文件。Dex~是~Dalvik~VM~executes~的全称,即~Dalvik~虚拟机可执行文件,并非~Java~ME~的字节码而是~Dalvik~字节码。\\
		一个APK文件结构为:
		\begin{enumerate}
		\item META-INF$\backslash$————签名信息,用来保证~apk~包的完整性和系统的安全,jar~文件经常可以看到;
		\item res$\backslash$————资源文件夹,包括程序中使用的图片,布局文件等;
		\item AndroidManifest.xml$\backslash$————项目配置清单,但不是明文的XML格式,无法直接打开阅读;
		\item classes.dex$\backslash$————Dalvik~可执行二进制文件,在运行时被动态优化为dey文件并由Dalvik虚拟机解释执行	 ;
		\item resources.arsc$\backslash$————编译后的二进制资源文件,资源文件打包而成,字符串值(源码中的/value/Strings.xml)就在其中;
		\item lib$\backslash$————动态链接库文件;
		\item assets$\backslash$————原始文件文件夹,其中的文件不会被压缩,也不能像~res~目录下的资源文件一样通过资源类引用。
		\end{enumerate}\par
		 图~\ref{apk}~是我们解压缩helloworld.apk文件后看到的内容,可以看到其结构跟工程结构有些类似。
		
\pic[htbp]{helloworld.apk的结构}{width=0.5\textwidth}{apk}

		
		 classes.dex~文件是~Java~源码编译后生成的~Java~字节码文件。但由于~Android~使用的~Dalvik~虚拟机与标准的~Java~虚拟机是不兼容的,Dex~文件与~Class~文件相比,不论是文件结构还是~opcode~都不一样。目前常见的~Java~ 反编译工具都不能处理~Dex~文件。Android~SDK~中提供了一个~Dex~文件的反编译工具~dexdump。用法为,dexdump -d -f -h  xxx.dex。
		\\
		指令参数解释:\\
		-d : 反编译程序段\\
		-f : 从文件头显示摘要信息\\
		-h : 显示文件头详情\\
		-C : 反编译低级符号名\\
		-S : 只计算大小 \par
		 在知道了程序安装包是~Zip~编码之后,我们就可以通过遍历~Zip~包中包含的项目名,找到程序的二进制文件,即~Dex~文件。并在内存中建立一段缓冲区,可将~Dex~文件读入内存,再写到指定的临时文件中。分解流程如图~\ref{flow1}~所示。
	
\pic[htbp]{程序分解流程}{height = 5.76cm}{flow1}	
		
	\subsection{构建特征向量模块}
		 特征向量是数据挖掘中的一个概念,一个数据集中的每个数据实例都可以用一组属性值来描述,每一个数据实例都具有一个特殊的目标属性,称为类属性,它表征每个数据实例归属的类。这一组属性值即是代表这个数据实例的特征向量。在我们的检测问题中,Android~系统~API~和~Java~标准函数就是我们定义的属性,而恶意和非恶意就是类属性。这一过程如图~\ref{flow3}~所示。
		
\pic[htbp]{构建特征向量流程}{height = 0.5\textwidth}{flow3}	
		
		 我们将特征向量空间($\Omega$)存储在数据库中,数据表的定义见表~\ref{omegatable}~所示。

\threelinetable[htbp]{omegatable}{0.8\textwidth}{llllX}{$\Omega$~数据表定义}
{字段&主键&类型&是否为空&备注\\
}{
ID&是&Int&NOT NULL&特征向量维度序号\\
MethodName&否&Text&NOT NULL&函数名\\
}{\item
}

		 构造特征向量时,先初始化一个全为0的特征向量,然后利用SQL查询语句确定代表某一函数名的维度序号:\\
		select ID from Omega where MethodName = “待查函数名”;\\
		 并将特征向量($\omega$)的第~ID~位设置为1。见式(\ref{omegai})。
		\begin{equation}\label{omegai}
		\omega_i =
		\begin{cases}
		1 & i = ID \\
		0 & else
		\end{cases}
		\end{equation}
	\subsection{特征库构建模块}
		 特征库构建模块实现于~PC~端,恶意样本来自各大权威机构公布的数据,详见第\pageref{omegai}页\ref{omegai}小节,其流程如图~\ref{flow5}~所示,特征库数据模型构建方法如下:
		
\begin{pics}[htbp]{特征库构建流程}{flow5}
  \addsubpic{KNN特征库构建流程}{width=0.3\textwidth}{flow5-1}
  \addsubpic{K-L~变换矩阵构建流程}{width=0.3\textwidth}{flow5-2}
  \addsubpic{LDA~投影矩阵构建流程}{width=0.3\textwidth}{flow5-3}
\end{pics}

		\begin{enumerate}
		\item 最近邻居(KNN)算法\par
		 KNN~算法的特征库就是恶意程序样本的特征向量集合,我们将这些特征向量存储到数据库中。
		
		\item 主成分分析(PCA)算法\par
		 考虑到文献\citeup{wangang1912}\citeup{zhaokaihua1995}中的适用条件,由于我们的特征向量维度远大于样本数量,所以需要去掉冗余数据,使训练数据矩阵为可逆矩阵才能使用线性判别分析(LDA)算法。

		 PCA~方法主要是通过对协方差矩阵进行本征分解,以得出数据的主成分(即本征矢量)与它们的权值(即本征值)。PCA~ 提供了一种降低数据维度的有效办法;如果分析者在原数据中除掉最小的本征值所对应的成分,那么所得的低维度数据必定是最优化的(也即这样降低维度必定是失去信息最少的方法)。

		 我们的目标是把高维的数据集~$\Omega_B$~和~$\Omega_M$~变换成具有较小维度的数据集~$Y_B$~和~$Y_M$。$Y_B$~和~$Y_M$~是矩阵~$\Omega_B$~和~$\Omega_M$~的~Karhunen–Loève~变换(K-L~变换)。即~$\mathbf{Y}=\mathbb{KLT}\{\mathbf{X}\}$。\\
		计算特征向量平均值见式(\ref{mean})。
		\begin{equation}\label{mean}
		u=\dfrac{1}{N} \sum_{\omega \in \Omega_B \cup \Omega_M} \omega
		\end{equation}
		 从~$\Omega_B$~和~$\Omega_M$~中减去平均值~$u$~见式(\ref{Omega_u})。
		\begin{equation}\label{Omega_u}
		\begin{split}
		B & =\begin{vmatrix}\Omega_B \\ \Omega_M \end{vmatrix}-hu\\
		  & \text{其中h是全为1的列向量。}
		\end{split}
		\end{equation}
		求协方差矩阵~C~见式(\ref{getC})。
		\begin{equation}\label{getC}
		C=B \cdot B^T
		\end{equation}
		 计算~C~的特征值和特征向量,提取不为0的特征值所对应的特征向量,构成~K-L~变换矩阵~W。\\
		 所以,$Y_B$~和~$Y_M$~可由式(\ref{ybym})计算。
		\begin{equation}\label{ybym}
		\begin{split}
		Y_B &= \Omega_B \cdot W \\
		Y_M &= \Omega_M \cdot W
		\end{split}
		\end{equation}
		
		最后我们将~K-L~变换矩阵~W~存储在数据库中。
				
		\item Fisher~线性判别分析(LDA)\par
\threelinetable[htbp]{ldaparameterdeftable}{\textwidth}{lXlX}{LDA算法变量定义}
{变量&定义&变量&定义\\
}{
        $S_b$ 	& 样本类间离散度矩阵 & $x$		& 一个程序\\
		$S_i$ 	& 样本类内离散度矩阵&$X_M$	& 恶意程序集合\\
		$S_w$ 	& 总类内离散度矩阵&$X_B$	& 非恶意程序集合\\
		$W$   	& 投影方向向量&$y_M$ 	& 恶意样本的投影值 \\
		$J_F(W)$& Fisher~准则函数&$y_B$ 	& 非恶意样本的投影值\\
		$M$		& 恶意(Malice)的缩写&$y_0$ 	& 识别阈值点\\
		$B$		& 非恶意(Benign)的缩写&&\\
}{
\item
}

应用统计方法解决模式识别问题时,一再碰到的问题之一是维数问题。在低维空间里解析上或计算上行得通的方法,在高维空间里往往行不通。因此,降低维数有时就成为处理实际问题的关键。在数学上总是可以把高维空间样本投影到一条直线上,形成一维空间,即把维数压缩到一维。但是投影方向有无数种,若把样本投影到一条任意的直线上,可能使几类样本混在一起无法区分,如图~\ref{fisher1}~所示。。但在一般情况下,总可以找到某个方向,使在这个方向的直线上,样本的投影能分开得最好,如图~\ref{fisher2}~所示。。 问题是如何根据实际情况找到这条最好的、最易于区分的投影线。这就是~Fisher~法所要解决的基本问题。\par

\begin{pics}[htbp]{Fisher~线性判别基本原理}{fisher}
  \addsubpic{最优方向投影}{width=0.4\textwidth}{fisher1}
  \addsubpic{K-L~任意方向投影}{width=0.4\textwidth}{fisher2}
\end{pics}

		 描述~LDA~算法前,首先定义几个基本变量,变量定义见表~\ref{ldaparameterdeftable}。LDA~算法步骤如下:

		\begin{enumerate}
		\item 计算样本均值向量~$m_i$:
		\begin{equation*}
		m_i=\dfrac{1}{N_i}\sum_{y \in Y_i}y ~~~~,i=B,M
		\end{equation*}
		\item 计算样本类内离散度矩阵~$S_i$~和总类内离散度矩阵~$S_w$:
		\begin{align}
		S_i & = \sum_{y \in Y_i}(y-m_i)(x-m_i)^T ~~~~,i=B,M\\
		S_w & = P(x|x \in X_B)S_B + P(x|x\in X_M)S_M
		\end{align}
		\item 计算样本类间离散度矩阵~$S_b$:
		\begin{equation*}
		S_b=P(x|x \in X_B)P(x|x\in X_M)(m_B-m_M)(m_B-m_M)^T
		\end{equation*}
		$P(x|x \in X_B)$~和~$P(x|x\in X_M)$~是恶意程序和非恶意程序的先验概率,根据目前~Android~市场的情况,我们取~$P(x|x\in X_M)=0.001$。
		\item Fisher准则函数为:
		\begin{equation}
		J_F(W) = \dfrac{W^T S_b W}{W^T S_w W}
		\end{equation}
		 为求函数取极大值时的~$W^*$。可用拉格朗日乘数法,定义拉格朗日函数为:
		\begin{equation*}
		\dfrac{L(W,\lambda)}{W} = S_bW-\lambda S_w W
		\end{equation*}
		另偏导数为零,得
		\begin{equation*}
		S_b W^* = \lambda S_w W^*
		\end{equation*}
		 其中~$W^*$~就是~$J_F(W)$~的极值解。因为~$S_w$~可逆,等式两边左乘~$S_w^{-1}$,可得
		\begin{equation*}
		S_w^{-1} S_b W^* = \lambda W^*
		\end{equation*}
		所以求~$W^*$~即求矩阵~$S_w^{-1} S_b$~的特征值问题。在我们这个特殊情况下,只有两种类别,故
		\begin{equation*}
		S_b W^* = (m_B-m_M)(m_B-m_M)^T W^*
		\end{equation*}
		 其值为一标量,所以对~$W$~投影方向无影响。忽略这个标量的比例因子可得,
		\begin{equation*}
		W^* = S_w^{-1} (m_B-m_M)
		\end{equation*}
		\item 求出~$W^*$~后即可计算:
		\begin{align}
		y_M & = mean(W^*T \cdot Y_M) \\
		y_B & = mean(W^*T \cdot Y_B) \\
		y_0 & = \dfrac{m_B+m_M}{2}  +  \dfrac{\ln (P(x|x\in X_B)/P(x|x\in X_M))}{N_B+N_M-2}
		\end{align}
		 最后将LDA变换矩阵~$W$、$y_M$、$y_B$、$y_0$保存在数据库中。
		\end{enumerate}	
		\end{enumerate}
\section{本章小结}
本章介绍了~AFace~系统的实现细节,它是一个由~Android~端和~PC~端两部分组成的系统。我们首先介绍了系统的设计原则。然后介绍了我们在此原则下设计的检测流程及其背后的检测原理。最后,作为一款完整的产品,还介绍了系统的产品软件架构以及产品的界面设计。
 % Section III

\bibliographystyle{plain}       %References go to end of Section III
\bibliography{refs}

\clearpage

\chapter{Members of the consortium}
\label{cha:members}

\instructions{
\textit{This section is not covered by the page limit.}
\vskip0.2cm
\textit{The information provided here will be used to judge the operational capacity.}
}

\section{Participants (applicants)}
\label{sec:participants}

\instructions{
Please provide, for each participant, the following (if available):\\
\begin{itemize}
\item a description of the legal entity and its main tasks, with an explanation of how its profile matches the tasks in the proposal;
\item a curriculum vitae or description of the profile of the persons, including their gender, who will be primarily responsible for carrying out the proposed research and/or innovation activities;
\item a list of up to 5 relevant publications, and/or products, services (including widely-used datasets or software), or other achievements relevant to the call content;
\item a list of up to 5 relevant previous projects or activities, connected to the subject of this proposal;
\item a description of any significant infrastructure and/or any major items of technical equipment, relevant to the proposed work;
\item any other supporting documents specified in the work programme for this call.
\end{itemize}
}

\section{Third parties involved in the project (including use of third party resources)}
\label{sec:third-parties}

\instructions{
\textit{Please complete, for each participant, the following table (or simply state "No third parties involved", if applicable).} \\
If yes in first row, please describe and justify the tasks to be subcontracted. If yes in second row, please describe the third party, the link of the participant to the third party, and describe and justify the foreseen tasks to be performed by the third party\footnote{A third party that is an affiliated entity or has a legal link to a participant implying a collaboration not limited to the action. (Article 14 of the Model Grant Agreement).}. If yes in third row, please describe the third party and their contributions.}

\begin{tabular}{|p{.85\textwidth}|p{.05\textwidth}|}
  \hline  
  \multicolumn{2}{|l|}{\cellcolor[gray]{0.8}\textbf{UoC}}\\
  \hline
  Does the participant plan to subcontract certain tasks (please note that core tasks of the project should not be sub-contracted) &
  \textbf{Y/N} \\
  \hline
Does the participant envisage that part of its work is performed by linked
third parties &
  \textbf{Y/N} \\
  \hline
  Does the participant envisage the use of contributions in kind provided by
third parties (Articles 11 and 12 of the General Model Grant Agreement) &
  \textbf{Y/N}\\
  \hline
\end{tabular}

\begin{tabular}{|p{.85\textwidth}|p{.05\textwidth}|}
  \hline  
  \multicolumn{2}{|l|}{\cellcolor[gray]{0.8}\textbf{UoP1}}\\
  \hline
  Does the participant plan to subcontract certain tasks (please note that core tasks of the project should not be sub-contracted) &
  \textbf{Y/N} \\
  \hline
Does the participant envisage that part of its work is performed by linked
third parties &
  \textbf{Y/N} \\
  \hline
  Does the participant envisage the use of contributions in kind provided by
third parties (Articles 11 and 12 of the General Model Grant Agreement) &
  \textbf{Y/N}\\
  \hline
\end{tabular}


\begin{tabular}{|p{.85\textwidth}|p{.05\textwidth}|}
  \hline  
  \multicolumn{2}{|l|}{\cellcolor[gray]{0.8}\textbf{UoP2}}\\
  \hline
  Does the participant plan to subcontract certain tasks (please note that core tasks of the project should not be sub-contracted) &
  \textbf{Y/N} \\
  \hline
Does the participant envisage that part of its work is performed by linked
third parties &
  \textbf{Y/N} \\
  \hline
  Does the participant envisage the use of contributions in kind provided by
third parties (Articles 11 and 12 of the General Model Grant Agreement) &
  \textbf{Y/N}\\
  \hline
\end{tabular}
 % Section IV
\chapter{Ethics and Security}
\label{cha:ethics}
\instructions{
\textit{This section is not covered by the page limit.}
}

\section{Ethics}
\label{sec:ethics}
\instructions{
If you have entered any ethics issues in the ethical issue table in the administrative proposal forms, you must:
\begin{itemize}
\item submit an ethics self-assessment, which:
\begin{itemize}
\item describes how the proposal meets the national legal and ethical requirements of the country or countries where the tasks raising ethical issues are to be carried out; 
\item explains in detail how you intend to address the issues in the ethical issues table, in particular as regard:
\begin{itemize}
\item research objectives (e.g. study of vulnerable populations, dual use, etc.)
\item research methodology (e.g. clinical trials, involvement of children and related consent procedures, protection of any data collected, etc.) 
\item the potential impact of the research (e.g. dual use issues, environmental damage, stigmatisation of particular social groups, political or financial retaliation, benefit-sharing,  malevolent use, etc.).
\end{itemize}
\end{itemize}
\item provide the documents that you need under national law(if you already have them), e.g.:
\begin{itemize}
\item an ethics committee opinion;
\item the document notifying activities raising ethical issues or authorising such activities;
\end{itemize}
\end{itemize}
\textit{\indent If these documents are not in English, you must also submit an English summary of them (containing, if available, the conclusions of the committee or authority concerned).}
\vskip0.2cm
\textit{If you plan to request these documents specifically for the project you are proposing, your request must contain an explicit reference to the project title.}
}

\section{Security}\footnote{Article 37.1 of the Model Grant Agreement: Before disclosing results of activities raising security issues to a third party (including affiliated entities), a beneficiary must inform the coordinator -- which must request written approval from the Commission/Agency. Article 37.2: Activities related to ``classified deliverables'' must comply with the ``security requirements'' until they are declassified. Action tasks related to classified deliverables may not be subcontracted without prior explicit written approval from the Commission/Agency. The beneficiaries must inform the coordinator -- which must immediately inform the Commission/Agency -- of
any changes in the security context and --if necessary -- request for Annex 1 to be amended (see Article 55).
}
\label{sec:security}
\instructions{
Please indicate if your project will involve:
\begin{itemize}
\item activities or results raising security issues: (YES/NO)
\item ``EU-classified information'' as background or results: (YES/NO)
\end{itemize}
}
 % Section V

\appendix

%% Proposal appendix

 % Appendix

\backmatter

\end{document}
