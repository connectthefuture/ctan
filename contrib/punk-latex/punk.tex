\documentclass{article}
\usepackage[T1]{fontenc}
\usepackage[utf8]{inputenc}
\usepackage{listings}
\title{The \texttt{punk} package\medskip\\
  \large\LaTeX\ support for the punk fonts}
\author{version 1.1 \today}
\lstset{language=[latex]tex,breaklines=true}
\date{Palle J\o rgensen}

\begin{document}
\maketitle
\section{Introduction}
\label{sec:introduction}
The \texttt{punk} package provides support for the punk fonts. The
punk fonts are already installed on many systems, this is only support
for using the punk fonts with \LaTeX.

The license of the punk package and the related files is GNU General
Public License.

\section{Using the punk package}
\label{sec:using-punk-fonts}

If you want some text typeset with the punk fonts for a short
text you can use one of the commands

\begin{lstlisting}
  \textpunk{...}, \textpunksl{...}, \textpunkbf{...}
\end{lstlisting}
which typesets the text with Punk, Punk Slanted and Punk Bold.

If you want to typeset longer passages of text with the punk fonts,
you can use the environment

\begin{lstlisting}
  punkfamily
\end{lstlisting}
Inside \texttt{punkfamily} the normal \LaTeX\ font switches
\verb+\slshape+ and \verb+\bfseries+ works. Furthermore \verb+\emph+
works too.

It is possible to use the command
\begin{lstlisting}
  \punkfamily
\end{lstlisting}
but this command also changes the current fontencoding; use with
caution\dots

\clearpage
\appendix
\enlargethispage*{5mm}
\section{Source of the files in the punk bundle}
\label{sec:source}

\subsection{punk.sty}
\label{sec:punk.sty}
\lstinputlisting{punk.sty}

\subsection{ot1pnk.fd}
\label{sec:ot1pnr.fd}
\lstinputlisting{ot1pnr.fd}

\end{document}

%%% Local Variables: 
%%% mode: latex
%%% TeX-master: t
%%% End: 

%%% Local Variables: 
%%% mode: latex
%%% TeX-master: t
%%% End: 
