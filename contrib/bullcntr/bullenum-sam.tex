%
% Copyright (C) 2007 by Gustavo MEZZETTI <gustavo.mezzetti@istruzione.it>.
%
% This file is part of a work named "bullcntr package".
%
% The bullcntr package may be distributed and/or modified under the
% conditions of the LaTeX Project Public License, either version 1.3 of
% this license or (at your option) any later version.
% The latest version of this license is in
%   http://www.latex-project.org/lppl.txt
% and version 1.3 or later is part of all distributions of LaTeX
% version 2005/12/01 or later.
%
% The bullcntr package has the LPPL maintenance status
%   "author-maintained".
%
% The file `manifest.txt' that comes along with this file specifies what
% the bullcntr package consists of; more precisely, it explains how the
% locutions "Work" and "Compiled Work", used in the LaTeX Project Public
% License, are to be interpreted in the case of this work.
%
% Note that the distribution of this file alone is "distribution of only
% part of the Work" in the sense of the LaTeX Project Public License,
% and should be avoided (see the LaTeX Project Public Licence for
% details).
%
\documentclass[a4paper]{article}
% Non-metric paper sizes have no right to exist! (;-)
\usepackage{bullenum}
% Of course, there is no need to load the hyperref package, but we
% like to use it:
\usepackage[bookmarksnumbered]{hyperref}

\title{Using \tbenu:\\an example}
\author{G.~Mezzetti}
\date{October~10, 2008}

% The following command definitions has nothing to do with the
% bullcntr package: you may ignore them.

\DeclareRobustCommand*{\command}[1]{%
	{\texorpdfstring{\normalfont\ttfamily \char\escapechar}{\pdfbslash}#1}%
}
\DeclareRobustCommand*{\packlass}[1]{%
	{\texorpdfstring{\normalfont \sffamily}{}#1}%
}
\DeclareRobustCommand*{\meta}[1]{%
	\texorpdfstring
		{\mbox{\normalfont \(\langle\textit{#1}\rangle\)}}%
		{<#1>}%
}
\DeclareRobustCommand*{\filedir}[1]{{\normalfont\ttfamily #1}}
\DeclareRobustCommand*{\cnt}[1]{\texttt{#1}}
\DeclareRobustCommand*{\env}[1]{\texttt{#1}}
\newcommand*{\Bullcntr}{bullcntr}
\newcommand*{\bull}{\packlass{\Bullcntr}}
\newcommand*{\tbull}{the \bull\ package}
\newcommand*{\Bullenum}{bullenum}
\newcommand*{\benu}{\packlass{\Bullenum}}
\newcommand*{\tbenu}{the \benu\ package}
\newcommand*{\Enumerat}{enumerate}
\newcommand*{\beenv}{\env{\Bullenum}}
\newcommand*{\enenv}{\env{\Enumerat}}

\newcommand*{\pdfbslash}{}
{\catcode`\|=0 |catcode`|\=12 |gdef|pdfbslash{\\}}

% End of command definitions.

% The following is just a neurotic symptom of the author:
\hfuzz = 0pt
\vfuzz = 0pt



\begin{document}

\maketitle

\begin{abstract}
	This document illustrates, by means of practical examples, the use
	of \tbenu.  You should read the \LaTeX\ source of this document
	and compare it with the output it produces; you may also find it
	useful as a starting base for your own coding.
\end{abstract}

\tableofcontents



\setcounter{secnumdepth}{0}

\section{Introduction}

If you are reading this document, you will probably know what \tbenu\
is and what purpose it fulfills.  In the sequel, we shall assume no
other knowledge about \tbenu\ besides this; in particular, we assume
that you did not bother reading the documentation, and prefer to learn
how to use this package by looking at some sample code.  We also take
for granted that you are reading the \LaTeX\ source of this document,
and comparing it with the output it produces.

You might find it useful to read also~\cite{bull-sam}, which contains
sample code that theaches you how to use \tbull.



\setcounter{secnumdepth}{3}

\section{Basic usage}

Here we illustrate the use of the \beenv\ environment to produce lists
whose items are numbered with the \verb|\bullcntr| command.  Different
``bullet styles'' are obtained by means of some declarations that
affects the size (or even the shape) of the bullets employed.



\subsection{The standard sizes (\command{smartctrbull})}
\label{SS:Xmp-Standard}

In our first example, actually, we do not alter the default bullet
style, that is, the output \tbenu\ will produce by itself, if you
invoke it without options and do not declare any style by your own; in
any case, it corresponds to the style selected by a
\verb|\smartctrbull| declaration.

\begin{bullenum}
	\item
		This is the first item of the list, that is, item~\#1.  It is
		therefore numbered with a single bullet.  Poor bullet!  How
		lonely it must feel!

	\item
		This is the second item of the list, that is, item~\#2.  It is
		therefore numbered with two bullets, one on the left and one
		on the right.  If you are a romantic nature, you might even
		think of them as Romeo and Juliet\ldots

	\item\label{Standard-3}
		This is the third item of the list, that is---guess
		what?---item~\#3.  Not surprisingly, it is numbered with three
		bullets neatly arranged in the shape of an equilateral
		triangle.

	\item
		This is the fourth item, item~\#4.  A square has exactly four
		vertices, so it seems quite appropriate to lay out the four
		bullets that must be used to number this item exactly in this
		form, as you can see in the label.

	\item
		If you have not lost the track, you will agree that this is
		the fifth item of the list.  A regular pentagon would probably
		have been the optimal choice for displaying this value of the
		counter, but for some reason the arrangement shown in the
		label of this item has been preferred by the author of \tbull.

	\item\label{Standard-6}
		Sixth item!  Item~\#6!  A regular hexagon is used here!  Note,
		however, that the bullets have shrunk to a smaller size,
		because the size used so far would have made the bullet
		cluster look a bit too crowded.

	\item
		In the opinion of the author, the shape used to display the
		value~7 is even more pleasant than the regular hexagon used
		for the value~6; indeed, this is a regular hexagon too, but
		with the center marked by an additional bullet, which makes
		the figure look more symmetric.

	\item
		If 7~was the perfect number of bullets, here we face the
		disruption of this perfection.  The solution adopted for
		arranging 8~bullets, which you can see here, is, in the
		author's opinion, the less inelegant among all possibilities.

	\item
		Item~\#9, the last value that the \verb|\bullcntr| command can
		display.  The bullets are just stacked in a $3\times3$ square
		pattern---the obvious choice!
\end{bullenum}



\subsection{Moderately large bullets (\command{largectrbull})}

Our next example illustrates the output produced by the style selected
with a \verb|\largectrbull| declaration; this style continues to use,
for the values over~5, the same bullet size adopted for the first five
values.  So, we say

\begin{bullenum}
	\largectrbull

	\item
		This is the first item of the list, that is, item~\#1.  It is
		therefore numbered with a single bullet.  Poor bullet!  How
		lonely it must feel!

	\item
		This is the second item of the list, that is, item~\#2.  It is
		therefore numbered with two bullets, one on the left and one
		on the right.  If you are a romantic nature, you might even
		think of them as Romeo and Juliet\ldots

	\item\label{Large-3}
		This is the third item of the list, that is---guess
		what?---item~\#3.  Not surprisingly, it is numbered with three
		bullets neatly arranged in the shape of an equilateral
		triangle.

	\item
		This is the fourth item, item~\#4.  A square has exactly four
		vertices, so it seems quite appropriate to lay out the four
		bullets that must be used to number this item exactly in this
		form, as you can see in the label.

	\item
		If you have not lost the track, you will agree that this is
		the fifth item of the list.  A regular pentagon would probably
		have been the optimal choice for displaying this value of the
		counter, but for some reason the arrangement shown in the
		label of this item has been preferred by the author of \tbull.

	\item\label{Large-6}
		Sixth item!  Item~\#6!  A regular hexagon is used here!  Note,
		once again, that the bullets have \emph{not} shrunk to a
		smaller size, so that the resulting picture looks pretty
		crowded, although not intolerably bad-looking.  Compare the
		label of this item with a reference to the corresponding
		item~\ref{Standard-6} under~\ref{SS:Xmp-Standard}.

	\item
		In the opinion of the author, the shape used to display the
		value~7 is even more pleasant than the regular hexagon used
		for the value~6; indeed, this is a regular hexagon too, but
		with the center marked by an additional bullet, which makes
		the figure look more symmetric.

	\item
		If 7~was the perfect number of bullets, here we face the
		disruption of this perfection.  The solution adopted for
		arranging 8~bullets, which you can see here, is, in the
		author's opinion, the less inelegant among all possibilities.

	\item
		Item~\#9, the last value that the \verb|\bullcntr| command can
		display.  The bullets are just stacked in a $3\times3$ square
		pattern---the obvious choice!
\end{bullenum}



\subsection{Small bullets (\command{smallctrbull})}

The reciprocal example is to use small bullets throughout, from 1 to~9.
Thus, we use the following source code:

\begin{bullenum}
	\smallctrbull

	\item
		This is the first item of the list, that is, item~\#1.  It is
		therefore numbered with a single bullet.  Poor bullet!  How
		lonely it must feel!

	\item
		This is the second item of the list, that is, item~\#2.  It is
		therefore numbered with two bullets, one on the left and one
		on the right.  If you are a romantic nature, you might even
		think of them as Romeo and Juliet\ldots

	\item\label{Small-3}
		This is the third item of the list, that is---guess
		what?---item~\#3.  Not surprisingly, it is numbered with three
		bullets neatly arranged in the shape of an equilateral
		triangle.

	\item
		This is the fourth item, item~\#4.  A square has exactly four
		vertices, so it seems quite appropriate to lay out the four
		bullets that must be used to number this item exactly in this
		form, as you can see in the label.

	\item
		If you have not lost the track, you will agree that this is
		the fifth item of the list.  A regular pentagon would probably
		have been the optimal choice for displaying this value of the
		counter, but for some reason the arrangement shown in the
		label of this item has been preferred by the author of \tbull.

	\item\label{Small-6}
		Sixth item!  Item~\#6!  A regular hexagon is used here!  Note,
		once more, that the bullets have \emph{not} shrunk to a
		smaller size, because they were \emph{already} small in the
		items shown thus far.

	\item
		In the opinion of the author, the shape used to display the
		value~7 is even more pleasant than the regular hexagon used
		for the value~6; indeed, this is a regular hexagon too, but
		with the center marked by an additional bullet, which makes
		the figure look more symmetric.

	\item
		If 7~was the perfect number of bullets, here we face the
		disruption of this perfection.  The solution adopted for
		arranging 8~bullets, which you can see here, is, in the
		author's opinion, the less inelegant among all possibilities.

	\item
		Item~\#9, the last value that the \verb|\bullcntr| command can
		display.  The bullets are just stacked in a $3\times3$ square
		pattern---the obvious choice!
\end{bullenum}

Again, compare the references to items \#3 and~\#6 from different
examples: standard bullets (\ref{Standard-3} and~\ref{Standard-6}),
moderately large bullets (\ref{Large-3} and~\ref{Large-6}), small
bullets (\ref{Small-3} and~\ref{Small-6}).



\subsection{Tiny bullets}

We come now to a sort of extreme example: the use of
\verb|\textperiodcentered|.  Since there is no predefined declaration
for this style, it is necessary to manually redefine the two
``hooks'', \verb|\counterlargebullet| and \verb|\countersmallbullet|,
that \tbull\ (and therefore, indirectly, also \tbenu) call when they
want to typeset a bullet.

\begin{bullenum}
	\renewcommand*{\counterlargebullet}{\textperiodcentered}
	\renewcommand*{\countersmallbullet}{\textperiodcentered}

	\item
		This is the first item of the list, that is, item~\#1.  It is
		therefore numbered with a single bullet.  Poor bullet!  How
		lonely it must feel!

	\item
		This is the second item of the list, that is, item~\#2.  It is
		therefore numbered with two bullets, one on the left and one
		on the right.  If you are a romantic nature, you might even
		think of them as Romeo and Juliet\ldots

	\item
		This is the third item of the list, that is---guess
		what?---item~\#3.  Not surprisingly, it is numbered with three
		bullets neatly arranged in the shape of an equilateral
		triangle.

	\item
		This is the fourth item, item~\#4.  A square has exactly four
		vertices, so it seems quite appropriate to lay out the four
		bullets that must be used to number this item exactly in this
		form, as you can see in the label.

	\item
		If you have not lost the track, you will agree that this is
		the fifth item of the list.  A regular pentagon would probably
		have been the optimal choice for displaying this value of the
		counter, but for some reason the arrangement shown in the
		label of this item has been preferred by the author of \tbull.

	\item
		Sixth item!  Item~\#6!  A regular hexagon is used here!  Note
		that it was really impossible, this time, for the bullets to
		shrunk to a still smaller size than the one that has been
		adopted in the above items!

	\item
		In the opinion of the author, the shape used to display the
		value~7 is even more pleasant than the regular hexagon used
		for the value~6; indeed, this is a regular hexagon too, but
		with the center marked by an additional bullet, which makes
		the figure look more symmetric.

	\item
		If 7~was the perfect number of bullets, here we face the
		disruption of this perfection.  The solution adopted for
		arranging 8~bullets, which you can see here, is, in the
		author's opinion, the less inelegant among all possibilities.

	\item
		Item~\#9, the last value that the \verb|\bullcntr| command can
		display.  The bullets are just stacked in a $3\times3$ square
		pattern---the obvious choice!
\end{bullenum}

This bullet style might be employed in a \beenv\ environment nested
inside another one.



\subsection{St.\texorpdfstring{\,}{ }Valentine (\command{heartctrbull})}
\label{SS:Xmp-Hearts}

To end this inventory of variations, let us illustrate the use of a
fancier ``bullet'': a little heart.  We obtain this with a
\verb|\heartctrbull| declaration.

\begin{bullenum}
	\heartctrbull

	\item\label{Hearts-1}
		This is the first item of the list, that is, item~\#1.  It is
		therefore numbered with a single heart.  Poor heart!  It is a
		true lonely heart!

	\item
		This is the second item of the list, that is, item~\#2.  It is
		therefore numbered with two hearts, one on the left and one on
		the right.  Thinking of Romeo and Juliet is absolutely natural
		in this context\ldots

	\item\label{Hearts-3}
		This is the third item of the list, that is---guess
		what?---item~\#3.  Not surprisingly, it is numbered with three
		hearts neatly arranged in the shape of an equilateral
		triangle.

	\item\label{Hearts-4}
		This is the fourth item, item~\#4.  A square has exactly four
		vertices, so it seems quite appropriate to lay out the four
		hearts that must be used to number this item exactly in this
		form, as you can see in the label.

	\item\label{Hearts-5}
		If you have not lost the track, you will agree that this is
		the fifth item of the list.  A regular pentagon would probably
		have been the optimal choice for displaying this value of the
		counter, but for some reason the arrangement shown in the
		label of this item has been preferred by the author of \tbull.

	\item
		Sixth item!  Item~\#6!  A regular hexagon is used here,
		adorned with hearts!  Of course, the hearts do not shrink, as
		doesn't yours, and as doesn't your ardent passion!

	\item\label{Hearts-7}
		In the opinion of the author, the shape used to display the
		value~7 is even more pleasant than the regular hexagon used
		for the value~6; indeed, this is a regular hexagon too, but
		with the center marked by an additional heart, which makes the
		figure look more symmetric.

	\item\label{Hearts-8}
		If 7~was the perfect number of hearts, here we face the
		disruption of this perfection.  The solution adopted for
		arranging 8~hearts, which you can see here, is, in the
		author's opinion, the less inelegant among all possibilities.

	\item
		Item~\#9, the last value that the \verb|\bullcntr| command can
		display.  The hearts are just stacked in a $3\times3$ square
		pattern---the obvious choice!
\end{bullenum}

% ( paren match
This could be used in valentines to enumerate to your beloved the
terms of your passion\ldots~:-)

Cross-references are a bit funny with this style; they look like this:
\ref{Hearts-1}, \ref{Hearts-3}, \ref{Hearts-4}, \ref{Hearts-5},
\ref{Hearts-7}, and~\ref{Hearts-8}.



\section{Nested enumerations}

Try not to overdo this: carrying it to excess could lead to confused
output.



\subsection{Nesting one \beenv\ inside another}
\label{SS:NestBullBull}

We use ``large'' bullets for the outer environment, and ``tiny''
bullets for the inner one.

\begin{bullenum}
	\largectrbull

	\item
		First item: let this one alone, just to have a term of
		comparison.

	\item\label{NestBull-2}
		Second item.  Here we nest another \beenv\ environment:
		\begin{bullenum}
			\renewcommand*{\counterlargebullet}{\textperiodcentered}
			\renewcommand*{\countersmallbullet}{\textperiodcentered}
			\item\label{NestBull-2.1}
				First sub-item.
			\item\label{NestBull-2.2}
				Second sub-item.
			\item\label{NestBull-2.3}
				Third sub-item.
			\item\label{NestBull-2.4}
				Fourth sub-item.  OK, this should suffice.
		\end{bullenum}
		Don't forget, we are still inside item~\ref{NestBull-2} of
		the outer enumeration.

	\item\label{NestBull-3}
		Third item (of the outer enumeration).  Let us repeat exactly
		the same nested enumeration as in item~\ref{NestBull-2}.
		\begin{bullenum}
			\renewcommand*{\counterlargebullet}{\textperiodcentered}
			\renewcommand*{\countersmallbullet}{\textperiodcentered}
			\item\label{NestBull-3.1}
				First sub-item.
			\item\label{NestBull-3.2}
				Second sub-item.
			\item\label{NestBull-3.3}
				Third sub-item.
			\item\label{NestBull-3.4}
				Fourth sub-item.  OK, this should suffice.
		\end{bullenum}
		Back to item~\ref{NestBull-3} of the outer enumeration.
\end{bullenum}

Now the bad news: you might want to refer to, say, item
\ref{NestBull-3.4} under item~\ref{NestBull-3}.  As you can see, the
reference comes out wrong, since the label from the outer enumeration
% ( paren match
is typeset with the style of the inner one.~:-(\spacefactor=\sfcode`.
\space This is a known bug: I\@ decided not to correct it, since doing
so would require to introduce the concept of bullet styles
distinguished by level, which I\@ deem intolerably burdensome.  Be
aware of this limitation when you nest \beenv\ environments inside one
another (this does not apply, however, to \beenv's nested within
\enenv's, as we'll see in the following examples).



\subsection{Nesting \beenv\ inside \enenv}
\label{SS:NestBullEnum}

Let us repeat the same example shown above, but changing the outer
environment to \enenv, and suppressing style variations.

\begin{enumerate}
	\item
		First item: let this one alone, just to have a term of
		comparison.

	\item\label{NestBuEn-2}
		Second item.  Here we nest a \beenv\ environment:
		\begin{bullenum}
			\item\label{NestBuEn-2.1}
				First sub-item.
			\item\label{NestBuEn-2.2}
				Second sub-item.
			\item\label{NestBuEn-2.3}
				Third sub-item.
			\item\label{NestBuEn-2.4}
				Fourth sub-item.  OK, this should suffice.
		\end{bullenum}
		Don't forget, we are still inside item~\ref{NestBuEn-2} of
		the outer enumeration.

	\item\label{NestBuEn-3}
		Third item (of the outer enumeration).  Let us repeat exactly
		the same nested enumeration as in item~\ref{NestBuEn-2}.
		\begin{bullenum}
			\item\label{NestBuEn-3.1}
				First sub-item.
			\item\label{NestBuEn-3.2}
				Second sub-item.
			\item\label{NestBuEn-3.3}
				Third sub-item.
			\item\label{NestBuEn-3.4}
				Fourth sub-item.  OK, this should suffice.
		\end{bullenum}
		Back to item~\ref{NestBuEn-3} of the outer enumeration.
\end{enumerate}

See Subsection~\ref{S:CrossXmp} for examples of cross-references to
the items of these lists.



\subsection{Nesting \enenv\ inside \beenv}
\label{SS:NestEnumBull}

As before, but the other way around: \beenv\ is now the outer
environment, and \enenv\ the inner one.

\begin{bullenum}
	\item
		First item: let this one alone, just to have a term of
		comparison.

	\item\label{NestEnBu-2}
		Second item.  Here we nest an \enenv\ environment:
		\begin{enumerate}
			\item\label{NestEnBu-2.1}
				First sub-item.
			\item\label{NestEnBu-2.2}
				Second sub-item.
			\item\label{NestEnBu-2.3}
				Third sub-item.
			\item\label{NestEnBu-2.4}
				Fourth sub-item.  OK, this should suffice.
		\end{enumerate}
		Don't forget, we are still inside item~\ref{NestEnBu-2} of
		the outer enumeration.

	\item\label{NestEnBu-3}
		Third item (of the outer enumeration).  Let us repeat exactly
		the same nested enumeration as in item~\ref{NestEnBu-2}.
		\begin{enumerate}
			\item\label{NestEnBu-3.1}
				First sub-item.
			\item\label{NestEnBu-3.2}
				Second sub-item.
			\item\label{NestEnBu-3.3}
				Third sub-item.
			\item\label{NestEnBu-3.4}
				Fourth sub-item.  OK, this should suffice.
		\end{enumerate}
		Back to item~\ref{NestEnBu-3} of the outer enumeration.
\end{bullenum}

Again, see Subsection~\ref{S:CrossXmp} for examples of
cross-references to the items of the above lists.



\section{Examples of cross-references}
\label{S:CrossXmp}

Cross-references are printed using the bullet style that was in effect
when the \verb|\label| command was given.  The following examples
illustrate this.

In~\ref{SS:Xmp-Standard} we had references to~\ref{Standard-3} and
to~\ref{Standard-6}.  In~\ref{SS:Xmp-Hearts}, on the other hand, the
references were to~\ref{Hearts-1}, to~\ref{Hearts-3},
to~\ref{Hearts-4}, to~\ref{Hearts-5}, to~\ref{Hearts-7}, and
to~\ref{Hearts-8}.

We turn now to nested lists.  We let alone the example
of~\ref{SS:NestBullBull}, we already know that references come out
wrong in this case.  In~\ref{SS:NestBullEnum} we had items
\ref{NestBuEn-2.1}, \ref{NestBuEn-2.2}, \ref{NestBuEn-2.3}, and
\ref{NestBuEn-2.4}, all listed under item~\ref{NestBuEn-2}, and
similarly items \ref{NestBuEn-3.1}, \ref{NestBuEn-3.2},
\ref{NestBuEn-3.3}, and \ref{NestBuEn-3.4}, as part of
item~\ref{NestBuEn-3}.  In~\ref{SS:NestEnumBull}, on the other hand,
we had items \ref{NestEnBu-2.1}, \ref{NestEnBu-2.2},
\ref{NestEnBu-2.3}, and \ref{NestEnBu-2.4}, all listed under
item~\ref{NestEnBu-2}: as you can see, the bullets now precede the
reference to the item of \enenv; similarly, items \ref{NestEnBu-3.1},
\ref{NestEnBu-3.2}, \ref{NestEnBu-3.3}, and \ref{NestEnBu-3.4} are now
a part of the ``bulletted'' item~\ref{NestEnBu-3}.



\setcounter{secnumdepth}{0}

\begin{thebibliography}{9}
	\providecommand*{\bysame}{\leavevmode\hbox to3em{\hrulefill}\thinspace}
	\addcontentsline{toc}{section}{\refname}

	\bibitem{dtx}
		G.~Mezzetti, \emph{The \bull\ package}, documented \TeX\
		source (dtx), included in every distribution of \tbull\ (in
		the file \filedir{\Bullcntr.dtx}).

	\bibitem{overview}
		\bysame, \emph{Overview of \tbull}, included in every
		distribution of \tbull, surely as \LaTeX\ source (in the file
		\filedir{\Bullcntr-man.tex}) and perhaps in precompiled PDF
		format too (in the file \filedir{\Bullcntr-man.pdf}).

	\bibitem{bull-sam}
		\bysame, \emph{Using \tbull: an example}, \LaTeX\ source,
		included in every distribution of \tbull\ (in the file
		\filedir{\Bullcntr-sam.tex}).
\end{thebibliography}

\end{document}
