\documentclass[compose]{exam-n}

\begin{document}

\begin{question}{20} \author{John Brown and Declan Diver}
\part An earth satellite in a highly eccentric orbit of (constant)
perigee distance $q$ undergoes a targential velocity impulse
$-\Delta V$ at each perigee passage.  By considering the mean rate
of change of velocity at perigee, show that the mean rate of
change of the semi-major axis  $a$ ($\gg q$) satisfies
\begin{equation*}
\frac{1}{a^2} \Diffl at =
\left(\frac{8}{GMq}\right)^{1/2}\frac{\Delta V}{T},
\end{equation*}
where $M$  is the Earth's mass and $T$ the orbital period.
\partmarks{3}
\begin{questiondata}
You may assume $\displaystyle v^2(r)=GM\left(\frac{2}{r}-\frac{1}{a}\right)$.
\end{questiondata}

Using $T=2\pi(a^3/GM)^{1/2}$  show that with $a_0=a(0)$, (where
$a(t)$ is the semimajor axis at time $t$)
\begin{equation*}
\frac{a(t)}{a_0}=\left[1-\frac{t\Delta V}{2^{1/2}\pi
a_0(1-e_0)^{1/2}}\right]^2
\end{equation*}
\partmarks*{2}
and
\begin{equation*}
\frac{T(t)}{T_0}=\left[1-\frac{t\Delta V}{2^{1/2}\pi
a_0(1-e_0)^{1/2}}\right]^3
\end{equation*}
\partmarks*{1}
and the eccentricity satisfies (with $e_0=e(0)$)
\begin{equation*}
e(t)=1-\frac{1-e_0}{\left[1-\frac{t\Delta V}{2^{1/2}\pi
a_0(1-e_0)^{1/2}}\right]^2}.
\end{equation*}
\partmarks*{2}

Show that, once the orbit is circular, its radius decays
exponentially with time on timescale  $m_0/2\dot{m}$  where  $m_0$
is the satellite mass and  $\dot{m}$ the mass of atmosphere
`stopped' by it per second. \partmarks{2}

\part      What is meant by (a) the sphere of influence of a star, and
(b) the passage distance?
\partmarks{2}

Consider a system of $N$ identical stars, each of mass $m$.

\part Given that the change $\delta u$ in the speed of one such star
due to the cumulative effect over time $t$  of many gravitational
encounters with other stars in the system can be approximated by
\begin{equation*}
(\delta u)^2 \propto [\nu tm^2\log(p_{\rm max}/p_{\rm
min})]/\bar{u},
\end{equation*}
where $\bar{u}$ is the rms mutual speed, $\nu$  is the stellar
number density, and $p_{\rm max, min}$  are the maximum, minimum
passage distances for the system, show that this leads to a natural
time $T$ for the system, where
\begin{equation*}
T\propto\frac{\bar{u}u^2}{m^2\nu\log N}.
\end{equation*}
\partmarks*{5}

\begin{questiondata}
You may assume that the sphere of influence radius of a star is
approximated by $(m/M)^{2/5}R$ where $R$  and $M$ are the radius
and mass of the whole system respectively.
\end{questiondata}

\part Deduce that $T$ is the disintegration timescale for the system,
by showing that a star with initial speed $u_0$ in a stable circular
orbit reaches escape speed after time  $T$.
\partmarks{3}

Dummy text, to lengthen the question to the extent that it spreads across three pages.
Dummy text, to lengthen the question to the extent that it spreads across three pages.
Dummy text, to lengthen the question to the extent that it spreads across three pages.
Dummy text, to lengthen the question to the extent that it spreads across three pages.
Dummy text, to lengthen the question to the extent that it spreads across three pages.
Dummy text, to lengthen the question to the extent that it spreads across three pages.
Dummy text, to lengthen the question to the extent that it spreads across three pages.
Dummy text, to lengthen the question to the extent that it spreads across three pages.
Dummy text, to lengthen the question to the extent that it spreads across three pages.
Dummy text, to lengthen the question to the extent that it spreads across three pages.
Dummy text, to lengthen the question to the extent that it spreads across three pages.
Dummy text, to lengthen the question to the extent that it spreads across three pages.
Dummy text, to lengthen the question to the extent that it spreads across three pages.
Dummy text, to lengthen the question to the extent that it spreads across three pages.
Dummy text, to lengthen the question to the extent that it spreads across three pages.
Dummy text, to lengthen the question to the extent that it spreads across three pages.
\end{question}
\end{document}
