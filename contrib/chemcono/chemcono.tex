\documentclass[DIV=9, pagesize=auto]{scrartcl}

\usepackage{fixltx2e}
\usepackage{etex}
\usepackage{xspace}
\usepackage{lmodern}
\usepackage[T1]{fontenc}
\usepackage{textcomp}
\usepackage{microtype}
\usepackage[unicode=true]{hyperref}

\newcommand*{\mail}[1]{\href{mailto:#1}{\texttt{#1}}}
\newcommand*{\pkg}[1]{\textsf{#1}}
\newcommand*{\cs}[1]{\texttt{\textbackslash#1}}
\makeatletter
\newcommand*{\cmd}[1]{\cs{\expandafter\@gobble\string#1}}
\makeatother
\newcommand*{\env}[1]{\texttt{#1}}
\newcommand*{\opt}[1]{\texttt{#1}}

\addtokomafont{title}{\rmfamily}

\title{The \pkg{chemcono} package\thanks{This manual corresponds to \pkg{chemcono}~v1.3, dated~9 Sep 1999.}}
\author{Stefan Schulz\\\mail{stefan.schulz@tu-bs.de}}
\date{9 Sep 1999}


\begin{document}

\maketitle

\section{History:}

Writing chemistry papers, very often a consecutive numbering for compounds
according to their appearance in the text is required. First, I have no
knowledge of any basic \TeX. I extracted the definitions for \cmd{\cite} and
\cmd{\thebibliography} from \texttt{latex.ltx} and \texttt{article.cls} and replaced every
occurrence of the string \texttt{cit} by \texttt{ffcit}
and \texttt{bib} by \texttt{ffbib}. In addition, I also performed some small formatting
changes. Thus, this package  worked nicely
in a document with 300 compound numbers and 700 citations in \LaTeX2.09 and \LaTeXe.

\medskip

\noindent
\begin{tabular}{@{}l@{ -- }p{11cm}@{}}
  Version 1.0 & \LaTeX2.09.                                                                                     \\
  Version 1.1 & Same thing for \LaTeXe.                                                                         \\
  Version 1.2 & Now works with \LaTeXe\ of 1996/06/01                                                           \\
  Version 1.3 & Added option \opt{[tight]}. Saves space in the compound number list. Default is \opt{[normal]}. \\
\end{tabular}


\section{Usage:}

Load package \pkg{chemcono}. For every compound, use \cmd{\fcite} like \cmd{\cite} for citations.
After a \cmd{\clearpage} at the end of your document use \env{theffbibliography} like
the normal \env{thebibliography}, with \cmd{\ffbibitem} instead of \cmd{\bibitem}:
%
\begin{verbatim}
\begin{theffbibliography}{99}
\ffbibitem{f1} danaidone
\ffbibitem{f2} thio ether
\end{theffbibliography}
\end{verbatim}
%
After compilation and printout, discard the last page.
If you want other formatting than bold face numbers, use
%
\begin{verbatim}
\renewcommand{\fcite}[1]{\underline{\ffcite{#1}}}
\end{verbatim}
%
which gives in this case  an underlined number.
Look at \texttt{example.tex}.
The formatting of the \env{ffbibliography} can be made tight to save space by the option \opt{[tight]}.

I have added the file \texttt{drftcono.sty}, which bases on the \texttt{drftcite.sty} file from D.~Arseneau.
When using \texttt{drftcono.sty}, you get the tags of the compound numbers instead the real numbers in the text.
In the compound number section, small superscripts show which label has been used first, second,
and so on. You have to load \texttt{drftcono.sty} after \texttt{chemcono.sty}. For the final version, just comment
out \texttt{drftcono.sty} as in \texttt{example.tex}.
I also added a file \texttt{showkeysff.sty}, which shows labels in the margin as in \pkg{showkeys}, but the \cmd{\fcite} tags
are also added. I was not able to include counting of occurrence present in \pkg{drftcono}
in the \pkg{showkeysff} package, which would be highly desirable.


\section{Complications:}

You must not use identical labels for \cmd{\fcite} and \cmd{\cite}. Otherwise \LaTeX\ gets
confused. As a hint, I identify every compound label by a preceding \texttt{f} as in
\verb+\fcite{f1}+, and every citation label with a \texttt{c}, like \verb+\cite{c1}+.


\section{Future:}

I would like to have the option for removing the printout of
\env{theffbibliography}. Than one would not need to discard the last page.
Because I have no knowledge of any programming, I do not know how to do
this. I also do not know whether the  goal of this style could be achieved
much more easily. Any hints are welcome.
Inclusion of order of occurrence counting in \pkg{showkeysff}.

\end{document}
