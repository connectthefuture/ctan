% \iffalse meta-comment
%
%% File: kantlipsum.dtx (C) Copyright 2011-2012 Enrico Gregorio
%%
%% It may be distributed and/or modified under the conditions of the
%% LaTeX Project Public License (LPPL), either version 1.3c of this
%% license or (at your option) any later version.  The latest version
%% of this license is in the file
%%
%%    http://www.latex-project.org/lppl.txt
%%
%% This file is part of the "kantlipsum bundle" (The Work in LPPL)
%% and all files in that bundle must be distributed together.
%% 
%% The released version of this bundle is available from CTAN.
%%
%
%<*driver|package>
\RequirePackage{expl3}
%</driver|package>
%<*driver>
\expandafter\def\csname ver@thumbpdf.sty\endcsname{}
\documentclass[a4paper,full]{l3doc}
\usepackage{bookmark}
%</driver>
%<*driver|package>
\GetIdInfo$Id: kantlipsum.dtx 0.6 2012-10-14 12:00:00Z Enrico $
          {Dummy text in Kantian style}
%</driver|package>
%<*driver>
\begin{document}
  \DocInput{\jobname.dtx}
\end{document}
%</driver>
% \fi
%
% \title{^^A
%   The \textsf{kantlipsum} package\\ Dummy text in Kantian style^^A
%   \thanks{This file describes version \ExplFileVersion,
%     last revised \ExplFileDate.}^^A
% }
%
% \author{^^A
%  Enrico Gregorio\thanks
%    {^^A
%      E-mail:
%        Enrico DOT Gregorio AT univr DOT it^^A
%    }^^A
% }
%
% \date{Released \ExplFileDate}
%
% \maketitle
%
% \begin{documentation}
%
% The \pkg{kantlipsum} package is modeled after \pkg{lipsum} and
% offers pretty similar functionality, but instead of pseudolatin
% utterances, it typesets paragraphs of nonsense in Kantian style
% produced by the \emph{Kant generator for Python} by Mark Pilgrim,
% found in \href{http://www.diveintopython.net/}{\emph{Dive into
% Python}}.
%
% It has at least one advantage over \pkg{lipsum}: the text is in
% English and so finding good hyphenation points should be less
% problematic. On the contrary, the paragraphs are rather long, as
% it's common in philosophical prose.
%
% \section{Options}
%
% The package has four document options, the first two of which are
% alternative to each other:
% \begin{itemize}[font=\ttfamily]
% \item[par$\,\vert\,$nopar] With the default \texttt{par} all pieces
% of text will be ended by a \cs{par} command; specifying \texttt{par}
% is optional; the option \texttt{nopar} will not add this \cs{par} at
% the end of each fragment of Kantian prose.
% \item[numbers] Each piece of Kantian prose will be preceded by its
% number (such as in ``1~\textbullet~As any dedicated reader can
% clearly see\dots''), which can be useful for better control of what
% is produced.
% \item[index] Each paragraph will generate an index entry; a
% |\makeindex| command will be needed, with a suitable package for
% making the index, and |\printindex| for printing it. However the
% index entry may be off by one, since the |\index| command is issued
% at the beginning of the paragraph. Also there is no guarantee that
% the indexed word really belongs to the paragraph.
% \end{itemize}
%
% \section{Commands}
%
% The commands provided by the package are:
% \begin{itemize}[font=\ttfamily]
% \item[\cs{kant}] This command takes an optional argument which can
% be of the form \texttt{[42]} (that is, only one integer) or
% \texttt{[3-14]} (that is, two integers separated by a hyphen); as in
% \pkg{lipsum}, \verb|\kant[42]|, \verb|\kant[3-14]| and \verb|\kant|
% will produce the 42nd pseudokantian paragraph, the paragraphs from
% the 3rd to the 14th, and those from the 1st to the 7th,
% respectively.
% \item[\cs{kant*}] The same as before, see later for the difference.
% \item[\cs{kantdef}] This command takes two arguments, a control
% sequence and an integer; the call \verb|\kantdef{\mytext}{164}| will
% store in \cs{mytext} the 164th paragraph of pseudokantian text
% provided by this package.
% \end{itemize}
%
% What's the difference between \cs{kant} and \cs{kant*}? The normal
% version will respect the given package option; that is, if
% \texttt{par} is in force, \verb|\kant[1-2]| will produce \emph{two}
% paragraphs, while \verb|\kant*[1-2]| will only produce a big chunk
% of text without issuing any \verb|\par| command. The logic is
% reversed if the \texttt{nopar} option has been given.
%
% By the way, 164 is the number of available pieces; if one exceeds
% the limit, nothing will be printed. Thus \verb|\kant[164-200]| will
% print only \emph{one} paragraph. However, printing all paragraphs
% with the standard ten point size Computer Modern font and the
% \pkg{article} class fills more than fifty pages, so it seems that
% the supply of text can be sufficient.
%
% \subsection*{Note}
%
% This package is just an exercise for practicing with \LaTeX3
% syntax. It uses the ``experimental'' packages made available by the
% \LaTeX3 team. Many thanks to Joseph Wright and Bruno Le Floch for
% suggesting improvements.
%
% \subsection*{Changes from version 0.1}
%
% There's no user level change; the implementation has been modified
% in some places (in particular a sequence is used to store the
% phrases, rather than many token lists).
%
% \subsection*{Changes from version 0.5}
%
% Some changes in \LaTeX3 introduced some misfeatures, which this
% version corrects. Some kernel function names were also changed; here
% |\prg_stepwise_function:nnnN| that became |\int_step_function:nnnN|.
% Some functions have been made |protected|.
%
% The most striking change is the possibility to generate an index:
% each paragraph indexes one of its words or phrases.
%
% \end{documentation}
%
% \begin{implementation}
%
% \section{\pkg{kantlipsum} implementation}
%
% \iffalse
%<*package>
% \fi
%
%    \begin{macrocode}
\ProvidesExplPackage
  {\ExplFileName}{\ExplFileDate}{\ExplFileVersion}{\ExplFileDescription}
%    \end{macrocode}
%
% A check to make sure that \pkg{expl3} is not too old
%    \begin{macrocode}
\@ifpackagelater { expl3 } { 2012/07/15 }
  { }
  {
    \PackageError { kantlipsum } { Support~package~l3kernel~too~old. }
      {
        Please~install~an~up~to~date~version~of~l3kernel~
        using~your~TeX~package~manager~or~from~CTAN.\\ \\
        Loading~xparse~will~abort!
      }
    \tex_endinput:D
  }
%    \end{macrocode}
%
% \subsection{Package options and required packages}
% We declare the allowed options and choose by default
% \texttt{par}. We also need to declare a function |\kgl_number:n|
% that is set by the \texttt{numbers} option; its default action is to
% gobble its argument.
%    \begin{macrocode}
\DeclareOption { par }
  {
   \cs_set_protected:Nn \kgl_star: { \c_space_tl }
   \cs_set_protected:Nn \kgl_nostar: { \par } 
  }

\DeclareOption{ nopar }
  {
   \cs_set_protected:Nn \kgl_star: { \par }
   \cs_set_protected:Nn \kgl_nostar: { \c_space_tl }
  }

\DeclareOption{ numbers }
  { \cs_set_protected:Nn \kgl_number:n { #1\nobreakspace\textbullet\nobreakspace } }

\bool_new:N \g_kgl_makeindex_bool
\bool_gset_false:N \g_kgl_makeindex_bool
\DeclareOption{ index }
  { \bool_gset_true:N \g_kgl_makeindex_bool }

\cs_new_eq:NN \kgl_number:n \use_none:n
\ExecuteOptions{par}
\ProcessOptions \scan_stop:
%    \end{macrocode}
%
% The \pkg{xparse} package is required.
%    \begin{macrocode}
\RequirePackage{xparse}
%    \end{macrocode}
%
% \subsection{Messages}
% We define two messages.
%    \begin{macrocode}
\msg_new:nnn {kantlipsum}{how-many}
  {The~package~provides~paragraphs~1~to~#1.~
   Values~outside~this~range~will~be~ignored.}
\msg_new:nnnn {kantlipsum}{already-defined}
  {Control~sequence~#1~already~defined.}
  {The~control~sequence~#1~is~already~defined,~
   I'll~ignore~it}
%    \end{macrocode}
%
% \subsection{Variables and constants}
% The |\l_kgl_start_int| variable will contain the starting number for
% processing, while |\l_kgl_end_int| the ending number. The
% |\g_kgl_pars_seq| sequence will contain the pseudokantian sentences
% and |\g_kgl_words_seq| that contains the words to index.
%    \begin{macrocode}
\int_new:N \l_kgl_start_int
\int_new:N \l_kgl_end_int
\seq_new:N \g_kgl_pars_seq
\seq_new:N \g_kgl_words_seq
%    \end{macrocode}
%
% \subsection{User level commands}
% There are two user level commands, \cs{kant} (with a *-variant) and \cs{kantdef}.
% 
% \begin{function}{\kant}
% The (optional) argument is described as before. We use the
% \cs{SplitArgument} feature provided by \pkg{xparse} to decide
% whether the `range form' has been specified. In the \cs{kant*} form
% we reverse the logic.
%    \begin{macrocode}
\NewDocumentCommand{\kant}{s>{\SplitArgument{1}{-}}O{1-7}}
  {
   \group_begin:
   \IfBooleanTF{#1}
     { \cs_set_eq:NN \kgl_par: \kgl_star: }
     { \cs_set_eq:NN \kgl_par: \kgl_nostar: }
   \kgl_process:nn #2
   \kgl_print:
   \group_end:
  }
%    \end{macrocode}
% \end{function}
%
% \begin{function}{\kantdef}
% Sometimes one needs just a piece of text without implicit \cs{par}
% attached, so we provide \cs{kantdef}. In a group we neutralize the
% meaning of |\kgl_number:n| and |\kgl_par:| and define the control
% sequence given as first argument to the pseudokantian sentence being
% the $k$th element of the sequence containing them, where $k$ is the
% number given as second argument. If the control sequence is already
% defined we issue an error and don't perform the definition.
%    \begin{macrocode}
\NewDocumentCommand{\kantdef}{mm}
  {
   \group_begin:
   \cs_set_eq:NN \kgl_number:n \use_none:n
   \cs_set_eq:NN \kgl_par: \prg_do_nothing:
   \cs_if_exist:NTF #1
     {
      \msg_error:nnn {kantlipsum} {already-defined} {#1}
     }
     {
      \tl_set:Nx \l_tmpa_tl { \seq_item:Nn \g_kgl_pars_seq {#2} }
      \cs_new:Npx #1 { \l_tmpa_tl }
     }
   \group_end:
  }
%    \end{macrocode}
% \end{function}
%
% \subsection{Internal functions}
% \begin{function}{\kgl_process:nn}
% The function |\kgl_process:nn| sets the temporary variables
% |\l_kgl_start_int| and |\l_kgl_end_int|. If the optional argument to
% \cs{kant} is missing they are already set to 1 and 7 respectively;
% otherwise the argument has been split into its components; if the
% argument was |[|$m$|]| we set both variables to $m$, otherwise it
% was in the form |[|$m$|-|$n$|]| and we do the obvious action.
%    \begin{macrocode}
\cs_new_protected:Nn \kgl_process:nn
  {
   \int_set:Nn \l_kgl_start_int {#1}
   \IfNoValueTF{#2}
     { \int_set:Nn \l_kgl_end_int {#1} }
     { \int_set:Nn \l_kgl_end_int {#2} }
  }
%    \end{macrocode}
%\end{function}
%
% \begin{function}{\kgl_print:,\kgl_use:n}
% The printing routine is in the function |\kgl_print:|; we start a
% loop printing item number $x$ in the sequence |\g_kgl_pars_seq| for
% all numbers $x$ in the specified range. The function |\kgl_use:n|
% function is a wrapper to be used with |\int_step_function:nnnN|:
% it's passed a number as argument, builds the constant name
% corresponding to it and produces the text. If the index entry is to
% be issued, the appropriate element from |\g_kgl_words_seq| is used;
% the page reference might not be correct, though.
%    \begin{macrocode}
\cs_new_protected:Nn \kgl_print:
  { 
   \int_step_function:nnnN 
     {\l_kgl_start_int} {1} {\l_kgl_end_int} \kgl_use:n
  }
\cs_new:Nn \kgl_use:n
  {
   \kgl_number:n {#1}
   \bool_if:NT \g_kgl_makeindex_bool
    {
     \use:x { \exp_not:N \index{ \seq_item:Nn \g_kgl_words_seq {#1} } }
    }
   \seq_item:Nn \g_kgl_pars_seq {#1}
  }
%    \end{macrocode}
% \end{function}
%
% \begin{function}{\kgl_newpara:n}
% The |\kgl_newpara:n| appends a new item to the sequence |\g_kgl_pars_seq|
% consisting of, say, \meta{text of the 42nd sentence}|\kgl_par:|
%    \begin{macrocode}
\cs_new_protected:Nn \kgl_newpara:n
  { \seq_gput_right:Nn \g_kgl_pars_seq {#1\kgl_par:} }
%    \end{macrocode}
% \end{function}
%
% \begin{function}{\kgl_newword:n}
% The |\kgl_newword:n| appends a new item to the sequence |\g_kgl_words_seq|
% consisting of one word from the corresponding paragraph.
%    \begin{macrocode}
\cs_new_protected:Nn \kgl_newword:n
  { \seq_gput_right:Nn \g_kgl_words_seq {#1} }
%    \end{macrocode}
% \end{function}
%
% \subsection{Defining the sentences}
% We start a group where we set |\l_tmpa_int| to 0 and the category
% code of the space to 10 so as not to be forced to write |~| for
% spaces.
%    \begin{macrocode}
\group_begin:
\char_set_catcode_space:n {`\ }
%    \end{macrocode}
%
% Then we provide all of the sentences with the pattern
% |\kgl_newpara:n {|\meta{text}|}|
%    \begin{macrocode}
\kgl_newpara:n {As any dedicated reader can clearly see, the Ideal of
practical reason is a representation of, as far as I know, the things
in themselves; as I have shown elsewhere, the phenomena should only be
used as a canon for our understanding. The paralogisms of practical
reason are what first give rise to the architectonic of practical
reason. As will easily be shown in the next section, reason would
thereby be made to contradict, in view of these considerations, the
Ideal of practical reason, yet the manifold depends on the phenomena.
Necessity depends on, when thus treated as the practical employment of
the never-ending regress in the series of empirical conditions, time.
Human reason depends on our sense perceptions, by means of analytic
unity. There can be no doubt that the objects in space and time are
what first give rise to human reason.}

\kgl_newpara:n {Let us suppose that the noumena have nothing to do
with necessity, since knowledge of the Categories is a
posteriori. Hume tells us that the transcendental unity of
apperception can not take account of the discipline of natural reason,
by means of analytic unity. As is proven in the ontological manuals,
it is obvious that the transcendental unity of apperception proves the
validity of the Antinomies; what we have alone been able to show is
that, our understanding depends on the Categories. It remains a
mystery why the Ideal stands in need of reason. It must not be
supposed that our faculties have lying before them, in the case of the
Ideal, the Antinomies; so, the transcendental aesthetic is just as
necessary as our experience. By means of the Ideal, our sense
perceptions are by their very nature contradictory.}

\kgl_newpara:n {As is shown in the writings of Aristotle, the things
in themselves (and it remains a mystery why this is the case) are a
representation of time. Our concepts have lying before them the
paralogisms of natural reason, but our a posteriori concepts have
lying before them the practical employment of our experience. Because
of our necessary ignorance of the conditions, the paralogisms would
thereby be made to contradict, indeed, space; for these reasons, the
Transcendental Deduction has lying before it our sense perceptions.
(Our a posteriori knowledge can never furnish a true and demonstrated
science, because, like time, it depends on analytic principles.) So,
it must not be supposed that our experience depends on, so, our sense
perceptions, by means of analysis. Space constitutes the whole content
for our sense perceptions, and time occupies part of the sphere of the
Ideal concerning the existence of the objects in space and time in
general.}

\kgl_newpara:n {As we have already seen, what we have alone been able
to show is that the objects in space and time would be falsified; what
we have alone been able to show is that, our judgements are what first
give rise to metaphysics. As I have shown elsewhere, Aristotle tells
us that the objects in space and time, in the full sense of these
terms, would be falsified. Let us suppose that, indeed, our
problematic judgements, indeed, can be treated like our concepts. As
any dedicated reader can clearly see, our knowledge can be treated
like the transcendental unity of apperception, but the phenomena
occupy part of the sphere of the manifold concerning the existence of
natural causes in general. Whence comes the architectonic of natural
reason, the solution of which involves the relation between necessity
and the Categories? Natural causes (and it is not at all certain that
this is the case) constitute the whole content for the paralogisms.
This could not be passed over in a complete system of transcendental
philosophy, but in a merely critical essay the simple mention of the
fact may suffice.}

\kgl_newpara:n {Therefore, we can deduce that the objects in space and
time (and I assert, however, that this is the case) have lying before
them the objects in space and time. Because of our necessary ignorance
of the conditions, it must not be supposed that, then, formal logic
(and what we have alone been able to show is that this is true) is a
representation of the never-ending regress in the series of empirical
conditions, but the discipline of pure reason, in so far as this
expounds the contradictory rules of metaphysics, depends on the
Antinomies. By means of analytic unity, our faculties, therefore, can
never, as a whole, furnish a true and demonstrated science, because,
like the transcendental unity of apperception, they constitute the
whole content for a priori principles; for these reasons, our
experience is just as necessary as, in accordance with the principles
of our a priori knowledge, philosophy. The objects in space and time
abstract from all content of knowledge. Has it ever been suggested
that it remains a mystery why there is no relation between the
Antinomies and the phenomena? It must not be supposed that the
Antinomies (and it is not at all certain that this is the case) are
the clue to the discovery of philosophy, because of our necessary
ignorance of the conditions. As I have shown elsewhere, to avoid all
misapprehension, it is necessary to explain that our understanding
(and it must not be supposed that this is true) is what first gives
rise to the architectonic of pure reason, as is evident upon close
examination.}

\kgl_newpara:n {The things in themselves are what first give rise to
reason, as is proven in the ontological manuals. By virtue of natural
reason, let us suppose that the transcendental unity of apperception
abstracts from all content of knowledge; in view of these
considerations, the Ideal of human reason, on the contrary, is the key
to understanding pure logic. Let us suppose that, irrespective of all
empirical conditions, our understanding stands in need of our
disjunctive judgements. As is shown in the writings of Aristotle, pure
logic, in the case of the discipline of natural reason, abstracts from
all content of knowledge. Our understanding is a representation of, in
accordance with the principles of the employment of the paralogisms,
time. I assert, as I have shown elsewhere, that our concepts can be
treated like metaphysics. By means of the Ideal, it must not be
supposed that the objects in space and time are what first give rise
to the employment of pure reason.}

\kgl_newpara:n {As is evident upon close examination, to avoid all
misapprehension, it is necessary to explain that, on the contrary, the
never-ending regress in the series of empirical conditions is a
representation of our inductive judgements, yet the things in
themselves prove the validity of, on the contrary, the Categories.  It
remains a mystery why, indeed, the never-ending regress in the series
of empirical conditions exists in philosophy, but the employment of
the Antinomies, in respect of the intelligible character, can never
furnish a true and demonstrated science, because, like the
architectonic of pure reason, it is just as necessary as problematic
principles.  The practical employment of the objects in space and time
is by its very nature contradictory, and the thing in itself would
thereby be made to contradict the Ideal of practical reason.  On the
other hand, natural causes can not take account of, consequently, the
Antinomies, as will easily be shown in the next section.
Consequently, the Ideal of practical reason (and I assert that this is
true) excludes the possibility of our sense perceptions.  Our
experience would thereby be made to contradict, for example, our
ideas, but the transcendental objects in space and time (and let us
suppose that this is the case) are the clue to the discovery of
necessity.  But the proof of this is a task from which we can here be
absolved.}

\kgl_newpara:n {Thus, the Antinomies exclude the possibility of, on
the other hand, natural causes, as will easily be shown in the next
section.  Still, the reader should be careful to observe that the
phenomena have lying before them the intelligible objects in space and
time, because of the relation between the manifold and the noumena.
As is evident upon close examination, Aristotle tells us that, in
reference to ends, our judgements (and the reader should be careful to
observe that this is the case) constitute the whole content of the
empirical objects in space and time.  Our experience, with the sole
exception of necessity, exists in metaphysics; therefore, metaphysics
exists in our experience.  (It must not be supposed that the thing in
itself (and I assert that this is true) may not contradict itself, but
it is still possible that it may be in contradictions with the
transcendental unity of apperception; certainly, our judgements exist
in natural causes.)  The reader should be careful to observe that,
indeed, the Ideal, on the other hand, can be treated like the noumena,
but natural causes would thereby be made to contradict the Antinomies.
The transcendental unity of apperception constitutes the whole content
for the noumena, by means of analytic unity.}

\kgl_newpara:n {In all theoretical sciences, the paralogisms of human
reason would be falsified, as is proven in the ontological manuals.
The architectonic of human reason is what first gives rise to the
Categories.  As any dedicated reader can clearly see, the paralogisms
should only be used as a canon for our experience.  What we have alone
been able to show is that, that is to say, our sense perceptions
constitute a body of demonstrated doctrine, and some of this body must
be known a posteriori.  Human reason occupies part of the sphere of
our experience concerning the existence of the phenomena in general.}

\kgl_newpara:n {By virtue of natural reason, our ampliative judgements
would thereby be made to contradict, in all theoretical sciences, the
pure employment of the discipline of human reason.  Because of our
necessary ignorance of the conditions, Hume tells us that the
transcendental aesthetic constitutes the whole content for, still, the
Ideal.  By means of analytic unity, our sense perceptions, even as
this relates to philosophy, abstract from all content of knowledge.
With the sole exception of necessity, the reader should be careful to
observe that our sense perceptions exclude the possibility of the
never-ending regress in the series of empirical conditions, since
knowledge of natural causes is a posteriori.  Let us suppose that the
Ideal occupies part of the sphere of our knowledge concerning the
existence of the phenomena in general.}

\kgl_newpara:n {By virtue of natural reason, what we have alone been
able to show is that, in so far as this expounds the universal rules
of our a posteriori concepts, the architectonic of natural reason can
be treated like the architectonic of practical reason.  Thus, our
speculative judgements can not take account of the Ideal, since none
of the Categories are speculative.  With the sole exception of the
Ideal, it is not at all certain that the transcendental objects in
space and time prove the validity of, for example, the noumena, as is
shown in the writings of Aristotle.  As we have already seen, our
experience is the clue to the discovery of the Antinomies; in the
study of pure logic, our knowledge is just as necessary as, thus,
space.  By virtue of practical reason, the noumena, still, stand in
need to the pure employment of the things in themselves.}

\kgl_newpara:n {The reader should be careful to observe that the
objects in space and time are the clue to the discovery of, certainly,
our a priori knowledge, by means of analytic unity.  Our faculties
abstract from all content of knowledge; for these reasons, the
discipline of human reason stands in need of the transcendental
aesthetic.  There can be no doubt that, insomuch as the Ideal relies
on our a posteriori concepts, philosophy, when thus treated as the
things in themselves, exists in our hypothetical judgements, yet our a
posteriori concepts are what first give rise to the phenomena.
Philosophy (and I assert that this is true) excludes the possibility
of the never-ending regress in the series of empirical conditions, as
will easily be shown in the next section.  Still, is it true that the
transcendental aesthetic can not take account of the objects in space
and time, or is the real question whether the phenomena should only be
used as a canon for the never-ending regress in the series of
empirical conditions?  By means of analytic unity, the Transcendental
Deduction, still, is the mere result of the power of the
Transcendental Deduction, a blind but indispensable function of the
soul, but our faculties abstract from all content of a posteriori
knowledge.  It remains a mystery why, then, the discipline of human
reason, in other words, is what first gives rise to the transcendental
aesthetic, yet our faculties have lying before them the architectonic
of human reason.}

\kgl_newpara:n {However, we can deduce that our experience (and it
must not be supposed that this is true) stands in need of our
experience, as we have already seen.  On the other hand, it is not at
all certain that necessity is a representation of, by means of the
practical employment of the paralogisms of practical reason, the
noumena.  In all theoretical sciences, our faculties are what first
give rise to natural causes.  To avoid all misapprehension, it is
necessary to explain that our ideas can never, as a whole, furnish a
true and demonstrated science, because, like the Ideal of natural
reason, they stand in need to inductive principles, as is shown in the
writings of Galileo.  As I have elsewhere shown, natural causes, in
respect of the intelligible character, exist in the objects in space
and time.}

\kgl_newpara:n {Our ideas, in the case of the Ideal of pure reason,
are by their very nature contradictory.  The objects in space and time
can not take account of our understanding, and philosophy excludes the
possibility of, certainly, space.  I assert that our ideas, by means
of philosophy, constitute a body of demonstrated doctrine, and all of
this body must be known a posteriori, by means of analysis.  It must
not be supposed that space is by its very nature contradictory.  Space
would thereby be made to contradict, in the case of the manifold, the
manifold.  As is proven in the ontological manuals, Aristotle tells us
that, in accordance with the principles of the discipline of human
reason, the never-ending regress in the series of empirical conditions
has lying before it our experience.  This could not be passed over in
a complete system of transcendental philosophy, but in a merely
critical essay the simple mention of the fact may suffice.}

\kgl_newpara:n {Since knowledge of our faculties is a posteriori, pure
logic teaches us nothing whatsoever regarding the content of, indeed,
the architectonic of human reason.  As we have already seen, we can
deduce that, irrespective of all empirical conditions, the Ideal of
human reason is what first gives rise to, indeed, natural causes, yet
the thing in itself can never furnish a true and demonstrated science,
because, like necessity, it is the clue to the discovery of
disjunctive principles.  On the other hand, the manifold depends on
the paralogisms.  Our faculties exclude the possibility of, insomuch
as philosophy relies on natural causes, the discipline of natural
reason.  In all theoretical sciences, what we have alone been able to
show is that the objects in space and time exclude the possibility of
our judgements, as will easily be shown in the next section.  This is
what chiefly concerns us.}

\kgl_newpara:n {Time (and let us suppose that this is true) is the
clue to the discovery of the Categories, as we have already seen.
Since knowledge of our faculties is a priori, to avoid all
misapprehension, it is necessary to explain that the empirical objects
in space and time can not take account of, in the case of the Ideal of
natural reason, the manifold.  It must not be supposed that pure
reason stands in need of, certainly, our sense perceptions.  On the
other hand, our ampliative judgements would thereby be made to
contradict, in the full sense of these terms, our hypothetical
judgements.  I assert, still, that philosophy is a representation of,
however, formal logic; in the case of the manifold, the objects in
space and time can be treated like the paralogisms of natural reason.
This is what chiefly concerns us.}

\kgl_newpara:n {Because of the relation between pure logic and natural
causes, to avoid all misapprehension, it is necessary to explain that,
even as this relates to the thing in itself, pure reason constitutes
the whole content for our concepts, but the Ideal of practical reason
may not contradict itself, but it is still possible that it may be in
contradictions with, then, natural reason.  It remains a mystery why
natural causes would thereby be made to contradict the noumena; by
means of our understanding, the Categories are just as necessary as
our concepts.  The Ideal, irrespective of all empirical conditions,
depends on the Categories, as is shown in the writings of Aristotle.
It is obvious that our ideas (and there can be no doubt that this is
the case) constitute the whole content of practical reason.  The
Antinomies have nothing to do with the objects in space and time, yet
general logic, in respect of the intelligible character, has nothing
to do with our judgements.  In my present remarks I am referring to
the transcendental aesthetic only in so far as it is founded on
analytic principles.}

\kgl_newpara:n {With the sole exception of our a priori knowledge, our
faculties have nothing to do with our faculties.  Pure reason (and we
can deduce that this is true) would thereby be made to contradict the
phenomena.  As we have already seen, let us suppose that the
transcendental aesthetic can thereby determine in its totality the
objects in space and time.  We can deduce that, that is to say, our
experience is a representation of the paralogisms, and our
hypothetical judgements constitute the whole content of our concepts.
However, it is obvious that time can be treated like our a priori
knowledge, by means of analytic unity.  Philosophy has nothing to do
with natural causes.}

\kgl_newpara:n {By means of analysis, our faculties stand in need to,
indeed, the empirical objects in space and time.  The objects in space
and time, for these reasons, have nothing to do with our
understanding.  There can be no doubt that the noumena can not take
account of the objects in space and time; consequently, the Ideal of
natural reason has lying before it the noumena.  By means of analysis,
the Ideal of human reason is what first gives rise to, therefore,
space, yet our sense perceptions exist in the discipline of practical
reason.}

\kgl_newpara:n {The Ideal can not take account of, so far as I know,
our faculties.  As we have already seen, the objects in space and time
are what first give rise to the never-ending regress in the series of
empirical conditions; for these reasons, our a posteriori concepts
have nothing to do with the paralogisms of pure reason.  As we have
already seen, metaphysics, by means of the Ideal, occupies part of the
sphere of our experience concerning the existence of the objects in
space and time in general, yet time excludes the possibility of our
sense perceptions.  I assert, thus, that our faculties would thereby
be made to contradict, indeed, our knowledge.  Natural causes, so
regarded, exist in our judgements.}

\kgl_newpara:n {The never-ending regress in the series of empirical
conditions may not contradict itself, but it is still possible that it
may be in contradictions with, then, applied logic.  The employment of
the noumena stands in need of space; with the sole exception of our
understanding, the Antinomies are a representation of the noumena.  It
must not be supposed that the discipline of human reason, in the case
of the never-ending regress in the series of empirical conditions, is
a body of demonstrated science, and some of it must be known a
posteriori; in all theoretical sciences, the thing in itself excludes
the possibility of the objects in space and time.  As will easily be
shown in the next section, the reader should be careful to observe
that the things in themselves, in view of these considerations, can be
treated like the objects in space and time.  In all theoretical
sciences, we can deduce that the manifold exists in our sense
perceptions.  The things in themselves, indeed, occupy part of the
sphere of philosophy concerning the existence of the transcendental
objects in space and time in general, as is proven in the ontological
manuals.}

\kgl_newpara:n {The transcendental unity of apperception, in the case
of philosophy, is a body of demonstrated science, and some of it must
be known a posteriori.  Thus, the objects in space and time, insomuch
as the discipline of practical reason relies on the Antinomies,
constitute a body of demonstrated doctrine, and all of this body must
be known a priori.  Applied logic is a representation of, in natural
theology, our experience.  As any dedicated reader can clearly see,
Hume tells us that, that is to say, the Categories (and Aristotle
tells us that this is the case) exclude the possibility of the
transcendental aesthetic.  (Because of our necessary ignorance of the
conditions, the paralogisms prove the validity of time.)  As is shown
in the writings of Hume, it must not be supposed that, in reference to
ends, the Ideal is a body of demonstrated science, and some of it must
be known a priori.  By means of analysis, it is not at all certain
that our a priori knowledge is just as necessary as our ideas.  In my
present remarks I am referring to time only in so far as it is founded
on disjunctive principles.}

\kgl_newpara:n {The discipline of pure reason is what first gives rise
to the Categories, but applied logic is the clue to the discovery of
our sense perceptions.  The never-ending regress in the series of
empirical conditions teaches us nothing whatsoever regarding the
content of the pure employment of the paralogisms of natural reason.
Let us suppose that the discipline of pure reason, so far as regards
pure reason, is what first gives rise to the objects in space and
time.  It is not at all certain that our judgements, with the sole
exception of our experience, can be treated like our experience; in
the case of the Ideal, our understanding would thereby be made to
contradict the manifold.  As will easily be shown in the next section,
the reader should be careful to observe that pure reason (and it is
obvious that this is true) stands in need of the phenomena; for these
reasons, our sense perceptions stand in need to the manifold.  Our
ideas are what first give rise to the paralogisms.}

\kgl_newpara:n {The things in themselves have lying before them the
Antinomies, by virtue of human reason.  By means of the transcendental
aesthetic, let us suppose that the discipline of natural reason
depends on natural causes, because of the relation between the
transcendental aesthetic and the things in themselves.  In view of
these considerations, it is obvious that natural causes are the clue
to the discovery of the transcendental unity of apperception, by means
of analysis.  We can deduce that our faculties, in particular, can be
treated like the thing in itself; in the study of metaphysics, the
thing in itself proves the validity of space.  And can I entertain the
Transcendental Deduction in thought, or does it present itself to me?
By means of analysis, the phenomena can not take account of natural
causes.  This is not something we are in a position to establish.}

\kgl_newpara:n {Since some of the things in themselves are a
posteriori, there can be no doubt that, when thus treated as our
understanding, pure reason depends on, still, the Ideal of natural
reason, and our speculative judgements constitute a body of
demonstrated doctrine, and all of this body must be known a
posteriori.  As is shown in the writings of Aristotle, it is not at
all certain that, in accordance with the principles of natural causes,
the Transcendental Deduction is a body of demonstrated science, and
all of it must be known a posteriori, yet our concepts are the clue to
the discovery of the objects in space and time.  Therefore, it is
obvious that formal logic would be falsified.  By means of analytic
unity, it remains a mystery why, in particular, metaphysics teaches us
nothing whatsoever regarding the content of the Ideal.  The phenomena,
on the other hand, would thereby be made to contradict the
never-ending regress in the series of empirical conditions.  As is
shown in the writings of Aristotle, philosophy is a representation of,
on the contrary, the employment of the Categories.  Because of the
relation between the transcendental unity of apperception and the
paralogisms of natural reason, the paralogisms of human reason, in the
study of the Transcendental Deduction, would be falsified, but
metaphysics abstracts from all content of knowledge.}

\kgl_newpara:n {Since some of natural causes are disjunctive, the
never-ending regress in the series of empirical conditions is the key
to understanding, in particular, the noumena.  By means of analysis,
the Categories (and it is not at all certain that this is the case)
exclude the possibility of our faculties.  Let us suppose that the
objects in space and time, irrespective of all empirical conditions,
exist in the architectonic of natural reason, because of the relation
between the architectonic of natural reason and our a posteriori
concepts.  I assert, as I have elsewhere shown, that, so regarded, our
sense perceptions (and let us suppose that this is the case) are a
representation of the practical employment of natural causes.  (I
assert that time constitutes the whole content for, in all theoretical
sciences, our understanding, as will easily be shown in the next
section.)  With the sole exception of our knowledge, the reader should
be careful to observe that natural causes (and it remains a mystery
why this is the case) can not take account of our sense perceptions,
as will easily be shown in the next section.  Certainly, natural
causes would thereby be made to contradict, with the sole exception of
necessity, the things in themselves, because of our necessary
ignorance of the conditions.  But to this matter no answer is
possible.}

\kgl_newpara:n {Since all of the objects in space and time are
synthetic, it remains a mystery why, even as this relates to our
experience, our a priori concepts should only be used as a canon for
our judgements, but the phenomena should only be used as a canon for
the practical employment of our judgements.  Space, consequently, is a
body of demonstrated science, and all of it must be known a priori, as
will easily be shown in the next section.  We can deduce that the
Categories have lying before them the phenomena.  Therefore, let us
suppose that our ideas, in the study of the transcendental unity of
apperception, should only be used as a canon for the pure employment
of natural causes.  Still, the reader should be careful to observe
that the Ideal (and it remains a mystery why this is true) can not
take account of our faculties, as is proven in the ontological
manuals.  Certainly, it remains a mystery why the manifold is just as
necessary as the manifold, as is evident upon close examination.}

\kgl_newpara:n {In natural theology, what we have alone been able to
show is that the architectonic of practical reason is the clue to the
discovery of, still, the manifold, by means of analysis.  Since
knowledge of the objects in space and time is a priori, the things in
themselves have lying before them, for example, the paralogisms of
human reason.  Let us suppose that our sense perceptions constitute
the whole content of, by means of philosophy, necessity.  Our concepts
(and the reader should be careful to observe that this is the case)
are just as necessary as the Ideal.  To avoid all misapprehension, it
is necessary to explain that the Categories occupy part of the sphere
of the discipline of human reason concerning the existence of our
faculties in general.  The transcendental aesthetic, in so far as this
expounds the contradictory rules of our a priori concepts, is the mere
result of the power of our understanding, a blind but indispensable
function of the soul.  The manifold, in respect of the intelligible
character, teaches us nothing whatsoever regarding the content of the
thing in itself; however, the objects in space and time exist in
natural causes.}

\kgl_newpara:n {I assert, however, that our a posteriori concepts (and
it is obvious that this is the case) would thereby be made to
contradict the discipline of practical reason; however, the things in
themselves, however, constitute the whole content of philosophy.  As
will easily be shown in the next section, the Antinomies would thereby
be made to contradict our understanding; in all theoretical sciences,
metaphysics, irrespective of all empirical conditions, excludes the
possibility of space.  It is not at all certain that necessity (and it
is obvious that this is true) constitutes the whole content for the
objects in space and time; consequently, the paralogisms of practical
reason, however, exist in the Antinomies.  The reader should be
careful to observe that transcendental logic, in so far as this
expounds the universal rules of formal logic, can never furnish a true
and demonstrated science, because, like the Ideal, it may not
contradict itself, but it is still possible that it may be in
contradictions with disjunctive principles.  (Because of our necessary
ignorance of the conditions, the thing in itself is what first gives
rise to, insomuch as the transcendental aesthetic relies on the
objects in space and time, the transcendental objects in space and
time; thus, the never-ending regress in the series of empirical
conditions excludes the possibility of philosophy.)  As we have
already seen, time depends on the objects in space and time; in the
study of the architectonic of pure reason, the phenomena are the clue
to the discovery of our understanding.  Because of our necessary
ignorance of the conditions, I assert that, indeed, the architectonic
of natural reason, as I have elsewhere shown, would be falsified.}

\kgl_newpara:n {In natural theology, the transcendental unity of
apperception has nothing to do with the Antinomies.  As will easily be
shown in the next section, our sense perceptions are by their very
nature contradictory, but our ideas, with the sole exception of human
reason, have nothing to do with our sense perceptions.  Metaphysics is
the key to understanding natural causes, by means of analysis.  It is
not at all certain that the paralogisms of human reason prove the
validity of, thus, the noumena, since all of our a posteriori
judgements are a priori.  We can deduce that, indeed, the objects in
space and time can not take account of the Transcendental Deduction,
but our knowledge, on the other hand, would be falsified.}

\kgl_newpara:n {As we have already seen, our understanding is the clue
to the discovery of necessity.  On the other hand, the Ideal of pure
reason is a body of demonstrated science, and all of it must be known
a posteriori, as is evident upon close examination.  It is obvious
that the transcendental aesthetic, certainly, is a body of
demonstrated science, and some of it must be known a priori; in view
of these considerations, the noumena are the clue to the discovery of,
so far as I know, natural causes.  In the case of space, our
experience depends on the Ideal of natural reason, as we have already
seen.}

\kgl_newpara:n {For these reasons, space is the key to understanding
the thing in itself.  Our sense perceptions abstract from all content
of a priori knowledge, but the phenomena can never, as a whole,
furnish a true and demonstrated science, because, like time, they are
just as necessary as disjunctive principles.  Our problematic
judgements constitute the whole content of time.  By means of
analysis, our ideas are by their very nature contradictory, and our a
posteriori concepts are a representation of natural causes.  I assert
that the objects in space and time would thereby be made to
contradict, so far as regards the thing in itself, the Transcendental
Deduction; in natural theology, the noumena are the clue to the
discovery of, so far as I know, the Transcendental Deduction.}

\kgl_newpara:n {To avoid all misapprehension, it is necessary to
explain that, in respect of the intelligible character, the
transcendental aesthetic depends on the objects in space and time, yet
the manifold is the clue to the discovery of the Transcendental
Deduction.  Therefore, the transcendental unity of apperception would
thereby be made to contradict, in the case of our understanding, our
ideas.  There can be no doubt that the things in themselves prove the
validity of the objects in space and time, as is shown in the writings
of Aristotle.  By means of analysis, there can be no doubt that,
insomuch as the discipline of pure reason relies on the Categories,
the transcendental unity of apperception would thereby be made to
contradict the never-ending regress in the series of empirical
conditions.  In the case of space, the Categories exist in time.  Our
faculties can be treated like our concepts.  As is shown in the
writings of Galileo, the transcendental unity of apperception stands
in need of, in the case of necessity, our speculative judgements.}

\kgl_newpara:n {The phenomena (and it is obvious that this is the
case) prove the validity of our sense perceptions; in natural
theology, philosophy teaches us nothing whatsoever regarding the
content of the transcendental objects in space and time.  In natural
theology, our sense perceptions are a representation of the
Antinomies.  The noumena exclude the possibility of, even as this
relates to the transcendental aesthetic, our knowledge.  Our concepts
would thereby be made to contradict, that is to say, the noumena; in
the study of philosophy, space is by its very nature contradictory.
Since some of the Antinomies are problematic, our ideas are a
representation of our a priori concepts, yet space, in other words,
has lying before it the things in themselves.  Aristotle tells us
that, in accordance with the principles of the phenomena, the
Antinomies are a representation of metaphysics.}

\kgl_newpara:n {The things in themselves can not take account of the
Transcendental Deduction.  By means of analytic unity, it is obvious
that, that is to say, our sense perceptions, in all theoretical
sciences, can not take account of the thing in itself, yet the
transcendental unity of apperception, in the full sense of these
terms, would thereby be made to contradict the employment of our sense
perceptions.  Our synthetic judgements would be falsified.  Since some
of our faculties are problematic, the things in themselves exclude the
possibility of the Ideal.  It must not be supposed that the things in
themselves are a representation of, in accordance with the principles
of philosophy, our sense perceptions.}

\kgl_newpara:n {As is proven in the ontological manuals, philosophy is
the mere result of the power of pure logic, a blind but indispensable
function of the soul; however, the phenomena can never, as a whole,
furnish a true and demonstrated science, because, like general logic,
they exclude the possibility of problematic principles.  To avoid all
misapprehension, it is necessary to explain that the never-ending
regress in the series of empirical conditions is by its very nature
contradictory.  It must not be supposed that our a priori concepts
stand in need to natural causes, because of the relation between the
Ideal and our ideas.  (We can deduce that the Antinomies would be
falsified.)  Since knowledge of the Categories is a posteriori, what
we have alone been able to show is that, in the full sense of these
terms, necessity (and we can deduce that this is true) is the key to
understanding time, but the Ideal of natural reason is just as
necessary as our experience.  As will easily be shown in the next
section, the thing in itself, with the sole exception of the manifold,
abstracts from all content of a posteriori knowledge.  The question of
this matter's relation to objects is not in any way under discussion.}

\kgl_newpara:n {By means of the transcendental aesthetic, it remains a
mystery why the phenomena (and it is not at all certain that this is
the case) are the clue to the discovery of the never-ending regress in
the series of empirical conditions.  In all theoretical sciences,
metaphysics exists in the objects in space and time, because of the
relation between formal logic and our synthetic judgements.  The
Categories would thereby be made to contradict the paralogisms, as any
dedicated reader can clearly see.  Therefore, there can be no doubt
that the paralogisms have nothing to do with, so far as regards the
Ideal and our faculties, the paralogisms, because of our necessary
ignorance of the conditions.  It must not be supposed that the objects
in space and time occupy part of the sphere of necessity concerning
the existence of the noumena in general.  In natural theology, the
things in themselves, therefore, are by their very nature
contradictory, by virtue of natural reason.  This is the sense in
which it is to be understood in this work.}

\kgl_newpara:n {As is evident upon close examination, let us suppose
that, in accordance with the principles of time, our a priori concepts
are the clue to the discovery of philosophy.  By means of analysis, to
avoid all misapprehension, it is necessary to explain that, in
particular, the transcendental aesthetic can not take account of
natural causes.  As we have already seen, the reader should be careful
to observe that, in accordance with the principles of the objects in
space and time, the noumena are the mere results of the power of our
understanding, a blind but indispensable function of the soul, and the
thing in itself abstracts from all content of a posteriori knowledge.
We can deduce that, indeed, our experience, in reference to ends, can
never furnish a true and demonstrated science, because, like the Ideal
of practical reason, it can thereby determine in its totality
speculative principles, yet our hypothetical judgements are just as
necessary as space.  It is not at all certain that, insomuch as the
Ideal of practical reason relies on the noumena, the Categories prove
the validity of philosophy, yet pure reason is the key to
understanding the Categories.  This is what chiefly concerns us.}

\kgl_newpara:n {Natural causes, when thus treated as the things in
themselves, abstract from all content of a posteriori knowledge, by
means of analytic unity.  Our a posteriori knowledge, in other words,
is the key to understanding the Antinomies.  As we have already seen,
what we have alone been able to show is that, so far as I know, the
objects in space and time are the clue to the discovery of the
manifold.  The things in themselves are the clue to the discovery of,
in the case of the Ideal of natural reason, our concepts.  To avoid
all misapprehension, it is necessary to explain that, so far as
regards philosophy, the discipline of human reason, for these reasons,
is a body of demonstrated science, and some of it must be known a
priori, but our faculties, consequently, would thereby be made to
contradict the Antinomies.  It remains a mystery why our understanding
excludes the possibility of, insomuch as the Ideal relies on the
objects in space and time, our concepts.  It is not at all certain
that the pure employment of the objects in space and time (and the
reader should be careful to observe that this is true) is the clue to
the discovery of the architectonic of pure reason.  Let us suppose
that natural reason is a representation of, insomuch as space relies
on the paralogisms, the Transcendental Deduction, by means of
analysis.}

\kgl_newpara:n {As we have already seen, the Ideal constitutes the
whole content for the transcendental unity of apperception.  By means
of analytic unity, let us suppose that, when thus treated as space,
our synthetic judgements, therefore, would be falsified, and the
objects in space and time are what first give rise to our sense
perceptions.  Let us suppose that, in the full sense of these terms,
the discipline of practical reason can not take account of our
experience, and our ideas have lying before them our inductive
judgements.  (Since all of the phenomena are speculative, to avoid all
misapprehension, it is necessary to explain that the noumena
constitute a body of demonstrated doctrine, and some of this body must
be known a posteriori; as I have elsewhere shown, the noumena are a
representation of the noumena.)  Let us suppose that practical reason
can thereby determine in its totality, by means of the Ideal, the pure
employment of the discipline of practical reason.  Galileo tells us
that the employment of the phenomena can be treated like our ideas;
still, the Categories, when thus treated as the paralogisms, exist in
the employment of the Antinomies.  Let us apply this to our
experience.}

\kgl_newpara:n {I assert, thus, that the discipline of natural reason
can be treated like the transcendental aesthetic, since some of the
Categories are speculative.  In the case of transcendental logic, our
ideas prove the validity of our understanding, as any dedicated reader
can clearly see.  In natural theology, our ideas can not take account
of general logic, because of the relation between philosophy and the
noumena.  As is evident upon close examination, natural causes should
only be used as a canon for the manifold, and our faculties, in
natural theology, are a representation of natural causes.  As is shown
in the writings of Aristotle, the Ideal of human reason, for these
reasons, would be falsified.  What we have alone been able to show is
that the Categories, so far as regards philosophy and the Categories,
are the mere results of the power of the Transcendental Deduction, a
blind but indispensable function of the soul, as is proven in the
ontological manuals.}

\kgl_newpara:n {The noumena have nothing to do with, thus, the
Antinomies.  What we have alone been able to show is that the things
in themselves constitute the whole content of human reason, as is
proven in the ontological manuals.  The noumena (and to avoid all
misapprehension, it is necessary to explain that this is the case) are
the clue to the discovery of the architectonic of natural reason.  As
we have already seen, let us suppose that our experience is what first
gives rise to, therefore, the transcendental unity of apperception; in
the study of the practical employment of the Antinomies, our
ampliative judgements are what first give rise to the objects in space
and time.  Necessity can never furnish a true and demonstrated
science, because, like our understanding, it can thereby determine in
its totality hypothetical principles, and the empirical objects in
space and time are what first give rise to, in all theoretical
sciences, our a posteriori concepts.}

\kgl_newpara:n {Our understanding excludes the possibility of
practical reason.  Our faculties stand in need to, consequently, the
never-ending regress in the series of empirical conditions; still, the
employment of necessity is what first gives rise to general logic.
With the sole exception of applied logic, to avoid all
misapprehension, it is necessary to explain that time, in view of
these considerations, can never furnish a true and demonstrated
science, because, like the Ideal of human reason, it is a
representation of ampliative principles, as is evident upon close
examination.  Since knowledge of the paralogisms of natural reason is
a priori, I assert, consequently, that, in so far as this expounds the
practical rules of the thing in itself, the things in themselves
exclude the possibility of the discipline of pure reason, yet the
empirical objects in space and time prove the validity of natural
causes.}

\kgl_newpara:n {Because of the relation between space and the noumena,
our experience is by its very nature contradictory.  It is obvious
that natural causes constitute the whole content of the transcendental
unity of apperception, as any dedicated reader can clearly see.  By
virtue of pure reason, our sense perceptions, in all theoretical
sciences, have lying before them human reason.  In view of these
considerations, let us suppose that the transcendental objects in
space and time, in the study of the architectonic of practical reason,
exclude the possibility of the objects in space and time, because of
our necessary ignorance of the conditions.  By means of philosophy, is
it true that formal logic can not take account of the manifold, or is
the real question whether our sense perceptions are the mere results
of the power of the transcendental aesthetic, a blind but
indispensable function of the soul?  The objects in space and time are
just as necessary as the Antinomies, because of the relation between
metaphysics and the things in themselves.  Human reason is a
representation of the transcendental aesthetic.  In my present remarks
I am referring to the pure employment of our disjunctive judgements
only in so far as it is founded on inductive principles.}

\kgl_newpara:n {What we have alone been able to show is that our sense
perceptions are the clue to the discovery of our understanding; in
natural theology, necessity, in all theoretical sciences, occupies
part of the sphere of the transcendental unity of apperception
concerning the existence of our faculties in general.  The
transcendental aesthetic is what first gives rise to the never-ending
regress in the series of empirical conditions, as any dedicated reader
can clearly see.  The transcendental unity of apperception is what
first gives rise to, in all theoretical sciences, the Antinomies.  The
phenomena, consequently, stand in need to the things in themselves.
By means of analytic unity, necessity, on the contrary, abstracts from
all content of a priori knowledge.  The phenomena (and it remains a
mystery why this is the case) are just as necessary as the Ideal of
human reason.}

\kgl_newpara:n {As any dedicated reader can clearly see, our
experience is the clue to the discovery of philosophy; in the study of
space, the Categories are what first give rise to the transcendental
aesthetic.  As any dedicated reader can clearly see, the reader should
be careful to observe that, so regarded, the never-ending regress in
the series of empirical conditions, as I have elsewhere shown, is the
mere result of the power of the transcendental unity of apperception,
a blind but indispensable function of the soul, but our judgements can
be treated like time.  We can deduce that the objects in space and
time are just as necessary as the objects in space and time.
Aristotle tells us that, even as this relates to time, the objects in
space and time, however, abstract from all content of a posteriori
knowledge.  To avoid all misapprehension, it is necessary to explain
that the phenomena (and it is not at all certain that this is the
case) stand in need to the discipline of practical reason; thus, our
knowledge, indeed, can not take account of our ideas.}

\kgl_newpara:n {In the study of time, our concepts prove the validity
of, as I have elsewhere shown, our understanding, as any dedicated
reader can clearly see.  As will easily be shown in the next section,
the reader should be careful to observe that, so far as regards our
knowledge, natural causes, so far as regards the never-ending regress
in the series of empirical conditions and our a priori judgements,
should only be used as a canon for the pure employment of the
Transcendental Deduction, and our understanding can not take account
of formal logic.  As any dedicated reader can clearly see, to avoid
all misapprehension, it is necessary to explain that the Antinomies
are just as necessary as, on the other hand, our ideas; however, the
Ideal, in the full sense of these terms, exists in the architectonic
of human reason.  As is evident upon close examination, to avoid all
misapprehension, it is necessary to explain that, in other words, our
faculties have nothing to do with the manifold, but our faculties
should only be used as a canon for space.  Our faculties prove the
validity of the Antinomies, and the things in themselves (and let us
suppose that this is the case) are the clue to the discovery of our
ideas.  It remains a mystery why, then, the architectonic of practical
reason proves the validity of, therefore, the noumena.}

\kgl_newpara:n {The paralogisms of practical reason can be treated
like the paralogisms.  The objects in space and time, therefore, are
what first give rise to the discipline of human reason; in all
theoretical sciences, the things in themselves (and we can deduce that
this is the case) have nothing to do with metaphysics.  Therefore,
Aristotle tells us that our understanding exists in the Ideal of human
reason, as is proven in the ontological manuals.  Thus, our sense
perceptions (and it remains a mystery why this is the case) would
thereby be made to contradict space.  I assert, on the other hand,
that, in reference to ends, the objects in space and time can not take
account of the Categories, yet natural causes are the mere results of
the power of the discipline of human reason, a blind but indispensable
function of the soul.  By virtue of practical reason, it must not be
supposed that, that is to say, our faculties would thereby be made to
contradict philosophy, yet our a posteriori concepts, insomuch as the
Ideal of pure reason relies on the intelligible objects in space and
time, are by their very nature contradictory.}

\kgl_newpara:n {Time, on the contrary, can never furnish a true and
demonstrated science, because, like the transcendental aesthetic, it
constitutes the whole content for ampliative principles, yet natural
reason, even as this relates to philosophy, proves the validity of the
thing in itself.  As is evident upon close examination, the Ideal of
practical reason, when thus treated as the things in themselves, is by
its very nature contradictory; as I have elsewhere shown, our
understanding may not contradict itself, but it is still possible that
it may be in contradictions with the Ideal of practical reason.  Since
all of the things in themselves are problematic, it remains a mystery
why, so regarded, our knowledge is the key to understanding our
problematic judgements, but our ideas (and to avoid all
misapprehension, it is necessary to explain that this is the case)
have lying before them our disjunctive judgements.  In the case of the
Ideal, we can deduce that the transcendental unity of apperception
excludes the possibility of the manifold, as we have already seen.
Consequently, the Ideal of pure reason can be treated like the
phenomena.  Let us apply this to the Transcendental Deduction.}

\kgl_newpara:n {What we have alone been able to show is that our a
posteriori concepts (and it is obvious that this is the case) are what
first give rise to the transcendental unity of apperception.  In the
case of necessity, the reader should be careful to observe that
metaphysics is a representation of natural causes, by means of
analysis.  In all theoretical sciences, the phenomena (and the reader
should be careful to observe that this is the case) would thereby be
made to contradict natural reason.  The transcendental aesthetic, in
the case of space, is by its very nature contradictory.  By virtue of
human reason, to avoid all misapprehension, it is necessary to explain
that the empirical objects in space and time exist in our judgements;
for these reasons, the Antinomies, by means of our experience, can be
treated like the architectonic of human reason.  It must not be
supposed that our ideas have lying before them metaphysics;
consequently, the architectonic of pure reason, in all theoretical
sciences, would be falsified.}

\kgl_newpara:n {The Transcendental Deduction stands in need of the
Ideal of pure reason, and the noumena, for these reasons, are by their
very nature contradictory.  The objects in space and time have lying
before them our ideas.  The transcendental unity of apperception,
indeed, proves the validity of our understanding.  The architectonic
of human reason, so regarded, would be falsified, as is evident upon
close examination.  Since knowledge of the noumena is a priori, Hume
tells us that, then, the Transcendental Deduction, when thus treated
as the architectonic of natural reason, abstracts from all content of
knowledge, but the objects in space and time, for these reasons, stand
in need to the transcendental aesthetic.  By means of analytic unity,
natural causes exclude the possibility of, consequently, metaphysics,
and the discipline of pure reason abstracts from all content of a
priori knowledge.  We thus have a pure synthesis of apprehension.}

\kgl_newpara:n {Because of our necessary ignorance of the conditions,
what we have alone been able to show is that formal logic can not take
account of the Categories; in the study of the transcendental
aesthetic, philosophy can thereby determine in its totality the
noumena.  In all theoretical sciences, I assert that necessity has
nothing to do with our sense perceptions.  Because of the relation
between our understanding and the phenomena, the Categories are what
first give rise to, so far as regards time and the phenomena, the
transcendental aesthetic; in view of these considerations, the
phenomena can not take account of the Antinomies.  As is proven in the
ontological manuals, the objects in space and time (and to avoid all
misapprehension, it is necessary to explain that this is the case) are
what first give rise to the Ideal.  In natural theology, let us
suppose that the Transcendental Deduction is the key to understanding,
so far as regards the thing in itself, the Ideal, as any dedicated
reader can clearly see.  This is the sense in which it is to be
understood in this work.}

\kgl_newpara:n {It must not be supposed that, in respect of the
intelligible character, the Antinomies (and we can deduce that this is
the case) constitute the whole content of the phenomena, yet the
Categories exist in natural causes.  The Ideal of natural reason, when
thus treated as metaphysics, can be treated like our faculties;
consequently, pure reason (and there can be no doubt that this is
true) is what first gives rise to our sense perceptions.  The
paralogisms of practical reason exist in the objects in space and
time.  As we have already seen, our sense perceptions stand in need to
space.  Still, our a priori concepts, in the case of metaphysics, have
nothing to do with the Categories.  Because of the relation between
the discipline of practical reason and our a posteriori concepts, we
can deduce that, when thus treated as the phenomena, our sense
perceptions (and there can be no doubt that this is the case) are what
first give rise to the discipline of practical reason.}

\kgl_newpara:n {Thus, the reader should be careful to observe that the
noumena would thereby be made to contradict necessity, because of our
necessary ignorance of the conditions.  Consequently, our sense
perceptions are just as necessary as the architectonic of natural
reason, as is shown in the writings of Galileo.  It remains a mystery
why, when thus treated as human reason, our concepts, when thus
treated as the Categories, can never, as a whole, furnish a true and
demonstrated science, because, like the Ideal, they are just as
necessary as synthetic principles, yet our sense perceptions would be
falsified.  The noumena, in all theoretical sciences, can not take
account of space, as is proven in the ontological manuals.  Since
knowledge of our analytic judgements is a priori, to avoid all
misapprehension, it is necessary to explain that the paralogisms
constitute a body of demonstrated doctrine, and none of this body must
be known a priori; in view of these considerations, the phenomena can
not take account of, for these reasons, the transcendental unity of
apperception.}

\kgl_newpara:n {The reader should be careful to observe that, for
example, pure logic depends on the transcendental unity of
apperception.  As any dedicated reader can clearly see, our a priori
concepts are what first give rise to the Categories.  Hume tells us
that our ideas are just as necessary as, on the other hand, natural
causes; however, natural causes should only be used as a canon for our
faculties.  For these reasons, to avoid all misapprehension, it is
necessary to explain that our ideas are the clue to the discovery of
our understanding, as is shown in the writings of Hume.  (By virtue of
natural reason, the employment of our disjunctive judgements, then, is
by its very nature contradictory.)  By virtue of natural reason, the
Categories can not take account of our hypothetical judgements.  The
transcendental aesthetic teaches us nothing whatsoever regarding the
content of, consequently, the transcendental unity of apperception, as
will easily be shown in the next section.  We thus have a pure
synthesis of apprehension.}

\kgl_newpara:n {The Antinomies have nothing to do with our faculties.
As is shown in the writings of Hume, we can deduce that, on the
contrary, the empirical objects in space and time prove the validity
of our ideas.  The manifold may not contradict itself, but it is still
possible that it may be in contradictions with our a posteriori
concepts.  For these reasons, the transcendental objects in space and
time (and it is obvious that this is the case) have nothing to do with
our faculties, as will easily be shown in the next section.  What we
have alone been able to show is that the phenomena constitute the
whole content of the Antinomies; with the sole exception of
philosophy, the Categories have lying before them formal logic.  Since
knowledge of the Antinomies is a posteriori, it remains a mystery why
the Antinomies (and there can be no doubt that this is the case) prove
the validity of the thing in itself; for these reasons, metaphysics is
the mere result of the power of the employment of our sense
perceptions, a blind but indispensable function of the soul.  As I
have elsewhere shown, philosophy proves the validity of our sense
perceptions.}

\kgl_newpara:n {What we have alone been able to show is that the
phenomena, so far as I know, exist in the noumena; however, our
concepts, however, exclude the possibility of our judgements.  Galileo
tells us that our a posteriori knowledge would thereby be made to
contradict transcendental logic; in the case of philosophy, our
judgements stand in need to applied logic.  On the other hand, to
avoid all misapprehension, it is necessary to explain that the objects
in space and time exclude the possibility of, insomuch as pure logic
relies on the objects in space and time, the transcendental unity of
apperception, by virtue of practical reason.  Has it ever been
suggested that, as will easily be shown in the next section, the
reader should be careful to observe that there is a causal connection
bewteen philosophy and pure reason?  In natural theology, it remains a
mystery why the discipline of natural reason is a body of demonstrated
science, and some of it must be known a posteriori, as will easily be
shown in the next section.  In view of these considerations, let us
suppose that our sense perceptions, then, would be falsified, because
of the relation between the never-ending regress in the series of
empirical conditions and the paralogisms.  This distinction must have
some ground in the nature of the never-ending regress in the series of
empirical conditions.}

\kgl_newpara:n {To avoid all misapprehension, it is necessary to
explain that time excludes the possibility of the discipline of human
reason; in the study of practical reason, the manifold has nothing to
do with time.  Because of the relation between our a priori knowledge
and the phenomena, what we have alone been able to show is that our
experience is what first gives rise to the phenomena; thus, natural
causes are the clue to the discovery of, with the sole exception of
our experience, the objects in space and time.  Our ideas are what
first give rise to our faculties.  On the other hand, the phenomena
have lying before them our ideas, as is evident upon close
examination.  The paralogisms of natural reason are a representation
of, thus, the manifold.  I assert that space is what first gives rise
to the paralogisms of pure reason.  As is shown in the writings of
Hume, space has nothing to do with, for example, necessity.}

\kgl_newpara:n {We can deduce that the Ideal of practical reason, even
as this relates to our knowledge, is a representation of the
discipline of human reason.  The things in themselves are just as
necessary as our understanding.  The noumena prove the validity of the
manifold.  As will easily be shown in the next section, natural causes
occupy part of the sphere of our a priori knowledge concerning the
existence of the Antinomies in general.  The Categories are the clue
to the discovery of, consequently, the Transcendental Deduction.  Our
ideas are the mere results of the power of the Ideal of pure reason, a
blind but indispensable function of the soul.  The divisions are thus
provided; all that is required is to fill them.}

\kgl_newpara:n {The never-ending regress in the series of empirical
conditions can be treated like the objects in space and time.  What we
have alone been able to show is that, then, the transcendental
aesthetic, in reference to ends, would thereby be made to contradict
the Transcendental Deduction.  The architectonic of practical reason
has nothing to do with our ideas; however, time can never furnish a
true and demonstrated science, because, like the Ideal, it depends on
hypothetical principles.  Space has nothing to do with the Antinomies,
because of our necessary ignorance of the conditions.  In all
theoretical sciences, to avoid all misapprehension, it is necessary to
explain that the things in themselves are a representation of, in
other words, necessity, as is evident upon close examination.}

\kgl_newpara:n {As is proven in the ontological manuals, it remains a
mystery why our experience is the mere result of the power of the
discipline of human reason, a blind but indispensable function of the
soul.  For these reasons, the employment of the thing in itself
teaches us nothing whatsoever regarding the content of the Ideal of
natural reason.  In the case of transcendental logic, there can be no
doubt that the Ideal of practical reason is just as necessary as the
Antinomies.  I assert that, insomuch as the Ideal relies on the
noumena, the empirical objects in space and time stand in need to our
a priori concepts.  (It must not be supposed that, so regarded, our
ideas exclude the possibility of, in the case of the Ideal, the
architectonic of human reason.)  The reader should be careful to
observe that, irrespective of all empirical conditions, our concepts
are what first give rise to our experience.  By means of analytic
unity, our faculties, in so far as this expounds the contradictory
rules of the objects in space and time, are the mere results of the
power of space, a blind but indispensable function of the soul, and
the transcendental unity of apperception can not take account of,
however, our faculties.  But at present we shall turn our attention to
the thing in itself.}

\kgl_newpara:n {As is evident upon close examination, we can deduce
that the transcendental unity of apperception depends on the Ideal of
practical reason.  Certainly, it is obvious that the Antinomies, in
accordance with the principles of the objects in space and time,
constitute a body of demonstrated doctrine, and none of this body must
be known a posteriori.  Because of the relation between the discipline
of pure reason and our a posteriori concepts, I assert that, for
example, metaphysics, consequently, is by its very nature
contradictory, yet the transcendental aesthetic is the key to
understanding our understanding.  By virtue of natural reason, the
objects in space and time are what first give rise to, when thus
treated as the paralogisms of human reason, the things in themselves,
but the never-ending regress in the series of empirical conditions can
not take account of the architectonic of human reason.  What we have
alone been able to show is that natural causes, irrespective of all
empirical conditions, exist in the objects in space and time, as is
shown in the writings of Hume.  By virtue of practical reason, our
sense perceptions are what first give rise to, irrespective of all
empirical conditions, necessity.  Our sense perceptions, in the study
of necessity, would thereby be made to contradict transcendental
logic; consequently, natural reason stands in need of the objects in
space and time.  There can be no doubt that, in other words, the
paralogisms of natural reason have nothing to do with the thing in
itself, but the paralogisms prove the validity of transcendental
logic.}

\kgl_newpara:n {We can deduce that, then, the noumena are just as
necessary as, so regarded, the practical employment of the objects in
space and time.  It is obvious that the manifold has nothing to do
with our ideas; with the sole exception of the employment of the
noumena, natural reason, in natural theology, is the mere result of
the power of time, a blind but indispensable function of the soul.
Because of the relation between our understanding and the things in
themselves, it is not at all certain that, so far as regards the
transcendental unity of apperception and the paralogisms, the
phenomena can not take account of, so regarded, our sense perceptions,
yet our sense perceptions can never, as a whole, furnish a true and
demonstrated science, because, like time, they constitute the whole
content of analytic principles.  Since knowledge of our sense
perceptions is a posteriori, it is obvious that, in accordance with
the principles of our faculties, metaphysics excludes the possibility
of the manifold, and the Ideal may not contradict itself, but it is
still possible that it may be in contradictions with, thus, our sense
perceptions.  To avoid all misapprehension, it is necessary to explain
that our ideas exclude the possibility of, irrespective of all
empirical conditions, our ideas.  Let us apply this to space.}

\kgl_newpara:n {It remains a mystery why our sense perceptions prove
the validity of our a priori concepts.  The objects in space and time,
then, exist in metaphysics; therefore, the things in themselves can
not take account of the transcendental aesthetic.  The Ideal of pure
reason can thereby determine in its totality, that is to say, our
ideas, and space constitutes the whole content for the discipline of
human reason.  The paralogisms of pure reason are just as necessary
as, in all theoretical sciences, our knowledge.  The things in
themselves constitute a body of demonstrated doctrine, and some of
this body must be known a posteriori.}

\kgl_newpara:n {As will easily be shown in the next section, the
Transcendental Deduction exists in the Ideal.  To avoid all
misapprehension, it is necessary to explain that pure reason (and it
is obvious that this is true) is the key to understanding the
transcendental unity of apperception.  The reader should be careful to
observe that our experience depends on necessity.  It is obvious that
space, thus, can be treated like the objects in space and time,
because of the relation between the transcendental unity of
apperception and the objects in space and time.  It must not be
supposed that, even as this relates to natural reason, the Antinomies
(and it remains a mystery why this is the case) exclude the
possibility of the empirical objects in space and time, yet philosophy
proves the validity of practical reason.  The things in themselves, on
the contrary, abstract from all content of a posteriori knowledge; in
all theoretical sciences, the noumena (and there can be no doubt that
this is the case) are just as necessary as the Antinomies.  As is
shown in the writings of Galileo, I assert, in natural theology, that
the transcendental aesthetic, thus, exists in our faculties.  Our
faculties are just as necessary as the Categories, yet the manifold
has lying before it, certainly, our understanding.}

\kgl_newpara:n {It is obvious that the never-ending regress in the
series of empirical conditions may not contradict itself, but it is
still possible that it may be in contradictions with the architectonic
of practical reason.  The objects in space and time, so regarded,
should only be used as a canon for the architectonic of human reason,
as is proven in the ontological manuals.  In all theoretical sciences,
the Antinomies can not take account of our concepts, because of our
necessary ignorance of the conditions.  By means of analysis, the
things in themselves are a representation of our experience; for these
reasons, the paralogisms of practical reason have lying before them
our inductive judgements.  Still, the architectonic of pure reason is
just as necessary as the never-ending regress in the series of
empirical conditions.}

\kgl_newpara:n {Thus, transcendental logic (and I assert, for these
reasons, that this is true) depends on the Antinomies.  Still, general
logic (and it remains a mystery why this is true) is what first gives
rise to the objects in space and time, because of the relation between
metaphysics and the Antinomies.  As will easily be shown in the next
section, the paralogisms constitute a body of demonstrated doctrine,
and some of this body must be known a priori.  On the other hand, the
never-ending regress in the series of empirical conditions, in the
case of the Transcendental Deduction, exists in the noumena, as is
proven in the ontological manuals.  By means of analytic unity, it
remains a mystery why our judgements are by their very nature
contradictory; however, the objects in space and time exclude the
possibility of the Categories.  As any dedicated reader can clearly
see, the Antinomies would thereby be made to contradict the
transcendental aesthetic; in natural theology, our faculties
constitute the whole content of, for these reasons, the noumena.
However, the objects in space and time are what first give rise to our
understanding, because of our necessary ignorance of the conditions.}

\kgl_newpara:n {On the other hand, the Antinomies have nothing to do
with pure reason, because of our necessary ignorance of the
conditions.  Our speculative judgements are what first give rise to
the Categories.  Time is the key to understanding natural causes, as
is evident upon close examination.  Galileo tells us that the objects
in space and time, irrespective of all empirical conditions, should
only be used as a canon for our sense perceptions, since knowledge of
the noumena is a priori.  I assert that the Transcendental Deduction
depends on our concepts.  By means of analytic unity, our sense
perceptions constitute the whole content of the manifold.  In natural
theology, the discipline of natural reason, on the other hand, would
be falsified, as any dedicated reader can clearly see.}

\kgl_newpara:n {In the case of the discipline of human reason, it is
obvious that the phenomena, still, are the mere results of the power
of the practical employment of the Transcendental Deduction, a blind
but indispensable function of the soul, by means of analysis.  As any
dedicated reader can clearly see, Aristotle tells us that natural
causes constitute the whole content of, as I have elsewhere shown, the
pure employment of the paralogisms.  Aristotle tells us that,
irrespective of all empirical conditions, the thing in itself, indeed,
can never furnish a true and demonstrated science, because, like the
architectonic of practical reason, it has lying before it analytic
principles, yet the Categories have nothing to do with the objects in
space and time.  Because of our necessary ignorance of the conditions,
human reason is just as necessary as our concepts, yet the practical
employment of the paralogisms is the mere result of the power of
metaphysics, a blind but indispensable function of the soul.  For
these reasons, Hume tells us that natural causes have nothing to do
with the transcendental unity of apperception, by means of analytic
unity.  The Antinomies can not take account of the Antinomies, because
of our necessary ignorance of the conditions.  I assert, in all
theoretical sciences, that, that is to say, natural causes would
thereby be made to contradict, so regarded, the Ideal of natural
reason.  Hume tells us that our ideas abstract from all content of a
posteriori knowledge, as is evident upon close examination.}

\kgl_newpara:n {The manifold is a representation of the phenomena.
Our judgements constitute the whole content of, on the other hand, the
things in themselves, as will easily be shown in the next section.  By
means of analytic unity, the phenomena, in the full sense of these
terms, should only be used as a canon for the Ideal of human reason.
It is obvious that, so far as regards metaphysics and our judgements,
pure reason (and there can be no doubt that this is true) is the key
to understanding time.  In the study of formal logic, the paralogisms
of pure reason are the clue to the discovery of, thus, the manifold.}

\kgl_newpara:n {There can be no doubt that the never-ending regress in
the series of empirical conditions may not contradict itself, but it
is still possible that it may be in contradictions with, indeed, our
sense perceptions.  As is proven in the ontological manuals, the
architectonic of practical reason proves the validity of, in all
theoretical sciences, metaphysics; in view of these considerations,
our knowledge depends on our faculties.  Since knowledge of our sense
perceptions is a priori, to avoid all misapprehension, it is necessary
to explain that natural reason is what first gives rise to our
faculties.  There can be no doubt that, in the full sense of these
terms, the Antinomies exclude the possibility of the Transcendental
Deduction.  (In view of these considerations, the empirical objects in
space and time are by their very nature contradictory.)  It is obvious
that the objects in space and time can not take account of the
transcendental objects in space and time, as is proven in the
ontological manuals.  As is evident upon close examination, what we
have alone been able to show is that the objects in space and time are
the mere results of the power of time, a blind but indispensable
function of the soul.  The divisions are thus provided; all that is
required is to fill them.}

\kgl_newpara:n {As we have already seen, the Antinomies are a
representation of the Categories.  Necessity stands in need of the
Antinomies.  By virtue of natural reason, the Antinomies have lying
before them the Ideal of pure reason; on the other hand, the
Antinomies have nothing to do with natural causes.  As I have
elsewhere shown, the reader should be careful to observe that the
things in themselves would thereby be made to contradict, in so far as
this expounds the universal rules of our faculties, our ideas.  I
assert that, in so far as this expounds the necessary rules of human
reason, our concepts (and we can deduce that this is the case) prove
the validity of space, but our sense perceptions, so far as regards
the transcendental unity of apperception, can never, as a whole,
furnish a true and demonstrated science, because, like the
never-ending regress in the series of empirical conditions, they have
nothing to do with disjunctive principles.  But we have fallen short
of the necessary interconnection that we have in mind when we speak of
necessity.}

\kgl_newpara:n {As is evident upon close examination, the paralogisms
abstract from all content of a posteriori knowledge.  Consequently,
the transcendental aesthetic, in reference to ends, occupies part of
the sphere of metaphysics concerning the existence of the Categories
in general.  The objects in space and time, in particular, constitute
a body of demonstrated doctrine, and all of this body must be known a
posteriori; by means of the thing in itself, the noumena can be
treated like the thing in itself.  The things in themselves, for
example, are the mere results of the power of philosophy, a blind but
indispensable function of the soul, as is shown in the writings of
Aristotle.  As will easily be shown in the next section, it must not
be supposed that, in the full sense of these terms, our faculties, in
view of these considerations, constitute the whole content of the
objects in space and time, and our sense perceptions, in respect of
the intelligible character, can be treated like space.  Because of our
necessary ignorance of the conditions, Hume tells us that the
manifold, irrespective of all empirical conditions, is what first
gives rise to space.}

\kgl_newpara:n {In view of these considerations, our experience
occupies part of the sphere of the Ideal concerning the existence of
the objects in space and time in general, as will easily be shown in
the next section.  It must not be supposed that our ideas (and it
remains a mystery why this is the case) are a representation of the
intelligible objects in space and time.  Consequently, the
Transcendental Deduction can thereby determine in its totality, in
other words, our ideas, because of our necessary ignorance of the
conditions.  (In natural theology, our concepts abstract from all
content of a priori knowledge, as is proven in the ontological
manuals.)  I assert, in the case of the manifold, that human reason is
a body of demonstrated science, and all of it must be known a
posteriori, by virtue of human reason.  As is proven in the
ontological manuals, Aristotle tells us that the thing in itself, so
far as I know, can never furnish a true and demonstrated science,
because, like the architectonic of pure reason, it is just as
necessary as a priori principles.}

\kgl_newpara:n {To avoid all misapprehension, it is necessary to
explain that philosophy can not take account of our sense perceptions;
in the study of the discipline of natural reason, our experience, in
the study of the architectonic of practical reason, is the mere result
of the power of pure logic, a blind but indispensable function of the
soul.  As is evident upon close examination, the noumena are what
first give rise to, on the contrary, the phenomena, but natural
reason, that is to say, excludes the possibility of our hypothetical
judgements.  The objects in space and time are the clue to the
discovery of the thing in itself, because of our necessary ignorance
of the conditions.  Therefore, there can be no doubt that the
architectonic of practical reason depends on the Antinomies, because
of our necessary ignorance of the conditions.  Human reason (and there
can be no doubt that this is true) depends on our understanding, but
the Ideal can thereby determine in its totality metaphysics.}

\kgl_newpara:n {Since knowledge of the objects in space and time is a
posteriori, general logic, in respect of the intelligible character,
is by its very nature contradictory.  By means of analytic unity, it
is not at all certain that space, insomuch as our understanding relies
on our sense perceptions, would thereby be made to contradict the
Ideal.  By virtue of natural reason, the Antinomies are just as
necessary as, indeed, the thing in itself.  The manifold, as I have
elsewhere shown, is a body of demonstrated science, and some of it
must be known a priori.  There can be no doubt that, in particular,
the phenomena are a representation of pure logic, yet our sense
perceptions have lying before them our sense perceptions.  I assert,
as I have elsewhere shown, that, indeed, our experience (and let us
suppose that this is true) excludes the possibility of the objects in
space and time, and the discipline of human reason, in accordance with
the principles of the transcendental unity of apperception, occupies
part of the sphere of our understanding concerning the existence of
the phenomena in general.}

\kgl_newpara:n {Human reason (and we can deduce that this is true)
proves the validity of the architectonic of natural reason.  To avoid
all misapprehension, it is necessary to explain that the employment of
the things in themselves can not take account of the phenomena.  The
transcendental aesthetic, on the contrary, can be treated like the
never-ending regress in the series of empirical conditions; certainly,
our faculties constitute the whole content of, in particular, the
never-ending regress in the series of empirical conditions.  What we
have alone been able to show is that, then, the objects in space and
time stand in need to metaphysics, and our experience, in accordance
with the principles of time, stands in need of the never-ending
regress in the series of empirical conditions.  Since knowledge of our
ideas is a posteriori, the phenomena are a representation of the
phenomena.}

\kgl_newpara:n {Necessity, as I have elsewhere shown, is the mere
result of the power of the architectonic of practical reason, a blind
but indispensable function of the soul.  The paralogisms of pure
reason are the clue to the discovery of the practical employment of
the thing in itself.  There can be no doubt that the never-ending
regress in the series of empirical conditions has lying before it the
paralogisms of human reason; with the sole exception of the
architectonic of pure reason, transcendental logic is just as
necessary as, then, our judgements.  What we have alone been able to
show is that our synthetic judgements have lying before them, when
thus treated as space, our knowledge, by means of analysis.  By virtue
of natural reason, the transcendental aesthetic can be treated like
general logic, yet the objects in space and time are just as necessary
as the noumena.  }

\kgl_newpara:n {In view of these considerations, let us suppose that
the Categories exclude the possibility of the never-ending regress in
the series of empirical conditions.  The manifold occupies part of the
sphere of the thing in itself concerning the existence of the things
in themselves in general, and formal logic, indeed, would be
falsified.  It is not at all certain that, in reference to ends, the
discipline of practical reason, for example, occupies part of the
sphere of the discipline of practical reason concerning the existence
of our ampliative judgements in general, yet general logic is by its
very nature contradictory.  Since all of our judgements are a priori,
there can be no doubt that, in the full sense of these terms, the
phenomena can not take account of the transcendental objects in space
and time.  The architectonic of pure reason (and it is not at all
certain that this is true) stands in need of the things in themselves.
Philosophy is the key to understanding, thus, our sense perceptions.
This is what chiefly concerns us.}

\kgl_newpara:n {Our understanding would thereby be made to contradict,
so far as regards the Ideal, necessity.  Our faculties, as I have
elsewhere shown, are the mere results of the power of time, a blind
but indispensable function of the soul.  Time, with the sole exception
of formal logic, would be falsified, but the Ideal can not take
account of our sense perceptions.  It is not at all certain that the
Antinomies are what first give rise to our experience; thus, our a
posteriori concepts are the clue to the discovery of, so regarded, the
practical employment of the Transcendental Deduction.  Natural causes
occupy part of the sphere of practical reason concerning the existence
of the paralogisms of pure reason in general; in view of these
considerations, the noumena exclude the possibility of the employment
of the objects in space and time.  The manifold is what first gives
rise to the paralogisms, but our judgements are the clue to the
discovery of, in the study of the thing in itself, the discipline of
practical reason.}

\kgl_newpara:n {Our a priori concepts, with the sole exception of our
experience, have lying before them our judgements.  It must not be
supposed that the Antinomies are a representation of the discipline of
human reason, by means of analytic unity.  In the study of the
transcendental aesthetic, the paralogisms constitute a body of
demonstrated doctrine, and some of this body must be known a
posteriori.  The Categories are the mere results of the power of the
thing in itself, a blind but indispensable function of the soul.
Because of the relation between pure reason and the paralogisms of
human reason, to avoid all misapprehension, it is necessary to explain
that, indeed, the objects in space and time (and to avoid all
misapprehension, it is necessary to explain that this is the case) are
a representation of our concepts, yet the Ideal can be treated like
our inductive judgements.  As is proven in the ontological manuals,
our understanding would thereby be made to contradict, thus, the
Transcendental Deduction; as I have elsewhere shown, the phenomena
abstract from all content of knowledge.  The thing in itself excludes
the possibility of philosophy; therefore, space, for example, teaches
us nothing whatsoever regarding the content of metaphysics.  We can
deduce that the noumena (and it must not be supposed that this is the
case) are a representation of the transcendental unity of
apperception; with the sole exception of the thing in itself, our
sense perceptions, as I have elsewhere shown, can never, as a whole,
furnish a true and demonstrated science, because, like the
transcendental unity of apperception, they exclude the possibility of
hypothetical principles.}

\kgl_newpara:n {Since none of our faculties are speculative, our ideas
should only be used as a canon for time.  With the sole exception of
the manifold, our concepts exclude the possibility of the practical
employment of metaphysics, by means of analysis.  Aristotle tells us
that necessity (and it is obvious that this is true) would thereby be
made to contradict the thing in itself, because of our necessary
ignorance of the conditions.  As is proven in the ontological manuals,
metaphysics (and it remains a mystery why this is true) can thereby
determine in its totality the Ideal.  In the study of the
transcendental unity of apperception, it is obvious that the phenomena
have nothing to do with, therefore, natural causes, by means of
analysis.  Has it ever been suggested that it must not be supposed
that there is no relation bewteen the paralogisms of practical reason
and the Antinomies?  Time, indeed, is a representation of the
Antinomies.  The paralogisms of human reason are the clue to the
discovery of natural causes, by means of analysis.  Let us suppose
that, in other words, the manifold, that is to say, abstracts from all
content of knowledge.}

\kgl_newpara:n {As is proven in the ontological manuals, Aristotle
tells us that the transcendental unity of apperception can be treated
like the discipline of pure reason; in the case of our understanding,
our sense perceptions are just as necessary as the noumena.  The
reader should be careful to observe that the discipline of human
reason occupies part of the sphere of our understanding concerning the
existence of natural causes in general.  The noumena prove the
validity of philosophy, and the paralogisms of human reason exclude
the possibility of our sense perceptions.  Our faculties exist in our
a posteriori concepts; still, the never-ending regress in the series
of empirical conditions has lying before it necessity.  Since
knowledge of our sense perceptions is a posteriori, the transcendental
aesthetic can never furnish a true and demonstrated science, because,
like the transcendental aesthetic, it has nothing to do with
ampliative principles.  Transcendental logic exists in our faculties.}

\kgl_newpara:n {There can be no doubt that the objects in space and
time have nothing to do with our judgements.  The architectonic of
human reason has nothing to do with the noumena.  What we have alone
been able to show is that natural causes have nothing to do with,
still, our a priori concepts, as we have already seen.  As any
dedicated reader can clearly see, it remains a mystery why, for
example, our ideas, with the sole exception of the thing in itself,
can not take account of the objects in space and time.  It remains a
mystery why our faculties are a representation of the transcendental
aesthetic.  Our ideas, in reference to ends, can never, as a whole,
furnish a true and demonstrated science, because, like the discipline
of natural reason, they are a representation of synthetic principles.
The transcendental unity of apperception is just as necessary as, in
view of these considerations, our ampliative judgements; with the sole
exception of the transcendental aesthetic, the thing in itself (and it
remains a mystery why this is true) is the clue to the discovery of
our speculative judgements.}

\kgl_newpara:n {As I have elsewhere shown, the Ideal is a body of
demonstrated science, and some of it must be known a priori, as is
evident upon close examination.  Our ideas abstract from all content
of knowledge, and the phenomena have nothing to do with, then,
necessity.  As is proven in the ontological manuals, the empirical
objects in space and time exclude the possibility of, in other words,
our sense perceptions.  It must not be supposed that, then, the
never-ending regress in the series of empirical conditions stands in
need of, certainly, the Ideal of natural reason, yet pure reason can
not take account of the objects in space and time.  The noumena, in
all theoretical sciences, prove the validity of the practical
employment of the manifold; in natural theology, the phenomena are
just as necessary as the paralogisms.  It is not at all certain that
our concepts have lying before them our faculties, by means of
analytic unity.  It is not at all certain that the architectonic of
practical reason, then, is what first gives rise to necessity; still,
our concepts stand in need to the objects in space and time.}

\kgl_newpara:n {It must not be supposed that our sense perceptions are
the clue to the discovery of the Antinomies.  As will easily be shown
in the next section, our experience, in particular, excludes the
possibility of natural causes, yet the architectonic of human reason
can never furnish a true and demonstrated science, because, like
philosophy, it can thereby determine in its totality problematic
principles.  Let us suppose that, even as this relates to philosophy,
our a posteriori concepts, in view of these considerations, exist in
natural causes, yet space may not contradict itself, but it is still
possible that it may be in contradictions with the Categories.  (The
thing in itself, in all theoretical sciences, exists in our ideas.)
Because of our necessary ignorance of the conditions, let us suppose
that the things in themselves should only be used as a canon for the
things in themselves; certainly, our ideas, therefore, abstract from
all content of a priori knowledge.  Necessity constitutes the whole
content for practical reason.  But we have fallen short of the
necessary interconnection that we have in mind when we speak of the
transcendental aesthetic.  }

\kgl_newpara:n {As we have already seen, Aristotle tells us that, when
thus treated as the phenomena, the transcendental unity of
apperception can thereby determine in its totality the Ideal of human
reason.  There can be no doubt that natural causes can not take
account of, certainly, the phenomena, since none of the paralogisms
are hypothetical.  We can deduce that the transcendental aesthetic is
a body of demonstrated science, and none of it must be known a priori.
Hume tells us that, for example, our a posteriori knowledge
constitutes the whole content for our sense perceptions, yet the
discipline of pure reason, when thus treated as our understanding,
constitutes the whole content for the empirical objects in space and
time.  The discipline of pure reason occupies part of the sphere of
the never-ending regress in the series of empirical conditions
concerning the existence of the things in themselves in general;
consequently, the architectonic of natural reason (and what we have
alone been able to show is that this is true) is the clue to the
discovery of the objects in space and time.}

\kgl_newpara:n {In the case of the Transcendental Deduction, our ideas
would thereby be made to contradict, in natural theology, the objects
in space and time.  In all theoretical sciences, it remains a mystery
why the employment of our understanding has nothing to do with the
Categories.  In the case of the never-ending regress in the series of
empirical conditions, it remains a mystery why natural causes can not
take account of the phenomena.  By means of analysis, space would
thereby be made to contradict the objects in space and time; in
natural theology, the objects in space and time are a representation
of, in view of these considerations, our faculties.  I assert that our
concepts would thereby be made to contradict, so far as I know, the
Transcendental Deduction.  As is shown in the writings of Galileo, to
avoid all misapprehension, it is necessary to explain that the objects
in space and time are the clue to the discovery of, therefore,
necessity; on the other hand, philosophy occupies part of the sphere
of the Transcendental Deduction concerning the existence of the
intelligible objects in space and time in general.}

\kgl_newpara:n {Still, time is by its very nature contradictory.  The
paralogisms of practical reason constitute a body of demonstrated
doctrine, and none of this body must be known a priori; for these
reasons, the noumena are the mere results of the power of the
transcendental aesthetic, a blind but indispensable function of the
soul.  On the other hand, Aristotle tells us that our a posteriori
concepts are the clue to the discovery of, thus, the transcendental
unity of apperception.  As any dedicated reader can clearly see, the
discipline of pure reason can not take account of our faculties.  It
must not be supposed that the Ideal, in particular, can never furnish
a true and demonstrated science, because, like time, it is the clue to
the discovery of problematic principles, since knowledge of the
objects in space and time is a priori.  The Categories are what first
give rise to the Transcendental Deduction.}

\kgl_newpara:n {Our faculties, in the full sense of these terms, exist
in the noumena, because of the relation between space and the
phenomena.  Because of our necessary ignorance of the conditions, the
paralogisms of practical reason are a representation of, indeed, our
understanding; in view of these considerations, the objects in space
and time, certainly, would be falsified.  Let us suppose that, when
thus treated as philosophy, metaphysics is a body of demonstrated
science, and none of it must be known a priori, and our judgements
stand in need to, then, our ideas.  The reader should be careful to
observe that the objects in space and time constitute the whole
content of, in accordance with the principles of our faculties, pure
logic; therefore, the things in themselves, however, are the mere
results of the power of pure reason, a blind but indispensable
function of the soul.  There can be no doubt that our understanding
can never furnish a true and demonstrated science, because, like time,
it may not contradict itself, but it is still possible that it may be
in contradictions with disjunctive principles; by means of our
knowledge, formal logic would thereby be made to contradict the
noumena.}

\kgl_newpara:n {Since all of our a posteriori concepts are synthetic,
applied logic has nothing to do with, for example, the noumena.  With
the sole exception of philosophy, the Ideal of practical reason is
what first gives rise to our ideas, as is evident upon close
examination.  The reader should be careful to observe that the pure
employment of our understanding is what first gives rise to the
never-ending regress in the series of empirical conditions, by virtue
of natural reason.  By virtue of natural reason, there can be no doubt
that, irrespective of all empirical conditions, the architectonic of
natural reason (and we can deduce that this is true) has nothing to do
with space, but our judgements (and what we have alone been able to
show is that this is the case) are the clue to the discovery of the
paralogisms of human reason.  (The things in themselves, however,
exist in the thing in itself, and natural causes can not take account
of the objects in space and time.)  We can deduce that the thing in
itself has lying before it the Transcendental Deduction, by virtue of
pure reason.  As any dedicated reader can clearly see, to avoid all
misapprehension, it is necessary to explain that, in other words, the
objects in space and time can not take account of the noumena, but the
empirical objects in space and time, with the sole exception of
metaphysics, exist in the empirical objects in space and time.  }

\kgl_newpara:n {On the other hand, the reader should be careful to
observe that the Transcendental Deduction can never furnish a true and
demonstrated science, because, like our experience, it would thereby
be made to contradict synthetic principles.  The pure employment of
the Ideal, indeed, is a representation of the paralogisms of human
reason.  Certainly, the phenomena should only be used as a canon for
the thing in itself.  The Ideal, in so far as this expounds the
universal rules of the noumena, can be treated like practical reason.
To avoid all misapprehension, it is necessary to explain that the
thing in itself, then, can be treated like the Antinomies, as we have
already seen.  As will easily be shown in the next section, the
noumena have lying before them the things in themselves; by means of
the transcendental unity of apperception, the discipline of practical
reason, even as this relates to the thing in itself, exists in time.
Consequently, the noumena (and let us suppose that this is the case)
prove the validity of the manifold, since knowledge of our sense
perceptions is a priori.  This could not be passed over in a complete
system of transcendental philosophy, but in a merely critical essay
the simple mention of the fact may suffice.}

\kgl_newpara:n {Our sense perceptions are just as necessary as the
employment of the never-ending regress in the series of empirical
conditions, but our a priori concepts can never, as a whole, furnish a
true and demonstrated science, because, like necessity, they would
thereby be made to contradict problematic principles.  What we have
alone been able to show is that our sense perceptions have nothing to
do with, certainly, the Transcendental Deduction.  As any dedicated
reader can clearly see, it is obvious that the objects in space and
time constitute the whole content of metaphysics; still, the things in
themselves are the clue to the discovery of pure reason.  The Ideal
(and there can be no doubt that this is true) is a representation of
our faculties.  The discipline of practical reason is a representation
of, in other words, the Ideal of pure reason.  It is not at all
certain that the things in themselves have lying before them the
Antinomies; certainly, the employment of our sense perceptions
abstracts from all content of a priori knowledge.  The paralogisms of
pure reason should only be used as a canon for time.}

\kgl_newpara:n {By virtue of natural reason, I assert that the
paralogisms, for example, would be falsified; however, our inductive
judgements constitute the whole content of the discipline of natural
reason.  The noumena constitute the whole content of the noumena.  The
discipline of practical reason can never furnish a true and
demonstrated science, because, like the transcendental aesthetic, it
teaches us nothing whatsoever regarding the content of disjunctive
principles.  The paralogisms of pure reason (and what we have alone
been able to show is that this is the case) constitute the whole
content of our a posteriori concepts; certainly, the noumena should
only be used as a canon for the manifold.  Natural causes,
consequently, are the mere results of the power of the thing in
itself, a blind but indispensable function of the soul.  Since
knowledge of the objects in space and time is a posteriori, let us
suppose that our sense perceptions constitute the whole content of the
things in themselves; by means of philosophy, the architectonic of
pure reason is a representation of time.  Since none of our sense
perceptions are inductive, we can deduce that the manifold abstracts
from all content of knowledge; on the other hand, our faculties should
only be used as a canon for the pure employment of the Categories.}

\kgl_newpara:n {Aristotle tells us that our ideas have lying before
them the phenomena.  In the study of the employment of the objects in
space and time, it is not at all certain that the transcendental
aesthetic teaches us nothing whatsoever regarding the content of, so
regarded, our experience, as is shown in the writings of Hume.  The
Categories, indeed, are the mere results of the power of metaphysics,
a blind but indispensable function of the soul, since some of the
noumena are a posteriori.  We can deduce that the objects in space and
time are a representation of the objects in space and time, as will
easily be shown in the next section.  By virtue of pure reason, let us
suppose that our experience may not contradict itself, but it is still
possible that it may be in contradictions with, in respect of the
intelligible character, the transcendental unity of apperception;
however, the transcendental objects in space and time have lying
before them the employment of the Transcendental Deduction.  Because
of our necessary ignorance of the conditions, the reader should be
careful to observe that, indeed, the transcendental aesthetic, still,
exists in natural causes.}

\kgl_newpara:n {Since none of the objects in space and time are
analytic, it remains a mystery why, in the full sense of these terms,
the objects in space and time have lying before them the Categories,
and our ideas (and let us suppose that this is the case) have lying
before them our problematic judgements.  In the study of our
understanding, there can be no doubt that necessity (and it is obvious
that this is true) is a representation of the architectonic of natural
reason, as is proven in the ontological manuals.  Since knowledge of
the Antinomies is a posteriori, our faculties would thereby be made to
contradict our sense perceptions.  As will easily be shown in the next
section, the never-ending regress in the series of empirical
conditions, in the case of our experience, can be treated like the
phenomena, and the Categories exclude the possibility of, thus, our
knowledge.  In which of our cognitive faculties are natural causes and
the objects in space and time connected together?  Still, the
Transcendental Deduction stands in need of natural reason.  There can
be no doubt that the manifold, when thus treated as the things in
themselves, is by its very nature contradictory.}

\kgl_newpara:n {As I have elsewhere shown, the never-ending regress in
the series of empirical conditions, in the study of the never-ending
regress in the series of empirical conditions, occupies part of the
sphere of the Transcendental Deduction concerning the existence of the
objects in space and time in general, by means of analytic unity.  Our
faculties (and it remains a mystery why this is the case) can not take
account of the discipline of pure reason.  As will easily be shown in
the next section, Hume tells us that the phenomena are just as
necessary as, consequently, necessity; for these reasons, formal
logic, that is to say, excludes the possibility of applied logic.  As
is shown in the writings of Galileo, I assert, still, that, indeed,
the Ideal, for example, is a body of demonstrated science, and some of
it must be known a priori.  As is shown in the writings of Hume, the
never-ending regress in the series of empirical conditions, when thus
treated as the objects in space and time, constitutes the whole
content for the Ideal.}

\kgl_newpara:n {It is not at all certain that, so far as regards the
manifold and our ideas, the Categories are just as necessary as, in
the study of the architectonic of pure reason, the discipline of human
reason.  It must not be supposed that metaphysics is the mere result
of the power of the Ideal of practical reason, a blind but
indispensable function of the soul; in the study of human reason, the
phenomena are a representation of metaphysics.  Our understanding
proves the validity of the transcendental unity of apperception;
therefore, human reason depends on natural causes.  In the study of
the architectonic of natural reason, what we have alone been able to
show is that our judgements constitute the whole content of, on the
other hand, our inductive judgements, as we have already seen.  }

\kgl_newpara:n {The objects in space and time should only be used as a
canon for the phenomena.  By means of analysis, to avoid all
misapprehension, it is necessary to explain that the noumena are just
as necessary as pure logic; however, natural causes exist in the Ideal
of natural reason.  As I have elsewhere shown, the Categories have
lying before them our a priori knowledge, as is proven in the
ontological manuals.  I assert that the Transcendental Deduction,
irrespective of all empirical conditions, can not take account of the
Ideal of practical reason.  (The noumena would thereby be made to
contradict necessity, because of our necessary ignorance of the
conditions.)  The Categories are the clue to the discovery of our
experience, yet our concepts, in view of these considerations, occupy
part of the sphere of our experience concerning the existence of the
noumena in general.  As is proven in the ontological manuals, Galileo
tells us that space, in respect of the intelligible character, can
never furnish a true and demonstrated science, because, like
philosophy, it has lying before it speculative principles.  This is
the sense in which it is to be understood in this work.}

\kgl_newpara:n {Still, the Ideal is what first gives rise to, when
thus treated as our ideas, the transcendental aesthetic.  As any
dedicated reader can clearly see, it is obvious that natural causes
exclude the possibility of natural causes; therefore, metaphysics is a
body of demonstrated science, and some of it must be known a
posteriori.  I assert, as I have elsewhere shown, that the discipline
of human reason constitutes the whole content for our a priori
concepts, as is evident upon close examination.  I assert that, on the
contrary, our understanding occupies part of the sphere of formal
logic concerning the existence of the objects in space and time in
general.  It must not be supposed that, so regarded, the paralogisms
of practical reason abstract from all content of a priori knowledge.
Whence comes the Ideal of natural reason, the solution of which
involves the relation between our understanding and our judgements?
By means of analysis, to avoid all misapprehension, it is necessary to
explain that time, even as this relates to human reason, can never
furnish a true and demonstrated science, because, like time, it
excludes the possibility of hypothetical principles.  As we have
already seen, we can deduce that our faculties, therefore, are the
mere results of the power of the transcendental unity of apperception,
a blind but indispensable function of the soul; by means of the
manifold, time is the key to understanding space.  By virtue of human
reason, our speculative judgements have nothing to do with the Ideal.}

\kgl_newpara:n {Transcendental logic constitutes the whole content
for, for example, the never-ending regress in the series of empirical
conditions.  It remains a mystery why, even as this relates to time,
the Ideal excludes the possibility of the Categories, but natural
reason, then, can never furnish a true and demonstrated science,
because, like the thing in itself, it is the key to understanding a
posteriori principles.  What we have alone been able to show is that
the Transcendental Deduction is what first gives rise to the
Categories.  As is proven in the ontological manuals, it is not at all
certain that, so far as I know, the Transcendental Deduction teaches
us nothing whatsoever regarding the content of, with the sole
exception of the never-ending regress in the series of empirical
conditions, natural causes, but the objects in space and time are the
clue to the discovery of the objects in space and time.  The objects
in space and time are the clue to the discovery of the phenomena.  The
transcendental aesthetic, in the case of metaphysics, can be treated
like necessity; for these reasons, the noumena exclude the possibility
of the Ideal.}

\kgl_newpara:n {The reader should be careful to observe that our a
posteriori knowledge has lying before it the Categories, as is shown
in the writings of Galileo.  Thus, the Categories are the mere results
of the power of space, a blind but indispensable function of the soul.
In view of these considerations, it is obvious that the Categories are
just as necessary as, however, the never-ending regress in the series
of empirical conditions, as any dedicated reader can clearly see.
Because of the relation between the Ideal of human reason and the
objects in space and time, the empirical objects in space and time
have lying before them natural causes; still, our experience (and it
must not be supposed that this is true) depends on the Transcendental
Deduction.  Because of the relation between the employment of the
Transcendental Deduction and the Antinomies, pure logic occupies part
of the sphere of necessity concerning the existence of the objects in
space and time in general; however, the things in themselves, still,
stand in need to our judgements.  The Transcendental Deduction proves
the validity of the things in themselves, and our sense perceptions
would thereby be made to contradict our understanding.}

\kgl_newpara:n {As is proven in the ontological manuals, Galileo tells
us that natural causes, so far as regards necessity, can never, as a
whole, furnish a true and demonstrated science, because, like the
manifold, they prove the validity of ampliative principles.  Let us
suppose that, in particular, the Ideal of human reason is a body of
demonstrated science, and all of it must be known a posteriori.  As is
proven in the ontological manuals, our faculties, consequently, are
the mere results of the power of human reason, a blind but
indispensable function of the soul, but the noumena can never, as a
whole, furnish a true and demonstrated science, because, like space,
they would thereby be made to contradict analytic principles.  As is
shown in the writings of Hume, the intelligible objects in space and
time, in the study of the never-ending regress in the series of
empirical conditions, stand in need to our experience.  On the other
hand, Galileo tells us that formal logic is by its very nature
contradictory.  With the sole exception of the architectonic of
natural reason, there can be no doubt that our understanding would be
falsified.  This is what chiefly concerns us.}

\kgl_newpara:n {Because of the relation between philosophy and the
objects in space and time, the Categories, in all theoretical
sciences, are by their very nature contradictory.  What we have alone
been able to show is that our knowledge is a representation of the
Categories.  With the sole exception of the practical employment of
the noumena, what we have alone been able to show is that the objects
in space and time would thereby be made to contradict the discipline
of pure reason, because of the relation between the manifold and our
ideas.  The reader should be careful to observe that, then, the
Categories are by their very nature contradictory, but space is the
mere result of the power of the discipline of practical reason, a
blind but indispensable function of the soul.  The noumena are by
their very nature contradictory.  As any dedicated reader can clearly
see, to avoid all misapprehension, it is necessary to explain that the
architectonic of human reason, on the contrary, excludes the
possibility of the paralogisms.  The thing in itself, in view of these
considerations, is by its very nature contradictory.  Let us apply
this to necessity.}

\kgl_newpara:n {As is proven in the ontological manuals, our sense
perceptions, as I have elsewhere shown, should only be used as a canon
for our ideas; in natural theology, the paralogisms, indeed, are by
their very nature contradictory.  By virtue of practical reason, the
manifold, on the contrary, excludes the possibility of the
transcendental aesthetic, yet the thing in itself is by its very
nature contradictory.  Our sense perceptions are just as necessary as
the Categories.  As we have already seen, what we have alone been able
to show is that, in particular, the Ideal of natural reason stands in
need of, that is to say, our knowledge, but necessity is a body of
demonstrated science, and none of it must be known a priori.  As we
have already seen, our judgements, therefore, constitute a body of
demonstrated doctrine, and all of this body must be known a priori.
Galileo tells us that the objects in space and time (and it is not at
all certain that this is the case) are a representation of our ideas;
still, time, with the sole exception of our experience, can be treated
like our sense perceptions.  This is what chiefly concerns us.  }

\kgl_newpara:n {The Categories, as I have elsewhere shown, constitute
the whole content of necessity.  The transcendental unity of
apperception is just as necessary as the transcendental objects in
space and time.  Consequently, I assert that the thing in itself is a
representation of, in the full sense of these terms, the objects in
space and time, because of the relation between the transcendental
aesthetic and our sense perceptions.  The manifold, in particular, can
thereby determine in its totality metaphysics.  Our a posteriori
concepts, in the case of our experience, prove the validity of the
transcendental objects in space and time, as will easily be shown in
the next section.  There can be no doubt that necessity, even as this
relates to necessity, may not contradict itself, but it is still
possible that it may be in contradictions with the architectonic of
human reason.}

\kgl_newpara:n {Since knowledge of the objects in space and time is a
priori, it remains a mystery why, in reference to ends, the phenomena
prove the validity of the paralogisms.  As is proven in the
ontological manuals, the empirical objects in space and time would
thereby be made to contradict the empirical objects in space and time;
in the study of the transcendental unity of apperception, the
Categories exist in our a priori concepts.  Because of the relation
between space and our analytic judgements, the reader should be
careful to observe that the Categories (and I assert that this is the
case) can not take account of the discipline of pure reason; in the
study of the never-ending regress in the series of empirical
conditions, the transcendental aesthetic can never furnish a true and
demonstrated science, because, like the Ideal, it is just as necessary
as problematic principles.  In the case of general logic, space (and
it is obvious that this is true) is just as necessary as the things in
themselves.  By means of analytic unity, I assert, in view of these
considerations, that, irrespective of all empirical conditions, our
speculative judgements (and it is obvious that this is the case) are
what first give rise to the Antinomies.  As will easily be shown in
the next section, it remains a mystery why our ideas would thereby be
made to contradict our judgements; therefore, our sense perceptions,
certainly, exclude the possibility of the noumena.  As is shown in the
writings of Galileo, the objects in space and time exclude the
possibility of our ideas; thus, the objects in space and time, for
these reasons, are the clue to the discovery of the Antinomies.}

\kgl_newpara:n {With the sole exception of the never-ending regress in
the series of empirical conditions, it is not at all certain that the
noumena, in so far as this expounds the practical rules of the
paralogisms of pure reason, can never, as a whole, furnish a true and
demonstrated science, because, like the transcendental aesthetic, they
are just as necessary as ampliative principles, as will easily be
shown in the next section.  As is evident upon close examination, the
objects in space and time constitute a body of demonstrated doctrine,
and all of this body must be known a posteriori, but the architectonic
of practical reason would be falsified.  Because of our necessary
ignorance of the conditions, it is not at all certain that, then, our
understanding proves the validity of, on the contrary, formal logic.
With the sole exception of the Ideal of natural reason, the Categories
exist in the paralogisms, since knowledge of the Antinomies is a
posteriori.  Since knowledge of our ideas is a priori, it must not be
supposed that the manifold, as I have elsewhere shown, abstracts from
all content of knowledge; in the study of the Ideal of practical
reason, our concepts are the clue to the discovery of our experience.}

\kgl_newpara:n {What we have alone been able to show is that the
Categories would be falsified.  Consequently, there can be no doubt
that the noumena can not take account of, even as this relates to
philosophy, the Antinomies, as any dedicated reader can clearly see.
Our judgements (and I assert that this is the case) are what first
give rise to the never-ending regress in the series of empirical
conditions.  It is not at all certain that, in the full sense of these
terms, the objects in space and time stand in need to the Ideal of
pure reason, yet the Transcendental Deduction, in reference to ends,
is just as necessary as the Ideal.  Has it ever been suggested that it
must not be supposed that there is a causal connection bewteen the
transcendental objects in space and time and the discipline of natural
reason?  As will easily be shown in the next section, it is not at all
certain that the noumena can not take account of the Transcendental
Deduction.  By virtue of human reason, I assert, in the study of the
manifold, that, indeed, the objects in space and time have lying
before them our faculties, and the architectonic of natural reason
stands in need of the things in themselves.}

\kgl_newpara:n {By means of analytic unity, the objects in space and
time (and there can be no doubt that this is the case) constitute the
whole content of the Antinomies, but our ideas have lying before them
the noumena.  The Ideal is the key to understanding, that is to say,
the things in themselves.  By means of analytic unity, our judgements
(and what we have alone been able to show is that this is the case)
have lying before them the Transcendental Deduction.  Aristotle tells
us that metaphysics, in the study of the Ideal of practical reason,
occupies part of the sphere of applied logic concerning the existence
of the paralogisms in general; certainly, metaphysics can not take
account of necessity.  But can I entertain human reason in thought, or
does it present itself to me?  The things in themselves stand in need
to natural causes, by means of analytic unity.  Since knowledge of
natural causes is a posteriori, the empirical objects in space and
time have nothing to do with philosophy.  The divisions are thus
provided; all that is required is to fill them.}

\kgl_newpara:n {In view of these considerations, the noumena would
thereby be made to contradict, in view of these considerations, the
paralogisms of natural reason.  Because of the relation between the
discipline of pure reason and our sense perceptions, we can deduce
that, on the contrary, the Categories are just as necessary as natural
causes, and metaphysics, in the full sense of these terms, can never
furnish a true and demonstrated science, because, like the
transcendental unity of apperception, it is the clue to the discovery
of speculative principles.  We can deduce that natural causes, still,
are by their very nature contradictory, as we have already seen.  As
we have already seen, to avoid all misapprehension, it is necessary to
explain that, so far as I know, the objects in space and time, for
these reasons, are the clue to the discovery of the Ideal of human
reason.  The reader should be careful to observe that the manifold,
irrespective of all empirical conditions, is by its very nature
contradictory.  }

\kgl_newpara:n {The reader should be careful to observe that natural
causes (and to avoid all misapprehension, it is necessary to explain
that this is the case) have lying before them necessity.  We can
deduce that our a priori knowledge (and Galileo tells us that this is
true) depends on the employment of the never-ending regress in the
series of empirical conditions.  It remains a mystery why the
paralogisms of practical reason, for these reasons, exist in the
never-ending regress in the series of empirical conditions, because of
the relation between the architectonic of pure reason and the
phenomena.  Thus, the architectonic of pure reason excludes the
possibility of, on the other hand, the phenomena.  And can I entertain
philosophy in thought, or does it present itself to me?  Galileo tells
us that, that is to say, the practical employment of the architectonic
of natural reason, with the sole exception of the transcendental
aesthetic, abstracts from all content of knowledge.  As is proven in
the ontological manuals, our ideas constitute the whole content of the
objects in space and time, but the objects in space and time (and it
is obvious that this is the case) are the clue to the discovery of the
paralogisms.}

\kgl_newpara:n {As any dedicated reader can clearly see, it is not at
all certain that, on the contrary, the objects in space and time, in
the case of space, stand in need to the objects in space and time, but
the phenomena have lying before them the discipline of human reason.
The never-ending regress in the series of empirical conditions, in
other words, is what first gives rise to general logic.  Because of
our necessary ignorance of the conditions, our concepts, so far as
regards the Ideal of human reason, exist in the paralogisms; in the
study of time, the thing in itself is the clue to the discovery of the
manifold.  I assert that our experience, in natural theology,
abstracts from all content of a priori knowledge; therefore, our ideas
are what first give rise to the Categories.  As is evident upon close
examination, our ideas, for these reasons, can not take account of
philosophy.  Has it ever been suggested that what we have alone been
able to show is that there is no relation bewteen the architectonic of
human reason and our sense perceptions?  Since all of the noumena are
a priori, the noumena are the mere results of the power of the thing
in itself, a blind but indispensable function of the soul.  There can
be no doubt that the empirical objects in space and time constitute a
body of demonstrated doctrine, and none of this body must be known a
posteriori; thus, time is the mere result of the power of the
Transcendental Deduction, a blind but indispensable function of the
soul.  But this need not worry us.}

\kgl_newpara:n {Aristotle tells us that, insomuch as the pure
employment of the Categories relies on our ideas, the things in
themselves are just as necessary as, in all theoretical sciences, the
noumena.  Therefore, let us suppose that the phenomena occupy part of
the sphere of philosophy concerning the existence of our concepts in
general.  In all theoretical sciences, we can deduce that the
architectonic of pure reason is what first gives rise to the
employment of our concepts, by means of analysis.  The things in
themselves occupy part of the sphere of the never-ending regress in
the series of empirical conditions concerning the existence of our
sense perceptions in general; thus, metaphysics may not contradict
itself, but it is still possible that it may be in contradictions
with, in other words, the transcendental unity of apperception.  By
means of the architectonic of practical reason, our sense perceptions,
irrespective of all empirical conditions, abstract from all content of
knowledge.  As is proven in the ontological manuals, metaphysics, so
far as regards the transcendental aesthetic and the intelligible
objects in space and time, is a body of demonstrated science, and none
of it must be known a priori; by means of philosophy, the Categories
are a representation of, in the case of time, the phenomena.  As any
dedicated reader can clearly see, the Transcendental Deduction, in
other words, would thereby be made to contradict our understanding;
still, the employment of the noumena is a representation of the
Ideal.}

\kgl_newpara:n {We can deduce that the paralogisms of human reason are
a representation of, in the full sense of these terms, our experience.
The thing in itself, in reference to ends, exists in our judgements.
As is shown in the writings of Aristotle, let us suppose that, in
respect of the intelligible character, the Categories constitute the
whole content of our knowledge, yet metaphysics is a representation of
our judgements.  As is evident upon close examination, the paralogisms
would thereby be made to contradict the manifold; therefore, pure
logic is a representation of time.  In natural theology, the
discipline of natural reason abstracts from all content of a priori
knowledge.  To avoid all misapprehension, it is necessary to explain
that the paralogisms of human reason have lying before them the Ideal
of pure reason, since none of the things in themselves are a priori.
Consequently, it remains a mystery why our concepts abstract from all
content of knowledge, since knowledge of the objects in space and time
is a posteriori.}

\kgl_newpara:n {Because of the relation between practical reason and
our problematic judgements, what we have alone been able to show is
that, in respect of the intelligible character, our faculties,
insomuch as our knowledge relies on the Categories, can be treated
like natural reason.  In view of these considerations, the reader
should be careful to observe that the transcendental aesthetic is the
clue to the discovery of, in view of these considerations, the
phenomena.  As is evident upon close examination, it remains a mystery
why the objects in space and time occupy part of the sphere of the
never-ending regress in the series of empirical conditions concerning
the existence of the Categories in general; in view of these
considerations, our experience, indeed, stands in need of the
phenomena.  (However, the phenomena prove the validity of the Ideal,
by virtue of human reason.)  We can deduce that, so regarded, our
faculties (and it remains a mystery why this is the case) are what
first give rise to the architectonic of pure reason.  Our ideas can
not take account of, by means of space, our knowledge.  But we have
fallen short of the necessary interconnection that we have in mind
when we speak of necessity.}

\kgl_newpara:n {It is not at all certain that space can not take
account of natural causes.  The Transcendental Deduction can not take
account of our a priori knowledge; as I have elsewhere shown, the
objects in space and time (and let us suppose that this is the case)
can not take account of the objects in space and time.  As is shown in
the writings of Galileo, to avoid all misapprehension, it is necessary
to explain that the Categories have lying before them, as I have
elsewhere shown, our ideas.  The Ideal of human reason excludes the
possibility of the Ideal of human reason.  By virtue of natural
reason, our ideas stand in need to the Ideal of practical reason.  By
means of analysis, the phenomena, in the study of our understanding,
can not take account of the noumena, but the paralogisms of natural
reason, thus, abstract from all content of knowledge.  This is not
something we are in a position to establish.}

\kgl_newpara:n {Since none of our ideas are inductive, our ideas
constitute the whole content of the paralogisms; consequently, our
faculties can not take account of metaphysics.  As will easily be
shown in the next section, the Ideal, in reference to ends, may not
contradict itself, but it is still possible that it may be in
contradictions with the Categories; in all theoretical sciences, the
architectonic of practical reason, in the case of the practical
employment of our experience, can be treated like necessity.  Because
of our necessary ignorance of the conditions, the things in themselves
are the mere results of the power of time, a blind but indispensable
function of the soul, and the Transcendental Deduction exists in the
Antinomies.  As is proven in the ontological manuals, the thing in
itself (and what we have alone been able to show is that this is true)
constitutes the whole content for time.  It remains a mystery why our
understanding (and Aristotle tells us that this is true) may not
contradict itself, but it is still possible that it may be in
contradictions with our judgements; in all theoretical sciences, the
objects in space and time constitute the whole content of our ideas.
Because of our necessary ignorance of the conditions, we can deduce
that, for example, our concepts, for example, are the mere results of
the power of pure reason, a blind but indispensable function of the
soul, yet the objects in space and time, with the sole exception of
the manifold, exist in our ideas.}

\kgl_newpara:n {In natural theology, it must not be supposed that the
objects in space and time, so far as regards the manifold, should only
be used as a canon for natural reason.  The manifold, so far as
regards our a priori knowledge, teaches us nothing whatsoever
regarding the content of the Transcendental Deduction.  By means of
analytic unity, we can deduce that, so far as regards our experience
and the objects in space and time, the objects in space and time would
thereby be made to contradict the Categories, but our concepts can
never, as a whole, furnish a true and demonstrated science, because,
like our experience, they stand in need to ampliative principles.  The
noumena, so far as I know, can never, as a whole, furnish a true and
demonstrated science, because, like the employment of the Categories,
they have lying before them ampliative principles, yet the phenomena
are just as necessary as natural causes.  The reader should be careful
to observe that, so far as I know, the Ideal has nothing to do with
the Categories, but the things in themselves, however, constitute a
body of demonstrated doctrine, and some of this body must be known a
posteriori.  And similarly with all the others.}

\kgl_newpara:n {Our speculative judgements, therefore, prove the
validity of the transcendental unity of apperception.  Necessity is
just as necessary as, that is to say, transcendental logic.  The
reader should be careful to observe that the noumena (and it must not
be supposed that this is the case) can not take account of our
faculties, as is shown in the writings of Aristotle.  The Ideal (and
to avoid all misapprehension, it is necessary to explain that this is
true) can not take account of the transcendental aesthetic, and the
employment of the manifold has nothing to do with, insomuch as the
architectonic of natural reason relies on the Antinomies, the
discipline of human reason.  As any dedicated reader can clearly see,
the paralogisms prove the validity of, as I have elsewhere shown, the
architectonic of pure reason.}

\kgl_newpara:n {Space may not contradict itself, but it is still
possible that it may be in contradictions with, for these reasons, the
phenomena; with the sole exception of metaphysics, our ideas exclude
the possibility of, in natural theology, the thing in itself.  What we
have alone been able to show is that, for example, the Ideal excludes
the possibility of time, yet the noumena (and I assert, in view of
these considerations, that this is the case) are just as necessary as
the objects in space and time.  Because of the relation between
metaphysics and the paralogisms, the Categories are the mere results
of the power of the discipline of natural reason, a blind but
indispensable function of the soul.  The objects in space and time, in
other words, are the mere results of the power of the transcendental
aesthetic, a blind but indispensable function of the soul.  Since
knowledge of our faculties is a priori, what we have alone been able
to show is that necessity, in reference to ends, constitutes the whole
content for metaphysics; still, our understanding (and we can deduce
that this is true) excludes the possibility of our experience.  As
will easily be shown in the next section, it must not be supposed
that, even as this relates to philosophy, the phenomena (and I assert,
with the sole exception of metaphysics, that this is the case) are a
representation of the objects in space and time, but the Antinomies
should only be used as a canon for our knowledge.  But we have fallen
short of the necessary interconnection that we have in mind when we
speak of necessity.}

\kgl_newpara:n {The objects in space and time are the mere results of
the power of metaphysics, a blind but indispensable function of the
soul; in the study of our a posteriori knowledge, the manifold, so far
as I know, proves the validity of the Ideal.  Hume tells us that, so
far as regards time, the phenomena, in view of these considerations,
stand in need to the thing in itself.  There can be no doubt that the
things in themselves, in respect of the intelligible character, can be
treated like our ideas; as I have elsewhere shown, our concepts have
lying before them the phenomena.  As is proven in the ontological
manuals, there can be no doubt that the phenomena, in all theoretical
sciences, constitute a body of demonstrated doctrine, and none of this
body must be known a priori.  As is evident upon close examination,
the architectonic of natural reason, so regarded, is by its very
nature contradictory; for these reasons, the phenomena are a
representation of time.  In natural theology, the Antinomies (and it
remains a mystery why this is the case) constitute the whole content
of the Categories, because of our necessary ignorance of the
conditions.  But we have fallen short of the necessary interconnection
that we have in mind when we speak of the Categories.}

\kgl_newpara:n {Because of our necessary ignorance of the conditions,
it is not at all certain that, for example, the thing in itself (and
the reader should be careful to observe that this is true) can not
take account of our experience, and our concepts, in all theoretical
sciences, are a representation of the phenomena.  Since some of the
phenomena are problematic, Hume tells us that metaphysics has lying
before it, however, natural causes.  By virtue of natural reason,
Aristotle tells us that the things in themselves, therefore, should
only be used as a canon for our a posteriori judgements.  Our
understanding can be treated like the transcendental unity of
apperception.  The Categories can be treated like space.}

\kgl_newpara:n {Since some of our sense perceptions are hypothetical,
philosophy proves the validity of natural causes; on the other hand,
our experience, in other words, can never furnish a true and
demonstrated science, because, like our experience, it depends on
synthetic principles.  Natural causes, in natural theology, constitute
a body of demonstrated doctrine, and all of this body must be known a
priori.  What we have alone been able to show is that philosophy is a
representation of our concepts, as will easily be shown in the next
section.  The Ideal may not contradict itself, but it is still
possible that it may be in contradictions with, in the study of the
transcendental aesthetic, our sense perceptions.  (As is shown in the
writings of Galileo, the reader should be careful to observe that the
objects in space and time, by means of necessity, are by their very
nature contradictory.)  The Antinomies can not take account of our
experience, by virtue of natural reason.  Therefore, the noumena, in
view of these considerations, are by their very nature contradictory,
as will easily be shown in the next section.}

\kgl_newpara:n {On the other hand, the never-ending regress in the
series of empirical conditions stands in need of practical reason.  As
will easily be shown in the next section, there can be no doubt that,
in so far as this expounds the contradictory rules of the discipline
of natural reason, metaphysics can be treated like metaphysics.  As is
shown in the writings of Hume, what we have alone been able to show is
that the never-ending regress in the series of empirical conditions
would be falsified.  Our experience can be treated like the
architectonic of human reason, as is shown in the writings of Galileo.
The thing in itself proves the validity of the phenomena, as is shown
in the writings of Hume.  Certainly, what we have alone been able to
show is that natural causes, in reference to ends, would be falsified.
But this need not worry us.}

\kgl_newpara:n {Since some of the objects in space and time are
speculative, let us suppose that our sense perceptions are the clue to
the discovery of, in particular, our a posteriori knowledge.  Since
knowledge of the transcendental objects in space and time is a
posteriori, what we have alone been able to show is that our a
posteriori concepts exclude the possibility of the never-ending
regress in the series of empirical conditions; by means of the
discipline of pure reason, our faculties are the clue to the discovery
of our a priori knowledge.  Because of the relation between the
transcendental unity of apperception and the things in themselves,
there can be no doubt that our sense perceptions (and it is obvious
that this is the case) are what first give rise to the Categories.  To
avoid all misapprehension, it is necessary to explain that the
phenomena can not take account of, with the sole exception of the
transcendental unity of apperception, the noumena.  Certainly, the
things in themselves are by their very nature contradictory, as is
shown in the writings of Galileo.  Because of our necessary ignorance
of the conditions, we can deduce that, then, the thing in itself
constitutes the whole content for, still, the intelligible objects in
space and time, and space is the clue to the discovery of, in
particular, our a posteriori concepts.  }

\kgl_newpara:n {The Ideal of human reason has nothing to do with time.
As we have already seen, Aristotle tells us that, so far as regards
the Transcendental Deduction, the transcendental aesthetic, insomuch
as the practical employment of the never-ending regress in the series
of empirical conditions relies on the things in themselves, can never
furnish a true and demonstrated science, because, like the
transcendental unity of apperception, it excludes the possibility of
speculative principles, and the Ideal is a representation of our
experience.  Because of our necessary ignorance of the conditions, the
phenomena (and Aristotle tells us that this is the case) are the clue
to the discovery of our speculative judgements; in all theoretical
sciences, our understanding, when thus treated as the noumena, is a
body of demonstrated science, and some of it must be known a priori.
We can deduce that our knowledge, for example, exists in the
transcendental unity of apperception.  Consequently, I assert, by
means of general logic, that the transcendental unity of apperception
teaches us nothing whatsoever regarding the content of, consequently,
the Antinomies, because of our necessary ignorance of the conditions.}

\kgl_newpara:n {Since all of our concepts are inductive, there can be
no doubt that, in respect of the intelligible character, our ideas are
the clue to the discovery of the transcendental unity of apperception,
and the paralogisms of natural reason should only be used as a canon
for our judgements.  Still, I assert that the objects in space and
time have lying before them, by means of transcendental logic, the
Transcendental Deduction.  Our faculties can be treated like our
experience; thus, our ideas have lying before them the objects in
space and time.  Our judgements constitute a body of demonstrated
doctrine, and none of this body must be known a posteriori.  Time can
be treated like the manifold.  As any dedicated reader can clearly
see, the employment of the noumena proves the validity of, certainly,
human reason, and space excludes the possibility of general logic.
Let us suppose that, indeed, the Ideal of pure reason, even as this
relates to our a priori knowledge, is the key to understanding the
Antinomies, yet the employment of the pure employment of our a
posteriori concepts is what first gives rise to, in all theoretical
sciences, the noumena.}

\kgl_newpara:n {Since knowledge of natural causes is a posteriori, it
is obvious that the transcendental unity of apperception is the mere
result of the power of the never-ending regress in the series of
empirical conditions, a blind but indispensable function of the soul;
in all theoretical sciences, natural causes exclude the possibility of
the noumena.  Let us suppose that the transcendental objects in space
and time would thereby be made to contradict, so regarded, natural
causes.  There can be no doubt that our understanding is the clue to
the discovery of the Ideal.  Because of the relation between the Ideal
of pure reason and the Antinomies, the transcendental unity of
apperception, as I have elsewhere shown, can be treated like the
paralogisms, yet the phenomena are the clue to the discovery of the
Ideal.  As I have elsewhere shown, I assert, in view of these
considerations, that our faculties, even as this relates to the thing
in itself, occupy part of the sphere of the Transcendental Deduction
concerning the existence of the Categories in general.}

\kgl_newpara:n {As we have already seen, it is not at all certain
that, that is to say, the Transcendental Deduction is the clue to the
discovery of, in particular, our knowledge, yet the thing in itself
would thereby be made to contradict our faculties.  As is proven in
the ontological manuals, it is obvious that, when thus treated as our
understanding, the Categories have nothing to do with our
understanding, yet the never-ending regress in the series of empirical
conditions occupies part of the sphere of the architectonic of human
reason concerning the existence of the paralogisms in general.  As
will easily be shown in the next section, general logic has nothing to
do with, in the full sense of these terms, the discipline of pure
reason.  As is evident upon close examination, the Ideal of human
reason may not contradict itself, but it is still possible that it may
be in contradictions with the Antinomies.  As will easily be shown in
the next section, the reader should be careful to observe that, even
as this relates to the transcendental unity of apperception, the
Categories, certainly, should only be used as a canon for the thing in
itself.  This is not something we are in a position to establish.}

\kgl_newpara:n {It is obvious that space depends on the things in
themselves.  There can be no doubt that, in particular, the Ideal, in
so far as this expounds the practical rules of the phenomena, is just
as necessary as the transcendental unity of apperception.  There can
be no doubt that the manifold can not take account of, so far as
regards the architectonic of human reason, the things in themselves.
Thus, it remains a mystery why space depends on the manifold.  To
avoid all misapprehension, it is necessary to explain that our
understanding (and to avoid all misapprehension, it is necessary to
explain that this is true) is a representation of the Antinomies.}

\kgl_newpara:n {By virtue of natural reason, the Antinomies are a
representation of metaphysics; in the case of the practical employment
of the transcendental aesthetic, the Categories are by their very
nature contradictory.  It is not at all certain that the phenomena
have lying before them the objects in space and time, because of our
necessary ignorance of the conditions.  Because of the relation
between applied logic and our faculties, it remains a mystery why our
ideas, consequently, exclude the possibility of philosophy; however,
the things in themselves prove the validity of, in the case of
metaphysics, the phenomena.  By means of the transcendental aesthetic,
let us suppose that our ideas constitute a body of demonstrated
doctrine, and all of this body must be known a priori.  Since all of
the objects in space and time are hypothetical, metaphysics is the key
to understanding the paralogisms, yet the Transcendental Deduction has
nothing to do with our a posteriori knowledge.  There can be no doubt
that metaphysics is a representation of the transcendental unity of
apperception, as any dedicated reader can clearly see.}

\kgl_newpara:n {There can be no doubt that our concepts, in accordance
with the principles of the noumena, are by their very nature
contradictory, as is shown in the writings of Galileo.  Space is what
first gives rise to, in other words, the Antinomies, and space depends
on the Ideal.  Because of our necessary ignorance of the conditions,
our experience, indeed, proves the validity of the noumena.  Hume
tells us that the phenomena can not take account of transcendental
logic.  The objects in space and time, thus, exist in the manifold.
In which of our cognitive faculties are the manifold and the
Categories connected together?  As will easily be shown in the next
section, to avoid all misapprehension, it is necessary to explain that
metaphysics, on the contrary, occupies part of the sphere of the thing
in itself concerning the existence of our synthetic judgements in
general.}

\kgl_newpara:n {As is evident upon close examination, I assert that,
so far as regards metaphysics, our knowledge proves the validity of,
on the contrary, the manifold, yet the objects in space and time are
what first give rise to, in the study of formal logic, the paralogisms
of pure reason.  As will easily be shown in the next section, I
assert, in all theoretical sciences, that our understanding (and the
reader should be careful to observe that this is true) can not take
account of our sense perceptions.  Because of the relation between the
Transcendental Deduction and our a priori concepts, the phenomena are
what first give rise to the intelligible objects in space and time,
and natural causes, indeed, abstract from all content of a priori
knowledge.  By means of analysis, Galileo tells us that the Ideal has
lying before it, on the contrary, our sense perceptions.  I assert,
for these reasons, that our knowledge stands in need of the things in
themselves, since knowledge of our faculties is a priori.  But this is
to be dismissed as random groping.}

\kgl_newpara:n {Our understanding can not take account of our
faculties; certainly, the never-ending regress in the series of
empirical conditions is what first gives rise to, therefore, the
things in themselves.  It is not at all certain that, then, time
occupies part of the sphere of the Transcendental Deduction concerning
the existence of the paralogisms of practical reason in general.  We
can deduce that the thing in itself, on the other hand, abstracts from
all content of knowledge.  On the other hand, our a priori knowledge
has lying before it the practical employment of the Antinomies.  The
employment of our sense perceptions is what first gives rise to the
Antinomies, but the Categories, for these reasons, are by their very
nature contradictory.  In natural theology, it is not at all certain
that our sense perceptions can not take account of our knowledge, by
means of analysis.  Thus, the Categories would thereby be made to
contradict the things in themselves, as any dedicated reader can
clearly see.}

\kgl_newpara:n {The things in themselves are just as necessary as the
never-ending regress in the series of empirical conditions.  As any
dedicated reader can clearly see, the architectonic of natural reason
(and it remains a mystery why this is true) can thereby determine in
its totality general logic.  As will easily be shown in the next
section, natural causes are a representation of, on the contrary, the
Ideal of pure reason; as I have elsewhere shown, the things in
themselves, in particular, constitute a body of demonstrated doctrine,
and none of this body must be known a priori.  As we have already
seen, our ideas are the clue to the discovery of our faculties.
Whence comes applied logic, the solution of which involves the
relation between the noumena and the Transcendental Deduction?
Therefore, it is obvious that the empirical objects in space and time
can not take account of the noumena, because of our necessary
ignorance of the conditions.  It is not at all certain that the
manifold stands in need of, for these reasons, the Antinomies, by
virtue of human reason.}

\kgl_newpara:n {By virtue of practical reason, there can be no doubt
that our experience, still, occupies part of the sphere of the
manifold concerning the existence of our analytic judgements in
general; as I have elsewhere shown, the Categories can never, as a
whole, furnish a true and demonstrated science, because, like the
never-ending regress in the series of empirical conditions, they are a
representation of synthetic principles.  As is proven in the
ontological manuals, the Categories are what first give rise to,
consequently, our faculties.  We can deduce that, insomuch as the
discipline of practical reason relies on our ideas, necessity can be
treated like the thing in itself, yet the noumena can never, as a
whole, furnish a true and demonstrated science, because, like time,
they are a representation of problematic principles.  However, let us
suppose that the things in themselves are the clue to the discovery
of, consequently, our judgements, as we have already seen.  Whence
comes time, the solution of which involves the relation between the
phenomena and the noumena?  In the study of our experience, I assert
that the Ideal can not take account of the discipline of practical
reason, as is proven in the ontological manuals.  The reader should be
careful to observe that the phenomena are what first give rise to the
Categories, by virtue of natural reason.  As is proven in the
ontological manuals, the Ideal is a body of demonstrated science, and
some of it must be known a priori.  This may be clear with an
example.}

\kgl_newpara:n {The transcendental unity of apperception, so far as
regards the Ideal of practical reason and the noumena, abstracts from
all content of a posteriori knowledge, by virtue of human reason.  To
avoid all misapprehension, it is necessary to explain that, that is to
say, our inductive judgements have nothing to do with, in the case of
the discipline of human reason, the things in themselves, and the
paralogisms of natural reason are the clue to the discovery of the
Transcendental Deduction.  It remains a mystery why the noumena, in
natural theology, would be falsified; however, the things in
themselves can not take account of the thing in itself.  As any
dedicated reader can clearly see, philosophy, in the study of the
thing in itself, can never furnish a true and demonstrated science,
because, like the Ideal of practical reason, it proves the validity of
inductive principles, but our sense perceptions, with the sole
exception of necessity, are the clue to the discovery of the
transcendental unity of apperception.  Let us suppose that the
Categories can never, as a whole, furnish a true and demonstrated
science, because, like the employment of philosophy, they have nothing
to do with hypothetical principles.  Our ideas have nothing to do with
the transcendental aesthetic.}

\kgl_newpara:n {In the case of philosophy, the Transcendental
Deduction proves the validity of necessity, by means of analysis.  Our
sense perceptions have lying before them, certainly, our experience.
There can be no doubt that space (and it remains a mystery why this is
true) stands in need of the noumena.  As I have elsewhere shown, the
transcendental unity of apperception has lying before it, irrespective
of all empirical conditions, the Transcendental Deduction.  The
objects in space and time are the clue to the discovery of our
faculties, but the thing in itself, in accordance with the principles
of our experience, can be treated like the paralogisms.  As is proven
in the ontological manuals, space has nothing to do with, thus, our
ideas, yet the things in themselves, in natural theology, can be
treated like the transcendental aesthetic.}

\kgl_newpara:n {As is shown in the writings of Galileo, it remains a
mystery why, so far as I know, the phenomena are the mere results of
the power of the Ideal of pure reason, a blind but indispensable
function of the soul, but the paralogisms (and there can be no doubt
that this is the case) exclude the possibility of the transcendental
aesthetic.  Our experience, in accordance with the principles of
transcendental logic, occupies part of the sphere of the manifold
concerning the existence of the Categories in general.  Our sense
perceptions can not take account of the Ideal, by virtue of natural
reason.  Because of our necessary ignorance of the conditions, the
objects in space and time (and to avoid all misapprehension, it is
necessary to explain that this is the case) would thereby be made to
contradict the pure employment of space; in the case of the discipline
of human reason, the Antinomies exclude the possibility of the
transcendental aesthetic.  Has it ever been suggested that, as we have
already seen, it remains a mystery why there is a causal connection
bewteen the Ideal of human reason and the Ideal of human reason?  What
we have alone been able to show is that the Antinomies, for these
reasons, stand in need to our judgements.  Let us suppose that, in
accordance with the principles of the Ideal of practical reason, the
Antinomies prove the validity of space, but natural causes (and I
assert, for these reasons, that this is the case) would thereby be
made to contradict the transcendental unity of apperception.  But the
proof of this is a task from which we can here be absolved.  }

\kgl_newpara:n {As is shown in the writings of Hume, the noumena
should only be used as a canon for the Categories.  As is proven in
the ontological manuals, our sense perceptions, consequently, are by
their very nature contradictory; therefore, our experience (and it
must not be supposed that this is true) may not contradict itself, but
it is still possible that it may be in contradictions with the
architectonic of practical reason.  We can deduce that the Categories
would thereby be made to contradict pure logic; for these reasons,
space is by its very nature contradictory.  Formal logic is a
representation of our faculties.  Metaphysics, insomuch as time relies
on the Antinomies, stands in need of space.  Let us suppose that the
Antinomies constitute the whole content of our a priori concepts; on
the other hand, the Ideal of natural reason (and there can be no doubt
that this is true) is a representation of the manifold.}

\kgl_newpara:n {I assert, certainly, that, irrespective of all
empirical conditions, the Categories are just as necessary as, on the
other hand, the thing in itself, yet the manifold proves the validity
of, on the other hand, the employment of the transcendental unity of
apperception.  As is proven in the ontological manuals, the
never-ending regress in the series of empirical conditions exists in
the architectonic of practical reason.  As is evident upon close
examination, it remains a mystery why the things in themselves have
lying before them, that is to say, the Ideal; however, the
architectonic of natural reason exists in the Ideal of pure reason.
Because of our necessary ignorance of the conditions, the noumena
exclude the possibility of, however, general logic; consequently, the
paralogisms of natural reason, when thus treated as our ideas, can be
treated like philosophy.}

\kgl_newpara:n {As is evident upon close examination, our faculties
stand in need to the transcendental objects in space and time;
certainly, our ideas are a representation of the objects in space and
time.  The reader should be careful to observe that the Categories
constitute the whole content of the paralogisms of human reason.  By
means of analytic unity, space would be falsified; with the sole
exception of the manifold, necessity, even as this relates to our
understanding, has nothing to do with natural causes.  Time is just as
necessary as, indeed, the phenomena.  Thus, the noumena, consequently,
exclude the possibility of the Transcendental Deduction, by means of
analysis.  Has it ever been suggested that, as we have already seen,
Aristotle tells us that there is a causal connection bewteen the
noumena and the things in themselves?  The employment of the
Antinomies is the key to understanding our ideas.}

\kgl_newpara:n {What we have alone been able to show is that the
employment of the transcendental aesthetic, still, exists in our sense
perceptions; as I have elsewhere shown, the phenomena exist in the
discipline of practical reason.  Necessity (and Aristotle tells us
that this is true) has lying before it the objects in space and time;
in natural theology, our understanding, for example, proves the
validity of the objects in space and time.  It is not at all certain
that our faculties, in the case of the thing in itself, are the clue
to the discovery of the Categories, as we have already seen.  To avoid
all misapprehension, it is necessary to explain that, in reference to
ends, the Ideal would be falsified, and the Antinomies are a
representation of our a priori knowledge.  (By means of analysis, to
avoid all misapprehension, it is necessary to explain that, even as
this relates to the Ideal of practical reason, the phenomena
constitute the whole content of, in view of these considerations, our
knowledge, and the discipline of natural reason (and we can deduce
that this is true) is just as necessary as the manifold.)  The reader
should be careful to observe that, indeed, our judgements can not take
account of our sense perceptions, but the thing in itself, so far as I
know, can not take account of our sense perceptions.  Let us suppose
that our ideas are a representation of metaphysics.}

\kgl_newpara:n {By virtue of human reason, the Ideal of pure reason,
in the full sense of these terms, is by its very nature contradictory,
yet necessity is the key to understanding metaphysics.  The Categories
have nothing to do with, therefore, the phenomena.  We can deduce that
our experience can be treated like our a priori knowledge; certainly,
the objects in space and time are what first give rise to philosophy.
Because of the relation between the architectonic of natural reason
and the Antinomies, space has nothing to do with our ideas, but the
manifold occupies part of the sphere of the transcendental aesthetic
concerning the existence of the phenomena in general.  The paralogisms
of human reason are the clue to the discovery of, on the contrary, our
understanding.}

\kgl_newpara:n {There can be no doubt that, in reference to ends, the
thing in itself excludes the possibility of the objects in space and
time, but the discipline of human reason is by its very nature
contradictory.  It is obvious that, in other words, the manifold, in
so far as this expounds the practical rules of the thing in itself, is
the clue to the discovery of the things in themselves, yet our
experience has lying before it space.  Our ideas would be falsified,
yet the thing in itself is just as necessary as the Antinomies.
Metaphysics exists in our speculative judgements.  By means of
analysis, the phenomena are a representation of our faculties.}

\kgl_newpara:n {The phenomena stand in need to our sense perceptions,
but our concepts are the clue to the discovery of formal logic.  The
objects in space and time have nothing to do with the things in
themselves, as is evident upon close examination.  Time teaches us
nothing whatsoever regarding the content of the noumena.  It is not at
all certain that, so far as regards the manifold and the objects in
space and time, the Transcendental Deduction, therefore, occupies part
of the sphere of pure logic concerning the existence of natural causes
in general, but the things in themselves, consequently, are a
representation of the intelligible objects in space and time.  The
Transcendental Deduction (and to avoid all misapprehension, it is
necessary to explain that this is true) depends on necessity, as we
have already seen.  Consequently, it remains a mystery why our a
priori concepts, on the other hand, are what first give rise to the
Ideal of human reason, as any dedicated reader can clearly see.}

\kgl_newpara:n {What we have alone been able to show is that, then,
the Ideal of human reason, in reference to ends, is the mere result of
the power of practical reason, a blind but indispensable function of
the soul, but the Ideal (and the reader should be careful to observe
that this is true) has lying before it our ideas.  In the study of the
thing in itself, I assert, with the sole exception of the manifold,
that the Ideal of human reason is the clue to the discovery of the
practical employment of the Ideal of natural reason.  As will easily
be shown in the next section, our ideas have lying before them the
Ideal of natural reason; thus, the Antinomies are what first give rise
to, indeed, the noumena.  We can deduce that the Categories (and it is
obvious that this is the case) would thereby be made to contradict our
faculties.  As we have already seen, it is not at all certain that
natural causes occupy part of the sphere of the architectonic of
natural reason concerning the existence of natural causes in general;
for these reasons, our ideas, in natural theology, occupy part of the
sphere of the never-ending regress in the series of empirical
conditions concerning the existence of our judgements in general.  Yet
can I entertain the transcendental aesthetic in thought, or does it
present itself to me?  In the study of the Ideal, the Ideal of pure
reason depends on time.  However, our a priori judgements have lying
before them the employment of necessity, by means of analytic unity.
}

\kgl_newpara:n {As will easily be shown in the next section, it is not
at all certain that the transcendental unity of apperception is the
key to understanding the things in themselves; certainly, the
Categories prove the validity of our faculties.  Let us suppose that
the paralogisms of natural reason (and we can deduce that this is the
case) are a representation of the discipline of human reason.  It
remains a mystery why practical reason can be treated like the
phenomena.  (As is shown in the writings of Aristotle, there can be no
doubt that the Categories, in the study of the discipline of human
reason, exclude the possibility of the Categories.)  As will easily be
shown in the next section, our ideas stand in need to our knowledge.
As any dedicated reader can clearly see, the Antinomies exist in our a
posteriori concepts, yet the thing in itself can not take account of,
as I have elsewhere shown, the Categories.  The question of this
matter's relation to objects is not in any way under discussion.}

\kgl_newpara:n {It must not be supposed that, so regarded, our
experience, in particular, can thereby determine in its totality our
analytic judgements, yet necessity has nothing to do with, in
reference to ends, the discipline of human reason.  It is not at all
certain that the never-ending regress in the series of empirical
conditions would thereby be made to contradict, in particular, pure
logic; with the sole exception of the Ideal, our ideas, that is to
say, should only be used as a canon for our judgements.  Since some of
the Antinomies are disjunctive, the Transcendental Deduction can be
treated like the never-ending regress in the series of empirical
conditions.  In the case of the Transcendental Deduction, it is not at
all certain that the Ideal of natural reason, in view of these
considerations, can be treated like the architectonic of human reason.
The Antinomies (and Aristotle tells us that this is the case) exclude
the possibility of the Ideal of human reason; in the case of the
discipline of natural reason, necessity would thereby be made to
contradict, so far as I know, the Ideal of pure reason.
Transcendental logic is a representation of the Transcendental
Deduction; by means of the transcendental aesthetic, the thing in
itself can thereby determine in its totality the Ideal of pure reason.
In my present remarks I am referring to the never-ending regress in
the series of empirical conditions only in so far as it is founded on
hypothetical principles.}

\kgl_newpara:n {The things in themselves prove the validity of, on the
other hand, transcendental logic; therefore, necessity has lying
before it, indeed, the paralogisms.  What we have alone been able to
show is that our ideas constitute a body of demonstrated doctrine, and
all of this body must be known a priori.  Our understanding has lying
before it, for these reasons, our ampliative judgements.  Because of
our necessary ignorance of the conditions, it is obvious that time may
not contradict itself, but it is still possible that it may be in
contradictions with, in view of these considerations, our ideas;
still, the practical employment of the transcendental objects in space
and time, that is to say, has lying before it the things in
themselves.  Natural causes prove the validity of necessity.}

\kgl_newpara:n {The reader should be careful to observe that our a
priori concepts, in other words, can never, as a whole, furnish a true
and demonstrated science, because, like general logic, they prove the
validity of hypothetical principles, by virtue of human reason.  There
can be no doubt that, indeed, the Antinomies, in other words, would be
falsified, and the phenomena constitute the whole content of the
discipline of natural reason.  The phenomena can not take account of,
in natural theology, the Ideal of practical reason.  Time can never
furnish a true and demonstrated science, because, like necessity, it
has nothing to do with a posteriori principles; in view of these
considerations, our a priori concepts stand in need to the discipline
of pure reason.  Our ideas constitute the whole content of the objects
in space and time, but the Ideal, indeed, is the key to understanding
our understanding.}

\kgl_newpara:n {As we have already seen, it is not at all certain that
the Ideal of pure reason is just as necessary as natural causes; in
the case of the Transcendental Deduction, our faculties, in natural
theology, abstract from all content of knowledge.  The Categories can
never, as a whole, furnish a true and demonstrated science, because,
like the manifold, they have lying before them a posteriori
principles, but time is by its very nature contradictory.  We can
deduce that the Categories, so regarded, are by their very nature
contradictory; for these reasons, time is what first gives rise to our
ideas.  Still, is it the case that pure logic constitutes the whole
content for the Transcendental Deduction, or is the real question
whether the paralogisms exist in our experience?  Still, natural
reason, so far as I know, would be falsified, because of our necessary
ignorance of the conditions.  Our faculties would be falsified.}

\kgl_newpara:n {The Ideal proves the validity of the objects in space
and time.  To avoid all misapprehension, it is necessary to explain
that our judgements are a representation of, however, the manifold.
The objects in space and time exclude the possibility of necessity.
The reader should be careful to observe that the Ideal, consequently,
abstracts from all content of knowledge.  There can be no doubt that,
indeed, the objects in space and time would thereby be made to
contradict human reason.}

\kgl_newpara:n {It is obvious that the transcendental unity of
apperception can be treated like the Ideal.  I assert that applied
logic (and it is not at all certain that this is true) stands in need
of the objects in space and time; certainly, the Ideal of practical
reason is what first gives rise to the Categories.  On the other hand,
our experience (and it remains a mystery why this is true) stands in
need of the transcendental unity of apperception.  It remains a
mystery why the Antinomies prove the validity of metaphysics.  There
can be no doubt that, in particular, the architectonic of pure reason,
in all theoretical sciences, can never furnish a true and demonstrated
science, because, like the manifold, it teaches us nothing whatsoever
regarding the content of hypothetical principles, but the phenomena,
with the sole exception of the transcendental aesthetic, have nothing
to do with philosophy.  It is obvious that our understanding, that is
to say, is the mere result of the power of space, a blind but
indispensable function of the soul, by means of analytic unity.  Since
knowledge of our sense perceptions is a priori, we can deduce that our
experience is what first gives rise to the architectonic of practical
reason.  This may be clear with an example.  }

\kgl_newpara:n {I assert, consequently, that the Transcendental
Deduction would thereby be made to contradict our faculties, as will
easily be shown in the next section.  Let us suppose that our ideas,
in the full sense of these terms, occupy part of the sphere of formal
logic concerning the existence of the noumena in general.  To avoid
all misapprehension, it is necessary to explain that the
Transcendental Deduction, so far as I know, occupies part of the
sphere of the architectonic of practical reason concerning the
existence of the Antinomies in general; certainly, the paralogisms
occupy part of the sphere of the architectonic of natural reason
concerning the existence of our ideas in general.  To avoid all
misapprehension, it is necessary to explain that the pure employment
of the architectonic of practical reason, still, is by its very nature
contradictory; consequently, the intelligible objects in space and
time would thereby be made to contradict the transcendental objects in
space and time.  We can deduce that the thing in itself exists in the
Antinomies.  As is evident upon close examination, the never-ending
regress in the series of empirical conditions depends on, therefore,
necessity.  I assert that our judgements are a representation of the
noumena; on the other hand, the transcendental unity of apperception
teaches us nothing whatsoever regarding the content of, then, the
Ideal of pure reason.}

\kgl_newpara:n {As is evident upon close examination, the things in
themselves are the clue to the discovery of the phenomena, and
philosophy (and what we have alone been able to show is that this is
true) teaches us nothing whatsoever regarding the content of the
phenomena.  Still, to avoid all misapprehension, it is necessary to
explain that natural causes (and it is obvious that this is the case)
have nothing to do with our faculties.  To avoid all misapprehension,
it is necessary to explain that, irrespective of all empirical
conditions, the employment of the objects in space and time can not
take account of, that is to say, our concepts, but the never-ending
regress in the series of empirical conditions constitutes the whole
content for our sense perceptions.  In the case of the discipline of
pure reason, let us suppose that general logic stands in need of the
Ideal of human reason, as we have already seen.  The noumena prove the
validity of, in the study of transcendental logic, our understanding.}

\kgl_newpara:n {Space (and what we have alone been able to show is
that this is true) stands in need of necessity, yet our understanding,
so far as regards the Ideal of practical reason, can never furnish a
true and demonstrated science, because, like the transcendental unity
of apperception, it has lying before it a priori principles.  Since
some of our judgements are disjunctive, it remains a mystery why the
phenomena stand in need to the objects in space and time.  In view of
these considerations, the Categories (and let us suppose that this is
the case) are just as necessary as the pure employment of the
phenomena.  Let us suppose that the things in themselves, so far as I
know, abstract from all content of a posteriori knowledge.  It is
obvious that, even as this relates to the thing in itself, natural
causes can never, as a whole, furnish a true and demonstrated science,
because, like metaphysics, they are just as necessary as inductive
principles.  The architectonic of practical reason (and it is not at
all certain that this is true) depends on the thing in itself, but the
objects in space and time, as I have elsewhere shown, are the mere
results of the power of the employment of the Antinomies, a blind but
indispensable function of the soul.  By means of analysis, there can
be no doubt that, in reference to ends, natural causes are a
representation of, in respect of the intelligible character, time, and
the pure employment of the discipline of natural reason has lying
before it our experience.}

\kgl_newpara:n {Still, it must not be supposed that our faculties are
a representation of the Ideal of practical reason, as is evident upon
close examination.  As is proven in the ontological manuals, the
reader should be careful to observe that the objects in space and time
are the mere results of the power of time, a blind but indispensable
function of the soul; in all theoretical sciences, the Ideal is a
representation of, so far as regards the architectonic of natural
reason, our sense perceptions.  Aristotle tells us that, in
particular, the objects in space and time, in the case of the
manifold, are a representation of the things in themselves, yet
natural causes stand in need to, irrespective of all empirical
conditions, the things in themselves.  Certainly, the transcendental
unity of apperception, in accordance with the principles of the
intelligible objects in space and time, exists in our sense
perceptions.  As we have already seen, the discipline of human reason
(and Galileo tells us that this is true) depends on the thing in
itself.  Since some of natural causes are synthetic, the reader should
be careful to observe that, for example, the things in themselves (and
it is not at all certain that this is the case) are the clue to the
discovery of our concepts.  But this need not worry us.}

\kgl_newpara:n {The architectonic of natural reason is the key to
understanding, so far as regards our a posteriori knowledge and the
paralogisms, time; still, the Categories, with the sole exception of
the never-ending regress in the series of empirical conditions, should
only be used as a canon for the transcendental unity of apperception.
However, the reader should be careful to observe that the noumena
exist in time.  Because of the relation between space and the
phenomena, let us suppose that our ideas are the clue to the discovery
of our faculties.  The phenomena constitute the whole content of the
phenomena, but the transcendental unity of apperception, on the other
hand, would be falsified.  (As is evident upon close examination, it
must not be supposed that our a posteriori knowledge is by its very
nature contradictory.)  There can be no doubt that the practical
employment of our problematic judgements can be treated like the
transcendental aesthetic.  Aristotle tells us that our faculties have
nothing to do with the objects in space and time.  We thus have a pure
synthesis of apprehension.}

\kgl_newpara:n {Since none of the noumena are hypothetical, there can
be no doubt that, in particular, our knowledge, in other words, is the
clue to the discovery of the things in themselves.  Therefore, the
Ideal is just as necessary as, then, the Ideal, as will easily be
shown in the next section.  We can deduce that, then, our knowledge,
in respect of the intelligible character, is by its very nature
contradictory, and the noumena, in particular, are by their very
nature contradictory.  The reader should be careful to observe that,
indeed, pure logic, still, is a body of demonstrated science, and none
of it must be known a posteriori, yet our speculative judgements exist
in the manifold.  In the case of time, the Categories, by means of
transcendental logic, constitute the whole content of the things in
themselves, as any dedicated reader can clearly see.}

\kgl_newpara:n {Transcendental logic can thereby determine in its
totality, consequently, our faculties, because of our necessary
ignorance of the conditions.  Since some of the paralogisms are
analytic, there can be no doubt that, in reference to ends, the
Antinomies, for these reasons, constitute the whole content of
necessity, yet the things in themselves constitute the whole content
of our understanding.  In view of these considerations, it is obvious
that the paralogisms are by their very nature contradictory, as any
dedicated reader can clearly see.  In natural theology, our ideas (and
it remains a mystery why this is the case) have nothing to do with the
discipline of pure reason, as any dedicated reader can clearly see.
What we have alone been able to show is that philosophy occupies part
of the sphere of the Transcendental Deduction concerning the existence
of natural causes in general.  Since knowledge of the phenomena is a
posteriori, our ideas, in all theoretical sciences, can be treated
like time, but our judgements are just as necessary as the Categories.
Our understanding is a representation of the objects in space and
time, and the paralogisms are just as necessary as our experience.}

\kgl_newpara:n {Philosophy (and it must not be supposed that this is
true) is a representation of the never-ending regress in the series of
empirical conditions; however, the Antinomies have nothing to do with,
in the study of philosophy, the discipline of practical reason.
Because of the relation between philosophy and our ideas, it remains a
mystery why, so regarded, metaphysics depends on the employment of
natural causes.  The pure employment of the Antinomies, in particular,
is a body of demonstrated science, and all of it must be known a
priori, but necessity is a representation of the Ideal.  As will
easily be shown in the next section, it remains a mystery why the
Antinomies are what first give rise to the transcendental aesthetic;
in all theoretical sciences, the architectonic of pure reason has
nothing to do with, therefore, the noumena.  The noumena are the clue
to the discovery of the Categories, yet the transcendental aesthetic,
for example, stands in need of natural causes.  The Categories can not
take account of, so far as regards the architectonic of natural
reason, the paralogisms; in the study of general logic, the
transcendental unity of apperception, insomuch as the architectonic of
human reason relies on the Antinomies, can thereby determine in its
totality natural causes.}

\kgl_newpara:n {As is shown in the writings of Hume, it remains a
mystery why our judgements exclude the possibility of the
transcendental aesthetic; therefore, the transcendental aesthetic can
not take account of the thing in itself.  Our knowledge depends on,
indeed, our knowledge.  It is not at all certain that space is just as
necessary as the noumena.  Is it true that metaphysics can not take
account of the paralogisms of human reason, or is the real question
whether the noumena are by their very nature contradictory?  On the
other hand, time constitutes the whole content for necessity, by means
of analytic unity.  There can be no doubt that the phenomena have
lying before them metaphysics.  As is proven in the ontological
manuals, it remains a mystery why space exists in the objects in space
and time; still, the noumena, in the case of necessity, constitute the
whole content of philosophy.}

%    \end{macrocode}
% 
% Now we define the sequence of index words.
%    \begin{macrocode}
\kgl_newword:n {Ideal}
\kgl_newword:n {noumena}
\kgl_newword:n {Aristotle}
\kgl_newword:n {transcendental}
\kgl_newword:n {metaphysics}
\kgl_newword:n {reason}
\kgl_newword:n {science}
\kgl_newword:n {necessity}
\kgl_newword:n {Categories}
\kgl_newword:n {philosophy}
\kgl_newword:n {knowledge}
\kgl_newword:n {regress}
\kgl_newword:n {paralogism}
\kgl_newword:n {empirical}
\kgl_newword:n {space}
\kgl_newword:n {manifold}
\kgl_newword:n {understanding}
\kgl_newword:n {aesthetic}
\kgl_newword:n {noumena}
\kgl_newword:n {sphere}
\kgl_newword:n {time}
\kgl_newword:n {practical reason}
\kgl_newword:n {perception}
\kgl_newword:n {things in themselves}
\kgl_newword:n {doctrine}
\kgl_newword:n {regress}
\kgl_newword:n {mystery}
\kgl_newword:n {existence}
\kgl_newword:n {contradiction}
\kgl_newword:n {a priori}
\kgl_newword:n {natural causes}
\kgl_newword:n {analysis}
\kgl_newword:n {apperception}
\kgl_newword:n {Antinomies}
\kgl_newword:n {Transcendental Deduction}
\kgl_newword:n {phenomena}
\kgl_newword:n {formal logic}
\kgl_newword:n {soul}
\kgl_newword:n {misapprehension}
\kgl_newword:n {elsewhere}
\kgl_newword:n {theology}
\kgl_newword:n {employment}
\kgl_newword:n {logic}
\kgl_newword:n {practical reason}
\kgl_newword:n {theoretical sciences}
\kgl_newword:n {a posteriori}
\kgl_newword:n {mystery}
\kgl_newword:n {philosophy}
\kgl_newword:n {things in themselves}
\kgl_newword:n {experience}
\kgl_newword:n {contradictory}
\kgl_newword:n {Categories}
\kgl_newword:n {perceptions}
\kgl_newword:n {Galileo}
\kgl_newword:n {apperception}
\kgl_newword:n {empirical objects}
\kgl_newword:n {judgements}
\kgl_newword:n {phenomena}
\kgl_newword:n {power}
\kgl_newword:n {hypothetical principles}
\kgl_newword:n {transcendental logic}
\kgl_newword:n {doctrine}
\kgl_newword:n {understanding}
\kgl_newword:n {totality}
\kgl_newword:n {manifold}
\kgl_newword:n {inductive judgements}
\kgl_newword:n {Transcendental Deduction}
\kgl_newword:n {analytic unity}
\kgl_newword:n {Hume}
\kgl_newword:n {canon}
\kgl_newword:n {knowledge}
\kgl_newword:n {universal}
\kgl_newword:n {section}
\kgl_newword:n {body}
\kgl_newword:n {ignorance}
\kgl_newword:n {sense perceptions}
\kgl_newword:n {natural reason}
\kgl_newword:n {exception}
\kgl_newword:n {ampliative judgements}
\kgl_newword:n {experience}
\kgl_newword:n {Categories}
\kgl_newword:n {analysis}
\kgl_newword:n {philosophy}
\kgl_newword:n {apperception}
\kgl_newword:n {paralogism}
\kgl_newword:n {ignorance}
\kgl_newword:n {true}
\kgl_newword:n {space}
\kgl_newword:n {Ideal}
\kgl_newword:n {accordance}
\kgl_newword:n {regress}
\kgl_newword:n {experience}
\kgl_newword:n {a priori}
\kgl_newword:n {disjunctive}
\kgl_newword:n {soul}
\kgl_newword:n {understanding}
\kgl_newword:n {analytic unity}
\kgl_newword:n {phenomena}
\kgl_newword:n {practical reason}
\kgl_newword:n {cause}
\kgl_newword:n {manuals}
\kgl_newword:n {dedicated reader}
\kgl_newword:n {a posteriori}
\kgl_newword:n {employment}
\kgl_newword:n {natural theology}
\kgl_newword:n {manifold}
\kgl_newword:n {transcendental aesthetic}
\kgl_newword:n {close}
\kgl_newword:n {full}
\kgl_newword:n {Aristotle}
\kgl_newword:n {clue}
\kgl_newword:n {me}
\kgl_newword:n {account}
\kgl_newword:n {things}
\kgl_newword:n {sense}
\kgl_newword:n {intelligible}
\kgl_newword:n {understanding}
\kgl_newword:n {Categories}
\kgl_newword:n {never}
\kgl_newword:n {apperception}
\kgl_newword:n {Ideal}
\kgl_newword:n {need}
\kgl_newword:n {space}
\kgl_newword:n {virtue}
\kgl_newword:n {Hume}
\kgl_newword:n {still}
\kgl_newword:n {whatsoever}
\kgl_newword:n {even}
\kgl_newword:n {sphere}
\kgl_newword:n {position}
\kgl_newword:n {ignorance}
\kgl_newword:n {word}
\kgl_newword:n {phenomena}
\kgl_newword:n {theology}
\kgl_newword:n {mystery}
\kgl_newword:n {Categories}
\kgl_newword:n {perception}
\kgl_newword:n {power}
\kgl_newword:n {experience}
\kgl_newword:n {never-ending}
\kgl_newword:n {analytic}
\kgl_newword:n {itself}
\kgl_newword:n {a priori}
\kgl_newword:n {rule}
\kgl_newword:n {Transcendental Deduction}
\kgl_newword:n {empirical conditions}
\kgl_newword:n {knowledge}
\kgl_newword:n {disjunctive}
\kgl_newword:n {transcendental}
\kgl_newword:n {science}
\kgl_newword:n {falsified}
\kgl_newword:n {reader}
\kgl_newword:n {blind}
\kgl_newword:n {employment}
\kgl_newword:n {discipline}
\kgl_newword:n {function}
\kgl_newword:n {careful}
\kgl_newword:n {Aristotle}
\kgl_newword:n {Categories}
\kgl_newword:n {part}
\kgl_newword:n {noumena}
\kgl_newword:n {doubt}
\kgl_newword:n {duck}
\kgl_newword:n {Kant}
%    \end{macrocode}
%
% Finally we close the group and issue a message in the log file
% stating how many sentences are available.
%    \begin{macrocode}
\group_end:
\msg_info:nnx {kantlipsum} {how-many}
  { \int_eval:n {\seq_count:N \g_kgl_pars_seq} }
%    \end{macrocode}
%
% \iffalse
%</package>
% \fi
% \end{implementation}
%
% \PrintIndex
