\documentclass[10pt,a4paper,final]{article}

\usepackage{listings}
\usepackage[protrusion=true,draft=false,final,verbose=true]{microtype}

\title{The \textbf{plantslabels} package}
\author{Merciadri Luca}
\date{\today}

%% - HYPERREF PACKAGE - ** MUST be LAST ONE **
\usepackage[a4paper,bookmarks=true,bookmarksnumbered=true,bookmarksopen=true,bookmarksopenlevel=1,breaklinks=true,colorlinks=true,final,menucolor=red,pdfauthor={Merciadri Luca},pdfcreator={Merciadri Luca},pdfkeywords={plants},pdftitle={The plantslabels package},pdfsubject={(La)TeX},pdftoolbar=true]{hyperref}
\hypersetup{urlcolor=red,linkcolor=blue,citecolor=blue,colorlinks=true}

\usepackage{breakurl}

\begin{document}


\maketitle

\tableofcontents

\newpage
\section{Introduction}
This package (\verb v1.0 ) \textit{helps you writing plants' labels} when needed. For example, you may want to give a label to each plant of your collection.

\section{Use}
\subsection{Loading the Package}
To \textit{load the package}, please use
\begin{center}
\begin{verbatim}
\usepackage{plantslabels}
\end{verbatim}
\end{center}
\subsection{Available Options}
The set of options is currently empty.

\newpage

\section{Examples}
There is only one command in this package: \verb \plant . This command takes $9$ arguments, and only the three first are mandatory. Here is the syntax:
\begin{center}
\begin{verbatim}
   \plant{cols_labels}{rows_labels}{no_labels}{generic_plant_name}
   {generic_price}{generic_currency}{$generic_temperature$}
   {generic_substratum}{generic_picture}
\end{verbatim}
\end{center}
where
\begin{enumerate}
 \item \verb cols_labels ~is the number of cols of labels,\marginpar{Mandatory!}
 \item \verb rows_labels ~is the number of rows of labels,\marginpar{Mandatory!}
 \item \verb no_labels ~is the number of labels (under the condition $\texttt{cols\_labels}\times\texttt{rows\_labels}=\texttt{no\_labels}$),\marginpar{Mandatory!}
 \item \verb generic_plant_name ~is the plant's name which will be written on each of the \verb no_labels ~labels,
 \item \verb generic_price ~is the plant's price which will be written on each of the \verb no_labels ~labels,
 \item \verb generic_currency ~is the price currency which will be written on each of the \verb no_labels ~labels, after \verb generic_price ,
 \item \verb $generic_temperature$ ~is the temperature which will be written on each of the \verb no_labels ~labels (it should be $t_{\mathrm{min}}\to t_{\mathrm{max}}$, \textit{i.e.} the min and max temperatures for the plant),
 \item \verb generic_substratum ~is the plant's substratum which will be written on each of the \verb no_labels ~labels,
 \item \verb generic_picture ~is the plant's picture which will be drawn on each of the \verb no_labels ~labels.
\end{enumerate}

As all the arguments after \verb no_labels ~are not mandatory, you can skip them. For this, you need to write brackets, though. For example,
\begin{center}
\begin{verbatim}
   \plant{cols_labels}{rows_labels}{no_labels}{Plant}{}{}{}{}{}
\end{verbatim}
\end{center}
will simply draw one \verb no_labels ~($=\texttt{cols\_labels}\times \texttt{rows\_labels}$) labels  with ``Plant'' into it.

\subsection{Practical Example}
Let's say that you have two kinds of plants that you want to label: ``Myplant1'' and ``Myplant2.'' One habitually lives in the desert, and the other lives in tropical regions. You have, say, $2$ specimens of the first, and $4$ of the second. You can invoke, assuming \verb cactus.eps ~is your image for the first one, that you have no image for the second one, and that they respect the conditions mentioned below:
\begin{verbatim}
 \plant{1}{1}{2}{Myplant1}{5}{EUR}{$-10\to +50$}{Peat moss, sand,
 perlite}{cactus.eps}
 \plant{2}{2}{4}{Myplant2}{10}{EUR}{$20\to +40$}{Peat moss,
 fertilizer}{}
\end{verbatim}


\newpage
\section{Implementation}

Here is the code of \verb plantslabels.sty :
\lstset{language=TEX, basicstyle=\tiny, keywordstyle=\bfseries, commentstyle=\itshape, keywords={}, emph={}, emphstyle=\bfseries, numbers=left, stringstyle=\ttseries, showstringspaces=false, stepnumber=2, numbersep=5pt, showspaces=false, showtabs=false, backgroundcolor=\color{white}}

%\begin{lstlisting}[frame=single]
\lstinputlisting[lastline=95]{plantslabels.forlisting}
%\end{lstlisting}


%\newpage
\section{Limitations}
This package has currently no limitation.

\section{Remarks}
The temperature unit is habitually so obvious that you do not need to specify it manually.

\section{Bugs}
Not yet.

\section{Version History}
\begin{enumerate}
 \item \verb v1.0 : package is introduced to the \LaTeX{} world.
\end{enumerate}


\section{Contact}
If you have any question concerning this package (limitations, bugs, \ldots), please contact me at \href{mailto:Luca.Merciadri@student.ulg.ac.be}{Luca.Merciadri@student.ulg.ac.be}.


\section{Credits}
Thanks to Philipp Stephani and \textbf{Enrico Gregorio} for their answers at
\begin{center}
\url{http://groups.google.com/group/comp.text.tex/browse_thread/thread/5703b5328b93a000#}.
\end{center}


\end{document}