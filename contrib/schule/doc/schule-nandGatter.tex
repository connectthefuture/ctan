\documentclass[a4paper]{scrartcl}
\usepackage[utf8]{inputenc}
\usepackage[ngerman]{babel}
%\usepackage[usenames,dvipsnames,svgnames,table]{xcolor}
\definecolor{blau}{rgb}{0,0,0.75}         
\definecolor{orange}{rgb}{0.8,0.3,0}      
\usepackage{hyperref}
\hypersetup{
 pdftitle = {\LaTeX-Klassen und Pakete für den Einsatz im Bereich der Schule},
 pdfsubject = {},
 pdfauthor = {Johannes Pieper, Johannes Kuhaupt, Ludger Humbert, Andr\'e Hilbig},
 colorlinks = true,
 hypertexnames = true,
 linkcolor=blau, %
 filecolor=orange, %
 citecolor=blau,
 menucolor=orange, %
 urlcolor=orange,
 breaklinks=true
}
\usepackage{caption,xparse,xargs}
\usepackage[german=guillemets]{csquotes}
\usepackage{schule,syntaxdi,schulinf,schulphy}
\usepackage{placeins,float,prettyref}
\usepackage{newfloat}

\lstset{  %
	language=[LaTeX]TeX,                 % the language of the code
	basicstyle=\small,            % the size of the fonts that are used for the code
	numbers=left,                    % where to put the line-numbers
	numberstyle=\footnotesize,           % the size of the fonts that are used for the line-numbers
	stepnumber=2,                    % the step between two line-numbers. If it's 1, each line will be numbered
	numbersep=5pt,                   % how far the line-numbers are from the code
	backgroundcolor=\color{white},       % choose the background color. You must add \usepackage{color}
	showspaces=false,                % show spaces adding particular underscores
	showstringspaces=false,          % underline spaces within strings
	showtabs=false,                  % show tabs within strings adding particular underscores
	frame=false,                    % adds a frame around the code
	tabsize=2,                       % sets default tabsize to 2 spaces
	resetmargins=true,
	captionpos=b,                    % sets the caption-position to bottom
	title=,                    % show the filename of files included with \lstinputlisting;
	breaklines=true,
	breakautoindent=true,
	prebreak=\mbox{ $\curvearrowright$},
	postbreak=\mbox{$\rightsquigarrow$ },
	linewidth=\columnwidth,
	breakatwhitespace=true,         % sets if automatic breaks should only happen at whitespace 
	numberstyle=\tiny\color{gray},         % line number style
	keywordstyle=\color{blue},           % keyword style
	commentstyle=\color{OliveGreen},        % comment style
	stringstyle=\color{mauve},          % string literal style
	morekeywords={
	zeitpunkt, punkteitem, scaleSequenzdiagramm, newthread, newthreadtwo, 
	newinst, node, chainin, draw, to, dokName, jahrgang, minisec, subsection, 
	glqq, grqq, euro
}                % if you want to add more keywords to the set
}
\newcommand{\materialsammlung}{\url{http://ddi.uni-wuppertal.de/material/materialsammlung/index.html}}

\begin{document}
 \section*{NAND-Schaltpläne mit dem Paket \texttt{relaycircuit} erstellen}
		\begin{lstlisting}[gobble=7,multicols=2,basicstyle=\footnotesize,caption={}]
			\begin{tikzpicture}
				\draw (0,6.8) node [left] {\(+\)} 
					-- (9,6.8);
				\draw (0,0) node [left] {\(-\)} 
					-- (9,0);
				\draw (4.5,0) to[short, *-] 
					(4.5,0) node [ground] {};
		
				\draw (7.4,2.5) to[short,*-] 
					(7.5,2.5) to[lamp] (9,2.5) 
					node[ground] {};

				\draw (2.5,5.8) node[arbeits 
					relais] (a1) {};
				\draw (2.5,4) node[arbeits relais] 
					(a2) {};
				\draw (2.4,6.8) to[short,*-] 
					(a1.anschluss);
				\draw (a1.ausgabe) -- 
					(a2.anschluss);

				\draw (2.5,1) node[ruhe relais] 
					(r1) {};
				\draw (a2.ausgabe) -- 
					(r1.anschluss);
				\draw (r1.ausgabe) to[short,-*] 
					(2.4,0);
				\draw (5,1) node[ruhe relais] 
					(r2) {};
				\draw (r2.ausgabe) to[short,-*] 
					(4.9,0);

				\draw (7.5,1) node[arbeits relais] 
					(a3) {};
				\draw (7.5,4) node[ruhe relais] 
					(r3) {};
				\draw (a3.anschluss) -- 
					(r3.ausgabe);
				\draw (a3.ausgabe) to[short,-*] 
					(7.4,0);
				\draw (r3.anschluss) to[short,-*] 
					(7.4,6.8);

				\draw (2.4,2.5) to[short,*-*] 
					(4.9,2.5) -| (a3.eingabe);
				\draw (r2.anschluss) |- 
					(r3.eingabe);

				\draw (0,4.7) node [left] {A} 
					to[short,-*] (0.2,4.7) 
					-- (a2.eingabe);
				\draw (0.2,4.7) |- (r1.eingabe);

				\draw (0,2.1) node [left] {B} 
					to[short,-*] (0.4,2.1)
					-| (r2.eingabe);
				\draw (0.4,2.1) |- (a1.eingabe);
			\end{tikzpicture}
		\end{lstlisting}
  \vspace{0.6cm}
  \hrule width \textwidth
  \vspace{1cm}
	\scalebox{0.8}{
		\begin{tikzpicture}
			\draw (0,6.8) node [left] {\(+\)} -- (9,6.8);
			\draw (0,0) node [left] {\(-\)} -- (9,0);
			\draw (4.5,0) to[short, *-] (4.5,0) node [ground] {};
		
			\draw (7.4,2.5) to[short,*-] (7.5,2.5) to[lamp] (9,2.5) node[ground] {};

			\draw (2.5,5.8) node[arbeits relais] (a1) {};
			\draw (2.5,4) node[arbeits relais] (a2) {};
			\draw (2.4,6.8) to[short,*-] (a1.anschluss);
			\draw (a1.ausgabe) -- (a2.anschluss);

			\draw (2.5,1) node[ruhe relais] (r1) {};
			\draw (a2.ausgabe) -- (r1.anschluss);
			\draw (r1.ausgabe) to[short,-*] (2.4,0);
			\draw (5,1) node[ruhe relais] (r2) {};
			\draw (r2.ausgabe) to[short,-*] (4.9,0);

			\draw (7.5,1) node[arbeits relais] (a3) {};
			\draw (7.5,4) node[ruhe relais] (r3) {};
			\draw (a3.anschluss) -- (r3.ausgabe);
			\draw (a3.ausgabe) to[short,-*] (7.4,0);
			\draw (r3.anschluss) to[short,-*] (7.4,6.8);

			\draw (2.4,2.5) to[short,*-*] (4.9,2.5) -| (a3.eingabe);
			\draw (r2.anschluss) |- (r3.eingabe);

			\draw (0,4.7) node [left] {A} to[short,-*] (0.2,4.7) -- (a2.eingabe);
			\draw (0.2,4.7) |- (r1.eingabe);

			\draw (0,2.1) node [left] {B} to[short,-*] (0.4,2.1) -| (r2.eingabe);
			\draw (0.4,2.1) |- (a1.eingabe);
		\end{tikzpicture}
	}
\end{document}
