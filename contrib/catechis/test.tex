% +AMDG  See catechis.sty for dedication of this test 
% file.  Pray for us!  Have mercy on us!
% 
% This document is copyright 2008 by Donald P. Goodman, and is
% released publicly under the LaTeX Project Public License.  The
% distribution and modification of this work is constrained by the
% conditions of that license.  See
% 	http://www.latex-project.org/lppl.txt
% for the text of the license.  This document is released
% under version 1.3 of that license, and this work may be distributed
% or modified under the terms of that license or, at your option, any
% later version.
% 
% This work has the LPPL maintenance status 'maintained'.
% 
% The Current Maintainer of this work is Donald P. Goodman.
% 
% This work consists of the files catechis.ins, catechis.sty, and
% test.tex, along with README.
\documentclass[12pt]{article}
\title{Testing the \texttt{catechis} Package}
\author{Donald P.\ Goodman III}
\date{\today}

\usepackage{catechis}

\begin{document}
\maketitle

\catques{Who made us?}{God made us.}

\catques{Why did God make us?}{God made us to know Him, love Him, 
and serve Him, and by so doing to gain Heaven.}
\catques{Are you going to write any actual examples, or just copy
them all from other catechisms?}{Wait for it; I'm going to write
some of my own.  The above are verbatim or nearly verbatim from the
Baltimore Catechism, for those who do not know.}
\catques{Did you make a package for writing catechisms?  For 
serious?}{Yes, I did.  I had a number of reasons for this:
\begin{enumerate}
\item I wanted to typeset one or two for myself.
\catcomment{This although I've already done so in a very hacked,
difficult to read and difficult to write style.  But at least I'm
(hopefully) making it easier for others.}
\begin{enumerate}
\item There's one in \textit{Thomistic Salvation}.
\catcomment{And it's not half bad.  I think it addresses the issues it
was meant to address pretty thoroughly, in a basic, catechism sort of
way.}
\item There's also one in \textit{Officium Parvum}.\label{testlabel}
\catcomment{This one could use some work.  But it still isn't
terrible.}
\end{enumerate}
\item It's a part of my plan to make a complete Catholic 
desktop (though what that plan is, exactly, is still somewhat 
up in the air).
\catcomment{I'm trying to flesh this idea out a little more.  It will
at least contain a couple of \LaTeX\ packages and a multilingual
Scripture-searching tool; hopefully also a Catholic calendar desklet
and similar items.}
\end{enumerate}
} % 

\catques{All right, I guess so.  But it still seems weird.}{Maybe.  
But I think (hope) it'll bear fruit.
\catcomment{And I'm very hopeful.  Hope is, after all, one of the
three theological virtues, and mentioned repeatedly in the Scriptures
as necessary for us to have.}
}% 

\catques{Here's a really long question that I'm trying to make 
wrap onto another line so that we can see how that will behave.  
Will it behave well?}{Yes, it will behave well.  The algorithm 
is reasonably obedient and well-behaved.}
\catques{What about the Scripture citations you often see in
catechisms?  Can this hooty-falooty package of yours handle those?}{%
I'm glad you asked.  Yes, this package can easily handle those.
\catcomment{Even in conjunction with a catcomment, you'll notice;
versatility (and ease of goal) is always the key.}
\citetitle
\scripture{In the beginning was the Word, and the Word was with God,
and the Word was God.}{St.~John 1:1.}
\scripture{Where wast thou when I laid up the foundations of the 
earth? tell me if thou hast understanding.}{Job 38:4}
}%
\catques{Finally, how do the labels work with the hacked
enumerates?}{Well, they work all right, thanks to \texttt{varioref}; 
in fact, here's one:  \ref{testlabel}.}
\catques{I want to see some double-digit question numbers.}{I'm
working on it; only one more question to go.}
\catques{Finally; do they indent properly and all, at least with the
defaults?}{Yes, they indent exactly as one would expect.
\catcomment{Although in this, my first real foray into serious
\LaTeX\ programming, I found that very little works as expected, and
one must have a very thorough knowledge of internals to avoid being
surprised and having to work quite hard to figure out some unexpected
results.}
}
 
\end{document}
