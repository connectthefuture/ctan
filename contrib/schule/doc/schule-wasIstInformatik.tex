\documentclass{schullzk}
\usepackage[utf8]{inputenc}
\usepackage[ngerman]{babel}
\definecolor{blau}{rgb}{0,0,0.75}         
\definecolor{orange}{rgb}{0.8,0.3,0}      
\usepackage{hyperref}
\hypersetup{
 pdftitle = {\LaTeX-Klassen und Pakete für den Einsatz im Bereich der Schule},
 pdfsubject = {},
 pdfauthor = {Johannes Pieper, Johannes Kuhaupt, Ludger Humbert, Andr\'e Hilbig},
 colorlinks = true,
 hypertexnames = true,
 linkcolor=blau, %
 filecolor=orange, %
 citecolor=blau,
 menucolor=orange, %
 urlcolor=orange,
 breaklinks=true
}
\usepackage{caption,xparse,xargs}
\usepackage[german=guillemets]{csquotes}
\usepackage{schule,syntaxdi,schulinf,schulphy}
\usepackage{placeins,float,prettyref}
\usepackage{newfloat}

\lstset{  %
	language=[LaTeX]TeX,                 % the language of the code
	basicstyle=\small,            % the size of the fonts that are used for the code
	numbers=left,                    % where to put the line-numbers
	numberstyle=\footnotesize,           % the size of the fonts that are used for the line-numbers
	stepnumber=2,                    % the step between two line-numbers. If it's 1, each line will be numbered
	numbersep=5pt,                   % how far the line-numbers are from the code
	backgroundcolor=\color{white},       % choose the background color. You must add \usepackage{color}
	showspaces=false,                % show spaces adding particular underscores
	showstringspaces=false,          % underline spaces within strings
	showtabs=false,                  % show tabs within strings adding particular underscores
	frame=false,                    % adds a frame around the code
	tabsize=2,                       % sets default tabsize to 2 spaces
	resetmargins=true,
	captionpos=b,                    % sets the caption-position to bottom
	title=,                    % show the filename of files included with \lstinputlisting;
	breaklines=true,
	breakautoindent=true,
	prebreak=\mbox{ $\curvearrowright$},
	postbreak=\mbox{$\rightsquigarrow$ },
	linewidth=\columnwidth,
	breakatwhitespace=true,         % sets if automatic breaks should only happen at whitespace 
	numberstyle=\tiny\color{gray},         % line number style
	keywordstyle=\color{blue},           % keyword style
	commentstyle=\color{OliveGreen},        % comment style
	stringstyle=\color{mauve},          % string literal style
	morekeywords={
	zeitpunkt, punkteitem, scaleSequenzdiagramm, newthread, newthreadtwo, 
	newinst, node, chainin, draw, to, dokName, jahrgang, minisec, subsection, 
	glqq, grqq, euro
}                % if you want to add more keywords to the set
}
\newcommand{\materialsammlung}{\url{http://ddi.uni-wuppertal.de/material/materialsammlung/index.html}}

\inhalt{Definition Informatik}
\begin{document}
 \thispagestyle{empty}
 \section*{Beispiel für eine Lernzielkontrolle in
 Informatik zum Thema »Was ist Informatik?«}
 \footnotesize{(entnommen aus \materialsammlung)}
 % Der folgende kenntlich gemachte Abschnitt ist in der Zusammenarbeit von 
 % Informatikreferendaren und ehemaligen Informatikreferendaren der 
 % Studienseminare (heute ZfsL) Arnsberg, Hamm und Solingen entstanden.
 %
 % Der Abschnitt steht unter der Lizenz: Creative Commons by-nc-sa Version 4.0
 % http://creativecommons.org/licenses/by-nc-sa/4.0/deed.de
 %
 % Nach dieser Lizenz darf der Abschnitt beliebig kopiert und bearbeitet werden,
 % sofern das Folgeprodukt wiederum unter gleichen Lizenzbedingungen vertrieben
 % und auf die ursprünglichen Urheber verwiesen wird.
 % Eine kommerzielle Nutzung ist ausdrücklich ausgeschlossen.
 %
 % Die Namensnennung durch einen Verweis und die Lizenzangabe der ursprünglichen
 % Urheber auf den Materialien für Schülerinnen und Schüler ist erforderlich.
 %
 % Die vollständige Sammlung der Dokumente steht unter
 % http://ddi.uni-wuppertal.de/material/materialsammlung/
 % zur Verfügung.
 %
 % Das LaTeX-Paket zum Setzen der Dokumente der Sammlung steht
 % unter  http://www.ctan.org/pkg/
 % zur Verfügung.
 \begin{lstlisting}[gobble=0,multicols=2,basicstyle=\footnotesize,caption={}]
\documentclass{schullzk}
\usepackage[utf8]{inputenc}
\inhalt{Definition Informatik}
\begin{document}
 \punktesec{Aufgabe 1}
 \begin{aufgaben}
   \punkteitem{8} \textbf{
      Informatik -- zum
      Begriff}
   \begin{enumerate}
    \item Geben Sie \textbf{Ihre}
      Definition für Informatik an.
    \item Ordnen Sie die folgenden
      beiden Aussagen einer der
      Ebenen \textbf{Pragmatik,
      Syntax} oder \textbf{
      Semantik} zu:
      \begin{itemize}
        \item »Eine Studentin 
          sucht Literatur zu 
          einem bestimmten
          Thema.«
        \item »Bildarchive werden 
          häufig von
          Journalistinnen
          in Anspruch genommen, 
          um einen Artikel zu
          illustrieren; dabei 
          ist meist das Thema
          vorgegeben, aber 
          nicht der Bildinhalt.«
      \end{itemize}
    \item Benennen Sie die 
       Fachgebiete, in die
       Informatik 
       üblicherweise
       aufgeteilt wird.
    \item Ordnen Sie die folgenden
       Begriffe den von Ihnen 
       in 1\,c) genannten 
       Fachgebieten zu:
	
       Fahrtroutenoptimierung, 
       Software,
       Programmiersprache,
       Datenschutz, Linux, 
       MP3-Player
    \item Grenzen Sie die Begriffe
       \textbf{Information, Daten}
       und \textbf{Wissen}
       voneinander ab.
   \end{enumerate}
  \punkteitem{8} \textbf{Informatik 
    -- zum Begriff}
   \begin{enumerate}
    \item Grenzen Sie die Begriffe 
      \textbf{Semantik, Pragmatik,
      Syntax} voneinander ab.
    \item Nennen Sie die
      Fachgebiete der Informatik
      und ordnen Sie die folgenden
      Begriffe zu: 
      Programmiersprache~Python,
      Datenbank, 
      Persönlichkeitsschutz,
      Informatische Bildung,
      Hardware, Betriebssystem
    \item Ordnen Sie die folgende
      Aussage einer der Ebenen
      \textbf{Daten}, \textbf{
      Wissen}, \textbf{Information}
      zu: »Ein Dokument wird als 
      Folge von Zeichen/Symbolen 
      aufgefasst. Auf dieser Ebene
      kann beispielsweise mit 
      Methoden agiert werden, die
      Zeichenketten in Texten oder
      die nach Merkmalen wie Farbe,
      Textur und Kontur suchen.«
    \item Geben Sie \textbf{Ihre}
      Definition für Informatik an.
   end{enumerate}
 \end{aufgaben}
\end{document}
		\end{lstlisting}
\clearpage
 \punktesec{Aufgabe 1}
 \begin{aufgaben}
   \punkteitem{8} \textbf{
      Informatik -- zum
      Begriff}
   \begin{enumerate}
    \item Geben Sie \textbf{Ihre}
      Definition für Informatik an.
    \item Ordnen Sie die folgenden
      beiden Aussagen einer der
      Ebenen \textbf{Pragmatik,
      Syntax} oder \textbf{
      Semantik} zu:
      \begin{itemize}
        \item »Eine Studentin 
          sucht Literatur zu 
          einem bestimmten
          Thema.«
        \item »Bildarchive werden 
          häufig von
          Journalistinnen
          in Anspruch genommen, 
          um einen Artikel zu
          illustrieren; dabei 
          ist meist das Thema
          vorgegeben, aber 
          nicht der Bildinhalt.«
      \end{itemize}
    \item Benennen Sie die 
       Fachgebiete, in die
       Informatik 
       üblicherweise
       aufgeteilt wird.
    \item Ordnen Sie die folgenden
       Begriffe den von Ihnen 
       in 1\,c) genannten 
       Fachgebieten zu:
	
       Fahrtroutenoptimierung, 
       Software,
       Programmiersprache,
       Datenschutz, Linux, 
       MP3-Player
    \item Grenzen Sie die Begriffe
       \textbf{Information, Daten}
       und \textbf{Wissen}
       voneinander ab.
   \end{enumerate}
  \punkteitem{8} \textbf{Informatik 
    -- zum Begriff}
   \begin{enumerate}
    \item Grenzen Sie die Begriffe 
      \textbf{Semantik, Pragmatik,
      Syntax} voneinander ab.
    \item Nennen Sie die
      Fachgebiete der Informatik
      und ordnen Sie die folgenden
      Begriffe zu: 
      Programmiersprache~Python,
      Datenbank, 
      Persönlichkeitsschutz,
      Informatische Bildung,
      Hardware, Betriebssystem
    \item Ordnen Sie die folgende
      Aussage einer der Ebenen
      \textbf{Daten}, \textbf{
      Wissen}, \textbf{Information}
      zu: »Ein Dokument wird als 
      Folge von Zeichen/Symbolen 
      aufgefasst. Auf dieser Ebene
      kann beispielsweise mit 
      Methoden agiert werden, die
      Zeichenketten in Texten oder
      die nach Merkmalen wie Farbe,
      Textur und Kontur suchen.«
    \item Geben Sie \textbf{Ihre}
      Definition für Informatik an.
   \end{enumerate}
 \end{aufgaben}
\end{document}
