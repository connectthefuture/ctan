\ProvidesFile{niceverb.tex}[2014/03/28 documenting niceverb.sty]
\title{\textsf{niceverb.sty}\\---\\Minimizing 
  Markup\\for Documenting \LaTeX\ packages\thanks{This 
    document describes version 
    \textcolor{blue}{\UseVersionOf{niceverb.sty}}
    of \pkgnamefmt{niceverb.sty} as of \UseDateOf{niceverb.sty}.}
}
% \listfiles 2010/03/19
{ \RequirePackage{makedoc} \ProcessLineMessage{} %% 2010/03/11
  \MakeJobDoc{19}{\SectionLevelThreeParseInput}  }
\documentclass[fleqn]{article}%% TODO paper dimensions!?
\ProvidesFile{makedoc.cfg}[2011/06/27 documentation settings] 

\author{Uwe L\"uck\thanks{\url{http://contact-ednotes.sty.de.vu}}}
% \author{Uwe L\"uck---{\tt http://contact-ednotes.sty.de.vu}}

%% hyperref:
\RequirePackage{ifpdf}
\usepackage[%
  \ifpdf
%     bookmarks=false,          %% 2010/12/22
%     bookmarksnumbered,
    bookmarksopen,              %% 2011/01/24!?
    bookmarksopenlevel=2,       %% 2011/01/23
%     pdfpagemode=UseNone,
%     pdfstartpage=10,
%     pdfstartview=FitH,
    citebordercolor={ .6 1    .6},
    filebordercolor={1    .6 1},
    linkbordercolor={1    .9  .7},
     urlbordercolor={ .7 1   1},   %% playing 2011/01/24
  \else
    draft
  \fi
]{hyperref}

\RequirePackage{niceverb}[2011/01/24] 
\RequirePackage{readprov}               %% 2010/12/08
\RequirePackage{hypertoc}               %% 2011/01/23
\RequirePackage{texlinks}               %% 2011/01/24
\makeatletter
  \@ifundefined{strong} 
               {\let\strong\textbf}     %% 2011/01/24
               {} 
  \@ifundefined{file} 
               {\let\file\texttt}       %% 2011/05/23
               {} 
\makeatother

\errorcontextlines=4
\pagestyle{headings}

\endinput

 %% shared formatting settings
\newcommand*{\secref}[1]{Sec.~\ref{sec:#1}}         %% 2014/03/27
%% 2011/08/22:
\MDkeywords{literate programming, syntactic sugar,
            .txt to .tex enhancement, macro programming}
\hypersetup{%% was `syntacic' 2011/10/07:
%     pdftitle=syntactic sugar for LaTeX documentation by niceverb.sty, 
    %% <- 2011/11/05 -> 
    pdftitle=niceverb.sty: syntactic sugar for LaTeX documentation,
    pdfsubject=documenting niceverb.sty
}%% /2011/08/22
\begin{document}
\maketitle
\begin{MDabstract}
%   \tracingmacros=1 \tracingonline=1
'niceverb.sty' provides very decent syntax (through active characters) 
for describing \LaTeX\ packages and the syntax of macros conforming to 
\LaTeX\ syntax conventions.
\end{MDabstract}
\tableofcontents

  %% TODO table listing of active characters
%% Were tests 2010/03/08:
% \section{Presenting Nasty's `Nasty' ``Nasty'' &\NVerb\ 'niceverb'}
% \section{Presenting \cs{NVerb} 'niceverb'}
\section{Presenting 'niceverb'}
\subsection{Purpose}
% \begin{abstract}\noindent
% The 'nicetext' bundle provides ``minimal" markup 
The 'niceverb' package provides ``minimal" markup for documenting \LaTeX\ 
packages, reducing the number of keystrokes/visible characters needed
% .\,.\,. %%% ... %% TODO nicedots 
(kind of poor man's \acro{WYSIWYG}).\footnote{``What you see is what you 
  get." Novices are always warned that \acro{WYSIWYG} is essentially 
  impossible with \LaTeX.} %% TODO UK FAQ 2010/03/11
% One feature---\verb'&\foo'%%% badly self-documenting, `&' fails
It conveniently handles command names in arguments of macros 
such as &\footnote or even of sectioning commands. 
% (`.aux'/`.toc' entries).
% 
% This is done by making some characters active. 
% 'niceverb.sty' thus resembles 'wiki.sty'; both are siblings. 
% \end{abstract}
If you use 'makedoc.sty' additionally, commands for typesetting a 
package's code are inserted automatically (just using \TeX). 
%%% \footnote{Stephan I. B\"ottcher used
%%% 'awk' instead to typeset the documentation of his 'lineno.sty'.} 
As opposed to tools that are rather common on UNIX/Linux, this 
operation should work at any \TeX\ installation, irrespective of 
platform.

Both packages may at least be useful while working at a very new package 
and may suffice with small, simple packages. After having edited your 
package's code 
%% <jobname> 2010/02/28:
(typically in a `.sty' file---<jobname>`.sty'), 
you just ``{`latex'}" the manual file 
(maybe some `.tex' file---<jobname>`.tex') 
and get instantly the corresponding updated documentation.

'niceverb' and 'makedoc' may also help to generate without much effort 
documentations of nowadays commonly expected typographical quality for 
packages that so far only had plain text documentations.

\subsection{Acknowledgement/Basic Ideas}
\emph{Four}                                         %% 2011/01/26
ideas of Stephan I. B\"ottcher's in documenting his 
\ctanpkgref{lineno} inspired the present work: 
\begin{enumerate}
\item 
The markup and its definitions are short and simple, 
markup commands are placed at the right ``margin" 
of the ASCII file, 
so you hardly see them in reading the source file, 
you rather just read the text that will be printed. 
\item 
An 'awk' script removes the `%'s starting \emph{documentation} lines 
and inserts the commands for typesetting the package's \emph{code} 
(you don't see these commands in the source).\footnote{The 
  corresponding part of the ``present work" is 'makedoc.sty'.} 
  %% <- clarified 2010/03/11
\item 
An active character (\lq&|\rq) issues a `\string' \emph{and} switches 
to typewriter typeface for typesetting a command verbatim---so this 
works without changing category codes (which is the usual idea of 
typesetting code), therefore it works even in macro arguments.
\item                                               %% 2011/01/26
\lq\HardNVerb+<meta-variable>+\rq\ produces \lq<meta-variable>\rq. 
(\qtd{&\lt} stores the original \qtd{&<}.)  %% was \lessthan 2014/03/28
\end{enumerate}

\subsection{The Commands and Features of 'niceverb'}
Actually, it is the main purpose of 'niceverb' to save you from 
``commands" $\dots$\par
Single quotes &`, &', ``less than" &< (accompanied 
with `>'), the ``vertical" &|, the hash mark `#', ampersand `&', 
and in an extended ``auto mode" even backslash `\' become `\active'
characters with ``special effects." 
% \qtd{&|$\dots$&|} (i.e., \GenCmdBox+|<code>|+) in general
% should highlight descriptions of user commands and their syntax. 

The package mainly aims at typesetting commands and descriptions of their 
syntax \emph{if the latter is ``standard \LaTeX-like"}, 
using ``meta-variables." A string to be 
typeset ``verbatim" thus is assumed to start with a single command like 
&\foo, maybe followed by stars (\lq`*'\rq) and pairs of 
square brackets (\lq`['<opt-arg>`]'\rq) 
or curly braces (\lq`{'<mand-arg>`}'\rq), 
where those pairs contain strings indicating the typical 
kinds of contents for the respective arguments of that command.
A typical example is this: 
\[\InlineCmdBox{&\foo*[<opt-arg>]{<mand-arg>}}\]
This was achieved by typing 
\[\HardVerbBox+&\foo*[<opt-arg>]{<mand-arg>}+\]
In ``auto mode" of the package, even typing 
% \tracingmacros=1 \tracingonline=1
\[\HardVerbBox+\foo*[<opt-arg>]{<mand-arg>}+\]
would have sufficed---\acro{WYSIWYG}! I call such mixtures of 
\emph{verbatim} and ``meta-variables" \textit{\qtd{meta-code}}.

Outside macro arguments, you obtain the same by typing 
% \[\verb+`\foo*[<opt-arg>]{<mand-arg>}'+\]
\[\HardVerbBox+`\foo*[<opt-arg>]{<mand-arg>}'+\]

Details:
\begin{description}

\item[``Meta-variables:"] The package supports the ``angle 
brackets" style of ``meta-variables" (as with <meta-variable>). 
You just type \lq\verb'<bar>'\rq\ to get \lq<bar>\rq.

This works due to a sloppy variant `\NVerb' of `\verb'
which doesn't care about possible ligatures and definitions of active 
characters. Instead, it assumes that the ``verbatim" font doesn't 
contain ligatures anyway.\footnote{On the other hand, &\NVerb is more 
  \emph{careful} with 'niceverb''s special characters.}
\lq\verb'\verb+<foo>+'\rq, by contrast, just yields \lq\verb'<foo>'\rq.

Almost the same feature is offered by 'ltxguide.cls' which formats the 
basic guides from the \LaTeX\ Project Team. The present feature, 
however, also works in plain text outside verbatim mode. 
% On the other hand: without << feature

\item[Single quotes (left/right) for ``short verb:"]
The package ``assumes" that \emph{quoting} refers to 
\emph{code}, therefore \lq\verb+`foo'+\rq\ is typeset as 
\lq`foo'\rq, or (generally) |`<content>'| turns <content> 
into meta-code with the meta-variable feature as above. 
This somewhat resembles the &\MakeShortVerb feature of 'doc.sty'.
%% Moved up here 2010/02/28:
You can ``abuse" our %%% ``single quotes" 
feature just to get typewriter 
typeface.{\sloppy\par}%% not so useful here 2010/02/28:
% \footnote{In macro arguments this requires that the right 
% single quote &' is &\active.}

Problems with this feature will typically arise %%% fail %% 2010/02/28
when you try 
to typeset commands (and their syntax) in \emph{macro arguments}---e.g., 
$$\verb+\footnote{`\bar' is a celebrated fake example!}+$$
will try to \emph{execute} &\bar instead of typesetting it, giving 
an ``undefined" error or so. %% TODO try! 2010/02/28
\verb+\verb+ fails in the same situation, for the same reason. 
\lq\verb+&+\rq\ (&\footnote{&&&\bar<remaining>}) or 
``auto mode" (see below) may then work better.\footnote{&\bar indeed!} 
More generally, the quoting feature still works in macro arguments in 
the sense that you then have to mark difficult characters with `&' 
(simply as short for `\string'). However, it still won't work with 
curly braces that don't follow a command name 
(such \emph{pairs} of braces will simply get lost, 
 \emph{single} braces will give errors or so).%%%\footnote{`{group}'}

Double quotes and apostrophes should still work the usual way.
% %% TODO doesn't work, inside runs into `}' 2010/02/28:
% otherwise you could control the parsing mechanisms using curly braces 
% (outside and inside don't interact: `Harry{'}s' for \qtd{Harry's}).
For difficult cases, you can still use the standard `\verb' 
command from \LaTeX.
To get \emph{usual} single quotes, you can use their standard substitutes 
`\lq' and `\rq', or for pairs of them, 
|\qtd{<text>}| in place of `\lq <text>\rq'---or even `\lq <text>\rq\ '. 
To get single quotes around some verbatim <verb>,
often `\qtd{&<verb>}' works. 
It is for this reason that I have refrained from different 
solutions as in \ctanpkgref{newverbs} (so far).

%% 2012/10/10:
v0.44 provides |\AddQuotes| after which single quotes \emph{both} 
turn their content into metacode \emph{and print} single quotes 
around them \emph{automatically.} This can be turned off again by 
|\DontAddQuotes|.

\item[Single right quotes for &\textsf:]
Package names are (by some convention I often yet not always 
 see working) 
typeset with `\textsf'; 
it was natural to use a remaining case of using single quotes 
for abbreviating $$&\textsf{<text>}$$ by |'<text>'|.
% \footnote{%
% Font switching by sequences of single quotes is a feature of the 
% syntax for editing \textit{Wikipedia} pages and of 'wiki.sty'.}
%% <- undoubled 2010/02/28 ->
This idea of switching fonts continues font switching of 'wiki.sty'
which uses the syntax for editing {\it Wikipedia} pages 
(font switching by sequences of right single quotes).

\item[Verticals for setting-off command descriptions:]%%%
\hskip0pt plus 2em
\GenCmdBox+|<code>|+ works like \qtd{&`<code>&'} except putting 
the result into a \emph{framed box} (just as all around 
here)---or something else that you can achieve using some \emph{hooks} 
described with the implementation. There are variants like 
\GenCmdBox+\cmdboxitem|<code>|+.

\item[Ampersand shows command syntax \&c. even in arguments:]
\hfil E.g., type \lq\verb+&\foo{<arg>}+\rq\ to get 
\lq`\foo{<arg>}'\rq. This may be even more convenient for typing than 
the single quotes method, although looking somewhat strange.
However, in macro arguments this does not work with 
\emph{private letters} (`@' and `_' here), for this case, 
use |\cs{<characters>}| or |\cstx{<characters>}<parameters>|.%%%
% `&' may terminate \textit{verbatim} unexpectedly, being designed for 
% displaying ``\LaTeX-like command syntax" in the first instance.
\footnote{Moreover, && currently has a limited 'xspace' 
functionality only.}%%%\footnote{You can even use && for referring to 
%   active characters like && in footnotes etc.!}
%% <- said elsewhere now 2010/03/07

\begin{sloppypar}
This choice of `&' rests on the assumption that there won't be many 
tables in the documenation. You can restore the usual meaning of `&' 
by `\MakeNormal\&' and turn the present special meaning on again by 
\[`\MakeActive\&' \mbox{\quad or\quad } 
  `\MakeActiveLet\&\CmdSyntaxVerb'\]
You could also 
redefine (&\renewcommand) &\descriptionlabel using `\CmdSyntaxVerb' 
(the ``normal command" that is equivalent to `&', its ``permanent 
 alias") 
so \verb+\item[\foo]+ works as wanted.
\end{sloppypar}

\textbf{Another} feature of 'niceverb''s `&' is getting 
(some of the) special characters    %% 2010/03/20
(as listed in the standard macro `\dospecials') verbatim in arguments 
(where `\verb' and the like fail). It just acts similarly as \TeX's 
    %% undoubled lines 2011/05/09
 primitive `\string' (which it actually invokes---cf. discussion on the 
 left quote feature above). 

\item[``Auto mode" typesets commands verbatim unless .\,.\,.]
\begin{sloppypar}
In~``auto mode," the backslash \lq`\'\rq\ is an active character that 
builds a command name from the ensuing letters and typesets the 
command (and its syntax, allowing meta-variables) verbatim. 
However, there are some exceptions, which are collected in a macro 
|\niceverbNoVerbList|. &\begin, &\end, and &\item belong to this list, 
you can redefine (`\renewcommand') it, or add <macros> to it by
|\AddToNoVerbList}{<macros>}|                           %% 2010/12/29
There is also a command |\NormalCommand{<letters>}| \emph{issuing} the 
command `\<letters>' instead of typesetting it.
Since auto mode is somewhat dangerous, you have to start it explicitly 
by |\AutoCmdSyntaxVerb|. You can end it by |\EndAutoCmdSyntaxVerb|.
|\AutoCmdInput{<file>}| is probably most important. 
\end{sloppypar}

Auto mode is motivated by the observation that there are package files 
containing their documentation as pure (well-readable) ASCII 
text---contain\-ing the names of the new commands without any kind of 
quotation marks or verbatim commands. 
Auto mode should typeset such documentation just from the same ASCII 
text.

\item[Hash mark \lq&#\rq\ comes verbatim.]
No macro definitions are expected in the `document' 
environment.\footnote{This idea appeared 2009 on the 'LATEX-L' 
                      mailing list. It may be wrong, 
                      as I have sometimes experienced $\dots$}
                      %% <- changed 2010/03/11
Rather, \lq`#'\rq\ is an active character for taking the next 
character (assuming it is a digit) to form a reference to a 
\emph{macro parameter}---\lq`#1'\rq\ becomes \lq#1\rq\---\acro{WYSIWYG} 
indeed! (So the general syntax is |#<digit>|.)
\item[Escaping from 'niceverb' (generally).] 
     To get rid of the functionality of some active character <char> 
     (\qtd{&&}, single quote, ampersand, hash mark---not 
      ``auto mode," see above) here, use |\MakeNormal\<char>|---may 
     be within a group. To revive it again, use |\MakeActive\<char>|. 
     This may fail when a different package overtook the active <char> 
     (but I expect more failures then), in this case 
     |\MakeActiveLet\<char>\<perm-alias>| 
     revives the 'niceverb' meaning of <char>
     where `\<perm-alias>' is the ``permanent alias" for that active 
     <char> according to the documentation below. 
     E.g., `\LQverb' is the ``permanent alias" for active single left 
     quote, 'niceverb' activates it by 
     \NVerb+\MakeActiveLet\'\LQverb+.---You can turn off 'niceverb' 
     syntax \emph{alltogether} by |\noNiceVerb| and revive it 
     by |\useNiceVerb| (without ``auto mode").{\sloppy\par}

     \strong{Right Quotes:} Disabling\slash reviving replacement 
     of `\textsf' by single right quotes requires 
     \[|\nvRightQuoteNormal| \mbox{\quad or\quad } |\nvRightQuoteSansSerif|\] 
     %% 2014/03/27:
     respectively.---The feature fails in certain occasions
     because a single right quote must not always be interpreted
     as `\textsf', and deciding this by macros became quite 
     laborious for me and is most likely still not perfect. 
     There is a command |\nvAllRightQuotesSansSerif| 
     to be used with care that interpretes \emph{all} single
     right quotes as `\textsf', which, e.g., means that you
     must use `\rq' for apostrophes.

     \strong{``Moving" arguments:}                  %% 2014/03/27
     |\NiceVerbMove{<text>}| with v0.6 is for ``moving" arguments so that 
     'niceverb' syntax operates \emph{locally} at the destination
     (table of contents or page headings).
     It is automatically used by 'niceverb''s variant of \LaTeX's 
     sectioning commands; while with `\markboth', `\markright', 
     `\addcontentsline' etc.\ you must it include yourself 
     (currently, \TODO?).
\end{description}

\subsection{Examples}
The file 'mdoccorr.cfg' providing some `.txt'$\to$\LaTeX\ 
functionality---i.e., typographical corrections---documents itself 
using 'niceverb' syntax. Its code and the documentation that is 
typeset from it are in the \qtd{examples} section of 
'makedoc.pdf'.---Moreover, 
the documentation 'niceverb.pdf' of 'niceverb.sty' was 
typeset from 'niceverb.tex' and 'niceverb.sty' using 'niceverb' 
syntax, likewise 'fifinddo.pdf' and 'makedoc.pdf'. 
The example of 'niceverb' shows the most frequent use of the `&' 
feature.{\sloppy\par}

'nicetext' bundle release v0.4 contains a file 'substr.tex' 
that should typeset the documentation of the version of 
Harald Harders'
'substr.sty'\footnote{\url{http://ctan.org/pkg/substr}}
that your \TeX\ finds first, as well as 'arseneau.tex' 
typesetting a few packages by Donald Arseneau. 
The outcomes (with me) are 'substr.pdf' and 'arseneau.pdf'.
These are the first applications of 'niceverb''s ``auto mode" to 
(unmodified) third-party package files.
(I also made a more ambitious documentation of Donald Arseneau's 
 'import.sty v3.0' before I found that CTAN already has a nicely 
 typeset documentation of 'import.sty v5.2'.)

%% removed 2010/03/11:
% It seems to me that I could type so many pages on 'fifinddo' and 
% 'makedoc' in little more than a week 
% % (2009/04/12, much of which was needed for debugging and reworking concepts) 
% only due to the ``minimal" \emph{verbatim} and syntax-display syntax. 
% 
\subsection{What is Wrong with the Present Version}
\begin{enumerate}
\item 'niceverb.sty' should be an extension of 'wiki.sty'; 
      yet their font selection mechanisms are currently not compatible. 
      %% 2010/02/28:
      Especially, the feature of \[\hbox\bgroup|''<text>''|\egroup\] 
      %% <- failed with \mbox as of 2010/03/23, first two rq missing 
      %%    2010/03/29
      replacing 
      `\textit{<text>}' or `\emph{<text>}' may be considered missing. 
\item Font switching or horizontal spacing may fail in certain 
      situations.
%       (parentheses, titles, footnotes; 
      You can correct spacing by \lq`\ '\rq. 
        %% <- \qtd{`&\ '}.
% \item 
% The feature of mixing high-quality-typeset comments into the 
% package code listing is implemented in a very rudimentary way only. 
% % just allowing for `\subsection's. 
% The ``comment detector" detects Wikipedia-style subsection titles 
% instead of lines beginning with percent characters.\footnote{%
% Percent characters will definitely not be ``ignored" as with &\DocInput, 
% rather they will hide rests of \emph{documentation} lines as usually, 
% while they will be typeset verbatim in \emph{package code} lines.} 
% Switching between plain and verbatim typesetting in the package 
% listings isn't settled yet, since there are different styles of using 
% percent symbols. I have sometimes used double percent symbols 
% (\lq\verb+%%+\rq) 
% for commenting text and single ones just for ``reversible deletion of 
% code," while usually single percent symbols indicate commenting text 
% indeed. Double percent symbols may, by contrast, mean that the text remains 
% visible in the `.sty' file only, suppressed in the typeset 
% documentation ('lineno.sty').
% For a while, it may be necessary to provide replacing macros for each 
% package separately instead of providing a single macro package 
% managing all of them. 
% \item 
% The code listing currently uses the `listing' and `listingcont' 
% environments of 'moreverb.sty'; 
% the code font and the line numbers may be too large. 
\item The ``vertical" character \qtd{&|} produces inline boxes 
      only at present. It might as well provide a version of the 
      `decl' tabular environment of 'ltxguide.cls'. 
%% changes 2010/03/10
%       coloured\slash framed boxes instead (2009/04/09). They have 
%       their merits! See 'fifinddo.pdf'  and 'makedoc.pdf'. However, 
%       they 
      The inline boxes
      badly deal with long command names and many arguments.
      Doubled verticals could ensure the `decl' mode. 
      Moreover, such a box might issue an \emph{index} entry.
\item One may have \emph{opposite} ideas about using quotes---maybe 
      rather `"<code>"' should typeset <code> \textit{verbatim}.
      There might be a package option for this. If ordinary 
      \qtd{\NVerb'``<text>"'} still should work, awful tricks as now with 
      the right quote feature would be needed. %% TODO 2010/03/06
% \item ``Auto mode" has \emph{not} been tested on a serious application yet. 
%% partially improved 2010/02/28:
% \item % 'niceverb''s font switching tricks sometimes turn against their 
%       % inventor (and other users?). There must be some switching 
%       % ``off'' (and ``on'' again).%
%       %   \footnote{\hspace{1sp}'fifinddo'\slash\hspace{1sp}'makedoc'
%       %     %% <- TODO oh, oh! 2009/04/11
%       %     allow inserting such commands from a driver script, 
%       %     invisible in the file that contains the ``contentual'' 
%       %     documentation.}
%       % Also, there 
%       There
%       might better help with weird errors, 
%       some syntax checks might intercept earlier. 
% 
%       Similarly, some choices reflect a %% rather OK 2010/02/28
%       personal style and should be modifiable, especially by package 
%       options.\footnote{Please sponsor the project or support it 
%         otherwise!}
\item ``auto mode" seems not to work in section titles. (2011/01/26)
      %% <- noted with edtnotesc
\item Certain difficulties with typesetting code in macro arguments 
      may be overcome easily using $\varepsilon$\mbox{-}\TeX\ 
      features, I need to find out $\dots$
\end{enumerate}

\newpage                                            %% 2014/03/28
\section{The Package File}                          %% 2014/03/19
% \section{The ``Package" File} %% test for page heading 2014/03/24
% \section{`nicetext.sty'}      %% tests for pageheadings 2014/03/27
\subsection{Preliminaries}                          %% 2014/03/19
\subsubsection{File Header}                         %% 2014/03/19
\NeedsTeXFormat{LaTeX2e}[1994/12/01]
\ProvidesPackage{niceverb}[2015/11/21 v0.62
                           minimize doc markup (UL)] 

%% Copyright (C) 2009-2012, 2014 2015 Uwe Lueck, 
%% http://www.contact-ednotes.sty.de.vu 
%% -- author-maintained in the sense of LPPL below -- 
%%
%% This file can be redistributed and/or modified under 
%% the terms of the LaTeX Project Public License; either 
%% version 1.3a of the License, or any later version.
%% The latest version of this license is in
%%     http://www.latex-project.org/lppl.txt
%% We did our best to help you, but there is NO WARRANTY. 
%%
%% Please report bugs, problems, and suggestions via 
%% 
%%   http://www.contact-ednotes.sty.de.vu 
%%
%% ==== &\newlet                             ====
%% |\newlet<cmd><cnd>| counters the risk of mistyping <cmd> 
%% with `\@ifdefinable', and even saves some code lines:
\providecommand*{\newlet}[2]{\@ifdefinable#1{\let#1#2}}
\@onlypreamble\newlet
%%
%% ==== Switching Category Codes             ====
%% Underscore as a ``private letter," using 'stacklet' with v0.5:
\RequirePackage{stacklet} \PushCatMakeLetter\_          %% 2012/08/27
%% v0.3 introduced `\AssignCatCodeTo' and `\MakeNormal'.
%% v0.5 abolishes the former again and uses             %% former 2012/08/28
%% 'actcodes' for some part of `\catcode' switching:
\RequirePackage{actcodes}
%% |\CatCode{\<character>}| 
%% (or simply |\CatCode\<character>|)
%% saves one token per use and works when the category code 
%% of \qtd{&`} (``single left quote") has changed. 
%% As of v0.5, it may be defined by a different package: 
\providecommand*{\CatCode}{\catcode`}     %% \provi... 2012/08/27
% \newcommand*{\CatCode}[1]{\catcode`#1 } %% no better 2010/02/27
%% `\CatCode' is near to be moved into the 'catcodes' bundle, 
%% and basic commands from 'stacklet' and 'actcodes' 
%% may be reimplemented using it 
%% ('manycats'; 'allcats' for loading entire 'catcodes' 
%%  in good order).
%% 
%% |\AssignCatCodeTo{<number>}{\<char>}| \              %% \ 2012/08/28
%% no longer is considered 
%% useful (counted tokens in `memory.tex') and replaced by 
%% `\CatCode'.
% \newcommand*{\AssignCatCodeTo}[2]{\catcode`#2=#1\relax}
%% |\MakeLetter\<char>| is replaced by the 'stacklet' package---I
%% thought, but \emph{here} it is also needed to declare 
%% the ``private letters" of the package that is documented. 
%% This should be ``variable." OK, the new (v0.5) `\private_letters'
%% is a step towards this:
\newcommand*{\private_letters}{\CatCode\@11\CatCode\_11\relax}
%% |\MakeOther\<char>| and |\MakeActive\<char>| were implemented 
%% here before v0.5, now they are in 'actcodes'~...
% \def        \MakeOther {\AssignCatCodeTo{12}}
%% % |\MakeActive\<char>| just revives the meaning of <char> 
%% % it had most recently 
%% % (as an &\active character ... 
%% %  maybe ``Undefined control sequence" unless ...) 
%% % This is fine for reviving 'niceverb' functionality 
%% % after having disabled it by `\MakeNormal'\linebreak[0]---provided 
%% % no other package used <char> actively in the meantime ...
%% % % \providecommand*{\MakeActive}[1]{\CatCode#1\active}
%% % We take a copy |\MakeActiveHere| of `\MakeActive' 
%% % as the latter may become a dangerous thing for compatibility 
%% % with 'hyperref'.
%% % % \@ifdefinable\MakeActiveHere{%
%% % %    \let\MakeActiveHere\MakeActive}
%% % %    %% <- TODO aliascid + elsewhere 2010/03/12
%% |\MakeActiveLet\<char>\<macro name>| 
%% % is provided by 'actcodes' 
%% % (which does not provide `\MakeActive').    %% rm. 2012/09/27
%% % We take a copy |\MakeActiveLetHere| as well:
%% likewise is in 'actcodes'. 'niceverb' takes a copy 
%% |\MakeActiveLetHere| of it for dealing with 'hyperref' 
%% (see \secref{hyperref}). 
%% 'hyperref'-compatibility of mere `\MakeActive' 
%% is not provided any longer:
\newlet\MakeActiveLetHere\MakeActiveLet
%% %% |\MakeAlign\&| can be used to restore the meaning of `&' after we 
%% %% have made it `\active'.
%% % \providecommand*{\MakeAlign} {\AssignCatCodeTo4}
%% For restoring the usual category codes of \TeX's special 
%% characters later, we store them now.
%% (I.e., these characters are listed in the macro `\dospecials' 
%%  that expands to
%%  $$\mbox{\tt\def\do{\string\do\unskip\string}\dospecials}$$
%%  their category codes are
%% {\CatCode\#6%% was \AssignCat... 2012/08/27
%%  \CatCode\$3%% fifinddo-interference!? TODO 2010/02/27
%%  \CatCode\&4%% fifinddo-interference!? TODO 2010/02/27
%%  \def\do#1{\def\do##1{, 
%%  \the\CatCode##1}\the\CatCode#1}\dospecials} 
%% respectively; ``end of line", ``ignored", ``letter", ``other", and 
%% ``invalid" are missing---cf. {\it \TeX book} Chap.\,7.)
\def\do#1{\expandafter 
%   \chardef \csname normal_catcode_\string#1\expandafter \endcsname
%% <- v0.6 2014/03/22: First I thought ``too few `\expandafter's"; 
%%    actually the original `\expandafter' has no effect ->
  \chardef \csname normal_catcode_\string#1\endcsname
    \CatCode#1\relax}
\dospecials
%% Tests: 
%% %\withcsname\show normal_catcode_\string\\\endcsname %% 2014/03/22
%% ``normal category code" of `\' is 
%% \expandafter\the\csname normal_catcode_\string\\\endcsname,
%% ``normal category code" of `$' is 
%% \expandafter\the\csname normal_catcode_\string\$\endcsname;
%% ``normal category code" of `&' is 
%% \expandafter\the\csname normal_catcode_\string\&\endcsname.%%%
%% \footnote{\LaTeX's \cs{nfss@catcodes} is similar, 
%%           but it makes space-like characters ignored. 
%%           Also cf. 'ltfinal.dtx'. 
%%           TODO: &\RestoreNormalCatcodes.} %% 2010/03/06
%%
% \newcommand*{\make_iii_other}{\MakeOther\\\MakeOther\{\MakeOther\}}
  %% <- replaced 2009/04/05
%% |\MakeNormal\<char>| saves you from remembering ...
\newcommand*{\MakeNormal}[1]{%
  \@ifundefined{\norm_catc_str#1}%
             {\MakeOther#1}%
             {\CatCode#1\csname\norm_catc_str#1\endcsname\relax}}
\newcommand*{\norm_catc_str}{normal_catcode_\string} 
  %% TODO add ^^I and ^^M
  %% TODO save char tokens  %% 2012/08/27
%% We take a copy |\MakeNormalHere| of `\MakeNormal' as 
%% with `\MakeActive'.
\newlet\MakeNormalHere\MakeNormal
%%
%% ==== Robustness by &\IfTypesetting\ or So ====
%% \label{sec:iftype}
%% It seems we need some own ways 
%% of robustifying (as opposed to \LaTeX's `\protect' and %% 2014/03/20
%% `\DeclareRobustCommand'---sometimes, 
%%  especially for certain active characters)
%% to achieve various 
%% compatibilities---using 
%% %\begin{center}
%%   \[|\IfTypesetting{<if>}{<unless>}|\]%%%.
%% %\end{center}
%% It also saves some `\expandafter's.
\providecommand*{\IfTypesetting}{%
%     \relax 
%% <- This `\relax' suppressed ligatures of single right quotes!
%% %%^ arrow 2014/03/20
%% %% removed 2010/03/23
    \ifx \protect\@typeset@protect
          \expandafter \@firstoftwo 
    \else \expandafter \@secondoftwo \fi}
%% %\begin{center}
%%   \[|\nvSelfProtect{<cmd-char>}{<typeset>}|\] 
%% %\end{center}
%% is another idea. 
%% In ``typesetting mode," <typeset> is run. 
%% Otherwise a single unexpanded token <cmd-char> remains.
%% TODO bad at `\shipout'.
%% No `\protect' appears, and as opposed to \LaTeX's 
%% protection mechanism, running <typeset> does not 
%% require a second macro name.
%% The idea is that 
%% %`\nvSelfProtect{<cmd-char>}{<typeset>}'
%% `\nvSelfProtect{#1}{#2}'
%% is the \emph{definition} (substitution text---on 
%% token level) of <cmd-char>.\footnote{This 
%%     may go into a separate package under a different name later.}
\newcommand*{\nvSelfProtect}[2]{%
    \ifx \protect\@typeset@protect
           \nv_expand_else{#2}%                 %% braces 2014/03/26
%   \else  \protect#1\fi}
    \else  \noexpand#1\fi}                      %%  works 2014/03/28
\def\nv_expand_else#1\else#2\fi{\fi#1}
%% %% 2014/03/27
%% Sometimes ``control sequences" get definitions with 
%% `\svSelfProtect' below whose first argument then is 
%% an active character---the ``control sequence" then 
%% is the ``permanent alias" of the active character.
%% This is a somewhat ``indirect self"-protection.
%% At other places, the ``self"-protection is more 
%% direct. Then |\NewSelfProtectedCommand{<cmd>}{<def>}| avoids 
%% mistakes from mistyping <cmd> and saves some code.
%% It works like `\newcommand*', provides the 
%% `\svSelfProtect', and your definition <def> needs to contain 
%% the second argument of `\svSelfProtect' only. 
%% \emph{Arguments} are not supported currently 
%% (TODO---well, 3 applications 2014/03/27):{\sloppy\par}
\newcommand*{\NewSelfProtectedCommand}[2]{%
    \newcommand*#1{\nvSelfProtect#1{#2}}}
\@onlypreamble\NewSelfProtectedCommand
%% TODO 3 applications for the permanent alias case,
%% saving catcode changes ... %% 2014/03/28
%%
%% \strong{Testing:}
\newcommand*{\nvShowProtectedEdef}[1]{%
    \protected@edef\@tempa{#1}\show\@tempa}
%%
%% ==== Shared Shorthand Macros              ====
%% |\begin_min_verb| is a beginning shared by some macros here. 
%% It begins like \LaTeX's `\verb', apart from the final `\tt'.
%% %% 2014/03/18:
%% `\bgroup' is needed for `\hbox' and must be balanced by 
%% an `\egroup' counterpart later.
\newcommand*{\begin_min_verb}{% 
  \relax \ifmmode \hbox \else \leavevmode\null \fi 
  \bgroup \tt}
%% For typographical additions (``decorations") to the verbatim material, 
%% we collect it in a box register addressed by |\niceverb_savebox|:
\newsavebox\niceverb_savebox
%% |\SetNiceVerbSaveBox| starts reading the (``meta-")verbatim 
%% material:
\newcommand*{\SetNiceVerbSaveBox}{%
    \setbox\niceverb_savebox\hbox\bgroup}
%% |\NVerb|, |\HardNVerb|, or |\NiceMaybeMetaVerb|
%% with an optional argument about as `[\<id>_egroup]' should follow, cf. 
%% \secref{nverb}.---There have been two applications 
%% only up to now (2014/03/19), but this may change soon.%%%\footnote{%
%%     % Thought of &\qtdnverb, but this doesn't need this kind 
%%     % of treatment.}                                     %% 2014/03/20
%%                                                  %% introduced in v0.6
%%
%% |\TheNiceVerbSaveBox| allows referring to the verbatim material
%% collected, in order to place it a \emph{single time}---and no
%% surrounding braces are needed:
\newcommand*{\TheNiceVerbSaveBox}{\box\niceverb_savebox}
%% TODO: 
%% \begin{center}
%%   |\NewNiceVerbDecoration{\<deco>}{\<end-name>}{<start>}{<end-code>}|
%% \end{center}
%% might save from typing `\<end-name>' twice and from typing 
%% the two `\egroup's.
%%
%% %\section{&\+*[<opt>]{<mand>} `{verbatim}' ``o''} %% 2014/03/28
%% TODO left quote verb moving: braces get lost.    %% 2014/03/28
%% %\nvShowProtectedEdef{``''&\+<name>}
%%  === Implementation of the ``Nice" Syntax ===
%% ==== &\NVerb                                  ====
%% \label{sec:nverb}
%% Discovered mistakes in this section 2014/03/19,              %% 2014/03/20
%% with respect to robustness. \ 
%% (i)~`\NVerb' not really was meant to be a user command,
%% to appear in documentation code (rather to be internal). \
%% (ii)~The attempt to make it robust was incomplete. \
%% (iii)~The code for ``not typesetting" was strange. \
%% (iv)~It is difficult to imagine that somebody attempts 
%%      to use ``verbatim" code, e.g., in a section title
%%      (while with our `&'/`\string' it's ok). 
%%      \emph{At least handy replacement for &\textsf.} \       %% 2014/03/28
%% Well, let's see. May be I once find a useful application. 
%% So I repair the code---`\protect' before &\_...`_false'.
%% |\NVerb<char><code><char>|:
\newcommand*{\NVerb}{%
    \protect\_no_nice_meta_verb_false \NiceMaybeMetaVerb}
%% |\HardNVerb<char><code><char>| does not recognize meta-variables:
\newcommand*{\HardNVerb}{%
    \protect\_no_nice_meta_verb_true \NiceMaybeMetaVerb}
\newif\if_no_nice_meta_verb_
%% v0.6 equips both |\NVerb[<end-cmd>]| and             %% 2014/03/18
%% |\HardNVerb[<end-cmd>]| with an optional argument for a single 
%% parameter-less macro for what to do after reading 
%% verbatim text---for boxing or quoting etc.\footnote{The 
%%   goal resembles that of &\collectverb in Martin Scharrer's
%%   \ctanpkgref{newverbs}. A difference in implementation is 
%%   that the character delimiting the verbatim text 
%%   is used as a parameter delimiter for a new/temporary macro. 
%%   So the verbatim characters are fixed. Our approach will be 
%%   collecting the verbatim material in a box, if we need 
%%   something more complex than %&\niceverb_normal_egroup.
%%                                \cs{niceverb_mormal_egroup}.
%%   This allows changing category codes with &\MetaVar again, 
%%   although there hasn't been a need for this so far. 
%%   It might be useful for allowing shorthand macros in 
%%   &\MetaVar's argument.}
%% |\niceverb_egroup| then is useless and removed.
%% Macros that were assigned to it before v0.6 move into 
%% the new optional arguments.---Actually, the next macro 
%% |\NiceMaybeMetaVerb| shared by `\NVerb' and `\HardNVerb' 
%% gets the optional argument: TODO!?
% \newcommand*{\nice_maybe_meta_verb}[1]{%
\newcommand*{\NiceMaybeMetaVerb}[2][\niceverb_normal_egroup]{%
%% `\newcommand' with v0.6 must suffice for robustness, 
%% so removing 2014/03/20:
% \IfTypesetting{%
%% Mainly avoid `\verb''s noligs list which overrides definitions 
%% of some active characters, while 'cmtt' doesn't have any 
%% ligatures anyway. 
    \begin_min_verb
      \let\do\MakeOther \dospecials
%% Turn off 'niceverb' specials:
      \MakeOther\|\MakeOther\`\MakeOther\'%
      \if_no_nice_meta_verb_ \MakeOther\<%
      %%% \else    \MakeActiveLet\<\MetaVar     %% 2010/12/31
      \else        \MakeActiveLetHere\<\MetaVar %% 2011/06/20
      \fi
%     \MakeActiveLetHere #2\niceverb_egroup
      \MakeActiveLetHere #2#1%                  %% 2014/03/18
%% After the previous line has worked, we use `\def' 
%% instead of `\let', so there is no longer a need 
%% to choose a command name for the verbatim delimiter
%% -- well, \strong{no}, don't define the same 
%% macro several times. Also, the same ``end" macro 
%% might be used for different purposes, e.g., 
%% when a macro in an eventual expansion of the ``end" macro
%% is modified.
%     \MakeActiveDef     #2{#1}%                %% 2014/03/18
      \verb@eol@error %% TODO change message 2009/04/09
%   }{\string\NVerb \string#1}}
%% <- both `\string' very strange 
%%    (second one finds <char>---maybe it's active---but 
%%     then its next occurrence delimiting the verbatim code 
%%     will harm too!), 
%%    also redirecting to `\NVerb'. 
%%    (May have been ok for entries to auxiliary files.)
%%    New difficulties come from the optional argument, 
%%    which needs protection as well.---Ok, the optional 
%%    argument is not protected, and active characters 
%%    <char> must ``protect themselves," so use of 
%%    `\IfTypesetting' changes, cf. \secref{iftype}.
}
%% %2009/04/11: about `etc.' [preceding a box!? 2010/03/14]
%% [2014/03/19 removing/hiding
%%  remarks from 2009f. that I don't understand anymore~...]
\newcommand*{\niceverb_normal_egroup}{%
    \egroup 
    \niceverb_maybe_rq                   %% 2011/09/09 for \AddQuotes
    \ifmmode\else\@\fi}
% \@ifdefinable\niceverb_egroup                      %% rm 2014/03/18
%     {\let\niceverb_egroup\niceverb_normal_egroup}
%%
%% ==== Single Quotes Typeset Meta-Code          ====
%% |\LQverb| will be a ``permanent alias" 
%% for the active left single quote. 
%%
%% The verbatim feature must not act when another single left 
%% quote is ahead---we assume a double quote is intended then, 
%% and we typeset it
%% (thus the left quote feature does not allow to typeset 
%%  something verbatim that starts with a single left quote). 
%% %Rather, double quotes should be typeset then.               rm. 2014/03/25
%% In page headers, a `\protect' 
%% could %%% may                                                %% 2014/03/28
%% be in the way %%%.\footnote{% 
%% %     See \code{\string\show\cs{let_token}} below---tried %% 2014/03/24
%% %     with section title in `niceverb.tex'.}
%% before v0.6.
%% (A hook for `\relax'ing certain things in 
%% `\markboth' and `\markright' would have been an alternative. 
%%  TODO)       %% 2014/03/24
\MakeActive\`
  \newcommand*{\LQverb}{%
%   \IfTypesetting{\lq_double_test}{\protect`}}
%% New approach v0.6:
    \nvSelfProtect`\lq_double_test}
%   \IfTypesetting{\lq_double_test}{\noexpand`}}
\MakeOther\`
\newcommand*{\lq_double_test}{%
%% This test settles the next catcode, so better switch to ``other" 
%% in advance (won't harm if left quote isn't next):
%% TODO switch what?                                %% 2014/03/27
  \begingroup 
    \let\do\MakeOther \dospecials 
    \MakeOther\|%% 2010/03/09!
    \futurelet\let_token \lq_double_decide}
\newcommand*{\lq_double_decide}{%
  \ifx\let_token\LQverb
    \endgroup
    ``\expandafter \@gobble
%% ... alternative ...
%     \expandafter `%
%% does not recognize next left quote---why? TODO---Corresponding 
%% right quotes will become ``other" due to 
%% having no space at the left. 
%% TODO to be changed with 'wiki.sty'.
  \else
%     \ifx\let_token\protect                      %% rm. 2014/03/28
% %     \show\let_token           %% indeed before v0.6, 2014/03/24
%       \expandafter\expandafter\expandafter \lq_double_decide_ii
%     \else
      \endgroup
      \niceverb_maybe_qs                              %% 2011/09/09
%       \expandafter\expandafter\expandafter \NVerb 
%         \expandafter\expandafter\expandafter \'%
%% <- 2015/11/20 with one conditional less, less \expandafter ->
      \expandafter \NVerb \expandafter \'%
%     \fi
  \fi}
%% `\lq_double_decide_ii' continues test behind `\protect'.
\newcommand*{\lq_double_decide_ii}[1]{%
    \futurelet\let_token \lq_double_decide}
%%
%% ==== Ampersand (or &\cstx) Typesets Meta-Code ====
%% |\CmdSyntaxVerb| will be a permanent alias for the active `&'.
\MakeActive\&
  \newcommand*{\CmdSyntaxVerb}{% 
    \IfTypesetting{%
      \begin_min_verb 
%% v0.3 moves the previous line from &\cmd_syntax_verb 
%% %% <- shows bug is fixed 2010/03/09
%% where it is too late to establish private letters 
%% according to next line which was in `\begin_min_verb' 
%% earlier---an important bug fix!
      \private_letters                                  %% v0.5
      \cmd_syntax_verb
%     }{\protect&\string}}
%     }{\noexpand&\string}}
%% ... with `\string', in an `\edef', the following 
%% command cannot be properly typeset, so 
    }{\noexpand&\noexpand}}                             %% 2014/03/26
%% TODO actually test non-typesetting, maybe introduce 
%% macros that perform tests anywhere ... 
%% %e.g., `\typeout'~... %%% \typeout{``}\show&         %% 2014/03/27
\MakeNormal\&
\newcommand*{\cmd_syntax_verb}[1]{%
  \string#1\futurelet\let_token \after_cs}
%% However, `&' (or `\CmdSyntaxVerb') 
%% may fail with private letters, 
%% % (there should be a hook for them),          %% todo 2010/02/28, 
%%                                               %% rm 2014/03/17 after v0.5
%% especially in \emph{macro arguments}\footnote{%% 2010/03/05
%%     TODO: 'vfoot2e.sty' -- see notes.}
%% and with 'hyperref' in titles of              %% 2010/03/11
%% \emph{sections bearing \cs{label}s},
%% so we provide something like |\cs{<characters>}| from 
%% '\ctanpkgref{tugboat}.sty'. %% 'doc.sty'.%% corr. 2011/05/27 
\DeclareRobustCommand*{\cs}[1]{%
%   \begin_min_verb \backslash_verb #1\egroup} 
%% ... fails with `_' in footnote today (2014/03/19) so:
  \begin_min_verb \withcsname\string#1\endcsname\egroup} %% v0.6
\newcommand*{\backslash_verb}{\char`\\}
%% %% 2011/06/27: undid 2011/05/27
%% Moreover, typing `&\par' in ``short" \emph{macro arguments}
%% fails, you better type `\cs{par}' then. Likewise, 
%% `\cs{if<letters>}' and `\cs{fi}' is safer in case 
%% you want to skip some part of the documentation 
%% (e.g., a package option skips commented code)
%% by \cs{if}`<letters>'\cs{fi}.
%% Finally, there will be PDF bookmarks support for `\cs' 
%% rather than for a real `&' or `\CmdSyntaxVerb' analogue like 
%% |\cstx{<characters>}*[<opt>]{<mand>}| as follows.      %% corr. 2014/03/27
\DeclareRobustCommand*{\cstx}[1]{%              %% corr. 2010/03/17
%   \begin_min_verb \backslash_verb #1%
%% v0.6 like above:
    \begin_min_verb \withcsname\string#1\endcsname
    \futurelet\let_token \after_cs}
\newcommand*{\after_cs}{%
  \ifcat\noexpand\let_token a\egroup \space
  \else \expandafter \decide_verb \fi}
\newcommand*{\test_more_verb}{\futurelet\let_token \decide_verb}
\newcommand*{\decide_verb}{%
    \jumpteg_on_with\bgroup\braces_verb
    \jumpteg_on_with[\brackets_verb
    \jumpteg_on_with*\star_verb
  \egroup}
    %% CAUTION/TODO wrong before (... if cmd without arg
    %%         use \ then or choose usual verb...
    %%         or \MakeLetter\( etc. ... or \xspace
\newcommand*{\jumpteg_on_with}[2]{%
  \ifx\let_token#1\do_jumpteg_with#2\fi}
%% TODO cf. 'xfor', 'xspace' (&\break@loop); 
%% `\DoOrBranch#1'\,...\,`#1' or so. %% found 2010/03/05
%% %% <- `...' fix 2011/01/19
\def\do_jumpteg_with#1#2\egroup{\fi#1}
\def\braces_verb#1{\string{#1\string}\test_more_verb}
\def\brackets_verb[#1]{[#1]\test_more_verb}
\def\star_verb*{*\test_more_verb}
  %% not needed with \Auto... OTHERWISE useful in args!
%% %%2010/03/15:
%% As 'latex.ltx' has `\endgraf' as a permanent alias for the 
%% primitive version of `\par' and `\endline' for `\cr', 
%% we offer |\endcell| as a replacement for the original `&':
\let\endcell&
%%
%% ==== Escape Character Typesets Meta-Code      ====
%% \label{sec:esc}
%% |\BuildCsSyntax| will be a permanent alias for the active escape 
%% character.
\DeclareRobustCommand*{\BuildCsSyntax}{% 
  \futurelet\let_token \build_cs_syntax_sp}
\newcommand*{\build_cs_syntax_sp}{%
  \ifx\let_token\@sptoken 
    \@%                                 %% 2010/12/30
  \else %% TODO ^^M!?
    \expandafter \start_build_cs_syntax
  \fi}
\newcommand*{\start_build_cs_syntax}[1]{%
  \edef\string_built{\string#1}%
%% #1 may be active.---With Donald Arseneau's 'import.sty' (e.g.), 
%% \qtd{&_} may be needed to be `\active' with the meaning of 
%% `\textunderscore', therefore restoring its category code
%% needs some more care than with v0.32 and earlier:
  \edef\before_build_cs_sub{\the\CatCode\_}%
  \private_letters                                  %% v0.5
  \test_more_cs}
\newcommand*{\test_more_cs}{%
  \futurelet\let_token \decide_more_cs}
\newcommand*{\decide_more_cs}{%
  \ifcat\noexpand\let_token a\expandafter \add_to_cs
  \else 
%     \MakeNormalHere\_ 
%% Restoring \qtd{&_} more carefully with v0.4 
%% (`\begingroup' ... `\endgroup'!?): 
%% %% 2010/03/27
    \CatCode\_\before_build_cs_sub
    \MakeOther\@%
%     \expandafter \in@ \expandafter
%       {\csname \string_built \expandafter \endcsname 
%         \expandafter}\expandafter{\niceverbNoVerbList}%
    %% <- useless braces 2014/07/17 ->
    \expandafter \in@ \csname \string_built \expandafter 
        \endcsname \expandafter {\niceverbNoVerbList}%
    \ifin@
      \csname \string_built 
        \expandafter\expandafter\expandafter \endcsname
    \else
      \begin_min_verb \backslash_verb\string_built
        \expandafter\expandafter\expandafter \test_more_verb
    \fi
  \fi}
  %% TODO such \if nestings with ifthen!? 
 %% cf.:
%  \let\let_token,\typeout{\meaning\let_token} 
  %% TEST TODO fuer xspace!? (\ifin@)
\newcommand*{\add_to_cs}[1]{% 
  \edef\string_built{\string_built#1}\test_more_cs}
%%
%% |\AutoCmdSyntaxVerb| starts, |\EndAutoCmdSyntaxVerb| \emph{ends}
%% ``auto mode." 
\newcommand*{\AutoCmdSyntaxVerb}{%
    \MakeActiveLetHere\\\BuildCsSyntax}
\newcommand*{\EndAutoCmdSyntaxVerb}{\CatCode\\\z@}
%% |\NormalCommand{<characters>}| executes `\<characters>'
%% in ``auto mode."
\newcommand*{\NormalCommand}{} \let\NormalCommand\@nameuse
%% %% 2010/03/11:
%% Once I may want to use this feature in {\it Wikipedia}-like 
%% section titles as supported by 'makedoc', yet I cannot really 
%% apply the present feature soon, so this must wait ... 
%% (There is a special problem with `\newlabel' and 'hyperref' ...)
%% 
%% Former tests: 
%  \futurelet\LetToken\relax \relax 
%  \show\LetToken \typeout{\ifcat\noexpand\LetToken aa\else x\fi}
%%
%% |\niceverbNoVerbList| is the list of macros that will be 
%% \emph{executed} instead of being typeset. 
\newcommand*{\niceverbNoVerbList}{%
  \begin\end\item\verb\EndAutoCmdSyntaxVerb\NormalCommand
  \section\subsection\subsubsection} %% TODO!?
%% |\AddToMacro{\niceverbNoVerbList}{<macros>}| can be used to 
%% add <macros> to that list. 
\providecommand*{\AddToMacro}[2]{%   %% TODO move to ... 2010/03/05
  \expandafter \def \expandafter #1\expandafter {#1#2}}
  %% <- was very wrong 2010/03/18
%% Hey, or just |\AddToNoVerbList{<macros>}|: %% 2010/03/28
\newcommand*{\AddToNoVerbList}{\AddToMacro\niceverbNoVerbList}
%%
%% ``Auto mode" probably ain't mean a thing if it ain't invoked using 
%% \[|\AutoCmdInput{<file>}|\] for typesetting <file> in ``auto mode:" 
\newcommand*{\AutoCmdInput}[1]{%
    \begingroup 
      \AddToMacro\niceverbNoVerbList{\ProvidesFile}% 
      %% <- removed `\endinput', will be code! 2010/04/05
      \AutoCmdSyntaxVerb
      \input{#1}%
      \EndAutoCmdSyntaxVerb
    \endgroup
}
%%
%% ==== Meta-Variables                           ====
%% |\MetaVar<var-id>>| will be a permanent alias for the active \qtd{&<}.
%% v0.6 simplifies `\pdfstringdefDisableCommands'. %% was \PDF...
% \def\MetaVar#1>{%
\MakeActive\<
\newcommand*{\MetaVar}{\nvSelfProtect>\nvMetaVar}
\MakeOther\<
\def\nvMetaVar#1>{%
    \mbox{\normalfont\itshape $\langle$#1\/$\rangle$}}
%% As opposed to 'ltxguide.cls', this works outside verbatim as well.
%% TODO: offer without angles as well %% moved down 2014/03/25
%%
%% ==== Hash Mark is Code                        ====
%% |\HashVerb<digit>| will be a permanent alias for the active hash mark. 
\newcommand*{\HashVerb}[1]{{\tt\##1}}
%%
%% ==== Single Right Quotes for &\textsf         ====
%% \label{sec:rqsf}
%% |\RQsansserif| will be a permanent alias for the active single 
%% right quote. 
%%
%% One essential %% 2014/03/25
%% problem with the single right quote feature
%% %problem with the ``single right quote feature" 
%% is that a single right quote may be meant to be an apostrophe. 
%% This is certainly the case at the right of a letter. 
%% On the other hand, we assume that it is \emph{not} an apostrophe 
%%  (i)~in vertical mode (opening a new paragraph), 
%% (ii)~after a horizontal skip. 
%%
%% %% 2014/03/25:
%% Another problem is that with and \LaTeX\ 
%% (as with Plain \TeX---\meta{The \TeX book}\ p.~357), 
%% the right single quote is needed for primes in math mode, 
%% and \LaTeX\ enforces this in `\@outputpage' preparing 
%% `\write's (why? TODO) as well as page headers.
\MakeActive\'
  \newcommand*{\niceverb_rq_choice}[1]{%                %% 2014/03/27
%% We make a deal with `\active@math@prime': in math mode, 
%% the prime functionality acts; outside, the ``right quote
%% sansserif mode" acts. 
%% Test: $a'$ now works with 'niceverb'.---For 
%% page headers, in expanding without typesetting, 
%% the expansion of `\RQsansserif' must contain another active 
%% single right quote.
    \nvSelfProtect'{\ifmmode
                      \expandafter\active@math@prime
                    \else
                      \expandafter#1%
                    \fi}}
%% The following `\do_rq_sansserif' is what |\DoRQsansserif|
%% below was before v0.6. This, too, must be changed for 
%% `\active@math@char', and earlier use of `\DoRQsansserif'
%% in `\niceverb_rq_sf_test' must be replaced.
  \@ifdefinable\do_rq_sansserif
    {\def\do_rq_sansserif#1'{\textsf{#1}}}
  \newcommand*{\RQsansserif}{%
%   \IfTypesetting{\niceverb_rq_sf_test}{\protect'}}
    \niceverb_rq_choice\niceverb_rq_sf_test}
\MakeOther\'
%% Another macro just to avoid more sequences of `\expandafter':
\newcommand*{\niceverb_rq_sf_test}{%
  \ifhmode 
    \ifdim\lastskip>\z@ 
%     \expandafter\expandafter\expandafter \DoRQsansserif
      \expandafter\expandafter\expandafter \do_rq_sansserif
    \else 
      \ifnum\niceverb_spacefactor
        \expandafter\expandafter\expandafter\expandafter
          \expandafter\expandafter\expandafter 
            \do_rq_sansserif %%% \DoRQsansserif
      \else '\fi
    \fi
  \else \ifvmode 
%   \expandafter\expandafter\expandafter \DoRQsansserif
    \expandafter\expandafter\expandafter \do_rq_sansserif
    \else '\fi
  \fi}
% \nvShowProtectedEdef{'niceverb'}
\MakeOther\'
%% |\DoRQsansserif| %% 2010/03/10
%% is \emph{another} (possible) alias for the 
%% active single right quote, below.
\newcommand*{\DoRQsansserif}{%
    \niceverb_rq_choice\do_rq_sansserif}                %% 2014/03/27
%% The following cases are typical and cannot be decided by the 
%% previous criteria:
%% (i)~parenthesis, (ii)~footnotes and after ``horizontal" 
%% environments like `\[<math>\]', (iii)~section titles, 
%% (iv)~`\noindent'. 
%% We introduce some dangerous tricks---redefinitions of 
%% % \LaTeX's internals `\@footnotetext' and `\@sect' 
%% %% <- 2010/03/16 ->
%% \LaTeX's internal `\@sect' and of \TeX's primitives 
%% `\noindent' and `\ignorespaces' as well as by 
%% a signal `\spacefactor' value of 1001. 
%% %In page headers, \LaTeX\ equips the single right quote with the 
%% %meaning of `\active@math@prime' which must be overridden.
%% %TODO more serious! v0.6
%%
%% |\nvAllowRQSS| becomes more powerful with v0.6, 
%% for \secref{listmv}:
\NewSelfProtectedCommand{\nvAllowRQSS}{%
    \MakeActiveLetHere\'\RQsansserif
    \niceverb_rqsf                      %% 2014/03/27
    \niceverb_ignore}                   %% 2010/03/16
%% These and the entire right quote functionality are
%% activated by                 %% removed todo 2010/03/10
%% \[|\nvRightQuoteSansSerif|
%%   \mbox{\quad and disabled by\quad} 
%%   |\nvRightQuoteNormal|\]---at 
%% `\begin{document}'---where we collect previous settings---or 
%% later:
\AtBeginDocument{%
    \edef\before_niceverb_parenthesis{\the\sfcode`\(}%
    \newlet \before_niceverb_ignore   \ignorespaces     %% 2010/03/16
    \newlet \before_niceverb_sect     \@sect    %% \newlet 2014/03/25
    \newlet \before_niceverb_noindent \noindent}        %% 2010/03/08
%% We assume that `\@sect' has the same 
%% % We assume that `\@footnotetext' and `\@sect' have the same 
%% parameters there as in \LaTeX\ 
%% (even if redefined by another package, like 'hyperref').
\def\niceverb_sect#1#2#3#4#5#6[#7]#8{%
    \before_niceverb_sect{#1}{#2}{#3}{#4}{#5}{#6}%
%                       [{\protect\nvAllowRQSS #7}]%
%                        {\protect\nvAllowRQSS #8}}
%% With v0.6, a more general |\NiceVerbMove{<text>}| 
%% is introduced, defined in \secref{listmv}:
                        [\NiceVerbMove{#7}]%
                        {\NiceVerbMove{#8}}}
%% 2010/03/20:
\newcommand*{\niceverb_spacefactor}{\spacefactor=1001\relax}
\newcommand*{\niceverb_noindent}{%
    \before_niceverb_noindent \niceverb_spacefactor} 
\newcommand*{\niceverb_ignore}{% 
    \ifhmode \niceverb_spacefactor \fi \before_niceverb_ignore}
%%
%% Here are the main switches. With v0.6, |\nvRightQuoteSansSerif| 
%% is divided into two parts, for \secref{listmv}:
\newcommand*{\niceverb_rqsf}{%                      %% 2014/03/27
%   \MakeActiveLet\'\RQsansserif
    \sfcode`\(=1001   %% enable in parentheses 2009/04/10
%% I also added \HardNVerb+\sfcode`/=1001+ in the preamble 
%% of 'makedoc.tex'. %% 2010/03/15
%   \let\@footnotetext\niceverb_footnotetext
    \let\ignorespaces\niceverb_ignore               %% 2010/03/16
%   \let\@sect\niceverb_sect
    \let\noindent\niceverb_noindent}                %% 2010/03/08
\newcommand*{\nvRightQuoteSansSerif}{%
    \niceverb_rqsf 
    \MakeActiveLet\'\RQsansserif
    \let\@sect\niceverb_sect
    \def\niceverb_rqsf_kind{\nvAllowRQSS}}
%% <- It really must be `\def' in order to transmit the 
%% choice to the table of contents.
%%
%% With v0.6, in dealing with moving things in 
%% \secref{listmv}, section titles are handled 
%% in a more complex way. We divide the former 
%% |\nvRightQuoteNormal| into two parts:
\newcommand*{\niceverb_rq_normal}{%
%   \MakeNormal\'%                                      %% 2010/03/21
    \sfcode`\(=\before_niceverb_parenthesis\relax
    \let\ignorespaces\before_niceverb_ignore            %% 2010/03/16
    \let\noindent\before_niceverb_noindent}             %% 2010/03/08
\MakeActive\`
\newcommand*{\nvRightQuoteNormal}{%
    \MakeNormal\'%                                      %% 2010/03/21
    \niceverb_rq_normal
    \let\nv_rqsf_kind\@empty
    \ifnum\CatCode\`=\active                %% `=' missing 2015/11/09
      \ifx`\LQverb \else
        \let\@sect\before_niceverb_sect
      \fi
    \else
        \let\@sect\before_niceverb_sect
    \fi}
\MakeOther\`
%% |\nvAllRightQuotesSansSerif| %% 2010/03/10
%% (after `\begin{document}'!)
%% forces the `\textsf' feature 
%% \emph{without} testing for apostrophes. You then must be 
%% sure---DANGER! CARE!---to 
%% use \qtd{&\rq} only for obtaining an apostrophe and the 
%% double quote character \qtd{&"} for closing double quotes, 
%% or our `\dqtd{<text>}' for the entire quoting.{\sloppy\par}
\newcommand*{\nvAllRightQuotesSansSerif}{%
    \niceverb_rq_normal %%% \nvRightQuoteNormal         %% 2014/03/27
%% That's one use of |\DoRQsansserif| with v0.6:
    \MakeActiveLet\'\DoRQsansserif
    \def\niceverb_rqsf_kind{\nvAllRQSS}}                %% 2014/03/27
%% <- must be `\def' for transmissions.
\NewSelfProtectedCommand{\nvAllRQSS}{%
        \niceverb_rq_normal
%% That's the other use of |\DoRQsansserif| with v0.6:
        \MakeActiveLetHere\'\DoRQsansserif}
%% [Hiding remarks from 2010f.\ (`\ctanpkgref') 2014/03/23]
%% %I started v0.31 (signal `\sfcode'=1000, lowercase letters 
%% %get `\sfcode'=1001) because 
%% %\NVerb+\href{http://ctan.org/pkg/<pkg>}{<pkg>}+ failed. 
%% %However, what I actually needed was |\ctanpkgref{<pack-name>}|: 
%% % \DeclareRobustCommand*{\ctanpkgref}[1]{%
%% %     \href{http://ctan.org/pkg/#1}{\textsf{#1}}}
%% %... moves to 'texlinks.sty' 2011/01/24.
%%
%% ==== Boxes Highlighting Commands and Syntax   ====
%% With v0.3, we include one kind of command syntax boxes 
%% whose <content> is (in 'niceverb' syntax) 
%% delimited as \GenCmdBox+|<content>|+.
%% %% 2010/03/14: %% \newsavebox!? moves up 2014/03/19
%% |\GenCmdBox<char><content><char>}| works like 
%% `\NVerb<char><content><char>' except putting the latter's result 
%% into a framed (or coloured or ...) box.{\sloppy\par}
\newcommand*{\GenCmdBox}  {\_no_nice_meta_verb_false \gen_cmd_box}
%% |\HardVerbBox| is a variant of `\GenCmdBox' with the meta-variable 
%% feature disabled 
%% (for the documentation of the present package).
\newcommand*{\HardVerbBox}{\_no_nice_meta_verb_true  \gen_cmd_box}
\newcommand*{\gen_cmd_box}{%
% \let\niceverb_egroup\nice_collect_verb_egroup %% rm 2014/03/18
% \setbox\niceverb_savebox \hbox\bgroup 
%% <- 2014/03/19 ->
  \SetNiceVerbSaveBox
%     \if_no_nice_meta_verb_ 
%           \expandafter \HardNVerb
%     \else \expandafter \NVerb     \fi
%% <- 2014/03/19 -> [TODO use generalization]
    \NiceMaybeMetaVerb[\nice_collect_verb_egroup]%
}
\newcommand*{\nice_collect_verb_egroup}{%
    \egroup \egroup
  \ifvmode \expandafter \VerticalCmdBox
  \else    \ifmmode \hbox \fi
           \expandafter \InlineCmdBox \fi
%               {\box\niceverb_savebox}%
%% <- 2014/03/19 -> 
                \TheNiceVerbSaveBox
%% %Modifying invocation of `\niceverb_normal_egroup' 2011/11/05
%% %according to remark of 2010/03/15 for saving nesting level:
%% (Removing a remark that I don't understand 2014/03/19.)
  \ifmmode\else\@\fi
% \let\niceverb_egroup\niceverb_normal_egroup %%     rm 2014/03/19
}
%% |\nvCmdBox| will be the permanent alias for \qtd{&|}.
\newcommand*{\nvCmdBox}{\GenCmdBox\|}
%% |\VerticalCmdBox{<content>}| 
%% may eventually start a `decl' environment 
%% as in 'ltxguide.cls', looking ahead for another \qtd{&|} 
%% in order to (perhaps) append another row.
%% Another possibility is first to do some
%% \[`\if@nobreak\else \pagebreak[2]\fi'\]
%% etc. and then invoke `\InlineCmdBox'.
%% The user can choose later by some `\renewcommand'. 
%% We do the perhaps most essential thing here 
%% (again cf. `\begin_min_verb'):{\sloppy\par}
\newcommand*{\VerticalCmdBox}{%
%% v0.6 encourages a page break here 
%% according to the above idea, in order to avoid 
%% a page break after explaining subsequent code 
%% (TODO: that's a major functionality change):
             \if@nobreak\else \pagebreak[2]\fi
             \leavevmode\InlineCmdBox}
%% (2011/11/05 removing `\null'.)
%% The command declaration boxes in the documentation of 
%% Nicola Talbot's \ctanpkgref{datatool}
%% would be an especially nice realization of 
%% `\VerticalCmdBox'.\footnote{I find the documentation 
%%     of Martin Scharrer's \ctanpkgref{newverbs} package 
%%     similarly impressive.}
%%
%% |\InlineCmdBox{<content>}|, according to our idea, should not 
%% change baseline skip, even with some `\fboxsep' and `\fboxrule'.
%% (However, it may be a good idea to increase the overall 
%%  normal baseline skip.)
%% We therefore replace actual height and depth of the content by 
%% the height and depth of math parentheses.
\newcommand*{\InlineCmdBox}[1]{%
  \bgroup
%% ... needed in math mode with `\begin_min_verb'.
    \fboxsep 1pt
    \kern\SetOffInlineCmdBoxOuter
    \smash{\SetOffInlineCmdBox{\kern\SetOffInlineCmdBoxInner
                               \InlineCmdBoxArea{#1}%
                               \kern\SetOffInlineCmdBoxInner}}%
    \mathstrut
    \kern\SetOffInlineCmdBoxOuter
  \egroup
}
%% The default choice for |\SetOffInlineCmdBox| is `\fbox':
\newlet\SetOffInlineCmdBox\fbox
%% You can `\renewcommand' it to change `\fboxsep', `\fboxrule'
%% etc. or to use a `\colorbox' with the 'color' package, e.g.,
%% I used the following setting so far:
%% %% 2010/03/10
%% \begin{verbatim}
%%     \RequirePackage{color}
%%     \renewcommand*{\SetOffInlineCmdBox}
%%                   {\colorbox[cmyk]{.1,0,.2,.05}}
%% \end{verbatim}
%% |\SetOffInlineCmdBoxInner| enables controlling the inner 
%% horizontal space to the box margin independently of 
%% `\fboxsep'. 
%% %% We set it to 0\,pt. as default (it is a macro only, 
%% %% for a while).
\newcommand*{\SetOffInlineCmdBoxInner}{-\fboxsep\thinspace}
%% This choice is inspired by `\cstok' for ``boxed" things 
%% in Knuth's 'manmac.tex' which formats {\it The \TeX book}.
%% %% <- 2010/03/10
%%
%% |\SetOffInlineCmdBoxOuter| allows that the box hangs out into the 
%% margin horizontally. 
%% We set it to 0\,pt as default (it is a macro only, for a while).
\newcommand*{\SetOffInlineCmdBoxOuter}{\z@}
%% The height and depth of the frame should be the same for all 
%% inline boxes, we think.
%% The present choice |\InnerCmdBoxArea| for the spacing
%% respects code characters rather than the height and depth 
%% of the angle brackets that surround meta-variable names.
\newcommand*{\InlineCmdBoxArea}[1]{%
    \smash{#1}\vphantom{gjpq\backslash_verb}}
%% \GenCmdBox+\cmdboxitem|<content>|+ is another variant of 
%% `\GenCmdBox'. It should replace `\item[<content>]' in the 
%% `description' environment. %% 2010/03/15
\newcommand*{\cmdboxitem}{%
% \bgroup 
%   \let\niceverb_egroup\cmd_item_egroup
%   \global %% TODO!? 2010/03/15
%   \setbox\niceverb_savebox \hbox\bgroup
%% <- 2014/03/19 ->
    \SetNiceVerbSaveBox
%     \NVerb}
%% <- 2014/03/19 ->
      \NVerb[\cmd_item_egroup]}
\newcommand*{\cmd_item_egroup}{%
      \egroup \egroup %%% \egroup                %% 1 less 2014/03/19
  \item[\InlineCmdBox\TheNiceVerbSaveBox]}
%% %% 2014/03/19:
%% Does it work?
%% \begin{description}
%%   \cmdboxitem+\foo{<arg>}+ could be defined for a test.
%%   \cmdboxitem+\bar{<arg>}+ could be defined for a test as well.
%% \end{description}
%%
%%  === When 'niceverb' Gets Nasty ===
%% These things are new with v0.3.
%% ==== Meta-Variables ====
%% This is even newer than v0.3.    %% 2011/05/09
%%
%% In case you actually need $\lt$ and $\gt$ in math mode, 
%% |\lt| and |\gt| are ``provided" as aliases:
\providecommand*{\gt}{>}
\providecommand*{\lt}{<}
%%
%% ==== Quotes         ====
%% %% WRONG 2010/03/05:
%% % The left quote feature for meta-code requires that the right quote 
%% % feature---for replacing `\textsf'---is activated---no, rather:
%% % that the right quote \qtd{&'} is `\active'.
%% %
%% In order to get \emph{real} single quotes, you could use `\lq <text>\rq', 
%% maybe appending a `\ ', but the code |\qtd{<text>}| may look better 
%% and be easier to type.
\providecommand*{\qtd}[1]{`#1'}             %% provide 2012/11/27
%% However, here we get the problem that the left quote in 
%% \NVerb\+\qtd{`<code>'}+ will be unable to switch into 
%% verbatim mode entirely---then use `&', 
%% e.g., \qtd{&\qtd{&&&&}} typesets \qtd{\qtd{&&}}, i.e., 
%% the ampersand in single (non-verbatim) quotes.
% todo \qtdverb!? alternative meaning for \LQverb!? 2010/03/06
%      rather rare, & takes less space              2010/03/09
%% ... see approaches below~...
%% 
%% |\AddQuotes| automatically surrounds code with single quotes. 
%% I have so often felt that it was a design mistake 
%% to drop them (2011/09/09):
\newcommand*{\AddQuotes}{%
    \let\niceverb_maybe_qs\niceverb_add_qs}
\newcommand*{\niceverb_add_qs}{%
%% In a math display, quotes are suppressed even with `\AddQuotes':
    \ifmmode\else
      `\let\niceverb_maybe_rq\niceverb_rq
    \fi}
\newlet\niceverb_maybe_rq\relax
\newcommand*{\niceverb_rq}{'\let\niceverb_maybe_rq\relax}
%% You can undo this by |\DontAddQuotes|:
\newcommand*{\DontAddQuotes}{\let\niceverb_maybe_qs\relax}
%% The default will be the behaviour that we had before:
\DontAddQuotes 
%% With v0.6, |\qtdnverb<char><m-verb><char>| encloses the 
%% ``meta-verbatim" material with single quotes:
\newcommand*{\qtdnverb}{%
%% Useless after `\AddQuotes':           %% mod. 2015/11/20
  \ifx\niceverb_maybe_qs\niceverb_add_qs
    \expandafter \NVerb
  \else
    \lq
    \expandafter\NVerb\expandafter
        [\expandafter\niceverb_egroup_rq\expandafter]%
%% <- The comment mark proved essential 2015/11/20.
%% The line was commented out previously, obviously I had forgotten 
%% that the feature didn't work.
  \fi}
%% Completing the work v0.62:
\newcommand*{\niceverb_egroup_rq}{\niceverb_normal_egroup\rq}
%% |\dqtd{<text>}| can be used for enclosing in \emph{double} quotes 
%% with the dangerous `\nvAllRightQuotesSansSerif' (see above).
\providecommand*{\dqtd}[1]{``#1"}                   %% 2012/11/27
%%
%% ==== 'hyperref'     ====
%% \label{sec:hyperref}
%% This is for\slash about compatibility with the 'hyperref' package. 
%% (One preliminary thing: in doubt, don't load 'niceverb' earlier 
%% than 'hyperref'.) %% TODO 2010/03/14 
%%
%% We need some substitutions for PDF bookmarks with 
%% 'hyperref'. We issue them at `\begin{document}' when we know 
%% if 'hyperref' is at work.\footnote{An alternative approach would be 
%%                                    using 
%%                                    \ctanpkgref{afterpackage}
%%                                    by Alex Rozhenko.}
\AtBeginDocument{%
  \@ifpackageloaded{hyperref}{%
    \newcommand*{\PDFcstring}{%         %% moved here 2010/03/09
      \134\expandafter\@gobble\string}% %% ASCII octal encoding
    \pdfstringdefDisableCommands{%
      \let\nvAllowRQSS\empty            %% not \relax 2010/03/12
      \let\NiceVerbGeneral\empty        %% 2014/03/27
      \let\nvAllRQSS\empty              %% 2014/03/27
      \MakeActiveLetHere\<<%            %% 2014/03/28
      %% 2010/03/12
      \MakeActiveLetHere\`\lq \MakeActiveLetHere\'\rq
      \MakeActiveLetHere\&\PDFcstring 
      \def\cs{\134}%                    %% 2010/03/17, 2011/06/27
%% The typesetting version of `\BuildCsSyntax' (\secref{esc}): 2014/07/16
%%                                             %% showfile.tex 2014/07/12
      \withcsname\def BuildCsSyntax \endcsname{\cs}%
%% ... disables `\niceverbNoVerbList'; better switch off 
%% auto mode with section headings TODO                 %% 2014/07/17
%% (modify `\@startsection')
      \let\decide_more_cs\bookmark_more_cs
    }%
%% Moreover, in order to avoid spurious                 %% 2010/03/14
%% \texttt{Label(s) may have changed} with 'hyperref', 
%% a single right quote must be \emph{read} as active 
%% by a `\newlabel' if and only if it has been active when 
%% `\@currentlabelname' was formed.\footnote{This uses 
%%      \cs{@onelevelsanitize}, therefore &\protect doesn't 
%%      change the behaviour of ``active" characters.} 
%% as `\active'. We use `\protected@write' as this cares for 
%% `\nofiles'. `\@auxout' may be `\@partaux' for `\include'.
    \newcommand*{\niceverb_aux_cat}[2]{%                %% 2010/03/14
        \protected@write\@auxout{}{\string#1\string#2}}%
%% v0.5 restricts ``activating" to `\MakeActiveLet':
%     \renewcommand*{\MakeActive}[1]{%
%         \MakeActiveHere#1%
%         \niceverb_aux_cat\MakeActiveHere#1}%
    \renewcommand*{\MakeActiveLet}[2]{%
        \MakeActiveLetHere#1#2%
%         \niceverb_aux_cat\MakeActiveHere#1}%
        \protected@write\@auxout{}{%
            \string\MakeActiveLetHere\string#1\string#2}}% 
    \renewcommand*{\MakeNormal}[1]{%
        \MakeNormalHere#1%
        \niceverb_aux_cat\MakeNormalHere#1}%
  }{}%
}
%% % todo doesn't 'babel' have the same problem? 2010/03/12
%% ==== 'hyper-xr'     ==== %% 2010/03/21
%% With the 'hyper-xr' package creating links into external documents, 
%% preceding `\externaldocument{<file>}' with 
%% `\MakeActiveLet\&\CmdSyntaxVerb' may be needed. 
%% I do not want to redefine something here right now as %% typo 2010/11/09 
%% I have too little experience with this situation.
%% 
%% ==== Listings and Moving ==== %% 2014/03/26f.
%% \label{sec:listmv}
%% Working on v0.6, in testing I discovered a problem with 
%% the \emph{listing environments}. The present documentation 
%% uses code listings with 'makedoc', which build on the 
%% 'moreverb' package and eventually call \LaTeX's 
%% `\@noligs' macro.\footnote{\ctanpkgref{moreverb}
%%     is used, and its listing environments use \cs{@verbatim} 
%%     from the \ctanpkgref{verbatim} package, then 
%%     \cs{verbatim@font} calls \cs{@noligs}~...}
%% The problem also appears with the `{verbatim}' environment 
%% from the \LaTeX\ kernel ('latex.ltx') as well as from the
%% \ctanpkgref{verbatim} package---with anything that calls 
%% \LaTeX's `\@noligs'. The latter assigns special meangings 
%% to the active characters listed in the `\verbatim@nolig@list', 
%% three of them need a different meaning with 'niceverb'.
%% When a page break happens after such an environment has been 
%% entered (this may well be when the environment falls to the 
%% beginning of the next page), these settings are used in 
%% \LaTeX's `\@outputpage' for running the `\write's of the page 
%% as well as for page headers. And this happens quite often 
%% in a package documentation!
%%
%% The problem was reported by Walter Schmidt with respect to 
%% math primes as `latex/3104' in 1999. I cannot reproduce    %% rm. to 2015/11/09
%% it, and I see two reasons in recent `latex.ltx' code 
%% why it cannot happen anymore. However, one remedy in `latex.ltx'
%% is activating `\active@math@prime' in `\@resetactivechars' 
%% of `\@outputpage'. 
%% But this is bad for 'niceverb''s single right quote. 
%% We override the `\active@math@prime' functionality and
%% verbatim `\@noligs' by appending 
%% some protection of the characters collected
%% in `\verbatim@nolig@list' 
%% % local restorement |\useNiceVerbHere| of our syntax.
%% to `\@resetactivechars'.
%% This solves the problem for `\write's at `\shipout'.
%% |\do_protect_noligs| is used for this purpose;
%% actually it is applied in |\useNiceVerb|
%% (\secref{allonoff}):{\sloppy\par}
\newcommand*{\do_protect_noligs}[1]{%                   %% 2014/03/28
    \MakeActiveLetHere#1\relax}          %% `Here' missing 2015/11/09
%% |\nvResetPages| can be used to restore \LaTeX's 
%% `\@resetactivechars'. I don't add it to `\noNiceVerb'
%% because it could corrupt `\write's, so should be used with care.
%% % Actually I think that our new version of `\@resetactivechars'
%% % is better than \LaTeX's, I think I can recommend to use it 
%% % always and even to modify `latex.ltx' accordingly.
%% % TODO!? math headings? without 'niceverb' restore right 
%% % single quote for math prime after `\shipout'.
\AtBeginDocument{%
    \newlet\latex_reset_actives\@resetactivechars}
\newcommand*{\nvResetPages}{%
    \let\@resetactivechars\latex_reset_actives}
%% |\NiceVerbMove{<text>}| with v0.6 is for ``moving" arguments so that 
%% 'niceverb' syntax operates \emph{locally} at the destination,
%% I think of table of contents and page headers.
%% It is automatically used by 'niceverb''s variant of \LaTeX's 
%% sectioning commands (\secref{rqsf}); while with `\markboth', `\markright', 
%% `\addcontentsline' etc. you must it include yourself (currently, TODO?).
%% This is meant as a remedy against \LaTeX's and \ctanpkgref{verbatim}'s 
%% `\@noligs' with respect to page headers. However, 
%% another purpose is that you could switch off the 'niceverb'
%% syntax at the beginning of your document (by `\noNiceVerb'), 
%% though certain entries to the table of contents can use 
%% 'niceverb' syntax without affecting other entries
%% (where some active characters may have different meanings, 
%%  perhaps from a different package).
\newcommand*{\NiceVerbMove}[1]{%
%% What goes to `.aux' files must not have underscores:
    {\NiceVerbGeneral\niceverb_rqsf_kind#1}}
\NewSelfProtectedCommand{\NiceVerbGeneral}{%
%% %... TODO only 2 applications of `\NewSelfProtectedCommand'!?
%% %
%% % Here is |\useNiceVerbHere|:
% \newcommand*{\useNiceVerbHere}{%                      %% 2014/03/28
    \let\MakeActiveLet\MakeActiveLetHere \useNiceVerbI}
% \newcommand*{\NiceVerbGeneral}{%
%     \nvSelfProtect\NiceVerbGeneral\useNiceVerbHere}
%%
%% ==== Turning off and on altogether ====
%% \label{sec:allonoff}
%% These commands are new with v0.3. 
%%
%% |\noNiceVerb| \emph{disables} all 'niceverb' features. 
\newcommand*{\noNiceVerb}  {\MakeNormal\`%
                            \MakeNormal\&%
                            \MakeNormal\<%
                            \MakeNormal\#%
                            \nvRightQuoteNormal
                            \MakeNormal\|%
                  \let\@sect\niceverb_before_sect}      %% 2014/03/27
%% |\useNiceVerb| \emph{activates} all the 'niceverb' features 
%% (apart from ``auto mode"). With v0.6, it is divided 
%% into two parts for `\NiceVerbMove' in \secref{listmv}:
\newcommand*{\useNiceVerbI}{\MakeActiveLet\`\LQverb
%% TODO to be changed with 'wiki.sty' v0.2
                            \MakeActiveLet\&\CmdSyntaxVerb
                            \MakeActiveLet\<\MetaVar
                            \MakeActiveLet\#\HashVerb
                            \nvRightQuoteSansSerif
                            \MakeActiveLet\|\nvCmdBox}
\newcommand*{\useNiceVerb} {\useNiceVerbI               %% 2014/03/27
                            \let\@sect\niceverb_sect
    \g@addto@macro\@resetactivechars{%
%                           \useNiceVerbHere            %% 2014/03/28
    \let\do\do_protect_noligs \verbatim@nolig@list      %% 2014/03/28
}}
%% 
%%
%%  === Minor Final Things         ===
%% ==== Activating the 'niceverb' Syntax ====
%% 'niceverb' features are activated at `\begin{document}' 
%% so (some) other packages can be loaded \emph{after} 'niceverb'. 
%% For v0.3, we do this after possible settings for 
%% compatibility with 'hyperref'.
\AtBeginDocument{\useNiceVerb}
%%
%% ==== Leave Package Mode               ====
\PopLetterCat\_                                         %% 2012/08/27
\endinput
%%
%%==== VERSION HISTORY ====

v0.1   2009/02/21   very first, sent to CTAN
v0.2   2009/04/04   ...NoVerbList: \subsubsection, \AddToMacro, 
       2009/04/05   \SimpleVerb makes more other than iii
       2009/04/06   just uses \dospecials
       2009/04/08   debugging code for rq/sf, +\relax
       2009/04/09   +\verb@eol@error, prepared for new doc method, 
                    removed spurious \makeat..., -\relax (ligature), 
       2009/04/10   ('-trick
       2009/04/11   \@ after \SimpleVerb
       2009/04/14   noted TODO below
       2009/04/15   change v0.1 to 2009/02/21
v0.30  2010/02/27   short, more explained, \AssignCatCodeTo,
                    use \MakeActive for re-activating, \MakeNormal
       2010/02/28   fixed @ and _ with & by moving \begin_min_verb;
                    replaced \lq by `; Capitals in Titles
       2010/03/05   \SimpleVerb -> \NVerb; 
                    use \MakeActive + \MakeNormal; \rq -> ';
                    renamed some sections; \lq_verb -> \LQverb, 
                    \niceverb_meta -> \MetaVar, 
                    \param_verb -> \HashVerb
       2010/03/06   removed \MakeAlign; removed @ and _ todo below;
                    \NVerb makes ` and ' other; 
                    \nvAllowRQSF allows ' in column titles,
       2010/03/08   \LQverb and & work in column titles, 
                    \RQverb works with \noindent; 
                    bookmark substitutions
       2010/03/09   extended notes on 'hyperref' (in)compatibility;
                    \MakeLetter\@ in \CmdSyntaxVerb only;
                    |...| implemented as \prepareCmdBox etc.!
       2010/03/10   \colorbox example, \thinspace; ltxguide!;
                    removed todo; ..._exec -> \DoRQsansserif; 
                    minor doc changes in ``Nasty"
       2010/03/11   doc changes in ``Escape Character ..." and 
                    ``Ampersand"
       2010/03/12   \niceverb_aux_cat, \MakeActiveHere etc., 
                    \IfTypesetting, \noNiceVerb, \useNiceVerb,
                    corr. bracing mistake in \MakeNormal!
       2010/03/14   0.31 -> 0.3; \HardNVerb, \GenCmdBox, 
                    \prepareCmdBox -> \nvCmdBox 
       2010/03/15   \endcell; \cmdboxitem; remark on \sfcode`/
       2010/03/16   corr. -> \endline; 
                    advice on \cs{par}, \cs{if...}, \cs{fi};
                    redefined \ignorespaces for RQ feature
       2010/03/17   corr. `\fututelet', corr. \cs PDF substitution
       2010/03/18   |\niceverbNoVerbList|, |\AddToMacro| etc.;
                    corr. \AddToMacro; 
                    \lastskip-fix of \niceverb_ignore, 
                    another fix of \niceverb_noindent
       2010/03/19   another fix of \niceverb_ignore: \spacefactor
       2010/03/20   ... again: \niceverb_spacefactor

NOT DISTRIBUTED, just stored saved as separate version

v0.31  2010/03/20   right quote feauture: letters get \sfcode=1001
                    `column title' -> `page headers', \ctanpkgref

NOT DISTRIBUTED, just stored as separate version

v0.32  2010/03/21   taking best things from v0.30 and v0.31
       2010/03/23   removed \relax from \IfTypesetting
SENT TO CTAN

v0.4   2010/03/27   restoring `_' with "auto mode" safer
       2010/03/28   \AddToNoVerbList
       2010/03/29   note above, renamed v0.4
SENT TO CTAN

v0.41  2010/04/03   v0.33 -> v0.4 
       2010/04/05   corrected \AutoCmdInput list
SENT TO CTAN as part of NICETEXT release r0.41

v0.41a 2010/11/09   typo corrected
v0.42  2010/12/30   corr. `\ ' emulation in auto mode
       2010/12/31   \MetaVar in ...maybe_meta...
       2011/01/19   `...' fix
       2011/01/24   \ctanpkgref moves to texlinks.sty
       2011/01/26   update (C)
with nicetext RELEASE r0.42
v0.43  2011/05/09   \gt, \lt
       2011/05/27   \cs uses \@backslashchar
       2011/06/20   \MakeActiveLetHere in \nice_maybe_meta_verb !!!
       2011/06/27   2011/05/27 undone
       2011/08/20   `r0.42', `v0.43'
with nicetext RELEASE r0.43
v0.44  2011/09/09   \AddQuotes, \DontAddQuotes
with nicetext RELEASE r0.44
v0.45  2011/11/05   mod. \niceverb_collect_egroup/\VerticalCmdBox, 
                    tried \output problem without avail
       2011/12/05   clarified "r0.44"
with nicetext RELEASE r0.5 

v0.5   2012/08/27   using 'catcodes', \providecommand\CatCode, 
                    rm. \AssignCatCodeTo, \private_letters 
       2012/08/28   fixed \private_letters; 
                    rewording for filling lines
       2012/09/27   corrections about \MakeActive...
with nicetext RELEASE r0.6

v0.6   2012/11/27   \[d]qtd only \provide'd 
v0.61  2014/03/18   doc.: rm. TODO on private letters hook,
                    folding history tighter, 
                    RM CODE COMMENTED OUT IN 2011;
                    \VerticalCmdBox gets \pagebreak[2]
       2014/03/19   doc.: strange replaced, restructured, 
                    Command-Highlighting Boxes -> Boxes 
                    Highlighting ...;
                    opt. arg. for \NVerb etc. replaces 
                    \niceverb_egroup, \cs/\cstx enhanced, 
                    reimpl.s with \SetNiceVerbSaveBox, 
                    \nice_maybe_meta_verb -> \NiceMaybeMetaVerb
       2014/03/20   reworking robustness -- doc., ...; doc. on 
                    `Shared ...', \qtdnverb
       2014/03/21   "debugging": \noexpand vs. \protect
       2014/03/22   ... continued; mod. \MakeNormal
       2014/03/23   ..., hiding ...
       2014/03/24   TODO on left quotes, doc. test there, 
                    rm. babel-TODO
       2014/03/25   doc. about left quotes shorter, rm. earlier page
                    breaks, doc. problems with right quotes; \newlet;
                    dealing with \active@math@prime
       2014/03/26   corr. test for right single quote, more about
                    \active@math@prime, corr. \CmdSyntaxVerb
       2014/03/27   different treatment of \active@math@prime ...
                    main work for sec:listmv and independent 
                    switching for rqsf, \NewSelfProtectedCommand
                    (3 applications); doc. corr., TODO; 
                    \typeout test
       2014/03/28   test section; [TODO?]; \nvShowProtectedEdef, 
                    \MetaVar: protection and hyperref version; 
                    remark \NVerb; [\NiceVerbHere]; \protect test
                    with RQ removed; \do_protect_noligs fixed 
                    and used
       2014/07/16   \BuildCsSyntax with hyperref
       2014/07/17   remarks on \BuildCsSyntax and removing useless 
                    braces there (hours of trying better)
       2015/02/23   doc.: \PDFstring -> \pdfstring
       2015/04/07   doc.: page headings -> page headers
       2015/11/09   bugfixes \nvRightQuoteNormal and 
                    \do_protect_noligs; doc. typo fix
with nicetext RELEASE r0.66
v0.62 2015/11/20f.  \qtdnverb was disabled! 
                    doc.: fn. on \qtdnverb obsolete


\end{document}

HISTORY

2009/04/09  adjusted to new doc-generation method
2009/04/12  examples, 'awk' lower-case
2009/04/15  example 'mdcorr.cfg', abstract, 
            \pagebreak to implementation
2010/02/27  replaced `|' by `+' with \verb 
            so `|' works as announced
2010/02/28  "Missing:" ''...'' 'wiki' feature, 
            somethings aren't missing anymore 
            (or otherwise removed); more on quotes; 
            applying |...| 
2010/03/05  \SimpleVerb -> \NVerb; after intro on `&' quotes as well
2010/03/06  typo in ``examples''; removed makedoc.cfg sample; 
            more on `&'
2010/03/07  without \listfiles
2010/03/09  hyperref ... \input{mdcorr.cfg}!, |...| settled
2010/03/10  moved pdf stuff to 'makedoc.cfg'; 
            do use 'mdcorr.cfg' for demo; future of |
2010/03/11  applied \MakeJobDoc and shortened preamble; 
            various minor doc changes
2010/03/12  ``Ampersand" improved; \noNiceVerb + \useNiceVerb
2010/03/14  use \InlineCmdBox and \HardVerbBox; |...| described
2010/03/18  \AddToMacro; ``auto mode" tested seriously (substr.sty) 
            - \AutoCmdInput
2010/03/19  line break changes; '' -> " 
2010/03/20  testing niceverb v0.31
2010/03/23  `mdoccorr.cfg' example again
2010/03/27  ``auto mode,"
2010/03/29  \mbox -> \hbox in display; arseneau.tex/pdf
2010/04/05  Harder -> Harders
2010/11/27  \ProvidesFile for myfilist
2010/12/29  \AddToNoVerbList
2011/01/26  using color.sty and readprov.sty; 
            ack. Stephan B. for <...>; auto headings issue
2011/05/09  undoubled lines about `&'
2011/08/22  using new makedoc.cfg features 
2011/10/07  `syntacic'
2011/11/05  modified pdftitle
2012/10/10  \AddQuotes, \DontAddQuotes

FOR V0.6:
2014/03/19  subsec. Preliminaries, subsubsec. Header, `Implementation
            of the Markup Syntax' -> `The Package File'
2014/03/24  experiment with former
2014/03/25  debugging in abstract
2014/03/26  trying class option `fleqn' / debugging again
2014/03/27  \secref; |\NiceVerbMove|, |\nvAllRightQuotesSansSerif|;
            tests
2014/03/28  debugging; \lessthan -> \lt; \newpage
