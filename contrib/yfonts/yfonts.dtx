%\CheckSum{194}
%
% \iffalse meta comment
%
% Copyright (c) Walter Schmidt 1997--2003
%
% This program can be redistributed and/or modified under the terms
% of the LaTeX Project Public License Distributed from CTAN
% archives in directory macros/latex/base/ as file lppl.txt; either
% version 1 of the License, or (at your option) any later version.
% 
% \fi
%
% \iffalse
%
%<*driver>
\ProvidesFile{yfonts.drv}
%</driver>
%<package>\ProvidesPackage{yfonts}
  [2003/01/08 v1.3 (WaS)]     
%
%<*driver> 
\documentclass[11pt,openbib]{ltxdoc}
\usepackage{german}
\setcounter{StandardModuleDepth}{1}
\CodelineNumbered
\hyphenation{Makro}
\begin{document}
 \DocInput{yfonts.dtx}
\end{document}
%</driver>
% \fi
%
% \newcommand{\MF}{\textsf{METAFONT}}
% \GetFileInfo{yfonts.drv}
% \DeleteShortVerb{\|}
% \title{Ein Makropaket f"ur die gebrochenen Schriften}
% \author{Walter Schmidt\\ $\langle$\texttt{w-a-schmidt@arcor.de}$\rangle$}
% \date{Version: \fileversion{} -- \filedate}
% \maketitle
% \tableofcontents
%
% \section{Gebrochene Schriften und \LaTeX}
% Im deutschen Sprachraum waren vom Mittelalter an bis ins 20.~Jahrhundert
% die sogenannten gebrochenen Schriften weit verbreitet.
% Innerhalb der Gruppe der gebrochenen Schriften unterscheidet man drei
% Untergruppen, n"amlich Gotisch, Schwabacher und Fraktur; manchmal
% werden aber auch alle drei pauschal als Frakturschriften bezeichnet.
% Zur korrekten Verwendung  der gebrochenen
% Schriften und zu ihrer Geschichte sei
% auf die am Schlu"s auf\/gef"uhrte Literatur verwiesen.
% F"ur die Benutzung mit \TeX{} und \LaTeX{} hat Yannis Haralambous
% derartige Schriften in allen drei Stilen unter Anlehnung an
% historische Vorbilder entworfen, dazu einen Satz an barocken
% Initialen. Eine ausf"uhrliche Beschreibung dieser Fonts findet man
% in \cite{file:oldgerm} und \cite{bk:kopka}.
% 
% Eine Eigenheit der gebrochenen Schriften sind ihre zahlreichen
% Ligaturen und einige Sonderzeichen, die in der Antiqua nicht vorkommen, 
% w"ahrend andere Sonderzeichen oder Akzente fehlen.
% Die Codierung entsprechender Fonts unterscheidet
% sich also zwangsl"aufig von "ublichen \TeX-Schriften.
% Deshalb war die 
% Benutzung gebrochener Schriften in \LaTeX{} fr"uher
% nicht einfach.
% Inzwischen unterst"utzt \LaTeX{} jedoch
% unterschiedliche  Codierungen;
% damit ist es m"oglich, auch die altdeutschen Schriften 
% in das Font-Auswahlschema von \LaTeX{} (NFSS)
% zu integrieren.
%
% \section{Das Makropaket \texttt{yfonts}}
% Das im folgenden beschriebene Makropaket {\tt yfonts} erm"oglicht
% den Zugriff auf Gotisch, Schwabacher, Fraktur und Initialen in \LaTeXe
% und vertr"agt sich mit der deutschen Sprachanpassung
% \texttt{german.sty}. Es wird auf die "ubliche Weise geladen:
% \begin{quote}
% \verb!\usepackage{yfonts}!
% \end{quote}
% und kennt eine Option,
% \begin{quote}
% \verb!\usepackage[varumlaut]{yfonts}!
% \end{quote}
% deren Bedeutung in Abschnitt \ref{sec:umlaute} beschrieben ist.
%
%
% \subsection{Unterst"utzte Fonts}
% Als Voreinstellung werden folgende Fonts unterst"utzt:
% \begin{quote}
% \begin{tabular}{l|l}
% \texttt{ygoth} & Gotisch\\
% \texttt{yswab} & Schwabacher\\
% \texttt{yfrak} & Fraktur\\
% \texttt{ysmfrak} & Fraktur (Variante)\\
% \texttt{yinitas} & Initialen\\
% \end{tabular}
% \end{quote}
% Im CTAN findet man die Quellen f"ur diese Schriften in den 
% Unterverzeichnissen von \texttt{fonts/gothic}.
% \begin{quote}
% \textbf{Achtung:}\\
% Die \MF-Quellen der Schrift \texttt{ygoth} sind fehlerhaft.
% Wenn man \MF{} anweist, diese Fehler bei der Bearbeitung zu ignorieren,
% entsteht normalerweise trotzdem ein brauchbares Ergebnis.
% In sehr wenigen F"allen wurden Probleme auch mit den anderen 
% Schriften beobachtet, 
% offenbar abh"angig vom \textit{mode} und der Verg"o"serung.
% \end{quote}
%
% \subsection{Schriftauswahl}
% \DescribeMacro{\gothfamily}
% \DescribeMacro{\swabfamily}
% \DescribeMacro{\frakfamily}
% \DescribeMacro{\initfamily}
% Das Makropaket  kennt zur Schriftauswahl die neuen Befehle
% \verb!\gothfamily!, \verb!\swabfamily!, 
% \verb!\frakfamily! und \verb!\initfamily!,
% die innerhalb einer Gruppe Gotisch, Schwabacher, Fraktur oder Initialen 
% aktivieren. 
%
% \DescribeMacro{\textgoth}\DescribeMacro{\textswab}
% \DescribeMacro{\textfrak}\DescribeMacro{\textinit}
% Die Schriftartbefehle gibt es auch in einer Variante mit Argument:
% \verb!\textgoth{...}!, \verb!\textswab{...}!, \verb!\textfrak{...}!  und 
% \verb!\textinit{...}!.
% 
% Die so angesprochenen Initialen stehen auf der Grundlinie.
% In den Absatz hineingezogene
% Initialen sind in Abschnitt \ref{sec:initialen} beschrieben.
%
% Befehle zur Gr"o"sensteuerung wirken auch auf die
% altdeutschen Schriften. Die definierten Schriftgrade entsprechen dem
% "ublichen Raster von 10\,pt, 10.95\,pt, 12\,pt, \dots\ bis 24.88\,pt.
%
% Keiner der Fonts ist in mehreren \textit{series} oder 
% \textit{shapes}
% verf"ugbar; ein Befehl wie \verb!\bfseries! hat also keine Wirkung.
% Man kann aber -- das ist durchaus vorbildgerecht -- innerhalb
% eines in Fraktur geschriebenen Textes Hervorhebungen oder 
% "Uberschriften  in Schwabacher setzen.
% 
% F"ur die Frakturschrift wird derjenige Font
% benutzt, der im Makro \verb!\frakdefault! angegeben ist. 
% \DescribeMacro{\frakdefault}
% Die Voreinstellung
% ist {\tt yfrak}. Es gibt auch eine Variante davon, {\tt ysmfrak}, die
% sich durch geringere Strichst"arken und einen etwas gr"o"seren Abstand der 
% einzelnen Zeichen unterscheidet.
% Nach
% \begin{quote}
% \verb!\renewcommand{\frakdefault}{ysmfrak}!\\
% \end{quote}
% wird als Frakturschrift {\tt ysmfrak} benutzt.
%
% \DescribeMacro{\gothdefault}
% \DescribeMacro{\swabdefault}
% \DescribeMacro{\initdefault}
% Analog dazu gibt es auch die Makros \verb!\gothdefault!,
% \verb!\swabdefault! und \verb!\initdefault! f"ur 
% Gotisch ({\tt ygoth}), Schwabacher ({\tt yswab}) 
% und die Initialen ({\tt yinitas}). Ein Abgehen von den 
% Voreinstellungen ist
% selbstverst"andlich nur dann sinnvoll, wenn passende Fonts und fd-Dateien
% verf"ugbar sind!
%
% \subsection{Sonderzeichen}
% \subsubsection{Umlaute}\label{sec:umlaute}
% Die Umlaute sind in den Fonts als Ligaturen definiert: 
%
% \DescribeMacro{"a}\DescribeMacro{*a}
% Ohne das
% Makroapket {\tt german} f"uhrt die Eingabe von \verb!"a! zum 
% "ublichen Umlaut mit P"unktchen, und
% {\tt *a} ergibt eine Variante, bei der statt der Punkte ein winziges e
% auf das Grundsymbol gesetzt wird. 
%
% Das Paket \texttt{yfonts} sorgt daf"ur, da"s der normale
% Akzentbefehl \verb!\"! den Umlaut mit P"unktchen erzeugt; 
% wenn es aber mit der Option {\tt varumlaut}
% geladen wurde, ergibt er die Variante mit dem aufgesetzten e.
%
% L"adt man zus"atzlich das dem Makropaket {\tt german},
% dann hat die Eingabe von z.\,B. \verb!"a!
% stets die gleiche Wirkung wie \verb!\"a!, ist also abh"angig von der Option.
% Die *-Ligaturen sind dabei weiterhin vorhanden.
%
% In der gotischen Schrift ist die Umlaut-Variante \emph{nicht}
% verf"ugbar; die Option hat hier \emph{keine} Wirkung.
%
% \subsubsection{Scharfes s}
% Das scharfe s ist sowohl als Ligatur \texttt{sz}
% wie auch als Ligatur
% \DescribeMacro{sz}\DescribeMacro{"s}
% \verb!"s! in den Fonts vorhanden. 
% Mit \texttt{yfonts.sty} wird auch 
% der \TeX-Befehl \verb!\ss! erkannt.
% Benutzt man zus"atzlich das \texttt{german}-Paket, wird \verb!"s! ebenfalls
% richtig interpretiert.
%
% \subsubsection{Langes und rundes s}
% Das in den gebrochenen Schriften f"ur eine korrekte Typographie 
% ben"otigte sogenannte Schlu"s-s wird, genauso wie auch im Makropaket
% {\tt oldgerm}, als Ligatur  
% \DescribeMacro{s:}
% {\tt s:} eingegeben.
%
% \subsubsection{Anf"uhrungszeichen}
% In den Schriften Schwabacher und Fraktur sind die doppelten deutschen
% Anf"uhrungszeichen auch als Ligatur der entsprechenden amerikanischen
% \textit{Quotes} vorhanden. Die Eingabe von \verb!``Zitat''! ist also
% gleichwertig mit der Ersatzdarstellung entsprechend \texttt{german.sty}.
%
% Die Anf"uhrungszeichen der Frakturschriften \texttt{yfrak} und \texttt{ysmfrak}
% unterscheiden sich nicht von der Computer-Modern Antiqua -- das 
% ist \emph{kein} Fehler, sondern es
% entspricht dem Stil der Breit"-kopf"=Fraktur, die als
% Vorbild diente. Sp"atere Schriften hatten Anf"uhrungszeichen,
% die ungef"ahr dem Fraktur-Komma "ahnelten.
% 
% Leider fehlt in allen Schriften das "`einfache "offnende"' Anf"uhrungszeichen,
% das mit dem Paket \texttt{german} als \verb!\glq! eingegeben wird.  
% Stattdessen wird das entsprechende Zeichen aus CM Roman benutzt, was
% zumindest bei der Frakturschrift nat"urlich nicht auf"|f"allt.
%
% \subsubsection{Sonstige Sonderzeichen}
% Sonstige Sonderzeichen und Akzente werden mit den "ublichen
% \LaTeX"=Befehlen oder Befehlen aus \texttt{german.sty}
% eingegeben. Falls sie in den altdeutschen Fonts
% nicht vorhanden sind, wird automatisch ein Ersatz aus einer anderen
% Schrift genommen (Ausnahmen siehe Abschnitt \ref{sec:err}) 
% und eine entsprechende Warnung angezeigt.
%
% Ein zus"atzliches Sonderzeichen
% ist \verb!\etc!, womit ein fr"uher
% \DescribeMacro{\etc}
% weithin "ubliches Kurzzeichen fur "`usw."' erzeugt wird. Dieses Zeichen
% ist nur in der Frakturschrift enthalten.  
%
% \subsection{Zeilenabstand}
% Die gebrochenen Schriften von Y.~Haralambous d"urfen 
% bei gleicher Nenngr"o"se viel enger gesetzt werde als Antiqua. 
% So pa"st zu einer Fraktur oder Schwabacher mit einer 
% Nenngr"o"se von 12\,pt ein Zeilenabstand von 12\,dd.
% Das ist ein "`klassischer"' Wert, den man in vielen alten B"uchern findet.
% \DescribeMacro{\fraklines}
% Das Paket definiert  den Befehl \verb!\fraklines!,
% der den Zeilenschritt \verb!\baselineskip! 
% in genau diesem Verh"altnis zur laufenden Schriftgr"o"se einstellt.
% Die Wirkung des Befehls beschr"ankt sich auf die laufende Gruppe.
% Bei einem Wechsel der Schriftgr"o"se mu"s der Befehl \verb!fraklines!
% wiederholt werden!
%
% F"ur Gotisch ist der geeignete Zeilenabstand im Vergleich zur
% Nenngr"o"se der Schrift noch erheblich geringer; eine entsprechende
% Unterst"utzung ist hier aber nicht vorgesehen.
% \texttt{ygoth} ist
% nach Vorbildern des 15.~Jh. gestaltet, so da"s f"ur eine vorbildgerechte
% Typographie sowieso zahlreiche weitere Anpassungen n"otig sind.
% Das vorliegende Paket stellt nur das NFSS-Interface zur
% Verf"ugung.
%
% \subsection{Verwendung der Initialen} \label{sec:initialen}
% In den Absatz hineingezogene Initialen kann man mit dem
% Befehl \verb!\yinipar! erzeugen. Er lehnt sich an einen Vorschlag von Kopka 
% \cite{bk:kopka} an. \DescribeMacro{\yinipar} 
% Ein Absatz mit Initiale sieht dann folgenderma"sen aus:
% \begin{quote}
% \begin{verbatim}
% {\frakfamily\fraklines
% \yinipar{D}ie Orgel, der Fl"ugel, das: Fortepiano und das:
% Clavicord sind die gebr"auchlichsten Instrumente zum ...
% \par}
% \end{verbatim}
% \end{quote}
% Diese Initialen stehen auf der Grundlinie der vierten Zeile des Absatzes,
% und sie schlie"sen genau
% mit der Oberkante der ersten Zeile ab, wenn man den Zeilenabstand 
% wie oben beschrieben mit \verb!\fraklines! einstellt. 
%
% \verb!\yinipar! beginnt einen neuen 
% Absatz (wie \verb!\par!)
% und hebt einen evtl. Absatzeinzug auf (wie \verb!\noindent!).
% Wichtig ist, da"s das Ende des Absatzes (\verb!\par! oder Leerzeile)
% innerhalb der Gruppe steht, wo \verb!\fraklines!
% noch g"ultig ist!
%
% Die Initialen kennen selbstverst"andlich nur Gro"sbuchstaben und keine
% Sonderzeichen. Auch Umlaute sind in diesem Zeichensatz nicht
% enthalten; es ist "ublich, stattdessen den Vokal
% als Initiale und nachfolgend ein "`e"' zu setzen.
%
% Der Befehl \verb!\yinipar! setzt voraus, da"s als Font f"ur die
% Initialen tats"achlich \texttt{yinitas} gew"ahlt ist, d.\,h., da"s 
% die Voreinstellung f"ur \verb!\initfamily! \emph{nicht} ver"andert wurde.
% (Eine andere Schrift mit gleicher Gr"o"se k"onnte jedoch verwendet werden.)
%
% Der Befehl \verb!\yinitpar! \DescribeMacro{\yinitpar}, 
% der vor Version~1.2 in diesem Paket vorhanden war,
% existiert weiterhin.  Er hat die gleiche Wirkung
% wie \verb!\yinipar!, beginnt aber nicht automatisch einen neuen
% Absatz und hebt einen vorhandenen \verb!\parindent! nicht selbst auf.
%
% \subsection{Bekannte Fehler und M"angel}\label{sec:err}
% \begin{itemize}
% \item 
% Das Symbol \verb!\etc! fehlt in der
% Schwabacher, obwohl es in der zugeh"origen Codierung LY definiert ist.
% Beim Versuch, dieses Zeichen anzusprechen, erh"alt man keinen Fehler, 
% sondern nur eine Warnung in der Protokolldatei.
% \item
% In allen Fonts fehlen einige der folgenden ASCII-Symbole;
% ein automatischer  Ersatz durch Zeichen aus anderen Schriften 
% ist hier nicht m"oglich:
% \verb!  ; ` [ ] / * @ & % !
%
% F"ur die "ublichen Anwendungen altdeutscher
% Schrift sollte das kein Problem darstellen.
% \item 
% Bei der Verwendung der deutschen Silben"-trenn"-muster
% zusammen mit gebrochenen Schriften werden schon allein
% wegen der zwei unterschiedlichen Varianten des Buchstabens "`s"'
% einige Trennstellen  nicht erkannt.
% Normalerweise bekommt man trotzdem 
% ein brauchbares Ergebnis, solange
% Wortwahl und Rechtschreibung nicht zu sehr vom heute "Ublichen
% abweichen.
% \end{itemize}
%
% \StopEventually{
% \begin{thebibliography}{9}
% \bibitem{file:oldgerm} Haralambous, Y.: Typesetting old german.
% \newblock CTAN: \texttt{macros/latex/packages/mfnfss/oldgerm.dtx}
% \bibitem{bk:kopka} Kopka, H.: LaTeX -- Erg"anzungen.
% \newblock Addison-Wesley Deutschland, Bonn 1995,
% \newblock Kapitel 2.2.3.
% \bibitem{bk:weichert} Weichert, J.: Druckschriften. 
% \newblock Bruckmann, M"unchen 1991.
% \bibitem{bk:duden} DUDEN -- Rechtschreibung der deutschen Sprache und der
% Fremd\-w"orter.
% \newblock Bibliographisches Institut, Mannheim 1986,
% \newblock Kapitel "`Richtlinien f"ur den Schriftsatz"'.
% \end{thebibliography}}
%
% \section{Implementierung}
% Das Makropaket setzt mindestens \LaTeXe\ in
% der Version 1994/12/01 voraus,
% denn es macht Gebrauch von der vollst"andigen Unterst"utzung 
% unterschiedlicher Font-Codierungen:
%    \begin{macrocode}
%<*package>
\NeedsTeXFormat{LaTeX2e}[1994/12/01]
%    \end{macrocode}
%
% Die Option {\tt varumlaut} steuert, ob die Umlaute mit P"unktchen
% oder in der Variante mit einem aufgesetzten e erscheinen:
%    \begin{macrocode}
\newif\ifyf@v
\DeclareOption{varumlaut}{\yf@vtrue}
\ProcessOptions\relax
%    \end{macrocode}
%
% \subsection{Das Interface zu den Schriften}
% Wir definieren lokale Codierungen f"ur Fraktur und Schwabacher \dots
%    \begin{macrocode}
\DeclareFontEncoding{LY}{}{}
\DeclareFontSubstitution{LY}{yfrak}{m}{n}
%    \end{macrocode}
% \dots{} und Gotisch:
%    \begin{macrocode}
\DeclareFontEncoding{LYG}{}{}
\DeclareFontSubstitution{LYG}{ygoth}{m}{n}
%    \end{macrocode}
%
% Die Schriftfamilie "`Gotisch"':
%    \begin{macrocode}
\DeclareFontFamily{LYG}{ygoth}{}
\DeclareFontShape{LYG}{ygoth}{m}{n}{%
<10><10.95><12><14.4><17.28><20.74><24.88>ygoth}{}
%    \end{macrocode}
%
% Die Schriftfamilie "`Fraktur"':
%    \begin{macrocode}
\DeclareFontFamily{LY}{yfrak}{}
\DeclareFontShape{LY}{yfrak}{m}{n}{%
<10><10.95><12><14.4><17.28><20.74><24.88>yfrak}{}
%
\DeclareFontFamily{LY}{ysmfrak}{}
\DeclareFontShape{LY}{ysmfrak}{m}{n}{%
 <10><10.95><12><14.4><17.28><20.74><24.88>ysmfrak}{}
%    \end{macrocode}
%
% Die Schriftfamilie "`Schwabacher"':
%    \begin{macrocode}
\DeclareFontFamily{LY}{yswab}{}
\DeclareFontShape{LY}{yswab}{m}{n}{%
<10><10.95><12><14.4><17.28><20.74><24.88>yswab}{}
%    \end{macrocode}

% Wir benutzen die von Andreas Schrell korrigierten Initialen
% \texttt{yinitas}, denn die urspr"ungliche Version \texttt{yinit}
% wies zahlreiche Ungereimtheiten auf.
%
% Die Codierung ist \texttt{U} (undefiniert), denn die Initialen
% gibt es nur als Gro"sbuchstaben, und es werden keine
% spezifischen Befehle f"ur diesen Zeichensatz definiert.
% 
% Die Initialen werden in den normalen Ver"-gr"o"serungs"-stufen
% benutzt, also 1, 1.095, 1.2 usw.
% Mit ihrer Entwurfsgr"o"se von $(1392/36)\,\mathrm{pt}$
% ergeben sich daraus die unten eingesetzten effektiven Gr"o"sen.
% Bei einem klassischen Zeilenabstand, wie ihn \verb!\fraklines! einstellt,
% sind diese Initialen dann wunderbarerweise exakt so gro"s, da"s
% sie von der Oberkante der ersten bis zur Unterkante der vierten
% Zeile des Absatzes reichen!
%    \begin{macrocode}
\DeclareFontFamily{U}{yinitas}{}
\DeclareFontShape{U}{yinitas}{m}{n}{%
<10>sfixed*yinitas%
<10.95>sfixed*[42.34]yinitas%
<12>sfixed*[46.39]yinitas%
<14.4>sfixed*[55.68]yinitas%
<17.28>sfixed*[66.82]yinitas%
<20.74>sfixed*[80.19]yinitas%
<24.88>sfixed*[96.20]yinitas%
}{}
%    \end{macrocode}
%
% \subsection{Die Makros}
% Es werden alle f"ur diese Codierungen spezifischen Befehle 
% definiert; zun"achst f"ur die gotische Schrift:
%    \begin{macrocode}
\DeclareTextSymbol{\textemdash}{LYG}{124}
\DeclareTextSymbol{\textendash}{LYG}{123}
\DeclareTextComposite{\"}{LYG}{a}{91}
\DeclareTextComposite{\"}{LYG}{o}{93}
\DeclareTextComposite{\"}{LYG}{u}{94}
\DeclareTextComposite{\"}{LYG}{e}{92}
\DeclareTextSymbol{\ss}{LYG}{25}
\DeclareTextCommand{\SS}{LYG}{SS}
\DeclareTextSymbol{\i}{LYG}{16}
\DeclareTextSymbol{\j}{LYG}{17}
\DeclareTextSymbol{\textquotedblleft}{LYG}{95}
\DeclareTextCommand{\grqq}{LYG}{\textquotedblleft}
\DeclareTextSymbol{\textquotedblright}{LYG}{34}
%    \end{macrocode}
%
% Die Makros f"ur Fraktur und Schwabacher:
%    \begin{macrocode}
\DeclareTextCommand{\"}{LY}[1]{\UseTextAccent{OT1}{\"}{#1}}
\DeclareTextAccent{\'}{LY}{19}
\DeclareTextAccent{\.}{LY}{95}
\DeclareTextAccent{\=}{LY}{22}
\DeclareTextAccent{\^}{LY}{94}
\DeclareTextAccent{\`}{LY}{18}
\DeclareTextAccent{\~}{LY}{126}
\DeclareTextAccent{\H}{LY}{125}
\DeclareTextAccent{\u}{LY}{21}
\DeclareTextAccent{\v}{LY}{20}
\DeclareTextAccent{\r}{LY}{23}
%
\DeclareTextSymbol{\textemdash}{LY}{124}
\DeclareTextSymbol{\textendash}{LY}{123}
%
\ifyf@v
\DeclareTextComposite{\"}{LY}{a}{137}
\DeclareTextComposite{\"}{LY}{o}{153}
\DeclareTextComposite{\"}{LY}{u}{158}
\DeclareTextComposite{\"}{LY}{e}{144}
\else
\DeclareTextComposite{\"}{LY}{a}{138}
\DeclareTextComposite{\"}{LY}{o}{154}
\DeclareTextComposite{\"}{LY}{u}{159}
\DeclareTextComposite{\"}{LY}{e}{145}
\fi
%
\DeclareTextSymbol{\textsection}{LY}{60}
\DeclareTextSymbol{\i}{LY}{16}
\DeclareTextSymbol{\j}{LY}{17}
\DeclareTextSymbol{\ss}{LY}{26}
\DeclareTextSymbol{\etc}{LY}{201}
%
\DeclareTextSymbol{\textquotedblleft}{LY}{34}
\DeclareTextCommand{\grqq}{LY}{\textquotedblleft}
%\DeclareTextSymbol{\quotesinglbase}{LY}{13}% missing!
\DeclareTextSymbol{\quotedblbase}{LY}{92}
\DeclareTextSymbol{\textquoteleft}{LY}{`\`}
\DeclareTextCommand{\grq}{LY}{\textquoteleft}
\DeclareTextSymbol{\textquoteright}{LY}{`\'}
%
\DeclareTextCommand{\SS}{LY}
   {SS}
%
%    \end{macrocode}
%
% Die folgenden Makros verweisen auf die Fonts,
% die als Voreinstellung benutzt werden:
%    \begin{macrocode}
\def\gothdefault{ygoth}
\def\swabdefault{yswab}
\def\frakdefault{yfrak}
\def\initdefault{yinitas}
%    \end{macrocode}
%
% Die Kommandos zur Schriftumschaltung werden unter Verwendung
% des Befehls \verb!\usefont! definiert, 
% weil in diesen Familien keine weiteren Schriftschnitte 
% (\textit{series}, \textit{shape}) vorhanden sind:
%    \begin{macrocode}
\def\gothfamily{\usefont{LYG}{\gothdefault}{m}{n}}
\def\swabfamily{\usefont{LY}{\swabdefault}{m}{n}}
\def\frakfamily{\usefont{LY}{\frakdefault}{m}{n}}
\def\initfamily{\usefont{U}{\initdefault}{m}{n}}
%
\DeclareTextFontCommand{\textgoth}{\gothfamily}
\DeclareTextFontCommand{\textswab}{\swabfamily}
\DeclareTextFontCommand{\textfrak}{\frakfamily}
\DeclareTextFontCommand{\textinit}{\initfamily}
%    \end{macrocode}
%
% Ein Makro zur Benutzung der Initialen:
% \verb!\yinipar! beginnt einen neuen Absatz, 
% unterdr"uckt einen evtl. definierten Einzug
% und ruft seinerseits \verb!\yinitpar! auf.
%    \begin{macrocode}
\newcommand{\yinipar}[1]{\par\noindent\yinitpar{#1}}
%    \end{macrocode}
% \verb!\yinitpar! setzt -- nach einem Vorschlag von 
% Kopka~\cite{bk:kopka} -- die Initiale auf die 
% Grundlinie der vierten Zeile: 
%    \begin{macrocode}
\newcommand{\yinitpar}[1]{\setbox0=\hbox{\textinit{#1}}%
\hangindent=\wd0\hangafter=-4\advance\hangindent by .25em
{\dimen@=-3\baselineskip
\dimen@=\baselinestretch\dimen@
\hskip-\wd0 \hskip-.25em
\raisebox{\dimen@}[0pt][0pt]{\unhbox0}\hskip.25em}}
%    \end{macrocode}
%
% Der Befehl \verb!\fraklines! stellt den Zeilenabstand so ein,
% da"s sein Wert in dd dem aktuellen Schriftgrad in pt entspricht:
%    \begin{macrocode}
\def\fraklines{\baselineskip\f@size dd\relax}
%</package>
%    \end{macrocode}
%
% \section*{Die vorliegende Beschreibung \ldots}
% \ldots{} wird erzeugt, indem man die Datei \texttt{yfonts.dtx}
% mit \LaTeX{} bearbeitet. Diese Datei enth"alt folgende
% \textsc{Docstrip}-Module:
%
% \begin{center}
% \begin{tabular}{ll}
% Modul: & Inhalt:\\[0.5ex]
% \textsf{package} & Makropaket \textsf{yfonts} \\
% \textsf{driver} &  Treiber f"ur Beschreibung \\
% \end{tabular}
% \end{center}\vspace{1ex}
% 
% Der n"achste Befehl verhindert, da"s die abschlie"sende
% \textit{Character Table} in die \textsc{Docstrip}-Module "ubernommen wird:
%    \begin{macrocode}
\endinput
%    \end{macrocode}
% \Finale 
%% \CharacterTable
%%  {Upper-case    \A\B\C\D\E\F\G\H\I\J\K\L\M\N\O\P\Q\R\S\T\U\V\W\X\Y\Z
%%   Lower-case    \a\b\c\d\e\f\g\h\i\j\k\l\m\n\o\p\q\r\s\t\u\v\w\x\y\z
%%   Digits        \0\1\2\3\4\5\6\7\8\9
%%   Exclamation   \!     Double quote  \"     Hash (number) \#
%%   Dollar        \$     Percent       \%     Ampersand     \&
%%   Acute accent  \'     Left paren    \(     Right paren   \)
%%   Asterisk      \*     Plus          \+     Comma         \,
%%   Minus         \-     Point         \.     Solidus       \/
%%   Colon         \:     Semicolon     \;     Less than     \<
%%   Equals        \=     Greater than  \>     Question mark \?
%%   Commercial at \@     Left bracket  \[     Backslash     \\
%%   Right bracket \]     Circumflex    \^     Underscore    \_
%%   Grave accent  \`     Left brace    \{     Vertical bar  \|
%%   Right brace   \}     Tilde         \~}
%%

