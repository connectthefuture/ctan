\documentclass{ltxguide}
\usepackage{txfonts,url}
\usepackage[shell,debug]%
  {pdftricks}

\begin{psinputs}
\usepackage{txfonts}
\usepackage{pstricks}
\usepackage{color}
\usepackage{pstcol}
\usepackage{pst-plot}
\usepackage{pst-tree}
\usepackage{pst-eps}
\usepackage{multido}
\usepackage{pst-node}
\usepackage{pst-eps}
\end{psinputs}

% 1. run pdfLaTeX
% 2. run script `pst2pdf' by invoking ./pst2pdf
% 3. second run of pdfLaTeX

\begin{document}

%\NoProcess[1-6]


\title{Trivial Experiments with psTricks manipulation}

\author{\bfseries Radhakrishnan CV and Rajagopal
           CV\\[-3pt]
       \small River Valley Technologies,
                       Trivandrum, India\\[-3pt]
       \small\url{http://www.river-valley.com}\\[6pt]
       \bfseries Antoine Chambert-Loir\\[-3pt]
       \small Ecole polytechnique, Palaiseau Cedex,
                       France\\[-3pt]
       \small \url{http://www.math.polytechnique.fr/~chambert}
}

\maketitle

\section{Objectives}

|psTricks| macros cannot be used with pdf\TeX, since the high level
PostScript language commands generated by the |psTricks| are unknown
to pdf\TeX. As such, a package \emph{viz.}, |pdfTricks.sty| has been
written to circumvent this limitation, so that the extensive
facilities offered by the powerful |psTricks| package can be made use
of in a pdf\TeX{} document. This is brought by making use of the shell
escape function available in the |web2c| \TeX{} compiler, while this
package is of no use to other commercial implementations.

Shell escape provides us the facility of stopping a \TeX{} run midway,
perform the functions that we do with a shell command, complete those
functions, return to the \TeX{} run and finish the job. Within a
package, this facility can be invoked by using 
|\write18{shell command}|. So if you use a line:
\begin{decl}
|write18 ls -l|
\end{decl}
\TeX{} will stop at the moment when it encounter the |\write18|
command, escapes to shell, executes |ls -l| command and return to the
compilation process.

\section{Shell escape}

In all the web2c \TeX{} implementations, shell escape is turned off by
default. You can either turn it on by changing the line in your
|texmf.cnf| (configuration file of your \TeX{} system) to:
\begin{decl}
|shell_escape = t|
\end{decl}
which will be `|f|' by default. You have to recreate all the formats
like |pdflatex.fmt| or whatever you might need. This is not a wise
step though, since it makes you open to the assault of Trojan macros
without your knowledge.

The more elegant way will be to invoke shell escape with a switch,
each time you run the compiler like:
\begin{decl}
|pdflatex -shell-escape <file name>|
\end{decl}
This eliminates the vulnerability to macros/packages written by
intelligent criminal minds.

\section{Usage}

The usage is very simple, as usual you have to load the package with a
package loading line in the preamble of your document:

\begin{decl}
|\usepackage{pdftricks}|
\end{decl}

All the packages that are needed for the compilation of the |pstricks|
code shall be enclosed within an environment |\begin{psinputs}| \dots
  |\end{psinputs}| in order to make it available to all the figure
documents during compilation. An example of this will look like
\begin{decl}
|\begin{psinputs}|\\
| \usepackage{pstricks}|\\
| \usepackage{color}   |\\
| \usepackage{pstcol}  |\\
| \usepackage{pst-plot}|\\
| \usepackage{pst-tree}|\\
| \usepackage{pst-eps} |\\
| \usepackage{multido} |\\
| \usepackage{pst-node}|\\
| \usepackage{pst-eps} |\\
| ... |\\
|\end{psinputs}|
\end{decl}

Each |pspicture| environment along with any macro code shall be
enclosed within an environment \emph{viz.}, |pdfpic|. This will enable
pdf\LaTeX{} to write external files like |fig1.tex|, |fig2.tex|,
\dots, depending on the number of |psTricks| figures you have included
in your document. |pdftricks| will intelligently find out whether your
pdf\TeX{} implementation has shell escape facility enabled or not.
Depending on this, there are two ways of approaching the problem which
are given in the succeeding sections.

\subsection{With shell escape}

With shell escape option invoked, one need not bother anything.  The
usual compilation with pdf\LaTeX{} will result in a |pdf| document
with all the figures appearing in the correct positions. pdf\TeX{}
will do the necessary job of writing all the |pstricks| code to self
standing external |*.tex| documents, escapes to shell, compiles the
|*.tex| document, converts to |*.eps|, translates to |*.pdf| and
includes in the location where the |pstricks| code appeared.

\subsection{Without shell escape}

A special shell script viz., |pst2pdf| is also provided. Now as the
second step, you have to execute this shell script, |pst2pdf| which
will compile all the newly written |fig1.tex|,\dots, and respective
|*.dvi| will be generated, which will be translated into respective
|*.eps| files and finally translated to |*.pdf| by calling |epstopdf|.
As a final step, you shall run pdf\LaTeX{} again, so that the |*.pdf|
figures will be included with the usual |\includegraphics| command
automatically in the locations wherever the |pstricks| code appeared.


See an example below:
\begin{verbatim}
\begin{pdfdisplay}
\definecolor{lightblue}{rgb}{0,0,.5}
\definecolor{Navy}{rgb}{0,0,0.5}
\definecolor{LemonChiffon}{rgb}{1,0.98,0.8}
\definecolor{ForestGreen}{rgb}{0.13,0.55,0.13}
\SpecialCoor
\begin{pspicture}(-6,-6)(6,6)
  \psaxes{<<->>}(0,0)(-6,-6)(6,6)
  \psset{arrows=->>}
  \multido{\ia=-5+1}{11}{%
\multido{\ib=-5+1}{11}{%
 \pstVerb{/x \ia\space def
    /y \ib\space def      
    y 0 eq                
      {/ValueTempA 0 def  
       /ValueTempB 0.5 def}
      {/ValueTempZ 2 1 x x mul y y mul div add sqrt mul def
       /ValueTempA 1 ValueTempZ div def        
       /ValueTempB x y ValueTempZ mul div def} 
      ifelse}                                  
 \psline(! x ValueTempA sub y ValueTempB sub)  
        (! x ValueTempA add y ValueTempB add)}}
\end{pspicture}
\end{pdfdisplay}
\end{verbatim}

This |psTricks| code will result in the following graphic:
\configure[pdfgraphic][color=red,rulesep=12pt,
                       linecolor=gray30,
                      linewidth=2pt]
\begin{pdfdisplay} 
\definecolor{lightblue}{rgb}{0,0,.5}
\definecolor{Navy}{rgb}{0,0,0.5}
\definecolor{LemonChiffon}{rgb}{1,0.98,0.8}
\definecolor{ForestGreen}{rgb}{0.13,0.55,0.13}
\def\transl#1#2{#1\space 72 2.54 div mul #2\space 72 2.54 div
     mul translate  0 0 moveto }
\newif\ifnotbothz
 \pspicture(-6,-6)(6,6)
 \psaxes{<->}(0,0)(-6,-6)(6,6)
 \multido{\n=-5+1}{11}{%
 \multido{\N=-5+1}{11}{%
 \notbothztrue%
 \ifnum\N=0 %
 \ifnum\n=0 \notbothzfalse\fi%
 \fi%
 \ifnotbothz%
 \psdot(0,0)%
 \pstverb{\transl{\n}{\N}\space
 \n\space neg \N\space
    atan rotate -9 0 moveto 18 0 rlineto stroke }
\fi}}
\endpspicture                      
\end{pdfdisplay}

The following are the steps in the sequential order:
\begin{decl}
1.~|latex <your document>|\\
2.~|pst2pdf|\\
3.~|latex <your document>|
\end{decl}

\section{Options and Hooks}

\begin{description}
  
\item[|shell|] This option will invoke the shell escape functions.

\item[|noshell|] This option will suppress the shell escape functions
  and will assume that there is no shell escape facility.

\item[|NoProcess|] A new command |\NoProcess| has been introduced in
  this version to facilitate the suppressing of pdf generation of
  those figures whose pdf's are already available. This might prove
  helpful when you have more figures to process and many of them are
  perfected and don't need recompilation and translation everytime you
  run pdf\LaTeX. The usage is:
\begin{decl}
|\NoProcess{<comma separated and/or hyphen separated ranges>}|
\end{decl}
If you have ten figures and if you want to suppress the processing of
the figure numbers 1, 2, 4 to 8 you can issue the command at the top
of the document as:
\begin{decl}
|\NoProcess{1,2,4-8}|
\end{decl}
\end{description}

\noindent There are few hooks to resize the graphic thus translated and
included.
\begin{decl}
|\configure[pdfgraphic][width=2in]|\\
|\configure[pdfgraphic][height=3in]|\\
|\configure[pdfgraphic][width=2in,height=3in]|
\end{decl}
the functionality is same as the |width| and |height| options as in
the |\includegraphics| command. But the graphic will be restricted to
aspect ratio.

\section{Limitations/Dependencies}

\begin{itemize}
\item Usage is limited to |web2c| \TeX{} implementations.
\item The shell escape commands used in this package are typical
  \textsc{Unix} commands and as such it best works in
  \textsc{Linux/Unix} flavours.
\item The package needs, |graphicx| and |keyval| packages; |epstopdf|,
  the Perl script that translates |eps| files into |pdf|.
\end{itemize}

\section{Licence}
The package is distributed under \textsc{gnu} General Public Licence.

\section{Examples}

Here are some examples of pstricks code and the figures generated by
pdf\LaTeX.
\subsection{Example 1}
\begin{verbatim}
\configure[pdfgraphic][width=2in,linecolor=red,background=gray10]
\begin{pdfdisplay}
\definecolor{lightblue}{rgb}{0,0,.5}
\definecolor{Navy}{rgb}{0,0,0.5}
\definecolor{LemonChiffon}{rgb}{1,0.98,0.8}
\definecolor{ForestGreen}{rgb}{0.13,0.55,0.13}
\SpecialCoor
\begin{pspicture}(-6,-6)(6,6)
  \psaxes{<<->>}(0,0)(-6,-6)(6,6)
  \psset{arrows=->>}
  \multido{\ia=-5+1}{11}{%
\multido{\ib=-5+1}{11}{%
 \pstVerb{/x \ia\space def
    /y \ib\space def
    y 0 eq
      {/ValueTempA 0 def
       /ValueTempB 0.5 def}
      {/ValueTempZ 2 1 x x mul y y mul div add sqrt mul def
       /ValueTempA 1 ValueTempZ div def
       /ValueTempB x y ValueTempZ mul div def}
      ifelse}
 \psline(! x ValueTempA sub y ValueTempB sub)
        (! x ValueTempA add y ValueTempB add)}}
\end{pspicture}
\end{pdfdisplay}
\end{verbatim}
\configure[pdfgraphic][width=.9\linewidth,linecolor=red,
                       rulesep=12pt,background=gray10,
                      linewidth=2pt]
\begin{pdfdisplay}
\definecolor{lightblue}{rgb}{0,0,.5}
\definecolor{Navy}{rgb}{0,0,0.5}
\definecolor{LemonChiffon}{rgb}{1,0.98,0.8}
\definecolor{ForestGreen}{rgb}{0.13,0.55,0.13}
\SpecialCoor
\begin{pspicture}(-6,-6)(6,6)
  \psaxes{<->}(0,0)(-6,-6)(6,6)
  \psset{arrows=->}
  \multido{\ia=-5+1}{11}{%
\multido{\ib=-5+1}{11}{%
 \pstVerb{/x \ia\space def
    /y \ib\space def
    y 0 eq
      {/ValueTempA 0 def
       /ValueTempB 0.5 def}
      {/ValueTempZ 2 1 x x mul y y mul div add sqrt mul def
       /ValueTempA 1 ValueTempZ div def
       /ValueTempB x y ValueTempZ mul div def}
      ifelse}
 \psline(! x ValueTempA sub y ValueTempB sub)
        (! x ValueTempA add y ValueTempB add)}}
\end{pspicture}
\end{pdfdisplay}

\subsection{Example 2}
\begin{verbatim}
\configure[pdfgraphic][linecolor=blue,background=gray10,width=4in]
\begin{pdfdisplay}
\definecolor{lightblue}{rgb}{0,0,.5}
\definecolor{Navy}{rgb}{0,0,0.5}
\definecolor{LemonChiffon}{rgb}{1,0.98,0.8}
\definecolor{ForestGreen}{rgb}{0.13,0.55,0.13}
\newcommand{\MyNode}[2]{%
  \Tr{\psshadowbox[fillstyle=solid,fillcolor=#1]{\tiny #2}}}
\newcommand{\NoeudXt}[1]{\MyNode{ForestGreen}{#1}}
\newcommand{\NoeudMotif}[1]{\MyNode{Navy}{\textcolor{white}{#1}}}
\psset{armB=5mm,angleA=90,angleB=-90,levelsep=2cm,treesep=5mm}
\renewcommand{\psedge}[2]{\ncangle{#2}{#1}}
\TeXtoEPS
\begin{pspicture}(-8cm,-9.5cm)(8cm,1cm)
\rput(0,0){\LARGE\textcolor{red}{Set of Motif widgets classes}}
\rput(0,-4.8){%
  \psframebox[fillstyle=solid,fillcolor=LemonChiffon,linearc=5mm,
             cornersize=absolute]
    {\pstree{\NoeudXt{Core}}
    {\pstree{\NoeudMotif{Primitive}}
            {\pstree{\NoeudMotif{Label}}
                    {\TC*}
             \NoeudMotif{Scrollbar}
             \NoeudMotif{List}
             \NoeudMotif{Text}
             \NoeudMotif{ArrowButton}}
     \pstree{\NoeudXt{Composite}}
            {\pstree{\NoeudXt{Constraint}}
                    {\pstree{\NoeudMotif{Manager}}
                            {\TC*}}}
     \pstree{\NoeudXt{Shell}}
            {\pstree{\NoeudXt{OverrideShell}}
                    {\NoeudMotif{MenuShell}}
             \pstree{\NoeudXt{WMShell}}
                    {\pstree{\NoeudXt{VendorShell}}
                            {\TC*}}}}}}
\rput(-2,-10){%
  \psshadowbox[fillstyle=solid,fillcolor=ForestGreen]{Core} Xt Class}
\rput(2,-10){%
  \psshadowbox[fillstyle=solid,fillcolor=Navy]{%
     \textcolor{white}{List}} Motif Class}
\end{pspicture}
\endTeXtoEPS
\end{pdfdisplay}
\end{verbatim}
\configure[pdfgraphic][linecolor=blue,
                       linewidth=2pt, rulesep=6pt,
                       background=gray10,scale=.8]
\begin{pdfdisplay}
\definecolor{lightblue}{rgb}{0,0,.5}
\definecolor{Navy}{rgb}{0,0,0.5}
\definecolor{LemonChiffon}{rgb}{1,0.98,0.8}
\definecolor{ForestGreen}{rgb}{0.13,0.55,0.13}
\newcommand{\MyNode}[2]{%
  \Tr{\psshadowbox[fillstyle=solid,fillcolor=#1]{\tiny #2}}}
\newcommand{\NoeudXt}[1]{\MyNode{ForestGreen}{#1}}
\newcommand{\NoeudMotif}[1]{\MyNode{Navy}{\textcolor{white}{#1}}}
\psset{armB=5mm,angleA=90,angleB=-90,levelsep=2cm,treesep=5mm}
\renewcommand{\psedge}[2]{\ncangle{#2}{#1}}
\TeXtoEPS
\begin{pspicture}(-8cm,-9.5cm)(8cm,1cm)
\rput(0,0){\LARGE\textcolor{red}{Set of Motif widgets classes}}
\rput(0,-4.8){%
  \psframebox[fillstyle=solid,fillcolor=LemonChiffon,linearc=5mm,
             cornersize=absolute]
    {\pstree{\NoeudXt{Core}}
    {\pstree{\NoeudMotif{Primitive}}
            {\pstree{\NoeudMotif{Label}}
                    {\TC*}
             \NoeudMotif{Scrollbar}
             \NoeudMotif{List}
             \NoeudMotif{Text}
             \NoeudMotif{ArrowButton}}
     \pstree{\NoeudXt{Composite}}
            {\pstree{\NoeudXt{Constraint}}
                    {\pstree{\NoeudMotif{Manager}}
                            {\TC*}}}
     \pstree{\NoeudXt{Shell}}
            {\pstree{\NoeudXt{OverrideShell}}
                    {\NoeudMotif{MenuShell}}
             \pstree{\NoeudXt{WMShell}}
                    {\pstree{\NoeudXt{VendorShell}}
                            {\TC*}}}}}}
\rput(-2,-10){%
  \psshadowbox[fillstyle=solid,fillcolor=ForestGreen]{Core} Xt Class}
\rput(2,-10){%
  \psshadowbox[fillstyle=solid,fillcolor=Navy]{%
     \textcolor{white}{List}} Motif Class}
\end{pspicture}
\endTeXtoEPS
\end{pdfdisplay}

\subsection{Example 3}
\begin{verbatim}
\configure[pdfgraphic][linecolor=black,background=gray30]
\begin{pdfdisplay}
\TeXtoEPS
\definecolor{lightblue}{rgb}{0,0,.5}
\definecolor{Navy}{rgb}{0,0,0.5}
\definecolor{LemonChiffon}{rgb}{1,0.98,0.8}
\definecolor{ForestGreen}{rgb}{0.13,0.55,0.13}
\psset{unit=.825cm}
\begin{pspicture}(10,8)
  \psset{fillstyle=solid,linestyle=none,linewidth=0}
  \psframe[fillcolor=lightblue](10,8)
  \pscircle[fillcolor=yellow](2,6){.8} % Sun
  {% Rays
   \psset{linecolor=yellow,linestyle=solid,linewidth=.3}
   \degrees[8]
   \multido{\i=1+1}{8}{\rput{\i}(2,6){\psline(1,0)(1.5,0)}}
  }%
  \pspolygon[fillcolor=green](6,0)(10,2)(10,0)% Grass
  \psdiamond[fillcolor=red,gangle=-45](8,6)(1.5,2.5)% Kite
  \rput{45}(8,6){\pnode(-2.5,0){Kitetail}}
  \rput{-10}(.8,1.5){\psdiamond[fillcolor=yellow](.6,.1)(.6,.3)}
  \rput{-80}(.8,1.5){\psdiamond[fillcolor=yellow](.6,.1)(.6,.3)}
  \pnode(.8,1.5){Tailend}
  \nccurve[fillstyle=none,angleA=270,angleB=125,ncurvB=.9,ncurvA=1.4,
           linestyle=dotted,dotstyle=square,linewidth=.25]{Kitetail}{Tailend}
  \newcommand{\bunting}{\pstriangle(.35,.35)}
  \psset{fillcolor=red,labelsep=.01}
  \naput[nrot=115,npos=.15]{\bunting}
  \nbput[nrot=25,npos=.15]{\bunting}
  \naput[nrot=75,npos=.4]{\bunting}
  \nbput[nrot=115,npos=.4]{\bunting}
  \naput[nrot=115,npos=.7]{\bunting}
  \nbput[nrot=25,npos=.7]{\bunting}
\end{pspicture}
\endTeXtoEPS
\end{pdfdisplay}
\end{verbatim}

\configure[pdfgraphic][width=\linewidth,
                       linewidth=2pt,
                       linecolor=gray30,                  
                       background=gray10,
                       rulesep=6pt]
\begin{pdfdisplay}
\TeXtoEPS
\definecolor{lightblue}{rgb}{0,0,.5}
\definecolor{Navy}{rgb}{0,0,0.5}
\definecolor{LemonChiffon}{rgb}{1,0.98,0.8}
\definecolor{ForestGreen}{rgb}{0.13,0.55,0.13}
\psset{unit=.825cm}
\begin{pspicture}(10,8)
  \psset{fillstyle=solid,linestyle=none,linewidth=0}
  \psframe[fillcolor=lightblue](10,8)
  \pscircle[fillcolor=yellow](2,6){.8} % Sun
  {% Rays
   \psset{linecolor=yellow,linestyle=solid,linewidth=.3}
   \degrees[8]
   \multido{\i=1+1}{8}{\rput{\i}(2,6){\psline(1,0)(1.5,0)}}
  }%
  \pspolygon[fillcolor=green](6,0)(10,2)(10,0)% Grass
  \psdiamond[fillcolor=red,gangle=-45](8,6)(1.5,2.5)% Kite
  \rput{45}(8,6){\pnode(-2.5,0){Kitetail}}
  \rput{-10}(.8,1.5){\psdiamond[fillcolor=yellow](.6,.1)(.6,.3)}
  \rput{-80}(.8,1.5){\psdiamond[fillcolor=yellow](.6,.1)(.6,.3)}
  \pnode(.8,1.5){Tailend}
  \nccurve[fillstyle=none,angleA=270,angleB=125,ncurvB=.9,ncurvA=1.4,
           linestyle=dotted,dotstyle=square,linewidth=.25]{Kitetail}{Tailend}
  \newcommand{\bunting}{\pstriangle(.35,.35)}
  \psset{fillcolor=red,labelsep=.01}
  \naput[nrot=115,npos=.15]{\bunting}
  \nbput[nrot=25,npos=.15]{\bunting}
  \naput[nrot=75,npos=.4]{\bunting}
  \nbput[nrot=115,npos=.4]{\bunting}
  \naput[nrot=115,npos=.7]{\bunting}
  \nbput[nrot=25,npos=.7]{\bunting}
\end{pspicture}
\endTeXtoEPS
\end{pdfdisplay}

\configure[pdfgraphic][width=\linewidth,
                       linewidth=2pt,
                       linecolor=gray30,                  
                       background=gray10,
                       rulesep=12pt]

\begin{pdfdisplay}

\definecolor{Pink}{rgb}{1.,0.75,0.8}
%  Flow diagram with the psmatrix environment\\[5mm]

  \psframebox[linearc=5mm,cornersize=absolute]{%
    \begin{psmatrix}[rowsep=0.5cm,colsep=0.8cm]
      \psovalbox[fillstyle=solid,fillcolor=yellow]{Begin} \\
      \psframebox{Initialisations} \\
      \psdiabox[fillstyle=solid,fillcolor=Pink]{Special} &
         \psframebox{Call to SP1} & \psframebox{Call to SP2} \\
      \psframebox{Action 1} \\
      \psframebox{Action 2} \\
      \psovalbox[fillstyle=solid,fillcolor=yellow]{End}
      % Links
      \ncline{1,1}{2,1}
      \ncline{2,1}{3,1}
      \ncline{3,1}{4,1}<{\textcolor{red}{No}}
      \ncline{4,1}{5,1}
      \ncline{5,1}{6,1}
      \ncline{->}{3,1}{3,2}^{\textcolor{red}{Yes}}
      \ncline{->}{3,2}{3,3}
      \ncbar[angleA=-90,armB=0,nodesepB=2.5mm]{->}{3,3}{4,1}
    \end{psmatrix}}
\end{pdfdisplay}

\begin{verbatim}
\begin{pdfdisplay}

\definecolor{Pink}{rgb}{1.,0.75,0.8}
%  Flow diagram with the psmatrix environment\\[5mm]

  \psframebox[linearc=5mm,cornersize=absolute]{%
    \begin{psmatrix}[rowsep=0.5cm,colsep=0.8cm]
      \psovalbox[fillstyle=solid,fillcolor=yellow]{Begin} \\
      \psframebox{Initialisations} \\
      \psdiabox[fillstyle=solid,fillcolor=Pink]{Special} &
         \psframebox{Call to SP1} & \psframebox{Call to SP2} \\
      \psframebox{Action 1} \\
      \psframebox{Action 2} \\
      \psovalbox[fillstyle=solid,fillcolor=yellow]{End}
      % Links
      \ncline{1,1}{2,1}
      \ncline{2,1}{3,1}
      \ncline{3,1}{4,1}<{\textcolor{red}{No}}
      \ncline{4,1}{5,1}
      \ncline{5,1}{6,1}
      \ncline{->}{3,1}{3,2}^{\textcolor{red}{Yes}}
      \ncline{->}{3,2}{3,3}
      \ncbar[angleA=-90,armB=0,nodesepB=2.5mm]{->}{3,3}{4,1}
    \end{psmatrix}}
\end{pdfdisplay}
\end{verbatim}
  
\end{document}
