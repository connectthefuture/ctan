\documentclass[pagesize=auto, fontsize=12pt, DIV=11]{scrartcl}

\usepackage{fixltx2e}
\usepackage{etex}
\usepackage{xspace}
\usepackage{lmodern}
\usepackage[T1]{fontenc}
\usepackage{textcomp}
\usepackage[svgnames]{xcolor}
\usepackage{listings}
\usepackage{microtype}
\usepackage{hyperref}

\newcommand*{\mail}[1]{\href{mailto:#1}{\texttt{#1}}}
\newcommand*{\pkg}[1]{\textsf{#1}}
\newcommand*{\cs}[1]{\texttt{\textbackslash#1}}
\makeatletter
\newcommand*{\cmd}[1]{\cs{\expandafter\@gobble\string#1}}
\makeatother
\newcommand*{\env}[1]{\texttt{#1}}

\addtokomafont{title}{\rmfamily}

\lstset{%
  language=[LaTeX]TeX,%
  columns=flexible,%
  upquote=true,%
  numbers=left,%
  basicstyle=\ttfamily,%
  keywordstyle=\color{Navy},%
  commentstyle=\color{DimGray},%
  stringstyle=\color{SeaGreen},%
  numberstyle=\scriptsize\color{SlateGray}%
}

\title{The \pkg{outline} package}
\subtitle{Simple Outline Package}
\author{%
  Peter Halvorson\thanks{Georgia Institute of Technology, Nuclear Engineering, \mail{peter@fission.gatech.edu}}%
  \and Seth Flaxman\thanks{\mail{seth@abisoft.com}}%
  \and Clea F. Rees%
}
\date{2002/08/23}


\begin{document}

\maketitle

\noindent

\begin{quote}
  \footnotesize
  This work may be distributed and/or modified under the
  conditions of the \LaTeX\ Project Public License, either version~1.3
  of this license or (at your option) any later version.
  The latest version of this license is in
  \url{http://www.latex-project.org/lppl.txt}
  and version~1.3 or later is part of all distributions of \LaTeX\ %
  version~2005/12/01 or later.
\end{quote}


\section{Summary}

The package defines an \env{outline} environment, which provides facilities 
similar to \env{enumerate}, but up to~6 levels deep.


\section{Description}

Create six-level list environment \env{\{outline\}} for making outlines; mark
each outline topic with \cmd{\item}.  Use of label/ref sequences provided.
A direct hack of the \env{enumerate} code from \texttt{latex.tex} (added more depth and
outline style numbering).  Use as you would use the \env{enumerate} environment.


\section{History}

\begin{labeling}[\hspace{\labelsep}--]{January 10, 1991}
\item[January 10, 1991] Copyright 1991 Peter Halvorson
\item[August 23, 2002] Updates for \LaTeXe\ copyright 2002 Seth Flaxman
\item[October 6, 2008] LPPL 1.3c or later by Clea F. Rees (for Seth Flaxman)
\item[May 16, 2010] \LaTeX\ version of documentation created by Philipp Stephani
\end{labeling}


\section{Example}

\begin{lstlisting}
\documentclass{report}
\usepackage{outline}

% [outline] includes new outline environment. I. A. 1. a. (1) (a)
% use \begin{outline} \item ... \end{outline}

\pagestyle{empty}

\begin{document}

\begin{outline}
  \item {\bf Introduction }
  \begin{outline}
    \item {\bf Applications } \\
      Motivation for research and applications related to the
      subject.
    \item {\bf Organization } \\
      Explain organization of the report, what is included, and what
      is not.
  \end{outline}
  \item {\bf Literature Survey }
  \begin{outline}
    \item {\bf Experimental Work } \\
      Literature describing experiments with something in common with
      my experiment.  My experiment is subdivided into section
      relating to each aspect of the whole.
    \begin{outline}
      \item {\bf Drop Delivery } \\
	Literature relating to the production of droplets.
      \begin{outline}
	\item {\bf Continuous } \\
	  Continuous drop production methods, i.e. jet methods.
	\item {\bf Drop on Demand } \\
	  Drop on demand methods, i.e. ink jet devices.  Produce drops
	  whenever needed, simplifies control of frequency.
	\item {\bf Flexibility } \\
	  Best methods in terms of flexible velocities, volumes, and
	  frequencies.
	\item {\bf Control Circuitry } \\
	  Circuitry necessary to control the drops, may include
	  control of generation, size, and frequency.  Divertors and
	  drop chargers.
	\item {\bf Extensibility } \\
	  Methods extensible to 2D applications.
	\item {\bf Recirculation } \\
	  Recirculation techniques, pump, none, capillary.
      \end{outline}
      \item {\bf Instrumentation } \\
	Literature dealing with measurement of various parameters.
      \begin{outline}
	\item {\bf Temperature }
        \begin{outline}
          \item {\bf Heater Surface }
	  \item {\bf Fluid Temperature }
	  \item {\bf Heat Flux }
	  \item {\bf Heat Transfer Coefficient }
        \end{outline}
	\item {\bf Drop Characteristics }
	\begin{outline}
	  \item {\bf Size }
	  \item {\bf Velocity }
	  \item {\bf Frequency }
        \end{outline}
      \end{outline}
      \item {\bf Heating Element } \\
	Literature dealing with the heating element.  Material
	properties, surface properties, heat sources.
      \begin{outline}
	\item {\bf Material }
	\item {\bf Heat Source }
      \end{outline}
    \end{outline}
    \item {\bf Analytical Work }
    \begin{outline}
      \item {\bf Evaporation }
      \item {\bf Boiling }
      \item {\bf Leidenfrost Temperatures }
      \item {\bf Heat Transfer }
      \item {\bf Numerical Analysis }
      \begin{outline}
	\item {\bf Drop Characteristics }
	\item {\bf Surface Wetting }
	\item {\bf Transient Temperatures }
      \end{outline}
    \end{outline}
  \end{outline}
  \item {\bf Proposed Research }
  \begin{outline}
    \item {\bf Experimental Work }
    \item {\bf Analytical Work }
  \end{outline}
\end{outline}

\end{document}
\end{lstlisting}

\end{document}
