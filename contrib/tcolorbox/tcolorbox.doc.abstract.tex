% !TeX root = tcolorbox.tex
% include file of tcolorbox.tex (manual of the LaTeX package tcolorbox)
\begin{tcboutputlisting}
% \usepackage{incgraph}
\begin{inctext}
\begin{tikzpicture}
\definecolorseries{boxcol}{rgb}{last}{blue}{red}
\resetcolorseries[28]{boxcol}
\coordinate (A) at (0,0); \coordinate (B) at (21,29.7);
\path[use as bounding box] (A) rectangle coordinate (C) (B);
\node[transform shape,xslant=0.7,rotate=-10,xshift=0cm] at (C) {%
  \begin{tcbraster}[raster columns=4,title=tcolorbox \version,
    fonttitle=\small\bfseries,raster width=50cm]
  \foreach \b in {1,...,28} {\begin{tcolorbox}[enhanced,
    watermark text=\thetcbrasternum,
    colframe=boxcol!30!white,
    colback=boxcol!25!white!30!white,
    colbacktitle=boxcol!!+!50!black!30!white,
    colupper=black!30!white]\lipsum[2]\end{tcolorbox}}
  \end{tcbraster}%
};
\node at (C) {%
  \begin{tcbitemize}[title=tcolorbox \version,fonttitle=\small\bfseries,
    enhanced jigsaw,opacityback=0.5,opacitybacktitle=0.75,
    halign=center,valign=center,arc=5mm,
    raster width=16cm,raster column skip=8mm,raster halign=center,
    raster force size=false,
    raster row 1/.style={height=6cm},
    raster row 2/.style={width=6cm,height=4cm},
    raster column 1/.style={flushright title,
      frame style={left color=yellow!50!black,right color=green!50!black},
      title style={left color=yellow!50!blue,right color=blue!50!green!50!black},
      interior style={left color=yellow!70,right color=green!70},
      underlay={\draw[line width=6mm,line cap=round,black!60]
        ([shift={(0.4,-0.15)}]frame.north east)
        --([shift={(0.4,0.15)}]frame.south east); }},
    raster column 2/.style={
      frame style={left color=green!50!black,right color=yellow!50!black},
      title style={left color=blue!50!green!50!black,right color=yellow!50!blue},
      interior style={left color=green!70,right color=yellow!70}}]
  \tcbitem[fontupper=\Huge\bfseries,sharp corners=east,
    underlay={\draw[line width=6mm,line cap=round,black!60]
      ([shift={(0.4,0.30)}]frame.north east)-- coordinate(A) +(0,0.2);
      \draw[line width=1mm,line cap=round,black!60](A) -- +(30:1.5cm);
      \draw[line width=1mm,line cap=round,black!60](A) -- +(150:1.5cm);}]
    tcolorbox
  \tcbitem[fontupper=\large\bfseries,sharp corners=west]
    Manual for\\ version\\ \version\\(\datum)
  \tcbitem[sharp corners=northeast]
  \tcbitem[sharp corners=northwest] Thomas F.~Sturm
  \end{tcbitemize}%
};
\end{tikzpicture}
\end{inctext}
\end{tcboutputlisting}
\tcbuselistingtext
\tcbinputlisting{title=Cover code,
  base example,coltitle=black,fonttitle=\itshape,titlerule=0pt,
  colbacktitle=Navy!15!ExampleBack,top=0mm,before=\par\smallskip,%
  listing style=mydocumentation,listing only}

%\bigskip
%\begin{marker}
%If you have trouble printing this document, the reason is quite likely the
%cover page. Printing the pages starting with page 2 or page 3 should work.
%\end{marker}

\clearpage
\begin{center}
\begin{tcolorbox}[enhanced,hbox,tikznode,left=8mm,right=8mm,boxrule=0.4pt,
  colback=white,colframe=black!50!yellow,
  %drop fuzzy midday shadow=black!50!yellow,
  drop lifted shadow=black!50!yellow,arc is angular,
  before=\par\vspace*{5mm},after=\par\bigskip]
{\bfseries\LARGE The \texttt{tcolorbox} package}\\[3mm]
{\large Manual for version \version\ (\datum)}
\end{tcolorbox}
{\large Thomas F.~Sturm%
  \footnote{Prof.~Dr.~Dr.~Thomas F.~Sturm, Institut f\"{u}r Mathematik und Informatik,
    Universit\"{a}t der Bundeswehr M\"{u}nchen, D-85577 Neubiberg, Germany;
     email: \href{mailto:thomas.sturm@unibw.de}{thomas.sturm@unibw.de}}\par\medskip
\normalsize\url{http://www.ctan.org/pkg/tcolorbox}\par
\url{https://github.com/T-F-S/tcolorbox}}
\end{center}
\bigskip
\begin{absquote}
  \begin{center}\bfseries Abstract\end{center}
  |tcolorbox| provides an environment for colored and framed text boxes with a
  heading line. Optionally, such a box can be split in an upper and a lower
  part. The package |tcolorbox| can be used for the setting of \LaTeX\ examples where
  one part of the box displays the source code and the other part shows the
  output. Another common use case is the setting of theorems. The package supports
  saving and reuse of source code and text parts.
\end{absquote}

\begin{tcolorbox}[breakable,enhanced jigsaw,title={Contents},fonttitle=\bfseries\Large,
  colback=yellow!10!white,colframe=red!50!black,before=\par\bigskip\noindent,
  interior style={fill overzoom image=goldshade.png,fill image opacity=0.25},
  colbacktitle=red!50!yellow!75!black,
  enlargepage flexible=\baselineskip,pad at break*=3mm,
  watermark color=yellow!75!red!25!white,
  watermark text={\bfseries\Large Contents},
  attach boxed title to top center={yshift=-0.25mm-\tcboxedtitleheight/2,yshifttext=2mm-\tcboxedtitleheight/2},
  boxed title style={enhanced,boxrule=0.5mm,
    frame code={ \path[tcb fill frame] ([xshift=-4mm]frame.west) -- (frame.north west)
    -- (frame.north east) -- ([xshift=4mm]frame.east)
    -- (frame.south east) -- (frame.south west) -- cycle; },
    interior code={ \path[tcb fill interior] ([xshift=-2mm]interior.west)
    -- (interior.north west) -- (interior.north east)
    -- ([xshift=2mm]interior.east) -- (interior.south east) -- (interior.south west)
    -- cycle;}  },
  drop fuzzy shadow]
\makeatletter
\@starttoc{toc}
\makeatother
\end{tcolorbox}
