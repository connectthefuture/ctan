\documentclass{book}
\usepackage{corridx}
\usepackage{makeidx}
%\usepackage[chemical,english]{babel}
\makeindex
\begin{document}

%\selectlanguage{english}

\tableofcontents

 \chapter{Testing of the Corridx Package}


\section{User Defined Commands}




We show now the use of these commands in the following sample
text. You should look in the original \LaTeX\space file.

Note that we have in the text also the commands
\begin{verbatim}
\index{acr  @\sectioncrrdx{Index of Acronyms}\swallow|swallow}%
\index{chem  @\sectioncrrdx{Index of Chemicals}\swallow|swallow}%
\index{gen  @\sectioncrrdx{General Index}\swallow|swallow}%
\end{verbatim}


Optionally you can use something like
\verb"\newcommand{\cis}{\textit{cis}}". It will also work.



\newcommand{\cis}{\textit{cis}}



\section{Sample Text}

There are various types of novolak resins with different \ig[
resins]{ortho} to para ratios of the methylene linkages, \ia{high
ortho novolak resins}{HON}, \ia{general-purpose novolak
resins}{GPN} and \ia{high para novolak resins}{HPN}.

The liquid-phase oxidation of cumene to \ib{cumene
hydroperoxide}{CHP} results in \ic{acetone} and \ic{phenol}. This
is used for \ic{bisphenol~A}, \ic{bisphenol~B}, \ic{resorcinol},
\ic{cresol}s, and \ic{xylenol}s. \ic{2-Cyclohexyl-5-methylphenol}
is used for \ig[!positive]{photoresist}s.
\ic{\textit{m}-Methoxyphenol}, \ic{2-naphthol}, \ic{cardanol}, and
\ic{cardol}, are other suitable \ig[!other]{phenols}.

Compounds, such as \ic{$\alpha$-methylstyrene} or
\ib{\textit{N},\textit{N}-dimethyl formamide}{DMF} are not used.
Also \ib{1,3-propanediol}{1,3-PD} is not used. Further
\ib{\cis-3-hexen-1-ol}{3-HXL} or \ib{2-pyridylcarbinol}{PC} are
not a reasonable solution.

\ic{2,5-Norbornadiene} is also known as
\ic{bicyclo[2.2.1]hepta-2,5-diene}. Another interesting compound
is \ic{[\textup{2.2.1.0$^{2,6}$.0$^{3,5}$}]quadricycloheptane}.


We switch now \newline \verb"\crrdxformatpage{chem}{|textit}" and
\newline\verb"\crrdxformatpage{gen}{|textbf}"

\crrdxformatpage{chem}{|textit}%
\crrdxformatpage{gen}{|textbf}%

and check:

 \ic{1,2-butanediol}
 \ia{1,2-butanediol}{1,2-BD}
 \ig[!unsaturated]{polyester}


\section{The Sample Text Verbatim}

\begin{verbatim}
There are various types of novolak resins with different \ig[
resins]{ortho} to para ratios of the methylene linkages, \ia{high
ortho novolak resins}{HON}, \ia{general-purpose novolak
resins}{GPN} and \ia{high para novolak resins}{HPN}.

The liquid-phase oxidation of cumene to \ib{cumene
hydroperoxide}{CHP} results in \ic{acetone} and \ic{phenol}. This
is used for \ic{bisphenol~A}, \ic{bisphenol~B}, \ic{resorcinol},
\ic{cresol}s, and \ic{xylenol}s. \ic{2-Cyclohexyl-5-methylphenol}
is used for \ig[!positive]{photoresist}s.
\ic{\textit{m}-Methoxyphenol}, \ic{2-naphthol}, \ic{cardanol}, and
\ic{cardol}, are other suitable \ig[!other]{phenols}.

Compounds, such as \ic{$\alpha$-methylstyrene} or
\ib{\textit{N},\textit{N}-dimethyl formamide}{DMF} are not used.
Also \ib{1,3-propanediol}{1,3-PD} is not used. Further
\ib{\cis-3-hexen-1-ol}{3-HXL} or \ib{2-pyridylcarbinol}{PC} are
not a reasonable solution.

\ic{2,5-Norbornadiene} is also known as
\ic{bicyclo[2.2.1]hepta-2,5-diene}. Another interesting compound
is \ic{[\textup{2.2.1.0$^{2,6}$.0$^{3,5}$}]quadricycloheptane}.


We switch now \newline \verb"\crrdxformatpage{chem}{|textit}" and
\newline\verb"\crrdxformatpage{gen}{|textbf}"

\crrdxformatpage{chem}{|textit}%
\crrdxformatpage{gen}{|textbf}%

and check:

 \ic{1,2-butanediol}
 \ia{1,2-butanediol}{1,2-BD}
 \ig[!unsaturated]{polyester}


\end{verbatim}




\index{acr  @\sectioncrrdx{Index of Acronyms}\swallow|swallow}%
\index{chem  @\sectioncrrdx{Index of Chemicals}\swallow|swallow}%
\index{gen  @\sectioncrrdx{General Index}\swallow|swallow}%

  \printindex
\end{document}
