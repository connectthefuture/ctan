%	\iffalse
%% File: fullblck.dtx	Copyright 1998,1999 James H. Cloos, Jr. <cloos@jhcloos.com>
%
%<*preamble>
%
%	(1)  If you do not already have 'fullblck.ins'
%	     run 'fullblck.dtx' through LaTeX to create it.
%
%	     Now run 'fullblck.ins' through (La)TeX to get the package.
%
%	     [Or use 'docstrip' and extract 'fullblck.sty' from 'fullblck.dtx'
%	     using option 'package']
%
%	(2)  Then run 'fullblck.dtx' three times thought LaTeX
%	     to get the documentation 'fullblck.dvi'.
%
%
%
%%% ====================================================================
%%%  @LaTeX-package-file{
%%%     author          = "James H. Cloos, Jr.",
%%%     version         = "1.03",
%%%     date            = "1999/May/25",
%%%     time            = "22:30:00 UTC",
%%%     filename        = "fullblck.dtx",
%%%     address         = "James H. Cloos, Jr.
%%%                        6607 Brodie Ln #414
%%%                        Austin, TX 78745-4650
%%%                        US",
%%%     telephone       = "+1 888.612.7791",
%%%     FAX             = "+1 888.612.7791",
%%%     email           = "cloos@jhcloos.com",
%%%     codetable       = "ISO/ASCII",
%%%     keywords        = "letter, fullblock",
%%%     supported       = "yes",
%%%     abstract        = "This package defines commands for using
%%%                        standard SI units in all your texts. The
%%%                        package provides different options for the
%%%                        spacing of the units.",
%%%     docstring       = "This is based heavily on LaTeX2e's letter.dtx, which is:
%%%                           Copyright 1993 1994 1995 1996 1997 The LaTeX3 Project
%%%                           and any individual authors listed in letter.dtx. 
%%%                        My modifications are Copyright 1998,1999 James H. Cloos, Jr.
%%%                        All rights reserved.",
%%%  }
%%% ====================================================================
%
%
% IMPORTANT COPYRIGHT NOTICE:
%
% This package is based on letter.dtx from LaTeX2e and may be used and 
% distributed under the same terms as LaTeX2e.
%
%</preamble>
%
%
%<*batchfile>
\begin{filecontents}{fullblck.ins}
\def\batchfile{fullblck.ins}
\input docstrip.tex
\keepsilent
\generate{\file{fullblck.sty}{\from{fullblck.dtx}{package}}}
\endbatchfile
\end{filecontents}
%</batchfile>
%
%<*driver>
\def\fileversion{1.03}
\def\filedate{1999/May/25}
\documentclass{ltxdoc}
%
\begin{document}
\DocInput{fullblck.dtx}
\end{document}
%</driver>
% \fi
%
%
% \CheckSum{0}
% \CharacterTable
%  {Upper-case    \A\B\C\D\E\F\G\H\I\J\K\L\M\N\O\P\Q\R\S\T\U\V\W\X\Y\Z
%   Lower-case    \a\b\c\d\e\f\g\h\i\j\k\l\m\n\o\p\q\r\s\t\u\v\w\x\y\z
%   Digits        \0\1\2\3\4\5\6\7\8\9
%   Exclamation   \!     Double quote  \"     Hash (number) \#
%   Dollar        \$     Percent       \%     Ampersand     \&
%   Acute accent  \'     Left paren    \(     Right paren   \)
%   Asterisk      \*     Plus          \+     Comma         \,
%   Minus         \-     Point         \.     Solidus       \/
%   Colon         \:     Semicolon     \;     Less than     \<
%   Equals        \=     Greater than  \>     Question mark \?
%   Commercial at \@     Left bracket  \[     Backslash     \\
%   Right bracket \]     Circumflex    \^     Underscore    \_
%   Grave accent  \`     Left brace    \{     Vertical bar  \|
%   Right brace   \}     Tilde         \~}
%
%
%
% \title{The \texttt{fullblck} package\thanks{This file
%        has version number \fileversion, last revised \filedate.}}
%
% \author{James H. Cloos, Jr.}
% \date{1998/November/24}
% \maketitle
%
%
% \changes{v1.1}{1998/11/17}{Initial version.}%
% \changes{v1.2}{1998/11/24}{Coverted to \textsc{dtx} file.}%
%
% \begin{abstract}
% This article describes the \texttt{fullblck} package which modifies
% the \texttt{letter} class to use a full block layout.
% Specifically, this means that the return address,
% date and closing will be set at the left margin.
% \end{abstract}
%
% \section{Introduction}
% In the US, there are three common layouts used for business letters.
% \LaTeX2e's \texttt{letter} class implements the indented layout.
% This package modifies
% \LaTeX2e's \texttt{letter} class to set the letter in the fullblock
% layout.  No attempt is made to support the semi-block layout.
%
%
% \section{How to use the package}
%
% Add \texttt{\char92usepackage\{fullblck\}} to the preamble of your letter.
% No options are required or supported, and no new commands are defined.
%
% \section{The implementation}
%
% First, the basic boilerplate for a \textsc{sty} file:
%
%    \begin{macrocode}
%<*package>
\NeedsTeXFormat{LaTeX2e}[1996/06/01]
\ProvidesPackage{fullblck}
  [1998/11/17 v1.1 JHCloos FullBlock Style for Letters]
%    \end{macrocode}
%
% Now we set the \texttt{\char92longindentation} to zero;
% this puts the closing at the left margin.
%
%    \begin{macrocode}
\longindentation=0pt%
%    \end{macrocode}
%
% Finally we redefine \texttt{\char92opening} to just dump the return
% address and date at the left margin in \texttt{\char92raggedright}
% mode.
%
%    \begin{macrocode}
\renewcommand*{\opening}[1]{\ifx\@empty\fromaddress
  \thispagestyle{firstpage}%
    {\raggedright\@date\par}%
  \else  % home address
    \thispagestyle{empty}%
    {\raggedright\ignorespaces
      \fromaddress \\*[2\parskip]%
      \@date \par}%
  \fi
  \vspace{2\parskip}%
  {\raggedright \toname \\ \toaddress \par}%
  \vspace{2\parskip}%
  #1\par\nobreak}
%</package>
%    \end{macrocode}
%
%
% \Finale

