% -*- mode: LaTeX; coding: iso-8859-15 -*-
\documentclass[a4paper,10pt]{article}
\usepackage[latin9]{inputenc}
\usepackage[T1]{fontenc}
\usepackage[frenchb]{babel}
\usepackage{lmodern}
\usepackage[tame]{engpron}
\usepackage{clefval}

\newcommand{\Saut}{\par
  \noindent\hspace*{\stretch{1}}%
  \makebox[0.75\linewidth][c]{\hrulefill}%
  \hspace*{\stretch{1}}\par}
\setlength{\parindent}{0pt}

\begin{document}

\makepoundletter
\newcommand{\LaPron}[1]{#1
  \pron{\ActiveLaLivre\expandafter{\TheValue{#1}}}} 
\newcommand{\LaCle}[2]{\TheKey{#1}{#2}\LaPron{#1}}
\makepoundother

\title{Exemple accompagnant le module \texttt{clefval}.}
\author{Le \TeX nicien de surface}
\def\today{2004-05-23}
\maketitle
\thispagestyle{empty}

Ceci est un exemple d'utilisation des modules \texttt{engpron}
---~avec l'option \texttt{tame}~--- et \texttt{clefval}.

\Saut\medskip

Ceci est l'essentiel de l'ent�te de ce document, on notera que l'on a
plac� la d�finition des commandes \verb|\LaPron| et \verb|\LaCle|
\textbf{apr�s} le \verb|\begin{document}|. 

\begin{verbatim}
\documentclass[a4paper,10pt]{article}
\usepackage[latin9]{inputenc}     \usepackage[T1]{fontenc}
\usepackage[frenchb]{babel}       \usepackage{lmodern}
\usepackage[tame]{engpron}        \usepackage{clefval}
\setlength{\parindent}{0pt}
\begin{document}
\makepoundletter
\newcommand{\LaPron}[1]{#1
\pron{\ActiveLaLivre\expandafter{\TheValue{#1}}}} 
\newcommand{\LaCle}[2]{\TheKey{#1}{#2}\LaPron{#1}}
\makepoundother
\end{verbatim}

\Saut
\medskip

On �crit une premi�re fois le mot et sa prononciation
\LaCle{baby}{b�qb�i}.  Plus tard, on n'�crit \emph{presque} plus que
le mot et on a �galement la prononciation, comme ici : \LaPron{baby}.

Un autre \LaCle{stationery}{�Hst�q�s�en�Xer�i}. Et donc
\LaPron{stationery}.

\bigskip

Ce qui est obtenu avec :

\bigskip

\verb|On �crit une premi�re fois le mot et sa prononciation |\\
\verb|\LaCle{baby}{b|\texttt{\�}\verb|qb|\texttt{\�}\verb|i}.| 
\verb|Plus tard, on n'�crit \emph{presque} plus que |\\ 
\verb|le mot et on a �galement la prononciation,| \verb|comme ici :|
\verb|\LaPron{baby}.| 

\verb|Un autre\LaCle{stationery}{|\texttt{\�}\verb|Hst|%
\texttt{\�}\verb|q|\texttt{\�}%
\verb|s|\texttt{\�}\verb|en|\texttt{\�}\verb|Xer|\texttt{\�}\verb|i}.|\\
\verb|Et donc |\verb|\LaPron{stationery}.|

\Saut\medskip

Contenu du fichier \texttt{aux} (extrait) :

\verb!\newkey{baby}{b!\texttt{\�}\verb!qb!\texttt{\�}\verb!i}!\\
\verb!\newkey{stationery}{!\texttt{\�}\verb!Hst!\texttt{\�}%
\verb!q!\texttt{\�}\verb!s!\texttt{\�}\verb!en!\texttt{\�}\verb!Xer!%
\texttt{\�}\verb!i}! 

%%% Faux verbatim pour � pour des raisons de codes !

\vspace*{\stretch{1}}
\end{document}

%%% Local Variables: 
%%% mode: latex
%%% TeX-master: t
%%% End: 
