%\iffalse
%<*package>
%% \CharacterTable
%%  {Upper-case    \A\B\C\D\E\F\G\H\I\J\K\L\M\N\O\P\Q\R\S\T\U\V\W\X\Y\Z
%%   Lower-case    \a\b\c\d\e\f\g\h\i\j\k\l\m\n\o\p\q\r\s\t\u\v\w\x\y\z
%%   Digits        \0\1\2\3\4\5\6\7\8\9
%%   Exclamation   \!     Double quote  \"     Hash (number) \#
%%   Dollar        \$     Percent       \%     Ampersand     \&
%%   Acute accent  \'     Left paren    \(     Right paren   \)
%%   Asterisk      \*     Plus          \+     Comma         \,
%%   Minus         \-     Point         \.     Solidus       \/
%%   Colon         \:     Semicolon     \;     Less than     \<
%%   Equals        \=     Greater than  \>     Question mark \?
%%   Commercial at \@     Left bracket  \[     Backslash     \\
%%   Right bracket \]     Circumflex    \^     Underscore    \_
%%   Grave accent  \`     Left brace    \{     Vertical bar  \|
%%   Right brace   \}     Tilde         \~}
%</package>
%\fi
% \iffalse
% Doc-Source file to use with LaTeX2e
% Copyright (C) 2015 Nicola Talbot, all rights reserved.
% (New maintainer add relevant lines here.)
% \fi
% \iffalse
%<*driver>
\documentclass{ltxdoc}

\usepackage{alltt}
\usepackage{graphicx}
\usepackage[utf8]{inputenc}
\usepackage[T1]{fontenc}
\usepackage[colorlinks,
            bookmarks,
            hyperindex=false,
            pdfauthor={Nicola L.C. Talbot},
            pdftitle={datetime2.sty Dutch Module}]{hyperref}


\CheckSum{246}

\renewcommand*{\usage}[1]{\hyperpage{#1}}
\renewcommand*{\main}[1]{\hyperpage{#1}}
\IndexPrologue{\section*{\indexname}\markboth{\indexname}{\indexname}}
\setcounter{IndexColumns}{2}

\newcommand*{\sty}[1]{\textsf{#1}}
\newcommand*{\opt}[1]{\texttt{#1}\index{#1=\texttt{#1}|main}}

\RecordChanges
\PageIndex
\CodelineNumbered

\begin{document}
\DocInput{datetime2-dutch.dtx}
\end{document}
%</driver>
%\fi
%
%\MakeShortVerb{"}
%
%\title{Dutch Module for datetime2 Package}
%\author{Nicola L. C. Talbot (inactive)}
%\date{2015-03-29 (v1.0)}
%\maketitle
%
%This module is currently unmaintained and may be subject to change.
%If you want to volunteer to take over maintanance, contact me at
%\url{http://www.dickimaw-books.com/contact.html}
%
%\begin{abstract}
%This is the Dutch language module for the \sty{datetime2}
%package. If you want to use the settings in this module you must
%install it in addition to installing \sty{datetime2}. If you use
%\sty{babel} or \sty{polyglossia}, you will need this module to
%prevent them from redefining \cs{today}. The \sty{datetime2}
% \opt{useregional} setting must be set to "text" or "numeric"
% for the language styles to be set.
% Alternatively, you can set the style in the document using
% \cs{DTMsetstyle}, but this may be changed by \cs{date}\meta{language}
% depending on the value of the \opt{useregional} setting.
%\end{abstract}
%
%I've copied the date style from \texttt{babel-dutch}'s \cs{today}.
%
%I don't know if these settings are correct.
%In particular, I don't know if the "dutch" time style is
%correct. Currently this just uses the "default" time style. Please
%be aware that this may change. Whoever takes over maintanance
%of this module may can change it as appropriate.
%
%The new maintainer should add the line:
%\begin{verbatim}
% The Current Maintainer of this work is Name.
%\end{verbatim}
%to the preamble part in \texttt{datetime2-dutch.ins} where Name
%is the name of the maintainer(s) and replace
%the `inactive' status to `maintained'.
%
%Currently there is only a regionless style.
%
%\StopEventually{%
%\clearpage
%\phantomsection
%\addcontentsline{toc}{section}{Change History}%
%\PrintChanges
%\addcontentsline{toc}{section}{\indexname}%
%\PrintIndex}
%\section{The Code}
%At the moment there is only the one ".ldf" file.
%
%\subsection{Main Dutch Module (\texttt{datetime2-dutch.ldf})}
%\changes{1.0}{2015-03-29}{Initial release}
%
%\iffalse
%    \begin{macrocode}
%<*datetime2-dutch.ldf>
%    \end{macrocode}
%\fi
%
% Identify Module
%    \begin{macrocode}
\ProvidesDateTimeModule{dutch}[2015/03/29 v1.0]
%    \end{macrocode}
%\begin{macro}{\DTMdutchordinal}
% I don't know if this is correct, but it's provided in case 
% a suffix is required.
%    \begin{macrocode}
\newcommand*{\DTMdutchordinal}[1]{%
  \number#1 % space intended
}
%    \end{macrocode}
%\end{macro}
%
%\begin{macro}{\DTMdutchmonthname}
% Dutch month names.
%    \begin{macrocode}
\newcommand*{\DTMdutchmonthname}[1]{%
  \ifcase#1
  \or
  januari%
  \or
  februari%
  \or
  maart%
  \or
  april%
  \or
  mei%
  \or
  juni%
  \or
  juli%
  \or
  augustus%
  \or
  september%
  \or
  oktober%
  \or
  november%
  \or
  december%
  \fi
}
%    \end{macrocode}
%\end{macro}
%
%\begin{macro}{\DTMdutchMonthname}
% As above but start with a capital.
%    \begin{macrocode}
\newcommand*{\DTMdutchMonthname}[1]{%
  \ifcase#1
  \or
  Januari%
  \or
  Februari%
  \or
  Maart%
  \or
  April%
  \or
  Mei%
  \or
  Juni%
  \or
  Juli%
  \or
  Augustus%
  \or
  September%
  \or
  Oktober%
  \or
  November%
  \or
  December%
  \fi
}
%    \end{macrocode}
%\end{macro}
%
% Week day names are provided but currently not used in the date style.
%
%\begin{macro}{\DTMdutchweekdayname}
% Dutch day of week names.
%    \begin{macrocode}
\newcommand*{\DTMdutchweekdayname}[1]{%
  \ifcase#1
  maandag%
  \or
  dinsdag%
  \or
  woensdag%
  \or
  donderdag%
  \or
  vrijdag%
  \or
  zaterdag%
  \or
  zondag%
  \fi
}
%    \end{macrocode}
%\end{macro}
%
%\begin{macro}{\DTMdutchWeekdayname}
% As above but start with a capital.
%    \begin{macrocode}
\newcommand*{\DTMdutchWeekdayname}[1]{%
  \ifcase#1
  Maandag%
  \or
  Dinsdag%
  \or
  Woensdag%
  \or
  Donderdag%
  \or
  Vrijdag%
  \or
  Zaterdag%
  \or
  Zondag%
  \fi
}
%    \end{macrocode}
%\end{macro}
%
%\begin{macro}{\DTMdutchshortweekdayname}
% Abbreviated day of week names.
%    \begin{macrocode}
\newcommand*{\DTMdutchshortweekdayname}[1]{%
  \ifcase#1
  ma%
  \or
  di%
  \or
  wo%
  \or
  do%
  \or
  vr%
  \or
  za%
  \or
  zo%
  \fi
}
%    \end{macrocode}
%\end{macro}
%
%\begin{macro}{\DTMdutchshortWeekdayname}
% As above but start with a capital.
%    \begin{macrocode}
\newcommand*{\DTMdutchshortWeekdayname}[1]{%
  \ifcase#1
  Ma%
  \or
  Di%
  \or
  Wo%
  \or
  Do%
  \or
  Vr%
  \or
  Za%
  \or
  Zo%
  \fi
}
%    \end{macrocode}
%\end{macro}
%
% Define the \texttt{dutch} style.
% The time style is the same as the "default" style
% provided by \sty{datetime2}. This may need correcting. For
% example, if a 12 hour style similar to the "englishampm" (from the
% "english-base" module) is required. 
%
% Allow the user a way of configuring the "dutch" and
% "dutch-numeric" styles. This doesn't use the package wide
% separators such as
% \cs{dtm@datetimesep} in case other date formats are also required.
%\begin{macro}{\DTMdutchdaymonthsep}
% The separator between the day and month for the text format.
%    \begin{macrocode}
\newcommand*{\DTMdutchdaymonthsep}{%
 \DTMtexorpdfstring{\protect~}{\space}}
%    \end{macrocode}
%\end{macro}
%
%\begin{macro}{\DTMdutchmonthyearsep}
% The separator between the month and year for the text format.
%    \begin{macrocode}
\newcommand*{\DTMdutchmonthyearsep}{\space}
%    \end{macrocode}
%\end{macro}
%
%\begin{macro}{\DTMdutchdatetimesep}
% The separator between the date and time blocks in the full format
% (either text or numeric).
%    \begin{macrocode}
\newcommand*{\DTMdutchdatetimesep}{\space}
%    \end{macrocode}
%\end{macro}
%
%\begin{macro}{\DTMdutchtimezonesep}
% The separator between the time and zone blocks in the full format
% (either text or numeric).
%    \begin{macrocode}
\newcommand*{\DTMdutchtimezonesep}{\space}
%    \end{macrocode}
%\end{macro}
%
%\begin{macro}{\DTMdutchdatesep}
% The separator for the numeric date format.
%    \begin{macrocode}
\newcommand*{\DTMdutchdatesep}{-}
%    \end{macrocode}
%\end{macro}
%
%\begin{macro}{\DTMdutchtimesep}
% The separator for the numeric time format.
%    \begin{macrocode}
\newcommand*{\DTMdutchtimesep}{:}
%    \end{macrocode}
%\end{macro}
%
%Provide keys that can be used in \cs{DTMlangsetup} to set these
%separators.
%    \begin{macrocode}
\DTMdefkey{dutch}{daymonthsep}{\renewcommand*{\DTMdutchdaymonthsep}{#1}}
\DTMdefkey{dutch}{monthyearsep}{\renewcommand*{\DTMdutchmonthyearsep}{#1}}
\DTMdefkey{dutch}{datetimesep}{\renewcommand*{\DTMdutchdatetimesep}{#1}}
\DTMdefkey{dutch}{timezonesep}{\renewcommand*{\DTMdutchtimezonesep}{#1}}
\DTMdefkey{dutch}{datesep}{\renewcommand*{\DTMdutchdatesep}{#1}}
\DTMdefkey{dutch}{timesep}{\renewcommand*{\DTMdutchtimesep}{#1}}
%    \end{macrocode}
%
% TODO: provide a boolean key to switch between full and abbreviated
% formats if appropriate. (I don't know how the date should be
% abbreviated.)
%
% Define a boolean key that determines if the time zone mappings
% should be used.
%    \begin{macrocode}
\DTMdefboolkey{dutch}{mapzone}[true]{}
%    \end{macrocode}
% The default is to use mappings.
%    \begin{macrocode}
\DTMsetbool{dutch}{mapzone}{true}
%    \end{macrocode}
%
% Define a boolean key that determines if the day of month should be
% displayed.
%    \begin{macrocode}
\DTMdefboolkey{dutch}{showdayofmonth}[true]{}
%    \end{macrocode}
% The default is to show the day of month.
%    \begin{macrocode}
\DTMsetbool{dutch}{showdayofmonth}{true}
%    \end{macrocode}
%
% Define a boolean key that determines if the year should be
% displayed.
%    \begin{macrocode}
\DTMdefboolkey{dutch}{showyear}[true]{}
%    \end{macrocode}
% The default is to show the year.
%    \begin{macrocode}
\DTMsetbool{dutch}{showyear}{true}
%    \end{macrocode}
%
% Define the "dutch" style. (TODO: implement day of week?)
%    \begin{macrocode}
\DTMnewstyle
 {dutch}% label
 {% date style
   \renewcommand*\DTMdisplaydate[4]{%
     \DTMifbool{dutch}{showdayofmonth}
     {\DTMdutchordinal{##3}\DTMdutchdaymonthsep}%
     {}%
     \DTMdutchmonthname{##2}%
     \DTMifbool{dutch}{showyear}%
     {%
       \DTMdutchmonthyearsep
       \number##1 % space intended
     }%
     {}%
   }%
   \renewcommand*\DTMDisplaydate[4]{%
     \DTMifbool{dutch}{showdayofmonth}
     {%
       \DTMdutchordinal{##3}\DTMdutchdaymonthsep
       \DTMdutchmonthname{##2}%
     }%
     {%
        \DTMdutchMonthname{##2}%
     }%
     \DTMifbool{dutch}{showyear}%
     {%
       \DTMdutchmonthyearsep
       \number##1 % space intended
     }%
     {}%
   }%
 }%
 {% time style (use default)
   \DTMsettimestyle{default}%
 }%
 {% zone style
   \DTMresetzones
   \DTMdutchzonemaps
   \renewcommand*{\DTMdisplayzone}[2]{%
     \DTMifbool{dutch}{mapzone}%
     {\DTMusezonemapordefault{##1}{##2}}%
     {%
       \ifnum##1<0\else+\fi\DTMtwodigits{##1}%
       \ifDTMshowzoneminutes\DTMdutchtimesep\DTMtwodigits{##2}\fi
     }%
   }%
 }%
 {% full style
   \renewcommand*{\DTMdisplay}[9]{%
    \ifDTMshowdate
     \DTMdisplaydate{##1}{##2}{##3}{##4}%
     \DTMdutchdatetimesep
    \fi
    \DTMdisplaytime{##5}{##6}{##7}%
    \ifDTMshowzone
     \DTMdutchtimezonesep
     \DTMdisplayzone{##8}{##9}%
    \fi
   }%
   \renewcommand*{\DTMDisplay}[9]{%
    \ifDTMshowdate
     \DTMDisplaydate{##1}{##2}{##3}{##4}%
     \DTMdutchdatetimesep
    \fi
    \DTMdisplaytime{##5}{##6}{##7}%
    \ifDTMshowzone
     \DTMdutchtimezonesep
     \DTMdisplayzone{##8}{##9}%
    \fi
   }%
 }%
%    \end{macrocode}
%
% Define numeric style.
%    \begin{macrocode}
\DTMnewstyle
 {dutch-numeric}% label
 {% date style
    \renewcommand*\DTMdisplaydate[4]{%
      \DTMifbool{dutch}{showdayofmonth}%
      {%
        \number##3 % space intended
        \DTMdutchdatesep
      }%
      {}%
      \number##2 % space intended
      \DTMifbool{dutch}{showyear}%
      {%
        \DTMdutchdatesep
        \number##1 % space intended
      }%
      {}%
    }%
    \renewcommand*{\DTMDisplaydate}{\DTMdisplaydate}%
 }%
 {% time style
    \renewcommand*\DTMdisplaytime[3]{%
      \number##1
      \DTMdutchtimesep\DTMtwodigits{##2}%
      \ifDTMshowseconds\DTMdutchtimesep\DTMtwodigits{##3}\fi
    }%
 }%
 {% zone style
   \DTMresetzones
   \DTMdutchzonemaps
   \renewcommand*{\DTMdisplayzone}[2]{%
     \DTMifbool{dutch}{mapzone}%
     {\DTMusezonemapordefault{##1}{##2}}%
     {%
       \ifnum##1<0\else+\fi\DTMtwodigits{##1}%
       \ifDTMshowzoneminutes\DTMdutchtimesep\DTMtwodigits{##2}\fi
     }%
   }%
 }%
 {% full style
   \renewcommand*{\DTMdisplay}[9]{%
    \ifDTMshowdate
     \DTMdisplaydate{##1}{##2}{##3}{##4}%
     \DTMdutchdatetimesep
    \fi
    \DTMdisplaytime{##5}{##6}{##7}%
    \ifDTMshowzone
     \DTMdutchtimezonesep
     \DTMdisplayzone{##8}{##9}%
    \fi
   }%
   \renewcommand*{\DTMDisplay}{\DTMdisplay}%
 }
%    \end{macrocode}
%
%\begin{macro}{\DTMdutchzonemaps}
% The time zone mappings are set through this command, which can be
% redefined if extra mappings are required or mappings need to be
% removed.
%    \begin{macrocode}
\newcommand*{\DTMdutchzonemaps}{%
  \DTMdefzonemap{01}{00}{CET}%
  \DTMdefzonemap{02}{00}{CEST}%
}
%    \end{macrocode}
%\end{macro}

% Switch style according to the \opt{useregional} setting.
%    \begin{macrocode}
\DTMifcaseregional
{}% do nothing
{\DTMsetstyle{dutch}}
{\DTMsetstyle{dutch-numeric}}
%    \end{macrocode}
%
% Redefine \cs{datedutch} (or \cs{date}\meta{dialect}) to prevent
% \sty{babel} from resetting \cs{today}. (For this to work,
% \sty{babel} must already have been loaded if it's required.)
%    \begin{macrocode}
\ifcsundef{date\CurrentTrackedDialect}
{%
  \ifundef\datedutch
  {% do nothing
  }%
  {%
    \def\datedutch{%
      \DTMifcaseregional
      {}% do nothing
      {\DTMsetstyle{dutch}}%
      {\DTMsetstyle{dutch-numeric}}%
    }%
  }%
}%
{%
  \csdef{date\CurrentTrackedDialect}{%
    \DTMifcaseregional
    {}% do nothing
    {\DTMsetstyle{dutch}}%
    {\DTMsetstyle{dutch-numeric}}
  }%
}%
%    \end{macrocode}
%\iffalse
%    \begin{macrocode}
%</datetime2-dutch.ldf>
%    \end{macrocode}
%\fi
%\Finale
\endinput
