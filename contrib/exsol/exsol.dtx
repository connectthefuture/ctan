% \iffalse meta-comment
%
% Copyright (C) 2016 by Walter Daems <walter.daems@uantwerpen.be>
%
% This work may be distributed and/or modified under the conditions of
% the LaTeX Project Public License, either version 1.3 of this license
% or (at your option) any later version.  The latest version of this
% license is in:
% 
%    http://www.latex-project.org/lppl.txt
% 
% and version 1.3 or later is part of all distributions of LaTeX version
% 2005/12/01 or later.
%
% This work has the LPPL maintenance status `maintained'.
% 
% The Current Maintainer of this work is Walter Daems.
%
% This work consists of the files exsol.dtx and exsol.ins and the derived 
% files:
%   - exsol.sty
%   - example.tex
%   - example-solutionbook.tex
%   - example-formulacollection.tex
%   - example-local.tex
% \fi
%
% \iffalse
%
%<package|driver>\NeedsTeXFormat{LaTeX2e}
%<driver>\ProvidesFile{exsol.dtx}
%<package>\ProvidesPackage{exsol}
%<package|driver>  [2016/05/13 v1.2 ExSol - Exercises and Solutions package (DMW)]
%<*driver> 
\documentclass[10pt,a4paper]{ltxdoc}
\usepackage[english]{babel}
\usepackage[exercisesfontsize=small]{exsol}
\usepackage{metalogo}
\EnableCrossrefs
\CodelineIndex
\RecordChanges
\usepackage{makeidx}
\usepackage{alltt}
\IfFileExists{tocbibind.sty}{\usepackage{tocbibind}}{}
\IfFileExists{hyperref.sty}{\usepackage[bookmarksopen]{hyperref}}{}
\EnableCrossrefs         
\CodelineIndex
\RecordChanges
\newcommand{\exsol}{\textsc{ExSol}}
\StopEventually{\PrintChanges\PrintIndex}
\def\fileversion{1.2}
\def\filedate{2016/05/05}
\begin{document}
 \DocInput{exsol.dtx}
\end{document}
%</driver>
% \fi
%
% \CheckSum{0}
%
% \CharacterTable
%  {Upper-case    \A\B\C\D\E\F\G\H\I\J\K\L\M\N\O\P\Q\R\S\T\U\V\W\X\Y\Z
%   Lower-case    \a\b\c\d\e\f\g\h\i\j\k\l\m\n\o\p\q\r\s\t\u\v\w\x\y\z
%   Digits        \0\1\2\3\4\5\6\7\8\9
%   Exclamation   \!     Double quote  \"     Hash (number) \#
%   Dollar        \$     Percent       \%     Ampersand     \&
%   Acute accent  \'     Left paren    \(     Right paren   \)
%   Asterisk      \*     Plus          \+     Comma         \,
%   Minus         \-     Point         \.     Solidus       \/
%   Colon         \:     Semicolon     \;     Less than     \<
%   Equals        \=     Greater than  \>     Question mark \?
%   Commercial at \@     Left bracket  \[     Backslash     \\
%   Right bracket \]     Circumflex    \^     Underscore    \_
%   Grave accent  \`     Left brace    \{     Vertical bar  \|
%   Right brace   \}     Tilde         \~}
%
%
% \changes{v0.1}{2012/01/05}{\@ Initial version}
% \changes{v0.2}{2012/01/06}{\@ Minor bug fixes based on first use by
% Paul Levrie}
% \changes{v0.3}{2012/01/07}{\@ Minor bug fixes based on second use by
% Paul}
% \changes{v0.4}{2012/01/09}{\@ Allowed for non-list formatting of
% exercises (as default)}
% \changes{v0.5}{2012/01/15}{\@ Added option to also send exercises to
% solutions file}
% \changes{v0.6}{2013/05/12}{\@ Prepared for CTAN publication}
% \changes{v0.7}{2014/07/14}{\@ Fixed UTF8 compatibility issues}
% \changes{v0.8}{2014/07/15}{\@ Fixed missing babel tag and running out
% of write handles}
% \changes{v0.9}{2014/07/28}{\@ Changed default behavior
% w.r.t. minipage-wrapping of exercises} 
% \changes{v0.91}{2014/08/31}{\@ Corrected minipage dependence, made }
% \changes{v1.0}{2015/09/16}{\@ First stable release}
% \changes{v1.1}{2015/12/30}{\@ Bugfix release}
% \changes{v1.2}{2016/05/05}{\@ Implemented multicolumn option for
% exercises, introduced new options (local, nolabels), separate
% counter for exerciseseries (only for use local mode), introduced
% local mode}
%
% \DoNotIndex{\newcommand,\newenvironment}
% \setlength{\parindent}{0em}
% \addtolength{\parskip}{0.5\baselineskip}
%
% \title{The \exsol{} package\thanks{This document
%   corresponds to exsol~\fileversion, dated \filedate.}}
% \author{Walter Daems (\texttt{walter.daems@uantwerpen.be})}
% \date{}
%
% \maketitle
%
% \section{Introduction}
% %%%%%%%%%%%%%%%%%%%%%%
%
% \subsection{Package goal}
% The package \exsol{} provides macros to allow
% embedding exercises and solutions in the \LaTeX{} source of an
% instructional text (e.g., a book or a course text) while keeping the
% exercises and the solutions apart in the typeset result.
% 
% In \emph{global mode} (the default), this corresponds to generating
% the following separate documents:
% \begin{itemize}
% \item your original text that only contains the exercises, and
% \item a solution book that only contains the solutions to the
% exercises (a package option exists to also copy the exercises
% themselves to the solution book). 
% \end{itemize}
% The former is generated when running \LaTeX{} on your document. This
% run writes the solutions to a secondary file that can be included
% into a simple document harness, such that when running \LaTeX{} on
% the latter, you can generate a nice solution book.
%
% In \emph{local mode} (invoked by specifying the package option 'local'),
% this corresponds to reading in the saved solutions in the current
% document at a later stage in the text.
% 
% \subsection{Why use \exsol{}?}
% \begin{itemize}
% \item It allows to keep the \LaTeX{} source of your exercises and their
% solutions together in a single file, next to each other. Away with
% the nightmare to keep your solutions in sync with the exercises.
% \item It separates exercises and solutions, allowing you
%   \begin{itemize}
%   \item to only release the solution book to the instructors of the
%     course (using global mode);
%   \item to encourage students that you provide with the solutions
%   separated from the exercises to first try solving the exercises
%   without peeking into the list of solutions (or the solution book).
%   \end{itemize}
% \end{itemize}
% 
% \subsection{Credits}
% The code of the \exsol{} package was taken almost literally
% from \textsf{fancyvrb} \cite{fancyvrb}. Therefore, all credits go to
% the authors/maintainers of \textsf{fancyvrb}.
%
% Thanks to Paul Levrie, Pieter Pareit, Pekka Pere, Benjamin
% Grinstein and Philippe Marti for signaling problems and making
% suggestions for the improvement of the package and the
% documentation.
%
% \subsection{Note}
% As of version 1.2, the package also contains facilities for
% generating a formula collection. The only limitation is that
% exercises and solutions cannot be part of the formula collection.
% Though this is a restriction, it is not a severe one in my
% opinion. Formula collections should be as concise as they can be.
% 
% Though generating formula collections goes beyond exercises and
% solutions, I chose not to change the package's name
% \exsol{}. Einstein's name was also not changed into Relativistic
% Einstein when he got to understand the theory of relativity.
%
% \section{Installation}
% %%%%%%%%%%%%%%%%%%%%%%%%
% Either you are a package manager and then you'll know how to
% prepare an installation package for \exsol{}.
%
% Either you are a normal user and then you have two options. First,
% check if there is a package that your favorite \LaTeX{}
% distributor has prepared for you. Second, grab the TDS package
% from CTAN \cite{CTAN} (|exsol.tds.zip|) and unzip it
% somewhere in your own TDS tree, regenerate your filename database
% and off you go. 
% In any case, make sure that \LaTeX{} finds the |exsol.sty|
% file. 
%
% The \exsol{} package uses some auxiliary packages:
% \textsf{fancyvrb}, 
% \textsf{ifthen}, \textsf{kvoptions} and, optionally,
% \textsf{babel}. Fetch them from CTAN \cite{CTAN} if your \TeX{}
% distributor does not provide them.
%
% \section{Usage}
% %%%%%%%%%%%%%%%%%
% 
% \subsection{Preparing your document source}
% %%%%%%%%%%%%%%%%%%%%%%%%%%%%%%%%%%%%%%%%%%%%%%%
%
% \subsubsection{Loading the package}
% The macro package exsol can be loaded with:
% \begin{verbatim}
% \usepackage{exsol}
% \end{verbatim}
%
% Your first choice to make is where you want your solutions to
% appear. The primary objective of the exsol package was 'global
% mode', i.e. separating your solutions from the exercises,
% gathering the solutions in a separate book.
% To this end, don't specify the package option '|[local]|', or
% specify '|[local=false]|'. 
%
% A second mode of operation is 'local mode'. This allows grouping
% your exercises in series and including them later in your text.
% Gathering exercises in (numbered) series and 'loading' them locally
% in your text, allows for simplifying the
% individual exercise numbers (omitting their prefix containing,
% chapter number, section number, subsection number a.s.o.). 
%
% \subsubsection{Global mode - flat grouping exercises}
% Adding exercises together with their solutions in your document is
% easy. Just embed them in a
% |exercise| and a corresponding |solution| environment.
% Optionally, you may embed several of them in a |exercises|
% environment to make them stand out in your text.
%
% \begin{verbatim}
% \begin{exercises}[columns=2]
%
%   \begin{exercise}
%     Calculate $y = 5 + 7$
%   \end{exercises}
%   \begin{solution}
%     $y = 12$
%   \end{solution}
%
%   \begin{exercise}
%     Calculate $y = 7 - 12$
%   \end{exercises}
%   \begin{solution}
%     $y = -5$
%   \end{solution}
%
% \end{exercises}
% \end{verbatim}
% 
% The optional argument of the exercises environment allows specifying
% the typesetting in multiple columns.
%
% On how to generate a solution book, take a look at the examples in
% section~\ref{examplesglobal}.
%
% \subsubsection{Local mode - grouping of exercises in series}
% One might also consider to keep the solutions in the same text, in
% local mode. In this case, we advise to gather the exercises in
% series (e.g. according to their degree of difficulty). This can be
% done by using the |exerciseseries| environment instead of the
% |exercises| environment.
% This environment takes and also takes a
% mandatory label argument. 
% In addition it takes two optional arguments:
% \begin{itemize}
% \item |columns| to specify the amount of clumns 
% \item |exsubrule| to specify the creation of a horizontal rule below
% the exercise series.
% \item |solsubrule| to specify the creation of a horizontal rule below
% the solution series.
% \item |subrule| to specify the creation of a horizontal rule below
% the exercise and the solution series.
% \end{itemize}
% 
% 
% \begin{verbatim}
% \begin{exerciseseries}[columns=2,subrule=\hrule]{Easy exercises}
%
%   \begin{exercise}
%     Calculate $y = 5 + 7$
%   \end{exercises}
%   \begin{solution}
%     $y = 12$
%   \end{solution}
%
%   \begin{exercise}
%     Calculate $y = 7 - 12$
%   \end{exercises}
%   \begin{solution}
%     $y = -5$
%   \end{solution}
%
% \end{exerciseseries}
%
% \begin{exerciseseries}{Difficult exercises}
%
%   \begin{exercise}
%     Calculate $y = 5 \cdot 7$
%   \end{exercises}
%   \begin{solution}
%     $y = 35$
%   \end{solution}
%
%   \begin{exercise}
%     Calculate $y = 8 / 4$
%   \end{exercises}
%   \begin{solution}
%     $y = 2$
%   \end{solution}
%
% \end{exerciseseries}
% \end{verbatim}
%
%
%
% \subsection{Examples}
%
% \subsubsection{Global mode}
% \label{examplesglobal}
% Below, you can find an example of a file that contains a number of
% exercises and solutions, with the goal to generate a separate
% solution book.
%
% \begin{Verbatim}[fontsize=\footnotesize]
%<*example>
\documentclass[a4paper,10pt]{article}

\usepackage{a4wide}
\usepackage[english]{babel}
\usepackage[copyexercisesinsolutions]{exsol}

\title{Gobal example, from the \textsf{ExSol} package}
\author{Walter Daems}
\setlength{\parindent}{0em}
\begin{document}
\maketitle

\section{Introduction}

In this text we explain how to wolve second-order polynomial
equations. 

\section{Solving second-order polynomial equations}

\begin{informulacollectiononly}
\section*{Solving second-order polynomial equations}
\end{informulacollectiononly}
\begin{informulacollection}
  The roots of the following equation
  \begin{equation}
    a x^2 + bx + c = 0
  \end{equation}
  can be determined as:
  \begin{equation}
    x_{1,2} = \frac{-b \pm \sqrt{b^2 - 4 a c}}{2 a}
  \end{equation}
\end{informulacollection}

\begin{exercises}[columns = 2]

  \begin{exercise}
    Solve the following equation for $x \in C$, with $C$ the set of
    complex numbers:
    \begin{equation}
      5 x^2 -3 x = 5
    \end{equation}
  \end{exercise}
  \begin{solution}
    Let's start by rearranging the equation, a bit:
    \begin{eqnarray}
      5.7 x^2 - 3.1 x &=& 5.3\\
      5.7 x^2 - 3.1 x -5.3 &=& 0
    \end{eqnarray}
    The equation is now in the standard form:
    \begin{equation}
      a x^2 + b x + c = 0
    \end{equation}
    For quadratic equations in the standard form, we know that two
    solutions exist:
    \begin{equation}
      x_{1,2} = \frac{ -b \pm \sqrt{d}}{2a}
    \end{equation}
    with
    \begin{equation}
      d = b^2 - 4 a c
    \end{equation}
    If we apply this to our case, we obtain:
    \begin{equation}
      d = (-3.1)^2 - 4 \cdot 5.7 \cdot (-5.3) = 130.45
    \end{equation}
    and
    \begin{eqnarray}
      x_1 &=& \frac{3.1 + \sqrt{130.45}}{11.4} = 1.27\\
      x_2 &=& \frac{3.1 - \sqrt{130.45}}{11.4} = -0.73
    \end{eqnarray}
    The proposed values $x = x_1, x_2$ are solutions to the given equation.
  \end{solution}
  \begin{exercise}
    Consider a 2-dimensional vector space equipped with a Euclidean
    distance function. Given a right-angled triangle, with the sides
    $A$ and $B$ adjacent to the right angle having lengths, $3$ and
    $4$, calculate the length of the hypotenuse, labeled $C$.
  \end{exercise}
  \begin{solution}
    This calls for application of Pythagoras' theorem, which 
    tells us:
    \begin{equation}
      \left\|A\right\|^2 + \left\|B\right\|^2 = \left\|C\right\|^2
    \end{equation}
    and therefore:
    \begin{eqnarray}
      \left\|C\right\| 
      &=& \sqrt{\left\|A\right\|^2 + \left\|B\right\|^2}\\
      &=& \sqrt{3^2 + 4^2}\\
      &=& \sqrt{25} = 5
    \end{eqnarray}
    Therefore, the length of the hypotenuse equals $5$.
  \end{solution}

\end{exercises}

\end{document}
%</example>
%\end{Verbatim}
%
% The result in the original document, can be seen in the file
% |example.pdf|. As you can see, the formulacollection entry and the
% exercise appear. There's no trace of the solution.
%
% When running \LaTeX{} on your document (e.g., the file example.tex
% as part of the exsol package, as a side effect files with the
% extension |.sol.tex| and |.fc.tex| have been written
% to disk (in our case, the files |example.sol.tex| and
% |example.fc.tex|), containing all solutions and entries for
% the formula collection in sequence.
%
% Generating a solution book is as simple as including the solution
% file into a simple \LaTeX{} harness, that allows you giving it a
% proper title page and to add other bells and whistles.
% This can be seen below:
% 
% \begin{Verbatim}[fontsize=\footnotesize]
%<*examplesol>
\documentclass[a4paper,10pt]{article}

\usepackage{a4wide}
\usepackage[english]{babel}
\usepackage{exsol}

\setlength{\parindent}{0em}

\title{Solutions to the exercises, specified in the example of the
  \textsf{ExSol} package}
\author{Walter Daems}

\begin{document}

\maketitle

\input{example.sol.tex}

\end{document}
%</examplesol>
% \end{Verbatim}
%
% Generating a formula collection is as simple as including the
% formula collectoin file into a simple \LaTeX{} harness, that allows
% you giving it a proper title page and to add other bells and
% whistles.
% This can be seen below:
% 
% \begin{Verbatim}[fontsize=\footnotesize]
%<*examplefor>
\documentclass[a4paper,10pt]{article}

\usepackage{a4wide}
\usepackage[english]{babel}
\usepackage{exsol}

\setlength{\parindent}{0em}

\title{Formula collection, specified in the example of the
  \textsf{ExSol} package}
\author{Walter Daems}

\begin{document}

\maketitle

\input{example.fc.tex}

\end{document}
%</examplefor> 
% \end{Verbatim}
%
% \subsubsection{Local mode}
%
% Below, you can find an example of a file that contains a number of
% exercises and solutions, with the goal to include them later in the
% same document.
% This is done by issuing the |loadSolutions| macro at the very end of
% the file. This macro can be called multiple times and will gobble up
% and include all solutions so far. As a consequence the solutions
% file is emptied and can be filled again by specifying new
% exercise/solution pairs.
%
%
% \begin{Verbatim}[fontsize=\footnotesize]
%<*example-local>
\documentclass[a4paper,10pt]{article}

\usepackage{a4wide}
\usepackage[german]{babel}
\usepackage[local,nolabels,exerciseaslist]{exsol}
\usepackage{enumitem}

\setlength{\exsolexercisetopbottomsep}{0pt plus 0pt minus 1pt}
\setlength{\exsolexerciseleftmargin}{2em}
\setlength{\exsolexerciserightmargin}{1em}
\setlength{\exsolexerciseparindent}{0em}
\setlength{\exsolexerciselabelsep}{1ex}
\setlength{\exsolexerciselabelwidth}{30pt}
\setlength{\exsolexerciseitemindent}{0pt}
\setlength{\exsolexerciseparsep}{\parskip}

\title{Local example, from the \textsf{ExSol} package}
\author{Philippe Marti}
\setlength{\parindent}{0em}
\begin{document}
\maketitle

\section{\exercisesname}

\begin{exerciseseries}[columns=2,solsubrule=\hrule]{Gleichungen $|$ Gleichungssysteme}
  
  \begin{exercise}
    Die Summe zweier Zahlen ist 17 und ihre Differenz 7. Bestimme die
    beiden Zahlen! 
  \end{exercise}
  \begin{solution}
    5 und 12
  \end{solution}
  
  \begin{exercise}
    Die Differenz einer Zahl und dem Dreifachen einer zweiten Zahl ist
    14. Bestimme die beiden Zahlen, falls die zweite Zahl ein Zehntel
    der ersten ist. 
  \end{exercise}
  \begin{solution}
    20 und 2
  \end{solution}
  
\end{exerciseseries}

\begin{exerciseseries}[columns=2]{Geraden}
  \begin{exercise}
    Berechne den Schnittpunkt von \mbox{$y=3x+1$} und \mbox{$y=3x-7$}.
  \end{exercise}
  \begin{solution}
    Es gibt keinen Schnittpunkt
  \end{solution}

  \begin{exercise}
    Die Familie Meier fordert Offerten f\"ur eine Heizungsreparatur
    ein. Firma A berechnet f\"ur die Fahrtkosten Fr. 42.- und f\"ur
    jede Arbeitsstunde 76.-. Bei der Firma B sind die Fahrtkosten
    Fr. 35.- und jede Arbeitsstunde wird mit Fr. 80.- berechnet. 
    \begin{enumerate}[label=\alph*)]
    \item Welche Kosten entstehen f\"ur beide Firmen, wenn ein Monteur
      3.5 Stunden  
      f\"ur die Arbeit benötigt? Welche Firma ist in diesem Fall
      kostengünstiger? 
      
    \item Wie lauten die Gleichungen derjenigen zwei linearen
      Funktionen, die jeder Arbeitszeit $x$ (in Stunden) die
      entstehenden Kosten $y$ (in Franken) zuordnet? 
      
    \item Berechne, bei welcher Arbeitszeit die Kosten bei beiden
      Firmen gleich sind.
    \end{enumerate}
  \end{exercise}
  \begin{solution}
    \begin{enumerate}[label=\alph*)]
    \item Firma A: 308.- $|$ Firma B: 315.-
    \item A: $y=76x+42$ $|$ B: $y=80x+35$
    \item Bei $1\frac{3}{4}$ Stunden
    \end{enumerate}
  \end{solution}
\end{exerciseseries}

\section{\solutionsname}
\loadSolutions

\end{document}
%</example-local>
% \end{Verbatim}
%
% \subsection{Fiddling with the spacing}
%
% The default spacing provided by the \textsf{ExSol} package should be
% fine for most users. However, if you like to tweak, below you can
% find the controls.
% 
% As the formula collection generation is intended to interfere as
% little as possible with the flow of the main document, you will
% notice that it is impossible to control the extra spacing, as no
% extra spacing should originate from using the
% |informulacollection| environment.
%
% \subsubsection{Spacing before and after the \texttt{exercises} environment}
%
% The lengths below control the spacing of the |exercises| environment:
% \begin{itemize}
% \item |exsolexerciseaboveskip|: rubber length controlling the
% vertical space after the top marker line of the environment
% \item |exsolexercisebelowskip|: rubber length controlling the
% vertical space before the bottom marker line of the environment
% \end{itemize}
%
% You can simply specify them like:
% \begin{verbatim}
% \setlength{\exsolexercisesaboveskip}{1ex plus 1pt minus 1pt}
% \setlength{\exsolexercisesbelowskip}{1ex plus 1pt minus 1pt}
% \end{verbatim}
% The spacings specified here are the package defaults.
%
% \subsubsection{Spacing of the individual exercises}
% Caution: the spacing can only be tuned, when one invokes the
% |exerciseaslist| package option!
%
% Then lengths below control the spacing of the |exercise| environment:
% \begin{itemize}
% \item |exercisetopbottomsep|: rubber length controlling the vertical
% space before and after individual exercises
% \item |exerciseleftmargin|: length controlling the horizontal
% space between the surrounding environment's left margin (most
% often the page margin) and the left edge of the exercise
% environment 
% \item |exerciseleftmargin|: length controlling the horizontal
% space between the surrounding environment's right margin (most
% often the page margin) and the right edge of the exercise
% environment
% \item |exerciseitemindent|: length controlling the first-line
% indentation of the first paragraph in the exercise environment
% (actually, the label is set w.r.t. this position, that we will
% conveniently call position 'x')
% \item |exerciseparindent|: length controlling the first-line
% indentation of the other paragraphs in the exercise environment.
% \item |exerciselabelsep|: length controlling the distance between
% the label and position 'x'
% \item |exerciselabelwidth|: minimal width of the (internally
% right-alligned) box to use for the exercises label; if the box is
% not sufficiently big, position 'x' is shifted to the right
% \item |exerciseparsep|: internal paragraph separation (vertically)
% \end{itemize}
% 
% You can simply specify them like:
% \begin{verbatim}
% \setlength{\exsolexercisetopbottomsep}{0pt plus 0pt minus 1pt}
% \setlength{\exsolexerciseleftmargin}{1em}
% \setlength{\exsolexerciserightmargin}{1em}
% \setlength{\exsolexerciseparindent}{0em}
% \setlength{\exsolexerciselabelsep}{0.5em}
% \setlength{\exsolexerciselabelwidth}{0pt}
% \setlength{\exsolexerciseitemindent}{0pt}
% \setlength{\exsolexerciseparsep}{\parskip}
% \end{verbatim}
% The spacings specified here are the package defaults.
%
% \subsection{Tips and tricks}
%
% If you want to include the solutions all at the
% end of the current document in global mode, you need to explicitly
% close the solution stream before including it:
% \begin{verbatim}
%   \closeout\solutionstream\input{\jobname.sol.tex}
% \end{verbatim}
%
% If you want to avoid exercises being split by a page boundary, then
% provide the package option 'minipage'. This causes the exercises to
% be wrapped in a minipage environment.
% 
% \clearpage
%
% \section{Implementation}
% %%%%%%%%%%%%%%%%%%%%%%%
%    \begin{macrocode}
%<*package>
%    \end{macrocode}
%
% \subsection{Auxiliary packages}
% %%%%%%%%%%%%%%%%%%%%%%%%%%%%%%%
% The package uses some auxiliary packages:
%    \begin{macrocode}
\RequirePackage{ifmtarg}
\RequirePackage{fancyvrb}
\RequirePackage{ifthen}
\RequirePackage{kvoptions}
\RequirePackage{multicol}
%    \end{macrocode}
%
% \subsection{Package options}
% %%%%%%%%%%%%%%%%%%%%%%%%%%%%
% The package offers some options:
%
% \changes{v1.2}{2016/05/05}{Added option local}
% \begin{macro}{local}
%  This boolean option (true, false) allows setting the mode of the
%  package into local, i.e. that the numbering of the exercises is not
%  related to the position in the document, but uses it's own local
%  counter in combination with a exerciseseries counter.
%    \begin{macrocode}
\DeclareBoolOption[false]{local}
%    \end{macrocode}
% \end{macro}
%
% \changes{v1.2}{2016/05/05}{Added option nolabels}
% \begin{macro}{nolabels}
%  This boolean option (true, false) allows suppressing the 'Exercise'
%  and 'Solution' label that normally appear before the exercise and
%  solution number.
%    \begin{macrocode}
\DeclareBoolOption[false]{nolabels}
%    \end{macrocode}
% \end{macro}
%
% \changes{v0.2}{2012/01/06}{Added option exercisesfont}
% \changes{v0.4}{2012/01/09}{Changed name of option to
% exercisesfontsize}
%
% \begin{macro}{exercisesfontsize}
%  This option allows setting the font of the |exercises|
%  environment. You may chopse one of tiny, scriptsize, footnotesize,
%  small, normalsize, large, etc.\\
%  E.g., |[exercisesfontsize=small]|.
%    \begin{macrocode}
\DeclareStringOption[normalsize]{exercisesfontsize}
%    \end{macrocode}
% \end{macro}
%
% \changes{v0.4}{2012/01/06}{Added option exercisesinlist}
% \changes{v0.5}{2012/01/09}{Changed option exercisesinlist to exerciseaslist}
%
% \begin{macro}{exerciseaslist}
%  This boolean option (true, false) allows setting the typesetting of
%  the |exercises| in a list environment. This causes the
%  exercises to be typeset in a more compact fashion, with indented
%  left and right margin. 
%    \begin{macrocode}
\DeclareBoolOption[false]{exerciseaslist}
%    \end{macrocode}
% \end{macro}
%
% \changes{v0.5}{2012/01/09}{Added option copyexercisesinsolutions}
% \begin{macro}{copyexercisesinsolutions}
%  This boolean option (true, false) allows copying the exercises in
%  the solutions file, to allow for making a complete stand-alone
%  exercises bundle.
%    \begin{macrocode}
\DeclareBoolOption[false]{copyexercisesinsolutions}
%    \end{macrocode}
% \end{macro}
%
% \changes{v0.9}{2014/07/28}{. Changed default behavior
% w.r.t. minipage-wraping of exercises}
% \begin{macro}{minipage}
%  This boolean option (true, false) causes the exercises to be
%  wrapped in minipages. This avoids them getting split by a page
%  boundary.
%    \begin{macrocode}
\DeclareBoolOption[false]{minipage}
%    \end{macrocode}
% \end{macro}
%
% The options are processed using:
%    \begin{macrocode}
\ProcessKeyvalOptions*
%    \end{macrocode}
%
% The options are subsequently handled
%    \begin{macrocode}
\newcommand{\exercisesfontsize}{\csname \exsol@exercisesfontsize\endcsname}
%    \end{macrocode}
%
%
% \subsection{Customization of lengths}
% %%%%%%%%%%%%%%%%%%%%%%%%%%%%%%%%%%%%%%%
% The commands below allow customizing many lengths that control the
% typesetting of the exercises.
%
% \changes{v0.91}{2014/08/31}{added user-accessible lengths}
% First some lengths to control the spacing before and after |exercises|.
%    \begin{macrocode}
\newlength{\exsolexercisesaboveskip}
\setlength{\exsolexercisesaboveskip}{0ex plus 1pt minus 1pt}
\newlength{\exsolexercisesbelowskip}
\setlength{\exsolexercisesbelowskip}{0ex plus 1pt minus 1pt}
%    \end{macrocode}
%
% Then some lengths to control the spacing for a single
% exercise. These lengths only work when the |exerciseaslist| package
% option has been specified. Sensible defaults have been set.
%    \begin{macrocode}
\newlength{\exsolexercisetopbottomsep}
\setlength{\exsolexercisetopbottomsep}{0pt plus 0pt minus 1pt}
\newlength{\exsolexerciseleftmargin}
\setlength{\exsolexerciseleftmargin}{1em}
\newlength{\exsolexerciserightmargin}
\setlength{\exsolexerciserightmargin}{1em}
\newlength{\exsolexerciseparindent}
\setlength{\exsolexerciseparindent}{0em}
\newlength{\exsolexerciselabelsep}
\setlength{\exsolexerciselabelsep}{0.5em}
\newlength{\exsolexerciselabelwidth}
\setlength{\exsolexerciselabelwidth}{0pt}
\newlength{\exsolexerciseitemindent}
\setlength{\exsolexerciseitemindent}{0pt}
\newlength{\exsolexerciseparsep}
\setlength{\exsolexerciseparsep}{\parskip}
%    \end{macrocode}
% 
% 
% \subsection{Con- and destruction of the auxiliary streams}
% %%%%%%%%%%%%%%%%%%%%%%%%%%%%%%%%%%%%%%%%%%%%%%%%%%%%%%%%%%
% At the beginning of your document, we start by opening a stream to a
% file that will be used to write the solutions to. At the end of your
% document, the package closes the stream.
% \changes{v0.8}{2014/07/15}{moved newwrite of exercise stream to this
% spot to avoid consuming all handles}
%    \begin{macrocode}
\AtBeginDocument{
  \typeout{Writing solutions to solution file \jobname.sol.tex}
  \newwrite\solutionstream
  \immediate\openout\solutionstream=\jobname.sol.tex
  \typeout{Using intermediate exercise file \jobname.exc.tex}
  \newwrite\exercisestream
  \typeout{Writing formulae to formula collection file \jobname.fc.tex}
  \newwrite\formulacollectionstream
  \immediate\openout\formulacollectionstream=\jobname.fc.tex
  \typeout{Using intermediate formula file \jobname.for.tex}
  \newwrite\formulastream
}
\AtEndDocument{
  \immediate\closeout\formulacollectionstream
  \immediate\closeout\solutionstream
}
%    \end{macrocode}
%
%
% In local mode we also want to close the solutionstream, read it and
% open it again:
%    \begin{macrocode}
\newcommand\loadSolutions{
  \immediate\closeout\solutionstream
  \input{\jobname.sol.tex}
%  \immediate\openout\solutionstream=\jobname.sol.tex
}
%    \end{macrocode}
%
% \subsection{Series counter}
% %%%%%%%%%%%%%%%%%%%%%%%%%%%%%%
% By providing an exerciseseries counter, proper numbering of the
% exercise series is provided. Note that separate series, render the
% numbering of the exercises from document-global, to series-local,
% therefore hindering the concordance of solutions to exercises. 
%
% \changes{v1.2}{2016/05/05}{Added series counter}
%    \begin{macrocode}
\newcounter{exerciseseries}[subsubsection]
\setcounter{exerciseseries}{0}
\renewcommand{\theexerciseseries}{\arabic{exerciseseries}}
%    \end{macrocode}
%
%
% \subsection{Exercise counter}
% %%%%%%%%%%%%%%%%%%%%%%%%%%%%%%
% By providing an exercise counter, proper numbering of the exercises
% is provided to allow for good cross referencing of the solutions to
% the exercises.
% \changes{v0.2}{2012/01/06}{Removed dash in counter when in document
% without sectioning commands}
%    \begin{macrocode}
\newcounter{exercise}[exerciseseries]
\setcounter{exercise}{0}
\renewcommand{\theexercise}{%
  \ifexsol@local
  \arabic{exerciseseries}.\arabic{exercise}%
  \else
  \@ifundefined{c@chapter}{}{\if0\arabic{chapter}\else\arabic{chapter}.\fi}%
  \if0\arabic{section}\else\arabic{section}\fi%
  \if0\arabic{subsection}\else.\arabic{subsection}\fi%
  \if0\arabic{subsubsection}\else.\arabic{subsubsection}\fi%
  \if0\arabic{exercise}\else%
  \@ifundefined{c@chapter}%
  {\if0\arabic{section}\else-\fi}%
  {-}%
  \arabic{exercise}%
  \fi
  \fi
}
%    \end{macrocode}
%
%
% \subsection{Detokenization in order to cope with utf8}
%
% Combining old-school \LaTeX{} (before \XeTeX{} and \LuaTeX{}) and
% UTF-8 is a pain.
% Detokenization has been suggested by Geoffrey Poore to solve issues
% with UTF-8 characters messing up the |fancyvrb| internals.
% \changes{v0.7}{2014/07/14}{Added detokenized writing}
% \changes{v1.0}{2014/09/13}{Added detokenized writing of formula and
% formula collection stream}
%    \begin{macrocode}
\newcommand{\GPES@write@detok}[1]{%
  \immediate\write\exercisestream{\detokenize{#1}}}%
\newcommand{\GPSS@write@detok}[1]{%
  \immediate\write\solutionstream{\detokenize{#1}}}%
\newcommand{\GPESS@write@detok}[1]{%
  \GPES@write@detok{#1}%
  \GPSS@write@detok{#1}}%
\newcommand{\GPFORCOL@write@detok}[1]{%
  \immediate\write\formulacollectionstream{\detokenize{#1}}%
  \immediate\write\formulastream{\detokenize{#1}}}%
\newcommand{\GPFORCOLONLY@write@detok}[1]{%
  \immediate\write\formulacollectionstream{\detokenize{#1}}}%
%    \end{macrocode}
%
%
% \section{The user environments}
%
% \begin{macro}{exercise}
%   The |exercise| environment is used to typeset your
%   exercises, provide them with a nice label and allow for copying
%   the exercise to the solutions file (if the package option
%   |copyexercisesinsolution|) is set. The label can be
%   set by redefining the \cs{exercisename} macro, or by relying on
%   the \textsf{Babel} provisions. The code is almost litteraly
%   taken from the \textsf{fancyvrb} package.
%    \begin{macrocode}
\def\exercise{\FV@Environment{}{exercise}}
\def\FVB@exercise{%
  \refstepcounter{exercise}%
  \immediate\openout\exercisestream=\jobname.exc.tex
  \ifexsol@local
%  \immediate\write\solutionstream{}
  \else
%  \immediate\write\solutionstream{\string\vspace*\string{2ex\string}%
%    \string\quad\string\newline}
  \fi
  \ifexsol@copyexercisesinsolutions
  % WDSC tofix
  % \typeout{Writing exercise to \jobname.sol.tex}
  \immediate\write\solutionstream{\string\begin{exsol@exercise}{\theexercise}}
  \fi
  \immediate\write\exercisestream{\string\begin{exsol@exercise}{\theexercise}}
  \@bsphack
  \begingroup
    \FV@UseKeyValues
    \FV@DefineWhiteSpace
    \def\FV@Space{\space}%
    \FV@DefineTabOut
    \ifexsol@copyexercisesinsolutions
      \let\FV@ProcessLine\GPESS@write@detok %
    \else
      \let\FV@ProcessLine\GPES@write@detok %
    \fi
    \relax
    \let\FV@FontScanPrep\relax
    \let\@noligs\relax
    \FV@Scan
  }
\def\FVE@exercise{
  \endgroup\@esphack
  \immediate\write\exercisestream{\string\end{exsol@exercise}}
  \ifexsol@copyexercisesinsolutions
    \immediate\write\solutionstream{\string\end{exsol@exercise}}
  \fi
  \immediate\closeout\exercisestream
  \input{\jobname.exc.tex}
}
\DefineVerbatimEnvironment{exercise}{exercise}{}
%    \end{macrocode}
% \end{macro}
%
% \begin{macro}{exsol@exercise}
%   The |exsol@exercise| environment is an internal macro used
%   to typeset your exercises and provide them with a nice label and
%   number. Do not use it directly. Use the proper environment
%   |exercise| instead.
%   \changes{v0.2}{2012/01/06}{Attempted to fix MiKTeX formatting problems}
%   \changes{v0.3}{2012/01/08}{Fixed labelsep to avoid cluttered
%   itemize environments}
%   \changes{v0.4}{2012/01/06}{Added option exercisesinlist such that
%   default results in non list formatting of exercise}
%   \changes{v0.5}{2012/01/09}{Changed implementation to allow for
%   copying the exercises to the solutions file.}
%    \begin{macrocode}
\newenvironment{exsol@exercise}[1]
{%
  \ifthenelse{\boolean{exsol@minipage}}{\begin{minipage}[t]{\textwidth}}{}%
    \ifthenelse{\boolean{exsol@exerciseaslist}}
               {\begin{list}%
                   {%
                   }%
                   {%
                     \setlength{\topsep}{\exsolexercisetopbottomsep}%
                     \setlength{\leftmargin}{\exsolexerciseleftmargin}%
                     \setlength{\rightmargin}{\exsolexerciserightmargin}%
                     \setlength{\listparindent}{\exsolexerciseparindent}%
                     \setlength{\itemindent}{\exsolexerciseitemindent}%
                     \setlength{\parsep}{\exsolexerciseparsep}
                     \setlength{\labelsep}{\exsolexerciselabelsep}
                     \setlength{\labelwidth}{\exsolexerciselabelwidth}}
                 \item[\ifexsol@nolabels~#1:\else%
                   \subparagraph{\exercisename{}~#1:}\fi]
               }%
               {\ifexsol@nolabels #1:\else%
                 \subparagraph{\exercisename{}~#1:}\fi}
}
{%
  \ifthenelse{\boolean{exsol@exerciseaslist}}%
             {\end{list}}{}%
  \ifthenelse{\boolean{exsol@minipage}}{\end{minipage}}{\par}%
}
%    \end{macrocode}
% \end{macro}
%
%
% \begin{macro}{solution}
%   The |solution| environment is used to typeset your solutions
%   and provide them with a nice label and number that corresponds to
%   the exercise that preceeded this solution. Theno label can be
%   set by redefining the \cs{solutionname} macro, or by relying on
%   the \textsf{Babel} provisions. The code is almost litteraly
%   taken from the \textsf{fancyvrb} package.
%    \begin{macrocode}
\def\solution{\FV@Environment{}{solution}}
\def\FVB@solution{%
  %\typeout{Writing solution to \jobname.sol.tex}
  \ifexsol@copyexercisesinsolutions
  \immediate\write\solutionstream{\string\begin{exsol@solution}{}}
  \else
  \immediate\write\solutionstream{\string\begin{exsol@solution}{\theexercise}}
  \fi
  \@bsphack
  \begingroup
    \FV@UseKeyValues
    \FV@DefineWhiteSpace
    \def\FV@Space{\space}%
    \FV@DefineTabOut
    \let\FV@ProcessLine\GPSS@write@detok %
    \relax
    \let\FV@FontScanPrep\relax
    \let\@noligs\relax
    \FV@Scan
  }
\def\FVE@solution{
  \endgroup\@esphack
  \immediate\write\solutionstream{\string\end{exsol@solution}}
}
\DefineVerbatimEnvironment{solution}{solution}{}
%    \end{macrocode}
% \end{macro}
%
% \begin{macro}{exsol@solution}
%   The |exsol@solution| environment is an internal macro used
%   to typeset your solutions. Do not use it directly. Use the proper
%   environment |solution| instead.
%   \changes{v1.2}{2016/05/05}{Added |exsol@solution| environment for
%   improved control on solution typesetting}
%    \begin{macrocode}
\newenvironment{exsol@solution}[1]
{%
  \ifthenelse{\boolean{exsol@minipage}}{\begin{minipage}[t]{\textwidth}}{}%
    \ifthenelse{\boolean{exsol@exerciseaslist}}
    {\begin{list}%
        {%
        }%
        {%
          \setlength{\topsep}{\exsolexercisetopbottomsep}%
          \setlength{\leftmargin}{\exsolexerciseleftmargin}%
          \setlength{\rightmargin}{\exsolexerciserightmargin}%
          \setlength{\listparindent}{\exsolexerciseparindent}%
          \setlength{\itemindent}{\exsolexerciseitemindent}%
          \setlength{\parsep}{\exsolexerciseparsep}
          \setlength{\labelsep}{\exsolexerciselabelsep}
          \setlength{\labelwidth}{\exsolexerciselabelwidth}}
      \item[\ifexsol@nolabels #1:\else%
        \subparagraph{\solutionname{}\@ifmtarg{#1}{}{~}#1:}\fi]
      }%
      {\ifexsol@nolabels #1:\else%
        \subparagraph{\solutionname{}\@ifmtarg{#1}{}{~}#1:}\fi}
    }
    {%
      \ifthenelse{\boolean{exsol@exerciseaslist}}%
      {\end{list}}{}%
    \ifthenelse{\boolean{exsol@minipage}}{\end{minipage}}{\par}%
}
%    \end{macrocode}
% \end{macro}
%
%
% \begin{macro}{exercises}
%   The |exercises| environment helps typesetting your exercises to
%   stand out from the rest of the text. You may use it at the end of
%   a chapter, or just to group some exercises in the text.
%   \changes{v0.2}{2012/01/06}{Attempted to fix MiKTeX formatting problems}
%   \changes{v0.3}{2012/01/07}{Added some extra whitespace below
%   exercisesname}
%   \changes{v1.2}{2016/05/05}{Added multicolumn feature}
%    \begin{macrocode}
\define@key{exercises}{columns}{\renewcommand\columncount{#1}}
\define@key{exercises}{exsubrule}{\renewcommand\exsubrule{#1}}
\define@key{exercises}{solsubrule}{\renewcommand\solsubrule{#1}}
\define@key{exercises}{subrule}{\renewcommand\exsubrule{#1}\renewcommand\solsubrule{#1}}
\newenvironment{exercises}[1][]
{
  \newcommand\columncount{1} % default
  \newcommand\exsubrule{} % default
  \newcommand\solsubrule{} % default
  \setkeys{exercises}{#1}
  \par\exercisesfontsize\rule{.25\linewidth}{0.15mm}%
  \vspace*{\exsolexercisesaboveskip}%
  \paragraph{\exercisesname}~\par
  \ifthenelse{\columncount > 1}{\begin{multicols}{\columncount}}{}
  }
  {
  \ifthenelse{\columncount > 1}{\end{multicols}}{}\relax
  \vspace*{-\baselineskip}\vspace*{\exsolexercisesbelowskip}%
  %\rule{.25\linewidth}{0.15mm}
  \exsubrule\par}
%    \end{macrocode}
% \end{macro}
% 
% \changes{v1.2}{2016/05/05}{Added exerciseseries environment}
% \begin{macro}{exerciseseries}
%   The |exerciseseries| environment helps typesetting your
%   exercises in series.
%    \begin{macrocode}
\define@key{exerciseseries}{columns}{\renewcommand\columncount{#1}}
\define@key{exerciseseries}{exsubrule}{\renewcommand\exsubrule{#1}}
\define@key{exerciseseries}{solsubrule}{\renewcommand\solsubrule{#1}}
\define@key{exerciseseries}{subrule}{\renewcommand\exsubrule{#1}\renewcommand\solsubrule{#1}}
\newenvironment{exerciseseries}[2][]
{
  \refstepcounter{exerciseseries}%
  \newcommand\columncount{1} % default
  \newcommand\exsubrule{} % default
  \newcommand\solsubrule{} % default
  \setkeys{exerciseseries}{#1}
  \paragraph{\seriesname~\theexerciseseries:~#2}~\par
  \ifthenelse{\columncount > 1}{\begin{multicols}{\columncount}}{}
    \immediate\write\solutionstream{\string\begin\string{solutionseries\string}%
        \string[#1\string]\string{#2\string}\string{\theexerciseseries\string}}
    }
    {
      \ifthenelse{\columncount > 1}{\end{multicols}}{}\relax
    \exsubrule\par
    \immediate\write\solutionstream{\string\end\string{solutionseries\string}}
}
\newenvironment{solutionseries}[3][]
{ 
  \newcommand\columncount{1} % default
  \newcommand\exsubrule{} % default
  \newcommand\solsubrule{} % default
  \setkeys{exercises}{#1}
  \paragraph{\seriesname~#3}~\par
  \ifthenelse{\columncount > 1}{\begin{multicols}{\columncount}}{}
  }
  {
    \ifthenelse{\columncount > 1}{\end{multicols}}{}
  \solsubrule\par
}
%    \end{macrocode}
% \end{macro}
% 
% \begin{macro}{informulacollection}
%   The |informulacollection| environment is used to write its
%   contents to the formula collection stream and load back into the
%   main text for typesetting. The code is almost litteraly
%   taken from the \textsf{fancyvrb} package.
%    \begin{macrocode}
\def\informulacollection{\FV@Environment{}{informulacollection}}
\def\FVB@informulacollection{%
  \immediate\openout\formulastream=\jobname.for.tex
  %\typeout{Writing formula to \jobname.for.tex and \jobname.fc.tex}
  \@bsphack
  \begingroup
    \FV@UseKeyValues
    \FV@DefineWhiteSpace
    \def\FV@Space{\space}%
    \FV@DefineTabOut
    \let\FV@ProcessLine\GPFORCOL@write@detok %
    \relax
    \let\FV@FontScanPrep\relax
    \let\@noligs\relax
    \FV@Scan
  }
\def\FVE@informulacollection{
  \endgroup\@esphack
  \immediate\closeout\formulastream
  \input{\jobname.for.tex}
}
\DefineVerbatimEnvironment{informulacollection}{informulacollection}{}
%    \end{macrocode}
% \end{macro}
%
% \begin{macro}{informulacollectiononly}
%   The |informulacollectiononly| environment is used to write its
%   contents to the formula collection stream \emph{without} loading
%   it back into the main text for typesetting. The code is almost litteraly
%   taken from the \textsf{fancyvrb} package.
%    \begin{macrocode}
\def\informulacollectiononly{\FV@Environment{}{informulacollectiononly}}
\def\FVB@informulacollectiononly{%
  %\typeout{Writing special to \jobname.fc.tex}
  \@bsphack
  \begingroup
    \FV@UseKeyValues
    \FV@DefineWhiteSpace
    \def\FV@Space{\space}%
    \FV@DefineTabOut
    \let\FV@ProcessLine\GPFORCOLONLY@write@detok %
    \relax
    \let\FV@FontScanPrep\relax
    \let\@noligs\relax
    \FV@Scan
  }
\def\FVE@informulacollectiononly{
  \endgroup\@esphack
}
\DefineVerbatimEnvironment{informulacollectiononly}{informulacollectiononly}{}
%    \end{macrocode}
% \end{macro}
%
% \subsection{Some Babel provisions}
% %%%%%%%%%%%%%%%%%%%%%%%%%%%%%%%%%%
% \changes{v0.2}{2012/01/06}{Fixed babel errors}
%    \begin{macrocode}
\newcommand{\exercisename}{Exercise}
\newcommand{\exercisesname}{Exercises}
\newcommand{\solutionname}{Solution}
\newcommand{\solutionsname}{Solutions}
\newcommand{\seriesname}{Series}
%    \end{macrocode}
% 
% 
% You may redefine these macros, but to help you out a little bit, we
% provide with some basic Babel auxiliaries. If you're a true polyglot
% and are willing to help me out by providing translations for other
% languages, I'm very willing to incorporate them into the code.
%
% \changes{v0.7}{2014/07/14}{Added Finnish language support}
% \changes{v1.2}{2016/05/05}{Added Spanish language support}
% \changes{v1.2}{2016/05/05}{Completed German language support
% (i.e. new spelling ngerman) support}
%    \begin{macrocode}
\addto\captionsdutch{%
  \renewcommand{\exercisename}{Oefening}%
  \renewcommand{\exercisesname}{Oefeningen}%
  \renewcommand{\solutionname}{Oplossing}%
  \renewcommand{\solutionsname}{Oplossingen}%
  \renewcommand{\seriesname}{Reeks}%
}
\addto\captionsgerman{%
  \renewcommand{\exercisename}{Aufgabe}%
  \renewcommand{\exercisesname}{Aufgaben}%
  \renewcommand{\solutionname}{L\"osung}%
  \renewcommand{\solutionsname}{L\"osungen}%
  \renewcommand{\seriesname}{Serie}%
}
\addto\captionsngerman{%
  \renewcommand{\exercisename}{Aufgabe}%
  \renewcommand{\exercisesname}{Aufgaben}%
  \renewcommand{\solutionname}{L\"osung}%
  \renewcommand{\solutionsname}{L\"osungen}%
  \renewcommand{\seriesname}{Serie}%
}
\addto\captionsfrench{%
  \renewcommand{\exercisename}{Exercice}%
  \renewcommand{\exercisesname}{Exercices}%
  \renewcommand{\solutionname}{Solution}%
  \renewcommand{\solutionsname}{Solutions}%
  \renewcommand{\seriesname}{Serie}%
}
\addto\captionsfinnish{
  \renewcommand{\exercisename}{Teht\"av\"a}%
  \renewcommand{\exercisesname}{Teht\"avi\"a}%
  \renewcommand{\solutionname}{Ratkaisu}%
  \renewcommand{\solutionsname}{Ratkaisut}%
  \renewcommand{\seriesname}{Sarja}
}
\addto\captionsspanish{%
  \renewcommand{\exercisename}{Ejercicio}%
  \renewcommand{\exercisesname}{Ejercicios}%
  \renewcommand{\solutionname}{Soluci\'on}%
  \renewcommand{\solutionsname}{Soluciones}%
  \renewcommand{\seriesname}{Serie}%
}
%    \end{macrocode}
%
%
%
% Now the final hack overloads the basic sectioning commands to make
% sure that they are copied into your solution book.
%
% \changes{v1.0}{2015/09/16}{Added congruence of chaptercounter of
% main document and chapter counter of exercises document}
% \changes{v1.1}{2015/09/16}{Corrected congruence of chaptercounter of
% main document and chapter counter of exercises document (taking
% intou account alphanumbered chapters}
%    \begin{macrocode}
\ifexsol@local
\else
\let\exsol@@makechapterhead\@makechapterhead
\def\@makechapterhead#1{%
  \exsol@@makechapterhead{#1}
  \immediate\write\solutionstream{\string\setcounter{chapter}{\arabic{chapter}}%
    \string\chapter{#1}}%
}
\ifdefined\frontmatter
  \let\exsol@@frontmatter\frontmatter
  \def\frontmatter{%
    \immediate\write\solutionstream{\string\frontmatter}%
    \exsol@@frontmatter
  }
\fi
\ifdefined\frontmatter
  \let\exsol@@mainmatter\mainmatter
  \def\mainmatter{%
    \immediate\write\solutionstream{\string\mainmatter}%
    \exsol@@mainmatter
  }
\fi
\ifdefined\backmatter
  \let\exsol@@backmatter\backmatter
  \def\backmatter{%
    \immediate\write\solutionstream{\string\backmatter}%
    \exsol@@backmatter
  }
\fi
\ifdefined\appendix
  \let\exsol@@appendix\appendix
  \def\appendix{%
    \immediate\write\solutionstream{\string\appendix}%
    \exsol@@appendix
  }
\fi
\fi
%    \end{macrocode}
%
% \begin{macro}{\noexercisesinchapter}
%   If you have chapters without exercises, you may want to indicate
%   this clearly into your source. Otherwise empty chapters may appear
%   in your solution book.
%    \begin{macrocode}
\newcommand{\noexercisesinchapter}
{
  \immediate\write\solutionstream{No exercises in this chapter}
}
%    \end{macrocode}
% \end{macro}
%
%    \begin{macrocode}
%</package>
%    \end{macrocode}
%
% \bibliographystyle{alpha}
%
% \begin{thebibliography}{99}
%
% \bibitem{fancyvrb}
% Timothy Van Zandt, Herbert Vo\ss, Denis Girou, Sebastian Rahtz, Niall
% Mansfield 
% \newblock The \texttt{fancyvrb} package.
% \newblock \url{http://ctan.org/pkg/fancyvrb}.
% \newblock online, accessed in January 2012.
%
% \bibitem{CTAN} 
% The Comprehensive TeX Archive Network.
% \newblock \url{http://www.ctan.org}.
% \newblock online, accessed in January 2012.
%
% \end{thebibliography}
%
% \Finale
\endinput
