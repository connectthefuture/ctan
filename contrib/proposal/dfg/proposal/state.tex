\section{State of the Art and Preliminary Work}\label{sec:state}
\begin{todo}{from the proposal guidelines}
  For new proposals please explain briefly and precisely the state of the art in your
  field in its direct relationship to your project. This description should make clear in
  which context you situate your own research and in what areas you intend to make a
  unique, innovative, promising contribution. This description must be concise and
  understand- able without referring to additional literature.

  For renewal proposals, please report on your previous work. This report should also be
  understandable without referring to additional literature.

  To illustrate and enhance your presentation you may refer to your own and others’ pub-
  lications. Indicate whenever you are referring to other researchers’ work.  Please list
  all cited publications in your bibliography under section 3. This reference list is not
  consid- ered your list of publications. Note that reviewers are not required to read any
  of the works you cite. This also applies to review sessions that are held on site. In
  this case, manuscripts and publications that provide more information on the progress
  reports and are published up to the review panel’s meeting may be made available at the
  meeting to enable reviewers to read through the information. Reviews will be based only
  on the text of the actual proposal.
\end{todo}
\subsection{List of Project-Related Publications}\label{sec:projpapers}

\begin{todo}{from the proposal template}
  Please include a list of own publications that are related to the proposed project. It
  serves as an important basis for assessing your proposal. The number of publications to
  cite here is determined as follows:
  \begin{compactdesc}
    \item[Single applicant] two publications per year of the funding duration
    \item[Multiple applicants] three publications per year of the funding duration
    \end{compactdesc}
    These rules refer to the proposed funding duration for new proposals and the completed
    duration for renewal proposals.
    
    If you are submitting a proposal to the DFG for the first time and have therefore not
    published in the proposed research area, please list the up to five most important
    publications so far.
\end{todo}

\subsubsection{Peer-Reviewed Articles}

\dfgprojpapers{Kohlhase:pdpl10,providemore}

\ednote{Anmerkung Jens: Ein nützliches Feature wäre hier, wenn das Paket eine (eventuell
  über Optionen der Dokumentklasse unterdrückbare) Warnung ausgeben würde, wenn zu viele
  Publikationen entsprechend DFG-Richtlinien angegeben werden. Die Anzahl ist sehr eng
  begrenzt.}

\subsubsection{Other Articles\qquad None.}
\subsubsection{Patents\qquad None.}

%%% Local Variables: 
%%% mode: LaTeX
%%% TeX-master: "proposal"
%%% End: 

% LocalWords:  subsubsections dfgprojpapers pdpl10 providemore compactdesc
% LocalWords:  ourpubs nociteprolist KohKoh ccbssmt09 KohRabZho tmlmrsca10
% LocalWords:  Hutter09 sifemp09
