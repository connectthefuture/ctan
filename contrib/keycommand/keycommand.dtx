% \iffalse meta-comment
% keycommand : key-value interface for commands and environments in LaTeX v3.1415 2010/04/27]
%
% This work may be distributed and/or modified under the
% conditions of the LaTeX Project Public License, either
% version 1.3 of this license or (at your option) any later
% version. The latest version of this license is in
%    http://www.latex-project.org/lppl.txt
%
% This work consists of the main source file keycommand.dtx
% and the derived files
%    keycommand.sty, keycommand.pdf, keycommand.ins,
%    keycommand-example.tex
%
% Unpacking:
%    (a) If keycommand.ins is present:
%           etex keycommand.ins
%    (b) Without keycommand.ins:
%           etex keycommand.dtx
%    (c) If you insist on using LaTeX
%           latex \let\install=y% \iffalse meta-comment
% keycommand : key-value interface for commands and environments in LaTeX v3.1415 2010/04/27]
%
% This work may be distributed and/or modified under the
% conditions of the LaTeX Project Public License, either
% version 1.3 of this license or (at your option) any later
% version. The latest version of this license is in
%    http://www.latex-project.org/lppl.txt
%
% This work consists of the main source file keycommand.dtx
% and the derived files
%    keycommand.sty, keycommand.pdf, keycommand.ins,
%    keycommand-example.tex
%
% Unpacking:
%    (a) If keycommand.ins is present:
%           etex keycommand.ins
%    (b) Without keycommand.ins:
%           etex keycommand.dtx
%    (c) If you insist on using LaTeX
%           latex \let\install=y% \iffalse meta-comment
% keycommand : key-value interface for commands and environments in LaTeX v3.1415 2010/04/27]
%
% This work may be distributed and/or modified under the
% conditions of the LaTeX Project Public License, either
% version 1.3 of this license or (at your option) any later
% version. The latest version of this license is in
%    http://www.latex-project.org/lppl.txt
%
% This work consists of the main source file keycommand.dtx
% and the derived files
%    keycommand.sty, keycommand.pdf, keycommand.ins,
%    keycommand-example.tex
%
% Unpacking:
%    (a) If keycommand.ins is present:
%           etex keycommand.ins
%    (b) Without keycommand.ins:
%           etex keycommand.dtx
%    (c) If you insist on using LaTeX
%           latex \let\install=y% \iffalse meta-comment
% keycommand : key-value interface for commands and environments in LaTeX v3.1415 2010/04/27]
%
% This work may be distributed and/or modified under the
% conditions of the LaTeX Project Public License, either
% version 1.3 of this license or (at your option) any later
% version. The latest version of this license is in
%    http://www.latex-project.org/lppl.txt
%
% This work consists of the main source file keycommand.dtx
% and the derived files
%    keycommand.sty, keycommand.pdf, keycommand.ins,
%    keycommand-example.tex
%
% Unpacking:
%    (a) If keycommand.ins is present:
%           etex keycommand.ins
%    (b) Without keycommand.ins:
%           etex keycommand.dtx
%    (c) If you insist on using LaTeX
%           latex \let\install=y\input{keycommand.dtx}
%        (quote the arguments according to the demands of your shell)
%
% Documentation:
%           (pdf)latex keycommand.dtx
% Copyright (C) 2009-2010 by Florent Chervet <florent.chervet@free.fr>
%<*ignore>
\begingroup
  \def\x{LaTeX2e}%
\expandafter\endgroup
\ifcase 0\ifx\install y1\fi\expandafter
         \ifx\csname processbatchFile\endcsname\relax\else1\fi
         \ifx\fmtname\x\else 1\fi\relax
\else\csname fi\endcsname
%</ignore>
%<*install>
\input docstrip.tex
\Msg{************************************************************************}
\Msg{* Installation}
\Msg{* Package: keycommand 2010/04/27 v3.1415 key-value interface for commands and environments in LaTeX}
\Msg{************************************************************************}

\keepsilent
\askforoverwritefalse

\let\MetaPrefix\relax
\preamble

This is a generated file.

keycommand : key-value interface for commands and environments in LaTeX [v3.1415 2010/04/27]

This work may be distributed and/or modified under the
conditions of the LaTeX Project Public License, either
version 1.3 of this license or (at your option) any later
version. The latest version of this license is in
   http://www.latex-project.org/lppl.txt

This work consists of the main source file keycommand.dtx
and the derived files
   keycommand.sty, keycommand.pdf, keycommand.ins,
   keycommand-example.tex

keycommand : an easy way to define commands with optional keys
Copyright (C) 2009-2010 by Florent Chervet <florent.chervet@free.fr>

\endpreamble
\let\MetaPrefix\DoubleperCent

\generate{%
   \file{keycommand.ins}{\from{keycommand.dtx}{install}}%
   \file{keycommand.sty}{\from{keycommand.dtx}{package}}%
   \file{keycommand-example.tex}{\from{keycommand.dtx}{example}}%
}

\generate{%
   \file{keycommand.drv}{\from{keycommand.dtx}{driver}}%
}

\obeyspaces
\Msg{************************************************************************}
\Msg{*}
\Msg{* To finish the installation you have to move the following}
\Msg{* file into a directory searched by TeX:}
\Msg{*}
\Msg{*     keycommand.sty}
\Msg{*}
\Msg{* To produce the documentation run the file `keycommand.dtx'}
\Msg{* through LaTeX.}
\Msg{*}
\Msg{* Happy TeXing!}
\Msg{*}
\Msg{************************************************************************}

\endbatchfile
%</install>
%<*ignore>
\fi
%</ignore>
%<*driver>
\edef\thisfile{\jobname}
\def\thisinfo{key-value interface for commands and environments in \LaTeX.}
\def\thisdate{2010/04/27}
\def\thisversion{3.1415}
\let\loadclass\LoadClass
\def\LoadClass#1{\loadclass[abstracton]{scrartcl}\let\scrmaketitle\maketitle\AtEndOfClass{\let\maketitle\scrmaketitle}}
\documentclass[a4paper,oneside]{ltxdoc}
\usepackage[latin9]{inputenc}
\usepackage[american]{babel}
\usepackage[T1]{fontenc}
\usepackage{etex,etoolbox,holtxdoc,geometry,tocloft,graphicx,xspace,fancyhdr,color,bbding,embedfile,framed,multirow,txfonts,makecell,enumitem,arydshln}
\CodelineNumbered
\usepackage{keyval}\makeatletter\let\keyval@setkeys\setkeys\makeatother
\usepackage{xkeyval}\let\xsetkeys\setkeys
\usepackage{kvsetkeys}
\usepackage{fancyvrb}
\lastlinefit999
\geometry{top=2cm,headheight=1cm,headsep=.3cm,bottom=1.4cm,footskip=.5cm,left=2.5cm,right=1cm}
\hypersetup{%
  pdftitle={The keycommand package},
  pdfsubject={key-value interface for commands and environments in LaTeX.},
  pdfauthor={Florent CHERVET},
  colorlinks,linkcolor=reflink,
  pdfstartview={FitH},
  pdfkeywords={tex, e-tex, latex, package, keys, keycommand, newcommand, keyval, kvsetkeys, programming},
  bookmarksopen=true,bookmarksopenlevel=3}
\embedfile{\thisfile.dtx}
\begin{document}
   \DocInput{\thisfile.dtx}
\end{document}
%</driver>
% \fi
%
% \CheckSum{1111}
%
% \CharacterTable
%  {Upper-case    \A\B\C\D\E\F\G\H\I\J\K\L\M\N\O\P\Q\R\S\T\U\V\W\X\Y\Z
%   Lower-case    \a\b\c\d\e\f\g\h\i\j\k\l\m\n\o\p\q\r\s\t\u\v\w\x\y\z
%   Digits        \0\1\2\3\4\5\6\7\8\9
%   Exclamation   \!     Double quote  \"     Hash (number) \#
%   Dollar        \$     Percent       \%     Ampersand     \&
%   Acute accent  \'     Left paren    \(     Right paren   \)
%   Asterisk      \*     Plus          \+     Comma         \,
%   Minus         \-     Point         \.     Solidus       \/
%   Colon         \:     Semicolon     \;     Less than     \<
%   Equals        \=     Greater than  \>     Question mark \?
%   Commercial at \@     Left bracket  \[     Backslash     \\
%   Right bracket \]     Circumflex    \^     Underscore    \_
%   Grave accent  \`     Left brace    \{     Vertical bar  \|
%   Right brace   \}     Tilde         \~}
%
% \DoNotIndex{\begin,\CodelineIndex,\CodelineNumbered,\def,\DisableCrossrefs,\~,\@ifpackagelater}
% \DoNotIndex{\DocInput,\documentclass,\EnableCrossrefs,\end,\GetFileInfo}
% \DoNotIndex{\NeedsTeXFormat,\OnlyDescription,\RecordChanges,\usepackage}
% \DoNotIndex{\ProvidesClass,\ProvidesPackage,\ProvidesFile,\RequirePackage}
% \DoNotIndex{\filename,\fileversion,\filedate,\let}
% \DoNotIndex{\@listctr,\@nameuse,\csname,\else,\endcsname,\expandafter}
% \DoNotIndex{\gdef,\global,\if,\item,\newcommand,\nobibliography}
% \DoNotIndex{\par,\providecommand,\relax,\renewcommand,\renewenvironment}
% \DoNotIndex{\stepcounter,\usecounter,\nocite,\fi}
% \DoNotIndex{\@fileswfalse,\@gobble,\@ifstar,\@unexpandable@protect}
% \DoNotIndex{\AtBeginDocument,\AtEndDocument,\begingroup,\endgroup}
% \DoNotIndex{\frenchspacing,\MessageBreak,\newif,\PackageWarningNoLine}
% \DoNotIndex{\protect,\string,\xdef,\ifx,\texttt,\@biblabel,\bibitem}
% \DoNotIndex{\z@,\wd,\wheremsg,\vrule,\voidb@x,\verb,\bibitem}
% \DoNotIndex{\FrameCommand,\MakeFramed,\FrameRestore,\hskip,\hfil,\hfill,\hsize,\hspace,\hss,\hbox,\hb@xt@,\endMakeFramed,\escapechar}
% \DoNotIndex{\do,\date,\if@tempswa,\@tempdima,\@tempboxa,\@tempswatrue,\@tempswafalse,\ifdefined,\ifhmode,\ifmmode,\cr}
% \DoNotIndex{\box,\author,\advance,\multiply,\Command,\outer,\next,\leavevmode,\kern,\title,\toks@,\trcg@where,\tt}
% \DoNotIndex{\the,\width,\star,\space,\section,\subsection,\textasteriskcentered,\textwidth}
% \DoNotIndex{\",\:,\@empty,\@for,\@gtempa,\@latex@error,\@namedef,\@nameuse,\@tempa,\@testopt,\@width,\\,\m@ne,\makeatletter,\makeatother}
% \DoNotIndex{\maketitle,\parindent,\setbox,\x,\kernel@ifnextchar}
% \DoNotIndex{\KVS@CommaComma,\KVS@CommaSpace,\KVS@EqualsSpace,\KVS@Equals,\KVS@Global,\KVS@SpaceEquals,\KVS@SpaceComma,\KVS@Comma}
% \DoNotIndex{\DefineShortVerb,\DeleteShortVerb,\UndefineShortVerb,\MakeShortVerb,\endinput}
% \let\ClearPage\clearpage
% \makeatletter
% \MakeShortVerb{\+}\DeleteShortVerb{\|}\DefineShortVerb{\|}
% \catcode`\� \active   \def�{\@ifnextchar �{\par\nobreak\vskip-2\parskip}{\par\nobreak\vskip-\parskip}}
% \def\thispackage{\xpackage{\thisfile}\xspace}
% \def\ThisPackage{\Xpackage{\thisfile}\xspace}
% \def\Xpackage{\@dblarg\X@package}
% \def\X@package[#1]#2{%
%     \xpackage{#2\footnote{\noindent\xpackage{#2}: \href{http://www.ctan.org/tex-archive/macros/latex/contrib/#1}{\nolinkurl{CTAN:macros/latex/contrib/#1}}}}}
% \def\Underbrace#1_#2{$\underbrace{\vtop to2ex{}\hbox{#1}}_{\footnotesize\hbox{#2}}$}
%
% \parindent\z@\parskip.4\baselineskip\topsep\parskip\partopsep\z@
% \g@addto@macro\macro@font{\macrocodecolor\let\AltMacroFont\macro@font}
% \g@addto@macro\@list@extra{\parsep\parskip\topsep\z@\itemsep\z@}
% \def\smex{\leavevmode\hb@xt@2em{\hfil$\longrightarrow$\hfil}}
% \newrobustcmd\verbfont{\usefont{T1}{\ttdefault}{\f@series}{n}}    \let\vb\verbfont
% \renewrobustcmd\#[1]{{\usefont{T1}{pcr}{bx}{n}\char`\##1}}
% \newrobustcmd\csred[1]{\textcolor{red}{\cs{#1}}}
% \renewrobustcmd\cs[2][]{\mbox{\vb#1\expandafter\@gobble\string\\#2}}
% \newrobustcmd\CSbf[1]{\textbf{\CS{#1}}}
% \newrobustcmd\csbf[2][]{\textbf{\cs[{#1}]{#2}}}
% \newrobustcmd\textttbf[1]{\textbf{\texttt{#1}}}
% \renewrobustcmd*\bf{\bfseries}\newcommand\nnn{\normalfont\mdseries\upshape}\newcommand\nbf{\normalfont\bfseries\upshape}
% \newrobustcmd*\blue{\color{blue}}\newcommand*\red{\color{dr}}\newcommand*\green{\color{green}}\newcommand\rred{\color{red}}
% \newrobustcmd\rrbf{\color{red}\bfseries}
% \definecolor{copper}{rgb}{0.67,0.33,0.00}  \newcommand\copper{\color{copper}}
% \definecolor{dg}{rgb}{0.16,0.33,0.00}      \newcommand\dg{\color{dg}}
% \definecolor{db}{rgb}{0,0,0.502}           \newcommand\db{\color{db}}
% \definecolor{dr}{rgb}{0.49,0.00,0.00}      \let\dr\red
% \newrobustcmd\bk{\color{black}}\newcommand\md{\mdseries}
%
% \fancyhf{}\fancyhead[L]{The \thispackage package -- \thisinfo}
% \fancyfoot[L]{\color[gray]{.35}\scriptsize\thispackage\quad[rev.\thisversion]\quad\copyright\oldstylenums{2009-2010}\,\lower.3ex\hbox{\NibRight}\,Florent Chervet}
% \fancyfoot[R]{\oldstylenums{\thepage} / \oldstylenums{\pageref{LastPage}}}
% \pagestyle{fancy}
% \fancypagestyle{plain}{%
%     \let\headrulewidth\z@
%     \fancyhf{}%
%     \fancyfoot[R]{\oldstylenums{\thepage} / \oldstylenums{\pageref{LastPage}}}}
%
% \newcommand\macrocodecolor{\color{macrocode}}\definecolor{macrocode}{rgb}{0.18,0.00,0.45}
% \newcommand\reflinkcolor{\color{reflink}}\definecolor{reflink}{rgb}{0.49,0.00,0.00}
% \font\umrandA=umranda at 20pt
% \def\@serp{\leavevmode\lower20pt\hbox{\umrandA\char'131}}
% \def\serp#1{\@serp\hfil #1\hfil\reflectbox{\@serp}}
% \newrobustcmd\stform{\@ifnextchar*{\@stform[]\textasteriskcentered\@gobble}\@stform}
% \newrobustcmd\@stform[2][\string]{\textttbf{\rred#1#2}\xspace}
%
% \makeatother
%
% \deffootnote{1em}{0pt}{\rlap{\textsuperscript{\thefootnotemark}}\kern1em}
%
% \title{\vskip-18pt\mdseries {\bfseries\ThisPackage}\kern.6em package}
% \author{\footnotesize\xemail{florent.chervet@free.fr}}
% \date{\thisdate~--~version \thisversion}
% \subtitle{\thisinfo}
% ^^A\subject{\vskip-2cm\serp{The completely redesigned}}
% \subject{\vskip-2cm\relax The \textit{free} and \textit{open source}}
%
% \maketitle
% 
% \makeatletter\begingroup\let\@thefnmark\@empty\let\@makefntext\@firstofone
% \footnotetext{\noindent
% This documentation is produced with the +DocStrip+ utility.
% \begin{tabbing}
% \qquad\=\smex\=To get the documentation, \= run (thrice):\quad\= \texttt{pdflatex keycommand.dtx} \\
% \qquad\>\>To get the index, \> run:\>\texttt{makeindex -s gind.ist keycommand.idx} \\
% \>\smex\>To get the package, \> run:\>        \texttt{etex keycommand.dtx}
% \end{tabbing}�
% The \xext{dtx} file is embedded into this pdf file thank to \xpackage{embedfile} by H. Oberdiek.}
% \endgroup\makeatother
% 
% \hypersetup{bookmarksopenlevel=3}
% \deffootnote{1em}{0pt}{\rlap{\thefootnotemark.}\kern1em}
% \vspace*{-18pt}
% \begin{abstract}\parindent0pt\noindent\leftskip1cm\rightskip\leftskip\lastlinefit0%
%
% \thispackage provides an easy way to define commands or environments
% with optional keys.
% \smallskip
%
% \csbf{newkeycommand} \cs{renewkeycommand} \cs{providekeycommand} and \csbf{newkeyenvironment},\linebreak 
% \cs{renewkeyenvironment} are macros to define such commands and environments with keys.
% 
% \thispackage is designed to make easier interface for user-defined commands.  In particular,
% \csbf{newkeycommand}\stform+ permits the use of key-commands in every context. 
% \medskip
%
% Keys are defined with the command itself in a very natural way.
% You can restrict the possible values for the keys by declaring them with a \textbf{type}.
% Available types for keys are : \textit{boolean}, \textit{enum} and \textit{choice} (see \ref{subsec:GeneralSyntax}).
%
% \smallskip
%
% The \thispackage package requires and is based on the package \xpackage{xkeyval} by Hendri Adriaens, and uses
% the \cs{kv@normalize} macro of \xpackage{kvsetkeys} (Heiko Oberdiek) for robustness, as shown
% in \ref{kvsetkeys-comparisons}).
%
% It works with an \eTeX{} distribution of \LaTeX.
% \end{abstract}
%
% \DeleteShortVerb{\+}\enlargethispage{2\baselineskip}
% \cftbeforesecskip=4pt plus2pt minus2pt
% \cftbeforesubsecskip=0pt plus2pt minus2pt
% \renewcommand\contentsname{Contents\quad\leaders\vrule height3.4pt depth-3pt\hfill\null\kern0pt\vskip-6pt}
% ^^A\vskip-.8\baselineskip
% \tableofcontents
%
% \clearpage\MakeShortVerb{\+}
%
% \def\B#1{\texttt{[}\meta{#1}\texttt{]}}
%
% \section{User Interface}
%
% \subsection{General syntax}\label{subsec:GeneralSyntax}
%
% \begin{declcs}{newkeycommand}%
%  \Underbrace{\textcolor{red}{\textasteriskcentered\string+[short-unexpand]}}_{\makecell[c]{modifiers \\ Optional}}\,%
%  \Underbrace{\M{command}}_{Required}\,%
%   {\color{db}\Underbrace{\B{keys=defaults}\,\B{OptKey}\,\B{<n>}}_{Optional}\,}%
%   \Underbrace{\M{definition}}_{Required}
% \end{declcs}
%
% \cs{newkeycommand} will define \cs{command} as a new key-command!\quad well...
%
% Use the \stform* form when you do not want it to be a \cs{long} macro (as for \LaTeX{}-\cs{newcommand}).
%
% The +[keys=defaults]+ argument define the keys with their default values. It is optional, but a key-command
% without keys seems to be useless (at least for me...). Keys may be defined as :
%
% \newlist{myenum}{enumerate}{1}
% \setlist[myenum]{label={},topsep=-\parskip,itemsep=-\parskip,parsep=\parskip,after=\vskip-\baselineskip}
%
% \renewcommand\theadfont{\tt\bfseries}
% \noindent\begin{tabular}{|c|>{\db}c|m{8cm}|}\hline
% \thead{Type} & \thead{exemple}                                              & \thead{value of \cs{commandkey}}                                                                                         \\ \hline
% general      & color{\dg=red}                                               & \cs{commandkey}\{{\db color}\} is `{\dg red}' and may be anything (text, number, macro...)                                           \\ \hline
% boolean      & {\rred bool} bold{\dg =true}                                 & \cs{commandkey}\{{\db bold}\} is:
%                                                                                                       \begin{myenum}
%                                                                                                       \item {\tt 0}\quad (for {\dg false})
%                                                                                                       \item {\tt 1}\quad (for {\dg true})
%                                                                                                       \end{myenum}          \\ \hline
% \multirow{2}*{enumerate} & {\rred enum} position{\dg=\{left,centered,right\}} & \cs{commandkey}\{{\db position}\} is:
%                                                                                                     \begin{myenum}
%                                                                                                     \item `{\dg left}'\quad by default and can be
%                                                                                                     \item `{\dg centered}' or
%                                                                                                     \item `{\dg right}'
%                                                                                                        \end{myenum}         \\ \cdashline{2-3}[1pt/2pt]
%                          & {\rred enum\textasteriskcentered} position{\dg=\{left,centered,right\}}  & This is the same, except match is case \textbf{in}sensitive   \bottopstrut                                   \\ \hline
% \multirow{2}*{choice}    & {\rred choice} position={\dg \{left,centered,right\}} & \cs{commandkey}\{{\db position}\} is:
%                                                                                                     \begin{myenum}
%                                                                                                     \item {\tt 0}\quad (for {\dg left} the default value),
%                                                                                                     \item {\tt 1}\quad (for {\dg centered})
%                                                                                                     \item {\tt 2}\quad (for {\dg right})
%                                                                                                     \end{myenum}                           \\ \cdashline{2-3}[1pt/2pt]
%                          & {\rred choice\textasteriskcentered} position={\dg\{left,centered,right\}} & This is the same, except match is case \textbf{in}sensitive   \bottopstrut                                   \\ \hline
% \end{tabular}
%
% The {\db+OptKey+} argument is used if you wish to capture the +key=value+ pairs that are not specifically defined (more on this in the examples section \ref{sec:examples}).
%
% The key-command may have {\tt 0} up to {\tt 9} \textbf{mandatory} arguments : specify the number by +<n>+ ({\tt 0} if omitted).
%
% The \stform+ form expands the \cs{commandkey} before executing the key-command itself, as explain in section \ref{sec:example:plus}.
%
% \subsection{First example :}
%
% \begin{tabbing}\label{textrule}
% \,\=\csbf{new}\=\textttbf{keycommand}\cs[\copper]{textrule}+[+{\color{db}+raise=.4ex,width=3em,thick=.4pt+}+][1]{%+ \\ ^^A+][1]{%+}\\
% \>\>\cs{rule}+[+\cs[\red]{commandkey}+{+{\db+raise+}+}]{+\cs[\red]{commandkey}+{+{\db+width+}+}{+\cs[\red]{commandkey}+{+{\db+thick+}+}}+\\
% \>\>\#1 \\
% \>\>\cs{rule}+[+\cs[\red]{commandkey}+{+{\db+raise+}+}]{+\cs[\red]{commandkey}+{+{\db+width+}+}}{+\cs[\red]{commandkey}+{+{\db+thick+}+}}}+
% \end{tabbing}
%
% defines the keys {\db+width+}, {\db+thick+} and {\db+raise+} with their default values (if not specified):
% {\db+3em+}, {\db+.4pt+} and {\db+.4ex+}. Now \cs[\copper]{textrule} can be used as follow:
% \begin{tabbing}
% \=1:\quad\=\cs[\copper]{textrule}+[width=2em]{hello}+\hskip2.5cm\=\smex\qquad\=        \rule[.4ex]{2em}{.4pt}hello\rule[.4ex]{2em}{.4pt} \\
% \>2:\>\cs[\copper]{textrule}+[thick=5pt,width=2em]{hello}+\>\smex\>                  \rule[.4ex]{2em}{5pt}hello\rule[.4ex]{2em}{5pt}\\
% \>3:\>\cs[\copper]{textrule}+{hello}+\quad \>\smex\>                                 \rule[.4ex]{3em}{.4pt}hello\rule[.4ex]{3em}{.4pt}\\
% \>4:\>\cs[\copper]{textrule}+[thick=2pt,raise=1ex]{hello}+\>\smex\>                  \rule[1ex]{3em}{2pt}hello\rule[1ex]{3em}{2pt} \\
% \> \textit{et c\ae tera}.
% \end{tabbing}
%
% \clearpage
%
% \subsection[Second example : the \string+ form]{Second example : the {\rred\bf\string+} form}
% \label{sec:example:plus}
%
% \DeleteShortVerb{\+}
% \begin{Verbatim}[gobble=1,commandchars=$(),frame=lines]
% ($bf\newkeycommand)($rred$bf+[\|])($copper\myfigure)[image,
%                              caption,
%                              enum placement={H,h,b,t,p},
%                              width=\textwidth,
%                              label=
%                             ][($db OtherKeys)]{%
%        ($rred|)($bf\begin){figure}($dr|)[($red\commandkey){placement}]
%           ($rred|)($bf\includegraphics)($dr|)[width=($red\commandkey){width},($red\commandkey){($db OtherKeys)}]{%
%                             ($red\commandkey){image}}%
%           ($dg\ifcommandkey){caption}{($rred|)\caption($rred|){($red\commandkey){caption}}}{}%
%           ($dg\ifcommandkey){label}{($rred|)\label($rred|){($red\commandkey){label}}}{}%
%        ($rred|)($bf\end){figure}($rred|)}
% \end{Verbatim}
% \MakeShortVerb{\+}
%
% With the \stform+ form of \cs{newkeycommand}, the definition will be expanded (at run time). The optional {\rred\bf+[\|]+} argument
% means that everything inside {\bf\rred+|+ ... +|+} is protected from expansion.
%
% {\dg\cs{ifcommandkey}}\{\meta{name}\}\{\meta{true}\}\{\meta{false}\}\quad expands \meta{true} if the commandkey \meta{name} is not blank.
%
% {\db \meta{Otherkeys}} captures the keys given by the user but not declared: they are simply given back to \cs{includegraphics} here...
%
%
% \subsection[Explanation of the \string+ form]{Explanation of the {\rred\bf\string+} form}
% \DeleteShortVerb{\+}
% The |\commankey{|\meta{name}|}| stuff is expanded at run time using the following scheme:��
% \begin{Verbatim}[gobble=1,commandchars=!(),frame=lines]
%     (!bf\newkeycommand)(!copper\keyMacro)[A=\defA,B=\defB,C=\defC,D=\defD][1]{(!dg\begingroup)
%        (!dg\edef)\keyMacro##1{(!dg\endgroup)
%            (!dg\noexpand)\Macro{(!red\getcommandkey){A}}
%                           {(!red\getcommandkey){B}}
%                           {(!red\getcommandkey){C}}
%                           {(!red\getcommandkey){D}}
%     }\keyMacro{#1}}
% \end{Verbatim}
% Therefore, the arguments of \cs{Macro} are ready: there is no more \cs{commandkey} stuff, but instead the values of the keys
% as you gave them to the key-command. \cs{getcommandkey}\{A\} is expanded to \cs{defA}.
%
% But \cs{defA} is not expanded of course: in the \stform+ form, \cs{commandkey} has the meaning of \cs{getcommandkey}.
%
% As you can see, the mandatory arguments \#1, \#2 etc. are \textbf{never expanded}: there is no need to protect them inside the special (usually {\rred\bf\textbar}) character.
%
%
% \MakeShortVerb{\+}
% \clearpage
%
% \subsection{key-environments}
%
% \begin{declcs}{newkeyenvironment}%
%  \Underbrace{\textcolor{red}{\textasteriskcentered\string+[short-unexpand]}}_{\makecell[c]{modifiers \\ Optional}}\,%
%  \Underbrace{\M{envir name}}_{Required}\,%
%   {\db\Underbrace{\B{keys=defaults}\,\B{OptKey}\,\B{<n>}}_{Optional}\,}%
%   \Underbrace{\M{begin}}_{Required}\Underbrace{\M{end}}_{Required}
% \end{declcs}
%
% In the same way, you may define environments with optional keys as follow:�
% \begin{tabbing}
% \qquad\=+\newkeyenvironment+\=+{EnvirWithKeys}[kOne=+default value,...+][n]+\\
% \>\>+{+ commands at begin +EnvirWithKeys }+ \\
% \>\>+{+ commands at end +EnvirWithKeys }+
% \end{tabbing}
%
% Where $n$ is the number of mandatory other arguments (\emph{ie} without keys), if any.
%
% Key-environments may be defined with the \stform+ form in the same way as \cs{newkeycommand} is used.
% Be aware that each part of the environment: \meta{begin} and \meta{end} are expanded at run time then, 
% and the optional {\rred\bf+[\|]+} argument protects from expansion in each of those parts.
% 
% \subsection[Example of a \string+ key-environment]{Example of a {\rred\bf\string+} key-environment}
% 
% \DeleteShortVerb{\+}
% \begin{Verbatim}[gobble=1,commandchars=$(),frame=lines]
% ($bf\newkeyenvironment)($rred$bf+[\|])({$copper myfigure)}[
%                              caption,
%                              enum placement={H,h,b,t,p},
%                              width=.5\linewidth,
%                              label
%                             ][($db OtherKeys)][1]%
%     {% ($nbf$dg begin part)
%        ($rred|)($bf\begin){figure}($rred|)[($red\commandkey){placement}]
%           ($rred|)($bf\includegraphics)($rred|)[($red\commandkey){($db OtherKeys)},width=($red\commandkey){width}]{$#1}%
%     }
%     {% ($nbf$dg end part)
%           ($dg\ifcommandkey){caption}{($rred|)\caption($rred|){($red\commandkey){caption} image file = $#1}}{}%
%           ($dg\ifcommandkey){label}{($rred|)\label($rred|){($red\commandkey){label}}}{}%
%        ($rred|)($bf\end){figure}($rred|)%
%     }
% \end{Verbatim}
% \MakeShortVerb{\+}
%
% As you can see, \cs{commandkey} and mandatory arguments (\#1 here) are available both in the \meta{begin} 
% and in the \meta{end} parts of the key-environment.
% 
%
% \DefineShortVerb{\+}
%
% \section{Messages and more}
%
% \subsection{Invalid keys}
%
% If you use the command +\textule+ (defined in \ref{textrule}) with a key say: +height+
% that has not been declared at the definition of the key-command, you will get an
% error message like this:
% \begin{quote}\tt
% The key-value pairs ``height=...''�
% cannot be processed for key-command \string\textrule!�
% See the definition of the keycommand!
% \end{quote}
% The error is recoverable: the key is ignored.
%
% If you assign a value to an \textit{enum} or a \textit{choice} key, which is not allowed in the definition,
% you will get the following message:
% \begin{quote}\tt
% The value ``...'' is not allowed in key ...�
% for key-command \string\command�
% I'll use the default value ``...'' for this key instead�
% See the definition of the key-command!
% \end{quote}
% The error is recoverable: the key is assigned its default value.
%
% If you use a \cs{commandkey}\{\meta{name}\} in a key-command where \meta{name} is not defined as a key,
% you will get the \TeX{} generic error message :�
% \qquad undefined control sequence : \cs{keycmd->...@name}.
%
%
% \subsection{Testing keys}
%
% \begin{declcs}{ifcommandkey}\,\M{key name}\,\M{commands if key is NOT blank}\,\M{commands if key is blank}
% \end{declcs}
%
% When you define a key command you may let the default value of a key empty. Then, you may wish to
% expand some commands only if the key has been given by the user (with a non empty value). This can
% be achieved using the macro |\ifcommandkey|.
%
% \clearpage
% \subsection{xkeyval, keyval and kvsetkeys comparison}
%
% \begin{tabbing}
% \quad\=\xpackage{xkeyval}: \expandafter\meaning\csname ver@xkeyval.sty\endcsname \\
% \>\xpackage{keyval}: \expandafter\meaning\csname ver@keyval.sty\endcsname \\
% \>\xpackage{kvsetkeys}: \expandafter\meaning\csname ver@kvsetkeys.sty\endcsname
% \end{tabbing}
%
% \makeatletter\def\theadfont{\tt\bfseries}
% \define@key{fam}{key}{\def\result{#1}}
% \begin{table}[h]\label{kvsetkeys-comparisons}
% \begin{tabular}{|l|l|>{\color{db}}l|>{\color{dg}}l|}\hline
% \thead{\bf Example} & \thead{keyval} & \thead{xkeyval} & \thead{\makecell{kvsetkeys\\and\\keycommand}} \\ \hline
% +\setkeys{fam}{key={{value}}}+
%     & \keyval@setkeys{fam}{key={{value}}}\meaning\result
%     & \xsetkeys{fam}{key={{value}}}\meaning\result
%     & \kvsetkeys{fam}{key={{value}}}\meaning\result \\\hline
% +\setkeys{fam}{key={{{value}}}}+
%     & \keyval@setkeys{fam}{key={{{value}}}}\meaning\result
%     & \xsetkeys{fam}{key={{{value}}}}\meaning\result
%     & \kvsetkeys{fam}{key={{{value}}}}\meaning\result \\\hline
% +\setkeys{fam}{key=+\textvisiblespace+{{{value}}}}+
%     & \keyval@setkeys{fam}{key= {{{value}}}}\meaning\result
%     & \xsetkeys{fam}{key= {{{value}}}}\meaning\result
%     & \kvsetkeys{fam}{key= {{{value}}}}\meaning\result \\\hline
% \end{tabular}
% \caption{Then it is clear that, at this time, \xpackage{kvsetkeys} has the only correct behaviour...}
% \end{table}
%
% In \thispackage the key-value pairs are first normalized using \xpackage{kvsetkeys}-\cs{kv@normalize}. Then braces are added
% around the values in order to keep the good behaviour of \xpackage{kvsetkeys} while using \xpackage{xkeyval}.
% \makeatother
%
%
%
%
% \StopEventually{
% }
%
% \begin{center}\vskip6pt$\star$\hskip4em\lower12pt\hbox{$\star$}\hskip4em$\star$\vadjust{\vskip12pt}\end{center}
%
% \section{Implementation} \label{Implementation}
% \csdef{HDorg@PrintMacroName}#1{\hbox to4em{\strut \MacroFont \string #1\ \hss}}
%
% \subsection{Identification}
%
% This package is intended to use with \LaTeX{} so we don't check if it is loaded twice.
%
%    \begin{macrocode}
%<*package>
\NeedsTeXFormat{LaTeX2e}% LaTeX 2.09 can't be used (nor non-LaTeX)
   [2005/12/01]% LaTeX must be 2005/12/01 or younger (see kvsetkeys.dtx).
\ProvidesPackage{keycommand}
   [2010/04/27 v3.1415 - key-value interface for commands and environments in LaTeX]
%    \end{macrocode}
%
% \subsection{Requirements}
%
% The package is based on \xpackage{xkeyval}. However, \xpackage{xkeyval} is far less reliable
% than \xpackage{kvsetkeys} as far as spaces and bracket (groups) are concerned, as shown in the section
% \ref{kvsetkeys-comparisons} of this documentation.
%
% Therefore, we also use the macros of \xpackage{kvsetkeys} in order to \textit{normalize} the \texttt{key=value}
% list before setting the keys. This way, we take advantage of both \xpackage{xkeyval} and \xpackage{kvsetkeys} !
%
% As long as we use \eTeX{} primitives in \xpackage{keycommand} we also load the
% \xpackage{etex} package in order to get an error message if \eTeX{} is not running.
%
% The \xpackage{etoolbox} package gives some facility to write \xpackage{keycommand}.
% 
% From version \texttt{3.141} onwards, \thispackage does not load \xpackage{etextools} anymore.
%
%    \begin{macrocode}
\def\kcmd@pkg@name{keycommand}
\RequirePackage{etex,kvsetkeys,xkeyval,etoolbox}
%    \end{macrocode}
%
% Save the \cs{setkeys} macro of \xpackage{xkeyval} package (in case it was overwritten by a
% subsequent load of \xpackage{kvsetkeys} or \xpackage{keyval} for example :
%    \begin{macrocode}
\protected\def\kcmd@Xsetkeys{\XKV@sttrue\XKV@plfalse\XKV@testoptc\XKV@setkeys}% in case \setkeys 
%                                                                                was overwritten
%    \end{macrocode}
% Some \cs{catcode} assertions internally used by \thispackage:
%    \begin{macrocode}
\let\kcmd@AtEnd\@empty
\def\TMP@EnsureCode#1#2{%
  \edef\kcmd@AtEnd{%
    \kcmd@AtEnd
    \catcode#1 \the\catcode#1\relax
  }%
  \catcode#1 #2\relax
}
\TMP@EnsureCode{32}{10}% space
\TMP@EnsureCode{61}{12}% = sign
\TMP@EnsureCode{45}{12}% - sign
\TMP@EnsureCode{62}{12}% > sign
\TMP@EnsureCode{46}{12}% . dot
\TMP@EnsureCode{47}{8}% / slash (etextools)
\AtEndOfPackage{\kcmd@AtEnd\undef\kcmd@AtEnd}
%    \end{macrocode}
% 
% \begin{macro}{\kcmd@ifstrdigit}\qquad\qquad
% This macro is used too test the optional arguments of \cs{newkeycommand}, 
% in particular, one must know in an argument is a single digit (representing
% the number of mandatory arguments) or anything else (representing the \texttt{key=value} 
% list or the ``special'' \texttt{OptKey} key:
%    \begin{macrocode}
\iffalse%\ifdefined\pdfmatch% use \pdfmatch if present
   \long\def\kcmd@ifstrdigit#1{\csname @\ifnum\pdfmatch
      {\detokenize{^[[:space:]]*[[:digit:]][[:space:]]*$}}{\detokenize{#1}}=1 %
      first\else second\fi oftwo\endcsname}
\else% use filter, very efficient !
\def\kcmd@ifstrdigit#1{%
   \kcmd@nbk#1//%
      {\expandafter\expandafter\expandafter\kcmd@ifstrdigit@i
         \expandafter\expandafter\expandafter{\detokenize\expandafter{\number\number0#1}}}%
      {\@secondoftwo}//%
}
\def\kcmd@ifstrdigit@i#1{%
   \def\kcmd@ifstrdigit@ii##1#1##2##3\kcmd@ifstrdigit@ii{%
      \csname @\ifx##20first\else second\fi oftwo\endcsname
      }\kcmd@ifstrdigit@ii 00 01 02 03 04 05 06 07 08 09 0#1 \relax\kcmd@ifstrdigit@ii
}
\fi
%    \end{macrocode}
% \end{macro}
%
% \subsection{Defining (and undefining) command-keys}
%\begin{macro}{\kcmd@keyfam}\qquad
% The macro expands to the family-name, given the keycommand name:
%    \begin{macrocode}
\def\kcmd@keyfam#1{\detokenize{keycmd->}\expandafter\@gobble\string#1}
%    \end{macrocode}
% \end{macro}
% \begin{macro}{\kcmd@nbk}\qquad is the optimized \cs{ifnotblank} macro of \xpackage{etoolbox}
% (with \textttbf{/} having a catcode of 8):
%    \begin{macrocode}
\def\kcmd@nbk#1#2/#3#4#5//{#4}%
%    \end{macrocode}
% \end{macro}
%
% \begin{macro}{\kcmd@normalize@setkeys}~\par
% This macro assigns the values to the keys (expansion of \xpackage{xkeyval}-\cs{setkeys}
% on the result of \xpackage{kvsetkeys}-\cs{kv@normalize}). Braces are normalized too so that
% \verb+key=+\textvisiblespace+{{{value}}}+ is the same as \verb+key={{{value}}}+ as explained in section \ref{kvsetkeys-comparisons}:
%    \begin{macrocode}
\newrobustcmd\kcmd@normalize@setkeys[4]{%
% #1 = key-command,
% #2 = family,
% #3 = other-key,
% #4 = key-values pairs
   \kv@normalize{#4}\toks@{}%
   \expandafter\kv@parse@normalized\expandafter{\kv@list}{\kcmd@normalize@braces{#2}}%
   \edef\kv@list{\kcmd@Xsetkeys{\unexpanded{#2}}{\the\toks@}}\kv@list
   \kcmd@nbk#3//% undeclared keys are assigned to "OtherKeys"
      {\cslet{#2@#3}\XKV@rm}% (if specified, ie not empty)
      {\expandafter\kcmd@nbk\XKV@rm//% (otherwise a recoverable error is thown)
         {\PackageError\kcmd@pkg@name{The key-value pairs :\MessageBreak
         \XKV@rm\MessageBreak
         cannot be processed for key-command \string#1\MessageBreak
         See the definition of the key-command!}{}}{}//}//%
}
\long\def\kcmd@normalize@braces#1#2#3{% This is kvsetkeys processor for normalizing braces
   \toks@\expandafter{\the\toks@,#2}%
   \ifx @\detokenize{#3}@\else \toks@\expandafter{\the\toks@={{{#3}}}}\fi
}
%    \end{macrocode}
% \end{macro}
% 
% \begin{macro}{\kcmd@definekey}~\par
% \CS{kcmd@definekey} define the keys declared for the key-command.
% It is used as the \emph{processor} for the \cs{kv@parse} macro of \xpackage{kvsetkeys}.
% The macro appends the key names to the key list: ``\textit{family}.keylist''.
%
% keys are first checked for their type (bool, enum, enum*, choice or choice*) :
%
%    \begin{macrocode}
\def\kcmd@check@typeofkey#1{% expands to
% 0 if key has no type,
% 1 if boolean,
% 2 if enum*,
% 3 if enum,
% 4 if choice*,
% 5 if choice
   \kcmd@check@typeofkey@bool#1bool //%
      {\kcmd@check@typeofkey@enumst#1enum* //%
         {\kcmd@check@typeofkey@enum#1enum //%
            {\kcmd@check@typeofkey@choicest#1choice* //%
               {\kcmd@check@typeofkey@choice#1choice //%
                  05//}4//}3//}2//}1//}
\def\kcmd@check@typeofkey@bool #1bool #2//{\kcmd@nbk#1//}
\def\kcmd@get@keyname@bool #1bool #2//{#2}
\def\kcmd@check@typeofkey@enumst #1enum* #2//{\kcmd@nbk#1//}
\def\kcmd@get@keyname@enumst #1enum* #2//{#2}
\def\kcmd@check@typeofkey@enum #1enum #2//{\kcmd@nbk#1//}
\def\kcmd@get@keyname@enum #1enum #2//{#2}
\def\kcmd@check@typeofkey@choicest #1choice* #2//{\kcmd@nbk#1//}
\def\kcmd@get@keyname@choicest #1choice* #2//{#2}
\def\kcmd@check@typeofkey@choice #1choice #2//{\kcmd@nbk#1//}
\def\kcmd@get@keyname@choice #1choice #2//{#2}
%
\protected\long\def\kcmd@definekey#1#2#3#4#5{% define the keys using xkeyval macros
% #1 = keycommand,
% #2 = \global,
% #3 = family,
% #4 = key (before = sign),
% #5 = default (after = sign)
   \ifcase\kcmd@check@typeofkey{#4}\relax% standard
      #2\csedef{#3.keylist}{\csname#3.keylist\endcsname,#4}%
      \define@cmdkey{#3}[{#3@}]{#4}[{#5}]{}%
   \or% bool
      #2\csedef{#3.keylist}{\csname#3.keylist\endcsname,\kcmd@get@keyname@bool#4//}%
      \kcmd@define@boolkey#1{#3}{\kcmd@get@keyname@bool#4//}{#5}%
   \or% enum*
      #2\csedef{#3.keylist}{\csname#3.keylist\endcsname,\kcmd@get@keyname@enumst#4//}%
      \kcmd@define@choicekey#1*{#3}{\kcmd@get@keyname@enumst#4//}{#5}{\expandonce\val}%
   \or% enum
      #2\csedef{#3.keylist}{\csname#3.keylist\endcsname,\kcmd@get@keyname@enum#4//}%
      \kcmd@define@choicekey#1{}{#3}{\kcmd@get@keyname@enum#4//}{#5}{\expandonce\val}%
   \or% choice*
      #2\csedef{#3.keylist}{\csname#3.keylist\endcsname,\kcmd@get@keyname@choicest#4//}%
      \kcmd@define@choicekey#1*{#3}{\kcmd@get@keyname@choicest#4//}{#5}{\number\nr}%
   \or% choice
      #2\csedef{#3.keylist}{\csname#3.keylist\endcsname,\kcmd@get@keyname@choice#4//}%
      \kcmd@define@choicekey#1{}{#3}{\kcmd@get@keyname@choice#4//}{#5}{\number\nr}%
   \fi
   \ifx#2\global\relax
      #2\csletcs{KV@#3@#4}{KV@#3@#4}% globalize
      #2\csletcs{KV@#3@#4@default}{KV@#3@#4@default}% globalize default value
   \fi
}
%
\long\def\kcmd@firstchoiceof#1,#2\kcmd@nil{\unexpanded{#1}}
%
\long\def\kcmd@define@choicekey#1#2#3#4#5#6{%
   \begingroup\edef\kcmd@define@choicekey{\endgroup
      \noexpand\define@choicekey#2+{#3}{#4}
            [\noexpand\val\noexpand\nr]%
            {\unexpanded{#5}}% list of allowed values
            [{\kcmd@firstchoiceof#5,\kcmd@nil}]% default value
            {\csedef{#3@#4}{\unexpanded{#6}}}% define key value if in the allowed list
            {\kcmd@error@handler\noexpand#1{#3}{#4}{\kcmd@firstchoiceof#5,\kcmd@nil}}% error handler
   }\kcmd@define@choicekey
}
\def\kcmd@define@boolkey#1#2#3#4{\begingroup
   \kcmd@nbk#4//{\def\default{#4}}{\def\default{true}}//%
   \edef\kcmd@define@boolkey{\endgroup
      \noexpand\define@choicekey*+{#2}{#3}[\noexpand\val\noexpand\nr]%
            {false,true}
            [{\unexpanded\expandafter{\default}}]%
            {\csedef{#2@#3}{\noexpand\number\noexpand\nr}}%
            {\kcmd@error@handler\noexpand#1{#2}{#3}{\unexpanded\expandafter{\default}}}%
   }\kcmd@define@boolkey
}
%
\protected\long\def\kcmd@error@handler#1#2#3#4{%
% #1 = key-command,
% #2 = family,
% #3 = key,
% #4 = default
   \PackageError\kcmd@pkg@name{%
      Value `\val\space' is not allowed in key #3\MessageBreak
      for key-command \string#1.\MessageBreak
      I'll use the default value `#4' for this key.\MessageBreak
      See the definition of the key-command!}{%
      \csdef{#2@#3}{#4}}}
%    \end{macrocode}
% \end{macro}
%
% \begin{macro}{\kcmd@undefinekeys}~\par
% Now in case we redefine a key-command, we would like the old keys (\emph{ie} the keys
% associated to the old definition of the command) to be cleared, undefined.
% That's the job of \cs{kcmd@undefinekeys}.
%    \begin{macrocode}
\protected\def\kcmd@undefinekeys#1#2{% #1 = global, #2 = family
   \ifcsundef{#2.keylist}
      {\cslet{#2.keylist}\@gobble}
      {\expandafter\expandafter\expandafter\docsvlist
         \expandafter\expandafter\expandafter{%
                        \csname #2.keylist\endcsname}%
      \cslet{#2.keylist}\@gobble}%
}
\def\kcmd@undefinekey#1#2#3{% #1 = global, #2 = family, #3 = key
   #1\csundef{KV@#2@#3}%
   #1\csundef{KV@#2@#3@default}%
}
%    \end{macrocode}
% \end{macro}
% 
%\begin{macro}{\kcmd@setdefaults}\qquad\qquad
% sets the defaults values for the keys at the very beginning of the keycommand:
%    \begin{macrocode}
\def\kcmd@setdefaults#1{%
   \ifcsundef{#1.keylist}{}
   {\expandafter\expandafter\expandafter\docsvlist
      \expandafter\expandafter\expandafter{%
                           \csname#1.keylist\endcsname}}%
}
%    \end{macrocode}
%\end{macro}
% 
% 
%
% \begin{macro}{\kcmd@def}
% checks \cs{@ifdefinable} and cancels definition if needed:
%    \begin{macrocode}
\protected\long\def\kcmd@def#1#2[#3][#4][#5]#6#7{%
   \ifx#1\kcmd@donot@provide  \endgroup
   \else
      \@tempswafalse\@ifdefinable#1{\@tempswatrue}%
      \if@tempswa
         \edef\kcmd@fam{\kcmd@keyfam{#1}}%
         \expandafter\kcmd@defcommand\expandafter{\kcmd@fam}#1[{#3}][{#4}][{#5}]{#6}{#2}{#7}%
      \else\endgroup
      \fi
   \fi
}
%    \end{macrocode}
% \end{macro}
% 
% \begin{macro}{\kcmd@defcommand}\qquad\qquad prepares (expands) the arguments before closing the group opened at the very beginning.
% Then it proceeds (\cs{kcmd@yargdef} (normal interface)  or \cs{kcmd@yargedef} (when \cs{newkeycommand}\stform+ is used))
%    \begin{macrocode}
\protected\long\def\kcmd@defcommand#1#2[#3][#4][#5]#6#7#8{%
   \let\commandkey\relax  \let\getcommandkey\relax  \let#2\relax   
   \cslet{#1}\relax  \cslet{#1.commankey}\relax  \cslet{#1.getcommandkey}\relax
   \def\do{\kcmd@undefinekey{\kcmd@gbl}{#1}}%
   \edef\kcmd@defcommand{\endgroup
      \kcmd@undefinekeys{\kcmd@gbl}{#1}% undefines all keys for this keycommand family
      \ifx\kcmd@unexpandchar\@empty\else
         \kcmd@mount@unexpandchar{#1}{\unexpanded\expandafter{\kcmd@unexpandchar}}%
      \fi
      \unexpanded{\kv@parse{#3,#4}}{\kcmd@definekey\noexpand#2{\kcmd@gbl}{#1}}% defines keys
      \csdef{#1.commandkey}####1{\noexpand\csname#1@####1\endcsname}%
      \csdef{#1.getcommandkey}####1{%
         \unexpanded{\unexpanded\expandafter\expandafter\expandafter}{%
                           \noexpand\csname#1@####1\endcsname}}%
      \kcmd@ifplus% \newkeycommand+ / \newkeyenvironment+
         \protected\csdef{#1}{%
            \kcmd@yargedef{\kcmd@gbl}{\kcmd@long}\csname#1\endcsname
                          {\number#5}{\noexpand#7}{\csname#1.unexpandchar\endcsname}}%
         \ifx#7\@gobble\else 
             \protected\def#7{\kcmd@yargedef#7}%
         \fi
      \else% \newkeycommand / \newkeyenvironment
         \csdef{#1}{%
            \kcmd@yargdef{\kcmd@gbl}{\kcmd@long}\csname#1\endcsname
                          {\number#5}{\noexpand#7}}%
         \ifx#7\@gobble\else \def#7####1{% that means we have to define a key-environment
            \def#7{%
               \let\getcommandkey\csname#1.getcommandkey\endcsname
               \let\commandkey\csname#1.commandkey\endcsname
               ####1}%
            }%
         \fi
      \fi
      \def\noexpand\do####1{\unexpanded{\expandafter\noexpand\csname}KV@#1@####1@default%
                                                                                     \endcsname}% 
      \let\commandkey\relax \let\getcommandkey\relax \let#2\relax
      \kcmd@gbl\protected\edef#2{% entry point
         \let\getcommandkey\noexpand\noexpand\csname#1.getcommandkey\endcsname
         \kcmd@ifplus  \let\commandkey\getcommandkey
         \else         \let\commandkey\noexpand\noexpand\csname#1.commandkey\endcsname
         \fi
         \noexpand\kcmd@setdefaults{#1}%
         \ifx#7\@gobble \noexpand\noexpand\noexpand\@testopt
                        {\kcmd@setkeys#2{#1}{\kcmd@otherkey{#4}}}{}%
         \else          \noexpand\noexpand\noexpand\@testopt
                        {\kcmd@setkeys#2{#1}{\kcmd@otherkey{#4}}}{}%
         \fi
         }%
      \csname#1\endcsname% expand \kcmd@yargedef or \kcmd@yargdef
   }\kcmd@defcommand{#6}{#8}% #6 = definition, #8 = definition end-envir
}
\protected\long\def\kcmd@setkeys#1#2#3[#4]{% #1=key-command, #2=family, #3=otherkey, #4=key=value pairs
   \kcmd@normalize@setkeys{#1}{#2}{#3}{#4}\csname#2\endcsname
}
\long\def\kcmd@otherkey#1{\kcmd@nbk#1//{\kcmd@otherkey@name#1=\kcmd@nil}{}//}
\long\def\kcmd@otherkey@name#1=#2\kcmd@nil{#1}
%    \end{macrocode}
% \end{macro}
%
% \begin{macro}{\kcmd@mount@unexpandchar}~\par
% This macro defines the macro \cs{"\textit{family.unexpandchar}"}. 
% \CS{"\textit{family.unexpandchar}"} activates the shortcut character 
% for \cs{unexpanded} and defines its meaning.
%    \begin{macrocode}
\protected \def \kcmd@mount@unexpandchar#1#2{%
   \protected\csdef{#1.unexpandchar}{\begingroup
      \catcode`\~\active \lccode`\~`#2 \lccode`#2 0\relax
         \lowercase{%
            \expandafter\endgroup\expandafter\def\expandafter~{%
               \catcode`#2\active
               \long\def~########1~{\unexpanded{########1}}}%
         ~}%
   }%
}
%    \end{macrocode}
% \end{macro}
%
%----------------------------------------------------------------------------
% \begin{macro}{\kcmd@yargdef}\qquad\qquad
% This is the ``{\tt argdef}'' macro for the normal (non \string+) form:
%    \begin{macrocode}
\protected \def \kcmd@yargdef #1#2#3#4#5{\begingroup
% #1 = global or {}
% #2 = long or {}
% #3 = Command
% #4 = nr of args
% #5 = endenvir (or \@gobble if not an environment, or \relax if #3 is endenvir)
   \def \kcmd@yargd@f ##1#4##2##{\afterassignment#5\endgroup
      #1#2\expandafter\def\expandafter#3\@gobble ##1#4%
   }\kcmd@yargd@f 0##1##2##3##4##5##6##7##8##9###4%
}
%    \end{macrocode}
% \end{macro}
%
% \begin{macro}{\kcmd@yargedef}\qquad\qquad
% This is the ``{\tt argdef}'' macro for the {\rred\bf\string+} form:
%    \begin{macrocode}
\protected \def \kcmd@yargedef#1#2#3#4#5#6{\begingroup
% #1 = global or {}
% #2 = long or {}
% #3 = Command
% #4 = nr of args
% #5 = endenvir (or \@gobble if not an environment, or \relax if #3 is endenvir)
% #6 = unexpandchar mounting macro
  \kcmd@nargs{#4}% 
   \protected\long\def\kcmd@yarg@edef##1##2{\endgroup
         #1\edef#3{\begingroup #6%
            #2\edef#3\unexpanded{##2}{\endgroup\unexpanded{##1}%
         }#3}%
   }%
   \protected\def\kcmd@envir##1{%
      \edef\next{\kcmd@yarg@edef{\def\noexpand#5{\expandonce{#5##1}}\expandonce{#3##1}}}\next
   }%
   \protected\def\kcmd@command##1{%
      \edef\next{\kcmd@yarg@edef{\expandonce{#3##1}}}\next
   }%
   \protected\def\kcmd@yargedef##1{%
      \kcmd@yargedef@##1 0####1####2####3####4####5####6####7####8####9#####4%
   }%
   \ifx#5\@gobble % keycommand
      \def\next{\kcmd@command}%
   \else          % key-environmment
      \def\next{\kcmd@envir}%
   \fi
   \let\@next\relax
   \def\kcmd@yargedef@##1##2#4##3##{%
      \ifx\@next\relax 
         \edef\@next{\next{\expandonce{\kcmd@nargs}}{\expandonce{\@gobble##2#4}}}%
         \ifx#5\@gobble \edef\@next{\expandonce\@next\noexpand#5}%
         \else \edef\@next{\edef\noexpand\@next{\noexpand\unexpanded{\expandonce\@next}}#5}%
         \fi
      \fi
      \afterassignment\@next
      \expandafter\def\expandafter##1\@gobble##2#4%
   }%
   \kcmd@yargedef#3%
}
%    \end{macrocode}
% \end{macro}
%
% \begin{macro}{\kcmd@nargs}\qquad
% Filter macros used by \cs{kcmd@yargedef} to get the correct number of arguments:
%    \begin{macrocode}
\def\kcmd@nargs#1{\edef\kcmd@nargs%##1##2##3##4##5##6##7##8##9%
        {\ifnum#1>0{####1%
         \ifnum#1>1}{####2%
         \ifnum#1>2}{####3%
         \ifnum#1>3}{####4%
         \ifnum#1>4}{####5%
         \ifnum#1>5}{####6%
         \ifnum#1>6}{####7%
         \ifnum#1>7}{####8%
         \ifnum#1>8}{####9%
         \fi\fi\fi\fi\fi\fi\fi\fi}\fi}%
}%
%    \end{macrocode}
% \end{macro}
%
% \subsection{new key-commands}
%
% \begin{macro}{\newkeycommand}\qquad\qquad
% Here are the entry points:
%    \begin{macrocode}
\newrobustcmd*\newkeycommand{\begingroup
   \let\kcmd@gbl\@empty\kcmd@star@or@long\new@keycommand}
\newrobustcmd*\renewkeycommand{\begingroup
   \let\kcmd@gbl\@empty\kcmd@star@or@long\renew@keycommand}
\newrobustcmd*\providekeycommand{\begingroup
   \let\kcmd@gbl\@empty\kcmd@star@or@long\provide@keycommand}
%    \end{macrocode}
% \end{macro}
%
% \begin{macro}{\kcmd@star@or@long}~\par
% This is the adaptation of \LaTeX's \cs{@star@or@long} macro:
%    \begin{macrocode}
\def\kcmd@star@or@long#1{\@ifstar
      {\let\kcmd@long\@empty\kcmd@plus#1}
      {\def\kcmd@long{\long}\kcmd@plus#1}}
\def\kcmd@@ifplus#1{\@ifnextchar +{\@firstoftwo{#1}}}% same as LaTeX's \@ifstar
\def\kcmd@plus#1{\kcmd@@ifplus
      {\def\kcmd@ifplus{\iftrue}\kcmd@testopt#1}
      {\def\kcmd@ifplus{\iffalse}\kcmd@testopt#1}}
\def\kcmd@testopt#1{\@testopt{\kcmd@unexpandchar#1}{}}
%    \end{macrocode}
% \end{macro}
%
%\begin{macro}{\kcmd@unexpandchar}\qquad\qquad\quad
% Reads the possible unexpand-char shortcut:
%    \begin{macrocode}
\def\kcmd@unexpandchar#1[#2]{%
   \kcmd@ifplus
      \kcmd@nbk#2//
         {\def\kcmd@unexpandchar{#2}% only once inside group...
          \def\kcmd@unexpandchar@activate{\catcode`#2 \active}%
         }{%
          \let\kcmd@unexpandchar\@empty
          \let\kcmd@unexpandchar@activate\relax
         }//%
   \else  \let\kcmd@unexpandchar\@empty
      \kcmd@nbk#2//%
         {\PackageError\kcmd@pkg@name{shortcut option for \string\unexpanded\MessageBreak
         You can't use a shortcut option for \string\unexpanded\MessageBreak
         without the \string+ form of \string\newkeycommand!}%
         {I will ignore this option and proceed.}%
         }%
         {}//%      
   \fi#1}
%    \end{macrocode}
%\end{macro}
%
% \begin{macro}{\new@keycommand}\qquad\qquad
% Reads the key-command name (cs-token):
%    \begin{macrocode}
\def\new@keycommand#1{\@testopt{\@newkeycommand#1}0}
%    \end{macrocode}
% \end{macro}
%
%\begin{macro}{\@newkeycommand}\qquad\qquad
% Reads the first optional parameter (keys or number of mandatory args):
%    \begin{macrocode}
\long\def\@newkeycommand#1[#2]{% #2 = key=values or N=mandatory args
   \kcmd@ifplus \kcmd@unexpandchar@activate \fi% activates unexpand-char before reading definition
   \kcmd@ifstrdigit{#2}%
      {\@new@key@command#1[][][{#2}]}% no kv, no optkey, number of args
      {\@testopt{\@new@keycommand#1[{#2}]}0}}% kv, check for optkey/nr of args
%    \end{macrocode}
% \end{macro}
%
%\begin{macro}{\@new@keycommand}\qquad\qquad
% Reads the second optional parameter (opt key or number of mandatory args):
%    \begin{macrocode}
\long\def\@new@keycommand#1[#2][#3]{%
   \kcmd@ifstrdigit{#3}%
      {\@new@key@command#1[{#2}][][{#3}]}% no optkey
      {\@testopt{\@new@key@command#1[{#2}][{#3}]}0}}
%    \end{macrocode}
%\end{macro}
%
%\begin{macro}{\@new@key@command}\qquad\qquad
% Reads the definition of the command (\cs{kcmd@def} handles both cases of commands and environements).
% The so called "unexpand-char shortcut" has been activated before reading command definition:
%    \begin{macrocode}
\long\def\@new@key@command#1[#2][#3][#4]#5{%
      \kcmd@def#1\@gobble[{#2}][{#3}][{#4}]{#5}{}}
%    \end{macrocode}
%\end{macro}
%
% \begin{macro}{\renew@keycommand}
%    \begin{macrocode}
\def\renew@keycommand#1{\begingroup
   \escapechar\m@ne\edef\@gtempa{{\string#1}}%
   \expandafter\@ifundefined\@gtempa
      {\endgroup\@latex@error{\noexpand#1undefined}\@ehc}
      \endgroup
   \let\@ifdefinable\@rc@ifdefinable
   \new@keycommand#1%
}
%    \end{macrocode}
% \end{macro}
%
% \begin{macro}{\provide@keycommand}
%    \begin{macrocode}
\def\provide@keycommand#1{\begingroup
   \escapechar\m@ne\edef\@gtempa{{\string#1}}%
   \expandafter\@ifundefined\@gtempa
      {\endgroup\new@keycommand#1}
      {\endgroup\def\kcmd@donot@provide{\renew@keycommand\kcmd@donot@provide
         }\kcmd@donot@provide}%
}
\let\kcmd@donot@provide\@empty% it must not be undefined
%    \end{macrocode}
% \end{macro}
%
% \subsection{new key-environments}
%
% \begin{macro}{\newkeyenvironment}
%    \begin{macrocode}
\newrobustcmd*\newkeyenvironment{\begingroup
   \let\kcmd@gbl\@empty\kcmd@star@or@long\new@keyenvironment}
\newrobustcmd\renewkeyenvironment{\begingroup
   \let\kcmd@gbl\@empty\kcmd@star@or@long\renew@keyenvironment}
%    \end{macrocode}
% \end{macro}
%
% \begin{macro}{\new@keyenvironment}
%    \begin{macrocode}
\def\new@keyenvironment#1{\@testopt{\@newkeyenva{#1}}{}}
\long\def\@newkeyenva#1[#2]{%
   \kcmd@ifstrdigit{#2}%
      {\@newkeyenv{#1}{[][][{#2}]}}
      {\@testopt{\@newkeyenvb{#1}[{#2}]}{}}}
\long\def\@newkeyenvb#1[#2][#3]{%
   \kcmd@ifstrdigit{#3}%
      {\@newkeyenv{#1}{[{#2}][][{#3}]}}
      {\@testopt{\@newkeyenvc{#1}{[{#2}][{#3}]}}0}}
\long\def\@newkeyenvc#1#2[#3]{\@newkeyenv{#1}{#2[{#3}]}}
\long\def\@newkeyenv#1#2{%
   \kcmd@ifplus \kcmd@unexpandchar@activate \fi
   \kcmd@keyenvir@def{#1}{#2}%
}
\long\def\kcmd@keyenvir@def#1#2#3#4{%
   \expandafter\let\csname end#1\endcsname\relax
   \expandafter\kcmd@def\csname #1\expandafter\endcsname\csname end#1\endcsname#2{#3}{#4}%
}
%    \end{macrocode}
% \end{macro}
%
% \begin{macro}{\renew@keyenvironment}
%    \begin{macrocode}
\def\renew@keyenvironment#1{%
  \@ifundefined{#1}%
     {\@latex@error{Environment #1 undefined}\@ehc
     }\relax
  \cslet{#1}\relax
  \new@keyenvironment{#1}}
%    \end{macrocode}
% \end{macro}
% \iffalse
%<package>
%<package>
% \fi
%
% \subsection{Tests on keys}
%
% \begin{macro}{\ifcommandkey}\qquad
% \{\meta{key-name}\}\{\meta{true}\}\{\meta{false}\}\quad expands \meta{true} only if the value of the key
% is not blank:
%    \begin{macrocode}
\newcommand*\ifcommandkey[1]{\csname @\expandafter\expandafter\expandafter
   \kcmd@nbk\commandkey{#1}//{first}{second}//%
   oftwo\endcsname}
%    \end{macrocode}
% \end{macro}
%
%
% \begin{macro}{\showcommandkeys}\qquad\qquad are helper macros essentially for debuging purpose...
%    \begin{macrocode}
\newrobustcmd*\showcommandkeys[1]{\let\do\showcommandkey\docsvlist{#1}}
\newrobustcmd*\showcommandkey[1]{key \string"#1\string" = %
   \detokenize\expandafter\expandafter\expandafter{\commandkey{#1}}\par}
%    \end{macrocode}
% \end{macro}
% 
%
%    \begin{macrocode}
%</package>
%    \end{macrocode}
%
% \section{Examples}
% \label{sec:examples}
%
%    \begin{macrocode}
%<*example>
\ProvidesFile{keycommand-example}
\documentclass[a4paper]{article}
\usepackage[T1]{fontenc}
\usepackage[latin1]{inputenc}
\usepackage[american]{babel}
\usepackage{keycommand,framed,fancyvrb}
%
\makeatletter
\parindent\z@
\newkeycommand*\Rule[raise=.4ex,width=1em,thick=.4pt][1]{%
   \rule[\commandkey{raise}]{\commandkey{width}}{\commandkey{thick}}%
   #1%
   \rule[\commandkey{raise}]{\commandkey{width}}{\commandkey{thick}}}

\newkeycommand*\charleads[sep=1][2]{%
   \ifhmode\else\leavevmode\fi\setbox\@tempboxa\hbox{#2}\@tempdima=1.584\wd\@tempboxa%
   \cleaders\hb@xt@\commandkey{sep}\@tempdima{\hss\box\@tempboxa\hss}#1%
   \setbox\@tempboxa\box\voidb@x}
\newcommand*\charfill[1][]{\charleads[{#1}]{\hfill\kern\z@}}
\newcommand*\charfil[1][]{\charleads[{#1}]{\hfil\kern\z@}}
%
\newkeyenvironment*{dblruled}[first=.4pt,second=.4pt,sep=1pt,left=\z@]{%
   \def\FrameCommand{%
      \vrule\@width\commandkey{first}%
      \hskip\commandkey{sep}
      \vrule\@width\commandkey{second}%
      \hspace{\commandkey{left}}}%
   \parindent\z@
   \MakeFramed {\advance\hsize-\width \FrameRestore}}
   {\endMakeFramed}
%
\makeatother
\begin{document}
\title{This is {\tt keycommand-example.tex}}
\author{Florent Chervet}
\date{July 22, 2009}

\maketitle

{\Large Please refer to {\tt keycommand-example.tex} for definitions.}

\section{Example of a keycommand : \texttt{\string\Rule}}

\begin{tabular*}\textwidth{rl}
\verb+\Rule[width=2em]{hello}+:&\Rule[width=2em]{hello}\cr
\verb+\Rule[thick=1pt,width=2em]{hello}+:&\Rule[thick=1pt,width=2em]{hello}\cr
\verb+\Rule{hello}+:&\Rule{hello}\cr
\verb+\Rule[thick=1pt,raise=1ex]{hello}+:&\Rule[thick=1pt,raise=1ex]{hello}
\end{tabular*}

\section{Example of a keycommand : \texttt{\string\charfill}}

\begin{tabular*}\textwidth{rp{.4\textwidth}}
\verb+\charfill{$\star$}+: & \charfill{$\star$}\cr
\verb+\charfill[sep=2]{$\star$}+: & \charfill[sep=2]{$\star$} \\
\verb+\charfill[sep=.7]{\textasteriskcentered}+: & \charfill[sep=.7]{\textasteriskcentered}
\end{tabular*}


\section{Example of a keyenvironment : \texttt{dblruled}}

Key environment \texttt{dblruled } uses \texttt{framed.sty} and therefore it can be used 
even if a pagebreak occurs inside the environment:
\medskip

\verb+\begin{dblruled}+\par
\verb+   test for dblruled key-environment\par+\par
\verb+   test for dblruled key-environment\par+\par
\verb+   test for dblruled key-environment+\par
\verb+\end{dblruled}+

\begin{dblruled}
 test for dblruled key-environment\par
 test for dblruled key-environment\par
 test for dblruled key-environment
\end{dblruled}


\verb+\begin{dblruled}[first=4pt,sep=2pt,second=.6pt,left=.2em]+\par
\verb+   test for dblruled key-environment\par+\par
\verb+   test for dblruled key-environment\par+\par
\verb+   test for dblruled key-environment+\par
\verb+\end{dblruled}+

\begin{dblruled}[first=4pt,sep=2pt,second=.6pt,left=.2em]
 test for dblruled key-environment\par
 test for dblruled key-environment\par
 test for dblruled key-environment
\end{dblruled}

\end{document}
%</example>
%    \end{macrocode}
% \DeleteShortVerb{\+}^^A\UndefineShortVerb{\+}
% \begin{History}
% 
%   \begin{Version}{2010/04/27 v3.1415}
%   \item Key-environment can now be nested ! (it's not too late... I hope so)
%   \item Keys and mandatory arguments as well can be used in both \texttt{begin} end \texttt{end} part of the environment.
%   \end{Version}
% 
%   \begin{Version}{2010/04/25 v3.141}
%   \item No new feature but a real improvement in optimization. \\
%         In particular, \thispackage does not load \xpackage{etextools} anymore. \\
%   \item Bug fix for \cs{providekeycommand}.
%         
%   \end{Version}
%
%   \begin{Version}{2010/04/18 v3.14}
%   \item Correction of bug in the normalization process. \\
%         Correction of a bug in \cs{ifcommandkey} (undesirable space...)
%   \item Modification of the pdf documentation for the \stform+ form of key-environments.
%   \end{Version}
%
%   \begin{Version}{2010/03/28 v3.0}
%   \item Complete redesign of the implementation. \\
%   \xpackage{keycommand} is now based on some macros of \xpackage{etoolbox}.
%
%   \item Adding the + prefix and the ability to capture keys that where not defined.
%
%   \end{Version}
%
%   \begin{Version}{2009/07/22 v1.0}
%   \item
%     First version.
%   \end{Version}
%
% \end{History}
%
% \begin{thebibliography}{9}
%
% \bibitem{xkeyval}
%   Hendri Adriaens:
%   \textit{The \xpackage{xkeyval} package};
%   2008/08/13 v2.6a;
%   \CTAN{macros/latex/contrib/xkeyval.dtx}
%
% \bibitem{kvsetkeys}
%   Heiko Oberdiek:
%   \textit{The \xpackage{kvsetkeys} package};
%   2007/09/29 v1.3;
%   \CTAN{macros/latex/contrib/oberdiek/kvsetkeys.dtx}.
%
% \bibitem{keyval}
%   David Carlisle:
%   \textit{The \xpackage{keyval} package};
%   1999/03/16 v1.13;
%   \CTAN{macros/latex/required/graphics/keyval.dtx}.
%
% \end{thebibliography}
%
% \PrintIndex
%
% \label{LastPage}
% \Finale
%        (quote the arguments according to the demands of your shell)
%
% Documentation:
%           (pdf)latex keycommand.dtx
% Copyright (C) 2009-2010 by Florent Chervet <florent.chervet@free.fr>
%<*ignore>
\begingroup
  \def\x{LaTeX2e}%
\expandafter\endgroup
\ifcase 0\ifx\install y1\fi\expandafter
         \ifx\csname processbatchFile\endcsname\relax\else1\fi
         \ifx\fmtname\x\else 1\fi\relax
\else\csname fi\endcsname
%</ignore>
%<*install>
\input docstrip.tex
\Msg{************************************************************************}
\Msg{* Installation}
\Msg{* Package: keycommand 2010/04/27 v3.1415 key-value interface for commands and environments in LaTeX}
\Msg{************************************************************************}

\keepsilent
\askforoverwritefalse

\let\MetaPrefix\relax
\preamble

This is a generated file.

keycommand : key-value interface for commands and environments in LaTeX [v3.1415 2010/04/27]

This work may be distributed and/or modified under the
conditions of the LaTeX Project Public License, either
version 1.3 of this license or (at your option) any later
version. The latest version of this license is in
   http://www.latex-project.org/lppl.txt

This work consists of the main source file keycommand.dtx
and the derived files
   keycommand.sty, keycommand.pdf, keycommand.ins,
   keycommand-example.tex

keycommand : an easy way to define commands with optional keys
Copyright (C) 2009-2010 by Florent Chervet <florent.chervet@free.fr>

\endpreamble
\let\MetaPrefix\DoubleperCent

\generate{%
   \file{keycommand.ins}{\from{keycommand.dtx}{install}}%
   \file{keycommand.sty}{\from{keycommand.dtx}{package}}%
   \file{keycommand-example.tex}{\from{keycommand.dtx}{example}}%
}

\generate{%
   \file{keycommand.drv}{\from{keycommand.dtx}{driver}}%
}

\obeyspaces
\Msg{************************************************************************}
\Msg{*}
\Msg{* To finish the installation you have to move the following}
\Msg{* file into a directory searched by TeX:}
\Msg{*}
\Msg{*     keycommand.sty}
\Msg{*}
\Msg{* To produce the documentation run the file `keycommand.dtx'}
\Msg{* through LaTeX.}
\Msg{*}
\Msg{* Happy TeXing!}
\Msg{*}
\Msg{************************************************************************}

\endbatchfile
%</install>
%<*ignore>
\fi
%</ignore>
%<*driver>
\edef\thisfile{\jobname}
\def\thisinfo{key-value interface for commands and environments in \LaTeX.}
\def\thisdate{2010/04/27}
\def\thisversion{3.1415}
\let\loadclass\LoadClass
\def\LoadClass#1{\loadclass[abstracton]{scrartcl}\let\scrmaketitle\maketitle\AtEndOfClass{\let\maketitle\scrmaketitle}}
\documentclass[a4paper,oneside]{ltxdoc}
\usepackage[latin9]{inputenc}
\usepackage[american]{babel}
\usepackage[T1]{fontenc}
\usepackage{etex,etoolbox,holtxdoc,geometry,tocloft,graphicx,xspace,fancyhdr,color,bbding,embedfile,framed,multirow,txfonts,makecell,enumitem,arydshln}
\CodelineNumbered
\usepackage{keyval}\makeatletter\let\keyval@setkeys\setkeys\makeatother
\usepackage{xkeyval}\let\xsetkeys\setkeys
\usepackage{kvsetkeys}
\usepackage{fancyvrb}
\lastlinefit999
\geometry{top=2cm,headheight=1cm,headsep=.3cm,bottom=1.4cm,footskip=.5cm,left=2.5cm,right=1cm}
\hypersetup{%
  pdftitle={The keycommand package},
  pdfsubject={key-value interface for commands and environments in LaTeX.},
  pdfauthor={Florent CHERVET},
  colorlinks,linkcolor=reflink,
  pdfstartview={FitH},
  pdfkeywords={tex, e-tex, latex, package, keys, keycommand, newcommand, keyval, kvsetkeys, programming},
  bookmarksopen=true,bookmarksopenlevel=3}
\embedfile{\thisfile.dtx}
\begin{document}
   \DocInput{\thisfile.dtx}
\end{document}
%</driver>
% \fi
%
% \CheckSum{1111}
%
% \CharacterTable
%  {Upper-case    \A\B\C\D\E\F\G\H\I\J\K\L\M\N\O\P\Q\R\S\T\U\V\W\X\Y\Z
%   Lower-case    \a\b\c\d\e\f\g\h\i\j\k\l\m\n\o\p\q\r\s\t\u\v\w\x\y\z
%   Digits        \0\1\2\3\4\5\6\7\8\9
%   Exclamation   \!     Double quote  \"     Hash (number) \#
%   Dollar        \$     Percent       \%     Ampersand     \&
%   Acute accent  \'     Left paren    \(     Right paren   \)
%   Asterisk      \*     Plus          \+     Comma         \,
%   Minus         \-     Point         \.     Solidus       \/
%   Colon         \:     Semicolon     \;     Less than     \<
%   Equals        \=     Greater than  \>     Question mark \?
%   Commercial at \@     Left bracket  \[     Backslash     \\
%   Right bracket \]     Circumflex    \^     Underscore    \_
%   Grave accent  \`     Left brace    \{     Vertical bar  \|
%   Right brace   \}     Tilde         \~}
%
% \DoNotIndex{\begin,\CodelineIndex,\CodelineNumbered,\def,\DisableCrossrefs,\~,\@ifpackagelater}
% \DoNotIndex{\DocInput,\documentclass,\EnableCrossrefs,\end,\GetFileInfo}
% \DoNotIndex{\NeedsTeXFormat,\OnlyDescription,\RecordChanges,\usepackage}
% \DoNotIndex{\ProvidesClass,\ProvidesPackage,\ProvidesFile,\RequirePackage}
% \DoNotIndex{\filename,\fileversion,\filedate,\let}
% \DoNotIndex{\@listctr,\@nameuse,\csname,\else,\endcsname,\expandafter}
% \DoNotIndex{\gdef,\global,\if,\item,\newcommand,\nobibliography}
% \DoNotIndex{\par,\providecommand,\relax,\renewcommand,\renewenvironment}
% \DoNotIndex{\stepcounter,\usecounter,\nocite,\fi}
% \DoNotIndex{\@fileswfalse,\@gobble,\@ifstar,\@unexpandable@protect}
% \DoNotIndex{\AtBeginDocument,\AtEndDocument,\begingroup,\endgroup}
% \DoNotIndex{\frenchspacing,\MessageBreak,\newif,\PackageWarningNoLine}
% \DoNotIndex{\protect,\string,\xdef,\ifx,\texttt,\@biblabel,\bibitem}
% \DoNotIndex{\z@,\wd,\wheremsg,\vrule,\voidb@x,\verb,\bibitem}
% \DoNotIndex{\FrameCommand,\MakeFramed,\FrameRestore,\hskip,\hfil,\hfill,\hsize,\hspace,\hss,\hbox,\hb@xt@,\endMakeFramed,\escapechar}
% \DoNotIndex{\do,\date,\if@tempswa,\@tempdima,\@tempboxa,\@tempswatrue,\@tempswafalse,\ifdefined,\ifhmode,\ifmmode,\cr}
% \DoNotIndex{\box,\author,\advance,\multiply,\Command,\outer,\next,\leavevmode,\kern,\title,\toks@,\trcg@where,\tt}
% \DoNotIndex{\the,\width,\star,\space,\section,\subsection,\textasteriskcentered,\textwidth}
% \DoNotIndex{\",\:,\@empty,\@for,\@gtempa,\@latex@error,\@namedef,\@nameuse,\@tempa,\@testopt,\@width,\\,\m@ne,\makeatletter,\makeatother}
% \DoNotIndex{\maketitle,\parindent,\setbox,\x,\kernel@ifnextchar}
% \DoNotIndex{\KVS@CommaComma,\KVS@CommaSpace,\KVS@EqualsSpace,\KVS@Equals,\KVS@Global,\KVS@SpaceEquals,\KVS@SpaceComma,\KVS@Comma}
% \DoNotIndex{\DefineShortVerb,\DeleteShortVerb,\UndefineShortVerb,\MakeShortVerb,\endinput}
% \let\ClearPage\clearpage
% \makeatletter
% \MakeShortVerb{\+}\DeleteShortVerb{\|}\DefineShortVerb{\|}
% \catcode`\� \active   \def�{\@ifnextchar �{\par\nobreak\vskip-2\parskip}{\par\nobreak\vskip-\parskip}}
% \def\thispackage{\xpackage{\thisfile}\xspace}
% \def\ThisPackage{\Xpackage{\thisfile}\xspace}
% \def\Xpackage{\@dblarg\X@package}
% \def\X@package[#1]#2{%
%     \xpackage{#2\footnote{\noindent\xpackage{#2}: \href{http://www.ctan.org/tex-archive/macros/latex/contrib/#1}{\nolinkurl{CTAN:macros/latex/contrib/#1}}}}}
% \def\Underbrace#1_#2{$\underbrace{\vtop to2ex{}\hbox{#1}}_{\footnotesize\hbox{#2}}$}
%
% \parindent\z@\parskip.4\baselineskip\topsep\parskip\partopsep\z@
% \g@addto@macro\macro@font{\macrocodecolor\let\AltMacroFont\macro@font}
% \g@addto@macro\@list@extra{\parsep\parskip\topsep\z@\itemsep\z@}
% \def\smex{\leavevmode\hb@xt@2em{\hfil$\longrightarrow$\hfil}}
% \newrobustcmd\verbfont{\usefont{T1}{\ttdefault}{\f@series}{n}}    \let\vb\verbfont
% \renewrobustcmd\#[1]{{\usefont{T1}{pcr}{bx}{n}\char`\##1}}
% \newrobustcmd\csred[1]{\textcolor{red}{\cs{#1}}}
% \renewrobustcmd\cs[2][]{\mbox{\vb#1\expandafter\@gobble\string\\#2}}
% \newrobustcmd\CSbf[1]{\textbf{\CS{#1}}}
% \newrobustcmd\csbf[2][]{\textbf{\cs[{#1}]{#2}}}
% \newrobustcmd\textttbf[1]{\textbf{\texttt{#1}}}
% \renewrobustcmd*\bf{\bfseries}\newcommand\nnn{\normalfont\mdseries\upshape}\newcommand\nbf{\normalfont\bfseries\upshape}
% \newrobustcmd*\blue{\color{blue}}\newcommand*\red{\color{dr}}\newcommand*\green{\color{green}}\newcommand\rred{\color{red}}
% \newrobustcmd\rrbf{\color{red}\bfseries}
% \definecolor{copper}{rgb}{0.67,0.33,0.00}  \newcommand\copper{\color{copper}}
% \definecolor{dg}{rgb}{0.16,0.33,0.00}      \newcommand\dg{\color{dg}}
% \definecolor{db}{rgb}{0,0,0.502}           \newcommand\db{\color{db}}
% \definecolor{dr}{rgb}{0.49,0.00,0.00}      \let\dr\red
% \newrobustcmd\bk{\color{black}}\newcommand\md{\mdseries}
%
% \fancyhf{}\fancyhead[L]{The \thispackage package -- \thisinfo}
% \fancyfoot[L]{\color[gray]{.35}\scriptsize\thispackage\quad[rev.\thisversion]\quad\copyright\oldstylenums{2009-2010}\,\lower.3ex\hbox{\NibRight}\,Florent Chervet}
% \fancyfoot[R]{\oldstylenums{\thepage} / \oldstylenums{\pageref{LastPage}}}
% \pagestyle{fancy}
% \fancypagestyle{plain}{%
%     \let\headrulewidth\z@
%     \fancyhf{}%
%     \fancyfoot[R]{\oldstylenums{\thepage} / \oldstylenums{\pageref{LastPage}}}}
%
% \newcommand\macrocodecolor{\color{macrocode}}\definecolor{macrocode}{rgb}{0.18,0.00,0.45}
% \newcommand\reflinkcolor{\color{reflink}}\definecolor{reflink}{rgb}{0.49,0.00,0.00}
% \font\umrandA=umranda at 20pt
% \def\@serp{\leavevmode\lower20pt\hbox{\umrandA\char'131}}
% \def\serp#1{\@serp\hfil #1\hfil\reflectbox{\@serp}}
% \newrobustcmd\stform{\@ifnextchar*{\@stform[]\textasteriskcentered\@gobble}\@stform}
% \newrobustcmd\@stform[2][\string]{\textttbf{\rred#1#2}\xspace}
%
% \makeatother
%
% \deffootnote{1em}{0pt}{\rlap{\textsuperscript{\thefootnotemark}}\kern1em}
%
% \title{\vskip-18pt\mdseries {\bfseries\ThisPackage}\kern.6em package}
% \author{\footnotesize\xemail{florent.chervet@free.fr}}
% \date{\thisdate~--~version \thisversion}
% \subtitle{\thisinfo}
% ^^A\subject{\vskip-2cm\serp{The completely redesigned}}
% \subject{\vskip-2cm\relax The \textit{free} and \textit{open source}}
%
% \maketitle
% 
% \makeatletter\begingroup\let\@thefnmark\@empty\let\@makefntext\@firstofone
% \footnotetext{\noindent
% This documentation is produced with the +DocStrip+ utility.
% \begin{tabbing}
% \qquad\=\smex\=To get the documentation, \= run (thrice):\quad\= \texttt{pdflatex keycommand.dtx} \\
% \qquad\>\>To get the index, \> run:\>\texttt{makeindex -s gind.ist keycommand.idx} \\
% \>\smex\>To get the package, \> run:\>        \texttt{etex keycommand.dtx}
% \end{tabbing}�
% The \xext{dtx} file is embedded into this pdf file thank to \xpackage{embedfile} by H. Oberdiek.}
% \endgroup\makeatother
% 
% \hypersetup{bookmarksopenlevel=3}
% \deffootnote{1em}{0pt}{\rlap{\thefootnotemark.}\kern1em}
% \vspace*{-18pt}
% \begin{abstract}\parindent0pt\noindent\leftskip1cm\rightskip\leftskip\lastlinefit0%
%
% \thispackage provides an easy way to define commands or environments
% with optional keys.
% \smallskip
%
% \csbf{newkeycommand} \cs{renewkeycommand} \cs{providekeycommand} and \csbf{newkeyenvironment},\linebreak 
% \cs{renewkeyenvironment} are macros to define such commands and environments with keys.
% 
% \thispackage is designed to make easier interface for user-defined commands.  In particular,
% \csbf{newkeycommand}\stform+ permits the use of key-commands in every context. 
% \medskip
%
% Keys are defined with the command itself in a very natural way.
% You can restrict the possible values for the keys by declaring them with a \textbf{type}.
% Available types for keys are : \textit{boolean}, \textit{enum} and \textit{choice} (see \ref{subsec:GeneralSyntax}).
%
% \smallskip
%
% The \thispackage package requires and is based on the package \xpackage{xkeyval} by Hendri Adriaens, and uses
% the \cs{kv@normalize} macro of \xpackage{kvsetkeys} (Heiko Oberdiek) for robustness, as shown
% in \ref{kvsetkeys-comparisons}).
%
% It works with an \eTeX{} distribution of \LaTeX.
% \end{abstract}
%
% \DeleteShortVerb{\+}\enlargethispage{2\baselineskip}
% \cftbeforesecskip=4pt plus2pt minus2pt
% \cftbeforesubsecskip=0pt plus2pt minus2pt
% \renewcommand\contentsname{Contents\quad\leaders\vrule height3.4pt depth-3pt\hfill\null\kern0pt\vskip-6pt}
% ^^A\vskip-.8\baselineskip
% \tableofcontents
%
% \clearpage\MakeShortVerb{\+}
%
% \def\B#1{\texttt{[}\meta{#1}\texttt{]}}
%
% \section{User Interface}
%
% \subsection{General syntax}\label{subsec:GeneralSyntax}
%
% \begin{declcs}{newkeycommand}%
%  \Underbrace{\textcolor{red}{\textasteriskcentered\string+[short-unexpand]}}_{\makecell[c]{modifiers \\ Optional}}\,%
%  \Underbrace{\M{command}}_{Required}\,%
%   {\color{db}\Underbrace{\B{keys=defaults}\,\B{OptKey}\,\B{<n>}}_{Optional}\,}%
%   \Underbrace{\M{definition}}_{Required}
% \end{declcs}
%
% \cs{newkeycommand} will define \cs{command} as a new key-command!\quad well...
%
% Use the \stform* form when you do not want it to be a \cs{long} macro (as for \LaTeX{}-\cs{newcommand}).
%
% The +[keys=defaults]+ argument define the keys with their default values. It is optional, but a key-command
% without keys seems to be useless (at least for me...). Keys may be defined as :
%
% \newlist{myenum}{enumerate}{1}
% \setlist[myenum]{label={},topsep=-\parskip,itemsep=-\parskip,parsep=\parskip,after=\vskip-\baselineskip}
%
% \renewcommand\theadfont{\tt\bfseries}
% \noindent\begin{tabular}{|c|>{\db}c|m{8cm}|}\hline
% \thead{Type} & \thead{exemple}                                              & \thead{value of \cs{commandkey}}                                                                                         \\ \hline
% general      & color{\dg=red}                                               & \cs{commandkey}\{{\db color}\} is `{\dg red}' and may be anything (text, number, macro...)                                           \\ \hline
% boolean      & {\rred bool} bold{\dg =true}                                 & \cs{commandkey}\{{\db bold}\} is:
%                                                                                                       \begin{myenum}
%                                                                                                       \item {\tt 0}\quad (for {\dg false})
%                                                                                                       \item {\tt 1}\quad (for {\dg true})
%                                                                                                       \end{myenum}          \\ \hline
% \multirow{2}*{enumerate} & {\rred enum} position{\dg=\{left,centered,right\}} & \cs{commandkey}\{{\db position}\} is:
%                                                                                                     \begin{myenum}
%                                                                                                     \item `{\dg left}'\quad by default and can be
%                                                                                                     \item `{\dg centered}' or
%                                                                                                     \item `{\dg right}'
%                                                                                                        \end{myenum}         \\ \cdashline{2-3}[1pt/2pt]
%                          & {\rred enum\textasteriskcentered} position{\dg=\{left,centered,right\}}  & This is the same, except match is case \textbf{in}sensitive   \bottopstrut                                   \\ \hline
% \multirow{2}*{choice}    & {\rred choice} position={\dg \{left,centered,right\}} & \cs{commandkey}\{{\db position}\} is:
%                                                                                                     \begin{myenum}
%                                                                                                     \item {\tt 0}\quad (for {\dg left} the default value),
%                                                                                                     \item {\tt 1}\quad (for {\dg centered})
%                                                                                                     \item {\tt 2}\quad (for {\dg right})
%                                                                                                     \end{myenum}                           \\ \cdashline{2-3}[1pt/2pt]
%                          & {\rred choice\textasteriskcentered} position={\dg\{left,centered,right\}} & This is the same, except match is case \textbf{in}sensitive   \bottopstrut                                   \\ \hline
% \end{tabular}
%
% The {\db+OptKey+} argument is used if you wish to capture the +key=value+ pairs that are not specifically defined (more on this in the examples section \ref{sec:examples}).
%
% The key-command may have {\tt 0} up to {\tt 9} \textbf{mandatory} arguments : specify the number by +<n>+ ({\tt 0} if omitted).
%
% The \stform+ form expands the \cs{commandkey} before executing the key-command itself, as explain in section \ref{sec:example:plus}.
%
% \subsection{First example :}
%
% \begin{tabbing}\label{textrule}
% \,\=\csbf{new}\=\textttbf{keycommand}\cs[\copper]{textrule}+[+{\color{db}+raise=.4ex,width=3em,thick=.4pt+}+][1]{%+ \\ ^^A+][1]{%+}\\
% \>\>\cs{rule}+[+\cs[\red]{commandkey}+{+{\db+raise+}+}]{+\cs[\red]{commandkey}+{+{\db+width+}+}{+\cs[\red]{commandkey}+{+{\db+thick+}+}}+\\
% \>\>\#1 \\
% \>\>\cs{rule}+[+\cs[\red]{commandkey}+{+{\db+raise+}+}]{+\cs[\red]{commandkey}+{+{\db+width+}+}}{+\cs[\red]{commandkey}+{+{\db+thick+}+}}}+
% \end{tabbing}
%
% defines the keys {\db+width+}, {\db+thick+} and {\db+raise+} with their default values (if not specified):
% {\db+3em+}, {\db+.4pt+} and {\db+.4ex+}. Now \cs[\copper]{textrule} can be used as follow:
% \begin{tabbing}
% \=1:\quad\=\cs[\copper]{textrule}+[width=2em]{hello}+\hskip2.5cm\=\smex\qquad\=        \rule[.4ex]{2em}{.4pt}hello\rule[.4ex]{2em}{.4pt} \\
% \>2:\>\cs[\copper]{textrule}+[thick=5pt,width=2em]{hello}+\>\smex\>                  \rule[.4ex]{2em}{5pt}hello\rule[.4ex]{2em}{5pt}\\
% \>3:\>\cs[\copper]{textrule}+{hello}+\quad \>\smex\>                                 \rule[.4ex]{3em}{.4pt}hello\rule[.4ex]{3em}{.4pt}\\
% \>4:\>\cs[\copper]{textrule}+[thick=2pt,raise=1ex]{hello}+\>\smex\>                  \rule[1ex]{3em}{2pt}hello\rule[1ex]{3em}{2pt} \\
% \> \textit{et c\ae tera}.
% \end{tabbing}
%
% \clearpage
%
% \subsection[Second example : the \string+ form]{Second example : the {\rred\bf\string+} form}
% \label{sec:example:plus}
%
% \DeleteShortVerb{\+}
% \begin{Verbatim}[gobble=1,commandchars=$(),frame=lines]
% ($bf\newkeycommand)($rred$bf+[\|])($copper\myfigure)[image,
%                              caption,
%                              enum placement={H,h,b,t,p},
%                              width=\textwidth,
%                              label=
%                             ][($db OtherKeys)]{%
%        ($rred|)($bf\begin){figure}($dr|)[($red\commandkey){placement}]
%           ($rred|)($bf\includegraphics)($dr|)[width=($red\commandkey){width},($red\commandkey){($db OtherKeys)}]{%
%                             ($red\commandkey){image}}%
%           ($dg\ifcommandkey){caption}{($rred|)\caption($rred|){($red\commandkey){caption}}}{}%
%           ($dg\ifcommandkey){label}{($rred|)\label($rred|){($red\commandkey){label}}}{}%
%        ($rred|)($bf\end){figure}($rred|)}
% \end{Verbatim}
% \MakeShortVerb{\+}
%
% With the \stform+ form of \cs{newkeycommand}, the definition will be expanded (at run time). The optional {\rred\bf+[\|]+} argument
% means that everything inside {\bf\rred+|+ ... +|+} is protected from expansion.
%
% {\dg\cs{ifcommandkey}}\{\meta{name}\}\{\meta{true}\}\{\meta{false}\}\quad expands \meta{true} if the commandkey \meta{name} is not blank.
%
% {\db \meta{Otherkeys}} captures the keys given by the user but not declared: they are simply given back to \cs{includegraphics} here...
%
%
% \subsection[Explanation of the \string+ form]{Explanation of the {\rred\bf\string+} form}
% \DeleteShortVerb{\+}
% The |\commankey{|\meta{name}|}| stuff is expanded at run time using the following scheme:��
% \begin{Verbatim}[gobble=1,commandchars=!(),frame=lines]
%     (!bf\newkeycommand)(!copper\keyMacro)[A=\defA,B=\defB,C=\defC,D=\defD][1]{(!dg\begingroup)
%        (!dg\edef)\keyMacro##1{(!dg\endgroup)
%            (!dg\noexpand)\Macro{(!red\getcommandkey){A}}
%                           {(!red\getcommandkey){B}}
%                           {(!red\getcommandkey){C}}
%                           {(!red\getcommandkey){D}}
%     }\keyMacro{#1}}
% \end{Verbatim}
% Therefore, the arguments of \cs{Macro} are ready: there is no more \cs{commandkey} stuff, but instead the values of the keys
% as you gave them to the key-command. \cs{getcommandkey}\{A\} is expanded to \cs{defA}.
%
% But \cs{defA} is not expanded of course: in the \stform+ form, \cs{commandkey} has the meaning of \cs{getcommandkey}.
%
% As you can see, the mandatory arguments \#1, \#2 etc. are \textbf{never expanded}: there is no need to protect them inside the special (usually {\rred\bf\textbar}) character.
%
%
% \MakeShortVerb{\+}
% \clearpage
%
% \subsection{key-environments}
%
% \begin{declcs}{newkeyenvironment}%
%  \Underbrace{\textcolor{red}{\textasteriskcentered\string+[short-unexpand]}}_{\makecell[c]{modifiers \\ Optional}}\,%
%  \Underbrace{\M{envir name}}_{Required}\,%
%   {\db\Underbrace{\B{keys=defaults}\,\B{OptKey}\,\B{<n>}}_{Optional}\,}%
%   \Underbrace{\M{begin}}_{Required}\Underbrace{\M{end}}_{Required}
% \end{declcs}
%
% In the same way, you may define environments with optional keys as follow:�
% \begin{tabbing}
% \qquad\=+\newkeyenvironment+\=+{EnvirWithKeys}[kOne=+default value,...+][n]+\\
% \>\>+{+ commands at begin +EnvirWithKeys }+ \\
% \>\>+{+ commands at end +EnvirWithKeys }+
% \end{tabbing}
%
% Where $n$ is the number of mandatory other arguments (\emph{ie} without keys), if any.
%
% Key-environments may be defined with the \stform+ form in the same way as \cs{newkeycommand} is used.
% Be aware that each part of the environment: \meta{begin} and \meta{end} are expanded at run time then, 
% and the optional {\rred\bf+[\|]+} argument protects from expansion in each of those parts.
% 
% \subsection[Example of a \string+ key-environment]{Example of a {\rred\bf\string+} key-environment}
% 
% \DeleteShortVerb{\+}
% \begin{Verbatim}[gobble=1,commandchars=$(),frame=lines]
% ($bf\newkeyenvironment)($rred$bf+[\|])({$copper myfigure)}[
%                              caption,
%                              enum placement={H,h,b,t,p},
%                              width=.5\linewidth,
%                              label
%                             ][($db OtherKeys)][1]%
%     {% ($nbf$dg begin part)
%        ($rred|)($bf\begin){figure}($rred|)[($red\commandkey){placement}]
%           ($rred|)($bf\includegraphics)($rred|)[($red\commandkey){($db OtherKeys)},width=($red\commandkey){width}]{$#1}%
%     }
%     {% ($nbf$dg end part)
%           ($dg\ifcommandkey){caption}{($rred|)\caption($rred|){($red\commandkey){caption} image file = $#1}}{}%
%           ($dg\ifcommandkey){label}{($rred|)\label($rred|){($red\commandkey){label}}}{}%
%        ($rred|)($bf\end){figure}($rred|)%
%     }
% \end{Verbatim}
% \MakeShortVerb{\+}
%
% As you can see, \cs{commandkey} and mandatory arguments (\#1 here) are available both in the \meta{begin} 
% and in the \meta{end} parts of the key-environment.
% 
%
% \DefineShortVerb{\+}
%
% \section{Messages and more}
%
% \subsection{Invalid keys}
%
% If you use the command +\textule+ (defined in \ref{textrule}) with a key say: +height+
% that has not been declared at the definition of the key-command, you will get an
% error message like this:
% \begin{quote}\tt
% The key-value pairs ``height=...''�
% cannot be processed for key-command \string\textrule!�
% See the definition of the keycommand!
% \end{quote}
% The error is recoverable: the key is ignored.
%
% If you assign a value to an \textit{enum} or a \textit{choice} key, which is not allowed in the definition,
% you will get the following message:
% \begin{quote}\tt
% The value ``...'' is not allowed in key ...�
% for key-command \string\command�
% I'll use the default value ``...'' for this key instead�
% See the definition of the key-command!
% \end{quote}
% The error is recoverable: the key is assigned its default value.
%
% If you use a \cs{commandkey}\{\meta{name}\} in a key-command where \meta{name} is not defined as a key,
% you will get the \TeX{} generic error message :�
% \qquad undefined control sequence : \cs{keycmd->...@name}.
%
%
% \subsection{Testing keys}
%
% \begin{declcs}{ifcommandkey}\,\M{key name}\,\M{commands if key is NOT blank}\,\M{commands if key is blank}
% \end{declcs}
%
% When you define a key command you may let the default value of a key empty. Then, you may wish to
% expand some commands only if the key has been given by the user (with a non empty value). This can
% be achieved using the macro |\ifcommandkey|.
%
% \clearpage
% \subsection{xkeyval, keyval and kvsetkeys comparison}
%
% \begin{tabbing}
% \quad\=\xpackage{xkeyval}: \expandafter\meaning\csname ver@xkeyval.sty\endcsname \\
% \>\xpackage{keyval}: \expandafter\meaning\csname ver@keyval.sty\endcsname \\
% \>\xpackage{kvsetkeys}: \expandafter\meaning\csname ver@kvsetkeys.sty\endcsname
% \end{tabbing}
%
% \makeatletter\def\theadfont{\tt\bfseries}
% \define@key{fam}{key}{\def\result{#1}}
% \begin{table}[h]\label{kvsetkeys-comparisons}
% \begin{tabular}{|l|l|>{\color{db}}l|>{\color{dg}}l|}\hline
% \thead{\bf Example} & \thead{keyval} & \thead{xkeyval} & \thead{\makecell{kvsetkeys\\and\\keycommand}} \\ \hline
% +\setkeys{fam}{key={{value}}}+
%     & \keyval@setkeys{fam}{key={{value}}}\meaning\result
%     & \xsetkeys{fam}{key={{value}}}\meaning\result
%     & \kvsetkeys{fam}{key={{value}}}\meaning\result \\\hline
% +\setkeys{fam}{key={{{value}}}}+
%     & \keyval@setkeys{fam}{key={{{value}}}}\meaning\result
%     & \xsetkeys{fam}{key={{{value}}}}\meaning\result
%     & \kvsetkeys{fam}{key={{{value}}}}\meaning\result \\\hline
% +\setkeys{fam}{key=+\textvisiblespace+{{{value}}}}+
%     & \keyval@setkeys{fam}{key= {{{value}}}}\meaning\result
%     & \xsetkeys{fam}{key= {{{value}}}}\meaning\result
%     & \kvsetkeys{fam}{key= {{{value}}}}\meaning\result \\\hline
% \end{tabular}
% \caption{Then it is clear that, at this time, \xpackage{kvsetkeys} has the only correct behaviour...}
% \end{table}
%
% In \thispackage the key-value pairs are first normalized using \xpackage{kvsetkeys}-\cs{kv@normalize}. Then braces are added
% around the values in order to keep the good behaviour of \xpackage{kvsetkeys} while using \xpackage{xkeyval}.
% \makeatother
%
%
%
%
% \StopEventually{
% }
%
% \begin{center}\vskip6pt$\star$\hskip4em\lower12pt\hbox{$\star$}\hskip4em$\star$\vadjust{\vskip12pt}\end{center}
%
% \section{Implementation} \label{Implementation}
% \csdef{HDorg@PrintMacroName}#1{\hbox to4em{\strut \MacroFont \string #1\ \hss}}
%
% \subsection{Identification}
%
% This package is intended to use with \LaTeX{} so we don't check if it is loaded twice.
%
%    \begin{macrocode}
%<*package>
\NeedsTeXFormat{LaTeX2e}% LaTeX 2.09 can't be used (nor non-LaTeX)
   [2005/12/01]% LaTeX must be 2005/12/01 or younger (see kvsetkeys.dtx).
\ProvidesPackage{keycommand}
   [2010/04/27 v3.1415 - key-value interface for commands and environments in LaTeX]
%    \end{macrocode}
%
% \subsection{Requirements}
%
% The package is based on \xpackage{xkeyval}. However, \xpackage{xkeyval} is far less reliable
% than \xpackage{kvsetkeys} as far as spaces and bracket (groups) are concerned, as shown in the section
% \ref{kvsetkeys-comparisons} of this documentation.
%
% Therefore, we also use the macros of \xpackage{kvsetkeys} in order to \textit{normalize} the \texttt{key=value}
% list before setting the keys. This way, we take advantage of both \xpackage{xkeyval} and \xpackage{kvsetkeys} !
%
% As long as we use \eTeX{} primitives in \xpackage{keycommand} we also load the
% \xpackage{etex} package in order to get an error message if \eTeX{} is not running.
%
% The \xpackage{etoolbox} package gives some facility to write \xpackage{keycommand}.
% 
% From version \texttt{3.141} onwards, \thispackage does not load \xpackage{etextools} anymore.
%
%    \begin{macrocode}
\def\kcmd@pkg@name{keycommand}
\RequirePackage{etex,kvsetkeys,xkeyval,etoolbox}
%    \end{macrocode}
%
% Save the \cs{setkeys} macro of \xpackage{xkeyval} package (in case it was overwritten by a
% subsequent load of \xpackage{kvsetkeys} or \xpackage{keyval} for example :
%    \begin{macrocode}
\protected\def\kcmd@Xsetkeys{\XKV@sttrue\XKV@plfalse\XKV@testoptc\XKV@setkeys}% in case \setkeys 
%                                                                                was overwritten
%    \end{macrocode}
% Some \cs{catcode} assertions internally used by \thispackage:
%    \begin{macrocode}
\let\kcmd@AtEnd\@empty
\def\TMP@EnsureCode#1#2{%
  \edef\kcmd@AtEnd{%
    \kcmd@AtEnd
    \catcode#1 \the\catcode#1\relax
  }%
  \catcode#1 #2\relax
}
\TMP@EnsureCode{32}{10}% space
\TMP@EnsureCode{61}{12}% = sign
\TMP@EnsureCode{45}{12}% - sign
\TMP@EnsureCode{62}{12}% > sign
\TMP@EnsureCode{46}{12}% . dot
\TMP@EnsureCode{47}{8}% / slash (etextools)
\AtEndOfPackage{\kcmd@AtEnd\undef\kcmd@AtEnd}
%    \end{macrocode}
% 
% \begin{macro}{\kcmd@ifstrdigit}\qquad\qquad
% This macro is used too test the optional arguments of \cs{newkeycommand}, 
% in particular, one must know in an argument is a single digit (representing
% the number of mandatory arguments) or anything else (representing the \texttt{key=value} 
% list or the ``special'' \texttt{OptKey} key:
%    \begin{macrocode}
\iffalse%\ifdefined\pdfmatch% use \pdfmatch if present
   \long\def\kcmd@ifstrdigit#1{\csname @\ifnum\pdfmatch
      {\detokenize{^[[:space:]]*[[:digit:]][[:space:]]*$}}{\detokenize{#1}}=1 %
      first\else second\fi oftwo\endcsname}
\else% use filter, very efficient !
\def\kcmd@ifstrdigit#1{%
   \kcmd@nbk#1//%
      {\expandafter\expandafter\expandafter\kcmd@ifstrdigit@i
         \expandafter\expandafter\expandafter{\detokenize\expandafter{\number\number0#1}}}%
      {\@secondoftwo}//%
}
\def\kcmd@ifstrdigit@i#1{%
   \def\kcmd@ifstrdigit@ii##1#1##2##3\kcmd@ifstrdigit@ii{%
      \csname @\ifx##20first\else second\fi oftwo\endcsname
      }\kcmd@ifstrdigit@ii 00 01 02 03 04 05 06 07 08 09 0#1 \relax\kcmd@ifstrdigit@ii
}
\fi
%    \end{macrocode}
% \end{macro}
%
% \subsection{Defining (and undefining) command-keys}
%\begin{macro}{\kcmd@keyfam}\qquad
% The macro expands to the family-name, given the keycommand name:
%    \begin{macrocode}
\def\kcmd@keyfam#1{\detokenize{keycmd->}\expandafter\@gobble\string#1}
%    \end{macrocode}
% \end{macro}
% \begin{macro}{\kcmd@nbk}\qquad is the optimized \cs{ifnotblank} macro of \xpackage{etoolbox}
% (with \textttbf{/} having a catcode of 8):
%    \begin{macrocode}
\def\kcmd@nbk#1#2/#3#4#5//{#4}%
%    \end{macrocode}
% \end{macro}
%
% \begin{macro}{\kcmd@normalize@setkeys}~\par
% This macro assigns the values to the keys (expansion of \xpackage{xkeyval}-\cs{setkeys}
% on the result of \xpackage{kvsetkeys}-\cs{kv@normalize}). Braces are normalized too so that
% \verb+key=+\textvisiblespace+{{{value}}}+ is the same as \verb+key={{{value}}}+ as explained in section \ref{kvsetkeys-comparisons}:
%    \begin{macrocode}
\newrobustcmd\kcmd@normalize@setkeys[4]{%
% #1 = key-command,
% #2 = family,
% #3 = other-key,
% #4 = key-values pairs
   \kv@normalize{#4}\toks@{}%
   \expandafter\kv@parse@normalized\expandafter{\kv@list}{\kcmd@normalize@braces{#2}}%
   \edef\kv@list{\kcmd@Xsetkeys{\unexpanded{#2}}{\the\toks@}}\kv@list
   \kcmd@nbk#3//% undeclared keys are assigned to "OtherKeys"
      {\cslet{#2@#3}\XKV@rm}% (if specified, ie not empty)
      {\expandafter\kcmd@nbk\XKV@rm//% (otherwise a recoverable error is thown)
         {\PackageError\kcmd@pkg@name{The key-value pairs :\MessageBreak
         \XKV@rm\MessageBreak
         cannot be processed for key-command \string#1\MessageBreak
         See the definition of the key-command!}{}}{}//}//%
}
\long\def\kcmd@normalize@braces#1#2#3{% This is kvsetkeys processor for normalizing braces
   \toks@\expandafter{\the\toks@,#2}%
   \ifx @\detokenize{#3}@\else \toks@\expandafter{\the\toks@={{{#3}}}}\fi
}
%    \end{macrocode}
% \end{macro}
% 
% \begin{macro}{\kcmd@definekey}~\par
% \CS{kcmd@definekey} define the keys declared for the key-command.
% It is used as the \emph{processor} for the \cs{kv@parse} macro of \xpackage{kvsetkeys}.
% The macro appends the key names to the key list: ``\textit{family}.keylist''.
%
% keys are first checked for their type (bool, enum, enum*, choice or choice*) :
%
%    \begin{macrocode}
\def\kcmd@check@typeofkey#1{% expands to
% 0 if key has no type,
% 1 if boolean,
% 2 if enum*,
% 3 if enum,
% 4 if choice*,
% 5 if choice
   \kcmd@check@typeofkey@bool#1bool //%
      {\kcmd@check@typeofkey@enumst#1enum* //%
         {\kcmd@check@typeofkey@enum#1enum //%
            {\kcmd@check@typeofkey@choicest#1choice* //%
               {\kcmd@check@typeofkey@choice#1choice //%
                  05//}4//}3//}2//}1//}
\def\kcmd@check@typeofkey@bool #1bool #2//{\kcmd@nbk#1//}
\def\kcmd@get@keyname@bool #1bool #2//{#2}
\def\kcmd@check@typeofkey@enumst #1enum* #2//{\kcmd@nbk#1//}
\def\kcmd@get@keyname@enumst #1enum* #2//{#2}
\def\kcmd@check@typeofkey@enum #1enum #2//{\kcmd@nbk#1//}
\def\kcmd@get@keyname@enum #1enum #2//{#2}
\def\kcmd@check@typeofkey@choicest #1choice* #2//{\kcmd@nbk#1//}
\def\kcmd@get@keyname@choicest #1choice* #2//{#2}
\def\kcmd@check@typeofkey@choice #1choice #2//{\kcmd@nbk#1//}
\def\kcmd@get@keyname@choice #1choice #2//{#2}
%
\protected\long\def\kcmd@definekey#1#2#3#4#5{% define the keys using xkeyval macros
% #1 = keycommand,
% #2 = \global,
% #3 = family,
% #4 = key (before = sign),
% #5 = default (after = sign)
   \ifcase\kcmd@check@typeofkey{#4}\relax% standard
      #2\csedef{#3.keylist}{\csname#3.keylist\endcsname,#4}%
      \define@cmdkey{#3}[{#3@}]{#4}[{#5}]{}%
   \or% bool
      #2\csedef{#3.keylist}{\csname#3.keylist\endcsname,\kcmd@get@keyname@bool#4//}%
      \kcmd@define@boolkey#1{#3}{\kcmd@get@keyname@bool#4//}{#5}%
   \or% enum*
      #2\csedef{#3.keylist}{\csname#3.keylist\endcsname,\kcmd@get@keyname@enumst#4//}%
      \kcmd@define@choicekey#1*{#3}{\kcmd@get@keyname@enumst#4//}{#5}{\expandonce\val}%
   \or% enum
      #2\csedef{#3.keylist}{\csname#3.keylist\endcsname,\kcmd@get@keyname@enum#4//}%
      \kcmd@define@choicekey#1{}{#3}{\kcmd@get@keyname@enum#4//}{#5}{\expandonce\val}%
   \or% choice*
      #2\csedef{#3.keylist}{\csname#3.keylist\endcsname,\kcmd@get@keyname@choicest#4//}%
      \kcmd@define@choicekey#1*{#3}{\kcmd@get@keyname@choicest#4//}{#5}{\number\nr}%
   \or% choice
      #2\csedef{#3.keylist}{\csname#3.keylist\endcsname,\kcmd@get@keyname@choice#4//}%
      \kcmd@define@choicekey#1{}{#3}{\kcmd@get@keyname@choice#4//}{#5}{\number\nr}%
   \fi
   \ifx#2\global\relax
      #2\csletcs{KV@#3@#4}{KV@#3@#4}% globalize
      #2\csletcs{KV@#3@#4@default}{KV@#3@#4@default}% globalize default value
   \fi
}
%
\long\def\kcmd@firstchoiceof#1,#2\kcmd@nil{\unexpanded{#1}}
%
\long\def\kcmd@define@choicekey#1#2#3#4#5#6{%
   \begingroup\edef\kcmd@define@choicekey{\endgroup
      \noexpand\define@choicekey#2+{#3}{#4}
            [\noexpand\val\noexpand\nr]%
            {\unexpanded{#5}}% list of allowed values
            [{\kcmd@firstchoiceof#5,\kcmd@nil}]% default value
            {\csedef{#3@#4}{\unexpanded{#6}}}% define key value if in the allowed list
            {\kcmd@error@handler\noexpand#1{#3}{#4}{\kcmd@firstchoiceof#5,\kcmd@nil}}% error handler
   }\kcmd@define@choicekey
}
\def\kcmd@define@boolkey#1#2#3#4{\begingroup
   \kcmd@nbk#4//{\def\default{#4}}{\def\default{true}}//%
   \edef\kcmd@define@boolkey{\endgroup
      \noexpand\define@choicekey*+{#2}{#3}[\noexpand\val\noexpand\nr]%
            {false,true}
            [{\unexpanded\expandafter{\default}}]%
            {\csedef{#2@#3}{\noexpand\number\noexpand\nr}}%
            {\kcmd@error@handler\noexpand#1{#2}{#3}{\unexpanded\expandafter{\default}}}%
   }\kcmd@define@boolkey
}
%
\protected\long\def\kcmd@error@handler#1#2#3#4{%
% #1 = key-command,
% #2 = family,
% #3 = key,
% #4 = default
   \PackageError\kcmd@pkg@name{%
      Value `\val\space' is not allowed in key #3\MessageBreak
      for key-command \string#1.\MessageBreak
      I'll use the default value `#4' for this key.\MessageBreak
      See the definition of the key-command!}{%
      \csdef{#2@#3}{#4}}}
%    \end{macrocode}
% \end{macro}
%
% \begin{macro}{\kcmd@undefinekeys}~\par
% Now in case we redefine a key-command, we would like the old keys (\emph{ie} the keys
% associated to the old definition of the command) to be cleared, undefined.
% That's the job of \cs{kcmd@undefinekeys}.
%    \begin{macrocode}
\protected\def\kcmd@undefinekeys#1#2{% #1 = global, #2 = family
   \ifcsundef{#2.keylist}
      {\cslet{#2.keylist}\@gobble}
      {\expandafter\expandafter\expandafter\docsvlist
         \expandafter\expandafter\expandafter{%
                        \csname #2.keylist\endcsname}%
      \cslet{#2.keylist}\@gobble}%
}
\def\kcmd@undefinekey#1#2#3{% #1 = global, #2 = family, #3 = key
   #1\csundef{KV@#2@#3}%
   #1\csundef{KV@#2@#3@default}%
}
%    \end{macrocode}
% \end{macro}
% 
%\begin{macro}{\kcmd@setdefaults}\qquad\qquad
% sets the defaults values for the keys at the very beginning of the keycommand:
%    \begin{macrocode}
\def\kcmd@setdefaults#1{%
   \ifcsundef{#1.keylist}{}
   {\expandafter\expandafter\expandafter\docsvlist
      \expandafter\expandafter\expandafter{%
                           \csname#1.keylist\endcsname}}%
}
%    \end{macrocode}
%\end{macro}
% 
% 
%
% \begin{macro}{\kcmd@def}
% checks \cs{@ifdefinable} and cancels definition if needed:
%    \begin{macrocode}
\protected\long\def\kcmd@def#1#2[#3][#4][#5]#6#7{%
   \ifx#1\kcmd@donot@provide  \endgroup
   \else
      \@tempswafalse\@ifdefinable#1{\@tempswatrue}%
      \if@tempswa
         \edef\kcmd@fam{\kcmd@keyfam{#1}}%
         \expandafter\kcmd@defcommand\expandafter{\kcmd@fam}#1[{#3}][{#4}][{#5}]{#6}{#2}{#7}%
      \else\endgroup
      \fi
   \fi
}
%    \end{macrocode}
% \end{macro}
% 
% \begin{macro}{\kcmd@defcommand}\qquad\qquad prepares (expands) the arguments before closing the group opened at the very beginning.
% Then it proceeds (\cs{kcmd@yargdef} (normal interface)  or \cs{kcmd@yargedef} (when \cs{newkeycommand}\stform+ is used))
%    \begin{macrocode}
\protected\long\def\kcmd@defcommand#1#2[#3][#4][#5]#6#7#8{%
   \let\commandkey\relax  \let\getcommandkey\relax  \let#2\relax   
   \cslet{#1}\relax  \cslet{#1.commankey}\relax  \cslet{#1.getcommandkey}\relax
   \def\do{\kcmd@undefinekey{\kcmd@gbl}{#1}}%
   \edef\kcmd@defcommand{\endgroup
      \kcmd@undefinekeys{\kcmd@gbl}{#1}% undefines all keys for this keycommand family
      \ifx\kcmd@unexpandchar\@empty\else
         \kcmd@mount@unexpandchar{#1}{\unexpanded\expandafter{\kcmd@unexpandchar}}%
      \fi
      \unexpanded{\kv@parse{#3,#4}}{\kcmd@definekey\noexpand#2{\kcmd@gbl}{#1}}% defines keys
      \csdef{#1.commandkey}####1{\noexpand\csname#1@####1\endcsname}%
      \csdef{#1.getcommandkey}####1{%
         \unexpanded{\unexpanded\expandafter\expandafter\expandafter}{%
                           \noexpand\csname#1@####1\endcsname}}%
      \kcmd@ifplus% \newkeycommand+ / \newkeyenvironment+
         \protected\csdef{#1}{%
            \kcmd@yargedef{\kcmd@gbl}{\kcmd@long}\csname#1\endcsname
                          {\number#5}{\noexpand#7}{\csname#1.unexpandchar\endcsname}}%
         \ifx#7\@gobble\else 
             \protected\def#7{\kcmd@yargedef#7}%
         \fi
      \else% \newkeycommand / \newkeyenvironment
         \csdef{#1}{%
            \kcmd@yargdef{\kcmd@gbl}{\kcmd@long}\csname#1\endcsname
                          {\number#5}{\noexpand#7}}%
         \ifx#7\@gobble\else \def#7####1{% that means we have to define a key-environment
            \def#7{%
               \let\getcommandkey\csname#1.getcommandkey\endcsname
               \let\commandkey\csname#1.commandkey\endcsname
               ####1}%
            }%
         \fi
      \fi
      \def\noexpand\do####1{\unexpanded{\expandafter\noexpand\csname}KV@#1@####1@default%
                                                                                     \endcsname}% 
      \let\commandkey\relax \let\getcommandkey\relax \let#2\relax
      \kcmd@gbl\protected\edef#2{% entry point
         \let\getcommandkey\noexpand\noexpand\csname#1.getcommandkey\endcsname
         \kcmd@ifplus  \let\commandkey\getcommandkey
         \else         \let\commandkey\noexpand\noexpand\csname#1.commandkey\endcsname
         \fi
         \noexpand\kcmd@setdefaults{#1}%
         \ifx#7\@gobble \noexpand\noexpand\noexpand\@testopt
                        {\kcmd@setkeys#2{#1}{\kcmd@otherkey{#4}}}{}%
         \else          \noexpand\noexpand\noexpand\@testopt
                        {\kcmd@setkeys#2{#1}{\kcmd@otherkey{#4}}}{}%
         \fi
         }%
      \csname#1\endcsname% expand \kcmd@yargedef or \kcmd@yargdef
   }\kcmd@defcommand{#6}{#8}% #6 = definition, #8 = definition end-envir
}
\protected\long\def\kcmd@setkeys#1#2#3[#4]{% #1=key-command, #2=family, #3=otherkey, #4=key=value pairs
   \kcmd@normalize@setkeys{#1}{#2}{#3}{#4}\csname#2\endcsname
}
\long\def\kcmd@otherkey#1{\kcmd@nbk#1//{\kcmd@otherkey@name#1=\kcmd@nil}{}//}
\long\def\kcmd@otherkey@name#1=#2\kcmd@nil{#1}
%    \end{macrocode}
% \end{macro}
%
% \begin{macro}{\kcmd@mount@unexpandchar}~\par
% This macro defines the macro \cs{"\textit{family.unexpandchar}"}. 
% \CS{"\textit{family.unexpandchar}"} activates the shortcut character 
% for \cs{unexpanded} and defines its meaning.
%    \begin{macrocode}
\protected \def \kcmd@mount@unexpandchar#1#2{%
   \protected\csdef{#1.unexpandchar}{\begingroup
      \catcode`\~\active \lccode`\~`#2 \lccode`#2 0\relax
         \lowercase{%
            \expandafter\endgroup\expandafter\def\expandafter~{%
               \catcode`#2\active
               \long\def~########1~{\unexpanded{########1}}}%
         ~}%
   }%
}
%    \end{macrocode}
% \end{macro}
%
%----------------------------------------------------------------------------
% \begin{macro}{\kcmd@yargdef}\qquad\qquad
% This is the ``{\tt argdef}'' macro for the normal (non \string+) form:
%    \begin{macrocode}
\protected \def \kcmd@yargdef #1#2#3#4#5{\begingroup
% #1 = global or {}
% #2 = long or {}
% #3 = Command
% #4 = nr of args
% #5 = endenvir (or \@gobble if not an environment, or \relax if #3 is endenvir)
   \def \kcmd@yargd@f ##1#4##2##{\afterassignment#5\endgroup
      #1#2\expandafter\def\expandafter#3\@gobble ##1#4%
   }\kcmd@yargd@f 0##1##2##3##4##5##6##7##8##9###4%
}
%    \end{macrocode}
% \end{macro}
%
% \begin{macro}{\kcmd@yargedef}\qquad\qquad
% This is the ``{\tt argdef}'' macro for the {\rred\bf\string+} form:
%    \begin{macrocode}
\protected \def \kcmd@yargedef#1#2#3#4#5#6{\begingroup
% #1 = global or {}
% #2 = long or {}
% #3 = Command
% #4 = nr of args
% #5 = endenvir (or \@gobble if not an environment, or \relax if #3 is endenvir)
% #6 = unexpandchar mounting macro
  \kcmd@nargs{#4}% 
   \protected\long\def\kcmd@yarg@edef##1##2{\endgroup
         #1\edef#3{\begingroup #6%
            #2\edef#3\unexpanded{##2}{\endgroup\unexpanded{##1}%
         }#3}%
   }%
   \protected\def\kcmd@envir##1{%
      \edef\next{\kcmd@yarg@edef{\def\noexpand#5{\expandonce{#5##1}}\expandonce{#3##1}}}\next
   }%
   \protected\def\kcmd@command##1{%
      \edef\next{\kcmd@yarg@edef{\expandonce{#3##1}}}\next
   }%
   \protected\def\kcmd@yargedef##1{%
      \kcmd@yargedef@##1 0####1####2####3####4####5####6####7####8####9#####4%
   }%
   \ifx#5\@gobble % keycommand
      \def\next{\kcmd@command}%
   \else          % key-environmment
      \def\next{\kcmd@envir}%
   \fi
   \let\@next\relax
   \def\kcmd@yargedef@##1##2#4##3##{%
      \ifx\@next\relax 
         \edef\@next{\next{\expandonce{\kcmd@nargs}}{\expandonce{\@gobble##2#4}}}%
         \ifx#5\@gobble \edef\@next{\expandonce\@next\noexpand#5}%
         \else \edef\@next{\edef\noexpand\@next{\noexpand\unexpanded{\expandonce\@next}}#5}%
         \fi
      \fi
      \afterassignment\@next
      \expandafter\def\expandafter##1\@gobble##2#4%
   }%
   \kcmd@yargedef#3%
}
%    \end{macrocode}
% \end{macro}
%
% \begin{macro}{\kcmd@nargs}\qquad
% Filter macros used by \cs{kcmd@yargedef} to get the correct number of arguments:
%    \begin{macrocode}
\def\kcmd@nargs#1{\edef\kcmd@nargs%##1##2##3##4##5##6##7##8##9%
        {\ifnum#1>0{####1%
         \ifnum#1>1}{####2%
         \ifnum#1>2}{####3%
         \ifnum#1>3}{####4%
         \ifnum#1>4}{####5%
         \ifnum#1>5}{####6%
         \ifnum#1>6}{####7%
         \ifnum#1>7}{####8%
         \ifnum#1>8}{####9%
         \fi\fi\fi\fi\fi\fi\fi\fi}\fi}%
}%
%    \end{macrocode}
% \end{macro}
%
% \subsection{new key-commands}
%
% \begin{macro}{\newkeycommand}\qquad\qquad
% Here are the entry points:
%    \begin{macrocode}
\newrobustcmd*\newkeycommand{\begingroup
   \let\kcmd@gbl\@empty\kcmd@star@or@long\new@keycommand}
\newrobustcmd*\renewkeycommand{\begingroup
   \let\kcmd@gbl\@empty\kcmd@star@or@long\renew@keycommand}
\newrobustcmd*\providekeycommand{\begingroup
   \let\kcmd@gbl\@empty\kcmd@star@or@long\provide@keycommand}
%    \end{macrocode}
% \end{macro}
%
% \begin{macro}{\kcmd@star@or@long}~\par
% This is the adaptation of \LaTeX's \cs{@star@or@long} macro:
%    \begin{macrocode}
\def\kcmd@star@or@long#1{\@ifstar
      {\let\kcmd@long\@empty\kcmd@plus#1}
      {\def\kcmd@long{\long}\kcmd@plus#1}}
\def\kcmd@@ifplus#1{\@ifnextchar +{\@firstoftwo{#1}}}% same as LaTeX's \@ifstar
\def\kcmd@plus#1{\kcmd@@ifplus
      {\def\kcmd@ifplus{\iftrue}\kcmd@testopt#1}
      {\def\kcmd@ifplus{\iffalse}\kcmd@testopt#1}}
\def\kcmd@testopt#1{\@testopt{\kcmd@unexpandchar#1}{}}
%    \end{macrocode}
% \end{macro}
%
%\begin{macro}{\kcmd@unexpandchar}\qquad\qquad\quad
% Reads the possible unexpand-char shortcut:
%    \begin{macrocode}
\def\kcmd@unexpandchar#1[#2]{%
   \kcmd@ifplus
      \kcmd@nbk#2//
         {\def\kcmd@unexpandchar{#2}% only once inside group...
          \def\kcmd@unexpandchar@activate{\catcode`#2 \active}%
         }{%
          \let\kcmd@unexpandchar\@empty
          \let\kcmd@unexpandchar@activate\relax
         }//%
   \else  \let\kcmd@unexpandchar\@empty
      \kcmd@nbk#2//%
         {\PackageError\kcmd@pkg@name{shortcut option for \string\unexpanded\MessageBreak
         You can't use a shortcut option for \string\unexpanded\MessageBreak
         without the \string+ form of \string\newkeycommand!}%
         {I will ignore this option and proceed.}%
         }%
         {}//%      
   \fi#1}
%    \end{macrocode}
%\end{macro}
%
% \begin{macro}{\new@keycommand}\qquad\qquad
% Reads the key-command name (cs-token):
%    \begin{macrocode}
\def\new@keycommand#1{\@testopt{\@newkeycommand#1}0}
%    \end{macrocode}
% \end{macro}
%
%\begin{macro}{\@newkeycommand}\qquad\qquad
% Reads the first optional parameter (keys or number of mandatory args):
%    \begin{macrocode}
\long\def\@newkeycommand#1[#2]{% #2 = key=values or N=mandatory args
   \kcmd@ifplus \kcmd@unexpandchar@activate \fi% activates unexpand-char before reading definition
   \kcmd@ifstrdigit{#2}%
      {\@new@key@command#1[][][{#2}]}% no kv, no optkey, number of args
      {\@testopt{\@new@keycommand#1[{#2}]}0}}% kv, check for optkey/nr of args
%    \end{macrocode}
% \end{macro}
%
%\begin{macro}{\@new@keycommand}\qquad\qquad
% Reads the second optional parameter (opt key or number of mandatory args):
%    \begin{macrocode}
\long\def\@new@keycommand#1[#2][#3]{%
   \kcmd@ifstrdigit{#3}%
      {\@new@key@command#1[{#2}][][{#3}]}% no optkey
      {\@testopt{\@new@key@command#1[{#2}][{#3}]}0}}
%    \end{macrocode}
%\end{macro}
%
%\begin{macro}{\@new@key@command}\qquad\qquad
% Reads the definition of the command (\cs{kcmd@def} handles both cases of commands and environements).
% The so called "unexpand-char shortcut" has been activated before reading command definition:
%    \begin{macrocode}
\long\def\@new@key@command#1[#2][#3][#4]#5{%
      \kcmd@def#1\@gobble[{#2}][{#3}][{#4}]{#5}{}}
%    \end{macrocode}
%\end{macro}
%
% \begin{macro}{\renew@keycommand}
%    \begin{macrocode}
\def\renew@keycommand#1{\begingroup
   \escapechar\m@ne\edef\@gtempa{{\string#1}}%
   \expandafter\@ifundefined\@gtempa
      {\endgroup\@latex@error{\noexpand#1undefined}\@ehc}
      \endgroup
   \let\@ifdefinable\@rc@ifdefinable
   \new@keycommand#1%
}
%    \end{macrocode}
% \end{macro}
%
% \begin{macro}{\provide@keycommand}
%    \begin{macrocode}
\def\provide@keycommand#1{\begingroup
   \escapechar\m@ne\edef\@gtempa{{\string#1}}%
   \expandafter\@ifundefined\@gtempa
      {\endgroup\new@keycommand#1}
      {\endgroup\def\kcmd@donot@provide{\renew@keycommand\kcmd@donot@provide
         }\kcmd@donot@provide}%
}
\let\kcmd@donot@provide\@empty% it must not be undefined
%    \end{macrocode}
% \end{macro}
%
% \subsection{new key-environments}
%
% \begin{macro}{\newkeyenvironment}
%    \begin{macrocode}
\newrobustcmd*\newkeyenvironment{\begingroup
   \let\kcmd@gbl\@empty\kcmd@star@or@long\new@keyenvironment}
\newrobustcmd\renewkeyenvironment{\begingroup
   \let\kcmd@gbl\@empty\kcmd@star@or@long\renew@keyenvironment}
%    \end{macrocode}
% \end{macro}
%
% \begin{macro}{\new@keyenvironment}
%    \begin{macrocode}
\def\new@keyenvironment#1{\@testopt{\@newkeyenva{#1}}{}}
\long\def\@newkeyenva#1[#2]{%
   \kcmd@ifstrdigit{#2}%
      {\@newkeyenv{#1}{[][][{#2}]}}
      {\@testopt{\@newkeyenvb{#1}[{#2}]}{}}}
\long\def\@newkeyenvb#1[#2][#3]{%
   \kcmd@ifstrdigit{#3}%
      {\@newkeyenv{#1}{[{#2}][][{#3}]}}
      {\@testopt{\@newkeyenvc{#1}{[{#2}][{#3}]}}0}}
\long\def\@newkeyenvc#1#2[#3]{\@newkeyenv{#1}{#2[{#3}]}}
\long\def\@newkeyenv#1#2{%
   \kcmd@ifplus \kcmd@unexpandchar@activate \fi
   \kcmd@keyenvir@def{#1}{#2}%
}
\long\def\kcmd@keyenvir@def#1#2#3#4{%
   \expandafter\let\csname end#1\endcsname\relax
   \expandafter\kcmd@def\csname #1\expandafter\endcsname\csname end#1\endcsname#2{#3}{#4}%
}
%    \end{macrocode}
% \end{macro}
%
% \begin{macro}{\renew@keyenvironment}
%    \begin{macrocode}
\def\renew@keyenvironment#1{%
  \@ifundefined{#1}%
     {\@latex@error{Environment #1 undefined}\@ehc
     }\relax
  \cslet{#1}\relax
  \new@keyenvironment{#1}}
%    \end{macrocode}
% \end{macro}
% \iffalse
%<package>
%<package>
% \fi
%
% \subsection{Tests on keys}
%
% \begin{macro}{\ifcommandkey}\qquad
% \{\meta{key-name}\}\{\meta{true}\}\{\meta{false}\}\quad expands \meta{true} only if the value of the key
% is not blank:
%    \begin{macrocode}
\newcommand*\ifcommandkey[1]{\csname @\expandafter\expandafter\expandafter
   \kcmd@nbk\commandkey{#1}//{first}{second}//%
   oftwo\endcsname}
%    \end{macrocode}
% \end{macro}
%
%
% \begin{macro}{\showcommandkeys}\qquad\qquad are helper macros essentially for debuging purpose...
%    \begin{macrocode}
\newrobustcmd*\showcommandkeys[1]{\let\do\showcommandkey\docsvlist{#1}}
\newrobustcmd*\showcommandkey[1]{key \string"#1\string" = %
   \detokenize\expandafter\expandafter\expandafter{\commandkey{#1}}\par}
%    \end{macrocode}
% \end{macro}
% 
%
%    \begin{macrocode}
%</package>
%    \end{macrocode}
%
% \section{Examples}
% \label{sec:examples}
%
%    \begin{macrocode}
%<*example>
\ProvidesFile{keycommand-example}
\documentclass[a4paper]{article}
\usepackage[T1]{fontenc}
\usepackage[latin1]{inputenc}
\usepackage[american]{babel}
\usepackage{keycommand,framed,fancyvrb}
%
\makeatletter
\parindent\z@
\newkeycommand*\Rule[raise=.4ex,width=1em,thick=.4pt][1]{%
   \rule[\commandkey{raise}]{\commandkey{width}}{\commandkey{thick}}%
   #1%
   \rule[\commandkey{raise}]{\commandkey{width}}{\commandkey{thick}}}

\newkeycommand*\charleads[sep=1][2]{%
   \ifhmode\else\leavevmode\fi\setbox\@tempboxa\hbox{#2}\@tempdima=1.584\wd\@tempboxa%
   \cleaders\hb@xt@\commandkey{sep}\@tempdima{\hss\box\@tempboxa\hss}#1%
   \setbox\@tempboxa\box\voidb@x}
\newcommand*\charfill[1][]{\charleads[{#1}]{\hfill\kern\z@}}
\newcommand*\charfil[1][]{\charleads[{#1}]{\hfil\kern\z@}}
%
\newkeyenvironment*{dblruled}[first=.4pt,second=.4pt,sep=1pt,left=\z@]{%
   \def\FrameCommand{%
      \vrule\@width\commandkey{first}%
      \hskip\commandkey{sep}
      \vrule\@width\commandkey{second}%
      \hspace{\commandkey{left}}}%
   \parindent\z@
   \MakeFramed {\advance\hsize-\width \FrameRestore}}
   {\endMakeFramed}
%
\makeatother
\begin{document}
\title{This is {\tt keycommand-example.tex}}
\author{Florent Chervet}
\date{July 22, 2009}

\maketitle

{\Large Please refer to {\tt keycommand-example.tex} for definitions.}

\section{Example of a keycommand : \texttt{\string\Rule}}

\begin{tabular*}\textwidth{rl}
\verb+\Rule[width=2em]{hello}+:&\Rule[width=2em]{hello}\cr
\verb+\Rule[thick=1pt,width=2em]{hello}+:&\Rule[thick=1pt,width=2em]{hello}\cr
\verb+\Rule{hello}+:&\Rule{hello}\cr
\verb+\Rule[thick=1pt,raise=1ex]{hello}+:&\Rule[thick=1pt,raise=1ex]{hello}
\end{tabular*}

\section{Example of a keycommand : \texttt{\string\charfill}}

\begin{tabular*}\textwidth{rp{.4\textwidth}}
\verb+\charfill{$\star$}+: & \charfill{$\star$}\cr
\verb+\charfill[sep=2]{$\star$}+: & \charfill[sep=2]{$\star$} \\
\verb+\charfill[sep=.7]{\textasteriskcentered}+: & \charfill[sep=.7]{\textasteriskcentered}
\end{tabular*}


\section{Example of a keyenvironment : \texttt{dblruled}}

Key environment \texttt{dblruled } uses \texttt{framed.sty} and therefore it can be used 
even if a pagebreak occurs inside the environment:
\medskip

\verb+\begin{dblruled}+\par
\verb+   test for dblruled key-environment\par+\par
\verb+   test for dblruled key-environment\par+\par
\verb+   test for dblruled key-environment+\par
\verb+\end{dblruled}+

\begin{dblruled}
 test for dblruled key-environment\par
 test for dblruled key-environment\par
 test for dblruled key-environment
\end{dblruled}


\verb+\begin{dblruled}[first=4pt,sep=2pt,second=.6pt,left=.2em]+\par
\verb+   test for dblruled key-environment\par+\par
\verb+   test for dblruled key-environment\par+\par
\verb+   test for dblruled key-environment+\par
\verb+\end{dblruled}+

\begin{dblruled}[first=4pt,sep=2pt,second=.6pt,left=.2em]
 test for dblruled key-environment\par
 test for dblruled key-environment\par
 test for dblruled key-environment
\end{dblruled}

\end{document}
%</example>
%    \end{macrocode}
% \DeleteShortVerb{\+}^^A\UndefineShortVerb{\+}
% \begin{History}
% 
%   \begin{Version}{2010/04/27 v3.1415}
%   \item Key-environment can now be nested ! (it's not too late... I hope so)
%   \item Keys and mandatory arguments as well can be used in both \texttt{begin} end \texttt{end} part of the environment.
%   \end{Version}
% 
%   \begin{Version}{2010/04/25 v3.141}
%   \item No new feature but a real improvement in optimization. \\
%         In particular, \thispackage does not load \xpackage{etextools} anymore. \\
%   \item Bug fix for \cs{providekeycommand}.
%         
%   \end{Version}
%
%   \begin{Version}{2010/04/18 v3.14}
%   \item Correction of bug in the normalization process. \\
%         Correction of a bug in \cs{ifcommandkey} (undesirable space...)
%   \item Modification of the pdf documentation for the \stform+ form of key-environments.
%   \end{Version}
%
%   \begin{Version}{2010/03/28 v3.0}
%   \item Complete redesign of the implementation. \\
%   \xpackage{keycommand} is now based on some macros of \xpackage{etoolbox}.
%
%   \item Adding the + prefix and the ability to capture keys that where not defined.
%
%   \end{Version}
%
%   \begin{Version}{2009/07/22 v1.0}
%   \item
%     First version.
%   \end{Version}
%
% \end{History}
%
% \begin{thebibliography}{9}
%
% \bibitem{xkeyval}
%   Hendri Adriaens:
%   \textit{The \xpackage{xkeyval} package};
%   2008/08/13 v2.6a;
%   \CTAN{macros/latex/contrib/xkeyval.dtx}
%
% \bibitem{kvsetkeys}
%   Heiko Oberdiek:
%   \textit{The \xpackage{kvsetkeys} package};
%   2007/09/29 v1.3;
%   \CTAN{macros/latex/contrib/oberdiek/kvsetkeys.dtx}.
%
% \bibitem{keyval}
%   David Carlisle:
%   \textit{The \xpackage{keyval} package};
%   1999/03/16 v1.13;
%   \CTAN{macros/latex/required/graphics/keyval.dtx}.
%
% \end{thebibliography}
%
% \PrintIndex
%
% \label{LastPage}
% \Finale
%        (quote the arguments according to the demands of your shell)
%
% Documentation:
%           (pdf)latex keycommand.dtx
% Copyright (C) 2009-2010 by Florent Chervet <florent.chervet@free.fr>
%<*ignore>
\begingroup
  \def\x{LaTeX2e}%
\expandafter\endgroup
\ifcase 0\ifx\install y1\fi\expandafter
         \ifx\csname processbatchFile\endcsname\relax\else1\fi
         \ifx\fmtname\x\else 1\fi\relax
\else\csname fi\endcsname
%</ignore>
%<*install>
\input docstrip.tex
\Msg{************************************************************************}
\Msg{* Installation}
\Msg{* Package: keycommand 2010/04/27 v3.1415 key-value interface for commands and environments in LaTeX}
\Msg{************************************************************************}

\keepsilent
\askforoverwritefalse

\let\MetaPrefix\relax
\preamble

This is a generated file.

keycommand : key-value interface for commands and environments in LaTeX [v3.1415 2010/04/27]

This work may be distributed and/or modified under the
conditions of the LaTeX Project Public License, either
version 1.3 of this license or (at your option) any later
version. The latest version of this license is in
   http://www.latex-project.org/lppl.txt

This work consists of the main source file keycommand.dtx
and the derived files
   keycommand.sty, keycommand.pdf, keycommand.ins,
   keycommand-example.tex

keycommand : an easy way to define commands with optional keys
Copyright (C) 2009-2010 by Florent Chervet <florent.chervet@free.fr>

\endpreamble
\let\MetaPrefix\DoubleperCent

\generate{%
   \file{keycommand.ins}{\from{keycommand.dtx}{install}}%
   \file{keycommand.sty}{\from{keycommand.dtx}{package}}%
   \file{keycommand-example.tex}{\from{keycommand.dtx}{example}}%
}

\generate{%
   \file{keycommand.drv}{\from{keycommand.dtx}{driver}}%
}

\obeyspaces
\Msg{************************************************************************}
\Msg{*}
\Msg{* To finish the installation you have to move the following}
\Msg{* file into a directory searched by TeX:}
\Msg{*}
\Msg{*     keycommand.sty}
\Msg{*}
\Msg{* To produce the documentation run the file `keycommand.dtx'}
\Msg{* through LaTeX.}
\Msg{*}
\Msg{* Happy TeXing!}
\Msg{*}
\Msg{************************************************************************}

\endbatchfile
%</install>
%<*ignore>
\fi
%</ignore>
%<*driver>
\edef\thisfile{\jobname}
\def\thisinfo{key-value interface for commands and environments in \LaTeX.}
\def\thisdate{2010/04/27}
\def\thisversion{3.1415}
\let\loadclass\LoadClass
\def\LoadClass#1{\loadclass[abstracton]{scrartcl}\let\scrmaketitle\maketitle\AtEndOfClass{\let\maketitle\scrmaketitle}}
\documentclass[a4paper,oneside]{ltxdoc}
\usepackage[latin9]{inputenc}
\usepackage[american]{babel}
\usepackage[T1]{fontenc}
\usepackage{etex,etoolbox,holtxdoc,geometry,tocloft,graphicx,xspace,fancyhdr,color,bbding,embedfile,framed,multirow,txfonts,makecell,enumitem,arydshln}
\CodelineNumbered
\usepackage{keyval}\makeatletter\let\keyval@setkeys\setkeys\makeatother
\usepackage{xkeyval}\let\xsetkeys\setkeys
\usepackage{kvsetkeys}
\usepackage{fancyvrb}
\lastlinefit999
\geometry{top=2cm,headheight=1cm,headsep=.3cm,bottom=1.4cm,footskip=.5cm,left=2.5cm,right=1cm}
\hypersetup{%
  pdftitle={The keycommand package},
  pdfsubject={key-value interface for commands and environments in LaTeX.},
  pdfauthor={Florent CHERVET},
  colorlinks,linkcolor=reflink,
  pdfstartview={FitH},
  pdfkeywords={tex, e-tex, latex, package, keys, keycommand, newcommand, keyval, kvsetkeys, programming},
  bookmarksopen=true,bookmarksopenlevel=3}
\embedfile{\thisfile.dtx}
\begin{document}
   \DocInput{\thisfile.dtx}
\end{document}
%</driver>
% \fi
%
% \CheckSum{1111}
%
% \CharacterTable
%  {Upper-case    \A\B\C\D\E\F\G\H\I\J\K\L\M\N\O\P\Q\R\S\T\U\V\W\X\Y\Z
%   Lower-case    \a\b\c\d\e\f\g\h\i\j\k\l\m\n\o\p\q\r\s\t\u\v\w\x\y\z
%   Digits        \0\1\2\3\4\5\6\7\8\9
%   Exclamation   \!     Double quote  \"     Hash (number) \#
%   Dollar        \$     Percent       \%     Ampersand     \&
%   Acute accent  \'     Left paren    \(     Right paren   \)
%   Asterisk      \*     Plus          \+     Comma         \,
%   Minus         \-     Point         \.     Solidus       \/
%   Colon         \:     Semicolon     \;     Less than     \<
%   Equals        \=     Greater than  \>     Question mark \?
%   Commercial at \@     Left bracket  \[     Backslash     \\
%   Right bracket \]     Circumflex    \^     Underscore    \_
%   Grave accent  \`     Left brace    \{     Vertical bar  \|
%   Right brace   \}     Tilde         \~}
%
% \DoNotIndex{\begin,\CodelineIndex,\CodelineNumbered,\def,\DisableCrossrefs,\~,\@ifpackagelater}
% \DoNotIndex{\DocInput,\documentclass,\EnableCrossrefs,\end,\GetFileInfo}
% \DoNotIndex{\NeedsTeXFormat,\OnlyDescription,\RecordChanges,\usepackage}
% \DoNotIndex{\ProvidesClass,\ProvidesPackage,\ProvidesFile,\RequirePackage}
% \DoNotIndex{\filename,\fileversion,\filedate,\let}
% \DoNotIndex{\@listctr,\@nameuse,\csname,\else,\endcsname,\expandafter}
% \DoNotIndex{\gdef,\global,\if,\item,\newcommand,\nobibliography}
% \DoNotIndex{\par,\providecommand,\relax,\renewcommand,\renewenvironment}
% \DoNotIndex{\stepcounter,\usecounter,\nocite,\fi}
% \DoNotIndex{\@fileswfalse,\@gobble,\@ifstar,\@unexpandable@protect}
% \DoNotIndex{\AtBeginDocument,\AtEndDocument,\begingroup,\endgroup}
% \DoNotIndex{\frenchspacing,\MessageBreak,\newif,\PackageWarningNoLine}
% \DoNotIndex{\protect,\string,\xdef,\ifx,\texttt,\@biblabel,\bibitem}
% \DoNotIndex{\z@,\wd,\wheremsg,\vrule,\voidb@x,\verb,\bibitem}
% \DoNotIndex{\FrameCommand,\MakeFramed,\FrameRestore,\hskip,\hfil,\hfill,\hsize,\hspace,\hss,\hbox,\hb@xt@,\endMakeFramed,\escapechar}
% \DoNotIndex{\do,\date,\if@tempswa,\@tempdima,\@tempboxa,\@tempswatrue,\@tempswafalse,\ifdefined,\ifhmode,\ifmmode,\cr}
% \DoNotIndex{\box,\author,\advance,\multiply,\Command,\outer,\next,\leavevmode,\kern,\title,\toks@,\trcg@where,\tt}
% \DoNotIndex{\the,\width,\star,\space,\section,\subsection,\textasteriskcentered,\textwidth}
% \DoNotIndex{\",\:,\@empty,\@for,\@gtempa,\@latex@error,\@namedef,\@nameuse,\@tempa,\@testopt,\@width,\\,\m@ne,\makeatletter,\makeatother}
% \DoNotIndex{\maketitle,\parindent,\setbox,\x,\kernel@ifnextchar}
% \DoNotIndex{\KVS@CommaComma,\KVS@CommaSpace,\KVS@EqualsSpace,\KVS@Equals,\KVS@Global,\KVS@SpaceEquals,\KVS@SpaceComma,\KVS@Comma}
% \DoNotIndex{\DefineShortVerb,\DeleteShortVerb,\UndefineShortVerb,\MakeShortVerb,\endinput}
% \let\ClearPage\clearpage
% \makeatletter
% \MakeShortVerb{\+}\DeleteShortVerb{\|}\DefineShortVerb{\|}
% \catcode`\� \active   \def�{\@ifnextchar �{\par\nobreak\vskip-2\parskip}{\par\nobreak\vskip-\parskip}}
% \def\thispackage{\xpackage{\thisfile}\xspace}
% \def\ThisPackage{\Xpackage{\thisfile}\xspace}
% \def\Xpackage{\@dblarg\X@package}
% \def\X@package[#1]#2{%
%     \xpackage{#2\footnote{\noindent\xpackage{#2}: \href{http://www.ctan.org/tex-archive/macros/latex/contrib/#1}{\nolinkurl{CTAN:macros/latex/contrib/#1}}}}}
% \def\Underbrace#1_#2{$\underbrace{\vtop to2ex{}\hbox{#1}}_{\footnotesize\hbox{#2}}$}
%
% \parindent\z@\parskip.4\baselineskip\topsep\parskip\partopsep\z@
% \g@addto@macro\macro@font{\macrocodecolor\let\AltMacroFont\macro@font}
% \g@addto@macro\@list@extra{\parsep\parskip\topsep\z@\itemsep\z@}
% \def\smex{\leavevmode\hb@xt@2em{\hfil$\longrightarrow$\hfil}}
% \newrobustcmd\verbfont{\usefont{T1}{\ttdefault}{\f@series}{n}}    \let\vb\verbfont
% \renewrobustcmd\#[1]{{\usefont{T1}{pcr}{bx}{n}\char`\##1}}
% \newrobustcmd\csred[1]{\textcolor{red}{\cs{#1}}}
% \renewrobustcmd\cs[2][]{\mbox{\vb#1\expandafter\@gobble\string\\#2}}
% \newrobustcmd\CSbf[1]{\textbf{\CS{#1}}}
% \newrobustcmd\csbf[2][]{\textbf{\cs[{#1}]{#2}}}
% \newrobustcmd\textttbf[1]{\textbf{\texttt{#1}}}
% \renewrobustcmd*\bf{\bfseries}\newcommand\nnn{\normalfont\mdseries\upshape}\newcommand\nbf{\normalfont\bfseries\upshape}
% \newrobustcmd*\blue{\color{blue}}\newcommand*\red{\color{dr}}\newcommand*\green{\color{green}}\newcommand\rred{\color{red}}
% \newrobustcmd\rrbf{\color{red}\bfseries}
% \definecolor{copper}{rgb}{0.67,0.33,0.00}  \newcommand\copper{\color{copper}}
% \definecolor{dg}{rgb}{0.16,0.33,0.00}      \newcommand\dg{\color{dg}}
% \definecolor{db}{rgb}{0,0,0.502}           \newcommand\db{\color{db}}
% \definecolor{dr}{rgb}{0.49,0.00,0.00}      \let\dr\red
% \newrobustcmd\bk{\color{black}}\newcommand\md{\mdseries}
%
% \fancyhf{}\fancyhead[L]{The \thispackage package -- \thisinfo}
% \fancyfoot[L]{\color[gray]{.35}\scriptsize\thispackage\quad[rev.\thisversion]\quad\copyright\oldstylenums{2009-2010}\,\lower.3ex\hbox{\NibRight}\,Florent Chervet}
% \fancyfoot[R]{\oldstylenums{\thepage} / \oldstylenums{\pageref{LastPage}}}
% \pagestyle{fancy}
% \fancypagestyle{plain}{%
%     \let\headrulewidth\z@
%     \fancyhf{}%
%     \fancyfoot[R]{\oldstylenums{\thepage} / \oldstylenums{\pageref{LastPage}}}}
%
% \newcommand\macrocodecolor{\color{macrocode}}\definecolor{macrocode}{rgb}{0.18,0.00,0.45}
% \newcommand\reflinkcolor{\color{reflink}}\definecolor{reflink}{rgb}{0.49,0.00,0.00}
% \font\umrandA=umranda at 20pt
% \def\@serp{\leavevmode\lower20pt\hbox{\umrandA\char'131}}
% \def\serp#1{\@serp\hfil #1\hfil\reflectbox{\@serp}}
% \newrobustcmd\stform{\@ifnextchar*{\@stform[]\textasteriskcentered\@gobble}\@stform}
% \newrobustcmd\@stform[2][\string]{\textttbf{\rred#1#2}\xspace}
%
% \makeatother
%
% \deffootnote{1em}{0pt}{\rlap{\textsuperscript{\thefootnotemark}}\kern1em}
%
% \title{\vskip-18pt\mdseries {\bfseries\ThisPackage}\kern.6em package}
% \author{\footnotesize\xemail{florent.chervet@free.fr}}
% \date{\thisdate~--~version \thisversion}
% \subtitle{\thisinfo}
% ^^A\subject{\vskip-2cm\serp{The completely redesigned}}
% \subject{\vskip-2cm\relax The \textit{free} and \textit{open source}}
%
% \maketitle
% 
% \makeatletter\begingroup\let\@thefnmark\@empty\let\@makefntext\@firstofone
% \footnotetext{\noindent
% This documentation is produced with the +DocStrip+ utility.
% \begin{tabbing}
% \qquad\=\smex\=To get the documentation, \= run (thrice):\quad\= \texttt{pdflatex keycommand.dtx} \\
% \qquad\>\>To get the index, \> run:\>\texttt{makeindex -s gind.ist keycommand.idx} \\
% \>\smex\>To get the package, \> run:\>        \texttt{etex keycommand.dtx}
% \end{tabbing}�
% The \xext{dtx} file is embedded into this pdf file thank to \xpackage{embedfile} by H. Oberdiek.}
% \endgroup\makeatother
% 
% \hypersetup{bookmarksopenlevel=3}
% \deffootnote{1em}{0pt}{\rlap{\thefootnotemark.}\kern1em}
% \vspace*{-18pt}
% \begin{abstract}\parindent0pt\noindent\leftskip1cm\rightskip\leftskip\lastlinefit0%
%
% \thispackage provides an easy way to define commands or environments
% with optional keys.
% \smallskip
%
% \csbf{newkeycommand} \cs{renewkeycommand} \cs{providekeycommand} and \csbf{newkeyenvironment},\linebreak 
% \cs{renewkeyenvironment} are macros to define such commands and environments with keys.
% 
% \thispackage is designed to make easier interface for user-defined commands.  In particular,
% \csbf{newkeycommand}\stform+ permits the use of key-commands in every context. 
% \medskip
%
% Keys are defined with the command itself in a very natural way.
% You can restrict the possible values for the keys by declaring them with a \textbf{type}.
% Available types for keys are : \textit{boolean}, \textit{enum} and \textit{choice} (see \ref{subsec:GeneralSyntax}).
%
% \smallskip
%
% The \thispackage package requires and is based on the package \xpackage{xkeyval} by Hendri Adriaens, and uses
% the \cs{kv@normalize} macro of \xpackage{kvsetkeys} (Heiko Oberdiek) for robustness, as shown
% in \ref{kvsetkeys-comparisons}).
%
% It works with an \eTeX{} distribution of \LaTeX.
% \end{abstract}
%
% \DeleteShortVerb{\+}\enlargethispage{2\baselineskip}
% \cftbeforesecskip=4pt plus2pt minus2pt
% \cftbeforesubsecskip=0pt plus2pt minus2pt
% \renewcommand\contentsname{Contents\quad\leaders\vrule height3.4pt depth-3pt\hfill\null\kern0pt\vskip-6pt}
% ^^A\vskip-.8\baselineskip
% \tableofcontents
%
% \clearpage\MakeShortVerb{\+}
%
% \def\B#1{\texttt{[}\meta{#1}\texttt{]}}
%
% \section{User Interface}
%
% \subsection{General syntax}\label{subsec:GeneralSyntax}
%
% \begin{declcs}{newkeycommand}%
%  \Underbrace{\textcolor{red}{\textasteriskcentered\string+[short-unexpand]}}_{\makecell[c]{modifiers \\ Optional}}\,%
%  \Underbrace{\M{command}}_{Required}\,%
%   {\color{db}\Underbrace{\B{keys=defaults}\,\B{OptKey}\,\B{<n>}}_{Optional}\,}%
%   \Underbrace{\M{definition}}_{Required}
% \end{declcs}
%
% \cs{newkeycommand} will define \cs{command} as a new key-command!\quad well...
%
% Use the \stform* form when you do not want it to be a \cs{long} macro (as for \LaTeX{}-\cs{newcommand}).
%
% The +[keys=defaults]+ argument define the keys with their default values. It is optional, but a key-command
% without keys seems to be useless (at least for me...). Keys may be defined as :
%
% \newlist{myenum}{enumerate}{1}
% \setlist[myenum]{label={},topsep=-\parskip,itemsep=-\parskip,parsep=\parskip,after=\vskip-\baselineskip}
%
% \renewcommand\theadfont{\tt\bfseries}
% \noindent\begin{tabular}{|c|>{\db}c|m{8cm}|}\hline
% \thead{Type} & \thead{exemple}                                              & \thead{value of \cs{commandkey}}                                                                                         \\ \hline
% general      & color{\dg=red}                                               & \cs{commandkey}\{{\db color}\} is `{\dg red}' and may be anything (text, number, macro...)                                           \\ \hline
% boolean      & {\rred bool} bold{\dg =true}                                 & \cs{commandkey}\{{\db bold}\} is:
%                                                                                                       \begin{myenum}
%                                                                                                       \item {\tt 0}\quad (for {\dg false})
%                                                                                                       \item {\tt 1}\quad (for {\dg true})
%                                                                                                       \end{myenum}          \\ \hline
% \multirow{2}*{enumerate} & {\rred enum} position{\dg=\{left,centered,right\}} & \cs{commandkey}\{{\db position}\} is:
%                                                                                                     \begin{myenum}
%                                                                                                     \item `{\dg left}'\quad by default and can be
%                                                                                                     \item `{\dg centered}' or
%                                                                                                     \item `{\dg right}'
%                                                                                                        \end{myenum}         \\ \cdashline{2-3}[1pt/2pt]
%                          & {\rred enum\textasteriskcentered} position{\dg=\{left,centered,right\}}  & This is the same, except match is case \textbf{in}sensitive   \bottopstrut                                   \\ \hline
% \multirow{2}*{choice}    & {\rred choice} position={\dg \{left,centered,right\}} & \cs{commandkey}\{{\db position}\} is:
%                                                                                                     \begin{myenum}
%                                                                                                     \item {\tt 0}\quad (for {\dg left} the default value),
%                                                                                                     \item {\tt 1}\quad (for {\dg centered})
%                                                                                                     \item {\tt 2}\quad (for {\dg right})
%                                                                                                     \end{myenum}                           \\ \cdashline{2-3}[1pt/2pt]
%                          & {\rred choice\textasteriskcentered} position={\dg\{left,centered,right\}} & This is the same, except match is case \textbf{in}sensitive   \bottopstrut                                   \\ \hline
% \end{tabular}
%
% The {\db+OptKey+} argument is used if you wish to capture the +key=value+ pairs that are not specifically defined (more on this in the examples section \ref{sec:examples}).
%
% The key-command may have {\tt 0} up to {\tt 9} \textbf{mandatory} arguments : specify the number by +<n>+ ({\tt 0} if omitted).
%
% The \stform+ form expands the \cs{commandkey} before executing the key-command itself, as explain in section \ref{sec:example:plus}.
%
% \subsection{First example :}
%
% \begin{tabbing}\label{textrule}
% \,\=\csbf{new}\=\textttbf{keycommand}\cs[\copper]{textrule}+[+{\color{db}+raise=.4ex,width=3em,thick=.4pt+}+][1]{%+ \\ ^^A+][1]{%+}\\
% \>\>\cs{rule}+[+\cs[\red]{commandkey}+{+{\db+raise+}+}]{+\cs[\red]{commandkey}+{+{\db+width+}+}{+\cs[\red]{commandkey}+{+{\db+thick+}+}}+\\
% \>\>\#1 \\
% \>\>\cs{rule}+[+\cs[\red]{commandkey}+{+{\db+raise+}+}]{+\cs[\red]{commandkey}+{+{\db+width+}+}}{+\cs[\red]{commandkey}+{+{\db+thick+}+}}}+
% \end{tabbing}
%
% defines the keys {\db+width+}, {\db+thick+} and {\db+raise+} with their default values (if not specified):
% {\db+3em+}, {\db+.4pt+} and {\db+.4ex+}. Now \cs[\copper]{textrule} can be used as follow:
% \begin{tabbing}
% \=1:\quad\=\cs[\copper]{textrule}+[width=2em]{hello}+\hskip2.5cm\=\smex\qquad\=        \rule[.4ex]{2em}{.4pt}hello\rule[.4ex]{2em}{.4pt} \\
% \>2:\>\cs[\copper]{textrule}+[thick=5pt,width=2em]{hello}+\>\smex\>                  \rule[.4ex]{2em}{5pt}hello\rule[.4ex]{2em}{5pt}\\
% \>3:\>\cs[\copper]{textrule}+{hello}+\quad \>\smex\>                                 \rule[.4ex]{3em}{.4pt}hello\rule[.4ex]{3em}{.4pt}\\
% \>4:\>\cs[\copper]{textrule}+[thick=2pt,raise=1ex]{hello}+\>\smex\>                  \rule[1ex]{3em}{2pt}hello\rule[1ex]{3em}{2pt} \\
% \> \textit{et c\ae tera}.
% \end{tabbing}
%
% \clearpage
%
% \subsection[Second example : the \string+ form]{Second example : the {\rred\bf\string+} form}
% \label{sec:example:plus}
%
% \DeleteShortVerb{\+}
% \begin{Verbatim}[gobble=1,commandchars=$(),frame=lines]
% ($bf\newkeycommand)($rred$bf+[\|])($copper\myfigure)[image,
%                              caption,
%                              enum placement={H,h,b,t,p},
%                              width=\textwidth,
%                              label=
%                             ][($db OtherKeys)]{%
%        ($rred|)($bf\begin){figure}($dr|)[($red\commandkey){placement}]
%           ($rred|)($bf\includegraphics)($dr|)[width=($red\commandkey){width},($red\commandkey){($db OtherKeys)}]{%
%                             ($red\commandkey){image}}%
%           ($dg\ifcommandkey){caption}{($rred|)\caption($rred|){($red\commandkey){caption}}}{}%
%           ($dg\ifcommandkey){label}{($rred|)\label($rred|){($red\commandkey){label}}}{}%
%        ($rred|)($bf\end){figure}($rred|)}
% \end{Verbatim}
% \MakeShortVerb{\+}
%
% With the \stform+ form of \cs{newkeycommand}, the definition will be expanded (at run time). The optional {\rred\bf+[\|]+} argument
% means that everything inside {\bf\rred+|+ ... +|+} is protected from expansion.
%
% {\dg\cs{ifcommandkey}}\{\meta{name}\}\{\meta{true}\}\{\meta{false}\}\quad expands \meta{true} if the commandkey \meta{name} is not blank.
%
% {\db \meta{Otherkeys}} captures the keys given by the user but not declared: they are simply given back to \cs{includegraphics} here...
%
%
% \subsection[Explanation of the \string+ form]{Explanation of the {\rred\bf\string+} form}
% \DeleteShortVerb{\+}
% The |\commankey{|\meta{name}|}| stuff is expanded at run time using the following scheme:��
% \begin{Verbatim}[gobble=1,commandchars=!(),frame=lines]
%     (!bf\newkeycommand)(!copper\keyMacro)[A=\defA,B=\defB,C=\defC,D=\defD][1]{(!dg\begingroup)
%        (!dg\edef)\keyMacro##1{(!dg\endgroup)
%            (!dg\noexpand)\Macro{(!red\getcommandkey){A}}
%                           {(!red\getcommandkey){B}}
%                           {(!red\getcommandkey){C}}
%                           {(!red\getcommandkey){D}}
%     }\keyMacro{#1}}
% \end{Verbatim}
% Therefore, the arguments of \cs{Macro} are ready: there is no more \cs{commandkey} stuff, but instead the values of the keys
% as you gave them to the key-command. \cs{getcommandkey}\{A\} is expanded to \cs{defA}.
%
% But \cs{defA} is not expanded of course: in the \stform+ form, \cs{commandkey} has the meaning of \cs{getcommandkey}.
%
% As you can see, the mandatory arguments \#1, \#2 etc. are \textbf{never expanded}: there is no need to protect them inside the special (usually {\rred\bf\textbar}) character.
%
%
% \MakeShortVerb{\+}
% \clearpage
%
% \subsection{key-environments}
%
% \begin{declcs}{newkeyenvironment}%
%  \Underbrace{\textcolor{red}{\textasteriskcentered\string+[short-unexpand]}}_{\makecell[c]{modifiers \\ Optional}}\,%
%  \Underbrace{\M{envir name}}_{Required}\,%
%   {\db\Underbrace{\B{keys=defaults}\,\B{OptKey}\,\B{<n>}}_{Optional}\,}%
%   \Underbrace{\M{begin}}_{Required}\Underbrace{\M{end}}_{Required}
% \end{declcs}
%
% In the same way, you may define environments with optional keys as follow:�
% \begin{tabbing}
% \qquad\=+\newkeyenvironment+\=+{EnvirWithKeys}[kOne=+default value,...+][n]+\\
% \>\>+{+ commands at begin +EnvirWithKeys }+ \\
% \>\>+{+ commands at end +EnvirWithKeys }+
% \end{tabbing}
%
% Where $n$ is the number of mandatory other arguments (\emph{ie} without keys), if any.
%
% Key-environments may be defined with the \stform+ form in the same way as \cs{newkeycommand} is used.
% Be aware that each part of the environment: \meta{begin} and \meta{end} are expanded at run time then, 
% and the optional {\rred\bf+[\|]+} argument protects from expansion in each of those parts.
% 
% \subsection[Example of a \string+ key-environment]{Example of a {\rred\bf\string+} key-environment}
% 
% \DeleteShortVerb{\+}
% \begin{Verbatim}[gobble=1,commandchars=$(),frame=lines]
% ($bf\newkeyenvironment)($rred$bf+[\|])({$copper myfigure)}[
%                              caption,
%                              enum placement={H,h,b,t,p},
%                              width=.5\linewidth,
%                              label
%                             ][($db OtherKeys)][1]%
%     {% ($nbf$dg begin part)
%        ($rred|)($bf\begin){figure}($rred|)[($red\commandkey){placement}]
%           ($rred|)($bf\includegraphics)($rred|)[($red\commandkey){($db OtherKeys)},width=($red\commandkey){width}]{$#1}%
%     }
%     {% ($nbf$dg end part)
%           ($dg\ifcommandkey){caption}{($rred|)\caption($rred|){($red\commandkey){caption} image file = $#1}}{}%
%           ($dg\ifcommandkey){label}{($rred|)\label($rred|){($red\commandkey){label}}}{}%
%        ($rred|)($bf\end){figure}($rred|)%
%     }
% \end{Verbatim}
% \MakeShortVerb{\+}
%
% As you can see, \cs{commandkey} and mandatory arguments (\#1 here) are available both in the \meta{begin} 
% and in the \meta{end} parts of the key-environment.
% 
%
% \DefineShortVerb{\+}
%
% \section{Messages and more}
%
% \subsection{Invalid keys}
%
% If you use the command +\textule+ (defined in \ref{textrule}) with a key say: +height+
% that has not been declared at the definition of the key-command, you will get an
% error message like this:
% \begin{quote}\tt
% The key-value pairs ``height=...''�
% cannot be processed for key-command \string\textrule!�
% See the definition of the keycommand!
% \end{quote}
% The error is recoverable: the key is ignored.
%
% If you assign a value to an \textit{enum} or a \textit{choice} key, which is not allowed in the definition,
% you will get the following message:
% \begin{quote}\tt
% The value ``...'' is not allowed in key ...�
% for key-command \string\command�
% I'll use the default value ``...'' for this key instead�
% See the definition of the key-command!
% \end{quote}
% The error is recoverable: the key is assigned its default value.
%
% If you use a \cs{commandkey}\{\meta{name}\} in a key-command where \meta{name} is not defined as a key,
% you will get the \TeX{} generic error message :�
% \qquad undefined control sequence : \cs{keycmd->...@name}.
%
%
% \subsection{Testing keys}
%
% \begin{declcs}{ifcommandkey}\,\M{key name}\,\M{commands if key is NOT blank}\,\M{commands if key is blank}
% \end{declcs}
%
% When you define a key command you may let the default value of a key empty. Then, you may wish to
% expand some commands only if the key has been given by the user (with a non empty value). This can
% be achieved using the macro |\ifcommandkey|.
%
% \clearpage
% \subsection{xkeyval, keyval and kvsetkeys comparison}
%
% \begin{tabbing}
% \quad\=\xpackage{xkeyval}: \expandafter\meaning\csname ver@xkeyval.sty\endcsname \\
% \>\xpackage{keyval}: \expandafter\meaning\csname ver@keyval.sty\endcsname \\
% \>\xpackage{kvsetkeys}: \expandafter\meaning\csname ver@kvsetkeys.sty\endcsname
% \end{tabbing}
%
% \makeatletter\def\theadfont{\tt\bfseries}
% \define@key{fam}{key}{\def\result{#1}}
% \begin{table}[h]\label{kvsetkeys-comparisons}
% \begin{tabular}{|l|l|>{\color{db}}l|>{\color{dg}}l|}\hline
% \thead{\bf Example} & \thead{keyval} & \thead{xkeyval} & \thead{\makecell{kvsetkeys\\and\\keycommand}} \\ \hline
% +\setkeys{fam}{key={{value}}}+
%     & \keyval@setkeys{fam}{key={{value}}}\meaning\result
%     & \xsetkeys{fam}{key={{value}}}\meaning\result
%     & \kvsetkeys{fam}{key={{value}}}\meaning\result \\\hline
% +\setkeys{fam}{key={{{value}}}}+
%     & \keyval@setkeys{fam}{key={{{value}}}}\meaning\result
%     & \xsetkeys{fam}{key={{{value}}}}\meaning\result
%     & \kvsetkeys{fam}{key={{{value}}}}\meaning\result \\\hline
% +\setkeys{fam}{key=+\textvisiblespace+{{{value}}}}+
%     & \keyval@setkeys{fam}{key= {{{value}}}}\meaning\result
%     & \xsetkeys{fam}{key= {{{value}}}}\meaning\result
%     & \kvsetkeys{fam}{key= {{{value}}}}\meaning\result \\\hline
% \end{tabular}
% \caption{Then it is clear that, at this time, \xpackage{kvsetkeys} has the only correct behaviour...}
% \end{table}
%
% In \thispackage the key-value pairs are first normalized using \xpackage{kvsetkeys}-\cs{kv@normalize}. Then braces are added
% around the values in order to keep the good behaviour of \xpackage{kvsetkeys} while using \xpackage{xkeyval}.
% \makeatother
%
%
%
%
% \StopEventually{
% }
%
% \begin{center}\vskip6pt$\star$\hskip4em\lower12pt\hbox{$\star$}\hskip4em$\star$\vadjust{\vskip12pt}\end{center}
%
% \section{Implementation} \label{Implementation}
% \csdef{HDorg@PrintMacroName}#1{\hbox to4em{\strut \MacroFont \string #1\ \hss}}
%
% \subsection{Identification}
%
% This package is intended to use with \LaTeX{} so we don't check if it is loaded twice.
%
%    \begin{macrocode}
%<*package>
\NeedsTeXFormat{LaTeX2e}% LaTeX 2.09 can't be used (nor non-LaTeX)
   [2005/12/01]% LaTeX must be 2005/12/01 or younger (see kvsetkeys.dtx).
\ProvidesPackage{keycommand}
   [2010/04/27 v3.1415 - key-value interface for commands and environments in LaTeX]
%    \end{macrocode}
%
% \subsection{Requirements}
%
% The package is based on \xpackage{xkeyval}. However, \xpackage{xkeyval} is far less reliable
% than \xpackage{kvsetkeys} as far as spaces and bracket (groups) are concerned, as shown in the section
% \ref{kvsetkeys-comparisons} of this documentation.
%
% Therefore, we also use the macros of \xpackage{kvsetkeys} in order to \textit{normalize} the \texttt{key=value}
% list before setting the keys. This way, we take advantage of both \xpackage{xkeyval} and \xpackage{kvsetkeys} !
%
% As long as we use \eTeX{} primitives in \xpackage{keycommand} we also load the
% \xpackage{etex} package in order to get an error message if \eTeX{} is not running.
%
% The \xpackage{etoolbox} package gives some facility to write \xpackage{keycommand}.
% 
% From version \texttt{3.141} onwards, \thispackage does not load \xpackage{etextools} anymore.
%
%    \begin{macrocode}
\def\kcmd@pkg@name{keycommand}
\RequirePackage{etex,kvsetkeys,xkeyval,etoolbox}
%    \end{macrocode}
%
% Save the \cs{setkeys} macro of \xpackage{xkeyval} package (in case it was overwritten by a
% subsequent load of \xpackage{kvsetkeys} or \xpackage{keyval} for example :
%    \begin{macrocode}
\protected\def\kcmd@Xsetkeys{\XKV@sttrue\XKV@plfalse\XKV@testoptc\XKV@setkeys}% in case \setkeys 
%                                                                                was overwritten
%    \end{macrocode}
% Some \cs{catcode} assertions internally used by \thispackage:
%    \begin{macrocode}
\let\kcmd@AtEnd\@empty
\def\TMP@EnsureCode#1#2{%
  \edef\kcmd@AtEnd{%
    \kcmd@AtEnd
    \catcode#1 \the\catcode#1\relax
  }%
  \catcode#1 #2\relax
}
\TMP@EnsureCode{32}{10}% space
\TMP@EnsureCode{61}{12}% = sign
\TMP@EnsureCode{45}{12}% - sign
\TMP@EnsureCode{62}{12}% > sign
\TMP@EnsureCode{46}{12}% . dot
\TMP@EnsureCode{47}{8}% / slash (etextools)
\AtEndOfPackage{\kcmd@AtEnd\undef\kcmd@AtEnd}
%    \end{macrocode}
% 
% \begin{macro}{\kcmd@ifstrdigit}\qquad\qquad
% This macro is used too test the optional arguments of \cs{newkeycommand}, 
% in particular, one must know in an argument is a single digit (representing
% the number of mandatory arguments) or anything else (representing the \texttt{key=value} 
% list or the ``special'' \texttt{OptKey} key:
%    \begin{macrocode}
\iffalse%\ifdefined\pdfmatch% use \pdfmatch if present
   \long\def\kcmd@ifstrdigit#1{\csname @\ifnum\pdfmatch
      {\detokenize{^[[:space:]]*[[:digit:]][[:space:]]*$}}{\detokenize{#1}}=1 %
      first\else second\fi oftwo\endcsname}
\else% use filter, very efficient !
\def\kcmd@ifstrdigit#1{%
   \kcmd@nbk#1//%
      {\expandafter\expandafter\expandafter\kcmd@ifstrdigit@i
         \expandafter\expandafter\expandafter{\detokenize\expandafter{\number\number0#1}}}%
      {\@secondoftwo}//%
}
\def\kcmd@ifstrdigit@i#1{%
   \def\kcmd@ifstrdigit@ii##1#1##2##3\kcmd@ifstrdigit@ii{%
      \csname @\ifx##20first\else second\fi oftwo\endcsname
      }\kcmd@ifstrdigit@ii 00 01 02 03 04 05 06 07 08 09 0#1 \relax\kcmd@ifstrdigit@ii
}
\fi
%    \end{macrocode}
% \end{macro}
%
% \subsection{Defining (and undefining) command-keys}
%\begin{macro}{\kcmd@keyfam}\qquad
% The macro expands to the family-name, given the keycommand name:
%    \begin{macrocode}
\def\kcmd@keyfam#1{\detokenize{keycmd->}\expandafter\@gobble\string#1}
%    \end{macrocode}
% \end{macro}
% \begin{macro}{\kcmd@nbk}\qquad is the optimized \cs{ifnotblank} macro of \xpackage{etoolbox}
% (with \textttbf{/} having a catcode of 8):
%    \begin{macrocode}
\def\kcmd@nbk#1#2/#3#4#5//{#4}%
%    \end{macrocode}
% \end{macro}
%
% \begin{macro}{\kcmd@normalize@setkeys}~\par
% This macro assigns the values to the keys (expansion of \xpackage{xkeyval}-\cs{setkeys}
% on the result of \xpackage{kvsetkeys}-\cs{kv@normalize}). Braces are normalized too so that
% \verb+key=+\textvisiblespace+{{{value}}}+ is the same as \verb+key={{{value}}}+ as explained in section \ref{kvsetkeys-comparisons}:
%    \begin{macrocode}
\newrobustcmd\kcmd@normalize@setkeys[4]{%
% #1 = key-command,
% #2 = family,
% #3 = other-key,
% #4 = key-values pairs
   \kv@normalize{#4}\toks@{}%
   \expandafter\kv@parse@normalized\expandafter{\kv@list}{\kcmd@normalize@braces{#2}}%
   \edef\kv@list{\kcmd@Xsetkeys{\unexpanded{#2}}{\the\toks@}}\kv@list
   \kcmd@nbk#3//% undeclared keys are assigned to "OtherKeys"
      {\cslet{#2@#3}\XKV@rm}% (if specified, ie not empty)
      {\expandafter\kcmd@nbk\XKV@rm//% (otherwise a recoverable error is thown)
         {\PackageError\kcmd@pkg@name{The key-value pairs :\MessageBreak
         \XKV@rm\MessageBreak
         cannot be processed for key-command \string#1\MessageBreak
         See the definition of the key-command!}{}}{}//}//%
}
\long\def\kcmd@normalize@braces#1#2#3{% This is kvsetkeys processor for normalizing braces
   \toks@\expandafter{\the\toks@,#2}%
   \ifx @\detokenize{#3}@\else \toks@\expandafter{\the\toks@={{{#3}}}}\fi
}
%    \end{macrocode}
% \end{macro}
% 
% \begin{macro}{\kcmd@definekey}~\par
% \CS{kcmd@definekey} define the keys declared for the key-command.
% It is used as the \emph{processor} for the \cs{kv@parse} macro of \xpackage{kvsetkeys}.
% The macro appends the key names to the key list: ``\textit{family}.keylist''.
%
% keys are first checked for their type (bool, enum, enum*, choice or choice*) :
%
%    \begin{macrocode}
\def\kcmd@check@typeofkey#1{% expands to
% 0 if key has no type,
% 1 if boolean,
% 2 if enum*,
% 3 if enum,
% 4 if choice*,
% 5 if choice
   \kcmd@check@typeofkey@bool#1bool //%
      {\kcmd@check@typeofkey@enumst#1enum* //%
         {\kcmd@check@typeofkey@enum#1enum //%
            {\kcmd@check@typeofkey@choicest#1choice* //%
               {\kcmd@check@typeofkey@choice#1choice //%
                  05//}4//}3//}2//}1//}
\def\kcmd@check@typeofkey@bool #1bool #2//{\kcmd@nbk#1//}
\def\kcmd@get@keyname@bool #1bool #2//{#2}
\def\kcmd@check@typeofkey@enumst #1enum* #2//{\kcmd@nbk#1//}
\def\kcmd@get@keyname@enumst #1enum* #2//{#2}
\def\kcmd@check@typeofkey@enum #1enum #2//{\kcmd@nbk#1//}
\def\kcmd@get@keyname@enum #1enum #2//{#2}
\def\kcmd@check@typeofkey@choicest #1choice* #2//{\kcmd@nbk#1//}
\def\kcmd@get@keyname@choicest #1choice* #2//{#2}
\def\kcmd@check@typeofkey@choice #1choice #2//{\kcmd@nbk#1//}
\def\kcmd@get@keyname@choice #1choice #2//{#2}
%
\protected\long\def\kcmd@definekey#1#2#3#4#5{% define the keys using xkeyval macros
% #1 = keycommand,
% #2 = \global,
% #3 = family,
% #4 = key (before = sign),
% #5 = default (after = sign)
   \ifcase\kcmd@check@typeofkey{#4}\relax% standard
      #2\csedef{#3.keylist}{\csname#3.keylist\endcsname,#4}%
      \define@cmdkey{#3}[{#3@}]{#4}[{#5}]{}%
   \or% bool
      #2\csedef{#3.keylist}{\csname#3.keylist\endcsname,\kcmd@get@keyname@bool#4//}%
      \kcmd@define@boolkey#1{#3}{\kcmd@get@keyname@bool#4//}{#5}%
   \or% enum*
      #2\csedef{#3.keylist}{\csname#3.keylist\endcsname,\kcmd@get@keyname@enumst#4//}%
      \kcmd@define@choicekey#1*{#3}{\kcmd@get@keyname@enumst#4//}{#5}{\expandonce\val}%
   \or% enum
      #2\csedef{#3.keylist}{\csname#3.keylist\endcsname,\kcmd@get@keyname@enum#4//}%
      \kcmd@define@choicekey#1{}{#3}{\kcmd@get@keyname@enum#4//}{#5}{\expandonce\val}%
   \or% choice*
      #2\csedef{#3.keylist}{\csname#3.keylist\endcsname,\kcmd@get@keyname@choicest#4//}%
      \kcmd@define@choicekey#1*{#3}{\kcmd@get@keyname@choicest#4//}{#5}{\number\nr}%
   \or% choice
      #2\csedef{#3.keylist}{\csname#3.keylist\endcsname,\kcmd@get@keyname@choice#4//}%
      \kcmd@define@choicekey#1{}{#3}{\kcmd@get@keyname@choice#4//}{#5}{\number\nr}%
   \fi
   \ifx#2\global\relax
      #2\csletcs{KV@#3@#4}{KV@#3@#4}% globalize
      #2\csletcs{KV@#3@#4@default}{KV@#3@#4@default}% globalize default value
   \fi
}
%
\long\def\kcmd@firstchoiceof#1,#2\kcmd@nil{\unexpanded{#1}}
%
\long\def\kcmd@define@choicekey#1#2#3#4#5#6{%
   \begingroup\edef\kcmd@define@choicekey{\endgroup
      \noexpand\define@choicekey#2+{#3}{#4}
            [\noexpand\val\noexpand\nr]%
            {\unexpanded{#5}}% list of allowed values
            [{\kcmd@firstchoiceof#5,\kcmd@nil}]% default value
            {\csedef{#3@#4}{\unexpanded{#6}}}% define key value if in the allowed list
            {\kcmd@error@handler\noexpand#1{#3}{#4}{\kcmd@firstchoiceof#5,\kcmd@nil}}% error handler
   }\kcmd@define@choicekey
}
\def\kcmd@define@boolkey#1#2#3#4{\begingroup
   \kcmd@nbk#4//{\def\default{#4}}{\def\default{true}}//%
   \edef\kcmd@define@boolkey{\endgroup
      \noexpand\define@choicekey*+{#2}{#3}[\noexpand\val\noexpand\nr]%
            {false,true}
            [{\unexpanded\expandafter{\default}}]%
            {\csedef{#2@#3}{\noexpand\number\noexpand\nr}}%
            {\kcmd@error@handler\noexpand#1{#2}{#3}{\unexpanded\expandafter{\default}}}%
   }\kcmd@define@boolkey
}
%
\protected\long\def\kcmd@error@handler#1#2#3#4{%
% #1 = key-command,
% #2 = family,
% #3 = key,
% #4 = default
   \PackageError\kcmd@pkg@name{%
      Value `\val\space' is not allowed in key #3\MessageBreak
      for key-command \string#1.\MessageBreak
      I'll use the default value `#4' for this key.\MessageBreak
      See the definition of the key-command!}{%
      \csdef{#2@#3}{#4}}}
%    \end{macrocode}
% \end{macro}
%
% \begin{macro}{\kcmd@undefinekeys}~\par
% Now in case we redefine a key-command, we would like the old keys (\emph{ie} the keys
% associated to the old definition of the command) to be cleared, undefined.
% That's the job of \cs{kcmd@undefinekeys}.
%    \begin{macrocode}
\protected\def\kcmd@undefinekeys#1#2{% #1 = global, #2 = family
   \ifcsundef{#2.keylist}
      {\cslet{#2.keylist}\@gobble}
      {\expandafter\expandafter\expandafter\docsvlist
         \expandafter\expandafter\expandafter{%
                        \csname #2.keylist\endcsname}%
      \cslet{#2.keylist}\@gobble}%
}
\def\kcmd@undefinekey#1#2#3{% #1 = global, #2 = family, #3 = key
   #1\csundef{KV@#2@#3}%
   #1\csundef{KV@#2@#3@default}%
}
%    \end{macrocode}
% \end{macro}
% 
%\begin{macro}{\kcmd@setdefaults}\qquad\qquad
% sets the defaults values for the keys at the very beginning of the keycommand:
%    \begin{macrocode}
\def\kcmd@setdefaults#1{%
   \ifcsundef{#1.keylist}{}
   {\expandafter\expandafter\expandafter\docsvlist
      \expandafter\expandafter\expandafter{%
                           \csname#1.keylist\endcsname}}%
}
%    \end{macrocode}
%\end{macro}
% 
% 
%
% \begin{macro}{\kcmd@def}
% checks \cs{@ifdefinable} and cancels definition if needed:
%    \begin{macrocode}
\protected\long\def\kcmd@def#1#2[#3][#4][#5]#6#7{%
   \ifx#1\kcmd@donot@provide  \endgroup
   \else
      \@tempswafalse\@ifdefinable#1{\@tempswatrue}%
      \if@tempswa
         \edef\kcmd@fam{\kcmd@keyfam{#1}}%
         \expandafter\kcmd@defcommand\expandafter{\kcmd@fam}#1[{#3}][{#4}][{#5}]{#6}{#2}{#7}%
      \else\endgroup
      \fi
   \fi
}
%    \end{macrocode}
% \end{macro}
% 
% \begin{macro}{\kcmd@defcommand}\qquad\qquad prepares (expands) the arguments before closing the group opened at the very beginning.
% Then it proceeds (\cs{kcmd@yargdef} (normal interface)  or \cs{kcmd@yargedef} (when \cs{newkeycommand}\stform+ is used))
%    \begin{macrocode}
\protected\long\def\kcmd@defcommand#1#2[#3][#4][#5]#6#7#8{%
   \let\commandkey\relax  \let\getcommandkey\relax  \let#2\relax   
   \cslet{#1}\relax  \cslet{#1.commankey}\relax  \cslet{#1.getcommandkey}\relax
   \def\do{\kcmd@undefinekey{\kcmd@gbl}{#1}}%
   \edef\kcmd@defcommand{\endgroup
      \kcmd@undefinekeys{\kcmd@gbl}{#1}% undefines all keys for this keycommand family
      \ifx\kcmd@unexpandchar\@empty\else
         \kcmd@mount@unexpandchar{#1}{\unexpanded\expandafter{\kcmd@unexpandchar}}%
      \fi
      \unexpanded{\kv@parse{#3,#4}}{\kcmd@definekey\noexpand#2{\kcmd@gbl}{#1}}% defines keys
      \csdef{#1.commandkey}####1{\noexpand\csname#1@####1\endcsname}%
      \csdef{#1.getcommandkey}####1{%
         \unexpanded{\unexpanded\expandafter\expandafter\expandafter}{%
                           \noexpand\csname#1@####1\endcsname}}%
      \kcmd@ifplus% \newkeycommand+ / \newkeyenvironment+
         \protected\csdef{#1}{%
            \kcmd@yargedef{\kcmd@gbl}{\kcmd@long}\csname#1\endcsname
                          {\number#5}{\noexpand#7}{\csname#1.unexpandchar\endcsname}}%
         \ifx#7\@gobble\else 
             \protected\def#7{\kcmd@yargedef#7}%
         \fi
      \else% \newkeycommand / \newkeyenvironment
         \csdef{#1}{%
            \kcmd@yargdef{\kcmd@gbl}{\kcmd@long}\csname#1\endcsname
                          {\number#5}{\noexpand#7}}%
         \ifx#7\@gobble\else \def#7####1{% that means we have to define a key-environment
            \def#7{%
               \let\getcommandkey\csname#1.getcommandkey\endcsname
               \let\commandkey\csname#1.commandkey\endcsname
               ####1}%
            }%
         \fi
      \fi
      \def\noexpand\do####1{\unexpanded{\expandafter\noexpand\csname}KV@#1@####1@default%
                                                                                     \endcsname}% 
      \let\commandkey\relax \let\getcommandkey\relax \let#2\relax
      \kcmd@gbl\protected\edef#2{% entry point
         \let\getcommandkey\noexpand\noexpand\csname#1.getcommandkey\endcsname
         \kcmd@ifplus  \let\commandkey\getcommandkey
         \else         \let\commandkey\noexpand\noexpand\csname#1.commandkey\endcsname
         \fi
         \noexpand\kcmd@setdefaults{#1}%
         \ifx#7\@gobble \noexpand\noexpand\noexpand\@testopt
                        {\kcmd@setkeys#2{#1}{\kcmd@otherkey{#4}}}{}%
         \else          \noexpand\noexpand\noexpand\@testopt
                        {\kcmd@setkeys#2{#1}{\kcmd@otherkey{#4}}}{}%
         \fi
         }%
      \csname#1\endcsname% expand \kcmd@yargedef or \kcmd@yargdef
   }\kcmd@defcommand{#6}{#8}% #6 = definition, #8 = definition end-envir
}
\protected\long\def\kcmd@setkeys#1#2#3[#4]{% #1=key-command, #2=family, #3=otherkey, #4=key=value pairs
   \kcmd@normalize@setkeys{#1}{#2}{#3}{#4}\csname#2\endcsname
}
\long\def\kcmd@otherkey#1{\kcmd@nbk#1//{\kcmd@otherkey@name#1=\kcmd@nil}{}//}
\long\def\kcmd@otherkey@name#1=#2\kcmd@nil{#1}
%    \end{macrocode}
% \end{macro}
%
% \begin{macro}{\kcmd@mount@unexpandchar}~\par
% This macro defines the macro \cs{"\textit{family.unexpandchar}"}. 
% \CS{"\textit{family.unexpandchar}"} activates the shortcut character 
% for \cs{unexpanded} and defines its meaning.
%    \begin{macrocode}
\protected \def \kcmd@mount@unexpandchar#1#2{%
   \protected\csdef{#1.unexpandchar}{\begingroup
      \catcode`\~\active \lccode`\~`#2 \lccode`#2 0\relax
         \lowercase{%
            \expandafter\endgroup\expandafter\def\expandafter~{%
               \catcode`#2\active
               \long\def~########1~{\unexpanded{########1}}}%
         ~}%
   }%
}
%    \end{macrocode}
% \end{macro}
%
%----------------------------------------------------------------------------
% \begin{macro}{\kcmd@yargdef}\qquad\qquad
% This is the ``{\tt argdef}'' macro for the normal (non \string+) form:
%    \begin{macrocode}
\protected \def \kcmd@yargdef #1#2#3#4#5{\begingroup
% #1 = global or {}
% #2 = long or {}
% #3 = Command
% #4 = nr of args
% #5 = endenvir (or \@gobble if not an environment, or \relax if #3 is endenvir)
   \def \kcmd@yargd@f ##1#4##2##{\afterassignment#5\endgroup
      #1#2\expandafter\def\expandafter#3\@gobble ##1#4%
   }\kcmd@yargd@f 0##1##2##3##4##5##6##7##8##9###4%
}
%    \end{macrocode}
% \end{macro}
%
% \begin{macro}{\kcmd@yargedef}\qquad\qquad
% This is the ``{\tt argdef}'' macro for the {\rred\bf\string+} form:
%    \begin{macrocode}
\protected \def \kcmd@yargedef#1#2#3#4#5#6{\begingroup
% #1 = global or {}
% #2 = long or {}
% #3 = Command
% #4 = nr of args
% #5 = endenvir (or \@gobble if not an environment, or \relax if #3 is endenvir)
% #6 = unexpandchar mounting macro
  \kcmd@nargs{#4}% 
   \protected\long\def\kcmd@yarg@edef##1##2{\endgroup
         #1\edef#3{\begingroup #6%
            #2\edef#3\unexpanded{##2}{\endgroup\unexpanded{##1}%
         }#3}%
   }%
   \protected\def\kcmd@envir##1{%
      \edef\next{\kcmd@yarg@edef{\def\noexpand#5{\expandonce{#5##1}}\expandonce{#3##1}}}\next
   }%
   \protected\def\kcmd@command##1{%
      \edef\next{\kcmd@yarg@edef{\expandonce{#3##1}}}\next
   }%
   \protected\def\kcmd@yargedef##1{%
      \kcmd@yargedef@##1 0####1####2####3####4####5####6####7####8####9#####4%
   }%
   \ifx#5\@gobble % keycommand
      \def\next{\kcmd@command}%
   \else          % key-environmment
      \def\next{\kcmd@envir}%
   \fi
   \let\@next\relax
   \def\kcmd@yargedef@##1##2#4##3##{%
      \ifx\@next\relax 
         \edef\@next{\next{\expandonce{\kcmd@nargs}}{\expandonce{\@gobble##2#4}}}%
         \ifx#5\@gobble \edef\@next{\expandonce\@next\noexpand#5}%
         \else \edef\@next{\edef\noexpand\@next{\noexpand\unexpanded{\expandonce\@next}}#5}%
         \fi
      \fi
      \afterassignment\@next
      \expandafter\def\expandafter##1\@gobble##2#4%
   }%
   \kcmd@yargedef#3%
}
%    \end{macrocode}
% \end{macro}
%
% \begin{macro}{\kcmd@nargs}\qquad
% Filter macros used by \cs{kcmd@yargedef} to get the correct number of arguments:
%    \begin{macrocode}
\def\kcmd@nargs#1{\edef\kcmd@nargs%##1##2##3##4##5##6##7##8##9%
        {\ifnum#1>0{####1%
         \ifnum#1>1}{####2%
         \ifnum#1>2}{####3%
         \ifnum#1>3}{####4%
         \ifnum#1>4}{####5%
         \ifnum#1>5}{####6%
         \ifnum#1>6}{####7%
         \ifnum#1>7}{####8%
         \ifnum#1>8}{####9%
         \fi\fi\fi\fi\fi\fi\fi\fi}\fi}%
}%
%    \end{macrocode}
% \end{macro}
%
% \subsection{new key-commands}
%
% \begin{macro}{\newkeycommand}\qquad\qquad
% Here are the entry points:
%    \begin{macrocode}
\newrobustcmd*\newkeycommand{\begingroup
   \let\kcmd@gbl\@empty\kcmd@star@or@long\new@keycommand}
\newrobustcmd*\renewkeycommand{\begingroup
   \let\kcmd@gbl\@empty\kcmd@star@or@long\renew@keycommand}
\newrobustcmd*\providekeycommand{\begingroup
   \let\kcmd@gbl\@empty\kcmd@star@or@long\provide@keycommand}
%    \end{macrocode}
% \end{macro}
%
% \begin{macro}{\kcmd@star@or@long}~\par
% This is the adaptation of \LaTeX's \cs{@star@or@long} macro:
%    \begin{macrocode}
\def\kcmd@star@or@long#1{\@ifstar
      {\let\kcmd@long\@empty\kcmd@plus#1}
      {\def\kcmd@long{\long}\kcmd@plus#1}}
\def\kcmd@@ifplus#1{\@ifnextchar +{\@firstoftwo{#1}}}% same as LaTeX's \@ifstar
\def\kcmd@plus#1{\kcmd@@ifplus
      {\def\kcmd@ifplus{\iftrue}\kcmd@testopt#1}
      {\def\kcmd@ifplus{\iffalse}\kcmd@testopt#1}}
\def\kcmd@testopt#1{\@testopt{\kcmd@unexpandchar#1}{}}
%    \end{macrocode}
% \end{macro}
%
%\begin{macro}{\kcmd@unexpandchar}\qquad\qquad\quad
% Reads the possible unexpand-char shortcut:
%    \begin{macrocode}
\def\kcmd@unexpandchar#1[#2]{%
   \kcmd@ifplus
      \kcmd@nbk#2//
         {\def\kcmd@unexpandchar{#2}% only once inside group...
          \def\kcmd@unexpandchar@activate{\catcode`#2 \active}%
         }{%
          \let\kcmd@unexpandchar\@empty
          \let\kcmd@unexpandchar@activate\relax
         }//%
   \else  \let\kcmd@unexpandchar\@empty
      \kcmd@nbk#2//%
         {\PackageError\kcmd@pkg@name{shortcut option for \string\unexpanded\MessageBreak
         You can't use a shortcut option for \string\unexpanded\MessageBreak
         without the \string+ form of \string\newkeycommand!}%
         {I will ignore this option and proceed.}%
         }%
         {}//%      
   \fi#1}
%    \end{macrocode}
%\end{macro}
%
% \begin{macro}{\new@keycommand}\qquad\qquad
% Reads the key-command name (cs-token):
%    \begin{macrocode}
\def\new@keycommand#1{\@testopt{\@newkeycommand#1}0}
%    \end{macrocode}
% \end{macro}
%
%\begin{macro}{\@newkeycommand}\qquad\qquad
% Reads the first optional parameter (keys or number of mandatory args):
%    \begin{macrocode}
\long\def\@newkeycommand#1[#2]{% #2 = key=values or N=mandatory args
   \kcmd@ifplus \kcmd@unexpandchar@activate \fi% activates unexpand-char before reading definition
   \kcmd@ifstrdigit{#2}%
      {\@new@key@command#1[][][{#2}]}% no kv, no optkey, number of args
      {\@testopt{\@new@keycommand#1[{#2}]}0}}% kv, check for optkey/nr of args
%    \end{macrocode}
% \end{macro}
%
%\begin{macro}{\@new@keycommand}\qquad\qquad
% Reads the second optional parameter (opt key or number of mandatory args):
%    \begin{macrocode}
\long\def\@new@keycommand#1[#2][#3]{%
   \kcmd@ifstrdigit{#3}%
      {\@new@key@command#1[{#2}][][{#3}]}% no optkey
      {\@testopt{\@new@key@command#1[{#2}][{#3}]}0}}
%    \end{macrocode}
%\end{macro}
%
%\begin{macro}{\@new@key@command}\qquad\qquad
% Reads the definition of the command (\cs{kcmd@def} handles both cases of commands and environements).
% The so called "unexpand-char shortcut" has been activated before reading command definition:
%    \begin{macrocode}
\long\def\@new@key@command#1[#2][#3][#4]#5{%
      \kcmd@def#1\@gobble[{#2}][{#3}][{#4}]{#5}{}}
%    \end{macrocode}
%\end{macro}
%
% \begin{macro}{\renew@keycommand}
%    \begin{macrocode}
\def\renew@keycommand#1{\begingroup
   \escapechar\m@ne\edef\@gtempa{{\string#1}}%
   \expandafter\@ifundefined\@gtempa
      {\endgroup\@latex@error{\noexpand#1undefined}\@ehc}
      \endgroup
   \let\@ifdefinable\@rc@ifdefinable
   \new@keycommand#1%
}
%    \end{macrocode}
% \end{macro}
%
% \begin{macro}{\provide@keycommand}
%    \begin{macrocode}
\def\provide@keycommand#1{\begingroup
   \escapechar\m@ne\edef\@gtempa{{\string#1}}%
   \expandafter\@ifundefined\@gtempa
      {\endgroup\new@keycommand#1}
      {\endgroup\def\kcmd@donot@provide{\renew@keycommand\kcmd@donot@provide
         }\kcmd@donot@provide}%
}
\let\kcmd@donot@provide\@empty% it must not be undefined
%    \end{macrocode}
% \end{macro}
%
% \subsection{new key-environments}
%
% \begin{macro}{\newkeyenvironment}
%    \begin{macrocode}
\newrobustcmd*\newkeyenvironment{\begingroup
   \let\kcmd@gbl\@empty\kcmd@star@or@long\new@keyenvironment}
\newrobustcmd\renewkeyenvironment{\begingroup
   \let\kcmd@gbl\@empty\kcmd@star@or@long\renew@keyenvironment}
%    \end{macrocode}
% \end{macro}
%
% \begin{macro}{\new@keyenvironment}
%    \begin{macrocode}
\def\new@keyenvironment#1{\@testopt{\@newkeyenva{#1}}{}}
\long\def\@newkeyenva#1[#2]{%
   \kcmd@ifstrdigit{#2}%
      {\@newkeyenv{#1}{[][][{#2}]}}
      {\@testopt{\@newkeyenvb{#1}[{#2}]}{}}}
\long\def\@newkeyenvb#1[#2][#3]{%
   \kcmd@ifstrdigit{#3}%
      {\@newkeyenv{#1}{[{#2}][][{#3}]}}
      {\@testopt{\@newkeyenvc{#1}{[{#2}][{#3}]}}0}}
\long\def\@newkeyenvc#1#2[#3]{\@newkeyenv{#1}{#2[{#3}]}}
\long\def\@newkeyenv#1#2{%
   \kcmd@ifplus \kcmd@unexpandchar@activate \fi
   \kcmd@keyenvir@def{#1}{#2}%
}
\long\def\kcmd@keyenvir@def#1#2#3#4{%
   \expandafter\let\csname end#1\endcsname\relax
   \expandafter\kcmd@def\csname #1\expandafter\endcsname\csname end#1\endcsname#2{#3}{#4}%
}
%    \end{macrocode}
% \end{macro}
%
% \begin{macro}{\renew@keyenvironment}
%    \begin{macrocode}
\def\renew@keyenvironment#1{%
  \@ifundefined{#1}%
     {\@latex@error{Environment #1 undefined}\@ehc
     }\relax
  \cslet{#1}\relax
  \new@keyenvironment{#1}}
%    \end{macrocode}
% \end{macro}
% \iffalse
%<package>
%<package>
% \fi
%
% \subsection{Tests on keys}
%
% \begin{macro}{\ifcommandkey}\qquad
% \{\meta{key-name}\}\{\meta{true}\}\{\meta{false}\}\quad expands \meta{true} only if the value of the key
% is not blank:
%    \begin{macrocode}
\newcommand*\ifcommandkey[1]{\csname @\expandafter\expandafter\expandafter
   \kcmd@nbk\commandkey{#1}//{first}{second}//%
   oftwo\endcsname}
%    \end{macrocode}
% \end{macro}
%
%
% \begin{macro}{\showcommandkeys}\qquad\qquad are helper macros essentially for debuging purpose...
%    \begin{macrocode}
\newrobustcmd*\showcommandkeys[1]{\let\do\showcommandkey\docsvlist{#1}}
\newrobustcmd*\showcommandkey[1]{key \string"#1\string" = %
   \detokenize\expandafter\expandafter\expandafter{\commandkey{#1}}\par}
%    \end{macrocode}
% \end{macro}
% 
%
%    \begin{macrocode}
%</package>
%    \end{macrocode}
%
% \section{Examples}
% \label{sec:examples}
%
%    \begin{macrocode}
%<*example>
\ProvidesFile{keycommand-example}
\documentclass[a4paper]{article}
\usepackage[T1]{fontenc}
\usepackage[latin1]{inputenc}
\usepackage[american]{babel}
\usepackage{keycommand,framed,fancyvrb}
%
\makeatletter
\parindent\z@
\newkeycommand*\Rule[raise=.4ex,width=1em,thick=.4pt][1]{%
   \rule[\commandkey{raise}]{\commandkey{width}}{\commandkey{thick}}%
   #1%
   \rule[\commandkey{raise}]{\commandkey{width}}{\commandkey{thick}}}

\newkeycommand*\charleads[sep=1][2]{%
   \ifhmode\else\leavevmode\fi\setbox\@tempboxa\hbox{#2}\@tempdima=1.584\wd\@tempboxa%
   \cleaders\hb@xt@\commandkey{sep}\@tempdima{\hss\box\@tempboxa\hss}#1%
   \setbox\@tempboxa\box\voidb@x}
\newcommand*\charfill[1][]{\charleads[{#1}]{\hfill\kern\z@}}
\newcommand*\charfil[1][]{\charleads[{#1}]{\hfil\kern\z@}}
%
\newkeyenvironment*{dblruled}[first=.4pt,second=.4pt,sep=1pt,left=\z@]{%
   \def\FrameCommand{%
      \vrule\@width\commandkey{first}%
      \hskip\commandkey{sep}
      \vrule\@width\commandkey{second}%
      \hspace{\commandkey{left}}}%
   \parindent\z@
   \MakeFramed {\advance\hsize-\width \FrameRestore}}
   {\endMakeFramed}
%
\makeatother
\begin{document}
\title{This is {\tt keycommand-example.tex}}
\author{Florent Chervet}
\date{July 22, 2009}

\maketitle

{\Large Please refer to {\tt keycommand-example.tex} for definitions.}

\section{Example of a keycommand : \texttt{\string\Rule}}

\begin{tabular*}\textwidth{rl}
\verb+\Rule[width=2em]{hello}+:&\Rule[width=2em]{hello}\cr
\verb+\Rule[thick=1pt,width=2em]{hello}+:&\Rule[thick=1pt,width=2em]{hello}\cr
\verb+\Rule{hello}+:&\Rule{hello}\cr
\verb+\Rule[thick=1pt,raise=1ex]{hello}+:&\Rule[thick=1pt,raise=1ex]{hello}
\end{tabular*}

\section{Example of a keycommand : \texttt{\string\charfill}}

\begin{tabular*}\textwidth{rp{.4\textwidth}}
\verb+\charfill{$\star$}+: & \charfill{$\star$}\cr
\verb+\charfill[sep=2]{$\star$}+: & \charfill[sep=2]{$\star$} \\
\verb+\charfill[sep=.7]{\textasteriskcentered}+: & \charfill[sep=.7]{\textasteriskcentered}
\end{tabular*}


\section{Example of a keyenvironment : \texttt{dblruled}}

Key environment \texttt{dblruled } uses \texttt{framed.sty} and therefore it can be used 
even if a pagebreak occurs inside the environment:
\medskip

\verb+\begin{dblruled}+\par
\verb+   test for dblruled key-environment\par+\par
\verb+   test for dblruled key-environment\par+\par
\verb+   test for dblruled key-environment+\par
\verb+\end{dblruled}+

\begin{dblruled}
 test for dblruled key-environment\par
 test for dblruled key-environment\par
 test for dblruled key-environment
\end{dblruled}


\verb+\begin{dblruled}[first=4pt,sep=2pt,second=.6pt,left=.2em]+\par
\verb+   test for dblruled key-environment\par+\par
\verb+   test for dblruled key-environment\par+\par
\verb+   test for dblruled key-environment+\par
\verb+\end{dblruled}+

\begin{dblruled}[first=4pt,sep=2pt,second=.6pt,left=.2em]
 test for dblruled key-environment\par
 test for dblruled key-environment\par
 test for dblruled key-environment
\end{dblruled}

\end{document}
%</example>
%    \end{macrocode}
% \DeleteShortVerb{\+}^^A\UndefineShortVerb{\+}
% \begin{History}
% 
%   \begin{Version}{2010/04/27 v3.1415}
%   \item Key-environment can now be nested ! (it's not too late... I hope so)
%   \item Keys and mandatory arguments as well can be used in both \texttt{begin} end \texttt{end} part of the environment.
%   \end{Version}
% 
%   \begin{Version}{2010/04/25 v3.141}
%   \item No new feature but a real improvement in optimization. \\
%         In particular, \thispackage does not load \xpackage{etextools} anymore. \\
%   \item Bug fix for \cs{providekeycommand}.
%         
%   \end{Version}
%
%   \begin{Version}{2010/04/18 v3.14}
%   \item Correction of bug in the normalization process. \\
%         Correction of a bug in \cs{ifcommandkey} (undesirable space...)
%   \item Modification of the pdf documentation for the \stform+ form of key-environments.
%   \end{Version}
%
%   \begin{Version}{2010/03/28 v3.0}
%   \item Complete redesign of the implementation. \\
%   \xpackage{keycommand} is now based on some macros of \xpackage{etoolbox}.
%
%   \item Adding the + prefix and the ability to capture keys that where not defined.
%
%   \end{Version}
%
%   \begin{Version}{2009/07/22 v1.0}
%   \item
%     First version.
%   \end{Version}
%
% \end{History}
%
% \begin{thebibliography}{9}
%
% \bibitem{xkeyval}
%   Hendri Adriaens:
%   \textit{The \xpackage{xkeyval} package};
%   2008/08/13 v2.6a;
%   \CTAN{macros/latex/contrib/xkeyval.dtx}
%
% \bibitem{kvsetkeys}
%   Heiko Oberdiek:
%   \textit{The \xpackage{kvsetkeys} package};
%   2007/09/29 v1.3;
%   \CTAN{macros/latex/contrib/oberdiek/kvsetkeys.dtx}.
%
% \bibitem{keyval}
%   David Carlisle:
%   \textit{The \xpackage{keyval} package};
%   1999/03/16 v1.13;
%   \CTAN{macros/latex/required/graphics/keyval.dtx}.
%
% \end{thebibliography}
%
% \PrintIndex
%
% \label{LastPage}
% \Finale
%        (quote the arguments according to the demands of your shell)
%
% Documentation:
%           (pdf)latex keycommand.dtx
% Copyright (C) 2009-2010 by Florent Chervet <florent.chervet@free.fr>
%<*ignore>
\begingroup
  \def\x{LaTeX2e}%
\expandafter\endgroup
\ifcase 0\ifx\install y1\fi\expandafter
         \ifx\csname processbatchFile\endcsname\relax\else1\fi
         \ifx\fmtname\x\else 1\fi\relax
\else\csname fi\endcsname
%</ignore>
%<*install>
\input docstrip.tex
\Msg{************************************************************************}
\Msg{* Installation}
\Msg{* Package: keycommand 2010/04/27 v3.1415 key-value interface for commands and environments in LaTeX}
\Msg{************************************************************************}

\keepsilent
\askforoverwritefalse

\let\MetaPrefix\relax
\preamble

This is a generated file.

keycommand : key-value interface for commands and environments in LaTeX [v3.1415 2010/04/27]

This work may be distributed and/or modified under the
conditions of the LaTeX Project Public License, either
version 1.3 of this license or (at your option) any later
version. The latest version of this license is in
   http://www.latex-project.org/lppl.txt

This work consists of the main source file keycommand.dtx
and the derived files
   keycommand.sty, keycommand.pdf, keycommand.ins,
   keycommand-example.tex

keycommand : an easy way to define commands with optional keys
Copyright (C) 2009-2010 by Florent Chervet <florent.chervet@free.fr>

\endpreamble
\let\MetaPrefix\DoubleperCent

\generate{%
   \file{keycommand.ins}{\from{keycommand.dtx}{install}}%
   \file{keycommand.sty}{\from{keycommand.dtx}{package}}%
   \file{keycommand-example.tex}{\from{keycommand.dtx}{example}}%
}

\generate{%
   \file{keycommand.drv}{\from{keycommand.dtx}{driver}}%
}

\obeyspaces
\Msg{************************************************************************}
\Msg{*}
\Msg{* To finish the installation you have to move the following}
\Msg{* file into a directory searched by TeX:}
\Msg{*}
\Msg{*     keycommand.sty}
\Msg{*}
\Msg{* To produce the documentation run the file `keycommand.dtx'}
\Msg{* through LaTeX.}
\Msg{*}
\Msg{* Happy TeXing!}
\Msg{*}
\Msg{************************************************************************}

\endbatchfile
%</install>
%<*ignore>
\fi
%</ignore>
%<*driver>
\edef\thisfile{\jobname}
\def\thisinfo{key-value interface for commands and environments in \LaTeX.}
\def\thisdate{2010/04/27}
\def\thisversion{3.1415}
\let\loadclass\LoadClass
\def\LoadClass#1{\loadclass[abstracton]{scrartcl}\let\scrmaketitle\maketitle\AtEndOfClass{\let\maketitle\scrmaketitle}}
\documentclass[a4paper,oneside]{ltxdoc}
\usepackage[latin9]{inputenc}
\usepackage[american]{babel}
\usepackage[T1]{fontenc}
\usepackage{etex,etoolbox,holtxdoc,geometry,tocloft,graphicx,xspace,fancyhdr,color,bbding,embedfile,framed,multirow,txfonts,makecell,enumitem,arydshln}
\CodelineNumbered
\usepackage{keyval}\makeatletter\let\keyval@setkeys\setkeys\makeatother
\usepackage{xkeyval}\let\xsetkeys\setkeys
\usepackage{kvsetkeys}
\usepackage{fancyvrb}
\lastlinefit999
\geometry{top=2cm,headheight=1cm,headsep=.3cm,bottom=1.4cm,footskip=.5cm,left=2.5cm,right=1cm}
\hypersetup{%
  pdftitle={The keycommand package},
  pdfsubject={key-value interface for commands and environments in LaTeX.},
  pdfauthor={Florent CHERVET},
  colorlinks,linkcolor=reflink,
  pdfstartview={FitH},
  pdfkeywords={tex, e-tex, latex, package, keys, keycommand, newcommand, keyval, kvsetkeys, programming},
  bookmarksopen=true,bookmarksopenlevel=3}
\embedfile{\thisfile.dtx}
\begin{document}
   \DocInput{\thisfile.dtx}
\end{document}
%</driver>
% \fi
%
% \CheckSum{1111}
%
% \CharacterTable
%  {Upper-case    \A\B\C\D\E\F\G\H\I\J\K\L\M\N\O\P\Q\R\S\T\U\V\W\X\Y\Z
%   Lower-case    \a\b\c\d\e\f\g\h\i\j\k\l\m\n\o\p\q\r\s\t\u\v\w\x\y\z
%   Digits        \0\1\2\3\4\5\6\7\8\9
%   Exclamation   \!     Double quote  \"     Hash (number) \#
%   Dollar        \$     Percent       \%     Ampersand     \&
%   Acute accent  \'     Left paren    \(     Right paren   \)
%   Asterisk      \*     Plus          \+     Comma         \,
%   Minus         \-     Point         \.     Solidus       \/
%   Colon         \:     Semicolon     \;     Less than     \<
%   Equals        \=     Greater than  \>     Question mark \?
%   Commercial at \@     Left bracket  \[     Backslash     \\
%   Right bracket \]     Circumflex    \^     Underscore    \_
%   Grave accent  \`     Left brace    \{     Vertical bar  \|
%   Right brace   \}     Tilde         \~}
%
% \DoNotIndex{\begin,\CodelineIndex,\CodelineNumbered,\def,\DisableCrossrefs,\~,\@ifpackagelater}
% \DoNotIndex{\DocInput,\documentclass,\EnableCrossrefs,\end,\GetFileInfo}
% \DoNotIndex{\NeedsTeXFormat,\OnlyDescription,\RecordChanges,\usepackage}
% \DoNotIndex{\ProvidesClass,\ProvidesPackage,\ProvidesFile,\RequirePackage}
% \DoNotIndex{\filename,\fileversion,\filedate,\let}
% \DoNotIndex{\@listctr,\@nameuse,\csname,\else,\endcsname,\expandafter}
% \DoNotIndex{\gdef,\global,\if,\item,\newcommand,\nobibliography}
% \DoNotIndex{\par,\providecommand,\relax,\renewcommand,\renewenvironment}
% \DoNotIndex{\stepcounter,\usecounter,\nocite,\fi}
% \DoNotIndex{\@fileswfalse,\@gobble,\@ifstar,\@unexpandable@protect}
% \DoNotIndex{\AtBeginDocument,\AtEndDocument,\begingroup,\endgroup}
% \DoNotIndex{\frenchspacing,\MessageBreak,\newif,\PackageWarningNoLine}
% \DoNotIndex{\protect,\string,\xdef,\ifx,\texttt,\@biblabel,\bibitem}
% \DoNotIndex{\z@,\wd,\wheremsg,\vrule,\voidb@x,\verb,\bibitem}
% \DoNotIndex{\FrameCommand,\MakeFramed,\FrameRestore,\hskip,\hfil,\hfill,\hsize,\hspace,\hss,\hbox,\hb@xt@,\endMakeFramed,\escapechar}
% \DoNotIndex{\do,\date,\if@tempswa,\@tempdima,\@tempboxa,\@tempswatrue,\@tempswafalse,\ifdefined,\ifhmode,\ifmmode,\cr}
% \DoNotIndex{\box,\author,\advance,\multiply,\Command,\outer,\next,\leavevmode,\kern,\title,\toks@,\trcg@where,\tt}
% \DoNotIndex{\the,\width,\star,\space,\section,\subsection,\textasteriskcentered,\textwidth}
% \DoNotIndex{\",\:,\@empty,\@for,\@gtempa,\@latex@error,\@namedef,\@nameuse,\@tempa,\@testopt,\@width,\\,\m@ne,\makeatletter,\makeatother}
% \DoNotIndex{\maketitle,\parindent,\setbox,\x,\kernel@ifnextchar}
% \DoNotIndex{\KVS@CommaComma,\KVS@CommaSpace,\KVS@EqualsSpace,\KVS@Equals,\KVS@Global,\KVS@SpaceEquals,\KVS@SpaceComma,\KVS@Comma}
% \DoNotIndex{\DefineShortVerb,\DeleteShortVerb,\UndefineShortVerb,\MakeShortVerb,\endinput}
% \let\ClearPage\clearpage
% \makeatletter
% \MakeShortVerb{\+}\DeleteShortVerb{\|}\DefineShortVerb{\|}
% \catcode`\� \active   \def�{\@ifnextchar �{\par\nobreak\vskip-2\parskip}{\par\nobreak\vskip-\parskip}}
% \def\thispackage{\xpackage{\thisfile}\xspace}
% \def\ThisPackage{\Xpackage{\thisfile}\xspace}
% \def\Xpackage{\@dblarg\X@package}
% \def\X@package[#1]#2{%
%     \xpackage{#2\footnote{\noindent\xpackage{#2}: \href{http://www.ctan.org/tex-archive/macros/latex/contrib/#1}{\nolinkurl{CTAN:macros/latex/contrib/#1}}}}}
% \def\Underbrace#1_#2{$\underbrace{\vtop to2ex{}\hbox{#1}}_{\footnotesize\hbox{#2}}$}
%
% \parindent\z@\parskip.4\baselineskip\topsep\parskip\partopsep\z@
% \g@addto@macro\macro@font{\macrocodecolor\let\AltMacroFont\macro@font}
% \g@addto@macro\@list@extra{\parsep\parskip\topsep\z@\itemsep\z@}
% \def\smex{\leavevmode\hb@xt@2em{\hfil$\longrightarrow$\hfil}}
% \newrobustcmd\verbfont{\usefont{T1}{\ttdefault}{\f@series}{n}}    \let\vb\verbfont
% \renewrobustcmd\#[1]{{\usefont{T1}{pcr}{bx}{n}\char`\##1}}
% \newrobustcmd\csred[1]{\textcolor{red}{\cs{#1}}}
% \renewrobustcmd\cs[2][]{\mbox{\vb#1\expandafter\@gobble\string\\#2}}
% \newrobustcmd\CSbf[1]{\textbf{\CS{#1}}}
% \newrobustcmd\csbf[2][]{\textbf{\cs[{#1}]{#2}}}
% \newrobustcmd\textttbf[1]{\textbf{\texttt{#1}}}
% \renewrobustcmd*\bf{\bfseries}\newcommand\nnn{\normalfont\mdseries\upshape}\newcommand\nbf{\normalfont\bfseries\upshape}
% \newrobustcmd*\blue{\color{blue}}\newcommand*\red{\color{dr}}\newcommand*\green{\color{green}}\newcommand\rred{\color{red}}
% \newrobustcmd\rrbf{\color{red}\bfseries}
% \definecolor{copper}{rgb}{0.67,0.33,0.00}  \newcommand\copper{\color{copper}}
% \definecolor{dg}{rgb}{0.16,0.33,0.00}      \newcommand\dg{\color{dg}}
% \definecolor{db}{rgb}{0,0,0.502}           \newcommand\db{\color{db}}
% \definecolor{dr}{rgb}{0.49,0.00,0.00}      \let\dr\red
% \newrobustcmd\bk{\color{black}}\newcommand\md{\mdseries}
%
% \fancyhf{}\fancyhead[L]{The \thispackage package -- \thisinfo}
% \fancyfoot[L]{\color[gray]{.35}\scriptsize\thispackage\quad[rev.\thisversion]\quad\copyright\oldstylenums{2009-2010}\,\lower.3ex\hbox{\NibRight}\,Florent Chervet}
% \fancyfoot[R]{\oldstylenums{\thepage} / \oldstylenums{\pageref{LastPage}}}
% \pagestyle{fancy}
% \fancypagestyle{plain}{%
%     \let\headrulewidth\z@
%     \fancyhf{}%
%     \fancyfoot[R]{\oldstylenums{\thepage} / \oldstylenums{\pageref{LastPage}}}}
%
% \newcommand\macrocodecolor{\color{macrocode}}\definecolor{macrocode}{rgb}{0.18,0.00,0.45}
% \newcommand\reflinkcolor{\color{reflink}}\definecolor{reflink}{rgb}{0.49,0.00,0.00}
% \font\umrandA=umranda at 20pt
% \def\@serp{\leavevmode\lower20pt\hbox{\umrandA\char'131}}
% \def\serp#1{\@serp\hfil #1\hfil\reflectbox{\@serp}}
% \newrobustcmd\stform{\@ifnextchar*{\@stform[]\textasteriskcentered\@gobble}\@stform}
% \newrobustcmd\@stform[2][\string]{\textttbf{\rred#1#2}\xspace}
%
% \makeatother
%
% \deffootnote{1em}{0pt}{\rlap{\textsuperscript{\thefootnotemark}}\kern1em}
%
% \title{\vskip-18pt\mdseries {\bfseries\ThisPackage}\kern.6em package}
% \author{\footnotesize\xemail{florent.chervet@free.fr}}
% \date{\thisdate~--~version \thisversion}
% \subtitle{\thisinfo}
% ^^A\subject{\vskip-2cm\serp{The completely redesigned}}
% \subject{\vskip-2cm\relax The \textit{free} and \textit{open source}}
%
% \maketitle
% 
% \makeatletter\begingroup\let\@thefnmark\@empty\let\@makefntext\@firstofone
% \footnotetext{\noindent
% This documentation is produced with the +DocStrip+ utility.
% \begin{tabbing}
% \qquad\=\smex\=To get the documentation, \= run (thrice):\quad\= \texttt{pdflatex keycommand.dtx} \\
% \qquad\>\>To get the index, \> run:\>\texttt{makeindex -s gind.ist keycommand.idx} \\
% \>\smex\>To get the package, \> run:\>        \texttt{etex keycommand.dtx}
% \end{tabbing}�
% The \xext{dtx} file is embedded into this pdf file thank to \xpackage{embedfile} by H. Oberdiek.}
% \endgroup\makeatother
% 
% \hypersetup{bookmarksopenlevel=3}
% \deffootnote{1em}{0pt}{\rlap{\thefootnotemark.}\kern1em}
% \vspace*{-18pt}
% \begin{abstract}\parindent0pt\noindent\leftskip1cm\rightskip\leftskip\lastlinefit0%
%
% \thispackage provides an easy way to define commands or environments
% with optional keys.
% \smallskip
%
% \csbf{newkeycommand} \cs{renewkeycommand} \cs{providekeycommand} and \csbf{newkeyenvironment},\linebreak 
% \cs{renewkeyenvironment} are macros to define such commands and environments with keys.
% 
% \thispackage is designed to make easier interface for user-defined commands.  In particular,
% \csbf{newkeycommand}\stform+ permits the use of key-commands in every context. 
% \medskip
%
% Keys are defined with the command itself in a very natural way.
% You can restrict the possible values for the keys by declaring them with a \textbf{type}.
% Available types for keys are : \textit{boolean}, \textit{enum} and \textit{choice} (see \ref{subsec:GeneralSyntax}).
%
% \smallskip
%
% The \thispackage package requires and is based on the package \xpackage{xkeyval} by Hendri Adriaens, and uses
% the \cs{kv@normalize} macro of \xpackage{kvsetkeys} (Heiko Oberdiek) for robustness, as shown
% in \ref{kvsetkeys-comparisons}).
%
% It works with an \eTeX{} distribution of \LaTeX.
% \end{abstract}
%
% \DeleteShortVerb{\+}\enlargethispage{2\baselineskip}
% \cftbeforesecskip=4pt plus2pt minus2pt
% \cftbeforesubsecskip=0pt plus2pt minus2pt
% \renewcommand\contentsname{Contents\quad\leaders\vrule height3.4pt depth-3pt\hfill\null\kern0pt\vskip-6pt}
% ^^A\vskip-.8\baselineskip
% \tableofcontents
%
% \clearpage\MakeShortVerb{\+}
%
% \def\B#1{\texttt{[}\meta{#1}\texttt{]}}
%
% \section{User Interface}
%
% \subsection{General syntax}\label{subsec:GeneralSyntax}
%
% \begin{declcs}{newkeycommand}%
%  \Underbrace{\textcolor{red}{\textasteriskcentered\string+[short-unexpand]}}_{\makecell[c]{modifiers \\ Optional}}\,%
%  \Underbrace{\M{command}}_{Required}\,%
%   {\color{db}\Underbrace{\B{keys=defaults}\,\B{OptKey}\,\B{<n>}}_{Optional}\,}%
%   \Underbrace{\M{definition}}_{Required}
% \end{declcs}
%
% \cs{newkeycommand} will define \cs{command} as a new key-command!\quad well...
%
% Use the \stform* form when you do not want it to be a \cs{long} macro (as for \LaTeX{}-\cs{newcommand}).
%
% The +[keys=defaults]+ argument define the keys with their default values. It is optional, but a key-command
% without keys seems to be useless (at least for me...). Keys may be defined as :
%
% \newlist{myenum}{enumerate}{1}
% \setlist[myenum]{label={},topsep=-\parskip,itemsep=-\parskip,parsep=\parskip,after=\vskip-\baselineskip}
%
% \renewcommand\theadfont{\tt\bfseries}
% \noindent\begin{tabular}{|c|>{\db}c|m{8cm}|}\hline
% \thead{Type} & \thead{exemple}                                              & \thead{value of \cs{commandkey}}                                                                                         \\ \hline
% general      & color{\dg=red}                                               & \cs{commandkey}\{{\db color}\} is `{\dg red}' and may be anything (text, number, macro...)                                           \\ \hline
% boolean      & {\rred bool} bold{\dg =true}                                 & \cs{commandkey}\{{\db bold}\} is:
%                                                                                                       \begin{myenum}
%                                                                                                       \item {\tt 0}\quad (for {\dg false})
%                                                                                                       \item {\tt 1}\quad (for {\dg true})
%                                                                                                       \end{myenum}          \\ \hline
% \multirow{2}*{enumerate} & {\rred enum} position{\dg=\{left,centered,right\}} & \cs{commandkey}\{{\db position}\} is:
%                                                                                                     \begin{myenum}
%                                                                                                     \item `{\dg left}'\quad by default and can be
%                                                                                                     \item `{\dg centered}' or
%                                                                                                     \item `{\dg right}'
%                                                                                                        \end{myenum}         \\ \cdashline{2-3}[1pt/2pt]
%                          & {\rred enum\textasteriskcentered} position{\dg=\{left,centered,right\}}  & This is the same, except match is case \textbf{in}sensitive   \bottopstrut                                   \\ \hline
% \multirow{2}*{choice}    & {\rred choice} position={\dg \{left,centered,right\}} & \cs{commandkey}\{{\db position}\} is:
%                                                                                                     \begin{myenum}
%                                                                                                     \item {\tt 0}\quad (for {\dg left} the default value),
%                                                                                                     \item {\tt 1}\quad (for {\dg centered})
%                                                                                                     \item {\tt 2}\quad (for {\dg right})
%                                                                                                     \end{myenum}                           \\ \cdashline{2-3}[1pt/2pt]
%                          & {\rred choice\textasteriskcentered} position={\dg\{left,centered,right\}} & This is the same, except match is case \textbf{in}sensitive   \bottopstrut                                   \\ \hline
% \end{tabular}
%
% The {\db+OptKey+} argument is used if you wish to capture the +key=value+ pairs that are not specifically defined (more on this in the examples section \ref{sec:examples}).
%
% The key-command may have {\tt 0} up to {\tt 9} \textbf{mandatory} arguments : specify the number by +<n>+ ({\tt 0} if omitted).
%
% The \stform+ form expands the \cs{commandkey} before executing the key-command itself, as explain in section \ref{sec:example:plus}.
%
% \subsection{First example :}
%
% \begin{tabbing}\label{textrule}
% \,\=\csbf{new}\=\textttbf{keycommand}\cs[\copper]{textrule}+[+{\color{db}+raise=.4ex,width=3em,thick=.4pt+}+][1]{%+ \\ ^^A+][1]{%+}\\
% \>\>\cs{rule}+[+\cs[\red]{commandkey}+{+{\db+raise+}+}]{+\cs[\red]{commandkey}+{+{\db+width+}+}{+\cs[\red]{commandkey}+{+{\db+thick+}+}}+\\
% \>\>\#1 \\
% \>\>\cs{rule}+[+\cs[\red]{commandkey}+{+{\db+raise+}+}]{+\cs[\red]{commandkey}+{+{\db+width+}+}}{+\cs[\red]{commandkey}+{+{\db+thick+}+}}}+
% \end{tabbing}
%
% defines the keys {\db+width+}, {\db+thick+} and {\db+raise+} with their default values (if not specified):
% {\db+3em+}, {\db+.4pt+} and {\db+.4ex+}. Now \cs[\copper]{textrule} can be used as follow:
% \begin{tabbing}
% \=1:\quad\=\cs[\copper]{textrule}+[width=2em]{hello}+\hskip2.5cm\=\smex\qquad\=        \rule[.4ex]{2em}{.4pt}hello\rule[.4ex]{2em}{.4pt} \\
% \>2:\>\cs[\copper]{textrule}+[thick=5pt,width=2em]{hello}+\>\smex\>                  \rule[.4ex]{2em}{5pt}hello\rule[.4ex]{2em}{5pt}\\
% \>3:\>\cs[\copper]{textrule}+{hello}+\quad \>\smex\>                                 \rule[.4ex]{3em}{.4pt}hello\rule[.4ex]{3em}{.4pt}\\
% \>4:\>\cs[\copper]{textrule}+[thick=2pt,raise=1ex]{hello}+\>\smex\>                  \rule[1ex]{3em}{2pt}hello\rule[1ex]{3em}{2pt} \\
% \> \textit{et c\ae tera}.
% \end{tabbing}
%
% \clearpage
%
% \subsection[Second example : the \string+ form]{Second example : the {\rred\bf\string+} form}
% \label{sec:example:plus}
%
% \DeleteShortVerb{\+}
% \begin{Verbatim}[gobble=1,commandchars=$(),frame=lines]
% ($bf\newkeycommand)($rred$bf+[\|])($copper\myfigure)[image,
%                              caption,
%                              enum placement={H,h,b,t,p},
%                              width=\textwidth,
%                              label=
%                             ][($db OtherKeys)]{%
%        ($rred|)($bf\begin){figure}($dr|)[($red\commandkey){placement}]
%           ($rred|)($bf\includegraphics)($dr|)[width=($red\commandkey){width},($red\commandkey){($db OtherKeys)}]{%
%                             ($red\commandkey){image}}%
%           ($dg\ifcommandkey){caption}{($rred|)\caption($rred|){($red\commandkey){caption}}}{}%
%           ($dg\ifcommandkey){label}{($rred|)\label($rred|){($red\commandkey){label}}}{}%
%        ($rred|)($bf\end){figure}($rred|)}
% \end{Verbatim}
% \MakeShortVerb{\+}
%
% With the \stform+ form of \cs{newkeycommand}, the definition will be expanded (at run time). The optional {\rred\bf+[\|]+} argument
% means that everything inside {\bf\rred+|+ ... +|+} is protected from expansion.
%
% {\dg\cs{ifcommandkey}}\{\meta{name}\}\{\meta{true}\}\{\meta{false}\}\quad expands \meta{true} if the commandkey \meta{name} is not blank.
%
% {\db \meta{Otherkeys}} captures the keys given by the user but not declared: they are simply given back to \cs{includegraphics} here...
%
%
% \subsection[Explanation of the \string+ form]{Explanation of the {\rred\bf\string+} form}
% \DeleteShortVerb{\+}
% The |\commankey{|\meta{name}|}| stuff is expanded at run time using the following scheme:��
% \begin{Verbatim}[gobble=1,commandchars=!(),frame=lines]
%     (!bf\newkeycommand)(!copper\keyMacro)[A=\defA,B=\defB,C=\defC,D=\defD][1]{(!dg\begingroup)
%        (!dg\edef)\keyMacro##1{(!dg\endgroup)
%            (!dg\noexpand)\Macro{(!red\getcommandkey){A}}
%                           {(!red\getcommandkey){B}}
%                           {(!red\getcommandkey){C}}
%                           {(!red\getcommandkey){D}}
%     }\keyMacro{#1}}
% \end{Verbatim}
% Therefore, the arguments of \cs{Macro} are ready: there is no more \cs{commandkey} stuff, but instead the values of the keys
% as you gave them to the key-command. \cs{getcommandkey}\{A\} is expanded to \cs{defA}.
%
% But \cs{defA} is not expanded of course: in the \stform+ form, \cs{commandkey} has the meaning of \cs{getcommandkey}.
%
% As you can see, the mandatory arguments \#1, \#2 etc. are \textbf{never expanded}: there is no need to protect them inside the special (usually {\rred\bf\textbar}) character.
%
%
% \MakeShortVerb{\+}
% \clearpage
%
% \subsection{key-environments}
%
% \begin{declcs}{newkeyenvironment}%
%  \Underbrace{\textcolor{red}{\textasteriskcentered\string+[short-unexpand]}}_{\makecell[c]{modifiers \\ Optional}}\,%
%  \Underbrace{\M{envir name}}_{Required}\,%
%   {\db\Underbrace{\B{keys=defaults}\,\B{OptKey}\,\B{<n>}}_{Optional}\,}%
%   \Underbrace{\M{begin}}_{Required}\Underbrace{\M{end}}_{Required}
% \end{declcs}
%
% In the same way, you may define environments with optional keys as follow:�
% \begin{tabbing}
% \qquad\=+\newkeyenvironment+\=+{EnvirWithKeys}[kOne=+default value,...+][n]+\\
% \>\>+{+ commands at begin +EnvirWithKeys }+ \\
% \>\>+{+ commands at end +EnvirWithKeys }+
% \end{tabbing}
%
% Where $n$ is the number of mandatory other arguments (\emph{ie} without keys), if any.
%
% Key-environments may be defined with the \stform+ form in the same way as \cs{newkeycommand} is used.
% Be aware that each part of the environment: \meta{begin} and \meta{end} are expanded at run time then, 
% and the optional {\rred\bf+[\|]+} argument protects from expansion in each of those parts.
% 
% \subsection[Example of a \string+ key-environment]{Example of a {\rred\bf\string+} key-environment}
% 
% \DeleteShortVerb{\+}
% \begin{Verbatim}[gobble=1,commandchars=$(),frame=lines]
% ($bf\newkeyenvironment)($rred$bf+[\|])({$copper myfigure)}[
%                              caption,
%                              enum placement={H,h,b,t,p},
%                              width=.5\linewidth,
%                              label
%                             ][($db OtherKeys)][1]%
%     {% ($nbf$dg begin part)
%        ($rred|)($bf\begin){figure}($rred|)[($red\commandkey){placement}]
%           ($rred|)($bf\includegraphics)($rred|)[($red\commandkey){($db OtherKeys)},width=($red\commandkey){width}]{$#1}%
%     }
%     {% ($nbf$dg end part)
%           ($dg\ifcommandkey){caption}{($rred|)\caption($rred|){($red\commandkey){caption} image file = $#1}}{}%
%           ($dg\ifcommandkey){label}{($rred|)\label($rred|){($red\commandkey){label}}}{}%
%        ($rred|)($bf\end){figure}($rred|)%
%     }
% \end{Verbatim}
% \MakeShortVerb{\+}
%
% As you can see, \cs{commandkey} and mandatory arguments (\#1 here) are available both in the \meta{begin} 
% and in the \meta{end} parts of the key-environment.
% 
%
% \DefineShortVerb{\+}
%
% \section{Messages and more}
%
% \subsection{Invalid keys}
%
% If you use the command +\textule+ (defined in \ref{textrule}) with a key say: +height+
% that has not been declared at the definition of the key-command, you will get an
% error message like this:
% \begin{quote}\tt
% The key-value pairs ``height=...''�
% cannot be processed for key-command \string\textrule!�
% See the definition of the keycommand!
% \end{quote}
% The error is recoverable: the key is ignored.
%
% If you assign a value to an \textit{enum} or a \textit{choice} key, which is not allowed in the definition,
% you will get the following message:
% \begin{quote}\tt
% The value ``...'' is not allowed in key ...�
% for key-command \string\command�
% I'll use the default value ``...'' for this key instead�
% See the definition of the key-command!
% \end{quote}
% The error is recoverable: the key is assigned its default value.
%
% If you use a \cs{commandkey}\{\meta{name}\} in a key-command where \meta{name} is not defined as a key,
% you will get the \TeX{} generic error message :�
% \qquad undefined control sequence : \cs{keycmd->...@name}.
%
%
% \subsection{Testing keys}
%
% \begin{declcs}{ifcommandkey}\,\M{key name}\,\M{commands if key is NOT blank}\,\M{commands if key is blank}
% \end{declcs}
%
% When you define a key command you may let the default value of a key empty. Then, you may wish to
% expand some commands only if the key has been given by the user (with a non empty value). This can
% be achieved using the macro |\ifcommandkey|.
%
% \clearpage
% \subsection{xkeyval, keyval and kvsetkeys comparison}
%
% \begin{tabbing}
% \quad\=\xpackage{xkeyval}: \expandafter\meaning\csname ver@xkeyval.sty\endcsname \\
% \>\xpackage{keyval}: \expandafter\meaning\csname ver@keyval.sty\endcsname \\
% \>\xpackage{kvsetkeys}: \expandafter\meaning\csname ver@kvsetkeys.sty\endcsname
% \end{tabbing}
%
% \makeatletter\def\theadfont{\tt\bfseries}
% \define@key{fam}{key}{\def\result{#1}}
% \begin{table}[h]\label{kvsetkeys-comparisons}
% \begin{tabular}{|l|l|>{\color{db}}l|>{\color{dg}}l|}\hline
% \thead{\bf Example} & \thead{keyval} & \thead{xkeyval} & \thead{\makecell{kvsetkeys\\and\\keycommand}} \\ \hline
% +\setkeys{fam}{key={{value}}}+
%     & \keyval@setkeys{fam}{key={{value}}}\meaning\result
%     & \xsetkeys{fam}{key={{value}}}\meaning\result
%     & \kvsetkeys{fam}{key={{value}}}\meaning\result \\\hline
% +\setkeys{fam}{key={{{value}}}}+
%     & \keyval@setkeys{fam}{key={{{value}}}}\meaning\result
%     & \xsetkeys{fam}{key={{{value}}}}\meaning\result
%     & \kvsetkeys{fam}{key={{{value}}}}\meaning\result \\\hline
% +\setkeys{fam}{key=+\textvisiblespace+{{{value}}}}+
%     & \keyval@setkeys{fam}{key= {{{value}}}}\meaning\result
%     & \xsetkeys{fam}{key= {{{value}}}}\meaning\result
%     & \kvsetkeys{fam}{key= {{{value}}}}\meaning\result \\\hline
% \end{tabular}
% \caption{Then it is clear that, at this time, \xpackage{kvsetkeys} has the only correct behaviour...}
% \end{table}
%
% In \thispackage the key-value pairs are first normalized using \xpackage{kvsetkeys}-\cs{kv@normalize}. Then braces are added
% around the values in order to keep the good behaviour of \xpackage{kvsetkeys} while using \xpackage{xkeyval}.
% \makeatother
%
%
%
%
% \StopEventually{
% }
%
% \begin{center}\vskip6pt$\star$\hskip4em\lower12pt\hbox{$\star$}\hskip4em$\star$\vadjust{\vskip12pt}\end{center}
%
% \section{Implementation} \label{Implementation}
% \csdef{HDorg@PrintMacroName}#1{\hbox to4em{\strut \MacroFont \string #1\ \hss}}
%
% \subsection{Identification}
%
% This package is intended to use with \LaTeX{} so we don't check if it is loaded twice.
%
%    \begin{macrocode}
%<*package>
\NeedsTeXFormat{LaTeX2e}% LaTeX 2.09 can't be used (nor non-LaTeX)
   [2005/12/01]% LaTeX must be 2005/12/01 or younger (see kvsetkeys.dtx).
\ProvidesPackage{keycommand}
   [2010/04/27 v3.1415 - key-value interface for commands and environments in LaTeX]
%    \end{macrocode}
%
% \subsection{Requirements}
%
% The package is based on \xpackage{xkeyval}. However, \xpackage{xkeyval} is far less reliable
% than \xpackage{kvsetkeys} as far as spaces and bracket (groups) are concerned, as shown in the section
% \ref{kvsetkeys-comparisons} of this documentation.
%
% Therefore, we also use the macros of \xpackage{kvsetkeys} in order to \textit{normalize} the \texttt{key=value}
% list before setting the keys. This way, we take advantage of both \xpackage{xkeyval} and \xpackage{kvsetkeys} !
%
% As long as we use \eTeX{} primitives in \xpackage{keycommand} we also load the
% \xpackage{etex} package in order to get an error message if \eTeX{} is not running.
%
% The \xpackage{etoolbox} package gives some facility to write \xpackage{keycommand}.
% 
% From version \texttt{3.141} onwards, \thispackage does not load \xpackage{etextools} anymore.
%
%    \begin{macrocode}
\def\kcmd@pkg@name{keycommand}
\RequirePackage{etex,kvsetkeys,xkeyval,etoolbox}
%    \end{macrocode}
%
% Save the \cs{setkeys} macro of \xpackage{xkeyval} package (in case it was overwritten by a
% subsequent load of \xpackage{kvsetkeys} or \xpackage{keyval} for example :
%    \begin{macrocode}
\protected\def\kcmd@Xsetkeys{\XKV@sttrue\XKV@plfalse\XKV@testoptc\XKV@setkeys}% in case \setkeys 
%                                                                                was overwritten
%    \end{macrocode}
% Some \cs{catcode} assertions internally used by \thispackage:
%    \begin{macrocode}
\let\kcmd@AtEnd\@empty
\def\TMP@EnsureCode#1#2{%
  \edef\kcmd@AtEnd{%
    \kcmd@AtEnd
    \catcode#1 \the\catcode#1\relax
  }%
  \catcode#1 #2\relax
}
\TMP@EnsureCode{32}{10}% space
\TMP@EnsureCode{61}{12}% = sign
\TMP@EnsureCode{45}{12}% - sign
\TMP@EnsureCode{62}{12}% > sign
\TMP@EnsureCode{46}{12}% . dot
\TMP@EnsureCode{47}{8}% / slash (etextools)
\AtEndOfPackage{\kcmd@AtEnd\undef\kcmd@AtEnd}
%    \end{macrocode}
% 
% \begin{macro}{\kcmd@ifstrdigit}\qquad\qquad
% This macro is used too test the optional arguments of \cs{newkeycommand}, 
% in particular, one must know in an argument is a single digit (representing
% the number of mandatory arguments) or anything else (representing the \texttt{key=value} 
% list or the ``special'' \texttt{OptKey} key:
%    \begin{macrocode}
\iffalse%\ifdefined\pdfmatch% use \pdfmatch if present
   \long\def\kcmd@ifstrdigit#1{\csname @\ifnum\pdfmatch
      {\detokenize{^[[:space:]]*[[:digit:]][[:space:]]*$}}{\detokenize{#1}}=1 %
      first\else second\fi oftwo\endcsname}
\else% use filter, very efficient !
\def\kcmd@ifstrdigit#1{%
   \kcmd@nbk#1//%
      {\expandafter\expandafter\expandafter\kcmd@ifstrdigit@i
         \expandafter\expandafter\expandafter{\detokenize\expandafter{\number\number0#1}}}%
      {\@secondoftwo}//%
}
\def\kcmd@ifstrdigit@i#1{%
   \def\kcmd@ifstrdigit@ii##1#1##2##3\kcmd@ifstrdigit@ii{%
      \csname @\ifx##20first\else second\fi oftwo\endcsname
      }\kcmd@ifstrdigit@ii 00 01 02 03 04 05 06 07 08 09 0#1 \relax\kcmd@ifstrdigit@ii
}
\fi
%    \end{macrocode}
% \end{macro}
%
% \subsection{Defining (and undefining) command-keys}
%\begin{macro}{\kcmd@keyfam}\qquad
% The macro expands to the family-name, given the keycommand name:
%    \begin{macrocode}
\def\kcmd@keyfam#1{\detokenize{keycmd->}\expandafter\@gobble\string#1}
%    \end{macrocode}
% \end{macro}
% \begin{macro}{\kcmd@nbk}\qquad is the optimized \cs{ifnotblank} macro of \xpackage{etoolbox}
% (with \textttbf{/} having a catcode of 8):
%    \begin{macrocode}
\def\kcmd@nbk#1#2/#3#4#5//{#4}%
%    \end{macrocode}
% \end{macro}
%
% \begin{macro}{\kcmd@normalize@setkeys}~\par
% This macro assigns the values to the keys (expansion of \xpackage{xkeyval}-\cs{setkeys}
% on the result of \xpackage{kvsetkeys}-\cs{kv@normalize}). Braces are normalized too so that
% \verb+key=+\textvisiblespace+{{{value}}}+ is the same as \verb+key={{{value}}}+ as explained in section \ref{kvsetkeys-comparisons}:
%    \begin{macrocode}
\newrobustcmd\kcmd@normalize@setkeys[4]{%
% #1 = key-command,
% #2 = family,
% #3 = other-key,
% #4 = key-values pairs
   \kv@normalize{#4}\toks@{}%
   \expandafter\kv@parse@normalized\expandafter{\kv@list}{\kcmd@normalize@braces{#2}}%
   \edef\kv@list{\kcmd@Xsetkeys{\unexpanded{#2}}{\the\toks@}}\kv@list
   \kcmd@nbk#3//% undeclared keys are assigned to "OtherKeys"
      {\cslet{#2@#3}\XKV@rm}% (if specified, ie not empty)
      {\expandafter\kcmd@nbk\XKV@rm//% (otherwise a recoverable error is thown)
         {\PackageError\kcmd@pkg@name{The key-value pairs :\MessageBreak
         \XKV@rm\MessageBreak
         cannot be processed for key-command \string#1\MessageBreak
         See the definition of the key-command!}{}}{}//}//%
}
\long\def\kcmd@normalize@braces#1#2#3{% This is kvsetkeys processor for normalizing braces
   \toks@\expandafter{\the\toks@,#2}%
   \ifx @\detokenize{#3}@\else \toks@\expandafter{\the\toks@={{{#3}}}}\fi
}
%    \end{macrocode}
% \end{macro}
% 
% \begin{macro}{\kcmd@definekey}~\par
% \CS{kcmd@definekey} define the keys declared for the key-command.
% It is used as the \emph{processor} for the \cs{kv@parse} macro of \xpackage{kvsetkeys}.
% The macro appends the key names to the key list: ``\textit{family}.keylist''.
%
% keys are first checked for their type (bool, enum, enum*, choice or choice*) :
%
%    \begin{macrocode}
\def\kcmd@check@typeofkey#1{% expands to
% 0 if key has no type,
% 1 if boolean,
% 2 if enum*,
% 3 if enum,
% 4 if choice*,
% 5 if choice
   \kcmd@check@typeofkey@bool#1bool //%
      {\kcmd@check@typeofkey@enumst#1enum* //%
         {\kcmd@check@typeofkey@enum#1enum //%
            {\kcmd@check@typeofkey@choicest#1choice* //%
               {\kcmd@check@typeofkey@choice#1choice //%
                  05//}4//}3//}2//}1//}
\def\kcmd@check@typeofkey@bool #1bool #2//{\kcmd@nbk#1//}
\def\kcmd@get@keyname@bool #1bool #2//{#2}
\def\kcmd@check@typeofkey@enumst #1enum* #2//{\kcmd@nbk#1//}
\def\kcmd@get@keyname@enumst #1enum* #2//{#2}
\def\kcmd@check@typeofkey@enum #1enum #2//{\kcmd@nbk#1//}
\def\kcmd@get@keyname@enum #1enum #2//{#2}
\def\kcmd@check@typeofkey@choicest #1choice* #2//{\kcmd@nbk#1//}
\def\kcmd@get@keyname@choicest #1choice* #2//{#2}
\def\kcmd@check@typeofkey@choice #1choice #2//{\kcmd@nbk#1//}
\def\kcmd@get@keyname@choice #1choice #2//{#2}
%
\protected\long\def\kcmd@definekey#1#2#3#4#5{% define the keys using xkeyval macros
% #1 = keycommand,
% #2 = \global,
% #3 = family,
% #4 = key (before = sign),
% #5 = default (after = sign)
   \ifcase\kcmd@check@typeofkey{#4}\relax% standard
      #2\csedef{#3.keylist}{\csname#3.keylist\endcsname,#4}%
      \define@cmdkey{#3}[{#3@}]{#4}[{#5}]{}%
   \or% bool
      #2\csedef{#3.keylist}{\csname#3.keylist\endcsname,\kcmd@get@keyname@bool#4//}%
      \kcmd@define@boolkey#1{#3}{\kcmd@get@keyname@bool#4//}{#5}%
   \or% enum*
      #2\csedef{#3.keylist}{\csname#3.keylist\endcsname,\kcmd@get@keyname@enumst#4//}%
      \kcmd@define@choicekey#1*{#3}{\kcmd@get@keyname@enumst#4//}{#5}{\expandonce\val}%
   \or% enum
      #2\csedef{#3.keylist}{\csname#3.keylist\endcsname,\kcmd@get@keyname@enum#4//}%
      \kcmd@define@choicekey#1{}{#3}{\kcmd@get@keyname@enum#4//}{#5}{\expandonce\val}%
   \or% choice*
      #2\csedef{#3.keylist}{\csname#3.keylist\endcsname,\kcmd@get@keyname@choicest#4//}%
      \kcmd@define@choicekey#1*{#3}{\kcmd@get@keyname@choicest#4//}{#5}{\number\nr}%
   \or% choice
      #2\csedef{#3.keylist}{\csname#3.keylist\endcsname,\kcmd@get@keyname@choice#4//}%
      \kcmd@define@choicekey#1{}{#3}{\kcmd@get@keyname@choice#4//}{#5}{\number\nr}%
   \fi
   \ifx#2\global\relax
      #2\csletcs{KV@#3@#4}{KV@#3@#4}% globalize
      #2\csletcs{KV@#3@#4@default}{KV@#3@#4@default}% globalize default value
   \fi
}
%
\long\def\kcmd@firstchoiceof#1,#2\kcmd@nil{\unexpanded{#1}}
%
\long\def\kcmd@define@choicekey#1#2#3#4#5#6{%
   \begingroup\edef\kcmd@define@choicekey{\endgroup
      \noexpand\define@choicekey#2+{#3}{#4}
            [\noexpand\val\noexpand\nr]%
            {\unexpanded{#5}}% list of allowed values
            [{\kcmd@firstchoiceof#5,\kcmd@nil}]% default value
            {\csedef{#3@#4}{\unexpanded{#6}}}% define key value if in the allowed list
            {\kcmd@error@handler\noexpand#1{#3}{#4}{\kcmd@firstchoiceof#5,\kcmd@nil}}% error handler
   }\kcmd@define@choicekey
}
\def\kcmd@define@boolkey#1#2#3#4{\begingroup
   \kcmd@nbk#4//{\def\default{#4}}{\def\default{true}}//%
   \edef\kcmd@define@boolkey{\endgroup
      \noexpand\define@choicekey*+{#2}{#3}[\noexpand\val\noexpand\nr]%
            {false,true}
            [{\unexpanded\expandafter{\default}}]%
            {\csedef{#2@#3}{\noexpand\number\noexpand\nr}}%
            {\kcmd@error@handler\noexpand#1{#2}{#3}{\unexpanded\expandafter{\default}}}%
   }\kcmd@define@boolkey
}
%
\protected\long\def\kcmd@error@handler#1#2#3#4{%
% #1 = key-command,
% #2 = family,
% #3 = key,
% #4 = default
   \PackageError\kcmd@pkg@name{%
      Value `\val\space' is not allowed in key #3\MessageBreak
      for key-command \string#1.\MessageBreak
      I'll use the default value `#4' for this key.\MessageBreak
      See the definition of the key-command!}{%
      \csdef{#2@#3}{#4}}}
%    \end{macrocode}
% \end{macro}
%
% \begin{macro}{\kcmd@undefinekeys}~\par
% Now in case we redefine a key-command, we would like the old keys (\emph{ie} the keys
% associated to the old definition of the command) to be cleared, undefined.
% That's the job of \cs{kcmd@undefinekeys}.
%    \begin{macrocode}
\protected\def\kcmd@undefinekeys#1#2{% #1 = global, #2 = family
   \ifcsundef{#2.keylist}
      {\cslet{#2.keylist}\@gobble}
      {\expandafter\expandafter\expandafter\docsvlist
         \expandafter\expandafter\expandafter{%
                        \csname #2.keylist\endcsname}%
      \cslet{#2.keylist}\@gobble}%
}
\def\kcmd@undefinekey#1#2#3{% #1 = global, #2 = family, #3 = key
   #1\csundef{KV@#2@#3}%
   #1\csundef{KV@#2@#3@default}%
}
%    \end{macrocode}
% \end{macro}
% 
%\begin{macro}{\kcmd@setdefaults}\qquad\qquad
% sets the defaults values for the keys at the very beginning of the keycommand:
%    \begin{macrocode}
\def\kcmd@setdefaults#1{%
   \ifcsundef{#1.keylist}{}
   {\expandafter\expandafter\expandafter\docsvlist
      \expandafter\expandafter\expandafter{%
                           \csname#1.keylist\endcsname}}%
}
%    \end{macrocode}
%\end{macro}
% 
% 
%
% \begin{macro}{\kcmd@def}
% checks \cs{@ifdefinable} and cancels definition if needed:
%    \begin{macrocode}
\protected\long\def\kcmd@def#1#2[#3][#4][#5]#6#7{%
   \ifx#1\kcmd@donot@provide  \endgroup
   \else
      \@tempswafalse\@ifdefinable#1{\@tempswatrue}%
      \if@tempswa
         \edef\kcmd@fam{\kcmd@keyfam{#1}}%
         \expandafter\kcmd@defcommand\expandafter{\kcmd@fam}#1[{#3}][{#4}][{#5}]{#6}{#2}{#7}%
      \else\endgroup
      \fi
   \fi
}
%    \end{macrocode}
% \end{macro}
% 
% \begin{macro}{\kcmd@defcommand}\qquad\qquad prepares (expands) the arguments before closing the group opened at the very beginning.
% Then it proceeds (\cs{kcmd@yargdef} (normal interface)  or \cs{kcmd@yargedef} (when \cs{newkeycommand}\stform+ is used))
%    \begin{macrocode}
\protected\long\def\kcmd@defcommand#1#2[#3][#4][#5]#6#7#8{%
   \let\commandkey\relax  \let\getcommandkey\relax  \let#2\relax   
   \cslet{#1}\relax  \cslet{#1.commankey}\relax  \cslet{#1.getcommandkey}\relax
   \def\do{\kcmd@undefinekey{\kcmd@gbl}{#1}}%
   \edef\kcmd@defcommand{\endgroup
      \kcmd@undefinekeys{\kcmd@gbl}{#1}% undefines all keys for this keycommand family
      \ifx\kcmd@unexpandchar\@empty\else
         \kcmd@mount@unexpandchar{#1}{\unexpanded\expandafter{\kcmd@unexpandchar}}%
      \fi
      \unexpanded{\kv@parse{#3,#4}}{\kcmd@definekey\noexpand#2{\kcmd@gbl}{#1}}% defines keys
      \csdef{#1.commandkey}####1{\noexpand\csname#1@####1\endcsname}%
      \csdef{#1.getcommandkey}####1{%
         \unexpanded{\unexpanded\expandafter\expandafter\expandafter}{%
                           \noexpand\csname#1@####1\endcsname}}%
      \kcmd@ifplus% \newkeycommand+ / \newkeyenvironment+
         \protected\csdef{#1}{%
            \kcmd@yargedef{\kcmd@gbl}{\kcmd@long}\csname#1\endcsname
                          {\number#5}{\noexpand#7}{\csname#1.unexpandchar\endcsname}}%
         \ifx#7\@gobble\else 
             \protected\def#7{\kcmd@yargedef#7}%
         \fi
      \else% \newkeycommand / \newkeyenvironment
         \csdef{#1}{%
            \kcmd@yargdef{\kcmd@gbl}{\kcmd@long}\csname#1\endcsname
                          {\number#5}{\noexpand#7}}%
         \ifx#7\@gobble\else \def#7####1{% that means we have to define a key-environment
            \def#7{%
               \let\getcommandkey\csname#1.getcommandkey\endcsname
               \let\commandkey\csname#1.commandkey\endcsname
               ####1}%
            }%
         \fi
      \fi
      \def\noexpand\do####1{\unexpanded{\expandafter\noexpand\csname}KV@#1@####1@default%
                                                                                     \endcsname}% 
      \let\commandkey\relax \let\getcommandkey\relax \let#2\relax
      \kcmd@gbl\protected\edef#2{% entry point
         \let\getcommandkey\noexpand\noexpand\csname#1.getcommandkey\endcsname
         \kcmd@ifplus  \let\commandkey\getcommandkey
         \else         \let\commandkey\noexpand\noexpand\csname#1.commandkey\endcsname
         \fi
         \noexpand\kcmd@setdefaults{#1}%
         \ifx#7\@gobble \noexpand\noexpand\noexpand\@testopt
                        {\kcmd@setkeys#2{#1}{\kcmd@otherkey{#4}}}{}%
         \else          \noexpand\noexpand\noexpand\@testopt
                        {\kcmd@setkeys#2{#1}{\kcmd@otherkey{#4}}}{}%
         \fi
         }%
      \csname#1\endcsname% expand \kcmd@yargedef or \kcmd@yargdef
   }\kcmd@defcommand{#6}{#8}% #6 = definition, #8 = definition end-envir
}
\protected\long\def\kcmd@setkeys#1#2#3[#4]{% #1=key-command, #2=family, #3=otherkey, #4=key=value pairs
   \kcmd@normalize@setkeys{#1}{#2}{#3}{#4}\csname#2\endcsname
}
\long\def\kcmd@otherkey#1{\kcmd@nbk#1//{\kcmd@otherkey@name#1=\kcmd@nil}{}//}
\long\def\kcmd@otherkey@name#1=#2\kcmd@nil{#1}
%    \end{macrocode}
% \end{macro}
%
% \begin{macro}{\kcmd@mount@unexpandchar}~\par
% This macro defines the macro \cs{"\textit{family.unexpandchar}"}. 
% \CS{"\textit{family.unexpandchar}"} activates the shortcut character 
% for \cs{unexpanded} and defines its meaning.
%    \begin{macrocode}
\protected \def \kcmd@mount@unexpandchar#1#2{%
   \protected\csdef{#1.unexpandchar}{\begingroup
      \catcode`\~\active \lccode`\~`#2 \lccode`#2 0\relax
         \lowercase{%
            \expandafter\endgroup\expandafter\def\expandafter~{%
               \catcode`#2\active
               \long\def~########1~{\unexpanded{########1}}}%
         ~}%
   }%
}
%    \end{macrocode}
% \end{macro}
%
%----------------------------------------------------------------------------
% \begin{macro}{\kcmd@yargdef}\qquad\qquad
% This is the ``{\tt argdef}'' macro for the normal (non \string+) form:
%    \begin{macrocode}
\protected \def \kcmd@yargdef #1#2#3#4#5{\begingroup
% #1 = global or {}
% #2 = long or {}
% #3 = Command
% #4 = nr of args
% #5 = endenvir (or \@gobble if not an environment, or \relax if #3 is endenvir)
   \def \kcmd@yargd@f ##1#4##2##{\afterassignment#5\endgroup
      #1#2\expandafter\def\expandafter#3\@gobble ##1#4%
   }\kcmd@yargd@f 0##1##2##3##4##5##6##7##8##9###4%
}
%    \end{macrocode}
% \end{macro}
%
% \begin{macro}{\kcmd@yargedef}\qquad\qquad
% This is the ``{\tt argdef}'' macro for the {\rred\bf\string+} form:
%    \begin{macrocode}
\protected \def \kcmd@yargedef#1#2#3#4#5#6{\begingroup
% #1 = global or {}
% #2 = long or {}
% #3 = Command
% #4 = nr of args
% #5 = endenvir (or \@gobble if not an environment, or \relax if #3 is endenvir)
% #6 = unexpandchar mounting macro
  \kcmd@nargs{#4}% 
   \protected\long\def\kcmd@yarg@edef##1##2{\endgroup
         #1\edef#3{\begingroup #6%
            #2\edef#3\unexpanded{##2}{\endgroup\unexpanded{##1}%
         }#3}%
   }%
   \protected\def\kcmd@envir##1{%
      \edef\next{\kcmd@yarg@edef{\def\noexpand#5{\expandonce{#5##1}}\expandonce{#3##1}}}\next
   }%
   \protected\def\kcmd@command##1{%
      \edef\next{\kcmd@yarg@edef{\expandonce{#3##1}}}\next
   }%
   \protected\def\kcmd@yargedef##1{%
      \kcmd@yargedef@##1 0####1####2####3####4####5####6####7####8####9#####4%
   }%
   \ifx#5\@gobble % keycommand
      \def\next{\kcmd@command}%
   \else          % key-environmment
      \def\next{\kcmd@envir}%
   \fi
   \let\@next\relax
   \def\kcmd@yargedef@##1##2#4##3##{%
      \ifx\@next\relax 
         \edef\@next{\next{\expandonce{\kcmd@nargs}}{\expandonce{\@gobble##2#4}}}%
         \ifx#5\@gobble \edef\@next{\expandonce\@next\noexpand#5}%
         \else \edef\@next{\edef\noexpand\@next{\noexpand\unexpanded{\expandonce\@next}}#5}%
         \fi
      \fi
      \afterassignment\@next
      \expandafter\def\expandafter##1\@gobble##2#4%
   }%
   \kcmd@yargedef#3%
}
%    \end{macrocode}
% \end{macro}
%
% \begin{macro}{\kcmd@nargs}\qquad
% Filter macros used by \cs{kcmd@yargedef} to get the correct number of arguments:
%    \begin{macrocode}
\def\kcmd@nargs#1{\edef\kcmd@nargs%##1##2##3##4##5##6##7##8##9%
        {\ifnum#1>0{####1%
         \ifnum#1>1}{####2%
         \ifnum#1>2}{####3%
         \ifnum#1>3}{####4%
         \ifnum#1>4}{####5%
         \ifnum#1>5}{####6%
         \ifnum#1>6}{####7%
         \ifnum#1>7}{####8%
         \ifnum#1>8}{####9%
         \fi\fi\fi\fi\fi\fi\fi\fi}\fi}%
}%
%    \end{macrocode}
% \end{macro}
%
% \subsection{new key-commands}
%
% \begin{macro}{\newkeycommand}\qquad\qquad
% Here are the entry points:
%    \begin{macrocode}
\newrobustcmd*\newkeycommand{\begingroup
   \let\kcmd@gbl\@empty\kcmd@star@or@long\new@keycommand}
\newrobustcmd*\renewkeycommand{\begingroup
   \let\kcmd@gbl\@empty\kcmd@star@or@long\renew@keycommand}
\newrobustcmd*\providekeycommand{\begingroup
   \let\kcmd@gbl\@empty\kcmd@star@or@long\provide@keycommand}
%    \end{macrocode}
% \end{macro}
%
% \begin{macro}{\kcmd@star@or@long}~\par
% This is the adaptation of \LaTeX's \cs{@star@or@long} macro:
%    \begin{macrocode}
\def\kcmd@star@or@long#1{\@ifstar
      {\let\kcmd@long\@empty\kcmd@plus#1}
      {\def\kcmd@long{\long}\kcmd@plus#1}}
\def\kcmd@@ifplus#1{\@ifnextchar +{\@firstoftwo{#1}}}% same as LaTeX's \@ifstar
\def\kcmd@plus#1{\kcmd@@ifplus
      {\def\kcmd@ifplus{\iftrue}\kcmd@testopt#1}
      {\def\kcmd@ifplus{\iffalse}\kcmd@testopt#1}}
\def\kcmd@testopt#1{\@testopt{\kcmd@unexpandchar#1}{}}
%    \end{macrocode}
% \end{macro}
%
%\begin{macro}{\kcmd@unexpandchar}\qquad\qquad\quad
% Reads the possible unexpand-char shortcut:
%    \begin{macrocode}
\def\kcmd@unexpandchar#1[#2]{%
   \kcmd@ifplus
      \kcmd@nbk#2//
         {\def\kcmd@unexpandchar{#2}% only once inside group...
          \def\kcmd@unexpandchar@activate{\catcode`#2 \active}%
         }{%
          \let\kcmd@unexpandchar\@empty
          \let\kcmd@unexpandchar@activate\relax
         }//%
   \else  \let\kcmd@unexpandchar\@empty
      \kcmd@nbk#2//%
         {\PackageError\kcmd@pkg@name{shortcut option for \string\unexpanded\MessageBreak
         You can't use a shortcut option for \string\unexpanded\MessageBreak
         without the \string+ form of \string\newkeycommand!}%
         {I will ignore this option and proceed.}%
         }%
         {}//%      
   \fi#1}
%    \end{macrocode}
%\end{macro}
%
% \begin{macro}{\new@keycommand}\qquad\qquad
% Reads the key-command name (cs-token):
%    \begin{macrocode}
\def\new@keycommand#1{\@testopt{\@newkeycommand#1}0}
%    \end{macrocode}
% \end{macro}
%
%\begin{macro}{\@newkeycommand}\qquad\qquad
% Reads the first optional parameter (keys or number of mandatory args):
%    \begin{macrocode}
\long\def\@newkeycommand#1[#2]{% #2 = key=values or N=mandatory args
   \kcmd@ifplus \kcmd@unexpandchar@activate \fi% activates unexpand-char before reading definition
   \kcmd@ifstrdigit{#2}%
      {\@new@key@command#1[][][{#2}]}% no kv, no optkey, number of args
      {\@testopt{\@new@keycommand#1[{#2}]}0}}% kv, check for optkey/nr of args
%    \end{macrocode}
% \end{macro}
%
%\begin{macro}{\@new@keycommand}\qquad\qquad
% Reads the second optional parameter (opt key or number of mandatory args):
%    \begin{macrocode}
\long\def\@new@keycommand#1[#2][#3]{%
   \kcmd@ifstrdigit{#3}%
      {\@new@key@command#1[{#2}][][{#3}]}% no optkey
      {\@testopt{\@new@key@command#1[{#2}][{#3}]}0}}
%    \end{macrocode}
%\end{macro}
%
%\begin{macro}{\@new@key@command}\qquad\qquad
% Reads the definition of the command (\cs{kcmd@def} handles both cases of commands and environements).
% The so called "unexpand-char shortcut" has been activated before reading command definition:
%    \begin{macrocode}
\long\def\@new@key@command#1[#2][#3][#4]#5{%
      \kcmd@def#1\@gobble[{#2}][{#3}][{#4}]{#5}{}}
%    \end{macrocode}
%\end{macro}
%
% \begin{macro}{\renew@keycommand}
%    \begin{macrocode}
\def\renew@keycommand#1{\begingroup
   \escapechar\m@ne\edef\@gtempa{{\string#1}}%
   \expandafter\@ifundefined\@gtempa
      {\endgroup\@latex@error{\noexpand#1undefined}\@ehc}
      \endgroup
   \let\@ifdefinable\@rc@ifdefinable
   \new@keycommand#1%
}
%    \end{macrocode}
% \end{macro}
%
% \begin{macro}{\provide@keycommand}
%    \begin{macrocode}
\def\provide@keycommand#1{\begingroup
   \escapechar\m@ne\edef\@gtempa{{\string#1}}%
   \expandafter\@ifundefined\@gtempa
      {\endgroup\new@keycommand#1}
      {\endgroup\def\kcmd@donot@provide{\renew@keycommand\kcmd@donot@provide
         }\kcmd@donot@provide}%
}
\let\kcmd@donot@provide\@empty% it must not be undefined
%    \end{macrocode}
% \end{macro}
%
% \subsection{new key-environments}
%
% \begin{macro}{\newkeyenvironment}
%    \begin{macrocode}
\newrobustcmd*\newkeyenvironment{\begingroup
   \let\kcmd@gbl\@empty\kcmd@star@or@long\new@keyenvironment}
\newrobustcmd\renewkeyenvironment{\begingroup
   \let\kcmd@gbl\@empty\kcmd@star@or@long\renew@keyenvironment}
%    \end{macrocode}
% \end{macro}
%
% \begin{macro}{\new@keyenvironment}
%    \begin{macrocode}
\def\new@keyenvironment#1{\@testopt{\@newkeyenva{#1}}{}}
\long\def\@newkeyenva#1[#2]{%
   \kcmd@ifstrdigit{#2}%
      {\@newkeyenv{#1}{[][][{#2}]}}
      {\@testopt{\@newkeyenvb{#1}[{#2}]}{}}}
\long\def\@newkeyenvb#1[#2][#3]{%
   \kcmd@ifstrdigit{#3}%
      {\@newkeyenv{#1}{[{#2}][][{#3}]}}
      {\@testopt{\@newkeyenvc{#1}{[{#2}][{#3}]}}0}}
\long\def\@newkeyenvc#1#2[#3]{\@newkeyenv{#1}{#2[{#3}]}}
\long\def\@newkeyenv#1#2{%
   \kcmd@ifplus \kcmd@unexpandchar@activate \fi
   \kcmd@keyenvir@def{#1}{#2}%
}
\long\def\kcmd@keyenvir@def#1#2#3#4{%
   \expandafter\let\csname end#1\endcsname\relax
   \expandafter\kcmd@def\csname #1\expandafter\endcsname\csname end#1\endcsname#2{#3}{#4}%
}
%    \end{macrocode}
% \end{macro}
%
% \begin{macro}{\renew@keyenvironment}
%    \begin{macrocode}
\def\renew@keyenvironment#1{%
  \@ifundefined{#1}%
     {\@latex@error{Environment #1 undefined}\@ehc
     }\relax
  \cslet{#1}\relax
  \new@keyenvironment{#1}}
%    \end{macrocode}
% \end{macro}
% \iffalse
%<package>
%<package>
% \fi
%
% \subsection{Tests on keys}
%
% \begin{macro}{\ifcommandkey}\qquad
% \{\meta{key-name}\}\{\meta{true}\}\{\meta{false}\}\quad expands \meta{true} only if the value of the key
% is not blank:
%    \begin{macrocode}
\newcommand*\ifcommandkey[1]{\csname @\expandafter\expandafter\expandafter
   \kcmd@nbk\commandkey{#1}//{first}{second}//%
   oftwo\endcsname}
%    \end{macrocode}
% \end{macro}
%
%
% \begin{macro}{\showcommandkeys}\qquad\qquad are helper macros essentially for debuging purpose...
%    \begin{macrocode}
\newrobustcmd*\showcommandkeys[1]{\let\do\showcommandkey\docsvlist{#1}}
\newrobustcmd*\showcommandkey[1]{key \string"#1\string" = %
   \detokenize\expandafter\expandafter\expandafter{\commandkey{#1}}\par}
%    \end{macrocode}
% \end{macro}
% 
%
%    \begin{macrocode}
%</package>
%    \end{macrocode}
%
% \section{Examples}
% \label{sec:examples}
%
%    \begin{macrocode}
%<*example>
\ProvidesFile{keycommand-example}
\documentclass[a4paper]{article}
\usepackage[T1]{fontenc}
\usepackage[latin1]{inputenc}
\usepackage[american]{babel}
\usepackage{keycommand,framed,fancyvrb}
%
\makeatletter
\parindent\z@
\newkeycommand*\Rule[raise=.4ex,width=1em,thick=.4pt][1]{%
   \rule[\commandkey{raise}]{\commandkey{width}}{\commandkey{thick}}%
   #1%
   \rule[\commandkey{raise}]{\commandkey{width}}{\commandkey{thick}}}

\newkeycommand*\charleads[sep=1][2]{%
   \ifhmode\else\leavevmode\fi\setbox\@tempboxa\hbox{#2}\@tempdima=1.584\wd\@tempboxa%
   \cleaders\hb@xt@\commandkey{sep}\@tempdima{\hss\box\@tempboxa\hss}#1%
   \setbox\@tempboxa\box\voidb@x}
\newcommand*\charfill[1][]{\charleads[{#1}]{\hfill\kern\z@}}
\newcommand*\charfil[1][]{\charleads[{#1}]{\hfil\kern\z@}}
%
\newkeyenvironment*{dblruled}[first=.4pt,second=.4pt,sep=1pt,left=\z@]{%
   \def\FrameCommand{%
      \vrule\@width\commandkey{first}%
      \hskip\commandkey{sep}
      \vrule\@width\commandkey{second}%
      \hspace{\commandkey{left}}}%
   \parindent\z@
   \MakeFramed {\advance\hsize-\width \FrameRestore}}
   {\endMakeFramed}
%
\makeatother
\begin{document}
\title{This is {\tt keycommand-example.tex}}
\author{Florent Chervet}
\date{July 22, 2009}

\maketitle

{\Large Please refer to {\tt keycommand-example.tex} for definitions.}

\section{Example of a keycommand : \texttt{\string\Rule}}

\begin{tabular*}\textwidth{rl}
\verb+\Rule[width=2em]{hello}+:&\Rule[width=2em]{hello}\cr
\verb+\Rule[thick=1pt,width=2em]{hello}+:&\Rule[thick=1pt,width=2em]{hello}\cr
\verb+\Rule{hello}+:&\Rule{hello}\cr
\verb+\Rule[thick=1pt,raise=1ex]{hello}+:&\Rule[thick=1pt,raise=1ex]{hello}
\end{tabular*}

\section{Example of a keycommand : \texttt{\string\charfill}}

\begin{tabular*}\textwidth{rp{.4\textwidth}}
\verb+\charfill{$\star$}+: & \charfill{$\star$}\cr
\verb+\charfill[sep=2]{$\star$}+: & \charfill[sep=2]{$\star$} \\
\verb+\charfill[sep=.7]{\textasteriskcentered}+: & \charfill[sep=.7]{\textasteriskcentered}
\end{tabular*}


\section{Example of a keyenvironment : \texttt{dblruled}}

Key environment \texttt{dblruled } uses \texttt{framed.sty} and therefore it can be used 
even if a pagebreak occurs inside the environment:
\medskip

\verb+\begin{dblruled}+\par
\verb+   test for dblruled key-environment\par+\par
\verb+   test for dblruled key-environment\par+\par
\verb+   test for dblruled key-environment+\par
\verb+\end{dblruled}+

\begin{dblruled}
 test for dblruled key-environment\par
 test for dblruled key-environment\par
 test for dblruled key-environment
\end{dblruled}


\verb+\begin{dblruled}[first=4pt,sep=2pt,second=.6pt,left=.2em]+\par
\verb+   test for dblruled key-environment\par+\par
\verb+   test for dblruled key-environment\par+\par
\verb+   test for dblruled key-environment+\par
\verb+\end{dblruled}+

\begin{dblruled}[first=4pt,sep=2pt,second=.6pt,left=.2em]
 test for dblruled key-environment\par
 test for dblruled key-environment\par
 test for dblruled key-environment
\end{dblruled}

\end{document}
%</example>
%    \end{macrocode}
% \DeleteShortVerb{\+}^^A\UndefineShortVerb{\+}
% \begin{History}
% 
%   \begin{Version}{2010/04/27 v3.1415}
%   \item Key-environment can now be nested ! (it's not too late... I hope so)
%   \item Keys and mandatory arguments as well can be used in both \texttt{begin} end \texttt{end} part of the environment.
%   \end{Version}
% 
%   \begin{Version}{2010/04/25 v3.141}
%   \item No new feature but a real improvement in optimization. \\
%         In particular, \thispackage does not load \xpackage{etextools} anymore. \\
%   \item Bug fix for \cs{providekeycommand}.
%         
%   \end{Version}
%
%   \begin{Version}{2010/04/18 v3.14}
%   \item Correction of bug in the normalization process. \\
%         Correction of a bug in \cs{ifcommandkey} (undesirable space...)
%   \item Modification of the pdf documentation for the \stform+ form of key-environments.
%   \end{Version}
%
%   \begin{Version}{2010/03/28 v3.0}
%   \item Complete redesign of the implementation. \\
%   \xpackage{keycommand} is now based on some macros of \xpackage{etoolbox}.
%
%   \item Adding the + prefix and the ability to capture keys that where not defined.
%
%   \end{Version}
%
%   \begin{Version}{2009/07/22 v1.0}
%   \item
%     First version.
%   \end{Version}
%
% \end{History}
%
% \begin{thebibliography}{9}
%
% \bibitem{xkeyval}
%   Hendri Adriaens:
%   \textit{The \xpackage{xkeyval} package};
%   2008/08/13 v2.6a;
%   \CTAN{macros/latex/contrib/xkeyval.dtx}
%
% \bibitem{kvsetkeys}
%   Heiko Oberdiek:
%   \textit{The \xpackage{kvsetkeys} package};
%   2007/09/29 v1.3;
%   \CTAN{macros/latex/contrib/oberdiek/kvsetkeys.dtx}.
%
% \bibitem{keyval}
%   David Carlisle:
%   \textit{The \xpackage{keyval} package};
%   1999/03/16 v1.13;
%   \CTAN{macros/latex/required/graphics/keyval.dtx}.
%
% \end{thebibliography}
%
% \PrintIndex
%
% \label{LastPage}
% \Finale