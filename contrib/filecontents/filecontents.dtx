% \iffalse meta-comment
%
% Copyright (C) 2011 Scott Pakin <scott+fc@pakin.org>
% -------------------------------------------------------
%
% This package may be distributed and/or modified under the
% conditions of the LaTeX Project Public License, either version 1.3c
% of this license or (at your option) any later version.
% The latest version of this license is in
%
%    http://www.latex-project.org/lppl.txt
%
% and version 1.3c or later is part of all distributions of LaTeX
% version 2008/05/04 or later.
%
% \fi
%
% \iffalse
%<*driver>
\ProvidesFile{filecontents.dtx}
%</driver>
%<package>\NeedsTeXFormat{LaTeX2e}[1999/12/01]
%<package>\ProvidesPackage{filecontents}
%<*package>
    [2011/10/08 v1.3 Create an external file from within a LaTeX document]
%</package>
%
%<*driver>
\documentclass{ltxdoc}
\EnableCrossrefs
\CodelineIndex
\RecordChanges
\begin{document}
\DocInput{filecontents.dtx}
\end{document}
%</driver>
% \fi
%
% %%%%%%%%%%%%%%%%%%%%%%%%%%%%%%%%%%%%%%%%%%%%%%%%%%%%%%%%%%%%%%%%%%%%
%
% \CheckSum{175}
%
% \CharacterTable
%  {Upper-case    \A\B\C\D\E\F\G\H\I\J\K\L\M\N\O\P\Q\R\S\T\U\V\W\X\Y\Z
%   Lower-case    \a\b\c\d\e\f\g\h\i\j\k\l\m\n\o\p\q\r\s\t\u\v\w\x\y\z
%   Digits        \0\1\2\3\4\5\6\7\8\9
%   Exclamation   \!     Double quote  \"     Hash (number) \#
%   Dollar        \$     Percent       \%     Ampersand     \&
%   Acute accent  \'     Left paren    \(     Right paren   \)
%   Asterisk      \*     Plus          \+     Comma         \,
%   Minus         \-     Point         \.     Solidus       \/
%   Colon         \:     Semicolon     \;     Less than     \<
%   Equals        \=     Greater than  \>     Question mark \?
%   Commercial at \@     Left bracket  \[     Backslash     \\
%   Right bracket \]     Circumflex    \^     Underscore    \_
%   Grave accent  \`     Left brace    \{     Vertical bar  \|
%   Right brace   \}     Tilde         \~}
%
% \GetFileInfo{filecontents.dtx}
%
% \title{The \textsf{filecontents} package\thanks{This file
%         has version number \fileversion, last
%         revised \filedate.}}
% \author{Scott Pakin\\scott+fc@pakin.org}
% \maketitle
%
% \changes{v1.0}{2001/07/31}{Initial version}
% \changes{v1.1a}{2006/03/11}{%
%   Clarified/corrected the \texttt{\string\string\string\documentclass}
%   restriction as per Robin Fairbairns's suggestion}
%
% ^^A  The following were copied verbatim from source2e.tex.
% \DoNotIndex{\def,\long,\edef,\xdef,\gdef,\let,\global}
% \DoNotIndex{\if,\ifnum,\ifdim,\ifcat,\ifmmode,\ifvmode,\ifhmode,%
%             \iftrue,\iffalse,\ifvoid,\ifx,\ifeof,\ifcase,\else,\or,\fi}
% \DoNotIndex{\box,\copy,\setbox,\unvbox,\unhbox,\hbox,%
%             \vbox,\vtop,\vcenter}
% \DoNotIndex{\@empty,\immediate,\write}
% \DoNotIndex{\egroup,\bgroup,\expandafter,\begingroup,\endgroup}
% \DoNotIndex{\divide,\advance,\multiply,\count,\dimen}
% \DoNotIndex{\relax,\space,\string}
% \DoNotIndex{\csname,\endcsname,\@spaces,\openin,\openout,%
%             \closein,\closeout}
% \DoNotIndex{\catcode,\endinput}
% \DoNotIndex{\jobname,\message,\read,\the,\m@ne,\noexpand}
% \DoNotIndex{\hsize,\vsize,\hskip,\vskip,\kern,\hfil,\hfill,\hss}
% \DoNotIndex{\m@ne,\z@,\z@skip,\@ne,\tw@,\p@}
% \DoNotIndex{\dp,\wd,\ht,\vss,\unskip}
%
% ^^A  The following are specific to filecontents.dtx.
% \DoNotIndex{\@currenvir,\@gobble,\@gobblefour,\@ifundefined,\@makeother}
% \DoNotIndex{\@undefined,\active,\chardef,\day,\do,\dospecials,\E,\end,\I}
% \DoNotIndex{\if@tempswa,\L,\LaTeX,\loop,\MessageBreak,\month}
% \DoNotIndex{\newenvironment,\number,\repeat,\reserved@b,\reserved@c}
% \DoNotIndex{\two@digits,\year,\*,\^}
%
% ^^A  Define some commands to help delineate my changes.
% \newcommand{\startfcchanges}{^^A
%   \centerline{^^A
%     \makebox[0pt]{^^A
%       \raisebox{-2\baselineskip}[0pt][0pt]{^^A
%       \makebox[0pt][l]{\rule{1pt}{2\baselineskip}}}^^A
%       \rule{1em}{1pt}^^A
%       \rule{\linewidth}{1pt}^^A
%       \rule{1em}{1pt}^^A
%       \raisebox{-2\baselineskip}[0pt][0pt]{^^A
%       \makebox[0pt][r]{\rule{1pt}{2\baselineskip}}}^^A
%     }^^A
%   }^^A
%  \noindent
% }
% \newcommand{\stopfcchanges}{^^A
%   \centerline{^^A
%     \makebox[0pt]{^^A
%       \raisebox{0pt}[0pt][0pt]{^^A
%       \makebox[0pt][l]{\rule{1pt}{2\baselineskip}}}^^A
%       \rule{1em}{1pt}^^A
%       \rule{\linewidth}{1pt}^^A
%       \rule{1em}{1pt}^^A
%       \raisebox{0pt}[0pt][0pt]{^^A
%       \makebox[0pt][r]{\rule{1pt}{2\baselineskip}}}^^A
%     }^^A
%   }^^A
% }
%
%
% \section{Introduction}
%
% \DescribeEnv{filecontents}
% There is a little-known environment called |filecontents| that is
% built into \LaTeXe.  Here is |filecontents|' description, which was
% taken from |ltclass.dtx|:
%
% \begin{quotation}
%   The environment |filecontents| is intended for passing the contents of
%   packages, options, or other files along with a document in a single
%   file.  It has one argument, which is the name of the file to
%   create.  If that file already exists (maybe only in the current
%   directory if the OS supports a notion of a `current directory' or
%   `default directory') then nothing happens (except for an information
%   message) and the body of the environment is bypassed.  Otherwise, the
%   body of the environment is written verbatim to the file name given as
%   the first argument, together with some comments about how it was
%   produced.
%
%   The environment is allowed only before |\documentclass| to ensure that
%   all packages or options necessary for this particular run are present
%   when needed.  The begin and end tags should each be on a line by
%   itself.  There is also a star-form; this does not write extra comments
%   into the file.
% \end{quotation}
%
% \noindent
% (The comment about |filecontents| being valid only before
% |\documentclass| is, in fact, untrue.  |filecontents| is allowed
% anywhere in the document's preamble.)
%
% \bigskip
%
% The \textsf{filecontents} package provides a hacked-up version of the
% |filecontents| and |filecontents*| environments that lifts the two
% restrictions stated above, namely that existing files are never
% overwritten and that |filecontents| must be used before the
% |\documentclass| declaration (really, the |\begin{document}|).
% \textsf{filecontents} is therefore a more convenient way to write
% external files from within a \LaTeX\ document than is provided by
% default by the \LaTeXe\ kernel.
%
%
% \paragraph{Sample usage}
% |filecontents| works much like |verbatim|, except that it takes
% a mandatory filename argument:
%
% \begin{verbatim}
%     \begin{filecontents}{myfile.tex}
%     This text gets written to \texttt{myfile.tex}.
%     \end{filecontents}
% \end{verbatim}
%
% \noindent
% The preceding code will write a |myfile.tex| file with contents resembling
% the following:
%
% \begin{verbatim}
%     %% LaTeX2e file `myfile.tex'
%     %% generated by the `filecontents' environment
%     %% from source `mydocument' on 2001/07/31.
%     %%
%     This text gets written to \texttt{myfile.tex}.
% \end{verbatim}
%
% \noindent
% |myfile.tex| can then be incorporated back into the document with
% |\include| or |\input|.  Had |filecontents*| been used instead of
% |filecontents|, the file would have contained only the
% ``\texttt{This text gets written to} \verb"\texttt{myfile.tex}."'' line.
% |filecontents*| is therefore useful for writing non-\LaTeX\ files such as
% Encapsulated PostScript files.
%
% If you use the \textsf{ltxtable} package you may find
% \textsf{filecontents} particularly useful.  \textsf{ltxtable} is a
% crude conglomeration of \textsf{longtable}, which allows tables to
% cross page boundaries, and \textsf{tabularx}, which enables tables to
% stretch to a specified width.  \textsf{ltxtable}'s interface is a bit
% cumbersome, however; it requires that the |longtable| environment be
% contained in a separate file.  With the \textsf{filecontents} package
% you can create this file right before the |\LTXtable| invocation---a
% far more convenient alternative than having to manually place the
% table within a separate file.
%
%
% \StopEventually{^^A
%   \PrintChanges
%   \PrintIndex
% }
%
% \section{Implementation}
%
% Most users can stop reading at this point.  The Implementation section
% contains the annotated source code for the \textsf{filecontents}
% package itself, which is useful only to people who want a detailed and
% precise explanation of how \textsf{filecontents} works.
%
% To give credit where credit is due, I wrote virtually none of the
% \textsf{filecontents} code myself.  It comes almost exclusively from
% the \LaTeXe\ source code, specifically from the file |ltclass.dtx|,
% which is attributed to Frank Mittelbach, Chris Rowley, Alan Jeffrey,
% and David Carlisle.  I merely made a few small changes (indicated below
% by bracketed blocks of code and comments) to make the |filecontents|
% environment more convenient to use.
%
%    \begin{macrocode}
%<*package>
%    \end{macrocode}
%
% \begin{macro}{\filec@ntents}
% Except where indicated, the source---including comments---to the
% |\filec@ntents| macro was taken verbatim from |ltclass.dtx|.
%    \begin{macrocode}
\begingroup%
\catcode`\*=11 %
\catcode`\^^M\active%
\catcode`\^^L\active\let^^L\relax%
\catcode`\^^I\active%
%    \end{macrocode}
%
%    \begin{macrocode}
\gdef\filec@ntents#1{%
  \openin\@inputcheck#1 %
%    \end{macrocode}
%
% \startfcchanges
% In the original code a pre-existing file would not be overwritten.
% In the new version the file existence check is used solely to decide
% whether to output ``\texttt{Writing file `\textrm{\meta{filename}}'}''
% or ``\texttt{Overwriting file `\textrm{\meta{filename}}'}''.  Control
% flow then always falls through to what used to be the |\ifeof| case
% (file does not exist; open it), never the |\else| case (file already
% exists; do nothing).
% \changes{v1.2}{2009/03/17}{Added percent signs after the
%   \texttt{\char`\\else} and \texttt{\char`\\fi}
%   lines as per Heiko Oberdiek's suggestion for getting \textsf{filecontents}
%   to work with the \textsf{guitar} package}
% \changes{v1.3}{2011/10/08}{Added a \texttt{\char`\\closein} to fix bug
%   \textsf{latex/1487} (reported by Ulrike Fischer and Heiko Oberdiek)}
%    \begin{macrocode}
  \ifeof\@inputcheck%
    \@latex@warning@no@line%
        {Writing file `\@currdir#1'}%
  \else %
    \@latex@warning@no@line%
        {Overwriting file `\@currdir#1'}%
  \fi %
  \closein\@inputcheck %
%    \end{macrocode}
%    \begin{macrocode}
  \chardef\reserved@c15 %
  \ch@ck7\reserved@c\write%
  \immediate\openout\reserved@c#1\relax%
%    \end{macrocode}
% \stopfcchanges
%
%    \begin{macrocode}
  \if@tempswa%
    \immediate\write\reserved@c{%
      \@percentchar\@percentchar\space%
          \expandafter\@gobble\string\LaTeX2e file `#1'^^J%
      \@percentchar\@percentchar\space  generated by the %
        `\@currenvir' \expandafter\@gobblefour\string\newenvironment^^J%
      \@percentchar\@percentchar\space from source `\jobname' on %
         \number\year/\two@digits\month/\two@digits\day.^^J%
      \@percentchar\@percentchar}%
  \fi%
  \let\do\@makeother\dospecials%
%    \end{macrocode}
% \startfcchanges
% The \textsf{inputenc} packages might have marked some of the upper~128
% character codes ``active'' (category code~13).  That confuses
% \textsf{filecontents}.  Hence, we locally mark each of the upper~128
% character codes as ``letter'' (category code~11) so that they can be
% written correctly to a file.
% \changes{v1.1}{2004/08/16}{Made it possible for \textsf{filecontents}
%   to write \mbox{Latin-1} characters as per Harry Schmidt's feature
%   request and Frank Mittelbach's suggestion of how to implement it.}
% \changes{v1.2}{2009/03/17}{Added percent signs after each line in
%   the loop as per Heiko Oberdiek's suggestion for getting
%   \textsf{filecontents} to work with the \textsf{guitar} package}
%    \begin{macrocode}
  \count0=128\relax %
  \loop %
    \catcode\count0=11\relax %
    \advance\count0 by 1\relax %
    \ifnum\count0<256 %
  \repeat %
%    \end{macrocode}
% \stopfcchanges
%
%    \begin{macrocode}
  \edef\E{\@backslashchar end\string{\@currenvir\string}}%
  \edef\reserved@b{%
    \def\noexpand\reserved@b%
         ####1\E####2\E####3\relax}%
  \reserved@b{%
    \ifx\relax##3\relax%
%    \end{macrocode}
% There was no |\end{filecontents}|
%    \begin{macrocode}
      \immediate\write\reserved@c{##1}%
    \else%
%    \end{macrocode}
% There was a |\end{filecontents}|, so stop this time.
%    \begin{macrocode}
      \edef^^M{\noexpand\end{\@currenvir}}%
      \ifx\relax##1\relax%
      \else%
%    \end{macrocode}
% Text before the |\end|, write it with a warning.
%    \begin{macrocode}
          \@latex@warning{Writing text `##1' before %
             \string\end{\@currenvir}\MessageBreak as last line of #1}%
        \immediate\write\reserved@c{##1}%
      \fi%
      \ifx\relax##2\relax%
      \else%
%    \end{macrocode}
% Text after the |\end|, ignore it with a warning.
%    \begin{macrocode}
         \@latex@warning{%
           Ignoring text `##2' after \string\end{\@currenvir}}%
      \fi%
    \fi%
    ^^M}%
%    \end{macrocode}
%
%    \begin{macrocode}
  \catcode`\^^L\active%
  \let\L\@undefined%
  \def^^L{\@ifundefined L^^J^^J^^J}%
  \catcode`\^^I\active%
  \let\I\@undefined%
  \def^^I{\@ifundefined I\space\space}%
  \catcode`\^^M\active%
  \edef^^M##1^^M{%
    \noexpand\reserved@b##1\E\E\relax}}%
\endgroup%
%    \end{macrocode}
% \end{macro}
%
%
% \begin{macro}{\fc@no@preamblecmds}
% \LaTeXe\ declares |\filecontents|, |\filecontents*|, and all of the
% related helper macros as |\@onlypreamble|, meaning they become invalid
% after the |\begin{document}|.  The following code re-enables their
% usage anywhere in the document.  It was taken from the
% \textsf{pkgindoc} package (which is generated from |ltclass.dtx|), but
% modified to re-enable only the commands needed by
% \textsf{filecontents}, not all of the class and package
% option-processing commands, as well.
%    \begin{macrocode}
\def\fc@no@preamblecmds#1\do\filecontents#2\do\filec@ntents#3\relax{%
  \gdef\@preamblecmds{#1#3}}
\expandafter\fc@no@preamblecmds\@preamblecmds\relax
%    \end{macrocode}
% \end{macro}
%
%    \begin{macrocode}
%</package>
%    \end{macrocode}
%
% \Finale
%
