% \iffalse meta-comment
% !TEX program  = pdfLaTeX
%
%<*internal>
\iffalse
%</internal>
%
%<*exampletex>
\documentclass[a4paper,openright,parskip=half-]{semproc}
\addglobalbib{example.bib}
\usepackage[ngerman,english]{babel}
\begin{document}
\author{Max Mustermann}
\title{Example}
\maketitle
\globaltoc
\chapter{Introduction}
This document serves as an example for the \verb|semproc| class. 

The reader should note that this introduction, which is typeset as a
\verb|\chapter| shows up in the table of contents. However, the numbering scheme
of talks and chapters is confusing at best. \verb|semproc| therefore offers a
macro \verb|\nonumchapter| for unnumbered chapters that show up in the table of
contents and may be used e.\,g.\ for introductions as in the following second
introduction.

\nonumchapter{Second Introduction}
This is a second introduction with no number but still showing up in the table
of contents.

\talk[speaker={Donald Knuth}]{Notation}
\begin{summary}
  This could be a manuscript for a talk entitled ``Notation'' given by
  Donald Knuth at Stanford university. A video recording is available at
  \cite{video}.
\end{summary}
\localtoc
\section{Introduction}
An introduction could go here.
\section{Examples of Notation}
\subsection{Notation for integrals}
One could compare Leibniz' and Newton's notation for integrals as Knuth does in
his talk.
\subsection{Frege's Grundgesetze}
Even if Knuth does not talk about the notation in \citeauthor{frege}'s
\cite{frege}, there is even a paper~\cite{onfrege} on it.
\section{Outlook}
A final section with final words.
\localbib

\talk[speaker={Somebody},notes={Somebody else}]{On Something}
\begin{summary}
  Some Summary
\end{summary}
\localtoc
\section{Section A}
\section{Section B}

\selectlanguage{ngerman}
\talk[speaker={Hans}]{\"Uber etwas Wichtiges}
\begin{summary}
  Hier sieht man, dass \verb|semproc| an andere Sprachen, wie zum Beispiel
  \verb|ngerman| angepasst werden kann. Momentan werden aber nur \verb|english|
  und \verb|ngerman| unterstützt.
\end{summary}
\end{document}
%</exampletex>
%<*examplebib>
@article{onfrege,
    AUTHOR = {Green, J. J. and Rossberg, Marcus and Ebert, Philip A.},
     TITLE = {The convenience of the typesetter; notation and typography in
              {F}rege's {\itshape {G}rundgesetze der {A}rithmetik}},
   JOURNAL = {Bull. Symb. Log.},
    VOLUME = {21},
      YEAR = {2015},
    NUMBER = {1},
     PAGES = {15--30},
      ISSN = {1079-8986},
       DOI = {10.1017/bsl.2015.5},
}
@book {frege,
    AUTHOR = {Frege, G.},
     TITLE = {Grundgesetze der {A}rithmetik. {B}and {I}, {II}},
 PUBLISHER = {Georg Olms Verlagsbuchhandlung, Hildesheim},
      YEAR = {1962},
     PAGES = {Vol. I, xxxii+254 pp.; Vol. II, xvi+266},
}
@misc {video,
  AUTHOR = {Donald Knuth},
  TITLE = {Notation},
  URL = {https://goo.gl/f122V5}
}
%</examplebib>
%<*readme>
----------------------------------------------------------------
semproc --- LaTeX class for seminar proceedings
E-mail: tobias.columbus@kit.edu
Released under the LaTeX Project Public License v1.3c or later
See http://www.latex-project.org/lppl.txt
----------------------------------------------------------------

This class provides some means for typesetting seminar proceedings based on
KOMA-Script's scrreprt class and etoc: It offers an alternative to \chapter that
typesets the speaker and if necessary the typist of the notes for the talk in
question. Moreover, the class provides two types of table of contents. A global
table of contents showing only the talks of the seminar and the respective
speakers and a local table of contents for each talk showing the sections and
subsections of the respective talk.
%</readme>
%
%<*internal>
\fi
\def\nameofplainTeX{plain}
\ifx\fmtname\nameofplainTeX\else
  \expandafter\begingroup
\fi
%</internal>
%
%<*install>
\input docstrip.tex
\keepsilent
\askforoverwritefalse
\preamble
----------------------------------------------------------------
semproc --- LaTeX class for seminar proceedings
E-mail: tobias.columbus@kit.edu
Released under the LaTeX Project Public License v1.3c or later
See http://www.latex-project.org/lppl.txt
----------------------------------------------------------------

\endpreamble
\postamble

Copyright (C) 2015 by Tobias Columbus <tobias.columbus@kit.edu>

This work may be distributed and/or modified under the
conditions of the LaTeX Project Public License (LPPL), either
version 1.3c of this license or (at your option) any later
version.  The latest version of this license is in the file:

http://www.latex-project.org/lppl.txt

This work is "maintained" (as per LPPL maintenance status) by
Tobias Columbus.

This work consists of the files semproc.dtx
and the derived files           semproc.ins,
                                semproc.pdf,
                                semproc.cls,
                                example.tex,
                                example.bib,
                                example.pdf

\endpostamble
\usedir{tex/latex/semproc}
\generate{
  \file{\jobname.cls}{\from{\jobname.dtx}{class}}
}
%</install>
%<install>\endbatchfile
%
%<*internal>
\usedir{source/latex/semproc}
\generate{
  \file{\jobname.ins}{\from{\jobname.dtx}{install}}
}
\generate{
  \file{example.tex}{\from{\jobname.dtx}{exampletex}}
}
\generate{
  \file{example.bib}{\from{\jobname.dtx}{examplebib}}
}
\nopreamble\nopostamble
\usedir{doc/latex/semproc}
\generate{
  \file{README.txt}{\from{\jobname.dtx}{readme}}
}
\ifx\fmtname\nameofplainTeX
  \expandafter\endbatchfile
\else
  \expandafter\endgroup
\fi
%</internal>
% 
%<*driver>
\ProvidesFile{\jobname.dtx}
%</driver>
%<class>\NeedsTeXFormat{LaTeX2e}
%<class>\ProvidesClass{semproc}
%<*class>
  [2015/06/15 v0.1 LaTeX class for seminar proceedings]
%</class>
%
%<*driver>
\documentclass{ltxdoc}
\usepackage[T1]{fontenc}
\usepackage{lmodern}
\usepackage[numbered]{hypdoc}
\usepackage{listings}
\usepackage{xcolor}
\usepackage{tabularx}
\parindent=0em
\parskip=0.5ex
\lstset{
  basicstyle=\small\ttfamily,
  language=[LaTeX]TeX,
}
\EnableCrossrefs
\CodelineIndex
\RecordChanges
\begin{document}
  \DocInput{\jobname.dtx}
\end{document}
%</driver>
% \fi
% 
%\GetFileInfo{\jobname.dtx}
%
%\title{
%  \textsf{semproc} --- LaTeX class for seminar proceedings\thanks{
%    This file describes version \fileversion, last revised \filedate.
%  }
%}
%\author{
%  Tobias Columbus\thanks{E-mail: tobias.columbus@kit.edu}
%}
%\date{Released \filedate}
%
%\maketitle
%
%\changes{v0.1}{2015/06/15}{First public release}
%
% \section{Documentation}
% The class |semproc| is derived from KOMA-Script's class |scrreprt|. In
% addition to |scrreprt|'s many features, it provides |\talk| as an alternative
% to the |\chapter| command and some means to have a table of contents and a
% bibliography for each such |\talk|. Currently, |semproc| supports German and
% English.
%
% \subsection{Macros}
% 
% \DescribeMacro{\talk} |\talk| \oarg{speaker=\{Speaker\},notes=\{Typist\}}
% \marg{title} starts a new talk with title \meta{title}, speaker \meta{Speaker}
% and typist \meta{Typist}. The |\talk| macro also starts a new
% |biblatex|-refsection.
%
% \DescribeMacro{\globaltoc} |\globaltoc| typesets a global table of contents
% consisting of the various talks and its speakers.
%
% \DescribeMacro{\localtoc} |\localtoc| typesets a local table of contents
% consisting of the sections and subsections in the current talk.
%
%
% \DescribeMacro{\localbib} |\localbib| typesets a bibliography for the current
% talk.
%
% \DescribeMacro{\nonumchapter} |\nonumchapter| \marg{title} starts a new
% unnumbered |\chapter*| with \meta{title} but makes sure the chapter occurs in
% the global table of contents. More precisely, |\nonumchapter| is just a
% wrapper around |\chapter*| and |\etoctoccontentsline*|.
%
% \subsection{Environments}
%
% \DescribeEnv{summary} The |summary| environment is a drop-in
% replacement for the |abstract| environment.
%
% \subsection{Customization}
% |semproc| allows for customization of the fonts used in the heading of a talk
% and in the local and global table of contents. The fonts may be customized
% with |\setkomafont| or |\addtokomafont| as described in the documentation of
% KOMA-Script. All fonts are shown in table~\ref{table:fonts}.
% \begin{table}
%   \begin{tabularx}{\textwidth}{lXX}
%     Font & Usage & Default  \\\hline
%     talkhead & Font of the head of a talk heading, i.\,e.\ the line
%     ``Talk~3''. & |\usekomafont{disposition}| |\usekomafont{section}| \\
%     talktitle & Font of the title of a talk &
%     |\usekomafont{disposition}| |\usekomafont{chapter}| \\
%     talkspeakers & Font of the speaker and typist of a talk &
%     |\usekomafont{disposition}| \\\\
%     globaltocspeaker & Speaker in an entry in the global table of contents &
%     |\sffamily| \\
%     globaltocnumber& Number of an entry in the global table of contents &
%     |\usekomafont{disposition}| \\
%     globaltoctitle& Title of an entry in the global table of contents &
%     |\usekomafont{disposition}| \\
%     globaltocpage& Page number of an entry in the global table of contents &
%     |\usekomafont{disposition}| \\\\
%     localtocheading & Heading of the local table of contents
%     typeset with |\localtoc| & |\usekomafont{disposition}|  \\
%     localtocnumber & Number of an entry in the local table of contents &
%     |\sffamily| \\
%     localtoctitle& Title of an entry in the local table of contents &
%     |\sffamily| \\
%     localtocpage & Page number of an entry in the local table of contents &
%     |\sffamily| \\\\
%     summaryheading & Font for the heading of a |summary| environment &
%     |\usekomafont{disposition}|
%   \end{tabularx}
%   \caption{Fonts that can be customized.}\label{table:fonts}
% \end{table}
% 
% \subsection{Packages}
% |semproc| loads the following packages: |xkeyval|, |xcolor| |etoc|,
% |hyperref|, |bookmark|, |etoolbox|, |biblatex| and |csquotes|. Moreover, it
% passes the option |dvipsnames| to the package |xcolor|, which is loaded
% automatically by hyperref. For the precise arguments to these packages, the
% reader is referred to the implementation.
%
% Note however, that bibliographies should be typeset using |biblatex| as
% opposed to |bibtex|.
%
% \subsection{Example}
% An example can be found in the file |example.tex|.
%
%\StopEventually{
%  \PrintChanges
%  \PrintIndex
%}
%
%
% \section{Implementation}
%    \begin{macrocode}
%<*class>
%    \end{macrocode}
%
%\begin{macro}{scrreprt}
% Load |scrreprt| and pass all options to |scrreprt|.
%    \begin{macrocode}
\DeclareOption*{\PassOptionsToClass{\CurrentOption}{scrreprt}}
\ProcessOptions\relax
\LoadClass{scrreprt}
%    \end{macrocode}
%\end{macro}
%
%\begin{macro}{Packages}
% We load |xkeyval| for processing the names of the speaker/typist of each talk.
%    \begin{macrocode}
\RequirePackage{xkeyval}
%    \end{macrocode}
% We load |xcolor| and pass option |dvipsnames| as we set |RoyalBlue| as default
% colour for all links.
%    \begin{macrocode}
\PassOptionsToPackage{dvipsnames}{xcolor}
\RequirePackage{xcolor}
%    \end{macrocode}
% We load |etoc| for creating the macros for the local table of contents per
% talk and the global table of contents.
%    \begin{macrocode}
\RequirePackage{etoc}
%    \end{macrocode}
% We load |hyperref| for references. Option |hypertexnames| must be set to
% |false|, otherwise referencing the talks is buggy -- at least the last time I
% tried.
%    \begin{macrocode}
\RequirePackage%
  [unicode=true,colorlinks=true,hypertexnames=false]
  {hyperref}
\RequirePackage{bookmark}
\hypersetup{allcolors={RoyalBlue}}
%    \end{macrocode}
% We load |etoolbox| for |\ifdefvoid| used in |\semproc@talk@head|.
%    \begin{macrocode}
\RequirePackage{etoolbox}
%    \end{macrocode}
% We load |biblatex| for managing bibliographies per talk via refsections and
% |csquotes| as it is recommended by |biblatex|.
%    \begin{macrocode}
\RequirePackage{biblatex}
\RequirePackage{csquotes}
%    \end{macrocode}
%\end{macro}
%
%\begin{macro}{Internationalization}
% Using |\providecaptionname| of KOMA-Script, we define language dependent
% strings. Currently only German and English are supported. 
%
% We first define the german variants:
%    \begin{macrocode}
\providecaptionname{german,ngerman}%
  {\talkname}{Vortrag}
\providecaptionname{german,ngerman}%
  {\notesname}{Notizen von}
\providecaptionname{german,ngerman}%
  {\nospeakername}{Kein Vortragender}
\providecaptionname{german,ngerman}%
  {\localtocheadingname}{Inhalt}
\providecaptionname{german,ngerman}%
  {\summaryname}{Zusammenfassung}
%    \end{macrocode}
% Next, we define the english variants:
%    \begin{macrocode}
\providecaptionname{USenglish,UKenglish,english}%
  {\talkname}{Talk}
\providecaptionname{USenglish,UKenglish,english}%
  {\notesname}{Notes by}
\providecaptionname{USenglish,UKenglish,english}%
  {\nospeakername}{No speaker}
\providecaptionname{USenglish,UKenglish,english}%
  {\localtocheadingname}{Contents}
\providecaptionname{USenglish,UKenglish,english}%
  {\summaryname}{Abstract}
%    \end{macrocode}
%\end{macro}
%
%\begin{macro}{Fonts}
% Using KOMA-Script's |\newkomafont|, we define various fonts that we use in
% this class. First, we define the fonts |talkhead|, |talktitle| and
% |talkspeakers|, which are used for the heading, i.e.\ the line ``Talk 1'',
% title and speakers/typists of the respective talk.
%    \begin{macrocode}
\newkomafont{talkhead}
  {\usekomafont{disposition}\usekomafont{section}}
\newkomafont{talktitle}
  {\usekomafont{disposition}\usekomafont{chapter}}
\newkomafont{talkspeakers}
  {\usekomafont{disposition}}
%    \end{macrocode}
% Next, we define the fonts |globaltocspeaker|, |globaltocnumber|,
% |globaltocpage| and |globaltoctitle|, which are used for the speaker, number,
% page and title of the talk in the global table of contents.
%    \begin{macrocode}
\newkomafont{globaltocspeaker}
  {\sffamily}
\newkomafont{globaltocnumber}
  {\usekomafont{disposition}}
\newkomafont{globaltocpage}
  {\usekomafont{disposition}}
\newkomafont{globaltoctitle}
  {\usekomafont{disposition}}
%    \end{macrocode}
% Similarly, we define the fonts |localtocnumber|, |localtocpage|,
% |localtoctitle| for the local table of contents. Additionally, we define
% |localtocheading|, which is the font of the heading of the local table of
% contents.
%    \begin{macrocode}
\newkomafont{localtocheading}
  {\usekomafont{disposition}}
\newkomafont{localtocnumber}
  {\sffamily}
\newkomafont{localtocpage}
  {\sffamily}
\newkomafont{localtoctitle}
  {\sffamily}
%    \end{macrocode}
% Finally, we define the font |summaryheading|, which is the font of the heading
% of the |summary| environment also defined in this class.
%    \begin{macrocode}
\newkomafont{summaryheading}
  {\usekomafont{disposition}}
%    \end{macrocode}
%\end{macro}
%
%\begin{macro}{\nonumchapter}
% |\nonumchapter|\marg{title} is a thin wrapper around |\chapter*| and
% |\etoctoccontentsline*|.
%    \begin{macrocode}
\newcommand{\nonumchapter}[1]{%
  \chapter*{#1}%
  \etoctoccontentsline*{chapter}{#1}{0}%
}
%    \end{macrocode}
%\end{macro}
%
%\begin{macro}{Talk counter}
% A new counter for the talks.
%    \begin{macrocode}
\newcounter{talk}
\setcounter{talk}{0}
%    \end{macrocode}
% Compatibility with autoref.
%    \begin{macrocode}
\def\talkautorefname{\talkname}
%    \end{macrocode}
% We let sections be numbered within the talk, equations within the section and
% without referring to the chapter. The latter is achieved by redefining
% |\thesection| and |\theHsection| for hyperref.
%    \begin{macrocode}
\@addtoreset{section}{talk}
\@addtoreset{equation}{section}
\renewcommand\thesection{\arabic{section}}
\renewcommand\theHsection{\arabic{section}}
\renewcommand\theequation{\thesection.\arabic{equation}}
%    \end{macrocode}
%\end{macro}
%
%
%\begin{macro}{\semproc@talk@head}
% The macro |\semproc@talk@head|\marg{title} typesets the heading of the talk
% with the given title. The counter |talk| has to be incremented  and the values
% of |\semproc@talk@speaker| and |\semproc@talk@notes| have to be set before
% calling |\semproc@talk@head|.
%    \begin{macrocode}
\newcommand{\semproc@talk@head}[1]{%
  {\noindent\usekomafont{talkhead}\talkname~\thetalk}%
  {\par\nobreak\vskip2ex}%
  {\usekomafont{talktitle}\noindent#1}
  \vskip1ex\noindent\hrulefill\vskip1ex
  {
    \usekomafont{talkspeakers}
    \noindent\semproc@talk@speaker%
    \ifdefvoid{\semproc@talk@notes}{}{%
      \hfill{\notesname~\semproc@talk@notes}%
    }
    \par\nobreak%
  }%
  \vskip 3.5ex
}
%    \end{macrocode}
%\end{macro}
%
%\begin{macro}{\talk}
% The main macro of this class is |\talk| \oarg{speaker={Speaker},
% notes={Notes}} \marg{title} intended to replace |\chapter|. We first prepare
% the optional key/value options.
%    \begin{macrocode}
\define@key{talk}{speaker}%
  [\nospeakername]%
  {\def\semproc@talk@speaker{#1}}
\define@key{talk}{notes}%
  []%
  {\def\semproc@talk@notes{#1}}
%    \end{macrocode}
% The macro |\talk| parses the key/value options, increments the counter |talk|
% and calls |\semproc@talk@head| for typesetting the heading. Finally, it starts
% a new |biblatex| refsection and writes all the data to the |.aux| file via
% |etoc|'s |\etoccontentsline| macro. The |\@currentlabelname| is set for
% correct references.
%    \begin{macrocode}
\newcommand*\talk[2][\empty]{%
  \presetkeys{talk}{speaker,notes}{}%
  \setkeys{talk}{#1}%
  \refstepcounter{talk}%
  \cleardoubleoddplainpage%
  \thispagestyle{\chapterpagestyle}%
  \semproc@talk@head{#2}
  \renewcommand\@currentlabelname{#2}
  \newrefsection%
  \etoctoccontentsline*{talk}%
    {\protect\numberline{\thetalk}#2}{0}
  \etoctoccontentsline{speaker}%
    {\protect\numberline{}\semproc@talk@speaker}
  \renewcommand\@currentlabelname{#2}
}
%    \end{macrocode}
%\end{macro}
%
%\begin{macro}{Various lengths and a box}
% Lengths for typesetting the local and global tables of contents.
%    \begin{macrocode}
\newlength{\semproc@toc@skip}
\setlength{\semproc@toc@skip}{4em}
\newlength{\semproc@toc@calc}
%    \end{macrocode}
%\end{macro}
%
%\begin{macro}{\semproc@toc@skipafter}
% |\semproc@toc@skipafter| \marg{style} typesets |\etocnumber| with the chosen
% \meta{style} and then skips the difference of |\semproc@toc@skip| and the
% width of |\etocnumber|.
%    \begin{macrocode}
\newcommand\semproc@toc@skipafter[1]{%
  \settowidth{\semproc@toc@calc}%
    {#1\etocthenumber}%
  \addtolength{\semproc@toc@calc}%
    {-\semproc@toc@skip}%
  \setlength{\semproc@toc@calc}%
    {-\semproc@toc@calc}%
  \noindent{#1\etocnumber}%
  \hskip\semproc@toc@calc%
}
%    \end{macrocode}
%\end{macro}
%
%\begin{macro}{\globaltoc}
% |\globaltoc| prints out a global table of contents showing only the talks and
% the respective speakers/typists.
%    \begin{macrocode}
\newcommand{\globaltoc}{%
  \begingroup
    \etocsetnexttocdepth{1}
    \etocsetlevel{speaker}{1}
    \etocsetlevel{section}{2}
    \etocsetlevel{talk}{0}
    \etocsetlevel{chapter}{0}
%    \end{macrocode}
% The speaker is written below the title of the talk with
% |\semproc@toc@skip| indentation and font |globaltocspeaker|.
%    \begin{macrocode}
    \etocsetstyle{speaker}
      {}
      {}
      {%
        \hskip\semproc@toc@skip%
        {\usekomafont{globaltocspeaker}\etocthename}%
        \nobreak\hfill\nobreak\par%
      }
      {}
%    \end{macrocode}
% Before each talk entry in the global table of contents, we have a vertical
% skip of |1ex|. Then, the number of the talk is written left-aligned and the
% title of the talk is written on the same line with indentation
% |\semproc@toc@skip|.
%    \begin{macrocode}
    \etocsetstyle{talk}
      {}
      {\vskip1ex\leavevmode\relax}
      {%
        \semproc@toc@skipafter{\usekomafont{globaltocnumber}}%
        {%
          \usekomafont{globaltoctitle}%
          \etocname%
        }%
        \nobreak%
        \leaders\hbox to .3cm{\hss.}\hfill%
        \nobreak%
        \makebox[1cm][r]{%
          \usekomafont{globaltocpage}%
          \hss\etocpage%
        }\par%
      }
      {}
    \tableofcontents
  \endgroup
}
%    \end{macrocode}
%\end{macro}
%
%\begin{macro}{\localtoc}
% |\localtoc| prints out a local table of contents consisting only of the
% sections and subsections in the current talk.
%    \begin{macrocode}
\newcommand{\localtoc}{%
  \begingroup
    \let\parskip@old%
    \parskip\parskip0ex%
    \etocsetnexttocdepth{2}
    \etocsetlevel{section}{1}
    \etocsetlevel{subsection}{2}
    \etocsetlevel{talk}{-2}
    \etocsetlevel{speaker}{-2}
    \etocsetlevel{chapter}{-2}
%    \end{macrocode}
% Every section except the first one has a |\vskip| of |1ex| before it is
% typeset. Then, sections are written with |\semproc@toc@skip| indentation.
%    \begin{macrocode}
    \etocsetstyle{section}
      {}
      {%
        \ifthenelse{\equal{\etocthenumber}{1}}%
        {}%
        {\vskip1ex}%
      }
      {%
        \semproc@toc@skipafter{%
          \usekomafont{localtocnumber}%
        }%
        {%
          \usekomafont{localtoctitle}%
          \etocname%
        }%
        \nobreak%
        \leaders\hbox to .3cm{\hss.}\hfill%
        \nobreak%
        \makebox[1cm][r]{%
          \usekomafont{localtocpage}%
          \hss\etocpage%
        }\vskip0.5ex%
      }
      {}
%    \end{macrocode}
% Subsections are written with |\semproc@toc@skip| indentation and no additional
% vertical space.
%    \begin{macrocode}
    \etocsetstyle{subsection}
      {}
      {}
      {%
        \semproc@toc@skipafter{%
          \usekomafont{localtocnumber}%
        }%
        {%
          \usekomafont{localtoctitle}%
          \etocname
        }%
        \nobreak%
        \leaders\hbox to .3cm{\hss.}\hfill%
        \nobreak%
        \makebox[1cm][r]{%
          \usekomafont{localtocpage}%
          \hss\etocpage%
        }\vskip0.5ex%
      }
      {}
%    \end{macrocode}
% Finally, we define the heading of the table of contents and typeset it.
%    \begin{macrocode}
      \etocsettocstyle{%
        \centerline{%
          \usekomafont{localtocheading}%
          \localtocheadingname%
        }
      }{}
    \localtableofcontents
    \parskip\parskip@old
  \endgroup
}
%    \end{macrocode}
%\end{macro}
%
%
%\begin{environment}{summary}
% Instead of using the |abstract| environment, we define a |summary| environment
% to be used for each talk.
%    \begin{macrocode}
\newenvironment{summary}
  {%
    \centerline{%
      \usekomafont{summaryheading}%
      \summaryname%
    }%
  \let\parindent@old\parindent\parindent0em%
  }{%
    \parindent\parindent@old\vskip3.5ex%
  }
%    \end{macrocode}
%\end{environment}
%
%\begin{macro}{\localbib}
%    \begin{macrocode}
\newcommand{\localbib}{%
  \printbibliography[heading=subbibliography]%
}
%    \end{macrocode}
%\end{macro}
%
%    \begin{macrocode}
%</class>
%    \end{macrocode}
%\Finale
