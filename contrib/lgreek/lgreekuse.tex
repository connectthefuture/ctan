\documentclass[a4paper]{article}
\usepackage[delims]{lgreek}
\usepackage{doc}
\MakeShortVerb\|
\newenvironment{TeXystuff}%
  {\small\renewcommand{\baselinestretch}{0.9}\smallskip}%
  {\par\smallskip}

\title{Using Greek Fonts with \LaTeX}
\author{Silvio Levy\\
  Princeton University \\
  Fine Hall, Washington Road\\
  Princeton, NJ, 08544\\
  (\texttt{levy@princeton.edu})\\[5pt]
  \emph{modified for \LaTeX2e by}\\[5pt]
  Timothy Murphy\\
  School of Mathematics\\
  Trinity College Dublin\\
  (\texttt{tim@maths.tcd.ie})}

\begin{document}

\maketitle

\begin{abstract}
In this document I hope to show that typesetting Greek in \LaTeX\
using the |lgreek| package (and the |gr| fonts)
can be as easy as typesetting English text, 
and leads to equally good results.
This is meant to be a tutorial, not an exhaustive discussion;
some \TeX nical remarks that should be useful after the reader
has acquired some familiarity with the fonts are printed in fine
print.
\end{abstract}

\section{The Alphabet}

In order to typeset Greek text, you need to go into ``Greek mode.''
This is achieved by typing |\begin{greek}| anywhere in your document;
Greek mode will remain in effect until you type a matching |\end{greek}|.
While in Greek mode, the letters `a' to `z' and `A' to `Z' come out
as Greek letters, according to the following code:

\begin{center}
\setlength{\tabcolsep}{3pt}
\begin{tabular}{cccccccccccccccccccccccc}
$a$&$b$&$g$&$d$&$e$&$z$&$h$&$j$&$i$&$k$&$l$&$m$&
$n$&$x$&$o$&$p$&$r$&$s$&$t$&$u$&$f$&$q$&$y$&$w$\\
|a|&|b|&|g|&|d|&|e|&|z|&|h|&|j|&|i|&|k|&|l|&|m|&
|n|&|x|&|o|&|p|&|r|&|s|&|t|&|u|&|f|&|q|&|y|&|w|
\end{tabular}
\end{center}

There is no digamma yet.  The same character `s' will print as
`$c$' or `$s$', depending on its position in a word.

\begin{TeXystuff}
The system does this by accessing a ligature of `s' with any other
letter that follows it.  If, for some reason, you want to print
an initial/medial sigma by itself (as in the table above), or
at the end of a word, you should type `c'.
\end{TeXystuff}

Try to typeset some simple text now.  Create a file
containing the following lines:

\begin{quote}
\begin{verbatim}
\documentclass{article}
\usepackage{lgreek}
\begin{document}
This is English text.
\begin{greek}
This is Greek text.
\end{greek}
\end{document}
\end{verbatim}
\end{quote}

When you \TeX\ this file, you get the following gibberish:
\begin{quote}
%\hbox{
This is English text.
\begin{greek}
This is Greek text.
\end{greek}
%}
\end{quote}

If you give the |delims| option for the package then
the character |$| can be used
in place of both |\begin{greek}| and |\end{greek}|,
as eg
\begin{quote}
\begin{verbatim}
This is English text.
$This is Greek text.$
\end{verbatim}
\end{quote}

The control sequences |\(|\dots|\)| are still available
for in-text math.

\section{Accents and Breathings}

To get an acute, grave or circumflex accent over a vowel,
type |'|, |`| or |~|, respectively, before the vowel.
To get a rough or smooth breathing, type |<| or |>| before
the vowel (or rho) and any accent that it may have.  To get an iota
subscript, type \verb"|" \emph{after} the vowel.  A diaeresis is
represented by |"|, and if accompanied by an accent it can come
before or after the accent.

For example, \hbox{\verb">en >arq\~h| >\~hn <o l'ogos"}
gives \hbox{$>en >arq~h| >~hn <o l'ogos$}.
Neat, ain't it?

\begin{TeXystuff}
Accents and breathings, too, are typeset by means of ligatures: a
vowel with a breathing, an accent and iota subscript, for example,
is realized as a four-character ligature.  The only exception is
when a breathing is followed by a grave accent, in which case the
breathing\({}+{}\)accent combination is typeset as a \TeX\ |\accent|
over the vowel.  This means that words containing such combinations
cannot be hyphenated in (standard) \TeX; but this is not a problem
because, with the exception of very rare cases of crasis, all such words
are monosyllables.
\end{TeXystuff}

\section{Punctuation}

Here's the table of correspondences for punctuation:

\begin{center}
\setlength{\tabcolsep}{4pt}
\begin{tabular}{ccccccccc}
$.$&$,$&$;$&$:$&$!$&$?$&$''$&$(($&$))$\\
|.|&|,|&|;|&|:|&|!|&|?|&|''|&|((|&|))|
\end{tabular}
\end{center}

The last three entries represent the apostrophe and quotations marks.
The other available non-letters are the ten digits, parentheses,
brackets, hyphen, em- and en-dashes, slash, percent sign, asterisk,
plus and equal signs.  All of these are accessible in the usual way.
In a future release there will be tick marks for
numbers (\(\hbox{$a$}'=1\), \({}_\prime\hbox{$a$}=1000\)).

\section{Hyphenation}

A hyphenation table for both modern and ancient Greek is currently
being debugged.  For now one can use the usual (English)
hyphenation table, which gives the right results about 90\% of the
time (amazing, isn't it?).  Be sure to proofread your text carefully,
unless you've turned hyphenation off.

\section{Interaction with other macros}

While in Greek mode you can do just about everything that you can
outside: go into math mode, create boxes, alignments, and so on.
The file |greekmacros.tex| sets things up so that in Greek mode
the control sequences |\tt| and |\bf| switch to a
typewriter and a bold Greek font, respectively: thus
|\texttt{s''>agap\~w}| gives \hbox{$\texttt{s''>agap~w}$}.  (Try it.)
On the other hand, there are no ``italic'' or slanted Greek fonts,
so |\it| and |\sl| will give you the same fonts as outside
Greek mode.  The various constructions under
%\AMSTeX\ and \LaTeX\ for increasing or decreasing point sizes don't
\LaTeX\ for increasing or decreasing point sizes don't
work yet; they will in a future release.

The characters that form diacritics (|<|, |>|,
|'|, |`|, |\~|, |"| and \verb"|") are treated differently
depending on whether or not you're in Greek mode.  More exactly,
under plain \TeX\ these characters (with the exception of
|\~|) have a |\catcode| of 12: they print as themselves, 
and they cannot appear in control words.  But in Greek mode
|'|, |`|, |\~|, |"| and \verb"|" are ``letters'', that is,
they have a |\catcode| of 11, while |<| and |>| are
active, with a |\catcode| of 13.  This may be important even
for beginners because it means that |'|, for example, can be taken
as part of a control word.  Thus the sequence
\begin{verbatim}
  \begin{greek}
  wm'ega\hfil'alfa
  \end{greek}
\end{verbatim}
\noindent
will cause an error message about an
undefined control sequence |\hfil'alfa|, instead of printing
\begin{quote}
\begin{greek}
wm'ega\hfil 'alfa
\end{greek}
\end{quote}
as you might expect.  (I hope classicists will forgive this use
of the modern Greek one-accent system.) The solution, of course,
is to remember to add a blank after the |\hfil|.

\begin{TeXystuff}
A more subtle problem arises when you use Greek text in macro
arguments, if the arguments are scanned while you're outside Greek
mode.  This is because \TeX\ assigns |\catcode|s to tokens
as it first reads them, so when the argument is plugged into
the body of the macro the characters above have the wrong 
|\catcode|.  If the legendary Jonathan Horatio Quick were
to write
\begin{quote}
\begin{verbatim}
\def\hellenize#1{\begin{greek}#1\end{greek}}
\hellenize{d'uo >`h tre~is,}
\end{verbatim}
\end{quote}
he would be unpleasantly surprised by the following output:
\begin{quote}
\def\hellenize#1{\begin{greek}#1\end{greek}}
\hellenize{d'uo >`h tre~is,}
\end{quote}
\enlargethispage{24pt}
which can be explained as follows: the |\~|, which should be
a letter, is seen as an active character, and expands to a blank
as in plain \TeX; while the breathing, which should be active,
is not, and in particular it doesn't do the right thing when
next to the grave accent.  Solutions to this problem require a
bit of wizardry, and will not be discussed here; see, for example,
Reinhard Wonneberger's article in the October, 1986 issue of
{\it TUGboat}, especially pages 179--180.
\end{TeXystuff}

\end{document}

