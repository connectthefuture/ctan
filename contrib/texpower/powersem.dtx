% \iffalse meta-comment
% --------------------------------------------------------------
% Part of the TeXPower bundle
% Copyright (C) 1999-2004 Stephan Lehmke
%
% This program is free software; you can redistribute it and/or
% modify it under the terms of the GNU General Public License
% as published by the Free Software Foundation; either version 2
% of the License, or (at your option) any later version.
%
% This program is distributed in the hope that it will be useful,
% but WITHOUT ANY WARRANTY; without even the implied warranty of
% MERCHANTABILITY or FITNESS FOR A PARTICULAR PURPOSE.  See the
% GNU General Public License for more details.
% --------------------------------------------------------------
%
% powersem.dtx,v 1.3 2005/03/28 22:31:01 hansfn Exp
%
% \fi
%
% \iffalse
%<*driver>
\ProvidesFile{powersem.dtx}
%</driver>
%<powersem>\NeedsTeXFormat{LaTeX2e}
%<powersem>\ProvidesClass{powersem}
%<*powersem>
    [2004/07/27 v0.8a Create online Presentations with seminar.]
%</powersem>
%
%<*driver>
\documentclass{ltxdoc}
\EnableCrossrefs
\CodelineIndex
\RecordChanges
\begin{document}
  \DocInput{powersem.dtx}
\end{document}
%</driver>
%
% \fi
%
% \CheckSum{366}
%
% \CharacterTable
%  {Upper-case    \A\B\C\D\E\F\G\H\I\J\K\L\M\N\O\P\Q\R\S\T\U\V\W\X\Y\Z
%   Lower-case    \a\b\c\d\e\f\g\h\i\j\k\l\m\n\o\p\q\r\s\t\u\v\w\x\y\z
%   Digits        \0\1\2\3\4\5\6\7\8\9
%   Exclamation   \!     Double quote  \"     Hash (number) \#
%   Dollar        \$     Percent       \%     Ampersand     \&
%   Acute accent  \'     Left paren    \(     Right paren   \)
%   Asterisk      \*     Plus          \+     Comma         \,
%   Minus         \-     Point         \.     Solidus       \/
%   Colon         \:     Semicolon     \;     Less than     \<
%   Equals        \=     Greater than  \>     Question mark \?
%   Commercial at \@     Left bracket  \[     Backslash     \\
%   Right bracket \]     Circumflex    \^     Underscore    \_
%   Grave accent  \`     Left brace    \{     Vertical bar  \|
%   Right brace   \}     Tilde         \~}
%
%
% \GetFileInfo{powersem.dtx}
%
% \DoNotIndex{\newcommand,\newenvironment}
%
% \title{The \textsf{powersem} class\thanks{This document
%   corresponds to \textsf{powersem}~\fileversion, dated \filedate.}}
% \author{ Stephan Lehmke \\ \texttt{Stephan.Lehmke@cs.uni-dortmund.de}}
%
% \maketitle
%
% \tableofcontents
%
% \section{Introduction}
%
% The user documentation is found in \texttt{manual.tex} and the FAQ.
% Only the implementation documentation is covered in this document.
%
% Make seminar ready for the Third Millennium ;-)
% This class loads seminar and tries to fix some problems which occur when trying to create dynamic presentations with
% the texpower package. 
%
% Some new features helpful for dynamic presentations are also added.
%
% This class is part of the TeXPower bundle, to be found at\\ % |http://texpower.sourceforge.net/|
%
% The TeXPower bundle also contains a package \texttt{fixseminar.sty} which applies some more fixes, in particular for
% hyperref. These can not be applied here because they have to be made after hyperref is loaded.
%
% \StopEventually{\PrintChanges\PrintIndex}
%
% \section{Implementation}
%
% \changes{v0.1}{1999/10/18}{First version. So far it doesn't do much more than load seminar.}
%
% \changes{v0.2}{2000/03/28}{Force seminar to treat \cs{paperwidth} and \cs{paperheight} in a
% sensible manner.}
%
% \changes{v0.3}{2000/05/10}{Added option `calcdimensions'.}
%
% \changes{v0.4}{2000/06/16}{Now separates text from footnotes when option `display' is given.}
%
% \changes{v0.5}{2000/06/26}{\cs{@colht} is ignored by seminar, which is bad. Added a patch to
% \cs{set@slidesize}.  calcdimensions now handles slide frames.}
%
% \changes{v0.5a}{2000/07/03}{The footnote patch with option display effectively disabled setting
% \cs{centerslidestrue}. Fixed.}
%
% \changes{v0.6}{2001/11/10}{seminar's \cs{output@slide} kernel copied (from seminar.bg2) and
% modified for better color handling and to remove some overfull \cs{vbox} warnings.  There was
% another underfull vbox warning lurking in the footnote patch. Removed.  New option truepagenumbers
% for cooperation with texpower in case `fragmented' slide numbers (like 3.5.1) are used.  Make
% seminar's form of raggedright lines (controlled by \cs{raggedslides}) active in parboxes and
% minipages (and p columns in tables or arrays).}
%
% \changes{v0.7}{2002/11/22}{Inserted \cs{nointerlineskip} in page building code to heal a slight
% problem with vertical spacing.  A small modification was neccessary wrt the KOMA option because
% with fixseminar v0.0.4, pdflatex slides can now be rotated.}
%
% \changes{v0.8}{2003/05/06}{Some fixes provided by Pascal Kockaert to (hopefully) get better
% behaviour with seminar's ``article'' option - many thanks!}
%
% \changes{v0.8a}{2004/07/27}{Moved to dtx format. No other code changes.}
%
%    \begin{macrocode}
\RequirePackage{ifthen}
%    \end{macrocode}
%
% The display option is understood by all relevant classes and packages from the TeXPower bundle. It means that
% `dynamic' features are to be turned on.
% There is a boolean register display (as defined in ifthen.sty) which is set by this option and can be used to
% differentiate between slides for display and slides for printout.
% Notes are disabled automatically.
%    \begin{macrocode}
\newboolean{display}
\DeclareOption{display}
{\setboolean{display}{true}\PassOptionsToClass{slidesonly}{seminar}}
%    \end{macrocode}
%
% The truepagenumbers option makes powersem count pages with the counter page, independently of the counter slide. This
% enables proper working of TeXPowers navigation buttons (some of which calculate relative page numbers) even when the
% counter slide is reset frequently (for slide numberings of the type |<l>.<n>.<m>|).
%    \begin{macrocode}
\newboolean{truepn@PS}
\DeclareOption{truepagenumbers}{\setboolean{truepn@PS}{true}}
%    \end{macrocode}
%
% The KOMA option tells powersem to load scrartcl (from the KOMA suite of document classes) instead of article.
%    \begin{macrocode}
\newboolean{BaseClass@PS}
\DeclareOption{KOMA}
{\setboolean{BaseClass@PS}{true}\providecommand{\baseclass}{scrartcl}}
%    \end{macrocode}
%
% The UseBaseClass option is more neutral: It tells powersem to load the class |\baseclass| (initially article) instead of
% article. 
%    \begin{macrocode}
\DeclareOption{UseBaseClass}
{\setboolean{BaseClass@PS}{true}\providecommand{\baseclass}{article}}
%    \end{macrocode}
%
% The reportclass option tells powersem to load the class |\baseclass| (initially report) instead of
% article. 
%    \begin{macrocode}
\DeclareOption{reportclass}
{\setboolean{BaseClass@PS}{true}\providecommand{\baseclass}{report}}
%    \end{macrocode}
%
% The bookclass option tells powersem to load the class |\baseclass| (initially report) instead of
% article. 
%    \begin{macrocode}
\DeclareOption{bookclass}
{\setboolean{BaseClass@PS}{true}\providecommand{\baseclass}{book}}
%    \end{macrocode}
%
% We need to redefine the a4paper option which is broken in seminar.cls
%    \begin{macrocode}
\DeclareOption{a4paper}{\PassOptionsToClass{a4}{seminar}}
%    \end{macrocode}
%
% The calcdimensions option tells powersem to automatically calculate slidewidth and slideheight from paper dimensions
% and margins.
%    \begin{macrocode}
\newboolean{calcdim@PS}
\DeclareOption{calcdimensions}{\setboolean{calcdim@PS}{true}}
%    \end{macrocode}
%
%    \begin{macrocode}
\DeclareOption*{\PassOptionsToClass{\CurrentOption}{seminar}}

\ProcessOptions
%    \end{macrocode}
%
% Now we need to hack a little to make seminar load |\baseclass| instead of article (if one of the respectice options was
% given). 
%    \begin{macrocode}
\let\oldLC@PS=\LoadClass

\ifthenelse{\boolean{BaseClass@PS}}
{%
  \def\article@PS{article}
%    \end{macrocode}
% Make |\PassOptionsToClass| and |\LoadClass| use |\baseclass|...
%    \begin{macrocode}
  \let\oldPOTC@PS=\PassOptionsToClass
  \def\PassOptionsToClass#1#2%
  {\edef\cls@PS{#2}%
   \ifx\article@PS\cls@PS\oldPOTC@PS{#1}{\baseclass}%
   \else\oldPOTC@PS{#1}{#2}%
   \fi}
%    \end{macrocode}
% Furthermore, seminar redefines |\paperheight| and |\paperwidth|, saving their value in |\sem@paperheight| and
% |\sem@paperwidth|. We undo this so the paper dimension calculation of typearea can be used.
%    \begin{macrocode}
  \renewcommand{\LoadClass}[2][]%
  {%
    \edef\cls@PS{#2}%
    \ifx\article@PS\cls@PS
      \let\paperheight\sem@paperheight\let\paperwidth\sem@paperwidth
      \oldLC@PS[#1]{\baseclass}%
      \ifarticle\else\portraittrue\fi
     \else
      \oldLC@PS[#1]{#2}%
    \fi
  }%
} 
{}
%    \end{macrocode}
%
% Finally, the seminar class and some bug fixes are loaded.
%    \begin{macrocode}
\oldLC@PS{seminar}

\AtBeginDocument
{%
%% BEGIN seminar.cls
%% LaTeX2e seminar class, derived by Sebastian Rahtz from
%% LaTeX document style `seminar', for use with LaTeX v2.09.
%% generated on <1993/10/8>. 
%% Fixes from David Carlisle, Denis Girou, 
%% and Frank Küster/Günter Milde (a4paper option fix)
%%
\def\fileversion{1.61}
\def\filedate{2014/01/17}
%%
%% This is a style for typesetting notes and slides.
%%
%% COPYRIGHT 1993, by Timothy Van Zandt, Timothy.VAN-ZANDT@insead.edu
%%
%% This file may be distributed and/or modified under the conditions of
%% the LaTeX Project Public License, either version 1.2 of this license
%% or (at your option) any later version.  The latest version of this
%% license is in:
%% 
%%    http://www.latex-project.org/lppl.txt
%% 
%% and version 1.2 or later is part of all distributions of LaTeX version
%% 1999/12/01 or later.
%%
%%
\NeedsTeXFormat{LaTeX2e}
\ProvidesClass{seminar}[\filedate, \fileversion]
\typeout{%
  Documentclass: `seminar' v\fileversion \space <\filedate> (tvz,hv)}
%%%%%%%%%%%%%%%% fix hv 
\def\tx@BOL{}
%%%%%%%%%%%%%%%
\def\test@member#1#2{%
  \edef\@tempg{,#2,#1,}%
  \edef\@temph{####1,#1,}%
  \expandafter\def\expandafter\@temph\@temph##2\@nil{%
    \def\@tempg{##2}%
    \ifx\@tempg\@empty\@testfalse\else\@testtrue\fi}%
  \expandafter\@temph\@tempg\@nil}
\def\addto@hook#1#2{#1\expandafter{\the#1#2}}
\@ifundefined{reset@font}{\def\reset@font{\normalsize\rm}}{}
\def\@seminarerr#1#2{%
  \edef\@tempc{#2}\expandafter\errhelp\expandafter{\@tempc}%
  \typeout{^^JSeminar.sty error.\space\space\space
    Type \space H <return> \space for immediate help.^^J}%
  \errmessage{#1^^J}}
\def\notslide@err#1{\message{Cannot use \string#1 in slide environments}}
\def\new@slidebox{\alloc@4\box\chardef\insc@unt}
\RequirePackage{pst-ovl}
%%
%\RequirePackage{geometry}
%\geometry{dvips,paperwidth=8.5in,paperheight=11in}
\newdimen\slidewidth \slidewidth 8.5in
\newdimen\slideheight \slideheight 6.3in
\paperwidth=8.5in
\paperheight=11in
%%
\def\addto@preamblecmds#1{%
  \begingroup
    \def\do{\noexpand\do\noexpand}%
    \xdef\@preamblecmds{\@preamblecmds\do#1}%
  \endgroup}
\addto@preamblecmds{\addto@preamblecmds\do\@preamblecmds}
\def\sem@temp#1{\@ifundefined{if#1}%
  {\def\@tempa{\csname newif\endcsname}
  \expandafter\@tempa\csname if#1\endcsname}{}}
\sem@temp{article}{}
\sem@temp{slidesonly}{}
\sem@temp{notes}{}
\sem@temp{notesonly}{}
\sem@temp{notesonlystar}{}
\sem@temp{portrait}{}
\DeclareOption{article}{\articletrue}
\DeclareOption{noxcomment}{\def\xcomment@hook{}}
\DeclareOption{slidesonly}{\slidesonlytrue\notesfalse\notesonlyfalse}
\DeclareOption{notes}{\notestrue\slidesonlyfalse\notesonlyfalse}
\DeclareOption{notesonly}{\notesonlytrue\slidesonlyfalse\notesfalse}
\DeclareOption{notesonly*}{%
  \notesonlytrue\slidesonlyfalse\notesfalse\notesonlystartrue}
\DeclareOption{portrait}{\portraittrue}
\DeclareOption{a4}{%
  \paperwidth 210mm
  \paperheight 297mm
%  \geometry{paperwidth=210mm,paperheight=297mm}
  \slidewidth 222mm
  \slideheight 152mm\relax}
\DeclareOption{a4paper}{\csname ds@a4\endcsname}%
\DeclareOption{semrot}{\AtEndOfClass{%% BEGIN sem-rot.sty
\def\fileversion{1.0}
\def\filedate{93/04/01}
%%
%% LaTeX style option `semrot.sty'.
%% For use with LaTeX v2.09 and the `seminar' document style.
%%
%% COPYRIGHT 1993, by Timothy Van Zandt, Timothy.VAN-ZANDT@insead.edu
%%
%% This file may be distributed and/or modified under the conditions of
%% the LaTeX Project Public License, either version 1.2 of this license
%% or (at your option) any later version.  The latest version of this
%% license is in:
%% 
%%    http://www.latex-project.org/lppl.txt
%% 
%% and version 1.2 or later is part of all distributions of LaTeX version
%% 1999/12/01 or later.
%%
%%
%% The following is just an interface for seminar.sty:
\def\leftsliderotation#1{\@rotate[l]{#1}}
\def\rightsliderotation#1{\@rotate[r]{#1}}
\sliderotation{left}
%%
%% And the rest is rotate.sty, as distributed with Rokicki's dvips.
%%
%   These macros allow you to rotate or flip a \TeX\ box.  Very useful for
%   sideways tables or upsidedown answers.
%
%   To use, create a box containing the information you want to rotate.
%   (An hbox or vbox will do.)  Now call \@rotr\boxnum to rotate the
%   material and create a new box with the appropriate (flipped) dimensions.
%   \@rotr rotates right, \@rotl rotates left, \@rotu turns upside down, and
%   \@rotf flips.  These boxes may contain other rotated boxes.
%
\newdimen\@rotdimen
\newbox\@rotbox

\def\@vspec#1{\special{ps:#1}}%  passes #1 verbatim to the output
\def\@rotstart#1{\@vspec{gsave currentpoint currentpoint translate
   #1 neg exch neg exch translate}}% #1 can be any origin-fixing transformation
\def\@rotfinish{\@vspec{currentpoint grestore moveto}}% gets back in synch
%
%   First, the rotation right. The reference point of the rotated box
%   is the lower right corner of the original box.
%
\def\@rotr#1{\@rotdimen=\ht#1\advance\@rotdimen by\dp#1%
   \hbox to\@rotdimen{\hskip\ht#1\vbox to\wd#1{\@rotstart{90 rotate}%
   \box#1\vss}\hss}\@rotfinish}
%
%   Next, the rotation left. The reference point of the rotated box
%   is the upper left corner of the original box.
%
\def\@rotl#1{\@rotdimen=\ht#1\advance\@rotdimen by\dp#1%
   \hbox to\@rotdimen{\vbox to\wd#1{\vskip\wd#1\@rotstart{270 rotate}%
   \box#1\vss}\hss}\@rotfinish}%
%
%   Upside down is simple. The reference point of the rotated box
%   is the upper right corner of the original box. (The box's height
%   should be the current font's xheight, \fontdimen5\font,
%   if you want that xheight to be at the baseline after rotation.)
%
\def\@rotu#1{\@rotdimen=\ht#1\advance\@rotdimen by\dp#1%
   \hbox to\wd#1{\hskip\wd#1\vbox to\@rotdimen{\vskip\@rotdimen
   \@rotstart{-1 dup scale}\box#1\vss}\hss}\@rotfinish}%
%
%   And flipped end for end is pretty ysae too. We retain the baseline.
%
\def\@rotf#1{\hbox to\wd#1{\hskip\wd#1\@rotstart{-1 1 scale}%
   \box#1\hss}\@rotfinish}%
%
%   Now the user interface for LaTeX:  \rotate[type]{text} where
%   ``type'' is `l' for left, `r' for right, `u' for upside-down, or
%   `f' for flip.  `l' is the default.
%
\def\rotate{\@ifnextchar[{\@rotate}{\@rotate[l]}}
\def\@rotate[#1]#2{\setbox\@rotbox=\hbox{#2}\@nameuse{@rot#1}\@rotbox}
\endinput
%% END sem-rot.sty
}}
\DeclareOption{semhelv}{\AtEndOfClass{%% BEGIN semhelv.sty
\def\fileversion{1.4}
\def\filedate{2000/05/08}
%%
%% LaTeX style option `semhelv.sty' for the `seminar' document class.
%% For use with LaTeX2e.
%%
%% COPYRIGHT 1994, by Timothy Van Zandt, Timothy.VAN-ZANDT@insead.edu
%%
%% This file may be distributed and/or modified under the conditions of
%% the LaTeX Project Public License, either version 1.2 of this license
%% or (at your option) any later version.  The latest version of this
%% license is in:
%% 
%%    http://www.latex-project.org/lppl.txt
%% 
%% and version 1.2 or later is part of all distributions of LaTeX version
%% 1999/12/01 or later.
%%
%% Edited by Sebastian Rahtz <sebastian.rahtz@oucs.ox.ac.uk>
%%   and Denis Girou <denis.girou@idris.fr>
%% To use OT1-encoding 7t names
%% ************************************************************************
%% ****                        DESCRIPTION                             ****
%% ************************************************************************
%%
% This file contains the external font information needed to load
% the PostScript Helvetica fonts for use by seminar.cls.
%
% IMPORTANT:
%   As distributed, the following font file names are assumed for the
%   `standard' TeX encoding:
%
%     Helvetica              phvr7t
%     Helvetica-Oblique      phvro7t
%     Helvetica-Bold         phvb7t
%     Helvetica-BoldOblique  phvbo7t
%     Courier                pcrr7t
%     Courier-Bold           pcrb7t
%     Courier-Oblique        pcrro7t
%     Courier-BoldOblique    pcrbo7t
%
%   With the NFSS2 and LaTeX2e, you can also use the CORK encoding. The
%   assumed file names are as above, but with `8t' instead of `7t'
%
%   This means, e.g., that phvr7t.tfm or phvr7t.vf must be the name of the
%   TeX .tfm or .vf (virtual font) file for the PostScript Helvetica font.
%   See the documentation of your dvi-to-ps driver for information on using
%   PostScript fonts.
%
% Fonts are scaled so that the x-heights match those of the cmr fonts;
% this makes in-line math look acceptable.
%
% Note: The Courier font is not that great.
%
% Preliminaries:
\csname semhelv@loaded\endcsname
\let\semhelv@loaded\endinput
\typeout{Style Option: `semhelv' for the `seminar' doc style
  \fileversion\space <\filedate> (tvz)}
\typeout{Modified by Stefan (see source for details)}
\@ifundefined{selectfont}
  {\@seminarerr{`semhelv' style option can only be used
    with the New Font Selection Scheme}\@eha\endinput}{}
%
% Font definitions.

\expandafter\ifx\csname DeclareSizeFunction\endcsname\relax

% FOR NFSS:

\@ifundefined{newpsfamily}{%% BEGIN npsfont.sty
%%
\def\fileversion{v1.0}
\def\filedate{91/11/20}
%%
%% Style option npsfont.sty, for use with LaTeX 2.09
%%   and the New Font Selection Scheme.
%%
%% COPYRIGHT 1992, by Timothy Van Zandt, Timothy.VAN-ZANDT@insead.edu
%%
%%
%% This file may be distributed and/or modified under the conditions of
%% the LaTeX Project Public License, either version 1.2 of this license
%% or (at your option) any later version.  The latest version of this
%% license is in:
%% 
%%    http://www.latex-project.org/lppl.txt
%% 
%% and version 1.2 or later is part of all distributions of LaTeX version
%% 1999/12/01 or later.
%%
%%
% DESCRIPTION:
%
% Defining new PostScript fonts with the New Font Selection Scheme is easy,
% but these macros make it easier. They save on keystrokes, and most
% importantly, they make it easy to change the file names used for fonts
% because you only have to type the file name once instead of 12 times (for
% the 12 standard LaTeX sizes). An optional argument lets you scale the fonts.
%
% These macros are substitutes for the NFSS's \new@fontshape command. For
% information on the purpose of this command, and on how to select fonts that
% are defined with it, see the documentation of the NFSS. If you don't have
% the NFSS, it is available from
%   rusmv1.rus.uni-stuttgart.de
% in
%   soft/tex/latex-style-supported/New-Font-Selection
% and from
%   ymir.claremont.edu
% in
%   [.anonymous.tex.inputs.latex-mainz]
% The NFSS is undoubtably the best way to use PostScript fonts (or any non-
% standard fonts) with LaTeX, and is well worth upgrading to.
%
% USAGE:
%
%   \newpsfont[<scaling>]{<family>}{<series>}{<shape>}{<filename>}
%
%   \newpsfamily[<scaling>]{<family>}{
%     {<series 1>}{<shape 1>}{<filename 1>},
%     {<series 2>}{<shape 2>}{<filename 2>},
%                 . . .
%     {<series n>}{<shape n>}{<filename n>}}
%
% The scaling argument is optional. Like the \new@fontshape command, these
% commands can only be used in the preamble.
%
% EXAMPLE:
%
% Suppose the Helvetica-Oblique virtual font on your system has the name
% phvro.vf. The following defines the <helvetica> family, <m>edium series, and
% <it>alic shape to be the Helvetica-Oblique font, at the standard LaTeX
% font sizes, scaled by .82:
%
%   \newpsfont[.82]{helvetica}{m}{it}{phvro}
%
% This is equivalent to:
%
%   \new@fontshape{helvetica}{m}{it}{%
%     <5>phvro at 4.1pt%
%     <6>phvro at 4.92pt%
%     <7>phvro at 5.74pt%
%     <8>phvro at 6.56pt%
%     <9>phvro at 7.38pt%
%     <10>phvro at 8.2pt%
%     <11>phvro at 8.979pt%
%     <12>phvro at 9.84pt%
%     <14>phvro at 11.808pt%
%     <17>phvro at 14.1696pt%
%     <20>phvro at 17.0068pt%
%     <25>phvro at 20.4016pt}{}
%
% The following command sets up the standard Helvetica family:
%
%   \newpsfamily[.82]{helvetica}{%
%     {m}{n}{phvr},                          %Helvetica
%     {m}{it}{phvro},                        %Helvetica-Oblique
%     {m}{sl}{phvro},                        %Helvetica-Oblique
%     {bx}{n}{phvb},                         %Helvetica-Bold
%     {bx}{it}{phvbo},                       %Helvetica-BoldOblique
%     {bx}{sl}{phvbo}}                       %Helvetica-BoldOblique
%
% (Don't forget the commas.) It is equivalent to:
%
%   \newpsfont[.82]{helvetica}{m}{n}{phvr}     %Helvetica
%   \newpsfont[.82]{helvetica}{m}{it}{phvro}   %Helvetica-Oblique
%   \newpsfont[.82]{helvetica}{m}{sl}{phvro}   %Helvetica-Oblique
%   \newpsfont[.82]{helvetica}{bx}{n}{phvb}    %Helvetica-Bold
%   \newpsfont[.82]{helvetica}{bx}{it}{phvbo}  %Helvetica-BoldOblique
%   \newpsfont[.82]{helvetica}{bx}{sl}{phvbo}  %Helvetica-BoldOblique
%   \extra@def{helvetica}{}{}
%
% WHY SCALE FONTS?:
%
% Disclaimer: The suggestions that follow come from someone who doesn't know
% anything about fonts.
%
% Mixing fonts that are not designed to be used together is not for purists,
% but many of us do it anyway. Scaling the fonts can at least correct for
% differences in the x-heights. This is important if you use Computer
% Modern fonts for in-line math and some other fonts for text.
%
% To find out the x-height of the 10pt size of a font, load this style and
% use the command
%
%   \showxheight{<filename>}
%
% E.g., \showxheight{phvr} and \showxheight{cmr10} displays
%
%   x-height of cmr10 at 10pt is 4.30554pt
%   x-height of phvr at 10pt is 5.24998pt
%
% on the terminal. Scaling the Helvetica fonts by 4.30554/5.24990 = .82
% gets the x-heights of these fonts in line. It is best to set the x-heights
% the same for the medium weight, normal shaped fonts, and then scale the
% rest of the fonts in the same family by the same amount.
%
% David Jones has a psfonts.sty that sets up the standard PostScript fonts,
% all scaled so that the x-heights match that of the Computer Modern medium
% roman font. It is available via anonymous ftp from theory.lcs.mit.edu in
% /pub/dmjones.
%
% IMPLEMENTATION:
%
% \scale@stdsizes has at its argument a scaling factor. It uses \font@scale
% to define \scaled@sizes to be the macro whose argument is the name of a
% font, and which expands to the definitions of the standard \LaTeX{} sizes
% for New Font Selection Schemes \new@fontshape command.
%
\message{\space\space\fileversion\space\space\filedate
  \space\space<tvz>}
\@ifundefined{new@psfontshape}{}{\endinput}
\def\font@scale#1#2#3{\dimen@=#2\dimen@=#3\dimen@
  \edef\scaled@sizes##1{\scaled@sizes{##1}<#1>##1 at\number\dimen@ sp}}
\def\scale@stdsizes#1{%
  \edef\scaled@sizes##1{}%
  \font@scale{5}{5pt}{#1}%
  \font@scale{6}{6pt}{#1}%
  \font@scale{7}{7pt}{#1}%
  \font@scale{8}{8pt}{#1}%
  \font@scale{9}{9pt}{#1}%
  \font@scale{10}{10pt}{#1}%
  \font@scale{11}{10.95pt}{#1}%
  \font@scale{12}{12pt}{#1}%
  \font@scale{14}{14.4pt}{#1}%
  \font@scale{17}{17.28pt}{#1}%
  \font@scale{20}{20.74pt}{#1}%
  \font@scale{25}{24.88pt}{#1}}
\def\new@psfontshape#1#2#3#4{%
  \new@fontshape{#1}{#2}{#3}{\scaled@sizes{#4}}{}}
\def\new@psfont[#1]#2#3#4#5{%
  \scale@stdsizes{#1}%
  \new@psfontshape{#2}{#3}{#4}{#5}}
\def\newpsfont{%
  \@ifnextchar [{\new@psfont}{\new@psfont[1]}}
\def\newpsfamily{%
  \@ifnextchar [{\new@psfamily}{\new@psfamily[1]}}
\def\new@psfamily[#1]#2#3{%
  \scale@stdsizes{#1}%
  \@for\@tempa:=#3\do{%
    \edef\@tempb{\noexpand\new@psfontshape{#2}\@tempa}
    \@tempb}%
  \extra@def{#2}{}{}}
\def\showxheight#1{%
  \font\@tempfont=#1 at 10pt%
  \typeout{^^J x-height of #1 at 10pt is \the\fontdimen5\@tempfont}}
% The commands are added to \@preamblecmds to save memory.
\begingroup
  \def\do{\noexpand\do\noexpand}
  \xdef\@preamblecmds{\@preamblecmds
    \do\font@scale
    \do\scale@stdsizes
    \do\scaled@sizes
    \do\new@psfontshape
    \do\new@psfont
    \do\newpsfont
    \do\newpsfamily
    \do\new@psfamily}
\endgroup
\endinput
%% END npsfont.sty
}{}%

% See npsfont.sty for the definition of \newpsfamily.

\newpsfamily[.82]{helvetica}{
  {m}{n}{phvr7t},          %Helvetica
  {m}{it}{phvro7t},        %Helvetica-Oblique
  {m}{sl}{phvro7t},        %Helvetica-Oblique
  {bx}{n}{phvb7t},         %Helvetica-Bold
  {bx}{it}{phvbo7t},       %Helvetica-BoldOblique
  {bx}{sl}{phvbo7t}}       %Helvetica-BoldOblique
\newpsfamily[.985]{courier}{
  {m}{n}{pcrr7t},          %Courier
  {bx}{n}{pcrb7t},         %Courier-Bold
  {m}{it}{pcrro7t},        %Courier-Oblique
  {bx}{it}{pcrbo7t}}       %Courier-BoldOblique

% Use fonts or use \slidefonts:
\ifarticle
  \renewcommand{\slidefonts}{%
    \renewcommand{\default@family}{helvetica}%
    \renewcommand{\rmdefault}{helvetica}%
    \renewcommand{\ttdefault}{courier}%
    \renewcommand{\sfdefault}{helvetica}%
    \family\default@family\selectfont}
\else
  \renewcommand{\default@family}{helvetica}%
  \renewcommand{\rmdefault}{helvetica}%
  \renewcommand{\ttdefault}{courier}%
  \renewcommand{\sfdefault}{helvetica}%
  \family\default@family\selectfont
\fi

\else

% FOR NFSS2 OR LATEX2E:

\def\helv@scale{.82}

\DeclareFontFamily{T1}{helvetica}{}%
\DeclareFontShape{T1}{helvetica}{m}{n}{<->s*[\helv@scale]phvr8t}{}%
\DeclareFontShape{T1}{helvetica}{m}{it}{<->s*[\helv@scale]phvro8t}{}%
\DeclareFontShape{T1}{helvetica}{m}{sc}{<->s*[\helv@scale]phvrc8t}{}%
\DeclareFontShape{T1}{helvetica}{b}{n}{<->s*[\helv@scale]phvb8t}{}%
\DeclareFontShape{T1}{helvetica}{b}{it}{<->s*[\helv@scale]phvbo8t}{}%
\DeclareFontShape{T1}{helvetica}{m}{sl}{<->ssub * helvetica/m/it}{}%
\DeclareFontShape{T1}{helvetica}{b}{sc}{<->sub * helvetica/m/sc}{}%
\DeclareFontShape{T1}{helvetica}{b}{sl}{<->ssub * helvetica/b/it}{}%
\DeclareFontShape{T1}{helvetica}{bx}{n}{<->ssub * helvetica/b/n}{}%
\DeclareFontShape{T1}{helvetica}{bx}{it}{<->ssub * helvetica/b/it}{}%
\DeclareFontShape{T1}{helvetica}{bx}{sc}{<->sub * helvetica/m/sc}{}%
\DeclareFontShape{T1}{helvetica}{bx}{sl}{<->ssub * helvetica/b/it}{}%

\DeclareFontFamily{OT1}{helvetica}{}%
\DeclareFontShape{OT1}{helvetica}{m}{n}{<->s*[\helv@scale]phvr7t}{}%
\DeclareFontShape{OT1}{helvetica}{m}{it}{<->s*[\helv@scale]phvro7t}{}%
\DeclareFontShape{OT1}{helvetica}{m}{sc}{<->s*[\helv@scale]phvrc7t}{}%
\DeclareFontShape{OT1}{helvetica}{b}{n}{<->s*[\helv@scale]phvb7t}{}%
\DeclareFontShape{OT1}{helvetica}{b}{it}{<->s*[\helv@scale]phvbo7t}{}%
\DeclareFontShape{OT1}{helvetica}{m}{sl}{<->ssub * helvetica/m/it}{}%
\DeclareFontShape{OT1}{helvetica}{b}{sc}{<->sub * helvetica/m/sc}{}%
\DeclareFontShape{OT1}{helvetica}{b}{sl}{<->ssub * helvetica/b/it}{}%
\DeclareFontShape{OT1}{helvetica}{bx}{n}{<->ssub * helvetica/b/n}{}%
\DeclareFontShape{OT1}{helvetica}{bx}{it}{<->ssub * helvetica/b/it}{}%
\DeclareFontShape{OT1}{helvetica}{bx}{sc}{<->sub * helvetica/m/sc}{}%
\DeclareFontShape{OT1}{helvetica}{bx}{sl}{<->ssub * helvetica/b/it}{}%

\def\courier@scale{.985}

% DG/SR modification begin - May. 13, 1998
% Update some names (phvrq -> phvr7t etc.)
% DG/SR modification end

\DeclareFontFamily{T1}{courier}{\hyphenchar \font\m@ne}%
\DeclareFontShape{T1}{courier}{m}{n}{<->s*[\courier@scale]pcrr8t}{}%
\DeclareFontShape{T1}{courier}{m}{it}{<->s*[\courier@scale]pcrro8t}{}%
\DeclareFontShape{T1}{courier}{m}{sc}{<->s*[\courier@scale]pcrrc8t}{}%
\DeclareFontShape{T1}{courier}{b}{n}{<->s*[\courier@scale]pcrb8t}{}%
\DeclareFontShape{T1}{courier}{b}{it}{<->s*[\courier@scale]pcrbo8t}{}%
\DeclareFontShape{T1}{courier}{m}{sl}{<->ssub * courier/m/it}{}%
\DeclareFontShape{T1}{courier}{b}{sc}{<->sub * courier/m/sc}{}%
\DeclareFontShape{T1}{courier}{b}{sl}{<->ssub * courier/b/it}{}%
\DeclareFontShape{T1}{courier}{bx}{n}{<->ssub * courier/b/n}{}%
\DeclareFontShape{T1}{courier}{bx}{it}{<->ssub * courier/b/it}{}%
\DeclareFontShape{T1}{courier}{bx}{sc}{<->sub * courier/m/sc}{}%
\DeclareFontShape{T1}{courier}{bx}{sl}{<->ssub * courier/b/it}{}%

\DeclareFontFamily{OT1}{courier}{\hyphenchar \font\m@ne}%
\DeclareFontShape{OT1}{courier}{m}{n}{<->s*[\courier@scale]pcrr7t}{}%
\DeclareFontShape{OT1}{courier}{m}{it}{<->s*[\courier@scale]pcrro7t}{}%
\DeclareFontShape{OT1}{courier}{m}{sc}{<->s*[\courier@scale]pcrrc7t}{}%
\DeclareFontShape{OT1}{courier}{b}{n}{<->s*[\courier@scale]pcrb7t}{}%
\DeclareFontShape{OT1}{courier}{b}{it}{<->s*[\courier@scale]pcrbo7t}{}%
\DeclareFontShape{OT1}{courier}{m}{sl}{<->ssub * courier/m/it}{}%
\DeclareFontShape{OT1}{courier}{b}{sc}{<->sub * courier/m/sc}{}%
\DeclareFontShape{OT1}{courier}{b}{sl}{<->ssub * courier/b/it}{}%
\DeclareFontShape{OT1}{courier}{bx}{n}{<->ssub * courier/b/n}{}%
\DeclareFontShape{OT1}{courier}{bx}{it}{<->ssub * courier/b/it}{}%
\DeclareFontShape{OT1}{courier}{bx}{sc}{<->sub * courier/m/sc}{}%
\DeclareFontShape{OT1}{courier}{bx}{sl}{<->ssub * courier/b/it}{}%

% Use fonts or use \slidefonts:
\ifarticle
  \renewcommand{\slidefonts}{%
    \renewcommand{\default@family}{helvetica}%
    \renewcommand{\rmdefault}{helvetica}%
    \renewcommand{\ttdefault}{courier}%
    \renewcommand{\sfdefault}{helvetica}%
    \fontfamily\default@family\selectfont}
\else
  \renewcommand{\default@family}{helvetica}%
  \renewcommand{\rmdefault}{helvetica}%
  \renewcommand{\ttdefault}{courier}%
  \renewcommand{\sfdefault}{helvetica}%
  \fontfamily\default@family\selectfont
\fi

\fi

\endinput
%% END semhelv.sty

}}
\DeclareOption{semcolor}{\AtEndOfClass{%%
%% This is file `semcolor.sty', version 1.01
%%
%% IMPORTANT NOTICE:
%%
%% semcolor.sty Copyright (C) 1993 Timothy Van Zandt
%%
%% This package may be distributed under the terms of the LaTeX Project 
%% Public License, as described in lppl.txt in the base LaTeX distribution.
%% Either version 1.0 or, at your option, any later version.
%%
\def\fileversion{1.02}
\def\filedate{2014/01/17}
%%
%% LaTeX style option `semcolor.sty'.
%% For use with LaTeX v2.09 and the `seminar' document style.
%%
%% COPYRIGHT 1993, by Timothy Van Zandt, Timothy.VAN-ZANDT@insead.edu
%%
%% This file may be distributed and/or modified under the conditions of
%% the LaTeX Project Public License, either version 1.2 of this license
%% or (at your option) any later version.  The latest version of this
%% license is in:
%% 
%%    http://www.latex-project.org/lppl.txt
%% 
%% and version 1.2 or later is part of all distributions of LaTeX version
%% 1999/12/01 or later.
%%
%%
%% You must install the PSTricks package v0.9 or later to use semcolor.sty.
%%
\@ifundefined{SemcolorFrameOps}{}{\endinput}
\typeout{Style Option: `semcolor' for doc style `seminar'
  \fileversion\space <\filedate> (tvz)}
% RN modification begin -- 2004/05/04; Rolf.Niepraschk@ptb.de
%\input{pstricks}
\ProvidesFile{semcolor.sty}[\filedate\space v\fileversion\space (tvz)]
%\RequirePackage{pstricks}
\RequirePackage{pst-ovl}
% RN modification end
%
% Set up the frame styles:
\def\SemcolorFrameOps{\psset{%
  fillstyle=none,%
  linewidth=\slideframewidth,%
  framesep=\slideframesep,%
  cornersize=absolute,%
  linearc=.5cm,%
  shadowsize=2pt,%
  shadowcolor=darkgray}}%
\newslideframe{scplain}[\SemcolorFrameOps]{\psframebox{#1}}
\newslideframe{scdouble}[\SemcolorFrameOps]{\psdblframebox{#1}}
\newslideframe{scshadow}[\SemcolorFrameOps]{\psshadowbox{#1}}
\slideframe{scplain}
%
% Rotation:
\def\leftsliderotation#1{\rotateleft{#1}}
\def\rightsliderotation#1{\rotateright{#1}}
\sliderotation{left}
%
% Color:
\addto@hook\after@slidepage{\pstVerb{0 setgray}}
%
% Overlays:
\def\make@overlay#1{%
  \setbox\@slidebox=\hbox{%
    \pst@initoverlay{#1}%
    \pst@Verb{ 0 \tx@BOL }%
    \copy\@@slidebox
    \pst@Verb{ -1 \tx@BOL }}}
\def\@overlay{\pst@overlay}
\addto@hook\before@newslide{\def\curr@overlay{0}}
\@ifundefined{overlaystrue}{}{\ifarticle\else\overlaystrue\layerstrue\fi}
%
\endinput
%% END semcolor.sty
}}
\DeclareOption{semlcmss}{\AtEndOfClass{%% BEGIN semlcmss.sty
\def\fileversion{1.0 (2e)}
\def\filedate{94/02/01}
%%
%% LaTeX style option `semlcmss.sty' for the `seminar' document style.
%% For use with LaTeX v2.09 and the New Font Selection Scheme (1 or 2)
%% or (some day) with LaTeX2e.
%%
%% COPYRIGHT 1994, by Timothy Van Zandt, Timothy.VAN-ZANDT@insead.edu
%%
%% This file may be distributed and/or modified under the conditions of
%% the LaTeX Project Public License, either version 1.2 of this license
%% or (at your option) any later version.  The latest version of this
%% license is in:
%% 
%%    http://www.latex-project.org/lppl.txt
%% 
%% and version 1.2 or later is part of all distributions of LaTeX version
%% 1999/12/01 or later.
%%
%%
%% This file contains the external font information needed to load
%% the SliTeX sans serif fonts for use by seminar.sty.
%%
\csname semlcmss@loaded\endcsname
\let\semlcmss@loaded\endinput
\typeout{Style Option: `semlcmss' \fileversion\space <\filedate> (tvz)}
\@ifundefined{selectfont}
  {\@seminarerr{`semlcmss' style option can only be used
    with the New Font Selection Scheme}\@eha\endinput}{}
%
\expandafter\ifx\csname DeclareSizeFunction\endcsname\relax

% First, the font definitions:
\new@fontshape{lcmss}{m}{n}{%
  <5>lcmss8 scaled 695%
  <6>lcmss8 scaled 833%
  <7>lcmss8 scaled 913%
  <8>lcmss8 %
  <9>lcmss8 scaled 1095%
  <10>lcmss8 scaled 1200%
  <11>lcmss8 scaled 1314%
  <12>lcmss8 scaled 1440%
  <14>lcmss8 scaled 1727%
  <17>lcmss8 scaled 2074%
  <20>lcmss8 scaled 2488%
  <25>lcmss8 scaled 2986%
  }{}
\extra@def{lcmss}{}{}
%
\new@fontshape{lcmss}{m}{it}{%
  <5>lcmssi8 scaled 695%
  <6>lcmssi8 scaled 833%
  <7>lcmssi8 scaled 913%
  <8>lcmssi8 %
  <9>lcmssi8 scaled 1095%
  <10>lcmssi8 scaled 1200%
  <11>lcmssi8 scaled 1314%
  <12>lcmssi8 scaled 1440%
  <14>lcmssi8 scaled 1727%
  <17>lcmssi8 scaled 2074%
  <20>lcmssi8 scaled 2488%
  <25>lcmssi8 scaled 2986%
  }{}
\extra@def{lcmss}{}{}
\subst@fontshape{lcmss}{m}{sl}{lcmss}{m}{it}
%
\new@fontshape{lcmss}{bx}{n}{%
  <5>lcmssb8 scaled 695%
  <6>lcmssb8 scaled 833%
  <7>lcmssb8 scaled 913%
  <8>lcmssb8 %
  <9>lcmssb8 scaled 1095%
  <10>lcmssb8 scaled 1200%
  <11>lcmssb8 scaled 1314%
  <12>lcmssb8 scaled 1440%
  <14>lcmssb8 scaled 1727%
  <17>lcmssb8 scaled 2074%
  <20>lcmssb8 scaled 2488%
  <25>lcmssb8 scaled 2986%
  }{}
\extra@def{cmssb}{}{}
\subst@fontshape{lcmss}{bx}{it}{lcmss}{bx}{n}
\subst@fontshape{lcmss}{bx}{sl}{lcmss}{bx}{n}

%
% Now define \slidefonts:
\ifarticle
  \def\slidefonts{%
    \def\default@family{lcmss}%
    \def\rmdefault{lcmss}%
    \def\sfdefault{lcmss}%
    \family\default@family\selectfont}
\else
    \def\default@family{lcmss}%
    \def\rmdefault{lcmss}%
    \def\sfdefault{lcmss}%
    \family\default@family\selectfont
\fi

\else

\DeclareFontFamily{OT1}{lcmss}{}{}
\DeclareFontShape{OT1}{lcmss}{m}{n}{
  <5>s*[1.1120]lcmss8
  <6>s*[1.1107]lcmss8
  <7>s*[1.0434]lcmss8
  <8>lcmss8
  <9>s*[0.9733]lcmss8
  <10><10.95><12><14.4><17.28><20.74><24.88>s*[0.96]lcmss8}{}

\DeclareFontShape{OT1}{lcmss}{m}{sl}{
  <5>s*[1.1120]lcmssi8
  <6>s*[1.1107]lcmssi8
  <7>s*[1.0434]lcmssi8
  <8>lcmssi8
  <9>s*[0.9733]lcmssi8
  <10><10.95><12><14.4><17.28><20.74><24.88>s*[0.96]lcmssi8}{}

\DeclareFontShape{OT1}{lcmss}{bx}{n}{
  <5>s*[1.1120]lcmssb8
  <6>s*[1.1107]lcmssb8
  <7>s*[1.0434]lcmssb8
  <8>lcmssb8
  <9>s*[0.9733]lcmssb8
  <10><10.95><12><14.4><17.28><20.74><24.88>s*[0.96]lcmssb8}{}

\DeclareFontShape{OT1}{lcmss}{m}{it}{<-> ssub * lcmss/m/sl }{}
\DeclareFontShape{OT1}{lcmss}{bx}{it}{<-> sub * lcmss/bx/n }{}
\DeclareFontShape{OT1}{lcmss}{bx}{sl}{<-> sub * lcmss/bx/n }{}

% Now define \slidefonts:
\ifarticle
  \def\slidefonts{%
    \def\default@family{lcmss}%
    \def\rmdefault{lcmss}%
    \def\sfdefault{lcmss}%
    \fontfamily\default@family\selectfont}
\else
    \def\default@family{lcmss}%
    \def\rmdefault{lcmss}%
    \def\sfdefault{lcmss}%
    \fontfamily\default@family\selectfont
\fi

\fi

% Now set up a few things for seminar.sty:
% These fonts are intended to be used with the 10pt font size
% declarations (and \magstep4), slightly modified:
\ptsize{10}\slidesmag{4}\articlemag{0}
\@namedef{10@semptsize}{%
  \def\@normalsize{\slide@@setsize\normalsize{12pt}\xpt\@xpt}%
  \def\small{\footnotesize}%
  \def\footnotesize{\slide@@setsize\footnotesize{9.5pt}\viiipt\@viiipt}%
  \def\scriptsize{\footnotesize}%
  \def\tiny{\slide@setsize\tiny{6pt}\vpt\@vpt}%
  \def\large{\slide@setsize\large{14pt}\xiipt\@xiipt}%
  \def\Large{\slide@setsize\Large{18pt}\xivpt\@xivpt}%
  \def\LARGE{\slide@setsize\LARGE{22pt}\xviipt\@xviipt}%
  \def\huge{\slide@setsize\huge{25pt}\xxpt\@xxpt}%
  \def\Huge{\huge}}
\def\semlcmss@warning#1{%
  \@warning{Can't use `\string#1' with the semlcmss option!}}
\def\ptsize#1{\semlcmss@warning\ptsize}
\def\slidesmag#1{\semlcmss@warning\slidesmag}
\def\articlemag#1{\semlcmss@warning\articlemag}

\endinput
%% END semlcmss.sty
}}
\DeclareOption{slidesec}{\AtEndOfClass{%% BEGIN slidesec.sty
\def\fileversion{v0.9}
\def\filedate{92/05/07}
%%
%% Style option slidesec.sty, for use with LaTeX 2.09 or SliTeX.
%%
%% COPYRIGHT 1992, by Timothy Van Zandt, Timothy.VAN-ZANDT@insead.edu
%%
%%
%% This file may be distributed and/or modified under the conditions of
%% the LaTeX Project Public License, either version 1.2 of this license
%% or (at your option) any later version.  The latest version of this
%% license is in:
%% 
%%    http://www.latex-project.org/lppl.txt
%% 
%% and version 1.2 or later is part of all distributions of LaTeX version
%% 1999/12/01 or later.
%%
%%
% DISCLAIMER:
%
% Someone who actually uses something like this should rewrite it.
%
% DESCRIPTION:
%
% This file contains macros for sections, headings, tables of contents,
% and lists of slides, for slides. It can probably be used with your
% favorite LaTeX slide package (e.g., seminar.sty).
% However:
%%   If the counter used to keep track of slides is not  slide  ,
%%   then change the following line:
\def\@SCTR{\theslide}
%%
% ACKNOWLEDGEMENTS:
%
% The idea comes from Young U. Ryu's lslides.sty.
%
% USAGE:
%
% \slideheading[#1]{#2}
%  * Analogous to \section.
%  * [#1] is optional; \slideheading{#1} is same as \slideheading[#1]{#2}.
%  * Advances slidesection counter.
%  * Sets \theslideheading to #1; can be used, e.g., in headers and footers.
%  * Adds #1 to slide contents file.
%  * If #2 is non-empty, invokes \makeslideheading{#2} to make a heading.
%    Default definition of \makeslideheading is
%       \begin{center}\large\bf #2 \end{center}
%    but you can redefine this to suit your tastes.
%
% \slidesubheading[#1]{#2}
%  * Analogous to \subsection.
%  * Like \slideheading, but add "sub" to all the names.
%
% \slideheading and \slidesubheading are meant to be used at most once in
% a slide, but this is not a binding restriction.
%
% You may want to define abbreviated forms. E.g.,
%   \newcommand{heading}{\slideheading}
%   \newcommand{subheading}{\slidesubheading}
%
% \listofslides
%  * Analogous to \listoftables.
%
% \slidecontents
%  * Analogous to \tableofcontents
%
% \Slidecontents
%  * List \slidecontents, but puts some check marks and stuff to
%    indicate what you have covered so far and what is coming up.
%    Meant to be used in the middle of a presentation.
%
% To change fonts and otherwise customize what you get from \listofslides,
% \slidecontents, and \Slidecontents, you have to hack at the definitions
% of these macros.
%
% BUGS:
%
% These macros have not been tested very much.
%
% IMPLEMENTATION:
%
\@ifundefined{@startlos}{}{\endinput}
\typeout{File `slidesec.sty' \fileversion\space\space
  \filedate\space\space <tvz>}
%
% If we use \@starttoc{los} to read in \jobname.los, then its contents
% are destroyed and we can't read it again. \@startlos is like \@starttoc,
% but it sets the flae \los@flag to 1, and then we add a few commands
% to \enddocument that open \filename.los if \los@flag=1.
%
\def\los@flag{0}%
\def\@startlos{%
  \begingroup\makeatletter\@input{\jobname.los}\endgroup
  \gdef\los@flag{1}\global\@nobreakfalse}
\newtoks\@temptokenb
\@temptokena{%
  \ifnum\los@flag=1\if@filesw
    \newwrite\tf@los\immediate\openout\tf@los\jobname.los\relax
  \fi\fi}
\expandafter\@temptokenb\expandafter{\enddocument}
\edef\enddocument{\the\@temptokena\the\@temptokenb}
%
\newcounter{slidesection}
\newcounter{slidesubsection}
\def\theslidesection{\arabic{slidesection}}
\def\thesubslidesection{\theslidesection.\arabic{subslidesection}}
%
\def\slideheading{\@dblarg\slide@heading}%
\def\slide@heading[#1]#2{%
  \stepcounter{slidesection}%
  \addtocontents{los}{\protect\l@slide
    {\the\c@slidesection}{\ignorespaces#1}{\@SCTR}}%
  \gdef\theslideheading{#1}%
  \def\@tempa{#2}%
  \ifx\@tempa\@empty\else
    {\edef\@currentlabel{\csname p@slidesection\endcsname\theslidesection}%
      \makeslideheading{#2}}%
  \fi}
\def\slidesubheading{\@dblarg\slide@subheading}
\def\slide@subheading[#1]#2{%
  \stepcounter{slidesection}%
  \addtocontents{los}{\protect\l@subslide
    {\the\c@slidesubsection}{\ignorespaces#1}{\@SCTR}}%
  \gdef\theslidesubheading{#1}%
  \def\@tempa{#2}%
  \ifx\@tempa\@empty\else
    {\edef\@currentlabel{\csname p@slidesubsection\endcsname
      \theslidesubsection}\makeslideheading{#2}}%
  \fi}
\def\theslideheading{}\def\theslidesubheading{}
%
\def\makeslideheading#1{%
  \begin{center}\large\bf #1\end{center}}
\def\makeslidesubheading{\makeslideheading}
%
\def\@undottedtocline#1#2#3#4#5{\vskip \z@ plus .2pt
  \bgroup
    \leftskip #2\relax \parindent #2\relax\@afterindenttrue
    \interlinepenalty\@M
    \leavevmode
    \@tempdima #3\relax \advance\leftskip \@tempdima
    \hbox{}\hskip -\leftskip
    #4\par
  \egroup}
\def\slide@dottedcline{\@dottedtocline{1}{1.5em}{2.3em}}
\def\slide@undottedcline{\@undottedtocline{1}{1.5em}{2.3em}}
\def\slide@cline{\slide@dottedcline}
%
\def\slidenumberline#1{\hbox to\@tempdima{\hss#1\hskip 1em}}
%
\def\listofslides{\section*{List of Slides%
  \@mkboth{LIST OF SLIDES}{LIST OF SLIDES}}%
  \def\l@slide##1##2##3{%
    \slide@undottedcline{\slidenumberline{##3}{##2}}{}}%
  \let\l@subslide\l@slide
  \@startlos}
%
\def\slidecontents{\@ifstar%
  {\def\slide@cline{\slide@undottedcline}\slide@contents}%
  {\def\slide@cline{\slide@dottedcline}\slide@contents}}
\def\slide@contents{%
  \def\l@slide##1##2##3{%
    \slide@cline{\slidenumberline{$\bullet$}{##2}}{##3}}%
  \let\l@subslide\@gobbletwo
  \@startlos}
%
\def\Slidecontents{\@ifstar%
  {\def\slide@cline{\slide@undottedcline}\Slide@contents}%
  {\def\slide@cline{\slide@dottedcline}\Slide@contents}}
\def\Slide@contents{%
  \def\l@slide##1##2##3{%
    \ifcase\lslide@flag
      \message{##1 ** \the\c@slidesection}
      \ifnum##1>\c@slidesection
        \def\lslide@flag{1}{\large\slide@cline{%
          \slidenumberline{$\Rightarrow\bullet$}{##2}}{##3}}%
      \else
        {\large\slide@cline{%
          \slidenumberline{$\surd\;\bullet$}{##2}}{##3}}%
      \fi
    \or
      \def\lslide@flag{2}{\large\slide@cline{%
        \slidenumberline{$\bullet$}{##2}}{##3}}%
    \or
      {\large\slide@cline{%
        \slidenumberline{$\bullet$}{##2}}{##3}}%
    \fi}%
  \def\l@subslide##1##2##3{%
    \ifnum\lslide@flag=1
      \@undottedtocline{2}{3.8em}{3.2em}{##2}{}%
    \fi}%
  \def\lslide@flag{0}%
  \@startlos}
%
\endinput
%% END slidesec.sty
}}
\DeclareOption{semlayer}{\AtEndOfClass{%% BEGIN semlayer.sty
%%
%% This is file `semlayer.sty', generated
%% on <1993/4/2> with the docstrip utility (2.0r).
%%
%% The original source files were:
%%
%% semlayer.doc
%%
\def\fileversion{1.0}
\def\filedate{93/04/01}
%%
%% LaTeX style option `semlayer.sty'.
%% For use with LaTeX v2.09 and the `seminar' document style.
%%
%% COPYRIGHT 1993, by Timothy Van Zandt, Timothy.VAN-ZANDT@insead.edu
%%
%% This file may be distributed and/or modified under the conditions of
%% the LaTeX Project Public License, either version 1.2 of this license
%% or (at your option) any later version.  The latest version of this
%% license is in:
%% 
%%    http://www.latex-project.org/lppl.txt
%% 
%% and version 1.2 or later is part of all distributions of LaTeX version
%% 1999/12/01 or later.
%%
%%
\addto@hook\before@newslide{%
  \global\def\overlay@list{}%
  \setcounter{overlay}{0}}
\newbox\@@slidebox
\def\process@@slide{%
  \ifx\overlay@list\@empty
    \finish@slide
    \output@slide
  \else
    \setbox\@@slidebox\box\@slidebox
    \make@overlay{0}%
    \finish@slide
    \output@slide
    {\let\theoverlay\theslide\layer@loop{0}}%
    \overlay@loop
  \fi}
\def\overlay@loop{%
  \begingroup
    \overlay@prep
    \c@overlay=1
    \@whilenum\c@overlay<10
      \do{%
        \test@member{\the\c@overlay}{\overlay@list}%
        \if@test
          \make@overlay{\the\c@overlay}%
          \finish@slide
          \output@slide
        \fi
        \layer@loop{\the\c@overlay}%
        \advance\c@overlay by 1}%
  \endgroup}
\def\layer@loop#1{%
  \begingroup
    \layer@prep
    \@for\currlayer:=\layer@list\do{%
    \test@member{#1-\currlayer}{\overlay@list}%
      \if@test
        \make@overlay{#1-\currlayer}%
        \finish@slide
        \output@slide
      \fi}%
  \endgroup}
\def\layer@prep{%
  \ifarticle\pagestyle{\layer@pagestyle}\fi
  \let\thepage\thelayer
  \ifx\thelayerframe\@empty\else\let\theslideframe\thelayerframe\fi
  \let\slidelabel\layerlabel}
\def\overlay@prep{%
  \ifarticle\pagestyle{\overlay@pagestyle}\fi
  \let\thepage\theoverlay
  \ifx\theoverlayframe\@empty\else\let\theslideframe\theoverlayframe\fi
  \let\slidelabel\overlaylabel}
\def\layerframe{\@slideframe{layer}}
\def\overlayframe{\@slideframe{overlay}}
\def\thelayerframe{}
\def\theoverlayframe{}
\def\overlaylabel{\bf Overlay \theoverlay}
\def\layerlabel{\bf Layer \thelayer}
\@ifundefined{c@overlay}{\newcounter{overlay}}{}%%% hv 20131224
%\newcounter{overlay}
\def\theoverlay{\theslide-\alph{overlay}}
\def\currlayer{}
\def\thelayer{\theoverlay-\currlayer}
\newif\ifoverlays
\newif\iflayers
\@ifundefined{@overlay}%
  {\let\@overlay\relax}%
  {\ifarticle\else\overlaystrue\layerstrue\fi}
\def\nooverlays@err{%
  \@seminarerr{You need to use additional macros for overlays,
    e.g., semcolor.sty}\@ehd
  \global\overlaysfalse
  \global\layersfalse}
\@ifundefined{@overlay}{}{}
\def\overlay#1{%
  \ifoverlays\ifx\@overlay\relax\nooverlays@err\fi\fi
  \ifslide
    \@tempcnta=#1
    \@testfalse
    \ifnum\@tempcnta>-1 \ifnum\@tempcnta<10 \@testtrue\fi\fi
    \if@test
      \c@overlay=\@tempcnta
      \ifnum\c@overlay=0
        \advance\c@slide by -1
        \refstepcounter{slide}%
      \else
        \advance\c@overlay by -1
        \refstepcounter{overlay}%
        \ifoverlays
          \xdef\overlay@list{\overlay@list,\the\c@overlay}%
        \fi
      \fi
      \ifoverlays
        \@overlay{\the\c@overlay}%
      \fi
    \else
      \@seminarerr{\string\overlay\space argument must be
        an integer between 0 and 9}\@eha
    \fi
  \else
    \@seminarerr{\string\overlay\space can only be used
      in slide environments}\@eha
  \fi
  \ignorespaces}
\def\colorlayers#1{%
  \edef\@tempc{#1}%
  \edef\layer@list{\layer@list,#1}%
  \def\@tempa##1{%
    \@namedef{##1}{%
      \iflayers\ifx\@overlay\relax\nooverlays@err\fi\fi
      \ifslide
        \iflayers
          \xdef\overlay@list{\overlay@list,\the\c@overlay-##1}%
          \@overlay{\the\c@overlay-##1}%
        \fi
      \else
        \@seminarerr{Color layer `##1' can only be used in
          slide environments}\@eha
      \fi}}
  \@for\@tempb:=\@tempc\do{%
    \expandafter\@tempa\expandafter{\@tempb}}}
\def\layer@list{}
\def\overlaypagestyle#1{\@ifgoodps{#1}{\edef\overlay@pagestyle{#1}}}
\def\layerpagestyle#1{\@ifgoodps{#1}{\edef\layer@pagestyle{#1}}}
\overlaypagestyle{}
\layerpagestyle{}

\endinput
%%
%% End of file `semlayer.sty'.
}}
\DeclareOption*{\PassOptionsToClass{\CurrentOption}{article}}
\ProcessOptions
\ifnotesonly\else\notesonlystarfalse\fi
\newdimen\footheight  % seminar.sty v1.0 uses it, but its not in LaTeX2e.
% Don't want these changed by article.cls:
\let\sem@paperheight\paperheight
\let\sem@paperwidth\paperwidth
\let\paperheight\footheight
\let\paperwidth\footheight
\LoadClass{article}

\RequirePackage{pst-ovl}%    hv 20131224

\renewcommand\normalsize{%
   \@setfontsize\normalsize\@xpt\@xiipt
   \abovedisplayskip 10\p@ \@plus2\p@ \@minus5\p@
   \abovedisplayshortskip \z@ \@plus3\p@
   \belowdisplayshortskip 6\p@ \@plus3\p@ \@minus3\p@
   \belowdisplayskip \abovedisplayskip
   \let\@listi\@slide@listi}
\let\paperheight\sem@paperheight
\let\paperwidth\sem@paperwidth
\RequirePackage{sem-page}
\newtoks\before@document
\newtoks\after@document
\let\xcomment@hook\relax
\before@document{\endgroup\the\before@document\begingroup}
\after@document{\the\after@document\xcomment@hook}
\expandafter\@temptokena\expandafter{\document}
\edef\document{\the\before@document\the\@temptokena\the\after@document}
\before@document{}
\after@document{}
\newtoks\before@enddocument
\before@enddocument{\the\before@enddocument}
\expandafter\@temptokena\expandafter{\enddocument}
\edef\enddocument{\the\before@enddocument\the\@temptokena}
\before@enddocument{}
\newif\ifprintlandscape
\ifportrait
  \ifarticle\printlandscapetrue\fi
\else
  \ifarticle\else\printlandscapetrue\fi
\fi
\addto@hook\before@document{\ifprintlandscape\printlandscape\fi}
\addto@preamblecmds{\printlandscape}
\def\printlandscape{\addto@hook\before@enddocument{%
  \typeout{^^J%
  *******************************************************^^J%
  ***** !! PRINT THIS DOCUMENT IN LANDSCAPE MODE !! *****^^J%
  *******************************************************}}}
\def\slide{%
  \NestedSlide@Error{slide}%
  \landscapetrue
  \@ifnextchar[{\begin@slide}{\begin@slide[\slidewidth,\slideheight]}}
\def\endslide{\end@slide}
\@namedef{slide*}{%
  \NestedSlide@Error{slide*}%
  \landscapefalse
  \@ifnextchar[{\begin@slide}{\begin@slide[\slidewidth,\slideheight]}}%
\@namedef{endslide*}{\end@slide}
\newcounter{slide}
\def\theslide{\arabic{slide}}
\newcount\slide@count
\newbox\@slidebox
\newbox\not@slidebox
\newif\ifslide
\newif\iflandscape
\def\@landscapeonly{0}
\def\landscapeonly{\def\@landscapeonly{1}}
\def\portraitonly{\def\@landscapeonly{-1}}
\def\NestedSlide@Error#1{%
  \ifslide
    \endgroup
    \@seminarerr{Nested slide environments. Perhaps missing
      \string\end{\@currenvir}. May be fatal}\@ehd
    \expandafter\end\expandafter{\@currenvir}%
    \ifslide\expandafter\end\expandafter{\@currenvir}\fi
    \begingroup
    \def\@currenvir{#1}%
  \fi}
\def\begin@slide[#1,#2]{%
  \slide@clearpage
  \setlength\slidewidth{#1}%
  \setlength\slideheight{#2}%
  \begingroup
    \ifarticle
      \output{%
        \advance\count@ 1
        \global\setbox\not@slidebox\box\@cclv}%
      \par\@@par\penalty-\@M
    \fi
    \output{\slide@output}%
    \slidetrue
    \ifarticle\global\slide@count=\z@\fi
    \refstepcounter{slide}%
    \ifnotesonlystar\xdef\first@slidemarker{\the\c@slide}\fi
    \def\do##1{\setcounter{##1}\z@}\slide@reset
    \ifarticle\else
      \edef\page@textheight{\number\textheight sp}%
      \edef\page@textwidth{\number\textwidth sp}%
    \fi
    \set@slidesize
    \slidebox@restore
    \the\slide@hook
    \the\before@newslide
    \everyslide}
\def\end@slide{%
    \par\penalty-\@M
    \xdef\@tempg{\@currenvir}%
  \endgroup
  \ifslide
    \@seminarerr{Perhaps missing `\string\end{\@tempg}',
      \iffalse{\fi`\string}' or `\string\endgroup'}\@ehd
    \def\next{\endgroup\ifslide\expandafter\next\fi}%
    \next
  \fi
  \begingroup
    \output{\setbox\@tempboxa\box\@cclv}%
    \@@par\penalty-\@M
  \endgroup
  \global\advance\c@slide-1
  \def\do##1{\setcounter{##1}{\@nameuse{saved@c@##1}}}%
  \slide@reset
  \ifarticle\outputloop@savedslides\fi
  \the\after@slide}
\let\slideclearpagetrue\relax
\let\slideclearpagefalse\relax
\ifarticle
  \def\slide@clearpage{\par\penalty\z@}
  \ifnotes
    \def\slideclearpagetrue{\def\slide@clearpage{\clearpage}}
    \def\slideclearpagefalse{\def\slide@clearpage{\par\penalty\z@}}
  \else
    \ifnotesonly
      \def\slideclearpagetrue{\def\slide@clearpage{\clearpage}}
      \def\slideclearpagefalse{\def\slide@clearpage{\par\penalty\z@}}
    \fi
  \fi
\else
  \def\slide@clearpage{\clearpage}
\fi
\newtoks\slide@hook
\def\everyslide{}
\newtoks\after@slide
\def\slidebox@restore{%
  \def\thepage{\theslide}%
  \def\newpage{\newslide}%
  \def\clearpage{\newslide}%
  \def\thispagestyle##1{\notslide@err{\thispagestyle}}%
  \pagestyle{\slide@pagestyle}%
  \@twocolumnfalse
  \def\twocolumn{\notslide@err{\twocolumn}}%
  \def\onecolumn{\notslide@err{\onecolumn}}%
  \def\marginpar##1{\notslide@err{\marginpar}}%
  \def\thanks{\slidethanks}%
  \def\maketitle{\slidemaketitle}%
  \fix@floats
  \fix@whatsits
  \slide@footnotes
  \def\do##1{\expandafter\xdef\csname
   saved@c@##1\endcsname{\the\@nameuse{c@##1}}}%
  \slide@reset
  \topskip\z@ \maxdepth\z@
  \slide@listparameters
  \slidefonts
  \def\baselinestretch{\slidestretch}%
  \def\arraystretch{\slidearraystretch}%
  \sem@ptsize{\slide@ptsize}}
\def\date#1{\gdef\@date{#1}\gdef\thedate{#1}}
\def\author#1{\gdef\@author{#1}\gdef\theauthor{#1}}
\def\title#1{\gdef\@title{#1}\gdef\thetitle{#1}}
\date{\today}
\let\slidethanks\thanks
\def\thethanks{\@thanks}%
\def\slidemaketitle{%
  \par
  \begin{center}\bf
    {\large \thetitle}\par\vskip 1ex
    \begin{tabular}[t]{c} \theauthor \end{tabular}\par\vskip 1ex
    \thedate
  \end{center}%
  \thethanks\par}
\def\fix@floats{%
  \def\@xfloat##1[##2]{%
    \expandafter\let\csname end##1\endcsname\end@float
    \par\medskip\vbox\bgroup\def\@captype{##1}\parindent\z@
    \ignorespaces}%
  \def\end@float{\par\vskip\z@\egroup\medskip}%
  \def\@dblfloat{\@float}\def\end@dblfloat{\end@float}%
  \def\endfigure{\end@float}\def\endtable{\end@float}}
\let\normal@write\write
\let\normal@read\read
\let\normal@openout\openout
\let\normal@closeout\closeout
% changed as per DPC 1997/09/16
\def\fix@whatsits{%
  \long\def \protected@write##1##2##3{%
      \begingroup
       ##2%
       \let\protect\noexpand
       \immediate\normal@write##1{##3}%
      \endgroup
      \if@nobreak\ifvmode\nobreak\fi\fi}%
}
% end change
\newinsert\slide@footins
\skip\slide@footins=\bigskipamount
\count\slide@footins=1000
\dimen\slide@footins=4in
\def\theslidefootnote{\alph{footnote}}
\def\slide@footnotes{%
  \def\thefootnote{\theslidefootnote}%
  \let\footins\slide@footins
  \interfootnotelinepenalty\@M}
\def\slidefonts{}
\def\slidestretch{1.18}
\def\slidearraystretch{1.2}
\def\raggedslides{\@ifnextchar[{\@raggedslides}{\@raggedslides[1fil]}}
\def\@raggedslides[#1]{%
  \edef\slide@@rightskip{#1}%
  \ifslide\slide@rightskip\fi}
\def\slide@rightskip{%
  \@rightskip\z@ plus \slide@@rightskip\relax \rightskip\@rightskip}
\def\slide@@rightskip{1fil}
\newcount\slide@listdepth
\def\slide@listparameters{%
  \let\@listdepth\slide@listdepth
  \slide@listdepth\z@
  \def\@listi{\slide@listi}%
  \def\@listii{\slide@listii}%
  \def\@listiii{\slide@listiii}%
  \let\@listiv\relax\let\@listv\relax\let\@listvi\relax}
\def\slide@listi{%
  \leftmargin\leftmargini
  \labelwidth\leftmargini \advance\labelwidth-\labelsep
  \parsep\parskip \divide\parsep2
  \partopsep\slidepartopsep\relax
  \advance\partopsep-\parskip
  \ifdim\partopsep<\z@\partopsep\z@\fi
  \itemsep\slideitemsep\relax
  \ifdim\parsep<\itemsep
    \topsep\itemsep \advance\topsep-\parsep
  \else
    \itemsep\parsep \topsep\z@
  \fi}
\def\slide@listii{%
  \leftmargin\leftmarginii
  \labelwidth\leftmarginii \advance\labelwidth-\labelsep
  \divide\itemsep2 \divide\parsep2
  \divide\topsep2 \divide\partopsep2\relax}%
\def\slide@listiii{%
  \leftmargin\leftmarginiii
  \labelwidth\leftmarginiii \advance\labelwidth-\labelsep
  \itemsep \z@ \parsep\z@ \topsep\z@}%
\def\slideleftmargini{1.8em}
\def\slideleftmarginii{1.4em}
\def\slideleftmarginiii{1em}
\def\slidelabelsep{.5em}
\def\slideitemsep{.8ex minus .2ex}
\def\slidepartopsep{1ex minus .2ex}
\newbox\saved@specials
\def\save@slidespecials{%
  \begingroup
    \output{%
      \global\setbox\saved@specials=\box\@cclv
      \global\wd\saved@specials=\z@
      \global\dp\saved@specials=\z@
      \global\ht\saved@specials=\z@}%
    \hbox{}\penalty-\@M
    \global\let\saved@texttop\@texttop
    \gdef\@texttop{%
      \ifvoid\saved@specials\else\box\saved@specials\nointerlineskip\fi
      \saved@texttop
      \global\let\@texttop\saved@texttop}%
  \endgroup}
\addto@hook\after@document{\save@slidespecials}
\ifarticle\else
  \def\insert@specials{%
    \ifvoid\saved@specials\else
      \setbox\@slidebox\hbox{\box\saved@specials\box\@slidebox}%
    \fi
    \global\let\insert@specials\relax}
\fi
\def\extraslideheight#1{%
  \@tempdima #1\relax
  \edef\X@SlideHeight{\number\@tempdima sp}
  \ifslide\set@slidesize\fi}
\extraslideheight{6pt}
\def\set@slidesize{%
  \iflandscape
    \hsize=\inverseslidesmag\slidewidth
    \vsize=\inverseslidesmag\slideheight
  \else
    \hsize=\inverseslidesmag\slideheight
    \vsize=\inverseslidesmag\slidewidth
  \fi
  \edef\slide@vsize{\number\vsize sp}%
  \textheight\vsize
  \advance\vsize\X@SlideHeight\relax
  \textwidth\hsize\columnwidth\hsize\linewidth\hsize}
\def\newslide{%
  \par\penalty-\@M
  \def\do##1{\setcounter{##1}\z@}\slide@reset
  \the\before@newslide
  \set@slidesize}
\newtoks\before@newslide
\def\slide@reset{\do{footnote}}
\def\slidereset#1{\def\slide@reset{}\addtoslidereset{#1}}
\def\addtoslidereset#1{%
  \edef\@tempa{#1}\expandafter\addto@slidereset\@tempa,\@nil,}
\def\addto@slidereset#1,{%
  \ifx\@nil#1\else
    \@ifundefined{c@#1}%
      {\@seminarerr{Counter `#1' not defined}\@ehd}%
      {\expandafter\def\expandafter\slide@reset\expandafter{%
        \slide@reset\do{#1}}}%
    \expandafter\addto@slidereset
  \fi}
\def\slide@output{%
  \@makeslide
  \@testfalse
  \ifnotesonly\else
    \iflandscape
      \ifnum\@landscapeonly>-1 \os@test\fi
    \else
      \ifnum\@landscapeonly<1 \os@test\fi
    \fi
  \fi
  \if@test
    \reset@slideoutput
    \@@makeslide
    \process@slide
  \fi
  \refstepcounter{slide}}
\def\reset@slideoutput{%
  \let\par\@@par
  \reset@font
  \def\baselinestretch{1}%
  \@nameuse{1\@ptsize @semptsize}%
  \catcode`\ =10
  \let\-\@dischyph \let\'\@acci \let\`\@accii \let\=\@acciii}
\newif\ifcenterslides
\centerslidestrue
\providecommand\@begindvi{} % added 1997/04/15 SPQR, to cope with very
                            % old LaTeXs
\def\@makeslide{%
  \setbox\@slidebox\vbox{%
    \@begindvi % added 1997/04/15 SPQR
    \unvbox\@cclv
    \ifvoid\slide@footins\else
      \vskip\skip\slide@footins
      \footnoterule
      \unvbox\slide@footins
    \fi
    \vskip\z@}}
\def\@@makeslide{%
  \overfullslide@warning
  \setbox\@slidebox\hbox{%
    \vfuzz=\slidefuzz\relax
    \vbox to\slide@vsize{%
    \ifcenterslides\vskip\z@ plus .0001fil \fi
    \unvbox\@slidebox
    \vskip\z@ plus .0001fil}%
    \the\after@slidepage}%
  \wd\@slidebox\textwidth}
\def\overfullslide@warning{%
  \dimen@\ht\@slidebox
  \advance\dimen@-\slide@vsize\relax
  \ifdim\dimen@>\slidefuzz\relax
    \@warning{Slide \theslide\space overfull by \the\dimen@}%
  \fi}
\def\slidefuzz{2pt}
\newtoks\after@slidepage
\newif\ifrotateheaders
\def\sliderotation#1{\@ifundefined{semsr@#1}%
  {\@latexerr{Slide rotation `#1' not defined.}\@eha}%
  {\@nameuse{semsr@#1}}}
\def\semsr@left{%
  \def\rotate@slide{%
    \setbox\@slidebox\hbox{\leftsliderotation{\box\@slidebox}}}}
\def\semsr@right{%
  \def\rotate@slide{%
    \setbox\@slidebox\hbox{\rightsliderotation{\box\@slidebox}}}}
\def\semsr@none{\let\rotate@slide\relax}
\sliderotation{none}
\def\leftsliderotation#1{%
  \@seminarerr{\string\leftsliderotation\space has not been defined}%
  \@ehd}
\let\rightsliderotation\leftsliderotation
\def\@ifrotateslide#1{%
  \ifx\rotate@slide\relax\else
    \iflandscape\ifportrait#1\fi\else\ifportrait\else#1\fi\fi
  \fi}
\def\process@slide{%
  \slideframewidth=\inverseslidesmag\slideframewidth
  \slideframesep=\inverseslidesmag\slideframesep
  \fboxrule\slideframewidth
  \fboxsep\slideframesep
  \ifarticle
    \@ifrotateslide\rotate@slide
  \else
    \ifrotateheaders\else\@ifrotateslide\rotate@slide\fi
  \fi
  \process@@slide}
\def\process@@slide{\finish@slide\output@slide}
\def\finish@slide{%
  \theslideframe
  \add@slidelabel\slidelabel}
\ifarticle
  \def\output@slide{%
    \global\advance\slide@count1
    \@ifundefined{slidebox@\the\slide@count}%
      {{\globaldefs=1\expandafter
        \new@slidebox\csname slidebox@\the\slide@count\endcsname}}{}%
    \expandafter\global\expandafter\setbox\csname
      slidebox@\the\slide@count\endcsname\box\@slidebox}
\else
  \def\output@slide{%
    \begingroup
      \hoffset=-\inverseslidesmag in
      \voffset=-\inverseslidesmag in
      \setslidelength\@tempdima{%
        \ifportrait\paperwidth\else\paperheight\fi}
      \setslidelength\@tempdimb{%
        \ifportrait\paperheight\else\paperwidth\fi}
      \ifrotateheaders
        \@ifrotateslide{%
          \dimen@=\@tempdima
          \@tempdima=\@tempdimb
          \@tempdimb=\dimen@}
      \fi
      % \oddsidemargin, \evensidemargin, \headheight, \footheight
      % used for scratch:
      \setslidelength\oddsidemargin\slideleftmargin
      \setslidelength\evensidemargin\sliderightmargin
      \setslidelength\headheight\slidetopmargin
      \setslidelength\footheight\slidebottommargin
      % Some page styles like to know \textwidth:
      \textwidth=\@tempdima
      \advance\textwidth-\oddsidemargin
      \advance\textwidth-\evensidemargin
      \setbox\@slidebox=\hbox to \@tempdima{%
        \kern\oddsidemargin
        \vbox to\@tempdimb{%
          \ifnum\fancyput@flag>-1
            \hbox{\kern-\oddsidemargin\do@fancyput}%
          \fi
          \let\label\@gobble
          \let\index\@gobble
          \let\glossary\@gobble
          \vbox to\headheight{%
            \vfill
            \hbox{%
              \slideheadfont\relax\strut
              \hbox to\textwidth{\@oddhead}}%
            \kern\z@}%
          \vfill
          \hbox to\textwidth{\hss\box\@slidebox\hss}%
          \vfill
          \vbox to\footheight{%
            \hbox{%
              \slidefootfont\relax\strut
              \hbox to\textwidth{\@oddfoot}}%
            \vfill}}%
        \hss}%
      \ifrotateheaders\@ifrotateslide\rotate@slide\fi
      \insert@specials
      \shipout\box\@slidebox
    \endgroup
    \let\firstmark\botmark}
  \@ifundefined{fancyput@flag}{\def\fancyput@flag{-1}}{}
\fi
\newskip\slidesep
\slidesep\intextsep
\ifarticle
  \def\fps@fslide{htbp}
  \def\ftype@fslide{32}
  \def\float@savedslide{%
    \begingroup\@float{fslide}%
      \centerline{\box\@slidebox}%
    \end@float\endgroup}%
  \@namedef{float*@savedslide}{%
    \begingroup\@dblfloat{fslide}%
      \centerline{\box\@slidebox}%
    \end@dblfloat\endgroup}%
  \def\center@slide{\hbox{%
    \kern-\@totalleftmargin
    \hbox to \columnwidth{\hss\box\@slidebox\hss}}}%
  \def\onepercol@savedslide{%
    \vbox to .996\textheight{\vss\center@slide\vss}\goodbreak}%
  \def\twopercol@savedslide{%
    \dimen@.5\textheight
    \advance\dimen@-\slidesep
    \ifdim\ht\@slidebox>\dimen@
      \onepercol@savedslide
    \else
      \vbox to .498\textheight{\vss\center@slide\vss}\goodbreak
    \fi}
  \def\here@savedslide{%
    \addvspace\slidesep\center@slide\addvspace\slidesep}
  \@namedef{here*@savedslide}{%
    \goodbreak \hrule \@height\z@ \nobreak \vskip\slidesep \nobreak
    \center@slide
    \nobreak \vskip\slidesep \nobreak \hrule\@height\z@ \goodbreak}
\fi
\ifarticle
  \def\slideplacement#1{\@ifundefined{#1@savedslide}%
    {\@seminarerr{Slide placement `#1' undefined}\@ehd}%
    {\expandafter\let\expandafter\output@savedslide
      \csname #1@savedslide\endcsname}}
\else
  \def\slideplacement#1{}
\fi
\ifarticle
  \ifnotes
    \ifportrait
      \slideplacement{float}
    \else
      \slideplacement{float*}
    \fi
  \else
    \ifportrait
      \slideplacement{onepercol}
    \else
      \slideplacement{twopercol}
    \fi
  \fi
\fi
\ifarticle
  \def\outputloop@savedslides{%
    \global\maxdepth\@maxdepth
    \ifvoid\not@slidebox\else
      \dimen@=\dp\not@slidebox
      \unvbox\not@slidebox
      \hrule height\z@
      \prevdepth\dimen@
      \penalty\z@
    \fi
    \edef\slide@@count{\the\slide@count\relax}%
    \slide@count\z@
    \loop
    \ifnum\slide@count<\slide@@count
      \advance\slide@count1
      \expandafter\setbox\expandafter\@slidebox\expandafter\box
        \csname slidebox@\the\slide@count\endcsname
      \output@savedslide
    \repeat
    \ifnotesonlystar\make@slidemarker\fi}
\fi
\def\make@slidemarker{%
  \addvspace\slidesep
  \moveleft\@totalleftmargin
  \vbox{%
    \hsize\columnwidth
    \hrule height 1pt
    \kern 8pt
    \hbox to \columnwidth{%
      \hss
      \LARGE\bf\the@slidemarker
      \hss}%
    \kern 8pt
    \hrule height 1pt}%
  \addvspace\slidesep}
\def\the@slidemarker{%
  Slide%
  \ifnum\c@slide=\first@slidemarker\else
    s {\c@slide\first@slidemarker\relax\theslide} --\fi
  { }\theslide}%
\ifarticle
  \let\c@note\c@page
  \def\p@note{\p@page}
  \def\thenote{\thepage}
\else
  \newcounter{note}
  \def\thenote{\theslide-\arabic{note}}
  \def\thepage{\thenote}
  \addto@hook\after@slide{\setcounter{note}{1}}
  \expandafter\def\expandafter\@outputpage\expandafter{%
    \@outputpage\stepcounter{note}}
\fi
\ifarticle\else
  \let\c@page\c@slide
  \countdef\c@slide=0
  \c@slide=0
  \c@page=1
\fi
\ifarticle
  \let\truepagenumbers\relax
\else
  \def\truepagenumbers{%
    \let\c@slide\c@page
    \countdef\c@page=0
    \c@page=1
    \c@slide=0
    \let\truepagenumbers\relax}
\fi
\addto@preamblecmds{\truepagenumbers}
\newdimen\slideframewidth \slideframewidth 4pt
\newdimen\slideframesep \slideframesep .3in
\def\newslideframe#1{%
  \@ifnextchar[{\@newslideframe{#1}}{\@newslideframe{#1}[]}}
\def\@newslideframe#1[#2]{%
  \@namedef{semsfops@#1}{#2}%
  \@namedef{semsf@#1}##1}
\newslideframe{plain}{\fbox{#1}}
\def\slideframe{\@slideframe{slide}}
\def\@slideframe#1{%
  \@ifstar{\@testtrue\@@slideframe{#1}}{\@testfalse\@@slideframe{#1}}}
\def\@@slideframe#1{%
  \@ifnextchar[{\@@@slideframe{#1}}{\@@@slideframe{#1}[]}}
\def\@@@slideframe#1[#2]#3{%
  \def\@tempa{none}%
  \def\@tempb{#3}%
  \ifx\@tempa\@tempb
    \@namedef{the#1frame}{\relax}%
  \else
    \ifx\@tempb\@empty
      \@namedef{the#1frame}{}%
    \else
      \@ifundefined{semsf@#3}%
        {\@seminarerr{Slide frame `#3' undefined}\@eha}%
        {\if@test
          \@@@@slideframe{#1}[#2]{#3}%
        \else
          \@namedef{the#1frame}{\setbox\@slidebox=\hbox{{%
            \@nameuse{semsfops@#3}#2\@nameuse{semsf@#3}{\box\@slidebox}}}}%
        \fi}%
     \fi
   \fi}
\def\@@@@slideframe#1[#2]#3{%
  \expandafter\let\expandafter\@tempa\csname the#1frame\endcsname
  \edef\next{%
    \noexpand\def\expandafter\noexpand\csname the#1frame\endcsname}%
  \expandafter\next\expandafter{\@tempa
    \setbox\@slidebox=\hbox{{%
      \@nameuse{semsfops@#3}%
      #2%
      \@nameuse{semsf@#3}{\box\@slidebox}}}}}%
\slideframe{plain}
\def\slidestyle#1{\@ifundefined{ss@#1}%
  {\@seminarerr{Slide style `#1' undefined}\@eha}%
  {\@nameuse{ss@#1}}}
\def\ss@empty{\let\add@slidelabel\@gobble}
\def\ss@left{\def\add@slidelabel##1{%
  \setbox\@slidebox=\hbox{%
    \vbox to \ht\@slidebox{\vss
    \hbox to 0pt{\hss##1\hskip 15pt}%
    \vss}\box\@slidebox}}}
\def\ss@bottom{\def\add@slidelabel##1{%
  \setbox\@slidebox=\vbox{\copy\@slidebox\vskip 9pt
    \hbox to\wd\@slidebox{\hss##1\hss}}}}%
\ifarticle
  \ifportrait\slidestyle{bottom}\else\slidestyle{left}\fi
\else
  \slidestyle{empty}
\fi
\def\slidelabel{\bf Slide \theslide}
\def\newpagestyle#1#2#3{%
  \expandafter\newcommand\csname ps@#1\endcsname{%
    \def\@oddhead{#2}\let\@evenhead\@oddhead
    \def\@oddfoot{#3}\let\@evenfoot\@oddfoot}}
\def\renewpagestyle#1#2#3{%
  \expandafter\renewcommand\csname ps@#1\endcsname{%
    \def\@oddhead{#2}\let\@evenhead\@oddhead
    \def\@oddfoot{#3}\let\@evenfoot\@oddfoot}}
\def\@ifgoodps#1{%
  \@ifundefined{ps@#1}{\@seminarerr{Page style `#1' undefined}\@eha}}
\def\slidepagestyle#1{%
  \@ifgoodps{#1}%
    {\ifslide\pagestyle{#1}\else\edef\slide@pagestyle{#1}\fi}}
\def\ps@{}
\slidepagestyle{}
\ifarticle
  \def\ps@align{}
\else
  \def\ps@align{%
    \def\@oddhead{\thepage\hfil+}\let\@evenhead\@oddhead
    \def\@oddfoot{+\hfil+}\let\@evenfoot\@oddfoot}
\fi
\def\slideheadfont{\scriptsize}
\def\slidefootfont{\scriptsize}
\def\magstep#1{\ifcase#1 \@m\or 1200\or 1440\or 1728\or
  2074\or 2488\or 2986\or 3583\or 4300\or 5160\fi\relax}
\def\magstepminus#1{%
  \ifcase#1 \@m\or 833\or 694\or 579\or 482\or 401\fi\relax}
\def\@magstep#1{%
  \ifnum#1<\z@\magstepminus{-#1}\else\magstep#1\fi}
{\catcode`\p=12\catcode`\t=12
  \gdef\@@inv@@mag#1pt#2{\def#2{#1}}}
\def\invert@mag#1{\@tempdima=1000pt
  \divide\@tempdima by #1\relax
  \expandafter\@@inv@@mag\the\@tempdima#1}
\def\@slidesmag#1{%
  \@tempcnta=#1\relax%
  \ifnum\@tempcnta>0
    \edef\inverseslidesmag{\the\@tempcnta}%
    \invert@mag\inverseslidesmag
    \ifarticle\else\mag\@tempcnta\fi
  \else
    \@seminarerr{\string\@slidesmag\space argument must be an
      integer equal to 1000 times the magnification}\@eha
  \fi}
\def\@articlemag#1{%
  \@tempcnta=#1\relax%
  \ifnum\@tempcnta>0
    \edef\inverseartmag{\the\@tempcnta}%
    \invert@mag\inverseartmag
    \ifarticle\mag\@tempcnta\fi
  \else
    \@seminarerr{\string\articlemag\space argument must be an
      integer equal to 1000 times the magnification}\@eha
  \fi}
\addto@preamblecmds{\@slidesmag\do\@articlemag}
\newdimen\semin
\newdimen\semcm
\def\@semmagerr#1{%
  \@seminarerr{\string#1 argument must be an integer
    between -5 and 9}\@eha}
\def\slidesmag#1{%
  \@tempcnta=#1\relax
  \ifnum\@tempcnta>-6
    \ifnum\@tempcnta<10
      \edef\the@slidesmag{\the\@tempcnta}%
      \@slidesmag{\@magstep\@tempcnta}%
    \else
      \@semmagerr\slidesmag
    \fi
  \else
    \@semmagerr\slidesmag
  \fi
  \setslidelength\semin\seminlength
  \setslidelength\semcm\semcmlength}
\def\seminlength{1in}
\def\semcmlength{1cm}
\def\articlemag#1{%
  \@tempcnta=#1\relax
  \ifnum\@tempcnta>-6
    \ifnum\@tempcnta<10
      \edef\the@articlemag{\the\@tempcnta}%
      \@articlemag{\@magstep\@tempcnta}%
    \else
      \@semmagerr\articlemag
    \fi
  \else
    \@semmagerr\articlemag
  \fi}
\addto@preamblecmds{\slidesmag\do\articlemag}
\def\setslidelength#1#2{%
  #1=#2\relax
  #1=\inverseslidesmag#1}%
\def\addtoslidelength#1#2{%
  \dimen@=#2\relax
  \advance#1 by \inverseslidesmag\dimen@}
\def\setartlength#1#2{%
  #1=#2\relax
  #1=\inverseartmag#1}
\def\addtoartlength#1#2{%
  \dimen@=#2\relax
  \advance#1 by \inverseartmag\dimen@}
\def\slide@epsfsize#1#2{%
  \ifdim\epsfxsize=0pt
    \ifdim\epsfysize=0pt
      \inverseslidesmag#1%
    \else
      0pt
    \fi
  \else
    \inverseslidesmag\epsfxsize
  \fi
  \epsfysize
  \ifdim\epsfysize=0pt
    \ifdim\epsfxsize=0pt
      \inverseslidesmag#2%
    \else
      0pt
    \fi
  \else
    \inverseslidesmag\epsfysize
  \fi}
\def\epsfslidesize{\let\epsfsize\slide@epsfsize}
\slidesmag{4}
\articlemag{0}
\def\do@pageparameters{%
  \do\oddsidemargin
  \do\evensidemargin
  \do\marginparwidth
  \do\marginparsep
  \do\topmargin
  \do\headheight
  \do\headsep
  \do\textheight
  \do\textwidth
  \do\topskip
  \do\footskip
  \do\footheight}
\ifarticle
  \def\scale@pageparameters{%
    \begingroup
      \def\do##1{\global##1=\inverseartmag##1\relax}%
      \do@pageparameters
    \endgroup}
\else
  \def\scale@pageparameters{%
    \begingroup
      \def\do##1{\global##1=\inverseslidesmag##1\relax}%
      \do@pageparameters
    \endgroup}
\fi
\addto@hook\before@document{\scale@pageparameters}
\addto@preamblecmds{\scale@pageparameters\do\do@pageparameters}
\def\allversions{}
\let\endallversions\relax
\@namedef{allversions*}{\@bsphack\globaldefs=1}
\@namedef{endallversions*}{\@esphack}
\def\slide@list{slide,slide*,allversions,allversions*}
\def\addtoslidelist#1{\xdef\slide@list{\slide@list,#1}}
\addto@preamblecmds{\addtoslidelist}
\ifslidesonly
  \@ifundefined{xcomment@@@}{\@@input xcomment.sty }{}
  \def\xcomment@hook{\@xcomment{@@@}{\slide@list}}
  \newxcomment[]{note}
\else
  \def\note{\@bsphack}%
  \def\endnote{\@esphack}%
\fi
\def\noxcomment{\def\xcomment@hook{}}
\def\os@list{}
\newif\if@os
\def\onlyslides#1{\def\os@list{#1}\@ostrue
  \def\os@warning{\@warning{\string\onlyslides\space argument
    contains undefined references}}}
\def\notslides#1{\def\os@list{#1}\@osfalse
  \def\os@warning{\@warning{\string\notslides\space argument
    contains undefined references}}}
\addto@preamblecmds{\onlyslides\do\notslides}
\addto@hook\after@document{%
  \ifx\os@list\@empty\else\os@expandlist\fi}
\def\os@expandlist{%
  \let\os@@warning\relax
  \begingroup
    \def\ref##1{\@ifundefined{r@##1}{?}%
      {\noexpand\@car\@nameuse{r@##1}\noexpand\@nil}}%
    \edef\@tempd{\os@list}%
    \xdef\os@list{}%
    \@for\@tempc:=\@tempd
      \do{\expandafter\os@expandrange\@tempc-:-:\@nil}%
    \os@@warning
  \endgroup
  \let\os@expandrange\relax
  \let\os@checknum\relax
  \let\os@expandlist\relax}
\def\os@expandrange#1-#2-#3\@nil{%
  \def\@tempa{?}\def\@tempb{#1}%
  \ifx\@tempa\@tempb
    \let\os@@warning\os@warning
  \else
    \@tempcnta=#1\relax
    \def\@tempb{#2}%
    \ifx\@tempa\@tempb
      \let\os@@warning\os@warning
    \else
      \def\@tempa{:}%
      \ifx\@tempa\@tempb
        \@tempcntb=\@tempcnta
      \else
        \@tempcntb=#2\relax
      \fi
      \advance\@tempcnta by -1
      \advance\@tempcntb by 1
      \ifx\os@list\@empty
        \xdef\os@list{\the\@tempcnta+\the\@tempcntb}%
      \else
        \xdef\os@list{\os@list,\the\@tempcnta+\the\@tempcntb}%
      \fi
    \fi
  \fi}
\def\os@test{%
  \@testtrue
  \iflandscape
    \ifnum\@landscapeonly=-1 \@testfalse\fi
  \else
    \ifnum\@landscapeonly=1 \@testfalse\fi
  \fi
  \if@test
    \ifx\os@list\@empty\else
      \if@os\@testfalse\fi
      \@for\@tempa:=\os@list\do{\expandafter\os@testrange\@tempa\@nil}%
    \fi
  \fi}
\def\os@testrange#1+#2\@nil{%
  \ifnum\c@slide>#1
    \ifnum\c@slide<#2
      \if@os\@testtrue\else\@testfalse\fi
    \fi
  \fi}
\def\onlynotestoo{%
  \ifnotes\@testtrue\else\ifnotesonly\@testtrue\else\@testfalse\fi\fi
  \if@test
    \@ifundefined{xcomment@@@}{%
      \edef\sem@temp{\the\catcode`\@}%
      \catcode`\@=11
      \@@input xcomment.sty
      \catcode`\@=\sem@temp\relax}{}%
    \def\xcomment@hook{\@xcomment{@@@}{\slide@list}}%
    \addto@hook\after@slide\onlynotes@too
  \fi}
\def\onlynotes@too{%
  \os@test
  \if@test\gdef\do@end{}\else\gdef\do@end{\xc@begin}\fi}
\addto@preamblecmds\onlynotestoo
\def\ptsize#1{%
  \@ifundefined{#1@semptsize}%
    {\@seminarerr{\string\ptsize\space `#1' not valid.}\@eha}%
    {\ifslide
      \sem@ptsize{#1}\large\normalsize
    \else
      \edef\slide@ptsize{#1}%
    \fi}}
\edef\slide@ptsize{1\@ptsize}%
\def\slidefontsizes{\ptsize} %For backwards compatibility??
\def\slide@setsize#1#2#3#4{%
  \@setsize{#1}{#2}{#3}{#4}%
  \set@slideskip{#2}}
\def\slide@@setsize#1#2#3#4{%
  \slide@setsize{#1}{#2}{#3}{#4}\slidedisplayskips}
\def\set@slideskip#1{%
  \normallineskiplimit=#1
  \advance\normallineskiplimit-\normalbaselineskip
  \multiply\normallineskiplimit-1
  \normallineskiplimit\slideskip\normallineskiplimit
  \ifdim\normallineskiplimit<1pt\normallineskiplimit=1pt\fi
  \normallineskip=\normallineskiplimit
    minus \slideshrink\normallineskiplimit
  \dimen@=\normalbaselineskip
  \normalbaselineskip=\dimen@ minus \slideshrink\normallineskiplimit
  \normalbaselines}
\def\slideskip{.75}
\def\slideshrink{.25}
\def\slidedisplayskips{%
  \abovedisplayskip 1.75ex minus .35ex
  \belowdisplayskip \abovedisplayskip
  \abovedisplayshortskip .2ex minus .2ex
  \belowdisplayshortskip 1ex minus .2ex}
\def\sem@ptsize#1{%
  \@nameuse{#1@semptsize}%
  \large\normalsize
  \leftmargini\slideleftmargini\relax
  \leftmarginii\slideleftmarginii\relax
  \leftmarginiii\slideleftmarginiii\relax
  \labelsep\slidelabelsep\relax
  \parskip\slideparskip\relax
  \parindent\slideparindent\relax
  \slide@rightskip
  \slide@listi
  \skip\footins\slidefootins\relax
  \footnotesep\slidefootnotesep\relax}
\def\slidefootins{2ex minus .8ex}
\def\slidefootnotesep{1.2ex}
\def\slideparindent{\z@}
\def\slideparskip{1ex minus .2ex}
\@namedef{8@semptsize}{%
  \def\normalsize{\slide@@setsize\normalsize{9.5pt}\viiipt\@viiipt}%
  \def\small{\slide@@setsize\small{8pt}\viipt\@viipt}%
  \def\footnotesize{\slide@@setsize\footnotesize{8pt}\vipt\@vipt}%
  \def\scriptsize{\slide@setsize\scriptsize{7pt}\vipt\@vipt}%
  \def\tiny{\slide@setsize\tiny{6pt}\vpt\@vpt}%
  \def\large{\slide@setsize\large{12pt}\xpt\@xpt}%
  \def\Large{\slide@setsize\Large{14pt}\xiipt\@xiipt}%
  \def\LARGE{\slide@setsize\LARGE{17pt}\xivpt\@xivpt}%
  \def\huge{\slide@setsize\huge{20pt}\xviipt\@xviipt}%
  \def\Huge{\slide@setsize\Huge{25pt}\xxpt\@xxpt}}
\@namedef{9@semptsize}{%
  \def\normalsize{\slide@@setsize\normalsize{11pt}\ixpt\@ixpt}%
  \def\small{\slide@@setsize\small{9.5pt}\viiipt\@viiipt}%
  \def\footnotesize{\slide@@setsize\footnotesize{8pt}\viipt\@viipt}%
  \def\scriptsize{\slide@setsize\scriptsize{7pt}\vipt\@vipt}%
  \def\tiny{\slide@setsize\tiny{6pt}\vpt\@vpt}%
  \def\large{\slide@setsize\large{12pt}\xpt\@xpt}%
  \def\Large{\slide@setsize\Large{14pt}\xiipt\@xiipt}%
  \def\LARGE{\slide@setsize\LARGE{17pt}\xivpt\@xivpt}%
  \def\huge{\slide@setsize\huge{20pt}\xviipt\@xviipt}%
  \def\Huge{\slide@setsize\Huge{25pt}\xxpt\@xxpt}}
\@namedef{10@semptsize}{%
  \def\normalsize{\slide@@setsize\normalsize{12pt}\xpt\@xpt}%
  \def\small{\slide@@setsize\small{11pt}\ixpt\@ixpt}%
  \def\footnotesize{\slide@@setsize\footnotesize{9.5pt}\viiipt\@viiipt}%
  \def\scriptsize{\slide@setsize\scriptsize{8pt}\viipt\@viipt}%
  \def\tiny{\slide@setsize\tiny{6pt}\vpt\@vpt}%
  \def\large{\slide@setsize\large{14pt}\xiipt\@xiipt}%
  \def\Large{\slide@setsize\Large{18pt}\xivpt\@xivpt}%
  \def\LARGE{\slide@setsize\LARGE{22pt}\xviipt\@xviipt}%
  \def\huge{\slide@setsize\huge{25pt}\xxpt\@xxpt}%
  \def\Huge{\slide@setsize\Huge{30pt}\xxvpt\@xxvpt}}
\@namedef{11@semptsize}{%
  \def\normalsize{\slide@@setsize\normalsize{13.6pt}\xipt\@xipt}%
  \def\small{\slide@@setsize\small{12pt}\xpt\@xpt}%
  \def\footnotesize{\slide@@setsize\footnotesize{11pt}\ixpt\@ixpt}%
  \def\scriptsize{\slide@setsize\scriptsize{9.5pt}\viiipt\@viiipt}%
  \def\tiny{\slide@setsize\tiny{7pt}\vipt\@vipt}%
  \def\large{\slide@setsize\large{14pt}\xiipt\@xiipt}%
  \def\Large{\slide@setsize\Large{18pt}\xivpt\@xivpt}%
  \def\LARGE{\slide@setsize\LARGE{22pt}\xviipt\@xviipt}%
  \def\huge{\slide@setsize\huge{25pt}\xxpt\@xxpt}%
  \def\Huge{\slide@setsize\Huge{30pt}\xxvpt\@xxvpt}}
\@namedef{12@semptsize}{%
  \def\normalsize{\slide@@setsize\normalsize{14.5pt}\xiipt\@xiipt}%
  \def\small{\slide@@setsize\small{13.6pt}\xipt\@xipt}%
  \def\footnotesize{\slide@@setsize\footnotesize{12pt}\xpt\@xpt}%
  \def\scriptsize{\slide@setsize\scriptsize{9.5pt}\viiipt\@viiipt}%
  \def\tiny{\slide@setsize\tiny{7pt}\vipt\@vipt}%
  \def\large{\slide@setsize\large{18pt}\xivpt\@xivpt}%
  \def\Large{\slide@setsize\Large{22pt}\xviipt\@xviipt}%
  \def\LARGE{\slide@setsize\LARGE{25pt}\xxpt\@xxpt}%
  \def\huge{\slide@setsize\huge{30pt}\xxvpt\@xxvpt}%
  \let\Huge\huge}
\@namedef{14@semptsize}{%
  \def\normalsize{\slide@@setsize\normalsize{18pt}\xivpt\@xivpt}%
  \def\small{\slide@@setsize\small{14.5pt}\xiipt\@xiipt}%
  \def\footnotesize{\slide@@setsize\footnotesize{13.6pt}\xipt\@xipt}%
  \def\scriptsize{\slide@setsize\scriptsize{12pt}\xpt\@xpt}%
  \def\tiny{\slide@setsize\tiny{9.5pt}\viiipt\@viiipt}%
  \def\large{\slide@setsize\large{22pt}\xviipt\@xviipt}%
  \def\Large{\slide@setsize\Large{25pt}\xxpt\@xxpt}%
  \def\LARGE{\slide@setsize\LARGE{30pt}\xxvpt\@xxvpt}%
  \let\huge\LARGE
  \let\Huge\LARGE}
\@namedef{17@semptsize}{%
  \def\normalsize{\slide@@setsize\normalsize{22pt}\xviipt\@xviipt}%
  \def\small{\slide@@setsize\small{18pt}\xivpt\@xivpt}%
  \def\footnotesize{\slide@@setsize\footnotesize{14.5pt}\xiipt\@xiipt}%
  \def\scriptsize{\slide@setsize\scriptsize{13.6pt}\xipt\@xipt}%
  \def\tiny{\slide@setsize\tiny{12pt}\xpt\@xpt}%
  \def\large{\slide@setsize\large{25pt}\xxpt\@xxpt}%
  \def\Large{\slide@setsize\Large{30pt}\xxvpt\@xxvpt}%
  \let\LARGE\Large
  \let\huge\Large
  \let\Huge\Large}
\def\twoup{\@ifnextchar[{\@twoup}{\@twoup[0]}}
\ifarticle
  \ifportrait
    \def\@twoup[#1]{%
      \@tempcnta=\the@articlemag\relax
      \@tempcntb=\@tempcnta
      \advance\@tempcntb by #1\relax
      \advance\@tempcntb by -2\relax
      \ifnum\@tempcntb>-6\relax
        \@@input 2up.tex
        \source{\@magstep\@tempcnta}{\paperheight}{\paperwidth}%
        \target{\@magstep\@tempcntb}{\paperwidth}{\paperheight}%
        \targetlayout{topbottom}%
        \printlandscapefalse
      \else
        \@seminarerr{Article magnification is too low for
          \string\twoup\space to handle}\@eha
      \fi
      \def\articlemag##1{\@semtwoupmagerr\articlemag}}
  \else
    \def\@twoup[#1]{%
      \@tempcnta=\the@articlemag\relax
      \@tempcntb=\@tempcnta
      \advance\@tempcntb by #1\relax
      \advance\@tempcntb by -2\relax
      \ifnum\@tempcntb>-6\relax
        \@@input 2up.tex
        \source{\@magstep\@tempcnta}{\paperwidth}{\paperheight}%
        \target{\@magstep\@tempcntb}{\paperheight}{\paperwidth}%
        \if@twoside
          \targetlayout{twosided}%
        \fi
        \printlandscapetrue
      \else
        \@seminarerr{Article magnification is too low for
          \string\twoup\space to handle}\@eha
      \fi
      \def\articlemag##1{\@semtwoupmagerr\articlemag}}
  \fi
\else
  \ifportrait
    \def\@twoup[#1]{%
      \@tempcnta=\the@slidesmag\relax
      \@tempcntb=\@tempcnta
      \advance\@tempcntb by #1\relax
      \advance\@tempcntb by -3\relax
      \ifnum\@tempcntb>-6\relax
        \@@input 2up.tex
        \source{\@magstep\@tempcnta}{\paperwidth}{\paperheight}%
        \advance\@tempcnta by -3
        \target{\@magstep\@tempcnta}{\paperheight}{\paperwidth}%
        \printlandscapetrue
      \else
        \@seminarerr{Slides magnification is too low for
          \string\twoup\space to handle}\@eha
      \fi
      \def\slidesmag##1{\@semtwoupmagerr\slidesmag}}
  \else
    \def\@twoup[#1]{%
      \@tempcnta=\the@slidesmag\relax
      \@tempcntb=\@tempcnta
      \advance\@tempcntb by #1\relax
      \advance\@tempcntb by -3\relax
      \ifnum\@tempcntb>-6\relax
        \@@input 2up.tex
        \source{\@magstep\@tempcnta}{\paperheight}{\paperwidth}%
        \target{\@magstep\@tempcntb}{\paperwidth}{\paperheight}%
        \targetlayout{topbottom}%
        \printlandscapefalse
      \else
        \@seminarerr{Slides magnification is too low for
          \string\twoup\space to handle}\@eha
      \fi
      \def\slidesmag##1{\@semtwoupmargerr\slidesmag}}%
  \fi
\fi
\def\@semtwoupmagerr#1{%
  \@seminarerr{\string#1 must come before \string\twoup}\@eha}
\addto@preamblecmds{\twoup\do\@twoup}
%
\InputIfFileExists{seminar.con}{}
%\ifeof1\else\closein1 \relax\@@input seminar.con\fi
%
\endinput
%%
%% END `seminar.cls'.

%
%% BEGIN seminar.cls
%% LaTeX2e seminar class, derived by Sebastian Rahtz from
%% LaTeX document style `seminar', for use with LaTeX v2.09.
%% generated on <1993/10/8>. 
%% Fixes from David Carlisle, Denis Girou, 
%% and Frank Küster/Günter Milde (a4paper option fix)
%%
\def\fileversion{1.61}
\def\filedate{2014/01/17}
%%
%% This is a style for typesetting notes and slides.
%%
%% COPYRIGHT 1993, by Timothy Van Zandt, Timothy.VAN-ZANDT@insead.edu
%%
%% This file may be distributed and/or modified under the conditions of
%% the LaTeX Project Public License, either version 1.2 of this license
%% or (at your option) any later version.  The latest version of this
%% license is in:
%% 
%%    http://www.latex-project.org/lppl.txt
%% 
%% and version 1.2 or later is part of all distributions of LaTeX version
%% 1999/12/01 or later.
%%
%%
\NeedsTeXFormat{LaTeX2e}
\ProvidesClass{seminar}[\filedate, \fileversion]
\typeout{%
  Documentclass: `seminar' v\fileversion \space <\filedate> (tvz,hv)}
%%%%%%%%%%%%%%%% fix hv 
\def\tx@BOL{}
%%%%%%%%%%%%%%%
\def\test@member#1#2{%
  \edef\@tempg{,#2,#1,}%
  \edef\@temph{####1,#1,}%
  \expandafter\def\expandafter\@temph\@temph##2\@nil{%
    \def\@tempg{##2}%
    \ifx\@tempg\@empty\@testfalse\else\@testtrue\fi}%
  \expandafter\@temph\@tempg\@nil}
\def\addto@hook#1#2{#1\expandafter{\the#1#2}}
\@ifundefined{reset@font}{\def\reset@font{\normalsize\rm}}{}
\def\@seminarerr#1#2{%
  \edef\@tempc{#2}\expandafter\errhelp\expandafter{\@tempc}%
  \typeout{^^JSeminar.sty error.\space\space\space
    Type \space H <return> \space for immediate help.^^J}%
  \errmessage{#1^^J}}
\def\notslide@err#1{\message{Cannot use \string#1 in slide environments}}
\def\new@slidebox{\alloc@4\box\chardef\insc@unt}
\RequirePackage{pst-ovl}
%%
%\RequirePackage{geometry}
%\geometry{dvips,paperwidth=8.5in,paperheight=11in}
\newdimen\slidewidth \slidewidth 8.5in
\newdimen\slideheight \slideheight 6.3in
\paperwidth=8.5in
\paperheight=11in
%%
\def\addto@preamblecmds#1{%
  \begingroup
    \def\do{\noexpand\do\noexpand}%
    \xdef\@preamblecmds{\@preamblecmds\do#1}%
  \endgroup}
\addto@preamblecmds{\addto@preamblecmds\do\@preamblecmds}
\def\sem@temp#1{\@ifundefined{if#1}%
  {\def\@tempa{\csname newif\endcsname}
  \expandafter\@tempa\csname if#1\endcsname}{}}
\sem@temp{article}{}
\sem@temp{slidesonly}{}
\sem@temp{notes}{}
\sem@temp{notesonly}{}
\sem@temp{notesonlystar}{}
\sem@temp{portrait}{}
\DeclareOption{article}{\articletrue}
\DeclareOption{noxcomment}{\def\xcomment@hook{}}
\DeclareOption{slidesonly}{\slidesonlytrue\notesfalse\notesonlyfalse}
\DeclareOption{notes}{\notestrue\slidesonlyfalse\notesonlyfalse}
\DeclareOption{notesonly}{\notesonlytrue\slidesonlyfalse\notesfalse}
\DeclareOption{notesonly*}{%
  \notesonlytrue\slidesonlyfalse\notesfalse\notesonlystartrue}
\DeclareOption{portrait}{\portraittrue}
\DeclareOption{a4}{%
  \paperwidth 210mm
  \paperheight 297mm
%  \geometry{paperwidth=210mm,paperheight=297mm}
  \slidewidth 222mm
  \slideheight 152mm\relax}
\DeclareOption{a4paper}{\csname ds@a4\endcsname}%
\DeclareOption{semrot}{\AtEndOfClass{%% BEGIN sem-rot.sty
\def\fileversion{1.0}
\def\filedate{93/04/01}
%%
%% LaTeX style option `semrot.sty'.
%% For use with LaTeX v2.09 and the `seminar' document style.
%%
%% COPYRIGHT 1993, by Timothy Van Zandt, Timothy.VAN-ZANDT@insead.edu
%%
%% This file may be distributed and/or modified under the conditions of
%% the LaTeX Project Public License, either version 1.2 of this license
%% or (at your option) any later version.  The latest version of this
%% license is in:
%% 
%%    http://www.latex-project.org/lppl.txt
%% 
%% and version 1.2 or later is part of all distributions of LaTeX version
%% 1999/12/01 or later.
%%
%%
%% The following is just an interface for seminar.sty:
\def\leftsliderotation#1{\@rotate[l]{#1}}
\def\rightsliderotation#1{\@rotate[r]{#1}}
\sliderotation{left}
%%
%% And the rest is rotate.sty, as distributed with Rokicki's dvips.
%%
%   These macros allow you to rotate or flip a \TeX\ box.  Very useful for
%   sideways tables or upsidedown answers.
%
%   To use, create a box containing the information you want to rotate.
%   (An hbox or vbox will do.)  Now call \@rotr\boxnum to rotate the
%   material and create a new box with the appropriate (flipped) dimensions.
%   \@rotr rotates right, \@rotl rotates left, \@rotu turns upside down, and
%   \@rotf flips.  These boxes may contain other rotated boxes.
%
\newdimen\@rotdimen
\newbox\@rotbox

\def\@vspec#1{\special{ps:#1}}%  passes #1 verbatim to the output
\def\@rotstart#1{\@vspec{gsave currentpoint currentpoint translate
   #1 neg exch neg exch translate}}% #1 can be any origin-fixing transformation
\def\@rotfinish{\@vspec{currentpoint grestore moveto}}% gets back in synch
%
%   First, the rotation right. The reference point of the rotated box
%   is the lower right corner of the original box.
%
\def\@rotr#1{\@rotdimen=\ht#1\advance\@rotdimen by\dp#1%
   \hbox to\@rotdimen{\hskip\ht#1\vbox to\wd#1{\@rotstart{90 rotate}%
   \box#1\vss}\hss}\@rotfinish}
%
%   Next, the rotation left. The reference point of the rotated box
%   is the upper left corner of the original box.
%
\def\@rotl#1{\@rotdimen=\ht#1\advance\@rotdimen by\dp#1%
   \hbox to\@rotdimen{\vbox to\wd#1{\vskip\wd#1\@rotstart{270 rotate}%
   \box#1\vss}\hss}\@rotfinish}%
%
%   Upside down is simple. The reference point of the rotated box
%   is the upper right corner of the original box. (The box's height
%   should be the current font's xheight, \fontdimen5\font,
%   if you want that xheight to be at the baseline after rotation.)
%
\def\@rotu#1{\@rotdimen=\ht#1\advance\@rotdimen by\dp#1%
   \hbox to\wd#1{\hskip\wd#1\vbox to\@rotdimen{\vskip\@rotdimen
   \@rotstart{-1 dup scale}\box#1\vss}\hss}\@rotfinish}%
%
%   And flipped end for end is pretty ysae too. We retain the baseline.
%
\def\@rotf#1{\hbox to\wd#1{\hskip\wd#1\@rotstart{-1 1 scale}%
   \box#1\hss}\@rotfinish}%
%
%   Now the user interface for LaTeX:  \rotate[type]{text} where
%   ``type'' is `l' for left, `r' for right, `u' for upside-down, or
%   `f' for flip.  `l' is the default.
%
\def\rotate{\@ifnextchar[{\@rotate}{\@rotate[l]}}
\def\@rotate[#1]#2{\setbox\@rotbox=\hbox{#2}\@nameuse{@rot#1}\@rotbox}
\endinput
%% END sem-rot.sty
}}
\DeclareOption{semhelv}{\AtEndOfClass{%% BEGIN semhelv.sty
\def\fileversion{1.4}
\def\filedate{2000/05/08}
%%
%% LaTeX style option `semhelv.sty' for the `seminar' document class.
%% For use with LaTeX2e.
%%
%% COPYRIGHT 1994, by Timothy Van Zandt, Timothy.VAN-ZANDT@insead.edu
%%
%% This file may be distributed and/or modified under the conditions of
%% the LaTeX Project Public License, either version 1.2 of this license
%% or (at your option) any later version.  The latest version of this
%% license is in:
%% 
%%    http://www.latex-project.org/lppl.txt
%% 
%% and version 1.2 or later is part of all distributions of LaTeX version
%% 1999/12/01 or later.
%%
%% Edited by Sebastian Rahtz <sebastian.rahtz@oucs.ox.ac.uk>
%%   and Denis Girou <denis.girou@idris.fr>
%% To use OT1-encoding 7t names
%% ************************************************************************
%% ****                        DESCRIPTION                             ****
%% ************************************************************************
%%
% This file contains the external font information needed to load
% the PostScript Helvetica fonts for use by seminar.cls.
%
% IMPORTANT:
%   As distributed, the following font file names are assumed for the
%   `standard' TeX encoding:
%
%     Helvetica              phvr7t
%     Helvetica-Oblique      phvro7t
%     Helvetica-Bold         phvb7t
%     Helvetica-BoldOblique  phvbo7t
%     Courier                pcrr7t
%     Courier-Bold           pcrb7t
%     Courier-Oblique        pcrro7t
%     Courier-BoldOblique    pcrbo7t
%
%   With the NFSS2 and LaTeX2e, you can also use the CORK encoding. The
%   assumed file names are as above, but with `8t' instead of `7t'
%
%   This means, e.g., that phvr7t.tfm or phvr7t.vf must be the name of the
%   TeX .tfm or .vf (virtual font) file for the PostScript Helvetica font.
%   See the documentation of your dvi-to-ps driver for information on using
%   PostScript fonts.
%
% Fonts are scaled so that the x-heights match those of the cmr fonts;
% this makes in-line math look acceptable.
%
% Note: The Courier font is not that great.
%
% Preliminaries:
\csname semhelv@loaded\endcsname
\let\semhelv@loaded\endinput
\typeout{Style Option: `semhelv' for the `seminar' doc style
  \fileversion\space <\filedate> (tvz)}
\typeout{Modified by Stefan (see source for details)}
\@ifundefined{selectfont}
  {\@seminarerr{`semhelv' style option can only be used
    with the New Font Selection Scheme}\@eha\endinput}{}
%
% Font definitions.

\expandafter\ifx\csname DeclareSizeFunction\endcsname\relax

% FOR NFSS:

\@ifundefined{newpsfamily}{%% BEGIN npsfont.sty
%%
\def\fileversion{v1.0}
\def\filedate{91/11/20}
%%
%% Style option npsfont.sty, for use with LaTeX 2.09
%%   and the New Font Selection Scheme.
%%
%% COPYRIGHT 1992, by Timothy Van Zandt, Timothy.VAN-ZANDT@insead.edu
%%
%%
%% This file may be distributed and/or modified under the conditions of
%% the LaTeX Project Public License, either version 1.2 of this license
%% or (at your option) any later version.  The latest version of this
%% license is in:
%% 
%%    http://www.latex-project.org/lppl.txt
%% 
%% and version 1.2 or later is part of all distributions of LaTeX version
%% 1999/12/01 or later.
%%
%%
% DESCRIPTION:
%
% Defining new PostScript fonts with the New Font Selection Scheme is easy,
% but these macros make it easier. They save on keystrokes, and most
% importantly, they make it easy to change the file names used for fonts
% because you only have to type the file name once instead of 12 times (for
% the 12 standard LaTeX sizes). An optional argument lets you scale the fonts.
%
% These macros are substitutes for the NFSS's \new@fontshape command. For
% information on the purpose of this command, and on how to select fonts that
% are defined with it, see the documentation of the NFSS. If you don't have
% the NFSS, it is available from
%   rusmv1.rus.uni-stuttgart.de
% in
%   soft/tex/latex-style-supported/New-Font-Selection
% and from
%   ymir.claremont.edu
% in
%   [.anonymous.tex.inputs.latex-mainz]
% The NFSS is undoubtably the best way to use PostScript fonts (or any non-
% standard fonts) with LaTeX, and is well worth upgrading to.
%
% USAGE:
%
%   \newpsfont[<scaling>]{<family>}{<series>}{<shape>}{<filename>}
%
%   \newpsfamily[<scaling>]{<family>}{
%     {<series 1>}{<shape 1>}{<filename 1>},
%     {<series 2>}{<shape 2>}{<filename 2>},
%                 . . .
%     {<series n>}{<shape n>}{<filename n>}}
%
% The scaling argument is optional. Like the \new@fontshape command, these
% commands can only be used in the preamble.
%
% EXAMPLE:
%
% Suppose the Helvetica-Oblique virtual font on your system has the name
% phvro.vf. The following defines the <helvetica> family, <m>edium series, and
% <it>alic shape to be the Helvetica-Oblique font, at the standard LaTeX
% font sizes, scaled by .82:
%
%   \newpsfont[.82]{helvetica}{m}{it}{phvro}
%
% This is equivalent to:
%
%   \new@fontshape{helvetica}{m}{it}{%
%     <5>phvro at 4.1pt%
%     <6>phvro at 4.92pt%
%     <7>phvro at 5.74pt%
%     <8>phvro at 6.56pt%
%     <9>phvro at 7.38pt%
%     <10>phvro at 8.2pt%
%     <11>phvro at 8.979pt%
%     <12>phvro at 9.84pt%
%     <14>phvro at 11.808pt%
%     <17>phvro at 14.1696pt%
%     <20>phvro at 17.0068pt%
%     <25>phvro at 20.4016pt}{}
%
% The following command sets up the standard Helvetica family:
%
%   \newpsfamily[.82]{helvetica}{%
%     {m}{n}{phvr},                          %Helvetica
%     {m}{it}{phvro},                        %Helvetica-Oblique
%     {m}{sl}{phvro},                        %Helvetica-Oblique
%     {bx}{n}{phvb},                         %Helvetica-Bold
%     {bx}{it}{phvbo},                       %Helvetica-BoldOblique
%     {bx}{sl}{phvbo}}                       %Helvetica-BoldOblique
%
% (Don't forget the commas.) It is equivalent to:
%
%   \newpsfont[.82]{helvetica}{m}{n}{phvr}     %Helvetica
%   \newpsfont[.82]{helvetica}{m}{it}{phvro}   %Helvetica-Oblique
%   \newpsfont[.82]{helvetica}{m}{sl}{phvro}   %Helvetica-Oblique
%   \newpsfont[.82]{helvetica}{bx}{n}{phvb}    %Helvetica-Bold
%   \newpsfont[.82]{helvetica}{bx}{it}{phvbo}  %Helvetica-BoldOblique
%   \newpsfont[.82]{helvetica}{bx}{sl}{phvbo}  %Helvetica-BoldOblique
%   \extra@def{helvetica}{}{}
%
% WHY SCALE FONTS?:
%
% Disclaimer: The suggestions that follow come from someone who doesn't know
% anything about fonts.
%
% Mixing fonts that are not designed to be used together is not for purists,
% but many of us do it anyway. Scaling the fonts can at least correct for
% differences in the x-heights. This is important if you use Computer
% Modern fonts for in-line math and some other fonts for text.
%
% To find out the x-height of the 10pt size of a font, load this style and
% use the command
%
%   \showxheight{<filename>}
%
% E.g., \showxheight{phvr} and \showxheight{cmr10} displays
%
%   x-height of cmr10 at 10pt is 4.30554pt
%   x-height of phvr at 10pt is 5.24998pt
%
% on the terminal. Scaling the Helvetica fonts by 4.30554/5.24990 = .82
% gets the x-heights of these fonts in line. It is best to set the x-heights
% the same for the medium weight, normal shaped fonts, and then scale the
% rest of the fonts in the same family by the same amount.
%
% David Jones has a psfonts.sty that sets up the standard PostScript fonts,
% all scaled so that the x-heights match that of the Computer Modern medium
% roman font. It is available via anonymous ftp from theory.lcs.mit.edu in
% /pub/dmjones.
%
% IMPLEMENTATION:
%
% \scale@stdsizes has at its argument a scaling factor. It uses \font@scale
% to define \scaled@sizes to be the macro whose argument is the name of a
% font, and which expands to the definitions of the standard \LaTeX{} sizes
% for New Font Selection Schemes \new@fontshape command.
%
\message{\space\space\fileversion\space\space\filedate
  \space\space<tvz>}
\@ifundefined{new@psfontshape}{}{\endinput}
\def\font@scale#1#2#3{\dimen@=#2\dimen@=#3\dimen@
  \edef\scaled@sizes##1{\scaled@sizes{##1}<#1>##1 at\number\dimen@ sp}}
\def\scale@stdsizes#1{%
  \edef\scaled@sizes##1{}%
  \font@scale{5}{5pt}{#1}%
  \font@scale{6}{6pt}{#1}%
  \font@scale{7}{7pt}{#1}%
  \font@scale{8}{8pt}{#1}%
  \font@scale{9}{9pt}{#1}%
  \font@scale{10}{10pt}{#1}%
  \font@scale{11}{10.95pt}{#1}%
  \font@scale{12}{12pt}{#1}%
  \font@scale{14}{14.4pt}{#1}%
  \font@scale{17}{17.28pt}{#1}%
  \font@scale{20}{20.74pt}{#1}%
  \font@scale{25}{24.88pt}{#1}}
\def\new@psfontshape#1#2#3#4{%
  \new@fontshape{#1}{#2}{#3}{\scaled@sizes{#4}}{}}
\def\new@psfont[#1]#2#3#4#5{%
  \scale@stdsizes{#1}%
  \new@psfontshape{#2}{#3}{#4}{#5}}
\def\newpsfont{%
  \@ifnextchar [{\new@psfont}{\new@psfont[1]}}
\def\newpsfamily{%
  \@ifnextchar [{\new@psfamily}{\new@psfamily[1]}}
\def\new@psfamily[#1]#2#3{%
  \scale@stdsizes{#1}%
  \@for\@tempa:=#3\do{%
    \edef\@tempb{\noexpand\new@psfontshape{#2}\@tempa}
    \@tempb}%
  \extra@def{#2}{}{}}
\def\showxheight#1{%
  \font\@tempfont=#1 at 10pt%
  \typeout{^^J x-height of #1 at 10pt is \the\fontdimen5\@tempfont}}
% The commands are added to \@preamblecmds to save memory.
\begingroup
  \def\do{\noexpand\do\noexpand}
  \xdef\@preamblecmds{\@preamblecmds
    \do\font@scale
    \do\scale@stdsizes
    \do\scaled@sizes
    \do\new@psfontshape
    \do\new@psfont
    \do\newpsfont
    \do\newpsfamily
    \do\new@psfamily}
\endgroup
\endinput
%% END npsfont.sty
}{}%

% See npsfont.sty for the definition of \newpsfamily.

\newpsfamily[.82]{helvetica}{
  {m}{n}{phvr7t},          %Helvetica
  {m}{it}{phvro7t},        %Helvetica-Oblique
  {m}{sl}{phvro7t},        %Helvetica-Oblique
  {bx}{n}{phvb7t},         %Helvetica-Bold
  {bx}{it}{phvbo7t},       %Helvetica-BoldOblique
  {bx}{sl}{phvbo7t}}       %Helvetica-BoldOblique
\newpsfamily[.985]{courier}{
  {m}{n}{pcrr7t},          %Courier
  {bx}{n}{pcrb7t},         %Courier-Bold
  {m}{it}{pcrro7t},        %Courier-Oblique
  {bx}{it}{pcrbo7t}}       %Courier-BoldOblique

% Use fonts or use \slidefonts:
\ifarticle
  \renewcommand{\slidefonts}{%
    \renewcommand{\default@family}{helvetica}%
    \renewcommand{\rmdefault}{helvetica}%
    \renewcommand{\ttdefault}{courier}%
    \renewcommand{\sfdefault}{helvetica}%
    \family\default@family\selectfont}
\else
  \renewcommand{\default@family}{helvetica}%
  \renewcommand{\rmdefault}{helvetica}%
  \renewcommand{\ttdefault}{courier}%
  \renewcommand{\sfdefault}{helvetica}%
  \family\default@family\selectfont
\fi

\else

% FOR NFSS2 OR LATEX2E:

\def\helv@scale{.82}

\DeclareFontFamily{T1}{helvetica}{}%
\DeclareFontShape{T1}{helvetica}{m}{n}{<->s*[\helv@scale]phvr8t}{}%
\DeclareFontShape{T1}{helvetica}{m}{it}{<->s*[\helv@scale]phvro8t}{}%
\DeclareFontShape{T1}{helvetica}{m}{sc}{<->s*[\helv@scale]phvrc8t}{}%
\DeclareFontShape{T1}{helvetica}{b}{n}{<->s*[\helv@scale]phvb8t}{}%
\DeclareFontShape{T1}{helvetica}{b}{it}{<->s*[\helv@scale]phvbo8t}{}%
\DeclareFontShape{T1}{helvetica}{m}{sl}{<->ssub * helvetica/m/it}{}%
\DeclareFontShape{T1}{helvetica}{b}{sc}{<->sub * helvetica/m/sc}{}%
\DeclareFontShape{T1}{helvetica}{b}{sl}{<->ssub * helvetica/b/it}{}%
\DeclareFontShape{T1}{helvetica}{bx}{n}{<->ssub * helvetica/b/n}{}%
\DeclareFontShape{T1}{helvetica}{bx}{it}{<->ssub * helvetica/b/it}{}%
\DeclareFontShape{T1}{helvetica}{bx}{sc}{<->sub * helvetica/m/sc}{}%
\DeclareFontShape{T1}{helvetica}{bx}{sl}{<->ssub * helvetica/b/it}{}%

\DeclareFontFamily{OT1}{helvetica}{}%
\DeclareFontShape{OT1}{helvetica}{m}{n}{<->s*[\helv@scale]phvr7t}{}%
\DeclareFontShape{OT1}{helvetica}{m}{it}{<->s*[\helv@scale]phvro7t}{}%
\DeclareFontShape{OT1}{helvetica}{m}{sc}{<->s*[\helv@scale]phvrc7t}{}%
\DeclareFontShape{OT1}{helvetica}{b}{n}{<->s*[\helv@scale]phvb7t}{}%
\DeclareFontShape{OT1}{helvetica}{b}{it}{<->s*[\helv@scale]phvbo7t}{}%
\DeclareFontShape{OT1}{helvetica}{m}{sl}{<->ssub * helvetica/m/it}{}%
\DeclareFontShape{OT1}{helvetica}{b}{sc}{<->sub * helvetica/m/sc}{}%
\DeclareFontShape{OT1}{helvetica}{b}{sl}{<->ssub * helvetica/b/it}{}%
\DeclareFontShape{OT1}{helvetica}{bx}{n}{<->ssub * helvetica/b/n}{}%
\DeclareFontShape{OT1}{helvetica}{bx}{it}{<->ssub * helvetica/b/it}{}%
\DeclareFontShape{OT1}{helvetica}{bx}{sc}{<->sub * helvetica/m/sc}{}%
\DeclareFontShape{OT1}{helvetica}{bx}{sl}{<->ssub * helvetica/b/it}{}%

\def\courier@scale{.985}

% DG/SR modification begin - May. 13, 1998
% Update some names (phvrq -> phvr7t etc.)
% DG/SR modification end

\DeclareFontFamily{T1}{courier}{\hyphenchar \font\m@ne}%
\DeclareFontShape{T1}{courier}{m}{n}{<->s*[\courier@scale]pcrr8t}{}%
\DeclareFontShape{T1}{courier}{m}{it}{<->s*[\courier@scale]pcrro8t}{}%
\DeclareFontShape{T1}{courier}{m}{sc}{<->s*[\courier@scale]pcrrc8t}{}%
\DeclareFontShape{T1}{courier}{b}{n}{<->s*[\courier@scale]pcrb8t}{}%
\DeclareFontShape{T1}{courier}{b}{it}{<->s*[\courier@scale]pcrbo8t}{}%
\DeclareFontShape{T1}{courier}{m}{sl}{<->ssub * courier/m/it}{}%
\DeclareFontShape{T1}{courier}{b}{sc}{<->sub * courier/m/sc}{}%
\DeclareFontShape{T1}{courier}{b}{sl}{<->ssub * courier/b/it}{}%
\DeclareFontShape{T1}{courier}{bx}{n}{<->ssub * courier/b/n}{}%
\DeclareFontShape{T1}{courier}{bx}{it}{<->ssub * courier/b/it}{}%
\DeclareFontShape{T1}{courier}{bx}{sc}{<->sub * courier/m/sc}{}%
\DeclareFontShape{T1}{courier}{bx}{sl}{<->ssub * courier/b/it}{}%

\DeclareFontFamily{OT1}{courier}{\hyphenchar \font\m@ne}%
\DeclareFontShape{OT1}{courier}{m}{n}{<->s*[\courier@scale]pcrr7t}{}%
\DeclareFontShape{OT1}{courier}{m}{it}{<->s*[\courier@scale]pcrro7t}{}%
\DeclareFontShape{OT1}{courier}{m}{sc}{<->s*[\courier@scale]pcrrc7t}{}%
\DeclareFontShape{OT1}{courier}{b}{n}{<->s*[\courier@scale]pcrb7t}{}%
\DeclareFontShape{OT1}{courier}{b}{it}{<->s*[\courier@scale]pcrbo7t}{}%
\DeclareFontShape{OT1}{courier}{m}{sl}{<->ssub * courier/m/it}{}%
\DeclareFontShape{OT1}{courier}{b}{sc}{<->sub * courier/m/sc}{}%
\DeclareFontShape{OT1}{courier}{b}{sl}{<->ssub * courier/b/it}{}%
\DeclareFontShape{OT1}{courier}{bx}{n}{<->ssub * courier/b/n}{}%
\DeclareFontShape{OT1}{courier}{bx}{it}{<->ssub * courier/b/it}{}%
\DeclareFontShape{OT1}{courier}{bx}{sc}{<->sub * courier/m/sc}{}%
\DeclareFontShape{OT1}{courier}{bx}{sl}{<->ssub * courier/b/it}{}%

% Use fonts or use \slidefonts:
\ifarticle
  \renewcommand{\slidefonts}{%
    \renewcommand{\default@family}{helvetica}%
    \renewcommand{\rmdefault}{helvetica}%
    \renewcommand{\ttdefault}{courier}%
    \renewcommand{\sfdefault}{helvetica}%
    \fontfamily\default@family\selectfont}
\else
  \renewcommand{\default@family}{helvetica}%
  \renewcommand{\rmdefault}{helvetica}%
  \renewcommand{\ttdefault}{courier}%
  \renewcommand{\sfdefault}{helvetica}%
  \fontfamily\default@family\selectfont
\fi

\fi

\endinput
%% END semhelv.sty

}}
\DeclareOption{semcolor}{\AtEndOfClass{%%
%% This is file `semcolor.sty', version 1.01
%%
%% IMPORTANT NOTICE:
%%
%% semcolor.sty Copyright (C) 1993 Timothy Van Zandt
%%
%% This package may be distributed under the terms of the LaTeX Project 
%% Public License, as described in lppl.txt in the base LaTeX distribution.
%% Either version 1.0 or, at your option, any later version.
%%
\def\fileversion{1.02}
\def\filedate{2014/01/17}
%%
%% LaTeX style option `semcolor.sty'.
%% For use with LaTeX v2.09 and the `seminar' document style.
%%
%% COPYRIGHT 1993, by Timothy Van Zandt, Timothy.VAN-ZANDT@insead.edu
%%
%% This file may be distributed and/or modified under the conditions of
%% the LaTeX Project Public License, either version 1.2 of this license
%% or (at your option) any later version.  The latest version of this
%% license is in:
%% 
%%    http://www.latex-project.org/lppl.txt
%% 
%% and version 1.2 or later is part of all distributions of LaTeX version
%% 1999/12/01 or later.
%%
%%
%% You must install the PSTricks package v0.9 or later to use semcolor.sty.
%%
\@ifundefined{SemcolorFrameOps}{}{\endinput}
\typeout{Style Option: `semcolor' for doc style `seminar'
  \fileversion\space <\filedate> (tvz)}
% RN modification begin -- 2004/05/04; Rolf.Niepraschk@ptb.de
%\input{pstricks}
\ProvidesFile{semcolor.sty}[\filedate\space v\fileversion\space (tvz)]
%\RequirePackage{pstricks}
\RequirePackage{pst-ovl}
% RN modification end
%
% Set up the frame styles:
\def\SemcolorFrameOps{\psset{%
  fillstyle=none,%
  linewidth=\slideframewidth,%
  framesep=\slideframesep,%
  cornersize=absolute,%
  linearc=.5cm,%
  shadowsize=2pt,%
  shadowcolor=darkgray}}%
\newslideframe{scplain}[\SemcolorFrameOps]{\psframebox{#1}}
\newslideframe{scdouble}[\SemcolorFrameOps]{\psdblframebox{#1}}
\newslideframe{scshadow}[\SemcolorFrameOps]{\psshadowbox{#1}}
\slideframe{scplain}
%
% Rotation:
\def\leftsliderotation#1{\rotateleft{#1}}
\def\rightsliderotation#1{\rotateright{#1}}
\sliderotation{left}
%
% Color:
\addto@hook\after@slidepage{\pstVerb{0 setgray}}
%
% Overlays:
\def\make@overlay#1{%
  \setbox\@slidebox=\hbox{%
    \pst@initoverlay{#1}%
    \pst@Verb{ 0 \tx@BOL }%
    \copy\@@slidebox
    \pst@Verb{ -1 \tx@BOL }}}
\def\@overlay{\pst@overlay}
\addto@hook\before@newslide{\def\curr@overlay{0}}
\@ifundefined{overlaystrue}{}{\ifarticle\else\overlaystrue\layerstrue\fi}
%
\endinput
%% END semcolor.sty
}}
\DeclareOption{semlcmss}{\AtEndOfClass{%% BEGIN semlcmss.sty
\def\fileversion{1.0 (2e)}
\def\filedate{94/02/01}
%%
%% LaTeX style option `semlcmss.sty' for the `seminar' document style.
%% For use with LaTeX v2.09 and the New Font Selection Scheme (1 or 2)
%% or (some day) with LaTeX2e.
%%
%% COPYRIGHT 1994, by Timothy Van Zandt, Timothy.VAN-ZANDT@insead.edu
%%
%% This file may be distributed and/or modified under the conditions of
%% the LaTeX Project Public License, either version 1.2 of this license
%% or (at your option) any later version.  The latest version of this
%% license is in:
%% 
%%    http://www.latex-project.org/lppl.txt
%% 
%% and version 1.2 or later is part of all distributions of LaTeX version
%% 1999/12/01 or later.
%%
%%
%% This file contains the external font information needed to load
%% the SliTeX sans serif fonts for use by seminar.sty.
%%
\csname semlcmss@loaded\endcsname
\let\semlcmss@loaded\endinput
\typeout{Style Option: `semlcmss' \fileversion\space <\filedate> (tvz)}
\@ifundefined{selectfont}
  {\@seminarerr{`semlcmss' style option can only be used
    with the New Font Selection Scheme}\@eha\endinput}{}
%
\expandafter\ifx\csname DeclareSizeFunction\endcsname\relax

% First, the font definitions:
\new@fontshape{lcmss}{m}{n}{%
  <5>lcmss8 scaled 695%
  <6>lcmss8 scaled 833%
  <7>lcmss8 scaled 913%
  <8>lcmss8 %
  <9>lcmss8 scaled 1095%
  <10>lcmss8 scaled 1200%
  <11>lcmss8 scaled 1314%
  <12>lcmss8 scaled 1440%
  <14>lcmss8 scaled 1727%
  <17>lcmss8 scaled 2074%
  <20>lcmss8 scaled 2488%
  <25>lcmss8 scaled 2986%
  }{}
\extra@def{lcmss}{}{}
%
\new@fontshape{lcmss}{m}{it}{%
  <5>lcmssi8 scaled 695%
  <6>lcmssi8 scaled 833%
  <7>lcmssi8 scaled 913%
  <8>lcmssi8 %
  <9>lcmssi8 scaled 1095%
  <10>lcmssi8 scaled 1200%
  <11>lcmssi8 scaled 1314%
  <12>lcmssi8 scaled 1440%
  <14>lcmssi8 scaled 1727%
  <17>lcmssi8 scaled 2074%
  <20>lcmssi8 scaled 2488%
  <25>lcmssi8 scaled 2986%
  }{}
\extra@def{lcmss}{}{}
\subst@fontshape{lcmss}{m}{sl}{lcmss}{m}{it}
%
\new@fontshape{lcmss}{bx}{n}{%
  <5>lcmssb8 scaled 695%
  <6>lcmssb8 scaled 833%
  <7>lcmssb8 scaled 913%
  <8>lcmssb8 %
  <9>lcmssb8 scaled 1095%
  <10>lcmssb8 scaled 1200%
  <11>lcmssb8 scaled 1314%
  <12>lcmssb8 scaled 1440%
  <14>lcmssb8 scaled 1727%
  <17>lcmssb8 scaled 2074%
  <20>lcmssb8 scaled 2488%
  <25>lcmssb8 scaled 2986%
  }{}
\extra@def{cmssb}{}{}
\subst@fontshape{lcmss}{bx}{it}{lcmss}{bx}{n}
\subst@fontshape{lcmss}{bx}{sl}{lcmss}{bx}{n}

%
% Now define \slidefonts:
\ifarticle
  \def\slidefonts{%
    \def\default@family{lcmss}%
    \def\rmdefault{lcmss}%
    \def\sfdefault{lcmss}%
    \family\default@family\selectfont}
\else
    \def\default@family{lcmss}%
    \def\rmdefault{lcmss}%
    \def\sfdefault{lcmss}%
    \family\default@family\selectfont
\fi

\else

\DeclareFontFamily{OT1}{lcmss}{}{}
\DeclareFontShape{OT1}{lcmss}{m}{n}{
  <5>s*[1.1120]lcmss8
  <6>s*[1.1107]lcmss8
  <7>s*[1.0434]lcmss8
  <8>lcmss8
  <9>s*[0.9733]lcmss8
  <10><10.95><12><14.4><17.28><20.74><24.88>s*[0.96]lcmss8}{}

\DeclareFontShape{OT1}{lcmss}{m}{sl}{
  <5>s*[1.1120]lcmssi8
  <6>s*[1.1107]lcmssi8
  <7>s*[1.0434]lcmssi8
  <8>lcmssi8
  <9>s*[0.9733]lcmssi8
  <10><10.95><12><14.4><17.28><20.74><24.88>s*[0.96]lcmssi8}{}

\DeclareFontShape{OT1}{lcmss}{bx}{n}{
  <5>s*[1.1120]lcmssb8
  <6>s*[1.1107]lcmssb8
  <7>s*[1.0434]lcmssb8
  <8>lcmssb8
  <9>s*[0.9733]lcmssb8
  <10><10.95><12><14.4><17.28><20.74><24.88>s*[0.96]lcmssb8}{}

\DeclareFontShape{OT1}{lcmss}{m}{it}{<-> ssub * lcmss/m/sl }{}
\DeclareFontShape{OT1}{lcmss}{bx}{it}{<-> sub * lcmss/bx/n }{}
\DeclareFontShape{OT1}{lcmss}{bx}{sl}{<-> sub * lcmss/bx/n }{}

% Now define \slidefonts:
\ifarticle
  \def\slidefonts{%
    \def\default@family{lcmss}%
    \def\rmdefault{lcmss}%
    \def\sfdefault{lcmss}%
    \fontfamily\default@family\selectfont}
\else
    \def\default@family{lcmss}%
    \def\rmdefault{lcmss}%
    \def\sfdefault{lcmss}%
    \fontfamily\default@family\selectfont
\fi

\fi

% Now set up a few things for seminar.sty:
% These fonts are intended to be used with the 10pt font size
% declarations (and \magstep4), slightly modified:
\ptsize{10}\slidesmag{4}\articlemag{0}
\@namedef{10@semptsize}{%
  \def\@normalsize{\slide@@setsize\normalsize{12pt}\xpt\@xpt}%
  \def\small{\footnotesize}%
  \def\footnotesize{\slide@@setsize\footnotesize{9.5pt}\viiipt\@viiipt}%
  \def\scriptsize{\footnotesize}%
  \def\tiny{\slide@setsize\tiny{6pt}\vpt\@vpt}%
  \def\large{\slide@setsize\large{14pt}\xiipt\@xiipt}%
  \def\Large{\slide@setsize\Large{18pt}\xivpt\@xivpt}%
  \def\LARGE{\slide@setsize\LARGE{22pt}\xviipt\@xviipt}%
  \def\huge{\slide@setsize\huge{25pt}\xxpt\@xxpt}%
  \def\Huge{\huge}}
\def\semlcmss@warning#1{%
  \@warning{Can't use `\string#1' with the semlcmss option!}}
\def\ptsize#1{\semlcmss@warning\ptsize}
\def\slidesmag#1{\semlcmss@warning\slidesmag}
\def\articlemag#1{\semlcmss@warning\articlemag}

\endinput
%% END semlcmss.sty
}}
\DeclareOption{slidesec}{\AtEndOfClass{%% BEGIN slidesec.sty
\def\fileversion{v0.9}
\def\filedate{92/05/07}
%%
%% Style option slidesec.sty, for use with LaTeX 2.09 or SliTeX.
%%
%% COPYRIGHT 1992, by Timothy Van Zandt, Timothy.VAN-ZANDT@insead.edu
%%
%%
%% This file may be distributed and/or modified under the conditions of
%% the LaTeX Project Public License, either version 1.2 of this license
%% or (at your option) any later version.  The latest version of this
%% license is in:
%% 
%%    http://www.latex-project.org/lppl.txt
%% 
%% and version 1.2 or later is part of all distributions of LaTeX version
%% 1999/12/01 or later.
%%
%%
% DISCLAIMER:
%
% Someone who actually uses something like this should rewrite it.
%
% DESCRIPTION:
%
% This file contains macros for sections, headings, tables of contents,
% and lists of slides, for slides. It can probably be used with your
% favorite LaTeX slide package (e.g., seminar.sty).
% However:
%%   If the counter used to keep track of slides is not  slide  ,
%%   then change the following line:
\def\@SCTR{\theslide}
%%
% ACKNOWLEDGEMENTS:
%
% The idea comes from Young U. Ryu's lslides.sty.
%
% USAGE:
%
% \slideheading[#1]{#2}
%  * Analogous to \section.
%  * [#1] is optional; \slideheading{#1} is same as \slideheading[#1]{#2}.
%  * Advances slidesection counter.
%  * Sets \theslideheading to #1; can be used, e.g., in headers and footers.
%  * Adds #1 to slide contents file.
%  * If #2 is non-empty, invokes \makeslideheading{#2} to make a heading.
%    Default definition of \makeslideheading is
%       \begin{center}\large\bf #2 \end{center}
%    but you can redefine this to suit your tastes.
%
% \slidesubheading[#1]{#2}
%  * Analogous to \subsection.
%  * Like \slideheading, but add "sub" to all the names.
%
% \slideheading and \slidesubheading are meant to be used at most once in
% a slide, but this is not a binding restriction.
%
% You may want to define abbreviated forms. E.g.,
%   \newcommand{heading}{\slideheading}
%   \newcommand{subheading}{\slidesubheading}
%
% \listofslides
%  * Analogous to \listoftables.
%
% \slidecontents
%  * Analogous to \tableofcontents
%
% \Slidecontents
%  * List \slidecontents, but puts some check marks and stuff to
%    indicate what you have covered so far and what is coming up.
%    Meant to be used in the middle of a presentation.
%
% To change fonts and otherwise customize what you get from \listofslides,
% \slidecontents, and \Slidecontents, you have to hack at the definitions
% of these macros.
%
% BUGS:
%
% These macros have not been tested very much.
%
% IMPLEMENTATION:
%
\@ifundefined{@startlos}{}{\endinput}
\typeout{File `slidesec.sty' \fileversion\space\space
  \filedate\space\space <tvz>}
%
% If we use \@starttoc{los} to read in \jobname.los, then its contents
% are destroyed and we can't read it again. \@startlos is like \@starttoc,
% but it sets the flae \los@flag to 1, and then we add a few commands
% to \enddocument that open \filename.los if \los@flag=1.
%
\def\los@flag{0}%
\def\@startlos{%
  \begingroup\makeatletter\@input{\jobname.los}\endgroup
  \gdef\los@flag{1}\global\@nobreakfalse}
\newtoks\@temptokenb
\@temptokena{%
  \ifnum\los@flag=1\if@filesw
    \newwrite\tf@los\immediate\openout\tf@los\jobname.los\relax
  \fi\fi}
\expandafter\@temptokenb\expandafter{\enddocument}
\edef\enddocument{\the\@temptokena\the\@temptokenb}
%
\newcounter{slidesection}
\newcounter{slidesubsection}
\def\theslidesection{\arabic{slidesection}}
\def\thesubslidesection{\theslidesection.\arabic{subslidesection}}
%
\def\slideheading{\@dblarg\slide@heading}%
\def\slide@heading[#1]#2{%
  \stepcounter{slidesection}%
  \addtocontents{los}{\protect\l@slide
    {\the\c@slidesection}{\ignorespaces#1}{\@SCTR}}%
  \gdef\theslideheading{#1}%
  \def\@tempa{#2}%
  \ifx\@tempa\@empty\else
    {\edef\@currentlabel{\csname p@slidesection\endcsname\theslidesection}%
      \makeslideheading{#2}}%
  \fi}
\def\slidesubheading{\@dblarg\slide@subheading}
\def\slide@subheading[#1]#2{%
  \stepcounter{slidesection}%
  \addtocontents{los}{\protect\l@subslide
    {\the\c@slidesubsection}{\ignorespaces#1}{\@SCTR}}%
  \gdef\theslidesubheading{#1}%
  \def\@tempa{#2}%
  \ifx\@tempa\@empty\else
    {\edef\@currentlabel{\csname p@slidesubsection\endcsname
      \theslidesubsection}\makeslideheading{#2}}%
  \fi}
\def\theslideheading{}\def\theslidesubheading{}
%
\def\makeslideheading#1{%
  \begin{center}\large\bf #1\end{center}}
\def\makeslidesubheading{\makeslideheading}
%
\def\@undottedtocline#1#2#3#4#5{\vskip \z@ plus .2pt
  \bgroup
    \leftskip #2\relax \parindent #2\relax\@afterindenttrue
    \interlinepenalty\@M
    \leavevmode
    \@tempdima #3\relax \advance\leftskip \@tempdima
    \hbox{}\hskip -\leftskip
    #4\par
  \egroup}
\def\slide@dottedcline{\@dottedtocline{1}{1.5em}{2.3em}}
\def\slide@undottedcline{\@undottedtocline{1}{1.5em}{2.3em}}
\def\slide@cline{\slide@dottedcline}
%
\def\slidenumberline#1{\hbox to\@tempdima{\hss#1\hskip 1em}}
%
\def\listofslides{\section*{List of Slides%
  \@mkboth{LIST OF SLIDES}{LIST OF SLIDES}}%
  \def\l@slide##1##2##3{%
    \slide@undottedcline{\slidenumberline{##3}{##2}}{}}%
  \let\l@subslide\l@slide
  \@startlos}
%
\def\slidecontents{\@ifstar%
  {\def\slide@cline{\slide@undottedcline}\slide@contents}%
  {\def\slide@cline{\slide@dottedcline}\slide@contents}}
\def\slide@contents{%
  \def\l@slide##1##2##3{%
    \slide@cline{\slidenumberline{$\bullet$}{##2}}{##3}}%
  \let\l@subslide\@gobbletwo
  \@startlos}
%
\def\Slidecontents{\@ifstar%
  {\def\slide@cline{\slide@undottedcline}\Slide@contents}%
  {\def\slide@cline{\slide@dottedcline}\Slide@contents}}
\def\Slide@contents{%
  \def\l@slide##1##2##3{%
    \ifcase\lslide@flag
      \message{##1 ** \the\c@slidesection}
      \ifnum##1>\c@slidesection
        \def\lslide@flag{1}{\large\slide@cline{%
          \slidenumberline{$\Rightarrow\bullet$}{##2}}{##3}}%
      \else
        {\large\slide@cline{%
          \slidenumberline{$\surd\;\bullet$}{##2}}{##3}}%
      \fi
    \or
      \def\lslide@flag{2}{\large\slide@cline{%
        \slidenumberline{$\bullet$}{##2}}{##3}}%
    \or
      {\large\slide@cline{%
        \slidenumberline{$\bullet$}{##2}}{##3}}%
    \fi}%
  \def\l@subslide##1##2##3{%
    \ifnum\lslide@flag=1
      \@undottedtocline{2}{3.8em}{3.2em}{##2}{}%
    \fi}%
  \def\lslide@flag{0}%
  \@startlos}
%
\endinput
%% END slidesec.sty
}}
\DeclareOption{semlayer}{\AtEndOfClass{%% BEGIN semlayer.sty
%%
%% This is file `semlayer.sty', generated
%% on <1993/4/2> with the docstrip utility (2.0r).
%%
%% The original source files were:
%%
%% semlayer.doc
%%
\def\fileversion{1.0}
\def\filedate{93/04/01}
%%
%% LaTeX style option `semlayer.sty'.
%% For use with LaTeX v2.09 and the `seminar' document style.
%%
%% COPYRIGHT 1993, by Timothy Van Zandt, Timothy.VAN-ZANDT@insead.edu
%%
%% This file may be distributed and/or modified under the conditions of
%% the LaTeX Project Public License, either version 1.2 of this license
%% or (at your option) any later version.  The latest version of this
%% license is in:
%% 
%%    http://www.latex-project.org/lppl.txt
%% 
%% and version 1.2 or later is part of all distributions of LaTeX version
%% 1999/12/01 or later.
%%
%%
\addto@hook\before@newslide{%
  \global\def\overlay@list{}%
  \setcounter{overlay}{0}}
\newbox\@@slidebox
\def\process@@slide{%
  \ifx\overlay@list\@empty
    \finish@slide
    \output@slide
  \else
    \setbox\@@slidebox\box\@slidebox
    \make@overlay{0}%
    \finish@slide
    \output@slide
    {\let\theoverlay\theslide\layer@loop{0}}%
    \overlay@loop
  \fi}
\def\overlay@loop{%
  \begingroup
    \overlay@prep
    \c@overlay=1
    \@whilenum\c@overlay<10
      \do{%
        \test@member{\the\c@overlay}{\overlay@list}%
        \if@test
          \make@overlay{\the\c@overlay}%
          \finish@slide
          \output@slide
        \fi
        \layer@loop{\the\c@overlay}%
        \advance\c@overlay by 1}%
  \endgroup}
\def\layer@loop#1{%
  \begingroup
    \layer@prep
    \@for\currlayer:=\layer@list\do{%
    \test@member{#1-\currlayer}{\overlay@list}%
      \if@test
        \make@overlay{#1-\currlayer}%
        \finish@slide
        \output@slide
      \fi}%
  \endgroup}
\def\layer@prep{%
  \ifarticle\pagestyle{\layer@pagestyle}\fi
  \let\thepage\thelayer
  \ifx\thelayerframe\@empty\else\let\theslideframe\thelayerframe\fi
  \let\slidelabel\layerlabel}
\def\overlay@prep{%
  \ifarticle\pagestyle{\overlay@pagestyle}\fi
  \let\thepage\theoverlay
  \ifx\theoverlayframe\@empty\else\let\theslideframe\theoverlayframe\fi
  \let\slidelabel\overlaylabel}
\def\layerframe{\@slideframe{layer}}
\def\overlayframe{\@slideframe{overlay}}
\def\thelayerframe{}
\def\theoverlayframe{}
\def\overlaylabel{\bf Overlay \theoverlay}
\def\layerlabel{\bf Layer \thelayer}
\@ifundefined{c@overlay}{\newcounter{overlay}}{}%%% hv 20131224
%\newcounter{overlay}
\def\theoverlay{\theslide-\alph{overlay}}
\def\currlayer{}
\def\thelayer{\theoverlay-\currlayer}
\newif\ifoverlays
\newif\iflayers
\@ifundefined{@overlay}%
  {\let\@overlay\relax}%
  {\ifarticle\else\overlaystrue\layerstrue\fi}
\def\nooverlays@err{%
  \@seminarerr{You need to use additional macros for overlays,
    e.g., semcolor.sty}\@ehd
  \global\overlaysfalse
  \global\layersfalse}
\@ifundefined{@overlay}{}{}
\def\overlay#1{%
  \ifoverlays\ifx\@overlay\relax\nooverlays@err\fi\fi
  \ifslide
    \@tempcnta=#1
    \@testfalse
    \ifnum\@tempcnta>-1 \ifnum\@tempcnta<10 \@testtrue\fi\fi
    \if@test
      \c@overlay=\@tempcnta
      \ifnum\c@overlay=0
        \advance\c@slide by -1
        \refstepcounter{slide}%
      \else
        \advance\c@overlay by -1
        \refstepcounter{overlay}%
        \ifoverlays
          \xdef\overlay@list{\overlay@list,\the\c@overlay}%
        \fi
      \fi
      \ifoverlays
        \@overlay{\the\c@overlay}%
      \fi
    \else
      \@seminarerr{\string\overlay\space argument must be
        an integer between 0 and 9}\@eha
    \fi
  \else
    \@seminarerr{\string\overlay\space can only be used
      in slide environments}\@eha
  \fi
  \ignorespaces}
\def\colorlayers#1{%
  \edef\@tempc{#1}%
  \edef\layer@list{\layer@list,#1}%
  \def\@tempa##1{%
    \@namedef{##1}{%
      \iflayers\ifx\@overlay\relax\nooverlays@err\fi\fi
      \ifslide
        \iflayers
          \xdef\overlay@list{\overlay@list,\the\c@overlay-##1}%
          \@overlay{\the\c@overlay-##1}%
        \fi
      \else
        \@seminarerr{Color layer `##1' can only be used in
          slide environments}\@eha
      \fi}}
  \@for\@tempb:=\@tempc\do{%
    \expandafter\@tempa\expandafter{\@tempb}}}
\def\layer@list{}
\def\overlaypagestyle#1{\@ifgoodps{#1}{\edef\overlay@pagestyle{#1}}}
\def\layerpagestyle#1{\@ifgoodps{#1}{\edef\layer@pagestyle{#1}}}
\overlaypagestyle{}
\layerpagestyle{}

\endinput
%%
%% End of file `semlayer.sty'.
}}
\DeclareOption*{\PassOptionsToClass{\CurrentOption}{article}}
\ProcessOptions
\ifnotesonly\else\notesonlystarfalse\fi
\newdimen\footheight  % seminar.sty v1.0 uses it, but its not in LaTeX2e.
% Don't want these changed by article.cls:
\let\sem@paperheight\paperheight
\let\sem@paperwidth\paperwidth
\let\paperheight\footheight
\let\paperwidth\footheight
\LoadClass{article}

\RequirePackage{pst-ovl}%    hv 20131224

\renewcommand\normalsize{%
   \@setfontsize\normalsize\@xpt\@xiipt
   \abovedisplayskip 10\p@ \@plus2\p@ \@minus5\p@
   \abovedisplayshortskip \z@ \@plus3\p@
   \belowdisplayshortskip 6\p@ \@plus3\p@ \@minus3\p@
   \belowdisplayskip \abovedisplayskip
   \let\@listi\@slide@listi}
\let\paperheight\sem@paperheight
\let\paperwidth\sem@paperwidth
\RequirePackage{sem-page}
\newtoks\before@document
\newtoks\after@document
\let\xcomment@hook\relax
\before@document{\endgroup\the\before@document\begingroup}
\after@document{\the\after@document\xcomment@hook}
\expandafter\@temptokena\expandafter{\document}
\edef\document{\the\before@document\the\@temptokena\the\after@document}
\before@document{}
\after@document{}
\newtoks\before@enddocument
\before@enddocument{\the\before@enddocument}
\expandafter\@temptokena\expandafter{\enddocument}
\edef\enddocument{\the\before@enddocument\the\@temptokena}
\before@enddocument{}
\newif\ifprintlandscape
\ifportrait
  \ifarticle\printlandscapetrue\fi
\else
  \ifarticle\else\printlandscapetrue\fi
\fi
\addto@hook\before@document{\ifprintlandscape\printlandscape\fi}
\addto@preamblecmds{\printlandscape}
\def\printlandscape{\addto@hook\before@enddocument{%
  \typeout{^^J%
  *******************************************************^^J%
  ***** !! PRINT THIS DOCUMENT IN LANDSCAPE MODE !! *****^^J%
  *******************************************************}}}
\def\slide{%
  \NestedSlide@Error{slide}%
  \landscapetrue
  \@ifnextchar[{\begin@slide}{\begin@slide[\slidewidth,\slideheight]}}
\def\endslide{\end@slide}
\@namedef{slide*}{%
  \NestedSlide@Error{slide*}%
  \landscapefalse
  \@ifnextchar[{\begin@slide}{\begin@slide[\slidewidth,\slideheight]}}%
\@namedef{endslide*}{\end@slide}
\newcounter{slide}
\def\theslide{\arabic{slide}}
\newcount\slide@count
\newbox\@slidebox
\newbox\not@slidebox
\newif\ifslide
\newif\iflandscape
\def\@landscapeonly{0}
\def\landscapeonly{\def\@landscapeonly{1}}
\def\portraitonly{\def\@landscapeonly{-1}}
\def\NestedSlide@Error#1{%
  \ifslide
    \endgroup
    \@seminarerr{Nested slide environments. Perhaps missing
      \string\end{\@currenvir}. May be fatal}\@ehd
    \expandafter\end\expandafter{\@currenvir}%
    \ifslide\expandafter\end\expandafter{\@currenvir}\fi
    \begingroup
    \def\@currenvir{#1}%
  \fi}
\def\begin@slide[#1,#2]{%
  \slide@clearpage
  \setlength\slidewidth{#1}%
  \setlength\slideheight{#2}%
  \begingroup
    \ifarticle
      \output{%
        \advance\count@ 1
        \global\setbox\not@slidebox\box\@cclv}%
      \par\@@par\penalty-\@M
    \fi
    \output{\slide@output}%
    \slidetrue
    \ifarticle\global\slide@count=\z@\fi
    \refstepcounter{slide}%
    \ifnotesonlystar\xdef\first@slidemarker{\the\c@slide}\fi
    \def\do##1{\setcounter{##1}\z@}\slide@reset
    \ifarticle\else
      \edef\page@textheight{\number\textheight sp}%
      \edef\page@textwidth{\number\textwidth sp}%
    \fi
    \set@slidesize
    \slidebox@restore
    \the\slide@hook
    \the\before@newslide
    \everyslide}
\def\end@slide{%
    \par\penalty-\@M
    \xdef\@tempg{\@currenvir}%
  \endgroup
  \ifslide
    \@seminarerr{Perhaps missing `\string\end{\@tempg}',
      \iffalse{\fi`\string}' or `\string\endgroup'}\@ehd
    \def\next{\endgroup\ifslide\expandafter\next\fi}%
    \next
  \fi
  \begingroup
    \output{\setbox\@tempboxa\box\@cclv}%
    \@@par\penalty-\@M
  \endgroup
  \global\advance\c@slide-1
  \def\do##1{\setcounter{##1}{\@nameuse{saved@c@##1}}}%
  \slide@reset
  \ifarticle\outputloop@savedslides\fi
  \the\after@slide}
\let\slideclearpagetrue\relax
\let\slideclearpagefalse\relax
\ifarticle
  \def\slide@clearpage{\par\penalty\z@}
  \ifnotes
    \def\slideclearpagetrue{\def\slide@clearpage{\clearpage}}
    \def\slideclearpagefalse{\def\slide@clearpage{\par\penalty\z@}}
  \else
    \ifnotesonly
      \def\slideclearpagetrue{\def\slide@clearpage{\clearpage}}
      \def\slideclearpagefalse{\def\slide@clearpage{\par\penalty\z@}}
    \fi
  \fi
\else
  \def\slide@clearpage{\clearpage}
\fi
\newtoks\slide@hook
\def\everyslide{}
\newtoks\after@slide
\def\slidebox@restore{%
  \def\thepage{\theslide}%
  \def\newpage{\newslide}%
  \def\clearpage{\newslide}%
  \def\thispagestyle##1{\notslide@err{\thispagestyle}}%
  \pagestyle{\slide@pagestyle}%
  \@twocolumnfalse
  \def\twocolumn{\notslide@err{\twocolumn}}%
  \def\onecolumn{\notslide@err{\onecolumn}}%
  \def\marginpar##1{\notslide@err{\marginpar}}%
  \def\thanks{\slidethanks}%
  \def\maketitle{\slidemaketitle}%
  \fix@floats
  \fix@whatsits
  \slide@footnotes
  \def\do##1{\expandafter\xdef\csname
   saved@c@##1\endcsname{\the\@nameuse{c@##1}}}%
  \slide@reset
  \topskip\z@ \maxdepth\z@
  \slide@listparameters
  \slidefonts
  \def\baselinestretch{\slidestretch}%
  \def\arraystretch{\slidearraystretch}%
  \sem@ptsize{\slide@ptsize}}
\def\date#1{\gdef\@date{#1}\gdef\thedate{#1}}
\def\author#1{\gdef\@author{#1}\gdef\theauthor{#1}}
\def\title#1{\gdef\@title{#1}\gdef\thetitle{#1}}
\date{\today}
\let\slidethanks\thanks
\def\thethanks{\@thanks}%
\def\slidemaketitle{%
  \par
  \begin{center}\bf
    {\large \thetitle}\par\vskip 1ex
    \begin{tabular}[t]{c} \theauthor \end{tabular}\par\vskip 1ex
    \thedate
  \end{center}%
  \thethanks\par}
\def\fix@floats{%
  \def\@xfloat##1[##2]{%
    \expandafter\let\csname end##1\endcsname\end@float
    \par\medskip\vbox\bgroup\def\@captype{##1}\parindent\z@
    \ignorespaces}%
  \def\end@float{\par\vskip\z@\egroup\medskip}%
  \def\@dblfloat{\@float}\def\end@dblfloat{\end@float}%
  \def\endfigure{\end@float}\def\endtable{\end@float}}
\let\normal@write\write
\let\normal@read\read
\let\normal@openout\openout
\let\normal@closeout\closeout
% changed as per DPC 1997/09/16
\def\fix@whatsits{%
  \long\def \protected@write##1##2##3{%
      \begingroup
       ##2%
       \let\protect\noexpand
       \immediate\normal@write##1{##3}%
      \endgroup
      \if@nobreak\ifvmode\nobreak\fi\fi}%
}
% end change
\newinsert\slide@footins
\skip\slide@footins=\bigskipamount
\count\slide@footins=1000
\dimen\slide@footins=4in
\def\theslidefootnote{\alph{footnote}}
\def\slide@footnotes{%
  \def\thefootnote{\theslidefootnote}%
  \let\footins\slide@footins
  \interfootnotelinepenalty\@M}
\def\slidefonts{}
\def\slidestretch{1.18}
\def\slidearraystretch{1.2}
\def\raggedslides{\@ifnextchar[{\@raggedslides}{\@raggedslides[1fil]}}
\def\@raggedslides[#1]{%
  \edef\slide@@rightskip{#1}%
  \ifslide\slide@rightskip\fi}
\def\slide@rightskip{%
  \@rightskip\z@ plus \slide@@rightskip\relax \rightskip\@rightskip}
\def\slide@@rightskip{1fil}
\newcount\slide@listdepth
\def\slide@listparameters{%
  \let\@listdepth\slide@listdepth
  \slide@listdepth\z@
  \def\@listi{\slide@listi}%
  \def\@listii{\slide@listii}%
  \def\@listiii{\slide@listiii}%
  \let\@listiv\relax\let\@listv\relax\let\@listvi\relax}
\def\slide@listi{%
  \leftmargin\leftmargini
  \labelwidth\leftmargini \advance\labelwidth-\labelsep
  \parsep\parskip \divide\parsep2
  \partopsep\slidepartopsep\relax
  \advance\partopsep-\parskip
  \ifdim\partopsep<\z@\partopsep\z@\fi
  \itemsep\slideitemsep\relax
  \ifdim\parsep<\itemsep
    \topsep\itemsep \advance\topsep-\parsep
  \else
    \itemsep\parsep \topsep\z@
  \fi}
\def\slide@listii{%
  \leftmargin\leftmarginii
  \labelwidth\leftmarginii \advance\labelwidth-\labelsep
  \divide\itemsep2 \divide\parsep2
  \divide\topsep2 \divide\partopsep2\relax}%
\def\slide@listiii{%
  \leftmargin\leftmarginiii
  \labelwidth\leftmarginiii \advance\labelwidth-\labelsep
  \itemsep \z@ \parsep\z@ \topsep\z@}%
\def\slideleftmargini{1.8em}
\def\slideleftmarginii{1.4em}
\def\slideleftmarginiii{1em}
\def\slidelabelsep{.5em}
\def\slideitemsep{.8ex minus .2ex}
\def\slidepartopsep{1ex minus .2ex}
\newbox\saved@specials
\def\save@slidespecials{%
  \begingroup
    \output{%
      \global\setbox\saved@specials=\box\@cclv
      \global\wd\saved@specials=\z@
      \global\dp\saved@specials=\z@
      \global\ht\saved@specials=\z@}%
    \hbox{}\penalty-\@M
    \global\let\saved@texttop\@texttop
    \gdef\@texttop{%
      \ifvoid\saved@specials\else\box\saved@specials\nointerlineskip\fi
      \saved@texttop
      \global\let\@texttop\saved@texttop}%
  \endgroup}
\addto@hook\after@document{\save@slidespecials}
\ifarticle\else
  \def\insert@specials{%
    \ifvoid\saved@specials\else
      \setbox\@slidebox\hbox{\box\saved@specials\box\@slidebox}%
    \fi
    \global\let\insert@specials\relax}
\fi
\def\extraslideheight#1{%
  \@tempdima #1\relax
  \edef\X@SlideHeight{\number\@tempdima sp}
  \ifslide\set@slidesize\fi}
\extraslideheight{6pt}
\def\set@slidesize{%
  \iflandscape
    \hsize=\inverseslidesmag\slidewidth
    \vsize=\inverseslidesmag\slideheight
  \else
    \hsize=\inverseslidesmag\slideheight
    \vsize=\inverseslidesmag\slidewidth
  \fi
  \edef\slide@vsize{\number\vsize sp}%
  \textheight\vsize
  \advance\vsize\X@SlideHeight\relax
  \textwidth\hsize\columnwidth\hsize\linewidth\hsize}
\def\newslide{%
  \par\penalty-\@M
  \def\do##1{\setcounter{##1}\z@}\slide@reset
  \the\before@newslide
  \set@slidesize}
\newtoks\before@newslide
\def\slide@reset{\do{footnote}}
\def\slidereset#1{\def\slide@reset{}\addtoslidereset{#1}}
\def\addtoslidereset#1{%
  \edef\@tempa{#1}\expandafter\addto@slidereset\@tempa,\@nil,}
\def\addto@slidereset#1,{%
  \ifx\@nil#1\else
    \@ifundefined{c@#1}%
      {\@seminarerr{Counter `#1' not defined}\@ehd}%
      {\expandafter\def\expandafter\slide@reset\expandafter{%
        \slide@reset\do{#1}}}%
    \expandafter\addto@slidereset
  \fi}
\def\slide@output{%
  \@makeslide
  \@testfalse
  \ifnotesonly\else
    \iflandscape
      \ifnum\@landscapeonly>-1 \os@test\fi
    \else
      \ifnum\@landscapeonly<1 \os@test\fi
    \fi
  \fi
  \if@test
    \reset@slideoutput
    \@@makeslide
    \process@slide
  \fi
  \refstepcounter{slide}}
\def\reset@slideoutput{%
  \let\par\@@par
  \reset@font
  \def\baselinestretch{1}%
  \@nameuse{1\@ptsize @semptsize}%
  \catcode`\ =10
  \let\-\@dischyph \let\'\@acci \let\`\@accii \let\=\@acciii}
\newif\ifcenterslides
\centerslidestrue
\providecommand\@begindvi{} % added 1997/04/15 SPQR, to cope with very
                            % old LaTeXs
\def\@makeslide{%
  \setbox\@slidebox\vbox{%
    \@begindvi % added 1997/04/15 SPQR
    \unvbox\@cclv
    \ifvoid\slide@footins\else
      \vskip\skip\slide@footins
      \footnoterule
      \unvbox\slide@footins
    \fi
    \vskip\z@}}
\def\@@makeslide{%
  \overfullslide@warning
  \setbox\@slidebox\hbox{%
    \vfuzz=\slidefuzz\relax
    \vbox to\slide@vsize{%
    \ifcenterslides\vskip\z@ plus .0001fil \fi
    \unvbox\@slidebox
    \vskip\z@ plus .0001fil}%
    \the\after@slidepage}%
  \wd\@slidebox\textwidth}
\def\overfullslide@warning{%
  \dimen@\ht\@slidebox
  \advance\dimen@-\slide@vsize\relax
  \ifdim\dimen@>\slidefuzz\relax
    \@warning{Slide \theslide\space overfull by \the\dimen@}%
  \fi}
\def\slidefuzz{2pt}
\newtoks\after@slidepage
\newif\ifrotateheaders
\def\sliderotation#1{\@ifundefined{semsr@#1}%
  {\@latexerr{Slide rotation `#1' not defined.}\@eha}%
  {\@nameuse{semsr@#1}}}
\def\semsr@left{%
  \def\rotate@slide{%
    \setbox\@slidebox\hbox{\leftsliderotation{\box\@slidebox}}}}
\def\semsr@right{%
  \def\rotate@slide{%
    \setbox\@slidebox\hbox{\rightsliderotation{\box\@slidebox}}}}
\def\semsr@none{\let\rotate@slide\relax}
\sliderotation{none}
\def\leftsliderotation#1{%
  \@seminarerr{\string\leftsliderotation\space has not been defined}%
  \@ehd}
\let\rightsliderotation\leftsliderotation
\def\@ifrotateslide#1{%
  \ifx\rotate@slide\relax\else
    \iflandscape\ifportrait#1\fi\else\ifportrait\else#1\fi\fi
  \fi}
\def\process@slide{%
  \slideframewidth=\inverseslidesmag\slideframewidth
  \slideframesep=\inverseslidesmag\slideframesep
  \fboxrule\slideframewidth
  \fboxsep\slideframesep
  \ifarticle
    \@ifrotateslide\rotate@slide
  \else
    \ifrotateheaders\else\@ifrotateslide\rotate@slide\fi
  \fi
  \process@@slide}
\def\process@@slide{\finish@slide\output@slide}
\def\finish@slide{%
  \theslideframe
  \add@slidelabel\slidelabel}
\ifarticle
  \def\output@slide{%
    \global\advance\slide@count1
    \@ifundefined{slidebox@\the\slide@count}%
      {{\globaldefs=1\expandafter
        \new@slidebox\csname slidebox@\the\slide@count\endcsname}}{}%
    \expandafter\global\expandafter\setbox\csname
      slidebox@\the\slide@count\endcsname\box\@slidebox}
\else
  \def\output@slide{%
    \begingroup
      \hoffset=-\inverseslidesmag in
      \voffset=-\inverseslidesmag in
      \setslidelength\@tempdima{%
        \ifportrait\paperwidth\else\paperheight\fi}
      \setslidelength\@tempdimb{%
        \ifportrait\paperheight\else\paperwidth\fi}
      \ifrotateheaders
        \@ifrotateslide{%
          \dimen@=\@tempdima
          \@tempdima=\@tempdimb
          \@tempdimb=\dimen@}
      \fi
      % \oddsidemargin, \evensidemargin, \headheight, \footheight
      % used for scratch:
      \setslidelength\oddsidemargin\slideleftmargin
      \setslidelength\evensidemargin\sliderightmargin
      \setslidelength\headheight\slidetopmargin
      \setslidelength\footheight\slidebottommargin
      % Some page styles like to know \textwidth:
      \textwidth=\@tempdima
      \advance\textwidth-\oddsidemargin
      \advance\textwidth-\evensidemargin
      \setbox\@slidebox=\hbox to \@tempdima{%
        \kern\oddsidemargin
        \vbox to\@tempdimb{%
          \ifnum\fancyput@flag>-1
            \hbox{\kern-\oddsidemargin\do@fancyput}%
          \fi
          \let\label\@gobble
          \let\index\@gobble
          \let\glossary\@gobble
          \vbox to\headheight{%
            \vfill
            \hbox{%
              \slideheadfont\relax\strut
              \hbox to\textwidth{\@oddhead}}%
            \kern\z@}%
          \vfill
          \hbox to\textwidth{\hss\box\@slidebox\hss}%
          \vfill
          \vbox to\footheight{%
            \hbox{%
              \slidefootfont\relax\strut
              \hbox to\textwidth{\@oddfoot}}%
            \vfill}}%
        \hss}%
      \ifrotateheaders\@ifrotateslide\rotate@slide\fi
      \insert@specials
      \shipout\box\@slidebox
    \endgroup
    \let\firstmark\botmark}
  \@ifundefined{fancyput@flag}{\def\fancyput@flag{-1}}{}
\fi
\newskip\slidesep
\slidesep\intextsep
\ifarticle
  \def\fps@fslide{htbp}
  \def\ftype@fslide{32}
  \def\float@savedslide{%
    \begingroup\@float{fslide}%
      \centerline{\box\@slidebox}%
    \end@float\endgroup}%
  \@namedef{float*@savedslide}{%
    \begingroup\@dblfloat{fslide}%
      \centerline{\box\@slidebox}%
    \end@dblfloat\endgroup}%
  \def\center@slide{\hbox{%
    \kern-\@totalleftmargin
    \hbox to \columnwidth{\hss\box\@slidebox\hss}}}%
  \def\onepercol@savedslide{%
    \vbox to .996\textheight{\vss\center@slide\vss}\goodbreak}%
  \def\twopercol@savedslide{%
    \dimen@.5\textheight
    \advance\dimen@-\slidesep
    \ifdim\ht\@slidebox>\dimen@
      \onepercol@savedslide
    \else
      \vbox to .498\textheight{\vss\center@slide\vss}\goodbreak
    \fi}
  \def\here@savedslide{%
    \addvspace\slidesep\center@slide\addvspace\slidesep}
  \@namedef{here*@savedslide}{%
    \goodbreak \hrule \@height\z@ \nobreak \vskip\slidesep \nobreak
    \center@slide
    \nobreak \vskip\slidesep \nobreak \hrule\@height\z@ \goodbreak}
\fi
\ifarticle
  \def\slideplacement#1{\@ifundefined{#1@savedslide}%
    {\@seminarerr{Slide placement `#1' undefined}\@ehd}%
    {\expandafter\let\expandafter\output@savedslide
      \csname #1@savedslide\endcsname}}
\else
  \def\slideplacement#1{}
\fi
\ifarticle
  \ifnotes
    \ifportrait
      \slideplacement{float}
    \else
      \slideplacement{float*}
    \fi
  \else
    \ifportrait
      \slideplacement{onepercol}
    \else
      \slideplacement{twopercol}
    \fi
  \fi
\fi
\ifarticle
  \def\outputloop@savedslides{%
    \global\maxdepth\@maxdepth
    \ifvoid\not@slidebox\else
      \dimen@=\dp\not@slidebox
      \unvbox\not@slidebox
      \hrule height\z@
      \prevdepth\dimen@
      \penalty\z@
    \fi
    \edef\slide@@count{\the\slide@count\relax}%
    \slide@count\z@
    \loop
    \ifnum\slide@count<\slide@@count
      \advance\slide@count1
      \expandafter\setbox\expandafter\@slidebox\expandafter\box
        \csname slidebox@\the\slide@count\endcsname
      \output@savedslide
    \repeat
    \ifnotesonlystar\make@slidemarker\fi}
\fi
\def\make@slidemarker{%
  \addvspace\slidesep
  \moveleft\@totalleftmargin
  \vbox{%
    \hsize\columnwidth
    \hrule height 1pt
    \kern 8pt
    \hbox to \columnwidth{%
      \hss
      \LARGE\bf\the@slidemarker
      \hss}%
    \kern 8pt
    \hrule height 1pt}%
  \addvspace\slidesep}
\def\the@slidemarker{%
  Slide%
  \ifnum\c@slide=\first@slidemarker\else
    s {\c@slide\first@slidemarker\relax\theslide} --\fi
  { }\theslide}%
\ifarticle
  \let\c@note\c@page
  \def\p@note{\p@page}
  \def\thenote{\thepage}
\else
  \newcounter{note}
  \def\thenote{\theslide-\arabic{note}}
  \def\thepage{\thenote}
  \addto@hook\after@slide{\setcounter{note}{1}}
  \expandafter\def\expandafter\@outputpage\expandafter{%
    \@outputpage\stepcounter{note}}
\fi
\ifarticle\else
  \let\c@page\c@slide
  \countdef\c@slide=0
  \c@slide=0
  \c@page=1
\fi
\ifarticle
  \let\truepagenumbers\relax
\else
  \def\truepagenumbers{%
    \let\c@slide\c@page
    \countdef\c@page=0
    \c@page=1
    \c@slide=0
    \let\truepagenumbers\relax}
\fi
\addto@preamblecmds{\truepagenumbers}
\newdimen\slideframewidth \slideframewidth 4pt
\newdimen\slideframesep \slideframesep .3in
\def\newslideframe#1{%
  \@ifnextchar[{\@newslideframe{#1}}{\@newslideframe{#1}[]}}
\def\@newslideframe#1[#2]{%
  \@namedef{semsfops@#1}{#2}%
  \@namedef{semsf@#1}##1}
\newslideframe{plain}{\fbox{#1}}
\def\slideframe{\@slideframe{slide}}
\def\@slideframe#1{%
  \@ifstar{\@testtrue\@@slideframe{#1}}{\@testfalse\@@slideframe{#1}}}
\def\@@slideframe#1{%
  \@ifnextchar[{\@@@slideframe{#1}}{\@@@slideframe{#1}[]}}
\def\@@@slideframe#1[#2]#3{%
  \def\@tempa{none}%
  \def\@tempb{#3}%
  \ifx\@tempa\@tempb
    \@namedef{the#1frame}{\relax}%
  \else
    \ifx\@tempb\@empty
      \@namedef{the#1frame}{}%
    \else
      \@ifundefined{semsf@#3}%
        {\@seminarerr{Slide frame `#3' undefined}\@eha}%
        {\if@test
          \@@@@slideframe{#1}[#2]{#3}%
        \else
          \@namedef{the#1frame}{\setbox\@slidebox=\hbox{{%
            \@nameuse{semsfops@#3}#2\@nameuse{semsf@#3}{\box\@slidebox}}}}%
        \fi}%
     \fi
   \fi}
\def\@@@@slideframe#1[#2]#3{%
  \expandafter\let\expandafter\@tempa\csname the#1frame\endcsname
  \edef\next{%
    \noexpand\def\expandafter\noexpand\csname the#1frame\endcsname}%
  \expandafter\next\expandafter{\@tempa
    \setbox\@slidebox=\hbox{{%
      \@nameuse{semsfops@#3}%
      #2%
      \@nameuse{semsf@#3}{\box\@slidebox}}}}}%
\slideframe{plain}
\def\slidestyle#1{\@ifundefined{ss@#1}%
  {\@seminarerr{Slide style `#1' undefined}\@eha}%
  {\@nameuse{ss@#1}}}
\def\ss@empty{\let\add@slidelabel\@gobble}
\def\ss@left{\def\add@slidelabel##1{%
  \setbox\@slidebox=\hbox{%
    \vbox to \ht\@slidebox{\vss
    \hbox to 0pt{\hss##1\hskip 15pt}%
    \vss}\box\@slidebox}}}
\def\ss@bottom{\def\add@slidelabel##1{%
  \setbox\@slidebox=\vbox{\copy\@slidebox\vskip 9pt
    \hbox to\wd\@slidebox{\hss##1\hss}}}}%
\ifarticle
  \ifportrait\slidestyle{bottom}\else\slidestyle{left}\fi
\else
  \slidestyle{empty}
\fi
\def\slidelabel{\bf Slide \theslide}
\def\newpagestyle#1#2#3{%
  \expandafter\newcommand\csname ps@#1\endcsname{%
    \def\@oddhead{#2}\let\@evenhead\@oddhead
    \def\@oddfoot{#3}\let\@evenfoot\@oddfoot}}
\def\renewpagestyle#1#2#3{%
  \expandafter\renewcommand\csname ps@#1\endcsname{%
    \def\@oddhead{#2}\let\@evenhead\@oddhead
    \def\@oddfoot{#3}\let\@evenfoot\@oddfoot}}
\def\@ifgoodps#1{%
  \@ifundefined{ps@#1}{\@seminarerr{Page style `#1' undefined}\@eha}}
\def\slidepagestyle#1{%
  \@ifgoodps{#1}%
    {\ifslide\pagestyle{#1}\else\edef\slide@pagestyle{#1}\fi}}
\def\ps@{}
\slidepagestyle{}
\ifarticle
  \def\ps@align{}
\else
  \def\ps@align{%
    \def\@oddhead{\thepage\hfil+}\let\@evenhead\@oddhead
    \def\@oddfoot{+\hfil+}\let\@evenfoot\@oddfoot}
\fi
\def\slideheadfont{\scriptsize}
\def\slidefootfont{\scriptsize}
\def\magstep#1{\ifcase#1 \@m\or 1200\or 1440\or 1728\or
  2074\or 2488\or 2986\or 3583\or 4300\or 5160\fi\relax}
\def\magstepminus#1{%
  \ifcase#1 \@m\or 833\or 694\or 579\or 482\or 401\fi\relax}
\def\@magstep#1{%
  \ifnum#1<\z@\magstepminus{-#1}\else\magstep#1\fi}
{\catcode`\p=12\catcode`\t=12
  \gdef\@@inv@@mag#1pt#2{\def#2{#1}}}
\def\invert@mag#1{\@tempdima=1000pt
  \divide\@tempdima by #1\relax
  \expandafter\@@inv@@mag\the\@tempdima#1}
\def\@slidesmag#1{%
  \@tempcnta=#1\relax%
  \ifnum\@tempcnta>0
    \edef\inverseslidesmag{\the\@tempcnta}%
    \invert@mag\inverseslidesmag
    \ifarticle\else\mag\@tempcnta\fi
  \else
    \@seminarerr{\string\@slidesmag\space argument must be an
      integer equal to 1000 times the magnification}\@eha
  \fi}
\def\@articlemag#1{%
  \@tempcnta=#1\relax%
  \ifnum\@tempcnta>0
    \edef\inverseartmag{\the\@tempcnta}%
    \invert@mag\inverseartmag
    \ifarticle\mag\@tempcnta\fi
  \else
    \@seminarerr{\string\articlemag\space argument must be an
      integer equal to 1000 times the magnification}\@eha
  \fi}
\addto@preamblecmds{\@slidesmag\do\@articlemag}
\newdimen\semin
\newdimen\semcm
\def\@semmagerr#1{%
  \@seminarerr{\string#1 argument must be an integer
    between -5 and 9}\@eha}
\def\slidesmag#1{%
  \@tempcnta=#1\relax
  \ifnum\@tempcnta>-6
    \ifnum\@tempcnta<10
      \edef\the@slidesmag{\the\@tempcnta}%
      \@slidesmag{\@magstep\@tempcnta}%
    \else
      \@semmagerr\slidesmag
    \fi
  \else
    \@semmagerr\slidesmag
  \fi
  \setslidelength\semin\seminlength
  \setslidelength\semcm\semcmlength}
\def\seminlength{1in}
\def\semcmlength{1cm}
\def\articlemag#1{%
  \@tempcnta=#1\relax
  \ifnum\@tempcnta>-6
    \ifnum\@tempcnta<10
      \edef\the@articlemag{\the\@tempcnta}%
      \@articlemag{\@magstep\@tempcnta}%
    \else
      \@semmagerr\articlemag
    \fi
  \else
    \@semmagerr\articlemag
  \fi}
\addto@preamblecmds{\slidesmag\do\articlemag}
\def\setslidelength#1#2{%
  #1=#2\relax
  #1=\inverseslidesmag#1}%
\def\addtoslidelength#1#2{%
  \dimen@=#2\relax
  \advance#1 by \inverseslidesmag\dimen@}
\def\setartlength#1#2{%
  #1=#2\relax
  #1=\inverseartmag#1}
\def\addtoartlength#1#2{%
  \dimen@=#2\relax
  \advance#1 by \inverseartmag\dimen@}
\def\slide@epsfsize#1#2{%
  \ifdim\epsfxsize=0pt
    \ifdim\epsfysize=0pt
      \inverseslidesmag#1%
    \else
      0pt
    \fi
  \else
    \inverseslidesmag\epsfxsize
  \fi
  \epsfysize
  \ifdim\epsfysize=0pt
    \ifdim\epsfxsize=0pt
      \inverseslidesmag#2%
    \else
      0pt
    \fi
  \else
    \inverseslidesmag\epsfysize
  \fi}
\def\epsfslidesize{\let\epsfsize\slide@epsfsize}
\slidesmag{4}
\articlemag{0}
\def\do@pageparameters{%
  \do\oddsidemargin
  \do\evensidemargin
  \do\marginparwidth
  \do\marginparsep
  \do\topmargin
  \do\headheight
  \do\headsep
  \do\textheight
  \do\textwidth
  \do\topskip
  \do\footskip
  \do\footheight}
\ifarticle
  \def\scale@pageparameters{%
    \begingroup
      \def\do##1{\global##1=\inverseartmag##1\relax}%
      \do@pageparameters
    \endgroup}
\else
  \def\scale@pageparameters{%
    \begingroup
      \def\do##1{\global##1=\inverseslidesmag##1\relax}%
      \do@pageparameters
    \endgroup}
\fi
\addto@hook\before@document{\scale@pageparameters}
\addto@preamblecmds{\scale@pageparameters\do\do@pageparameters}
\def\allversions{}
\let\endallversions\relax
\@namedef{allversions*}{\@bsphack\globaldefs=1}
\@namedef{endallversions*}{\@esphack}
\def\slide@list{slide,slide*,allversions,allversions*}
\def\addtoslidelist#1{\xdef\slide@list{\slide@list,#1}}
\addto@preamblecmds{\addtoslidelist}
\ifslidesonly
  \@ifundefined{xcomment@@@}{\@@input xcomment.sty }{}
  \def\xcomment@hook{\@xcomment{@@@}{\slide@list}}
  \newxcomment[]{note}
\else
  \def\note{\@bsphack}%
  \def\endnote{\@esphack}%
\fi
\def\noxcomment{\def\xcomment@hook{}}
\def\os@list{}
\newif\if@os
\def\onlyslides#1{\def\os@list{#1}\@ostrue
  \def\os@warning{\@warning{\string\onlyslides\space argument
    contains undefined references}}}
\def\notslides#1{\def\os@list{#1}\@osfalse
  \def\os@warning{\@warning{\string\notslides\space argument
    contains undefined references}}}
\addto@preamblecmds{\onlyslides\do\notslides}
\addto@hook\after@document{%
  \ifx\os@list\@empty\else\os@expandlist\fi}
\def\os@expandlist{%
  \let\os@@warning\relax
  \begingroup
    \def\ref##1{\@ifundefined{r@##1}{?}%
      {\noexpand\@car\@nameuse{r@##1}\noexpand\@nil}}%
    \edef\@tempd{\os@list}%
    \xdef\os@list{}%
    \@for\@tempc:=\@tempd
      \do{\expandafter\os@expandrange\@tempc-:-:\@nil}%
    \os@@warning
  \endgroup
  \let\os@expandrange\relax
  \let\os@checknum\relax
  \let\os@expandlist\relax}
\def\os@expandrange#1-#2-#3\@nil{%
  \def\@tempa{?}\def\@tempb{#1}%
  \ifx\@tempa\@tempb
    \let\os@@warning\os@warning
  \else
    \@tempcnta=#1\relax
    \def\@tempb{#2}%
    \ifx\@tempa\@tempb
      \let\os@@warning\os@warning
    \else
      \def\@tempa{:}%
      \ifx\@tempa\@tempb
        \@tempcntb=\@tempcnta
      \else
        \@tempcntb=#2\relax
      \fi
      \advance\@tempcnta by -1
      \advance\@tempcntb by 1
      \ifx\os@list\@empty
        \xdef\os@list{\the\@tempcnta+\the\@tempcntb}%
      \else
        \xdef\os@list{\os@list,\the\@tempcnta+\the\@tempcntb}%
      \fi
    \fi
  \fi}
\def\os@test{%
  \@testtrue
  \iflandscape
    \ifnum\@landscapeonly=-1 \@testfalse\fi
  \else
    \ifnum\@landscapeonly=1 \@testfalse\fi
  \fi
  \if@test
    \ifx\os@list\@empty\else
      \if@os\@testfalse\fi
      \@for\@tempa:=\os@list\do{\expandafter\os@testrange\@tempa\@nil}%
    \fi
  \fi}
\def\os@testrange#1+#2\@nil{%
  \ifnum\c@slide>#1
    \ifnum\c@slide<#2
      \if@os\@testtrue\else\@testfalse\fi
    \fi
  \fi}
\def\onlynotestoo{%
  \ifnotes\@testtrue\else\ifnotesonly\@testtrue\else\@testfalse\fi\fi
  \if@test
    \@ifundefined{xcomment@@@}{%
      \edef\sem@temp{\the\catcode`\@}%
      \catcode`\@=11
      \@@input xcomment.sty
      \catcode`\@=\sem@temp\relax}{}%
    \def\xcomment@hook{\@xcomment{@@@}{\slide@list}}%
    \addto@hook\after@slide\onlynotes@too
  \fi}
\def\onlynotes@too{%
  \os@test
  \if@test\gdef\do@end{}\else\gdef\do@end{\xc@begin}\fi}
\addto@preamblecmds\onlynotestoo
\def\ptsize#1{%
  \@ifundefined{#1@semptsize}%
    {\@seminarerr{\string\ptsize\space `#1' not valid.}\@eha}%
    {\ifslide
      \sem@ptsize{#1}\large\normalsize
    \else
      \edef\slide@ptsize{#1}%
    \fi}}
\edef\slide@ptsize{1\@ptsize}%
\def\slidefontsizes{\ptsize} %For backwards compatibility??
\def\slide@setsize#1#2#3#4{%
  \@setsize{#1}{#2}{#3}{#4}%
  \set@slideskip{#2}}
\def\slide@@setsize#1#2#3#4{%
  \slide@setsize{#1}{#2}{#3}{#4}\slidedisplayskips}
\def\set@slideskip#1{%
  \normallineskiplimit=#1
  \advance\normallineskiplimit-\normalbaselineskip
  \multiply\normallineskiplimit-1
  \normallineskiplimit\slideskip\normallineskiplimit
  \ifdim\normallineskiplimit<1pt\normallineskiplimit=1pt\fi
  \normallineskip=\normallineskiplimit
    minus \slideshrink\normallineskiplimit
  \dimen@=\normalbaselineskip
  \normalbaselineskip=\dimen@ minus \slideshrink\normallineskiplimit
  \normalbaselines}
\def\slideskip{.75}
\def\slideshrink{.25}
\def\slidedisplayskips{%
  \abovedisplayskip 1.75ex minus .35ex
  \belowdisplayskip \abovedisplayskip
  \abovedisplayshortskip .2ex minus .2ex
  \belowdisplayshortskip 1ex minus .2ex}
\def\sem@ptsize#1{%
  \@nameuse{#1@semptsize}%
  \large\normalsize
  \leftmargini\slideleftmargini\relax
  \leftmarginii\slideleftmarginii\relax
  \leftmarginiii\slideleftmarginiii\relax
  \labelsep\slidelabelsep\relax
  \parskip\slideparskip\relax
  \parindent\slideparindent\relax
  \slide@rightskip
  \slide@listi
  \skip\footins\slidefootins\relax
  \footnotesep\slidefootnotesep\relax}
\def\slidefootins{2ex minus .8ex}
\def\slidefootnotesep{1.2ex}
\def\slideparindent{\z@}
\def\slideparskip{1ex minus .2ex}
\@namedef{8@semptsize}{%
  \def\normalsize{\slide@@setsize\normalsize{9.5pt}\viiipt\@viiipt}%
  \def\small{\slide@@setsize\small{8pt}\viipt\@viipt}%
  \def\footnotesize{\slide@@setsize\footnotesize{8pt}\vipt\@vipt}%
  \def\scriptsize{\slide@setsize\scriptsize{7pt}\vipt\@vipt}%
  \def\tiny{\slide@setsize\tiny{6pt}\vpt\@vpt}%
  \def\large{\slide@setsize\large{12pt}\xpt\@xpt}%
  \def\Large{\slide@setsize\Large{14pt}\xiipt\@xiipt}%
  \def\LARGE{\slide@setsize\LARGE{17pt}\xivpt\@xivpt}%
  \def\huge{\slide@setsize\huge{20pt}\xviipt\@xviipt}%
  \def\Huge{\slide@setsize\Huge{25pt}\xxpt\@xxpt}}
\@namedef{9@semptsize}{%
  \def\normalsize{\slide@@setsize\normalsize{11pt}\ixpt\@ixpt}%
  \def\small{\slide@@setsize\small{9.5pt}\viiipt\@viiipt}%
  \def\footnotesize{\slide@@setsize\footnotesize{8pt}\viipt\@viipt}%
  \def\scriptsize{\slide@setsize\scriptsize{7pt}\vipt\@vipt}%
  \def\tiny{\slide@setsize\tiny{6pt}\vpt\@vpt}%
  \def\large{\slide@setsize\large{12pt}\xpt\@xpt}%
  \def\Large{\slide@setsize\Large{14pt}\xiipt\@xiipt}%
  \def\LARGE{\slide@setsize\LARGE{17pt}\xivpt\@xivpt}%
  \def\huge{\slide@setsize\huge{20pt}\xviipt\@xviipt}%
  \def\Huge{\slide@setsize\Huge{25pt}\xxpt\@xxpt}}
\@namedef{10@semptsize}{%
  \def\normalsize{\slide@@setsize\normalsize{12pt}\xpt\@xpt}%
  \def\small{\slide@@setsize\small{11pt}\ixpt\@ixpt}%
  \def\footnotesize{\slide@@setsize\footnotesize{9.5pt}\viiipt\@viiipt}%
  \def\scriptsize{\slide@setsize\scriptsize{8pt}\viipt\@viipt}%
  \def\tiny{\slide@setsize\tiny{6pt}\vpt\@vpt}%
  \def\large{\slide@setsize\large{14pt}\xiipt\@xiipt}%
  \def\Large{\slide@setsize\Large{18pt}\xivpt\@xivpt}%
  \def\LARGE{\slide@setsize\LARGE{22pt}\xviipt\@xviipt}%
  \def\huge{\slide@setsize\huge{25pt}\xxpt\@xxpt}%
  \def\Huge{\slide@setsize\Huge{30pt}\xxvpt\@xxvpt}}
\@namedef{11@semptsize}{%
  \def\normalsize{\slide@@setsize\normalsize{13.6pt}\xipt\@xipt}%
  \def\small{\slide@@setsize\small{12pt}\xpt\@xpt}%
  \def\footnotesize{\slide@@setsize\footnotesize{11pt}\ixpt\@ixpt}%
  \def\scriptsize{\slide@setsize\scriptsize{9.5pt}\viiipt\@viiipt}%
  \def\tiny{\slide@setsize\tiny{7pt}\vipt\@vipt}%
  \def\large{\slide@setsize\large{14pt}\xiipt\@xiipt}%
  \def\Large{\slide@setsize\Large{18pt}\xivpt\@xivpt}%
  \def\LARGE{\slide@setsize\LARGE{22pt}\xviipt\@xviipt}%
  \def\huge{\slide@setsize\huge{25pt}\xxpt\@xxpt}%
  \def\Huge{\slide@setsize\Huge{30pt}\xxvpt\@xxvpt}}
\@namedef{12@semptsize}{%
  \def\normalsize{\slide@@setsize\normalsize{14.5pt}\xiipt\@xiipt}%
  \def\small{\slide@@setsize\small{13.6pt}\xipt\@xipt}%
  \def\footnotesize{\slide@@setsize\footnotesize{12pt}\xpt\@xpt}%
  \def\scriptsize{\slide@setsize\scriptsize{9.5pt}\viiipt\@viiipt}%
  \def\tiny{\slide@setsize\tiny{7pt}\vipt\@vipt}%
  \def\large{\slide@setsize\large{18pt}\xivpt\@xivpt}%
  \def\Large{\slide@setsize\Large{22pt}\xviipt\@xviipt}%
  \def\LARGE{\slide@setsize\LARGE{25pt}\xxpt\@xxpt}%
  \def\huge{\slide@setsize\huge{30pt}\xxvpt\@xxvpt}%
  \let\Huge\huge}
\@namedef{14@semptsize}{%
  \def\normalsize{\slide@@setsize\normalsize{18pt}\xivpt\@xivpt}%
  \def\small{\slide@@setsize\small{14.5pt}\xiipt\@xiipt}%
  \def\footnotesize{\slide@@setsize\footnotesize{13.6pt}\xipt\@xipt}%
  \def\scriptsize{\slide@setsize\scriptsize{12pt}\xpt\@xpt}%
  \def\tiny{\slide@setsize\tiny{9.5pt}\viiipt\@viiipt}%
  \def\large{\slide@setsize\large{22pt}\xviipt\@xviipt}%
  \def\Large{\slide@setsize\Large{25pt}\xxpt\@xxpt}%
  \def\LARGE{\slide@setsize\LARGE{30pt}\xxvpt\@xxvpt}%
  \let\huge\LARGE
  \let\Huge\LARGE}
\@namedef{17@semptsize}{%
  \def\normalsize{\slide@@setsize\normalsize{22pt}\xviipt\@xviipt}%
  \def\small{\slide@@setsize\small{18pt}\xivpt\@xivpt}%
  \def\footnotesize{\slide@@setsize\footnotesize{14.5pt}\xiipt\@xiipt}%
  \def\scriptsize{\slide@setsize\scriptsize{13.6pt}\xipt\@xipt}%
  \def\tiny{\slide@setsize\tiny{12pt}\xpt\@xpt}%
  \def\large{\slide@setsize\large{25pt}\xxpt\@xxpt}%
  \def\Large{\slide@setsize\Large{30pt}\xxvpt\@xxvpt}%
  \let\LARGE\Large
  \let\huge\Large
  \let\Huge\Large}
\def\twoup{\@ifnextchar[{\@twoup}{\@twoup[0]}}
\ifarticle
  \ifportrait
    \def\@twoup[#1]{%
      \@tempcnta=\the@articlemag\relax
      \@tempcntb=\@tempcnta
      \advance\@tempcntb by #1\relax
      \advance\@tempcntb by -2\relax
      \ifnum\@tempcntb>-6\relax
        \@@input 2up.tex
        \source{\@magstep\@tempcnta}{\paperheight}{\paperwidth}%
        \target{\@magstep\@tempcntb}{\paperwidth}{\paperheight}%
        \targetlayout{topbottom}%
        \printlandscapefalse
      \else
        \@seminarerr{Article magnification is too low for
          \string\twoup\space to handle}\@eha
      \fi
      \def\articlemag##1{\@semtwoupmagerr\articlemag}}
  \else
    \def\@twoup[#1]{%
      \@tempcnta=\the@articlemag\relax
      \@tempcntb=\@tempcnta
      \advance\@tempcntb by #1\relax
      \advance\@tempcntb by -2\relax
      \ifnum\@tempcntb>-6\relax
        \@@input 2up.tex
        \source{\@magstep\@tempcnta}{\paperwidth}{\paperheight}%
        \target{\@magstep\@tempcntb}{\paperheight}{\paperwidth}%
        \if@twoside
          \targetlayout{twosided}%
        \fi
        \printlandscapetrue
      \else
        \@seminarerr{Article magnification is too low for
          \string\twoup\space to handle}\@eha
      \fi
      \def\articlemag##1{\@semtwoupmagerr\articlemag}}
  \fi
\else
  \ifportrait
    \def\@twoup[#1]{%
      \@tempcnta=\the@slidesmag\relax
      \@tempcntb=\@tempcnta
      \advance\@tempcntb by #1\relax
      \advance\@tempcntb by -3\relax
      \ifnum\@tempcntb>-6\relax
        \@@input 2up.tex
        \source{\@magstep\@tempcnta}{\paperwidth}{\paperheight}%
        \advance\@tempcnta by -3
        \target{\@magstep\@tempcnta}{\paperheight}{\paperwidth}%
        \printlandscapetrue
      \else
        \@seminarerr{Slides magnification is too low for
          \string\twoup\space to handle}\@eha
      \fi
      \def\slidesmag##1{\@semtwoupmagerr\slidesmag}}
  \else
    \def\@twoup[#1]{%
      \@tempcnta=\the@slidesmag\relax
      \@tempcntb=\@tempcnta
      \advance\@tempcntb by #1\relax
      \advance\@tempcntb by -3\relax
      \ifnum\@tempcntb>-6\relax
        \@@input 2up.tex
        \source{\@magstep\@tempcnta}{\paperheight}{\paperwidth}%
        \target{\@magstep\@tempcntb}{\paperwidth}{\paperheight}%
        \targetlayout{topbottom}%
        \printlandscapefalse
      \else
        \@seminarerr{Slides magnification is too low for
          \string\twoup\space to handle}\@eha
      \fi
      \def\slidesmag##1{\@semtwoupmargerr\slidesmag}}%
  \fi
\fi
\def\@semtwoupmagerr#1{%
  \@seminarerr{\string#1 must come before \string\twoup}\@eha}
\addto@preamblecmds{\twoup\do\@twoup}
%
\InputIfFileExists{seminar.con}{}
%\ifeof1\else\closein1 \relax\@@input seminar.con\fi
%
\endinput
%%
%% END `seminar.cls'.

%
}

\expandafter\def\expandafter\set@slidesize\expandafter
{\set@slidesize\@colht\vsize}

\g@addto@macro\@arrayparboxrestore\slide@rightskip

\ifthenelse{\boolean{truepn@PS}}{\truepagenumbers}{}

\AtBeginDocument{%
\ifthenelse{\boolean{BaseClass@PS}}{\sliderotation{none}}{}}
%    \end{macrocode} 
%
% Make some changes in the seminar kernel to hopefully get better handling of text colors and avoid overfull box
% warnings when |\slidetopmargin| and |\slidebottommargin| are set too small.
%    \begin{macrocode}
\AtBeginDocument{%
\ifarticle
\else
  \def\output@slide{%
    \begingroup
      \hoffset=-\inverseslidesmag in
      \voffset=-\inverseslidesmag in
      \setslidelength\@tempdima{%
        \ifportrait\paperwidth\else\paperheight\fi}%
      \setslidelength\@tempdimb{%
        \ifportrait\paperheight\else\paperwidth\fi}%
      \ifrotateheaders
        \@ifrotateslide{%
          \dimen@=\@tempdima
          \@tempdima=\@tempdimb
          \@tempdimb=\dimen@}%
      \fi
      % \oddsidemargin, \evensidemargin, \headheight, \footheight
      % used for scratch:
      \setslidelength\headheight\slidetopmargin
      \setslidelength\footheight\slidebottommargin
      \ifautoslidemargins
        \textwidth=\wd\@slidebox
        \oddsidemargin=\@tempdima
        \advance\oddsidemargin-\textwidth
        \divide\oddsidemargin 2
      \else
        \setslidelength\oddsidemargin\slideleftmargin
        \setslidelength\evensidemargin\sliderightmargin
        \textwidth=\@tempdima
        \advance\textwidth-\oddsidemargin
        \advance\textwidth-\evensidemargin
      \fi
      \setbox\@slidebox=\hbox to \@tempdima{{%    
%    \end{macrocode}
% All double \{\{ around box contents added for better color handling (STL).
%    \begin{macrocode}
      %                                                         
        \kern\oddsidemargin
        \vbox to\@tempdimb{{%
          \ifnum\fancyput@flag>-1
            \hbox{\kern-\oddsidemargin\do@fancyput}%
          \fi
          \let\label\@gobble
          \let\index\@gobble
          \let\glossary\@gobble
          \vbox to\headheight{{%
            \vfill
            \color@hbox
            \normalcolor
           \slideheadfont\relax%\strut         Removed (STL)
           \hb@xt@\textwidth{\@oddhead}%
            \color@endbox
            \kern\z@}}%
          \nointerlineskip
          \vss
          \hbox to\textwidth{{\hss\box\@slidebox\hss}}%
          \vss
          \nointerlineskip
          \vbox to\footheight{{%
            \vfill
              \color@hbox
              \normalcolor
             \slidefootfont\relax%\strut         Removed (STL)
             \hb@xt@\textwidth{\@oddfoot}%
              \color@endbox
            \vfill}}%
        }}\hss}}%
      \ifrotateheaders\@ifrotateslide\rotate@slide\fi
      \insert@specials
      \shipout\box\@slidebox
    \endgroup
    \let\firstmark\botmark
    \ifthenelse{\boolean{truepn@PS}}{\stepcounter{page}}{}%
%    \end{macrocode} 
% The above will seriously upset notes! (STL)
%    \begin{macrocode} 
    }%
  \@ifundefined{fancyput@flag}{\def\fancyput@flag{-1}}{}%
\fi
}
%    \end{macrocode} 
%
% When producing slides to be displayed interactively, we must make absolutely sure that interactive building of slides
% doesn't move text which has already been set. So slides are not to be centered. 
%    \begin{macrocode}
\ifthenelse{\boolean{display}}
{%
  \centerslidesfalse
  \extraslideheight{0pt}%
  \renewcommand{\slideshrink}{0}%
  \def\@makeslide{%
    \ifcenterslides
      \setbox\@slidebox\vbox{{%
        \@begindvi % added 1997/04/15 SPQR
        \unvbox\@cclv
        \ifvoid\slide@footins\else
          \vskip\skip\slide@footins
          \footnoterule
          \unvbox\slide@footins
        \fi
        \vskip\z@}}
     \else
      \setbox\@slidebox\vbox to \slide@vsize{{%
        \@begindvi % added 1997/04/15 SPQR
        \unvbox\@cclv
        \ifvoid\slide@footins\vfil\else
          \vfil\vskip\skip\slide@footins
          \footnoterule
          \unvbox\slide@footins
        \fi
        \vskip\z@}}%
    \fi
    }%
  }
{}
%    \end{macrocode} 
%
% If powersem is asked to calculate the slide dimensions...
%    \begin{macrocode}
\def\relax@PS{\relax}

\ifthenelse{\boolean{calcdim@PS}}
{%
  \AtBeginDocument
  {%
    \setlength{\slidewidth}{\paperwidth}%
    \advance\slidewidth by -\slideleftmargin\relax
    \advance\slidewidth by -\sliderightmargin\relax
    \ifx\theslideframe\relax@PS
     \else
      \advance\slidewidth by -2\slideframewidth\relax
      \advance\slidewidth by -2\slideframesep\relax
    \fi
    \setlength{\slideheight}{\paperheight}%
    \advance\slideheight by -\slidetopmargin\relax
    \advance\slideheight by -\slidebottommargin\relax
    \ifx\theslideframe\relax@PS
     \else
      \advance\slideheight by -2\slideframewidth\relax
      \advance\slideheight by -2\slideframesep\relax
    \fi
    }%
  }
{}

\AtBeginDocument
{%
  \@ifundefined{headwidth}{}
  {%
    \setslidelength{\headwidth}{\paperwidth}%
    \addtoslidelength{\headwidth}{-\slideleftmargin}%
    \addtoslidelength{\headwidth}{-\sliderightmargin}%
  }%
}
%    \end{macrocode} 
% \Finale
\endinput
