% !arara: pdflatex
% !arara: biber
% arara: pdflatex
% arara: pdflatex
% --------------------------------------------------------------------------
% the ENOTEZ package
% 
%   Endnotes for LaTeX2e
% 
% --------------------------------------------------------------------------
% Clemens Niederberger
% Web:    http://www.mychemistry.eu/forums/forum/enotez/
% E-Mail: contact@mychemistry.eu
% --------------------------------------------------------------------------
% Copyright 2012-2017 Clemens Niederberger
% 
% This work may be distributed and/or modified under the
% conditions of the LaTeX Project Public License, either version 1.3
% of this license or (at your option) any later version.
% The latest version of this license is in
%   http://www.latex-project.org/lppl.txt
% and version 1.3 or later is part of all distributions of LaTeX
% version 2005/12/01 or later.
% 
% This work has the LPPL maintenance status `maintained'.
% 
% The Current Maintainer of this work is Clemens Niederberger.
% --------------------------------------------------------------------------
% The enotez package consists of the files
%  - enotez.sty, enotez_en.tex, enotez_en.pdf, README
% --------------------------------------------------------------------------
% If you have any ideas, questions, suggestions or bugs to report, please
% feel free to contact me.
% --------------------------------------------------------------------------
\documentclass[load-preamble+]{cnltx-doc}
\usepackage[utf8]{inputenc}
\usepackage{enotez}

\setcnltx{
  package = enotez ,
  info    = {Endnotes for \LaTeXe} ,
  authors = Clemens Niederberger ,
  email   = contact@mychemistry.eu ,
  url     = https://bitbucket.org/cgnieder/enotez/ ,
  pre-output = \setfnpct{dont-mess-around} ,
  add-cmds = {
    @endnotemark,
    endnote ,
    endnotemark ,
    endnotetext ,
    enmark ,
    enmarkstyle ,
    enotezwritemark ,
    NewSplitTitleTag ,
    printendnotes ,
    setenotez ,
    splitendnotes ,
    theendnote
  } ,
  add-silent-cmds = {
    @currentlabel ,
    appendix ,
    cs ,
    DeclareInstance ,
    DeclareTemplateInterface ,
    DeclareTranslation ,
    kant
  } ,
  index-setup = {
     othercode = \footnotesize ,
     level = \addsec ,
     % noclearpage
  } ,
  makeindex-setup = {
     columns = 3 ,
     columnsep = 1em
  }
}
\setenotez{mark-cs=\textsu,backref}
\DeclareInstance{enotez-list}{addsec}{paragraph}{heading=\addsec{#1}}

\defbibheading{bibliography}{\addsec{References}}

\usepackage{kantlipsum}
\usepackage{enumitem}

\begin{document}

\section{Licence and Requirements}
\license

\enotez\ needs and loads the following packages:
\bnd{l3kernel}~\cite{bnd:l3kernel}, \pkg{xparse}, \pkg{xtemplate} and
\pkg{l3keys2e} from the \bnd{l3packages} bundle~\cite{bnd:l3packages},
\pkg{xpatch}~\cite{pkg:xpatch}, \pkg{scrlfile} from the \KOMAScript\
bundle~\cite{bnd:koma-script} and \pkg{translations}~\cite{pkg:translations}.

\section{Motivation}
\enotez\ is a new implementation of endnotes for \LaTeXe\ since the
\pkg{endnotes} package~\cite{pkg:endnotes} has some deficiencies.  Nested
endnotes, for example, are not supported, neither is
\pkg{hyperref}~\cite{pkg:hyperref}.  The \pkg{sepfootnotes}
package~\cite{pkg:sepfootnotes} also provides means for endnotes but actually
has a different purpose: to separate input and usage both of footnotes and
endnotes.  So it might not be the best solution in every case\footnote{You
  have to write the actual notes in the preamble or a separate file and
  reference them in the text.}.  It also does not allow nested endnotes.

While \enotez\ worked in tests nicely with the
\cls{memoir}~\cite{cls:memoir} class please keep in mind that
\cls{memoir} provides its own endnote mechanism.

\enotez\ enables nested endnotes properly and has another mechanism of
customizing the list of endnotes which is easily extendable.  One of the main
features of \enotez\ is a split list of endnotes in which the notes are
automatically separated by the sections or chapters they were set in, see
section~\ref{sec:split} for more information.

As an aside: \enotez\ is nicely compatible with the \pkg{fnpct}
package~\cite{pkg:fnpct}.  Version~0.2j or newer of \pkg{fnpct} automatically
detects \enotez\ and adapts the \cs{endnote} command.


\section{Usage}
\subsection{Placing the Notes}
The usage is simple: use \cs{endnote} in the text where you want to place the
note mark.
\begin{commands}
  \command{endnote}[\oarg{mark}\marg{text}]
    Add an endnote in the text.
  \command{endnotemark}[\oarg{mark}]
    \sinceversion{0.9}Add an endnotemark.
  \command{endnotetext}[\marg{text}]
    \sinceversion{0.9}Add text to an endnote placed with \cs{endnotemark}.
\end{commands}
\begin{example}
  This is some text.\endnote{With an endnote.}
\end{example}
There's not really much more to it.  It is possible to add a custom mark by
using the optional argument but that shouldn't be needed too often.
% \cs{endnote} works fine inside tables, minipages, floats and
% captions\footnote{This has been tested with the standard classes,
%   \cls{memoir}, and the \cls{KOMA-Script} classes, with and without the
%   \pkg*{caption} package.  If you're using another package that redefines
%   \cs*{caption} or are using another class it might not work.  Before you
%   place a note in a caption you should re-think the idea anyway.}.
Endnotes
can also be nested.

% Since this functionality seemed making a pair
% \cs{endnotemark}/\cs{endnotetext} superfluous they are \emph{not} defined by
% \enotez.
\begin{example}
  This is some text.\endnote{With another endnote.\endnote{This is a
      nested\endnote{And another level deeper\ldots} endnote!}}
  % uses package `kantlipsum' to produce dummy text:
  Of course you can have several paragraphs\endnote{\kant[1-3]} in an endnote.
\end{example}

The marks of the endnotes in the running text are printed through the command
\cs{enotezwritemark} which defaults to \cs*{textsuperscript}.  Its argument
contains the current mark which is preceded by \cs{enmarkstyle}.  Both of
these commands can be redefined of course to adapt to custom settings.  This
can also be done using options, see section~\ref{sec:options}.  The mark of
the endnote that has been set last is stored in % \cs{theendnote} and in
\verbcode+\@currentlabel+.\sinceversion{0.6}

% This could be used to define a command which writes a mark:
% \begin{example}
%   % we cannot use \newcommand -- the command starts with \end...
%   \makeatletter
%   \def\endnotemark{\@ifnextchar[{\@endnotemark}{\@endnotemark[\theendnote]}}
%   \def\@endnotemark[#1]{\enotezwritemark{\enmarkstyle#1}}
%   \makeatother
%   Text\endnotemark
% \end{example}
% Please note that this definition \emph{does not} step the endnote counter but
% either refers to the last number (no optional argument) or uses the one
% provided in the optional argument.

% \makeatletter
% \def\endnotemark{\@ifnextchar[{\@endnotemark}{\@endnotemark[\theendnote]}}
% \def\@endnotemark[#1]{\enotezwritemark{\enmarkstyle#1}}
% \makeatother

Endnotes can also be labelled and later be referred to:
\begin{example}
  The next endnote\endnote{This endnote gets a label.}\label{en:test} has
  the number~\ref{en:test}.  Let's now test
  \cs{endnotemark}\endnotemark[\ref{en:test}].
\end{example}

\subsection{Printing the Notes}
The notes are printed by using the command \cs{printendnotes}.
\begin{commands}
  \command{printendnotes}[\sarg\oarg{style}]
    Print the list of endnotes. \meta{style} is one of the instances
    explained in section~\ref{sec:customizing_the_list}.
\end{commands}
If used without argument it prints all notes set so far with
\cs{endnote}. The current list will then be cleared.  All endnotes set after
it are stored again for the next usage of \cs{printendnotes}.  The starred
version will print \emph{all} endnotes but shouldn't be used more than once if
you have nested endnotes.

It may take several compilation runs until all notes are printed correctly.
In a first run they are written to the \code{aux} file.  In the second run
they are available to \cs{printendnotes}.  If you have nested endnotes they
will be written to the \code{aux} file the first time they're printed with
\cs{printendnotes} which means you might have to compile your file once more.
If you change any of the endnotes or add another one you again will need at
least two runs, maybe more.  \enotez\ tries to warn you in these cases by
invoking the warning
\begin{flushleft}
  \ttfamily
  Endnotes may have changed.  Rerun to get them right.
\end{flushleft}
but may not catch all cases.

\enotez\ provides two commands that allow to set some kinds of preamble and
postamble to a list, either to every list or only to the next one:
\begin{commands}
  \command{AtEveryEndnotesList}[\marg{text}]
    \sinceversion{0.5}inserts \meta{text} between heading and the actual notes
    every time \cs{printendnotes} is used.
  \command{AtNextEndnotesList}[\marg{text}]
    \sinceversion{0.5} inserts \meta{text} between heading and the actual
    notes the next time \cs{printendnotes} is used.  This overwrites a
    possible preamble set with \cs{AtEveryEndnotesList} for this instance of
    \cs{printendnotes}.
  \command{AfterEveryEndnotesList}[\marg{text}]
    \sinceversion{0.5} inserts \meta{text} after the notes list every time
    \cs{printendnotes} is used.
  \command{AfterNextEndnotesList}[\marg{text}]
    \sinceversion{0.5} inserts \meta{text} after the notes list the next time
    \cs{printendnotes} is used.  This overwrites a possible postamble set with
    \cs{AfterEveryEndnotesList} for this instance of \cs{printendnotes}.
\end{commands}
If something is inserted with one of these commands the inserted \meta{text}
will be followed by a \cs*{par} and a vertical skip for the preamble.  The
postambles follow a \cs*{par} and a vertical skip.  The skips can be set using
an option, see section~\ref{sec:options}.

\section{Options}\label{sec:options}
\subsection{Package Options}
\enotez\ has a few package options which should be pretty self-explanatory.
They can be set either as package options with
\cs*{usepackage}\oarg{options}\Marg{enotez} or with the setup command.
\begin{commands}
  \command{setenotez}[\marg{options}]
    Setup command for setting \enotez' options.
\end{commands}
\begin{options}
  \keyval{list-name}{list name}\Default{Notes}
    The name of the notes list. This name is used for the heading of the
    list.
  \keybool{reset}\Default{false}
    If set to \code{true} the notes numbers will start from 1 again after
    \cs{printendnotes} has been invoked.
  \keychoice{counter-format}{arabic,alph,Alph,roman,Roman,symbols}\Default{arabic}
    Change the format of the endnote counter.  Please be aware that there are
    only 26 alphabetic counter symbols (options \code{alph} and \code{Alph}
    and \emph{only 9} symbols (option \code{symbols}).
  \keyval{mark-format}{code}\Default
    Redefine \cs{enmarkstyle} to execute \meta{code}.  This command is placed
    directly before the endnote mark in the text.
  \keyval{mark-cs}{command}\Default{\cs*{textsuperscript}}
    Lets \cs{enotezwritemark} to be equal to \meta{command}.  This command
    is used to typeset the endnote marks in the text and should take one
    argument.
  \keybool{backref}\Default{false}\label{key:backref}
    \sinceversion{0.7}If set to \code{true} and \pkg{hyperref} has been
    loaded backlinks from the notes in the list to the marks in the text are
    added.
  \keychoice{totoc}{section,chapter,false}\Default{false}\label{key:totoc}
    Add an entry to the table of contents.
  \keyval{list-heading}{sectioning command including argument}
    You can use this option to manually set the list heading command, \eg,
    \keyis{list-heading}{\cs{chapter}\Marg{\#1}} for a numbered heading.  The
    default depends upon if the class you're using provides \cs*{chapter} or
    not.  It either uses \cs*{chapter}\sarg\ or \cs*{section}\sarg. You can
    see that you have to refer to the actual heading with \code{\#1}.
  \keyval{list-style}{style}\Default{plain}
    Sets the default list style, see section~\ref{sec:customizing_the_list}
    for details.
  \keyval{list-preamble-skip}{skip}\Default{\cs*{medskipamount}}
    \sinceversion{0.5}Sets the vertical skip (a rubber length) that is
    inserted if a list preamble is inserted by using either
    \cs{AtNextEndnotesList} or \cs{AtEveryEndnotesList}.  It's default is set
    equal to \cs*{medskipamount}.
  \keyval{list-postamble-skip}{skip}\Default{\cs*{medskipamount}}
    \sinceversion{0.5}Sets the vertical skip (a rubber length) that is
    inserted if a list postamble is inserted by using either
    \cs{AfterNextEndnotesList} or \cs{AfterEveryEndnotesList}.  It's default
    is set equal to \cs*{medskipamount}.
\end{options}

\subsection{Customizing the List}\label{sec:customizing_the_list}
The list is typeset with \pkg{xtemplate}'s possibilities.  \enotez\ declares
the object \code{enotez-list} and two templates for it, the template
\code{paragraph} and the template \code{list}.

\subsubsection{The \code{paragraph} Template}
The \code{paragraph} template's interface is defined as follows:
\begin{sourcecode}
  \DeclareTemplateInterface{enotez-list}{paragraph}{1}
    {
      % parameter   : type       = default
      heading       : function 1 = \section*{#1}   ,
      format        : tokenlist  = \footnotesize   ,
      number        : function 1 = \enmark{#1}     ,
      number-format : tokenlist  = \normalfont     ,
      notes-sep     : length     = .5\baselineskip ,
    }
\end{sourcecode}
The parameters functions are these:
\begin{description}[style=nextline]
  \item[\code{heading}] The command with which the heading is typeset.
  \item[\code{format}] The format of the whole list.
  \item[\code{number}] The command that is used to typeset the numbers of the
    notes.  The command \cs{enmark} is explained soon.
  \item[\code{numbers-format}] The format of the numbers.
  \item[\code{notes-sep}] Additional space between the notes.
\end{description}

\enotez\ uses this template to define the instance \code{plain}:
\begin{sourcecode}
  \DeclareInstance{enotez-list}{plain}{paragraph}{}
\end{sourcecode}
This is the default style of the list.

You can easily define your own instances, though:
\begin{sourcecode}
  \DeclareInstance{enotez-list}{custom}{paragraph}
    {
      heading   = \chapter*{#1}        ,
      notes-sep = \baselineskip        ,
      format    = \normalfont          ,
      number    = \textsuperscript{#1}
    }
\end{sourcecode}
This would use a chapter heading for the title, separate the notes with
\cs*{baselineskip} and typeset them with \cs*{normalfont}.  The numbers would
be typeset with \cs*{textsuperscript}.  You could now use it like this:
\begin{sourcecode}
  \printendnotes[custom]
\end{sourcecode}

If you wanted superscripted numbers, you could also redefine \cs{enmark}. 
\begin{commands}
  \command{enmark}
    this command is initially defined like this:
    \verbcode=\newcommand*\enmark[1]{#1.}=
\end{commands}

\subsubsection{The \code{list} Template}
The \code{list} template's interface is defined as follows:
\begin{sourcecode}
  \DeclareTemplateInterface{enotez-list}{list}{1}
    {
      % parameter   : type       = default
      heading       : function 1 = \section*{#1} ,
      format        : tokenlist  = \footnotesize ,
      number        : function 1 = \enmark{#1}   ,
      number-format : tokenlist  = \normalfont   ,
      list-type     : tokenlist  = description   ,
    }
\end{sourcecode}
This template uses a list to typeset the notes.  As you can see the default
list is a \code{description} list.

\enotez\ defines two instances of this template:
\begin{sourcecode}
  \DeclareInstance{enotez-list}{description}{list}{}
  \DeclareInstance{enotez-list}{itemize}{list}{list-type = itemize}
\end{sourcecode}
They're available through \cs{printendnotes}\Oarg{description} and
\cs{printendnotes}\Oarg{itemize}, respectively.

Again you can define your own instances using whatever list you want, possibly
one defined with the power of \pkg{enumitem}.

\section{Collect Notes Section-wise and Print List Stepwise}\label{sec:split}
\emph{This feature is experimental and surely has some limitations.  Please
  let me know if something doesn't work as expected}.

Not to be misunderstood: you can use \cs{printendnotes} as often as you like,
possibly after each section.  That is \emph{not} what is meant here.  Let's
suppose you are writing a book and have many endnotes in many chapters.  It
would be nice if the list of endnotes at the end of the book could be split
into parts for each chapter.  This section describes how you can achieve that
with \enotez.

First of all \enotez\ will rely on the fact that you use \cs{printendnotes}
only \emph{once}!  If you call it more times nobody knows what will
happen\ldots

You'll need to tell \enotez\ that you want to split the notes into groups.
\begin{options}
  \keychoice{split}{section,chapter,false}\Default{false}
    Enable the automatic splitting.
  \keyval{split-sectioning}{csname}\Default
    \emph{This option is deprecated and may be dropped in future versions!}
    The command that is used to display the titles between the splits.  It
    needs to be a command that takes one argument and should be entered
    without the leading backslash.  If the option is not used \enotez\ will
    choose \code{subsection*} for \keyis{split}{section} and \code{section*}
    for \keyis{split}{chapter}.
  \keyval{split-heading}{sectioning command including argument}
    \sinceversion{0.3}The command that is used to display the titles between
    the splits.  It is entered with argument and the actual title is referred
    to with \code{\#1}, \eg,
    \keyis{split-heading}{\cs*{subsection}\sarg\Marg{\#1}}.  If the option is
    not used \enotez\ will choose \cs*{subsection}\sarg\Marg{\#1} for
    \keyis{split}{section} and \cs*{section}\sarg\Marg{\#1} for
    \keyis{split}{chapter}.
  \keyval{split-title}{tokenlist}\Default{Notes for <name> <ref>}
    The title that will be inserted between the splits.  \code{<name>} is
    replaced by \code{section} for \keyis{split}{section} and \code{chapter}
    for \keyis{split}{chapter}. \code{<ref>} is replaced by the corresponding
    \cs*{thesection} or \cs*{thechapter}.
\end{options}
Set the \option{split} option:
\begin{sourcecode}
  \setenotez{split=section}
\end{sourcecode}
Well -- that's it, basically.  You'll have to be careful, though: if you're
having nested endnotes the nested ones appear first in the ``Notes'' section
(or chapter, respectively).  In this case you should have a numbered section
title for the notes, presumably in the appendix.  You'll need to create a new
list style:
\begin{sourcecode}
  % preamble:
  \usepackage{enotez}
  \DeclareInstance{enotez-list}{section}{paragraph}{heading=\section{#1}}
  \setenotez{list-style=section,split=section}
  % document:
  \appendix
  \printendnotes
\end{sourcecode}

Please beware that the option \option{reset} also impacts here: the numbering
will be reset for each section or chapter, depending on the choice you made
for \option{split}.

There are additional commands:
\begin{commands}
  \command{AtEveryListSplit}[\marg{code}]
    Inserts\sinceversion{0.7} \meta{code} before each sub-heading in a
    splitted list.
  \command{AfterEveryListSplit}[\marg{code}]
    Inserts\sinceversion{0.7} \meta{code} after each sub-heading in a splitted
    list.
  \command{EnotezCurrentSplitTitle}
    Holds\sinceversion{0.7} the current sub-heading in a splitted list and may
    be used in \cs{AtEveryListSplit} and \cs{AfterEveryListSplit}.
  \command{NewSplitTitleTag}[\marg{tag}\marg{replacement}]
    In\sinceversion{0.8} the sub-headings of a splitted list, the tags
    \code{<ref>}, \code{<name>} and \code{<split-level-id>} can be used per
    default.  This command allows to define additional tags.
\end{commands}

The tags that can be used for setting the sub-headings of a splitted list have
the following meaning:
\begin{description}
  \item[\code{<ref>}] The counter representation of the current section or
    chapter, \ie, either the corresponding expansion of \cs*{thesection} or
    \cs*{thechapter}, depending on the split-level.
  \item[\code{<name>}] The name of the split-level, \ie, either ``section'' or
    ``chapter'', depending on the split-level.  This is language sensitive
    (see section~\ref{sec:language-support}).  The corresponding translation
    ids are \code{enotez-section} and \code{enotez-chapter}, respectively.
  \item[\code{<split-level-id>}] The number of the current section or chapter,
    \ie, either the corresponding expansion of \verbcode+\arabic{section}+ or
    \verbcode+\arabic{chapter}+, depending on the split-level.
\end{description}
The command \cs{NewSplitTitleTag} let's you define your own. You can use the
three existing tags inside the replacement.  Note that the \meta{tag} argument
is input without opening \code{<} and closing \code{>}!
\cs{NewSplitTitleTag}\Marg{own}\Marg{...} would define the tag \code{<own>}.

\enotez\ comes with an example document for a split list which you should find
in the same folder as this documentation.

\section{Language Support}\label{sec:language-support}
\enotez\ uses the \pkg{translations} package~\cite{pkg:translations} to
translate language dependent strings.  The advantage of this is that language
settings with \pkg{babel}~\cite{pkg:babel} or
\pkg{polyglossia}~\cite{pkg:polyglossia} are detected automatically.  However,
the available translations are somewhat limited due to my limited language
knowledge.  If you find missing or wrong translations you can try to add or
correct them by adding the corresponding versions of the following lines to
your preamble:
\begin{sourcecode}
  \DeclareTranslation{English}{enotez-title}{Notes}
  % ``<name> <ref>'' is a placeholder for e.g. ``section 1'':
  \DeclareTranslation{English}{enotez-splitted-title}{Notes for <name> <ref>}
  \DeclareTranslation{English}{enotez-section}{section}
  \DeclareTranslation{English}{enotez-chapter}{chapter}
\end{sourcecode}
If you like you can also drop me an email at
\href{mailto:contact@mychemistry.eu}{contact@mychemistry.eu} and I'll add the
correct translations to \enotez.

\section{hyperref Support}
If \pkg{hyperref} is loaded and you are using the option \option{totoc} (see
page~\pageref{key:totoc}) the list title is linked via a \cs*{phantomsection}.

If \pkg{hyperref} is used with \code{hyperfootnotes} set to \code{true} the
endnote marks are linked to the respective entries in the
list\changedversion{0.7}.  If you also set \enotez' option \option{backref}
(see page~\pageref{key:backref}) the notes in the list are themselves linked
to the marks in the text.

\AtNextEndnotesList{This is an example of a preamble to the list set with
\cs{AtNextEndnotesList}.}
\AfterEveryEndnotesList{\noindent This is an example of a postamble to the
  list set with \cs{AfterEveryEndnotesList}.  Note that it would have started
  with a paragraph indent which was prevented here by using \cs*{noindent}.}
\printendnotes[addsec]

\end{document}
