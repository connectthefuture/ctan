% The main file for developing the proposal. 
% Variants with different class options are 
% - submit.tex (no draft stuff, no ednotes, no revision information)
% - public.tex (like submit.tex, but no financials either) 
\providecommand{\classoptions}{,keys} % to be overwritten in variants
\documentclass[noworkareas,deliverables,gitinfo,propB\classoptions]{euproposal}
\usepackage[T1]{fontenc}
\usepackage[utf8]{inputenc}
\addbibresource{../lib/dummy}
\input{../lib/WApersons}% Some sections of the included files depend on this.

\begin{document}
\begin{center}\color{red}\huge
  This mock proposal is just an example for \texttt{euproposal.cls} it reflects the ICT
  template of January 2012
\end{center}
\begin{proposal}[site=jacu,jacuRM=36,
  site=efo,efoRM=36,
  site=bar,barRM=36,
  site=baz,bazRM=36,
  coordinator=miko,
  acronym={iPoWr},
  acrolong={\underline{I}ntelligent} {\underline{P}r\underline{o}sal} {\underline{Wr}iting},
  title=\pn: \protect\pnlong,
  callname = ICT Call 1,
  callid = FP7-???-200?-?,
  instrument= Small or Medium-Scale Focused Research Project (STREP), 
  challengeid = 4,
  challenge = ICT for EU Proposals,
  objectiveid={ICT-2012.4.4}, 
  objective = Technology-enhanced Documents,
  outcomeid = b1,
  outcome = {More time for Research, not Proposal writing},
  coordinator=miko,
  months=24,
  compactht]
\begin{abstract}
  Writing grant proposals is a collaborative effort that requires the integration of
  contributions from many individuals. The use of an ASCII-based format like {\LaTeX}
  allows to coordinate the process via a source code control system like
  {\textsc{Subversion}}, allowing the proposal writing team to concentrate on the contents
  rather than the mechanics of wrangling with text fragments and revisions.
\end{abstract}

\tableofcontents

\begin{todo}{from the proposal template}
  Recommended length for the whole part B: 50--60 pages (including tables, references,
  etc.)
\end{todo}
\chapter{Scientific and Technical Quality}\label{chap:quality}
\begin{todo}{from the proposal template}
  Maximum length for the whole of Section 1 –-- twenty pages, not including the tables in
  Section 1.3
\end{todo}

\section{Concept and Objectives}\label{sec:objectives}
\begin{todo}{from the proposal template}
  Explain the concept of your project. What are the main ideas that led you to propose
  this work?  Describe in detail the S\&T objectives. Show how they relate to the topics
  addressed by the call. The objectives should be those achievable within the project, not
  through subsequent development. They should be stated in a measurable and verifiable
  form, including through the milestones that will be indicated under Section 1.3 below.
\end{todo}

%%% Local Variables: 
%%% mode: latex
%%% TeX-master: "propB"
%%% End: 

\section{Progress beyond the State-of-the-Art}\label{sec:progress}
\begin{todo}{from the proposal template}
 Describe the state-of-the-art in the area concerned, and the advance that the proposed
  project would bring about. If applicable, refer to the results of any patent search you
  might have carried out.
\end{todo}

%%% Local Variables: 
%%% mode: latex
%%% TeX-master: "propB"
%%% End: 

\section{Scientific/Technical Methodology and Work Plan}\label{sec:methodology}
\begin{todo}{from the proposal template}
  A detailed work plan should be presented, broken down into work packages\footnote{A work
    package is a major sub-division of the proposed project with a verifiable end-point –
    normally a deliverable or an important milestone in the overall project.} (WPs) which
  should follow the logical phases of the implementation of the project, and include
  consortium management and assessment of progress and results. (Note that your overall
  approach to management will be described later, in Section 2).

Notes: The number of work packages used must be appropriate to the complexity of the work
and the overall value of the proposed project. The planning should be sufficiently
detailed to justify the proposed effort and allow progress monitoring by the Commission.

Any significant risks should be identified, and contingency plans described
\end{todo}
\newpage\section{Objectives and Work Programme}\label{sec:workplan}

\subsection{Anticipated Total Duration of the Project}\label{sec:duration}

\begin{todo}{from the proposal template}
Please state
\begin{itemize}
 \item the project's intended duration 1 and how long DFG funds will be necessary,
 \item for ongoing projects: since when the project has been active.
\end{itemize}
\end{todo}

\subsection{Objectives}\label{sec:objectives}

\begin{objective}[id=firstobj,title=Supporting Authors]
  This is the first objective, after all we have to write proposals all the time, and we
  would rather spend time on research. 
\end{objective}

\begin{objective}[id=secondobj,title=Supporting Reviewers]
  They are only human too, so let's have a heart for them as well. 
\end{objective}


\subsection{Work Programme Including Proposed Research Methods}\label{sec:wawp}

%%%%%%%%%%%%%%%%%%%%%%%%%%%%%%%%%%%%%%%%%%%%%%%%%%%%%%%%%%%%%%%%%%%%%%%%%%%%%%%%%
\LaTeX is the best document markup language, it can even be used for literate
programming~\cite{DK:LP,Lamport:ladps94,Knuth:ttb84}

\begin{todo}{from the proposal template}
 review the state of the art in the and your own contribution to it; probably you want to
  divide this into subsubsections. 
\end{todo}

\begin{todo}{from the proposal template}
For each applicant

Please give a detailed account of the steps planned during the proposed funding pe-
riod. (For experimental projects, a schedule detailing all planned experiments should
be provided.)

The quality of the work programme is critical to the success of a funding proposal. The
work programme should clearly state how much funding will be requested, why the
funds are needed, and how they will be used, providing details on individual items
where applicable.

Please provide a detailed description of the methods that you plan to use in the project:
What methods are already available? What methods need to be developed? What as-
sistance is needed from outside your own group/institute?
Please list all cited publications pertaining to the description of your work programme
in your bibliography under section 3.
\end{todo}

The project is organized around \pdatacount{all}{wa} large-scale work areas which correspond
to the objectives formulated above. These are subdivided into \pdatacount{all}{wp} work
packages, which we summarize in Figure~\ref{fig:wplist}. Work area
\WAref{mansubsus} will run over the whole project\ednote{come up with a better
  example, this is still oriented towards an EU project} duration of {\pn}. All
{\pdatacount{systems}{wp}} work packages in {\WAref{systems}} will and have to be
covered simultaneously in order to benefit from design-implementation-application feedback
loops.

\wpfig

\begin{workplan}
\begin{workarea}[id=mansubsus,title={Management, Support \& Sustainability}, short=Management]
  This work-group corresponds to Objective \OBJref{firstobj} and has two work packages:
  one for management proper ({\WPref{management}}), and one each for
  dissemination ({\WPref{dissem}})
   
  This work group ensures the dissemination and creation of the periodic integrative
  reports containing the periodic Project Management Report, the Project Management
  Handbook, an Knowledge Dissemination Plan ({\WPref{management}}), the Proceedings of the
  Annual {\pn} Summer School as well as non-public Dissemination and Exploitation plans
  ({\WPref{dissem}}), as well as a report of the {\pn} project milestones.
   
\begin{workpackage}[id=management,lead=jacu,
  title=Project Management,
 jacuRM=2,jacuRAM=8,pcgRM=2]
  Based on the ``Bewilligungsbescheid'' of the DFG, and based on the financial and
  administrative data agreed, the project manager will carry out the overall project
  management, including administrative management.  A project quality handbook will be
  defined, and a {\pn} help-desk for answering questions about the format (first
  project-internal, and after month 12 public) will be established. The project management
  will consist of the following tasks
\begin{tasklist} 
\begin{task}[id=foo,wphases=0-3,requires=\taskin{t1}{dissem}]
  To perform the administrative, scientific/technical, and financial management of the
  project 
\end{task}
\begin{task}[wphases=13-17!.5]
  To co-ordinate the contacts with the DFG and other funding bodies, building on the
  results in \taskref{management}{foo}
\end{task}
\begin{task}
  To control quality and timing of project results and to resolve conflicts
\end{task}
\begin{task}
  To set up inter-project communication rules and mechanisms
\end{task}
\end{tasklist}

\end{workpackage}
 
\begin{workpackage}[id=dissem,lead=pcg,
 title=Dissemination and Exploitation,
pcgRM=8,jacuRAM=2] 
Much of the activity of a project involves small groups of nodes in joint work. This work
 package is set up to ensure their best wide-scale integration, communication, and
 synergetic presentation of the results. Clearly identified means of dissemination of
 work-in-progress as well as final results will serve the effectiveness of work within the
 project and steadily improve the visibility and usage of the emerging semantic services.


 The work package members set up events for dissemination of the research and
 work-in-progress results for researchers (workshops and summer schools), and for industry
 (trade fairs). An in-depth evaluation will be undertaken of the response of test-users.
 
 \begin{tasklist}
  \begin{task}[id=t1,wphases=6-7]
    sdfkj
  \end{task}
  \begin{task}[wphases=12-13]
    sdflkjsdf
  \end{task}
  \begin{task}[wphases=18-19]
    sdflkjsdf
  \end{task}
 \begin{task}[wphases=22-24] 
 \end{task}
\end{tasklist}

Within two months of the start of the project, a project website will go live. This
website will have two areas: a members' area and a public area.\ldots
\end{workpackage}
\end{workarea}
 

\begin{workarea}[id=systems,title={System Development}]
  This workarea does not correspond to \OBJtref{secondobj}, but it has two work packages:
  one for the development of the {\LaTeX} class ({\WPref{class}}), and for the
  proposal template ({\WPref{temple}})

  This work group coordinates the system development.

\begin{workpackage}[id=class,lead=jacu,
                    title=A LaTeX class for EU Proposals,short=Class,
                   jacuRM=12,jacuRAM=8,pcgRM=12,pcgRAM=2]
We plan to develop a {\LaTeX} class for marking up EU Proposals

We will follow strict software design principles, first comes a
requirements analys, then \ldots
\begin{tasklist}
  \begin{task}[wphases=0-2]
    sdfsdf
  \end{task}
  \begin{task}[wphases=4-8]
    sdfsdf
  \end{task}
  \begin{task}[id=t3,wphases=10-14]
    sdfsdf
  \end{task}
  \begin{task}[wphases=20-24]
    sdfsdfd
  \end{task}
\end{tasklist}
\end{workpackage} 

\begin{workpackage}[id=temple,lead=pcg,
  title= Proposal Template,short=Template,jacuRM=12]

We plan to develop a template file for {\pn} proposals

We abstract an example from existing proposals
\begin{tasklist}
  \begin{task}[wphases=6-12]
    sdfdsf 
  \end{task}
  \begin{task}[id=temple2,wphases=18-24,requires=\taskin{t3}{class}]
    sdfsdf
  \end{task} 
\end{tasklist}
\end{workpackage}

\begin{workpackage}[id=workphase,title=A work package without tasks,
  wphases=0-4!.5]
  
  And finally, a work package without tasks, so we can see the effect on the gantt chart
  in fig~\ref{fig:gantt}.
\end{workpackage}
\end{workarea}
\end{workplan} 

\ganttchart[draft,xscale=.45] 

\subsection{Data Handling}\label{sec:data}

The \pn project will not systematically produce researchdata. All project results will be
published for at least $x$ years at our archive at \url{http://example.org}.

\subsection{-- 2.7 (Other Information / Explanations on the Proposed Investigations / Information on Scientific and Financial Involvement of International
  Cooperation Partners) \qquad \sf n/a}


%%% Local Variables: 
%%% mode: LaTeX
%%% TeX-master: "proposal"
%%% End: 

% LocalWords:  workplan.tex wplist dfgcount wa mansubsus duratio ipower wpfig
% LocalWords:  ganttchart xscale workplan workarea pdataref dissem workpackage foo
% LocalWords:  tasklist taskin taskref sdfkj sdflkjsdf sdfsdf sdfsdfd sdfdsf pn
% LocalWords:  firstobj secondobj pdatacount WAref ednote OBJref pcgRM pcg
% LocalWords:  ldots OBJtref workphase


\newpage
\subsection{Work Package List}\label{sec:wplist}

\begin{todo}{from the proposal template}
Please indicate one activity per work package:
RTD = Research and technological development; DEM = Demonstration; MGT = Management of the consortium
\end{todo}

%\makeatletter\wp@total@RM{management}\makeatother
\wpfigstyle{\footnotesize}
\wpfig[pages,type,start,end]

\newpage\subsection{List of Deliverables}\label{sec:deliverables}

\begin{todo}{from the proposal template}
\begin{compactenum}
\item Deliverable numbers in order of delivery dates. Please use the numbering convention <WP number>.<number of deliverable within
that WP>. For example, deliverable 4.2 would be the second deliverable from work package 4.
\item Please indicate the nature of the deliverable using one of the following codes:
R = Report, P = Prototype, D = Demonstrator, O = Other
\item Please indicate the dissemination level using one of the following codes:
PU = Public
PP = Restricted to other programme participants (including the Commission Services).
RE = Restricted to a group specified by the consortium (including the Commission Services).
CO = Confidential, only for members of the consortium (including the Commission Services).
\end{compactenum}
\end{todo}
We will now give an overview over the deliverables and milestones of the work
packages. Note that the times of deliverables after month 24 are estimates and may change
as the work packages progress.

In the table below, {\emph{integrating work deliverables}} (see top of
section~\ref{sec:wplist}) are printed in boldface to mark them. They integrate
contributions from multiple work packages. \ednote{CL: the rest of this paragraph does not
  comply with the EU guide for applicants, needs to be rewritten}These can have the
dissemination level ``partial'', which indicates that it contains parts of level
``project'' that are to be disseminated to the project and evaluators only. In such
reports, two versions are prepared, and disseminated accordingly.

{\footnotesize\inputdelivs{8cm}}


%%% Local Variables: 
%%% mode: latex
%%% TeX-master: "propB"
%%% End: 

\newpage\subsection{List of Milestones}\label{sec:milestones}

\begin{todo}{from the proposal template}
  Milestones are control points where decisions are needed with regard to the next stage
  of the project. For example, a milestone may occur when a major result has been
  achieved, if its successful attainment is a requirement for the next phase of
  work. Another example would be a point when the consortium must decide which of several
  technologies to adopt for further development.

  Means of verification: Show how you will confirm that the milestone has been
  attained. Refer to indicators if appropriate. For examples: a laboratory prototype
  completed and running flawlessly, software released and validated by a user group, field
  survey complete and data quality validated.
\end{todo}


The work in the {\pn} project is structured by seven milestones, which coincide with the
project meetings in summer and fall.  Since the meetings are the main face-to-face
interaction points in the project, it is suitable to schedule the milestones for these
events, where they can be discussed in detail. We envision that this setup will give the
project the vital coherence in spite of the broad mix of disciplinary backgrounds of the
participants.\ednote{maybe automate the milestones}

\begin{milestones}
  \milestone[id=kickoff,verif=Inspection,month=1]
    {Initial Infrastructure}
    {Set up the organizational infrastructure, in particular: Web Presence, project TRAC,\ldots}
  \milestone[id=consensus,verif=Inspection,month=24]{Consensus} {Reach Consensus on the
    way the project goes}
  \milestone[id=exploitation,verif=Inspection,month=36]{Exploitation}{The exploitation
    plan should be clear so that we can start on this in the last year.}
  \milestone[id=final,verif=Inspection,month=48]{Final Results}{all is done}
\end{milestones}

%%% Local Variables: 
%%% mode: latex
%%% TeX-master: "propB"
%%% End: 

% LocalWords:  pn ednote verif ldots


\subsection{Work Package Descriptions}\label{sec:workpackages}
\begin{workplan}
%%%%%%%%%%%%%%%%%%%%%%%%%%%%%%
%  Work Package Description  %
%%%%%%%%%%%%%%%%%%%%%%%%%%%%%%

\begin{workpackage}{MANAGEMENT  WORK PACKAGE}
  \label{wp:management} %change and use appropriate description

  %%%%%%%%%%%%%%%%%% TOP TABLE %%%%%%%%%%%%%%%%%%%%%%%%%%%%%
  % Data for the top table
  \wpstart{1} %Starting Month
  \wpend{36} %End Month
  \wptype{Activity type} %RTD, DEM, MGT, or OTHER

  % Person Months per participant (required, max 7, * for leader)  
  % syntax: \personmonths{Participant number}{value}    (not wp leader)
  %     or  \personmonths{Participant short name}{value} (not wp leader)
  %         \personmonths*{Participant number}{value}    (wp leader)
  % for example:
  \personmonths*{UoC}{12}
  \personmonths{UoP2}{3}
  \personmonths{UoP3}{2}
  % etc.

  \makewptable % Work package summary table
    
  % Work Package Objectives
  \begin{wpobjectives}
    This work package has the following objectives:
    \begin{enumerate}
    \item To develop ....
    \item To apply this ....
    \item etc.
    \end{enumerate}
  \end{wpobjectives}
  
  % Work Package Description
  \begin{wpdescription}
    % Divide work package into multiple tasks.
    % Use \wptask command
    % syntax: \wptask{leader}{contributors}{start-m}{end-m}{title}{description}   
 
    Description of work carried out in WP, broken down into tasks, and
    with role of partners list. Use the \texttt{\textbackslash wptask} command.

    \wptask{UoC}{UoC}{1}{12}{Test}{
      \label{task:wp1test}
      Here we will test the WP Task code. 
    }
    \wptask{UoC}{UoC}{6}{9}{Integrate}{
      \label{task:wp1integrate}
      In this task UZH will integrate the work done in ~\ref{task:wp1test}.
    }    
    \wptask{UoP3}{All other}{9}{12}{Apply}{
      Here all the WP participants will apply the results to...
    }
    
    \paragraph{Role of partners}
    \begin{description}
    \item[Participant short name] will lead Task~\ref{task:wp1integrate}.
    \item[UoC] will..
    \end{description}
  \end{wpdescription}
  
  % Work Package Deliverable
  \begin{wpdeliverables}
    % Data for the deliverables and milestones  tables
    % syntax: \deliverable[delivery date]{nature}{dissemination
    % level}{description} 
    %
    % nature: R = Report, P = Prototype, D = Demonstrator, O = Other
    % dissemination level: PU = Public, PP = Restricted to other
    % programme participants (including the Commission Services), RE =
    % Restricted to a group specified by the consortium (including the
    % Commission Services), CO = Confidential, only for members of the
    % consortium (including the Commission Services).
    % 
    % \wpdeliverable[date]{R}{PU}{A report on \ldots}

    \wpdeliverable[36]{UoC}{R}{PU}{Report on the definition of the model
      specifications.}\label{dev:wp1specs}
    
    \wpdeliverable[12]{UoP3}{R}{PU}{Report on Feasibility study for the model
      implementation.}\label{dev:wp1implementation}

    \wpdeliverable[24]{UoP2}{R}{PU}{Prototype of model
      implementation.}\label{dev:wp1prototype}

  \end{wpdeliverables}

\end{workpackage}


%%% Local Variables:
%%% mode: latex
%%% TeX-master: "proposal-main"
%%% End:
\newpage
\begin{workpackage}%
[id=dissem,type=RTD,lead=efo,
 wphases=10-24!1,
 title=Dissemination and Exploitation,short=Dissemination,
 efoRM=8,jacuRM=2,barRM=2,bazRM=2]
We can state the state of the art and similar things before the summary in the boxes
here. 
\wpheadertable

\begin{wpobjectives}
  Much of the activity of a project involves small groups of nodes in joint work. This
  work package is set up to ensure their best wide-scale integration, communication, and
  synergetic presentation of the results. Clearly identified means of dissemination of
  work-in-progress as well as final results will serve the effectiveness of work within
  the project and steadily improve the visibility and usage of the emerging semantic
  services.
\end{wpobjectives}

\begin{wpdescription}
  The work package members set up events for dissemination of the research and
  work-in-progress results for researchers (workshops and summer schools), and for
  industry (trade fairs). An in-depth evaluation will be undertaken of the response of
  test-users.

  Within two months of the start of the project, a project website will go live. This
  website will have two areas: a members' area and a public area.\ldots
\end{wpdescription}

\begin{wpdelivs}
  \begin{wpdeliv}[due=2,id=website,nature=O,dissem=PU,miles=kickoff]
     {Set-up of the Project web server}
   \end{wpdeliv}
   \begin{wpdeliv}[due=8,id=ws1proc,nature=R,dissem=PU,miles={kickoff}]
     {Proceedings of the first {\pn} Summer School.}
   \end{wpdeliv}
   \begin{wpdeliv}[due=9,id=dissem,nature=R,dissem=PP]
     {Dissemination Plan}
   \end{wpdeliv}
   \begin{wpdeliv}[due=9,id=exploitplan,nature=R,dissem=PP,miles=exploitation]
     {Scientific and Commercial Exploitation Plan}
   \end{wpdeliv}
   \begin{wpdeliv}[due=20,id=ws2proc,nature=R,dissem=PU,miles={exploitation}]
     {Proceedings of the second {\pn} Summer School.}
   \end{wpdeliv}
   \begin{wpdeliv}[due=32,id=ss1proc,nature=R,dissem=PU,miles={exploitation}]
     {Proceedings of the third {\pn} Summer School.}
   \end{wpdeliv}
   \begin{wpdeliv}[due=44,id=ws3proc,nature=R,dissem=PU,miles=exploitation]
     {Proceedings of the fourth {\pn} Summer School.}
   \end{wpdeliv}
 \end{wpdelivs}
\end{workpackage}

%%% Local Variables: 
%%% mode: LaTeX
%%% TeX-master: "propB"
%%% End: 

% LocalWords:  wp-dissem.tex workpackage dissem efo fromto bazRM wpheadertable
% LocalWords:  wpobjectives wpdescription ldots wpdelivs wpdeliv ws1proc pn
% LocalWords:  exploitplan ws2proc ss1proc ws3proc pdataRef deliv
% LocalWords:  mansubsusintReport
\newpage
\begin{workpackage}[id=class,type=RTD,lead=jacu,
                    wphases=3-9!1,
                    title=A {\LaTeX} class for EU Proposals,short=Class,
                    jacuRM=12,barRM=12]
We can state the state of the art and similar things before the summary in the boxes
here. 
\wpheadertable
\begin{wpobjectives}
\LaTeX is the best document markup language, it can even be used for literate
programming~\cite{DK:LP,Lamport:ladps94,Knuth:ttb84}

  To develop a {\LaTeX} class for marking up EU Proposals
\end{wpobjectives}

\begin{wpdescription}
  We will follow strict software design principles, first comes a requirements analys,
  then \ldots
\end{wpdescription}

\begin{wpdelivs}
  \begin{wpdeliv}[due=6,id=req,nature=R,dissem=PP,miles=kickoff]
     {Requirements analysis}
   \end{wpdeliv}
   \begin{wpdeliv}[due=12,id=spec,nature=R,dissem=PU,miles=consensus]
     {{\pn} Specification }
   \end{wpdeliv}
   \begin{wpdeliv}[due=18,id=demonstrator,nature=P,dissem=PU,miles={consensus,final}]
     {First demonstrator ({\tt{article.cls}} really)}
   \end{wpdeliv}
   \begin{wpdeliv}[due=24,id=proto,nature=P,dissem=PU,miles=final]
     {First prototype}
   \end{wpdeliv}
    \begin{wpdeliv}[due=36,id=release,nature=P,dissem=PU,miles=final]
      {Final {\LaTeX} class, ready for release}
    \end{wpdeliv}
  \end{wpdelivs}
Furthermore, this work package contributes to {\pdataRef{deliv}{managementreport2}{label}} and
{\pdataRef{deliv}{managementreport7}{label}}.
\end{workpackage}

%%% Local Variables: 
%%% mode: LaTeX
%%% TeX-master: "propB"
%%% End: 
\newpage
\begin{workpackage}[id=temple,type=DEM,lead=bar,
  wphases=6-12!1,
  title={\pn} Proposal Template,short=Template,barRM=6,bazRM=6]
We can state the state of the art and similar things before the summary in the boxes
here. 
\wpheadertable

\begin{wpobjectives}
  To develop a template file for {\pn} proposals
\end{wpobjectives}

\begin{wpdescription}
  We abstract an example from existing proposals
\end{wpdescription}

\begin{wpdelivs}
  \begin{wpdeliv}[due=6,id=req,nature=R,dissem=PP,miles=kickoff]
    {Requirements analysis}
  \end{wpdeliv}
  \begin{wpdeliv}[due=12,id=spec,nature=R,dissem=PU,miles=consensus]
    {{\pn} Specification }
  \end{wpdeliv}
  \begin{wpdeliv}[due=18,id=demonstrator,nature=D,dissem=PU,miles={consensus,final}]
    {First demonstrator ({\tt{article.cls}} really)}
  \end{wpdeliv}
  \begin{wpdeliv}[due=24,id=proto,nature=P,dissem=PU,miles=final]
    {First prototype}
  \end{wpdeliv}
  \begin{wpdeliv}[due=36,id=release,nature=P,dissem=PU,miles=final]
    {Final Template, ready for release}
  \end{wpdeliv}
\end{wpdelivs}
Furthermore, this work package contributes to {\pdataRef{deliv}{managementreport2}{label}} and
{\pdataRef{deliv}{managementreport7}{label}}.
\end{workpackage}

%%% Local Variables: 
%%% mode: LaTeX
%%% TeX-master: "propB"
%%% End: 

% LocalWords:  wp-temple.tex workpackage fromto pn bazRM wpheadertable wpdelivs
% LocalWords:  wpobjectives wpdescription wpdeliv req dissem tt article.cls
% LocalWords:  pdataRef deliv systemsintReport
\newpage
\end{workplan}
\newpage\subsection{Significant Risks and Associated Contingency Plans}\label{sec:risks}
\begin{todo}{from the proposal template}
  Describe any significant risks, and associated contingency plans
\end{todo}
\begin{oldpart}{need to integrate this somewhere. CL: I will check other proposals to see how they did it; the Guide does not really prescribe anything.}
\paragraph{Global Risk Management}
The crucial problem of \pn (and similar endeavors that offer a new basis for communication
and interaction) is that of community uptake: Unless we can convince scientists and
knowledge workers industry to use the new tools and interactions, we will
never be able to assemble the large repositories of flexiformal mathematical knowledge we
envision. We will consider uptake to be the main ongoing evaluation criterion for the network.
\end{oldpart}

%%% Local Variables: 
%%% mode: latex
%%% TeX-master: "propB"
%%% End: 



%%% Local Variables: 
%%% mode: latex
%%% TeX-master: "propB"
%%% End: 

% LocalWords:  workplan newpage wplist makeatletter makeatother wpfig
% LocalWords:  workpackages wp-dissem wp-class wp-temple

%%% Local Variables: 
%%% mode: LaTeX
%%% TeX-master: "propB"
%%% End: 

\newpage
% !Mode:: "TeX:UTF-8"

\chapter{示例:实现方案}
本章首先介绍了本系统实现的设计原则与目标,然后描述了系统的整体架构与总体设计方案,接着阐述了系统的各个模块实现时的解决方案,最后给出了系统在~Android~具体实现时的实现方法和使用说明。
\section{设计目标与原则}
	\subsection{设计目标}
	本系统能够检测已知恶意软件及其变种,并能通过模糊检测发现具有相似恶意行为的未知恶意软件,为~Android~平台这样的开放式移动平台提供安全保障,可广泛用于各种型号的~Android~设备。系统具体设计目标如下:
	\begin{enumerate}
	\item 适用于目前主流的~Android~平板及手机,至少可运行于3.0版本系统。
	\item 能够检测用户指定的程序是否为恶意程序。
	\item 能够自动检测设备上的所有程序,并可定时检测。
	\item 能够监控设备的程序安装行为,自动检测安装的程序是否为恶意程序。
	\item 能够保证本程序自身的特征库不被破坏,并能及时修复和更新特征库。
	\item 能够保证用户使用方便。
	\end{enumerate}
	
	\subsection{设计原则}
	 从安全产品的特点出发,本系统设计与实现将遵循下列一些设计原则:
	\begin{enumerate}
	\item 高效性
	
	 系统运行效率高,可快速实现对目标程序的特征提取与检测,并将结果用最清晰简洁的方式告知用户。
	\item 灵活性
	
	 为用户提供的各种功能具有可选性,用户可根据自己的需要选择其中的功能,而且用户可自行决定如何处理检测结果为恶意的程序。
	
	\item 实用性
	
	本系统应按照实用性原则进行设计,在保证对程序检测的同时,力求用户界面简洁友好。
	
	\item 可扩展性
	
	 目前本系统只检测程序的~Java~实现部分,但是还有极少数程序代码是用~C~语言编写的,在后续的开发过程中,可以在不改变程序结构的前提下,实现这一部分的检测功能。
	
	\item 健壮性 \par
	 系统应具有应对非法操作的能力,并且当针对于本系统的恶意攻击到来时,可以及时防御,防止自身特征库遭到损坏。
	\end{enumerate}
	
\section{系统方案}
	 本系统由两部分构成,第一部分是产品部分,即~Android~应用程序,采用~Java~作为编程语言。第二部分是检测算法模型构建部分,采用~Matlab~实现,其输出的模型数据供产品部分作为特征库使用。故本系统的特征库构建于~PC~端,而对目标程序的检测运行于~Android~端。从而将构建过程中包含的巨大计算量留在~PC~端。其具体结构如图~\ref{system}~所示。
\pic[htbp]{系统结构}{height=5.86cm}{system}

	 其中程序信息抽取模块是本系统检测恶意软件的基础,特征检测模块是系统的核心与实现难点。特征检测模块中又分为变种检测模块和模糊检测模块,变种检测模块检测待检软件是否是已知恶意软件的变种,而模糊检测模块实现的是本系统从人脸匹配中引入的新型检测手段,可对特征库中不存在的未知恶意软件进行检测。
	
\section{系统模块的实现}
	\subsection{程序分解模块}
		
		 Android~的程序文件为~APK~格式,APK~文件是~Android~最终的运行程序,是~Android~Package~的全称。类似于~Symbian~ 操作系统中~sis~文件,APK~文件其实是~Zip~文件格式,但后缀名被修改为~APK。通过解压,可以看到~Dex~ 文件。Dex~是~Dalvik~VM~executes~的全称,即~Dalvik~虚拟机可执行文件,并非~Java~ME~的字节码而是~Dalvik~字节码。\\
		一个APK文件结构为:
		\begin{enumerate}
		\item META-INF$\backslash$————签名信息,用来保证~apk~包的完整性和系统的安全,jar~文件经常可以看到;
		\item res$\backslash$————资源文件夹,包括程序中使用的图片,布局文件等;
		\item AndroidManifest.xml$\backslash$————项目配置清单,但不是明文的XML格式,无法直接打开阅读;
		\item classes.dex$\backslash$————Dalvik~可执行二进制文件,在运行时被动态优化为dey文件并由Dalvik虚拟机解释执行	 ;
		\item resources.arsc$\backslash$————编译后的二进制资源文件,资源文件打包而成,字符串值(源码中的/value/Strings.xml)就在其中;
		\item lib$\backslash$————动态链接库文件;
		\item assets$\backslash$————原始文件文件夹,其中的文件不会被压缩,也不能像~res~目录下的资源文件一样通过资源类引用。
		\end{enumerate}\par
		 图~\ref{apk}~是我们解压缩helloworld.apk文件后看到的内容,可以看到其结构跟工程结构有些类似。
		
\pic[htbp]{helloworld.apk的结构}{width=0.5\textwidth}{apk}

		
		 classes.dex~文件是~Java~源码编译后生成的~Java~字节码文件。但由于~Android~使用的~Dalvik~虚拟机与标准的~Java~虚拟机是不兼容的,Dex~文件与~Class~文件相比,不论是文件结构还是~opcode~都不一样。目前常见的~Java~ 反编译工具都不能处理~Dex~文件。Android~SDK~中提供了一个~Dex~文件的反编译工具~dexdump。用法为,dexdump -d -f -h  xxx.dex。
		\\
		指令参数解释:\\
		-d : 反编译程序段\\
		-f : 从文件头显示摘要信息\\
		-h : 显示文件头详情\\
		-C : 反编译低级符号名\\
		-S : 只计算大小 \par
		 在知道了程序安装包是~Zip~编码之后,我们就可以通过遍历~Zip~包中包含的项目名,找到程序的二进制文件,即~Dex~文件。并在内存中建立一段缓冲区,可将~Dex~文件读入内存,再写到指定的临时文件中。分解流程如图~\ref{flow1}~所示。
	
\pic[htbp]{程序分解流程}{height = 5.76cm}{flow1}	
		
	\subsection{构建特征向量模块}
		 特征向量是数据挖掘中的一个概念,一个数据集中的每个数据实例都可以用一组属性值来描述,每一个数据实例都具有一个特殊的目标属性,称为类属性,它表征每个数据实例归属的类。这一组属性值即是代表这个数据实例的特征向量。在我们的检测问题中,Android~系统~API~和~Java~标准函数就是我们定义的属性,而恶意和非恶意就是类属性。这一过程如图~\ref{flow3}~所示。
		
\pic[htbp]{构建特征向量流程}{height = 0.5\textwidth}{flow3}	
		
		 我们将特征向量空间($\Omega$)存储在数据库中,数据表的定义见表~\ref{omegatable}~所示。

\threelinetable[htbp]{omegatable}{0.8\textwidth}{llllX}{$\Omega$~数据表定义}
{字段&主键&类型&是否为空&备注\\
}{
ID&是&Int&NOT NULL&特征向量维度序号\\
MethodName&否&Text&NOT NULL&函数名\\
}{\item
}

		 构造特征向量时,先初始化一个全为0的特征向量,然后利用SQL查询语句确定代表某一函数名的维度序号:\\
		select ID from Omega where MethodName = “待查函数名”;\\
		 并将特征向量($\omega$)的第~ID~位设置为1。见式(\ref{omegai})。
		\begin{equation}\label{omegai}
		\omega_i =
		\begin{cases}
		1 & i = ID \\
		0 & else
		\end{cases}
		\end{equation}
	\subsection{特征库构建模块}
		 特征库构建模块实现于~PC~端,恶意样本来自各大权威机构公布的数据,详见第\pageref{omegai}页\ref{omegai}小节,其流程如图~\ref{flow5}~所示,特征库数据模型构建方法如下:
		
\begin{pics}[htbp]{特征库构建流程}{flow5}
  \addsubpic{KNN特征库构建流程}{width=0.3\textwidth}{flow5-1}
  \addsubpic{K-L~变换矩阵构建流程}{width=0.3\textwidth}{flow5-2}
  \addsubpic{LDA~投影矩阵构建流程}{width=0.3\textwidth}{flow5-3}
\end{pics}

		\begin{enumerate}
		\item 最近邻居(KNN)算法\par
		 KNN~算法的特征库就是恶意程序样本的特征向量集合,我们将这些特征向量存储到数据库中。
		
		\item 主成分分析(PCA)算法\par
		 考虑到文献\citeup{wangang1912}\citeup{zhaokaihua1995}中的适用条件,由于我们的特征向量维度远大于样本数量,所以需要去掉冗余数据,使训练数据矩阵为可逆矩阵才能使用线性判别分析(LDA)算法。

		 PCA~方法主要是通过对协方差矩阵进行本征分解,以得出数据的主成分(即本征矢量)与它们的权值(即本征值)。PCA~ 提供了一种降低数据维度的有效办法;如果分析者在原数据中除掉最小的本征值所对应的成分,那么所得的低维度数据必定是最优化的(也即这样降低维度必定是失去信息最少的方法)。

		 我们的目标是把高维的数据集~$\Omega_B$~和~$\Omega_M$~变换成具有较小维度的数据集~$Y_B$~和~$Y_M$。$Y_B$~和~$Y_M$~是矩阵~$\Omega_B$~和~$\Omega_M$~的~Karhunen–Loève~变换(K-L~变换)。即~$\mathbf{Y}=\mathbb{KLT}\{\mathbf{X}\}$。\\
		计算特征向量平均值见式(\ref{mean})。
		\begin{equation}\label{mean}
		u=\dfrac{1}{N} \sum_{\omega \in \Omega_B \cup \Omega_M} \omega
		\end{equation}
		 从~$\Omega_B$~和~$\Omega_M$~中减去平均值~$u$~见式(\ref{Omega_u})。
		\begin{equation}\label{Omega_u}
		\begin{split}
		B & =\begin{vmatrix}\Omega_B \\ \Omega_M \end{vmatrix}-hu\\
		  & \text{其中h是全为1的列向量。}
		\end{split}
		\end{equation}
		求协方差矩阵~C~见式(\ref{getC})。
		\begin{equation}\label{getC}
		C=B \cdot B^T
		\end{equation}
		 计算~C~的特征值和特征向量,提取不为0的特征值所对应的特征向量,构成~K-L~变换矩阵~W。\\
		 所以,$Y_B$~和~$Y_M$~可由式(\ref{ybym})计算。
		\begin{equation}\label{ybym}
		\begin{split}
		Y_B &= \Omega_B \cdot W \\
		Y_M &= \Omega_M \cdot W
		\end{split}
		\end{equation}
		
		最后我们将~K-L~变换矩阵~W~存储在数据库中。
				
		\item Fisher~线性判别分析(LDA)\par
\threelinetable[htbp]{ldaparameterdeftable}{\textwidth}{lXlX}{LDA算法变量定义}
{变量&定义&变量&定义\\
}{
        $S_b$ 	& 样本类间离散度矩阵 & $x$		& 一个程序\\
		$S_i$ 	& 样本类内离散度矩阵&$X_M$	& 恶意程序集合\\
		$S_w$ 	& 总类内离散度矩阵&$X_B$	& 非恶意程序集合\\
		$W$   	& 投影方向向量&$y_M$ 	& 恶意样本的投影值 \\
		$J_F(W)$& Fisher~准则函数&$y_B$ 	& 非恶意样本的投影值\\
		$M$		& 恶意(Malice)的缩写&$y_0$ 	& 识别阈值点\\
		$B$		& 非恶意(Benign)的缩写&&\\
}{
\item
}

应用统计方法解决模式识别问题时,一再碰到的问题之一是维数问题。在低维空间里解析上或计算上行得通的方法,在高维空间里往往行不通。因此,降低维数有时就成为处理实际问题的关键。在数学上总是可以把高维空间样本投影到一条直线上,形成一维空间,即把维数压缩到一维。但是投影方向有无数种,若把样本投影到一条任意的直线上,可能使几类样本混在一起无法区分,如图~\ref{fisher1}~所示。。但在一般情况下,总可以找到某个方向,使在这个方向的直线上,样本的投影能分开得最好,如图~\ref{fisher2}~所示。。 问题是如何根据实际情况找到这条最好的、最易于区分的投影线。这就是~Fisher~法所要解决的基本问题。\par

\begin{pics}[htbp]{Fisher~线性判别基本原理}{fisher}
  \addsubpic{最优方向投影}{width=0.4\textwidth}{fisher1}
  \addsubpic{K-L~任意方向投影}{width=0.4\textwidth}{fisher2}
\end{pics}

		 描述~LDA~算法前,首先定义几个基本变量,变量定义见表~\ref{ldaparameterdeftable}。LDA~算法步骤如下:

		\begin{enumerate}
		\item 计算样本均值向量~$m_i$:
		\begin{equation*}
		m_i=\dfrac{1}{N_i}\sum_{y \in Y_i}y ~~~~,i=B,M
		\end{equation*}
		\item 计算样本类内离散度矩阵~$S_i$~和总类内离散度矩阵~$S_w$:
		\begin{align}
		S_i & = \sum_{y \in Y_i}(y-m_i)(x-m_i)^T ~~~~,i=B,M\\
		S_w & = P(x|x \in X_B)S_B + P(x|x\in X_M)S_M
		\end{align}
		\item 计算样本类间离散度矩阵~$S_b$:
		\begin{equation*}
		S_b=P(x|x \in X_B)P(x|x\in X_M)(m_B-m_M)(m_B-m_M)^T
		\end{equation*}
		$P(x|x \in X_B)$~和~$P(x|x\in X_M)$~是恶意程序和非恶意程序的先验概率,根据目前~Android~市场的情况,我们取~$P(x|x\in X_M)=0.001$。
		\item Fisher准则函数为:
		\begin{equation}
		J_F(W) = \dfrac{W^T S_b W}{W^T S_w W}
		\end{equation}
		 为求函数取极大值时的~$W^*$。可用拉格朗日乘数法,定义拉格朗日函数为:
		\begin{equation*}
		\dfrac{L(W,\lambda)}{W} = S_bW-\lambda S_w W
		\end{equation*}
		另偏导数为零,得
		\begin{equation*}
		S_b W^* = \lambda S_w W^*
		\end{equation*}
		 其中~$W^*$~就是~$J_F(W)$~的极值解。因为~$S_w$~可逆,等式两边左乘~$S_w^{-1}$,可得
		\begin{equation*}
		S_w^{-1} S_b W^* = \lambda W^*
		\end{equation*}
		所以求~$W^*$~即求矩阵~$S_w^{-1} S_b$~的特征值问题。在我们这个特殊情况下,只有两种类别,故
		\begin{equation*}
		S_b W^* = (m_B-m_M)(m_B-m_M)^T W^*
		\end{equation*}
		 其值为一标量,所以对~$W$~投影方向无影响。忽略这个标量的比例因子可得,
		\begin{equation*}
		W^* = S_w^{-1} (m_B-m_M)
		\end{equation*}
		\item 求出~$W^*$~后即可计算:
		\begin{align}
		y_M & = mean(W^*T \cdot Y_M) \\
		y_B & = mean(W^*T \cdot Y_B) \\
		y_0 & = \dfrac{m_B+m_M}{2}  +  \dfrac{\ln (P(x|x\in X_B)/P(x|x\in X_M))}{N_B+N_M-2}
		\end{align}
		 最后将LDA变换矩阵~$W$、$y_M$、$y_B$、$y_0$保存在数据库中。
		\end{enumerate}	
		\end{enumerate}
\section{本章小结}
本章介绍了~AFace~系统的实现细节,它是一个由~Android~端和~PC~端两部分组成的系统。我们首先介绍了系统的设计原则。然后介绍了我们在此原则下设计的检测流程及其背后的检测原理。最后,作为一款完整的产品,还介绍了系统的产品软件架构以及产品的界面设计。
\newpage
\chapter{Impact}
\label{cha:impact}


\section{Expected impact} 
\label{sec:expected-impact}
\instructions{
\textit{Please be specific, and provide only information that applies to the proposal and its objectives. Wherever possible, use quantified indicators and targets.}\\
\begin{itemize}
\item Describe how your project will contribute to the expected impacts set out in the work programme under the relevant topic. 
\item Describe the importance of the technological outcome with regards to its transformational impact on science, technology and/or society.
\item Describe the empowerment of new and high-potential actors towards future technological leadership.
\end{itemize}
}





\section{Measures to maximize impact} 
\label{sec:maximize-impact}

\subsection{Dissemination and exploitation of results}
\label{sec:dissemination-exploitation}
\instructions{
\begin{itemize}
\item Provide a plan for disseminating and exploiting the project results. The plan, which should be proportionate to the scale of the project, should contain measures to be implemented both during and after the project. 
\item Explain how the proposed measures will help to achieve the expected impact of the project. 
\item Where relevant, include information on how the participants will manage the research data generated and/or collected during the project, in particular addressing the following issues:\footnote{For further guidance on research data management, please refer to the H2020 Online Manual on the Participant Portal.}
\begin{itemize}
\item What types of data will the project generate/collect?
\item What standards will be used?
\item How will this data be exploited and/or shared/made accessible for verification and re-use? If data cannot be made available, explain why.
\item How will this data be curated and preserved?
\end{itemize}
\emph{You will need an appropriate consortium agreement to manage (amongst other things) the ownership and access to key knowledge (IPR, data etc.). Where relevant, these will allow you, collectively and individually, to pursue market opportunities arising from the project's results.} \\
\emph{The appropriate structure of the consortium to support exploitation is addressed in section 3.3.}\\
\begin{itemize}
\item Outline the strategy for knowledge management and protection. Include measures to provide open access (free on-line access, such as the ``green'' or ``gold'' model) to peer-reviewed scientific publications which might result from the project.\footnote{Open access must be granted to all scientific publications resulting from Horizon 2020 actions. Further guidance on open access is available in the H2020 Online Manual on the Participant Portal.}\\
\end{itemize} 
\emph{Open access publishing (also called 'gold' open access) means that an article is immediately provided in open access mode by the scientific publisher. The associated costs are usually shifted away from readers, and instead (for example) to the university or research institute to which the researcher is affiliated, or to the funding agency supporting the research.}\\
\emph{Self-archiving (also called 'green' open access) means that the published article or the final peer-reviewed manuscript is archived by the researcher - or a representative - in an online repository before, after or alongside its publication. Access to this article is often - but not necessarily - delayed (``embargo period''), as some scientific publishers may wish to recoup their investment by selling subscriptions and charging pay-per-download/view fees during an exclusivity period.}
\end{itemize}
}

\subsection{Communication activities}
\label{sec:communication}
\instructions{
\begin{itemize}
\item Describe the proposed communication measures for promoting the project and its findings during the period of the grant. Measures should be proportionate to the scale of the project, with clear objectives.  They should be tailored to the needs of various audiences, including groups beyond the project's own community. Where relevant, include measures for public/societal engagement on issues related to the project. 
\end{itemize}
}\newpage
\chapter{Ethical Issues}\label{chap:ethical}
\begin{todo}{from the proposal template}
  Describe any ethical issues that may arise in the project. In particular, you should
  explain the benefit and burden of the experiments and the effects it may have on the
  research subject. Identify the countries where research will be undertaken and which
  ethical committees and regulatory organisations will need to be approached during the
  life of the project.

  Include the Ethical issues table below.  If you indicate YES to any issue, please
  identify the pages in the proposal where this ethical issue is described. Answering
  'YES' to some of these boxes does not automatically lead to an ethical review1.  It
  enables the independent experts to decide if an ethical review is required. If you are
  sure that none of the issues apply to your proposal, simply tick the YES box in the last
  row.
\end{todo}

\begin{small}
\begin{tabular}{|p{1em}p{11cm}|l|l|}\hline
  \multicolumn{2}{|l|}{\cellcolor{lightgray}{\strut}} & 
  \cellcolor{lightgray}{YES} & 
  \cellcolor{lightgray}{PAGE}\\\hline 
  \multicolumn{2}{|l|}{\bf{Informed Consent}} & & \\\hline
  & Does the proposal involve children?  & & \\\hline
  & Does the proposal involve patients or persons not able to give consent? & & \\\hline
  & Does the proposal involve adult healthy volunteers? & & \\\hline
  & Does the proposal involve Human Genetic Material? & & \\\hline
  & Does the proposal involve Human biological samples? & & \\\hline
  & Does the proposal involve Human data collection? & & \\\hline
  \multicolumn{2}{|l|}{\bf{Research on Human embryo/foetus}}  & & \\\hline
  & Does the proposal involve Human Embryos? & & \\\hline
  & Does the proposal involve Human Foetal Tissue / Cells? & & \\\hline
  & Does the proposal involve Human Embryonic Stem Cells? & & \\\hline
  \multicolumn{2}{|l|}{\bf{Privacy}} & & \\\hline
  & Does the proposal involve processing of genetic information 
         or personal data (eg. health, sexual lifestyle, ethnicity, 
         political opinion, religious or philosophical conviction)  & & \\\hline 
  & Does the proposal involve tracking the location or observation 
         of people? & & \\\hline 
  \multicolumn{2}{|l|}{\bf{Research on Animals}} & & \\\hline 
  & Does the proposal involve research on animals? & & \\\hline 
  & Are those animals transgenic small laboratory animals? & & \\\hline 
  & Are those animals transgenic farm animals? & & \\\hline 
  & Are those animals cloned farm animals? & & \\\hline 
  & Are those animals non-human primates?  & & \\\hline 
  \multicolumn{2}{|l|}{\bf{Research Involving Developing Countries}} & & \\\hline 
  & Use of local resources (genetic, animal, plant etc) & & \\\hline 
  & Benefit to local community (capacity building 
         i.e. access to healthcare, education etc) & & \\\hline 
  \multicolumn{2}{|l|}{\bf{Dual Use}} & & \\\hline 
  & Research having direct military application  & & \\\hline 
  & Research having the potential for terrorist abuse & & \\\hline 
  \multicolumn{2}{|l|}{\bf{ICT Implants}} & & \\\hline 
  & Does the proposal involve clinical trials of ICT implants?  & & \\\hline 
  \multicolumn{2}{|l|}{\bf\footnotesize{I CONFIRM THAT NONE OF THE ABOVE ISSUES APPLY TO MY PROPOSAL}} 
      & &\cellcolor{lightgray}{} \\\hline 
\end{tabular}
\end{small}

\section{Personal Data}

\end{proposal}
\end{document}

%%% Local Variables: 
%%% mode: LaTeX
%%% TeX-master: t
%%% End: 

% LocalWords:  efo efoRM baz bazRM miko acrolong ntelligent iting pn pnlong
% LocalWords:  textsc newpage compactht texttt euproposal.cls callname callid
% LocalWords:  challengeid objectiveid outcomeid tableofcontents
