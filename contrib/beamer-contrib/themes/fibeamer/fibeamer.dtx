% \iffalse\begin{macrocode}
%<*driver>

\documentclass{ltxdoc}
\usepackage[utf8]{inputenc} % this file uses UTF-8
\usepackage[english]{babel}
\usepackage{tgpagella}
\usepackage{tabularx}
\usepackage{hologo}
\usepackage{booktabs}
\usepackage[scaled=0.86]{berasans}
\usepackage[scaled=1.03]{inconsolata}
\usepackage[resetfonts]{cmap}
\usepackage[T1]{fontenc} % use 8bit fonts
\emergencystretch 2dd
\usepackage{hypdoc}
\usepackage{microtype}
\usepackage{ragged2e}

% Making paragraphs numbered
\makeatletter
\renewcommand\paragraph{\@startsection{paragraph}{4}{\z@}%
            {-2.5ex\@plus -1ex \@minus -.25ex}%
            {1.25ex \@plus .25ex}%
            {\normalfont\normalsize\bfseries}}
\makeatother
\setcounter{secnumdepth}{4} % how many sectioning levels to assign
\setcounter{tocdepth}{4}    % how many sectioning levels to show

% ltxdoc class options
\CodelineIndex
\MakeShortVerb{|}
\EnableCrossrefs
\DoNotIndex{}
\makeatletter
\c@IndexColumns=2
\makeatother

\begin{document}
  \RecordChanges
  \DocInput{fibeamer.dtx}
  \PrintIndex
  \RaggedRight
  \PrintChanges
\end{document}

%</driver>
%    \end{macrocode}
%<*class>
\NeedsTeXFormat{LaTeX2e}
%
% Define `\fibeamer@version` and store it in the `VERSION.tex` file \fi
{\def\fibeamer@versiondef#1#2{
  \gdef\fibeamer@version@number{#1}
  \gdef\fibeamer@version@date{#2}
  \gdef\fibeamer@version{#2 #1 fibeamer MU beamer theme}}
\fibeamer@versiondef{v1.1.7}{2017/07/05}}
% {\newwrite\f\openout\f=VERSION\write\f{\fibeamer@version}\closeout\f}
%
% \changes{v1.1.5:1} {2016/05/18}{Added \texttt{pdfencoding} to the
%   \textsf{hyperref} package setup. This fixes the problem with garbled
%   characters in PDF metadata with Lua\TeX.}
% \iffalse Use `\fibeamer@version` as the PDF creator key.
\hypersetup{%
  pdfcreator=\fibeamer@version,
  pdfencoding=auto}
% \fi
%
%%%%%%%%%%%%%%%%%%%%%%%%%%%%%%%%%%%%%%%%%%%%%%%%%%%%%%%%%%%%%%%%%%%%%%%%%%%%%%%
% 
% \changes{v1.1.3}   {2016/02/24}{Added visual tests of output
%   PDFs.}
% \changes{v1.1.1:3} {2016/01/14}{Refactored the themes of the
%   Masaryk University in Brno. Removed the useless uses of
%   \texttt{use} in the color themes. Replaced
%   \cs{setbeamertemplate} in the inner theme with the more
%   appropriate \cs{defbeamertemplate*}. Added missing
%   \texttt{\%}s. [VN]}
% \changes{v1.1.0:7} {2016/01/12}{Updated the example documents of
%   the Masaryk University in Brno to showcase the use of
%   \cs{alert}. [VN]}
% \changes{v1.1.0:6} {2016/01/11}{Updated the example documents of
%   the Masaryk University in Brno to showcase the use of the
%   \cs{tableofcontents}. [VN]}
% \changes{v1.1.0:1} {2015/11/24}{Added the new logo files for the
%   faculties of the Masaryk University in Brno and removed the old
%   logo files. [VN]}
% \changes{v1.0.4}   {2015/11/21}{Updated the user guide of the
%   Masaryk University in Brno. [VN]}
% \changes{v1.0.3}   {2015/11/20}{The example documents from the
%   \texttt{example} directory and the user guide from the
%   \texttt{guide} directory are now a part of the CTAN archive.
%   [VN]}
%
%%%%%%%%%%%%%%%%%%%%%%%%%%%%%%%%%%%%%%%%%%%%%%%%%%%%%%%%%%%%%%%%%%%%%%%%%%%%%%%
%
% \title{The beamer theme for the typesetting of thesis defense
%   presentations at the Masaryk University in Brno}
% \author{Vít Novotný}
% \date{\today}
% \maketitle
%
% \begin{abstract}
% \noindent This document details the design and the implementation
% of the \textsf{fibeamer} theme for the \textsf{beamer} document
% class. Included are technical information for anyone who wishes
% to extend the theme with their own beamer themes as well as
% information for ordinary users.
% \end{abstract}
%
% \tableofcontents
%
% \section{Basic usage}
% In order to use the \textsf{fibeamer} theme, insert
% |\usetheme|\oarg{options}|{fibeamer}| into the preamble of a
% \LaTeX\ document that uses the \textsf{beamer} document class.
% Refer to Section \ref{sec:options} for the list of available
% \textit{options}.
% 
% \section{Package options}\label{sec:options}
% \subsection{The \texttt{fonts} option}
% \begin{macro}{\iffibeamer@fonts}
% The |fonts| option instructs the package to set up the
% combination of the font families of Carlito, Arev, Iwona, Dsfont,
% and DejaVu for the typesetting of text and mathematics. This
% option is enabled by default.
%    \begin{macrocode}
\ProvidesPackage{fibeamer/beamerthemefibeamer}[\fibeamer@version]
\newif\iffibeamer@fonts
\DeclareOptionBeamer{fonts}{\fibeamer@fontstrue}
\ExecuteOptionsBeamer{fonts}
%    \end{macrocode}
% \end{macro}
% \subsection{The \texttt{nofonts} option}
% The |nofonts| option instructs the package not to alter the
% currently set text and mathematics font families.
%    \begin{macrocode}
\DeclareOptionBeamer{nofonts}{\fibeamer@fontsfalse}
%    \end{macrocode}
% \subsection{The \texttt{microtype} option}
% \changes{v1.0.2}{2015/11/18}{Added the opt-out \texttt{microtype
%   option}. [VN]}
% \begin{macro}{\iffibeamer@microtype}
% The |microtype| option instructs the package to use the
% microtypographic extensions of modern \TeX\ engines, such as
% \hologo{pdfTeX}, \Hologo{LuaTeX}, and (with only partial support)
% \Hologo{XeLaTeX}. This option is enabled by default.
%    \begin{macrocode}
\newif\iffibeamer@microtype
\DeclareOptionBeamer{microtype}{\fibeamer@microtypetrue}
\ExecuteOptionsBeamer{microtype}
%    \end{macrocode}
% \end{macro}
% \subsection{The \texttt{nomicrotype} option}
% The |nomicrotype| option disables the microtypographic
% extensions. This may be necessary, if an older \TeX\ engine,
% such as \hologo{TeX} or \hologo{eTeX}, is being used.
%    \begin{macrocode}
\DeclareOptionBeamer{nomicrotype}{\fibeamer@microtypefalse}
%    \end{macrocode}
% \begin{macro}{\fibeamer@university}
% \subsection{The \texttt{university} option}
% The \marg{\texttt{university}=identifier} option pair sets the
% identifier of the university, at which the presentation is being
% written, to \textit{identifier}. The \textit{identifier} is
% stored within the |\fibeamer@university| macro, whose
% implicit value is \texttt{mu}. This value corresponds to the
% Masaryk University in Brno.
%    \begin{macrocode}
\DeclareOptionBeamer{university}{\def\fibeamer@university{#1}}
\ExecuteOptionsBeamer{university=mu}
%    \end{macrocode}
% \end{macro}
% \begin{macro}{\fibeamer@faculty}
% \subsection{The \texttt{faculty} option}
% The \marg{\texttt{faculty}=identifier} pair sets the faculty, at
% which the thesis is being written, to \textit{domain}. The
% following faculty \textit{identifier}s are recognized at the
% Masaryk University in Brno:
% \begin{center}\begin{tabularx}{\textwidth}{Xc}\toprule
%   The faculty & The \textit{domain} name \\\midrule
%   The Faculty of Informatics & \texttt{fi} \\
%   The Faculty of Science & \texttt{sci} \\
%   The Faculty of Law & \texttt{law} \\
%   The Faculty of Economics and Administration & \texttt{econ} \\
%   The Faculty of Social Studies & \texttt{fss} \\
%   The Faculty of Medicine & \texttt{med} \\
%   The Faculty of Education & \texttt{ped} \\
%   The Faculty of Arts & \texttt{phil} \\
%   The Faculty of Sports Studies & \texttt{fsps} \\\bottomrule
% \end{tabularx}\end{center}
% The \textit{identifier} is stored within the |\fibeamer@faculty|
% macro, whose implicit value is \texttt{fi}.
%    \begin{macrocode}
\DeclareOptionBeamer{faculty}{\def\fibeamer@faculty{#1}}
\ExecuteOptionsBeamer{faculty=fi}
%    \end{macrocode}
% \end{macro}
% \begin{macro}{\fibeamer@locale}
% \subsection{The \texttt{locale} option}
% \changes{v1.1.0:1}{2015/11/24}{Added the \texttt{locale} option
%   and the \cs{fibeamer@locale} macro. [VN]}
% The \marg{\texttt{locale}=name} pair sets the name of the main
% locale to \textit{name}. The \textit{name} is stored within the
% |\fibeamer@locale| macro, whose implicit value is the main
% language of either the \textsf{babel} or the \textsf{polyglossia}
% package, or \texttt{english}, when undefined.
%
% With regards to the themes of the Masaryk University in Brno, the
% |\fibeamer@locale| macro is not used; it only provides the
% default value to the |\fibeamer@logoLocale| macro.
%    \begin{macrocode}
\def\fibeamer@locale{%
  % Babel / polyglossia detection
  \ifx\languagename\undefined%
  english\else\languagename\fi}
\DeclareOptionBeamer{locale}{%
  \def\fibeamer@locale{#1}}
%    \end{macrocode}
% \end{macro}
% \begin{macro}{\fibeamer@logoLocale}
% \subsection{The \texttt{logoLocale} option}
% The \marg{\texttt{logoLocale}=name} pair sets the logo file
% locale to \textit{name}. The \textit{name} is stored within the
% |\fibeamer@logoLocale| macro, whose implicit value is
% |\fibeamer@locale|.
% \changes{v1.1.0:2}{2016/01/11}{Added the \texttt{logoLocale}
%   option and the \cs{fibeamer@logoLocale} macro. [VN]}
% \begin{macrocode}
\def\fibeamer@logoLocale{\fibeamer@locale}
\DeclareOptionBeamer{logoLocale}{%
  \def\fibeamer@logoLocale{#1}}
%    \end{macrocode}
% \end{macro}
% \begin{macro}{\fibeamer@basePath}
% \subsection{The \texttt{basePath} option}
% The \marg{\texttt{basePath}=path} pair sets the \textit{path}
% containing the package files. The \textit{path} is prepended to
% every other path (|\fibeamer@logopath| and |\fibeamer@themePath|)
% used by the package. If non-empty, the
% \textit{path} gets normalized to \textit{path/}. The normalized
% \textit{path} is stored within the |\fibeamer@basePath| macro,
% whose implicit value is |fibeamer/|.
%    \begin{macrocode}
\DeclareOptionBeamer{basePath}{%
  \ifx\fibeamer@empty#1\fibeamer@empty%
    \def\fibeamer@basePath{}%
  \else%
    \def\fibeamer@basePath{#1/}%
  \fi}
\ExecuteOptionsBeamer{basePath=fibeamer}
%    \end{macrocode}
% \end{macro}
% \begin{macro}{\fibeamer@subdir}
% The |\fibeamer@subdir| macro returns |/| unchanged, coerces
% |.|, |..|, |/|\textit{path}, |./|\textit{path} and
% |../|\textit{path} to |./|, |../|, |/|\textit{path}|/|,
% |./|\textit{path}|/| and |../|\textit{path}|/|, respectively, and
% prefixes any other \textit{path} with |\fibeamer@basePath|. This
% macro is used within the definition of the \texttt{themePath} and
% \texttt{logoPath} options.
%    \begin{macrocode}
\def\fibeamer@subdir#1#2#3#4\empty{%
  \ifx#1\empty%           <empty> -> <basePath>
    \fibeamer@basePath
  \else
    \if#1/%
      \ifx#2\empty%             / -> /
        /%
      \else%              /<path> -> /<path>/
        #1#2#3#4/%
      \fi
    \else
      \if#1.%
        \ifx#2\empty%           . -> ./
          ./%
        \else
          \if#2.%
            \ifx#3\empty%      .. -> ../
              ../%
            \else
              \if#3/%   ../<path> -> ../<path>/
                ../#4/%
              \else
                \fibeamer@basePath#1#2#3#4/%
              \fi
            \fi
          \else
            \if#2/%      ./<path> -> ./<path>/
              ./#3#4/%
            \else
              \fibeamer@basePath#1#2#3#4/%
            \fi
          \fi
        \fi
      \else
        \fibeamer@basePath#1#2#3#4/%
      \fi
    \fi
  \fi}
%    \end{macrocode}
% \end{macro}
% \begin{macro}{\fibeamer@themePath}
% \subsection{The \texttt{themePath} option}
% The \marg{\texttt{themePath}=path} pair sets the \textit{path}
% containing the theme files. The \textit{path} is normalized using
% the |\fibeamer@subdir| macro and stored within the
% |\fibeamer@stylePath| macro, whose implicit value is
% |\fibeamer@basePath theme/|. By default, this expands to
% \texttt{fibeamer/theme/}.
%    \begin{macrocode}
\DeclareOptionBeamer{themePath}{%
  \def\fibeamer@themePath{\fibeamer@subdir#1%
    \empty\empty\empty\empty}}
\ExecuteOptionsBeamer{themePath=theme}
%    \end{macrocode}
% \end{macro}
% \begin{macro}{\fibeamer@logoPath}
% \subsection{The \texttt{logoPath} option}
% The \marg{\texttt{logoPath}=path} pair sets the \textit{path}
% containing the logo files, which is used by the outer themes to
% load the main logo. The \textit{path} is normalized using the
% |\fibeamer@subdir| macro and stored within the
% |\fibeamer@logoPath| macro, whose implicit value is
% |\fibeamer@basePath| followed by |logo/\fibeamer@university|. By
% default, this expands to \texttt{fibeamer/logo/mu/}.
%    \begin{macrocode}
\DeclareOptionBeamer{logoPath}{%
  \def\fibeamer@logoPath{\fibeamer@subdir#1%
    \empty\empty\empty\empty}}
\ExecuteOptionsBeamer{logoPath=logo/\fibeamer@university}
%    \end{macrocode}
% \end{macro}
% \begin{macro}{\fibeamer@logo}
% \subsection{The \texttt{logo} option}
% The \marg{\texttt{logo}=filename} pair sets the prefix of the
% filename of the main logo to \textit{filename}. The
% \textit{filename} is stored within the |\fibeamer@logo| macro,
% whose implicit value is
% \changes{v1.1.0:1}{2015/11/24}{Changed the implicit value of the
%   \cs{fibeamer@logo} macro. [VN]}
% |\fibeamer@logoPath| followed by
% \texttt{fibeamer-}|\fibeamer@university|\texttt{^^A
% \discretionary{-}{-}{-}}|\fibeamer@faculty|\texttt{-}^^A
% |\fibeamer@logoLocale|. By default, this expands to
% \texttt{fibeamer\discretionary{/}{/}{/}logo/fibeamer-mu-fi^^A
% -english}.
%    \begin{macrocode}
\DeclareOptionBeamer{logo}{\def\fibeamer@logo{#1}}
\ExecuteOptionsBeamer{%
  logo=\fibeamer@logoPath fibeamer-\fibeamer@university-%
    \fibeamer@faculty-\fibeamer@logoLocale}
%    \end{macrocode}
% \end{macro}
% \begin{macro}{\fibeamer@fallbackLogo}
% \subsection{The \texttt{fallbackLogo} option}
% \changes{v1.1.0:3}{2016/01/11}{Added the \cs{fibeamer@fallbackLogo}
%   macro. [VN]}
% \changes{v1.1.1:1}{2016/01/14}{Added the \texttt{fallbackLogo}
%   option. [VN]}
% The |\fibeamer@fallbackLogo| macro contains the filename of the
% logo file to be used, if |\fibeamer@logo| does not exist. The
% implicit value of the |\fibeamer@fall|\discretionary{-}{}{}^^A
% |backLogo| macro is |\fibeamer@logoPath| followed by
% \texttt{fibeamer-}|\fibeamer@uni|\texttt{\discretionary{-}{}{}}^^A
% |versity\fibeamer@faculty|\texttt{-english}. By default, this
% expands to \texttt{fibeamer/logo\discretionary{/}{/}{/}fi\-bea^^A
% mer-mu-fi-english}.
%    \begin{macrocode}
\DeclareOptionBeamer{fallbackLogo}{\def\fibeamer@fallbackLogo{#1}}
\ExecuteOptionsBeamer{%
  fallbackLogo=\fibeamer@logoPath fibeamer-\fibeamer@university-%
    \fibeamer@faculty-english}
%    \end{macrocode}
% \end{macro}
% \section{Logic}
% \subsection{Macros}
% \begin{macro}{\fibeamer@require}
% The |\fibeamer@require|\marg{package} macro is used to gracefully
% load a \textit{package}. Packages that have already been loaded
% or do not exist are ignored by the macro.
%    \begin{macrocode}
\def\fibeamer@require#1{\IfFileExists{#1.sty}{%
  \@ifpackageloaded{#1}{}{\RequirePackage{#1}}}{}}
%    \end{macrocode}
% \end{macro}
% \begin{macro}{\fibeamer@requireTheme}
% The |\fibeamer@requireTheme|\marg{class} macro is used to load a
% \textit{class} of beamer themes. The following packages are
% loaded, provided they exist:
% \begin{itemize}
%  \item\texttt{beamer}\textit{class}\texttt{themefibeamer} from
%    the |\fibeamer@themePath| directory -- The base theme file. By
%    default, this expands to \texttt{fibeamer/theme/beamer}^^A
%    \textit{class}\texttt{themefibeamer}.
%  \item\texttt{beamer}\textit{class}\texttt{themefibeamer-}^^A
%    |\fibeamer@university| from the |\fibeamer@themePath|^^A
%    \discretionary{}{}{}|\fibeamer@university/| directory -- The
%    university theme file. By default, this expands to
%    \texttt{fibeamer/theme/mu/beamer}\textit{class}^^A
%    \texttt{themefibeamer-mu}.
%  \item\texttt{beamer}\textit{class}\texttt{themefibeamer-}^^A
%    |\fibeamer@university|\texttt{-}|\fibeamer@faculty| from the
%    |\fibeamer@themePath\fibeamer@university/| directory -- The
%    faculty theme file. By default, this expands to \texttt{^^A
%    fibeamer/theme/mu/beamer}\textit{class}\texttt{themefibeam^^A
%    er-mu-fi}.
% \end{itemize}
%    \begin{macrocode}
\def\fibeamer@requireTheme#1{%
  \fibeamer@require{\fibeamer@themePath beamer#1themefibeamer}
  \fibeamer@require{\fibeamer@themePath\fibeamer@university%
    /beamer#1themefibeamer-\fibeamer@university}
  \fibeamer@require{\fibeamer@themePath\fibeamer@university%
    /beamer#1themefibeamer-\fibeamer@university-\fibeamer@faculty}}
%    \end{macrocode}
% \end{macro}
% \begin{macro}{\fibeamer@includeLogo}
% \changes{v1.1.0:5}{2016/01/11}{Added the
%   \cs{fibeamer@includeLogo} macro. [VN]}
% The |\fibeamer@includeLogo|\oarg{options} macro includes the
% main logo into the document. If specified, the \textit{options}
% are passed to the |\includegraphics| macro.  The
% |\fibeamer@includeLogo| macro requires the \textsf{etoolbox}
% package to function.
%    \begin{macrocode}
\fibeamer@require{etoolbox}
\newcommand\fibeamer@includeLogo[1][]{{
  % See <http://tex.stackexchange.com/a/39987/70941>.
  \patchcmd{\Gin@ii}% Make `\includegraphics` use `@fallbackLogo`.
    {\begingroup}% <search>
    {\begingroup\renewcommand{\@latex@error}[2]{%
       \includegraphics[#1]\fibeamer@fallbackLogo}}% <replace>
    {}% <success>
    {}% <failure>
  \includegraphics[#1]\fibeamer@logo}}
%    \end{macrocode}
% \end{macro}
% \begin{macro}{\fibeamer@patch}
% \changes{v1.1.2}{2016/02/19}{Added \cs{fibeamer@patch}.}
% The |\fibeamer@patch|\oarg{versions}\oarg{patch} macro expands
% \textit{patch}, if |\fibeamer|\texttt{\discretionary{@}{@}{@}}^^A
% |version@number| (defined at the top of the file
% \texttt{beamerthemefibeamer.sty}) matches any of the
% comma-delimited \textit{versions}. This macro enables the simple
% deployment of version-targeted patches.
%    \begin{macrocode}
\def\fibeamer@patch#1#2{%
  \def\fibeamer@patch@versions{#1}%
  \def\fibeamer@patch@action{#2}%
  \def\fibeamer@patch@next##1,{%
    \def\fibeamer@patch@arg{##1}%
    \def\fibeamer@patch@relax{\relax}%
    \ifx\fibeamer@patch@arg\fibeamer@version@number
      \def\fibeamer@patch@next####1\relax,{}%
      \expandafter\fibeamer@patch@action
      \expandafter\fibeamer@patch@next
    \else\ifx\fibeamer@patch@arg\fibeamer@patch@relax\else
      \expandafter\expandafter\expandafter\fibeamer@patch@next
    \fi\fi}%
  \expandafter\expandafter\expandafter\fibeamer@patch@next
  \expandafter\fibeamer@patch@versions\expandafter,\relax,}
%    \end{macrocode}
% \end{macro}
% \subsection{Main routine}
% First, the theme processes the options.
%    \begin{macrocode}
\ProcessOptionsBeamer
%    \end{macrocode}
% When the |microtype| option is specified, the \textsf{microtype}
% package gets loaded.
%    \begin{macrocode}
  % Set up the microtypographic extensions
  \iffibeamer@microtype
    \RequirePackage{microtype}
  \fi
%    \end{macrocode}
% When the |fonts| option is specified, the following packages will
% be used by the package to configure the fonts in the presentation
% mode:
% \begin{itemize}
%   \item\textsf{ifthen} -- This package is used to construct
%     compound conditionals.
%   \item\textsf{ifxetex}, \textsf{ifluatex} -- These packages are
%     used to detect the used \TeX\ engine.
%   \item\textsf{lmodern} -- The Latin Modern font family is used as a
%     fallback for missing glyphs.
%   \item\textsf{carlito} -- The Carlito font family is used as the
%     primary text and math font face.
%   \item\textsf{arevmath} -- The Arev Sans math font family is used for
%     various symbols and greek alphabet.
%   \item\textsf{iwona} -- The Iwona font family is used for large
%     math symbols.
%   \item\textsf{dejavu} -- The DejaVu Sans Mono font family is
%     used for the typesetting of monospaced text.
%   \item\textsf{setspace} -- This package is used to adjust the
%     leading to 115 \%. Loading this package breaks any top-level
%     footnotes without the \texttt{[frame]} optional parameter
%     specified. This is a known bug.\footnote{See
%       \url{http://tex.stackexchange.com/a/118005/70941}.}
%   \item\textsf{fontenc} -- This package is used to set the font
%     encoding to Cork. This package is only used outside the
%     \Hologo{XeTeX} and \Hologo{LuaTeX} engines.
%   \item\textsf{fontspec} -- This package is used to load fonts.
%     This package is only used with the \Hologo{XeTeX} and
%     \Hologo{LuaTeX} engines.
% \end{itemize}
%    \begin{macrocode}
\mode<presentation>
  % Set up the fonts
  \iffibeamer@fonts
    \RequirePackage{ifthen}
    \RequirePackage{ifxetex}
    \RequirePackage{ifluatex}
    \RequirePackage{lmodern}
    \RequirePackage[sfdefault,lf]{carlito}
    \renewcommand*\oldstylenums[1]{{\carlitoOsF #1}}

    %% Load arev with scaling factor of .85
    %% See <http://tex.stackexchange.com/a/181240/70941>
    \DeclareFontFamily{OML}{zavm}{\skewchar\font=127 }
    \DeclareFontShape{OML}{zavm}{m}{it}{<-> s*[.85] zavmri7m}{}
    \DeclareFontShape{OML}{zavm}{b}{it}{<-> s*[.85] zavmbi7m}{}
    \DeclareFontShape{OML}{zavm}{m}{sl}{<->ssub * zavm/m/it}{}
    \DeclareFontShape{OML}{zavm}{bx}{it}{<->ssub * zavm/b/it}{}
    \DeclareFontShape{OML}{zavm}{b}{sl}{<->ssub * zavm/b/it}{}
    \DeclareFontShape{OML}{zavm}{bx}{sl}{<->ssub * zavm/b/sl}{}

    \AtBeginDocument{
      \SetSymbolFont{operators}   {normal}{OT1}{zavm}{m}{n}
      \SetSymbolFont{letters}     {normal}{OML}{zavm}{m}{it}
      \SetSymbolFont{symbols}     {normal}{OMS}{zavm}{m}{n}
      \SetSymbolFont{largesymbols}{normal}{OMX}{iwona}{m}{n}}
    \RequirePackage[sans]{dsfont}

    \ifthenelse{\boolean{xetex}\OR\boolean{luatex}}{
      \RequirePackage{fontspec}
%    \end{macrocode}
% \changes{v1.1.7}{2017/04/28}{Removed undesirable TeX ligatures from the mono
%   font. [VN]}
%    \begin{macrocode}
      \setmonofont[Scale=0.85]{DejaVu Sans Mono}
    }{
      \RequirePackage[scaled=0.85]{DejaVuSansMono}
      \RequirePackage[resetfonts]{cmap}
      \RequirePackage[T1]{fontenc}
    }
    \RequirePackage{setspace}
    \setstretch{1.15}
  \fi
\mode
  <all>
%    \end{macrocode}
% Finally, the color, font, inner and outer themes of the
% respective university and faculty will be loaded.
%    \begin{macrocode}
\fibeamer@requireTheme{color}
\fibeamer@requireTheme{font}
\fibeamer@requireTheme{inner}
\fibeamer@requireTheme{outer}
%    \end{macrocode}
%
% \section{Themes}
% This section contains the combined documentation of all available
% themes.  When creating a new theme file, it is advisable to
% create one self-contained \texttt{dtx} file, which is then
% partitioned into locale files via the \textsf{docstrip} tool
% based on the respective \texttt{ins} file. A
% \DescribeMacro{\file} macro |\file|\marg{filename} is available
% for the sectioning of the documentation of various files within
% the \texttt{dtx} file. For more information about \texttt{dtx}
% files and the \textsf{docstrip} tool, consult the \textsf{dtxtut,
% docstrip, doc} and \textsf{ltxdoc} manuals.
%
% \def\file#1{\subsubsection{The \texttt{#1} file}}
%
% \subsection{Base files}
% % \iffalse
%<*color>
% \fi\file{theme/mu/beamercolorthemefibeamer-mu.sty}
% This is the base color theme for presentations written at the
% Masaryk University in Brno.
%    \begin{macrocode}
\NeedsTeXFormat{LaTeX2e}
\ProvidesPackage{fibeamer/theme/mu/%
  beamercolorthemefibeamer-mu}[2016/05/06]
%    \end{macrocode}
% \begin{macro}{\darkframes}
% The |darkframes| environment switches the color definitions to
% render the enclosed frames in dark colors. This is a dummy
% definition, which will be overridden by the subsequently loaded
% color theme in the presentation mode.
%    \begin{macrocode}
\newenvironment{darkframes}{}{}
%    \end{macrocode}
% \end{macro}
% The rest of the theme will be ignored outside the presentation
% mode.
%    \begin{macrocode}
\mode<presentation>
%    \end{macrocode}
% The theme loads the following packages, which will be used by the
% subsequently loaded color theme specific to a faculty:
% \begin{itemize}
%   \item\textsf{listings} -- This package is used for code
%     listings. The subsequently loaded color theme will specify
%     source code coloring for the package.
%   \item\textsf{ifthen} -- This package is used to construct
%     compound conditionals.
%   \item\textsf{tikz} -- This package is used to create gradient
%     background for dark slides.
% \end{itemize}
%    \begin{macrocode}
  \RequirePackage{listings}
  \RequirePackage{ifthen}
  \RequirePackage{tikz}
%    \end{macrocode}
% \begin{macro}{\iffibeamer@dark}
% The |\iffibeamer@dark| conditional will be switched on and off by
% the subsequently loaded color theme based on whether or not the
% given frame is being typeset in light or dark colors. This
% information will be used by outer themes to insert the correct
% logo into each frame.
%    \begin{macrocode}
  \newif\iffibeamer@dark\fibeamer@darkfalse
%    \end{macrocode}
% \end{macro}
% A frame that is either title or dark, as specified by the value
% of the |\iffibeamer@dark| conditional, will have a gradient
% background as specified by the |fibeamer@dark@backgroundInner|
% and |fibeamer@light@backgroundOuter| colors that shall be defined
% by the subsequently loaded color theme.
%
% A frame that is neither title nor dark, as specified by the value
% of the |\iffibeamer@dark| conditional, will have a gradient
% background as specified by the |fibeamer@light@backgroundInner|
% and |fibeamer@light@backgroundOuter| colors that shall be defined
% by the subsequently loaded color theme.
%    \begin{macrocode}
  \defbeamertemplate*{background canvas}{fibeamer}{%
    \ifthenelse{%
      \boolean{fibeamer@dark} \OR \c@framenumber=0
    }{%
      \begin{tikzpicture}
        \clip (0,\fibeamer@lengths@clipbottom) rectangle
          (\paperwidth,\fibeamer@lengths@cliptop);
        \path [inner color = fibeamer@dark@backgroundInner,
               outer color = fibeamer@dark@backgroundOuter]
          (0,0) rectangle (\paperwidth,\paperwidth);
      \end{tikzpicture}
    }{%
      \begin{tikzpicture}
        \clip (0,\fibeamer@lengths@clipbottom) rectangle
          (\paperwidth,\fibeamer@lengths@cliptop);
        \path [inner color = fibeamer@light@backgroundInner,
               outer color = fibeamer@light@backgroundOuter]
          (0,0) rectangle (\paperwidth,\paperwidth);
      \end{tikzpicture}
    }}
%    \end{macrocode}
% The |\qed| symbol inserted at the end of proofs will have the
% same color as the rest of the proof.
% \changes{v1.1.1:2}{2016/01/14}{Added proper coloring of
%   \cs{qed} to the themes of the Masaryk University in Brno. [VN]}
%    \begin{macrocode}
  \setbeamercolor{qed symbol}{%
    use=block body,
    fg=block body.fg,
    bg=block body.bg}
%    \end{macrocode}
% The links can be colored by the subsequently loaded color themes.
% \changes{v1.1.4:3}{2016/05/06}{Added proper link coloring for the
%   color themes of the Masaryk University in Brno. [VN]}
%    \begin{macrocode}
  \hypersetup{colorlinks,linkcolor=}
\mode
<all>
%    \end{macrocode}
% \iffalse
%</color>
%<*font>
% \fi\file{theme/mu/beamerfontthemefibeamer-mu.sty}
% This is the base font theme for presentations written at the
% Masaryk University in Brno. The theme has no effect outside the
% presentation mode.
%    \begin{macrocode}
\NeedsTeXFormat{LaTeX2e}
\ProvidesPackage{fibeamer/theme/mu/%
  beamerfontthemefibeamer-mu}[2016/01/12]
\mode<presentation>
  \setbeamerfont{normal text}{size=\normalsize}
  \setbeamerfont{title}{size=\LARGE, series=\bfseries}
  \setbeamerfont{subtitle}{parent=normal text, size=\Large}
  \setbeamerfont{frametitle}{size=\Large}
  \setbeamerfont{framesubtitle}{size=\large, shape=\itshape}
  \setbeamerfont{description item}{series=\bfseries}
  \setbeamerfont{author}{size=\large}
\mode
<all>
%    \end{macrocode}
% \iffalse
%</font>
%<*inner>
% \fi\file{theme/mu/beamerinnerthemefibeamer-mu.sty}
% This is the base inner theme for presentations written at the
% Masaryk University in Brno. The theme has no effect outside the
% presentation mode.
%    \begin{macrocode}
\NeedsTeXFormat{LaTeX2e}
\ProvidesPackage{fibeamer/theme/mu/%
  beamerinnerthemefibeamer-mu}[2016/01/14]
\mode<presentation>
%    \end{macrocode}
% This part of the inner theme defines the design of lists.
%    \begin{macrocode}
\defbeamertemplate*{itemize item}{fibeamer}{$\bullet$}
\defbeamertemplate*{itemize subitem}{fibeamer}{\---}
\defbeamertemplate*{itemize subsubitem}{fibeamer}{\guillemotright}
%    \end{macrocode}
% This part of the inner theme defines the design of bibliography
% items and citations.^^A
% \changes{v1.1.0:8}{2016/01/12}{Added support for colored
%   citations to the themes of the Masaryk University in Brno.
%   [VN]}
%    \begin{macrocode}
\defbeamertemplate*{bibliography item}{fibeamer}{\insertbiblabel}
\AtBeginDocument{%
  \let\fibeamer@oldcite\cite
  \def\cite#1{{%
    \usebeamercolor[fg]{item}%
    \fibeamer@oldcite{#1}}}}
%    \end{macrocode}
% This part of the inner theme defines the design of the table of
% contents.
% \changes{v1.1.0:6}{2016/01/11}{Added support for the
%   \cs{tableofcontents} to the themes of the Masaryk University in
%   Brno. [VN]}
%    \begin{macrocode}
\defbeamertemplate*{section in toc}{fibeamer}{%
  \usebeamercolor[fg]{item}%
    \inserttocsectionnumber.%
  \usebeamercolor[fg]{structure}%
  \kern1.25ex\inserttocsection\par}
\defbeamertemplate*{subsection in toc}{fibeamer}{%
  \hspace\leftmargini
  \usebeamercolor[fg]{item}%
    \inserttocsectionnumber.\inserttocsubsectionnumber%
  \usebeamercolor[fg]{structure}%
  \kern1.25ex\inserttocsubsection\par}
\defbeamertemplate*{subsubsection in toc}{fibeamer}{%
  \hspace\leftmargini
  \hspace\leftmarginii
  \usebeamercolor[fg]{item}%
    \inserttocsectionnumber.\inserttocsubsectionnumber.%
    \inserttocsubsubsectionnumber%
  \usebeamercolor[fg]{structure}%
  \kern1.25ex\inserttocsubsubsection\par}
\mode
<all>
%    \end{macrocode}
% \iffalse
%</inner>
%<*outer>
% \fi\file{theme/mu/beamerouterthemefibeamer-mu.sty}
% This is the base outer theme for presentations written at the
% Masaryk University in Brno. The theme has no effect outside the
% presentation mode.
%    \begin{macrocode}
\NeedsTeXFormat{LaTeX2e}
\ProvidesPackage{fibeamer/theme/mu/%
  beamerouterthemefibeamer-mu}[2016/01/12]
\mode<presentation>
%    \end{macrocode}
% The theme uses the following packages:
% \begin{itemize}
%   \item\textsf{ifthen} -- This package is used to construct
%     compound conditionals.
%   \item\textsf{ifpdf} -- This package is used to check, whether
%     the document is being typeset in DVI mode. If it is, then
%     the |\pdfpagewidth| and |\pdfpageheight| dimensions are
%     defined, so that positioning in TikZ works correctly.
%     \changes{v1.0.1}{2015/10/03}{Added DVI output support. [VN]}
%     ^^A <http://tex.stackexchange.com/a/246631/70941>
%   \item\textsf{tikz} -- This package is used to position the
%     logo and the frame number on a frame.
%   \item\textsf{pgfcore} -- This package is used to draw the
%     dashed line at the title frame.
% \end{itemize}
%    \begin{macrocode}
  \RequirePackage{ifthen}
  \RequirePackage{ifpdf}
  \ifpdf\else
    \@ifundefined{pdfpagewidth}{\newdimen\pdfpagewidth}{}
    \@ifundefined{pdfpageheight}{\newdimen\pdfpageheight}{}
    \pdfpagewidth=\paperwidth
    \pdfpageheight=\paperheight
  \fi
  \RequirePackage{tikz}
  \RequirePackage{pgfcore}
%    \end{macrocode}
% This part of the outer theme defines the geometry of the frames
% along with other dimensions.
% \changes{v1.1.0:4}{2016/01/11}{Length definitions within the
%   themes of the Masaryk University in Brno are no longer based on
%   the dimensions of the (now unused) logo in the upper right
%   corner. [VN]}
%    \begin{macrocode}
  \newlength\fibeamer@lengths@baseunit
  \fibeamer@lengths@baseunit=3.75mm
  % The footer padding
  \newlength\fibeamer@lengths@footerpad
  \setlength\fibeamer@lengths@footerpad{%
    \fibeamer@lengths@baseunit}
  % The side margins
  \newlength\fibeamer@lengths@margin
  \setlength\fibeamer@lengths@margin{%
    3\fibeamer@lengths@baseunit}
  \setbeamersize{
    text margin left=\fibeamer@lengths@margin,
    text margin right=\fibeamer@lengths@margin}
  % The upper margin
  \newlength\fibeamer@lengths@titleline
  \setlength\fibeamer@lengths@titleline{%
    3\fibeamer@lengths@baseunit}
  % The background clipping
  \newlength\fibeamer@lengths@clipbottom
  \setlength\fibeamer@lengths@clipbottom\paperwidth
  \addtolength\fibeamer@lengths@clipbottom{-\paperheight}
  \setlength\fibeamer@lengths@clipbottom{%
    0.5\fibeamer@lengths@clipbottom}
  \newlength\fibeamer@lengths@cliptop
  \setlength\fibeamer@lengths@cliptop\paperwidth
  \addtolength\fibeamer@lengths@cliptop{%
    -\fibeamer@lengths@clipbottom}
%    \end{macrocode}
% \changes{v1.1.0:6}{2016/01/11}{Added the logo dimension
% definitions to the themes of the Masaryk University in Brno.
% [VN]}
%    \begin{macrocode}
  % The logo size
  \newlength\fibeamer@lengths@logowidth
  \setlength\fibeamer@lengths@logowidth{%
    14\fibeamer@lengths@baseunit}
  \newlength\fibeamer@lengths@logoheight
  \setlength\fibeamer@lengths@logoheight{%
    0.4\fibeamer@lengths@logowidth}
%    \end{macrocode}
% The outer theme completely culls the bottom navigation.
%    \begin{macrocode}
  \defbeamertemplate*{navigation symbols}{fibeamer}{}
%    \end{macrocode}
% The outer theme also culls the headline.
% \changes{v1.1.0:1}{2015/11/24}{The faculty logos are no longer
%   displayed on regular slides, as per the new unified design of
%   the Masaryk University in Brno. [VN]}
%    \begin{macrocode}
  \defbeamertemplate*{headline}{fibeamer}{}
%    \end{macrocode}
% The frame title.
%    \begin{macrocode}
  \defbeamertemplate*{frametitle}{fibeamer}{%
    \vskip-1em % Align the text with the top border
    \vskip\fibeamer@lengths@titleline
    \usebeamercolor[fg]{frametitle}%
    \usebeamerfont{frametitle}%
      \insertframetitle\par%
    \usebeamercolor[fg]{framesubtitle}%
    \usebeamerfont{framesubtitle}%
      \insertframesubtitle}
%    \end{macrocode}
% The footline contains the frame number. It is flushed right.
%    \begin{macrocode}
  \defbeamertemplate*{footline}{fibeamer}{%
    \ifnum\c@framenumber=0\else%
      \begin{tikzpicture}[overlay]
        \node[anchor=south east,
          yshift=\fibeamer@lengths@footerpad,
          xshift=-\fibeamer@lengths@footerpad] at
          (current page.south east) {
            \usebeamercolor[fg]{framenumber}%
            \usebeamerfont{framenumber}%
            \insertframenumber/\inserttotalframenumber};
      \end{tikzpicture}
    \fi}
%    \end{macrocode}
% The title frame contains the main logo, the |\title|, the
% |\subtitle|, and the |\author|.
%    \begin{macrocode}
  \defbeamertemplate*{title page}{fibeamer}{%
%    \end{macrocode}
% \changes{v1.1.6}{2016/09/27}{The title frame now properly loads the
%   color theme definitions for dark frames. As a result, the
%   \textsf{hyperref} \textsc{url} color for light frames is no
%   longer used in the title frame. [VN]}
%    \begin{macrocode}
    \begin{darkframes}

    % This is slide 0
    \setcounter{framenumber}{0}

    % Input the university logo
    \begin{tikzpicture}[
      remember picture,
      overlay,
      xshift=0.5\fibeamer@lengths@logowidth,
      yshift=0.5\fibeamer@lengths@logoheight
    ]
      \node at (0,0) {
        \fibeamer@includeLogo[
          width=\fibeamer@lengths@logowidth,
          height=\fibeamer@lengths@logoheight
        ]};
    \end{tikzpicture}

    % Input the title
    \usebeamerfont{title}%
    \usebeamercolor[fg]{title}%
    \begin{minipage}[b][2\baselineskip][b]{\textwidth}%
      \raggedright\inserttitle
    \end{minipage}
    \vskip-.5\baselineskip

    % Input the dashed line
    \begin{pgfpicture}
      \pgfsetlinewidth{2pt}
      \pgfsetroundcap
      \pgfsetdash{{0pt}{4pt}}{0cm}

      \pgfpathmoveto{\pgfpoint{0mm}{0mm}}
      \pgfpathlineto{\pgfpoint{\textwidth}{0mm}}

      \pgfusepath{stroke}
    \end{pgfpicture}
    \vfill
%    \end{macrocode}
% \changes{v1.1.4:4}{2016/05/06}{Added support for subtitle and
%   author name coloring within the color themes of the Masaryk
%   University in Brno. [VN]}
%    \begin{macrocode}
    % Input the subtitle
    \usebeamerfont{subtitle}%
    \usebeamercolor[fg]{subtitle}%
    \begin{minipage}{\textwidth}
      \raggedright%
      \insertsubtitle%
    \end{minipage}\vskip.25\baselineskip

    % Input the author's name
    \usebeamerfont{author}%
    \usebeamercolor[fg]{author}%
    \begin{minipage}{\textwidth}
      \raggedright%
      \insertauthor%
    \end{minipage}
    \end{darkframes}}

\mode
<all>
%    \end{macrocode}
% \iffalse
%</outer>
% \fi

% \subsection{The Faculty of Informatics}
% % \file{style/mu/fithesis-fi.sty}
% This is the style file for the theses written at the Faculty of
% Informatics at the Masaryk University in Brno. It has been
% prepared in accordance with the formal requirements published at
% the website of the faculty\footnote{See
% \url{http://www.fi.muni.cz/docs/BP_DP_na_FI.pdf}}.
%    \begin{macrocode}
\NeedsTeXFormat{LaTeX2e}
\ProvidesPackage{fithesis/style/mu/fithesis-fi}[2016/04/18]
%    \end{macrocode}
% The file defines the color scheme of the respective faculty. Note
% the the color definitions are in RGB, which makes the resulting
% files generally unsuitable for printing.
%    \begin{macrocode}
\thesis@color@setup{
  links={HTML}{FFD451},
  tableEmph={HTML}{FFD451},
  tableOdd={HTML}{FFF9E5},
  tableEven={HTML}{FFECB3}}
%    \end{macrocode}
% The bibliography support is enabled. The |numeric| citations are
% used and the bibliography is sorted in citation order.
%    \begin{macrocode}
\thesis@bibliography@setup{
  style=iso-numeric,
  sorting=none}
\thesis@bibliography@load
%    \end{macrocode}
% In case of rigorous and doctoral theses, the style file hides the
% thesis assignment in accordance with the formal requirements of
% the faculty.
% \begin{macrocode}
\ifx\thesis@type\thesis@rigorous
  \thesis@blocks@assignment@false
\else\ifx\thesis@type\thesis@doctoral
  \thesis@blocks@assignment@false
\fi\fi
%    \end{macrocode}
% \begin{macro}{\thesis@blocks@titlePage}
% The style file redefines the cover and title page footers to
% include the thesis advisor's name and signature in case of a
% rigorous thesis. Along with the macros required by the
% locale file interface, the locale files need to define the
% following strings:
% \begin{itemize}
%   \item\texttt{advisorSignature} -- The label of the advisor
%     signature field typeset in the case of rigorous theses
% \end{itemize}
% \begin{macrocode}
\def\thesis@blocks@advisor{%
  {\thesis@titlePage@large\\[0.3in]
    {\bf\thesis@@{advisorTitle}:} \thesis@advisor}}
\def\thesis@blocks@titlePage@content{%
    {\thesis@titlePage@Huge\bf\thesis@TeXtitle\par\vfil}\vskip 0.8in
    {\thesis@titlePage@large\sc\thesis@@{typeName}\\[0.3in]}
    {\thesis@titlePage@Large\bf\thesis@author}
    % If this is a rigorous thesis or a PhD thesis proposal,
    % typeset the name of the thesis advisor.
    \ifx\thesis@type\thesis@rigorous
      \thesis@blocks@advisor
    \else\ifx\thesis@type\thesis@proposal
      \thesis@blocks@advisor
    \fi\fi}%
\def\thesis@blocks@advisorSignature{%
  \let\@A\relax\newlength{\@A}
    \settowidth{\@A}{\thesis@@{advisorSignature}}
    \setlength{\@A}{\@A+1cm}
  \hfill\raisebox{-0.5em}{\parbox{\@A}{
    \centering
    \rule{\@A}{1pt}\\
    \thesis@@{advisorSignature}
  }}}%
\def\thesis@blocks@titlePage@footer{%
  {\thesis@titlePage@large\thesis@place, \thesis@@{semester}
  % If this is a rigorous thesis or a PhD thesis proposal,
  % create space for the advisor's signature.
  \ifx\thesis@type\thesis@rigorous
    \thesis@blocks@advisorSignature
  \else\ifx\thesis@type\thesis@proposal
    \thesis@blocks@advisorSignature
  \fi\fi}}
%    \end{macrocode}
% \end{macro}\begin{macro}{\thesis@blocks@declaration}
% The |\thesis@blocks@declaration| macro typesets the
% declaration text. Compared to the definition within the
% \texttt{style/mu/base.sty} file, this macro also
% typesets the advisor's name at the bottom of the page.
% \begin{macrocode}
\def\thesis@blocks@declaration{%
  \thesis@blocks@clear
  \begin{alwayssingle}%
    \chapter*{\thesis@@{declarationTitle}}%
    \thesis@declaration
    \vskip 2cm%
    \hfill\thesis@author
    \par\vfill\noindent
    \textbf{\thesis@@{advisorTitle}:} \thesis@advisor
    \par\vfil
  \end{alwayssingle}}
%    \end{macrocode}
% \end{macro}
% In Ph.D. theses, only the table of contents will be typeset in
% the front matter as per the formal requirements of the
% faculty\footnote{See
% \url{http://is.muni.cz/www/2575/dtedi/index_en.html}}.
%
% Note that there is no direct support for the seminar paper type.
% If you would like to change the contents of the preamble and the
% postamble, you should modify the |\thesis@blocks@preamble| and
% |\thesis@blocks@postamble| macros.
%
% All blocks within the autolayout preamble and postamble that are
% not defined within this file are defined in the
% \texttt{style/mu/fithesis-base.sty} file.
%    \begin{macrocode}
\def\thesis@blocks@preamble{%
  \thesis@blocks@coverMatter
    \thesis@blocks@cover
    \thesis@blocks@titlePage
  \thesis@blocks@frontMatter
    \ifx\thesis@type\thesis@proposal
      \thesis@blocks@toc
    \else
      \thesis@blocks@assignment
      \thesis@blocks@declaration
      \thesis@blocks@thanks
      \thesis@blocks@clearRight
        \thesis@blocks@abstract
        \thesis@blocks@keywords
      \thesis@blocks@tables
    \fi}
\def\thesis@blocks@postamble{%
  \thesis@blocks@bibliography}
%    \end{macrocode}

% \subsection{The Faculty of Science}
% % \file{style/mu/fithesis-sci.sty}
% This is the style file for the theses written at the Faculty of
% Science at the Masaryk University in Brno. It has been
% prepared in accordance with the formal requirements published at
% the website of the faculty\footnote{See
% \url{http://www.sci.muni.cz/NW/predpisy/od/OD-2014-05.pdf}}.
%    \begin{macrocode}
\NeedsTeXFormat{LaTeX2e}
\ProvidesPackage{fithesis/style/mu/fithesis-sci}[2017/06/02]
%    \end{macrocode}
% The file defines the color scheme of the respective faculty. Note
% the the color definitions are in RGB, which makes the resulting
% files generally unsuitable for printing.
%    \begin{macrocode}
\thesis@color@setup{
  links={HTML}{20E366},
  tableEmph={HTML}{8EDEAA},
  tableOdd={HTML}{EDF7F1},
  tableEven={HTML}{CCEDD8}}
%    \end{macrocode}
% The bibliography support is enabled. The |numeric| citations are
% used and the bibliography is sorted in citation order.
%    \begin{macrocode}
\thesis@bibliography@setup{
  style=iso-numeric,
  sorting=none}
\thesis@bibliography@load
%    \end{macrocode}
% The file uses Czech locale strings within the macros.
%    \begin{macrocode}
\thesis@requireLocale{czech}
%    \end{macrocode}
% \begin{macro}{\ifthesis@czech}
% The |\ifthesis@czech| \ldots |\else| \ldots |\fi| conditional is made
% available for testing, whether or not the current locale is Czech.
% \changes{v0.3.45}{2017/05/23}{Defined the
%   \cs{ifthesis@czech} macro in
%   \texttt{style/mu/fithesis-sci.sty}. The patch was submitted by
%   Juraj Pálenik. [VN]}
%    \begin{macrocode}
\def\ifthesis@czech{
  \expandafter\def\expandafter\@czech\expandafter{\string
  \czech}%
  \expandafter\expandafter\expandafter\def\expandafter
  \expandafter\expandafter\@locale\expandafter\expandafter
  \expandafter{\expandafter\string\csname\thesis@locale\endcsname}%
  \expandafter\csname\expandafter i\expandafter f\ifx\@locale
  \@czech
    true%
  \else
    false%
  \fi\endcsname}
\ifthesis@czech
  \expandafter\expandafter\expandafter\let\expandafter\expandafter
    \csname ifthesis@czech\endcsname\csname iftrue\endcsname
\else
  \expandafter\expandafter\expandafter\let\expandafter\expandafter
    \csname ifthesis@czech\endcsname\csname iffalse\endcsname
\fi
%    \end{macrocode}
% The file recognizes the following options: \begin{itemize}
%   \item\texttt{abstractonsinglepage} -- The abstracts are going
%   to be typeset on a single page as opposed to being spread
%   across several pages. This is the default for theses whose main
%   locale is neither Czech nor English.
% \end{itemize}
% \changes{v0.3.45}{2017/05/24}{Defined the
%   \texttt{abstractonsinglepage} option in
%   \texttt{style/mu/fithesis-sci.sty}. The patch was submitted by
%   Juraj Pálenik. [VN]}
%    \begin{macrocode}
\newif\ifthesis@abstractonsinglepage@
\DeclareOption{abstractonsinglepage}{\thesis@abstractonsinglepage@true}
\ifthesis@czech\else\ifthesis@english\else
  \ExecuteOptions{abstractonsinglepage}
\fi\fi
\ProcessOptions*
%    \end{macrocode}
% \end{macro}
% The file loads the following packages:
% \begin{itemize}
%   \item\textsf{tikz} -- Used for dimension arithmetic.
%   \item\textsf{changepage} -- Used for width adjustments.
% \end{itemize}
%    \begin{macrocode}
\thesis@require{tikz}
\thesis@require{changepage}
%    \end{macrocode}
% In case of rigorous and doctoral theses, the style file hides the
% thesis assignment in accordance with the formal requirements of
% the faculty.
% \begin{macrocode}
\ifx\thesis@type\thesis@bachelors\else
\ifx\thesis@type\thesis@masters\else
  \thesis@blocks@assignment@false
\fi\fi
%    \end{macrocode}
% Enable the inclusion of the scanned assignment inside the digital
% version of the document.
% \begin{macrocode}
\thesis@blocks@assignment@hideIfDigital@false
%    \end{macrocode}
% \begin{macro}{\thesis@blocks@bibEntry}
% The |\thesis@blocks@bibEntry| macro typesets a bibliographical
% entry. Along with the macros required by the locale file
% interface, the locale files need to define the following macros:
% \begin{itemize}
%   \item|\thesis@|\textit{locale}|@bib@title| -- The title of the
%     entire block
%   \item|\thesis@|\textit{locale}|@bib@author| -- The label of the
%     author name entry
%   \item|\thesis@|\textit{locale}|@bib@title| -- The label of the
%     title name entry
%   \item|\thesis@|\textit{locale}|@bib@programme| -- The label of
%     the programme name entry
%   \item|\thesis@|\textit{locale}|@bib@field| -- The label of the
%     field of study name entry
%   \item|\thesis@|\textit{locale}|@bib@advisor| -- The label of
%     the advisor name entry
%   \item|\thesis@|\textit{locale}|@bib@academicYear| -- The label
%     of the academic year entry
%   \item|\thesis@|\textit{locale}|@bib@pages| -- The label of the
%     number of pages entry
%   \item|\thesis@|\textit{locale}|@bib@keywords| -- The label of
%     the keywords entry
% \end{itemize}
% \changes{v0.3.45}{2017/05/26}{Bibliographical entries in
%   \texttt{style/mu/fithesis-sci.sty} now face each other when the
%   main locale is either Czech or English. [VN]}
%    \begin{macrocode}
\def\thesis@blocks@bibEntry{%
  \begin{alwayssingle}%
    % Clear only the right page, if the main locale is Czech.
    \ifthesis@czech
      \begingroup
      \let\thesis@blocks@clear\thesis@blocks@clearRight
    \fi
    \chapter*{\thesis@@{bib@title}}%
    \ifthesis@czech
      \endgroup
    \fi
    {% Calculate the width of the columns
    \let\@A\relax\newlength{\@A}\settowidth{\@A}{{%
      \bf\thesis@@{bib@author}:}}
    \let\@B\relax\newlength{\@B}\settowidth{\@B}{{%
      \bf\thesis@@{bib@thesisTitle}:}}
    \let\@C\relax\newlength{\@C}\settowidth{\@C}{{%
      \bf\thesis@@{bib@programme}:}}
    \let\@D\relax\newlength{\@D}\settowidth{\@D}{{%
      \bf\thesis@@{bib@field}:}}
    % Unless this is a rigorous thesis, we will be typesetting the
    % name of the thesis advisor.
    \let\@E\relax\newlength{\@E}
      \ifx\thesis@type\thesis@rigorous
        \setlength{\@E}{0pt}%
      \else
        \settowidth{\@E}{{\bf\thesis@@{bib@advisor}:}}
      \fi
    \let\@F\relax\newlength{\@F}\settowidth{\@F}{{%
      \bf\thesis@@{bib@academicYear}:}}
    \let\@G\relax\newlength{\@G}\settowidth{\@G}{{%
      \bf\thesis@@{bib@pages}:}}
    \let\@H\relax\newlength{\@H}\settowidth{\@H}{{%
      \bf\thesis@@{bib@keywords}:}}
    \let\@skip\relax\newlength{\@skip}\setlength{\@skip}{16pt}
    \let\@left\relax\newlength{\@left}\pgfmathsetlength{\@left}{%
      max(\@A,\@B,\@C,\@D,\@E,\@F,\@G,\@H)}
    \let\@right\relax\newlength{\@right}\setlength{\@right}{%
      \textwidth-\@left-\@skip}
    % Typeset the table
    \renewcommand{\arraystretch}{2}
    \noindent\begin{thesis@newtable@old}%
      {@{}p{\@left}@{\hskip\@skip}p{\@right}@{}}
      \textbf{\thesis@@{bib@author}:} &
        \noindent\parbox[t]{\@right}{
          \thesis@author\\
          \thesis@@{facultyName},
          \thesis@@{universityName}\\
          \thesis@department@name
        }\\
      \textbf{\thesis@@{bib@thesisTitle}:}
        & \thesis@title \\
      \textbf{\thesis@@{bib@programme}:}
        & \thesis@programme \\
      \textbf{\thesis@@{bib@field}:}
        & \thesis@field@name \\
      % Unless this is a rigorous thesis, typeset the name of the
      % thesis advisor.
      \ifx\thesis@type\thesis@rigorous\else
        \textbf{\thesis@@{bib@advisor}:}
          & \thesis@advisor \\
      \fi
      \textbf{\thesis@@{bib@academicYear}:}
        & \thesis@academicYear \\
      \textbf{\thesis@@{bib@pages}:}
        & \thesis@pages@preamble{} + \thesis@pages \\
      \textbf{\thesis@@{bib@keywords}:}
        & \thesis@TeXkeywords \\
    \end{thesis@newtable@old}}
  \end{alwayssingle}}
%    \end{macrocode}
% \end{macro}\begin{macro}{\thesis@blocks@bibEntryEn}
% The |\thesis@blocks@bibEntryEn| macro typesets a bibliographical
% entry in English unless the current locale is English.
%    \begin{macrocode}
\def\thesis@blocks@bibEntryEn{%
  \ifthesis@english\else
    {\thesis@selectLocale{english}
    \begin{alwayssingle}
      \chapter*{\thesis@english@bib@title}%
      {% Calculate the width of the columns
      \let\@A\relax\newlength{\@A}\settowidth{\@A}{{%
        \bf\thesis@english@bib@author:}}
      \let\@B\relax\newlength{\@B}\settowidth{\@B}{{%
        \bf\thesis@english@bib@thesisTitle:}}
      \let\@C\relax\newlength{\@C}\settowidth{\@C}{{%
        \bf\thesis@english@bib@programme:}}
      \let\@D\relax\newlength{\@D}\settowidth{\@D}{{%
        \bf\thesis@english@bib@field:}}
      % Unless this is a rigorous thesis, we will be typesetting
      % the name of the thesis advisor.
      \let\@E\relax\newlength{\@E}
        \ifx\thesis@type\thesis@rigorous
          \setlength{\@E}{0pt}%
        \else
          \settowidth{\@E}{{\bf\thesis@english@bib@advisor:}}
        \fi
      \let\@F\relax\newlength{\@F}\settowidth{\@F}{{%
        \bf\thesis@english@bib@academicYear:}}
      \let\@G\relax\newlength{\@G}\settowidth{\@G}{{%
        \bf\thesis@english@bib@pages:}}
      \let\@H\relax\newlength{\@H}\settowidth{\@H}{{%
        \bf\thesis@english@bib@keywords:}}
      \let\@skip\relax\newlength{\@skip}\setlength{\@skip}{16pt}
      \let\@left\relax\newlength{\@left}\pgfmathsetlength{\@left}{%
        max(\@A,\@B,\@C,\@D,\@E,\@F,\@G,\@H)}
      \let\@right\relax\newlength{\@right}\setlength{\@right}{%
        \textwidth-\@left-\@skip}
      % Typeset the table
      \renewcommand{\arraystretch}{2}
      \noindent\begin{thesis@newtable@old}%
        {@{}p{\@left}@{\hskip\@skip}p{\@right}@{}}
          \textbf{\thesis@english@bib@author:} &
            \noindent\parbox[t]{\@right}{
              \thesis@author\\
              \thesis@english@facultyName,
              \thesis@english@universityName\\
              \thesis@departmentEn@name
            }\\
          \textbf{\thesis@english@bib@thesisTitle:}
            & \thesis@titleEn \\
          \textbf{\thesis@english@bib@programme:}
            & \thesis@programmeEn \\
          \textbf{\thesis@english@bib@field:}
            & \thesis@fieldEn@name \\
          % Unless this is a rigorous thesis, typeset the name of the
          % thesis advisor.
          \ifx\thesis@type\thesis@rigorous\else
            \textbf{\thesis@english@bib@advisor:}
              & \thesis@advisor \\
          \fi
          \textbf{\thesis@english@bib@academicYear:}
            & \thesis@academicYear \\
          \textbf{\thesis@english@bib@pages:}
            & \thesis@pages@preamble{} + \thesis@pages \\
          \textbf{\thesis@english@bib@keywords:}
            & \thesis@TeXkeywordsEn \\
        \end{thesis@newtable@old}}
      \end{alwayssingle}
    }%
  \fi}
%    \end{macrocode}
% \end{macro}\begin{macro}{\thesis@blocks@abstractCs}
% The |\thesis@blocks@abstractCs| macro typesets the
% abstract in Czech. If the current locale is Czech, the
% macro produces no output. The following extra data field is
% defined for the macro: \begin{itemize}
%   \item|abstractCs| -- the Czech title of the thesis used for the
%     typesetting. This extra data field will expand to
%     |\thesis@abstract| if the current locale of the thesis
%     is Czech.
% \end{itemize}
% \changes{v0.3.45}{2017/05/28}{Defined the
%   \cs{thesis@blocks@abstractCs} macro in
%   \texttt{style/mu/fithesis-sci.sty}. The patch was submitted by
%   Juraj Pálenik. [VN]}
%    \begin{macrocode}
\thesis@def@extra[{
  \ifthesis@czech
    \thesis@abstract
  \else
    \thesis@placeholder@extra@abstractCs
  \fi
}]{abstractCs}
\def\thesis@blocks@abstractCs{%
  \ifthesis@czech\else
    {\thesis@selectLocale{czech}%
    \begin{alwayssingle}%
      \ifthesis@abstractonsinglepage@
        \thesis@blocks@clear
      \else
        % Start the new chapter without clearing the left page.
        \thesis@blocks@clearRight
      \fi
      {\let\thesis@blocks@clear\relax
      \chapter*{\thesis@czech@abstractTitle}%
      \thesis@extra@abstractCs}%
      \par\vfil\null
    \end{alwayssingle}}%
  \fi}
%    \end{macrocode}
% \end{macro}\begin{macro}{\thesis@blocks@bibEntryCs}
% The |\thesis@blocks@bibEntryCs| macro typesets a bibliographical
% entry in English unless the current locale is Czech. The
% macro uses the following extra data fields:\begin{itemize}
%   \item|programmeCs| -- the Czech name of the author's study
%     programme. This extra data field will expand to
%     |\thesis@programme| if the current locale of the thesis
%     is Czech.
%   \item|fieldCs| -- the Czech name of the author's field of
%     study. This extra data field will expand to
%     |\thesis@field@name| if the current locale of the thesis
%     is Czech.
%   \item|keywordsCs| -- the Czech keywords of the thesis.
%     This extra data field will expand to |\thesis@keywords| if
%     the current locale of the thesis is Czech.
%   \item|TeXkeywordsCs| -- the Czech \TeX{} keywords of the thesis.
%     This extra data field will expand to |\thesis@TeXkeywords| if
%     the current locale of the thesis is Czech.
% \end{itemize}
% \changes{v0.3.45}{2017/05/21}{Defined the
%   \cs{thesis@blocks@bibEntryCs} macro in
%   \texttt{style/mu/fithesis-sci.sty}. The patch was submitted by
%   Juraj Pálenik. [VN]}
%    \begin{macrocode}
\thesis@def@extra[{
  \ifthesis@czech
    \thesis@programme
  \else
    \thesis@placeholder@extra@programmeCs
  \fi
}]{programmeCs}
\thesis@def@extra[{
  \ifthesis@czech
    \thesis@field@name
  \else
    \thesis@placeholder@extra@fieldCs
  \fi
}]{fieldCs}
\thesis@def@extra[{
  \ifthesis@czech
    \thesis@title
  \else
    \thesis@placeholder@extra@titleCs
  \fi
}]{titleCs}
\thesis@def@extra[{
  \ifthesis@czech
    \thesis@keywords
  \else
    \thesis@placeholder@extra@keywordsCs
  \fi
}]{keywordsCs}
\thesis@def@extra[{
  \ifthesis@czech
    \thesis@TeXkeywords
  \else
    \thesis@placeholder@extra@keywordsCs
  \fi
}]{TeXkeywordsCs}
%    \end{macrocode}
% \changes{v0.3.45}{2017/05/26}{Bibliographical entries in
%   \texttt{style/mu/fithesis-sci.sty} now face each other when the
%   main locale is either Czech or English. [VN]}
%    \begin{macrocode}
\def\thesis@blocks@bibEntryCs{%
  \ifthesis@czech\else
    {\thesis@selectLocale{czech}
    \begin{alwayssingle}
      % Clear only the right page, if the main locale is English.
      \ifthesis@english
        \begingroup
        \let\thesis@blocks@clear\thesis@blocks@clearRight
      \fi
      \chapter*{\thesis@czech@bib@title}%
      \ifthesis@english
        \endgroup
      \fi
      {% Calculate the width of the columns
      \let\@A\relax\newlength{\@A}\settowidth{\@A}{{%
        \bf\thesis@czech@bib@author:}}
      \let\@B\relax\newlength{\@B}\settowidth{\@B}{{%
        \bf\thesis@czech@bib@thesisTitle:}}
      \let\@C\relax\newlength{\@C}\settowidth{\@C}{{%
        \bf\thesis@czech@bib@programme:}}
      \let\@D\relax\newlength{\@D}\settowidth{\@D}{{%
        \bf\thesis@czech@bib@field:}}
      % Unless this is a rigorous thesis, we will be typesetting
      % the name of the thesis advisor.
      \let\@E\relax\newlength{\@E}
        \ifx\thesis@type\thesis@rigorous
          \setlength{\@E}{0pt}%
        \else
          \settowidth{\@E}{{\bf\thesis@czech@bib@advisor:}}
        \fi
      \let\@F\relax\newlength{\@F}\settowidth{\@F}{{%
        \bf\thesis@czech@bib@academicYear:}}
      \let\@G\relax\newlength{\@G}\settowidth{\@G}{{%
        \bf\thesis@czech@bib@pages:}}
      \let\@H\relax\newlength{\@H}\settowidth{\@H}{{%
        \bf\thesis@czech@bib@keywords:}}
      \let\@skip\relax\newlength{\@skip}\setlength{\@skip}{16pt}
      \let\@left\relax\newlength{\@left}\pgfmathsetlength{\@left}{%
        max(\@A,\@B,\@C,\@D,\@E,\@F,\@G,\@H)}
      \let\@right\relax\newlength{\@right}\setlength{\@right}{%
        \textwidth-\@left-\@skip}
      % Typeset the table
      \renewcommand{\arraystretch}{2}
      \noindent\begin{thesis@newtable@old}%
        {@{}p{\@left}@{\hskip\@skip}p{\@right}@{}}
          \textbf{\thesis@czech@bib@author:} &
            \noindent\parbox[t]{\@right}{
              \thesis@author\\
              \thesis@czech@facultyName,
              \thesis@czech@universityName\\
              \thesis@extra@departmentCs
            }\\
          \textbf{\thesis@czech@bib@thesisTitle:}
            & \thesis@extra@titleCs \\
          \textbf{\thesis@czech@bib@programme:}
            & \thesis@extra@programmeCs \\
          \textbf{\thesis@czech@bib@field:}
            & \thesis@extra@fieldCs \\
          % Unless this is a rigorous thesis, typeset the name of the
          % thesis advisor.
          \ifx\thesis@type\thesis@rigorous\else
            \textbf{\thesis@czech@bib@advisor:}
              & \thesis@advisor \\
          \fi
          \textbf{\thesis@czech@bib@academicYear:}
            & \thesis@academicYear \\
          \textbf{\thesis@czech@bib@pages:}
            & \thesis@pages@preamble{} + \thesis@pages \\
          \textbf{\thesis@czech@bib@keywords:}
            & \thesis@extra@TeXkeywordsCs \\
        \end{thesis@newtable@old}}
      \end{alwayssingle}
    }%
  \fi}
%    \end{macrocode}
% \end{macro}\begin{macro}{\thesis@blocks@frontMatter}
% The |\thesis@blocks@frontMatter| macro sets up the style
% of the front matter front matter of the thesis. The front matter
% is typeset without any visible numbering, as mandated by the
% formal requirements of the faculty.
%    \begin{macrocode}
\def\thesis@blocks@frontMatter{%
  \thesis@blocks@clear
  \pagestyle{empty}
  \parindent 1.5em
  \setcounter{page}{1}
  \pagenumbering{roman}}
%    \end{macrocode}
% \end{macro}\begin{macro}{\thesis@blocks@cover}
% The |\thesis@blocks@cover| macro typesets the thesis
% cover. The following extra data field is defined for the macro:
% \begin{itemize}
%   \item|departmentCs| -- the Czech name of the department at
%     which the thesis is being written. This extra data field will
%     expand to |\thesis@department@name| if the main locale of the
%     thesis is Czech.
% \end{itemize}
% \begin{macrocode}
\thesis@def@extra[{
  \ifthesis@czech
    \thesis@department@name
  \else
    \thesis@placeholder@extra@departmentCs
  \fi
}]{departmentCs}
\def\thesis@blocks@cover{{%
  \thesis@selectLocale{czech}
  \ifthesis@cover@
    \thesis@blocks@clear
    \begin{alwayssingle}
      \begin{center}
      {\sc\thesis@titlePage@LARGE\thesis@czech@universityName\\%
          \thesis@titlePage@Large\thesis@czech@facultyName\\[0.3em]%
          \thesis@titlePage@normalsize\thesis@extra@departmentCs}
      \vfill
      {\bf\thesis@titlePage@Huge\thesis@czech@typeName}
      \vfill
      {\thesis@titlePage@large\thesis@place
       \ \thesis@year\hfill\thesis@author}
      \end{center}
    \end{alwayssingle}
  \fi}}
%    \end{macrocode}
% \end{macro}\begin{macro}{\thesis@blocks@titlePage}
% The |\thesis@blocks@titlePage| macro typesets the thesis
% title page. Depending on the value of the |\ifthesis@color@|
% conditional, the faculty logo is loaded from either
% |\thesis@logopath|, if \texttt{false}, or from
% |\thesis@logopath color/|, if \texttt{true}.
% The following extra data field is defined for the macro:
% \begin{itemize}
%   \item|TeXtitleCs| -- the Czech title of the thesis used for the
%     typesetting. This extra data field will expand to
%     |\thesis@TeXtitle| if the main locale of the thesis is Czech.
% \end{itemize}
% \begin{macrocode}
\thesis@def@extra[{
  \ifthesis@czech
    \thesis@TeXtitle
  \else
    \thesis@placeholder@extra@titleCs
  \fi
}]{TeXtitleCs}
\def\thesis@blocks@titlePage{{%
  \thesis@blocks@clear
  \thesis@selectLocale{czech}
  \begin{alwayssingle}
    % The top of the page
    \begin{adjustwidth}{-12mm}{}
      \begin{minipage}{30mm}
        \thesis@blocks@universityLogo@color[width=30mm]
      \end{minipage}\begin{minipage}{89mm}
        \begin{center}
          {\sc\thesis@titlePage@LARGE\thesis@czech@universityName\\%
              \thesis@titlePage@Large\thesis@czech@facultyName\\[0.3em]%
              \thesis@titlePage@normalsize\thesis@extra@departmentCs}
          \rule{\textwidth}{2pt}\vspace*{2mm}
        \end{center}
      \end{minipage}\begin{minipage}{30mm}
        \thesis@blocks@facultyLogo@color[width=30mm]
      \end{minipage}
    \end{adjustwidth}
    % The middle of the page
    \vfill
    \parbox\textwidth{% Prevent vfills from squashing the leading
      \bf\thesis@titlePage@Huge\thesis@extra@TeXtitleCs}
    {\thesis@titlePage@Huge\\[0.8em]}
    {\thesis@titlePage@large\thesis@czech@typeName\\[1em]}
    {\bf\thesis@titlePage@LARGE\thesis@author\\}
    \vfill\noindent
    % The bottom of the page
    {\bf\thesis@titlePage@normalsize
      % Unless this is a rigorous thesis, typeset the name of the
      % thesis advisor.
      \ifx\thesis@type\thesis@rigorous\else
          \thesis@czech@advisorTitle: \thesis@advisor\hfill
      \fi
      \thesis@place\ \thesis@year}
  \end{alwayssingle}}}
%    \end{macrocode}
% \end{macro}\begin{macro}{\thesis@blocks@thanks}
% The |\thesis@blocks@thanks| macro typesets the
% acknowledgement, if the |\thesis@thanks| macro is
% defined. Otherwise, the macro produces no output.
% As per the faculty requirements, the acknowledgement is
% positioned at the top of the page.
% \changes{v0.3.45}{2017/05/24}{Redefined the
%   \cs{thesis@blocks@thanks} and \cs{thesis@blocks@declaration}
%   macros in \texttt{style/mu/fithesis-sci.sty}. The patch was
%   submitted by Juraj Pálenik. [VN]}
%    \begin{macrocode}
\def\thesis@blocks@thanks{%
  \thesis@blocks@clear
  \ifx\thesis@thanks\undefined\else
    \begin{alwayssingle}%
      \chapter*{\thesis@@{thanksTitle}}%
      \leavevmode\thesis@thanks
    \end{alwayssingle}%
  \fi}
%    \end{macrocode}
% \end{macro}\begin{macro}{\thesis@blocks@declaration}
% The |\thesis@blocks@declaration| macro typesets the declaration
% text. Unlike the generic |\thesis@blocks@declaration| macro from
% the \texttt{style/mu/fithesis-sci.sty} file, this definition
% includes the date and a blank line for the author's signature, as
% per the requirements of the faculty.
%    \begin{macrocode}
\def\thesis@blocks@declaration{%
  \begin{alwayssingle}%
    \leavevmode\vfill
    % Start the new chapter without clearing any page.
    {\let\thesis@blocks@clear\relax
    \chapter*{\thesis@@{declarationTitle}}}%
    \thesis@declaration
    \vskip 2cm%
    {\let\@A\relax\newlength{\@A}
      \settowidth{\@A}{\thesis@@{authorSignature}}
      \setlength{\@A}{\@A+1cm}
    \noindent\thesis@place, \thesis@@{formattedDate}\hfill
    \begin{minipage}[t]{\@A}%
      \centering\rule{\@A}{1pt}\\
      \thesis@@{authorSignature}\par
    \end{minipage}}
  \end{alwayssingle}}
%    \end{macrocode}
% \end{macro}
% Note that there is no direct support for the seminar paper and
% thesis proposal types.  If you would like to change the contents
% of the preamble and the postamble, you should modify the
% |\thesis@blocks@preamble| and |\thesis@blocks@postamble| macros.
%
% All blocks within the autolayout preamble and postamble that are
% not defined within this file are defined in the
% \texttt{style/mu/fithesis-base.sty} file. The entire front matter
% is typeset as though the locale were Czech in accordance with the
% formal requirements of the faculty.
%    \begin{macrocode}
\def\thesis@blocks@preamble{
  \thesis@blocks@coverMatter
    \thesis@blocks@cover
  \thesis@blocks@frontMatter
    \thesis@blocks@titlePage
    \thesis@blocks@clearRight
      \thesis@blocks@bibEntryCs
      \thesis@blocks@bibEntry
      \thesis@blocks@bibEntryEn
      \thesis@blocks@abstractCs
      \ifthesis@abstractonsinglepage@
        \begingroup
          \let\clearpage\relax
      \fi
          \thesis@blocks@abstract
          \thesis@blocks@abstractEn
      \ifthesis@abstractonsinglepage@
        \endgroup
      \fi
    \thesis@blocks@assignment
    {\thesis@selectLocale{czech}%
    \thesis@blocks@thanks
    \thesis@blocks@declaration
    \thesis@blocks@clear
      \pagestyle{plain}%
      \thesis@blocks@tables}}
\def\thesis@blocks@postamble{%
  \thesis@blocks@bibliography}
%    \end{macrocode}

% \subsection{The Faculty of Arts}
% % \file{theme/mu/beamercolorthemefibeamer-phil.sty}
% This is the color theme for presentations written at the Faculty
% of Arts at the Masaryk University in Brno. This theme has no
% effect outside the presentation mode.
%    \begin{macrocode}
\NeedsTeXFormat{LaTeX2e}
\ProvidesPackage{fibeamer/theme/mu/%
  beamercolorthemefibeamer-mu-phil}[2016/05/06]
\mode<presentation>
%    \end{macrocode}
% This color theme uses the combination of yellow and shades of gray.  The
% |fibeamer@{dark,|\-|light}@background{Inner,|\-|Outer}| colors are used
% within the background canvas template, which is defined within the base
% color theme of the Masaryk University and which draws the gradient
% background of the frames.
%    \begin{macrocode}
  \definecolor{fibeamer@black}{HTML}{000000}
  \definecolor{fibeamer@white}{HTML}{FFFFFF}
  \definecolor{fibeamer@blue}{HTML}{0071B2}
  \colorlet{fibeamer@lightBlue}{fibeamer@blue!30!fibeamer@white}
  \colorlet{fibeamer@darkBlue}{fibeamer@blue!60!fibeamer@black}
  \definecolor{fibeamer@gray}{HTML}{999999}
  \definecolor{fibeamer@lightOrange}{HTML}{FFA25E}
  \colorlet{fibeamer@orange}{fibeamer@lightOrange!80!fibeamer@darkBlue}
%    \end{macrocode}
% \changes{v1.1.4:2}{2016/05/06}{Removed gradient backgrounds from
%   the color themes of the Masaryk University in Brno. [VN]}
%    \begin{macrocode}
  %% Background gradients
  \colorlet{fibeamer@dark@backgroundInner}{fibeamer@darkBlue}
  \colorlet{fibeamer@dark@backgroundOuter}{fibeamer@darkBlue}
  \colorlet{fibeamer@light@backgroundInner}{fibeamer@white}
  \colorlet{fibeamer@light@backgroundOuter}{fibeamer@white}
%    \end{macrocode}
% The |darkframes| environment switches the |\iffibeamer@darktrue|
% conditional on and sets a dark color theme.
%    \begin{macrocode}
  \renewenvironment{darkframes}{%
    \begingroup
      \fibeamer@darktrue
      %% Structures
      \setbeamercolor*{frametitle}{fg=fibeamer@lightBlue}
      \setbeamercolor*{framesubtitle}{fg=fibeamer@white}
      %% Text
      \setbeamercolor*{normal text}{fg=fibeamer@white, bg=fibeamer@blue}
      \setbeamercolor*{structure}{fg=fibeamer@white, bg=fibeamer@blue}
%    \end{macrocode}
% \changes{v1.1.0:7}{2016/01/12}{Added support for \cs{alert} to
%   the themes of the Masaryk University in Brno. [VN]}
% \changes{v1.1.4:5}{2016/05/06}{Unified the alert colors in the
%   color themes of the Masaryk University in Brno. [VN]}
%    \begin{macrocode}
      \setbeamercolor*{alerted text}{fg=fibeamer@lightOrange}
%    \end{macrocode}
% \changes{v1.1.4:3}{2016/05/06}{Added proper link coloring for the
%   color themes of the Masaryk University in Brno. [VN]}
% \changes{v1.1.6}{2017/04/23}{Added proper citation coloring for the
%   color themes of the Masaryk University in Brno. [VN]}
%    \begin{macrocode}
      %% Items, footnotes and links
      \setbeamercolor*{item}{fg=fibeamer@lightBlue}
      \setbeamercolor*{footnote mark}{fg=fibeamer@lightBlue}
      \hypersetup{urlcolor=fibeamer@lightBlue, citecolor=fibeamer@lightBlue}
      %% Blocks
      \setbeamercolor*{block title}{%
        fg=fibeamer@white, bg=fibeamer@blue!60!fibeamer@white}
      \setbeamercolor*{block title example}{%
        fg=fibeamer@white, bg=fibeamer@blue!60!fibeamer@white}
      \setbeamercolor*{block title alerted}{%
        fg=fibeamer@darkBlue, bg=fibeamer@lightOrange}
      \setbeamercolor*{block body}{%
        fg=fibeamer@blue, 
        bg=fibeamer@gray!15!fibeamer@white}
      \usebeamercolor*{normal text}
      % Code listings
      \lstset{%
        commentstyle=\color{green!30!fibeamer@white},
        keywordstyle=\color{fibeamer@lightBlue},
        stringstyle=\color{red!30!fibeamer@white}}
      }{%
    \endgroup}
%    \end{macrocode}
% Outside the |darkframes| environment, the light theme is used.
%    \begin{macrocode}
  %% Structures
  \setbeamercolor{frametitle}{fg=fibeamer@blue}
  \setbeamercolor{framesubtitle}{fg=fibeamer@black!75!fibeamer@white}
  %% Text
  \setbeamercolor{normal text}{fg=fibeamer@black, bg=fibeamer@white}
  \setbeamercolor{structure}{fg=fibeamer@black, bg=fibeamer@white}
%    \end{macrocode}
% \changes{v1.1.0:7}{2016/01/12}{Added support for \cs{alert} to
%   the themes of the Masaryk University in Brno. [VN]}
% \changes{v1.1.4:5}{2016/05/06}{Unified the alert colors in the
%   color themes of the Masaryk University in Brno. [VN]}
%    \begin{macrocode}
  \setbeamercolor{alerted text}{fg=red}
  \addtobeamertemplate{block begin}{%
    \iffibeamer@dark % alerted text in plain block at dark slides
      \setbeamercolor{alerted text}{fg=fibeamer@orange}%
    \fi}{}
%    \end{macrocode}
% \changes{v1.1.4:3}{2016/05/06}{Added proper link coloring for the
%   color themes of the Masaryk University in Brno. [VN]}
% \changes{v1.1.6}{2017/04/23}{Added proper citation coloring for the
%   color themes of the Masaryk University in Brno. [VN]}
%    \begin{macrocode}
  %% Items, footnotes and links
  \setbeamercolor*{item}{fg=fibeamer@blue}
  \setbeamercolor*{footnote mark}{fg=fibeamer@blue}
  \hypersetup{urlcolor=fibeamer@blue, citecolor=fibeamer@blue}
  %% Blocks
  \setbeamercolor{block title}{%
    fg=fibeamer@white, bg=fibeamer@blue}
  \setbeamercolor{block title example}{%
    fg=fibeamer@white, bg=fibeamer@blue}
  \setbeamercolor{block title alerted}{%
    fg=fibeamer@white, bg=red}
  \setbeamercolor{block body}{%
    fg=fibeamer@blue, bg=fibeamer@gray!20!fibeamer@white}
  %% Title
  \setbeamercolor{title}{fg=fibeamer@white, bg=fibeamer@blue}
  % Code listings
  \lstset{%
    basicstyle=\footnotesize\ttfamily,
    breakatwhitespace=false,
    breaklines=true,
    commentstyle=\color{green!60!fibeamer@black},
    extendedchars=true,
    keywordstyle=\color{fibeamer@blue},
    showspaces=false,
    showstringspaces=false,
    showtabs=false,
    stringstyle=\color{violet}}
\mode
<all>
%    \end{macrocode}

% \subsection{The Faculty of Education}
% % \file{style/mu/fithesis-ped.sty}
% This is the style file for the theses written at the Faculty of
% Education at the Masaryk University in Brno. It has been prepared
% in accordance with the formal requirements published at the
% of the faculty\footnote{See \url{https://is.muni.cz/auth/do/p^^A
% ed/VPAN/pokdek/Pokyn_dekana_zaverecne_prace_finale__1_.pdf}}.
%    \begin{macrocode}
\NeedsTeXFormat{LaTeX2e}
\ProvidesPackage{fithesis/style/mu/fithesis-ped}[2016/04/18]
%    \end{macrocode}
% The file defines the color scheme of the respective faculty. Note
% the the color definitions are in RGB, which makes the resulting
% files generally unsuitable for printing.
%    \begin{macrocode}
\thesis@color@setup{
  links={HTML}{FFA02F},
  tableEmph={HTML}{FFBB6B},
  tableOdd={HTML}{FFF1E0},
  tableEven={HTML}{FFDEB7}}
%    \end{macrocode}
% The bibliography support is enabled. The |numeric| citations are
% used and the bibliography is sorted by name, title, and year.
%    \begin{macrocode}
\thesis@bibliography@setup{
  style=iso-numeric,
  sorting=nty}
\thesis@bibliography@load
%    \end{macrocode}
% The style file configures the title page header to include the
% department name and the title page content to include the
% advisor's name.
%    \begin{macrocode}
\thesis@blocks@titlePage@department@true
\def\thesis@blocks@titlePage@content{%
    {\thesis@titlePage@Huge\bf\thesis@TeXtitle\par\vfil}\vskip 0.8in
    {\thesis@titlePage@large\sc\thesis@@{typeName}\\[0.3in]}
    {\thesis@titlePage@Large\bf\thesis@author}
    % Typeset the name of the thesis advisor.
    {\thesis@titlePage@large\\[0.3in]
      {\bf\thesis@@{advisorTitle}:} \thesis@advisor}}
%    \end{macrocode}
% Note that there is no direct support for the seminar paper and
% thesis proposal types.  If you would like to change the contents
% of the preamble and the postamble, you should modify the
% |\thesis@blocks@preamble| and |\thesis@blocks@postamble| macros.
%
% All blocks within the autolayout preamble and postamble that are
% not defined within this file are defined in the
% \texttt{style/mu/fithesis-base.sty} file.
%    \begin{macrocode}
\def\thesis@blocks@preamble{%
  \thesis@blocks@coverMatter
    \thesis@blocks@cover
    \thesis@blocks@titlePage
  \thesis@blocks@frontMatter
    \thesis@blocks@bibEntry
    \thesis@blocks@abstract
    \thesis@blocks@abstractEn
    \thesis@blocks@keywords
    \thesis@blocks@keywordsEn
    \thesis@blocks@declaration
    \thesis@blocks@thanks
    \thesis@blocks@tables}
\def\thesis@blocks@postamble{%
  \thesis@blocks@bibliography}
%    \end{macrocode}

% \subsection{The Faculty of Social Studies}
% % \file{theme/mu/beamercolorthemefibeamer-fss.sty}
% This is the color theme for presentations written at the Faculty
% of Social Studies at the Masaryk University in Brno.  This theme
% has no effect outside the presentation mode.
%    \begin{macrocode}
\NeedsTeXFormat{LaTeX2e}
\ProvidesPackage{fibeamer/theme/mu/%
  beamercolorthemefibeamer-mu-fss}[2016/05/06]
\mode<presentation>
%    \end{macrocode}
% This color theme uses the combination of yellow and shades of gray.  The
% |fibeamer@{dark,|\-|light}@background{Inner,|\-|Outer}| colors are used
% within the background canvas template, which is defined within the base
% color theme of the Masaryk University and which draws the gradient
% background of the frames.
%    \begin{macrocode}
  \definecolor{fibeamer@black}{HTML}{000000}
  \definecolor{fibeamer@white}{HTML}{FFFFFF}
  \definecolor{fibeamer@cyan}{HTML}{00796E}
  \colorlet{fibeamer@darkCyan}{fibeamer@cyan!70!fibeamer@black}
  \colorlet{fibeamer@lightCyan}{fibeamer@cyan!30!fibeamer@white}
  \definecolor{fibeamer@gray}{HTML}{999999}
  \definecolor{fibeamer@lightOrange}{HTML}{FFA25E}
  \colorlet{fibeamer@orange}{fibeamer@lightOrange!80!fibeamer@darkCyan}
%    \end{macrocode}
% \changes{v1.1.4:2}{2016/05/06}{Removed gradient backgrounds from
%   the color themes of the Masaryk University in Brno. [VN]}
%    \begin{macrocode}
  %% Background gradients
  \colorlet{fibeamer@dark@backgroundInner}{fibeamer@darkCyan}
  \colorlet{fibeamer@dark@backgroundOuter}{fibeamer@darkCyan}
  \colorlet{fibeamer@light@backgroundInner}{fibeamer@white}
  \colorlet{fibeamer@light@backgroundOuter}{fibeamer@white}
%    \end{macrocode}
% The |darkframes| environment switches the |\iffibeamer@darktrue|
% conditional on and sets a dark color theme.
%    \begin{macrocode}
  \renewenvironment{darkframes}{%
    \begingroup
      \fibeamer@darktrue
      %% Structures
      \setbeamercolor*{frametitle}{fg=fibeamer@lightCyan}
      \setbeamercolor*{framesubtitle}{fg=fibeamer@white}
      %% Text
      \setbeamercolor*{normal text}{fg=fibeamer@white, bg=fibeamer@cyan}
      \setbeamercolor*{structure}{fg=fibeamer@white, bg=fibeamer@cyan}
%    \end{macrocode}
% \changes{v1.1.0:7}{2016/01/12}{Added support for \cs{alert} to
%   the themes of the Masaryk University in Brno. [VN]}
% \changes{v1.1.4:5}{2016/05/06}{Unified the alert colors in the
%   color themes of the Masaryk University in Brno. [VN]}
%    \begin{macrocode}
      \setbeamercolor*{alerted text}{fg=fibeamer@lightOrange}
%    \end{macrocode}
% \changes{v1.1.4:3}{2016/05/06}{Added proper link coloring for the
%   color themes of the Masaryk University in Brno. [VN]}
% \changes{v1.1.6}{2017/04/23}{Added proper citation coloring for the
%   color themes of the Masaryk University in Brno. [VN]}
%    \begin{macrocode}
      %% Items, footnotes and links
      \setbeamercolor{item}{fg=fibeamer@lightCyan}
      \setbeamercolor{footnote mark}{fg=fibeamer@lightCyan}
      \hypersetup{urlcolor=fibeamer@lightCyan, citecolor=fibeamer@lightCyan}
      %% Blocks
      \setbeamercolor*{block title}{%
        fg=fibeamer@white, bg=fibeamer@cyan!60!fibeamer@white}
      \setbeamercolor*{block title example}{%
        fg=fibeamer@white, bg=fibeamer@cyan!60!fibeamer@white}
      \setbeamercolor*{block title alerted}{%
        fg=fibeamer@darkCyan, bg=fibeamer@lightOrange}
      \setbeamercolor*{block body}{%
        fg=fibeamer@cyan, 
        bg=fibeamer@gray!15!fibeamer@white}
      \usebeamercolor*{normal text}
      % Code listings
      \lstset{%
        commentstyle=\color{green!30!fibeamer@white},
        keywordstyle=\color{blue!30!fibeamer@white},
        stringstyle=\color{red!30!fibeamer@white}}
      }{%
    \endgroup}
%    \end{macrocode}
% Outside the |darkframes| environment, the light theme is used.
%    \begin{macrocode}
  %% Structures
  \setbeamercolor{frametitle}{fg=fibeamer@cyan}
  \setbeamercolor{framesubtitle}{fg=fibeamer@black!75!fibeamer@white}
  %% Text
  \setbeamercolor{normal text}{fg=fibeamer@black, bg=fibeamer@white}
  \setbeamercolor{structure}{fg=fibeamer@black, bg=fibeamer@white}
%    \end{macrocode}
% \changes{v1.1.0:7}{2016/01/12}{Added support for \cs{alert} to
%   the themes of the Masaryk University in Brno. [VN]}
% \changes{v1.1.4:5}{2016/05/06}{Unified the alert colors in the
%   color themes of the Masaryk University in Brno. [VN]}
%    \begin{macrocode}
  \setbeamercolor{alerted text}{fg=red}
  \addtobeamertemplate{block begin}{%
    \iffibeamer@dark % alerted text in plain block at dark slides
      \setbeamercolor{alerted text}{fg=fibeamer@orange}%
    \fi}{}
%    \end{macrocode}
% \changes{v1.1.4:3}{2016/05/06}{Added proper link coloring for the
%   color themes of the Masaryk University in Brno. [VN]}
% \changes{v1.1.6}{2017/04/23}{Added proper citation coloring for the
%   color themes of the Masaryk University in Brno. [VN]}
%    \begin{macrocode}
  %% Items, footnotes and links
  \setbeamercolor{item}{fg=fibeamer@cyan}
  \setbeamercolor{footnote mark}{fg=fibeamer@cyan}
  \hypersetup{urlcolor=fibeamer@cyan, citecolor=fibeamer@cyan}
  %% Blocks
  \setbeamercolor{block title}{%
    fg=fibeamer@white, bg=fibeamer@cyan}
  \setbeamercolor{block title example}{%
    fg=fibeamer@white, bg=fibeamer@cyan}
  \setbeamercolor{block title alerted}{%
    fg=fibeamer@white, bg=red}
  \setbeamercolor{block body}{%
    fg=fibeamer@cyan, bg=fibeamer@gray!20!fibeamer@white}
  %% Title
  \setbeamercolor{title}{fg=fibeamer@white, bg=fibeamer@cyan}
  % Code listings
  \lstset{%
    basicstyle=\footnotesize\ttfamily,
    breakatwhitespace=false,
    breaklines=true,
    commentstyle=\color{green!60!fibeamer@black},
    extendedchars=true,
    keywordstyle=\color{blue},
    showspaces=false,
    showstringspaces=false,
    showtabs=false,
    stringstyle=\color{violet}}
\mode
<all>
%    \end{macrocode}

% \subsection{The Faculty of Law}
% % \iffalse
%<*driver>
\documentclass{ltxdoc}
\usepackage{camel}
\usepackage{array}

\def\fileversion{0.1d}
\def\filedate{1997/01/09}
\def\docdate {1997/01/09}

\EnableCrossrefs
\RecordChanges                  % Gather update information
\CodelineIndex                  % Index code by line number
%\OnlyDescription  % comment out for implementation details
\let\lawhline=\hline
\begin{document}
\citationstyle{law}
\citationdata{camel}
\title{The {\sc Law} module for the {\sc Camel} Bibliography
  Engine\thanks{%
   This file has version number \fileversion{} dated
   \filedate{}.
   The documentation was last revised on \docdate.
   The documentation and the code for {\sc Camel}
   are \copyright{} 1992--95 Frank Bennett, Jr.  Distribution
   and use are freely welcomed, on the sole condition that
   acknowledgement of the {\sc Camel} package, its {\sc Law}
   module and of their author
   be made in any published using these utilities.
      }}
\author{Frank G Bennett, Jr.\thanks{Lecturer in the Commercial
    Laws of the Far East, School of
    Oriental and African Studies, University of London.
    Acknowledgements of the numerous people who have provided
    comments and suggestions in the development of this module
    and of the {\sc Camel} package itself are listed in the users'
    guide to that software.}}
\maketitle
\DocInput{law.dtx}
\end{document}
%</driver>
% \fi
% \CheckSum{741}
% 
%\iffalse
%
% Copyright (C) 1992-1995 by Frank Bennett.  All rights reserved.
%
% IMPORTANT NOTICE:
% 
% You are not allowed to change this file.  You may however copy
% this file to a file with a different name and then change the
% copy if (a) you do not charge for the modified code, (b) you
% acknowledge Camel and its author(s) in the new file, if it
% is distributed to others, and (c) you attach these same
% conditions to the new file.
% 
% You are not allowed to distribute this file alone.  You are not
% allowed to take money for the distribution or use of this file
% (or a changed version) except for a nominal charge for copying
% etc.
% 
% You are allowed to distribute this file under the condition that
% it is distributed with all of its contents, intact.
% 
% For error reports, or offers to help make Camel a more powerful,
% friendlier, and altogether more thrilling package, please contact me on
% fb@soas.ac.uk
% 
%\fi
%
%
% \DoNotIndex{\@,\@@par,\@beginparpenalty,\@empty}
% \DoNotIndex{\@flushglue,\@gobble,\@input}
% \DoNotIndex{\@makefnmark,\@makeother,\@maketitle}
% \DoNotIndex{\@namedef,\@ne,\@spaces,\@tempa}
% \DoNotIndex{\@tempb,\@tempswafalse,\@tempswatrue}
% \DoNotIndex{\@thanks,\@thefnmark,\@topnum}
% \DoNotIndex{\@@,\@elt,\@forloop,\@fortmp,\@gtempa,\@totalleftmargin}
% \DoNotIndex{\",\/,\@ifundefined,\@nil,\@verbatim,\@vobeyspaces}
% \DoNotIndex{\|,\~,\ ,\active,\advance,\aftergroup,\begingroup,\bgroup}
% \DoNotIndex{\cal,\csname,\def,\documentstyle,\dospecials,\edef}
% \DoNotIndex{\egroup}
% \DoNotIndex{\else,\endcsname,\endgroup,\endinput,\endtrivlist}
% \DoNotIndex{\expandafter,\fi,\fnsymbol,\futurelet,\gdef,\global}
% \DoNotIndex{\hbox,\hss,\if,\if@inlabel,\if@tempswa,\if@twocolumn}
% \DoNotIndex{\ifcase}
% \DoNotIndex{\ifcat,\iffalse,\ifx,\ignorespaces,\index,\input,\item}
% \DoNotIndex{\jobname,\kern,\leavevmode,\leftskip,\let,\llap,\lower}
% \DoNotIndex{\m@ne,\next,\newpage,\nobreak,\noexpand,\nonfrenchspacing}
% \DoNotIndex{\obeylines,\or,\protect,\raggedleft,\rightskip,\rm,\sc}
% \DoNotIndex{\setbox,\setcounter,\small,\space,\string,\strut}
% \DoNotIndex{\strutbox}
% \DoNotIndex{\thefootnote,\thispagestyle,\topmargin,\trivlist,\tt}
% \DoNotIndex{\twocolumn,\typeout,\vss,\vtop,\xdef,\z@}
% \DoNotIndex{\,,\@bsphack,\@esphack,\@noligs,\@vobeyspaces,\@xverbatim}
% \DoNotIndex{\`,\catcode,\end,\escapechar,\frenchspacing,\glossary}
% \DoNotIndex{\hangindent,\hfil,\hfill,\hskip,\hspace,\ht,\it,\langle}
% \DoNotIndex{\leaders,\long,\makelabel,\marginpar,\markboth,\mathcode}
% \DoNotIndex{\mathsurround,\mbox,\newcount,\newdimen,\newskip}
% \DoNotIndex{\nopagebreak}
% \DoNotIndex{\parfillskip,\parindent,\parskip,\penalty,\raise,\rangle}
% \DoNotIndex{\section,\setlength,\TeX,\topsep,\underline,\unskip,\verb}
% \DoNotIndex{\vskip,\vspace,\widetilde,\\,\%,\@date,\@defpar}
% \DoNotIndex{\[,\{,\},\]}
% \DoNotIndex{\count@,\ifnum,\loop,\today,\uppercase,\uccode}
% \DoNotIndex{\baselineskip,\begin,\tw@}
% \DoNotIndex{\a,\b,\c,\d,\e,\f,\g,\h,\i,\j,\k,\l,\m,\n,\o,\p,\q}
% \DoNotIndex{\r,\s,\t,\u,\v,\w,\x,\y,\z,\A,\B,\C,\D,\E,\F,\G,\H}
% \DoNotIndex{\I,\J,\K,\L,\M,\N,\O,\P,\Q,\R,\S,\T,\U,\V,\W,\X,\Y,\Z}
% \DoNotIndex{\1,\2,\3,\4,\5,\6,\7,\8,\9,\0}
% \DoNotIndex{\!,\#,\$,\&,\',\(,\),\+,\.,\:,\;,\<,\=,\>,\?,\_}
% \DoNotIndex{\discretionary,\immediate,\makeatletter,\makeatother}
% \DoNotIndex{\meaning,\newenvironment,\par,\relax,\renewenvironment}
% \DoNotIndex{\repeat,\scriptsize,\selectfont,\the,\undefined}
% \DoNotIndex{\arabic,\do,\makeindex,\null,\number,\show,\write,\@ehc}
% \DoNotIndex{\@author,\@ehc,\@ifstar,\@sanitize,\@title,\everypar}
% \DoNotIndex{\if@minipage,\if@restonecol,\ifeof,\ifmmode}
% \DoNotIndex{\lccode,\newtoks,\onecolumn,\openin,\p@,\SelfDocumenting}
% \DoNotIndex{\settowidth,\@resetonecoltrue,\@resetonecolfalse,\bf}
% \DoNotIndex{\clearpage,\closein,\lowercase,\@inlabelfalse}
% \DoNotIndex{\selectfont,\mathcode,\newmathalphabet,\rmdefault}
% \DoNotIndex{\bfdefault}
% \setcounter{StandardModuleDepth}{1}
% \MakeShortVerb{\"}
% 
% \begin{abstract}
% \noindent The {\tt Law} module for the {\sc Camel} bibliography
% package and \BibTeX{} attempts to implement fully automated typesetting
% of citations in the so-called Blue Book style used in
% the publication of legal materials.  This
% demanding style requires context sensitive
% in-footnote cross-referencing
% between citations.
% \end{abstract}
%
% \noindent An adage in office management is that you should
% only touch incoming paper once; to respond to it, to file
% it, to forward it, or to destroy it.  A number of 
% commercial citation
% database managers provide a facility for ``filing'' citations
% in a flexible form, the idea being to extend this principle
% to citations as well as paper.  A hanging point for this
% strategy as been in-text context-sensitive citation styles.
% Database managers are at their best in exporting 
% entire bibliographies
% and lists of authorities.  Some packages are
% capable of scanning a document for citation ``tags'', 
% which eliminates the need to separately select bibliography
% items in the database.  Some, too, can replace ``tags''
% in the document with the formatted text of a citation.
% But once the text is replaced, the format of citations
% added in this way is fixed; conversion is a one-way process.
%
% A more serious problem is that, while the database manager
% can easily identify the {\em tag}, it is far more difficult,
% without logical markup, to identify its 
% {\em context\/}---whether it occurs in a footnote, how many
% items were in the preceding footnote, how many articles
% by the same author are cited in the document, and so forth.
% As a result, citation formatting of cross-referenced styles
% is still generally done by hand.
%
% One of the most demanding cross-referenced styles is that
% laid down in {\sl A Uniform System of Citation}, or ``the Blue
% Book'', for the citation of legal materials.  This style
% has survived the era of computerization largely because
% most U.S. law journals using the style are edited by highly
% competitive law students.  Staff members contribute their
% editorial time to their journal free of charge, because of
% the value of listing law journal membership on their resume.
% The Blue Book style minimizes the bulk of citation
% text, while conveying sufficient information to the reader
% for the location of cited material.  It is also designed
% to provide all information required for the location of
% {\em primary\/} legal authority from the face of any citation
% to it, without tracing down cross-references.
%
% A first attempt to address this problem was made in 
% the complementary \LexiTeX{}
% and \LexiBib{} packages, for \LaTeX{} and \BibTeX{}
% respectively.  Lessons learned from developing these
% packages suggested that a comprehensive formatting package
% for legal citations could serve as the foundation for
% a modular, highly generalized citation and bibliography
% formatting system.  The logical code from these packages
% was excised and used in the drafting of the {\sc Camel}
% bibliography engine.  The typesetting code was then used
% to assemble the {\sc Law} module for {\sc Camel}.
%
% The implications of such a system are particularly interesting
% if it comes to be widely used in the publication of
% law journals or, even better, in the publication of court
% judgements.  The efficiency with which
% citations can be reported to citation services would be
% increased, since \BibTeX-format database entries
% are in a standard format that can
% be processed electronically.  These same lists could
% be made available as text-searchable databases of authority.
% Network discussions of legal issues could be accompanied by
% growing lists of annotated authority, available for all to
% use.
%
% To use this package, all you should need to do is unpack
% the files by running {\tt law.ins} (thanks to Robin Fairbairns
% for pushing me toward a standard-ish method of installation!),
% copy the style files to their respective homes, and follow the
% markup conventions described in the {\sc Camel} manual.  The
% same should apply to any other bibliography
% modules produced for {\sc Camel} in the future, no matter what
% sort of contortions they put your citation data through before
% they end up on the page.  If it has extensive documentation,
% it's {\em not\/} a {\sc Camel} module!
%
% You are not under an obligation to enjoy this package.  If you
% have complaints, you can contact me on {\tt fb@soas.ac.uk}.
%
%\StopEventually{}
% \section{The style code}
% \subsubsection{Entry type specific functions}
% The functions below should not be used as general utilities;
% they are designed specifically for use with a particular entry
% type.  While they could be placed directly into the entry
% functions to which they apply, defining them separately
% helps improve the transparency of the code.
%
% \DescribeEnv{build.bridges}
% I'm not actually sure whether this function belongs in {\tt
% camel.dtx} or in {\tt law.dtx}.  What it does is to insert (or
% not insert) an appropriate "\bridges" declaration into the {\tt
% .bbl} file for the current citation.  It is governed by a
% string existing on the stack when it is invoked, and leaves
% nothing on the stack.  Toggles are {\tt schedules}, {\tt sections} and {\tt
% articles}.  Season to taste.
%    \begin{macrocode}
%<*bstfunctions>
FUNCTION {build.bridges}
  { duplicate$ empty$
    { pop$ skip$ }
    { duplicate$ "sections" =
      { pop$ "\bridges{\ \S~}{}{\ \S\S~}{\ }{\ \S~}{\ \S\S~}"
      "" "" must.must.must
      }

      { duplicate$ "articles" =
        { pop$
          "\bridges{\ art.~}{}{\ arts.~}{\ }{\ art.~}{\ arts.~}"
          "" "" must.must.must
        }
        { "schedules" =
          { "\bridges{\ sched.~}{}{\ scheds.~}{\ }{\ sched.~}{\ scheds.~}"
            "" "" must.must.must
          }
          'skip$
          if$
        }if$
      }if$
    }if$
 }
%    \end{macrocode}
% \DescribeEnv{get.a.kinda.sort.key}
% It is early days for the sorting of bibliographies in {\sc
% Camel}.  This will give a rough key.
%    \begin{macrocode}
FUNCTION { get.a.kinda.sort.key }
{     author empty$
     { title empty$
       { "0000" }
       { title "*" "forward" gather.chars pop$
         duplicate$ "l" change.letter.case "the" =
         { pop$ "*" "forward" gather.chars pop$ swap$ pop$ }
         { duplicate$ "l" change.letter.case "a" =
           { pop$ "*" "forward" gather.chars pop$ swap$ pop$ }
           { swap$ pop$
           }if$
         }if$
       }if$
     }
     { author #1 "{ll}" format.name$
     }if$
}
%    \end{macrocode}
% \DescribeEnv{j.format.division}
% This function assembles the various fields relevant to the
% division of a Japanese court into a single string.
%    \begin{macrocode}
FUNCTION {j.format.division}
 { division empty$
   'skip$
   { dc.. court "end" first.in.second
     { pop$ " No.~" 's := "endlabel" 't := }
     { pop$ sc.. court "start" first.in.second
       { pop$ "No.\ " 's := "frontlabel" 't := }
       {  pop$ " " 's :=  "endlabel" 't :=
       }if$
     }if$
   }if$
   division s divno t field.tag.no.combine
 }
%    \end{macrocode}
%
% \subsubsection{Entry type functions}
% Now we define the type functions for all entry types that may
% appear
% in the {\tt*.bib} file---e.g., functions like `"article"' and 
% `"book"'.  These
% are the routines that actually generate the "*.bbl" file output
% for
% the entry.  These must all precede the "READ" command.  In
% addition, the
% style designer should have a function `"default.type"' for
% unknown types.\footnote{This comment by Oren Patashnik.}
% 
% \DescribeEnv{article}
% This function performs the necessary operations for
% exporting a valid \LexiTeX\ citation to an article.
% For this and for all citation types defined in the
% \LexiBib{} style, the goal is to provide reasonably
% complete commentary, so that anyone wanting to
% alter the behaviour of the style can set to work
% with a fair degree of confidence about what needs to
% be done to achieve a particular result.
%    \begin{macrocode}
FUNCTION {article}
%    \end{macrocode}
% The "article" entry is used for all kinds of material, so
% it ends up as one of the most complex entries.  Before we
% do anything, we have to check whether the default {\sc Camel}
% bridges are acceptable.  There are two types of situations to watch out
% for.  First, if the volume and the number are both
% non-empty, we need to add a special set of substitute bridges.
% Alternatively, if both "volume" and "number" are empty,
% but the date is more specific than just a year, we
% assume we are dealing with a newspaper.  In either case,
% we use the same special set of bridges.
%    \begin{macrocode}
{ volume empty$ not number empty$ not and
  volume empty$ number empty$ year "mo.dd.yy" format.date
  pop$ itemcount #1 = not and and or
  { "\bridges{,\ p.~}{,\ }{,\ }{\ }{\ at~}{\ at~}" 
    "" "" must.must.must 
    newline$
  }
%    \end{macrocode}
% Second, the {\tt @article} entry type, like many entry types in
% the Blue Book style normally places 
% white space between the title
% and a pinpoint page number.  If the title ends in a numeral
% this will be confusing, so the Blue Book requires that we
% separate the two with a comma in this case.  The following
% adjustment to bridging punctuation accomplishes this purpose.
%    \begin{macrocode}
  { title type.last.char "numeral" =
    { "\bridges{,\ }{,\ }{,\ }{\ }{\ at~}{\ at~}"
      "" "" must.must.must
      newline$
    }
    'skip$
    if$
  }if$
%    \end{macrocode}
% \changes{1995/07/08}{v0.1a}{Function {\tt type.last.char} and
% {\tt\protect\string\protect\bridges} declaration used to handle titles ending
% in a numeral correctly (by the insertion of a comma).  Added to
% the {\tt @article} and {@book} entries as a trial; will
% propagate to other entry types once this change is trusted.}
% Then we write the citation leader, to prepare for outputting the
% actual content of the citation text.
%    \begin{macrocode}
 "\lexibib{article}{" cite$ "}{" must.must.must
 get.a.kinda.sort.key 
 "}{" "" must.must.must 
%    \end{macrocode}
% The author name is pushed to the stack, followed by a toggle
% to trigger last-name-only formatting.  Then the "format.names"
% reduces this to a single, appropriately-formatted string,
% possibly the null string.  All that is left to do is push a
% set a braces, a null string to make up three arguments to the
% export routine, and write the lot on the output
% file unconditionally, using "must.must.must".
%    \begin{macrocode}
author  "lastonly" format.names
"}{"
""
must.must.must
%    \end{macrocode}
% Next comes the title of the article.  This is not specially
% formatted; we simply push the title field, then a warning
% string
% followed by a check for whether it is empty or not, then
% braces and a null string, and write again.
%    \begin{macrocode}
title "title" check
"}{"
""
must.must.must
%    \end{macrocode}
% The next bit is actually rather thorny.  There are three
% possible cases.  The first is where both a volume number
% and an issue number exist.  In this case, we use a
% verbose form of reference. In the second, there is no
% volume number, but possibly an issue number.  This is
% the proper form for Commonwealth legal materials,
% and requires the year in brackets, followed by the 
% issue number to show the volume within the year.
% Third, we may simply have a volume number by itself.
% This calls for the citation form for most journals,
% and U.S. case reporters.
%    \begin{macrocode}
  volume empty$ not number empty$ not and
  { journal "journal" check ", v.~" volume must.must.must
    ", n.~" number "}{" must.must.must }
  { volume empty$ not
    { volume "\ " journal "journal" check empty.to.null
      might.ifone.must
      "}{" "" "" must.must.must}
    { year "mo.dd.yy" format.date itemcount #1 =
      { "[" swap$ "] " iftwo.might.iftwo
      number "\ " journal "journal" check might.ifone.must
      "}{" "" "" must.must.must}
      { pop$ number empty$
        { "\\" journal "journal" check "\\" must.must.must
          "}{" "" "" must.must.must }
        { number "\ " journal "journal" check might.ifone.must
          "}{" "" "" must.must.must
        }if$
      }if$
    }if$
  }if$ 
%    \end{macrocode}
% The Blue Book does not like page ranges
% so we need to clean
% out anything following a dash in the "pages" field.  The
% "short" option
% toggles this behaviour on.  We also check to see that the page
% is not empty.  This is followed by braces, a null string, and
% output.
%    \begin{macrocode}
pages "short" format.pages "pages" check
"}{"
""
must.must.must
%    \end{macrocode}
% We add the year next, but only if the "volume" field is
% non-empty (if "volume" is empty, we'll have put the year
% is as a bracketed volume number, Commonwealth-style.
% We could use just the year (this is normal for Blue Book
% style, but we'll add the month for good measure, if it's
% been provided in the "year" field.
%    \begin{macrocode}
volume empty$ not
  { "(" "" "" must.must.must
    publisher ":\ " "" might.ifone.must
    "" year "mo.dd.yy" format.date ")" must.must.must }
  { year "mo.dd.yy" format.date itemcount #1 =
    'pop$
    { "" "" must.must.must
    }if$
  }if$
%    \end{macrocode}
% Finally, we have to tangle with cross-references.
% Yuck.  (Not a complaint about \BibTeX, just a general
% response to the design problems inherent in the task).
% Formatting depends on whether there is a "crossref" entry.
% Hear that, guys?  If you don't use the "crossref" field,
% we'll short-change you on formatting service.
%    \begin{macrocode}
crossref empty$
%    \end{macrocode}
% If there is no "crossref", we just push a brace and a couple
% of nulls, and write.  Done!  Hurray!
%    \begin{macrocode}
{ "}" "" "" must.must.must }
%    \end{macrocode}
% But if there {\em was\/} a "crossref", we've got work to do.
% Darn.  The first thing we do is have a look at "booktitle".
% This should be non-null in this situation; there's no sense
% setting up a cross-reference to an individual volume of
% a journal unless there's something special to be said
% about it.
%    \begin{macrocode}
{ booktitle empty$
%    \end{macrocode}
% So if no "booktitle" is found, we whinge and format
% as for a no-cross-reference entry.
%    \begin{macrocode}
  { "no booktitle (name of special issue) for "
     cite$ "/" crossref ".  Why a crossref?" * * * * warning$
    "}" "" "" must.must.must }
%    \end{macrocode}
% If there {\em is\/} a "booktitle", though, 
% and if the "volume" field is non-empty (which
% means that we just printed, or at least should
% have printed, the year), we close the
% parens following the year (which is opened by \LexiTeX),
% and open another (which will be closed by \LexiTeX).
% That's it for conditional punctuation; we follow
% with "booktitle", which should be 
% the title or subject description of the special
% issue of the journal.
%    \begin{macrocode}
  { volume empty$ {""} {") ("} if$ booktitle "}" must.must.must}
  if$
}if$
%    \end{macrocode}
% Add a fresh new line in the export file, and we're done!  Whew!
%    \begin{macrocode}
newline$
}
%    \end{macrocode}
%
% \DescribeEnv{book}
% This is the entry for books, which includes individual
% volumes in a series, and multi-volume works with a single
% title.  Correct me if I'm wrong, but I think this latter
% citation type is not supported by \BibTeX.  \LexiBib{}
% manages it by allowing the user to specify the volume
% number in the text using the optional ":<number>:" argument to
% the "\lexicite" tag.
%    \begin{macrocode}
FUNCTION {book}
%    \end{macrocode}
% This entry type normally places white space between the title
% and a pinpoint page number.  If the title ends in a numeral
% this will be confusing, so the Blue Book requires that we
% separate the two with a comma in this case.  The following
% adjustment to bridging punctuation accomplishes this purpose.
%    \begin{macrocode}
{ title type.last.char "numeral" =
    { "\bridges{,\ }{}{,\ }{\ }{\ at~}{\ at~}"
      "" "" must.must.must
      newline$
    }
    { units empty.to.null "paras" =
      { "\bridges{,\ para.~}{}{,\ paras.~}{\ }{,\ para.~}{,\ paras.}"
        "" "" must.must.must
        newline$
      }
      'skip$
      if$
    }if$
%    \end{macrocode}
% Next, after the opener, we push the opening macro tag for a book
% entry, the nickname of the citation, and a couple of
% braces.  This is all mandatory and can safely be given
% unconditional export.
%    \begin{macrocode}
  "\lexibib{book}{" cite$ "}{" must.must.must
   get.a.kinda.sort.key "" "}{" must.must.must
%    \end{macrocode}
% A non-empty "volume" field means we need a leading
% volume number in the Blue Book style.  If a volume
% number (or anything else) is found in the "volume"
% field of a book entry, we replace it with a 
% "volno" macro.  This will expand in the document to
% whatever the author has specified using the optional
% ":<number>" argument to "\lexicite".  For example,
% volume 8 of "holdsworth" would be: "\lexicite:8:{holdsworth}".
%    \begin{macrocode}
  volume empty$
  { "" }
  { "\volno\ " }
  if$
%    \end{macrocode}
% The author comes first.  We push the contents of the
% "author" field, then the toggle string ``"firstinitial"'',
% and run  the "format.names" function to produce the name
% formatted properly for a book entry.  Then we push
% fa couple of braces
% and force all three items ("\volno" or null, "author", and
% braces) onto the output.
%    \begin{macrocode}
  author
  "firstinitial" format.names "}{" must.must.must
%    \end{macrocode}
% The title may be a simple name for a single volume,
% or it may be in title/subtitle format.  Title/subtitle
% can be written into the "title" entry directly as
% ``\meta{title}: \meta{subtitle}''.  We provide for
% an alternative form here, just in case "booktitle"
% is used for the main title --- conservation of
% data.  If there is no "booktitle", we push the title, then
% check to be sure it's non-empty, then write it onto output.
% If "booktitle" exists, we push it, then a colon, then
% "title", followed by a conditional write on output that
% will suppress the colon if "title" is empty.
% Finally, we push the closing braces and an opening
% parens (for the year info) and write that stuff on output.
%    \begin{macrocode}
  booktitle empty$
    { title "title" check "" ""  must.must.must }
    { booktitle ": " title must.ifthree.might
    }if$
  type empty.to.null "cmnd" =
  type empty.to.null "command" =
  type empty.to.null "command papers" =
  or or
  { ", " year "mo.dd.yy" format.date
    ", " "Cmnd. "
    * * *
    series empty$
    { "no.~" }
    { "Ser.~" series ", No.~" * *
    }if$
  number "number" check
   * *
   "" "" must.must.must
   }
   'skip$
   if$
  "}{}{}{" "(" "" must.must.must
%    \end{macrocode}
% We're now in the final ``field'' of the \LexiTeX{} entry.
% This is mainly for the year, but we also give the 
% name of the editor(s) or translator(s) 
% if present.  If there is no
% editor or translator, we'll put a series name here, to help identify
% the source.  We don't put both, because this would
% confuse things (there can be book editors and
% series editors too, and the Blue Book style is 
% too streamlined to distinguish the two elegantly).
% So our first task is to see if there is an editor\ldots
%    \begin{macrocode}
    editor empty$ translator empty$ and
%    \end{macrocode}
% \ldots{} and if there is none, we put in a
% series name if it exists.  
% The "series" \BibTeX{} field should be used for the
% name of the series of which a volume forms a part,
% but some folks might accidentally use "booktitle".
% We'll be forgiving and accept it anyway.
%    \begin{macrocode}
    { series empty$
%    \end{macrocode}
%  If both
% "series" is empty, we've nothing to do.
%    \begin{macrocode}
      'skip$
%    \end{macrocode}
%
% If "series" exists, we push it, followed by
% a bridge and a series item number,
% and a toggle for the "field.tag.no.combine" function.
% The "endlabel" toggle causes this function to put
% the bridge and the number after the series name,
% if a number exists, and push the lot back as a single
% item on the stack.  Otherwise it leaves just the series
% name.
%
% And last, we put up a comma to close, and do a mandatory
% export of the lot.
%    \begin{macrocode}
      { series " No.~" number
      "endlabel" field.tag.no.combine ", " "" must.must.must }
      if$
    }
%    \end{macrocode}
% If either the editor or the translator fields were not empty,
% we format the editor or translator name
% instead, and put those details here.
%    \begin{macrocode}
    { editor translator either.or "firstinitial" format.names
%    \end{macrocode}
% We need to append the correct designator, either ``ed.''\ or
% ``trans.''.  The "either.or" function will use the second
% item pushed if both are non-empty, so we take advantage of
% this ``feature'' in making our choice of designators; the
% "ed." or "eds." strings are only used if the "translator"
% field is empty.  And finally, we push a null string to
% round out, and do a compulsory export.
%    \begin{macrocode}
      translator empty$
      { editor num.names$ #1 >
        { " eds.\ " }
        { " ed.\ " }
        if$
      }
      { " trans.\ " }
      if$
      "" must.must.must
    }if$
%    \end{macrocode}
% We also need to indicate the edition, if any.
%    \begin{macrocode}
  edition " ed.\ " "" might.ifone.must
%    \end{macrocode}
% The year itself is easy.  We push the year, do a check to issue
% a warning if necessary, then run format.date over it,
% which yields the year in "theyear", which can be pushed
% back onto the stack.  Then we fill out to six
% \LexiTeX{} fields in
% all, and do a compulsory export.
%    \begin{macrocode}
  type empty.to.null "cmnd" =
  type empty.to.null "command" =
  type empty.to.null "command papers" =
  or or
  { "" }
  { year "mo.dd.yy" format.date
  }if$
  ")}" "" must.must.must
%    \end{macrocode}
% A new line for a new macro, and we're done!  Rejoice!
% On to the next function definition!
%    \begin{macrocode}
  newline$
}
%    \end{macrocode}
%
% \DescribeEnv{incollection}
% This is for those nasty entries that are created when someone
% publishes an article in a collection of essays edited by someone
% else.
%    \begin{macrocode}
FUNCTION{incollection}
{ "\lexibib{incollection}{" cite$ "}{" must.must.must
   get.a.kinda.sort.key "" "}{" must.must.must
  author "lastonly" format.names "author" check
  "}{" "" must.must.must
  title "title" check
  "}{" "" must.must.must
  chapter empty$
    { "\\in \\" }
    { "\\" type empty$
      { "chapter " chapter "chapter" check " of \\" * * * }
      { type " " chapter "chapter" check " of \\" * * * * 
      }if$
    }if$
  booktitle "booktitle" check
  "}{" must.must.must
  pages "short" format.pages "pages" check
  "}{" "(" must.must.must
% We're now in the final ``field'' of the \LexiTeX{} entry.
% The coding here is the same as for a "book" entry; the
% reader is referred to that entry for the commentary
% on the following code.
%    \begin{macrocode}
    editor empty$ translator empty$ and
    { series empty$
      'skip$
      { series " No.~" number
      "endlabel" field.tag.no.combine ", " "" must.must.must }
      if$
    }
    { editor booktranslator either.or "firstinitial" format.names
      booktranslator empty$
      { editor num.names$ #1 >
        { " eds.\ " }
        { " ed.\ " }
        if$
      }
      { " trans.\ " }
      if$
      "" must.must.must
    }if$
  edition " ed.\ " "" might.ifone.must
  year "mo.dd.yy" format.date ")}" "" must.must.must
  newline$
}
%    \end{macrocode}
% \DescribeEnv{inbook}
% There is significant massaging of the use to which
% "title" and "booktitle" is put here.  If both occur,
% but are different, the form is ``"title" in "booktitle"'',
% or ``"title", chapter X of "booktitle"''.
% If both occur, but are the same, the treatment depends
% on whether "chapter" occurs.  If it does not occur,
% the style complains --- an "@inbook" entry should always
% refer to a chapter or other specific unit if the
% "title" and "booktitle" are the same.  If it does occur,
% then the chapter or other specifier is treated as a
% volume number inserted as an option to "\source" using "v=",
% the "title" being inserted as the title of the work.
%    \begin{macrocode}
FUNCTION{inbook}
{ "\lexibib{inbook}{" cite$ "}{" must.must.must
  get.a.kinda.sort.key "" "}{" must.must.must
  author "firstinitial" format.names "author" check
  "}{" "" must.must.must
  title booktitle =
    { chapter empty$
      { "no chapter or separate title for inbook entry " cite$ * warning$
      }
      { "{\rm " "Chapter" type either.or.nowarning 
        " " chapter " of} " * * * *
      }if$
    }
    { ""
    }if$
  title "title" check
  "}{" must.must.must
  title booktitle =
    { "" }
    { chapter empty$
      { "{\rm in} " }
      { type empty$
        { "{\rm Chapter " chapter " of} " * * }
        { "{\rm " type " " chapter " of} " * * * *
        }if$
      }if$
    }if$
  booktitle "booktitle" check
  "}{" must.must.must
  pages "short" format.pages "pages" check
  "}{" "(" must.must.must
% We're now in the final ``field'' of the \LexiTeX{} entry.
% The coding here is the same as for a "book" entry; the
% reader is referred to that entry for the commentary
% on the following code.
%    \begin{macrocode}
    translator empty$
    { series empty$
      'skip$
      { series " No.~" number
      "endlabel" field.tag.no.combine ", " "" must.must.must }
      if$
    }
    { translator "firstinitial" format.names
      " trans.\ "
      "" must.must.must
    }if$
  edition " ed.\ " "" might.ifone.must
  year "mo.dd.yy" format.date ")" "}" must.must.must
  newline$
}
%    \end{macrocode}
%
%    \begin{macrocode}
FUNCTION{booklet}
{ "\lexibib{booklet}{" cite$ "}{" must.must.must
  get.a.kinda.sort.key "" "}{" must.must.must
  author "full" format.names "}{" "" must.must.must
  "\\" title "\\}{}{}{(" must.must.must
  howpublished ", " "" might.ifone.must
  year "mo.dd.yy" format.date
  ")}" "" must.must.must
  newline$
}
%    \end{macrocode}
% \changes{1994/12/12}{1.0g}{Added the `techreport' function,
% to support draft article sent to Pedro Aphalo for comments.}
%    \begin{macrocode}
FUNCTION {techreport}
{"\lexibib{techreport}{" cite$ "}{" must.must.must
  get.a.kinda.sort.key "" "}{" must.must.must
  institution author either.or.nowarning
  "full" format.names "author & institution" check
  "}{" title "title" check must.must.must
  "}{}{}{" "(" "" must.must.must
  author empty$
    'skip$
    {institution "\ "  "" might.ifone.must
    }if$
  type empty$
    { "Technical report" }
    { type
    }if$
  type empty.to.null "Cmnd" =
  { "\ " }
  { " No.~" 
  }if$
  number "endlabel" field.tag.no.combine
  ", " "" must.must.must
  year "mo.dd.yy" format.date
  "" ")}" must.must.must
  newline$
  }
%    \end{macrocode}
%    \begin{macrocode}
FUNCTION {mastersthesis}
{"\lexibib{mastersthesis}{" cite$ "}{" must.must.must
   get.a.kinda.sort.key "}{" "" must.must.must
  author
  "full" format.names "author" check
  "}{" title "title" check must.must.must
  "}{}{}{("
  type empty$
    { "Master's Thesis" }
    { type
    }if$
  ", " must.must.must
  institution "institution" check ", " "" might.ifone.must
  year "mo.dd.yy" format.date
  ")" "}" must.must.must
  newline$
  }
%    \end{macrocode}
%
% \paragraph{Cases} Law cases are all entered using the
% {\tt @CASE} entry type.  The formatting of citations
% varies from jurisdiction to jurisdiction, so the behaviour
% of citations of this type is controlled via a "jurisdiction"
% field.  Supported jurisdictions are listed in the user
% guide.  Below, the functions for each jurisdiction are
% defined first, followed by the "case" function itself.
%
%    \begin{macrocode}
FUNCTION {clear.cite.vars}
 { 
"" 'volume.var :=
"" 'number.var :=
"" 'journal.var :=
"" 'pages.var :=
"" 'year.var :=
 }
FUNCTION {case}
%    \end{macrocode}
% Like the "article" entry, the "case" entry must make some
% pretty fine decisions about how to format material
% fed to it.  As a consequence, it is complex, but similar
% to the "article" entry in many particulars.
%    \begin{macrocode}
{ cites empty.to.null "=" *
  journal empty$
  { parse.one.cite }
  { volume empty.to.null 'volume.var :=
    number empty.to.null 'number.var :=
    journal 'journal.var :=
    pages empty.to.null 'pages.var :=
    year empty.to.null 'year.var :=
  }if$
  volume.var empty$ not number.var empty$ not and
  { "\bridges{,\ p.~}{,\ }{,\ }{\ }{\ at~}{\ at~}"
    "" "" must.must.must
    newline$ }
  'skip$
  if$
 "\lexibib{case}{" cite$ "}{" must.must.must
  get.a.kinda.sort.key "" "}{}{" must.must.must
  title empty$
  {  "Decision of the " court "court" check "" must.must.must
      " (" j.format.division ")" iftwo.might.iftwo
      ", " "" "" must.must.must
      casedate "month.dd.yy" format.date "" "" must.must.must
      "}{" "" "" must.must.must
  }
  { title "}{" "" must.must.must }
  if$
  volume.var empty$ not number.var empty$ not and
  { journal.var "journal" check ", v.~" volume.var must.must.must
    ", n.~" number.var "}{" must.must.must }
  { volume.var empty$
    { "[" year.var "mo.dd.yy" format.date "] " iftwo.might.iftwo
      number.var "\ " journal.var "journal" check might.ifone.must
      "}{" "" "" must.must.must }
    { volume.var "\ " journal.var "journal" check empty.to.null
      might.ifone.must
      "}{" "" "" must.must.must
    }if$
  }if$
pages.var "short" format.pages "pages" check
"}{"
"("
must.must.must
volume.var empty$
  'skip$
  { year.var "mo.dd.yy" format.date "" "" must.must.must
  }if$
crossref empty$
{ ")}" "" "" must.must.must }
{ booktitle empty$
  { "no booktitle (name of special issue) for "
     cite$ "/" crossref * * * warning$
    ")}" "" "" must.must.must }
  { volume.var empty$ {""} {") ("} if$ booktitle "}" must.must.must}
  if$
}if$
{ duplicate$ "=" = not }
{ parse.one.cite
  "={" "" "" must.must.must
  volume.var empty$ not number.var empty$ not and
  { journal.var "journal" check ", v.~" volume.var must.must.must
    ", n.~" number.var "}{" must.must.must }
  { volume.var empty$
    { "[" year.var "mo.dd.yy" format.date "] " iftwo.might.iftwo
      number.var "\ " journal.var "journal" check might.ifone.must
      "}{" "" "" must.must.must }
    { volume.var "\ " journal.var "journal" check empty.to.null
      might.ifone.must
      "}{" "" "" must.must.must
    }if$
  }if$
pages.var "short" format.pages "pages" check
"}{"
""
must.must.must
volume.var empty$
  'skip$
  { "(" year.var "mo.dd.yy" format.date ")" must.must.must
  }if$
crossref empty$
{ "}" "" "" must.must.must }
{ booktitle empty$
  { "no booktitle (name of special issue) for "
     cite$ "/" crossref * * * warning$
    "}" "" "" must.must.must }
  { volume.var empty$ {""} {") ("} if$ booktitle "}" must.must.must}
  if$
}if$
}while$
pop$
clear.cite.vars
newline$
annote empty.to.null write$ newline$
}
%    \end{macrocode}
% The following item adds annotations; this may be eliminated by
% stripping with "noannotes".
%    \begin{macrocode}
%<*!noannotes>
% annote empty.to.null write$ newline$
%</!noannotes>
% }
%    \end{macrocode}
%
% \DescribeEnv{j.statute}
% This function applies to Japanese statutory materials.
% \changes{1994/08/03}{v0.1c}{Added support for Japanese statutes.}
%    \begin{macrocode}
FUNCTION {j.statute}
 { "\lexibib{jstatute}{" cite$ "}{" must.must.must
   get.a.kinda.sort.key "" "}{}{" must.must.must
   title "title" check empty.to.null "}{}{}{" "" must.must.must
   title empty$ {""} {"("} if$
   "" "" must.must.must
   number empty$
     'skip$
     { type empty$
       { "Law" }
       { type
       }if$
       "\ no.~" *
       number "number" check " of "
       iftwo.might.iftwo
     }
     if$
   year "yy" format.date 
   title empty$ {""} {")"} if$
   "}" must.must.must
   newline$
 }
%    \end{macrocode}
%
% \DescribeEnv{s.statute}
% This function formats a statute entry for Singapore.
%    \begin{macrocode}
FUNCTION { s.statute }
{ "\lexibib{statute}{" cite$ "}{" must.must.must
   get.a.kinda.sort.key "" "}{}{" must.must.must
  title "title" check ", No.~" number
  "endlabel" field.tag.no.combine
  number empty$
    { "\ " * }
    { "\ of " * }
    if$
  year "mo.dd.yy" format.date "}{}{}{}" must.must.must
  newline$
}
%    \end{macrocode}
%    \begin{macrocode}
FUNCTION { e.statute }
{ "\lexibib{statute}{" cite$ "}{" must.must.must
   get.a.kinda.sort.key "" "}{}{" must.must.must
  title "title" check "\ " 
  year "mo.dd.yy" format.date must.must.must
  "}{}{}{}" "" "" must.must.must
  newline$
}
%    \end{macrocode}
%
% \DescribeEnv{statute}
% This function selects the correct statute entry function.
%    \begin{macrocode}
FUNCTION { statute }
 { type build.bridges
   jurisdiction empty.to.null duplicate$
   "japan" =
   { pop$ j.statute }
   { duplicate$ "singapore" =
     { pop$ s.statute }
     { duplicate$ "england" =
       { pop$ e.statute }
       { pop$ "IMPORTANT: unknown jurisdiction for " cite$ * warning$
       }if$
     }if$
   }if$
 }
%    \end{macrocode}
%
% \DescribeEnv{unpublished}
%
% A hastily drafted function type for unpublished materials.
%
%    \begin{macrocode}
FUNCTION {unpublished}
{ "\lexibib{book}{" cite$ "}{" must.must.must
   get.a.kinda.sort.key "" "}{" must.must.must
  author "firstinitial" format.names "}{" "" must.must.must
  title "title" check "" ""  must.must.must
  "}{}{}{" "(" "" must.must.must
   note
  ")}" "" must.must.must
  newline$
}
%    \end{macrocode}
%
% \DescribeEnv{default.type}
% We use the "book" type as our default type.  When "manual" is
% completed, we should probably use that type instead.
%    \begin{macrocode}
FUNCTION {default.type} { book }
%</bstfunctions>
%    \end{macrocode}
%
% 
% \subsection{Macro definitions}
% We don't define any macros for abbreviating law journal
% names.  Instead, we will use Blue Book abbreviations
% ``native'', with a special syntax (probably the full
% form in syntax: 
% ``"\gobble{Accountant}{}"'' immediately after
% the abbreviation) for resolving
% ambiguous abbreviations.  Meanwhile, trust me: use the
% Blue Book abbreviations and take this upcoming facility
% on faith.  And besides, do you ever {\em need\/} to spell
% out journal and reporter names?
%
% \subsection{Execution}
% With all preliminaries out of the way, our first act is
% to read in the entries from {\tt *.bib}..
%    \begin{macrocode}
%<*bstfunctions>
READ
%    \end{macrocode}
% Then we say ``Hi'' to the user.  It would be nice to make this
% the first message, but the structure of \BibTeX{} style files
% dictates that it will follow any warnings about missing
% entries.
%    \begin{macrocode}
EXECUTE {hello}
%</bstfunctions>
%    \end{macrocode}
% \section{Camel style code}
%    \begin{macrocode}
%<*lawcitestyle>
\ProvidesFile{law.cst}[1995/01/08]
%    \end{macrocode}
% \subsection{Word list}
% A list of inter-words and their corresponding expansions
% must be provided.
%    \begin{macrocode}
{\catcode`\_=13%
 \catcode`\^=13%
\gdef\@law@wordlist{%
  \\{\item}{\item}%
  \\{and}{_\ ^And }%
  \\{but-see}{_\ ^But see }%
  \\{,}{_; }%
  \\{;}{_; }%
  \\{:}{_; }%
  \\{eg}{_; ^E.g.~}%
  \\{accord}{_; ^Accord }%
  \\{see}{_; ^See }%
  \\{see-also}{_; ^See also }%
  \\{cited-in}{ cited in }%
  \\{citing}{ citing }%
  \\{cf}{_; ^Cf.~}%
  \\{compare}{_; ^Compare }%
  \\{with}{ with }%
  \\{contra}{_; ^Contra }%
  \\{but-cf}{_\ ^But cf.~}%
  \\{see-generally}{_. See generally }%
  \\{affirmed}{_, ^aff'd }%
  \\{reprinted-in}{_; ^Reprinted in }%
  \@law@nomatch}
}
%    \end{macrocode}
%
% \subsection{Print routines}
% Every ".cst" file must define a "\@law@print" macro.
% This is the macro that prints the citation, both in the
% text of the document and in the bibliography.
% The toggles set by the {\sc Camel} engine allow
% a high degree of refinement in the formatting of
% citations.  For the styles currently used in
% publishing with \LaTeX{}, relatively little of this
% power is required.  In-document citation styles
% are much more demanding; if you digest the operation
% the following "\@law@print" macro, which is suitable for
% formatting legal citations, you will find the drafting
% of author-date styles and the like quite straightforward
% by comparison.
%    \begin{macrocode}
\gdef\@law@print{%
%    \end{macrocode}
% In this particular style, a long citation (for the
% bibliography, for example) and a first in-text citation
% are identical.  These forms have separate toggles, so
% we start by equating them.
%    \begin{macrocode}
  \xdef\@law@argtwolist{\the\@ltok@argtwo}%
  \if@law@firstuseofcite\@law@longcitetrue\fi%
%    \end{macrocode}
% This toggles the printing on and off.  This toggle is
% set by the "n" option to the "\source" command.
%    \begin{macrocode}
\if@law@printcite%
  \begingroup%
  \def\@law@firstslash{\begingroup\def\\{\@law@secondslash}%
   \the\ltokspecialface}%
  \def\@law@secondslash{\endgroup\def\\{\@law@firstslash}}%
  \def\\{\@law@firstslash}%
%    \end{macrocode}
% There are two halves to the macro; one for long cite forms,
% the other for short.  Long cite forms are in the first
% half.
%    \begin{macrocode}
\if@law@longcite%
%    \end{macrocode}
% Long citations are pretty straightforward; we've gathered all
% the information we needed; now we just need to plunk it
% all down in order, pretty much.
% First to print is the author field (the second argument to the
% citation declaration command), followed by the
% author-to-title punctuation bridge.  The enclosing
% braces limit the scope of the special active character
% definitions of "^", "_" and "|".
%    \begin{macrocode}
 \global\ltokspecialface=\@ltok@authoroptionface%
 {\the\@ltok@authormainface%
 \@law@barinfull\the\@ltok@author}\the\@ltok@atot%
%    \end{macrocode}
%  Next comes similar treatment for the title.
%    \begin{macrocode}
 \global\ltokspecialface=\@ltok@titleoptionface%
 {\the\@ltok@titlemainface%
 \@law@barinfull\the\@ltok@name}\the\@ltok@ttocone%
%    \end{macrocode}
%    \begin{macrocode}
  \@law@longrecurse%
%    \end{macrocode}
% This else marks the boundary between long-form citations
% (which we have seen are relatively simple to print)
% and short-form citations (which are rather complex).
% This "\else" matches the "\if@law@longcite" conditional.
%    \begin{macrocode}
\else%
%    \end{macrocode}
% The footnote number and the accompanying bridge
% should not appear in a short-form citation if
% the citation being printed first occurred in the
% current footnote.  Rather than suppressing printing,
% we just set the relevant tokens to nil.
%    \begin{macrocode}
  \ifnum\the\@ltok@pageorfootno=\the\c@law@footnote\relax%
    \xdef\@law@temp{\the\@ltok@whereitsat}%
    \xdef\@law@temptwo{\the\@ltok@infoot}%
    \ifx\@law@temp\@law@temptwo%
     \global\@ltok@whereitsat{}%
     \global\@ltok@pageorfootno{}%
    \fi%
  \fi%
%    \end{macrocode}
% Special forms of tidying-up may be appropriate to
% each of the four classes of citation.  An appropriate
% cleaning macro is called here.
%    \begin{macrocode}
\csname @law@\the\@ltok@citetype preformat\endcsname%
%    \end{macrocode}
% If it has been found that the same citation has been
% used immediately previous to this instance,
% we use {\em Id.}  If the "\@justabove" test showed
% that {\em Id.} by itself is appropriate, we prepare
% to eat any period immediately following the citation
% macro.  Otherwise, we tack on the pinpoint reference
% and its accompanying bridge.  A fail-safe conditional
% reverts to the original plan if the pinpoint reference
% is empty.
%    \begin{macrocode}
  \if@justabove%
   {\def\,{,}%
    \Id%
     \if@l@quiteexact%
      \gdef\@law@gobble{\@ifnextchar.{\@gobble}{}}%
     \else%
      \ifcat$\the\@ltok@argtwo$%
        \gdef\@law@gobble{\@ifnextchar.{\@gobble}{}}%
      \else%

        \the\@ltok@atbridge{\@law@barkill\the\@ltok@argtwo\relax}%
        \@law@fetchparas%
        \@law@shiftparas%
        \@law@shortrecurse%
      \fi%
     \fi}%
%    \end{macrocode}
% If {\em Id.} was not called for, we have to create
% a proper short-form reference.  This "\else" corresponds
% with the "\@justabove" toggle, above.
%    \begin{macrocode}
  \else%
%    \end{macrocode}
% The author will be used in any case.  Note that the author
% and its following bridge may have been set to nil, above.
%    \begin{macrocode}
   \global\ltokspecialface=\@ltok@authoroptionface%
   {\the\@ltok@authormainface%
%  print whatever there if for the author
   \@law@barinshort\the\@ltok@author}\the\@ltok@atot%
%    \end{macrocode}
% The title information is also used, if it is present.
%    \begin{macrocode}
   \global\ltokspecialface=\@ltok@titleoptionface%
   {\the\@ltok@titlemainface%
%    \end{macrocode}
% If we have specified a nickname for this cite for
% `hereinafter' references, we use it instead of the
% title of the work.
%<*parabeta>
[Some day this might be hooked up.]
%</parabeta>
%    \begin{macrocode}
   \@law@barinshort\the\@ltok@name}%
%    \end{macrocode}
% At this point, we must make decisions concerning whether
% to use the {\em supra\/} cross-referencing form.  This
% depends on the type of citation we are working on.
% The "\@nosupra" condition is set to true if we are working
% on a case or a statute, otherwise it is set to false.
%    \begin{macrocode}
   \xdef\@law@temp{\the\@ltok@citetype}%
    \ifx\@law@temp\@law@case%
     \@nosupratrue%
    \else%
      \ifx\@law@temp\@law@statute%
       \@nosupratrue%
      \else%
       \@nosuprafalse%
      \fi%
    \fi%
%    \end{macrocode}
% Now we put the result of the above test to work.
% First, the citation form for cases or statutes, which
% do not permit the {\em supra\/} form.
%    \begin{macrocode}
    \if@nosupra%
%    \end{macrocode}
%    \begin{macrocode}
     \the\@ltok@ttocone%
     \@law@shortprint%
     \@law@shortrecurse%
    \else%
     \supra\the\@ltok@whereitsat\the\@ltok@pageorfootno%
     \the\@ltok@atbridge%
     {\def\,{,}%
     {\@law@barkill\the\@ltok@argtwo\relax}}%
    \fi%
%    \end{macrocode}
% In order, these close "\if@justabove", "\if@law@longcite"
% and "\if@law@printcite".
%    \begin{macrocode}
     \xdef\@law@temp{\the\@ltok@citetype}%
      \ifx\@law@temp\@law@statute%
       \ifcat$\the\@ltok@citefirst$%
         \the\@ltok@ptoctwo{\@law@barinshort\the\@ltok@citelast}%
       \fi%
      \fi%
  \fi%
\fi%
\endgroup%
\fi%  <- end of if@law@printcite
%    \end{macrocode}
% The "\@law@gobble" here will consume a period inserted
% automatically after a forced footnote, if the printed
% form turned out to be an {\em Id.}.  See the concluding
% code of "\@law@setup".
% \changes{1994/07/26}{2.0c}{Moved {\tt\protect\string%
% \protect\@law@forcingfalse} outside of a conditional
% expression at the end of the print routine, to correct
% the failure of the forcing mechanism to work more than once.}
% \changes{1994/12/14}{2.0l}{Fairly drastic simplification
% of the forcing mechanism, in the course of providing for
% unified handling of citation strings.  Code now much
% easier to follow.}
% \changes{1996/03/09}{0.1c}{Corrected a fatal bug in the
% parallel pinpointing mechanmism.  This required changes to both
% Camel and the Law module; the list macro of pinpoints was
% being parsed in both!}
%    \begin{macrocode}
\global\@law@longcitefalse\gdef\volno{\message{(No volume %
    for \the\@ltok@nickname)}}}%
%    \end{macrocode}
% \begin{macro}{\@law@longrecurse}
% This macro prints the tail end of long citations until
% parallels are exhausted.
%    \begin{macrocode}
\def\@law@longrecurse{%
  \@law@longprint%
  \ifnum\the\c@law@parapin>0\relax%
    \loop%
      \addtocounter{law@paracounter}{1}%
    \ifnum\the\c@law@paracounter<\the\c@law@parapin\relax%
      ; \@law@longprint%
    \repeat%
  \else%
    \loop%
      \addtocounter{law@paracounter}{1}%
    \ifnum\the\c@law@paracounter>\the\c@law@paranormal\relax%
    \else%
      ; \@law@longprint%
    \repeat%
  \fi%
%    \end{macrocode}
%
% \textit{Finally, we print the last date item, which might
% be blank if it's already been printed inside "\@law@longprint".}
% \changes{1997/01/09}{v2.0m}{Fixed loss of citelast in parallel citations.}
%
%    \begin{macrocode}
  \@law@temp}
\def\@law@shortrecurse{%
  \ifnum\the\c@law@parapin>0\relax%
    \loop%
      \addtocounter{law@paracounter}{1}%
    \ifnum\the\c@law@paracounter<\the\c@law@parapin\relax%
      \@law@pincut\@ltok@argtwo\frompinlist%
      ; \@law@shortprint%
    \repeat%
  \else%
    \loop%
      \addtocounter{law@paracounter}{1}%
    \ifnum\the\c@law@paracounter>\the\c@law@paranormal\relax%
    \else%
      ; \@law@shortprint%
    \repeat%
  \fi}
%    \end{macrocode}
% \end{macro}
%
% \begin{macro}{\@law@longprint}
%    \begin{macrocode}
\def\@law@longprint{%
  \begingroup%
  \@law@fetchparas%
  \@law@tidybridges%
%    \end{macrocode}
% If for some reason either the pinpoint argument is nil,
% or it {\em and\/} the location page token register is nil,
% we set the accompanying bridges to nil. 
%    \begin{macrocode}
  \ifcat$\the\@ltok@argtwo$%
   \global\@ltok@ptop{}%
   \global\@ltok@atbridge{}%
   \ifcat$\the\@ltok@citepage$%
     \global\@ltok@conetop{}%
    \fi%
   \fi%
%    \end{macrocode}
%    \begin{macrocode}
 \global\ltokspecialface=\@ltok@citefirstoptionface%
 {\the\@ltok@citefirstmainface%
 \@law@barinfull\the\@ltok@citefirst}%
 \@law@conetopsetup%
 \the\@ltok@conetop%
%    \end{macrocode}
% Next comes the location page and its following 
% bridge.  Both of these may be blank.  No funny
% business with the shorthand active characters is
% required, since we assume this will not contain
% any special text for which they will be required.
%    \begin{macrocode}
 \the\@ltok@citepage\the\@ltok@ptop%
%    \end{macrocode}
% A special use of "\," is defined before we expand
% the optional argument stuff.
%    \begin{macrocode}
  {\def\,{,}\@law@barkill%
   \expandafter\the\@ltok@argtwo%
   \relax}%
%    \end{macrocode}
% \textit{And finally, we print the final portion of the citation.
%  We package the citation part together with its preceding bridge.
%  If it is printed, we blank it; this gives us a simple command that
%  we can put at the end of the recursive routine that invokes this
%  code, so that the tail end of a string cite will be printed if
%  necessary, and only once.}
%  \changes{1997/01/09}{v2.0m}{Modified current-and-last comparison
%   algorithm for the last portion of the cite in parallel citations.}
%
%    \begin{macrocode}
  \xdef\@law@temp{%
    \the\@ltok@ptoctwo%
    \noexpand\@law@barinfull%
    \the\@ltok@citelast}%
  \xdef\@law@temptwo{%
    \the\@ltok@ptoctwo%
    \noexpand\@law@barinfull%
    \the\@ltok@usercitelast}%
  \ifx\@law@temp\@law@temptwo%
  \else%
    {\@law@temp}%
    \gdef\@law@temp{}%
  \fi%
    \@law@shiftparas%
  \endgroup}%
%    \end{macrocode}
% \end{macro}
%
% \begin{macro}{\@law@shortprint}
%    \begin{macrocode}
\def\@law@shortprint{%
% If {\sc Camel} sees more than one pinpoint page or section,
% the bridges preceding
% the page references must be set to their plural form.
% This change applies to both long and short form citations.
% It has to be cloned in each, however, because the decision has
% to be taken with respect to each cite in a string of parallels.
%    \begin{macrocode}
  \begingroup%
  \@law@fetchparas%
  \@law@tidybridges%
%    \end{macrocode}
% The actual recursive printing works.
%    \begin{macrocode}
     \global\ltokspecialface=\@ltok@citefirstoptionface%
     {\the\@ltok@citefirstmainface%
     \@law@barinshort\the\@ltok@citefirst}%
%    \end{macrocode}
% If the pinpoint reference is empty, we tack on the
% location page after the appropriate bridge.  Otherwise,
% we do nothing for the present; any pinpoint reference
% will be produced later on.
%  Note that the bridge
% used here will only be in plural form if something was
% given for use as a pinpoint reference.
% NB: THIS IS A TEST!!!! ******** BRACKETS TO BE PULLED!!!!
%    \begin{macrocode}
     \@law@conetopsetup%
     \the\@ltok@conetop%
     \ifcat$\the\@ltok@argtwo$%
       \the\@ltok@citepage%
     \else%
       {\@law@barkill\the\@ltok@argtwo}%
     \fi%
    \@law@shiftparas%
  \endgroup}%
%    \end{macrocode}
% \end{macro}
%
% \paragraph{Print format subroutines}
% \begin{macro}{\@law@casepreformat}
% This is empty; no special preparations are necessary
% for the printing of a case citation.
%    \begin{macrocode}
\gdef\@law@casepreformat{}%
%    \end{macrocode}
% \end{macro}
% \begin{macro}{\@law@statutepreformat}
% If short statutory references are in force,
% the title and the following bridge are set to nil,
% if some reference is provided in the first citation
% part.
%    \begin{macrocode}
\gdef\@law@statutepreformat{%
\if@law@statuteverbose%
\else%
  \ifcat$\the\@ltok@citefirst$%
  \else%
   \global\@ltok@name{}%
   \global\@ltok@ttocone{}%
  \fi% 
\fi%
}%
%    \end{macrocode}
% \end{macro}
% \begin{macro}{\@law@articlepreformat}
% If a work by the identical author has been cited more than
% once, we leave
% everything intact (the print routine does the necessary
% culling);
% otherwise, we cut the title here.  The use of
% "\@law@authortracing"
% is explained above.
%    \begin{macrocode}
\gdef\@law@articlepreformat{%
\ifcat$\the\@ltok@author$%
\else%
{\@law@clean\@ltok@author\@law@authortracing%
\expandafter\expandafter\expandafter\if\expandafter%
\csname\@law@authortracing\endcsname2%
 \else%
  \global\@ltok@atot{}\global\@ltok@name{}%
\fi}\fi}%
%    \end{macrocode}
% \end{macro}
% \begin{macro}{\@law@bookpreformat}
% This provides the same treatment given to articles.
%    \begin{macrocode}
\gdef\@law@bookpreformat{%
\ifcat$\the\@ltok@author$%
\else%
{\@law@clean\@ltok@author\@law@authortracing%
\expandafter\expandafter\expandafter\if\expandafter%
\csname\@law@authortracing\endcsname2%
 \else%
  \global\@ltok@atot{}\global\@ltok@name{}%
 \fi}\fi}%
%</lawcitestyle>
%    \end{macrocode}
% \end{macro}
%
% \subsection{Camel citation formats}
% The style definitions are stored in a separate file,
% to make it easier and less risky for users to play with
% the styles to produce desired output.  Table 3
% provides a guide to the macro arguments and their functions.
% \subsubsection{Books} The styles used for citing books and
% book-like things follow.
%    \begin{macrocode}
%<*lawcite>
\ProvidesFile{law.cit}[1994/12/07]
\newcitestyle{book}%
 {srsrrrB}
 {[a],\ [t][c]\ [p](pl)\ [rp]\ [e]:[id]\ at~(pl)\ at~[xrf]}
%
\newcitestyle{booklet}%
 {riririB}
 {[a],\ [t][c]\ [p](pl)\ [rp]\ [e]:[id]\ at~(pl)\ at~[xrf]}
%
\newcitestyle{techreport}%
 {riririB}
 {[a],\ [t][c]\ [p](pl)\ [rp]\ [e]:[id]\ at~(pl)\ at~[xrf]}
%
\newcitestyle{mastersthesis}%
 {riririB}
 {[a],\ [t][c]\ [p](pl)\ [rp]\ [e]:[id]\ at~(pl)\ at~[xrf]}
%    \end{macrocode}
% \subsubsection{Articles} Styles used for articles 
% and similar shortish things follow.  They're all
% the same, but the clones help keep things clear when following
% the action between \LaTeX\ and \BibTeX.
%    \begin{macrocode}
\newcitestyle{article}%
 {rsirsrA}
 {[a],\ [t],\ [c]\ [p],\ (pl),\ [rp]\ [e]:[id]\ at~(pl)\
at~[xrf]}
\newcitestyle{incollection}%
 {rsirsrA}
 {[a],\ [t],\ [c]\ [p],\ (pl),\ [rp]\ [e]:[id]\ at~(pl)\
at~[xrf]}
\newcitestyle{inbook}%
 {srirsrA}
 {[a],\ [t],\ [c]\ [p],\ (pl),\ [rp]\ [e]:[id]\ at~(pl)\
at~[xrf]}
%    \end{macrocode}
% \subsubsection{Cases} There are several styles for cases; we pretty
% much need a separate style for each jurisdiction.
%    \begin{macrocode}
\newcitestyle{case}%
 {rrirrsC}
 {[a][t],\ [c]\ [p],\ (pl),\ [rp]\ [e]:[id]\ at~(pl)\ at~[xrf]}
%    \end{macrocode}
% \subsubsection{Statutes} There are several of these.  More may
% need to be added on an {\em ad hoc\/} basis.
%    \begin{macrocode}
\newcitestyle{statute}%
 {rrrsrsS}
 {[a][t],\ [c]\ \S~[p](pl)\ \S\S~[rp]\ [e]:[id]\ \S~(pl)\
\S\S~[xrf]}
%
\newcitestyle{jstatute}%
 {rrrsrsS}
 {[a][t][c]\ \S~[p](pl)\ \S\S~[rp]\ [e]:[id]\ \S~(pl)\
\S\S~[xrf]}
%</lawcite>
%    \end{macrocode}
%
% \section{Extraction utilities}
% \subsection{The Driver}
% Here is a simple driver for extracting the files in the
% package.
% \changes{1997/01/09}{v2.0m}{Brought installation file up to
%   date for use with the new docstrip utility.}
%
%    \begin{macrocode}
%<*installer>
\def\batchfile{law.ins}
\input docstrip.tex
\preamble

Copyright (C) 1992--97 Frank Bennett, Jr.
All rights reserved.

This file is part of the Law module for the Camel package.
\endpreamble

\def\batchfile{camel.dst}      % ignored in distribution
\input docstrip.tex              % ignored in distribution

\keepsilent

\preamble
This file is part of the Law module of the Camel package.
---------------------------------------------------------
This is a generated file.  
IMPORTANT NOTICE:

You are allowed to change this file, subject to the following
conditions.  Under any circumstances, new macro definitions
should not be added to this file.  You are welcome to modify
the macro definitions contained in this file for your own
use.  If you pass a copy of the modified version to someone
else, you should (a) let me know about the change on
fb@soas.ac.uk, and (b) put a note of the changes and of your
own contact details in the file.  Furthermore, you must
acknowledge Camel and its author(s) in the new file (if it
is distributed to others), and you must attach these same
conditions to the new file.

You are not allowed to distribute this file alone.  You are not
allowed to take money for the distribution or use of this file
(or a changed version) except for a nominal charge for copying
etc.

You are allowed to distribute this file under the condition that
it is distributed with all of its contents, intact.

For error reports, or offers to help make this a more powerful,
friendlier, and better package, please contact me on
`fb' at soas.ac.uk

\endpreamble


\generate{\file{law.cst} {\from{law.dtx}{lawcitestyle}}
          \file{law.cit} {\from{law.dtx}{lawcite}}}


\postamble
\endpostamble

%<installer>% Change eng to jse below for JBibTeX support

\generate{\file{law.bst} {\from{camel.dtx}{bstheader,eng}
                          \from{law.dtx}{bstheader,eng}
                          \from{camel.dtx}{bstlibrary,eng}
                          \from{law.dtx}{bstfunctions,eng}
                          \from{camel.dtx}{bsttrailer,eng}}}

\keepsilent


\ifToplevel{
\Msg{***********************************************************}
\Msg{*}
\Msg{* To finish the installation you have to move the following}
\Msg{* file into a directory searched by TeX:}
\Msg{*}
\Msg{* \space\space law.cst}
\Msg{* \space\space law.cit}
\Msg{*}
\Msg{* You should also move the following file into a directory}
\Msg{* searched for style files by BibTeX:}
\Msg{*}
\Msg{* \space\space law.bst}
\Msg{*}
\Msg{* Other style modules can be found on CTAN in the `styles'}
\Msg{* subdirectory below Camel itself.}
\Msg{*}
\Msg{***********************************************************}
}
%</installer>
%    \end{macrocode}
% \Finale \PrintIndex \PrintChanges

% \subsection{The Faculty of Economics and Administration}
% % \file{style/mu/fithesis-econ.sty}
% This is the style file for the theses written at the Faculty of
% Economics and Administration at the Masaryk University in Brno.
% It has been prepared in accordance with the formal requirements
% \changes{v0.3.46}{2017/06/02}{The documentation now points to the
%   2/2017 dean's directive for the Faculty of Economics and
%   Administration, Masaryk University, Brno. [VN]}
% published at the website of the faculty\footnote{See \url{ht^^A
% tps://is.muni.cz/auth/do/econ/predpisy/smernice/prehled/6715^^A
% 9928/SmerniceDekana2017-c.2-o_zaverecnych_pracich_2017.pdf}}.
%    \begin{macrocode}
\NeedsTeXFormat{LaTeX2e}
\ProvidesPackage{fithesis/style/mu/fithesis-econ}[2017/05/21]
%    \end{macrocode}
% The file defines the color scheme of the respective faculty. Note
% the the color definitions are in RGB, which makes the resulting
% files generally unsuitable for printing.
%    \begin{macrocode}
\thesis@color@setup{
  links={HTML}{F27995},
  tableEmph={HTML}{E8B88B},
  tableOdd={HTML}{F5ECEB},
  tableEven={HTML}{EBD8D5}}
%    \end{macrocode}
% The bibliography support is enabled. The |authoryear| citations
% are used and the bibliography is sorted by name, title, and year.
%    \begin{macrocode}
\thesis@bibliography@setup{
  style=iso-authoryear,
  sorting=nty}
\thesis@bibliography@load
%    \end{macrocode}
% The file loads the following packages:
% \begin{itemize}
%   \item\textsf{tikz} -- Used for dimension arithmetic.
%   \item\textsf{geometry} -- Allows for modifications of the type
%     area dimensions.
%   \item\textsf{array} -- Enables |<{decl.}| and |>{decl.}|
%     declarations in table preambles.
% \end{itemize}
% In addition to this, the type area width is set to
% 16\,cm in accordance with the formal requirements of the faculty.
% This leads to overfull lines and is against the good conscience
% of the author of this style.
%    \begin{macrocode}
\thesis@require{tikz}
\thesis@require{geometry}
\thesis@require{array}
\geometry{top=25mm,bottom=20mm,left=25mm,right=25mm,includeheadfoot}
%    \end{macrocode}
% \begin{macro}{\thesis@blocks@cover}
% The |\thesis@blocks@cover| macro typesets the thesis
% cover.
%    \begin{macrocode}
\def\thesis@blocks@cover{%
  \ifthesis@cover@
    \thesis@blocks@clear
    \begin{alwayssingle}
      \thispagestyle{empty}
      \begin{center}
      {\sc\thesis@titlePage@LARGE\thesis@@{universityName}\\%
          \thesis@titlePage@Large\thesis@@{facultyName}\\}
      \vfill
      {\bf\thesis@titlePage@Huge\thesis@@{typeName}}
      \vfill
      {\thesis@titlePage@large\thesis@place
       \ \thesis@year\hfill\thesis@author}
      \end{center}
    \end{alwayssingle}
  \fi}
%    \end{macrocode}
% \end{macro}
% The style file configures the title page header to include the
% name of the field of study and redefines the title page content
% not to include the author's name and the title page footer
% to include both the author's and advisor's name, the year and
% place of the thesis defense in accordance with the formal
% requirements of the faculty.
%    \begin{macrocode}
\thesis@blocks@titlePage@field@true
\def\thesis@blocks@titlePage@content{%
  {\thesis@titlePage@Huge\bf\thesis@TeXtitle}
  \ifthesis@english\else
    {\\[0.1in]\thesis@titlePage@Large\bf\thesis@TeXtitleEn}
  \fi {\\[0.3in]\thesis@titlePage@large\sc\thesis@@{typeName}\\}}
\def\thesis@blocks@titlePage@footer{%
  {\thesis@titlePage@large
    {% Calculate the width of the thesis author and advisor boxes
     \let\@A\relax\newlength{\@A}\settowidth{\@A}{{%
       \bf\thesis@@{advisorTitle}:}}
     \let\@B\relax\newlength{\@B}\settowidth{\@B}{\thesis@advisor}
     \let\@C\relax\newlength{\@C}\settowidth{\@C}{{%
       \bf\thesis@@{authorTitle}:}}
     \let\@D\relax\newlength{\@D}\settowidth{\@D}{\thesis@author}
    \let\@left\relax\newlength{\@left}\pgfmathsetlength{\@left}{%
      max(\@A,\@B)}
    \let\@right\relax\newlength{\@right}\pgfmathsetlength{\@right}{%
      max(\@C,\@D)}
    % Typeset the thesis author and advisor boxes
    \vskip 2in\begin{minipage}[t]{\@left}
      {\bf\thesis@@{advisorTitle}:}\\\thesis@advisor
    \end{minipage}\hfill\begin{minipage}[t]{\@right}
      {\bf\thesis@@{authorTitle}:}\\\thesis@author
    \end{minipage}}\\[4em]\thesis@place, \thesis@year}}
%    \end{macrocode}
% \begin{macro}{\thesis@blocks@frontMatter}
% The |\thesis@blocks@frontMatter| macro sets up the style
% of the front matter of the thesis. The page numbering is arabic
% as per the formal requirements and it is hidden. In case of
% double-sided typesetting, the geometry is altered according to
% the requirements of the faculty.
% \begin{macrocode}
\def\thesis@blocks@frontMatter{%
  \thesis@blocks@clear
  % In case of double-sided typesetting, change the geometry
  \ifthesis@twoside@
    \newgeometry{top=25mm,bottom=20mm,left=35mm,
      right=15mm, includeheadfoot}
  \fi\pagestyle{empty}
  \parindent 1.5em
  \setcounter{page}{1}
  \pagenumbering{arabic}}
%    \end{macrocode}
% \end{macro}\begin{macro}{\thesis@blocks@mainMatter}
% The |\thesis@blocks@mainMatter| macro sets up the style
% of the main matter of the thesis. The page numbering doesn't
% reset at the beginning of the main thesis as per the formal
% requirements.
% \begin{macrocode}
\def\thesis@blocks@mainMatter{%
  \thesis@blocks@clear
  % In case of double-sided typesetting, change the geometry
  \ifthesis@twoside@
    \newgeometry{top=25mm,bottom=20mm,left=35mm,
      right=15mm, includeheadfoot}
  \fi\pagestyle{thesisheadings}
  \parindent 1.5em\relax}
%    \end{macrocode}
% \end{macro}\begin{macro}{\thesis@blocks@tables}
% The |\thesis@blocks@tables| macro optionally typesets the
% |\listoftables| and |\listoffigures|.
% \begin{macrocode}
\def\thesis@blocks@tables{%
  \thesis@blocks@lot
  \thesis@blocks@lof}
%    \end{macrocode}
% \end{macro}
% If the |nolot| and |nolof| options haven't been specified, the
% |\thesis@blocks@lot| and |\thesis@blocks@lof| macros are
% redefined to create an entry in the table of contents.
% \begin{macrocode}
\ifx\thesis@blocks@lot\relax\else
  \def\thesis@blocks@lot{%
    \thesis@blocks@clear
    \phantomsection
    \addcontentsline{toc}{chapter}{\listtablename}%
    \listoftables}
\fi

\ifx\thesis@blocks@lof\relax\else
  \def\thesis@blocks@lof{%
    \thesis@blocks@clear
    \phantomsection
    \addcontentsline{toc}{chapter}{\listfigurename}%
    \listoffigures}
\fi
%    \end{macrocode}
% \begin{macro}{\thesis@blocks@declaration}
% The |\thesis@blocks@declaration| macro typesets the declaration
% text. Unlike the generic |\thesis@blocks@declaration| macro from
% the \texttt{style/mu/fithesis-sci.sty} file, this definition
% includes the date and a blank line for the author's signature, as
% per the requirements of the faculty.
% \changes{v0.3.46}{2017/06/02}{Redefined
%   \cs{thesis@blocks@declaration} in
%   \texttt{style/mu/fithesis-econ.sty} in accordance with the
%   example documents. The patch was submitted by Jana Ratajská.
%   [VN]}
%    \begin{macrocode}
\def\thesis@blocks@declaration{%
  \begin{alwayssingle}%
    \thesis@blocks@clear
    \leavevmode\vfill
    % Start the new chapter without clearing any page.
    {\let\thesis@blocks@clear\relax
    \chapter*{\thesis@@{declarationTitle}}}%
    \thesis@declaration
    \vskip 2cm%
    {\let\@A\relax\newlength{\@A}
      \settowidth{\@A}{\thesis@@{authorSignature}}
      \setlength{\@A}{\@A+1cm}
    \noindent\thesis@place, \thesis@@{formattedDate}\hfill
    \begin{minipage}[t]{\@A}%
      \centering\rule{\@A}{1pt}\\
      \thesis@@{authorSignature}\par
    \end{minipage}}
  \end{alwayssingle}}
%    \end{macrocode}
% \end{macro}\begin{macro}{\thesis@blocks@abstract}
% \changes{v0.3.46}{2017/06/02}{Redefined
%   \cs{thesis@blocks@abstract}, \cs{thesis@blocks@abstractEn},
%   \cs{thesis@blocks@keywords}, and \cs{thesis@blocks@keywordsEn}
%   in \texttt{style/mu/fithesis-econ.sty} in accordance with the
%   example documents. The patch was submitted by Jana Ratajská.
%   [VN]}
% The |\thesis@blocks@abstract| macro typesets the
% abstract. This definition typesets the abstract on the same page.
% \begin{macrocode}
\def\thesis@blocks@abstract{%
  \begin{alwayssingle}%
    \vskip 40\p@
    {\let\thesis@blocks@clear\relax
    \chapter*{\thesis@@{abstractTitle}}}%
    \noindent\thesis@abstract
  \end{alwayssingle}}
%    \end{macrocode}
% \end{macro}\begin{macro}{\thesis@blocks@abstractEn}
% The |\thesis@blocks@abstractEn| macro typesets the abstract in
% English. If the current locale is English, the macro produces no
% output. This macro typesets the abstract on the same page.
% \begin{macrocode}
\def\thesis@blocks@abstractEn{%
  \ifthesis@english\else
    {\thesis@selectLocale{english}%
    \begin{alwayssingle}%
      \vskip 20\p@
      {\let\thesis@blocks@clear\relax
      \chapter*{\thesis@english@abstractTitle}}%
      \noindent\thesis@abstractEn
    \end{alwayssingle}}%
  \fi}
%    \end{macrocode}
% \end{macro}\begin{macro}{\thesis@blocks@keywords}
% The |\thesis@blocks@keywords| macro typesets the keywords. This
% definition typesets the keywords on the same page.
% \begin{macrocode}
\def\thesis@blocks@keywords{%
  \begin{alwayssingle}%
    \vskip 40\p@
    {\let\thesis@blocks@clear\relax
    \chapter*{\thesis@@{keywordsTitle}}%
    \noindent\thesis@TeXkeywords}%
  \end{alwayssingle}}
%    \end{macrocode}
% \end{macro}\begin{macro}{\thesis@blocks@keywordsEn}
% The |\thesis@blocks@keywordsEn| macro typesets the keywords in
% English. If the current locale is English, the macro produces no
% output.
% \begin{macrocode}
\def\thesis@blocks@keywordsEn{%
  \ifthesis@english\else
    {\thesis@selectLocale{english}%
    \begin{alwayssingle}%
      \vskip 20\p@
      {\let\thesis@blocks@clear\relax%
      \chapter*{\thesis@english@keywordsTitle}}%
      \noindent\thesis@TeXkeywordsEn
    \end{alwayssingle}}%
  \fi}
%    \end{macrocode}
% \end{macro}\begin{macro}{\thesis@blocks@bibEntry}
% The |\thesis@blocks@bibEntry| macro typesets a bibliographical
% entry. Along with the macros required by the locale file
% interface, the locale files need to define the following macros:
% \begin{itemize}
%   \item|\thesis@|\textit{locale}|@bib@author| -- The label of the
%     author name entry
%   \item|\thesis@|\textit{locale}|@bib@title| -- The label of the
%     title name entry
%   \item|\thesis@|\textit{locale}|@bib@titleEn| -- The label of the
%     English title name entry (\cs{thesis@english@bib@titleEn}
%     does not need to be defined)
%   \item|\thesis@|\textit{locale}|@bib@department| -- The label of
%     the department name entry
%   \item|\thesis@|\textit{locale}|@bib@advisor| -- The label of
%     the advisor name entry
%   \item|\thesis@|\textit{locale}|@bib@year| -- The label of the
%     year entry
% \end{itemize}
% \changes{v0.3.46}{2017/06/02}{Defined \cs{thesis@blocks@bibEntry}
%   in \texttt{style/mu/fithesis-econ.sty} in accordance with the
%   example documents. The patch was submitted by Jana Ratajská.
%   [VN]}
%    \begin{macrocode}
\def\thesis@blocks@bibEntry{%
  \thesis@blocks@clear
  \noindent\begin{thesis@newtable@old}{@{}>{\bfseries}ll@{}}
    \thesis@@{bib@author}:        & \thesis@author     \\
    \thesis@@{bib@thesisTitle}:   & \thesis@title      \\
  \ifthesis@english\else
    \thesis@@{bib@thesisTitleEn}: & \thesis@titleEn    \\
  \fi
    \thesis@@{bib@department}:    & \thesis@department \\
    \thesis@@{bib@advisor}:       & \thesis@advisor    \\
    \thesis@@{bib@year}:          & \thesis@year       \\
  \end{thesis@newtable@old}}
%    \end{macrocode}
% \end{macro}
% Note that there is no direct support for the seminar paper and
% thesis proposal types.  If you would like to change the contents
% of the preamble and the postamble, you should modify the
% |\thesis@blocks@preamble| and |\thesis@blocks@postamble| macros.
%
% All blocks within the autolayout preamble that are not defined
% within this file are defined in the
% \texttt{style/mu/fithesis-base.sty} file.
%    \begin{macrocode}
\def\thesis@blocks@preamble{%
  \thesis@blocks@coverMatter
    \thesis@blocks@cover
  \thesis@blocks@frontMatter
    \thesis@blocks@titlePage
    \thesis@blocks@assignment
    \thesis@blocks@bibEntry
    \thesis@blocks@abstract
    \thesis@blocks@abstractEn
    \thesis@blocks@keywords
    \thesis@blocks@keywordsEn
    \thesis@blocks@declaration
    \thesis@blocks@thanks
    \thesis@blocks@toc}
%    \end{macrocode}
% All blocks within the autolayout postamble that are not defined
% within this file are defined in the \texttt{style/mu/base.sty}
% file.
%    \begin{macrocode}
\def\thesis@blocks@postamble{%
  \thesis@blocks@bibliography
  \thesis@blocks@tables}
%    \end{macrocode}

% \subsection{The Faculty of Medicine}
% % \file{theme/mu/beamercolorthemefibeamer-med.sty}
% This is the color theme for presentations written at the Faculty
% of Medicine at the Masaryk University in Brno. This theme has no
% effect outside the presentation mode.
%    \begin{macrocode}
\NeedsTeXFormat{LaTeX2e}
\ProvidesPackage{fibeamer/theme/mu/%
  beamercolorthemefibeamer-mu-med}[2016/05/06]
\mode<presentation>
%    \end{macrocode}
% This color theme uses the combination of yellow and shades of gray.  The
% |fibeamer@{dark,|\-|light}@background{Inner,|\-|Outer}| colors are used
% within the background canvas template, which is defined within the base
% color theme of the Masaryk University and which draws the gradient
% background of the frames.
%    \begin{macrocode}
  \definecolor{fibeamer@black}{HTML}{000000}
  \definecolor{fibeamer@white}{HTML}{FFFFFF}
  \definecolor{fibeamer@red}{HTML}{c82600}
  \colorlet{fibeamer@lightRed}{fibeamer@red!30!fibeamer@white}
  \colorlet{fibeamer@darkRed}{fibeamer@red!60!fibeamer@black}
  \definecolor{fibeamer@gray}{HTML}{999999}
  \definecolor{fibeamer@lightOrange}{HTML}{FFA25E}
  \colorlet{fibeamer@orange}{fibeamer@lightOrange!80!fibeamer@darkRed}
%    \end{macrocode}
% \changes{v1.1.4:2}{2016/05/06}{Removed gradient backgrounds from
%   the color themes of the Masaryk University in Brno. [VN]}
%    \begin{macrocode}
  %% Background gradients
  \colorlet{fibeamer@dark@backgroundInner}{fibeamer@darkRed}
  \colorlet{fibeamer@dark@backgroundOuter}{fibeamer@darkRed}
  \colorlet{fibeamer@light@backgroundInner}{fibeamer@white}
  \colorlet{fibeamer@light@backgroundOuter}{fibeamer@white}
%    \end{macrocode}
% The |darkframes| environment switches the |\iffibeamer@darktrue|
% conditional on and sets a dark color theme.
%    \begin{macrocode}
  \renewenvironment{darkframes}{%
    \begingroup
      \fibeamer@darktrue
      %% Structures
      \setbeamercolor*{frametitle}{fg=fibeamer@lightRed}
      \setbeamercolor*{framesubtitle}{fg=fibeamer@white}
      %% Text
      \setbeamercolor*{normal text}{fg=fibeamer@white, bg=fibeamer@red}
      \setbeamercolor*{structure}{fg=fibeamer@white, bg=fibeamer@red}
%    \end{macrocode}
% \changes{v1.1.0:7}{2016/01/12}{Added support for \cs{alert} to
%   the themes of the Masaryk University in Brno. [VN]}
% \changes{v1.1.4:5}{2016/05/06}{Unified the alert colors in the
%   color themes of the Masaryk University in Brno. [VN]}
%    \begin{macrocode}
      \setbeamercolor*{alerted text}{fg=fibeamer@lightOrange}
%    \end{macrocode}
% \changes{v1.1.4:3}{2016/05/06}{Added proper link coloring for the
%   color themes of the Masaryk University in Brno. [VN]}
% \changes{v1.1.6}{2017/04/23}{Added proper citation coloring for the
%   color themes of the Masaryk University in Brno. [VN]}
%    \begin{macrocode}
      %% Items, footnotes and links
      \setbeamercolor*{item}{fg=fibeamer@lightRed}
      \setbeamercolor*{footnote mark}{fg=fibeamer@lightRed}
      \hypersetup{urlcolor=fibeamer@lightRed, citecolor=fibeamer@lightRed}
      %% Blocks
      \setbeamercolor*{block title}{%
        fg=fibeamer@white, bg=fibeamer@red!60!fibeamer@white}
      \setbeamercolor*{block title example}{%
        fg=fibeamer@white, bg=fibeamer@red!60!fibeamer@white}
      \setbeamercolor*{block title alerted}{%
        fg=fibeamer@darkRed, bg=fibeamer@lightOrange}
      \setbeamercolor*{block body}{%
        fg=fibeamer@red, 
        bg=fibeamer@gray!15!fibeamer@white}
      \usebeamercolor*{normal text}
      % Code listings
      \lstset{%
        commentstyle=\color{green!30!fibeamer@white},
        keywordstyle=\color{blue!30!fibeamer@white},
        stringstyle=\color{fibeamer@lightRed}}
      }{%
    \endgroup}
%    \end{macrocode}
% Outside the |darkframes| environment, the light theme is used.
%    \begin{macrocode}
  %% Structures
  \setbeamercolor{frametitle}{fg=fibeamer@red}
  \setbeamercolor{framesubtitle}{fg=fibeamer@black!75!fibeamer@white}
  %% Text
  \setbeamercolor{normal text}{fg=fibeamer@black, bg=fibeamer@white}
  \setbeamercolor{structure}{fg=fibeamer@black, bg=fibeamer@white}
%    \end{macrocode}
% \changes{v1.1.0:7}{2016/01/12}{Added support for \cs{alert} to
%   the themes of the Masaryk University in Brno. [VN]}
% \changes{v1.1.4:5}{2016/05/06}{Unified the alert colors in the
%   color themes of the Masaryk University in Brno. [VN]}
%    \begin{macrocode}
  \setbeamercolor{alerted text}{fg=orange}
  \addtobeamertemplate{block begin}{%
    \iffibeamer@dark % alerted text in plain block at dark slides
      \setbeamercolor{alerted text}{fg=fibeamer@orange}%
    \fi}{}
%    \end{macrocode}
% \changes{v1.1.4:3}{2016/05/06}{Added proper link coloring for the
%   color themes of the Masaryk University in Brno. [VN]}
% \changes{v1.1.6}{2017/04/23}{Added proper citation coloring for the
%   color themes of the Masaryk University in Brno. [VN]}
%    \begin{macrocode}
  %% Items, footnotes and links
  \setbeamercolor*{item}{fg=fibeamer@red}
  \setbeamercolor*{footnote mark}{fg=fibeamer@red}
  \hypersetup{urlcolor=fibeamer@red, citecolor=fibeamer@red}
  %% Blocks
  \setbeamercolor{block title}{%
    fg=fibeamer@white, bg=fibeamer@red}
  \setbeamercolor{block title example}{%
    fg=fibeamer@white, bg=fibeamer@red}
  \setbeamercolor{block title alerted}{%
    fg=fibeamer@white, bg=orange}
  \setbeamercolor{block body}{%
    fg=fibeamer@red!90!fibeamer@black, 
    bg=fibeamer@gray!20!fibeamer@white}
  %% Title
  \setbeamercolor{title}{fg=fibeamer@white, bg=fibeamer@red}
  % Code listings
  \lstset{%
    basicstyle=\footnotesize\ttfamily,
    breakatwhitespace=false,
    breaklines=true,
    commentstyle=\color{green!60!fibeamer@black},
    extendedchars=true,
    keywordstyle=\color{blue},
    showspaces=false,
    showstringspaces=false,
    showtabs=false,
    stringstyle=\color{violet}}
\mode
<all>
%    \end{macrocode}

% \subsection{The Faculty of Sports Studies}
% % \file{style/mu/fithesis-fsps.sty}
% This is the style file for the theses written at the Faculty of
% Sports Studies at the Masaryk University in Brno. It has been
% prepared in accordance with the formal requirements published at
% the website of the faculty\footnote{See \url{https://is.muni.cz/^^A
% auth/do/fsps/studijni/info-stud/SZZ/pokyny_ZP_13-5-2013.pdf}}.
%    \begin{macrocode}
\NeedsTeXFormat{LaTeX2e}
\ProvidesPackage{fithesis/style/mu/fithesis-fsps}[2017/05/21]
%    \end{macrocode}
% The file defines the color scheme of the respective faculty. Note
% the the color definitions are in RGB, which makes the resulting
% files generally unsuitable for printing.
%    \begin{macrocode}
\thesis@color@setup{
  links={HTML}{93BCF5},
  tableEmph={HTML}{A8BDE3},
  tableOdd={HTML}{EBEFF5},
  tableEven={HTML}{D1DAEB}}
%    \end{macrocode}
% The bibliography support is enabled. The |authoryear| citations
% are used and the bibliography is sorted by name, title, and year.
%    \begin{macrocode}
\thesis@bibliography@setup{
  style=iso-authoryear,
  sorting=nty}
\thesis@bibliography@load
%    \end{macrocode}
% The file loads the following packages:
% \begin{itemize}
%   \item\textsf{tikz} -- Used for dimension arithmetic.
%   \item\textsf{geometry} -- Allows for modifications of the type
%     area dimensions.
%   \item\textsf{setspace} -- Allows for line height modifications.
% \end{itemize}
% In addition to this, the type area width is set to
% 14\,cm in accordance with the formal requirements of the faculty.
%    \begin{macrocode}
\thesis@require{tikz}
\thesis@require{geometry}
\thesis@require{setspace}
\geometry{top=30mm,bottom=30mm,left=40mm,right=30mm,includeheadfoot}
%    \end{macrocode}
% The paragraph indentation is 1.25\,cm as per the requirements of the faculty.
%    \begin{macrocode}
\setlength{\parindent}{1.25cm}
%    \end{macrocode}
% The style file redefines the title page content
% not to include the author's name and the title page footer
% to include both the author's and advisor's name, the year and
% place of the thesis defense in accordance with the formal
% requirements of the faculty.
%    \begin{macrocode}
\def\thesis@blocks@titlePage@footer{%
  {\thesis@titlePage@large
    {% Calculate the width of the thesis author and advisor boxes
     \let\@A\relax\newlength{\@A}\settowidth{\@A}{{%
       \bf\thesis@@{advisorTitle}:}}
     \let\@B\relax\newlength{\@B}\settowidth{\@B}{\thesis@advisor}
     \let\@C\relax\newlength{\@C}\settowidth{\@C}{{%
       \bf\thesis@@{authorTitle}:}}
     \let\@D\relax\newlength{\@D}\settowidth{\@D}{\thesis@author}
     \let\@E\relax\newlength{\@E}\settowidth{\@E}{\thesis@field}
     \let\@F\relax\newlength{\@F}\pgfmathsetlength{\@F}{max(\@D,\@E)}
    \let\@left\relax\newlength{\@left}\pgfmathsetlength{\@left}{%
      max(\@A,\@B)}
    \let\@right\relax\newlength{\@right}\pgfmathsetlength{\@right}{%
      max(\@C,\@F)}
    % Typeset the thesis author and advisor boxes
    \vskip 2in\begin{minipage}[t]{\@left}
      {\bf\thesis@@{advisorTitle}:}\\\thesis@advisor
    \end{minipage}\hfill\begin{minipage}[t]{\@right}
      {\bf\thesis@@{authorTitle}:}\\\thesis@author\\\thesis@field
    \end{minipage}}\\[4em]\thesis@place, \thesis@year}}
%    \end{macrocode}
% \begin{macro}{\thesis@blocks@frontMatter}
% The |\thesis@blocks@frontMatter| macro sets up the style of the
% front matter of the thesis. The leading is adjusted in
% accordance with the requirements of the faculty.
% \begin{macrocode}
\def\thesis@blocks@frontMatter{%
  \thesis@blocks@clear
  \pagestyle{plain}
  \parindent 1.5em
  \setcounter{page}{1}
  \pagenumbering{roman}
  \onehalfspacing}
%    \end{macrocode}
% \end{macro}\begin{macro}{\thesis@blocks@mainMatter}
% The |\thesis@blocks@mainMatter| macro sets up the style
% of the main matter of the thesis. The leading is adjusted in
% accordance with the requirements of the faculty.
% \begin{macrocode}
\def\thesis@blocks@mainMatter{%
  \thesis@blocks@clear
  \setcounter{page}{1}
  \pagenumbering{arabic}
  \pagestyle{thesisheadings}
  \parindent 1.5em
  \onehalfspacing}
%    \end{macrocode}
% \end{macro}\begin{macro}{\thesis@blocks@bibliography}
% The |\thesis@blocks@bibliography| macro typesets the
% bibliography. The leading is adjusted in accordance
% with the requirements of the faculty.
% \begin{macrocode}
\def\thesis@blocks@bibliography{%
  \ifthesis@bibliography@loaded@
    \ifthesis@bibliography@included@\else
      \singlespacing
      \thesis@blocks@clear
      {\emergencystretch=3em%
      \printbibliography[heading=bibintoc]}%
    \fi
  \fi}
%    \end{macrocode}
% \end{macro}\begin{macro}{\thesis@blocks@declaration}
% The |\thesis@blocks@declaration| macro typesets the declaration
% text. Unlike the generic |\thesis@blocks@declaration| macro from
% the \texttt{style/mu/fithesis-sci.sty} file, this definition
% includes the date and a blank line for the author's signature, as
% per the requirements of the faculty.
%
% Along with the macros required by the locale file interface, the
% locale files need to define the following macros:
% \begin{itemize}
%   \item|\thesis@|\textit{locale}|@authorSignature| -- The
%     label of the author's signature field
%   \item|\thesis@|\textit{locale}|@formattedDate| -- A
%     formatted date
% \end{itemize}
%    \begin{macrocode}
\def\thesis@blocks@declaration{%
  \thesis@blocks@clear
  \begin{alwayssingle}%
    \chapter*{\thesis@@{declarationTitle}}%
    \thesis@declaration
    \vskip 2cm%
    {\let\@A\relax\newlength{\@A}
      \settowidth{\@A}{\thesis@@{authorSignature}}
      \setlength{\@A}{\@A+1cm}
    \noindent\thesis@place, \thesis@@{formattedDate}\hfill
    \begin{minipage}[t]{\@A}%
      \centering\rule{\@A}{1pt}\\
      \thesis@@{authorSignature}\par
    \end{minipage}}
  \end{alwayssingle}}
%    \end{macrocode}
% \end{macro}
% Note that there is no direct support for the seminar paper and
% thesis proposal types.  If you would like to change the contents
% of the preamble and the postamble, you should modify the
% |\thesis@blocks@preamble| and |\thesis@blocks@postamble| macros.
%
% All blocks within the autolayout preamble and postamble that are
% not defined within this file are defined in the
% \texttt{style/mu/fithesis-base.sty} file.
%    \begin{macrocode}
\def\thesis@blocks@preamble{%
  \thesis@blocks@coverMatter
    \thesis@blocks@cover
    \thesis@blocks@titlePage
  \thesis@blocks@frontMatter
    \thesis@blocks@declaration
    \thesis@blocks@thanks
    \thesis@blocks@tables}
\def\thesis@blocks@postamble{%
  \thesis@blocks@bibliography}
%    \end{macrocode}

% \iffalse
%</class>
% \fi
