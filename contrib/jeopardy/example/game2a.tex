\pdfoutput=1

\documentclass{article}

\usepackage{color}

\usepackage{amsmath}
\DeclareMathOperator{\arctg}{arctg}
\everymath{\displaystyle}

\usepackage[pdftex,designi]{web}
\usepackage[noxcolor,pdftex]{exerquiz}
\usepackage[ImplMulti]{dljslib}

\parindent 0pt
\def\correctColor{color.green}
\def\wrongColor{color.red}

\usepackage[picture,finetune]{jeopardy}
\def\JeopardyPictureFile{joke.jpg}

\def\ChampionMsg{Have a fun with math!}
\GrandPoohbahtoks{\textColor{1 1 0 rg}\textSize{0}}

\everyCategoryHead{\color{darkgreen}\tiny}
\pagestyle{empty}

\def\logoB{\rotatebox{90}{%
    \vbox{\normalsize\hbox{\color{red}{Robert Ma\v{r}\'{i}k}}
      \hbox{\color{green}Jeopardy game}}}
  \vfill}

\let\rmdefault\sfdefault
\hypersetup{pdfpagemode=Window, pdfnewwindow=true, pdfmenubar=true,%
  pdftoolbar=true,colorlinks,  pdfwindowui=false,
  pdfpagemode=Window}


\AditionalShift=5pt
\begin{document}
%\CellHeight=24bp

\Celltoks{\BG{1 1 0.8}\textSize{0}\BC{1 1 0.8}\W{0}}
\SetGameWidth{0.7\linewidth}
\everyPushButton{\S{I}\BC{}}

\MakeGameBoard

\begin{category}{Precalculus}

\begin{question}
$\ln \frac xy=$
\Ans0 $\ln x+\ln y$
\Ans1 $\ln x-\ln y$
\Ans0 $x\ln y$
\Ans0 $y\ln x$
\Ans0 none of them
\end{question}

\begin{question}
The function $y=x^2\cdot \sin x$ is
\Ans1 odd
\Ans0 even
\Ans0 neither odd nor even
\end{question}

\begin{question}
$\arctan 1=$
\Ans0 $\infty$ 
\Ans0 $\frac \pi3$
\Ans1 $\frac \pi4$
\Ans0 $\frac \pi6$
\Ans0 none of them
\end{question}

\begin{question}
The equivalence "$a<b$ if and only if $f(a)<f(b)$" is the property of
\Ans0 even functions
\Ans0 one-to-one functions
\Ans0 continuous functions
\Ans1 increasing functions
\Ans0 none of them
\end{question}

\end{category}


\begin{category}{Functions}

\begin{question}
How many points of inflection is on the graph of the function $y=\sin x$ in
the open interval $(0,2\pi)$
\Ans0 none
\Ans1 one
\Ans0 two
\Ans0 three
\Ans0 none of them
\end{question}

\begin{question}
Find points of discontinuity of the function $y=\frac {x-4}{(x-2)\ln x}$
\Ans0 none
\Ans0 $0$
\Ans0 $0$, $1$
\Ans1 $0$, $1$, $2$
\Ans0 $0$, $2$
\Ans0 $0$, $1$, $4$
\Ans0 $0$, $4$
\Ans0 none of them
\end{question}

\begin{question}
Let $f$ be a function and $f^{-1}$ be its inverse. Then $f^{-1}\bigl(f(x)\bigr)=$
\Ans0 $0$
\Ans0 $1$
\Ans1 $x$
\Ans0 $f(x)$
\Ans0 $f^{-1}(x)$
\Ans0 none of them
\end{question}

\begin{question}
$\arcsin(\sin x)=x$ for every $x\in\mathbf{R}$
\Ans0 Yes
\Ans1 No
\end{question}

\end{category}


\begin{category}{Limits}

\begin{question}
$\lim_{x\to-\infty} \arctg x=$
\Ans0 $0$
\Ans0 $\frac\pi2$
\Ans1 $-\frac\pi2$
\Ans0 $\infty$
\Ans0 $-\infty$
\Ans0 none of them
\end{question}

\begin{question}
$\lim_{x\to\infty}\sin x=$
\Ans0 $1$
\Ans0 $-1$
\Ans1 does not exist
\Ans0 none of them
\end{question}

\begin{question}
$\lim_{x\to\infty}\frac{2x^3+x^2+4}{x^2-x+2}=$
\Ans1 $\infty$
\Ans0 $2$
\Ans0 $0$
\Ans0 none of them
\end{question}

\begin{question}
$\lim_{x\to 0^+}\frac{e^{1/x}(x-1)}{x}$
\Ans0 $0$
\Ans0 $1$
\Ans0 $e$
\Ans0 $\infty$
\Ans0 $-1$
\Ans0 $-e$
\Ans1 $-\infty$
\Ans0 none of them
\end{question}

\end{category}

\begin{category}{Derivative}


\begin{question}
$\left(\frac 1{\sqrt[3]x}\right)'=$
\Ans0 $\frac 13 x^{-2/3}$
\Ans0 $-\frac 13 x^{-2/3}$
%\Ans0 $\frac 13 x^{1/3}$
\Ans0 $-\frac 13 x^{1/3}$
\Ans0 $\frac 13 x^{-4/3}$
\Ans1 $-\frac 13 x^{-4/3}$
\Ans0 none of them
\end{question}


\begin{question}
$(x-x\ln x)'=$
\Ans0 $\ln x$
\Ans1 $-\ln x$
\Ans0 $1+\ln x$
\Ans0 $1-\ln x$
\Ans0 $0$
\Ans0 $1-\frac 1x$
\Ans0 none of them
\end{question}

\begin{question}
$\left(x^2e^{x^2}\right)'$
\Ans0 $2xe^{2x}$
\Ans0 $2xe^{x^2}2x$
\Ans0 $2xe^{x^2}+x^2e^{x^2}$
\Ans1 $2xe^{x^2}+x^2e^{x^2}2x$
\Ans0 $2xe^{x^2}2x+x^2e^{x^2}2x$
\Ans0 none of them
\end{question}

\begin{question}
The definition of the derivative of the function $f$ at the point $a$ is
\Ans0 $\lim_{h\to 0}\frac{f(x+h)+f(x)}{h}$
\Ans0 $\lim_{h\to 0}\frac{f(x+h)}{h}$
\Ans1 $\lim_{h\to 0}\frac{f(x+h)-f(x)}{h}$
\Ans0 $\lim_{h\to 0}\frac{f(x)-f(x+h)}{h}$
\Ans0 $\lim_{h\to 0}\frac{f(x-h)-f(x)}{h}$
\Ans0 none of them
\end{question}

\end{category}

\everyRespBoxMath{\BG{1 1 1}}
\begin{category}{Evaluation of derivatives}
  \begin{question}$(x^2+1)'=$
    \RespBoxMath{2x}{3}{0.001}{[1,5]}
  \end{question}
  \begin{question}$(xe^x)'=$
    \RespBoxMath{(x+1)e^x}{3}{0.001}{[1,5]}
  \end{question}
  \begin{question}$\ln(\sin x)=$
    \RespBoxMath{cos(x)/sin(x)}{3}{0.001}{[1,2]} 
  \end{question}
  \begin{question}$(xe^{-x})'=$
    \RespBoxMath{(1-x)e^{-x}}{3}{0.001}{[1,5]}  
  \end{question}
\end{category}



\begin{category}{Theory}

\begin{question} By theorem of Bolzano, the polynomial $y=x^3+2x+4$ has zero on
\Ans0 $(0,1)$
\Ans0 $(1,2)$
\Ans0 $(2,3)$
\Ans0 $(-1,0)$
\Ans1 $(-2,-1)$
\Ans0 $(-3,-2)$
\Ans0 none of them
\end{question}


\begin{question}
Let $a\in Im(f)$. Then the solution of the equation $f(x)=a$ exists. This
solution is unique if and only if
\Ans1 $f$ is one-to-one
\Ans0 $f$ is increasing
\Ans0 $f$ continuous
\Ans0 $f$ differentiable
\Ans0 none of them
\end{question}


\begin{question}
If the function has a derivative at the point $x=a$, then it is
\Ans0 increasing at $a$.
\Ans0 decreasing at $a$.
\Ans0 one-to-one at $a$.
\Ans1 continuous at $a$.
\Ans0 undefined at $a$.
\end{question}

\begin{question}
If both $y(a)=y'(a)=y''(a)=0$, then the function
\Ans0 has local maximum at $a$.
\Ans0 has local minimum at $a$.
\Ans0 has point of inflection at $a$.
\Ans1 any of these possibilites may be true, we need more informations.
\end{question}

\end{category}


\end{document}
%%% Local Variables: 
%%% mode: latex
%%% TeX-master: t
%%% End: 


