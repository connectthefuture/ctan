%% land-sample-slide.tex
%% Copyright 2013 Behzad Salimi
%
% This work may be distributed and/or modified under the
% conditions of the LaTeX Project Public License, either version 1.3
% of this license or any later version.
% The latest version of this license is in
%   http://www.latex-project.org/lppl.txt
% and version 1.3 or later is part of all distributions of LaTeX
% version 2005/12/01 or later.
%
% This work has the LPPL maintenance status `maintained'.
% 
% The Current Maintainer of this work is Behzad Salimi
%  sites.google.com/site/quadratures
%
% This work consists of the files:
%
%        land-sample-slide.pdf
%        land-sample-slide.tex
%        port-sample-slide.pdf
%        port-sample-slide.tex
%        sslides.cls

\documentclass{sslides}
\usepackage{graphics}

   \addtolength{\topmargin}{-15mm}
   \setlength{\textheight}{7.00in}
   \setlength{\textwidth}{9.50in}
   \setlength{\oddsidemargin}{-0.25in}
%  \addtolength{\headheight}{4mm}
%  \setlength\evensidemargin{0mm}
%  \setlength{\parindent}{0mm}
%  \addtolength{\parskip}{2mm}

\renewcommand{\thefootnote}{\fnsymbol{footnote}}
\newcommand{\foot}[1]{\footnote{#1}%
\ifthenelse{\value{footnote}=9}{\setcounter{footnote}{0}}{}}

\newsavebox{\tl}
\newcommand{\topleft}[1]{\savebox{\tl}{\tiny #1}}
\newcommand{\tleft}{\usebox{\tl}}
\newsavebox{\tc}
\newcommand{\topcenter}[1]{\savebox{\tc}{\tiny #1}}
\newcommand{\tcenter}{\usebox{\tc}}
\newsavebox{\tr}
\newcommand{\topright}[1]{\savebox{\tr}{\tiny #1}}
\newcommand{\tright}{\usebox{\tr}}

\newsavebox{\bl}
\newcommand{\footleft}[1]{\savebox{\bl}{\tiny #1}}
\newcommand{\bleft}{\usebox{\bl}}
\newsavebox{\bc}
\newcommand{\footcenter}[1]{\savebox{\bc}{\tiny #1}}
\newcommand{\bcenter}{\usebox{\bc}}

%  \newsavebox{\br}
%  \newcommand{\footright}[1]{\savebox{\br}{\tiny #1}}
%  \newcommand{\bright}{\usebox{\br}}
%  uncomment above three lines to use \footright{}


%  Use this portion to create any header or footer style.
%%%%%%%%%%%%%%%%%%%%%%%%%%%%%%%%%%%%%%%%%%%%%%%%%%%%%%%%%%%%%%%%%%%%
\oddh{%
\parbox[t]{\textwidth}{%
{\makebox[10mm][l]{\tleft}\hfill\makebox[10mm][c]{\tcenter}
\hfill\makebox[10mm][r]{\tright}}\\[-9mm]
\makebox[0mm]{\ }\hrulefill}}
%%%%%%%%%%%%%%%%%%%%%%%%%%%%%%%%%%%%%%%%%%%%%%%%%%%%%%%%%%%%%%%%%%%%
\oddf{\parbox[b]{\textwidth}{\hrulefill\vspace{-5mm}\newline%
{\makebox[10mm][l]{\bleft}\hfill\makebox[10mm][c]{\bcenter}\hfill
\makebox[10mm][r]{\tiny\theslide}}}}
%  \makebox[10mm][r]{\bright}}}} % replace with the line above,
%  then use \footright{...} for a fixed field in the right footer
%%%%%%%%%%%%%%%%%%%%%%%%%%%%%%%%%%%%%%%%%%%%%%%%%%%%%%%%%%%%%%%%%%%%

%  commands to set header/footer:
%  \topleft{}
%  \topcenter{}
%  \topright{}
%  \footleft{}
%  \footcenter{}
%  \footright{}

%  \onlyslides{3}
\begin{document}



\begin{slide}
\thispagestyle{empty}

\begin{center}
{\large\LaTeX\\
Slide Class\\
with\\
Header and Footer}\\[7mm]
\normalfont\normalsize
Behzad Salimi\\[1mm]
{\small sites.google.com/site/quadratures}
\end{center}

\

\centerline{Abstract}
\vspace{-7mm}
A simple slide ``documentclass" to use the standard \LaTeX\ 
slide class with the added option of including header and footer.

\end{slide}



\topleft{\raisebox{1mm}{\LaTeX}}
\topcenter{sslides class}
\topright{\today}
\footleft{sites.google.com/site/quadratures}
%  \footcenter{x}
%\footright{\theslide}  % this position is reserved to advance page number
\begin{slide}
\centerline{\textbf{\LaTeX\ slides with header/footer}}

The standard \LaTeX\ {\tt slide} class offers a very clean, sharp
and easy-to-read font for slides. However, the standard class does
not have a provision for including header and footer on slides.

The slide class {\tt sslides} allows creation of  arbitrary header
and footer while preserving all of the functionality of the
standard {\tt slide} class.

\LaTeX\ code can be used to generate any header and footer.  The
{\tt .tex} file for this manual provides a simple example of
creating a template to independently specify the left-, center-,
or right-region of both header and footer.

%  In the example file, the right footer region is reserved for page number, but this can
%  be easily changed.

\end{slide}










\topcenter{Example Slide}
%  \footcenter{Example Slide}
\begin{slide}

\centerline{\textbf{How to use}}
Copy the class file {\tt sslides.cls}
%  (that came with this package)
in a directory visible to \LaTeX, for example, the current
directory where the slide file is run.  The following two commands
in the slide input file generate header and footer:\\
{\small\verb+\oddh{ . . . }+\hfill
{\footnotesize generates text/objects in the header;}\\
\verb+\oddf{ . . . }+\hfill
{\footnotesize generates text/objects in the footer}.}\\
The header/footer specified will show on all slides until they are
changed.

\

\centerline{\textbf{Example}}
{\small
\begin{verbatim}
\documentclass{sslides}
\oddh{ header field }
\oddf{ footer field }
\begin{slide}
. . . body of slide . . .
\end{slide}
\end{document}
\end{verbatim}
}
\end{slide}

























\end{document}

