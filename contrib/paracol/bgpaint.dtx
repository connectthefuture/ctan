% \backgroundcolor{t}[rgb]{0.7,0,0}
% \backgroundcolor{b}[rgb]{0.8,0.6,0}
% \backgroundcolor{l}[rgb]{0,0,0.7}
% \backgroundcolor{r}[rgb]{0,0.7,0}
% \backgroundcolor{c[0]}[rgb]{1,0.8,1}
% \backgroundcolor{c[1]}[rgb]{1,1,0.8}
% \backgroundcolor{g}[rgb]{0.8,1,1}
% \backgroundcolor{f}[rgb]{0.8,0,1}
% \backgroundcolor{n}[rgb]{0.8,0.6,1}
% \backgroundcolor{p}[rgb]{0.8,1,0.6}
% \backgroundcolor{s}[rgb]{0.8,0.8,0.8}
% \pagerim5pt
% 
% \section{Examples of Background Painting}
% \label{sec:bgpaint}
% \subsection{Fundamental Painting}
% \label{sec:bgpaint-fund}
% \twosided[pcm]
% 
% As you undoubtedly notice, this page and a few pages following it are
% colorfully painted.  For this and the next three pages, the author
% declared the \bground{} color of each region as follows.
% 
% \begin{itemize}\item[]
% |\backgroundcolor{t}[rgb]{0.7,0,0}       % dark red for top margin|\\
% |\backgroundcolor{b}[rgb]{0.8,0.6,0}     % dark orange for bottom margin|\\
% |\backgroundcolor{l}[rgb]{0,0,0.7}       % dark blue for left margin|\\
% |\backgroundcolor{r}[rgb]{0,0.7,0}       % dark green for right margin|\\
% |\backgroundcolor{c[0]}[rgb]{1,0.8,1}    % pink for colunmn-0|\\
% |\backgroundcolor{c[1]}[rgb]{1,1,0.8}    % cream yellow for column-1|\\
% |\backgroundcolor{g}[rgb]{0.8,1,1}       % light blue for the gap|\\
% |\backgroundcolor{f}[rgb]{0.8,0,1}       % purple for page-wise floats|\\
% |\backgroundcolor{n}[rgb]{0.8,0.6,1}     |
%     |% light purple for page-wise footnotes|\\
% |\backgroundcolor{p}[rgb]{0.8,1,0.6}     |
%     |% pale green for pre/post-environment|\\
% |\backgroundcolor{s}[rgb]{0.8,0.8,0.8}   % light gray for spanning texts|
% \end{itemize}
% 
% \SpecialUsageIndex{\backgroundcolor}
% 
% Therefore, the \bground{} of this |p|re-environment paragraph and other
% stuff above is painted by pale green.
% 
% \Index{pre-environment stuff}
% 
% Since the author set \Uidx{\!\pagerim!} to be 5\,|pt|, you will see
% unpainted strips of 5\,|pt| wide at all paper edges surrounding painted
% regions.  For this and the next three pages, \Uidx{\!\twosided!}|[pcm]| is
% declared to enable |p|, |c| and |m| features but to disable the |b|
% feature.  Therefore, though this page \pageref{sec:bgpaint} is even and
% thus the left outside margin is wider than the right inside one, the
% \bground{}s of |l|(eft) and |r|(ight) margins are painted by dark blue and
% dark green respectively.
% \par\bigskip
% 
% \begin{paracol}{2}
% This column-0 is now right and inside because of the |c| feature of
% \!\twosided! is enabled.  On the other hand, the \bground{} is this column
% is painted by pink because \!\backgroundcolor! for |c[0]| specifies so.
% That is, the column ordinals optionally given to |c|(olumn) (and |g|(ap))
% regions are \emph{logical} ones not always corresponding to their
% \emph{physical} positions in a page.
% 
% \switchcolumn
% \begingroup\it
% As explained in the right column-0, the \bground{} of this left and
% outside column-1 is painted by cream yellow as
% {\rm\!\backgroundcolor!|{c[1]}|} specifies.  Now we have a
% {\rm\!\switchcolumn!|*|} with a \mctext{} to show the \bgpaint{} for
% it\footnote{
% 
% Since the footnotes in this \env{paracol} environment are \scfnote{} and
% \mgfnote{}, and \!\backgroundcolor!\texttt{\char`\{n\char`\}} specifies
% light purple, the \bground{} of this (foot)|n|(ote) region is painted by
% the color.}.
% 
% \par\endgroup
% \switchcolumn*[\subsection*{The background of this |s|(panning text)
% region is painted by light gray}\medskip]
% 
% \begin{figure*}\nosv
% \def\arraystretch{0.8}
% \centerline{\begin{tabular}[b]{|c|}\hline
%     \hbox to.9\textwidth{}\\
%     \texttt{f}(loat) region for this page-wise figure is painted by purple\\ 
%     \\\hline
%     \end{tabular}}
% \caption{A Page-Wise Figure}
% \end{figure*}
% 
% This paragraph is to show how the first line of a paragraph just below a
% \mctext{} is placed in the painted region.
% \par\vfill
% 
% \switchcolumn
% \begingroup\it
% See the right column for the reason why this paragraph is here.
% \par\vfill
% 
% See the right column for what we are now doing.
% \par\endgroup
% \switchcolumn
% 
% Now we have a \!\flushpage! to see the \bgpaint{} for a material not shown
% in the page, i.e., a page-wise float.
% \flushpage
% 
% Since we are now in an odd-numbered page \pageref{page:bgpaint2}, this
% column-0 is now a left one and is still painted by pink of course.
% \par\vfill\label{page:bgpaint2}
% 
% This paragraph is to show how the last line of a page without \Scfnote{}s
% is placed in the painted region.
% \par\newpage
% 
% This page is to show how the page without any \pwstuff{} looks like.
% \par\vfill
% 
% Shortly we will close this \env{paracol} environment in the next page.
% \par\newpage
% 
% Now we are closing this \env{paracol} environment to show how its
% \postenv{} is painted.
% 
% \switchcolumn
% \begingroup\it
% As expected, the \bground{} of this column-1 is still painted by cream
% yellow.
% \par\vfill
% 
% See the comment in the left column.
% \par\newpage
% 
% See the right column for the reason why we have this almost blank page.
% \par\vfill
% 
% See the right column for what will happen shortly.
% \par\newpage
% 
% See the left column for the reason why we are now closing the environment.
% \endgroup
% \end{paracol}
% \bigskip
% 
% The \bground{} of this paragraph in |p|(ost-environment) region is also
% painted by pale green, because \postenv{} can be \preenv{} at the same
% time as we see shortly.  \par\bigskip
% 
% \begin{paracol}{2}
% This short \env{paracol} environment illustrates how the \preenv{} of this
% environment, or the \postenv{} of the last environment in other words, is
% painted.
% 
% \switchcolumn
% \begingroup\it
% Therefore, the author does not have much to say in this column, except for
% giving a footnote here\footnote{
% 
% Since this footnote is \mgfnote{} with that in the \postenv{}, it is
% considered as a part of \postenv{} and thus painted by pale green rather
% than light purple.\label{fn:bgpaint1}}.
% \endgroup
% \end{paracol}
% \bigskip
% 
% Before moving to the next example, one caution is given for \bgpaint{} of
% \Mgfnote{}s.  As the footnote \ref{fn:bgpaint1} itself says, \Mgfnote{}s
% given in the \lpage{} of a \env{paracol} environment are considered as
% belonging to \postenv{}.  Therefore, the footnote \ref{fn:bgpaint1} is
% painted by pale green as well as another footnote given now\footnote{
% 
% Since this footnote really belongs to \postenv{}, its \bground{} is painted
% by pale green naturally.}.
% \par\label{page:bgpaint4}
% 
% 
% 
% \newpage
% \backgroundcolor{t(0pt,0pt)(0pt,-4pt)}[rgb]{0.7,0,0}
% \backgroundcolor{b(0pt,-4pt)(0pt,0pt)}[rgb]{0.8,0.6,0}
% \backgroundcolor{l(0pt,4pt)(-4pt,4pt)}[rgb]{0,0,0.7}
% \backgroundcolor{r(-4pt,4pt)(0pt,4pt)}[rgb]{0,0.7,0}
% \backgroundcolor{c[0](4pt,4pt)}[rgb]{1,0.8,1}
% \backgroundcolor{c[1](4pt,4pt)}[rgb]{1,1,0.8}
% \backgroundcolor{g(-4pt,4pt)}[rgb]{0.8,1,1}
% \backgroundcolor{f(4pt,4pt)(4pt,-4pt)}[rgb]{0.8,0,1}
% \backgroundcolor{n(4pt,-4pt)(4pt,4pt)}[rgb]{0.8,0.6,1}
% \backgroundcolor{p(4pt,4pt)}[rgb]{0.8,1,0.6}
% \backgroundcolor{s(4pt,-4pt)}[rgb]{0.8,0.8,0.8}
% 
% \subsection{Mirrored Painting and Enlarging/Shrinking/Shifting Regions}
% \label{sec:bgpaint-me}
% \twosided
% 
% At a glance, this and the next three pages look painted similarly to
% previous four pages, but by a careful examination you should notice
% two important differences.  The first one is found in the colors
% of left and right margins.  As the author enabled all features of
% \Uidx{\!\twosided!} including |b| for \mirror{}ing and we are now in an
% even-numbered page \pageref{sec:bgpaint-me}, the left and outside margin
% is painted by dark green for the region |r|(ight margin), while the right
% and inside one is painted by dark blue for |l|(eft margin).
% 
% The other is that regions are enlarged, shrunk or shifted by 4\,|pt| by
% the following \!\backgroundcolor! commands with extensions.
% 
% \begin{itemize}\item[]
% |\backgroundcolor{t(0pt,0pt)(0pt,-4pt)}[rgb]{0.7,0,0}   |
%     |% B up|\\
% |\backgroundcolor{b(0pt,-4pt)(0pt,0pt)}[rgb]{0.8,0.6,0} |
%     |% T down|\\
% |\backgroundcolor{l(0pt,4pt)(-4pt,4pt)}[rgb]{0,0,0.7}   |
%     |% R left T/B outside|\\
% |\backgroundcolor{r(-4pt,4pt)(0pt,4pt)}[rgb]{0,0.7,0}   |
%     |% L right T/B outside|\\
% |\backgroundcolor{c[0](4pt,4pt)}[rgb]{1,0.8,1}          |
%     |% all edges outside|\\
% |\backgroundcolor{c[1](4pt,4pt)}[rgb]{1,1,0.8}          |
%     |% all edges outside|\\
% |\backgroundcolor{g(-4pt,4pt)}[rgb]{0.8,1,1}            |
%     |% L/R inside & T/B outside|\\
% |\backgroundcolor{f(4pt,4pt)(4pt,-4pt)}[rgb]{0.8,0,1}   |
%     |% L/R outside & T/B up|\\
% |\backgroundcolor{n(4pt,-4pt)(4pt,4pt)}[rgb]{0.8,0.6,1} |
%     |% L/R outside & T/B down|\\
% |\backgroundcolor{p(4pt,4pt)}[rgb]{0.8,1,0.6}           |
%     |% all edges outside|\\
% |\backgroundcolor{s(4pt,-4pt)}[rgb]{0.8,0.8,0.8}        |
%     |% L/R outside & T/B inside|
% \end{itemize}
% 
% \SpecialUsageIndex{\backgroundcolor}
% 
% In the comments above, |L|(eft), |R|(ight), |T|(op) and |B|(ottom) mean
% edges moved by a given extension.  Therefore, for example,
% ``|L/R outside & T/B up|'' for |f|(loat) region means it is enlarged
% horizontally and shifted up vertically by the asymmetric extension
% |(4pt,4pt)(4pt,-4pt)|.  These a little bit complicated setting of
% extensions are to solve the problems in the fundamental example shown in
% previous four pages, namely too strict definition of the regions to be
% painted.  That is, both vertical edges of regions having texts, e.g.,
% |c|(olumn) regions, should look too close to the first and last letters.
% Similarly both horizontal edges of those regions seem too close especially
% when the first line is tall (e.g., the section title in
% p.\Tie\pageref{sec:bgpaint} and the page-wise figure in
% p.\Tie\pageref{page:bgpaint2}) and the last line of a column is followed by
% \mctext{} or \postenv.  Therefore, the author made fine tuning moving
% inside edges of margins outside, and so on.  We will come back this issue
% after exemplifying the effect of the tuning.
% \par\bigskip
% 
% \advance\skip\footins4pt\relax
% \begin{paracol}{2}
% By the tuning to enlarge this |c|(olumn) region, this paragraph has
% comfortable spaces above and below it, as well as at the both side edges.
% 
% \switchcolumn
% \begingroup\it
% This paragraph is surrounded by spaces of a small but comfortable amount as
% well.\footnote{
% 
% Shifting this (foot)|n|(ote) region down a little bit, the space below this
% footnote and above the top edge of the |b|(ottom margin) region is enlarged.}.
% 
% \par\endgroup
% \switchcolumn*[\subsection*{The background of this |s|(panning text)
% region is painted by light gray and enlarged horizontally but shrunk
% vertically}\par\medskip]
% 
% \begin{figure*}\nosv
% \def\arraystretch{0.8}
% \centerline{\begin{tabular}[b]{|c|}\hline
%     \hbox to.9\textwidth{}\\
%     shifting up this \texttt{f}(loat) region gives us a small space above
%     the top edge of the rectangle\\
%     \\\hline
%     \end{tabular}}
% \caption{A Page-Wise Figure}
% \end{figure*}
% 
% This paragraph is to show how well the first line of a paragraph just below a
% \mctext{} is separated from the boundary of two painted regions.
% \par\vfill
% 
% \switchcolumn
% \begingroup\it
% See the right column for the reason why this paragraph is here.
% \par\vfill
% 
% See the right column for what we are now doing.
% \par\endgroup
% \switchcolumn
% 
% By enlarging this |c|(olumn) region and shift the (foot)|n|(ote) region
% down, this paragraph has a comfortable amount of space below it.
% \flushpage
% 
% Similarly to other paragraphs below \pwstuff, this paragraph is well
% separated from the bottom edge of the |f|(loat) region above.
% 
% \par\vfill\label{page:bgpaint-me2}
% 
% As in the case of the line above \Scfnote{}s, the last line of this
% paragraph has a sufficient space separating it from the top edge of the
% |b|(ottom margin) region.
% \par\newpage
% 
% This page is to show how the page without any \pwstuff{} looks like.  As
% you are seeing, the space above this paragraph is sufficient and
% comfortable.
% \par\vfill
% 
% Shortly we will close this \env{paracol} environment in the next page.
% \par\newpage
% 
% Now we are closing this \env{paracol} environment to show how this
% paragraph is separated from the boundary of |c|(olumn) and
% |p|(ost-environment) regions.
% 
% \switchcolumn
% \begingroup\it
% See the comment in the left column for the intention of placing this
% paragraph here.
% \par\vfill
% 
% See the comment in the left column, too.
% \par\newpage
% 
% See the right column for the reason why we have this almost blank page.
% \par\vfill
% 
% See the right column for what will happen shortly.
% \par\newpage
% 
% See the left column for the reason why we are now closing the environment.
% \endgroup
% \end{paracol}
% \bigskip
% 
% The \bground{} of this paragraph in |p|(ost-environment) region is
% painted by pale green as done in p.\Tie\pageref{page:bgpaint4}, but its top
% and bottom edges \emph{look} shifted down and up to give spaces below and
% above the last and first paragraphs in \env{paracol} environments,
% respectively.
% \par\bigskip
% 
% \begin{paracol}{2}
% This short \env{paracol} environment illustrates how the \preenv{} of this
% environment, or the \postenv{} of the last environment in other words, is
% painted.
% 
% \switchcolumn
% \begingroup\it
% Therefore, the author does not have much to say in this column, except for
% giving a footnote here\footnote{
% 
% As the footnote \ref{fn:bgpaint1} in p.\Tie\pageref{fn:bgpaint1}, this
% \Mgfnote{} is a part of \postenv{} and thus painted by pale green rather
% than light purple.\label{fn:bgpaint-me1}}.
% \endgroup
% \end{paracol}
% \bigskip
% 
% In the setting with \!\backgroundcolor! commands in
% p.\Tie\pageref{sec:bgpaint-me}, the author carefully moved contacting edges
% of regions.  For example, to enlarge |c|(olumn) regions, the inside edges
% of |l|(eft margin) and |r|(ight margin) regions are moved outside, and both
% vertical edges of the |g|(ap) region shifted toward its inside.  As for
% the horizontal edges, the bottom edges of |t|(op margin) and |f|(loat)
% regions are moved up, the top edges of |b|(ottom margin) and
% (foot)|n|(ote) regions are moved down, and both top and bottom edges of
% the |s|(panning text) region are shifted toward its inside.
% 
% These edge shifting could make a region too narrow or too much shifted
% resulting in a material in it overreaching its boundary, especially in
% vertical shifting of horizontal edges.  However we can exploit some large
% space automatically or manually inserted above and/or below the material
% to avoid overreaching.  That is the author exploited the following spaces;
% \!\headsep! below the page head (though it is empty in this document);
% \!\dbltextfloatsep! below the bottom-most page-wise float; spaces that
% \!\subsection!|*| inserts above and below it together with manually
% inserted \!\medskip! below it; \!\skip!\!\footins!\footnote{
% 
% This is a kind of ``length command'' maybe not widely known.}
% 
% above the first footnote which the author enlarged by 4\,|pt| temporarily
% for this and the next subsections; and \!\footskip! from the bottom edge
% of text area to that of the page number.
% 
% Now you might notice that the explanation above does not mention the |p|
% region for \Preenv{} and \postenv.  As you should find in the settings,
% this region is enlarged horizontally \emph{and vertically} so that its top
% and bottom edges are moved up and down when the region is at the top or
% bottom of a page, as you are seeing now and find in
% p.\Tie\pageref{sec:bgpaint-me}.  However, this enlargement of course has a
% side effect that the region collides against |c|(olumn) and |g|(ap) regions
% also enlarged vertically making them overlapped.  This overlap will be
% invisible with most of \emph{printers} because, as shown in
% Section\Tie\ref{sec:ref-bgpaint}, |p| region is painted \emph{before} |c|
% and |g| regions are painted.  In addition, since relatively large spaces
% of \!\bigskip! are manually inserted before each \beginparacol{} and after
% each \Endparacol{}, texts in \Preenv{} and \postenv{} are well separated
% from region boundaries.
% 
% This overlay painting |c| and |g| over |p|, however, might produce an
% unexpected result with some printer with which, for example, two colors
% are \emph{blended} in the thin overlapped strip\footnote{
% 
% For example, a dvi previewer |dviout| produces such a blended result with
% the default setting of coloring.}.
% 
% Unfortunately, this overlay painting is inevitable in the current version
% 1.3, but in a future version, hopefully 1.4, more sophisticated
% \emph{position-dependent} region definition, for example, to shift the top
% edge of |p| region only when the region is at the top of page, could be
% introduced.
% 
% Another remark is that the \mirror{}ing specified by the |b| feature of
% \!\twosided! works not only on the colors of side margins but also on
% their asymmetric shrinkage.  That is, the asymmetric shifts of vertical
% edges of |l| and |r| regions correctly performed irrespective of their
% physical positions, i.e., even when the |l| (resp.\ |r|) region is at
% the right (resp.\ left) margin and the edge to be shift is the left
% (resp.\ right) one rather than right (resp.\ left).
% 
% 
% 
% \newpage \suppressfloats
% \nobackgroundcolor{t}
% \nobackgroundcolor{b}
% \nobackgroundcolor{l}
% \nobackgroundcolor{r}
% \nobackgroundcolor{g}
% \backgroundcolor{c[0](4pt,4pt)(0.5\columnsep,4pt)}[rgb]{1,0.8,1}
% \backgroundcolor{c[1](0.5\columnsep,4pt)(4pt,4pt)}[rgb]{1,1,0.8}
% \backgroundcolor{C[0](10000pt,10000pt)(0.5\columnsep,10000pt)}[rgb]{1,0.8,1}
% \backgroundcolor{C[1](0.5\columnsep,10000pt)(10000pt,10000pt)}[rgb]{1,1,0.8}
% 
% \subsection{Regions with Infinite Extensions}
% \label{sec:bgpaint-inf}
% 
% You are now seeing another \bgpaint{} much different from previous two
% examples.  That is, after disabling painting of |t|, |b|, |l|, |r| and |g|
% regions by \Uidx{\!\nobackgroundcolor!}, the author gave the followings
% for painting this and the next pages.
% 
% \begin{itemize}\item[]
% |\backgroundcolor|
%     |{c[0](4pt,4pt)(0.5\columnsep,4pt)}[rgb]{1,0.8,1}|\\
% |\backgroundcolor|
%     |{c[1](0.5\columnsep,4pt)(4pt,4pt)}[rgb]{1,1,0.8}|\\
% |\backgroundcolor|
%     |{C[0](10000pt,10000pt)(0.5\columnsep,10000pt)}[rgb]{1,0.8,1}|\\
% |\backgroundcolor|
%     |{C[1](0.5\columnsep,10000pt)(10000pt,10000pt)}[rgb]{1,1,0.8}|
% \end{itemize}
% 
% \SpecialUsageIndex{\backgroundcolor}
% 
% The first two lines above is different from the previous declaration
% because inside edges of |c[0]| and |c[1]| regions are shifted toward
% outside of them and thus inside of unpainted |g| region so that the edges
% are contacted.  On the other hand, the last two lines are for
% \emph{under-painting} of columns and has \emph{\bginfext} to make top,
% bottom and outside edges of |C| regions reaching to the corresponding
% paper edges.  Since this under-painting is done with colors same as those
% of over-painting of |c| regions, you will have an impression that the
% paper is two-toned and \pwstuff{} are pasted on the paper\footnote{
% 
% This footnote is given outside \env{paracol} environment but its
% \bground{} is painted by light purple because it is merged with the
% footnote \ref{fn:bgpaint-inf2}.\label{fn:bgpaint-inf1}}.
% 
% \par\bigskip
% 
% \begin{figure}\nosv
% \def\arraystretch{0.8}
% \centerline{\begin{tabular}[b]{|c|}\hline
%     \hbox to.9\textwidth{}\\
%     \parbox{.8\textwidth}{
% 	This \texttt{f}(loat) region could be extended to both side edges
%	and the top edge of the paper if its extension were
%	\texttt{(10000pt,10000pt)(10000pt,-4pt)}.}\\
%     \\\hline
%     \end{tabular}}
% \caption{A Page-Wise Figure \emph{Imported} from Pre-Environment}
% \label{fig:bgpaint-inf}
% \end{figure}
% 
% \begin{paracol}{2}
% Though you cannot see, the right edge of this over-painted |c[0]| region
% is shifted right by 4\,|pt| to hide the small patch at the right bottom
% corner of the |p| region above by overlaying.
% 
% \switchcolumn
% \begingroup\it
% As explained in the right column, this {\rm|c[1]|} region also has an
% invisible left edge shifted left by {\rm4\,|pt|}\footnote{
% 
% This (foot)|n|(ote) region could be extended to both side edges and the
% bottom edge of the paper if its extension were
% \texttt{(10000pt,-4pt)(10000pt,10000pt)}.\label{fn:bgpaint-inf2}}.
% \endgroup
% 
% \switchcolumn*[\subsection*{This \texttt{s}(panning text) region could be
% extended to both side edges of the paper if its extension were
% \texttt{(10000pt,-4pt)}.}\par\medskip]
% 
% The author does not have much to say now for this column chunk.
% \par\vfill
% 
% Still nothing to say particular to the page break we will have shortly.
% \par\newpage
% 
% This paragraph is just for keeping the \env{paracol} environment alive in
% this page.
% \switchcolumn
% 
% \begingroup\it
% Little to say as well.
% \par\vfill
% 
% Nothing to say as well.
% \par\newpage
% 
% This paragraph is not necessary for keeping alive the environment but is
% given for consistent view.
% \endgroup
% 
% \begin{figure*}\nosv
% \def\arraystretch{0.8}
% \centerline{\begin{tabular}[b]{|c|}\hline
%     \hbox to.9\textwidth{}\\
%     \parbox{.8\textwidth}{
% 	This figure is given in the \env{paracol} environment closed in the
%	previous page but its background is not painted.}\\
%     \\\hline
%     \end{tabular}}
% \caption{A Page-Wise Figure \emph{Exported} to Post-Environment}
% \label{fig:bgpaint-inf2}
% \end{figure*}
% \end{paracol}
% \bigskip
% 
% Note that overlay painting is inevitable for two-toned page painting, as
% far as you want to paint \bground{} of \pwstuff.
% 
% The last issue of \bgpaint{} is about painting materials given outside
% \env{paracol}.  As you have seen, \Preenv{} and \postenv{} are painted but
% it is done only when they reside in a page having a portion of a
% \env{paracol} environment (maybe) of course.  Therefore, the next page is
% \emph{not} painted because the page does not have any parallel-columned
% stuff.  Therefore, even if you wish to paint the whole of your document
% including pages without \env{paracol} stuff, you cannot do it just with
% \Paracol{} package, at least so far.
% 
% On the other hand, some materials given outside \env{paracol} environments
% are painted as if they are given in the environment when they are
% \emph{imported} into the environment.  One category has footnotes given in
% \preenv{} when \!\footnotelayout!|{m}| is specified for merging, as
% exemplified by the footnote \ref{fn:bgpaint-inf1} in the previous page.
% Note that such a footnote is painted by the color for |n| region rather
% than |p| region even when there are no footnotes in the \env{paracol}
% environment.  The other category has ordinary floats given by \env{figure}
% and/or \env{table}
% (i.e., neither \env{figure*} nor \env{table*}) environments outside
% \env{paracol} and then \emph{deferred} to a page having (a portion of)
% stuff produced by \env{paracol}.  Since such a float, e.g.,
% Figure\Tie\ref{fig:bgpaint-inf} in this page, is considered as a page-wise
% float given in the \env{paracol} environment in this section, its
% background is painted by the color for the |f| region, rather than that
% for the |p| region which would be used if the float were is placed in the
% previous page.  Note that such a deferred float import could occur not
% only from the page having \beginparacol{} but also from pages preceding
% it.  For example, if you have three \env{figure} environments in a page
% $p-1$ just preceding the page $p$ in which you start a \env{paracol}
% environment, it could happen that first one is placed in $p-1$ without
% painting, the second is placed in $p$ and painted by the color for |p|,
% and the third is placed in $p+1$ and painted by the color for |f|.
% 
% Finally some materials \emph{exported} from a \env{paracol} environment
% are painted as if they are in \postenv.  In previous two subsections, we
% saw \Mgfnote{}s (e.g., \ref{fn:bgpaint1} in p.\Tie\pageref{fn:bgpaint1}
% and \ref{fn:bgpaint-me1} in p.\Tie\pageref{fn:bgpaint-me1}) are painted by
% the color of |p| rather than |n|.  The other kind of exportation is of
% page-wise floats given in a \env{paracol} environment but deferred to the
% page next to the page having \Endparacol, or further.  For example,
% Figure~\ref{fig:bgpaint-inf2} is given in the \env{paracol} environment
% above in this page, but its \bground{} is not painted because the next page
% in which the figure is placed does not have any parallel-columned
% stuff\footnote{
% 
% If it has, the background is painted by the color for |p|.}.
% 
% \newpage\vspace*{\fill}
% \centerline{(intentionally blanked page to show this page is \emph{not}
% painted)}
% \vfill
% \advance\skip\footins-4pt\relax
% \endinput
