% \iffalse meta-comment
%
% Copyright 2013-2016 by Jonas L. Dabelow
% 
% Dieses Werk darf nach den Bedingungen der LaTeX Project Public License,
% entweder Version 1.3c oder (nach Ihrer Wahl) jeder spaeteren Version, 
% verteilt und/oder veraendert werden.
% Die neueste Version dieser Lizenz ist:
% 
% http://www.latex-project.org/lppl.txt
% 
% Dieses Werk hat den LPPL-Betreuungs-Status "author-maintained" (vom Autor betreut).
% 
% Dieses Werk besteht aus den Dateien sr-vorl.dtx and sr-vorl.ins
% and the derived files sr-vorl.cls, frontmatter_sr-vorl.tex,
% mainmatter_sr-vorl.tex, backmatter_sr-vorl.tex und hauptdatei_sr-vorl.tex.
% 
% 
% 
% This work may be distributed and/or modified under the
% conditions of the LaTeX Project Public License, either version 1.3c
% of this license or (at your option) any later version.
% The latest version of this license is in:
% 
% http://www.latex-project.org/lppl.txt
% 
% This work has the LPPL maintenance status "author-maintained".
% 
% This work consists of the files sr-vorl.dtx and sr-vorl.ins
% and the derived files sr-vorl.cls, frontmatter_sr-vorl.tex,
% mainmatter_sr-vorl.tex, backmatter_sr-vorl.tex and hauptdatei_sr-vorl.tex.
% 
%
%<*driver>
	\ProvidesFile{sr-vorl.dtx}
%</driver>
%<class>\NeedsTeXFormat{LaTeX2e}
%<class>\ProvidesClass{sr-vorl}
%<*class>
	[2016/01/30 v1.1 Vorlage fuer Autoren bei Springer Research]
%</class>
%
%<*driver>
%
\documentclass[a4paper]{ltxdoc}
\usepackage{lmodern}
\usepackage[T1]{fontenc}
\usepackage[utf8]{inputenc}
\usepackage[british,ngerman]{babel}
\usepackage{etoolbox}
\newtoggle{auxdatei}
\IfFileExists{\jobname.aux}% Abfrage zur Vermeidung von biber-Fehlern beim ersten Durchlauf
	{\global\toggletrue{auxdatei}}
	{\global\togglefalse{auxdatei}}
\iftoggle{auxdatei}
	{\usepackage[sortcites=true,maxnames=5,backref=true,backend=biber,maxnames=10]{biblatex}}
	{}
\usepackage{ragged2e}
\usepackage[german=quotes,autostyle=true]{csquotes}
\usepackage{microtype}
\usepackage{filecontents}
\usepackage{scrextend}
\usepackage{doc}
\usepackage{hyperref}
\usepackage[german]{cleveref}


\hypersetup{%
	colorlinks=true,
	bookmarksnumbered=true,
	bookmarksopen=true,
	pdftoolbar=true,
	pdfmenubar=true,
	pdffitwindow=true,
	pdfdisplaydoctitle=true,
	pdflang=ngerman,
	pdfcenterwindow=true,
	pdfpagelayout=SinglePage,
	pdftitle={Vorlage für Springer Research und Springer VS - Dokumentation},
	pdfsubject={Vorlage für Springer Research und Springer VS},
	pdfkeywords={LaTeX, Springer Research, Springer VS, Vorlage, Template, Klasse},
	pdfauthor={Jonas L. Dabelow},
	pdfcreator={TeXMaker mit  TeXLive},
	linkcolor=blue,
	urlcolor=red,
}


\newcommand{\env}[1]{\texttt{#1}}% Umgebungsname im Fließtext (ohne Verlinkung)
\newcommand{\fmacro}[1]{\texttt{\textbackslash{}#1}}% Makro im Fliesstext (ohne Verlinkung)
\newcommand{\farg}[1]{\ensuremath{\langle}\textit{#1}\ensuremath{\rangle}}% freies Argument (ohne Klammern)
\newcommand{\sarg}[1]{\texttt{\{#1\}}}% spezielles Argument in geschweiften Klammern (kein Meta-Argument!)
\newcommand{\soarg}[1]{\texttt{[#1]}}% spezielles optionales Argument in geschweiften Klammern (kein Meta-Argument!)
\newcommand{\macropar}{\par\medskip}% Absatz nach jedem Makro
\newcommand{\descriptionpar}{\par\smallskip}% Absatz nach \macrodescription
\newcommand{\datei}[1]{\textit{#1}}% setzt einen Dateinamen
\newcommand{\option}[1]{\texttt{#1}}% setzt einen Optionsnamen
\newcommand{\programm}[1]{\texttt{#1}}% setzt einen Programmnamen (z. B. pdflatex)




\newcommand{\envdescription}[2][]{% Beschreibung einer Umgebung (mit automatischer Verlinkung)
	\DescribeMacro{#2-\textrm{Umgebung}}%
	\ifblank{#1}%
		{\hypertarget{env:#2}{}}%
		{\hypertarget{env:#1}{}}%
}


\newcommand{\macrodescription}[2][]{% Beschreibung eines Makros (mit automatischer Verlinkung)
	{%
		\def\PrintDescribeMacro#1{\strut \MacroFont \textbackslash#1}%
		\DescribeMacro{#2}%
		\ifblank{#1}%
			{\hypertarget{macro:#2}{}}%
			{\hypertarget{macro:#1}{}}%
	}%
}

\newcommand{\optdescription}[2][]{% Beschreibung einer Option (mit automatischer Verlinkung)
	\DescribeMacro{#2}%
	\ifblank{#1}%
		{\hypertarget{option:#2}{}}%
		{\hypertarget{option:#1}{}}%
}

\newcommand{\macroref}[2][]{% Referenzierung auf ein Makro
	\ifblank{#1}%
		{\hyperlink{macro:#2}{\fmacro{#2}}}%
		{\hyperlink{macro:#2}{#1}}%
}

\newcommand{\envref}[2][]{% Referenzierung auf eine Umgebung
	\ifblank{#1}%
		{\hyperlink{env:#2}{\env{#2}}}%
		{\hyperlink{env:#2}{#1}}%
}

\newcommand{\optref}[2][]{% Referenzierung auf eine Umgebung
	\ifblank{#1}%
		{\hyperlink{option:#2}{\env{#2}}}%
		{\hyperlink{option:#2}{#1}}%
}

\newcommand{\counter}[1]{\texttt{#1}}% Beschreibung eines Counters

\newcommand{\paket}[2][]{% Paketnamen setzen (opt. Arg. kann Link auf Doku enthalten)
	\ifblank{#1}%
		{\texttt{#2}}%
		{\href{#1}{\texttt{#2}}}%
}

% #1: Versionsnummer
% #2: was wurde geaendert?
% #3: Kommentar
\newcommand{\aenderung}[3]{% Aenderung fuer Liste der Aenderungen
	\item[#2~(Version #1)]~\\
	#3%
}

\setlength{\parindent}{1em}

\iftoggle{auxdatei}
	{%
		\defbibnote{literatur}{%
			\addcontentsline{toc}{section}{Literatur}%
			Die meisten Einträge in diesem Literaturverzeichnis verweisen auf \LaTeX-Pakete.
			Bitte beachten Sie, dass je nach Version der Pakete das Verhalten auf Ihrem Rechner von dem in den Dokumentationen beschriebenen leicht abweichen kann oder in der Dokumentation beschriebene Befehle auf Ihrem System (noch) nicht verfügbar sein können.\par
			Soweit möglich sind für englische Text hier die deutschen Übersetzungen aufgeführt.
			In einigen Fällen kann es passieren, dass die Übersetzung nicht die aktuellste Version wiedergibt.
		}
	}
	{}

\begin{filecontents}{sr-vorl.bib}
	@Book{latex-begleiter,
	author = {F. Mittelbach and M. Goossens and J. Braams and D. Carlisle and C. Rowley},
	title = {Der \LaTeX Begleiter},
	publisher = {Pearson Studium},
	address = {München},
	edition = {2.~Auf"|lage},
	year = {2005},
	}
	
	@online{l2tabu,
	author = {M. Ensenbach and M. Trettin},
	title = {Das \LaTeX2$_\varepsilon$-Sündenregister},
	year = {2011},
	url = {http://ftp.fernuni-hagen.de/ftp-dir/pub/mirrors/www.ctan.org/info/l2tabu/german/l2tabu.pdf},
	urldate = {2016-01-30},
	}
	
	@online{l2kurz,
	author = {M. Daniel and P. Gundlach and W. Schmidt and J. Knappen and H. Partl and I. Hyna},
	title = {\LaTeX2$_\varepsilon$-Kurzbeschreibung},
	year = {2013},
	url = {ftp://ftp.rrzn.uni-hannover.de/pub/mirror/tex-archive/info/lshort/german/l2kurz.pdf},
	urldate = {2016-01-30},
	}
	
	@Book{koma,
	author = {M. Kohm},
	title = {\textsf{KOMA}-Script},
	publisher = {Lehmanns Media},
	year = {2014},
	edition = {5.~Auf"|lage},
	address = {Berlin},
	url = {ftp://ftp.tu-chemnitz.de/pub/tex/macros/latex/contrib/koma-script/doc/scrguide.pdf},
	urldate = {2016-01-30},
	}
	
	@online{ctan,
	author = {},
	title = {Comprehensive \TeX\ Archive Network},
	year = {1993},
	url = {http://www.ctan.org/},
	urldate = {2016-01-30},
	}
	
	@online{caption,
	author = {A. Sommerfeldt},
	title = {Anpassen der Abbildungs- und Tabellenbeschriftungen},
	year = {2011},
	url = {ftp://ftp.fu-berlin.de/tex/CTAN/macros/latex/contrib/caption/caption-deu.pdf},
	urldate = {2016-01-30},
	}
	
	@online{geometry,
	author = {H. Umeki},
	translator = {H.-M. Haase},
	title = {Das \paket{geometry} Paket},
	year = {2010},
	url = {http://ftp.fernuni-hagen.de/ftp-dir/pub/mirrors/www.ctan.org/macros/latex/contrib/geometry-de/geometry-de.pdf},
	urldate = {2016-01-30},
	}
	
	@online{enumitem,
	author = {J. Bezos},
	translator = {M. Ludwig and C. Römer},
	title = {Anpassen von Listen mit dem Paket \paket{enumitem}},
	year = {2011},
	url = {ftp://ftp.mpi-sb.mpg.de/pub/tex/mirror/ftp.dante.de/pub/tex/info/translations/enumitem/de/enumitem-de.pdf},
	urldate = {2016-01-30},
	}
	
	@online{amsldoc,
	author = {American Mathematical Society},
	title = {User's Guide for the \paket{amsmath} Package},
	year = {2002},
	url = {ftp://ftp.mpi-sb.mpg.de/pub/tex/mirror/ftp.dante.de/pub/tex/macros/latex/required/amslatex/math/amsldoc.pdf},
	urldate = {2016-01-30},
	}
	
	@online{wikibooks-floats,
	author = {Wikibooks},
	title = {\LaTeX/Floats, Figures and Captions},
	url = {http://en.wikibooks.org/wiki/LaTeX/Floats,_Figures_and_Captions},
	urldate = {2016-01-30},
	}
	
	@online{float,
	author = {A. Lingnau},
	title = {An Improved Environment for Floats},
	year = {2001},
	url = {http://ftp.fau.de/ctan/macros/latex/contrib/float/float.pdf},
	urldate = {2016-01-30},
	}
	
	@Book{texbook,
	author = {D. E. Knuth},
	title = {The \TeX book},
	publisher = {American Mathematical Society},
	year = {1989},
	edition = {9.~Auf"|lage},
	}
	
	@online{rotating,
	author = {R. Fairbairns and S. Rahtz and L. Barroca},
	title = {A package for rotated objects in \LaTeX},
	year = {2010},
	url = {ftp://ftp.rrzn.uni-hannover.de/pub/mirror/tex-archive/macros/latex/contrib/rotating/rotating.pdf},
	urldate = {2016-01-30},
	}
	
	@Book{voss-mathe,
	author = {H. Voß},
	title = {Mathematiksatz mit \LaTeX},
	publisher = {Lehmanns Media},
	year = {2009},
	address = {Berlin},
	edition = {1.~Auf"|lage},
	}
	
	@online{voss-mathe-online,
	author = {H. Voß},
	title = {Math Mode},
	year = {2014},
	url = {http://sunsite.informatik.rwth-aachen.de/ftp/pub/mirror/ctan/info/math/voss/mathmode/Mathmode.pdf},
	urldate = {2016-01-30},
	}
	
	@online{microtype,
	author = {R. Schlicht},
	title = {Das \paket{microtype} Paket},
	url = {http://ftp.gwdg.de/pub/ctan/info/translations/microtype/de/microtype-DE.pdf},
	urldate = {2016-01-30},
	}
	
	@online{babel,
	author = {J. Braams and J. Bezos},
	title = {Babel},
	year = {2016},
	url = {ftp://ftp.fu-berlin.de/tex/CTAN/macros/latex/required/babel/base/babel.pdf},
	urldate = {2016-01-30},
	}
	
	@online{onlyamsmath,
	author = {H. Harders},
	title = {The \paket{onlyamsmath} package},
	year = {2012},
	url = {http://ftp.fau.de/ctan/macros/latex/contrib/onlyamsmath/onlyamsmath.pdf},
	urldate = {2016-01-30},
	}
	
	@online{wikibooks-silbentrennung,
	author = {Wikibooks},
	title = {\LaTeX-Wörterbuch: Silbentrennung},
	url = {http://de.wikibooks.org/wiki/LaTeX-Wörterbuch:_Silbentrennung},
	urldate = {2016-01-30},
	}
	
	@online{wikibooks-latex,
	author = {Wikibooks},
	title = {\LaTeX},
	url = {http://en.wikibooks.org/wiki/LaTeX},
	urldate = {2016-01-30},
	}
	
	@online{booktabs,
	author = {S. Fear},
	translator = {T. Manderla and C. Römer},
	title = {Anfertigen von hochwertigen Tabellen mit \LaTeX},
	year = {2011},
	url = {ftp://ftp.rrzn.uni-hannover.de/pub/mirror/tex-archive/macros/latex/contrib/booktabs-de/booktabs-de.pdf},
	urldate = {2016-01-30},
	}
	
	@online{csquotes,
	author = {P. Lehman},
	translator = {T. Conrad and P. Faßbender and C. Römer},
	title = {Das \paket{csquotes}-Paket},
	year = {2011},
	url = {ftp://ftp.rrzn.uni-hannover.de/pub/mirror/tex-archive/info/translations/csquotes/de/csquotes-DE.pdf},
	urldate = {2016-01-30},
	}
	
	@online{cleveref,
	author = {T. Cubitt},
	title = {The \paket{cleveref} package},
	year = {2013},
	url = {http://ftp.uni-erlangen.de/ctan/macros/latex/contrib/cleveref/cleveref.pdf},
	urldate = {2016-01-30},
	}
	
	@online{varioref,
	author = {F. Mittelbach},
	title = {The \paket{varioref} package},
	year = {2015},
	url = {http://sunsite.informatik.rwth-aachen.de/ftp/pub/mirror/ctan/macros/latex/required/tools/varioref.pdf},
	urldate = {2016-01-30},
	}

	@online{siunitx,
	author = {J. Wright},
	title = {\paket{siunitx} --~A comprehensive (SI) units package},
	year = {2016},
	url = {ftp://ftp.rrzn.uni-hannover.de/pub/mirror/tex-archive/macros/latex/contrib/siunitx/siunitx.pdf},
	urldate = {2016-01-30},
	}

	@online{etoolbox,
	author = {P. Lehman},
	translator = {T. Enderling},
	title = {Das Paket \paket{etoolbox}},
	year = {2011},
	url = {ftp://ftp.fu-berlin.de/tex/CTAN/info/translations/etoolbox/de/etoolbox-DE.pdf},
	urldate = {2016-01-30},
	}
	
	@online{ragged2e,
	author = {M. Schröder},
	title = {The \paket{ragged2e}-package},
	year = {2009},
	url = {http://ftp.fernuni-hagen.de/ftp-dir/pub/mirrors/www.ctan.org/macros/latex/contrib/ms/ragged2e.pdf},
	urldate = {2016-01-30},
	}
	
	@online{juergens1,
	author = {M. Jürgens},
	title = {\LaTeX\ --~eine Einführung und ein bisschen mehr\dots},
	year = {2000},
	url = {http://www.wiwiss.fu-berlin.de/fachbereich/vwl/iso/links/latex_einfuehrung_manuela_juergens.pdf},
	urldate = {2016-01-30},
	}

	@online{juergens2,
	author = {M. Jürgens},
	title = {\LaTeX\ --~Fortgeschrittene Anwendungen},
	year = {1995},
	url = {ftp://ftp.fernuni-hagen.de/pub/pdf/urz-broschueren/broschueren/a0279510.pdf},
	urldate = {2016-01-30},
	}
	
	@online{buxbaum-presentation,
	author = {E. Buxbaum},
	title = {The \LaTeX\ document preparation system},
	year = {2008},
	url = {ftp://ftp.rrzn.uni-hannover.de/pub/mirror/tex-archive/info/latex-course/LaTeX-Course.pdf},
	urldate = {2016-01-30},
	}
	
	@online{simplified-introduction,
	author = {H. J. Greenberg},
	title = {A Simplified Introduction to \LaTeX},
	year = {2010},
	url = {http://sunsite.informatik.rwth-aachen.de/ftp/pub/mirror/ctan/info/simplified-latex/simplified-intro.pdf},
	urldate = {2016-01-30},
	}
	
	@online{beginners-introduction,
	author = {P. Flynn},
	title = {A beginner's introduction to typesetting with \LaTeX},
	year = {2005},
	url = {http://sunsite.informatik.rwth-aachen.de/ftp/pub/mirror/ctan/info/beginlatex/beginlatex-3.6.pdf},
	urldate = {2016-01-30},
	}
	
	@Book{more-math,
	author = {G. Grätzer},
	title = {More Math Into \LaTeX},
	publisher = {Springer},
	year = {2007},
	address = {New York},
	edition = {4.~Auf"|lage},
	url = {http://ctan.space-pro.be/tex-archive/info/Math_into_LaTeX-4/Short_Course.pdf},
	urldate = {2016-01-30},
	}
	
	@online{xkeyval,
	author = {H. Adriaens},
	title = {The \paket{xkeyval} package},
	year = {2014},
	url = {http://ftp.gwdg.de/pub/ctan/macros/latex/contrib/xkeyval/xkeyval.pdf},
	urldate = {2016-01-30},
	}
	
	@online{xstring,
	author = {C. Tellechea},
	title = {\paket{xstring}},
	year = {2013},
	url = {http://sunsite.informatik.rwth-aachen.de/ftp/pub/mirror/ctan/macros/generic/xstring/xstring_doc_en.pdf},
	urldate = {2016-01-30},
	}
	
	@online{varwidth,
	author = {D. Arseneau},
	title = {The \paket{varwidth} package},
	year = {2011},
	url = {ftp://ftp.mpi-sb.mpg.de/pub/tex/mirror/ftp.dante.de/pub/tex/macros/latex/contrib/varwidth/varwidth-doc.pdf},
	urldate = {2016-01-30},
	}
	
	@online{chngcntr,
	author = {P. Wilson},
	title = {The \paket{chngcntr} package},
	year = {2009},
	url = {http://ctan.space-pro.be/tex-archive/macros/latex/contrib/chngcntr/chngcntr.pdf},
	urldate = {2016-01-30},
	}
\end{filecontents}

\iftoggle{auxdatei}
	{\addbibresource{sr-vorl.bib}}
	{}

\clubpenalty=10000
\widowpenalty=10000

\DisableCrossrefs
%\CodelineIndex
\RecordChanges
\OnlyDescription
\begin{document}


\title{sr-vorl.cls\\[1ex]\normalsize Eine \LaTeX-Klasse für Bücher bei Springer Research (Springer Gabler Research, Springer Vieweg Research, Springer Spektrum Research) und Springer VS (Springer VS Research, Springer VS Forschung)
}
\author{Copyright 2013--2016 by Jonas~L.~Dabelow\footnote{\href{mailto:J.L.Dabelow@gmx.net}{\texttt{J.L.Dabelow@gmx.net}}}}
\date{Version~1.1 (31.\,Januar~2016)}
\maketitle


\pdfbookmark[1]{Inhaltsverzeichnis}{tableofcontents}
\tableofcontents
\newpage
\DocInput{sr-vorl.dtx}

\end{document}

%</driver>
%\fi
%
%
%\GetFileInfo{sr-vorl.dtx}
%
%
% \section{Einleitung}
% Diese \LaTeX-Klasse wurde erstellt, um Bücher zu setzen, die bei Springer Research (\textbf{Springer Gabler Research}, \textbf{Springer Vieweg Research}, \textbf{Springer Spektrum Research}) oder Springer~VS (\textbf{Springer VS Research}, \textbf{Springer VS Forschung}) veröffentlicht werden sollen (Beachte hierzu auch die Klassenoption \optref{format}!).
% Mit dieser Klasse werden viele Gestaltungsvorgaben des Verlags bereits umgesetzt.
% Der Autor der Klasse gibt jedoch keine Garantie, dass damit alle Vorgaben bereits umgesetzt sind.
% Dies ist in einigen Fällen auch nicht möglich, da die Autoren eines Buches einige Dinge \enquote{von Hand} umsetzen müssen, da \LaTeX\ hierfür keine vollständige Automatisierung ermöglicht (z.\,B. Silbentrennung, Platzierung von Gleitobjekten\dots).
% In den meisten Fällen ist der Autor dieser Klasse jedoch bemüht, hilfreiche Hinweise zu geben, wie eine solche händische Umsetzung sinnvoll erfolgen kann (siehe \cref{sec:hinweise}).\par
% In dieser Dokumenation werden zunächst in \cref{sec:befehle} die Befehle und Umgebungen vorgestellt, die diese Klasse bereitstellt.
% In \cref{sec:optionen} werden die Einstellungen gezeigt, die mit Hilfe von Klassenoptionen vorgenommen werden können.
% Weiterführende Hinweise zur Umsetzung der Gestaltungsvorgaben werden in \cref{sec:hinweise} gegeben.
% Weiterführende Informationen zu dieser Dokumentenklasse sowie weitere Tipps und Hinweise finden sich in \cref{sec:weiterfuehrendes}.
% In \cref{sec:lizenz} finden Sie zudem die Lizenz, unter der diese Dokumentation, die zugehörige Klasse und alle weiteren dazu gehörenden Dateien stehen.\par
% Dieses Dokument ist keine Einführung in \LaTeX!
% Es werden grundlegende \LaTeX-Kenntnisse vorausgesetzt (siehe auch \cref{sec:literaturtipps}).\par
% Da diese \LaTeX-Klasse noch in der Entwicklung ist, freut sich der Autor grundsätzlich über Hinweise, Fehlerreports und Feature-Anfragen (unabhängig davon, ob er sie umsetzt oder nicht).
%
% 
% \subsection{Nutzung der Klasse}
% Sie können diese Klasse wie jede andere \LaTeX-Klasse auch nutzen.
% Binden Sie sie einfach mit dem Befehl \fmacro{documentclass}\oarg{Optionen}\sarg{sr-vorl} ein.
% Dazu muss die Datei \datei{sr-vorl.cls} in einem Ordner liegen, der von \LaTeX\ durchsucht wird (z.\,B. der Ordner, in dem sich Ihre Hauptdatei befindet).\par
% Sie können zusätzlich entweder die bereitgestellten Dateien \datei{hauptdatei\_sr-vorl.tex}, \datei{frontmatter\_sr-vorl.tex} (für den Vorspann), \datei{mainmatter\_sr-vorl.tex} (für den Inhalt) und \datei{backmatter\_sr-vorl.tex} (für den Anhang) verwenden oder die Klasse in Verbindung mit Ihren eigenen Dateien nutzen.
% In diesem Fall sollten Sie zur Gliederung des Dokuments auch die \textsf{KOMA}-Script-Befehle \fmacro{frontmatter}, \fmacro{mainmatter} und \fmacro{backmatter} für Vorspann, Hauptteil und Nachspann Ihres Dokuments verwenden \cite{koma}, da diese Befehle wichtige Funktionen erfüllen (z.\,B. Umschalten der Seitennummerierung).\par
% Sollten Sie keine der genannten Dateien haben, sondern lediglich die Dateien \datei{sr-vorl.dtx} und \datei{sr-vorl.ins}, ist dies kein Problem.
% Sie müssen lediglich die Datei \datei{sr-vorl.ins} mit \programm{tex}, \programm{latex} oder \programm{pdflatex} kompilieren, um die erwähnten Dateien zu erzeugen (siehe \cref{sec:erzeugung}).\par
% Diese Klasse wurde für den Einsatz mit \programm{pdflatex} optimiert.
% Sie ist prinzipiell auch mit \programm{latex} nutzbar (beachte dazu auch \cref{sec:querformat}), der Autor empfiehlt aber dringend den Einsatz von \programm{pdflatex}.
% Die Nutzung mit anderen Treibern wie \programm{xetex}, \programm{luatex} oder \programm{context} ist nicht getestet!
% 
% \subsection{Erzeugung der Klassendatei und Hilfsdateien sowie der Dokumentation (falls notwendig)}
% \label{sec:erzeugung}
% Die vorliegende Klasse inklusive aller zugehörigen Dateien kann --~wie üblicherweise jede \LaTeX-Klasse~-- aus einer \datei{.dtx}- und einer \datei{.ins}-Datei erzeugt werden.\par
% Sofern Sie keinen Zugriff auf die Klassendatei \datei{sr-vorl.cls}, aber auf die Dateien \datei{sr-vorl.dtx} und \datei{sr-vorl.ins} haben, können Sie die Klassendatei erzeugen, indem Sie \datei{sr-vorl.ins} mit \programm{tex}, \programm{latex} oder \programm{pdflatex} kompilieren:\\
% \texttt{tex sr-vorl.ins}\\
% oder\\
% \texttt{latex sr-vorl.ins}\\
% oder\\
% \texttt{pdflatex sr-vorl.ins}\\
% Durch einen (beliebigen) dieser Aufrufe werden die Dateien \datei{sr-vorl.cls}, \datei{frontmatter\_sr-vorl.tex}, \datei{mainmatter\_sr-vorl.tex}, \datei{backmatter\_sr-vorl.tex} und \datei{hauptdatei\_sr-vorl.tex} erzeugt.\par
% Um diese Dokumentation zu erzeugen, müssen Sie die Datei \datei{sr-vorl.dtx} mit \programm{pdflatex} kompilieren: \texttt{pdflatex sr-vorl.dtx}\\
% Für die vollständige Dokumentation benötigen Sie auch \programm{biber}.
% Es ergibt sich zur Erzeugung der vollständigen Dokumentation der folgende Ablauf:\\
% \texttt{pdflatex sr-vorl.dtx}\\
% \texttt{pdflatex sr-vorl.dtx}\\
% \texttt{biber sr-vorl.bcf}\\
% \texttt{pdflatex sr-vorl.dtx}\\
% \texttt{pdflatex sr-vorl.dtx}\par
% Um \datei{sr-vorl.dtx} kompilieren zu können, müssen die Klasse \paket{ltxdoc} sowie die folgenden Pakete auf Ihrem System verfügbar sein: \paket{lmodern}, \paket{fontenc}, \paket{inputenc}, \paket{babel}, \paket{etoolbox}, \paket{biblatex}, \paket{ragged2e}, \paket{csquotes}, \paket{microtype}, \paket{filecontents}, \paket{scrextend}, \paket{doc}, \paket{hyperref}, \paket{cleveref}.
%
% \section{Befehle und Umgebungen}
% \label{sec:befehle}
% Diese Klasse basiert auf der Klasse \paket{scrbook} aus dem \textsf{KOMA}-Script-Bündel \cite{koma}.
% Alle Befehle und Umgebungen, die von \textsf{KOMA}-Script bereit gestellt werden, können hier genutzt werden.\par
% Zusätzlich sind einige hilfreiche Befehle und Umgebungen in dieser Klasse implementiert worden.\par\bigskip
%
% \subsection{Der Vorspann}
% \label{sec:vorspann}
% \noindent\envdescription{widmung}\texttt{\fmacro{begin}\{widmung\}}\oarg{Überschrift}\texttt{\dots}\descriptionpar
% In dieser Umgebung wird die Widmung gesetzt.
% Sie wird rechtsbündig, in kursiver Schrift auf eine rechte Seite gesetzt.
% Vertikal ist sie so ausgerichtet, dass das Verhältnis von oberem und unterem Abstand den Goldenen Schnitt bildet.\par
% Als optionales Argument kann eine Überschrift angegeben werden (die ebenfalls rechtsbündig gesetzt wird).
% Wir kein optionales Argument angegeben, so erhält die Widmung keine Überschrift.\macropar
%
% \pagebreak
% \macrodescription{geleitwort}%
% Dieser Befehl setzt lediglich die Überschrift \enquote{Geleitwort}.
% Die Sprache der Überschrift wird anhand der gewählten Klassenoption ausgewählt (siehe \cref{sec:optionen}).
% Die Überschrift kann verändert werden, indem der Befehl \fmacro{GeleitwortTitel} umdefiniert wird.
% Dies muss nach der Präambel gemacht werden.\macropar
%
% \noindent\macrodescription{verfasser}\oarg{Ortbreite, Namebreite}\marg{Ort}\marg{Name}\descriptionpar
% Mit diesem Befehl wird am Ende des Geleitwortes der Verfasser desselben sowie der Ort, an dem es geschrieben wurde, gesetzt.
% Der Befehl steht direkt hinter dem Geleitwort, ein vertikaler Abstand wird automatisch eingefügt.
% Der Ort wird linksbündig nach links gesetzt, der Verfasser rechtsbündig nach rechts.\par
% Dabei ist die Breite des Ortes standardmäßig bis zu 40\,\% der Zeilenbreite, die des Verfassers bis zu 50\,\% der Zeilenbreite.
% Dies kann mit Hilfe des optionalen Arguments geändert werden.
% Hier müssen beide Längen durch ein Komma getrennt angegeben werden.
% Es empfiehlt sich --~wie in vielen Fällen in \LaTeX~--, die Längen relativ zur Zeilenlänge zu wählen.
% (Die Standardeinstellung wird z.\,B. durch \verb+[0.4\linewidth,0.5\linewidth]+ realisiert.)\par
% Übersteigen die Längen in der Summe die Zeilenbreite, so erscheint eine Warnung.\macropar
%
% \macrodescription{vorwort}%
% Dieser Befehl setzt lediglich die Überschrift \enquote{Vorwort}.
% Die Sprache der Überschrift wird anhand der gewählten Klassenoption ausgewählt (siehe \cref{sec:optionen}).
% Die Überschrift kann verändert werden, indem der Befehl \fmacro{VorwortTitel} umdefiniert wird.
% Dies muss nach der Präambel gemacht werden.\macropar
%
% \subsection{Der Hauptteil}
% \label{sec:hauptteil}
% \noindent\macrodescription{kapitel}\oarg{Titel im Inhaltsverzeichnis}\marg{Titel}\marg{Autor(en)}\descriptionpar
% Dieser Befehl ist für die Verwendung in Mehrautorenbüchern (siehe Klassenoption \optref{format}).
% Er ist dort statt des Standard-\LaTeX-Befehls \fmacro{chapter} zu nutzen!
% Die ersten beiden Argumente funktionieren wie die von \fmacro{chapter}.
% Im dritten Argument wird der Autor bzw. die Autoren des Kapitels angegeben.
% Es erscheint unterhalb der Kapitelüberschrift.\macropar
%
% \section{Klassenoptionen}
% \label{sec:optionen}
% Die vorliegende \LaTeX-Klasse ist für deutsch- und englischsprachige Dokumente entwickelt worden.
% Damit die Klasse nutzbar ist, ist es empfohlen --~wenn auch nicht zwingend notwendig~--, eine der Sprachen als Option anzugeben.\par\bigskip
%
% \optdescription{deutsch}
% Diese Option nimmt alle Sprachvoreinstellungen für ein deutsches Dokument mit neuer Rechtschreibung vor.
% Mit Hilfe des \paket{babel}-Pakets wird \option{ngerman} als Sprache eingestellt \cite{babel}.\descriptionpar
%
% \optdescription{deutsch-ar}
% Diese Option nimmt alle Sprachvoreinstellungen für ein deutsches Dokument mit alter Rechtschreibung vor.
% Mit Hilfe des \paket{babel}-Pakets wird \option{german} als Sprache eingestellt \cite{babel}.\descriptionpar
%
% \optdescription{englisch}
% Diese Option nimmt alle Sprachvoreinstellungen für ein englisches Dokument vor.
% Mit Hilfe des \paket{babel}-Pakets wird \option{british} als Sprache eingestellt \cite{babel}.\descriptionpar
%
% \optdescription[format]{format=\farg{Buchart}}
% Mit der vorliegenden Klasse können sowohl Monografien in verschiedenen Formaten als auch Mehrautorenbücher gesetzt werden.
% Damit alle Formatierungen entsprechend angepasst werden, muss die Art des Buches mit Hilfe der \option{format}-Option angegeben werden.\par
% Folgende Angaben sind möglich: \option{a5-monografie}, \option{handbuch-monografie} und \option{a5-mehrautorenbuch} für Bücher bei \textbf{Springer Gabler Research}, \textbf{Springer Vieweg Research} und \textbf{Springer Spektrum Research} sowie \option{vs} für Bücher bei \textbf{Springer VS Research} und \textbf{Springer VS Forschung}.\par
% Für die Einstellung \option{a5-mehrautorenbuch} wird zusätzlich der Befehl \fmacro{kapitel} bereitgestellt (siehe \cref{sec:vorspann}).\par
% Wird kein Format angegeben, so wird automatisch \option{a5-monografie} eingestellt.\descriptionpar
%
% \optdescription{no-shorthands}
% Wird die Klasse mit der Option \optref{englisch} geladen, so werden automatisch deutsche \enquote{shorthands} aktiviert.
% Dies sind z.\,B. die in \cref{sec:zeilenumbrueche} beschriebenen Kurzbefehle, die mit dem Zeichen \texttt{"} (ASCII-Strichelchen) beginnen.
% Für die händische Silbentrennung ist  dies praktisch, wie in \cref{sec:zeilenumbrueche} beschrieben wird.
% Es kann allerdings in Einzelfällen zu Fehlern führen.
% Mit dieser Option lassen sich diese shorthands vollständig ausschalten.\par
% Möchte man die shorthands nur temporär ausschalten, so kann man den Befehl \fmacro{shorthandoff}\sarg{"} aus dem \paket{babel}-Paket nutzen \cite{babel}.
% Mit \fmacro{shorthandon}\sarg{"} werden sie wieder aktiviert.\descriptionpar
% 
% \optdescription{no-microtype}
% Die Klasse bindet automatisch das Paket \paket{microtype} ein, um die Mikrotypographie des Dokuments zu verbessern (siehe \cref{sec:umbrueche}) \cite{microtype}.
% Diese Option verhindert das Einbinden.\descriptionpar
%
% \optdescription{non-ams}
% Die Klasse bindet automatisch das Paket \paket{onlyamsmath} ein, das einen Fehler ausgibt, sobald eine Matheumgebung genutzt wird, die nicht von \paket{amsmath} stammt \cite{onlyamsmath}.
% Warum das sinnvoll ist, wird in \cref{sec:mathesatz} erklärt.
% Sollten Sie dennoch solche Matheumgebungen nutzen wollen, so können Sie die Fehler mit dieser Option verhindern.\descriptionpar
%
% \optdescription{test}
% Diese Option ist dazu da, den Satz eines Dokuments zu überprüfen.
% Sie schaltet einen Rahmen um den Satzspiegel ein, so dass Abweichungen davon leichter überprüft werden können.
% Zudem werden \emph{overfull boxes} (z.\,B. überlange Zeilen) \cite{texbook,latex-begleiter} durch schwarze Kästchen markiert.\descriptionpar
% 
% \optdescription{schriftgroesse=\farg{Wert}}
% Mit dieser Option kann die (Standard-)Schriftgröße Ihres Dokuments eingestellt werden.
% Die Angabe \farg{Wert} gibt die Schriftgröße in pt an.
% Diese sollte nicht kleiner als 10\,pt sein, voreingestellt sind 11\,pt.\descriptionpar
%
% \section{Gestaltungshinweise}
% \label{sec:hinweise}
% In diesem Kapitel werden Hinweise zu einigen Themen gegeben, die zwar für die Gestaltung eines Buches bei Springer Research wichtig sind, jedoch nicht automatisiert in dieser Klasse implementiert werden konnten und einige Arbeit durch den Nutzer der Klasse erfordern.
% Dieses Kapitel soll Ihnen die Arbeit erleichtern.
%
% \subsection{Abbildungen und Tabellen}
% \label{sec:gleitobjekte}
% Abbildungen und Tabellen werden in \LaTeX\ meist als sogenannte \emph{Gleitobjekte} gesetzt.
% Das bedeutet, dass diese Objekte nicht an einer festen Stelle im Text gesetzt werden, sondern \emph{gleiten} können und dort gesetzt werden, wo es für den Satz am sinnvollsten ist.\par
% \subsubsection{Platzierung von Gleitobjekten}
% \label{sec:platzierung}
% Gleitobjekte für Abbildungen werden in der Regel durch die \env{figure}-Umgebung erzeugt, solche für Tabellen durch die \env{table}-Umgebung.
% Die Platzierung der Gleitobjekte lässt sich mit Hilfe des optionalen Arguments dieser Umgebungen beeinflussen.
% Dafür gibt es folgende Möglichkeiten:
% \begin{description}
% 	\item[\texttt{h}]
% 	\LaTeX\ soll versuchen, das Gleitobjekt \textbf{h}ier zu platzieren
% 	\item[\texttt{t}]
% 	\LaTeX\ soll versuchen, das Gleitobjekt oben auf einer Seite zu platzieren (\textbf{t}op)
% 	\item[\texttt{b}]
% 	\LaTeX\ soll versuchen, das Gleitobjekt unten auf einer Seite zu platzieren (\textbf{b}ottom)
% 	\item[\texttt{p}]
% 	\LaTeX\ soll das Gleitobjekt auf einer eigenen Seite (\textbf{p}age) --~die keinerlei Fließtext, sondern nur (eines oder mehrere) Gleitobjekte enthält~-- platzieren
% \end{description}
% Die Reihenfolge der angegebenen Parameter ist unerheblich, sie werden immer in der hier aufgeführten Reihenfolge abgearbeitet:\\
% \fmacro{begin}\sarg{figure}\soarg{ht} ist das gleiche wie \fmacro{begin}\sarg{figure}\soarg{th}.\par
% Dabei beachtet \LaTeX\ einige voreingestellte Bedingungen (z.\,B. wie viel Prozent einer Seite mit Gleitobjekten gefüllt sein dürfen).
% Um diese Bedingungen temporär abzuschalten, kann zusätzlich zu den oben aufgeführten Parametern ein \enquote{\texttt{!}} genutzt werden (z.\,B. \mbox{\fmacro{begin}\sarg{figure}\soarg{ht!}})\cite{latex-begleiter,l2kurz,wikibooks-floats}.\par
% Da Gleitobjekte gleichen Typs nicht aneinander vorbei gleiten können (Gleitobjekte verschiedenen Typs aber sehr wohl!), d.\,h. die Reihenfolge der \env{table}-Umgebungen und die Reihenfolge der \env{figure}-Umgebungen bleiben erhalten, kann es manchmal sinnvoll sein, den Parameter \enquote{\texttt{p}} zu nutzen, da sonst ein großes Gleitobjekt die anderen aufhalten kann, so dass viele Gleitobjekte auf"|laufen, die dann alle erst am Ende des Kapitels gesetzt werden.\par
% Grundsätzlich sollten Abbildungen und Tabellen möglichst oben auf einer Seite erscheinen (Vorgabe von Springer Research).\par
% Manche Abbildungen oder Tabellen sollen auch nicht gleiten, sondern an einem festen Platz gesetzt werden.
% Dafür gibt es mehrere Möglichkeiten: Zum einen gibt es keinen Zwang, Abbildungen und Tabellen in Gleitumgebungen zu stecken --~was nicht gleiten soll, sollte nicht gleitbar gemacht werden!
% Zum anderen gibt es mit Hilfe des \paket{float}-Pakets die Möglichkeit, im optionalen Argument der Gleitumgebungen den Parameter \enquote{\texttt{H}} zu nutzen.
% Dieser sorgt dafür, dass das Gleitobjekt definitiv an der Stelle gesetzt wird, an der es im Text auftaucht \cite{float}.
%
% \subsubsection{Über- und Unterschriften}
% \label{sec:beschriftung}
% Abbildungen sind mit \textbf{Unter}schriften zu versehen, Tabellen mit \textbf{Über}schriften.
% Diese Klasse ist zwar so eingestellt, dass die vertikalen Abstände vor und nach den Beschriftungen passend sind, allerdings kann die Platzierung der Beschriftungen von \LaTeX\ nicht automatisch gesteuert werden!
% Nutzen Sie den Befehl \fmacro{caption} oder \fmacro{caption*} und setzen Sie diesen entsprechend \textbf{nach} einer Abbildung und \textbf{vor} einer Tabelle.\par
% Wenn Sie, wie in \cref{sec:platzierung} beschrieben, Tabellen oder Abbildungen außerhalb von Gleitumgebungen setzen, so verwenden Sie den Befehl \fmacro{captionof} bzw. \fmacro{captionof*} aus dem \paket{caption}-Paket, welches von der Klasse ohnehin geladen wird \cite{caption}.\par
% Wenn Sie den von \LaTeX\ bereitgestellten \fmacro{label}-\fmacro{ref}-Mechanismus (siehe auch \cref{sec:paketempfehlungen}) verwenden, um auf Abbildungen und Tabellen zu verweisen, was zu empfehlen ist, so müssen Sie das Label immer \textbf{nach} dem entsprechenden \fmacro{caption}- bzw. \fmacro{captionof}-Befehl setzen.
%
% \subsubsection{Abbildungen und Tabellen im Querformat}
% \label{sec:querformat}
% Für Abbildungen und Tabellen im Querformat gibt es mit \LaTeX\ verschiedene Lösungen.
% Der Autor dieser Klasse empfiehlt dafür, das Paket \paket{rotating} mit der Paketoption \option{figuresright} einzubinden und die von diesem Paket bereitgestellten Umgebungen \env{sidewaysfigure} und \env{sidewaystable} zu nutzen \cite{rotating}.\par
% Sollten Sie für Ihr Dokument statt \programm{pdflatex} den Workflow \programm{latex} \mbox{$\rightarrow$ \programm{dvips}} \mbox{$\rightarrow$ \programm{ps2pdf}} nutzen (wovon der Autor dieses Dokuments dringend abrät), so beachten Sie, dass Sie \programm{ps2pdf} mit der Option \texttt{-dAutoRotatePages=/None} aufrufen, damit die Querformat-Seiten im pdf-Dokument nicht automatisch rotiert angezeigt werden.
%
% \subsection{Seiten- und Zeilenumbrüche}
% \label{sec:umbrueche}
% \LaTeX\ (bzw. \TeX) verwendet einen sehr ausgefeilten Algorithmus, um den Zeilen- und Seitenumbruch zu optimieren \cite{texbook}.
% Allerdings muss auch beim besten Algorithmus manchmal an einigen Stellen von Hand nachgearbeitet werden.
% Um die Arbeit dabei zu erleichtern, kann die Option \optref{test} genutzt werden.
% Unabhängig davon, wie man die Umbrüche bearbeitet, sollte man dies als letzte Arbeit an einem Werk vornehmen, da sich durch andere Arbeiten, z.\,B. das Verschieben von Abbildungen, der Zeilen- und Seitenumbruch wieder verändern kann.\par
% Diese Klasse nutzt das Paket \paket{microtype}, das einen Randausgleich vornimmt, der dafür sorgt, dass der rechte Rand einer Seite (im Blocksatz) weniger \enquote{flatterig} aussieht \cite{microtype}.
% Dies geschieht, indem einige Zeichen (z.\,B. Bindestriche und Satzzeichen) minimal verschoben werden (was im Einzelfall kaum erkennbar ist, aber insgesamt einen beeindruckenden Effekt hat).
% Sollte Ihnen dies nicht gefallen, so nutzen Sie die Klassenoption \optref{no-microtype}.
%
% \subsubsection{Zeilenumbrüche und Silbentrennung}
% \label{sec:zeilenumbrueche}
% Die einfachste Möglichkeit, auf Zeilenumbrüche Einfluss zu nehmen, ist es, falsche Silbentrennung zu verbessern.
% Dafür gibt es in \LaTeX\ mehrere Möglichkeiten.\par
% Kommt ein Wort, das von \LaTeX\ falsch getrennt wird, sehr häufig im Text vor, so ist es sinnvoll, \LaTeX\ die korrekte Trennung beizubringen.
% Dies funktioniert mit dem Befehl \fmacro{hyphenation}.
% So könnte man \LaTeX\ die Trennung des Wortes \enquote{Femtosekundenlaser} folgendermaßen erklären: \fmacro{hyphenation}\sarg{Fem-to-se-kun-den-la-ser}.
% Dieses Wort wird dann nur dort getrennt, wo in dem Befehl ein Bindestrich steht.
% Dies gilt jedoch immer nur für exakt dieses Wort (nicht für abgeleitete Formen) und nur in der aktuell eingestellten Sprache \cite{babel,latex-begleiter}.\par
% Für die manuelle Trennung einzelner Wörter kann man \LaTeX\ im Einzelfall auch neue Trennregeln beibringen.
% Dies geschieht im Fließtext mit Hilfe so genannter \enquote{shorthands}:\shorthandoff{"}
% \begin{description}
% 	\item[\texttt{\textbackslash-}]
% 	Das Wort darf \textbf{nur} an dieser Stelle getrennt werden.
% 	\item[\texttt{"-}]
% 	Dies gibt eine \textbf{zusätzliche} Trennstelle an (zu denen, die \LaTeX\ für dieses Wort bereits kennt/gefunden hat).
% 	\item[\texttt{"=}]
%	Es wird ein Bindestrich erzeugt. Die übrigen Trennstellen bleiben davon unbeeinflusst.
% 	\item[\texttt{""}]
%	Dies gibt eine Stelle an, an der getrennt werden darf, ohne dass ein Bindestrich gesetzt wird.
% 	\item[\texttt{"\textasciitilde}]
%	Dies erzeugt einen Bindestrich, an dem \textbf{nicht} getrennt werden darf.
% \end{description}
% Die aufgeführten shorthands werden einfach innerhalb eines Wortes verwendet, z.\,B.: \texttt{Femto"-sekunden"=Laser} \cite{babel,latex-begleiter,wikibooks-silbentrennung}.\par\shorthandon{"}
% Der Vollständigkeit halber sei noch der Standard-\TeX-Befehl \texttt{\textasciitilde}~(Tilde) erwähnt.
% Dieser erzeugt ein Leerzeichen, an dem nicht getrennt werden darf (geschütztes Leerzeichen), was zum Beispiel bei Konstrukten wie \enquote{Seite~5} sinnvoll sein kann.\par
% Manchmal kommt man mit der einfachen Verbesserung der Silbentrennung nicht weiter, sondern muss manuell Zeilenumbrüche einfügen.
% Dabei ist es sinnvoll, \LaTeX\ noch eine gewisse Entscheidungsfreiheit zu lassen.
% Der Befehl \fmacro{linebreak} ermöglicht genau dies.
% Er zeigt \LaTeX\ an, dass an dieser Stelle ein Zeilenumbruch sinnvoll wäre.
% Mit Hilfe eines optionalen Arguments kann angegeben werden, wie sinnvoll oder dringend der Zeilenumbruch an dieser Stelle ist.
% Dieses Argument kann eine Zahl zwischen null und vier sein, wobei eine höhere Zahl den Zeilenumbruch wahrscheinlicher (dringender) macht und \fmacro{linebreak} gleichbedeutend mit \fmacro{linebreak}\soarg{4} ist.\par
% Entgegengesetzt zu \fmacro{linebreak} gibt es den Befehl \fmacro{nolinebreak}.
% Sein optionales Argument funktioniert genau komplementär zu dem von \fmacro{linebreak} (\fmacro{nolinebreak}\soarg{4} verhindert den Zeilenumbruch am ehesten).\par
% Um die Trennung eines Wortes (oder einer Fließtextformel) zu verhindern, kann der Befehl \fmacro{mbox} verwendet werden.
% Dieser nimmt ein Argument auf, das dann nicht getrennt werden kann, z.\,B. \fmacro{mbox}\sarg{\textdollar 1+1=2\textdollar}.
% Dies kann jedoch dazu führen, dass der Text über den Zeilenrand hinaus geht, was weitere Umarbeitungen erfordert.
%
%
% \subsubsection{Seitenumbrüche}
% \label{sec:seitenumbrueche}
% An manchen Stellen werden Seitenumbrüche von Hand eingefügt, weil der jeweilige Autor (oder der Lektor) dies möchte.
% Diese Seitenumbrüche werden meist mit den Befehlen \fmacro{newpage} oder \fmacro{clearpage} erzeugt, wobei der Unterschied darin besteht, dass letzterer vor dem Seitenumbruch alle noch nicht positionierten Gleitobjekte (siehe \cref{sec:gleitobjekte}) setzt.
% Analog dazu gibt es den Befehl \fmacro{cleardoublepage}, der im Unterschied zu \fmacro{clearpage} dafür sorgt, dass die neue Seite eine ungerade (rechte) Seite ist, wozu manchmal eine Leerseite (Vakatseite) eingefügt werden muss.\par
% Soll jedoch nicht unbedingt eine neue Seite angefangen werden, sondern erscheint dies sinnvoll, um den Textsatz zu optimieren, so kann es praktisch sein, \LaTeX\ zwar Hinweise zu geben, wo eine Seite umgebrochen werden kann, die Entscheidung aber dem Programm zu überlassen.
% Für diese Fälle gibt es analog zu den Befehlen \fmacro{linebreak} und \fmacro{nolinebreak} (siehe \cref{sec:zeilenumbrueche}) die Befehle \fmacro{pagebreak} und \fmacro{nopagebreak}.
% Diese funktionieren vollständig analog zueinander (inklusive optionalem Argument).\par
% Ein Grund dafür, diese Befehle zu nutzen kann die Vermeidung von \emph{Schusterjungen} und \emph{Hurenkindern} sein.
% Das ist eine einzelne Zeile eines Absatzanfangs am Ende einer Seite bzw. eine einzelne Zeile eines Absatzendes am Anfang einer Seite.
% Sie sind unter allen Umständen (notfalls Umformulierung einzelner Sätze) zu vermeiden!
% Der Autor dieser \LaTeX-Klasse versucht, Sie dabei zu unterstützen, indem er die Werte \fmacro{clubpenalty}, \fmacro{widowpenalty} und \fmacro{displaywidowpenalty} (siehe \cite{texbook,latex-begleiter}) auf 9999 gesetzt hat.
%
% 
% \subsection{Mathematiksatz}
% \label{sec:mathesatz}
% Die \LaTeX-Gemeinde ist sich heutzutage nahezu einig, dass für abgesetzte Formeln die Umgebungen verwendet werden sollten, die von der American Mathematical Society (AMS) im Paket \paket{amsmath} bereitgestellt werden, da bei diesen die Abstände am konsistentesten sind und Gleichungsnummern (im Gegensatz zu veralteten Umgebungen wie \env{eqnarray}) nicht überdruckt werden \cite{amsldoc,l2tabu}.
% Für einfache abgesetzte Formeln sind dies im Wesentlichen die \env{equation}-Umgebung (für einzelne Formeln) und die \env{align}-Umgebung (für mehrere Formeln) bzw. deren Sternversionen \env{equation*} und \env{align*} für nicht nummerierte Formeln.\par
% In dieser Klasse wird das Paket \paket{onlyamsmath} verwendet, das einen Fehler ausgibt, wenn Matheumgebungen genutzt werden, die nicht von der AMS stammen \cite{onlyamsmath}.
% Möchten Sie dieses Verhalten vermeiden, so nutzen Sie die Klassenoption \optref{non-ams}.\par
% Nützliche Hinweise zum Mathematiksatz in \LaTeX\ finden Sie in \cite{amsldoc,voss-mathe,voss-mathe-online,latex-begleiter,more-math}.
%
%
% \section{Weiterführendes}
% \label{sec:weiterfuehrendes}
% Hier finden Sie noch einige Tipps und Literaturempfehlungen sowie ein paar Hintergründe zu dieser Klasse.
%
% \subsection{Paketempfehlungen}
% \label{sec:paketempfehlungen}
% Hier werden nun einige Pakete aufgeführt, die der Autor dieser Klasse allgemein als nützlich ansieht.
% Sie werden nicht von der Klasse direkt geladen.
% Ob der Nutzer der Klasse sie verwendet, hängt also von ihm selbst ab.
% \begin{labeling}{\paket{booktabs}}
% 	\item[\paket{booktabs}]
% 	Verbesserte horizontale Linien für übersichtliche Tabellen \cite{booktabs}.
% 	\item[\paket{csquotes}]
% 	Einfache Schnittstelle für die Nutzung von Anführungszeichen \cite{csquotes}.
% 	\item[\paket{cleveref}]
% 	Komfortable Alternative zum \fmacro{ref}-Befehl (kompatibel mit \paket{varioref}) \cite{cleveref}.
% 	\item[\paket{varioref}]
% 	Praktische Ergänzung zum \fmacro{ref}-Befehl, gerade für gedruckte Werke bei Verweisen weit entfernt vom Label \cite{varioref}.
% 	\item[\paket{amsmath}]
% 	Sinnvolle Mathematik-Umgebungen und Weiteres (siehe auch \cref{sec:mathesatz}) \cite{amsldoc}.
% 	\item[\paket{siunitx}]
%	Größen und Einheiten einfach und konsistent setzen \cite{siunitx}.
% \end{labeling}
%
% \subsection{Verwendete Pakete}
% \label{sec:verwendete-pakete}
% Hier werden kurz die von der Klasse geladenen Pakete (in der Reihenfolge, in der sie geladen werden) und ihre Funktionen aufgeführt.
% Das Wissen, welche Pakete geladen werden, kann wichtig sein, wenn der Nutzer der Klasse andere Pakete laden möchte, die mit den von der Klasse geladenen in Konflikt stehen könnten.\par
% Die Optionen, mit denen diese Pakete geladen werden, sind teilweise von den verwendeten Klassenoptionen abhängig (besonders von \optref{format}) und deshalb hier nicht aufgeführt.
% Sie sind bei Bedarf der Quelltextdatei \datei{sr-vorl.dtx} zu entnehmen.
% \begin{labeling}{\paket{onlyamsmath}\cite{onlyamsmath}}
% 	\item[\paket{scrbook} \cite{koma}]
%	Dies ist die verwendete Dokumentenklasse aus dem \textsf{KOMA}-Script-Bündel.
% 	\item[\paket{xkeyval} \cite{xkeyval}]
% 	Notwendig zum Erstellen einiger Klassenoptionen.
% 	\item[\paket{etoolbox} \cite{etoolbox}]
% 	Stellt verschiedene if-Abfragen und Hooks zur Verfügung.
% 	\item[\paket{babel} \cite{babel}]
% 	Lädt die Sprachen.
% 	\item[\paket{geometry} \cite{geometry}]
% 	Wird zur Einstellung des Satzspiegels genutzt.
% 	\item[\paket{scrpage2} \cite{koma}]
% 	Für die Definition der Kopf- und Fußzeilen.
% 	Der definierte Standard-Seitenstil heißt \option{sr-standard}.
% 	\item[\paket{caption} \cite{caption}]
% 	Ermöglicht Einstellungen der Bild- und Tabellenbeschriftungen.
% 	\item[\paket{ragged2e} \cite{ragged2e}]
% 	Verbesserter Flattersatz.
% 	\item[\paket{enumitem} \cite{enumitem}]
% 	Einstellungen für Listen.
% 	\item[\paket{xstring} \cite{xstring}]
%	Das Paket bietet viele Befehle zur String-Verarbeitung und "~auswertung.
%	Es wird zur unter anderem Auswertung der Klassenoptionen verwendet.
% 	\item[\paket{chngcntr} \cite{chngcntr}]
% 	Das Paket dient der Modifikation von Zählern.
% 	Es wird zur Anpassung der Kapitel-, Abbildungs- und anderer Zähler bei Mehrautorenbüchern verwendet.
% 	\item[\paket{varwidth}] \cite{varwidth}
% 	Stellt die \env{varwidth}-Umgebung als Erweiterung der \env{minipage}-Umgebung bereit.
% 	\item[\paket{onlyamsmath} \cite{onlyamsmath}]
% 	Verhindert die Nutzung von Mathe-Umgebungen, die nicht von der AMS stammen (siehe auch die Option \optref{non-ams} in \cref{sec:optionen}).
% 	\item[\paket{microtype} \cite{microtype}]
% 	Verbesserte Mikrotypographie (siehe auch die Option \optref{no-microtype} in \cref{sec:optionen}).
% \end{labeling}
% 
% \subsection{Literaturtipps}
% \label{sec:literaturtipps}
% Wie bereits eingangs erwähnt, ist dieses Dokument keine \LaTeX-Einführung.
% Man findet im Internet diverse Einführungen von verschiedener Länge und Qualität.
% Beispiele für Einführungen --~für deren Qualität der Autor dieses Dokuments keinerlei Garantie übernimmt~-- sind: \cite{l2kurz,beginners-introduction,buxbaum-presentation,simplified-introduction,juergens1,juergens2}.
% Gerade bei Werken älteren Datums muss man vor veralteten Befehlen auf der Hut sein.
% Eine Hilfe hierzu bietet \cite{l2tabu}.
% Dort sind einige veraltete Befehle und Pakete sowie sonstige Fehlerquellen benannt.\par
% Für den Mathematiksatz sei auf die bereits in \cref{sec:mathesatz} erwähnten Werke \cite{amsldoc,voss-mathe,voss-mathe-online,latex-begleiter,more-math} verwiesen.\par
% Weitere Orientierung bieten \cite{latex-begleiter,texbook,wikibooks-latex}.
% Nahezu alle Pakete inklusive ihrer Beschreibungen sowie weitere nützliche Dokumente finden Sie im \emph{Comprehensive \TeX\ Archive Network} \cite{ctan}.
% Viele dieser Dokumente und Beschreibungen finden Sie auch in Ihrem \LaTeX-System auf Ihrem Rechner, indem Sie in der Konsole (Terminal, Bash, Eingabeaufforderung) folgendes eingeben: \texttt{texdoc \meta{Paketname}}\par
% Hilft das alles nicht weiter, so tut es manchmal auch eine Internetsuche.
% Neben einschlägig bekannten Suchmaschinen sind hier spezielle Foren nützlich, in denen sich hilfsbereite \LaTeX-Spezialisten tummeln, z.\,B. \url{http://tex.stackexchange.com/} (englischsprachig, sehr viele aktive Nutzer) und \url{www.golatex.de} (deutschsprachig, weniger aktive Nutzer).
% 
%\section{Lizenz}
% \label{sec:lizenz}
% Dieses Werk darf nach den Bedingungen der LaTeX Project Public License, entweder Version 1.3c oder (nach Ihrer Wahl) jeder späteren Version, verteilt und/oder verändert werden.
% Die neueste Version dieser Lizenz ist:\\
% \url{http://www.latex-project.org/lppl.txt}\\
% Dieses Werk hat den LPPL-Betreuungs-Status \enquote{author-maintained} (vom Autor betreut).
%
%
% \begin{otherlanguage}{british}
% 	This work may be distributed and/or modified under the conditions of the LaTeX Project Public License, either version 1.3c of this license or (at your option) any later version.
% 	The latest version of this license is in:\\
% 	\url{http://www.latex-project.org/lppl.txt}\\
% 	This work has the LPPL maintenance status \foreignquote{english}{author-maintained}.
% \end{otherlanguage}
%
% \nocite{*}
% \iftoggle{auxdatei}
% 	{\printbibliography[prenote=literatur]}
% 	{}
%
% \section*{Liste der Änderungen}
% \addcontentsline{toc}{section}{Liste der Änderungen}
% Hier finden Sie eine Übersicht über die Änderungen, die an dieser Klasse im Verlauf ihrer Versionen vorgenommen wurden.
%
% \begin{description}
% 	\aenderung{1.1}{Option \optref{format}}{Neue Einstellungsmöglichkeit \option{vs} für Bücher bei Springer VS.}
% 	\aenderung{1.1}{Befehl \macroref{verfasser}}{Neuer Befehl zum Satz von Autor und Ort des Geleitwortes.}
% 	\aenderung{1.1}{Umstellung auf \programm{biber}}{Die Bibliografie dieser Dokumentation wird nun mit \programm{biber} erzeugt.}
% \end{description}
%
%\StopEventually{}
%
%<*class>
\LoadClassWithOptions{scrbook}


\RequirePackage{xkeyval}
\RequirePackage{etoolbox}

\DeclareOptionX[sr]{no-shorthands}{%
	\global\Germanshorthandsfalse%
}

\DeclareOptionX[sr]{test}{%
	\PassOptionsToClass{draft}{scrbook}
	\PassOptionsToPackage{showframe}{geometry}
}

\DeclareOptionX[sr]{deutsch}{%
	\PassOptionsToPackage{ngerman}{babel}%
	\ifGermanshorthands%
	\else%
		\AfterEndPreamble{\shorthandoff{"}}
	\fi%
	\gdef\GeleitwortTitel{Geleitwort}%
	\gdef\VorwortTitel{Vorwort}%
}

\DeclareOptionX[sr]{deutsch-ar}{%
	\PassOptionsToPackage{german}{babel}%
	\ifGermanshorthands%
	\else%
		\AfterEndPreamble{\shorthandoff{"}}
	\fi%
	\gdef\GeleitwortTitel{Geleitwort}%
	\gdef\VorwortTitel{Vorwort}%
}

\DeclareOptionX[sr]{englisch}{%
	\PassOptionsToPackage{ngerman,british}{babel}%
	\ifGermanshorthands%
		\AfterPackage+{babel}{%
			\useshorthands{"}%
			\addto\extrasbritish{\languageshorthands{ngerman}}%
		}%
	\fi%
	\gdef\GeleitwortTitel{Foreword}%
	\gdef\VorwortTitel{Preface}%
}

\DeclareOptionX[sr]{no-microtype}{%
	\global\Microtypefalse%
}

\DeclareOptionX[sr]{non-ams}{%
	\global\AMSfalse%
}


\gdef\sr@schriftgroesse{11}% Standard-Schriftgroesse
\gdef\sr@format{a5-monografie}% Standard-Format

\newif\ifMicrotype% soll microtype eingebunden werden
\Microtypetrue

\newif\ifAMS% soll onlyamsmath eingebunden werden
\AMStrue

\newif\ifGermanshorthands% sollen die deutschen shorthands auch im Englischen nutzbar sein
\Germanshorthandstrue

\DeclareOptionX[sr]{schriftgroesse}[11]{\gdef\sr@schriftgroesse{#1}}

\DeclareOptionX[sr]{format}[a5-monografie]{\gdef\sr@format{#1}}

\ProcessOptions\relax
\ProcessOptionsX[sr]\relax




\RequirePackage{babel}
\RequirePackage{geometry}
\RequirePackage{scrpage2}
\RequirePackage{caption}
\RequirePackage{ragged2e}
\RequirePackage{enumitem}
\RequirePackage{xstring}
\RequirePackage{chngcntr}
\RequirePackage{varwidth}
\ifAMS
	\RequirePackage[all,error]{onlyamsmath}
	\AtEndPreamble{%
		\@ifpackageloaded{tikz}% Test, ob TikZ geladen wurde
			{% Catcode für $ in tikz-Befehlen anpassen, um Konflikt zwischen onlyamsmath und TikZ zu vermeiden
				\preto\tikzpicture{\catcode`$=3 }
				\preto\tikz{\catcode`$=3 }
			}
			{}
	}
\fi
\ifMicrotype
	\RequirePackage{microtype}
\fi



\ifundef{\GeleitwortTitel}% falls keine Sprachoption gesetzt wurde, wird der Befehl \GeleitwortTitel definiert
	{\gdef\GeleitwortTitel{}}
	{}
\ifundef{\VorwortTitel}% falls keine Sprachoption gesetzt wurde, wird der Befehl \VorwortTitel definiert
	{\gdef\VorwortTitel{}}
	{}



\defpagestyle{plain}
	{%
		(0pt,0pt)%
		{}%
		{}%
		{}%
		(0pt,0pt)%
	}%
	{%
		(0pt,0pt)%
		{}%
		{}%
		{}%
		(0pt,0pt)%
	}%


\KOMAoptions{% Einstellungen der KOMA-Klasse
	fontsize=\sr@schriftgroesse pt,%
	open=right,%
	captions=signature,%
	captions=abovetable,%
	appendixprefix=false,%
}%

\gdef\frontmatter{%
	\pagenumbering{Roman}%
	\setcounter{page}{5}%
}


\captionsetup{% Einstellungen fuer Bild- und Tabellenbeschriftungen (mittels caption-Paket)
	format=hang,%
	font=small,%
	labelfont=bf,%
	justification=RaggedRight ,%
	singlelinecheck=false,%
}%

\setlist{labelindent=0em,leftmargin=*}

\AfterEndPreamble{%
	\pagestyle{sr-standard}% Standard-pagestyle einschalten
	\raggedbottom% vertikalen Seitenausgleich abschalten
}


% Einstellungen fuer die verschiedenen Formate
\IfStrEqCase{\sr@format}{%
	{a5-monografie}%
		{%
			\KOMAoptions{%
				paper=a4,%
				twoside,%
				pagesize=auto,%
			}%
			\recalctypearea
			\geometry{% Einstellungen fuer Satzspiegel (mittels geometry-Paket)
				includehead,%
				textwidth=11.5cm,%
				height=18cm,%
			}%
			\automark[section]{chapter}
			\defpagestyle{sr-standard}% Standard pagestyle
				{%
					(0pt,0pt)%
					{\pagemark\hfill\headmark}%
					{\headmark\hfill\pagemark}%
					{}%
					(\textwidth,0.5pt)%
				}%
				{%
					(0pt,0pt)%
					{}%
					{}%
					{}%
					(0pt,0pt)%
				}%
		}%
	{handbuch-monografie}%
		{%
			\KOMAoptions{%
				paper=16.8cm:24cm,%
				twoside,%
				pagesize=auto,%
			}%
			\recalctypearea
			\geometry{% Einstellungen fuer Satzspiegel (mittels geometry-Paket)
				includehead,%
				textwidth=12.5cm,%
				height=20cm,%
			}%
			\automark[section]{chapter}
			\defpagestyle{sr-standard}% Standard pagestyle
				{%
					(0pt,0pt)%
					{\pagemark\hfill\headmark}%
					{\headmark\hfill\pagemark}%
					{}%
					(\textwidth,0.5pt)%
				}%
				{%
					(0pt,0pt)%
					{}%
					{}%
					{}%
					(0pt,0pt)%
				}%
		}%
	{a5-mehrautorenbuch}% Einstellungen fuer Mehrautorenbuecher
		{%
			\KOMAoptions{%
				paper=a4,%
				twoside,%
				pagesize=auto,%
			}%
			\recalctypearea
			\geometry{% Einstellungen fuer Satzspiegel (mittels geometry-Paket)
				includehead,%
				textwidth=11.5cm,%
				height=18cm,%
			}%
			\automark{chapter}
			\defpagestyle{sr-standard}% Standard pagestyle
				{%
					(0pt,0pt)%
					{\pagemark\hfill\headmark}%
					{\KapitelAutor\hfill\pagemark}%
					{}%
					(\textwidth,0.5pt)%
				}%
				{%
					(0pt,0pt)%
					{}%
					{}%
					{}%
					(0pt,0pt)%
				}%
			\gdef\KapitelAutor{}%
		    \newcommand{\sr@chapterheadstartvskip}{% Definition eines neuen Abstands vor Kapitelanfaengen
				\vspace*{1.8\baselineskip}%
			}%
			\newcommand{\sr@chapterheadendvskip}{% Definition eines neuen Abstands nach Kapitelanfaengen
				\vspace{.5\baselineskip plus 0.1\baselineskip minus .05\baselineskip}%
			}%
			\let\old@chapterheadstartvskip\chapterheadstartvskip% Sicherung des alten Abstandes vor Kapitelanfaengen
			\let\old@chapterheadendvskip\chapterheadendvskip% Sicherung des alten Abstandes nach Kapitelanfaengen
			\counterwithout{section}{chapter}%
			\counterwithout{figure}{chapter}%
			\counterwithout{table}{chapter}%
			\counterwithout{equation}{chapter}%
			\counterwithin*{section}{chapter}%
			\counterwithin*{figure}{chapter}%
			\counterwithin*{table}{chapter}%
			\counterwithin*{equation}{chapter}%
			\counterwithin*{footnote}{chapter}%
		}%
	{vs}% Einstellungen fuer Buch bei Springer VS
		{%
			\KOMAoptions{%
				paper=a4,%
				twoside,%
				pagesize=auto,%
			}%
			\recalctypearea
			\geometry{% Einstellungen fuer Satzspiegel (mittels geometry-Paket)
				includehead,%
				textwidth=11.5cm,%
				height=18cm,%
			}%
			\automark[section]{chapter}
			\defpagestyle{sr-standard}% Standard pagestyle
				{%
					(0pt,0pt)%
					{\pagemark\hfill\headmark}%
					{\headmark\hfill\pagemark}%
					{}%
					(\textwidth,0.5pt)%
				}%
				{%
					(0pt,0pt)%
					{}%
					{}%
					{}%
					(0pt,0pt)%
				}%
			\gdef\frontmatter{%
				\pagenumbering{arabic}%
				\setcounter{page}{5}%
			}%
			\let\old@chapterheadstartvskip\chapterheadstartvskip%
			\let\old@chapterheadendvskip\chapterheadendvskip%
			\renewcommand*{\chapterheadstartvskip}{\vspace*{-\topskip}}%
			\renewcommand*{\chapterheadendvskip}{\vspace{72pt}}%
			\deffootnote%
				{1em}%
				{1em}%
				{\thefootnotemark\ }%
		}%
}


\newlength{\sr@widmungstretch@oben}%	vertikaler Abstand vor der Widmung

\newlength{\sr@verfasserbreite@ort}% Breite des Ortes im Befehl \verfasser
\newlength{\sr@verfasserbreite@name}% Breite des Namens im Befehl \verfasser


% \begin{environment}{widmung}
% Widmung
%    \begin{macrocode}
\newenvironment{widmung}[1][]
	{%
		\ifblank{#1}%
			{%
				\cleardoublepage%
				\thispagestyle{plain}%
				\setlength{\sr@widmungstretch@oben}{-2\baselineskip}%
				\addtolength{\sr@widmungstretch@oben}{\stretch{1}}%
				\vspace*{\sr@widmungstretch@oben}%
				\begin{flushright}%
			}%
			{%
				\let\raggedsection\raggedleft%
				\setlength{\sr@widmungstretch@oben}{-\baselineskip}%
				\addtolength{\sr@widmungstretch@oben}{\stretch{1}}%
				\def\chapterheadstartvskip{\vspace*{\sr@widmungstretch@oben}}%
				\chapter*{\raggedleft #1}%
				\begin{flushright}%
			}%
		\itshape%
	}%
	{%
		\end{flushright}%
		\vspace*{\stretch{1.618}}%
	}%
%    \end{macrocode}
% \end{environment}


% \begin{macro}{\vorwort}
% Vorwort-Kapitel
%    \begin{macrocode}
\newcommand{\vorwort}{%
	\addchap*{\VorwortTitel}%
	\markboth{\VorwortTitel}{\VorwortTitel}
}
%    \end{macrocode}
% \end{macro}


% \begin{macro}{\geleitwort}
% Geleitwort-Kapitel
%    \begin{macrocode}
\newcommand{\geleitwort}{%
	\addchap*{\GeleitwortTitel}%
	\markboth{\GeleitwortTitel}{\GeleitwortTitel}
}
%    \end{macrocode}
% \end{macro}

% \begin{macro}{\verfasser}
% Geleitwort-Kapitel
%    \begin{macrocode}
\newcommand{\verfasser}[3][0.4\linewidth,0.5\linewidth]{%
	\StrBefore{#1}{,}[\sr@verfasserbefehl@ort]%
	\StrBehind{#1}{,}[\sr@verfasserbefehl@name]%
	\setlength{\sr@verfasserbreite@ort}{\sr@verfasserbefehl@ort}%
	\setlength{\sr@verfasserbreite@name}{\sr@verfasserbefehl@name}%
	\ifdimcomp{\sr@verfasserbreite@name + \sr@verfasserbreite@ort}{>}{\linewidth}% gucken, ob alles in eine Zeile passt
		{% es passt nicht in eine Zeile
			\ClassWarning{sr-vorl}{Die Summe der im Befehl '\verfasser' genutzten Laengen ist groesser als die Zeilenbreite!}% Warnung, wenn die Laengen zu gross sind
		}%
		{}% es passt in eine Zeile
	%
	\nopagebreak%

	\nopagebreak%
	\vspace*{1.5\baselineskip plus 0.75\baselineskip minus 0.75\baselineskip}%
	\nopagebreak%
	\noindent
	\begin{varwidth}[t]{\sr@verfasserbreite@ort}%
		\raggedright%
		#2%
	\end{varwidth}%
	\hfill%
	\begin{varwidth}[t]{\sr@verfasserbreite@name}%
		\raggedleft%
		#3%
	\end{varwidth}%
}
%    \end{macrocode}
% \end{macro}


% \begin{macro}{\kapitel}
% Kapitel-Befehl fuer Mehrautorenbuecher
%    \begin{macrocode}
\newcommand{\kapitel}[3][]{%
	\gdef\KapitelAutor{#3}%
	\stepcounter{chapter}
	\let\chapterheadstartvskip\sr@chapterheadstartvskip%
	\let\chapterheadendvskip\sr@chapterheadendvskip%
	\ifblank{#1}%
		{\addchap{#2}}%
		{\addchap[#1]{#2}}%
	\let\chapterheadstartvskip\old@chapterheadstartvskip%
	\let\chapterheadendvskip\old@chapterheadendvskip%
	{\usekomafont{section}\itshape#3}\par%
	\vspace*{1.35\baselineskip plus 0.09\baselineskip minus .15\baselineskip}%
	\noindent%
}
%    \end{macrocode}
% \end{macro}


% Einstellungen fuer Gleitobjekte
\setcounter{topnumber}{3}
\setcounter{bottomnumber}{1}
\setcounter{totalnumber}{5}

\renewcommand{\topfraction}{1}
\renewcommand{\bottomfraction}{0.4}
\renewcommand{\textfraction}{0.05}
\renewcommand{\floatpagefraction}{0.7}

\setlength{\@fptop}{0pt}
\setlength{\@fpsep}{8pt}
\setlength{\@fpbot}{0pt plus 1fil}

\clubpenalty=9999
\widowpenalty=9999
\displaywidowpenalty=9999

%</class>
%
%
%<*frontmatter>
%% Diese Datei sollte den Vorspann Ihrer Arbeit enthalten, wie z. B. Titelei, Vorwort, Geleitwort, Danksagung, Inhaltsverzeichnis, Abbildungsverzeichnis.
\frontmatter
\begin{widmung}% Innerhalb dieser Umgebung koennen Sie Ihre Widmung schreiben.

\end{widmung}

\geleitwort% Hier koennen Sie Ihr Geleitwort schreiben.


\vorwort% Hier koennen Sie Ihr Vorwort schreiben.


%% Hier koennen Inhaltsverzeichnis, Abbildungsverzeichnis und Aehnliches folgen.

%</frontmatter>
%
%
%<*mainmatter>
%% Diese Datei sollte den Hauptteil Ihrer Arbeit enthalten.
\mainmatter

%</mainmatter>
%
%
%<*backmatter>
%% Diese Datei sollte den Nachspann (Anhang, Literaturverzeichnis etc.) Ihrer Arbeit enthalten.
\backmatter
%% 

%</backmatter>
%
%
%<*hauptdatei>
%% Dies ist die Hauptdatei Ihrer Arbeit. Hier koennen Sie Pakete einbinden, Definitionen vornehmen und Aehnliches.
\documentclass{sr-vorl}

\begin{document}
\input{frontmatter_sr-vorl.tex}%% Die Datei, die den Vorspann Ihrer Arbeit enthalten sollte, wird eingebunden.

\input{mainmatter_sr-vorl.tex}%% Die Datei, die den Hauptteil Ihrer Arbeit enthalten sollte, wird eingebunden.

\input{backmatter_sr-vorl.tex}%% Die Datei, die den Nachspann Ihrer Arbeit enthalten sollte, wird eingebunden.

\end{document}

%</hauptdatei>
%
%\Finale
