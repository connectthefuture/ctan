% \iffalse meta-comment
%
% File: pageslts.dtx
% Version: 2015/12/21 v1.2f
%
% Copyright (C) 2010 - 2015 by
%    H.-Martin M"unch <Martin dot Muench at Uni-Bonn dot de>
% Portions of code copyrighted by other people as marked.
%
% This package first started as a revision of the lastpage
% package of Jeffrey P. Goldberg (jeffrey+news at goldmark dot org),
% but it became obvious that a replacement was needed.
%
% This work may be distributed and/or modified under the
% conditions of the LaTeX Project Public License, either
% version 1.3c of this license or (at your option) any later
% version. This version of this license is in
%    http://www.latex-project.org/lppl/lppl-1-3c.txt
% and the latest version of this license is in
%    http://www.latex-project.org/lppl.txt
% and version 1.3c or later is part of all distributions of
% LaTeX version 2005/12/01 or later.
%
% This work has the LPPL maintenance status "maintained".
%
% The Current Maintainer of this work is H.-Martin Muench.
%
% This work consists of the main source file pageslts.dtx,
% the README, and the derived files
%    pageslts.sty, pageslts.pdf,
%    pageslts.ins, pageslts.drv,
%    pageslts-example.tex, pageslts-example.pdf.
%
% 'pageslts' is available on CTAN:
% https://www.ctan.org/pkg/pageslts
%
% Also a TDS.ZIP file is provided that contains all the files
% already sorted in a TDS tree:
% http://mirrors.ctan.org/install/macros/latex/contrib/pageslts.tds.zip
%
%<*ignore>
\begingroup
  \catcode123=1 %
  \catcode125=2 %
  \def\x{LaTeX2e}%
\expandafter\endgroup
\ifcase 0\ifx\install y1\fi\expandafter
         \ifx\csname processbatchFile\endcsname\relax\else1\fi
         \ifx\fmtname\x\else 1\fi\relax
\else\csname fi\endcsname
%</ignore>
%<*install>
\input docstrip.tex
\Msg{*********************************************************************************}
\Msg{* Installation}
\Msg{* Package: pageslts 2015/12/21 v1.2f Refers to special pages' numbers/names (HMM)}
\Msg{*********************************************************************************}

\keepsilent
\askforoverwritefalse

\let\MetaPrefix\relax
\preamble

This is a generated file.

Project: pageslts
Version: 2015/12/21 v1.2f

Copyright (C) 2010 - 2015 by
    H.-Martin M"unch <Martin dot Muench at Uni-Bonn dot de>
Portions of code copyrighted by other people as marked.

IMPORTANT NOTICE:
On the first page backslash pagenumbering should be called
(with the appropriate argument out of e.g.
 * roman    - Lowercase Roman numerals: i, ii, iii, iv,...
 * Roman    - Uppercase Roman numerals: I, II, III, IV,...
 * arabic   - Arabic numerals: 1, 2, 3, 4,...
 * alph     - Lowercase letters: a, b, c, d,...
 * Alph     - Uppercase letters: A, B, C, D,...
 * fnsymbol - Footnote symbols: *, † (ddagger), ‡ (dddagger),
                                 § (mathsection),...
). The command backslash pagenumbering is renewed, but no change
should be visible to the output text, because only labels are added.
backslash OrigPagenumbering can be used to get the original meaning
of backslash pagenumbering, but in that case this package does not
see the change in the page numbering scheme and will make mistakes.
The package takes options.

The usual disclaimer applies:
If it doesn't work right that's your problem.
(Nevertheless, send an e-mail to the maintainer
 when you find an error in this package.)

This work may be distributed and/or modified under the
conditions of the LaTeX Project Public License, either
version 1.3c of this license or (at your option) any later
version. This version of this license is in
   http://www.latex-project.org/lppl/lppl-1-3c.txt
and the latest version of this license is in
   http://www.latex-project.org/lppl.txt
and version 1.3c or later is part of all distributions of
LaTeX version 2005/12/01 or later.

This work has the LPPL maintenance status "maintained".

The Current Maintainer of this work is H.-Martin Muench.

This package first started as a revision of the lastpage
package of Jeffrey P. Goldberg (jeffrey+news at goldmark dot org),
but it became obvious that a replacement was needed.

This work consists of the main source file pageslts.dtx,
the README, and the derived files
   pageslts.sty, pageslts.pdf,
   pageslts.ins, pageslts.drv,
   pageslts-example.tex, pageslts-example.pdf.

In memoriam
 Claudia Simone Barth + 1996/01/30
 Tommy Muench + 2014/01/02
 Hans-Klaus Muench + 2014/08/24

\endpreamble
\let\MetaPrefix\DoubleperCent

\generate{%
  \file{pageslts.ins}{\from{pageslts.dtx}{install}}%
  \file{pageslts.drv}{\from{pageslts.dtx}{driver}}%
  \usedir{tex/latex/pageslts}%
  \file{pageslts.sty}{\from{pageslts.dtx}{package}}%
  \usedir{doc/latex/pageslts}%
  \file{pageslts-example.tex}{\from{pageslts.dtx}{example}}%
}

\catcode32=13\relax% active space
\let =\space%
\Msg{************************************************************************}
\Msg{*}
\Msg{* To finish the installation you have to move the following}
\Msg{* file into a directory searched by TeX:}
\Msg{*}
\Msg{*  pageslts.sty}
\Msg{*}
\Msg{* To produce the documentation run the file `pageslts.drv'}
\Msg{* through (pdf)LaTeX, e.g.}
\Msg{*  pdflatex pageslts.drv}
\Msg{*  makeindex -s gind.ist pageslts.idx}
\Msg{*  pdflatex pageslts.drv}
\Msg{*  makeindex -s gind.ist pageslts.idx}
\Msg{*  pdflatex pageslts.drv}
\Msg{*}
\Msg{* At least three runs are necessary e. g. to get the}
\Msg{*  references right!}
\Msg{*}
\Msg{* Happy TeXing!}
\Msg{*}
\Msg{************************************************************************}

\endbatchfile
%</install>
%<*ignore>
\fi
%</ignore>
%
% \section{The documentation driver file}
%
% The next bit of code contains the documentation driver file for
% \TeX , i.\,e., the file that will produce the documentation you
% are currently reading. It will be extracted from this file by the
% \texttt{docstrip} programme. That is, run \LaTeX{} on \texttt{docstrip}
% and specify the \texttt{driver} option when \texttt{docstrip}
% asks for options.
%
%    \begin{macrocode}
%<*driver>
\NeedsTeXFormat{LaTeX2e}[2015/01/01]
\ProvidesFile{pageslts.drv}[2015/12/21 v1.2f
            Refers to special pages' numbers/names (HMM)]
\documentclass[landscape]{ltxdoc}[2015/03/26]% v2.0w
\usepackage{geometry}[2010/09/12]%             v5.6
\usepackage{holtxdoc}[2012/03/21]%             v0.24
%% pageslts may work with earlier versions of LaTeX2e and those
%% class and packages, but this was not tested.
%% Please consider updating your LaTeX, class, and packages
%% to the most recent version (if they are not already the most
%% recent version).
\hypersetup{%
 pdfsubject={Refers to special pages' numbers/names (HMM)},%
 pdfkeywords={LaTeX, pageslts, pagesLTS, lastpage, page number, page name, H.-Martin Muench},%
 pdfencoding=auto,%
 pdflang={en},%
 breaklinks=true,%
 linktoc=all,%
 pdfstartview=FitH,%
 pdfpagelayout=OneColumn,%
 bookmarksnumbered=true,%
 bookmarksopen=true,%
 bookmarksopenlevel=2,%
 pdfmenubar=true,%
 pdftoolbar=true,%
 pdfwindowui=true,%
 pdfnewwindow=true%
}
\usepackage{ulem}[2012/05/18]% no version is given in the ulem.sty file
\CodelineIndex
\hyphenation{created every-thing ignored}
\gdef\unit#1{\mathord{\thinspace\mathrm{#1}}}
\makeatletter
\@ifundefined{eTeX}{\gdef\eTeX{$\m@th \varepsilon $-\TeX }}{% else \relax
}
\makeatother
\begin{document}
  \DocInput{pageslts.dtx}%
\end{document}
%</driver>
%    \end{macrocode}
%
% \fi
%
% \CheckSum{1790}
%
% \CharacterTable
%  {Upper-case    \A\B\C\D\E\F\G\H\I\J\K\L\M\N\O\P\Q\R\S\T\U\V\W\X\Y\Z
%   Lower-case    \a\b\c\d\e\f\g\h\i\j\k\l\m\n\o\p\q\r\s\t\u\v\w\x\y\z
%   Digits        \0\1\2\3\4\5\6\7\8\9
%   Exclamation   \!     Double quote  \"     Hash (number) \#
%   Dollar        \$     Percent       \%     Ampersand     \&
%   Acute accent  \'     Left paren    \(     Right paren   \)
%   Asterisk      \*     Plus          \+     Comma         \,
%   Minus         \-     Point         \.     Solidus       \/
%   Colon         \:     Semicolon     \;     Less than     \<
%   Equals        \=     Greater than  \>     Question mark \?
%   Commercial at \@     Left bracket  \[     Backslash     \\
%   Right bracket \]     Circumflex    \^     Underscore    \_
%   Grave accent  \`     Left brace    \{     Vertical bar  \|
%   Right brace   \}     Tilde         \~}
%
% \GetFileInfo{pageslts.drv}
%
% \begingroup
%   \def\x{\#,\$,\^,\_,\~,\ ,\&,\{,\},\%}%
%   \makeatletter
%   \@onelevel@sanitize\x
% \expandafter\endgroup
% \expandafter\DoNotIndex\expandafter{\x}
% \expandafter\DoNotIndex\expandafter{\string\ }
% \begingroup
%   \makeatletter
%     \lccode`9=32\relax
%     \lowercase{%^^A
%       \edef\x{\noexpand\DoNotIndex{\@backslashchar9}}%^^A
%     }%^^A
%   \expandafter\endgroup\x
%
% \DoNotIndex{\",\-,\,,\\,\noindent}
% \DoNotIndex{\documentclass,\usepackage,\ProvidesPackage}
% \DoNotIndex{\NeedsTeXFormat,\TeX,\LaTeX}
% \DoNotIndex{\ldots,\thinspace,\textbackslash}
% \DoNotIndex{\begin,\end}
% \DoNotIndex{\textbf,\textit,\textquotedblleft,\textquotedblright,\textsc,\textsf,\texttt,\underline}
% \DoNotIndex{\ifx,\ifnum,\ifodd,\enduremath}
% \DoNotIndex{\newpage,\pagebreak,\clearpage,\newline,\linebreak,\nolinebreak}
% \DoNotIndex{\lipsum,\MessageBreak,\pageref,\protect}
% \DoNotIndex{\smallskip,\medskip,\bigskip,\hskip,\space,\hline}
% \DoNotIndex{\advance,\@ne}
% \DoNotIndex{\item}
% \DoNotIndex{\@PackageInfoNoLine,\@bsphack,\@esphack,\@ehc,\@ehd,\@ifundefined,%
%  \addcontentsline,\arabic,\enddocument,\endinput,\href,\makeatletter,\makeatother,%
%  \section,\subsection,\tableofcontents,\verb}
%
% \title{The \xpackage{pageslts} package}
% \date{2015/12/21 v1.2f}
% \author{H.-Martin M\"{u}nch\\\xemail{Martin.Muench at Uni-Bonn.de}}
%
% \maketitle
%
% \begin{abstract}
%  \noindent This \LaTeX{} package puts the labels \texttt{LastPage}
%  (|\AtEndDocument|) and \texttt{VeryLastPage} (|\AfterLastShipout|)
%  into the \xfile{.aux} file, allowing the user to refer
%  to the (very) last page of a document. This might be
%  particularly useful in places like headers or footers.
%  When more than one page numbering scheme is used,
%  these references do not give the total \textit{number} of pages.
%  For this case the label \texttt{LastPages} is introduced.
%  Additionally, at the last page of each page numbering
%  scheme a label \texttt{pagesLTS.<numbering scheme>} is
%  placed, where \texttt{<numbering scheme>} is e.\,g.
%  arabic, roman, Roman, alph, or Alph. For fnsymbol
%  please use |\lastpageref{pagesLTS.fnsymbol}|
%  instead of |\pageref{pagesLTS.fnsymbol}|.
%  When the same numbering scheme is used twice,
%  the page numbers are either reset to one or continued
%  automatically, depending on the option given when the package
%  is called. The command |\theCurrentPage|
%  prints the current total/absolute page number -- in contrast to
%  |\thepage|, which gives only the page
%  \textit{name} in the current page numbering scheme.
%  |\theCurrentPageLocal| gives the current
%  number of pages in the current page numbering scheme.
%  |\thepage| and |\theCurrentPageLocal| are different e.\,g.
%  when |\addtocounter{page}{|\ldots|}| or
%  |\setcounter{page}{|\ldots|}| were used.
%  At the first page of the document a label \texttt{pagesLTS.0} is
%  created. This label can be referred to, too.
%  Further labels are provided for special cases.\\
%  The \xpackage{alphalph} package is supported, i.\,e.
%  page numbers alph or Alph $>26$ and fnsymbol $>9$
%  can be used (with according options set). Even zero
%  and negative page numbers can be used with \texttt{arabic},
%  \texttt{alph}, \texttt{Alph}, \texttt{roman}, \texttt{Roman},
%  and \texttt{fnsymbol} page numbering (with \xpackage{alphalph} package
%  and according options).\\
%  |\pageref*| and |\lastpageref*|, for using \xpackage{hyperref}
%  but suppressing links, are supported.
% \end{abstract}
%
% \noindent \textbf{Please make sure to first deinstall the obsolete
% \xpackage{pagesLTS} package before installing this \xpackage{pageslts}
% package!}\\
% (There is at least one operating system which otherwise automatically
% renames \xfile{pageslts} to \xfile{pagesLTS}.)
%
% \noindent Right after |\begin{document}| a |\pagenumbering{|\ldots|}| should
% be called -- with the appropriate argument out of e.\,g.\\
% arabic (Arabic numerals: 1, 2, 3, 4,\ldots),\\
% roman (Lowercase Roman numerals: i, ii, iii, iv,\ldots),
% Roman (Uppercase Roman numerals: I, II, III, IV,\ldots),\\
% alph (Lowercase letters: a, b, c, d,\ldots),
% Alph (Uppercase letters: A, B, C, D,\ldots),\\
% fnsymbol (Footnote symbols: $*$, $\dagger $, $\ddagger $, $\mathsection $,\ldots).
%
% \noindent This package first started as a revision of the
% \xpackage{lastpage} package of \textbf{Jeffrey P. Goldberg} (Thanks!),
% but then it became obvious that a replacement was needed.
%
% \noindent Disclaimer for web links: The author is not responsible for any contents
% referred to in this work unless he has full knowledge of illegal contents.
% If any damage occurs by the use of information presented there, only the
% author of the respective pages might be liable, not the one who has referred
% to these pages.
%
% \noindent Save per page about $200\unit{ml}$ water,
% $2\unit{g}$ CO$_{2}$ and $2\unit{g}$ wood:
% Therefore please print only if this is really necessary.
%
% \bigskip
%
% \tableofcontents
%
% \newpage
%
% \section{Introduction}
%
% \indent This package puts the labels \texttt{LastPage}
% (|\AtEndDocument|) (same as my \xpackage{LastPage} package, invented by
% \textsc{Jeffrey P. Goldberg}) and \texttt{VeryLastPage} (|\AfterLastShipout|)
% into the \xfile{.aux} file, allowing the user to refer to the (very)
% last page of a document via |\lastpageref{LastPage}| and
% |\lastpageref{VeryLastPage}|. This might be particularly useful in
% places like headers or footers. When more than one page numbering scheme
% is used, these references do not give the total \textit{number} of pages.
% For this case the label \texttt{LastPages} is introduced (similar to the
% label \texttt{TotPages} of the \xpackage{TotPages} package, but
% the label \texttt{LastPages} is set later in the document).
% Additionally, at the last page of each page numbering scheme a label
% \texttt{pagesLTS.<}\textit{numbering scheme}\texttt{>} is placed,
% where \texttt{<}\textit{numbering scheme}\texttt{>} is e.\,g.
% arabic, roman, Roman, alph, or Alph. For fnsymbol please use
% |\lastpageref{pagesLTS.fnsymbol}| instead of |\pageref{pagesLTS.fnsymbol}|.
% When the same numbering scheme is used twice, the page numbers are
% either reset to one or continued automatically, depending on the option
% given when the package is called. The command |\theCurrentPage| prints
% the current total/absolute page number - in contrast to |\thepage|,
% which gives only the page \textit{name} in the current page numbering
% scheme. |\theCurrentPageLocal| gives the current number of pages
% in the current page numbering scheme. |\thepage| and |\theCurrentPageLocal|
% are different e.\,g. when |\addtocounter{page}{|\ldots|}| or
% |\setcounter{page}{|\ldots|}| were used. (See also \LaTeX\ bug~3421:
% 3rd~page is even (twoside, titlepage, abstract),
% \url{http://www.latex-project.org/cgi-bin/ltxbugs2html?category=LaTeX&responsible=anyone&state=anything&keyword=pagenumber&pr=latex%2F3421&search=}.)
% At the first page of the document a label \texttt{pagesLTS.0} is
% created. This label can be referred to, too. Further labels are
% provided for special cases.\\
% The \xpackage{alphalph} package is supported, i.\,e. page numbers
% \texttt{alph} or \texttt{Alph} $>26$ and \texttt{fnsymbol} $>9$ can
% be used (with the according options set). Even zero or negative page
% numbers can be used with \texttt{arabic}, \texttt{alph}, \texttt{Alph}, and
% \texttt{fnsymbol} page numbering (with \xpackage{alphalph} package
% and according options), and zero \texttt{roman} and \texttt{Roman} pages,
% too.
%
% \bigskip
% \noindent Right after |\begin{document}| a |\pagenumbering{|\ldots|}|
% should be called -- with the appropriate argument out of e.\,g.\\
% arabic (Arabic numerals: 1, 2, 3, 4,\ldots),\\
% roman (Lowercase Roman numerals: i, ii, iii, iv,\ldots),
% Roman (Uppercase Roman numerals: I, II, III, IV,\ldots),\\
% alph (Lowercase letters: a, b, c, d,\ldots),
% Alph (Uppercase letters: A, B, C, D,\ldots),\\
% fnsymbol (Footnote symbols: $*$, $\dagger $, $\ddagger $, $\mathsection $,\ldots).
%
% \bigskip
% \noindent This package first started as a revision of the
% \xpackage{lastpage} package of \textsc{Jeffrey P. Goldberg} (Thanks!),
% but then it became obvious that a replacement was needed to accomplish
% what this package does.
%
% \bigskip
% \noindent \textbf{Trademarks} appear throughout this documentation
% without any trademark symbol; they are the property of their respective
% trademark owner. There is no intention of infringement; the usage is
% to the benefit of the trademark owner.
%
% \bigskip
% \phantomsection
% \DescribeMacro{logical page numbers}
% \noindent \textbf{Tip}: For the display of the \xfile{pdf}~file use
% \textbf{logical page numbers} together with \xpackage{hyperref}!
% \label{logical page numbers}
% \begin{description}
% \item [-] In Adobe Reader DC 2015.008.20082 enable:\\
% Edit $>$ Preferences $>$ Categories: Page Display $>$
% Page Content and Information: Use logical page numbers
%
% \item[-] Use the \xpackage{hyperref} package with option |plainpages=false|\,.
% \end{description}
% The display will be e.\,g.\ \textquotedblleft 7~(7~of~9)\textquotedblright,
% or, in case of Roman instead of arabic numbers,
% \textquotedblleft VII~(7~of~9)\textquotedblright, and when different
% page numbers are used (see below) e.\,g. arabic after 10~Roman pages:
% \textquotedblleft 17~(27~of~30)\textquotedblright.
% Please try this with the compiled \xpackage{pageslts-example} file!
%
% \bigskip
% \noindent The name of the \xpackage{pageslts} package refers to
% Last, Total, and page numbering Schemes pages. \xpackage{pagesLTS}
% was a former name of this package.
%
% \pagebreak
%
% \section{Usage}
%
% \indent Just load the package placing
% \begin{quote}
%   |\usepackage[<|\textit{options}|>]{pageslts}|
% \end{quote}
% \noindent in the preamble of your \LaTeXe\ source file
% (about |\AtEndDocument| see subsection~\ref{ssec:aed})
% and place a |\pagenumbering{...}| with appropriate argument
% (e.\,g.~arabic, roman, Roman, fnsymbol, alph, or Alph)
% right behind |\begin{document}| (see subsubsection~\ref{sss:nopnc})!\\
%
% \indent For example for various draft forms it is desirable to have a
% page reference to the last page, so that e.\,g. page footers can
% contain something like \textquotedblleft page $N$ of $K$\textquotedblright,
% where $N$ is the current page and $K$ is the last page. Once the package
% is loaded, anywhere in the text references can be made to the labels
% \texttt{LastPage}, \texttt{VeryLastPage}, and \texttt{LastPages}
% (most times with |\pageref{...}|, but more save with |\lastpageref{...}|).
% In particular one can use the \xpackage{fancyhdr} or \xpackage{nccfancyhdr}
% package, or redefinitions of the page headings and footings to get
% a reference to the (very) last page.\\
% \DescribeMacro{\pageref*}
% \DescribeMacro{\lastpageref*}
% If the \xpackage{hyperref} package is used, the references are hyperlinked
% to their aims. If these hyperlinks shall be suppressed, |\pageref*{...}|
% and |\lastpageref*{...}| can be used.
%
% \subsection{Options}
% \DescribeMacro{options}
% \indent The \xpackage{pageslts} package takes the following options:
%
% \subsubsection{pagecontinue\label{sss:pagecontinue}}
% \DescribeMacro{pagecontinue}
% \indent When option \texttt{pagecontinue=false} is \textbf{not} given
% (i.\,e. \texttt{pagecontinue} or \texttt{pagecontinue=true}
% or no \texttt{pagecontinue} option at all), at each
% |\pagenumbering{...}| command the number of the page will be continued
% with the page number following the last page of the same page numbering
% scheme. For example, if there are V~Roman pages in the frontmatter,
% some arabic ones in the mainmatter, and then Roman ones again in the
% backmatter, the last ones will start with VI instead of~I again.\\
% \indent If you want to start with I (or i, 1, a, A, *,\ldots ) again,
% set option \texttt{pagecontinue=false}. If you want to generally continue
% the numbers, but for some page numbering scheme do not want this,
% use \texttt{pagecontinue=true} and say |\setcounter{page}{1}|
% after |\pagenumbering{...}| for that page numbering scheme.
%
% \subsubsection{alphMult, AlphMulti, fnsymbolmult\label{sss:alphalph}}
%
% \indent The page number printed in
% \nolinebreak{\texttt{fnsymbol}\footnote{%
% \ensuremath {*},%
% \ensuremath {\dagger },%
% \ensuremath {\ddagger },%
% \ensuremath {\mathsection },%
% \ensuremath {\mathparagraph },%
% \ensuremath {\delimiter "026B30D },%
% \ensuremath {**},%
% \ensuremath {\dagger \dagger },%
% \ensuremath {\ddagger \ddagger }}}
% must be $>0$ and $<10$ and those printed in
% \nolinebreak{\texttt{alph}\footnote{a, b, c, d, e, f, g, h,%
% i, j, k, l, m, n, o, p, q, r, s, t, u, v, w, x, y, z}} and
% \nolinebreak{\texttt{Alph}\footnote{A, B, C, D, E, F, G, H,%
% I, J, K, L, M, N, O, P, Q, R, S, T, U, V, W, X, Y, Z}} must
% be $>0$ and $<27$. After page~Z\quad \LaTeX\ \textit{should}
% continue with AA, AB, AC,\ldots\ Some people prefer
% AA, BB, CC,\ldots, but in hexadecimal it is $AA_{16}=170_{10}$
% and $171_{10}=AB_{16}$, whereas $BB_{16}=187_{10}$. In any way
% it should continue at all (maybe even with an user option to
% choose between the two continuations), but instead only gives
% an error:
% \begin{quote}
%   |LaTeX Error: Counter too large|\\
%   |See the LaTeX manual or LaTeX Companion for explanation.|\\
%   |You've lost some text.  Try typing  <return>  to proceed.|\\
%   |If that doesn't work, type  X <return>  to quit.|
% \end{quote}
% But thanks to the \xpackage{alphalph} package by \textsc{Heiko Oberdiek}
% these limitation no longer hold. With his |\erroralph| command
% now even negative or zero page
% \textquotedblleft numbers\textquotedblright{} are possible.\\
%
% \DescribeMacro{alphMult}
% The string option \texttt{alphMult} takes three values: |ab|, |bb|, |0|:
% \begin{description}
% \item[ab] After page z, the page
%  \textquotedblleft numbers\textquotedblright{} continue with
%  aa, ab, ac, ad,\ldots, fxshrxw (the default), and before a with
%  0, -a, -b,\ldots, -z, -aa, -ab,\ldots, -fxshrxw ($=-21\,474\,836\,47$).
% \item[bb] After page z, the page
%  \textquotedblleft numbers\textquotedblright{} continue with
%  aa, bb, cc, dd,\ldots, and before a with
%  0, -a, -b,\ldots, -z, -aa, -bb,\ldots\\
%  (Internally up~to $\pm 55\,834\,558$ is allowed, but when printed will
%  exceed the \LaTeX\ capacity even for smaller numbers --
%  in the example file this happens at about $6\,500$.)\\
%  (If you have a document with more than $6\,500$~pages, you might
%   think about splitting it in volumes. And page
%   \textquotedblleft numbers\textquotedblright{} with about 100~digits
%   are probably not easy to grasp for the reader, too.)\\
% \item[0] (zero) The \xpackage{pageslts} package does nothing,
%  thus the user is free to define the page
%  \textquotedblleft numbers\textquotedblright{} after~z and before~a.\\
%  (But if the user does not do anything at all, the\\
%  \ |LaTeX Error: Counter too large|\\
%  will appear again.)
% \end{description}
%
% \DescribeMacro{AlphMulti}
% The string option \texttt{AlphMulti} takes three values: |AB|, |BB|, |0|:
% \begin{description}
% \item[AB] After page Z, the page
%  \textquotedblleft numbers\textquotedblright{} continue with
%  AA, AB, AC, AD,\ldots, FXSHRXW (the default), and before A with
%  0, -A, -B,\ldots, -Z, -AA, -AB,\ldots, -FXSHRXW.
% \item[BB] After page Z, the page
%  \textquotedblleft numbers\textquotedblright{} continue with
%  AA, BB, CC, DD,\ldots, and before A with
%  0, -A, -B,\ldots, -Z, -AA, -BB,\ldots \\
%  (About the limits please see |alphMult| above.)
% \item[0] (zero) The \xpackage{pageslts} package does nothing,
%  thus the user is free to define the page
%  \textquotedblleft numbers\textquotedblright{} after~Z and before~A.\\
%  (But if the user does not do anything at all, the\\
%  \ |LaTeX Error: Counter too large|\\
%  will appear again.)
% \end{description}
%
% \DescribeMacro{fnsymbolmult}
% When option \texttt{fnsymbolmult=false} is \textbf{not} given
% (i.\,e. \texttt{fnsymbolmult} or \texttt{fnsymbolmult=true}
% or no \texttt{fnsymbolmult} option at all),
% after 5 (\ensuremath {\mathparagraph }) the page
% \textquotedblleft number\textquotedblright{} is continued with
% the doubled \textquotedblleft number\textquotedblright{} of the
% first, second, third,\ldots\ page (\ensuremath {**},
% \ensuremath {\dagger \dagger }, \ensuremath {\ddagger \ddagger },
% \ensuremath {\mathsection \mathsection },
% \ensuremath {\mathparagraph \mathparagraph }),
% and after the tenth page the
% \textquotedblleft number\textquotedblright{} is tripled
% (\ensuremath {***}, \ensuremath {\dagger \dagger \dagger },\ldots).
% Compile the \textsf{pageslts-example.tex} with \pdfLaTeX\ and
% see the resulting \xfile{pdf} file.\\
% Before \ensuremath {*} (page~1) the page
% \textquotedblleft numbers\textquotedblright{} are continued with
% 0, \ensuremath {- * }, \ensuremath {- \dagger },\ldots,
% \ensuremath {- \mathparagraph }, \ensuremath {- **},
% \ensuremath {- \dagger \dagger },\ldots \\
% If this is not wanted, set option \texttt{fnsymbolmult=false},
% and \xpackage{pageslts} will do nothing and allow the user to change
% the page \textquotedblleft number\textquotedblright . (But if the
% user does not do anything at all, the\\
% \ |LaTeX Error: Counter too large|\\
% will appear again.)\\
% While in \LaTeXe\ arabic (page) numbers are possible up~to
% |MAX|\,$ = 2\,147\,483\,647$ (cf.~the \xpackage{alphalph} package),\\
% |\erroralphalph{\fnsymbolmult}{...}| numbers are possible up~to
% 10\,737\,415 only. If this number is not only used internally
% but printed, after number about 11\,705 (which is
% 2\,341 times \ensuremath {\mathparagraph }) the \LaTeXe\ capacity
% is exceeded, depending on the remaining file and its use of
% \TeX\ capacity, of course. (If you have a document with more than
% 11\,705~pages, you might think about splitting it in volumes.
% And page \textquotedblleft numbers\textquotedblright{} with
% 2\,341~digits are probably not easy to grasp for the reader, too.)\\
%
% \subsubsection{romanMult, RomanMulti\label{sss:romanMult}}
%
% \DescribeMacro{romanMult}
% \DescribeMacro{RomanMulti}
% The options \texttt{romanMult(=true)} and \texttt{RomanMulti(=true)}
% expand the |\roman| and |\Roman| page numbering scheme to values
% below one ($<1$), i.\,e. 0, -i, -ii, -iii, -iv,\ldots\ and
% 0, -I, -II, -III, -IV,\ldots, respectively.\\
% Again the \TeX\ capacity will be exceeded before
% \nolinebreak{$\pm$ |MAX|\,}$ = \pm 2\,147\,483\,647$, and even if
% $1\,000\,000\,000$ is internally possible, this would print
% $1\,000\,000$ times the letter~\textsf{m} (or~\textsf{M}),
% which would require either very small print or quite huge paper size.\\
% (If you have a document with so many pages, you might think
% about splitting it in volumes. And page
% \textquotedblleft numbers\textquotedblright{} with
% thousands of digits are probably not easy to grasp for the reader,
% too.)\\
% If the expansion below $1$ is not wanted, set options
% \texttt{romanMult=false} and/or \texttt{RomanMulti=false},
% and \xpackage{pageslts} will do nothing and allow the user
% to change the page \textquotedblleft number\textquotedblright .
% (But if the user does not do anything at all, \LaTeX\ will just
% ignore the values - not even a warning will be issued.)
%
% \subsubsection{Arabic page numbers}
%
% \DescribeMacro{Arabic page numbers}
% In \LaTeXe\ arabic (page) numbers are already possible between
% |-MAX...MAX|, where |MAX|\,$ = 2\,147\,483\,647$ (cf.~the
% \xpackage{alphalph} package), without any expansion necessary.
% (But if you have a document with so many pages, you might think
% about splitting it in volumes!)
%
% \subsection{Labels}
%
% \DescribeMacro{pagesLTS.0}
% \indent At the first page a label \texttt{pagesLTS.0} is created.
% If |\pagenumbering{...}| is used right after
% |\begin{document}|, this is much easier for the
% \xpackage{pageslts} package (and chances for successful placing
% of all labels are much higher;
% \nolinebreak{cf.~subsubsection~\ref{sss:once}).}\\
%
% \DescribeMacro{LastPage}
% |\AtEndDocument| (see subsection~\ref{ssec:aed}) this package defines
% a label, \texttt{LastPage}, which the user can refer to with the\linebreak
% |\lastpageref{LastPage}| command. While |\pageref{LastPage}| is also
% possible (especially for backward compatibility with the \xpackage{LastPage}
% package), this is discouraged, because it will not work when it is used
% together with the \xpackage{hyperref} package and the \xpackage{fnsymbol}
% page numbering scheme. (The \xpackage{LastPage} package did not work with
% this combination, too, so if you want to, you can reproduce the old error
% -- but you do not have to do it, but can use |\lastpageref{LastPage}|.)\\
%
% \DescribeMacro{VeryLastPage}
% |\AfterLastShipout| the label \texttt{VeryLastPage} is defined, which
% the user can also refer to with the |\lastpageref{VeryLastPage}| command.
% Depending on usage of |\AtEndDocument| by other packages, \texttt{LastPage}
% might not point to the very last page, but |\lastpageref{VeryLastPage}|
% should do this (cf.~subsection~\ref{ssec:aed}).\\
%
% \DescribeMacro{LastPages}
% \DescribeMacro{page number}
% \DescribeMacro{number of pages}
% When more than one page numbering scheme is used, neither \texttt{LastPage}
% nor \texttt{VeryLastPage} give the total \textbf{number} of pages.
% For example, for a document with VI+36 pages, both give
% \textquotedblleft 36\textquotedblright{} as reference to the last page.
% While this is correct, the total number of pages is 42, and this is given
% by the reference to \texttt{LastPage\underline{s}}:
% \nolinebreak{|\lastpageref{LastPage|\underline{|s|}|}|} (note the
% \textquotedblleft s\textquotedblright{} at the end). When the
% page number was manipulated by |\addtocounter{page}{...}|
% or |\setcounter{page}{...}|, \texttt{LastPages} ignores this.
% (At a page numbering change the page is reset to one (without option
% |pagecontinue|). This is done by |\setcounter{page}{1}|, thus this is
% ignored, too.)\\
% |\pageref{totpages}| of the \xpackage{totpages} package is similar to
% |\lastpageref{LastPages}|, but while the target for \linebreak[4]
% |\pageref{totpages}| is placed |\AtEndDocument|, the target for
% |\lastpageref{LastPages}| is placed |\AfterLastShipout|, therefore
% |\lastpageref{LastPages}| is safer to really get the total page number.\\
%
% \pagebreak
%
% \DescribeMacro{\theCurrentPage}
% |\theCurrentPage| gives the current total/absolute page, in contrast to
% |\thepage|, which gives only the page \textit{name} in the current page
% numbering scheme. For example, when there are Roman VII pages in the
% frontmatter and afterwards in the mainmatter you are at arabic page~9,
% then |\theCurrentPage| is~16, whereas |\thepage| is~9. When the page
% \textquotedblleft number\textquotedblright{} (name) is manipulated by
% |\addtocounter{page}{...}| or |\setcounter{page}{...}|, |\theCurrentPage|
% \nolinebreak{ignores} this. Because \texttt{CurrentPage} is a normal
% counter, you can also say e.\,g.~|\Roman{CurrentPage}| to get the value
% in Roman page numbering scheme (e.\,g.~VIII for~8).\\
%
% \phantomsection
% \label{CurrentPageLocal}
% \DescribeMacro{\theCurrentPageLocal}
% |\theCurrentPageLocal| gives the current (arabic) number of pages in the
% current page numbering scheme. |\thepage| and |\theCurrentPageLocal| are
% different e.\,g. when |\addtocounter{page}{...}| or
% |\setcounter{page}{...}| were used.\\
% |\theCurrentPageLocal| can be printed in other formats, e.\,g.
% |\roman{pagesLTS.current.local.roman}|, but probably it only makes
% sense if page numbering scheme and format are the same, e.\,g.
% |\Roman{pagesLTS.current.local.Roman}| or\\
% |\Alph{pagesLTS.current.local.Alph}|.\\
% |\arabic{pagesLTS.current.local.|\ldots|}| probably make sense even
% when combined with another page numbering scheme. And this is exactly
% what |\theCurrentPageLocal| does:\\
% |\def\theCurrentPageLocal{\arabic{pagesLTS.current.local.\pagesLTS@pnc}}|.\\
%
% \DescribeMacro{pagesLTS. page numbering scheme . number }
% If you want to refer to the last page of the first, second,\ldots\ use
% of a page numbering scheme, you can refer to \linebreak
% |pagesLTS.<page numbering scheme>.<number>|, e.\,g.
% |\lastpageref{pagesLTS.Roman.1}|, where |<number>| is the occurrence
% of the page numbering scheme. For details please see
% page~\pageref{pagesLTS.pnscheme.nr}.\\
%
% \bigskip
% \DescribeMacro{\lastpageref}
% For pages with the \texttt{fnsymbol} page numbering scheme,
% |\lastpageref{...}| \nolinebreak{instead} of |\pageref{...}|
% \textit{must} be used. This is required for pages some\-where
% inside of the document as well as the (very) last page(s).
% Because |\lastpageref{...}| is a synonym for |\pageref{...}|,
% where no \xpackage{fnsymbol} page numbering scheme is used,
% it is save(r) to use it for all references to labels provided
% by the \xpackage{pageslts} package.
%
% \subsection[\texttt{\textbackslash pagenumbering\{\ldots \}}]{%
% \texttt{\textbackslash pagenumbering\{\ldots \}}}
% \DescribeMacro{\pagenumbering}
% \subsubsection{If \texttt{\textbackslash pagenumbering\{\ldots \}} is \textbf{not} used\label{sss:nopnc}}
%
% \indent When the \xpackage{pageslts} package is used, but |\pagenumbering{...}|
% (with an argument like arabic, roman, Roman, fnsymbol, alph, or Alph)
% is not used, there should be no problem, except that you might need
% more~(!) compiler runs to get all references right, and some references
% might even be missing (see below). The \xpackage{pageslts}
% package tries to determine the page numbering scheme at the first
% shipout, but success is not guaranteed. Thus please use |\pagenumbering{...}|
% at the beginning of your document!\\
% \indent Without |\pagenumbering{<something>}| (|<something>| e.\,g. = arabic)
% at the beginning of the document, the page numbers might be given in
% arabic \textit{by} (class) \textit{default}, but the \xpackage{pageslts}
% package does not know about this without |\pagenumbering{arabic}|.~--\\
% \indent The label \texttt{pagesLTS.0} is created at the first
% page even if no |\pagenumbering{...}| command is given.
% Maybe have a look at the \xfile{.aux} file after compiling your
% document to detect further labels (of other packages, too).\\
%
% \pagebreak
%
% \subsubsection{If \texttt{\textbackslash pagenumbering\{\ldots \}} is used once\label{sss:once}}
%
% \DescribeMacro{pagesLTS.0}
% At the first page a label \texttt{pagesLTS.0} is created.
% If |\pagenumbering{...}| is used right after |\begin{document}|,
% this is much easier for the \xpackage{pageslts} package
% (and chances for successful placing of all labels are much higher).
%
% \subsubsection{If \texttt{\textbackslash pagenumbering\{\ldots \}} is used more than once}
%
% \indent Everything from the preceding subsubsections applies and additionally
% the following:\\
% \indent When different page numbering schemes are used, e.\,g.
% Roman numbers for the frontmatter and arabic numbers for the
% mainmatter, please use |\pagenumbering{...}| for each of them!
% Even if you do this, the reference to neither the label \texttt{LastPage}
% nor the label \texttt{VeryLastPage} gives the \textbf{total}
% number of pages, but only the number of pages of the last used page
% numbering scheme (which could be exactly what you want, e.\,g. if
% you want to refer to the last page itself and do not want to
% give the total number of pages).\\
% \DescribeMacro{LastPages}
% \indent For remediation the label
% \nolinebreak{\texttt{LastPage}\textbf{\underline{s}}}
% (with \textquotedblleft s\textquotedblright{} at its end) is
% introduced. Please then refer to this label by \newline
% |\lastpageref{LastPages}| instead of \texttt{LastPage} or
% \texttt{VeryLastPage}.\\
%
% \DescribeMacro{pagesLTS.arabic}
% \DescribeMacro{pagesLTS.roman}
% \DescribeMacro{pagesLTS.Roman}
% \DescribeMacro{pagesLTS.alph}
% \DescribeMacro{pagesLTS.Alph}
% \DescribeMacro{pagesLTS.fnsymbol}
% Additionally, at the last page of each page numbering scheme a label
% \texttt{pagesLTS.<numbering scheme>} is placed, where
% \texttt{<numbering scheme>} is e.\,g.\ arabic, roman, Roman, alph,
% Alph,\ldots .\\
% For the \textbf{fnsymbol} page numbering scheme
% |\lastpageref{pagesLTS.fnsymbol}| is needed instead of \newline
% |\pageref{pagesLTS.fnsymbol}|. You can and should use
% |\lastpageref{...}| also for the other page numbering schemes.\\
%
% \bigskip
%
% While at the time of the last revision of the \xpackage{pageslts} package
% no other page numbering schemes were known to the maintainer,
% this package in principle works with every scheme which is
% recognized by the original |\pagenumbering| command.
% But the \xpackage{hyperref} package only then works with crazy page names,
% if the references to those pages are given in a certain way, thus the
% combination of a new page numbering scheme, the \xpackage{hyperref}
% and the \xpackage{pageslts} package might not work.~--
% The \xpackage{pageslts} package by itself also works with schemes,
% which the original |\pagenumbering{...}| does not recognize, but because
% the original |\pagenumbering{...}| is called by the \xpackage{pageslts}
% package, this might cause an error, see subsection~\ref{sec:unknown}!
% (And if the number format is unknown to \LaTeX{}, the pages will have
% no number, and therefore cannot be referenced. You might be able to
% help yourself by using the \xpackage{hyperref} package and
% manually placing |\hypertarget|s and |\href|s.)
%
% \subsubsection{If the same \texttt{\textbackslash pagenumbering\{\ldots \}}\ %
%  scheme is used more than once\label{sss:twice}}
%
% \indent Everything from the preceding subsubsections applies and additionally
% the following:\\
% \DescribeMacro{pagecontinue}
% \indent If the same page numbering scheme is used twice (or even more often)
% in one document (e.\,g. in the frontmatter Roman: I--V, in the
% mainmatter arabic: 1--20, and in the backmatter again Roman: VI--X),
% the second time it is used, the page numbering is either continued (option
% \texttt{pagecontinue} or \texttt{pagecontinue=true} or no option
% \texttt{pagecontinue}; the default) or reset to one (option
% \texttt{pagecontinue=false}). It is even possible to use
% a~page numbering scheme more than twice.\\
%
% \phantomsection
% \label{pagesLTS.pnscheme.nr}
% \DescribeMacro{pagesLTS. page numbering scheme . number }
% If you want to refer to the last page of the first, second,\ldots\ use
% of a page numbering scheme, page~V in the example above, you can refer to
% |pagesLTS.<page numbering scheme>.<number>|, e.\,g.
% |\lastpageref{pagesLTS.Roman.1}|, where |<number>| is the occurrence
% of the page numbering scheme.\\
%
% \pagebreak
%
% If you want to refer to the first page of a page numbering scheme,
% just place a label there, e.\,g.\\
% \indent |\pagenumbering{Roman}| \\
% \indent |\section{Section title\label{RomanSection}}| \\
% (You know where you use |\pagenumbering{...}| and this is the
%  \xpackage{pageslts} package, not the \xpackage{firstpage} one).\\
%
% When you want to give the number of pages of each
% \textquotedblleft sector\textquotedblright{} of the page numbering
% scheme, you can use\\
% \DescribeMacro{pagesLTS. page numbering scheme . number .local.cnt}
% |\lastpages{<page numbering scheme>}{<number>}|, \\
% where |<page numbering scheme>| is e.\,g. Roman, arabic,\ldots\ and
% |<number>| the \textquotedblleft sector\textquotedblright{} number,
% e.\,g.~|\lastpages{Roman}{2}|.\\
% (Internally, the counter has the format
% |pagesLTS.<page numbering scheme>.<number>.local.cnt|.)
%
% \noindent If you used the page numbering scheme Roman for three times,
% you could say
% \begin{verbatim}
% Last Roman page (pagesLTS.Roman): \lastpageref{pagesLTS.Roman}\\
% There are \lastpageref{pagesLTS.Roman.local}~pages with Roman numbers:\\
% \lastpages{Roman}{1}~pages in the first Roman sector
% (\pageref{Roman} -- \lastpageref{pagesLTS.Roman.1}),\\
% \lastpages{Roman}{2}~pages in the second Roman sector
% (\pageref{Roman2} -- \lastpageref{pagesLTS.Roman.2}), and\\
% \lastpages{Roman}{3}~pages in the third Roman sector
% (\pageref{Roman3} -- \lastpageref{pagesLTS.Roman.3}.\\
% \end{verbatim}
% to get
% \begin{quote}
% Last Roman page (pagesLTS.Roman): VIII\\
% There are 8 pages with Roman numbers:\\
% 3 pages in the first Roman sector (I -- III),\\
% 4 pages in the second Roman sector (IV -- VII), and\\
% 3 pages in the third Roman sector (VIII -- X).\\
% \end{quote}
% (see e.\,g. the compiled \textsf{pageslts-example} file).\\
%
% If you want to continue one page numbering scheme, but later on
% (third use of it, or for another page numbering scheme) want to
% reset the page number, just say |\setcounter{page}{1}| there.\\
%
% \pagebreak
%
% \noindent In your document the code
% \begin{verbatim}
% \makeatletter
% \renewcommand{\@evenfoot}%
%  {\normalsize\slshape DRAFT \today\hfil \upshape page {\thepage} (\theCurrentPage) of\ %
%   \lastpageref{pagesLTS.Roman} + \lastpageref{pagesLTS.arabic}\ = \lastpageref{LastPages} pages%
%  }
% \renewcommand{\@oddfoot}{\@evenfoot}
% \makeatother
%\end{verbatim}
% \noindent creates footers like\\
%
% \textquotedblleft\mbox{\textsl{DRAFT \today}\hspace{1cm}page V (5) of VII + 35 = 42 pages}\textquotedblright\\
%
% \noindent or\\
%
% \textquotedblleft\mbox{\textsl{DRAFT \today}\hspace{1cm}page 10 (17) of VII + 35 = 42 pages}\textquotedblright\\
%
% \noindent in the compiled document (cf.~the \texttt{pageslts-example} file).
%
% \noindent Code like
% \begin{verbatim}
%  This book has \lastpageref{pagesLTS.Roman}+\lastpageref{pagesLTS.arabic} pages %
%  (\lastpageref{LastPages} pages in total).
% \end{verbatim}
% \noindent produces output like
% \begin{quote}
% This book has X+85 pages (95 pages in total).
% \end{quote}
% \noindent (when using the \xpackage{hyperref} package,
% the references are even hyperlinked).\\
%
% If |\addtocounter{page}{...}| or |\setcounter{page}{...}| have been used,
% the local version of \texttt{CurrentPage} can be used, |\theCurrentPageLocal|,
% see subsection~\ref{CurrentPageLocal}.\\
%
% \subsection{\xpackage{papermas}(s) package\label{ss:papermas}}
% There is a kind of an add-on to this package, the \xpackage{papermas}
% package, which can be used to compute the number of sheets of paper
% needed to print a document (you can print more than one page of a
% document on one sheet of paper) as well as the approximate mass
% of the printout. Please see the \ref{ss:Downloads} subsection.
%
% \newpage
%
% \section{A few warnings\label{sec:warn}}
%
% \subsection{Hyperref and repeated page numbers}
%
% When two (or more) different page numbering schemes are used, or the page number
% is reset, or for any other reason there are two pages with the same number
% (maybe in different format, e.\,g. $1$ and \textsc{I}), and \xpackage{hyperref} has
% not been configured right, this can cause problems. Use \xpackage{hyperref} with
% \texttt{plainpages=false} and \texttt{pdfpagelabels=true}, and everything should
% be fine.\\
% More details can be found at
% \ \url{http://www.tex.ac.uk/cgi-bin/texfaq2html?label=pdfpagelabels}.
%
% \subsection{\texttt{\textbackslash AtEndDocument}\label{ssec:aed}}
%
% \indent The output of a \LaTeXe\ run is not independent of the order
% in which packages are loaded. It is often the case that the same
% formats for which one must put tables and figure at the end,
% are the ones in which endnotes are also required.
% If one wants to use |\AtEndDocument| here as well (as done for
% |\lastpageref{LastPage}|), then it is easy to get to three separate
% uses of |\AtEndDocument| (assuming one uses this for the endnotes
% as well). Clearly it is not safe for any package writer or user
% to assume that no material will follow what they put into
% |\AtEndDocument|. Therefore a message, which begins with
% \texttt{AED}, is included in every usage of |\AtEndDocument|,
% and it is tried to minimize any side effects the usage may have.\\
%
% As now \textsc{Heiko Oberdiek}'s \texttt{atveryend} package is used,
% the references |\lastpageref{VeryLastPage}| and \newline
% |\lastpageref{LastPages}| should work all right. About how to get the
% \texttt{atveryend} package, please see subsection~\ref{ss:Downloads}.
%
% \subsection{Interaction with very old versions of the \xpackage{endfloat} %
%  package\label{sec:endfloat}}
%
% \indent The \emph{very} old version~2.0 (and earlier) of the \xpackage{endfloat}
% package actually redefined the |\enddocument| command,
% and so interfered drastically with the \LaTeXe\ commands which make use
% of |\AtEndDocument|. Newer versions of \xpackage{endfloat} exist
% (at the time of writing this documentation: v2.5d as of 2011/12/25)
% in modern documentation form, which should be available from
% the same source where you received this file, see subsection~\ref{ss:Downloads}.\\
% A note is placed in the style file at the |\RequirePackage| section,
% and later it is even checked whether a (very) old \xpackage{endfloat}
% package is in use. If it is, a warning or even an error message
% is given, depending on \xpackage{endfloat} version.
% This assumes, that the old versions of \xpackage{endfloat} at least gave
% a version date, of course.\\
% If you want your \texttt{LastPage} to label the last page of
% these end floats, you need to load \xpackage{pageslts} \emph{after}
% loading \xpackage{endfloat}, or to use \texttt{VeryLastPage} instead.
% If, on the other hand, you \emph{want} \texttt{LastPage} to refer
% to the (not so) last page, exclusive of the floats at the end,
% then load in the reverse order. Independent from the order of
% \xpackage{pageslts} and \xpackage{endfloat}, you will still need the
% modified\footnote{New versions are available for over $15$~years now,
% so it might be time to update, if you did not do it already.} version of
% \xpackage{endfloat}.\\
% Using the
% \nolinebreak{\texttt{LastPage}\textbf{\underline{s}}\ (\textit{s}!)}\ label
% should get you to the last page in all cases: |\lastpageref{LastPages}|.\\
%
% Other \LaTeX209\ (!) packages also seem to like to redefine
% |\enddocument|. In addition to the old \xpackage{endfloat},
% \xpackage{harvard} comes to mind. All of these will need to be
% modified swiftly. \textbf{If possible, update to \LaTeXe{}!}
%
% \pagebreak
%
% \subsection{\xpackage{showkeys} package\label{ssec:showkeys}}
%
% When the \xpackage{showkeys} package has been loaded in \texttt{draft} mode,
% in the margin for each label a~box is displayed with the name of the label.
% \xpackage{showkeys} accomplishes this by redefining |\label|, but
% \xpackage{pageslts} does not use |\label|, but writes directly to the
% {\nobreak |\jobname|\xfile{.aux}-}\allowbreak file, and this is generally done
% \emph{after} the according page has shipped out, therefore no box can be placed
% on the \emph{preceding} page. At~least \xpackage{pageslts} gives a warning, that
% \xpackage{showkeys} cannot present the respective label.
%
% \subsection{\xpackage{lastpage} package\label{ssec:lastpage}}
%
% This package first started as a revision of the \xpackage{lastpage}
% package of \textsc{Jeffrey P. Goldberg} (jeffrey+news at goldmark dot org),
% but it became obvious that a replacement was needed to accomplish what this
% package does. For backward compatibility, a~label named |LastPage| is provided.
% Thus |\usepackage{lastpage}| can be replaced by\\
% |\usepackage[pagecontinue=false,%|\newline
% |alphMult=0,AlphMulti=0,fnsymbolmult=false,romanMult=false,RomanMulti=false]{pageslts}|,\\
% if the behaviour of the \xpackage{lastpage} package should be simulated.
% Using \textbf{old}~(!) versions of the \xpackage{lastpage} before the \xpackage{pageslts}
% before the \xpackage{hyperref} [2012/11/06 v6.83m]
% package results in multiply definitions of the |LastPage| label.
% While the \xpackage{pageslts} package cancels the command |\lastpage@putlabel|
% from the old \xpackage{lastpage} package (because it does this itself, and better),
% \xpackage{hyperref} redefines |\lastpage@putlabel| and thereby reintroduces it again
% (\xpackage{hyperref} should probably check for the version of the \xpackage{lastpage}
% package and/or whether the \xpackage{pageslts} package was also loaded.)
%
% \subsection{Using an unknown page numbering scheme\label{sec:unknown}}
%
% \indent I do not know whether \LaTeXe\ can handle another page numbering
% scheme (e.\,g.~Hebraic), but if you want to use it, this should be
% no problem for the \xpackage{pageslts} package. But the original
% |\pagenumbering{...}| as well as the \xpackage{hyperref} package
% (if~used) might want to vote against it, especially when used together
% with the \xpackage{pageslts} package. Especially especially
% (\textit{sic}!) if the last page uses this new page numbering scheme,
% you should check everything double (at least).\\
% (And if the number format is unknown to \LaTeX{}, the pages will have
% no number, and therefore cannot be referenced. You might be able to
% help yourself by using the \xpackage{hyperref} package and
% manually placing |\hypertarget|s and |\href|s.)
%
% \subsection{Page counter overflow\label{ss:overflow}}
%
% \indent Without the use of the \xpackage{alphalph} package, the\\
% \textquotedblleft ranges of supported counter values are more or less
% restricted. Only \cs{arabic} can be used with any counter
% value \TeX\ supports.
% \begin{quote}
% \catcode`\|=12 %
% \begin{tabular}{@{}l|l|l|l@{}}
% Presentation & Supported & Ignored & Error message\\
% command      & domain    & values  & `Counter too large'\\
% \hline
% \cs{arabic}
%   & \ttfamily -MAX..MAX & &\\
% \cs{roman}, \cs{Roman}
%   & \ttfamily 1..MAX & \ttfamily -MAX..0 &\\
% \cs{alph}, \cs{Alph}
%   & \ttfamily 1..26 & 0 & \ttfamily -MAX..-1, 27..MAX\\
% \cs{fnsymbol}
%   & \ttfamily 1..9 & 0 & \ttfamily -MAX..-1, 10..MAX\\
% \hline
% \end{tabular}\\[1ex]
% \texttt{MAX} = \texttt{2147483647}
% \end{quote}
% \textquotedblright{} (\textsc{Heiko Oberdiek}:
% The \xpackage{alphalph} package, 2010/04/18, v2.3, first table, p.~2).\\
% Please see subsubsections~\ref{sss:alphalph} and \ref{sss:romanMult} for
% instructions how to overcome these limitations.
%
% \newpage
%
% \subsection{Using the \xpackage{fnsymbol} page numbering scheme\label{ss:fnsymbol}}
%
% \indent Using the \texttt{fnsymbol} page numbering scheme can result in
% problems -- big ones!\\
% \textbf{When using this page numbering scheme, it is very important to use}
% \ |\lastpageref{...}| \textbf{instead of}\ |\pageref{...}| \textbf{for any%
% link to any label provided by the \xpackage{pageslts} package.}\\
% While the \xpackage{pageslts} package tries really very hard to circumvent
% any problem, other packages might screw up -- and quite totally for that.
% So,~you have been warned!\\
%
% \begin{description}
% \item[-] There can be a counter overflow, see preceding subsection~\ref{ss:overflow}.
%
% \item[-] Adobe Reader DC 2015.008.20082 does not show the correct page names
% for all pages with \texttt{fnsymbol} page numbering scheme (see the example file),
% while at least the (\ldots\ of \ldots )  part of the page number is displayed correctly
% (see page~\pageref{logical page numbers}, tip about logical page numbers).
% When the \xpackage{alphalph} package and the \xpackage{pageslts} package with
% |fnsymbolmult| option are used, more page names are presented correctly by the Reader.
% (Adobe Reader~X even got all pages right.)
% \end{description}
%
% \newpage
%
% \section{Alternatives\label{sec:Alternatives}}
%
% There are similar packages, which do (or do not) similar things. As I
% neither know what exactly you want to accomplish when using this package
% (e.\,g.~page number vs. page name, hyperlinks or not), nor what resources
% you have (e.\,g.~\eTeX{}), here is a list of some possible alternatives:\\
%
% \DescribeMacro{LastPage}
% \begin{description}
% \item[-] The \xpackage{LastPage} package also provides the
%  \texttt{LastPage} label (but not \texttt{VeryLastPage} or
%  \texttt{LastPages}). If you only want this and have a quite
%  limited amount of \TeX\ resources, you might want to use that
%  package instead.
%
% \item[-] If \LaTeX2.09\ is still used, and if you are unable
%  to switch to \LaTeXe{}, the \LaTeX2.09{} compatible
%  \xpackage{lastpage209.sty} can be used (which is also contained
%  in the recent \xpackage{LastPage} package).
% \end{description}
%
% \DescribeMacro{totpages}
% \begin{description}
% \item[-] The \xpackage{totpages} package provides a
%  \texttt{totpages} label similar to \texttt{LastPages},
%  but |\AtEndDocument| instead of |\AfterLastShipout|. Therefore
%  you should stay with \xpackage{pageslts}. The \xpackage{totpages}
%  package additionally computes the number of paper sheets needed to
%  (double) print the document (with one, two, three,\ldots\ pages
%  on one sheet of paper). This can also be accomplished with the
%  \xpackage{papermas} package.
% \end{description}
%
% \DescribeMacro{totcount}
% \begin{description}
% \item[-] The \xpackage{totcount} package provides the last value
%  of a counter, thus also the value of the \texttt{page} counter.
%  You do not get a hyperlink to the last page, only the numerical
%  value of the last page name is given (i.\,e. X+72 pages gives 72
%  instead of 82 as total number of pages), and the number of pages
%  can be changed e.\,g. by |\addtocounter|.
% \end{description}
%
% \DescribeMacro{nofm}
% \begin{description}
% \item[-] \textquotedblleft There is a package \xpackage{nofm.sty}
%  available, but some versions of it are defective, and most don't
%  work with \xpackage{fancyhdr} because they take over the complete
%  page layout.\textquotedblright{} (\textsc{Piet van Oostrum}:
%  Page layout in \LaTeX{}, March~2, 2004, section~16; fancyhdr.pdf)\\
%  \xpackage{nofm} as of 1991/02/25 (without version number), available at\\
%  \url{http://mirror.ctan.org/obsolete/macros/latex209/contrib/misc/nofm.sty},\\
%  does not work with e.\,g. \xpackage{hyperref}, redefines |\enddocument|
%  as well as |\@oddhead|, |\@evenhead|, |\@oddfoot|, and |\@evenfoot|.\\
%  If you know the (\url{https://www.CTAN.org}) location of a%
%  \textbf{working}~(!) version, please send me an e-mail, thanks!
% \end{description}
%
% \DescribeMacro{count1to}
% \begin{description}
% \item[-] You may want to have a look at the \xpackage{count1to} package.
% \end{description}
%
% \DescribeMacro{zref}
% \begin{description}
% \item[-] The \xpackage{zref} package of \textsc{Heiko Oberdiek} requires
%  \eTeX . \xpackage{pageslts} does not require \eTeX , but if you
%  already have \eTeX , you may have a look at the extensive
%  \xpackage{zref} package, whether it suits your needs better
%  (or additionally or whatsoever).
% \end{description}
%
% \bigskip
%
% \noindent (You programmed or found another alternative,
%  which is available at \url{https://www.CTAN.org}?\\
%  OK, send an e-mail to me with the name, location at CTAN,
%  and a short notice, and I will probably include it in the list above.)\\
%
% \smallskip
%
% \noindent About how to get those packages, please see subsection~\ref{ss:Downloads}.
%
% \newpage
%
% \section{Example}
%
%    \begin{macrocode}
%<*example>
\documentclass[british]{article}[2014/09/29]% v1.4h
\usepackage{lipsum}[2014/07/27]%              v1.3
\usepackage[draft]{showkeys}[2014/10/28]%     v3.17
%%      Use final instead of draft to hide the keys. %%
\usepackage{hyperref}[2012/11/06]%            v6.83m
\hypersetup{%
 extension=pdf,%
 plainpages=false,%
 pdfpagelabels=true,%
 hyperindex=false,%
 pdflang={en},%
 pdftitle={pageslts package example},%
 pdfauthor={Hans-Martin Muench},%
 pdfsubject={Example for the pageslts package},%
 pdfkeywords={LaTeX, pageslts, H.-Martin Muench},%
 pdfview=Fit,%
 pdfstartview=Fit,%
 pdfpagelayout=SinglePage,%
 bookmarksopen=true%
}
\usepackage[pagecontinue=true,alphMult=ab,AlphMulti=AB,fnsymbolmult=true,%
 romanMult=true,RomanMulti=true]{pageslts}[2015/12/21]% v1.2f
%% These are the default options. %%

\makeatletter
  \renewcommand{\@evenfoot}%
   {Page \thepage\ (\theCurrentPage; local: \theCurrentPageLocal) of %
    \lastpageref{pagesLTS.roman}(\lastpageref{pagesLTS.roman.local}) + %
    \lastpageref{pagesLTS.Roman}(\lastpageref{pagesLTS.Roman.local}) + %
    \lastpageref{pagesLTS.arabic}(\lastpageref{pagesLTS.arabic.local}) + %
    \lastpageref{pagesLTS.fnsymbol}(\lastpageref{pagesLTS.fnsymbol.local}) + %
    \lastpageref{pagesLTS.alph}(\lastpageref{pagesLTS.alph.local}) + %
    \lastpageref{pagesLTS.Alph}(\lastpageref{pagesLTS.Alph.local}) = %
    \lastpageref{LastPages} pages.%
   }
  \renewcommand{\@oddfoot}{\@evenfoot}
  \def\pagesLTSexampleArabic{3}
  \def\pagesLTSexamplealph{23}
  \gdef\unit#1{\mathord{\thinspace\mathrm{#1}}}%
\makeatother

\listfiles

\begin{document}
\pagenumbering{roman}
%% Note the first \pagenumbering immediately behind \begin{document}! %%
%%\addtocounter{page}{-2} %%

\section*{Example for pageslts}
\addcontentsline{toc}{section}{Example for pageslts}
\markboth{Example for pageslts}{Example for pageslts}

This example demonstrates the most common uses of package\\
\textsf{pageslts}, v1.2f as of 2015/12/21 (HMM).\\
The used options were \texttt{pagecontinue=true},
\texttt{alphMult=ab}, \texttt{AlphMulti=AB}, \linebreak
\texttt{fnsymbolmult=true},
\texttt{romanMult=true}, and \texttt{RomanMulti=true}
(the default ones).\\
For more details please see the documentation!\\

\label{keys} To hide the \pageref{keys}{\qquad } use option
\texttt{final} instead of \texttt{draft} with the \textsf{showkeys}
package (or remove the package call from the preamble of
this document).\\

\textbf{Hyperlinks or not:} If the \textsf{hyperref} package is loaded,
the references are also hyperlinked:\\
\smallskip
Page \thepage\ (\theCurrentPage; local: \theCurrentPageLocal) of %
\lastpageref{pagesLTS.roman}(\lastpageref{pagesLTS.roman.local}) + %
\lastpageref{pagesLTS.Roman}(\lastpageref{pagesLTS.Roman.local}) + %
\lastpageref{pagesLTS.arabic}(\lastpageref{pagesLTS.arabic.local}) + %
\lastpageref{pagesLTS.fnsymbol}(\lastpageref{pagesLTS.fnsymbol.local}) + %
\lastpageref{pagesLTS.alph}(\lastpageref{pagesLTS.alph.local}) + %
\lastpageref{pagesLTS.Alph}(\lastpageref{pagesLTS.Alph.local}) = %
\lastpageref{LastPages} pages.\\
If the \textsf{hyperref} package is loaded, but the hyperlinks of the
references shall be suppressed, \verb|\pageref*{...}|
and \verb|\lastpageref*{...}| can be used:\\
Page \thepage\ (\theCurrentPage; local: \theCurrentPageLocal) of %
\lastpageref*{pagesLTS.roman}(\lastpageref*{pagesLTS.roman.local}) + %
\lastpageref*{pagesLTS.Roman}(\lastpageref*{pagesLTS.Roman.local}) + %
\lastpageref*{pagesLTS.arabic}(\lastpageref*{pagesLTS.arabic.local}) + %
\lastpageref*{pagesLTS.fnsymbol}(\lastpageref*{pagesLTS.fnsymbol.local}) + %
\lastpageref*{pagesLTS.alph}(\lastpageref*{pagesLTS.alph.local}) + %
\lastpageref*{pagesLTS.Alph}(\lastpageref*{pagesLTS.Alph.local}) = %
\lastpageref*{LastPages} pages.\\


\textbf{Trademarks} appear throughout this example without any
trademark symbol; they are the property of their respective
trademark owner. There is no intention of infringement; the
usage is to the benefit of the trademark owner.\\

\textbf{Tip}: Use \textit{logical page numbers}
for the display of the pdf (in Adobe Reader DC 2015.008.20082:
Edit $>$ Preferences $>$ Categories: Page Display $>$
Page Content and Information: Use logical page numbers)!\\

You want negative page numbers? Not only arabic, but even roman,
Roman, alph, Alph or fnsymbol ones? No problem, e.\,g. just give a\\
\verb|\addtocounter{page}{|-\textit{some number}\verb|}| in the
source code of this example file (or uncomment the prepared line)!

\bigskip

Save per page about $200\unit{ml}$ water, $2\unit{g}$ CO$_{2}$
and $2\unit{g}$ wood:\\
Therefore please print only if this is really necessary.

\pagebreak

\tableofcontents

\newpage

\pagenumbering{roman}
%% in case the page numbering is changed before,
%% otherwise pagesLTS.current.local.roman on this page
%% would be undefined

\section{roman}

\noindent (\texttt{roman} page numbering was started before,
because the page numbering scheme was needed to start at
the first page, of course.)\\

\noindent First page (\verb|\lastpageref{pagesLTS.0}|):
\lastpageref{pagesLTS.0}\\

\noindent The page (\verb|\thepage|): \thepage \\

\noindent Current page (\verb|\theCurrentPage|),
i.\,e. counted continuously from the first page): \theCurrentPage \\
You can get this also in other formats:
\roman{CurrentPage}, \Roman{CurrentPage}, \arabic{CurrentPage},
\fnsymbol{CurrentPage}, \alph{CurrentPage}, \Alph{CurrentPage}.

\noindent CurrentPageLocal (\verb|\theCurrentPageLocal|),
i.\,e. counted continuously from the first page of the
current page numbering scheme): \theCurrentPageLocal \\
You can get also this in other formats, too:
\roman{pagesLTS.current.local.roman}, \Roman{pagesLTS.current.local.roman},
\arabic{pagesLTS.current.local.roman}, \fnsymbol{pagesLTS.current.local.roman},
\alph{pagesLTS.current.local.roman}, \Alph{pagesLTS.current.local.roman},
but probably it only makes sense if page numbering scheme and format are
the same, e.\,g.\\
\verb|\Roman{pagesLTS.current.local.Roman}|\\
or \verb|\Alph{pagesLTS.current.local.Alph}|. \verb|\arabic{...}| could
make sense even if combined with another page numbering scheme.
And this is exactly what \verb|\theCurrentPageLocal| does:\\
\nolinebreak{\verb|\def\theCurrentPageLocal{\arabic{pagesLTS.current.local.\pagesLTS@pnc}}|.}\\

\noindent Last roman page (pagesLTS.roman): \lastpageref{pagesLTS.roman}{\hskip4em }
(There are \lastpageref{pagesLTS.roman.local} pages with roman numbers.)\\

\noindent Last Roman page (pagesLTS.Roman): \lastpageref{pagesLTS.Roman}{\hskip3em }
(There are \lastpageref{pagesLTS.Roman.local}~pages with Roman numbers:\\
\lastpages{Roman}{1}~pages in the first Roman sector
(\pageref{Roman}{\hskip3em }-\lastpageref{pagesLTS.Roman.1}{\hskip3em }),\\
\lastpages{Roman}{2}~pages in the second Roman sector
(\pageref{Roman2}{\hskip3em }-\lastpageref{pagesLTS.Roman.2}{\hskip3em }), and\\
\lastpages{Roman}{3}~pages in the third Roman sector
(\pageref{Roman3}{\hskip3em }-\lastpageref{pagesLTS.Roman.3}{\hskip3em }).\\

When the option \texttt{pagecontinue=false} is used with the
\textsf{pageslts} package, the \verb|\lastpageref{pagesLTS.Roman}| will
point to the same page as before, but this will have a lower number.\\
The \verb|\lastpageref{pagesLTS.Roman.local}| will not change,
because the number of pages does not change (only the page numbers).\\

\noindent Last arabic page (pagesLTS.arabic): \lastpageref{pagesLTS.arabic}{\hskip5em }
(There are only \lastpageref{pagesLTS.arabic.local} pages with arabic numbers,
because an \verb|\addtocounter{page}{|\pagesLTSexampleArabic\verb|}| was used.)\\

\noindent Last fnsymbol page (pagesLTS.fnsymbol): \lastpageref{pagesLTS.fnsymbol} \\
(\verb|\lastpageref{pagesLTS.fnsymbol}| -- never \verb|\pageref{pagesLTS.fnsymbol}|!)\\
(There are \lastpageref{pagesLTS.fnsymbol.local} pages with fnsymbol numbers.)\\

\noindent Last alph page (pagesLTS.alph): \lastpageref{pagesLTS.alph}{\hskip4em }
(There are only \lastpageref{pagesLTS.alph.local} pages with alph numbers,
because an \verb|\addtocounter{page}{|\pagesLTSexamplealph\verb|}| was used.)\\

\noindent Last Alph page (pagesLTS.Alph): \lastpageref{pagesLTS.Alph}{\hskip4em }
(There are \lastpageref{pagesLTS.Alph.local} pages with Alph numbers.)\\

\noindent Last page's \textit{name} (LastPage): \lastpageref{LastPage}\\

\noindent Very last page's \textit{name} (VeryLastPage): \lastpageref{VeryLastPage}\\
(\texttt{lastpage} and \texttt{VeryLastPage} are identical, unless
a package with output \linebreak
\verb|\AtEndDocument| after the \textsf{pageslts} package was added.)\\

\noindent Last page's \textit{number} (LastPages): \lastpageref{LastPages}{\hskip3em }
(=~total number of pages)\\

\lipsum[1-3]

\newpage

\pagenumbering{Roman}

\section{Roman\label{Roman}}
\subsection{Common Roman page numbering}

\noindent First page (\verb|\lastpageref{pagesLTS.0}|):
\lastpageref{pagesLTS.0}\\

\noindent The page (\verb|\thepage|): \thepage \\

\noindent Current page (\verb|\theCurrentPage|),
i.\,e. counted continuously from the first page): \theCurrentPage \\

\noindent CurrentPageLocal (\verb|\theCurrentPageLocal|),
i.\,e. counted continuously from the first page of the
current page numbering scheme): \theCurrentPageLocal \\

\noindent Last roman page (pagesLTS.roman): \lastpageref{pagesLTS.roman}{\hskip4em }
(There are \lastpageref{pagesLTS.roman.local} pages with roman numbers.)\\

\noindent Last Roman page (pagesLTS.Roman): \lastpageref{pagesLTS.Roman}{\hskip3em }
(There are \lastpageref{pagesLTS.Roman.local}~pages with Roman numbers:\\
\lastpages{Roman}{1}~pages in the first Roman sector
(\pageref{Roman}{\hskip3em }-\lastpageref{pagesLTS.Roman.1}{\hskip3em }),\\
\lastpages{Roman}{2}~pages in the second Roman sector
(\pageref{Roman2}{\hskip3em }-\lastpageref{pagesLTS.Roman.2}{\hskip3em }), and\\
\lastpages{Roman}{3}~pages in the third Roman sector
(\pageref{Roman3}{\hskip3em }-\lastpageref{pagesLTS.Roman.3}{\hskip3em }).\\

\noindent Last arabic page (pagesLTS.arabic): \lastpageref{pagesLTS.arabic}{\hskip5em }
(There are only \lastpageref{pagesLTS.arabic.local} pages with arabic numbers,
because an \verb|\addtocounter{page}{|\pagesLTSexampleArabic\verb|}| was used.)\\

\noindent Last fnsymbol page (pagesLTS.fnsymbol): \lastpageref{pagesLTS.fnsymbol} \\
(\verb|\lastpageref{pagesLTS.fnsymbol}| -- never \verb|\pageref{pagesLTS.fnsymbol}|!)\\
(There are \lastpageref{pagesLTS.fnsymbol.local} pages with fnsymbol numbers.)\\

\noindent Last alph page (pagesLTS.alph): \lastpageref{pagesLTS.alph}{\hskip4em }
(There are only \lastpageref{pagesLTS.alph.local} pages with alph numbers,
because an \verb|\addtocounter{page}{|\pagesLTSexamplealph\verb|}| was used.)\\

\noindent Last Alph page (pagesLTS.Alph): \lastpageref{pagesLTS.Alph}{\hskip4em }
(There are \lastpageref{pagesLTS.Alph.local} pages with Alph numbers.)\\

\noindent Last page's \textit{name} (LastPage): \lastpageref{LastPage}\\

\noindent Very last page's \textit{name} (VeryLastPage): \lastpageref{VeryLastPage}\\
(\texttt{LastPage} and \texttt{VeryLastPage} are identical, unless
a package with output \linebreak
\verb|\AtEndDocument| after the \textsf{pageslts} package was added.)\\

\noindent Last page's \textit{number} (LastPages): \lastpageref{LastPages}{\hskip3em }
(=~total number of pages)\\

\lipsum[1-4]

\newpage

\subsection{Last page of first Roman sector}
\verb|\lastpageref{pagesLTS.Roman}| does \textbf{not}
refer to this page (but there: \lastpageref{pagesLTS.Roman}),
because the option \texttt{pagecontinue=true}
was chosen. When a reference to this page is wanted,\\
\verb|\lastpageref{pagesLTS.Roman.1}| can be used: \lastpageref{pagesLTS.Roman.1}.\\

\bigskip
There are \lastpages{Roman}{1}~pages (\verb|\lastpages{Roman}{1}|)
in this first Roman sector.\\
The Roman page numbering scheme is continued later in section~\ref{Roman2})!

\newpage

\pagenumbering{arabic}

\section{arabic}

\subsection{Standard page numbering}

\noindent First page (\verb|\lastpageref{pagesLTS.0}|):
\lastpageref{pagesLTS.0}\\

\noindent The page (\verb|\thepage|): \thepage \\

\noindent Current page (\verb|\theCurrentPage|),
i.\,e. counted continuously from the first page): \theCurrentPage \\

\noindent CurrentPageLocal (\verb|\theCurrentPageLocal|),
i.\,e. counted continuously from the first page of the
current page numbering scheme): \theCurrentPageLocal \\

\noindent Last roman page (pagesLTS.roman): \lastpageref{pagesLTS.roman}{\hskip4em }
(There are \lastpageref{pagesLTS.roman.local} pages with roman numbers.)\\

\noindent Last Roman page (pagesLTS.Roman): \lastpageref{pagesLTS.Roman}{\hskip3em }
(There are \lastpageref{pagesLTS.Roman.local}~pages with Roman numbers:\\
\lastpages{Roman}{1}~pages in the first Roman sector
(\pageref{Roman}{\hskip3em }-\lastpageref{pagesLTS.Roman.1}{\hskip3em }),\\
\lastpages{Roman}{2}~pages in the second Roman sector
(\pageref{Roman2}{\hskip3em }-\lastpageref{pagesLTS.Roman.2}{\hskip3em }), and\\
\lastpages{Roman}{3}~pages in the third Roman sector
(\pageref{Roman3}{\hskip3em }-\lastpageref{pagesLTS.Roman.3}{\hskip3em }).\\

\noindent Last arabic page (pagesLTS.arabic): \lastpageref{pagesLTS.arabic}{\hskip5em }
(There are only \lastpageref{pagesLTS.arabic.local} pages with arabic numbers,
because an \verb|\addtocounter{page}{|\pagesLTSexampleArabic\verb|}| was used.)\\

\noindent Last fnsymbol page (pagesLTS.fnsymbol): \lastpageref{pagesLTS.fnsymbol} \\
(\verb|\lastpageref{pagesLTS.fnsymbol}| -- never \verb|\pageref{pagesLTS.fnsymbol}|!)\\
(There are \lastpageref{pagesLTS.fnsymbol.local} pages with fnsymbol numbers.)\\

\noindent Last alph page (pagesLTS.alph): \lastpageref{pagesLTS.alph}{\hskip4em }
(There are only \lastpageref{pagesLTS.alph.local} pages with alph numbers,
because an \verb|\addtocounter{page}{|\pagesLTSexamplealph\verb|}| was used.)\\

\noindent Last Alph page (pagesLTS.Alph): \lastpageref{pagesLTS.Alph}{\hskip4em }
(There are \lastpageref{pagesLTS.Alph.local} pages with Alph numbers.)\\

\noindent Last page's \textit{name} (LastPage): \lastpageref{LastPage}\\

\noindent Very last page's \textit{name} (VeryLastPage): \lastpageref{VeryLastPage}\\
(\texttt{LastPage} and \texttt{VeryLastPage} are identical, unless
a package with output \linebreak
\verb|\AtEndDocument| after the \textsf{pagesLTS} package was added.)\\

\noindent Last page's \textit{number} (LastPages): \lastpageref{LastPages}{\hskip3em }
(=~total number of pages)\\

\lipsum[1-4]
\newpage

\subsection[Empty page style]{Also an empty page style is no problem %
for the current or total page count}

\bigskip

\thispagestyle{empty}

\noindent First page (\verb|\lastpageref{pagesLTS.0}|):
\lastpageref{pagesLTS.0}\\

\noindent The page (\verb|\thepage|): \thepage \\

\noindent Current page (\verb|\theCurrentPage|),
i.\,e. counted continuously from the first page): \theCurrentPage \\

\noindent CurrentPageLocal (\verb|\theCurrentPageLocal|),
i.\,e. counted continuously from the first page of the
current page numbering scheme): \theCurrentPageLocal \\

\noindent Last roman page (pagesLTS.roman): \lastpageref{pagesLTS.roman}{\hskip4em }
(There are \lastpageref{pagesLTS.roman.local} pages with roman numbers.)\\

\noindent Last Roman page (pagesLTS.Roman): \lastpageref{pagesLTS.Roman}{\hskip3em }
(There are \lastpageref{pagesLTS.Roman.local}~pages with Roman numbers:\\
\lastpages{Roman}{1}~pages in the first Roman sector
(\pageref{Roman}{\hskip3em }-\lastpageref{pagesLTS.Roman.1}{\hskip3em }),\\
\lastpages{Roman}{2}~pages in the second Roman sector
(\pageref{Roman2}{\hskip3em }-\lastpageref{pagesLTS.Roman.2}{\hskip3em }), and\\
\lastpages{Roman}{3}~pages in the third Roman sector
(\pageref{Roman3}{\hskip3em }-\lastpageref{pagesLTS.Roman.3}{\hskip3em }).\\

\noindent Last arabic page (pagesLTS.arabic): \lastpageref{pagesLTS.arabic}{\hskip5em }
(There are only \lastpageref{pagesLTS.arabic.local} pages with arabic numbers,
because an \verb|\addtocounter{page}{|\pagesLTSexampleArabic\verb|}| was used.)\\

\noindent Last fnsymbol page (pagesLTS.fnsymbol): \lastpageref{pagesLTS.fnsymbol} \\
(\verb|\lastpageref{pagesLTS.fnsymbol}| -- never \verb|\pageref{pagesLTS.fnsymbol}|!)\\
(There are \lastpageref{pagesLTS.fnsymbol.local} pages with fnsymbol numbers.)\\

\noindent Last alph page (pagesLTS.alph): \lastpageref{pagesLTS.alph}{\hskip4em }
(There are only \lastpageref{pagesLTS.alph.local} pages with alph numbers,
because an \verb|\addtocounter{page}{|\pagesLTSexamplealph\verb|}| was used.)\\

\noindent Last Alph page (pagesLTS.Alph): \lastpageref{pagesLTS.Alph}{\hskip4em }
(There are \lastpageref{pagesLTS.Alph.local} pages with Alph numbers.)\\

\noindent Last page's \textit{name} (LastPage): \lastpageref{LastPage}\\

\noindent Very last page's \textit{name} (VeryLastPage): \lastpageref{VeryLastPage}\\
(\texttt{LastPage} and \texttt{VeryLastPage} are identical, unless
a package with output \linebreak
\verb|\AtEndDocument| after the \textsf{pageslts} package was added.)\\

\noindent Last page's \textit{number} (LastPages): \lastpageref{LastPages}{\hskip3em }
(=~total number of pages)\\

\lipsum[1-4]

\newpage

\subsection[addtocounter, setcounter]{Neither %
\texttt{\textbackslash addtocounter\{page\}} nor %
\texttt{\textbackslash setcounter\{page\}} is a problem for the %
current or total page numbers}

(Here is an \verb|\addtocounter{page}{|\pagesLTSexampleArabic\verb|}|
in the source code.)\\
\addtocounter{page}{\pagesLTSexampleArabic}

\noindent The page (from \verb|\thepage| command): \thepage \\

\noindent Current page (from \verb|\theCurrentPage|\ %
command), i.\,e. counted continuously from the first page): \theCurrentPage \\

\noindent CurrentPageLocal (from \verb|\theCurrentPageLocal|\ %
command), i.\,e. counted continuously from the first page of the
current page numbering scheme): \theCurrentPageLocal\\

\noindent Last page's number (LastPages): \lastpageref{LastPages}{\hskip3em }
(= total number of pages)\\

\lipsum[1-7]

\newpage

\pagenumbering{fnsymbol}

\section{fnsymbol}

Adobe Reader DC 2015.008.20082 does not show the correct page names
for all pages with \texttt{fnsymbol} page numbering scheme,
while at least the (\ldots\ of \ldots )  part of the page number is displayed correctly.
(Adobe Reader~X even got all pages right.)

\bigskip

Without option \texttt{fnsymbolmult=true} of the \textsf{pageslts} package
(and the help of \textsc{Heiko Oberdiek}'s \textsf{alphalph} package),
after page~9 (\textquotedblleft \ensuremath {\ddagger \ddagger }\textquotedblright )
(and also for negative page numbers) there would just appear a
\begin{quote}
\begin{verbatim}
LaTeX Error: Counter too large
See the LaTeX manual or LaTeX Companion for explanation.
You've lost some text.  Try typing  <return>  to proceed.
If that doesn't work, type  X <return>  to quit.
\end{verbatim}
\end{quote}
Now the page numbers after 5 (\ensuremath {\mathparagraph }) are
continued  with the doubled \textquotedblleft number\textquotedblright{} of
the first, second, third,\ldots\ page (\ensuremath {**},
\ensuremath {\dagger \dagger }, \ensuremath {\ddagger \ddagger },
\ensuremath {\mathsection \mathsection },
\ensuremath {\mathparagraph \mathparagraph }),
and after the tenth page the \textquotedblleft number\textquotedblright{} is
tripled (\ensuremath {***}, \ensuremath {\dagger \dagger \dagger },\ldots).
Page zero is named 0 and negative pages just named like the positive ones
with addition of a minus sign~($-$).

\bigskip

\noindent First page (\verb|\lastpageref{pagesLTS.0}|):
\lastpageref{pagesLTS.0}\\

\noindent The page (\verb|\thepage|): \thepage \\

\noindent Current page (\verb|\theCurrentPage|),
i.\,e. counted continuously from the first page): \theCurrentPage \\

\noindent CurrentPageLocal (\verb|\theCurrentPageLocal|),
i.\,e. counted continuously from the first page of the
current page numbering scheme): \theCurrentPageLocal \\

\noindent Last roman page (pagesLTS.roman): \lastpageref{pagesLTS.roman}{\hskip4em }
(There are \lastpageref{pagesLTS.roman.local} pages with roman numbers.)\\

\noindent Last Roman page (pagesLTS.Roman): \lastpageref{pagesLTS.Roman}{\hskip3em }
(There are \lastpageref{pagesLTS.Roman.local}~pages with Roman numbers:\\
\lastpages{Roman}{1}~pages in the first Roman sector
(\pageref{Roman}{\hskip3em }-\lastpageref{pagesLTS.Roman.1}{\hskip3em }),\\
\lastpages{Roman}{2}~pages in the second Roman sector
(\pageref{Roman2}{\hskip3em }-\lastpageref{pagesLTS.Roman.2}{\hskip3em }), and\\
\lastpages{Roman}{3}~pages in the third Roman sector
(\pageref{Roman3}{\hskip3em }-\lastpageref{pagesLTS.Roman.3}{\hskip3em }).\\

\noindent Last arabic page (pagesLTS.arabic): \lastpageref{pagesLTS.arabic}{\hskip5em }
(There are only \lastpageref{pagesLTS.arabic.local} pages with arabic numbers,
because an \verb|\addtocounter{page}{|\pagesLTSexampleArabic\verb|}| was used.)\\

\noindent Last fnsymbol page (pagesLTS.fnsymbol): \lastpageref{pagesLTS.fnsymbol} \\
(\verb|\lastpageref{pagesLTS.fnsymbol}| -- never \verb|\pageref{pagesLTS.fnsymbol}|!)\\
(There are \lastpageref{pagesLTS.fnsymbol.local} pages with fnsymbol numbers.)\\

\noindent Last alph page (pagesLTS.alph): \lastpageref{pagesLTS.alph}{\hskip4em }
(There are only \lastpageref{pagesLTS.alph.local} pages with alph numbers,
because an \verb|\addtocounter{page}{|\pagesLTSexamplealph\verb|}| was used.)\\

\noindent Last Alph page (pagesLTS.Alph): \lastpageref{pagesLTS.Alph}{\hskip4em }
(There are \lastpageref{pagesLTS.Alph.local} pages with Alph numbers.)\\

\noindent Last page's \textit{name} (LastPage): \lastpageref{LastPage}\\

\noindent Very last page's \textit{name} (VeryLastPage): \lastpageref{VeryLastPage}\\
(\texttt{LastPage} and \texttt{VeryLastPage} are identical, unless
a package with output \linebreak
\verb|\AtEndDocument| after the \textsf{pageslts} package was added.)\\

\noindent Last page's \textit{number} (LastPages): \lastpageref{LastPages}{\hskip3em }
(=~total number of pages)\\

\lipsum[1-60]

\newpage

\pagenumbering{Roman}

\section{Roman - again!\label{Roman2}}

The page number would start with \textquotedblleft I\textquotedblright{} again --
but for the \textsf{pageslts} package (with option \texttt{pagecontinue=true},
or with option just \texttt{pagecontinue}, or even just
\nolinebreak{with\textbf{out}} option \texttt{pagecontinue=false}).
This package remembered the (\arabic{pagesLTS.double.Roman}-1)\footnote{%
OK, here you have to compute this value for yourself, but %
subtracting one should be manageable for \TeX{}nicians.} pages already
done in Roman output, and therefore continues with page
\textquotedblleft \thepage \textquotedblright .\\
If you want to start with \textquotedblleft I\textquotedblright{} all
over again, you will have two pages with the same name,
but nevertheless you can do this by using option \texttt{pagecontinue=false}
or a \verb|\setcounter{page}{1}|\ here (not demonstrated in this example file).\\

\noindent First page (\verb|\lastpageref{pagesLTS.0}|):
 \lastpageref{pagesLTS.0}\\

\noindent The page (\verb|\thepage|): \thepage \\

\noindent Current page (\verb|\theCurrentPage|),
i.\,e. counted continuously from the first page): \theCurrentPage \\

\noindent CurrentPageLocal (\verb|\theCurrentPageLocal|),
i.\,e. counted continuously from the first page of the
current page numbering scheme): \theCurrentPageLocal \\

\noindent Last roman page (pagesLTS.roman): \lastpageref{pagesLTS.roman}{\hskip4em }
(There are \lastpageref{pagesLTS.roman.local} pages with roman numbers.)\\

\noindent Last Roman page (pagesLTS.Roman): \lastpageref{pagesLTS.Roman}{\hskip3em }
(There are \lastpageref{pagesLTS.Roman.local}~pages with Roman numbers:\\
\lastpages{Roman}{1}~pages in the first Roman sector
(\pageref{Roman}{\hskip3em }-\lastpageref{pagesLTS.Roman.1}{\hskip3em }),\\
\lastpages{Roman}{2}~pages in the second Roman sector
(\pageref{Roman2}{\hskip3em }-\lastpageref{pagesLTS.Roman.2}{\hskip3em }), and\\
\lastpages{Roman}{3}~pages in the third Roman sector
(\pageref{Roman3}{\hskip3em }-\lastpageref{pagesLTS.Roman.3}{\hskip3em }).\\

\noindent Last arabic page (pagesLTS.arabic): \lastpageref{pagesLTS.arabic}{\hskip5em }
(There are only \lastpageref{pagesLTS.arabic.local} pages with arabic numbers,
because an \verb|\addtocounter{page}{|\pagesLTSexampleArabic\verb|}| was used.)\\

\noindent Last fnsymbol page (pagesLTS.fnsymbol): \lastpageref{pagesLTS.fnsymbol} \\
(\verb|\lastpageref{pagesLTS.fnsymbol}| -- never
\verb|\pageref{pagesLTS.fnsymbol}|!)\\
(There are \lastpageref{pagesLTS.fnsymbol.local} pages with fnsymbol numbers.)\\

\noindent Last alph page (pagesLTS.alph): \lastpageref{pagesLTS.alph}{\hskip4em }
(There are only \lastpageref{pagesLTS.alph.local} pages with alph numbers,
because an \verb|\addtocounter{page}{|\pagesLTSexamplealph\verb|}| was used.)\\

\noindent Last Alph page (pagesLTS.Alph): \lastpageref{pagesLTS.Alph}{\hskip4em }
(There are \lastpageref{pagesLTS.Alph.local} pages with Alph numbers.)\\

\noindent Last page's \textit{name} (LastPage): \lastpageref{LastPage}\\

\noindent Very last page's \textit{name} (VeryLastPage): \lastpageref{VeryLastPage}\\
(\texttt{LastPage} and \texttt{VeryLastPage} are identical, unless
a package with output \linebreak
\verb|\AtEndDocument| after the \textsf{pageslts} package was added.)\\

\noindent Last page's \textit{number} (LastPages): \lastpageref{LastPages}{\hskip3em }
(=~total number of pages)\\

\lipsum[1-6]

\newpage

\verb|\lastpageref{pagesLTS.Roman}| does \textbf{not}
refer to this page (but there: \lastpageref{pagesLTS.Roman}),
because the option \texttt{pagecontinue=true}
was chosen. When a reference to this page is wanted,\\
\verb|\lastpageref{pagesLTS.Roman.2}| can be used: \lastpageref{pagesLTS.Roman.2}.\\

\bigskip
There are \lastpages{Roman}{2}~pages (\verb|\lastpages{Roman}{2}|) in this
second Roman sector.\\
The Roman page numbering scheme is continued later in section~\ref{Roman3})!

\newpage

\pagenumbering{alph}

\section{alph\label{alph}}

\noindent First page (\verb|\lastpageref{pagesLTS.0}|):
 \lastpageref{pagesLTS.0}\\

\noindent The page (\verb|\thepage|): \thepage \\

\noindent Current page (\verb|\theCurrentPage|),
i.\,e. counted continuously from the first page): \theCurrentPage \\

\noindent CurrentPageLocal (\verb|\theCurrentPageLocal|),
i.\,e. counted continuously from the first page of the
current page numbering scheme): \theCurrentPageLocal \\

\noindent Last roman page (pagesLTS.roman): \lastpageref{pagesLTS.roman}{\hskip4em }
(There are \lastpageref{pagesLTS.roman.local} pages with roman numbers.)\\

\noindent Last Roman page (pagesLTS.Roman): \lastpageref{pagesLTS.Roman}{\hskip3em }
(There are \lastpageref{pagesLTS.Roman.local}~pages with Roman numbers:\\
\lastpages{Roman}{1}~pages in the first Roman sector
(\pageref{Roman}{\hskip3em }-\lastpageref{pagesLTS.Roman.1}{\hskip3em }),\\
\lastpages{Roman}{2}~pages in the second Roman sector
(\pageref{Roman2}{\hskip3em }-\lastpageref{pagesLTS.Roman.2}{\hskip3em }), and\\
\lastpages{Roman}{3}~pages in the third Roman sector
(\pageref{Roman3}{\hskip3em }-\lastpageref{pagesLTS.Roman.3}{\hskip3em }).\\

\noindent Last arabic page (pagesLTS.arabic): \lastpageref{pagesLTS.arabic}{\hskip5em }
(There are only \lastpageref{pagesLTS.arabic.local} pages with arabic numbers,
because an \verb|\addtocounter{page}{|\pagesLTSexampleArabic\verb|}| was used.)\\

\noindent Last fnsymbol page (pagesLTS.fnsymbol): \lastpageref{pagesLTS.fnsymbol} \\
(\verb|\lastpageref{pagesLTS.fnsymbol}| -- never \verb|\pageref{pagesLTS.fnsymbol}|!)\\
(There are \lastpageref{pagesLTS.fnsymbol.local} pages with fnsymbol numbers.)\\

\noindent Last alph page (pagesLTS.alph): \lastpageref{pagesLTS.alph}{\hskip4em }
(There are only \lastpageref{pagesLTS.alph.local} pages with alph numbers,
because an \verb|\addtocounter{page}{|\pagesLTSexamplealph\verb|}| was used.)\\

\noindent Last Alph page (pagesLTS.Alph): \lastpageref{pagesLTS.Alph}{\hskip4em }
(There are \lastpageref{pagesLTS.Alph.local} pages with Alph numbers.)\\

\noindent Last page's \textit{name} (LastPage): \lastpageref{LastPage}\\

\noindent Very last page's \textit{name} (VeryLastPage): \lastpageref{VeryLastPage}\\
(\texttt{LastPage} and \texttt{VeryLastPage} are identical, unless
a package with output \linebreak
\verb|\AtEndDocument| after the \textsf{pageslts} package was added.)\\

\noindent Last page's \textit{number} (LastPages): \lastpageref{LastPages}{\hskip3em }
(=~total number of pages)\\

\lipsum[1-4]

\newpage

Without option \texttt{alphMult=ab} of the \textsf{pageslts} (and the help of
\textsc{Heiko Oberdiek's} \textsf{alphalph} package), after page
\textquotedblleft z\textquotedblright{} there would just appear a
\begin{quote}
\begin{verbatim}
LaTeX Error: Counter too large
See the LaTeX manual or LaTeX Companion for explanation.
You've lost some text.  Try typing  <return>  to proceed.
If that doesn't work, type  X <return>  to quit.
\end{verbatim}
\end{quote}
Now the page numbers are continued aa, ab, ac,\ldots\ (aa, bb, cc,\ldots\ is
also possible, see the \textsf{pageslts} documentation).\\
To demonstrate this, we add a\\
\verb|\addtocounter{page}{|\pagesLTSexamplealph\verb|}|\\
in the source code here.

\addtocounter{page}{\pagesLTSexamplealph}

\bigskip

\lipsum[1-18]

\newpage

\pagenumbering{Roman}

\section{Roman - third time!\label{Roman3}}

The page number would start with
\textquotedblleft I\textquotedblright{} again -- but
for the \textsf{pageslts} package (with option \texttt{pagecontinue=true},
or with option just \texttt{pagecontinue}, or even just
\nolinebreak{with\textbf{out}} option \texttt{pagecontinue=false}).
This package remembered the
(\arabic{pagesLTS.double.Roman}-1)\footnote{OK, here %
you have to compute this value for yourself, but %
subtracting one should be managable for \TeX nicians.} pages already
done in Roman output, and therefore continues with page
\textquotedblleft \thepage \textquotedblright .\\
If you want to start with \textquotedblleft I\textquotedblright{} all
over again, you will have (at least) two pages with the same name,
but nevertheless you can do this by using option
\texttt{pagecontinue=false} instead of \texttt{pagecontinue=true}
(not demonstrated here).\\

\noindent First page (\verb|\lastpageref{pagesLTS.0}|):
 \lastpageref{pagesLTS.0}\\

\noindent The page (\verb|\thepage|): \thepage \\

\noindent Current page (\verb|\theCurrentPage|),
i.\,e. counted continuously from the first page): \theCurrentPage \\

\noindent CurrentPageLocal (\verb|\theCurrentPageLocal|),
i.\,e. counted continuously from the first page of the
current page numbering scheme): \theCurrentPageLocal \\

\noindent Last roman page (pagesLTS.roman): \lastpageref{pagesLTS.roman}{\hskip4em }
(There are \lastpageref{pagesLTS.roman.local} pages with roman numbers.)\\

\noindent Last Roman page (pagesLTS.Roman): \lastpageref{pagesLTS.Roman}{\hskip3em }
(There are \lastpageref{pagesLTS.Roman.local}~pages with Roman numbers:\\
\lastpages{Roman}{1}~pages in the first Roman sector
(\pageref{Roman}{\hskip3em }-\lastpageref{pagesLTS.Roman.1}{\hskip3em }),\\
\lastpages{Roman}{2}~pages in the second Roman sector
(\pageref{Roman2}{\hskip3em }-\lastpageref{pagesLTS.Roman.2}{\hskip3em }), and\\
\lastpages{Roman}{3}~pages in the third Roman sector
(\pageref{Roman3}{\hskip3em }-\lastpageref{pagesLTS.Roman.3}{\hskip3em }).\\

\noindent Last arabic page (pagesLTS.arabic): \lastpageref{pagesLTS.arabic}{\hskip5em }
(There are only \lastpageref{pagesLTS.arabic.local} pages with arabic numbers,
because an \verb|\addtocounter{page}{|\pagesLTSexampleArabic\verb|}| was used.)\\

\noindent Last fnsymbol page (pagesLTS.fnsymbol): \lastpageref{pagesLTS.fnsymbol} \\
(\verb|\lastpageref{pagesLTS.fnsymbol}| -- never \verb|\pageref{pagesLTS.fnsymbol}|!)\\
(There are \lastpageref{pagesLTS.fnsymbol.local} pages with fnsymbol numbers.)\\

\noindent Last alph page (pagesLTS.alph): \lastpageref{pagesLTS.alph}{\hskip4em }
(There are only \lastpageref{pagesLTS.alph.local} pages with alph numbers,
because an \verb|\addtocounter{page}{|\pagesLTSexamplealph\verb|}| was used.)\\

\noindent Last Alph page (pagesLTS.Alph): \lastpageref{pagesLTS.Alph}{\hskip4em }
(There are \lastpageref{pagesLTS.Alph.local} pages with Alph numbers.)\\

\noindent Last page's \textit{name} (LastPage): \lastpageref{LastPage}\\

\noindent Very last page's \textit{name} (VeryLastPage): \lastpageref{VeryLastPage}\\
(\texttt{LastPage} and \texttt{VeryLastPage} are identical, unless
a package with output \linebreak
\verb|\AtEndDocument| after the \textsf{pageslts} package was added.)\\

\noindent Last page's \textit{number} (LastPages): \lastpageref{LastPages}{\hskip3em }
(=~total number of pages)\\

\lipsum[1-3]

\newpage

\lastpageref{pagesLTS.Roman}
(\verb|\lastpageref{pagesLTS.Roman}|)
\textbf{does} refers to this page, because the option
\texttt{pagecontinue=true} was chosen. Also\\
\verb|\lastpageref{pagesLTS.Roman.3}| can be used: \lastpageref{pagesLTS.Roman.3}.\\

\bigskip

There are \lastpages{Roman}{3}~pages (\verb|\lastpages{Roman}{3}|) in this
third Roman sector.\\

\newpage

\pagenumbering{Alph}

\section{Alph}

\noindent First page (\verb|\lastpageref{pagesLTS.0}|):
 \lastpageref{pagesLTS.0}\\

\noindent The page (\verb|\thepage|): \thepage \\

\noindent Current page (\verb|\theCurrentPage|),
i.\,e. counted continuously from the first page): \theCurrentPage \\

\noindent CurrentPageLocal (\verb|\theCurrentPageLocal|),
i.\,e. counted continuously from the first page of the
current page numbering scheme): \theCurrentPageLocal \\

\noindent Last roman page (pagesLTS.roman): \lastpageref{pagesLTS.roman}{\hskip4em }
(There are \lastpageref{pagesLTS.roman.local} pages with roman numbers.)\\

\noindent Last Roman page (pagesLTS.Roman): \lastpageref{pagesLTS.Roman}{\hskip3em }
(There are \lastpageref{pagesLTS.Roman.local}~pages with Roman numbers:\\
\lastpages{Roman}{1}~pages in the first Roman sector
(\pageref{Roman}{\hskip3em }-\lastpageref{pagesLTS.Roman.1}{\hskip3em }),\\
\lastpages{Roman}{2}~pages in the second Roman sector
(\pageref{Roman2}{\hskip3em }-\lastpageref{pagesLTS.Roman.2}{\hskip3em }), and\\
\lastpages{Roman}{3}~pages in the third Roman sector
(\pageref{Roman3}{\hskip3em }-\lastpageref{pagesLTS.Roman.3}{\hskip3em }).\\

\noindent Last arabic page (pagesLTS.arabic): \lastpageref{pagesLTS.arabic}{\hskip5em }
(There are only \lastpageref{pagesLTS.arabic.local} pages with arabic numbers,
because an \verb|\addtocounter{page}{|\pagesLTSexampleArabic\verb|}| was used.)\\

\noindent Last fnsymbol page (pagesLTS.fnsymbol): \lastpageref{pagesLTS.fnsymbol} \\
(\verb|\lastpageref{pagesLTS.fnsymbol}| -- never \verb|\pageref{pagesLTS.fnsymbol}|!)\\
(There are \lastpageref{pagesLTS.fnsymbol.local} pages with fnsymbol numbers.)\\

\noindent Last alph page (pagesLTS.alph): \lastpageref{pagesLTS.alph}{\hskip4em }
(There are only \lastpageref{pagesLTS.alph.local} pages with alph numbers,
because an \verb|\addtocounter{page}{|\pagesLTSexamplealph\verb|}| was used.)\\

\noindent Last Alph page (pagesLTS.Alph): \lastpageref{pagesLTS.Alph}{\hskip4em }
(There are \lastpageref{pagesLTS.Alph.local} pages with Alph numbers.)\\

\noindent Last page's \textit{name} (LastPage): \lastpageref{LastPage}\\

\noindent Very last page's \textit{name} (VeryLastPage): \lastpageref{VeryLastPage}\\
(\texttt{LastPage} and \texttt{VeryLastPage} are identical, unless
a package with output \linebreak
\verb|\AtEndDocument| after the \textsf{pageslts} package was added.)\\

\noindent Last page's \textit{number} (LastPages): \lastpageref{LastPages}{\hskip3em }
(=~total number of pages)\\

\lipsum[1-3]

Without option \texttt{alphMulti=AB} of the \textsf{pageslts} (and the help of
\textsc{Heiko Oberdiek's} \textsf{alphalph} package), after page
\textquotedblleft Z\textquotedblright{} there would just appear a
\begin{quote}
\begin{verbatim}
LaTeX Error: Counter too large
See the LaTeX manual or LaTeX Companion for explanation.
You've lost some text.  Try typing  <return>  to proceed.
If that doesn't work, type  X <return>  to quit.
\end{verbatim}
\end{quote}
Now the page numbers are continued AA, AB, AC,\ldots\ (AA, BB, CC,\ldots\ is
also possible, see the \textsf{pageslts} documentation).\\
This is not demonstrated here, but see section~\ref{alph}.

\newpage

\section{The End}

\noindent First page (\verb|\lastpageref{pagesLTS.0}|):
 \lastpageref{pagesLTS.0}\\

\noindent The page (\verb|\thepage|): \thepage \\

\noindent Current page (\verb|\theCurrentPage|),
i.\,e. counted continuously from the first page): \theCurrentPage \\

\noindent CurrentPageLocal (\verb|\theCurrentPageLocal|),
i.\,e. counted continuously from the first page of the
current page numbering scheme): \theCurrentPageLocal \\

\noindent Last roman page (pagesLTS.roman): \lastpageref{pagesLTS.roman}{\hskip4em }
(There are \lastpageref{pagesLTS.roman.local} pages with roman numbers.)\\

\noindent Last Roman page (pagesLTS.Roman): \lastpageref{pagesLTS.Roman}{\hskip3em }
(There are \lastpageref{pagesLTS.Roman.local}~pages with Roman numbers:\\
\lastpages{Roman}{1}~pages in the first Roman sector
(\pageref{Roman}{\hskip3em }-\lastpageref{pagesLTS.Roman.1}{\hskip3em }),\\
\lastpages{Roman}{2}~pages in the second Roman sector
(\pageref{Roman2}{\hskip3em }-\lastpageref{pagesLTS.Roman.2}{\hskip3em }), and\\
\lastpages{Roman}{3}~pages in the third Roman sector
(\pageref{Roman3}{\hskip3em }-\lastpageref{pagesLTS.Roman.3}{\hskip3em }).\\

\noindent Last arabic page (pagesLTS.arabic): \lastpageref{pagesLTS.arabic}{\hskip5em }
(There are only \lastpageref{pagesLTS.arabic.local} pages with arabic numbers,
because an \verb|\addtocounter{page}{|\pagesLTSexampleArabic\verb|}| was used.)\\

\noindent Last fnsymbol page (pagesLTS.fnsymbol): \lastpageref{pagesLTS.fnsymbol} \\
(\verb|\lastpageref{pagesLTS.fnsymbol}| -- never \verb|\pageref{pagesLTS.fnsymbol}|!)\\
(There are \lastpageref{pagesLTS.fnsymbol.local} pages with fnsymbol numbers.)\\

\noindent Last alph page (pagesLTS.alph): \lastpageref{pagesLTS.alph}{\hskip4em }
(There are only \lastpageref{pagesLTS.alph.local} pages with alph numbers,
because an \verb|\addtocounter{page}{|\pagesLTSexamplealph\verb|}| was used.)\\

\noindent Last Alph page (pagesLTS.Alph): \lastpageref{pagesLTS.Alph}{\hskip4em }
(There are \lastpageref{pagesLTS.Alph.local} pages with Alph numbers.)\\

\noindent Last page's \textit{name} (LastPage): \lastpageref{LastPage}\\

\noindent Very last page's \textit{name} (VeryLastPage): \lastpageref{VeryLastPage}\\
(\texttt{LastPage} and \texttt{VeryLastPage} are identical, unless
a package with output \linebreak
\verb|\AtEndDocument| after the \textsf{pageslts} package was added.)\\

\noindent Last page's \textit{number} (LastPages): \lastpageref{LastPages}{\hskip3em }
(=~total number of pages)\\

\medskip

\noindent Page \thepage\ (\theCurrentPage; local: \theCurrentPageLocal) of %
\lastpageref{pagesLTS.roman} (\lastpageref{pagesLTS.roman.local}) + %
\lastpageref{pagesLTS.Roman} (\lastpageref{pagesLTS.Roman.local}) + %
\lastpageref{pagesLTS.arabic} (\lastpageref{pagesLTS.arabic.local}) + %
\lastpageref{pagesLTS.fnsymbol} (\lastpageref{pagesLTS.fnsymbol.local}) + %
\lastpageref{pagesLTS.alph} (\lastpageref{pagesLTS.alph.local}) + %
\lastpageref{pagesLTS.Alph} (\lastpageref{pagesLTS.Alph.local}) = %
\lastpageref{LastPages} pages.

\end{document}
%</example>
%    \end{macrocode}
%
% \pagebreak
%
% \StopEventually{}
%
% \section{The implementation}
%
% (This and the source code of the example file are the reasons
% for printing the documentation in landscape format instead of portrait.)\\
%
% We start off by checking that we are loading into \LaTeXe{} and
% announcing the name and version of this package.
%
%    \begin{macrocode}
%<*package>
%    \end{macrocode}
%
%    \begin{macrocode}
\NeedsTeXFormat{LaTeX2e}[2015/01/01]
\ProvidesPackage{pageslts}[2015/12/21 v1.2f
            Refers to special pages' numbers/names (HMM)]

%    \end{macrocode}
%
% A short description of the \xpackage{pageslts} package:
%
%    \begin{macrocode}
%% Allows for things like\\
%% |Page \thepage\ (\theCurrentPage; local: \theCurrentPageLocal) of %
%%  \lastpageref{pagesLTS.roman}(\lastpageref{pagesLTS.roman.local}) + %
%%  \lastpageref{pagesLTS.Roman}(\lastpageref{pagesLTS.Roman.local}) + %
%%  \lastpageref{pagesLTS.arabic}(\lastpageref{pagesLTS.arabic.local}) + %
%%  \lastpageref{pagesLTS.fnsymbol}(\lastpageref{pagesLTS.fnsymbol.local}) + %
%%  \lastpageref{pagesLTS.alph}(\lastpageref{pagesLTS.alph.local}) + %
%%  \lastpageref{pagesLTS.Alph}(\lastpageref{pagesLTS.Alph.local}) = %
%%  \lastpageref{LastPages} pages.|\\
%% to get\\
%% 'Page d (57; local: 4) of ii(2) + XX(20) + *(1) + 30(30) + e(5) + C(3) = 61 pages.'.

%    \end{macrocode}
%
% The package is now named `pageslts' but had been named `pagesLTS' before.
% \LaTeX\ does not load a package two times, but it is case sensitive,
% i.\,e. \emph{would} load `pageslts' and `pagesLTS', resulting in
% possible problems. Therefore we check for a loaded `pagesLTS' package
% (assuming |\pagesLTS@loaded| is not |p@gesLTSlo@ded| and not defined to
% |p@gesLTSlo@ded| by another package). This check might work only for
% the second and following compilation runs.
%
%    \begin{macrocode}
\@ifpackageloaded{pagesLTS}{%
  \def\pagesLTS@doubleload{p@gesLTSlo@ded}
  \ifx\pagesLTS@loaded\pagesLTS@doubleload%
    \PackageError{pageslts}{Requested pagesLTS instead of pageslts}{%
      You have requested package `pagesLTS'. This package is now named `pageslts'.\MessageBreak%
      Requesting `pagesLTS' as well as `pageslts' leads to loading the same package\MessageBreak%
      twice (and results in errors). Loading of `pageslts' will therefore be\MessageBreak%
      abandoned now. To fix this problem, please look in the \jobname.log file for\MessageBreak%
      LaTeX Warning: You have requested package `pagesLTS',\MessageBreak%
      \space \space \space \space \space \space \space \space \space \space \space \space \space%
      \space \space but the package provides `pageslts'.\MessageBreak%
      and change the package call in the calling style/document.\MessageBreak%
      If a style file made the mistake, please inform its maintainer and/or the\MessageBreak%
      maintainer of the pageslts package about it (after making sure you have\MessageBreak%
      the recent version of that style file).\MessageBreak%
      Thank you and sorry for the inconvenience!\MessageBreak%
     }
    \endinput
  \fi
  }{% \relax
   }

\gdef\pagesLTS@loaded{p@gesLTSlo@ded}

%    \end{macrocode}
%
% For |\ltx@ifpackageloaded| the \xpackage{ltxcmds} package is needed,
% also by \textsc{Heiko Oberdiek} (see subsection~\ref{ss:Downloads}):
%
%    \begin{macrocode}
\RequirePackage{ltxcmds}[2011/11/09]%        v1.22
%    \end{macrocode}
%
% For its |\AfterLastShipout| (as well as its |\AtEndAfterFileList|) command
% we need the \xpackage{atveryend} package by \textsc{Heiko Oberdiek}
% (see subsection~\ref{ss:Downloads}):
%
%    \begin{macrocode}
\RequirePackage{atveryend}[2011/06/30]%      v1.8
\AtBeginDocument{%
  %% https://tex.stackexchange.com/a/45654/6865
  %% and https://tex.stackexchange.com/q/261445/6865
  \def\pagesLTS@etb{etoolbox.sty}%
  \let\pagesLTS@ave\@empty%
  \@for\@pageslts@currname:=\@filelist\do{%
     \ifx\@pageslts@currname\pagesLTS@etb\relax%
       \def\pagesLTS@ave{atveryend.sty}%
     \fi%
     \ifx\@pageslts@currname\pagesLTS@ave%
       \@ifpackagelater{atveryend}{2015/08/01}{%\relax
          \PackageInfo{pageslts}{etoolbox package loaded before atveryend package\MessageBreak%
            With atveryend version 1.8 of 2011/06/30 this caused a problem,\MessageBreak%
            which is probably solved in the new version used here.\MessageBreak%
            To play it safe, you could load the atveryend package\MessageBreak%
            before the etoolbox package.\MessageBreak%
            }%
        }{% else
          \PackageError{pageslts}{etoolbox package loaded before atveryend package}{%
            You can use both packages, but when etoolbox is loaded before atveryend,\MessageBreak%
            atveryend no longer detects \string\enddocument\space\MessageBreak% 
            and hook "AtVeryVeryEnd" is not executed.\MessageBreak%
            Just load the atveryend package before the etoolbox package!\MessageBreak%
            }%
         }%
     \fi%
   }%
  }

%    \end{macrocode}
%
% For its |\EveryShipout| command we need the \xpackage{everyshi}
% package by \textsc{Martin Schr\"{o}der} (see subsection~\ref{ss:Downloads}):
%
%    \begin{macrocode}
\RequirePackage{everyshi}[2001/05/15]%       v3.00
%    \end{macrocode}
%
% For its |\LetLtxMacro| command we need the \xpackage{letltxmacro}
% package by \textsc{Heiko Oberdiek} (see subsection~\ref{ss:Downloads}):
%
%    \begin{macrocode}
\RequirePackage{letltxmacro}[2010/09/02]%    v1.4
%    \end{macrocode}
%
% For the handling of the options we need the \xpackage{kvoptions}
% package also by \textsc{Heiko Oberdiek} (see subsection~\ref{ss:Downloads}):
%
%    \begin{macrocode}
\RequirePackage{kvoptions}[2011/06/30]%      v3.11
%    \end{macrocode}
%
% The \xpackage{undolabl} package of \textsc{H.-Martin M\"{u}nch} (i.\,e. myself),
% with code from \textsc{Ulrich Diez}, (see subsection~\ref{ss:Downloads})
% is needed to overwrite labels, when the same page numbering scheme
% is used twice (or even more often).
%
%    \begin{macrocode}
\RequirePackage{undolabl}[2015/03/29]%       v1.0l
%    \end{macrocode}
%
% We use the \xpackage{rerunfilecheck} package by \textsc{Heiko Oberdiek}
% to make sure that the user gets and sees the rerun warnings (if any).
%
%    \begin{macrocode}
\RequirePackage{rerunfilecheck}[2011/04/15]% v1.7
%    \end{macrocode}
%
% We must not forget to give the source of |Prelim@EveryShipout|:
%
%    \begin{macrocode}
%% pageslts package uses Prelim@EveryShipout code from the
%% prelim2e package [2009/05/29 v1.3] by Martin Schr\"{o}der, thanks!

%    \end{macrocode}
%
% About the \xpackage{prelim2e} package by \textsc{Martin Schr\"{o}der}
% see subsection~\ref{ss:Downloads}.\\
%
% A last information for the user(s):
%
%    \begin{macrocode}
%% pageslts may work with earlier versions of LaTeX2e and those packages,
%% but this was not tested. Please consider updating your LaTeX
%% and packages to the most recent version (if they are not already
%% the most recent version).
%    \end{macrocode}
% See subsection~\ref{ss:Downloads} about how to get them.\\
%
% \pagebreak
%
% The very old version~2.0 (and earlier) of the \xpackage{endfloat}
% package actually redefined the |\enddocument| command,
% and so interfered drastically with the \LaTeXe\ commands which
% make use of |\AtEndDocument|. Newer versions of \xpackage{endfloat}
% exists (at the time of writing this documentation: v2.5d as of 2011/12/25)
% in modern documentation form, which are available from \url{https://www.CTAN.org}
% (see subsection~\ref{ss:Downloads}).
% A~note is placed here, and later it is checked whether a (very) old
% \xpackage{endfloat} package is in use. If it is, a warning or even
% an error message is given, depending on \xpackage{endfloat} version.
% This assumes, that the old versions of \xpackage{endfloat} at least gave
% a version date, of course.
%
%    \begin{macrocode}
%% The recent version of the endfloat package is v2.5d as of 2011/12/25.
%% The pageslts package is not fully compatible with version 2.0
%% (and earlier) of the endfloat package, because those versions
%% redefined the \enddocument command.

%    \end{macrocode}
%
% The options are introduced:
%
%    \begin{macrocode}
\SetupKeyvalOptions{family = pagesLTS,prefix = pagesLTS@}
\DeclareBoolOption[true]{pagecontinue}% \pagesLTS@pagecontinue
\DeclareStringOption[ab]{alphMult}
\DeclareStringOption[AB]{AlphMulti}
\DeclareBoolOption[true]{romanMult}
\DeclareBoolOption[true]{RomanMulti}
\DeclareBoolOption[true]{fnsymbolmult}

\ProcessKeyvalOptions*

%    \end{macrocode}
%
% For comparisons, zero, one, two and three are defined (|\z@|, |\@ne| and so on
% do not work for this).
%
%    \begin{macrocode}
\def\pagesLTS@zero{0}
\def\pagesLTS@one{1}
\def\pagesLTS@two{2}
\def\pagesLTS@three{3}

%    \end{macrocode}
%
% The traditional behaviour is a reset of the page number to one,
% each time the page numbering scheme changes. The option
% |pagecontinue| changes this to a continuation with the
% number/name following the last page number/name of the same
% page numbering scheme. The user is informed accordingly.
%
%    \begin{macrocode}
\ifpagesLTS@pagecontinue%
  \PackageInfo{pageslts}{Option pagecontinue enabled\MessageBreak%
    (maybe by default):\MessageBreak%
    The pageslts package will continue the page numbering,\MessageBreak%
    when the same page numbering scheme is used twice.\MessageBreak%
    If you do not want this, call pageslts with option\MessageBreak%
    pagecontinue=false (or use \string\setcounter{page}=1).\MessageBreak%
  }
\else
%    \end{macrocode}
%
% \pagebreak
%
%    \begin{macrocode}
  \PackageWarningNoLine{pageslts}{Option pagecontinue is false:\MessageBreak%
    The pageslts package was used, but the option\MessageBreak%
    pagecontinue was set to false.\MessageBreak%
    If you want the page numbers to be continued,\MessageBreak%
    when the same page numbering scheme is used twice,\MessageBreak%
    please call pageslts with option pagecontinue=true,\MessageBreak%
    otherwise the page number is reset to one each time\MessageBreak%
    the page numbering scheme is changed.\MessageBreak%
    For details please see the documentation%
  }
\fi

%    \end{macrocode}
%
% The page number printed in \texttt{alph} or in \texttt{Alph}
% page numbering scheme had to be $>0$ and \nolinebreak{$<27$.}\ %
% Now the \xpackage{alphalph} package allows to extend the
% numbering scheme (not only for pages). Because some users
% prefer aa, ab, ac, ad,\ldots\ and some aa, bb, cc, dd,\ldots,
% both schemes can be chosen via the options.
% The \texttt{fnsymbol} page numbering scheme was restricted to values
% $>0$ and \nolinebreak{$<10$.}\ The \xpackage{alphalph} package allows
% to extend this page numbering scheme, too. Option \texttt{fnsymbolmult}
% can be chosen with the \xpackage{pageslts} package.
% If no extension is wished (or another extension is wished
% and implemented manually), \xpackage{pageslts} can be called with
% options set to 0 (zero) and false: |alphMult=0|, |AlphMulti=0|,
% |fnsymbolmult=false|.
%
%    \begin{macrocode}
\def\pagesLTS@ab{ab}
\def\pagesLTS@bb{bb}
\def\pagesLTS@ABi{AB}
\def\pagesLTS@BBi{BB}

\ifx\pagesLTS@alphMult\pagesLTS@ab%
\else
  \ifx\pagesLTS@alphMult\pagesLTS@bb%
  \else
    \ifx\pagesLTS@alphMult\pagesLTS@zero%
      \PackageWarningNoLine{pageslts}{%
        Option alphMult=0 found:\MessageBreak%
        The pageslts package was used, but the option\MessageBreak%
        alphMult was set to 0 (zero).\MessageBreak%
        If you want the page numbers to be extended\MessageBreak%
        after z, you have to organize this yourself now.\MessageBreak%
        For automatic continuation, please use the\MessageBreak%
        alphalph package and call pageslts\MessageBreak%
        with option alphMult=ab (for aa, ab, ac, ad,...) or\MessageBreak%
        with option alphMult=bb (for aa, bb, cc, dd,...).\MessageBreak%
        For details please see the documentation!%
       }
    \else
      \PackageError{pageslts}{Unknown option value}{%
        The pageslts package was used with option\MessageBreak%
        alphMult= \pagesLTS@alphMult . Only values\MessageBreak%
        ab, bb, and 0 (zero) are valid.\MessageBreak%
        The default ab is set.\MessageBreak%
        For details please see the documentation!%
       }
      \setkeys{pagesLTS}{alphMult=ab}
    \fi
  \fi
\fi

\ifx\pagesLTS@AlphMulti\pagesLTS@ABi%
\else
  \ifx\pagesLTS@AlphMulti\pagesLTS@BBi%
  \else
    \ifx\pagesLTS@AlphMulti\pagesLTS@zero%
      \PackageWarningNoLine{pageslts}{%
        Option AlphMulti=0 found:\MessageBreak%
        The pageslts package was used, but the option\MessageBreak%
        AlphMulti was set to 0 (zero).\MessageBreak%
        If you want the page numbers to be extended\MessageBreak%
        after Z, you have to organize this yourself now.\MessageBreak%
        For automatic continuation, please use the\MessageBreak%
        alphalph package and call pageslts\MessageBreak%
        with option AlphMulti=AB (for AA, AB, AC, AD,...) or\MessageBreak%
        with option AlphMulti=BB (for AA, BB, CC, DD,...).\MessageBreak%
        For details please see the documentation!%
       }
    \else
      \PackageError{pageslts}{Unknown option value}{%
        The pageslts package was used with option\MessageBreak%
        AlphMulti= \pagesLTS@AlphMulti . Only values\MessageBreak%
        AB, BB, and 0 (zero) are valid.\MessageBreak%
        The default AB is set.\MessageBreak%
        For details please see the documentation!%
       }
      \setkeys{pagesLTS}{AlphMulti=AB}
    \fi
  \fi
\fi

%    \end{macrocode}
%
% If |alph| or |Alph| or |fnsymbol| page numbers shall be continued,
% the \xpackage{alphalph} package is required.
%
%    \begin{macrocode}
\ifx\pagesLTS@alphMult\pagesLTS@zero% \relax
\else \RequirePackage{alphalph}[2011/05/13]% v2.4
\fi

\ifx\pagesLTS@AlphMulti\pagesLTS@zero% \relax
\else \RequirePackage{alphalph}[2011/05/13]% v2.4
\fi

\ifpagesLTS@fnsymbolmult%
  \RequirePackage{alphalph}[2011/05/13]% v2.4
\fi

%    \end{macrocode}
%
% For the roman page numbering scheme, it is just the
% choice of an extension by \xpackage{pageslts} or not.
%
%    \begin{macrocode}
\ifpagesLTS@romanMult%
  \PackageInfo{pageslts}{Option romanMult enabled\MessageBreak%
    (maybe by default):\MessageBreak%
    The pageslts package will extend the page numbering\MessageBreak%
    of the roman scheme below i with\MessageBreak%
     0, -i, -ii, -iii, -iv,...\MessageBreak%
    If you do not want this, call pageslts with option\MessageBreak%
    romanMult=false.\MessageBreak%
  }
\else
  \PackageWarningNoLine{pageslts}{Option romanMult is set to false:\MessageBreak%
  The pageslts package was used, but the option\MessageBreak%
  romanMult was set to false.\MessageBreak%
  If you want the page numbering of the roman scheme\MessageBreak%
  to be extended below i,\MessageBreak%
  please call pageslts with option romanMult=true,\MessageBreak%
  otherwise zero and negative page numbers of the\MessageBreak%
  roman scheme will need to be defined otherwise.\MessageBreak%
  For details please see the documentation!\MessageBreak%
  }
\fi

%    \end{macrocode}
%
% Same for the Roman page numbering scheme.
%
%    \begin{macrocode}
\ifpagesLTS@RomanMulti%
  \PackageInfo{pageslts}{Option RomanMulti enabled\MessageBreak%
    (maybe by default):\MessageBreak%
    The pageslts package will extend the page numbering\MessageBreak%
    of the Roman scheme below I with\MessageBreak%
    0, -I, -II, -III, -IV,...\MessageBreak%
    If you do not want this, call pageslts with option\MessageBreak%
    RomanMulti=false.\MessageBreak%
   }
\else
  \PackageWarningNoLine{pageslts}{Option RomanMulti is set to false:\MessageBreak%
    The pageslts package was used, but the option\MessageBreak%
    RomanMulti was set to false.\MessageBreak%
    If you want the page numbering of the Roman scheme\MessageBreak%
    to be extended below i,\MessageBreak%
    please call pageslts with option RomanMulti=true,\MessageBreak%
    otherwise zero and negative page numbers of the\MessageBreak%
    Roman scheme will need to be defined otherwise.\MessageBreak%
    For details please see the documentation!\MessageBreak%
   }
\fi

%    \end{macrocode}
%
% For the footnotesymbol page numbering scheme, it is also
% just the choice of a extension by \xpackage{pageslts} or not.
%
%    \begin{macrocode}
\ifpagesLTS@fnsymbolmult%
  \PackageInfo{pageslts}{Option fnsymbolmult enabled\MessageBreak%
    (maybe by default):\MessageBreak%
    The pageslts package will extend the page numbering\MessageBreak%
    of the footnotesymbol scheme using the alphalph\MessageBreak%
    package.\MessageBreak%
    If you do not want this, call pageslts with option\MessageBreak%
    fnsymbolmult=false.\MessageBreak%
   }
\else
  \PackageWarningNoLine{pageslts}{%
    Option fnsymbolmult is set to false:\MessageBreak%
    The pageslts package was used, but the option\MessageBreak%
    fnsymbolmult was set to false.\MessageBreak%
    If you want the page numbering of the footnotesymbol\MessageBreak%
    scheme to be extended using the alphalph package,\MessageBreak%
    please call pageslts with option fnsymbolmult=true,\MessageBreak%
    otherwise page numbers of the footnotesymbol scheme\MessageBreak%
    greater than nine will need to be defined otherwise.\MessageBreak%
    For details please see the documentation!\MessageBreak%
   }
\fi

%    \end{macrocode}
%
% Now defining some variables, place-holders, and abbreviations:
%
%    \begin{macrocode}
\def\pagesLTS@pnc{0}
\def\pagesLTS@called{0}
\def\pagesLTS@fns{fnsymbol}
\def\pagesLTS@alph{alph}
\def\pagesLTS@Alph{Alph}
\def\pagesLTS@rerun{0}
\def\pagesLTS@eso{0}
\def\pagesLTS@esov{0}
\def\lastpageref{\lastpagereftxt}
\def\pagesLTS@undolable{none}
\def\pncmissing{0}
\def\pagesLTS@SK{0}
\def\pagesLTS@messageNPN{%
 The pageslts package was used, but\MessageBreak%
 \string\pagenumbering \MessageBreak%
 was not called at the beginning of the document\MessageBreak%
 (maybe earlier or later).\MessageBreak%
 Please use \string\pagenumbering \MessageBreak%
 (with an argument like arabic, roman, Roman,\MessageBreak%
 fnsymbol, alph, or Alph) at the beginning\MessageBreak%
 of your document! Otherwise your document\MessageBreak%
 will probably compile, but the pageslts\MessageBreak%
 package might not be able to get all labels\MessageBreak%
 for the references to the respective pages\MessageBreak%
 right.\MessageBreak%
 }

%    \end{macrocode}
%
% It is checked whether writing to files is allowed.
% The \xpackage{pageslts} package cannot be used without that!
% Some packages (e.\,g. \xpackage{tikz} and \xpackage{selectp}) sometimes prevent
% the output to the \xfile{aux} file. In that case a warning or an error message
% is issued. This is no problem as long as there is/was another compilation run
% where the labels can/could be processed via the \xfile{aux} file.\\
% If it is allowed to write to the \xfile{aux} file, we define
% |\pagesLTS@loaded| as |p@gesLTSnotlo@ded| for the next compilation run.
%
%    \begin{macrocode}
\AtBeginDocument{%
  \if@filesw%
%    \end{macrocode}
%
% \noindent then |\relax|.\newline
% When writing to files is not allowed, nothing can be done.
% But when the labels were already processed via the \xfile{aux} file,
% nothing needs to be done (if enough compilation runs have been done before).
%
%    \begin{macrocode}
  \else%
    \@ifpackageloaded{tikz}{%
      \PackageWarning{pageslts}%
        {It was not allowed to write to an\MessageBreak%
         .aux file. This package does not work without access\MessageBreak%
         to an .aux file.\MessageBreak%
         It is OK if the .aux file was already updated\MessageBreak%
         by a previouse compiler run\MessageBreak%
         and would not have changed anyway.\MessageBreak%
        }%
    }{\PackageError{pageslts}{No writing to auxiliary file allowed}%
        {It was not allowed to write to an .aux file.\MessageBreak%
         This package does not work without access to an .aux file.\MessageBreak%
         Press Ctrl+Z to exit.\MessageBreak%
         But it is OK if the .aux file was already updated\MessageBreak%
         by previouse compiler runs\MessageBreak%
         and would not have changed anyway.\MessageBreak%
         (In that case just press Enter or Return\MessageBreak%
         to continue the compilation.)\MessageBreak%
        }%
     }%
  \fi%
  }

%    \end{macrocode}
%
% \begin{macro}{\pagenumbering}
% \indent \qquad To keep the original meaning of |\pagenumbering|:
%
%    \begin{macrocode}
\LetLtxMacro{\OrigPagenumbering}{\pagenumbering}

%    \end{macrocode}
% \end{macro}
%
% Defining some new counters (and doing related things):
%
%    \begin{macrocode}
\newcounter{CurrentPage}
\setcounter{CurrentPage}{1}
\def\theCurrentPageLocal{\arabic{pagesLTS.current.local.\pagesLTS@pnc}}
%    \end{macrocode}
%
% The counter |pagesLTS.pagenr| is for saving the total page
% number of the last page in the \xfile{.aux} file.
%
%    \begin{macrocode}
\newcounter{pagesLTS.pagenr}
%    \end{macrocode}
%
% While generally
% |\pagesLTS@ifcounter{pagesLTS.current.local.\pagesLTS@pnc}|
% is used, for the beginning of the document
% |pagesLTS.current.local.0| is predefined.
% (A~|\pagesLTS@ifcounter{pagesLTS.current.local.\pagesLTS@pnc}|
% could be used for this, too, but we know that |pagesLTS.current.local.0|
% was not defined, so we can just do the definition here.)
% And the first local page gets the number one.
%
%    \begin{macrocode}
\newcounter{pagesLTS.current.local.0}
\setcounter{pagesLTS.current.local.0}{1}
%    \end{macrocode}
%
% And the same again for |pagesLTS.pnc.0|.
%
%    \begin{macrocode}
\newcounter{pagesLTS.pnc.0}

%    \end{macrocode}
%
% \phantomsection
% \begin{macro}{\xroman}\label{xroman}
% When |\roman{...}| is used with a value $<1$, \LaTeX\ just ignores this
% (see subsection~\ref{ss:overflow}). Here we provide a command
% |\xroman{...}| \nolinebreak{(e\textbf{x}panded} roman),
% which gives the usual |\roman| numbers (i, ii, iii, iv,\ldots)
% for positive values,
% $-\left\vert \ldots \right\vert $ (i.\,e. -i, -ii, -iii, -iv,\ldots)
% for negative values, and 0 for all other values (which should be zero).
%
%    \begin{macrocode}
\newcommand{\xroman}[1]{%
  \ifnum\value{#1}>0%
    \roman{#1}%
  \else%
    \ifnum\value{#1}<0%
%    \end{macrocode}
%
% |\arabic{#1}| gives the arabic number of argument |#1|,
% which is negative here (for example $-7$),
% \textquotedblleft |-|\textquotedblright{} puts another minus
% sign in front of it (for example $--7$), |\number| removes
% all unnecessary preceding zeros, plus and minus signs
% (for example $7$), |\romannumeral| turns it into a |roman|
% number (for example vii), and
% \textquotedblleft |-|\textquotedblright{} puts the minus
% sign back in front of it (for example -vii).
%
%    \begin{macrocode}
      -\romannumeral\number-\arabic{#1}%
    \else%
      0%
    \fi%
  \fi%
  }

%    \end{macrocode}
%  \end{macro}
%
% \pagebreak
%
% \begin{macro}{\XRoman}
% |\XRoman| does the same for uppercase |\Roman| numbers.
% |-\uppercase{\romannumeral\number-\arabic{#1}}| cannot be used,
% because the result in the example is |-\uppercase{vii}| and
% not |-VII|.\footnote{This does not matter for the print out,
% but for the display of the logical page numbers
% as well as the \xfile{.aux} file.} Therefore we have a look
% at \LaTeX's own |\@Roman\FOOcounter|,\\
% |\def\@Roman#1{\expandafter\@slowromancap\romannumeral #1@}|,\\
% and use |\@slowromancap|, which is a fully expandable macro,
% to do the trick for this:
%
% \noindent \textquotedblleft
% \begin{quote}
%\begin{verbatim}
%\def\@slowromancap#1{\ifx @#1% then terminate
%\else
%\if i#1I\else\if v#1V\else\if x#1X\else\if l#1L\else\if
%c#1C\else\if d#1D\else \if m#1M\else#1\fi\fi\fi\fi\fi\fi\fi
%\expandafter\@slowromancap
%\fi
%}
%\end{verbatim}
% \end{quote}
% \textquotedblright{} (1998/05/16 Version v1.1g LaTeX Kernel File~m
% ltcounts.dtx 105 Counters and Lengths).
%
%    \begin{macrocode}
\newcommand{\XRoman}[1]{%
  \ifnum\value{#1}>0%
    \Roman{#1}%
  \else%
    \ifnum\value{#1}<0%
      -\expandafter\@slowromancap\romannumeral\number-\arabic{#1}@%
    \else%
      0%
    \fi%
  \fi%
  }

%    \end{macrocode}
%  \end{macro}
%
% \begin{macro}{\XXRoman}
% In older versions |\XXRoman| was used. For compatibility, it is forwarded to |\XRoman|
% and an error message is given.
%
%    \begin{macrocode}
\newcommand{\XXRoman}[1]{\XRoman{#1}%
  \PackageError{pageslts}{Old command \string\XXRoman\space found}{Replaced by \string\XRoman.}%
  }

%    \end{macrocode}
%  \end{macro}
%
% \begin{macro}{\pagesLTS@ifcounter}
% We provide a way to create counters like
% \begin{itemize}
% \item[-] \DescribeMacro{pagesLTS.pnc. page numbering scheme }
%   |pagesLTS.pnc.<page numbering scheme>|, e.\,g. |pagesLTS.pnc.Roman|,
% \item[-] \DescribeMacro{pagesLTS.double. page numbering scheme }
%   |pagesLTS.double.<page numbering scheme>|, e.\,g. |pagesLTS.double.Roman|,
% \item[-] \DescribeMacro{PageCurrentLocal. page numbering scheme }
%   |PageCurrentLocal.<page numbering scheme>|, e.\,g. |PageCurrentLocal.Roman|,
% \end{itemize}
% for all page numbering schemes, even those not supported
% by the current original |\pagenumbering|
% (1994/05/19 v1.1a LaTeX Kernel File w ltpageno.dtx 52 Page Numbering),
% which is\\
% \begin{quote}
%\begin{verbatim}
%\countdef\c@page=0 \c@page=1
%\def\cl@page{}
%\def\pagenumbering#1{%
%  \global\c@page \@ne \gdef\thepage{\csname @#1\endcsname
%   \c@page}}
%\end{verbatim}
% \end{quote}
%
%    \begin{macrocode}
\newcommand{\pagesLTS@ifcounter}[1]{%
  \@ifundefined{c@#1}{\newcounter{#1}}{\relax}%
  }

%    \end{macrocode}
%  \end{macro}
%
% \begin{macro}{\lastpages}
% We provide a command to give the number of pages in a sector
% of a split page numbering scheme (see page~\pageref{pagesLTS.pnscheme.nr},\\
% |pagesLTS.<page numbering scheme>.<number>.local.cnt|):
%
%    \begin{macrocode}
\newcommand{\lastpages}[2]{%
  \pagesLTS@ifcounter{pagesLTS.#1.#2.local.cnt}%
  \arabic{pagesLTS.#1.#2.local.cnt}%
  }

%    \end{macrocode}
%  \end{macro}
%
% \begin{macro}{\pagesLTS@writelabel}
% At last defining the writing of a label:
%
%    \begin{macrocode}
\newcommand{\pagesLTS@writelabel}[1]{%
  \addtocounter{page}{+1}%
%    \end{macrocode}
%
% \noindent |\addtocounter{page}{+1}| because |\pagesLTS@putlabel| includes an\\
% |\addtocounter{page}{-1}|, which is not necessary here.\\
% Into the \xfile{.aux} file something like\\
% |\newlabel{pagesLTS.Roman}{{}{VIII}{}{page.VIII}{}}|\\
% is written, thus |\lastpageref{pagesLTS.Roman}| prints |VIII| and
% links to |page.VIII|.
%
%    \begin{macrocode}
  \pagesLTS@putlabel{pagesLTS.#1}{\thepage}{1}
  \addtocounter{page}{-1}%
  \ifx\pagesLTS@pnc\pagesLTS@zero% \relax
%    \end{macrocode}
%
% i.\,e. if the current page numbering scheme is
% \textquotedblleft 0\textquotedblright, i.\,e. before the first
% |\pagenumbering{...}| command, do nothing,
%
%    \begin{macrocode}
  \else%
    \addtocounter{page}{+1}%
    \pagesLTS@putlabel{pagesLTS.#1.local}{\theCurrentPageLocal}{1}
%    \end{macrocode}
%
% otherwise write into the \xfile{.aux} file something like\\
% |\newlabel{pagesLTS.arabic.local}{{}{5}{}{page.8}{}}|,\\
% thus |\lastpageref{pagesLTS.arabic.local}| prints |5| and
% links to |page.8|. Here (and in the example file) it is not
% \textquotedblleft print |8| and link to |page.8|\textquotedblright ,
% because |\addtocounter{page}{3}| has been used, thus the page with
% \textquotedblleft number\textquotedblright{} (name) |8| is the
% \textbf{fifth} $(=8-3)$ page.
%
%    \begin{macrocode}
    \addtocounter{page}{-1}%
  \fi%
  }

%    \end{macrocode}
% \end{macro}
%
% \begin{macro}{\erroralphalph}
% \indent |\erroralphalph| extends the
% \textquotedblleft numbers\textquotedblright{} of counters
% to zero and negative values for representations usually
% not supporting this: |\alphalph|, |\AlphAlph|, and
% |\fnsymbolmult| of the \xpackage{alphalph} package.\\
% |\alph|, |\Alph|, and |\fnsymbol| would not support
% \textquotedblleft numbers\textquotedblright{} below one.\\
% |\arabic| already supports negative numbers and zero.\\
% |\roman| and |\Roman| support neither negative numbers
% nor zero, but are expanded in this package
% (|\xroman| and |\XRoman|), see page~\pageref{xroman}.\\
%
%    \begin{macrocode}

%% The following code is from Heiko Oberdiek [2010/04/18],              %%
%% expanding his alphalph package as of 2010/04/18, v2.3. (Thanks!)     %%
\newcommand*{\erroralphalph}[2]{%
  \ifnum\value{#2}>0%
    #1{\value{#2}}%
  \else%
    \ifnum\value{#2}<0%
      -#1{\expandafter\@gobble\the\value{#2}}%
    \else%
      0%
    \fi%
  \fi%
  }
%% End of code from Heiko Oberdiek                                      %%
%% Check and Error/Warning messages have been moved to \EveryShipout,   %%
%% because messages inside e.g. the \pageref command can cause trouble. %%

%    \end{macrocode}
% \end{macro}
%
% \begin{macro}{\expandPagenumbering}
% \indent Here the |\erroralphalph| command is called with the
% appropriate arguments for each page numbering scheme.
%
%    \begin{macrocode}

\newcommand{\expandPagenumbering}[1]{%
  \let\Origthepage\thepage%
  \def\pagesLTS@tmpC{arabic}%
  \ifx\pagesLTS@pnc\pagesLTS@tmpC% \relax
%    \end{macrocode}
%
% |\arabic| already supports negative numbers and zero
% (|-MAX...MAX|, where |MAX|\,$ = 2\,147\,483\,647$).
%
%    \begin{macrocode}
  \else%
    \def\pagesLTS@tmpC{roman}%
    \ifx\pagesLTS@pnc\pagesLTS@tmpC%
      \ifpagesLTS@romanMult%
%    \end{macrocode}
%
% |\erroralphalph{\roman}{page}| cannot be used, because
% |-\roman{\expandafter\@gobble\the\value{page}}| does not work.
% If option |romanMult| is not false,
% |\xroman| (see page~\pageref{xroman}) expands the usable
% roman page numbers to values below 1 (i, I, respectively),
% see subsubsection~\ref{sss:romanMult}.
%
%    \begin{macrocode}
        \renewcommand*{\thepage}{\xroman{page}}%
      \fi%
    \else%
      \def\pagesLTS@tmpC{Roman}%
      \ifx\pagesLTS@pnc\pagesLTS@tmpC%
        \ifpagesLTS@RomanMulti%
%    \end{macrocode}
%
% The same for |\Roman| page numbering, expanded by |\XRoman|
% (see page~\pageref{xroman}).
%
%    \begin{macrocode}
          \renewcommand*{\thepage}{\XRoman{page}}%
        \fi%
      \else%
        \ifx\pagesLTS@pnc\pagesLTS@alph%
%    \end{macrocode}
%
% |\alph| and |\Alph| page numberings are expanded to negative
% and zero values, and to values greater than
% \textquotedblleft z\textquotedblright{} or
% \textquotedblleft Z\textquotedblright{} with the
% \xpackage{alphalph} package.~--
% If |\pagesLTS@alphMult| was zero, nothing is done.
%
%    \begin{macrocode}
          \ifx\pagesLTS@alphMult\pagesLTS@ab%
            \renewcommand*{\thepage}{\erroralphalph{\alphalph}{page}}%
          \else \ifx\pagesLTS@alphMult\pagesLTS@bb%
                  \renewcommand*{\thepage}{\erroralphalph{\alphMult}{page}}%
                \fi%
          \fi%
        \else%
          \ifx\pagesLTS@pnc\pagesLTS@Alph%
            \ifx\pagesLTS@AlphMulti\pagesLTS@ABi%
              \renewcommand*{\thepage}{\erroralphalph{\AlphAlph}{page}}%
            \else \ifx\pagesLTS@AlphMulti\pagesLTS@BBi%
                    \renewcommand*{\thepage}{\erroralphalph{\AlphMult}{page}}%
                  \fi%
            \fi%
          \else%
            \ifx\pagesLTS@pnc\pagesLTS@fns%
%    \end{macrocode}
%
% Same for |\fnsymbol| page numbers.
%
%    \begin{macrocode}
              \ifpagesLTS@fnsymbolmult%
                \renewcommand*{\thepage}{\erroralphalph{\fnsymbolmult}{page}}%
              \fi%
            \else%
%    \end{macrocode}
%
% If the used page numbering scheme has not been recognized by the
% \xpackage{pageslts} package so far, we can do nothing, and problems
% might result.\\
%
%    \begin{macrocode}
              \PackageError{pageslts}{unknown page numbering scheme}{%
                The pageslts package encountered the unknown\MessageBreak%
                page numbering scheme\MessageBreak%
                `#1'. \MessageBreak%
                If this is no typing mistake, it might work\MessageBreak%
                - or it might not work.\MessageBreak%
                \@ehc%
                }
            \fi%
          \fi%
        \fi%
      \fi%
    \fi%
  \fi%
  \let\pagesLTS@tmpC\undefined%
  }

%    \end{macrocode}
% \end{macro}
%
% \begin{macro}{\pagenumbering}
% \indent Now for the \textbf{new} version of the |\pagenumbering| command:
%
%    \begin{macrocode}
\renewcommand{\pagenumbering}[1]{%
%    \end{macrocode}
%
% If the current page numbering scheme, |\pagesLTS@pnc|, or the
% requested page numbering scheme, |#1|, is |\pagesLTS@fns|,
% i.\,e. \texttt{fnsymbol}, the counter\texttt{pagesLTS.fnsymbol.local}
% is needed. If it does not exists yet, it is created here.
%
%    \begin{macrocode}
  \edef\pagesLTS@tmpA{#1}%
  \ifx\pagesLTS@pnc\pagesLTS@fns%
    \pagesLTS@ifcounter{pagesLTS.fnsymbol.local}%
  \fi%
  \ifx\pagesLTS@tmpA\pagesLTS@fns%
    \pagesLTS@ifcounter{pagesLTS.fnsymbol.local}%
  \fi%
%    \end{macrocode}
%
% If the current page numbering scheme, |\pagesLTS@pnc|, and the
% requested page numbering scheme, |#1|, is the same one, nothing
% further is done, otherwise the real action begins.
%
%    \begin{macrocode}
  \ifx\pagesLTS@pnc\pagesLTS@tmpA% \relax
  \else%
%    \end{macrocode}
%
% The next code is executed, when we are at a page after the first one.
% This distinction is done for two reasons:\\
% On the one hand, |\pagenumbering| could be called \textit{before}
% |\begin{document}| (where the current page should not be
% greater than one), and on the other hand we go one page back to
% aim all references to that page. Obviously at the first page
% there is no going backward.
%
%    \begin{macrocode}
    \ifnum \value{CurrentPage}>1%
      \addtocounter{page}{-1}%
      \addtocounter{pagesLTS.current.local.\pagesLTS@pnc}{-1}%
%    \end{macrocode}
%
% For the case that the page numbering scheme is or will be split,
% like e.\,g. the Roman one in the |pageslts-example.tex|, a counter
% like
% |pagesLTS.Roman.1.local.count|
% (or |pagesLTS.Roman.2.local.count|,
% |pagesLTS.Roman.3.local.count|,\ldots ) is introduced and set to the
% number of the local page.
%
%    \begin{macrocode}
      \newcounter{pagesLTS.\pagesLTS@pnc.\arabic{pagesLTS.pnc.\pagesLTS@pnc}.local.count}%
      \setcounter{pagesLTS.\pagesLTS@pnc.\arabic{pagesLTS.pnc.\pagesLTS@pnc}.local.count}{%
        \value{pagesLTS.current.local.\pagesLTS@pnc}}%
%    \end{macrocode}
%
% If the page numbering scheme is \xpackage{fnsymbol},
% and if it \textit{was} used before,
% from said counter the number of pages of the preceding uses
% of the same page numbering scheme, |pagesLTS.\pagesLTS@pnc.done|,
% is subtracted (same as for the other schemes, see below). Instead of introducing a new
% counter (which can be problematic, when the number of available counters is limited),
% we borrow the \texttt{pagesLTS.pnc.0} counter, i.\,e. we save its value to |\pagesLTS@tmpa|,
% (ab)use the counter, and then set it back to its former value as saved in |\pagesLTS@tmpa|.
%
%    \begin{macrocode}
      \ifx\pagesLTS@pnc\pagesLTS@fns%
        \ifnum \value{pagesLTS.pnc.\pagesLTS@pnc}>1%
          \mathchardef\pagesLTS@tmpa=\arabic{pagesLTS.pnc.0}%
          \setcounter{pagesLTS.pnc.0}{\value{pagesLTS.pnc.\pagesLTS@pnc}}%
          \addtocounter{pagesLTS.pnc.0}{-1}%
          \addtocounter{pagesLTS.\pagesLTS@pnc.\arabic{pagesLTS.pnc.\pagesLTS@pnc}.local.count}{%
            -\value{pagesLTS.\pagesLTS@pnc.\arabic{pagesLTS.pnc.0}.local.count}}%
          \setcounter{pagesLTS.pnc.0}{\pagesLTS@tmpa}%
        \fi%
%    \end{macrocode}
%
% If the page numbering scheme is \textit{not} \xpackage{fnsymbol},
% a~numbered label is written:
%
%    \begin{macrocode}
      \else%
        \pagesLTS@writelabel{\pagesLTS@pnc.\arabic{pagesLTS.pnc.\pagesLTS@pnc}}%
%    \end{macrocode}
%
% If the page numbering scheme was \textit{not} used before,
%
%    \begin{macrocode}
        \ifnum \value{pagesLTS.pnc.\pagesLTS@pnc}<2%
%    \end{macrocode}
%
% an unnumbered label is also written:
%
%    \begin{macrocode}
          \pagesLTS@writelabel{\pagesLTS@pnc}%
%    \end{macrocode}
%
% If the page numbering scheme \textit{was} used before,
% from said counter the number of pages of the preceding uses
% of the same page numbering scheme, |pagesLTS.\pagesLTS@pnc.done|,
% is subtracted. Instead of introducing a new
% counter (which can be problematic, when the number of available counters is limited),
% we again borrow the \texttt{pagesLTS.pnc.0} counter (see above).
%
%    \begin{macrocode}
        \else%
          \mathchardef\pagesLTS@tmpa=\arabic{pagesLTS.pnc.0}%
          \setcounter{pagesLTS.pnc.0}{\value{pagesLTS.pnc.\pagesLTS@pnc}}%
          \addtocounter{pagesLTS.pnc.0}{-1}%
          \pagesLTS@ifcounter{pagesLTS.\pagesLTS@pnc.done}%
          \addtocounter{pagesLTS.\pagesLTS@pnc.done}{%
            \value{pagesLTS.\pagesLTS@pnc.\arabic{pagesLTS.pnc.0}.local.count}}%
          \addtocounter{pagesLTS.\pagesLTS@pnc.\arabic{pagesLTS.pnc.\pagesLTS@pnc}.local.count}{%
            -\value{pagesLTS.\pagesLTS@pnc.done}}%
          \setcounter{pagesLTS.pnc.0}{\pagesLTS@tmpa}%
        \fi%
%    \end{macrocode}
%
% The values are written to the \xfile{.aux} file
% (if writing is allowed: |\if@filesw|),
% because they must be available at the beginning of the document:
%
%    \begin{macrocode}
        \if@filesw%
          \immediate\write\@auxout{\string
            \pagesLTS@ifcounter{pagesLTS.\pagesLTS@pnc.\arabic{pagesLTS.pnc.\pagesLTS@pnc}.local.cnt}}%
        \fi%
        \edef\pagesLTS@tmpB{\arabic{pagesLTS.\pagesLTS@pnc.\arabic{pagesLTS.pnc.\pagesLTS@pnc}.local.count}}%
        \if@filesw%
          \immediate\write\@auxout{\string
            \setcounter{pagesLTS.\pagesLTS@pnc.\arabic{pagesLTS.pnc.\pagesLTS@pnc}.local.cnt}{\pagesLTS@tmpB}}%
        \fi%
      \fi%
%    \end{macrocode}
%
% For further code for the case of \xpackage{fnsymbol} please see below
% (|\lastpagereftext|, page~\pageref{lastpagereftextlabel}).\\
% The last page number is saved, in case the same page
% numbering scheme is continued later.
%
%    \begin{macrocode}
      \pagesLTS@ifcounter{pagesLTS.double.\pagesLTS@pnc}%
      \setcounter{pagesLTS.double.\pagesLTS@pnc}{\value{page}}%
%    \end{macrocode}
%
% We went back one page, so we must go forward again:
%
%    \begin{macrocode}
      \addtocounter{pagesLTS.current.local.\pagesLTS@pnc}{+1}%
      \addtocounter{page}{+1}%
%    \end{macrocode}
%
% The page numbering scheme |\pagesLTS@pnc| is now set to
% the new one, given by the user as argument with the
% |\pagenumbering{...}| command:
%
%    \begin{macrocode}
      \xdef\pagesLTS@pnc{#1}%
%    \end{macrocode}
%
% The new page numbering scheme is now started for real:
%
%    \begin{macrocode}
      \OrigPagenumbering{#1}%
%    \end{macrocode}
%
% If a page numbering scheme not known by the original
% |\pagenumbering{...}| command is used, an error will
% arise here - but maybe without error message.\\
%
% If page numbering schem |\alph|, |\Alph|, or |\fnsymbol| is used,
% \xpackage{pageslts} extends the page numbers according to
% the given options, using the \xpackage{alphalph} package.
% |\arabic| does not need any expansion.
% |\roman| and |\Roman| at least receive a definition for
% zero.
%
%    \begin{macrocode}
      \expandPagenumbering{#1}%
%    \end{macrocode}
%
% Counters like |pagesLTS.pnc.Roman| are introduced:
%
%    \begin{macrocode}
      \pagesLTS@ifcounter{pagesLTS.pnc.\pagesLTS@pnc}%
%    \end{macrocode}
%
% The saved number of times this page numbering scheme was used
% is increased by one:
%
%    \begin{macrocode}
      \addtocounter{pagesLTS.pnc.\pagesLTS@pnc}{1}%
%    \end{macrocode}
%
% \indent Now defining the counter |pagesLTS.double.\pagesLTS@pnc|,
% if it did not exist already, adding~1, because this is
% the first page of it (or another one, if the scheme is
% continued):
%
%    \begin{macrocode}
      \pagesLTS@ifcounter{pagesLTS.double.\pagesLTS@pnc}%
      \addtocounter{pagesLTS.double.\pagesLTS@pnc}{1}%
%    \end{macrocode}
%
% The page number is continued, if the option |pagecontinue=false|
% is \textbf{not} set, otherwise it is reset to one. Note that neither
% the local nor the current counter are reset, as they contain
% the real \textit{values} and not the \textit{names} of the pages.
%
%    \begin{macrocode}
      \ifpagesLTS@pagecontinue%
        \setcounter{page}{\value{pagesLTS.double.\pagesLTS@pnc}}%
      \else%
        \setcounter{page}{1}%
      \fi%
%    \end{macrocode}
%
% If it does not exist already, the counter
% |pagesLTS.current.local.\pagesLTS@pnc|
% (e.\,g. |pagesLTS.current.local.Roman|) is created.
%
%    \begin{macrocode}
      \pagesLTS@ifcounter{pagesLTS.current.local.\pagesLTS@pnc}%
%    \end{macrocode}
%
% If |pagesLTS.double.\pagesLTS@pnc| of the current page numbering
% scheme is equal to one, this is the first page of this page
% numbering scheme. Then |pagesLTS.current.local.\pagesLTS@pnc|
% (which was zero) is set to one.
%
%    \begin{macrocode}
      \ifnum \value{pagesLTS.double.\pagesLTS@pnc}=1%
        \setcounter{pagesLTS.current.local.\pagesLTS@pnc}{1}%
      \fi%
%    \end{macrocode}
%
% Otherwise, i.\,e. if |\value{CurrentPage}| is not |>1|,
% i.\,e. before the first page has shiped out:
%
%    \begin{macrocode}
    \else%
      %% before the first page has shipped out
%    \end{macrocode}
%
% The current page numbering scheme is defined by the
% argument of |\pagenumbering{...}|, which the user gave:
%
%    \begin{macrocode}
      \xdef\pagesLTS@pnc{#1}%
%    \end{macrocode}
%
% \noindent and the page numbering scheme set by the original
% page numbering command
% (1994/05/19 v1.1a LaTeX Kernel File w ltpageno.dtx 52 Page Numbering),
% which resets the page number to one,
% but at the first page continuation does not make sense).
% Well, nearly the original page numbering command:
% |\OrigPagenumbering{\pagesLTS@pnc}| does not work,
% so we \textquotedblleft expand\textquotedblright{} the
% |\OrigPagenumbering| command:
%
%    \begin{macrocode}
      \global\c@page \@ne\relax%
      \global\def\thepage{\csname \expandafter @\pagesLTS@pnc \endcsname \c@page}%
%    \end{macrocode}
%
% If a page numbering scheme is used, which is not known by
% \LaTeX , an error might arise here -- but maybe without error message.\\
%
% If page numbering scheme |\alph|, |\Alph|, or |\fnsymbol| is used,
% \xpackage{pageslts} extends the page numbers according to
% the given options, using the \xpackage{alphalph} package.
% |\arabic| does not need any expansion.
% |\roman| and |\Roman| at least receive a definition for
% zero.
%
%    \begin{macrocode}
      \expandPagenumbering{#1}
%    \end{macrocode}
%
% We are at the first page, so the page counters are set to one:
%
%    \begin{macrocode}
      \pagesLTS@ifcounter{pagesLTS.pnc.\pagesLTS@pnc}%
      \setcounter{pagesLTS.pnc.\pagesLTS@pnc}{1}%
      \pagesLTS@ifcounter{pagesLTS.double.\pagesLTS@pnc}%
      \setcounter{pagesLTS.double.\pagesLTS@pnc}{1}%
      \pagesLTS@ifcounter{pagesLTS.current.local.\pagesLTS@pnc}%
      \setcounter{pagesLTS.current.local.\pagesLTS@pnc}{1}%
    \fi%
%    \end{macrocode}
%
% Whether |\pagenumbering{...}| is called in the preamble,
% |\AtBeginDocument|, right after |\begin{document}|,
% or somewhere in the document, we want to remember
% \textit{whether} it was called at all:
%
%    \begin{macrocode}
    \gdef\pagesLTS@called{1}%
  \fi%
%    \end{macrocode}
%
% We do not need the temporary definitions any more.
%
%    \begin{macrocode}
  \let\pagesLTS@tmpA\undefined%
  \let\pagesLTS@tmpB\undefined%
  }

%    \end{macrocode}
% \end{macro}
%
% \begin{macro}{\lastpageref*}
% If \xpackage{hyperref} is used, but (some) references to some last page shall
% not be hyperlinked, a command |\lastpageref*| (analogous to |\pageref*|) is
% needed. Therefore we define (analogous to |\HyPsd@pageref| from
% the \xpackage{hyperref} package by \textsc{Heiko Oberdiek})
%
%    \begin{macrocode}
%% analogous to \HyPsd@pageref from the hyperref package by Heiko Oberdiek:
\def\lastpagereftxt#1{\pagesLTS@@pageref#1*\END}

%    \end{macrocode}
%
% Macro |\pagesLTSpageref| checks, whether a star is present (analogous to
% |\HyPsd@@pageref| again from the \xpackage{hyperref} package
% of \textsc{Heiko Oberdiek}):
%
%    \begin{macrocode}
\def\pagesLTS@@pageref#1*#2\END{%
  \ifx\\#2\\% no star
    \pagesLTS@@@pageref{#1}%
  \else% star
    \expandafter\pagesLTS@@@pagerefstar%
  \fi%
}

\def\pagesLTS@@@pageref#1{\lastpagereftext{#1}}
\def\pagesLTS@@@pagerefstar#1{\lastpagereftextstar{#1}}

%    \end{macrocode}
% \end{macro}
%
% \phantomsection
% \begin{macro}{\lastpagereftext} \label{lastpagereftextlabel}
% \indent When |\lastpageref| is used somewhere inside the \texttt{txt} (text),
% i.\,e. not at the last page, it is defined as |\lastpagereftxt| (see above).
% When the page numbering scheme is \xpackage{fnsymbol}, and the \xpackage{hyperref}
% package has been loaded, a hyperref instead of a label is used for the reference
% to |pagesLTS.fnsymbol.local.\arabic{pagesLTS.fnsymbol.local}|.
% (And if the \texttt{pagesLTS.fnsymbol.local} counter did not exist yet,
% it is created here.)\\
%
%    \begin{macrocode}
\newcommand{\lastpagereftext}[1]{%
  \def\pagesLTS@tmpA{#1}%
  \def\pagesLTS@tmpB{pagesLTS.fnsymbol.local}%
  \ifx\pagesLTS@tmpA\pagesLTS@tmpB%
    \pagesLTS@ifcounter{pagesLTS.fnsymbol.local}%
    \ltx@ifpackageloaded{hyperref}{%
      \href{\#pagesLTS.fnsymbol.local.\arabic{pagesLTS.fnsymbol.local}}%
        {\arabic{pagesLTS.fnsymbol.local}}%
%    \end{macrocode}
%
% When the page numbering scheme is \xpackage{fnsymbol}, but the \xpackage{hyperref} package
% has \textit{not} been loaded, just the arabic number of the
% \texttt{pagesLTS.fnsymbol.local} counter is given (because there will be
% no hyperlink anyway).
%
%    \begin{macrocode}
     }{\arabic{pagesLTS.fnsymbol.local}}%
%    \end{macrocode}
%
% Otherwise just the common |\pageref| is applied:
%
%    \begin{macrocode}
  \else%
    \pageref{#1}%
  \fi%
%    \end{macrocode}
%
% We do not need the temporary definitions any more.
%
%    \begin{macrocode}
  \let\pagesLTS@tmpA\undefined%
  \let\pagesLTS@tmpB\undefined%
  }

%    \end{macrocode}
% \end{macro}
%
% \begin{macro}{\lastpagereftextstar}
% And the same for the starred version, where no hyperlink is generated:
%
%    \begin{macrocode}
\newcommand{\lastpagereftextstar}[1]{%
  \def\pagesLTS@tmpA{#1}%
  \def\pagesLTS@tmpB{pagesLTS.fnsymbol.local}%
  \ifx\pagesLTS@tmpA\pagesLTS@tmpB%
     \pagesLTS@ifcounter{pagesLTS.fnsymbol.local}%
     \arabic{pagesLTS.fnsymbol.local}%
  \else%
%    \end{macrocode}
%
% There is no |\pageref*| without \xpackage{hyperref}.
%
%    \begin{macrocode}
    \ltx@ifpackageloaded{hyperref}{\pageref*{#1}}{\pageref{#1}}%
  \fi%
  \let\pagesLTS@tmpA\undefined%
  \let\pagesLTS@tmpB\undefined%
  }

%    \end{macrocode}
% \end{macro}
%
% \begin{macro}{\lastpagerefend}
% \indent When the \xpackage{hyperref} package is used and the page numbering scheme
% of the last page is \xpackage{fnsymbol}, |\lastpageref| is defined as
% |\lastpagerefend|.
% Hyperrefs instead of labels are used for the reference to
% \xpackage{fnsymbol} pages (including the last one).\\
%
% Again it must be discriminated between unstarred form and starred form:
%
%    \begin{macrocode}
\def\lastpagerefend#1{\pagesLTS@@pagerefend#1*\END}

\def\pagesLTS@@pagerefend#1*#2\END{%
  \ifx\\#2\\% no star
    \pagesLTS@@@pagerefend{#1}%
  \else% star
    \expandafter\pagesLTS@@@pagerefendstar%
  \fi%
}

\def\pagesLTS@@@pagerefend#1{\l@stpagerefend{#1}}
\def\pagesLTS@@@pagerefendstar#1{\l@stpagerefendstar{#1}}

%    \end{macrocode}
% \end{macro}
%
% \begin{macro}{\l@stpagerefend}
% The unstarred form (i.\,e. with hyperlinks, if \xpackage{hyperref} is loaded,
% otherwise without hyperlinks):
%
%    \begin{macrocode}
\newcommand{\l@stpagerefend}[1]{%
  \def\pagesLTS@tmpA{#1}%
  \def\pagesLTS@tmpB{pagesLTS.fnsymbol.local}%
  \ifx\pagesLTS@tmpA\pagesLTS@tmpB%
    \pagesLTS@ifcounter{pagesLTS.fnsymbol.local}%
    \ltx@ifpackageloaded{hyperref}{%
      \href{\#pagesLTS.fnsymbol.local.\arabic{pagesLTS.fnsymbol.local}}%
        {\arabic{pagesLTS.fnsymbol.local}}%
     }{\arabic{pagesLTS.fnsymbol.local}}%
  \else%
    \def\pagesLTS@tmpB{pagesLTS.fnsymbol}%
    \ifx\pagesLTS@tmpA\pagesLTS@tmpB%
      \ltx@ifpackageloaded{hyperref}{%
        \href{\#pagesLTS.fnsymbol.local.\arabic{pagesLTS.fnsymbol.local}}%
          {\pagesLTS.lastpage}%
       }{\pageref{pagesLTS.fnsymbol}}%
    \else%
      \def\pagesLTS@tmpB{LastPage}%
      \ifx\pagesLTS@tmpA\pagesLTS@tmpB%
        \ltx@ifpackageloaded{hyperref}{%
          \href{\#pagesLTS.fnsymbol.local.\pagesLTS@eso}%
            {\pagesLTS.lastpage}%
         }{\pageref{LastPage}}%
      \else%
        \def\pagesLTS@tmpB{VeryLastPage}%
        \ifx\pagesLTS@tmpA\pagesLTS@tmpB%
          \ltx@ifpackageloaded{hyperref}{%
            \href{\#pagesLTS.fnsymbol.local.\arabic{pagesLTS.fnsymbol.local}}%
              {\pagesLTS.lastpage}%
           }{\pageref{VeryLastPage}}%
        \else%
          \def\pagesLTS@tmpB{LastPages}%
          \ifx\pagesLTS@tmpA\pagesLTS@tmpB%
            \ltx@ifpackageloaded{hyperref}{%
              \href{\#pagesLTS.fnsymbol.local.\arabic{pagesLTS.fnsymbol.local}}%
                {\arabic{pagesLTS.pagenr}}%
             }{\pageref{LastPages}}%
          \else%
            \pageref{#1}%
          \fi%
        \fi%
      \fi%
    \fi%
  \fi%
%    \end{macrocode}
%
% We do not need the temporary definitions any more.
%
%    \begin{macrocode}
  \let\pagesLTS@tmpA\undefined%
  \let\pagesLTS@tmpB\undefined%
  }

%    \end{macrocode}
% \end{macro}
%
% \pagebreak
%
% \begin{macro}{\l@stpagerefendstar}%
% And the starred form, without hyperlinks, even if \xpackage{hyperref} is loaded,
% otherwise (i.\,e. without loaded \xpackage{hyperref}) this command is not called:
%
%    \begin{macrocode}
\newcommand{\l@stpagerefendstar}[1]{%
  \def\pagesLTS@tmpA{#1}%
  \def\pagesLTS@tmpB{pagesLTS.fnsymbol.local}%
  \ifx\pagesLTS@tmpA\pagesLTS@tmpB%
    \pagesLTS@ifcounter{pagesLTS.fnsymbol.local}%
    \arabic{pagesLTS.fnsymbol.local}%
  \else%
    \def\pagesLTS@tmpB{pagesLTS.fnsymbol}%
    \ifx\pagesLTS@tmpA\pagesLTS@tmpB%
      \ltx@ifpackageloaded{hyperref}{\pageref*{pagesLTS.fnsymbol}}{%
        \pageref{pagesLTS.fnsymbol}}%
    \else%
      \def\pagesLTS@tmpB{LastPage}%
      \ifx\pagesLTS@tmpA\pagesLTS@tmpB%
        \ltx@ifpackageloaded{hyperref}{\pageref*{LastPage}}{\pageref{LastPage}}%
      \else%
        \def\pagesLTS@tmpB{VeryLastPage}%
        \ifx\pagesLTS@tmpA\pagesLTS@tmpB%
          \ltx@ifpackageloaded{hyperref}{\pageref*{VeryLastPage}}{%
            \pageref{VeryLastPage}}%
        \else%
          \def\pagesLTS@tmpB{LastPages}%
          \ifx\pagesLTS@tmpA\pagesLTS@tmpB%
            \ltx@ifpackageloaded{hyperref}{\pageref*{LastPages}}{\pageref{LastPages}}%
          \else%
            \ltx@ifpackageloaded{hyperref}{\pageref*{#1}}{\pageref{#1}}%
            \fi%
          \fi%
        \fi%
      \fi%
    \fi%
  \fi%
  \let\pagesLTS@tmpA\undefined%
  \let\pagesLTS@tmpB\undefined%
  }

%    \end{macrocode}
% \end{macro}
%
% \pagebreak
%
% \begin{macro}{\overrideLTSlabel}
% |\overridelabel| from the \xpackage{undolabl} package just |\undonewlabel|s
% a~label and places a new |\label{#1}|, but we need to place a
% |\pagesLTS@putlabel{#1}{#2}|, therefore we need another command instead of
% (but somewhat similar to) |\overridelabel|:
%
%    \begin{macrocode}
% somewhat analogous to \overridelabel from the undolabl package:
\newcommand\overrideLTSlabel[2]{%
  \@bsphack
    \ifnum \value{pagesLTS.pnc.\pagesLTS@pnc}>1%
      \edef\pagesLTStmpA{#1}%
      \edef\pagesLTStmpB{pagesLTS.\pagesLTS@pnc.local}%
      \ifx\pagesLTStmpA\pagesLTStmpB%
          \immediate\write\@auxout{\string\undonewlabel{#1}}%
          \@overriddenmessage s{#1}%
      \fi%
    \fi%
    \pagesLTS@putlabel{#1}{#2}{0}%
  \@esphack%
  }

%    \end{macrocode}
% \end{macro}
%
% \begin{macro}{\@pagesLTS@Prelim@EveryShipout}
% \indent Because we cannot make references to pages with \xpackage{fnsymbol}
% page \textquotedblleft numbers\textquotedblright{} manually
% with \xpackage{hyperref}, we use \nolinebreak{|\phantomsection|\,s} and
% refer to one of those. But because we do not know how many
% \nolinebreak{|\phantomsection|\,s} and \nolinebreak{|\section*|\,s}
% are introduced by the user (or other packages;
% cf.~\LaTeX~bug~2298: knowing level of |section*|,
% \ \url{http://www.latex-project.org/cgi-bin/ltxbugs2html?category=LaTeX&responsible=anyone&state=open&keyword=&pr=latex%2F2298&search=}),
% we cannot refer to the last one as we did with the pages.
%
%    \begin{macrocode}
\newcommand{\@pagesLTS@Prelim@EveryShipout}{%
%% The following code is from the prelim2e package             %%
%% [2009/05/29 v1.3] by Martin Schr\"{o}der (Thanks!):         %%
  \bgroup
    \dimen\z@=\wd\@cclv
    \dimen\@ne=\ht\@cclv
    \dimen\tw@=\dp\@cclv
    \dimen\thr@@=\dimen1
    \advance\dimen\thr@@ by \dimen\tw@
    \global\setbox\@cclv\vbox to \dimen\thr@@{%
      \hb@xt@\dimen\z@{%
        \box\@cclv%
        \hss%
        }%
      \vbox to \z@{%
        \hb@xt@\dimen\z@{%
          \let\protect\relax
%% Code not from prelim2e package:                             %%
%    \end{macrocode}
%
% Therefore each page with \xpackage{fnsymbol} page
% \textquotedblleft number\textquotedblright{} receives a
% |\phantomsection| and a label, which includes a number increased
% by one for each page. This is done for
% |pagesLTS.fnsymbol.local.\arabic{pagesLTS.fnsymbol.cont}|
% as well as
% |pagesLTS.fnsymbol|, |pagesLTS.\pagesLTS@pnc|, and
% |pagesLTS.\pagesLTS@pnc.local|.\\
% In case an older label already existed, it is overwritten by an
% |\overridelabel| command.
%
%    \begin{macrocode}
          \ifx\pagesLTS@pnc\pagesLTS@fns%
            \pagesLTS@ifcounter{pagesLTS.fnsymbol.cont}%
            \addtocounter{pagesLTS.fnsymbol.cont}{1}%
            \ltx@ifpackageloaded{hyperref}{%
              \phantomsection%
              \hypertarget{pagesLTS.fnsymbol.local.\arabic{pagesLTS.fnsymbol.cont}}{}%
              }{}%
            \ifnum \pagesLTS@esov=\pagesLTS@zero%
              \label{pagesLTS.fnsymbol}%
            \else%
              \overridelabel{pagesLTS.fnsymbol}%
            \fi%
          \else%
            \ltx@ifpackageloaded{hyperref}{\phantomsection}{}%
            \if@filesw%
              \overridelabel{pagesLTS.\pagesLTS@pnc}%
%    \end{macrocode}
%
% We need to go forward one page (and later backward again), because
% |\overrideLTSlabel| calls a |\pagesLTS@putlabel|, and that one uses
% |\addtocounter{page}{-1}...\addtocounter{page}{+1}|, which is not
% needed here.
%
%    \begin{macrocode}
              \addtocounter{page}{+1}%
                \overrideLTSlabel{pagesLTS.\pagesLTS@pnc.local}{\theCurrentPageLocal}%
              \addtocounter{page}{-1}%
            \fi%
          \fi%
%% Code from prelim2e package again:                           %%
        }%
        \vss%
      }%
      \vss%
    }%
    \wd\@cclv=\dimen\z@
    \ht\@cclv=\dimen\@ne
    \dp\@cclv=\dimen\tw@
  \egroup
%% End of code from the prelim2e package.                      %%
  }

%    \end{macrocode}
% \end{macro}
%
% \pagebreak
%
% \begin{macro}{\EveryShipout}
% \indent At the end of each shipout, the following commands are executed:
%
%    \begin{macrocode}
\EveryShipout{%
  \ifnum\value{page}>0% \relax
  \else%
    \ifnum\value{page}=0%
      \PackageWarning{pageslts}{%
        Counter `page' is zero!\MessageBreak%
        If the page numbering scheme is not arabic\MessageBreak%
        and further not extended\MessageBreak%
        (see Page counter overflow in the pageslts\MessageBreak%
        documentation), without other measures\MessageBreak%
        this will lead to a counter overflow.\MessageBreak%
      }%
    \else%
      \ifnum\value{page}<0%
        \PackageWarning{pageslts}{%
          Counter `page' is negative: `\the\value{page}'!\MessageBreak%
          If the page numbering scheme is not arabic\MessageBreak%
          and further not extended\MessageBreak%
          (see Page counter overflow in the pageslts\MessageBreak%
          documentation), without other measures\MessageBreak%
          this will lead to a counter overflow.\MessageBreak%
        }%
      \else%
        \PackageError{pageslts}{%
          Counter `page' does not have a recognized value:\MessageBreak%
          `\the\value{page}'\MessageBreak%
          \@ehd \MessageBreak%
          }%
      \fi%
    \fi%
  \fi%
%    \end{macrocode}
%
% If the |CurrentPage| is equal to one, this is the first shipout.
%
%    \begin{macrocode}
  \ifnum \value{CurrentPage}=1% This is the first shipout!
%    \end{macrocode}
%
% We check whether some page numbering scheme was defined by
% |\pagenumbering{...}| (\textbf{as it schould be!}):
%
%    \begin{macrocode}
    \ifx\pagesLTS@called\pagesLTS@zero%
%    \end{macrocode}
%
% If it was not defined (i.\,e. |\pagesLTS@called| is zero),
% the user should be informed, that a |\pagenumbering{...}|
% is missing behind |\begin{document}|.
% Of course, it is possible that some package did some
% pages of output with |\AtBeginDocument|. In that case,
% one |\pagenumbering{...}| before |\begin{document}| and
% one |\pagenumbering{...}| (with the same argument, of course!)
% behind |\begin{document}| could help somewhat. When |\PackageError|
% was used here, the error message was not written to the screen and
% the \xfile{.log}-file, but into the document. Therefore we just make
% a note to give the error message later (|\AtEndDocument|). At that
% time unfortunately most of the document has already been compiled
% (or did not compile due to this error), but I do not know how to
% change that.
%
%    \begin{macrocode}
      \global\def\pncmissing{1}%
%    \end{macrocode}
%
% We save the current value of the page,
%
%    \begin{macrocode}
      \mathchardef\pagesLTS@tmpD=\arabic{page}%
%    \end{macrocode}
%
% determine the current page numbering scheme,
%
%    \begin{macrocode}
%% Code from Andres L\"{o}h, Universiteit Utrecht (NL)         %%
      \def\extract#1{\expandafter\extract@ #1\END}%
      \def\extract@#1\csname @#2\endcsname#3\END{#2}%
      \edef\pagesLTS@tmpQ{\extract\thepage}%
%% End of code from Andres L\"{o}h                             %%
      \let\pagesLTS@tmpP\pagesLTS@tmpQ%
%    \end{macrocode}
%
% set the current page numbering scheme to~0
% (because before the beginning of the document it should be~0),
%
%    \begin{macrocode}
      \def\pagesLTS@pnc{0}%
%    \end{macrocode}
%
% and then issue a |\pagenumbering| command with the determined
% page numbering scheme as argument:
%
%    \begin{macrocode}
      \pagenumbering{\pagesLTS@tmpP}%
%    \end{macrocode}
%
% This resets the page to one (if option |pagecontinue=false|
% was chosen), but because we do not start a new page numbering
% scheme here but manifest a page numbering scheme, which
% the user forgot to define, the page number should not have
% been reset to one. (This is the first page, but maybe the user
% wants it to have page number 2001?) Therefore we revert this here
% and set the page number to its value, which was saved before the
% |\pagenumbering| command.
%
%    \begin{macrocode}
      \setcounter{page}{\pagesLTS@tmpD}%
    \fi%
%    \end{macrocode}
%
% We are at the first page, so we put the label here.
%
%    \begin{macrocode}
    \pagesLTS@writelabel{0}%
  \fi%
%    \end{macrocode}
%
% If the current page numbering scheme |\pagesLTS@pnc| is
% |\pagesLTS@fns| (which is defined as \texttt{fnsymbol}),
% the label is set by \linebreak
% |\@pagesLTS@Prelim@EveryShipout| (see just above),
% and |\pagesLTS@esov| is set to the (real) number (not the name) of
% this page numbering scheme, |\arabic{pagesLTS.fnsymbol.cont}|.\\
% When no more pages with \xpackage{fnsymbol} page
% \textquotedblleft number\textquotedblright{} are shipped out,
% the value remains fixed and we have our reference to the last page
% of the \xpackage{fnsymbol} page numbering range.
% (At least we will have that reference after some more work,
% see below).
%
%    \begin{macrocode}
  \ifx\pagesLTS@pnc\pagesLTS@fns%
    \@pagesLTS@Prelim@EveryShipout%
    \gdef\pagesLTS@esov{\arabic{pagesLTS.fnsymbol.cont}}%
%    \end{macrocode}
%
% When another page numbering scheme was reused (in the example file
% \xpackage{Roman}), we also need to apply\\
% |\@pagesLTS@Prelim@EveryShipout|,
% because otherwise we would get multiply defined labels.
%
%    \begin{macrocode}
  \else%
    \ifnum \value{pagesLTS.pnc.\pagesLTS@pnc}>1%
      \@pagesLTS@Prelim@EveryShipout%
    \fi%
  \fi%
%    \end{macrocode}
%
% The |CurrentPage| as well as the |pagesLTS.current.local.\pagesLTS@pnc|
% are advanced by one (because one page was shipped out and the next is
% about to begin).
%
%    \begin{macrocode}
  \addtocounter{CurrentPage}{1}%
  \addtocounter{pagesLTS.current.local.\pagesLTS@pnc}{1}%
  }

%    \end{macrocode}
% \end{macro}
%
% \begin{macro}{\pagesLTS@putlabelhyper}
% \indent Here the labels are set, if the \xpackage{hyperref} package
% was loaded. Simply using |\label| would not work, because labels
% wait for the output routines to work, and there may not be any
% more invocations of the output routines. To force the write out
% we need to do an |\immediate| write.
%
%    \begin{macrocode}
\newcommand{\pagesLTS@putlabelhyper}[2]{%
  \ifHy@pageanchor \relax%
  \else%
%    \end{macrocode}
%
% If the \xpackage{hyperref} package is used, but pageanchors are disabled,
% the hyperlinking will not work.
%
%    \begin{macrocode}
    \PackageError{pagesLTS}{hyperref option pageanchor disabled}{%
      The \string\lastpageref{#1} link doesn't work\MessageBreak%
      using hyperref with disabled option `pageanchor'.\MessageBreak%
    }%
  \fi
%    \end{macrocode}
%
% If use of the \xfile{.aux}-file is allowed, the label for
% \texttt{LastPage} is written into that file, the page
% reference depending on the options, which where set for
% the \xpackage{hyperref} package.
%
%    \begin{macrocode}
%% The following code is from the hyperref package             %%
%% [2010/04/17 v6.80x; newer versions are available]           %%
%% by Heiko Oberdiek (Big Thanks!).                            %%
  \if@filesw
    \begingroup
      \let\@number\@firstofone
      \ifHy@pageanchor
        \ifHy@hypertexnames
          \ifHy@plainpages
            \def\Hy@temp{\arabic{page}}%
          \else
            \Hy@unicodefalse
%% Code not from hyperref package:                             %%
%% The following lines are modified from the hyperref package. %%
%% Without the modification, after the first shipout "PD1" is  %%
%% inserted each time |\pdfstringdef\Hy@temp{\thepage}| is     %%
%% executed (if |fnsymbol| is not used).                       %%
            \ifnum \value{CurrentPage}=1%
              \ifx\pagesLTS@pnc\pagesLTS@fns%
                \pdfstringdef\Hy@temp{\thepage}%
              \else%
                \def\Hy@temp{\thepage}%
              \fi%
            \else%
              \pdfstringdef\Hy@temp{\thepage}%
            \fi%
%% Code from hyperref package again:                           %%
          \fi
        \else
          \def\Hy@temp{\the\Hy@pagecounter}%
        \fi
      \fi
%% End of code from the hyperref package.                      %%
%% (The following four lines are modified                      %%
%%  from the hyperref package.)                                %%
      \immediate\write\@auxout{\string
        \newlabel{#1}{{}{#2}{}{\ifHy@pageanchor page.\Hy@temp\fi}{}}}%
    \endgroup%
  \fi%
  }

%    \end{macrocode}
% \end{macro}
%
% \phantomsection
% \begin{macro}{\pagesLTS@putlabel}\label{pagesLTSputlabel}
% \indent Since the page has been put out, we are on the page after
% that page. We therefore subtract one from the page counter.
%
%    \begin{macrocode}
\newcommand{\pagesLTS@putlabel}[3]{%
  \addtocounter{page}{-1}%
%    \end{macrocode}
%
% When the \xpackage{showkeys} package has been loaded in \texttt{draft} mode,
% in the margin for each label a box is displayed with the name of the label.
% \xpackage{showkeys} accomplishes this by redefining |\label|, but
% \xpackage{pageslts} does not use |\label|, but writes directly to the
% {\nobreak |\jobname|\xfile{.aux}-}\allowbreak file, and this is generally done
% \emph{after} the according page has shipped out, therefore no box can be placed
% on the \emph{preceding} page. At least \xpackage{pageslts} gives a warning, that
% \xpackage{showkeys} cannot present the respective label.
%
%    \begin{macrocode}
  \ifx\pagesLTS@SK\pagesLTS@one\relax%
    \message{^^J}%
    \message{Package pageslts Warning: Package showkeys without option final loaded,}%
    \ifnum\value{pagesLTS.pagenr}<1%
      \message{(pageslts)                but label #1 on page \thepage\space(about \theCurrentPage)}%
    \else%
      \message{(pageslts)                but label #1 on page \thepage\space(about \theCurrentPage\space of \arabic{pagesLTS.pagenr})}%
    \fi%
    \message{(pageslts)                cannot be shown, because pageslts does not use \string\label,}%
    \message{(pageslts)                but writes directly to the \jobname.aux file. ^^J}%
  \fi%
%    \end{macrocode}
%
% If the \xpackage{hyperref} package is used, the format of the labels is somewhat longer.
%
%    \begin{macrocode}
  \ltx@ifpackageloaded{hyperref}{\pagesLTS@putlabelhyper{#1}{#2}}{% else
%    \end{macrocode}
%
% \pagebreak
%
% If the \xpackage{hyperref} package is not used, there will be no hyperlinks,
% and the label is written in the way of the old \xpackage{lastpage} package.
% But we must remember to undo the label first, if it already exists.
%
%    \begin{macrocode}
    \if@filesw%
      \ifnum \value{pagesLTS.pnc.\pagesLTS@pnc}<2%
%    \end{macrocode}
%
% When the \xpackage{nameref} package is used, |\newlabel| needs five instead of two arguments:
%
%    \begin{macrocode}
        \ltx@ifpackageloaded{nameref}{%
          \immediate\write\@auxout{\string\newlabel{#1}{{}{#2}{}{}{}}}}{% else
          \immediate\write\@auxout{\string\newlabel{#1}{{}{#2}}}}%
      \else%
        \edef\pagesLTStmpA{#1}%
        \edef\pagesLTStmpB{pagesLTS.\pagesLTS@pnc.local}%
        \ifx\pagesLTStmpA\pagesLTStmpB%
          \edef\pagesLTStmpA{#3}%
          \ifx\pagesLTStmpA\pagesLTS@one%
%    \end{macrocode}
%
% Only when the third argument of |\pagesLTS@putlabel| is $1$,
% we do need to undo the label. Otherwise there is no label to
% undo, and the \xpackage{undolabl} package would give an error.
%
%    \begin{macrocode}
            \immediate\write\@auxout{\string\undonewlabel{#1}}%
          \fi%
        \fi%
        \ltx@ifpackageloaded{nameref}{%
          \immediate\write\@auxout{\string\newlabel{#1}{{}{#2}{}{}{}}}}{% else
          \immediate\write\@auxout{\string\newlabel{#1}{{}{#2}}}}%
      \fi%
    \fi%
  }%
%    \end{macrocode}
%
% After the writeout we restore the page number again,
% since there might be other things still to be done.
%
%    \begin{macrocode}
  \addtocounter{page}{+1}%
  }

%    \end{macrocode}
% \end{macro}
%
% \pagebreak
%
% \begin{macro}{\pagesLTS@putlabels}
% \indent |\pagesLTS@putlabels| is nearly identical to |\pagesLTS@putlabelV|:
%
%    \begin{macrocode}
\newcommand{\pagesLTS@putlabels}{%
  \addtocounter{page}{-1}%
  \addtocounter{CurrentPage}{-1}%
  \addtocounter{pagesLTS.current.local.\pagesLTS@pnc}{-1}%
%    \end{macrocode}
%
% If |\pagenumbering{...}| has not been used,
% |\pagesLTS@pnc| is still zero ($0$, |\pagesLTS@zero|),
% and the according warning message is given.
%
%    \begin{macrocode}
  \ifx\pagesLTS@pnc\pagesLTS@zero%
   \PackageWarning{pagesLTS}{No page numbering scheme found:\MessageBreak%
     \pagesLTS@messageNPN }%
%    \end{macrocode}
%
% \noindent otherwise the numbered label is written,
% and if the page numbering scheme was not used before,
% the unnumbered label is written, too.
%
%
%    \begin{macrocode}
  \else%
    \pagesLTS@writelabel{\pagesLTS@pnc.\arabic{pagesLTS.pnc.\pagesLTS@pnc}}%
    \ifnum \value{pagesLTS.pnc.\pagesLTS@pnc}<2%
      \ifx\pagesLTS@pnc\pagesLTS@fns% \relax
      \else%
        \pagesLTS@writelabel{\pagesLTS@pnc}%
      \fi%
    \fi%
  \fi%
%    \end{macrocode}
%
% \noindent Before the label for the \texttt{LastPages} can be put,
% we must advance one page again, because |\pagesLTS@putlabel| itself
% goes back one page (and at its end forward again).
%
%    \begin{macrocode}
  \addtocounter{page}{+1}%
  \pagesLTS@putlabel{LastPages}{\theCurrentPage}{1}%
%    \end{macrocode}
%
% \noindent Here should follow a\\
%  |\addtocounter{page}{-1}|,\\
% but we have to remember to increase the page counters again,
% which were decreased at the start of this |\pagesLTS@putlabels|
% command, and that would include\\
%  |\addtocounter{page}{+1}|,\\
% therefore this two lines cancel each other and
% therefore just can be skipped. But the other counters have to
% be increased:
%
%    \begin{macrocode}
  \addtocounter{CurrentPage}{+1}%
  \addtocounter{pagesLTS.current.local.\pagesLTS@pnc}{1}%
  }

%    \end{macrocode}
% \end{macro}
%
% \pagebreak
%
% \begin{macro}{\AtBeginDocument}
% \indent |\AtBeginDocument| we write into the \xfile{aux} file regarding
% the double loading of the package.
% Then it is checked whether the \texttt{endfloat}
% package has been loaded, whether it is newer than March 1992 (i.\,e. at least
% April 1992 v2.0), in which case it is compatible with this
% \xpackage{pageslts} package.\\
% If it is even newer than 2011/12/24, it is the recent version
% (as of the time of last revision of this documentation:
% 2011/12/25 v2.5d).
%
%    \begin{macrocode}
\AtBeginDocument{%
  \if@filesw%
    \immediate\write\@auxout{\string\gdef\string\pagesLTS@loaded{p@gesLTSnotlo@ded}}%
  \fi%
  \@ifpackageloaded{endfloat}%
   {\@ifpackagelater{endfloat}{1992/03/31}% April 1992 v2.0
    {\@ifpackagelater{endfloat}{2011/12/24}{% 2011/12/25 v2.5d
       \relax}%
%    \end{macrocode}
%
% If it is compatible, but not the recent version, a warning is given:
%
%    \begin{macrocode}
     {\PackageWarningNoLine{pageslts}{Old endfloat package detected:\MessageBreak%
        There is a newer version of the endfloat package available.\MessageBreak%
        Please consider updating your version.\MessageBreak%
        The pageslts package might be incompatible with\MessageBreak%
        your current endfloat package.\MessageBreak%
      }%
     }%
%    \end{macrocode}
%
% If it is so very old, that it is not compatible,
% an Error message is given:
%
%    \begin{macrocode}
    }{\PackageError{pageslts}{Incompatible, very old endfloat package detected.}%
      {The very old version 2.0 (and earlier) of the\MessageBreak%
       endfloat package actually redefined the \string\enddocument ,\MessageBreak%
       and so interfered drastically with the LaTeX2e commands\MessageBreak%
       which make use of \string\AtEndDocument .\MessageBreak%
       Newer versions of the endfloat package exists\MessageBreak%
       (at least: v2.5d as of 2011/12/25)\MessageBreak%
       in modern documentation form,\MessageBreak%
       which should be available from CTAN.\MessageBreak%
       Please update your endfloat package\MessageBreak%
       for use with the pageslts package.\MessageBreak %
      }%
     }%
   }{}%
%    \end{macrocode}
%
% \pagebreak
%
% It is checked whether the old \xpackage{lastpage} package has been loaded.\\
% (If it has been loaded indeed, the |\lastpage@putlabel| is
% \textquotedblleft killed\textquotedblright ,
% see subsection~\ref{ssec:lastpage}.)
%
%    \begin{macrocode}
  \@ifpackageloaded{lastpage}%
   {\@ifpackagelater{lastpage}{2010/07/28}% 2010/07/29 v1.2a
      {\@ifpackagelater{lastpage}{2013/01/27}% 2013/01/28 v1.2l
         {\PackageWarning{pageslts}{lastpage package detected.\MessageBreak%
           With pageslts package in use, lastpage has no function.\MessageBreak%
           Just remove the lastpage package from your document.\MessageBreak%
           }%
         }{%
         \PackageWarning{pageslts}{Old lastpage package detected.\MessageBreak%
           With pageslts package in use, lastpage has no function.\MessageBreak%
           Just remove the lastpage package from your document.\MessageBreak%
           At least update it!\MessageBreak%
           }%
         }%
      }{%
       \PackageWarning{pageslts}{Incompatible package lastpage detected:\MessageBreak%
         Package pageslts was loaded, but also an old\MessageBreak%
         version of the lastpage package.\MessageBreak%
         pageslts has all functionality of the lastpage\MessageBreak%
         package (and more), so just remove the lastpage\MessageBreak%
         package from your document.\MessageBreak%
         (At least update it!)\MessageBreak%
         pageslts will now ''kill'' the lastpage@putlabel\MessageBreak%
         command of the lastpage package.\MessageBreak%
       }%
       \gdef\lastpage@putlabel{\relax}%
      }%
   }{}%
%    \end{macrocode}
%
% Further it is checked whether the \xpackage{alphalph} package has been loaded.
% If that is the case, the commands are defined accordingly.\\
%
%    \begin{macrocode}
  \ltx@ifpackageloaded{alphalph}%
    {\newalphalph{\AlphMult}[mult]{\@Alph}{26}%
     \newalphalph{\alphMult}[mult]{\@alph}{26}%
     \newalphalph{\fnsymbolmult}[mult]{\@fnsymbol}{5}%
    }{}%
%    \end{macrocode}
%
% \pagebreak
%
% Further it is checked whether the \xpackage{hyperref} package has been loaded:
%
%    \begin{macrocode}
  \ltx@ifpackageloaded{hyperref}{%
%    \end{macrocode}
%
% and whether the \xpackage{pdfpages} package is loaded:
%
%    \begin{macrocode}
     \@ifpackageloaded{pdfpages}%
       {\PackageWarningNoLine{pageslts}{Package pdfpages detected.\MessageBreak%
          Using hyperref with pdfpages can cause problems. See\MessageBreak%
          https://www.ctan.org/pkg/pax\MessageBreak%
          for the Pdf Annotations eXtractor, pax%
          }%
       }{% \relax
        }%
%    \end{macrocode}
%
% The \xpackage{undolabl} package has been updated and now uses |\undonewlabel| with only
% one argument.
%
%    \begin{macrocode}
     \@ifpackageloaded{undolabl}%
       {\@ifpackagelater{undolabl}{2010/07/14}%    2010/07/15 v1.0d
          {\@ifpackagelater{undolabl}{2015/03/29}% 2015/03/29 v1.0l
             {% recent version as of the time of last revision of this package: OK
             }{% old, but not obsolete version
               \PackageWarningNoLine{pageslts}{Old version of undolabl package used.\MessageBreak%
                 See https://www.ctan.org/pkg/undolabl\MessageBreak%
                 for a new version.\MessageBreak%
                }%
             }
          }{\PackageError{pageslts}{Incompatible, obsolete version of undolabl package used.}{%
              See https://www.ctan.org/pkg/undolabl\MessageBreak%
              for a new version.\MessageBreak%
              Type  X <return>  to quit.\MessageBreak%
             }%
          }
       }{\PackageError{pageslts}{Package undolabl missing}{%
           Package undolabl not found.\MessageBreak%
           The pageslts package needs the undolabl package.\MessageBreak%
           See https://www.ctan.org/pkg/undolabl\MessageBreak%
           Type  X <return>  to quit.\MessageBreak%
           }%
       }
%    \end{macrocode}
%
% \pagebreak
%
% Additionally a version check of the available \xpackage{hyperref}
% package is performed and if need be a warning is issued:
%
%    \begin{macrocode}
     \@ifpackagelater{hyperref}{2012/11/05}{% 2012/11/06 v6.83m
       \relax}{%
       \PackageWarningNoLine{pageslts}{Old hyperref package detected:\MessageBreak%
         There is a newer version of the\MessageBreak%
         hyperref package available:\MessageBreak%
         https://www.ctan.org/pkg/hyperref\MessageBreak%
         Please consider updating your version.\MessageBreak%
       }%
     }%
     %% pageslts supports the use of the package hyperref by
     %% Heiko Oberdiek (hyperref version 2012/11/06 v6.83m).
     %% pageslts may work with earlier versions of this packages,
     %% but this was not tested.
    }{%
%    \end{macrocode}
%
% If no \xpackage{hyperref} package in use is detected, a warning is issued, too:
%
%    \begin{macrocode}
      \PackageWarning{pageslts}{Package hyperref NOT detected.\MessageBreak%
        pageslts would support hyperref. The page references\MessageBreak%
        will NOT be hyperlinked!\MessageBreak }%
     }%
%    \end{macrocode}
%
% If the \xpackage{showkeys} package has been loaded in \texttt{draft} mode,
% warnings about missing label boxes will be issued (see p.~\pageref{pagesLTSputlabel}),
% but because it cannot be checked for \xpackage{showkeys} after |\AtBeginDocument|,
% the check must be done here and the result remembered.
%
%    \begin{macrocode}
  \ltx@ifpackageloaded{showkeys}{%
    \@ifpackagewith{showkeys}{final}{}{\gdef\pagesLTS@SK{1}}}{% else \relax
    }%
  }

%    \end{macrocode}
% \end{macro}
%
% \begin{macro}{\AtEndDocument}
% \indent |\AtEndDocument| the check is repeated. If \xpackage{showkeys} was
% loaded after the check performed before, an error message is issued.
%
%    \begin{macrocode}
\AtEndDocument{%
  \ltx@ifpackageloaded{showkeys}{%
    \ifx\pagesLTS@SK\pagesLTS@one\relax%
    \else%
      \providecommand*\pagesLTS@sklf[1]{}%
      \ifx\showkeyslabelformat\pagesLTS@sklf\relax%
      \else%
        \PackageError{pageslts}{Package showkeys loaded after pageslts}{%
          The showkeys package with option "final" has been loaded\MessageBreak%
          after the pageslts package.\MessageBreak%
          Please first load showkeys and then pageslts}%
      \fi%
    \fi%
  }{% else no showcase, \relax
  }%
%    \end{macrocode}
%
% We also give the error message about the missing (i.\,e.~not found)
% page numbering scheme, which could not be given in |\EveryShipout|.
%
%    \begin{macrocode}
  \ifx\pncmissing\pagesLTS@one\relax%
    \PackageError{pageslts}{pagenumbering scheme missing}{\pagesLTS@messageNPN }%
  \fi%
%    \end{macrocode}
%
% Then we put in a |\message| to show, in what order things (which were called)
% are done (see subsection~\ref{ssec:aed}).
%
%    \begin{macrocode}
  \message{^^J%
    AED: pageslts setting LastPage ^^J}%
%    \end{macrocode}
%
% After this we issue a |\clearpage| to put out all floats,
% which are still floating, remember the page number (if |fnsymbol|),
% and after that we place the \texttt{LastPage} label.
%
%    \begin{macrocode}
  \clearpage%
  \ifx\pagesLTS@pnc\pagesLTS@fns%
    \def\pagesLTS@tmpA{\arabic{pagesLTS.fnsymbol.local}}%
    \ifnum \pagesLTS@eso=\pagesLTS@tmpA%
      \gdef\pagesLTS@rerun{0}%
    \else%
      \gdef\pagesLTS@rerun{1}%
    \fi%
    \if@filesw%
      \immediate\write\@auxout{\string\gdef\string\pagesLTS@eso{\pagesLTS@tmpA}}%
    \fi%
  \fi%
  \pagesLTS@putlabel{LastPage}{\thepage}{1}%
%    \end{macrocode}
%
% We do not need the temporary definition any more.
%
%    \begin{macrocode}
  \let\pagesLTS@tmpA\undefined%
  }

%    \end{macrocode}
% \end{macro}
%
% \begin{macro}{\AfterLastShipout}
% \indent |\AfterLastShipout| is a command from \textsc{Heiko Oberdiek's}
% \texttt{atveryend} package (see above).
%
%    \begin{macrocode}
\AfterLastShipout{%
%    \end{macrocode}
%
% If writing to the \xfile{.aux} file is allowed:
%
%    \begin{macrocode}
  \if@filesw%
%    \end{macrocode}
%
% The number of pages with the \xpackage{fnsymbol} page numbering scheme,
% |\pagesLTS@esov|, is saved via the \xfile{.aux} file (if it is not zero):
%
%    \begin{macrocode}
    \ifx\pagesLTS@esov\pagesLTS@zero%
    \else%
      \immediate\write\@auxout{\string
        \pagesLTS@ifcounter{pagesLTS.fnsymbol.local}}%
      \immediate\write\@auxout{\string
        \setcounter{pagesLTS.fnsymbol.local}{\pagesLTS@esov}}%
    \fi%
%    \end{macrocode}
%
% If the \xpackage{hyperref} package is in use, and the page numbering scheme
% of the last page is \xpackage{fnsymbol}, everything is quite more complicated.
% Therefore |\lastpageref| is switched from simple |\lastpagereftxt|
% to the more difficult |\lastpagerefend|.
%
%    \begin{macrocode}
    \ltx@ifpackageloaded{hyperref}{%
      \ifx\pagesLTS@pnc\pagesLTS@fns%
          \immediate\write\@auxout{\string
            \gdef\string\lastpageref{\string\lastpagerefend}}%
      \fi%
     }{}%
  \fi%
%    \end{macrocode}
%
% At the call of a |\pagenumbering{...}| command, everything
% for a split page numbering scheme is organized. For the last
% page numbering scheme, there is no |\pagenumbering{...}| command
% at the end, so we need to handle this here:
%
%    \begin{macrocode}
  \pagesLTS@ifcounter{pagesLTS.\pagesLTS@pnc.\arabic{pagesLTS.pnc.\pagesLTS@pnc}.local.count}%
  \setcounter{pagesLTS.\pagesLTS@pnc.\arabic{pagesLTS.pnc.\pagesLTS@pnc}.local.count}{%
    \value{pagesLTS.current.local.\pagesLTS@pnc}}%
%    \end{macrocode}
%
% And we are one page after the last one (|\AfterLastShipout|!),
% so we go back one page. (We again borrow the \texttt{pagesLTS.pnc.0} counter for the computations
% instead of defining yet another one.)
%
%    \begin{macrocode}
  \addtocounter{pagesLTS.\pagesLTS@pnc.\arabic{pagesLTS.pnc.\pagesLTS@pnc}.local.count}{-1}
  \ifnum \value{pagesLTS.pnc.\pagesLTS@pnc}>1%
    \mathchardef\pagesLTS@tmpD=\arabic{pagesLTS.pnc.0}%
    \setcounter{pagesLTS.pnc.0}{\value{pagesLTS.pnc.\pagesLTS@pnc}}%
    \@tempcnta=\value{pagesLTS.pnc.0}\relax%
    \loop%
      \ifnum\@tempcnta>1\relax%
        \addtocounter{pagesLTS.pnc.0}{-1}%
        \addtocounter{pagesLTS.\pagesLTS@pnc.\arabic{pagesLTS.pnc.\pagesLTS@pnc}.local.count}{%
          -\value{pagesLTS.\pagesLTS@pnc.\arabic{pagesLTS.pnc.0}.local.count}}%
        \@tempcnta=\value{pagesLTS.pnc.0}\relax%
    \repeat%
    \setcounter{pagesLTS.pnc.0}{\pagesLTS@tmpD}%
  \fi%
  \if@filesw%
    \immediate\write\@auxout{\string
      \pagesLTS@ifcounter{pagesLTS.\pagesLTS@pnc.\arabic{pagesLTS.pnc.\pagesLTS@pnc}.local.cnt}}%
    \edef\pagesLTS@tmpA{\arabic{pagesLTS.\pagesLTS@pnc.\arabic{pagesLTS.pnc.\pagesLTS@pnc}.local.count}}%
    \immediate\write\@auxout{\string
      \setcounter{pagesLTS.\pagesLTS@pnc.\arabic{pagesLTS.pnc.\pagesLTS@pnc}.local.cnt}{\pagesLTS@tmpA}}%
    \let\pagesLTS@tmpA\undefined%
  \fi%
%    \end{macrocode}
%
% We need to save (via the \xfile{.aux} file) the page name |\thepage| and
% the page number |\arabic{CurrentPage}| of the last page,
% in case the last page has \xpackage{fnsymbol} page numbering scheme.
%
%    \begin{macrocode}
  \addtocounter{page}{-1}%
  \edef\pagesLTS@tmpA{\thepage}%
  \if@filesw%
    \immediate\write\@auxout{\string
      \gdef\string\pagesLTS.lastpage{\pagesLTS@tmpA}}%
  \fi%
  \addtocounter{page}{+1}%
  \addtocounter{CurrentPage}{-1}%
  \edef\pagesLTS@tmpB{\arabic{CurrentPage}}%
  \if@filesw%
    \immediate\write\@auxout{\string
      \setcounter{pagesLTS.pagenr}{\pagesLTS@tmpB}}%
  \fi%
  \addtocounter{CurrentPage}{+1}%
%    \end{macrocode}
%
% The \texttt{VeryLastPage} label is set here, and when
% |\lastpageref{VeryLastPage}| instead of |\lastpageref{LastPage}| is used,
% it should really point to the last page.
% |LastPage| and |VeryLastPage| should be identical,
% unless a package was active with output |\AtEndDocument|
% \textit{after} the \xpackage{pageslts} package.
%
%    \begin{macrocode}
  \message{^^J%
    AED: pageslts setting VeryLastPage via AfterLastShipout ^^J}%
  \pagesLTS@putlabel{VeryLastPage}{\thepage}{1}%
%    \end{macrocode}
%
% The \texttt{LastPages} label is set here, and
% |\lastpageref{LastPages}| gives the total number of pages
% and points to the (very) last page.
%
%    \begin{macrocode}
  \message{^^J%
    AED: pageslts setting LastPages via AfterLastShipout ^^J}%
  \pagesLTS@putlabels%
  \typeout{^^J}%
  \ifodd\pagesLTS@tmpB%
    \@PackageInfoNoLine{pageslts}{Total number of pages is odd}%
  \else%
    \@PackageInfoNoLine{pageslts}{Total number of pages is even}%
  \fi%
  \typeout{^^J}%
%    \end{macrocode}
%
% We do not need the temporary definitions any more.
%
%    \begin{macrocode}
  \let\pagesLTS@tmpA\undefined%
  \let\pagesLTS@tmpB\undefined%
  }

%    \end{macrocode}
% \end{macro}
%
% \pagebreak
%
% \begin{macro}{\AtEndAfterFileList}
% \indent |\AtEndAfterFileList{...}| is even later:
% \begin{quote}
% \textquotedblleft After the .aux file closing and reading \LaTeX{} prints the
% file list if requested by |\listfiles|. Then this hook is executed.\textquotedblright
% \end{quote}
% (\xpackage{atveryend} package of \textsc{Heiko Oberdiek}, v1.7 as of 2011/04/23,
% newer version available.) Here it is used for a rerun hint.\\
% For example if the page numbering scheme of the last page of the
% \textsf{pageslts-example.tex} file is changed to |fnsymbol| and two runs
% of \pdfLaTeX{} are done, \pdfLaTeX{} will be happy and will not complain
% about changed labels. But indeed, a \textit{third} run is necessary
% and indicated by the warning message below.
%
%    \begin{macrocode}
\AtEndAfterFileList{%
  \ifx\pagesLTS@rerun\pagesLTS@one%
    \PackageWarningNoLine{pageslts}{%
      Label(s) may have changed.\MessageBreak%
      Rerun to get cross-references right%
     }%
  \fi%
  }

%    \end{macrocode}
% \end{macro}
%
% \begin{macro}{\frontmatter}
% \indent |\frontmatter| often contains |\pagenumbering|, but for some unknown reason
% there are problems when another |\pagenumbering| with different page numbering scheme
% has been used before on the same page. (This would not make any sense anyway,
% because one page can only have one page numbering scheme.) This problem does not
% occur when two |\pagenumbering| commands are used inside normal text.
% Thus we need to check whether |\frontmatter| has been defined,
% whether it changes the page numbering scheme, and whether the page numbering scheme
% before |\frontmatter| was initiated at the same page.
% \end{macro}
%
% We need an |\ifundefined|, which neither requires \eTeX{} nor
% defines the command, which is to be tested, to |\relax| (in which case
% it would be no longer undefined for further tests). For this we use
% code from \textsc{Markus Kohm} similar to his |\scr@ifundefinedorrelax| from %
% \xpackage{scrbase.sty} of the KOMA script bundle,
% but without the necessarity to use \eTeX :
%
%    \begin{macrocode}
\newcommand{\pagesLTS@ifundefinedorrelax}[1]{%
  % Modified Markus Kohm code for use without e-TeX
  \begingroup\expandafter\expandafter\expandafter\endgroup
  \expandafter\ifx\csname #1\endcsname\relax%
    \expandafter\ifx\csname #1\endcsname\relax%
      \expandafter\expandafter\expandafter\@firstoftwo
    \else%
      \expandafter\expandafter\expandafter\@secondoftwo
    \fi%
  \else%
    \expandafter\@firstoftwo
  \fi%
  }

\pagesLTS@ifundefinedorrelax{frontmatter}{% \relax
}{% else
  \ltx@GlobalPrependToMacro{\frontmatter}{\pagesLTS@prefrontmatter}%
  \ltx@GlobalAppendToMacro{\frontmatter}{\pagesLTS@postfrontmatter}%
 }

%    \end{macrocode}
%
% \begin{macro}{\pagesLTS@prefrontmatter}
%
% At the beginning of the real (i.e. unchanged) |\frontmatter| (FMB), we remember the
% current (c) page numbering (pn) scheme: its name (n) and page number (p).
% |\pagesLTS@FMBpncp| would be |1| if the according |\pagenumbering| command
% was used on the same page at the |\frontmatter|.
%
%    \begin{macrocode}
\newcommand{\pagesLTS@prefrontmatter}{%
  \xdef\pagesLTS@FMBpncn{\pagesLTS@pnc}%
  \xdef\pagesLTS@FMBpncp{\arabic{pagesLTS.current.local.\pagesLTS@pnc}}%
  }

%    \end{macrocode}
% \end{macro}
%
% \begin{macro}{\pagesLTS@postfrontmatter}
%
% The named checks are performed and in case of the named problem an error message
% is given. Any idea how to automatically solve this issue?
%
%    \begin{macrocode}
\newcommand{\pagesLTS@postfrontmatter}{%
  \xdef\pagesLTS@FMEpncn{\pagesLTS@pnc}%
  \ifx\pagesLTS@FMBpncn\pagesLTS@FMEpncn%
  \else%
    \ifx\pagesLTS@FMBpncn\pagesLTS@zero%
    \else%
      \ifx\pagesLTS@FMBpncp\pagesLTS@one%
        \PackageError{pageslts}{\string\pagenumbering\space before \string\frontmatter}{%
          Do not use \string\pagenumbering{\pagesLTS@FMBpncn} before \string\frontmatter \MessageBreak%
          on the same page!\MessageBreak%
          \string\frontmatter\space (re)defines the page numbering scheme to \pagesLTS@FMEpncn , \MessageBreak%
          thus earlier use of \string\pagenumbering{\pagesLTS@FMBpncn} on the same page is useless anyway.}%
      \fi%
    \fi%
  \fi%
}

%    \end{macrocode}
% \end{macro}
%
%    \begin{macrocode}
%</package>
%    \end{macrocode}
%
% \newpage
%
% \section{Installation}
%
% \subsection{Downloads\label{ss:Downloads}}
%
% Everything is available at \url{https://www.ctan.org},
% but may need additional packages themselves.\\
%
% \DescribeMacro{pageslts.dtx}
% For unpacking the |thumbs.dtx| file and constructing the documentation it is required:
% \begin{description}
% \item[-] \TeX Format \LaTeXe{}: \url{https://www.CTAN.org}
%
% \item[-] document class \xclass{ltxdoc}, 2015-03-26, 2.0w, \url{https://www.ctan.org/pkg/ltxdoc}
%
% \item[-] package \xpackage{holtxdoc}, 2012/03/21, v0.24, \url{https://www.ctan.org/pkg/holtxdoc}
%
% \item[-] package \xpackage{hypdoc}, 2011/08/19, v1.11, \url{https://www.ctan.org/pkg/hypdoc}
%
% \item[-] package \xpackage{geometry}, 2010/09/12, v5.6, \url{https://www.ctan.org/pkg/geometry}
%
% \item[-] package \xpackage{ulem}, 2012/05/18, no version number given, \url{https://www.ctan.org/pkg/ulem}
% \end{description}
%
% \DescribeMacro{pageslts.sty}
% The |pageslts.sty| for \LaTeXe{} (i.\,e. all documents using
% the \xpackage{pageslts} package) requires:
% \begin{description}
% \item[-] \TeX{} Format \LaTeXe{}, \url{https://www.CTAN.org}
%
% \item[-] package \xpackage{atveryend}, 2011/06/30, v1.8, \url{https://www.ctan.org/pkg/atveryend}
%
% \item[-] package \xpackage{everyshi}, 2001/05/15, v3.00, \url{https://www.ctan.org/pkg/everyshi}
%
% \item[-] package \xpackage{kvoptions}, 2011/06/30, v3.11, \url{https://www.ctan.org/pkg/kvoptions}
%
% \item[-] package \xpackage{letltxmacro}, 2010/09/02, v1.4, \url{https://www.ctan.org/pkg/letltxmacro}
%
% \item[-] package \xpackage{ltxcmds}, 2011/11/09, v1.22, \url{https://www.ctan.org/pkg/ltxcmds}
%
% \item[-] package \xpackage{rerunfilecheck}, 2011/04/15, v1.7, \url{https://www.ctan.org/pkg/rerunfilecheck}
%
% \item[-] package \xpackage{undolabl}, 2015/03/29, v1.0l, \url{https://www.ctan.org/pkg/undolabl}\\
% \end{description}
%
% \DescribeMacro{pageslts-example.tex}
% The |pageslts-example.tex| requires the same files as all
% documents using the \xpackage{pageslts} package, and additionally:
% \begin{description}
% \item[-] class \textsf{article}, 2014/09/29, v1.4h, from \textsf{classes.dtx}: \url{https://www.ctan.org/pkg/classes}
%
% \item[-] package \xpackage{alphalph}, 2011/05/13, v2.4, \url{https://www.ctan.org/pkg/alphalph}
%
% \item[-] package \xpackage{lipsum}, 2014/07/27, v1.3, \url{https://www.ctan.org/pkg/lipsum}
%
% \item[-] package \xpackage{showkeys}, 2014/10/28, v3.17, \url{https://www.ctan.org/pkg/showkeys}
%
% \item[-] package \xpackage{hyperref}, 2012/11/06, v6.83m, \url{https://www.ctan.org/pkg/hyperref}
%
% \item[-] package \xpackage{pageslts}, 2015/12/21, v1.2f, \url{https://www.ctan.org/pkg/pageslts}\\
%   (Well, it is the example file for this package, and because you are reading the
%    documentation for the \xpackage{pageslts} package, it can be assumed that you already
%    have some version of it~-- is it the current one?)
% \end{description}
%
% \pagebreak
%
% \DescribeMacro{papermas}
% The \xpackage{papermas} package is not required, but requires itself the
% the \xpackage{pageslts} package and can be considered as kind of add-on:
% \begin{description}
% \item[-] package \xpackage{papermas}, 2011/08/22, v1.0h, \url{https://www.ctan.org/pkg/papermas}
% \end{description}
%
% \DescribeMacro{endfloat}
% The \xpackage{endfloat} package is not required, but because
% the \xpackage{pageslts} package is incompatibel with very old versions
% of the \xpackage{endfloat} package (see subsection~\ref{sec:endfloat}),
% here the recent one is listed:
% \begin{description}
% \item[-] package \xpackage{endfloat}, 2011/12/25, v2.5d, \url{https://www.ctan.org/pkg/endfloat}
% \end{description}
%
% \DescribeMacro{prelim2e}
% The \xpackage{prelim2e} package is not required either, but because
% |Prelim@EveryShipout| code was taken from that package, it is listed, too:
% \begin{description}
% \item[-] package \xpackage{prelim2e}, 2009/05/29, v1.3, \url{https://www.ctan.org/pkg/prelim2e}
% \end{description}
%
% \DescribeMacro{fancyhdr}
% \DescribeMacro{nccfancyhdr}
% Neither \xpackage{fancyhdr} nor \xpackage{nccfancyhdr} package is required
% (the \xpackage{lastpage} package used its predecessor \xpackage{fancyheadings}),
% but because they were mentioned, also they are listed here:
% \begin{description}
% \item[-] package \xpackage{fancyhdr}, \url{https://www.ctan.org/pkg/fancyhdr}
%
% \item[-] package \xpackage{nccfancyhdr}, \url{https://www.ctan.org/pkg/nccfancyhdr}
% \end{description}
%
% \DescribeMacro{lastpage}
% \DescribeMacro{totpages}
% \DescribeMacro{totcount}
% \DescribeMacro{nofm}
% \DescribeMacro{count1to}
% \DescribeMacro{zref}
% As possible alternatives in section \ref{sec:Alternatives} there are listed
% \begin{description}
% \item[-] package \xpackage{lastpage}, 2015/03/29, v1.2m, \url{https://www.ctan.org/pkg/lastpage}
%
% \item[-] package \xpackage{totpages}, 2005/09/19, v2.00, \url{https://www.ctan.org/pkg/totpages}
%
% \item[-] package \xpackage{totcount}, 2011/01/25, v1.2, \url{https://www.ctan.org/pkg/totcount}
%
% \item[-] package \xpackage{nofm}, 1991/02/25, without version number,
%  \url{http://mirror.ctan.org/obsolete/macros/latex209/contrib/misc/nofm.sty},
%  does not work with e.\,g. \xpackage{hyperref}
%
% \item[-] package \xpackage{count1to}, 2009/05/24, v2.1, \url{https://www.ctan.org/pkg/count1to}
%
% \item[-] package \xpackage{zref}, 2012/04/04, v2.24, \url{https://www.ctan.org/pkg/zref}
% \end{description}
%
% \DescribeMacro{Oberdiek}
% \DescribeMacro{alphalph}
% \DescribeMacro{atveryend}
% \DescribeMacro{holtxdoc}
% \DescribeMacro{letltxmacro}
% \DescribeMacro{ltxcmds}
% \DescribeMacro{kvoptions}
% \DescribeMacro{rerunfilecheck}
% \DescribeMacro{zref}
% All packages of \textsc{Heiko Oberdiek's} bundle `oberdiek'
% (especially \xpackage{alphalph}, \xpackage{atveryend}, \xpackage{holtxdoc}, \xpackage{letltxmacro},
% \xpackage{ltxcmds}, \xpackage{kvoptions}, \xpackage{rerunfilecheck}, and \xpackage{zref})
% are also available in a TDS compliant ZIP archive:\\
% \url{http://mirrors.ctan.org/install/macros/latex/contrib/oberdiek.tds.zip}.\\
% It is probably best to download and use this, because the packages in there
% are quite probably both recent and compatible among themselves.
%
% \DescribeMacro{hyperref}
% \noindent \xpackage{hyperref} is not included in that bundle and needs to be
% downloaded separately,\\
% \url{http://mirrors.ctan.org/install/macros/latex/contrib/hyperref.tds.zip}.
%
% \DescribeMacro{M\"{u}nch}
% A hyperlinked list of my (other) packages can be found at
% \url{https://www.ctan.org/author/muench-hm}.
%
% \subsection{Package, unpacking TDS}
% \paragraph{Package.} This package is available on \url{CTAN.org}.
% \begin{description}
% \item[\url{http://mirrors.ctan.org/macros/latex/contrib/pageslts/pageslts.dtx}]\hspace*{0.1cm} \\
%       The source file.
% \item[\url{http://mirrors.ctan.org/macros/latex/contrib/pageslts/pageslts.pdf}]\hspace*{0.1cm} \\
%       The documentation.
% \item[\url{http://mirrors.ctan.org/macros/latex/contrib/pageslts/pageslts-example.pdf}]\hspace*{0.1cm} \\
%       The compiled example file, as it should look like.
% \item[\url{http://mirrors.ctan.org/macros/latex/contrib/pageslts/README}]\hspace*{0.1cm} \\
%       The README file.
% \end{description}
% There is also a |thumbs.tds.zip| available:
% \begin{description}
% \item[\url{http://mirror.ctan.org/install/macros/latex/contrib/pageslts.tds.zip}]\hspace*{0.1cm} \\
%       Everything in TDS compliant, compiled format.
% \end{description}
% which additionally contains\\
% \begin{tabular}{ll}
% pageslts.ins & The installation file.\\
% pageslts.drv & The driver to generate the documentation.\\
% pageslts.sty & The \xext{sty}le file.\\
% pageslts-example.tex & The example file.%
% \end{tabular}
%
% \bigskip
%
% \noindent For required other packages see the preceding subsection.
%
% \paragraph{Unpacking.} The \xfile{.dtx} file is a self-extracting
% \docstrip{} archive. The files are extracted by running the
% \xfile{.dtx} through \plainTeX :
% \begin{quote}
%   \verb|tex pageslts.dtx|
% \end{quote}
%
% About generating the documentation see paragraph~\ref{GenDoc} below.\\
%
% \paragraph{TDS.} Now the different files must be moved into
% the different directories in your installation TDS tree
% (also known as \xfile{texmf} tree):
% \begin{quote}
% \def\t{^^A
% \begin{tabular}{@{}>{\ttfamily}l@{ $\rightarrow$ }>{\ttfamily}l@{}}
%   pageslts.sty & tex/latex/pageslts.sty\\
%   pageslts.pdf & doc/latex/pageslts.pdf\\
%   pageslts-example.tex & doc/latex/pageslts-example.tex\\
%   pageslts-example.pdf & doc/latex/pageslts-example.pdf\\
%   pageslts.dtx & source/latex/pageslts.dtx\\
% \end{tabular}^^A
% }^^A
% \sbox0{\t}^^A
% \ifdim\wd0>\linewidth
%   \begingroup
%     \advance\linewidth by\leftmargin
%     \advance\linewidth by\rightmargin
%   \edef\x{\endgroup
%     \def\noexpand\lw{\the\linewidth}^^A
%   }\x
%   \def\lwbox{^^A
%     \leavevmode
%     \hbox to \linewidth{^^A
%       \kern-\leftmargin\relax
%       \hss
%       \usebox0
%       \hss
%       \kern-\rightmargin\relax
%     }^^A
%   }^^A
%   \ifdim\wd0>\lw
%     \sbox0{\small\t}^^A
%     \ifdim\wd0>\linewidth
%       \ifdim\wd0>\lw
%         \sbox0{\footnotesize\t}^^A
%         \ifdim\wd0>\linewidth
%           \ifdim\wd0>\lw
%             \sbox0{\scriptsize\t}^^A
%             \ifdim\wd0>\linewidth
%               \ifdim\wd0>\lw
%                 \sbox0{\tiny\t}^^A
%                 \ifdim\wd0>\linewidth
%                   \lwbox
%                 \else
%                   \usebox0
%                 \fi
%               \else
%                 \lwbox
%               \fi
%             \else
%               \usebox0
%             \fi
%           \else
%             \lwbox
%           \fi
%         \else
%           \usebox0
%         \fi
%       \else
%         \lwbox
%       \fi
%     \else
%       \usebox0
%     \fi
%   \else
%     \lwbox
%   \fi
% \else
%   \usebox0
% \fi
% \end{quote}
% If you have a \xfile{docstrip.cfg} that configures and enables \docstrip's
% TDS installing feature, then some files can already be in the right
% place, see the documentation of \docstrip.
%
% \subsection{Refresh file name databases}
%
% If your \TeX~distribution (\teTeX, \mikTeX,\dots) relies on file name
% databases, you must refresh these. For example, \teTeX\ users run
% \verb|texhash| or \verb|mktexlsr|.
%
% \subsection{Some details for the interested}
%
% \paragraph{Unpacking with \LaTeX.}
% The \xfile{.dtx} chooses its action depending on the format:
% \begin{description}
% \item[\plainTeX:] Run \docstrip\ and extract the files.
% \item[\LaTeX:] Generate the documentation.
% \end{description}
% If you insist on using \LaTeX\ for \docstrip\ (really,
% \docstrip\ does not need \LaTeX), then inform the autodetect routine
% about your intention:
% \begin{quote}
%   \verb|latex \let\install=y% \iffalse meta-comment
%
% File: pageslts.dtx
% Version: 2015/12/21 v1.2f
%
% Copyright (C) 2010 - 2015 by
%    H.-Martin M"unch <Martin dot Muench at Uni-Bonn dot de>
% Portions of code copyrighted by other people as marked.
%
% This package first started as a revision of the lastpage
% package of Jeffrey P. Goldberg (jeffrey+news at goldmark dot org),
% but it became obvious that a replacement was needed.
%
% This work may be distributed and/or modified under the
% conditions of the LaTeX Project Public License, either
% version 1.3c of this license or (at your option) any later
% version. This version of this license is in
%    http://www.latex-project.org/lppl/lppl-1-3c.txt
% and the latest version of this license is in
%    http://www.latex-project.org/lppl.txt
% and version 1.3c or later is part of all distributions of
% LaTeX version 2005/12/01 or later.
%
% This work has the LPPL maintenance status "maintained".
%
% The Current Maintainer of this work is H.-Martin Muench.
%
% This work consists of the main source file pageslts.dtx,
% the README, and the derived files
%    pageslts.sty, pageslts.pdf,
%    pageslts.ins, pageslts.drv,
%    pageslts-example.tex, pageslts-example.pdf.
%
% 'pageslts' is available on CTAN:
% https://www.ctan.org/pkg/pageslts
%
% Also a TDS.ZIP file is provided that contains all the files
% already sorted in a TDS tree:
% http://mirrors.ctan.org/install/macros/latex/contrib/pageslts.tds.zip
%
%<*ignore>
\begingroup
  \catcode123=1 %
  \catcode125=2 %
  \def\x{LaTeX2e}%
\expandafter\endgroup
\ifcase 0\ifx\install y1\fi\expandafter
         \ifx\csname processbatchFile\endcsname\relax\else1\fi
         \ifx\fmtname\x\else 1\fi\relax
\else\csname fi\endcsname
%</ignore>
%<*install>
\input docstrip.tex
\Msg{*********************************************************************************}
\Msg{* Installation}
\Msg{* Package: pageslts 2015/12/21 v1.2f Refers to special pages' numbers/names (HMM)}
\Msg{*********************************************************************************}

\keepsilent
\askforoverwritefalse

\let\MetaPrefix\relax
\preamble

This is a generated file.

Project: pageslts
Version: 2015/12/21 v1.2f

Copyright (C) 2010 - 2015 by
    H.-Martin M"unch <Martin dot Muench at Uni-Bonn dot de>
Portions of code copyrighted by other people as marked.

IMPORTANT NOTICE:
On the first page backslash pagenumbering should be called
(with the appropriate argument out of e.g.
 * roman    - Lowercase Roman numerals: i, ii, iii, iv,...
 * Roman    - Uppercase Roman numerals: I, II, III, IV,...
 * arabic   - Arabic numerals: 1, 2, 3, 4,...
 * alph     - Lowercase letters: a, b, c, d,...
 * Alph     - Uppercase letters: A, B, C, D,...
 * fnsymbol - Footnote symbols: *, † (ddagger), ‡ (dddagger),
                                 § (mathsection),...
). The command backslash pagenumbering is renewed, but no change
should be visible to the output text, because only labels are added.
backslash OrigPagenumbering can be used to get the original meaning
of backslash pagenumbering, but in that case this package does not
see the change in the page numbering scheme and will make mistakes.
The package takes options.

The usual disclaimer applies:
If it doesn't work right that's your problem.
(Nevertheless, send an e-mail to the maintainer
 when you find an error in this package.)

This work may be distributed and/or modified under the
conditions of the LaTeX Project Public License, either
version 1.3c of this license or (at your option) any later
version. This version of this license is in
   http://www.latex-project.org/lppl/lppl-1-3c.txt
and the latest version of this license is in
   http://www.latex-project.org/lppl.txt
and version 1.3c or later is part of all distributions of
LaTeX version 2005/12/01 or later.

This work has the LPPL maintenance status "maintained".

The Current Maintainer of this work is H.-Martin Muench.

This package first started as a revision of the lastpage
package of Jeffrey P. Goldberg (jeffrey+news at goldmark dot org),
but it became obvious that a replacement was needed.

This work consists of the main source file pageslts.dtx,
the README, and the derived files
   pageslts.sty, pageslts.pdf,
   pageslts.ins, pageslts.drv,
   pageslts-example.tex, pageslts-example.pdf.

In memoriam
 Claudia Simone Barth + 1996/01/30
 Tommy Muench + 2014/01/02
 Hans-Klaus Muench + 2014/08/24

\endpreamble
\let\MetaPrefix\DoubleperCent

\generate{%
  \file{pageslts.ins}{\from{pageslts.dtx}{install}}%
  \file{pageslts.drv}{\from{pageslts.dtx}{driver}}%
  \usedir{tex/latex/pageslts}%
  \file{pageslts.sty}{\from{pageslts.dtx}{package}}%
  \usedir{doc/latex/pageslts}%
  \file{pageslts-example.tex}{\from{pageslts.dtx}{example}}%
}

\catcode32=13\relax% active space
\let =\space%
\Msg{************************************************************************}
\Msg{*}
\Msg{* To finish the installation you have to move the following}
\Msg{* file into a directory searched by TeX:}
\Msg{*}
\Msg{*  pageslts.sty}
\Msg{*}
\Msg{* To produce the documentation run the file `pageslts.drv'}
\Msg{* through (pdf)LaTeX, e.g.}
\Msg{*  pdflatex pageslts.drv}
\Msg{*  makeindex -s gind.ist pageslts.idx}
\Msg{*  pdflatex pageslts.drv}
\Msg{*  makeindex -s gind.ist pageslts.idx}
\Msg{*  pdflatex pageslts.drv}
\Msg{*}
\Msg{* At least three runs are necessary e. g. to get the}
\Msg{*  references right!}
\Msg{*}
\Msg{* Happy TeXing!}
\Msg{*}
\Msg{************************************************************************}

\endbatchfile
%</install>
%<*ignore>
\fi
%</ignore>
%
% \section{The documentation driver file}
%
% The next bit of code contains the documentation driver file for
% \TeX , i.\,e., the file that will produce the documentation you
% are currently reading. It will be extracted from this file by the
% \texttt{docstrip} programme. That is, run \LaTeX{} on \texttt{docstrip}
% and specify the \texttt{driver} option when \texttt{docstrip}
% asks for options.
%
%    \begin{macrocode}
%<*driver>
\NeedsTeXFormat{LaTeX2e}[2015/01/01]
\ProvidesFile{pageslts.drv}[2015/12/21 v1.2f
            Refers to special pages' numbers/names (HMM)]
\documentclass[landscape]{ltxdoc}[2015/03/26]% v2.0w
\usepackage{geometry}[2010/09/12]%             v5.6
\usepackage{holtxdoc}[2012/03/21]%             v0.24
%% pageslts may work with earlier versions of LaTeX2e and those
%% class and packages, but this was not tested.
%% Please consider updating your LaTeX, class, and packages
%% to the most recent version (if they are not already the most
%% recent version).
\hypersetup{%
 pdfsubject={Refers to special pages' numbers/names (HMM)},%
 pdfkeywords={LaTeX, pageslts, pagesLTS, lastpage, page number, page name, H.-Martin Muench},%
 pdfencoding=auto,%
 pdflang={en},%
 breaklinks=true,%
 linktoc=all,%
 pdfstartview=FitH,%
 pdfpagelayout=OneColumn,%
 bookmarksnumbered=true,%
 bookmarksopen=true,%
 bookmarksopenlevel=2,%
 pdfmenubar=true,%
 pdftoolbar=true,%
 pdfwindowui=true,%
 pdfnewwindow=true%
}
\usepackage{ulem}[2012/05/18]% no version is given in the ulem.sty file
\CodelineIndex
\hyphenation{created every-thing ignored}
\gdef\unit#1{\mathord{\thinspace\mathrm{#1}}}
\makeatletter
\@ifundefined{eTeX}{\gdef\eTeX{$\m@th \varepsilon $-\TeX }}{% else \relax
}
\makeatother
\begin{document}
  \DocInput{pageslts.dtx}%
\end{document}
%</driver>
%    \end{macrocode}
%
% \fi
%
% \CheckSum{1790}
%
% \CharacterTable
%  {Upper-case    \A\B\C\D\E\F\G\H\I\J\K\L\M\N\O\P\Q\R\S\T\U\V\W\X\Y\Z
%   Lower-case    \a\b\c\d\e\f\g\h\i\j\k\l\m\n\o\p\q\r\s\t\u\v\w\x\y\z
%   Digits        \0\1\2\3\4\5\6\7\8\9
%   Exclamation   \!     Double quote  \"     Hash (number) \#
%   Dollar        \$     Percent       \%     Ampersand     \&
%   Acute accent  \'     Left paren    \(     Right paren   \)
%   Asterisk      \*     Plus          \+     Comma         \,
%   Minus         \-     Point         \.     Solidus       \/
%   Colon         \:     Semicolon     \;     Less than     \<
%   Equals        \=     Greater than  \>     Question mark \?
%   Commercial at \@     Left bracket  \[     Backslash     \\
%   Right bracket \]     Circumflex    \^     Underscore    \_
%   Grave accent  \`     Left brace    \{     Vertical bar  \|
%   Right brace   \}     Tilde         \~}
%
% \GetFileInfo{pageslts.drv}
%
% \begingroup
%   \def\x{\#,\$,\^,\_,\~,\ ,\&,\{,\},\%}%
%   \makeatletter
%   \@onelevel@sanitize\x
% \expandafter\endgroup
% \expandafter\DoNotIndex\expandafter{\x}
% \expandafter\DoNotIndex\expandafter{\string\ }
% \begingroup
%   \makeatletter
%     \lccode`9=32\relax
%     \lowercase{%^^A
%       \edef\x{\noexpand\DoNotIndex{\@backslashchar9}}%^^A
%     }%^^A
%   \expandafter\endgroup\x
%
% \DoNotIndex{\",\-,\,,\\,\noindent}
% \DoNotIndex{\documentclass,\usepackage,\ProvidesPackage}
% \DoNotIndex{\NeedsTeXFormat,\TeX,\LaTeX}
% \DoNotIndex{\ldots,\thinspace,\textbackslash}
% \DoNotIndex{\begin,\end}
% \DoNotIndex{\textbf,\textit,\textquotedblleft,\textquotedblright,\textsc,\textsf,\texttt,\underline}
% \DoNotIndex{\ifx,\ifnum,\ifodd,\enduremath}
% \DoNotIndex{\newpage,\pagebreak,\clearpage,\newline,\linebreak,\nolinebreak}
% \DoNotIndex{\lipsum,\MessageBreak,\pageref,\protect}
% \DoNotIndex{\smallskip,\medskip,\bigskip,\hskip,\space,\hline}
% \DoNotIndex{\advance,\@ne}
% \DoNotIndex{\item}
% \DoNotIndex{\@PackageInfoNoLine,\@bsphack,\@esphack,\@ehc,\@ehd,\@ifundefined,%
%  \addcontentsline,\arabic,\enddocument,\endinput,\href,\makeatletter,\makeatother,%
%  \section,\subsection,\tableofcontents,\verb}
%
% \title{The \xpackage{pageslts} package}
% \date{2015/12/21 v1.2f}
% \author{H.-Martin M\"{u}nch\\\xemail{Martin.Muench at Uni-Bonn.de}}
%
% \maketitle
%
% \begin{abstract}
%  \noindent This \LaTeX{} package puts the labels \texttt{LastPage}
%  (|\AtEndDocument|) and \texttt{VeryLastPage} (|\AfterLastShipout|)
%  into the \xfile{.aux} file, allowing the user to refer
%  to the (very) last page of a document. This might be
%  particularly useful in places like headers or footers.
%  When more than one page numbering scheme is used,
%  these references do not give the total \textit{number} of pages.
%  For this case the label \texttt{LastPages} is introduced.
%  Additionally, at the last page of each page numbering
%  scheme a label \texttt{pagesLTS.<numbering scheme>} is
%  placed, where \texttt{<numbering scheme>} is e.\,g.
%  arabic, roman, Roman, alph, or Alph. For fnsymbol
%  please use |\lastpageref{pagesLTS.fnsymbol}|
%  instead of |\pageref{pagesLTS.fnsymbol}|.
%  When the same numbering scheme is used twice,
%  the page numbers are either reset to one or continued
%  automatically, depending on the option given when the package
%  is called. The command |\theCurrentPage|
%  prints the current total/absolute page number -- in contrast to
%  |\thepage|, which gives only the page
%  \textit{name} in the current page numbering scheme.
%  |\theCurrentPageLocal| gives the current
%  number of pages in the current page numbering scheme.
%  |\thepage| and |\theCurrentPageLocal| are different e.\,g.
%  when |\addtocounter{page}{|\ldots|}| or
%  |\setcounter{page}{|\ldots|}| were used.
%  At the first page of the document a label \texttt{pagesLTS.0} is
%  created. This label can be referred to, too.
%  Further labels are provided for special cases.\\
%  The \xpackage{alphalph} package is supported, i.\,e.
%  page numbers alph or Alph $>26$ and fnsymbol $>9$
%  can be used (with according options set). Even zero
%  and negative page numbers can be used with \texttt{arabic},
%  \texttt{alph}, \texttt{Alph}, \texttt{roman}, \texttt{Roman},
%  and \texttt{fnsymbol} page numbering (with \xpackage{alphalph} package
%  and according options).\\
%  |\pageref*| and |\lastpageref*|, for using \xpackage{hyperref}
%  but suppressing links, are supported.
% \end{abstract}
%
% \noindent \textbf{Please make sure to first deinstall the obsolete
% \xpackage{pagesLTS} package before installing this \xpackage{pageslts}
% package!}\\
% (There is at least one operating system which otherwise automatically
% renames \xfile{pageslts} to \xfile{pagesLTS}.)
%
% \noindent Right after |\begin{document}| a |\pagenumbering{|\ldots|}| should
% be called -- with the appropriate argument out of e.\,g.\\
% arabic (Arabic numerals: 1, 2, 3, 4,\ldots),\\
% roman (Lowercase Roman numerals: i, ii, iii, iv,\ldots),
% Roman (Uppercase Roman numerals: I, II, III, IV,\ldots),\\
% alph (Lowercase letters: a, b, c, d,\ldots),
% Alph (Uppercase letters: A, B, C, D,\ldots),\\
% fnsymbol (Footnote symbols: $*$, $\dagger $, $\ddagger $, $\mathsection $,\ldots).
%
% \noindent This package first started as a revision of the
% \xpackage{lastpage} package of \textbf{Jeffrey P. Goldberg} (Thanks!),
% but then it became obvious that a replacement was needed.
%
% \noindent Disclaimer for web links: The author is not responsible for any contents
% referred to in this work unless he has full knowledge of illegal contents.
% If any damage occurs by the use of information presented there, only the
% author of the respective pages might be liable, not the one who has referred
% to these pages.
%
% \noindent Save per page about $200\unit{ml}$ water,
% $2\unit{g}$ CO$_{2}$ and $2\unit{g}$ wood:
% Therefore please print only if this is really necessary.
%
% \bigskip
%
% \tableofcontents
%
% \newpage
%
% \section{Introduction}
%
% \indent This package puts the labels \texttt{LastPage}
% (|\AtEndDocument|) (same as my \xpackage{LastPage} package, invented by
% \textsc{Jeffrey P. Goldberg}) and \texttt{VeryLastPage} (|\AfterLastShipout|)
% into the \xfile{.aux} file, allowing the user to refer to the (very)
% last page of a document via |\lastpageref{LastPage}| and
% |\lastpageref{VeryLastPage}|. This might be particularly useful in
% places like headers or footers. When more than one page numbering scheme
% is used, these references do not give the total \textit{number} of pages.
% For this case the label \texttt{LastPages} is introduced (similar to the
% label \texttt{TotPages} of the \xpackage{TotPages} package, but
% the label \texttt{LastPages} is set later in the document).
% Additionally, at the last page of each page numbering scheme a label
% \texttt{pagesLTS.<}\textit{numbering scheme}\texttt{>} is placed,
% where \texttt{<}\textit{numbering scheme}\texttt{>} is e.\,g.
% arabic, roman, Roman, alph, or Alph. For fnsymbol please use
% |\lastpageref{pagesLTS.fnsymbol}| instead of |\pageref{pagesLTS.fnsymbol}|.
% When the same numbering scheme is used twice, the page numbers are
% either reset to one or continued automatically, depending on the option
% given when the package is called. The command |\theCurrentPage| prints
% the current total/absolute page number - in contrast to |\thepage|,
% which gives only the page \textit{name} in the current page numbering
% scheme. |\theCurrentPageLocal| gives the current number of pages
% in the current page numbering scheme. |\thepage| and |\theCurrentPageLocal|
% are different e.\,g. when |\addtocounter{page}{|\ldots|}| or
% |\setcounter{page}{|\ldots|}| were used. (See also \LaTeX\ bug~3421:
% 3rd~page is even (twoside, titlepage, abstract),
% \url{http://www.latex-project.org/cgi-bin/ltxbugs2html?category=LaTeX&responsible=anyone&state=anything&keyword=pagenumber&pr=latex%2F3421&search=}.)
% At the first page of the document a label \texttt{pagesLTS.0} is
% created. This label can be referred to, too. Further labels are
% provided for special cases.\\
% The \xpackage{alphalph} package is supported, i.\,e. page numbers
% \texttt{alph} or \texttt{Alph} $>26$ and \texttt{fnsymbol} $>9$ can
% be used (with the according options set). Even zero or negative page
% numbers can be used with \texttt{arabic}, \texttt{alph}, \texttt{Alph}, and
% \texttt{fnsymbol} page numbering (with \xpackage{alphalph} package
% and according options), and zero \texttt{roman} and \texttt{Roman} pages,
% too.
%
% \bigskip
% \noindent Right after |\begin{document}| a |\pagenumbering{|\ldots|}|
% should be called -- with the appropriate argument out of e.\,g.\\
% arabic (Arabic numerals: 1, 2, 3, 4,\ldots),\\
% roman (Lowercase Roman numerals: i, ii, iii, iv,\ldots),
% Roman (Uppercase Roman numerals: I, II, III, IV,\ldots),\\
% alph (Lowercase letters: a, b, c, d,\ldots),
% Alph (Uppercase letters: A, B, C, D,\ldots),\\
% fnsymbol (Footnote symbols: $*$, $\dagger $, $\ddagger $, $\mathsection $,\ldots).
%
% \bigskip
% \noindent This package first started as a revision of the
% \xpackage{lastpage} package of \textsc{Jeffrey P. Goldberg} (Thanks!),
% but then it became obvious that a replacement was needed to accomplish
% what this package does.
%
% \bigskip
% \noindent \textbf{Trademarks} appear throughout this documentation
% without any trademark symbol; they are the property of their respective
% trademark owner. There is no intention of infringement; the usage is
% to the benefit of the trademark owner.
%
% \bigskip
% \phantomsection
% \DescribeMacro{logical page numbers}
% \noindent \textbf{Tip}: For the display of the \xfile{pdf}~file use
% \textbf{logical page numbers} together with \xpackage{hyperref}!
% \label{logical page numbers}
% \begin{description}
% \item [-] In Adobe Reader DC 2015.008.20082 enable:\\
% Edit $>$ Preferences $>$ Categories: Page Display $>$
% Page Content and Information: Use logical page numbers
%
% \item[-] Use the \xpackage{hyperref} package with option |plainpages=false|\,.
% \end{description}
% The display will be e.\,g.\ \textquotedblleft 7~(7~of~9)\textquotedblright,
% or, in case of Roman instead of arabic numbers,
% \textquotedblleft VII~(7~of~9)\textquotedblright, and when different
% page numbers are used (see below) e.\,g. arabic after 10~Roman pages:
% \textquotedblleft 17~(27~of~30)\textquotedblright.
% Please try this with the compiled \xpackage{pageslts-example} file!
%
% \bigskip
% \noindent The name of the \xpackage{pageslts} package refers to
% Last, Total, and page numbering Schemes pages. \xpackage{pagesLTS}
% was a former name of this package.
%
% \pagebreak
%
% \section{Usage}
%
% \indent Just load the package placing
% \begin{quote}
%   |\usepackage[<|\textit{options}|>]{pageslts}|
% \end{quote}
% \noindent in the preamble of your \LaTeXe\ source file
% (about |\AtEndDocument| see subsection~\ref{ssec:aed})
% and place a |\pagenumbering{...}| with appropriate argument
% (e.\,g.~arabic, roman, Roman, fnsymbol, alph, or Alph)
% right behind |\begin{document}| (see subsubsection~\ref{sss:nopnc})!\\
%
% \indent For example for various draft forms it is desirable to have a
% page reference to the last page, so that e.\,g. page footers can
% contain something like \textquotedblleft page $N$ of $K$\textquotedblright,
% where $N$ is the current page and $K$ is the last page. Once the package
% is loaded, anywhere in the text references can be made to the labels
% \texttt{LastPage}, \texttt{VeryLastPage}, and \texttt{LastPages}
% (most times with |\pageref{...}|, but more save with |\lastpageref{...}|).
% In particular one can use the \xpackage{fancyhdr} or \xpackage{nccfancyhdr}
% package, or redefinitions of the page headings and footings to get
% a reference to the (very) last page.\\
% \DescribeMacro{\pageref*}
% \DescribeMacro{\lastpageref*}
% If the \xpackage{hyperref} package is used, the references are hyperlinked
% to their aims. If these hyperlinks shall be suppressed, |\pageref*{...}|
% and |\lastpageref*{...}| can be used.
%
% \subsection{Options}
% \DescribeMacro{options}
% \indent The \xpackage{pageslts} package takes the following options:
%
% \subsubsection{pagecontinue\label{sss:pagecontinue}}
% \DescribeMacro{pagecontinue}
% \indent When option \texttt{pagecontinue=false} is \textbf{not} given
% (i.\,e. \texttt{pagecontinue} or \texttt{pagecontinue=true}
% or no \texttt{pagecontinue} option at all), at each
% |\pagenumbering{...}| command the number of the page will be continued
% with the page number following the last page of the same page numbering
% scheme. For example, if there are V~Roman pages in the frontmatter,
% some arabic ones in the mainmatter, and then Roman ones again in the
% backmatter, the last ones will start with VI instead of~I again.\\
% \indent If you want to start with I (or i, 1, a, A, *,\ldots ) again,
% set option \texttt{pagecontinue=false}. If you want to generally continue
% the numbers, but for some page numbering scheme do not want this,
% use \texttt{pagecontinue=true} and say |\setcounter{page}{1}|
% after |\pagenumbering{...}| for that page numbering scheme.
%
% \subsubsection{alphMult, AlphMulti, fnsymbolmult\label{sss:alphalph}}
%
% \indent The page number printed in
% \nolinebreak{\texttt{fnsymbol}\footnote{%
% \ensuremath {*},%
% \ensuremath {\dagger },%
% \ensuremath {\ddagger },%
% \ensuremath {\mathsection },%
% \ensuremath {\mathparagraph },%
% \ensuremath {\delimiter "026B30D },%
% \ensuremath {**},%
% \ensuremath {\dagger \dagger },%
% \ensuremath {\ddagger \ddagger }}}
% must be $>0$ and $<10$ and those printed in
% \nolinebreak{\texttt{alph}\footnote{a, b, c, d, e, f, g, h,%
% i, j, k, l, m, n, o, p, q, r, s, t, u, v, w, x, y, z}} and
% \nolinebreak{\texttt{Alph}\footnote{A, B, C, D, E, F, G, H,%
% I, J, K, L, M, N, O, P, Q, R, S, T, U, V, W, X, Y, Z}} must
% be $>0$ and $<27$. After page~Z\quad \LaTeX\ \textit{should}
% continue with AA, AB, AC,\ldots\ Some people prefer
% AA, BB, CC,\ldots, but in hexadecimal it is $AA_{16}=170_{10}$
% and $171_{10}=AB_{16}$, whereas $BB_{16}=187_{10}$. In any way
% it should continue at all (maybe even with an user option to
% choose between the two continuations), but instead only gives
% an error:
% \begin{quote}
%   |LaTeX Error: Counter too large|\\
%   |See the LaTeX manual or LaTeX Companion for explanation.|\\
%   |You've lost some text.  Try typing  <return>  to proceed.|\\
%   |If that doesn't work, type  X <return>  to quit.|
% \end{quote}
% But thanks to the \xpackage{alphalph} package by \textsc{Heiko Oberdiek}
% these limitation no longer hold. With his |\erroralph| command
% now even negative or zero page
% \textquotedblleft numbers\textquotedblright{} are possible.\\
%
% \DescribeMacro{alphMult}
% The string option \texttt{alphMult} takes three values: |ab|, |bb|, |0|:
% \begin{description}
% \item[ab] After page z, the page
%  \textquotedblleft numbers\textquotedblright{} continue with
%  aa, ab, ac, ad,\ldots, fxshrxw (the default), and before a with
%  0, -a, -b,\ldots, -z, -aa, -ab,\ldots, -fxshrxw ($=-21\,474\,836\,47$).
% \item[bb] After page z, the page
%  \textquotedblleft numbers\textquotedblright{} continue with
%  aa, bb, cc, dd,\ldots, and before a with
%  0, -a, -b,\ldots, -z, -aa, -bb,\ldots\\
%  (Internally up~to $\pm 55\,834\,558$ is allowed, but when printed will
%  exceed the \LaTeX\ capacity even for smaller numbers --
%  in the example file this happens at about $6\,500$.)\\
%  (If you have a document with more than $6\,500$~pages, you might
%   think about splitting it in volumes. And page
%   \textquotedblleft numbers\textquotedblright{} with about 100~digits
%   are probably not easy to grasp for the reader, too.)\\
% \item[0] (zero) The \xpackage{pageslts} package does nothing,
%  thus the user is free to define the page
%  \textquotedblleft numbers\textquotedblright{} after~z and before~a.\\
%  (But if the user does not do anything at all, the\\
%  \ |LaTeX Error: Counter too large|\\
%  will appear again.)
% \end{description}
%
% \DescribeMacro{AlphMulti}
% The string option \texttt{AlphMulti} takes three values: |AB|, |BB|, |0|:
% \begin{description}
% \item[AB] After page Z, the page
%  \textquotedblleft numbers\textquotedblright{} continue with
%  AA, AB, AC, AD,\ldots, FXSHRXW (the default), and before A with
%  0, -A, -B,\ldots, -Z, -AA, -AB,\ldots, -FXSHRXW.
% \item[BB] After page Z, the page
%  \textquotedblleft numbers\textquotedblright{} continue with
%  AA, BB, CC, DD,\ldots, and before A with
%  0, -A, -B,\ldots, -Z, -AA, -BB,\ldots \\
%  (About the limits please see |alphMult| above.)
% \item[0] (zero) The \xpackage{pageslts} package does nothing,
%  thus the user is free to define the page
%  \textquotedblleft numbers\textquotedblright{} after~Z and before~A.\\
%  (But if the user does not do anything at all, the\\
%  \ |LaTeX Error: Counter too large|\\
%  will appear again.)
% \end{description}
%
% \DescribeMacro{fnsymbolmult}
% When option \texttt{fnsymbolmult=false} is \textbf{not} given
% (i.\,e. \texttt{fnsymbolmult} or \texttt{fnsymbolmult=true}
% or no \texttt{fnsymbolmult} option at all),
% after 5 (\ensuremath {\mathparagraph }) the page
% \textquotedblleft number\textquotedblright{} is continued with
% the doubled \textquotedblleft number\textquotedblright{} of the
% first, second, third,\ldots\ page (\ensuremath {**},
% \ensuremath {\dagger \dagger }, \ensuremath {\ddagger \ddagger },
% \ensuremath {\mathsection \mathsection },
% \ensuremath {\mathparagraph \mathparagraph }),
% and after the tenth page the
% \textquotedblleft number\textquotedblright{} is tripled
% (\ensuremath {***}, \ensuremath {\dagger \dagger \dagger },\ldots).
% Compile the \textsf{pageslts-example.tex} with \pdfLaTeX\ and
% see the resulting \xfile{pdf} file.\\
% Before \ensuremath {*} (page~1) the page
% \textquotedblleft numbers\textquotedblright{} are continued with
% 0, \ensuremath {- * }, \ensuremath {- \dagger },\ldots,
% \ensuremath {- \mathparagraph }, \ensuremath {- **},
% \ensuremath {- \dagger \dagger },\ldots \\
% If this is not wanted, set option \texttt{fnsymbolmult=false},
% and \xpackage{pageslts} will do nothing and allow the user to change
% the page \textquotedblleft number\textquotedblright . (But if the
% user does not do anything at all, the\\
% \ |LaTeX Error: Counter too large|\\
% will appear again.)\\
% While in \LaTeXe\ arabic (page) numbers are possible up~to
% |MAX|\,$ = 2\,147\,483\,647$ (cf.~the \xpackage{alphalph} package),\\
% |\erroralphalph{\fnsymbolmult}{...}| numbers are possible up~to
% 10\,737\,415 only. If this number is not only used internally
% but printed, after number about 11\,705 (which is
% 2\,341 times \ensuremath {\mathparagraph }) the \LaTeXe\ capacity
% is exceeded, depending on the remaining file and its use of
% \TeX\ capacity, of course. (If you have a document with more than
% 11\,705~pages, you might think about splitting it in volumes.
% And page \textquotedblleft numbers\textquotedblright{} with
% 2\,341~digits are probably not easy to grasp for the reader, too.)\\
%
% \subsubsection{romanMult, RomanMulti\label{sss:romanMult}}
%
% \DescribeMacro{romanMult}
% \DescribeMacro{RomanMulti}
% The options \texttt{romanMult(=true)} and \texttt{RomanMulti(=true)}
% expand the |\roman| and |\Roman| page numbering scheme to values
% below one ($<1$), i.\,e. 0, -i, -ii, -iii, -iv,\ldots\ and
% 0, -I, -II, -III, -IV,\ldots, respectively.\\
% Again the \TeX\ capacity will be exceeded before
% \nolinebreak{$\pm$ |MAX|\,}$ = \pm 2\,147\,483\,647$, and even if
% $1\,000\,000\,000$ is internally possible, this would print
% $1\,000\,000$ times the letter~\textsf{m} (or~\textsf{M}),
% which would require either very small print or quite huge paper size.\\
% (If you have a document with so many pages, you might think
% about splitting it in volumes. And page
% \textquotedblleft numbers\textquotedblright{} with
% thousands of digits are probably not easy to grasp for the reader,
% too.)\\
% If the expansion below $1$ is not wanted, set options
% \texttt{romanMult=false} and/or \texttt{RomanMulti=false},
% and \xpackage{pageslts} will do nothing and allow the user
% to change the page \textquotedblleft number\textquotedblright .
% (But if the user does not do anything at all, \LaTeX\ will just
% ignore the values - not even a warning will be issued.)
%
% \subsubsection{Arabic page numbers}
%
% \DescribeMacro{Arabic page numbers}
% In \LaTeXe\ arabic (page) numbers are already possible between
% |-MAX...MAX|, where |MAX|\,$ = 2\,147\,483\,647$ (cf.~the
% \xpackage{alphalph} package), without any expansion necessary.
% (But if you have a document with so many pages, you might think
% about splitting it in volumes!)
%
% \subsection{Labels}
%
% \DescribeMacro{pagesLTS.0}
% \indent At the first page a label \texttt{pagesLTS.0} is created.
% If |\pagenumbering{...}| is used right after
% |\begin{document}|, this is much easier for the
% \xpackage{pageslts} package (and chances for successful placing
% of all labels are much higher;
% \nolinebreak{cf.~subsubsection~\ref{sss:once}).}\\
%
% \DescribeMacro{LastPage}
% |\AtEndDocument| (see subsection~\ref{ssec:aed}) this package defines
% a label, \texttt{LastPage}, which the user can refer to with the\linebreak
% |\lastpageref{LastPage}| command. While |\pageref{LastPage}| is also
% possible (especially for backward compatibility with the \xpackage{LastPage}
% package), this is discouraged, because it will not work when it is used
% together with the \xpackage{hyperref} package and the \xpackage{fnsymbol}
% page numbering scheme. (The \xpackage{LastPage} package did not work with
% this combination, too, so if you want to, you can reproduce the old error
% -- but you do not have to do it, but can use |\lastpageref{LastPage}|.)\\
%
% \DescribeMacro{VeryLastPage}
% |\AfterLastShipout| the label \texttt{VeryLastPage} is defined, which
% the user can also refer to with the |\lastpageref{VeryLastPage}| command.
% Depending on usage of |\AtEndDocument| by other packages, \texttt{LastPage}
% might not point to the very last page, but |\lastpageref{VeryLastPage}|
% should do this (cf.~subsection~\ref{ssec:aed}).\\
%
% \DescribeMacro{LastPages}
% \DescribeMacro{page number}
% \DescribeMacro{number of pages}
% When more than one page numbering scheme is used, neither \texttt{LastPage}
% nor \texttt{VeryLastPage} give the total \textbf{number} of pages.
% For example, for a document with VI+36 pages, both give
% \textquotedblleft 36\textquotedblright{} as reference to the last page.
% While this is correct, the total number of pages is 42, and this is given
% by the reference to \texttt{LastPage\underline{s}}:
% \nolinebreak{|\lastpageref{LastPage|\underline{|s|}|}|} (note the
% \textquotedblleft s\textquotedblright{} at the end). When the
% page number was manipulated by |\addtocounter{page}{...}|
% or |\setcounter{page}{...}|, \texttt{LastPages} ignores this.
% (At a page numbering change the page is reset to one (without option
% |pagecontinue|). This is done by |\setcounter{page}{1}|, thus this is
% ignored, too.)\\
% |\pageref{totpages}| of the \xpackage{totpages} package is similar to
% |\lastpageref{LastPages}|, but while the target for \linebreak[4]
% |\pageref{totpages}| is placed |\AtEndDocument|, the target for
% |\lastpageref{LastPages}| is placed |\AfterLastShipout|, therefore
% |\lastpageref{LastPages}| is safer to really get the total page number.\\
%
% \pagebreak
%
% \DescribeMacro{\theCurrentPage}
% |\theCurrentPage| gives the current total/absolute page, in contrast to
% |\thepage|, which gives only the page \textit{name} in the current page
% numbering scheme. For example, when there are Roman VII pages in the
% frontmatter and afterwards in the mainmatter you are at arabic page~9,
% then |\theCurrentPage| is~16, whereas |\thepage| is~9. When the page
% \textquotedblleft number\textquotedblright{} (name) is manipulated by
% |\addtocounter{page}{...}| or |\setcounter{page}{...}|, |\theCurrentPage|
% \nolinebreak{ignores} this. Because \texttt{CurrentPage} is a normal
% counter, you can also say e.\,g.~|\Roman{CurrentPage}| to get the value
% in Roman page numbering scheme (e.\,g.~VIII for~8).\\
%
% \phantomsection
% \label{CurrentPageLocal}
% \DescribeMacro{\theCurrentPageLocal}
% |\theCurrentPageLocal| gives the current (arabic) number of pages in the
% current page numbering scheme. |\thepage| and |\theCurrentPageLocal| are
% different e.\,g. when |\addtocounter{page}{...}| or
% |\setcounter{page}{...}| were used.\\
% |\theCurrentPageLocal| can be printed in other formats, e.\,g.
% |\roman{pagesLTS.current.local.roman}|, but probably it only makes
% sense if page numbering scheme and format are the same, e.\,g.
% |\Roman{pagesLTS.current.local.Roman}| or\\
% |\Alph{pagesLTS.current.local.Alph}|.\\
% |\arabic{pagesLTS.current.local.|\ldots|}| probably make sense even
% when combined with another page numbering scheme. And this is exactly
% what |\theCurrentPageLocal| does:\\
% |\def\theCurrentPageLocal{\arabic{pagesLTS.current.local.\pagesLTS@pnc}}|.\\
%
% \DescribeMacro{pagesLTS. page numbering scheme . number }
% If you want to refer to the last page of the first, second,\ldots\ use
% of a page numbering scheme, you can refer to \linebreak
% |pagesLTS.<page numbering scheme>.<number>|, e.\,g.
% |\lastpageref{pagesLTS.Roman.1}|, where |<number>| is the occurrence
% of the page numbering scheme. For details please see
% page~\pageref{pagesLTS.pnscheme.nr}.\\
%
% \bigskip
% \DescribeMacro{\lastpageref}
% For pages with the \texttt{fnsymbol} page numbering scheme,
% |\lastpageref{...}| \nolinebreak{instead} of |\pageref{...}|
% \textit{must} be used. This is required for pages some\-where
% inside of the document as well as the (very) last page(s).
% Because |\lastpageref{...}| is a synonym for |\pageref{...}|,
% where no \xpackage{fnsymbol} page numbering scheme is used,
% it is save(r) to use it for all references to labels provided
% by the \xpackage{pageslts} package.
%
% \subsection[\texttt{\textbackslash pagenumbering\{\ldots \}}]{%
% \texttt{\textbackslash pagenumbering\{\ldots \}}}
% \DescribeMacro{\pagenumbering}
% \subsubsection{If \texttt{\textbackslash pagenumbering\{\ldots \}} is \textbf{not} used\label{sss:nopnc}}
%
% \indent When the \xpackage{pageslts} package is used, but |\pagenumbering{...}|
% (with an argument like arabic, roman, Roman, fnsymbol, alph, or Alph)
% is not used, there should be no problem, except that you might need
% more~(!) compiler runs to get all references right, and some references
% might even be missing (see below). The \xpackage{pageslts}
% package tries to determine the page numbering scheme at the first
% shipout, but success is not guaranteed. Thus please use |\pagenumbering{...}|
% at the beginning of your document!\\
% \indent Without |\pagenumbering{<something>}| (|<something>| e.\,g. = arabic)
% at the beginning of the document, the page numbers might be given in
% arabic \textit{by} (class) \textit{default}, but the \xpackage{pageslts}
% package does not know about this without |\pagenumbering{arabic}|.~--\\
% \indent The label \texttt{pagesLTS.0} is created at the first
% page even if no |\pagenumbering{...}| command is given.
% Maybe have a look at the \xfile{.aux} file after compiling your
% document to detect further labels (of other packages, too).\\
%
% \pagebreak
%
% \subsubsection{If \texttt{\textbackslash pagenumbering\{\ldots \}} is used once\label{sss:once}}
%
% \DescribeMacro{pagesLTS.0}
% At the first page a label \texttt{pagesLTS.0} is created.
% If |\pagenumbering{...}| is used right after |\begin{document}|,
% this is much easier for the \xpackage{pageslts} package
% (and chances for successful placing of all labels are much higher).
%
% \subsubsection{If \texttt{\textbackslash pagenumbering\{\ldots \}} is used more than once}
%
% \indent Everything from the preceding subsubsections applies and additionally
% the following:\\
% \indent When different page numbering schemes are used, e.\,g.
% Roman numbers for the frontmatter and arabic numbers for the
% mainmatter, please use |\pagenumbering{...}| for each of them!
% Even if you do this, the reference to neither the label \texttt{LastPage}
% nor the label \texttt{VeryLastPage} gives the \textbf{total}
% number of pages, but only the number of pages of the last used page
% numbering scheme (which could be exactly what you want, e.\,g. if
% you want to refer to the last page itself and do not want to
% give the total number of pages).\\
% \DescribeMacro{LastPages}
% \indent For remediation the label
% \nolinebreak{\texttt{LastPage}\textbf{\underline{s}}}
% (with \textquotedblleft s\textquotedblright{} at its end) is
% introduced. Please then refer to this label by \newline
% |\lastpageref{LastPages}| instead of \texttt{LastPage} or
% \texttt{VeryLastPage}.\\
%
% \DescribeMacro{pagesLTS.arabic}
% \DescribeMacro{pagesLTS.roman}
% \DescribeMacro{pagesLTS.Roman}
% \DescribeMacro{pagesLTS.alph}
% \DescribeMacro{pagesLTS.Alph}
% \DescribeMacro{pagesLTS.fnsymbol}
% Additionally, at the last page of each page numbering scheme a label
% \texttt{pagesLTS.<numbering scheme>} is placed, where
% \texttt{<numbering scheme>} is e.\,g.\ arabic, roman, Roman, alph,
% Alph,\ldots .\\
% For the \textbf{fnsymbol} page numbering scheme
% |\lastpageref{pagesLTS.fnsymbol}| is needed instead of \newline
% |\pageref{pagesLTS.fnsymbol}|. You can and should use
% |\lastpageref{...}| also for the other page numbering schemes.\\
%
% \bigskip
%
% While at the time of the last revision of the \xpackage{pageslts} package
% no other page numbering schemes were known to the maintainer,
% this package in principle works with every scheme which is
% recognized by the original |\pagenumbering| command.
% But the \xpackage{hyperref} package only then works with crazy page names,
% if the references to those pages are given in a certain way, thus the
% combination of a new page numbering scheme, the \xpackage{hyperref}
% and the \xpackage{pageslts} package might not work.~--
% The \xpackage{pageslts} package by itself also works with schemes,
% which the original |\pagenumbering{...}| does not recognize, but because
% the original |\pagenumbering{...}| is called by the \xpackage{pageslts}
% package, this might cause an error, see subsection~\ref{sec:unknown}!
% (And if the number format is unknown to \LaTeX{}, the pages will have
% no number, and therefore cannot be referenced. You might be able to
% help yourself by using the \xpackage{hyperref} package and
% manually placing |\hypertarget|s and |\href|s.)
%
% \subsubsection{If the same \texttt{\textbackslash pagenumbering\{\ldots \}}\ %
%  scheme is used more than once\label{sss:twice}}
%
% \indent Everything from the preceding subsubsections applies and additionally
% the following:\\
% \DescribeMacro{pagecontinue}
% \indent If the same page numbering scheme is used twice (or even more often)
% in one document (e.\,g. in the frontmatter Roman: I--V, in the
% mainmatter arabic: 1--20, and in the backmatter again Roman: VI--X),
% the second time it is used, the page numbering is either continued (option
% \texttt{pagecontinue} or \texttt{pagecontinue=true} or no option
% \texttt{pagecontinue}; the default) or reset to one (option
% \texttt{pagecontinue=false}). It is even possible to use
% a~page numbering scheme more than twice.\\
%
% \phantomsection
% \label{pagesLTS.pnscheme.nr}
% \DescribeMacro{pagesLTS. page numbering scheme . number }
% If you want to refer to the last page of the first, second,\ldots\ use
% of a page numbering scheme, page~V in the example above, you can refer to
% |pagesLTS.<page numbering scheme>.<number>|, e.\,g.
% |\lastpageref{pagesLTS.Roman.1}|, where |<number>| is the occurrence
% of the page numbering scheme.\\
%
% \pagebreak
%
% If you want to refer to the first page of a page numbering scheme,
% just place a label there, e.\,g.\\
% \indent |\pagenumbering{Roman}| \\
% \indent |\section{Section title\label{RomanSection}}| \\
% (You know where you use |\pagenumbering{...}| and this is the
%  \xpackage{pageslts} package, not the \xpackage{firstpage} one).\\
%
% When you want to give the number of pages of each
% \textquotedblleft sector\textquotedblright{} of the page numbering
% scheme, you can use\\
% \DescribeMacro{pagesLTS. page numbering scheme . number .local.cnt}
% |\lastpages{<page numbering scheme>}{<number>}|, \\
% where |<page numbering scheme>| is e.\,g. Roman, arabic,\ldots\ and
% |<number>| the \textquotedblleft sector\textquotedblright{} number,
% e.\,g.~|\lastpages{Roman}{2}|.\\
% (Internally, the counter has the format
% |pagesLTS.<page numbering scheme>.<number>.local.cnt|.)
%
% \noindent If you used the page numbering scheme Roman for three times,
% you could say
% \begin{verbatim}
% Last Roman page (pagesLTS.Roman): \lastpageref{pagesLTS.Roman}\\
% There are \lastpageref{pagesLTS.Roman.local}~pages with Roman numbers:\\
% \lastpages{Roman}{1}~pages in the first Roman sector
% (\pageref{Roman} -- \lastpageref{pagesLTS.Roman.1}),\\
% \lastpages{Roman}{2}~pages in the second Roman sector
% (\pageref{Roman2} -- \lastpageref{pagesLTS.Roman.2}), and\\
% \lastpages{Roman}{3}~pages in the third Roman sector
% (\pageref{Roman3} -- \lastpageref{pagesLTS.Roman.3}.\\
% \end{verbatim}
% to get
% \begin{quote}
% Last Roman page (pagesLTS.Roman): VIII\\
% There are 8 pages with Roman numbers:\\
% 3 pages in the first Roman sector (I -- III),\\
% 4 pages in the second Roman sector (IV -- VII), and\\
% 3 pages in the third Roman sector (VIII -- X).\\
% \end{quote}
% (see e.\,g. the compiled \textsf{pageslts-example} file).\\
%
% If you want to continue one page numbering scheme, but later on
% (third use of it, or for another page numbering scheme) want to
% reset the page number, just say |\setcounter{page}{1}| there.\\
%
% \pagebreak
%
% \noindent In your document the code
% \begin{verbatim}
% \makeatletter
% \renewcommand{\@evenfoot}%
%  {\normalsize\slshape DRAFT \today\hfil \upshape page {\thepage} (\theCurrentPage) of\ %
%   \lastpageref{pagesLTS.Roman} + \lastpageref{pagesLTS.arabic}\ = \lastpageref{LastPages} pages%
%  }
% \renewcommand{\@oddfoot}{\@evenfoot}
% \makeatother
%\end{verbatim}
% \noindent creates footers like\\
%
% \textquotedblleft\mbox{\textsl{DRAFT \today}\hspace{1cm}page V (5) of VII + 35 = 42 pages}\textquotedblright\\
%
% \noindent or\\
%
% \textquotedblleft\mbox{\textsl{DRAFT \today}\hspace{1cm}page 10 (17) of VII + 35 = 42 pages}\textquotedblright\\
%
% \noindent in the compiled document (cf.~the \texttt{pageslts-example} file).
%
% \noindent Code like
% \begin{verbatim}
%  This book has \lastpageref{pagesLTS.Roman}+\lastpageref{pagesLTS.arabic} pages %
%  (\lastpageref{LastPages} pages in total).
% \end{verbatim}
% \noindent produces output like
% \begin{quote}
% This book has X+85 pages (95 pages in total).
% \end{quote}
% \noindent (when using the \xpackage{hyperref} package,
% the references are even hyperlinked).\\
%
% If |\addtocounter{page}{...}| or |\setcounter{page}{...}| have been used,
% the local version of \texttt{CurrentPage} can be used, |\theCurrentPageLocal|,
% see subsection~\ref{CurrentPageLocal}.\\
%
% \subsection{\xpackage{papermas}(s) package\label{ss:papermas}}
% There is a kind of an add-on to this package, the \xpackage{papermas}
% package, which can be used to compute the number of sheets of paper
% needed to print a document (you can print more than one page of a
% document on one sheet of paper) as well as the approximate mass
% of the printout. Please see the \ref{ss:Downloads} subsection.
%
% \newpage
%
% \section{A few warnings\label{sec:warn}}
%
% \subsection{Hyperref and repeated page numbers}
%
% When two (or more) different page numbering schemes are used, or the page number
% is reset, or for any other reason there are two pages with the same number
% (maybe in different format, e.\,g. $1$ and \textsc{I}), and \xpackage{hyperref} has
% not been configured right, this can cause problems. Use \xpackage{hyperref} with
% \texttt{plainpages=false} and \texttt{pdfpagelabels=true}, and everything should
% be fine.\\
% More details can be found at
% \ \url{http://www.tex.ac.uk/cgi-bin/texfaq2html?label=pdfpagelabels}.
%
% \subsection{\texttt{\textbackslash AtEndDocument}\label{ssec:aed}}
%
% \indent The output of a \LaTeXe\ run is not independent of the order
% in which packages are loaded. It is often the case that the same
% formats for which one must put tables and figure at the end,
% are the ones in which endnotes are also required.
% If one wants to use |\AtEndDocument| here as well (as done for
% |\lastpageref{LastPage}|), then it is easy to get to three separate
% uses of |\AtEndDocument| (assuming one uses this for the endnotes
% as well). Clearly it is not safe for any package writer or user
% to assume that no material will follow what they put into
% |\AtEndDocument|. Therefore a message, which begins with
% \texttt{AED}, is included in every usage of |\AtEndDocument|,
% and it is tried to minimize any side effects the usage may have.\\
%
% As now \textsc{Heiko Oberdiek}'s \texttt{atveryend} package is used,
% the references |\lastpageref{VeryLastPage}| and \newline
% |\lastpageref{LastPages}| should work all right. About how to get the
% \texttt{atveryend} package, please see subsection~\ref{ss:Downloads}.
%
% \subsection{Interaction with very old versions of the \xpackage{endfloat} %
%  package\label{sec:endfloat}}
%
% \indent The \emph{very} old version~2.0 (and earlier) of the \xpackage{endfloat}
% package actually redefined the |\enddocument| command,
% and so interfered drastically with the \LaTeXe\ commands which make use
% of |\AtEndDocument|. Newer versions of \xpackage{endfloat} exist
% (at the time of writing this documentation: v2.5d as of 2011/12/25)
% in modern documentation form, which should be available from
% the same source where you received this file, see subsection~\ref{ss:Downloads}.\\
% A note is placed in the style file at the |\RequirePackage| section,
% and later it is even checked whether a (very) old \xpackage{endfloat}
% package is in use. If it is, a warning or even an error message
% is given, depending on \xpackage{endfloat} version.
% This assumes, that the old versions of \xpackage{endfloat} at least gave
% a version date, of course.\\
% If you want your \texttt{LastPage} to label the last page of
% these end floats, you need to load \xpackage{pageslts} \emph{after}
% loading \xpackage{endfloat}, or to use \texttt{VeryLastPage} instead.
% If, on the other hand, you \emph{want} \texttt{LastPage} to refer
% to the (not so) last page, exclusive of the floats at the end,
% then load in the reverse order. Independent from the order of
% \xpackage{pageslts} and \xpackage{endfloat}, you will still need the
% modified\footnote{New versions are available for over $15$~years now,
% so it might be time to update, if you did not do it already.} version of
% \xpackage{endfloat}.\\
% Using the
% \nolinebreak{\texttt{LastPage}\textbf{\underline{s}}\ (\textit{s}!)}\ label
% should get you to the last page in all cases: |\lastpageref{LastPages}|.\\
%
% Other \LaTeX209\ (!) packages also seem to like to redefine
% |\enddocument|. In addition to the old \xpackage{endfloat},
% \xpackage{harvard} comes to mind. All of these will need to be
% modified swiftly. \textbf{If possible, update to \LaTeXe{}!}
%
% \pagebreak
%
% \subsection{\xpackage{showkeys} package\label{ssec:showkeys}}
%
% When the \xpackage{showkeys} package has been loaded in \texttt{draft} mode,
% in the margin for each label a~box is displayed with the name of the label.
% \xpackage{showkeys} accomplishes this by redefining |\label|, but
% \xpackage{pageslts} does not use |\label|, but writes directly to the
% {\nobreak |\jobname|\xfile{.aux}-}\allowbreak file, and this is generally done
% \emph{after} the according page has shipped out, therefore no box can be placed
% on the \emph{preceding} page. At~least \xpackage{pageslts} gives a warning, that
% \xpackage{showkeys} cannot present the respective label.
%
% \subsection{\xpackage{lastpage} package\label{ssec:lastpage}}
%
% This package first started as a revision of the \xpackage{lastpage}
% package of \textsc{Jeffrey P. Goldberg} (jeffrey+news at goldmark dot org),
% but it became obvious that a replacement was needed to accomplish what this
% package does. For backward compatibility, a~label named |LastPage| is provided.
% Thus |\usepackage{lastpage}| can be replaced by\\
% |\usepackage[pagecontinue=false,%|\newline
% |alphMult=0,AlphMulti=0,fnsymbolmult=false,romanMult=false,RomanMulti=false]{pageslts}|,\\
% if the behaviour of the \xpackage{lastpage} package should be simulated.
% Using \textbf{old}~(!) versions of the \xpackage{lastpage} before the \xpackage{pageslts}
% before the \xpackage{hyperref} [2012/11/06 v6.83m]
% package results in multiply definitions of the |LastPage| label.
% While the \xpackage{pageslts} package cancels the command |\lastpage@putlabel|
% from the old \xpackage{lastpage} package (because it does this itself, and better),
% \xpackage{hyperref} redefines |\lastpage@putlabel| and thereby reintroduces it again
% (\xpackage{hyperref} should probably check for the version of the \xpackage{lastpage}
% package and/or whether the \xpackage{pageslts} package was also loaded.)
%
% \subsection{Using an unknown page numbering scheme\label{sec:unknown}}
%
% \indent I do not know whether \LaTeXe\ can handle another page numbering
% scheme (e.\,g.~Hebraic), but if you want to use it, this should be
% no problem for the \xpackage{pageslts} package. But the original
% |\pagenumbering{...}| as well as the \xpackage{hyperref} package
% (if~used) might want to vote against it, especially when used together
% with the \xpackage{pageslts} package. Especially especially
% (\textit{sic}!) if the last page uses this new page numbering scheme,
% you should check everything double (at least).\\
% (And if the number format is unknown to \LaTeX{}, the pages will have
% no number, and therefore cannot be referenced. You might be able to
% help yourself by using the \xpackage{hyperref} package and
% manually placing |\hypertarget|s and |\href|s.)
%
% \subsection{Page counter overflow\label{ss:overflow}}
%
% \indent Without the use of the \xpackage{alphalph} package, the\\
% \textquotedblleft ranges of supported counter values are more or less
% restricted. Only \cs{arabic} can be used with any counter
% value \TeX\ supports.
% \begin{quote}
% \catcode`\|=12 %
% \begin{tabular}{@{}l|l|l|l@{}}
% Presentation & Supported & Ignored & Error message\\
% command      & domain    & values  & `Counter too large'\\
% \hline
% \cs{arabic}
%   & \ttfamily -MAX..MAX & &\\
% \cs{roman}, \cs{Roman}
%   & \ttfamily 1..MAX & \ttfamily -MAX..0 &\\
% \cs{alph}, \cs{Alph}
%   & \ttfamily 1..26 & 0 & \ttfamily -MAX..-1, 27..MAX\\
% \cs{fnsymbol}
%   & \ttfamily 1..9 & 0 & \ttfamily -MAX..-1, 10..MAX\\
% \hline
% \end{tabular}\\[1ex]
% \texttt{MAX} = \texttt{2147483647}
% \end{quote}
% \textquotedblright{} (\textsc{Heiko Oberdiek}:
% The \xpackage{alphalph} package, 2010/04/18, v2.3, first table, p.~2).\\
% Please see subsubsections~\ref{sss:alphalph} and \ref{sss:romanMult} for
% instructions how to overcome these limitations.
%
% \newpage
%
% \subsection{Using the \xpackage{fnsymbol} page numbering scheme\label{ss:fnsymbol}}
%
% \indent Using the \texttt{fnsymbol} page numbering scheme can result in
% problems -- big ones!\\
% \textbf{When using this page numbering scheme, it is very important to use}
% \ |\lastpageref{...}| \textbf{instead of}\ |\pageref{...}| \textbf{for any%
% link to any label provided by the \xpackage{pageslts} package.}\\
% While the \xpackage{pageslts} package tries really very hard to circumvent
% any problem, other packages might screw up -- and quite totally for that.
% So,~you have been warned!\\
%
% \begin{description}
% \item[-] There can be a counter overflow, see preceding subsection~\ref{ss:overflow}.
%
% \item[-] Adobe Reader DC 2015.008.20082 does not show the correct page names
% for all pages with \texttt{fnsymbol} page numbering scheme (see the example file),
% while at least the (\ldots\ of \ldots )  part of the page number is displayed correctly
% (see page~\pageref{logical page numbers}, tip about logical page numbers).
% When the \xpackage{alphalph} package and the \xpackage{pageslts} package with
% |fnsymbolmult| option are used, more page names are presented correctly by the Reader.
% (Adobe Reader~X even got all pages right.)
% \end{description}
%
% \newpage
%
% \section{Alternatives\label{sec:Alternatives}}
%
% There are similar packages, which do (or do not) similar things. As I
% neither know what exactly you want to accomplish when using this package
% (e.\,g.~page number vs. page name, hyperlinks or not), nor what resources
% you have (e.\,g.~\eTeX{}), here is a list of some possible alternatives:\\
%
% \DescribeMacro{LastPage}
% \begin{description}
% \item[-] The \xpackage{LastPage} package also provides the
%  \texttt{LastPage} label (but not \texttt{VeryLastPage} or
%  \texttt{LastPages}). If you only want this and have a quite
%  limited amount of \TeX\ resources, you might want to use that
%  package instead.
%
% \item[-] If \LaTeX2.09\ is still used, and if you are unable
%  to switch to \LaTeXe{}, the \LaTeX2.09{} compatible
%  \xpackage{lastpage209.sty} can be used (which is also contained
%  in the recent \xpackage{LastPage} package).
% \end{description}
%
% \DescribeMacro{totpages}
% \begin{description}
% \item[-] The \xpackage{totpages} package provides a
%  \texttt{totpages} label similar to \texttt{LastPages},
%  but |\AtEndDocument| instead of |\AfterLastShipout|. Therefore
%  you should stay with \xpackage{pageslts}. The \xpackage{totpages}
%  package additionally computes the number of paper sheets needed to
%  (double) print the document (with one, two, three,\ldots\ pages
%  on one sheet of paper). This can also be accomplished with the
%  \xpackage{papermas} package.
% \end{description}
%
% \DescribeMacro{totcount}
% \begin{description}
% \item[-] The \xpackage{totcount} package provides the last value
%  of a counter, thus also the value of the \texttt{page} counter.
%  You do not get a hyperlink to the last page, only the numerical
%  value of the last page name is given (i.\,e. X+72 pages gives 72
%  instead of 82 as total number of pages), and the number of pages
%  can be changed e.\,g. by |\addtocounter|.
% \end{description}
%
% \DescribeMacro{nofm}
% \begin{description}
% \item[-] \textquotedblleft There is a package \xpackage{nofm.sty}
%  available, but some versions of it are defective, and most don't
%  work with \xpackage{fancyhdr} because they take over the complete
%  page layout.\textquotedblright{} (\textsc{Piet van Oostrum}:
%  Page layout in \LaTeX{}, March~2, 2004, section~16; fancyhdr.pdf)\\
%  \xpackage{nofm} as of 1991/02/25 (without version number), available at\\
%  \url{http://mirror.ctan.org/obsolete/macros/latex209/contrib/misc/nofm.sty},\\
%  does not work with e.\,g. \xpackage{hyperref}, redefines |\enddocument|
%  as well as |\@oddhead|, |\@evenhead|, |\@oddfoot|, and |\@evenfoot|.\\
%  If you know the (\url{https://www.CTAN.org}) location of a%
%  \textbf{working}~(!) version, please send me an e-mail, thanks!
% \end{description}
%
% \DescribeMacro{count1to}
% \begin{description}
% \item[-] You may want to have a look at the \xpackage{count1to} package.
% \end{description}
%
% \DescribeMacro{zref}
% \begin{description}
% \item[-] The \xpackage{zref} package of \textsc{Heiko Oberdiek} requires
%  \eTeX . \xpackage{pageslts} does not require \eTeX , but if you
%  already have \eTeX , you may have a look at the extensive
%  \xpackage{zref} package, whether it suits your needs better
%  (or additionally or whatsoever).
% \end{description}
%
% \bigskip
%
% \noindent (You programmed or found another alternative,
%  which is available at \url{https://www.CTAN.org}?\\
%  OK, send an e-mail to me with the name, location at CTAN,
%  and a short notice, and I will probably include it in the list above.)\\
%
% \smallskip
%
% \noindent About how to get those packages, please see subsection~\ref{ss:Downloads}.
%
% \newpage
%
% \section{Example}
%
%    \begin{macrocode}
%<*example>
\documentclass[british]{article}[2014/09/29]% v1.4h
\usepackage{lipsum}[2014/07/27]%              v1.3
\usepackage[draft]{showkeys}[2014/10/28]%     v3.17
%%      Use final instead of draft to hide the keys. %%
\usepackage{hyperref}[2012/11/06]%            v6.83m
\hypersetup{%
 extension=pdf,%
 plainpages=false,%
 pdfpagelabels=true,%
 hyperindex=false,%
 pdflang={en},%
 pdftitle={pageslts package example},%
 pdfauthor={Hans-Martin Muench},%
 pdfsubject={Example for the pageslts package},%
 pdfkeywords={LaTeX, pageslts, H.-Martin Muench},%
 pdfview=Fit,%
 pdfstartview=Fit,%
 pdfpagelayout=SinglePage,%
 bookmarksopen=true%
}
\usepackage[pagecontinue=true,alphMult=ab,AlphMulti=AB,fnsymbolmult=true,%
 romanMult=true,RomanMulti=true]{pageslts}[2015/12/21]% v1.2f
%% These are the default options. %%

\makeatletter
  \renewcommand{\@evenfoot}%
   {Page \thepage\ (\theCurrentPage; local: \theCurrentPageLocal) of %
    \lastpageref{pagesLTS.roman}(\lastpageref{pagesLTS.roman.local}) + %
    \lastpageref{pagesLTS.Roman}(\lastpageref{pagesLTS.Roman.local}) + %
    \lastpageref{pagesLTS.arabic}(\lastpageref{pagesLTS.arabic.local}) + %
    \lastpageref{pagesLTS.fnsymbol}(\lastpageref{pagesLTS.fnsymbol.local}) + %
    \lastpageref{pagesLTS.alph}(\lastpageref{pagesLTS.alph.local}) + %
    \lastpageref{pagesLTS.Alph}(\lastpageref{pagesLTS.Alph.local}) = %
    \lastpageref{LastPages} pages.%
   }
  \renewcommand{\@oddfoot}{\@evenfoot}
  \def\pagesLTSexampleArabic{3}
  \def\pagesLTSexamplealph{23}
  \gdef\unit#1{\mathord{\thinspace\mathrm{#1}}}%
\makeatother

\listfiles

\begin{document}
\pagenumbering{roman}
%% Note the first \pagenumbering immediately behind \begin{document}! %%
%%\addtocounter{page}{-2} %%

\section*{Example for pageslts}
\addcontentsline{toc}{section}{Example for pageslts}
\markboth{Example for pageslts}{Example for pageslts}

This example demonstrates the most common uses of package\\
\textsf{pageslts}, v1.2f as of 2015/12/21 (HMM).\\
The used options were \texttt{pagecontinue=true},
\texttt{alphMult=ab}, \texttt{AlphMulti=AB}, \linebreak
\texttt{fnsymbolmult=true},
\texttt{romanMult=true}, and \texttt{RomanMulti=true}
(the default ones).\\
For more details please see the documentation!\\

\label{keys} To hide the \pageref{keys}{\qquad } use option
\texttt{final} instead of \texttt{draft} with the \textsf{showkeys}
package (or remove the package call from the preamble of
this document).\\

\textbf{Hyperlinks or not:} If the \textsf{hyperref} package is loaded,
the references are also hyperlinked:\\
\smallskip
Page \thepage\ (\theCurrentPage; local: \theCurrentPageLocal) of %
\lastpageref{pagesLTS.roman}(\lastpageref{pagesLTS.roman.local}) + %
\lastpageref{pagesLTS.Roman}(\lastpageref{pagesLTS.Roman.local}) + %
\lastpageref{pagesLTS.arabic}(\lastpageref{pagesLTS.arabic.local}) + %
\lastpageref{pagesLTS.fnsymbol}(\lastpageref{pagesLTS.fnsymbol.local}) + %
\lastpageref{pagesLTS.alph}(\lastpageref{pagesLTS.alph.local}) + %
\lastpageref{pagesLTS.Alph}(\lastpageref{pagesLTS.Alph.local}) = %
\lastpageref{LastPages} pages.\\
If the \textsf{hyperref} package is loaded, but the hyperlinks of the
references shall be suppressed, \verb|\pageref*{...}|
and \verb|\lastpageref*{...}| can be used:\\
Page \thepage\ (\theCurrentPage; local: \theCurrentPageLocal) of %
\lastpageref*{pagesLTS.roman}(\lastpageref*{pagesLTS.roman.local}) + %
\lastpageref*{pagesLTS.Roman}(\lastpageref*{pagesLTS.Roman.local}) + %
\lastpageref*{pagesLTS.arabic}(\lastpageref*{pagesLTS.arabic.local}) + %
\lastpageref*{pagesLTS.fnsymbol}(\lastpageref*{pagesLTS.fnsymbol.local}) + %
\lastpageref*{pagesLTS.alph}(\lastpageref*{pagesLTS.alph.local}) + %
\lastpageref*{pagesLTS.Alph}(\lastpageref*{pagesLTS.Alph.local}) = %
\lastpageref*{LastPages} pages.\\


\textbf{Trademarks} appear throughout this example without any
trademark symbol; they are the property of their respective
trademark owner. There is no intention of infringement; the
usage is to the benefit of the trademark owner.\\

\textbf{Tip}: Use \textit{logical page numbers}
for the display of the pdf (in Adobe Reader DC 2015.008.20082:
Edit $>$ Preferences $>$ Categories: Page Display $>$
Page Content and Information: Use logical page numbers)!\\

You want negative page numbers? Not only arabic, but even roman,
Roman, alph, Alph or fnsymbol ones? No problem, e.\,g. just give a\\
\verb|\addtocounter{page}{|-\textit{some number}\verb|}| in the
source code of this example file (or uncomment the prepared line)!

\bigskip

Save per page about $200\unit{ml}$ water, $2\unit{g}$ CO$_{2}$
and $2\unit{g}$ wood:\\
Therefore please print only if this is really necessary.

\pagebreak

\tableofcontents

\newpage

\pagenumbering{roman}
%% in case the page numbering is changed before,
%% otherwise pagesLTS.current.local.roman on this page
%% would be undefined

\section{roman}

\noindent (\texttt{roman} page numbering was started before,
because the page numbering scheme was needed to start at
the first page, of course.)\\

\noindent First page (\verb|\lastpageref{pagesLTS.0}|):
\lastpageref{pagesLTS.0}\\

\noindent The page (\verb|\thepage|): \thepage \\

\noindent Current page (\verb|\theCurrentPage|),
i.\,e. counted continuously from the first page): \theCurrentPage \\
You can get this also in other formats:
\roman{CurrentPage}, \Roman{CurrentPage}, \arabic{CurrentPage},
\fnsymbol{CurrentPage}, \alph{CurrentPage}, \Alph{CurrentPage}.

\noindent CurrentPageLocal (\verb|\theCurrentPageLocal|),
i.\,e. counted continuously from the first page of the
current page numbering scheme): \theCurrentPageLocal \\
You can get also this in other formats, too:
\roman{pagesLTS.current.local.roman}, \Roman{pagesLTS.current.local.roman},
\arabic{pagesLTS.current.local.roman}, \fnsymbol{pagesLTS.current.local.roman},
\alph{pagesLTS.current.local.roman}, \Alph{pagesLTS.current.local.roman},
but probably it only makes sense if page numbering scheme and format are
the same, e.\,g.\\
\verb|\Roman{pagesLTS.current.local.Roman}|\\
or \verb|\Alph{pagesLTS.current.local.Alph}|. \verb|\arabic{...}| could
make sense even if combined with another page numbering scheme.
And this is exactly what \verb|\theCurrentPageLocal| does:\\
\nolinebreak{\verb|\def\theCurrentPageLocal{\arabic{pagesLTS.current.local.\pagesLTS@pnc}}|.}\\

\noindent Last roman page (pagesLTS.roman): \lastpageref{pagesLTS.roman}{\hskip4em }
(There are \lastpageref{pagesLTS.roman.local} pages with roman numbers.)\\

\noindent Last Roman page (pagesLTS.Roman): \lastpageref{pagesLTS.Roman}{\hskip3em }
(There are \lastpageref{pagesLTS.Roman.local}~pages with Roman numbers:\\
\lastpages{Roman}{1}~pages in the first Roman sector
(\pageref{Roman}{\hskip3em }-\lastpageref{pagesLTS.Roman.1}{\hskip3em }),\\
\lastpages{Roman}{2}~pages in the second Roman sector
(\pageref{Roman2}{\hskip3em }-\lastpageref{pagesLTS.Roman.2}{\hskip3em }), and\\
\lastpages{Roman}{3}~pages in the third Roman sector
(\pageref{Roman3}{\hskip3em }-\lastpageref{pagesLTS.Roman.3}{\hskip3em }).\\

When the option \texttt{pagecontinue=false} is used with the
\textsf{pageslts} package, the \verb|\lastpageref{pagesLTS.Roman}| will
point to the same page as before, but this will have a lower number.\\
The \verb|\lastpageref{pagesLTS.Roman.local}| will not change,
because the number of pages does not change (only the page numbers).\\

\noindent Last arabic page (pagesLTS.arabic): \lastpageref{pagesLTS.arabic}{\hskip5em }
(There are only \lastpageref{pagesLTS.arabic.local} pages with arabic numbers,
because an \verb|\addtocounter{page}{|\pagesLTSexampleArabic\verb|}| was used.)\\

\noindent Last fnsymbol page (pagesLTS.fnsymbol): \lastpageref{pagesLTS.fnsymbol} \\
(\verb|\lastpageref{pagesLTS.fnsymbol}| -- never \verb|\pageref{pagesLTS.fnsymbol}|!)\\
(There are \lastpageref{pagesLTS.fnsymbol.local} pages with fnsymbol numbers.)\\

\noindent Last alph page (pagesLTS.alph): \lastpageref{pagesLTS.alph}{\hskip4em }
(There are only \lastpageref{pagesLTS.alph.local} pages with alph numbers,
because an \verb|\addtocounter{page}{|\pagesLTSexamplealph\verb|}| was used.)\\

\noindent Last Alph page (pagesLTS.Alph): \lastpageref{pagesLTS.Alph}{\hskip4em }
(There are \lastpageref{pagesLTS.Alph.local} pages with Alph numbers.)\\

\noindent Last page's \textit{name} (LastPage): \lastpageref{LastPage}\\

\noindent Very last page's \textit{name} (VeryLastPage): \lastpageref{VeryLastPage}\\
(\texttt{lastpage} and \texttt{VeryLastPage} are identical, unless
a package with output \linebreak
\verb|\AtEndDocument| after the \textsf{pageslts} package was added.)\\

\noindent Last page's \textit{number} (LastPages): \lastpageref{LastPages}{\hskip3em }
(=~total number of pages)\\

\lipsum[1-3]

\newpage

\pagenumbering{Roman}

\section{Roman\label{Roman}}
\subsection{Common Roman page numbering}

\noindent First page (\verb|\lastpageref{pagesLTS.0}|):
\lastpageref{pagesLTS.0}\\

\noindent The page (\verb|\thepage|): \thepage \\

\noindent Current page (\verb|\theCurrentPage|),
i.\,e. counted continuously from the first page): \theCurrentPage \\

\noindent CurrentPageLocal (\verb|\theCurrentPageLocal|),
i.\,e. counted continuously from the first page of the
current page numbering scheme): \theCurrentPageLocal \\

\noindent Last roman page (pagesLTS.roman): \lastpageref{pagesLTS.roman}{\hskip4em }
(There are \lastpageref{pagesLTS.roman.local} pages with roman numbers.)\\

\noindent Last Roman page (pagesLTS.Roman): \lastpageref{pagesLTS.Roman}{\hskip3em }
(There are \lastpageref{pagesLTS.Roman.local}~pages with Roman numbers:\\
\lastpages{Roman}{1}~pages in the first Roman sector
(\pageref{Roman}{\hskip3em }-\lastpageref{pagesLTS.Roman.1}{\hskip3em }),\\
\lastpages{Roman}{2}~pages in the second Roman sector
(\pageref{Roman2}{\hskip3em }-\lastpageref{pagesLTS.Roman.2}{\hskip3em }), and\\
\lastpages{Roman}{3}~pages in the third Roman sector
(\pageref{Roman3}{\hskip3em }-\lastpageref{pagesLTS.Roman.3}{\hskip3em }).\\

\noindent Last arabic page (pagesLTS.arabic): \lastpageref{pagesLTS.arabic}{\hskip5em }
(There are only \lastpageref{pagesLTS.arabic.local} pages with arabic numbers,
because an \verb|\addtocounter{page}{|\pagesLTSexampleArabic\verb|}| was used.)\\

\noindent Last fnsymbol page (pagesLTS.fnsymbol): \lastpageref{pagesLTS.fnsymbol} \\
(\verb|\lastpageref{pagesLTS.fnsymbol}| -- never \verb|\pageref{pagesLTS.fnsymbol}|!)\\
(There are \lastpageref{pagesLTS.fnsymbol.local} pages with fnsymbol numbers.)\\

\noindent Last alph page (pagesLTS.alph): \lastpageref{pagesLTS.alph}{\hskip4em }
(There are only \lastpageref{pagesLTS.alph.local} pages with alph numbers,
because an \verb|\addtocounter{page}{|\pagesLTSexamplealph\verb|}| was used.)\\

\noindent Last Alph page (pagesLTS.Alph): \lastpageref{pagesLTS.Alph}{\hskip4em }
(There are \lastpageref{pagesLTS.Alph.local} pages with Alph numbers.)\\

\noindent Last page's \textit{name} (LastPage): \lastpageref{LastPage}\\

\noindent Very last page's \textit{name} (VeryLastPage): \lastpageref{VeryLastPage}\\
(\texttt{LastPage} and \texttt{VeryLastPage} are identical, unless
a package with output \linebreak
\verb|\AtEndDocument| after the \textsf{pageslts} package was added.)\\

\noindent Last page's \textit{number} (LastPages): \lastpageref{LastPages}{\hskip3em }
(=~total number of pages)\\

\lipsum[1-4]

\newpage

\subsection{Last page of first Roman sector}
\verb|\lastpageref{pagesLTS.Roman}| does \textbf{not}
refer to this page (but there: \lastpageref{pagesLTS.Roman}),
because the option \texttt{pagecontinue=true}
was chosen. When a reference to this page is wanted,\\
\verb|\lastpageref{pagesLTS.Roman.1}| can be used: \lastpageref{pagesLTS.Roman.1}.\\

\bigskip
There are \lastpages{Roman}{1}~pages (\verb|\lastpages{Roman}{1}|)
in this first Roman sector.\\
The Roman page numbering scheme is continued later in section~\ref{Roman2})!

\newpage

\pagenumbering{arabic}

\section{arabic}

\subsection{Standard page numbering}

\noindent First page (\verb|\lastpageref{pagesLTS.0}|):
\lastpageref{pagesLTS.0}\\

\noindent The page (\verb|\thepage|): \thepage \\

\noindent Current page (\verb|\theCurrentPage|),
i.\,e. counted continuously from the first page): \theCurrentPage \\

\noindent CurrentPageLocal (\verb|\theCurrentPageLocal|),
i.\,e. counted continuously from the first page of the
current page numbering scheme): \theCurrentPageLocal \\

\noindent Last roman page (pagesLTS.roman): \lastpageref{pagesLTS.roman}{\hskip4em }
(There are \lastpageref{pagesLTS.roman.local} pages with roman numbers.)\\

\noindent Last Roman page (pagesLTS.Roman): \lastpageref{pagesLTS.Roman}{\hskip3em }
(There are \lastpageref{pagesLTS.Roman.local}~pages with Roman numbers:\\
\lastpages{Roman}{1}~pages in the first Roman sector
(\pageref{Roman}{\hskip3em }-\lastpageref{pagesLTS.Roman.1}{\hskip3em }),\\
\lastpages{Roman}{2}~pages in the second Roman sector
(\pageref{Roman2}{\hskip3em }-\lastpageref{pagesLTS.Roman.2}{\hskip3em }), and\\
\lastpages{Roman}{3}~pages in the third Roman sector
(\pageref{Roman3}{\hskip3em }-\lastpageref{pagesLTS.Roman.3}{\hskip3em }).\\

\noindent Last arabic page (pagesLTS.arabic): \lastpageref{pagesLTS.arabic}{\hskip5em }
(There are only \lastpageref{pagesLTS.arabic.local} pages with arabic numbers,
because an \verb|\addtocounter{page}{|\pagesLTSexampleArabic\verb|}| was used.)\\

\noindent Last fnsymbol page (pagesLTS.fnsymbol): \lastpageref{pagesLTS.fnsymbol} \\
(\verb|\lastpageref{pagesLTS.fnsymbol}| -- never \verb|\pageref{pagesLTS.fnsymbol}|!)\\
(There are \lastpageref{pagesLTS.fnsymbol.local} pages with fnsymbol numbers.)\\

\noindent Last alph page (pagesLTS.alph): \lastpageref{pagesLTS.alph}{\hskip4em }
(There are only \lastpageref{pagesLTS.alph.local} pages with alph numbers,
because an \verb|\addtocounter{page}{|\pagesLTSexamplealph\verb|}| was used.)\\

\noindent Last Alph page (pagesLTS.Alph): \lastpageref{pagesLTS.Alph}{\hskip4em }
(There are \lastpageref{pagesLTS.Alph.local} pages with Alph numbers.)\\

\noindent Last page's \textit{name} (LastPage): \lastpageref{LastPage}\\

\noindent Very last page's \textit{name} (VeryLastPage): \lastpageref{VeryLastPage}\\
(\texttt{LastPage} and \texttt{VeryLastPage} are identical, unless
a package with output \linebreak
\verb|\AtEndDocument| after the \textsf{pagesLTS} package was added.)\\

\noindent Last page's \textit{number} (LastPages): \lastpageref{LastPages}{\hskip3em }
(=~total number of pages)\\

\lipsum[1-4]
\newpage

\subsection[Empty page style]{Also an empty page style is no problem %
for the current or total page count}

\bigskip

\thispagestyle{empty}

\noindent First page (\verb|\lastpageref{pagesLTS.0}|):
\lastpageref{pagesLTS.0}\\

\noindent The page (\verb|\thepage|): \thepage \\

\noindent Current page (\verb|\theCurrentPage|),
i.\,e. counted continuously from the first page): \theCurrentPage \\

\noindent CurrentPageLocal (\verb|\theCurrentPageLocal|),
i.\,e. counted continuously from the first page of the
current page numbering scheme): \theCurrentPageLocal \\

\noindent Last roman page (pagesLTS.roman): \lastpageref{pagesLTS.roman}{\hskip4em }
(There are \lastpageref{pagesLTS.roman.local} pages with roman numbers.)\\

\noindent Last Roman page (pagesLTS.Roman): \lastpageref{pagesLTS.Roman}{\hskip3em }
(There are \lastpageref{pagesLTS.Roman.local}~pages with Roman numbers:\\
\lastpages{Roman}{1}~pages in the first Roman sector
(\pageref{Roman}{\hskip3em }-\lastpageref{pagesLTS.Roman.1}{\hskip3em }),\\
\lastpages{Roman}{2}~pages in the second Roman sector
(\pageref{Roman2}{\hskip3em }-\lastpageref{pagesLTS.Roman.2}{\hskip3em }), and\\
\lastpages{Roman}{3}~pages in the third Roman sector
(\pageref{Roman3}{\hskip3em }-\lastpageref{pagesLTS.Roman.3}{\hskip3em }).\\

\noindent Last arabic page (pagesLTS.arabic): \lastpageref{pagesLTS.arabic}{\hskip5em }
(There are only \lastpageref{pagesLTS.arabic.local} pages with arabic numbers,
because an \verb|\addtocounter{page}{|\pagesLTSexampleArabic\verb|}| was used.)\\

\noindent Last fnsymbol page (pagesLTS.fnsymbol): \lastpageref{pagesLTS.fnsymbol} \\
(\verb|\lastpageref{pagesLTS.fnsymbol}| -- never \verb|\pageref{pagesLTS.fnsymbol}|!)\\
(There are \lastpageref{pagesLTS.fnsymbol.local} pages with fnsymbol numbers.)\\

\noindent Last alph page (pagesLTS.alph): \lastpageref{pagesLTS.alph}{\hskip4em }
(There are only \lastpageref{pagesLTS.alph.local} pages with alph numbers,
because an \verb|\addtocounter{page}{|\pagesLTSexamplealph\verb|}| was used.)\\

\noindent Last Alph page (pagesLTS.Alph): \lastpageref{pagesLTS.Alph}{\hskip4em }
(There are \lastpageref{pagesLTS.Alph.local} pages with Alph numbers.)\\

\noindent Last page's \textit{name} (LastPage): \lastpageref{LastPage}\\

\noindent Very last page's \textit{name} (VeryLastPage): \lastpageref{VeryLastPage}\\
(\texttt{LastPage} and \texttt{VeryLastPage} are identical, unless
a package with output \linebreak
\verb|\AtEndDocument| after the \textsf{pageslts} package was added.)\\

\noindent Last page's \textit{number} (LastPages): \lastpageref{LastPages}{\hskip3em }
(=~total number of pages)\\

\lipsum[1-4]

\newpage

\subsection[addtocounter, setcounter]{Neither %
\texttt{\textbackslash addtocounter\{page\}} nor %
\texttt{\textbackslash setcounter\{page\}} is a problem for the %
current or total page numbers}

(Here is an \verb|\addtocounter{page}{|\pagesLTSexampleArabic\verb|}|
in the source code.)\\
\addtocounter{page}{\pagesLTSexampleArabic}

\noindent The page (from \verb|\thepage| command): \thepage \\

\noindent Current page (from \verb|\theCurrentPage|\ %
command), i.\,e. counted continuously from the first page): \theCurrentPage \\

\noindent CurrentPageLocal (from \verb|\theCurrentPageLocal|\ %
command), i.\,e. counted continuously from the first page of the
current page numbering scheme): \theCurrentPageLocal\\

\noindent Last page's number (LastPages): \lastpageref{LastPages}{\hskip3em }
(= total number of pages)\\

\lipsum[1-7]

\newpage

\pagenumbering{fnsymbol}

\section{fnsymbol}

Adobe Reader DC 2015.008.20082 does not show the correct page names
for all pages with \texttt{fnsymbol} page numbering scheme,
while at least the (\ldots\ of \ldots )  part of the page number is displayed correctly.
(Adobe Reader~X even got all pages right.)

\bigskip

Without option \texttt{fnsymbolmult=true} of the \textsf{pageslts} package
(and the help of \textsc{Heiko Oberdiek}'s \textsf{alphalph} package),
after page~9 (\textquotedblleft \ensuremath {\ddagger \ddagger }\textquotedblright )
(and also for negative page numbers) there would just appear a
\begin{quote}
\begin{verbatim}
LaTeX Error: Counter too large
See the LaTeX manual or LaTeX Companion for explanation.
You've lost some text.  Try typing  <return>  to proceed.
If that doesn't work, type  X <return>  to quit.
\end{verbatim}
\end{quote}
Now the page numbers after 5 (\ensuremath {\mathparagraph }) are
continued  with the doubled \textquotedblleft number\textquotedblright{} of
the first, second, third,\ldots\ page (\ensuremath {**},
\ensuremath {\dagger \dagger }, \ensuremath {\ddagger \ddagger },
\ensuremath {\mathsection \mathsection },
\ensuremath {\mathparagraph \mathparagraph }),
and after the tenth page the \textquotedblleft number\textquotedblright{} is
tripled (\ensuremath {***}, \ensuremath {\dagger \dagger \dagger },\ldots).
Page zero is named 0 and negative pages just named like the positive ones
with addition of a minus sign~($-$).

\bigskip

\noindent First page (\verb|\lastpageref{pagesLTS.0}|):
\lastpageref{pagesLTS.0}\\

\noindent The page (\verb|\thepage|): \thepage \\

\noindent Current page (\verb|\theCurrentPage|),
i.\,e. counted continuously from the first page): \theCurrentPage \\

\noindent CurrentPageLocal (\verb|\theCurrentPageLocal|),
i.\,e. counted continuously from the first page of the
current page numbering scheme): \theCurrentPageLocal \\

\noindent Last roman page (pagesLTS.roman): \lastpageref{pagesLTS.roman}{\hskip4em }
(There are \lastpageref{pagesLTS.roman.local} pages with roman numbers.)\\

\noindent Last Roman page (pagesLTS.Roman): \lastpageref{pagesLTS.Roman}{\hskip3em }
(There are \lastpageref{pagesLTS.Roman.local}~pages with Roman numbers:\\
\lastpages{Roman}{1}~pages in the first Roman sector
(\pageref{Roman}{\hskip3em }-\lastpageref{pagesLTS.Roman.1}{\hskip3em }),\\
\lastpages{Roman}{2}~pages in the second Roman sector
(\pageref{Roman2}{\hskip3em }-\lastpageref{pagesLTS.Roman.2}{\hskip3em }), and\\
\lastpages{Roman}{3}~pages in the third Roman sector
(\pageref{Roman3}{\hskip3em }-\lastpageref{pagesLTS.Roman.3}{\hskip3em }).\\

\noindent Last arabic page (pagesLTS.arabic): \lastpageref{pagesLTS.arabic}{\hskip5em }
(There are only \lastpageref{pagesLTS.arabic.local} pages with arabic numbers,
because an \verb|\addtocounter{page}{|\pagesLTSexampleArabic\verb|}| was used.)\\

\noindent Last fnsymbol page (pagesLTS.fnsymbol): \lastpageref{pagesLTS.fnsymbol} \\
(\verb|\lastpageref{pagesLTS.fnsymbol}| -- never \verb|\pageref{pagesLTS.fnsymbol}|!)\\
(There are \lastpageref{pagesLTS.fnsymbol.local} pages with fnsymbol numbers.)\\

\noindent Last alph page (pagesLTS.alph): \lastpageref{pagesLTS.alph}{\hskip4em }
(There are only \lastpageref{pagesLTS.alph.local} pages with alph numbers,
because an \verb|\addtocounter{page}{|\pagesLTSexamplealph\verb|}| was used.)\\

\noindent Last Alph page (pagesLTS.Alph): \lastpageref{pagesLTS.Alph}{\hskip4em }
(There are \lastpageref{pagesLTS.Alph.local} pages with Alph numbers.)\\

\noindent Last page's \textit{name} (LastPage): \lastpageref{LastPage}\\

\noindent Very last page's \textit{name} (VeryLastPage): \lastpageref{VeryLastPage}\\
(\texttt{LastPage} and \texttt{VeryLastPage} are identical, unless
a package with output \linebreak
\verb|\AtEndDocument| after the \textsf{pageslts} package was added.)\\

\noindent Last page's \textit{number} (LastPages): \lastpageref{LastPages}{\hskip3em }
(=~total number of pages)\\

\lipsum[1-60]

\newpage

\pagenumbering{Roman}

\section{Roman - again!\label{Roman2}}

The page number would start with \textquotedblleft I\textquotedblright{} again --
but for the \textsf{pageslts} package (with option \texttt{pagecontinue=true},
or with option just \texttt{pagecontinue}, or even just
\nolinebreak{with\textbf{out}} option \texttt{pagecontinue=false}).
This package remembered the (\arabic{pagesLTS.double.Roman}-1)\footnote{%
OK, here you have to compute this value for yourself, but %
subtracting one should be manageable for \TeX{}nicians.} pages already
done in Roman output, and therefore continues with page
\textquotedblleft \thepage \textquotedblright .\\
If you want to start with \textquotedblleft I\textquotedblright{} all
over again, you will have two pages with the same name,
but nevertheless you can do this by using option \texttt{pagecontinue=false}
or a \verb|\setcounter{page}{1}|\ here (not demonstrated in this example file).\\

\noindent First page (\verb|\lastpageref{pagesLTS.0}|):
 \lastpageref{pagesLTS.0}\\

\noindent The page (\verb|\thepage|): \thepage \\

\noindent Current page (\verb|\theCurrentPage|),
i.\,e. counted continuously from the first page): \theCurrentPage \\

\noindent CurrentPageLocal (\verb|\theCurrentPageLocal|),
i.\,e. counted continuously from the first page of the
current page numbering scheme): \theCurrentPageLocal \\

\noindent Last roman page (pagesLTS.roman): \lastpageref{pagesLTS.roman}{\hskip4em }
(There are \lastpageref{pagesLTS.roman.local} pages with roman numbers.)\\

\noindent Last Roman page (pagesLTS.Roman): \lastpageref{pagesLTS.Roman}{\hskip3em }
(There are \lastpageref{pagesLTS.Roman.local}~pages with Roman numbers:\\
\lastpages{Roman}{1}~pages in the first Roman sector
(\pageref{Roman}{\hskip3em }-\lastpageref{pagesLTS.Roman.1}{\hskip3em }),\\
\lastpages{Roman}{2}~pages in the second Roman sector
(\pageref{Roman2}{\hskip3em }-\lastpageref{pagesLTS.Roman.2}{\hskip3em }), and\\
\lastpages{Roman}{3}~pages in the third Roman sector
(\pageref{Roman3}{\hskip3em }-\lastpageref{pagesLTS.Roman.3}{\hskip3em }).\\

\noindent Last arabic page (pagesLTS.arabic): \lastpageref{pagesLTS.arabic}{\hskip5em }
(There are only \lastpageref{pagesLTS.arabic.local} pages with arabic numbers,
because an \verb|\addtocounter{page}{|\pagesLTSexampleArabic\verb|}| was used.)\\

\noindent Last fnsymbol page (pagesLTS.fnsymbol): \lastpageref{pagesLTS.fnsymbol} \\
(\verb|\lastpageref{pagesLTS.fnsymbol}| -- never
\verb|\pageref{pagesLTS.fnsymbol}|!)\\
(There are \lastpageref{pagesLTS.fnsymbol.local} pages with fnsymbol numbers.)\\

\noindent Last alph page (pagesLTS.alph): \lastpageref{pagesLTS.alph}{\hskip4em }
(There are only \lastpageref{pagesLTS.alph.local} pages with alph numbers,
because an \verb|\addtocounter{page}{|\pagesLTSexamplealph\verb|}| was used.)\\

\noindent Last Alph page (pagesLTS.Alph): \lastpageref{pagesLTS.Alph}{\hskip4em }
(There are \lastpageref{pagesLTS.Alph.local} pages with Alph numbers.)\\

\noindent Last page's \textit{name} (LastPage): \lastpageref{LastPage}\\

\noindent Very last page's \textit{name} (VeryLastPage): \lastpageref{VeryLastPage}\\
(\texttt{LastPage} and \texttt{VeryLastPage} are identical, unless
a package with output \linebreak
\verb|\AtEndDocument| after the \textsf{pageslts} package was added.)\\

\noindent Last page's \textit{number} (LastPages): \lastpageref{LastPages}{\hskip3em }
(=~total number of pages)\\

\lipsum[1-6]

\newpage

\verb|\lastpageref{pagesLTS.Roman}| does \textbf{not}
refer to this page (but there: \lastpageref{pagesLTS.Roman}),
because the option \texttt{pagecontinue=true}
was chosen. When a reference to this page is wanted,\\
\verb|\lastpageref{pagesLTS.Roman.2}| can be used: \lastpageref{pagesLTS.Roman.2}.\\

\bigskip
There are \lastpages{Roman}{2}~pages (\verb|\lastpages{Roman}{2}|) in this
second Roman sector.\\
The Roman page numbering scheme is continued later in section~\ref{Roman3})!

\newpage

\pagenumbering{alph}

\section{alph\label{alph}}

\noindent First page (\verb|\lastpageref{pagesLTS.0}|):
 \lastpageref{pagesLTS.0}\\

\noindent The page (\verb|\thepage|): \thepage \\

\noindent Current page (\verb|\theCurrentPage|),
i.\,e. counted continuously from the first page): \theCurrentPage \\

\noindent CurrentPageLocal (\verb|\theCurrentPageLocal|),
i.\,e. counted continuously from the first page of the
current page numbering scheme): \theCurrentPageLocal \\

\noindent Last roman page (pagesLTS.roman): \lastpageref{pagesLTS.roman}{\hskip4em }
(There are \lastpageref{pagesLTS.roman.local} pages with roman numbers.)\\

\noindent Last Roman page (pagesLTS.Roman): \lastpageref{pagesLTS.Roman}{\hskip3em }
(There are \lastpageref{pagesLTS.Roman.local}~pages with Roman numbers:\\
\lastpages{Roman}{1}~pages in the first Roman sector
(\pageref{Roman}{\hskip3em }-\lastpageref{pagesLTS.Roman.1}{\hskip3em }),\\
\lastpages{Roman}{2}~pages in the second Roman sector
(\pageref{Roman2}{\hskip3em }-\lastpageref{pagesLTS.Roman.2}{\hskip3em }), and\\
\lastpages{Roman}{3}~pages in the third Roman sector
(\pageref{Roman3}{\hskip3em }-\lastpageref{pagesLTS.Roman.3}{\hskip3em }).\\

\noindent Last arabic page (pagesLTS.arabic): \lastpageref{pagesLTS.arabic}{\hskip5em }
(There are only \lastpageref{pagesLTS.arabic.local} pages with arabic numbers,
because an \verb|\addtocounter{page}{|\pagesLTSexampleArabic\verb|}| was used.)\\

\noindent Last fnsymbol page (pagesLTS.fnsymbol): \lastpageref{pagesLTS.fnsymbol} \\
(\verb|\lastpageref{pagesLTS.fnsymbol}| -- never \verb|\pageref{pagesLTS.fnsymbol}|!)\\
(There are \lastpageref{pagesLTS.fnsymbol.local} pages with fnsymbol numbers.)\\

\noindent Last alph page (pagesLTS.alph): \lastpageref{pagesLTS.alph}{\hskip4em }
(There are only \lastpageref{pagesLTS.alph.local} pages with alph numbers,
because an \verb|\addtocounter{page}{|\pagesLTSexamplealph\verb|}| was used.)\\

\noindent Last Alph page (pagesLTS.Alph): \lastpageref{pagesLTS.Alph}{\hskip4em }
(There are \lastpageref{pagesLTS.Alph.local} pages with Alph numbers.)\\

\noindent Last page's \textit{name} (LastPage): \lastpageref{LastPage}\\

\noindent Very last page's \textit{name} (VeryLastPage): \lastpageref{VeryLastPage}\\
(\texttt{LastPage} and \texttt{VeryLastPage} are identical, unless
a package with output \linebreak
\verb|\AtEndDocument| after the \textsf{pageslts} package was added.)\\

\noindent Last page's \textit{number} (LastPages): \lastpageref{LastPages}{\hskip3em }
(=~total number of pages)\\

\lipsum[1-4]

\newpage

Without option \texttt{alphMult=ab} of the \textsf{pageslts} (and the help of
\textsc{Heiko Oberdiek's} \textsf{alphalph} package), after page
\textquotedblleft z\textquotedblright{} there would just appear a
\begin{quote}
\begin{verbatim}
LaTeX Error: Counter too large
See the LaTeX manual or LaTeX Companion for explanation.
You've lost some text.  Try typing  <return>  to proceed.
If that doesn't work, type  X <return>  to quit.
\end{verbatim}
\end{quote}
Now the page numbers are continued aa, ab, ac,\ldots\ (aa, bb, cc,\ldots\ is
also possible, see the \textsf{pageslts} documentation).\\
To demonstrate this, we add a\\
\verb|\addtocounter{page}{|\pagesLTSexamplealph\verb|}|\\
in the source code here.

\addtocounter{page}{\pagesLTSexamplealph}

\bigskip

\lipsum[1-18]

\newpage

\pagenumbering{Roman}

\section{Roman - third time!\label{Roman3}}

The page number would start with
\textquotedblleft I\textquotedblright{} again -- but
for the \textsf{pageslts} package (with option \texttt{pagecontinue=true},
or with option just \texttt{pagecontinue}, or even just
\nolinebreak{with\textbf{out}} option \texttt{pagecontinue=false}).
This package remembered the
(\arabic{pagesLTS.double.Roman}-1)\footnote{OK, here %
you have to compute this value for yourself, but %
subtracting one should be managable for \TeX nicians.} pages already
done in Roman output, and therefore continues with page
\textquotedblleft \thepage \textquotedblright .\\
If you want to start with \textquotedblleft I\textquotedblright{} all
over again, you will have (at least) two pages with the same name,
but nevertheless you can do this by using option
\texttt{pagecontinue=false} instead of \texttt{pagecontinue=true}
(not demonstrated here).\\

\noindent First page (\verb|\lastpageref{pagesLTS.0}|):
 \lastpageref{pagesLTS.0}\\

\noindent The page (\verb|\thepage|): \thepage \\

\noindent Current page (\verb|\theCurrentPage|),
i.\,e. counted continuously from the first page): \theCurrentPage \\

\noindent CurrentPageLocal (\verb|\theCurrentPageLocal|),
i.\,e. counted continuously from the first page of the
current page numbering scheme): \theCurrentPageLocal \\

\noindent Last roman page (pagesLTS.roman): \lastpageref{pagesLTS.roman}{\hskip4em }
(There are \lastpageref{pagesLTS.roman.local} pages with roman numbers.)\\

\noindent Last Roman page (pagesLTS.Roman): \lastpageref{pagesLTS.Roman}{\hskip3em }
(There are \lastpageref{pagesLTS.Roman.local}~pages with Roman numbers:\\
\lastpages{Roman}{1}~pages in the first Roman sector
(\pageref{Roman}{\hskip3em }-\lastpageref{pagesLTS.Roman.1}{\hskip3em }),\\
\lastpages{Roman}{2}~pages in the second Roman sector
(\pageref{Roman2}{\hskip3em }-\lastpageref{pagesLTS.Roman.2}{\hskip3em }), and\\
\lastpages{Roman}{3}~pages in the third Roman sector
(\pageref{Roman3}{\hskip3em }-\lastpageref{pagesLTS.Roman.3}{\hskip3em }).\\

\noindent Last arabic page (pagesLTS.arabic): \lastpageref{pagesLTS.arabic}{\hskip5em }
(There are only \lastpageref{pagesLTS.arabic.local} pages with arabic numbers,
because an \verb|\addtocounter{page}{|\pagesLTSexampleArabic\verb|}| was used.)\\

\noindent Last fnsymbol page (pagesLTS.fnsymbol): \lastpageref{pagesLTS.fnsymbol} \\
(\verb|\lastpageref{pagesLTS.fnsymbol}| -- never \verb|\pageref{pagesLTS.fnsymbol}|!)\\
(There are \lastpageref{pagesLTS.fnsymbol.local} pages with fnsymbol numbers.)\\

\noindent Last alph page (pagesLTS.alph): \lastpageref{pagesLTS.alph}{\hskip4em }
(There are only \lastpageref{pagesLTS.alph.local} pages with alph numbers,
because an \verb|\addtocounter{page}{|\pagesLTSexamplealph\verb|}| was used.)\\

\noindent Last Alph page (pagesLTS.Alph): \lastpageref{pagesLTS.Alph}{\hskip4em }
(There are \lastpageref{pagesLTS.Alph.local} pages with Alph numbers.)\\

\noindent Last page's \textit{name} (LastPage): \lastpageref{LastPage}\\

\noindent Very last page's \textit{name} (VeryLastPage): \lastpageref{VeryLastPage}\\
(\texttt{LastPage} and \texttt{VeryLastPage} are identical, unless
a package with output \linebreak
\verb|\AtEndDocument| after the \textsf{pageslts} package was added.)\\

\noindent Last page's \textit{number} (LastPages): \lastpageref{LastPages}{\hskip3em }
(=~total number of pages)\\

\lipsum[1-3]

\newpage

\lastpageref{pagesLTS.Roman}
(\verb|\lastpageref{pagesLTS.Roman}|)
\textbf{does} refers to this page, because the option
\texttt{pagecontinue=true} was chosen. Also\\
\verb|\lastpageref{pagesLTS.Roman.3}| can be used: \lastpageref{pagesLTS.Roman.3}.\\

\bigskip

There are \lastpages{Roman}{3}~pages (\verb|\lastpages{Roman}{3}|) in this
third Roman sector.\\

\newpage

\pagenumbering{Alph}

\section{Alph}

\noindent First page (\verb|\lastpageref{pagesLTS.0}|):
 \lastpageref{pagesLTS.0}\\

\noindent The page (\verb|\thepage|): \thepage \\

\noindent Current page (\verb|\theCurrentPage|),
i.\,e. counted continuously from the first page): \theCurrentPage \\

\noindent CurrentPageLocal (\verb|\theCurrentPageLocal|),
i.\,e. counted continuously from the first page of the
current page numbering scheme): \theCurrentPageLocal \\

\noindent Last roman page (pagesLTS.roman): \lastpageref{pagesLTS.roman}{\hskip4em }
(There are \lastpageref{pagesLTS.roman.local} pages with roman numbers.)\\

\noindent Last Roman page (pagesLTS.Roman): \lastpageref{pagesLTS.Roman}{\hskip3em }
(There are \lastpageref{pagesLTS.Roman.local}~pages with Roman numbers:\\
\lastpages{Roman}{1}~pages in the first Roman sector
(\pageref{Roman}{\hskip3em }-\lastpageref{pagesLTS.Roman.1}{\hskip3em }),\\
\lastpages{Roman}{2}~pages in the second Roman sector
(\pageref{Roman2}{\hskip3em }-\lastpageref{pagesLTS.Roman.2}{\hskip3em }), and\\
\lastpages{Roman}{3}~pages in the third Roman sector
(\pageref{Roman3}{\hskip3em }-\lastpageref{pagesLTS.Roman.3}{\hskip3em }).\\

\noindent Last arabic page (pagesLTS.arabic): \lastpageref{pagesLTS.arabic}{\hskip5em }
(There are only \lastpageref{pagesLTS.arabic.local} pages with arabic numbers,
because an \verb|\addtocounter{page}{|\pagesLTSexampleArabic\verb|}| was used.)\\

\noindent Last fnsymbol page (pagesLTS.fnsymbol): \lastpageref{pagesLTS.fnsymbol} \\
(\verb|\lastpageref{pagesLTS.fnsymbol}| -- never \verb|\pageref{pagesLTS.fnsymbol}|!)\\
(There are \lastpageref{pagesLTS.fnsymbol.local} pages with fnsymbol numbers.)\\

\noindent Last alph page (pagesLTS.alph): \lastpageref{pagesLTS.alph}{\hskip4em }
(There are only \lastpageref{pagesLTS.alph.local} pages with alph numbers,
because an \verb|\addtocounter{page}{|\pagesLTSexamplealph\verb|}| was used.)\\

\noindent Last Alph page (pagesLTS.Alph): \lastpageref{pagesLTS.Alph}{\hskip4em }
(There are \lastpageref{pagesLTS.Alph.local} pages with Alph numbers.)\\

\noindent Last page's \textit{name} (LastPage): \lastpageref{LastPage}\\

\noindent Very last page's \textit{name} (VeryLastPage): \lastpageref{VeryLastPage}\\
(\texttt{LastPage} and \texttt{VeryLastPage} are identical, unless
a package with output \linebreak
\verb|\AtEndDocument| after the \textsf{pageslts} package was added.)\\

\noindent Last page's \textit{number} (LastPages): \lastpageref{LastPages}{\hskip3em }
(=~total number of pages)\\

\lipsum[1-3]

Without option \texttt{alphMulti=AB} of the \textsf{pageslts} (and the help of
\textsc{Heiko Oberdiek's} \textsf{alphalph} package), after page
\textquotedblleft Z\textquotedblright{} there would just appear a
\begin{quote}
\begin{verbatim}
LaTeX Error: Counter too large
See the LaTeX manual or LaTeX Companion for explanation.
You've lost some text.  Try typing  <return>  to proceed.
If that doesn't work, type  X <return>  to quit.
\end{verbatim}
\end{quote}
Now the page numbers are continued AA, AB, AC,\ldots\ (AA, BB, CC,\ldots\ is
also possible, see the \textsf{pageslts} documentation).\\
This is not demonstrated here, but see section~\ref{alph}.

\newpage

\section{The End}

\noindent First page (\verb|\lastpageref{pagesLTS.0}|):
 \lastpageref{pagesLTS.0}\\

\noindent The page (\verb|\thepage|): \thepage \\

\noindent Current page (\verb|\theCurrentPage|),
i.\,e. counted continuously from the first page): \theCurrentPage \\

\noindent CurrentPageLocal (\verb|\theCurrentPageLocal|),
i.\,e. counted continuously from the first page of the
current page numbering scheme): \theCurrentPageLocal \\

\noindent Last roman page (pagesLTS.roman): \lastpageref{pagesLTS.roman}{\hskip4em }
(There are \lastpageref{pagesLTS.roman.local} pages with roman numbers.)\\

\noindent Last Roman page (pagesLTS.Roman): \lastpageref{pagesLTS.Roman}{\hskip3em }
(There are \lastpageref{pagesLTS.Roman.local}~pages with Roman numbers:\\
\lastpages{Roman}{1}~pages in the first Roman sector
(\pageref{Roman}{\hskip3em }-\lastpageref{pagesLTS.Roman.1}{\hskip3em }),\\
\lastpages{Roman}{2}~pages in the second Roman sector
(\pageref{Roman2}{\hskip3em }-\lastpageref{pagesLTS.Roman.2}{\hskip3em }), and\\
\lastpages{Roman}{3}~pages in the third Roman sector
(\pageref{Roman3}{\hskip3em }-\lastpageref{pagesLTS.Roman.3}{\hskip3em }).\\

\noindent Last arabic page (pagesLTS.arabic): \lastpageref{pagesLTS.arabic}{\hskip5em }
(There are only \lastpageref{pagesLTS.arabic.local} pages with arabic numbers,
because an \verb|\addtocounter{page}{|\pagesLTSexampleArabic\verb|}| was used.)\\

\noindent Last fnsymbol page (pagesLTS.fnsymbol): \lastpageref{pagesLTS.fnsymbol} \\
(\verb|\lastpageref{pagesLTS.fnsymbol}| -- never \verb|\pageref{pagesLTS.fnsymbol}|!)\\
(There are \lastpageref{pagesLTS.fnsymbol.local} pages with fnsymbol numbers.)\\

\noindent Last alph page (pagesLTS.alph): \lastpageref{pagesLTS.alph}{\hskip4em }
(There are only \lastpageref{pagesLTS.alph.local} pages with alph numbers,
because an \verb|\addtocounter{page}{|\pagesLTSexamplealph\verb|}| was used.)\\

\noindent Last Alph page (pagesLTS.Alph): \lastpageref{pagesLTS.Alph}{\hskip4em }
(There are \lastpageref{pagesLTS.Alph.local} pages with Alph numbers.)\\

\noindent Last page's \textit{name} (LastPage): \lastpageref{LastPage}\\

\noindent Very last page's \textit{name} (VeryLastPage): \lastpageref{VeryLastPage}\\
(\texttt{LastPage} and \texttt{VeryLastPage} are identical, unless
a package with output \linebreak
\verb|\AtEndDocument| after the \textsf{pageslts} package was added.)\\

\noindent Last page's \textit{number} (LastPages): \lastpageref{LastPages}{\hskip3em }
(=~total number of pages)\\

\medskip

\noindent Page \thepage\ (\theCurrentPage; local: \theCurrentPageLocal) of %
\lastpageref{pagesLTS.roman} (\lastpageref{pagesLTS.roman.local}) + %
\lastpageref{pagesLTS.Roman} (\lastpageref{pagesLTS.Roman.local}) + %
\lastpageref{pagesLTS.arabic} (\lastpageref{pagesLTS.arabic.local}) + %
\lastpageref{pagesLTS.fnsymbol} (\lastpageref{pagesLTS.fnsymbol.local}) + %
\lastpageref{pagesLTS.alph} (\lastpageref{pagesLTS.alph.local}) + %
\lastpageref{pagesLTS.Alph} (\lastpageref{pagesLTS.Alph.local}) = %
\lastpageref{LastPages} pages.

\end{document}
%</example>
%    \end{macrocode}
%
% \pagebreak
%
% \StopEventually{}
%
% \section{The implementation}
%
% (This and the source code of the example file are the reasons
% for printing the documentation in landscape format instead of portrait.)\\
%
% We start off by checking that we are loading into \LaTeXe{} and
% announcing the name and version of this package.
%
%    \begin{macrocode}
%<*package>
%    \end{macrocode}
%
%    \begin{macrocode}
\NeedsTeXFormat{LaTeX2e}[2015/01/01]
\ProvidesPackage{pageslts}[2015/12/21 v1.2f
            Refers to special pages' numbers/names (HMM)]

%    \end{macrocode}
%
% A short description of the \xpackage{pageslts} package:
%
%    \begin{macrocode}
%% Allows for things like\\
%% |Page \thepage\ (\theCurrentPage; local: \theCurrentPageLocal) of %
%%  \lastpageref{pagesLTS.roman}(\lastpageref{pagesLTS.roman.local}) + %
%%  \lastpageref{pagesLTS.Roman}(\lastpageref{pagesLTS.Roman.local}) + %
%%  \lastpageref{pagesLTS.arabic}(\lastpageref{pagesLTS.arabic.local}) + %
%%  \lastpageref{pagesLTS.fnsymbol}(\lastpageref{pagesLTS.fnsymbol.local}) + %
%%  \lastpageref{pagesLTS.alph}(\lastpageref{pagesLTS.alph.local}) + %
%%  \lastpageref{pagesLTS.Alph}(\lastpageref{pagesLTS.Alph.local}) = %
%%  \lastpageref{LastPages} pages.|\\
%% to get\\
%% 'Page d (57; local: 4) of ii(2) + XX(20) + *(1) + 30(30) + e(5) + C(3) = 61 pages.'.

%    \end{macrocode}
%
% The package is now named `pageslts' but had been named `pagesLTS' before.
% \LaTeX\ does not load a package two times, but it is case sensitive,
% i.\,e. \emph{would} load `pageslts' and `pagesLTS', resulting in
% possible problems. Therefore we check for a loaded `pagesLTS' package
% (assuming |\pagesLTS@loaded| is not |p@gesLTSlo@ded| and not defined to
% |p@gesLTSlo@ded| by another package). This check might work only for
% the second and following compilation runs.
%
%    \begin{macrocode}
\@ifpackageloaded{pagesLTS}{%
  \def\pagesLTS@doubleload{p@gesLTSlo@ded}
  \ifx\pagesLTS@loaded\pagesLTS@doubleload%
    \PackageError{pageslts}{Requested pagesLTS instead of pageslts}{%
      You have requested package `pagesLTS'. This package is now named `pageslts'.\MessageBreak%
      Requesting `pagesLTS' as well as `pageslts' leads to loading the same package\MessageBreak%
      twice (and results in errors). Loading of `pageslts' will therefore be\MessageBreak%
      abandoned now. To fix this problem, please look in the \jobname.log file for\MessageBreak%
      LaTeX Warning: You have requested package `pagesLTS',\MessageBreak%
      \space \space \space \space \space \space \space \space \space \space \space \space \space%
      \space \space but the package provides `pageslts'.\MessageBreak%
      and change the package call in the calling style/document.\MessageBreak%
      If a style file made the mistake, please inform its maintainer and/or the\MessageBreak%
      maintainer of the pageslts package about it (after making sure you have\MessageBreak%
      the recent version of that style file).\MessageBreak%
      Thank you and sorry for the inconvenience!\MessageBreak%
     }
    \endinput
  \fi
  }{% \relax
   }

\gdef\pagesLTS@loaded{p@gesLTSlo@ded}

%    \end{macrocode}
%
% For |\ltx@ifpackageloaded| the \xpackage{ltxcmds} package is needed,
% also by \textsc{Heiko Oberdiek} (see subsection~\ref{ss:Downloads}):
%
%    \begin{macrocode}
\RequirePackage{ltxcmds}[2011/11/09]%        v1.22
%    \end{macrocode}
%
% For its |\AfterLastShipout| (as well as its |\AtEndAfterFileList|) command
% we need the \xpackage{atveryend} package by \textsc{Heiko Oberdiek}
% (see subsection~\ref{ss:Downloads}):
%
%    \begin{macrocode}
\RequirePackage{atveryend}[2011/06/30]%      v1.8
\AtBeginDocument{%
  %% https://tex.stackexchange.com/a/45654/6865
  %% and https://tex.stackexchange.com/q/261445/6865
  \def\pagesLTS@etb{etoolbox.sty}%
  \let\pagesLTS@ave\@empty%
  \@for\@pageslts@currname:=\@filelist\do{%
     \ifx\@pageslts@currname\pagesLTS@etb\relax%
       \def\pagesLTS@ave{atveryend.sty}%
     \fi%
     \ifx\@pageslts@currname\pagesLTS@ave%
       \@ifpackagelater{atveryend}{2015/08/01}{%\relax
          \PackageInfo{pageslts}{etoolbox package loaded before atveryend package\MessageBreak%
            With atveryend version 1.8 of 2011/06/30 this caused a problem,\MessageBreak%
            which is probably solved in the new version used here.\MessageBreak%
            To play it safe, you could load the atveryend package\MessageBreak%
            before the etoolbox package.\MessageBreak%
            }%
        }{% else
          \PackageError{pageslts}{etoolbox package loaded before atveryend package}{%
            You can use both packages, but when etoolbox is loaded before atveryend,\MessageBreak%
            atveryend no longer detects \string\enddocument\space\MessageBreak% 
            and hook "AtVeryVeryEnd" is not executed.\MessageBreak%
            Just load the atveryend package before the etoolbox package!\MessageBreak%
            }%
         }%
     \fi%
   }%
  }

%    \end{macrocode}
%
% For its |\EveryShipout| command we need the \xpackage{everyshi}
% package by \textsc{Martin Schr\"{o}der} (see subsection~\ref{ss:Downloads}):
%
%    \begin{macrocode}
\RequirePackage{everyshi}[2001/05/15]%       v3.00
%    \end{macrocode}
%
% For its |\LetLtxMacro| command we need the \xpackage{letltxmacro}
% package by \textsc{Heiko Oberdiek} (see subsection~\ref{ss:Downloads}):
%
%    \begin{macrocode}
\RequirePackage{letltxmacro}[2010/09/02]%    v1.4
%    \end{macrocode}
%
% For the handling of the options we need the \xpackage{kvoptions}
% package also by \textsc{Heiko Oberdiek} (see subsection~\ref{ss:Downloads}):
%
%    \begin{macrocode}
\RequirePackage{kvoptions}[2011/06/30]%      v3.11
%    \end{macrocode}
%
% The \xpackage{undolabl} package of \textsc{H.-Martin M\"{u}nch} (i.\,e. myself),
% with code from \textsc{Ulrich Diez}, (see subsection~\ref{ss:Downloads})
% is needed to overwrite labels, when the same page numbering scheme
% is used twice (or even more often).
%
%    \begin{macrocode}
\RequirePackage{undolabl}[2015/03/29]%       v1.0l
%    \end{macrocode}
%
% We use the \xpackage{rerunfilecheck} package by \textsc{Heiko Oberdiek}
% to make sure that the user gets and sees the rerun warnings (if any).
%
%    \begin{macrocode}
\RequirePackage{rerunfilecheck}[2011/04/15]% v1.7
%    \end{macrocode}
%
% We must not forget to give the source of |Prelim@EveryShipout|:
%
%    \begin{macrocode}
%% pageslts package uses Prelim@EveryShipout code from the
%% prelim2e package [2009/05/29 v1.3] by Martin Schr\"{o}der, thanks!

%    \end{macrocode}
%
% About the \xpackage{prelim2e} package by \textsc{Martin Schr\"{o}der}
% see subsection~\ref{ss:Downloads}.\\
%
% A last information for the user(s):
%
%    \begin{macrocode}
%% pageslts may work with earlier versions of LaTeX2e and those packages,
%% but this was not tested. Please consider updating your LaTeX
%% and packages to the most recent version (if they are not already
%% the most recent version).
%    \end{macrocode}
% See subsection~\ref{ss:Downloads} about how to get them.\\
%
% \pagebreak
%
% The very old version~2.0 (and earlier) of the \xpackage{endfloat}
% package actually redefined the |\enddocument| command,
% and so interfered drastically with the \LaTeXe\ commands which
% make use of |\AtEndDocument|. Newer versions of \xpackage{endfloat}
% exists (at the time of writing this documentation: v2.5d as of 2011/12/25)
% in modern documentation form, which are available from \url{https://www.CTAN.org}
% (see subsection~\ref{ss:Downloads}).
% A~note is placed here, and later it is checked whether a (very) old
% \xpackage{endfloat} package is in use. If it is, a warning or even
% an error message is given, depending on \xpackage{endfloat} version.
% This assumes, that the old versions of \xpackage{endfloat} at least gave
% a version date, of course.
%
%    \begin{macrocode}
%% The recent version of the endfloat package is v2.5d as of 2011/12/25.
%% The pageslts package is not fully compatible with version 2.0
%% (and earlier) of the endfloat package, because those versions
%% redefined the \enddocument command.

%    \end{macrocode}
%
% The options are introduced:
%
%    \begin{macrocode}
\SetupKeyvalOptions{family = pagesLTS,prefix = pagesLTS@}
\DeclareBoolOption[true]{pagecontinue}% \pagesLTS@pagecontinue
\DeclareStringOption[ab]{alphMult}
\DeclareStringOption[AB]{AlphMulti}
\DeclareBoolOption[true]{romanMult}
\DeclareBoolOption[true]{RomanMulti}
\DeclareBoolOption[true]{fnsymbolmult}

\ProcessKeyvalOptions*

%    \end{macrocode}
%
% For comparisons, zero, one, two and three are defined (|\z@|, |\@ne| and so on
% do not work for this).
%
%    \begin{macrocode}
\def\pagesLTS@zero{0}
\def\pagesLTS@one{1}
\def\pagesLTS@two{2}
\def\pagesLTS@three{3}

%    \end{macrocode}
%
% The traditional behaviour is a reset of the page number to one,
% each time the page numbering scheme changes. The option
% |pagecontinue| changes this to a continuation with the
% number/name following the last page number/name of the same
% page numbering scheme. The user is informed accordingly.
%
%    \begin{macrocode}
\ifpagesLTS@pagecontinue%
  \PackageInfo{pageslts}{Option pagecontinue enabled\MessageBreak%
    (maybe by default):\MessageBreak%
    The pageslts package will continue the page numbering,\MessageBreak%
    when the same page numbering scheme is used twice.\MessageBreak%
    If you do not want this, call pageslts with option\MessageBreak%
    pagecontinue=false (or use \string\setcounter{page}=1).\MessageBreak%
  }
\else
%    \end{macrocode}
%
% \pagebreak
%
%    \begin{macrocode}
  \PackageWarningNoLine{pageslts}{Option pagecontinue is false:\MessageBreak%
    The pageslts package was used, but the option\MessageBreak%
    pagecontinue was set to false.\MessageBreak%
    If you want the page numbers to be continued,\MessageBreak%
    when the same page numbering scheme is used twice,\MessageBreak%
    please call pageslts with option pagecontinue=true,\MessageBreak%
    otherwise the page number is reset to one each time\MessageBreak%
    the page numbering scheme is changed.\MessageBreak%
    For details please see the documentation%
  }
\fi

%    \end{macrocode}
%
% The page number printed in \texttt{alph} or in \texttt{Alph}
% page numbering scheme had to be $>0$ and \nolinebreak{$<27$.}\ %
% Now the \xpackage{alphalph} package allows to extend the
% numbering scheme (not only for pages). Because some users
% prefer aa, ab, ac, ad,\ldots\ and some aa, bb, cc, dd,\ldots,
% both schemes can be chosen via the options.
% The \texttt{fnsymbol} page numbering scheme was restricted to values
% $>0$ and \nolinebreak{$<10$.}\ The \xpackage{alphalph} package allows
% to extend this page numbering scheme, too. Option \texttt{fnsymbolmult}
% can be chosen with the \xpackage{pageslts} package.
% If no extension is wished (or another extension is wished
% and implemented manually), \xpackage{pageslts} can be called with
% options set to 0 (zero) and false: |alphMult=0|, |AlphMulti=0|,
% |fnsymbolmult=false|.
%
%    \begin{macrocode}
\def\pagesLTS@ab{ab}
\def\pagesLTS@bb{bb}
\def\pagesLTS@ABi{AB}
\def\pagesLTS@BBi{BB}

\ifx\pagesLTS@alphMult\pagesLTS@ab%
\else
  \ifx\pagesLTS@alphMult\pagesLTS@bb%
  \else
    \ifx\pagesLTS@alphMult\pagesLTS@zero%
      \PackageWarningNoLine{pageslts}{%
        Option alphMult=0 found:\MessageBreak%
        The pageslts package was used, but the option\MessageBreak%
        alphMult was set to 0 (zero).\MessageBreak%
        If you want the page numbers to be extended\MessageBreak%
        after z, you have to organize this yourself now.\MessageBreak%
        For automatic continuation, please use the\MessageBreak%
        alphalph package and call pageslts\MessageBreak%
        with option alphMult=ab (for aa, ab, ac, ad,...) or\MessageBreak%
        with option alphMult=bb (for aa, bb, cc, dd,...).\MessageBreak%
        For details please see the documentation!%
       }
    \else
      \PackageError{pageslts}{Unknown option value}{%
        The pageslts package was used with option\MessageBreak%
        alphMult= \pagesLTS@alphMult . Only values\MessageBreak%
        ab, bb, and 0 (zero) are valid.\MessageBreak%
        The default ab is set.\MessageBreak%
        For details please see the documentation!%
       }
      \setkeys{pagesLTS}{alphMult=ab}
    \fi
  \fi
\fi

\ifx\pagesLTS@AlphMulti\pagesLTS@ABi%
\else
  \ifx\pagesLTS@AlphMulti\pagesLTS@BBi%
  \else
    \ifx\pagesLTS@AlphMulti\pagesLTS@zero%
      \PackageWarningNoLine{pageslts}{%
        Option AlphMulti=0 found:\MessageBreak%
        The pageslts package was used, but the option\MessageBreak%
        AlphMulti was set to 0 (zero).\MessageBreak%
        If you want the page numbers to be extended\MessageBreak%
        after Z, you have to organize this yourself now.\MessageBreak%
        For automatic continuation, please use the\MessageBreak%
        alphalph package and call pageslts\MessageBreak%
        with option AlphMulti=AB (for AA, AB, AC, AD,...) or\MessageBreak%
        with option AlphMulti=BB (for AA, BB, CC, DD,...).\MessageBreak%
        For details please see the documentation!%
       }
    \else
      \PackageError{pageslts}{Unknown option value}{%
        The pageslts package was used with option\MessageBreak%
        AlphMulti= \pagesLTS@AlphMulti . Only values\MessageBreak%
        AB, BB, and 0 (zero) are valid.\MessageBreak%
        The default AB is set.\MessageBreak%
        For details please see the documentation!%
       }
      \setkeys{pagesLTS}{AlphMulti=AB}
    \fi
  \fi
\fi

%    \end{macrocode}
%
% If |alph| or |Alph| or |fnsymbol| page numbers shall be continued,
% the \xpackage{alphalph} package is required.
%
%    \begin{macrocode}
\ifx\pagesLTS@alphMult\pagesLTS@zero% \relax
\else \RequirePackage{alphalph}[2011/05/13]% v2.4
\fi

\ifx\pagesLTS@AlphMulti\pagesLTS@zero% \relax
\else \RequirePackage{alphalph}[2011/05/13]% v2.4
\fi

\ifpagesLTS@fnsymbolmult%
  \RequirePackage{alphalph}[2011/05/13]% v2.4
\fi

%    \end{macrocode}
%
% For the roman page numbering scheme, it is just the
% choice of an extension by \xpackage{pageslts} or not.
%
%    \begin{macrocode}
\ifpagesLTS@romanMult%
  \PackageInfo{pageslts}{Option romanMult enabled\MessageBreak%
    (maybe by default):\MessageBreak%
    The pageslts package will extend the page numbering\MessageBreak%
    of the roman scheme below i with\MessageBreak%
     0, -i, -ii, -iii, -iv,...\MessageBreak%
    If you do not want this, call pageslts with option\MessageBreak%
    romanMult=false.\MessageBreak%
  }
\else
  \PackageWarningNoLine{pageslts}{Option romanMult is set to false:\MessageBreak%
  The pageslts package was used, but the option\MessageBreak%
  romanMult was set to false.\MessageBreak%
  If you want the page numbering of the roman scheme\MessageBreak%
  to be extended below i,\MessageBreak%
  please call pageslts with option romanMult=true,\MessageBreak%
  otherwise zero and negative page numbers of the\MessageBreak%
  roman scheme will need to be defined otherwise.\MessageBreak%
  For details please see the documentation!\MessageBreak%
  }
\fi

%    \end{macrocode}
%
% Same for the Roman page numbering scheme.
%
%    \begin{macrocode}
\ifpagesLTS@RomanMulti%
  \PackageInfo{pageslts}{Option RomanMulti enabled\MessageBreak%
    (maybe by default):\MessageBreak%
    The pageslts package will extend the page numbering\MessageBreak%
    of the Roman scheme below I with\MessageBreak%
    0, -I, -II, -III, -IV,...\MessageBreak%
    If you do not want this, call pageslts with option\MessageBreak%
    RomanMulti=false.\MessageBreak%
   }
\else
  \PackageWarningNoLine{pageslts}{Option RomanMulti is set to false:\MessageBreak%
    The pageslts package was used, but the option\MessageBreak%
    RomanMulti was set to false.\MessageBreak%
    If you want the page numbering of the Roman scheme\MessageBreak%
    to be extended below i,\MessageBreak%
    please call pageslts with option RomanMulti=true,\MessageBreak%
    otherwise zero and negative page numbers of the\MessageBreak%
    Roman scheme will need to be defined otherwise.\MessageBreak%
    For details please see the documentation!\MessageBreak%
   }
\fi

%    \end{macrocode}
%
% For the footnotesymbol page numbering scheme, it is also
% just the choice of a extension by \xpackage{pageslts} or not.
%
%    \begin{macrocode}
\ifpagesLTS@fnsymbolmult%
  \PackageInfo{pageslts}{Option fnsymbolmult enabled\MessageBreak%
    (maybe by default):\MessageBreak%
    The pageslts package will extend the page numbering\MessageBreak%
    of the footnotesymbol scheme using the alphalph\MessageBreak%
    package.\MessageBreak%
    If you do not want this, call pageslts with option\MessageBreak%
    fnsymbolmult=false.\MessageBreak%
   }
\else
  \PackageWarningNoLine{pageslts}{%
    Option fnsymbolmult is set to false:\MessageBreak%
    The pageslts package was used, but the option\MessageBreak%
    fnsymbolmult was set to false.\MessageBreak%
    If you want the page numbering of the footnotesymbol\MessageBreak%
    scheme to be extended using the alphalph package,\MessageBreak%
    please call pageslts with option fnsymbolmult=true,\MessageBreak%
    otherwise page numbers of the footnotesymbol scheme\MessageBreak%
    greater than nine will need to be defined otherwise.\MessageBreak%
    For details please see the documentation!\MessageBreak%
   }
\fi

%    \end{macrocode}
%
% Now defining some variables, place-holders, and abbreviations:
%
%    \begin{macrocode}
\def\pagesLTS@pnc{0}
\def\pagesLTS@called{0}
\def\pagesLTS@fns{fnsymbol}
\def\pagesLTS@alph{alph}
\def\pagesLTS@Alph{Alph}
\def\pagesLTS@rerun{0}
\def\pagesLTS@eso{0}
\def\pagesLTS@esov{0}
\def\lastpageref{\lastpagereftxt}
\def\pagesLTS@undolable{none}
\def\pncmissing{0}
\def\pagesLTS@SK{0}
\def\pagesLTS@messageNPN{%
 The pageslts package was used, but\MessageBreak%
 \string\pagenumbering \MessageBreak%
 was not called at the beginning of the document\MessageBreak%
 (maybe earlier or later).\MessageBreak%
 Please use \string\pagenumbering \MessageBreak%
 (with an argument like arabic, roman, Roman,\MessageBreak%
 fnsymbol, alph, or Alph) at the beginning\MessageBreak%
 of your document! Otherwise your document\MessageBreak%
 will probably compile, but the pageslts\MessageBreak%
 package might not be able to get all labels\MessageBreak%
 for the references to the respective pages\MessageBreak%
 right.\MessageBreak%
 }

%    \end{macrocode}
%
% It is checked whether writing to files is allowed.
% The \xpackage{pageslts} package cannot be used without that!
% Some packages (e.\,g. \xpackage{tikz} and \xpackage{selectp}) sometimes prevent
% the output to the \xfile{aux} file. In that case a warning or an error message
% is issued. This is no problem as long as there is/was another compilation run
% where the labels can/could be processed via the \xfile{aux} file.\\
% If it is allowed to write to the \xfile{aux} file, we define
% |\pagesLTS@loaded| as |p@gesLTSnotlo@ded| for the next compilation run.
%
%    \begin{macrocode}
\AtBeginDocument{%
  \if@filesw%
%    \end{macrocode}
%
% \noindent then |\relax|.\newline
% When writing to files is not allowed, nothing can be done.
% But when the labels were already processed via the \xfile{aux} file,
% nothing needs to be done (if enough compilation runs have been done before).
%
%    \begin{macrocode}
  \else%
    \@ifpackageloaded{tikz}{%
      \PackageWarning{pageslts}%
        {It was not allowed to write to an\MessageBreak%
         .aux file. This package does not work without access\MessageBreak%
         to an .aux file.\MessageBreak%
         It is OK if the .aux file was already updated\MessageBreak%
         by a previouse compiler run\MessageBreak%
         and would not have changed anyway.\MessageBreak%
        }%
    }{\PackageError{pageslts}{No writing to auxiliary file allowed}%
        {It was not allowed to write to an .aux file.\MessageBreak%
         This package does not work without access to an .aux file.\MessageBreak%
         Press Ctrl+Z to exit.\MessageBreak%
         But it is OK if the .aux file was already updated\MessageBreak%
         by previouse compiler runs\MessageBreak%
         and would not have changed anyway.\MessageBreak%
         (In that case just press Enter or Return\MessageBreak%
         to continue the compilation.)\MessageBreak%
        }%
     }%
  \fi%
  }

%    \end{macrocode}
%
% \begin{macro}{\pagenumbering}
% \indent \qquad To keep the original meaning of |\pagenumbering|:
%
%    \begin{macrocode}
\LetLtxMacro{\OrigPagenumbering}{\pagenumbering}

%    \end{macrocode}
% \end{macro}
%
% Defining some new counters (and doing related things):
%
%    \begin{macrocode}
\newcounter{CurrentPage}
\setcounter{CurrentPage}{1}
\def\theCurrentPageLocal{\arabic{pagesLTS.current.local.\pagesLTS@pnc}}
%    \end{macrocode}
%
% The counter |pagesLTS.pagenr| is for saving the total page
% number of the last page in the \xfile{.aux} file.
%
%    \begin{macrocode}
\newcounter{pagesLTS.pagenr}
%    \end{macrocode}
%
% While generally
% |\pagesLTS@ifcounter{pagesLTS.current.local.\pagesLTS@pnc}|
% is used, for the beginning of the document
% |pagesLTS.current.local.0| is predefined.
% (A~|\pagesLTS@ifcounter{pagesLTS.current.local.\pagesLTS@pnc}|
% could be used for this, too, but we know that |pagesLTS.current.local.0|
% was not defined, so we can just do the definition here.)
% And the first local page gets the number one.
%
%    \begin{macrocode}
\newcounter{pagesLTS.current.local.0}
\setcounter{pagesLTS.current.local.0}{1}
%    \end{macrocode}
%
% And the same again for |pagesLTS.pnc.0|.
%
%    \begin{macrocode}
\newcounter{pagesLTS.pnc.0}

%    \end{macrocode}
%
% \phantomsection
% \begin{macro}{\xroman}\label{xroman}
% When |\roman{...}| is used with a value $<1$, \LaTeX\ just ignores this
% (see subsection~\ref{ss:overflow}). Here we provide a command
% |\xroman{...}| \nolinebreak{(e\textbf{x}panded} roman),
% which gives the usual |\roman| numbers (i, ii, iii, iv,\ldots)
% for positive values,
% $-\left\vert \ldots \right\vert $ (i.\,e. -i, -ii, -iii, -iv,\ldots)
% for negative values, and 0 for all other values (which should be zero).
%
%    \begin{macrocode}
\newcommand{\xroman}[1]{%
  \ifnum\value{#1}>0%
    \roman{#1}%
  \else%
    \ifnum\value{#1}<0%
%    \end{macrocode}
%
% |\arabic{#1}| gives the arabic number of argument |#1|,
% which is negative here (for example $-7$),
% \textquotedblleft |-|\textquotedblright{} puts another minus
% sign in front of it (for example $--7$), |\number| removes
% all unnecessary preceding zeros, plus and minus signs
% (for example $7$), |\romannumeral| turns it into a |roman|
% number (for example vii), and
% \textquotedblleft |-|\textquotedblright{} puts the minus
% sign back in front of it (for example -vii).
%
%    \begin{macrocode}
      -\romannumeral\number-\arabic{#1}%
    \else%
      0%
    \fi%
  \fi%
  }

%    \end{macrocode}
%  \end{macro}
%
% \pagebreak
%
% \begin{macro}{\XRoman}
% |\XRoman| does the same for uppercase |\Roman| numbers.
% |-\uppercase{\romannumeral\number-\arabic{#1}}| cannot be used,
% because the result in the example is |-\uppercase{vii}| and
% not |-VII|.\footnote{This does not matter for the print out,
% but for the display of the logical page numbers
% as well as the \xfile{.aux} file.} Therefore we have a look
% at \LaTeX's own |\@Roman\FOOcounter|,\\
% |\def\@Roman#1{\expandafter\@slowromancap\romannumeral #1@}|,\\
% and use |\@slowromancap|, which is a fully expandable macro,
% to do the trick for this:
%
% \noindent \textquotedblleft
% \begin{quote}
%\begin{verbatim}
%\def\@slowromancap#1{\ifx @#1% then terminate
%\else
%\if i#1I\else\if v#1V\else\if x#1X\else\if l#1L\else\if
%c#1C\else\if d#1D\else \if m#1M\else#1\fi\fi\fi\fi\fi\fi\fi
%\expandafter\@slowromancap
%\fi
%}
%\end{verbatim}
% \end{quote}
% \textquotedblright{} (1998/05/16 Version v1.1g LaTeX Kernel File~m
% ltcounts.dtx 105 Counters and Lengths).
%
%    \begin{macrocode}
\newcommand{\XRoman}[1]{%
  \ifnum\value{#1}>0%
    \Roman{#1}%
  \else%
    \ifnum\value{#1}<0%
      -\expandafter\@slowromancap\romannumeral\number-\arabic{#1}@%
    \else%
      0%
    \fi%
  \fi%
  }

%    \end{macrocode}
%  \end{macro}
%
% \begin{macro}{\XXRoman}
% In older versions |\XXRoman| was used. For compatibility, it is forwarded to |\XRoman|
% and an error message is given.
%
%    \begin{macrocode}
\newcommand{\XXRoman}[1]{\XRoman{#1}%
  \PackageError{pageslts}{Old command \string\XXRoman\space found}{Replaced by \string\XRoman.}%
  }

%    \end{macrocode}
%  \end{macro}
%
% \begin{macro}{\pagesLTS@ifcounter}
% We provide a way to create counters like
% \begin{itemize}
% \item[-] \DescribeMacro{pagesLTS.pnc. page numbering scheme }
%   |pagesLTS.pnc.<page numbering scheme>|, e.\,g. |pagesLTS.pnc.Roman|,
% \item[-] \DescribeMacro{pagesLTS.double. page numbering scheme }
%   |pagesLTS.double.<page numbering scheme>|, e.\,g. |pagesLTS.double.Roman|,
% \item[-] \DescribeMacro{PageCurrentLocal. page numbering scheme }
%   |PageCurrentLocal.<page numbering scheme>|, e.\,g. |PageCurrentLocal.Roman|,
% \end{itemize}
% for all page numbering schemes, even those not supported
% by the current original |\pagenumbering|
% (1994/05/19 v1.1a LaTeX Kernel File w ltpageno.dtx 52 Page Numbering),
% which is\\
% \begin{quote}
%\begin{verbatim}
%\countdef\c@page=0 \c@page=1
%\def\cl@page{}
%\def\pagenumbering#1{%
%  \global\c@page \@ne \gdef\thepage{\csname @#1\endcsname
%   \c@page}}
%\end{verbatim}
% \end{quote}
%
%    \begin{macrocode}
\newcommand{\pagesLTS@ifcounter}[1]{%
  \@ifundefined{c@#1}{\newcounter{#1}}{\relax}%
  }

%    \end{macrocode}
%  \end{macro}
%
% \begin{macro}{\lastpages}
% We provide a command to give the number of pages in a sector
% of a split page numbering scheme (see page~\pageref{pagesLTS.pnscheme.nr},\\
% |pagesLTS.<page numbering scheme>.<number>.local.cnt|):
%
%    \begin{macrocode}
\newcommand{\lastpages}[2]{%
  \pagesLTS@ifcounter{pagesLTS.#1.#2.local.cnt}%
  \arabic{pagesLTS.#1.#2.local.cnt}%
  }

%    \end{macrocode}
%  \end{macro}
%
% \begin{macro}{\pagesLTS@writelabel}
% At last defining the writing of a label:
%
%    \begin{macrocode}
\newcommand{\pagesLTS@writelabel}[1]{%
  \addtocounter{page}{+1}%
%    \end{macrocode}
%
% \noindent |\addtocounter{page}{+1}| because |\pagesLTS@putlabel| includes an\\
% |\addtocounter{page}{-1}|, which is not necessary here.\\
% Into the \xfile{.aux} file something like\\
% |\newlabel{pagesLTS.Roman}{{}{VIII}{}{page.VIII}{}}|\\
% is written, thus |\lastpageref{pagesLTS.Roman}| prints |VIII| and
% links to |page.VIII|.
%
%    \begin{macrocode}
  \pagesLTS@putlabel{pagesLTS.#1}{\thepage}{1}
  \addtocounter{page}{-1}%
  \ifx\pagesLTS@pnc\pagesLTS@zero% \relax
%    \end{macrocode}
%
% i.\,e. if the current page numbering scheme is
% \textquotedblleft 0\textquotedblright, i.\,e. before the first
% |\pagenumbering{...}| command, do nothing,
%
%    \begin{macrocode}
  \else%
    \addtocounter{page}{+1}%
    \pagesLTS@putlabel{pagesLTS.#1.local}{\theCurrentPageLocal}{1}
%    \end{macrocode}
%
% otherwise write into the \xfile{.aux} file something like\\
% |\newlabel{pagesLTS.arabic.local}{{}{5}{}{page.8}{}}|,\\
% thus |\lastpageref{pagesLTS.arabic.local}| prints |5| and
% links to |page.8|. Here (and in the example file) it is not
% \textquotedblleft print |8| and link to |page.8|\textquotedblright ,
% because |\addtocounter{page}{3}| has been used, thus the page with
% \textquotedblleft number\textquotedblright{} (name) |8| is the
% \textbf{fifth} $(=8-3)$ page.
%
%    \begin{macrocode}
    \addtocounter{page}{-1}%
  \fi%
  }

%    \end{macrocode}
% \end{macro}
%
% \begin{macro}{\erroralphalph}
% \indent |\erroralphalph| extends the
% \textquotedblleft numbers\textquotedblright{} of counters
% to zero and negative values for representations usually
% not supporting this: |\alphalph|, |\AlphAlph|, and
% |\fnsymbolmult| of the \xpackage{alphalph} package.\\
% |\alph|, |\Alph|, and |\fnsymbol| would not support
% \textquotedblleft numbers\textquotedblright{} below one.\\
% |\arabic| already supports negative numbers and zero.\\
% |\roman| and |\Roman| support neither negative numbers
% nor zero, but are expanded in this package
% (|\xroman| and |\XRoman|), see page~\pageref{xroman}.\\
%
%    \begin{macrocode}

%% The following code is from Heiko Oberdiek [2010/04/18],              %%
%% expanding his alphalph package as of 2010/04/18, v2.3. (Thanks!)     %%
\newcommand*{\erroralphalph}[2]{%
  \ifnum\value{#2}>0%
    #1{\value{#2}}%
  \else%
    \ifnum\value{#2}<0%
      -#1{\expandafter\@gobble\the\value{#2}}%
    \else%
      0%
    \fi%
  \fi%
  }
%% End of code from Heiko Oberdiek                                      %%
%% Check and Error/Warning messages have been moved to \EveryShipout,   %%
%% because messages inside e.g. the \pageref command can cause trouble. %%

%    \end{macrocode}
% \end{macro}
%
% \begin{macro}{\expandPagenumbering}
% \indent Here the |\erroralphalph| command is called with the
% appropriate arguments for each page numbering scheme.
%
%    \begin{macrocode}

\newcommand{\expandPagenumbering}[1]{%
  \let\Origthepage\thepage%
  \def\pagesLTS@tmpC{arabic}%
  \ifx\pagesLTS@pnc\pagesLTS@tmpC% \relax
%    \end{macrocode}
%
% |\arabic| already supports negative numbers and zero
% (|-MAX...MAX|, where |MAX|\,$ = 2\,147\,483\,647$).
%
%    \begin{macrocode}
  \else%
    \def\pagesLTS@tmpC{roman}%
    \ifx\pagesLTS@pnc\pagesLTS@tmpC%
      \ifpagesLTS@romanMult%
%    \end{macrocode}
%
% |\erroralphalph{\roman}{page}| cannot be used, because
% |-\roman{\expandafter\@gobble\the\value{page}}| does not work.
% If option |romanMult| is not false,
% |\xroman| (see page~\pageref{xroman}) expands the usable
% roman page numbers to values below 1 (i, I, respectively),
% see subsubsection~\ref{sss:romanMult}.
%
%    \begin{macrocode}
        \renewcommand*{\thepage}{\xroman{page}}%
      \fi%
    \else%
      \def\pagesLTS@tmpC{Roman}%
      \ifx\pagesLTS@pnc\pagesLTS@tmpC%
        \ifpagesLTS@RomanMulti%
%    \end{macrocode}
%
% The same for |\Roman| page numbering, expanded by |\XRoman|
% (see page~\pageref{xroman}).
%
%    \begin{macrocode}
          \renewcommand*{\thepage}{\XRoman{page}}%
        \fi%
      \else%
        \ifx\pagesLTS@pnc\pagesLTS@alph%
%    \end{macrocode}
%
% |\alph| and |\Alph| page numberings are expanded to negative
% and zero values, and to values greater than
% \textquotedblleft z\textquotedblright{} or
% \textquotedblleft Z\textquotedblright{} with the
% \xpackage{alphalph} package.~--
% If |\pagesLTS@alphMult| was zero, nothing is done.
%
%    \begin{macrocode}
          \ifx\pagesLTS@alphMult\pagesLTS@ab%
            \renewcommand*{\thepage}{\erroralphalph{\alphalph}{page}}%
          \else \ifx\pagesLTS@alphMult\pagesLTS@bb%
                  \renewcommand*{\thepage}{\erroralphalph{\alphMult}{page}}%
                \fi%
          \fi%
        \else%
          \ifx\pagesLTS@pnc\pagesLTS@Alph%
            \ifx\pagesLTS@AlphMulti\pagesLTS@ABi%
              \renewcommand*{\thepage}{\erroralphalph{\AlphAlph}{page}}%
            \else \ifx\pagesLTS@AlphMulti\pagesLTS@BBi%
                    \renewcommand*{\thepage}{\erroralphalph{\AlphMult}{page}}%
                  \fi%
            \fi%
          \else%
            \ifx\pagesLTS@pnc\pagesLTS@fns%
%    \end{macrocode}
%
% Same for |\fnsymbol| page numbers.
%
%    \begin{macrocode}
              \ifpagesLTS@fnsymbolmult%
                \renewcommand*{\thepage}{\erroralphalph{\fnsymbolmult}{page}}%
              \fi%
            \else%
%    \end{macrocode}
%
% If the used page numbering scheme has not been recognized by the
% \xpackage{pageslts} package so far, we can do nothing, and problems
% might result.\\
%
%    \begin{macrocode}
              \PackageError{pageslts}{unknown page numbering scheme}{%
                The pageslts package encountered the unknown\MessageBreak%
                page numbering scheme\MessageBreak%
                `#1'. \MessageBreak%
                If this is no typing mistake, it might work\MessageBreak%
                - or it might not work.\MessageBreak%
                \@ehc%
                }
            \fi%
          \fi%
        \fi%
      \fi%
    \fi%
  \fi%
  \let\pagesLTS@tmpC\undefined%
  }

%    \end{macrocode}
% \end{macro}
%
% \begin{macro}{\pagenumbering}
% \indent Now for the \textbf{new} version of the |\pagenumbering| command:
%
%    \begin{macrocode}
\renewcommand{\pagenumbering}[1]{%
%    \end{macrocode}
%
% If the current page numbering scheme, |\pagesLTS@pnc|, or the
% requested page numbering scheme, |#1|, is |\pagesLTS@fns|,
% i.\,e. \texttt{fnsymbol}, the counter\texttt{pagesLTS.fnsymbol.local}
% is needed. If it does not exists yet, it is created here.
%
%    \begin{macrocode}
  \edef\pagesLTS@tmpA{#1}%
  \ifx\pagesLTS@pnc\pagesLTS@fns%
    \pagesLTS@ifcounter{pagesLTS.fnsymbol.local}%
  \fi%
  \ifx\pagesLTS@tmpA\pagesLTS@fns%
    \pagesLTS@ifcounter{pagesLTS.fnsymbol.local}%
  \fi%
%    \end{macrocode}
%
% If the current page numbering scheme, |\pagesLTS@pnc|, and the
% requested page numbering scheme, |#1|, is the same one, nothing
% further is done, otherwise the real action begins.
%
%    \begin{macrocode}
  \ifx\pagesLTS@pnc\pagesLTS@tmpA% \relax
  \else%
%    \end{macrocode}
%
% The next code is executed, when we are at a page after the first one.
% This distinction is done for two reasons:\\
% On the one hand, |\pagenumbering| could be called \textit{before}
% |\begin{document}| (where the current page should not be
% greater than one), and on the other hand we go one page back to
% aim all references to that page. Obviously at the first page
% there is no going backward.
%
%    \begin{macrocode}
    \ifnum \value{CurrentPage}>1%
      \addtocounter{page}{-1}%
      \addtocounter{pagesLTS.current.local.\pagesLTS@pnc}{-1}%
%    \end{macrocode}
%
% For the case that the page numbering scheme is or will be split,
% like e.\,g. the Roman one in the |pageslts-example.tex|, a counter
% like
% |pagesLTS.Roman.1.local.count|
% (or |pagesLTS.Roman.2.local.count|,
% |pagesLTS.Roman.3.local.count|,\ldots ) is introduced and set to the
% number of the local page.
%
%    \begin{macrocode}
      \newcounter{pagesLTS.\pagesLTS@pnc.\arabic{pagesLTS.pnc.\pagesLTS@pnc}.local.count}%
      \setcounter{pagesLTS.\pagesLTS@pnc.\arabic{pagesLTS.pnc.\pagesLTS@pnc}.local.count}{%
        \value{pagesLTS.current.local.\pagesLTS@pnc}}%
%    \end{macrocode}
%
% If the page numbering scheme is \xpackage{fnsymbol},
% and if it \textit{was} used before,
% from said counter the number of pages of the preceding uses
% of the same page numbering scheme, |pagesLTS.\pagesLTS@pnc.done|,
% is subtracted (same as for the other schemes, see below). Instead of introducing a new
% counter (which can be problematic, when the number of available counters is limited),
% we borrow the \texttt{pagesLTS.pnc.0} counter, i.\,e. we save its value to |\pagesLTS@tmpa|,
% (ab)use the counter, and then set it back to its former value as saved in |\pagesLTS@tmpa|.
%
%    \begin{macrocode}
      \ifx\pagesLTS@pnc\pagesLTS@fns%
        \ifnum \value{pagesLTS.pnc.\pagesLTS@pnc}>1%
          \mathchardef\pagesLTS@tmpa=\arabic{pagesLTS.pnc.0}%
          \setcounter{pagesLTS.pnc.0}{\value{pagesLTS.pnc.\pagesLTS@pnc}}%
          \addtocounter{pagesLTS.pnc.0}{-1}%
          \addtocounter{pagesLTS.\pagesLTS@pnc.\arabic{pagesLTS.pnc.\pagesLTS@pnc}.local.count}{%
            -\value{pagesLTS.\pagesLTS@pnc.\arabic{pagesLTS.pnc.0}.local.count}}%
          \setcounter{pagesLTS.pnc.0}{\pagesLTS@tmpa}%
        \fi%
%    \end{macrocode}
%
% If the page numbering scheme is \textit{not} \xpackage{fnsymbol},
% a~numbered label is written:
%
%    \begin{macrocode}
      \else%
        \pagesLTS@writelabel{\pagesLTS@pnc.\arabic{pagesLTS.pnc.\pagesLTS@pnc}}%
%    \end{macrocode}
%
% If the page numbering scheme was \textit{not} used before,
%
%    \begin{macrocode}
        \ifnum \value{pagesLTS.pnc.\pagesLTS@pnc}<2%
%    \end{macrocode}
%
% an unnumbered label is also written:
%
%    \begin{macrocode}
          \pagesLTS@writelabel{\pagesLTS@pnc}%
%    \end{macrocode}
%
% If the page numbering scheme \textit{was} used before,
% from said counter the number of pages of the preceding uses
% of the same page numbering scheme, |pagesLTS.\pagesLTS@pnc.done|,
% is subtracted. Instead of introducing a new
% counter (which can be problematic, when the number of available counters is limited),
% we again borrow the \texttt{pagesLTS.pnc.0} counter (see above).
%
%    \begin{macrocode}
        \else%
          \mathchardef\pagesLTS@tmpa=\arabic{pagesLTS.pnc.0}%
          \setcounter{pagesLTS.pnc.0}{\value{pagesLTS.pnc.\pagesLTS@pnc}}%
          \addtocounter{pagesLTS.pnc.0}{-1}%
          \pagesLTS@ifcounter{pagesLTS.\pagesLTS@pnc.done}%
          \addtocounter{pagesLTS.\pagesLTS@pnc.done}{%
            \value{pagesLTS.\pagesLTS@pnc.\arabic{pagesLTS.pnc.0}.local.count}}%
          \addtocounter{pagesLTS.\pagesLTS@pnc.\arabic{pagesLTS.pnc.\pagesLTS@pnc}.local.count}{%
            -\value{pagesLTS.\pagesLTS@pnc.done}}%
          \setcounter{pagesLTS.pnc.0}{\pagesLTS@tmpa}%
        \fi%
%    \end{macrocode}
%
% The values are written to the \xfile{.aux} file
% (if writing is allowed: |\if@filesw|),
% because they must be available at the beginning of the document:
%
%    \begin{macrocode}
        \if@filesw%
          \immediate\write\@auxout{\string
            \pagesLTS@ifcounter{pagesLTS.\pagesLTS@pnc.\arabic{pagesLTS.pnc.\pagesLTS@pnc}.local.cnt}}%
        \fi%
        \edef\pagesLTS@tmpB{\arabic{pagesLTS.\pagesLTS@pnc.\arabic{pagesLTS.pnc.\pagesLTS@pnc}.local.count}}%
        \if@filesw%
          \immediate\write\@auxout{\string
            \setcounter{pagesLTS.\pagesLTS@pnc.\arabic{pagesLTS.pnc.\pagesLTS@pnc}.local.cnt}{\pagesLTS@tmpB}}%
        \fi%
      \fi%
%    \end{macrocode}
%
% For further code for the case of \xpackage{fnsymbol} please see below
% (|\lastpagereftext|, page~\pageref{lastpagereftextlabel}).\\
% The last page number is saved, in case the same page
% numbering scheme is continued later.
%
%    \begin{macrocode}
      \pagesLTS@ifcounter{pagesLTS.double.\pagesLTS@pnc}%
      \setcounter{pagesLTS.double.\pagesLTS@pnc}{\value{page}}%
%    \end{macrocode}
%
% We went back one page, so we must go forward again:
%
%    \begin{macrocode}
      \addtocounter{pagesLTS.current.local.\pagesLTS@pnc}{+1}%
      \addtocounter{page}{+1}%
%    \end{macrocode}
%
% The page numbering scheme |\pagesLTS@pnc| is now set to
% the new one, given by the user as argument with the
% |\pagenumbering{...}| command:
%
%    \begin{macrocode}
      \xdef\pagesLTS@pnc{#1}%
%    \end{macrocode}
%
% The new page numbering scheme is now started for real:
%
%    \begin{macrocode}
      \OrigPagenumbering{#1}%
%    \end{macrocode}
%
% If a page numbering scheme not known by the original
% |\pagenumbering{...}| command is used, an error will
% arise here - but maybe without error message.\\
%
% If page numbering schem |\alph|, |\Alph|, or |\fnsymbol| is used,
% \xpackage{pageslts} extends the page numbers according to
% the given options, using the \xpackage{alphalph} package.
% |\arabic| does not need any expansion.
% |\roman| and |\Roman| at least receive a definition for
% zero.
%
%    \begin{macrocode}
      \expandPagenumbering{#1}%
%    \end{macrocode}
%
% Counters like |pagesLTS.pnc.Roman| are introduced:
%
%    \begin{macrocode}
      \pagesLTS@ifcounter{pagesLTS.pnc.\pagesLTS@pnc}%
%    \end{macrocode}
%
% The saved number of times this page numbering scheme was used
% is increased by one:
%
%    \begin{macrocode}
      \addtocounter{pagesLTS.pnc.\pagesLTS@pnc}{1}%
%    \end{macrocode}
%
% \indent Now defining the counter |pagesLTS.double.\pagesLTS@pnc|,
% if it did not exist already, adding~1, because this is
% the first page of it (or another one, if the scheme is
% continued):
%
%    \begin{macrocode}
      \pagesLTS@ifcounter{pagesLTS.double.\pagesLTS@pnc}%
      \addtocounter{pagesLTS.double.\pagesLTS@pnc}{1}%
%    \end{macrocode}
%
% The page number is continued, if the option |pagecontinue=false|
% is \textbf{not} set, otherwise it is reset to one. Note that neither
% the local nor the current counter are reset, as they contain
% the real \textit{values} and not the \textit{names} of the pages.
%
%    \begin{macrocode}
      \ifpagesLTS@pagecontinue%
        \setcounter{page}{\value{pagesLTS.double.\pagesLTS@pnc}}%
      \else%
        \setcounter{page}{1}%
      \fi%
%    \end{macrocode}
%
% If it does not exist already, the counter
% |pagesLTS.current.local.\pagesLTS@pnc|
% (e.\,g. |pagesLTS.current.local.Roman|) is created.
%
%    \begin{macrocode}
      \pagesLTS@ifcounter{pagesLTS.current.local.\pagesLTS@pnc}%
%    \end{macrocode}
%
% If |pagesLTS.double.\pagesLTS@pnc| of the current page numbering
% scheme is equal to one, this is the first page of this page
% numbering scheme. Then |pagesLTS.current.local.\pagesLTS@pnc|
% (which was zero) is set to one.
%
%    \begin{macrocode}
      \ifnum \value{pagesLTS.double.\pagesLTS@pnc}=1%
        \setcounter{pagesLTS.current.local.\pagesLTS@pnc}{1}%
      \fi%
%    \end{macrocode}
%
% Otherwise, i.\,e. if |\value{CurrentPage}| is not |>1|,
% i.\,e. before the first page has shiped out:
%
%    \begin{macrocode}
    \else%
      %% before the first page has shipped out
%    \end{macrocode}
%
% The current page numbering scheme is defined by the
% argument of |\pagenumbering{...}|, which the user gave:
%
%    \begin{macrocode}
      \xdef\pagesLTS@pnc{#1}%
%    \end{macrocode}
%
% \noindent and the page numbering scheme set by the original
% page numbering command
% (1994/05/19 v1.1a LaTeX Kernel File w ltpageno.dtx 52 Page Numbering),
% which resets the page number to one,
% but at the first page continuation does not make sense).
% Well, nearly the original page numbering command:
% |\OrigPagenumbering{\pagesLTS@pnc}| does not work,
% so we \textquotedblleft expand\textquotedblright{} the
% |\OrigPagenumbering| command:
%
%    \begin{macrocode}
      \global\c@page \@ne\relax%
      \global\def\thepage{\csname \expandafter @\pagesLTS@pnc \endcsname \c@page}%
%    \end{macrocode}
%
% If a page numbering scheme is used, which is not known by
% \LaTeX , an error might arise here -- but maybe without error message.\\
%
% If page numbering scheme |\alph|, |\Alph|, or |\fnsymbol| is used,
% \xpackage{pageslts} extends the page numbers according to
% the given options, using the \xpackage{alphalph} package.
% |\arabic| does not need any expansion.
% |\roman| and |\Roman| at least receive a definition for
% zero.
%
%    \begin{macrocode}
      \expandPagenumbering{#1}
%    \end{macrocode}
%
% We are at the first page, so the page counters are set to one:
%
%    \begin{macrocode}
      \pagesLTS@ifcounter{pagesLTS.pnc.\pagesLTS@pnc}%
      \setcounter{pagesLTS.pnc.\pagesLTS@pnc}{1}%
      \pagesLTS@ifcounter{pagesLTS.double.\pagesLTS@pnc}%
      \setcounter{pagesLTS.double.\pagesLTS@pnc}{1}%
      \pagesLTS@ifcounter{pagesLTS.current.local.\pagesLTS@pnc}%
      \setcounter{pagesLTS.current.local.\pagesLTS@pnc}{1}%
    \fi%
%    \end{macrocode}
%
% Whether |\pagenumbering{...}| is called in the preamble,
% |\AtBeginDocument|, right after |\begin{document}|,
% or somewhere in the document, we want to remember
% \textit{whether} it was called at all:
%
%    \begin{macrocode}
    \gdef\pagesLTS@called{1}%
  \fi%
%    \end{macrocode}
%
% We do not need the temporary definitions any more.
%
%    \begin{macrocode}
  \let\pagesLTS@tmpA\undefined%
  \let\pagesLTS@tmpB\undefined%
  }

%    \end{macrocode}
% \end{macro}
%
% \begin{macro}{\lastpageref*}
% If \xpackage{hyperref} is used, but (some) references to some last page shall
% not be hyperlinked, a command |\lastpageref*| (analogous to |\pageref*|) is
% needed. Therefore we define (analogous to |\HyPsd@pageref| from
% the \xpackage{hyperref} package by \textsc{Heiko Oberdiek})
%
%    \begin{macrocode}
%% analogous to \HyPsd@pageref from the hyperref package by Heiko Oberdiek:
\def\lastpagereftxt#1{\pagesLTS@@pageref#1*\END}

%    \end{macrocode}
%
% Macro |\pagesLTSpageref| checks, whether a star is present (analogous to
% |\HyPsd@@pageref| again from the \xpackage{hyperref} package
% of \textsc{Heiko Oberdiek}):
%
%    \begin{macrocode}
\def\pagesLTS@@pageref#1*#2\END{%
  \ifx\\#2\\% no star
    \pagesLTS@@@pageref{#1}%
  \else% star
    \expandafter\pagesLTS@@@pagerefstar%
  \fi%
}

\def\pagesLTS@@@pageref#1{\lastpagereftext{#1}}
\def\pagesLTS@@@pagerefstar#1{\lastpagereftextstar{#1}}

%    \end{macrocode}
% \end{macro}
%
% \phantomsection
% \begin{macro}{\lastpagereftext} \label{lastpagereftextlabel}
% \indent When |\lastpageref| is used somewhere inside the \texttt{txt} (text),
% i.\,e. not at the last page, it is defined as |\lastpagereftxt| (see above).
% When the page numbering scheme is \xpackage{fnsymbol}, and the \xpackage{hyperref}
% package has been loaded, a hyperref instead of a label is used for the reference
% to |pagesLTS.fnsymbol.local.\arabic{pagesLTS.fnsymbol.local}|.
% (And if the \texttt{pagesLTS.fnsymbol.local} counter did not exist yet,
% it is created here.)\\
%
%    \begin{macrocode}
\newcommand{\lastpagereftext}[1]{%
  \def\pagesLTS@tmpA{#1}%
  \def\pagesLTS@tmpB{pagesLTS.fnsymbol.local}%
  \ifx\pagesLTS@tmpA\pagesLTS@tmpB%
    \pagesLTS@ifcounter{pagesLTS.fnsymbol.local}%
    \ltx@ifpackageloaded{hyperref}{%
      \href{\#pagesLTS.fnsymbol.local.\arabic{pagesLTS.fnsymbol.local}}%
        {\arabic{pagesLTS.fnsymbol.local}}%
%    \end{macrocode}
%
% When the page numbering scheme is \xpackage{fnsymbol}, but the \xpackage{hyperref} package
% has \textit{not} been loaded, just the arabic number of the
% \texttt{pagesLTS.fnsymbol.local} counter is given (because there will be
% no hyperlink anyway).
%
%    \begin{macrocode}
     }{\arabic{pagesLTS.fnsymbol.local}}%
%    \end{macrocode}
%
% Otherwise just the common |\pageref| is applied:
%
%    \begin{macrocode}
  \else%
    \pageref{#1}%
  \fi%
%    \end{macrocode}
%
% We do not need the temporary definitions any more.
%
%    \begin{macrocode}
  \let\pagesLTS@tmpA\undefined%
  \let\pagesLTS@tmpB\undefined%
  }

%    \end{macrocode}
% \end{macro}
%
% \begin{macro}{\lastpagereftextstar}
% And the same for the starred version, where no hyperlink is generated:
%
%    \begin{macrocode}
\newcommand{\lastpagereftextstar}[1]{%
  \def\pagesLTS@tmpA{#1}%
  \def\pagesLTS@tmpB{pagesLTS.fnsymbol.local}%
  \ifx\pagesLTS@tmpA\pagesLTS@tmpB%
     \pagesLTS@ifcounter{pagesLTS.fnsymbol.local}%
     \arabic{pagesLTS.fnsymbol.local}%
  \else%
%    \end{macrocode}
%
% There is no |\pageref*| without \xpackage{hyperref}.
%
%    \begin{macrocode}
    \ltx@ifpackageloaded{hyperref}{\pageref*{#1}}{\pageref{#1}}%
  \fi%
  \let\pagesLTS@tmpA\undefined%
  \let\pagesLTS@tmpB\undefined%
  }

%    \end{macrocode}
% \end{macro}
%
% \begin{macro}{\lastpagerefend}
% \indent When the \xpackage{hyperref} package is used and the page numbering scheme
% of the last page is \xpackage{fnsymbol}, |\lastpageref| is defined as
% |\lastpagerefend|.
% Hyperrefs instead of labels are used for the reference to
% \xpackage{fnsymbol} pages (including the last one).\\
%
% Again it must be discriminated between unstarred form and starred form:
%
%    \begin{macrocode}
\def\lastpagerefend#1{\pagesLTS@@pagerefend#1*\END}

\def\pagesLTS@@pagerefend#1*#2\END{%
  \ifx\\#2\\% no star
    \pagesLTS@@@pagerefend{#1}%
  \else% star
    \expandafter\pagesLTS@@@pagerefendstar%
  \fi%
}

\def\pagesLTS@@@pagerefend#1{\l@stpagerefend{#1}}
\def\pagesLTS@@@pagerefendstar#1{\l@stpagerefendstar{#1}}

%    \end{macrocode}
% \end{macro}
%
% \begin{macro}{\l@stpagerefend}
% The unstarred form (i.\,e. with hyperlinks, if \xpackage{hyperref} is loaded,
% otherwise without hyperlinks):
%
%    \begin{macrocode}
\newcommand{\l@stpagerefend}[1]{%
  \def\pagesLTS@tmpA{#1}%
  \def\pagesLTS@tmpB{pagesLTS.fnsymbol.local}%
  \ifx\pagesLTS@tmpA\pagesLTS@tmpB%
    \pagesLTS@ifcounter{pagesLTS.fnsymbol.local}%
    \ltx@ifpackageloaded{hyperref}{%
      \href{\#pagesLTS.fnsymbol.local.\arabic{pagesLTS.fnsymbol.local}}%
        {\arabic{pagesLTS.fnsymbol.local}}%
     }{\arabic{pagesLTS.fnsymbol.local}}%
  \else%
    \def\pagesLTS@tmpB{pagesLTS.fnsymbol}%
    \ifx\pagesLTS@tmpA\pagesLTS@tmpB%
      \ltx@ifpackageloaded{hyperref}{%
        \href{\#pagesLTS.fnsymbol.local.\arabic{pagesLTS.fnsymbol.local}}%
          {\pagesLTS.lastpage}%
       }{\pageref{pagesLTS.fnsymbol}}%
    \else%
      \def\pagesLTS@tmpB{LastPage}%
      \ifx\pagesLTS@tmpA\pagesLTS@tmpB%
        \ltx@ifpackageloaded{hyperref}{%
          \href{\#pagesLTS.fnsymbol.local.\pagesLTS@eso}%
            {\pagesLTS.lastpage}%
         }{\pageref{LastPage}}%
      \else%
        \def\pagesLTS@tmpB{VeryLastPage}%
        \ifx\pagesLTS@tmpA\pagesLTS@tmpB%
          \ltx@ifpackageloaded{hyperref}{%
            \href{\#pagesLTS.fnsymbol.local.\arabic{pagesLTS.fnsymbol.local}}%
              {\pagesLTS.lastpage}%
           }{\pageref{VeryLastPage}}%
        \else%
          \def\pagesLTS@tmpB{LastPages}%
          \ifx\pagesLTS@tmpA\pagesLTS@tmpB%
            \ltx@ifpackageloaded{hyperref}{%
              \href{\#pagesLTS.fnsymbol.local.\arabic{pagesLTS.fnsymbol.local}}%
                {\arabic{pagesLTS.pagenr}}%
             }{\pageref{LastPages}}%
          \else%
            \pageref{#1}%
          \fi%
        \fi%
      \fi%
    \fi%
  \fi%
%    \end{macrocode}
%
% We do not need the temporary definitions any more.
%
%    \begin{macrocode}
  \let\pagesLTS@tmpA\undefined%
  \let\pagesLTS@tmpB\undefined%
  }

%    \end{macrocode}
% \end{macro}
%
% \pagebreak
%
% \begin{macro}{\l@stpagerefendstar}%
% And the starred form, without hyperlinks, even if \xpackage{hyperref} is loaded,
% otherwise (i.\,e. without loaded \xpackage{hyperref}) this command is not called:
%
%    \begin{macrocode}
\newcommand{\l@stpagerefendstar}[1]{%
  \def\pagesLTS@tmpA{#1}%
  \def\pagesLTS@tmpB{pagesLTS.fnsymbol.local}%
  \ifx\pagesLTS@tmpA\pagesLTS@tmpB%
    \pagesLTS@ifcounter{pagesLTS.fnsymbol.local}%
    \arabic{pagesLTS.fnsymbol.local}%
  \else%
    \def\pagesLTS@tmpB{pagesLTS.fnsymbol}%
    \ifx\pagesLTS@tmpA\pagesLTS@tmpB%
      \ltx@ifpackageloaded{hyperref}{\pageref*{pagesLTS.fnsymbol}}{%
        \pageref{pagesLTS.fnsymbol}}%
    \else%
      \def\pagesLTS@tmpB{LastPage}%
      \ifx\pagesLTS@tmpA\pagesLTS@tmpB%
        \ltx@ifpackageloaded{hyperref}{\pageref*{LastPage}}{\pageref{LastPage}}%
      \else%
        \def\pagesLTS@tmpB{VeryLastPage}%
        \ifx\pagesLTS@tmpA\pagesLTS@tmpB%
          \ltx@ifpackageloaded{hyperref}{\pageref*{VeryLastPage}}{%
            \pageref{VeryLastPage}}%
        \else%
          \def\pagesLTS@tmpB{LastPages}%
          \ifx\pagesLTS@tmpA\pagesLTS@tmpB%
            \ltx@ifpackageloaded{hyperref}{\pageref*{LastPages}}{\pageref{LastPages}}%
          \else%
            \ltx@ifpackageloaded{hyperref}{\pageref*{#1}}{\pageref{#1}}%
            \fi%
          \fi%
        \fi%
      \fi%
    \fi%
  \fi%
  \let\pagesLTS@tmpA\undefined%
  \let\pagesLTS@tmpB\undefined%
  }

%    \end{macrocode}
% \end{macro}
%
% \pagebreak
%
% \begin{macro}{\overrideLTSlabel}
% |\overridelabel| from the \xpackage{undolabl} package just |\undonewlabel|s
% a~label and places a new |\label{#1}|, but we need to place a
% |\pagesLTS@putlabel{#1}{#2}|, therefore we need another command instead of
% (but somewhat similar to) |\overridelabel|:
%
%    \begin{macrocode}
% somewhat analogous to \overridelabel from the undolabl package:
\newcommand\overrideLTSlabel[2]{%
  \@bsphack
    \ifnum \value{pagesLTS.pnc.\pagesLTS@pnc}>1%
      \edef\pagesLTStmpA{#1}%
      \edef\pagesLTStmpB{pagesLTS.\pagesLTS@pnc.local}%
      \ifx\pagesLTStmpA\pagesLTStmpB%
          \immediate\write\@auxout{\string\undonewlabel{#1}}%
          \@overriddenmessage s{#1}%
      \fi%
    \fi%
    \pagesLTS@putlabel{#1}{#2}{0}%
  \@esphack%
  }

%    \end{macrocode}
% \end{macro}
%
% \begin{macro}{\@pagesLTS@Prelim@EveryShipout}
% \indent Because we cannot make references to pages with \xpackage{fnsymbol}
% page \textquotedblleft numbers\textquotedblright{} manually
% with \xpackage{hyperref}, we use \nolinebreak{|\phantomsection|\,s} and
% refer to one of those. But because we do not know how many
% \nolinebreak{|\phantomsection|\,s} and \nolinebreak{|\section*|\,s}
% are introduced by the user (or other packages;
% cf.~\LaTeX~bug~2298: knowing level of |section*|,
% \ \url{http://www.latex-project.org/cgi-bin/ltxbugs2html?category=LaTeX&responsible=anyone&state=open&keyword=&pr=latex%2F2298&search=}),
% we cannot refer to the last one as we did with the pages.
%
%    \begin{macrocode}
\newcommand{\@pagesLTS@Prelim@EveryShipout}{%
%% The following code is from the prelim2e package             %%
%% [2009/05/29 v1.3] by Martin Schr\"{o}der (Thanks!):         %%
  \bgroup
    \dimen\z@=\wd\@cclv
    \dimen\@ne=\ht\@cclv
    \dimen\tw@=\dp\@cclv
    \dimen\thr@@=\dimen1
    \advance\dimen\thr@@ by \dimen\tw@
    \global\setbox\@cclv\vbox to \dimen\thr@@{%
      \hb@xt@\dimen\z@{%
        \box\@cclv%
        \hss%
        }%
      \vbox to \z@{%
        \hb@xt@\dimen\z@{%
          \let\protect\relax
%% Code not from prelim2e package:                             %%
%    \end{macrocode}
%
% Therefore each page with \xpackage{fnsymbol} page
% \textquotedblleft number\textquotedblright{} receives a
% |\phantomsection| and a label, which includes a number increased
% by one for each page. This is done for
% |pagesLTS.fnsymbol.local.\arabic{pagesLTS.fnsymbol.cont}|
% as well as
% |pagesLTS.fnsymbol|, |pagesLTS.\pagesLTS@pnc|, and
% |pagesLTS.\pagesLTS@pnc.local|.\\
% In case an older label already existed, it is overwritten by an
% |\overridelabel| command.
%
%    \begin{macrocode}
          \ifx\pagesLTS@pnc\pagesLTS@fns%
            \pagesLTS@ifcounter{pagesLTS.fnsymbol.cont}%
            \addtocounter{pagesLTS.fnsymbol.cont}{1}%
            \ltx@ifpackageloaded{hyperref}{%
              \phantomsection%
              \hypertarget{pagesLTS.fnsymbol.local.\arabic{pagesLTS.fnsymbol.cont}}{}%
              }{}%
            \ifnum \pagesLTS@esov=\pagesLTS@zero%
              \label{pagesLTS.fnsymbol}%
            \else%
              \overridelabel{pagesLTS.fnsymbol}%
            \fi%
          \else%
            \ltx@ifpackageloaded{hyperref}{\phantomsection}{}%
            \if@filesw%
              \overridelabel{pagesLTS.\pagesLTS@pnc}%
%    \end{macrocode}
%
% We need to go forward one page (and later backward again), because
% |\overrideLTSlabel| calls a |\pagesLTS@putlabel|, and that one uses
% |\addtocounter{page}{-1}...\addtocounter{page}{+1}|, which is not
% needed here.
%
%    \begin{macrocode}
              \addtocounter{page}{+1}%
                \overrideLTSlabel{pagesLTS.\pagesLTS@pnc.local}{\theCurrentPageLocal}%
              \addtocounter{page}{-1}%
            \fi%
          \fi%
%% Code from prelim2e package again:                           %%
        }%
        \vss%
      }%
      \vss%
    }%
    \wd\@cclv=\dimen\z@
    \ht\@cclv=\dimen\@ne
    \dp\@cclv=\dimen\tw@
  \egroup
%% End of code from the prelim2e package.                      %%
  }

%    \end{macrocode}
% \end{macro}
%
% \pagebreak
%
% \begin{macro}{\EveryShipout}
% \indent At the end of each shipout, the following commands are executed:
%
%    \begin{macrocode}
\EveryShipout{%
  \ifnum\value{page}>0% \relax
  \else%
    \ifnum\value{page}=0%
      \PackageWarning{pageslts}{%
        Counter `page' is zero!\MessageBreak%
        If the page numbering scheme is not arabic\MessageBreak%
        and further not extended\MessageBreak%
        (see Page counter overflow in the pageslts\MessageBreak%
        documentation), without other measures\MessageBreak%
        this will lead to a counter overflow.\MessageBreak%
      }%
    \else%
      \ifnum\value{page}<0%
        \PackageWarning{pageslts}{%
          Counter `page' is negative: `\the\value{page}'!\MessageBreak%
          If the page numbering scheme is not arabic\MessageBreak%
          and further not extended\MessageBreak%
          (see Page counter overflow in the pageslts\MessageBreak%
          documentation), without other measures\MessageBreak%
          this will lead to a counter overflow.\MessageBreak%
        }%
      \else%
        \PackageError{pageslts}{%
          Counter `page' does not have a recognized value:\MessageBreak%
          `\the\value{page}'\MessageBreak%
          \@ehd \MessageBreak%
          }%
      \fi%
    \fi%
  \fi%
%    \end{macrocode}
%
% If the |CurrentPage| is equal to one, this is the first shipout.
%
%    \begin{macrocode}
  \ifnum \value{CurrentPage}=1% This is the first shipout!
%    \end{macrocode}
%
% We check whether some page numbering scheme was defined by
% |\pagenumbering{...}| (\textbf{as it schould be!}):
%
%    \begin{macrocode}
    \ifx\pagesLTS@called\pagesLTS@zero%
%    \end{macrocode}
%
% If it was not defined (i.\,e. |\pagesLTS@called| is zero),
% the user should be informed, that a |\pagenumbering{...}|
% is missing behind |\begin{document}|.
% Of course, it is possible that some package did some
% pages of output with |\AtBeginDocument|. In that case,
% one |\pagenumbering{...}| before |\begin{document}| and
% one |\pagenumbering{...}| (with the same argument, of course!)
% behind |\begin{document}| could help somewhat. When |\PackageError|
% was used here, the error message was not written to the screen and
% the \xfile{.log}-file, but into the document. Therefore we just make
% a note to give the error message later (|\AtEndDocument|). At that
% time unfortunately most of the document has already been compiled
% (or did not compile due to this error), but I do not know how to
% change that.
%
%    \begin{macrocode}
      \global\def\pncmissing{1}%
%    \end{macrocode}
%
% We save the current value of the page,
%
%    \begin{macrocode}
      \mathchardef\pagesLTS@tmpD=\arabic{page}%
%    \end{macrocode}
%
% determine the current page numbering scheme,
%
%    \begin{macrocode}
%% Code from Andres L\"{o}h, Universiteit Utrecht (NL)         %%
      \def\extract#1{\expandafter\extract@ #1\END}%
      \def\extract@#1\csname @#2\endcsname#3\END{#2}%
      \edef\pagesLTS@tmpQ{\extract\thepage}%
%% End of code from Andres L\"{o}h                             %%
      \let\pagesLTS@tmpP\pagesLTS@tmpQ%
%    \end{macrocode}
%
% set the current page numbering scheme to~0
% (because before the beginning of the document it should be~0),
%
%    \begin{macrocode}
      \def\pagesLTS@pnc{0}%
%    \end{macrocode}
%
% and then issue a |\pagenumbering| command with the determined
% page numbering scheme as argument:
%
%    \begin{macrocode}
      \pagenumbering{\pagesLTS@tmpP}%
%    \end{macrocode}
%
% This resets the page to one (if option |pagecontinue=false|
% was chosen), but because we do not start a new page numbering
% scheme here but manifest a page numbering scheme, which
% the user forgot to define, the page number should not have
% been reset to one. (This is the first page, but maybe the user
% wants it to have page number 2001?) Therefore we revert this here
% and set the page number to its value, which was saved before the
% |\pagenumbering| command.
%
%    \begin{macrocode}
      \setcounter{page}{\pagesLTS@tmpD}%
    \fi%
%    \end{macrocode}
%
% We are at the first page, so we put the label here.
%
%    \begin{macrocode}
    \pagesLTS@writelabel{0}%
  \fi%
%    \end{macrocode}
%
% If the current page numbering scheme |\pagesLTS@pnc| is
% |\pagesLTS@fns| (which is defined as \texttt{fnsymbol}),
% the label is set by \linebreak
% |\@pagesLTS@Prelim@EveryShipout| (see just above),
% and |\pagesLTS@esov| is set to the (real) number (not the name) of
% this page numbering scheme, |\arabic{pagesLTS.fnsymbol.cont}|.\\
% When no more pages with \xpackage{fnsymbol} page
% \textquotedblleft number\textquotedblright{} are shipped out,
% the value remains fixed and we have our reference to the last page
% of the \xpackage{fnsymbol} page numbering range.
% (At least we will have that reference after some more work,
% see below).
%
%    \begin{macrocode}
  \ifx\pagesLTS@pnc\pagesLTS@fns%
    \@pagesLTS@Prelim@EveryShipout%
    \gdef\pagesLTS@esov{\arabic{pagesLTS.fnsymbol.cont}}%
%    \end{macrocode}
%
% When another page numbering scheme was reused (in the example file
% \xpackage{Roman}), we also need to apply\\
% |\@pagesLTS@Prelim@EveryShipout|,
% because otherwise we would get multiply defined labels.
%
%    \begin{macrocode}
  \else%
    \ifnum \value{pagesLTS.pnc.\pagesLTS@pnc}>1%
      \@pagesLTS@Prelim@EveryShipout%
    \fi%
  \fi%
%    \end{macrocode}
%
% The |CurrentPage| as well as the |pagesLTS.current.local.\pagesLTS@pnc|
% are advanced by one (because one page was shipped out and the next is
% about to begin).
%
%    \begin{macrocode}
  \addtocounter{CurrentPage}{1}%
  \addtocounter{pagesLTS.current.local.\pagesLTS@pnc}{1}%
  }

%    \end{macrocode}
% \end{macro}
%
% \begin{macro}{\pagesLTS@putlabelhyper}
% \indent Here the labels are set, if the \xpackage{hyperref} package
% was loaded. Simply using |\label| would not work, because labels
% wait for the output routines to work, and there may not be any
% more invocations of the output routines. To force the write out
% we need to do an |\immediate| write.
%
%    \begin{macrocode}
\newcommand{\pagesLTS@putlabelhyper}[2]{%
  \ifHy@pageanchor \relax%
  \else%
%    \end{macrocode}
%
% If the \xpackage{hyperref} package is used, but pageanchors are disabled,
% the hyperlinking will not work.
%
%    \begin{macrocode}
    \PackageError{pagesLTS}{hyperref option pageanchor disabled}{%
      The \string\lastpageref{#1} link doesn't work\MessageBreak%
      using hyperref with disabled option `pageanchor'.\MessageBreak%
    }%
  \fi
%    \end{macrocode}
%
% If use of the \xfile{.aux}-file is allowed, the label for
% \texttt{LastPage} is written into that file, the page
% reference depending on the options, which where set for
% the \xpackage{hyperref} package.
%
%    \begin{macrocode}
%% The following code is from the hyperref package             %%
%% [2010/04/17 v6.80x; newer versions are available]           %%
%% by Heiko Oberdiek (Big Thanks!).                            %%
  \if@filesw
    \begingroup
      \let\@number\@firstofone
      \ifHy@pageanchor
        \ifHy@hypertexnames
          \ifHy@plainpages
            \def\Hy@temp{\arabic{page}}%
          \else
            \Hy@unicodefalse
%% Code not from hyperref package:                             %%
%% The following lines are modified from the hyperref package. %%
%% Without the modification, after the first shipout "PD1" is  %%
%% inserted each time |\pdfstringdef\Hy@temp{\thepage}| is     %%
%% executed (if |fnsymbol| is not used).                       %%
            \ifnum \value{CurrentPage}=1%
              \ifx\pagesLTS@pnc\pagesLTS@fns%
                \pdfstringdef\Hy@temp{\thepage}%
              \else%
                \def\Hy@temp{\thepage}%
              \fi%
            \else%
              \pdfstringdef\Hy@temp{\thepage}%
            \fi%
%% Code from hyperref package again:                           %%
          \fi
        \else
          \def\Hy@temp{\the\Hy@pagecounter}%
        \fi
      \fi
%% End of code from the hyperref package.                      %%
%% (The following four lines are modified                      %%
%%  from the hyperref package.)                                %%
      \immediate\write\@auxout{\string
        \newlabel{#1}{{}{#2}{}{\ifHy@pageanchor page.\Hy@temp\fi}{}}}%
    \endgroup%
  \fi%
  }

%    \end{macrocode}
% \end{macro}
%
% \phantomsection
% \begin{macro}{\pagesLTS@putlabel}\label{pagesLTSputlabel}
% \indent Since the page has been put out, we are on the page after
% that page. We therefore subtract one from the page counter.
%
%    \begin{macrocode}
\newcommand{\pagesLTS@putlabel}[3]{%
  \addtocounter{page}{-1}%
%    \end{macrocode}
%
% When the \xpackage{showkeys} package has been loaded in \texttt{draft} mode,
% in the margin for each label a box is displayed with the name of the label.
% \xpackage{showkeys} accomplishes this by redefining |\label|, but
% \xpackage{pageslts} does not use |\label|, but writes directly to the
% {\nobreak |\jobname|\xfile{.aux}-}\allowbreak file, and this is generally done
% \emph{after} the according page has shipped out, therefore no box can be placed
% on the \emph{preceding} page. At least \xpackage{pageslts} gives a warning, that
% \xpackage{showkeys} cannot present the respective label.
%
%    \begin{macrocode}
  \ifx\pagesLTS@SK\pagesLTS@one\relax%
    \message{^^J}%
    \message{Package pageslts Warning: Package showkeys without option final loaded,}%
    \ifnum\value{pagesLTS.pagenr}<1%
      \message{(pageslts)                but label #1 on page \thepage\space(about \theCurrentPage)}%
    \else%
      \message{(pageslts)                but label #1 on page \thepage\space(about \theCurrentPage\space of \arabic{pagesLTS.pagenr})}%
    \fi%
    \message{(pageslts)                cannot be shown, because pageslts does not use \string\label,}%
    \message{(pageslts)                but writes directly to the \jobname.aux file. ^^J}%
  \fi%
%    \end{macrocode}
%
% If the \xpackage{hyperref} package is used, the format of the labels is somewhat longer.
%
%    \begin{macrocode}
  \ltx@ifpackageloaded{hyperref}{\pagesLTS@putlabelhyper{#1}{#2}}{% else
%    \end{macrocode}
%
% \pagebreak
%
% If the \xpackage{hyperref} package is not used, there will be no hyperlinks,
% and the label is written in the way of the old \xpackage{lastpage} package.
% But we must remember to undo the label first, if it already exists.
%
%    \begin{macrocode}
    \if@filesw%
      \ifnum \value{pagesLTS.pnc.\pagesLTS@pnc}<2%
%    \end{macrocode}
%
% When the \xpackage{nameref} package is used, |\newlabel| needs five instead of two arguments:
%
%    \begin{macrocode}
        \ltx@ifpackageloaded{nameref}{%
          \immediate\write\@auxout{\string\newlabel{#1}{{}{#2}{}{}{}}}}{% else
          \immediate\write\@auxout{\string\newlabel{#1}{{}{#2}}}}%
      \else%
        \edef\pagesLTStmpA{#1}%
        \edef\pagesLTStmpB{pagesLTS.\pagesLTS@pnc.local}%
        \ifx\pagesLTStmpA\pagesLTStmpB%
          \edef\pagesLTStmpA{#3}%
          \ifx\pagesLTStmpA\pagesLTS@one%
%    \end{macrocode}
%
% Only when the third argument of |\pagesLTS@putlabel| is $1$,
% we do need to undo the label. Otherwise there is no label to
% undo, and the \xpackage{undolabl} package would give an error.
%
%    \begin{macrocode}
            \immediate\write\@auxout{\string\undonewlabel{#1}}%
          \fi%
        \fi%
        \ltx@ifpackageloaded{nameref}{%
          \immediate\write\@auxout{\string\newlabel{#1}{{}{#2}{}{}{}}}}{% else
          \immediate\write\@auxout{\string\newlabel{#1}{{}{#2}}}}%
      \fi%
    \fi%
  }%
%    \end{macrocode}
%
% After the writeout we restore the page number again,
% since there might be other things still to be done.
%
%    \begin{macrocode}
  \addtocounter{page}{+1}%
  }

%    \end{macrocode}
% \end{macro}
%
% \pagebreak
%
% \begin{macro}{\pagesLTS@putlabels}
% \indent |\pagesLTS@putlabels| is nearly identical to |\pagesLTS@putlabelV|:
%
%    \begin{macrocode}
\newcommand{\pagesLTS@putlabels}{%
  \addtocounter{page}{-1}%
  \addtocounter{CurrentPage}{-1}%
  \addtocounter{pagesLTS.current.local.\pagesLTS@pnc}{-1}%
%    \end{macrocode}
%
% If |\pagenumbering{...}| has not been used,
% |\pagesLTS@pnc| is still zero ($0$, |\pagesLTS@zero|),
% and the according warning message is given.
%
%    \begin{macrocode}
  \ifx\pagesLTS@pnc\pagesLTS@zero%
   \PackageWarning{pagesLTS}{No page numbering scheme found:\MessageBreak%
     \pagesLTS@messageNPN }%
%    \end{macrocode}
%
% \noindent otherwise the numbered label is written,
% and if the page numbering scheme was not used before,
% the unnumbered label is written, too.
%
%
%    \begin{macrocode}
  \else%
    \pagesLTS@writelabel{\pagesLTS@pnc.\arabic{pagesLTS.pnc.\pagesLTS@pnc}}%
    \ifnum \value{pagesLTS.pnc.\pagesLTS@pnc}<2%
      \ifx\pagesLTS@pnc\pagesLTS@fns% \relax
      \else%
        \pagesLTS@writelabel{\pagesLTS@pnc}%
      \fi%
    \fi%
  \fi%
%    \end{macrocode}
%
% \noindent Before the label for the \texttt{LastPages} can be put,
% we must advance one page again, because |\pagesLTS@putlabel| itself
% goes back one page (and at its end forward again).
%
%    \begin{macrocode}
  \addtocounter{page}{+1}%
  \pagesLTS@putlabel{LastPages}{\theCurrentPage}{1}%
%    \end{macrocode}
%
% \noindent Here should follow a\\
%  |\addtocounter{page}{-1}|,\\
% but we have to remember to increase the page counters again,
% which were decreased at the start of this |\pagesLTS@putlabels|
% command, and that would include\\
%  |\addtocounter{page}{+1}|,\\
% therefore this two lines cancel each other and
% therefore just can be skipped. But the other counters have to
% be increased:
%
%    \begin{macrocode}
  \addtocounter{CurrentPage}{+1}%
  \addtocounter{pagesLTS.current.local.\pagesLTS@pnc}{1}%
  }

%    \end{macrocode}
% \end{macro}
%
% \pagebreak
%
% \begin{macro}{\AtBeginDocument}
% \indent |\AtBeginDocument| we write into the \xfile{aux} file regarding
% the double loading of the package.
% Then it is checked whether the \texttt{endfloat}
% package has been loaded, whether it is newer than March 1992 (i.\,e. at least
% April 1992 v2.0), in which case it is compatible with this
% \xpackage{pageslts} package.\\
% If it is even newer than 2011/12/24, it is the recent version
% (as of the time of last revision of this documentation:
% 2011/12/25 v2.5d).
%
%    \begin{macrocode}
\AtBeginDocument{%
  \if@filesw%
    \immediate\write\@auxout{\string\gdef\string\pagesLTS@loaded{p@gesLTSnotlo@ded}}%
  \fi%
  \@ifpackageloaded{endfloat}%
   {\@ifpackagelater{endfloat}{1992/03/31}% April 1992 v2.0
    {\@ifpackagelater{endfloat}{2011/12/24}{% 2011/12/25 v2.5d
       \relax}%
%    \end{macrocode}
%
% If it is compatible, but not the recent version, a warning is given:
%
%    \begin{macrocode}
     {\PackageWarningNoLine{pageslts}{Old endfloat package detected:\MessageBreak%
        There is a newer version of the endfloat package available.\MessageBreak%
        Please consider updating your version.\MessageBreak%
        The pageslts package might be incompatible with\MessageBreak%
        your current endfloat package.\MessageBreak%
      }%
     }%
%    \end{macrocode}
%
% If it is so very old, that it is not compatible,
% an Error message is given:
%
%    \begin{macrocode}
    }{\PackageError{pageslts}{Incompatible, very old endfloat package detected.}%
      {The very old version 2.0 (and earlier) of the\MessageBreak%
       endfloat package actually redefined the \string\enddocument ,\MessageBreak%
       and so interfered drastically with the LaTeX2e commands\MessageBreak%
       which make use of \string\AtEndDocument .\MessageBreak%
       Newer versions of the endfloat package exists\MessageBreak%
       (at least: v2.5d as of 2011/12/25)\MessageBreak%
       in modern documentation form,\MessageBreak%
       which should be available from CTAN.\MessageBreak%
       Please update your endfloat package\MessageBreak%
       for use with the pageslts package.\MessageBreak %
      }%
     }%
   }{}%
%    \end{macrocode}
%
% \pagebreak
%
% It is checked whether the old \xpackage{lastpage} package has been loaded.\\
% (If it has been loaded indeed, the |\lastpage@putlabel| is
% \textquotedblleft killed\textquotedblright ,
% see subsection~\ref{ssec:lastpage}.)
%
%    \begin{macrocode}
  \@ifpackageloaded{lastpage}%
   {\@ifpackagelater{lastpage}{2010/07/28}% 2010/07/29 v1.2a
      {\@ifpackagelater{lastpage}{2013/01/27}% 2013/01/28 v1.2l
         {\PackageWarning{pageslts}{lastpage package detected.\MessageBreak%
           With pageslts package in use, lastpage has no function.\MessageBreak%
           Just remove the lastpage package from your document.\MessageBreak%
           }%
         }{%
         \PackageWarning{pageslts}{Old lastpage package detected.\MessageBreak%
           With pageslts package in use, lastpage has no function.\MessageBreak%
           Just remove the lastpage package from your document.\MessageBreak%
           At least update it!\MessageBreak%
           }%
         }%
      }{%
       \PackageWarning{pageslts}{Incompatible package lastpage detected:\MessageBreak%
         Package pageslts was loaded, but also an old\MessageBreak%
         version of the lastpage package.\MessageBreak%
         pageslts has all functionality of the lastpage\MessageBreak%
         package (and more), so just remove the lastpage\MessageBreak%
         package from your document.\MessageBreak%
         (At least update it!)\MessageBreak%
         pageslts will now ''kill'' the lastpage@putlabel\MessageBreak%
         command of the lastpage package.\MessageBreak%
       }%
       \gdef\lastpage@putlabel{\relax}%
      }%
   }{}%
%    \end{macrocode}
%
% Further it is checked whether the \xpackage{alphalph} package has been loaded.
% If that is the case, the commands are defined accordingly.\\
%
%    \begin{macrocode}
  \ltx@ifpackageloaded{alphalph}%
    {\newalphalph{\AlphMult}[mult]{\@Alph}{26}%
     \newalphalph{\alphMult}[mult]{\@alph}{26}%
     \newalphalph{\fnsymbolmult}[mult]{\@fnsymbol}{5}%
    }{}%
%    \end{macrocode}
%
% \pagebreak
%
% Further it is checked whether the \xpackage{hyperref} package has been loaded:
%
%    \begin{macrocode}
  \ltx@ifpackageloaded{hyperref}{%
%    \end{macrocode}
%
% and whether the \xpackage{pdfpages} package is loaded:
%
%    \begin{macrocode}
     \@ifpackageloaded{pdfpages}%
       {\PackageWarningNoLine{pageslts}{Package pdfpages detected.\MessageBreak%
          Using hyperref with pdfpages can cause problems. See\MessageBreak%
          https://www.ctan.org/pkg/pax\MessageBreak%
          for the Pdf Annotations eXtractor, pax%
          }%
       }{% \relax
        }%
%    \end{macrocode}
%
% The \xpackage{undolabl} package has been updated and now uses |\undonewlabel| with only
% one argument.
%
%    \begin{macrocode}
     \@ifpackageloaded{undolabl}%
       {\@ifpackagelater{undolabl}{2010/07/14}%    2010/07/15 v1.0d
          {\@ifpackagelater{undolabl}{2015/03/29}% 2015/03/29 v1.0l
             {% recent version as of the time of last revision of this package: OK
             }{% old, but not obsolete version
               \PackageWarningNoLine{pageslts}{Old version of undolabl package used.\MessageBreak%
                 See https://www.ctan.org/pkg/undolabl\MessageBreak%
                 for a new version.\MessageBreak%
                }%
             }
          }{\PackageError{pageslts}{Incompatible, obsolete version of undolabl package used.}{%
              See https://www.ctan.org/pkg/undolabl\MessageBreak%
              for a new version.\MessageBreak%
              Type  X <return>  to quit.\MessageBreak%
             }%
          }
       }{\PackageError{pageslts}{Package undolabl missing}{%
           Package undolabl not found.\MessageBreak%
           The pageslts package needs the undolabl package.\MessageBreak%
           See https://www.ctan.org/pkg/undolabl\MessageBreak%
           Type  X <return>  to quit.\MessageBreak%
           }%
       }
%    \end{macrocode}
%
% \pagebreak
%
% Additionally a version check of the available \xpackage{hyperref}
% package is performed and if need be a warning is issued:
%
%    \begin{macrocode}
     \@ifpackagelater{hyperref}{2012/11/05}{% 2012/11/06 v6.83m
       \relax}{%
       \PackageWarningNoLine{pageslts}{Old hyperref package detected:\MessageBreak%
         There is a newer version of the\MessageBreak%
         hyperref package available:\MessageBreak%
         https://www.ctan.org/pkg/hyperref\MessageBreak%
         Please consider updating your version.\MessageBreak%
       }%
     }%
     %% pageslts supports the use of the package hyperref by
     %% Heiko Oberdiek (hyperref version 2012/11/06 v6.83m).
     %% pageslts may work with earlier versions of this packages,
     %% but this was not tested.
    }{%
%    \end{macrocode}
%
% If no \xpackage{hyperref} package in use is detected, a warning is issued, too:
%
%    \begin{macrocode}
      \PackageWarning{pageslts}{Package hyperref NOT detected.\MessageBreak%
        pageslts would support hyperref. The page references\MessageBreak%
        will NOT be hyperlinked!\MessageBreak }%
     }%
%    \end{macrocode}
%
% If the \xpackage{showkeys} package has been loaded in \texttt{draft} mode,
% warnings about missing label boxes will be issued (see p.~\pageref{pagesLTSputlabel}),
% but because it cannot be checked for \xpackage{showkeys} after |\AtBeginDocument|,
% the check must be done here and the result remembered.
%
%    \begin{macrocode}
  \ltx@ifpackageloaded{showkeys}{%
    \@ifpackagewith{showkeys}{final}{}{\gdef\pagesLTS@SK{1}}}{% else \relax
    }%
  }

%    \end{macrocode}
% \end{macro}
%
% \begin{macro}{\AtEndDocument}
% \indent |\AtEndDocument| the check is repeated. If \xpackage{showkeys} was
% loaded after the check performed before, an error message is issued.
%
%    \begin{macrocode}
\AtEndDocument{%
  \ltx@ifpackageloaded{showkeys}{%
    \ifx\pagesLTS@SK\pagesLTS@one\relax%
    \else%
      \providecommand*\pagesLTS@sklf[1]{}%
      \ifx\showkeyslabelformat\pagesLTS@sklf\relax%
      \else%
        \PackageError{pageslts}{Package showkeys loaded after pageslts}{%
          The showkeys package with option "final" has been loaded\MessageBreak%
          after the pageslts package.\MessageBreak%
          Please first load showkeys and then pageslts}%
      \fi%
    \fi%
  }{% else no showcase, \relax
  }%
%    \end{macrocode}
%
% We also give the error message about the missing (i.\,e.~not found)
% page numbering scheme, which could not be given in |\EveryShipout|.
%
%    \begin{macrocode}
  \ifx\pncmissing\pagesLTS@one\relax%
    \PackageError{pageslts}{pagenumbering scheme missing}{\pagesLTS@messageNPN }%
  \fi%
%    \end{macrocode}
%
% Then we put in a |\message| to show, in what order things (which were called)
% are done (see subsection~\ref{ssec:aed}).
%
%    \begin{macrocode}
  \message{^^J%
    AED: pageslts setting LastPage ^^J}%
%    \end{macrocode}
%
% After this we issue a |\clearpage| to put out all floats,
% which are still floating, remember the page number (if |fnsymbol|),
% and after that we place the \texttt{LastPage} label.
%
%    \begin{macrocode}
  \clearpage%
  \ifx\pagesLTS@pnc\pagesLTS@fns%
    \def\pagesLTS@tmpA{\arabic{pagesLTS.fnsymbol.local}}%
    \ifnum \pagesLTS@eso=\pagesLTS@tmpA%
      \gdef\pagesLTS@rerun{0}%
    \else%
      \gdef\pagesLTS@rerun{1}%
    \fi%
    \if@filesw%
      \immediate\write\@auxout{\string\gdef\string\pagesLTS@eso{\pagesLTS@tmpA}}%
    \fi%
  \fi%
  \pagesLTS@putlabel{LastPage}{\thepage}{1}%
%    \end{macrocode}
%
% We do not need the temporary definition any more.
%
%    \begin{macrocode}
  \let\pagesLTS@tmpA\undefined%
  }

%    \end{macrocode}
% \end{macro}
%
% \begin{macro}{\AfterLastShipout}
% \indent |\AfterLastShipout| is a command from \textsc{Heiko Oberdiek's}
% \texttt{atveryend} package (see above).
%
%    \begin{macrocode}
\AfterLastShipout{%
%    \end{macrocode}
%
% If writing to the \xfile{.aux} file is allowed:
%
%    \begin{macrocode}
  \if@filesw%
%    \end{macrocode}
%
% The number of pages with the \xpackage{fnsymbol} page numbering scheme,
% |\pagesLTS@esov|, is saved via the \xfile{.aux} file (if it is not zero):
%
%    \begin{macrocode}
    \ifx\pagesLTS@esov\pagesLTS@zero%
    \else%
      \immediate\write\@auxout{\string
        \pagesLTS@ifcounter{pagesLTS.fnsymbol.local}}%
      \immediate\write\@auxout{\string
        \setcounter{pagesLTS.fnsymbol.local}{\pagesLTS@esov}}%
    \fi%
%    \end{macrocode}
%
% If the \xpackage{hyperref} package is in use, and the page numbering scheme
% of the last page is \xpackage{fnsymbol}, everything is quite more complicated.
% Therefore |\lastpageref| is switched from simple |\lastpagereftxt|
% to the more difficult |\lastpagerefend|.
%
%    \begin{macrocode}
    \ltx@ifpackageloaded{hyperref}{%
      \ifx\pagesLTS@pnc\pagesLTS@fns%
          \immediate\write\@auxout{\string
            \gdef\string\lastpageref{\string\lastpagerefend}}%
      \fi%
     }{}%
  \fi%
%    \end{macrocode}
%
% At the call of a |\pagenumbering{...}| command, everything
% for a split page numbering scheme is organized. For the last
% page numbering scheme, there is no |\pagenumbering{...}| command
% at the end, so we need to handle this here:
%
%    \begin{macrocode}
  \pagesLTS@ifcounter{pagesLTS.\pagesLTS@pnc.\arabic{pagesLTS.pnc.\pagesLTS@pnc}.local.count}%
  \setcounter{pagesLTS.\pagesLTS@pnc.\arabic{pagesLTS.pnc.\pagesLTS@pnc}.local.count}{%
    \value{pagesLTS.current.local.\pagesLTS@pnc}}%
%    \end{macrocode}
%
% And we are one page after the last one (|\AfterLastShipout|!),
% so we go back one page. (We again borrow the \texttt{pagesLTS.pnc.0} counter for the computations
% instead of defining yet another one.)
%
%    \begin{macrocode}
  \addtocounter{pagesLTS.\pagesLTS@pnc.\arabic{pagesLTS.pnc.\pagesLTS@pnc}.local.count}{-1}
  \ifnum \value{pagesLTS.pnc.\pagesLTS@pnc}>1%
    \mathchardef\pagesLTS@tmpD=\arabic{pagesLTS.pnc.0}%
    \setcounter{pagesLTS.pnc.0}{\value{pagesLTS.pnc.\pagesLTS@pnc}}%
    \@tempcnta=\value{pagesLTS.pnc.0}\relax%
    \loop%
      \ifnum\@tempcnta>1\relax%
        \addtocounter{pagesLTS.pnc.0}{-1}%
        \addtocounter{pagesLTS.\pagesLTS@pnc.\arabic{pagesLTS.pnc.\pagesLTS@pnc}.local.count}{%
          -\value{pagesLTS.\pagesLTS@pnc.\arabic{pagesLTS.pnc.0}.local.count}}%
        \@tempcnta=\value{pagesLTS.pnc.0}\relax%
    \repeat%
    \setcounter{pagesLTS.pnc.0}{\pagesLTS@tmpD}%
  \fi%
  \if@filesw%
    \immediate\write\@auxout{\string
      \pagesLTS@ifcounter{pagesLTS.\pagesLTS@pnc.\arabic{pagesLTS.pnc.\pagesLTS@pnc}.local.cnt}}%
    \edef\pagesLTS@tmpA{\arabic{pagesLTS.\pagesLTS@pnc.\arabic{pagesLTS.pnc.\pagesLTS@pnc}.local.count}}%
    \immediate\write\@auxout{\string
      \setcounter{pagesLTS.\pagesLTS@pnc.\arabic{pagesLTS.pnc.\pagesLTS@pnc}.local.cnt}{\pagesLTS@tmpA}}%
    \let\pagesLTS@tmpA\undefined%
  \fi%
%    \end{macrocode}
%
% We need to save (via the \xfile{.aux} file) the page name |\thepage| and
% the page number |\arabic{CurrentPage}| of the last page,
% in case the last page has \xpackage{fnsymbol} page numbering scheme.
%
%    \begin{macrocode}
  \addtocounter{page}{-1}%
  \edef\pagesLTS@tmpA{\thepage}%
  \if@filesw%
    \immediate\write\@auxout{\string
      \gdef\string\pagesLTS.lastpage{\pagesLTS@tmpA}}%
  \fi%
  \addtocounter{page}{+1}%
  \addtocounter{CurrentPage}{-1}%
  \edef\pagesLTS@tmpB{\arabic{CurrentPage}}%
  \if@filesw%
    \immediate\write\@auxout{\string
      \setcounter{pagesLTS.pagenr}{\pagesLTS@tmpB}}%
  \fi%
  \addtocounter{CurrentPage}{+1}%
%    \end{macrocode}
%
% The \texttt{VeryLastPage} label is set here, and when
% |\lastpageref{VeryLastPage}| instead of |\lastpageref{LastPage}| is used,
% it should really point to the last page.
% |LastPage| and |VeryLastPage| should be identical,
% unless a package was active with output |\AtEndDocument|
% \textit{after} the \xpackage{pageslts} package.
%
%    \begin{macrocode}
  \message{^^J%
    AED: pageslts setting VeryLastPage via AfterLastShipout ^^J}%
  \pagesLTS@putlabel{VeryLastPage}{\thepage}{1}%
%    \end{macrocode}
%
% The \texttt{LastPages} label is set here, and
% |\lastpageref{LastPages}| gives the total number of pages
% and points to the (very) last page.
%
%    \begin{macrocode}
  \message{^^J%
    AED: pageslts setting LastPages via AfterLastShipout ^^J}%
  \pagesLTS@putlabels%
  \typeout{^^J}%
  \ifodd\pagesLTS@tmpB%
    \@PackageInfoNoLine{pageslts}{Total number of pages is odd}%
  \else%
    \@PackageInfoNoLine{pageslts}{Total number of pages is even}%
  \fi%
  \typeout{^^J}%
%    \end{macrocode}
%
% We do not need the temporary definitions any more.
%
%    \begin{macrocode}
  \let\pagesLTS@tmpA\undefined%
  \let\pagesLTS@tmpB\undefined%
  }

%    \end{macrocode}
% \end{macro}
%
% \pagebreak
%
% \begin{macro}{\AtEndAfterFileList}
% \indent |\AtEndAfterFileList{...}| is even later:
% \begin{quote}
% \textquotedblleft After the .aux file closing and reading \LaTeX{} prints the
% file list if requested by |\listfiles|. Then this hook is executed.\textquotedblright
% \end{quote}
% (\xpackage{atveryend} package of \textsc{Heiko Oberdiek}, v1.7 as of 2011/04/23,
% newer version available.) Here it is used for a rerun hint.\\
% For example if the page numbering scheme of the last page of the
% \textsf{pageslts-example.tex} file is changed to |fnsymbol| and two runs
% of \pdfLaTeX{} are done, \pdfLaTeX{} will be happy and will not complain
% about changed labels. But indeed, a \textit{third} run is necessary
% and indicated by the warning message below.
%
%    \begin{macrocode}
\AtEndAfterFileList{%
  \ifx\pagesLTS@rerun\pagesLTS@one%
    \PackageWarningNoLine{pageslts}{%
      Label(s) may have changed.\MessageBreak%
      Rerun to get cross-references right%
     }%
  \fi%
  }

%    \end{macrocode}
% \end{macro}
%
% \begin{macro}{\frontmatter}
% \indent |\frontmatter| often contains |\pagenumbering|, but for some unknown reason
% there are problems when another |\pagenumbering| with different page numbering scheme
% has been used before on the same page. (This would not make any sense anyway,
% because one page can only have one page numbering scheme.) This problem does not
% occur when two |\pagenumbering| commands are used inside normal text.
% Thus we need to check whether |\frontmatter| has been defined,
% whether it changes the page numbering scheme, and whether the page numbering scheme
% before |\frontmatter| was initiated at the same page.
% \end{macro}
%
% We need an |\ifundefined|, which neither requires \eTeX{} nor
% defines the command, which is to be tested, to |\relax| (in which case
% it would be no longer undefined for further tests). For this we use
% code from \textsc{Markus Kohm} similar to his |\scr@ifundefinedorrelax| from %
% \xpackage{scrbase.sty} of the KOMA script bundle,
% but without the necessarity to use \eTeX :
%
%    \begin{macrocode}
\newcommand{\pagesLTS@ifundefinedorrelax}[1]{%
  % Modified Markus Kohm code for use without e-TeX
  \begingroup\expandafter\expandafter\expandafter\endgroup
  \expandafter\ifx\csname #1\endcsname\relax%
    \expandafter\ifx\csname #1\endcsname\relax%
      \expandafter\expandafter\expandafter\@firstoftwo
    \else%
      \expandafter\expandafter\expandafter\@secondoftwo
    \fi%
  \else%
    \expandafter\@firstoftwo
  \fi%
  }

\pagesLTS@ifundefinedorrelax{frontmatter}{% \relax
}{% else
  \ltx@GlobalPrependToMacro{\frontmatter}{\pagesLTS@prefrontmatter}%
  \ltx@GlobalAppendToMacro{\frontmatter}{\pagesLTS@postfrontmatter}%
 }

%    \end{macrocode}
%
% \begin{macro}{\pagesLTS@prefrontmatter}
%
% At the beginning of the real (i.e. unchanged) |\frontmatter| (FMB), we remember the
% current (c) page numbering (pn) scheme: its name (n) and page number (p).
% |\pagesLTS@FMBpncp| would be |1| if the according |\pagenumbering| command
% was used on the same page at the |\frontmatter|.
%
%    \begin{macrocode}
\newcommand{\pagesLTS@prefrontmatter}{%
  \xdef\pagesLTS@FMBpncn{\pagesLTS@pnc}%
  \xdef\pagesLTS@FMBpncp{\arabic{pagesLTS.current.local.\pagesLTS@pnc}}%
  }

%    \end{macrocode}
% \end{macro}
%
% \begin{macro}{\pagesLTS@postfrontmatter}
%
% The named checks are performed and in case of the named problem an error message
% is given. Any idea how to automatically solve this issue?
%
%    \begin{macrocode}
\newcommand{\pagesLTS@postfrontmatter}{%
  \xdef\pagesLTS@FMEpncn{\pagesLTS@pnc}%
  \ifx\pagesLTS@FMBpncn\pagesLTS@FMEpncn%
  \else%
    \ifx\pagesLTS@FMBpncn\pagesLTS@zero%
    \else%
      \ifx\pagesLTS@FMBpncp\pagesLTS@one%
        \PackageError{pageslts}{\string\pagenumbering\space before \string\frontmatter}{%
          Do not use \string\pagenumbering{\pagesLTS@FMBpncn} before \string\frontmatter \MessageBreak%
          on the same page!\MessageBreak%
          \string\frontmatter\space (re)defines the page numbering scheme to \pagesLTS@FMEpncn , \MessageBreak%
          thus earlier use of \string\pagenumbering{\pagesLTS@FMBpncn} on the same page is useless anyway.}%
      \fi%
    \fi%
  \fi%
}

%    \end{macrocode}
% \end{macro}
%
%    \begin{macrocode}
%</package>
%    \end{macrocode}
%
% \newpage
%
% \section{Installation}
%
% \subsection{Downloads\label{ss:Downloads}}
%
% Everything is available at \url{https://www.ctan.org},
% but may need additional packages themselves.\\
%
% \DescribeMacro{pageslts.dtx}
% For unpacking the |thumbs.dtx| file and constructing the documentation it is required:
% \begin{description}
% \item[-] \TeX Format \LaTeXe{}: \url{https://www.CTAN.org}
%
% \item[-] document class \xclass{ltxdoc}, 2015-03-26, 2.0w, \url{https://www.ctan.org/pkg/ltxdoc}
%
% \item[-] package \xpackage{holtxdoc}, 2012/03/21, v0.24, \url{https://www.ctan.org/pkg/holtxdoc}
%
% \item[-] package \xpackage{hypdoc}, 2011/08/19, v1.11, \url{https://www.ctan.org/pkg/hypdoc}
%
% \item[-] package \xpackage{geometry}, 2010/09/12, v5.6, \url{https://www.ctan.org/pkg/geometry}
%
% \item[-] package \xpackage{ulem}, 2012/05/18, no version number given, \url{https://www.ctan.org/pkg/ulem}
% \end{description}
%
% \DescribeMacro{pageslts.sty}
% The |pageslts.sty| for \LaTeXe{} (i.\,e. all documents using
% the \xpackage{pageslts} package) requires:
% \begin{description}
% \item[-] \TeX{} Format \LaTeXe{}, \url{https://www.CTAN.org}
%
% \item[-] package \xpackage{atveryend}, 2011/06/30, v1.8, \url{https://www.ctan.org/pkg/atveryend}
%
% \item[-] package \xpackage{everyshi}, 2001/05/15, v3.00, \url{https://www.ctan.org/pkg/everyshi}
%
% \item[-] package \xpackage{kvoptions}, 2011/06/30, v3.11, \url{https://www.ctan.org/pkg/kvoptions}
%
% \item[-] package \xpackage{letltxmacro}, 2010/09/02, v1.4, \url{https://www.ctan.org/pkg/letltxmacro}
%
% \item[-] package \xpackage{ltxcmds}, 2011/11/09, v1.22, \url{https://www.ctan.org/pkg/ltxcmds}
%
% \item[-] package \xpackage{rerunfilecheck}, 2011/04/15, v1.7, \url{https://www.ctan.org/pkg/rerunfilecheck}
%
% \item[-] package \xpackage{undolabl}, 2015/03/29, v1.0l, \url{https://www.ctan.org/pkg/undolabl}\\
% \end{description}
%
% \DescribeMacro{pageslts-example.tex}
% The |pageslts-example.tex| requires the same files as all
% documents using the \xpackage{pageslts} package, and additionally:
% \begin{description}
% \item[-] class \textsf{article}, 2014/09/29, v1.4h, from \textsf{classes.dtx}: \url{https://www.ctan.org/pkg/classes}
%
% \item[-] package \xpackage{alphalph}, 2011/05/13, v2.4, \url{https://www.ctan.org/pkg/alphalph}
%
% \item[-] package \xpackage{lipsum}, 2014/07/27, v1.3, \url{https://www.ctan.org/pkg/lipsum}
%
% \item[-] package \xpackage{showkeys}, 2014/10/28, v3.17, \url{https://www.ctan.org/pkg/showkeys}
%
% \item[-] package \xpackage{hyperref}, 2012/11/06, v6.83m, \url{https://www.ctan.org/pkg/hyperref}
%
% \item[-] package \xpackage{pageslts}, 2015/12/21, v1.2f, \url{https://www.ctan.org/pkg/pageslts}\\
%   (Well, it is the example file for this package, and because you are reading the
%    documentation for the \xpackage{pageslts} package, it can be assumed that you already
%    have some version of it~-- is it the current one?)
% \end{description}
%
% \pagebreak
%
% \DescribeMacro{papermas}
% The \xpackage{papermas} package is not required, but requires itself the
% the \xpackage{pageslts} package and can be considered as kind of add-on:
% \begin{description}
% \item[-] package \xpackage{papermas}, 2011/08/22, v1.0h, \url{https://www.ctan.org/pkg/papermas}
% \end{description}
%
% \DescribeMacro{endfloat}
% The \xpackage{endfloat} package is not required, but because
% the \xpackage{pageslts} package is incompatibel with very old versions
% of the \xpackage{endfloat} package (see subsection~\ref{sec:endfloat}),
% here the recent one is listed:
% \begin{description}
% \item[-] package \xpackage{endfloat}, 2011/12/25, v2.5d, \url{https://www.ctan.org/pkg/endfloat}
% \end{description}
%
% \DescribeMacro{prelim2e}
% The \xpackage{prelim2e} package is not required either, but because
% |Prelim@EveryShipout| code was taken from that package, it is listed, too:
% \begin{description}
% \item[-] package \xpackage{prelim2e}, 2009/05/29, v1.3, \url{https://www.ctan.org/pkg/prelim2e}
% \end{description}
%
% \DescribeMacro{fancyhdr}
% \DescribeMacro{nccfancyhdr}
% Neither \xpackage{fancyhdr} nor \xpackage{nccfancyhdr} package is required
% (the \xpackage{lastpage} package used its predecessor \xpackage{fancyheadings}),
% but because they were mentioned, also they are listed here:
% \begin{description}
% \item[-] package \xpackage{fancyhdr}, \url{https://www.ctan.org/pkg/fancyhdr}
%
% \item[-] package \xpackage{nccfancyhdr}, \url{https://www.ctan.org/pkg/nccfancyhdr}
% \end{description}
%
% \DescribeMacro{lastpage}
% \DescribeMacro{totpages}
% \DescribeMacro{totcount}
% \DescribeMacro{nofm}
% \DescribeMacro{count1to}
% \DescribeMacro{zref}
% As possible alternatives in section \ref{sec:Alternatives} there are listed
% \begin{description}
% \item[-] package \xpackage{lastpage}, 2015/03/29, v1.2m, \url{https://www.ctan.org/pkg/lastpage}
%
% \item[-] package \xpackage{totpages}, 2005/09/19, v2.00, \url{https://www.ctan.org/pkg/totpages}
%
% \item[-] package \xpackage{totcount}, 2011/01/25, v1.2, \url{https://www.ctan.org/pkg/totcount}
%
% \item[-] package \xpackage{nofm}, 1991/02/25, without version number,
%  \url{http://mirror.ctan.org/obsolete/macros/latex209/contrib/misc/nofm.sty},
%  does not work with e.\,g. \xpackage{hyperref}
%
% \item[-] package \xpackage{count1to}, 2009/05/24, v2.1, \url{https://www.ctan.org/pkg/count1to}
%
% \item[-] package \xpackage{zref}, 2012/04/04, v2.24, \url{https://www.ctan.org/pkg/zref}
% \end{description}
%
% \DescribeMacro{Oberdiek}
% \DescribeMacro{alphalph}
% \DescribeMacro{atveryend}
% \DescribeMacro{holtxdoc}
% \DescribeMacro{letltxmacro}
% \DescribeMacro{ltxcmds}
% \DescribeMacro{kvoptions}
% \DescribeMacro{rerunfilecheck}
% \DescribeMacro{zref}
% All packages of \textsc{Heiko Oberdiek's} bundle `oberdiek'
% (especially \xpackage{alphalph}, \xpackage{atveryend}, \xpackage{holtxdoc}, \xpackage{letltxmacro},
% \xpackage{ltxcmds}, \xpackage{kvoptions}, \xpackage{rerunfilecheck}, and \xpackage{zref})
% are also available in a TDS compliant ZIP archive:\\
% \url{http://mirrors.ctan.org/install/macros/latex/contrib/oberdiek.tds.zip}.\\
% It is probably best to download and use this, because the packages in there
% are quite probably both recent and compatible among themselves.
%
% \DescribeMacro{hyperref}
% \noindent \xpackage{hyperref} is not included in that bundle and needs to be
% downloaded separately,\\
% \url{http://mirrors.ctan.org/install/macros/latex/contrib/hyperref.tds.zip}.
%
% \DescribeMacro{M\"{u}nch}
% A hyperlinked list of my (other) packages can be found at
% \url{https://www.ctan.org/author/muench-hm}.
%
% \subsection{Package, unpacking TDS}
% \paragraph{Package.} This package is available on \url{CTAN.org}.
% \begin{description}
% \item[\url{http://mirrors.ctan.org/macros/latex/contrib/pageslts/pageslts.dtx}]\hspace*{0.1cm} \\
%       The source file.
% \item[\url{http://mirrors.ctan.org/macros/latex/contrib/pageslts/pageslts.pdf}]\hspace*{0.1cm} \\
%       The documentation.
% \item[\url{http://mirrors.ctan.org/macros/latex/contrib/pageslts/pageslts-example.pdf}]\hspace*{0.1cm} \\
%       The compiled example file, as it should look like.
% \item[\url{http://mirrors.ctan.org/macros/latex/contrib/pageslts/README}]\hspace*{0.1cm} \\
%       The README file.
% \end{description}
% There is also a |thumbs.tds.zip| available:
% \begin{description}
% \item[\url{http://mirror.ctan.org/install/macros/latex/contrib/pageslts.tds.zip}]\hspace*{0.1cm} \\
%       Everything in TDS compliant, compiled format.
% \end{description}
% which additionally contains\\
% \begin{tabular}{ll}
% pageslts.ins & The installation file.\\
% pageslts.drv & The driver to generate the documentation.\\
% pageslts.sty & The \xext{sty}le file.\\
% pageslts-example.tex & The example file.%
% \end{tabular}
%
% \bigskip
%
% \noindent For required other packages see the preceding subsection.
%
% \paragraph{Unpacking.} The \xfile{.dtx} file is a self-extracting
% \docstrip{} archive. The files are extracted by running the
% \xfile{.dtx} through \plainTeX :
% \begin{quote}
%   \verb|tex pageslts.dtx|
% \end{quote}
%
% About generating the documentation see paragraph~\ref{GenDoc} below.\\
%
% \paragraph{TDS.} Now the different files must be moved into
% the different directories in your installation TDS tree
% (also known as \xfile{texmf} tree):
% \begin{quote}
% \def\t{^^A
% \begin{tabular}{@{}>{\ttfamily}l@{ $\rightarrow$ }>{\ttfamily}l@{}}
%   pageslts.sty & tex/latex/pageslts.sty\\
%   pageslts.pdf & doc/latex/pageslts.pdf\\
%   pageslts-example.tex & doc/latex/pageslts-example.tex\\
%   pageslts-example.pdf & doc/latex/pageslts-example.pdf\\
%   pageslts.dtx & source/latex/pageslts.dtx\\
% \end{tabular}^^A
% }^^A
% \sbox0{\t}^^A
% \ifdim\wd0>\linewidth
%   \begingroup
%     \advance\linewidth by\leftmargin
%     \advance\linewidth by\rightmargin
%   \edef\x{\endgroup
%     \def\noexpand\lw{\the\linewidth}^^A
%   }\x
%   \def\lwbox{^^A
%     \leavevmode
%     \hbox to \linewidth{^^A
%       \kern-\leftmargin\relax
%       \hss
%       \usebox0
%       \hss
%       \kern-\rightmargin\relax
%     }^^A
%   }^^A
%   \ifdim\wd0>\lw
%     \sbox0{\small\t}^^A
%     \ifdim\wd0>\linewidth
%       \ifdim\wd0>\lw
%         \sbox0{\footnotesize\t}^^A
%         \ifdim\wd0>\linewidth
%           \ifdim\wd0>\lw
%             \sbox0{\scriptsize\t}^^A
%             \ifdim\wd0>\linewidth
%               \ifdim\wd0>\lw
%                 \sbox0{\tiny\t}^^A
%                 \ifdim\wd0>\linewidth
%                   \lwbox
%                 \else
%                   \usebox0
%                 \fi
%               \else
%                 \lwbox
%               \fi
%             \else
%               \usebox0
%             \fi
%           \else
%             \lwbox
%           \fi
%         \else
%           \usebox0
%         \fi
%       \else
%         \lwbox
%       \fi
%     \else
%       \usebox0
%     \fi
%   \else
%     \lwbox
%   \fi
% \else
%   \usebox0
% \fi
% \end{quote}
% If you have a \xfile{docstrip.cfg} that configures and enables \docstrip's
% TDS installing feature, then some files can already be in the right
% place, see the documentation of \docstrip.
%
% \subsection{Refresh file name databases}
%
% If your \TeX~distribution (\teTeX, \mikTeX,\dots) relies on file name
% databases, you must refresh these. For example, \teTeX\ users run
% \verb|texhash| or \verb|mktexlsr|.
%
% \subsection{Some details for the interested}
%
% \paragraph{Unpacking with \LaTeX.}
% The \xfile{.dtx} chooses its action depending on the format:
% \begin{description}
% \item[\plainTeX:] Run \docstrip\ and extract the files.
% \item[\LaTeX:] Generate the documentation.
% \end{description}
% If you insist on using \LaTeX\ for \docstrip\ (really,
% \docstrip\ does not need \LaTeX), then inform the autodetect routine
% about your intention:
% \begin{quote}
%   \verb|latex \let\install=y% \iffalse meta-comment
%
% File: pageslts.dtx
% Version: 2015/12/21 v1.2f
%
% Copyright (C) 2010 - 2015 by
%    H.-Martin M"unch <Martin dot Muench at Uni-Bonn dot de>
% Portions of code copyrighted by other people as marked.
%
% This package first started as a revision of the lastpage
% package of Jeffrey P. Goldberg (jeffrey+news at goldmark dot org),
% but it became obvious that a replacement was needed.
%
% This work may be distributed and/or modified under the
% conditions of the LaTeX Project Public License, either
% version 1.3c of this license or (at your option) any later
% version. This version of this license is in
%    http://www.latex-project.org/lppl/lppl-1-3c.txt
% and the latest version of this license is in
%    http://www.latex-project.org/lppl.txt
% and version 1.3c or later is part of all distributions of
% LaTeX version 2005/12/01 or later.
%
% This work has the LPPL maintenance status "maintained".
%
% The Current Maintainer of this work is H.-Martin Muench.
%
% This work consists of the main source file pageslts.dtx,
% the README, and the derived files
%    pageslts.sty, pageslts.pdf,
%    pageslts.ins, pageslts.drv,
%    pageslts-example.tex, pageslts-example.pdf.
%
% 'pageslts' is available on CTAN:
% https://www.ctan.org/pkg/pageslts
%
% Also a TDS.ZIP file is provided that contains all the files
% already sorted in a TDS tree:
% http://mirrors.ctan.org/install/macros/latex/contrib/pageslts.tds.zip
%
%<*ignore>
\begingroup
  \catcode123=1 %
  \catcode125=2 %
  \def\x{LaTeX2e}%
\expandafter\endgroup
\ifcase 0\ifx\install y1\fi\expandafter
         \ifx\csname processbatchFile\endcsname\relax\else1\fi
         \ifx\fmtname\x\else 1\fi\relax
\else\csname fi\endcsname
%</ignore>
%<*install>
\input docstrip.tex
\Msg{*********************************************************************************}
\Msg{* Installation}
\Msg{* Package: pageslts 2015/12/21 v1.2f Refers to special pages' numbers/names (HMM)}
\Msg{*********************************************************************************}

\keepsilent
\askforoverwritefalse

\let\MetaPrefix\relax
\preamble

This is a generated file.

Project: pageslts
Version: 2015/12/21 v1.2f

Copyright (C) 2010 - 2015 by
    H.-Martin M"unch <Martin dot Muench at Uni-Bonn dot de>
Portions of code copyrighted by other people as marked.

IMPORTANT NOTICE:
On the first page backslash pagenumbering should be called
(with the appropriate argument out of e.g.
 * roman    - Lowercase Roman numerals: i, ii, iii, iv,...
 * Roman    - Uppercase Roman numerals: I, II, III, IV,...
 * arabic   - Arabic numerals: 1, 2, 3, 4,...
 * alph     - Lowercase letters: a, b, c, d,...
 * Alph     - Uppercase letters: A, B, C, D,...
 * fnsymbol - Footnote symbols: *, † (ddagger), ‡ (dddagger),
                                 § (mathsection),...
). The command backslash pagenumbering is renewed, but no change
should be visible to the output text, because only labels are added.
backslash OrigPagenumbering can be used to get the original meaning
of backslash pagenumbering, but in that case this package does not
see the change in the page numbering scheme and will make mistakes.
The package takes options.

The usual disclaimer applies:
If it doesn't work right that's your problem.
(Nevertheless, send an e-mail to the maintainer
 when you find an error in this package.)

This work may be distributed and/or modified under the
conditions of the LaTeX Project Public License, either
version 1.3c of this license or (at your option) any later
version. This version of this license is in
   http://www.latex-project.org/lppl/lppl-1-3c.txt
and the latest version of this license is in
   http://www.latex-project.org/lppl.txt
and version 1.3c or later is part of all distributions of
LaTeX version 2005/12/01 or later.

This work has the LPPL maintenance status "maintained".

The Current Maintainer of this work is H.-Martin Muench.

This package first started as a revision of the lastpage
package of Jeffrey P. Goldberg (jeffrey+news at goldmark dot org),
but it became obvious that a replacement was needed.

This work consists of the main source file pageslts.dtx,
the README, and the derived files
   pageslts.sty, pageslts.pdf,
   pageslts.ins, pageslts.drv,
   pageslts-example.tex, pageslts-example.pdf.

In memoriam
 Claudia Simone Barth + 1996/01/30
 Tommy Muench + 2014/01/02
 Hans-Klaus Muench + 2014/08/24

\endpreamble
\let\MetaPrefix\DoubleperCent

\generate{%
  \file{pageslts.ins}{\from{pageslts.dtx}{install}}%
  \file{pageslts.drv}{\from{pageslts.dtx}{driver}}%
  \usedir{tex/latex/pageslts}%
  \file{pageslts.sty}{\from{pageslts.dtx}{package}}%
  \usedir{doc/latex/pageslts}%
  \file{pageslts-example.tex}{\from{pageslts.dtx}{example}}%
}

\catcode32=13\relax% active space
\let =\space%
\Msg{************************************************************************}
\Msg{*}
\Msg{* To finish the installation you have to move the following}
\Msg{* file into a directory searched by TeX:}
\Msg{*}
\Msg{*  pageslts.sty}
\Msg{*}
\Msg{* To produce the documentation run the file `pageslts.drv'}
\Msg{* through (pdf)LaTeX, e.g.}
\Msg{*  pdflatex pageslts.drv}
\Msg{*  makeindex -s gind.ist pageslts.idx}
\Msg{*  pdflatex pageslts.drv}
\Msg{*  makeindex -s gind.ist pageslts.idx}
\Msg{*  pdflatex pageslts.drv}
\Msg{*}
\Msg{* At least three runs are necessary e. g. to get the}
\Msg{*  references right!}
\Msg{*}
\Msg{* Happy TeXing!}
\Msg{*}
\Msg{************************************************************************}

\endbatchfile
%</install>
%<*ignore>
\fi
%</ignore>
%
% \section{The documentation driver file}
%
% The next bit of code contains the documentation driver file for
% \TeX , i.\,e., the file that will produce the documentation you
% are currently reading. It will be extracted from this file by the
% \texttt{docstrip} programme. That is, run \LaTeX{} on \texttt{docstrip}
% and specify the \texttt{driver} option when \texttt{docstrip}
% asks for options.
%
%    \begin{macrocode}
%<*driver>
\NeedsTeXFormat{LaTeX2e}[2015/01/01]
\ProvidesFile{pageslts.drv}[2015/12/21 v1.2f
            Refers to special pages' numbers/names (HMM)]
\documentclass[landscape]{ltxdoc}[2015/03/26]% v2.0w
\usepackage{geometry}[2010/09/12]%             v5.6
\usepackage{holtxdoc}[2012/03/21]%             v0.24
%% pageslts may work with earlier versions of LaTeX2e and those
%% class and packages, but this was not tested.
%% Please consider updating your LaTeX, class, and packages
%% to the most recent version (if they are not already the most
%% recent version).
\hypersetup{%
 pdfsubject={Refers to special pages' numbers/names (HMM)},%
 pdfkeywords={LaTeX, pageslts, pagesLTS, lastpage, page number, page name, H.-Martin Muench},%
 pdfencoding=auto,%
 pdflang={en},%
 breaklinks=true,%
 linktoc=all,%
 pdfstartview=FitH,%
 pdfpagelayout=OneColumn,%
 bookmarksnumbered=true,%
 bookmarksopen=true,%
 bookmarksopenlevel=2,%
 pdfmenubar=true,%
 pdftoolbar=true,%
 pdfwindowui=true,%
 pdfnewwindow=true%
}
\usepackage{ulem}[2012/05/18]% no version is given in the ulem.sty file
\CodelineIndex
\hyphenation{created every-thing ignored}
\gdef\unit#1{\mathord{\thinspace\mathrm{#1}}}
\makeatletter
\@ifundefined{eTeX}{\gdef\eTeX{$\m@th \varepsilon $-\TeX }}{% else \relax
}
\makeatother
\begin{document}
  \DocInput{pageslts.dtx}%
\end{document}
%</driver>
%    \end{macrocode}
%
% \fi
%
% \CheckSum{1790}
%
% \CharacterTable
%  {Upper-case    \A\B\C\D\E\F\G\H\I\J\K\L\M\N\O\P\Q\R\S\T\U\V\W\X\Y\Z
%   Lower-case    \a\b\c\d\e\f\g\h\i\j\k\l\m\n\o\p\q\r\s\t\u\v\w\x\y\z
%   Digits        \0\1\2\3\4\5\6\7\8\9
%   Exclamation   \!     Double quote  \"     Hash (number) \#
%   Dollar        \$     Percent       \%     Ampersand     \&
%   Acute accent  \'     Left paren    \(     Right paren   \)
%   Asterisk      \*     Plus          \+     Comma         \,
%   Minus         \-     Point         \.     Solidus       \/
%   Colon         \:     Semicolon     \;     Less than     \<
%   Equals        \=     Greater than  \>     Question mark \?
%   Commercial at \@     Left bracket  \[     Backslash     \\
%   Right bracket \]     Circumflex    \^     Underscore    \_
%   Grave accent  \`     Left brace    \{     Vertical bar  \|
%   Right brace   \}     Tilde         \~}
%
% \GetFileInfo{pageslts.drv}
%
% \begingroup
%   \def\x{\#,\$,\^,\_,\~,\ ,\&,\{,\},\%}%
%   \makeatletter
%   \@onelevel@sanitize\x
% \expandafter\endgroup
% \expandafter\DoNotIndex\expandafter{\x}
% \expandafter\DoNotIndex\expandafter{\string\ }
% \begingroup
%   \makeatletter
%     \lccode`9=32\relax
%     \lowercase{%^^A
%       \edef\x{\noexpand\DoNotIndex{\@backslashchar9}}%^^A
%     }%^^A
%   \expandafter\endgroup\x
%
% \DoNotIndex{\",\-,\,,\\,\noindent}
% \DoNotIndex{\documentclass,\usepackage,\ProvidesPackage}
% \DoNotIndex{\NeedsTeXFormat,\TeX,\LaTeX}
% \DoNotIndex{\ldots,\thinspace,\textbackslash}
% \DoNotIndex{\begin,\end}
% \DoNotIndex{\textbf,\textit,\textquotedblleft,\textquotedblright,\textsc,\textsf,\texttt,\underline}
% \DoNotIndex{\ifx,\ifnum,\ifodd,\enduremath}
% \DoNotIndex{\newpage,\pagebreak,\clearpage,\newline,\linebreak,\nolinebreak}
% \DoNotIndex{\lipsum,\MessageBreak,\pageref,\protect}
% \DoNotIndex{\smallskip,\medskip,\bigskip,\hskip,\space,\hline}
% \DoNotIndex{\advance,\@ne}
% \DoNotIndex{\item}
% \DoNotIndex{\@PackageInfoNoLine,\@bsphack,\@esphack,\@ehc,\@ehd,\@ifundefined,%
%  \addcontentsline,\arabic,\enddocument,\endinput,\href,\makeatletter,\makeatother,%
%  \section,\subsection,\tableofcontents,\verb}
%
% \title{The \xpackage{pageslts} package}
% \date{2015/12/21 v1.2f}
% \author{H.-Martin M\"{u}nch\\\xemail{Martin.Muench at Uni-Bonn.de}}
%
% \maketitle
%
% \begin{abstract}
%  \noindent This \LaTeX{} package puts the labels \texttt{LastPage}
%  (|\AtEndDocument|) and \texttt{VeryLastPage} (|\AfterLastShipout|)
%  into the \xfile{.aux} file, allowing the user to refer
%  to the (very) last page of a document. This might be
%  particularly useful in places like headers or footers.
%  When more than one page numbering scheme is used,
%  these references do not give the total \textit{number} of pages.
%  For this case the label \texttt{LastPages} is introduced.
%  Additionally, at the last page of each page numbering
%  scheme a label \texttt{pagesLTS.<numbering scheme>} is
%  placed, where \texttt{<numbering scheme>} is e.\,g.
%  arabic, roman, Roman, alph, or Alph. For fnsymbol
%  please use |\lastpageref{pagesLTS.fnsymbol}|
%  instead of |\pageref{pagesLTS.fnsymbol}|.
%  When the same numbering scheme is used twice,
%  the page numbers are either reset to one or continued
%  automatically, depending on the option given when the package
%  is called. The command |\theCurrentPage|
%  prints the current total/absolute page number -- in contrast to
%  |\thepage|, which gives only the page
%  \textit{name} in the current page numbering scheme.
%  |\theCurrentPageLocal| gives the current
%  number of pages in the current page numbering scheme.
%  |\thepage| and |\theCurrentPageLocal| are different e.\,g.
%  when |\addtocounter{page}{|\ldots|}| or
%  |\setcounter{page}{|\ldots|}| were used.
%  At the first page of the document a label \texttt{pagesLTS.0} is
%  created. This label can be referred to, too.
%  Further labels are provided for special cases.\\
%  The \xpackage{alphalph} package is supported, i.\,e.
%  page numbers alph or Alph $>26$ and fnsymbol $>9$
%  can be used (with according options set). Even zero
%  and negative page numbers can be used with \texttt{arabic},
%  \texttt{alph}, \texttt{Alph}, \texttt{roman}, \texttt{Roman},
%  and \texttt{fnsymbol} page numbering (with \xpackage{alphalph} package
%  and according options).\\
%  |\pageref*| and |\lastpageref*|, for using \xpackage{hyperref}
%  but suppressing links, are supported.
% \end{abstract}
%
% \noindent \textbf{Please make sure to first deinstall the obsolete
% \xpackage{pagesLTS} package before installing this \xpackage{pageslts}
% package!}\\
% (There is at least one operating system which otherwise automatically
% renames \xfile{pageslts} to \xfile{pagesLTS}.)
%
% \noindent Right after |\begin{document}| a |\pagenumbering{|\ldots|}| should
% be called -- with the appropriate argument out of e.\,g.\\
% arabic (Arabic numerals: 1, 2, 3, 4,\ldots),\\
% roman (Lowercase Roman numerals: i, ii, iii, iv,\ldots),
% Roman (Uppercase Roman numerals: I, II, III, IV,\ldots),\\
% alph (Lowercase letters: a, b, c, d,\ldots),
% Alph (Uppercase letters: A, B, C, D,\ldots),\\
% fnsymbol (Footnote symbols: $*$, $\dagger $, $\ddagger $, $\mathsection $,\ldots).
%
% \noindent This package first started as a revision of the
% \xpackage{lastpage} package of \textbf{Jeffrey P. Goldberg} (Thanks!),
% but then it became obvious that a replacement was needed.
%
% \noindent Disclaimer for web links: The author is not responsible for any contents
% referred to in this work unless he has full knowledge of illegal contents.
% If any damage occurs by the use of information presented there, only the
% author of the respective pages might be liable, not the one who has referred
% to these pages.
%
% \noindent Save per page about $200\unit{ml}$ water,
% $2\unit{g}$ CO$_{2}$ and $2\unit{g}$ wood:
% Therefore please print only if this is really necessary.
%
% \bigskip
%
% \tableofcontents
%
% \newpage
%
% \section{Introduction}
%
% \indent This package puts the labels \texttt{LastPage}
% (|\AtEndDocument|) (same as my \xpackage{LastPage} package, invented by
% \textsc{Jeffrey P. Goldberg}) and \texttt{VeryLastPage} (|\AfterLastShipout|)
% into the \xfile{.aux} file, allowing the user to refer to the (very)
% last page of a document via |\lastpageref{LastPage}| and
% |\lastpageref{VeryLastPage}|. This might be particularly useful in
% places like headers or footers. When more than one page numbering scheme
% is used, these references do not give the total \textit{number} of pages.
% For this case the label \texttt{LastPages} is introduced (similar to the
% label \texttt{TotPages} of the \xpackage{TotPages} package, but
% the label \texttt{LastPages} is set later in the document).
% Additionally, at the last page of each page numbering scheme a label
% \texttt{pagesLTS.<}\textit{numbering scheme}\texttt{>} is placed,
% where \texttt{<}\textit{numbering scheme}\texttt{>} is e.\,g.
% arabic, roman, Roman, alph, or Alph. For fnsymbol please use
% |\lastpageref{pagesLTS.fnsymbol}| instead of |\pageref{pagesLTS.fnsymbol}|.
% When the same numbering scheme is used twice, the page numbers are
% either reset to one or continued automatically, depending on the option
% given when the package is called. The command |\theCurrentPage| prints
% the current total/absolute page number - in contrast to |\thepage|,
% which gives only the page \textit{name} in the current page numbering
% scheme. |\theCurrentPageLocal| gives the current number of pages
% in the current page numbering scheme. |\thepage| and |\theCurrentPageLocal|
% are different e.\,g. when |\addtocounter{page}{|\ldots|}| or
% |\setcounter{page}{|\ldots|}| were used. (See also \LaTeX\ bug~3421:
% 3rd~page is even (twoside, titlepage, abstract),
% \url{http://www.latex-project.org/cgi-bin/ltxbugs2html?category=LaTeX&responsible=anyone&state=anything&keyword=pagenumber&pr=latex%2F3421&search=}.)
% At the first page of the document a label \texttt{pagesLTS.0} is
% created. This label can be referred to, too. Further labels are
% provided for special cases.\\
% The \xpackage{alphalph} package is supported, i.\,e. page numbers
% \texttt{alph} or \texttt{Alph} $>26$ and \texttt{fnsymbol} $>9$ can
% be used (with the according options set). Even zero or negative page
% numbers can be used with \texttt{arabic}, \texttt{alph}, \texttt{Alph}, and
% \texttt{fnsymbol} page numbering (with \xpackage{alphalph} package
% and according options), and zero \texttt{roman} and \texttt{Roman} pages,
% too.
%
% \bigskip
% \noindent Right after |\begin{document}| a |\pagenumbering{|\ldots|}|
% should be called -- with the appropriate argument out of e.\,g.\\
% arabic (Arabic numerals: 1, 2, 3, 4,\ldots),\\
% roman (Lowercase Roman numerals: i, ii, iii, iv,\ldots),
% Roman (Uppercase Roman numerals: I, II, III, IV,\ldots),\\
% alph (Lowercase letters: a, b, c, d,\ldots),
% Alph (Uppercase letters: A, B, C, D,\ldots),\\
% fnsymbol (Footnote symbols: $*$, $\dagger $, $\ddagger $, $\mathsection $,\ldots).
%
% \bigskip
% \noindent This package first started as a revision of the
% \xpackage{lastpage} package of \textsc{Jeffrey P. Goldberg} (Thanks!),
% but then it became obvious that a replacement was needed to accomplish
% what this package does.
%
% \bigskip
% \noindent \textbf{Trademarks} appear throughout this documentation
% without any trademark symbol; they are the property of their respective
% trademark owner. There is no intention of infringement; the usage is
% to the benefit of the trademark owner.
%
% \bigskip
% \phantomsection
% \DescribeMacro{logical page numbers}
% \noindent \textbf{Tip}: For the display of the \xfile{pdf}~file use
% \textbf{logical page numbers} together with \xpackage{hyperref}!
% \label{logical page numbers}
% \begin{description}
% \item [-] In Adobe Reader DC 2015.008.20082 enable:\\
% Edit $>$ Preferences $>$ Categories: Page Display $>$
% Page Content and Information: Use logical page numbers
%
% \item[-] Use the \xpackage{hyperref} package with option |plainpages=false|\,.
% \end{description}
% The display will be e.\,g.\ \textquotedblleft 7~(7~of~9)\textquotedblright,
% or, in case of Roman instead of arabic numbers,
% \textquotedblleft VII~(7~of~9)\textquotedblright, and when different
% page numbers are used (see below) e.\,g. arabic after 10~Roman pages:
% \textquotedblleft 17~(27~of~30)\textquotedblright.
% Please try this with the compiled \xpackage{pageslts-example} file!
%
% \bigskip
% \noindent The name of the \xpackage{pageslts} package refers to
% Last, Total, and page numbering Schemes pages. \xpackage{pagesLTS}
% was a former name of this package.
%
% \pagebreak
%
% \section{Usage}
%
% \indent Just load the package placing
% \begin{quote}
%   |\usepackage[<|\textit{options}|>]{pageslts}|
% \end{quote}
% \noindent in the preamble of your \LaTeXe\ source file
% (about |\AtEndDocument| see subsection~\ref{ssec:aed})
% and place a |\pagenumbering{...}| with appropriate argument
% (e.\,g.~arabic, roman, Roman, fnsymbol, alph, or Alph)
% right behind |\begin{document}| (see subsubsection~\ref{sss:nopnc})!\\
%
% \indent For example for various draft forms it is desirable to have a
% page reference to the last page, so that e.\,g. page footers can
% contain something like \textquotedblleft page $N$ of $K$\textquotedblright,
% where $N$ is the current page and $K$ is the last page. Once the package
% is loaded, anywhere in the text references can be made to the labels
% \texttt{LastPage}, \texttt{VeryLastPage}, and \texttt{LastPages}
% (most times with |\pageref{...}|, but more save with |\lastpageref{...}|).
% In particular one can use the \xpackage{fancyhdr} or \xpackage{nccfancyhdr}
% package, or redefinitions of the page headings and footings to get
% a reference to the (very) last page.\\
% \DescribeMacro{\pageref*}
% \DescribeMacro{\lastpageref*}
% If the \xpackage{hyperref} package is used, the references are hyperlinked
% to their aims. If these hyperlinks shall be suppressed, |\pageref*{...}|
% and |\lastpageref*{...}| can be used.
%
% \subsection{Options}
% \DescribeMacro{options}
% \indent The \xpackage{pageslts} package takes the following options:
%
% \subsubsection{pagecontinue\label{sss:pagecontinue}}
% \DescribeMacro{pagecontinue}
% \indent When option \texttt{pagecontinue=false} is \textbf{not} given
% (i.\,e. \texttt{pagecontinue} or \texttt{pagecontinue=true}
% or no \texttt{pagecontinue} option at all), at each
% |\pagenumbering{...}| command the number of the page will be continued
% with the page number following the last page of the same page numbering
% scheme. For example, if there are V~Roman pages in the frontmatter,
% some arabic ones in the mainmatter, and then Roman ones again in the
% backmatter, the last ones will start with VI instead of~I again.\\
% \indent If you want to start with I (or i, 1, a, A, *,\ldots ) again,
% set option \texttt{pagecontinue=false}. If you want to generally continue
% the numbers, but for some page numbering scheme do not want this,
% use \texttt{pagecontinue=true} and say |\setcounter{page}{1}|
% after |\pagenumbering{...}| for that page numbering scheme.
%
% \subsubsection{alphMult, AlphMulti, fnsymbolmult\label{sss:alphalph}}
%
% \indent The page number printed in
% \nolinebreak{\texttt{fnsymbol}\footnote{%
% \ensuremath {*},%
% \ensuremath {\dagger },%
% \ensuremath {\ddagger },%
% \ensuremath {\mathsection },%
% \ensuremath {\mathparagraph },%
% \ensuremath {\delimiter "026B30D },%
% \ensuremath {**},%
% \ensuremath {\dagger \dagger },%
% \ensuremath {\ddagger \ddagger }}}
% must be $>0$ and $<10$ and those printed in
% \nolinebreak{\texttt{alph}\footnote{a, b, c, d, e, f, g, h,%
% i, j, k, l, m, n, o, p, q, r, s, t, u, v, w, x, y, z}} and
% \nolinebreak{\texttt{Alph}\footnote{A, B, C, D, E, F, G, H,%
% I, J, K, L, M, N, O, P, Q, R, S, T, U, V, W, X, Y, Z}} must
% be $>0$ and $<27$. After page~Z\quad \LaTeX\ \textit{should}
% continue with AA, AB, AC,\ldots\ Some people prefer
% AA, BB, CC,\ldots, but in hexadecimal it is $AA_{16}=170_{10}$
% and $171_{10}=AB_{16}$, whereas $BB_{16}=187_{10}$. In any way
% it should continue at all (maybe even with an user option to
% choose between the two continuations), but instead only gives
% an error:
% \begin{quote}
%   |LaTeX Error: Counter too large|\\
%   |See the LaTeX manual or LaTeX Companion for explanation.|\\
%   |You've lost some text.  Try typing  <return>  to proceed.|\\
%   |If that doesn't work, type  X <return>  to quit.|
% \end{quote}
% But thanks to the \xpackage{alphalph} package by \textsc{Heiko Oberdiek}
% these limitation no longer hold. With his |\erroralph| command
% now even negative or zero page
% \textquotedblleft numbers\textquotedblright{} are possible.\\
%
% \DescribeMacro{alphMult}
% The string option \texttt{alphMult} takes three values: |ab|, |bb|, |0|:
% \begin{description}
% \item[ab] After page z, the page
%  \textquotedblleft numbers\textquotedblright{} continue with
%  aa, ab, ac, ad,\ldots, fxshrxw (the default), and before a with
%  0, -a, -b,\ldots, -z, -aa, -ab,\ldots, -fxshrxw ($=-21\,474\,836\,47$).
% \item[bb] After page z, the page
%  \textquotedblleft numbers\textquotedblright{} continue with
%  aa, bb, cc, dd,\ldots, and before a with
%  0, -a, -b,\ldots, -z, -aa, -bb,\ldots\\
%  (Internally up~to $\pm 55\,834\,558$ is allowed, but when printed will
%  exceed the \LaTeX\ capacity even for smaller numbers --
%  in the example file this happens at about $6\,500$.)\\
%  (If you have a document with more than $6\,500$~pages, you might
%   think about splitting it in volumes. And page
%   \textquotedblleft numbers\textquotedblright{} with about 100~digits
%   are probably not easy to grasp for the reader, too.)\\
% \item[0] (zero) The \xpackage{pageslts} package does nothing,
%  thus the user is free to define the page
%  \textquotedblleft numbers\textquotedblright{} after~z and before~a.\\
%  (But if the user does not do anything at all, the\\
%  \ |LaTeX Error: Counter too large|\\
%  will appear again.)
% \end{description}
%
% \DescribeMacro{AlphMulti}
% The string option \texttt{AlphMulti} takes three values: |AB|, |BB|, |0|:
% \begin{description}
% \item[AB] After page Z, the page
%  \textquotedblleft numbers\textquotedblright{} continue with
%  AA, AB, AC, AD,\ldots, FXSHRXW (the default), and before A with
%  0, -A, -B,\ldots, -Z, -AA, -AB,\ldots, -FXSHRXW.
% \item[BB] After page Z, the page
%  \textquotedblleft numbers\textquotedblright{} continue with
%  AA, BB, CC, DD,\ldots, and before A with
%  0, -A, -B,\ldots, -Z, -AA, -BB,\ldots \\
%  (About the limits please see |alphMult| above.)
% \item[0] (zero) The \xpackage{pageslts} package does nothing,
%  thus the user is free to define the page
%  \textquotedblleft numbers\textquotedblright{} after~Z and before~A.\\
%  (But if the user does not do anything at all, the\\
%  \ |LaTeX Error: Counter too large|\\
%  will appear again.)
% \end{description}
%
% \DescribeMacro{fnsymbolmult}
% When option \texttt{fnsymbolmult=false} is \textbf{not} given
% (i.\,e. \texttt{fnsymbolmult} or \texttt{fnsymbolmult=true}
% or no \texttt{fnsymbolmult} option at all),
% after 5 (\ensuremath {\mathparagraph }) the page
% \textquotedblleft number\textquotedblright{} is continued with
% the doubled \textquotedblleft number\textquotedblright{} of the
% first, second, third,\ldots\ page (\ensuremath {**},
% \ensuremath {\dagger \dagger }, \ensuremath {\ddagger \ddagger },
% \ensuremath {\mathsection \mathsection },
% \ensuremath {\mathparagraph \mathparagraph }),
% and after the tenth page the
% \textquotedblleft number\textquotedblright{} is tripled
% (\ensuremath {***}, \ensuremath {\dagger \dagger \dagger },\ldots).
% Compile the \textsf{pageslts-example.tex} with \pdfLaTeX\ and
% see the resulting \xfile{pdf} file.\\
% Before \ensuremath {*} (page~1) the page
% \textquotedblleft numbers\textquotedblright{} are continued with
% 0, \ensuremath {- * }, \ensuremath {- \dagger },\ldots,
% \ensuremath {- \mathparagraph }, \ensuremath {- **},
% \ensuremath {- \dagger \dagger },\ldots \\
% If this is not wanted, set option \texttt{fnsymbolmult=false},
% and \xpackage{pageslts} will do nothing and allow the user to change
% the page \textquotedblleft number\textquotedblright . (But if the
% user does not do anything at all, the\\
% \ |LaTeX Error: Counter too large|\\
% will appear again.)\\
% While in \LaTeXe\ arabic (page) numbers are possible up~to
% |MAX|\,$ = 2\,147\,483\,647$ (cf.~the \xpackage{alphalph} package),\\
% |\erroralphalph{\fnsymbolmult}{...}| numbers are possible up~to
% 10\,737\,415 only. If this number is not only used internally
% but printed, after number about 11\,705 (which is
% 2\,341 times \ensuremath {\mathparagraph }) the \LaTeXe\ capacity
% is exceeded, depending on the remaining file and its use of
% \TeX\ capacity, of course. (If you have a document with more than
% 11\,705~pages, you might think about splitting it in volumes.
% And page \textquotedblleft numbers\textquotedblright{} with
% 2\,341~digits are probably not easy to grasp for the reader, too.)\\
%
% \subsubsection{romanMult, RomanMulti\label{sss:romanMult}}
%
% \DescribeMacro{romanMult}
% \DescribeMacro{RomanMulti}
% The options \texttt{romanMult(=true)} and \texttt{RomanMulti(=true)}
% expand the |\roman| and |\Roman| page numbering scheme to values
% below one ($<1$), i.\,e. 0, -i, -ii, -iii, -iv,\ldots\ and
% 0, -I, -II, -III, -IV,\ldots, respectively.\\
% Again the \TeX\ capacity will be exceeded before
% \nolinebreak{$\pm$ |MAX|\,}$ = \pm 2\,147\,483\,647$, and even if
% $1\,000\,000\,000$ is internally possible, this would print
% $1\,000\,000$ times the letter~\textsf{m} (or~\textsf{M}),
% which would require either very small print or quite huge paper size.\\
% (If you have a document with so many pages, you might think
% about splitting it in volumes. And page
% \textquotedblleft numbers\textquotedblright{} with
% thousands of digits are probably not easy to grasp for the reader,
% too.)\\
% If the expansion below $1$ is not wanted, set options
% \texttt{romanMult=false} and/or \texttt{RomanMulti=false},
% and \xpackage{pageslts} will do nothing and allow the user
% to change the page \textquotedblleft number\textquotedblright .
% (But if the user does not do anything at all, \LaTeX\ will just
% ignore the values - not even a warning will be issued.)
%
% \subsubsection{Arabic page numbers}
%
% \DescribeMacro{Arabic page numbers}
% In \LaTeXe\ arabic (page) numbers are already possible between
% |-MAX...MAX|, where |MAX|\,$ = 2\,147\,483\,647$ (cf.~the
% \xpackage{alphalph} package), without any expansion necessary.
% (But if you have a document with so many pages, you might think
% about splitting it in volumes!)
%
% \subsection{Labels}
%
% \DescribeMacro{pagesLTS.0}
% \indent At the first page a label \texttt{pagesLTS.0} is created.
% If |\pagenumbering{...}| is used right after
% |\begin{document}|, this is much easier for the
% \xpackage{pageslts} package (and chances for successful placing
% of all labels are much higher;
% \nolinebreak{cf.~subsubsection~\ref{sss:once}).}\\
%
% \DescribeMacro{LastPage}
% |\AtEndDocument| (see subsection~\ref{ssec:aed}) this package defines
% a label, \texttt{LastPage}, which the user can refer to with the\linebreak
% |\lastpageref{LastPage}| command. While |\pageref{LastPage}| is also
% possible (especially for backward compatibility with the \xpackage{LastPage}
% package), this is discouraged, because it will not work when it is used
% together with the \xpackage{hyperref} package and the \xpackage{fnsymbol}
% page numbering scheme. (The \xpackage{LastPage} package did not work with
% this combination, too, so if you want to, you can reproduce the old error
% -- but you do not have to do it, but can use |\lastpageref{LastPage}|.)\\
%
% \DescribeMacro{VeryLastPage}
% |\AfterLastShipout| the label \texttt{VeryLastPage} is defined, which
% the user can also refer to with the |\lastpageref{VeryLastPage}| command.
% Depending on usage of |\AtEndDocument| by other packages, \texttt{LastPage}
% might not point to the very last page, but |\lastpageref{VeryLastPage}|
% should do this (cf.~subsection~\ref{ssec:aed}).\\
%
% \DescribeMacro{LastPages}
% \DescribeMacro{page number}
% \DescribeMacro{number of pages}
% When more than one page numbering scheme is used, neither \texttt{LastPage}
% nor \texttt{VeryLastPage} give the total \textbf{number} of pages.
% For example, for a document with VI+36 pages, both give
% \textquotedblleft 36\textquotedblright{} as reference to the last page.
% While this is correct, the total number of pages is 42, and this is given
% by the reference to \texttt{LastPage\underline{s}}:
% \nolinebreak{|\lastpageref{LastPage|\underline{|s|}|}|} (note the
% \textquotedblleft s\textquotedblright{} at the end). When the
% page number was manipulated by |\addtocounter{page}{...}|
% or |\setcounter{page}{...}|, \texttt{LastPages} ignores this.
% (At a page numbering change the page is reset to one (without option
% |pagecontinue|). This is done by |\setcounter{page}{1}|, thus this is
% ignored, too.)\\
% |\pageref{totpages}| of the \xpackage{totpages} package is similar to
% |\lastpageref{LastPages}|, but while the target for \linebreak[4]
% |\pageref{totpages}| is placed |\AtEndDocument|, the target for
% |\lastpageref{LastPages}| is placed |\AfterLastShipout|, therefore
% |\lastpageref{LastPages}| is safer to really get the total page number.\\
%
% \pagebreak
%
% \DescribeMacro{\theCurrentPage}
% |\theCurrentPage| gives the current total/absolute page, in contrast to
% |\thepage|, which gives only the page \textit{name} in the current page
% numbering scheme. For example, when there are Roman VII pages in the
% frontmatter and afterwards in the mainmatter you are at arabic page~9,
% then |\theCurrentPage| is~16, whereas |\thepage| is~9. When the page
% \textquotedblleft number\textquotedblright{} (name) is manipulated by
% |\addtocounter{page}{...}| or |\setcounter{page}{...}|, |\theCurrentPage|
% \nolinebreak{ignores} this. Because \texttt{CurrentPage} is a normal
% counter, you can also say e.\,g.~|\Roman{CurrentPage}| to get the value
% in Roman page numbering scheme (e.\,g.~VIII for~8).\\
%
% \phantomsection
% \label{CurrentPageLocal}
% \DescribeMacro{\theCurrentPageLocal}
% |\theCurrentPageLocal| gives the current (arabic) number of pages in the
% current page numbering scheme. |\thepage| and |\theCurrentPageLocal| are
% different e.\,g. when |\addtocounter{page}{...}| or
% |\setcounter{page}{...}| were used.\\
% |\theCurrentPageLocal| can be printed in other formats, e.\,g.
% |\roman{pagesLTS.current.local.roman}|, but probably it only makes
% sense if page numbering scheme and format are the same, e.\,g.
% |\Roman{pagesLTS.current.local.Roman}| or\\
% |\Alph{pagesLTS.current.local.Alph}|.\\
% |\arabic{pagesLTS.current.local.|\ldots|}| probably make sense even
% when combined with another page numbering scheme. And this is exactly
% what |\theCurrentPageLocal| does:\\
% |\def\theCurrentPageLocal{\arabic{pagesLTS.current.local.\pagesLTS@pnc}}|.\\
%
% \DescribeMacro{pagesLTS. page numbering scheme . number }
% If you want to refer to the last page of the first, second,\ldots\ use
% of a page numbering scheme, you can refer to \linebreak
% |pagesLTS.<page numbering scheme>.<number>|, e.\,g.
% |\lastpageref{pagesLTS.Roman.1}|, where |<number>| is the occurrence
% of the page numbering scheme. For details please see
% page~\pageref{pagesLTS.pnscheme.nr}.\\
%
% \bigskip
% \DescribeMacro{\lastpageref}
% For pages with the \texttt{fnsymbol} page numbering scheme,
% |\lastpageref{...}| \nolinebreak{instead} of |\pageref{...}|
% \textit{must} be used. This is required for pages some\-where
% inside of the document as well as the (very) last page(s).
% Because |\lastpageref{...}| is a synonym for |\pageref{...}|,
% where no \xpackage{fnsymbol} page numbering scheme is used,
% it is save(r) to use it for all references to labels provided
% by the \xpackage{pageslts} package.
%
% \subsection[\texttt{\textbackslash pagenumbering\{\ldots \}}]{%
% \texttt{\textbackslash pagenumbering\{\ldots \}}}
% \DescribeMacro{\pagenumbering}
% \subsubsection{If \texttt{\textbackslash pagenumbering\{\ldots \}} is \textbf{not} used\label{sss:nopnc}}
%
% \indent When the \xpackage{pageslts} package is used, but |\pagenumbering{...}|
% (with an argument like arabic, roman, Roman, fnsymbol, alph, or Alph)
% is not used, there should be no problem, except that you might need
% more~(!) compiler runs to get all references right, and some references
% might even be missing (see below). The \xpackage{pageslts}
% package tries to determine the page numbering scheme at the first
% shipout, but success is not guaranteed. Thus please use |\pagenumbering{...}|
% at the beginning of your document!\\
% \indent Without |\pagenumbering{<something>}| (|<something>| e.\,g. = arabic)
% at the beginning of the document, the page numbers might be given in
% arabic \textit{by} (class) \textit{default}, but the \xpackage{pageslts}
% package does not know about this without |\pagenumbering{arabic}|.~--\\
% \indent The label \texttt{pagesLTS.0} is created at the first
% page even if no |\pagenumbering{...}| command is given.
% Maybe have a look at the \xfile{.aux} file after compiling your
% document to detect further labels (of other packages, too).\\
%
% \pagebreak
%
% \subsubsection{If \texttt{\textbackslash pagenumbering\{\ldots \}} is used once\label{sss:once}}
%
% \DescribeMacro{pagesLTS.0}
% At the first page a label \texttt{pagesLTS.0} is created.
% If |\pagenumbering{...}| is used right after |\begin{document}|,
% this is much easier for the \xpackage{pageslts} package
% (and chances for successful placing of all labels are much higher).
%
% \subsubsection{If \texttt{\textbackslash pagenumbering\{\ldots \}} is used more than once}
%
% \indent Everything from the preceding subsubsections applies and additionally
% the following:\\
% \indent When different page numbering schemes are used, e.\,g.
% Roman numbers for the frontmatter and arabic numbers for the
% mainmatter, please use |\pagenumbering{...}| for each of them!
% Even if you do this, the reference to neither the label \texttt{LastPage}
% nor the label \texttt{VeryLastPage} gives the \textbf{total}
% number of pages, but only the number of pages of the last used page
% numbering scheme (which could be exactly what you want, e.\,g. if
% you want to refer to the last page itself and do not want to
% give the total number of pages).\\
% \DescribeMacro{LastPages}
% \indent For remediation the label
% \nolinebreak{\texttt{LastPage}\textbf{\underline{s}}}
% (with \textquotedblleft s\textquotedblright{} at its end) is
% introduced. Please then refer to this label by \newline
% |\lastpageref{LastPages}| instead of \texttt{LastPage} or
% \texttt{VeryLastPage}.\\
%
% \DescribeMacro{pagesLTS.arabic}
% \DescribeMacro{pagesLTS.roman}
% \DescribeMacro{pagesLTS.Roman}
% \DescribeMacro{pagesLTS.alph}
% \DescribeMacro{pagesLTS.Alph}
% \DescribeMacro{pagesLTS.fnsymbol}
% Additionally, at the last page of each page numbering scheme a label
% \texttt{pagesLTS.<numbering scheme>} is placed, where
% \texttt{<numbering scheme>} is e.\,g.\ arabic, roman, Roman, alph,
% Alph,\ldots .\\
% For the \textbf{fnsymbol} page numbering scheme
% |\lastpageref{pagesLTS.fnsymbol}| is needed instead of \newline
% |\pageref{pagesLTS.fnsymbol}|. You can and should use
% |\lastpageref{...}| also for the other page numbering schemes.\\
%
% \bigskip
%
% While at the time of the last revision of the \xpackage{pageslts} package
% no other page numbering schemes were known to the maintainer,
% this package in principle works with every scheme which is
% recognized by the original |\pagenumbering| command.
% But the \xpackage{hyperref} package only then works with crazy page names,
% if the references to those pages are given in a certain way, thus the
% combination of a new page numbering scheme, the \xpackage{hyperref}
% and the \xpackage{pageslts} package might not work.~--
% The \xpackage{pageslts} package by itself also works with schemes,
% which the original |\pagenumbering{...}| does not recognize, but because
% the original |\pagenumbering{...}| is called by the \xpackage{pageslts}
% package, this might cause an error, see subsection~\ref{sec:unknown}!
% (And if the number format is unknown to \LaTeX{}, the pages will have
% no number, and therefore cannot be referenced. You might be able to
% help yourself by using the \xpackage{hyperref} package and
% manually placing |\hypertarget|s and |\href|s.)
%
% \subsubsection{If the same \texttt{\textbackslash pagenumbering\{\ldots \}}\ %
%  scheme is used more than once\label{sss:twice}}
%
% \indent Everything from the preceding subsubsections applies and additionally
% the following:\\
% \DescribeMacro{pagecontinue}
% \indent If the same page numbering scheme is used twice (or even more often)
% in one document (e.\,g. in the frontmatter Roman: I--V, in the
% mainmatter arabic: 1--20, and in the backmatter again Roman: VI--X),
% the second time it is used, the page numbering is either continued (option
% \texttt{pagecontinue} or \texttt{pagecontinue=true} or no option
% \texttt{pagecontinue}; the default) or reset to one (option
% \texttt{pagecontinue=false}). It is even possible to use
% a~page numbering scheme more than twice.\\
%
% \phantomsection
% \label{pagesLTS.pnscheme.nr}
% \DescribeMacro{pagesLTS. page numbering scheme . number }
% If you want to refer to the last page of the first, second,\ldots\ use
% of a page numbering scheme, page~V in the example above, you can refer to
% |pagesLTS.<page numbering scheme>.<number>|, e.\,g.
% |\lastpageref{pagesLTS.Roman.1}|, where |<number>| is the occurrence
% of the page numbering scheme.\\
%
% \pagebreak
%
% If you want to refer to the first page of a page numbering scheme,
% just place a label there, e.\,g.\\
% \indent |\pagenumbering{Roman}| \\
% \indent |\section{Section title\label{RomanSection}}| \\
% (You know where you use |\pagenumbering{...}| and this is the
%  \xpackage{pageslts} package, not the \xpackage{firstpage} one).\\
%
% When you want to give the number of pages of each
% \textquotedblleft sector\textquotedblright{} of the page numbering
% scheme, you can use\\
% \DescribeMacro{pagesLTS. page numbering scheme . number .local.cnt}
% |\lastpages{<page numbering scheme>}{<number>}|, \\
% where |<page numbering scheme>| is e.\,g. Roman, arabic,\ldots\ and
% |<number>| the \textquotedblleft sector\textquotedblright{} number,
% e.\,g.~|\lastpages{Roman}{2}|.\\
% (Internally, the counter has the format
% |pagesLTS.<page numbering scheme>.<number>.local.cnt|.)
%
% \noindent If you used the page numbering scheme Roman for three times,
% you could say
% \begin{verbatim}
% Last Roman page (pagesLTS.Roman): \lastpageref{pagesLTS.Roman}\\
% There are \lastpageref{pagesLTS.Roman.local}~pages with Roman numbers:\\
% \lastpages{Roman}{1}~pages in the first Roman sector
% (\pageref{Roman} -- \lastpageref{pagesLTS.Roman.1}),\\
% \lastpages{Roman}{2}~pages in the second Roman sector
% (\pageref{Roman2} -- \lastpageref{pagesLTS.Roman.2}), and\\
% \lastpages{Roman}{3}~pages in the third Roman sector
% (\pageref{Roman3} -- \lastpageref{pagesLTS.Roman.3}.\\
% \end{verbatim}
% to get
% \begin{quote}
% Last Roman page (pagesLTS.Roman): VIII\\
% There are 8 pages with Roman numbers:\\
% 3 pages in the first Roman sector (I -- III),\\
% 4 pages in the second Roman sector (IV -- VII), and\\
% 3 pages in the third Roman sector (VIII -- X).\\
% \end{quote}
% (see e.\,g. the compiled \textsf{pageslts-example} file).\\
%
% If you want to continue one page numbering scheme, but later on
% (third use of it, or for another page numbering scheme) want to
% reset the page number, just say |\setcounter{page}{1}| there.\\
%
% \pagebreak
%
% \noindent In your document the code
% \begin{verbatim}
% \makeatletter
% \renewcommand{\@evenfoot}%
%  {\normalsize\slshape DRAFT \today\hfil \upshape page {\thepage} (\theCurrentPage) of\ %
%   \lastpageref{pagesLTS.Roman} + \lastpageref{pagesLTS.arabic}\ = \lastpageref{LastPages} pages%
%  }
% \renewcommand{\@oddfoot}{\@evenfoot}
% \makeatother
%\end{verbatim}
% \noindent creates footers like\\
%
% \textquotedblleft\mbox{\textsl{DRAFT \today}\hspace{1cm}page V (5) of VII + 35 = 42 pages}\textquotedblright\\
%
% \noindent or\\
%
% \textquotedblleft\mbox{\textsl{DRAFT \today}\hspace{1cm}page 10 (17) of VII + 35 = 42 pages}\textquotedblright\\
%
% \noindent in the compiled document (cf.~the \texttt{pageslts-example} file).
%
% \noindent Code like
% \begin{verbatim}
%  This book has \lastpageref{pagesLTS.Roman}+\lastpageref{pagesLTS.arabic} pages %
%  (\lastpageref{LastPages} pages in total).
% \end{verbatim}
% \noindent produces output like
% \begin{quote}
% This book has X+85 pages (95 pages in total).
% \end{quote}
% \noindent (when using the \xpackage{hyperref} package,
% the references are even hyperlinked).\\
%
% If |\addtocounter{page}{...}| or |\setcounter{page}{...}| have been used,
% the local version of \texttt{CurrentPage} can be used, |\theCurrentPageLocal|,
% see subsection~\ref{CurrentPageLocal}.\\
%
% \subsection{\xpackage{papermas}(s) package\label{ss:papermas}}
% There is a kind of an add-on to this package, the \xpackage{papermas}
% package, which can be used to compute the number of sheets of paper
% needed to print a document (you can print more than one page of a
% document on one sheet of paper) as well as the approximate mass
% of the printout. Please see the \ref{ss:Downloads} subsection.
%
% \newpage
%
% \section{A few warnings\label{sec:warn}}
%
% \subsection{Hyperref and repeated page numbers}
%
% When two (or more) different page numbering schemes are used, or the page number
% is reset, or for any other reason there are two pages with the same number
% (maybe in different format, e.\,g. $1$ and \textsc{I}), and \xpackage{hyperref} has
% not been configured right, this can cause problems. Use \xpackage{hyperref} with
% \texttt{plainpages=false} and \texttt{pdfpagelabels=true}, and everything should
% be fine.\\
% More details can be found at
% \ \url{http://www.tex.ac.uk/cgi-bin/texfaq2html?label=pdfpagelabels}.
%
% \subsection{\texttt{\textbackslash AtEndDocument}\label{ssec:aed}}
%
% \indent The output of a \LaTeXe\ run is not independent of the order
% in which packages are loaded. It is often the case that the same
% formats for which one must put tables and figure at the end,
% are the ones in which endnotes are also required.
% If one wants to use |\AtEndDocument| here as well (as done for
% |\lastpageref{LastPage}|), then it is easy to get to three separate
% uses of |\AtEndDocument| (assuming one uses this for the endnotes
% as well). Clearly it is not safe for any package writer or user
% to assume that no material will follow what they put into
% |\AtEndDocument|. Therefore a message, which begins with
% \texttt{AED}, is included in every usage of |\AtEndDocument|,
% and it is tried to minimize any side effects the usage may have.\\
%
% As now \textsc{Heiko Oberdiek}'s \texttt{atveryend} package is used,
% the references |\lastpageref{VeryLastPage}| and \newline
% |\lastpageref{LastPages}| should work all right. About how to get the
% \texttt{atveryend} package, please see subsection~\ref{ss:Downloads}.
%
% \subsection{Interaction with very old versions of the \xpackage{endfloat} %
%  package\label{sec:endfloat}}
%
% \indent The \emph{very} old version~2.0 (and earlier) of the \xpackage{endfloat}
% package actually redefined the |\enddocument| command,
% and so interfered drastically with the \LaTeXe\ commands which make use
% of |\AtEndDocument|. Newer versions of \xpackage{endfloat} exist
% (at the time of writing this documentation: v2.5d as of 2011/12/25)
% in modern documentation form, which should be available from
% the same source where you received this file, see subsection~\ref{ss:Downloads}.\\
% A note is placed in the style file at the |\RequirePackage| section,
% and later it is even checked whether a (very) old \xpackage{endfloat}
% package is in use. If it is, a warning or even an error message
% is given, depending on \xpackage{endfloat} version.
% This assumes, that the old versions of \xpackage{endfloat} at least gave
% a version date, of course.\\
% If you want your \texttt{LastPage} to label the last page of
% these end floats, you need to load \xpackage{pageslts} \emph{after}
% loading \xpackage{endfloat}, or to use \texttt{VeryLastPage} instead.
% If, on the other hand, you \emph{want} \texttt{LastPage} to refer
% to the (not so) last page, exclusive of the floats at the end,
% then load in the reverse order. Independent from the order of
% \xpackage{pageslts} and \xpackage{endfloat}, you will still need the
% modified\footnote{New versions are available for over $15$~years now,
% so it might be time to update, if you did not do it already.} version of
% \xpackage{endfloat}.\\
% Using the
% \nolinebreak{\texttt{LastPage}\textbf{\underline{s}}\ (\textit{s}!)}\ label
% should get you to the last page in all cases: |\lastpageref{LastPages}|.\\
%
% Other \LaTeX209\ (!) packages also seem to like to redefine
% |\enddocument|. In addition to the old \xpackage{endfloat},
% \xpackage{harvard} comes to mind. All of these will need to be
% modified swiftly. \textbf{If possible, update to \LaTeXe{}!}
%
% \pagebreak
%
% \subsection{\xpackage{showkeys} package\label{ssec:showkeys}}
%
% When the \xpackage{showkeys} package has been loaded in \texttt{draft} mode,
% in the margin for each label a~box is displayed with the name of the label.
% \xpackage{showkeys} accomplishes this by redefining |\label|, but
% \xpackage{pageslts} does not use |\label|, but writes directly to the
% {\nobreak |\jobname|\xfile{.aux}-}\allowbreak file, and this is generally done
% \emph{after} the according page has shipped out, therefore no box can be placed
% on the \emph{preceding} page. At~least \xpackage{pageslts} gives a warning, that
% \xpackage{showkeys} cannot present the respective label.
%
% \subsection{\xpackage{lastpage} package\label{ssec:lastpage}}
%
% This package first started as a revision of the \xpackage{lastpage}
% package of \textsc{Jeffrey P. Goldberg} (jeffrey+news at goldmark dot org),
% but it became obvious that a replacement was needed to accomplish what this
% package does. For backward compatibility, a~label named |LastPage| is provided.
% Thus |\usepackage{lastpage}| can be replaced by\\
% |\usepackage[pagecontinue=false,%|\newline
% |alphMult=0,AlphMulti=0,fnsymbolmult=false,romanMult=false,RomanMulti=false]{pageslts}|,\\
% if the behaviour of the \xpackage{lastpage} package should be simulated.
% Using \textbf{old}~(!) versions of the \xpackage{lastpage} before the \xpackage{pageslts}
% before the \xpackage{hyperref} [2012/11/06 v6.83m]
% package results in multiply definitions of the |LastPage| label.
% While the \xpackage{pageslts} package cancels the command |\lastpage@putlabel|
% from the old \xpackage{lastpage} package (because it does this itself, and better),
% \xpackage{hyperref} redefines |\lastpage@putlabel| and thereby reintroduces it again
% (\xpackage{hyperref} should probably check for the version of the \xpackage{lastpage}
% package and/or whether the \xpackage{pageslts} package was also loaded.)
%
% \subsection{Using an unknown page numbering scheme\label{sec:unknown}}
%
% \indent I do not know whether \LaTeXe\ can handle another page numbering
% scheme (e.\,g.~Hebraic), but if you want to use it, this should be
% no problem for the \xpackage{pageslts} package. But the original
% |\pagenumbering{...}| as well as the \xpackage{hyperref} package
% (if~used) might want to vote against it, especially when used together
% with the \xpackage{pageslts} package. Especially especially
% (\textit{sic}!) if the last page uses this new page numbering scheme,
% you should check everything double (at least).\\
% (And if the number format is unknown to \LaTeX{}, the pages will have
% no number, and therefore cannot be referenced. You might be able to
% help yourself by using the \xpackage{hyperref} package and
% manually placing |\hypertarget|s and |\href|s.)
%
% \subsection{Page counter overflow\label{ss:overflow}}
%
% \indent Without the use of the \xpackage{alphalph} package, the\\
% \textquotedblleft ranges of supported counter values are more or less
% restricted. Only \cs{arabic} can be used with any counter
% value \TeX\ supports.
% \begin{quote}
% \catcode`\|=12 %
% \begin{tabular}{@{}l|l|l|l@{}}
% Presentation & Supported & Ignored & Error message\\
% command      & domain    & values  & `Counter too large'\\
% \hline
% \cs{arabic}
%   & \ttfamily -MAX..MAX & &\\
% \cs{roman}, \cs{Roman}
%   & \ttfamily 1..MAX & \ttfamily -MAX..0 &\\
% \cs{alph}, \cs{Alph}
%   & \ttfamily 1..26 & 0 & \ttfamily -MAX..-1, 27..MAX\\
% \cs{fnsymbol}
%   & \ttfamily 1..9 & 0 & \ttfamily -MAX..-1, 10..MAX\\
% \hline
% \end{tabular}\\[1ex]
% \texttt{MAX} = \texttt{2147483647}
% \end{quote}
% \textquotedblright{} (\textsc{Heiko Oberdiek}:
% The \xpackage{alphalph} package, 2010/04/18, v2.3, first table, p.~2).\\
% Please see subsubsections~\ref{sss:alphalph} and \ref{sss:romanMult} for
% instructions how to overcome these limitations.
%
% \newpage
%
% \subsection{Using the \xpackage{fnsymbol} page numbering scheme\label{ss:fnsymbol}}
%
% \indent Using the \texttt{fnsymbol} page numbering scheme can result in
% problems -- big ones!\\
% \textbf{When using this page numbering scheme, it is very important to use}
% \ |\lastpageref{...}| \textbf{instead of}\ |\pageref{...}| \textbf{for any%
% link to any label provided by the \xpackage{pageslts} package.}\\
% While the \xpackage{pageslts} package tries really very hard to circumvent
% any problem, other packages might screw up -- and quite totally for that.
% So,~you have been warned!\\
%
% \begin{description}
% \item[-] There can be a counter overflow, see preceding subsection~\ref{ss:overflow}.
%
% \item[-] Adobe Reader DC 2015.008.20082 does not show the correct page names
% for all pages with \texttt{fnsymbol} page numbering scheme (see the example file),
% while at least the (\ldots\ of \ldots )  part of the page number is displayed correctly
% (see page~\pageref{logical page numbers}, tip about logical page numbers).
% When the \xpackage{alphalph} package and the \xpackage{pageslts} package with
% |fnsymbolmult| option are used, more page names are presented correctly by the Reader.
% (Adobe Reader~X even got all pages right.)
% \end{description}
%
% \newpage
%
% \section{Alternatives\label{sec:Alternatives}}
%
% There are similar packages, which do (or do not) similar things. As I
% neither know what exactly you want to accomplish when using this package
% (e.\,g.~page number vs. page name, hyperlinks or not), nor what resources
% you have (e.\,g.~\eTeX{}), here is a list of some possible alternatives:\\
%
% \DescribeMacro{LastPage}
% \begin{description}
% \item[-] The \xpackage{LastPage} package also provides the
%  \texttt{LastPage} label (but not \texttt{VeryLastPage} or
%  \texttt{LastPages}). If you only want this and have a quite
%  limited amount of \TeX\ resources, you might want to use that
%  package instead.
%
% \item[-] If \LaTeX2.09\ is still used, and if you are unable
%  to switch to \LaTeXe{}, the \LaTeX2.09{} compatible
%  \xpackage{lastpage209.sty} can be used (which is also contained
%  in the recent \xpackage{LastPage} package).
% \end{description}
%
% \DescribeMacro{totpages}
% \begin{description}
% \item[-] The \xpackage{totpages} package provides a
%  \texttt{totpages} label similar to \texttt{LastPages},
%  but |\AtEndDocument| instead of |\AfterLastShipout|. Therefore
%  you should stay with \xpackage{pageslts}. The \xpackage{totpages}
%  package additionally computes the number of paper sheets needed to
%  (double) print the document (with one, two, three,\ldots\ pages
%  on one sheet of paper). This can also be accomplished with the
%  \xpackage{papermas} package.
% \end{description}
%
% \DescribeMacro{totcount}
% \begin{description}
% \item[-] The \xpackage{totcount} package provides the last value
%  of a counter, thus also the value of the \texttt{page} counter.
%  You do not get a hyperlink to the last page, only the numerical
%  value of the last page name is given (i.\,e. X+72 pages gives 72
%  instead of 82 as total number of pages), and the number of pages
%  can be changed e.\,g. by |\addtocounter|.
% \end{description}
%
% \DescribeMacro{nofm}
% \begin{description}
% \item[-] \textquotedblleft There is a package \xpackage{nofm.sty}
%  available, but some versions of it are defective, and most don't
%  work with \xpackage{fancyhdr} because they take over the complete
%  page layout.\textquotedblright{} (\textsc{Piet van Oostrum}:
%  Page layout in \LaTeX{}, March~2, 2004, section~16; fancyhdr.pdf)\\
%  \xpackage{nofm} as of 1991/02/25 (without version number), available at\\
%  \url{http://mirror.ctan.org/obsolete/macros/latex209/contrib/misc/nofm.sty},\\
%  does not work with e.\,g. \xpackage{hyperref}, redefines |\enddocument|
%  as well as |\@oddhead|, |\@evenhead|, |\@oddfoot|, and |\@evenfoot|.\\
%  If you know the (\url{https://www.CTAN.org}) location of a%
%  \textbf{working}~(!) version, please send me an e-mail, thanks!
% \end{description}
%
% \DescribeMacro{count1to}
% \begin{description}
% \item[-] You may want to have a look at the \xpackage{count1to} package.
% \end{description}
%
% \DescribeMacro{zref}
% \begin{description}
% \item[-] The \xpackage{zref} package of \textsc{Heiko Oberdiek} requires
%  \eTeX . \xpackage{pageslts} does not require \eTeX , but if you
%  already have \eTeX , you may have a look at the extensive
%  \xpackage{zref} package, whether it suits your needs better
%  (or additionally or whatsoever).
% \end{description}
%
% \bigskip
%
% \noindent (You programmed or found another alternative,
%  which is available at \url{https://www.CTAN.org}?\\
%  OK, send an e-mail to me with the name, location at CTAN,
%  and a short notice, and I will probably include it in the list above.)\\
%
% \smallskip
%
% \noindent About how to get those packages, please see subsection~\ref{ss:Downloads}.
%
% \newpage
%
% \section{Example}
%
%    \begin{macrocode}
%<*example>
\documentclass[british]{article}[2014/09/29]% v1.4h
\usepackage{lipsum}[2014/07/27]%              v1.3
\usepackage[draft]{showkeys}[2014/10/28]%     v3.17
%%      Use final instead of draft to hide the keys. %%
\usepackage{hyperref}[2012/11/06]%            v6.83m
\hypersetup{%
 extension=pdf,%
 plainpages=false,%
 pdfpagelabels=true,%
 hyperindex=false,%
 pdflang={en},%
 pdftitle={pageslts package example},%
 pdfauthor={Hans-Martin Muench},%
 pdfsubject={Example for the pageslts package},%
 pdfkeywords={LaTeX, pageslts, H.-Martin Muench},%
 pdfview=Fit,%
 pdfstartview=Fit,%
 pdfpagelayout=SinglePage,%
 bookmarksopen=true%
}
\usepackage[pagecontinue=true,alphMult=ab,AlphMulti=AB,fnsymbolmult=true,%
 romanMult=true,RomanMulti=true]{pageslts}[2015/12/21]% v1.2f
%% These are the default options. %%

\makeatletter
  \renewcommand{\@evenfoot}%
   {Page \thepage\ (\theCurrentPage; local: \theCurrentPageLocal) of %
    \lastpageref{pagesLTS.roman}(\lastpageref{pagesLTS.roman.local}) + %
    \lastpageref{pagesLTS.Roman}(\lastpageref{pagesLTS.Roman.local}) + %
    \lastpageref{pagesLTS.arabic}(\lastpageref{pagesLTS.arabic.local}) + %
    \lastpageref{pagesLTS.fnsymbol}(\lastpageref{pagesLTS.fnsymbol.local}) + %
    \lastpageref{pagesLTS.alph}(\lastpageref{pagesLTS.alph.local}) + %
    \lastpageref{pagesLTS.Alph}(\lastpageref{pagesLTS.Alph.local}) = %
    \lastpageref{LastPages} pages.%
   }
  \renewcommand{\@oddfoot}{\@evenfoot}
  \def\pagesLTSexampleArabic{3}
  \def\pagesLTSexamplealph{23}
  \gdef\unit#1{\mathord{\thinspace\mathrm{#1}}}%
\makeatother

\listfiles

\begin{document}
\pagenumbering{roman}
%% Note the first \pagenumbering immediately behind \begin{document}! %%
%%\addtocounter{page}{-2} %%

\section*{Example for pageslts}
\addcontentsline{toc}{section}{Example for pageslts}
\markboth{Example for pageslts}{Example for pageslts}

This example demonstrates the most common uses of package\\
\textsf{pageslts}, v1.2f as of 2015/12/21 (HMM).\\
The used options were \texttt{pagecontinue=true},
\texttt{alphMult=ab}, \texttt{AlphMulti=AB}, \linebreak
\texttt{fnsymbolmult=true},
\texttt{romanMult=true}, and \texttt{RomanMulti=true}
(the default ones).\\
For more details please see the documentation!\\

\label{keys} To hide the \pageref{keys}{\qquad } use option
\texttt{final} instead of \texttt{draft} with the \textsf{showkeys}
package (or remove the package call from the preamble of
this document).\\

\textbf{Hyperlinks or not:} If the \textsf{hyperref} package is loaded,
the references are also hyperlinked:\\
\smallskip
Page \thepage\ (\theCurrentPage; local: \theCurrentPageLocal) of %
\lastpageref{pagesLTS.roman}(\lastpageref{pagesLTS.roman.local}) + %
\lastpageref{pagesLTS.Roman}(\lastpageref{pagesLTS.Roman.local}) + %
\lastpageref{pagesLTS.arabic}(\lastpageref{pagesLTS.arabic.local}) + %
\lastpageref{pagesLTS.fnsymbol}(\lastpageref{pagesLTS.fnsymbol.local}) + %
\lastpageref{pagesLTS.alph}(\lastpageref{pagesLTS.alph.local}) + %
\lastpageref{pagesLTS.Alph}(\lastpageref{pagesLTS.Alph.local}) = %
\lastpageref{LastPages} pages.\\
If the \textsf{hyperref} package is loaded, but the hyperlinks of the
references shall be suppressed, \verb|\pageref*{...}|
and \verb|\lastpageref*{...}| can be used:\\
Page \thepage\ (\theCurrentPage; local: \theCurrentPageLocal) of %
\lastpageref*{pagesLTS.roman}(\lastpageref*{pagesLTS.roman.local}) + %
\lastpageref*{pagesLTS.Roman}(\lastpageref*{pagesLTS.Roman.local}) + %
\lastpageref*{pagesLTS.arabic}(\lastpageref*{pagesLTS.arabic.local}) + %
\lastpageref*{pagesLTS.fnsymbol}(\lastpageref*{pagesLTS.fnsymbol.local}) + %
\lastpageref*{pagesLTS.alph}(\lastpageref*{pagesLTS.alph.local}) + %
\lastpageref*{pagesLTS.Alph}(\lastpageref*{pagesLTS.Alph.local}) = %
\lastpageref*{LastPages} pages.\\


\textbf{Trademarks} appear throughout this example without any
trademark symbol; they are the property of their respective
trademark owner. There is no intention of infringement; the
usage is to the benefit of the trademark owner.\\

\textbf{Tip}: Use \textit{logical page numbers}
for the display of the pdf (in Adobe Reader DC 2015.008.20082:
Edit $>$ Preferences $>$ Categories: Page Display $>$
Page Content and Information: Use logical page numbers)!\\

You want negative page numbers? Not only arabic, but even roman,
Roman, alph, Alph or fnsymbol ones? No problem, e.\,g. just give a\\
\verb|\addtocounter{page}{|-\textit{some number}\verb|}| in the
source code of this example file (or uncomment the prepared line)!

\bigskip

Save per page about $200\unit{ml}$ water, $2\unit{g}$ CO$_{2}$
and $2\unit{g}$ wood:\\
Therefore please print only if this is really necessary.

\pagebreak

\tableofcontents

\newpage

\pagenumbering{roman}
%% in case the page numbering is changed before,
%% otherwise pagesLTS.current.local.roman on this page
%% would be undefined

\section{roman}

\noindent (\texttt{roman} page numbering was started before,
because the page numbering scheme was needed to start at
the first page, of course.)\\

\noindent First page (\verb|\lastpageref{pagesLTS.0}|):
\lastpageref{pagesLTS.0}\\

\noindent The page (\verb|\thepage|): \thepage \\

\noindent Current page (\verb|\theCurrentPage|),
i.\,e. counted continuously from the first page): \theCurrentPage \\
You can get this also in other formats:
\roman{CurrentPage}, \Roman{CurrentPage}, \arabic{CurrentPage},
\fnsymbol{CurrentPage}, \alph{CurrentPage}, \Alph{CurrentPage}.

\noindent CurrentPageLocal (\verb|\theCurrentPageLocal|),
i.\,e. counted continuously from the first page of the
current page numbering scheme): \theCurrentPageLocal \\
You can get also this in other formats, too:
\roman{pagesLTS.current.local.roman}, \Roman{pagesLTS.current.local.roman},
\arabic{pagesLTS.current.local.roman}, \fnsymbol{pagesLTS.current.local.roman},
\alph{pagesLTS.current.local.roman}, \Alph{pagesLTS.current.local.roman},
but probably it only makes sense if page numbering scheme and format are
the same, e.\,g.\\
\verb|\Roman{pagesLTS.current.local.Roman}|\\
or \verb|\Alph{pagesLTS.current.local.Alph}|. \verb|\arabic{...}| could
make sense even if combined with another page numbering scheme.
And this is exactly what \verb|\theCurrentPageLocal| does:\\
\nolinebreak{\verb|\def\theCurrentPageLocal{\arabic{pagesLTS.current.local.\pagesLTS@pnc}}|.}\\

\noindent Last roman page (pagesLTS.roman): \lastpageref{pagesLTS.roman}{\hskip4em }
(There are \lastpageref{pagesLTS.roman.local} pages with roman numbers.)\\

\noindent Last Roman page (pagesLTS.Roman): \lastpageref{pagesLTS.Roman}{\hskip3em }
(There are \lastpageref{pagesLTS.Roman.local}~pages with Roman numbers:\\
\lastpages{Roman}{1}~pages in the first Roman sector
(\pageref{Roman}{\hskip3em }-\lastpageref{pagesLTS.Roman.1}{\hskip3em }),\\
\lastpages{Roman}{2}~pages in the second Roman sector
(\pageref{Roman2}{\hskip3em }-\lastpageref{pagesLTS.Roman.2}{\hskip3em }), and\\
\lastpages{Roman}{3}~pages in the third Roman sector
(\pageref{Roman3}{\hskip3em }-\lastpageref{pagesLTS.Roman.3}{\hskip3em }).\\

When the option \texttt{pagecontinue=false} is used with the
\textsf{pageslts} package, the \verb|\lastpageref{pagesLTS.Roman}| will
point to the same page as before, but this will have a lower number.\\
The \verb|\lastpageref{pagesLTS.Roman.local}| will not change,
because the number of pages does not change (only the page numbers).\\

\noindent Last arabic page (pagesLTS.arabic): \lastpageref{pagesLTS.arabic}{\hskip5em }
(There are only \lastpageref{pagesLTS.arabic.local} pages with arabic numbers,
because an \verb|\addtocounter{page}{|\pagesLTSexampleArabic\verb|}| was used.)\\

\noindent Last fnsymbol page (pagesLTS.fnsymbol): \lastpageref{pagesLTS.fnsymbol} \\
(\verb|\lastpageref{pagesLTS.fnsymbol}| -- never \verb|\pageref{pagesLTS.fnsymbol}|!)\\
(There are \lastpageref{pagesLTS.fnsymbol.local} pages with fnsymbol numbers.)\\

\noindent Last alph page (pagesLTS.alph): \lastpageref{pagesLTS.alph}{\hskip4em }
(There are only \lastpageref{pagesLTS.alph.local} pages with alph numbers,
because an \verb|\addtocounter{page}{|\pagesLTSexamplealph\verb|}| was used.)\\

\noindent Last Alph page (pagesLTS.Alph): \lastpageref{pagesLTS.Alph}{\hskip4em }
(There are \lastpageref{pagesLTS.Alph.local} pages with Alph numbers.)\\

\noindent Last page's \textit{name} (LastPage): \lastpageref{LastPage}\\

\noindent Very last page's \textit{name} (VeryLastPage): \lastpageref{VeryLastPage}\\
(\texttt{lastpage} and \texttt{VeryLastPage} are identical, unless
a package with output \linebreak
\verb|\AtEndDocument| after the \textsf{pageslts} package was added.)\\

\noindent Last page's \textit{number} (LastPages): \lastpageref{LastPages}{\hskip3em }
(=~total number of pages)\\

\lipsum[1-3]

\newpage

\pagenumbering{Roman}

\section{Roman\label{Roman}}
\subsection{Common Roman page numbering}

\noindent First page (\verb|\lastpageref{pagesLTS.0}|):
\lastpageref{pagesLTS.0}\\

\noindent The page (\verb|\thepage|): \thepage \\

\noindent Current page (\verb|\theCurrentPage|),
i.\,e. counted continuously from the first page): \theCurrentPage \\

\noindent CurrentPageLocal (\verb|\theCurrentPageLocal|),
i.\,e. counted continuously from the first page of the
current page numbering scheme): \theCurrentPageLocal \\

\noindent Last roman page (pagesLTS.roman): \lastpageref{pagesLTS.roman}{\hskip4em }
(There are \lastpageref{pagesLTS.roman.local} pages with roman numbers.)\\

\noindent Last Roman page (pagesLTS.Roman): \lastpageref{pagesLTS.Roman}{\hskip3em }
(There are \lastpageref{pagesLTS.Roman.local}~pages with Roman numbers:\\
\lastpages{Roman}{1}~pages in the first Roman sector
(\pageref{Roman}{\hskip3em }-\lastpageref{pagesLTS.Roman.1}{\hskip3em }),\\
\lastpages{Roman}{2}~pages in the second Roman sector
(\pageref{Roman2}{\hskip3em }-\lastpageref{pagesLTS.Roman.2}{\hskip3em }), and\\
\lastpages{Roman}{3}~pages in the third Roman sector
(\pageref{Roman3}{\hskip3em }-\lastpageref{pagesLTS.Roman.3}{\hskip3em }).\\

\noindent Last arabic page (pagesLTS.arabic): \lastpageref{pagesLTS.arabic}{\hskip5em }
(There are only \lastpageref{pagesLTS.arabic.local} pages with arabic numbers,
because an \verb|\addtocounter{page}{|\pagesLTSexampleArabic\verb|}| was used.)\\

\noindent Last fnsymbol page (pagesLTS.fnsymbol): \lastpageref{pagesLTS.fnsymbol} \\
(\verb|\lastpageref{pagesLTS.fnsymbol}| -- never \verb|\pageref{pagesLTS.fnsymbol}|!)\\
(There are \lastpageref{pagesLTS.fnsymbol.local} pages with fnsymbol numbers.)\\

\noindent Last alph page (pagesLTS.alph): \lastpageref{pagesLTS.alph}{\hskip4em }
(There are only \lastpageref{pagesLTS.alph.local} pages with alph numbers,
because an \verb|\addtocounter{page}{|\pagesLTSexamplealph\verb|}| was used.)\\

\noindent Last Alph page (pagesLTS.Alph): \lastpageref{pagesLTS.Alph}{\hskip4em }
(There are \lastpageref{pagesLTS.Alph.local} pages with Alph numbers.)\\

\noindent Last page's \textit{name} (LastPage): \lastpageref{LastPage}\\

\noindent Very last page's \textit{name} (VeryLastPage): \lastpageref{VeryLastPage}\\
(\texttt{LastPage} and \texttt{VeryLastPage} are identical, unless
a package with output \linebreak
\verb|\AtEndDocument| after the \textsf{pageslts} package was added.)\\

\noindent Last page's \textit{number} (LastPages): \lastpageref{LastPages}{\hskip3em }
(=~total number of pages)\\

\lipsum[1-4]

\newpage

\subsection{Last page of first Roman sector}
\verb|\lastpageref{pagesLTS.Roman}| does \textbf{not}
refer to this page (but there: \lastpageref{pagesLTS.Roman}),
because the option \texttt{pagecontinue=true}
was chosen. When a reference to this page is wanted,\\
\verb|\lastpageref{pagesLTS.Roman.1}| can be used: \lastpageref{pagesLTS.Roman.1}.\\

\bigskip
There are \lastpages{Roman}{1}~pages (\verb|\lastpages{Roman}{1}|)
in this first Roman sector.\\
The Roman page numbering scheme is continued later in section~\ref{Roman2})!

\newpage

\pagenumbering{arabic}

\section{arabic}

\subsection{Standard page numbering}

\noindent First page (\verb|\lastpageref{pagesLTS.0}|):
\lastpageref{pagesLTS.0}\\

\noindent The page (\verb|\thepage|): \thepage \\

\noindent Current page (\verb|\theCurrentPage|),
i.\,e. counted continuously from the first page): \theCurrentPage \\

\noindent CurrentPageLocal (\verb|\theCurrentPageLocal|),
i.\,e. counted continuously from the first page of the
current page numbering scheme): \theCurrentPageLocal \\

\noindent Last roman page (pagesLTS.roman): \lastpageref{pagesLTS.roman}{\hskip4em }
(There are \lastpageref{pagesLTS.roman.local} pages with roman numbers.)\\

\noindent Last Roman page (pagesLTS.Roman): \lastpageref{pagesLTS.Roman}{\hskip3em }
(There are \lastpageref{pagesLTS.Roman.local}~pages with Roman numbers:\\
\lastpages{Roman}{1}~pages in the first Roman sector
(\pageref{Roman}{\hskip3em }-\lastpageref{pagesLTS.Roman.1}{\hskip3em }),\\
\lastpages{Roman}{2}~pages in the second Roman sector
(\pageref{Roman2}{\hskip3em }-\lastpageref{pagesLTS.Roman.2}{\hskip3em }), and\\
\lastpages{Roman}{3}~pages in the third Roman sector
(\pageref{Roman3}{\hskip3em }-\lastpageref{pagesLTS.Roman.3}{\hskip3em }).\\

\noindent Last arabic page (pagesLTS.arabic): \lastpageref{pagesLTS.arabic}{\hskip5em }
(There are only \lastpageref{pagesLTS.arabic.local} pages with arabic numbers,
because an \verb|\addtocounter{page}{|\pagesLTSexampleArabic\verb|}| was used.)\\

\noindent Last fnsymbol page (pagesLTS.fnsymbol): \lastpageref{pagesLTS.fnsymbol} \\
(\verb|\lastpageref{pagesLTS.fnsymbol}| -- never \verb|\pageref{pagesLTS.fnsymbol}|!)\\
(There are \lastpageref{pagesLTS.fnsymbol.local} pages with fnsymbol numbers.)\\

\noindent Last alph page (pagesLTS.alph): \lastpageref{pagesLTS.alph}{\hskip4em }
(There are only \lastpageref{pagesLTS.alph.local} pages with alph numbers,
because an \verb|\addtocounter{page}{|\pagesLTSexamplealph\verb|}| was used.)\\

\noindent Last Alph page (pagesLTS.Alph): \lastpageref{pagesLTS.Alph}{\hskip4em }
(There are \lastpageref{pagesLTS.Alph.local} pages with Alph numbers.)\\

\noindent Last page's \textit{name} (LastPage): \lastpageref{LastPage}\\

\noindent Very last page's \textit{name} (VeryLastPage): \lastpageref{VeryLastPage}\\
(\texttt{LastPage} and \texttt{VeryLastPage} are identical, unless
a package with output \linebreak
\verb|\AtEndDocument| after the \textsf{pagesLTS} package was added.)\\

\noindent Last page's \textit{number} (LastPages): \lastpageref{LastPages}{\hskip3em }
(=~total number of pages)\\

\lipsum[1-4]
\newpage

\subsection[Empty page style]{Also an empty page style is no problem %
for the current or total page count}

\bigskip

\thispagestyle{empty}

\noindent First page (\verb|\lastpageref{pagesLTS.0}|):
\lastpageref{pagesLTS.0}\\

\noindent The page (\verb|\thepage|): \thepage \\

\noindent Current page (\verb|\theCurrentPage|),
i.\,e. counted continuously from the first page): \theCurrentPage \\

\noindent CurrentPageLocal (\verb|\theCurrentPageLocal|),
i.\,e. counted continuously from the first page of the
current page numbering scheme): \theCurrentPageLocal \\

\noindent Last roman page (pagesLTS.roman): \lastpageref{pagesLTS.roman}{\hskip4em }
(There are \lastpageref{pagesLTS.roman.local} pages with roman numbers.)\\

\noindent Last Roman page (pagesLTS.Roman): \lastpageref{pagesLTS.Roman}{\hskip3em }
(There are \lastpageref{pagesLTS.Roman.local}~pages with Roman numbers:\\
\lastpages{Roman}{1}~pages in the first Roman sector
(\pageref{Roman}{\hskip3em }-\lastpageref{pagesLTS.Roman.1}{\hskip3em }),\\
\lastpages{Roman}{2}~pages in the second Roman sector
(\pageref{Roman2}{\hskip3em }-\lastpageref{pagesLTS.Roman.2}{\hskip3em }), and\\
\lastpages{Roman}{3}~pages in the third Roman sector
(\pageref{Roman3}{\hskip3em }-\lastpageref{pagesLTS.Roman.3}{\hskip3em }).\\

\noindent Last arabic page (pagesLTS.arabic): \lastpageref{pagesLTS.arabic}{\hskip5em }
(There are only \lastpageref{pagesLTS.arabic.local} pages with arabic numbers,
because an \verb|\addtocounter{page}{|\pagesLTSexampleArabic\verb|}| was used.)\\

\noindent Last fnsymbol page (pagesLTS.fnsymbol): \lastpageref{pagesLTS.fnsymbol} \\
(\verb|\lastpageref{pagesLTS.fnsymbol}| -- never \verb|\pageref{pagesLTS.fnsymbol}|!)\\
(There are \lastpageref{pagesLTS.fnsymbol.local} pages with fnsymbol numbers.)\\

\noindent Last alph page (pagesLTS.alph): \lastpageref{pagesLTS.alph}{\hskip4em }
(There are only \lastpageref{pagesLTS.alph.local} pages with alph numbers,
because an \verb|\addtocounter{page}{|\pagesLTSexamplealph\verb|}| was used.)\\

\noindent Last Alph page (pagesLTS.Alph): \lastpageref{pagesLTS.Alph}{\hskip4em }
(There are \lastpageref{pagesLTS.Alph.local} pages with Alph numbers.)\\

\noindent Last page's \textit{name} (LastPage): \lastpageref{LastPage}\\

\noindent Very last page's \textit{name} (VeryLastPage): \lastpageref{VeryLastPage}\\
(\texttt{LastPage} and \texttt{VeryLastPage} are identical, unless
a package with output \linebreak
\verb|\AtEndDocument| after the \textsf{pageslts} package was added.)\\

\noindent Last page's \textit{number} (LastPages): \lastpageref{LastPages}{\hskip3em }
(=~total number of pages)\\

\lipsum[1-4]

\newpage

\subsection[addtocounter, setcounter]{Neither %
\texttt{\textbackslash addtocounter\{page\}} nor %
\texttt{\textbackslash setcounter\{page\}} is a problem for the %
current or total page numbers}

(Here is an \verb|\addtocounter{page}{|\pagesLTSexampleArabic\verb|}|
in the source code.)\\
\addtocounter{page}{\pagesLTSexampleArabic}

\noindent The page (from \verb|\thepage| command): \thepage \\

\noindent Current page (from \verb|\theCurrentPage|\ %
command), i.\,e. counted continuously from the first page): \theCurrentPage \\

\noindent CurrentPageLocal (from \verb|\theCurrentPageLocal|\ %
command), i.\,e. counted continuously from the first page of the
current page numbering scheme): \theCurrentPageLocal\\

\noindent Last page's number (LastPages): \lastpageref{LastPages}{\hskip3em }
(= total number of pages)\\

\lipsum[1-7]

\newpage

\pagenumbering{fnsymbol}

\section{fnsymbol}

Adobe Reader DC 2015.008.20082 does not show the correct page names
for all pages with \texttt{fnsymbol} page numbering scheme,
while at least the (\ldots\ of \ldots )  part of the page number is displayed correctly.
(Adobe Reader~X even got all pages right.)

\bigskip

Without option \texttt{fnsymbolmult=true} of the \textsf{pageslts} package
(and the help of \textsc{Heiko Oberdiek}'s \textsf{alphalph} package),
after page~9 (\textquotedblleft \ensuremath {\ddagger \ddagger }\textquotedblright )
(and also for negative page numbers) there would just appear a
\begin{quote}
\begin{verbatim}
LaTeX Error: Counter too large
See the LaTeX manual or LaTeX Companion for explanation.
You've lost some text.  Try typing  <return>  to proceed.
If that doesn't work, type  X <return>  to quit.
\end{verbatim}
\end{quote}
Now the page numbers after 5 (\ensuremath {\mathparagraph }) are
continued  with the doubled \textquotedblleft number\textquotedblright{} of
the first, second, third,\ldots\ page (\ensuremath {**},
\ensuremath {\dagger \dagger }, \ensuremath {\ddagger \ddagger },
\ensuremath {\mathsection \mathsection },
\ensuremath {\mathparagraph \mathparagraph }),
and after the tenth page the \textquotedblleft number\textquotedblright{} is
tripled (\ensuremath {***}, \ensuremath {\dagger \dagger \dagger },\ldots).
Page zero is named 0 and negative pages just named like the positive ones
with addition of a minus sign~($-$).

\bigskip

\noindent First page (\verb|\lastpageref{pagesLTS.0}|):
\lastpageref{pagesLTS.0}\\

\noindent The page (\verb|\thepage|): \thepage \\

\noindent Current page (\verb|\theCurrentPage|),
i.\,e. counted continuously from the first page): \theCurrentPage \\

\noindent CurrentPageLocal (\verb|\theCurrentPageLocal|),
i.\,e. counted continuously from the first page of the
current page numbering scheme): \theCurrentPageLocal \\

\noindent Last roman page (pagesLTS.roman): \lastpageref{pagesLTS.roman}{\hskip4em }
(There are \lastpageref{pagesLTS.roman.local} pages with roman numbers.)\\

\noindent Last Roman page (pagesLTS.Roman): \lastpageref{pagesLTS.Roman}{\hskip3em }
(There are \lastpageref{pagesLTS.Roman.local}~pages with Roman numbers:\\
\lastpages{Roman}{1}~pages in the first Roman sector
(\pageref{Roman}{\hskip3em }-\lastpageref{pagesLTS.Roman.1}{\hskip3em }),\\
\lastpages{Roman}{2}~pages in the second Roman sector
(\pageref{Roman2}{\hskip3em }-\lastpageref{pagesLTS.Roman.2}{\hskip3em }), and\\
\lastpages{Roman}{3}~pages in the third Roman sector
(\pageref{Roman3}{\hskip3em }-\lastpageref{pagesLTS.Roman.3}{\hskip3em }).\\

\noindent Last arabic page (pagesLTS.arabic): \lastpageref{pagesLTS.arabic}{\hskip5em }
(There are only \lastpageref{pagesLTS.arabic.local} pages with arabic numbers,
because an \verb|\addtocounter{page}{|\pagesLTSexampleArabic\verb|}| was used.)\\

\noindent Last fnsymbol page (pagesLTS.fnsymbol): \lastpageref{pagesLTS.fnsymbol} \\
(\verb|\lastpageref{pagesLTS.fnsymbol}| -- never \verb|\pageref{pagesLTS.fnsymbol}|!)\\
(There are \lastpageref{pagesLTS.fnsymbol.local} pages with fnsymbol numbers.)\\

\noindent Last alph page (pagesLTS.alph): \lastpageref{pagesLTS.alph}{\hskip4em }
(There are only \lastpageref{pagesLTS.alph.local} pages with alph numbers,
because an \verb|\addtocounter{page}{|\pagesLTSexamplealph\verb|}| was used.)\\

\noindent Last Alph page (pagesLTS.Alph): \lastpageref{pagesLTS.Alph}{\hskip4em }
(There are \lastpageref{pagesLTS.Alph.local} pages with Alph numbers.)\\

\noindent Last page's \textit{name} (LastPage): \lastpageref{LastPage}\\

\noindent Very last page's \textit{name} (VeryLastPage): \lastpageref{VeryLastPage}\\
(\texttt{LastPage} and \texttt{VeryLastPage} are identical, unless
a package with output \linebreak
\verb|\AtEndDocument| after the \textsf{pageslts} package was added.)\\

\noindent Last page's \textit{number} (LastPages): \lastpageref{LastPages}{\hskip3em }
(=~total number of pages)\\

\lipsum[1-60]

\newpage

\pagenumbering{Roman}

\section{Roman - again!\label{Roman2}}

The page number would start with \textquotedblleft I\textquotedblright{} again --
but for the \textsf{pageslts} package (with option \texttt{pagecontinue=true},
or with option just \texttt{pagecontinue}, or even just
\nolinebreak{with\textbf{out}} option \texttt{pagecontinue=false}).
This package remembered the (\arabic{pagesLTS.double.Roman}-1)\footnote{%
OK, here you have to compute this value for yourself, but %
subtracting one should be manageable for \TeX{}nicians.} pages already
done in Roman output, and therefore continues with page
\textquotedblleft \thepage \textquotedblright .\\
If you want to start with \textquotedblleft I\textquotedblright{} all
over again, you will have two pages with the same name,
but nevertheless you can do this by using option \texttt{pagecontinue=false}
or a \verb|\setcounter{page}{1}|\ here (not demonstrated in this example file).\\

\noindent First page (\verb|\lastpageref{pagesLTS.0}|):
 \lastpageref{pagesLTS.0}\\

\noindent The page (\verb|\thepage|): \thepage \\

\noindent Current page (\verb|\theCurrentPage|),
i.\,e. counted continuously from the first page): \theCurrentPage \\

\noindent CurrentPageLocal (\verb|\theCurrentPageLocal|),
i.\,e. counted continuously from the first page of the
current page numbering scheme): \theCurrentPageLocal \\

\noindent Last roman page (pagesLTS.roman): \lastpageref{pagesLTS.roman}{\hskip4em }
(There are \lastpageref{pagesLTS.roman.local} pages with roman numbers.)\\

\noindent Last Roman page (pagesLTS.Roman): \lastpageref{pagesLTS.Roman}{\hskip3em }
(There are \lastpageref{pagesLTS.Roman.local}~pages with Roman numbers:\\
\lastpages{Roman}{1}~pages in the first Roman sector
(\pageref{Roman}{\hskip3em }-\lastpageref{pagesLTS.Roman.1}{\hskip3em }),\\
\lastpages{Roman}{2}~pages in the second Roman sector
(\pageref{Roman2}{\hskip3em }-\lastpageref{pagesLTS.Roman.2}{\hskip3em }), and\\
\lastpages{Roman}{3}~pages in the third Roman sector
(\pageref{Roman3}{\hskip3em }-\lastpageref{pagesLTS.Roman.3}{\hskip3em }).\\

\noindent Last arabic page (pagesLTS.arabic): \lastpageref{pagesLTS.arabic}{\hskip5em }
(There are only \lastpageref{pagesLTS.arabic.local} pages with arabic numbers,
because an \verb|\addtocounter{page}{|\pagesLTSexampleArabic\verb|}| was used.)\\

\noindent Last fnsymbol page (pagesLTS.fnsymbol): \lastpageref{pagesLTS.fnsymbol} \\
(\verb|\lastpageref{pagesLTS.fnsymbol}| -- never
\verb|\pageref{pagesLTS.fnsymbol}|!)\\
(There are \lastpageref{pagesLTS.fnsymbol.local} pages with fnsymbol numbers.)\\

\noindent Last alph page (pagesLTS.alph): \lastpageref{pagesLTS.alph}{\hskip4em }
(There are only \lastpageref{pagesLTS.alph.local} pages with alph numbers,
because an \verb|\addtocounter{page}{|\pagesLTSexamplealph\verb|}| was used.)\\

\noindent Last Alph page (pagesLTS.Alph): \lastpageref{pagesLTS.Alph}{\hskip4em }
(There are \lastpageref{pagesLTS.Alph.local} pages with Alph numbers.)\\

\noindent Last page's \textit{name} (LastPage): \lastpageref{LastPage}\\

\noindent Very last page's \textit{name} (VeryLastPage): \lastpageref{VeryLastPage}\\
(\texttt{LastPage} and \texttt{VeryLastPage} are identical, unless
a package with output \linebreak
\verb|\AtEndDocument| after the \textsf{pageslts} package was added.)\\

\noindent Last page's \textit{number} (LastPages): \lastpageref{LastPages}{\hskip3em }
(=~total number of pages)\\

\lipsum[1-6]

\newpage

\verb|\lastpageref{pagesLTS.Roman}| does \textbf{not}
refer to this page (but there: \lastpageref{pagesLTS.Roman}),
because the option \texttt{pagecontinue=true}
was chosen. When a reference to this page is wanted,\\
\verb|\lastpageref{pagesLTS.Roman.2}| can be used: \lastpageref{pagesLTS.Roman.2}.\\

\bigskip
There are \lastpages{Roman}{2}~pages (\verb|\lastpages{Roman}{2}|) in this
second Roman sector.\\
The Roman page numbering scheme is continued later in section~\ref{Roman3})!

\newpage

\pagenumbering{alph}

\section{alph\label{alph}}

\noindent First page (\verb|\lastpageref{pagesLTS.0}|):
 \lastpageref{pagesLTS.0}\\

\noindent The page (\verb|\thepage|): \thepage \\

\noindent Current page (\verb|\theCurrentPage|),
i.\,e. counted continuously from the first page): \theCurrentPage \\

\noindent CurrentPageLocal (\verb|\theCurrentPageLocal|),
i.\,e. counted continuously from the first page of the
current page numbering scheme): \theCurrentPageLocal \\

\noindent Last roman page (pagesLTS.roman): \lastpageref{pagesLTS.roman}{\hskip4em }
(There are \lastpageref{pagesLTS.roman.local} pages with roman numbers.)\\

\noindent Last Roman page (pagesLTS.Roman): \lastpageref{pagesLTS.Roman}{\hskip3em }
(There are \lastpageref{pagesLTS.Roman.local}~pages with Roman numbers:\\
\lastpages{Roman}{1}~pages in the first Roman sector
(\pageref{Roman}{\hskip3em }-\lastpageref{pagesLTS.Roman.1}{\hskip3em }),\\
\lastpages{Roman}{2}~pages in the second Roman sector
(\pageref{Roman2}{\hskip3em }-\lastpageref{pagesLTS.Roman.2}{\hskip3em }), and\\
\lastpages{Roman}{3}~pages in the third Roman sector
(\pageref{Roman3}{\hskip3em }-\lastpageref{pagesLTS.Roman.3}{\hskip3em }).\\

\noindent Last arabic page (pagesLTS.arabic): \lastpageref{pagesLTS.arabic}{\hskip5em }
(There are only \lastpageref{pagesLTS.arabic.local} pages with arabic numbers,
because an \verb|\addtocounter{page}{|\pagesLTSexampleArabic\verb|}| was used.)\\

\noindent Last fnsymbol page (pagesLTS.fnsymbol): \lastpageref{pagesLTS.fnsymbol} \\
(\verb|\lastpageref{pagesLTS.fnsymbol}| -- never \verb|\pageref{pagesLTS.fnsymbol}|!)\\
(There are \lastpageref{pagesLTS.fnsymbol.local} pages with fnsymbol numbers.)\\

\noindent Last alph page (pagesLTS.alph): \lastpageref{pagesLTS.alph}{\hskip4em }
(There are only \lastpageref{pagesLTS.alph.local} pages with alph numbers,
because an \verb|\addtocounter{page}{|\pagesLTSexamplealph\verb|}| was used.)\\

\noindent Last Alph page (pagesLTS.Alph): \lastpageref{pagesLTS.Alph}{\hskip4em }
(There are \lastpageref{pagesLTS.Alph.local} pages with Alph numbers.)\\

\noindent Last page's \textit{name} (LastPage): \lastpageref{LastPage}\\

\noindent Very last page's \textit{name} (VeryLastPage): \lastpageref{VeryLastPage}\\
(\texttt{LastPage} and \texttt{VeryLastPage} are identical, unless
a package with output \linebreak
\verb|\AtEndDocument| after the \textsf{pageslts} package was added.)\\

\noindent Last page's \textit{number} (LastPages): \lastpageref{LastPages}{\hskip3em }
(=~total number of pages)\\

\lipsum[1-4]

\newpage

Without option \texttt{alphMult=ab} of the \textsf{pageslts} (and the help of
\textsc{Heiko Oberdiek's} \textsf{alphalph} package), after page
\textquotedblleft z\textquotedblright{} there would just appear a
\begin{quote}
\begin{verbatim}
LaTeX Error: Counter too large
See the LaTeX manual or LaTeX Companion for explanation.
You've lost some text.  Try typing  <return>  to proceed.
If that doesn't work, type  X <return>  to quit.
\end{verbatim}
\end{quote}
Now the page numbers are continued aa, ab, ac,\ldots\ (aa, bb, cc,\ldots\ is
also possible, see the \textsf{pageslts} documentation).\\
To demonstrate this, we add a\\
\verb|\addtocounter{page}{|\pagesLTSexamplealph\verb|}|\\
in the source code here.

\addtocounter{page}{\pagesLTSexamplealph}

\bigskip

\lipsum[1-18]

\newpage

\pagenumbering{Roman}

\section{Roman - third time!\label{Roman3}}

The page number would start with
\textquotedblleft I\textquotedblright{} again -- but
for the \textsf{pageslts} package (with option \texttt{pagecontinue=true},
or with option just \texttt{pagecontinue}, or even just
\nolinebreak{with\textbf{out}} option \texttt{pagecontinue=false}).
This package remembered the
(\arabic{pagesLTS.double.Roman}-1)\footnote{OK, here %
you have to compute this value for yourself, but %
subtracting one should be managable for \TeX nicians.} pages already
done in Roman output, and therefore continues with page
\textquotedblleft \thepage \textquotedblright .\\
If you want to start with \textquotedblleft I\textquotedblright{} all
over again, you will have (at least) two pages with the same name,
but nevertheless you can do this by using option
\texttt{pagecontinue=false} instead of \texttt{pagecontinue=true}
(not demonstrated here).\\

\noindent First page (\verb|\lastpageref{pagesLTS.0}|):
 \lastpageref{pagesLTS.0}\\

\noindent The page (\verb|\thepage|): \thepage \\

\noindent Current page (\verb|\theCurrentPage|),
i.\,e. counted continuously from the first page): \theCurrentPage \\

\noindent CurrentPageLocal (\verb|\theCurrentPageLocal|),
i.\,e. counted continuously from the first page of the
current page numbering scheme): \theCurrentPageLocal \\

\noindent Last roman page (pagesLTS.roman): \lastpageref{pagesLTS.roman}{\hskip4em }
(There are \lastpageref{pagesLTS.roman.local} pages with roman numbers.)\\

\noindent Last Roman page (pagesLTS.Roman): \lastpageref{pagesLTS.Roman}{\hskip3em }
(There are \lastpageref{pagesLTS.Roman.local}~pages with Roman numbers:\\
\lastpages{Roman}{1}~pages in the first Roman sector
(\pageref{Roman}{\hskip3em }-\lastpageref{pagesLTS.Roman.1}{\hskip3em }),\\
\lastpages{Roman}{2}~pages in the second Roman sector
(\pageref{Roman2}{\hskip3em }-\lastpageref{pagesLTS.Roman.2}{\hskip3em }), and\\
\lastpages{Roman}{3}~pages in the third Roman sector
(\pageref{Roman3}{\hskip3em }-\lastpageref{pagesLTS.Roman.3}{\hskip3em }).\\

\noindent Last arabic page (pagesLTS.arabic): \lastpageref{pagesLTS.arabic}{\hskip5em }
(There are only \lastpageref{pagesLTS.arabic.local} pages with arabic numbers,
because an \verb|\addtocounter{page}{|\pagesLTSexampleArabic\verb|}| was used.)\\

\noindent Last fnsymbol page (pagesLTS.fnsymbol): \lastpageref{pagesLTS.fnsymbol} \\
(\verb|\lastpageref{pagesLTS.fnsymbol}| -- never \verb|\pageref{pagesLTS.fnsymbol}|!)\\
(There are \lastpageref{pagesLTS.fnsymbol.local} pages with fnsymbol numbers.)\\

\noindent Last alph page (pagesLTS.alph): \lastpageref{pagesLTS.alph}{\hskip4em }
(There are only \lastpageref{pagesLTS.alph.local} pages with alph numbers,
because an \verb|\addtocounter{page}{|\pagesLTSexamplealph\verb|}| was used.)\\

\noindent Last Alph page (pagesLTS.Alph): \lastpageref{pagesLTS.Alph}{\hskip4em }
(There are \lastpageref{pagesLTS.Alph.local} pages with Alph numbers.)\\

\noindent Last page's \textit{name} (LastPage): \lastpageref{LastPage}\\

\noindent Very last page's \textit{name} (VeryLastPage): \lastpageref{VeryLastPage}\\
(\texttt{LastPage} and \texttt{VeryLastPage} are identical, unless
a package with output \linebreak
\verb|\AtEndDocument| after the \textsf{pageslts} package was added.)\\

\noindent Last page's \textit{number} (LastPages): \lastpageref{LastPages}{\hskip3em }
(=~total number of pages)\\

\lipsum[1-3]

\newpage

\lastpageref{pagesLTS.Roman}
(\verb|\lastpageref{pagesLTS.Roman}|)
\textbf{does} refers to this page, because the option
\texttt{pagecontinue=true} was chosen. Also\\
\verb|\lastpageref{pagesLTS.Roman.3}| can be used: \lastpageref{pagesLTS.Roman.3}.\\

\bigskip

There are \lastpages{Roman}{3}~pages (\verb|\lastpages{Roman}{3}|) in this
third Roman sector.\\

\newpage

\pagenumbering{Alph}

\section{Alph}

\noindent First page (\verb|\lastpageref{pagesLTS.0}|):
 \lastpageref{pagesLTS.0}\\

\noindent The page (\verb|\thepage|): \thepage \\

\noindent Current page (\verb|\theCurrentPage|),
i.\,e. counted continuously from the first page): \theCurrentPage \\

\noindent CurrentPageLocal (\verb|\theCurrentPageLocal|),
i.\,e. counted continuously from the first page of the
current page numbering scheme): \theCurrentPageLocal \\

\noindent Last roman page (pagesLTS.roman): \lastpageref{pagesLTS.roman}{\hskip4em }
(There are \lastpageref{pagesLTS.roman.local} pages with roman numbers.)\\

\noindent Last Roman page (pagesLTS.Roman): \lastpageref{pagesLTS.Roman}{\hskip3em }
(There are \lastpageref{pagesLTS.Roman.local}~pages with Roman numbers:\\
\lastpages{Roman}{1}~pages in the first Roman sector
(\pageref{Roman}{\hskip3em }-\lastpageref{pagesLTS.Roman.1}{\hskip3em }),\\
\lastpages{Roman}{2}~pages in the second Roman sector
(\pageref{Roman2}{\hskip3em }-\lastpageref{pagesLTS.Roman.2}{\hskip3em }), and\\
\lastpages{Roman}{3}~pages in the third Roman sector
(\pageref{Roman3}{\hskip3em }-\lastpageref{pagesLTS.Roman.3}{\hskip3em }).\\

\noindent Last arabic page (pagesLTS.arabic): \lastpageref{pagesLTS.arabic}{\hskip5em }
(There are only \lastpageref{pagesLTS.arabic.local} pages with arabic numbers,
because an \verb|\addtocounter{page}{|\pagesLTSexampleArabic\verb|}| was used.)\\

\noindent Last fnsymbol page (pagesLTS.fnsymbol): \lastpageref{pagesLTS.fnsymbol} \\
(\verb|\lastpageref{pagesLTS.fnsymbol}| -- never \verb|\pageref{pagesLTS.fnsymbol}|!)\\
(There are \lastpageref{pagesLTS.fnsymbol.local} pages with fnsymbol numbers.)\\

\noindent Last alph page (pagesLTS.alph): \lastpageref{pagesLTS.alph}{\hskip4em }
(There are only \lastpageref{pagesLTS.alph.local} pages with alph numbers,
because an \verb|\addtocounter{page}{|\pagesLTSexamplealph\verb|}| was used.)\\

\noindent Last Alph page (pagesLTS.Alph): \lastpageref{pagesLTS.Alph}{\hskip4em }
(There are \lastpageref{pagesLTS.Alph.local} pages with Alph numbers.)\\

\noindent Last page's \textit{name} (LastPage): \lastpageref{LastPage}\\

\noindent Very last page's \textit{name} (VeryLastPage): \lastpageref{VeryLastPage}\\
(\texttt{LastPage} and \texttt{VeryLastPage} are identical, unless
a package with output \linebreak
\verb|\AtEndDocument| after the \textsf{pageslts} package was added.)\\

\noindent Last page's \textit{number} (LastPages): \lastpageref{LastPages}{\hskip3em }
(=~total number of pages)\\

\lipsum[1-3]

Without option \texttt{alphMulti=AB} of the \textsf{pageslts} (and the help of
\textsc{Heiko Oberdiek's} \textsf{alphalph} package), after page
\textquotedblleft Z\textquotedblright{} there would just appear a
\begin{quote}
\begin{verbatim}
LaTeX Error: Counter too large
See the LaTeX manual or LaTeX Companion for explanation.
You've lost some text.  Try typing  <return>  to proceed.
If that doesn't work, type  X <return>  to quit.
\end{verbatim}
\end{quote}
Now the page numbers are continued AA, AB, AC,\ldots\ (AA, BB, CC,\ldots\ is
also possible, see the \textsf{pageslts} documentation).\\
This is not demonstrated here, but see section~\ref{alph}.

\newpage

\section{The End}

\noindent First page (\verb|\lastpageref{pagesLTS.0}|):
 \lastpageref{pagesLTS.0}\\

\noindent The page (\verb|\thepage|): \thepage \\

\noindent Current page (\verb|\theCurrentPage|),
i.\,e. counted continuously from the first page): \theCurrentPage \\

\noindent CurrentPageLocal (\verb|\theCurrentPageLocal|),
i.\,e. counted continuously from the first page of the
current page numbering scheme): \theCurrentPageLocal \\

\noindent Last roman page (pagesLTS.roman): \lastpageref{pagesLTS.roman}{\hskip4em }
(There are \lastpageref{pagesLTS.roman.local} pages with roman numbers.)\\

\noindent Last Roman page (pagesLTS.Roman): \lastpageref{pagesLTS.Roman}{\hskip3em }
(There are \lastpageref{pagesLTS.Roman.local}~pages with Roman numbers:\\
\lastpages{Roman}{1}~pages in the first Roman sector
(\pageref{Roman}{\hskip3em }-\lastpageref{pagesLTS.Roman.1}{\hskip3em }),\\
\lastpages{Roman}{2}~pages in the second Roman sector
(\pageref{Roman2}{\hskip3em }-\lastpageref{pagesLTS.Roman.2}{\hskip3em }), and\\
\lastpages{Roman}{3}~pages in the third Roman sector
(\pageref{Roman3}{\hskip3em }-\lastpageref{pagesLTS.Roman.3}{\hskip3em }).\\

\noindent Last arabic page (pagesLTS.arabic): \lastpageref{pagesLTS.arabic}{\hskip5em }
(There are only \lastpageref{pagesLTS.arabic.local} pages with arabic numbers,
because an \verb|\addtocounter{page}{|\pagesLTSexampleArabic\verb|}| was used.)\\

\noindent Last fnsymbol page (pagesLTS.fnsymbol): \lastpageref{pagesLTS.fnsymbol} \\
(\verb|\lastpageref{pagesLTS.fnsymbol}| -- never \verb|\pageref{pagesLTS.fnsymbol}|!)\\
(There are \lastpageref{pagesLTS.fnsymbol.local} pages with fnsymbol numbers.)\\

\noindent Last alph page (pagesLTS.alph): \lastpageref{pagesLTS.alph}{\hskip4em }
(There are only \lastpageref{pagesLTS.alph.local} pages with alph numbers,
because an \verb|\addtocounter{page}{|\pagesLTSexamplealph\verb|}| was used.)\\

\noindent Last Alph page (pagesLTS.Alph): \lastpageref{pagesLTS.Alph}{\hskip4em }
(There are \lastpageref{pagesLTS.Alph.local} pages with Alph numbers.)\\

\noindent Last page's \textit{name} (LastPage): \lastpageref{LastPage}\\

\noindent Very last page's \textit{name} (VeryLastPage): \lastpageref{VeryLastPage}\\
(\texttt{LastPage} and \texttt{VeryLastPage} are identical, unless
a package with output \linebreak
\verb|\AtEndDocument| after the \textsf{pageslts} package was added.)\\

\noindent Last page's \textit{number} (LastPages): \lastpageref{LastPages}{\hskip3em }
(=~total number of pages)\\

\medskip

\noindent Page \thepage\ (\theCurrentPage; local: \theCurrentPageLocal) of %
\lastpageref{pagesLTS.roman} (\lastpageref{pagesLTS.roman.local}) + %
\lastpageref{pagesLTS.Roman} (\lastpageref{pagesLTS.Roman.local}) + %
\lastpageref{pagesLTS.arabic} (\lastpageref{pagesLTS.arabic.local}) + %
\lastpageref{pagesLTS.fnsymbol} (\lastpageref{pagesLTS.fnsymbol.local}) + %
\lastpageref{pagesLTS.alph} (\lastpageref{pagesLTS.alph.local}) + %
\lastpageref{pagesLTS.Alph} (\lastpageref{pagesLTS.Alph.local}) = %
\lastpageref{LastPages} pages.

\end{document}
%</example>
%    \end{macrocode}
%
% \pagebreak
%
% \StopEventually{}
%
% \section{The implementation}
%
% (This and the source code of the example file are the reasons
% for printing the documentation in landscape format instead of portrait.)\\
%
% We start off by checking that we are loading into \LaTeXe{} and
% announcing the name and version of this package.
%
%    \begin{macrocode}
%<*package>
%    \end{macrocode}
%
%    \begin{macrocode}
\NeedsTeXFormat{LaTeX2e}[2015/01/01]
\ProvidesPackage{pageslts}[2015/12/21 v1.2f
            Refers to special pages' numbers/names (HMM)]

%    \end{macrocode}
%
% A short description of the \xpackage{pageslts} package:
%
%    \begin{macrocode}
%% Allows for things like\\
%% |Page \thepage\ (\theCurrentPage; local: \theCurrentPageLocal) of %
%%  \lastpageref{pagesLTS.roman}(\lastpageref{pagesLTS.roman.local}) + %
%%  \lastpageref{pagesLTS.Roman}(\lastpageref{pagesLTS.Roman.local}) + %
%%  \lastpageref{pagesLTS.arabic}(\lastpageref{pagesLTS.arabic.local}) + %
%%  \lastpageref{pagesLTS.fnsymbol}(\lastpageref{pagesLTS.fnsymbol.local}) + %
%%  \lastpageref{pagesLTS.alph}(\lastpageref{pagesLTS.alph.local}) + %
%%  \lastpageref{pagesLTS.Alph}(\lastpageref{pagesLTS.Alph.local}) = %
%%  \lastpageref{LastPages} pages.|\\
%% to get\\
%% 'Page d (57; local: 4) of ii(2) + XX(20) + *(1) + 30(30) + e(5) + C(3) = 61 pages.'.

%    \end{macrocode}
%
% The package is now named `pageslts' but had been named `pagesLTS' before.
% \LaTeX\ does not load a package two times, but it is case sensitive,
% i.\,e. \emph{would} load `pageslts' and `pagesLTS', resulting in
% possible problems. Therefore we check for a loaded `pagesLTS' package
% (assuming |\pagesLTS@loaded| is not |p@gesLTSlo@ded| and not defined to
% |p@gesLTSlo@ded| by another package). This check might work only for
% the second and following compilation runs.
%
%    \begin{macrocode}
\@ifpackageloaded{pagesLTS}{%
  \def\pagesLTS@doubleload{p@gesLTSlo@ded}
  \ifx\pagesLTS@loaded\pagesLTS@doubleload%
    \PackageError{pageslts}{Requested pagesLTS instead of pageslts}{%
      You have requested package `pagesLTS'. This package is now named `pageslts'.\MessageBreak%
      Requesting `pagesLTS' as well as `pageslts' leads to loading the same package\MessageBreak%
      twice (and results in errors). Loading of `pageslts' will therefore be\MessageBreak%
      abandoned now. To fix this problem, please look in the \jobname.log file for\MessageBreak%
      LaTeX Warning: You have requested package `pagesLTS',\MessageBreak%
      \space \space \space \space \space \space \space \space \space \space \space \space \space%
      \space \space but the package provides `pageslts'.\MessageBreak%
      and change the package call in the calling style/document.\MessageBreak%
      If a style file made the mistake, please inform its maintainer and/or the\MessageBreak%
      maintainer of the pageslts package about it (after making sure you have\MessageBreak%
      the recent version of that style file).\MessageBreak%
      Thank you and sorry for the inconvenience!\MessageBreak%
     }
    \endinput
  \fi
  }{% \relax
   }

\gdef\pagesLTS@loaded{p@gesLTSlo@ded}

%    \end{macrocode}
%
% For |\ltx@ifpackageloaded| the \xpackage{ltxcmds} package is needed,
% also by \textsc{Heiko Oberdiek} (see subsection~\ref{ss:Downloads}):
%
%    \begin{macrocode}
\RequirePackage{ltxcmds}[2011/11/09]%        v1.22
%    \end{macrocode}
%
% For its |\AfterLastShipout| (as well as its |\AtEndAfterFileList|) command
% we need the \xpackage{atveryend} package by \textsc{Heiko Oberdiek}
% (see subsection~\ref{ss:Downloads}):
%
%    \begin{macrocode}
\RequirePackage{atveryend}[2011/06/30]%      v1.8
\AtBeginDocument{%
  %% https://tex.stackexchange.com/a/45654/6865
  %% and https://tex.stackexchange.com/q/261445/6865
  \def\pagesLTS@etb{etoolbox.sty}%
  \let\pagesLTS@ave\@empty%
  \@for\@pageslts@currname:=\@filelist\do{%
     \ifx\@pageslts@currname\pagesLTS@etb\relax%
       \def\pagesLTS@ave{atveryend.sty}%
     \fi%
     \ifx\@pageslts@currname\pagesLTS@ave%
       \@ifpackagelater{atveryend}{2015/08/01}{%\relax
          \PackageInfo{pageslts}{etoolbox package loaded before atveryend package\MessageBreak%
            With atveryend version 1.8 of 2011/06/30 this caused a problem,\MessageBreak%
            which is probably solved in the new version used here.\MessageBreak%
            To play it safe, you could load the atveryend package\MessageBreak%
            before the etoolbox package.\MessageBreak%
            }%
        }{% else
          \PackageError{pageslts}{etoolbox package loaded before atveryend package}{%
            You can use both packages, but when etoolbox is loaded before atveryend,\MessageBreak%
            atveryend no longer detects \string\enddocument\space\MessageBreak% 
            and hook "AtVeryVeryEnd" is not executed.\MessageBreak%
            Just load the atveryend package before the etoolbox package!\MessageBreak%
            }%
         }%
     \fi%
   }%
  }

%    \end{macrocode}
%
% For its |\EveryShipout| command we need the \xpackage{everyshi}
% package by \textsc{Martin Schr\"{o}der} (see subsection~\ref{ss:Downloads}):
%
%    \begin{macrocode}
\RequirePackage{everyshi}[2001/05/15]%       v3.00
%    \end{macrocode}
%
% For its |\LetLtxMacro| command we need the \xpackage{letltxmacro}
% package by \textsc{Heiko Oberdiek} (see subsection~\ref{ss:Downloads}):
%
%    \begin{macrocode}
\RequirePackage{letltxmacro}[2010/09/02]%    v1.4
%    \end{macrocode}
%
% For the handling of the options we need the \xpackage{kvoptions}
% package also by \textsc{Heiko Oberdiek} (see subsection~\ref{ss:Downloads}):
%
%    \begin{macrocode}
\RequirePackage{kvoptions}[2011/06/30]%      v3.11
%    \end{macrocode}
%
% The \xpackage{undolabl} package of \textsc{H.-Martin M\"{u}nch} (i.\,e. myself),
% with code from \textsc{Ulrich Diez}, (see subsection~\ref{ss:Downloads})
% is needed to overwrite labels, when the same page numbering scheme
% is used twice (or even more often).
%
%    \begin{macrocode}
\RequirePackage{undolabl}[2015/03/29]%       v1.0l
%    \end{macrocode}
%
% We use the \xpackage{rerunfilecheck} package by \textsc{Heiko Oberdiek}
% to make sure that the user gets and sees the rerun warnings (if any).
%
%    \begin{macrocode}
\RequirePackage{rerunfilecheck}[2011/04/15]% v1.7
%    \end{macrocode}
%
% We must not forget to give the source of |Prelim@EveryShipout|:
%
%    \begin{macrocode}
%% pageslts package uses Prelim@EveryShipout code from the
%% prelim2e package [2009/05/29 v1.3] by Martin Schr\"{o}der, thanks!

%    \end{macrocode}
%
% About the \xpackage{prelim2e} package by \textsc{Martin Schr\"{o}der}
% see subsection~\ref{ss:Downloads}.\\
%
% A last information for the user(s):
%
%    \begin{macrocode}
%% pageslts may work with earlier versions of LaTeX2e and those packages,
%% but this was not tested. Please consider updating your LaTeX
%% and packages to the most recent version (if they are not already
%% the most recent version).
%    \end{macrocode}
% See subsection~\ref{ss:Downloads} about how to get them.\\
%
% \pagebreak
%
% The very old version~2.0 (and earlier) of the \xpackage{endfloat}
% package actually redefined the |\enddocument| command,
% and so interfered drastically with the \LaTeXe\ commands which
% make use of |\AtEndDocument|. Newer versions of \xpackage{endfloat}
% exists (at the time of writing this documentation: v2.5d as of 2011/12/25)
% in modern documentation form, which are available from \url{https://www.CTAN.org}
% (see subsection~\ref{ss:Downloads}).
% A~note is placed here, and later it is checked whether a (very) old
% \xpackage{endfloat} package is in use. If it is, a warning or even
% an error message is given, depending on \xpackage{endfloat} version.
% This assumes, that the old versions of \xpackage{endfloat} at least gave
% a version date, of course.
%
%    \begin{macrocode}
%% The recent version of the endfloat package is v2.5d as of 2011/12/25.
%% The pageslts package is not fully compatible with version 2.0
%% (and earlier) of the endfloat package, because those versions
%% redefined the \enddocument command.

%    \end{macrocode}
%
% The options are introduced:
%
%    \begin{macrocode}
\SetupKeyvalOptions{family = pagesLTS,prefix = pagesLTS@}
\DeclareBoolOption[true]{pagecontinue}% \pagesLTS@pagecontinue
\DeclareStringOption[ab]{alphMult}
\DeclareStringOption[AB]{AlphMulti}
\DeclareBoolOption[true]{romanMult}
\DeclareBoolOption[true]{RomanMulti}
\DeclareBoolOption[true]{fnsymbolmult}

\ProcessKeyvalOptions*

%    \end{macrocode}
%
% For comparisons, zero, one, two and three are defined (|\z@|, |\@ne| and so on
% do not work for this).
%
%    \begin{macrocode}
\def\pagesLTS@zero{0}
\def\pagesLTS@one{1}
\def\pagesLTS@two{2}
\def\pagesLTS@three{3}

%    \end{macrocode}
%
% The traditional behaviour is a reset of the page number to one,
% each time the page numbering scheme changes. The option
% |pagecontinue| changes this to a continuation with the
% number/name following the last page number/name of the same
% page numbering scheme. The user is informed accordingly.
%
%    \begin{macrocode}
\ifpagesLTS@pagecontinue%
  \PackageInfo{pageslts}{Option pagecontinue enabled\MessageBreak%
    (maybe by default):\MessageBreak%
    The pageslts package will continue the page numbering,\MessageBreak%
    when the same page numbering scheme is used twice.\MessageBreak%
    If you do not want this, call pageslts with option\MessageBreak%
    pagecontinue=false (or use \string\setcounter{page}=1).\MessageBreak%
  }
\else
%    \end{macrocode}
%
% \pagebreak
%
%    \begin{macrocode}
  \PackageWarningNoLine{pageslts}{Option pagecontinue is false:\MessageBreak%
    The pageslts package was used, but the option\MessageBreak%
    pagecontinue was set to false.\MessageBreak%
    If you want the page numbers to be continued,\MessageBreak%
    when the same page numbering scheme is used twice,\MessageBreak%
    please call pageslts with option pagecontinue=true,\MessageBreak%
    otherwise the page number is reset to one each time\MessageBreak%
    the page numbering scheme is changed.\MessageBreak%
    For details please see the documentation%
  }
\fi

%    \end{macrocode}
%
% The page number printed in \texttt{alph} or in \texttt{Alph}
% page numbering scheme had to be $>0$ and \nolinebreak{$<27$.}\ %
% Now the \xpackage{alphalph} package allows to extend the
% numbering scheme (not only for pages). Because some users
% prefer aa, ab, ac, ad,\ldots\ and some aa, bb, cc, dd,\ldots,
% both schemes can be chosen via the options.
% The \texttt{fnsymbol} page numbering scheme was restricted to values
% $>0$ and \nolinebreak{$<10$.}\ The \xpackage{alphalph} package allows
% to extend this page numbering scheme, too. Option \texttt{fnsymbolmult}
% can be chosen with the \xpackage{pageslts} package.
% If no extension is wished (or another extension is wished
% and implemented manually), \xpackage{pageslts} can be called with
% options set to 0 (zero) and false: |alphMult=0|, |AlphMulti=0|,
% |fnsymbolmult=false|.
%
%    \begin{macrocode}
\def\pagesLTS@ab{ab}
\def\pagesLTS@bb{bb}
\def\pagesLTS@ABi{AB}
\def\pagesLTS@BBi{BB}

\ifx\pagesLTS@alphMult\pagesLTS@ab%
\else
  \ifx\pagesLTS@alphMult\pagesLTS@bb%
  \else
    \ifx\pagesLTS@alphMult\pagesLTS@zero%
      \PackageWarningNoLine{pageslts}{%
        Option alphMult=0 found:\MessageBreak%
        The pageslts package was used, but the option\MessageBreak%
        alphMult was set to 0 (zero).\MessageBreak%
        If you want the page numbers to be extended\MessageBreak%
        after z, you have to organize this yourself now.\MessageBreak%
        For automatic continuation, please use the\MessageBreak%
        alphalph package and call pageslts\MessageBreak%
        with option alphMult=ab (for aa, ab, ac, ad,...) or\MessageBreak%
        with option alphMult=bb (for aa, bb, cc, dd,...).\MessageBreak%
        For details please see the documentation!%
       }
    \else
      \PackageError{pageslts}{Unknown option value}{%
        The pageslts package was used with option\MessageBreak%
        alphMult= \pagesLTS@alphMult . Only values\MessageBreak%
        ab, bb, and 0 (zero) are valid.\MessageBreak%
        The default ab is set.\MessageBreak%
        For details please see the documentation!%
       }
      \setkeys{pagesLTS}{alphMult=ab}
    \fi
  \fi
\fi

\ifx\pagesLTS@AlphMulti\pagesLTS@ABi%
\else
  \ifx\pagesLTS@AlphMulti\pagesLTS@BBi%
  \else
    \ifx\pagesLTS@AlphMulti\pagesLTS@zero%
      \PackageWarningNoLine{pageslts}{%
        Option AlphMulti=0 found:\MessageBreak%
        The pageslts package was used, but the option\MessageBreak%
        AlphMulti was set to 0 (zero).\MessageBreak%
        If you want the page numbers to be extended\MessageBreak%
        after Z, you have to organize this yourself now.\MessageBreak%
        For automatic continuation, please use the\MessageBreak%
        alphalph package and call pageslts\MessageBreak%
        with option AlphMulti=AB (for AA, AB, AC, AD,...) or\MessageBreak%
        with option AlphMulti=BB (for AA, BB, CC, DD,...).\MessageBreak%
        For details please see the documentation!%
       }
    \else
      \PackageError{pageslts}{Unknown option value}{%
        The pageslts package was used with option\MessageBreak%
        AlphMulti= \pagesLTS@AlphMulti . Only values\MessageBreak%
        AB, BB, and 0 (zero) are valid.\MessageBreak%
        The default AB is set.\MessageBreak%
        For details please see the documentation!%
       }
      \setkeys{pagesLTS}{AlphMulti=AB}
    \fi
  \fi
\fi

%    \end{macrocode}
%
% If |alph| or |Alph| or |fnsymbol| page numbers shall be continued,
% the \xpackage{alphalph} package is required.
%
%    \begin{macrocode}
\ifx\pagesLTS@alphMult\pagesLTS@zero% \relax
\else \RequirePackage{alphalph}[2011/05/13]% v2.4
\fi

\ifx\pagesLTS@AlphMulti\pagesLTS@zero% \relax
\else \RequirePackage{alphalph}[2011/05/13]% v2.4
\fi

\ifpagesLTS@fnsymbolmult%
  \RequirePackage{alphalph}[2011/05/13]% v2.4
\fi

%    \end{macrocode}
%
% For the roman page numbering scheme, it is just the
% choice of an extension by \xpackage{pageslts} or not.
%
%    \begin{macrocode}
\ifpagesLTS@romanMult%
  \PackageInfo{pageslts}{Option romanMult enabled\MessageBreak%
    (maybe by default):\MessageBreak%
    The pageslts package will extend the page numbering\MessageBreak%
    of the roman scheme below i with\MessageBreak%
     0, -i, -ii, -iii, -iv,...\MessageBreak%
    If you do not want this, call pageslts with option\MessageBreak%
    romanMult=false.\MessageBreak%
  }
\else
  \PackageWarningNoLine{pageslts}{Option romanMult is set to false:\MessageBreak%
  The pageslts package was used, but the option\MessageBreak%
  romanMult was set to false.\MessageBreak%
  If you want the page numbering of the roman scheme\MessageBreak%
  to be extended below i,\MessageBreak%
  please call pageslts with option romanMult=true,\MessageBreak%
  otherwise zero and negative page numbers of the\MessageBreak%
  roman scheme will need to be defined otherwise.\MessageBreak%
  For details please see the documentation!\MessageBreak%
  }
\fi

%    \end{macrocode}
%
% Same for the Roman page numbering scheme.
%
%    \begin{macrocode}
\ifpagesLTS@RomanMulti%
  \PackageInfo{pageslts}{Option RomanMulti enabled\MessageBreak%
    (maybe by default):\MessageBreak%
    The pageslts package will extend the page numbering\MessageBreak%
    of the Roman scheme below I with\MessageBreak%
    0, -I, -II, -III, -IV,...\MessageBreak%
    If you do not want this, call pageslts with option\MessageBreak%
    RomanMulti=false.\MessageBreak%
   }
\else
  \PackageWarningNoLine{pageslts}{Option RomanMulti is set to false:\MessageBreak%
    The pageslts package was used, but the option\MessageBreak%
    RomanMulti was set to false.\MessageBreak%
    If you want the page numbering of the Roman scheme\MessageBreak%
    to be extended below i,\MessageBreak%
    please call pageslts with option RomanMulti=true,\MessageBreak%
    otherwise zero and negative page numbers of the\MessageBreak%
    Roman scheme will need to be defined otherwise.\MessageBreak%
    For details please see the documentation!\MessageBreak%
   }
\fi

%    \end{macrocode}
%
% For the footnotesymbol page numbering scheme, it is also
% just the choice of a extension by \xpackage{pageslts} or not.
%
%    \begin{macrocode}
\ifpagesLTS@fnsymbolmult%
  \PackageInfo{pageslts}{Option fnsymbolmult enabled\MessageBreak%
    (maybe by default):\MessageBreak%
    The pageslts package will extend the page numbering\MessageBreak%
    of the footnotesymbol scheme using the alphalph\MessageBreak%
    package.\MessageBreak%
    If you do not want this, call pageslts with option\MessageBreak%
    fnsymbolmult=false.\MessageBreak%
   }
\else
  \PackageWarningNoLine{pageslts}{%
    Option fnsymbolmult is set to false:\MessageBreak%
    The pageslts package was used, but the option\MessageBreak%
    fnsymbolmult was set to false.\MessageBreak%
    If you want the page numbering of the footnotesymbol\MessageBreak%
    scheme to be extended using the alphalph package,\MessageBreak%
    please call pageslts with option fnsymbolmult=true,\MessageBreak%
    otherwise page numbers of the footnotesymbol scheme\MessageBreak%
    greater than nine will need to be defined otherwise.\MessageBreak%
    For details please see the documentation!\MessageBreak%
   }
\fi

%    \end{macrocode}
%
% Now defining some variables, place-holders, and abbreviations:
%
%    \begin{macrocode}
\def\pagesLTS@pnc{0}
\def\pagesLTS@called{0}
\def\pagesLTS@fns{fnsymbol}
\def\pagesLTS@alph{alph}
\def\pagesLTS@Alph{Alph}
\def\pagesLTS@rerun{0}
\def\pagesLTS@eso{0}
\def\pagesLTS@esov{0}
\def\lastpageref{\lastpagereftxt}
\def\pagesLTS@undolable{none}
\def\pncmissing{0}
\def\pagesLTS@SK{0}
\def\pagesLTS@messageNPN{%
 The pageslts package was used, but\MessageBreak%
 \string\pagenumbering \MessageBreak%
 was not called at the beginning of the document\MessageBreak%
 (maybe earlier or later).\MessageBreak%
 Please use \string\pagenumbering \MessageBreak%
 (with an argument like arabic, roman, Roman,\MessageBreak%
 fnsymbol, alph, or Alph) at the beginning\MessageBreak%
 of your document! Otherwise your document\MessageBreak%
 will probably compile, but the pageslts\MessageBreak%
 package might not be able to get all labels\MessageBreak%
 for the references to the respective pages\MessageBreak%
 right.\MessageBreak%
 }

%    \end{macrocode}
%
% It is checked whether writing to files is allowed.
% The \xpackage{pageslts} package cannot be used without that!
% Some packages (e.\,g. \xpackage{tikz} and \xpackage{selectp}) sometimes prevent
% the output to the \xfile{aux} file. In that case a warning or an error message
% is issued. This is no problem as long as there is/was another compilation run
% where the labels can/could be processed via the \xfile{aux} file.\\
% If it is allowed to write to the \xfile{aux} file, we define
% |\pagesLTS@loaded| as |p@gesLTSnotlo@ded| for the next compilation run.
%
%    \begin{macrocode}
\AtBeginDocument{%
  \if@filesw%
%    \end{macrocode}
%
% \noindent then |\relax|.\newline
% When writing to files is not allowed, nothing can be done.
% But when the labels were already processed via the \xfile{aux} file,
% nothing needs to be done (if enough compilation runs have been done before).
%
%    \begin{macrocode}
  \else%
    \@ifpackageloaded{tikz}{%
      \PackageWarning{pageslts}%
        {It was not allowed to write to an\MessageBreak%
         .aux file. This package does not work without access\MessageBreak%
         to an .aux file.\MessageBreak%
         It is OK if the .aux file was already updated\MessageBreak%
         by a previouse compiler run\MessageBreak%
         and would not have changed anyway.\MessageBreak%
        }%
    }{\PackageError{pageslts}{No writing to auxiliary file allowed}%
        {It was not allowed to write to an .aux file.\MessageBreak%
         This package does not work without access to an .aux file.\MessageBreak%
         Press Ctrl+Z to exit.\MessageBreak%
         But it is OK if the .aux file was already updated\MessageBreak%
         by previouse compiler runs\MessageBreak%
         and would not have changed anyway.\MessageBreak%
         (In that case just press Enter or Return\MessageBreak%
         to continue the compilation.)\MessageBreak%
        }%
     }%
  \fi%
  }

%    \end{macrocode}
%
% \begin{macro}{\pagenumbering}
% \indent \qquad To keep the original meaning of |\pagenumbering|:
%
%    \begin{macrocode}
\LetLtxMacro{\OrigPagenumbering}{\pagenumbering}

%    \end{macrocode}
% \end{macro}
%
% Defining some new counters (and doing related things):
%
%    \begin{macrocode}
\newcounter{CurrentPage}
\setcounter{CurrentPage}{1}
\def\theCurrentPageLocal{\arabic{pagesLTS.current.local.\pagesLTS@pnc}}
%    \end{macrocode}
%
% The counter |pagesLTS.pagenr| is for saving the total page
% number of the last page in the \xfile{.aux} file.
%
%    \begin{macrocode}
\newcounter{pagesLTS.pagenr}
%    \end{macrocode}
%
% While generally
% |\pagesLTS@ifcounter{pagesLTS.current.local.\pagesLTS@pnc}|
% is used, for the beginning of the document
% |pagesLTS.current.local.0| is predefined.
% (A~|\pagesLTS@ifcounter{pagesLTS.current.local.\pagesLTS@pnc}|
% could be used for this, too, but we know that |pagesLTS.current.local.0|
% was not defined, so we can just do the definition here.)
% And the first local page gets the number one.
%
%    \begin{macrocode}
\newcounter{pagesLTS.current.local.0}
\setcounter{pagesLTS.current.local.0}{1}
%    \end{macrocode}
%
% And the same again for |pagesLTS.pnc.0|.
%
%    \begin{macrocode}
\newcounter{pagesLTS.pnc.0}

%    \end{macrocode}
%
% \phantomsection
% \begin{macro}{\xroman}\label{xroman}
% When |\roman{...}| is used with a value $<1$, \LaTeX\ just ignores this
% (see subsection~\ref{ss:overflow}). Here we provide a command
% |\xroman{...}| \nolinebreak{(e\textbf{x}panded} roman),
% which gives the usual |\roman| numbers (i, ii, iii, iv,\ldots)
% for positive values,
% $-\left\vert \ldots \right\vert $ (i.\,e. -i, -ii, -iii, -iv,\ldots)
% for negative values, and 0 for all other values (which should be zero).
%
%    \begin{macrocode}
\newcommand{\xroman}[1]{%
  \ifnum\value{#1}>0%
    \roman{#1}%
  \else%
    \ifnum\value{#1}<0%
%    \end{macrocode}
%
% |\arabic{#1}| gives the arabic number of argument |#1|,
% which is negative here (for example $-7$),
% \textquotedblleft |-|\textquotedblright{} puts another minus
% sign in front of it (for example $--7$), |\number| removes
% all unnecessary preceding zeros, plus and minus signs
% (for example $7$), |\romannumeral| turns it into a |roman|
% number (for example vii), and
% \textquotedblleft |-|\textquotedblright{} puts the minus
% sign back in front of it (for example -vii).
%
%    \begin{macrocode}
      -\romannumeral\number-\arabic{#1}%
    \else%
      0%
    \fi%
  \fi%
  }

%    \end{macrocode}
%  \end{macro}
%
% \pagebreak
%
% \begin{macro}{\XRoman}
% |\XRoman| does the same for uppercase |\Roman| numbers.
% |-\uppercase{\romannumeral\number-\arabic{#1}}| cannot be used,
% because the result in the example is |-\uppercase{vii}| and
% not |-VII|.\footnote{This does not matter for the print out,
% but for the display of the logical page numbers
% as well as the \xfile{.aux} file.} Therefore we have a look
% at \LaTeX's own |\@Roman\FOOcounter|,\\
% |\def\@Roman#1{\expandafter\@slowromancap\romannumeral #1@}|,\\
% and use |\@slowromancap|, which is a fully expandable macro,
% to do the trick for this:
%
% \noindent \textquotedblleft
% \begin{quote}
%\begin{verbatim}
%\def\@slowromancap#1{\ifx @#1% then terminate
%\else
%\if i#1I\else\if v#1V\else\if x#1X\else\if l#1L\else\if
%c#1C\else\if d#1D\else \if m#1M\else#1\fi\fi\fi\fi\fi\fi\fi
%\expandafter\@slowromancap
%\fi
%}
%\end{verbatim}
% \end{quote}
% \textquotedblright{} (1998/05/16 Version v1.1g LaTeX Kernel File~m
% ltcounts.dtx 105 Counters and Lengths).
%
%    \begin{macrocode}
\newcommand{\XRoman}[1]{%
  \ifnum\value{#1}>0%
    \Roman{#1}%
  \else%
    \ifnum\value{#1}<0%
      -\expandafter\@slowromancap\romannumeral\number-\arabic{#1}@%
    \else%
      0%
    \fi%
  \fi%
  }

%    \end{macrocode}
%  \end{macro}
%
% \begin{macro}{\XXRoman}
% In older versions |\XXRoman| was used. For compatibility, it is forwarded to |\XRoman|
% and an error message is given.
%
%    \begin{macrocode}
\newcommand{\XXRoman}[1]{\XRoman{#1}%
  \PackageError{pageslts}{Old command \string\XXRoman\space found}{Replaced by \string\XRoman.}%
  }

%    \end{macrocode}
%  \end{macro}
%
% \begin{macro}{\pagesLTS@ifcounter}
% We provide a way to create counters like
% \begin{itemize}
% \item[-] \DescribeMacro{pagesLTS.pnc. page numbering scheme }
%   |pagesLTS.pnc.<page numbering scheme>|, e.\,g. |pagesLTS.pnc.Roman|,
% \item[-] \DescribeMacro{pagesLTS.double. page numbering scheme }
%   |pagesLTS.double.<page numbering scheme>|, e.\,g. |pagesLTS.double.Roman|,
% \item[-] \DescribeMacro{PageCurrentLocal. page numbering scheme }
%   |PageCurrentLocal.<page numbering scheme>|, e.\,g. |PageCurrentLocal.Roman|,
% \end{itemize}
% for all page numbering schemes, even those not supported
% by the current original |\pagenumbering|
% (1994/05/19 v1.1a LaTeX Kernel File w ltpageno.dtx 52 Page Numbering),
% which is\\
% \begin{quote}
%\begin{verbatim}
%\countdef\c@page=0 \c@page=1
%\def\cl@page{}
%\def\pagenumbering#1{%
%  \global\c@page \@ne \gdef\thepage{\csname @#1\endcsname
%   \c@page}}
%\end{verbatim}
% \end{quote}
%
%    \begin{macrocode}
\newcommand{\pagesLTS@ifcounter}[1]{%
  \@ifundefined{c@#1}{\newcounter{#1}}{\relax}%
  }

%    \end{macrocode}
%  \end{macro}
%
% \begin{macro}{\lastpages}
% We provide a command to give the number of pages in a sector
% of a split page numbering scheme (see page~\pageref{pagesLTS.pnscheme.nr},\\
% |pagesLTS.<page numbering scheme>.<number>.local.cnt|):
%
%    \begin{macrocode}
\newcommand{\lastpages}[2]{%
  \pagesLTS@ifcounter{pagesLTS.#1.#2.local.cnt}%
  \arabic{pagesLTS.#1.#2.local.cnt}%
  }

%    \end{macrocode}
%  \end{macro}
%
% \begin{macro}{\pagesLTS@writelabel}
% At last defining the writing of a label:
%
%    \begin{macrocode}
\newcommand{\pagesLTS@writelabel}[1]{%
  \addtocounter{page}{+1}%
%    \end{macrocode}
%
% \noindent |\addtocounter{page}{+1}| because |\pagesLTS@putlabel| includes an\\
% |\addtocounter{page}{-1}|, which is not necessary here.\\
% Into the \xfile{.aux} file something like\\
% |\newlabel{pagesLTS.Roman}{{}{VIII}{}{page.VIII}{}}|\\
% is written, thus |\lastpageref{pagesLTS.Roman}| prints |VIII| and
% links to |page.VIII|.
%
%    \begin{macrocode}
  \pagesLTS@putlabel{pagesLTS.#1}{\thepage}{1}
  \addtocounter{page}{-1}%
  \ifx\pagesLTS@pnc\pagesLTS@zero% \relax
%    \end{macrocode}
%
% i.\,e. if the current page numbering scheme is
% \textquotedblleft 0\textquotedblright, i.\,e. before the first
% |\pagenumbering{...}| command, do nothing,
%
%    \begin{macrocode}
  \else%
    \addtocounter{page}{+1}%
    \pagesLTS@putlabel{pagesLTS.#1.local}{\theCurrentPageLocal}{1}
%    \end{macrocode}
%
% otherwise write into the \xfile{.aux} file something like\\
% |\newlabel{pagesLTS.arabic.local}{{}{5}{}{page.8}{}}|,\\
% thus |\lastpageref{pagesLTS.arabic.local}| prints |5| and
% links to |page.8|. Here (and in the example file) it is not
% \textquotedblleft print |8| and link to |page.8|\textquotedblright ,
% because |\addtocounter{page}{3}| has been used, thus the page with
% \textquotedblleft number\textquotedblright{} (name) |8| is the
% \textbf{fifth} $(=8-3)$ page.
%
%    \begin{macrocode}
    \addtocounter{page}{-1}%
  \fi%
  }

%    \end{macrocode}
% \end{macro}
%
% \begin{macro}{\erroralphalph}
% \indent |\erroralphalph| extends the
% \textquotedblleft numbers\textquotedblright{} of counters
% to zero and negative values for representations usually
% not supporting this: |\alphalph|, |\AlphAlph|, and
% |\fnsymbolmult| of the \xpackage{alphalph} package.\\
% |\alph|, |\Alph|, and |\fnsymbol| would not support
% \textquotedblleft numbers\textquotedblright{} below one.\\
% |\arabic| already supports negative numbers and zero.\\
% |\roman| and |\Roman| support neither negative numbers
% nor zero, but are expanded in this package
% (|\xroman| and |\XRoman|), see page~\pageref{xroman}.\\
%
%    \begin{macrocode}

%% The following code is from Heiko Oberdiek [2010/04/18],              %%
%% expanding his alphalph package as of 2010/04/18, v2.3. (Thanks!)     %%
\newcommand*{\erroralphalph}[2]{%
  \ifnum\value{#2}>0%
    #1{\value{#2}}%
  \else%
    \ifnum\value{#2}<0%
      -#1{\expandafter\@gobble\the\value{#2}}%
    \else%
      0%
    \fi%
  \fi%
  }
%% End of code from Heiko Oberdiek                                      %%
%% Check and Error/Warning messages have been moved to \EveryShipout,   %%
%% because messages inside e.g. the \pageref command can cause trouble. %%

%    \end{macrocode}
% \end{macro}
%
% \begin{macro}{\expandPagenumbering}
% \indent Here the |\erroralphalph| command is called with the
% appropriate arguments for each page numbering scheme.
%
%    \begin{macrocode}

\newcommand{\expandPagenumbering}[1]{%
  \let\Origthepage\thepage%
  \def\pagesLTS@tmpC{arabic}%
  \ifx\pagesLTS@pnc\pagesLTS@tmpC% \relax
%    \end{macrocode}
%
% |\arabic| already supports negative numbers and zero
% (|-MAX...MAX|, where |MAX|\,$ = 2\,147\,483\,647$).
%
%    \begin{macrocode}
  \else%
    \def\pagesLTS@tmpC{roman}%
    \ifx\pagesLTS@pnc\pagesLTS@tmpC%
      \ifpagesLTS@romanMult%
%    \end{macrocode}
%
% |\erroralphalph{\roman}{page}| cannot be used, because
% |-\roman{\expandafter\@gobble\the\value{page}}| does not work.
% If option |romanMult| is not false,
% |\xroman| (see page~\pageref{xroman}) expands the usable
% roman page numbers to values below 1 (i, I, respectively),
% see subsubsection~\ref{sss:romanMult}.
%
%    \begin{macrocode}
        \renewcommand*{\thepage}{\xroman{page}}%
      \fi%
    \else%
      \def\pagesLTS@tmpC{Roman}%
      \ifx\pagesLTS@pnc\pagesLTS@tmpC%
        \ifpagesLTS@RomanMulti%
%    \end{macrocode}
%
% The same for |\Roman| page numbering, expanded by |\XRoman|
% (see page~\pageref{xroman}).
%
%    \begin{macrocode}
          \renewcommand*{\thepage}{\XRoman{page}}%
        \fi%
      \else%
        \ifx\pagesLTS@pnc\pagesLTS@alph%
%    \end{macrocode}
%
% |\alph| and |\Alph| page numberings are expanded to negative
% and zero values, and to values greater than
% \textquotedblleft z\textquotedblright{} or
% \textquotedblleft Z\textquotedblright{} with the
% \xpackage{alphalph} package.~--
% If |\pagesLTS@alphMult| was zero, nothing is done.
%
%    \begin{macrocode}
          \ifx\pagesLTS@alphMult\pagesLTS@ab%
            \renewcommand*{\thepage}{\erroralphalph{\alphalph}{page}}%
          \else \ifx\pagesLTS@alphMult\pagesLTS@bb%
                  \renewcommand*{\thepage}{\erroralphalph{\alphMult}{page}}%
                \fi%
          \fi%
        \else%
          \ifx\pagesLTS@pnc\pagesLTS@Alph%
            \ifx\pagesLTS@AlphMulti\pagesLTS@ABi%
              \renewcommand*{\thepage}{\erroralphalph{\AlphAlph}{page}}%
            \else \ifx\pagesLTS@AlphMulti\pagesLTS@BBi%
                    \renewcommand*{\thepage}{\erroralphalph{\AlphMult}{page}}%
                  \fi%
            \fi%
          \else%
            \ifx\pagesLTS@pnc\pagesLTS@fns%
%    \end{macrocode}
%
% Same for |\fnsymbol| page numbers.
%
%    \begin{macrocode}
              \ifpagesLTS@fnsymbolmult%
                \renewcommand*{\thepage}{\erroralphalph{\fnsymbolmult}{page}}%
              \fi%
            \else%
%    \end{macrocode}
%
% If the used page numbering scheme has not been recognized by the
% \xpackage{pageslts} package so far, we can do nothing, and problems
% might result.\\
%
%    \begin{macrocode}
              \PackageError{pageslts}{unknown page numbering scheme}{%
                The pageslts package encountered the unknown\MessageBreak%
                page numbering scheme\MessageBreak%
                `#1'. \MessageBreak%
                If this is no typing mistake, it might work\MessageBreak%
                - or it might not work.\MessageBreak%
                \@ehc%
                }
            \fi%
          \fi%
        \fi%
      \fi%
    \fi%
  \fi%
  \let\pagesLTS@tmpC\undefined%
  }

%    \end{macrocode}
% \end{macro}
%
% \begin{macro}{\pagenumbering}
% \indent Now for the \textbf{new} version of the |\pagenumbering| command:
%
%    \begin{macrocode}
\renewcommand{\pagenumbering}[1]{%
%    \end{macrocode}
%
% If the current page numbering scheme, |\pagesLTS@pnc|, or the
% requested page numbering scheme, |#1|, is |\pagesLTS@fns|,
% i.\,e. \texttt{fnsymbol}, the counter\texttt{pagesLTS.fnsymbol.local}
% is needed. If it does not exists yet, it is created here.
%
%    \begin{macrocode}
  \edef\pagesLTS@tmpA{#1}%
  \ifx\pagesLTS@pnc\pagesLTS@fns%
    \pagesLTS@ifcounter{pagesLTS.fnsymbol.local}%
  \fi%
  \ifx\pagesLTS@tmpA\pagesLTS@fns%
    \pagesLTS@ifcounter{pagesLTS.fnsymbol.local}%
  \fi%
%    \end{macrocode}
%
% If the current page numbering scheme, |\pagesLTS@pnc|, and the
% requested page numbering scheme, |#1|, is the same one, nothing
% further is done, otherwise the real action begins.
%
%    \begin{macrocode}
  \ifx\pagesLTS@pnc\pagesLTS@tmpA% \relax
  \else%
%    \end{macrocode}
%
% The next code is executed, when we are at a page after the first one.
% This distinction is done for two reasons:\\
% On the one hand, |\pagenumbering| could be called \textit{before}
% |\begin{document}| (where the current page should not be
% greater than one), and on the other hand we go one page back to
% aim all references to that page. Obviously at the first page
% there is no going backward.
%
%    \begin{macrocode}
    \ifnum \value{CurrentPage}>1%
      \addtocounter{page}{-1}%
      \addtocounter{pagesLTS.current.local.\pagesLTS@pnc}{-1}%
%    \end{macrocode}
%
% For the case that the page numbering scheme is or will be split,
% like e.\,g. the Roman one in the |pageslts-example.tex|, a counter
% like
% |pagesLTS.Roman.1.local.count|
% (or |pagesLTS.Roman.2.local.count|,
% |pagesLTS.Roman.3.local.count|,\ldots ) is introduced and set to the
% number of the local page.
%
%    \begin{macrocode}
      \newcounter{pagesLTS.\pagesLTS@pnc.\arabic{pagesLTS.pnc.\pagesLTS@pnc}.local.count}%
      \setcounter{pagesLTS.\pagesLTS@pnc.\arabic{pagesLTS.pnc.\pagesLTS@pnc}.local.count}{%
        \value{pagesLTS.current.local.\pagesLTS@pnc}}%
%    \end{macrocode}
%
% If the page numbering scheme is \xpackage{fnsymbol},
% and if it \textit{was} used before,
% from said counter the number of pages of the preceding uses
% of the same page numbering scheme, |pagesLTS.\pagesLTS@pnc.done|,
% is subtracted (same as for the other schemes, see below). Instead of introducing a new
% counter (which can be problematic, when the number of available counters is limited),
% we borrow the \texttt{pagesLTS.pnc.0} counter, i.\,e. we save its value to |\pagesLTS@tmpa|,
% (ab)use the counter, and then set it back to its former value as saved in |\pagesLTS@tmpa|.
%
%    \begin{macrocode}
      \ifx\pagesLTS@pnc\pagesLTS@fns%
        \ifnum \value{pagesLTS.pnc.\pagesLTS@pnc}>1%
          \mathchardef\pagesLTS@tmpa=\arabic{pagesLTS.pnc.0}%
          \setcounter{pagesLTS.pnc.0}{\value{pagesLTS.pnc.\pagesLTS@pnc}}%
          \addtocounter{pagesLTS.pnc.0}{-1}%
          \addtocounter{pagesLTS.\pagesLTS@pnc.\arabic{pagesLTS.pnc.\pagesLTS@pnc}.local.count}{%
            -\value{pagesLTS.\pagesLTS@pnc.\arabic{pagesLTS.pnc.0}.local.count}}%
          \setcounter{pagesLTS.pnc.0}{\pagesLTS@tmpa}%
        \fi%
%    \end{macrocode}
%
% If the page numbering scheme is \textit{not} \xpackage{fnsymbol},
% a~numbered label is written:
%
%    \begin{macrocode}
      \else%
        \pagesLTS@writelabel{\pagesLTS@pnc.\arabic{pagesLTS.pnc.\pagesLTS@pnc}}%
%    \end{macrocode}
%
% If the page numbering scheme was \textit{not} used before,
%
%    \begin{macrocode}
        \ifnum \value{pagesLTS.pnc.\pagesLTS@pnc}<2%
%    \end{macrocode}
%
% an unnumbered label is also written:
%
%    \begin{macrocode}
          \pagesLTS@writelabel{\pagesLTS@pnc}%
%    \end{macrocode}
%
% If the page numbering scheme \textit{was} used before,
% from said counter the number of pages of the preceding uses
% of the same page numbering scheme, |pagesLTS.\pagesLTS@pnc.done|,
% is subtracted. Instead of introducing a new
% counter (which can be problematic, when the number of available counters is limited),
% we again borrow the \texttt{pagesLTS.pnc.0} counter (see above).
%
%    \begin{macrocode}
        \else%
          \mathchardef\pagesLTS@tmpa=\arabic{pagesLTS.pnc.0}%
          \setcounter{pagesLTS.pnc.0}{\value{pagesLTS.pnc.\pagesLTS@pnc}}%
          \addtocounter{pagesLTS.pnc.0}{-1}%
          \pagesLTS@ifcounter{pagesLTS.\pagesLTS@pnc.done}%
          \addtocounter{pagesLTS.\pagesLTS@pnc.done}{%
            \value{pagesLTS.\pagesLTS@pnc.\arabic{pagesLTS.pnc.0}.local.count}}%
          \addtocounter{pagesLTS.\pagesLTS@pnc.\arabic{pagesLTS.pnc.\pagesLTS@pnc}.local.count}{%
            -\value{pagesLTS.\pagesLTS@pnc.done}}%
          \setcounter{pagesLTS.pnc.0}{\pagesLTS@tmpa}%
        \fi%
%    \end{macrocode}
%
% The values are written to the \xfile{.aux} file
% (if writing is allowed: |\if@filesw|),
% because they must be available at the beginning of the document:
%
%    \begin{macrocode}
        \if@filesw%
          \immediate\write\@auxout{\string
            \pagesLTS@ifcounter{pagesLTS.\pagesLTS@pnc.\arabic{pagesLTS.pnc.\pagesLTS@pnc}.local.cnt}}%
        \fi%
        \edef\pagesLTS@tmpB{\arabic{pagesLTS.\pagesLTS@pnc.\arabic{pagesLTS.pnc.\pagesLTS@pnc}.local.count}}%
        \if@filesw%
          \immediate\write\@auxout{\string
            \setcounter{pagesLTS.\pagesLTS@pnc.\arabic{pagesLTS.pnc.\pagesLTS@pnc}.local.cnt}{\pagesLTS@tmpB}}%
        \fi%
      \fi%
%    \end{macrocode}
%
% For further code for the case of \xpackage{fnsymbol} please see below
% (|\lastpagereftext|, page~\pageref{lastpagereftextlabel}).\\
% The last page number is saved, in case the same page
% numbering scheme is continued later.
%
%    \begin{macrocode}
      \pagesLTS@ifcounter{pagesLTS.double.\pagesLTS@pnc}%
      \setcounter{pagesLTS.double.\pagesLTS@pnc}{\value{page}}%
%    \end{macrocode}
%
% We went back one page, so we must go forward again:
%
%    \begin{macrocode}
      \addtocounter{pagesLTS.current.local.\pagesLTS@pnc}{+1}%
      \addtocounter{page}{+1}%
%    \end{macrocode}
%
% The page numbering scheme |\pagesLTS@pnc| is now set to
% the new one, given by the user as argument with the
% |\pagenumbering{...}| command:
%
%    \begin{macrocode}
      \xdef\pagesLTS@pnc{#1}%
%    \end{macrocode}
%
% The new page numbering scheme is now started for real:
%
%    \begin{macrocode}
      \OrigPagenumbering{#1}%
%    \end{macrocode}
%
% If a page numbering scheme not known by the original
% |\pagenumbering{...}| command is used, an error will
% arise here - but maybe without error message.\\
%
% If page numbering schem |\alph|, |\Alph|, or |\fnsymbol| is used,
% \xpackage{pageslts} extends the page numbers according to
% the given options, using the \xpackage{alphalph} package.
% |\arabic| does not need any expansion.
% |\roman| and |\Roman| at least receive a definition for
% zero.
%
%    \begin{macrocode}
      \expandPagenumbering{#1}%
%    \end{macrocode}
%
% Counters like |pagesLTS.pnc.Roman| are introduced:
%
%    \begin{macrocode}
      \pagesLTS@ifcounter{pagesLTS.pnc.\pagesLTS@pnc}%
%    \end{macrocode}
%
% The saved number of times this page numbering scheme was used
% is increased by one:
%
%    \begin{macrocode}
      \addtocounter{pagesLTS.pnc.\pagesLTS@pnc}{1}%
%    \end{macrocode}
%
% \indent Now defining the counter |pagesLTS.double.\pagesLTS@pnc|,
% if it did not exist already, adding~1, because this is
% the first page of it (or another one, if the scheme is
% continued):
%
%    \begin{macrocode}
      \pagesLTS@ifcounter{pagesLTS.double.\pagesLTS@pnc}%
      \addtocounter{pagesLTS.double.\pagesLTS@pnc}{1}%
%    \end{macrocode}
%
% The page number is continued, if the option |pagecontinue=false|
% is \textbf{not} set, otherwise it is reset to one. Note that neither
% the local nor the current counter are reset, as they contain
% the real \textit{values} and not the \textit{names} of the pages.
%
%    \begin{macrocode}
      \ifpagesLTS@pagecontinue%
        \setcounter{page}{\value{pagesLTS.double.\pagesLTS@pnc}}%
      \else%
        \setcounter{page}{1}%
      \fi%
%    \end{macrocode}
%
% If it does not exist already, the counter
% |pagesLTS.current.local.\pagesLTS@pnc|
% (e.\,g. |pagesLTS.current.local.Roman|) is created.
%
%    \begin{macrocode}
      \pagesLTS@ifcounter{pagesLTS.current.local.\pagesLTS@pnc}%
%    \end{macrocode}
%
% If |pagesLTS.double.\pagesLTS@pnc| of the current page numbering
% scheme is equal to one, this is the first page of this page
% numbering scheme. Then |pagesLTS.current.local.\pagesLTS@pnc|
% (which was zero) is set to one.
%
%    \begin{macrocode}
      \ifnum \value{pagesLTS.double.\pagesLTS@pnc}=1%
        \setcounter{pagesLTS.current.local.\pagesLTS@pnc}{1}%
      \fi%
%    \end{macrocode}
%
% Otherwise, i.\,e. if |\value{CurrentPage}| is not |>1|,
% i.\,e. before the first page has shiped out:
%
%    \begin{macrocode}
    \else%
      %% before the first page has shipped out
%    \end{macrocode}
%
% The current page numbering scheme is defined by the
% argument of |\pagenumbering{...}|, which the user gave:
%
%    \begin{macrocode}
      \xdef\pagesLTS@pnc{#1}%
%    \end{macrocode}
%
% \noindent and the page numbering scheme set by the original
% page numbering command
% (1994/05/19 v1.1a LaTeX Kernel File w ltpageno.dtx 52 Page Numbering),
% which resets the page number to one,
% but at the first page continuation does not make sense).
% Well, nearly the original page numbering command:
% |\OrigPagenumbering{\pagesLTS@pnc}| does not work,
% so we \textquotedblleft expand\textquotedblright{} the
% |\OrigPagenumbering| command:
%
%    \begin{macrocode}
      \global\c@page \@ne\relax%
      \global\def\thepage{\csname \expandafter @\pagesLTS@pnc \endcsname \c@page}%
%    \end{macrocode}
%
% If a page numbering scheme is used, which is not known by
% \LaTeX , an error might arise here -- but maybe without error message.\\
%
% If page numbering scheme |\alph|, |\Alph|, or |\fnsymbol| is used,
% \xpackage{pageslts} extends the page numbers according to
% the given options, using the \xpackage{alphalph} package.
% |\arabic| does not need any expansion.
% |\roman| and |\Roman| at least receive a definition for
% zero.
%
%    \begin{macrocode}
      \expandPagenumbering{#1}
%    \end{macrocode}
%
% We are at the first page, so the page counters are set to one:
%
%    \begin{macrocode}
      \pagesLTS@ifcounter{pagesLTS.pnc.\pagesLTS@pnc}%
      \setcounter{pagesLTS.pnc.\pagesLTS@pnc}{1}%
      \pagesLTS@ifcounter{pagesLTS.double.\pagesLTS@pnc}%
      \setcounter{pagesLTS.double.\pagesLTS@pnc}{1}%
      \pagesLTS@ifcounter{pagesLTS.current.local.\pagesLTS@pnc}%
      \setcounter{pagesLTS.current.local.\pagesLTS@pnc}{1}%
    \fi%
%    \end{macrocode}
%
% Whether |\pagenumbering{...}| is called in the preamble,
% |\AtBeginDocument|, right after |\begin{document}|,
% or somewhere in the document, we want to remember
% \textit{whether} it was called at all:
%
%    \begin{macrocode}
    \gdef\pagesLTS@called{1}%
  \fi%
%    \end{macrocode}
%
% We do not need the temporary definitions any more.
%
%    \begin{macrocode}
  \let\pagesLTS@tmpA\undefined%
  \let\pagesLTS@tmpB\undefined%
  }

%    \end{macrocode}
% \end{macro}
%
% \begin{macro}{\lastpageref*}
% If \xpackage{hyperref} is used, but (some) references to some last page shall
% not be hyperlinked, a command |\lastpageref*| (analogous to |\pageref*|) is
% needed. Therefore we define (analogous to |\HyPsd@pageref| from
% the \xpackage{hyperref} package by \textsc{Heiko Oberdiek})
%
%    \begin{macrocode}
%% analogous to \HyPsd@pageref from the hyperref package by Heiko Oberdiek:
\def\lastpagereftxt#1{\pagesLTS@@pageref#1*\END}

%    \end{macrocode}
%
% Macro |\pagesLTSpageref| checks, whether a star is present (analogous to
% |\HyPsd@@pageref| again from the \xpackage{hyperref} package
% of \textsc{Heiko Oberdiek}):
%
%    \begin{macrocode}
\def\pagesLTS@@pageref#1*#2\END{%
  \ifx\\#2\\% no star
    \pagesLTS@@@pageref{#1}%
  \else% star
    \expandafter\pagesLTS@@@pagerefstar%
  \fi%
}

\def\pagesLTS@@@pageref#1{\lastpagereftext{#1}}
\def\pagesLTS@@@pagerefstar#1{\lastpagereftextstar{#1}}

%    \end{macrocode}
% \end{macro}
%
% \phantomsection
% \begin{macro}{\lastpagereftext} \label{lastpagereftextlabel}
% \indent When |\lastpageref| is used somewhere inside the \texttt{txt} (text),
% i.\,e. not at the last page, it is defined as |\lastpagereftxt| (see above).
% When the page numbering scheme is \xpackage{fnsymbol}, and the \xpackage{hyperref}
% package has been loaded, a hyperref instead of a label is used for the reference
% to |pagesLTS.fnsymbol.local.\arabic{pagesLTS.fnsymbol.local}|.
% (And if the \texttt{pagesLTS.fnsymbol.local} counter did not exist yet,
% it is created here.)\\
%
%    \begin{macrocode}
\newcommand{\lastpagereftext}[1]{%
  \def\pagesLTS@tmpA{#1}%
  \def\pagesLTS@tmpB{pagesLTS.fnsymbol.local}%
  \ifx\pagesLTS@tmpA\pagesLTS@tmpB%
    \pagesLTS@ifcounter{pagesLTS.fnsymbol.local}%
    \ltx@ifpackageloaded{hyperref}{%
      \href{\#pagesLTS.fnsymbol.local.\arabic{pagesLTS.fnsymbol.local}}%
        {\arabic{pagesLTS.fnsymbol.local}}%
%    \end{macrocode}
%
% When the page numbering scheme is \xpackage{fnsymbol}, but the \xpackage{hyperref} package
% has \textit{not} been loaded, just the arabic number of the
% \texttt{pagesLTS.fnsymbol.local} counter is given (because there will be
% no hyperlink anyway).
%
%    \begin{macrocode}
     }{\arabic{pagesLTS.fnsymbol.local}}%
%    \end{macrocode}
%
% Otherwise just the common |\pageref| is applied:
%
%    \begin{macrocode}
  \else%
    \pageref{#1}%
  \fi%
%    \end{macrocode}
%
% We do not need the temporary definitions any more.
%
%    \begin{macrocode}
  \let\pagesLTS@tmpA\undefined%
  \let\pagesLTS@tmpB\undefined%
  }

%    \end{macrocode}
% \end{macro}
%
% \begin{macro}{\lastpagereftextstar}
% And the same for the starred version, where no hyperlink is generated:
%
%    \begin{macrocode}
\newcommand{\lastpagereftextstar}[1]{%
  \def\pagesLTS@tmpA{#1}%
  \def\pagesLTS@tmpB{pagesLTS.fnsymbol.local}%
  \ifx\pagesLTS@tmpA\pagesLTS@tmpB%
     \pagesLTS@ifcounter{pagesLTS.fnsymbol.local}%
     \arabic{pagesLTS.fnsymbol.local}%
  \else%
%    \end{macrocode}
%
% There is no |\pageref*| without \xpackage{hyperref}.
%
%    \begin{macrocode}
    \ltx@ifpackageloaded{hyperref}{\pageref*{#1}}{\pageref{#1}}%
  \fi%
  \let\pagesLTS@tmpA\undefined%
  \let\pagesLTS@tmpB\undefined%
  }

%    \end{macrocode}
% \end{macro}
%
% \begin{macro}{\lastpagerefend}
% \indent When the \xpackage{hyperref} package is used and the page numbering scheme
% of the last page is \xpackage{fnsymbol}, |\lastpageref| is defined as
% |\lastpagerefend|.
% Hyperrefs instead of labels are used for the reference to
% \xpackage{fnsymbol} pages (including the last one).\\
%
% Again it must be discriminated between unstarred form and starred form:
%
%    \begin{macrocode}
\def\lastpagerefend#1{\pagesLTS@@pagerefend#1*\END}

\def\pagesLTS@@pagerefend#1*#2\END{%
  \ifx\\#2\\% no star
    \pagesLTS@@@pagerefend{#1}%
  \else% star
    \expandafter\pagesLTS@@@pagerefendstar%
  \fi%
}

\def\pagesLTS@@@pagerefend#1{\l@stpagerefend{#1}}
\def\pagesLTS@@@pagerefendstar#1{\l@stpagerefendstar{#1}}

%    \end{macrocode}
% \end{macro}
%
% \begin{macro}{\l@stpagerefend}
% The unstarred form (i.\,e. with hyperlinks, if \xpackage{hyperref} is loaded,
% otherwise without hyperlinks):
%
%    \begin{macrocode}
\newcommand{\l@stpagerefend}[1]{%
  \def\pagesLTS@tmpA{#1}%
  \def\pagesLTS@tmpB{pagesLTS.fnsymbol.local}%
  \ifx\pagesLTS@tmpA\pagesLTS@tmpB%
    \pagesLTS@ifcounter{pagesLTS.fnsymbol.local}%
    \ltx@ifpackageloaded{hyperref}{%
      \href{\#pagesLTS.fnsymbol.local.\arabic{pagesLTS.fnsymbol.local}}%
        {\arabic{pagesLTS.fnsymbol.local}}%
     }{\arabic{pagesLTS.fnsymbol.local}}%
  \else%
    \def\pagesLTS@tmpB{pagesLTS.fnsymbol}%
    \ifx\pagesLTS@tmpA\pagesLTS@tmpB%
      \ltx@ifpackageloaded{hyperref}{%
        \href{\#pagesLTS.fnsymbol.local.\arabic{pagesLTS.fnsymbol.local}}%
          {\pagesLTS.lastpage}%
       }{\pageref{pagesLTS.fnsymbol}}%
    \else%
      \def\pagesLTS@tmpB{LastPage}%
      \ifx\pagesLTS@tmpA\pagesLTS@tmpB%
        \ltx@ifpackageloaded{hyperref}{%
          \href{\#pagesLTS.fnsymbol.local.\pagesLTS@eso}%
            {\pagesLTS.lastpage}%
         }{\pageref{LastPage}}%
      \else%
        \def\pagesLTS@tmpB{VeryLastPage}%
        \ifx\pagesLTS@tmpA\pagesLTS@tmpB%
          \ltx@ifpackageloaded{hyperref}{%
            \href{\#pagesLTS.fnsymbol.local.\arabic{pagesLTS.fnsymbol.local}}%
              {\pagesLTS.lastpage}%
           }{\pageref{VeryLastPage}}%
        \else%
          \def\pagesLTS@tmpB{LastPages}%
          \ifx\pagesLTS@tmpA\pagesLTS@tmpB%
            \ltx@ifpackageloaded{hyperref}{%
              \href{\#pagesLTS.fnsymbol.local.\arabic{pagesLTS.fnsymbol.local}}%
                {\arabic{pagesLTS.pagenr}}%
             }{\pageref{LastPages}}%
          \else%
            \pageref{#1}%
          \fi%
        \fi%
      \fi%
    \fi%
  \fi%
%    \end{macrocode}
%
% We do not need the temporary definitions any more.
%
%    \begin{macrocode}
  \let\pagesLTS@tmpA\undefined%
  \let\pagesLTS@tmpB\undefined%
  }

%    \end{macrocode}
% \end{macro}
%
% \pagebreak
%
% \begin{macro}{\l@stpagerefendstar}%
% And the starred form, without hyperlinks, even if \xpackage{hyperref} is loaded,
% otherwise (i.\,e. without loaded \xpackage{hyperref}) this command is not called:
%
%    \begin{macrocode}
\newcommand{\l@stpagerefendstar}[1]{%
  \def\pagesLTS@tmpA{#1}%
  \def\pagesLTS@tmpB{pagesLTS.fnsymbol.local}%
  \ifx\pagesLTS@tmpA\pagesLTS@tmpB%
    \pagesLTS@ifcounter{pagesLTS.fnsymbol.local}%
    \arabic{pagesLTS.fnsymbol.local}%
  \else%
    \def\pagesLTS@tmpB{pagesLTS.fnsymbol}%
    \ifx\pagesLTS@tmpA\pagesLTS@tmpB%
      \ltx@ifpackageloaded{hyperref}{\pageref*{pagesLTS.fnsymbol}}{%
        \pageref{pagesLTS.fnsymbol}}%
    \else%
      \def\pagesLTS@tmpB{LastPage}%
      \ifx\pagesLTS@tmpA\pagesLTS@tmpB%
        \ltx@ifpackageloaded{hyperref}{\pageref*{LastPage}}{\pageref{LastPage}}%
      \else%
        \def\pagesLTS@tmpB{VeryLastPage}%
        \ifx\pagesLTS@tmpA\pagesLTS@tmpB%
          \ltx@ifpackageloaded{hyperref}{\pageref*{VeryLastPage}}{%
            \pageref{VeryLastPage}}%
        \else%
          \def\pagesLTS@tmpB{LastPages}%
          \ifx\pagesLTS@tmpA\pagesLTS@tmpB%
            \ltx@ifpackageloaded{hyperref}{\pageref*{LastPages}}{\pageref{LastPages}}%
          \else%
            \ltx@ifpackageloaded{hyperref}{\pageref*{#1}}{\pageref{#1}}%
            \fi%
          \fi%
        \fi%
      \fi%
    \fi%
  \fi%
  \let\pagesLTS@tmpA\undefined%
  \let\pagesLTS@tmpB\undefined%
  }

%    \end{macrocode}
% \end{macro}
%
% \pagebreak
%
% \begin{macro}{\overrideLTSlabel}
% |\overridelabel| from the \xpackage{undolabl} package just |\undonewlabel|s
% a~label and places a new |\label{#1}|, but we need to place a
% |\pagesLTS@putlabel{#1}{#2}|, therefore we need another command instead of
% (but somewhat similar to) |\overridelabel|:
%
%    \begin{macrocode}
% somewhat analogous to \overridelabel from the undolabl package:
\newcommand\overrideLTSlabel[2]{%
  \@bsphack
    \ifnum \value{pagesLTS.pnc.\pagesLTS@pnc}>1%
      \edef\pagesLTStmpA{#1}%
      \edef\pagesLTStmpB{pagesLTS.\pagesLTS@pnc.local}%
      \ifx\pagesLTStmpA\pagesLTStmpB%
          \immediate\write\@auxout{\string\undonewlabel{#1}}%
          \@overriddenmessage s{#1}%
      \fi%
    \fi%
    \pagesLTS@putlabel{#1}{#2}{0}%
  \@esphack%
  }

%    \end{macrocode}
% \end{macro}
%
% \begin{macro}{\@pagesLTS@Prelim@EveryShipout}
% \indent Because we cannot make references to pages with \xpackage{fnsymbol}
% page \textquotedblleft numbers\textquotedblright{} manually
% with \xpackage{hyperref}, we use \nolinebreak{|\phantomsection|\,s} and
% refer to one of those. But because we do not know how many
% \nolinebreak{|\phantomsection|\,s} and \nolinebreak{|\section*|\,s}
% are introduced by the user (or other packages;
% cf.~\LaTeX~bug~2298: knowing level of |section*|,
% \ \url{http://www.latex-project.org/cgi-bin/ltxbugs2html?category=LaTeX&responsible=anyone&state=open&keyword=&pr=latex%2F2298&search=}),
% we cannot refer to the last one as we did with the pages.
%
%    \begin{macrocode}
\newcommand{\@pagesLTS@Prelim@EveryShipout}{%
%% The following code is from the prelim2e package             %%
%% [2009/05/29 v1.3] by Martin Schr\"{o}der (Thanks!):         %%
  \bgroup
    \dimen\z@=\wd\@cclv
    \dimen\@ne=\ht\@cclv
    \dimen\tw@=\dp\@cclv
    \dimen\thr@@=\dimen1
    \advance\dimen\thr@@ by \dimen\tw@
    \global\setbox\@cclv\vbox to \dimen\thr@@{%
      \hb@xt@\dimen\z@{%
        \box\@cclv%
        \hss%
        }%
      \vbox to \z@{%
        \hb@xt@\dimen\z@{%
          \let\protect\relax
%% Code not from prelim2e package:                             %%
%    \end{macrocode}
%
% Therefore each page with \xpackage{fnsymbol} page
% \textquotedblleft number\textquotedblright{} receives a
% |\phantomsection| and a label, which includes a number increased
% by one for each page. This is done for
% |pagesLTS.fnsymbol.local.\arabic{pagesLTS.fnsymbol.cont}|
% as well as
% |pagesLTS.fnsymbol|, |pagesLTS.\pagesLTS@pnc|, and
% |pagesLTS.\pagesLTS@pnc.local|.\\
% In case an older label already existed, it is overwritten by an
% |\overridelabel| command.
%
%    \begin{macrocode}
          \ifx\pagesLTS@pnc\pagesLTS@fns%
            \pagesLTS@ifcounter{pagesLTS.fnsymbol.cont}%
            \addtocounter{pagesLTS.fnsymbol.cont}{1}%
            \ltx@ifpackageloaded{hyperref}{%
              \phantomsection%
              \hypertarget{pagesLTS.fnsymbol.local.\arabic{pagesLTS.fnsymbol.cont}}{}%
              }{}%
            \ifnum \pagesLTS@esov=\pagesLTS@zero%
              \label{pagesLTS.fnsymbol}%
            \else%
              \overridelabel{pagesLTS.fnsymbol}%
            \fi%
          \else%
            \ltx@ifpackageloaded{hyperref}{\phantomsection}{}%
            \if@filesw%
              \overridelabel{pagesLTS.\pagesLTS@pnc}%
%    \end{macrocode}
%
% We need to go forward one page (and later backward again), because
% |\overrideLTSlabel| calls a |\pagesLTS@putlabel|, and that one uses
% |\addtocounter{page}{-1}...\addtocounter{page}{+1}|, which is not
% needed here.
%
%    \begin{macrocode}
              \addtocounter{page}{+1}%
                \overrideLTSlabel{pagesLTS.\pagesLTS@pnc.local}{\theCurrentPageLocal}%
              \addtocounter{page}{-1}%
            \fi%
          \fi%
%% Code from prelim2e package again:                           %%
        }%
        \vss%
      }%
      \vss%
    }%
    \wd\@cclv=\dimen\z@
    \ht\@cclv=\dimen\@ne
    \dp\@cclv=\dimen\tw@
  \egroup
%% End of code from the prelim2e package.                      %%
  }

%    \end{macrocode}
% \end{macro}
%
% \pagebreak
%
% \begin{macro}{\EveryShipout}
% \indent At the end of each shipout, the following commands are executed:
%
%    \begin{macrocode}
\EveryShipout{%
  \ifnum\value{page}>0% \relax
  \else%
    \ifnum\value{page}=0%
      \PackageWarning{pageslts}{%
        Counter `page' is zero!\MessageBreak%
        If the page numbering scheme is not arabic\MessageBreak%
        and further not extended\MessageBreak%
        (see Page counter overflow in the pageslts\MessageBreak%
        documentation), without other measures\MessageBreak%
        this will lead to a counter overflow.\MessageBreak%
      }%
    \else%
      \ifnum\value{page}<0%
        \PackageWarning{pageslts}{%
          Counter `page' is negative: `\the\value{page}'!\MessageBreak%
          If the page numbering scheme is not arabic\MessageBreak%
          and further not extended\MessageBreak%
          (see Page counter overflow in the pageslts\MessageBreak%
          documentation), without other measures\MessageBreak%
          this will lead to a counter overflow.\MessageBreak%
        }%
      \else%
        \PackageError{pageslts}{%
          Counter `page' does not have a recognized value:\MessageBreak%
          `\the\value{page}'\MessageBreak%
          \@ehd \MessageBreak%
          }%
      \fi%
    \fi%
  \fi%
%    \end{macrocode}
%
% If the |CurrentPage| is equal to one, this is the first shipout.
%
%    \begin{macrocode}
  \ifnum \value{CurrentPage}=1% This is the first shipout!
%    \end{macrocode}
%
% We check whether some page numbering scheme was defined by
% |\pagenumbering{...}| (\textbf{as it schould be!}):
%
%    \begin{macrocode}
    \ifx\pagesLTS@called\pagesLTS@zero%
%    \end{macrocode}
%
% If it was not defined (i.\,e. |\pagesLTS@called| is zero),
% the user should be informed, that a |\pagenumbering{...}|
% is missing behind |\begin{document}|.
% Of course, it is possible that some package did some
% pages of output with |\AtBeginDocument|. In that case,
% one |\pagenumbering{...}| before |\begin{document}| and
% one |\pagenumbering{...}| (with the same argument, of course!)
% behind |\begin{document}| could help somewhat. When |\PackageError|
% was used here, the error message was not written to the screen and
% the \xfile{.log}-file, but into the document. Therefore we just make
% a note to give the error message later (|\AtEndDocument|). At that
% time unfortunately most of the document has already been compiled
% (or did not compile due to this error), but I do not know how to
% change that.
%
%    \begin{macrocode}
      \global\def\pncmissing{1}%
%    \end{macrocode}
%
% We save the current value of the page,
%
%    \begin{macrocode}
      \mathchardef\pagesLTS@tmpD=\arabic{page}%
%    \end{macrocode}
%
% determine the current page numbering scheme,
%
%    \begin{macrocode}
%% Code from Andres L\"{o}h, Universiteit Utrecht (NL)         %%
      \def\extract#1{\expandafter\extract@ #1\END}%
      \def\extract@#1\csname @#2\endcsname#3\END{#2}%
      \edef\pagesLTS@tmpQ{\extract\thepage}%
%% End of code from Andres L\"{o}h                             %%
      \let\pagesLTS@tmpP\pagesLTS@tmpQ%
%    \end{macrocode}
%
% set the current page numbering scheme to~0
% (because before the beginning of the document it should be~0),
%
%    \begin{macrocode}
      \def\pagesLTS@pnc{0}%
%    \end{macrocode}
%
% and then issue a |\pagenumbering| command with the determined
% page numbering scheme as argument:
%
%    \begin{macrocode}
      \pagenumbering{\pagesLTS@tmpP}%
%    \end{macrocode}
%
% This resets the page to one (if option |pagecontinue=false|
% was chosen), but because we do not start a new page numbering
% scheme here but manifest a page numbering scheme, which
% the user forgot to define, the page number should not have
% been reset to one. (This is the first page, but maybe the user
% wants it to have page number 2001?) Therefore we revert this here
% and set the page number to its value, which was saved before the
% |\pagenumbering| command.
%
%    \begin{macrocode}
      \setcounter{page}{\pagesLTS@tmpD}%
    \fi%
%    \end{macrocode}
%
% We are at the first page, so we put the label here.
%
%    \begin{macrocode}
    \pagesLTS@writelabel{0}%
  \fi%
%    \end{macrocode}
%
% If the current page numbering scheme |\pagesLTS@pnc| is
% |\pagesLTS@fns| (which is defined as \texttt{fnsymbol}),
% the label is set by \linebreak
% |\@pagesLTS@Prelim@EveryShipout| (see just above),
% and |\pagesLTS@esov| is set to the (real) number (not the name) of
% this page numbering scheme, |\arabic{pagesLTS.fnsymbol.cont}|.\\
% When no more pages with \xpackage{fnsymbol} page
% \textquotedblleft number\textquotedblright{} are shipped out,
% the value remains fixed and we have our reference to the last page
% of the \xpackage{fnsymbol} page numbering range.
% (At least we will have that reference after some more work,
% see below).
%
%    \begin{macrocode}
  \ifx\pagesLTS@pnc\pagesLTS@fns%
    \@pagesLTS@Prelim@EveryShipout%
    \gdef\pagesLTS@esov{\arabic{pagesLTS.fnsymbol.cont}}%
%    \end{macrocode}
%
% When another page numbering scheme was reused (in the example file
% \xpackage{Roman}), we also need to apply\\
% |\@pagesLTS@Prelim@EveryShipout|,
% because otherwise we would get multiply defined labels.
%
%    \begin{macrocode}
  \else%
    \ifnum \value{pagesLTS.pnc.\pagesLTS@pnc}>1%
      \@pagesLTS@Prelim@EveryShipout%
    \fi%
  \fi%
%    \end{macrocode}
%
% The |CurrentPage| as well as the |pagesLTS.current.local.\pagesLTS@pnc|
% are advanced by one (because one page was shipped out and the next is
% about to begin).
%
%    \begin{macrocode}
  \addtocounter{CurrentPage}{1}%
  \addtocounter{pagesLTS.current.local.\pagesLTS@pnc}{1}%
  }

%    \end{macrocode}
% \end{macro}
%
% \begin{macro}{\pagesLTS@putlabelhyper}
% \indent Here the labels are set, if the \xpackage{hyperref} package
% was loaded. Simply using |\label| would not work, because labels
% wait for the output routines to work, and there may not be any
% more invocations of the output routines. To force the write out
% we need to do an |\immediate| write.
%
%    \begin{macrocode}
\newcommand{\pagesLTS@putlabelhyper}[2]{%
  \ifHy@pageanchor \relax%
  \else%
%    \end{macrocode}
%
% If the \xpackage{hyperref} package is used, but pageanchors are disabled,
% the hyperlinking will not work.
%
%    \begin{macrocode}
    \PackageError{pagesLTS}{hyperref option pageanchor disabled}{%
      The \string\lastpageref{#1} link doesn't work\MessageBreak%
      using hyperref with disabled option `pageanchor'.\MessageBreak%
    }%
  \fi
%    \end{macrocode}
%
% If use of the \xfile{.aux}-file is allowed, the label for
% \texttt{LastPage} is written into that file, the page
% reference depending on the options, which where set for
% the \xpackage{hyperref} package.
%
%    \begin{macrocode}
%% The following code is from the hyperref package             %%
%% [2010/04/17 v6.80x; newer versions are available]           %%
%% by Heiko Oberdiek (Big Thanks!).                            %%
  \if@filesw
    \begingroup
      \let\@number\@firstofone
      \ifHy@pageanchor
        \ifHy@hypertexnames
          \ifHy@plainpages
            \def\Hy@temp{\arabic{page}}%
          \else
            \Hy@unicodefalse
%% Code not from hyperref package:                             %%
%% The following lines are modified from the hyperref package. %%
%% Without the modification, after the first shipout "PD1" is  %%
%% inserted each time |\pdfstringdef\Hy@temp{\thepage}| is     %%
%% executed (if |fnsymbol| is not used).                       %%
            \ifnum \value{CurrentPage}=1%
              \ifx\pagesLTS@pnc\pagesLTS@fns%
                \pdfstringdef\Hy@temp{\thepage}%
              \else%
                \def\Hy@temp{\thepage}%
              \fi%
            \else%
              \pdfstringdef\Hy@temp{\thepage}%
            \fi%
%% Code from hyperref package again:                           %%
          \fi
        \else
          \def\Hy@temp{\the\Hy@pagecounter}%
        \fi
      \fi
%% End of code from the hyperref package.                      %%
%% (The following four lines are modified                      %%
%%  from the hyperref package.)                                %%
      \immediate\write\@auxout{\string
        \newlabel{#1}{{}{#2}{}{\ifHy@pageanchor page.\Hy@temp\fi}{}}}%
    \endgroup%
  \fi%
  }

%    \end{macrocode}
% \end{macro}
%
% \phantomsection
% \begin{macro}{\pagesLTS@putlabel}\label{pagesLTSputlabel}
% \indent Since the page has been put out, we are on the page after
% that page. We therefore subtract one from the page counter.
%
%    \begin{macrocode}
\newcommand{\pagesLTS@putlabel}[3]{%
  \addtocounter{page}{-1}%
%    \end{macrocode}
%
% When the \xpackage{showkeys} package has been loaded in \texttt{draft} mode,
% in the margin for each label a box is displayed with the name of the label.
% \xpackage{showkeys} accomplishes this by redefining |\label|, but
% \xpackage{pageslts} does not use |\label|, but writes directly to the
% {\nobreak |\jobname|\xfile{.aux}-}\allowbreak file, and this is generally done
% \emph{after} the according page has shipped out, therefore no box can be placed
% on the \emph{preceding} page. At least \xpackage{pageslts} gives a warning, that
% \xpackage{showkeys} cannot present the respective label.
%
%    \begin{macrocode}
  \ifx\pagesLTS@SK\pagesLTS@one\relax%
    \message{^^J}%
    \message{Package pageslts Warning: Package showkeys without option final loaded,}%
    \ifnum\value{pagesLTS.pagenr}<1%
      \message{(pageslts)                but label #1 on page \thepage\space(about \theCurrentPage)}%
    \else%
      \message{(pageslts)                but label #1 on page \thepage\space(about \theCurrentPage\space of \arabic{pagesLTS.pagenr})}%
    \fi%
    \message{(pageslts)                cannot be shown, because pageslts does not use \string\label,}%
    \message{(pageslts)                but writes directly to the \jobname.aux file. ^^J}%
  \fi%
%    \end{macrocode}
%
% If the \xpackage{hyperref} package is used, the format of the labels is somewhat longer.
%
%    \begin{macrocode}
  \ltx@ifpackageloaded{hyperref}{\pagesLTS@putlabelhyper{#1}{#2}}{% else
%    \end{macrocode}
%
% \pagebreak
%
% If the \xpackage{hyperref} package is not used, there will be no hyperlinks,
% and the label is written in the way of the old \xpackage{lastpage} package.
% But we must remember to undo the label first, if it already exists.
%
%    \begin{macrocode}
    \if@filesw%
      \ifnum \value{pagesLTS.pnc.\pagesLTS@pnc}<2%
%    \end{macrocode}
%
% When the \xpackage{nameref} package is used, |\newlabel| needs five instead of two arguments:
%
%    \begin{macrocode}
        \ltx@ifpackageloaded{nameref}{%
          \immediate\write\@auxout{\string\newlabel{#1}{{}{#2}{}{}{}}}}{% else
          \immediate\write\@auxout{\string\newlabel{#1}{{}{#2}}}}%
      \else%
        \edef\pagesLTStmpA{#1}%
        \edef\pagesLTStmpB{pagesLTS.\pagesLTS@pnc.local}%
        \ifx\pagesLTStmpA\pagesLTStmpB%
          \edef\pagesLTStmpA{#3}%
          \ifx\pagesLTStmpA\pagesLTS@one%
%    \end{macrocode}
%
% Only when the third argument of |\pagesLTS@putlabel| is $1$,
% we do need to undo the label. Otherwise there is no label to
% undo, and the \xpackage{undolabl} package would give an error.
%
%    \begin{macrocode}
            \immediate\write\@auxout{\string\undonewlabel{#1}}%
          \fi%
        \fi%
        \ltx@ifpackageloaded{nameref}{%
          \immediate\write\@auxout{\string\newlabel{#1}{{}{#2}{}{}{}}}}{% else
          \immediate\write\@auxout{\string\newlabel{#1}{{}{#2}}}}%
      \fi%
    \fi%
  }%
%    \end{macrocode}
%
% After the writeout we restore the page number again,
% since there might be other things still to be done.
%
%    \begin{macrocode}
  \addtocounter{page}{+1}%
  }

%    \end{macrocode}
% \end{macro}
%
% \pagebreak
%
% \begin{macro}{\pagesLTS@putlabels}
% \indent |\pagesLTS@putlabels| is nearly identical to |\pagesLTS@putlabelV|:
%
%    \begin{macrocode}
\newcommand{\pagesLTS@putlabels}{%
  \addtocounter{page}{-1}%
  \addtocounter{CurrentPage}{-1}%
  \addtocounter{pagesLTS.current.local.\pagesLTS@pnc}{-1}%
%    \end{macrocode}
%
% If |\pagenumbering{...}| has not been used,
% |\pagesLTS@pnc| is still zero ($0$, |\pagesLTS@zero|),
% and the according warning message is given.
%
%    \begin{macrocode}
  \ifx\pagesLTS@pnc\pagesLTS@zero%
   \PackageWarning{pagesLTS}{No page numbering scheme found:\MessageBreak%
     \pagesLTS@messageNPN }%
%    \end{macrocode}
%
% \noindent otherwise the numbered label is written,
% and if the page numbering scheme was not used before,
% the unnumbered label is written, too.
%
%
%    \begin{macrocode}
  \else%
    \pagesLTS@writelabel{\pagesLTS@pnc.\arabic{pagesLTS.pnc.\pagesLTS@pnc}}%
    \ifnum \value{pagesLTS.pnc.\pagesLTS@pnc}<2%
      \ifx\pagesLTS@pnc\pagesLTS@fns% \relax
      \else%
        \pagesLTS@writelabel{\pagesLTS@pnc}%
      \fi%
    \fi%
  \fi%
%    \end{macrocode}
%
% \noindent Before the label for the \texttt{LastPages} can be put,
% we must advance one page again, because |\pagesLTS@putlabel| itself
% goes back one page (and at its end forward again).
%
%    \begin{macrocode}
  \addtocounter{page}{+1}%
  \pagesLTS@putlabel{LastPages}{\theCurrentPage}{1}%
%    \end{macrocode}
%
% \noindent Here should follow a\\
%  |\addtocounter{page}{-1}|,\\
% but we have to remember to increase the page counters again,
% which were decreased at the start of this |\pagesLTS@putlabels|
% command, and that would include\\
%  |\addtocounter{page}{+1}|,\\
% therefore this two lines cancel each other and
% therefore just can be skipped. But the other counters have to
% be increased:
%
%    \begin{macrocode}
  \addtocounter{CurrentPage}{+1}%
  \addtocounter{pagesLTS.current.local.\pagesLTS@pnc}{1}%
  }

%    \end{macrocode}
% \end{macro}
%
% \pagebreak
%
% \begin{macro}{\AtBeginDocument}
% \indent |\AtBeginDocument| we write into the \xfile{aux} file regarding
% the double loading of the package.
% Then it is checked whether the \texttt{endfloat}
% package has been loaded, whether it is newer than March 1992 (i.\,e. at least
% April 1992 v2.0), in which case it is compatible with this
% \xpackage{pageslts} package.\\
% If it is even newer than 2011/12/24, it is the recent version
% (as of the time of last revision of this documentation:
% 2011/12/25 v2.5d).
%
%    \begin{macrocode}
\AtBeginDocument{%
  \if@filesw%
    \immediate\write\@auxout{\string\gdef\string\pagesLTS@loaded{p@gesLTSnotlo@ded}}%
  \fi%
  \@ifpackageloaded{endfloat}%
   {\@ifpackagelater{endfloat}{1992/03/31}% April 1992 v2.0
    {\@ifpackagelater{endfloat}{2011/12/24}{% 2011/12/25 v2.5d
       \relax}%
%    \end{macrocode}
%
% If it is compatible, but not the recent version, a warning is given:
%
%    \begin{macrocode}
     {\PackageWarningNoLine{pageslts}{Old endfloat package detected:\MessageBreak%
        There is a newer version of the endfloat package available.\MessageBreak%
        Please consider updating your version.\MessageBreak%
        The pageslts package might be incompatible with\MessageBreak%
        your current endfloat package.\MessageBreak%
      }%
     }%
%    \end{macrocode}
%
% If it is so very old, that it is not compatible,
% an Error message is given:
%
%    \begin{macrocode}
    }{\PackageError{pageslts}{Incompatible, very old endfloat package detected.}%
      {The very old version 2.0 (and earlier) of the\MessageBreak%
       endfloat package actually redefined the \string\enddocument ,\MessageBreak%
       and so interfered drastically with the LaTeX2e commands\MessageBreak%
       which make use of \string\AtEndDocument .\MessageBreak%
       Newer versions of the endfloat package exists\MessageBreak%
       (at least: v2.5d as of 2011/12/25)\MessageBreak%
       in modern documentation form,\MessageBreak%
       which should be available from CTAN.\MessageBreak%
       Please update your endfloat package\MessageBreak%
       for use with the pageslts package.\MessageBreak %
      }%
     }%
   }{}%
%    \end{macrocode}
%
% \pagebreak
%
% It is checked whether the old \xpackage{lastpage} package has been loaded.\\
% (If it has been loaded indeed, the |\lastpage@putlabel| is
% \textquotedblleft killed\textquotedblright ,
% see subsection~\ref{ssec:lastpage}.)
%
%    \begin{macrocode}
  \@ifpackageloaded{lastpage}%
   {\@ifpackagelater{lastpage}{2010/07/28}% 2010/07/29 v1.2a
      {\@ifpackagelater{lastpage}{2013/01/27}% 2013/01/28 v1.2l
         {\PackageWarning{pageslts}{lastpage package detected.\MessageBreak%
           With pageslts package in use, lastpage has no function.\MessageBreak%
           Just remove the lastpage package from your document.\MessageBreak%
           }%
         }{%
         \PackageWarning{pageslts}{Old lastpage package detected.\MessageBreak%
           With pageslts package in use, lastpage has no function.\MessageBreak%
           Just remove the lastpage package from your document.\MessageBreak%
           At least update it!\MessageBreak%
           }%
         }%
      }{%
       \PackageWarning{pageslts}{Incompatible package lastpage detected:\MessageBreak%
         Package pageslts was loaded, but also an old\MessageBreak%
         version of the lastpage package.\MessageBreak%
         pageslts has all functionality of the lastpage\MessageBreak%
         package (and more), so just remove the lastpage\MessageBreak%
         package from your document.\MessageBreak%
         (At least update it!)\MessageBreak%
         pageslts will now ''kill'' the lastpage@putlabel\MessageBreak%
         command of the lastpage package.\MessageBreak%
       }%
       \gdef\lastpage@putlabel{\relax}%
      }%
   }{}%
%    \end{macrocode}
%
% Further it is checked whether the \xpackage{alphalph} package has been loaded.
% If that is the case, the commands are defined accordingly.\\
%
%    \begin{macrocode}
  \ltx@ifpackageloaded{alphalph}%
    {\newalphalph{\AlphMult}[mult]{\@Alph}{26}%
     \newalphalph{\alphMult}[mult]{\@alph}{26}%
     \newalphalph{\fnsymbolmult}[mult]{\@fnsymbol}{5}%
    }{}%
%    \end{macrocode}
%
% \pagebreak
%
% Further it is checked whether the \xpackage{hyperref} package has been loaded:
%
%    \begin{macrocode}
  \ltx@ifpackageloaded{hyperref}{%
%    \end{macrocode}
%
% and whether the \xpackage{pdfpages} package is loaded:
%
%    \begin{macrocode}
     \@ifpackageloaded{pdfpages}%
       {\PackageWarningNoLine{pageslts}{Package pdfpages detected.\MessageBreak%
          Using hyperref with pdfpages can cause problems. See\MessageBreak%
          https://www.ctan.org/pkg/pax\MessageBreak%
          for the Pdf Annotations eXtractor, pax%
          }%
       }{% \relax
        }%
%    \end{macrocode}
%
% The \xpackage{undolabl} package has been updated and now uses |\undonewlabel| with only
% one argument.
%
%    \begin{macrocode}
     \@ifpackageloaded{undolabl}%
       {\@ifpackagelater{undolabl}{2010/07/14}%    2010/07/15 v1.0d
          {\@ifpackagelater{undolabl}{2015/03/29}% 2015/03/29 v1.0l
             {% recent version as of the time of last revision of this package: OK
             }{% old, but not obsolete version
               \PackageWarningNoLine{pageslts}{Old version of undolabl package used.\MessageBreak%
                 See https://www.ctan.org/pkg/undolabl\MessageBreak%
                 for a new version.\MessageBreak%
                }%
             }
          }{\PackageError{pageslts}{Incompatible, obsolete version of undolabl package used.}{%
              See https://www.ctan.org/pkg/undolabl\MessageBreak%
              for a new version.\MessageBreak%
              Type  X <return>  to quit.\MessageBreak%
             }%
          }
       }{\PackageError{pageslts}{Package undolabl missing}{%
           Package undolabl not found.\MessageBreak%
           The pageslts package needs the undolabl package.\MessageBreak%
           See https://www.ctan.org/pkg/undolabl\MessageBreak%
           Type  X <return>  to quit.\MessageBreak%
           }%
       }
%    \end{macrocode}
%
% \pagebreak
%
% Additionally a version check of the available \xpackage{hyperref}
% package is performed and if need be a warning is issued:
%
%    \begin{macrocode}
     \@ifpackagelater{hyperref}{2012/11/05}{% 2012/11/06 v6.83m
       \relax}{%
       \PackageWarningNoLine{pageslts}{Old hyperref package detected:\MessageBreak%
         There is a newer version of the\MessageBreak%
         hyperref package available:\MessageBreak%
         https://www.ctan.org/pkg/hyperref\MessageBreak%
         Please consider updating your version.\MessageBreak%
       }%
     }%
     %% pageslts supports the use of the package hyperref by
     %% Heiko Oberdiek (hyperref version 2012/11/06 v6.83m).
     %% pageslts may work with earlier versions of this packages,
     %% but this was not tested.
    }{%
%    \end{macrocode}
%
% If no \xpackage{hyperref} package in use is detected, a warning is issued, too:
%
%    \begin{macrocode}
      \PackageWarning{pageslts}{Package hyperref NOT detected.\MessageBreak%
        pageslts would support hyperref. The page references\MessageBreak%
        will NOT be hyperlinked!\MessageBreak }%
     }%
%    \end{macrocode}
%
% If the \xpackage{showkeys} package has been loaded in \texttt{draft} mode,
% warnings about missing label boxes will be issued (see p.~\pageref{pagesLTSputlabel}),
% but because it cannot be checked for \xpackage{showkeys} after |\AtBeginDocument|,
% the check must be done here and the result remembered.
%
%    \begin{macrocode}
  \ltx@ifpackageloaded{showkeys}{%
    \@ifpackagewith{showkeys}{final}{}{\gdef\pagesLTS@SK{1}}}{% else \relax
    }%
  }

%    \end{macrocode}
% \end{macro}
%
% \begin{macro}{\AtEndDocument}
% \indent |\AtEndDocument| the check is repeated. If \xpackage{showkeys} was
% loaded after the check performed before, an error message is issued.
%
%    \begin{macrocode}
\AtEndDocument{%
  \ltx@ifpackageloaded{showkeys}{%
    \ifx\pagesLTS@SK\pagesLTS@one\relax%
    \else%
      \providecommand*\pagesLTS@sklf[1]{}%
      \ifx\showkeyslabelformat\pagesLTS@sklf\relax%
      \else%
        \PackageError{pageslts}{Package showkeys loaded after pageslts}{%
          The showkeys package with option "final" has been loaded\MessageBreak%
          after the pageslts package.\MessageBreak%
          Please first load showkeys and then pageslts}%
      \fi%
    \fi%
  }{% else no showcase, \relax
  }%
%    \end{macrocode}
%
% We also give the error message about the missing (i.\,e.~not found)
% page numbering scheme, which could not be given in |\EveryShipout|.
%
%    \begin{macrocode}
  \ifx\pncmissing\pagesLTS@one\relax%
    \PackageError{pageslts}{pagenumbering scheme missing}{\pagesLTS@messageNPN }%
  \fi%
%    \end{macrocode}
%
% Then we put in a |\message| to show, in what order things (which were called)
% are done (see subsection~\ref{ssec:aed}).
%
%    \begin{macrocode}
  \message{^^J%
    AED: pageslts setting LastPage ^^J}%
%    \end{macrocode}
%
% After this we issue a |\clearpage| to put out all floats,
% which are still floating, remember the page number (if |fnsymbol|),
% and after that we place the \texttt{LastPage} label.
%
%    \begin{macrocode}
  \clearpage%
  \ifx\pagesLTS@pnc\pagesLTS@fns%
    \def\pagesLTS@tmpA{\arabic{pagesLTS.fnsymbol.local}}%
    \ifnum \pagesLTS@eso=\pagesLTS@tmpA%
      \gdef\pagesLTS@rerun{0}%
    \else%
      \gdef\pagesLTS@rerun{1}%
    \fi%
    \if@filesw%
      \immediate\write\@auxout{\string\gdef\string\pagesLTS@eso{\pagesLTS@tmpA}}%
    \fi%
  \fi%
  \pagesLTS@putlabel{LastPage}{\thepage}{1}%
%    \end{macrocode}
%
% We do not need the temporary definition any more.
%
%    \begin{macrocode}
  \let\pagesLTS@tmpA\undefined%
  }

%    \end{macrocode}
% \end{macro}
%
% \begin{macro}{\AfterLastShipout}
% \indent |\AfterLastShipout| is a command from \textsc{Heiko Oberdiek's}
% \texttt{atveryend} package (see above).
%
%    \begin{macrocode}
\AfterLastShipout{%
%    \end{macrocode}
%
% If writing to the \xfile{.aux} file is allowed:
%
%    \begin{macrocode}
  \if@filesw%
%    \end{macrocode}
%
% The number of pages with the \xpackage{fnsymbol} page numbering scheme,
% |\pagesLTS@esov|, is saved via the \xfile{.aux} file (if it is not zero):
%
%    \begin{macrocode}
    \ifx\pagesLTS@esov\pagesLTS@zero%
    \else%
      \immediate\write\@auxout{\string
        \pagesLTS@ifcounter{pagesLTS.fnsymbol.local}}%
      \immediate\write\@auxout{\string
        \setcounter{pagesLTS.fnsymbol.local}{\pagesLTS@esov}}%
    \fi%
%    \end{macrocode}
%
% If the \xpackage{hyperref} package is in use, and the page numbering scheme
% of the last page is \xpackage{fnsymbol}, everything is quite more complicated.
% Therefore |\lastpageref| is switched from simple |\lastpagereftxt|
% to the more difficult |\lastpagerefend|.
%
%    \begin{macrocode}
    \ltx@ifpackageloaded{hyperref}{%
      \ifx\pagesLTS@pnc\pagesLTS@fns%
          \immediate\write\@auxout{\string
            \gdef\string\lastpageref{\string\lastpagerefend}}%
      \fi%
     }{}%
  \fi%
%    \end{macrocode}
%
% At the call of a |\pagenumbering{...}| command, everything
% for a split page numbering scheme is organized. For the last
% page numbering scheme, there is no |\pagenumbering{...}| command
% at the end, so we need to handle this here:
%
%    \begin{macrocode}
  \pagesLTS@ifcounter{pagesLTS.\pagesLTS@pnc.\arabic{pagesLTS.pnc.\pagesLTS@pnc}.local.count}%
  \setcounter{pagesLTS.\pagesLTS@pnc.\arabic{pagesLTS.pnc.\pagesLTS@pnc}.local.count}{%
    \value{pagesLTS.current.local.\pagesLTS@pnc}}%
%    \end{macrocode}
%
% And we are one page after the last one (|\AfterLastShipout|!),
% so we go back one page. (We again borrow the \texttt{pagesLTS.pnc.0} counter for the computations
% instead of defining yet another one.)
%
%    \begin{macrocode}
  \addtocounter{pagesLTS.\pagesLTS@pnc.\arabic{pagesLTS.pnc.\pagesLTS@pnc}.local.count}{-1}
  \ifnum \value{pagesLTS.pnc.\pagesLTS@pnc}>1%
    \mathchardef\pagesLTS@tmpD=\arabic{pagesLTS.pnc.0}%
    \setcounter{pagesLTS.pnc.0}{\value{pagesLTS.pnc.\pagesLTS@pnc}}%
    \@tempcnta=\value{pagesLTS.pnc.0}\relax%
    \loop%
      \ifnum\@tempcnta>1\relax%
        \addtocounter{pagesLTS.pnc.0}{-1}%
        \addtocounter{pagesLTS.\pagesLTS@pnc.\arabic{pagesLTS.pnc.\pagesLTS@pnc}.local.count}{%
          -\value{pagesLTS.\pagesLTS@pnc.\arabic{pagesLTS.pnc.0}.local.count}}%
        \@tempcnta=\value{pagesLTS.pnc.0}\relax%
    \repeat%
    \setcounter{pagesLTS.pnc.0}{\pagesLTS@tmpD}%
  \fi%
  \if@filesw%
    \immediate\write\@auxout{\string
      \pagesLTS@ifcounter{pagesLTS.\pagesLTS@pnc.\arabic{pagesLTS.pnc.\pagesLTS@pnc}.local.cnt}}%
    \edef\pagesLTS@tmpA{\arabic{pagesLTS.\pagesLTS@pnc.\arabic{pagesLTS.pnc.\pagesLTS@pnc}.local.count}}%
    \immediate\write\@auxout{\string
      \setcounter{pagesLTS.\pagesLTS@pnc.\arabic{pagesLTS.pnc.\pagesLTS@pnc}.local.cnt}{\pagesLTS@tmpA}}%
    \let\pagesLTS@tmpA\undefined%
  \fi%
%    \end{macrocode}
%
% We need to save (via the \xfile{.aux} file) the page name |\thepage| and
% the page number |\arabic{CurrentPage}| of the last page,
% in case the last page has \xpackage{fnsymbol} page numbering scheme.
%
%    \begin{macrocode}
  \addtocounter{page}{-1}%
  \edef\pagesLTS@tmpA{\thepage}%
  \if@filesw%
    \immediate\write\@auxout{\string
      \gdef\string\pagesLTS.lastpage{\pagesLTS@tmpA}}%
  \fi%
  \addtocounter{page}{+1}%
  \addtocounter{CurrentPage}{-1}%
  \edef\pagesLTS@tmpB{\arabic{CurrentPage}}%
  \if@filesw%
    \immediate\write\@auxout{\string
      \setcounter{pagesLTS.pagenr}{\pagesLTS@tmpB}}%
  \fi%
  \addtocounter{CurrentPage}{+1}%
%    \end{macrocode}
%
% The \texttt{VeryLastPage} label is set here, and when
% |\lastpageref{VeryLastPage}| instead of |\lastpageref{LastPage}| is used,
% it should really point to the last page.
% |LastPage| and |VeryLastPage| should be identical,
% unless a package was active with output |\AtEndDocument|
% \textit{after} the \xpackage{pageslts} package.
%
%    \begin{macrocode}
  \message{^^J%
    AED: pageslts setting VeryLastPage via AfterLastShipout ^^J}%
  \pagesLTS@putlabel{VeryLastPage}{\thepage}{1}%
%    \end{macrocode}
%
% The \texttt{LastPages} label is set here, and
% |\lastpageref{LastPages}| gives the total number of pages
% and points to the (very) last page.
%
%    \begin{macrocode}
  \message{^^J%
    AED: pageslts setting LastPages via AfterLastShipout ^^J}%
  \pagesLTS@putlabels%
  \typeout{^^J}%
  \ifodd\pagesLTS@tmpB%
    \@PackageInfoNoLine{pageslts}{Total number of pages is odd}%
  \else%
    \@PackageInfoNoLine{pageslts}{Total number of pages is even}%
  \fi%
  \typeout{^^J}%
%    \end{macrocode}
%
% We do not need the temporary definitions any more.
%
%    \begin{macrocode}
  \let\pagesLTS@tmpA\undefined%
  \let\pagesLTS@tmpB\undefined%
  }

%    \end{macrocode}
% \end{macro}
%
% \pagebreak
%
% \begin{macro}{\AtEndAfterFileList}
% \indent |\AtEndAfterFileList{...}| is even later:
% \begin{quote}
% \textquotedblleft After the .aux file closing and reading \LaTeX{} prints the
% file list if requested by |\listfiles|. Then this hook is executed.\textquotedblright
% \end{quote}
% (\xpackage{atveryend} package of \textsc{Heiko Oberdiek}, v1.7 as of 2011/04/23,
% newer version available.) Here it is used for a rerun hint.\\
% For example if the page numbering scheme of the last page of the
% \textsf{pageslts-example.tex} file is changed to |fnsymbol| and two runs
% of \pdfLaTeX{} are done, \pdfLaTeX{} will be happy and will not complain
% about changed labels. But indeed, a \textit{third} run is necessary
% and indicated by the warning message below.
%
%    \begin{macrocode}
\AtEndAfterFileList{%
  \ifx\pagesLTS@rerun\pagesLTS@one%
    \PackageWarningNoLine{pageslts}{%
      Label(s) may have changed.\MessageBreak%
      Rerun to get cross-references right%
     }%
  \fi%
  }

%    \end{macrocode}
% \end{macro}
%
% \begin{macro}{\frontmatter}
% \indent |\frontmatter| often contains |\pagenumbering|, but for some unknown reason
% there are problems when another |\pagenumbering| with different page numbering scheme
% has been used before on the same page. (This would not make any sense anyway,
% because one page can only have one page numbering scheme.) This problem does not
% occur when two |\pagenumbering| commands are used inside normal text.
% Thus we need to check whether |\frontmatter| has been defined,
% whether it changes the page numbering scheme, and whether the page numbering scheme
% before |\frontmatter| was initiated at the same page.
% \end{macro}
%
% We need an |\ifundefined|, which neither requires \eTeX{} nor
% defines the command, which is to be tested, to |\relax| (in which case
% it would be no longer undefined for further tests). For this we use
% code from \textsc{Markus Kohm} similar to his |\scr@ifundefinedorrelax| from %
% \xpackage{scrbase.sty} of the KOMA script bundle,
% but without the necessarity to use \eTeX :
%
%    \begin{macrocode}
\newcommand{\pagesLTS@ifundefinedorrelax}[1]{%
  % Modified Markus Kohm code for use without e-TeX
  \begingroup\expandafter\expandafter\expandafter\endgroup
  \expandafter\ifx\csname #1\endcsname\relax%
    \expandafter\ifx\csname #1\endcsname\relax%
      \expandafter\expandafter\expandafter\@firstoftwo
    \else%
      \expandafter\expandafter\expandafter\@secondoftwo
    \fi%
  \else%
    \expandafter\@firstoftwo
  \fi%
  }

\pagesLTS@ifundefinedorrelax{frontmatter}{% \relax
}{% else
  \ltx@GlobalPrependToMacro{\frontmatter}{\pagesLTS@prefrontmatter}%
  \ltx@GlobalAppendToMacro{\frontmatter}{\pagesLTS@postfrontmatter}%
 }

%    \end{macrocode}
%
% \begin{macro}{\pagesLTS@prefrontmatter}
%
% At the beginning of the real (i.e. unchanged) |\frontmatter| (FMB), we remember the
% current (c) page numbering (pn) scheme: its name (n) and page number (p).
% |\pagesLTS@FMBpncp| would be |1| if the according |\pagenumbering| command
% was used on the same page at the |\frontmatter|.
%
%    \begin{macrocode}
\newcommand{\pagesLTS@prefrontmatter}{%
  \xdef\pagesLTS@FMBpncn{\pagesLTS@pnc}%
  \xdef\pagesLTS@FMBpncp{\arabic{pagesLTS.current.local.\pagesLTS@pnc}}%
  }

%    \end{macrocode}
% \end{macro}
%
% \begin{macro}{\pagesLTS@postfrontmatter}
%
% The named checks are performed and in case of the named problem an error message
% is given. Any idea how to automatically solve this issue?
%
%    \begin{macrocode}
\newcommand{\pagesLTS@postfrontmatter}{%
  \xdef\pagesLTS@FMEpncn{\pagesLTS@pnc}%
  \ifx\pagesLTS@FMBpncn\pagesLTS@FMEpncn%
  \else%
    \ifx\pagesLTS@FMBpncn\pagesLTS@zero%
    \else%
      \ifx\pagesLTS@FMBpncp\pagesLTS@one%
        \PackageError{pageslts}{\string\pagenumbering\space before \string\frontmatter}{%
          Do not use \string\pagenumbering{\pagesLTS@FMBpncn} before \string\frontmatter \MessageBreak%
          on the same page!\MessageBreak%
          \string\frontmatter\space (re)defines the page numbering scheme to \pagesLTS@FMEpncn , \MessageBreak%
          thus earlier use of \string\pagenumbering{\pagesLTS@FMBpncn} on the same page is useless anyway.}%
      \fi%
    \fi%
  \fi%
}

%    \end{macrocode}
% \end{macro}
%
%    \begin{macrocode}
%</package>
%    \end{macrocode}
%
% \newpage
%
% \section{Installation}
%
% \subsection{Downloads\label{ss:Downloads}}
%
% Everything is available at \url{https://www.ctan.org},
% but may need additional packages themselves.\\
%
% \DescribeMacro{pageslts.dtx}
% For unpacking the |thumbs.dtx| file and constructing the documentation it is required:
% \begin{description}
% \item[-] \TeX Format \LaTeXe{}: \url{https://www.CTAN.org}
%
% \item[-] document class \xclass{ltxdoc}, 2015-03-26, 2.0w, \url{https://www.ctan.org/pkg/ltxdoc}
%
% \item[-] package \xpackage{holtxdoc}, 2012/03/21, v0.24, \url{https://www.ctan.org/pkg/holtxdoc}
%
% \item[-] package \xpackage{hypdoc}, 2011/08/19, v1.11, \url{https://www.ctan.org/pkg/hypdoc}
%
% \item[-] package \xpackage{geometry}, 2010/09/12, v5.6, \url{https://www.ctan.org/pkg/geometry}
%
% \item[-] package \xpackage{ulem}, 2012/05/18, no version number given, \url{https://www.ctan.org/pkg/ulem}
% \end{description}
%
% \DescribeMacro{pageslts.sty}
% The |pageslts.sty| for \LaTeXe{} (i.\,e. all documents using
% the \xpackage{pageslts} package) requires:
% \begin{description}
% \item[-] \TeX{} Format \LaTeXe{}, \url{https://www.CTAN.org}
%
% \item[-] package \xpackage{atveryend}, 2011/06/30, v1.8, \url{https://www.ctan.org/pkg/atveryend}
%
% \item[-] package \xpackage{everyshi}, 2001/05/15, v3.00, \url{https://www.ctan.org/pkg/everyshi}
%
% \item[-] package \xpackage{kvoptions}, 2011/06/30, v3.11, \url{https://www.ctan.org/pkg/kvoptions}
%
% \item[-] package \xpackage{letltxmacro}, 2010/09/02, v1.4, \url{https://www.ctan.org/pkg/letltxmacro}
%
% \item[-] package \xpackage{ltxcmds}, 2011/11/09, v1.22, \url{https://www.ctan.org/pkg/ltxcmds}
%
% \item[-] package \xpackage{rerunfilecheck}, 2011/04/15, v1.7, \url{https://www.ctan.org/pkg/rerunfilecheck}
%
% \item[-] package \xpackage{undolabl}, 2015/03/29, v1.0l, \url{https://www.ctan.org/pkg/undolabl}\\
% \end{description}
%
% \DescribeMacro{pageslts-example.tex}
% The |pageslts-example.tex| requires the same files as all
% documents using the \xpackage{pageslts} package, and additionally:
% \begin{description}
% \item[-] class \textsf{article}, 2014/09/29, v1.4h, from \textsf{classes.dtx}: \url{https://www.ctan.org/pkg/classes}
%
% \item[-] package \xpackage{alphalph}, 2011/05/13, v2.4, \url{https://www.ctan.org/pkg/alphalph}
%
% \item[-] package \xpackage{lipsum}, 2014/07/27, v1.3, \url{https://www.ctan.org/pkg/lipsum}
%
% \item[-] package \xpackage{showkeys}, 2014/10/28, v3.17, \url{https://www.ctan.org/pkg/showkeys}
%
% \item[-] package \xpackage{hyperref}, 2012/11/06, v6.83m, \url{https://www.ctan.org/pkg/hyperref}
%
% \item[-] package \xpackage{pageslts}, 2015/12/21, v1.2f, \url{https://www.ctan.org/pkg/pageslts}\\
%   (Well, it is the example file for this package, and because you are reading the
%    documentation for the \xpackage{pageslts} package, it can be assumed that you already
%    have some version of it~-- is it the current one?)
% \end{description}
%
% \pagebreak
%
% \DescribeMacro{papermas}
% The \xpackage{papermas} package is not required, but requires itself the
% the \xpackage{pageslts} package and can be considered as kind of add-on:
% \begin{description}
% \item[-] package \xpackage{papermas}, 2011/08/22, v1.0h, \url{https://www.ctan.org/pkg/papermas}
% \end{description}
%
% \DescribeMacro{endfloat}
% The \xpackage{endfloat} package is not required, but because
% the \xpackage{pageslts} package is incompatibel with very old versions
% of the \xpackage{endfloat} package (see subsection~\ref{sec:endfloat}),
% here the recent one is listed:
% \begin{description}
% \item[-] package \xpackage{endfloat}, 2011/12/25, v2.5d, \url{https://www.ctan.org/pkg/endfloat}
% \end{description}
%
% \DescribeMacro{prelim2e}
% The \xpackage{prelim2e} package is not required either, but because
% |Prelim@EveryShipout| code was taken from that package, it is listed, too:
% \begin{description}
% \item[-] package \xpackage{prelim2e}, 2009/05/29, v1.3, \url{https://www.ctan.org/pkg/prelim2e}
% \end{description}
%
% \DescribeMacro{fancyhdr}
% \DescribeMacro{nccfancyhdr}
% Neither \xpackage{fancyhdr} nor \xpackage{nccfancyhdr} package is required
% (the \xpackage{lastpage} package used its predecessor \xpackage{fancyheadings}),
% but because they were mentioned, also they are listed here:
% \begin{description}
% \item[-] package \xpackage{fancyhdr}, \url{https://www.ctan.org/pkg/fancyhdr}
%
% \item[-] package \xpackage{nccfancyhdr}, \url{https://www.ctan.org/pkg/nccfancyhdr}
% \end{description}
%
% \DescribeMacro{lastpage}
% \DescribeMacro{totpages}
% \DescribeMacro{totcount}
% \DescribeMacro{nofm}
% \DescribeMacro{count1to}
% \DescribeMacro{zref}
% As possible alternatives in section \ref{sec:Alternatives} there are listed
% \begin{description}
% \item[-] package \xpackage{lastpage}, 2015/03/29, v1.2m, \url{https://www.ctan.org/pkg/lastpage}
%
% \item[-] package \xpackage{totpages}, 2005/09/19, v2.00, \url{https://www.ctan.org/pkg/totpages}
%
% \item[-] package \xpackage{totcount}, 2011/01/25, v1.2, \url{https://www.ctan.org/pkg/totcount}
%
% \item[-] package \xpackage{nofm}, 1991/02/25, without version number,
%  \url{http://mirror.ctan.org/obsolete/macros/latex209/contrib/misc/nofm.sty},
%  does not work with e.\,g. \xpackage{hyperref}
%
% \item[-] package \xpackage{count1to}, 2009/05/24, v2.1, \url{https://www.ctan.org/pkg/count1to}
%
% \item[-] package \xpackage{zref}, 2012/04/04, v2.24, \url{https://www.ctan.org/pkg/zref}
% \end{description}
%
% \DescribeMacro{Oberdiek}
% \DescribeMacro{alphalph}
% \DescribeMacro{atveryend}
% \DescribeMacro{holtxdoc}
% \DescribeMacro{letltxmacro}
% \DescribeMacro{ltxcmds}
% \DescribeMacro{kvoptions}
% \DescribeMacro{rerunfilecheck}
% \DescribeMacro{zref}
% All packages of \textsc{Heiko Oberdiek's} bundle `oberdiek'
% (especially \xpackage{alphalph}, \xpackage{atveryend}, \xpackage{holtxdoc}, \xpackage{letltxmacro},
% \xpackage{ltxcmds}, \xpackage{kvoptions}, \xpackage{rerunfilecheck}, and \xpackage{zref})
% are also available in a TDS compliant ZIP archive:\\
% \url{http://mirrors.ctan.org/install/macros/latex/contrib/oberdiek.tds.zip}.\\
% It is probably best to download and use this, because the packages in there
% are quite probably both recent and compatible among themselves.
%
% \DescribeMacro{hyperref}
% \noindent \xpackage{hyperref} is not included in that bundle and needs to be
% downloaded separately,\\
% \url{http://mirrors.ctan.org/install/macros/latex/contrib/hyperref.tds.zip}.
%
% \DescribeMacro{M\"{u}nch}
% A hyperlinked list of my (other) packages can be found at
% \url{https://www.ctan.org/author/muench-hm}.
%
% \subsection{Package, unpacking TDS}
% \paragraph{Package.} This package is available on \url{CTAN.org}.
% \begin{description}
% \item[\url{http://mirrors.ctan.org/macros/latex/contrib/pageslts/pageslts.dtx}]\hspace*{0.1cm} \\
%       The source file.
% \item[\url{http://mirrors.ctan.org/macros/latex/contrib/pageslts/pageslts.pdf}]\hspace*{0.1cm} \\
%       The documentation.
% \item[\url{http://mirrors.ctan.org/macros/latex/contrib/pageslts/pageslts-example.pdf}]\hspace*{0.1cm} \\
%       The compiled example file, as it should look like.
% \item[\url{http://mirrors.ctan.org/macros/latex/contrib/pageslts/README}]\hspace*{0.1cm} \\
%       The README file.
% \end{description}
% There is also a |thumbs.tds.zip| available:
% \begin{description}
% \item[\url{http://mirror.ctan.org/install/macros/latex/contrib/pageslts.tds.zip}]\hspace*{0.1cm} \\
%       Everything in TDS compliant, compiled format.
% \end{description}
% which additionally contains\\
% \begin{tabular}{ll}
% pageslts.ins & The installation file.\\
% pageslts.drv & The driver to generate the documentation.\\
% pageslts.sty & The \xext{sty}le file.\\
% pageslts-example.tex & The example file.%
% \end{tabular}
%
% \bigskip
%
% \noindent For required other packages see the preceding subsection.
%
% \paragraph{Unpacking.} The \xfile{.dtx} file is a self-extracting
% \docstrip{} archive. The files are extracted by running the
% \xfile{.dtx} through \plainTeX :
% \begin{quote}
%   \verb|tex pageslts.dtx|
% \end{quote}
%
% About generating the documentation see paragraph~\ref{GenDoc} below.\\
%
% \paragraph{TDS.} Now the different files must be moved into
% the different directories in your installation TDS tree
% (also known as \xfile{texmf} tree):
% \begin{quote}
% \def\t{^^A
% \begin{tabular}{@{}>{\ttfamily}l@{ $\rightarrow$ }>{\ttfamily}l@{}}
%   pageslts.sty & tex/latex/pageslts.sty\\
%   pageslts.pdf & doc/latex/pageslts.pdf\\
%   pageslts-example.tex & doc/latex/pageslts-example.tex\\
%   pageslts-example.pdf & doc/latex/pageslts-example.pdf\\
%   pageslts.dtx & source/latex/pageslts.dtx\\
% \end{tabular}^^A
% }^^A
% \sbox0{\t}^^A
% \ifdim\wd0>\linewidth
%   \begingroup
%     \advance\linewidth by\leftmargin
%     \advance\linewidth by\rightmargin
%   \edef\x{\endgroup
%     \def\noexpand\lw{\the\linewidth}^^A
%   }\x
%   \def\lwbox{^^A
%     \leavevmode
%     \hbox to \linewidth{^^A
%       \kern-\leftmargin\relax
%       \hss
%       \usebox0
%       \hss
%       \kern-\rightmargin\relax
%     }^^A
%   }^^A
%   \ifdim\wd0>\lw
%     \sbox0{\small\t}^^A
%     \ifdim\wd0>\linewidth
%       \ifdim\wd0>\lw
%         \sbox0{\footnotesize\t}^^A
%         \ifdim\wd0>\linewidth
%           \ifdim\wd0>\lw
%             \sbox0{\scriptsize\t}^^A
%             \ifdim\wd0>\linewidth
%               \ifdim\wd0>\lw
%                 \sbox0{\tiny\t}^^A
%                 \ifdim\wd0>\linewidth
%                   \lwbox
%                 \else
%                   \usebox0
%                 \fi
%               \else
%                 \lwbox
%               \fi
%             \else
%               \usebox0
%             \fi
%           \else
%             \lwbox
%           \fi
%         \else
%           \usebox0
%         \fi
%       \else
%         \lwbox
%       \fi
%     \else
%       \usebox0
%     \fi
%   \else
%     \lwbox
%   \fi
% \else
%   \usebox0
% \fi
% \end{quote}
% If you have a \xfile{docstrip.cfg} that configures and enables \docstrip's
% TDS installing feature, then some files can already be in the right
% place, see the documentation of \docstrip.
%
% \subsection{Refresh file name databases}
%
% If your \TeX~distribution (\teTeX, \mikTeX,\dots) relies on file name
% databases, you must refresh these. For example, \teTeX\ users run
% \verb|texhash| or \verb|mktexlsr|.
%
% \subsection{Some details for the interested}
%
% \paragraph{Unpacking with \LaTeX.}
% The \xfile{.dtx} chooses its action depending on the format:
% \begin{description}
% \item[\plainTeX:] Run \docstrip\ and extract the files.
% \item[\LaTeX:] Generate the documentation.
% \end{description}
% If you insist on using \LaTeX\ for \docstrip\ (really,
% \docstrip\ does not need \LaTeX), then inform the autodetect routine
% about your intention:
% \begin{quote}
%   \verb|latex \let\install=y% \iffalse meta-comment
%
% File: pageslts.dtx
% Version: 2015/12/21 v1.2f
%
% Copyright (C) 2010 - 2015 by
%    H.-Martin M"unch <Martin dot Muench at Uni-Bonn dot de>
% Portions of code copyrighted by other people as marked.
%
% This package first started as a revision of the lastpage
% package of Jeffrey P. Goldberg (jeffrey+news at goldmark dot org),
% but it became obvious that a replacement was needed.
%
% This work may be distributed and/or modified under the
% conditions of the LaTeX Project Public License, either
% version 1.3c of this license or (at your option) any later
% version. This version of this license is in
%    http://www.latex-project.org/lppl/lppl-1-3c.txt
% and the latest version of this license is in
%    http://www.latex-project.org/lppl.txt
% and version 1.3c or later is part of all distributions of
% LaTeX version 2005/12/01 or later.
%
% This work has the LPPL maintenance status "maintained".
%
% The Current Maintainer of this work is H.-Martin Muench.
%
% This work consists of the main source file pageslts.dtx,
% the README, and the derived files
%    pageslts.sty, pageslts.pdf,
%    pageslts.ins, pageslts.drv,
%    pageslts-example.tex, pageslts-example.pdf.
%
% 'pageslts' is available on CTAN:
% https://www.ctan.org/pkg/pageslts
%
% Also a TDS.ZIP file is provided that contains all the files
% already sorted in a TDS tree:
% http://mirrors.ctan.org/install/macros/latex/contrib/pageslts.tds.zip
%
%<*ignore>
\begingroup
  \catcode123=1 %
  \catcode125=2 %
  \def\x{LaTeX2e}%
\expandafter\endgroup
\ifcase 0\ifx\install y1\fi\expandafter
         \ifx\csname processbatchFile\endcsname\relax\else1\fi
         \ifx\fmtname\x\else 1\fi\relax
\else\csname fi\endcsname
%</ignore>
%<*install>
\input docstrip.tex
\Msg{*********************************************************************************}
\Msg{* Installation}
\Msg{* Package: pageslts 2015/12/21 v1.2f Refers to special pages' numbers/names (HMM)}
\Msg{*********************************************************************************}

\keepsilent
\askforoverwritefalse

\let\MetaPrefix\relax
\preamble

This is a generated file.

Project: pageslts
Version: 2015/12/21 v1.2f

Copyright (C) 2010 - 2015 by
    H.-Martin M"unch <Martin dot Muench at Uni-Bonn dot de>
Portions of code copyrighted by other people as marked.

IMPORTANT NOTICE:
On the first page backslash pagenumbering should be called
(with the appropriate argument out of e.g.
 * roman    - Lowercase Roman numerals: i, ii, iii, iv,...
 * Roman    - Uppercase Roman numerals: I, II, III, IV,...
 * arabic   - Arabic numerals: 1, 2, 3, 4,...
 * alph     - Lowercase letters: a, b, c, d,...
 * Alph     - Uppercase letters: A, B, C, D,...
 * fnsymbol - Footnote symbols: *, † (ddagger), ‡ (dddagger),
                                 § (mathsection),...
). The command backslash pagenumbering is renewed, but no change
should be visible to the output text, because only labels are added.
backslash OrigPagenumbering can be used to get the original meaning
of backslash pagenumbering, but in that case this package does not
see the change in the page numbering scheme and will make mistakes.
The package takes options.

The usual disclaimer applies:
If it doesn't work right that's your problem.
(Nevertheless, send an e-mail to the maintainer
 when you find an error in this package.)

This work may be distributed and/or modified under the
conditions of the LaTeX Project Public License, either
version 1.3c of this license or (at your option) any later
version. This version of this license is in
   http://www.latex-project.org/lppl/lppl-1-3c.txt
and the latest version of this license is in
   http://www.latex-project.org/lppl.txt
and version 1.3c or later is part of all distributions of
LaTeX version 2005/12/01 or later.

This work has the LPPL maintenance status "maintained".

The Current Maintainer of this work is H.-Martin Muench.

This package first started as a revision of the lastpage
package of Jeffrey P. Goldberg (jeffrey+news at goldmark dot org),
but it became obvious that a replacement was needed.

This work consists of the main source file pageslts.dtx,
the README, and the derived files
   pageslts.sty, pageslts.pdf,
   pageslts.ins, pageslts.drv,
   pageslts-example.tex, pageslts-example.pdf.

In memoriam
 Claudia Simone Barth + 1996/01/30
 Tommy Muench + 2014/01/02
 Hans-Klaus Muench + 2014/08/24

\endpreamble
\let\MetaPrefix\DoubleperCent

\generate{%
  \file{pageslts.ins}{\from{pageslts.dtx}{install}}%
  \file{pageslts.drv}{\from{pageslts.dtx}{driver}}%
  \usedir{tex/latex/pageslts}%
  \file{pageslts.sty}{\from{pageslts.dtx}{package}}%
  \usedir{doc/latex/pageslts}%
  \file{pageslts-example.tex}{\from{pageslts.dtx}{example}}%
}

\catcode32=13\relax% active space
\let =\space%
\Msg{************************************************************************}
\Msg{*}
\Msg{* To finish the installation you have to move the following}
\Msg{* file into a directory searched by TeX:}
\Msg{*}
\Msg{*  pageslts.sty}
\Msg{*}
\Msg{* To produce the documentation run the file `pageslts.drv'}
\Msg{* through (pdf)LaTeX, e.g.}
\Msg{*  pdflatex pageslts.drv}
\Msg{*  makeindex -s gind.ist pageslts.idx}
\Msg{*  pdflatex pageslts.drv}
\Msg{*  makeindex -s gind.ist pageslts.idx}
\Msg{*  pdflatex pageslts.drv}
\Msg{*}
\Msg{* At least three runs are necessary e. g. to get the}
\Msg{*  references right!}
\Msg{*}
\Msg{* Happy TeXing!}
\Msg{*}
\Msg{************************************************************************}

\endbatchfile
%</install>
%<*ignore>
\fi
%</ignore>
%
% \section{The documentation driver file}
%
% The next bit of code contains the documentation driver file for
% \TeX , i.\,e., the file that will produce the documentation you
% are currently reading. It will be extracted from this file by the
% \texttt{docstrip} programme. That is, run \LaTeX{} on \texttt{docstrip}
% and specify the \texttt{driver} option when \texttt{docstrip}
% asks for options.
%
%    \begin{macrocode}
%<*driver>
\NeedsTeXFormat{LaTeX2e}[2015/01/01]
\ProvidesFile{pageslts.drv}[2015/12/21 v1.2f
            Refers to special pages' numbers/names (HMM)]
\documentclass[landscape]{ltxdoc}[2015/03/26]% v2.0w
\usepackage{geometry}[2010/09/12]%             v5.6
\usepackage{holtxdoc}[2012/03/21]%             v0.24
%% pageslts may work with earlier versions of LaTeX2e and those
%% class and packages, but this was not tested.
%% Please consider updating your LaTeX, class, and packages
%% to the most recent version (if they are not already the most
%% recent version).
\hypersetup{%
 pdfsubject={Refers to special pages' numbers/names (HMM)},%
 pdfkeywords={LaTeX, pageslts, pagesLTS, lastpage, page number, page name, H.-Martin Muench},%
 pdfencoding=auto,%
 pdflang={en},%
 breaklinks=true,%
 linktoc=all,%
 pdfstartview=FitH,%
 pdfpagelayout=OneColumn,%
 bookmarksnumbered=true,%
 bookmarksopen=true,%
 bookmarksopenlevel=2,%
 pdfmenubar=true,%
 pdftoolbar=true,%
 pdfwindowui=true,%
 pdfnewwindow=true%
}
\usepackage{ulem}[2012/05/18]% no version is given in the ulem.sty file
\CodelineIndex
\hyphenation{created every-thing ignored}
\gdef\unit#1{\mathord{\thinspace\mathrm{#1}}}
\makeatletter
\@ifundefined{eTeX}{\gdef\eTeX{$\m@th \varepsilon $-\TeX }}{% else \relax
}
\makeatother
\begin{document}
  \DocInput{pageslts.dtx}%
\end{document}
%</driver>
%    \end{macrocode}
%
% \fi
%
% \CheckSum{1790}
%
% \CharacterTable
%  {Upper-case    \A\B\C\D\E\F\G\H\I\J\K\L\M\N\O\P\Q\R\S\T\U\V\W\X\Y\Z
%   Lower-case    \a\b\c\d\e\f\g\h\i\j\k\l\m\n\o\p\q\r\s\t\u\v\w\x\y\z
%   Digits        \0\1\2\3\4\5\6\7\8\9
%   Exclamation   \!     Double quote  \"     Hash (number) \#
%   Dollar        \$     Percent       \%     Ampersand     \&
%   Acute accent  \'     Left paren    \(     Right paren   \)
%   Asterisk      \*     Plus          \+     Comma         \,
%   Minus         \-     Point         \.     Solidus       \/
%   Colon         \:     Semicolon     \;     Less than     \<
%   Equals        \=     Greater than  \>     Question mark \?
%   Commercial at \@     Left bracket  \[     Backslash     \\
%   Right bracket \]     Circumflex    \^     Underscore    \_
%   Grave accent  \`     Left brace    \{     Vertical bar  \|
%   Right brace   \}     Tilde         \~}
%
% \GetFileInfo{pageslts.drv}
%
% \begingroup
%   \def\x{\#,\$,\^,\_,\~,\ ,\&,\{,\},\%}%
%   \makeatletter
%   \@onelevel@sanitize\x
% \expandafter\endgroup
% \expandafter\DoNotIndex\expandafter{\x}
% \expandafter\DoNotIndex\expandafter{\string\ }
% \begingroup
%   \makeatletter
%     \lccode`9=32\relax
%     \lowercase{%^^A
%       \edef\x{\noexpand\DoNotIndex{\@backslashchar9}}%^^A
%     }%^^A
%   \expandafter\endgroup\x
%
% \DoNotIndex{\",\-,\,,\\,\noindent}
% \DoNotIndex{\documentclass,\usepackage,\ProvidesPackage}
% \DoNotIndex{\NeedsTeXFormat,\TeX,\LaTeX}
% \DoNotIndex{\ldots,\thinspace,\textbackslash}
% \DoNotIndex{\begin,\end}
% \DoNotIndex{\textbf,\textit,\textquotedblleft,\textquotedblright,\textsc,\textsf,\texttt,\underline}
% \DoNotIndex{\ifx,\ifnum,\ifodd,\enduremath}
% \DoNotIndex{\newpage,\pagebreak,\clearpage,\newline,\linebreak,\nolinebreak}
% \DoNotIndex{\lipsum,\MessageBreak,\pageref,\protect}
% \DoNotIndex{\smallskip,\medskip,\bigskip,\hskip,\space,\hline}
% \DoNotIndex{\advance,\@ne}
% \DoNotIndex{\item}
% \DoNotIndex{\@PackageInfoNoLine,\@bsphack,\@esphack,\@ehc,\@ehd,\@ifundefined,%
%  \addcontentsline,\arabic,\enddocument,\endinput,\href,\makeatletter,\makeatother,%
%  \section,\subsection,\tableofcontents,\verb}
%
% \title{The \xpackage{pageslts} package}
% \date{2015/12/21 v1.2f}
% \author{H.-Martin M\"{u}nch\\\xemail{Martin.Muench at Uni-Bonn.de}}
%
% \maketitle
%
% \begin{abstract}
%  \noindent This \LaTeX{} package puts the labels \texttt{LastPage}
%  (|\AtEndDocument|) and \texttt{VeryLastPage} (|\AfterLastShipout|)
%  into the \xfile{.aux} file, allowing the user to refer
%  to the (very) last page of a document. This might be
%  particularly useful in places like headers or footers.
%  When more than one page numbering scheme is used,
%  these references do not give the total \textit{number} of pages.
%  For this case the label \texttt{LastPages} is introduced.
%  Additionally, at the last page of each page numbering
%  scheme a label \texttt{pagesLTS.<numbering scheme>} is
%  placed, where \texttt{<numbering scheme>} is e.\,g.
%  arabic, roman, Roman, alph, or Alph. For fnsymbol
%  please use |\lastpageref{pagesLTS.fnsymbol}|
%  instead of |\pageref{pagesLTS.fnsymbol}|.
%  When the same numbering scheme is used twice,
%  the page numbers are either reset to one or continued
%  automatically, depending on the option given when the package
%  is called. The command |\theCurrentPage|
%  prints the current total/absolute page number -- in contrast to
%  |\thepage|, which gives only the page
%  \textit{name} in the current page numbering scheme.
%  |\theCurrentPageLocal| gives the current
%  number of pages in the current page numbering scheme.
%  |\thepage| and |\theCurrentPageLocal| are different e.\,g.
%  when |\addtocounter{page}{|\ldots|}| or
%  |\setcounter{page}{|\ldots|}| were used.
%  At the first page of the document a label \texttt{pagesLTS.0} is
%  created. This label can be referred to, too.
%  Further labels are provided for special cases.\\
%  The \xpackage{alphalph} package is supported, i.\,e.
%  page numbers alph or Alph $>26$ and fnsymbol $>9$
%  can be used (with according options set). Even zero
%  and negative page numbers can be used with \texttt{arabic},
%  \texttt{alph}, \texttt{Alph}, \texttt{roman}, \texttt{Roman},
%  and \texttt{fnsymbol} page numbering (with \xpackage{alphalph} package
%  and according options).\\
%  |\pageref*| and |\lastpageref*|, for using \xpackage{hyperref}
%  but suppressing links, are supported.
% \end{abstract}
%
% \noindent \textbf{Please make sure to first deinstall the obsolete
% \xpackage{pagesLTS} package before installing this \xpackage{pageslts}
% package!}\\
% (There is at least one operating system which otherwise automatically
% renames \xfile{pageslts} to \xfile{pagesLTS}.)
%
% \noindent Right after |\begin{document}| a |\pagenumbering{|\ldots|}| should
% be called -- with the appropriate argument out of e.\,g.\\
% arabic (Arabic numerals: 1, 2, 3, 4,\ldots),\\
% roman (Lowercase Roman numerals: i, ii, iii, iv,\ldots),
% Roman (Uppercase Roman numerals: I, II, III, IV,\ldots),\\
% alph (Lowercase letters: a, b, c, d,\ldots),
% Alph (Uppercase letters: A, B, C, D,\ldots),\\
% fnsymbol (Footnote symbols: $*$, $\dagger $, $\ddagger $, $\mathsection $,\ldots).
%
% \noindent This package first started as a revision of the
% \xpackage{lastpage} package of \textbf{Jeffrey P. Goldberg} (Thanks!),
% but then it became obvious that a replacement was needed.
%
% \noindent Disclaimer for web links: The author is not responsible for any contents
% referred to in this work unless he has full knowledge of illegal contents.
% If any damage occurs by the use of information presented there, only the
% author of the respective pages might be liable, not the one who has referred
% to these pages.
%
% \noindent Save per page about $200\unit{ml}$ water,
% $2\unit{g}$ CO$_{2}$ and $2\unit{g}$ wood:
% Therefore please print only if this is really necessary.
%
% \bigskip
%
% \tableofcontents
%
% \newpage
%
% \section{Introduction}
%
% \indent This package puts the labels \texttt{LastPage}
% (|\AtEndDocument|) (same as my \xpackage{LastPage} package, invented by
% \textsc{Jeffrey P. Goldberg}) and \texttt{VeryLastPage} (|\AfterLastShipout|)
% into the \xfile{.aux} file, allowing the user to refer to the (very)
% last page of a document via |\lastpageref{LastPage}| and
% |\lastpageref{VeryLastPage}|. This might be particularly useful in
% places like headers or footers. When more than one page numbering scheme
% is used, these references do not give the total \textit{number} of pages.
% For this case the label \texttt{LastPages} is introduced (similar to the
% label \texttt{TotPages} of the \xpackage{TotPages} package, but
% the label \texttt{LastPages} is set later in the document).
% Additionally, at the last page of each page numbering scheme a label
% \texttt{pagesLTS.<}\textit{numbering scheme}\texttt{>} is placed,
% where \texttt{<}\textit{numbering scheme}\texttt{>} is e.\,g.
% arabic, roman, Roman, alph, or Alph. For fnsymbol please use
% |\lastpageref{pagesLTS.fnsymbol}| instead of |\pageref{pagesLTS.fnsymbol}|.
% When the same numbering scheme is used twice, the page numbers are
% either reset to one or continued automatically, depending on the option
% given when the package is called. The command |\theCurrentPage| prints
% the current total/absolute page number - in contrast to |\thepage|,
% which gives only the page \textit{name} in the current page numbering
% scheme. |\theCurrentPageLocal| gives the current number of pages
% in the current page numbering scheme. |\thepage| and |\theCurrentPageLocal|
% are different e.\,g. when |\addtocounter{page}{|\ldots|}| or
% |\setcounter{page}{|\ldots|}| were used. (See also \LaTeX\ bug~3421:
% 3rd~page is even (twoside, titlepage, abstract),
% \url{http://www.latex-project.org/cgi-bin/ltxbugs2html?category=LaTeX&responsible=anyone&state=anything&keyword=pagenumber&pr=latex%2F3421&search=}.)
% At the first page of the document a label \texttt{pagesLTS.0} is
% created. This label can be referred to, too. Further labels are
% provided for special cases.\\
% The \xpackage{alphalph} package is supported, i.\,e. page numbers
% \texttt{alph} or \texttt{Alph} $>26$ and \texttt{fnsymbol} $>9$ can
% be used (with the according options set). Even zero or negative page
% numbers can be used with \texttt{arabic}, \texttt{alph}, \texttt{Alph}, and
% \texttt{fnsymbol} page numbering (with \xpackage{alphalph} package
% and according options), and zero \texttt{roman} and \texttt{Roman} pages,
% too.
%
% \bigskip
% \noindent Right after |\begin{document}| a |\pagenumbering{|\ldots|}|
% should be called -- with the appropriate argument out of e.\,g.\\
% arabic (Arabic numerals: 1, 2, 3, 4,\ldots),\\
% roman (Lowercase Roman numerals: i, ii, iii, iv,\ldots),
% Roman (Uppercase Roman numerals: I, II, III, IV,\ldots),\\
% alph (Lowercase letters: a, b, c, d,\ldots),
% Alph (Uppercase letters: A, B, C, D,\ldots),\\
% fnsymbol (Footnote symbols: $*$, $\dagger $, $\ddagger $, $\mathsection $,\ldots).
%
% \bigskip
% \noindent This package first started as a revision of the
% \xpackage{lastpage} package of \textsc{Jeffrey P. Goldberg} (Thanks!),
% but then it became obvious that a replacement was needed to accomplish
% what this package does.
%
% \bigskip
% \noindent \textbf{Trademarks} appear throughout this documentation
% without any trademark symbol; they are the property of their respective
% trademark owner. There is no intention of infringement; the usage is
% to the benefit of the trademark owner.
%
% \bigskip
% \phantomsection
% \DescribeMacro{logical page numbers}
% \noindent \textbf{Tip}: For the display of the \xfile{pdf}~file use
% \textbf{logical page numbers} together with \xpackage{hyperref}!
% \label{logical page numbers}
% \begin{description}
% \item [-] In Adobe Reader DC 2015.008.20082 enable:\\
% Edit $>$ Preferences $>$ Categories: Page Display $>$
% Page Content and Information: Use logical page numbers
%
% \item[-] Use the \xpackage{hyperref} package with option |plainpages=false|\,.
% \end{description}
% The display will be e.\,g.\ \textquotedblleft 7~(7~of~9)\textquotedblright,
% or, in case of Roman instead of arabic numbers,
% \textquotedblleft VII~(7~of~9)\textquotedblright, and when different
% page numbers are used (see below) e.\,g. arabic after 10~Roman pages:
% \textquotedblleft 17~(27~of~30)\textquotedblright.
% Please try this with the compiled \xpackage{pageslts-example} file!
%
% \bigskip
% \noindent The name of the \xpackage{pageslts} package refers to
% Last, Total, and page numbering Schemes pages. \xpackage{pagesLTS}
% was a former name of this package.
%
% \pagebreak
%
% \section{Usage}
%
% \indent Just load the package placing
% \begin{quote}
%   |\usepackage[<|\textit{options}|>]{pageslts}|
% \end{quote}
% \noindent in the preamble of your \LaTeXe\ source file
% (about |\AtEndDocument| see subsection~\ref{ssec:aed})
% and place a |\pagenumbering{...}| with appropriate argument
% (e.\,g.~arabic, roman, Roman, fnsymbol, alph, or Alph)
% right behind |\begin{document}| (see subsubsection~\ref{sss:nopnc})!\\
%
% \indent For example for various draft forms it is desirable to have a
% page reference to the last page, so that e.\,g. page footers can
% contain something like \textquotedblleft page $N$ of $K$\textquotedblright,
% where $N$ is the current page and $K$ is the last page. Once the package
% is loaded, anywhere in the text references can be made to the labels
% \texttt{LastPage}, \texttt{VeryLastPage}, and \texttt{LastPages}
% (most times with |\pageref{...}|, but more save with |\lastpageref{...}|).
% In particular one can use the \xpackage{fancyhdr} or \xpackage{nccfancyhdr}
% package, or redefinitions of the page headings and footings to get
% a reference to the (very) last page.\\
% \DescribeMacro{\pageref*}
% \DescribeMacro{\lastpageref*}
% If the \xpackage{hyperref} package is used, the references are hyperlinked
% to their aims. If these hyperlinks shall be suppressed, |\pageref*{...}|
% and |\lastpageref*{...}| can be used.
%
% \subsection{Options}
% \DescribeMacro{options}
% \indent The \xpackage{pageslts} package takes the following options:
%
% \subsubsection{pagecontinue\label{sss:pagecontinue}}
% \DescribeMacro{pagecontinue}
% \indent When option \texttt{pagecontinue=false} is \textbf{not} given
% (i.\,e. \texttt{pagecontinue} or \texttt{pagecontinue=true}
% or no \texttt{pagecontinue} option at all), at each
% |\pagenumbering{...}| command the number of the page will be continued
% with the page number following the last page of the same page numbering
% scheme. For example, if there are V~Roman pages in the frontmatter,
% some arabic ones in the mainmatter, and then Roman ones again in the
% backmatter, the last ones will start with VI instead of~I again.\\
% \indent If you want to start with I (or i, 1, a, A, *,\ldots ) again,
% set option \texttt{pagecontinue=false}. If you want to generally continue
% the numbers, but for some page numbering scheme do not want this,
% use \texttt{pagecontinue=true} and say |\setcounter{page}{1}|
% after |\pagenumbering{...}| for that page numbering scheme.
%
% \subsubsection{alphMult, AlphMulti, fnsymbolmult\label{sss:alphalph}}
%
% \indent The page number printed in
% \nolinebreak{\texttt{fnsymbol}\footnote{%
% \ensuremath {*},%
% \ensuremath {\dagger },%
% \ensuremath {\ddagger },%
% \ensuremath {\mathsection },%
% \ensuremath {\mathparagraph },%
% \ensuremath {\delimiter "026B30D },%
% \ensuremath {**},%
% \ensuremath {\dagger \dagger },%
% \ensuremath {\ddagger \ddagger }}}
% must be $>0$ and $<10$ and those printed in
% \nolinebreak{\texttt{alph}\footnote{a, b, c, d, e, f, g, h,%
% i, j, k, l, m, n, o, p, q, r, s, t, u, v, w, x, y, z}} and
% \nolinebreak{\texttt{Alph}\footnote{A, B, C, D, E, F, G, H,%
% I, J, K, L, M, N, O, P, Q, R, S, T, U, V, W, X, Y, Z}} must
% be $>0$ and $<27$. After page~Z\quad \LaTeX\ \textit{should}
% continue with AA, AB, AC,\ldots\ Some people prefer
% AA, BB, CC,\ldots, but in hexadecimal it is $AA_{16}=170_{10}$
% and $171_{10}=AB_{16}$, whereas $BB_{16}=187_{10}$. In any way
% it should continue at all (maybe even with an user option to
% choose between the two continuations), but instead only gives
% an error:
% \begin{quote}
%   |LaTeX Error: Counter too large|\\
%   |See the LaTeX manual or LaTeX Companion for explanation.|\\
%   |You've lost some text.  Try typing  <return>  to proceed.|\\
%   |If that doesn't work, type  X <return>  to quit.|
% \end{quote}
% But thanks to the \xpackage{alphalph} package by \textsc{Heiko Oberdiek}
% these limitation no longer hold. With his |\erroralph| command
% now even negative or zero page
% \textquotedblleft numbers\textquotedblright{} are possible.\\
%
% \DescribeMacro{alphMult}
% The string option \texttt{alphMult} takes three values: |ab|, |bb|, |0|:
% \begin{description}
% \item[ab] After page z, the page
%  \textquotedblleft numbers\textquotedblright{} continue with
%  aa, ab, ac, ad,\ldots, fxshrxw (the default), and before a with
%  0, -a, -b,\ldots, -z, -aa, -ab,\ldots, -fxshrxw ($=-21\,474\,836\,47$).
% \item[bb] After page z, the page
%  \textquotedblleft numbers\textquotedblright{} continue with
%  aa, bb, cc, dd,\ldots, and before a with
%  0, -a, -b,\ldots, -z, -aa, -bb,\ldots\\
%  (Internally up~to $\pm 55\,834\,558$ is allowed, but when printed will
%  exceed the \LaTeX\ capacity even for smaller numbers --
%  in the example file this happens at about $6\,500$.)\\
%  (If you have a document with more than $6\,500$~pages, you might
%   think about splitting it in volumes. And page
%   \textquotedblleft numbers\textquotedblright{} with about 100~digits
%   are probably not easy to grasp for the reader, too.)\\
% \item[0] (zero) The \xpackage{pageslts} package does nothing,
%  thus the user is free to define the page
%  \textquotedblleft numbers\textquotedblright{} after~z and before~a.\\
%  (But if the user does not do anything at all, the\\
%  \ |LaTeX Error: Counter too large|\\
%  will appear again.)
% \end{description}
%
% \DescribeMacro{AlphMulti}
% The string option \texttt{AlphMulti} takes three values: |AB|, |BB|, |0|:
% \begin{description}
% \item[AB] After page Z, the page
%  \textquotedblleft numbers\textquotedblright{} continue with
%  AA, AB, AC, AD,\ldots, FXSHRXW (the default), and before A with
%  0, -A, -B,\ldots, -Z, -AA, -AB,\ldots, -FXSHRXW.
% \item[BB] After page Z, the page
%  \textquotedblleft numbers\textquotedblright{} continue with
%  AA, BB, CC, DD,\ldots, and before A with
%  0, -A, -B,\ldots, -Z, -AA, -BB,\ldots \\
%  (About the limits please see |alphMult| above.)
% \item[0] (zero) The \xpackage{pageslts} package does nothing,
%  thus the user is free to define the page
%  \textquotedblleft numbers\textquotedblright{} after~Z and before~A.\\
%  (But if the user does not do anything at all, the\\
%  \ |LaTeX Error: Counter too large|\\
%  will appear again.)
% \end{description}
%
% \DescribeMacro{fnsymbolmult}
% When option \texttt{fnsymbolmult=false} is \textbf{not} given
% (i.\,e. \texttt{fnsymbolmult} or \texttt{fnsymbolmult=true}
% or no \texttt{fnsymbolmult} option at all),
% after 5 (\ensuremath {\mathparagraph }) the page
% \textquotedblleft number\textquotedblright{} is continued with
% the doubled \textquotedblleft number\textquotedblright{} of the
% first, second, third,\ldots\ page (\ensuremath {**},
% \ensuremath {\dagger \dagger }, \ensuremath {\ddagger \ddagger },
% \ensuremath {\mathsection \mathsection },
% \ensuremath {\mathparagraph \mathparagraph }),
% and after the tenth page the
% \textquotedblleft number\textquotedblright{} is tripled
% (\ensuremath {***}, \ensuremath {\dagger \dagger \dagger },\ldots).
% Compile the \textsf{pageslts-example.tex} with \pdfLaTeX\ and
% see the resulting \xfile{pdf} file.\\
% Before \ensuremath {*} (page~1) the page
% \textquotedblleft numbers\textquotedblright{} are continued with
% 0, \ensuremath {- * }, \ensuremath {- \dagger },\ldots,
% \ensuremath {- \mathparagraph }, \ensuremath {- **},
% \ensuremath {- \dagger \dagger },\ldots \\
% If this is not wanted, set option \texttt{fnsymbolmult=false},
% and \xpackage{pageslts} will do nothing and allow the user to change
% the page \textquotedblleft number\textquotedblright . (But if the
% user does not do anything at all, the\\
% \ |LaTeX Error: Counter too large|\\
% will appear again.)\\
% While in \LaTeXe\ arabic (page) numbers are possible up~to
% |MAX|\,$ = 2\,147\,483\,647$ (cf.~the \xpackage{alphalph} package),\\
% |\erroralphalph{\fnsymbolmult}{...}| numbers are possible up~to
% 10\,737\,415 only. If this number is not only used internally
% but printed, after number about 11\,705 (which is
% 2\,341 times \ensuremath {\mathparagraph }) the \LaTeXe\ capacity
% is exceeded, depending on the remaining file and its use of
% \TeX\ capacity, of course. (If you have a document with more than
% 11\,705~pages, you might think about splitting it in volumes.
% And page \textquotedblleft numbers\textquotedblright{} with
% 2\,341~digits are probably not easy to grasp for the reader, too.)\\
%
% \subsubsection{romanMult, RomanMulti\label{sss:romanMult}}
%
% \DescribeMacro{romanMult}
% \DescribeMacro{RomanMulti}
% The options \texttt{romanMult(=true)} and \texttt{RomanMulti(=true)}
% expand the |\roman| and |\Roman| page numbering scheme to values
% below one ($<1$), i.\,e. 0, -i, -ii, -iii, -iv,\ldots\ and
% 0, -I, -II, -III, -IV,\ldots, respectively.\\
% Again the \TeX\ capacity will be exceeded before
% \nolinebreak{$\pm$ |MAX|\,}$ = \pm 2\,147\,483\,647$, and even if
% $1\,000\,000\,000$ is internally possible, this would print
% $1\,000\,000$ times the letter~\textsf{m} (or~\textsf{M}),
% which would require either very small print or quite huge paper size.\\
% (If you have a document with so many pages, you might think
% about splitting it in volumes. And page
% \textquotedblleft numbers\textquotedblright{} with
% thousands of digits are probably not easy to grasp for the reader,
% too.)\\
% If the expansion below $1$ is not wanted, set options
% \texttt{romanMult=false} and/or \texttt{RomanMulti=false},
% and \xpackage{pageslts} will do nothing and allow the user
% to change the page \textquotedblleft number\textquotedblright .
% (But if the user does not do anything at all, \LaTeX\ will just
% ignore the values - not even a warning will be issued.)
%
% \subsubsection{Arabic page numbers}
%
% \DescribeMacro{Arabic page numbers}
% In \LaTeXe\ arabic (page) numbers are already possible between
% |-MAX...MAX|, where |MAX|\,$ = 2\,147\,483\,647$ (cf.~the
% \xpackage{alphalph} package), without any expansion necessary.
% (But if you have a document with so many pages, you might think
% about splitting it in volumes!)
%
% \subsection{Labels}
%
% \DescribeMacro{pagesLTS.0}
% \indent At the first page a label \texttt{pagesLTS.0} is created.
% If |\pagenumbering{...}| is used right after
% |\begin{document}|, this is much easier for the
% \xpackage{pageslts} package (and chances for successful placing
% of all labels are much higher;
% \nolinebreak{cf.~subsubsection~\ref{sss:once}).}\\
%
% \DescribeMacro{LastPage}
% |\AtEndDocument| (see subsection~\ref{ssec:aed}) this package defines
% a label, \texttt{LastPage}, which the user can refer to with the\linebreak
% |\lastpageref{LastPage}| command. While |\pageref{LastPage}| is also
% possible (especially for backward compatibility with the \xpackage{LastPage}
% package), this is discouraged, because it will not work when it is used
% together with the \xpackage{hyperref} package and the \xpackage{fnsymbol}
% page numbering scheme. (The \xpackage{LastPage} package did not work with
% this combination, too, so if you want to, you can reproduce the old error
% -- but you do not have to do it, but can use |\lastpageref{LastPage}|.)\\
%
% \DescribeMacro{VeryLastPage}
% |\AfterLastShipout| the label \texttt{VeryLastPage} is defined, which
% the user can also refer to with the |\lastpageref{VeryLastPage}| command.
% Depending on usage of |\AtEndDocument| by other packages, \texttt{LastPage}
% might not point to the very last page, but |\lastpageref{VeryLastPage}|
% should do this (cf.~subsection~\ref{ssec:aed}).\\
%
% \DescribeMacro{LastPages}
% \DescribeMacro{page number}
% \DescribeMacro{number of pages}
% When more than one page numbering scheme is used, neither \texttt{LastPage}
% nor \texttt{VeryLastPage} give the total \textbf{number} of pages.
% For example, for a document with VI+36 pages, both give
% \textquotedblleft 36\textquotedblright{} as reference to the last page.
% While this is correct, the total number of pages is 42, and this is given
% by the reference to \texttt{LastPage\underline{s}}:
% \nolinebreak{|\lastpageref{LastPage|\underline{|s|}|}|} (note the
% \textquotedblleft s\textquotedblright{} at the end). When the
% page number was manipulated by |\addtocounter{page}{...}|
% or |\setcounter{page}{...}|, \texttt{LastPages} ignores this.
% (At a page numbering change the page is reset to one (without option
% |pagecontinue|). This is done by |\setcounter{page}{1}|, thus this is
% ignored, too.)\\
% |\pageref{totpages}| of the \xpackage{totpages} package is similar to
% |\lastpageref{LastPages}|, but while the target for \linebreak[4]
% |\pageref{totpages}| is placed |\AtEndDocument|, the target for
% |\lastpageref{LastPages}| is placed |\AfterLastShipout|, therefore
% |\lastpageref{LastPages}| is safer to really get the total page number.\\
%
% \pagebreak
%
% \DescribeMacro{\theCurrentPage}
% |\theCurrentPage| gives the current total/absolute page, in contrast to
% |\thepage|, which gives only the page \textit{name} in the current page
% numbering scheme. For example, when there are Roman VII pages in the
% frontmatter and afterwards in the mainmatter you are at arabic page~9,
% then |\theCurrentPage| is~16, whereas |\thepage| is~9. When the page
% \textquotedblleft number\textquotedblright{} (name) is manipulated by
% |\addtocounter{page}{...}| or |\setcounter{page}{...}|, |\theCurrentPage|
% \nolinebreak{ignores} this. Because \texttt{CurrentPage} is a normal
% counter, you can also say e.\,g.~|\Roman{CurrentPage}| to get the value
% in Roman page numbering scheme (e.\,g.~VIII for~8).\\
%
% \phantomsection
% \label{CurrentPageLocal}
% \DescribeMacro{\theCurrentPageLocal}
% |\theCurrentPageLocal| gives the current (arabic) number of pages in the
% current page numbering scheme. |\thepage| and |\theCurrentPageLocal| are
% different e.\,g. when |\addtocounter{page}{...}| or
% |\setcounter{page}{...}| were used.\\
% |\theCurrentPageLocal| can be printed in other formats, e.\,g.
% |\roman{pagesLTS.current.local.roman}|, but probably it only makes
% sense if page numbering scheme and format are the same, e.\,g.
% |\Roman{pagesLTS.current.local.Roman}| or\\
% |\Alph{pagesLTS.current.local.Alph}|.\\
% |\arabic{pagesLTS.current.local.|\ldots|}| probably make sense even
% when combined with another page numbering scheme. And this is exactly
% what |\theCurrentPageLocal| does:\\
% |\def\theCurrentPageLocal{\arabic{pagesLTS.current.local.\pagesLTS@pnc}}|.\\
%
% \DescribeMacro{pagesLTS. page numbering scheme . number }
% If you want to refer to the last page of the first, second,\ldots\ use
% of a page numbering scheme, you can refer to \linebreak
% |pagesLTS.<page numbering scheme>.<number>|, e.\,g.
% |\lastpageref{pagesLTS.Roman.1}|, where |<number>| is the occurrence
% of the page numbering scheme. For details please see
% page~\pageref{pagesLTS.pnscheme.nr}.\\
%
% \bigskip
% \DescribeMacro{\lastpageref}
% For pages with the \texttt{fnsymbol} page numbering scheme,
% |\lastpageref{...}| \nolinebreak{instead} of |\pageref{...}|
% \textit{must} be used. This is required for pages some\-where
% inside of the document as well as the (very) last page(s).
% Because |\lastpageref{...}| is a synonym for |\pageref{...}|,
% where no \xpackage{fnsymbol} page numbering scheme is used,
% it is save(r) to use it for all references to labels provided
% by the \xpackage{pageslts} package.
%
% \subsection[\texttt{\textbackslash pagenumbering\{\ldots \}}]{%
% \texttt{\textbackslash pagenumbering\{\ldots \}}}
% \DescribeMacro{\pagenumbering}
% \subsubsection{If \texttt{\textbackslash pagenumbering\{\ldots \}} is \textbf{not} used\label{sss:nopnc}}
%
% \indent When the \xpackage{pageslts} package is used, but |\pagenumbering{...}|
% (with an argument like arabic, roman, Roman, fnsymbol, alph, or Alph)
% is not used, there should be no problem, except that you might need
% more~(!) compiler runs to get all references right, and some references
% might even be missing (see below). The \xpackage{pageslts}
% package tries to determine the page numbering scheme at the first
% shipout, but success is not guaranteed. Thus please use |\pagenumbering{...}|
% at the beginning of your document!\\
% \indent Without |\pagenumbering{<something>}| (|<something>| e.\,g. = arabic)
% at the beginning of the document, the page numbers might be given in
% arabic \textit{by} (class) \textit{default}, but the \xpackage{pageslts}
% package does not know about this without |\pagenumbering{arabic}|.~--\\
% \indent The label \texttt{pagesLTS.0} is created at the first
% page even if no |\pagenumbering{...}| command is given.
% Maybe have a look at the \xfile{.aux} file after compiling your
% document to detect further labels (of other packages, too).\\
%
% \pagebreak
%
% \subsubsection{If \texttt{\textbackslash pagenumbering\{\ldots \}} is used once\label{sss:once}}
%
% \DescribeMacro{pagesLTS.0}
% At the first page a label \texttt{pagesLTS.0} is created.
% If |\pagenumbering{...}| is used right after |\begin{document}|,
% this is much easier for the \xpackage{pageslts} package
% (and chances for successful placing of all labels are much higher).
%
% \subsubsection{If \texttt{\textbackslash pagenumbering\{\ldots \}} is used more than once}
%
% \indent Everything from the preceding subsubsections applies and additionally
% the following:\\
% \indent When different page numbering schemes are used, e.\,g.
% Roman numbers for the frontmatter and arabic numbers for the
% mainmatter, please use |\pagenumbering{...}| for each of them!
% Even if you do this, the reference to neither the label \texttt{LastPage}
% nor the label \texttt{VeryLastPage} gives the \textbf{total}
% number of pages, but only the number of pages of the last used page
% numbering scheme (which could be exactly what you want, e.\,g. if
% you want to refer to the last page itself and do not want to
% give the total number of pages).\\
% \DescribeMacro{LastPages}
% \indent For remediation the label
% \nolinebreak{\texttt{LastPage}\textbf{\underline{s}}}
% (with \textquotedblleft s\textquotedblright{} at its end) is
% introduced. Please then refer to this label by \newline
% |\lastpageref{LastPages}| instead of \texttt{LastPage} or
% \texttt{VeryLastPage}.\\
%
% \DescribeMacro{pagesLTS.arabic}
% \DescribeMacro{pagesLTS.roman}
% \DescribeMacro{pagesLTS.Roman}
% \DescribeMacro{pagesLTS.alph}
% \DescribeMacro{pagesLTS.Alph}
% \DescribeMacro{pagesLTS.fnsymbol}
% Additionally, at the last page of each page numbering scheme a label
% \texttt{pagesLTS.<numbering scheme>} is placed, where
% \texttt{<numbering scheme>} is e.\,g.\ arabic, roman, Roman, alph,
% Alph,\ldots .\\
% For the \textbf{fnsymbol} page numbering scheme
% |\lastpageref{pagesLTS.fnsymbol}| is needed instead of \newline
% |\pageref{pagesLTS.fnsymbol}|. You can and should use
% |\lastpageref{...}| also for the other page numbering schemes.\\
%
% \bigskip
%
% While at the time of the last revision of the \xpackage{pageslts} package
% no other page numbering schemes were known to the maintainer,
% this package in principle works with every scheme which is
% recognized by the original |\pagenumbering| command.
% But the \xpackage{hyperref} package only then works with crazy page names,
% if the references to those pages are given in a certain way, thus the
% combination of a new page numbering scheme, the \xpackage{hyperref}
% and the \xpackage{pageslts} package might not work.~--
% The \xpackage{pageslts} package by itself also works with schemes,
% which the original |\pagenumbering{...}| does not recognize, but because
% the original |\pagenumbering{...}| is called by the \xpackage{pageslts}
% package, this might cause an error, see subsection~\ref{sec:unknown}!
% (And if the number format is unknown to \LaTeX{}, the pages will have
% no number, and therefore cannot be referenced. You might be able to
% help yourself by using the \xpackage{hyperref} package and
% manually placing |\hypertarget|s and |\href|s.)
%
% \subsubsection{If the same \texttt{\textbackslash pagenumbering\{\ldots \}}\ %
%  scheme is used more than once\label{sss:twice}}
%
% \indent Everything from the preceding subsubsections applies and additionally
% the following:\\
% \DescribeMacro{pagecontinue}
% \indent If the same page numbering scheme is used twice (or even more often)
% in one document (e.\,g. in the frontmatter Roman: I--V, in the
% mainmatter arabic: 1--20, and in the backmatter again Roman: VI--X),
% the second time it is used, the page numbering is either continued (option
% \texttt{pagecontinue} or \texttt{pagecontinue=true} or no option
% \texttt{pagecontinue}; the default) or reset to one (option
% \texttt{pagecontinue=false}). It is even possible to use
% a~page numbering scheme more than twice.\\
%
% \phantomsection
% \label{pagesLTS.pnscheme.nr}
% \DescribeMacro{pagesLTS. page numbering scheme . number }
% If you want to refer to the last page of the first, second,\ldots\ use
% of a page numbering scheme, page~V in the example above, you can refer to
% |pagesLTS.<page numbering scheme>.<number>|, e.\,g.
% |\lastpageref{pagesLTS.Roman.1}|, where |<number>| is the occurrence
% of the page numbering scheme.\\
%
% \pagebreak
%
% If you want to refer to the first page of a page numbering scheme,
% just place a label there, e.\,g.\\
% \indent |\pagenumbering{Roman}| \\
% \indent |\section{Section title\label{RomanSection}}| \\
% (You know where you use |\pagenumbering{...}| and this is the
%  \xpackage{pageslts} package, not the \xpackage{firstpage} one).\\
%
% When you want to give the number of pages of each
% \textquotedblleft sector\textquotedblright{} of the page numbering
% scheme, you can use\\
% \DescribeMacro{pagesLTS. page numbering scheme . number .local.cnt}
% |\lastpages{<page numbering scheme>}{<number>}|, \\
% where |<page numbering scheme>| is e.\,g. Roman, arabic,\ldots\ and
% |<number>| the \textquotedblleft sector\textquotedblright{} number,
% e.\,g.~|\lastpages{Roman}{2}|.\\
% (Internally, the counter has the format
% |pagesLTS.<page numbering scheme>.<number>.local.cnt|.)
%
% \noindent If you used the page numbering scheme Roman for three times,
% you could say
% \begin{verbatim}
% Last Roman page (pagesLTS.Roman): \lastpageref{pagesLTS.Roman}\\
% There are \lastpageref{pagesLTS.Roman.local}~pages with Roman numbers:\\
% \lastpages{Roman}{1}~pages in the first Roman sector
% (\pageref{Roman} -- \lastpageref{pagesLTS.Roman.1}),\\
% \lastpages{Roman}{2}~pages in the second Roman sector
% (\pageref{Roman2} -- \lastpageref{pagesLTS.Roman.2}), and\\
% \lastpages{Roman}{3}~pages in the third Roman sector
% (\pageref{Roman3} -- \lastpageref{pagesLTS.Roman.3}.\\
% \end{verbatim}
% to get
% \begin{quote}
% Last Roman page (pagesLTS.Roman): VIII\\
% There are 8 pages with Roman numbers:\\
% 3 pages in the first Roman sector (I -- III),\\
% 4 pages in the second Roman sector (IV -- VII), and\\
% 3 pages in the third Roman sector (VIII -- X).\\
% \end{quote}
% (see e.\,g. the compiled \textsf{pageslts-example} file).\\
%
% If you want to continue one page numbering scheme, but later on
% (third use of it, or for another page numbering scheme) want to
% reset the page number, just say |\setcounter{page}{1}| there.\\
%
% \pagebreak
%
% \noindent In your document the code
% \begin{verbatim}
% \makeatletter
% \renewcommand{\@evenfoot}%
%  {\normalsize\slshape DRAFT \today\hfil \upshape page {\thepage} (\theCurrentPage) of\ %
%   \lastpageref{pagesLTS.Roman} + \lastpageref{pagesLTS.arabic}\ = \lastpageref{LastPages} pages%
%  }
% \renewcommand{\@oddfoot}{\@evenfoot}
% \makeatother
%\end{verbatim}
% \noindent creates footers like\\
%
% \textquotedblleft\mbox{\textsl{DRAFT \today}\hspace{1cm}page V (5) of VII + 35 = 42 pages}\textquotedblright\\
%
% \noindent or\\
%
% \textquotedblleft\mbox{\textsl{DRAFT \today}\hspace{1cm}page 10 (17) of VII + 35 = 42 pages}\textquotedblright\\
%
% \noindent in the compiled document (cf.~the \texttt{pageslts-example} file).
%
% \noindent Code like
% \begin{verbatim}
%  This book has \lastpageref{pagesLTS.Roman}+\lastpageref{pagesLTS.arabic} pages %
%  (\lastpageref{LastPages} pages in total).
% \end{verbatim}
% \noindent produces output like
% \begin{quote}
% This book has X+85 pages (95 pages in total).
% \end{quote}
% \noindent (when using the \xpackage{hyperref} package,
% the references are even hyperlinked).\\
%
% If |\addtocounter{page}{...}| or |\setcounter{page}{...}| have been used,
% the local version of \texttt{CurrentPage} can be used, |\theCurrentPageLocal|,
% see subsection~\ref{CurrentPageLocal}.\\
%
% \subsection{\xpackage{papermas}(s) package\label{ss:papermas}}
% There is a kind of an add-on to this package, the \xpackage{papermas}
% package, which can be used to compute the number of sheets of paper
% needed to print a document (you can print more than one page of a
% document on one sheet of paper) as well as the approximate mass
% of the printout. Please see the \ref{ss:Downloads} subsection.
%
% \newpage
%
% \section{A few warnings\label{sec:warn}}
%
% \subsection{Hyperref and repeated page numbers}
%
% When two (or more) different page numbering schemes are used, or the page number
% is reset, or for any other reason there are two pages with the same number
% (maybe in different format, e.\,g. $1$ and \textsc{I}), and \xpackage{hyperref} has
% not been configured right, this can cause problems. Use \xpackage{hyperref} with
% \texttt{plainpages=false} and \texttt{pdfpagelabels=true}, and everything should
% be fine.\\
% More details can be found at
% \ \url{http://www.tex.ac.uk/cgi-bin/texfaq2html?label=pdfpagelabels}.
%
% \subsection{\texttt{\textbackslash AtEndDocument}\label{ssec:aed}}
%
% \indent The output of a \LaTeXe\ run is not independent of the order
% in which packages are loaded. It is often the case that the same
% formats for which one must put tables and figure at the end,
% are the ones in which endnotes are also required.
% If one wants to use |\AtEndDocument| here as well (as done for
% |\lastpageref{LastPage}|), then it is easy to get to three separate
% uses of |\AtEndDocument| (assuming one uses this for the endnotes
% as well). Clearly it is not safe for any package writer or user
% to assume that no material will follow what they put into
% |\AtEndDocument|. Therefore a message, which begins with
% \texttt{AED}, is included in every usage of |\AtEndDocument|,
% and it is tried to minimize any side effects the usage may have.\\
%
% As now \textsc{Heiko Oberdiek}'s \texttt{atveryend} package is used,
% the references |\lastpageref{VeryLastPage}| and \newline
% |\lastpageref{LastPages}| should work all right. About how to get the
% \texttt{atveryend} package, please see subsection~\ref{ss:Downloads}.
%
% \subsection{Interaction with very old versions of the \xpackage{endfloat} %
%  package\label{sec:endfloat}}
%
% \indent The \emph{very} old version~2.0 (and earlier) of the \xpackage{endfloat}
% package actually redefined the |\enddocument| command,
% and so interfered drastically with the \LaTeXe\ commands which make use
% of |\AtEndDocument|. Newer versions of \xpackage{endfloat} exist
% (at the time of writing this documentation: v2.5d as of 2011/12/25)
% in modern documentation form, which should be available from
% the same source where you received this file, see subsection~\ref{ss:Downloads}.\\
% A note is placed in the style file at the |\RequirePackage| section,
% and later it is even checked whether a (very) old \xpackage{endfloat}
% package is in use. If it is, a warning or even an error message
% is given, depending on \xpackage{endfloat} version.
% This assumes, that the old versions of \xpackage{endfloat} at least gave
% a version date, of course.\\
% If you want your \texttt{LastPage} to label the last page of
% these end floats, you need to load \xpackage{pageslts} \emph{after}
% loading \xpackage{endfloat}, or to use \texttt{VeryLastPage} instead.
% If, on the other hand, you \emph{want} \texttt{LastPage} to refer
% to the (not so) last page, exclusive of the floats at the end,
% then load in the reverse order. Independent from the order of
% \xpackage{pageslts} and \xpackage{endfloat}, you will still need the
% modified\footnote{New versions are available for over $15$~years now,
% so it might be time to update, if you did not do it already.} version of
% \xpackage{endfloat}.\\
% Using the
% \nolinebreak{\texttt{LastPage}\textbf{\underline{s}}\ (\textit{s}!)}\ label
% should get you to the last page in all cases: |\lastpageref{LastPages}|.\\
%
% Other \LaTeX209\ (!) packages also seem to like to redefine
% |\enddocument|. In addition to the old \xpackage{endfloat},
% \xpackage{harvard} comes to mind. All of these will need to be
% modified swiftly. \textbf{If possible, update to \LaTeXe{}!}
%
% \pagebreak
%
% \subsection{\xpackage{showkeys} package\label{ssec:showkeys}}
%
% When the \xpackage{showkeys} package has been loaded in \texttt{draft} mode,
% in the margin for each label a~box is displayed with the name of the label.
% \xpackage{showkeys} accomplishes this by redefining |\label|, but
% \xpackage{pageslts} does not use |\label|, but writes directly to the
% {\nobreak |\jobname|\xfile{.aux}-}\allowbreak file, and this is generally done
% \emph{after} the according page has shipped out, therefore no box can be placed
% on the \emph{preceding} page. At~least \xpackage{pageslts} gives a warning, that
% \xpackage{showkeys} cannot present the respective label.
%
% \subsection{\xpackage{lastpage} package\label{ssec:lastpage}}
%
% This package first started as a revision of the \xpackage{lastpage}
% package of \textsc{Jeffrey P. Goldberg} (jeffrey+news at goldmark dot org),
% but it became obvious that a replacement was needed to accomplish what this
% package does. For backward compatibility, a~label named |LastPage| is provided.
% Thus |\usepackage{lastpage}| can be replaced by\\
% |\usepackage[pagecontinue=false,%|\newline
% |alphMult=0,AlphMulti=0,fnsymbolmult=false,romanMult=false,RomanMulti=false]{pageslts}|,\\
% if the behaviour of the \xpackage{lastpage} package should be simulated.
% Using \textbf{old}~(!) versions of the \xpackage{lastpage} before the \xpackage{pageslts}
% before the \xpackage{hyperref} [2012/11/06 v6.83m]
% package results in multiply definitions of the |LastPage| label.
% While the \xpackage{pageslts} package cancels the command |\lastpage@putlabel|
% from the old \xpackage{lastpage} package (because it does this itself, and better),
% \xpackage{hyperref} redefines |\lastpage@putlabel| and thereby reintroduces it again
% (\xpackage{hyperref} should probably check for the version of the \xpackage{lastpage}
% package and/or whether the \xpackage{pageslts} package was also loaded.)
%
% \subsection{Using an unknown page numbering scheme\label{sec:unknown}}
%
% \indent I do not know whether \LaTeXe\ can handle another page numbering
% scheme (e.\,g.~Hebraic), but if you want to use it, this should be
% no problem for the \xpackage{pageslts} package. But the original
% |\pagenumbering{...}| as well as the \xpackage{hyperref} package
% (if~used) might want to vote against it, especially when used together
% with the \xpackage{pageslts} package. Especially especially
% (\textit{sic}!) if the last page uses this new page numbering scheme,
% you should check everything double (at least).\\
% (And if the number format is unknown to \LaTeX{}, the pages will have
% no number, and therefore cannot be referenced. You might be able to
% help yourself by using the \xpackage{hyperref} package and
% manually placing |\hypertarget|s and |\href|s.)
%
% \subsection{Page counter overflow\label{ss:overflow}}
%
% \indent Without the use of the \xpackage{alphalph} package, the\\
% \textquotedblleft ranges of supported counter values are more or less
% restricted. Only \cs{arabic} can be used with any counter
% value \TeX\ supports.
% \begin{quote}
% \catcode`\|=12 %
% \begin{tabular}{@{}l|l|l|l@{}}
% Presentation & Supported & Ignored & Error message\\
% command      & domain    & values  & `Counter too large'\\
% \hline
% \cs{arabic}
%   & \ttfamily -MAX..MAX & &\\
% \cs{roman}, \cs{Roman}
%   & \ttfamily 1..MAX & \ttfamily -MAX..0 &\\
% \cs{alph}, \cs{Alph}
%   & \ttfamily 1..26 & 0 & \ttfamily -MAX..-1, 27..MAX\\
% \cs{fnsymbol}
%   & \ttfamily 1..9 & 0 & \ttfamily -MAX..-1, 10..MAX\\
% \hline
% \end{tabular}\\[1ex]
% \texttt{MAX} = \texttt{2147483647}
% \end{quote}
% \textquotedblright{} (\textsc{Heiko Oberdiek}:
% The \xpackage{alphalph} package, 2010/04/18, v2.3, first table, p.~2).\\
% Please see subsubsections~\ref{sss:alphalph} and \ref{sss:romanMult} for
% instructions how to overcome these limitations.
%
% \newpage
%
% \subsection{Using the \xpackage{fnsymbol} page numbering scheme\label{ss:fnsymbol}}
%
% \indent Using the \texttt{fnsymbol} page numbering scheme can result in
% problems -- big ones!\\
% \textbf{When using this page numbering scheme, it is very important to use}
% \ |\lastpageref{...}| \textbf{instead of}\ |\pageref{...}| \textbf{for any%
% link to any label provided by the \xpackage{pageslts} package.}\\
% While the \xpackage{pageslts} package tries really very hard to circumvent
% any problem, other packages might screw up -- and quite totally for that.
% So,~you have been warned!\\
%
% \begin{description}
% \item[-] There can be a counter overflow, see preceding subsection~\ref{ss:overflow}.
%
% \item[-] Adobe Reader DC 2015.008.20082 does not show the correct page names
% for all pages with \texttt{fnsymbol} page numbering scheme (see the example file),
% while at least the (\ldots\ of \ldots )  part of the page number is displayed correctly
% (see page~\pageref{logical page numbers}, tip about logical page numbers).
% When the \xpackage{alphalph} package and the \xpackage{pageslts} package with
% |fnsymbolmult| option are used, more page names are presented correctly by the Reader.
% (Adobe Reader~X even got all pages right.)
% \end{description}
%
% \newpage
%
% \section{Alternatives\label{sec:Alternatives}}
%
% There are similar packages, which do (or do not) similar things. As I
% neither know what exactly you want to accomplish when using this package
% (e.\,g.~page number vs. page name, hyperlinks or not), nor what resources
% you have (e.\,g.~\eTeX{}), here is a list of some possible alternatives:\\
%
% \DescribeMacro{LastPage}
% \begin{description}
% \item[-] The \xpackage{LastPage} package also provides the
%  \texttt{LastPage} label (but not \texttt{VeryLastPage} or
%  \texttt{LastPages}). If you only want this and have a quite
%  limited amount of \TeX\ resources, you might want to use that
%  package instead.
%
% \item[-] If \LaTeX2.09\ is still used, and if you are unable
%  to switch to \LaTeXe{}, the \LaTeX2.09{} compatible
%  \xpackage{lastpage209.sty} can be used (which is also contained
%  in the recent \xpackage{LastPage} package).
% \end{description}
%
% \DescribeMacro{totpages}
% \begin{description}
% \item[-] The \xpackage{totpages} package provides a
%  \texttt{totpages} label similar to \texttt{LastPages},
%  but |\AtEndDocument| instead of |\AfterLastShipout|. Therefore
%  you should stay with \xpackage{pageslts}. The \xpackage{totpages}
%  package additionally computes the number of paper sheets needed to
%  (double) print the document (with one, two, three,\ldots\ pages
%  on one sheet of paper). This can also be accomplished with the
%  \xpackage{papermas} package.
% \end{description}
%
% \DescribeMacro{totcount}
% \begin{description}
% \item[-] The \xpackage{totcount} package provides the last value
%  of a counter, thus also the value of the \texttt{page} counter.
%  You do not get a hyperlink to the last page, only the numerical
%  value of the last page name is given (i.\,e. X+72 pages gives 72
%  instead of 82 as total number of pages), and the number of pages
%  can be changed e.\,g. by |\addtocounter|.
% \end{description}
%
% \DescribeMacro{nofm}
% \begin{description}
% \item[-] \textquotedblleft There is a package \xpackage{nofm.sty}
%  available, but some versions of it are defective, and most don't
%  work with \xpackage{fancyhdr} because they take over the complete
%  page layout.\textquotedblright{} (\textsc{Piet van Oostrum}:
%  Page layout in \LaTeX{}, March~2, 2004, section~16; fancyhdr.pdf)\\
%  \xpackage{nofm} as of 1991/02/25 (without version number), available at\\
%  \url{http://mirror.ctan.org/obsolete/macros/latex209/contrib/misc/nofm.sty},\\
%  does not work with e.\,g. \xpackage{hyperref}, redefines |\enddocument|
%  as well as |\@oddhead|, |\@evenhead|, |\@oddfoot|, and |\@evenfoot|.\\
%  If you know the (\url{https://www.CTAN.org}) location of a%
%  \textbf{working}~(!) version, please send me an e-mail, thanks!
% \end{description}
%
% \DescribeMacro{count1to}
% \begin{description}
% \item[-] You may want to have a look at the \xpackage{count1to} package.
% \end{description}
%
% \DescribeMacro{zref}
% \begin{description}
% \item[-] The \xpackage{zref} package of \textsc{Heiko Oberdiek} requires
%  \eTeX . \xpackage{pageslts} does not require \eTeX , but if you
%  already have \eTeX , you may have a look at the extensive
%  \xpackage{zref} package, whether it suits your needs better
%  (or additionally or whatsoever).
% \end{description}
%
% \bigskip
%
% \noindent (You programmed or found another alternative,
%  which is available at \url{https://www.CTAN.org}?\\
%  OK, send an e-mail to me with the name, location at CTAN,
%  and a short notice, and I will probably include it in the list above.)\\
%
% \smallskip
%
% \noindent About how to get those packages, please see subsection~\ref{ss:Downloads}.
%
% \newpage
%
% \section{Example}
%
%    \begin{macrocode}
%<*example>
\documentclass[british]{article}[2014/09/29]% v1.4h
\usepackage{lipsum}[2014/07/27]%              v1.3
\usepackage[draft]{showkeys}[2014/10/28]%     v3.17
%%      Use final instead of draft to hide the keys. %%
\usepackage{hyperref}[2012/11/06]%            v6.83m
\hypersetup{%
 extension=pdf,%
 plainpages=false,%
 pdfpagelabels=true,%
 hyperindex=false,%
 pdflang={en},%
 pdftitle={pageslts package example},%
 pdfauthor={Hans-Martin Muench},%
 pdfsubject={Example for the pageslts package},%
 pdfkeywords={LaTeX, pageslts, H.-Martin Muench},%
 pdfview=Fit,%
 pdfstartview=Fit,%
 pdfpagelayout=SinglePage,%
 bookmarksopen=true%
}
\usepackage[pagecontinue=true,alphMult=ab,AlphMulti=AB,fnsymbolmult=true,%
 romanMult=true,RomanMulti=true]{pageslts}[2015/12/21]% v1.2f
%% These are the default options. %%

\makeatletter
  \renewcommand{\@evenfoot}%
   {Page \thepage\ (\theCurrentPage; local: \theCurrentPageLocal) of %
    \lastpageref{pagesLTS.roman}(\lastpageref{pagesLTS.roman.local}) + %
    \lastpageref{pagesLTS.Roman}(\lastpageref{pagesLTS.Roman.local}) + %
    \lastpageref{pagesLTS.arabic}(\lastpageref{pagesLTS.arabic.local}) + %
    \lastpageref{pagesLTS.fnsymbol}(\lastpageref{pagesLTS.fnsymbol.local}) + %
    \lastpageref{pagesLTS.alph}(\lastpageref{pagesLTS.alph.local}) + %
    \lastpageref{pagesLTS.Alph}(\lastpageref{pagesLTS.Alph.local}) = %
    \lastpageref{LastPages} pages.%
   }
  \renewcommand{\@oddfoot}{\@evenfoot}
  \def\pagesLTSexampleArabic{3}
  \def\pagesLTSexamplealph{23}
  \gdef\unit#1{\mathord{\thinspace\mathrm{#1}}}%
\makeatother

\listfiles

\begin{document}
\pagenumbering{roman}
%% Note the first \pagenumbering immediately behind \begin{document}! %%
%%\addtocounter{page}{-2} %%

\section*{Example for pageslts}
\addcontentsline{toc}{section}{Example for pageslts}
\markboth{Example for pageslts}{Example for pageslts}

This example demonstrates the most common uses of package\\
\textsf{pageslts}, v1.2f as of 2015/12/21 (HMM).\\
The used options were \texttt{pagecontinue=true},
\texttt{alphMult=ab}, \texttt{AlphMulti=AB}, \linebreak
\texttt{fnsymbolmult=true},
\texttt{romanMult=true}, and \texttt{RomanMulti=true}
(the default ones).\\
For more details please see the documentation!\\

\label{keys} To hide the \pageref{keys}{\qquad } use option
\texttt{final} instead of \texttt{draft} with the \textsf{showkeys}
package (or remove the package call from the preamble of
this document).\\

\textbf{Hyperlinks or not:} If the \textsf{hyperref} package is loaded,
the references are also hyperlinked:\\
\smallskip
Page \thepage\ (\theCurrentPage; local: \theCurrentPageLocal) of %
\lastpageref{pagesLTS.roman}(\lastpageref{pagesLTS.roman.local}) + %
\lastpageref{pagesLTS.Roman}(\lastpageref{pagesLTS.Roman.local}) + %
\lastpageref{pagesLTS.arabic}(\lastpageref{pagesLTS.arabic.local}) + %
\lastpageref{pagesLTS.fnsymbol}(\lastpageref{pagesLTS.fnsymbol.local}) + %
\lastpageref{pagesLTS.alph}(\lastpageref{pagesLTS.alph.local}) + %
\lastpageref{pagesLTS.Alph}(\lastpageref{pagesLTS.Alph.local}) = %
\lastpageref{LastPages} pages.\\
If the \textsf{hyperref} package is loaded, but the hyperlinks of the
references shall be suppressed, \verb|\pageref*{...}|
and \verb|\lastpageref*{...}| can be used:\\
Page \thepage\ (\theCurrentPage; local: \theCurrentPageLocal) of %
\lastpageref*{pagesLTS.roman}(\lastpageref*{pagesLTS.roman.local}) + %
\lastpageref*{pagesLTS.Roman}(\lastpageref*{pagesLTS.Roman.local}) + %
\lastpageref*{pagesLTS.arabic}(\lastpageref*{pagesLTS.arabic.local}) + %
\lastpageref*{pagesLTS.fnsymbol}(\lastpageref*{pagesLTS.fnsymbol.local}) + %
\lastpageref*{pagesLTS.alph}(\lastpageref*{pagesLTS.alph.local}) + %
\lastpageref*{pagesLTS.Alph}(\lastpageref*{pagesLTS.Alph.local}) = %
\lastpageref*{LastPages} pages.\\


\textbf{Trademarks} appear throughout this example without any
trademark symbol; they are the property of their respective
trademark owner. There is no intention of infringement; the
usage is to the benefit of the trademark owner.\\

\textbf{Tip}: Use \textit{logical page numbers}
for the display of the pdf (in Adobe Reader DC 2015.008.20082:
Edit $>$ Preferences $>$ Categories: Page Display $>$
Page Content and Information: Use logical page numbers)!\\

You want negative page numbers? Not only arabic, but even roman,
Roman, alph, Alph or fnsymbol ones? No problem, e.\,g. just give a\\
\verb|\addtocounter{page}{|-\textit{some number}\verb|}| in the
source code of this example file (or uncomment the prepared line)!

\bigskip

Save per page about $200\unit{ml}$ water, $2\unit{g}$ CO$_{2}$
and $2\unit{g}$ wood:\\
Therefore please print only if this is really necessary.

\pagebreak

\tableofcontents

\newpage

\pagenumbering{roman}
%% in case the page numbering is changed before,
%% otherwise pagesLTS.current.local.roman on this page
%% would be undefined

\section{roman}

\noindent (\texttt{roman} page numbering was started before,
because the page numbering scheme was needed to start at
the first page, of course.)\\

\noindent First page (\verb|\lastpageref{pagesLTS.0}|):
\lastpageref{pagesLTS.0}\\

\noindent The page (\verb|\thepage|): \thepage \\

\noindent Current page (\verb|\theCurrentPage|),
i.\,e. counted continuously from the first page): \theCurrentPage \\
You can get this also in other formats:
\roman{CurrentPage}, \Roman{CurrentPage}, \arabic{CurrentPage},
\fnsymbol{CurrentPage}, \alph{CurrentPage}, \Alph{CurrentPage}.

\noindent CurrentPageLocal (\verb|\theCurrentPageLocal|),
i.\,e. counted continuously from the first page of the
current page numbering scheme): \theCurrentPageLocal \\
You can get also this in other formats, too:
\roman{pagesLTS.current.local.roman}, \Roman{pagesLTS.current.local.roman},
\arabic{pagesLTS.current.local.roman}, \fnsymbol{pagesLTS.current.local.roman},
\alph{pagesLTS.current.local.roman}, \Alph{pagesLTS.current.local.roman},
but probably it only makes sense if page numbering scheme and format are
the same, e.\,g.\\
\verb|\Roman{pagesLTS.current.local.Roman}|\\
or \verb|\Alph{pagesLTS.current.local.Alph}|. \verb|\arabic{...}| could
make sense even if combined with another page numbering scheme.
And this is exactly what \verb|\theCurrentPageLocal| does:\\
\nolinebreak{\verb|\def\theCurrentPageLocal{\arabic{pagesLTS.current.local.\pagesLTS@pnc}}|.}\\

\noindent Last roman page (pagesLTS.roman): \lastpageref{pagesLTS.roman}{\hskip4em }
(There are \lastpageref{pagesLTS.roman.local} pages with roman numbers.)\\

\noindent Last Roman page (pagesLTS.Roman): \lastpageref{pagesLTS.Roman}{\hskip3em }
(There are \lastpageref{pagesLTS.Roman.local}~pages with Roman numbers:\\
\lastpages{Roman}{1}~pages in the first Roman sector
(\pageref{Roman}{\hskip3em }-\lastpageref{pagesLTS.Roman.1}{\hskip3em }),\\
\lastpages{Roman}{2}~pages in the second Roman sector
(\pageref{Roman2}{\hskip3em }-\lastpageref{pagesLTS.Roman.2}{\hskip3em }), and\\
\lastpages{Roman}{3}~pages in the third Roman sector
(\pageref{Roman3}{\hskip3em }-\lastpageref{pagesLTS.Roman.3}{\hskip3em }).\\

When the option \texttt{pagecontinue=false} is used with the
\textsf{pageslts} package, the \verb|\lastpageref{pagesLTS.Roman}| will
point to the same page as before, but this will have a lower number.\\
The \verb|\lastpageref{pagesLTS.Roman.local}| will not change,
because the number of pages does not change (only the page numbers).\\

\noindent Last arabic page (pagesLTS.arabic): \lastpageref{pagesLTS.arabic}{\hskip5em }
(There are only \lastpageref{pagesLTS.arabic.local} pages with arabic numbers,
because an \verb|\addtocounter{page}{|\pagesLTSexampleArabic\verb|}| was used.)\\

\noindent Last fnsymbol page (pagesLTS.fnsymbol): \lastpageref{pagesLTS.fnsymbol} \\
(\verb|\lastpageref{pagesLTS.fnsymbol}| -- never \verb|\pageref{pagesLTS.fnsymbol}|!)\\
(There are \lastpageref{pagesLTS.fnsymbol.local} pages with fnsymbol numbers.)\\

\noindent Last alph page (pagesLTS.alph): \lastpageref{pagesLTS.alph}{\hskip4em }
(There are only \lastpageref{pagesLTS.alph.local} pages with alph numbers,
because an \verb|\addtocounter{page}{|\pagesLTSexamplealph\verb|}| was used.)\\

\noindent Last Alph page (pagesLTS.Alph): \lastpageref{pagesLTS.Alph}{\hskip4em }
(There are \lastpageref{pagesLTS.Alph.local} pages with Alph numbers.)\\

\noindent Last page's \textit{name} (LastPage): \lastpageref{LastPage}\\

\noindent Very last page's \textit{name} (VeryLastPage): \lastpageref{VeryLastPage}\\
(\texttt{lastpage} and \texttt{VeryLastPage} are identical, unless
a package with output \linebreak
\verb|\AtEndDocument| after the \textsf{pageslts} package was added.)\\

\noindent Last page's \textit{number} (LastPages): \lastpageref{LastPages}{\hskip3em }
(=~total number of pages)\\

\lipsum[1-3]

\newpage

\pagenumbering{Roman}

\section{Roman\label{Roman}}
\subsection{Common Roman page numbering}

\noindent First page (\verb|\lastpageref{pagesLTS.0}|):
\lastpageref{pagesLTS.0}\\

\noindent The page (\verb|\thepage|): \thepage \\

\noindent Current page (\verb|\theCurrentPage|),
i.\,e. counted continuously from the first page): \theCurrentPage \\

\noindent CurrentPageLocal (\verb|\theCurrentPageLocal|),
i.\,e. counted continuously from the first page of the
current page numbering scheme): \theCurrentPageLocal \\

\noindent Last roman page (pagesLTS.roman): \lastpageref{pagesLTS.roman}{\hskip4em }
(There are \lastpageref{pagesLTS.roman.local} pages with roman numbers.)\\

\noindent Last Roman page (pagesLTS.Roman): \lastpageref{pagesLTS.Roman}{\hskip3em }
(There are \lastpageref{pagesLTS.Roman.local}~pages with Roman numbers:\\
\lastpages{Roman}{1}~pages in the first Roman sector
(\pageref{Roman}{\hskip3em }-\lastpageref{pagesLTS.Roman.1}{\hskip3em }),\\
\lastpages{Roman}{2}~pages in the second Roman sector
(\pageref{Roman2}{\hskip3em }-\lastpageref{pagesLTS.Roman.2}{\hskip3em }), and\\
\lastpages{Roman}{3}~pages in the third Roman sector
(\pageref{Roman3}{\hskip3em }-\lastpageref{pagesLTS.Roman.3}{\hskip3em }).\\

\noindent Last arabic page (pagesLTS.arabic): \lastpageref{pagesLTS.arabic}{\hskip5em }
(There are only \lastpageref{pagesLTS.arabic.local} pages with arabic numbers,
because an \verb|\addtocounter{page}{|\pagesLTSexampleArabic\verb|}| was used.)\\

\noindent Last fnsymbol page (pagesLTS.fnsymbol): \lastpageref{pagesLTS.fnsymbol} \\
(\verb|\lastpageref{pagesLTS.fnsymbol}| -- never \verb|\pageref{pagesLTS.fnsymbol}|!)\\
(There are \lastpageref{pagesLTS.fnsymbol.local} pages with fnsymbol numbers.)\\

\noindent Last alph page (pagesLTS.alph): \lastpageref{pagesLTS.alph}{\hskip4em }
(There are only \lastpageref{pagesLTS.alph.local} pages with alph numbers,
because an \verb|\addtocounter{page}{|\pagesLTSexamplealph\verb|}| was used.)\\

\noindent Last Alph page (pagesLTS.Alph): \lastpageref{pagesLTS.Alph}{\hskip4em }
(There are \lastpageref{pagesLTS.Alph.local} pages with Alph numbers.)\\

\noindent Last page's \textit{name} (LastPage): \lastpageref{LastPage}\\

\noindent Very last page's \textit{name} (VeryLastPage): \lastpageref{VeryLastPage}\\
(\texttt{LastPage} and \texttt{VeryLastPage} are identical, unless
a package with output \linebreak
\verb|\AtEndDocument| after the \textsf{pageslts} package was added.)\\

\noindent Last page's \textit{number} (LastPages): \lastpageref{LastPages}{\hskip3em }
(=~total number of pages)\\

\lipsum[1-4]

\newpage

\subsection{Last page of first Roman sector}
\verb|\lastpageref{pagesLTS.Roman}| does \textbf{not}
refer to this page (but there: \lastpageref{pagesLTS.Roman}),
because the option \texttt{pagecontinue=true}
was chosen. When a reference to this page is wanted,\\
\verb|\lastpageref{pagesLTS.Roman.1}| can be used: \lastpageref{pagesLTS.Roman.1}.\\

\bigskip
There are \lastpages{Roman}{1}~pages (\verb|\lastpages{Roman}{1}|)
in this first Roman sector.\\
The Roman page numbering scheme is continued later in section~\ref{Roman2})!

\newpage

\pagenumbering{arabic}

\section{arabic}

\subsection{Standard page numbering}

\noindent First page (\verb|\lastpageref{pagesLTS.0}|):
\lastpageref{pagesLTS.0}\\

\noindent The page (\verb|\thepage|): \thepage \\

\noindent Current page (\verb|\theCurrentPage|),
i.\,e. counted continuously from the first page): \theCurrentPage \\

\noindent CurrentPageLocal (\verb|\theCurrentPageLocal|),
i.\,e. counted continuously from the first page of the
current page numbering scheme): \theCurrentPageLocal \\

\noindent Last roman page (pagesLTS.roman): \lastpageref{pagesLTS.roman}{\hskip4em }
(There are \lastpageref{pagesLTS.roman.local} pages with roman numbers.)\\

\noindent Last Roman page (pagesLTS.Roman): \lastpageref{pagesLTS.Roman}{\hskip3em }
(There are \lastpageref{pagesLTS.Roman.local}~pages with Roman numbers:\\
\lastpages{Roman}{1}~pages in the first Roman sector
(\pageref{Roman}{\hskip3em }-\lastpageref{pagesLTS.Roman.1}{\hskip3em }),\\
\lastpages{Roman}{2}~pages in the second Roman sector
(\pageref{Roman2}{\hskip3em }-\lastpageref{pagesLTS.Roman.2}{\hskip3em }), and\\
\lastpages{Roman}{3}~pages in the third Roman sector
(\pageref{Roman3}{\hskip3em }-\lastpageref{pagesLTS.Roman.3}{\hskip3em }).\\

\noindent Last arabic page (pagesLTS.arabic): \lastpageref{pagesLTS.arabic}{\hskip5em }
(There are only \lastpageref{pagesLTS.arabic.local} pages with arabic numbers,
because an \verb|\addtocounter{page}{|\pagesLTSexampleArabic\verb|}| was used.)\\

\noindent Last fnsymbol page (pagesLTS.fnsymbol): \lastpageref{pagesLTS.fnsymbol} \\
(\verb|\lastpageref{pagesLTS.fnsymbol}| -- never \verb|\pageref{pagesLTS.fnsymbol}|!)\\
(There are \lastpageref{pagesLTS.fnsymbol.local} pages with fnsymbol numbers.)\\

\noindent Last alph page (pagesLTS.alph): \lastpageref{pagesLTS.alph}{\hskip4em }
(There are only \lastpageref{pagesLTS.alph.local} pages with alph numbers,
because an \verb|\addtocounter{page}{|\pagesLTSexamplealph\verb|}| was used.)\\

\noindent Last Alph page (pagesLTS.Alph): \lastpageref{pagesLTS.Alph}{\hskip4em }
(There are \lastpageref{pagesLTS.Alph.local} pages with Alph numbers.)\\

\noindent Last page's \textit{name} (LastPage): \lastpageref{LastPage}\\

\noindent Very last page's \textit{name} (VeryLastPage): \lastpageref{VeryLastPage}\\
(\texttt{LastPage} and \texttt{VeryLastPage} are identical, unless
a package with output \linebreak
\verb|\AtEndDocument| after the \textsf{pagesLTS} package was added.)\\

\noindent Last page's \textit{number} (LastPages): \lastpageref{LastPages}{\hskip3em }
(=~total number of pages)\\

\lipsum[1-4]
\newpage

\subsection[Empty page style]{Also an empty page style is no problem %
for the current or total page count}

\bigskip

\thispagestyle{empty}

\noindent First page (\verb|\lastpageref{pagesLTS.0}|):
\lastpageref{pagesLTS.0}\\

\noindent The page (\verb|\thepage|): \thepage \\

\noindent Current page (\verb|\theCurrentPage|),
i.\,e. counted continuously from the first page): \theCurrentPage \\

\noindent CurrentPageLocal (\verb|\theCurrentPageLocal|),
i.\,e. counted continuously from the first page of the
current page numbering scheme): \theCurrentPageLocal \\

\noindent Last roman page (pagesLTS.roman): \lastpageref{pagesLTS.roman}{\hskip4em }
(There are \lastpageref{pagesLTS.roman.local} pages with roman numbers.)\\

\noindent Last Roman page (pagesLTS.Roman): \lastpageref{pagesLTS.Roman}{\hskip3em }
(There are \lastpageref{pagesLTS.Roman.local}~pages with Roman numbers:\\
\lastpages{Roman}{1}~pages in the first Roman sector
(\pageref{Roman}{\hskip3em }-\lastpageref{pagesLTS.Roman.1}{\hskip3em }),\\
\lastpages{Roman}{2}~pages in the second Roman sector
(\pageref{Roman2}{\hskip3em }-\lastpageref{pagesLTS.Roman.2}{\hskip3em }), and\\
\lastpages{Roman}{3}~pages in the third Roman sector
(\pageref{Roman3}{\hskip3em }-\lastpageref{pagesLTS.Roman.3}{\hskip3em }).\\

\noindent Last arabic page (pagesLTS.arabic): \lastpageref{pagesLTS.arabic}{\hskip5em }
(There are only \lastpageref{pagesLTS.arabic.local} pages with arabic numbers,
because an \verb|\addtocounter{page}{|\pagesLTSexampleArabic\verb|}| was used.)\\

\noindent Last fnsymbol page (pagesLTS.fnsymbol): \lastpageref{pagesLTS.fnsymbol} \\
(\verb|\lastpageref{pagesLTS.fnsymbol}| -- never \verb|\pageref{pagesLTS.fnsymbol}|!)\\
(There are \lastpageref{pagesLTS.fnsymbol.local} pages with fnsymbol numbers.)\\

\noindent Last alph page (pagesLTS.alph): \lastpageref{pagesLTS.alph}{\hskip4em }
(There are only \lastpageref{pagesLTS.alph.local} pages with alph numbers,
because an \verb|\addtocounter{page}{|\pagesLTSexamplealph\verb|}| was used.)\\

\noindent Last Alph page (pagesLTS.Alph): \lastpageref{pagesLTS.Alph}{\hskip4em }
(There are \lastpageref{pagesLTS.Alph.local} pages with Alph numbers.)\\

\noindent Last page's \textit{name} (LastPage): \lastpageref{LastPage}\\

\noindent Very last page's \textit{name} (VeryLastPage): \lastpageref{VeryLastPage}\\
(\texttt{LastPage} and \texttt{VeryLastPage} are identical, unless
a package with output \linebreak
\verb|\AtEndDocument| after the \textsf{pageslts} package was added.)\\

\noindent Last page's \textit{number} (LastPages): \lastpageref{LastPages}{\hskip3em }
(=~total number of pages)\\

\lipsum[1-4]

\newpage

\subsection[addtocounter, setcounter]{Neither %
\texttt{\textbackslash addtocounter\{page\}} nor %
\texttt{\textbackslash setcounter\{page\}} is a problem for the %
current or total page numbers}

(Here is an \verb|\addtocounter{page}{|\pagesLTSexampleArabic\verb|}|
in the source code.)\\
\addtocounter{page}{\pagesLTSexampleArabic}

\noindent The page (from \verb|\thepage| command): \thepage \\

\noindent Current page (from \verb|\theCurrentPage|\ %
command), i.\,e. counted continuously from the first page): \theCurrentPage \\

\noindent CurrentPageLocal (from \verb|\theCurrentPageLocal|\ %
command), i.\,e. counted continuously from the first page of the
current page numbering scheme): \theCurrentPageLocal\\

\noindent Last page's number (LastPages): \lastpageref{LastPages}{\hskip3em }
(= total number of pages)\\

\lipsum[1-7]

\newpage

\pagenumbering{fnsymbol}

\section{fnsymbol}

Adobe Reader DC 2015.008.20082 does not show the correct page names
for all pages with \texttt{fnsymbol} page numbering scheme,
while at least the (\ldots\ of \ldots )  part of the page number is displayed correctly.
(Adobe Reader~X even got all pages right.)

\bigskip

Without option \texttt{fnsymbolmult=true} of the \textsf{pageslts} package
(and the help of \textsc{Heiko Oberdiek}'s \textsf{alphalph} package),
after page~9 (\textquotedblleft \ensuremath {\ddagger \ddagger }\textquotedblright )
(and also for negative page numbers) there would just appear a
\begin{quote}
\begin{verbatim}
LaTeX Error: Counter too large
See the LaTeX manual or LaTeX Companion for explanation.
You've lost some text.  Try typing  <return>  to proceed.
If that doesn't work, type  X <return>  to quit.
\end{verbatim}
\end{quote}
Now the page numbers after 5 (\ensuremath {\mathparagraph }) are
continued  with the doubled \textquotedblleft number\textquotedblright{} of
the first, second, third,\ldots\ page (\ensuremath {**},
\ensuremath {\dagger \dagger }, \ensuremath {\ddagger \ddagger },
\ensuremath {\mathsection \mathsection },
\ensuremath {\mathparagraph \mathparagraph }),
and after the tenth page the \textquotedblleft number\textquotedblright{} is
tripled (\ensuremath {***}, \ensuremath {\dagger \dagger \dagger },\ldots).
Page zero is named 0 and negative pages just named like the positive ones
with addition of a minus sign~($-$).

\bigskip

\noindent First page (\verb|\lastpageref{pagesLTS.0}|):
\lastpageref{pagesLTS.0}\\

\noindent The page (\verb|\thepage|): \thepage \\

\noindent Current page (\verb|\theCurrentPage|),
i.\,e. counted continuously from the first page): \theCurrentPage \\

\noindent CurrentPageLocal (\verb|\theCurrentPageLocal|),
i.\,e. counted continuously from the first page of the
current page numbering scheme): \theCurrentPageLocal \\

\noindent Last roman page (pagesLTS.roman): \lastpageref{pagesLTS.roman}{\hskip4em }
(There are \lastpageref{pagesLTS.roman.local} pages with roman numbers.)\\

\noindent Last Roman page (pagesLTS.Roman): \lastpageref{pagesLTS.Roman}{\hskip3em }
(There are \lastpageref{pagesLTS.Roman.local}~pages with Roman numbers:\\
\lastpages{Roman}{1}~pages in the first Roman sector
(\pageref{Roman}{\hskip3em }-\lastpageref{pagesLTS.Roman.1}{\hskip3em }),\\
\lastpages{Roman}{2}~pages in the second Roman sector
(\pageref{Roman2}{\hskip3em }-\lastpageref{pagesLTS.Roman.2}{\hskip3em }), and\\
\lastpages{Roman}{3}~pages in the third Roman sector
(\pageref{Roman3}{\hskip3em }-\lastpageref{pagesLTS.Roman.3}{\hskip3em }).\\

\noindent Last arabic page (pagesLTS.arabic): \lastpageref{pagesLTS.arabic}{\hskip5em }
(There are only \lastpageref{pagesLTS.arabic.local} pages with arabic numbers,
because an \verb|\addtocounter{page}{|\pagesLTSexampleArabic\verb|}| was used.)\\

\noindent Last fnsymbol page (pagesLTS.fnsymbol): \lastpageref{pagesLTS.fnsymbol} \\
(\verb|\lastpageref{pagesLTS.fnsymbol}| -- never \verb|\pageref{pagesLTS.fnsymbol}|!)\\
(There are \lastpageref{pagesLTS.fnsymbol.local} pages with fnsymbol numbers.)\\

\noindent Last alph page (pagesLTS.alph): \lastpageref{pagesLTS.alph}{\hskip4em }
(There are only \lastpageref{pagesLTS.alph.local} pages with alph numbers,
because an \verb|\addtocounter{page}{|\pagesLTSexamplealph\verb|}| was used.)\\

\noindent Last Alph page (pagesLTS.Alph): \lastpageref{pagesLTS.Alph}{\hskip4em }
(There are \lastpageref{pagesLTS.Alph.local} pages with Alph numbers.)\\

\noindent Last page's \textit{name} (LastPage): \lastpageref{LastPage}\\

\noindent Very last page's \textit{name} (VeryLastPage): \lastpageref{VeryLastPage}\\
(\texttt{LastPage} and \texttt{VeryLastPage} are identical, unless
a package with output \linebreak
\verb|\AtEndDocument| after the \textsf{pageslts} package was added.)\\

\noindent Last page's \textit{number} (LastPages): \lastpageref{LastPages}{\hskip3em }
(=~total number of pages)\\

\lipsum[1-60]

\newpage

\pagenumbering{Roman}

\section{Roman - again!\label{Roman2}}

The page number would start with \textquotedblleft I\textquotedblright{} again --
but for the \textsf{pageslts} package (with option \texttt{pagecontinue=true},
or with option just \texttt{pagecontinue}, or even just
\nolinebreak{with\textbf{out}} option \texttt{pagecontinue=false}).
This package remembered the (\arabic{pagesLTS.double.Roman}-1)\footnote{%
OK, here you have to compute this value for yourself, but %
subtracting one should be manageable for \TeX{}nicians.} pages already
done in Roman output, and therefore continues with page
\textquotedblleft \thepage \textquotedblright .\\
If you want to start with \textquotedblleft I\textquotedblright{} all
over again, you will have two pages with the same name,
but nevertheless you can do this by using option \texttt{pagecontinue=false}
or a \verb|\setcounter{page}{1}|\ here (not demonstrated in this example file).\\

\noindent First page (\verb|\lastpageref{pagesLTS.0}|):
 \lastpageref{pagesLTS.0}\\

\noindent The page (\verb|\thepage|): \thepage \\

\noindent Current page (\verb|\theCurrentPage|),
i.\,e. counted continuously from the first page): \theCurrentPage \\

\noindent CurrentPageLocal (\verb|\theCurrentPageLocal|),
i.\,e. counted continuously from the first page of the
current page numbering scheme): \theCurrentPageLocal \\

\noindent Last roman page (pagesLTS.roman): \lastpageref{pagesLTS.roman}{\hskip4em }
(There are \lastpageref{pagesLTS.roman.local} pages with roman numbers.)\\

\noindent Last Roman page (pagesLTS.Roman): \lastpageref{pagesLTS.Roman}{\hskip3em }
(There are \lastpageref{pagesLTS.Roman.local}~pages with Roman numbers:\\
\lastpages{Roman}{1}~pages in the first Roman sector
(\pageref{Roman}{\hskip3em }-\lastpageref{pagesLTS.Roman.1}{\hskip3em }),\\
\lastpages{Roman}{2}~pages in the second Roman sector
(\pageref{Roman2}{\hskip3em }-\lastpageref{pagesLTS.Roman.2}{\hskip3em }), and\\
\lastpages{Roman}{3}~pages in the third Roman sector
(\pageref{Roman3}{\hskip3em }-\lastpageref{pagesLTS.Roman.3}{\hskip3em }).\\

\noindent Last arabic page (pagesLTS.arabic): \lastpageref{pagesLTS.arabic}{\hskip5em }
(There are only \lastpageref{pagesLTS.arabic.local} pages with arabic numbers,
because an \verb|\addtocounter{page}{|\pagesLTSexampleArabic\verb|}| was used.)\\

\noindent Last fnsymbol page (pagesLTS.fnsymbol): \lastpageref{pagesLTS.fnsymbol} \\
(\verb|\lastpageref{pagesLTS.fnsymbol}| -- never
\verb|\pageref{pagesLTS.fnsymbol}|!)\\
(There are \lastpageref{pagesLTS.fnsymbol.local} pages with fnsymbol numbers.)\\

\noindent Last alph page (pagesLTS.alph): \lastpageref{pagesLTS.alph}{\hskip4em }
(There are only \lastpageref{pagesLTS.alph.local} pages with alph numbers,
because an \verb|\addtocounter{page}{|\pagesLTSexamplealph\verb|}| was used.)\\

\noindent Last Alph page (pagesLTS.Alph): \lastpageref{pagesLTS.Alph}{\hskip4em }
(There are \lastpageref{pagesLTS.Alph.local} pages with Alph numbers.)\\

\noindent Last page's \textit{name} (LastPage): \lastpageref{LastPage}\\

\noindent Very last page's \textit{name} (VeryLastPage): \lastpageref{VeryLastPage}\\
(\texttt{LastPage} and \texttt{VeryLastPage} are identical, unless
a package with output \linebreak
\verb|\AtEndDocument| after the \textsf{pageslts} package was added.)\\

\noindent Last page's \textit{number} (LastPages): \lastpageref{LastPages}{\hskip3em }
(=~total number of pages)\\

\lipsum[1-6]

\newpage

\verb|\lastpageref{pagesLTS.Roman}| does \textbf{not}
refer to this page (but there: \lastpageref{pagesLTS.Roman}),
because the option \texttt{pagecontinue=true}
was chosen. When a reference to this page is wanted,\\
\verb|\lastpageref{pagesLTS.Roman.2}| can be used: \lastpageref{pagesLTS.Roman.2}.\\

\bigskip
There are \lastpages{Roman}{2}~pages (\verb|\lastpages{Roman}{2}|) in this
second Roman sector.\\
The Roman page numbering scheme is continued later in section~\ref{Roman3})!

\newpage

\pagenumbering{alph}

\section{alph\label{alph}}

\noindent First page (\verb|\lastpageref{pagesLTS.0}|):
 \lastpageref{pagesLTS.0}\\

\noindent The page (\verb|\thepage|): \thepage \\

\noindent Current page (\verb|\theCurrentPage|),
i.\,e. counted continuously from the first page): \theCurrentPage \\

\noindent CurrentPageLocal (\verb|\theCurrentPageLocal|),
i.\,e. counted continuously from the first page of the
current page numbering scheme): \theCurrentPageLocal \\

\noindent Last roman page (pagesLTS.roman): \lastpageref{pagesLTS.roman}{\hskip4em }
(There are \lastpageref{pagesLTS.roman.local} pages with roman numbers.)\\

\noindent Last Roman page (pagesLTS.Roman): \lastpageref{pagesLTS.Roman}{\hskip3em }
(There are \lastpageref{pagesLTS.Roman.local}~pages with Roman numbers:\\
\lastpages{Roman}{1}~pages in the first Roman sector
(\pageref{Roman}{\hskip3em }-\lastpageref{pagesLTS.Roman.1}{\hskip3em }),\\
\lastpages{Roman}{2}~pages in the second Roman sector
(\pageref{Roman2}{\hskip3em }-\lastpageref{pagesLTS.Roman.2}{\hskip3em }), and\\
\lastpages{Roman}{3}~pages in the third Roman sector
(\pageref{Roman3}{\hskip3em }-\lastpageref{pagesLTS.Roman.3}{\hskip3em }).\\

\noindent Last arabic page (pagesLTS.arabic): \lastpageref{pagesLTS.arabic}{\hskip5em }
(There are only \lastpageref{pagesLTS.arabic.local} pages with arabic numbers,
because an \verb|\addtocounter{page}{|\pagesLTSexampleArabic\verb|}| was used.)\\

\noindent Last fnsymbol page (pagesLTS.fnsymbol): \lastpageref{pagesLTS.fnsymbol} \\
(\verb|\lastpageref{pagesLTS.fnsymbol}| -- never \verb|\pageref{pagesLTS.fnsymbol}|!)\\
(There are \lastpageref{pagesLTS.fnsymbol.local} pages with fnsymbol numbers.)\\

\noindent Last alph page (pagesLTS.alph): \lastpageref{pagesLTS.alph}{\hskip4em }
(There are only \lastpageref{pagesLTS.alph.local} pages with alph numbers,
because an \verb|\addtocounter{page}{|\pagesLTSexamplealph\verb|}| was used.)\\

\noindent Last Alph page (pagesLTS.Alph): \lastpageref{pagesLTS.Alph}{\hskip4em }
(There are \lastpageref{pagesLTS.Alph.local} pages with Alph numbers.)\\

\noindent Last page's \textit{name} (LastPage): \lastpageref{LastPage}\\

\noindent Very last page's \textit{name} (VeryLastPage): \lastpageref{VeryLastPage}\\
(\texttt{LastPage} and \texttt{VeryLastPage} are identical, unless
a package with output \linebreak
\verb|\AtEndDocument| after the \textsf{pageslts} package was added.)\\

\noindent Last page's \textit{number} (LastPages): \lastpageref{LastPages}{\hskip3em }
(=~total number of pages)\\

\lipsum[1-4]

\newpage

Without option \texttt{alphMult=ab} of the \textsf{pageslts} (and the help of
\textsc{Heiko Oberdiek's} \textsf{alphalph} package), after page
\textquotedblleft z\textquotedblright{} there would just appear a
\begin{quote}
\begin{verbatim}
LaTeX Error: Counter too large
See the LaTeX manual or LaTeX Companion for explanation.
You've lost some text.  Try typing  <return>  to proceed.
If that doesn't work, type  X <return>  to quit.
\end{verbatim}
\end{quote}
Now the page numbers are continued aa, ab, ac,\ldots\ (aa, bb, cc,\ldots\ is
also possible, see the \textsf{pageslts} documentation).\\
To demonstrate this, we add a\\
\verb|\addtocounter{page}{|\pagesLTSexamplealph\verb|}|\\
in the source code here.

\addtocounter{page}{\pagesLTSexamplealph}

\bigskip

\lipsum[1-18]

\newpage

\pagenumbering{Roman}

\section{Roman - third time!\label{Roman3}}

The page number would start with
\textquotedblleft I\textquotedblright{} again -- but
for the \textsf{pageslts} package (with option \texttt{pagecontinue=true},
or with option just \texttt{pagecontinue}, or even just
\nolinebreak{with\textbf{out}} option \texttt{pagecontinue=false}).
This package remembered the
(\arabic{pagesLTS.double.Roman}-1)\footnote{OK, here %
you have to compute this value for yourself, but %
subtracting one should be managable for \TeX nicians.} pages already
done in Roman output, and therefore continues with page
\textquotedblleft \thepage \textquotedblright .\\
If you want to start with \textquotedblleft I\textquotedblright{} all
over again, you will have (at least) two pages with the same name,
but nevertheless you can do this by using option
\texttt{pagecontinue=false} instead of \texttt{pagecontinue=true}
(not demonstrated here).\\

\noindent First page (\verb|\lastpageref{pagesLTS.0}|):
 \lastpageref{pagesLTS.0}\\

\noindent The page (\verb|\thepage|): \thepage \\

\noindent Current page (\verb|\theCurrentPage|),
i.\,e. counted continuously from the first page): \theCurrentPage \\

\noindent CurrentPageLocal (\verb|\theCurrentPageLocal|),
i.\,e. counted continuously from the first page of the
current page numbering scheme): \theCurrentPageLocal \\

\noindent Last roman page (pagesLTS.roman): \lastpageref{pagesLTS.roman}{\hskip4em }
(There are \lastpageref{pagesLTS.roman.local} pages with roman numbers.)\\

\noindent Last Roman page (pagesLTS.Roman): \lastpageref{pagesLTS.Roman}{\hskip3em }
(There are \lastpageref{pagesLTS.Roman.local}~pages with Roman numbers:\\
\lastpages{Roman}{1}~pages in the first Roman sector
(\pageref{Roman}{\hskip3em }-\lastpageref{pagesLTS.Roman.1}{\hskip3em }),\\
\lastpages{Roman}{2}~pages in the second Roman sector
(\pageref{Roman2}{\hskip3em }-\lastpageref{pagesLTS.Roman.2}{\hskip3em }), and\\
\lastpages{Roman}{3}~pages in the third Roman sector
(\pageref{Roman3}{\hskip3em }-\lastpageref{pagesLTS.Roman.3}{\hskip3em }).\\

\noindent Last arabic page (pagesLTS.arabic): \lastpageref{pagesLTS.arabic}{\hskip5em }
(There are only \lastpageref{pagesLTS.arabic.local} pages with arabic numbers,
because an \verb|\addtocounter{page}{|\pagesLTSexampleArabic\verb|}| was used.)\\

\noindent Last fnsymbol page (pagesLTS.fnsymbol): \lastpageref{pagesLTS.fnsymbol} \\
(\verb|\lastpageref{pagesLTS.fnsymbol}| -- never \verb|\pageref{pagesLTS.fnsymbol}|!)\\
(There are \lastpageref{pagesLTS.fnsymbol.local} pages with fnsymbol numbers.)\\

\noindent Last alph page (pagesLTS.alph): \lastpageref{pagesLTS.alph}{\hskip4em }
(There are only \lastpageref{pagesLTS.alph.local} pages with alph numbers,
because an \verb|\addtocounter{page}{|\pagesLTSexamplealph\verb|}| was used.)\\

\noindent Last Alph page (pagesLTS.Alph): \lastpageref{pagesLTS.Alph}{\hskip4em }
(There are \lastpageref{pagesLTS.Alph.local} pages with Alph numbers.)\\

\noindent Last page's \textit{name} (LastPage): \lastpageref{LastPage}\\

\noindent Very last page's \textit{name} (VeryLastPage): \lastpageref{VeryLastPage}\\
(\texttt{LastPage} and \texttt{VeryLastPage} are identical, unless
a package with output \linebreak
\verb|\AtEndDocument| after the \textsf{pageslts} package was added.)\\

\noindent Last page's \textit{number} (LastPages): \lastpageref{LastPages}{\hskip3em }
(=~total number of pages)\\

\lipsum[1-3]

\newpage

\lastpageref{pagesLTS.Roman}
(\verb|\lastpageref{pagesLTS.Roman}|)
\textbf{does} refers to this page, because the option
\texttt{pagecontinue=true} was chosen. Also\\
\verb|\lastpageref{pagesLTS.Roman.3}| can be used: \lastpageref{pagesLTS.Roman.3}.\\

\bigskip

There are \lastpages{Roman}{3}~pages (\verb|\lastpages{Roman}{3}|) in this
third Roman sector.\\

\newpage

\pagenumbering{Alph}

\section{Alph}

\noindent First page (\verb|\lastpageref{pagesLTS.0}|):
 \lastpageref{pagesLTS.0}\\

\noindent The page (\verb|\thepage|): \thepage \\

\noindent Current page (\verb|\theCurrentPage|),
i.\,e. counted continuously from the first page): \theCurrentPage \\

\noindent CurrentPageLocal (\verb|\theCurrentPageLocal|),
i.\,e. counted continuously from the first page of the
current page numbering scheme): \theCurrentPageLocal \\

\noindent Last roman page (pagesLTS.roman): \lastpageref{pagesLTS.roman}{\hskip4em }
(There are \lastpageref{pagesLTS.roman.local} pages with roman numbers.)\\

\noindent Last Roman page (pagesLTS.Roman): \lastpageref{pagesLTS.Roman}{\hskip3em }
(There are \lastpageref{pagesLTS.Roman.local}~pages with Roman numbers:\\
\lastpages{Roman}{1}~pages in the first Roman sector
(\pageref{Roman}{\hskip3em }-\lastpageref{pagesLTS.Roman.1}{\hskip3em }),\\
\lastpages{Roman}{2}~pages in the second Roman sector
(\pageref{Roman2}{\hskip3em }-\lastpageref{pagesLTS.Roman.2}{\hskip3em }), and\\
\lastpages{Roman}{3}~pages in the third Roman sector
(\pageref{Roman3}{\hskip3em }-\lastpageref{pagesLTS.Roman.3}{\hskip3em }).\\

\noindent Last arabic page (pagesLTS.arabic): \lastpageref{pagesLTS.arabic}{\hskip5em }
(There are only \lastpageref{pagesLTS.arabic.local} pages with arabic numbers,
because an \verb|\addtocounter{page}{|\pagesLTSexampleArabic\verb|}| was used.)\\

\noindent Last fnsymbol page (pagesLTS.fnsymbol): \lastpageref{pagesLTS.fnsymbol} \\
(\verb|\lastpageref{pagesLTS.fnsymbol}| -- never \verb|\pageref{pagesLTS.fnsymbol}|!)\\
(There are \lastpageref{pagesLTS.fnsymbol.local} pages with fnsymbol numbers.)\\

\noindent Last alph page (pagesLTS.alph): \lastpageref{pagesLTS.alph}{\hskip4em }
(There are only \lastpageref{pagesLTS.alph.local} pages with alph numbers,
because an \verb|\addtocounter{page}{|\pagesLTSexamplealph\verb|}| was used.)\\

\noindent Last Alph page (pagesLTS.Alph): \lastpageref{pagesLTS.Alph}{\hskip4em }
(There are \lastpageref{pagesLTS.Alph.local} pages with Alph numbers.)\\

\noindent Last page's \textit{name} (LastPage): \lastpageref{LastPage}\\

\noindent Very last page's \textit{name} (VeryLastPage): \lastpageref{VeryLastPage}\\
(\texttt{LastPage} and \texttt{VeryLastPage} are identical, unless
a package with output \linebreak
\verb|\AtEndDocument| after the \textsf{pageslts} package was added.)\\

\noindent Last page's \textit{number} (LastPages): \lastpageref{LastPages}{\hskip3em }
(=~total number of pages)\\

\lipsum[1-3]

Without option \texttt{alphMulti=AB} of the \textsf{pageslts} (and the help of
\textsc{Heiko Oberdiek's} \textsf{alphalph} package), after page
\textquotedblleft Z\textquotedblright{} there would just appear a
\begin{quote}
\begin{verbatim}
LaTeX Error: Counter too large
See the LaTeX manual or LaTeX Companion for explanation.
You've lost some text.  Try typing  <return>  to proceed.
If that doesn't work, type  X <return>  to quit.
\end{verbatim}
\end{quote}
Now the page numbers are continued AA, AB, AC,\ldots\ (AA, BB, CC,\ldots\ is
also possible, see the \textsf{pageslts} documentation).\\
This is not demonstrated here, but see section~\ref{alph}.

\newpage

\section{The End}

\noindent First page (\verb|\lastpageref{pagesLTS.0}|):
 \lastpageref{pagesLTS.0}\\

\noindent The page (\verb|\thepage|): \thepage \\

\noindent Current page (\verb|\theCurrentPage|),
i.\,e. counted continuously from the first page): \theCurrentPage \\

\noindent CurrentPageLocal (\verb|\theCurrentPageLocal|),
i.\,e. counted continuously from the first page of the
current page numbering scheme): \theCurrentPageLocal \\

\noindent Last roman page (pagesLTS.roman): \lastpageref{pagesLTS.roman}{\hskip4em }
(There are \lastpageref{pagesLTS.roman.local} pages with roman numbers.)\\

\noindent Last Roman page (pagesLTS.Roman): \lastpageref{pagesLTS.Roman}{\hskip3em }
(There are \lastpageref{pagesLTS.Roman.local}~pages with Roman numbers:\\
\lastpages{Roman}{1}~pages in the first Roman sector
(\pageref{Roman}{\hskip3em }-\lastpageref{pagesLTS.Roman.1}{\hskip3em }),\\
\lastpages{Roman}{2}~pages in the second Roman sector
(\pageref{Roman2}{\hskip3em }-\lastpageref{pagesLTS.Roman.2}{\hskip3em }), and\\
\lastpages{Roman}{3}~pages in the third Roman sector
(\pageref{Roman3}{\hskip3em }-\lastpageref{pagesLTS.Roman.3}{\hskip3em }).\\

\noindent Last arabic page (pagesLTS.arabic): \lastpageref{pagesLTS.arabic}{\hskip5em }
(There are only \lastpageref{pagesLTS.arabic.local} pages with arabic numbers,
because an \verb|\addtocounter{page}{|\pagesLTSexampleArabic\verb|}| was used.)\\

\noindent Last fnsymbol page (pagesLTS.fnsymbol): \lastpageref{pagesLTS.fnsymbol} \\
(\verb|\lastpageref{pagesLTS.fnsymbol}| -- never \verb|\pageref{pagesLTS.fnsymbol}|!)\\
(There are \lastpageref{pagesLTS.fnsymbol.local} pages with fnsymbol numbers.)\\

\noindent Last alph page (pagesLTS.alph): \lastpageref{pagesLTS.alph}{\hskip4em }
(There are only \lastpageref{pagesLTS.alph.local} pages with alph numbers,
because an \verb|\addtocounter{page}{|\pagesLTSexamplealph\verb|}| was used.)\\

\noindent Last Alph page (pagesLTS.Alph): \lastpageref{pagesLTS.Alph}{\hskip4em }
(There are \lastpageref{pagesLTS.Alph.local} pages with Alph numbers.)\\

\noindent Last page's \textit{name} (LastPage): \lastpageref{LastPage}\\

\noindent Very last page's \textit{name} (VeryLastPage): \lastpageref{VeryLastPage}\\
(\texttt{LastPage} and \texttt{VeryLastPage} are identical, unless
a package with output \linebreak
\verb|\AtEndDocument| after the \textsf{pageslts} package was added.)\\

\noindent Last page's \textit{number} (LastPages): \lastpageref{LastPages}{\hskip3em }
(=~total number of pages)\\

\medskip

\noindent Page \thepage\ (\theCurrentPage; local: \theCurrentPageLocal) of %
\lastpageref{pagesLTS.roman} (\lastpageref{pagesLTS.roman.local}) + %
\lastpageref{pagesLTS.Roman} (\lastpageref{pagesLTS.Roman.local}) + %
\lastpageref{pagesLTS.arabic} (\lastpageref{pagesLTS.arabic.local}) + %
\lastpageref{pagesLTS.fnsymbol} (\lastpageref{pagesLTS.fnsymbol.local}) + %
\lastpageref{pagesLTS.alph} (\lastpageref{pagesLTS.alph.local}) + %
\lastpageref{pagesLTS.Alph} (\lastpageref{pagesLTS.Alph.local}) = %
\lastpageref{LastPages} pages.

\end{document}
%</example>
%    \end{macrocode}
%
% \pagebreak
%
% \StopEventually{}
%
% \section{The implementation}
%
% (This and the source code of the example file are the reasons
% for printing the documentation in landscape format instead of portrait.)\\
%
% We start off by checking that we are loading into \LaTeXe{} and
% announcing the name and version of this package.
%
%    \begin{macrocode}
%<*package>
%    \end{macrocode}
%
%    \begin{macrocode}
\NeedsTeXFormat{LaTeX2e}[2015/01/01]
\ProvidesPackage{pageslts}[2015/12/21 v1.2f
            Refers to special pages' numbers/names (HMM)]

%    \end{macrocode}
%
% A short description of the \xpackage{pageslts} package:
%
%    \begin{macrocode}
%% Allows for things like\\
%% |Page \thepage\ (\theCurrentPage; local: \theCurrentPageLocal) of %
%%  \lastpageref{pagesLTS.roman}(\lastpageref{pagesLTS.roman.local}) + %
%%  \lastpageref{pagesLTS.Roman}(\lastpageref{pagesLTS.Roman.local}) + %
%%  \lastpageref{pagesLTS.arabic}(\lastpageref{pagesLTS.arabic.local}) + %
%%  \lastpageref{pagesLTS.fnsymbol}(\lastpageref{pagesLTS.fnsymbol.local}) + %
%%  \lastpageref{pagesLTS.alph}(\lastpageref{pagesLTS.alph.local}) + %
%%  \lastpageref{pagesLTS.Alph}(\lastpageref{pagesLTS.Alph.local}) = %
%%  \lastpageref{LastPages} pages.|\\
%% to get\\
%% 'Page d (57; local: 4) of ii(2) + XX(20) + *(1) + 30(30) + e(5) + C(3) = 61 pages.'.

%    \end{macrocode}
%
% The package is now named `pageslts' but had been named `pagesLTS' before.
% \LaTeX\ does not load a package two times, but it is case sensitive,
% i.\,e. \emph{would} load `pageslts' and `pagesLTS', resulting in
% possible problems. Therefore we check for a loaded `pagesLTS' package
% (assuming |\pagesLTS@loaded| is not |p@gesLTSlo@ded| and not defined to
% |p@gesLTSlo@ded| by another package). This check might work only for
% the second and following compilation runs.
%
%    \begin{macrocode}
\@ifpackageloaded{pagesLTS}{%
  \def\pagesLTS@doubleload{p@gesLTSlo@ded}
  \ifx\pagesLTS@loaded\pagesLTS@doubleload%
    \PackageError{pageslts}{Requested pagesLTS instead of pageslts}{%
      You have requested package `pagesLTS'. This package is now named `pageslts'.\MessageBreak%
      Requesting `pagesLTS' as well as `pageslts' leads to loading the same package\MessageBreak%
      twice (and results in errors). Loading of `pageslts' will therefore be\MessageBreak%
      abandoned now. To fix this problem, please look in the \jobname.log file for\MessageBreak%
      LaTeX Warning: You have requested package `pagesLTS',\MessageBreak%
      \space \space \space \space \space \space \space \space \space \space \space \space \space%
      \space \space but the package provides `pageslts'.\MessageBreak%
      and change the package call in the calling style/document.\MessageBreak%
      If a style file made the mistake, please inform its maintainer and/or the\MessageBreak%
      maintainer of the pageslts package about it (after making sure you have\MessageBreak%
      the recent version of that style file).\MessageBreak%
      Thank you and sorry for the inconvenience!\MessageBreak%
     }
    \endinput
  \fi
  }{% \relax
   }

\gdef\pagesLTS@loaded{p@gesLTSlo@ded}

%    \end{macrocode}
%
% For |\ltx@ifpackageloaded| the \xpackage{ltxcmds} package is needed,
% also by \textsc{Heiko Oberdiek} (see subsection~\ref{ss:Downloads}):
%
%    \begin{macrocode}
\RequirePackage{ltxcmds}[2011/11/09]%        v1.22
%    \end{macrocode}
%
% For its |\AfterLastShipout| (as well as its |\AtEndAfterFileList|) command
% we need the \xpackage{atveryend} package by \textsc{Heiko Oberdiek}
% (see subsection~\ref{ss:Downloads}):
%
%    \begin{macrocode}
\RequirePackage{atveryend}[2011/06/30]%      v1.8
\AtBeginDocument{%
  %% https://tex.stackexchange.com/a/45654/6865
  %% and https://tex.stackexchange.com/q/261445/6865
  \def\pagesLTS@etb{etoolbox.sty}%
  \let\pagesLTS@ave\@empty%
  \@for\@pageslts@currname:=\@filelist\do{%
     \ifx\@pageslts@currname\pagesLTS@etb\relax%
       \def\pagesLTS@ave{atveryend.sty}%
     \fi%
     \ifx\@pageslts@currname\pagesLTS@ave%
       \@ifpackagelater{atveryend}{2015/08/01}{%\relax
          \PackageInfo{pageslts}{etoolbox package loaded before atveryend package\MessageBreak%
            With atveryend version 1.8 of 2011/06/30 this caused a problem,\MessageBreak%
            which is probably solved in the new version used here.\MessageBreak%
            To play it safe, you could load the atveryend package\MessageBreak%
            before the etoolbox package.\MessageBreak%
            }%
        }{% else
          \PackageError{pageslts}{etoolbox package loaded before atveryend package}{%
            You can use both packages, but when etoolbox is loaded before atveryend,\MessageBreak%
            atveryend no longer detects \string\enddocument\space\MessageBreak% 
            and hook "AtVeryVeryEnd" is not executed.\MessageBreak%
            Just load the atveryend package before the etoolbox package!\MessageBreak%
            }%
         }%
     \fi%
   }%
  }

%    \end{macrocode}
%
% For its |\EveryShipout| command we need the \xpackage{everyshi}
% package by \textsc{Martin Schr\"{o}der} (see subsection~\ref{ss:Downloads}):
%
%    \begin{macrocode}
\RequirePackage{everyshi}[2001/05/15]%       v3.00
%    \end{macrocode}
%
% For its |\LetLtxMacro| command we need the \xpackage{letltxmacro}
% package by \textsc{Heiko Oberdiek} (see subsection~\ref{ss:Downloads}):
%
%    \begin{macrocode}
\RequirePackage{letltxmacro}[2010/09/02]%    v1.4
%    \end{macrocode}
%
% For the handling of the options we need the \xpackage{kvoptions}
% package also by \textsc{Heiko Oberdiek} (see subsection~\ref{ss:Downloads}):
%
%    \begin{macrocode}
\RequirePackage{kvoptions}[2011/06/30]%      v3.11
%    \end{macrocode}
%
% The \xpackage{undolabl} package of \textsc{H.-Martin M\"{u}nch} (i.\,e. myself),
% with code from \textsc{Ulrich Diez}, (see subsection~\ref{ss:Downloads})
% is needed to overwrite labels, when the same page numbering scheme
% is used twice (or even more often).
%
%    \begin{macrocode}
\RequirePackage{undolabl}[2015/03/29]%       v1.0l
%    \end{macrocode}
%
% We use the \xpackage{rerunfilecheck} package by \textsc{Heiko Oberdiek}
% to make sure that the user gets and sees the rerun warnings (if any).
%
%    \begin{macrocode}
\RequirePackage{rerunfilecheck}[2011/04/15]% v1.7
%    \end{macrocode}
%
% We must not forget to give the source of |Prelim@EveryShipout|:
%
%    \begin{macrocode}
%% pageslts package uses Prelim@EveryShipout code from the
%% prelim2e package [2009/05/29 v1.3] by Martin Schr\"{o}der, thanks!

%    \end{macrocode}
%
% About the \xpackage{prelim2e} package by \textsc{Martin Schr\"{o}der}
% see subsection~\ref{ss:Downloads}.\\
%
% A last information for the user(s):
%
%    \begin{macrocode}
%% pageslts may work with earlier versions of LaTeX2e and those packages,
%% but this was not tested. Please consider updating your LaTeX
%% and packages to the most recent version (if they are not already
%% the most recent version).
%    \end{macrocode}
% See subsection~\ref{ss:Downloads} about how to get them.\\
%
% \pagebreak
%
% The very old version~2.0 (and earlier) of the \xpackage{endfloat}
% package actually redefined the |\enddocument| command,
% and so interfered drastically with the \LaTeXe\ commands which
% make use of |\AtEndDocument|. Newer versions of \xpackage{endfloat}
% exists (at the time of writing this documentation: v2.5d as of 2011/12/25)
% in modern documentation form, which are available from \url{https://www.CTAN.org}
% (see subsection~\ref{ss:Downloads}).
% A~note is placed here, and later it is checked whether a (very) old
% \xpackage{endfloat} package is in use. If it is, a warning or even
% an error message is given, depending on \xpackage{endfloat} version.
% This assumes, that the old versions of \xpackage{endfloat} at least gave
% a version date, of course.
%
%    \begin{macrocode}
%% The recent version of the endfloat package is v2.5d as of 2011/12/25.
%% The pageslts package is not fully compatible with version 2.0
%% (and earlier) of the endfloat package, because those versions
%% redefined the \enddocument command.

%    \end{macrocode}
%
% The options are introduced:
%
%    \begin{macrocode}
\SetupKeyvalOptions{family = pagesLTS,prefix = pagesLTS@}
\DeclareBoolOption[true]{pagecontinue}% \pagesLTS@pagecontinue
\DeclareStringOption[ab]{alphMult}
\DeclareStringOption[AB]{AlphMulti}
\DeclareBoolOption[true]{romanMult}
\DeclareBoolOption[true]{RomanMulti}
\DeclareBoolOption[true]{fnsymbolmult}

\ProcessKeyvalOptions*

%    \end{macrocode}
%
% For comparisons, zero, one, two and three are defined (|\z@|, |\@ne| and so on
% do not work for this).
%
%    \begin{macrocode}
\def\pagesLTS@zero{0}
\def\pagesLTS@one{1}
\def\pagesLTS@two{2}
\def\pagesLTS@three{3}

%    \end{macrocode}
%
% The traditional behaviour is a reset of the page number to one,
% each time the page numbering scheme changes. The option
% |pagecontinue| changes this to a continuation with the
% number/name following the last page number/name of the same
% page numbering scheme. The user is informed accordingly.
%
%    \begin{macrocode}
\ifpagesLTS@pagecontinue%
  \PackageInfo{pageslts}{Option pagecontinue enabled\MessageBreak%
    (maybe by default):\MessageBreak%
    The pageslts package will continue the page numbering,\MessageBreak%
    when the same page numbering scheme is used twice.\MessageBreak%
    If you do not want this, call pageslts with option\MessageBreak%
    pagecontinue=false (or use \string\setcounter{page}=1).\MessageBreak%
  }
\else
%    \end{macrocode}
%
% \pagebreak
%
%    \begin{macrocode}
  \PackageWarningNoLine{pageslts}{Option pagecontinue is false:\MessageBreak%
    The pageslts package was used, but the option\MessageBreak%
    pagecontinue was set to false.\MessageBreak%
    If you want the page numbers to be continued,\MessageBreak%
    when the same page numbering scheme is used twice,\MessageBreak%
    please call pageslts with option pagecontinue=true,\MessageBreak%
    otherwise the page number is reset to one each time\MessageBreak%
    the page numbering scheme is changed.\MessageBreak%
    For details please see the documentation%
  }
\fi

%    \end{macrocode}
%
% The page number printed in \texttt{alph} or in \texttt{Alph}
% page numbering scheme had to be $>0$ and \nolinebreak{$<27$.}\ %
% Now the \xpackage{alphalph} package allows to extend the
% numbering scheme (not only for pages). Because some users
% prefer aa, ab, ac, ad,\ldots\ and some aa, bb, cc, dd,\ldots,
% both schemes can be chosen via the options.
% The \texttt{fnsymbol} page numbering scheme was restricted to values
% $>0$ and \nolinebreak{$<10$.}\ The \xpackage{alphalph} package allows
% to extend this page numbering scheme, too. Option \texttt{fnsymbolmult}
% can be chosen with the \xpackage{pageslts} package.
% If no extension is wished (or another extension is wished
% and implemented manually), \xpackage{pageslts} can be called with
% options set to 0 (zero) and false: |alphMult=0|, |AlphMulti=0|,
% |fnsymbolmult=false|.
%
%    \begin{macrocode}
\def\pagesLTS@ab{ab}
\def\pagesLTS@bb{bb}
\def\pagesLTS@ABi{AB}
\def\pagesLTS@BBi{BB}

\ifx\pagesLTS@alphMult\pagesLTS@ab%
\else
  \ifx\pagesLTS@alphMult\pagesLTS@bb%
  \else
    \ifx\pagesLTS@alphMult\pagesLTS@zero%
      \PackageWarningNoLine{pageslts}{%
        Option alphMult=0 found:\MessageBreak%
        The pageslts package was used, but the option\MessageBreak%
        alphMult was set to 0 (zero).\MessageBreak%
        If you want the page numbers to be extended\MessageBreak%
        after z, you have to organize this yourself now.\MessageBreak%
        For automatic continuation, please use the\MessageBreak%
        alphalph package and call pageslts\MessageBreak%
        with option alphMult=ab (for aa, ab, ac, ad,...) or\MessageBreak%
        with option alphMult=bb (for aa, bb, cc, dd,...).\MessageBreak%
        For details please see the documentation!%
       }
    \else
      \PackageError{pageslts}{Unknown option value}{%
        The pageslts package was used with option\MessageBreak%
        alphMult= \pagesLTS@alphMult . Only values\MessageBreak%
        ab, bb, and 0 (zero) are valid.\MessageBreak%
        The default ab is set.\MessageBreak%
        For details please see the documentation!%
       }
      \setkeys{pagesLTS}{alphMult=ab}
    \fi
  \fi
\fi

\ifx\pagesLTS@AlphMulti\pagesLTS@ABi%
\else
  \ifx\pagesLTS@AlphMulti\pagesLTS@BBi%
  \else
    \ifx\pagesLTS@AlphMulti\pagesLTS@zero%
      \PackageWarningNoLine{pageslts}{%
        Option AlphMulti=0 found:\MessageBreak%
        The pageslts package was used, but the option\MessageBreak%
        AlphMulti was set to 0 (zero).\MessageBreak%
        If you want the page numbers to be extended\MessageBreak%
        after Z, you have to organize this yourself now.\MessageBreak%
        For automatic continuation, please use the\MessageBreak%
        alphalph package and call pageslts\MessageBreak%
        with option AlphMulti=AB (for AA, AB, AC, AD,...) or\MessageBreak%
        with option AlphMulti=BB (for AA, BB, CC, DD,...).\MessageBreak%
        For details please see the documentation!%
       }
    \else
      \PackageError{pageslts}{Unknown option value}{%
        The pageslts package was used with option\MessageBreak%
        AlphMulti= \pagesLTS@AlphMulti . Only values\MessageBreak%
        AB, BB, and 0 (zero) are valid.\MessageBreak%
        The default AB is set.\MessageBreak%
        For details please see the documentation!%
       }
      \setkeys{pagesLTS}{AlphMulti=AB}
    \fi
  \fi
\fi

%    \end{macrocode}
%
% If |alph| or |Alph| or |fnsymbol| page numbers shall be continued,
% the \xpackage{alphalph} package is required.
%
%    \begin{macrocode}
\ifx\pagesLTS@alphMult\pagesLTS@zero% \relax
\else \RequirePackage{alphalph}[2011/05/13]% v2.4
\fi

\ifx\pagesLTS@AlphMulti\pagesLTS@zero% \relax
\else \RequirePackage{alphalph}[2011/05/13]% v2.4
\fi

\ifpagesLTS@fnsymbolmult%
  \RequirePackage{alphalph}[2011/05/13]% v2.4
\fi

%    \end{macrocode}
%
% For the roman page numbering scheme, it is just the
% choice of an extension by \xpackage{pageslts} or not.
%
%    \begin{macrocode}
\ifpagesLTS@romanMult%
  \PackageInfo{pageslts}{Option romanMult enabled\MessageBreak%
    (maybe by default):\MessageBreak%
    The pageslts package will extend the page numbering\MessageBreak%
    of the roman scheme below i with\MessageBreak%
     0, -i, -ii, -iii, -iv,...\MessageBreak%
    If you do not want this, call pageslts with option\MessageBreak%
    romanMult=false.\MessageBreak%
  }
\else
  \PackageWarningNoLine{pageslts}{Option romanMult is set to false:\MessageBreak%
  The pageslts package was used, but the option\MessageBreak%
  romanMult was set to false.\MessageBreak%
  If you want the page numbering of the roman scheme\MessageBreak%
  to be extended below i,\MessageBreak%
  please call pageslts with option romanMult=true,\MessageBreak%
  otherwise zero and negative page numbers of the\MessageBreak%
  roman scheme will need to be defined otherwise.\MessageBreak%
  For details please see the documentation!\MessageBreak%
  }
\fi

%    \end{macrocode}
%
% Same for the Roman page numbering scheme.
%
%    \begin{macrocode}
\ifpagesLTS@RomanMulti%
  \PackageInfo{pageslts}{Option RomanMulti enabled\MessageBreak%
    (maybe by default):\MessageBreak%
    The pageslts package will extend the page numbering\MessageBreak%
    of the Roman scheme below I with\MessageBreak%
    0, -I, -II, -III, -IV,...\MessageBreak%
    If you do not want this, call pageslts with option\MessageBreak%
    RomanMulti=false.\MessageBreak%
   }
\else
  \PackageWarningNoLine{pageslts}{Option RomanMulti is set to false:\MessageBreak%
    The pageslts package was used, but the option\MessageBreak%
    RomanMulti was set to false.\MessageBreak%
    If you want the page numbering of the Roman scheme\MessageBreak%
    to be extended below i,\MessageBreak%
    please call pageslts with option RomanMulti=true,\MessageBreak%
    otherwise zero and negative page numbers of the\MessageBreak%
    Roman scheme will need to be defined otherwise.\MessageBreak%
    For details please see the documentation!\MessageBreak%
   }
\fi

%    \end{macrocode}
%
% For the footnotesymbol page numbering scheme, it is also
% just the choice of a extension by \xpackage{pageslts} or not.
%
%    \begin{macrocode}
\ifpagesLTS@fnsymbolmult%
  \PackageInfo{pageslts}{Option fnsymbolmult enabled\MessageBreak%
    (maybe by default):\MessageBreak%
    The pageslts package will extend the page numbering\MessageBreak%
    of the footnotesymbol scheme using the alphalph\MessageBreak%
    package.\MessageBreak%
    If you do not want this, call pageslts with option\MessageBreak%
    fnsymbolmult=false.\MessageBreak%
   }
\else
  \PackageWarningNoLine{pageslts}{%
    Option fnsymbolmult is set to false:\MessageBreak%
    The pageslts package was used, but the option\MessageBreak%
    fnsymbolmult was set to false.\MessageBreak%
    If you want the page numbering of the footnotesymbol\MessageBreak%
    scheme to be extended using the alphalph package,\MessageBreak%
    please call pageslts with option fnsymbolmult=true,\MessageBreak%
    otherwise page numbers of the footnotesymbol scheme\MessageBreak%
    greater than nine will need to be defined otherwise.\MessageBreak%
    For details please see the documentation!\MessageBreak%
   }
\fi

%    \end{macrocode}
%
% Now defining some variables, place-holders, and abbreviations:
%
%    \begin{macrocode}
\def\pagesLTS@pnc{0}
\def\pagesLTS@called{0}
\def\pagesLTS@fns{fnsymbol}
\def\pagesLTS@alph{alph}
\def\pagesLTS@Alph{Alph}
\def\pagesLTS@rerun{0}
\def\pagesLTS@eso{0}
\def\pagesLTS@esov{0}
\def\lastpageref{\lastpagereftxt}
\def\pagesLTS@undolable{none}
\def\pncmissing{0}
\def\pagesLTS@SK{0}
\def\pagesLTS@messageNPN{%
 The pageslts package was used, but\MessageBreak%
 \string\pagenumbering \MessageBreak%
 was not called at the beginning of the document\MessageBreak%
 (maybe earlier or later).\MessageBreak%
 Please use \string\pagenumbering \MessageBreak%
 (with an argument like arabic, roman, Roman,\MessageBreak%
 fnsymbol, alph, or Alph) at the beginning\MessageBreak%
 of your document! Otherwise your document\MessageBreak%
 will probably compile, but the pageslts\MessageBreak%
 package might not be able to get all labels\MessageBreak%
 for the references to the respective pages\MessageBreak%
 right.\MessageBreak%
 }

%    \end{macrocode}
%
% It is checked whether writing to files is allowed.
% The \xpackage{pageslts} package cannot be used without that!
% Some packages (e.\,g. \xpackage{tikz} and \xpackage{selectp}) sometimes prevent
% the output to the \xfile{aux} file. In that case a warning or an error message
% is issued. This is no problem as long as there is/was another compilation run
% where the labels can/could be processed via the \xfile{aux} file.\\
% If it is allowed to write to the \xfile{aux} file, we define
% |\pagesLTS@loaded| as |p@gesLTSnotlo@ded| for the next compilation run.
%
%    \begin{macrocode}
\AtBeginDocument{%
  \if@filesw%
%    \end{macrocode}
%
% \noindent then |\relax|.\newline
% When writing to files is not allowed, nothing can be done.
% But when the labels were already processed via the \xfile{aux} file,
% nothing needs to be done (if enough compilation runs have been done before).
%
%    \begin{macrocode}
  \else%
    \@ifpackageloaded{tikz}{%
      \PackageWarning{pageslts}%
        {It was not allowed to write to an\MessageBreak%
         .aux file. This package does not work without access\MessageBreak%
         to an .aux file.\MessageBreak%
         It is OK if the .aux file was already updated\MessageBreak%
         by a previouse compiler run\MessageBreak%
         and would not have changed anyway.\MessageBreak%
        }%
    }{\PackageError{pageslts}{No writing to auxiliary file allowed}%
        {It was not allowed to write to an .aux file.\MessageBreak%
         This package does not work without access to an .aux file.\MessageBreak%
         Press Ctrl+Z to exit.\MessageBreak%
         But it is OK if the .aux file was already updated\MessageBreak%
         by previouse compiler runs\MessageBreak%
         and would not have changed anyway.\MessageBreak%
         (In that case just press Enter or Return\MessageBreak%
         to continue the compilation.)\MessageBreak%
        }%
     }%
  \fi%
  }

%    \end{macrocode}
%
% \begin{macro}{\pagenumbering}
% \indent \qquad To keep the original meaning of |\pagenumbering|:
%
%    \begin{macrocode}
\LetLtxMacro{\OrigPagenumbering}{\pagenumbering}

%    \end{macrocode}
% \end{macro}
%
% Defining some new counters (and doing related things):
%
%    \begin{macrocode}
\newcounter{CurrentPage}
\setcounter{CurrentPage}{1}
\def\theCurrentPageLocal{\arabic{pagesLTS.current.local.\pagesLTS@pnc}}
%    \end{macrocode}
%
% The counter |pagesLTS.pagenr| is for saving the total page
% number of the last page in the \xfile{.aux} file.
%
%    \begin{macrocode}
\newcounter{pagesLTS.pagenr}
%    \end{macrocode}
%
% While generally
% |\pagesLTS@ifcounter{pagesLTS.current.local.\pagesLTS@pnc}|
% is used, for the beginning of the document
% |pagesLTS.current.local.0| is predefined.
% (A~|\pagesLTS@ifcounter{pagesLTS.current.local.\pagesLTS@pnc}|
% could be used for this, too, but we know that |pagesLTS.current.local.0|
% was not defined, so we can just do the definition here.)
% And the first local page gets the number one.
%
%    \begin{macrocode}
\newcounter{pagesLTS.current.local.0}
\setcounter{pagesLTS.current.local.0}{1}
%    \end{macrocode}
%
% And the same again for |pagesLTS.pnc.0|.
%
%    \begin{macrocode}
\newcounter{pagesLTS.pnc.0}

%    \end{macrocode}
%
% \phantomsection
% \begin{macro}{\xroman}\label{xroman}
% When |\roman{...}| is used with a value $<1$, \LaTeX\ just ignores this
% (see subsection~\ref{ss:overflow}). Here we provide a command
% |\xroman{...}| \nolinebreak{(e\textbf{x}panded} roman),
% which gives the usual |\roman| numbers (i, ii, iii, iv,\ldots)
% for positive values,
% $-\left\vert \ldots \right\vert $ (i.\,e. -i, -ii, -iii, -iv,\ldots)
% for negative values, and 0 for all other values (which should be zero).
%
%    \begin{macrocode}
\newcommand{\xroman}[1]{%
  \ifnum\value{#1}>0%
    \roman{#1}%
  \else%
    \ifnum\value{#1}<0%
%    \end{macrocode}
%
% |\arabic{#1}| gives the arabic number of argument |#1|,
% which is negative here (for example $-7$),
% \textquotedblleft |-|\textquotedblright{} puts another minus
% sign in front of it (for example $--7$), |\number| removes
% all unnecessary preceding zeros, plus and minus signs
% (for example $7$), |\romannumeral| turns it into a |roman|
% number (for example vii), and
% \textquotedblleft |-|\textquotedblright{} puts the minus
% sign back in front of it (for example -vii).
%
%    \begin{macrocode}
      -\romannumeral\number-\arabic{#1}%
    \else%
      0%
    \fi%
  \fi%
  }

%    \end{macrocode}
%  \end{macro}
%
% \pagebreak
%
% \begin{macro}{\XRoman}
% |\XRoman| does the same for uppercase |\Roman| numbers.
% |-\uppercase{\romannumeral\number-\arabic{#1}}| cannot be used,
% because the result in the example is |-\uppercase{vii}| and
% not |-VII|.\footnote{This does not matter for the print out,
% but for the display of the logical page numbers
% as well as the \xfile{.aux} file.} Therefore we have a look
% at \LaTeX's own |\@Roman\FOOcounter|,\\
% |\def\@Roman#1{\expandafter\@slowromancap\romannumeral #1@}|,\\
% and use |\@slowromancap|, which is a fully expandable macro,
% to do the trick for this:
%
% \noindent \textquotedblleft
% \begin{quote}
%\begin{verbatim}
%\def\@slowromancap#1{\ifx @#1% then terminate
%\else
%\if i#1I\else\if v#1V\else\if x#1X\else\if l#1L\else\if
%c#1C\else\if d#1D\else \if m#1M\else#1\fi\fi\fi\fi\fi\fi\fi
%\expandafter\@slowromancap
%\fi
%}
%\end{verbatim}
% \end{quote}
% \textquotedblright{} (1998/05/16 Version v1.1g LaTeX Kernel File~m
% ltcounts.dtx 105 Counters and Lengths).
%
%    \begin{macrocode}
\newcommand{\XRoman}[1]{%
  \ifnum\value{#1}>0%
    \Roman{#1}%
  \else%
    \ifnum\value{#1}<0%
      -\expandafter\@slowromancap\romannumeral\number-\arabic{#1}@%
    \else%
      0%
    \fi%
  \fi%
  }

%    \end{macrocode}
%  \end{macro}
%
% \begin{macro}{\XXRoman}
% In older versions |\XXRoman| was used. For compatibility, it is forwarded to |\XRoman|
% and an error message is given.
%
%    \begin{macrocode}
\newcommand{\XXRoman}[1]{\XRoman{#1}%
  \PackageError{pageslts}{Old command \string\XXRoman\space found}{Replaced by \string\XRoman.}%
  }

%    \end{macrocode}
%  \end{macro}
%
% \begin{macro}{\pagesLTS@ifcounter}
% We provide a way to create counters like
% \begin{itemize}
% \item[-] \DescribeMacro{pagesLTS.pnc. page numbering scheme }
%   |pagesLTS.pnc.<page numbering scheme>|, e.\,g. |pagesLTS.pnc.Roman|,
% \item[-] \DescribeMacro{pagesLTS.double. page numbering scheme }
%   |pagesLTS.double.<page numbering scheme>|, e.\,g. |pagesLTS.double.Roman|,
% \item[-] \DescribeMacro{PageCurrentLocal. page numbering scheme }
%   |PageCurrentLocal.<page numbering scheme>|, e.\,g. |PageCurrentLocal.Roman|,
% \end{itemize}
% for all page numbering schemes, even those not supported
% by the current original |\pagenumbering|
% (1994/05/19 v1.1a LaTeX Kernel File w ltpageno.dtx 52 Page Numbering),
% which is\\
% \begin{quote}
%\begin{verbatim}
%\countdef\c@page=0 \c@page=1
%\def\cl@page{}
%\def\pagenumbering#1{%
%  \global\c@page \@ne \gdef\thepage{\csname @#1\endcsname
%   \c@page}}
%\end{verbatim}
% \end{quote}
%
%    \begin{macrocode}
\newcommand{\pagesLTS@ifcounter}[1]{%
  \@ifundefined{c@#1}{\newcounter{#1}}{\relax}%
  }

%    \end{macrocode}
%  \end{macro}
%
% \begin{macro}{\lastpages}
% We provide a command to give the number of pages in a sector
% of a split page numbering scheme (see page~\pageref{pagesLTS.pnscheme.nr},\\
% |pagesLTS.<page numbering scheme>.<number>.local.cnt|):
%
%    \begin{macrocode}
\newcommand{\lastpages}[2]{%
  \pagesLTS@ifcounter{pagesLTS.#1.#2.local.cnt}%
  \arabic{pagesLTS.#1.#2.local.cnt}%
  }

%    \end{macrocode}
%  \end{macro}
%
% \begin{macro}{\pagesLTS@writelabel}
% At last defining the writing of a label:
%
%    \begin{macrocode}
\newcommand{\pagesLTS@writelabel}[1]{%
  \addtocounter{page}{+1}%
%    \end{macrocode}
%
% \noindent |\addtocounter{page}{+1}| because |\pagesLTS@putlabel| includes an\\
% |\addtocounter{page}{-1}|, which is not necessary here.\\
% Into the \xfile{.aux} file something like\\
% |\newlabel{pagesLTS.Roman}{{}{VIII}{}{page.VIII}{}}|\\
% is written, thus |\lastpageref{pagesLTS.Roman}| prints |VIII| and
% links to |page.VIII|.
%
%    \begin{macrocode}
  \pagesLTS@putlabel{pagesLTS.#1}{\thepage}{1}
  \addtocounter{page}{-1}%
  \ifx\pagesLTS@pnc\pagesLTS@zero% \relax
%    \end{macrocode}
%
% i.\,e. if the current page numbering scheme is
% \textquotedblleft 0\textquotedblright, i.\,e. before the first
% |\pagenumbering{...}| command, do nothing,
%
%    \begin{macrocode}
  \else%
    \addtocounter{page}{+1}%
    \pagesLTS@putlabel{pagesLTS.#1.local}{\theCurrentPageLocal}{1}
%    \end{macrocode}
%
% otherwise write into the \xfile{.aux} file something like\\
% |\newlabel{pagesLTS.arabic.local}{{}{5}{}{page.8}{}}|,\\
% thus |\lastpageref{pagesLTS.arabic.local}| prints |5| and
% links to |page.8|. Here (and in the example file) it is not
% \textquotedblleft print |8| and link to |page.8|\textquotedblright ,
% because |\addtocounter{page}{3}| has been used, thus the page with
% \textquotedblleft number\textquotedblright{} (name) |8| is the
% \textbf{fifth} $(=8-3)$ page.
%
%    \begin{macrocode}
    \addtocounter{page}{-1}%
  \fi%
  }

%    \end{macrocode}
% \end{macro}
%
% \begin{macro}{\erroralphalph}
% \indent |\erroralphalph| extends the
% \textquotedblleft numbers\textquotedblright{} of counters
% to zero and negative values for representations usually
% not supporting this: |\alphalph|, |\AlphAlph|, and
% |\fnsymbolmult| of the \xpackage{alphalph} package.\\
% |\alph|, |\Alph|, and |\fnsymbol| would not support
% \textquotedblleft numbers\textquotedblright{} below one.\\
% |\arabic| already supports negative numbers and zero.\\
% |\roman| and |\Roman| support neither negative numbers
% nor zero, but are expanded in this package
% (|\xroman| and |\XRoman|), see page~\pageref{xroman}.\\
%
%    \begin{macrocode}

%% The following code is from Heiko Oberdiek [2010/04/18],              %%
%% expanding his alphalph package as of 2010/04/18, v2.3. (Thanks!)     %%
\newcommand*{\erroralphalph}[2]{%
  \ifnum\value{#2}>0%
    #1{\value{#2}}%
  \else%
    \ifnum\value{#2}<0%
      -#1{\expandafter\@gobble\the\value{#2}}%
    \else%
      0%
    \fi%
  \fi%
  }
%% End of code from Heiko Oberdiek                                      %%
%% Check and Error/Warning messages have been moved to \EveryShipout,   %%
%% because messages inside e.g. the \pageref command can cause trouble. %%

%    \end{macrocode}
% \end{macro}
%
% \begin{macro}{\expandPagenumbering}
% \indent Here the |\erroralphalph| command is called with the
% appropriate arguments for each page numbering scheme.
%
%    \begin{macrocode}

\newcommand{\expandPagenumbering}[1]{%
  \let\Origthepage\thepage%
  \def\pagesLTS@tmpC{arabic}%
  \ifx\pagesLTS@pnc\pagesLTS@tmpC% \relax
%    \end{macrocode}
%
% |\arabic| already supports negative numbers and zero
% (|-MAX...MAX|, where |MAX|\,$ = 2\,147\,483\,647$).
%
%    \begin{macrocode}
  \else%
    \def\pagesLTS@tmpC{roman}%
    \ifx\pagesLTS@pnc\pagesLTS@tmpC%
      \ifpagesLTS@romanMult%
%    \end{macrocode}
%
% |\erroralphalph{\roman}{page}| cannot be used, because
% |-\roman{\expandafter\@gobble\the\value{page}}| does not work.
% If option |romanMult| is not false,
% |\xroman| (see page~\pageref{xroman}) expands the usable
% roman page numbers to values below 1 (i, I, respectively),
% see subsubsection~\ref{sss:romanMult}.
%
%    \begin{macrocode}
        \renewcommand*{\thepage}{\xroman{page}}%
      \fi%
    \else%
      \def\pagesLTS@tmpC{Roman}%
      \ifx\pagesLTS@pnc\pagesLTS@tmpC%
        \ifpagesLTS@RomanMulti%
%    \end{macrocode}
%
% The same for |\Roman| page numbering, expanded by |\XRoman|
% (see page~\pageref{xroman}).
%
%    \begin{macrocode}
          \renewcommand*{\thepage}{\XRoman{page}}%
        \fi%
      \else%
        \ifx\pagesLTS@pnc\pagesLTS@alph%
%    \end{macrocode}
%
% |\alph| and |\Alph| page numberings are expanded to negative
% and zero values, and to values greater than
% \textquotedblleft z\textquotedblright{} or
% \textquotedblleft Z\textquotedblright{} with the
% \xpackage{alphalph} package.~--
% If |\pagesLTS@alphMult| was zero, nothing is done.
%
%    \begin{macrocode}
          \ifx\pagesLTS@alphMult\pagesLTS@ab%
            \renewcommand*{\thepage}{\erroralphalph{\alphalph}{page}}%
          \else \ifx\pagesLTS@alphMult\pagesLTS@bb%
                  \renewcommand*{\thepage}{\erroralphalph{\alphMult}{page}}%
                \fi%
          \fi%
        \else%
          \ifx\pagesLTS@pnc\pagesLTS@Alph%
            \ifx\pagesLTS@AlphMulti\pagesLTS@ABi%
              \renewcommand*{\thepage}{\erroralphalph{\AlphAlph}{page}}%
            \else \ifx\pagesLTS@AlphMulti\pagesLTS@BBi%
                    \renewcommand*{\thepage}{\erroralphalph{\AlphMult}{page}}%
                  \fi%
            \fi%
          \else%
            \ifx\pagesLTS@pnc\pagesLTS@fns%
%    \end{macrocode}
%
% Same for |\fnsymbol| page numbers.
%
%    \begin{macrocode}
              \ifpagesLTS@fnsymbolmult%
                \renewcommand*{\thepage}{\erroralphalph{\fnsymbolmult}{page}}%
              \fi%
            \else%
%    \end{macrocode}
%
% If the used page numbering scheme has not been recognized by the
% \xpackage{pageslts} package so far, we can do nothing, and problems
% might result.\\
%
%    \begin{macrocode}
              \PackageError{pageslts}{unknown page numbering scheme}{%
                The pageslts package encountered the unknown\MessageBreak%
                page numbering scheme\MessageBreak%
                `#1'. \MessageBreak%
                If this is no typing mistake, it might work\MessageBreak%
                - or it might not work.\MessageBreak%
                \@ehc%
                }
            \fi%
          \fi%
        \fi%
      \fi%
    \fi%
  \fi%
  \let\pagesLTS@tmpC\undefined%
  }

%    \end{macrocode}
% \end{macro}
%
% \begin{macro}{\pagenumbering}
% \indent Now for the \textbf{new} version of the |\pagenumbering| command:
%
%    \begin{macrocode}
\renewcommand{\pagenumbering}[1]{%
%    \end{macrocode}
%
% If the current page numbering scheme, |\pagesLTS@pnc|, or the
% requested page numbering scheme, |#1|, is |\pagesLTS@fns|,
% i.\,e. \texttt{fnsymbol}, the counter\texttt{pagesLTS.fnsymbol.local}
% is needed. If it does not exists yet, it is created here.
%
%    \begin{macrocode}
  \edef\pagesLTS@tmpA{#1}%
  \ifx\pagesLTS@pnc\pagesLTS@fns%
    \pagesLTS@ifcounter{pagesLTS.fnsymbol.local}%
  \fi%
  \ifx\pagesLTS@tmpA\pagesLTS@fns%
    \pagesLTS@ifcounter{pagesLTS.fnsymbol.local}%
  \fi%
%    \end{macrocode}
%
% If the current page numbering scheme, |\pagesLTS@pnc|, and the
% requested page numbering scheme, |#1|, is the same one, nothing
% further is done, otherwise the real action begins.
%
%    \begin{macrocode}
  \ifx\pagesLTS@pnc\pagesLTS@tmpA% \relax
  \else%
%    \end{macrocode}
%
% The next code is executed, when we are at a page after the first one.
% This distinction is done for two reasons:\\
% On the one hand, |\pagenumbering| could be called \textit{before}
% |\begin{document}| (where the current page should not be
% greater than one), and on the other hand we go one page back to
% aim all references to that page. Obviously at the first page
% there is no going backward.
%
%    \begin{macrocode}
    \ifnum \value{CurrentPage}>1%
      \addtocounter{page}{-1}%
      \addtocounter{pagesLTS.current.local.\pagesLTS@pnc}{-1}%
%    \end{macrocode}
%
% For the case that the page numbering scheme is or will be split,
% like e.\,g. the Roman one in the |pageslts-example.tex|, a counter
% like
% |pagesLTS.Roman.1.local.count|
% (or |pagesLTS.Roman.2.local.count|,
% |pagesLTS.Roman.3.local.count|,\ldots ) is introduced and set to the
% number of the local page.
%
%    \begin{macrocode}
      \newcounter{pagesLTS.\pagesLTS@pnc.\arabic{pagesLTS.pnc.\pagesLTS@pnc}.local.count}%
      \setcounter{pagesLTS.\pagesLTS@pnc.\arabic{pagesLTS.pnc.\pagesLTS@pnc}.local.count}{%
        \value{pagesLTS.current.local.\pagesLTS@pnc}}%
%    \end{macrocode}
%
% If the page numbering scheme is \xpackage{fnsymbol},
% and if it \textit{was} used before,
% from said counter the number of pages of the preceding uses
% of the same page numbering scheme, |pagesLTS.\pagesLTS@pnc.done|,
% is subtracted (same as for the other schemes, see below). Instead of introducing a new
% counter (which can be problematic, when the number of available counters is limited),
% we borrow the \texttt{pagesLTS.pnc.0} counter, i.\,e. we save its value to |\pagesLTS@tmpa|,
% (ab)use the counter, and then set it back to its former value as saved in |\pagesLTS@tmpa|.
%
%    \begin{macrocode}
      \ifx\pagesLTS@pnc\pagesLTS@fns%
        \ifnum \value{pagesLTS.pnc.\pagesLTS@pnc}>1%
          \mathchardef\pagesLTS@tmpa=\arabic{pagesLTS.pnc.0}%
          \setcounter{pagesLTS.pnc.0}{\value{pagesLTS.pnc.\pagesLTS@pnc}}%
          \addtocounter{pagesLTS.pnc.0}{-1}%
          \addtocounter{pagesLTS.\pagesLTS@pnc.\arabic{pagesLTS.pnc.\pagesLTS@pnc}.local.count}{%
            -\value{pagesLTS.\pagesLTS@pnc.\arabic{pagesLTS.pnc.0}.local.count}}%
          \setcounter{pagesLTS.pnc.0}{\pagesLTS@tmpa}%
        \fi%
%    \end{macrocode}
%
% If the page numbering scheme is \textit{not} \xpackage{fnsymbol},
% a~numbered label is written:
%
%    \begin{macrocode}
      \else%
        \pagesLTS@writelabel{\pagesLTS@pnc.\arabic{pagesLTS.pnc.\pagesLTS@pnc}}%
%    \end{macrocode}
%
% If the page numbering scheme was \textit{not} used before,
%
%    \begin{macrocode}
        \ifnum \value{pagesLTS.pnc.\pagesLTS@pnc}<2%
%    \end{macrocode}
%
% an unnumbered label is also written:
%
%    \begin{macrocode}
          \pagesLTS@writelabel{\pagesLTS@pnc}%
%    \end{macrocode}
%
% If the page numbering scheme \textit{was} used before,
% from said counter the number of pages of the preceding uses
% of the same page numbering scheme, |pagesLTS.\pagesLTS@pnc.done|,
% is subtracted. Instead of introducing a new
% counter (which can be problematic, when the number of available counters is limited),
% we again borrow the \texttt{pagesLTS.pnc.0} counter (see above).
%
%    \begin{macrocode}
        \else%
          \mathchardef\pagesLTS@tmpa=\arabic{pagesLTS.pnc.0}%
          \setcounter{pagesLTS.pnc.0}{\value{pagesLTS.pnc.\pagesLTS@pnc}}%
          \addtocounter{pagesLTS.pnc.0}{-1}%
          \pagesLTS@ifcounter{pagesLTS.\pagesLTS@pnc.done}%
          \addtocounter{pagesLTS.\pagesLTS@pnc.done}{%
            \value{pagesLTS.\pagesLTS@pnc.\arabic{pagesLTS.pnc.0}.local.count}}%
          \addtocounter{pagesLTS.\pagesLTS@pnc.\arabic{pagesLTS.pnc.\pagesLTS@pnc}.local.count}{%
            -\value{pagesLTS.\pagesLTS@pnc.done}}%
          \setcounter{pagesLTS.pnc.0}{\pagesLTS@tmpa}%
        \fi%
%    \end{macrocode}
%
% The values are written to the \xfile{.aux} file
% (if writing is allowed: |\if@filesw|),
% because they must be available at the beginning of the document:
%
%    \begin{macrocode}
        \if@filesw%
          \immediate\write\@auxout{\string
            \pagesLTS@ifcounter{pagesLTS.\pagesLTS@pnc.\arabic{pagesLTS.pnc.\pagesLTS@pnc}.local.cnt}}%
        \fi%
        \edef\pagesLTS@tmpB{\arabic{pagesLTS.\pagesLTS@pnc.\arabic{pagesLTS.pnc.\pagesLTS@pnc}.local.count}}%
        \if@filesw%
          \immediate\write\@auxout{\string
            \setcounter{pagesLTS.\pagesLTS@pnc.\arabic{pagesLTS.pnc.\pagesLTS@pnc}.local.cnt}{\pagesLTS@tmpB}}%
        \fi%
      \fi%
%    \end{macrocode}
%
% For further code for the case of \xpackage{fnsymbol} please see below
% (|\lastpagereftext|, page~\pageref{lastpagereftextlabel}).\\
% The last page number is saved, in case the same page
% numbering scheme is continued later.
%
%    \begin{macrocode}
      \pagesLTS@ifcounter{pagesLTS.double.\pagesLTS@pnc}%
      \setcounter{pagesLTS.double.\pagesLTS@pnc}{\value{page}}%
%    \end{macrocode}
%
% We went back one page, so we must go forward again:
%
%    \begin{macrocode}
      \addtocounter{pagesLTS.current.local.\pagesLTS@pnc}{+1}%
      \addtocounter{page}{+1}%
%    \end{macrocode}
%
% The page numbering scheme |\pagesLTS@pnc| is now set to
% the new one, given by the user as argument with the
% |\pagenumbering{...}| command:
%
%    \begin{macrocode}
      \xdef\pagesLTS@pnc{#1}%
%    \end{macrocode}
%
% The new page numbering scheme is now started for real:
%
%    \begin{macrocode}
      \OrigPagenumbering{#1}%
%    \end{macrocode}
%
% If a page numbering scheme not known by the original
% |\pagenumbering{...}| command is used, an error will
% arise here - but maybe without error message.\\
%
% If page numbering schem |\alph|, |\Alph|, or |\fnsymbol| is used,
% \xpackage{pageslts} extends the page numbers according to
% the given options, using the \xpackage{alphalph} package.
% |\arabic| does not need any expansion.
% |\roman| and |\Roman| at least receive a definition for
% zero.
%
%    \begin{macrocode}
      \expandPagenumbering{#1}%
%    \end{macrocode}
%
% Counters like |pagesLTS.pnc.Roman| are introduced:
%
%    \begin{macrocode}
      \pagesLTS@ifcounter{pagesLTS.pnc.\pagesLTS@pnc}%
%    \end{macrocode}
%
% The saved number of times this page numbering scheme was used
% is increased by one:
%
%    \begin{macrocode}
      \addtocounter{pagesLTS.pnc.\pagesLTS@pnc}{1}%
%    \end{macrocode}
%
% \indent Now defining the counter |pagesLTS.double.\pagesLTS@pnc|,
% if it did not exist already, adding~1, because this is
% the first page of it (or another one, if the scheme is
% continued):
%
%    \begin{macrocode}
      \pagesLTS@ifcounter{pagesLTS.double.\pagesLTS@pnc}%
      \addtocounter{pagesLTS.double.\pagesLTS@pnc}{1}%
%    \end{macrocode}
%
% The page number is continued, if the option |pagecontinue=false|
% is \textbf{not} set, otherwise it is reset to one. Note that neither
% the local nor the current counter are reset, as they contain
% the real \textit{values} and not the \textit{names} of the pages.
%
%    \begin{macrocode}
      \ifpagesLTS@pagecontinue%
        \setcounter{page}{\value{pagesLTS.double.\pagesLTS@pnc}}%
      \else%
        \setcounter{page}{1}%
      \fi%
%    \end{macrocode}
%
% If it does not exist already, the counter
% |pagesLTS.current.local.\pagesLTS@pnc|
% (e.\,g. |pagesLTS.current.local.Roman|) is created.
%
%    \begin{macrocode}
      \pagesLTS@ifcounter{pagesLTS.current.local.\pagesLTS@pnc}%
%    \end{macrocode}
%
% If |pagesLTS.double.\pagesLTS@pnc| of the current page numbering
% scheme is equal to one, this is the first page of this page
% numbering scheme. Then |pagesLTS.current.local.\pagesLTS@pnc|
% (which was zero) is set to one.
%
%    \begin{macrocode}
      \ifnum \value{pagesLTS.double.\pagesLTS@pnc}=1%
        \setcounter{pagesLTS.current.local.\pagesLTS@pnc}{1}%
      \fi%
%    \end{macrocode}
%
% Otherwise, i.\,e. if |\value{CurrentPage}| is not |>1|,
% i.\,e. before the first page has shiped out:
%
%    \begin{macrocode}
    \else%
      %% before the first page has shipped out
%    \end{macrocode}
%
% The current page numbering scheme is defined by the
% argument of |\pagenumbering{...}|, which the user gave:
%
%    \begin{macrocode}
      \xdef\pagesLTS@pnc{#1}%
%    \end{macrocode}
%
% \noindent and the page numbering scheme set by the original
% page numbering command
% (1994/05/19 v1.1a LaTeX Kernel File w ltpageno.dtx 52 Page Numbering),
% which resets the page number to one,
% but at the first page continuation does not make sense).
% Well, nearly the original page numbering command:
% |\OrigPagenumbering{\pagesLTS@pnc}| does not work,
% so we \textquotedblleft expand\textquotedblright{} the
% |\OrigPagenumbering| command:
%
%    \begin{macrocode}
      \global\c@page \@ne\relax%
      \global\def\thepage{\csname \expandafter @\pagesLTS@pnc \endcsname \c@page}%
%    \end{macrocode}
%
% If a page numbering scheme is used, which is not known by
% \LaTeX , an error might arise here -- but maybe without error message.\\
%
% If page numbering scheme |\alph|, |\Alph|, or |\fnsymbol| is used,
% \xpackage{pageslts} extends the page numbers according to
% the given options, using the \xpackage{alphalph} package.
% |\arabic| does not need any expansion.
% |\roman| and |\Roman| at least receive a definition for
% zero.
%
%    \begin{macrocode}
      \expandPagenumbering{#1}
%    \end{macrocode}
%
% We are at the first page, so the page counters are set to one:
%
%    \begin{macrocode}
      \pagesLTS@ifcounter{pagesLTS.pnc.\pagesLTS@pnc}%
      \setcounter{pagesLTS.pnc.\pagesLTS@pnc}{1}%
      \pagesLTS@ifcounter{pagesLTS.double.\pagesLTS@pnc}%
      \setcounter{pagesLTS.double.\pagesLTS@pnc}{1}%
      \pagesLTS@ifcounter{pagesLTS.current.local.\pagesLTS@pnc}%
      \setcounter{pagesLTS.current.local.\pagesLTS@pnc}{1}%
    \fi%
%    \end{macrocode}
%
% Whether |\pagenumbering{...}| is called in the preamble,
% |\AtBeginDocument|, right after |\begin{document}|,
% or somewhere in the document, we want to remember
% \textit{whether} it was called at all:
%
%    \begin{macrocode}
    \gdef\pagesLTS@called{1}%
  \fi%
%    \end{macrocode}
%
% We do not need the temporary definitions any more.
%
%    \begin{macrocode}
  \let\pagesLTS@tmpA\undefined%
  \let\pagesLTS@tmpB\undefined%
  }

%    \end{macrocode}
% \end{macro}
%
% \begin{macro}{\lastpageref*}
% If \xpackage{hyperref} is used, but (some) references to some last page shall
% not be hyperlinked, a command |\lastpageref*| (analogous to |\pageref*|) is
% needed. Therefore we define (analogous to |\HyPsd@pageref| from
% the \xpackage{hyperref} package by \textsc{Heiko Oberdiek})
%
%    \begin{macrocode}
%% analogous to \HyPsd@pageref from the hyperref package by Heiko Oberdiek:
\def\lastpagereftxt#1{\pagesLTS@@pageref#1*\END}

%    \end{macrocode}
%
% Macro |\pagesLTSpageref| checks, whether a star is present (analogous to
% |\HyPsd@@pageref| again from the \xpackage{hyperref} package
% of \textsc{Heiko Oberdiek}):
%
%    \begin{macrocode}
\def\pagesLTS@@pageref#1*#2\END{%
  \ifx\\#2\\% no star
    \pagesLTS@@@pageref{#1}%
  \else% star
    \expandafter\pagesLTS@@@pagerefstar%
  \fi%
}

\def\pagesLTS@@@pageref#1{\lastpagereftext{#1}}
\def\pagesLTS@@@pagerefstar#1{\lastpagereftextstar{#1}}

%    \end{macrocode}
% \end{macro}
%
% \phantomsection
% \begin{macro}{\lastpagereftext} \label{lastpagereftextlabel}
% \indent When |\lastpageref| is used somewhere inside the \texttt{txt} (text),
% i.\,e. not at the last page, it is defined as |\lastpagereftxt| (see above).
% When the page numbering scheme is \xpackage{fnsymbol}, and the \xpackage{hyperref}
% package has been loaded, a hyperref instead of a label is used for the reference
% to |pagesLTS.fnsymbol.local.\arabic{pagesLTS.fnsymbol.local}|.
% (And if the \texttt{pagesLTS.fnsymbol.local} counter did not exist yet,
% it is created here.)\\
%
%    \begin{macrocode}
\newcommand{\lastpagereftext}[1]{%
  \def\pagesLTS@tmpA{#1}%
  \def\pagesLTS@tmpB{pagesLTS.fnsymbol.local}%
  \ifx\pagesLTS@tmpA\pagesLTS@tmpB%
    \pagesLTS@ifcounter{pagesLTS.fnsymbol.local}%
    \ltx@ifpackageloaded{hyperref}{%
      \href{\#pagesLTS.fnsymbol.local.\arabic{pagesLTS.fnsymbol.local}}%
        {\arabic{pagesLTS.fnsymbol.local}}%
%    \end{macrocode}
%
% When the page numbering scheme is \xpackage{fnsymbol}, but the \xpackage{hyperref} package
% has \textit{not} been loaded, just the arabic number of the
% \texttt{pagesLTS.fnsymbol.local} counter is given (because there will be
% no hyperlink anyway).
%
%    \begin{macrocode}
     }{\arabic{pagesLTS.fnsymbol.local}}%
%    \end{macrocode}
%
% Otherwise just the common |\pageref| is applied:
%
%    \begin{macrocode}
  \else%
    \pageref{#1}%
  \fi%
%    \end{macrocode}
%
% We do not need the temporary definitions any more.
%
%    \begin{macrocode}
  \let\pagesLTS@tmpA\undefined%
  \let\pagesLTS@tmpB\undefined%
  }

%    \end{macrocode}
% \end{macro}
%
% \begin{macro}{\lastpagereftextstar}
% And the same for the starred version, where no hyperlink is generated:
%
%    \begin{macrocode}
\newcommand{\lastpagereftextstar}[1]{%
  \def\pagesLTS@tmpA{#1}%
  \def\pagesLTS@tmpB{pagesLTS.fnsymbol.local}%
  \ifx\pagesLTS@tmpA\pagesLTS@tmpB%
     \pagesLTS@ifcounter{pagesLTS.fnsymbol.local}%
     \arabic{pagesLTS.fnsymbol.local}%
  \else%
%    \end{macrocode}
%
% There is no |\pageref*| without \xpackage{hyperref}.
%
%    \begin{macrocode}
    \ltx@ifpackageloaded{hyperref}{\pageref*{#1}}{\pageref{#1}}%
  \fi%
  \let\pagesLTS@tmpA\undefined%
  \let\pagesLTS@tmpB\undefined%
  }

%    \end{macrocode}
% \end{macro}
%
% \begin{macro}{\lastpagerefend}
% \indent When the \xpackage{hyperref} package is used and the page numbering scheme
% of the last page is \xpackage{fnsymbol}, |\lastpageref| is defined as
% |\lastpagerefend|.
% Hyperrefs instead of labels are used for the reference to
% \xpackage{fnsymbol} pages (including the last one).\\
%
% Again it must be discriminated between unstarred form and starred form:
%
%    \begin{macrocode}
\def\lastpagerefend#1{\pagesLTS@@pagerefend#1*\END}

\def\pagesLTS@@pagerefend#1*#2\END{%
  \ifx\\#2\\% no star
    \pagesLTS@@@pagerefend{#1}%
  \else% star
    \expandafter\pagesLTS@@@pagerefendstar%
  \fi%
}

\def\pagesLTS@@@pagerefend#1{\l@stpagerefend{#1}}
\def\pagesLTS@@@pagerefendstar#1{\l@stpagerefendstar{#1}}

%    \end{macrocode}
% \end{macro}
%
% \begin{macro}{\l@stpagerefend}
% The unstarred form (i.\,e. with hyperlinks, if \xpackage{hyperref} is loaded,
% otherwise without hyperlinks):
%
%    \begin{macrocode}
\newcommand{\l@stpagerefend}[1]{%
  \def\pagesLTS@tmpA{#1}%
  \def\pagesLTS@tmpB{pagesLTS.fnsymbol.local}%
  \ifx\pagesLTS@tmpA\pagesLTS@tmpB%
    \pagesLTS@ifcounter{pagesLTS.fnsymbol.local}%
    \ltx@ifpackageloaded{hyperref}{%
      \href{\#pagesLTS.fnsymbol.local.\arabic{pagesLTS.fnsymbol.local}}%
        {\arabic{pagesLTS.fnsymbol.local}}%
     }{\arabic{pagesLTS.fnsymbol.local}}%
  \else%
    \def\pagesLTS@tmpB{pagesLTS.fnsymbol}%
    \ifx\pagesLTS@tmpA\pagesLTS@tmpB%
      \ltx@ifpackageloaded{hyperref}{%
        \href{\#pagesLTS.fnsymbol.local.\arabic{pagesLTS.fnsymbol.local}}%
          {\pagesLTS.lastpage}%
       }{\pageref{pagesLTS.fnsymbol}}%
    \else%
      \def\pagesLTS@tmpB{LastPage}%
      \ifx\pagesLTS@tmpA\pagesLTS@tmpB%
        \ltx@ifpackageloaded{hyperref}{%
          \href{\#pagesLTS.fnsymbol.local.\pagesLTS@eso}%
            {\pagesLTS.lastpage}%
         }{\pageref{LastPage}}%
      \else%
        \def\pagesLTS@tmpB{VeryLastPage}%
        \ifx\pagesLTS@tmpA\pagesLTS@tmpB%
          \ltx@ifpackageloaded{hyperref}{%
            \href{\#pagesLTS.fnsymbol.local.\arabic{pagesLTS.fnsymbol.local}}%
              {\pagesLTS.lastpage}%
           }{\pageref{VeryLastPage}}%
        \else%
          \def\pagesLTS@tmpB{LastPages}%
          \ifx\pagesLTS@tmpA\pagesLTS@tmpB%
            \ltx@ifpackageloaded{hyperref}{%
              \href{\#pagesLTS.fnsymbol.local.\arabic{pagesLTS.fnsymbol.local}}%
                {\arabic{pagesLTS.pagenr}}%
             }{\pageref{LastPages}}%
          \else%
            \pageref{#1}%
          \fi%
        \fi%
      \fi%
    \fi%
  \fi%
%    \end{macrocode}
%
% We do not need the temporary definitions any more.
%
%    \begin{macrocode}
  \let\pagesLTS@tmpA\undefined%
  \let\pagesLTS@tmpB\undefined%
  }

%    \end{macrocode}
% \end{macro}
%
% \pagebreak
%
% \begin{macro}{\l@stpagerefendstar}%
% And the starred form, without hyperlinks, even if \xpackage{hyperref} is loaded,
% otherwise (i.\,e. without loaded \xpackage{hyperref}) this command is not called:
%
%    \begin{macrocode}
\newcommand{\l@stpagerefendstar}[1]{%
  \def\pagesLTS@tmpA{#1}%
  \def\pagesLTS@tmpB{pagesLTS.fnsymbol.local}%
  \ifx\pagesLTS@tmpA\pagesLTS@tmpB%
    \pagesLTS@ifcounter{pagesLTS.fnsymbol.local}%
    \arabic{pagesLTS.fnsymbol.local}%
  \else%
    \def\pagesLTS@tmpB{pagesLTS.fnsymbol}%
    \ifx\pagesLTS@tmpA\pagesLTS@tmpB%
      \ltx@ifpackageloaded{hyperref}{\pageref*{pagesLTS.fnsymbol}}{%
        \pageref{pagesLTS.fnsymbol}}%
    \else%
      \def\pagesLTS@tmpB{LastPage}%
      \ifx\pagesLTS@tmpA\pagesLTS@tmpB%
        \ltx@ifpackageloaded{hyperref}{\pageref*{LastPage}}{\pageref{LastPage}}%
      \else%
        \def\pagesLTS@tmpB{VeryLastPage}%
        \ifx\pagesLTS@tmpA\pagesLTS@tmpB%
          \ltx@ifpackageloaded{hyperref}{\pageref*{VeryLastPage}}{%
            \pageref{VeryLastPage}}%
        \else%
          \def\pagesLTS@tmpB{LastPages}%
          \ifx\pagesLTS@tmpA\pagesLTS@tmpB%
            \ltx@ifpackageloaded{hyperref}{\pageref*{LastPages}}{\pageref{LastPages}}%
          \else%
            \ltx@ifpackageloaded{hyperref}{\pageref*{#1}}{\pageref{#1}}%
            \fi%
          \fi%
        \fi%
      \fi%
    \fi%
  \fi%
  \let\pagesLTS@tmpA\undefined%
  \let\pagesLTS@tmpB\undefined%
  }

%    \end{macrocode}
% \end{macro}
%
% \pagebreak
%
% \begin{macro}{\overrideLTSlabel}
% |\overridelabel| from the \xpackage{undolabl} package just |\undonewlabel|s
% a~label and places a new |\label{#1}|, but we need to place a
% |\pagesLTS@putlabel{#1}{#2}|, therefore we need another command instead of
% (but somewhat similar to) |\overridelabel|:
%
%    \begin{macrocode}
% somewhat analogous to \overridelabel from the undolabl package:
\newcommand\overrideLTSlabel[2]{%
  \@bsphack
    \ifnum \value{pagesLTS.pnc.\pagesLTS@pnc}>1%
      \edef\pagesLTStmpA{#1}%
      \edef\pagesLTStmpB{pagesLTS.\pagesLTS@pnc.local}%
      \ifx\pagesLTStmpA\pagesLTStmpB%
          \immediate\write\@auxout{\string\undonewlabel{#1}}%
          \@overriddenmessage s{#1}%
      \fi%
    \fi%
    \pagesLTS@putlabel{#1}{#2}{0}%
  \@esphack%
  }

%    \end{macrocode}
% \end{macro}
%
% \begin{macro}{\@pagesLTS@Prelim@EveryShipout}
% \indent Because we cannot make references to pages with \xpackage{fnsymbol}
% page \textquotedblleft numbers\textquotedblright{} manually
% with \xpackage{hyperref}, we use \nolinebreak{|\phantomsection|\,s} and
% refer to one of those. But because we do not know how many
% \nolinebreak{|\phantomsection|\,s} and \nolinebreak{|\section*|\,s}
% are introduced by the user (or other packages;
% cf.~\LaTeX~bug~2298: knowing level of |section*|,
% \ \url{http://www.latex-project.org/cgi-bin/ltxbugs2html?category=LaTeX&responsible=anyone&state=open&keyword=&pr=latex%2F2298&search=}),
% we cannot refer to the last one as we did with the pages.
%
%    \begin{macrocode}
\newcommand{\@pagesLTS@Prelim@EveryShipout}{%
%% The following code is from the prelim2e package             %%
%% [2009/05/29 v1.3] by Martin Schr\"{o}der (Thanks!):         %%
  \bgroup
    \dimen\z@=\wd\@cclv
    \dimen\@ne=\ht\@cclv
    \dimen\tw@=\dp\@cclv
    \dimen\thr@@=\dimen1
    \advance\dimen\thr@@ by \dimen\tw@
    \global\setbox\@cclv\vbox to \dimen\thr@@{%
      \hb@xt@\dimen\z@{%
        \box\@cclv%
        \hss%
        }%
      \vbox to \z@{%
        \hb@xt@\dimen\z@{%
          \let\protect\relax
%% Code not from prelim2e package:                             %%
%    \end{macrocode}
%
% Therefore each page with \xpackage{fnsymbol} page
% \textquotedblleft number\textquotedblright{} receives a
% |\phantomsection| and a label, which includes a number increased
% by one for each page. This is done for
% |pagesLTS.fnsymbol.local.\arabic{pagesLTS.fnsymbol.cont}|
% as well as
% |pagesLTS.fnsymbol|, |pagesLTS.\pagesLTS@pnc|, and
% |pagesLTS.\pagesLTS@pnc.local|.\\
% In case an older label already existed, it is overwritten by an
% |\overridelabel| command.
%
%    \begin{macrocode}
          \ifx\pagesLTS@pnc\pagesLTS@fns%
            \pagesLTS@ifcounter{pagesLTS.fnsymbol.cont}%
            \addtocounter{pagesLTS.fnsymbol.cont}{1}%
            \ltx@ifpackageloaded{hyperref}{%
              \phantomsection%
              \hypertarget{pagesLTS.fnsymbol.local.\arabic{pagesLTS.fnsymbol.cont}}{}%
              }{}%
            \ifnum \pagesLTS@esov=\pagesLTS@zero%
              \label{pagesLTS.fnsymbol}%
            \else%
              \overridelabel{pagesLTS.fnsymbol}%
            \fi%
          \else%
            \ltx@ifpackageloaded{hyperref}{\phantomsection}{}%
            \if@filesw%
              \overridelabel{pagesLTS.\pagesLTS@pnc}%
%    \end{macrocode}
%
% We need to go forward one page (and later backward again), because
% |\overrideLTSlabel| calls a |\pagesLTS@putlabel|, and that one uses
% |\addtocounter{page}{-1}...\addtocounter{page}{+1}|, which is not
% needed here.
%
%    \begin{macrocode}
              \addtocounter{page}{+1}%
                \overrideLTSlabel{pagesLTS.\pagesLTS@pnc.local}{\theCurrentPageLocal}%
              \addtocounter{page}{-1}%
            \fi%
          \fi%
%% Code from prelim2e package again:                           %%
        }%
        \vss%
      }%
      \vss%
    }%
    \wd\@cclv=\dimen\z@
    \ht\@cclv=\dimen\@ne
    \dp\@cclv=\dimen\tw@
  \egroup
%% End of code from the prelim2e package.                      %%
  }

%    \end{macrocode}
% \end{macro}
%
% \pagebreak
%
% \begin{macro}{\EveryShipout}
% \indent At the end of each shipout, the following commands are executed:
%
%    \begin{macrocode}
\EveryShipout{%
  \ifnum\value{page}>0% \relax
  \else%
    \ifnum\value{page}=0%
      \PackageWarning{pageslts}{%
        Counter `page' is zero!\MessageBreak%
        If the page numbering scheme is not arabic\MessageBreak%
        and further not extended\MessageBreak%
        (see Page counter overflow in the pageslts\MessageBreak%
        documentation), without other measures\MessageBreak%
        this will lead to a counter overflow.\MessageBreak%
      }%
    \else%
      \ifnum\value{page}<0%
        \PackageWarning{pageslts}{%
          Counter `page' is negative: `\the\value{page}'!\MessageBreak%
          If the page numbering scheme is not arabic\MessageBreak%
          and further not extended\MessageBreak%
          (see Page counter overflow in the pageslts\MessageBreak%
          documentation), without other measures\MessageBreak%
          this will lead to a counter overflow.\MessageBreak%
        }%
      \else%
        \PackageError{pageslts}{%
          Counter `page' does not have a recognized value:\MessageBreak%
          `\the\value{page}'\MessageBreak%
          \@ehd \MessageBreak%
          }%
      \fi%
    \fi%
  \fi%
%    \end{macrocode}
%
% If the |CurrentPage| is equal to one, this is the first shipout.
%
%    \begin{macrocode}
  \ifnum \value{CurrentPage}=1% This is the first shipout!
%    \end{macrocode}
%
% We check whether some page numbering scheme was defined by
% |\pagenumbering{...}| (\textbf{as it schould be!}):
%
%    \begin{macrocode}
    \ifx\pagesLTS@called\pagesLTS@zero%
%    \end{macrocode}
%
% If it was not defined (i.\,e. |\pagesLTS@called| is zero),
% the user should be informed, that a |\pagenumbering{...}|
% is missing behind |\begin{document}|.
% Of course, it is possible that some package did some
% pages of output with |\AtBeginDocument|. In that case,
% one |\pagenumbering{...}| before |\begin{document}| and
% one |\pagenumbering{...}| (with the same argument, of course!)
% behind |\begin{document}| could help somewhat. When |\PackageError|
% was used here, the error message was not written to the screen and
% the \xfile{.log}-file, but into the document. Therefore we just make
% a note to give the error message later (|\AtEndDocument|). At that
% time unfortunately most of the document has already been compiled
% (or did not compile due to this error), but I do not know how to
% change that.
%
%    \begin{macrocode}
      \global\def\pncmissing{1}%
%    \end{macrocode}
%
% We save the current value of the page,
%
%    \begin{macrocode}
      \mathchardef\pagesLTS@tmpD=\arabic{page}%
%    \end{macrocode}
%
% determine the current page numbering scheme,
%
%    \begin{macrocode}
%% Code from Andres L\"{o}h, Universiteit Utrecht (NL)         %%
      \def\extract#1{\expandafter\extract@ #1\END}%
      \def\extract@#1\csname @#2\endcsname#3\END{#2}%
      \edef\pagesLTS@tmpQ{\extract\thepage}%
%% End of code from Andres L\"{o}h                             %%
      \let\pagesLTS@tmpP\pagesLTS@tmpQ%
%    \end{macrocode}
%
% set the current page numbering scheme to~0
% (because before the beginning of the document it should be~0),
%
%    \begin{macrocode}
      \def\pagesLTS@pnc{0}%
%    \end{macrocode}
%
% and then issue a |\pagenumbering| command with the determined
% page numbering scheme as argument:
%
%    \begin{macrocode}
      \pagenumbering{\pagesLTS@tmpP}%
%    \end{macrocode}
%
% This resets the page to one (if option |pagecontinue=false|
% was chosen), but because we do not start a new page numbering
% scheme here but manifest a page numbering scheme, which
% the user forgot to define, the page number should not have
% been reset to one. (This is the first page, but maybe the user
% wants it to have page number 2001?) Therefore we revert this here
% and set the page number to its value, which was saved before the
% |\pagenumbering| command.
%
%    \begin{macrocode}
      \setcounter{page}{\pagesLTS@tmpD}%
    \fi%
%    \end{macrocode}
%
% We are at the first page, so we put the label here.
%
%    \begin{macrocode}
    \pagesLTS@writelabel{0}%
  \fi%
%    \end{macrocode}
%
% If the current page numbering scheme |\pagesLTS@pnc| is
% |\pagesLTS@fns| (which is defined as \texttt{fnsymbol}),
% the label is set by \linebreak
% |\@pagesLTS@Prelim@EveryShipout| (see just above),
% and |\pagesLTS@esov| is set to the (real) number (not the name) of
% this page numbering scheme, |\arabic{pagesLTS.fnsymbol.cont}|.\\
% When no more pages with \xpackage{fnsymbol} page
% \textquotedblleft number\textquotedblright{} are shipped out,
% the value remains fixed and we have our reference to the last page
% of the \xpackage{fnsymbol} page numbering range.
% (At least we will have that reference after some more work,
% see below).
%
%    \begin{macrocode}
  \ifx\pagesLTS@pnc\pagesLTS@fns%
    \@pagesLTS@Prelim@EveryShipout%
    \gdef\pagesLTS@esov{\arabic{pagesLTS.fnsymbol.cont}}%
%    \end{macrocode}
%
% When another page numbering scheme was reused (in the example file
% \xpackage{Roman}), we also need to apply\\
% |\@pagesLTS@Prelim@EveryShipout|,
% because otherwise we would get multiply defined labels.
%
%    \begin{macrocode}
  \else%
    \ifnum \value{pagesLTS.pnc.\pagesLTS@pnc}>1%
      \@pagesLTS@Prelim@EveryShipout%
    \fi%
  \fi%
%    \end{macrocode}
%
% The |CurrentPage| as well as the |pagesLTS.current.local.\pagesLTS@pnc|
% are advanced by one (because one page was shipped out and the next is
% about to begin).
%
%    \begin{macrocode}
  \addtocounter{CurrentPage}{1}%
  \addtocounter{pagesLTS.current.local.\pagesLTS@pnc}{1}%
  }

%    \end{macrocode}
% \end{macro}
%
% \begin{macro}{\pagesLTS@putlabelhyper}
% \indent Here the labels are set, if the \xpackage{hyperref} package
% was loaded. Simply using |\label| would not work, because labels
% wait for the output routines to work, and there may not be any
% more invocations of the output routines. To force the write out
% we need to do an |\immediate| write.
%
%    \begin{macrocode}
\newcommand{\pagesLTS@putlabelhyper}[2]{%
  \ifHy@pageanchor \relax%
  \else%
%    \end{macrocode}
%
% If the \xpackage{hyperref} package is used, but pageanchors are disabled,
% the hyperlinking will not work.
%
%    \begin{macrocode}
    \PackageError{pagesLTS}{hyperref option pageanchor disabled}{%
      The \string\lastpageref{#1} link doesn't work\MessageBreak%
      using hyperref with disabled option `pageanchor'.\MessageBreak%
    }%
  \fi
%    \end{macrocode}
%
% If use of the \xfile{.aux}-file is allowed, the label for
% \texttt{LastPage} is written into that file, the page
% reference depending on the options, which where set for
% the \xpackage{hyperref} package.
%
%    \begin{macrocode}
%% The following code is from the hyperref package             %%
%% [2010/04/17 v6.80x; newer versions are available]           %%
%% by Heiko Oberdiek (Big Thanks!).                            %%
  \if@filesw
    \begingroup
      \let\@number\@firstofone
      \ifHy@pageanchor
        \ifHy@hypertexnames
          \ifHy@plainpages
            \def\Hy@temp{\arabic{page}}%
          \else
            \Hy@unicodefalse
%% Code not from hyperref package:                             %%
%% The following lines are modified from the hyperref package. %%
%% Without the modification, after the first shipout "PD1" is  %%
%% inserted each time |\pdfstringdef\Hy@temp{\thepage}| is     %%
%% executed (if |fnsymbol| is not used).                       %%
            \ifnum \value{CurrentPage}=1%
              \ifx\pagesLTS@pnc\pagesLTS@fns%
                \pdfstringdef\Hy@temp{\thepage}%
              \else%
                \def\Hy@temp{\thepage}%
              \fi%
            \else%
              \pdfstringdef\Hy@temp{\thepage}%
            \fi%
%% Code from hyperref package again:                           %%
          \fi
        \else
          \def\Hy@temp{\the\Hy@pagecounter}%
        \fi
      \fi
%% End of code from the hyperref package.                      %%
%% (The following four lines are modified                      %%
%%  from the hyperref package.)                                %%
      \immediate\write\@auxout{\string
        \newlabel{#1}{{}{#2}{}{\ifHy@pageanchor page.\Hy@temp\fi}{}}}%
    \endgroup%
  \fi%
  }

%    \end{macrocode}
% \end{macro}
%
% \phantomsection
% \begin{macro}{\pagesLTS@putlabel}\label{pagesLTSputlabel}
% \indent Since the page has been put out, we are on the page after
% that page. We therefore subtract one from the page counter.
%
%    \begin{macrocode}
\newcommand{\pagesLTS@putlabel}[3]{%
  \addtocounter{page}{-1}%
%    \end{macrocode}
%
% When the \xpackage{showkeys} package has been loaded in \texttt{draft} mode,
% in the margin for each label a box is displayed with the name of the label.
% \xpackage{showkeys} accomplishes this by redefining |\label|, but
% \xpackage{pageslts} does not use |\label|, but writes directly to the
% {\nobreak |\jobname|\xfile{.aux}-}\allowbreak file, and this is generally done
% \emph{after} the according page has shipped out, therefore no box can be placed
% on the \emph{preceding} page. At least \xpackage{pageslts} gives a warning, that
% \xpackage{showkeys} cannot present the respective label.
%
%    \begin{macrocode}
  \ifx\pagesLTS@SK\pagesLTS@one\relax%
    \message{^^J}%
    \message{Package pageslts Warning: Package showkeys without option final loaded,}%
    \ifnum\value{pagesLTS.pagenr}<1%
      \message{(pageslts)                but label #1 on page \thepage\space(about \theCurrentPage)}%
    \else%
      \message{(pageslts)                but label #1 on page \thepage\space(about \theCurrentPage\space of \arabic{pagesLTS.pagenr})}%
    \fi%
    \message{(pageslts)                cannot be shown, because pageslts does not use \string\label,}%
    \message{(pageslts)                but writes directly to the \jobname.aux file. ^^J}%
  \fi%
%    \end{macrocode}
%
% If the \xpackage{hyperref} package is used, the format of the labels is somewhat longer.
%
%    \begin{macrocode}
  \ltx@ifpackageloaded{hyperref}{\pagesLTS@putlabelhyper{#1}{#2}}{% else
%    \end{macrocode}
%
% \pagebreak
%
% If the \xpackage{hyperref} package is not used, there will be no hyperlinks,
% and the label is written in the way of the old \xpackage{lastpage} package.
% But we must remember to undo the label first, if it already exists.
%
%    \begin{macrocode}
    \if@filesw%
      \ifnum \value{pagesLTS.pnc.\pagesLTS@pnc}<2%
%    \end{macrocode}
%
% When the \xpackage{nameref} package is used, |\newlabel| needs five instead of two arguments:
%
%    \begin{macrocode}
        \ltx@ifpackageloaded{nameref}{%
          \immediate\write\@auxout{\string\newlabel{#1}{{}{#2}{}{}{}}}}{% else
          \immediate\write\@auxout{\string\newlabel{#1}{{}{#2}}}}%
      \else%
        \edef\pagesLTStmpA{#1}%
        \edef\pagesLTStmpB{pagesLTS.\pagesLTS@pnc.local}%
        \ifx\pagesLTStmpA\pagesLTStmpB%
          \edef\pagesLTStmpA{#3}%
          \ifx\pagesLTStmpA\pagesLTS@one%
%    \end{macrocode}
%
% Only when the third argument of |\pagesLTS@putlabel| is $1$,
% we do need to undo the label. Otherwise there is no label to
% undo, and the \xpackage{undolabl} package would give an error.
%
%    \begin{macrocode}
            \immediate\write\@auxout{\string\undonewlabel{#1}}%
          \fi%
        \fi%
        \ltx@ifpackageloaded{nameref}{%
          \immediate\write\@auxout{\string\newlabel{#1}{{}{#2}{}{}{}}}}{% else
          \immediate\write\@auxout{\string\newlabel{#1}{{}{#2}}}}%
      \fi%
    \fi%
  }%
%    \end{macrocode}
%
% After the writeout we restore the page number again,
% since there might be other things still to be done.
%
%    \begin{macrocode}
  \addtocounter{page}{+1}%
  }

%    \end{macrocode}
% \end{macro}
%
% \pagebreak
%
% \begin{macro}{\pagesLTS@putlabels}
% \indent |\pagesLTS@putlabels| is nearly identical to |\pagesLTS@putlabelV|:
%
%    \begin{macrocode}
\newcommand{\pagesLTS@putlabels}{%
  \addtocounter{page}{-1}%
  \addtocounter{CurrentPage}{-1}%
  \addtocounter{pagesLTS.current.local.\pagesLTS@pnc}{-1}%
%    \end{macrocode}
%
% If |\pagenumbering{...}| has not been used,
% |\pagesLTS@pnc| is still zero ($0$, |\pagesLTS@zero|),
% and the according warning message is given.
%
%    \begin{macrocode}
  \ifx\pagesLTS@pnc\pagesLTS@zero%
   \PackageWarning{pagesLTS}{No page numbering scheme found:\MessageBreak%
     \pagesLTS@messageNPN }%
%    \end{macrocode}
%
% \noindent otherwise the numbered label is written,
% and if the page numbering scheme was not used before,
% the unnumbered label is written, too.
%
%
%    \begin{macrocode}
  \else%
    \pagesLTS@writelabel{\pagesLTS@pnc.\arabic{pagesLTS.pnc.\pagesLTS@pnc}}%
    \ifnum \value{pagesLTS.pnc.\pagesLTS@pnc}<2%
      \ifx\pagesLTS@pnc\pagesLTS@fns% \relax
      \else%
        \pagesLTS@writelabel{\pagesLTS@pnc}%
      \fi%
    \fi%
  \fi%
%    \end{macrocode}
%
% \noindent Before the label for the \texttt{LastPages} can be put,
% we must advance one page again, because |\pagesLTS@putlabel| itself
% goes back one page (and at its end forward again).
%
%    \begin{macrocode}
  \addtocounter{page}{+1}%
  \pagesLTS@putlabel{LastPages}{\theCurrentPage}{1}%
%    \end{macrocode}
%
% \noindent Here should follow a\\
%  |\addtocounter{page}{-1}|,\\
% but we have to remember to increase the page counters again,
% which were decreased at the start of this |\pagesLTS@putlabels|
% command, and that would include\\
%  |\addtocounter{page}{+1}|,\\
% therefore this two lines cancel each other and
% therefore just can be skipped. But the other counters have to
% be increased:
%
%    \begin{macrocode}
  \addtocounter{CurrentPage}{+1}%
  \addtocounter{pagesLTS.current.local.\pagesLTS@pnc}{1}%
  }

%    \end{macrocode}
% \end{macro}
%
% \pagebreak
%
% \begin{macro}{\AtBeginDocument}
% \indent |\AtBeginDocument| we write into the \xfile{aux} file regarding
% the double loading of the package.
% Then it is checked whether the \texttt{endfloat}
% package has been loaded, whether it is newer than March 1992 (i.\,e. at least
% April 1992 v2.0), in which case it is compatible with this
% \xpackage{pageslts} package.\\
% If it is even newer than 2011/12/24, it is the recent version
% (as of the time of last revision of this documentation:
% 2011/12/25 v2.5d).
%
%    \begin{macrocode}
\AtBeginDocument{%
  \if@filesw%
    \immediate\write\@auxout{\string\gdef\string\pagesLTS@loaded{p@gesLTSnotlo@ded}}%
  \fi%
  \@ifpackageloaded{endfloat}%
   {\@ifpackagelater{endfloat}{1992/03/31}% April 1992 v2.0
    {\@ifpackagelater{endfloat}{2011/12/24}{% 2011/12/25 v2.5d
       \relax}%
%    \end{macrocode}
%
% If it is compatible, but not the recent version, a warning is given:
%
%    \begin{macrocode}
     {\PackageWarningNoLine{pageslts}{Old endfloat package detected:\MessageBreak%
        There is a newer version of the endfloat package available.\MessageBreak%
        Please consider updating your version.\MessageBreak%
        The pageslts package might be incompatible with\MessageBreak%
        your current endfloat package.\MessageBreak%
      }%
     }%
%    \end{macrocode}
%
% If it is so very old, that it is not compatible,
% an Error message is given:
%
%    \begin{macrocode}
    }{\PackageError{pageslts}{Incompatible, very old endfloat package detected.}%
      {The very old version 2.0 (and earlier) of the\MessageBreak%
       endfloat package actually redefined the \string\enddocument ,\MessageBreak%
       and so interfered drastically with the LaTeX2e commands\MessageBreak%
       which make use of \string\AtEndDocument .\MessageBreak%
       Newer versions of the endfloat package exists\MessageBreak%
       (at least: v2.5d as of 2011/12/25)\MessageBreak%
       in modern documentation form,\MessageBreak%
       which should be available from CTAN.\MessageBreak%
       Please update your endfloat package\MessageBreak%
       for use with the pageslts package.\MessageBreak %
      }%
     }%
   }{}%
%    \end{macrocode}
%
% \pagebreak
%
% It is checked whether the old \xpackage{lastpage} package has been loaded.\\
% (If it has been loaded indeed, the |\lastpage@putlabel| is
% \textquotedblleft killed\textquotedblright ,
% see subsection~\ref{ssec:lastpage}.)
%
%    \begin{macrocode}
  \@ifpackageloaded{lastpage}%
   {\@ifpackagelater{lastpage}{2010/07/28}% 2010/07/29 v1.2a
      {\@ifpackagelater{lastpage}{2013/01/27}% 2013/01/28 v1.2l
         {\PackageWarning{pageslts}{lastpage package detected.\MessageBreak%
           With pageslts package in use, lastpage has no function.\MessageBreak%
           Just remove the lastpage package from your document.\MessageBreak%
           }%
         }{%
         \PackageWarning{pageslts}{Old lastpage package detected.\MessageBreak%
           With pageslts package in use, lastpage has no function.\MessageBreak%
           Just remove the lastpage package from your document.\MessageBreak%
           At least update it!\MessageBreak%
           }%
         }%
      }{%
       \PackageWarning{pageslts}{Incompatible package lastpage detected:\MessageBreak%
         Package pageslts was loaded, but also an old\MessageBreak%
         version of the lastpage package.\MessageBreak%
         pageslts has all functionality of the lastpage\MessageBreak%
         package (and more), so just remove the lastpage\MessageBreak%
         package from your document.\MessageBreak%
         (At least update it!)\MessageBreak%
         pageslts will now ''kill'' the lastpage@putlabel\MessageBreak%
         command of the lastpage package.\MessageBreak%
       }%
       \gdef\lastpage@putlabel{\relax}%
      }%
   }{}%
%    \end{macrocode}
%
% Further it is checked whether the \xpackage{alphalph} package has been loaded.
% If that is the case, the commands are defined accordingly.\\
%
%    \begin{macrocode}
  \ltx@ifpackageloaded{alphalph}%
    {\newalphalph{\AlphMult}[mult]{\@Alph}{26}%
     \newalphalph{\alphMult}[mult]{\@alph}{26}%
     \newalphalph{\fnsymbolmult}[mult]{\@fnsymbol}{5}%
    }{}%
%    \end{macrocode}
%
% \pagebreak
%
% Further it is checked whether the \xpackage{hyperref} package has been loaded:
%
%    \begin{macrocode}
  \ltx@ifpackageloaded{hyperref}{%
%    \end{macrocode}
%
% and whether the \xpackage{pdfpages} package is loaded:
%
%    \begin{macrocode}
     \@ifpackageloaded{pdfpages}%
       {\PackageWarningNoLine{pageslts}{Package pdfpages detected.\MessageBreak%
          Using hyperref with pdfpages can cause problems. See\MessageBreak%
          https://www.ctan.org/pkg/pax\MessageBreak%
          for the Pdf Annotations eXtractor, pax%
          }%
       }{% \relax
        }%
%    \end{macrocode}
%
% The \xpackage{undolabl} package has been updated and now uses |\undonewlabel| with only
% one argument.
%
%    \begin{macrocode}
     \@ifpackageloaded{undolabl}%
       {\@ifpackagelater{undolabl}{2010/07/14}%    2010/07/15 v1.0d
          {\@ifpackagelater{undolabl}{2015/03/29}% 2015/03/29 v1.0l
             {% recent version as of the time of last revision of this package: OK
             }{% old, but not obsolete version
               \PackageWarningNoLine{pageslts}{Old version of undolabl package used.\MessageBreak%
                 See https://www.ctan.org/pkg/undolabl\MessageBreak%
                 for a new version.\MessageBreak%
                }%
             }
          }{\PackageError{pageslts}{Incompatible, obsolete version of undolabl package used.}{%
              See https://www.ctan.org/pkg/undolabl\MessageBreak%
              for a new version.\MessageBreak%
              Type  X <return>  to quit.\MessageBreak%
             }%
          }
       }{\PackageError{pageslts}{Package undolabl missing}{%
           Package undolabl not found.\MessageBreak%
           The pageslts package needs the undolabl package.\MessageBreak%
           See https://www.ctan.org/pkg/undolabl\MessageBreak%
           Type  X <return>  to quit.\MessageBreak%
           }%
       }
%    \end{macrocode}
%
% \pagebreak
%
% Additionally a version check of the available \xpackage{hyperref}
% package is performed and if need be a warning is issued:
%
%    \begin{macrocode}
     \@ifpackagelater{hyperref}{2012/11/05}{% 2012/11/06 v6.83m
       \relax}{%
       \PackageWarningNoLine{pageslts}{Old hyperref package detected:\MessageBreak%
         There is a newer version of the\MessageBreak%
         hyperref package available:\MessageBreak%
         https://www.ctan.org/pkg/hyperref\MessageBreak%
         Please consider updating your version.\MessageBreak%
       }%
     }%
     %% pageslts supports the use of the package hyperref by
     %% Heiko Oberdiek (hyperref version 2012/11/06 v6.83m).
     %% pageslts may work with earlier versions of this packages,
     %% but this was not tested.
    }{%
%    \end{macrocode}
%
% If no \xpackage{hyperref} package in use is detected, a warning is issued, too:
%
%    \begin{macrocode}
      \PackageWarning{pageslts}{Package hyperref NOT detected.\MessageBreak%
        pageslts would support hyperref. The page references\MessageBreak%
        will NOT be hyperlinked!\MessageBreak }%
     }%
%    \end{macrocode}
%
% If the \xpackage{showkeys} package has been loaded in \texttt{draft} mode,
% warnings about missing label boxes will be issued (see p.~\pageref{pagesLTSputlabel}),
% but because it cannot be checked for \xpackage{showkeys} after |\AtBeginDocument|,
% the check must be done here and the result remembered.
%
%    \begin{macrocode}
  \ltx@ifpackageloaded{showkeys}{%
    \@ifpackagewith{showkeys}{final}{}{\gdef\pagesLTS@SK{1}}}{% else \relax
    }%
  }

%    \end{macrocode}
% \end{macro}
%
% \begin{macro}{\AtEndDocument}
% \indent |\AtEndDocument| the check is repeated. If \xpackage{showkeys} was
% loaded after the check performed before, an error message is issued.
%
%    \begin{macrocode}
\AtEndDocument{%
  \ltx@ifpackageloaded{showkeys}{%
    \ifx\pagesLTS@SK\pagesLTS@one\relax%
    \else%
      \providecommand*\pagesLTS@sklf[1]{}%
      \ifx\showkeyslabelformat\pagesLTS@sklf\relax%
      \else%
        \PackageError{pageslts}{Package showkeys loaded after pageslts}{%
          The showkeys package with option "final" has been loaded\MessageBreak%
          after the pageslts package.\MessageBreak%
          Please first load showkeys and then pageslts}%
      \fi%
    \fi%
  }{% else no showcase, \relax
  }%
%    \end{macrocode}
%
% We also give the error message about the missing (i.\,e.~not found)
% page numbering scheme, which could not be given in |\EveryShipout|.
%
%    \begin{macrocode}
  \ifx\pncmissing\pagesLTS@one\relax%
    \PackageError{pageslts}{pagenumbering scheme missing}{\pagesLTS@messageNPN }%
  \fi%
%    \end{macrocode}
%
% Then we put in a |\message| to show, in what order things (which were called)
% are done (see subsection~\ref{ssec:aed}).
%
%    \begin{macrocode}
  \message{^^J%
    AED: pageslts setting LastPage ^^J}%
%    \end{macrocode}
%
% After this we issue a |\clearpage| to put out all floats,
% which are still floating, remember the page number (if |fnsymbol|),
% and after that we place the \texttt{LastPage} label.
%
%    \begin{macrocode}
  \clearpage%
  \ifx\pagesLTS@pnc\pagesLTS@fns%
    \def\pagesLTS@tmpA{\arabic{pagesLTS.fnsymbol.local}}%
    \ifnum \pagesLTS@eso=\pagesLTS@tmpA%
      \gdef\pagesLTS@rerun{0}%
    \else%
      \gdef\pagesLTS@rerun{1}%
    \fi%
    \if@filesw%
      \immediate\write\@auxout{\string\gdef\string\pagesLTS@eso{\pagesLTS@tmpA}}%
    \fi%
  \fi%
  \pagesLTS@putlabel{LastPage}{\thepage}{1}%
%    \end{macrocode}
%
% We do not need the temporary definition any more.
%
%    \begin{macrocode}
  \let\pagesLTS@tmpA\undefined%
  }

%    \end{macrocode}
% \end{macro}
%
% \begin{macro}{\AfterLastShipout}
% \indent |\AfterLastShipout| is a command from \textsc{Heiko Oberdiek's}
% \texttt{atveryend} package (see above).
%
%    \begin{macrocode}
\AfterLastShipout{%
%    \end{macrocode}
%
% If writing to the \xfile{.aux} file is allowed:
%
%    \begin{macrocode}
  \if@filesw%
%    \end{macrocode}
%
% The number of pages with the \xpackage{fnsymbol} page numbering scheme,
% |\pagesLTS@esov|, is saved via the \xfile{.aux} file (if it is not zero):
%
%    \begin{macrocode}
    \ifx\pagesLTS@esov\pagesLTS@zero%
    \else%
      \immediate\write\@auxout{\string
        \pagesLTS@ifcounter{pagesLTS.fnsymbol.local}}%
      \immediate\write\@auxout{\string
        \setcounter{pagesLTS.fnsymbol.local}{\pagesLTS@esov}}%
    \fi%
%    \end{macrocode}
%
% If the \xpackage{hyperref} package is in use, and the page numbering scheme
% of the last page is \xpackage{fnsymbol}, everything is quite more complicated.
% Therefore |\lastpageref| is switched from simple |\lastpagereftxt|
% to the more difficult |\lastpagerefend|.
%
%    \begin{macrocode}
    \ltx@ifpackageloaded{hyperref}{%
      \ifx\pagesLTS@pnc\pagesLTS@fns%
          \immediate\write\@auxout{\string
            \gdef\string\lastpageref{\string\lastpagerefend}}%
      \fi%
     }{}%
  \fi%
%    \end{macrocode}
%
% At the call of a |\pagenumbering{...}| command, everything
% for a split page numbering scheme is organized. For the last
% page numbering scheme, there is no |\pagenumbering{...}| command
% at the end, so we need to handle this here:
%
%    \begin{macrocode}
  \pagesLTS@ifcounter{pagesLTS.\pagesLTS@pnc.\arabic{pagesLTS.pnc.\pagesLTS@pnc}.local.count}%
  \setcounter{pagesLTS.\pagesLTS@pnc.\arabic{pagesLTS.pnc.\pagesLTS@pnc}.local.count}{%
    \value{pagesLTS.current.local.\pagesLTS@pnc}}%
%    \end{macrocode}
%
% And we are one page after the last one (|\AfterLastShipout|!),
% so we go back one page. (We again borrow the \texttt{pagesLTS.pnc.0} counter for the computations
% instead of defining yet another one.)
%
%    \begin{macrocode}
  \addtocounter{pagesLTS.\pagesLTS@pnc.\arabic{pagesLTS.pnc.\pagesLTS@pnc}.local.count}{-1}
  \ifnum \value{pagesLTS.pnc.\pagesLTS@pnc}>1%
    \mathchardef\pagesLTS@tmpD=\arabic{pagesLTS.pnc.0}%
    \setcounter{pagesLTS.pnc.0}{\value{pagesLTS.pnc.\pagesLTS@pnc}}%
    \@tempcnta=\value{pagesLTS.pnc.0}\relax%
    \loop%
      \ifnum\@tempcnta>1\relax%
        \addtocounter{pagesLTS.pnc.0}{-1}%
        \addtocounter{pagesLTS.\pagesLTS@pnc.\arabic{pagesLTS.pnc.\pagesLTS@pnc}.local.count}{%
          -\value{pagesLTS.\pagesLTS@pnc.\arabic{pagesLTS.pnc.0}.local.count}}%
        \@tempcnta=\value{pagesLTS.pnc.0}\relax%
    \repeat%
    \setcounter{pagesLTS.pnc.0}{\pagesLTS@tmpD}%
  \fi%
  \if@filesw%
    \immediate\write\@auxout{\string
      \pagesLTS@ifcounter{pagesLTS.\pagesLTS@pnc.\arabic{pagesLTS.pnc.\pagesLTS@pnc}.local.cnt}}%
    \edef\pagesLTS@tmpA{\arabic{pagesLTS.\pagesLTS@pnc.\arabic{pagesLTS.pnc.\pagesLTS@pnc}.local.count}}%
    \immediate\write\@auxout{\string
      \setcounter{pagesLTS.\pagesLTS@pnc.\arabic{pagesLTS.pnc.\pagesLTS@pnc}.local.cnt}{\pagesLTS@tmpA}}%
    \let\pagesLTS@tmpA\undefined%
  \fi%
%    \end{macrocode}
%
% We need to save (via the \xfile{.aux} file) the page name |\thepage| and
% the page number |\arabic{CurrentPage}| of the last page,
% in case the last page has \xpackage{fnsymbol} page numbering scheme.
%
%    \begin{macrocode}
  \addtocounter{page}{-1}%
  \edef\pagesLTS@tmpA{\thepage}%
  \if@filesw%
    \immediate\write\@auxout{\string
      \gdef\string\pagesLTS.lastpage{\pagesLTS@tmpA}}%
  \fi%
  \addtocounter{page}{+1}%
  \addtocounter{CurrentPage}{-1}%
  \edef\pagesLTS@tmpB{\arabic{CurrentPage}}%
  \if@filesw%
    \immediate\write\@auxout{\string
      \setcounter{pagesLTS.pagenr}{\pagesLTS@tmpB}}%
  \fi%
  \addtocounter{CurrentPage}{+1}%
%    \end{macrocode}
%
% The \texttt{VeryLastPage} label is set here, and when
% |\lastpageref{VeryLastPage}| instead of |\lastpageref{LastPage}| is used,
% it should really point to the last page.
% |LastPage| and |VeryLastPage| should be identical,
% unless a package was active with output |\AtEndDocument|
% \textit{after} the \xpackage{pageslts} package.
%
%    \begin{macrocode}
  \message{^^J%
    AED: pageslts setting VeryLastPage via AfterLastShipout ^^J}%
  \pagesLTS@putlabel{VeryLastPage}{\thepage}{1}%
%    \end{macrocode}
%
% The \texttt{LastPages} label is set here, and
% |\lastpageref{LastPages}| gives the total number of pages
% and points to the (very) last page.
%
%    \begin{macrocode}
  \message{^^J%
    AED: pageslts setting LastPages via AfterLastShipout ^^J}%
  \pagesLTS@putlabels%
  \typeout{^^J}%
  \ifodd\pagesLTS@tmpB%
    \@PackageInfoNoLine{pageslts}{Total number of pages is odd}%
  \else%
    \@PackageInfoNoLine{pageslts}{Total number of pages is even}%
  \fi%
  \typeout{^^J}%
%    \end{macrocode}
%
% We do not need the temporary definitions any more.
%
%    \begin{macrocode}
  \let\pagesLTS@tmpA\undefined%
  \let\pagesLTS@tmpB\undefined%
  }

%    \end{macrocode}
% \end{macro}
%
% \pagebreak
%
% \begin{macro}{\AtEndAfterFileList}
% \indent |\AtEndAfterFileList{...}| is even later:
% \begin{quote}
% \textquotedblleft After the .aux file closing and reading \LaTeX{} prints the
% file list if requested by |\listfiles|. Then this hook is executed.\textquotedblright
% \end{quote}
% (\xpackage{atveryend} package of \textsc{Heiko Oberdiek}, v1.7 as of 2011/04/23,
% newer version available.) Here it is used for a rerun hint.\\
% For example if the page numbering scheme of the last page of the
% \textsf{pageslts-example.tex} file is changed to |fnsymbol| and two runs
% of \pdfLaTeX{} are done, \pdfLaTeX{} will be happy and will not complain
% about changed labels. But indeed, a \textit{third} run is necessary
% and indicated by the warning message below.
%
%    \begin{macrocode}
\AtEndAfterFileList{%
  \ifx\pagesLTS@rerun\pagesLTS@one%
    \PackageWarningNoLine{pageslts}{%
      Label(s) may have changed.\MessageBreak%
      Rerun to get cross-references right%
     }%
  \fi%
  }

%    \end{macrocode}
% \end{macro}
%
% \begin{macro}{\frontmatter}
% \indent |\frontmatter| often contains |\pagenumbering|, but for some unknown reason
% there are problems when another |\pagenumbering| with different page numbering scheme
% has been used before on the same page. (This would not make any sense anyway,
% because one page can only have one page numbering scheme.) This problem does not
% occur when two |\pagenumbering| commands are used inside normal text.
% Thus we need to check whether |\frontmatter| has been defined,
% whether it changes the page numbering scheme, and whether the page numbering scheme
% before |\frontmatter| was initiated at the same page.
% \end{macro}
%
% We need an |\ifundefined|, which neither requires \eTeX{} nor
% defines the command, which is to be tested, to |\relax| (in which case
% it would be no longer undefined for further tests). For this we use
% code from \textsc{Markus Kohm} similar to his |\scr@ifundefinedorrelax| from %
% \xpackage{scrbase.sty} of the KOMA script bundle,
% but without the necessarity to use \eTeX :
%
%    \begin{macrocode}
\newcommand{\pagesLTS@ifundefinedorrelax}[1]{%
  % Modified Markus Kohm code for use without e-TeX
  \begingroup\expandafter\expandafter\expandafter\endgroup
  \expandafter\ifx\csname #1\endcsname\relax%
    \expandafter\ifx\csname #1\endcsname\relax%
      \expandafter\expandafter\expandafter\@firstoftwo
    \else%
      \expandafter\expandafter\expandafter\@secondoftwo
    \fi%
  \else%
    \expandafter\@firstoftwo
  \fi%
  }

\pagesLTS@ifundefinedorrelax{frontmatter}{% \relax
}{% else
  \ltx@GlobalPrependToMacro{\frontmatter}{\pagesLTS@prefrontmatter}%
  \ltx@GlobalAppendToMacro{\frontmatter}{\pagesLTS@postfrontmatter}%
 }

%    \end{macrocode}
%
% \begin{macro}{\pagesLTS@prefrontmatter}
%
% At the beginning of the real (i.e. unchanged) |\frontmatter| (FMB), we remember the
% current (c) page numbering (pn) scheme: its name (n) and page number (p).
% |\pagesLTS@FMBpncp| would be |1| if the according |\pagenumbering| command
% was used on the same page at the |\frontmatter|.
%
%    \begin{macrocode}
\newcommand{\pagesLTS@prefrontmatter}{%
  \xdef\pagesLTS@FMBpncn{\pagesLTS@pnc}%
  \xdef\pagesLTS@FMBpncp{\arabic{pagesLTS.current.local.\pagesLTS@pnc}}%
  }

%    \end{macrocode}
% \end{macro}
%
% \begin{macro}{\pagesLTS@postfrontmatter}
%
% The named checks are performed and in case of the named problem an error message
% is given. Any idea how to automatically solve this issue?
%
%    \begin{macrocode}
\newcommand{\pagesLTS@postfrontmatter}{%
  \xdef\pagesLTS@FMEpncn{\pagesLTS@pnc}%
  \ifx\pagesLTS@FMBpncn\pagesLTS@FMEpncn%
  \else%
    \ifx\pagesLTS@FMBpncn\pagesLTS@zero%
    \else%
      \ifx\pagesLTS@FMBpncp\pagesLTS@one%
        \PackageError{pageslts}{\string\pagenumbering\space before \string\frontmatter}{%
          Do not use \string\pagenumbering{\pagesLTS@FMBpncn} before \string\frontmatter \MessageBreak%
          on the same page!\MessageBreak%
          \string\frontmatter\space (re)defines the page numbering scheme to \pagesLTS@FMEpncn , \MessageBreak%
          thus earlier use of \string\pagenumbering{\pagesLTS@FMBpncn} on the same page is useless anyway.}%
      \fi%
    \fi%
  \fi%
}

%    \end{macrocode}
% \end{macro}
%
%    \begin{macrocode}
%</package>
%    \end{macrocode}
%
% \newpage
%
% \section{Installation}
%
% \subsection{Downloads\label{ss:Downloads}}
%
% Everything is available at \url{https://www.ctan.org},
% but may need additional packages themselves.\\
%
% \DescribeMacro{pageslts.dtx}
% For unpacking the |thumbs.dtx| file and constructing the documentation it is required:
% \begin{description}
% \item[-] \TeX Format \LaTeXe{}: \url{https://www.CTAN.org}
%
% \item[-] document class \xclass{ltxdoc}, 2015-03-26, 2.0w, \url{https://www.ctan.org/pkg/ltxdoc}
%
% \item[-] package \xpackage{holtxdoc}, 2012/03/21, v0.24, \url{https://www.ctan.org/pkg/holtxdoc}
%
% \item[-] package \xpackage{hypdoc}, 2011/08/19, v1.11, \url{https://www.ctan.org/pkg/hypdoc}
%
% \item[-] package \xpackage{geometry}, 2010/09/12, v5.6, \url{https://www.ctan.org/pkg/geometry}
%
% \item[-] package \xpackage{ulem}, 2012/05/18, no version number given, \url{https://www.ctan.org/pkg/ulem}
% \end{description}
%
% \DescribeMacro{pageslts.sty}
% The |pageslts.sty| for \LaTeXe{} (i.\,e. all documents using
% the \xpackage{pageslts} package) requires:
% \begin{description}
% \item[-] \TeX{} Format \LaTeXe{}, \url{https://www.CTAN.org}
%
% \item[-] package \xpackage{atveryend}, 2011/06/30, v1.8, \url{https://www.ctan.org/pkg/atveryend}
%
% \item[-] package \xpackage{everyshi}, 2001/05/15, v3.00, \url{https://www.ctan.org/pkg/everyshi}
%
% \item[-] package \xpackage{kvoptions}, 2011/06/30, v3.11, \url{https://www.ctan.org/pkg/kvoptions}
%
% \item[-] package \xpackage{letltxmacro}, 2010/09/02, v1.4, \url{https://www.ctan.org/pkg/letltxmacro}
%
% \item[-] package \xpackage{ltxcmds}, 2011/11/09, v1.22, \url{https://www.ctan.org/pkg/ltxcmds}
%
% \item[-] package \xpackage{rerunfilecheck}, 2011/04/15, v1.7, \url{https://www.ctan.org/pkg/rerunfilecheck}
%
% \item[-] package \xpackage{undolabl}, 2015/03/29, v1.0l, \url{https://www.ctan.org/pkg/undolabl}\\
% \end{description}
%
% \DescribeMacro{pageslts-example.tex}
% The |pageslts-example.tex| requires the same files as all
% documents using the \xpackage{pageslts} package, and additionally:
% \begin{description}
% \item[-] class \textsf{article}, 2014/09/29, v1.4h, from \textsf{classes.dtx}: \url{https://www.ctan.org/pkg/classes}
%
% \item[-] package \xpackage{alphalph}, 2011/05/13, v2.4, \url{https://www.ctan.org/pkg/alphalph}
%
% \item[-] package \xpackage{lipsum}, 2014/07/27, v1.3, \url{https://www.ctan.org/pkg/lipsum}
%
% \item[-] package \xpackage{showkeys}, 2014/10/28, v3.17, \url{https://www.ctan.org/pkg/showkeys}
%
% \item[-] package \xpackage{hyperref}, 2012/11/06, v6.83m, \url{https://www.ctan.org/pkg/hyperref}
%
% \item[-] package \xpackage{pageslts}, 2015/12/21, v1.2f, \url{https://www.ctan.org/pkg/pageslts}\\
%   (Well, it is the example file for this package, and because you are reading the
%    documentation for the \xpackage{pageslts} package, it can be assumed that you already
%    have some version of it~-- is it the current one?)
% \end{description}
%
% \pagebreak
%
% \DescribeMacro{papermas}
% The \xpackage{papermas} package is not required, but requires itself the
% the \xpackage{pageslts} package and can be considered as kind of add-on:
% \begin{description}
% \item[-] package \xpackage{papermas}, 2011/08/22, v1.0h, \url{https://www.ctan.org/pkg/papermas}
% \end{description}
%
% \DescribeMacro{endfloat}
% The \xpackage{endfloat} package is not required, but because
% the \xpackage{pageslts} package is incompatibel with very old versions
% of the \xpackage{endfloat} package (see subsection~\ref{sec:endfloat}),
% here the recent one is listed:
% \begin{description}
% \item[-] package \xpackage{endfloat}, 2011/12/25, v2.5d, \url{https://www.ctan.org/pkg/endfloat}
% \end{description}
%
% \DescribeMacro{prelim2e}
% The \xpackage{prelim2e} package is not required either, but because
% |Prelim@EveryShipout| code was taken from that package, it is listed, too:
% \begin{description}
% \item[-] package \xpackage{prelim2e}, 2009/05/29, v1.3, \url{https://www.ctan.org/pkg/prelim2e}
% \end{description}
%
% \DescribeMacro{fancyhdr}
% \DescribeMacro{nccfancyhdr}
% Neither \xpackage{fancyhdr} nor \xpackage{nccfancyhdr} package is required
% (the \xpackage{lastpage} package used its predecessor \xpackage{fancyheadings}),
% but because they were mentioned, also they are listed here:
% \begin{description}
% \item[-] package \xpackage{fancyhdr}, \url{https://www.ctan.org/pkg/fancyhdr}
%
% \item[-] package \xpackage{nccfancyhdr}, \url{https://www.ctan.org/pkg/nccfancyhdr}
% \end{description}
%
% \DescribeMacro{lastpage}
% \DescribeMacro{totpages}
% \DescribeMacro{totcount}
% \DescribeMacro{nofm}
% \DescribeMacro{count1to}
% \DescribeMacro{zref}
% As possible alternatives in section \ref{sec:Alternatives} there are listed
% \begin{description}
% \item[-] package \xpackage{lastpage}, 2015/03/29, v1.2m, \url{https://www.ctan.org/pkg/lastpage}
%
% \item[-] package \xpackage{totpages}, 2005/09/19, v2.00, \url{https://www.ctan.org/pkg/totpages}
%
% \item[-] package \xpackage{totcount}, 2011/01/25, v1.2, \url{https://www.ctan.org/pkg/totcount}
%
% \item[-] package \xpackage{nofm}, 1991/02/25, without version number,
%  \url{http://mirror.ctan.org/obsolete/macros/latex209/contrib/misc/nofm.sty},
%  does not work with e.\,g. \xpackage{hyperref}
%
% \item[-] package \xpackage{count1to}, 2009/05/24, v2.1, \url{https://www.ctan.org/pkg/count1to}
%
% \item[-] package \xpackage{zref}, 2012/04/04, v2.24, \url{https://www.ctan.org/pkg/zref}
% \end{description}
%
% \DescribeMacro{Oberdiek}
% \DescribeMacro{alphalph}
% \DescribeMacro{atveryend}
% \DescribeMacro{holtxdoc}
% \DescribeMacro{letltxmacro}
% \DescribeMacro{ltxcmds}
% \DescribeMacro{kvoptions}
% \DescribeMacro{rerunfilecheck}
% \DescribeMacro{zref}
% All packages of \textsc{Heiko Oberdiek's} bundle `oberdiek'
% (especially \xpackage{alphalph}, \xpackage{atveryend}, \xpackage{holtxdoc}, \xpackage{letltxmacro},
% \xpackage{ltxcmds}, \xpackage{kvoptions}, \xpackage{rerunfilecheck}, and \xpackage{zref})
% are also available in a TDS compliant ZIP archive:\\
% \url{http://mirrors.ctan.org/install/macros/latex/contrib/oberdiek.tds.zip}.\\
% It is probably best to download and use this, because the packages in there
% are quite probably both recent and compatible among themselves.
%
% \DescribeMacro{hyperref}
% \noindent \xpackage{hyperref} is not included in that bundle and needs to be
% downloaded separately,\\
% \url{http://mirrors.ctan.org/install/macros/latex/contrib/hyperref.tds.zip}.
%
% \DescribeMacro{M\"{u}nch}
% A hyperlinked list of my (other) packages can be found at
% \url{https://www.ctan.org/author/muench-hm}.
%
% \subsection{Package, unpacking TDS}
% \paragraph{Package.} This package is available on \url{CTAN.org}.
% \begin{description}
% \item[\url{http://mirrors.ctan.org/macros/latex/contrib/pageslts/pageslts.dtx}]\hspace*{0.1cm} \\
%       The source file.
% \item[\url{http://mirrors.ctan.org/macros/latex/contrib/pageslts/pageslts.pdf}]\hspace*{0.1cm} \\
%       The documentation.
% \item[\url{http://mirrors.ctan.org/macros/latex/contrib/pageslts/pageslts-example.pdf}]\hspace*{0.1cm} \\
%       The compiled example file, as it should look like.
% \item[\url{http://mirrors.ctan.org/macros/latex/contrib/pageslts/README}]\hspace*{0.1cm} \\
%       The README file.
% \end{description}
% There is also a |thumbs.tds.zip| available:
% \begin{description}
% \item[\url{http://mirror.ctan.org/install/macros/latex/contrib/pageslts.tds.zip}]\hspace*{0.1cm} \\
%       Everything in TDS compliant, compiled format.
% \end{description}
% which additionally contains\\
% \begin{tabular}{ll}
% pageslts.ins & The installation file.\\
% pageslts.drv & The driver to generate the documentation.\\
% pageslts.sty & The \xext{sty}le file.\\
% pageslts-example.tex & The example file.%
% \end{tabular}
%
% \bigskip
%
% \noindent For required other packages see the preceding subsection.
%
% \paragraph{Unpacking.} The \xfile{.dtx} file is a self-extracting
% \docstrip{} archive. The files are extracted by running the
% \xfile{.dtx} through \plainTeX :
% \begin{quote}
%   \verb|tex pageslts.dtx|
% \end{quote}
%
% About generating the documentation see paragraph~\ref{GenDoc} below.\\
%
% \paragraph{TDS.} Now the different files must be moved into
% the different directories in your installation TDS tree
% (also known as \xfile{texmf} tree):
% \begin{quote}
% \def\t{^^A
% \begin{tabular}{@{}>{\ttfamily}l@{ $\rightarrow$ }>{\ttfamily}l@{}}
%   pageslts.sty & tex/latex/pageslts.sty\\
%   pageslts.pdf & doc/latex/pageslts.pdf\\
%   pageslts-example.tex & doc/latex/pageslts-example.tex\\
%   pageslts-example.pdf & doc/latex/pageslts-example.pdf\\
%   pageslts.dtx & source/latex/pageslts.dtx\\
% \end{tabular}^^A
% }^^A
% \sbox0{\t}^^A
% \ifdim\wd0>\linewidth
%   \begingroup
%     \advance\linewidth by\leftmargin
%     \advance\linewidth by\rightmargin
%   \edef\x{\endgroup
%     \def\noexpand\lw{\the\linewidth}^^A
%   }\x
%   \def\lwbox{^^A
%     \leavevmode
%     \hbox to \linewidth{^^A
%       \kern-\leftmargin\relax
%       \hss
%       \usebox0
%       \hss
%       \kern-\rightmargin\relax
%     }^^A
%   }^^A
%   \ifdim\wd0>\lw
%     \sbox0{\small\t}^^A
%     \ifdim\wd0>\linewidth
%       \ifdim\wd0>\lw
%         \sbox0{\footnotesize\t}^^A
%         \ifdim\wd0>\linewidth
%           \ifdim\wd0>\lw
%             \sbox0{\scriptsize\t}^^A
%             \ifdim\wd0>\linewidth
%               \ifdim\wd0>\lw
%                 \sbox0{\tiny\t}^^A
%                 \ifdim\wd0>\linewidth
%                   \lwbox
%                 \else
%                   \usebox0
%                 \fi
%               \else
%                 \lwbox
%               \fi
%             \else
%               \usebox0
%             \fi
%           \else
%             \lwbox
%           \fi
%         \else
%           \usebox0
%         \fi
%       \else
%         \lwbox
%       \fi
%     \else
%       \usebox0
%     \fi
%   \else
%     \lwbox
%   \fi
% \else
%   \usebox0
% \fi
% \end{quote}
% If you have a \xfile{docstrip.cfg} that configures and enables \docstrip's
% TDS installing feature, then some files can already be in the right
% place, see the documentation of \docstrip.
%
% \subsection{Refresh file name databases}
%
% If your \TeX~distribution (\teTeX, \mikTeX,\dots) relies on file name
% databases, you must refresh these. For example, \teTeX\ users run
% \verb|texhash| or \verb|mktexlsr|.
%
% \subsection{Some details for the interested}
%
% \paragraph{Unpacking with \LaTeX.}
% The \xfile{.dtx} chooses its action depending on the format:
% \begin{description}
% \item[\plainTeX:] Run \docstrip\ and extract the files.
% \item[\LaTeX:] Generate the documentation.
% \end{description}
% If you insist on using \LaTeX\ for \docstrip\ (really,
% \docstrip\ does not need \LaTeX), then inform the autodetect routine
% about your intention:
% \begin{quote}
%   \verb|latex \let\install=y\input{pageslts.dtx}|
% \end{quote}
% Do not forget to quote the argument according to the demands
% of your shell.
%
% \paragraph{Generating the documentation.\label{GenDoc}}
% You can use both the \xfile{.dtx} or the \xfile{.drv} to generate
% the documentation. The process can be configured by the
% configuration file \xfile{ltxdoc.cfg}. For instance, put the following
% line into this file, if you want to have A4 as paper format:
% \begin{quote}
%   \verb|\PassOptionsToClass{a4paper}{article}|
% \end{quote}
%
% \noindent An example follows how to generate the
% documentation with \pdfLaTeX :
%
% \begin{quote}
%\begin{verbatim}
%pdflatex pageslts.dtx
%makeindex -s gind.ist pageslts.idx
%pdflatex pageslts.dtx
%makeindex -s gind.ist pageslts.idx
%pdflatex pageslts.dtx
%\end{verbatim}
% \end{quote}
%
% \subsection{Compiling the example}
%
% The example file, \textsf{pageslts-example.tex}, can be compiled via\\
% \indent |latex pageslts-example.tex|\\
% or (recommended)\\
% \indent |pdflatex pageslts-example.tex|\\
% and will need \textit{at least} (!) three compiler runs to get
% all references right.
%
% \pagebreak
%
% \section{Acknowledgements}
%
% I (H.-Martin M\"{u}nch) would like to thank \textsc{Jeffrey P. Goldberg}
% for inventing the \xpackage{lastpage} package. This package first started
% as a revision of the \xpackage{lastpage} package, but it became obvious
% that a replacement was needed to accomplish what this package does.
% Further I would like to thank \textsc{Heiko Oberdiek} for providing the
% |\erroralphalph| command as well as a~lot~(!) of useful packages
% (from which I also got everything I know about creating a file in
% \xfile{dtx} format, ok, say it: copying), \textsc{Martin Schr\"{o}der}
% for his \xpackage{prelim2e} package, from which I got the
% |Prelim@EveryShipout| code, \textsc{Ulrich Diez} for his code for the
% \xpackage{undolabl} package, which allows overwriting of labels,
% \textsc{Andres L\"{o}h} for the code to determine the current page
% numbering scheme, and the \Newsgroup{comp.text.tex} and
% \Newsgroup{de.comp.text.tex} newsgroups for their help in all things \TeX{}.
% For bug reports I thank Micha\l{} Herman, kwikwi, Joshua Ellis,
% and Dr. Clea F. Rees. For telling me how to fix a bug
% (and for all his shared whisdom at \url{https://tex.stackexchange.com})
% thanks go to Prof. Enrico Gregorio.
%
% \phantomsection
% \begin{History}\label{History}
%
%   Some old versions have been archived at
%   \url{http://ctanhg.scharrer-online.de/pkg/pageslts.html}.
%
%   \begin{Version}{1994/06/17, \xpackage{lastpage}}
%     \item \xpackage{lastpage} v0.99a: First shot by \textsc{Jeffrey P. Goldberg}.
%   \end{Version}
%   \begin{Version}{1994/06/25, \xpackage{lastpage}}
%     \item \xpackage{lastpage} v0.1b: Last version number created by \textsc{Jeffrey P. Goldberg}.
%   \end{Version}
%   \begin{Version}{1994/07/20, \xpackage{lastpage}}
%     \item \xpackage{lastpage} v0.1b (again): Documentation updated by \textsc{Jeffrey P. Goldberg}.\\
%       The main source code of the \xpackage{lastpage} package 1994/07/20 v0.1b was:\\
%       \begin{verbatim}
%        \NeedsTeXFormat{LaTeX2e}[1994/06/01]
%        \ProvidesPackage{lastpage}[1994/07/20 v0.1b
%          LaTeX2e package for refs to last page number (JPG)]
%        \def\lastpage@putlabel{\addtocounter{page}{-1}%
%          \immediate\write\@auxout{\string
%          \newlabel{LastPage}{{}{\thepage}}}%
%          \addtocounter{page}{1}}
%        \AtEndDocument{%
%          \message{AED: lastpage setting LastPage}%
%          \clearpage\lastpage@putlabel}%
%        \endinput
%       \end{verbatim}
%       and then \xpackage{hyperref} and \xpackage{revtex} even redefine |\lastpage@putlabel|.
%   \end{Version}
%   \begin{Version}{2010/02/18, \xpackage{lastpage}}
%     \item \xpackage{lastpage} v1.1: Proposed |LastPages| label by H.-Martin M\"{u}nch
%       on \Newsgroup{comp.text.tex}, see e.\,g.
%       \url{http://groups.google.com/group/comp.text.tex/msg/4407493da9c747f0?dmode=source};
%       now available in this \xpackage{pageslts} package.
%   \end{Version}
%   \begin{Version}{2010/05/15 v1.0 \xpackage{pagesLTS}}
%     \item \xpackage{pagesLTS} Complete rewriting of the package, so as to work with
%       \textbf{more than one page numbering scheme}; using |\AtVeryEnd|
%       for \texttt{VeryLastPage}; upgrade from \xpackage{fancyheadings} to
%       \xpackage{fancyhdr} package, then removed the need for a \xpackage{fancyhdr}
%       package at all.
%     \item Rewriting of the package, so as to work with the \xpackage{fnsymbol}
%       page numbering scheme (even on the last page).
%     \item Introduction of \xpackage{kvoptions} into this package.
%     \item Check for incompatible \xpackage{endfloat} package.
%     \item \xpackage{lastpage209.sty} for \LaTeX209 .
%     \item Replacement of |\filedate|, |-version|, |-name|,\ldots{} because
%       of \LaTeX~bug 2705:\\
%       Synopsis: Possible problem with |\fileversion| and |\filedate|\\
%       \url{http://www.latex-project.org/cgi-bin/ltxbugs2html?category=LaTeX&responsible=anyone&state=anything&keyword=lastpage&pr=latex%2F2705&search=}
%     \item \xpackage{alphalph} support included.
%     \item Page numbering extension |\erroralph| by \textsc{Heiko Oberdiek} included.
%     \item (Page-) Numbering extension for |roman| and |Roman| numbers included.
%     \item Incompatible, old \xpackage{lastpage} package \textquotedblleft killed\textquotedblright .
%     \item Example |pagesLTS-example.tex| added.
%     \item Alternatives listing (section \ref{sec:Alternatives}).
%     \item Listing of \TeX{} sources (subsection \ref{ss:Downloads}).
%     \item Complete rewriting of the documentation.
%     \item Everything in DTX framework.
%     \item New package name: \xpackage{pagesLTS} for Last, Total,
%            and page numbering Schemes pages.
%   \end{Version}
%   \begin{Version}{2010/06/01 v1.1(a) \xpackage{pagesLTS}}
%     \item Abstract changed: Negative \texttt{roman} and \texttt{Roman} page
%            numbers are now possible.
%     \item Some references to other packages have been updated.
%     \item Several typing mistakes have been corrected~-
%            both in the \xfile{sty}le file as well as in this documentation.
%   \end{Version}
%   \begin{Version}{2010/06/03 v1.1b \xpackage{pagesLTS}}
%     \item Corrected a bug in |\XXRoman|, where |\roman| instead of |\Roman| had been used.
%     \item New \xpackage{papermas} package mentioned.
%     \item Several changes in the documentation and the Readme file.
%   \end{Version}
%   \begin{Version}{2010/06/24 v1.1c \xpackage{pagesLTS}}
%     \item \xpackage{holtxdoc} warning in \xfile{drv} updated.
%     \item Removed CRLF line endings from the \xfile{dtx} file.
%     \item Corrected the location of the package at CTAN.
%           (In this version \xfile{TDS} was still missing due to packaging error.)
%     \item Corrected Message format in pagesLTS.\xfile{ins}.
%     \item Updates to the documentation.
%   \end{Version}
%   \begin{Version}{2010/07/15 v1.1d \xpackage{pagesLTS}}
%     \item Added the |\@ifclassloaded{revtex4}| code for |\lastpage@putlabel| (changed to
%            |\pagesLTS@putlabelhyper|) from the \xpackage{hyperref} package \textbf{as comment}~-
%            what is the meaning of that code?
%     \item In the documentation added the explanation of the occurence of multiply definitions
%            of the |LastPage| label with \xpackage{lastpage}, \xpackage{pagesLTS}, \xpackage{hyperref}
%            package (in that order).
%     \item Updated to (then) new version of \xpackage{undolabl} package [2010/07/15] v1.0d,
%            which uses |\undonewlabel| with only one instead of two arguments.
%     \item Added a warning message, if \xpackage{hyperref} \textit{and} \xpackage{pdfpages} are \textit{both} used.
%            (Should not \xpackage{hyperref} give this warning?)
%     \item Updates to the documentation.
%   \end{Version}
%   \begin{Version}{2010/07/29 v1.1e \xpackage{pagesLTS}}
%     \item Removed \xpackage{lastpage209.sty}, because it is now contained
%             in the \xpackage{lastpage}\xfile{.dtx}\ file, v\,$\geq 1.2$a.
%     \item Removed the |\@ifclassloaded{revtex4}| code for |\lastpage@putlabel|.
%     \item Handling of \xpackage{lastpage} package adapted to updated version~1.2(a).
%     \item Corrected error in \xpackage{lastpage} code [1994/07/20 v0.1b] given in \ref{History}~History.
%     \item Version handling for \xpackage{undolabl} package updated.
%     \item Included a |\CheckSum|.
%     \item Some minor details.
%   \end{Version}
%   \begin{Version}{2010/08/08 v1.1f \xpackage{pagesLTS}}
%     \item Version 1.1e had a bug: AlphAlph was replaced by alphalph (because that package is named like this),
%             but this was done also in commands and definitions~- now reverted.
%     \item Updates to the documentation.
%   \end{Version}
%   \begin{Version}{2010/08/12 v1.1g \xpackage{pagesLTS}}
%     \item Now the rerun warning is given \textit{after} e.\,g. the |\listfiles|,
%             increasing the chance of the user to read it (trick found in
%             \textsc{Harald Harders'} \xpackage{fnbreak} package, thanks!).
%   \end{Version}
%   \begin{Version}{2010/08/23 v1.1h \xpackage{pagesLTS}}
%     \item Renamed |\XXRoman| to |\XRoman|.
%     \item Reduced the number of needed counters.
%     \item Removed wrong \% from the driver file.
%     \item Changed the |\unit| definition (got rid of an old |\rm|).
%     \item Without use of the \xpackage{hyperref} package, labels of type
%             \texttt{pagesLTS.}\meta{page numbering scheme}\texttt{.local}
%             became multiply defined. Now an |\undolabl| was inserted.
%     \item Diverse details.
%   \end{Version}
%   \begin{Version}{2010/08/25 v1.1i \xpackage{pagesLTS}}
%     \item Bug fix: \textsf{tcilatex} defines the |\hyperref| command, therefore
%             for \xpackage{hyperref} package detection this had to be changed to
%             |\Hy@Warning|.
%   \end{Version}
%   \begin{Version}{2010/09/12 v1.1j \xpackage{pagesLTS}}
%     \item Bug fix: \LaTeX{} issued a \textquotedblleft \texttt{Label(s) may have changed.%
%             Rerun to get cross-references right.}\textquotedblright -warning,
%             even if labels had not changed but were overwritten.
%     \item Starred version of |\lastpageref| for suppressing hyperlinks introduced.
%     \item A lot of details.
%   \end{Version}
%   \begin{Version}{2010/09/22 v1.1k \xpackage{pagesLTS}}
%     \item When no \texttt{fnsymbol} pagenumbering scheme is used, the respective
%             counters are not defined, saving three counters.
%     \item Moved the package from \texttt{.../latex/muench/pagesLTS/...} to
%             \texttt{.../latex/pagesLTS/...}.\\
%             (Please make sure that the old version of the \xpackage{pagesLTS} package
%              was properly uninstalled from your system.)
%   \end{Version}
%   \begin{Version}{2010/09/27 v1.1l \xpackage{pagesLTS}}
%     \item Bug fix: |\PackageError{pagesLTS}{pagenumbering missing}{\pagesLTS@messageNPN }|
%             had to be moved to the outside of |\EveryShipout|, because it wrote its
%             message into the document instead to the screen and the \xfile{.log}-file.
%     \item Updated to version 2010/04/24 v0.19 of the \xpackage{holtxdoc} package.
%   \end{Version}
%   \begin{Version}{2011/02/01 v1.1m \xpackage{pagesLTS}}
%     \item Added a new warning subsection about \xpackage{hyperref} and repeated page
%             numbers.
%     \item Bug fix: Missing \% after |-\romannumeral\number-\arabic{#1}| added.
%     \item The (then) new version v2.4i of the \xpackage{endfloat} package was then
%             even older than 15~years.
%     \item Put a warning in the documentation as well as in the \xfile{log}-file and
%             at the screen during compilation about the \xpackage{showkeys} package.
%             (The labels of the \xpackage{pagesLTS} package cannot be shown by the
%             \xpackage{showkeys} package.)
%     \item Bug fix: In some situations a rerun warning was given even if no rerun
%             was necessary.
%     \item The recent version of the Adobe Reader \sout{\ is}\uwave{\ was} X (10.0.0)
%             (instead of 9.3.3). Its handling of special page numbers was improved.
%     \item The option \texttt{alphMult}\ \ is now set to \texttt{ab} by default.
%     \item The option \texttt{AlphMulti}\ is now set to \texttt{AB} by default.
%     \item Some details.
%   \end{Version}
%   \begin{Version}{2011/03/16 v1.1n \xpackage{pagesLTS}}
%     \item Bug fix: Handling of option |pagecontinue=false| changed.
%             When |pagecontinue=false| was used, but also a
%             |alphMult|, |AlphMulti|, |fnsymbolmult|, |romanMult| or |RomanMulti| option
%             other than |0| or |false|, respectively, was used, the page numbering
%             \textit{was} continued/extended. Now a warning is issued in case of such
%             option clash and |pagecontinue=false| is heeded, disabling all continuation.
%     \item Bug fix: |\ProvidesPackage{pagesLTS}| contained an older date (2010/09/27 of v1.1l
%             instead of 2011/02/01 of v1.1m).
%     \item Bug fix: The \xpackage{ulem} package is needed to generate the documentation from the
%             \xfile{pagesLTS.dtx} file, but was not listed as necessary package.
%     \item Bug fix: One reference to an outdated version of \xpackage{undolabl} package,
%             replaced by the (then) recent version.
%     \item Some minor details.
%   \end{Version}
%   \begin{Version}{2011/03/17 v1.1o \xpackage{pagesLTS}}
%     \item Documentation and ReadMe bug fix: This \xpackage{pagesLTS} package is located at
%             \url{https://www.ctan.org/pkg/pageslts} instead of \ldots /pagesLTS/.
%     \item There is a new (possible) alternative package, \xpackage{totcount},
%             see section~\ref{sec:Alternatives}.
%     \item Bug fix: There was a reference to \xpackage{lastpage} 1994/07/20, v0.1b,
%             instead of the current version.
%   \end{Version}
%   \begin{Version}{2011/08/08 v1.2a}
%     \item \textbf{Renamed the package from \xpackage{pagesLTS} to \xpackage{pageslts}
%             (keeping family, prefix, internal commands,\ldots{} as |pagesLTS|).}
%             Added checking against double loading as \xpackage{pagesLTS} and \xpackage{pageslts}.
%     \item The \xpackage{holtxdoc} package was fixed (recent: 2011/02/04, v0.21), therefore
%             the warning in \xfile{drv} could be removed.
%     \item |\AtEndAfterFileList| from the \uwave{then} new version of the \xpackage{atveryend} package,
%             2011/04/23, v1.7, by \textsc{Heiko Oberdiek}, is now used for the rerun hint
%             instead of appending to |\@dofilelist|.
%     \item Now defining $2$: |\def\pagesLTS@two{2}| (for example for the
%             \xpackage{thumbs} package) and $3$: |\def\pagesLTS@three{3}|.
%     \item Replaced |\texttt{\textbackslash|\ldots|}| by
%             \textbar \textbackslash \ldots \textbar{} in the \xfile{dtx} and by
%             |\verb|\textbar \textbackslash \ldots \textbar{} in the example (where possible).
%     \item When the \xpackage{alphalph} package is needed, it is loaded via |\RequirePackage|
%             instead of crashing with an error message.
%     \item A lot of details (also in the documentation).
%   \end{Version}
%   \begin{Version}{2013/01/28 v1.2b}
%     \item Updated to \TeX{} live 2012 (for compiling the documentation and example)
%             and installed the available updates. Therefore I can no longer test
%             whether \xpackage{pageslts} works with earlier versions of \LaTeX{}.
%     \item Replaced |\let| by |\LetLtxMacro|.
%     \item The \xpackage{nameref} package redefines |\label| to have five arguments
%             instead of two, therefore |\newlabel{LastPage}{{}{\thepage}{}{}{}}| instead of
%             |\newlabel{LastPage}{{}{\thepage}}| must be used.
%             (Bug reported at \url{http://tex.stackexchange.com/q/95541/6865},
%             thanks to Micha\l{} Herman!) Fixed.
%     \item Updates to a lot of details, also in the documentation.
%   \end{Version}
%   \begin{Version}{2014/01/19 v1.2c}
%     \item Bug: missing loop, fix: inserted.
%     \item Bug: when option |pagecontinue=false| was set, the extension of the
%             page numbering schemes was disabled by mistake, fixed.
%     \item Now using |\ltx@ifpackageloaded| from the \xpackage{ltxcmds} package
%             for checking (even \emph{after} |\AtBeginDocument|)
%             whether a package has been loaded.
%     \item Bug: incompatibility with \xpackage{lineno} because of a mistake
%             in the redefined |\pagenumbering|, fixed. (Bug reported by kwikwi, thanks!)
%     \item Bug: When |\pagenumbering| preceded |\frontmatter| on the same page
%             but with different argument than the |\pagenumbering|, which was inside
%             |\frontmatter|, then some labelling got mixed up.
%             (Also this bug reported by kwikwi, thanks!) While this is not fixed automatically,
%             now an appropriate error message is given. (Two different page numbering schemes
%             on the same page make no sense anyway.)
%     \item Updates of minor details.
%   \end{Version}
%   \begin{Version}{2015/08/02 v1.2d}
%     \item Updated to \TeX{} Live 2015 (for compiling the documentation and example)
%             and installed the available updates. Therefore I can no longer test
%             whether \xpackage{thumbs} works with earlier versions of \LaTeX .
%            (It probably does, but there is no guarantee.)
%     \item A |ifundefinedorrelax| similar to the one from \xpackage{scrbase.sty}
%             of the KOMA script bundle as 2013/12/19 v3.12 is used now,
%             without the need for \eTeX .
%     \item New versions of Adobe Reader and of some packages have become available.
%     \item Changed the message (type) to be displayed if writing to files is disallowed
%             (as pointed out by Joshua Ellis, thanks!).
%     \item Updates to several details, also in the documentation.
%   \end{Version}
%   \begin{Version}{2015/08/17 v1.2e}
%     \item Bug fix, see \url{https://tex.stackexchange.com/q/261445/6865},
%             thank you to \textsc{Prof. Enrico Gregorio} for providing the fix
%             and to \textsc{Dr. Clea F. Rees} for bringing this to my attention.
%   \end{Version}
%   \begin{Version}{2015/12/21 v1.2f}
%     \item Replaced |\next| with |\@pageslts@currname|.
%     \item Bug fix, see \url{https://tex.stackexchange.com/q/140235/6865}.
%             If the bug had been actually reported (instead of waiting until I see
%             the question), I would have fixed it earlier, of course.
%     \item Changed minor details like fixing urls in the manual.
%   \end{Version}
% \end{History}
%
% \bigskip
%
% When you find a mistake or have a suggestion for an improvement of this package,
% please send an e-mail to the maintainer, thanks!\\
% (Please see BUG REPORTS in the README.)
%
% \bigskip
%
% Note: \textsf{Y} is not missing in the following index, but no command beginning
% with this letter has been used in this \xpackage{pageslts} package.
%
% \pagebreak
%
% \PrintIndex
%
% \Finale
\endinput|
% \end{quote}
% Do not forget to quote the argument according to the demands
% of your shell.
%
% \paragraph{Generating the documentation.\label{GenDoc}}
% You can use both the \xfile{.dtx} or the \xfile{.drv} to generate
% the documentation. The process can be configured by the
% configuration file \xfile{ltxdoc.cfg}. For instance, put the following
% line into this file, if you want to have A4 as paper format:
% \begin{quote}
%   \verb|\PassOptionsToClass{a4paper}{article}|
% \end{quote}
%
% \noindent An example follows how to generate the
% documentation with \pdfLaTeX :
%
% \begin{quote}
%\begin{verbatim}
%pdflatex pageslts.dtx
%makeindex -s gind.ist pageslts.idx
%pdflatex pageslts.dtx
%makeindex -s gind.ist pageslts.idx
%pdflatex pageslts.dtx
%\end{verbatim}
% \end{quote}
%
% \subsection{Compiling the example}
%
% The example file, \textsf{pageslts-example.tex}, can be compiled via\\
% \indent |latex pageslts-example.tex|\\
% or (recommended)\\
% \indent |pdflatex pageslts-example.tex|\\
% and will need \textit{at least} (!) three compiler runs to get
% all references right.
%
% \pagebreak
%
% \section{Acknowledgements}
%
% I (H.-Martin M\"{u}nch) would like to thank \textsc{Jeffrey P. Goldberg}
% for inventing the \xpackage{lastpage} package. This package first started
% as a revision of the \xpackage{lastpage} package, but it became obvious
% that a replacement was needed to accomplish what this package does.
% Further I would like to thank \textsc{Heiko Oberdiek} for providing the
% |\erroralphalph| command as well as a~lot~(!) of useful packages
% (from which I also got everything I know about creating a file in
% \xfile{dtx} format, ok, say it: copying), \textsc{Martin Schr\"{o}der}
% for his \xpackage{prelim2e} package, from which I got the
% |Prelim@EveryShipout| code, \textsc{Ulrich Diez} for his code for the
% \xpackage{undolabl} package, which allows overwriting of labels,
% \textsc{Andres L\"{o}h} for the code to determine the current page
% numbering scheme, and the \Newsgroup{comp.text.tex} and
% \Newsgroup{de.comp.text.tex} newsgroups for their help in all things \TeX{}.
% For bug reports I thank Micha\l{} Herman, kwikwi, Joshua Ellis,
% and Dr. Clea F. Rees. For telling me how to fix a bug
% (and for all his shared whisdom at \url{https://tex.stackexchange.com})
% thanks go to Prof. Enrico Gregorio.
%
% \phantomsection
% \begin{History}\label{History}
%
%   Some old versions have been archived at
%   \url{http://ctanhg.scharrer-online.de/pkg/pageslts.html}.
%
%   \begin{Version}{1994/06/17, \xpackage{lastpage}}
%     \item \xpackage{lastpage} v0.99a: First shot by \textsc{Jeffrey P. Goldberg}.
%   \end{Version}
%   \begin{Version}{1994/06/25, \xpackage{lastpage}}
%     \item \xpackage{lastpage} v0.1b: Last version number created by \textsc{Jeffrey P. Goldberg}.
%   \end{Version}
%   \begin{Version}{1994/07/20, \xpackage{lastpage}}
%     \item \xpackage{lastpage} v0.1b (again): Documentation updated by \textsc{Jeffrey P. Goldberg}.\\
%       The main source code of the \xpackage{lastpage} package 1994/07/20 v0.1b was:\\
%       \begin{verbatim}
%        \NeedsTeXFormat{LaTeX2e}[1994/06/01]
%        \ProvidesPackage{lastpage}[1994/07/20 v0.1b
%          LaTeX2e package for refs to last page number (JPG)]
%        \def\lastpage@putlabel{\addtocounter{page}{-1}%
%          \immediate\write\@auxout{\string
%          \newlabel{LastPage}{{}{\thepage}}}%
%          \addtocounter{page}{1}}
%        \AtEndDocument{%
%          \message{AED: lastpage setting LastPage}%
%          \clearpage\lastpage@putlabel}%
%        \endinput
%       \end{verbatim}
%       and then \xpackage{hyperref} and \xpackage{revtex} even redefine |\lastpage@putlabel|.
%   \end{Version}
%   \begin{Version}{2010/02/18, \xpackage{lastpage}}
%     \item \xpackage{lastpage} v1.1: Proposed |LastPages| label by H.-Martin M\"{u}nch
%       on \Newsgroup{comp.text.tex}, see e.\,g.
%       \url{http://groups.google.com/group/comp.text.tex/msg/4407493da9c747f0?dmode=source};
%       now available in this \xpackage{pageslts} package.
%   \end{Version}
%   \begin{Version}{2010/05/15 v1.0 \xpackage{pagesLTS}}
%     \item \xpackage{pagesLTS} Complete rewriting of the package, so as to work with
%       \textbf{more than one page numbering scheme}; using |\AtVeryEnd|
%       for \texttt{VeryLastPage}; upgrade from \xpackage{fancyheadings} to
%       \xpackage{fancyhdr} package, then removed the need for a \xpackage{fancyhdr}
%       package at all.
%     \item Rewriting of the package, so as to work with the \xpackage{fnsymbol}
%       page numbering scheme (even on the last page).
%     \item Introduction of \xpackage{kvoptions} into this package.
%     \item Check for incompatible \xpackage{endfloat} package.
%     \item \xpackage{lastpage209.sty} for \LaTeX209 .
%     \item Replacement of |\filedate|, |-version|, |-name|,\ldots{} because
%       of \LaTeX~bug 2705:\\
%       Synopsis: Possible problem with |\fileversion| and |\filedate|\\
%       \url{http://www.latex-project.org/cgi-bin/ltxbugs2html?category=LaTeX&responsible=anyone&state=anything&keyword=lastpage&pr=latex%2F2705&search=}
%     \item \xpackage{alphalph} support included.
%     \item Page numbering extension |\erroralph| by \textsc{Heiko Oberdiek} included.
%     \item (Page-) Numbering extension for |roman| and |Roman| numbers included.
%     \item Incompatible, old \xpackage{lastpage} package \textquotedblleft killed\textquotedblright .
%     \item Example |pagesLTS-example.tex| added.
%     \item Alternatives listing (section \ref{sec:Alternatives}).
%     \item Listing of \TeX{} sources (subsection \ref{ss:Downloads}).
%     \item Complete rewriting of the documentation.
%     \item Everything in DTX framework.
%     \item New package name: \xpackage{pagesLTS} for Last, Total,
%            and page numbering Schemes pages.
%   \end{Version}
%   \begin{Version}{2010/06/01 v1.1(a) \xpackage{pagesLTS}}
%     \item Abstract changed: Negative \texttt{roman} and \texttt{Roman} page
%            numbers are now possible.
%     \item Some references to other packages have been updated.
%     \item Several typing mistakes have been corrected~-
%            both in the \xfile{sty}le file as well as in this documentation.
%   \end{Version}
%   \begin{Version}{2010/06/03 v1.1b \xpackage{pagesLTS}}
%     \item Corrected a bug in |\XXRoman|, where |\roman| instead of |\Roman| had been used.
%     \item New \xpackage{papermas} package mentioned.
%     \item Several changes in the documentation and the Readme file.
%   \end{Version}
%   \begin{Version}{2010/06/24 v1.1c \xpackage{pagesLTS}}
%     \item \xpackage{holtxdoc} warning in \xfile{drv} updated.
%     \item Removed CRLF line endings from the \xfile{dtx} file.
%     \item Corrected the location of the package at CTAN.
%           (In this version \xfile{TDS} was still missing due to packaging error.)
%     \item Corrected Message format in pagesLTS.\xfile{ins}.
%     \item Updates to the documentation.
%   \end{Version}
%   \begin{Version}{2010/07/15 v1.1d \xpackage{pagesLTS}}
%     \item Added the |\@ifclassloaded{revtex4}| code for |\lastpage@putlabel| (changed to
%            |\pagesLTS@putlabelhyper|) from the \xpackage{hyperref} package \textbf{as comment}~-
%            what is the meaning of that code?
%     \item In the documentation added the explanation of the occurence of multiply definitions
%            of the |LastPage| label with \xpackage{lastpage}, \xpackage{pagesLTS}, \xpackage{hyperref}
%            package (in that order).
%     \item Updated to (then) new version of \xpackage{undolabl} package [2010/07/15] v1.0d,
%            which uses |\undonewlabel| with only one instead of two arguments.
%     \item Added a warning message, if \xpackage{hyperref} \textit{and} \xpackage{pdfpages} are \textit{both} used.
%            (Should not \xpackage{hyperref} give this warning?)
%     \item Updates to the documentation.
%   \end{Version}
%   \begin{Version}{2010/07/29 v1.1e \xpackage{pagesLTS}}
%     \item Removed \xpackage{lastpage209.sty}, because it is now contained
%             in the \xpackage{lastpage}\xfile{.dtx}\ file, v\,$\geq 1.2$a.
%     \item Removed the |\@ifclassloaded{revtex4}| code for |\lastpage@putlabel|.
%     \item Handling of \xpackage{lastpage} package adapted to updated version~1.2(a).
%     \item Corrected error in \xpackage{lastpage} code [1994/07/20 v0.1b] given in \ref{History}~History.
%     \item Version handling for \xpackage{undolabl} package updated.
%     \item Included a |\CheckSum|.
%     \item Some minor details.
%   \end{Version}
%   \begin{Version}{2010/08/08 v1.1f \xpackage{pagesLTS}}
%     \item Version 1.1e had a bug: AlphAlph was replaced by alphalph (because that package is named like this),
%             but this was done also in commands and definitions~- now reverted.
%     \item Updates to the documentation.
%   \end{Version}
%   \begin{Version}{2010/08/12 v1.1g \xpackage{pagesLTS}}
%     \item Now the rerun warning is given \textit{after} e.\,g. the |\listfiles|,
%             increasing the chance of the user to read it (trick found in
%             \textsc{Harald Harders'} \xpackage{fnbreak} package, thanks!).
%   \end{Version}
%   \begin{Version}{2010/08/23 v1.1h \xpackage{pagesLTS}}
%     \item Renamed |\XXRoman| to |\XRoman|.
%     \item Reduced the number of needed counters.
%     \item Removed wrong \% from the driver file.
%     \item Changed the |\unit| definition (got rid of an old |\rm|).
%     \item Without use of the \xpackage{hyperref} package, labels of type
%             \texttt{pagesLTS.}\meta{page numbering scheme}\texttt{.local}
%             became multiply defined. Now an |\undolabl| was inserted.
%     \item Diverse details.
%   \end{Version}
%   \begin{Version}{2010/08/25 v1.1i \xpackage{pagesLTS}}
%     \item Bug fix: \textsf{tcilatex} defines the |\hyperref| command, therefore
%             for \xpackage{hyperref} package detection this had to be changed to
%             |\Hy@Warning|.
%   \end{Version}
%   \begin{Version}{2010/09/12 v1.1j \xpackage{pagesLTS}}
%     \item Bug fix: \LaTeX{} issued a \textquotedblleft \texttt{Label(s) may have changed.%
%             Rerun to get cross-references right.}\textquotedblright -warning,
%             even if labels had not changed but were overwritten.
%     \item Starred version of |\lastpageref| for suppressing hyperlinks introduced.
%     \item A lot of details.
%   \end{Version}
%   \begin{Version}{2010/09/22 v1.1k \xpackage{pagesLTS}}
%     \item When no \texttt{fnsymbol} pagenumbering scheme is used, the respective
%             counters are not defined, saving three counters.
%     \item Moved the package from \texttt{.../latex/muench/pagesLTS/...} to
%             \texttt{.../latex/pagesLTS/...}.\\
%             (Please make sure that the old version of the \xpackage{pagesLTS} package
%              was properly uninstalled from your system.)
%   \end{Version}
%   \begin{Version}{2010/09/27 v1.1l \xpackage{pagesLTS}}
%     \item Bug fix: |\PackageError{pagesLTS}{pagenumbering missing}{\pagesLTS@messageNPN }|
%             had to be moved to the outside of |\EveryShipout|, because it wrote its
%             message into the document instead to the screen and the \xfile{.log}-file.
%     \item Updated to version 2010/04/24 v0.19 of the \xpackage{holtxdoc} package.
%   \end{Version}
%   \begin{Version}{2011/02/01 v1.1m \xpackage{pagesLTS}}
%     \item Added a new warning subsection about \xpackage{hyperref} and repeated page
%             numbers.
%     \item Bug fix: Missing \% after |-\romannumeral\number-\arabic{#1}| added.
%     \item The (then) new version v2.4i of the \xpackage{endfloat} package was then
%             even older than 15~years.
%     \item Put a warning in the documentation as well as in the \xfile{log}-file and
%             at the screen during compilation about the \xpackage{showkeys} package.
%             (The labels of the \xpackage{pagesLTS} package cannot be shown by the
%             \xpackage{showkeys} package.)
%     \item Bug fix: In some situations a rerun warning was given even if no rerun
%             was necessary.
%     \item The recent version of the Adobe Reader \sout{\ is}\uwave{\ was} X (10.0.0)
%             (instead of 9.3.3). Its handling of special page numbers was improved.
%     \item The option \texttt{alphMult}\ \ is now set to \texttt{ab} by default.
%     \item The option \texttt{AlphMulti}\ is now set to \texttt{AB} by default.
%     \item Some details.
%   \end{Version}
%   \begin{Version}{2011/03/16 v1.1n \xpackage{pagesLTS}}
%     \item Bug fix: Handling of option |pagecontinue=false| changed.
%             When |pagecontinue=false| was used, but also a
%             |alphMult|, |AlphMulti|, |fnsymbolmult|, |romanMult| or |RomanMulti| option
%             other than |0| or |false|, respectively, was used, the page numbering
%             \textit{was} continued/extended. Now a warning is issued in case of such
%             option clash and |pagecontinue=false| is heeded, disabling all continuation.
%     \item Bug fix: |\ProvidesPackage{pagesLTS}| contained an older date (2010/09/27 of v1.1l
%             instead of 2011/02/01 of v1.1m).
%     \item Bug fix: The \xpackage{ulem} package is needed to generate the documentation from the
%             \xfile{pagesLTS.dtx} file, but was not listed as necessary package.
%     \item Bug fix: One reference to an outdated version of \xpackage{undolabl} package,
%             replaced by the (then) recent version.
%     \item Some minor details.
%   \end{Version}
%   \begin{Version}{2011/03/17 v1.1o \xpackage{pagesLTS}}
%     \item Documentation and ReadMe bug fix: This \xpackage{pagesLTS} package is located at
%             \url{https://www.ctan.org/pkg/pageslts} instead of \ldots /pagesLTS/.
%     \item There is a new (possible) alternative package, \xpackage{totcount},
%             see section~\ref{sec:Alternatives}.
%     \item Bug fix: There was a reference to \xpackage{lastpage} 1994/07/20, v0.1b,
%             instead of the current version.
%   \end{Version}
%   \begin{Version}{2011/08/08 v1.2a}
%     \item \textbf{Renamed the package from \xpackage{pagesLTS} to \xpackage{pageslts}
%             (keeping family, prefix, internal commands,\ldots{} as |pagesLTS|).}
%             Added checking against double loading as \xpackage{pagesLTS} and \xpackage{pageslts}.
%     \item The \xpackage{holtxdoc} package was fixed (recent: 2011/02/04, v0.21), therefore
%             the warning in \xfile{drv} could be removed.
%     \item |\AtEndAfterFileList| from the \uwave{then} new version of the \xpackage{atveryend} package,
%             2011/04/23, v1.7, by \textsc{Heiko Oberdiek}, is now used for the rerun hint
%             instead of appending to |\@dofilelist|.
%     \item Now defining $2$: |\def\pagesLTS@two{2}| (for example for the
%             \xpackage{thumbs} package) and $3$: |\def\pagesLTS@three{3}|.
%     \item Replaced |\texttt{\textbackslash|\ldots|}| by
%             \textbar \textbackslash \ldots \textbar{} in the \xfile{dtx} and by
%             |\verb|\textbar \textbackslash \ldots \textbar{} in the example (where possible).
%     \item When the \xpackage{alphalph} package is needed, it is loaded via |\RequirePackage|
%             instead of crashing with an error message.
%     \item A lot of details (also in the documentation).
%   \end{Version}
%   \begin{Version}{2013/01/28 v1.2b}
%     \item Updated to \TeX{} live 2012 (for compiling the documentation and example)
%             and installed the available updates. Therefore I can no longer test
%             whether \xpackage{pageslts} works with earlier versions of \LaTeX{}.
%     \item Replaced |\let| by |\LetLtxMacro|.
%     \item The \xpackage{nameref} package redefines |\label| to have five arguments
%             instead of two, therefore |\newlabel{LastPage}{{}{\thepage}{}{}{}}| instead of
%             |\newlabel{LastPage}{{}{\thepage}}| must be used.
%             (Bug reported at \url{http://tex.stackexchange.com/q/95541/6865},
%             thanks to Micha\l{} Herman!) Fixed.
%     \item Updates to a lot of details, also in the documentation.
%   \end{Version}
%   \begin{Version}{2014/01/19 v1.2c}
%     \item Bug: missing loop, fix: inserted.
%     \item Bug: when option |pagecontinue=false| was set, the extension of the
%             page numbering schemes was disabled by mistake, fixed.
%     \item Now using |\ltx@ifpackageloaded| from the \xpackage{ltxcmds} package
%             for checking (even \emph{after} |\AtBeginDocument|)
%             whether a package has been loaded.
%     \item Bug: incompatibility with \xpackage{lineno} because of a mistake
%             in the redefined |\pagenumbering|, fixed. (Bug reported by kwikwi, thanks!)
%     \item Bug: When |\pagenumbering| preceded |\frontmatter| on the same page
%             but with different argument than the |\pagenumbering|, which was inside
%             |\frontmatter|, then some labelling got mixed up.
%             (Also this bug reported by kwikwi, thanks!) While this is not fixed automatically,
%             now an appropriate error message is given. (Two different page numbering schemes
%             on the same page make no sense anyway.)
%     \item Updates of minor details.
%   \end{Version}
%   \begin{Version}{2015/08/02 v1.2d}
%     \item Updated to \TeX{} Live 2015 (for compiling the documentation and example)
%             and installed the available updates. Therefore I can no longer test
%             whether \xpackage{thumbs} works with earlier versions of \LaTeX .
%            (It probably does, but there is no guarantee.)
%     \item A |ifundefinedorrelax| similar to the one from \xpackage{scrbase.sty}
%             of the KOMA script bundle as 2013/12/19 v3.12 is used now,
%             without the need for \eTeX .
%     \item New versions of Adobe Reader and of some packages have become available.
%     \item Changed the message (type) to be displayed if writing to files is disallowed
%             (as pointed out by Joshua Ellis, thanks!).
%     \item Updates to several details, also in the documentation.
%   \end{Version}
%   \begin{Version}{2015/08/17 v1.2e}
%     \item Bug fix, see \url{https://tex.stackexchange.com/q/261445/6865},
%             thank you to \textsc{Prof. Enrico Gregorio} for providing the fix
%             and to \textsc{Dr. Clea F. Rees} for bringing this to my attention.
%   \end{Version}
%   \begin{Version}{2015/12/21 v1.2f}
%     \item Replaced |\next| with |\@pageslts@currname|.
%     \item Bug fix, see \url{https://tex.stackexchange.com/q/140235/6865}.
%             If the bug had been actually reported (instead of waiting until I see
%             the question), I would have fixed it earlier, of course.
%     \item Changed minor details like fixing urls in the manual.
%   \end{Version}
% \end{History}
%
% \bigskip
%
% When you find a mistake or have a suggestion for an improvement of this package,
% please send an e-mail to the maintainer, thanks!\\
% (Please see BUG REPORTS in the README.)
%
% \bigskip
%
% Note: \textsf{Y} is not missing in the following index, but no command beginning
% with this letter has been used in this \xpackage{pageslts} package.
%
% \pagebreak
%
% \PrintIndex
%
% \Finale
\endinput|
% \end{quote}
% Do not forget to quote the argument according to the demands
% of your shell.
%
% \paragraph{Generating the documentation.\label{GenDoc}}
% You can use both the \xfile{.dtx} or the \xfile{.drv} to generate
% the documentation. The process can be configured by the
% configuration file \xfile{ltxdoc.cfg}. For instance, put the following
% line into this file, if you want to have A4 as paper format:
% \begin{quote}
%   \verb|\PassOptionsToClass{a4paper}{article}|
% \end{quote}
%
% \noindent An example follows how to generate the
% documentation with \pdfLaTeX :
%
% \begin{quote}
%\begin{verbatim}
%pdflatex pageslts.dtx
%makeindex -s gind.ist pageslts.idx
%pdflatex pageslts.dtx
%makeindex -s gind.ist pageslts.idx
%pdflatex pageslts.dtx
%\end{verbatim}
% \end{quote}
%
% \subsection{Compiling the example}
%
% The example file, \textsf{pageslts-example.tex}, can be compiled via\\
% \indent |latex pageslts-example.tex|\\
% or (recommended)\\
% \indent |pdflatex pageslts-example.tex|\\
% and will need \textit{at least} (!) three compiler runs to get
% all references right.
%
% \pagebreak
%
% \section{Acknowledgements}
%
% I (H.-Martin M\"{u}nch) would like to thank \textsc{Jeffrey P. Goldberg}
% for inventing the \xpackage{lastpage} package. This package first started
% as a revision of the \xpackage{lastpage} package, but it became obvious
% that a replacement was needed to accomplish what this package does.
% Further I would like to thank \textsc{Heiko Oberdiek} for providing the
% |\erroralphalph| command as well as a~lot~(!) of useful packages
% (from which I also got everything I know about creating a file in
% \xfile{dtx} format, ok, say it: copying), \textsc{Martin Schr\"{o}der}
% for his \xpackage{prelim2e} package, from which I got the
% |Prelim@EveryShipout| code, \textsc{Ulrich Diez} for his code for the
% \xpackage{undolabl} package, which allows overwriting of labels,
% \textsc{Andres L\"{o}h} for the code to determine the current page
% numbering scheme, and the \Newsgroup{comp.text.tex} and
% \Newsgroup{de.comp.text.tex} newsgroups for their help in all things \TeX{}.
% For bug reports I thank Micha\l{} Herman, kwikwi, Joshua Ellis,
% and Dr. Clea F. Rees. For telling me how to fix a bug
% (and for all his shared whisdom at \url{https://tex.stackexchange.com})
% thanks go to Prof. Enrico Gregorio.
%
% \phantomsection
% \begin{History}\label{History}
%
%   Some old versions have been archived at
%   \url{http://ctanhg.scharrer-online.de/pkg/pageslts.html}.
%
%   \begin{Version}{1994/06/17, \xpackage{lastpage}}
%     \item \xpackage{lastpage} v0.99a: First shot by \textsc{Jeffrey P. Goldberg}.
%   \end{Version}
%   \begin{Version}{1994/06/25, \xpackage{lastpage}}
%     \item \xpackage{lastpage} v0.1b: Last version number created by \textsc{Jeffrey P. Goldberg}.
%   \end{Version}
%   \begin{Version}{1994/07/20, \xpackage{lastpage}}
%     \item \xpackage{lastpage} v0.1b (again): Documentation updated by \textsc{Jeffrey P. Goldberg}.\\
%       The main source code of the \xpackage{lastpage} package 1994/07/20 v0.1b was:\\
%       \begin{verbatim}
%        \NeedsTeXFormat{LaTeX2e}[1994/06/01]
%        \ProvidesPackage{lastpage}[1994/07/20 v0.1b
%          LaTeX2e package for refs to last page number (JPG)]
%        \def\lastpage@putlabel{\addtocounter{page}{-1}%
%          \immediate\write\@auxout{\string
%          \newlabel{LastPage}{{}{\thepage}}}%
%          \addtocounter{page}{1}}
%        \AtEndDocument{%
%          \message{AED: lastpage setting LastPage}%
%          \clearpage\lastpage@putlabel}%
%        \endinput
%       \end{verbatim}
%       and then \xpackage{hyperref} and \xpackage{revtex} even redefine |\lastpage@putlabel|.
%   \end{Version}
%   \begin{Version}{2010/02/18, \xpackage{lastpage}}
%     \item \xpackage{lastpage} v1.1: Proposed |LastPages| label by H.-Martin M\"{u}nch
%       on \Newsgroup{comp.text.tex}, see e.\,g.
%       \url{http://groups.google.com/group/comp.text.tex/msg/4407493da9c747f0?dmode=source};
%       now available in this \xpackage{pageslts} package.
%   \end{Version}
%   \begin{Version}{2010/05/15 v1.0 \xpackage{pagesLTS}}
%     \item \xpackage{pagesLTS} Complete rewriting of the package, so as to work with
%       \textbf{more than one page numbering scheme}; using |\AtVeryEnd|
%       for \texttt{VeryLastPage}; upgrade from \xpackage{fancyheadings} to
%       \xpackage{fancyhdr} package, then removed the need for a \xpackage{fancyhdr}
%       package at all.
%     \item Rewriting of the package, so as to work with the \xpackage{fnsymbol}
%       page numbering scheme (even on the last page).
%     \item Introduction of \xpackage{kvoptions} into this package.
%     \item Check for incompatible \xpackage{endfloat} package.
%     \item \xpackage{lastpage209.sty} for \LaTeX209 .
%     \item Replacement of |\filedate|, |-version|, |-name|,\ldots{} because
%       of \LaTeX~bug 2705:\\
%       Synopsis: Possible problem with |\fileversion| and |\filedate|\\
%       \url{http://www.latex-project.org/cgi-bin/ltxbugs2html?category=LaTeX&responsible=anyone&state=anything&keyword=lastpage&pr=latex%2F2705&search=}
%     \item \xpackage{alphalph} support included.
%     \item Page numbering extension |\erroralph| by \textsc{Heiko Oberdiek} included.
%     \item (Page-) Numbering extension for |roman| and |Roman| numbers included.
%     \item Incompatible, old \xpackage{lastpage} package \textquotedblleft killed\textquotedblright .
%     \item Example |pagesLTS-example.tex| added.
%     \item Alternatives listing (section \ref{sec:Alternatives}).
%     \item Listing of \TeX{} sources (subsection \ref{ss:Downloads}).
%     \item Complete rewriting of the documentation.
%     \item Everything in DTX framework.
%     \item New package name: \xpackage{pagesLTS} for Last, Total,
%            and page numbering Schemes pages.
%   \end{Version}
%   \begin{Version}{2010/06/01 v1.1(a) \xpackage{pagesLTS}}
%     \item Abstract changed: Negative \texttt{roman} and \texttt{Roman} page
%            numbers are now possible.
%     \item Some references to other packages have been updated.
%     \item Several typing mistakes have been corrected~-
%            both in the \xfile{sty}le file as well as in this documentation.
%   \end{Version}
%   \begin{Version}{2010/06/03 v1.1b \xpackage{pagesLTS}}
%     \item Corrected a bug in |\XXRoman|, where |\roman| instead of |\Roman| had been used.
%     \item New \xpackage{papermas} package mentioned.
%     \item Several changes in the documentation and the Readme file.
%   \end{Version}
%   \begin{Version}{2010/06/24 v1.1c \xpackage{pagesLTS}}
%     \item \xpackage{holtxdoc} warning in \xfile{drv} updated.
%     \item Removed CRLF line endings from the \xfile{dtx} file.
%     \item Corrected the location of the package at CTAN.
%           (In this version \xfile{TDS} was still missing due to packaging error.)
%     \item Corrected Message format in pagesLTS.\xfile{ins}.
%     \item Updates to the documentation.
%   \end{Version}
%   \begin{Version}{2010/07/15 v1.1d \xpackage{pagesLTS}}
%     \item Added the |\@ifclassloaded{revtex4}| code for |\lastpage@putlabel| (changed to
%            |\pagesLTS@putlabelhyper|) from the \xpackage{hyperref} package \textbf{as comment}~-
%            what is the meaning of that code?
%     \item In the documentation added the explanation of the occurence of multiply definitions
%            of the |LastPage| label with \xpackage{lastpage}, \xpackage{pagesLTS}, \xpackage{hyperref}
%            package (in that order).
%     \item Updated to (then) new version of \xpackage{undolabl} package [2010/07/15] v1.0d,
%            which uses |\undonewlabel| with only one instead of two arguments.
%     \item Added a warning message, if \xpackage{hyperref} \textit{and} \xpackage{pdfpages} are \textit{both} used.
%            (Should not \xpackage{hyperref} give this warning?)
%     \item Updates to the documentation.
%   \end{Version}
%   \begin{Version}{2010/07/29 v1.1e \xpackage{pagesLTS}}
%     \item Removed \xpackage{lastpage209.sty}, because it is now contained
%             in the \xpackage{lastpage}\xfile{.dtx}\ file, v\,$\geq 1.2$a.
%     \item Removed the |\@ifclassloaded{revtex4}| code for |\lastpage@putlabel|.
%     \item Handling of \xpackage{lastpage} package adapted to updated version~1.2(a).
%     \item Corrected error in \xpackage{lastpage} code [1994/07/20 v0.1b] given in \ref{History}~History.
%     \item Version handling for \xpackage{undolabl} package updated.
%     \item Included a |\CheckSum|.
%     \item Some minor details.
%   \end{Version}
%   \begin{Version}{2010/08/08 v1.1f \xpackage{pagesLTS}}
%     \item Version 1.1e had a bug: AlphAlph was replaced by alphalph (because that package is named like this),
%             but this was done also in commands and definitions~- now reverted.
%     \item Updates to the documentation.
%   \end{Version}
%   \begin{Version}{2010/08/12 v1.1g \xpackage{pagesLTS}}
%     \item Now the rerun warning is given \textit{after} e.\,g. the |\listfiles|,
%             increasing the chance of the user to read it (trick found in
%             \textsc{Harald Harders'} \xpackage{fnbreak} package, thanks!).
%   \end{Version}
%   \begin{Version}{2010/08/23 v1.1h \xpackage{pagesLTS}}
%     \item Renamed |\XXRoman| to |\XRoman|.
%     \item Reduced the number of needed counters.
%     \item Removed wrong \% from the driver file.
%     \item Changed the |\unit| definition (got rid of an old |\rm|).
%     \item Without use of the \xpackage{hyperref} package, labels of type
%             \texttt{pagesLTS.}\meta{page numbering scheme}\texttt{.local}
%             became multiply defined. Now an |\undolabl| was inserted.
%     \item Diverse details.
%   \end{Version}
%   \begin{Version}{2010/08/25 v1.1i \xpackage{pagesLTS}}
%     \item Bug fix: \textsf{tcilatex} defines the |\hyperref| command, therefore
%             for \xpackage{hyperref} package detection this had to be changed to
%             |\Hy@Warning|.
%   \end{Version}
%   \begin{Version}{2010/09/12 v1.1j \xpackage{pagesLTS}}
%     \item Bug fix: \LaTeX{} issued a \textquotedblleft \texttt{Label(s) may have changed.%
%             Rerun to get cross-references right.}\textquotedblright -warning,
%             even if labels had not changed but were overwritten.
%     \item Starred version of |\lastpageref| for suppressing hyperlinks introduced.
%     \item A lot of details.
%   \end{Version}
%   \begin{Version}{2010/09/22 v1.1k \xpackage{pagesLTS}}
%     \item When no \texttt{fnsymbol} pagenumbering scheme is used, the respective
%             counters are not defined, saving three counters.
%     \item Moved the package from \texttt{.../latex/muench/pagesLTS/...} to
%             \texttt{.../latex/pagesLTS/...}.\\
%             (Please make sure that the old version of the \xpackage{pagesLTS} package
%              was properly uninstalled from your system.)
%   \end{Version}
%   \begin{Version}{2010/09/27 v1.1l \xpackage{pagesLTS}}
%     \item Bug fix: |\PackageError{pagesLTS}{pagenumbering missing}{\pagesLTS@messageNPN }|
%             had to be moved to the outside of |\EveryShipout|, because it wrote its
%             message into the document instead to the screen and the \xfile{.log}-file.
%     \item Updated to version 2010/04/24 v0.19 of the \xpackage{holtxdoc} package.
%   \end{Version}
%   \begin{Version}{2011/02/01 v1.1m \xpackage{pagesLTS}}
%     \item Added a new warning subsection about \xpackage{hyperref} and repeated page
%             numbers.
%     \item Bug fix: Missing \% after |-\romannumeral\number-\arabic{#1}| added.
%     \item The (then) new version v2.4i of the \xpackage{endfloat} package was then
%             even older than 15~years.
%     \item Put a warning in the documentation as well as in the \xfile{log}-file and
%             at the screen during compilation about the \xpackage{showkeys} package.
%             (The labels of the \xpackage{pagesLTS} package cannot be shown by the
%             \xpackage{showkeys} package.)
%     \item Bug fix: In some situations a rerun warning was given even if no rerun
%             was necessary.
%     \item The recent version of the Adobe Reader \sout{\ is}\uwave{\ was} X (10.0.0)
%             (instead of 9.3.3). Its handling of special page numbers was improved.
%     \item The option \texttt{alphMult}\ \ is now set to \texttt{ab} by default.
%     \item The option \texttt{AlphMulti}\ is now set to \texttt{AB} by default.
%     \item Some details.
%   \end{Version}
%   \begin{Version}{2011/03/16 v1.1n \xpackage{pagesLTS}}
%     \item Bug fix: Handling of option |pagecontinue=false| changed.
%             When |pagecontinue=false| was used, but also a
%             |alphMult|, |AlphMulti|, |fnsymbolmult|, |romanMult| or |RomanMulti| option
%             other than |0| or |false|, respectively, was used, the page numbering
%             \textit{was} continued/extended. Now a warning is issued in case of such
%             option clash and |pagecontinue=false| is heeded, disabling all continuation.
%     \item Bug fix: |\ProvidesPackage{pagesLTS}| contained an older date (2010/09/27 of v1.1l
%             instead of 2011/02/01 of v1.1m).
%     \item Bug fix: The \xpackage{ulem} package is needed to generate the documentation from the
%             \xfile{pagesLTS.dtx} file, but was not listed as necessary package.
%     \item Bug fix: One reference to an outdated version of \xpackage{undolabl} package,
%             replaced by the (then) recent version.
%     \item Some minor details.
%   \end{Version}
%   \begin{Version}{2011/03/17 v1.1o \xpackage{pagesLTS}}
%     \item Documentation and ReadMe bug fix: This \xpackage{pagesLTS} package is located at
%             \url{https://www.ctan.org/pkg/pageslts} instead of \ldots /pagesLTS/.
%     \item There is a new (possible) alternative package, \xpackage{totcount},
%             see section~\ref{sec:Alternatives}.
%     \item Bug fix: There was a reference to \xpackage{lastpage} 1994/07/20, v0.1b,
%             instead of the current version.
%   \end{Version}
%   \begin{Version}{2011/08/08 v1.2a}
%     \item \textbf{Renamed the package from \xpackage{pagesLTS} to \xpackage{pageslts}
%             (keeping family, prefix, internal commands,\ldots{} as |pagesLTS|).}
%             Added checking against double loading as \xpackage{pagesLTS} and \xpackage{pageslts}.
%     \item The \xpackage{holtxdoc} package was fixed (recent: 2011/02/04, v0.21), therefore
%             the warning in \xfile{drv} could be removed.
%     \item |\AtEndAfterFileList| from the \uwave{then} new version of the \xpackage{atveryend} package,
%             2011/04/23, v1.7, by \textsc{Heiko Oberdiek}, is now used for the rerun hint
%             instead of appending to |\@dofilelist|.
%     \item Now defining $2$: |\def\pagesLTS@two{2}| (for example for the
%             \xpackage{thumbs} package) and $3$: |\def\pagesLTS@three{3}|.
%     \item Replaced |\texttt{\textbackslash|\ldots|}| by
%             \textbar \textbackslash \ldots \textbar{} in the \xfile{dtx} and by
%             |\verb|\textbar \textbackslash \ldots \textbar{} in the example (where possible).
%     \item When the \xpackage{alphalph} package is needed, it is loaded via |\RequirePackage|
%             instead of crashing with an error message.
%     \item A lot of details (also in the documentation).
%   \end{Version}
%   \begin{Version}{2013/01/28 v1.2b}
%     \item Updated to \TeX{} live 2012 (for compiling the documentation and example)
%             and installed the available updates. Therefore I can no longer test
%             whether \xpackage{pageslts} works with earlier versions of \LaTeX{}.
%     \item Replaced |\let| by |\LetLtxMacro|.
%     \item The \xpackage{nameref} package redefines |\label| to have five arguments
%             instead of two, therefore |\newlabel{LastPage}{{}{\thepage}{}{}{}}| instead of
%             |\newlabel{LastPage}{{}{\thepage}}| must be used.
%             (Bug reported at \url{http://tex.stackexchange.com/q/95541/6865},
%             thanks to Micha\l{} Herman!) Fixed.
%     \item Updates to a lot of details, also in the documentation.
%   \end{Version}
%   \begin{Version}{2014/01/19 v1.2c}
%     \item Bug: missing loop, fix: inserted.
%     \item Bug: when option |pagecontinue=false| was set, the extension of the
%             page numbering schemes was disabled by mistake, fixed.
%     \item Now using |\ltx@ifpackageloaded| from the \xpackage{ltxcmds} package
%             for checking (even \emph{after} |\AtBeginDocument|)
%             whether a package has been loaded.
%     \item Bug: incompatibility with \xpackage{lineno} because of a mistake
%             in the redefined |\pagenumbering|, fixed. (Bug reported by kwikwi, thanks!)
%     \item Bug: When |\pagenumbering| preceded |\frontmatter| on the same page
%             but with different argument than the |\pagenumbering|, which was inside
%             |\frontmatter|, then some labelling got mixed up.
%             (Also this bug reported by kwikwi, thanks!) While this is not fixed automatically,
%             now an appropriate error message is given. (Two different page numbering schemes
%             on the same page make no sense anyway.)
%     \item Updates of minor details.
%   \end{Version}
%   \begin{Version}{2015/08/02 v1.2d}
%     \item Updated to \TeX{} Live 2015 (for compiling the documentation and example)
%             and installed the available updates. Therefore I can no longer test
%             whether \xpackage{thumbs} works with earlier versions of \LaTeX .
%            (It probably does, but there is no guarantee.)
%     \item A |ifundefinedorrelax| similar to the one from \xpackage{scrbase.sty}
%             of the KOMA script bundle as 2013/12/19 v3.12 is used now,
%             without the need for \eTeX .
%     \item New versions of Adobe Reader and of some packages have become available.
%     \item Changed the message (type) to be displayed if writing to files is disallowed
%             (as pointed out by Joshua Ellis, thanks!).
%     \item Updates to several details, also in the documentation.
%   \end{Version}
%   \begin{Version}{2015/08/17 v1.2e}
%     \item Bug fix, see \url{https://tex.stackexchange.com/q/261445/6865},
%             thank you to \textsc{Prof. Enrico Gregorio} for providing the fix
%             and to \textsc{Dr. Clea F. Rees} for bringing this to my attention.
%   \end{Version}
%   \begin{Version}{2015/12/21 v1.2f}
%     \item Replaced |\next| with |\@pageslts@currname|.
%     \item Bug fix, see \url{https://tex.stackexchange.com/q/140235/6865}.
%             If the bug had been actually reported (instead of waiting until I see
%             the question), I would have fixed it earlier, of course.
%     \item Changed minor details like fixing urls in the manual.
%   \end{Version}
% \end{History}
%
% \bigskip
%
% When you find a mistake or have a suggestion for an improvement of this package,
% please send an e-mail to the maintainer, thanks!\\
% (Please see BUG REPORTS in the README.)
%
% \bigskip
%
% Note: \textsf{Y} is not missing in the following index, but no command beginning
% with this letter has been used in this \xpackage{pageslts} package.
%
% \pagebreak
%
% \PrintIndex
%
% \Finale
\endinput|
% \end{quote}
% Do not forget to quote the argument according to the demands
% of your shell.
%
% \paragraph{Generating the documentation.\label{GenDoc}}
% You can use both the \xfile{.dtx} or the \xfile{.drv} to generate
% the documentation. The process can be configured by the
% configuration file \xfile{ltxdoc.cfg}. For instance, put the following
% line into this file, if you want to have A4 as paper format:
% \begin{quote}
%   \verb|\PassOptionsToClass{a4paper}{article}|
% \end{quote}
%
% \noindent An example follows how to generate the
% documentation with \pdfLaTeX :
%
% \begin{quote}
%\begin{verbatim}
%pdflatex pageslts.dtx
%makeindex -s gind.ist pageslts.idx
%pdflatex pageslts.dtx
%makeindex -s gind.ist pageslts.idx
%pdflatex pageslts.dtx
%\end{verbatim}
% \end{quote}
%
% \subsection{Compiling the example}
%
% The example file, \textsf{pageslts-example.tex}, can be compiled via\\
% \indent |latex pageslts-example.tex|\\
% or (recommended)\\
% \indent |pdflatex pageslts-example.tex|\\
% and will need \textit{at least} (!) three compiler runs to get
% all references right.
%
% \pagebreak
%
% \section{Acknowledgements}
%
% I (H.-Martin M\"{u}nch) would like to thank \textsc{Jeffrey P. Goldberg}
% for inventing the \xpackage{lastpage} package. This package first started
% as a revision of the \xpackage{lastpage} package, but it became obvious
% that a replacement was needed to accomplish what this package does.
% Further I would like to thank \textsc{Heiko Oberdiek} for providing the
% |\erroralphalph| command as well as a~lot~(!) of useful packages
% (from which I also got everything I know about creating a file in
% \xfile{dtx} format, ok, say it: copying), \textsc{Martin Schr\"{o}der}
% for his \xpackage{prelim2e} package, from which I got the
% |Prelim@EveryShipout| code, \textsc{Ulrich Diez} for his code for the
% \xpackage{undolabl} package, which allows overwriting of labels,
% \textsc{Andres L\"{o}h} for the code to determine the current page
% numbering scheme, and the \Newsgroup{comp.text.tex} and
% \Newsgroup{de.comp.text.tex} newsgroups for their help in all things \TeX{}.
% For bug reports I thank Micha\l{} Herman, kwikwi, Joshua Ellis,
% and Dr. Clea F. Rees. For telling me how to fix a bug
% (and for all his shared whisdom at \url{https://tex.stackexchange.com})
% thanks go to Prof. Enrico Gregorio.
%
% \phantomsection
% \begin{History}\label{History}
%
%   Some old versions have been archived at
%   \url{http://ctanhg.scharrer-online.de/pkg/pageslts.html}.
%
%   \begin{Version}{1994/06/17, \xpackage{lastpage}}
%     \item \xpackage{lastpage} v0.99a: First shot by \textsc{Jeffrey P. Goldberg}.
%   \end{Version}
%   \begin{Version}{1994/06/25, \xpackage{lastpage}}
%     \item \xpackage{lastpage} v0.1b: Last version number created by \textsc{Jeffrey P. Goldberg}.
%   \end{Version}
%   \begin{Version}{1994/07/20, \xpackage{lastpage}}
%     \item \xpackage{lastpage} v0.1b (again): Documentation updated by \textsc{Jeffrey P. Goldberg}.\\
%       The main source code of the \xpackage{lastpage} package 1994/07/20 v0.1b was:\\
%       \begin{verbatim}
%        \NeedsTeXFormat{LaTeX2e}[1994/06/01]
%        \ProvidesPackage{lastpage}[1994/07/20 v0.1b
%          LaTeX2e package for refs to last page number (JPG)]
%        \def\lastpage@putlabel{\addtocounter{page}{-1}%
%          \immediate\write\@auxout{\string
%          \newlabel{LastPage}{{}{\thepage}}}%
%          \addtocounter{page}{1}}
%        \AtEndDocument{%
%          \message{AED: lastpage setting LastPage}%
%          \clearpage\lastpage@putlabel}%
%        \endinput
%       \end{verbatim}
%       and then \xpackage{hyperref} and \xpackage{revtex} even redefine |\lastpage@putlabel|.
%   \end{Version}
%   \begin{Version}{2010/02/18, \xpackage{lastpage}}
%     \item \xpackage{lastpage} v1.1: Proposed |LastPages| label by H.-Martin M\"{u}nch
%       on \Newsgroup{comp.text.tex}, see e.\,g.
%       \url{http://groups.google.com/group/comp.text.tex/msg/4407493da9c747f0?dmode=source};
%       now available in this \xpackage{pageslts} package.
%   \end{Version}
%   \begin{Version}{2010/05/15 v1.0 \xpackage{pagesLTS}}
%     \item \xpackage{pagesLTS} Complete rewriting of the package, so as to work with
%       \textbf{more than one page numbering scheme}; using |\AtVeryEnd|
%       for \texttt{VeryLastPage}; upgrade from \xpackage{fancyheadings} to
%       \xpackage{fancyhdr} package, then removed the need for a \xpackage{fancyhdr}
%       package at all.
%     \item Rewriting of the package, so as to work with the \xpackage{fnsymbol}
%       page numbering scheme (even on the last page).
%     \item Introduction of \xpackage{kvoptions} into this package.
%     \item Check for incompatible \xpackage{endfloat} package.
%     \item \xpackage{lastpage209.sty} for \LaTeX209 .
%     \item Replacement of |\filedate|, |-version|, |-name|,\ldots{} because
%       of \LaTeX~bug 2705:\\
%       Synopsis: Possible problem with |\fileversion| and |\filedate|\\
%       \url{http://www.latex-project.org/cgi-bin/ltxbugs2html?category=LaTeX&responsible=anyone&state=anything&keyword=lastpage&pr=latex%2F2705&search=}
%     \item \xpackage{alphalph} support included.
%     \item Page numbering extension |\erroralph| by \textsc{Heiko Oberdiek} included.
%     \item (Page-) Numbering extension for |roman| and |Roman| numbers included.
%     \item Incompatible, old \xpackage{lastpage} package \textquotedblleft killed\textquotedblright .
%     \item Example |pagesLTS-example.tex| added.
%     \item Alternatives listing (section \ref{sec:Alternatives}).
%     \item Listing of \TeX{} sources (subsection \ref{ss:Downloads}).
%     \item Complete rewriting of the documentation.
%     \item Everything in DTX framework.
%     \item New package name: \xpackage{pagesLTS} for Last, Total,
%            and page numbering Schemes pages.
%   \end{Version}
%   \begin{Version}{2010/06/01 v1.1(a) \xpackage{pagesLTS}}
%     \item Abstract changed: Negative \texttt{roman} and \texttt{Roman} page
%            numbers are now possible.
%     \item Some references to other packages have been updated.
%     \item Several typing mistakes have been corrected~-
%            both in the \xfile{sty}le file as well as in this documentation.
%   \end{Version}
%   \begin{Version}{2010/06/03 v1.1b \xpackage{pagesLTS}}
%     \item Corrected a bug in |\XXRoman|, where |\roman| instead of |\Roman| had been used.
%     \item New \xpackage{papermas} package mentioned.
%     \item Several changes in the documentation and the Readme file.
%   \end{Version}
%   \begin{Version}{2010/06/24 v1.1c \xpackage{pagesLTS}}
%     \item \xpackage{holtxdoc} warning in \xfile{drv} updated.
%     \item Removed CRLF line endings from the \xfile{dtx} file.
%     \item Corrected the location of the package at CTAN.
%           (In this version \xfile{TDS} was still missing due to packaging error.)
%     \item Corrected Message format in pagesLTS.\xfile{ins}.
%     \item Updates to the documentation.
%   \end{Version}
%   \begin{Version}{2010/07/15 v1.1d \xpackage{pagesLTS}}
%     \item Added the |\@ifclassloaded{revtex4}| code for |\lastpage@putlabel| (changed to
%            |\pagesLTS@putlabelhyper|) from the \xpackage{hyperref} package \textbf{as comment}~-
%            what is the meaning of that code?
%     \item In the documentation added the explanation of the occurence of multiply definitions
%            of the |LastPage| label with \xpackage{lastpage}, \xpackage{pagesLTS}, \xpackage{hyperref}
%            package (in that order).
%     \item Updated to (then) new version of \xpackage{undolabl} package [2010/07/15] v1.0d,
%            which uses |\undonewlabel| with only one instead of two arguments.
%     \item Added a warning message, if \xpackage{hyperref} \textit{and} \xpackage{pdfpages} are \textit{both} used.
%            (Should not \xpackage{hyperref} give this warning?)
%     \item Updates to the documentation.
%   \end{Version}
%   \begin{Version}{2010/07/29 v1.1e \xpackage{pagesLTS}}
%     \item Removed \xpackage{lastpage209.sty}, because it is now contained
%             in the \xpackage{lastpage}\xfile{.dtx}\ file, v\,$\geq 1.2$a.
%     \item Removed the |\@ifclassloaded{revtex4}| code for |\lastpage@putlabel|.
%     \item Handling of \xpackage{lastpage} package adapted to updated version~1.2(a).
%     \item Corrected error in \xpackage{lastpage} code [1994/07/20 v0.1b] given in \ref{History}~History.
%     \item Version handling for \xpackage{undolabl} package updated.
%     \item Included a |\CheckSum|.
%     \item Some minor details.
%   \end{Version}
%   \begin{Version}{2010/08/08 v1.1f \xpackage{pagesLTS}}
%     \item Version 1.1e had a bug: AlphAlph was replaced by alphalph (because that package is named like this),
%             but this was done also in commands and definitions~- now reverted.
%     \item Updates to the documentation.
%   \end{Version}
%   \begin{Version}{2010/08/12 v1.1g \xpackage{pagesLTS}}
%     \item Now the rerun warning is given \textit{after} e.\,g. the |\listfiles|,
%             increasing the chance of the user to read it (trick found in
%             \textsc{Harald Harders'} \xpackage{fnbreak} package, thanks!).
%   \end{Version}
%   \begin{Version}{2010/08/23 v1.1h \xpackage{pagesLTS}}
%     \item Renamed |\XXRoman| to |\XRoman|.
%     \item Reduced the number of needed counters.
%     \item Removed wrong \% from the driver file.
%     \item Changed the |\unit| definition (got rid of an old |\rm|).
%     \item Without use of the \xpackage{hyperref} package, labels of type
%             \texttt{pagesLTS.}\meta{page numbering scheme}\texttt{.local}
%             became multiply defined. Now an |\undolabl| was inserted.
%     \item Diverse details.
%   \end{Version}
%   \begin{Version}{2010/08/25 v1.1i \xpackage{pagesLTS}}
%     \item Bug fix: \textsf{tcilatex} defines the |\hyperref| command, therefore
%             for \xpackage{hyperref} package detection this had to be changed to
%             |\Hy@Warning|.
%   \end{Version}
%   \begin{Version}{2010/09/12 v1.1j \xpackage{pagesLTS}}
%     \item Bug fix: \LaTeX{} issued a \textquotedblleft \texttt{Label(s) may have changed.%
%             Rerun to get cross-references right.}\textquotedblright -warning,
%             even if labels had not changed but were overwritten.
%     \item Starred version of |\lastpageref| for suppressing hyperlinks introduced.
%     \item A lot of details.
%   \end{Version}
%   \begin{Version}{2010/09/22 v1.1k \xpackage{pagesLTS}}
%     \item When no \texttt{fnsymbol} pagenumbering scheme is used, the respective
%             counters are not defined, saving three counters.
%     \item Moved the package from \texttt{.../latex/muench/pagesLTS/...} to
%             \texttt{.../latex/pagesLTS/...}.\\
%             (Please make sure that the old version of the \xpackage{pagesLTS} package
%              was properly uninstalled from your system.)
%   \end{Version}
%   \begin{Version}{2010/09/27 v1.1l \xpackage{pagesLTS}}
%     \item Bug fix: |\PackageError{pagesLTS}{pagenumbering missing}{\pagesLTS@messageNPN }|
%             had to be moved to the outside of |\EveryShipout|, because it wrote its
%             message into the document instead to the screen and the \xfile{.log}-file.
%     \item Updated to version 2010/04/24 v0.19 of the \xpackage{holtxdoc} package.
%   \end{Version}
%   \begin{Version}{2011/02/01 v1.1m \xpackage{pagesLTS}}
%     \item Added a new warning subsection about \xpackage{hyperref} and repeated page
%             numbers.
%     \item Bug fix: Missing \% after |-\romannumeral\number-\arabic{#1}| added.
%     \item The (then) new version v2.4i of the \xpackage{endfloat} package was then
%             even older than 15~years.
%     \item Put a warning in the documentation as well as in the \xfile{log}-file and
%             at the screen during compilation about the \xpackage{showkeys} package.
%             (The labels of the \xpackage{pagesLTS} package cannot be shown by the
%             \xpackage{showkeys} package.)
%     \item Bug fix: In some situations a rerun warning was given even if no rerun
%             was necessary.
%     \item The recent version of the Adobe Reader \sout{\ is}\uwave{\ was} X (10.0.0)
%             (instead of 9.3.3). Its handling of special page numbers was improved.
%     \item The option \texttt{alphMult}\ \ is now set to \texttt{ab} by default.
%     \item The option \texttt{AlphMulti}\ is now set to \texttt{AB} by default.
%     \item Some details.
%   \end{Version}
%   \begin{Version}{2011/03/16 v1.1n \xpackage{pagesLTS}}
%     \item Bug fix: Handling of option |pagecontinue=false| changed.
%             When |pagecontinue=false| was used, but also a
%             |alphMult|, |AlphMulti|, |fnsymbolmult|, |romanMult| or |RomanMulti| option
%             other than |0| or |false|, respectively, was used, the page numbering
%             \textit{was} continued/extended. Now a warning is issued in case of such
%             option clash and |pagecontinue=false| is heeded, disabling all continuation.
%     \item Bug fix: |\ProvidesPackage{pagesLTS}| contained an older date (2010/09/27 of v1.1l
%             instead of 2011/02/01 of v1.1m).
%     \item Bug fix: The \xpackage{ulem} package is needed to generate the documentation from the
%             \xfile{pagesLTS.dtx} file, but was not listed as necessary package.
%     \item Bug fix: One reference to an outdated version of \xpackage{undolabl} package,
%             replaced by the (then) recent version.
%     \item Some minor details.
%   \end{Version}
%   \begin{Version}{2011/03/17 v1.1o \xpackage{pagesLTS}}
%     \item Documentation and ReadMe bug fix: This \xpackage{pagesLTS} package is located at
%             \url{https://www.ctan.org/pkg/pageslts} instead of \ldots /pagesLTS/.
%     \item There is a new (possible) alternative package, \xpackage{totcount},
%             see section~\ref{sec:Alternatives}.
%     \item Bug fix: There was a reference to \xpackage{lastpage} 1994/07/20, v0.1b,
%             instead of the current version.
%   \end{Version}
%   \begin{Version}{2011/08/08 v1.2a}
%     \item \textbf{Renamed the package from \xpackage{pagesLTS} to \xpackage{pageslts}
%             (keeping family, prefix, internal commands,\ldots{} as |pagesLTS|).}
%             Added checking against double loading as \xpackage{pagesLTS} and \xpackage{pageslts}.
%     \item The \xpackage{holtxdoc} package was fixed (recent: 2011/02/04, v0.21), therefore
%             the warning in \xfile{drv} could be removed.
%     \item |\AtEndAfterFileList| from the \uwave{then} new version of the \xpackage{atveryend} package,
%             2011/04/23, v1.7, by \textsc{Heiko Oberdiek}, is now used for the rerun hint
%             instead of appending to |\@dofilelist|.
%     \item Now defining $2$: |\def\pagesLTS@two{2}| (for example for the
%             \xpackage{thumbs} package) and $3$: |\def\pagesLTS@three{3}|.
%     \item Replaced |\texttt{\textbackslash|\ldots|}| by
%             \textbar \textbackslash \ldots \textbar{} in the \xfile{dtx} and by
%             |\verb|\textbar \textbackslash \ldots \textbar{} in the example (where possible).
%     \item When the \xpackage{alphalph} package is needed, it is loaded via |\RequirePackage|
%             instead of crashing with an error message.
%     \item A lot of details (also in the documentation).
%   \end{Version}
%   \begin{Version}{2013/01/28 v1.2b}
%     \item Updated to \TeX{} live 2012 (for compiling the documentation and example)
%             and installed the available updates. Therefore I can no longer test
%             whether \xpackage{pageslts} works with earlier versions of \LaTeX{}.
%     \item Replaced |\let| by |\LetLtxMacro|.
%     \item The \xpackage{nameref} package redefines |\label| to have five arguments
%             instead of two, therefore |\newlabel{LastPage}{{}{\thepage}{}{}{}}| instead of
%             |\newlabel{LastPage}{{}{\thepage}}| must be used.
%             (Bug reported at \url{http://tex.stackexchange.com/q/95541/6865},
%             thanks to Micha\l{} Herman!) Fixed.
%     \item Updates to a lot of details, also in the documentation.
%   \end{Version}
%   \begin{Version}{2014/01/19 v1.2c}
%     \item Bug: missing loop, fix: inserted.
%     \item Bug: when option |pagecontinue=false| was set, the extension of the
%             page numbering schemes was disabled by mistake, fixed.
%     \item Now using |\ltx@ifpackageloaded| from the \xpackage{ltxcmds} package
%             for checking (even \emph{after} |\AtBeginDocument|)
%             whether a package has been loaded.
%     \item Bug: incompatibility with \xpackage{lineno} because of a mistake
%             in the redefined |\pagenumbering|, fixed. (Bug reported by kwikwi, thanks!)
%     \item Bug: When |\pagenumbering| preceded |\frontmatter| on the same page
%             but with different argument than the |\pagenumbering|, which was inside
%             |\frontmatter|, then some labelling got mixed up.
%             (Also this bug reported by kwikwi, thanks!) While this is not fixed automatically,
%             now an appropriate error message is given. (Two different page numbering schemes
%             on the same page make no sense anyway.)
%     \item Updates of minor details.
%   \end{Version}
%   \begin{Version}{2015/08/02 v1.2d}
%     \item Updated to \TeX{} Live 2015 (for compiling the documentation and example)
%             and installed the available updates. Therefore I can no longer test
%             whether \xpackage{thumbs} works with earlier versions of \LaTeX .
%            (It probably does, but there is no guarantee.)
%     \item A |ifundefinedorrelax| similar to the one from \xpackage{scrbase.sty}
%             of the KOMA script bundle as 2013/12/19 v3.12 is used now,
%             without the need for \eTeX .
%     \item New versions of Adobe Reader and of some packages have become available.
%     \item Changed the message (type) to be displayed if writing to files is disallowed
%             (as pointed out by Joshua Ellis, thanks!).
%     \item Updates to several details, also in the documentation.
%   \end{Version}
%   \begin{Version}{2015/08/17 v1.2e}
%     \item Bug fix, see \url{https://tex.stackexchange.com/q/261445/6865},
%             thank you to \textsc{Prof. Enrico Gregorio} for providing the fix
%             and to \textsc{Dr. Clea F. Rees} for bringing this to my attention.
%   \end{Version}
%   \begin{Version}{2015/12/21 v1.2f}
%     \item Replaced |\next| with |\@pageslts@currname|.
%     \item Bug fix, see \url{https://tex.stackexchange.com/q/140235/6865}.
%             If the bug had been actually reported (instead of waiting until I see
%             the question), I would have fixed it earlier, of course.
%     \item Changed minor details like fixing urls in the manual.
%   \end{Version}
% \end{History}
%
% \bigskip
%
% When you find a mistake or have a suggestion for an improvement of this package,
% please send an e-mail to the maintainer, thanks!\\
% (Please see BUG REPORTS in the README.)
%
% \bigskip
%
% Note: \textsf{Y} is not missing in the following index, but no command beginning
% with this letter has been used in this \xpackage{pageslts} package.
%
% \pagebreak
%
% \PrintIndex
%
% \Finale
\endinput