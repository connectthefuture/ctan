%  encoding : utf8 
%  doc de tkz-euclide.sty
%  Created by Alain Matthes  on 2010-04-04.
%  Copyright (C) 2010 Alain Matthes  
%
% This file may be distributed and/or modified
%
% 1. under the LaTeX Project Public License , either version 1.3
% of this license or (at your option) any later version and/or
% 2. under the GNU Public License.
%
% See the file doc/generic/pgf/licenses/LICENSE for more details.%
% See http://www.latex-project.org/lppl.txt for details.
%
%
% TKZdoc-euclide-main is the french  doc of tkz-euclide

\documentclass[DIV         = 12,
               fontsize    = 10,
               headinclude = false,
               index       = totoc,
               footinclude = false,
               twoside,
               headings    = small]{tkz-doc}
%\usepackage{svn-multi}
\usepackage{tkz-euclide}
\usetkzobj{all}
\tkzSetUpColors[background=fondpaille,text=Maroon]
\usepackage[frenchb]{babel}
\usepackage[autolanguage]{numprint}
\usepackage[pdftex,
            unicode,
            colorlinks    = true,
            pdfpagelabels, 
            urlcolor      = blue,
            filecolor     = pdffilecolor,
            linkcolor     = blue,
            breaklinks    = false,
            hyperfootnotes= false,
            bookmarks     = false,
            bookmarksopen = false, 
            linktocpage   = true,
            pdfsubject    ={Euclidean geometry},
            pdfauthor     ={Alain Matthes},
            pdftitle      ={tkz-euclide},
            pdfkeywords   ={euclide,compass,rule,point,line},
            pdfcreator    ={pdfeTeX}
            ]{hyperref}    
\usepackage{url}
\def\UrlFont{\small\ttfamily}
\usepackage[protrusion = true,
            expansion,
            final,
            verbose    = false]{microtype}

\DisableLigatures{encoding = T1, family = tt*} 
\usepackage{tkzexample}  
% \usepackage[saved]{tkzexample}  
% \def\tkzFileSavedPrefix{tkzEucl}%    
\usepackage[parfill]{parskip}
\gdef\nameofpack{tkz-euclide}
\gdef\versionofpack{1.13 c}
\gdef\dateofpack{2011/01/20}
\gdef\nameofdoc{doc-tkz-euclide}
\gdef\dateofdoc{2011/02/18}
\gdef\authorofpack{Alain Matthes}
\gdef\adressofauthor{}
\gdef\namecollection{AlterMundus}
\gdef\urlauthor{http://altermundus.fr}
\gdef\urlauthorcom{http://altermundus.com}
\title{The package : tkz-euclide.sty}
\author{Alain Matthes}

%\usepackage{hvindex}
   
\usepackage{shortvrb,fancyvrb}
\makeatletter
\renewcommand*\l@subsubsection{\bprot@dottedtocline{3}{3.8em}{4em}}
\makeatother
\AtBeginDocument{\MakeShortVerb{\|}}

\pdfcompresslevel=9
\pdfinfo{
    /Title (doc-tkz-euclide.pdf)
    /Creator (TeX)
    /Producer (pdfeTeX)
    /Author (Alain Matthes)
    /CreationDate (18 février 2011)
    /Subject (Documentation du package tkz-euclide v 1.13 c)
    /Keywords (pdfeTeX, geometry, compass, triangle, segment, line, pdflatex) }

%<---------------------------------------------------------------------------> 
\begin{document}

\title{\nameofpack}
\date{\today}
\clearpage
\thispagestyle{empty}
\maketitle

\clearpage
\pagecolor{fondpaille} 
\color{Maroon}    
\colorlet{graphicbackground}{fondpaille}
\colorlet{codebackground}{Peach!30}
\colorlet{codeonlybackground}{Peach!30}    
\colorlet{numbackground}{fondpaille}
\colorlet{textcodecolor}{Maroon}
\colorlet{numcolor}{gray}  

\nameoffile{\nameofpack} 
\defoffile{Le package \textbf{tkz-euclide.sty} est un ensemble de macros spécialisées permettant de construire des objets géométriques en 2D dans un plan muni d'un repère. Il est construit au-dessus de PGF et son interface TikZ. Ce document fournit les définitions des différentes macros ainsi que des exemples dont la complexité est graduée. \textbf{tkz-euclide.sty} remplace \textbf{tkz-2d.sty} dont le code n'est plus maintenu. Ce package nécessite la version 2.1 de \TIKZ.}

\presentation  

\vspace*{1cm}  
\noindent\lefthand\ Je souhaite remercier \textbf{Till Tantau} pour avoir créé le merveilleux outil \href{http://sourceforge.net/projects/pgf/}{Ti\emph{k}Z}, ainsi que \tkzimp{Michel Bovani} pour \tkzname{fourier}, dont l'association avec \tkzname{utopia} est excellente. 

\vspace*{12pt} 
\noindent\lefthand\ Je  remercie \textbf{Yve Combe} pour avoir partagé son travail sur le rapporteur et les constructions à l'aide du compas. Je souhaite remercier également,  \tkzimp{David Arnold} qui a corrigé un grand nombre d'erreurs et qui a testé de nombreux exemples,  \tkzimp{Wolfgang Büchel} qui a corrigé également des erreurs et a construit de superbes scripts pour obtenir les fichiers d'exemples,  \tkzimp{John Kitzmiller} et \tkzimp{Dimitri Kapetas}  pour leurs exemples, et enfin  \tkzimp{Gaétan Marris} pour ses remarques et corrections.

\vspace*{12pt}
\noindent\lefthand\ Vous trouverez de nombreux exemples sur mes sites~: 
\href{http://altermundus.com/pages/download.html}{altermundus.com} ou 
\href{http://altermundus.fr/pages/download.html}{altermundus.fr}    

\vfill   
Vous pouvez envoyer vos remarques, et les rapports sur des erreurs que vous aurez constatées à l'adresse suivante~: \href{mailto:al.ma@mac.com}{\textcolor{blue}{Alain Matthes}}.
 
This file can be redistributed and/or modified under the terms of the LATEX 
Project Public License Distributed from \href{http://www.ctan.org/}{CTAN}\  archives.

\vspace{1cm}
\begin{center}
  \begin{tikzpicture}[decoration=snake,color=Peach,line width=1pt]
    \draw[decorate] (0,0)--(\textwidth-1cm,0);
  \end{tikzpicture}
\end{center}

\clearpage
\tableofcontents

\vspace{1cm}
\begin{center}
  \begin{tikzpicture}[decoration=snake,color=Peach,line width=1pt]
    \draw[decorate] (0,0)--(\textwidth-1cm,0);
  \end{tikzpicture}
\end{center}
\clearpage   \newpage

\setlength{\parskip}{1ex plus 0.5ex minus 0.2ex}
%!TEX root = /Users/ego/Boulot/TKZ/tkz-euclide/doc_fr/TKZdoc-euclide-main.tex
\section{Installation}    \NameDist{TeXLive}


Lorsque vous lirez ce document, il est possible que \tkzname{tkz-euclide}  soit présent sur le serveur du \tkzname{CTAN}\footnote{\tkzname{tkz-euclide} ne fait pas encore partie de \tkzname{TeXLive}} alors  \tkzname{tlmgr} vous permettra de l'installer.  Si  \tkzname{tkz-euclide} ne fait pas encore partie de votre distribution, cette section vous montre comment l'installer, elle est aussi nécessaire si vous avez envie d'installer une version beta  ou personnalisée de \tkzname{tkz-euclide}. Si le package est présent 
 sur le serveur du \tkzname{CTAN} et que vous n'utilisez pas \tkzname{tlmgr},  je vous conseille de la télécharger à partir de ce serveur, sinon vous le trouverez sur mon site.
 Pour distinguer les anciennes versions de la nouvelle, j'ai repris la numérotation à 1.00 et j'ai ajouté « c »\footnote{pour CTAN}  . Vous allez donc installer la version \tkzname{1.13 c}.

Le plus simple est de créer un dossier \tikz[remember picture,baseline=(n1.base)]\node [fill=blue!30,draw] (n1) {tkz};\footnote{ou bien un autre nom}  avec comme chemin : \colorbox{blue!20}{ texmf/tex/latex/tkz}.

\medskip
\begin{enumerate}
\item Après l'avoir décompressé, placez le dossier \tikz[remember picture,baseline=(n2.base)]\node [fill=blue!20,draw] (n2) {tkzeuclide}; dans le dossier \tikz[baseline=(tk.base)]\node [fill=blue!30,draw] (tk) {tkz};. Le dossier \tkzname{tkzbase} doit se trouver aussi dans le dossier \tkzname{tkz}.


\medskip
\begin{tikzpicture} [remember picture,rotate=90] 

\node (texmf)   at (4,2)  [draw,fill=blue!30 ] {texmf};

\node (tex)     at (6,0)   [draw ] {tex}; 
\node (doc)     at (2,0)   [draw ] {doc};

\node (texgen)  at (7,-2)  [draw ] {generic};
\node (docgen)  at (0,-2)  [draw ] {generic};

\node (latex)   at (4,-2)  [draw ] {latex}; 

\node (genpgf)  at (7,-4)  [draw] {pgf};
\node (latpgf)  at (5,-4)  [draw] {pgf};
\node (tkz)     at (4,-4)  [draw,fill=blue!20 ] {tkz};

\node (docpgf)  at (0,-4)  [draw] {pgf};

\node (fct)     at (6,-6)  [draw,fill=orange!20] {tkz-fct.sty};
\node (tkb)     at (4,-6)  [draw,fill=blue!20] {tkzeuclide};
\node (tke)     at (2,-6)  [draw,fill=blue!20] {tkzbase};

\node (sym)     at (10,-11)  [draw,fill=green!20] {tkz-lib-symbols.tex};
\node (add)     at (9,-11)  [draw,fill=green!20] {tkz-obj-addpoints.tex};  
\node (tuti)     at (8,-11)  [draw,fill=green!20] {tkz-obj-angles.tex}; 
\node (tmisc)    at (7,-11)  [draw,fill=green!20] {tkz-obj-arcs.tex};
\node (tmath)    at (6,-11)  [draw,fill=green!20] {tkz-obj-circles.tex};
\node (tbas)     at (5,-11)  [draw,fill=green!20] {tkz-obj-lines.tex};
\node (base)     at (4,-11)  [draw,fill=green!20] {tkz-euclide.sty}; 
\node (cfg)      at (3,-11)  [draw,fill=green!20]   {tkz-obj-protractor.tex};
\node (mark)     at (2,-11)  [draw,fill=green!20]   {tkz-obj-polygons.tex}; 
\node (pts)      at (1,-11)  [draw,fill=green!20]   {tkz-obj-sectors.tex};
\node (int) at (0,-11)  [draw,fill=green!20]   {tkz-tools-intersections.tex}; 
\node (tsf) at (-1,-11) [draw,fill=green!20]  {tkz-tools-transformations.tex};

\draw[-open triangle 90](texmf.north east) --(tex.south west)    ;
\draw[-open triangle 90](texmf.south east) -- (doc.north west)   ;

\draw[-open triangle 90](tex.north east) --(texgen.south west)    ;
\draw[-open triangle 90](tex.south east) -- (latex.north west)   ; 
\draw[-open triangle 90](texgen.east) -- (genpgf.west)   ;  

\draw[-open triangle 90](doc.south east) -- (docgen.north west)   ; 
\draw[-open triangle 90](docgen.east) -- (docpgf.west)   ; 

\draw[-open triangle 90](latex.north east) -- (latpgf.south west)   ; 
\draw[-open triangle 90](latex.east) -- (tkz.west)   ;    
 
\draw[-open triangle 90,blue!40](tkz.east) to[out=-90,in=90](fct.west) ;
\draw[-open triangle 90,blue!40](tkz.east) to[out=-90,in=90](tkb.west) ; 
\draw[-open triangle 90,blue!40](tkz.east) to[out=-90,in=90](tke.west) ; 

\draw[-open triangle 90,blue!40](tkb.east) to[out=-90,in=90](sym.west) ; 
\draw[-open triangle 90,blue!40](tkb.east) to[out=-90,in=90](add.west) ;
\draw[-open triangle 90,blue!40](tkb.east) to[out=-90,in=90](tuti.west) ; 
\draw[-open triangle 90,blue!40](tkb.east) to[out=-90,in=90](tmisc.west) ; 
\draw[-open triangle 90,blue!40](tkb.east) to[out=-90,in=90](tmath.west) ; 
\draw[-open triangle 90,blue!40](tkb.east) to[out=-90,in=90](tbas.west) ; 
\draw[-open triangle 90,blue!40](tkb.east) to[out=-90,in=90](base.west) ; 
\draw[-open triangle 90,blue!40](tkb.east) to[out=-90,in=90](cfg.west) ; 
\draw[-open triangle 90,blue!40](tkb.east) to[out=-90,in=90](mark.west) ; 
\draw[-open triangle 90,blue!40](tkb.east) to[out=-90,in=90](pts.west) ; 
\draw[-open triangle 90,blue!40](tkb.east) to[out=-90,in=90](int.west) ; 
\draw[-open triangle 90,blue!40](tkb.east) to[out=-90,in=90](tsf.west) ; 

\end{tikzpicture}
\begin{tikzpicture}[remember picture,overlay]
        \path[->,thin,red!40,>=latex] (n1) edge [bend left] (tkz);
        \path[->,thin,red!40,>=latex] (n2) edge [bend left] (tkb);
\end{tikzpicture}    

\newpage 

Il est nécessaire   que \tkzname{tkz-base} soit aussi installé. Le plus simple est d'installer \tkzname{tkz} complètement.   

\item Ouvrir un terminal, puis faire \colorbox{red!20}{|sudo texhash|} si nécessaire.
\item Vérifier que  \tkzname{fp}, \tkzname{numprint} et \tkzname{tikz 2.10} sont installés car ils sont obligatoires, pour le bon fonctionnement de \tkzname{tkz-euclide}.

\end{enumerate}  

 Voici les chemins du dossier tkz sur mes deux ordinateurs:      

\medskip
\begin{itemize}\setlength{\itemsep}{5pt}
\item   sous OS X \colorbox{blue!30}{\textbf{/Users/ego/Library/texmf}}; 
\item   sous Ubuntu \colorbox{blue!30}{\textbf{/home/ego/texmf}}.
\end{itemize}\NameSys{Linux}\NameSys{OS X} 


Je suppose que si vous mettez vos packages ailleurs, vous savez pourquoi !

\emph{remarque : l'installation proposée n'est valable que pour un utilisateur.}  

\subsection{Avec MikTeX sous Windows XP}\NameDist{MikTeX}\NameSys{Windows XP}

Je ne connais pas grand-chose à ce système, mais un utilisateur de mes packages \tkzimp{Wolfgang Buechel} a eu la gentillesse de me faire parvenir ce qui suit~:

Pour ajouter \tkzname{tkzeuclide} à MiKTeX\footnote{Essai réalisé avec la version \tkzname{2.7}}:

\begin{itemize}\setlength{\itemsep}{10pt}
  \item ajouter un dossier \tkzname{tkz} dans le dossier
       \textcolor{blue!60!black}{\texttt{[MiKTeX-dir]/tex/latex}}
  \item copier \tkzname{tkzeuclide} et tous les fichiers présents  dans le dossier \tkzname{tkz},
  \item mettre à jour  MiKTeX, pour cela dans shell DOS lancer la commande   \textbf{\textcolor{red}{|mktexlsr -u|}} 
  
   ou bien encore, choisir \textcolor{red!50}{|Start/Programs/Miktex/Settings/General|}
   
    puis appuyer sur le bouton  \textbf{\textcolor{red}{|Refresh FNDB|}}.
\end{itemize} 

\subsection{Liste des fichiers des dossiers \tkzname{tkzbase}  et \tkzname{tkzeuclide}}

Dans le dossier \tkzname{base}  :

\begin{itemize}
\item  \tkzname{tkz-base.cfg            }
\item  \tkzname{tkz-base.sty            }
\item  \tkzname{tkz-obj-marks.tex       }
\item  \tkzname{tkz-obj-points.tex      }
\item  \tkzname{tkz-obj-segments.tex    }
\item  \tkzname{tkz-tools-arith.tex     }
\item  \tkzname{tkz-tools-base.tex      }
\item  \tkzname{tkz-tools-math.tex      }
\item  \tkzname{tkz-tools-misc.tex      }
\item  \tkzname{tkz-tools-utilities.tex }
\end{itemize}

Dans le dossier \tkzname{euclide}  : 
 
\begin{itemize} 
\item   \tkzname{tkz-euclide.sty              }
\item   \tkzname{tkz-lib-symbols.tex          }
\item   \tkzname{tkz-obj-addpoints.tex        }
\item   \tkzname{tkz-obj-angles.tex           }
\item   \tkzname{tkz-obj-arcs.tex             }
\item   \tkzname{tkz-obj-circles.tex          }
\item   \tkzname{tkz-obj-lines.tex            }
\item   \tkzname{tkz-obj-protractor.tex       }
\item   \tkzname{tkz-obj-polygons.tex         }
\item   \tkzname{tkz-obj-sectors.tex          }
\item   \tkzname{tkz-obj-vectors.tex          }
\item   \tkzname{tkz-tools-intersections.tex  }
\item   \tkzname{tkz-tools-transformations.tex}
\end{itemize}

\subsection{Chargement des fichiers avec \tkzname{usetkzobj}}
Il n'était pas nécessaire de tout charger en une seule fois, seuls les fichiers indispensables sont installés. \tkzcname{usepackage\{tkz-base\}} charge tous les fichiers présents dans le dossier  \tkzname{tkzbase}; en particulier, les fichiers "objets" \tkzname{tkz-obj-points.tex} et \tkzname{tkz-obj-segments.tex} et \tkzname{tkz-obj-marks.tex}.
\tkzcname{usepackage\{tkz-euclide\}} va ajouter des outils indispensables, mais vous devrez indiquer quels objets vous seront utiles. Pour tout charger, vous pouvez écrire :  \tkzcname{usetkzobj\{all\}}  mais sinon vous pouvez demander :
   \tkzcname{usetkzobj\{cercles, arcs, protractor\}}.
\endinput


%!TEX root = /Users/ego/Boulot/TKZ/tkz-euclide/doc_fr/TKZdoc-euclide-main.tex

%<–––––––––––––––––––––––––––––––––––––––––––––––––––––––––––––––––––––––––>
\section{Présentation}
%<–––––––––––––––––––––––––––––––––––––––––––––––––––––––––––––––––––––––––>

\subsection{À propos de Ti\emph{k}Z et que peut apporter \tkzname{tkz-euclide.sty} ? }
\TIKZ\ est un outil que je trouve très agréable à utiliser. J'ai trouvé si simple son utilisation que je me suis demandé si cela avait un sens de créer un package pour  la création de dessins en 2d et en particulier pour créer des dessins liés à la géométrie euclidienne.  Quels arguments peuvent intervenir? 

\begin{enumerate}

\item Certains utilisateurs n'ont pas envie d'apprendre quoi que ce soit sur  \TIKZ, cela est respectable et une simplification du code par l'intermédiaire d'un package peut avoir une certaine utilité. La syntaxe n'est plus tout à fait celle de  \TIKZ, mais ressemble davantage à celle de \LATEX.

\item  Les noms des macros ont une signification plus mathématique.

\item La grande différence avec \TIKZ\ est qu'il est possible d'utiliser des grandes valeurs ainsi que des très petites, car la majorité des calculs sont faits à l'aide de \tkzNamePack{fp.sty}. C'est plus lent, mais nettement plus précis.

\item Il est possible de modifier facilement les styles pour  les objets principaux que sont les points, les droites, les cercles, les arcs, etc. 

\item Des exemples de constructions géométriques sont fournies et peuvent être utiles au débutant.
  
\item  Et pour terminer, cela peut être une approche en douceur de l'utilisation de \TIKZ\, par l'intermédiaire des options. Dans cette nouvelle version, j'ai essayé que les options de \TIKZ\ soient pratiquement toujours disponibles.

\end{enumerate}

Je vous encourage toutefois à étudier \TIKZ. En effet, l'utilisation de \tkzname{tkz-euclide.sty} fait perdre la notion de \tkzname{path}. Je donnerai quelques exemples pour voir les différences entre les codes. Cela dit, il est toujours possible de mélanger les différents codes et différentes syntaxes, cela n'est pas franchement satisfaisant, mais peut permettre de résoudre certains problèmes.


\subsection{À propos de \tkzname{tkz-euclide}}

Le but est donc de créer des dessins en 2D sur une page à priori A4, mais si je me suis préoccupé d'utiliser une surface inférieure, j'avoue ne pas avoir testé la possibilité de travailler sur une page de taille supérieure.

Avec \tkzname{tkz-euclide}, l'unité est le centimètre. Si votre travail ne concerne que de la géométrie classique, je vous conseille de conserver cette unité.

\emph{Pourquoi \tkzname{tkz-2d} disparait-il?}

Je n'étais pas content de la syntaxe qui était confuse, je n'avais pas utilisé pgf 2.00 et surtout j'ai généralisé l'utilisation de \tkzname{fp.sty}.

\clearpage \newpage
\section{Syntaxe}
Quelques mots sur la syntaxe.

 Les accolades sont réservés pour la création d'objets et les parenthèses ne sont utilisées que pour des objets, déjà existants~:
 
 \tkzcname{tkzDefPoint(1,2)\{A\}} crée le point nommé $A$.
 
\tkzcname{tkzLabelSegment[below](O,A)\{\$1\$\}} crée le label $1$ pour le segment $[OA]$.
 
 Enfin des macros comme \tkzcname{tkzDefMidPoint(O,A)} crée un point, qui est ici, le milieu d'un segment. Le point est nommé \tkzname{tkzPointResult}. 

Soit la création est une étape intermédiaire, et vous n'avez pas besoin de conserver ce point, alors tant qu'aucune macro ne modifie  l'attribution de \tkzname{tkzPointResult}, vous pouvez utiliser ce nom pour faire référence au milieu; soit   vous voulez conserver ce point, car il sera utilisé plusieurs fois, alors la macro  \tkzcname{tkzGetPoint\{M\}} permet d'attribuer le nom \tkzname{M} au point.

 Quant une macro donne comme résultat deux points, le premier est nommé \tkzname{tkzFirstPointResult} et le second \tkzname{tkzSecondPointResult}, la macro qui permet de récupérer les points est :
 
 \begin{itemize}
   \item \tkzcname{tkzGetPoints\{M\}\{N\}} qui attribue deux noms;
   \item \tkzcname{tkzGetFirstPoint\{M\}} seul le premier point sera utilisé;
   \item \tkzcname{tkzGetSecondPoint\{N\}} cette fois, seul le second point est nommé.
 \end{itemize}
Il est difficile de conserver un découpage du code comme dans l'exemple, si on ne veut pas nommer un point par exemple H dans l'\hyperlink{firstex}{exemple} minimal, mais complet de la section suivante.

Le code pourrait devenir :

\begin{tkzltxexample}[]
 \tkzDefPointWith[orthogonal](I,M) %\tkzGetPoint{H}
 \tkzDrawSegment[style=dashed](I,tkzPointResult)
 \tkzInterLC(I,tkzPointResult)(M,A)    \tkzGetSecondPoint{B}
\end{tkzltxexample}

\subsection{Notions générales}

Le principe est de définir des points en utilisant des coordonnées cartésiennes ou des coordonnées polaires et même des coordonnées barycentriques. 

Ensuite, il est possible d'obtenir d'autres points comme intersections d'objets, comme images d'autres points à l'aide de transformations ou bien encore des points issus de propriétés vectorielles.

\begin{itemize}
   \item \tkzcname{tkzDefPoint} pour l'usage de coordonnées,
      \item \tkzcname{tkzDefPointBy} pour l'usage des transformations,
      \item \tkzcname{tkzDefPointWith} pour l'usage des propriétés vectorielles,
   \item et enfin \tkzcname{tkzInterLL}, \tkzcname{tkzInterLC} et \tkzcname{tkzInterCC} sont les trois types d'intersections possibles  de droites et de cercles. Pour ces trois macros, j'ai préféré utiliser \tkzname{fp.sty} afin d'obtenir des résultats plus précis.
\end{itemize}


Puis à l'aide de ces points, nous pouvons tracer des objets comme des segments, des demi-droites, des droites, des triangles,  des cercles, des arcs etc.

Cela se fait à l'aide de macros dont le nom commence par \tkzcname{tkzDraw...}.

Enfin il est possible de placer des labels à l'aide de macros dont le nom commence par \tkzcname{tkzLabel...}.

Cela permet à ceux qui le souhaitent, de décomposer la création des figures en quatre étapes~:
\begin{enumerate}
   \item Définir les points dont les coordonnées sont connues ou bien calculables.
   \item Création de nouveaux points à l'aide de méthodes (intersection, transformation,etc.).
   \item Tracés des objets dans un ordre choisi.
   \item Placement des labels.
\end{enumerate}


Les coordonnées peuvent être obtenues à l'aide de calculs en utilisant pgfmath, fp ou encore \TEX. Toutes les macros n'acceptent pas que les calculs soient  faits pendant leurs assignations. Après avoir toléré ce comportement, je l'ai abandonné afin de laisser plus de souplesse à l'utilisateur. \tkzNamePack{fp.sty} est plus précis \tkzNamePack{pgfmath}, plus rapide aussi tout dépend des constructions demandées.

D'une façon générale, la syntaxe est plus homogène. Les noms des points créés sont entre accolades alors que les noms des points utilisés sont entre parenthèses. 

Après beaucoup d'hésitations, j'ai choisi le procédé suivant. Quand une macro crée un point, deux points, donne la mesure d'un angle alors le résultat est rangé dans un nom de générique. Ainsi l'intersection de deux droites définit un point appelé \tkzname{tkzPointResult}, celle de deux cercles donne \tkzname{tkzFirstPointResult} et \tkzname{tkzSecondPointResult}. Certaines macros définissent  une mesure de rayon qui sera alors dans une macro \tkzcname{tkzLengthResult} et d'autres la mesure d'un angle \tkzcname{tkzAngleResult}.
 Des macros sont fournies pour nommer différemment ces résultats et les conserver. Il pourrait paraître plus simple de donner un paramètre supplémentaire à la macro pour nommer directement le résultat, mais par exemple, on peut n'avoir besoin que d'un point sur deux après une intersection, une macro peut définir trois résultats un angle , une longueur et un point. Ensuite il est facile à l'utilisateur de créer des macros qui feront tout cela d'un seul coup si cela est nécessaire. 

\tkzcname{tkzDefPoint} utilise des accolades ainsi que les macros créant des labels. Il en est de même des transformations quand elles agissent sur une liste de points. 


\clearpage \newpage
\section{Exemple minimal, mais complet}
Cet exemple se trouve dans le dossier du package, et vous permet de tester votre installation.

Une unité de longueur étant choisie, l'exemple montre comment obtenir un segment de longueur $\sqrt{a}$ à partir d'un segment de longueur $a$, à l'aide d'une règle et d'un compas. 

$IM=a$, $OI=1$

\vspace{12pt}
\hypertarget{firstex}{}
\begin{center}
\begin{tikzpicture}[scale=.8]
   \tkzInit[ymin=-1,ymax=6,xmin=-1,xmax=10]    
   \tkzClip[space=.5]
   \tkzDefPoint(0,0){O}
   \tkzDefPoint(1,0){I}
   \tkzDefPoint(10,0){A}
   \tkzDefMidPoint(O,A)  \tkzGetPoint{M} 
   \tkzDefPointWith[orthogonal](I,M) \tkzGetPoint{H} 
   \tkzInterLC(I,H)(M,A)  \tkzGetSecondPoint{B}
   \tkzDrawSegment(O,A)
   \tkzDrawSegment[style=dashed](I,H)
   \tkzDrawPoints(O,I,A,B,M)
   \tkzDrawArc(M,A)(O)
   \tkzMarkRightAngle(A,I,B)
   \tkzLabelSegment[right=4pt](I,B){$\sqrt{a}$}
   \tkzLabelSegment[below](O,I){$1$} 
   \tkzLabelSegment[below](I,M){$a/2$}
   \tkzLabelSegment[below](M,A){$a/2$}
   \tkzLabelPoints(I,M,B,A) 
   \tkzLabelPoint[below left](O){$O$} 
\end{tikzpicture} 
\end{center}

\emph{Commentaires}

 Voyons tout d'abord le préambule. Il faut charger \tkzname{xcolor.sty} avant  \tkzname{tkz-euclide.sty} c'est à dire avant \TIKZ. Les options de \tkzname{xcolor.sty} dépendent des couleurs que vous utiliserez. Sinon, Il n'y  rien de particulier à signaler, à l'exception du fait que \TIKZ{} peut poser des problèmes avec les caractères actifs de \tkzname{frenchb} de \tkzNamePack{babel}, aussi j'ai créé deux macros \tkzNameMacro{tkzActivOff}  et \tkzNameMacro{tkzActivOn} pour désactiver puis réactiver ces caractères.

\begin{center}
\begin{tkzltxexample}[]
\documentclass{scrartcl}
\usepackage[utf8]{inputenc} 
\usepackage[upright]{fourier} 
\usepackage[usenames,dvipsnames,svgnames]{xcolor}
\usepackage{tkz-euclide} 
\usetkzobj{all} % on charge tous les objets
\usepackage[frenchb]{babel}
\end{tkzltxexample}
\end{center}


\emph{Commentaires}

Le code suivant comprend quatre parties : 
\begin{itemize}
   \item la première prépare le support. Ici, les deux  lignes \tkzimp{2} et \tkzimp{3} permettent de limiter la taille du dessin.
  \item la deuxième comprend les définitions de points nécessaires à la contruction, ce sont les lignes qui vont de \tkzimp{4} et \tkzimp{9};
  
  \item la troisième comprend les différents tracés,  les lignes de \textcolor{brown}{10} et \textcolor{brown}{14};
 
 \item la dernière ne s'occupe que du placement des labels.
\end{itemize}

\begin{enumerate}
\item  Mise en place
\begin{tkzltxexample}[num]
\begin{tikzpicture}[scale=.8]
   \tkzInit[ymin=-1,ymax=5,xmin=-1,xmax=10]
   \tkzClip
   \end{tkzltxexample}
\item  Création des points
\begin{tkzltxexample}[global num]
   \tkzDefPoint(0,0){O}
   \tkzDefPoint(1,0){I}
   \tkzDefPointBy[homothety=center O ratio  10 ](I)  \tkzGetPoint{A}
   \tkzDefMidPoint(O,A)              \tkzGetPoint{M}
   \tkzDefPointWith[orthogonal](I,M) \tkzGetPoint{H}
   \tkzInterLC(I,H)(M,A)             \tkzGetSecondPoint{B}
   \end{tkzltxexample}
\item  Tracés
\begin{tkzltxexample}[global num]
   \tkzDrawSegment(O,A)
   \tkzDrawSegment[style=dashed](I,H)
   \tkzDrawPoints(O,I,A,B,M)
   \tkzDrawArc(M,A)(O)
   \tkzMarkRightAngle(A,I,B)\end{tkzltxexample}
\item  Création des labels pour les points et les segments
\begin{tkzltxexample}[global num]
   \tkzLabelSegment[right=4pt](I,B){$\sqrt{a}$}
   \tkzLabelSegment[below](O,I){$1$} 
   \tkzLabelSegment[below](I,M){$a/2$}
   \tkzLabelSegment[below](M,A){$a/2$}
   \tkzLabelPoints(I,M,B,A) 
   \tkzLabelPoint[below left](O){$O$}
\end{tikzpicture}\end{tkzltxexample}
\end{enumerate}

\endinput

%<–––––––––––––––––––––––––––––––––––––––––––––––––––––––––––––––––––––––––>
\section{Syntaxe}
%<–––––––––––––––––––––––––––––––––––––––––––––––––––––––––––––––––––––––––>
% \macro[options](){}
Toutes les macros commencent par \tkzname{tkz}. Elles permettent de définir un objet ( en général un point), de tracer cette objet, de placer un label ou encore une marque
\begin{itemize}
  \item \tkzhname{\hyperlink{defpt}{tkzDefPoint}}
  \item \tkzhname{\hyperlink{defpts}{tkzDefPoints}}
  \item \tkzhname{\hyperlink{defptby}{tkzDefPointBy}}
  \item \tkzhname{\hyperlink{defptwith}{tkzDefPointWith}}
  \item \tkzhname{\hyperlink{defptmid}{tkzDefMidPoint}}
  \item \tkzhname{\hyperlink{defptequi}{tkzDefEquiPoint}}
  \item \tkzhname{\hyperlink{defl}{tkzDefLine}}
  \item \tkzhname{\hyperlink{defc}{tkzDefCircle}}
  \item \tkzhname{\hyperlink{defequi}{tkzDefEquilateral}}
  \item \tkzhname{\hyperlink{defsq}{tkzDefSquare}}
  \item \tkzhname{\hyperlink{defll}{tkzDefLLgram}}
  \item \tkzhname{\hyperlink{drpt}{tkzDrawPoint}}
  \item \tkzhname{\hyperlink{drpts}{tkzDrawPoints}}
  \item \tkzhname{\hyperlink{drl}{tkzDrawLine}}
  \item \tkzhname{\hyperlink{drls}{tkzDrawLines}}
  \item \tkzhname{\hyperlink{drs}{tkzDrawSegment}}
  \item \tkzhname{\hyperlink{drss}{tkzDrawSegments}}
  \item \tkzhname{\hyperlink{drc}{tkzDrawCircle}}
  \item \tkzhname{\hyperlink{dra}{tkzDrawArc}}
  \item \tkzhname{\hyperlink{drsec}{tkzDrawSector}}   
  \item \tkzhname{\hyperlink{drp}{tkzDrawPolygon}}
  \item \tkzhname{\hyperlink{drps}{tkzDrawPolySeg}}
  \item \tkzhname{\hyperlink{drm}{tkzDrawMark}}
  \item \tkzhname{\hyperlink{drms}{tkzDrawMarks}}    
  \item \tkzhname{\hyperlink{fa}{tkzFindAngle}}
\end{itemize} 

%!TEX root = /Users/ego/Boulot/TKZ/tkz-euclide/doc_fr/TKZdoc-euclide-main.tex

%<–––––––––––––––––––––––––––––––––––––––––––––––––––––––––––––––––––––––––>
\section{Résumé de tkz-base}
%<–––––––––––––––––––––––––––––––––––––––––––––––––––––––––––––––––––––––––>


\subsection{Utilité de \tkzname{tkz-base}}
\tkzNamePack{tkz-base} permet de simplifier l'utilisation d'intervalles de valeurs divers, ce package est nécessaire pour utiliser \tkzname{tkz-tukey}, un package pour dessiner les représentations graphiques en statistiques élémentaires (ce package n'est pas encore en version officielle). Il est aussi nécessaire  avec \tkzNamePack{tkz-fct}, pas plus officiel que le précédent et qui permet de dessiner les représentations graphiques des fonctions. Il utile également avec \tkzname{tkz-euclide}, mais pas pour les mêmes raisons, car l'unité par défaut, le cm, convient parfaitement.

Premièrement, il faut savoir qu'il n'est pas nécessaire de s'occuper avec \TIKZ\ de la taille du support (background). Cependant il est parfois nécessaire, soit de tracer une grille, soit de tracer des axes, soit de travailler avec une unité différente que le centimètre, soit finalement de contrôler la taille de ce qui sera affiché.
 Pour cela, il faut avoir préparé le repère dans lequel vous allez travailler, c'est le rôle de \tkzNamePack{tkz-base} et de sa macro principale  \tkzNameMacro{tkzInit}. Par exemple, si l'on veut travailler sur un carré de 10 cm de côté, mais  tel que l'unité soit le dm alors il faudra utiliser.

\tkzcname{tkzInit[xmax=1,ymax=1,xstep=0.1,ystep=0.1]}

 en revanche pour des valeurs de $x$ comprises entre \numprint{0} et \numprint{10000} et des valeurs de $y$ comprises entre \numprint{0} et \numprint{100000}, il faudra écrire
 
\tkzcname{tkzInit[xmax=10000,ymax=100000,xstep=1000,ystep=10000]}

Tout cela a peu de sens pour faire de la géométrie euclidienne, et dans ce cas, il est recommandé de laisser l'unité graphique égale à 1 cm. Je n'ai d'ailleurs pas testé si toutes les macros  destinées à la géométrie euclidienne, acceptaient d'autres valeurs que \tkzname{xstep=1} et \tkzname{ystep=1}. En revanche pour certains dessins, il est intéressant de fixer les valeurs extrêmes et de « clipper » le rectangle de définition afin de contrôler au mieux la taille de la figure.

Les principales macros de \tkzNamePack{tkz-base} sont:
\begin{itemize}
   \item \tkzcname{tkzInit}
   \item \tkzcname{tkzClip}
   \item \tkzcname{tkzAxeXY}
   \item \tkzcname{tkzAxeX}
   \item \tkzcname{tkzAxeY}
   \item \tkzcname{tkzDrawX}
   \item \tkzcname{tkzDrawY}
   \item \tkzcname{tkzLabelX}
   \item \tkzcname{tkzLabelY}
   \item \tkzcname{tkzGrid}
   \item \tkzcname{tkzRep}
\end{itemize}

Vous trouverez de multiples exemples dans la documentation de \tkzname{tkz-base}.
 
\newpage
\subsection{Exemple avec \tkzcname{tkzInit}} 

\begin{center}
\begin{tkzexample}[latex=8cm]
\begin{tikzpicture}
 \tkzInit[xmax=3,ymax=3]  
 \tkzAxeXY 
 \tkzGrid
\end{tikzpicture}
\end{tkzexample}
\end{center}  


\subsection{\tkzcname{tkzClip}}
Le rôle de cette macro est de « clipper » le rectangle initial afin que ne soient affichés que les tracés contenus dans ce rectangle.

\begin{tkzexample}[latex=8cm]
\begin{tikzpicture}
 \tkzInit[xmax=4, ymax=3]
 \tkzAxeXY 
 \tkzGrid
 \tkzClip
 \draw[red] (-1,-1)--(5,5);
\end{tikzpicture}
\end{tkzexample} 

Il est possible d'ajouter un peu d'espace
\begin{tkzltxexample}[]
  \tkzClip[space=1]
\end{tkzltxexample} 

\subsection{\tkzcname{tkzClip} et l'option \tkzname{space}} 

\begin{tkzexample}[latex=8cm]
\begin{tikzpicture}
 \tkzInit[xmax=4, ymax=3]
 \tkzAxeXY 
 \tkzGrid
 \tkzClip[space=-1]
 \draw[red] (-1,-1)--(5,5);
\end{tikzpicture}
\end{tkzexample}   
les dimensions du rectangle clippé sont \tkzname{xmin-1}, \tkzname{ymin-1}, \tkzname{xmax+1} et \tkzname{ymax+1}. 

\subsection{\tkzcname{tkzGrid} et l'option \tkzname{sub}}
L'option \tkzname{sub} permet d'afficher un grille secondaire plus fine.
\begin{tkzexample}[latex=8cm]
\begin{tikzpicture}
 \tkzInit[xmax=4, ymax=3]
 \tkzAxeXY
 \tkzGrid[sub]
\end{tikzpicture}
\end{tkzexample}  

\subsection{\tkzcname{tkzGrid} et les couleurs}
L'option \tkzname{sub} permet d'afficher un grille secondaire plus fine.
\begin{tkzexample}[latex=8cm]
\begin{tikzpicture}
 \tkzInit[xmax=4, ymax=3]
 \tkzAxeXY 
 \tkzGrid[sub,color=bistre,
          subxstep=.5,subystep=.5]
\end{tikzpicture}
\end{tkzexample}  

\endinput
%!TEX root = /Users/ego/Boulot/TKZ/tkz-euclide/doc_fr/TKZdoc-euclide-main.tex

\section{Les points}


J'ai fait une distinction entre le point utilisé en géométrie euclidienne et le point pour représenter un élément d'un nuage statistique. Dans le premier cas, j'utilise comme objet un \tkzname{node}, ce qui se traduit par le fait que la représentation du point ne peut être modifiée par un \tkzname{scale}; dans le second cas, j'utilise comme objet un \tkzname{plot mark}. Ce dernier peut être mis à l'échelle et posséder des formes plus variées que le node.

La nouvelle macro est \tkzNameMacro{tkzDefPoint}, celle-ci permet d'utiliser des options propres à \TIKZ\ comme shift et les valeurs sont traitées avec tkz-base. De plus, si des calculs sont nécessaires alors c'est le package \tkzNamePack{fp.sty} qui s'en charge. On peut utiliser les coordonnées cartésiennes ou polaires.

\subsection{Définition d'un point en coordonnées cartésiennes : \tkzcname{tkzDefPoint}} \hypertarget{tdp}{}

\begin{NewMacroBox}{tkzDefPoint}{\oarg{local options}\parg{x,y}\marg{name} ou \parg{a:r}\marg{name}}

\begin{tabular}{lll}
\toprule
arguments &  défaut  & définition  \\ 
\midrule
\TAline{x,y}{no default}{x et y sont deux dimensions, par défaut en cm.}
\TAline{a:r}{no default}{a est un angle en degré, r une dimension}
\bottomrule
\end{tabular}

\medskip
\noindent\emph{Les arguments obligatoires de cette macro sont  deux dimensions exprimées avec des décimaux, dans le premier cas ce sont deux mesures de longueur, dans le second ce sont une mesure de longueur et la mesure d'un angle en degré}

\medskip
\begin{tabular}{lll}
\toprule
options             & défaut & définition   \\ 
\midrule
\TOline{shift} {(0,0)} {espacement entre deux valeurs}
\TOline{label} {no default} {permet de placer un label à une distance prédéfinie}
 \bottomrule
\end{tabular}

\medskip
\noindent\emph{Toutes les options de \TIKZ\ que l'on peut appliquer à \tkzname{coordinate}, sont applicables (enfin je l'espère!)}
\end{NewMacroBox}

\subsubsection{Utilisation de \tkzname{shift} et \tkzname{label} }
\tkzname{shift} permet de placer les points par rapport à un autre.  Je n'aime guère utiliser l'option \tkzname{label} mais en tout cas c'est possible. Attention à l'utilisation de \tkzname{shift}, dans certains comme celui ci-dessous, une transformation générale de la figure n'est pas possible. Voir la méthode 

\begin{tkzexample}[latex=7cm]
\begin{tikzpicture}
 \tkzDefPoint[label=-60:$A_n$](2,3){A}
 \tkzDefPoint[shift={(2,3)},%
     label=above left:$B_n$](31:3){B}  
 \tkzDefPoint[shift={(2,3)},%
     label=above right:$C_n$](158:3){C}
 \tkzDrawSegments[color=red,%
          line width=1pt](A,B A,C) 
 \tkzDrawPoints[color=red](A,B,C)
\end{tikzpicture}
\end{tkzexample}

Préférable  pour effectuer une rotation, est d'utiliser un environnement \tkzNameEnv{scope}. 
                    
\begin{tkzexample}[latex=5cm] 
\begin{tikzpicture}[rotate=90] 
 \tkzDefPoint[label=right:$A_n$](2,3){A} 
 \begin{scope}[shift={(A)}]
  \tkzDefPoint[label= right:$B_n$](31:3){B} 
  \tkzDefPoint[label= right:$C_n$](158:3){C} 
 \end{scope}
  \tkzDrawSegments[color=red,%
           line width=1pt](A,B A,C) 
  \tkzDrawPoints[color=red](A,B,C)
 \end{tikzpicture}
\end{tkzexample}

\subsubsection{Formules et coordonnées}
Il faut ici respecter la syntaxe de \tkzNamePack{fp.sty}. Il est toujours possible de passer par \tkzNamePack{pgfmath.sty} mais dans ce cas, il faut calculer les coordonnées avant d'utiliser la macro \tkzcname{tkzDefPoint}.

\begin{tkzexample}[latex=6cm]
\begin{tikzpicture}[scale=1]
  \tkzInit[xmax=6,ymax=6]
  \tkzGrid
  \tkzSetUpPoint[shape = circle,color = red,%
                 size = 8,fill = red!30]
  \tkzDefPoint(-1+1,-1+4){O}
  \tkzDefPoint({3*ln(exp(1))},{exp(1)}){A}
  \tkzDefPoint({4*sin(FPpi/6)},{4*cos(FPpi/6)}){B}
  \tkzDefPoint({4*sin(FPpi/3)},{4*cos(FPpi/3)}){B'}
  \tkzDefPoint(30:5){C}
  \tkzDefPoint[shift={(1,3)}](45:4){A'} 
  \begin{scope}[shift=(A)]
      \tkzDefPoint(30:3){C'} 
  \end{scope}
  \tkzDrawPoints[color=blue](O,B,C) 
  \tkzDrawPoints[color=red,%
                 shape=cross out](B',A,A',C') 
  \tkzLabelPoints(A,O,B,B',A',C,C') 
\end{tikzpicture}
\end{tkzexample}


\subsubsection{Scope et \tkzcname{tkzDefPoint} }
On peut tout d'abord utiliser l'environnement \tkzNameEnv{scope} de \TIKZ\
Dans l'exemple suivant, nous avons un moyen de définir un triangle isocèle.

\begin{tkzexample}[latex=7cm]
\begin{tikzpicture}[scale=1]
  \tkzSetUpLine[color=blue!60]
 \begin{scope}[rotate=30]
  \tkzDefPoint(2,3){A}
  \begin{scope}[shift=(A)]
     \tkzDefPoint(90:5){B}
     \tkzDefPoint(30:5){C}
  \end{scope}
 \end{scope}
 \tkzDrawPolygon(A,B,C)
\tkzLabelPoints[above](B,C)
\tkzLabelPoints[below](A) 
\end{tikzpicture}
\end{tkzexample}
%<--------------------------------------------------------------------------->
\subsection{Définition de points multiples en coordonnées cartésiennes : \tkzcname{tkzDefPoints}} 
 
\begin{NewMacroBox}{tkzDefPoints}{\oarg{local options}\marg{$x_1/y_1/n_1,x_2/y_2/n_2$, ...}}
$x_1$ et $y_1$ sont les coordonnées d'un point référencé $n_1$ 

\begin{tabular}{lll}
\toprule
arguments &  exemple  &   \\ 
\midrule
\TAline{$x_i/y_i/n_i$}{\tkzcname{tkzDefPoints\{0/0/O,2/2/A\}}}{}
\end{tabular}
\end{NewMacroBox}

\begin{tkzexample}[latex=6cm,small]
\begin{tikzpicture}[scale=1]
 \tkzDefPoints{0/0/A,
               2/0/B,
               2/2/C,
               0/2/D}
 \tkzDrawSegments(D,A A,B B,C C,D)
 \tkzDrawPoints(A,B,C,D) 
\end{tikzpicture}
\end{tkzexample}   

\newpage 
%<--------------------------------------------------------------------------->
\subsection{Point relativement à un autre : \tkzcname{tkzDefShiftPoint}}
\begin{NewMacroBox}{tkzDefShiftPoint}{\oarg{Point}\parg{x,y}\marg{name} ou \parg{a:r}\marg{name}}
\begin{tabular}{lll}
arguments &  défaut  & définition \\ 
\midrule
\TAline{(x,y)}{no default}{x et y sont deux dimensions, par défaut en cm.}
\TAline{(a:r)}{no default}{a est un angle en degré, r une dimension}
\TOline{point} {no default} {\tkzcname{tkzDefShiftPoint}[A](0:4)\{B\}} 
\bottomrule
\end{tabular}

\emph{Pas d'option. Le nom du point est obligatoire.}
\end{NewMacroBox}

\subsubsection{Exemple avec  \tkzcname{tkzDefShiftPoint}}
Cette macro permet de placer un point relativement à un autre. Cela revient à une translation. Voici comment construire un triangle isocèle de sommet principal A et d'angle au sommet de $30$ degrés. 

\begin{tkzexample}[vbox]
\begin{tikzpicture}[scale=2,rotate=-30]
 \tkzDefPoint(2,3){A}
 \tkzDefShiftPoint[A](0:4){B}
 \tkzDefShiftPoint[A](30:4){C} 
 \tkzDrawSegments(A,B B,C C,A)
 \tkzMarkSegments[mark=|,color=red](A,B A,C)
 \tkzDrawPoints(A,B,C) 
 \tkzLabelPoints(B,C) \tkzLabelPoints[above left](A)
\end{tikzpicture}
\end{tkzexample}

\newpage
\subsection{Point relativement à un autre : \tkzcname{tkzDefShiftPointCoord}}

\begin{NewMacroBox}{tkzDefShiftPointCoord}{\oarg{a,b}\parg{x,y}\marg{name} ou \parg{a:r}\marg{name}}
\emph{Il s'agit d'effectuer une translation de vecteur $(a,b)$ au point défini par rapport à l'oigine.}

\medskip
\begin{tabular}{lll}
\toprule
arguments &  défaut  & définition \\ 
\midrule
\TAline{(x,y)}{no default}{x et y sont deux dimensions, par défaut en cm.}
\TAline{(a:r)}{no default}{a est un angle en degré, r une dimension}
\end{tabular}

\medskip
\begin{tabular}{lll}
\toprule
options             & défaut & exemple   \\ 
\midrule
\TOline{a,b} {no default} {\tkzcname{tkzDefShiftPointCoord}[2,3](0:4)\{B\}}
\end{tabular}
\emph{L'option est obligatoire}  
\end{NewMacroBox}

  
\subsubsection{Triangle équilatéral avec \tkzcname{tkzDefShiftPointCoord}}
Voyons comment obtenir un triangle équilatéral (il y a beaucoup plus simple)

\begin{tkzexample}[latex=7cm]
\begin{tikzpicture}[scale=1]
 \tkzDefPoint(2,3){A}
 \tkzDefShiftPointCoord[2,3](30:4){B}
 \tkzDefShiftPointCoord[2,3](-30:4){C} 
 \tkzDrawPolygon(A,B,C) 
\end{tikzpicture}
\end{tkzexample} 

\subsubsection{Triangle isocèle avec \tkzcname{tkzDefShiftPointCoord}}
Voyons comment obtenir un triangle isocèle dont l'angle principal est de 30 degrés. La rotation est possible. $AB=AC=5$ et $\widehat{BAC}$

\begin{tkzexample}[latex=7cm]
\begin{tikzpicture}[rotate=15]
 \tkzDefPoint(2,3){A}
 \tkzDefShiftPointCoord[2,3](15:5){B}
 \tkzDefShiftPointCoord[2,3](-15:5){C} 
 \tkzDrawSegments(A,B B,C C,A) 
 \tkzDrawPoints(A,B,C)
 \tkzLabelPoints(B,C)
 \tkzLabelPoint[left](A){$A$}
\end{tikzpicture}
\end{tkzexample}
%<--------------------------------------------------------------------------->
%<--------------------------------------------------------------------------->
%<--------------------------------------------------------------------------->

\clearpage \newpage
\subsection{Tracer des points \tkzcname{tkzDrawPoint}} \hypertarget{tdrp}{}

\begin{NewMacroBox}{tkzDrawPoint}{\oarg{local options}\parg{name}}
\begin{tabular}{lll}
arguments &  défaut  & définition                 \\ 
\midrule
\TAline{name of point} {no default}  {Un seul nom de point est accepté}
\bottomrule
\end{tabular}

\medskip
\noindent\emph{L'argument est  obligatoire. Le disque prend la couleur du cercle mais 50\% plus clair. Il est possible de tout modifier. Le point est un node et donc il est invariant si le dessin est modifié par une mise à l'échelle.}

\medskip
\begin{tabular}{lll}
\toprule
options             & défaut & définition  \\ 
\midrule
\TOline{shape}  {circle}{Possible \tkzname{cross} ou \tkzname{cross out}} 
\TOline{size}  {6}{$6 \times$ \tkzcname{pgflinewidth}}
\TOline{color}  {black}{la couleur par défaut peut être changée}
\bottomrule
\end{tabular}

\medskip
\noindent\emph{On peut créer d'autres formes comme \tkzname{cross}}
\end{NewMacroBox}

\subsubsection{Exemple de tracés de points}
Il faut remarquer que \tkzname{scale} ne touche pas à la forme des points. Ce qui est normal.  La plupart du temps, on se contente d'une seule forme de points que l'on pourra définir dès le début, soit avec une macro, soit en modifiant un fichier de configuration. 


\begin{tkzexample}[latex=5cm]
  \begin{tikzpicture}[scale=.5]
   \tkzDefPoint(1,3){A}
   \tkzDefPoint(4,1){B}
   \tkzDefPoint(0,0){O}
   \tkzDrawPoint[shape=cross out,size=12,color=red](A)
   \tkzDrawPoint[shape=cross,size=12,color=blue](B)
   \tkzDrawPoint[size=12,color=green](O)
  \end{tikzpicture}
\end{tkzexample}

Il est possible de tracer plusieurs points en une seule fois mais cette macro est un peu plus lente que la précédente. De plus on doit se contenter des mêmes options pour tous les points.                               

\hypertarget{tdrps}{}
\begin{NewMacroBox}{tkzDrawPoints}{\oarg{local options}\parg{liste}}
\begin{tabular}{lll}
arguments &  défaut  & définition \\ 
\midrule
\TAline{liste de  points}{no default}{exemple \tkzcname{tkzDrawPoints(A,B,C)}}
\bottomrule
\end{tabular}

\medskip
\emph{Attention au « s » final, un oubli entraîne des erreurs en cascade si vous tentez de tracer des points multiples. Les options sont les mêmes que pour la macro précédente. }
\end{NewMacroBox}

\subsubsection{Exemple avec \tkzcname{tkzDefPoint} et \tkzcname{tkzDrawPoints} } 

\begin{tkzexample}[latex=5cm]
  \begin{tikzpicture}[scale=.5]
   \tkzDefPoint(1,3){A}
   \tkzDefPoint(4,1){B}
   \tkzDefPoint(0,0){O}
   \tkzDrawPoints[size=8,color=red](A,B,C)
  \end{tikzpicture}
\end{tkzexample} 

\begin{tkzexample}[latex=7cm]
\begin{tikzpicture}[scale=.5]
 \tkzDefPoint(2,3){A}  \tkzDefPoint(5,-1){B}  
 \tkzDefPoint[label=below:$\mathcal{C}$,
               shift={(2,3)}](-30:5.5){E}
 \begin{scope}[shift=(A)]
    \tkzDefPoint(30:5){C}
 \end{scope}
 \tkzCalcLength[cm](A,B)\tkzGetLength{rAB}
 \tkzDrawCircle[R](A,\rAB cm)
 \tkzDrawSegment(A,B)
 \tkzDrawPoints(A,B,C) 
 \tkzLabelPoints(B,C)
 \tkzLabelPoints[above](A)
\end{tikzpicture}
\end{tkzexample}  
%<--------------------------------------------------------------------------->
%<--------------------------------------------------------------------------->
%<--------------------------------------------------------------------------->

\clearpage  \newpage
\subsection{Ajouter des labels aux  points \tkzcname{tkzLabelPoint}} 
\hypertarget{tlp}{}

 \begin{NewMacroBox}{tkzLabelPoint}{\oarg{local options}\parg{point}\marg{label}}
\begin{tabular}{lll}
arguments &  exemple  &                  \\ 
\midrule
\TAline{point}{\tkzcname{tkzLabelPoint(A)\{$A_1$\}}}{}
\bottomrule
\end{tabular}

\medskip
\emph{En option, on peut utiliser tous les styles de \TIKZ\ , en particulier le placement avec \tkzname{above}, \tkzname{right}, \dots}

 \end{NewMacroBox}

\subsubsection{Exemple avec \tkzcname{tkzLabelPoint}} 

\begin{tkzexample}[latex=6cm]  
\begin{tikzpicture}
  \tkzDefPoint(0,0){A}
  \tkzDefPoint(4,0){B}
  \tkzDefPoint(0,3){C}
  \tkzDrawSegments(A,B B,C C,A)
  % \tkzDrawPolygon with 
  % \usetkzobj{polygons}
  \tkzDrawPoints(A,B,C)
  \tkzLabelPoint[left,red](A){$A$}
  \tkzLabelPoint[right,blue](B){$B$}
  \tkzLabelPoint[above,purple](C){$C$}  
\end{tikzpicture} 
\end{tkzexample} 

\subsubsection{label et référence}
 La référence d'un point est l'objet qui permet d'utiliser le point, le label est le nom du point qui sera affiché.
 
\begin{tkzexample}[latex=8cm]
 \begin{tikzpicture}
    \tkzInit[xmax=1,xstep=0.15,ymax=.5]
    \tkzAxeX \tkzDrawY
    \tkzDefPoint(0.22,0.25){A} 
    \tkzDrawPoint(A)
    \tkzLabelPoint[above](A){$A_1$}  
  \end{tikzpicture}
 \end{tkzexample}


\newpage  

Il est possible de placer plusieurs labels rapidement quand les références des points sont identiques aux labels et quand les labels sont placés de la même manière par rapport aux points. Par défaut, c'est \tkzname{below right} qui a été choisi.
\hypertarget{tlps}{}  

\begin{NewMacroBox}{tkzLabelPoints}{\oarg{local options}\parg{$A_1,A_2,...$}}
\begin{tabular}{lll}
arguments &  exemple  & résultat                 \\ 
\midrule
\TAline{list of points}{\tkzcname{tkzLabelPoint(A,B,C)}}{Affichage de A, B et C}
\bottomrule
\end{tabular}

\medskip
 \emph{Cette macro diminue le nombre de lignes de codes mais il n'est pas évident que tous les points aient besoin du même positionnement des labels.}
\end{NewMacroBox}

\subsubsection{Exemple avec \tkzcname{tkzLabelPoints}}   
\begin{tkzexample}[latex = 6cm]  
\begin{tikzpicture}
  \tkzDefPoint(2,3){A}
  \tkzDefShiftPoint[A](30:2){B}
  \tkzDefShiftPoint[A](30:5){C}
  \tkzDrawPoints(A,B,C)
  \tkzLabelPoints(A,B,C) 
\end{tikzpicture} 
\end{tkzexample}
  
\subsection{Style des points  avec \tkzcname{tkzSetUpPoint}}

\begin{NewMacroBox}{tkzSetUpPoint}{\oarg{local options}}
\begin{tabular}{lll}
options &  défaut  & définition                 \\ 
\midrule
\TOline{liste}{no default}{exemple \tkzcname{tkzLabelPoint(A,B,C)}}
\bottomrule
\end{tabular}

\end{NewMacroBox}

Il s'agit d'une macro permettant de choisir un \hypertarget{setupoint}{style} pour les points. La macro \tkzcname{tkzDrawSegments}  est décrite \hyperlink{segs}{ici}.

\begin{tkzexample}[latex=6cm]
\begin{tikzpicture}
  \tkzInit[ymin=-0.5,ymax=3,xmin=-0.5,xmax=7]
  \tkzDefPoint(0,0){A}
  \tkzDefPoint(02.25,04.25){B}
  \tkzDefPoint(4,0){C}
  \tkzDefPoint(3,2){D}
  \tkzDrawSegments(A,B A,C A,D)  
  \tkzSetUpPoint[shape=cross out,size=10,color=red]
  \tkzDrawPoints(A,B,C,D)
  \tkzLabelPoints(A,B,C,D) 
\end{tikzpicture}
\end{tkzexample}



\section{Points particuliers}
L'introduction des points a été réalisée dans \tkzname{tkz-base}. La macro la plus importante étant \tkzcname{tkzDefPoint}. \tkzcname{tkzDrawPoint} permet de tracer les points, quant à \tkzcname{tkzLabelPoint}, elle permet d'afficher un label, lié au point. Voici quelques points particuliers.

%<--------------------------------------------------------------------------->
\subsection{Milieu d'un segment \tkzcname{tkzDefMidPoint}}
Il s'agit de déterminer le milieu d'un segment.
 
\begin{NewMacroBox}{tkzDefMidPoint}{\parg{pt1,pt2}}
Le résultat est dans \tkzname{tkzPointResult}. On peut le récupérer avec \tkzcname{tkzGetPoint}. Soit vous ne voulez pas conserver ce point et dans ce cas, vous pouvez immédiatement travailler avec \tkzname{tkzPointResult}, soit vous aurez besoin untéreurement
 
 \medskip
\begin{tabular}{lll}
\toprule
arguments &  défaut  & définition \\ 
\midrule
\TAline{(pt1,pt2)}{no default}{pt1 et pt2 sont deux points}
\end{tabular}
\end{NewMacroBox}

\subsubsection{Utilisation de \tkzcname{tkzDefMidPoint}}
Revoir l'utilisation de  \tkzcname{tkzDefPoint} dans \NamePack{tkz-base}.
\begin{tkzexample}[latex=7cm]
\begin{tikzpicture}[scale=1]
 \tkzDefPoint(2,3){A}
 \tkzDefPoint(4,0){B} 
 \tkzDefMidPoint(A,B) \tkzGetPoint{C}
 \tkzDrawSegment(A,B)
 \tkzDrawPoints(A,B,C)
 \tkzLabelPoints[right](A,B,C) 
\end{tikzpicture}
\end{tkzexample}

\subsection{Coordonnées barycentriques \tkzcname{tkzDefBarycentricPoint}}

$pt_1$, $pt_2$, \dots, $pt_n$ étant $n$ points, ils définissent $n$ vecteurs $\overrightarrow{v_1}$, $\overrightarrow{v_2}$, \dots, $\overrightarrow{v_n}$ avec comme extrémité commune l'origine du repère. $\alpha_1$, $\alpha_2$,
\dots, $\alpha_n$ étant $n$ nombres, le vecteur obtenu par :
\begin{align*}
  \frac{\alpha_1 \overrightarrow{v_1} + \alpha_2 \overrightarrow{v_2} + \cdots + \alpha_n \overrightarrow{v_n}}{\alpha_1
    + \alpha_2 + \cdots + \alpha_n}
\end{align*}
définit un point unique.

\begin{NewMacroBox}{tkzDefBarycentricPoint}{\parg{pt1=nb1,pt2=nb2,\ldots}}
\begin{tabular}{lll}
arguments &  défaut  & définition \\ 
\midrule
\TAline{(pt1=$\alpha_1$,pt2=$\alpha_2$,\ldots)}{no default}{Chaque point a une pondération} 
\bottomrule
\end{tabular}

\medskip
\emph{Il faut au moins deux points.}  
\end{NewMacroBox}

  
\subsubsection{Utilisation de \tkzcname{tkzDefBarycentricPoint} avec deux points}
Nous obtenons dans l'exemple suivant le barycentre des points A et B affectés des coefficients 1 et 2, autrement dit:
\[
  \overrightarrow{AI}= \frac{2}{3}\overrightarrow{AB}
\]   

\begin{tkzexample}[latex=7cm]
\begin{tikzpicture}
  \tkzDefPoint(2,3){A}
  \tkzDefShiftPointCoord[2,3](30:4){B}
  \tkzDefBarycentricPoint(A=1,B=2)
  \tkzGetPoint{I}
  \tkzDrawPoints(A,B,I)
  \tkzDrawLine(A,B)
   \tkzLabelPoints(A,B,I)  
\end{tikzpicture}  
\end{tkzexample} 

\subsubsection{Utilisation de \tkzcname{tkzDefBarycentricPoint} avec trois points}

Cette fois M est simplement le centre de gravité du triangle. Pour des raisons de simplification et d'homogénéité, il existe aussi \tkzcname{tkzCentroid}
\begin{tkzexample}[latex=7cm]   
\begin{tikzpicture}[scale=.8] 
 \tkzInit[xmax=6,ymax=6]
 \tkzDefPoint(2,1){A} 
 \tkzDefPoint(5,3){B} 
 \tkzDefPoint(0,6){C} 
 \tkzDrawPolygon(A,B,C)
 \tkzDefBarycentricPoint(A=1,B=1,C=1)
 \tkzGetPoint{M}
 \tkzDrawLines[add=0 and 1](A,M B,M C,M) 
 \tkzDrawPoints(A,B,C,M) 
 \tkzLabelPoints(A,B,C,M)
 \tkzDefMidPoint(A,B)  \tkzGetPoint{C'} 
 \tkzDefMidPoint(A,C)  \tkzGetPoint{B'} 
 \tkzDefMidPoint(C,B)  \tkzGetPoint{A'}
 \tkzDrawPoints(A',B',C') 
 \tkzLabelPoints(A',B',C') 
\end{tikzpicture}
\end{tkzexample}

\clearpage \newpage
\subsection{\tkzcname{tkzCentroid}}
 On obtient le centre de gravité du triangle. Le résultat est bien sûr dans \tkzname{tkzPointResult}. On peut le récupérer avec \tkzcname{tkzGetPoint}.
 
\begin{NewMacroBox}{tkzCentroid}{\parg{pt1,pt2,pt3}}
\begin{tabular}{lll}
arguments &  défaut  & définition \\ 
\midrule
\TAline{(pt1,pt2,pt3)}{no default}{liste non ordonnée de trois points}
\bottomrule
\end{tabular}
\end{NewMacroBox}

  
\subsubsection{Utilisation de \tkzcname{tkzCentroid}}

\begin{tkzexample}[latex=5cm]
 \begin{tikzpicture}[scale=.75]
   \tkzDefPoint(-1,1){A}
   \tkzDefPoint(5,1){B}
   \tkzDefEquilateral(A,B)\tkzGetPoint{C}
   \tkzDrawPolygon[color=Maroon](A,B,C)
   \tkzCentroid(A,B,C)\tkzGetPoint{G}
   \tkzDrawPoint(G)
   \tkzDrawLines[add = 0 and 2/3](A,G B,G C,G)
 \end{tikzpicture}
\end{tkzexample}         

\subsection{\tkzcname{tkzCircumCenter}}
 On obtient le centre du cercle circonscrit à un triangle. Le résultat est bien sûr dans \tkzname{tkzPointResult}. On peut le récupérer avec \tkzcname{tkzGetPoint}.
 
\begin{NewMacroBox}{tkzCircumCenter}{\parg{pt1,pt2,pt3}}
\begin{tabular}{lll}
arguments &  défaut  & définition \\ 
\midrule
\TAline{(pt1,pt2,pt3)}{no default}{liste non ordonnée de trois points}
\end{tabular}
\end{NewMacroBox}

\subsubsection{Utilisation de \tkzcname{tkzCircumCenter}}

\begin{tkzexample}[latex=6cm]
 \begin{tikzpicture}
  \tkzDefPoint(0,1){A} \tkzDefPoint(3,2){B}
  \tkzDefPoint(1,4){C}
  \tkzDrawPolygon[color=Maroon](A,B,C)
  \tkzCircumCenter(A,B,C)\tkzGetPoint{G}
  \tkzDrawPoint(G)
  \tkzDrawCircle(G,A)
 \end{tikzpicture}
\end{tkzexample}  


\subsection{\tkzcname{tkzInCenter}}
 On obtient le centre du cercle inscrit du triangle. Le résultat est bien sûr dans \tkzname{tkzPointResult}. On peut le récupérer avec \tkzcname{tkzGetPoint}.
 
\begin{NewMacroBox}{tkzInCenter}{\parg{pt1,pt2,pt3}}
\begin{tabular}{lll}
arguments &  défaut  & définition \\ 
\midrule
\TAline{(pt1,pt2,pt3)}{no default}{liste non ordonnée de trois points}
\bottomrule
\end{tabular}
\end{NewMacroBox}

  
\subsubsection{Utilisation de \tkzcname{tkzInCenter} avec trois points} 
Les trois points sont donnés dans le sens direct 
\begin{tkzexample}[latex=6cm]
\begin{tikzpicture}
  \tkzInit[xmax=6,ymax=6]
  \tkzClip
  \tkzDefPoint(0,0){A}
  \tkzDefPoint(5,1){B} 
  \tkzDefPoint(1,4){C} 
  \tkzDrawPolygon[color=Maroon](A,B,C)
  \tkzInCenter(A,B,C)\tkzGetPoint{G}
  \tkzDrawPoint(G)
  \tkzDrawLines[add = 0 and 2/3](A,G B,G C,G)
\end{tikzpicture} 
\end{tkzexample}


\endinput


 
%!TEX root = /Users/ego/Boulot/TKZ/tkz-euclide/doc_fr/TKZdoc-euclide-main.tex


\section{Définition aléatoire de points}
Il y a pour le moment quatre possibilités :
\begin{enumerate}
  \item point dans un rectangle,
  \item sur un segment,
  \item sur une droite,
  \item sur un cercle.
\end{enumerate}

\begin{NewMacroBox}{tkzGetRandPointOn}{\oarg{local options}\marg{name} }


\medskip
\begin{tabular}{lll}
\toprule
options     &     & définition                         \\ 
\midrule
\TOline{rectangle =  \#1 and \#2}{}{\#1 et \#2 sont des noms de points}
\TOline{segment =  \#1--\#2}{}{\#1 et \#2 sont des noms de points}
\TOline{line =  \#1--\#2}{}{\#1 et \#2 sont des noms de points}
\TOline{circle = center \#1 radius \#1 }{}{\#1 est un point et \#1 une mesure}
 \bottomrule
\end{tabular}

\medskip
\noindent\emph{Cette macro est assez simple à utiliser, voyez les exemples.}
\end{NewMacroBox} 

\subsection{Point aléatoire dans un rectangle} 

\begin{center}
\begin{tkzexample}[vbox]
\begin{tikzpicture}
  \tkzInit[xmax=5,ymax=5]  \tkzGrid   
  \tkzDefPoint(0,0){A}  \tkzDefPoint(2,2){B}
  \tkzDefPoint(5,5){C}
  \tkzGetRandPointOn[rectangle = A and B]{a}
  \tkzGetRandPointOn[rectangle = B and C]{d}
  \tkzDrawLine(a,d)
  \tkzDrawPoints(A,B,C,a,d) 
  \tkzLabelPoints(A,B,C,a,d)  
\end{tikzpicture} 
\end{tkzexample} 
\end{center}


\subsection{Point aléatoire sur un segment}  
\begin{tkzexample}[latex=6cm] 
\begin{tikzpicture}  
  \tkzInit[xmax=5,ymax=5] \tkzGrid   
  \tkzDefPoint(0,0){A} \tkzDefPoint(2,2){B}
  \tkzDefPoint(3,3){C} \tkzDefPoint(5,5){D}
  \tkzGetRandPointOn[segment = A--B]{a}
  \tkzGetRandPointOn[segment = C--D]{d}
  \tkzDrawPoints(A,B,C,D,a,d) 
  \tkzLabelPoints(A,B,C,D,a,d)
\end{tikzpicture} 
\end{tkzexample}

\subsection{Point aléatoire sur une droite}  
\begin{tkzexample}[latex=6cm] 
\begin{tikzpicture} 
  \tkzInit[xmax=5,ymax=5] \tkzGrid   
  \tkzDefPoint(0,0){A}  \tkzDefPoint(2,2){B}
  \tkzDefPoint(3,3){C}  \tkzDefPoint(5,5){D}
  \tkzGetRandPointOn[line = A--B]{a}
  \tkzGetRandPointOn[line = C--D]{d}
  \tkzDrawPoints(A,B,C,D,a,d) 
  \tkzLabelPoints(A,B,C,D,a,d)   
\end{tikzpicture}    
\end{tkzexample}

\subsection{Point aléatoire sur un cercle}  

\begin{tkzexample}[latex=5cm] 
\begin{tikzpicture} 
  \tkzInit[xmax=5,ymax=5]  \tkzGrid   
  \tkzDefPoint(3,2){A}  \tkzDefPoint(1,1){B}
  \tkzCalcLength[cm](A,B) \tkzGetLength{rAB}
  \tkzDrawCircle[R](A,\rAB cm) 
  \tkzGetRandPointOn[circle = center A radius \rAB cm]{a}
  \tkzDrawSegment(A,a)
  \tkzDrawPoints(A,B,a) 
  \tkzLabelPoints(A,B,a)  
\end{tikzpicture}
\end{tkzexample}


\newpage
\subsection{Milieu d'un segment au compas}  
 Pour terminer cette section, voici un exemple plus complexe. Il s'agit de déterminer le milieu d'un segment, uniquement avec un compas. 
 
\begin{center}
\begin{tkzexample}[vbox]
\begin{tikzpicture}[scale=.75]
  \tkzDefPoint(0,0){A}  
  \tkzGetRandPointOn[circle= center A radius 4cm]{B}
  \tkzDrawPoints(A,B)
  \tkzDefPointBy[rotation= center A angle 180](B) 
  \tkzGetPoint{C}
  \tkzInterCC[R](A,4 cm)(B,4 cm) 
  \tkzGetPoints{I}{I'}
  \tkzInterCC[R](A,4 cm)(I,4 cm) 
  \tkzGetPoints{J}{B}
  \tkzInterCC(B,A)(C,B) 
  \tkzGetPoints{D}{E}
  \tkzInterCC(D,B)(E,B) 
  \tkzGetPoints{M}{M'} 
  \tikzset{arc/.style={color=brown,style=dashed,delta=10}} 
  \tkzDrawArc[arc](C,D)(E) 
  \tkzDrawArc[arc](B,E)(D)
  \tkzDrawCircle[color=brown,line width=.2pt](A,B) 
  \tkzDrawArc[arc](D,B)(M) 
  \tkzDrawArc[arc](E,M)(B)
  \tkzCompasss[color=red,style=solid](B,I I,J J,C) 
  \tkzDrawPoints(B,C,D,E,M)    
 \end{tikzpicture}  
 \end{tkzexample}
\end{center}

\endinput
%!TEX root = /Users/ego/Boulot/TKZ/tkz-euclide/doc_fr/TKZdoc-euclide-main.tex


\section{Définition de points par transformation; \tkzcname{tkzDefPointBy} }
Ces transformations sont au nombre de sept :

\begin{enumerate}
   \item la translation;
   \item l'homothetie;
   \item la réflexion  ou symétrie orthogonale;
   \item la symétrie centrale;
   \item la projection orthogonale;
   \item la rotation;
   \item la rotation en radian;
   \item l'inversion par rapport à un cercle
\end{enumerate}

Le choix des transformations se fait par l'intermédiaire des options. Il y a deux macros l'une pour la transformation d'un unique point \tkzcname{tkzDefPointBy} et l'autre pour la transformation d'une liste de points \tkzcname{tkzDefPointsBy}. Dans le second cas, il faut donner en argument, les noms des images ou bien encore indiquer que le nom des images est formé à partir du nom des antécédents. Par défaut l'image de $A$ est $A'$. Par exemple, on écrira~:
\begin{tkzltxexample}[]
\tkzDefPointBy[translation= from A to A'](B) le résultat est dans tkzPointResult}
\tkzDefPointsBy[translation= from A to A'](B,C){} les images sont B' et C'
\tkzDefPointsBy[translation= from A to A'](B,C){D,E} les images sont D et E
\tkzDefPointsBy[translation= from A to A'](B) l'image est B'
\end{tkzltxexample}

La variante sans (s), évite l'usage d'une boucle et d'un test et est donc plus efficace.
 
\bigskip
\begin{NewMacroBox}{tkzDefPointBy}{\oarg{local options}\parg{pt}}
\emph{L'argument est un simple point existant et son image est stockée dans \tkzname{tkzPointResult}. Soit la création est une étape intermédiaire et vous n'avez pas besoin de conserver ce point alors tant qu'aucune macro ne modifie  l'attribution de \tkzname{tkzPointResult}, vous pouvez utiliser ce nom pour faire référence au point obtenu. Si vous voulez conserver ce point alors la macro  \tkzcname{tkzGetPoint\{M\}} permet d'attribuer le nom \tkzname{M} au point.}

\medskip
\begin{tabular}{lll}
\toprule
arguments &  définition  &   exemples               \\ 
\midrule
\TAline{pt}   {nom d'un point existant}   {$(A)$}
\bottomrule
\end{tabular}


\medskip
\begin{tabular}{lll}
options     &     & exemples                         \\ 
\midrule
\TOline{translation}{= from \#1 to \#2}{[translation=from A to B](E)}
\TOline{homothety}  {= center \#1 ratio \#2}{[homothety=center A ratio .5](E)}
\TOline{reflection} {= over \#1--\#2}{[reflection=over A--B](E)}
\TOline{symmetry }  {= center \#1}{[symmetry=center A](E)}
\TOline{projection }{= onto \#1--\#2}{[projection=onto A--B](E)}
\TOline{rotation }  {= center \#1 angle \#2}{[rotation=center O angle 30](E)}
\TOline{rotation in rad}{= center \#1 angle \#2}{rotation=center O angle pi/3} 
\TOline{inversion}{= center \#1 through \#2}{[inversion =center O through A](E)} 
\bottomrule
\end{tabular}

\medskip
\noindent\emph{ L'image est seulement définie et non tracée.}
\end{NewMacroBox} 

\newpage 
\subsection{La réflexion ou symétrie orthogonale } 

\subsubsection{Exemple de réflexion} 
\begin{center}
\begin{tkzexample}[vbox]
\begin{tikzpicture}[scale=1]
 \tkzInit[ymin=-4,ymax=6,xmin=-7,xmax=3]
 \tkzClip
 \tkzDefPoints{1.5/-1.5/C,-4.5/2/D}    
 \tkzDefPoint(-4,-2){O} 
 \tkzDefPoint(-2,-2){A}
 \foreach \i in {0,1,...,4}{%
 \pgfmathparse{0+\i * 72}
 \tkzDefPointBy[rotation=center O angle \pgfmathresult](A) \tkzGetPoint{A\i} 
 \tkzDefPointBy[reflection = over C--D](A\i) \tkzGetPoint{A\i'}}
 \tkzDrawPolygon(A0, A2, A4, A1, A3)    
 \tkzDrawPolygon(A0', A2', A4', A1', A3')
 \tkzDrawLine[add= .5 and .5](C,D)
\end{tikzpicture}
\end{tkzexample}
\end{center}
 
\newpage 
\subsection{L'homothétie}  
\subsubsection{Exemple d'homothétie et de projection}

\begin{center}
\begin{tkzexample}[vbox] 
\begin{tikzpicture}[scale=1.25] 
  \tkzInit   \tkzClip 
  \tkzDefPoint(0,1){A}   \tkzDefPoint(6,3){B}   \tkzDefPoint(3,6){C} 
  \tkzDrawLines[add= 0 and .3](A,B A,C) 
  \tkzDefLine[bisector](B,A,C)                     \tkzGetPoint{a} 
  \tkzDrawLine[add=0 and 0,color=magenta!50 ](A,a) 
  \tkzDefPointBy[homothety=center A ratio .5](a)   \tkzGetPoint{a'} 
  \tkzDefPointBy[projection = onto A--B](a')       \tkzGetPoint{k}   
  \tkzDrawSegment[style=dashed](a',k) 
  \tkzShowLine[bisector,size=2,gap=3](B,A,C) 
  \tkzDrawCircle(a',k)  
\end{tikzpicture}
\end{tkzexample}  
\end{center}


\newpage  
\subsection{La projection }  
\subsubsection{Exemple de projection}

\begin{center}
\begin{tkzexample}[vbox] 
\begin{tikzpicture}[scale=1.5]  
 \tkzInit[xmin=-3,xmax=5,ymax=4] \tkzClip[space=.5]
 \tkzDefPoint(0,0){A}
 \tkzDefPoint(0,4){B}
 \tkzDrawTriangle[pythagore](B,A) \tkzGetPoint{C}
 \tkzDefLine[bisector](B,C,A) \tkzGetPoint{c}
 \tkzInterLL(C,c)(A,B)        \tkzGetPoint{D}
 \tkzDrawSegment(C,D)
 \tkzDrawCircle(D,A)
 \tkzDefPointBy[projection=onto B--C](D) \tkzGetPoint{G}
 \tkzInterLC(C,D)(D,A) \tkzGetPoints{E}{F}
 \tkzDrawPoints(A,C,F) \tkzLabelPoints(A,C,F)
 \tkzDrawPoints(B,D,E,G)   
 \tkzLabelPoints[above right](B,D,E,G)
 \end{tikzpicture}
 \end{tkzexample} 
\end{center}


\newpage 
\subsection{La symétrie }  
\subsubsection{Exemple de symétrie}
 
 \begin{center}
\begin{tkzexample}[vbox] 
\begin{tikzpicture}[scale=2]
  \tkzDefPoint(0,0){O}
  \tkzDefPoint(2,-1){A}
  \tkzDefPoint(2,2){B}
  \tkzDefPointsBy[symmetry=center O](B,A){}
  \tkzDrawLine(A,A')
  \tkzDrawLine(B,B')
  \tkzMarkAngle[mark=s,arc=lll,size=2 cm,mkcolor=red](A,O,B) 
  \tkzLabelAngle[pos=1,circle,draw,fill=blue!10](A,O,B){$60^{\circ}$}  
\end{tikzpicture}  
\end{tkzexample}
\end{center}

\newpage  
\subsection{La rotation }  
\subsubsection{Exemple de rotation} 

 \begin{center}
\begin{tkzexample}[vbox] 
 \begin{tikzpicture}[scale=1.2,rotate=-90] 
 \tkzInit
 \tkzPoint(0,0){A} \tkzPoint(5,0){B}
 \tkzDrawSegment(A,B)
 \tkzDefPointBy[rotation= center A angle 60](B) 
 \tkzGetPoint{C} 
 \tkzDefPointBy[symmetry= center C](A) 
 \tkzGetPoint{D} 
 \tkzDrawSegment(A,tkzPointResult) 
 \tkzDrawLine(B,D)
 \tkzDrawArc[delta=10](A,B)(C) 
 \tkzDrawArc[delta=10](B,C)(A)
 \tkzDrawArc[delta=10](C,D)(D)  
 \tkzMarkRightAngle(D,B,A)  
\end{tikzpicture}  
\end{tkzexample}  
 \end{center} 

\newpage 
\subsection{La rotation en radian }  
\subsubsection{Exemple de rotation en radian} 
 
\begin{center}
\begin{tkzexample}[vbox]
\begin{tikzpicture} 
  \tkzInit\tkzGrid[sub]
  \tkzPoint[pos=left](1,5){A} 
  \tkzPoint(5,2){B}
  \tkzDrawSegment(A,B)
  \tkzDefPointBy[rotation in rad= center A angle pi/3](B)
  \tkzGetPoint{C}  
  \tkzCompass[color=red](A,C)
  \tkzCompass[color=red](B,C) 
\end{tikzpicture}
\end{tkzexample} 
\end{center}

\newpage 
\subsection{L'inversion par rapport à un cercle }
\subsubsection{Inversion de points}

\begin{center}
\begin{tkzexample}[vbox]  
\begin{tikzpicture}[scale=2]
  \tkzDefPoint(0,0){O}
  \tkzDefPoint(1,0){A}
  \tkzDrawCircle(O,A) 
  \tkzDefPoint(-1.5,-1.5){z1}
  \tkzDefPoint(0.35,0){z2} 
  \tkzDrawPoints[fill=red,color=black,size=8](O,z1,z2)   
  \tkzDefPointBy[inversion = center O through A](z1)
  \tkzGetPoint{Z1} 
  \tkzDefPointBy[inversion = center O through A](z2)
  \tkzGetPoint{Z2} 
  \tkzDrawPoints[fill=red,color=black,size=8](Z1,Z2)    
  \tkzDrawSegments(z1,Z1 z2,Z2)
  \tkzLabelPoints(O,A,z1,z2,Z1,Z2)  
\end{tikzpicture}
\end{tkzexample} 
\end{center}  

\subsubsection{Inversion de point : cercles orthogonaux} 

\begin{center}
\begin{tkzexample}[vbox]
\begin{tikzpicture}[scale=3]
  \tkzDefPoint(0,0){O}
  \tkzDefPoint(1,0){A}
  \tkzDrawCircle(O,A) 
  \tkzDefPoint(0.5,-0.25){z1}
  \tkzDefPoint(-0.5,-0.5){z2}
  \tkzDefPointBy[inversion = center O through A](z1)
  \tkzGetPoint{Z1} 
  \tkzCircumCenter(z1,z2,Z1)\tkzGetPoint{c}
  \tkzDrawCircle(c,Z1)
  \tkzDrawPoints[color=black,fill=red,size=12](O,z1,z2,Z1,O,A) 
\end{tikzpicture}
\end{tkzexample}
\end{center}


\newpage
Il existe une variante de cette macro pour la définition de multiples images

\begin{NewMacroBox}{tkzDefPointsBy}{\oarg{local options}\parg{liste de pts}\marg{liste de pts}}
\begin{tabular}{lll}
\toprule
arguments &  exemples  &                  \\ 
\midrule
\TAline{\parg{liste de pts}\marg{liste de pts}}{(A,B)\{E,F\}}{E est l'image de A et F celle de B.}   \\
\bottomrule
\end{tabular}

\medskip
\emph{Si la liste des images est vide alors le nom de l'image est le nom de l'antécédent auquel on ajoute « ' »}

\medskip
\begin{tabular}{lll}
\toprule
options     &     & exemples                         \\ 
\midrule
\TOline{translation = from \#1 to \#2}{}{[translation=from A to B](E)\{\}}
\TOline{homothety = center \#1 ratio \#2}{}{[homothety=center A ratio .5](E)\{F\}}
\TOline{reflection = over \#1--\#2}{}{[reflection=over A--B](E)\{F\}}
\TOline{symmetry = center \#1}{}{[symmetry=center A](E)\{F\}}
\TOline{projection = onto \#1--\#2}{}{[projection=onto A--B](E)\{F\}}
\TOline{rotation = center \#1 angle \#2}{}{[rotation=center  angle 30](E)\{F\}}
\TOline{rotation in rad = center \#1 angle \#2}{}{par exemple angle pi/3}
\bottomrule
\end{tabular}

\medskip
\noindent\emph{ Les points sont seulement définis et non tracés.}
\end{NewMacroBox}

\subsection{Exemple de translation}

\begin{tkzexample}[vbox,small]
\begin{tikzpicture} 
 \tkzDefPoint(0,0){A}  \tkzDefPoint(5,2){A'}
 \tkzDefPoint(3,0){B}  \tkzDefPoint(1,2){C} 
 \tkzDefPointsBy[translation= from A to A'](B,C){} 
 \tkzDrawPolygon[color=blue](A,B,C)
 \tkzDrawPolygon[color=red](A',B',C')
 \tkzDrawPoints[color=blue](A,B,C)
 \tkzDrawPoints[color=red](A',B',C') 
 \tkzLabelPoints(A,B,A',B')  \tkzLabelPoints[above](C,C')
 \tkzDrawSegments[color = gray,->,style=dashed](A,A' B,B' C,C')   
\end{tikzpicture}
\end{tkzexample}

\newpage
\subsection{Fruit of Life}
\begin{center}
\begin{tkzexample}[vbox] 
\begin{tikzpicture}[scale=.8]
 \tkzDefPoint(0,0){O}  \tkzDefPoint(1.5,0){A}
 \tkzDrawCircle(O,A)
 \foreach \i in {0,...,5}{
  \tkzDefPointBy[rotation  = center O  angle 30+60*\i](A) \tkzGetPoint{a\i}
  \tkzDefPointBy[homothety = center O  ratio 2](a\i) \tkzGetPoint{b\i}
  \tkzDefPointBy[homothety = center O  ratio 3](a\i) \tkzGetPoint{c\i}
  \tkzDefPointBy[homothety = center O  ratio 4](a\i) \tkzGetPoint{d\i}
  \tkzDrawCircle(b\i,a\i) \tkzDrawCircle(d\i,c\i)
  }
\tkzDrawPolygon[color=red!50!Gold,ultra thick](d0,d1,d2,d3,d4,d5) 
\tkzDrawPolygon[color=red!50!Gold,ultra thick](b0,b2,b4)
\tkzDrawSegments[color=red!50!Gold,ultra thick](b0,d5 b0,d0 b0,d1 %
                              b2,d1 b2,d2 b2,d3 b4,d3 b4,d4 b4,d5)
\tkzDrawPoints[color=red!50!Gold,size=20](b0,b2,b4,d0,d1,d2,d3,d4,d5)
\end{tikzpicture}
\end{tkzexample} 
\end{center}

\newpage
\subsection{Flower of Life}

\begin{center}
\begin{tkzexample}[vbox]
\begin{tikzpicture}[scale=.6]
 \tkzSetUpLine[line width=2pt,color=orange!80!black] 
 \tkzSetUpCompass[line width=2pt,color=orange!80!black]
 \tkzDefPoint(0,0){O} \tkzDefPoint(2.25,0){A}
 \tkzDrawCircle(O,A)
 \foreach \i in {0,...,5}{
  \tkzDefPointBy[rotation= center O angle 30+60*\i](A)   \tkzGetPoint{a\i}
  \tkzDefPointBy[rotation= center {a\i} angle  120](O)   \tkzGetPoint{b\i}
  \tkzDefPointBy[rotation= center {a\i} angle  180](O)   \tkzGetPoint{c\i}
  \tkzDefPointBy[rotation= center {c\i} angle  120](a\i) \tkzGetPoint{d\i}
  \tkzDefPointBy[rotation= center {c\i} angle   60](d\i) \tkzGetPoint{f\i}
  \tkzDefPointBy[rotation= center {d\i} angle   60](b\i) \tkzGetPoint{e\i} 
  \tkzDefPointBy[rotation= center {f\i} angle   60](d\i) \tkzGetPoint{g\i} 
  \tkzDefPointBy[rotation= center {d\i} angle   60](e\i) \tkzGetPoint{h\i}
  \tkzDefPointBy[rotation= center {e\i} angle  180](b\i) \tkzGetPoint{k\i}
  
  \tkzDrawCircle(a\i,O) \tkzDrawCircle(b\i,a\i)
  \tkzDrawCircle(c\i,a\i)
  \tkzDrawArc[rotate](f\i,d\i)(-120)
  \tkzDrawArc[rotate](e\i,d\i)(180)
  \tkzDrawArc[rotate](d\i,f\i)(180)
  \tkzDrawArc[rotate](g\i,f\i)(60)
  \tkzDrawArc[rotate](h\i,d\i)(60)
  \tkzDrawArc[rotate](k\i,e\i)(60) }
 \tkzClipCircle(O,f0)
\end{tikzpicture} 
\end{tkzexample}
\end{center}

\clearpage\newpage
\subsection{Sangaku cercle et carré}
Dans cet exemple, on peut voir comment utiliser un point sans le nommer

\begin{center}
\begin{tkzexample}[vbox]
\begin{tikzpicture}[scale = 1]
   \tkzInit[xmax = 8] \tkzClip
   \tkzDefPoint(0,0){B}
   \tkzDefPoint(0,8){A}
   \tkzDefSquare(A,B)
   \tkzGetPoints{C}{D}
   \tkzDrawSquare(A,B)
   \tkzClipPolygon(A,B,C,D)
   \tkzDefPoint(4,8){F}
   \tkzDefPoint(4,0){E}
   \tkzDefPoint(4,4){Q}
   \tkzFillPolygon[color = green](A,B,C,D)
   \tkzDrawCircle[fill   = orange](B,A)
   \tkzDrawCircle[fill   = purple](E,B)  
   \tkzTgtFromP(F,A)(B)
   \tkzInterLL(F,tkzFirstPointResult)(C,D)
   \tkzInterLL(A,tkzPointResult)(F,E) 
   \tkzDrawCircle[fill = yellow](tkzPointResult,Q)  
   \tkzDefPointBy[projection= onto B--A](tkzPointResult)
   \tkzDrawCircle[fill = blue!50!black](tkzPointResult,A)
\end{tikzpicture}
\end{tkzexample}
\end{center}   

\newpage
\subsection{Constructions de certaines  transformations \addbs{tkzShowTransformation}}

 \begin{NewMacroBox}{tkzShowTransformation}{\oarg{local options}\parg{pt1,pt2} ou \parg{pt1,pt2,pt3}}
\emph{Ces constructions concernent les symétries  orthogonales, les symétries centrales, les projections orthogonales et les translations. Plusieurs options permettent l'ajustement des constructions. L'idée de cette macro revient à \tkzimp{Yves Combe}}
  

\medskip 
\begin{tabular}{lll}
\toprule
options             & défaut & définition                         \\ 
\midrule
\TOline{reflection= over pt1--pt2}{reflection}{constructions d'une symétrie orthogonale} 
\TOline{symmetry=center pt}{reflection}{constructions d'une symétrie centrale} 
\TOline{projection=onto pt1--pt2}{reflection}{constructions d'une projection}
\TOline{translation=from pt1 to pt2}{reflection}{constructions d'une translation}
\TOline{K}{1}{cercle inscrit dans à un triangle }
\TOline{length}{1}{longueur d'un arc}
\TOline{ratio} {.5}{rapport entre les longueurs des arcs}
\TOline{gap}{2}{placement le point de construction}
\TOline{size}{1}{rayon d'un arc (voir bissectrice)}
 \bottomrule
\end{tabular}

\emph{Il faut ajouter bien sûr tous les styles de \TIKZ\ pour les tracés}
\end{NewMacroBox}

\subsubsection{Exemple d'utilisation de \tkzcname{tkzShowTransformation}} 

\begin{center}
\begin{tkzexample}[latex=6cm,small]
\begin{tikzpicture}[scale=.8]
  \tkzInit[xmin=-4,xmax=4,ymin=-5,ymax=5]
  \tkzGrid \tkzClip \tkzPoint(0,0){O} \tkzPoint(2,-2){A}
  \tkzDefPoint(70:4){B} \tkzDrawPoints(A,O,B)
  \tkzLabelPoints(A,O,B)
  \tkzDrawLine[add= 2 and 2](O,A)
  \tkzDefPointBy[translation=from O to A](B) 
  \tkzGetPoint{C}
  \tkzDrawPoint[color=orange](C)  \tkzLabelPoints(C)
  \tkzShowTransformation[translation=from O to A,%
             length=2](B) 
  \tkzDrawVectors[color=orange](O,A B,C)  
  \tkzDefPointBy[reflection=over O--A](B) \tkzGetPoint{E}
  \tkzDrawSegment[blue](B,E)
  \tkzDrawPoint[color=blue](E)\tkzLabelPoints(E) 
  \tkzShowTransformation[reflection=over O--A,size=2](B)   
  \tkzDefPointBy[symmetry=center O](B) \tkzGetPoint{F} 
  \tkzDrawSegment[color=green](B,F)
  \tkzDrawPoint[color=green](F)\tkzLabelPoints(F)
  \tkzShowTransformation[symmetry=center O,%
                      length=2](B) 
  \tkzDefPointBy[projection=onto O--A](C) 
  \tkzGetPoint{H}    
  \tkzDrawSegments[color=magenta](C,H)
  \tkzDrawPoint[color=magenta](H)\tkzLabelPoints(H)
  \tkzShowTransformation[projection=onto O--A,%
                         color=red,size=3,gap=-2](C)   
\end{tikzpicture}
\end{tkzexample}
\end{center}

\subsubsection{Autre exemple d'utilisation de \tkzcname{tkzShowTransformation}} 

Vous retouverez cette figure, mais sans les traits de construction
\begin{tkzexample}[vbox]  
  \begin{tikzpicture}[scale=1.25]
  % on définit les points nécessaires 
  \tkzInit[ymin=-3]
  \tkzClip[space=1]
  \tkzDefPoint(0,0){A}
  \tkzDefPoint(8,0){B}
  \tkzDefPoint(3.5,10){I}
  \tkzDefMidPoint(A,B) \tkzGetPoint{O} 
  % syntaxe (liste de points) {liste des images} si vide on met des '
  \tkzDefPointBy[projection=onto A--B](I) \tkzGetPoint{J}
  \tkzInterLC(I,A)(O,A) \tkzGetPoints{M'}{M}
  \tkzInterLC(I,B)(O,A)  \tkzGetPoints{N}{N'}    
  \tkzDrawCircle[diameter](A,B)
   % attention plusieurs segments donc (s) espace entre les objets 
   % virgule entre les points
  \tkzDrawSegments(I,A I,B A,B B,M A,N) 
  % idem (s) et espace entre les objets
  \tkzMarkRightAngles(A,M,B A,N,B)  
  \tkzDrawSegment[style=dashed,color=blue](I,J)
  % tkzShowTransformation il y a aussi tkzShowLine 
  \tkzShowTransformation[projection=onto A--B,color=red,size=3,gap=-3](I)
  % on trace les points à la fin ainsi c'est plus propre, il n'y a rien 
  % par-dessus 
  \tkzDrawPoints[color=red](M,N)
  \tkzDrawPoints[color=blue](O,A,B,I) 
  %  \tkzLabelPoints version rapide de  \tkzLabelPoint on met automatiquement
  % $O$ etc ... sinon on traite chaque point l'un après l'autre avec
  %  \tkzLabelPoint(le point){son label}
  \tkzLabelPoints(O)  \tkzLabelPoints[above right](N,I) 
  \tkzLabelPoints[below left](M,A) 
\end{tikzpicture} 
\end{tkzexample} 
\endinput
%!TEX root = /Users/ego/Boulot/TKZ/tkz-euclide/doc_fr/TKZdoc-euclide-main.tex


\section{Intersections}



Il est possible de déterminer les coordonnées des points d'intersection entre deux droites, une droite et un cercle et deux cercles.

Les commandes associées n'ont pas d'arguments optionnels et l'usager doit lui même déterminer l'existence des points d'intersection.


\subsection{Intersection de deux droites}


 \begin{NewMacroBox}{tkzInterLL}{\parg{$A,B$}\parg{$C,D$}}
\emph{Définit le point d'intersection \tkzname{tkzPointResult} des deux droites $(AB)$ and $(CD)$. Les points connus sont donnés en couple (deux par droite) entre parenthèses, quant au point obtenu, son nom est placé entre accolades.}       

 \end{NewMacroBox}   
% 

\medskip
\subsubsection{exemple d'intersection entre deux droites}
\begin{center}
\begin{tkzexample}[vbox]
\begin{tikzpicture}[rotate=-30]
   \tkzDefPoint(2,1){A}   \tkzDefPoint(6,5){B}
   \tkzDefPoint(3,6){C}   \tkzDefPoint(5,2){D}
   \tkzDrawLines(A,B C,D)
   \tkzInterLL(A,B)(C,D)  \tkzGetPoint{I}
   \tkzDrawPoints[color=blue](A,B,C,D) \tkzDrawPoint[color=red](I)
\end{tikzpicture}
\end{tkzexample}
\end{center}  

De nombreux points particuliers sont obtenus avec cette macro par exemple l'orthocentre (OrthoCenter) voir \tkzcname{tkzOrthoCenter}, le centre du cercle circonscrit à un triangle \tkzcname{tkzCircumCenter}. 

\newpage
\subsection{Intersection d'une droite et d'un cercle} % (fold)
\label{sub:intersection_d_une_droite_et_d_un_cercle}
Pour avoir une syntaxe homogène, l'option pour définir le cercle à l'aide de la mesure du rayon est \tkzname{R} comme pour les macros pour  le cercle , les arcs et les secteurs.    

Comme précédemment, la droite est définie par un couple de points. Le cercle
 est aussi défini par un un couple :
 \begin{itemize}
  \item $(O,C)$ qui est un couple de points, le premier désigne le centre et le second est un point quelconque du cercle.
  \item $(O,r)$  La mesure $r$ est celle du rayon. Elle est exprimée soint en \emph{cm}, soit en \emph{pt}.
 \end{itemize}
 

\begin{NewMacroBox}{tkzInterLC}{\parg{$A,B$}\parg{$O,C/r$}\marg{$I$}\marg{$J$}}
Les arguments sont donc deux couples. Le premier couple est un couple de points, le second est soit un couple de points si aucune option n'est utilisée ou bien si l'option \tkzname{N} est utilisée sinon le couple est constitué d'un point (le centre du cercle et d'une mesure, celle du rayon).

\medskip
\begin{tabular}{lll}
\toprule
options            & défaut  & définition                         \\ 
\midrule
\TOline{N}        {N}    { (O,C) détermine le cercle}
\TOline{R}        {N}    { (O, 1 cm) ou (O, 120 pt)}  
\bottomrule
\end{tabular}

\medskip   
\emph{La macro définit les points d' intersection $I$ et $J$ de la droite $(AB)$ et du cercle de centre $O$ de rayon $r$ s'ils existent; dans le cas contraire, une erreur sera signalée dans le fichier .log}
\end{NewMacroBox}

\subsubsection{Exemple simple d'intersection droite-cercle}

Dans l'exemple suivant, le tracé du cercle utilise deux points et  l'intersection de la droite et du cercle utilise deux couples de points

\begin{tkzexample}[latex=7cm]
\begin{tikzpicture}
   \tkzInit[xmax=5,ymax=4]
 \tkzDefPoint(1,1){O} 
 \tkzDefPoint(0,4){A} 
 \tkzDefPoint(5,0){B} 
 \tkzDefPoint(3,3){C}
 \tkzInterLC(A,B)(O,C)  \tkzGetPoints{D}{E}  
 \tkzDrawCircle(O,C)
 \tkzDrawPoints[color=blue](O,A,B,C)
 \tkzDrawPoints[color=red](D,E)
 \tkzDrawLine(A,B)
 \tkzLabelPoints[above right](O,A,B,C,D,E)
\end{tikzpicture} 
\end{tkzexample}  

\subsubsection{Exemple plus complexe d'intersection droite-cercle}
\url{http://gogeometry.com/problem/p190_tangent_circle}

\begin{center}
\begin{tkzexample}[vbox]
\begin{tikzpicture}[scale=1.25]
  \tkzInit[xmin=0,xmax=8,ymin=-4,ymax=4]  \tkzClip[space=.4]
  \tkzDefPoint(0,0){A}  \tkzDefPoint(8,0){B}
  \tkzDefMidPoint(A,B)  \tkzGetPoint{O}
  \tkzDrawCircle(O,B)
  \tkzDefMidPoint(O,B)  \tkzGetPoint{O'}
  \tkzDrawCircle(O',B)
  \tkzTangent[from=A](O',B) \tkzGetSecondPoint{E}
  \tkzInterLC(A,E)(O,B)     \tkzGetSecondPoint{D}
  \tkzDefPointBy[projection=onto A--B](D)  \tkzGetPoint{F}
  \tkzMarkRightAngle(D,F,B)
  \tkzDrawSegments(A,D A,B D,F) 
  \tkzDrawSegments[color=red,line width=1pt,opacity=.4](A,O F,B)
  \tkzDrawPoints(A,B,O,O',E,D)  \tkzLabelPoints(A,B,O,O',E,D) 
\end{tikzpicture}
\end{tkzexample}
\end{center}

 

\newpage
\subsubsection{Cercle défini par un centre et une mesure, et cas particuliers}
Regardons quelques cas particuliers comme des droites tangentes au cercle. 

\begin{center}
 
\begin{tkzexample}[vbox]
\begin{tikzpicture}[scale=.75]
  \tkzDefPoint(0,8){A}  \tkzDefPoint(8,0){B}
  \tkzDefPoint(8,8){C}  \tkzDefPoint(4,4){I}
  \tkzDefPoint(2,7){E}  \tkzDefPoint(6,4){F}  
  \tkzDrawCircle[R](I,4 cm)
  \tkzInterLC[R](A,C)(I,4 cm)  \tkzGetPoints{I1}{I2}
  \tkzInterLC[R](B,C)(I,4 cm)  \tkzGetPoints{J1}{J2}
  \tkzInterLC[R](A,B)(I,4 cm)  \tkzGetPoints{K1}{K2}
  \tkzDrawPoints[color=red](I1,J1,K1,K2)
  \tkzDrawLines(A,B B,C A,C)
  \tkzInterLC[R](E,F)(I,4 cm)  \tkzGetPoints{I2}{J2}  
  \tkzDrawPoints[color=blue](E,F)
  \tkzDrawPoints[color=red](I2,J2)
  \tkzDrawLine(I2,J2)\end{tikzpicture}
\end{tkzexample}  
 
\end{center}

\newpage
\subsubsection{Exemple plus complexe}
Attention à la syntaxe. Tout d'abord, les calculs pour les points peuvent être faits pendant le passage des arguments, mais il faut respecter la syntaxe de \tkzname{fp}. Vous pouvez constater que j'utilise la macro  \tkzcname{FPpi} car \tkzname{fp} travaille en radians !. De plus quand des calculs nécéssitent l'emploi de parenthèses, celles-ci doivent être insérées dans un groupe \TEX \{ \dots \}.


\begin{center}
\begin{tkzexample}[vbox]
\begin{tikzpicture}[scale=2.5,rotate=180]
  \tkzDefPoint(0,1){J} \tkzDefPoint(0,0){O}
  \tkzDrawCircle[R](O,1 cm)
  \tkzDrawArc[R,line width=1pt,color=Gold](J,2.5 cm)(180,0)
  \foreach \i in {0,-5,-10,...,-85}{
     \tkzDefPoint({2.5*cos(\i*\FPpi/180)},{1+2.5*sin(\i*\FPpi/180)}){P}
     \tkzDrawSegment[color=orange](J,P)
     \tkzInterLC[R](P,J)(O,1 cm) \tkzGetPoints{M}{N}
     \tkzDrawPoints(N)} 
  \foreach \i in {-90,-95,...,-175,-180}{
    \tkzDefPoint({2.5*cos(\i*\FPpi/180)},{1+2.5*sin(\i*\FPpi/180)}){P} 
    \tkzDrawSegment[color=orange](J,P)
    \tkzInterLC[R](P,J)(O,1 cm) \tkzGetPoints{M}{N}
    \tkzDrawPoints(M)}   
\end{tikzpicture}
\end{tkzexample} 
\end{center}

\newpage
\subsubsection{Calcul de la mesure du rayon} 
 Avec \tkzname{pgfmath} et \tkzcname{pgfmathsetmacro}   
 
La mesure du rayon peut être le résultat d'un calcul que l'on ne fera pas au sein de la macro d'intersection, mais avant. 
On peut calculer une longueur de plusieurs façons. Il est possible bien sûr,
 d'utiliser le module \tkzname{pgfmath} et la macro \tkzcname{pgfmathsetmacro}. Dans certains, les résultats obtenus ne sont pas assez précis ainsi le calcul suivant $0.0002 \div 0.0001$ donne 1.98 avec pgfmath alors que fp.sty donnera 2. C'est pour cela que j'ai préféré interdire le calcul pendant le passage de paramètres, cela permet à chacun de choisir sa méthode.
   
\begin{tkzexample}[latex=7cm]
\begin{tikzpicture}  
  \tkzDefPoint(2,2){A}
  \tkzDefPoint(5,4){B}
  \tkzDefPoint(4,4){O}
  \pgfmathsetmacro{\tkzLen}{0.0002/0.0001}
  \tkzDrawCircle[R](O,\tkzLen cm)
  \tkzInterLC[R](A,B)(O, \tkzLen cm) 
  \tkzGetPoints{I}{J}
  \tkzDrawPoints[color=blue](A,B)
  \tkzDrawPoints[color=red](I,J)
  \tkzDrawLine(I,J) 
\end{tikzpicture}
\end{tkzexample}

\subsubsection{Calcul de la mesure du rayon} 
Avec \tkzname{fp} et \tkzcname{FPeval}
  
\begin{tkzexample}[latex=7cm]
  \begin{tikzpicture}  
  \tkzDefPoint(2,2){A}
  \tkzDefPoint(5,4){B}
  \tkzDefPoint(4,4){O}
  \FPeval{\tkzLen}{0.0002/0.0001} 
  \tkzDrawCircle[R](O,\tkzLen cm)
  \tkzInterLC[R](A,B)(O, \tkzLen cm) 
  \tkzGetPoints{I}{J}
  \tkzDrawPoints[color=blue](A,B)
  \tkzDrawPoints[color=red](I,J)
  \tkzDrawLine(I,J) 
\end{tikzpicture}
  \end{tkzexample}

\subsubsection{Calcul de la mesure du rayon} 
 Avec \TEX\ et \tkzcname{tkzLength}. 
 
 Cette dimension a été créée avec \tkzcname{newdimen}. 2 cm a été transformé en points. Il est bien sûr possible  d'utiliser \TEX\ pour calculer.

\begin{tkzexample}[latex=7cm]   
\begin{tikzpicture}
  \tkzDefPoint(2,2){A}
  \tkzDefPoint(5,4){B}
  \tkzDefPoint(4,4){O}
  \tkzLength=2cm 
  \tkzDrawCircle[R](O,\tkzLength pt)
  \tkzInterLC[R](A,B)(O, \tkzLength pt)
   \tkzGetPoints{I}{J}
  \tkzDrawPoints[color=blue](A,B)
  \tkzDrawPoints[color=red](I,J)
  \tkzDrawLine(I,J) 
\end{tikzpicture}
\end{tkzexample} 



\subsubsection{Des carrés dans un demi-disque}
Un air de Sangaku ! Il s'agit de prouver que l'on peut inscrire dans un demi-disque, deux carrés, et de déterminer la longueur de leurs côtés respectifs en fonction du rayon.

\begin{center}
\begin{tkzexample}[vbox]
\begin{tikzpicture}[scale=1.5]
 \tkzInit[xmax=8,ymax=5]\tkzClip[space=.25] 
 \tkzDefPoint(0,0){A}
 \tkzDefPoint(8,0){B}
 \tkzDefPoint(4,0){I}
 \tkzDefSquare(A,B)    
   \tkzGetPoints{C}{D}
 \tkzInterLC(I,C)(I,B) 
   \tkzGetPoints{E'}{E}
 \tkzInterLC(I,D)(I,B) 
   \tkzGetPoints{F'}{F} 
 \tkzDefPointsBy[projection = onto A--B](E,F){H,G}
 \tkzDefPointsBy[symmetry   = center H](I){J}
 \tkzDefSquare(H,J)
   \tkzGetPoints{K}{L}
 \tkzDrawSector[fill=Maroon!30](I,B)(A)
 \tkzFillPolygon[color=red!40](H,E,F,G)
 \tkzFillPolygon[color=blue!40](H,J,K,L)
 \tkzDrawPolySeg[color=red](H,E,F,G) 
 \tkzDrawPolySeg[color=red](J,K,L)
 \tkzDrawPoints(E,G,H,F,J,K,L)
\end{tikzpicture}
\end{tkzexample}      
\end{center}


\clearpage \newpage
\subsection{Intersection de deux cercles} 

Le cas le plus fréquent est celui de deux cercles définis par leur centre et un point, mais comme précédemment l'option \tkzname{R} permet d'utiliser les mesures des rayons

\begin{NewMacroBox}{tkzInterCC}{\oarg{options}\parg{$O,A/r$}\parg{$O',A'/r'$}\marg{$I$}\marg{$J$}}

\medskip
\begin{tabular}{lll}
\toprule
options       & défaut  & définition                         \\ 
\midrule
\TOline{N}   {N}    {OA et O'A' sont des rayons, O et O' les centres}
\TOline{R}   {N}    {$r$ et $r'$ sont des dimensions et mesurent les rayons}   
\bottomrule
\end{tabular}

\medskip
\noindent
\emph{Cette macro définit le(s) point(s) d' intersection $I$ et $J$ des deux cercles de centre $O$ et $O'$. Si les deux cercles n'ont pas de point commun alors la macro se termine par une erreur qui n'est pas gérée. \\ 
Il est également possible d'utiliser directement \tkzcname{tkzInterCCN} et  \tkzcname{tkzInterCCR}.}
\end{NewMacroBox}   

\subsubsection{Construction d'un triangle connaissant les mesures des côtés}
On veut obtenir le triangle de Pythagore (3,4,5)  
\begin{center}
\begin{tkzexample}[vbox]  
\begin{tikzpicture}[scale=.8]
  \tkzDefPoint(0,0){A} \tkzDefPoint(5,0){B}
  \tkzDrawCircle[R,dashed](A,4 cm) \tkzDrawCircle[R,dashed](B,3 cm)
  \tkzInterCC[R](A,4 cm)(B,3 cm) \tkzGetPoints{C}{D}
  \tkzDrawPolygon(A,B,C)
  \tkzCompasss(A,C B,C) 
  \tkzLabelSegment[below](A,B){$5$ cm}
  \tkzLabelSegment[above left](A,C){$4$ cm}
  \tkzLabelSegment[above right](B,C){$3$ cm}
  \tkzDrawPoints[color=red](C) 
  \tkzDrawPoints[color=blue](A,B)
\end{tikzpicture}
\end{tkzexample}
\end{center}

\subsubsection{Dupliquer un triangle} 
Trois segments étant donnés, construire un triangle. Il s'agit de récupérer les mesures des longueurs avec \tkzcname{tkzCalcLength}.

\begin{tkzexample}[vbox]
\begin{tikzpicture}
 \tkzDefPoint(1,0){A}  \tkzDefPoint(4,0){B}   % On place les points   
 \tkzDefPoint(1,1){C}  \tkzDefPoint(5,1){D}
 \tkzDefPoint(1,2){E}  \tkzDefPoint(6,2){F}
 \tkzDefPoint(0,4){A'} \tkzDefPoint(3,4){B'}
 \tkzCalcLength[cm](C,D)\tkzGetLength{rCD}
 \tkzCalcLength[cm](E,F)\tkzGetLength{rEF}
 \tkzInterCC[R](A',\rCD cm)(B',\rEF cm)\tkzGetPoints{I}{J}
 \tkzDrawSegments[red](A,B C,D E,F) % Les tracés   
 \tkzDrawLine(A',B')    
 \tkzDrawPoints(D,E,I,J)
 \tkzDrawPolygon[color=red](A',B',I)
 \tkzSetUpLine[color=gray]
 \tkzCompass(A',B')
 \tkzDrawCircle[R](A',\rCD cm)
 \tkzDrawCircle[R](B',\rEF cm)
 \tkzDrawPoints(A,B,C,D,E,F,A',B',I)
 \tkzLabelPoints[left](A,C,E)
 \tkzLabelPoints[right](B,D,F)
 \tkzLabelPoints[below](A',B')
 \tkzLabelPoint[above left](I){$C'$}   
\end{tikzpicture} 
\end{tkzexample}

\subsubsection{Construction d'un triangle équilatéral}

\begin{tkzexample}[vbox] 
\begin{tikzpicture}[rotate=30] 
 \tkzDefPoint(1,1){A}
 \tkzDefPoint(5,1){B}
 \tkzInterCC(A,B)(B,A)\tkzGetPoints{C}{D}
 \tkzDrawPoint[color=black](C)
 \tkzDrawCircle[dashed](A,B)
 \tkzDrawCircle[dashed](B,A)
 \tkzCompass[color=red](A,C)
 \tkzCompass[color=red](B,C)
 \tkzDrawPolygon(A,B,C)
 \tkzLabelSegment[above left](A,C){$4$ cm}
 \tkzLabelSegment[above right](B,C){$4$ cm}
 \tkzLabelSegment[below](A,B){$4$ cm} 
 \tkzLabelPoints[](A,B)
 \tkzLabelPoint[above](C){$C$} 
\end{tikzpicture}
\end{tkzexample}

\subsubsection{Un triangle isocèle.}

\begin{tkzexample}[vbox]
\begin{tikzpicture}[rotate=30] 
 \tkzDefPoint(1,2){A}
 \tkzDefPoint(5,1){B}
 \tkzInterCC[R](A,5cm)(B,5cm)\tkzGetPoints{C}{D}
 \tkzDrawCircle[R,dashed](A,5 cm)
 \tkzDrawCircle[R,dashed](B,5 cm) 
 \tkzDrawPoint[color=blue](C) 
 \tkzCompass[color=red](A,C)
 \tkzCompass[color=red](B,C)
 \tkzDrawPolygon(A,B,C)
 \tkzLabelSegment[above left](A,C){$5$ cm}
 \tkzLabelSegment[above right](B,C){$5$ cm}
 \tkzLabelPoints[](A,B)
 \tkzLabelPoint[above](C){$C$}     
\end{tikzpicture}
\end{tkzexample} 

\subsubsection{Exemple une médiatrice}

\begin{center}
\begin{tkzexample}[]
\begin{tikzpicture}
  \tkzDefPoint(0,0){A} 
  \tkzDefPoint(3,3){B}  
  \tkzDrawCircle[color=blue](B,A)
  \tkzDrawCircle[color=blue](A,B)
  \tkzInterCC(B,A)(A,B)\tkzGetPoints{M}{N}
  \tkzDrawLine(A,B)
  \tkzDrawPoints(M,N)
  \tkzDrawLine[color=red](M,N)
\end{tikzpicture}
\end{tkzexample}
\end{center} 

\newpage
\subsubsection{Trisection d'un segment}
Voici un exemple complet utilisant toutes les macros précédentes. Il s'agit de partager avec une règle et un compas, un segment en trois segments de même longueur. 

\begin{center}
\begin{tkzexample}[vbox]
\begin{tikzpicture}[scale=.8] 
 \tkzDefPoint(0,0){A}  \tkzDefPoint(3,2){B}
 \tkzInterCC(A,B)(B,A) \tkzGetPoints{C}{D}
 \tkzInterCC(D,B)(B,A) \tkzGetPoints{A}{E}  
 \tkzInterCC(D,B)(A,B) \tkzGetPoints{F}{B}
 \tkzInterLC(E,F)(F,A) \tkzGetPoints{D}{G}   
 \tkzInterLL(A,G)(B,E) \tkzGetPoint{O}      
 \tkzInterLL(O,D)(A,B) \tkzGetPoint{J}
 \tkzInterLL(O,F)(A,B) \tkzGetPoint{I}
 \tkzDrawCircle(D,A)    \tkzDrawCircle(A,B)
 \tkzDrawCircle(B,A)    \tkzDrawCircle(F,A)
 \tkzDrawSegments[color=red](O,G O,B O,D O,F)
 \tkzDrawPoints(A,B,D,E,F,G,I,J)  \tkzLabelPoints(A,B,D,E,F,G,I,J)
 \tkzDrawSegments[blue](A,B B,D A,D A,F F,G E,G B,E)
 \tkzMarkSegments[mark=s|](A,I I,J J,B)
\end{tikzpicture}
\end{tkzexample} 
\end{center}
 
 \endinput 


%!TEX root = /Users/ego/Boulot/TKZ/tkz-euclide/doc_fr/TKZdoc-euclide-main.tex

\section{Les droites}

Il est bien sûr essentiel de tracer des droites, mais avant il faut pouvoir définir certaines droites particulières comme des médiatrices, des bissectrices, des parallèles ou encore des perpendiculaires. Le principe consiste à déterminer deux points de la droite. 
   

\subsection{Définition de droites}

\begin{NewMacroBox}{tkzDefLine}{\oarg{local options}\parg{pt1,pt2} ou \parg{pt1,pt2,pt3}}
\noindent\emph{L' argument est une liste de deux  ou trois points.    Suivant les cas, la macro définit un ou deux points nécessaires pour obtenir la droite cherchée. Il faut utiliser soit la macro \tkzcname{tkzGetPoint}, soit la macro \tkzcname{tkzGetPoints}.}
  

\medskip
\begin{tabular}{lll}
\toprule
options             & défaut & définition                         \\ 
\midrule
\TOline{mediator}{}{médiatrice. Deux points sont définis} 
\TOline{perpendicular=through\ldots}{}{perpendiculaire à une droite passant par un point} 
\TOline{orthogonal=through\ldots}{}{voir ci-dessus }
\TOline{parallel=through\ldots}{}{parallèle à une droite passant par un point}
\TOline{bisector}{}{bissectrice d'un angle défini par trois points}
\TOline{bisector out}{}{bissectrice extérieure}
\TOline{K}{1}{Coefficient  pour la droite perpendiculaire}
 \bottomrule
\end{tabular}
\end{NewMacroBox}  

\subsubsection{Exemple avec \tkzname{mediator}}  
\begin{tkzexample}[latex=5 cm]
\begin{tikzpicture}[rotate=25]
  \tkzInit
  \tkzDefPoints{-2/0/A,1/2/B}
  \tkzDefLine[mediator](A,B)          \tkzGetPoints{C}{D}
  \tkzDefPointWith[linear,K=.75](C,D) \tkzGetPoint{D}
  \tkzDefMidPoint(A,B)                \tkzGetPoint{I}
  \tkzFillPolygon[color=orange!30](A,C,B,D)
  \tkzDrawSegments(A,B C,D)
  \tkzMarkRightAngle(B,I,C) 
  \tkzDrawSegments(D,B D,A)
  \tkzDrawSegments(C,B C,A)
\end{tikzpicture}
\end{tkzexample}  

\subsubsection{Exemple avec \tkzname{orthogonal} et \tkzname{parallel}}    
\begin{tkzexample}[latex=5 cm]
\begin{tikzpicture}
   \tkzDefPoints{-1.5/-0.25/A,1/-0.75/B,-0.7/1/C}
   \tkzDrawLine[end   = $(d_1)$](A,B)
   \tkzDrawPoints(A,B,C)
   \tkzDefLine[orthogonal=through C](B,A) \tkzGetPoint{c}
   \tkzDrawLine[end   = $(\delta)$](C,c)
   \tkzInterLL(A,B)(C,c) \tkzGetPoint{I}
   \tkzMarkRightAngle(C,I,B) 
   \tkzDefLine[parallel=through C](A,B) \tkzGetPoint{c'}
   \tkzDrawLine[end   = $(d_2)$](C,c') 
   \tkzMarkRightAngle(I,C,c')   
\end{tikzpicture}
\end{tkzexample}

\subsection{Tracer une droite}

Pour tracer une droite, il suffit de donner les deux points et d'utiliser l'option \tkzname{add}. Cette option est due à Mark Wibrow 

\begin{tkzltxexample}[]
  \tikzset{%
    add/.style args={#1 and #2}{
        to path={%
 ($(\tikztostart)!-#1!(\tikztotarget)$)--($(\tikztotarget)!-#2!(\tikztostart)$)%
  \tikztonodes}}}
\end{tkzltxexample}
  
  Cela permet de tracer une partie d'une droite définie par deux points. On utilise pour cela deux valeurs, qui sont des pourcentages par rapport à la longueur du segment défini par les deux points.
  
\begin{tkzexample}[]
\begin{tikzpicture}
   \tkzDefPoints{0/0/A,5/0/B}
   \tkzDrawLine[color=blue,thin, add=1 and 1,end   = $(\delta)$](A,B) 
   \tkzDrawLine[color=red,thick, add=.5 and .5](A,B)
   \tkzDrawPoints(A,B)  \tkzLabelPoints(A,B)
    \tkzDrawLine[color=Maroon,line width=2pt, add=-.2 and -.2 ](A,B)  
  \end{tikzpicture} 
\end{tkzexample} 

 \begin{NewMacroBox}{tkzDrawLine}{\oarg{local options}\parg{pt1,pt2}}
\emph{Les arguments sont une liste de deux points.}

\begin{tabular}{lll}
\toprule
options             & défaut & définition                         \\ 
\midrule
\TOline{add= nb1 and nb2}{.2 and .2}{Permet de prolonger le segment} 
 \bottomrule
\end{tabular}

\medskip 
\emph{\tkzname{add} permet de définir la longueur du trait passant par les points pt1 et pt2. Les deux nombres sont des pourcentages. Les styles de \TIKZ\ sont accessibles pour les tracés}
\end{NewMacroBox}

\subsubsection{Exemple de tracer de droite avec \tkzname{add}}

\begin{tkzexample}[latex=5cm]
\begin{tikzpicture}
 \tkzInit[xmin=-2,xmax=3,ymin=-2.25,ymax=2.25]
 \tkzClip[space=.25]
 \tkzDefPoint(0,0){A} \tkzDefPoint(2,0.5){B}
 \tkzDefPoint(0,-1){C}\tkzDefPoint(2,-0.5){D} 
 \tkzDefPoint(0,1){E} \tkzDefPoint(2,1.5){F} 
 \tkzDefPoint(0,-2){G} \tkzDefPoint(2,-1.5){H}
  \tkzDrawLine(A,B)    \tkzDrawLine[add = 0 and .5](C,D) 
 \tkzDrawLine[add = 1 and 0](E,F)
  \tkzDrawLine[add = 0 and 0](G,H) 
 \tkzDrawPoints(A,B,C,D,E,F,G,H)    
 \tkzLabelPoints(A,B,C,D,E,F,G,H)  
\end{tikzpicture}
\end{tkzexample} 

\newpage
Il est possible de tracer plusieurs droites, mais avec les mêmes options.
\begin{NewMacroBox}{tkzDrawLines}{\oarg{local options}\parg{pt1,pt2 pt3,pt4 ...}}
\emph{Les arguments sont une liste de couples de deux points séparés par des espaces. Les styles de \TIKZ\ sont accessibles pour les tracés.}
\end{NewMacroBox}      

\subsubsection{Exemple avec \tkzcname{tkzDrawLines}}    
\begin{center}
\begin{tkzexample}[latex=7cm]
\begin{tikzpicture}
  \tkzDefPoint(0,0){A}
  \tkzDefPoint(2,0){B}
  \tkzDefPoint(1,2){C}
  \tkzDefPoint(3,2){D}   
  \tkzDrawLines(A,B C,D A,C B,D)
  \tkzLabelPoints(A,B,C,D)
\end{tikzpicture}
\end{tkzexample}
\end{center} 
 
\begin{center}
\begin{tkzexample}[vbox] 
\begin{tikzpicture}
 \tkzInit[xmin=-3,xmax=6, ymin=-1,ymax=6]
 \tkzClip
 \tkzDefPoint(0,0){O}
 \tkzDefPoint(3,1){I}
 \tkzDefPoint(1,4){J}
 \tkzDefLine[bisector](I,O,J)     \tkzGetPoint{i}   
 \tkzDefLine[bisector out](I,O,J) \tkzGetPoint{j}
 \tkzDrawLines[add = 1 and 1,color=red](O,I O,J) 
 \tkzDrawLines[add = 5 and 5,color=blue](O,i O,j) 
\end{tikzpicture} 
\end{tkzexample}
\end{center} 

\newpage
\subsubsection{Une enveloppe}
D'après une figure d'O. Reboux  avec pst-eucl de D Rodriguez
\begin{center}
\begin{tkzexample}[vbox]
\begin{tikzpicture}[scale=1.25]
  \tkzInit[xmin=-6,ymin=-6,xmax=6,ymax=6]  
  \tkzClip 
  \tkzDefPoint(0,0){O} 
  \tkzDefPoint(132:4){A}
  \tkzDefPoint(5,0){B}
  \foreach \ang in {5,10,...,360}{%
    \tkzDefPoint(\ang:5){M}
    \tkzDefLine[mediator](A,M)
    \tkzDrawLine[color=magenta,add= 4 and 4](tkzFirstPointResult,tkzSecondPointResult)}
\end{tikzpicture}
\end{tkzexample}
\end{center}

\newpage
\subsubsection{Une parabole}
D'après une figure d'O. Reboux  avec pst-eucl de D Rodriguez.
Il n'est pas nécessaire de nommer les deux points qui définissent la médiatrice.

\begin{center}
\begin{tkzexample}[vbox]
\begin{tikzpicture}[scale=1.25]
  \tkzInit[xmin=-6,ymin=-6,xmax=6,ymax=6]  
  \tkzClip 
  \tkzDefPoint(0,0){O} 
  \tkzDefPoint(132:5){A}
  \tkzDefPoint(4,0){B}
  \foreach \ang in {5,10,...,360}{%
    \tkzDefPoint(\ang:4){M}
    \tkzDefLine[mediator](A,M) 
    \tkzDrawLine[color=magenta,
             add= 4 and 4](tkzFirstPointResult,tkzSecondPointResult)}
   \end{tikzpicture}
\end{tkzexample}
\end{center}


\subsection{Ajouter des labels aux  droites \tkzcname{tkzLabelLine}} 

 \begin{NewMacroBox}{tkzLabelLine}{\oarg{local options}\parg{pt1,pt2}\marg{label}}

 \begin{tabular}{lll}
 \toprule
 arguments &  défaut  & définition                 \\ 
 \midrule
 \TAline{label}{}{exemple \tkzcname{tkzLabelLine(A,B)\{$\delta$\}}}
 \bottomrule
 \end{tabular}

\medskip
\begin{tabular}{lll}
\toprule
options             & défaut & définition                         \\ 
\midrule
\TOline{pos}{.5}{pos est une option de \TIKZ\ mais essentielle dans ce cas} 
 \bottomrule
\end{tabular}

\medskip
\emph{En option et en plus de \tkzname{pos}, on peut utiliser tous les styles de \TIKZ\ , en particulier le placement avec \tkzname{above}, \tkzname{right}, \dots}

 \end{NewMacroBox}

\subsubsection{Exemple avec \tkzcname{tkzLabelLine}}
Une option importante est \tkzname{pos}, c'est elle qui permet de placer le label le long de la droite. La valeur de \tkzname{pos} peut être supérieure à 1 ou négative.

\begin{tkzexample}[latex=4cm]
\begin{tikzpicture}
   \tkzInit[ymin=-1,ymax=1.5,xmin=-2,xmax=2.5]
   \tkzDefPoints{0/0/A,3/0/B,1/1/C}
   \tkzDefLine[perpendicular=through C,K=-1](A,B)
   \tkzGetPoint{c}
   \tkzDrawLines(A,B C,c)
   \tkzLabelLine[pos=1.25,blue,right](C,c){$(\delta)$} 
   \tkzLabelLine[pos=-0.25,red,left](C,c){encore $(\delta)$} 
\end{tikzpicture}
\end{tkzexample}


\subsection{Configurer les options pour les lignes \tkzcname{tkzSetUpLine}}
voir  \ref{tkzsetupline}
 
\newpage
\subsection{Montrer les constructions de certaines  lignes \tkzcname{tkzShowLine}}

 \begin{NewMacroBox}{tkzShowLine}{\oarg{local options}\parg{pt1,pt2} ou \parg{pt1,pt2,pt3}}
\emph{Ces constructions concernent les médiatrices, les droites perpendiculaires ou parallèles passant par un point donné et les bissectrices. Les arguments sont donc des listes de deux ou bien de trois points. Plusieurs options permettent l'ajustement des constructions. L'idée de cette macro revient à \tkzimp{Yves Combe}}
  

\medskip 
\begin{tabular}{lll}
\toprule
options             & défaut & définition                         \\ 
\midrule
\TOline{mediator}{mediator}{affiche les constructions d'une médiatrice} 
\TOline{perpendicular}{mediator}{constructions pour une perpendiculaire} 
\TOline{orthogonal}{mediator}{idem}
\TOline{bisector}{mediator}{constructions pour une bissectrice}
\TOline{K}{1}{cercle inscrit dans à un triangle }
\TOline{length}{1}{ en cm, longueur d'un arc}
\TOline{ratio} {.5}{rapport entre les longueurs des arcs}
\TOline{gap}{2}{placement le point de construction}
\TOline{size}{1}{rayon d'un arc (voir bissectrice)}
 \bottomrule
\end{tabular}

\emph{Il faut ajouter bien sûr tous les styles de \TIKZ\ pour les tracés}
\end{NewMacroBox}

\subsubsection{Exemple de \tkzcname{tkzShowLine} et \tkzname{parallel}} 

\begin{tkzexample}[latex=5cm]
\begin{tikzpicture}
    \tkzDefPoints{-1.5/-0.25/A,1/-0.75/B,-1.5/2/C}
    \tkzDrawLine(A,B)
    \tkzDefLine[parallel=through C](A,B)  \tkzGetPoint{c} 
    \tkzShowLine[parallel=through C](A,B)
    \tkzDrawLine(C,c)
    \tkzDrawPoints(A,B,C,c)
\end{tikzpicture}
\end{tkzexample}



\subsubsection{Exemple de \tkzcname{tkzShowLine} et \tkzname{perpendicular}} 

\begin{tkzexample}[latex=6cm]
\begin{tikzpicture}
  \tkzInit[xmin=0,xmax=6,ymin=0,ymax=6]
  \tkzClip
  \tkzDefPoint(0,0){A}
  \tkzDefPoint(3,4){B}  
  \tkzDefPoint(2,4){C}
  \tkzDefLine[perpendicular=through C,%
              K=-.5](A,B)
  \tkzGetPoint{c}
  \tkzDefPointBy[projection=onto A--B](c) 
  \tkzGetPoint{h}
  \tkzMarkRightAngle[fill=lightgray](A,h,C)
  \tkzDrawLines[](A,B C,c)
  \tkzDrawPoints(A,B,C,h,c)
\end{tikzpicture}
\end{tkzexample}

\subsubsection{Exemple de \tkzcname{tkzShowLine} et \tkzname{bisector}} 

\begin{tkzexample}[latex=5.25 cm]
\begin{tikzpicture}
 \tkzInit[xmin=0,xmax=7,ymin=0,ymax=7]
 \tkzClip 
 \tkzDefPoints{0/0/A, 6/2/B, 1/6/C}
 \tkzDrawPolygon(A,B,C)  
 \tkzSetUpCompass[color=brown,line width=.1 pt]
 \tkzDefLine[bisector](B,A,C)  \tkzGetPoint{a}
 \tkzDefLine[bisector](C,B,A)  \tkzGetPoint{b}
 \tkzShowLine[bisector,size=2,gap=3](B,A,C)
 \tkzShowLine[bisector,size=1,gap=3](C,B,A)   
 \tkzInterLL(A,a)(B,b) \tkzGetPoint{I}
 \tkzDefPointBy[projection = onto A--B](I) 
 \tkzDrawCircle[radius,color=red,%
 line width=.2pt](I,tkzPointResult) 
 \tkzDrawSegments[color=Maroon!50](I,tkzPointResult)
 \tkzDrawLines[add=0 and 5,color=Maroon!50](A,a B,b) 
\end{tikzpicture}

\end{tkzexample}

\subsubsection{Exemple de \tkzcname{tkzShowLine} et \tkzname{mediator}} 
\begin{tkzexample}[latex=6 cm]
\begin{tikzpicture}
 \tkzInit[xmax=6,ymax=7]
 \tkzGrid
 \tkzDefPoint(2,2){A} 
 \tkzDefPoint(5,4){B}  
 \tkzDrawPoints(A,B)    
 \tkzShowLine[mediator,color=orange,length=1](A,B)
 \tkzGetPoints{i}{j}
 \tkzLabelPoints[below =3pt](A,B)
 \tkzDrawLines[](A,B i,j) 
\end{tikzpicture}
\end{tkzexample}
\endinput
%!TEX root = /Users/ego/Boulot/TKZ/tkz-euclide/doc_fr/TKZdoc-euclide-main.tex

\section{Les segments}

Il existe bien sûr, une macro pour tracer simplement un segment (il serait possible comme pour une demi-droite, de créer un style avec \tkzcname{add}) .

\subsection{Tracer un segment \tkzcname{tkzDrawSegment}} 
 \hypertarget{tds}{}      

 \begin{NewMacroBox}{tkzDrawSegment}{\oarg{local options}\parg{pt1,pt2}}
\emph{Les arguments sont une liste de deux points. Les styles de \TIKZ\ sont accessibles pour les tracés}
 
\medskip
\begin{tabular}{lll}
argument    & exemple & définition    \\
\midrule
\TAline{(pt1,pt2)}{(A,B)}{trace le segment $[A,B]$}
\bottomrule 
\end{tabular}

C'est bien sûr équivalent à \tkzcname{draw (A)--(B);} 
\end{NewMacroBox}

\subsubsection{Exemple avec des références de points}     

\begin{tkzexample}[latex=6cm]
\begin{tikzpicture}[scale=1.5]
  \tkzInit[xmin=-1,xmax=3,ymin=-1,ymax=2]
  \tkzClip
  \tkzDefPoint(0,0){A}
  \tkzDefPoint(2,1){B}
  \tkzDrawSegment[color=red,thin](A,B)
  \tkzDrawPoints(A,B)    
  \tkzLabelPoints(A,B)  
\end{tikzpicture}
\end{tkzexample}
  

\subsubsection{Exemple avec des références de points} 
 Il est préférable de référencer les points, car les points sont
 placées en tenant compte de  \tkzcname{tkzInit}.
 
\begin{tkzexample}[latex=6cm]
\begin{tikzpicture}[scale=1.5]
  \tkzInit[xmin=-1,xmax=3,ymin=-1,ymax=2]
  \tkzClip
  \tkzDrawSegment[color=red,thin]({0,0},{2,1})  
\end{tikzpicture}
\end{tkzexample} 

\bigskip
Si les options sont les mêmes on peut tracer plusieurs \hypertarget{segs}{segments} avec la même macro. 
 
\newpage
\subsection{Tracer des segments \tkzcname{tkzDrawSegments}} 
 \hypertarget{tdss}{}      

 \begin{NewMacroBox}{tkzDrawSegments}{\oarg{local options}\parg{pt1,pt2 pt3,pt4 ...}}
\emph{Les arguments sont une liste de couple de deux points. Les styles de \TIKZ\ sont accessibles pour les tracés}
\end{NewMacroBox}

\begin{center}
\begin{tkzexample}[latex=6cm]
\begin{tikzpicture}
  \tkzInit[xmin=-1,xmax=3,ymin=-1,ymax=2]
  \tkzClip[space=1]
  \tkzDefPoint(0,0){A}
  \tkzDefPoint(2,1){B} 
  \tkzDefPoint(3,0){C} 
  \tkzDrawSegments(A,B B,C)
  \tkzDrawPoints(A,B,C)    
  \tkzLabelPoints(A,C) 
  \tkzLabelPoints[above](B)  
\end{tikzpicture}
\end{tkzexample}
\end{center} 

\subsection{Marquer un segment \tkzcname{tkzMarkSegment}}
\hypertarget{tms}{}  
  
 \begin{NewMacroBox}{tkzMarkSegment}{\oarg{local options}\parg{pt1,pt2}} 
\emph{La macro permet de placer une marque sur un segment.}

\medskip
\begin{tabular}{lll}
\toprule
options             & défaut & définition    \\
\midrule
\TOline{pos}{.5}{position de la marque} 
\TOline{color}{black}{couleur de la marque} 
\TOline{mark}{none}{choix de la marque} 
\TOline{size}{4pt}{taille de la marque} 
\bottomrule
\end{tabular}

\emph{Les marques possibles sont celles fournies par \TIKZ, mais d'autres marques ont été crées d'après une idée de Yves Combe.}
\end{NewMacroBox} 

\subsubsection{Marques multiples}
\begin{tkzexample}[latex=6cm,small] 
\begin{tikzpicture}
  \tkzDefPoint(2,1){A}
  \tkzDefPoint(6,4){B}
  \tkzDrawSegment(A,B)
  \tkzMarkSegment[color=Maroon,size=2pt,
        pos=0.4, mark=z](A,B) 
  \tkzMarkSegment[color=blue,
        pos=0.2, mark=oo](A,B)
  \tkzMarkSegment[pos=0.8,
        mark=s,color=red](A,B) 
\end{tikzpicture}
\end{tkzexample}

\subsubsection{Utilisation de \tkzname{mark}}      
\begin{tkzexample}[latex=6cm,small] 
\begin{tikzpicture}
  \tkzDefPoint(2,1){A} 
  \tkzDefPoint(6,4){B}
  \tkzDrawSegment(A,B)
  \tkzMarkSegment[color=gray,
                  pos=0.2,mark=s|](A,B)
  \tkzMarkSegment[color=gray,
                  pos=0.4,mark=s||](A,B)
  \tkzMarkSegment[color=Maroon,
                  pos=0.6,mark=||](A,B)
  \tkzMarkSegment[color=red,
                  pos=0.8,mark=|||](A,B)
\end{tikzpicture}
\end{tkzexample}


\subsection{Marquer des segments \tkzcname{tkzMarkSegments}}
\hypertarget{tmss}{} 
 
\begin{NewMacroBox}{tkzMarkSegments}{\oarg{local options}\parg{pt1,pt2 pt3,pt4 ...}}
\emph{Les arguments sont une liste de couple de deux points séparés par des espaces. Les styles de \TIKZ\ sont accessibles pour les tracés.}
\end{NewMacroBox} 

\subsubsection{Marques pour un triangle isocèle}      
\begin{tkzexample}[latex=6cm,small]
\begin{tikzpicture}[scale=1]
 \tkzDefPoints{0/0/O,2/2/A,4/0/B,6/2/C}
 \tkzDrawSegments(O,A A,B)
 \tkzDrawPoints(O,A,B)
 \tkzDrawLine(O,B)   
 \tkzMarkSegments[mark=||,size=6pt](O,A A,B)
\end{tikzpicture}
\end{tkzexample} 

\subsection{Exemple de rotation}   
\begin{center}
\begin{tkzexample}[latex=7cm,small] 
 \begin{tikzpicture}[scale=0.5]
  \tkzDefPoint(0,0){A}\tkzDefPoint(3,2){B} 
  \tkzDefPoint(4,0){C}\tkzDefPoint(2.5,1){P}
  \tkzDrawPolygon(A,B,C)
  \tkzDefEquilateral(A,P) \tkzGetPoint{P'}
  \tkzDefPointsBy[rotation=center A angle 60](P,B){P',C'}
  \tkzDrawPolygon(A,P,P')
  \tkzDrawPolySeg(P',C',A,P,B)
  \tkzDrawSegment(C,P)
  \tkzDrawPoints(A,B,C,C',P,P')
  \tkzMarkSegments[mark=s|,mark size=6pt,
  color=blue](A,P P,P' P',A) 
  \tkzMarkSegments[mark=||,color=orange](B,P P',C')
  \tkzLabelPoints(A,C) \tkzLabelPoints[below](P) 
  \tkzLabelPoints[above right](P',C',B) 
  
\end{tikzpicture} 
\end{tkzexample}  
\end{center}  
\newpage  
\hypertarget{tls}{}  
 \begin{NewMacroBox}{tkzLabelSegment}{\oarg{local options}\parg{pt1,pt2}\marg{label}}
\emph{Cette macro permet de placer une étiquette le long d'un segment ou encore d'une ligne. Les options sont celles de \TIKZ\ par exemple \tkzname{pos} } 

\medskip
\begin{tabular}{lll}
argument    & exemple & définition    \\
\midrule
\TAline{label}{\tkzcname{tkzLabelSegment(A,B)\{$5$\}}}{texte de l'étiquette} 
\TAline{(pt1,pt2)}{(A,B)}{étiquette le long de $[A,B]$} 
\bottomrule
\end{tabular}


\medskip
\begin{tabular}{lll}
options  & défaut & définition    \\
\midrule
\TOline{pos}{.5}{position du label} 
\end{tabular}
\end{NewMacroBox}  

 \subsubsection{Labels multiples}      
\begin{tkzexample}[latex=6 cm,small]
\begin{tikzpicture}
\tkzInit
\tkzDefPoint(0,0){A}
\tkzDefPoint(6,0){B}
\tkzDrawSegment(A,B)
\tkzLabelSegment[above,pos=.8](A,B){$a$}
\tkzLabelSegment[below,pos=.2](A,B){$4$}
\end{tikzpicture} 
\end{tkzexample}  

\subsubsection{Labels et Pythagore}
Cet exemple nécessite \tkzcname{usetkzobj{polygons}}
      
\begin{tkzexample}[latex=7cm]
\begin{tikzpicture}[scale=.75]
\tkzInit[xmax=5,ymax=5]
\tkzDefPoint(0,0){C}
\tkzDefPoint(4,0){A}
\tkzDefPoint(0,3){B}
\tkzDefSquare(B,A)\tkzGetPoints{E}{F}
\tkzDefSquare(A,C)\tkzGetPoints{G}{H}
\tkzDefSquare(C,B)\tkzGetPoints{I}{J}
\tkzFillPolygon[draw,
                fill = red!50 ](A,C,G,H)
\tkzFillPolygon[draw,
                fill = blue!50 ](C,B,I,J)
\tkzFillPolygon[draw,
                fill = purple!50](B,A,E,F)
\tkzFillPolygon[draw,opacity=.5,
                fill = orange](A,B,C)
\tkzDrawPolygon[line width = 1pt](A,B,C)
\tkzLabelSegment[above](C,A){$a$}
\tkzLabelSegment[right](B,C){$b$}
\tkzLabelSegment[below left](B,A){$c$}
\end{tikzpicture} 
\end{tkzexample}

\newpage 
\hypertarget{tlss}{} 
 \begin{NewMacroBox}{tkzLabelSegments}{\oarg{local options}\parg{pt1,pt2 pt3,pt4 ...}}
\emph{Les arguments sont une liste de couple de deux points. Les styles de \TIKZ\ sont accessibles pour les tracés.}
\end{NewMacroBox} 
 
\subsubsection{Labels pour un triangle isocèle}      
\begin{center}
\begin{tkzexample}[latex=6cm,small]
\begin{tikzpicture}[scale=1]
 \tkzDefPoints{0/0/O,2/2/A,4/0/B,6/2/C}
 \tkzDrawSegments(O,A A,B)
 \tkzDrawPoints(O,A,B)
 \tkzDrawLine(O,B)   
 \tkzLabelSegments[color=red,above=4pt](O,A A,B){$a$}
\end{tikzpicture}
\end{tkzexample}  
\end{center}   
\endinput



%!TEX root = /Users/ego/Boulot/TKZ/tkz-euclide/doc_fr/TKZdoc-euclide-main.tex


\section{Définition de points à l'aide d'un vecteur}

\subsection{\tkzcname{tkzDefPointWith}}
Il y a plusieurs possibilités pour créer des points qui répondent à certaines conditions vectorielles.
Cela peut se faire avec  \tkzcname{tkzDefPointWith}. Le principe général est le suivant, deux points sont passés en argument, autrement dit un vecteur. Les différentes options permettent d'obtenir  un nouveau point formant avec le premier point (sauf exception) un vecteur colinéaire  ou bien orthogonal au premier vecteur. Ensuite la longueur est soit proportionnelle à celle du premier, ou bien proportionnelle à l'unité. Dans la mesure ou ce point n'est utilisé que temporairement, il n'est pas obligé de le nommer immédiatement. Le résultat est dans \tkzcname{tkzPointResult}. La macro \tkzNameMacro{tkzGetPoint} permet de récupérer le point et de le nommer différemment.

\begin{NewMacroBox}{tkzDefPointWith}{\parg{pt1,pt2}}
 Il s'agit en fait de la définition d'un point  répondant à des conditions vectorielles.

\medskip
  
\begin{tabular}{lll}
\toprule
arguments             & définition & explication                         \\ 
\midrule
\TAline{(pt1,pt2)} {couple de points}{le résultat est un point dans \tkzcname{tkzPointResult} } \\                                                                         
 \bottomrule
\end{tabular}

\medskip
Dans ce qui suit, on suppose que le point est récupéré par \tkzNameMacro{tkzGetPoint\{C\}}

\begin{tabular}{lll}
\toprule
options             & exemple & explication                         \\ 
\midrule
\TOline{orthogonal}{[orthogonal](A,B)}{$AC=AB$ et $\overrightarrow{AC} \perp \overrightarrow{AB}$}
\TOline{orthogonal normed}{[orthogonal normed](A,B)}{$AC=1$ et $\overrightarrow{AC} \perp \overrightarrow{AB}$ } 
\TOline{linear}{[linear](A,B)}{ $\overrightarrow{AC}=K \times \overrightarrow{AB}$}
\TOline{linear normed}{[linear normed](A,B)}{$AC=K$ et $\overrightarrow{AC}=k\times \overrightarrow{AB}$ }  
\TOline{colinear= at \#1}{[colinear= at C](A,B)}{$\overrightarrow{CD}= \overrightarrow{AB}$ }
\TOline{K}{[linear](A,B),K=2}{$\overrightarrow{AC}=2\times \overrightarrow{AB}$}     
  \bottomrule
\end{tabular}

\medskip
\noindent\emph{Pour la linéarité, K est obligatoire. Sa valeur par défaut est égale à 1.}   


\end{NewMacroBox}

\subsubsection{\tkzcname{tkzDefPointWith} et \tkzname{orthogonal}} 
$K=-1$ c'est pour que $(\overrightarrow{AC},\overrightarrow{AB})$ détermine un angle positif. AB=AC puisque $|K|=1$
\begin{tkzexample}[latex=6cm]
\begin{tikzpicture}[scale=1.2]
   \tkzInit[xmax=5,ymax=4] \tkzGrid
   \tkzDefPoint(2,3){A}   \tkzDefPoint(4,2){B}
   \tkzDefPointWith[orthogonal,K=-1](A,B)
   \tkzGetPoint{C}
   \tkzDrawPoints[color=red](A,B,C)
   \tkzLabelPoints[above right=3pt](A,B,C)
\end{tikzpicture} 
\end{tkzexample}

\subsubsection{\tkzcname{tkzDefPointWith}  \tkzname{orthogonal normed}} 
AC=1

\begin{tkzexample}[latex=6cm]
\begin{tikzpicture}[scale=1.2]
   \tkzInit[ymin=1,xmax=5,ymax=5] \tkzGrid
   \tkzDefPoint(2,3){A}   \tkzDefPoint(4,2){B}
   \tkzDefPointWith[orthogonal normed](A,B)
   \tkzGetPoint{C}
   \tkzDrawPoints[color=red](A,B,C)
   \tkzLabelPoints[above right=3pt](A,B,C)
\end{tikzpicture} 
\end{tkzexample}

\subsubsection{\tkzcname{tkzDefPointWith} et  \tkzname{orthogonal normed}} 
$K=2$ donc AC=2.

\begin{tkzexample}[latex=6cm]
\begin{tikzpicture}[scale=1.2]
   \tkzInit[ymin=1,xmax=5,ymax=5] \tkzGrid
   \tkzDefPoint(2,3){A}   \tkzDefPoint(4,2){B}
   \tkzDefPointWith[orthogonal normed,K=2](A,B)
   \tkzGetPoint{C}
   \tkzDrawPoints[color=red](A,B,C)
   \tkzLabelPoints[above right=3pt](A,B,C)
\end{tikzpicture} 
\end{tkzexample}

\subsubsection{\tkzcname{tkzDefPointWith} et \tkzname{colinear}} 
$K=2$ donc AC=2.

\begin{tkzexample}[latex=6cm]
\begin{tikzpicture}[scale=1.2]
   \tkzInit[xmax=5,ymax=4] \tkzGrid
   \tkzDefPoint(2,3){A}   \tkzDefPoint(4,2){B}
   \tkzDefPoint(0,1){C}
   \tkzDefPointWith[colinear=at C](A,B)
   \tkzGetPoint{D}
   \tkzDrawPoints[color=red](A,B,C,D)
   \tkzLabelPoints[above right=3pt](A,B,C,D)
\end{tikzpicture} 
\end{tkzexample}

\subsubsection{\tkzcname{tkzDefPointWith}  \tkzname{linear} } 
 Ici $K=0.5$
Cela revient à appliquer une homothétie ou bien encore une multiplication d'un vecteur par un réel. C est ici le milieu de $[AB]$.

\begin{tkzexample}[latex=6cm]
\begin{tikzpicture}[scale=1.2]
   \tkzInit[ymin=1,xmax=5,ymax=4] \tkzGrid
   \tkzDefPoint(1,3){A}   \tkzDefPoint(4,2){B}
   \tkzDefPointWith[linear,K=0.5](A,B)
   \tkzGetPoint{C}
   \tkzDrawPoints[color=red](A,B,C)
   \tkzLabelPoints[above right=3pt](A,B,C)
\end{tikzpicture} 
\end{tkzexample}

\subsubsection{\tkzcname{tkzDefPointWith}  \tkzname{linear normed}}
Dans l'exemple suivant AC=1 et C appartient à $(AB)$.

\begin{tkzexample}[latex=6cm]
\begin{tikzpicture}[scale=1.2]
   \tkzInit[ymin=1,xmax=5,ymax=4] \tkzGrid
   \tkzDefPoint(1,3){A}   \tkzDefPoint(4,2){B}
   \tkzDefPointWith[linear normed](A,B)
   \tkzGetPoint{C}
   \tkzDrawPoints[color=red](A,B,C)
   \tkzLabelPoints[above right=3pt](A,B,C)
\end{tikzpicture} 
\end{tkzexample}
\endinput
  
%!TEX root = /Users/ego/Boulot/TKZ/tkz-euclide/doc_fr/TKZdoc-euclide-main.tex

%<–––––––––––––––––––––––––––––––––––––––––––––––––––––––––––––––––––––––––>
\section{Polygones}
%<–––––––––––––––––––––––––––––––––––––––––––––––––––––––––––––––––––––––––>
\subsection{Définition des  triangles} 
Les macros suivantes  vont permettre de définir ou de construire un triangle à partir \tkzname{au moins} de deux points. 

 Pour le moment, il est possible de définir les triangles suivants :
 \begin{itemize}
\item  \tkzname{two angles}  détermine un triangle connaissant deux angles,
\item  \tkzname{equilateral}  détermine un triangle équilatéral,
\item \tkzname{half} détermine un triangle rectangle tel que le rapport des mesures des deux côtés adjacents à l'angle droit soit égal à $2$,
\item \tkzname{pythagore} détermine un triangle rectangle dont les mesures des côtés sont proportionnelles à 3, 4 et 5,
\item \tkzname{school} détermine un triangle rectangle dont les angles sont 30, 60 et 90 degrés,
\item \tkzname{golden} détermine un triangle rectangle tel que le rapport des mesures des deux côtés adjacents à l'angle droit soit égal $\Phi=1,618034$, J'ai choisi comme dénomination « triangle doré » car il rpovient du rectangle d'or et j'ai conservé la dénomination « triangle d'or »  ou encore « triangle d'Euclide » pour le triangle isocèle dont les angles à la base sont de 72 degrés,

\item \tkzname{gold} ou \tkzname{euclide} pour le triangle d'or,

\item \tkzname{cheops} détermine un troisième point tel que le triangle soit isocèle  dont les mesures des côtés sont proportionnelles à $2$, $\Phi$ et $\Phi$.
\end{itemize}    

\begin{NewMacroBox}{tkzDefTriangle}{\oarg{local options}\parg{A,B}}
\emph{les points sont ordonnés car le triangle est construit en suivant le sens direct du cercle trigonométrique. Cette macro est soit utilisée en partenariat  avec \tkzcname{tkzGetPoint} soit en utilisant \tkzname{tkzPointResult} s'il n'est pas nécessaire de conserver le nom. }
  

\medskip
\begin{tabular}{lll}
\toprule
options             & défaut & définition                         \\ 
\midrule
\TOline{two angles= \#1 and \#2}{no defaut}{triangle connaissant deux angles} 
\TOline{equilateral} {no defaut}{triangle équilatéral }
\TOline{pythagore}{no defaut}{proportionnel au triangle de pythagore 3-4-5}
\TOline{school} {no defaut}{ angles de 30, 60 et 90 degrés }
\TOline{gold}{no defaut}{ angles de 72, 72 et 36 degrés, $A$ est le sommet }
\TOline{euclide} {no defaut}{identique au précédent mais $[AB]$ est la base}
\TOline{golden} {no defaut}{rectangle en B et $AB/AC = \Phi$} 
\TOline{cheops} {no defaut}{AC=BC, AC et BC sont proportionnels à $2$ et $\Phi$.} 
\end{tabular}
\end{NewMacroBox}  

\subsubsection{triangle doré (golden)}
\begin{tkzexample}[latex=6 cm,small]
\begin{tikzpicture}[scale=.8]
\tkzInit[xmax=5,ymax=3] \tkzClip[space=.5]
  \tkzDefPoint(0,0){A}      \tkzDefPoint(4,0){B}
  \tkzDefTriangle[golden](A,B)\tkzGetPoint{C}
  \tkzDrawPolygon(A,B,C) \tkzDrawPoints(A,B,C)
  \tkzLabelPoints(A,B) \tkzDrawBisector(A,C,B)
  \tkzLabelPoints[above](C) 
\end{tikzpicture}
\end{tkzexample} 

\subsubsection{triangle équilatéral}
\begin{tkzexample}[latex=7 cm,small]
\begin{tikzpicture}
  \tkzDefPoint(0,0){A}
  \tkzDefPoint(4,0){B}
  \tkzDefTriangle[equilateral](A,B) 
  \tkzGetPoint{C}
  \tkzDrawPolygon(A,B,C)
  \tkzDefTriangle[equilateral](B,A) 
  \tkzGetPoint{D}
  \tkzDrawPolygon(B,A,D)
  \tkzDrawPoints(A,B,C,D)
  \tkzLabelPoints(A,B,C,D) 
\end{tikzpicture}
\end{tkzexample} 

\subsubsection{triangle d'or  (euclide)}
\begin{tkzexample}[latex=7 cm,small] 
\begin{tikzpicture}
 \tkzDefPoint(0,0){A} \tkzDefPoint(4,0){B}
 \tkzDefTriangle[euclide](A,B)\tkzGetPoint{C}
 \tkzDrawPolygon(A,B,C)
 \tkzDrawPoints(A,B,C)
 \tkzLabelPoints(A,B)
 \tkzLabelPoints[above](C)
 \tkzDrawBisector(A,C,B) 
\end{tikzpicture}
\end{tkzexample}

\newpage
\subsection{Tracé des  triangles}          
 \begin{NewMacroBox}{tkzDrawTriangle}{\oarg{local options}\parg{A,B}}
\emph{Macro semblable à la macro précédente mais les côtés sont tracés.}

\medskip
\begin{tabular}{lll}
\toprule
options             & défaut & définition                         \\ 
\midrule
\TOline{two angles= \#1 and \#2}{no defaut}{triangle connaissant deux angles} 
\TOline{equilateral} {no defaut}{triangle équilatéral }
\TOline{pythagore}{no defaut}{proportionnel au triangle de pythagore 3-4-5}
\TOline{school} {no defaut}{les angles sont 30, 60 et 90 degrés }
\TOline{gold}{no defaut}{les angles sont 72, 72 et 36 degrés, $A$ est le sommet }
\TOline{euclide} {no defaut}{identique au précédent mais $[AB]$ est la base}
\TOline{golden} {no defaut}{rectangle en B et $AB/AC = \Phi$} 
\TOline{cheops} {no defaut}{isocèle en C et $AC/AB = \frac{\Phi}{2}$} 
\bottomrule
 \end{tabular}

\medskip
\emph{Dans toutes ses définitions, les dimensions du triangle dépendent des deux points de départ.}
\end{NewMacroBox}
 
 
\subsubsection{triangle de Pythagore}
Ce triangle a des côtés dont les longueurs sont proportionnelles à 3, 4 et 5.

\begin{tkzexample}[latex=6 cm,small]
\begin{tikzpicture}
 \tkzDefPoint(0,0){A}
 \tkzDefPoint(4,0){B}
 \tkzDrawTriangle[pythagore,fill=blue!30](A,B)
\end{tikzpicture}
\end{tkzexample}

 
 \subsubsection{triangle 30 60 90 (school)}
 Les angles font 30, 60 et 90 degrés.
 
\begin{tkzexample}[latex=6 cm,small]
\begin{tikzpicture}
  \tkzInit[ymin=-2.5,ymax=0,xmin=-5,xmax=0]
  \tkzClip[space=.5] 
    \begin{scope}[rotate=-180] 
  \tkzDefPoint(0,0){A} \tkzDefPoint(4,0){B}
  	\tkzDrawTriangle[school,fill=red!30](A,B) 
  \end{scope}
  \end{tikzpicture}
\end{tkzexample}

\newpage
\subsection{Les médianes}

 \begin{NewMacroBox}{tkzDrawMedian}{\oarg{local options}\parg{point,point}\parg{point}}
\emph{Il y aura sans doute une autre syntaxe pour ces segments.}

\medskip
\begin{tabular}{lll}
\toprule
arguments             & exemple & explication                         \\ 
\midrule
\TAline{\parg{pt1,pt2}\parg{pt3}}{\parg{A,B}\parg{C}}{[AB] est le segment cible C est le sommet}
\bottomrule
 \end{tabular}
\end{NewMacroBox}

\subsubsection{Médiane}
\begin{tkzexample}[latex=7 cm,small]
   \begin{tikzpicture}[scale=1.25]
    \tkzInit[xmin=0,xmax=4,ymin=0,ymax=3] \tkzClip 
    \tkzDefPoint(0,0){A} \tkzDefPoint(4,0){B}
    \tkzDefPoint(1,3){C} \tkzDrawPolygon(A,B,C)
    \tkzSetUpLine[color=blue]
    \tkzDrawMedian(A,B)(C)
    \tkzDrawMedian(A,C)(B)
    \tkzDrawMedian(B,C)(A)
   \end{tikzpicture}
\end{tkzexample}


\subsection{Les hauteurs}

 \begin{NewMacroBox}{tkzDrawAltitude}{\oarg{local options}\parg{point,point}\parg{point}}
\emph{Il y aura sans doute une autre syntaxe pour ces segments }

\medskip
\begin{tabular}{lll}
\toprule
options             & exemple & explication                         \\ 
\midrule
\TAline{\parg{pt1,pt2}\parg{pt3}}{\parg{A,B}\parg{C}}{[AB] est le segment cible C est le sommet}
\bottomrule
 \end{tabular}
\end{NewMacroBox}

\subsubsection{Hauteur}

\begin{tkzexample}[latex=7 cm,small]
\begin{tikzpicture}[scale=1.25]
 \tkzInit[xmin=0,xmax=4,ymin=0,ymax=3] \tkzClip 
 \tkzDefPoint(0,0){A} \tkzDefPoint(4,0){B}
 \tkzDefPoint(1,3){C} \tkzDrawPolygon(A,B,C)
 \tkzSetUpLine[color=magenta]
 \tkzDrawAltitude(A,B)(C)
 \tkzDrawAltitude(A,C)(B)
 \tkzDrawAltitude(B,C)(A)
\end{tikzpicture}
\end{tkzexample}

\subsection{Les bissectrices}

 \begin{NewMacroBox}{tkzDrawBisector}{\oarg{local options}\parg{point,point}\parg{point}}
\emph{Il faut donner l'angle dans le sens direct}

\medskip
\begin{tabular}{lll}
\toprule
options             & exemple & explication                         \\ 
\midrule
\TAline{\parg{pt1,pt2,pt3}}{\parg{A,B,C}}{Le sommet est B}
\bottomrule
 \end{tabular}
\end{NewMacroBox}

\subsubsection{Bissectrices dans un triangle}
Il faut donner les angles dans le sens direct.

\begin{tkzexample}[latex=7 cm,small]
\begin{tikzpicture}[scale=1.5]
 \tkzInit[xmin=0,xmax=4,ymin=0,ymax=3] \tkzClip 
 \tkzDefPoint(0,0){A} \tkzDefPoint(4,0){B}
 \tkzDefPoint(1,3){C} \tkzDrawPolygon(A,B,C)
 \tkzSetUpLine[color=purple]
 \tkzDrawBisector(C,B,A)
 \tkzDrawBisector(B,A,C)
 \tkzDrawBisector(A,C,B)
\end{tikzpicture}
\end{tkzexample}

\subsection{Le parallélogramme} 

Il n'y a pas de macro particulière pour tracer un parallélogramme. Le plus simple est d'employer 

 \tkzcname{tkzDefPointWith[colinear= at ..]}


\subsubsection{Exemple simple avec \tkzcname{colinear= at}}

\begin{tkzexample}[latex=7 cm,small]
\begin{tikzpicture}[scale=1.5]
 \tkzInit[xmin=0,xmax=4,ymin=0,ymax=2]
 \tkzClip[space=.5]   \tkzDefPoint(0,0){A} 
 \tkzDefPoint(3,0){B} \tkzDefPoint(4,2){C}
 \tkzDefPointWith[colinear= at C](B,A) 
 \tkzGetPoint{D}
 \tkzDrawPolygon(A,B,C,D)
 \tkzLabelPoints(A,B) 
 \tkzLabelPoints[above right](C,D)
\end{tikzpicture}
\end{tkzexample}

\subsubsection{Construction du rectangle d'or avec \tkzcname{colinear= at}}

\begin{tkzexample}[latex=7cm,small]
\begin{tikzpicture}[scale=.5]
  \tkzInit[xmax=14,ymax=10]
  \tkzClip[space=1]
  \tkzDefPoint(0,0){A}
  \tkzDefPoint(8,0){B}
  \tkzDefMidPoint(A,B)\tkzGetPoint{I}
  \tkzDefSquare(A,B)\tkzGetPoints{C}{D}
  \tkzDrawSquare(A,B)
  \tkzInterLC(A,B)(I,C)\tkzGetPoints{G}{E}
  \tkzDrawArc[style=dashed,color=gray](I,E)(D)
  \tkzDefPointWith[colinear= at C](E,B)
  \tkzGetPoint{F}
  \tkzDrawPoints(C,D,E,F)
  \tkzLabelPoints(A,B,C,D,E,F)
  \tkzDrawSegments[style=dashed,color=gray]%
(E,F C,F B,E)  
\end{tikzpicture}
\end{tkzexample}


\subsection{Définir les points d'un carré} 
 \begin{NewMacroBox}{tkzDefSquare}{\parg{pt1,pt2}}
 \emph{Le carré est défini dans le sens direct. À partir de deux points, on obtient deux autres points tel que les quatre pris dans l'ordre forme un carré. Le carré est défini dans le sens direct.    Les résultats sont dans \tkzname{tkzFirstPointResult} et \tkzname{tkzSecondPointResult}.\\
On peut les renommer avec \tkzcname{tkzGetPoints}}

\medskip
\begin{tabular}{lll}
\toprule
options             & exemple & explication                         \\ 
\midrule
\TAline{\parg{pt1,pt2}}{\tkzcname{tkzDefSquare}\parg{A,B}}{Le carré est défini dans le sens direct}
\bottomrule
 \end{tabular}
\end{NewMacroBox}

\subsubsection{Utilisation de \tkzcname{tkzDefSquare} avec deux points}

Il faut remarquer l'inversion des deux premiers points et le résultat.

\begin{tkzexample}[latex=4cm,small]
\begin{tikzpicture}[scale=.5]
  \tkzDefPoint(0,0){A} \tkzDefPoint(3,0){B}
  \tkzDefSquare(A,B)
  \tkzDrawPolygon[color=Maroon](A,B,tkzFirstPointResult,%
               tkzSecondPointResult)
  \tkzDefSquare(B,A)
  \tkzDrawPolygon[color=Gold](B,A,tkzFirstPointResult,%
               tkzSecondPointResult) 
\end{tikzpicture} 
\end{tkzexample}

 On peut n'avoir besoin que d'un point pour tracer un triangle isocèle rectangle alors on utilise \tkzcname{tkzGetFirstPoint} ou \tkzcname{tkzGetSecondPoint}

\subsubsection{Utilisation de \tkzcname{tkzDefSquare} pour obtenir un triangle isocèle rectangle}
\begin{tkzexample}[latex=7cm,small]
\begin{tikzpicture}[scale=1.5]
  \tkzDefPoint(0,0){A}
  \tkzDefPoint(3,0){B}
  \tkzDefSquare(A,B) \tkzGetFirstPoint{C}
  \tkzDrawPolygon[color=Maroon,fill=bistre](A,B,C)
\end{tikzpicture}
\end{tkzexample}

\subsubsection{Théorème de Pythagore et \tkzcname{tkzDefSquare} }
\begin{tkzexample}[latex=7cm,small]
\begin{tikzpicture}[scale=.75]
\tkzInit
\tkzDefPoint(0,0){C}
\tkzDefPoint(4,0){A}
\tkzDefPoint(0,3){B}
\tkzDefSquare(B,A)\tkzGetPoints{E}{F}
\tkzDefSquare(A,C)\tkzGetPoints{G}{H}
\tkzDefSquare(C,B)\tkzGetPoints{I}{J}
\tkzFillPolygon[fill = red!50 ](A,C,G,H)
\tkzFillPolygon[fill = blue!50 ](C,B,I,J)
\tkzFillPolygon[fill = purple!50](B,A,E,F)
\tkzFillPolygon[fill = orange,opacity=.5](A,B,C)
\tkzDrawPolygon[line width = 1pt](A,B,C)
\tkzDrawPolygon[line width = 1pt](A,C,G,H)
\tkzDrawPolygon[line width = 1pt](C,B,I,J)
\tkzDrawPolygon[line width = 1pt](B,A,E,F)
\tkzLabelSegment[above](C,A){$a$}
\tkzLabelSegment[right](B,C){$b$}
\tkzLabelSegment[below left](B,A){$c$}
\end{tikzpicture}
\end{tkzexample}
 
\newpage
\subsection{Tracé un carré} 

 \begin{NewMacroBox}{tkzDrawSquare}{\oarg{local options}\parg{pt1,pt2}}
 \emph{La macro trace un carré mais pas les sommets. Il est possible de colorier l'intérieur. L'ordre des points est celui du sens direct du cercle trigonométrique}

\medskip
\begin{tabular}{lll}
\toprule
options             & exemple & explication                         \\ 
\midrule
\TAline{\parg{pt1,pt2}}{\tkzcname{tkzDrawSquare}\parg{A,B}}{}
\bottomrule
 \end{tabular}
\end{NewMacroBox}

\subsubsection{Il s'agit d'inscrire deux carrés dans un demi-cercle.}

\begin{tkzexample}[latex=6 cm,small]
\begin{tikzpicture}[scale=.75] 
   \tkzInit[ymax=8,xmax=8]
 \tkzClip[space=.25]    \tkzDefPoint(0,0){A}
 \tkzDefPoint(8,0){B}  \tkzDefPoint(4,0){I}
 \tkzDefSquare(A,B)    \tkzGetPoints{C}{D}
 \tkzInterLC(I,C)(I,B) \tkzGetPoints{E'}{E}
 \tkzInterLC(I,D)(I,B) \tkzGetPoints{F'}{F} 
 \tkzDefPointsBy[projection=onto A--B](E,F){H,G}
 \tkzDefPointsBy[symmetry   = center H](I){J}
 \tkzDefSquare(H,J)    \tkzGetPoints{K}{L}
 \tkzDrawSector[fill=yellow](I,B)(A)
 \tkzFillPolygon[color=red!40](H,E,F,G)
 \tkzFillPolygon[color=blue!40](H,J,K,L)
 \tkzDrawPolySeg[color=red](H,E,F,G) 
 \tkzDrawPolySeg[color=red](J,K,L)
 \tkzDrawPoints(E,G,H,F,J,K,L)
\end{tikzpicture}
\end{tkzexample}

\subsection{Le rectangle d'or} 
 \begin{NewMacroBox}{tkzDefGoldRectangle}{\parg{point,point}}
\emph{La macro détermine un rectangle dont le rapport des dimensions est le nombre $\Phi$. Les points créés sont dans \tkzname{tkzFirstPointResult} et \tkzname{tkzSecondPointResult}. On peut les obtenir avec la macro \tkzcname{tkzGetPoints}. La macro suivante permet de tracer le rectangle.}

\begin{tabular}{lll}
\toprule
options             & exemple & explication                         \\
\midrule
\TAline{\parg{pt1,pt2}}{\parg{A,B}}{Si C et D sont créés alors $AB/BC=\Phi$}
 \end{tabular}
\end{NewMacroBox}

 \begin{NewMacroBox}{tkzDrawGoldRectangle}{\oarg{local options}\parg{point,point}}
\begin{tabular}{lll}
options             & exemple & explication                         \\
\midrule
\TAline{\parg{pt1,pt2}}{\parg{A,B}}{Trace le rectangle d'or basé sur le segment $[AB]$}
 \end{tabular}
\end{NewMacroBox}

% 
\subsubsection{Rectangles d'or}
 
\begin{tkzexample}[latex=6 cm,small]
\begin{tikzpicture}[scale=.6]
 \tkzDefPoint(0,0){A}      \tkzDefPoint(8,0){B}
 \tkzDefGoldRectangle(A,B) \tkzGetPoints{C}{D}
 \tkzDefGoldRectangle(B,C) \tkzGetPoints{E}{F}
 \tkzDrawPolygon[color=red,fill=red!20](A,B,C,D)
 \tkzDrawPolygon[color=blue,fill=blue!20](B,C,E,F)
\end{tikzpicture}
\end{tkzexample}

\subsection{Tracer un polygone} 

 \begin{NewMacroBox}{tkzDrawPolygon}{\oarg{local options}\parg{liste de points}}
\emph{Il suffit de donner une liste de points et la macro trace le polygone en utilisant les options de \TIKZ\ présentes.}

\begin{tabular}{lll}
\toprule
options             & exemple & explication                         \\
\midrule
\TAline{\parg{pt1,pt2}}{\parg{A,B}}{}
 \end{tabular}
\end{NewMacroBox}

\subsubsection{Tracer un polygone}

\begin{tkzexample}[latex=7 cm,small]
  \begin{tikzpicture}[rotate=25,scale=1.25]
\tkzDefPoints{-1/0/A,0/-2/B,4/0/C,0/1/D}
\tkzDrawPolygon[fill=green!50!blue,
line width=10pt,rounded corners](A,B,C,D)
\end{tikzpicture}
\end{tkzexample}

\begin{tkzexample}[latex=7cm, small]  
\begin{tikzpicture} [rotate=18,scale=1.5]
 \tkzDefPoint(0,0){A}
 \tkzDefPoint(2.25,0.2){B}
 \tkzDefPoint(2.5,2.75){C}
 \tkzDefPoint(-0.75,2){D}
 \tkzDrawPolygon[fill=black!50!blue!20!](A,B,C,D)
 \tkzDrawSegments[style=dashed](A,C B,D) 
\end{tikzpicture}\end{tkzexample}

\begin{tkzexample}[latex=7cm, small]
\begin{tikzpicture} [shift={(0,-5)},
                     rotate=-28,scale=1.5]
 \tkzDefPoint(0,0){A}
 \tkzDefPoint(2.25,0.2){C}
 \tkzDefPoint(2.5,2.75){B}
 \tkzDefPoint(-0.75,2){D}
 \tkzDrawPolygon[fill=black!50!blue!20!](A,B,C,D)   
 \tkzDrawSegments[style=dashed](A,C B,D)
\end{tikzpicture}\end{tkzexample}

\begin{tkzexample}[latex=7cm, small]
\begin{tikzpicture} [shift={(0,-9)},
                    rotate=-58,scale=1.5]
 \tkzDefPoint(1.5,1.5){A}
 \tkzDefPoint(2.25,0.2){B}
 \tkzDefPoint(2.5,2.75){C}
 \tkzDefPoint(-0.75,2){D}
 \tkzDrawPolygon[fill=black!50!blue!20!,%
    opacity=.5](A,B,C,D) 
 \tkzDrawSegments[style=dashed](A,C B,D)
\end{tikzpicture}
\end{tkzexample}

 
\subsection{Clipper un polygone} 
 \begin{NewMacroBox}{tkzClipPolygon}{\oarg{local options}\parg{liste de points}}
\emph{Cette macro permet de contenir les différentes tracés dans le polygone désigné.}

\medskip
\begin{tabular}{lll}
\toprule
options             & exemple & explication                         \\ 
\midrule
\TAline{\parg{pt1,pt2}}{\parg{A,B}}{}
%\bottomrule
 \end{tabular}
\end{NewMacroBox}
\subsubsection{Exemple simple avec \tkzcname{tkzClipPolygon}} 
\begin{tkzexample}[latex=7 cm,small]
\begin{tikzpicture}[scale=1.25]
 \tkzInit[xmin=0,xmax=4,ymin=0,ymax=3] 
 \tkzClip[space=.5] 
 \tkzDefPoint(0,0){A} \tkzDefPoint(4,0){B}
 \tkzDefPoint(1,3){C} \tkzDrawPolygon(A,B,C)
 \tkzDefPoint(0,2){D}  \tkzDefPoint(2,0){E}
 \tkzDrawPoints(D,E) \tkzLabelPoints(D,E) 
 \tkzClipPolygon(A,B,C)
 \tkzDrawLine[color=red](D,E)
\end{tikzpicture}
\end{tkzexample}

\subsubsection{Exemple Sangaku dans un carré} 
\begin{tkzexample}[latex=7cm, small]  
\begin{tikzpicture}[scale=.75]
 \tkzDefPoint(0,0){A} \tkzDefPoint(8,0){B}
 \tkzDefSquare(A,B) \tkzGetPoints{C}{D}
 \tkzDrawPolygon(B,C,D,A)
 \tkzClipPolygon(B,C,D,A)
 \tkzDefPoint(4,8){F}
 \tkzDefTriangle[equilateral](C,D) 
 \tkzGetPoint{I}
 \tkzDrawPoint(I)
 \tkzDefPointBy[projection=onto B--C](I) 
 \tkzGetPoint{J}
 \tkzInterLL(D,B)(I,J)  \tkzGetPoint{K}
 \tkzDefPointBy[symmetry=center K](B) 
 \tkzGetPoint{M}
 \tkzDrawCircle(M,I)
 \tkzCalcLength(M,I)   \tkzGetLength{dMI}
 \tkzFillPolygon[color = orange](A,B,C,D)
 \tkzFillCircle[R,color = yellow](M,\dMI pt)
 \tkzFillCircle[R,color = blue!50!black](F,4 cm)%
\end{tikzpicture}
\end{tkzexample}
 
\subsection{Colorier un polygone} 
 \begin{NewMacroBox}{tkzFillPolygon}{\oarg{local options}\parg{liste de points}}
    \emph{On peut colorier en traçant le polygone mais là on colorie l'intrieur du polygone sans le tracer.}
    
    \medskip
\begin{tabular}{lll}
\toprule
options             & exemple & explication                         \\ 
\midrule
\TAline{\parg{pt1,pt2,\dots}}{\parg{A,B,\dots}}{}
%\bottomrule
 \end{tabular}
\end{NewMacroBox} 

\subsubsection{Colorier un polygone} 
\begin{tkzexample}[latex=7cm, small]  
\begin{tikzpicture}[scale=0.7]
\tkzInit[xmin=-3,xmax=6,ymin=-1,ymax=6]
\tkzDrawX[noticks]
\tkzDrawY[noticks]    
\tkzDefPoint(0,0){O}  \tkzDefPoint(4,2){A}
\tkzDefPoint(-2,6){B}
\tkzPointShowCoord[xlabel=$x$,ylabel=$y$](A)
\tkzPointShowCoord[xlabel=$x'$,ylabel=$y'$,%
                   ystyle={right=2pt}](B) 
\tkzDrawVectors(O,A O,B)
\tkzLabelSegment[above=3pt](O,A){$\vec{u}$}
\tkzLabelSegment[above=3pt](O,B){$\vec{v}$}
\tkzMarkAngle[fill= yellow,size=1.8cm,%
              opacity=.5](A,O,B)
\tkzFillPolygon[red!30,opacity=0.25](A,B,O)
\tkzLabelAngle[pos = 1.5](A,O,B){$\alpha$} 
\end{tikzpicture}
\end{tkzexample}
\endinput  
%!TEX root = /Users/ego/Boulot/TKZ/tkz-euclide/doc_fr/TKZdoc-euclide-main.tex

\section{Les Cercles}

Parmi les macros suivantes, l'une  va permettre de tracer un cercle, ce qui n'est pas un réel exploit. Pour cela, il va falloir connaître le centre du cercle et soit le rayon du cercle, soit un point de la circonférence. Il m'a semblé que l'utilisation la plus fréquente était de tracer un cercle de centre donné passant par un point donné. Ce sera la méthode par défaut, sinon il faudra utiliser l'option \tkzname{R}. Il existe un grand nombre de cercles particuliers, par exemple le cercle circonscrit à un triangle.

\begin{itemize}
  \item  J'ai  créé une première macro \tkzcname{tkzDefCircle} qui permet en fonction d'un cercle
 particulier de récupérer son centre et la mesure du rayon en cm. Cette récupération  se fait avec les macros \tkzcname{tkzGetPoint} et \tkzcname{tkzGetLength},
 
 \item ensuite une macro \tkzcname{tkzDrawCircle},
 
 \item puis une macro qui permet de colorier un disque, mais sans tracer le cercle \tkzcname{tkzFillCircle},
 
 \item parfois, il est nécessaire qu'un dessin soit contenu dans un disque c'est le rôle attribuer à \tkzcname{tkzClipCircle},

 
 \item  Il reste enfin à pouvoir donner un label pour désigner un cercle et si plusieurs possibilités sont offertes, nous verrons ici \tkzcname{tkzLabelCircle}.
\end{itemize}
 

\subsection{Caractéristiques d'un cercle : \tkzcname{tkzDefCircle}}
 
Pour le moment, il est possible de récupérer les caractéristiques des cercles suivants (le premier est là pour que l'ensemble soit homogène)
\begin{itemize}
\item  \tkzname{radius}  cercle caractérisé par deux points définissant un rayon,
\item  \tkzname{diameter}  cercle caractérisé par  deux points définissant un diamètre,
\item \tkzname{circum} cercle circonscrit à un triangle,
\item \tkzname{in} cercle inscrit dans à un triangle,
\item \tkzname{euler} cercle d'Euler d'un triangle,
\item \tkzname{apollonius} cercle d'Apollonius caractérisé par un segment et un ratio.  
\end{itemize}   

\begin{NewMacroBox}{tkzDefCircle}{\oarg{local options}\parg{A,B} ou \parg{A,B,C}}
\emph{Attention les arguments sont des listes de deux ou bien de trois points. Cette macro est, soit utilisée en partenariat  avec \tkzcname{tkzGetPoint} et/ou \tkzcname{tkzGetLength} pour obtenir le centre et le rayon du cercle, soit en utilisant \tkzname{tkzPointResult} et \tkzname{tkzLengthResult} s'il n'est pas nécessaire de conserver les résultats.}
  

\medskip
\begin{tabular}{lll}
\toprule
options             & défaut & définition                         \\ 
\midrule
\TOline{radius}  {radius}{cercle caractérisé par deux points définissant un rayon} 
\TOline{diameter} {radius}{cercle caractérisé par  deux points définissant un diamètre }
\TOline{circum}{radius}{cercle circonscrit à un triangle} 
\TOline{in}    {radius}{cercle inscrit dans à un triangle } 
\TOline{euler}{radius}{Cercle d'Euler }
\TOline{apollonius} {radius}{Cercle d'Apollonius} 
\TOline{orthogonal} {radius}{Cercle de centre donné orthogonal à un autre cercle}
\TOline{orthogonal through}{radius}{Cercle orthogonal à un autre cercle passant par deux points} 
\TOline{K} {2}{Coefficient utilisé pour un cercle d'Apollonius} 
\TOline{color}   {black}{couleur du cercle} 
\TOline{fill}   {}{couleur du disque, si présent }  
\TOline{line width}   {.4pt}{épaisseur du trait }    \bottomrule
\end{tabular}

\medskip\emph{Dans les exemples suivants, je trace les cercles avec une macro pas encore présentée, mais ce n'est pas nécessaire. Dans certains cas on peut seulement avoir besoin du centre ou encore du rayon.}
\end{NewMacroBox}  

\subsubsection{Exemple}
\begin{tkzexample}[latex=7 cm]
\begin{tikzpicture}
   \tkzDefPoint(0,4){A}
   \tkzDefPoint(3,2){B}
   \tkzDefCircle[radius](A,B) 
   \tkzGetLength{rABpt}
   \tkzpttocm(\rABpt){rABcm}
   \tkzDrawCircle(A,B)
   \tkzDrawPoints(A,B)
   \tkzLabelPoints(A,B)
   \tkzLabelCircle[draw,fill=Gold,%
    text width=3cm,text centered](A,B)(-90)%
  {La mesure du rayon est : 
  \rABpt pt soit \rABcm cm}
\end{tikzpicture} 
 \end{tkzexample}   
% 
 \subsubsection{Exemple avec un point aléatoire} 
% 
\begin{tkzexample}[latex=7 cm]
 \begin{tikzpicture}
    \tkzDefPoint(0,4){A}
    \tkzDefPoint(3,2){B}
    \tkzDefMidPoint(A,B) \tkzGetPoint{I}
    \tkzGetRandPointOn[segment = I--B]{C}
    \tkzDefCircle[radius](A,C) 
   \tkzGetLength{rACpt}
   \tkzpttocm(\rACpt){rACcm} 
    \tkzDrawCircle(A,C)
    \tkzDrawPoints(A,B,C)
    \tkzLabelPoints(A,B,C) 
    \tkzLabelCircle[draw,fill=Gold,%
    text width=3cm,text centered](A,C)(-90)%
    {La mesure du rayon est :
     \rACpt pt soit \rACcm cm}  
 \end{tikzpicture}     
 \end{tkzexample}      

\newpage
 \subsubsection{Cercles inscrit et circonscrit pour un triangle donné}
\begin{tkzexample}[vbox]  
\begin{tikzpicture}[scale=1.5]
   \tkzDefPoint(2,2){A} 
   \tkzDefPoint(5,-2){B}
   \tkzDefPoint(1,-2){C}
   \tkzDefCircle[in](A,B,C)
   \tkzGetPoint{I}    \tkzGetLength{rIN}
   \tkzDefCircle[circum](A,B,C)
   \tkzGetPoint{K}   \tkzGetLength{rCI}
   \tkzDrawPoints(A,B,C,I,K)    
   \tkzDrawCircle[R,blue](I,\rIN pt) 
   \tkzDrawCircle[R,red](K,\rCI pt) 
   \tkzLabelPoints[below](B,C)
   \tkzLabelPoints[above left](A,I,K)
   \tkzDrawPolygon(A,B,C)
\end{tikzpicture} 
\end{tkzexample}

   
\newpage
 \subsubsection{Cercles d'Apollonius colorié pour un segment donné} 
Wikipedia donne comme définition :
 
Apollonius de Perga propose de définir le cercle comme l'ensemble des points M du plan pour lesquels le rapport des distances MA/MB reste constant, les points A et B étant donnés. 
Théorème — Si A et B sont deux points distincts et $k$ est un réel autre que 0 et 1, le cercle d'Apollonius du triplet (A,B,$k$) est l'ensemble des points M du plan tels que MA/MB = $k$.
                                      

\begin{tkzexample}[vbox]    
\begin{tikzpicture}[scale=1.25]
  \tkzDefPoint(0,0){A} 
  \tkzDefPoint(4,0){B}
  \tkzDefCircle[apollonius,K=2](A,B)
  \tkzGetPoint{K1}
  \tkzGetLength{rAp}
  \tkzDrawCircle[R,color = blue!50!black,fill=blue!20,opacity=.4](K1,\rAp pt)
  \tkzDefCircle[apollonius,K=3](A,B)
  \tkzGetPoint{K2}   \tkzGetLength{rAp}
  \tkzDrawCircle[R,color=red!50!black,fill=red!20,opacity=.4](K2,\rAp pt) 
  \tkzLabelPoints[below](A,B,K1,K2)
  \tkzDrawPoints(A,B,K1,K2) 
  \tkzDrawLine[add=.2 and 1](A,B)  
\end{tikzpicture}
\end{tkzexample}   

Les cercles ont été tracés et les disques coloriés, simplement avec les outils de \TIKZ.       

\newpage
 \subsubsection{Cercle d'Euler pour un triangle donné}
\begin{center}
\begin{tkzexample}[vbox]  
\begin{tikzpicture}[scale=1.5]
   \tkzInit[xmin=-1,ymin=-1,xmax=8,ymax=6] \tkzClip
   \tkzDefPoint(5,3.5){A} \tkzDefPoint(0,0){B} \tkzDefPoint(7,0){C}
   \tkzDefCircle[euler](A,B,C)
   \tkzGetPoint{E}  \tkzGetLength{rEuler}
   \tkzDrawPoints(A,B,C,E)    
   \tkzDrawCircle[R,blue](E,\rEuler pt)
   \tkzDrawPolygon(A,B,C)    
   \tkzLabelPoints[below](B,C)  \tkzLabelPoints[left](A,E)   
\end{tikzpicture}
\end{tkzexample}
 \end{center}  
     
 Il est possible avec les outils d'intersection de déterminer les points communs du cercle d'Euler et du triangle.   

\newpage
\subsubsection{Cercle orthogonal de centre donné} 
Nous allons chercher deux cercles orthogonaux au  cercle de centre O passant par A, leurs centres B et C étant donnés.   
 
\begin{center}
  \begin{tkzexample}[vbox]  
\begin{tikzpicture}[scale=1.5]
  \tkzDefPoint(0,0){O}  \tkzDefPoint(1,0){A} 
  \tkzDefPoint(1.5,1.25){B}  \tkzDefPoint(-2,-3){C} 
  \tkzDrawCircle(O,A)
  \tkzDefCircle[orthogonal from=B](O,A)
  \tkzDrawCircle[thick,color=red](B,tkzFirstPointResult)
  \tkzDefCircle[orthogonal from=C](O,A)
  \tkzDrawCircle[thick,color=red](C,tkzFirstPointResult)
  \tkzDrawPoints(tkzFirstPointResult,tkzSecondPointResult,O,A,B,C) 
  \tkzLabelPoints(O,A,C,B)  
\end{tikzpicture} 
\end{tkzexample}
\end{center}  

\newpage  
 \subsubsection{Cercle orthogonal passant par deux points donnés} 
 Nous allons cette fois récupéré le centre.  
 
\begin{center}
\begin{tkzexample}[vbox]  
\begin{tikzpicture}[scale=3]
  \tkzDefPoint(0,0){O}
  \tkzDefPoint(1,0){A}
  \tkzDrawCircle(O,A) 
  \tkzDefPoint(-1.5,-1.5){z1}
  \tkzDefPoint(1.5,-1.25){z2} 
  \tkzDefCircle[orthogonal through=z1 and z2](O,A) \tkzGetPoint{c}    
  \tkzDrawCircle[thick,color=red](tkzPointResult,z1)
  \tkzDrawPoints[fill=red,color=black,size=4](O,A,z1,z2,c)   
  \tkzLabelPoints(O,A,z1,z2,c) 
\end{tikzpicture}   
\end{tkzexample}
\end{center}  


\newpage     
\subsection{Tracer un cercle}  
\begin{NewMacroBox}{tkzDrawCircle}{\oarg{local options}\parg{A,B} ou \parg{A,B,C}}
\noindent\emph{Attention les arguments sont des listes de deux ou bien de trois points. Les cercles que l'on peut tracer sont les mêmes que pour la macro précédente. Une option supplémentaire \tkzname{R} afin de donner directement une mesure.}
  

\medskip
\begin{tabular}{lll}
\toprule
options             & défaut & définition                         \\ 
\midrule
\TOline{radius}{radius}{cercle avec deux points définissant un rayon}
\TOline{diameter}{radius}{cercle avec deux points définissant un diamètre}
\TOline{R} {radius}{cercle caractérisé par  un point et la mesure d'un rayon}
\TOline{circum}{radius}{cercle circonscrit à un triangle}
\TOline{in}{radius}{cercle inscrit dans à un triangle}
\TOline{euler}{radius}{Le cercle d'Euler}
\TOline{apollonius}{radius}{Le cercle d'Apollonius}
\TOline{K}{2}{Coefficient utilisé pour un cercle d'Apollonius}
\TOline{orthogonal}{radius}{Cercle de centre donné orthogonal à un autre cercle}
\TOline{orthogonal through}{radius}{Cercle orthogonal à un autre cercle passant par deux points}    
 \bottomrule
\end{tabular}

\medskip
\emph{Il faut ajouter bien sûr tous les styles de \TIKZ pour les tracés}
\end{NewMacroBox}
 
 \subsubsection{Cercles et styles, tracer un cercle et colorier le disque}
 On va voir qu'il est possible de colorier un disque, tout en traçant le cercle.
 
\begin{tkzexample}[latex=7cm]
\begin{tikzpicture}
  \tkzDefPoint(0,0){O} 
  \tkzDefPoint(3,0){A}
 % cercle de centre O et passant par A
  \tkzDrawCircle[color=blue,style=dashed](O,A) 
 % cercle de diamètre $[OA]$
  \tkzDrawCircle[diameter,color=red,%
                 line width=2pt,fill=red!40,%
                 opacity=.5](O,A)
 % cercle de centre O et de rayon = exp(1) cm
  \FPeval\rayon{exp(1)}
  \tkzDrawCircle[R,color=orange](O,\rayon cm) 
\end{tikzpicture} 
\end{tkzexample}  

 \subsubsection{Cercle orthogonal à un cercle donné passant par deux points donnés } 
 
\begin{center}
\begin{tkzexample}[vbox]
\begin{tikzpicture}[scale=2]
  \tkzDefPoint(0,0){O}
  \tkzDefPoint(1,0){A}
  \tkzDrawCircle(O,A) 
  \tkzDefPoint(0.5,-0.25){z1}
  \tkzDefPoint(-0.5,-0.5){z2}
  \tkzDrawPoints[color = black,fill   = red,size=12](O,z1,z2)
  \tkzDefPointBy[inversion = center O through A](z1) \tkzGetPoint{Z1} 
  \tkzCircumCenter(z1,z2,Z1) \tkzGetPoint{c}
  \tkzDrawCircle(c,Z1)
  \tkzDrawPoints(c,Z1)
  \tkzLabelPoints(O,A,z1,z2,Z1,c) 
\end{tikzpicture} 
\end{tkzexample}
\end{center}

 
 \newpage
 \subsubsection{Cardioïde}  
 D'après une idée d'O. Reboux réalisée avec pst-eucl ( module de Pstricks)  de D. Rodriguez.

Son nom vient du grec kardia (cœur), en référence à sa forme, et lui fut donné par Johan Castillon. Wikipedia     
% 
% %   BibTeX
% % 
% %  @misc{ wiki:xxx,
% %    author = "Wikipédia",
% %    title = "Cardioïde --- Wikipédia{,} l'encyclopédie libre",
% %    year = "2010",
% %    url = "http://fr.wikipedia.org/w/index.php?title=Cardio%C3%AFde&oldid=53868968",
% %    note = "[En ligne; Page disponible le 27-juin-2010]"
% %  }
% % 
% % Lorsque vous utilisez la package url sous LaTeX (\usepackage{url} quelque part dans le préambule) qui améliore l'affichage des adresses internet, veuillez utiliser de préférence ce format:
% % 
% % 
% %  @misc{ wiki:xxx,
% %    author = "Wikipédia",
% %    title = "Cardioïde --- Wikipédia{,} l'encyclopédie libre",
% %    year = "2010",
% %    url = "\url{http://fr.wikipedia.org/w/index.php?title=Cardio%C3%AFde&oldid=53868968}",
% %    note = "[En ligne; Page disponible le 27-juin-2010]"
% %  }   

\begin{center}
  \begin{tkzexample}[vbox]
  \begin{tikzpicture}[scale=1.25]
    \tkzDefPoint(0,0){O} 
    \tkzDefPoint(2,0){A}
    \foreach \ang in {5,10,...,360}{%
       \tkzDefPoint(\ang:2){M}
       \tkzDrawCircle(M,A) 
     }  
  \end{tikzpicture} 
  \end{tkzexample}
\end{center}
  

 \subsubsection{Ceci est une mappemonde }
  
\begin{tkzexample}[latex=5.5cm,small]
\begin{tikzpicture}[scale=.333]   
 \tkzInit[xmin=-10,xmax=10,ymin=-10,ymax=10]
 \tkzDefPoint(0 , 0){O}
 \tkzDefPoint(9 , 0){A}
 \tkzDefPoint(-9, 0){C} 
 \tkzDefPoint(0 , 9){B}
 \tkzDefPoint(0 ,-9){D}
 \tkzClipCircle(O,A) 
 \foreach \pti in {1,2,...,8}{
 \tkzDefPoint(10*\pti:9){P\pti}
 \tkzDefPoint(90:\pti){MP\pti}
 \tkzDefPoint(0: \pti){NP\pti}
 \tkzDefLine[mediator](MP\pti,P\pti) 
 \tkzInterLL(B,D)(tkzFirstPointResult,tkzSecondPointResult) 
 \tkzDrawCircle[color=Maroon](tkzPointResult,P\pti)
 } 
 \foreach \pti in {-1,-2,...,-8}{
 \tkzDefPoint(10*\pti:9){P\pti}
 \tkzDefPoint(-90:-\pti){MP\pti}
 \tkzDefPoint(0: -\pti){NP\pti}
 \tkzDefLine[mediator](MP\pti,P\pti)
 \tkzInterLL(B,D)(tkzFirstPointResult,tkzSecondPointResult)
 \tkzDrawCircle[color=Maroon](tkzPointResult,P\pti)
 } 
 \foreach \pti in {1,2,...,8}{
 \tkzDefLine[mediator](B,NP\pti)  
 \tkzInterLL(A,C)(tkzFirstPointResult,tkzSecondPointResult)
 \tkzDrawCircle[color=Maroon](tkzPointResult,NP\pti)
 }
 \foreach \pti in {1,2,...,8}{
 \tkzDefPoint(0: -\pti){NP\pti}
 \tkzDefLine[mediator](B,NP\pti) 
 \tkzInterLL(A,C)(tkzFirstPointResult,tkzSecondPointResult)
 \tkzDrawCircle[color=Maroon](tkzPointResult,NP\pti)
 }  
  \tkzDrawCircle[R,color=Maroon](O,9 cm)
  \tkzDrawSegments[color=Maroon](A,C B,D)  
\end{tikzpicture}     
 \end{tkzexample}


\clearpage \newpage


\newpage     
\subsection{Colorier un disque}
C'était possible avec la macro précédente, mais le tracé du disque était obligatoire, là ce n'est plus le cas.
  
\begin{NewMacroBox}{tkzFillCircle}{\oarg{local options}\parg{A,B}}
\begin{tabular}{lll}
options             & défaut & définition                         \\ 
\midrule
\TOline{radius}  {radius}{deux points définissent un rayon}
\TOline{R} {radius}{un point et la mesure d'un rayon }
\bottomrule
\end{tabular}

\medskip
\emph{Il n'est pas nécessaire de mettre \tkzname{radius} car c'est l'option par défaut. Il faut ajouter bien sûr tous les styles de \TIKZ pour les tracés}
\end{NewMacroBox}  

 \subsubsection{Exemple de \tkzcname{tkzFillCircle} provenant d'un sangaku} 
\begin{center}
  \begin{tkzexample}[vbox,small]
  \begin{tikzpicture}
   \tkzInit[xmin=0,xmax = 6,ymin=0,ymax=6] \tkzClip
   \tkzDefPoint(0,0){B}  \tkzDefPoint(6,0){C}%
   \tkzDefSquare(B,C)    \tkzGetPoints{D}{A} 
   \tkzClipPolygon(B,C,D,A) 
   \tkzDefMidPoint(A,D)  \tkzGetPoint{F}
   \tkzDefMidPoint(B,C)  \tkzGetPoint{E}
   \tkzDefMidPoint(B,D)  \tkzGetPoint{Q}           
   \tkzTangent[from = B](F,A) \tkzGetPoints{G}{H} 
   % \tkzTgtFromP(F,A)(B) est obsolète
   \tkzInterLL(F,G)(C,D) \tkzGetPoint{J}
   \tkzInterLL(A,J)(F,E) \tkzGetPoint{K}
   \tkzDefPointBy[projection=onto B--A](K)   \tkzGetPoint{M}  
   \tkzFillPolygon[color = green](A,B,C,D)
   \tkzFillCircle[color = orange](B,A)
   \tkzFillCircle[color = blue!50!black](M,A)
   \tkzFillCircle[color = purple](E,B)
   \tkzFillCircle[color = yellow](K,Q)  
  \end{tikzpicture}
  \end{tkzexample} 
\end{center}

  
\newpage     
\subsection{Clipper un disque}

\begin{NewMacroBox}{tkzClipCircle}{\oarg{local options}\parg{A,B}}
\begin{tabular}{lll}
%\toprule
options             & défaut & définition                         \\ 
\midrule
\TOline{radius}  {radius}{cercle caractérisé par deux points définissant un rayon} 
\TOline{R} {radius}{cercle caractérisé par  un point et la mesure d'un rayon }  
\bottomrule
\end{tabular}

\medskip
\emph{Il n'est pas nécessaire de mettre \tkzname{radius} car c'est l'option par défaut.}
\end{NewMacroBox}

 \subsubsection{Exemple 1 de \tkzcname{tkzClipCircle}} 
\begin{tkzexample}[latex=6cm,small] 
  \begin{tikzpicture}
  \tkzInit[xmax=5,ymax=5]
  \tkzGrid \tkzClip 
  \tkzDefPoint(0,0){A}
  \tkzDefPoint(2,2){O}
  \tkzDefPoint(4,4){B}
  \tkzDefPoint(6,6){C}
  \tkzDrawPoints(O,A,B,C) 
  \tkzLabelPoints(O,A,B,C)
  \tkzDrawCircle(O,A) 
  \tkzClipCircle(O,A)
  \tkzDrawLine(A,C)
  \tkzDrawCircle[fill=red!20,opacity=.5](C,O) 
\end{tikzpicture} 
\end{tkzexample}

 \subsubsection{Exemple 2 de \tkzcname{tkzClipCircle}}
\begin{tkzexample}[latex=6cm,small] 
  \begin{tikzpicture}
  \tkzInit[xmax=6,ymax=6]
  \tkzGrid \tkzClip
   \tkzDefPoint(0,0){A}
   \tkzDefPoint(2,2){O}
   \tkzDefPoint(4,4){B}
   \tkzDefPoint(6,6){C}
   \tkzDrawPoints(O,A,B,C) 
   \tkzLabelPoints(O,A,B,C)
   \tkzDrawCircle(O,A) 
   \begin{scope}
    \tkzClipCircle(O,A)
    \tkzDrawLine(A,C)
   \end{scope}
   \tkzClipCircle[R](B,1cm)
  \tkzDrawCircle[fill=red!20,opacity=.5](C,B)
\end{tikzpicture} 
\end{tkzexample}  

 \subsubsection{Exemple 3 de \tkzcname{tkzClipCircle}}
\begin{tkzexample}[latex=8cm,small]  
\begin{tikzpicture}[scale=.5]
 \tkzDefPoint(0,0){A} 
 \tkzDefPoint(8,0){B}
 \tkzDefSquare(A,B)\tkzGetPoints{C}{D} 
 \tkzDrawPolygon(A,B,C,D)
 \tkzClipPolygon(A,B,C,D)
 \begin{scope}
  \tkzClipCircle(D,C)
  \tkzFillCircle[color=gray!50,%
                opacity=.5](B,A) 
 \end{scope}
 \tkzDrawCircle(B,C) 
 \tkzDrawCircle(D,C)
\end{tikzpicture}
\end{tkzexample}

 \subsubsection{Exemple 4 de \tkzcname{tkzClipCircle} provenant d'un sangaku}
\begin{tkzexample}[latex=7cm,small] 
\begin{tikzpicture}[scale=.75]
 \tkzInit[xmin=-5,ymin=-5,xmax=5,ymax=5]
 \tkzClip
 \tkzDefPoint(0,0){O}
 \tkzDefPoint(-2,-3){A}
 \tkzDefPoint(2,-3){B}
 \tkzDefPoint(0,3){Q}
 \tkzDrawCircle[R](O,5 cm)
 \tkzInterLC[R](A,B)(O,5 cm) 
     \tkzGetPoints{M}{N}
 \tkzDrawPoints(M,N)
 \tkzClipCircle[R](O,5 cm)
 \tkzDrawLines[add= 1 and 1](A,B M,Q N,Q)
 \tkzDefMidPoint(M,N) \tkzGetPoint{R}
 \tkzDefLine[orthogonal=through Q](O,Q)
 \tkzGetPoint(q)
 \tkzCalcLength(R,Q) \tkzGetLength{dRQ}
 \tkzCalcLength(M,Q) \tkzGetLength{dMQ} 
 \pgfmathparse{(\dMQ)/(\dRQ)*1.5}
 \edef\tkz@q{\pgfmathresult}%
 \tkzDefPoint(\tkz@q,3){K}
 \tkzDefPointBy[projection=onto N--Q](K) 
    \tkzGetPoint{G}
  \tkzDrawCircle[R](K,1.5cm)
  \tkzFillCircle[R,color=purple!50,%
  opacity=.5](K,1.5 cm)
\end{tikzpicture}
  \end{tkzexample}

\newpage
\subsection{Donner un label à un cercle}
\begin{NewMacroBox}{tkzLabelCircle}{\oarg{local options}\parg{A,B}\parg{angle}\marg{label}}
\begin{tabular}{lll}
%\toprule
options             & défaut & définition                         \\
\midrule
\TOline{radius}  {radius}{cercle caractérisé par deux points définissant un rayon}
\TOline{R} {radius}{cercle caractérisé par  un point et la mesure d'un rayon }
\bottomrule
\end{tabular} 

\medskip
\emph{Il n'est pas nécessaire de mettre \tkzname{radius} car c'est l'option par défaut. On peut utiliser les styles de \TIKZ. Le label est créé et donc "passé" entre accolades. }
\end{NewMacroBox} 

 \subsubsection{Exemple  de \tkzcname{tkzLabelCircle}}  
\begin{tkzexample}[latex=5cm,small] 
\begin{tikzpicture}
  \tkzInit[ymin=-2.25,ymax=2.25,xmin=-2.25,xmax=2.25]
  \tkzDefPoint(0,0){O}
  \tkzDefPoint(2,0){N}
  \tkzDefPointBy[rotation=center O angle 50](N) 
      \tkzGetPoint{M}
  \tkzDefPointBy[rotation=center O angle -20](N) 
       \tkzGetPoint{P}
  \tkzDefPointBy[rotation=center O angle 125](N) 
       \tkzGetPoint{P'}
  \tkzLabelCircle[above=4pt](O,N)(120){$\mathcal{C}$}
  \tkzDrawCircle(O,M) 
  \tkzFillCircle[color=blue!20,opacity=.4](O,M) 
  \tkzLabelCircle[R,draw,fill=Gold,%
  text width=2cm,text centered](O,3 cm)(-60)%
          {Le cercle\\ $\mathcal{C}$}  
  \tkzDrawSegment[dashed](O,P)
  \tkzDrawPoints(M,P)\tkzLabelPoints[right](M,P)   
\end{tikzpicture}      
\end{tkzexample} 

\subsection{Tangente à un cercle}
Deux constructions sont proposées. La première est la construction d'une tangente à un cercle en un point donné de ce cercle et la seconde est la construction d'une tangente à un cercle passant par un point donné hors d'un disque. Ces macros remplacent d'anciennes macros qui existent encore \tkzcname{tkzTgtFromP} ou  \tkzcname{tkzTgtFromPR} ainsi que   \tkzcname{tkzTgtAt}. 

\begin{NewMacroBox}{tkzTangent}{\oarg{local options}\parg{pt1,pt2} ou \parg{pt1,dim}}
\emph{Le paramètre entre parenthèses est le centre du cercle ou bien le centre du cercle et un point du cercle ou encore le centre et le rayon.}

\begin{tabular}{lll}
\toprule
options             & défaut & définition                         \\ 
\midrule
\TOline{at=pt}{at}{tangente en un point du cercle} 
\TOline{from=pt} {at}{tangente à un cercle passant par un point}
\TOline{from with R=pt} {at}{idem, mais le cercle est défini par centre+rayon}  
\bottomrule
\end{tabular}

\emph{La tangente n'est pas tracée.Un second point de celle-ci est donné par \tkzname{tkzPointResult}.}
\end{NewMacroBox}

 \subsubsection{Exemple  de tangente passant par un point du cercle } 
\begin{tkzexample}[latex=7cm,small]
\begin{tikzpicture}[scale=.5]
  \tkzInit
  \tkzDefPoint(0,0){O}
   \tkzDefPoint(6,6){E}
  \tkzGetRandPointOn[circle=center O radius 4cm]{A}
  \tkzDrawSegment(O,A)
   \tkzDrawCircle(O,A)
  \tkzTangent[at=A](O) 
   \tkzGetPoint{h}
  \tkzTangent[from=E](O,A)  \tkzGetPoints{e}{f} 
    \tkzTangent[from with R=E](O,4 cm)  
    \tkzGetPoints{k}{l} 
    \tkzDrawLine[add = 5 and 4](A,h)
    \tkzMarkRightAngle[fill=red!30](O,A,h)
    \tkzDrawLines[](E,e E,l)
\end{tikzpicture}
\end{tkzexample} 


 \subsubsection{Exemple  de tangentes passant par un point extérieur } 

\begin{tkzexample}[latex=6cm,small]
\begin{tikzpicture}[scale=0.75] 
  \tkzDefPoint(3,3){c}
  \tkzDefPoint(6,3){a0}  
  \tkzRadius=1 cm 
  \tkzDrawCircle[R](c,\tkzRadius) 
\foreach \an in {0,10,...,350}{
    \tkzDefPointBy[rotation=center c angle \an](a0)  
    \tkzGetPoint{a}  
    \tkzTangent[from with R = a](c,\tkzRadius)  
    \tkzGetPoints{e}{f} 
    \tkzDrawLines[color=magenta](a,f a,e) 
   \tkzDrawSegments(c,e c,f)}
\end{tikzpicture} 
\end{tkzexample}

\newpage
 \subsubsection{Exemple  d'Andrew Mertz }

\begin{tkzexample}[vbox]
 \begin{tikzpicture}[scale=1] 
   \tkzInit[xmin=-4.1,xmax=5.2,ymin=-4.1,ymax=8]
   \tkzClip[space=.5]
   \tkzDefPoint(100:8){A}\tkzDefPoint(50:8){B}  
   \tkzDefPoint(0,0){C} \tkzDefPoint(0,4){R} 
   \tkzDrawCircle(C,R)
   \tkzTangent[from = A](C,R)  \tkzGetPoints{D}{E}
   \tkzTangent[from = B](C,R)  \tkzGetPoints{F}{G}
   \tkzDrawSector[fill=blue!80!black,opacity=0.5](A,D)(E)
   \tkzFillSector[color=red!80!black,opacity=0.5](B,F)(G)
   \tkzInterCC(A,D)(B,F) \tkzGetSecondPoint{I}
   \tkzDrawPoint[color=black](I)
 \end{tikzpicture}
\end{tkzexample}
\url{http://www.texample.net/tikz/examples/}  

\endinput 
%!TEX root = /Users/ego/Boulot/TKZ/tkz-euclide/doc_fr/TKZdoc-euclide-main.tex 

\section{Utilisation du compas}    

\subsection{Macro principale \tkzcname{tkzCompass}} 
\begin{NewMacroBox}{tkzCompass}{\oarg{local options}\parg{A,B}}
\emph{Attention les arguments sont des listes de deux ou bien de trois points. Cette macro est, soit utilisée en partenariat  avec \tkzcname{tkzGetPoint} et/ou \tkzcname{tkzGetLength}, soit en utilisant \tkzname{tkzPointResult} s'il n'est pas nécessaire de conserver le nom.}
  

\medskip
\begin{tabular}{lll}
\toprule
options             & défaut & définition                         \\ 
\midrule
\TOline{delta} {0}{} 
\TOline{length}{0.75}{} 
\TOline{ratio} {.5}{} 
\bottomrule
\end{tabular}
\end{NewMacroBox} 

\subsubsection{Option \tkzname{length}} 
\begin{tkzexample}[latex=7cm]
  \begin{tikzpicture}
      \tkzInit[xmax=7,ymax=6]
      \tkzDefPoint[pos=left](1,1){A}
      \tkzDefPoint(6,1){B}
      \tkzInterCC[R](A,4cm)(B,3cm)
      \tkzGetPoints{C}{D}
      \tkzDrawPoint(C)
      \tkzCompass[color=red,length=1.5](A,C)
      \tkzCompass[color=red](B,C)
      \tkzDrawSegments(A,B A,C B,C)
  \end{tikzpicture}
\end{tkzexample}

\subsubsection{Option \tkzname{delta}} 
\begin{tkzexample}[latex=7cm]
  \begin{tikzpicture} 
    \tkzInit[xmax=5,ymax=5]\tkzGrid[sub]
    \tkzClip
    \tkzDefPoint(0,0){A} 
    \tkzDefPoint(5,0){B}
    \tkzInterCC[R](A,4cm)(B,3cm)
    \tkzGetPoints{C}{D}
    \tkzDrawPoints(A,B,C) 
    \tkzCompass[color=red,delta=20](A,C)
    \tkzCompass[color=red,delta=20](B,C) 
    \tkzDrawPolygon(A,B,C)  
    \tkzMarkAngle(A,C,B)
  \end{tikzpicture}
\end{tkzexample} 

\newpage
\subsection{Multiples constructions \tkzcname{tkzCompasss}} 
\begin{NewMacroBox}{tkzCompasss}{\oarg{local options}\parg{pt1,pt2 pt3,pt4,...}}
\emph{Attention les arguments sont des listes de deux points. Cela permet d'économiser quelques lignes de codes.}
  

\medskip
\begin{tabular}{lll}
\toprule
options             & défaut & définition                         \\ 
\midrule
\TOline{delta} {0}{} 
\TOline{length}{0.75}{} 
\TOline{ratio} {.5}{} 
\end{tabular}
\end{NewMacroBox} 


\begin{center}
\begin{tkzexample}[vbox]
\begin{tikzpicture}[scale=.75]
 \tkzDefPoint(2,2){A}  \tkzDefPoint(5,-2){B}
 \tkzDefPoint(3,4){C}  \tkzDrawPoints(A,B) 
 \tkzDrawPoint[color=red,shape=cross out](C)    
 \tkzCompasss[color  = orange,length = 1](A,B A,C B,C C,B) 
 \tkzShowLine[mediator,color=red,dashed,length = 2](A,B)
 \tkzShowLine[parallel = through C,color    = blue,length   = 2](A,B)
 \tkzDefLine[mediator](A,B)           \tkzGetPoints{i}{j}
 \tkzDefLine[parallel=through C](A,B) \tkzGetPoint{D}
 \tkzDrawLines[add=.6 and .6](C,D A,C B,D)
 \tkzDrawLines(i,j) \tkzDrawPoints(A,B,C,i,j,D)  
 \tkzLabelPoints(A,B,C,i,j,D)
\end{tikzpicture}
\end{tkzexample} 
\end{center}



\newpage 

\subsection{Macro de configuration \tkzcname{tkzSetUpCompass}} 

\begin{NewMacroBox}{tkzSetUpCompass}{\oarg{local options}\parg{A,B} ou \parg{A,B,C}}
\begin{tabular}{lll}
options             & défaut & définition                         \\ 
\midrule
\TOline{line width}  {0.4pt}{épaisseur du trait} 
\TOline{color}  {black!50}{couleur du trait} 
\TOline{style}  {solid}{style du trait solid, dashed,dotted,...}
\end{tabular}
\end{NewMacroBox} 

\begin{center}
\begin{tkzexample}[vbox]
\begin{tikzpicture}
  \tkzInit[xmax=9,ymax=7] \tkzClip 
  \tkzDefPoints{0/1/A, 8/3/B, 3/6/C}
  \tkzDrawPolygon(A,B,C)  
  \tkzSetUpCompass[color=brown,line width=.3 pt,style=dashed]
  \tkzDefLine[bisector](B,A,C)  \tkzGetPoint{a}
  \tkzDefLine[bisector](C,B,A)  \tkzGetPoint{b}
  \tkzShowLine[bisector,size=2,gap=3](B,A,C)
  \tkzShowLine[bisector,size=1,gap=3](C,B,A)   
  \tkzInterLL(A,a)(B,b) \tkzGetPoint{I}
  \tkzDefPointBy[projection= onto A--B](I) \tkzGetPoint{H}
  \tkzDrawCircle[radius,color=red](I,H) 
  \tkzDrawSegments[color=Maroon!50](I,H)
  \tkzDrawLines[add=0 and 5,color=Maroon!50 ](A,a B,b) 
\end{tikzpicture} 
\end{tkzexample} 
\end{center}



%!TEX root = /Users/ego/Boulot/TKZ/tkz-euclide/doc_fr/TKZdoc-euclide-main.tex

\section{Les secteurs}

\begin{NewMacroBox}{tkzDrawSector}{\oarg{local options}\parg{O,\dots}\parg{\dots}}
\noindent\emph{Attention les arguments varient en fonction des options.}

\medskip
\begin{tabular}{lll}
\toprule
options             & défaut & définition                         \\ 
\midrule
\TOline{towards}{towards}{O est le centre et l'arc par de A vers (OB)}
\TOline{rotate} {towards}{l'arc part de A et l'angle détermine sa longueur } 
\TOline{R}{towards}{On donne le rayon et deux angles}
\TOline{R with nodes}{towards}{On donne le rayon et deux points}
\bottomrule
\end{tabular} 

\medskip
\emph{Il faut ajouter bien sûr tous les styles de \TIKZ\ pour les tracés}

\medskip

\begin{tabular}{lll}
\toprule
options             & arguments & exemple                         \\ 
\midrule
\TOline{towards}{\parg{pt,pt}\parg{pt}}{\tkzcname{tkzDrawSector(O,A)(B)}}
\TOline{rotate} {\parg{pt,pt}\parg{an}}{\tkzcname{tkzDrawSector[rotate,color=red](O,A)(90)}} 
\TOline{R}{\parg{pt,$r$}\parg{an,an}}{\tkzcname{tkzDrawSector[R,color=blue](O,2 cm)(30,90)}}
\TOline{R with nodes}{\parg{pt,$r$}\parg{pt,pt}}{\tkzcname{tkzDrawSector[R with nodes](O,2 cm)(A,B)}}
\bottomrule
\end{tabular}
\end{NewMacroBox}

Quelques exemples : 

\subsection{\tkzcname{tkzDrawSector} et \tkzname{towards}} 
Il est inutile de mettre \tkzname{towards}.
\begin{tkzexample}[latex=7cm,small]
\begin{tikzpicture}[scale=1]
  \tkzDefPoint(0,0){O}
  \tkzDefPoint(-30:3){A} 
  \tkzDefPointBy[rotation = center O angle -60](A) 
  \tkzDrawSector[fill=red!50](O,A)(tkzPointResult)
 \begin{scope}[shift={(-60:1cm)}]
  \tkzDefPoint(0,0){O}
  \tkzDefPoint(-30:3){A} 
  \tkzDefPointBy[rotation = center O angle -60](A) 
  \tkzDrawSector[fill=blue!50](O,tkzPointResult)(A)
  \end{scope}
\end{tikzpicture}   
\end{tkzexample}

\subsection{\tkzcname{tkzDrawSector} et \tkzname{rotate}}  

\begin{tkzexample}[latex=7cm]
\begin{tikzpicture}[scale=2]       
   \tkzDefPoint(0,0){O}
   \tkzDefPoint(2,2){A}
   \tkzDrawSector[rotate,draw=red!50!black,%
   fill=red!20](O,A)(30)
   \tkzDrawSector[rotate,draw=blue!50!black,%
   fill=blue!20](O,A)(-30)
\end{tikzpicture} 
\end{tkzexample}  

\subsection{\tkzcname{tkzDrawSector} et \tkzname{R}}  
\begin{tkzexample}[latex=7cm]   
\begin{tikzpicture}[scale=1.25]
   \tkzDefPoint(0,0){O}
   \tkzDefPoint(2,-1){A}
   \tkzDrawSector[R,draw=white,%
   fill=red!50](O,2cm)(30,90)
   \tkzDrawSector[R,draw=white,%
   fill=red!60](O,2cm)(90,180)
   \tkzDrawSector[R,draw=white,%
   fill=red!70](O,2cm)(180,270)
   \tkzDrawSector[R,draw=white,%
   fill=red!90](O,2cm)(270,360) 
\end{tikzpicture}
\end{tkzexample}

\subsection{\tkzcname{tkzDrawSector} et \tkzname{R}}  
\begin{tkzexample}[latex=7cm]     
\begin{tikzpicture}[scale=1.25]
 \tkzDefPoint[pos=left](0,0){O}
 \tkzDefPoint(4,-2){A}
 \tkzDefPoint(4,1){B}
 \tkzDefPoint(3,3){C}
 \tkzDrawSector[R with nodes,%
                fill=blue!20](O,1 cm)(B,C)
 \tkzDrawSector[R with nodes,%
                fill=red!20](O,1 cm)(A,B)  
\tkzDrawSegments(O,A O,B O,C)
\tkzDrawPoints(O,A,B,C) 
\tkzLabelPoints(A,B,C) 
\tkzLabelPoints[left](O) 
\end{tikzpicture}
\end{tkzexample}

\begin{NewMacroBox}{tkzFillSector}{\oarg{local options}\parg{O,\dots}\parg{\dots}}
\noindent\emph{Attention les arguments varient en fonction des options.}

\medskip

\begin{tabular}{lll}
\toprule
options             & défaut & définition                         \\ 
\midrule
\TOline{towards}{towards}{O est le centre et l'arc par de A vers (OB)}
\TOline{rotate} {towards}{l'arc part de A et l'angle détermine sa longueur } 
\TOline{R}{towards}{On donne le rayon et deux angles}
\TOline{R with nodes}{towards}{On donne le rayon et deux points}
\bottomrule
\end{tabular} 

\medskip
\emph{Il faut ajouter bien sûr tous les styles de \TIKZ pour les tracés}

\medskip
\begin{tabular}{lll}
\toprule
options             & arguments & exemple                         \\ 
\midrule
\TOline{towards}{\parg{pt,pt}\parg{pt}}{\tkzcname{tkzFillSector(O,A)(B)}}
\TOline{rotate} {\parg{pt,pt}\parg{an}}{\tkzcname{tkzFillSector[rotate,color=red](O,A)(90)}}
\TOline{R}{\parg{pt,$r$}\parg{an,an}}{\tkzcname{tkzFillSector[R,color=blue](O,2 cm)(30,90)}} 
\TOline{R with nodes}{\parg{pt,$r$}\parg{pt,pt}}{\tkzcname{tkzFillSector[R with nodes](O,2 cm)(A,B)}}
\bottomrule
\end{tabular}   
\end{NewMacroBox} 

\subsection{\tkzcname{tkzFillSector} et \tkzname{towards}} 
Il est inutile de mettre \tkzname{towards} et vous remarquerez que les contours ne sont pas tracés,seule la surface est colorée.
\begin{tkzexample}[latex=5.75cm,small]
\begin{tikzpicture}[scale=.6]
  \tkzDefPoint(0,0){O}
  \tkzDefPoint(-30:3){A} 
  \tkzDefPointBy[rotation = center O angle -60](A) 
  \tkzFillSector[fill=red!50](O,A)(tkzPointResult)
  \begin{scope}[shift={(-60:1cm)}]
   \tkzDefPoint(0,0){O}
   \tkzDefPoint(-30:3){A} 
   \tkzDefPointBy[rotation = center O angle -60](A) 
   \tkzFillSector[color=blue!50](O,tkzPointResult)(A)
  \end{scope}
\end{tikzpicture}
\end{tkzexample}


\subsection{\tkzcname{tkzFillSector} et \tkzname{rotate}} 
\begin{tkzexample}[latex=5.75cm,small]
\begin{tikzpicture}[scale=1.5]
 \tkzDefPoint(0,0){O} \tkzDefPoint(2,2){A}
 \tkzFillSector[rotate,color=red!20](O,A)(30)
 \tkzFillSector[rotate,color=blue!20](O,A)(-30)
\end{tikzpicture}    
\end{tkzexample} 

\newpage
\begin{NewMacroBox}{tkzClipSector}{\oarg{local options}\parg{O,\dots}\parg{\dots}}
\noindent\emph{Attention les arguments varient en fonction des options.}

\medskip

\begin{tabular}{lll}
\toprule
options             & défaut & définition                         \\ 
\midrule
\TOline{towards}{towards}{O est le centre et le secteur part de A vers (OB)}
\TOline{rotate} {towards}{le secteur part de A et l'angle détermine son amplitude } 
\TOline{R}{towards}{On donne le rayon et deux angles} 
\bottomrule
\end{tabular} 


\medskip
\emph{Il faut ajouter bien sûr tous les styles de \TIKZ\ pour les tracés}

\medskip   
\begin{tabular}{lll}
\toprule
options             & arguments & exemple                         \\ 
\midrule
\TOline{towards}{\parg{pt,pt}\parg{pt}}{\tkzcname{tkzClipSector(O,A)(B)}}
\TOline{rotate} {\parg{pt,pt}\parg{angle}}{\tkzcname{tkzClipSector[rotate](O,A)(90)}} 
\TOline{R}{\parg{pt,$r$}\parg{angle 1,angle 2}}{\tkzcname{tkzClipSector[R](O,2 cm)(30,90)}}
\bottomrule
\end{tabular}
\end{NewMacroBox}

\begin{center}
\begin{tkzexample}[vbox] 
\begin{tikzpicture}[scale=2] 
  \tkzDefPoint(0,0){O}
  \tkzDefPoint(2,-1){A}
  \tkzDefPoint(1,1){B} 
  \tkzDrawSector[color=bistre,dashed](O,A)(B)
  \tkzDrawSector[color=Maroon](O,B)(A)
  \tkzDrawPoints(A,B,O)
  \tkzClipSector(O,B)(A)
\draw[fill=red!20] (-1,0) rectangle (3,3);
\end{tikzpicture} 
\end{tkzexample}
\end{center}
 

 
  \endinput  
  

%!TEX root = /Users/ego/Boulot/TKZ/tkz-euclide/doc_fr/TKZdoc-euclide-main.tex

\section{Les arcs} 

\begin{NewMacroBox}{tkzDrawArc}{\oarg{local options}\parg{O,\dots}\parg{\dots} }

 \emph{Cette macro trace un arc de centre O. Suivant les options, les arguments diffèrent.   Il s'agit de déterminer un point de départ et un point d'arrivée. Soit le point de départ est donné, c'est ce qu'il y a de plus simple, soit on donne le rayon de l'arc. Dans ce dernier cas, il est nécessaire d'avoir deux angles. On peut soit donner directement les angles, soit donner des nodes qui associés au centre permettront de les déterminer.}   
  

\medskip

\begin{tabular}{lll}
\toprule
options             & défaut & définition                         \\ 
\midrule
\TOline{towards}{towards}{O est le centre et l'arc par de A vers (OB)} 
\TOline{rotate} {towards}{l'arc part de A et l'angle détermine sa longueur } 
\TOline{R}{towards}{On donne le rayon et deux angles} 
\TOline{R with nodes}{towards}{On donne le rayon et deux points}
\TOline{delta}{0}{angle ajouté de chaque côté } 
\bottomrule
\end{tabular}

\medskip
\emph{Il faut ajouter bien sûr tous les styles de \TIKZ pour les tracés}

\medskip

\begin{tabular}{lll}
\toprule
options             & arguments & exemple                         \\ 
\midrule
\TOline{towards}{\parg{pt,pt}\parg{pt}}{\tkzcname{tkzDrawArc[delta=10](O,A)(B)}} 
\TOline{rotate} {\parg{pt,pt}\parg{an}}{\tkzcname{tkzDrawArc[rotate,color=red](O,A)(90)}}
\TOline{R}{\parg{pt,$r$}\parg{an,an}}{\tkzcname{tkzDrawArc[R,color=blue](O,2 cm)(30,90)}}
\TOline{R with nodes}{\parg{pt,$r$}\parg{pt,pt}}{\tkzcname{tkzDrawArc[R with nodes](O,2 cm)(A,B)}}
\bottomrule
\end{tabular}
\end{NewMacroBox}

Quelques exemples : 

\subsection{\tkzcname{tkzDrawArc} et \tkzname{towards}}
Il est inutile de mettre \tkzname{towards}. Dans ce premier exemple l'arc part de A et va sur B. L'arc qui va de B vers A est différent. On obtient le saillant en allant dans le sens direct du cercle trigonométrique.
\begin{tkzexample}[latex=6cm]
\begin{tikzpicture}
  \tkzDefPoint(0,0){O}
  \tkzDefPoint(2,-1){A}
  \tkzDefPointBy[rotation= center O angle 90](A)
  \tkzGetPoint{B}
  \tkzDrawArc[color=blue](O,A)(B) 
  \tkzDrawArc(O,B)(A)
  \tkzDrawLines[add = 0 and .5](O,A O,B)
  \tkzDrawPoints(O,A,B)
  \tkzLabelPoints[below](O,A,B)  
\end{tikzpicture}
\end{tkzexample}


\subsection{\tkzcname{tkzDrawArc} et \tkzname{towards}}
Dans celui-ci, l'arc part de A mais s'arrête sur la droite (OB).
 
\begin{tkzexample}[latex=6cm]
\begin{tikzpicture}[scale=1.5] 
  \tkzDefPoint(0,0){O}
  \tkzDefPoint(2,-1){A}
  \tkzDefPoint(1,1){B} 
  \tkzDrawArc[color=blue](O,A)(B)
  \tkzDrawArc[color=Maroon](O,B)(A)
  \tkzDrawArc(O,B)(A)
  \tkzDrawLines[add = 0 and .5](O,A O,B) 
  \tkzDrawPoints(O,A,B)
  \tkzLabelPoints[below](O,A,B)  
\end{tikzpicture}
\end{tkzexample}

\subsection{\tkzcname{tkzDrawArc} et \tkzname{rotate}}
\begin{tkzexample}[latex=5cm] 
\begin{tikzpicture} 
  \tkzDefPoint(0,0){O}
  \tkzDefPoint(2,-2){A}
  \tkzDefPoint(60:2){B}
  \tkzDrawLines[add = 0 and .5](O,A O,B)
  \tkzDrawArc[rotate,color=red](O,A)(180)
  \tkzDrawPoints(O,A,B)
  \tkzLabelPoints[below](O,A,B) 
\end{tikzpicture}
\end{tkzexample} 


\subsection{\tkzcname{tkzDrawArc} et \tkzname{R}} 
\begin{tkzexample}[latex=5cm]   
\begin{tikzpicture}
  \tkzDefPoints{0/0/O}
  \tikzset{compass style/.append style={<->}}   
  \tkzDrawArc[R, color=orange,double](O,3cm)(270,360)
  \tkzDrawArc[R, color=blue,double](O,2cm)(0,270) 
  \tkzDrawPoint(O)
  \tkzLabelPoint[below](O){$O$}  
\end{tikzpicture} 
\end{tkzexample}

\subsection{\tkzcname{tkzDrawArc} et \tkzname{R with nodes}} 
\begin{tkzexample}[latex=5cm]
\begin{tikzpicture}
  \tkzDefPoint(0,0){O}
  \tkzDefPoint(2,-1){A}
  \tkzDefPoint(1,1){B}
  \tkzCalcLength(B,A)\tkzGetLength{radius}
  \tkzDrawArc[R with nodes](B,\radius pt)(A,O)
\end{tikzpicture}
\end{tkzexample}

\subsection{\tkzcname{tkzDrawArc} et \tkzname{delta}}
Cette option permet un peu comme \tkzcname{tkzCompass} de placer un arc et de déborder de chaque côté. delta est une mesure en degré.

\begin{tkzexample}[latex=7cm] 
\begin{tikzpicture} 
 \tkzInit
 \tkzDefPoint(0,0){A}
 \tkzDefPoint(5,0){B}
 \tkzDefPointBy[rotation= center A%
                angle 60](B) \tkzGetPoint{C} 
 \tkzSetUpLine[color=gray]
 \tkzDefPointBy[symmetry= center C](A)
    \tkzGetPoint{D} 
 \tkzDrawSegments(A,B A,D)
 \tkzDrawLine(B,D)
 \tkzSetUpCompass[color=orange]
 \tkzDrawArc[delta=10](A,B)(C)
 \tkzDrawArc[delta=10](B,C)(A)
 \tkzDrawArc[delta=10](C,D)(D)
 \tkzDrawPoints(A,B,C,D)
 \tkzLabelPoints(A,B,C,D)
 \tkzMarkRightAngle(D,B,A)
\end{tikzpicture}
\end{tkzexample} 


 \endinput
 

%!TEX root = /Users/ego/Boulot/TKZ/tkz-euclide/doc_fr/TKZdoc-euclide-main.tex


\section{Rapporteurs} % (fold)
\label{sec:rapporteurs}

D'après une idée de Yves Combe., la  macro suivante permet de dessiner un rapporteur. J'ai ajouté mon propre rapporteur qui est obtenu avec l'option \tkzname{full} (par défaut), celui de Yves est obtenu avec \tkzname{half}.


\begin{NewMacroBox}{tkzProtractor}{\oarg{local options}\parg{$O,A$}}
 
\medskip
\begin{tabular}{lll}
\toprule
options            & défaut  & définition                         \\ 
\midrule
\TOline{with}     {full}    { full ou bien half}
\TOline{lw}  {0.4 pt} { épaisseur des lignes}
\TOline{scale}   {1} { ratio : permet d'ajuster la taille du rapporteur} \TOline{return} {false} { sens indirect du cercle trigonométrique}
\bottomrule
\end{tabular}

\medskip
\emph{Le principe de fonctionnement est encore plus simple. Il suffit de nommer une demi-droite. Le rapporteur sera placé sur l'origine $O$ la direction de la demi-droites est donnée par $A$. L'angle est mesuré dans le sens direct du cercle trigonométrique} 
\end{NewMacroBox}


\subsection{Le rapporteur circulaire} 

Mesure dans le sens direct

\begin{tkzltxexample}[] 
\begin{tikzpicture}[scale=.75]
\tkzDefPoint(2,3){A}
\tkzDefPoint[shift={(2,3)}](31:8){B}
\tkzDefPoint[shift={(2,3)}](158:8){C}
\tkzDrawSegments[color = red,
           line width = 1pt](A,B A,C)
\tkzProtractor[with  = full,
               scale = 1.25](A,B)  
\end{tikzpicture}  
\end{tkzltxexample}
 
\vspace*{6cm}\hspace*{6cm}   
\begin{tikzpicture}[scale=.75,overlay]
\tkzDefPoint(2,3){A}
\tkzDefPoint[shift={(2,3)}](31:8){B}
\tkzDefPoint[shift={(2,3)}](158:8){C}
\tkzDrawSegments[color = red,
           line width = 1pt](A,B A,C)
\tkzProtractor[with  = full,
               scale = 1.25](A,B)  
\end{tikzpicture}  

\newpage
\subsection{Le rapporteur circulaire, transparent et retourné}
Mesure dans le sens indirect, on retourne le rapporteur.

\begin{center}
  \begin{tkzexample}[vbox] 
\begin{tikzpicture}
  \tkzInit[xmin=-4,xmax=9,ymin=-3,ymax=9]
  \tkzClip
  \tkzDefPoint(2,3){A}
  \tkzDefPoint[shift={(2,3)}](31:8){B}  
  \tkzDefPoint[shift={(2,3)}](158:8){C}   
  \tkzDrawSegments[color=red,line width=1pt](A,B A,C)
  \tkzProtractor[scale=1.25,with=full,return](A,C) 
\end{tikzpicture}
\end{tkzexample}
\end{center}    
 
\newpage
\subsection{Le rapporteur original semi-circulaire (Yves Combes)}

Mesure dans le sens direct avec un rapporteur semi-circulaire
\begin{center} 
\begin{tkzexample}[vbox]
\begin{tikzpicture}
  \tkzInit[xmin=-5,xmax=9,ymin=-3,ymax=10]
  \tkzClip     
  \tkzDefPoint(2,3){A}
  \tkzDefPoint[shift={(2,3)}](31:8){B}  
  \tkzDefPoint[shift={(2,3)}](158:8){C} 
  \tkzDrawSegments[color=red,line width=1pt](A,B A,C)
  \tkzProtractor[scale=1.25,with=half](A,B) 
\end{tikzpicture}
\end{tkzexample}  
\end{center}
\subsection{Le rapporteur semi-circulaire dans le sens indirect}

\begin{center} 
\begin{tkzexample}[vbox]
\begin{tikzpicture}
  \tkzInit[xmin=-5,xmax=9,ymin=-3,ymax=10]
  \tkzClip      
  \tkzDefPoint(2,3){A}
  \tkzDefPoint[shift={(2,3)}](31:8){B}  
  \tkzDefPoint[shift={(2,3)}](158:8){C}  
  \tkzDrawSegments[color=red,line width=1pt](A,B A,C)
  \tkzProtractor[scale=1.25,with=half,return](A,C) 
\end{tikzpicture}
\end{tkzexample}
\end{center}

le cas échéant vous pouvez utiliser la macro originale de Yves

\begin{NewMacroBox}{tkzOriProtractor}{\oarg{local options}}
 
\medskip
\begin{tabular}{lll}
\toprule
options            & défaut  & définition                         \\ 
\midrule
\TOline{with}  {full} {full ou bien half}   
\TOline{lw}  {0.4 pt} {épaisseur des lignes} 
\TOline{shift} {(x;y)}{permet de faire glisser le rapporteur} 
\TOline{rotate}  {0}  {permet de faire pivoter le rapporteur}
\TOline{scale}   {1}  {ratio : permet d'ajuster la taille du rapporteur} \TOline{return}{false}{sens indirect du cercle trigonométrique} 
\bottomrule
\end{tabular}

\medskip
\emph{Le principe de fonctionnement est encore plus simple. Il suffit de nommer une demi-droite. Le rapporteur sera placé sur l'origine.} 
\end{NewMacroBox}

\subsection{Le rapporteur semi-circulaire avec la macro originale} 
\begin{center} 
  \begin{tkzexample}[vbox] 
\begin{tikzpicture}
  \tkzInit[xmin=-5,xmax=9,ymin=-3,ymax=10]
  \tkzClip  
  \tkzDefPoint(2,3){A} 
  \tkzDefPoint[shift={(2,3)}](158:8){B}
  \tkzDefPoint[shift={(2,3)}](31:8){C}  
  \tkzDrawSegments[color=red,line width=1pt](A,B A,C)
  \tkzOriProtractor[shift = {(2,3)},scale=1.25, rotate = +31,with=half]
\end{tikzpicture}
\end{tkzexample}
\end{center} 

\subsection{Le rapporteur semi-circulaire avec la macro originale dans le sens indirect} 
\begin{center} 
  \begin{tkzexample}[vbox] 
\begin{tikzpicture}
  \tkzInit[xmin=-5,xmax=9,ymin=-3,ymax=10]
  \tkzClip  
  \tkzDefPoint(2,3){A} 
  \tkzDefPoint[shift={(2,3)}](158:8){B}
  \tkzDefPoint[shift={(2,3)}](31:8){C}  
  \tkzDrawSegments[color=red,line width=1pt](A,B A,C)
  \tkzOriProtractor[shift = {(2,3)},scale=1.25, rotate = -22,with=half]
\end{tikzpicture}
   \end{tkzexample}
\end{center}    
\endinput
%!TEX root = /Users/ego/Boulot/TKZ/tkz-euclide/doc_fr/TKZdoc-euclide-main.tex


\section{Quelques outils}


\subsection{Dupliquer un segment} 

Il s'agit de construire un segment sur une demi-droite donnée  de même longueur qu'un segment donné.

\begin{NewMacroBox}{tkzDuplicateLen}{\parg{pt1,pt2}\parg{pt3,pt4}\marg{pt5}}
Il s'agit de créer un segment  sur une demi-droite  donnée de même longueur qu'un segment donné . Il s'agit en fait de la définition d'un point.

\medskip
  
\begin{tabular}{lll}
\toprule
arguments             & exemple & explication                         \\ 
\midrule
\TAline{(pt1,pt2)(pt3,pt4)\{pt5\}} {\tkzcname{tkzDuplicateLen}(A,B)(E,F)\{C\}}{AC=EF et $C \in [AB)$} \\                                                                         
\bottomrule
\end{tabular}

\medskip
\emph{La macro \tkzcname{tkzDuplicateSegment} est identique à celle-ci. }
\end{NewMacroBox}

\subsubsection{Proportion d'or avec \tkzcname{tkzDuplicateLen}} 

\begin{tkzexample}[latex=7cm,small]
\begin{tikzpicture}[rotate=-90]
 \tkzInit[xmax=10,ymax=10]
 \tkzClip[space=1]
 \tkzDefPoint(0,0){A}
 \tkzDefPoint(10,0){B}
 \tkzDefMidPoint(A,B)       \tkzGetPoint{I}
 \tkzDefPointWith[orthogonal,K=-.75](B,A)
    \tkzGetPoint{C}
 \tkzInterLC(B,C)(B,I)      \tkzGetSecondPoint{D}
 \tkzDuplicateLen(B,D)(D,A) \tkzGetPoint{E}
 \tkzInterLC(A,B)(A,E)      \tkzGetPoints{N}{M}
 \tkzDrawArc[delta=10](D,E)(B)
 \tkzDrawArc[delta=10](A,M)(E)
 \tkzDrawLines(A,B B,C A,D)
 \tkzDrawArc[delta=10](B,D)(I)
 \tkzDrawPoints(A,B,D,C,M,I,N)
 \tkzLabelPoints(A,B,D,C,M,I,N)
\end{tikzpicture}
\end{tkzexample}
  

\subsection{Déterminer une pente}
Il s'agit de déterminer si elle existe, la pente d'une droite définie par deux points. Aucune vérification de l'existence n'est faite.

\begin{NewMacroBox}{tkzFindSlope}{\parg{pt1,pt2}\marg{name of macro}}
Le résultat est stocké dans une macro.

\medskip
  
\begin{tabular}{lll}
\toprule
arguments             & exemple & explication                         \\ 
\midrule
\TAline{(pt1,pt2){pt3}} {\tkzcname{tkzFindSlope}(A,B)\{slope\}}{\tkzcname{slope} donnera le résultat de $\frac{y_B-y_A}{x_B-x_A}$} \\                                                                         
\bottomrule
\end{tabular}

\medskip
\emph{Attention à ne pas avoir $x_B=x_A$ }
\end{NewMacroBox} 

\begin{center}
\begin{tkzexample}[vbox] 
\begin{tikzpicture}[scale=1.5]
  \tkzInit[xmax=5,ymax=5]\tkzGrid[sub]
  \tkzDefPoint(1,2){A}    \tkzDefPoint(3,4){B} 
  \tkzDefPoint(3,2){C}    \tkzDefPoint(3,1){D}
  \tkzDrawSegments(A,B A,C A,D)
  \tkzDrawPoints[color=red](A,B,C,D)  \tkzLabelPoints(A,B,C,D)
  \tkzFindSlope(A,B){SAB} \tkzFindSlope(A,C){SAC}\tkzFindSlope(A,D){SAD}
  \tkzText[fill=Gold!50,draw=brown](2.5,0){La pente de (AB) est : \SAB}
  \tkzText[fill=Gold!50,draw=brown](2.5,-.5){La pente de (AC) est : \SAC}
  \tkzText[fill=Gold!50,draw=brown](2.5,-1){La pente de (AD) est : \SAD}   
\end{tikzpicture}
\end{tkzexample} 
\end{center}


\newpage
\subsection{Angle formé par une droite avec l'axe horizontal} 
Beaucoup plus intéressante que la précédente. Le résultat est compris entre -180 degrés et +180 degrés. 

\begin{NewMacroBox}{tkzFindSlopeAngle}{\parg{pt1,pt2}}
Le résultat est stocké dans une macro \tkzcname{tkzAngleResult}.

\medskip
  
\begin{tabular}{lll}
\toprule
arguments             & exemple & explication                         \\ 
\midrule
\TAline{(pt1,pt2)} {\tkzcname{tkzFindSlopeAngle}(A,B)}{\tkzcname{tkzGetAngle} peut récupèrer le résultat}                                                                       
\bottomrule
\end{tabular}

\medskip
\emph{Si la récupération n'est pas nécessaire, il est possible d'utiliser \tkzcname{tkzAngleResult}}
\end{NewMacroBox}  


\subsubsection{exemple d'utilisation de \tkzcname{tkzFindSlopeAngle}}
Voici une autre version de la construction d'une médiatrice

\begin{center}
\begin{tkzexample}[vbox]
\begin{tikzpicture}
 \tkzInit   
 \tkzDefPoint(0,0){A}        \tkzDefPoint(3,2){B}
 \tkzDefLine[mediator](A,B)  \tkzGetPoints{I}{J} 
 \tkzCalcLength[cm](A,B)     \tkzGetLength{dAB}
 \tkzFindSlopeAngle(A,B)     \tkzGetAngle{tkzangle}  
 \begin{scope}[rotate=\tkzangle] 
   \tikzset{arc/.style={color=gray,delta=10}}
   \tkzDrawArc[R,arc](B,3/4*\dAB)(120,240) 
   \tkzDrawArc[R,arc](A,3/4*\dAB)(-45,60) 
   \tkzDrawLine(I,J)         \tkzDrawSegment(A,B) 
  \end{scope}
  \tkzDrawPoints(A,B,I,J)    \tkzLabelPoints(A,B) 
   \tkzLabelPoints[right](I,J) 
\end{tikzpicture}
\end{tkzexample}
\end{center}

  

\newpage
\subsection{Récupérer un angle} 
Dans l'exemple précédent, j'ai utilisé la macro  \tkzcname{tkzGetAngle} qui permet de récupérer un angle.

\begin{NewMacroBox}{tkzGetAngle}{\marg{name of macro}}
Cette macro récupère  \tkzcname{tkzAngleResult} et stocke le résultat dans une nouvelle macro.

\medskip
  
\begin{tabular}{lll}
\toprule
arguments             & exemple & explication                         \\ 
\midrule
\TAline{name of macro} {\tkzcname{tkzGetAngle}\{ang\}}{\tkzcname{ang} contient la valeur de l'angle.}                                                                       
\end{tabular}
\end{NewMacroBox}

\subsection{exemple d'utilisation de \tkzcname{tkzGetAngle}}
 Il s'agit ici que $(AB)$ soit la bissectrice de $\widehat{CAD}$, tel que la pente $AD$ soit nulle. On récupère la pente de $(AB)$ puis on effectue deux rotations.
 
\begin{center}
\begin{tkzexample}[vbox]
\begin{tikzpicture} 
  \tkzInit
  \tkzDefPoint(1,5){A} \tkzDefPoint(5,2){B}  \tkzDrawSegment(A,B) 
  \tkzFindSlopeAngle(A,B)\tkzGetAngle{tkzang}
  \tkzDefPointBy[rotation= center A angle \tkzang ](B) \tkzGetPoint{C} 
  \tkzDefPointBy[rotation= center A angle -\tkzang ](B) \tkzGetPoint{D} 
  \tkzCompass[length=1,dashed,color=red](A,C) 
  \tkzCompass[delta=10,Maroon](B,C)   \tkzDrawPoints(A,B,C,D) 
  \tkzLabelPoints(B,C,D)  \tkzLabelPoints[above left](A) 
  \tkzDrawSegments[style=dashed,color=bistre](A,C A,D)
\end{tikzpicture}
\end{tkzexample} 
\end{center}




\newpage
\subsection{Angle formé par trois points} 


\begin{NewMacroBox}{tkzFindAngle}{\parg{pt1,pt2,pt3}}
Le résultat est stocké dans une macro \tkzcname{tkzAngleResult}. 

\medskip
  
\begin{tabular}{lll}
\toprule
arguments             & exemple & explication                         \\ 
\midrule
\TAline{(pt1,pt2,pt3)} {\tkzcname{tkzFindAngle}(A,B,C)}{\tkzcname{tkzAngleResult} donne l'angle ($\overrightarrow{BA},\overrightarrow{BC}$)}                                                                         
\bottomrule
\end{tabular}

\medskip
\emph{Le résultat est compris entre -180 degrés et +180 degrés. pt2 est le sommet et  \tkzcname{tkzGetAngle} peut récupérer l'angle. }
\end{NewMacroBox}   

\subsection{Exemple d'utilisation de \tkzcname{tkzFindAngle} }

\begin{center}
\begin{tkzexample}[vbox,small]
\begin{tikzpicture}
  \tkzInit[xmin=-1,ymin=-1,xmax=7,ymax=7]
  \tkzClip  
  \tkzDefPoint (0,0){O}  \tkzDefPoint (6,0){A}
  \tkzDefPoint (5,5){B}  \tkzDefPoint (3,4){M}
  \tkzFindAngle (A,O,M)  \tkzGetAngle{an}   
  \tkzDefPointBy[rotation=center O angle \an](A) \tkzGetPoint{C}
  \tkzDrawSector[fill = blue!50,opacity=.5](O,A)(C)
  \tkzFindAngle(M,B,A)   \tkzGetAngle{am}
  \tkzDefPointBy[rotation = center O angle \am](A) \tkzGetPoint{D} 
  \tkzDrawSector[fill = red!50,opacity = .5](O,A)(D) 
  \tkzDrawPoints(O,A,B,M,C,D)   \tkzLabelPoints(O,A,B,M,C,D) 
  \FPround\an\an{2} \FPround\am\am{2} \tkzDrawSegments(M,B B,A)
  \tkzText(4,2){$\widehat{AOC}=\widehat{AOM}=\an^{\circ}$} 
  \tkzText(1,4){$\widehat{AOD}=\widehat{MBA}=\am^{\circ}$}  
\end{tikzpicture}
\end{tkzexample}
\end{center}

 

\newpage
\subsection{Longueur d'un segment \tkzcname{tkzVecLen}}
Il existe dans \TIKZ\ une option \tkzname{veclen}. Cette option
 permet de calculer AB si A et B sont deux points.

Le seul problème pour moi est que la version de \TIKZ\ n'est pas assez précise dans certains cas particuliers. Ma version utilise le package \tkzNamePack{fp.sty} et est plus lente, mais plus précise

\begin{NewMacroBox}{tkzVecLen}{\oarg{local options}\parg{pt1,pt2}\marg{name of macro}}
Le résultat est stocké dans une macro.

\medskip
  
\begin{tabular}{lll}
\toprule
arguments             & exemple & explication                         \\ 
\midrule
\TAline{(pt1,pt2)\{name of macro\}} {\tkzcname{tkzVecLen}(A,B)\{dAB\}}{\tkzcname{dAB} donne  $AB$ en pt}                                                                        
\bottomrule
\end{tabular}

\medskip
Une seule option 

\begin{tabular}{lll}
\toprule
options             & défaut & exemple                         \\ 
\midrule
\TOline{cm}  {false}{\tkzcname{tkzVecLen}[cm](A,B)\{dAB\} \tkzcname{dAB} donne AB en cm}
\end{tabular}
\end{NewMacroBox} 

\subsubsection{Construction d'un carré au compas}
 
\begin{tkzexample}[vbox,small]    
\begin{tikzpicture}[scale=1.2] 
  \tkzDefPoint(0,0){A} \tkzDefPoint(4,0){B}
  \tkzDrawLine[add= .6 and .2](A,B)
  \tkzCalcLength[cm](A,B)\tkzGetLength{dAB}
  \tkzDefLine[perpendicular=through A](A,B)
  \tkzDrawLine(A,tkzPointResult) \tkzGetPoint{D}
  \tkzShowLine[orthogonal=through A,gap=2](A,B)
  \tkzMarkRightAngle(B,A,D)
  \tkzVecKOrth[-1](B,A){C}
  \tkzCompasss(A,D D,C)   \tkzDrawArc[R](B,\dAB)(80,110) 
  \tkzDrawPoints(A,B,C,D) \tkzDrawSegments[color=gray,style=dashed](B,C C,D) 
  \tkzLabelPoints(A,B,C,D)
\end{tikzpicture}
\end{tkzexample} 

\newpage
\subsection{Transformation de pt en cm ou de cm en pt}
Pas sûr que cela soit nécessaire et il ne s'agit que d'une division par 28,45274 et d'un multiplication par ce même nombre. Les macros sont :

\begin{NewMacroBox}{tkzpttocm}{\parg{nombre}\marg{name of macro}}
Le résultat est stocké dans une macro.

\medskip
  
\begin{tabular}{lll}
\toprule
arguments             & exemple & explication                         \\ 
\midrule
\TAline{(nombre){name of macro}} {\tkzcname{tkzpttocm}(120)\{len\}}{\tkzcname{len} donne un nombre de tkzname{cm}}                                                                        
\bottomrule
\end{tabular}

\medskip
\noindent\emph{Il faudra utiliser \tkzcname{len} accompagné de  \tkzname{cm}} 
\end{NewMacroBox}  

\begin{NewMacroBox}{tkzcmtopt}{\parg{nombre}\marg{name of macro}}
Le résultat est stocké dans une macro.

\medskip
  
\begin{tabular}{lll}
\toprule
arguments             & exemple & explication                         \\ 
\midrule
\TAline{(nombre){name of macro}}{\tkzcname{tkzcmtopt}(5)\{len\}}{\tkzcname{len} donne un nombre de tkzname{pt}}                                                                        
\bottomrule
\end{tabular}

\medskip
\noindent\emph{Il faudra utiliser \tkzcname{len} accompagné de  \tkzname{pt}} 
\end{NewMacroBox}

\subsubsection{Exemple}
La macro    \tkzcname{tkzDefCircle[radius](A,B)} définit le rayon que l'on récupère avec     \tkzcname{tkzGetLength}, mais ce résultat est en \tkzname{pt}. 

\begin{tkzexample}[latex=8cm]
  \begin{tikzpicture}
     \tkzDefPoint(0,4){A}
     \tkzDefPoint(3,2){B}
     \tkzDefCircle[radius](A,B) 
     \tkzGetLength{rABpt}
     \tkzpttocm(\rABpt){rABcm}
     \tkzDrawCircle(A,B)
     \tkzDrawPoints(A,B)
     \tkzLabelPoints(A,B)
  \end{tikzpicture}
\end{tkzexample}
  
   
\endinput
%!TEX root = /Users/ego/Boulot/TKZ/tkz-euclide/doc_fr/TKZdoc-euclide-main.tex

\section{Personnalisation}
   

 
\subsection{Fichier de configuration: \tkzname{tkz-base.cfg}}

Vous pouvez créer votre propre fichier \tkzname{tkz-base.cfg} que vous placerez dans un dossier qui sera prioritaire au sein du \tkzname{texmf}.
Dans \tkzname{tkz-base.cfg}, il est possible de modifier les couleurs, ls épaisseurs des lignes. La lecture de ce fichier doit suffire à déterminer le rôle de chaque variable.

\subsection{\tkzcname{tkzSetUpLine}} \label{tkzsetupline}
\begin{NewMacroBox}{tkzSetUpLine}{\oarg{local options}}
\begin{tabular}{lll}
options &  défaut  & définition                 \\ 
\midrule
\TOline{color}{black}{couleur des arcs de cercle de construction} 
\TOline{line width}{0.4pt}{épaisseur des arcs de cercle de construction} 
\TOline{style}{solid}{style des arcs de cercle de construction}
\TOline{add}{.2 and .2}{modification de la longueur d'un segment} 
\end{tabular}  
\end{NewMacroBox}
 
Construire un triangle avec trois segments donnés

\begin{tkzexample}[latex=7cm,small]
\begin{tikzpicture}[scale=.6] 
 \tkzDefPoint(1,0){A} \tkzDefPoint(4,0){B}
 \tkzDefPoint(1,1){C} \tkzDefPoint(5,1){D}
 \tkzDefPoint(1,2){E} \tkzDefPoint(6,2){F}
 \tkzDefPoint(0,4){A'}\tkzDefPoint(3,4){B'}
 \tkzDrawSegments(A,B C,D E,F)
 \tkzDrawLine(A',B')
 \tkzSetUpLine[style=dashed,color=gray]
 \tkzCompass(A',B')
 \tkzCalcLength[cm](C,D)  \tkzGetLength{rCD} 
 \tkzDrawCircle[R](A',\rCD cm)
 \tkzCalcLength[cm](E,F)  \tkzGetLength{rEF} 
 \tkzDrawCircle[R](B',\rEF cm)
 \tkzInterCC[R](A',\rCD cm)(B',\rEF cm) 
 \tkzGetPoints{I}{J} 
 \tkzSetUpLine[color=red] \tkzDrawLine(A',B')
 \tkzDrawSegments(A',I B',I)
 \tkzDrawPoints(A,B,C,D,E,F,A',B',I,J)
 \tkzLabelPoints(A,B,C,D,E,F,A',B',I,J)
\end{tikzpicture}
\end{tkzexample}

Par défaut, dans \tkzname{tkz-base.cfg}, ces styles sont définis par :

\begin{tkzltxexample}[]
\global\edef\tkz@euc@linecolor{\tkz@mainlinecolor}
\global\def\tkz@euc@linewidth{0.6pt}
\global\def\tkz@euc@linestyle{solid}
\global\def\tkz@euc@lineleft{.2}
\global\def\tkz@euc@lineright{.2}  
\end{tkzltxexample}  



\subsection{\tkzcname{tkzSetUpCompass}}

\begin{NewMacroBox}{tkzSetUpCompass}{\oarg{local options}}
\begin{tabular}{lll}
options &  défaut  & définition                 \\ 
\midrule
\TOline{color}{black}{couleur des arcs de cercle de construction} 
\TOline{line width}{0.4pt}{épaisseur des arcs de cercle de construction} 
\TOline{style}{solid}{style des arcs de cercle de construction} 
\end{tabular}
\end{NewMacroBox}   

Par défaut, dans \tkzname{tkz-base.cfg}, ces styles sont définis par :

\begin{tkzltxexample}[]
  \global\edef\tkz@euc@compasscolor{\tkz@otherlinecolor}
  \global\def\tkz@euc@compasswidth{0.4pt}
  \global\def\tkz@euc@compassstyle{solid} 
\end{tkzltxexample}

Vous pouvez créer votre propre fichier \tkzname{tkz-base.cfg} que vous placerez dans un dossier qui sera prioritaire au sein du \tkzname{texmf}. 

\begin{center}
  \begin{tkzexample}[vbox]
  \begin{tikzpicture}[scale=0.75]
    \tkzInit[ymax=8] \tkzClip 
    \tkzDefPoints{0/1/A, 8/3/B, 3/6/C}      
    \tkzDrawPolygon(A,B,C)  
    \tkzSetUpCompass[color=red,line width=.2 pt] 
    \tkzDefLine[bisector](A,C,B) \tkzGetPoint{c}
    \tkzDefLine[bisector](B,A,C) \tkzGetPoint{a}
    \tkzDefLine[bisector](C,B,A) \tkzGetPoint{b} 
    \tkzShowLine[bisector,size=2,gap=3](A,C,B)
    \tkzShowLine[bisector,size=2,gap=3](B,A,C)
    \tkzShowLine[bisector,size=1,gap=2](C,B,A)
    \tkzDrawLines[add=0 and 0 ](B,b C,c)    
    \tkzDrawLine[add=0 and -.4 ](A,a)  
    \tkzLabelPoints(A,B) \tkzLabelPoints[above](C)
  \end{tikzpicture}      
  \end{tkzexample}
\end{center}



% section: (end)
\endinput
%!TEX root = /Users/ego/Boulot/TKZ/tkz-euclide/doc_fr/TKZdoc-euclide-main.tex


\section{Quelques exemples intéressants}

\subsection{Triangles isocèles semblables}

Ce qui suit provient de l'excellent site \textbf{Descartes et les Mathématiques}. Je n'ai pas modifié le texte  et je ne suis l'auteur que de la programmation des figures.

\url{http://debart.pagesperso-orange.fr/seconde/triangle.html}

Bibliographie : Géométrie au Bac - Tangente, hors série no 8 - Exercice 11, page 11

Élisabeth Busser et Gilles Cohen : 200 nouveaux problèmes du Monde - POLE 2007

Affaire de logique n° 364 - Le Monde 17 février 2004


Deux énoncés ont été proposés, l'un par la revue \emph{Tangente}, et l'autre par le journal \emph{Le Monde}.

\vspace*{2cm}
\emph{Rédaction de la revue Tangente} : \textcolor{orange}{On construit deux triangles isocèles semblables AXB et BYC de sommets principaux X et Y, tels que A, B et C soient alignés et que ces triangles soient « indirect ». Soit $\alpha$ l'angle au sommet $\widehat{AXB}$ = $\widehat{BYC}$. On construit ensuite un troisième triangle isocèle XZY semblable aux deux premiers, de sommet principal Z et « indirect ».\\
On demande de démontrer que le point Z appartient à la droite (AC).}

\vspace*{2cm}
\emph{Rédaction du Monde} : \textcolor{orange}{On construit deux triangles isocèles semblables AXB et BYC de sommets principaux X et Y, tels que A, B et C soient alignés et que ces triangles soient « indirect ». Soit $\alpha$ l'angle au sommet $\widehat{AXB}$ = $\widehat{BYC}$. Le point Z du segment [AC] est équidistant des deux sommets X et Y.\\
Sous quel angle voit-il ces deux sommets ?}

\vspace*{2cm}  Les constructions et leurs codes associés sont sur les deux pages suivantes, mais vous pouvez chercher avant de regarder. La programmation respecte (il me semble ...), mon raisonnement dans les deux cas.
\newpage  

 \subsubsection{version revue "Tangente"} 
\begin{tkzexample}[vbox]
\begin{tikzpicture}[scale=.8,rotate=60]
  \tkzDefPoint(6,0){X}   \tkzDefPoint(3,3){Y}
  \tkzDefShiftPoint[X](-110:6){A}    \tkzDefShiftPoint[X](-70:6){B}
  \tkzDefShiftPoint[Y](-110:4.2){A'} \tkzDefShiftPoint[Y](-70:4.2){B'}
  \tkzDefPointBy[translation= from A' to B ](Y) \tkzGetPoint{Y}
  \tkzDefPointBy[translation= from A' to B ](B') \tkzGetPoint{C}
  \tkzInterLL(A,B)(X,Y) \tkzGetPoint{O}
  \tkzDefMidPoint(X,Y) \tkzGetPoint{I}
  \tkzDefPointWith[orthogonal](I,Y)
  \tkzInterLL(I,tkzPointResult)(A,B) \tkzGetPoint{Z}
  \tkzDrawCircle[circum](X,Y,B)
  \tkzDrawLines[add = 0 and 1.5](A,C) \tkzDrawLines[add = 0 and 3](X,Y)
  \tkzDrawSegments(A,X B,X B,Y C,Y)   \tkzDrawSegments[color=red](X,Z Y,Z)
  \tkzDrawPoints(A,B,C,X,Y,O,Z)
  \tkzLabelPoints(A,B,C,Z)   \tkzLabelPoints[above right](X,Y,O)
\end{tikzpicture} 
\end{tkzexample}  


 \newpage 
 
  \subsubsection{version  "Le Monde"} 
\begin{center}
\begin{tkzexample}[vbox]
\begin{tikzpicture}[scale=1.25]
  \tkzDefPoint(0,0){A} 
  \tkzDefPoint(3,0){B}
  \tkzDefPoint(9,0){C}
  \tkzDefPoint(1.5,2){X}
  \tkzDefPoint(6,4){Y}
   \tkzDefCircle[circum](X,Y,B) \tkzGetPoint{O}
  \tkzDefMidPoint(X,Y)               \tkzGetPoint{I}
  \tkzDefPointWith[orthogonal](I,Y)  \tkzGetPoint{i}
  \tkzDrawLines[add = 2 and 1,color=orange](I,i)
  \tkzInterLL(I,i)(A,B)              \tkzGetPoint{Z}
  \tkzInterLC(I,i)(O,B)              \tkzGetSecondPoint{M}
    \tkzDefPointWith[orthogonal](B,Z)  \tkzGetPoint{b}
  \tkzDrawCircle(O,B)
  \tkzDrawLines[add = 0 and 2,color=orange](B,b)
   \tkzDrawSegments(A,X B,X B,Y C,Y A,C X,Y)
   \tkzDrawSegments[color=red](X,Z Y,Z)
  \tkzDrawPoints(A,B,C,X,Y,Z,M,I)
   \tkzLabelPoints(A,B,C,Z)
   \tkzLabelPoints[above right](X,Y,M,I)
\end{tikzpicture}
\end{tkzexample} 
\end{center}


\newpage
\subsection{Hauteurs d'un triangle}

Ce qui suit provient encore de l'excellent site \textbf{Descartes et les Mathématiques}. 

\url{http://debart.pagesperso-orange.fr/geoplan/geometrie_triangle.html}

Les trois hauteurs d'un triangle sont concourantes au même point H.

\begin{center}
\begin{tkzexample}[vbox]
\begin{tikzpicture}[scale=1.25]
  \tkzInit[xmin= 0,xmax=8 ,ymin=0 ,ymax=7 ] \tkzClip[space=.5]
   \tkzDefPoint(0,0){C} 
   \tkzDefPoint(7,0){B}
   \tkzDefPoint(5,6){A}
   \tkzDrawPolygon(A,B,C)
   \tkzDefMidPoint(C,B)          \tkzGetPoint{I}
   \tkzDrawArc(I,B)(C)
   \tkzInterLC(A,C)(I,B)        \tkzGetSecondPoint{B'}
   \tkzInterLC(A,B)(I,B)        \tkzGetFirstPoint{C'}
   \tkzInterLL(B,B')(C,C')      \tkzGetPoint{H}
   \tkzInterLL(A,H)(C,B)        \tkzGetPoint{A'}
   \tkzDrawCircle[circum,color=red](A,B',C') 
   \tkzDrawSegments[color=orange](B,B' C,C' A,A')
   \tkzMarkRightAngles(C,B',B B,C',C C,A',A)
   \tkzDrawPoints(A,B,C,A',B',C',H)
   \tkzLabelPoints(A,B,C,A',B',C',H)
\end{tikzpicture}
\end{tkzexample}
\end{center}

\newpage
\subsection{Hauteurs - autre construction}

\begin{center}
\begin{tkzexample}[vbox]
\begin{tikzpicture}
  \tkzClip[space=1]
  \tkzDefPoint(0,0){A}\tkzDefPoint(8,0){B}\tkzDefPoint(3.5,10){C} 
  \tkzDefMidPoint(A,B) \tkzGetPoint{O} 
  \tkzDefPointBy[projection=onto A--B](C) \tkzGetPoint{P}
  \tkzInterLC(C,A)(O,A)  \tkzGetSecondPoint{M}
  \tkzInterLC(C,B)(O,A)  \tkzGetFirstPoint{N}
  \tkzInterLL(B,M)(A,N)  \tkzGetPoint{I}
  \tkzDrawCircle[diameter](A,B)
  \tkzDrawSegments(C,A C,B A,B B,M A,N) 
  \tkzMarkRightAngles[fill=Maroon!20](A,M,B A,N,B A,P,C)
  \tkzDrawSegment[style=dashed,color=orange](C,P)
  \tkzLabelPoints(O,A,B,P)
  \tkzLabelPoint[left](M){$M$} 
  \tkzLabelPoint[right](N){$N$}
  \tkzLabelPoint[above](C){$C$}  
  \tkzLabelPoint[fill=fondpaille,above right](I){$I$}
  \tkzDrawPoints[color=red](M,N,P,I) \tkzDrawPoints[color=Maroon](O,A,B,C)  
\end{tikzpicture}
\end{tkzexample}  
\end{center}

\endinput
%!TEX root = /Users/ego/Boulot/TKZ/tkz-euclide/doc_fr/TKZdoc-euclide-main.tex

\section{Gallery  : Some examples}

Some examples with explanations in english.
%–––––––––––––––––––––––––––––––––––––––––––––––––––––––––––––––––––––––––––>

\subsection{White on Black}
This example shows how to get a segment with a length equal at $\sqrt{a}$ from a segment of length $a$, only with a rule and a compass.


\begin{center}
\begin{tkzexample}[]
  \tikzset{background rectangle/.style={fill=black}} 
\begin{tikzpicture}[show background rectangle]
   \tkzInit[ymin=-1.5,ymax=7,xmin=-1,xmax=+11]
   \tkzClip 
   \tkzDefPoint(0,0){O}
   \tkzDefPoint(1,0){I}
   \tkzDefPoint(10,0){A}
   \tkzDefPointWith[orthogonal](I,A) \tkzGetPoint{H}
   \tkzDefMidPoint(O,A) \tkzGetPoint{M}
   \tkzInterLC(I,H)(M,A)\tkzGetPoints{C}{B}
   \tkzDrawSegments[color=white,line width=1pt](I,H O,A)
   \tkzDrawPoints[color=white](O,I,A,B,M) 
   \tkzMarkRightAngle[color=white,line width=1pt](A,I,B) 
   \tkzDrawArc[color=white,line width=1pt,style=dashed](M,A)(O) 
  \tkzLabelSegment[white,right=1ex,pos=.5](I,B){$\sqrt{a}$} 
  \tkzLabelSegment[white,below=1ex,pos=.5](O,I){$1$}   
  \tkzLabelSegment[pos=.6,white,below=1ex](I,A){$a$} 
\end{tikzpicture}
\end{tkzexample}
\end{center}

\vfill\newpage
%<–––––––––––––––––––––––––––––––––––––––––––––––––––––––––––––––––––––––––––>

\subsection{ Square root of the integers }       
How to get $1$, $\sqrt{2}$, $\sqrt{3}$ with a rule and a compass.
\begin{center}
\begin{tkzexample}[]
\begin{tikzpicture}[scale=1.75]
   \tkzInit[xmin=-3,xmax=4,ymin=-2,ymax=4]
   \tkzGrid
   \tkzDefPoint(0,0){O}
   \tkzDefPoint(1,0){a0}
   \newcounter{tkzcounter}
   \setcounter{tkzcounter}{0}
   \newcounter{density}
   \setcounter{density}{20}
   \foreach \i in {0,...,15}{%
      \pgfmathsetcounter{density}{\thedensity+2}
      \setcounter{density}{\thedensity}    
      \stepcounter{tkzcounter}
      \tkzDefPointWith[orthogonal normed](a\i,O)
      \tkzGetPoint{a\thetkzcounter}
      \tkzDrawPolySeg[color=Maroon!\thedensity,%
         fill=Maroon!\thedensity,opacity=.5](a\i,a\thetkzcounter,O)}
 \end{tikzpicture}  
\end{tkzexample}
\end{center}

%<–––––––––––––––––––––––––––––––––––––––––––––––––––––––––––––––––––––––––––>
 \vfill\newpage
%<–––––––––––––––––––––––––––––––––––––––––––––––––––––––––––––––––––––––––––>
% 
\subsection{How to construct the tangent lines from a point to a circle with a rule and a compass.}
\begin{center}
\begin{tkzexample}[] 
  \begin{tikzpicture}
    \tkzPoint(0,0){O}
    \tkzPoint(9,2){P}
    \tkzDefMidPoint(O,P) \tkzGetPoint{I}
    \tkzDrawCircle[R](O,4cm)
    \tkzDrawCircle[diameter](O,P)
    \tkzCalcLength(I,P)  \tkzGetLength{dIP}
    \tkzInterCC[R](O,4cm)(I,\dIP pt)\tkzGetPoints{Q1}{Q2}
    \tkzDrawPoint[color=red](Q1)
    \tkzDrawPoint[color=red](Q2)
    \tkzDrawLine(P,Q1) 
    \tkzDrawLine(P,Q2) 
    \tkzDrawSegments(O,Q1 O,Q2)
    \tkzDrawLine(P,O)
\end{tikzpicture}
\end{tkzexample}
\end{center} 
% 
% %<–––––––––––––––––––––––––––––––––––––––––––––––––––––––––––––––––––––––––––>
 \vfill\newpage
%<–––––––––––––––––––––––––––––––––––––––––––––––––––––––––––––––––––––––––––>

\subsection{Circle and tangent}
We have a point A $(8,2)$, a circle with center A and radius=3cm and a line
  $\delta$ $y=4$. The line intercepts the circle at B. We want to draw the tangent at the circle in B.
   
\begin{center}
\begin{tkzexample}[]
\begin{tikzpicture}
  \tkzInit[xmax=14,ymin=-2,ymax=6]
  \tkzDrawX[noticks,label=$(d)$]
  \tkzPoint[pos=above right](8,2){A};
  \tkzPoint[color=red,pos=above right](0,0){O};
  \tkzDrawCircle[R,color=blue,line width=.8pt](A,3 cm)
  \tkzHLine[color=red,style=dashed]{4} 
  \tkzText[above](12,4){$\delta$}
  \FPeval\alphaR{arcsin(2/3)}% on a les bonnes valeurs
  \FPeval\xB{8-3*cos(\alphaR)}
  \tkzPoint[pos=above left](\xB,4){B};
  \tkzDrawSegment[line width=1pt](A,B)
  \tkzDefLine[orthogonal=through B](A,B) \tkzGetPoint{b}
  \tkzDefPoint(1,0){i}
  \tkzInterLL(B,b)(O,i) \tkzGetPoint{B'}
  \tkzDrawPoint(B')
  \tkzDrawLine(B,B')
 \end{tikzpicture}
\end{tkzexample}
\end{center}

 \vfill\newpage
%<–––––––––––––––––––––––––––––––––––––––––––––––––––––––––––––––––––––––––––>

\subsection{About right triangle}

We have a segment $[AB]$ and we want to determine a point $C$ such as $AC=8 cm$ and $ABC$ is a right triangle in $B$.

\begin{center}
\begin{tkzexample}[]
\begin{tikzpicture}
  \tkzInit
  \tkzClip
  \tkzPoint[pos=left](2,1){A}
  \tkzPoint(6,4){B} 
  \tkzDrawSegment(A,B)
  \tkzDrawPoint[color=red](A)
  \tkzDrawPoint[color=red](B)
  \tkzDefPointWith[orthogonal,K=-1](B,A)    
  \tkzDrawLine[add = .5 and .5](B,tkzPointResult)
  \tkzInterLC[R](B,tkzPointResult)(A,8 cm) \tkzGetPoints{C}{J}
  \tkzDrawPoint[color=red](C)
  \tkzCompass(A,C)
  \tkzMarkRightAngle(A,B,C)
  \tkzDrawLine[color=gray,style=dashed](A,C)
\end{tikzpicture} 
\end{tkzexample}
\end{center}

 %<–––––––––––––––––––––––––––––––––––––––––––––––––––––––––––––––––––––––––––>
 \vfill\newpage %<–––––––––––––––––––––––––––––––––––––––––––––––––––––––––––––––––––––––––––>

\subsection{Archimedes}

This is an ancient problem  proved by the great Greek mathematician Archimedes .
The figure below shows a semicircle, with diameter $AB$. A tangent line is drawn and  touches the semicircle at $B$.  An other tangent line at a point, $C$, on the semicircle is drawn. We project the point $C$ on the segment$[AB]$  on a point $D$ . The two tangent lines intersect at the point $T$.

Prove that the line $(AT)$ bisects $(CD)$

\begin{center}
\begin{tkzexample}[]  
\begin{tikzpicture}[scale=1.25] 
   \tkzInit[ymin=-1,ymax=7]
   \tkzClip
   \tkzDefPoint(0,0){A}\tkzDefPoint(6,0){D} 
   \tkzDefPoint(8,0){B}\tkzDefPoint(4,0){I}
   \tkzDefLine[orthogonal=through D](A,D)
   \tkzInterLC[R](D,tkzPointResult)(I,4 cm) \tkzGetFirstPoint{C}
   \tkzDefLine[orthogonal=through C](I,C)   \tkzGetPoint{c}
   \tkzDefLine[orthogonal=through B](A,B)   \tkzGetPoint{b}
   \tkzInterLL(C,c)(B,b) \tkzGetPoint{T} 
   \tkzInterLL(A,T)(C,D) \tkzGetPoint{P}
   \tkzDrawArc(I,B)(A) 
   \tkzDrawSegments(A,B A,T C,D I,C) \tkzDrawSegment[color=orange](I,C)
   \tkzDrawLine[add = 1 and 0](C,T)  \tkzDrawLine[add = 0 and 1](B,T)
   \tkzMarkRightAngle(I,C,T)
   \tkzDrawPoints(A,B,I,D,C,T)  
   \tkzLabelPoints(A,B,I,D)  \tkzLabelPoints[above right](C,T)
   \tkzMarkSegment[pos=.25,mark=s|](C,D) \tkzMarkSegment[pos=.75,mark=s|](C,D)
\end{tikzpicture}  
\end{tkzexample}
\end{center}  

\subsection{Example from Dimitris Kapeta}

You need in this example to use \tkzname{mkpos=.2} with \tkzcname{tkzMarkAngle} because the measure of $ \widehat{CAM}$ is too small.
Another possiblity is to use \tkzcname{tkzFillAngle}.

\begin{center}
\begin{tkzexample}[]
\begin{tikzpicture}[scale=1.25]
  \tkzInit[xmin=-5.2,xmax=3.2,ymin=-3.2,ymax=3.3]
  \tkzClip 
  \tkzDefPoint(0,0){O}
  \tkzDefPoint(2.5,0){N}
  \tkzDefPoint(-4.2,0.5){M}
  \tkzDefPointBy[rotation=center O angle 30](N)
  \tkzGetPoint{B}
  \tkzDefPointBy[rotation=center O angle -50](N)
  \tkzGetPoint{A}
  \tkzInterLC(M,B)(O,N) \tkzGetFirstPoint{C}
  \tkzInterLC(M,A)(O,N) \tkzGetSecondPoint{A'} 
  \tkzMarkAngle[fill=blue!25,mkpos=.2, size=0.5](A,C,B) 
  \tkzMarkAngle[fill=green!25,mkpos=.2, size=0.5](A,M,C)
  \tkzDrawSegments(A,C M,A M,B)
  \tkzDrawCircle(O,N)
  \tkzLabelCircle[above left](O,N)(120){$\mathcal{C}$}
  \tkzMarkAngle[fill=red!25,mkpos=.2, size=0.5cm](C,A,M)
  \tkzDrawPoints(O, A, B, M, B, C)
  \tkzLabelPoints[right](O,A,B)
  \tkzLabelPoints[above left](M,C)
  \tkzLabelPoint[below left](A'){$A'$}
\end{tikzpicture}
\end{tkzexample}
\end{center}

\newpage
\subsection{Example 1 from John Kitzmiller }
This figure is the last of beamer document. You can find the document on  my site 

Prove $\bigtriangleup LKJ$ is equilateral
  
\begin{center}
\begin{tkzexample}[vbox]
\begin{tikzpicture}[scale=1.5]
  \tkzDefPoint[label=below left:A](0,0){A}
  \tkzDefPoint[label=below right:B](6,0){B}
  \tkzDefTriangle[equilateral](A,B) \tkzGetPoint{C}
  \tkzMarkSegments[mark=|](A,B A,C B,C)
  \tkzDefBarycentricPoint(A=1,B=2) \tkzGetPoint{C'}
  \tkzDefBarycentricPoint(A=2,C=1) \tkzGetPoint{B'}
  \tkzDefBarycentricPoint(C=2,B=1) \tkzGetPoint{A'}
  \tkzInterLL(A,A')(C,C') \tkzGetPoint{J}
  \tkzInterLL(C,C')(B,B') \tkzGetPoint{K}
  \tkzInterLL(B,B')(A,A') \tkzGetPoint{L}
  \tkzLabelPoint[above](C){C}
  \tkzDrawPolygon(A,B,C) \tkzDrawSegments(A,J B,L C,K)
  \tkzMarkAngles[fill= orange,size=1cm,opacity=.3](J,A,C K,C,B L,B,A)
  \tkzLabelPoint[right](J){J}
  \tkzLabelPoint[below](K){K}
  \tkzLabelPoint[above left](L){L}
  \tkzMarkAngles[fill=orange, opacity=.3,thick,size=1,](A,C,J C,B,K B,A,L)
  \tkzMarkAngles[fill=green, size=1, opacity=.5](A,C,J C,B,K B,A,L)
  \tkzFillPolygon[color=yellow, opacity=.2](J,A,C)
  \tkzFillPolygon[color=yellow, opacity=.2](K,B,C)
  \tkzFillPolygon[color=yellow, opacity=.2](L,A,B)
  \tkzDrawSegments[line width=3pt,color=cyan,opacity=0.4](A,J C,K B,L)
  \tkzDrawSegments[line width=3pt,color=red,opacity=0.4](A,L B,K C,J)
  \tkzMarkSegments[mark=o](J,K K,L L,J)
\end{tikzpicture}  
\end{tkzexample}  

\end{center} 

\newpage
\subsection{Example 2 from John Kitzmiller }    
Prove $\dfrac{AC}{CE}=\dfrac{BD}{DF} \qquad$

Another interesting example from John, you can see how to use some extra options like \tkzname{decoration} and \tkzname{postaction}  from \TIKZ\ with \tkzname{tkz-euclide}.

\begin{center}
\begin{tkzexample}[vbox]
\begin{tikzpicture}[scale=1.5,decoration={markings,
  mark=at position 3cm with {\arrow[scale=2]{>}};}]
  \tkzInit[xmin=-0.25,xmax=6.25, ymin=-0.5,ymax=4]
  \tkzClip
  \tkzDefPoints{0/0/E, 6/0/F, 0/1.8/P, 6/1.8/Q, 0/3/R, 6/3/S}
  \tkzDrawLines[postaction={decorate}](E,F P,Q R,S)
  \tkzDefPoints{3.5/3/A, 5/3/B}
  \tkzDrawSegments(E,A F,B)
  \tkzInterLL(E,A)(P,Q) \tkzGetPoint{C}
  \tkzInterLL(B,F)(P,Q) \tkzGetPoint{D}
  \tkzLabelPoints[above right](A,B)
  \tkzLabelPoints[below](E,F)
  \tkzLabelPoints[above left](C)
  \tkzDrawSegments[style=dashed](A,F)
  \tkzInterLL(A,F)(P,Q) \tkzGetPoint{G}
  \tkzLabelPoints[above right](D,G) 
  \tkzDrawSegments[color=teal, line width=3pt, opacity=0.4](A,C A,G)
  \tkzDrawSegments[color=magenta, line width=3pt, opacity=0.4](C,E G,F)
  \tkzDrawSegments[color=teal, line width=3pt, opacity=0.4](B,D)
  \tkzDrawSegments[color=magenta, line width=3pt, opacity=0.4](D,F)
\end{tikzpicture} 
\end{tkzexample}
\end{center}

\newpage
\subsection{Example 3 from John Kitzmiller }    
Prove $\dfrac{BC}{CD}=\dfrac{AB}{AD} \qquad$ (Angle Bisector)


\begin{center}
\begin{tkzexample}[vbox]
\begin{tikzpicture}[scale=1.5] 
  \tkzInit[xmin=-4,xmax=5,ymax=4.5]   \tkzClip[space=.5]
  \tkzDefPoints{0/0/B, 5/0/D}         \tkzDefPoint(70:3){A}
  \tkzDrawPolygon(B,D,A)
  \tkzDefLine[bisector](B,A,D)         \tkzGetPoint{a}
  \tkzInterLL(A,a)(B,D)                \tkzGetPoint{C}
  \tkzDefLine[parallel=through B](A,C) \tkzGetPoint{b}
  \tkzInterLL(A,D)(B,b)                \tkzGetPoint{P}
  \begin{scope}[decoration={markings,
    mark=at position .5 with {\arrow[scale=2]{>}};}]
    \tkzDrawSegments[postaction={decorate},dashed](C,A P,B) 
  \end{scope}
  \tkzDrawSegment(A,C) \tkzDrawSegment[style=dashed](A,P)  
  \tkzLabelPoints[below](B,C,D) \tkzLabelPoints[above](A,P)
  \tkzDrawSegments[color=magenta, line width=3pt, opacity=0.4](B,C P,A)
  \tkzDrawSegments[color=teal,     line width=3pt, opacity=0.4](C,D A,D)
  \tkzDrawSegments[color=magenta, line width=3pt, opacity=0.4](A,B)
  \tkzMarkAngles[size=0.7](B,A,C C,A,D)
  \tkzMarkAngles[size=0.7, fill=green,  opacity=0.5](B,A,C A,B,P)
  \tkzMarkAngles[size=0.7, fill=yellow, opacity=0.3](B,P,A C,A,D)
  \tkzMarkAngles[size=0.7, fill=green,  opacity=0.6](B,A,C A,B,P B,P,A C,A,D)
  \tkzLabelAngle[pos=1](B,A,C){1}     \tkzLabelAngle[pos=1](C,A,D){2}
  \tkzLabelAngle[pos=1](A,B,P){3})    \tkzLabelAngle[pos=1](B,P,A){4}
  \tkzMarkSegments[mark=|](A,B A,P) 
\end{tikzpicture}   
\end{tkzexample}
\end{center} 

\newpage
\subsection{Example 4 from John Kitzmiller }    
Prove $\overline{AG}\cong\overline{EF} \qquad$ (Detour)

\begin{center}
\begin{tkzexample}[vbox]
\begin{tikzpicture}[scale=2]
  \tkzInit[xmax=5, ymax=5]
  \tkzDefPoint(0,3){A}    \tkzDefPoint(6,3){E}  \tkzDefPoint(1.35,3){B}
  \tkzDefPoint(4.65,3){D} \tkzDefPoint(1,1){G}  \tkzDefPoint(5,5){F} 
  \tkzDefMidPoint(A,E)    \tkzGetPoint{C}       
  \tkzFillPolygon[yellow, opacity=0.4](B,G,C)
  \tkzFillPolygon[yellow, opacity=0.4](D,F,C)
  \tkzFillPolygon[blue, opacity=0.3](A,B,G)
  \tkzFillPolygon[blue, opacity=0.3](E,D,F)
  \tkzMarkAngles[size=0.6,fill=green](B,G,A D,F,E)
  \tkzMarkAngles[size=0.6,fill=orange](B,C,G D,C,F)
  \tkzMarkAngles[size=0.6,fill=yellow](G,B,C F,D,C)
  \tkzMarkAngles[size=0.6,fill=red](A,B,G E,D,F)
  \tkzMarkSegments[mark=|](B,C D,C)  \tkzMarkSegments[mark=s||](G,C F,C)
  \tkzMarkSegments[mark=o](A,G E,F)  \tkzMarkSegments[mark=s](B,G D,F)
  \tkzDrawSegment[color=red](A,E)
  \tkzDrawSegment[color=blue](F,G)
  \tkzDrawSegments(A,G G,B E,F F,D) 
  \tkzLabelPoints[below](C,D,E,G)    \tkzLabelPoints[above](A,B,F)  
\end{tikzpicture}
\end{tkzexample}
\end{center}   
\endinput
%!TEX root = /Users/ego/Boulot/TKZ/tkz-euclide/doc_fr/TKZdoc-euclide-main.tex

\section{FAQ} 
\subsection{Erreurs les plus fréquentes}
 Je me base pour le moment sur les miennes, car ayant changé plusieurs fois de syntaxes, j'ai commis un certain nombre d'erreurs. Cette section est amenée à se développer.
 
 \begin{itemize}\setlength{\itemsep}{10pt}
  \item \tkzcname{tkzDrawPoint(A,B)} alors qu'il faut  \tkzcname{tkzDrawPoints}
  \item  \tkzcname{tkzGetPoint(A)} Quand on définit un objet, il faut utiliser des accolades et non des parenthèses, il faut donc écrire~: \tkzcname{tkzGetPoint\{A\}}
  
    \item \tkzcname{tkzGetPoint\{A\}} à la place de \tkzcname{tkzGetFirstPoint\{A\}}. Quant une macro donne deux points comme résultats, soit on récupère ces points  à l'aide de \tkzcname{tkzGetPoints\{A\}\{B\}}, soit on ne récupère que l'un des deux points, à l'aide  \tkzcname{tkzGetFirstPoint\{A\}} ou bien de \tkzcname{tkzGetSecondPoint\{A\}}. Ces deux points peuvent être utilisés avec comme référence \tkzname{tkzFirstPointResult} ou  \tkzname{tkzSecondPointResult}. Il est possible qu'un troisième point soit donné sous la référence \tkzname{tkzPointResult}  
     
  \item \tkzcname{tkzDrawSegment(A,B A,C)} alors qu'il faut  \tkzcname{tkzDrawSegments}. Il est possible de n'utiliser que les versions avec un « s » mais c'est moins efficace!
  \item Mélange option et arguments; toutes les macros  qui utilisent un cercle ont besoin de connaître le rayon de celui-ci. Si le rayon est donné par une mesure alors l'option comprend un \tkzname{R}.

\item  \tkzcname{tkzDrawSegments[color = gray,style=dashed]\{B,B' C,C'\}} est une erreur. Seules, les macros qui définissent un objet utilisent des accolades.   
  \item Les angles sont donnés en degrés 
  
  \item Si une erreur survient dans un calcul lors d'un passage de paramètres, alors il est préférable de faire ces calculs avant d'appeler la macro.
  \item Ne pas mélanger la syntaxe de \tkzNamePack{pgfmath} et celle de \tkzNamePack{fp.sty}. J'ai choisi souvent \tkzname{fp.sty} mais si vous préférez  pgfmath alors effectuez vos calculs avant le passage de paramètres.

%\tkzDrawLines[add=0 and 8]( A,a B,b) au lieu de \tkzDrawLines[add=0 and 8](A,a B,b)

%\tkzActivOff 
%\tkzDrawSegment[color=Maroon!50](I,H)

\item usage de \tkzcname{tkzClip} : Afin d'avoir des résultats  précis, j'ai évité de passer par des vecteurs normalisés. L'avantage de la normalisation est de contrôler la dimension des objets manipulés, le désavantage est qu'avec TeX, cela implique des erreurs. Ces erreurs sont souvent minimes, de l'ordre du millième, mais entraînent des catastrophes si le dessin est agrandi. Ne pas normaliser implique que certains points se trouvent bien loin de la zone de travail et seul \tkzcname{tkzClip} permet de réduire la taille du dessin. 


\item  une erreur se produit si vous utilisez la macro \tkzcname{tkzDrawAngle}
 avec un angle trop petit. L'erreur est produite par la librairie  \NameLib{decoration} quand on veut placer une marque sur un arc. Même si la marque est absente, l'erreur, elle, reste présente. Il est possible de contourner cette difficulté avec l'option \tkzname{mkpos=.2} par exemple, qui placera la marque avant l'arc. Une autre possibilité est d'utiliser la macro \tkzcname{tkzFillAngle}
\item Somme de deux vecteurs

Comment obtenir le point D tel que $\overrightarrow{AD} = \overrightarrow{AB} + \overrightarrow{AC}$?

\begin{tkzexample}[latex=5 cm,small]
  \begin{tikzpicture}[scale=.5]
  \tkzDefPoint(1,1){A}
  \tkzDefPoint(8,0){B}
  \tkzDefPoint(3,4){C} 
  \tkzDefVector[colinear= at C](A,B){D}
  \tkzDrawVectors[color=blue](A,B A,C) 
  \tkzDrawVector[color=red](A,D) 
  \tkzLabelPoints(A,B,C,D)  
\end{tikzpicture}
\end{tkzexample}

  \end{itemize}    
\endinput         
\clearpage\newpage
\printindex

\end{document}
