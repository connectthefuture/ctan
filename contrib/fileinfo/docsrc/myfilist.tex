\ProvidesFile{myfilist.tex}[2012/11/30 documenting myfilist.sty]
\title{\textsf{myfilist}\\---\\
       List Infos on just the Files\\You Want to Know About%
       \GetFileInfo{myfilist.sty}\thanks{%
            This file describes version 
            \textcolor{blue}{\fileversion} % \textbf{\fileversion}
            of \textsf{\filename} as of \filedate.}}
% \listfiles

%% Preprocessing:
\RequirePackage{makedoc}
\CopyFDconditionFromTo{PScomment}{comment}
% \MainDocParser{\ProcessInputWith{PScomment}}
%% Replacements:
\SetPatternCodes{\MakeOther\\\MakeOther\ } %% CARE!
\MakeExpandableAllReplacer{TeX}{TeX }{\TeX{} }{LEAVE}   %% 2012/09/28
% \let\do\PrependExpandableAllReplacer                    %% 2012/09/28
\renewcommand*{\do}{\PrependExpandableAllReplacer*}  %% re 2012/11/22
\do{LaTeX }{\LaTeX{} } 
\do{LaTeX2e }{\LaTeXe{} } 
\do{ .log}{ `.log'} 
\do{ .tex}{ `.tex'} 
\do{ .cls}{ `.cls'} 
\do{ .sty}{ `.sty'} 
\do{NOTE: }{\textbf{Note: }} 
\SetPatternCodes{\MakeOther\\}             %% less care needed 
\do{...}{$\dots$} 
\do{\ProvidesFile}{`\ProvidesFile'} 
\do{"loads"}{``loads''} 
\do{USAGE:}{\section{Usage}} 
\do{TRICKS:}{\section{Tricks, Package Option}} 
%% 2012/09/28 for v0.5, generation support:
\do{VARIANTS:}{\section{Variants}} 
% \do{<txt-file>}{\MetaVar{txt-file}}     %% 2012/09/28
\do{IMPLEMENTATION:}{\section{Implementation}} 
\do{gather.tex}{`gather.tex'}
\do{readprov.pdf}{`readprov.pdf'}
\do{readprov-.sty}{readprov.sty}            %% 2010/11/26
\do{readprov.sty}{'readprov.sty'}
\do{myfilist-.sty}{myfilist.sty}            %% 2010/11/26
\do{myfilist.sty}{'myfilist.sty'}           %% 2012/09/28
\do{ifnextok.sty}{'ifnextok.sty'} 
\do{adhocfilelist}{'adhocfilelist'} 
\do{filedate.sty}{'filedate.sty'}           %% 2012/10/25
\do[\MakeOther\ ]%
{run gather.tex}{\emph{run} gather.tex}
  %% <- TODO \PrependStandardEnhancement...
\SetCorrectHookJobLast
\ResetPatternCodes

\MainDocParser{\ProcessInputWith{PScomment}}
\LaTeXresultFile{myfilist.doc}
\MakeCloseDoc*{myfilist.sty}                        %% 2012/03/18
  %% <- we must NOT read mdoccorr.cfg here! 
  %%    (or put the above setup into a new one!) ->
% \MakeJobDoc{0}{\ProcessInputWith{PScomment}}

%% Typesetting:
\documentclass{article} 
\usepackage{color} %% for highlighting package version!? TODO
\ProvidesFile{makedoc.cfg}[2011/06/27 documentation settings] 

\author{Uwe L\"uck\thanks{\url{http://contact-ednotes.sty.de.vu}}}
% \author{Uwe L\"uck---{\tt http://contact-ednotes.sty.de.vu}}

%% hyperref:
\RequirePackage{ifpdf}
\usepackage[%
  \ifpdf
%     bookmarks=false,          %% 2010/12/22
%     bookmarksnumbered,
    bookmarksopen,              %% 2011/01/24!?
    bookmarksopenlevel=2,       %% 2011/01/23
%     pdfpagemode=UseNone,
%     pdfstartpage=10,
%     pdfstartview=FitH,
    citebordercolor={ .6 1    .6},
    filebordercolor={1    .6 1},
    linkbordercolor={1    .9  .7},
     urlbordercolor={ .7 1   1},   %% playing 2011/01/24
  \else
    draft
  \fi
]{hyperref}

\RequirePackage{niceverb}[2011/01/24] 
\RequirePackage{readprov}               %% 2010/12/08
\RequirePackage{hypertoc}               %% 2011/01/23
\RequirePackage{texlinks}               %% 2011/01/24
\makeatletter
  \@ifundefined{strong} 
               {\let\strong\textbf}     %% 2011/01/24
               {} 
  \@ifundefined{file} 
               {\let\file\texttt}       %% 2011/05/23
               {} 
\makeatother

\errorcontextlines=4
\pagestyle{headings}

\endinput


\ReadPackageInfos{myfilist}
\usepackage{wiki}
%% TeX markup inserted by the txt-to-TeX function must be 
%% declared here for using 'niceverb''s ``auto mode.'' 
\AddToNoVerbList{\LaTeXe\dots\textbf\emph\ 
                 %% <- at end of line! 2012/11/22 TODO not working!?!
                 \\                         %% 2012/11/22
                 \TeX\LaTeX}                %% 2012/09/28
\begingroup \MakeActive\" 
 \gdef\fineDQ{``\begingroup
   \let\do\MakeOther \dospecials \tt
   \def"{\endgroup''}}%
\endgroup

\pagestyle{headings}
\usepackage{parskip}

\MDfinaldatechecks
\begin{document}

\maketitle 

\begin{abstract}\sloppy\noindent
  'myfilist' addresses lazy file versions management, 
  when you move your package or chapter files through 
  various computers and various directories and after a while 
  wonder where the most recent versions are.

  Like Paul Ebermann's 
  'dateiliste',\urlfoot{CtanPkgRef}{dateiliste}
  'myfilist' varies \LaTeX's `\listfiles' 
  for listing file (especially version) informations. 
  Differences to 'dateiliste' and \LaTeX\ are:
  \begin{enumerate}
    \item You choose the files (and their order) to be listed;
    \item indeed (according to original intention): 
          this has very little to do with files used in 
          typesetting some document;
    \item output is just screen, `.log', or a `.txt'-type file 
          that you choose. 
  \end{enumerate}
  %% 2012/11/22 while mentioning 'nicetext' moves to final section:
  However, `\ListFileInfos[<txt-file>]' from here can also be used 
  to \emph{with} typesetting in order to get `\listfiles' output 
  in a separate plain text file `<txt-file>' 
  (\emph{without} remaining `.log' content).

\end{abstract}

% \pagebreak
\tableofcontents

\section{Installing}
The file '\jobname.sty' is provided ready, installation only requires 
putting it somewhere where \TeX\ finds it 
(which may need updating the filename data 
 base).\urlfoot{ukfaqref}{inst-wlcf}

\section{File Info Header}
\sloppy 
\let\\\MDdocnewline                         %% 2012/11/22
\setlength\topsep{\smallskipamount}         %% 2012/11/22
\enlargethispage{2\baselineskip}            %% 2012/11/22
\wikiEnvironments      %% TODO fails with "or" 2010/03/31
\MakeActiveLetHere\"\fineDQ %% fails with "loads"
\AutoCmdInput{myfilist.doc}
\nowikiEnvironments

\section{Example}
% \setlength{\parskip}{0pt}           %% 2012/11/13
\AddQuotes
`gather.tex' for the present bundle has been as follows.
%   %% TODO 2010/11/27:
% (added development versions of 'fifinddo.sty' etc.):
%% 2012/11/12:
`filedate' and `fdatechk' refer to checking ``date consistency''
with the \ctanpkgref{filedate} package, at the same time 
as updating the ``source file list'' `SrcFILEs.txt' 
which is generated by
\[`\NoStopListInfos[SrcFILEs.txt]'\]
Some entries refer to the \ctanpkgref{nicetext} bundle 
that has been used (``\code{----USED.---}'') 
to generate the documentation file `myfilist.pdf'.
\setlength{\topsep}{.5\bigskipamount}   %% 2012/11/30
\MDsamplecodeinput{gather}              %% 2010/11/27 2012/11/13
As to `fdatechk.tex':
\MDsamplecodeinput{fdatechk}

\section{Relation to 'nicetext' bundle.}
%% 2012/11/22 moves from abstract
The code of this package was ready in spring 2008; 
in spring 2010 I presented it as a kind of study on improving 
'nicetext''s\urlfoot{CtanPkgRef}{nicetext}
`txt'-to-\LaTeX\ processing after 'nicetext v0.4' 
(aim was to do this without modifying the documented file, 
 yet I did not obey this rule strictly here).
The new idea is adding 'wiki.sty''s ``environments'' 
feature to 'makedoc' and 'niceverb''s ``auto mode'' %% was " 2012/11/22
in order to interprete `txt' comment indents in the package file, 
while 'wiki.sty''s \emph{font} switching 
still is not compatible with 'niceverb''s. Easy script 
commands for achieving this are still missing (sorry; see the code 
in 'myfilist.tex' that achieved the present formatting.)

\end{document}
