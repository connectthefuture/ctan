\documentclass[10pt,private]{hletter}
%\documentclass[11pt,german,bruni]{hletter}

\begin{document}

\signature{}

\reference{Impressions of Lausanne}

\date{Lausanne, le 15 septembre 2008}

\begin{letter}{Sir F. Treves, Bart.,\\
               \textbf{Vevey.}\\
               Switzerland}

% if you want the top-right text to be in [..]s use {..} as well:
\opening[\textsc{[draft]}]{Sir,}

%The value of \verb+\chaptername+ is \chaptername.

Lausanne, the capital of the Canton of Vaud, is the smaller of the two
cities of the Lake. It stands on a green slope which glides, in
leisurely fashion, from the wood which crowns its summit to the beach
at its foot. The town is far up on this slope, being about a mile and a
quarter above the port of Ouchy.

Seen from the Lake, it is so discursive a city that no one could venture to
define its outlines. Its houses are scattered in all directions, among trees
and lawns, gardens and green fields. It is as if a drop of stone- coloured
paint, falling from a height, had been spattered over a green cloth.

From ancient prints it can be seen that old Lausanne was a very romantic
looking town. Its three hills were crowned with castle and spire, with
turrets and high- soaring roofs; while around it ran a zigzag wall pierced by
gates and surmounted by many towers. The dwellings that made up the mass of
the city were of dark wood with lofty gables. They huddled in the valleys
like a drift of autumn leaves in a gully. Of the fortifications, no trace
remains with the exception of one tower, the Tour de l'Ale, which stands near
the Place du Chauderon on the St.~Laurent hill. It is a high round tower of
the days of the musketeers, which finds itself now very inappropriately
placed in a modest street of private houses.

\closing{Yours sincerely,}

\vspace{2cm}
\cc{All Smiths in London\\ Mademoiselle S. Curchod}

\encl{Tourist guide to Switzerland.\\ Plan of Cully.}

\end{letter}

\end{document}
