\documentclass[pagesize=auto, fontsize=12pt, DIV=10]{scrartcl}

\usepackage{fixltx2e}
\usepackage{etex}
\usepackage{xspace}
\usepackage{lmodern}
\usepackage[T1]{fontenc}
\usepackage{textcomp}
\usepackage{microtype}
\usepackage{hyperref}

\newcommand*{\mail}[1]{\href{mailto:#1}{\texttt{#1}}}
\newcommand*{\pkg}[1]{\textsf{#1}}
\newcommand*{\cs}[1]{\texttt{\textbackslash#1}}
\makeatletter
\newcommand*{\cmd}[1]{\cs{\expandafter\@gobble\string#1}}
\makeatother
\newcommand*{\opt}[1]{\texttt{#1}}

\addtokomafont{title}{\rmfamily}

\title{The \pkg{optional} package\thanks{This manual corresponds to \pkg{optional}~v2.2b, dated~2005/01/26.}}
\author{Donald Arseneau\thanks{\mail{asnd@triumf.ca}}}
\date{2005/01/26}


\begin{document}

\maketitle

\begin{abstract}
  \noindent
  Enable multiple versions of a document to be printed from one source file,
  especially if most of the text is shared between versions.
\end{abstract}

\begin{quote}
  \scriptsize
  Copyright 1993, 1999, 2001, 2005 by Donald Arseneau (\mail{asnd@triumf.ca}).
  This software is released under the terms of the \LaTeX\ Project Public 
  License  (\url{ftp://ctan.tug.org/tex-archive/macros/latex/base/lppl.txt}).
  (Essentially: Free to use, copy, distribute (sell) and change, but, if
  changed, that fact must be made apparent to the user.)  It has a
  status of ``maintained''.
\end{quote}



\section{How to use}

One way to use this package is to declare (for example)
%
\begin{verbatim}
\usepackage[opta]{optional}
\end{verbatim}
%
at the beginning of your document, and flag optional text throughout 
your document like:
%
\begin{verbatim}
\opt{opta}{Do this if option opta was declared}
\opt{optb}{Do this if option optb was declared}
\opt{optx,opty}{Do this if either option optx or opty}
\opt{}{Never print this text!}
\opt{opta}{%% The "\appendix" call has already been made in the declaration
%% of the "appendices" environment (see thesis.tex).
\chapter{Pointless extras}
\label{app:Pointless}

\chapterquote{%
Le savant n'\'etudie pas la nature parce que cela est utile; \\
\indent il l'\'etudie parce qu'il y prend plaisir, \\
\indent et il y prend plaisir parce qu'elle est belle.}%
{Henri Poincar\'e, 1854--1912}

Appendixes (or should that be ``appendices''?) make you look really clever, 'cos
it's like you had more clever stuff to say than could be fitted into the main
bit of your thesis. Yeah. So everyone should have at least three of them\dots

\section{Like, duh}
\label{sec:Duh}
Padding? What do you mean?

\section{$y = \alpha x^2$}
\label{sec:EqnTitle}
See, maths in titles automatically goes bold where it should (and check the
table of contents: it \emph{isn't} bold there!) Check the source: nothing
needs to be specified to make this work. Thanks to Donald Arsenau for the
teeny hack that makes this work.

%% Big appendixes should be split off into separate files, just like chapters
%\input{app-myreallybigappendix}
}
\optv{xam}{Type: \verb|[root /]$ rm -r *|.}
\end{verbatim}
%
Note that both the package option and the ``\cmd{\opt}'' argument can contain
lists of options although, in practice, one or the other should be a
single option name.  Lists are allowed in both places to allow more
flexibility in the style of use. (But making the definitions much more
difficult, Grrr.)

Just as for ``\cmd{\includeonly}'', you will have to edit the main document
file to switch option codes (i.\,e.,\ change the ``\cmd{\usepackage}'' line).  
There are, however, several ways to use this package without altering 
the main document file: separate files, file-name sensing, interactive 
prompting, and command-line option selection.

Typically, different versions of a document will require different 
document class and package setup, besides the different tags for
\pkg{optional.sty}.  In that case it is best to have a separate main file
for each version of the document.  Each stub file will declare the 
document class and load some packages (including this one) and then 
input the rest of the document from a file common to all versions.
%
\begin{verbatim}
\documentclass[A0]{poster}
\usepackage[poster]{optional}
\input{my_paper}
\end{verbatim}
%
If the different opt-tags match the different stub file names (file
\texttt{poster.tex} will typeset the ``\opt{poster}'' version) then you can specify
%
\begin{verbatim}
\usepackage[\jobname]{optional}
\end{verbatim}
%
Alternatively, this ``\cmd{\jobname}'' technique can make use of symbolic links, 
if your computer system supports them, by having a single main input
file accessed under different names (and different ``\cmd{\jobname}''s).

Another scheme is to invoke \LaTeX\ with the command line such as:
%
\begin{verbatim}
latex "\def\UseOption{opta,optb}\input{file}"
\end{verbatim}
%
(with quoting appropriate to your operating system) then options ``\opt{opta}'' 
and ``\opt{optb}'' will be used in addition to any options specified with the
``\cmd{\usepackage}'' command.

You can prompt yourself to specify the option(s) with every run 
through \LaTeX:
%
\begin{verbatim}
\usepackage{optional}
\newcommand{\ExplainOptions}{man = users manual, check = checklist,
      ref = reference card, post = poster.}
\AskOption
\end{verbatim}
%
The definition of ``\cmd{\ExplainOptions}'' is optional; it only serves to help
the person who answers the question.  The ``\cmd{\AskOption}'' is also optional;
it will be executed automatically whenever optional.sty sees no list of
options.  This method is too tedious to use much.

The normal restrictions forbidding special characters in package options
and reference tags apply also the the tags used by the ``\cmd{\opt}'' command.

These are not `comment' macros: The optional text must be well-formed
with balanced braces, even if not printed.  The ``\cmd{\opt}'' command \emph{is}
completely `expandable' which means  it is robust and can even be used
in messages (``\cmd{\typeout}'').  

As usual, ``\cmd{\verb}'' commands and verbatim environments cannot be used 
in the argument to ``\cmd{\opt}''.  For this purpose there is a variant form
of ``\cmd{\opt}'' called ``\cmd{\optv}'' (optional verbatim) which may have a limited
class of verbatim material in the argument.  It can do so by leaving
the braces around the argument, which may have undesired side effects.
For an ``\cmd{\optv}'' argument to be successfully ignored, the verbatim material
must have balanced braces etc.

The ``\cmd{\opt}'' command is only intended for small sections of text.  If you
need to optionally include whole sections or chapters, put that material
in a separate file, and ``\cmd{\opt}''-ionally use an ``\cmd{\input}'' command:
%
\begin{verbatim}
\opt{internal}{\input{prog_listings}}
\end{verbatim}

\end{document}
