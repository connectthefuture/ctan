% \iffalse meta-comment
%
% File: romanbar.dtx
% Version: 2012/01/01 v1.0f
%
% Copyright (C) 2011 by
%    H.-Martin M"unch <Martin dot Muench at Uni-Bonn dot de>
%
% The original code for the bars was written by Prof. Enrico Gregorio
% (http://profs.sci.univr.it/~gregorio/), and published at
% http://tex.stackexchange.com/questions/24065/roman-numerals-formatting/24084#24084
% I (H.-Martin Münch) submitted this package to CTAN
% (after diverse updates of the code and writing a documentation).
%
% This work may be distributed and/or modified under the
% conditions of the LaTeX Project Public License, either
% version 1.3c of this license or (at your option) any later
% version. This version of this license is in
%    http://www.latex-project.org/lppl/lppl-1-3c.txt
% and the latest version of this license is in
%    http://www.latex-project.org/lppl.txt
% and version 1.3c or later is part of all distributions of
% LaTeX version 2005/12/01 or later.
%
% This work has the LPPL maintenance status "maintained".
%
% The Current Maintainer of this work is H.-Martin Muench.
%
% This work consists of the main source file romanbar.dtx,
% the README, and the derived files
%    romanbar.sty, romanbar.pdf,
%    romanbar.ins, romanbar.drv,
%    romanbar-example.tex, romanbar-example.pdf.
%
% Distribution:
%    CTAN:macros/latex/contrib/romanbar/romanbar.dtx
%    CTAN:macros/latex/contrib/romanbar/romanbar.pdf
%    CTAN:install/macros/latex/contrib/romanbar.tds.zip
%
% Cf. http://ctan.org/pkg/romanbar and
% http://www.tex.ac.uk/tex-archive/help/Catalogue/entries/romanbar.html
% (It might take some time for the entries to be created.)
%
% Unpacking:
%    (a) If romanbar.ins is present:
%           tex romanbar.ins
%    (b) Without romanbar.ins:
%           tex romanbar.dtx
%    (c) If you insist on using LaTeX
%           latex \let\install=y% \iffalse meta-comment
%
% File: romanbar.dtx
% Version: 2012/01/01 v1.0f
%
% Copyright (C) 2011 by
%    H.-Martin M"unch <Martin dot Muench at Uni-Bonn dot de>
%
% The original code for the bars was written by Prof. Enrico Gregorio
% (http://profs.sci.univr.it/~gregorio/), and published at
% http://tex.stackexchange.com/questions/24065/roman-numerals-formatting/24084#24084
% I (H.-Martin Münch) submitted this package to CTAN
% (after diverse updates of the code and writing a documentation).
%
% This work may be distributed and/or modified under the
% conditions of the LaTeX Project Public License, either
% version 1.3c of this license or (at your option) any later
% version. This version of this license is in
%    http://www.latex-project.org/lppl/lppl-1-3c.txt
% and the latest version of this license is in
%    http://www.latex-project.org/lppl.txt
% and version 1.3c or later is part of all distributions of
% LaTeX version 2005/12/01 or later.
%
% This work has the LPPL maintenance status "maintained".
%
% The Current Maintainer of this work is H.-Martin Muench.
%
% This work consists of the main source file romanbar.dtx,
% the README, and the derived files
%    romanbar.sty, romanbar.pdf,
%    romanbar.ins, romanbar.drv,
%    romanbar-example.tex, romanbar-example.pdf.
%
% Distribution:
%    CTAN:macros/latex/contrib/romanbar/romanbar.dtx
%    CTAN:macros/latex/contrib/romanbar/romanbar.pdf
%    CTAN:install/macros/latex/contrib/romanbar.tds.zip
%
% Cf. http://ctan.org/pkg/romanbar and
% http://www.tex.ac.uk/tex-archive/help/Catalogue/entries/romanbar.html
% (It might take some time for the entries to be created.)
%
% Unpacking:
%    (a) If romanbar.ins is present:
%           tex romanbar.ins
%    (b) Without romanbar.ins:
%           tex romanbar.dtx
%    (c) If you insist on using LaTeX
%           latex \let\install=y% \iffalse meta-comment
%
% File: romanbar.dtx
% Version: 2012/01/01 v1.0f
%
% Copyright (C) 2011 by
%    H.-Martin M"unch <Martin dot Muench at Uni-Bonn dot de>
%
% The original code for the bars was written by Prof. Enrico Gregorio
% (http://profs.sci.univr.it/~gregorio/), and published at
% http://tex.stackexchange.com/questions/24065/roman-numerals-formatting/24084#24084
% I (H.-Martin Münch) submitted this package to CTAN
% (after diverse updates of the code and writing a documentation).
%
% This work may be distributed and/or modified under the
% conditions of the LaTeX Project Public License, either
% version 1.3c of this license or (at your option) any later
% version. This version of this license is in
%    http://www.latex-project.org/lppl/lppl-1-3c.txt
% and the latest version of this license is in
%    http://www.latex-project.org/lppl.txt
% and version 1.3c or later is part of all distributions of
% LaTeX version 2005/12/01 or later.
%
% This work has the LPPL maintenance status "maintained".
%
% The Current Maintainer of this work is H.-Martin Muench.
%
% This work consists of the main source file romanbar.dtx,
% the README, and the derived files
%    romanbar.sty, romanbar.pdf,
%    romanbar.ins, romanbar.drv,
%    romanbar-example.tex, romanbar-example.pdf.
%
% Distribution:
%    CTAN:macros/latex/contrib/romanbar/romanbar.dtx
%    CTAN:macros/latex/contrib/romanbar/romanbar.pdf
%    CTAN:install/macros/latex/contrib/romanbar.tds.zip
%
% Cf. http://ctan.org/pkg/romanbar and
% http://www.tex.ac.uk/tex-archive/help/Catalogue/entries/romanbar.html
% (It might take some time for the entries to be created.)
%
% Unpacking:
%    (a) If romanbar.ins is present:
%           tex romanbar.ins
%    (b) Without romanbar.ins:
%           tex romanbar.dtx
%    (c) If you insist on using LaTeX
%           latex \let\install=y% \iffalse meta-comment
%
% File: romanbar.dtx
% Version: 2012/01/01 v1.0f
%
% Copyright (C) 2011 by
%    H.-Martin M"unch <Martin dot Muench at Uni-Bonn dot de>
%
% The original code for the bars was written by Prof. Enrico Gregorio
% (http://profs.sci.univr.it/~gregorio/), and published at
% http://tex.stackexchange.com/questions/24065/roman-numerals-formatting/24084#24084
% I (H.-Martin Münch) submitted this package to CTAN
% (after diverse updates of the code and writing a documentation).
%
% This work may be distributed and/or modified under the
% conditions of the LaTeX Project Public License, either
% version 1.3c of this license or (at your option) any later
% version. This version of this license is in
%    http://www.latex-project.org/lppl/lppl-1-3c.txt
% and the latest version of this license is in
%    http://www.latex-project.org/lppl.txt
% and version 1.3c or later is part of all distributions of
% LaTeX version 2005/12/01 or later.
%
% This work has the LPPL maintenance status "maintained".
%
% The Current Maintainer of this work is H.-Martin Muench.
%
% This work consists of the main source file romanbar.dtx,
% the README, and the derived files
%    romanbar.sty, romanbar.pdf,
%    romanbar.ins, romanbar.drv,
%    romanbar-example.tex, romanbar-example.pdf.
%
% Distribution:
%    CTAN:macros/latex/contrib/romanbar/romanbar.dtx
%    CTAN:macros/latex/contrib/romanbar/romanbar.pdf
%    CTAN:install/macros/latex/contrib/romanbar.tds.zip
%
% Cf. http://ctan.org/pkg/romanbar and
% http://www.tex.ac.uk/tex-archive/help/Catalogue/entries/romanbar.html
% (It might take some time for the entries to be created.)
%
% Unpacking:
%    (a) If romanbar.ins is present:
%           tex romanbar.ins
%    (b) Without romanbar.ins:
%           tex romanbar.dtx
%    (c) If you insist on using LaTeX
%           latex \let\install=y\input{romanbar.dtx}
%        (quote the arguments according to the demands of your shell)
%
% Documentation:
%    (a) If romanbar.drv is present:
%           (pdf)latex romanbar.drv
%           makeindex -s gind.ist romanbar.idx
%           (pdf)latex romanbar.drv
%           makeindex -s gind.ist romanbar.idx
%           (pdf)latex romanbar.drv
%    (b) Without romanbar.drv:
%           (pdf)latex romanbar.dtx
%           makeindex -s gind.ist romanbar.idx
%           (pdf)latex romanbar.dtx
%           makeindex -s gind.ist romanbar.idx
%           (pdf)latex romanbar.dtx
%
%    The class ltxdoc loads the configuration file ltxdoc.cfg
%    if available. Here you can specify further options, e.g.
%    use DIN A4 as paper format:
%       \PassOptionsToClass{a4paper}{article}
%
% Installation:
%    TDS:tex/latex/romanbar/romanbar.sty
%    TDS:doc/latex/romanbar/romanbar.pdf
%    TDS:doc/latex/romanbar/romanbar-example.tex
%    TDS:doc/latex/romanbar/romanbar-example.pdf
%    TDS:source/latex/romanbar/romanbar.dtx
%
%<*ignore>
\begingroup
  \catcode123=1 %
  \catcode125=2 %
  \def\x{LaTeX2e}%
\expandafter\endgroup
\ifcase 0\ifx\install y1\fi\expandafter
         \ifx\csname processbatchFile\endcsname\relax\else1\fi
         \ifx\fmtname\x\else 1\fi\relax
\else\csname fi\endcsname
%</ignore>
%<*install>
\input docstrip.tex
\Msg{********************************************************************}
\Msg{* Installation                                                     *}
\Msg{* Package: romanbar 2012/01/01 v1.0f Roman numbers with bars (HMM) *}
\Msg{********************************************************************}

\keepsilent
\askforoverwritefalse

\let\MetaPrefix\relax
\preamble

This is a generated file.

Project: romanbar
Version: 2012/01/01 v1.0f

Copyright (C) 2011 by
    H.-Martin M"unch <Martin dot Muench at Uni-Bonn dot de>

The usual disclaimer applies:
If it doesn't work right that's your problem.
(Nevertheless, send an e-mail to the maintainer
 when you find an error in this package.)

This work may be distributed and/or modified under the
conditions of the LaTeX Project Public License, either
version 1.3c of this license or (at your option) any later
version. This version of this license is in
   http://www.latex-project.org/lppl/lppl-1-3c.txt
and the latest version of this license is in
   http://www.latex-project.org/lppl.txt
and version 1.3c or later is part of all distributions of
LaTeX version 2005/12/01 or later.

This work has the LPPL maintenance status "maintained".

The Current Maintainer of this work is H.-Martin Muench.

This work consists of the main source file romanbar.dtx,
the README, and the derived files
   romanbar.sty, romanbar.pdf,
   romanbar.ins, romanbar.drv,
   romanbar-example.tex, romanbar-example.pdf.

\endpreamble
\let\MetaPrefix\DoubleperCent

\generate{%
  \file{romanbar.ins}{\from{romanbar.dtx}{install}}%
  \file{romanbar.drv}{\from{romanbar.dtx}{driver}}%
  \usedir{tex/latex/romanbar}%
  \file{romanbar.sty}{\from{romanbar.dtx}{package}}%
  \usedir{doc/latex/romanbar}%
  \file{romanbar-example.tex}{\from{romanbar.dtx}{example}}%
}

\catcode32=13\relax% active space
\let =\space%
\Msg{************************************************************************}
\Msg{*}
\Msg{* To finish the installation you have to move the following}
\Msg{* file into a directory searched by TeX:}
\Msg{*}
\Msg{*     romanbar.sty}
\Msg{*}
\Msg{* To produce the documentation run the file `romanbar.drv'}
\Msg{* through (pdf)LaTeX, e.g.}
\Msg{*  pdflatex romanbar.drv}
\Msg{*  makeindex -s gind.ist romanbar.idx}
\Msg{*  pdflatex romanbar.drv}
\Msg{*  makeindex -s gind.ist romanbar.idx}
\Msg{*  pdflatex romanbar.drv}
\Msg{*}
\Msg{* At least three runs are necessary e.g. to get the}
\Msg{*  references right!}
\Msg{*}
\Msg{* Happy TeXing!}
\Msg{*}
\Msg{************************************************************************}

\endbatchfile
%</install>
%<*ignore>
\fi
%</ignore>
%
% \section{The documentation driver file}
%
% The next bit of code contains the documentation driver file for
% \TeX{}, i.\,e., the file that will produce the documentation you
% are currently reading. It will be extracted from this file by the
% \texttt{docstrip} programme. That is, run \LaTeX{} on \texttt{docstrip}
% and specify the \texttt{driver} option when \texttt{docstrip}
% asks for options.
%
%    \begin{macrocode}
%<*driver>
\NeedsTeXFormat{LaTeX2e}[2011/06/27]
\ProvidesFile{romanbar.drv}%
  [2012/01/01 v1.0f Roman numbers with bars (HMM)]
\documentclass{ltxdoc}[2007/11/11]% v2.0u
\usepackage{holtxdoc}[2011/02/04]%  v0.21
%% romanbar may work with earlier versions of LaTeX2e and those
%% class and package, but this was not tested.
%% Please consider updating your LaTeX, class, and package
%% to the most recent version (if they are not already the most
%% recent version).
\hypersetup{%
 pdfsubject={Writing Roman numbers with bars (HMM)},%
 pdfkeywords={LaTeX, romanbar, roman, Roman, bars, H.-Martin Muench},%
 pdfencoding=auto,%
 pdflang={en},%
 breaklinks=true,%
 linktoc=all,%
 pdfstartview=FitH,%
 pdfpagelayout=OneColumn,%
 bookmarksnumbered=true,%
 bookmarksopen=true,%
 bookmarksopenlevel=2,%
 pdfmenubar=true,%
 pdftoolbar=true,%
 pdfwindowui=true,%
 pdfnewwindow=true%
}
\CodelineIndex
\hyphenation{docu-ment ana-lysed}
\gdef\unit#1{\mathord{\thinspace\mathrm{#1}}}%
\makeatletter
\@ifundefined{eTeX}{\gdef\eTeX{$\m@th \varepsilon $-\TeX }}{% else \relax
}
\makeatother
\begin{document}
  \DocInput{romanbar.dtx}%
\end{document}
%</driver>
%    \end{macrocode}
%
% \fi
%
% \CheckSum{179}
%
% \CharacterTable
%  {Upper-case    \A\B\C\D\E\F\G\H\I\J\K\L\M\N\O\P\Q\R\S\T\U\V\W\X\Y\Z
%   Lower-case    \a\b\c\d\e\f\g\h\i\j\k\l\m\n\o\p\q\r\s\t\u\v\w\x\y\z
%   Digits        \0\1\2\3\4\5\6\7\8\9
%   Exclamation   \!     Double quote  \"     Hash (number) \#
%   Dollar        \$     Percent       \%     Ampersand     \&
%   Acute accent  \'     Left paren    \(     Right paren   \)
%   Asterisk      \*     Plus          \+     Comma         \,
%   Minus         \-     Point         \.     Solidus       \/
%   Colon         \:     Semicolon     \;     Less than     \<
%   Equals        \=     Greater than  \>     Question mark \?
%   Commercial at \@     Left bracket  \[     Backslash     \\
%   Right bracket \]     Circumflex    \^     Underscore    \_
%   Grave accent  \`     Left brace    \{     Vertical bar  \|
%   Right brace   \}     Tilde         \~}
%
% \GetFileInfo{romanbar.drv}
%
% \begingroup
%   \def\x{\#,\$,\^,\_,\~,\ ,\&,\{,\},\%}%
%   \makeatletter
%   \@onelevel@sanitize\x
% \expandafter\endgroup
% \expandafter\DoNotIndex\expandafter{\x}
% \expandafter\DoNotIndex\expandafter{\string\ }
% \begingroup
%   \makeatletter
%     \lccode`9=32\relax
%     \lowercase{%^^A
%       \edef\x{\noexpand\DoNotIndex{\@backslashchar9}}%^^A
%     }%^^A
%   \expandafter\endgroup\x
%
% \DoNotIndex{\\,\,}
% \DoNotIndex{\documentclass,\usepackage,\ProvidesPackage,\begin,\end}
% \DoNotIndex{\MessageBreak}
% \DoNotIndex{\NeedsTeXFormat,\DoNotIndex,\verb}
% \DoNotIndex{\def,\edef,\gdef,\global}
% \DoNotIndex{\ifx,\listfiles,\mathord,\mathrm}
% \DoNotIndex{\bigskip,\space,\thinspace,\ldots}
% \DoNotIndex{\indent,\noindent,\newline,\linebreak,\pagebreak,\newpage}
% \DoNotIndex{\textbf,\textit,\textsf,\texttt,\textquotedblleft,\textquotedblright}
% \DoNotIndex{\plainTeX,\TeX,\LaTeX,\pdfLaTeX}
% \DoNotIndex{\item}
%
% \title{The \xpackage{romanbar} package}
% \date{2012/01/01 v1.0f}
% \author{H.-Martin M\"{u}nch\\\xemail{Martin.Muench at Uni-Bonn.de}}
%
% \maketitle
%
% \begin{abstract}
%  This \LaTeX{} package allows to write Roman numbers
%  (or any other text) with bars. (Additionally, commands for converting Arabic numbers
%  into Roman ones are provided and an |\ifnumeric| test function.)
% \end{abstract}
%
% \bigskip
%
% \noindent Disclaimer for web links: The author is not responsible for any contents
% referred to in this work unless he has full knowledge of illegal contents.
% If any damage occurs by the use of information presented there, only the
% author of the respective pages might be liable, not the one who has referred
% to these pages.
%
% \bigskip
%
% \noindent {\color{green} Save per page about $200\unit{ml}$ water,
% $2\unit{g}$ CO$_{2}$ and $2\unit{g}$ wood:\\
% Therefore please print only if this is really necessary.}
%
% \pagebreak
%
% \tableofcontents
%
% \pagebreak
%
% \section{Introduction}
%
% \indent This \LaTeX{} package allows to write Roman numbers
% (or any other text) with bars. (Additionally, command for converting Arabic numbers
% into Roman ones are provided and an |\ifnumeric| test function.)
%
% \bigskip
%
% \section{Usage}
%
% \indent Just load the package placing
% \begin{quote}
%   |\usepackage{romanbar}|
% \end{quote}
% \noindent in the preamble of your \LaTeXe{} source file.\\
% |\Romanbar{...}| then produces a Roman number with bars (please see the example file).
% |\ifnumeric{test}{true}{false}| tests for |test| being numeric.
% |\romannum{...}| turns an Arabic number into a lowercase Roman one, and
% |\Romannum{...}| turns an Arabic number into an uppercase Roman one.
%
% \subsection*{Options}
% \DescribeMacro{options}
% \indent The \xpackage{romanbar} package takes no options.
%
% \section{Alternatives\label{sec:Alternatives}}
%
% \begin{description}
% \item[-] the original code written by Prof. Enrico Gregorio\\
%   (\url{http://profs.sci.univr.it/~gregorio/}),\\
%   \url{http://tex.stackexchange.com/questions/24065/roman-numerals-formatting/24084#24084}
%\begin{verbatim}
%\def\barroman#1{\sbox0{#1}\dimen0=\dimexpr\wd0+1pt\relax
%  \makebox[\dimen0]{\rlap{\vrule width\dimen0 height 0.06ex depth 0.06ex}%
%    \rlap{\vrule width\dimen0 height\dimexpr\ht0+0.03ex\relax
%            depth\dimexpr-\ht0+0.09ex\relax}%
%    \kern.5pt#1\kern.5pt}}
%
%\barroman{I} \barroman{XI}
%\end{verbatim}
%
% \item[-]  Some fonts provide single characters for Roman numerals,
%   cf.~e.\,g.
% \url{http://tex.stackexchange.com/questions/38695/using-unicode-roman-numerals-in-xetex}.
% \end{description}
%
% \noindent (You programmed or found another alternative,
%  which is available at \CTAN{}?\\
%  OK, send an e-mail to me with the name, location at \CTAN{},
%  and a short notice, and I will probably include it in
%  the list above.)
%
% \newpage
%
% \section{Example}
%
%    \begin{macrocode}
%<*example>
\documentclass[british]{article}[2007/10/19]% v1.4h
%%%%%%%%%%%%%%%%%%%%%%%%%%%%%%%%%%%%%%%%%%%%%%%%%%%%%%%%%%%%%%%%%%%%%
\usepackage[extension=pdf,%
 plainpages=false,%
 pdfpagelabels=true,%
 hyperindex=false,%
 pdflang={en},%
 pdftitle={romanbar package example},%
 pdfauthor={H.-Martin Muench},%
 pdfsubject={Example for the romanbar package},%
 pdfkeywords={LaTeX, romanbar, roman, Roman, bars, H.-Martin Muench},%
 pdfview={XYZ null null 1},%
 pdfstartview={XYZ null null 1},%
 pdfpagelayout=SinglePage]{hyperref}[2011/12/04]% v6.82m
\usepackage{romanbar}[2012/01/01]% v1.0f
\gdef\unit#1{\mathord{\thinspace\mathrm{#1}}}%
\listfiles
\begin{document}
\pagenumbering{arabic}
\section*{Example for romanbar}

This example demonstrates the use of package\newline
\textsf{romanbar}, v1.0f as of 2012/01/01 (HMM).\newline
There are no options to be used.\newline

\noindent For more details please see the documentation!\newline

\noindent Save per page about $200\unit{ml}$ water,
$2\unit{g}$ CO$_{2}$ and $2\unit{g}$ wood:\newline
Therefore please print only if this is really necessary.\newline

\noindent This package provides the command \verb|\Romanbar|
to print bars below and over the following:

\begin{description}
\item[-] Roman numbers: \verb|\Romanbar{MMXII}| prints \Romanbar{MMXII}

\item[-] Arabic numbers turned into upper-case Roman numbers:\newline
           \verb|\Romanbar{2012}| prints \Romanbar{2012}

\item[-] negative Arabic numbers turned into upper-case Roman numbers
           \newline
           (although historically there were no negative Roman numbers):
           \newline
           \verb|\Romanbar{-12}| prints \Romanbar{-12}

\item[-] zero Arabic number ($0$; although historically
           there was no Roman zero):\newline
           \verb|\Romanbar{0}| prints \Romanbar{0}

\item[-] some arbitrary text:
           \verb|\Romanbar{Caesar}| prints \Romanbar{Caesar}\newline
           (with descenders: \Romanbar{AgjpqyW})


\newcounter{example}
\setcounter{example}{21}

\item[-] some counter's value:
           \verb|\Romanbar{\theexample}| prints \Romanbar{\theexample}
           \newline
           (where the value of \texttt{example} is \theexample)

\item[-] Arabic numbers, without turning them into upper-case Roman
           numbers:\newline
           \verb|\Romanbar{\relax 2012}| prints \Romanbar{\relax  2012}
\end{description}

Special care was taken for "L" (50), e.\,g. in 555/DLV: \Romanbar{555}.\\


\verb|\romannum{...}| converts an Arabic number into a lower-case Roman one,
e.\,g. \verb|\romannum{2012}| prints \romannum{2012}, and
\verb|\Romannum{...}| converts an Arabic number into an upper-case Roman one,
e.\,g. \verb|\Romannum{2012}| prints \Romannum{2012}.

\end{document}
%</example>
%    \end{macrocode}
%
% \StopEventually{}
%
% \newpage
%
% \section{The implementation}
%
% We start off by checking that we are loading into \LaTeXe{} and
% announcing the name and version of this package.
%
%    \begin{macrocode}
%<*package>
%    \end{macrocode}
%
%    \begin{macrocode}
\NeedsTeXFormat{LaTeX2e}[2011/06/27]
\ProvidesPackage{romanbar}[2012/01/01 v1.0f
              Roman numbers with bars (HMM)]

%    \end{macrocode}
%
% A short description of the \xpackage{romanbar} package:
%
%    \begin{macrocode}
%% Allows to write Roman numbers (or any other text) with bars
%% and to convert Arabic numbers into Roman ones.
%    \end{macrocode}
%
% A last information for the user:
%
%    \begin{macrocode}
%% romanbar may work with earlier versions of LaTeX,
%% but this was not tested. Please consider updating
%% your LaTeX to the most recent version
%% (if it is not already the most recent version).

%    \end{macrocode}
%
% \indent See subsection~\ref{ss:Downloads} about how to get it.\\
%
% There are no options to be processed.
%
% A command to gobble a possible minus sign is needed and
% |\gobbleminus| from \url{http://www.tex.ac.uk/cgi-bin/texfaq2html?label=isitanum}
% is used.
%
%    \begin{macrocode}
\def\gobbleminus#1{\ifx-#1\else#1\fi}
%% from http://www.tex.ac.uk/cgi-bin/texfaq2html?label=isitanum

%    \end{macrocode}
%
% We want to test whether the argument passed to |\Romanbar| is numeric
% and therefore define |\ifnumeric{test}{true}{false}|,
% where |test| is to be analysed, |true| is the code to be executed,
% if |test| is numeric, and |false| is the code to be executed,
% if |test| is not numeric. This is done similar to\\
% \url{http://tex.stackexchange.com/a/17119}.
%
%    \begin{macrocode}
\newcommand{\ifnumeric}[3]{%
%% similar to http://tex.stackexchange.com/a/17119
\sbox\z@{\@tempcnta=0\gobbleminus#1\relax}%
\ifdim\wd0>\z@\relax#3% is not numeric
\else#2% is numeric
\fi%
}

%    \end{macrocode}
%
% We define |\r@iseL| to raise any L (50), otherwise a good deal
% of the horizontal part of the letter would be swallowed by the lower bar
% under the \textquotedblleft{}number\textquotedblright{}.
%
%    \begin{macrocode}
\def\r@iseL#1{\ifx @#1% then terminate
\else%
\if L#1\raisebox{0.05ex}{L}\else #1\fi%
\expandafter\r@iseL%
\fi%
}

%    \end{macrocode}
%
% \pagebreak
%
% We define the |\Romanbar| command.
%
%    \begin{macrocode}
\DeclareRobustCommand{\Romanbar}[1]{%
\ifnumeric{#1}{% is numeric
%    \end{macrocode}
%
% If its parameter is numeric, the temporary counter |\@tempcnta| is set to the value.
%
%    \begin{macrocode}
\@tempcnta=#1\relax%
%    \end{macrocode}
%
% Normally there are no non-positive Roman numbers, therefore we do some trick
% similar to |\XRoman| from the \url{http://ctan.org/pkg/pageslts} package.
%
%    \begin{macrocode}
%% similar to \XRoman from the http://ctan.org/pkg/pageslts package
\ifnum\@tempcnta<1%
  \ifnum #1>0%
    \relax \Roman@bar{\Roman{#1}}%
  \else%
    \ifnum #1<0%
     -\Roman@bar{\expandafter\@slowromancap\romannumeral\number-#1@%
     }%
    \else%
      \Roman@bar{0}%
    \fi%
  \fi%
\else\Roman@bar{\@Roman\@tempcnta}%
\fi%
}{% is not numeric
\Roman@bar{#1}%
}%
}

%    \end{macrocode}
%
% Whether it is numeric or not, in the end |\Roman@bar| is applied to it.
%
%    \begin{macrocode}
\DeclareRobustCommand{\Roman@bar}[1]{% #1 is in Roman, i.e. MMXII
%% (or any other text, "Caesar" would work, too).
%% similar to code of  by Prof. Enrico Gregorio (egreg) at
%% http://tex.stackexchange.com/questions/24065/roman-numerals-formatting/24084#24084
\@bsphack%
\edef\romanbartmp{#1}%
%% height:
\sbox0{\textsf{\romanbartmp}}%
%% top line:
%% + top of top line:
\@tempdima=\ht0%
\advance\@tempdima+0.05ex%0.03
\dimen1=\@tempdima%
%% + bottom of top line:
\@tempdima=-\ht0%
\advance\@tempdima+0.05ex%0.07
\dimen2=\@tempdima%
%% width:
\sbox0{\textsf{\expandafter\r@iseL\romanbartmp @}}%
\@tempdima=\wd0%
\advance\@tempdima+1pt%
\dimen0=\@tempdima%
\@esphack%
\makebox[\dimen0]{%
\rlap{\vrule width\dimen0 height\dimen1 depth\dimen2}% top    line
\rlap{\vrule width\dimen0 height 0.06ex depth 0.03ex}% bottom line
\kern0.5pt\textsf{\expandafter\r@iseL\romanbartmp @}\kern0.5pt}%
}

%    \end{macrocode}
%
% While it is not needed for the bars, it is sometimes asked for a command
% to convert an Arabic number into a Roman one. |\roman{...}| can only be applied
% to a counter, i.\,e. |\roman{12}| does not work, but
%\begin{verbatim}
%\newcounter{examplecounter}
%\setcounter{examplecounter}{12}
%\roman{examplecounter}
%\end{verbatim}
% would be necessary. Further non-negative values would not work.
% |\romannum| accepts any number as argument. (If the argument is no number,
% an error message is given and the argument printed (executed)
%  without applying anything to it.)
%
%    \begin{macrocode}
\DeclareRobustCommand{\romannum}[1]{%
\ifnumeric{#1}{% is numeric
\@tempcnta=#1\relax%
%% similar to \xroman from the http://ctan.org/pkg/pageslts package
\ifnum\@tempcnta<1%
  \ifnum #1>0%
    \relax \roman{#1}%
  \else%
    \ifnum #1<0%
     -\romannumeral\number-#1@%
    \else%
      0%
    \fi%
  \fi%
\else\@roman\@tempcnta%
\fi%
}{% is not numeric
\PackageError{romanbar}{%
Argument of \string\romannum\space is not a number}{%
The command \string\romannum\space converts an Arabic number into a%
 lower-case Roman one,\MessageBreak%
but the used argument of \string\romannum\space is not an Arabic number%
 but\MessageBreak%
`#1`,\MessageBreak%
which will now be printed unchanged.}%
#1%
}%
}

%    \end{macrocode}
%
% The same for upper-case Roman numbers, |\Romannum| instead of |\Roman|:
%
%    \begin{macrocode}
\DeclareRobustCommand{\Romannum}[1]{%
\ifnumeric{#1}{% is numeric
\@tempcnta=#1\relax%
%% similar to \XRoman from the http://ctan.org/pkg/pageslts package
\ifnum\@tempcnta<1%
  \ifnum #1>0%
    \relax \Roman{#1}%
  \else%
    \ifnum #1<0%
     -\expandafter\@slowromancap\romannumeral\number-#1@%
    \else%
      0%
    \fi%
  \fi%
\else\@Roman\@tempcnta%
\fi%
}{% is not numeric
\PackageError{romanbar}{%
Argument of \string\Romannum\space is not a number}{%
The command \string\Romannum\space converts an Arabic number into an%
 upper-case Roman one,\MessageBreak%
but the used argument of \string\Romannum\space is not an Arabic number%
 but\MessageBreak%
`#1`.}%
#1%
}%
}

%    \end{macrocode}
%
%    \begin{macrocode}
%</package>
%    \end{macrocode}
%
% \newpage
%
% \section{Installation}
%
% \subsection{Downloads\label{ss:Downloads}}
%
% Everything should be available on \CTAN{}, \url{http://www.ctan.org/tex-archive/},
% but may need additional packages themselves.\\
%
% \DescribeMacro{romanbar.dtx}
% For unpacking the |romanbar.dtx| file and constructing the documentation
% it is required:
% \begin{description}
% \item[-] \TeX{} Format \LaTeXe{}, \url{http://www.CTAN.org/}
%
% \item[-] document class \xpackage{ltxdoc}, 2007/11/11, v2.0u,
%           \url{http://ctan.org/pkg/ltxdoc}
%
% \item[-] package \xpackage{holtxdoc}, 2011/02/04, v0.21,
%           \url{http://ctan.org/pkg/holtxdoc}
%
% \item[-] package \xpackage{hypdoc}, 2010/03/26, v1.9,
%           \url{http://ctan.org/pkg/hypdoc}
% \end{description}
%
% \DescribeMacro{romanbar.sty}
% The \texttt{romanbar.sty} for \LaTeXe{} (i.\,e.~each document using
% the \xpackage{romanbar} package) requires:
% \begin{description}
% \item[-] \TeX Format \LaTeXe{}, \url{http://www.CTAN.org/}
% \end{description}
%
% \DescribeMacro{romanbar-example.tex}
% The \texttt{romanbar-example.tex} requires the same files as all
% documents using the \xpackage{romanbar} package and additionally:
% \begin{description}
% \item[-] class \xpackage{article}, 2007/10/19, v1.4h, from \xpackage{classes.dtx}:\\
%           \CTAN{macros/latex/base/classes.dtx}
%
% \item[-] package \xpackage{romanbar}, 2012/01/01, v1.0f,\\
%           %%\url{http://ctan.org/pkg/romanbar}\\
%           \CTAN{macros/latex/contrib/romanbar/}
%   (Well, it is the example file for this package, and because you are reading the
%    documentation for the \xpackage{romanbar} package, it can be assumed that you already
%    have some version of it -- is it the current one?)
% \end{description}
%
% \DescribeMacro{Alternative}
% As possible alternative package in section \ref{sec:Alternatives} there is listed
% \begin{description}
% \item[-] none.
% \end{description}
%
% \DescribeMacro{Oberdiek}
% \DescribeMacro{holtxdoc}
% All packages of \textsc{Heiko Oberdiek's} bundle `oberdiek'
% (especially \xpackage{holtxdoc}) are also available in a TDS compliant ZIP archive:\\
% \CTAN{install/macros/latex/contrib/oberdiek.tds.zip}.\\
% It is probably best to download and use this, because the packages in there
% are quite probably both recent and compatible among themselves.\\
%
% \DescribeMacro{hyperref}
% \noindent \xpackage{hyperref} is not included in that bundle and needs to be
% downloaded separately,\\
% \url{http://mirror.ctan.org/install/macros/latex/contrib/hyperref.tds.zip}.\\
%
% \DescribeMacro{M\"{u}nch}
% A hyperlinked list of my (other) packages can be found at
% \url{http://www.Uni-Bonn.de/~uzs5pv/LaTeX.html}.\\
%
% \pagebreak
%
% \subsection{Package, unpacking TDS}
% \paragraph{Package.} This package should become available on \CTAN{} soon:
% \begin{description}
% \item[\CTAN{macros/latex/contrib/romanbar/romanbar.dtx}]\hspace*{0.1cm} \\
%       The source file.
% \item[\CTAN{macros/latex/contrib/romanbar/romanbar.pdf}]\hspace*{0.1cm} \\
%       The documentation.
% \item[\CTAN{macros/latex/contrib/romanbar/romanbar-example.pdf}]\hspace*{0.1cm} \\
%       The compiled example file, as it should look like.
% \item[\CTAN{macros/latex/contrib/romanbar/README}]\hspace*{0.1cm} \\
%       The README file.
% \end{description}
% There is also a romanbar.tds.zip available:
% \begin{description}
% \item[\CTAN{install/macros/latex/contrib/romanbar.tds.zip}]\hspace*{0.1cm} \\
%       Everything in \xfile{TDS} compliant, compiled format.
% \end{description}
% which additionally contains\\
% \begin{tabular}{ll}
% romanbar.ins & The installation file.\\
% romanbar.drv & The driver to generate the documentation.\\
% romanbar.sty & The \xext{sty}le file.\\
% romanbar-example.tex & The example file.
% \end{tabular}
%
% \bigskip
%
% \noindent For required other packages, please see the preceding subsection.
%
% \paragraph{Unpacking.} The \xfile{.dtx} file is a self-extracting
% \docstrip{} archive. The files are extracted by running the
% \xext{.dtx} through \plainTeX{}:
% \begin{quote}
%   \verb|tex romanbar.dtx|
% \end{quote}
%
% About generating the documentation see paragraph~\ref{GenDoc} below.\\
%
% \paragraph{TDS.} Now the different files must be moved into
% the different directories in your installation TDS tree
% (also known as \xfile{texmf} tree):
% \begin{quote}
% \def\t{^^A
% \begin{tabular}{@{}>{\ttfamily}l@{ $\rightarrow$ }>{\ttfamily}l@{}}
%   romanbar.sty & tex/latex/romanbar/romanbar.sty\\
%   romanbar.pdf & doc/latex/romanbar/romanbar.pdf\\
%   romanbar-example.tex & doc/latex/romanbar/romanbar-example.tex\\
%   romanbar-example.pdf & doc/latex/romanbar/romanbar-example.pdf\\
%   romanbar.dtx & source/latex/romanbar/romanbar.dtx\\
% \end{tabular}^^A
% }^^A
% \sbox0{\t}^^A
% \ifdim\wd0>\linewidth
%   \begingroup
%     \advance\linewidth by\leftmargin
%     \advance\linewidth by\rightmargin
%   \edef\x{\endgroup
%     \def\noexpand\lw{\the\linewidth}^^A
%   }\x
%   \def\lwbox{^^A
%     \leavevmode
%     \hbox to \linewidth{^^A
%       \kern-\leftmargin\relax
%       \hss
%       \usebox0
%       \hss
%       \kern-\rightmargin\relax
%     }^^A
%   }^^A
%   \ifdim\wd0>\lw
%     \sbox0{\small\t}^^A
%     \ifdim\wd0>\linewidth
%       \ifdim\wd0>\lw
%         \sbox0{\footnotesize\t}^^A
%         \ifdim\wd0>\linewidth
%           \ifdim\wd0>\lw
%             \sbox0{\scriptsize\t}^^A
%             \ifdim\wd0>\linewidth
%               \ifdim\wd0>\lw
%                 \sbox0{\tiny\t}^^A
%                 \ifdim\wd0>\linewidth
%                   \lwbox
%                 \else
%                   \usebox0
%                 \fi
%               \else
%                 \lwbox
%               \fi
%             \else
%               \usebox0
%             \fi
%           \else
%             \lwbox
%           \fi
%         \else
%           \usebox0
%         \fi
%       \else
%         \lwbox
%       \fi
%     \else
%       \usebox0
%     \fi
%   \else
%     \lwbox
%   \fi
% \else
%   \usebox0
% \fi
% \end{quote}
% If you have a \xfile{docstrip.cfg} that configures and enables \docstrip{}'s
% \xfile{TDS} installing feature, then some files can already be in the right
% place, see the documentation of \docstrip{}.
%
% \subsection{Refresh file name databases}
%
% If your \TeX{}~distribution (\teTeX{}, \mikTeX{},\dots{}) relies on file name
% databases, you must refresh these. For example, \teTeX{} users run
% \verb|texhash| or \verb|mktexlsr|.
%
% \subsection{Some details for the interested}
%
% \paragraph{Unpacking with \LaTeX{}.}
% The \xfile{.dtx} chooses its action depending on the format:
% \begin{description}
% \item[\plainTeX:] Run \docstrip{} and extract the files.
% \item[\LaTeX:] Generate the documentation.
% \end{description}
% If you insist on using \LaTeX{} for \docstrip{} (really,
% \docstrip{} does not need \LaTeX{}), then inform the autodetect routine
% about your intention:
% \begin{quote}
%   \verb|latex \let\install=y\input{romanbar.dtx}|
% \end{quote}
% Do not forget to quote the argument according to the demands
% of your shell.
%
% \paragraph{Generating the documentation.\label{GenDoc}}
% You can use both the \xfile{.dtx} or the \xfile{.drv} to generate
% the documentation. The process can be configured by a
% configuration file \xfile{ltxdoc.cfg}. For instance, put the following
% line into this file, if you want to have A4 as paper format:
% \begin{quote}
%   \verb|\PassOptionsToClass{a4paper}{article}|
% \end{quote}
%
% \noindent An example follows how to generate the
% documentation with \pdfLaTeX{}:
%
% \begin{quote}
%\begin{verbatim}
%pdflatex romanbar.dtx
%makeindex -s gind.ist romanbar.idx
%pdflatex romanbar.dtx
%makeindex -s gind.ist romanbar.idx
%pdflatex romanbar.dtx
%\end{verbatim}
% \end{quote}
%
% \subsection{Compiling the example}
%
% The example file, \textsf{romanbar-example.tex}, can be compiled via\\
% |(pdf)(la)tex romanbar-example.tex|.
%
% \section{Acknowledgements}
%
% I would like to thank \textsc{Heiko Oberdiek} for providing
% the \xpackage{hyperref} as well as a~lot~(!) of other useful packages
% (from which I also got everything I know about creating a file in \xext{dtx}
% format, ok, say it: copying),
% Prof.~\textsc{Enrico Gregorio} (egreg) for his answer
% \url{http://tex.stackexchange.com/questions/24065/roman-numerals-formatting/24084#24084},
% the \Newsgroup{comp.text.tex} and \Newsgroup{de.comp.text.tex}
% newsgroups as well as \url{http://tex.stackexchange.com}
% for their help in all things \TeX{}.
%
% \pagebreak
%
% \phantomsection
% \begin{History}\label{History}
%   \begin{Version}{2011/07/25 v1.0a}
%     \item First code by Prof.~\textsc{Enrico Gregorio} (egreg) at
%             \url{http://tex.stackexchange.com/questions/24065/roman-numerals-formatting/24084#24084}
%   \end{Version}
%   \begin{Version}{2011/07/26 v1.0b}
%     \item Packed into a \xext{sty} file.
%     \item Removed the requirement of \eTeX{} for the package.
%   \end{Version}
%   \begin{Version}{2011/08/25 v1.0c}
%     \item Renamed \xpackage{romanbar} package.
%     \item Minor details.
%   \end{Version}
%   \begin{Version}{2011/09/16 v1.0d}
%     \item Made commands robust.
%     \item Minor details.
%   \end{Version}
%   \begin{Version}{2011/12/16 v1.0e}
%     \item Additionally accepts now Arabic numbers and converts them to Roman ones with bars.
%     \item Also non-positive numbers are now accepted.
%   \end{Version}
%   \begin{Version}{2012/01/01 v1.0f}
%     \item Added |\romannum| and |\Romannum| to convert Arabic to Roman numbers (without bars).
%     \item |\r@iseL|.
%     \item Created a \xfile{dtx}, example, and README.
%     \item Upload to \CTAN{}.
%   \end{Version}
% \end{History}
%
% \bigskip
%
% When you find a mistake or have a suggestion for an improvement of this package,
% please send an e-mail to the maintainer, thanks! (Please see BUG REPORTS in the README.)
%
% \pagebreak
%
% \PrintIndex
%
% \Finale
\endinput
%        (quote the arguments according to the demands of your shell)
%
% Documentation:
%    (a) If romanbar.drv is present:
%           (pdf)latex romanbar.drv
%           makeindex -s gind.ist romanbar.idx
%           (pdf)latex romanbar.drv
%           makeindex -s gind.ist romanbar.idx
%           (pdf)latex romanbar.drv
%    (b) Without romanbar.drv:
%           (pdf)latex romanbar.dtx
%           makeindex -s gind.ist romanbar.idx
%           (pdf)latex romanbar.dtx
%           makeindex -s gind.ist romanbar.idx
%           (pdf)latex romanbar.dtx
%
%    The class ltxdoc loads the configuration file ltxdoc.cfg
%    if available. Here you can specify further options, e.g.
%    use DIN A4 as paper format:
%       \PassOptionsToClass{a4paper}{article}
%
% Installation:
%    TDS:tex/latex/romanbar/romanbar.sty
%    TDS:doc/latex/romanbar/romanbar.pdf
%    TDS:doc/latex/romanbar/romanbar-example.tex
%    TDS:doc/latex/romanbar/romanbar-example.pdf
%    TDS:source/latex/romanbar/romanbar.dtx
%
%<*ignore>
\begingroup
  \catcode123=1 %
  \catcode125=2 %
  \def\x{LaTeX2e}%
\expandafter\endgroup
\ifcase 0\ifx\install y1\fi\expandafter
         \ifx\csname processbatchFile\endcsname\relax\else1\fi
         \ifx\fmtname\x\else 1\fi\relax
\else\csname fi\endcsname
%</ignore>
%<*install>
\input docstrip.tex
\Msg{********************************************************************}
\Msg{* Installation                                                     *}
\Msg{* Package: romanbar 2012/01/01 v1.0f Roman numbers with bars (HMM) *}
\Msg{********************************************************************}

\keepsilent
\askforoverwritefalse

\let\MetaPrefix\relax
\preamble

This is a generated file.

Project: romanbar
Version: 2012/01/01 v1.0f

Copyright (C) 2011 by
    H.-Martin M"unch <Martin dot Muench at Uni-Bonn dot de>

The usual disclaimer applies:
If it doesn't work right that's your problem.
(Nevertheless, send an e-mail to the maintainer
 when you find an error in this package.)

This work may be distributed and/or modified under the
conditions of the LaTeX Project Public License, either
version 1.3c of this license or (at your option) any later
version. This version of this license is in
   http://www.latex-project.org/lppl/lppl-1-3c.txt
and the latest version of this license is in
   http://www.latex-project.org/lppl.txt
and version 1.3c or later is part of all distributions of
LaTeX version 2005/12/01 or later.

This work has the LPPL maintenance status "maintained".

The Current Maintainer of this work is H.-Martin Muench.

This work consists of the main source file romanbar.dtx,
the README, and the derived files
   romanbar.sty, romanbar.pdf,
   romanbar.ins, romanbar.drv,
   romanbar-example.tex, romanbar-example.pdf.

\endpreamble
\let\MetaPrefix\DoubleperCent

\generate{%
  \file{romanbar.ins}{\from{romanbar.dtx}{install}}%
  \file{romanbar.drv}{\from{romanbar.dtx}{driver}}%
  \usedir{tex/latex/romanbar}%
  \file{romanbar.sty}{\from{romanbar.dtx}{package}}%
  \usedir{doc/latex/romanbar}%
  \file{romanbar-example.tex}{\from{romanbar.dtx}{example}}%
}

\catcode32=13\relax% active space
\let =\space%
\Msg{************************************************************************}
\Msg{*}
\Msg{* To finish the installation you have to move the following}
\Msg{* file into a directory searched by TeX:}
\Msg{*}
\Msg{*     romanbar.sty}
\Msg{*}
\Msg{* To produce the documentation run the file `romanbar.drv'}
\Msg{* through (pdf)LaTeX, e.g.}
\Msg{*  pdflatex romanbar.drv}
\Msg{*  makeindex -s gind.ist romanbar.idx}
\Msg{*  pdflatex romanbar.drv}
\Msg{*  makeindex -s gind.ist romanbar.idx}
\Msg{*  pdflatex romanbar.drv}
\Msg{*}
\Msg{* At least three runs are necessary e.g. to get the}
\Msg{*  references right!}
\Msg{*}
\Msg{* Happy TeXing!}
\Msg{*}
\Msg{************************************************************************}

\endbatchfile
%</install>
%<*ignore>
\fi
%</ignore>
%
% \section{The documentation driver file}
%
% The next bit of code contains the documentation driver file for
% \TeX{}, i.\,e., the file that will produce the documentation you
% are currently reading. It will be extracted from this file by the
% \texttt{docstrip} programme. That is, run \LaTeX{} on \texttt{docstrip}
% and specify the \texttt{driver} option when \texttt{docstrip}
% asks for options.
%
%    \begin{macrocode}
%<*driver>
\NeedsTeXFormat{LaTeX2e}[2011/06/27]
\ProvidesFile{romanbar.drv}%
  [2012/01/01 v1.0f Roman numbers with bars (HMM)]
\documentclass{ltxdoc}[2007/11/11]% v2.0u
\usepackage{holtxdoc}[2011/02/04]%  v0.21
%% romanbar may work with earlier versions of LaTeX2e and those
%% class and package, but this was not tested.
%% Please consider updating your LaTeX, class, and package
%% to the most recent version (if they are not already the most
%% recent version).
\hypersetup{%
 pdfsubject={Writing Roman numbers with bars (HMM)},%
 pdfkeywords={LaTeX, romanbar, roman, Roman, bars, H.-Martin Muench},%
 pdfencoding=auto,%
 pdflang={en},%
 breaklinks=true,%
 linktoc=all,%
 pdfstartview=FitH,%
 pdfpagelayout=OneColumn,%
 bookmarksnumbered=true,%
 bookmarksopen=true,%
 bookmarksopenlevel=2,%
 pdfmenubar=true,%
 pdftoolbar=true,%
 pdfwindowui=true,%
 pdfnewwindow=true%
}
\CodelineIndex
\hyphenation{docu-ment ana-lysed}
\gdef\unit#1{\mathord{\thinspace\mathrm{#1}}}%
\makeatletter
\@ifundefined{eTeX}{\gdef\eTeX{$\m@th \varepsilon $-\TeX }}{% else \relax
}
\makeatother
\begin{document}
  \DocInput{romanbar.dtx}%
\end{document}
%</driver>
%    \end{macrocode}
%
% \fi
%
% \CheckSum{179}
%
% \CharacterTable
%  {Upper-case    \A\B\C\D\E\F\G\H\I\J\K\L\M\N\O\P\Q\R\S\T\U\V\W\X\Y\Z
%   Lower-case    \a\b\c\d\e\f\g\h\i\j\k\l\m\n\o\p\q\r\s\t\u\v\w\x\y\z
%   Digits        \0\1\2\3\4\5\6\7\8\9
%   Exclamation   \!     Double quote  \"     Hash (number) \#
%   Dollar        \$     Percent       \%     Ampersand     \&
%   Acute accent  \'     Left paren    \(     Right paren   \)
%   Asterisk      \*     Plus          \+     Comma         \,
%   Minus         \-     Point         \.     Solidus       \/
%   Colon         \:     Semicolon     \;     Less than     \<
%   Equals        \=     Greater than  \>     Question mark \?
%   Commercial at \@     Left bracket  \[     Backslash     \\
%   Right bracket \]     Circumflex    \^     Underscore    \_
%   Grave accent  \`     Left brace    \{     Vertical bar  \|
%   Right brace   \}     Tilde         \~}
%
% \GetFileInfo{romanbar.drv}
%
% \begingroup
%   \def\x{\#,\$,\^,\_,\~,\ ,\&,\{,\},\%}%
%   \makeatletter
%   \@onelevel@sanitize\x
% \expandafter\endgroup
% \expandafter\DoNotIndex\expandafter{\x}
% \expandafter\DoNotIndex\expandafter{\string\ }
% \begingroup
%   \makeatletter
%     \lccode`9=32\relax
%     \lowercase{%^^A
%       \edef\x{\noexpand\DoNotIndex{\@backslashchar9}}%^^A
%     }%^^A
%   \expandafter\endgroup\x
%
% \DoNotIndex{\\,\,}
% \DoNotIndex{\documentclass,\usepackage,\ProvidesPackage,\begin,\end}
% \DoNotIndex{\MessageBreak}
% \DoNotIndex{\NeedsTeXFormat,\DoNotIndex,\verb}
% \DoNotIndex{\def,\edef,\gdef,\global}
% \DoNotIndex{\ifx,\listfiles,\mathord,\mathrm}
% \DoNotIndex{\bigskip,\space,\thinspace,\ldots}
% \DoNotIndex{\indent,\noindent,\newline,\linebreak,\pagebreak,\newpage}
% \DoNotIndex{\textbf,\textit,\textsf,\texttt,\textquotedblleft,\textquotedblright}
% \DoNotIndex{\plainTeX,\TeX,\LaTeX,\pdfLaTeX}
% \DoNotIndex{\item}
%
% \title{The \xpackage{romanbar} package}
% \date{2012/01/01 v1.0f}
% \author{H.-Martin M\"{u}nch\\\xemail{Martin.Muench at Uni-Bonn.de}}
%
% \maketitle
%
% \begin{abstract}
%  This \LaTeX{} package allows to write Roman numbers
%  (or any other text) with bars. (Additionally, commands for converting Arabic numbers
%  into Roman ones are provided and an |\ifnumeric| test function.)
% \end{abstract}
%
% \bigskip
%
% \noindent Disclaimer for web links: The author is not responsible for any contents
% referred to in this work unless he has full knowledge of illegal contents.
% If any damage occurs by the use of information presented there, only the
% author of the respective pages might be liable, not the one who has referred
% to these pages.
%
% \bigskip
%
% \noindent {\color{green} Save per page about $200\unit{ml}$ water,
% $2\unit{g}$ CO$_{2}$ and $2\unit{g}$ wood:\\
% Therefore please print only if this is really necessary.}
%
% \pagebreak
%
% \tableofcontents
%
% \pagebreak
%
% \section{Introduction}
%
% \indent This \LaTeX{} package allows to write Roman numbers
% (or any other text) with bars. (Additionally, command for converting Arabic numbers
% into Roman ones are provided and an |\ifnumeric| test function.)
%
% \bigskip
%
% \section{Usage}
%
% \indent Just load the package placing
% \begin{quote}
%   |\usepackage{romanbar}|
% \end{quote}
% \noindent in the preamble of your \LaTeXe{} source file.\\
% |\Romanbar{...}| then produces a Roman number with bars (please see the example file).
% |\ifnumeric{test}{true}{false}| tests for |test| being numeric.
% |\romannum{...}| turns an Arabic number into a lowercase Roman one, and
% |\Romannum{...}| turns an Arabic number into an uppercase Roman one.
%
% \subsection*{Options}
% \DescribeMacro{options}
% \indent The \xpackage{romanbar} package takes no options.
%
% \section{Alternatives\label{sec:Alternatives}}
%
% \begin{description}
% \item[-] the original code written by Prof. Enrico Gregorio\\
%   (\url{http://profs.sci.univr.it/~gregorio/}),\\
%   \url{http://tex.stackexchange.com/questions/24065/roman-numerals-formatting/24084#24084}
%\begin{verbatim}
%\def\barroman#1{\sbox0{#1}\dimen0=\dimexpr\wd0+1pt\relax
%  \makebox[\dimen0]{\rlap{\vrule width\dimen0 height 0.06ex depth 0.06ex}%
%    \rlap{\vrule width\dimen0 height\dimexpr\ht0+0.03ex\relax
%            depth\dimexpr-\ht0+0.09ex\relax}%
%    \kern.5pt#1\kern.5pt}}
%
%\barroman{I} \barroman{XI}
%\end{verbatim}
%
% \item[-]  Some fonts provide single characters for Roman numerals,
%   cf.~e.\,g.
% \url{http://tex.stackexchange.com/questions/38695/using-unicode-roman-numerals-in-xetex}.
% \end{description}
%
% \noindent (You programmed or found another alternative,
%  which is available at \CTAN{}?\\
%  OK, send an e-mail to me with the name, location at \CTAN{},
%  and a short notice, and I will probably include it in
%  the list above.)
%
% \newpage
%
% \section{Example}
%
%    \begin{macrocode}
%<*example>
\documentclass[british]{article}[2007/10/19]% v1.4h
%%%%%%%%%%%%%%%%%%%%%%%%%%%%%%%%%%%%%%%%%%%%%%%%%%%%%%%%%%%%%%%%%%%%%
\usepackage[extension=pdf,%
 plainpages=false,%
 pdfpagelabels=true,%
 hyperindex=false,%
 pdflang={en},%
 pdftitle={romanbar package example},%
 pdfauthor={H.-Martin Muench},%
 pdfsubject={Example for the romanbar package},%
 pdfkeywords={LaTeX, romanbar, roman, Roman, bars, H.-Martin Muench},%
 pdfview={XYZ null null 1},%
 pdfstartview={XYZ null null 1},%
 pdfpagelayout=SinglePage]{hyperref}[2011/12/04]% v6.82m
\usepackage{romanbar}[2012/01/01]% v1.0f
\gdef\unit#1{\mathord{\thinspace\mathrm{#1}}}%
\listfiles
\begin{document}
\pagenumbering{arabic}
\section*{Example for romanbar}

This example demonstrates the use of package\newline
\textsf{romanbar}, v1.0f as of 2012/01/01 (HMM).\newline
There are no options to be used.\newline

\noindent For more details please see the documentation!\newline

\noindent Save per page about $200\unit{ml}$ water,
$2\unit{g}$ CO$_{2}$ and $2\unit{g}$ wood:\newline
Therefore please print only if this is really necessary.\newline

\noindent This package provides the command \verb|\Romanbar|
to print bars below and over the following:

\begin{description}
\item[-] Roman numbers: \verb|\Romanbar{MMXII}| prints \Romanbar{MMXII}

\item[-] Arabic numbers turned into upper-case Roman numbers:\newline
           \verb|\Romanbar{2012}| prints \Romanbar{2012}

\item[-] negative Arabic numbers turned into upper-case Roman numbers
           \newline
           (although historically there were no negative Roman numbers):
           \newline
           \verb|\Romanbar{-12}| prints \Romanbar{-12}

\item[-] zero Arabic number ($0$; although historically
           there was no Roman zero):\newline
           \verb|\Romanbar{0}| prints \Romanbar{0}

\item[-] some arbitrary text:
           \verb|\Romanbar{Caesar}| prints \Romanbar{Caesar}\newline
           (with descenders: \Romanbar{AgjpqyW})


\newcounter{example}
\setcounter{example}{21}

\item[-] some counter's value:
           \verb|\Romanbar{\theexample}| prints \Romanbar{\theexample}
           \newline
           (where the value of \texttt{example} is \theexample)

\item[-] Arabic numbers, without turning them into upper-case Roman
           numbers:\newline
           \verb|\Romanbar{\relax 2012}| prints \Romanbar{\relax  2012}
\end{description}

Special care was taken for "L" (50), e.\,g. in 555/DLV: \Romanbar{555}.\\


\verb|\romannum{...}| converts an Arabic number into a lower-case Roman one,
e.\,g. \verb|\romannum{2012}| prints \romannum{2012}, and
\verb|\Romannum{...}| converts an Arabic number into an upper-case Roman one,
e.\,g. \verb|\Romannum{2012}| prints \Romannum{2012}.

\end{document}
%</example>
%    \end{macrocode}
%
% \StopEventually{}
%
% \newpage
%
% \section{The implementation}
%
% We start off by checking that we are loading into \LaTeXe{} and
% announcing the name and version of this package.
%
%    \begin{macrocode}
%<*package>
%    \end{macrocode}
%
%    \begin{macrocode}
\NeedsTeXFormat{LaTeX2e}[2011/06/27]
\ProvidesPackage{romanbar}[2012/01/01 v1.0f
              Roman numbers with bars (HMM)]

%    \end{macrocode}
%
% A short description of the \xpackage{romanbar} package:
%
%    \begin{macrocode}
%% Allows to write Roman numbers (or any other text) with bars
%% and to convert Arabic numbers into Roman ones.
%    \end{macrocode}
%
% A last information for the user:
%
%    \begin{macrocode}
%% romanbar may work with earlier versions of LaTeX,
%% but this was not tested. Please consider updating
%% your LaTeX to the most recent version
%% (if it is not already the most recent version).

%    \end{macrocode}
%
% \indent See subsection~\ref{ss:Downloads} about how to get it.\\
%
% There are no options to be processed.
%
% A command to gobble a possible minus sign is needed and
% |\gobbleminus| from \url{http://www.tex.ac.uk/cgi-bin/texfaq2html?label=isitanum}
% is used.
%
%    \begin{macrocode}
\def\gobbleminus#1{\ifx-#1\else#1\fi}
%% from http://www.tex.ac.uk/cgi-bin/texfaq2html?label=isitanum

%    \end{macrocode}
%
% We want to test whether the argument passed to |\Romanbar| is numeric
% and therefore define |\ifnumeric{test}{true}{false}|,
% where |test| is to be analysed, |true| is the code to be executed,
% if |test| is numeric, and |false| is the code to be executed,
% if |test| is not numeric. This is done similar to\\
% \url{http://tex.stackexchange.com/a/17119}.
%
%    \begin{macrocode}
\newcommand{\ifnumeric}[3]{%
%% similar to http://tex.stackexchange.com/a/17119
\sbox\z@{\@tempcnta=0\gobbleminus#1\relax}%
\ifdim\wd0>\z@\relax#3% is not numeric
\else#2% is numeric
\fi%
}

%    \end{macrocode}
%
% We define |\r@iseL| to raise any L (50), otherwise a good deal
% of the horizontal part of the letter would be swallowed by the lower bar
% under the \textquotedblleft{}number\textquotedblright{}.
%
%    \begin{macrocode}
\def\r@iseL#1{\ifx @#1% then terminate
\else%
\if L#1\raisebox{0.05ex}{L}\else #1\fi%
\expandafter\r@iseL%
\fi%
}

%    \end{macrocode}
%
% \pagebreak
%
% We define the |\Romanbar| command.
%
%    \begin{macrocode}
\DeclareRobustCommand{\Romanbar}[1]{%
\ifnumeric{#1}{% is numeric
%    \end{macrocode}
%
% If its parameter is numeric, the temporary counter |\@tempcnta| is set to the value.
%
%    \begin{macrocode}
\@tempcnta=#1\relax%
%    \end{macrocode}
%
% Normally there are no non-positive Roman numbers, therefore we do some trick
% similar to |\XRoman| from the \url{http://ctan.org/pkg/pageslts} package.
%
%    \begin{macrocode}
%% similar to \XRoman from the http://ctan.org/pkg/pageslts package
\ifnum\@tempcnta<1%
  \ifnum #1>0%
    \relax \Roman@bar{\Roman{#1}}%
  \else%
    \ifnum #1<0%
     -\Roman@bar{\expandafter\@slowromancap\romannumeral\number-#1@%
     }%
    \else%
      \Roman@bar{0}%
    \fi%
  \fi%
\else\Roman@bar{\@Roman\@tempcnta}%
\fi%
}{% is not numeric
\Roman@bar{#1}%
}%
}

%    \end{macrocode}
%
% Whether it is numeric or not, in the end |\Roman@bar| is applied to it.
%
%    \begin{macrocode}
\DeclareRobustCommand{\Roman@bar}[1]{% #1 is in Roman, i.e. MMXII
%% (or any other text, "Caesar" would work, too).
%% similar to code of  by Prof. Enrico Gregorio (egreg) at
%% http://tex.stackexchange.com/questions/24065/roman-numerals-formatting/24084#24084
\@bsphack%
\edef\romanbartmp{#1}%
%% height:
\sbox0{\textsf{\romanbartmp}}%
%% top line:
%% + top of top line:
\@tempdima=\ht0%
\advance\@tempdima+0.05ex%0.03
\dimen1=\@tempdima%
%% + bottom of top line:
\@tempdima=-\ht0%
\advance\@tempdima+0.05ex%0.07
\dimen2=\@tempdima%
%% width:
\sbox0{\textsf{\expandafter\r@iseL\romanbartmp @}}%
\@tempdima=\wd0%
\advance\@tempdima+1pt%
\dimen0=\@tempdima%
\@esphack%
\makebox[\dimen0]{%
\rlap{\vrule width\dimen0 height\dimen1 depth\dimen2}% top    line
\rlap{\vrule width\dimen0 height 0.06ex depth 0.03ex}% bottom line
\kern0.5pt\textsf{\expandafter\r@iseL\romanbartmp @}\kern0.5pt}%
}

%    \end{macrocode}
%
% While it is not needed for the bars, it is sometimes asked for a command
% to convert an Arabic number into a Roman one. |\roman{...}| can only be applied
% to a counter, i.\,e. |\roman{12}| does not work, but
%\begin{verbatim}
%\newcounter{examplecounter}
%\setcounter{examplecounter}{12}
%\roman{examplecounter}
%\end{verbatim}
% would be necessary. Further non-negative values would not work.
% |\romannum| accepts any number as argument. (If the argument is no number,
% an error message is given and the argument printed (executed)
%  without applying anything to it.)
%
%    \begin{macrocode}
\DeclareRobustCommand{\romannum}[1]{%
\ifnumeric{#1}{% is numeric
\@tempcnta=#1\relax%
%% similar to \xroman from the http://ctan.org/pkg/pageslts package
\ifnum\@tempcnta<1%
  \ifnum #1>0%
    \relax \roman{#1}%
  \else%
    \ifnum #1<0%
     -\romannumeral\number-#1@%
    \else%
      0%
    \fi%
  \fi%
\else\@roman\@tempcnta%
\fi%
}{% is not numeric
\PackageError{romanbar}{%
Argument of \string\romannum\space is not a number}{%
The command \string\romannum\space converts an Arabic number into a%
 lower-case Roman one,\MessageBreak%
but the used argument of \string\romannum\space is not an Arabic number%
 but\MessageBreak%
`#1`,\MessageBreak%
which will now be printed unchanged.}%
#1%
}%
}

%    \end{macrocode}
%
% The same for upper-case Roman numbers, |\Romannum| instead of |\Roman|:
%
%    \begin{macrocode}
\DeclareRobustCommand{\Romannum}[1]{%
\ifnumeric{#1}{% is numeric
\@tempcnta=#1\relax%
%% similar to \XRoman from the http://ctan.org/pkg/pageslts package
\ifnum\@tempcnta<1%
  \ifnum #1>0%
    \relax \Roman{#1}%
  \else%
    \ifnum #1<0%
     -\expandafter\@slowromancap\romannumeral\number-#1@%
    \else%
      0%
    \fi%
  \fi%
\else\@Roman\@tempcnta%
\fi%
}{% is not numeric
\PackageError{romanbar}{%
Argument of \string\Romannum\space is not a number}{%
The command \string\Romannum\space converts an Arabic number into an%
 upper-case Roman one,\MessageBreak%
but the used argument of \string\Romannum\space is not an Arabic number%
 but\MessageBreak%
`#1`.}%
#1%
}%
}

%    \end{macrocode}
%
%    \begin{macrocode}
%</package>
%    \end{macrocode}
%
% \newpage
%
% \section{Installation}
%
% \subsection{Downloads\label{ss:Downloads}}
%
% Everything should be available on \CTAN{}, \url{http://www.ctan.org/tex-archive/},
% but may need additional packages themselves.\\
%
% \DescribeMacro{romanbar.dtx}
% For unpacking the |romanbar.dtx| file and constructing the documentation
% it is required:
% \begin{description}
% \item[-] \TeX{} Format \LaTeXe{}, \url{http://www.CTAN.org/}
%
% \item[-] document class \xpackage{ltxdoc}, 2007/11/11, v2.0u,
%           \url{http://ctan.org/pkg/ltxdoc}
%
% \item[-] package \xpackage{holtxdoc}, 2011/02/04, v0.21,
%           \url{http://ctan.org/pkg/holtxdoc}
%
% \item[-] package \xpackage{hypdoc}, 2010/03/26, v1.9,
%           \url{http://ctan.org/pkg/hypdoc}
% \end{description}
%
% \DescribeMacro{romanbar.sty}
% The \texttt{romanbar.sty} for \LaTeXe{} (i.\,e.~each document using
% the \xpackage{romanbar} package) requires:
% \begin{description}
% \item[-] \TeX Format \LaTeXe{}, \url{http://www.CTAN.org/}
% \end{description}
%
% \DescribeMacro{romanbar-example.tex}
% The \texttt{romanbar-example.tex} requires the same files as all
% documents using the \xpackage{romanbar} package and additionally:
% \begin{description}
% \item[-] class \xpackage{article}, 2007/10/19, v1.4h, from \xpackage{classes.dtx}:\\
%           \CTAN{macros/latex/base/classes.dtx}
%
% \item[-] package \xpackage{romanbar}, 2012/01/01, v1.0f,\\
%           %%\url{http://ctan.org/pkg/romanbar}\\
%           \CTAN{macros/latex/contrib/romanbar/}
%   (Well, it is the example file for this package, and because you are reading the
%    documentation for the \xpackage{romanbar} package, it can be assumed that you already
%    have some version of it -- is it the current one?)
% \end{description}
%
% \DescribeMacro{Alternative}
% As possible alternative package in section \ref{sec:Alternatives} there is listed
% \begin{description}
% \item[-] none.
% \end{description}
%
% \DescribeMacro{Oberdiek}
% \DescribeMacro{holtxdoc}
% All packages of \textsc{Heiko Oberdiek's} bundle `oberdiek'
% (especially \xpackage{holtxdoc}) are also available in a TDS compliant ZIP archive:\\
% \CTAN{install/macros/latex/contrib/oberdiek.tds.zip}.\\
% It is probably best to download and use this, because the packages in there
% are quite probably both recent and compatible among themselves.\\
%
% \DescribeMacro{hyperref}
% \noindent \xpackage{hyperref} is not included in that bundle and needs to be
% downloaded separately,\\
% \url{http://mirror.ctan.org/install/macros/latex/contrib/hyperref.tds.zip}.\\
%
% \DescribeMacro{M\"{u}nch}
% A hyperlinked list of my (other) packages can be found at
% \url{http://www.Uni-Bonn.de/~uzs5pv/LaTeX.html}.\\
%
% \pagebreak
%
% \subsection{Package, unpacking TDS}
% \paragraph{Package.} This package should become available on \CTAN{} soon:
% \begin{description}
% \item[\CTAN{macros/latex/contrib/romanbar/romanbar.dtx}]\hspace*{0.1cm} \\
%       The source file.
% \item[\CTAN{macros/latex/contrib/romanbar/romanbar.pdf}]\hspace*{0.1cm} \\
%       The documentation.
% \item[\CTAN{macros/latex/contrib/romanbar/romanbar-example.pdf}]\hspace*{0.1cm} \\
%       The compiled example file, as it should look like.
% \item[\CTAN{macros/latex/contrib/romanbar/README}]\hspace*{0.1cm} \\
%       The README file.
% \end{description}
% There is also a romanbar.tds.zip available:
% \begin{description}
% \item[\CTAN{install/macros/latex/contrib/romanbar.tds.zip}]\hspace*{0.1cm} \\
%       Everything in \xfile{TDS} compliant, compiled format.
% \end{description}
% which additionally contains\\
% \begin{tabular}{ll}
% romanbar.ins & The installation file.\\
% romanbar.drv & The driver to generate the documentation.\\
% romanbar.sty & The \xext{sty}le file.\\
% romanbar-example.tex & The example file.
% \end{tabular}
%
% \bigskip
%
% \noindent For required other packages, please see the preceding subsection.
%
% \paragraph{Unpacking.} The \xfile{.dtx} file is a self-extracting
% \docstrip{} archive. The files are extracted by running the
% \xext{.dtx} through \plainTeX{}:
% \begin{quote}
%   \verb|tex romanbar.dtx|
% \end{quote}
%
% About generating the documentation see paragraph~\ref{GenDoc} below.\\
%
% \paragraph{TDS.} Now the different files must be moved into
% the different directories in your installation TDS tree
% (also known as \xfile{texmf} tree):
% \begin{quote}
% \def\t{^^A
% \begin{tabular}{@{}>{\ttfamily}l@{ $\rightarrow$ }>{\ttfamily}l@{}}
%   romanbar.sty & tex/latex/romanbar/romanbar.sty\\
%   romanbar.pdf & doc/latex/romanbar/romanbar.pdf\\
%   romanbar-example.tex & doc/latex/romanbar/romanbar-example.tex\\
%   romanbar-example.pdf & doc/latex/romanbar/romanbar-example.pdf\\
%   romanbar.dtx & source/latex/romanbar/romanbar.dtx\\
% \end{tabular}^^A
% }^^A
% \sbox0{\t}^^A
% \ifdim\wd0>\linewidth
%   \begingroup
%     \advance\linewidth by\leftmargin
%     \advance\linewidth by\rightmargin
%   \edef\x{\endgroup
%     \def\noexpand\lw{\the\linewidth}^^A
%   }\x
%   \def\lwbox{^^A
%     \leavevmode
%     \hbox to \linewidth{^^A
%       \kern-\leftmargin\relax
%       \hss
%       \usebox0
%       \hss
%       \kern-\rightmargin\relax
%     }^^A
%   }^^A
%   \ifdim\wd0>\lw
%     \sbox0{\small\t}^^A
%     \ifdim\wd0>\linewidth
%       \ifdim\wd0>\lw
%         \sbox0{\footnotesize\t}^^A
%         \ifdim\wd0>\linewidth
%           \ifdim\wd0>\lw
%             \sbox0{\scriptsize\t}^^A
%             \ifdim\wd0>\linewidth
%               \ifdim\wd0>\lw
%                 \sbox0{\tiny\t}^^A
%                 \ifdim\wd0>\linewidth
%                   \lwbox
%                 \else
%                   \usebox0
%                 \fi
%               \else
%                 \lwbox
%               \fi
%             \else
%               \usebox0
%             \fi
%           \else
%             \lwbox
%           \fi
%         \else
%           \usebox0
%         \fi
%       \else
%         \lwbox
%       \fi
%     \else
%       \usebox0
%     \fi
%   \else
%     \lwbox
%   \fi
% \else
%   \usebox0
% \fi
% \end{quote}
% If you have a \xfile{docstrip.cfg} that configures and enables \docstrip{}'s
% \xfile{TDS} installing feature, then some files can already be in the right
% place, see the documentation of \docstrip{}.
%
% \subsection{Refresh file name databases}
%
% If your \TeX{}~distribution (\teTeX{}, \mikTeX{},\dots{}) relies on file name
% databases, you must refresh these. For example, \teTeX{} users run
% \verb|texhash| or \verb|mktexlsr|.
%
% \subsection{Some details for the interested}
%
% \paragraph{Unpacking with \LaTeX{}.}
% The \xfile{.dtx} chooses its action depending on the format:
% \begin{description}
% \item[\plainTeX:] Run \docstrip{} and extract the files.
% \item[\LaTeX:] Generate the documentation.
% \end{description}
% If you insist on using \LaTeX{} for \docstrip{} (really,
% \docstrip{} does not need \LaTeX{}), then inform the autodetect routine
% about your intention:
% \begin{quote}
%   \verb|latex \let\install=y% \iffalse meta-comment
%
% File: romanbar.dtx
% Version: 2012/01/01 v1.0f
%
% Copyright (C) 2011 by
%    H.-Martin M"unch <Martin dot Muench at Uni-Bonn dot de>
%
% The original code for the bars was written by Prof. Enrico Gregorio
% (http://profs.sci.univr.it/~gregorio/), and published at
% http://tex.stackexchange.com/questions/24065/roman-numerals-formatting/24084#24084
% I (H.-Martin Münch) submitted this package to CTAN
% (after diverse updates of the code and writing a documentation).
%
% This work may be distributed and/or modified under the
% conditions of the LaTeX Project Public License, either
% version 1.3c of this license or (at your option) any later
% version. This version of this license is in
%    http://www.latex-project.org/lppl/lppl-1-3c.txt
% and the latest version of this license is in
%    http://www.latex-project.org/lppl.txt
% and version 1.3c or later is part of all distributions of
% LaTeX version 2005/12/01 or later.
%
% This work has the LPPL maintenance status "maintained".
%
% The Current Maintainer of this work is H.-Martin Muench.
%
% This work consists of the main source file romanbar.dtx,
% the README, and the derived files
%    romanbar.sty, romanbar.pdf,
%    romanbar.ins, romanbar.drv,
%    romanbar-example.tex, romanbar-example.pdf.
%
% Distribution:
%    CTAN:macros/latex/contrib/romanbar/romanbar.dtx
%    CTAN:macros/latex/contrib/romanbar/romanbar.pdf
%    CTAN:install/macros/latex/contrib/romanbar.tds.zip
%
% Cf. http://ctan.org/pkg/romanbar and
% http://www.tex.ac.uk/tex-archive/help/Catalogue/entries/romanbar.html
% (It might take some time for the entries to be created.)
%
% Unpacking:
%    (a) If romanbar.ins is present:
%           tex romanbar.ins
%    (b) Without romanbar.ins:
%           tex romanbar.dtx
%    (c) If you insist on using LaTeX
%           latex \let\install=y\input{romanbar.dtx}
%        (quote the arguments according to the demands of your shell)
%
% Documentation:
%    (a) If romanbar.drv is present:
%           (pdf)latex romanbar.drv
%           makeindex -s gind.ist romanbar.idx
%           (pdf)latex romanbar.drv
%           makeindex -s gind.ist romanbar.idx
%           (pdf)latex romanbar.drv
%    (b) Without romanbar.drv:
%           (pdf)latex romanbar.dtx
%           makeindex -s gind.ist romanbar.idx
%           (pdf)latex romanbar.dtx
%           makeindex -s gind.ist romanbar.idx
%           (pdf)latex romanbar.dtx
%
%    The class ltxdoc loads the configuration file ltxdoc.cfg
%    if available. Here you can specify further options, e.g.
%    use DIN A4 as paper format:
%       \PassOptionsToClass{a4paper}{article}
%
% Installation:
%    TDS:tex/latex/romanbar/romanbar.sty
%    TDS:doc/latex/romanbar/romanbar.pdf
%    TDS:doc/latex/romanbar/romanbar-example.tex
%    TDS:doc/latex/romanbar/romanbar-example.pdf
%    TDS:source/latex/romanbar/romanbar.dtx
%
%<*ignore>
\begingroup
  \catcode123=1 %
  \catcode125=2 %
  \def\x{LaTeX2e}%
\expandafter\endgroup
\ifcase 0\ifx\install y1\fi\expandafter
         \ifx\csname processbatchFile\endcsname\relax\else1\fi
         \ifx\fmtname\x\else 1\fi\relax
\else\csname fi\endcsname
%</ignore>
%<*install>
\input docstrip.tex
\Msg{********************************************************************}
\Msg{* Installation                                                     *}
\Msg{* Package: romanbar 2012/01/01 v1.0f Roman numbers with bars (HMM) *}
\Msg{********************************************************************}

\keepsilent
\askforoverwritefalse

\let\MetaPrefix\relax
\preamble

This is a generated file.

Project: romanbar
Version: 2012/01/01 v1.0f

Copyright (C) 2011 by
    H.-Martin M"unch <Martin dot Muench at Uni-Bonn dot de>

The usual disclaimer applies:
If it doesn't work right that's your problem.
(Nevertheless, send an e-mail to the maintainer
 when you find an error in this package.)

This work may be distributed and/or modified under the
conditions of the LaTeX Project Public License, either
version 1.3c of this license or (at your option) any later
version. This version of this license is in
   http://www.latex-project.org/lppl/lppl-1-3c.txt
and the latest version of this license is in
   http://www.latex-project.org/lppl.txt
and version 1.3c or later is part of all distributions of
LaTeX version 2005/12/01 or later.

This work has the LPPL maintenance status "maintained".

The Current Maintainer of this work is H.-Martin Muench.

This work consists of the main source file romanbar.dtx,
the README, and the derived files
   romanbar.sty, romanbar.pdf,
   romanbar.ins, romanbar.drv,
   romanbar-example.tex, romanbar-example.pdf.

\endpreamble
\let\MetaPrefix\DoubleperCent

\generate{%
  \file{romanbar.ins}{\from{romanbar.dtx}{install}}%
  \file{romanbar.drv}{\from{romanbar.dtx}{driver}}%
  \usedir{tex/latex/romanbar}%
  \file{romanbar.sty}{\from{romanbar.dtx}{package}}%
  \usedir{doc/latex/romanbar}%
  \file{romanbar-example.tex}{\from{romanbar.dtx}{example}}%
}

\catcode32=13\relax% active space
\let =\space%
\Msg{************************************************************************}
\Msg{*}
\Msg{* To finish the installation you have to move the following}
\Msg{* file into a directory searched by TeX:}
\Msg{*}
\Msg{*     romanbar.sty}
\Msg{*}
\Msg{* To produce the documentation run the file `romanbar.drv'}
\Msg{* through (pdf)LaTeX, e.g.}
\Msg{*  pdflatex romanbar.drv}
\Msg{*  makeindex -s gind.ist romanbar.idx}
\Msg{*  pdflatex romanbar.drv}
\Msg{*  makeindex -s gind.ist romanbar.idx}
\Msg{*  pdflatex romanbar.drv}
\Msg{*}
\Msg{* At least three runs are necessary e.g. to get the}
\Msg{*  references right!}
\Msg{*}
\Msg{* Happy TeXing!}
\Msg{*}
\Msg{************************************************************************}

\endbatchfile
%</install>
%<*ignore>
\fi
%</ignore>
%
% \section{The documentation driver file}
%
% The next bit of code contains the documentation driver file for
% \TeX{}, i.\,e., the file that will produce the documentation you
% are currently reading. It will be extracted from this file by the
% \texttt{docstrip} programme. That is, run \LaTeX{} on \texttt{docstrip}
% and specify the \texttt{driver} option when \texttt{docstrip}
% asks for options.
%
%    \begin{macrocode}
%<*driver>
\NeedsTeXFormat{LaTeX2e}[2011/06/27]
\ProvidesFile{romanbar.drv}%
  [2012/01/01 v1.0f Roman numbers with bars (HMM)]
\documentclass{ltxdoc}[2007/11/11]% v2.0u
\usepackage{holtxdoc}[2011/02/04]%  v0.21
%% romanbar may work with earlier versions of LaTeX2e and those
%% class and package, but this was not tested.
%% Please consider updating your LaTeX, class, and package
%% to the most recent version (if they are not already the most
%% recent version).
\hypersetup{%
 pdfsubject={Writing Roman numbers with bars (HMM)},%
 pdfkeywords={LaTeX, romanbar, roman, Roman, bars, H.-Martin Muench},%
 pdfencoding=auto,%
 pdflang={en},%
 breaklinks=true,%
 linktoc=all,%
 pdfstartview=FitH,%
 pdfpagelayout=OneColumn,%
 bookmarksnumbered=true,%
 bookmarksopen=true,%
 bookmarksopenlevel=2,%
 pdfmenubar=true,%
 pdftoolbar=true,%
 pdfwindowui=true,%
 pdfnewwindow=true%
}
\CodelineIndex
\hyphenation{docu-ment ana-lysed}
\gdef\unit#1{\mathord{\thinspace\mathrm{#1}}}%
\makeatletter
\@ifundefined{eTeX}{\gdef\eTeX{$\m@th \varepsilon $-\TeX }}{% else \relax
}
\makeatother
\begin{document}
  \DocInput{romanbar.dtx}%
\end{document}
%</driver>
%    \end{macrocode}
%
% \fi
%
% \CheckSum{179}
%
% \CharacterTable
%  {Upper-case    \A\B\C\D\E\F\G\H\I\J\K\L\M\N\O\P\Q\R\S\T\U\V\W\X\Y\Z
%   Lower-case    \a\b\c\d\e\f\g\h\i\j\k\l\m\n\o\p\q\r\s\t\u\v\w\x\y\z
%   Digits        \0\1\2\3\4\5\6\7\8\9
%   Exclamation   \!     Double quote  \"     Hash (number) \#
%   Dollar        \$     Percent       \%     Ampersand     \&
%   Acute accent  \'     Left paren    \(     Right paren   \)
%   Asterisk      \*     Plus          \+     Comma         \,
%   Minus         \-     Point         \.     Solidus       \/
%   Colon         \:     Semicolon     \;     Less than     \<
%   Equals        \=     Greater than  \>     Question mark \?
%   Commercial at \@     Left bracket  \[     Backslash     \\
%   Right bracket \]     Circumflex    \^     Underscore    \_
%   Grave accent  \`     Left brace    \{     Vertical bar  \|
%   Right brace   \}     Tilde         \~}
%
% \GetFileInfo{romanbar.drv}
%
% \begingroup
%   \def\x{\#,\$,\^,\_,\~,\ ,\&,\{,\},\%}%
%   \makeatletter
%   \@onelevel@sanitize\x
% \expandafter\endgroup
% \expandafter\DoNotIndex\expandafter{\x}
% \expandafter\DoNotIndex\expandafter{\string\ }
% \begingroup
%   \makeatletter
%     \lccode`9=32\relax
%     \lowercase{%^^A
%       \edef\x{\noexpand\DoNotIndex{\@backslashchar9}}%^^A
%     }%^^A
%   \expandafter\endgroup\x
%
% \DoNotIndex{\\,\,}
% \DoNotIndex{\documentclass,\usepackage,\ProvidesPackage,\begin,\end}
% \DoNotIndex{\MessageBreak}
% \DoNotIndex{\NeedsTeXFormat,\DoNotIndex,\verb}
% \DoNotIndex{\def,\edef,\gdef,\global}
% \DoNotIndex{\ifx,\listfiles,\mathord,\mathrm}
% \DoNotIndex{\bigskip,\space,\thinspace,\ldots}
% \DoNotIndex{\indent,\noindent,\newline,\linebreak,\pagebreak,\newpage}
% \DoNotIndex{\textbf,\textit,\textsf,\texttt,\textquotedblleft,\textquotedblright}
% \DoNotIndex{\plainTeX,\TeX,\LaTeX,\pdfLaTeX}
% \DoNotIndex{\item}
%
% \title{The \xpackage{romanbar} package}
% \date{2012/01/01 v1.0f}
% \author{H.-Martin M\"{u}nch\\\xemail{Martin.Muench at Uni-Bonn.de}}
%
% \maketitle
%
% \begin{abstract}
%  This \LaTeX{} package allows to write Roman numbers
%  (or any other text) with bars. (Additionally, commands for converting Arabic numbers
%  into Roman ones are provided and an |\ifnumeric| test function.)
% \end{abstract}
%
% \bigskip
%
% \noindent Disclaimer for web links: The author is not responsible for any contents
% referred to in this work unless he has full knowledge of illegal contents.
% If any damage occurs by the use of information presented there, only the
% author of the respective pages might be liable, not the one who has referred
% to these pages.
%
% \bigskip
%
% \noindent {\color{green} Save per page about $200\unit{ml}$ water,
% $2\unit{g}$ CO$_{2}$ and $2\unit{g}$ wood:\\
% Therefore please print only if this is really necessary.}
%
% \pagebreak
%
% \tableofcontents
%
% \pagebreak
%
% \section{Introduction}
%
% \indent This \LaTeX{} package allows to write Roman numbers
% (or any other text) with bars. (Additionally, command for converting Arabic numbers
% into Roman ones are provided and an |\ifnumeric| test function.)
%
% \bigskip
%
% \section{Usage}
%
% \indent Just load the package placing
% \begin{quote}
%   |\usepackage{romanbar}|
% \end{quote}
% \noindent in the preamble of your \LaTeXe{} source file.\\
% |\Romanbar{...}| then produces a Roman number with bars (please see the example file).
% |\ifnumeric{test}{true}{false}| tests for |test| being numeric.
% |\romannum{...}| turns an Arabic number into a lowercase Roman one, and
% |\Romannum{...}| turns an Arabic number into an uppercase Roman one.
%
% \subsection*{Options}
% \DescribeMacro{options}
% \indent The \xpackage{romanbar} package takes no options.
%
% \section{Alternatives\label{sec:Alternatives}}
%
% \begin{description}
% \item[-] the original code written by Prof. Enrico Gregorio\\
%   (\url{http://profs.sci.univr.it/~gregorio/}),\\
%   \url{http://tex.stackexchange.com/questions/24065/roman-numerals-formatting/24084#24084}
%\begin{verbatim}
%\def\barroman#1{\sbox0{#1}\dimen0=\dimexpr\wd0+1pt\relax
%  \makebox[\dimen0]{\rlap{\vrule width\dimen0 height 0.06ex depth 0.06ex}%
%    \rlap{\vrule width\dimen0 height\dimexpr\ht0+0.03ex\relax
%            depth\dimexpr-\ht0+0.09ex\relax}%
%    \kern.5pt#1\kern.5pt}}
%
%\barroman{I} \barroman{XI}
%\end{verbatim}
%
% \item[-]  Some fonts provide single characters for Roman numerals,
%   cf.~e.\,g.
% \url{http://tex.stackexchange.com/questions/38695/using-unicode-roman-numerals-in-xetex}.
% \end{description}
%
% \noindent (You programmed or found another alternative,
%  which is available at \CTAN{}?\\
%  OK, send an e-mail to me with the name, location at \CTAN{},
%  and a short notice, and I will probably include it in
%  the list above.)
%
% \newpage
%
% \section{Example}
%
%    \begin{macrocode}
%<*example>
\documentclass[british]{article}[2007/10/19]% v1.4h
%%%%%%%%%%%%%%%%%%%%%%%%%%%%%%%%%%%%%%%%%%%%%%%%%%%%%%%%%%%%%%%%%%%%%
\usepackage[extension=pdf,%
 plainpages=false,%
 pdfpagelabels=true,%
 hyperindex=false,%
 pdflang={en},%
 pdftitle={romanbar package example},%
 pdfauthor={H.-Martin Muench},%
 pdfsubject={Example for the romanbar package},%
 pdfkeywords={LaTeX, romanbar, roman, Roman, bars, H.-Martin Muench},%
 pdfview={XYZ null null 1},%
 pdfstartview={XYZ null null 1},%
 pdfpagelayout=SinglePage]{hyperref}[2011/12/04]% v6.82m
\usepackage{romanbar}[2012/01/01]% v1.0f
\gdef\unit#1{\mathord{\thinspace\mathrm{#1}}}%
\listfiles
\begin{document}
\pagenumbering{arabic}
\section*{Example for romanbar}

This example demonstrates the use of package\newline
\textsf{romanbar}, v1.0f as of 2012/01/01 (HMM).\newline
There are no options to be used.\newline

\noindent For more details please see the documentation!\newline

\noindent Save per page about $200\unit{ml}$ water,
$2\unit{g}$ CO$_{2}$ and $2\unit{g}$ wood:\newline
Therefore please print only if this is really necessary.\newline

\noindent This package provides the command \verb|\Romanbar|
to print bars below and over the following:

\begin{description}
\item[-] Roman numbers: \verb|\Romanbar{MMXII}| prints \Romanbar{MMXII}

\item[-] Arabic numbers turned into upper-case Roman numbers:\newline
           \verb|\Romanbar{2012}| prints \Romanbar{2012}

\item[-] negative Arabic numbers turned into upper-case Roman numbers
           \newline
           (although historically there were no negative Roman numbers):
           \newline
           \verb|\Romanbar{-12}| prints \Romanbar{-12}

\item[-] zero Arabic number ($0$; although historically
           there was no Roman zero):\newline
           \verb|\Romanbar{0}| prints \Romanbar{0}

\item[-] some arbitrary text:
           \verb|\Romanbar{Caesar}| prints \Romanbar{Caesar}\newline
           (with descenders: \Romanbar{AgjpqyW})


\newcounter{example}
\setcounter{example}{21}

\item[-] some counter's value:
           \verb|\Romanbar{\theexample}| prints \Romanbar{\theexample}
           \newline
           (where the value of \texttt{example} is \theexample)

\item[-] Arabic numbers, without turning them into upper-case Roman
           numbers:\newline
           \verb|\Romanbar{\relax 2012}| prints \Romanbar{\relax  2012}
\end{description}

Special care was taken for "L" (50), e.\,g. in 555/DLV: \Romanbar{555}.\\


\verb|\romannum{...}| converts an Arabic number into a lower-case Roman one,
e.\,g. \verb|\romannum{2012}| prints \romannum{2012}, and
\verb|\Romannum{...}| converts an Arabic number into an upper-case Roman one,
e.\,g. \verb|\Romannum{2012}| prints \Romannum{2012}.

\end{document}
%</example>
%    \end{macrocode}
%
% \StopEventually{}
%
% \newpage
%
% \section{The implementation}
%
% We start off by checking that we are loading into \LaTeXe{} and
% announcing the name and version of this package.
%
%    \begin{macrocode}
%<*package>
%    \end{macrocode}
%
%    \begin{macrocode}
\NeedsTeXFormat{LaTeX2e}[2011/06/27]
\ProvidesPackage{romanbar}[2012/01/01 v1.0f
              Roman numbers with bars (HMM)]

%    \end{macrocode}
%
% A short description of the \xpackage{romanbar} package:
%
%    \begin{macrocode}
%% Allows to write Roman numbers (or any other text) with bars
%% and to convert Arabic numbers into Roman ones.
%    \end{macrocode}
%
% A last information for the user:
%
%    \begin{macrocode}
%% romanbar may work with earlier versions of LaTeX,
%% but this was not tested. Please consider updating
%% your LaTeX to the most recent version
%% (if it is not already the most recent version).

%    \end{macrocode}
%
% \indent See subsection~\ref{ss:Downloads} about how to get it.\\
%
% There are no options to be processed.
%
% A command to gobble a possible minus sign is needed and
% |\gobbleminus| from \url{http://www.tex.ac.uk/cgi-bin/texfaq2html?label=isitanum}
% is used.
%
%    \begin{macrocode}
\def\gobbleminus#1{\ifx-#1\else#1\fi}
%% from http://www.tex.ac.uk/cgi-bin/texfaq2html?label=isitanum

%    \end{macrocode}
%
% We want to test whether the argument passed to |\Romanbar| is numeric
% and therefore define |\ifnumeric{test}{true}{false}|,
% where |test| is to be analysed, |true| is the code to be executed,
% if |test| is numeric, and |false| is the code to be executed,
% if |test| is not numeric. This is done similar to\\
% \url{http://tex.stackexchange.com/a/17119}.
%
%    \begin{macrocode}
\newcommand{\ifnumeric}[3]{%
%% similar to http://tex.stackexchange.com/a/17119
\sbox\z@{\@tempcnta=0\gobbleminus#1\relax}%
\ifdim\wd0>\z@\relax#3% is not numeric
\else#2% is numeric
\fi%
}

%    \end{macrocode}
%
% We define |\r@iseL| to raise any L (50), otherwise a good deal
% of the horizontal part of the letter would be swallowed by the lower bar
% under the \textquotedblleft{}number\textquotedblright{}.
%
%    \begin{macrocode}
\def\r@iseL#1{\ifx @#1% then terminate
\else%
\if L#1\raisebox{0.05ex}{L}\else #1\fi%
\expandafter\r@iseL%
\fi%
}

%    \end{macrocode}
%
% \pagebreak
%
% We define the |\Romanbar| command.
%
%    \begin{macrocode}
\DeclareRobustCommand{\Romanbar}[1]{%
\ifnumeric{#1}{% is numeric
%    \end{macrocode}
%
% If its parameter is numeric, the temporary counter |\@tempcnta| is set to the value.
%
%    \begin{macrocode}
\@tempcnta=#1\relax%
%    \end{macrocode}
%
% Normally there are no non-positive Roman numbers, therefore we do some trick
% similar to |\XRoman| from the \url{http://ctan.org/pkg/pageslts} package.
%
%    \begin{macrocode}
%% similar to \XRoman from the http://ctan.org/pkg/pageslts package
\ifnum\@tempcnta<1%
  \ifnum #1>0%
    \relax \Roman@bar{\Roman{#1}}%
  \else%
    \ifnum #1<0%
     -\Roman@bar{\expandafter\@slowromancap\romannumeral\number-#1@%
     }%
    \else%
      \Roman@bar{0}%
    \fi%
  \fi%
\else\Roman@bar{\@Roman\@tempcnta}%
\fi%
}{% is not numeric
\Roman@bar{#1}%
}%
}

%    \end{macrocode}
%
% Whether it is numeric or not, in the end |\Roman@bar| is applied to it.
%
%    \begin{macrocode}
\DeclareRobustCommand{\Roman@bar}[1]{% #1 is in Roman, i.e. MMXII
%% (or any other text, "Caesar" would work, too).
%% similar to code of  by Prof. Enrico Gregorio (egreg) at
%% http://tex.stackexchange.com/questions/24065/roman-numerals-formatting/24084#24084
\@bsphack%
\edef\romanbartmp{#1}%
%% height:
\sbox0{\textsf{\romanbartmp}}%
%% top line:
%% + top of top line:
\@tempdima=\ht0%
\advance\@tempdima+0.05ex%0.03
\dimen1=\@tempdima%
%% + bottom of top line:
\@tempdima=-\ht0%
\advance\@tempdima+0.05ex%0.07
\dimen2=\@tempdima%
%% width:
\sbox0{\textsf{\expandafter\r@iseL\romanbartmp @}}%
\@tempdima=\wd0%
\advance\@tempdima+1pt%
\dimen0=\@tempdima%
\@esphack%
\makebox[\dimen0]{%
\rlap{\vrule width\dimen0 height\dimen1 depth\dimen2}% top    line
\rlap{\vrule width\dimen0 height 0.06ex depth 0.03ex}% bottom line
\kern0.5pt\textsf{\expandafter\r@iseL\romanbartmp @}\kern0.5pt}%
}

%    \end{macrocode}
%
% While it is not needed for the bars, it is sometimes asked for a command
% to convert an Arabic number into a Roman one. |\roman{...}| can only be applied
% to a counter, i.\,e. |\roman{12}| does not work, but
%\begin{verbatim}
%\newcounter{examplecounter}
%\setcounter{examplecounter}{12}
%\roman{examplecounter}
%\end{verbatim}
% would be necessary. Further non-negative values would not work.
% |\romannum| accepts any number as argument. (If the argument is no number,
% an error message is given and the argument printed (executed)
%  without applying anything to it.)
%
%    \begin{macrocode}
\DeclareRobustCommand{\romannum}[1]{%
\ifnumeric{#1}{% is numeric
\@tempcnta=#1\relax%
%% similar to \xroman from the http://ctan.org/pkg/pageslts package
\ifnum\@tempcnta<1%
  \ifnum #1>0%
    \relax \roman{#1}%
  \else%
    \ifnum #1<0%
     -\romannumeral\number-#1@%
    \else%
      0%
    \fi%
  \fi%
\else\@roman\@tempcnta%
\fi%
}{% is not numeric
\PackageError{romanbar}{%
Argument of \string\romannum\space is not a number}{%
The command \string\romannum\space converts an Arabic number into a%
 lower-case Roman one,\MessageBreak%
but the used argument of \string\romannum\space is not an Arabic number%
 but\MessageBreak%
`#1`,\MessageBreak%
which will now be printed unchanged.}%
#1%
}%
}

%    \end{macrocode}
%
% The same for upper-case Roman numbers, |\Romannum| instead of |\Roman|:
%
%    \begin{macrocode}
\DeclareRobustCommand{\Romannum}[1]{%
\ifnumeric{#1}{% is numeric
\@tempcnta=#1\relax%
%% similar to \XRoman from the http://ctan.org/pkg/pageslts package
\ifnum\@tempcnta<1%
  \ifnum #1>0%
    \relax \Roman{#1}%
  \else%
    \ifnum #1<0%
     -\expandafter\@slowromancap\romannumeral\number-#1@%
    \else%
      0%
    \fi%
  \fi%
\else\@Roman\@tempcnta%
\fi%
}{% is not numeric
\PackageError{romanbar}{%
Argument of \string\Romannum\space is not a number}{%
The command \string\Romannum\space converts an Arabic number into an%
 upper-case Roman one,\MessageBreak%
but the used argument of \string\Romannum\space is not an Arabic number%
 but\MessageBreak%
`#1`.}%
#1%
}%
}

%    \end{macrocode}
%
%    \begin{macrocode}
%</package>
%    \end{macrocode}
%
% \newpage
%
% \section{Installation}
%
% \subsection{Downloads\label{ss:Downloads}}
%
% Everything should be available on \CTAN{}, \url{http://www.ctan.org/tex-archive/},
% but may need additional packages themselves.\\
%
% \DescribeMacro{romanbar.dtx}
% For unpacking the |romanbar.dtx| file and constructing the documentation
% it is required:
% \begin{description}
% \item[-] \TeX{} Format \LaTeXe{}, \url{http://www.CTAN.org/}
%
% \item[-] document class \xpackage{ltxdoc}, 2007/11/11, v2.0u,
%           \url{http://ctan.org/pkg/ltxdoc}
%
% \item[-] package \xpackage{holtxdoc}, 2011/02/04, v0.21,
%           \url{http://ctan.org/pkg/holtxdoc}
%
% \item[-] package \xpackage{hypdoc}, 2010/03/26, v1.9,
%           \url{http://ctan.org/pkg/hypdoc}
% \end{description}
%
% \DescribeMacro{romanbar.sty}
% The \texttt{romanbar.sty} for \LaTeXe{} (i.\,e.~each document using
% the \xpackage{romanbar} package) requires:
% \begin{description}
% \item[-] \TeX Format \LaTeXe{}, \url{http://www.CTAN.org/}
% \end{description}
%
% \DescribeMacro{romanbar-example.tex}
% The \texttt{romanbar-example.tex} requires the same files as all
% documents using the \xpackage{romanbar} package and additionally:
% \begin{description}
% \item[-] class \xpackage{article}, 2007/10/19, v1.4h, from \xpackage{classes.dtx}:\\
%           \CTAN{macros/latex/base/classes.dtx}
%
% \item[-] package \xpackage{romanbar}, 2012/01/01, v1.0f,\\
%           %%\url{http://ctan.org/pkg/romanbar}\\
%           \CTAN{macros/latex/contrib/romanbar/}
%   (Well, it is the example file for this package, and because you are reading the
%    documentation for the \xpackage{romanbar} package, it can be assumed that you already
%    have some version of it -- is it the current one?)
% \end{description}
%
% \DescribeMacro{Alternative}
% As possible alternative package in section \ref{sec:Alternatives} there is listed
% \begin{description}
% \item[-] none.
% \end{description}
%
% \DescribeMacro{Oberdiek}
% \DescribeMacro{holtxdoc}
% All packages of \textsc{Heiko Oberdiek's} bundle `oberdiek'
% (especially \xpackage{holtxdoc}) are also available in a TDS compliant ZIP archive:\\
% \CTAN{install/macros/latex/contrib/oberdiek.tds.zip}.\\
% It is probably best to download and use this, because the packages in there
% are quite probably both recent and compatible among themselves.\\
%
% \DescribeMacro{hyperref}
% \noindent \xpackage{hyperref} is not included in that bundle and needs to be
% downloaded separately,\\
% \url{http://mirror.ctan.org/install/macros/latex/contrib/hyperref.tds.zip}.\\
%
% \DescribeMacro{M\"{u}nch}
% A hyperlinked list of my (other) packages can be found at
% \url{http://www.Uni-Bonn.de/~uzs5pv/LaTeX.html}.\\
%
% \pagebreak
%
% \subsection{Package, unpacking TDS}
% \paragraph{Package.} This package should become available on \CTAN{} soon:
% \begin{description}
% \item[\CTAN{macros/latex/contrib/romanbar/romanbar.dtx}]\hspace*{0.1cm} \\
%       The source file.
% \item[\CTAN{macros/latex/contrib/romanbar/romanbar.pdf}]\hspace*{0.1cm} \\
%       The documentation.
% \item[\CTAN{macros/latex/contrib/romanbar/romanbar-example.pdf}]\hspace*{0.1cm} \\
%       The compiled example file, as it should look like.
% \item[\CTAN{macros/latex/contrib/romanbar/README}]\hspace*{0.1cm} \\
%       The README file.
% \end{description}
% There is also a romanbar.tds.zip available:
% \begin{description}
% \item[\CTAN{install/macros/latex/contrib/romanbar.tds.zip}]\hspace*{0.1cm} \\
%       Everything in \xfile{TDS} compliant, compiled format.
% \end{description}
% which additionally contains\\
% \begin{tabular}{ll}
% romanbar.ins & The installation file.\\
% romanbar.drv & The driver to generate the documentation.\\
% romanbar.sty & The \xext{sty}le file.\\
% romanbar-example.tex & The example file.
% \end{tabular}
%
% \bigskip
%
% \noindent For required other packages, please see the preceding subsection.
%
% \paragraph{Unpacking.} The \xfile{.dtx} file is a self-extracting
% \docstrip{} archive. The files are extracted by running the
% \xext{.dtx} through \plainTeX{}:
% \begin{quote}
%   \verb|tex romanbar.dtx|
% \end{quote}
%
% About generating the documentation see paragraph~\ref{GenDoc} below.\\
%
% \paragraph{TDS.} Now the different files must be moved into
% the different directories in your installation TDS tree
% (also known as \xfile{texmf} tree):
% \begin{quote}
% \def\t{^^A
% \begin{tabular}{@{}>{\ttfamily}l@{ $\rightarrow$ }>{\ttfamily}l@{}}
%   romanbar.sty & tex/latex/romanbar/romanbar.sty\\
%   romanbar.pdf & doc/latex/romanbar/romanbar.pdf\\
%   romanbar-example.tex & doc/latex/romanbar/romanbar-example.tex\\
%   romanbar-example.pdf & doc/latex/romanbar/romanbar-example.pdf\\
%   romanbar.dtx & source/latex/romanbar/romanbar.dtx\\
% \end{tabular}^^A
% }^^A
% \sbox0{\t}^^A
% \ifdim\wd0>\linewidth
%   \begingroup
%     \advance\linewidth by\leftmargin
%     \advance\linewidth by\rightmargin
%   \edef\x{\endgroup
%     \def\noexpand\lw{\the\linewidth}^^A
%   }\x
%   \def\lwbox{^^A
%     \leavevmode
%     \hbox to \linewidth{^^A
%       \kern-\leftmargin\relax
%       \hss
%       \usebox0
%       \hss
%       \kern-\rightmargin\relax
%     }^^A
%   }^^A
%   \ifdim\wd0>\lw
%     \sbox0{\small\t}^^A
%     \ifdim\wd0>\linewidth
%       \ifdim\wd0>\lw
%         \sbox0{\footnotesize\t}^^A
%         \ifdim\wd0>\linewidth
%           \ifdim\wd0>\lw
%             \sbox0{\scriptsize\t}^^A
%             \ifdim\wd0>\linewidth
%               \ifdim\wd0>\lw
%                 \sbox0{\tiny\t}^^A
%                 \ifdim\wd0>\linewidth
%                   \lwbox
%                 \else
%                   \usebox0
%                 \fi
%               \else
%                 \lwbox
%               \fi
%             \else
%               \usebox0
%             \fi
%           \else
%             \lwbox
%           \fi
%         \else
%           \usebox0
%         \fi
%       \else
%         \lwbox
%       \fi
%     \else
%       \usebox0
%     \fi
%   \else
%     \lwbox
%   \fi
% \else
%   \usebox0
% \fi
% \end{quote}
% If you have a \xfile{docstrip.cfg} that configures and enables \docstrip{}'s
% \xfile{TDS} installing feature, then some files can already be in the right
% place, see the documentation of \docstrip{}.
%
% \subsection{Refresh file name databases}
%
% If your \TeX{}~distribution (\teTeX{}, \mikTeX{},\dots{}) relies on file name
% databases, you must refresh these. For example, \teTeX{} users run
% \verb|texhash| or \verb|mktexlsr|.
%
% \subsection{Some details for the interested}
%
% \paragraph{Unpacking with \LaTeX{}.}
% The \xfile{.dtx} chooses its action depending on the format:
% \begin{description}
% \item[\plainTeX:] Run \docstrip{} and extract the files.
% \item[\LaTeX:] Generate the documentation.
% \end{description}
% If you insist on using \LaTeX{} for \docstrip{} (really,
% \docstrip{} does not need \LaTeX{}), then inform the autodetect routine
% about your intention:
% \begin{quote}
%   \verb|latex \let\install=y\input{romanbar.dtx}|
% \end{quote}
% Do not forget to quote the argument according to the demands
% of your shell.
%
% \paragraph{Generating the documentation.\label{GenDoc}}
% You can use both the \xfile{.dtx} or the \xfile{.drv} to generate
% the documentation. The process can be configured by a
% configuration file \xfile{ltxdoc.cfg}. For instance, put the following
% line into this file, if you want to have A4 as paper format:
% \begin{quote}
%   \verb|\PassOptionsToClass{a4paper}{article}|
% \end{quote}
%
% \noindent An example follows how to generate the
% documentation with \pdfLaTeX{}:
%
% \begin{quote}
%\begin{verbatim}
%pdflatex romanbar.dtx
%makeindex -s gind.ist romanbar.idx
%pdflatex romanbar.dtx
%makeindex -s gind.ist romanbar.idx
%pdflatex romanbar.dtx
%\end{verbatim}
% \end{quote}
%
% \subsection{Compiling the example}
%
% The example file, \textsf{romanbar-example.tex}, can be compiled via\\
% |(pdf)(la)tex romanbar-example.tex|.
%
% \section{Acknowledgements}
%
% I would like to thank \textsc{Heiko Oberdiek} for providing
% the \xpackage{hyperref} as well as a~lot~(!) of other useful packages
% (from which I also got everything I know about creating a file in \xext{dtx}
% format, ok, say it: copying),
% Prof.~\textsc{Enrico Gregorio} (egreg) for his answer
% \url{http://tex.stackexchange.com/questions/24065/roman-numerals-formatting/24084#24084},
% the \Newsgroup{comp.text.tex} and \Newsgroup{de.comp.text.tex}
% newsgroups as well as \url{http://tex.stackexchange.com}
% for their help in all things \TeX{}.
%
% \pagebreak
%
% \phantomsection
% \begin{History}\label{History}
%   \begin{Version}{2011/07/25 v1.0a}
%     \item First code by Prof.~\textsc{Enrico Gregorio} (egreg) at
%             \url{http://tex.stackexchange.com/questions/24065/roman-numerals-formatting/24084#24084}
%   \end{Version}
%   \begin{Version}{2011/07/26 v1.0b}
%     \item Packed into a \xext{sty} file.
%     \item Removed the requirement of \eTeX{} for the package.
%   \end{Version}
%   \begin{Version}{2011/08/25 v1.0c}
%     \item Renamed \xpackage{romanbar} package.
%     \item Minor details.
%   \end{Version}
%   \begin{Version}{2011/09/16 v1.0d}
%     \item Made commands robust.
%     \item Minor details.
%   \end{Version}
%   \begin{Version}{2011/12/16 v1.0e}
%     \item Additionally accepts now Arabic numbers and converts them to Roman ones with bars.
%     \item Also non-positive numbers are now accepted.
%   \end{Version}
%   \begin{Version}{2012/01/01 v1.0f}
%     \item Added |\romannum| and |\Romannum| to convert Arabic to Roman numbers (without bars).
%     \item |\r@iseL|.
%     \item Created a \xfile{dtx}, example, and README.
%     \item Upload to \CTAN{}.
%   \end{Version}
% \end{History}
%
% \bigskip
%
% When you find a mistake or have a suggestion for an improvement of this package,
% please send an e-mail to the maintainer, thanks! (Please see BUG REPORTS in the README.)
%
% \pagebreak
%
% \PrintIndex
%
% \Finale
\endinput|
% \end{quote}
% Do not forget to quote the argument according to the demands
% of your shell.
%
% \paragraph{Generating the documentation.\label{GenDoc}}
% You can use both the \xfile{.dtx} or the \xfile{.drv} to generate
% the documentation. The process can be configured by a
% configuration file \xfile{ltxdoc.cfg}. For instance, put the following
% line into this file, if you want to have A4 as paper format:
% \begin{quote}
%   \verb|\PassOptionsToClass{a4paper}{article}|
% \end{quote}
%
% \noindent An example follows how to generate the
% documentation with \pdfLaTeX{}:
%
% \begin{quote}
%\begin{verbatim}
%pdflatex romanbar.dtx
%makeindex -s gind.ist romanbar.idx
%pdflatex romanbar.dtx
%makeindex -s gind.ist romanbar.idx
%pdflatex romanbar.dtx
%\end{verbatim}
% \end{quote}
%
% \subsection{Compiling the example}
%
% The example file, \textsf{romanbar-example.tex}, can be compiled via\\
% |(pdf)(la)tex romanbar-example.tex|.
%
% \section{Acknowledgements}
%
% I would like to thank \textsc{Heiko Oberdiek} for providing
% the \xpackage{hyperref} as well as a~lot~(!) of other useful packages
% (from which I also got everything I know about creating a file in \xext{dtx}
% format, ok, say it: copying),
% Prof.~\textsc{Enrico Gregorio} (egreg) for his answer
% \url{http://tex.stackexchange.com/questions/24065/roman-numerals-formatting/24084#24084},
% the \Newsgroup{comp.text.tex} and \Newsgroup{de.comp.text.tex}
% newsgroups as well as \url{http://tex.stackexchange.com}
% for their help in all things \TeX{}.
%
% \pagebreak
%
% \phantomsection
% \begin{History}\label{History}
%   \begin{Version}{2011/07/25 v1.0a}
%     \item First code by Prof.~\textsc{Enrico Gregorio} (egreg) at
%             \url{http://tex.stackexchange.com/questions/24065/roman-numerals-formatting/24084#24084}
%   \end{Version}
%   \begin{Version}{2011/07/26 v1.0b}
%     \item Packed into a \xext{sty} file.
%     \item Removed the requirement of \eTeX{} for the package.
%   \end{Version}
%   \begin{Version}{2011/08/25 v1.0c}
%     \item Renamed \xpackage{romanbar} package.
%     \item Minor details.
%   \end{Version}
%   \begin{Version}{2011/09/16 v1.0d}
%     \item Made commands robust.
%     \item Minor details.
%   \end{Version}
%   \begin{Version}{2011/12/16 v1.0e}
%     \item Additionally accepts now Arabic numbers and converts them to Roman ones with bars.
%     \item Also non-positive numbers are now accepted.
%   \end{Version}
%   \begin{Version}{2012/01/01 v1.0f}
%     \item Added |\romannum| and |\Romannum| to convert Arabic to Roman numbers (without bars).
%     \item |\r@iseL|.
%     \item Created a \xfile{dtx}, example, and README.
%     \item Upload to \CTAN{}.
%   \end{Version}
% \end{History}
%
% \bigskip
%
% When you find a mistake or have a suggestion for an improvement of this package,
% please send an e-mail to the maintainer, thanks! (Please see BUG REPORTS in the README.)
%
% \pagebreak
%
% \PrintIndex
%
% \Finale
\endinput
%        (quote the arguments according to the demands of your shell)
%
% Documentation:
%    (a) If romanbar.drv is present:
%           (pdf)latex romanbar.drv
%           makeindex -s gind.ist romanbar.idx
%           (pdf)latex romanbar.drv
%           makeindex -s gind.ist romanbar.idx
%           (pdf)latex romanbar.drv
%    (b) Without romanbar.drv:
%           (pdf)latex romanbar.dtx
%           makeindex -s gind.ist romanbar.idx
%           (pdf)latex romanbar.dtx
%           makeindex -s gind.ist romanbar.idx
%           (pdf)latex romanbar.dtx
%
%    The class ltxdoc loads the configuration file ltxdoc.cfg
%    if available. Here you can specify further options, e.g.
%    use DIN A4 as paper format:
%       \PassOptionsToClass{a4paper}{article}
%
% Installation:
%    TDS:tex/latex/romanbar/romanbar.sty
%    TDS:doc/latex/romanbar/romanbar.pdf
%    TDS:doc/latex/romanbar/romanbar-example.tex
%    TDS:doc/latex/romanbar/romanbar-example.pdf
%    TDS:source/latex/romanbar/romanbar.dtx
%
%<*ignore>
\begingroup
  \catcode123=1 %
  \catcode125=2 %
  \def\x{LaTeX2e}%
\expandafter\endgroup
\ifcase 0\ifx\install y1\fi\expandafter
         \ifx\csname processbatchFile\endcsname\relax\else1\fi
         \ifx\fmtname\x\else 1\fi\relax
\else\csname fi\endcsname
%</ignore>
%<*install>
\input docstrip.tex
\Msg{********************************************************************}
\Msg{* Installation                                                     *}
\Msg{* Package: romanbar 2012/01/01 v1.0f Roman numbers with bars (HMM) *}
\Msg{********************************************************************}

\keepsilent
\askforoverwritefalse

\let\MetaPrefix\relax
\preamble

This is a generated file.

Project: romanbar
Version: 2012/01/01 v1.0f

Copyright (C) 2011 by
    H.-Martin M"unch <Martin dot Muench at Uni-Bonn dot de>

The usual disclaimer applies:
If it doesn't work right that's your problem.
(Nevertheless, send an e-mail to the maintainer
 when you find an error in this package.)

This work may be distributed and/or modified under the
conditions of the LaTeX Project Public License, either
version 1.3c of this license or (at your option) any later
version. This version of this license is in
   http://www.latex-project.org/lppl/lppl-1-3c.txt
and the latest version of this license is in
   http://www.latex-project.org/lppl.txt
and version 1.3c or later is part of all distributions of
LaTeX version 2005/12/01 or later.

This work has the LPPL maintenance status "maintained".

The Current Maintainer of this work is H.-Martin Muench.

This work consists of the main source file romanbar.dtx,
the README, and the derived files
   romanbar.sty, romanbar.pdf,
   romanbar.ins, romanbar.drv,
   romanbar-example.tex, romanbar-example.pdf.

\endpreamble
\let\MetaPrefix\DoubleperCent

\generate{%
  \file{romanbar.ins}{\from{romanbar.dtx}{install}}%
  \file{romanbar.drv}{\from{romanbar.dtx}{driver}}%
  \usedir{tex/latex/romanbar}%
  \file{romanbar.sty}{\from{romanbar.dtx}{package}}%
  \usedir{doc/latex/romanbar}%
  \file{romanbar-example.tex}{\from{romanbar.dtx}{example}}%
}

\catcode32=13\relax% active space
\let =\space%
\Msg{************************************************************************}
\Msg{*}
\Msg{* To finish the installation you have to move the following}
\Msg{* file into a directory searched by TeX:}
\Msg{*}
\Msg{*     romanbar.sty}
\Msg{*}
\Msg{* To produce the documentation run the file `romanbar.drv'}
\Msg{* through (pdf)LaTeX, e.g.}
\Msg{*  pdflatex romanbar.drv}
\Msg{*  makeindex -s gind.ist romanbar.idx}
\Msg{*  pdflatex romanbar.drv}
\Msg{*  makeindex -s gind.ist romanbar.idx}
\Msg{*  pdflatex romanbar.drv}
\Msg{*}
\Msg{* At least three runs are necessary e.g. to get the}
\Msg{*  references right!}
\Msg{*}
\Msg{* Happy TeXing!}
\Msg{*}
\Msg{************************************************************************}

\endbatchfile
%</install>
%<*ignore>
\fi
%</ignore>
%
% \section{The documentation driver file}
%
% The next bit of code contains the documentation driver file for
% \TeX{}, i.\,e., the file that will produce the documentation you
% are currently reading. It will be extracted from this file by the
% \texttt{docstrip} programme. That is, run \LaTeX{} on \texttt{docstrip}
% and specify the \texttt{driver} option when \texttt{docstrip}
% asks for options.
%
%    \begin{macrocode}
%<*driver>
\NeedsTeXFormat{LaTeX2e}[2011/06/27]
\ProvidesFile{romanbar.drv}%
  [2012/01/01 v1.0f Roman numbers with bars (HMM)]
\documentclass{ltxdoc}[2007/11/11]% v2.0u
\usepackage{holtxdoc}[2011/02/04]%  v0.21
%% romanbar may work with earlier versions of LaTeX2e and those
%% class and package, but this was not tested.
%% Please consider updating your LaTeX, class, and package
%% to the most recent version (if they are not already the most
%% recent version).
\hypersetup{%
 pdfsubject={Writing Roman numbers with bars (HMM)},%
 pdfkeywords={LaTeX, romanbar, roman, Roman, bars, H.-Martin Muench},%
 pdfencoding=auto,%
 pdflang={en},%
 breaklinks=true,%
 linktoc=all,%
 pdfstartview=FitH,%
 pdfpagelayout=OneColumn,%
 bookmarksnumbered=true,%
 bookmarksopen=true,%
 bookmarksopenlevel=2,%
 pdfmenubar=true,%
 pdftoolbar=true,%
 pdfwindowui=true,%
 pdfnewwindow=true%
}
\CodelineIndex
\hyphenation{docu-ment ana-lysed}
\gdef\unit#1{\mathord{\thinspace\mathrm{#1}}}%
\makeatletter
\@ifundefined{eTeX}{\gdef\eTeX{$\m@th \varepsilon $-\TeX }}{% else \relax
}
\makeatother
\begin{document}
  \DocInput{romanbar.dtx}%
\end{document}
%</driver>
%    \end{macrocode}
%
% \fi
%
% \CheckSum{179}
%
% \CharacterTable
%  {Upper-case    \A\B\C\D\E\F\G\H\I\J\K\L\M\N\O\P\Q\R\S\T\U\V\W\X\Y\Z
%   Lower-case    \a\b\c\d\e\f\g\h\i\j\k\l\m\n\o\p\q\r\s\t\u\v\w\x\y\z
%   Digits        \0\1\2\3\4\5\6\7\8\9
%   Exclamation   \!     Double quote  \"     Hash (number) \#
%   Dollar        \$     Percent       \%     Ampersand     \&
%   Acute accent  \'     Left paren    \(     Right paren   \)
%   Asterisk      \*     Plus          \+     Comma         \,
%   Minus         \-     Point         \.     Solidus       \/
%   Colon         \:     Semicolon     \;     Less than     \<
%   Equals        \=     Greater than  \>     Question mark \?
%   Commercial at \@     Left bracket  \[     Backslash     \\
%   Right bracket \]     Circumflex    \^     Underscore    \_
%   Grave accent  \`     Left brace    \{     Vertical bar  \|
%   Right brace   \}     Tilde         \~}
%
% \GetFileInfo{romanbar.drv}
%
% \begingroup
%   \def\x{\#,\$,\^,\_,\~,\ ,\&,\{,\},\%}%
%   \makeatletter
%   \@onelevel@sanitize\x
% \expandafter\endgroup
% \expandafter\DoNotIndex\expandafter{\x}
% \expandafter\DoNotIndex\expandafter{\string\ }
% \begingroup
%   \makeatletter
%     \lccode`9=32\relax
%     \lowercase{%^^A
%       \edef\x{\noexpand\DoNotIndex{\@backslashchar9}}%^^A
%     }%^^A
%   \expandafter\endgroup\x
%
% \DoNotIndex{\\,\,}
% \DoNotIndex{\documentclass,\usepackage,\ProvidesPackage,\begin,\end}
% \DoNotIndex{\MessageBreak}
% \DoNotIndex{\NeedsTeXFormat,\DoNotIndex,\verb}
% \DoNotIndex{\def,\edef,\gdef,\global}
% \DoNotIndex{\ifx,\listfiles,\mathord,\mathrm}
% \DoNotIndex{\bigskip,\space,\thinspace,\ldots}
% \DoNotIndex{\indent,\noindent,\newline,\linebreak,\pagebreak,\newpage}
% \DoNotIndex{\textbf,\textit,\textsf,\texttt,\textquotedblleft,\textquotedblright}
% \DoNotIndex{\plainTeX,\TeX,\LaTeX,\pdfLaTeX}
% \DoNotIndex{\item}
%
% \title{The \xpackage{romanbar} package}
% \date{2012/01/01 v1.0f}
% \author{H.-Martin M\"{u}nch\\\xemail{Martin.Muench at Uni-Bonn.de}}
%
% \maketitle
%
% \begin{abstract}
%  This \LaTeX{} package allows to write Roman numbers
%  (or any other text) with bars. (Additionally, commands for converting Arabic numbers
%  into Roman ones are provided and an |\ifnumeric| test function.)
% \end{abstract}
%
% \bigskip
%
% \noindent Disclaimer for web links: The author is not responsible for any contents
% referred to in this work unless he has full knowledge of illegal contents.
% If any damage occurs by the use of information presented there, only the
% author of the respective pages might be liable, not the one who has referred
% to these pages.
%
% \bigskip
%
% \noindent {\color{green} Save per page about $200\unit{ml}$ water,
% $2\unit{g}$ CO$_{2}$ and $2\unit{g}$ wood:\\
% Therefore please print only if this is really necessary.}
%
% \pagebreak
%
% \tableofcontents
%
% \pagebreak
%
% \section{Introduction}
%
% \indent This \LaTeX{} package allows to write Roman numbers
% (or any other text) with bars. (Additionally, command for converting Arabic numbers
% into Roman ones are provided and an |\ifnumeric| test function.)
%
% \bigskip
%
% \section{Usage}
%
% \indent Just load the package placing
% \begin{quote}
%   |\usepackage{romanbar}|
% \end{quote}
% \noindent in the preamble of your \LaTeXe{} source file.\\
% |\Romanbar{...}| then produces a Roman number with bars (please see the example file).
% |\ifnumeric{test}{true}{false}| tests for |test| being numeric.
% |\romannum{...}| turns an Arabic number into a lowercase Roman one, and
% |\Romannum{...}| turns an Arabic number into an uppercase Roman one.
%
% \subsection*{Options}
% \DescribeMacro{options}
% \indent The \xpackage{romanbar} package takes no options.
%
% \section{Alternatives\label{sec:Alternatives}}
%
% \begin{description}
% \item[-] the original code written by Prof. Enrico Gregorio\\
%   (\url{http://profs.sci.univr.it/~gregorio/}),\\
%   \url{http://tex.stackexchange.com/questions/24065/roman-numerals-formatting/24084#24084}
%\begin{verbatim}
%\def\barroman#1{\sbox0{#1}\dimen0=\dimexpr\wd0+1pt\relax
%  \makebox[\dimen0]{\rlap{\vrule width\dimen0 height 0.06ex depth 0.06ex}%
%    \rlap{\vrule width\dimen0 height\dimexpr\ht0+0.03ex\relax
%            depth\dimexpr-\ht0+0.09ex\relax}%
%    \kern.5pt#1\kern.5pt}}
%
%\barroman{I} \barroman{XI}
%\end{verbatim}
%
% \item[-]  Some fonts provide single characters for Roman numerals,
%   cf.~e.\,g.
% \url{http://tex.stackexchange.com/questions/38695/using-unicode-roman-numerals-in-xetex}.
% \end{description}
%
% \noindent (You programmed or found another alternative,
%  which is available at \CTAN{}?\\
%  OK, send an e-mail to me with the name, location at \CTAN{},
%  and a short notice, and I will probably include it in
%  the list above.)
%
% \newpage
%
% \section{Example}
%
%    \begin{macrocode}
%<*example>
\documentclass[british]{article}[2007/10/19]% v1.4h
%%%%%%%%%%%%%%%%%%%%%%%%%%%%%%%%%%%%%%%%%%%%%%%%%%%%%%%%%%%%%%%%%%%%%
\usepackage[extension=pdf,%
 plainpages=false,%
 pdfpagelabels=true,%
 hyperindex=false,%
 pdflang={en},%
 pdftitle={romanbar package example},%
 pdfauthor={H.-Martin Muench},%
 pdfsubject={Example for the romanbar package},%
 pdfkeywords={LaTeX, romanbar, roman, Roman, bars, H.-Martin Muench},%
 pdfview={XYZ null null 1},%
 pdfstartview={XYZ null null 1},%
 pdfpagelayout=SinglePage]{hyperref}[2011/12/04]% v6.82m
\usepackage{romanbar}[2012/01/01]% v1.0f
\gdef\unit#1{\mathord{\thinspace\mathrm{#1}}}%
\listfiles
\begin{document}
\pagenumbering{arabic}
\section*{Example for romanbar}

This example demonstrates the use of package\newline
\textsf{romanbar}, v1.0f as of 2012/01/01 (HMM).\newline
There are no options to be used.\newline

\noindent For more details please see the documentation!\newline

\noindent Save per page about $200\unit{ml}$ water,
$2\unit{g}$ CO$_{2}$ and $2\unit{g}$ wood:\newline
Therefore please print only if this is really necessary.\newline

\noindent This package provides the command \verb|\Romanbar|
to print bars below and over the following:

\begin{description}
\item[-] Roman numbers: \verb|\Romanbar{MMXII}| prints \Romanbar{MMXII}

\item[-] Arabic numbers turned into upper-case Roman numbers:\newline
           \verb|\Romanbar{2012}| prints \Romanbar{2012}

\item[-] negative Arabic numbers turned into upper-case Roman numbers
           \newline
           (although historically there were no negative Roman numbers):
           \newline
           \verb|\Romanbar{-12}| prints \Romanbar{-12}

\item[-] zero Arabic number ($0$; although historically
           there was no Roman zero):\newline
           \verb|\Romanbar{0}| prints \Romanbar{0}

\item[-] some arbitrary text:
           \verb|\Romanbar{Caesar}| prints \Romanbar{Caesar}\newline
           (with descenders: \Romanbar{AgjpqyW})


\newcounter{example}
\setcounter{example}{21}

\item[-] some counter's value:
           \verb|\Romanbar{\theexample}| prints \Romanbar{\theexample}
           \newline
           (where the value of \texttt{example} is \theexample)

\item[-] Arabic numbers, without turning them into upper-case Roman
           numbers:\newline
           \verb|\Romanbar{\relax 2012}| prints \Romanbar{\relax  2012}
\end{description}

Special care was taken for "L" (50), e.\,g. in 555/DLV: \Romanbar{555}.\\


\verb|\romannum{...}| converts an Arabic number into a lower-case Roman one,
e.\,g. \verb|\romannum{2012}| prints \romannum{2012}, and
\verb|\Romannum{...}| converts an Arabic number into an upper-case Roman one,
e.\,g. \verb|\Romannum{2012}| prints \Romannum{2012}.

\end{document}
%</example>
%    \end{macrocode}
%
% \StopEventually{}
%
% \newpage
%
% \section{The implementation}
%
% We start off by checking that we are loading into \LaTeXe{} and
% announcing the name and version of this package.
%
%    \begin{macrocode}
%<*package>
%    \end{macrocode}
%
%    \begin{macrocode}
\NeedsTeXFormat{LaTeX2e}[2011/06/27]
\ProvidesPackage{romanbar}[2012/01/01 v1.0f
              Roman numbers with bars (HMM)]

%    \end{macrocode}
%
% A short description of the \xpackage{romanbar} package:
%
%    \begin{macrocode}
%% Allows to write Roman numbers (or any other text) with bars
%% and to convert Arabic numbers into Roman ones.
%    \end{macrocode}
%
% A last information for the user:
%
%    \begin{macrocode}
%% romanbar may work with earlier versions of LaTeX,
%% but this was not tested. Please consider updating
%% your LaTeX to the most recent version
%% (if it is not already the most recent version).

%    \end{macrocode}
%
% \indent See subsection~\ref{ss:Downloads} about how to get it.\\
%
% There are no options to be processed.
%
% A command to gobble a possible minus sign is needed and
% |\gobbleminus| from \url{http://www.tex.ac.uk/cgi-bin/texfaq2html?label=isitanum}
% is used.
%
%    \begin{macrocode}
\def\gobbleminus#1{\ifx-#1\else#1\fi}
%% from http://www.tex.ac.uk/cgi-bin/texfaq2html?label=isitanum

%    \end{macrocode}
%
% We want to test whether the argument passed to |\Romanbar| is numeric
% and therefore define |\ifnumeric{test}{true}{false}|,
% where |test| is to be analysed, |true| is the code to be executed,
% if |test| is numeric, and |false| is the code to be executed,
% if |test| is not numeric. This is done similar to\\
% \url{http://tex.stackexchange.com/a/17119}.
%
%    \begin{macrocode}
\newcommand{\ifnumeric}[3]{%
%% similar to http://tex.stackexchange.com/a/17119
\sbox\z@{\@tempcnta=0\gobbleminus#1\relax}%
\ifdim\wd0>\z@\relax#3% is not numeric
\else#2% is numeric
\fi%
}

%    \end{macrocode}
%
% We define |\r@iseL| to raise any L (50), otherwise a good deal
% of the horizontal part of the letter would be swallowed by the lower bar
% under the \textquotedblleft{}number\textquotedblright{}.
%
%    \begin{macrocode}
\def\r@iseL#1{\ifx @#1% then terminate
\else%
\if L#1\raisebox{0.05ex}{L}\else #1\fi%
\expandafter\r@iseL%
\fi%
}

%    \end{macrocode}
%
% \pagebreak
%
% We define the |\Romanbar| command.
%
%    \begin{macrocode}
\DeclareRobustCommand{\Romanbar}[1]{%
\ifnumeric{#1}{% is numeric
%    \end{macrocode}
%
% If its parameter is numeric, the temporary counter |\@tempcnta| is set to the value.
%
%    \begin{macrocode}
\@tempcnta=#1\relax%
%    \end{macrocode}
%
% Normally there are no non-positive Roman numbers, therefore we do some trick
% similar to |\XRoman| from the \url{http://ctan.org/pkg/pageslts} package.
%
%    \begin{macrocode}
%% similar to \XRoman from the http://ctan.org/pkg/pageslts package
\ifnum\@tempcnta<1%
  \ifnum #1>0%
    \relax \Roman@bar{\Roman{#1}}%
  \else%
    \ifnum #1<0%
     -\Roman@bar{\expandafter\@slowromancap\romannumeral\number-#1@%
     }%
    \else%
      \Roman@bar{0}%
    \fi%
  \fi%
\else\Roman@bar{\@Roman\@tempcnta}%
\fi%
}{% is not numeric
\Roman@bar{#1}%
}%
}

%    \end{macrocode}
%
% Whether it is numeric or not, in the end |\Roman@bar| is applied to it.
%
%    \begin{macrocode}
\DeclareRobustCommand{\Roman@bar}[1]{% #1 is in Roman, i.e. MMXII
%% (or any other text, "Caesar" would work, too).
%% similar to code of  by Prof. Enrico Gregorio (egreg) at
%% http://tex.stackexchange.com/questions/24065/roman-numerals-formatting/24084#24084
\@bsphack%
\edef\romanbartmp{#1}%
%% height:
\sbox0{\textsf{\romanbartmp}}%
%% top line:
%% + top of top line:
\@tempdima=\ht0%
\advance\@tempdima+0.05ex%0.03
\dimen1=\@tempdima%
%% + bottom of top line:
\@tempdima=-\ht0%
\advance\@tempdima+0.05ex%0.07
\dimen2=\@tempdima%
%% width:
\sbox0{\textsf{\expandafter\r@iseL\romanbartmp @}}%
\@tempdima=\wd0%
\advance\@tempdima+1pt%
\dimen0=\@tempdima%
\@esphack%
\makebox[\dimen0]{%
\rlap{\vrule width\dimen0 height\dimen1 depth\dimen2}% top    line
\rlap{\vrule width\dimen0 height 0.06ex depth 0.03ex}% bottom line
\kern0.5pt\textsf{\expandafter\r@iseL\romanbartmp @}\kern0.5pt}%
}

%    \end{macrocode}
%
% While it is not needed for the bars, it is sometimes asked for a command
% to convert an Arabic number into a Roman one. |\roman{...}| can only be applied
% to a counter, i.\,e. |\roman{12}| does not work, but
%\begin{verbatim}
%\newcounter{examplecounter}
%\setcounter{examplecounter}{12}
%\roman{examplecounter}
%\end{verbatim}
% would be necessary. Further non-negative values would not work.
% |\romannum| accepts any number as argument. (If the argument is no number,
% an error message is given and the argument printed (executed)
%  without applying anything to it.)
%
%    \begin{macrocode}
\DeclareRobustCommand{\romannum}[1]{%
\ifnumeric{#1}{% is numeric
\@tempcnta=#1\relax%
%% similar to \xroman from the http://ctan.org/pkg/pageslts package
\ifnum\@tempcnta<1%
  \ifnum #1>0%
    \relax \roman{#1}%
  \else%
    \ifnum #1<0%
     -\romannumeral\number-#1@%
    \else%
      0%
    \fi%
  \fi%
\else\@roman\@tempcnta%
\fi%
}{% is not numeric
\PackageError{romanbar}{%
Argument of \string\romannum\space is not a number}{%
The command \string\romannum\space converts an Arabic number into a%
 lower-case Roman one,\MessageBreak%
but the used argument of \string\romannum\space is not an Arabic number%
 but\MessageBreak%
`#1`,\MessageBreak%
which will now be printed unchanged.}%
#1%
}%
}

%    \end{macrocode}
%
% The same for upper-case Roman numbers, |\Romannum| instead of |\Roman|:
%
%    \begin{macrocode}
\DeclareRobustCommand{\Romannum}[1]{%
\ifnumeric{#1}{% is numeric
\@tempcnta=#1\relax%
%% similar to \XRoman from the http://ctan.org/pkg/pageslts package
\ifnum\@tempcnta<1%
  \ifnum #1>0%
    \relax \Roman{#1}%
  \else%
    \ifnum #1<0%
     -\expandafter\@slowromancap\romannumeral\number-#1@%
    \else%
      0%
    \fi%
  \fi%
\else\@Roman\@tempcnta%
\fi%
}{% is not numeric
\PackageError{romanbar}{%
Argument of \string\Romannum\space is not a number}{%
The command \string\Romannum\space converts an Arabic number into an%
 upper-case Roman one,\MessageBreak%
but the used argument of \string\Romannum\space is not an Arabic number%
 but\MessageBreak%
`#1`.}%
#1%
}%
}

%    \end{macrocode}
%
%    \begin{macrocode}
%</package>
%    \end{macrocode}
%
% \newpage
%
% \section{Installation}
%
% \subsection{Downloads\label{ss:Downloads}}
%
% Everything should be available on \CTAN{}, \url{http://www.ctan.org/tex-archive/},
% but may need additional packages themselves.\\
%
% \DescribeMacro{romanbar.dtx}
% For unpacking the |romanbar.dtx| file and constructing the documentation
% it is required:
% \begin{description}
% \item[-] \TeX{} Format \LaTeXe{}, \url{http://www.CTAN.org/}
%
% \item[-] document class \xpackage{ltxdoc}, 2007/11/11, v2.0u,
%           \url{http://ctan.org/pkg/ltxdoc}
%
% \item[-] package \xpackage{holtxdoc}, 2011/02/04, v0.21,
%           \url{http://ctan.org/pkg/holtxdoc}
%
% \item[-] package \xpackage{hypdoc}, 2010/03/26, v1.9,
%           \url{http://ctan.org/pkg/hypdoc}
% \end{description}
%
% \DescribeMacro{romanbar.sty}
% The \texttt{romanbar.sty} for \LaTeXe{} (i.\,e.~each document using
% the \xpackage{romanbar} package) requires:
% \begin{description}
% \item[-] \TeX Format \LaTeXe{}, \url{http://www.CTAN.org/}
% \end{description}
%
% \DescribeMacro{romanbar-example.tex}
% The \texttt{romanbar-example.tex} requires the same files as all
% documents using the \xpackage{romanbar} package and additionally:
% \begin{description}
% \item[-] class \xpackage{article}, 2007/10/19, v1.4h, from \xpackage{classes.dtx}:\\
%           \CTAN{macros/latex/base/classes.dtx}
%
% \item[-] package \xpackage{romanbar}, 2012/01/01, v1.0f,\\
%           %%\url{http://ctan.org/pkg/romanbar}\\
%           \CTAN{macros/latex/contrib/romanbar/}
%   (Well, it is the example file for this package, and because you are reading the
%    documentation for the \xpackage{romanbar} package, it can be assumed that you already
%    have some version of it -- is it the current one?)
% \end{description}
%
% \DescribeMacro{Alternative}
% As possible alternative package in section \ref{sec:Alternatives} there is listed
% \begin{description}
% \item[-] none.
% \end{description}
%
% \DescribeMacro{Oberdiek}
% \DescribeMacro{holtxdoc}
% All packages of \textsc{Heiko Oberdiek's} bundle `oberdiek'
% (especially \xpackage{holtxdoc}) are also available in a TDS compliant ZIP archive:\\
% \CTAN{install/macros/latex/contrib/oberdiek.tds.zip}.\\
% It is probably best to download and use this, because the packages in there
% are quite probably both recent and compatible among themselves.\\
%
% \DescribeMacro{hyperref}
% \noindent \xpackage{hyperref} is not included in that bundle and needs to be
% downloaded separately,\\
% \url{http://mirror.ctan.org/install/macros/latex/contrib/hyperref.tds.zip}.\\
%
% \DescribeMacro{M\"{u}nch}
% A hyperlinked list of my (other) packages can be found at
% \url{http://www.Uni-Bonn.de/~uzs5pv/LaTeX.html}.\\
%
% \pagebreak
%
% \subsection{Package, unpacking TDS}
% \paragraph{Package.} This package should become available on \CTAN{} soon:
% \begin{description}
% \item[\CTAN{macros/latex/contrib/romanbar/romanbar.dtx}]\hspace*{0.1cm} \\
%       The source file.
% \item[\CTAN{macros/latex/contrib/romanbar/romanbar.pdf}]\hspace*{0.1cm} \\
%       The documentation.
% \item[\CTAN{macros/latex/contrib/romanbar/romanbar-example.pdf}]\hspace*{0.1cm} \\
%       The compiled example file, as it should look like.
% \item[\CTAN{macros/latex/contrib/romanbar/README}]\hspace*{0.1cm} \\
%       The README file.
% \end{description}
% There is also a romanbar.tds.zip available:
% \begin{description}
% \item[\CTAN{install/macros/latex/contrib/romanbar.tds.zip}]\hspace*{0.1cm} \\
%       Everything in \xfile{TDS} compliant, compiled format.
% \end{description}
% which additionally contains\\
% \begin{tabular}{ll}
% romanbar.ins & The installation file.\\
% romanbar.drv & The driver to generate the documentation.\\
% romanbar.sty & The \xext{sty}le file.\\
% romanbar-example.tex & The example file.
% \end{tabular}
%
% \bigskip
%
% \noindent For required other packages, please see the preceding subsection.
%
% \paragraph{Unpacking.} The \xfile{.dtx} file is a self-extracting
% \docstrip{} archive. The files are extracted by running the
% \xext{.dtx} through \plainTeX{}:
% \begin{quote}
%   \verb|tex romanbar.dtx|
% \end{quote}
%
% About generating the documentation see paragraph~\ref{GenDoc} below.\\
%
% \paragraph{TDS.} Now the different files must be moved into
% the different directories in your installation TDS tree
% (also known as \xfile{texmf} tree):
% \begin{quote}
% \def\t{^^A
% \begin{tabular}{@{}>{\ttfamily}l@{ $\rightarrow$ }>{\ttfamily}l@{}}
%   romanbar.sty & tex/latex/romanbar/romanbar.sty\\
%   romanbar.pdf & doc/latex/romanbar/romanbar.pdf\\
%   romanbar-example.tex & doc/latex/romanbar/romanbar-example.tex\\
%   romanbar-example.pdf & doc/latex/romanbar/romanbar-example.pdf\\
%   romanbar.dtx & source/latex/romanbar/romanbar.dtx\\
% \end{tabular}^^A
% }^^A
% \sbox0{\t}^^A
% \ifdim\wd0>\linewidth
%   \begingroup
%     \advance\linewidth by\leftmargin
%     \advance\linewidth by\rightmargin
%   \edef\x{\endgroup
%     \def\noexpand\lw{\the\linewidth}^^A
%   }\x
%   \def\lwbox{^^A
%     \leavevmode
%     \hbox to \linewidth{^^A
%       \kern-\leftmargin\relax
%       \hss
%       \usebox0
%       \hss
%       \kern-\rightmargin\relax
%     }^^A
%   }^^A
%   \ifdim\wd0>\lw
%     \sbox0{\small\t}^^A
%     \ifdim\wd0>\linewidth
%       \ifdim\wd0>\lw
%         \sbox0{\footnotesize\t}^^A
%         \ifdim\wd0>\linewidth
%           \ifdim\wd0>\lw
%             \sbox0{\scriptsize\t}^^A
%             \ifdim\wd0>\linewidth
%               \ifdim\wd0>\lw
%                 \sbox0{\tiny\t}^^A
%                 \ifdim\wd0>\linewidth
%                   \lwbox
%                 \else
%                   \usebox0
%                 \fi
%               \else
%                 \lwbox
%               \fi
%             \else
%               \usebox0
%             \fi
%           \else
%             \lwbox
%           \fi
%         \else
%           \usebox0
%         \fi
%       \else
%         \lwbox
%       \fi
%     \else
%       \usebox0
%     \fi
%   \else
%     \lwbox
%   \fi
% \else
%   \usebox0
% \fi
% \end{quote}
% If you have a \xfile{docstrip.cfg} that configures and enables \docstrip{}'s
% \xfile{TDS} installing feature, then some files can already be in the right
% place, see the documentation of \docstrip{}.
%
% \subsection{Refresh file name databases}
%
% If your \TeX{}~distribution (\teTeX{}, \mikTeX{},\dots{}) relies on file name
% databases, you must refresh these. For example, \teTeX{} users run
% \verb|texhash| or \verb|mktexlsr|.
%
% \subsection{Some details for the interested}
%
% \paragraph{Unpacking with \LaTeX{}.}
% The \xfile{.dtx} chooses its action depending on the format:
% \begin{description}
% \item[\plainTeX:] Run \docstrip{} and extract the files.
% \item[\LaTeX:] Generate the documentation.
% \end{description}
% If you insist on using \LaTeX{} for \docstrip{} (really,
% \docstrip{} does not need \LaTeX{}), then inform the autodetect routine
% about your intention:
% \begin{quote}
%   \verb|latex \let\install=y% \iffalse meta-comment
%
% File: romanbar.dtx
% Version: 2012/01/01 v1.0f
%
% Copyright (C) 2011 by
%    H.-Martin M"unch <Martin dot Muench at Uni-Bonn dot de>
%
% The original code for the bars was written by Prof. Enrico Gregorio
% (http://profs.sci.univr.it/~gregorio/), and published at
% http://tex.stackexchange.com/questions/24065/roman-numerals-formatting/24084#24084
% I (H.-Martin Münch) submitted this package to CTAN
% (after diverse updates of the code and writing a documentation).
%
% This work may be distributed and/or modified under the
% conditions of the LaTeX Project Public License, either
% version 1.3c of this license or (at your option) any later
% version. This version of this license is in
%    http://www.latex-project.org/lppl/lppl-1-3c.txt
% and the latest version of this license is in
%    http://www.latex-project.org/lppl.txt
% and version 1.3c or later is part of all distributions of
% LaTeX version 2005/12/01 or later.
%
% This work has the LPPL maintenance status "maintained".
%
% The Current Maintainer of this work is H.-Martin Muench.
%
% This work consists of the main source file romanbar.dtx,
% the README, and the derived files
%    romanbar.sty, romanbar.pdf,
%    romanbar.ins, romanbar.drv,
%    romanbar-example.tex, romanbar-example.pdf.
%
% Distribution:
%    CTAN:macros/latex/contrib/romanbar/romanbar.dtx
%    CTAN:macros/latex/contrib/romanbar/romanbar.pdf
%    CTAN:install/macros/latex/contrib/romanbar.tds.zip
%
% Cf. http://ctan.org/pkg/romanbar and
% http://www.tex.ac.uk/tex-archive/help/Catalogue/entries/romanbar.html
% (It might take some time for the entries to be created.)
%
% Unpacking:
%    (a) If romanbar.ins is present:
%           tex romanbar.ins
%    (b) Without romanbar.ins:
%           tex romanbar.dtx
%    (c) If you insist on using LaTeX
%           latex \let\install=y% \iffalse meta-comment
%
% File: romanbar.dtx
% Version: 2012/01/01 v1.0f
%
% Copyright (C) 2011 by
%    H.-Martin M"unch <Martin dot Muench at Uni-Bonn dot de>
%
% The original code for the bars was written by Prof. Enrico Gregorio
% (http://profs.sci.univr.it/~gregorio/), and published at
% http://tex.stackexchange.com/questions/24065/roman-numerals-formatting/24084#24084
% I (H.-Martin Münch) submitted this package to CTAN
% (after diverse updates of the code and writing a documentation).
%
% This work may be distributed and/or modified under the
% conditions of the LaTeX Project Public License, either
% version 1.3c of this license or (at your option) any later
% version. This version of this license is in
%    http://www.latex-project.org/lppl/lppl-1-3c.txt
% and the latest version of this license is in
%    http://www.latex-project.org/lppl.txt
% and version 1.3c or later is part of all distributions of
% LaTeX version 2005/12/01 or later.
%
% This work has the LPPL maintenance status "maintained".
%
% The Current Maintainer of this work is H.-Martin Muench.
%
% This work consists of the main source file romanbar.dtx,
% the README, and the derived files
%    romanbar.sty, romanbar.pdf,
%    romanbar.ins, romanbar.drv,
%    romanbar-example.tex, romanbar-example.pdf.
%
% Distribution:
%    CTAN:macros/latex/contrib/romanbar/romanbar.dtx
%    CTAN:macros/latex/contrib/romanbar/romanbar.pdf
%    CTAN:install/macros/latex/contrib/romanbar.tds.zip
%
% Cf. http://ctan.org/pkg/romanbar and
% http://www.tex.ac.uk/tex-archive/help/Catalogue/entries/romanbar.html
% (It might take some time for the entries to be created.)
%
% Unpacking:
%    (a) If romanbar.ins is present:
%           tex romanbar.ins
%    (b) Without romanbar.ins:
%           tex romanbar.dtx
%    (c) If you insist on using LaTeX
%           latex \let\install=y\input{romanbar.dtx}
%        (quote the arguments according to the demands of your shell)
%
% Documentation:
%    (a) If romanbar.drv is present:
%           (pdf)latex romanbar.drv
%           makeindex -s gind.ist romanbar.idx
%           (pdf)latex romanbar.drv
%           makeindex -s gind.ist romanbar.idx
%           (pdf)latex romanbar.drv
%    (b) Without romanbar.drv:
%           (pdf)latex romanbar.dtx
%           makeindex -s gind.ist romanbar.idx
%           (pdf)latex romanbar.dtx
%           makeindex -s gind.ist romanbar.idx
%           (pdf)latex romanbar.dtx
%
%    The class ltxdoc loads the configuration file ltxdoc.cfg
%    if available. Here you can specify further options, e.g.
%    use DIN A4 as paper format:
%       \PassOptionsToClass{a4paper}{article}
%
% Installation:
%    TDS:tex/latex/romanbar/romanbar.sty
%    TDS:doc/latex/romanbar/romanbar.pdf
%    TDS:doc/latex/romanbar/romanbar-example.tex
%    TDS:doc/latex/romanbar/romanbar-example.pdf
%    TDS:source/latex/romanbar/romanbar.dtx
%
%<*ignore>
\begingroup
  \catcode123=1 %
  \catcode125=2 %
  \def\x{LaTeX2e}%
\expandafter\endgroup
\ifcase 0\ifx\install y1\fi\expandafter
         \ifx\csname processbatchFile\endcsname\relax\else1\fi
         \ifx\fmtname\x\else 1\fi\relax
\else\csname fi\endcsname
%</ignore>
%<*install>
\input docstrip.tex
\Msg{********************************************************************}
\Msg{* Installation                                                     *}
\Msg{* Package: romanbar 2012/01/01 v1.0f Roman numbers with bars (HMM) *}
\Msg{********************************************************************}

\keepsilent
\askforoverwritefalse

\let\MetaPrefix\relax
\preamble

This is a generated file.

Project: romanbar
Version: 2012/01/01 v1.0f

Copyright (C) 2011 by
    H.-Martin M"unch <Martin dot Muench at Uni-Bonn dot de>

The usual disclaimer applies:
If it doesn't work right that's your problem.
(Nevertheless, send an e-mail to the maintainer
 when you find an error in this package.)

This work may be distributed and/or modified under the
conditions of the LaTeX Project Public License, either
version 1.3c of this license or (at your option) any later
version. This version of this license is in
   http://www.latex-project.org/lppl/lppl-1-3c.txt
and the latest version of this license is in
   http://www.latex-project.org/lppl.txt
and version 1.3c or later is part of all distributions of
LaTeX version 2005/12/01 or later.

This work has the LPPL maintenance status "maintained".

The Current Maintainer of this work is H.-Martin Muench.

This work consists of the main source file romanbar.dtx,
the README, and the derived files
   romanbar.sty, romanbar.pdf,
   romanbar.ins, romanbar.drv,
   romanbar-example.tex, romanbar-example.pdf.

\endpreamble
\let\MetaPrefix\DoubleperCent

\generate{%
  \file{romanbar.ins}{\from{romanbar.dtx}{install}}%
  \file{romanbar.drv}{\from{romanbar.dtx}{driver}}%
  \usedir{tex/latex/romanbar}%
  \file{romanbar.sty}{\from{romanbar.dtx}{package}}%
  \usedir{doc/latex/romanbar}%
  \file{romanbar-example.tex}{\from{romanbar.dtx}{example}}%
}

\catcode32=13\relax% active space
\let =\space%
\Msg{************************************************************************}
\Msg{*}
\Msg{* To finish the installation you have to move the following}
\Msg{* file into a directory searched by TeX:}
\Msg{*}
\Msg{*     romanbar.sty}
\Msg{*}
\Msg{* To produce the documentation run the file `romanbar.drv'}
\Msg{* through (pdf)LaTeX, e.g.}
\Msg{*  pdflatex romanbar.drv}
\Msg{*  makeindex -s gind.ist romanbar.idx}
\Msg{*  pdflatex romanbar.drv}
\Msg{*  makeindex -s gind.ist romanbar.idx}
\Msg{*  pdflatex romanbar.drv}
\Msg{*}
\Msg{* At least three runs are necessary e.g. to get the}
\Msg{*  references right!}
\Msg{*}
\Msg{* Happy TeXing!}
\Msg{*}
\Msg{************************************************************************}

\endbatchfile
%</install>
%<*ignore>
\fi
%</ignore>
%
% \section{The documentation driver file}
%
% The next bit of code contains the documentation driver file for
% \TeX{}, i.\,e., the file that will produce the documentation you
% are currently reading. It will be extracted from this file by the
% \texttt{docstrip} programme. That is, run \LaTeX{} on \texttt{docstrip}
% and specify the \texttt{driver} option when \texttt{docstrip}
% asks for options.
%
%    \begin{macrocode}
%<*driver>
\NeedsTeXFormat{LaTeX2e}[2011/06/27]
\ProvidesFile{romanbar.drv}%
  [2012/01/01 v1.0f Roman numbers with bars (HMM)]
\documentclass{ltxdoc}[2007/11/11]% v2.0u
\usepackage{holtxdoc}[2011/02/04]%  v0.21
%% romanbar may work with earlier versions of LaTeX2e and those
%% class and package, but this was not tested.
%% Please consider updating your LaTeX, class, and package
%% to the most recent version (if they are not already the most
%% recent version).
\hypersetup{%
 pdfsubject={Writing Roman numbers with bars (HMM)},%
 pdfkeywords={LaTeX, romanbar, roman, Roman, bars, H.-Martin Muench},%
 pdfencoding=auto,%
 pdflang={en},%
 breaklinks=true,%
 linktoc=all,%
 pdfstartview=FitH,%
 pdfpagelayout=OneColumn,%
 bookmarksnumbered=true,%
 bookmarksopen=true,%
 bookmarksopenlevel=2,%
 pdfmenubar=true,%
 pdftoolbar=true,%
 pdfwindowui=true,%
 pdfnewwindow=true%
}
\CodelineIndex
\hyphenation{docu-ment ana-lysed}
\gdef\unit#1{\mathord{\thinspace\mathrm{#1}}}%
\makeatletter
\@ifundefined{eTeX}{\gdef\eTeX{$\m@th \varepsilon $-\TeX }}{% else \relax
}
\makeatother
\begin{document}
  \DocInput{romanbar.dtx}%
\end{document}
%</driver>
%    \end{macrocode}
%
% \fi
%
% \CheckSum{179}
%
% \CharacterTable
%  {Upper-case    \A\B\C\D\E\F\G\H\I\J\K\L\M\N\O\P\Q\R\S\T\U\V\W\X\Y\Z
%   Lower-case    \a\b\c\d\e\f\g\h\i\j\k\l\m\n\o\p\q\r\s\t\u\v\w\x\y\z
%   Digits        \0\1\2\3\4\5\6\7\8\9
%   Exclamation   \!     Double quote  \"     Hash (number) \#
%   Dollar        \$     Percent       \%     Ampersand     \&
%   Acute accent  \'     Left paren    \(     Right paren   \)
%   Asterisk      \*     Plus          \+     Comma         \,
%   Minus         \-     Point         \.     Solidus       \/
%   Colon         \:     Semicolon     \;     Less than     \<
%   Equals        \=     Greater than  \>     Question mark \?
%   Commercial at \@     Left bracket  \[     Backslash     \\
%   Right bracket \]     Circumflex    \^     Underscore    \_
%   Grave accent  \`     Left brace    \{     Vertical bar  \|
%   Right brace   \}     Tilde         \~}
%
% \GetFileInfo{romanbar.drv}
%
% \begingroup
%   \def\x{\#,\$,\^,\_,\~,\ ,\&,\{,\},\%}%
%   \makeatletter
%   \@onelevel@sanitize\x
% \expandafter\endgroup
% \expandafter\DoNotIndex\expandafter{\x}
% \expandafter\DoNotIndex\expandafter{\string\ }
% \begingroup
%   \makeatletter
%     \lccode`9=32\relax
%     \lowercase{%^^A
%       \edef\x{\noexpand\DoNotIndex{\@backslashchar9}}%^^A
%     }%^^A
%   \expandafter\endgroup\x
%
% \DoNotIndex{\\,\,}
% \DoNotIndex{\documentclass,\usepackage,\ProvidesPackage,\begin,\end}
% \DoNotIndex{\MessageBreak}
% \DoNotIndex{\NeedsTeXFormat,\DoNotIndex,\verb}
% \DoNotIndex{\def,\edef,\gdef,\global}
% \DoNotIndex{\ifx,\listfiles,\mathord,\mathrm}
% \DoNotIndex{\bigskip,\space,\thinspace,\ldots}
% \DoNotIndex{\indent,\noindent,\newline,\linebreak,\pagebreak,\newpage}
% \DoNotIndex{\textbf,\textit,\textsf,\texttt,\textquotedblleft,\textquotedblright}
% \DoNotIndex{\plainTeX,\TeX,\LaTeX,\pdfLaTeX}
% \DoNotIndex{\item}
%
% \title{The \xpackage{romanbar} package}
% \date{2012/01/01 v1.0f}
% \author{H.-Martin M\"{u}nch\\\xemail{Martin.Muench at Uni-Bonn.de}}
%
% \maketitle
%
% \begin{abstract}
%  This \LaTeX{} package allows to write Roman numbers
%  (or any other text) with bars. (Additionally, commands for converting Arabic numbers
%  into Roman ones are provided and an |\ifnumeric| test function.)
% \end{abstract}
%
% \bigskip
%
% \noindent Disclaimer for web links: The author is not responsible for any contents
% referred to in this work unless he has full knowledge of illegal contents.
% If any damage occurs by the use of information presented there, only the
% author of the respective pages might be liable, not the one who has referred
% to these pages.
%
% \bigskip
%
% \noindent {\color{green} Save per page about $200\unit{ml}$ water,
% $2\unit{g}$ CO$_{2}$ and $2\unit{g}$ wood:\\
% Therefore please print only if this is really necessary.}
%
% \pagebreak
%
% \tableofcontents
%
% \pagebreak
%
% \section{Introduction}
%
% \indent This \LaTeX{} package allows to write Roman numbers
% (or any other text) with bars. (Additionally, command for converting Arabic numbers
% into Roman ones are provided and an |\ifnumeric| test function.)
%
% \bigskip
%
% \section{Usage}
%
% \indent Just load the package placing
% \begin{quote}
%   |\usepackage{romanbar}|
% \end{quote}
% \noindent in the preamble of your \LaTeXe{} source file.\\
% |\Romanbar{...}| then produces a Roman number with bars (please see the example file).
% |\ifnumeric{test}{true}{false}| tests for |test| being numeric.
% |\romannum{...}| turns an Arabic number into a lowercase Roman one, and
% |\Romannum{...}| turns an Arabic number into an uppercase Roman one.
%
% \subsection*{Options}
% \DescribeMacro{options}
% \indent The \xpackage{romanbar} package takes no options.
%
% \section{Alternatives\label{sec:Alternatives}}
%
% \begin{description}
% \item[-] the original code written by Prof. Enrico Gregorio\\
%   (\url{http://profs.sci.univr.it/~gregorio/}),\\
%   \url{http://tex.stackexchange.com/questions/24065/roman-numerals-formatting/24084#24084}
%\begin{verbatim}
%\def\barroman#1{\sbox0{#1}\dimen0=\dimexpr\wd0+1pt\relax
%  \makebox[\dimen0]{\rlap{\vrule width\dimen0 height 0.06ex depth 0.06ex}%
%    \rlap{\vrule width\dimen0 height\dimexpr\ht0+0.03ex\relax
%            depth\dimexpr-\ht0+0.09ex\relax}%
%    \kern.5pt#1\kern.5pt}}
%
%\barroman{I} \barroman{XI}
%\end{verbatim}
%
% \item[-]  Some fonts provide single characters for Roman numerals,
%   cf.~e.\,g.
% \url{http://tex.stackexchange.com/questions/38695/using-unicode-roman-numerals-in-xetex}.
% \end{description}
%
% \noindent (You programmed or found another alternative,
%  which is available at \CTAN{}?\\
%  OK, send an e-mail to me with the name, location at \CTAN{},
%  and a short notice, and I will probably include it in
%  the list above.)
%
% \newpage
%
% \section{Example}
%
%    \begin{macrocode}
%<*example>
\documentclass[british]{article}[2007/10/19]% v1.4h
%%%%%%%%%%%%%%%%%%%%%%%%%%%%%%%%%%%%%%%%%%%%%%%%%%%%%%%%%%%%%%%%%%%%%
\usepackage[extension=pdf,%
 plainpages=false,%
 pdfpagelabels=true,%
 hyperindex=false,%
 pdflang={en},%
 pdftitle={romanbar package example},%
 pdfauthor={H.-Martin Muench},%
 pdfsubject={Example for the romanbar package},%
 pdfkeywords={LaTeX, romanbar, roman, Roman, bars, H.-Martin Muench},%
 pdfview={XYZ null null 1},%
 pdfstartview={XYZ null null 1},%
 pdfpagelayout=SinglePage]{hyperref}[2011/12/04]% v6.82m
\usepackage{romanbar}[2012/01/01]% v1.0f
\gdef\unit#1{\mathord{\thinspace\mathrm{#1}}}%
\listfiles
\begin{document}
\pagenumbering{arabic}
\section*{Example for romanbar}

This example demonstrates the use of package\newline
\textsf{romanbar}, v1.0f as of 2012/01/01 (HMM).\newline
There are no options to be used.\newline

\noindent For more details please see the documentation!\newline

\noindent Save per page about $200\unit{ml}$ water,
$2\unit{g}$ CO$_{2}$ and $2\unit{g}$ wood:\newline
Therefore please print only if this is really necessary.\newline

\noindent This package provides the command \verb|\Romanbar|
to print bars below and over the following:

\begin{description}
\item[-] Roman numbers: \verb|\Romanbar{MMXII}| prints \Romanbar{MMXII}

\item[-] Arabic numbers turned into upper-case Roman numbers:\newline
           \verb|\Romanbar{2012}| prints \Romanbar{2012}

\item[-] negative Arabic numbers turned into upper-case Roman numbers
           \newline
           (although historically there were no negative Roman numbers):
           \newline
           \verb|\Romanbar{-12}| prints \Romanbar{-12}

\item[-] zero Arabic number ($0$; although historically
           there was no Roman zero):\newline
           \verb|\Romanbar{0}| prints \Romanbar{0}

\item[-] some arbitrary text:
           \verb|\Romanbar{Caesar}| prints \Romanbar{Caesar}\newline
           (with descenders: \Romanbar{AgjpqyW})


\newcounter{example}
\setcounter{example}{21}

\item[-] some counter's value:
           \verb|\Romanbar{\theexample}| prints \Romanbar{\theexample}
           \newline
           (where the value of \texttt{example} is \theexample)

\item[-] Arabic numbers, without turning them into upper-case Roman
           numbers:\newline
           \verb|\Romanbar{\relax 2012}| prints \Romanbar{\relax  2012}
\end{description}

Special care was taken for "L" (50), e.\,g. in 555/DLV: \Romanbar{555}.\\


\verb|\romannum{...}| converts an Arabic number into a lower-case Roman one,
e.\,g. \verb|\romannum{2012}| prints \romannum{2012}, and
\verb|\Romannum{...}| converts an Arabic number into an upper-case Roman one,
e.\,g. \verb|\Romannum{2012}| prints \Romannum{2012}.

\end{document}
%</example>
%    \end{macrocode}
%
% \StopEventually{}
%
% \newpage
%
% \section{The implementation}
%
% We start off by checking that we are loading into \LaTeXe{} and
% announcing the name and version of this package.
%
%    \begin{macrocode}
%<*package>
%    \end{macrocode}
%
%    \begin{macrocode}
\NeedsTeXFormat{LaTeX2e}[2011/06/27]
\ProvidesPackage{romanbar}[2012/01/01 v1.0f
              Roman numbers with bars (HMM)]

%    \end{macrocode}
%
% A short description of the \xpackage{romanbar} package:
%
%    \begin{macrocode}
%% Allows to write Roman numbers (or any other text) with bars
%% and to convert Arabic numbers into Roman ones.
%    \end{macrocode}
%
% A last information for the user:
%
%    \begin{macrocode}
%% romanbar may work with earlier versions of LaTeX,
%% but this was not tested. Please consider updating
%% your LaTeX to the most recent version
%% (if it is not already the most recent version).

%    \end{macrocode}
%
% \indent See subsection~\ref{ss:Downloads} about how to get it.\\
%
% There are no options to be processed.
%
% A command to gobble a possible minus sign is needed and
% |\gobbleminus| from \url{http://www.tex.ac.uk/cgi-bin/texfaq2html?label=isitanum}
% is used.
%
%    \begin{macrocode}
\def\gobbleminus#1{\ifx-#1\else#1\fi}
%% from http://www.tex.ac.uk/cgi-bin/texfaq2html?label=isitanum

%    \end{macrocode}
%
% We want to test whether the argument passed to |\Romanbar| is numeric
% and therefore define |\ifnumeric{test}{true}{false}|,
% where |test| is to be analysed, |true| is the code to be executed,
% if |test| is numeric, and |false| is the code to be executed,
% if |test| is not numeric. This is done similar to\\
% \url{http://tex.stackexchange.com/a/17119}.
%
%    \begin{macrocode}
\newcommand{\ifnumeric}[3]{%
%% similar to http://tex.stackexchange.com/a/17119
\sbox\z@{\@tempcnta=0\gobbleminus#1\relax}%
\ifdim\wd0>\z@\relax#3% is not numeric
\else#2% is numeric
\fi%
}

%    \end{macrocode}
%
% We define |\r@iseL| to raise any L (50), otherwise a good deal
% of the horizontal part of the letter would be swallowed by the lower bar
% under the \textquotedblleft{}number\textquotedblright{}.
%
%    \begin{macrocode}
\def\r@iseL#1{\ifx @#1% then terminate
\else%
\if L#1\raisebox{0.05ex}{L}\else #1\fi%
\expandafter\r@iseL%
\fi%
}

%    \end{macrocode}
%
% \pagebreak
%
% We define the |\Romanbar| command.
%
%    \begin{macrocode}
\DeclareRobustCommand{\Romanbar}[1]{%
\ifnumeric{#1}{% is numeric
%    \end{macrocode}
%
% If its parameter is numeric, the temporary counter |\@tempcnta| is set to the value.
%
%    \begin{macrocode}
\@tempcnta=#1\relax%
%    \end{macrocode}
%
% Normally there are no non-positive Roman numbers, therefore we do some trick
% similar to |\XRoman| from the \url{http://ctan.org/pkg/pageslts} package.
%
%    \begin{macrocode}
%% similar to \XRoman from the http://ctan.org/pkg/pageslts package
\ifnum\@tempcnta<1%
  \ifnum #1>0%
    \relax \Roman@bar{\Roman{#1}}%
  \else%
    \ifnum #1<0%
     -\Roman@bar{\expandafter\@slowromancap\romannumeral\number-#1@%
     }%
    \else%
      \Roman@bar{0}%
    \fi%
  \fi%
\else\Roman@bar{\@Roman\@tempcnta}%
\fi%
}{% is not numeric
\Roman@bar{#1}%
}%
}

%    \end{macrocode}
%
% Whether it is numeric or not, in the end |\Roman@bar| is applied to it.
%
%    \begin{macrocode}
\DeclareRobustCommand{\Roman@bar}[1]{% #1 is in Roman, i.e. MMXII
%% (or any other text, "Caesar" would work, too).
%% similar to code of  by Prof. Enrico Gregorio (egreg) at
%% http://tex.stackexchange.com/questions/24065/roman-numerals-formatting/24084#24084
\@bsphack%
\edef\romanbartmp{#1}%
%% height:
\sbox0{\textsf{\romanbartmp}}%
%% top line:
%% + top of top line:
\@tempdima=\ht0%
\advance\@tempdima+0.05ex%0.03
\dimen1=\@tempdima%
%% + bottom of top line:
\@tempdima=-\ht0%
\advance\@tempdima+0.05ex%0.07
\dimen2=\@tempdima%
%% width:
\sbox0{\textsf{\expandafter\r@iseL\romanbartmp @}}%
\@tempdima=\wd0%
\advance\@tempdima+1pt%
\dimen0=\@tempdima%
\@esphack%
\makebox[\dimen0]{%
\rlap{\vrule width\dimen0 height\dimen1 depth\dimen2}% top    line
\rlap{\vrule width\dimen0 height 0.06ex depth 0.03ex}% bottom line
\kern0.5pt\textsf{\expandafter\r@iseL\romanbartmp @}\kern0.5pt}%
}

%    \end{macrocode}
%
% While it is not needed for the bars, it is sometimes asked for a command
% to convert an Arabic number into a Roman one. |\roman{...}| can only be applied
% to a counter, i.\,e. |\roman{12}| does not work, but
%\begin{verbatim}
%\newcounter{examplecounter}
%\setcounter{examplecounter}{12}
%\roman{examplecounter}
%\end{verbatim}
% would be necessary. Further non-negative values would not work.
% |\romannum| accepts any number as argument. (If the argument is no number,
% an error message is given and the argument printed (executed)
%  without applying anything to it.)
%
%    \begin{macrocode}
\DeclareRobustCommand{\romannum}[1]{%
\ifnumeric{#1}{% is numeric
\@tempcnta=#1\relax%
%% similar to \xroman from the http://ctan.org/pkg/pageslts package
\ifnum\@tempcnta<1%
  \ifnum #1>0%
    \relax \roman{#1}%
  \else%
    \ifnum #1<0%
     -\romannumeral\number-#1@%
    \else%
      0%
    \fi%
  \fi%
\else\@roman\@tempcnta%
\fi%
}{% is not numeric
\PackageError{romanbar}{%
Argument of \string\romannum\space is not a number}{%
The command \string\romannum\space converts an Arabic number into a%
 lower-case Roman one,\MessageBreak%
but the used argument of \string\romannum\space is not an Arabic number%
 but\MessageBreak%
`#1`,\MessageBreak%
which will now be printed unchanged.}%
#1%
}%
}

%    \end{macrocode}
%
% The same for upper-case Roman numbers, |\Romannum| instead of |\Roman|:
%
%    \begin{macrocode}
\DeclareRobustCommand{\Romannum}[1]{%
\ifnumeric{#1}{% is numeric
\@tempcnta=#1\relax%
%% similar to \XRoman from the http://ctan.org/pkg/pageslts package
\ifnum\@tempcnta<1%
  \ifnum #1>0%
    \relax \Roman{#1}%
  \else%
    \ifnum #1<0%
     -\expandafter\@slowromancap\romannumeral\number-#1@%
    \else%
      0%
    \fi%
  \fi%
\else\@Roman\@tempcnta%
\fi%
}{% is not numeric
\PackageError{romanbar}{%
Argument of \string\Romannum\space is not a number}{%
The command \string\Romannum\space converts an Arabic number into an%
 upper-case Roman one,\MessageBreak%
but the used argument of \string\Romannum\space is not an Arabic number%
 but\MessageBreak%
`#1`.}%
#1%
}%
}

%    \end{macrocode}
%
%    \begin{macrocode}
%</package>
%    \end{macrocode}
%
% \newpage
%
% \section{Installation}
%
% \subsection{Downloads\label{ss:Downloads}}
%
% Everything should be available on \CTAN{}, \url{http://www.ctan.org/tex-archive/},
% but may need additional packages themselves.\\
%
% \DescribeMacro{romanbar.dtx}
% For unpacking the |romanbar.dtx| file and constructing the documentation
% it is required:
% \begin{description}
% \item[-] \TeX{} Format \LaTeXe{}, \url{http://www.CTAN.org/}
%
% \item[-] document class \xpackage{ltxdoc}, 2007/11/11, v2.0u,
%           \url{http://ctan.org/pkg/ltxdoc}
%
% \item[-] package \xpackage{holtxdoc}, 2011/02/04, v0.21,
%           \url{http://ctan.org/pkg/holtxdoc}
%
% \item[-] package \xpackage{hypdoc}, 2010/03/26, v1.9,
%           \url{http://ctan.org/pkg/hypdoc}
% \end{description}
%
% \DescribeMacro{romanbar.sty}
% The \texttt{romanbar.sty} for \LaTeXe{} (i.\,e.~each document using
% the \xpackage{romanbar} package) requires:
% \begin{description}
% \item[-] \TeX Format \LaTeXe{}, \url{http://www.CTAN.org/}
% \end{description}
%
% \DescribeMacro{romanbar-example.tex}
% The \texttt{romanbar-example.tex} requires the same files as all
% documents using the \xpackage{romanbar} package and additionally:
% \begin{description}
% \item[-] class \xpackage{article}, 2007/10/19, v1.4h, from \xpackage{classes.dtx}:\\
%           \CTAN{macros/latex/base/classes.dtx}
%
% \item[-] package \xpackage{romanbar}, 2012/01/01, v1.0f,\\
%           %%\url{http://ctan.org/pkg/romanbar}\\
%           \CTAN{macros/latex/contrib/romanbar/}
%   (Well, it is the example file for this package, and because you are reading the
%    documentation for the \xpackage{romanbar} package, it can be assumed that you already
%    have some version of it -- is it the current one?)
% \end{description}
%
% \DescribeMacro{Alternative}
% As possible alternative package in section \ref{sec:Alternatives} there is listed
% \begin{description}
% \item[-] none.
% \end{description}
%
% \DescribeMacro{Oberdiek}
% \DescribeMacro{holtxdoc}
% All packages of \textsc{Heiko Oberdiek's} bundle `oberdiek'
% (especially \xpackage{holtxdoc}) are also available in a TDS compliant ZIP archive:\\
% \CTAN{install/macros/latex/contrib/oberdiek.tds.zip}.\\
% It is probably best to download and use this, because the packages in there
% are quite probably both recent and compatible among themselves.\\
%
% \DescribeMacro{hyperref}
% \noindent \xpackage{hyperref} is not included in that bundle and needs to be
% downloaded separately,\\
% \url{http://mirror.ctan.org/install/macros/latex/contrib/hyperref.tds.zip}.\\
%
% \DescribeMacro{M\"{u}nch}
% A hyperlinked list of my (other) packages can be found at
% \url{http://www.Uni-Bonn.de/~uzs5pv/LaTeX.html}.\\
%
% \pagebreak
%
% \subsection{Package, unpacking TDS}
% \paragraph{Package.} This package should become available on \CTAN{} soon:
% \begin{description}
% \item[\CTAN{macros/latex/contrib/romanbar/romanbar.dtx}]\hspace*{0.1cm} \\
%       The source file.
% \item[\CTAN{macros/latex/contrib/romanbar/romanbar.pdf}]\hspace*{0.1cm} \\
%       The documentation.
% \item[\CTAN{macros/latex/contrib/romanbar/romanbar-example.pdf}]\hspace*{0.1cm} \\
%       The compiled example file, as it should look like.
% \item[\CTAN{macros/latex/contrib/romanbar/README}]\hspace*{0.1cm} \\
%       The README file.
% \end{description}
% There is also a romanbar.tds.zip available:
% \begin{description}
% \item[\CTAN{install/macros/latex/contrib/romanbar.tds.zip}]\hspace*{0.1cm} \\
%       Everything in \xfile{TDS} compliant, compiled format.
% \end{description}
% which additionally contains\\
% \begin{tabular}{ll}
% romanbar.ins & The installation file.\\
% romanbar.drv & The driver to generate the documentation.\\
% romanbar.sty & The \xext{sty}le file.\\
% romanbar-example.tex & The example file.
% \end{tabular}
%
% \bigskip
%
% \noindent For required other packages, please see the preceding subsection.
%
% \paragraph{Unpacking.} The \xfile{.dtx} file is a self-extracting
% \docstrip{} archive. The files are extracted by running the
% \xext{.dtx} through \plainTeX{}:
% \begin{quote}
%   \verb|tex romanbar.dtx|
% \end{quote}
%
% About generating the documentation see paragraph~\ref{GenDoc} below.\\
%
% \paragraph{TDS.} Now the different files must be moved into
% the different directories in your installation TDS tree
% (also known as \xfile{texmf} tree):
% \begin{quote}
% \def\t{^^A
% \begin{tabular}{@{}>{\ttfamily}l@{ $\rightarrow$ }>{\ttfamily}l@{}}
%   romanbar.sty & tex/latex/romanbar/romanbar.sty\\
%   romanbar.pdf & doc/latex/romanbar/romanbar.pdf\\
%   romanbar-example.tex & doc/latex/romanbar/romanbar-example.tex\\
%   romanbar-example.pdf & doc/latex/romanbar/romanbar-example.pdf\\
%   romanbar.dtx & source/latex/romanbar/romanbar.dtx\\
% \end{tabular}^^A
% }^^A
% \sbox0{\t}^^A
% \ifdim\wd0>\linewidth
%   \begingroup
%     \advance\linewidth by\leftmargin
%     \advance\linewidth by\rightmargin
%   \edef\x{\endgroup
%     \def\noexpand\lw{\the\linewidth}^^A
%   }\x
%   \def\lwbox{^^A
%     \leavevmode
%     \hbox to \linewidth{^^A
%       \kern-\leftmargin\relax
%       \hss
%       \usebox0
%       \hss
%       \kern-\rightmargin\relax
%     }^^A
%   }^^A
%   \ifdim\wd0>\lw
%     \sbox0{\small\t}^^A
%     \ifdim\wd0>\linewidth
%       \ifdim\wd0>\lw
%         \sbox0{\footnotesize\t}^^A
%         \ifdim\wd0>\linewidth
%           \ifdim\wd0>\lw
%             \sbox0{\scriptsize\t}^^A
%             \ifdim\wd0>\linewidth
%               \ifdim\wd0>\lw
%                 \sbox0{\tiny\t}^^A
%                 \ifdim\wd0>\linewidth
%                   \lwbox
%                 \else
%                   \usebox0
%                 \fi
%               \else
%                 \lwbox
%               \fi
%             \else
%               \usebox0
%             \fi
%           \else
%             \lwbox
%           \fi
%         \else
%           \usebox0
%         \fi
%       \else
%         \lwbox
%       \fi
%     \else
%       \usebox0
%     \fi
%   \else
%     \lwbox
%   \fi
% \else
%   \usebox0
% \fi
% \end{quote}
% If you have a \xfile{docstrip.cfg} that configures and enables \docstrip{}'s
% \xfile{TDS} installing feature, then some files can already be in the right
% place, see the documentation of \docstrip{}.
%
% \subsection{Refresh file name databases}
%
% If your \TeX{}~distribution (\teTeX{}, \mikTeX{},\dots{}) relies on file name
% databases, you must refresh these. For example, \teTeX{} users run
% \verb|texhash| or \verb|mktexlsr|.
%
% \subsection{Some details for the interested}
%
% \paragraph{Unpacking with \LaTeX{}.}
% The \xfile{.dtx} chooses its action depending on the format:
% \begin{description}
% \item[\plainTeX:] Run \docstrip{} and extract the files.
% \item[\LaTeX:] Generate the documentation.
% \end{description}
% If you insist on using \LaTeX{} for \docstrip{} (really,
% \docstrip{} does not need \LaTeX{}), then inform the autodetect routine
% about your intention:
% \begin{quote}
%   \verb|latex \let\install=y\input{romanbar.dtx}|
% \end{quote}
% Do not forget to quote the argument according to the demands
% of your shell.
%
% \paragraph{Generating the documentation.\label{GenDoc}}
% You can use both the \xfile{.dtx} or the \xfile{.drv} to generate
% the documentation. The process can be configured by a
% configuration file \xfile{ltxdoc.cfg}. For instance, put the following
% line into this file, if you want to have A4 as paper format:
% \begin{quote}
%   \verb|\PassOptionsToClass{a4paper}{article}|
% \end{quote}
%
% \noindent An example follows how to generate the
% documentation with \pdfLaTeX{}:
%
% \begin{quote}
%\begin{verbatim}
%pdflatex romanbar.dtx
%makeindex -s gind.ist romanbar.idx
%pdflatex romanbar.dtx
%makeindex -s gind.ist romanbar.idx
%pdflatex romanbar.dtx
%\end{verbatim}
% \end{quote}
%
% \subsection{Compiling the example}
%
% The example file, \textsf{romanbar-example.tex}, can be compiled via\\
% |(pdf)(la)tex romanbar-example.tex|.
%
% \section{Acknowledgements}
%
% I would like to thank \textsc{Heiko Oberdiek} for providing
% the \xpackage{hyperref} as well as a~lot~(!) of other useful packages
% (from which I also got everything I know about creating a file in \xext{dtx}
% format, ok, say it: copying),
% Prof.~\textsc{Enrico Gregorio} (egreg) for his answer
% \url{http://tex.stackexchange.com/questions/24065/roman-numerals-formatting/24084#24084},
% the \Newsgroup{comp.text.tex} and \Newsgroup{de.comp.text.tex}
% newsgroups as well as \url{http://tex.stackexchange.com}
% for their help in all things \TeX{}.
%
% \pagebreak
%
% \phantomsection
% \begin{History}\label{History}
%   \begin{Version}{2011/07/25 v1.0a}
%     \item First code by Prof.~\textsc{Enrico Gregorio} (egreg) at
%             \url{http://tex.stackexchange.com/questions/24065/roman-numerals-formatting/24084#24084}
%   \end{Version}
%   \begin{Version}{2011/07/26 v1.0b}
%     \item Packed into a \xext{sty} file.
%     \item Removed the requirement of \eTeX{} for the package.
%   \end{Version}
%   \begin{Version}{2011/08/25 v1.0c}
%     \item Renamed \xpackage{romanbar} package.
%     \item Minor details.
%   \end{Version}
%   \begin{Version}{2011/09/16 v1.0d}
%     \item Made commands robust.
%     \item Minor details.
%   \end{Version}
%   \begin{Version}{2011/12/16 v1.0e}
%     \item Additionally accepts now Arabic numbers and converts them to Roman ones with bars.
%     \item Also non-positive numbers are now accepted.
%   \end{Version}
%   \begin{Version}{2012/01/01 v1.0f}
%     \item Added |\romannum| and |\Romannum| to convert Arabic to Roman numbers (without bars).
%     \item |\r@iseL|.
%     \item Created a \xfile{dtx}, example, and README.
%     \item Upload to \CTAN{}.
%   \end{Version}
% \end{History}
%
% \bigskip
%
% When you find a mistake or have a suggestion for an improvement of this package,
% please send an e-mail to the maintainer, thanks! (Please see BUG REPORTS in the README.)
%
% \pagebreak
%
% \PrintIndex
%
% \Finale
\endinput
%        (quote the arguments according to the demands of your shell)
%
% Documentation:
%    (a) If romanbar.drv is present:
%           (pdf)latex romanbar.drv
%           makeindex -s gind.ist romanbar.idx
%           (pdf)latex romanbar.drv
%           makeindex -s gind.ist romanbar.idx
%           (pdf)latex romanbar.drv
%    (b) Without romanbar.drv:
%           (pdf)latex romanbar.dtx
%           makeindex -s gind.ist romanbar.idx
%           (pdf)latex romanbar.dtx
%           makeindex -s gind.ist romanbar.idx
%           (pdf)latex romanbar.dtx
%
%    The class ltxdoc loads the configuration file ltxdoc.cfg
%    if available. Here you can specify further options, e.g.
%    use DIN A4 as paper format:
%       \PassOptionsToClass{a4paper}{article}
%
% Installation:
%    TDS:tex/latex/romanbar/romanbar.sty
%    TDS:doc/latex/romanbar/romanbar.pdf
%    TDS:doc/latex/romanbar/romanbar-example.tex
%    TDS:doc/latex/romanbar/romanbar-example.pdf
%    TDS:source/latex/romanbar/romanbar.dtx
%
%<*ignore>
\begingroup
  \catcode123=1 %
  \catcode125=2 %
  \def\x{LaTeX2e}%
\expandafter\endgroup
\ifcase 0\ifx\install y1\fi\expandafter
         \ifx\csname processbatchFile\endcsname\relax\else1\fi
         \ifx\fmtname\x\else 1\fi\relax
\else\csname fi\endcsname
%</ignore>
%<*install>
\input docstrip.tex
\Msg{********************************************************************}
\Msg{* Installation                                                     *}
\Msg{* Package: romanbar 2012/01/01 v1.0f Roman numbers with bars (HMM) *}
\Msg{********************************************************************}

\keepsilent
\askforoverwritefalse

\let\MetaPrefix\relax
\preamble

This is a generated file.

Project: romanbar
Version: 2012/01/01 v1.0f

Copyright (C) 2011 by
    H.-Martin M"unch <Martin dot Muench at Uni-Bonn dot de>

The usual disclaimer applies:
If it doesn't work right that's your problem.
(Nevertheless, send an e-mail to the maintainer
 when you find an error in this package.)

This work may be distributed and/or modified under the
conditions of the LaTeX Project Public License, either
version 1.3c of this license or (at your option) any later
version. This version of this license is in
   http://www.latex-project.org/lppl/lppl-1-3c.txt
and the latest version of this license is in
   http://www.latex-project.org/lppl.txt
and version 1.3c or later is part of all distributions of
LaTeX version 2005/12/01 or later.

This work has the LPPL maintenance status "maintained".

The Current Maintainer of this work is H.-Martin Muench.

This work consists of the main source file romanbar.dtx,
the README, and the derived files
   romanbar.sty, romanbar.pdf,
   romanbar.ins, romanbar.drv,
   romanbar-example.tex, romanbar-example.pdf.

\endpreamble
\let\MetaPrefix\DoubleperCent

\generate{%
  \file{romanbar.ins}{\from{romanbar.dtx}{install}}%
  \file{romanbar.drv}{\from{romanbar.dtx}{driver}}%
  \usedir{tex/latex/romanbar}%
  \file{romanbar.sty}{\from{romanbar.dtx}{package}}%
  \usedir{doc/latex/romanbar}%
  \file{romanbar-example.tex}{\from{romanbar.dtx}{example}}%
}

\catcode32=13\relax% active space
\let =\space%
\Msg{************************************************************************}
\Msg{*}
\Msg{* To finish the installation you have to move the following}
\Msg{* file into a directory searched by TeX:}
\Msg{*}
\Msg{*     romanbar.sty}
\Msg{*}
\Msg{* To produce the documentation run the file `romanbar.drv'}
\Msg{* through (pdf)LaTeX, e.g.}
\Msg{*  pdflatex romanbar.drv}
\Msg{*  makeindex -s gind.ist romanbar.idx}
\Msg{*  pdflatex romanbar.drv}
\Msg{*  makeindex -s gind.ist romanbar.idx}
\Msg{*  pdflatex romanbar.drv}
\Msg{*}
\Msg{* At least three runs are necessary e.g. to get the}
\Msg{*  references right!}
\Msg{*}
\Msg{* Happy TeXing!}
\Msg{*}
\Msg{************************************************************************}

\endbatchfile
%</install>
%<*ignore>
\fi
%</ignore>
%
% \section{The documentation driver file}
%
% The next bit of code contains the documentation driver file for
% \TeX{}, i.\,e., the file that will produce the documentation you
% are currently reading. It will be extracted from this file by the
% \texttt{docstrip} programme. That is, run \LaTeX{} on \texttt{docstrip}
% and specify the \texttt{driver} option when \texttt{docstrip}
% asks for options.
%
%    \begin{macrocode}
%<*driver>
\NeedsTeXFormat{LaTeX2e}[2011/06/27]
\ProvidesFile{romanbar.drv}%
  [2012/01/01 v1.0f Roman numbers with bars (HMM)]
\documentclass{ltxdoc}[2007/11/11]% v2.0u
\usepackage{holtxdoc}[2011/02/04]%  v0.21
%% romanbar may work with earlier versions of LaTeX2e and those
%% class and package, but this was not tested.
%% Please consider updating your LaTeX, class, and package
%% to the most recent version (if they are not already the most
%% recent version).
\hypersetup{%
 pdfsubject={Writing Roman numbers with bars (HMM)},%
 pdfkeywords={LaTeX, romanbar, roman, Roman, bars, H.-Martin Muench},%
 pdfencoding=auto,%
 pdflang={en},%
 breaklinks=true,%
 linktoc=all,%
 pdfstartview=FitH,%
 pdfpagelayout=OneColumn,%
 bookmarksnumbered=true,%
 bookmarksopen=true,%
 bookmarksopenlevel=2,%
 pdfmenubar=true,%
 pdftoolbar=true,%
 pdfwindowui=true,%
 pdfnewwindow=true%
}
\CodelineIndex
\hyphenation{docu-ment ana-lysed}
\gdef\unit#1{\mathord{\thinspace\mathrm{#1}}}%
\makeatletter
\@ifundefined{eTeX}{\gdef\eTeX{$\m@th \varepsilon $-\TeX }}{% else \relax
}
\makeatother
\begin{document}
  \DocInput{romanbar.dtx}%
\end{document}
%</driver>
%    \end{macrocode}
%
% \fi
%
% \CheckSum{179}
%
% \CharacterTable
%  {Upper-case    \A\B\C\D\E\F\G\H\I\J\K\L\M\N\O\P\Q\R\S\T\U\V\W\X\Y\Z
%   Lower-case    \a\b\c\d\e\f\g\h\i\j\k\l\m\n\o\p\q\r\s\t\u\v\w\x\y\z
%   Digits        \0\1\2\3\4\5\6\7\8\9
%   Exclamation   \!     Double quote  \"     Hash (number) \#
%   Dollar        \$     Percent       \%     Ampersand     \&
%   Acute accent  \'     Left paren    \(     Right paren   \)
%   Asterisk      \*     Plus          \+     Comma         \,
%   Minus         \-     Point         \.     Solidus       \/
%   Colon         \:     Semicolon     \;     Less than     \<
%   Equals        \=     Greater than  \>     Question mark \?
%   Commercial at \@     Left bracket  \[     Backslash     \\
%   Right bracket \]     Circumflex    \^     Underscore    \_
%   Grave accent  \`     Left brace    \{     Vertical bar  \|
%   Right brace   \}     Tilde         \~}
%
% \GetFileInfo{romanbar.drv}
%
% \begingroup
%   \def\x{\#,\$,\^,\_,\~,\ ,\&,\{,\},\%}%
%   \makeatletter
%   \@onelevel@sanitize\x
% \expandafter\endgroup
% \expandafter\DoNotIndex\expandafter{\x}
% \expandafter\DoNotIndex\expandafter{\string\ }
% \begingroup
%   \makeatletter
%     \lccode`9=32\relax
%     \lowercase{%^^A
%       \edef\x{\noexpand\DoNotIndex{\@backslashchar9}}%^^A
%     }%^^A
%   \expandafter\endgroup\x
%
% \DoNotIndex{\\,\,}
% \DoNotIndex{\documentclass,\usepackage,\ProvidesPackage,\begin,\end}
% \DoNotIndex{\MessageBreak}
% \DoNotIndex{\NeedsTeXFormat,\DoNotIndex,\verb}
% \DoNotIndex{\def,\edef,\gdef,\global}
% \DoNotIndex{\ifx,\listfiles,\mathord,\mathrm}
% \DoNotIndex{\bigskip,\space,\thinspace,\ldots}
% \DoNotIndex{\indent,\noindent,\newline,\linebreak,\pagebreak,\newpage}
% \DoNotIndex{\textbf,\textit,\textsf,\texttt,\textquotedblleft,\textquotedblright}
% \DoNotIndex{\plainTeX,\TeX,\LaTeX,\pdfLaTeX}
% \DoNotIndex{\item}
%
% \title{The \xpackage{romanbar} package}
% \date{2012/01/01 v1.0f}
% \author{H.-Martin M\"{u}nch\\\xemail{Martin.Muench at Uni-Bonn.de}}
%
% \maketitle
%
% \begin{abstract}
%  This \LaTeX{} package allows to write Roman numbers
%  (or any other text) with bars. (Additionally, commands for converting Arabic numbers
%  into Roman ones are provided and an |\ifnumeric| test function.)
% \end{abstract}
%
% \bigskip
%
% \noindent Disclaimer for web links: The author is not responsible for any contents
% referred to in this work unless he has full knowledge of illegal contents.
% If any damage occurs by the use of information presented there, only the
% author of the respective pages might be liable, not the one who has referred
% to these pages.
%
% \bigskip
%
% \noindent {\color{green} Save per page about $200\unit{ml}$ water,
% $2\unit{g}$ CO$_{2}$ and $2\unit{g}$ wood:\\
% Therefore please print only if this is really necessary.}
%
% \pagebreak
%
% \tableofcontents
%
% \pagebreak
%
% \section{Introduction}
%
% \indent This \LaTeX{} package allows to write Roman numbers
% (or any other text) with bars. (Additionally, command for converting Arabic numbers
% into Roman ones are provided and an |\ifnumeric| test function.)
%
% \bigskip
%
% \section{Usage}
%
% \indent Just load the package placing
% \begin{quote}
%   |\usepackage{romanbar}|
% \end{quote}
% \noindent in the preamble of your \LaTeXe{} source file.\\
% |\Romanbar{...}| then produces a Roman number with bars (please see the example file).
% |\ifnumeric{test}{true}{false}| tests for |test| being numeric.
% |\romannum{...}| turns an Arabic number into a lowercase Roman one, and
% |\Romannum{...}| turns an Arabic number into an uppercase Roman one.
%
% \subsection*{Options}
% \DescribeMacro{options}
% \indent The \xpackage{romanbar} package takes no options.
%
% \section{Alternatives\label{sec:Alternatives}}
%
% \begin{description}
% \item[-] the original code written by Prof. Enrico Gregorio\\
%   (\url{http://profs.sci.univr.it/~gregorio/}),\\
%   \url{http://tex.stackexchange.com/questions/24065/roman-numerals-formatting/24084#24084}
%\begin{verbatim}
%\def\barroman#1{\sbox0{#1}\dimen0=\dimexpr\wd0+1pt\relax
%  \makebox[\dimen0]{\rlap{\vrule width\dimen0 height 0.06ex depth 0.06ex}%
%    \rlap{\vrule width\dimen0 height\dimexpr\ht0+0.03ex\relax
%            depth\dimexpr-\ht0+0.09ex\relax}%
%    \kern.5pt#1\kern.5pt}}
%
%\barroman{I} \barroman{XI}
%\end{verbatim}
%
% \item[-]  Some fonts provide single characters for Roman numerals,
%   cf.~e.\,g.
% \url{http://tex.stackexchange.com/questions/38695/using-unicode-roman-numerals-in-xetex}.
% \end{description}
%
% \noindent (You programmed or found another alternative,
%  which is available at \CTAN{}?\\
%  OK, send an e-mail to me with the name, location at \CTAN{},
%  and a short notice, and I will probably include it in
%  the list above.)
%
% \newpage
%
% \section{Example}
%
%    \begin{macrocode}
%<*example>
\documentclass[british]{article}[2007/10/19]% v1.4h
%%%%%%%%%%%%%%%%%%%%%%%%%%%%%%%%%%%%%%%%%%%%%%%%%%%%%%%%%%%%%%%%%%%%%
\usepackage[extension=pdf,%
 plainpages=false,%
 pdfpagelabels=true,%
 hyperindex=false,%
 pdflang={en},%
 pdftitle={romanbar package example},%
 pdfauthor={H.-Martin Muench},%
 pdfsubject={Example for the romanbar package},%
 pdfkeywords={LaTeX, romanbar, roman, Roman, bars, H.-Martin Muench},%
 pdfview={XYZ null null 1},%
 pdfstartview={XYZ null null 1},%
 pdfpagelayout=SinglePage]{hyperref}[2011/12/04]% v6.82m
\usepackage{romanbar}[2012/01/01]% v1.0f
\gdef\unit#1{\mathord{\thinspace\mathrm{#1}}}%
\listfiles
\begin{document}
\pagenumbering{arabic}
\section*{Example for romanbar}

This example demonstrates the use of package\newline
\textsf{romanbar}, v1.0f as of 2012/01/01 (HMM).\newline
There are no options to be used.\newline

\noindent For more details please see the documentation!\newline

\noindent Save per page about $200\unit{ml}$ water,
$2\unit{g}$ CO$_{2}$ and $2\unit{g}$ wood:\newline
Therefore please print only if this is really necessary.\newline

\noindent This package provides the command \verb|\Romanbar|
to print bars below and over the following:

\begin{description}
\item[-] Roman numbers: \verb|\Romanbar{MMXII}| prints \Romanbar{MMXII}

\item[-] Arabic numbers turned into upper-case Roman numbers:\newline
           \verb|\Romanbar{2012}| prints \Romanbar{2012}

\item[-] negative Arabic numbers turned into upper-case Roman numbers
           \newline
           (although historically there were no negative Roman numbers):
           \newline
           \verb|\Romanbar{-12}| prints \Romanbar{-12}

\item[-] zero Arabic number ($0$; although historically
           there was no Roman zero):\newline
           \verb|\Romanbar{0}| prints \Romanbar{0}

\item[-] some arbitrary text:
           \verb|\Romanbar{Caesar}| prints \Romanbar{Caesar}\newline
           (with descenders: \Romanbar{AgjpqyW})


\newcounter{example}
\setcounter{example}{21}

\item[-] some counter's value:
           \verb|\Romanbar{\theexample}| prints \Romanbar{\theexample}
           \newline
           (where the value of \texttt{example} is \theexample)

\item[-] Arabic numbers, without turning them into upper-case Roman
           numbers:\newline
           \verb|\Romanbar{\relax 2012}| prints \Romanbar{\relax  2012}
\end{description}

Special care was taken for "L" (50), e.\,g. in 555/DLV: \Romanbar{555}.\\


\verb|\romannum{...}| converts an Arabic number into a lower-case Roman one,
e.\,g. \verb|\romannum{2012}| prints \romannum{2012}, and
\verb|\Romannum{...}| converts an Arabic number into an upper-case Roman one,
e.\,g. \verb|\Romannum{2012}| prints \Romannum{2012}.

\end{document}
%</example>
%    \end{macrocode}
%
% \StopEventually{}
%
% \newpage
%
% \section{The implementation}
%
% We start off by checking that we are loading into \LaTeXe{} and
% announcing the name and version of this package.
%
%    \begin{macrocode}
%<*package>
%    \end{macrocode}
%
%    \begin{macrocode}
\NeedsTeXFormat{LaTeX2e}[2011/06/27]
\ProvidesPackage{romanbar}[2012/01/01 v1.0f
              Roman numbers with bars (HMM)]

%    \end{macrocode}
%
% A short description of the \xpackage{romanbar} package:
%
%    \begin{macrocode}
%% Allows to write Roman numbers (or any other text) with bars
%% and to convert Arabic numbers into Roman ones.
%    \end{macrocode}
%
% A last information for the user:
%
%    \begin{macrocode}
%% romanbar may work with earlier versions of LaTeX,
%% but this was not tested. Please consider updating
%% your LaTeX to the most recent version
%% (if it is not already the most recent version).

%    \end{macrocode}
%
% \indent See subsection~\ref{ss:Downloads} about how to get it.\\
%
% There are no options to be processed.
%
% A command to gobble a possible minus sign is needed and
% |\gobbleminus| from \url{http://www.tex.ac.uk/cgi-bin/texfaq2html?label=isitanum}
% is used.
%
%    \begin{macrocode}
\def\gobbleminus#1{\ifx-#1\else#1\fi}
%% from http://www.tex.ac.uk/cgi-bin/texfaq2html?label=isitanum

%    \end{macrocode}
%
% We want to test whether the argument passed to |\Romanbar| is numeric
% and therefore define |\ifnumeric{test}{true}{false}|,
% where |test| is to be analysed, |true| is the code to be executed,
% if |test| is numeric, and |false| is the code to be executed,
% if |test| is not numeric. This is done similar to\\
% \url{http://tex.stackexchange.com/a/17119}.
%
%    \begin{macrocode}
\newcommand{\ifnumeric}[3]{%
%% similar to http://tex.stackexchange.com/a/17119
\sbox\z@{\@tempcnta=0\gobbleminus#1\relax}%
\ifdim\wd0>\z@\relax#3% is not numeric
\else#2% is numeric
\fi%
}

%    \end{macrocode}
%
% We define |\r@iseL| to raise any L (50), otherwise a good deal
% of the horizontal part of the letter would be swallowed by the lower bar
% under the \textquotedblleft{}number\textquotedblright{}.
%
%    \begin{macrocode}
\def\r@iseL#1{\ifx @#1% then terminate
\else%
\if L#1\raisebox{0.05ex}{L}\else #1\fi%
\expandafter\r@iseL%
\fi%
}

%    \end{macrocode}
%
% \pagebreak
%
% We define the |\Romanbar| command.
%
%    \begin{macrocode}
\DeclareRobustCommand{\Romanbar}[1]{%
\ifnumeric{#1}{% is numeric
%    \end{macrocode}
%
% If its parameter is numeric, the temporary counter |\@tempcnta| is set to the value.
%
%    \begin{macrocode}
\@tempcnta=#1\relax%
%    \end{macrocode}
%
% Normally there are no non-positive Roman numbers, therefore we do some trick
% similar to |\XRoman| from the \url{http://ctan.org/pkg/pageslts} package.
%
%    \begin{macrocode}
%% similar to \XRoman from the http://ctan.org/pkg/pageslts package
\ifnum\@tempcnta<1%
  \ifnum #1>0%
    \relax \Roman@bar{\Roman{#1}}%
  \else%
    \ifnum #1<0%
     -\Roman@bar{\expandafter\@slowromancap\romannumeral\number-#1@%
     }%
    \else%
      \Roman@bar{0}%
    \fi%
  \fi%
\else\Roman@bar{\@Roman\@tempcnta}%
\fi%
}{% is not numeric
\Roman@bar{#1}%
}%
}

%    \end{macrocode}
%
% Whether it is numeric or not, in the end |\Roman@bar| is applied to it.
%
%    \begin{macrocode}
\DeclareRobustCommand{\Roman@bar}[1]{% #1 is in Roman, i.e. MMXII
%% (or any other text, "Caesar" would work, too).
%% similar to code of  by Prof. Enrico Gregorio (egreg) at
%% http://tex.stackexchange.com/questions/24065/roman-numerals-formatting/24084#24084
\@bsphack%
\edef\romanbartmp{#1}%
%% height:
\sbox0{\textsf{\romanbartmp}}%
%% top line:
%% + top of top line:
\@tempdima=\ht0%
\advance\@tempdima+0.05ex%0.03
\dimen1=\@tempdima%
%% + bottom of top line:
\@tempdima=-\ht0%
\advance\@tempdima+0.05ex%0.07
\dimen2=\@tempdima%
%% width:
\sbox0{\textsf{\expandafter\r@iseL\romanbartmp @}}%
\@tempdima=\wd0%
\advance\@tempdima+1pt%
\dimen0=\@tempdima%
\@esphack%
\makebox[\dimen0]{%
\rlap{\vrule width\dimen0 height\dimen1 depth\dimen2}% top    line
\rlap{\vrule width\dimen0 height 0.06ex depth 0.03ex}% bottom line
\kern0.5pt\textsf{\expandafter\r@iseL\romanbartmp @}\kern0.5pt}%
}

%    \end{macrocode}
%
% While it is not needed for the bars, it is sometimes asked for a command
% to convert an Arabic number into a Roman one. |\roman{...}| can only be applied
% to a counter, i.\,e. |\roman{12}| does not work, but
%\begin{verbatim}
%\newcounter{examplecounter}
%\setcounter{examplecounter}{12}
%\roman{examplecounter}
%\end{verbatim}
% would be necessary. Further non-negative values would not work.
% |\romannum| accepts any number as argument. (If the argument is no number,
% an error message is given and the argument printed (executed)
%  without applying anything to it.)
%
%    \begin{macrocode}
\DeclareRobustCommand{\romannum}[1]{%
\ifnumeric{#1}{% is numeric
\@tempcnta=#1\relax%
%% similar to \xroman from the http://ctan.org/pkg/pageslts package
\ifnum\@tempcnta<1%
  \ifnum #1>0%
    \relax \roman{#1}%
  \else%
    \ifnum #1<0%
     -\romannumeral\number-#1@%
    \else%
      0%
    \fi%
  \fi%
\else\@roman\@tempcnta%
\fi%
}{% is not numeric
\PackageError{romanbar}{%
Argument of \string\romannum\space is not a number}{%
The command \string\romannum\space converts an Arabic number into a%
 lower-case Roman one,\MessageBreak%
but the used argument of \string\romannum\space is not an Arabic number%
 but\MessageBreak%
`#1`,\MessageBreak%
which will now be printed unchanged.}%
#1%
}%
}

%    \end{macrocode}
%
% The same for upper-case Roman numbers, |\Romannum| instead of |\Roman|:
%
%    \begin{macrocode}
\DeclareRobustCommand{\Romannum}[1]{%
\ifnumeric{#1}{% is numeric
\@tempcnta=#1\relax%
%% similar to \XRoman from the http://ctan.org/pkg/pageslts package
\ifnum\@tempcnta<1%
  \ifnum #1>0%
    \relax \Roman{#1}%
  \else%
    \ifnum #1<0%
     -\expandafter\@slowromancap\romannumeral\number-#1@%
    \else%
      0%
    \fi%
  \fi%
\else\@Roman\@tempcnta%
\fi%
}{% is not numeric
\PackageError{romanbar}{%
Argument of \string\Romannum\space is not a number}{%
The command \string\Romannum\space converts an Arabic number into an%
 upper-case Roman one,\MessageBreak%
but the used argument of \string\Romannum\space is not an Arabic number%
 but\MessageBreak%
`#1`.}%
#1%
}%
}

%    \end{macrocode}
%
%    \begin{macrocode}
%</package>
%    \end{macrocode}
%
% \newpage
%
% \section{Installation}
%
% \subsection{Downloads\label{ss:Downloads}}
%
% Everything should be available on \CTAN{}, \url{http://www.ctan.org/tex-archive/},
% but may need additional packages themselves.\\
%
% \DescribeMacro{romanbar.dtx}
% For unpacking the |romanbar.dtx| file and constructing the documentation
% it is required:
% \begin{description}
% \item[-] \TeX{} Format \LaTeXe{}, \url{http://www.CTAN.org/}
%
% \item[-] document class \xpackage{ltxdoc}, 2007/11/11, v2.0u,
%           \url{http://ctan.org/pkg/ltxdoc}
%
% \item[-] package \xpackage{holtxdoc}, 2011/02/04, v0.21,
%           \url{http://ctan.org/pkg/holtxdoc}
%
% \item[-] package \xpackage{hypdoc}, 2010/03/26, v1.9,
%           \url{http://ctan.org/pkg/hypdoc}
% \end{description}
%
% \DescribeMacro{romanbar.sty}
% The \texttt{romanbar.sty} for \LaTeXe{} (i.\,e.~each document using
% the \xpackage{romanbar} package) requires:
% \begin{description}
% \item[-] \TeX Format \LaTeXe{}, \url{http://www.CTAN.org/}
% \end{description}
%
% \DescribeMacro{romanbar-example.tex}
% The \texttt{romanbar-example.tex} requires the same files as all
% documents using the \xpackage{romanbar} package and additionally:
% \begin{description}
% \item[-] class \xpackage{article}, 2007/10/19, v1.4h, from \xpackage{classes.dtx}:\\
%           \CTAN{macros/latex/base/classes.dtx}
%
% \item[-] package \xpackage{romanbar}, 2012/01/01, v1.0f,\\
%           %%\url{http://ctan.org/pkg/romanbar}\\
%           \CTAN{macros/latex/contrib/romanbar/}
%   (Well, it is the example file for this package, and because you are reading the
%    documentation for the \xpackage{romanbar} package, it can be assumed that you already
%    have some version of it -- is it the current one?)
% \end{description}
%
% \DescribeMacro{Alternative}
% As possible alternative package in section \ref{sec:Alternatives} there is listed
% \begin{description}
% \item[-] none.
% \end{description}
%
% \DescribeMacro{Oberdiek}
% \DescribeMacro{holtxdoc}
% All packages of \textsc{Heiko Oberdiek's} bundle `oberdiek'
% (especially \xpackage{holtxdoc}) are also available in a TDS compliant ZIP archive:\\
% \CTAN{install/macros/latex/contrib/oberdiek.tds.zip}.\\
% It is probably best to download and use this, because the packages in there
% are quite probably both recent and compatible among themselves.\\
%
% \DescribeMacro{hyperref}
% \noindent \xpackage{hyperref} is not included in that bundle and needs to be
% downloaded separately,\\
% \url{http://mirror.ctan.org/install/macros/latex/contrib/hyperref.tds.zip}.\\
%
% \DescribeMacro{M\"{u}nch}
% A hyperlinked list of my (other) packages can be found at
% \url{http://www.Uni-Bonn.de/~uzs5pv/LaTeX.html}.\\
%
% \pagebreak
%
% \subsection{Package, unpacking TDS}
% \paragraph{Package.} This package should become available on \CTAN{} soon:
% \begin{description}
% \item[\CTAN{macros/latex/contrib/romanbar/romanbar.dtx}]\hspace*{0.1cm} \\
%       The source file.
% \item[\CTAN{macros/latex/contrib/romanbar/romanbar.pdf}]\hspace*{0.1cm} \\
%       The documentation.
% \item[\CTAN{macros/latex/contrib/romanbar/romanbar-example.pdf}]\hspace*{0.1cm} \\
%       The compiled example file, as it should look like.
% \item[\CTAN{macros/latex/contrib/romanbar/README}]\hspace*{0.1cm} \\
%       The README file.
% \end{description}
% There is also a romanbar.tds.zip available:
% \begin{description}
% \item[\CTAN{install/macros/latex/contrib/romanbar.tds.zip}]\hspace*{0.1cm} \\
%       Everything in \xfile{TDS} compliant, compiled format.
% \end{description}
% which additionally contains\\
% \begin{tabular}{ll}
% romanbar.ins & The installation file.\\
% romanbar.drv & The driver to generate the documentation.\\
% romanbar.sty & The \xext{sty}le file.\\
% romanbar-example.tex & The example file.
% \end{tabular}
%
% \bigskip
%
% \noindent For required other packages, please see the preceding subsection.
%
% \paragraph{Unpacking.} The \xfile{.dtx} file is a self-extracting
% \docstrip{} archive. The files are extracted by running the
% \xext{.dtx} through \plainTeX{}:
% \begin{quote}
%   \verb|tex romanbar.dtx|
% \end{quote}
%
% About generating the documentation see paragraph~\ref{GenDoc} below.\\
%
% \paragraph{TDS.} Now the different files must be moved into
% the different directories in your installation TDS tree
% (also known as \xfile{texmf} tree):
% \begin{quote}
% \def\t{^^A
% \begin{tabular}{@{}>{\ttfamily}l@{ $\rightarrow$ }>{\ttfamily}l@{}}
%   romanbar.sty & tex/latex/romanbar/romanbar.sty\\
%   romanbar.pdf & doc/latex/romanbar/romanbar.pdf\\
%   romanbar-example.tex & doc/latex/romanbar/romanbar-example.tex\\
%   romanbar-example.pdf & doc/latex/romanbar/romanbar-example.pdf\\
%   romanbar.dtx & source/latex/romanbar/romanbar.dtx\\
% \end{tabular}^^A
% }^^A
% \sbox0{\t}^^A
% \ifdim\wd0>\linewidth
%   \begingroup
%     \advance\linewidth by\leftmargin
%     \advance\linewidth by\rightmargin
%   \edef\x{\endgroup
%     \def\noexpand\lw{\the\linewidth}^^A
%   }\x
%   \def\lwbox{^^A
%     \leavevmode
%     \hbox to \linewidth{^^A
%       \kern-\leftmargin\relax
%       \hss
%       \usebox0
%       \hss
%       \kern-\rightmargin\relax
%     }^^A
%   }^^A
%   \ifdim\wd0>\lw
%     \sbox0{\small\t}^^A
%     \ifdim\wd0>\linewidth
%       \ifdim\wd0>\lw
%         \sbox0{\footnotesize\t}^^A
%         \ifdim\wd0>\linewidth
%           \ifdim\wd0>\lw
%             \sbox0{\scriptsize\t}^^A
%             \ifdim\wd0>\linewidth
%               \ifdim\wd0>\lw
%                 \sbox0{\tiny\t}^^A
%                 \ifdim\wd0>\linewidth
%                   \lwbox
%                 \else
%                   \usebox0
%                 \fi
%               \else
%                 \lwbox
%               \fi
%             \else
%               \usebox0
%             \fi
%           \else
%             \lwbox
%           \fi
%         \else
%           \usebox0
%         \fi
%       \else
%         \lwbox
%       \fi
%     \else
%       \usebox0
%     \fi
%   \else
%     \lwbox
%   \fi
% \else
%   \usebox0
% \fi
% \end{quote}
% If you have a \xfile{docstrip.cfg} that configures and enables \docstrip{}'s
% \xfile{TDS} installing feature, then some files can already be in the right
% place, see the documentation of \docstrip{}.
%
% \subsection{Refresh file name databases}
%
% If your \TeX{}~distribution (\teTeX{}, \mikTeX{},\dots{}) relies on file name
% databases, you must refresh these. For example, \teTeX{} users run
% \verb|texhash| or \verb|mktexlsr|.
%
% \subsection{Some details for the interested}
%
% \paragraph{Unpacking with \LaTeX{}.}
% The \xfile{.dtx} chooses its action depending on the format:
% \begin{description}
% \item[\plainTeX:] Run \docstrip{} and extract the files.
% \item[\LaTeX:] Generate the documentation.
% \end{description}
% If you insist on using \LaTeX{} for \docstrip{} (really,
% \docstrip{} does not need \LaTeX{}), then inform the autodetect routine
% about your intention:
% \begin{quote}
%   \verb|latex \let\install=y% \iffalse meta-comment
%
% File: romanbar.dtx
% Version: 2012/01/01 v1.0f
%
% Copyright (C) 2011 by
%    H.-Martin M"unch <Martin dot Muench at Uni-Bonn dot de>
%
% The original code for the bars was written by Prof. Enrico Gregorio
% (http://profs.sci.univr.it/~gregorio/), and published at
% http://tex.stackexchange.com/questions/24065/roman-numerals-formatting/24084#24084
% I (H.-Martin Münch) submitted this package to CTAN
% (after diverse updates of the code and writing a documentation).
%
% This work may be distributed and/or modified under the
% conditions of the LaTeX Project Public License, either
% version 1.3c of this license or (at your option) any later
% version. This version of this license is in
%    http://www.latex-project.org/lppl/lppl-1-3c.txt
% and the latest version of this license is in
%    http://www.latex-project.org/lppl.txt
% and version 1.3c or later is part of all distributions of
% LaTeX version 2005/12/01 or later.
%
% This work has the LPPL maintenance status "maintained".
%
% The Current Maintainer of this work is H.-Martin Muench.
%
% This work consists of the main source file romanbar.dtx,
% the README, and the derived files
%    romanbar.sty, romanbar.pdf,
%    romanbar.ins, romanbar.drv,
%    romanbar-example.tex, romanbar-example.pdf.
%
% Distribution:
%    CTAN:macros/latex/contrib/romanbar/romanbar.dtx
%    CTAN:macros/latex/contrib/romanbar/romanbar.pdf
%    CTAN:install/macros/latex/contrib/romanbar.tds.zip
%
% Cf. http://ctan.org/pkg/romanbar and
% http://www.tex.ac.uk/tex-archive/help/Catalogue/entries/romanbar.html
% (It might take some time for the entries to be created.)
%
% Unpacking:
%    (a) If romanbar.ins is present:
%           tex romanbar.ins
%    (b) Without romanbar.ins:
%           tex romanbar.dtx
%    (c) If you insist on using LaTeX
%           latex \let\install=y\input{romanbar.dtx}
%        (quote the arguments according to the demands of your shell)
%
% Documentation:
%    (a) If romanbar.drv is present:
%           (pdf)latex romanbar.drv
%           makeindex -s gind.ist romanbar.idx
%           (pdf)latex romanbar.drv
%           makeindex -s gind.ist romanbar.idx
%           (pdf)latex romanbar.drv
%    (b) Without romanbar.drv:
%           (pdf)latex romanbar.dtx
%           makeindex -s gind.ist romanbar.idx
%           (pdf)latex romanbar.dtx
%           makeindex -s gind.ist romanbar.idx
%           (pdf)latex romanbar.dtx
%
%    The class ltxdoc loads the configuration file ltxdoc.cfg
%    if available. Here you can specify further options, e.g.
%    use DIN A4 as paper format:
%       \PassOptionsToClass{a4paper}{article}
%
% Installation:
%    TDS:tex/latex/romanbar/romanbar.sty
%    TDS:doc/latex/romanbar/romanbar.pdf
%    TDS:doc/latex/romanbar/romanbar-example.tex
%    TDS:doc/latex/romanbar/romanbar-example.pdf
%    TDS:source/latex/romanbar/romanbar.dtx
%
%<*ignore>
\begingroup
  \catcode123=1 %
  \catcode125=2 %
  \def\x{LaTeX2e}%
\expandafter\endgroup
\ifcase 0\ifx\install y1\fi\expandafter
         \ifx\csname processbatchFile\endcsname\relax\else1\fi
         \ifx\fmtname\x\else 1\fi\relax
\else\csname fi\endcsname
%</ignore>
%<*install>
\input docstrip.tex
\Msg{********************************************************************}
\Msg{* Installation                                                     *}
\Msg{* Package: romanbar 2012/01/01 v1.0f Roman numbers with bars (HMM) *}
\Msg{********************************************************************}

\keepsilent
\askforoverwritefalse

\let\MetaPrefix\relax
\preamble

This is a generated file.

Project: romanbar
Version: 2012/01/01 v1.0f

Copyright (C) 2011 by
    H.-Martin M"unch <Martin dot Muench at Uni-Bonn dot de>

The usual disclaimer applies:
If it doesn't work right that's your problem.
(Nevertheless, send an e-mail to the maintainer
 when you find an error in this package.)

This work may be distributed and/or modified under the
conditions of the LaTeX Project Public License, either
version 1.3c of this license or (at your option) any later
version. This version of this license is in
   http://www.latex-project.org/lppl/lppl-1-3c.txt
and the latest version of this license is in
   http://www.latex-project.org/lppl.txt
and version 1.3c or later is part of all distributions of
LaTeX version 2005/12/01 or later.

This work has the LPPL maintenance status "maintained".

The Current Maintainer of this work is H.-Martin Muench.

This work consists of the main source file romanbar.dtx,
the README, and the derived files
   romanbar.sty, romanbar.pdf,
   romanbar.ins, romanbar.drv,
   romanbar-example.tex, romanbar-example.pdf.

\endpreamble
\let\MetaPrefix\DoubleperCent

\generate{%
  \file{romanbar.ins}{\from{romanbar.dtx}{install}}%
  \file{romanbar.drv}{\from{romanbar.dtx}{driver}}%
  \usedir{tex/latex/romanbar}%
  \file{romanbar.sty}{\from{romanbar.dtx}{package}}%
  \usedir{doc/latex/romanbar}%
  \file{romanbar-example.tex}{\from{romanbar.dtx}{example}}%
}

\catcode32=13\relax% active space
\let =\space%
\Msg{************************************************************************}
\Msg{*}
\Msg{* To finish the installation you have to move the following}
\Msg{* file into a directory searched by TeX:}
\Msg{*}
\Msg{*     romanbar.sty}
\Msg{*}
\Msg{* To produce the documentation run the file `romanbar.drv'}
\Msg{* through (pdf)LaTeX, e.g.}
\Msg{*  pdflatex romanbar.drv}
\Msg{*  makeindex -s gind.ist romanbar.idx}
\Msg{*  pdflatex romanbar.drv}
\Msg{*  makeindex -s gind.ist romanbar.idx}
\Msg{*  pdflatex romanbar.drv}
\Msg{*}
\Msg{* At least three runs are necessary e.g. to get the}
\Msg{*  references right!}
\Msg{*}
\Msg{* Happy TeXing!}
\Msg{*}
\Msg{************************************************************************}

\endbatchfile
%</install>
%<*ignore>
\fi
%</ignore>
%
% \section{The documentation driver file}
%
% The next bit of code contains the documentation driver file for
% \TeX{}, i.\,e., the file that will produce the documentation you
% are currently reading. It will be extracted from this file by the
% \texttt{docstrip} programme. That is, run \LaTeX{} on \texttt{docstrip}
% and specify the \texttt{driver} option when \texttt{docstrip}
% asks for options.
%
%    \begin{macrocode}
%<*driver>
\NeedsTeXFormat{LaTeX2e}[2011/06/27]
\ProvidesFile{romanbar.drv}%
  [2012/01/01 v1.0f Roman numbers with bars (HMM)]
\documentclass{ltxdoc}[2007/11/11]% v2.0u
\usepackage{holtxdoc}[2011/02/04]%  v0.21
%% romanbar may work with earlier versions of LaTeX2e and those
%% class and package, but this was not tested.
%% Please consider updating your LaTeX, class, and package
%% to the most recent version (if they are not already the most
%% recent version).
\hypersetup{%
 pdfsubject={Writing Roman numbers with bars (HMM)},%
 pdfkeywords={LaTeX, romanbar, roman, Roman, bars, H.-Martin Muench},%
 pdfencoding=auto,%
 pdflang={en},%
 breaklinks=true,%
 linktoc=all,%
 pdfstartview=FitH,%
 pdfpagelayout=OneColumn,%
 bookmarksnumbered=true,%
 bookmarksopen=true,%
 bookmarksopenlevel=2,%
 pdfmenubar=true,%
 pdftoolbar=true,%
 pdfwindowui=true,%
 pdfnewwindow=true%
}
\CodelineIndex
\hyphenation{docu-ment ana-lysed}
\gdef\unit#1{\mathord{\thinspace\mathrm{#1}}}%
\makeatletter
\@ifundefined{eTeX}{\gdef\eTeX{$\m@th \varepsilon $-\TeX }}{% else \relax
}
\makeatother
\begin{document}
  \DocInput{romanbar.dtx}%
\end{document}
%</driver>
%    \end{macrocode}
%
% \fi
%
% \CheckSum{179}
%
% \CharacterTable
%  {Upper-case    \A\B\C\D\E\F\G\H\I\J\K\L\M\N\O\P\Q\R\S\T\U\V\W\X\Y\Z
%   Lower-case    \a\b\c\d\e\f\g\h\i\j\k\l\m\n\o\p\q\r\s\t\u\v\w\x\y\z
%   Digits        \0\1\2\3\4\5\6\7\8\9
%   Exclamation   \!     Double quote  \"     Hash (number) \#
%   Dollar        \$     Percent       \%     Ampersand     \&
%   Acute accent  \'     Left paren    \(     Right paren   \)
%   Asterisk      \*     Plus          \+     Comma         \,
%   Minus         \-     Point         \.     Solidus       \/
%   Colon         \:     Semicolon     \;     Less than     \<
%   Equals        \=     Greater than  \>     Question mark \?
%   Commercial at \@     Left bracket  \[     Backslash     \\
%   Right bracket \]     Circumflex    \^     Underscore    \_
%   Grave accent  \`     Left brace    \{     Vertical bar  \|
%   Right brace   \}     Tilde         \~}
%
% \GetFileInfo{romanbar.drv}
%
% \begingroup
%   \def\x{\#,\$,\^,\_,\~,\ ,\&,\{,\},\%}%
%   \makeatletter
%   \@onelevel@sanitize\x
% \expandafter\endgroup
% \expandafter\DoNotIndex\expandafter{\x}
% \expandafter\DoNotIndex\expandafter{\string\ }
% \begingroup
%   \makeatletter
%     \lccode`9=32\relax
%     \lowercase{%^^A
%       \edef\x{\noexpand\DoNotIndex{\@backslashchar9}}%^^A
%     }%^^A
%   \expandafter\endgroup\x
%
% \DoNotIndex{\\,\,}
% \DoNotIndex{\documentclass,\usepackage,\ProvidesPackage,\begin,\end}
% \DoNotIndex{\MessageBreak}
% \DoNotIndex{\NeedsTeXFormat,\DoNotIndex,\verb}
% \DoNotIndex{\def,\edef,\gdef,\global}
% \DoNotIndex{\ifx,\listfiles,\mathord,\mathrm}
% \DoNotIndex{\bigskip,\space,\thinspace,\ldots}
% \DoNotIndex{\indent,\noindent,\newline,\linebreak,\pagebreak,\newpage}
% \DoNotIndex{\textbf,\textit,\textsf,\texttt,\textquotedblleft,\textquotedblright}
% \DoNotIndex{\plainTeX,\TeX,\LaTeX,\pdfLaTeX}
% \DoNotIndex{\item}
%
% \title{The \xpackage{romanbar} package}
% \date{2012/01/01 v1.0f}
% \author{H.-Martin M\"{u}nch\\\xemail{Martin.Muench at Uni-Bonn.de}}
%
% \maketitle
%
% \begin{abstract}
%  This \LaTeX{} package allows to write Roman numbers
%  (or any other text) with bars. (Additionally, commands for converting Arabic numbers
%  into Roman ones are provided and an |\ifnumeric| test function.)
% \end{abstract}
%
% \bigskip
%
% \noindent Disclaimer for web links: The author is not responsible for any contents
% referred to in this work unless he has full knowledge of illegal contents.
% If any damage occurs by the use of information presented there, only the
% author of the respective pages might be liable, not the one who has referred
% to these pages.
%
% \bigskip
%
% \noindent {\color{green} Save per page about $200\unit{ml}$ water,
% $2\unit{g}$ CO$_{2}$ and $2\unit{g}$ wood:\\
% Therefore please print only if this is really necessary.}
%
% \pagebreak
%
% \tableofcontents
%
% \pagebreak
%
% \section{Introduction}
%
% \indent This \LaTeX{} package allows to write Roman numbers
% (or any other text) with bars. (Additionally, command for converting Arabic numbers
% into Roman ones are provided and an |\ifnumeric| test function.)
%
% \bigskip
%
% \section{Usage}
%
% \indent Just load the package placing
% \begin{quote}
%   |\usepackage{romanbar}|
% \end{quote}
% \noindent in the preamble of your \LaTeXe{} source file.\\
% |\Romanbar{...}| then produces a Roman number with bars (please see the example file).
% |\ifnumeric{test}{true}{false}| tests for |test| being numeric.
% |\romannum{...}| turns an Arabic number into a lowercase Roman one, and
% |\Romannum{...}| turns an Arabic number into an uppercase Roman one.
%
% \subsection*{Options}
% \DescribeMacro{options}
% \indent The \xpackage{romanbar} package takes no options.
%
% \section{Alternatives\label{sec:Alternatives}}
%
% \begin{description}
% \item[-] the original code written by Prof. Enrico Gregorio\\
%   (\url{http://profs.sci.univr.it/~gregorio/}),\\
%   \url{http://tex.stackexchange.com/questions/24065/roman-numerals-formatting/24084#24084}
%\begin{verbatim}
%\def\barroman#1{\sbox0{#1}\dimen0=\dimexpr\wd0+1pt\relax
%  \makebox[\dimen0]{\rlap{\vrule width\dimen0 height 0.06ex depth 0.06ex}%
%    \rlap{\vrule width\dimen0 height\dimexpr\ht0+0.03ex\relax
%            depth\dimexpr-\ht0+0.09ex\relax}%
%    \kern.5pt#1\kern.5pt}}
%
%\barroman{I} \barroman{XI}
%\end{verbatim}
%
% \item[-]  Some fonts provide single characters for Roman numerals,
%   cf.~e.\,g.
% \url{http://tex.stackexchange.com/questions/38695/using-unicode-roman-numerals-in-xetex}.
% \end{description}
%
% \noindent (You programmed or found another alternative,
%  which is available at \CTAN{}?\\
%  OK, send an e-mail to me with the name, location at \CTAN{},
%  and a short notice, and I will probably include it in
%  the list above.)
%
% \newpage
%
% \section{Example}
%
%    \begin{macrocode}
%<*example>
\documentclass[british]{article}[2007/10/19]% v1.4h
%%%%%%%%%%%%%%%%%%%%%%%%%%%%%%%%%%%%%%%%%%%%%%%%%%%%%%%%%%%%%%%%%%%%%
\usepackage[extension=pdf,%
 plainpages=false,%
 pdfpagelabels=true,%
 hyperindex=false,%
 pdflang={en},%
 pdftitle={romanbar package example},%
 pdfauthor={H.-Martin Muench},%
 pdfsubject={Example for the romanbar package},%
 pdfkeywords={LaTeX, romanbar, roman, Roman, bars, H.-Martin Muench},%
 pdfview={XYZ null null 1},%
 pdfstartview={XYZ null null 1},%
 pdfpagelayout=SinglePage]{hyperref}[2011/12/04]% v6.82m
\usepackage{romanbar}[2012/01/01]% v1.0f
\gdef\unit#1{\mathord{\thinspace\mathrm{#1}}}%
\listfiles
\begin{document}
\pagenumbering{arabic}
\section*{Example for romanbar}

This example demonstrates the use of package\newline
\textsf{romanbar}, v1.0f as of 2012/01/01 (HMM).\newline
There are no options to be used.\newline

\noindent For more details please see the documentation!\newline

\noindent Save per page about $200\unit{ml}$ water,
$2\unit{g}$ CO$_{2}$ and $2\unit{g}$ wood:\newline
Therefore please print only if this is really necessary.\newline

\noindent This package provides the command \verb|\Romanbar|
to print bars below and over the following:

\begin{description}
\item[-] Roman numbers: \verb|\Romanbar{MMXII}| prints \Romanbar{MMXII}

\item[-] Arabic numbers turned into upper-case Roman numbers:\newline
           \verb|\Romanbar{2012}| prints \Romanbar{2012}

\item[-] negative Arabic numbers turned into upper-case Roman numbers
           \newline
           (although historically there were no negative Roman numbers):
           \newline
           \verb|\Romanbar{-12}| prints \Romanbar{-12}

\item[-] zero Arabic number ($0$; although historically
           there was no Roman zero):\newline
           \verb|\Romanbar{0}| prints \Romanbar{0}

\item[-] some arbitrary text:
           \verb|\Romanbar{Caesar}| prints \Romanbar{Caesar}\newline
           (with descenders: \Romanbar{AgjpqyW})


\newcounter{example}
\setcounter{example}{21}

\item[-] some counter's value:
           \verb|\Romanbar{\theexample}| prints \Romanbar{\theexample}
           \newline
           (where the value of \texttt{example} is \theexample)

\item[-] Arabic numbers, without turning them into upper-case Roman
           numbers:\newline
           \verb|\Romanbar{\relax 2012}| prints \Romanbar{\relax  2012}
\end{description}

Special care was taken for "L" (50), e.\,g. in 555/DLV: \Romanbar{555}.\\


\verb|\romannum{...}| converts an Arabic number into a lower-case Roman one,
e.\,g. \verb|\romannum{2012}| prints \romannum{2012}, and
\verb|\Romannum{...}| converts an Arabic number into an upper-case Roman one,
e.\,g. \verb|\Romannum{2012}| prints \Romannum{2012}.

\end{document}
%</example>
%    \end{macrocode}
%
% \StopEventually{}
%
% \newpage
%
% \section{The implementation}
%
% We start off by checking that we are loading into \LaTeXe{} and
% announcing the name and version of this package.
%
%    \begin{macrocode}
%<*package>
%    \end{macrocode}
%
%    \begin{macrocode}
\NeedsTeXFormat{LaTeX2e}[2011/06/27]
\ProvidesPackage{romanbar}[2012/01/01 v1.0f
              Roman numbers with bars (HMM)]

%    \end{macrocode}
%
% A short description of the \xpackage{romanbar} package:
%
%    \begin{macrocode}
%% Allows to write Roman numbers (or any other text) with bars
%% and to convert Arabic numbers into Roman ones.
%    \end{macrocode}
%
% A last information for the user:
%
%    \begin{macrocode}
%% romanbar may work with earlier versions of LaTeX,
%% but this was not tested. Please consider updating
%% your LaTeX to the most recent version
%% (if it is not already the most recent version).

%    \end{macrocode}
%
% \indent See subsection~\ref{ss:Downloads} about how to get it.\\
%
% There are no options to be processed.
%
% A command to gobble a possible minus sign is needed and
% |\gobbleminus| from \url{http://www.tex.ac.uk/cgi-bin/texfaq2html?label=isitanum}
% is used.
%
%    \begin{macrocode}
\def\gobbleminus#1{\ifx-#1\else#1\fi}
%% from http://www.tex.ac.uk/cgi-bin/texfaq2html?label=isitanum

%    \end{macrocode}
%
% We want to test whether the argument passed to |\Romanbar| is numeric
% and therefore define |\ifnumeric{test}{true}{false}|,
% where |test| is to be analysed, |true| is the code to be executed,
% if |test| is numeric, and |false| is the code to be executed,
% if |test| is not numeric. This is done similar to\\
% \url{http://tex.stackexchange.com/a/17119}.
%
%    \begin{macrocode}
\newcommand{\ifnumeric}[3]{%
%% similar to http://tex.stackexchange.com/a/17119
\sbox\z@{\@tempcnta=0\gobbleminus#1\relax}%
\ifdim\wd0>\z@\relax#3% is not numeric
\else#2% is numeric
\fi%
}

%    \end{macrocode}
%
% We define |\r@iseL| to raise any L (50), otherwise a good deal
% of the horizontal part of the letter would be swallowed by the lower bar
% under the \textquotedblleft{}number\textquotedblright{}.
%
%    \begin{macrocode}
\def\r@iseL#1{\ifx @#1% then terminate
\else%
\if L#1\raisebox{0.05ex}{L}\else #1\fi%
\expandafter\r@iseL%
\fi%
}

%    \end{macrocode}
%
% \pagebreak
%
% We define the |\Romanbar| command.
%
%    \begin{macrocode}
\DeclareRobustCommand{\Romanbar}[1]{%
\ifnumeric{#1}{% is numeric
%    \end{macrocode}
%
% If its parameter is numeric, the temporary counter |\@tempcnta| is set to the value.
%
%    \begin{macrocode}
\@tempcnta=#1\relax%
%    \end{macrocode}
%
% Normally there are no non-positive Roman numbers, therefore we do some trick
% similar to |\XRoman| from the \url{http://ctan.org/pkg/pageslts} package.
%
%    \begin{macrocode}
%% similar to \XRoman from the http://ctan.org/pkg/pageslts package
\ifnum\@tempcnta<1%
  \ifnum #1>0%
    \relax \Roman@bar{\Roman{#1}}%
  \else%
    \ifnum #1<0%
     -\Roman@bar{\expandafter\@slowromancap\romannumeral\number-#1@%
     }%
    \else%
      \Roman@bar{0}%
    \fi%
  \fi%
\else\Roman@bar{\@Roman\@tempcnta}%
\fi%
}{% is not numeric
\Roman@bar{#1}%
}%
}

%    \end{macrocode}
%
% Whether it is numeric or not, in the end |\Roman@bar| is applied to it.
%
%    \begin{macrocode}
\DeclareRobustCommand{\Roman@bar}[1]{% #1 is in Roman, i.e. MMXII
%% (or any other text, "Caesar" would work, too).
%% similar to code of  by Prof. Enrico Gregorio (egreg) at
%% http://tex.stackexchange.com/questions/24065/roman-numerals-formatting/24084#24084
\@bsphack%
\edef\romanbartmp{#1}%
%% height:
\sbox0{\textsf{\romanbartmp}}%
%% top line:
%% + top of top line:
\@tempdima=\ht0%
\advance\@tempdima+0.05ex%0.03
\dimen1=\@tempdima%
%% + bottom of top line:
\@tempdima=-\ht0%
\advance\@tempdima+0.05ex%0.07
\dimen2=\@tempdima%
%% width:
\sbox0{\textsf{\expandafter\r@iseL\romanbartmp @}}%
\@tempdima=\wd0%
\advance\@tempdima+1pt%
\dimen0=\@tempdima%
\@esphack%
\makebox[\dimen0]{%
\rlap{\vrule width\dimen0 height\dimen1 depth\dimen2}% top    line
\rlap{\vrule width\dimen0 height 0.06ex depth 0.03ex}% bottom line
\kern0.5pt\textsf{\expandafter\r@iseL\romanbartmp @}\kern0.5pt}%
}

%    \end{macrocode}
%
% While it is not needed for the bars, it is sometimes asked for a command
% to convert an Arabic number into a Roman one. |\roman{...}| can only be applied
% to a counter, i.\,e. |\roman{12}| does not work, but
%\begin{verbatim}
%\newcounter{examplecounter}
%\setcounter{examplecounter}{12}
%\roman{examplecounter}
%\end{verbatim}
% would be necessary. Further non-negative values would not work.
% |\romannum| accepts any number as argument. (If the argument is no number,
% an error message is given and the argument printed (executed)
%  without applying anything to it.)
%
%    \begin{macrocode}
\DeclareRobustCommand{\romannum}[1]{%
\ifnumeric{#1}{% is numeric
\@tempcnta=#1\relax%
%% similar to \xroman from the http://ctan.org/pkg/pageslts package
\ifnum\@tempcnta<1%
  \ifnum #1>0%
    \relax \roman{#1}%
  \else%
    \ifnum #1<0%
     -\romannumeral\number-#1@%
    \else%
      0%
    \fi%
  \fi%
\else\@roman\@tempcnta%
\fi%
}{% is not numeric
\PackageError{romanbar}{%
Argument of \string\romannum\space is not a number}{%
The command \string\romannum\space converts an Arabic number into a%
 lower-case Roman one,\MessageBreak%
but the used argument of \string\romannum\space is not an Arabic number%
 but\MessageBreak%
`#1`,\MessageBreak%
which will now be printed unchanged.}%
#1%
}%
}

%    \end{macrocode}
%
% The same for upper-case Roman numbers, |\Romannum| instead of |\Roman|:
%
%    \begin{macrocode}
\DeclareRobustCommand{\Romannum}[1]{%
\ifnumeric{#1}{% is numeric
\@tempcnta=#1\relax%
%% similar to \XRoman from the http://ctan.org/pkg/pageslts package
\ifnum\@tempcnta<1%
  \ifnum #1>0%
    \relax \Roman{#1}%
  \else%
    \ifnum #1<0%
     -\expandafter\@slowromancap\romannumeral\number-#1@%
    \else%
      0%
    \fi%
  \fi%
\else\@Roman\@tempcnta%
\fi%
}{% is not numeric
\PackageError{romanbar}{%
Argument of \string\Romannum\space is not a number}{%
The command \string\Romannum\space converts an Arabic number into an%
 upper-case Roman one,\MessageBreak%
but the used argument of \string\Romannum\space is not an Arabic number%
 but\MessageBreak%
`#1`.}%
#1%
}%
}

%    \end{macrocode}
%
%    \begin{macrocode}
%</package>
%    \end{macrocode}
%
% \newpage
%
% \section{Installation}
%
% \subsection{Downloads\label{ss:Downloads}}
%
% Everything should be available on \CTAN{}, \url{http://www.ctan.org/tex-archive/},
% but may need additional packages themselves.\\
%
% \DescribeMacro{romanbar.dtx}
% For unpacking the |romanbar.dtx| file and constructing the documentation
% it is required:
% \begin{description}
% \item[-] \TeX{} Format \LaTeXe{}, \url{http://www.CTAN.org/}
%
% \item[-] document class \xpackage{ltxdoc}, 2007/11/11, v2.0u,
%           \url{http://ctan.org/pkg/ltxdoc}
%
% \item[-] package \xpackage{holtxdoc}, 2011/02/04, v0.21,
%           \url{http://ctan.org/pkg/holtxdoc}
%
% \item[-] package \xpackage{hypdoc}, 2010/03/26, v1.9,
%           \url{http://ctan.org/pkg/hypdoc}
% \end{description}
%
% \DescribeMacro{romanbar.sty}
% The \texttt{romanbar.sty} for \LaTeXe{} (i.\,e.~each document using
% the \xpackage{romanbar} package) requires:
% \begin{description}
% \item[-] \TeX Format \LaTeXe{}, \url{http://www.CTAN.org/}
% \end{description}
%
% \DescribeMacro{romanbar-example.tex}
% The \texttt{romanbar-example.tex} requires the same files as all
% documents using the \xpackage{romanbar} package and additionally:
% \begin{description}
% \item[-] class \xpackage{article}, 2007/10/19, v1.4h, from \xpackage{classes.dtx}:\\
%           \CTAN{macros/latex/base/classes.dtx}
%
% \item[-] package \xpackage{romanbar}, 2012/01/01, v1.0f,\\
%           %%\url{http://ctan.org/pkg/romanbar}\\
%           \CTAN{macros/latex/contrib/romanbar/}
%   (Well, it is the example file for this package, and because you are reading the
%    documentation for the \xpackage{romanbar} package, it can be assumed that you already
%    have some version of it -- is it the current one?)
% \end{description}
%
% \DescribeMacro{Alternative}
% As possible alternative package in section \ref{sec:Alternatives} there is listed
% \begin{description}
% \item[-] none.
% \end{description}
%
% \DescribeMacro{Oberdiek}
% \DescribeMacro{holtxdoc}
% All packages of \textsc{Heiko Oberdiek's} bundle `oberdiek'
% (especially \xpackage{holtxdoc}) are also available in a TDS compliant ZIP archive:\\
% \CTAN{install/macros/latex/contrib/oberdiek.tds.zip}.\\
% It is probably best to download and use this, because the packages in there
% are quite probably both recent and compatible among themselves.\\
%
% \DescribeMacro{hyperref}
% \noindent \xpackage{hyperref} is not included in that bundle and needs to be
% downloaded separately,\\
% \url{http://mirror.ctan.org/install/macros/latex/contrib/hyperref.tds.zip}.\\
%
% \DescribeMacro{M\"{u}nch}
% A hyperlinked list of my (other) packages can be found at
% \url{http://www.Uni-Bonn.de/~uzs5pv/LaTeX.html}.\\
%
% \pagebreak
%
% \subsection{Package, unpacking TDS}
% \paragraph{Package.} This package should become available on \CTAN{} soon:
% \begin{description}
% \item[\CTAN{macros/latex/contrib/romanbar/romanbar.dtx}]\hspace*{0.1cm} \\
%       The source file.
% \item[\CTAN{macros/latex/contrib/romanbar/romanbar.pdf}]\hspace*{0.1cm} \\
%       The documentation.
% \item[\CTAN{macros/latex/contrib/romanbar/romanbar-example.pdf}]\hspace*{0.1cm} \\
%       The compiled example file, as it should look like.
% \item[\CTAN{macros/latex/contrib/romanbar/README}]\hspace*{0.1cm} \\
%       The README file.
% \end{description}
% There is also a romanbar.tds.zip available:
% \begin{description}
% \item[\CTAN{install/macros/latex/contrib/romanbar.tds.zip}]\hspace*{0.1cm} \\
%       Everything in \xfile{TDS} compliant, compiled format.
% \end{description}
% which additionally contains\\
% \begin{tabular}{ll}
% romanbar.ins & The installation file.\\
% romanbar.drv & The driver to generate the documentation.\\
% romanbar.sty & The \xext{sty}le file.\\
% romanbar-example.tex & The example file.
% \end{tabular}
%
% \bigskip
%
% \noindent For required other packages, please see the preceding subsection.
%
% \paragraph{Unpacking.} The \xfile{.dtx} file is a self-extracting
% \docstrip{} archive. The files are extracted by running the
% \xext{.dtx} through \plainTeX{}:
% \begin{quote}
%   \verb|tex romanbar.dtx|
% \end{quote}
%
% About generating the documentation see paragraph~\ref{GenDoc} below.\\
%
% \paragraph{TDS.} Now the different files must be moved into
% the different directories in your installation TDS tree
% (also known as \xfile{texmf} tree):
% \begin{quote}
% \def\t{^^A
% \begin{tabular}{@{}>{\ttfamily}l@{ $\rightarrow$ }>{\ttfamily}l@{}}
%   romanbar.sty & tex/latex/romanbar/romanbar.sty\\
%   romanbar.pdf & doc/latex/romanbar/romanbar.pdf\\
%   romanbar-example.tex & doc/latex/romanbar/romanbar-example.tex\\
%   romanbar-example.pdf & doc/latex/romanbar/romanbar-example.pdf\\
%   romanbar.dtx & source/latex/romanbar/romanbar.dtx\\
% \end{tabular}^^A
% }^^A
% \sbox0{\t}^^A
% \ifdim\wd0>\linewidth
%   \begingroup
%     \advance\linewidth by\leftmargin
%     \advance\linewidth by\rightmargin
%   \edef\x{\endgroup
%     \def\noexpand\lw{\the\linewidth}^^A
%   }\x
%   \def\lwbox{^^A
%     \leavevmode
%     \hbox to \linewidth{^^A
%       \kern-\leftmargin\relax
%       \hss
%       \usebox0
%       \hss
%       \kern-\rightmargin\relax
%     }^^A
%   }^^A
%   \ifdim\wd0>\lw
%     \sbox0{\small\t}^^A
%     \ifdim\wd0>\linewidth
%       \ifdim\wd0>\lw
%         \sbox0{\footnotesize\t}^^A
%         \ifdim\wd0>\linewidth
%           \ifdim\wd0>\lw
%             \sbox0{\scriptsize\t}^^A
%             \ifdim\wd0>\linewidth
%               \ifdim\wd0>\lw
%                 \sbox0{\tiny\t}^^A
%                 \ifdim\wd0>\linewidth
%                   \lwbox
%                 \else
%                   \usebox0
%                 \fi
%               \else
%                 \lwbox
%               \fi
%             \else
%               \usebox0
%             \fi
%           \else
%             \lwbox
%           \fi
%         \else
%           \usebox0
%         \fi
%       \else
%         \lwbox
%       \fi
%     \else
%       \usebox0
%     \fi
%   \else
%     \lwbox
%   \fi
% \else
%   \usebox0
% \fi
% \end{quote}
% If you have a \xfile{docstrip.cfg} that configures and enables \docstrip{}'s
% \xfile{TDS} installing feature, then some files can already be in the right
% place, see the documentation of \docstrip{}.
%
% \subsection{Refresh file name databases}
%
% If your \TeX{}~distribution (\teTeX{}, \mikTeX{},\dots{}) relies on file name
% databases, you must refresh these. For example, \teTeX{} users run
% \verb|texhash| or \verb|mktexlsr|.
%
% \subsection{Some details for the interested}
%
% \paragraph{Unpacking with \LaTeX{}.}
% The \xfile{.dtx} chooses its action depending on the format:
% \begin{description}
% \item[\plainTeX:] Run \docstrip{} and extract the files.
% \item[\LaTeX:] Generate the documentation.
% \end{description}
% If you insist on using \LaTeX{} for \docstrip{} (really,
% \docstrip{} does not need \LaTeX{}), then inform the autodetect routine
% about your intention:
% \begin{quote}
%   \verb|latex \let\install=y\input{romanbar.dtx}|
% \end{quote}
% Do not forget to quote the argument according to the demands
% of your shell.
%
% \paragraph{Generating the documentation.\label{GenDoc}}
% You can use both the \xfile{.dtx} or the \xfile{.drv} to generate
% the documentation. The process can be configured by a
% configuration file \xfile{ltxdoc.cfg}. For instance, put the following
% line into this file, if you want to have A4 as paper format:
% \begin{quote}
%   \verb|\PassOptionsToClass{a4paper}{article}|
% \end{quote}
%
% \noindent An example follows how to generate the
% documentation with \pdfLaTeX{}:
%
% \begin{quote}
%\begin{verbatim}
%pdflatex romanbar.dtx
%makeindex -s gind.ist romanbar.idx
%pdflatex romanbar.dtx
%makeindex -s gind.ist romanbar.idx
%pdflatex romanbar.dtx
%\end{verbatim}
% \end{quote}
%
% \subsection{Compiling the example}
%
% The example file, \textsf{romanbar-example.tex}, can be compiled via\\
% |(pdf)(la)tex romanbar-example.tex|.
%
% \section{Acknowledgements}
%
% I would like to thank \textsc{Heiko Oberdiek} for providing
% the \xpackage{hyperref} as well as a~lot~(!) of other useful packages
% (from which I also got everything I know about creating a file in \xext{dtx}
% format, ok, say it: copying),
% Prof.~\textsc{Enrico Gregorio} (egreg) for his answer
% \url{http://tex.stackexchange.com/questions/24065/roman-numerals-formatting/24084#24084},
% the \Newsgroup{comp.text.tex} and \Newsgroup{de.comp.text.tex}
% newsgroups as well as \url{http://tex.stackexchange.com}
% for their help in all things \TeX{}.
%
% \pagebreak
%
% \phantomsection
% \begin{History}\label{History}
%   \begin{Version}{2011/07/25 v1.0a}
%     \item First code by Prof.~\textsc{Enrico Gregorio} (egreg) at
%             \url{http://tex.stackexchange.com/questions/24065/roman-numerals-formatting/24084#24084}
%   \end{Version}
%   \begin{Version}{2011/07/26 v1.0b}
%     \item Packed into a \xext{sty} file.
%     \item Removed the requirement of \eTeX{} for the package.
%   \end{Version}
%   \begin{Version}{2011/08/25 v1.0c}
%     \item Renamed \xpackage{romanbar} package.
%     \item Minor details.
%   \end{Version}
%   \begin{Version}{2011/09/16 v1.0d}
%     \item Made commands robust.
%     \item Minor details.
%   \end{Version}
%   \begin{Version}{2011/12/16 v1.0e}
%     \item Additionally accepts now Arabic numbers and converts them to Roman ones with bars.
%     \item Also non-positive numbers are now accepted.
%   \end{Version}
%   \begin{Version}{2012/01/01 v1.0f}
%     \item Added |\romannum| and |\Romannum| to convert Arabic to Roman numbers (without bars).
%     \item |\r@iseL|.
%     \item Created a \xfile{dtx}, example, and README.
%     \item Upload to \CTAN{}.
%   \end{Version}
% \end{History}
%
% \bigskip
%
% When you find a mistake or have a suggestion for an improvement of this package,
% please send an e-mail to the maintainer, thanks! (Please see BUG REPORTS in the README.)
%
% \pagebreak
%
% \PrintIndex
%
% \Finale
\endinput|
% \end{quote}
% Do not forget to quote the argument according to the demands
% of your shell.
%
% \paragraph{Generating the documentation.\label{GenDoc}}
% You can use both the \xfile{.dtx} or the \xfile{.drv} to generate
% the documentation. The process can be configured by a
% configuration file \xfile{ltxdoc.cfg}. For instance, put the following
% line into this file, if you want to have A4 as paper format:
% \begin{quote}
%   \verb|\PassOptionsToClass{a4paper}{article}|
% \end{quote}
%
% \noindent An example follows how to generate the
% documentation with \pdfLaTeX{}:
%
% \begin{quote}
%\begin{verbatim}
%pdflatex romanbar.dtx
%makeindex -s gind.ist romanbar.idx
%pdflatex romanbar.dtx
%makeindex -s gind.ist romanbar.idx
%pdflatex romanbar.dtx
%\end{verbatim}
% \end{quote}
%
% \subsection{Compiling the example}
%
% The example file, \textsf{romanbar-example.tex}, can be compiled via\\
% |(pdf)(la)tex romanbar-example.tex|.
%
% \section{Acknowledgements}
%
% I would like to thank \textsc{Heiko Oberdiek} for providing
% the \xpackage{hyperref} as well as a~lot~(!) of other useful packages
% (from which I also got everything I know about creating a file in \xext{dtx}
% format, ok, say it: copying),
% Prof.~\textsc{Enrico Gregorio} (egreg) for his answer
% \url{http://tex.stackexchange.com/questions/24065/roman-numerals-formatting/24084#24084},
% the \Newsgroup{comp.text.tex} and \Newsgroup{de.comp.text.tex}
% newsgroups as well as \url{http://tex.stackexchange.com}
% for their help in all things \TeX{}.
%
% \pagebreak
%
% \phantomsection
% \begin{History}\label{History}
%   \begin{Version}{2011/07/25 v1.0a}
%     \item First code by Prof.~\textsc{Enrico Gregorio} (egreg) at
%             \url{http://tex.stackexchange.com/questions/24065/roman-numerals-formatting/24084#24084}
%   \end{Version}
%   \begin{Version}{2011/07/26 v1.0b}
%     \item Packed into a \xext{sty} file.
%     \item Removed the requirement of \eTeX{} for the package.
%   \end{Version}
%   \begin{Version}{2011/08/25 v1.0c}
%     \item Renamed \xpackage{romanbar} package.
%     \item Minor details.
%   \end{Version}
%   \begin{Version}{2011/09/16 v1.0d}
%     \item Made commands robust.
%     \item Minor details.
%   \end{Version}
%   \begin{Version}{2011/12/16 v1.0e}
%     \item Additionally accepts now Arabic numbers and converts them to Roman ones with bars.
%     \item Also non-positive numbers are now accepted.
%   \end{Version}
%   \begin{Version}{2012/01/01 v1.0f}
%     \item Added |\romannum| and |\Romannum| to convert Arabic to Roman numbers (without bars).
%     \item |\r@iseL|.
%     \item Created a \xfile{dtx}, example, and README.
%     \item Upload to \CTAN{}.
%   \end{Version}
% \end{History}
%
% \bigskip
%
% When you find a mistake or have a suggestion for an improvement of this package,
% please send an e-mail to the maintainer, thanks! (Please see BUG REPORTS in the README.)
%
% \pagebreak
%
% \PrintIndex
%
% \Finale
\endinput|
% \end{quote}
% Do not forget to quote the argument according to the demands
% of your shell.
%
% \paragraph{Generating the documentation.\label{GenDoc}}
% You can use both the \xfile{.dtx} or the \xfile{.drv} to generate
% the documentation. The process can be configured by a
% configuration file \xfile{ltxdoc.cfg}. For instance, put the following
% line into this file, if you want to have A4 as paper format:
% \begin{quote}
%   \verb|\PassOptionsToClass{a4paper}{article}|
% \end{quote}
%
% \noindent An example follows how to generate the
% documentation with \pdfLaTeX{}:
%
% \begin{quote}
%\begin{verbatim}
%pdflatex romanbar.dtx
%makeindex -s gind.ist romanbar.idx
%pdflatex romanbar.dtx
%makeindex -s gind.ist romanbar.idx
%pdflatex romanbar.dtx
%\end{verbatim}
% \end{quote}
%
% \subsection{Compiling the example}
%
% The example file, \textsf{romanbar-example.tex}, can be compiled via\\
% |(pdf)(la)tex romanbar-example.tex|.
%
% \section{Acknowledgements}
%
% I would like to thank \textsc{Heiko Oberdiek} for providing
% the \xpackage{hyperref} as well as a~lot~(!) of other useful packages
% (from which I also got everything I know about creating a file in \xext{dtx}
% format, ok, say it: copying),
% Prof.~\textsc{Enrico Gregorio} (egreg) for his answer
% \url{http://tex.stackexchange.com/questions/24065/roman-numerals-formatting/24084#24084},
% the \Newsgroup{comp.text.tex} and \Newsgroup{de.comp.text.tex}
% newsgroups as well as \url{http://tex.stackexchange.com}
% for their help in all things \TeX{}.
%
% \pagebreak
%
% \phantomsection
% \begin{History}\label{History}
%   \begin{Version}{2011/07/25 v1.0a}
%     \item First code by Prof.~\textsc{Enrico Gregorio} (egreg) at
%             \url{http://tex.stackexchange.com/questions/24065/roman-numerals-formatting/24084#24084}
%   \end{Version}
%   \begin{Version}{2011/07/26 v1.0b}
%     \item Packed into a \xext{sty} file.
%     \item Removed the requirement of \eTeX{} for the package.
%   \end{Version}
%   \begin{Version}{2011/08/25 v1.0c}
%     \item Renamed \xpackage{romanbar} package.
%     \item Minor details.
%   \end{Version}
%   \begin{Version}{2011/09/16 v1.0d}
%     \item Made commands robust.
%     \item Minor details.
%   \end{Version}
%   \begin{Version}{2011/12/16 v1.0e}
%     \item Additionally accepts now Arabic numbers and converts them to Roman ones with bars.
%     \item Also non-positive numbers are now accepted.
%   \end{Version}
%   \begin{Version}{2012/01/01 v1.0f}
%     \item Added |\romannum| and |\Romannum| to convert Arabic to Roman numbers (without bars).
%     \item |\r@iseL|.
%     \item Created a \xfile{dtx}, example, and README.
%     \item Upload to \CTAN{}.
%   \end{Version}
% \end{History}
%
% \bigskip
%
% When you find a mistake or have a suggestion for an improvement of this package,
% please send an e-mail to the maintainer, thanks! (Please see BUG REPORTS in the README.)
%
% \pagebreak
%
% \PrintIndex
%
% \Finale
\endinput
%        (quote the arguments according to the demands of your shell)
%
% Documentation:
%    (a) If romanbar.drv is present:
%           (pdf)latex romanbar.drv
%           makeindex -s gind.ist romanbar.idx
%           (pdf)latex romanbar.drv
%           makeindex -s gind.ist romanbar.idx
%           (pdf)latex romanbar.drv
%    (b) Without romanbar.drv:
%           (pdf)latex romanbar.dtx
%           makeindex -s gind.ist romanbar.idx
%           (pdf)latex romanbar.dtx
%           makeindex -s gind.ist romanbar.idx
%           (pdf)latex romanbar.dtx
%
%    The class ltxdoc loads the configuration file ltxdoc.cfg
%    if available. Here you can specify further options, e.g.
%    use DIN A4 as paper format:
%       \PassOptionsToClass{a4paper}{article}
%
% Installation:
%    TDS:tex/latex/romanbar/romanbar.sty
%    TDS:doc/latex/romanbar/romanbar.pdf
%    TDS:doc/latex/romanbar/romanbar-example.tex
%    TDS:doc/latex/romanbar/romanbar-example.pdf
%    TDS:source/latex/romanbar/romanbar.dtx
%
%<*ignore>
\begingroup
  \catcode123=1 %
  \catcode125=2 %
  \def\x{LaTeX2e}%
\expandafter\endgroup
\ifcase 0\ifx\install y1\fi\expandafter
         \ifx\csname processbatchFile\endcsname\relax\else1\fi
         \ifx\fmtname\x\else 1\fi\relax
\else\csname fi\endcsname
%</ignore>
%<*install>
\input docstrip.tex
\Msg{********************************************************************}
\Msg{* Installation                                                     *}
\Msg{* Package: romanbar 2012/01/01 v1.0f Roman numbers with bars (HMM) *}
\Msg{********************************************************************}

\keepsilent
\askforoverwritefalse

\let\MetaPrefix\relax
\preamble

This is a generated file.

Project: romanbar
Version: 2012/01/01 v1.0f

Copyright (C) 2011 by
    H.-Martin M"unch <Martin dot Muench at Uni-Bonn dot de>

The usual disclaimer applies:
If it doesn't work right that's your problem.
(Nevertheless, send an e-mail to the maintainer
 when you find an error in this package.)

This work may be distributed and/or modified under the
conditions of the LaTeX Project Public License, either
version 1.3c of this license or (at your option) any later
version. This version of this license is in
   http://www.latex-project.org/lppl/lppl-1-3c.txt
and the latest version of this license is in
   http://www.latex-project.org/lppl.txt
and version 1.3c or later is part of all distributions of
LaTeX version 2005/12/01 or later.

This work has the LPPL maintenance status "maintained".

The Current Maintainer of this work is H.-Martin Muench.

This work consists of the main source file romanbar.dtx,
the README, and the derived files
   romanbar.sty, romanbar.pdf,
   romanbar.ins, romanbar.drv,
   romanbar-example.tex, romanbar-example.pdf.

\endpreamble
\let\MetaPrefix\DoubleperCent

\generate{%
  \file{romanbar.ins}{\from{romanbar.dtx}{install}}%
  \file{romanbar.drv}{\from{romanbar.dtx}{driver}}%
  \usedir{tex/latex/romanbar}%
  \file{romanbar.sty}{\from{romanbar.dtx}{package}}%
  \usedir{doc/latex/romanbar}%
  \file{romanbar-example.tex}{\from{romanbar.dtx}{example}}%
}

\catcode32=13\relax% active space
\let =\space%
\Msg{************************************************************************}
\Msg{*}
\Msg{* To finish the installation you have to move the following}
\Msg{* file into a directory searched by TeX:}
\Msg{*}
\Msg{*     romanbar.sty}
\Msg{*}
\Msg{* To produce the documentation run the file `romanbar.drv'}
\Msg{* through (pdf)LaTeX, e.g.}
\Msg{*  pdflatex romanbar.drv}
\Msg{*  makeindex -s gind.ist romanbar.idx}
\Msg{*  pdflatex romanbar.drv}
\Msg{*  makeindex -s gind.ist romanbar.idx}
\Msg{*  pdflatex romanbar.drv}
\Msg{*}
\Msg{* At least three runs are necessary e.g. to get the}
\Msg{*  references right!}
\Msg{*}
\Msg{* Happy TeXing!}
\Msg{*}
\Msg{************************************************************************}

\endbatchfile
%</install>
%<*ignore>
\fi
%</ignore>
%
% \section{The documentation driver file}
%
% The next bit of code contains the documentation driver file for
% \TeX{}, i.\,e., the file that will produce the documentation you
% are currently reading. It will be extracted from this file by the
% \texttt{docstrip} programme. That is, run \LaTeX{} on \texttt{docstrip}
% and specify the \texttt{driver} option when \texttt{docstrip}
% asks for options.
%
%    \begin{macrocode}
%<*driver>
\NeedsTeXFormat{LaTeX2e}[2011/06/27]
\ProvidesFile{romanbar.drv}%
  [2012/01/01 v1.0f Roman numbers with bars (HMM)]
\documentclass{ltxdoc}[2007/11/11]% v2.0u
\usepackage{holtxdoc}[2011/02/04]%  v0.21
%% romanbar may work with earlier versions of LaTeX2e and those
%% class and package, but this was not tested.
%% Please consider updating your LaTeX, class, and package
%% to the most recent version (if they are not already the most
%% recent version).
\hypersetup{%
 pdfsubject={Writing Roman numbers with bars (HMM)},%
 pdfkeywords={LaTeX, romanbar, roman, Roman, bars, H.-Martin Muench},%
 pdfencoding=auto,%
 pdflang={en},%
 breaklinks=true,%
 linktoc=all,%
 pdfstartview=FitH,%
 pdfpagelayout=OneColumn,%
 bookmarksnumbered=true,%
 bookmarksopen=true,%
 bookmarksopenlevel=2,%
 pdfmenubar=true,%
 pdftoolbar=true,%
 pdfwindowui=true,%
 pdfnewwindow=true%
}
\CodelineIndex
\hyphenation{docu-ment ana-lysed}
\gdef\unit#1{\mathord{\thinspace\mathrm{#1}}}%
\makeatletter
\@ifundefined{eTeX}{\gdef\eTeX{$\m@th \varepsilon $-\TeX }}{% else \relax
}
\makeatother
\begin{document}
  \DocInput{romanbar.dtx}%
\end{document}
%</driver>
%    \end{macrocode}
%
% \fi
%
% \CheckSum{179}
%
% \CharacterTable
%  {Upper-case    \A\B\C\D\E\F\G\H\I\J\K\L\M\N\O\P\Q\R\S\T\U\V\W\X\Y\Z
%   Lower-case    \a\b\c\d\e\f\g\h\i\j\k\l\m\n\o\p\q\r\s\t\u\v\w\x\y\z
%   Digits        \0\1\2\3\4\5\6\7\8\9
%   Exclamation   \!     Double quote  \"     Hash (number) \#
%   Dollar        \$     Percent       \%     Ampersand     \&
%   Acute accent  \'     Left paren    \(     Right paren   \)
%   Asterisk      \*     Plus          \+     Comma         \,
%   Minus         \-     Point         \.     Solidus       \/
%   Colon         \:     Semicolon     \;     Less than     \<
%   Equals        \=     Greater than  \>     Question mark \?
%   Commercial at \@     Left bracket  \[     Backslash     \\
%   Right bracket \]     Circumflex    \^     Underscore    \_
%   Grave accent  \`     Left brace    \{     Vertical bar  \|
%   Right brace   \}     Tilde         \~}
%
% \GetFileInfo{romanbar.drv}
%
% \begingroup
%   \def\x{\#,\$,\^,\_,\~,\ ,\&,\{,\},\%}%
%   \makeatletter
%   \@onelevel@sanitize\x
% \expandafter\endgroup
% \expandafter\DoNotIndex\expandafter{\x}
% \expandafter\DoNotIndex\expandafter{\string\ }
% \begingroup
%   \makeatletter
%     \lccode`9=32\relax
%     \lowercase{%^^A
%       \edef\x{\noexpand\DoNotIndex{\@backslashchar9}}%^^A
%     }%^^A
%   \expandafter\endgroup\x
%
% \DoNotIndex{\\,\,}
% \DoNotIndex{\documentclass,\usepackage,\ProvidesPackage,\begin,\end}
% \DoNotIndex{\MessageBreak}
% \DoNotIndex{\NeedsTeXFormat,\DoNotIndex,\verb}
% \DoNotIndex{\def,\edef,\gdef,\global}
% \DoNotIndex{\ifx,\listfiles,\mathord,\mathrm}
% \DoNotIndex{\bigskip,\space,\thinspace,\ldots}
% \DoNotIndex{\indent,\noindent,\newline,\linebreak,\pagebreak,\newpage}
% \DoNotIndex{\textbf,\textit,\textsf,\texttt,\textquotedblleft,\textquotedblright}
% \DoNotIndex{\plainTeX,\TeX,\LaTeX,\pdfLaTeX}
% \DoNotIndex{\item}
%
% \title{The \xpackage{romanbar} package}
% \date{2012/01/01 v1.0f}
% \author{H.-Martin M\"{u}nch\\\xemail{Martin.Muench at Uni-Bonn.de}}
%
% \maketitle
%
% \begin{abstract}
%  This \LaTeX{} package allows to write Roman numbers
%  (or any other text) with bars. (Additionally, commands for converting Arabic numbers
%  into Roman ones are provided and an |\ifnumeric| test function.)
% \end{abstract}
%
% \bigskip
%
% \noindent Disclaimer for web links: The author is not responsible for any contents
% referred to in this work unless he has full knowledge of illegal contents.
% If any damage occurs by the use of information presented there, only the
% author of the respective pages might be liable, not the one who has referred
% to these pages.
%
% \bigskip
%
% \noindent {\color{green} Save per page about $200\unit{ml}$ water,
% $2\unit{g}$ CO$_{2}$ and $2\unit{g}$ wood:\\
% Therefore please print only if this is really necessary.}
%
% \pagebreak
%
% \tableofcontents
%
% \pagebreak
%
% \section{Introduction}
%
% \indent This \LaTeX{} package allows to write Roman numbers
% (or any other text) with bars. (Additionally, command for converting Arabic numbers
% into Roman ones are provided and an |\ifnumeric| test function.)
%
% \bigskip
%
% \section{Usage}
%
% \indent Just load the package placing
% \begin{quote}
%   |\usepackage{romanbar}|
% \end{quote}
% \noindent in the preamble of your \LaTeXe{} source file.\\
% |\Romanbar{...}| then produces a Roman number with bars (please see the example file).
% |\ifnumeric{test}{true}{false}| tests for |test| being numeric.
% |\romannum{...}| turns an Arabic number into a lowercase Roman one, and
% |\Romannum{...}| turns an Arabic number into an uppercase Roman one.
%
% \subsection*{Options}
% \DescribeMacro{options}
% \indent The \xpackage{romanbar} package takes no options.
%
% \section{Alternatives\label{sec:Alternatives}}
%
% \begin{description}
% \item[-] the original code written by Prof. Enrico Gregorio\\
%   (\url{http://profs.sci.univr.it/~gregorio/}),\\
%   \url{http://tex.stackexchange.com/questions/24065/roman-numerals-formatting/24084#24084}
%\begin{verbatim}
%\def\barroman#1{\sbox0{#1}\dimen0=\dimexpr\wd0+1pt\relax
%  \makebox[\dimen0]{\rlap{\vrule width\dimen0 height 0.06ex depth 0.06ex}%
%    \rlap{\vrule width\dimen0 height\dimexpr\ht0+0.03ex\relax
%            depth\dimexpr-\ht0+0.09ex\relax}%
%    \kern.5pt#1\kern.5pt}}
%
%\barroman{I} \barroman{XI}
%\end{verbatim}
%
% \item[-]  Some fonts provide single characters for Roman numerals,
%   cf.~e.\,g.
% \url{http://tex.stackexchange.com/questions/38695/using-unicode-roman-numerals-in-xetex}.
% \end{description}
%
% \noindent (You programmed or found another alternative,
%  which is available at \CTAN{}?\\
%  OK, send an e-mail to me with the name, location at \CTAN{},
%  and a short notice, and I will probably include it in
%  the list above.)
%
% \newpage
%
% \section{Example}
%
%    \begin{macrocode}
%<*example>
\documentclass[british]{article}[2007/10/19]% v1.4h
%%%%%%%%%%%%%%%%%%%%%%%%%%%%%%%%%%%%%%%%%%%%%%%%%%%%%%%%%%%%%%%%%%%%%
\usepackage[extension=pdf,%
 plainpages=false,%
 pdfpagelabels=true,%
 hyperindex=false,%
 pdflang={en},%
 pdftitle={romanbar package example},%
 pdfauthor={H.-Martin Muench},%
 pdfsubject={Example for the romanbar package},%
 pdfkeywords={LaTeX, romanbar, roman, Roman, bars, H.-Martin Muench},%
 pdfview={XYZ null null 1},%
 pdfstartview={XYZ null null 1},%
 pdfpagelayout=SinglePage]{hyperref}[2011/12/04]% v6.82m
\usepackage{romanbar}[2012/01/01]% v1.0f
\gdef\unit#1{\mathord{\thinspace\mathrm{#1}}}%
\listfiles
\begin{document}
\pagenumbering{arabic}
\section*{Example for romanbar}

This example demonstrates the use of package\newline
\textsf{romanbar}, v1.0f as of 2012/01/01 (HMM).\newline
There are no options to be used.\newline

\noindent For more details please see the documentation!\newline

\noindent Save per page about $200\unit{ml}$ water,
$2\unit{g}$ CO$_{2}$ and $2\unit{g}$ wood:\newline
Therefore please print only if this is really necessary.\newline

\noindent This package provides the command \verb|\Romanbar|
to print bars below and over the following:

\begin{description}
\item[-] Roman numbers: \verb|\Romanbar{MMXII}| prints \Romanbar{MMXII}

\item[-] Arabic numbers turned into upper-case Roman numbers:\newline
           \verb|\Romanbar{2012}| prints \Romanbar{2012}

\item[-] negative Arabic numbers turned into upper-case Roman numbers
           \newline
           (although historically there were no negative Roman numbers):
           \newline
           \verb|\Romanbar{-12}| prints \Romanbar{-12}

\item[-] zero Arabic number ($0$; although historically
           there was no Roman zero):\newline
           \verb|\Romanbar{0}| prints \Romanbar{0}

\item[-] some arbitrary text:
           \verb|\Romanbar{Caesar}| prints \Romanbar{Caesar}\newline
           (with descenders: \Romanbar{AgjpqyW})


\newcounter{example}
\setcounter{example}{21}

\item[-] some counter's value:
           \verb|\Romanbar{\theexample}| prints \Romanbar{\theexample}
           \newline
           (where the value of \texttt{example} is \theexample)

\item[-] Arabic numbers, without turning them into upper-case Roman
           numbers:\newline
           \verb|\Romanbar{\relax 2012}| prints \Romanbar{\relax  2012}
\end{description}

Special care was taken for "L" (50), e.\,g. in 555/DLV: \Romanbar{555}.\\


\verb|\romannum{...}| converts an Arabic number into a lower-case Roman one,
e.\,g. \verb|\romannum{2012}| prints \romannum{2012}, and
\verb|\Romannum{...}| converts an Arabic number into an upper-case Roman one,
e.\,g. \verb|\Romannum{2012}| prints \Romannum{2012}.

\end{document}
%</example>
%    \end{macrocode}
%
% \StopEventually{}
%
% \newpage
%
% \section{The implementation}
%
% We start off by checking that we are loading into \LaTeXe{} and
% announcing the name and version of this package.
%
%    \begin{macrocode}
%<*package>
%    \end{macrocode}
%
%    \begin{macrocode}
\NeedsTeXFormat{LaTeX2e}[2011/06/27]
\ProvidesPackage{romanbar}[2012/01/01 v1.0f
              Roman numbers with bars (HMM)]

%    \end{macrocode}
%
% A short description of the \xpackage{romanbar} package:
%
%    \begin{macrocode}
%% Allows to write Roman numbers (or any other text) with bars
%% and to convert Arabic numbers into Roman ones.
%    \end{macrocode}
%
% A last information for the user:
%
%    \begin{macrocode}
%% romanbar may work with earlier versions of LaTeX,
%% but this was not tested. Please consider updating
%% your LaTeX to the most recent version
%% (if it is not already the most recent version).

%    \end{macrocode}
%
% \indent See subsection~\ref{ss:Downloads} about how to get it.\\
%
% There are no options to be processed.
%
% A command to gobble a possible minus sign is needed and
% |\gobbleminus| from \url{http://www.tex.ac.uk/cgi-bin/texfaq2html?label=isitanum}
% is used.
%
%    \begin{macrocode}
\def\gobbleminus#1{\ifx-#1\else#1\fi}
%% from http://www.tex.ac.uk/cgi-bin/texfaq2html?label=isitanum

%    \end{macrocode}
%
% We want to test whether the argument passed to |\Romanbar| is numeric
% and therefore define |\ifnumeric{test}{true}{false}|,
% where |test| is to be analysed, |true| is the code to be executed,
% if |test| is numeric, and |false| is the code to be executed,
% if |test| is not numeric. This is done similar to\\
% \url{http://tex.stackexchange.com/a/17119}.
%
%    \begin{macrocode}
\newcommand{\ifnumeric}[3]{%
%% similar to http://tex.stackexchange.com/a/17119
\sbox\z@{\@tempcnta=0\gobbleminus#1\relax}%
\ifdim\wd0>\z@\relax#3% is not numeric
\else#2% is numeric
\fi%
}

%    \end{macrocode}
%
% We define |\r@iseL| to raise any L (50), otherwise a good deal
% of the horizontal part of the letter would be swallowed by the lower bar
% under the \textquotedblleft{}number\textquotedblright{}.
%
%    \begin{macrocode}
\def\r@iseL#1{\ifx @#1% then terminate
\else%
\if L#1\raisebox{0.05ex}{L}\else #1\fi%
\expandafter\r@iseL%
\fi%
}

%    \end{macrocode}
%
% \pagebreak
%
% We define the |\Romanbar| command.
%
%    \begin{macrocode}
\DeclareRobustCommand{\Romanbar}[1]{%
\ifnumeric{#1}{% is numeric
%    \end{macrocode}
%
% If its parameter is numeric, the temporary counter |\@tempcnta| is set to the value.
%
%    \begin{macrocode}
\@tempcnta=#1\relax%
%    \end{macrocode}
%
% Normally there are no non-positive Roman numbers, therefore we do some trick
% similar to |\XRoman| from the \url{http://ctan.org/pkg/pageslts} package.
%
%    \begin{macrocode}
%% similar to \XRoman from the http://ctan.org/pkg/pageslts package
\ifnum\@tempcnta<1%
  \ifnum #1>0%
    \relax \Roman@bar{\Roman{#1}}%
  \else%
    \ifnum #1<0%
     -\Roman@bar{\expandafter\@slowromancap\romannumeral\number-#1@%
     }%
    \else%
      \Roman@bar{0}%
    \fi%
  \fi%
\else\Roman@bar{\@Roman\@tempcnta}%
\fi%
}{% is not numeric
\Roman@bar{#1}%
}%
}

%    \end{macrocode}
%
% Whether it is numeric or not, in the end |\Roman@bar| is applied to it.
%
%    \begin{macrocode}
\DeclareRobustCommand{\Roman@bar}[1]{% #1 is in Roman, i.e. MMXII
%% (or any other text, "Caesar" would work, too).
%% similar to code of  by Prof. Enrico Gregorio (egreg) at
%% http://tex.stackexchange.com/questions/24065/roman-numerals-formatting/24084#24084
\@bsphack%
\edef\romanbartmp{#1}%
%% height:
\sbox0{\textsf{\romanbartmp}}%
%% top line:
%% + top of top line:
\@tempdima=\ht0%
\advance\@tempdima+0.05ex%0.03
\dimen1=\@tempdima%
%% + bottom of top line:
\@tempdima=-\ht0%
\advance\@tempdima+0.05ex%0.07
\dimen2=\@tempdima%
%% width:
\sbox0{\textsf{\expandafter\r@iseL\romanbartmp @}}%
\@tempdima=\wd0%
\advance\@tempdima+1pt%
\dimen0=\@tempdima%
\@esphack%
\makebox[\dimen0]{%
\rlap{\vrule width\dimen0 height\dimen1 depth\dimen2}% top    line
\rlap{\vrule width\dimen0 height 0.06ex depth 0.03ex}% bottom line
\kern0.5pt\textsf{\expandafter\r@iseL\romanbartmp @}\kern0.5pt}%
}

%    \end{macrocode}
%
% While it is not needed for the bars, it is sometimes asked for a command
% to convert an Arabic number into a Roman one. |\roman{...}| can only be applied
% to a counter, i.\,e. |\roman{12}| does not work, but
%\begin{verbatim}
%\newcounter{examplecounter}
%\setcounter{examplecounter}{12}
%\roman{examplecounter}
%\end{verbatim}
% would be necessary. Further non-negative values would not work.
% |\romannum| accepts any number as argument. (If the argument is no number,
% an error message is given and the argument printed (executed)
%  without applying anything to it.)
%
%    \begin{macrocode}
\DeclareRobustCommand{\romannum}[1]{%
\ifnumeric{#1}{% is numeric
\@tempcnta=#1\relax%
%% similar to \xroman from the http://ctan.org/pkg/pageslts package
\ifnum\@tempcnta<1%
  \ifnum #1>0%
    \relax \roman{#1}%
  \else%
    \ifnum #1<0%
     -\romannumeral\number-#1@%
    \else%
      0%
    \fi%
  \fi%
\else\@roman\@tempcnta%
\fi%
}{% is not numeric
\PackageError{romanbar}{%
Argument of \string\romannum\space is not a number}{%
The command \string\romannum\space converts an Arabic number into a%
 lower-case Roman one,\MessageBreak%
but the used argument of \string\romannum\space is not an Arabic number%
 but\MessageBreak%
`#1`,\MessageBreak%
which will now be printed unchanged.}%
#1%
}%
}

%    \end{macrocode}
%
% The same for upper-case Roman numbers, |\Romannum| instead of |\Roman|:
%
%    \begin{macrocode}
\DeclareRobustCommand{\Romannum}[1]{%
\ifnumeric{#1}{% is numeric
\@tempcnta=#1\relax%
%% similar to \XRoman from the http://ctan.org/pkg/pageslts package
\ifnum\@tempcnta<1%
  \ifnum #1>0%
    \relax \Roman{#1}%
  \else%
    \ifnum #1<0%
     -\expandafter\@slowromancap\romannumeral\number-#1@%
    \else%
      0%
    \fi%
  \fi%
\else\@Roman\@tempcnta%
\fi%
}{% is not numeric
\PackageError{romanbar}{%
Argument of \string\Romannum\space is not a number}{%
The command \string\Romannum\space converts an Arabic number into an%
 upper-case Roman one,\MessageBreak%
but the used argument of \string\Romannum\space is not an Arabic number%
 but\MessageBreak%
`#1`.}%
#1%
}%
}

%    \end{macrocode}
%
%    \begin{macrocode}
%</package>
%    \end{macrocode}
%
% \newpage
%
% \section{Installation}
%
% \subsection{Downloads\label{ss:Downloads}}
%
% Everything should be available on \CTAN{}, \url{http://www.ctan.org/tex-archive/},
% but may need additional packages themselves.\\
%
% \DescribeMacro{romanbar.dtx}
% For unpacking the |romanbar.dtx| file and constructing the documentation
% it is required:
% \begin{description}
% \item[-] \TeX{} Format \LaTeXe{}, \url{http://www.CTAN.org/}
%
% \item[-] document class \xpackage{ltxdoc}, 2007/11/11, v2.0u,
%           \url{http://ctan.org/pkg/ltxdoc}
%
% \item[-] package \xpackage{holtxdoc}, 2011/02/04, v0.21,
%           \url{http://ctan.org/pkg/holtxdoc}
%
% \item[-] package \xpackage{hypdoc}, 2010/03/26, v1.9,
%           \url{http://ctan.org/pkg/hypdoc}
% \end{description}
%
% \DescribeMacro{romanbar.sty}
% The \texttt{romanbar.sty} for \LaTeXe{} (i.\,e.~each document using
% the \xpackage{romanbar} package) requires:
% \begin{description}
% \item[-] \TeX Format \LaTeXe{}, \url{http://www.CTAN.org/}
% \end{description}
%
% \DescribeMacro{romanbar-example.tex}
% The \texttt{romanbar-example.tex} requires the same files as all
% documents using the \xpackage{romanbar} package and additionally:
% \begin{description}
% \item[-] class \xpackage{article}, 2007/10/19, v1.4h, from \xpackage{classes.dtx}:\\
%           \CTAN{macros/latex/base/classes.dtx}
%
% \item[-] package \xpackage{romanbar}, 2012/01/01, v1.0f,\\
%           %%\url{http://ctan.org/pkg/romanbar}\\
%           \CTAN{macros/latex/contrib/romanbar/}
%   (Well, it is the example file for this package, and because you are reading the
%    documentation for the \xpackage{romanbar} package, it can be assumed that you already
%    have some version of it -- is it the current one?)
% \end{description}
%
% \DescribeMacro{Alternative}
% As possible alternative package in section \ref{sec:Alternatives} there is listed
% \begin{description}
% \item[-] none.
% \end{description}
%
% \DescribeMacro{Oberdiek}
% \DescribeMacro{holtxdoc}
% All packages of \textsc{Heiko Oberdiek's} bundle `oberdiek'
% (especially \xpackage{holtxdoc}) are also available in a TDS compliant ZIP archive:\\
% \CTAN{install/macros/latex/contrib/oberdiek.tds.zip}.\\
% It is probably best to download and use this, because the packages in there
% are quite probably both recent and compatible among themselves.\\
%
% \DescribeMacro{hyperref}
% \noindent \xpackage{hyperref} is not included in that bundle and needs to be
% downloaded separately,\\
% \url{http://mirror.ctan.org/install/macros/latex/contrib/hyperref.tds.zip}.\\
%
% \DescribeMacro{M\"{u}nch}
% A hyperlinked list of my (other) packages can be found at
% \url{http://www.Uni-Bonn.de/~uzs5pv/LaTeX.html}.\\
%
% \pagebreak
%
% \subsection{Package, unpacking TDS}
% \paragraph{Package.} This package should become available on \CTAN{} soon:
% \begin{description}
% \item[\CTAN{macros/latex/contrib/romanbar/romanbar.dtx}]\hspace*{0.1cm} \\
%       The source file.
% \item[\CTAN{macros/latex/contrib/romanbar/romanbar.pdf}]\hspace*{0.1cm} \\
%       The documentation.
% \item[\CTAN{macros/latex/contrib/romanbar/romanbar-example.pdf}]\hspace*{0.1cm} \\
%       The compiled example file, as it should look like.
% \item[\CTAN{macros/latex/contrib/romanbar/README}]\hspace*{0.1cm} \\
%       The README file.
% \end{description}
% There is also a romanbar.tds.zip available:
% \begin{description}
% \item[\CTAN{install/macros/latex/contrib/romanbar.tds.zip}]\hspace*{0.1cm} \\
%       Everything in \xfile{TDS} compliant, compiled format.
% \end{description}
% which additionally contains\\
% \begin{tabular}{ll}
% romanbar.ins & The installation file.\\
% romanbar.drv & The driver to generate the documentation.\\
% romanbar.sty & The \xext{sty}le file.\\
% romanbar-example.tex & The example file.
% \end{tabular}
%
% \bigskip
%
% \noindent For required other packages, please see the preceding subsection.
%
% \paragraph{Unpacking.} The \xfile{.dtx} file is a self-extracting
% \docstrip{} archive. The files are extracted by running the
% \xext{.dtx} through \plainTeX{}:
% \begin{quote}
%   \verb|tex romanbar.dtx|
% \end{quote}
%
% About generating the documentation see paragraph~\ref{GenDoc} below.\\
%
% \paragraph{TDS.} Now the different files must be moved into
% the different directories in your installation TDS tree
% (also known as \xfile{texmf} tree):
% \begin{quote}
% \def\t{^^A
% \begin{tabular}{@{}>{\ttfamily}l@{ $\rightarrow$ }>{\ttfamily}l@{}}
%   romanbar.sty & tex/latex/romanbar/romanbar.sty\\
%   romanbar.pdf & doc/latex/romanbar/romanbar.pdf\\
%   romanbar-example.tex & doc/latex/romanbar/romanbar-example.tex\\
%   romanbar-example.pdf & doc/latex/romanbar/romanbar-example.pdf\\
%   romanbar.dtx & source/latex/romanbar/romanbar.dtx\\
% \end{tabular}^^A
% }^^A
% \sbox0{\t}^^A
% \ifdim\wd0>\linewidth
%   \begingroup
%     \advance\linewidth by\leftmargin
%     \advance\linewidth by\rightmargin
%   \edef\x{\endgroup
%     \def\noexpand\lw{\the\linewidth}^^A
%   }\x
%   \def\lwbox{^^A
%     \leavevmode
%     \hbox to \linewidth{^^A
%       \kern-\leftmargin\relax
%       \hss
%       \usebox0
%       \hss
%       \kern-\rightmargin\relax
%     }^^A
%   }^^A
%   \ifdim\wd0>\lw
%     \sbox0{\small\t}^^A
%     \ifdim\wd0>\linewidth
%       \ifdim\wd0>\lw
%         \sbox0{\footnotesize\t}^^A
%         \ifdim\wd0>\linewidth
%           \ifdim\wd0>\lw
%             \sbox0{\scriptsize\t}^^A
%             \ifdim\wd0>\linewidth
%               \ifdim\wd0>\lw
%                 \sbox0{\tiny\t}^^A
%                 \ifdim\wd0>\linewidth
%                   \lwbox
%                 \else
%                   \usebox0
%                 \fi
%               \else
%                 \lwbox
%               \fi
%             \else
%               \usebox0
%             \fi
%           \else
%             \lwbox
%           \fi
%         \else
%           \usebox0
%         \fi
%       \else
%         \lwbox
%       \fi
%     \else
%       \usebox0
%     \fi
%   \else
%     \lwbox
%   \fi
% \else
%   \usebox0
% \fi
% \end{quote}
% If you have a \xfile{docstrip.cfg} that configures and enables \docstrip{}'s
% \xfile{TDS} installing feature, then some files can already be in the right
% place, see the documentation of \docstrip{}.
%
% \subsection{Refresh file name databases}
%
% If your \TeX{}~distribution (\teTeX{}, \mikTeX{},\dots{}) relies on file name
% databases, you must refresh these. For example, \teTeX{} users run
% \verb|texhash| or \verb|mktexlsr|.
%
% \subsection{Some details for the interested}
%
% \paragraph{Unpacking with \LaTeX{}.}
% The \xfile{.dtx} chooses its action depending on the format:
% \begin{description}
% \item[\plainTeX:] Run \docstrip{} and extract the files.
% \item[\LaTeX:] Generate the documentation.
% \end{description}
% If you insist on using \LaTeX{} for \docstrip{} (really,
% \docstrip{} does not need \LaTeX{}), then inform the autodetect routine
% about your intention:
% \begin{quote}
%   \verb|latex \let\install=y% \iffalse meta-comment
%
% File: romanbar.dtx
% Version: 2012/01/01 v1.0f
%
% Copyright (C) 2011 by
%    H.-Martin M"unch <Martin dot Muench at Uni-Bonn dot de>
%
% The original code for the bars was written by Prof. Enrico Gregorio
% (http://profs.sci.univr.it/~gregorio/), and published at
% http://tex.stackexchange.com/questions/24065/roman-numerals-formatting/24084#24084
% I (H.-Martin Münch) submitted this package to CTAN
% (after diverse updates of the code and writing a documentation).
%
% This work may be distributed and/or modified under the
% conditions of the LaTeX Project Public License, either
% version 1.3c of this license or (at your option) any later
% version. This version of this license is in
%    http://www.latex-project.org/lppl/lppl-1-3c.txt
% and the latest version of this license is in
%    http://www.latex-project.org/lppl.txt
% and version 1.3c or later is part of all distributions of
% LaTeX version 2005/12/01 or later.
%
% This work has the LPPL maintenance status "maintained".
%
% The Current Maintainer of this work is H.-Martin Muench.
%
% This work consists of the main source file romanbar.dtx,
% the README, and the derived files
%    romanbar.sty, romanbar.pdf,
%    romanbar.ins, romanbar.drv,
%    romanbar-example.tex, romanbar-example.pdf.
%
% Distribution:
%    CTAN:macros/latex/contrib/romanbar/romanbar.dtx
%    CTAN:macros/latex/contrib/romanbar/romanbar.pdf
%    CTAN:install/macros/latex/contrib/romanbar.tds.zip
%
% Cf. http://ctan.org/pkg/romanbar and
% http://www.tex.ac.uk/tex-archive/help/Catalogue/entries/romanbar.html
% (It might take some time for the entries to be created.)
%
% Unpacking:
%    (a) If romanbar.ins is present:
%           tex romanbar.ins
%    (b) Without romanbar.ins:
%           tex romanbar.dtx
%    (c) If you insist on using LaTeX
%           latex \let\install=y% \iffalse meta-comment
%
% File: romanbar.dtx
% Version: 2012/01/01 v1.0f
%
% Copyright (C) 2011 by
%    H.-Martin M"unch <Martin dot Muench at Uni-Bonn dot de>
%
% The original code for the bars was written by Prof. Enrico Gregorio
% (http://profs.sci.univr.it/~gregorio/), and published at
% http://tex.stackexchange.com/questions/24065/roman-numerals-formatting/24084#24084
% I (H.-Martin Münch) submitted this package to CTAN
% (after diverse updates of the code and writing a documentation).
%
% This work may be distributed and/or modified under the
% conditions of the LaTeX Project Public License, either
% version 1.3c of this license or (at your option) any later
% version. This version of this license is in
%    http://www.latex-project.org/lppl/lppl-1-3c.txt
% and the latest version of this license is in
%    http://www.latex-project.org/lppl.txt
% and version 1.3c or later is part of all distributions of
% LaTeX version 2005/12/01 or later.
%
% This work has the LPPL maintenance status "maintained".
%
% The Current Maintainer of this work is H.-Martin Muench.
%
% This work consists of the main source file romanbar.dtx,
% the README, and the derived files
%    romanbar.sty, romanbar.pdf,
%    romanbar.ins, romanbar.drv,
%    romanbar-example.tex, romanbar-example.pdf.
%
% Distribution:
%    CTAN:macros/latex/contrib/romanbar/romanbar.dtx
%    CTAN:macros/latex/contrib/romanbar/romanbar.pdf
%    CTAN:install/macros/latex/contrib/romanbar.tds.zip
%
% Cf. http://ctan.org/pkg/romanbar and
% http://www.tex.ac.uk/tex-archive/help/Catalogue/entries/romanbar.html
% (It might take some time for the entries to be created.)
%
% Unpacking:
%    (a) If romanbar.ins is present:
%           tex romanbar.ins
%    (b) Without romanbar.ins:
%           tex romanbar.dtx
%    (c) If you insist on using LaTeX
%           latex \let\install=y% \iffalse meta-comment
%
% File: romanbar.dtx
% Version: 2012/01/01 v1.0f
%
% Copyright (C) 2011 by
%    H.-Martin M"unch <Martin dot Muench at Uni-Bonn dot de>
%
% The original code for the bars was written by Prof. Enrico Gregorio
% (http://profs.sci.univr.it/~gregorio/), and published at
% http://tex.stackexchange.com/questions/24065/roman-numerals-formatting/24084#24084
% I (H.-Martin Münch) submitted this package to CTAN
% (after diverse updates of the code and writing a documentation).
%
% This work may be distributed and/or modified under the
% conditions of the LaTeX Project Public License, either
% version 1.3c of this license or (at your option) any later
% version. This version of this license is in
%    http://www.latex-project.org/lppl/lppl-1-3c.txt
% and the latest version of this license is in
%    http://www.latex-project.org/lppl.txt
% and version 1.3c or later is part of all distributions of
% LaTeX version 2005/12/01 or later.
%
% This work has the LPPL maintenance status "maintained".
%
% The Current Maintainer of this work is H.-Martin Muench.
%
% This work consists of the main source file romanbar.dtx,
% the README, and the derived files
%    romanbar.sty, romanbar.pdf,
%    romanbar.ins, romanbar.drv,
%    romanbar-example.tex, romanbar-example.pdf.
%
% Distribution:
%    CTAN:macros/latex/contrib/romanbar/romanbar.dtx
%    CTAN:macros/latex/contrib/romanbar/romanbar.pdf
%    CTAN:install/macros/latex/contrib/romanbar.tds.zip
%
% Cf. http://ctan.org/pkg/romanbar and
% http://www.tex.ac.uk/tex-archive/help/Catalogue/entries/romanbar.html
% (It might take some time for the entries to be created.)
%
% Unpacking:
%    (a) If romanbar.ins is present:
%           tex romanbar.ins
%    (b) Without romanbar.ins:
%           tex romanbar.dtx
%    (c) If you insist on using LaTeX
%           latex \let\install=y\input{romanbar.dtx}
%        (quote the arguments according to the demands of your shell)
%
% Documentation:
%    (a) If romanbar.drv is present:
%           (pdf)latex romanbar.drv
%           makeindex -s gind.ist romanbar.idx
%           (pdf)latex romanbar.drv
%           makeindex -s gind.ist romanbar.idx
%           (pdf)latex romanbar.drv
%    (b) Without romanbar.drv:
%           (pdf)latex romanbar.dtx
%           makeindex -s gind.ist romanbar.idx
%           (pdf)latex romanbar.dtx
%           makeindex -s gind.ist romanbar.idx
%           (pdf)latex romanbar.dtx
%
%    The class ltxdoc loads the configuration file ltxdoc.cfg
%    if available. Here you can specify further options, e.g.
%    use DIN A4 as paper format:
%       \PassOptionsToClass{a4paper}{article}
%
% Installation:
%    TDS:tex/latex/romanbar/romanbar.sty
%    TDS:doc/latex/romanbar/romanbar.pdf
%    TDS:doc/latex/romanbar/romanbar-example.tex
%    TDS:doc/latex/romanbar/romanbar-example.pdf
%    TDS:source/latex/romanbar/romanbar.dtx
%
%<*ignore>
\begingroup
  \catcode123=1 %
  \catcode125=2 %
  \def\x{LaTeX2e}%
\expandafter\endgroup
\ifcase 0\ifx\install y1\fi\expandafter
         \ifx\csname processbatchFile\endcsname\relax\else1\fi
         \ifx\fmtname\x\else 1\fi\relax
\else\csname fi\endcsname
%</ignore>
%<*install>
\input docstrip.tex
\Msg{********************************************************************}
\Msg{* Installation                                                     *}
\Msg{* Package: romanbar 2012/01/01 v1.0f Roman numbers with bars (HMM) *}
\Msg{********************************************************************}

\keepsilent
\askforoverwritefalse

\let\MetaPrefix\relax
\preamble

This is a generated file.

Project: romanbar
Version: 2012/01/01 v1.0f

Copyright (C) 2011 by
    H.-Martin M"unch <Martin dot Muench at Uni-Bonn dot de>

The usual disclaimer applies:
If it doesn't work right that's your problem.
(Nevertheless, send an e-mail to the maintainer
 when you find an error in this package.)

This work may be distributed and/or modified under the
conditions of the LaTeX Project Public License, either
version 1.3c of this license or (at your option) any later
version. This version of this license is in
   http://www.latex-project.org/lppl/lppl-1-3c.txt
and the latest version of this license is in
   http://www.latex-project.org/lppl.txt
and version 1.3c or later is part of all distributions of
LaTeX version 2005/12/01 or later.

This work has the LPPL maintenance status "maintained".

The Current Maintainer of this work is H.-Martin Muench.

This work consists of the main source file romanbar.dtx,
the README, and the derived files
   romanbar.sty, romanbar.pdf,
   romanbar.ins, romanbar.drv,
   romanbar-example.tex, romanbar-example.pdf.

\endpreamble
\let\MetaPrefix\DoubleperCent

\generate{%
  \file{romanbar.ins}{\from{romanbar.dtx}{install}}%
  \file{romanbar.drv}{\from{romanbar.dtx}{driver}}%
  \usedir{tex/latex/romanbar}%
  \file{romanbar.sty}{\from{romanbar.dtx}{package}}%
  \usedir{doc/latex/romanbar}%
  \file{romanbar-example.tex}{\from{romanbar.dtx}{example}}%
}

\catcode32=13\relax% active space
\let =\space%
\Msg{************************************************************************}
\Msg{*}
\Msg{* To finish the installation you have to move the following}
\Msg{* file into a directory searched by TeX:}
\Msg{*}
\Msg{*     romanbar.sty}
\Msg{*}
\Msg{* To produce the documentation run the file `romanbar.drv'}
\Msg{* through (pdf)LaTeX, e.g.}
\Msg{*  pdflatex romanbar.drv}
\Msg{*  makeindex -s gind.ist romanbar.idx}
\Msg{*  pdflatex romanbar.drv}
\Msg{*  makeindex -s gind.ist romanbar.idx}
\Msg{*  pdflatex romanbar.drv}
\Msg{*}
\Msg{* At least three runs are necessary e.g. to get the}
\Msg{*  references right!}
\Msg{*}
\Msg{* Happy TeXing!}
\Msg{*}
\Msg{************************************************************************}

\endbatchfile
%</install>
%<*ignore>
\fi
%</ignore>
%
% \section{The documentation driver file}
%
% The next bit of code contains the documentation driver file for
% \TeX{}, i.\,e., the file that will produce the documentation you
% are currently reading. It will be extracted from this file by the
% \texttt{docstrip} programme. That is, run \LaTeX{} on \texttt{docstrip}
% and specify the \texttt{driver} option when \texttt{docstrip}
% asks for options.
%
%    \begin{macrocode}
%<*driver>
\NeedsTeXFormat{LaTeX2e}[2011/06/27]
\ProvidesFile{romanbar.drv}%
  [2012/01/01 v1.0f Roman numbers with bars (HMM)]
\documentclass{ltxdoc}[2007/11/11]% v2.0u
\usepackage{holtxdoc}[2011/02/04]%  v0.21
%% romanbar may work with earlier versions of LaTeX2e and those
%% class and package, but this was not tested.
%% Please consider updating your LaTeX, class, and package
%% to the most recent version (if they are not already the most
%% recent version).
\hypersetup{%
 pdfsubject={Writing Roman numbers with bars (HMM)},%
 pdfkeywords={LaTeX, romanbar, roman, Roman, bars, H.-Martin Muench},%
 pdfencoding=auto,%
 pdflang={en},%
 breaklinks=true,%
 linktoc=all,%
 pdfstartview=FitH,%
 pdfpagelayout=OneColumn,%
 bookmarksnumbered=true,%
 bookmarksopen=true,%
 bookmarksopenlevel=2,%
 pdfmenubar=true,%
 pdftoolbar=true,%
 pdfwindowui=true,%
 pdfnewwindow=true%
}
\CodelineIndex
\hyphenation{docu-ment ana-lysed}
\gdef\unit#1{\mathord{\thinspace\mathrm{#1}}}%
\makeatletter
\@ifundefined{eTeX}{\gdef\eTeX{$\m@th \varepsilon $-\TeX }}{% else \relax
}
\makeatother
\begin{document}
  \DocInput{romanbar.dtx}%
\end{document}
%</driver>
%    \end{macrocode}
%
% \fi
%
% \CheckSum{179}
%
% \CharacterTable
%  {Upper-case    \A\B\C\D\E\F\G\H\I\J\K\L\M\N\O\P\Q\R\S\T\U\V\W\X\Y\Z
%   Lower-case    \a\b\c\d\e\f\g\h\i\j\k\l\m\n\o\p\q\r\s\t\u\v\w\x\y\z
%   Digits        \0\1\2\3\4\5\6\7\8\9
%   Exclamation   \!     Double quote  \"     Hash (number) \#
%   Dollar        \$     Percent       \%     Ampersand     \&
%   Acute accent  \'     Left paren    \(     Right paren   \)
%   Asterisk      \*     Plus          \+     Comma         \,
%   Minus         \-     Point         \.     Solidus       \/
%   Colon         \:     Semicolon     \;     Less than     \<
%   Equals        \=     Greater than  \>     Question mark \?
%   Commercial at \@     Left bracket  \[     Backslash     \\
%   Right bracket \]     Circumflex    \^     Underscore    \_
%   Grave accent  \`     Left brace    \{     Vertical bar  \|
%   Right brace   \}     Tilde         \~}
%
% \GetFileInfo{romanbar.drv}
%
% \begingroup
%   \def\x{\#,\$,\^,\_,\~,\ ,\&,\{,\},\%}%
%   \makeatletter
%   \@onelevel@sanitize\x
% \expandafter\endgroup
% \expandafter\DoNotIndex\expandafter{\x}
% \expandafter\DoNotIndex\expandafter{\string\ }
% \begingroup
%   \makeatletter
%     \lccode`9=32\relax
%     \lowercase{%^^A
%       \edef\x{\noexpand\DoNotIndex{\@backslashchar9}}%^^A
%     }%^^A
%   \expandafter\endgroup\x
%
% \DoNotIndex{\\,\,}
% \DoNotIndex{\documentclass,\usepackage,\ProvidesPackage,\begin,\end}
% \DoNotIndex{\MessageBreak}
% \DoNotIndex{\NeedsTeXFormat,\DoNotIndex,\verb}
% \DoNotIndex{\def,\edef,\gdef,\global}
% \DoNotIndex{\ifx,\listfiles,\mathord,\mathrm}
% \DoNotIndex{\bigskip,\space,\thinspace,\ldots}
% \DoNotIndex{\indent,\noindent,\newline,\linebreak,\pagebreak,\newpage}
% \DoNotIndex{\textbf,\textit,\textsf,\texttt,\textquotedblleft,\textquotedblright}
% \DoNotIndex{\plainTeX,\TeX,\LaTeX,\pdfLaTeX}
% \DoNotIndex{\item}
%
% \title{The \xpackage{romanbar} package}
% \date{2012/01/01 v1.0f}
% \author{H.-Martin M\"{u}nch\\\xemail{Martin.Muench at Uni-Bonn.de}}
%
% \maketitle
%
% \begin{abstract}
%  This \LaTeX{} package allows to write Roman numbers
%  (or any other text) with bars. (Additionally, commands for converting Arabic numbers
%  into Roman ones are provided and an |\ifnumeric| test function.)
% \end{abstract}
%
% \bigskip
%
% \noindent Disclaimer for web links: The author is not responsible for any contents
% referred to in this work unless he has full knowledge of illegal contents.
% If any damage occurs by the use of information presented there, only the
% author of the respective pages might be liable, not the one who has referred
% to these pages.
%
% \bigskip
%
% \noindent {\color{green} Save per page about $200\unit{ml}$ water,
% $2\unit{g}$ CO$_{2}$ and $2\unit{g}$ wood:\\
% Therefore please print only if this is really necessary.}
%
% \pagebreak
%
% \tableofcontents
%
% \pagebreak
%
% \section{Introduction}
%
% \indent This \LaTeX{} package allows to write Roman numbers
% (or any other text) with bars. (Additionally, command for converting Arabic numbers
% into Roman ones are provided and an |\ifnumeric| test function.)
%
% \bigskip
%
% \section{Usage}
%
% \indent Just load the package placing
% \begin{quote}
%   |\usepackage{romanbar}|
% \end{quote}
% \noindent in the preamble of your \LaTeXe{} source file.\\
% |\Romanbar{...}| then produces a Roman number with bars (please see the example file).
% |\ifnumeric{test}{true}{false}| tests for |test| being numeric.
% |\romannum{...}| turns an Arabic number into a lowercase Roman one, and
% |\Romannum{...}| turns an Arabic number into an uppercase Roman one.
%
% \subsection*{Options}
% \DescribeMacro{options}
% \indent The \xpackage{romanbar} package takes no options.
%
% \section{Alternatives\label{sec:Alternatives}}
%
% \begin{description}
% \item[-] the original code written by Prof. Enrico Gregorio\\
%   (\url{http://profs.sci.univr.it/~gregorio/}),\\
%   \url{http://tex.stackexchange.com/questions/24065/roman-numerals-formatting/24084#24084}
%\begin{verbatim}
%\def\barroman#1{\sbox0{#1}\dimen0=\dimexpr\wd0+1pt\relax
%  \makebox[\dimen0]{\rlap{\vrule width\dimen0 height 0.06ex depth 0.06ex}%
%    \rlap{\vrule width\dimen0 height\dimexpr\ht0+0.03ex\relax
%            depth\dimexpr-\ht0+0.09ex\relax}%
%    \kern.5pt#1\kern.5pt}}
%
%\barroman{I} \barroman{XI}
%\end{verbatim}
%
% \item[-]  Some fonts provide single characters for Roman numerals,
%   cf.~e.\,g.
% \url{http://tex.stackexchange.com/questions/38695/using-unicode-roman-numerals-in-xetex}.
% \end{description}
%
% \noindent (You programmed or found another alternative,
%  which is available at \CTAN{}?\\
%  OK, send an e-mail to me with the name, location at \CTAN{},
%  and a short notice, and I will probably include it in
%  the list above.)
%
% \newpage
%
% \section{Example}
%
%    \begin{macrocode}
%<*example>
\documentclass[british]{article}[2007/10/19]% v1.4h
%%%%%%%%%%%%%%%%%%%%%%%%%%%%%%%%%%%%%%%%%%%%%%%%%%%%%%%%%%%%%%%%%%%%%
\usepackage[extension=pdf,%
 plainpages=false,%
 pdfpagelabels=true,%
 hyperindex=false,%
 pdflang={en},%
 pdftitle={romanbar package example},%
 pdfauthor={H.-Martin Muench},%
 pdfsubject={Example for the romanbar package},%
 pdfkeywords={LaTeX, romanbar, roman, Roman, bars, H.-Martin Muench},%
 pdfview={XYZ null null 1},%
 pdfstartview={XYZ null null 1},%
 pdfpagelayout=SinglePage]{hyperref}[2011/12/04]% v6.82m
\usepackage{romanbar}[2012/01/01]% v1.0f
\gdef\unit#1{\mathord{\thinspace\mathrm{#1}}}%
\listfiles
\begin{document}
\pagenumbering{arabic}
\section*{Example for romanbar}

This example demonstrates the use of package\newline
\textsf{romanbar}, v1.0f as of 2012/01/01 (HMM).\newline
There are no options to be used.\newline

\noindent For more details please see the documentation!\newline

\noindent Save per page about $200\unit{ml}$ water,
$2\unit{g}$ CO$_{2}$ and $2\unit{g}$ wood:\newline
Therefore please print only if this is really necessary.\newline

\noindent This package provides the command \verb|\Romanbar|
to print bars below and over the following:

\begin{description}
\item[-] Roman numbers: \verb|\Romanbar{MMXII}| prints \Romanbar{MMXII}

\item[-] Arabic numbers turned into upper-case Roman numbers:\newline
           \verb|\Romanbar{2012}| prints \Romanbar{2012}

\item[-] negative Arabic numbers turned into upper-case Roman numbers
           \newline
           (although historically there were no negative Roman numbers):
           \newline
           \verb|\Romanbar{-12}| prints \Romanbar{-12}

\item[-] zero Arabic number ($0$; although historically
           there was no Roman zero):\newline
           \verb|\Romanbar{0}| prints \Romanbar{0}

\item[-] some arbitrary text:
           \verb|\Romanbar{Caesar}| prints \Romanbar{Caesar}\newline
           (with descenders: \Romanbar{AgjpqyW})


\newcounter{example}
\setcounter{example}{21}

\item[-] some counter's value:
           \verb|\Romanbar{\theexample}| prints \Romanbar{\theexample}
           \newline
           (where the value of \texttt{example} is \theexample)

\item[-] Arabic numbers, without turning them into upper-case Roman
           numbers:\newline
           \verb|\Romanbar{\relax 2012}| prints \Romanbar{\relax  2012}
\end{description}

Special care was taken for "L" (50), e.\,g. in 555/DLV: \Romanbar{555}.\\


\verb|\romannum{...}| converts an Arabic number into a lower-case Roman one,
e.\,g. \verb|\romannum{2012}| prints \romannum{2012}, and
\verb|\Romannum{...}| converts an Arabic number into an upper-case Roman one,
e.\,g. \verb|\Romannum{2012}| prints \Romannum{2012}.

\end{document}
%</example>
%    \end{macrocode}
%
% \StopEventually{}
%
% \newpage
%
% \section{The implementation}
%
% We start off by checking that we are loading into \LaTeXe{} and
% announcing the name and version of this package.
%
%    \begin{macrocode}
%<*package>
%    \end{macrocode}
%
%    \begin{macrocode}
\NeedsTeXFormat{LaTeX2e}[2011/06/27]
\ProvidesPackage{romanbar}[2012/01/01 v1.0f
              Roman numbers with bars (HMM)]

%    \end{macrocode}
%
% A short description of the \xpackage{romanbar} package:
%
%    \begin{macrocode}
%% Allows to write Roman numbers (or any other text) with bars
%% and to convert Arabic numbers into Roman ones.
%    \end{macrocode}
%
% A last information for the user:
%
%    \begin{macrocode}
%% romanbar may work with earlier versions of LaTeX,
%% but this was not tested. Please consider updating
%% your LaTeX to the most recent version
%% (if it is not already the most recent version).

%    \end{macrocode}
%
% \indent See subsection~\ref{ss:Downloads} about how to get it.\\
%
% There are no options to be processed.
%
% A command to gobble a possible minus sign is needed and
% |\gobbleminus| from \url{http://www.tex.ac.uk/cgi-bin/texfaq2html?label=isitanum}
% is used.
%
%    \begin{macrocode}
\def\gobbleminus#1{\ifx-#1\else#1\fi}
%% from http://www.tex.ac.uk/cgi-bin/texfaq2html?label=isitanum

%    \end{macrocode}
%
% We want to test whether the argument passed to |\Romanbar| is numeric
% and therefore define |\ifnumeric{test}{true}{false}|,
% where |test| is to be analysed, |true| is the code to be executed,
% if |test| is numeric, and |false| is the code to be executed,
% if |test| is not numeric. This is done similar to\\
% \url{http://tex.stackexchange.com/a/17119}.
%
%    \begin{macrocode}
\newcommand{\ifnumeric}[3]{%
%% similar to http://tex.stackexchange.com/a/17119
\sbox\z@{\@tempcnta=0\gobbleminus#1\relax}%
\ifdim\wd0>\z@\relax#3% is not numeric
\else#2% is numeric
\fi%
}

%    \end{macrocode}
%
% We define |\r@iseL| to raise any L (50), otherwise a good deal
% of the horizontal part of the letter would be swallowed by the lower bar
% under the \textquotedblleft{}number\textquotedblright{}.
%
%    \begin{macrocode}
\def\r@iseL#1{\ifx @#1% then terminate
\else%
\if L#1\raisebox{0.05ex}{L}\else #1\fi%
\expandafter\r@iseL%
\fi%
}

%    \end{macrocode}
%
% \pagebreak
%
% We define the |\Romanbar| command.
%
%    \begin{macrocode}
\DeclareRobustCommand{\Romanbar}[1]{%
\ifnumeric{#1}{% is numeric
%    \end{macrocode}
%
% If its parameter is numeric, the temporary counter |\@tempcnta| is set to the value.
%
%    \begin{macrocode}
\@tempcnta=#1\relax%
%    \end{macrocode}
%
% Normally there are no non-positive Roman numbers, therefore we do some trick
% similar to |\XRoman| from the \url{http://ctan.org/pkg/pageslts} package.
%
%    \begin{macrocode}
%% similar to \XRoman from the http://ctan.org/pkg/pageslts package
\ifnum\@tempcnta<1%
  \ifnum #1>0%
    \relax \Roman@bar{\Roman{#1}}%
  \else%
    \ifnum #1<0%
     -\Roman@bar{\expandafter\@slowromancap\romannumeral\number-#1@%
     }%
    \else%
      \Roman@bar{0}%
    \fi%
  \fi%
\else\Roman@bar{\@Roman\@tempcnta}%
\fi%
}{% is not numeric
\Roman@bar{#1}%
}%
}

%    \end{macrocode}
%
% Whether it is numeric or not, in the end |\Roman@bar| is applied to it.
%
%    \begin{macrocode}
\DeclareRobustCommand{\Roman@bar}[1]{% #1 is in Roman, i.e. MMXII
%% (or any other text, "Caesar" would work, too).
%% similar to code of  by Prof. Enrico Gregorio (egreg) at
%% http://tex.stackexchange.com/questions/24065/roman-numerals-formatting/24084#24084
\@bsphack%
\edef\romanbartmp{#1}%
%% height:
\sbox0{\textsf{\romanbartmp}}%
%% top line:
%% + top of top line:
\@tempdima=\ht0%
\advance\@tempdima+0.05ex%0.03
\dimen1=\@tempdima%
%% + bottom of top line:
\@tempdima=-\ht0%
\advance\@tempdima+0.05ex%0.07
\dimen2=\@tempdima%
%% width:
\sbox0{\textsf{\expandafter\r@iseL\romanbartmp @}}%
\@tempdima=\wd0%
\advance\@tempdima+1pt%
\dimen0=\@tempdima%
\@esphack%
\makebox[\dimen0]{%
\rlap{\vrule width\dimen0 height\dimen1 depth\dimen2}% top    line
\rlap{\vrule width\dimen0 height 0.06ex depth 0.03ex}% bottom line
\kern0.5pt\textsf{\expandafter\r@iseL\romanbartmp @}\kern0.5pt}%
}

%    \end{macrocode}
%
% While it is not needed for the bars, it is sometimes asked for a command
% to convert an Arabic number into a Roman one. |\roman{...}| can only be applied
% to a counter, i.\,e. |\roman{12}| does not work, but
%\begin{verbatim}
%\newcounter{examplecounter}
%\setcounter{examplecounter}{12}
%\roman{examplecounter}
%\end{verbatim}
% would be necessary. Further non-negative values would not work.
% |\romannum| accepts any number as argument. (If the argument is no number,
% an error message is given and the argument printed (executed)
%  without applying anything to it.)
%
%    \begin{macrocode}
\DeclareRobustCommand{\romannum}[1]{%
\ifnumeric{#1}{% is numeric
\@tempcnta=#1\relax%
%% similar to \xroman from the http://ctan.org/pkg/pageslts package
\ifnum\@tempcnta<1%
  \ifnum #1>0%
    \relax \roman{#1}%
  \else%
    \ifnum #1<0%
     -\romannumeral\number-#1@%
    \else%
      0%
    \fi%
  \fi%
\else\@roman\@tempcnta%
\fi%
}{% is not numeric
\PackageError{romanbar}{%
Argument of \string\romannum\space is not a number}{%
The command \string\romannum\space converts an Arabic number into a%
 lower-case Roman one,\MessageBreak%
but the used argument of \string\romannum\space is not an Arabic number%
 but\MessageBreak%
`#1`,\MessageBreak%
which will now be printed unchanged.}%
#1%
}%
}

%    \end{macrocode}
%
% The same for upper-case Roman numbers, |\Romannum| instead of |\Roman|:
%
%    \begin{macrocode}
\DeclareRobustCommand{\Romannum}[1]{%
\ifnumeric{#1}{% is numeric
\@tempcnta=#1\relax%
%% similar to \XRoman from the http://ctan.org/pkg/pageslts package
\ifnum\@tempcnta<1%
  \ifnum #1>0%
    \relax \Roman{#1}%
  \else%
    \ifnum #1<0%
     -\expandafter\@slowromancap\romannumeral\number-#1@%
    \else%
      0%
    \fi%
  \fi%
\else\@Roman\@tempcnta%
\fi%
}{% is not numeric
\PackageError{romanbar}{%
Argument of \string\Romannum\space is not a number}{%
The command \string\Romannum\space converts an Arabic number into an%
 upper-case Roman one,\MessageBreak%
but the used argument of \string\Romannum\space is not an Arabic number%
 but\MessageBreak%
`#1`.}%
#1%
}%
}

%    \end{macrocode}
%
%    \begin{macrocode}
%</package>
%    \end{macrocode}
%
% \newpage
%
% \section{Installation}
%
% \subsection{Downloads\label{ss:Downloads}}
%
% Everything should be available on \CTAN{}, \url{http://www.ctan.org/tex-archive/},
% but may need additional packages themselves.\\
%
% \DescribeMacro{romanbar.dtx}
% For unpacking the |romanbar.dtx| file and constructing the documentation
% it is required:
% \begin{description}
% \item[-] \TeX{} Format \LaTeXe{}, \url{http://www.CTAN.org/}
%
% \item[-] document class \xpackage{ltxdoc}, 2007/11/11, v2.0u,
%           \url{http://ctan.org/pkg/ltxdoc}
%
% \item[-] package \xpackage{holtxdoc}, 2011/02/04, v0.21,
%           \url{http://ctan.org/pkg/holtxdoc}
%
% \item[-] package \xpackage{hypdoc}, 2010/03/26, v1.9,
%           \url{http://ctan.org/pkg/hypdoc}
% \end{description}
%
% \DescribeMacro{romanbar.sty}
% The \texttt{romanbar.sty} for \LaTeXe{} (i.\,e.~each document using
% the \xpackage{romanbar} package) requires:
% \begin{description}
% \item[-] \TeX Format \LaTeXe{}, \url{http://www.CTAN.org/}
% \end{description}
%
% \DescribeMacro{romanbar-example.tex}
% The \texttt{romanbar-example.tex} requires the same files as all
% documents using the \xpackage{romanbar} package and additionally:
% \begin{description}
% \item[-] class \xpackage{article}, 2007/10/19, v1.4h, from \xpackage{classes.dtx}:\\
%           \CTAN{macros/latex/base/classes.dtx}
%
% \item[-] package \xpackage{romanbar}, 2012/01/01, v1.0f,\\
%           %%\url{http://ctan.org/pkg/romanbar}\\
%           \CTAN{macros/latex/contrib/romanbar/}
%   (Well, it is the example file for this package, and because you are reading the
%    documentation for the \xpackage{romanbar} package, it can be assumed that you already
%    have some version of it -- is it the current one?)
% \end{description}
%
% \DescribeMacro{Alternative}
% As possible alternative package in section \ref{sec:Alternatives} there is listed
% \begin{description}
% \item[-] none.
% \end{description}
%
% \DescribeMacro{Oberdiek}
% \DescribeMacro{holtxdoc}
% All packages of \textsc{Heiko Oberdiek's} bundle `oberdiek'
% (especially \xpackage{holtxdoc}) are also available in a TDS compliant ZIP archive:\\
% \CTAN{install/macros/latex/contrib/oberdiek.tds.zip}.\\
% It is probably best to download and use this, because the packages in there
% are quite probably both recent and compatible among themselves.\\
%
% \DescribeMacro{hyperref}
% \noindent \xpackage{hyperref} is not included in that bundle and needs to be
% downloaded separately,\\
% \url{http://mirror.ctan.org/install/macros/latex/contrib/hyperref.tds.zip}.\\
%
% \DescribeMacro{M\"{u}nch}
% A hyperlinked list of my (other) packages can be found at
% \url{http://www.Uni-Bonn.de/~uzs5pv/LaTeX.html}.\\
%
% \pagebreak
%
% \subsection{Package, unpacking TDS}
% \paragraph{Package.} This package should become available on \CTAN{} soon:
% \begin{description}
% \item[\CTAN{macros/latex/contrib/romanbar/romanbar.dtx}]\hspace*{0.1cm} \\
%       The source file.
% \item[\CTAN{macros/latex/contrib/romanbar/romanbar.pdf}]\hspace*{0.1cm} \\
%       The documentation.
% \item[\CTAN{macros/latex/contrib/romanbar/romanbar-example.pdf}]\hspace*{0.1cm} \\
%       The compiled example file, as it should look like.
% \item[\CTAN{macros/latex/contrib/romanbar/README}]\hspace*{0.1cm} \\
%       The README file.
% \end{description}
% There is also a romanbar.tds.zip available:
% \begin{description}
% \item[\CTAN{install/macros/latex/contrib/romanbar.tds.zip}]\hspace*{0.1cm} \\
%       Everything in \xfile{TDS} compliant, compiled format.
% \end{description}
% which additionally contains\\
% \begin{tabular}{ll}
% romanbar.ins & The installation file.\\
% romanbar.drv & The driver to generate the documentation.\\
% romanbar.sty & The \xext{sty}le file.\\
% romanbar-example.tex & The example file.
% \end{tabular}
%
% \bigskip
%
% \noindent For required other packages, please see the preceding subsection.
%
% \paragraph{Unpacking.} The \xfile{.dtx} file is a self-extracting
% \docstrip{} archive. The files are extracted by running the
% \xext{.dtx} through \plainTeX{}:
% \begin{quote}
%   \verb|tex romanbar.dtx|
% \end{quote}
%
% About generating the documentation see paragraph~\ref{GenDoc} below.\\
%
% \paragraph{TDS.} Now the different files must be moved into
% the different directories in your installation TDS tree
% (also known as \xfile{texmf} tree):
% \begin{quote}
% \def\t{^^A
% \begin{tabular}{@{}>{\ttfamily}l@{ $\rightarrow$ }>{\ttfamily}l@{}}
%   romanbar.sty & tex/latex/romanbar/romanbar.sty\\
%   romanbar.pdf & doc/latex/romanbar/romanbar.pdf\\
%   romanbar-example.tex & doc/latex/romanbar/romanbar-example.tex\\
%   romanbar-example.pdf & doc/latex/romanbar/romanbar-example.pdf\\
%   romanbar.dtx & source/latex/romanbar/romanbar.dtx\\
% \end{tabular}^^A
% }^^A
% \sbox0{\t}^^A
% \ifdim\wd0>\linewidth
%   \begingroup
%     \advance\linewidth by\leftmargin
%     \advance\linewidth by\rightmargin
%   \edef\x{\endgroup
%     \def\noexpand\lw{\the\linewidth}^^A
%   }\x
%   \def\lwbox{^^A
%     \leavevmode
%     \hbox to \linewidth{^^A
%       \kern-\leftmargin\relax
%       \hss
%       \usebox0
%       \hss
%       \kern-\rightmargin\relax
%     }^^A
%   }^^A
%   \ifdim\wd0>\lw
%     \sbox0{\small\t}^^A
%     \ifdim\wd0>\linewidth
%       \ifdim\wd0>\lw
%         \sbox0{\footnotesize\t}^^A
%         \ifdim\wd0>\linewidth
%           \ifdim\wd0>\lw
%             \sbox0{\scriptsize\t}^^A
%             \ifdim\wd0>\linewidth
%               \ifdim\wd0>\lw
%                 \sbox0{\tiny\t}^^A
%                 \ifdim\wd0>\linewidth
%                   \lwbox
%                 \else
%                   \usebox0
%                 \fi
%               \else
%                 \lwbox
%               \fi
%             \else
%               \usebox0
%             \fi
%           \else
%             \lwbox
%           \fi
%         \else
%           \usebox0
%         \fi
%       \else
%         \lwbox
%       \fi
%     \else
%       \usebox0
%     \fi
%   \else
%     \lwbox
%   \fi
% \else
%   \usebox0
% \fi
% \end{quote}
% If you have a \xfile{docstrip.cfg} that configures and enables \docstrip{}'s
% \xfile{TDS} installing feature, then some files can already be in the right
% place, see the documentation of \docstrip{}.
%
% \subsection{Refresh file name databases}
%
% If your \TeX{}~distribution (\teTeX{}, \mikTeX{},\dots{}) relies on file name
% databases, you must refresh these. For example, \teTeX{} users run
% \verb|texhash| or \verb|mktexlsr|.
%
% \subsection{Some details for the interested}
%
% \paragraph{Unpacking with \LaTeX{}.}
% The \xfile{.dtx} chooses its action depending on the format:
% \begin{description}
% \item[\plainTeX:] Run \docstrip{} and extract the files.
% \item[\LaTeX:] Generate the documentation.
% \end{description}
% If you insist on using \LaTeX{} for \docstrip{} (really,
% \docstrip{} does not need \LaTeX{}), then inform the autodetect routine
% about your intention:
% \begin{quote}
%   \verb|latex \let\install=y\input{romanbar.dtx}|
% \end{quote}
% Do not forget to quote the argument according to the demands
% of your shell.
%
% \paragraph{Generating the documentation.\label{GenDoc}}
% You can use both the \xfile{.dtx} or the \xfile{.drv} to generate
% the documentation. The process can be configured by a
% configuration file \xfile{ltxdoc.cfg}. For instance, put the following
% line into this file, if you want to have A4 as paper format:
% \begin{quote}
%   \verb|\PassOptionsToClass{a4paper}{article}|
% \end{quote}
%
% \noindent An example follows how to generate the
% documentation with \pdfLaTeX{}:
%
% \begin{quote}
%\begin{verbatim}
%pdflatex romanbar.dtx
%makeindex -s gind.ist romanbar.idx
%pdflatex romanbar.dtx
%makeindex -s gind.ist romanbar.idx
%pdflatex romanbar.dtx
%\end{verbatim}
% \end{quote}
%
% \subsection{Compiling the example}
%
% The example file, \textsf{romanbar-example.tex}, can be compiled via\\
% |(pdf)(la)tex romanbar-example.tex|.
%
% \section{Acknowledgements}
%
% I would like to thank \textsc{Heiko Oberdiek} for providing
% the \xpackage{hyperref} as well as a~lot~(!) of other useful packages
% (from which I also got everything I know about creating a file in \xext{dtx}
% format, ok, say it: copying),
% Prof.~\textsc{Enrico Gregorio} (egreg) for his answer
% \url{http://tex.stackexchange.com/questions/24065/roman-numerals-formatting/24084#24084},
% the \Newsgroup{comp.text.tex} and \Newsgroup{de.comp.text.tex}
% newsgroups as well as \url{http://tex.stackexchange.com}
% for their help in all things \TeX{}.
%
% \pagebreak
%
% \phantomsection
% \begin{History}\label{History}
%   \begin{Version}{2011/07/25 v1.0a}
%     \item First code by Prof.~\textsc{Enrico Gregorio} (egreg) at
%             \url{http://tex.stackexchange.com/questions/24065/roman-numerals-formatting/24084#24084}
%   \end{Version}
%   \begin{Version}{2011/07/26 v1.0b}
%     \item Packed into a \xext{sty} file.
%     \item Removed the requirement of \eTeX{} for the package.
%   \end{Version}
%   \begin{Version}{2011/08/25 v1.0c}
%     \item Renamed \xpackage{romanbar} package.
%     \item Minor details.
%   \end{Version}
%   \begin{Version}{2011/09/16 v1.0d}
%     \item Made commands robust.
%     \item Minor details.
%   \end{Version}
%   \begin{Version}{2011/12/16 v1.0e}
%     \item Additionally accepts now Arabic numbers and converts them to Roman ones with bars.
%     \item Also non-positive numbers are now accepted.
%   \end{Version}
%   \begin{Version}{2012/01/01 v1.0f}
%     \item Added |\romannum| and |\Romannum| to convert Arabic to Roman numbers (without bars).
%     \item |\r@iseL|.
%     \item Created a \xfile{dtx}, example, and README.
%     \item Upload to \CTAN{}.
%   \end{Version}
% \end{History}
%
% \bigskip
%
% When you find a mistake or have a suggestion for an improvement of this package,
% please send an e-mail to the maintainer, thanks! (Please see BUG REPORTS in the README.)
%
% \pagebreak
%
% \PrintIndex
%
% \Finale
\endinput
%        (quote the arguments according to the demands of your shell)
%
% Documentation:
%    (a) If romanbar.drv is present:
%           (pdf)latex romanbar.drv
%           makeindex -s gind.ist romanbar.idx
%           (pdf)latex romanbar.drv
%           makeindex -s gind.ist romanbar.idx
%           (pdf)latex romanbar.drv
%    (b) Without romanbar.drv:
%           (pdf)latex romanbar.dtx
%           makeindex -s gind.ist romanbar.idx
%           (pdf)latex romanbar.dtx
%           makeindex -s gind.ist romanbar.idx
%           (pdf)latex romanbar.dtx
%
%    The class ltxdoc loads the configuration file ltxdoc.cfg
%    if available. Here you can specify further options, e.g.
%    use DIN A4 as paper format:
%       \PassOptionsToClass{a4paper}{article}
%
% Installation:
%    TDS:tex/latex/romanbar/romanbar.sty
%    TDS:doc/latex/romanbar/romanbar.pdf
%    TDS:doc/latex/romanbar/romanbar-example.tex
%    TDS:doc/latex/romanbar/romanbar-example.pdf
%    TDS:source/latex/romanbar/romanbar.dtx
%
%<*ignore>
\begingroup
  \catcode123=1 %
  \catcode125=2 %
  \def\x{LaTeX2e}%
\expandafter\endgroup
\ifcase 0\ifx\install y1\fi\expandafter
         \ifx\csname processbatchFile\endcsname\relax\else1\fi
         \ifx\fmtname\x\else 1\fi\relax
\else\csname fi\endcsname
%</ignore>
%<*install>
\input docstrip.tex
\Msg{********************************************************************}
\Msg{* Installation                                                     *}
\Msg{* Package: romanbar 2012/01/01 v1.0f Roman numbers with bars (HMM) *}
\Msg{********************************************************************}

\keepsilent
\askforoverwritefalse

\let\MetaPrefix\relax
\preamble

This is a generated file.

Project: romanbar
Version: 2012/01/01 v1.0f

Copyright (C) 2011 by
    H.-Martin M"unch <Martin dot Muench at Uni-Bonn dot de>

The usual disclaimer applies:
If it doesn't work right that's your problem.
(Nevertheless, send an e-mail to the maintainer
 when you find an error in this package.)

This work may be distributed and/or modified under the
conditions of the LaTeX Project Public License, either
version 1.3c of this license or (at your option) any later
version. This version of this license is in
   http://www.latex-project.org/lppl/lppl-1-3c.txt
and the latest version of this license is in
   http://www.latex-project.org/lppl.txt
and version 1.3c or later is part of all distributions of
LaTeX version 2005/12/01 or later.

This work has the LPPL maintenance status "maintained".

The Current Maintainer of this work is H.-Martin Muench.

This work consists of the main source file romanbar.dtx,
the README, and the derived files
   romanbar.sty, romanbar.pdf,
   romanbar.ins, romanbar.drv,
   romanbar-example.tex, romanbar-example.pdf.

\endpreamble
\let\MetaPrefix\DoubleperCent

\generate{%
  \file{romanbar.ins}{\from{romanbar.dtx}{install}}%
  \file{romanbar.drv}{\from{romanbar.dtx}{driver}}%
  \usedir{tex/latex/romanbar}%
  \file{romanbar.sty}{\from{romanbar.dtx}{package}}%
  \usedir{doc/latex/romanbar}%
  \file{romanbar-example.tex}{\from{romanbar.dtx}{example}}%
}

\catcode32=13\relax% active space
\let =\space%
\Msg{************************************************************************}
\Msg{*}
\Msg{* To finish the installation you have to move the following}
\Msg{* file into a directory searched by TeX:}
\Msg{*}
\Msg{*     romanbar.sty}
\Msg{*}
\Msg{* To produce the documentation run the file `romanbar.drv'}
\Msg{* through (pdf)LaTeX, e.g.}
\Msg{*  pdflatex romanbar.drv}
\Msg{*  makeindex -s gind.ist romanbar.idx}
\Msg{*  pdflatex romanbar.drv}
\Msg{*  makeindex -s gind.ist romanbar.idx}
\Msg{*  pdflatex romanbar.drv}
\Msg{*}
\Msg{* At least three runs are necessary e.g. to get the}
\Msg{*  references right!}
\Msg{*}
\Msg{* Happy TeXing!}
\Msg{*}
\Msg{************************************************************************}

\endbatchfile
%</install>
%<*ignore>
\fi
%</ignore>
%
% \section{The documentation driver file}
%
% The next bit of code contains the documentation driver file for
% \TeX{}, i.\,e., the file that will produce the documentation you
% are currently reading. It will be extracted from this file by the
% \texttt{docstrip} programme. That is, run \LaTeX{} on \texttt{docstrip}
% and specify the \texttt{driver} option when \texttt{docstrip}
% asks for options.
%
%    \begin{macrocode}
%<*driver>
\NeedsTeXFormat{LaTeX2e}[2011/06/27]
\ProvidesFile{romanbar.drv}%
  [2012/01/01 v1.0f Roman numbers with bars (HMM)]
\documentclass{ltxdoc}[2007/11/11]% v2.0u
\usepackage{holtxdoc}[2011/02/04]%  v0.21
%% romanbar may work with earlier versions of LaTeX2e and those
%% class and package, but this was not tested.
%% Please consider updating your LaTeX, class, and package
%% to the most recent version (if they are not already the most
%% recent version).
\hypersetup{%
 pdfsubject={Writing Roman numbers with bars (HMM)},%
 pdfkeywords={LaTeX, romanbar, roman, Roman, bars, H.-Martin Muench},%
 pdfencoding=auto,%
 pdflang={en},%
 breaklinks=true,%
 linktoc=all,%
 pdfstartview=FitH,%
 pdfpagelayout=OneColumn,%
 bookmarksnumbered=true,%
 bookmarksopen=true,%
 bookmarksopenlevel=2,%
 pdfmenubar=true,%
 pdftoolbar=true,%
 pdfwindowui=true,%
 pdfnewwindow=true%
}
\CodelineIndex
\hyphenation{docu-ment ana-lysed}
\gdef\unit#1{\mathord{\thinspace\mathrm{#1}}}%
\makeatletter
\@ifundefined{eTeX}{\gdef\eTeX{$\m@th \varepsilon $-\TeX }}{% else \relax
}
\makeatother
\begin{document}
  \DocInput{romanbar.dtx}%
\end{document}
%</driver>
%    \end{macrocode}
%
% \fi
%
% \CheckSum{179}
%
% \CharacterTable
%  {Upper-case    \A\B\C\D\E\F\G\H\I\J\K\L\M\N\O\P\Q\R\S\T\U\V\W\X\Y\Z
%   Lower-case    \a\b\c\d\e\f\g\h\i\j\k\l\m\n\o\p\q\r\s\t\u\v\w\x\y\z
%   Digits        \0\1\2\3\4\5\6\7\8\9
%   Exclamation   \!     Double quote  \"     Hash (number) \#
%   Dollar        \$     Percent       \%     Ampersand     \&
%   Acute accent  \'     Left paren    \(     Right paren   \)
%   Asterisk      \*     Plus          \+     Comma         \,
%   Minus         \-     Point         \.     Solidus       \/
%   Colon         \:     Semicolon     \;     Less than     \<
%   Equals        \=     Greater than  \>     Question mark \?
%   Commercial at \@     Left bracket  \[     Backslash     \\
%   Right bracket \]     Circumflex    \^     Underscore    \_
%   Grave accent  \`     Left brace    \{     Vertical bar  \|
%   Right brace   \}     Tilde         \~}
%
% \GetFileInfo{romanbar.drv}
%
% \begingroup
%   \def\x{\#,\$,\^,\_,\~,\ ,\&,\{,\},\%}%
%   \makeatletter
%   \@onelevel@sanitize\x
% \expandafter\endgroup
% \expandafter\DoNotIndex\expandafter{\x}
% \expandafter\DoNotIndex\expandafter{\string\ }
% \begingroup
%   \makeatletter
%     \lccode`9=32\relax
%     \lowercase{%^^A
%       \edef\x{\noexpand\DoNotIndex{\@backslashchar9}}%^^A
%     }%^^A
%   \expandafter\endgroup\x
%
% \DoNotIndex{\\,\,}
% \DoNotIndex{\documentclass,\usepackage,\ProvidesPackage,\begin,\end}
% \DoNotIndex{\MessageBreak}
% \DoNotIndex{\NeedsTeXFormat,\DoNotIndex,\verb}
% \DoNotIndex{\def,\edef,\gdef,\global}
% \DoNotIndex{\ifx,\listfiles,\mathord,\mathrm}
% \DoNotIndex{\bigskip,\space,\thinspace,\ldots}
% \DoNotIndex{\indent,\noindent,\newline,\linebreak,\pagebreak,\newpage}
% \DoNotIndex{\textbf,\textit,\textsf,\texttt,\textquotedblleft,\textquotedblright}
% \DoNotIndex{\plainTeX,\TeX,\LaTeX,\pdfLaTeX}
% \DoNotIndex{\item}
%
% \title{The \xpackage{romanbar} package}
% \date{2012/01/01 v1.0f}
% \author{H.-Martin M\"{u}nch\\\xemail{Martin.Muench at Uni-Bonn.de}}
%
% \maketitle
%
% \begin{abstract}
%  This \LaTeX{} package allows to write Roman numbers
%  (or any other text) with bars. (Additionally, commands for converting Arabic numbers
%  into Roman ones are provided and an |\ifnumeric| test function.)
% \end{abstract}
%
% \bigskip
%
% \noindent Disclaimer for web links: The author is not responsible for any contents
% referred to in this work unless he has full knowledge of illegal contents.
% If any damage occurs by the use of information presented there, only the
% author of the respective pages might be liable, not the one who has referred
% to these pages.
%
% \bigskip
%
% \noindent {\color{green} Save per page about $200\unit{ml}$ water,
% $2\unit{g}$ CO$_{2}$ and $2\unit{g}$ wood:\\
% Therefore please print only if this is really necessary.}
%
% \pagebreak
%
% \tableofcontents
%
% \pagebreak
%
% \section{Introduction}
%
% \indent This \LaTeX{} package allows to write Roman numbers
% (or any other text) with bars. (Additionally, command for converting Arabic numbers
% into Roman ones are provided and an |\ifnumeric| test function.)
%
% \bigskip
%
% \section{Usage}
%
% \indent Just load the package placing
% \begin{quote}
%   |\usepackage{romanbar}|
% \end{quote}
% \noindent in the preamble of your \LaTeXe{} source file.\\
% |\Romanbar{...}| then produces a Roman number with bars (please see the example file).
% |\ifnumeric{test}{true}{false}| tests for |test| being numeric.
% |\romannum{...}| turns an Arabic number into a lowercase Roman one, and
% |\Romannum{...}| turns an Arabic number into an uppercase Roman one.
%
% \subsection*{Options}
% \DescribeMacro{options}
% \indent The \xpackage{romanbar} package takes no options.
%
% \section{Alternatives\label{sec:Alternatives}}
%
% \begin{description}
% \item[-] the original code written by Prof. Enrico Gregorio\\
%   (\url{http://profs.sci.univr.it/~gregorio/}),\\
%   \url{http://tex.stackexchange.com/questions/24065/roman-numerals-formatting/24084#24084}
%\begin{verbatim}
%\def\barroman#1{\sbox0{#1}\dimen0=\dimexpr\wd0+1pt\relax
%  \makebox[\dimen0]{\rlap{\vrule width\dimen0 height 0.06ex depth 0.06ex}%
%    \rlap{\vrule width\dimen0 height\dimexpr\ht0+0.03ex\relax
%            depth\dimexpr-\ht0+0.09ex\relax}%
%    \kern.5pt#1\kern.5pt}}
%
%\barroman{I} \barroman{XI}
%\end{verbatim}
%
% \item[-]  Some fonts provide single characters for Roman numerals,
%   cf.~e.\,g.
% \url{http://tex.stackexchange.com/questions/38695/using-unicode-roman-numerals-in-xetex}.
% \end{description}
%
% \noindent (You programmed or found another alternative,
%  which is available at \CTAN{}?\\
%  OK, send an e-mail to me with the name, location at \CTAN{},
%  and a short notice, and I will probably include it in
%  the list above.)
%
% \newpage
%
% \section{Example}
%
%    \begin{macrocode}
%<*example>
\documentclass[british]{article}[2007/10/19]% v1.4h
%%%%%%%%%%%%%%%%%%%%%%%%%%%%%%%%%%%%%%%%%%%%%%%%%%%%%%%%%%%%%%%%%%%%%
\usepackage[extension=pdf,%
 plainpages=false,%
 pdfpagelabels=true,%
 hyperindex=false,%
 pdflang={en},%
 pdftitle={romanbar package example},%
 pdfauthor={H.-Martin Muench},%
 pdfsubject={Example for the romanbar package},%
 pdfkeywords={LaTeX, romanbar, roman, Roman, bars, H.-Martin Muench},%
 pdfview={XYZ null null 1},%
 pdfstartview={XYZ null null 1},%
 pdfpagelayout=SinglePage]{hyperref}[2011/12/04]% v6.82m
\usepackage{romanbar}[2012/01/01]% v1.0f
\gdef\unit#1{\mathord{\thinspace\mathrm{#1}}}%
\listfiles
\begin{document}
\pagenumbering{arabic}
\section*{Example for romanbar}

This example demonstrates the use of package\newline
\textsf{romanbar}, v1.0f as of 2012/01/01 (HMM).\newline
There are no options to be used.\newline

\noindent For more details please see the documentation!\newline

\noindent Save per page about $200\unit{ml}$ water,
$2\unit{g}$ CO$_{2}$ and $2\unit{g}$ wood:\newline
Therefore please print only if this is really necessary.\newline

\noindent This package provides the command \verb|\Romanbar|
to print bars below and over the following:

\begin{description}
\item[-] Roman numbers: \verb|\Romanbar{MMXII}| prints \Romanbar{MMXII}

\item[-] Arabic numbers turned into upper-case Roman numbers:\newline
           \verb|\Romanbar{2012}| prints \Romanbar{2012}

\item[-] negative Arabic numbers turned into upper-case Roman numbers
           \newline
           (although historically there were no negative Roman numbers):
           \newline
           \verb|\Romanbar{-12}| prints \Romanbar{-12}

\item[-] zero Arabic number ($0$; although historically
           there was no Roman zero):\newline
           \verb|\Romanbar{0}| prints \Romanbar{0}

\item[-] some arbitrary text:
           \verb|\Romanbar{Caesar}| prints \Romanbar{Caesar}\newline
           (with descenders: \Romanbar{AgjpqyW})


\newcounter{example}
\setcounter{example}{21}

\item[-] some counter's value:
           \verb|\Romanbar{\theexample}| prints \Romanbar{\theexample}
           \newline
           (where the value of \texttt{example} is \theexample)

\item[-] Arabic numbers, without turning them into upper-case Roman
           numbers:\newline
           \verb|\Romanbar{\relax 2012}| prints \Romanbar{\relax  2012}
\end{description}

Special care was taken for "L" (50), e.\,g. in 555/DLV: \Romanbar{555}.\\


\verb|\romannum{...}| converts an Arabic number into a lower-case Roman one,
e.\,g. \verb|\romannum{2012}| prints \romannum{2012}, and
\verb|\Romannum{...}| converts an Arabic number into an upper-case Roman one,
e.\,g. \verb|\Romannum{2012}| prints \Romannum{2012}.

\end{document}
%</example>
%    \end{macrocode}
%
% \StopEventually{}
%
% \newpage
%
% \section{The implementation}
%
% We start off by checking that we are loading into \LaTeXe{} and
% announcing the name and version of this package.
%
%    \begin{macrocode}
%<*package>
%    \end{macrocode}
%
%    \begin{macrocode}
\NeedsTeXFormat{LaTeX2e}[2011/06/27]
\ProvidesPackage{romanbar}[2012/01/01 v1.0f
              Roman numbers with bars (HMM)]

%    \end{macrocode}
%
% A short description of the \xpackage{romanbar} package:
%
%    \begin{macrocode}
%% Allows to write Roman numbers (or any other text) with bars
%% and to convert Arabic numbers into Roman ones.
%    \end{macrocode}
%
% A last information for the user:
%
%    \begin{macrocode}
%% romanbar may work with earlier versions of LaTeX,
%% but this was not tested. Please consider updating
%% your LaTeX to the most recent version
%% (if it is not already the most recent version).

%    \end{macrocode}
%
% \indent See subsection~\ref{ss:Downloads} about how to get it.\\
%
% There are no options to be processed.
%
% A command to gobble a possible minus sign is needed and
% |\gobbleminus| from \url{http://www.tex.ac.uk/cgi-bin/texfaq2html?label=isitanum}
% is used.
%
%    \begin{macrocode}
\def\gobbleminus#1{\ifx-#1\else#1\fi}
%% from http://www.tex.ac.uk/cgi-bin/texfaq2html?label=isitanum

%    \end{macrocode}
%
% We want to test whether the argument passed to |\Romanbar| is numeric
% and therefore define |\ifnumeric{test}{true}{false}|,
% where |test| is to be analysed, |true| is the code to be executed,
% if |test| is numeric, and |false| is the code to be executed,
% if |test| is not numeric. This is done similar to\\
% \url{http://tex.stackexchange.com/a/17119}.
%
%    \begin{macrocode}
\newcommand{\ifnumeric}[3]{%
%% similar to http://tex.stackexchange.com/a/17119
\sbox\z@{\@tempcnta=0\gobbleminus#1\relax}%
\ifdim\wd0>\z@\relax#3% is not numeric
\else#2% is numeric
\fi%
}

%    \end{macrocode}
%
% We define |\r@iseL| to raise any L (50), otherwise a good deal
% of the horizontal part of the letter would be swallowed by the lower bar
% under the \textquotedblleft{}number\textquotedblright{}.
%
%    \begin{macrocode}
\def\r@iseL#1{\ifx @#1% then terminate
\else%
\if L#1\raisebox{0.05ex}{L}\else #1\fi%
\expandafter\r@iseL%
\fi%
}

%    \end{macrocode}
%
% \pagebreak
%
% We define the |\Romanbar| command.
%
%    \begin{macrocode}
\DeclareRobustCommand{\Romanbar}[1]{%
\ifnumeric{#1}{% is numeric
%    \end{macrocode}
%
% If its parameter is numeric, the temporary counter |\@tempcnta| is set to the value.
%
%    \begin{macrocode}
\@tempcnta=#1\relax%
%    \end{macrocode}
%
% Normally there are no non-positive Roman numbers, therefore we do some trick
% similar to |\XRoman| from the \url{http://ctan.org/pkg/pageslts} package.
%
%    \begin{macrocode}
%% similar to \XRoman from the http://ctan.org/pkg/pageslts package
\ifnum\@tempcnta<1%
  \ifnum #1>0%
    \relax \Roman@bar{\Roman{#1}}%
  \else%
    \ifnum #1<0%
     -\Roman@bar{\expandafter\@slowromancap\romannumeral\number-#1@%
     }%
    \else%
      \Roman@bar{0}%
    \fi%
  \fi%
\else\Roman@bar{\@Roman\@tempcnta}%
\fi%
}{% is not numeric
\Roman@bar{#1}%
}%
}

%    \end{macrocode}
%
% Whether it is numeric or not, in the end |\Roman@bar| is applied to it.
%
%    \begin{macrocode}
\DeclareRobustCommand{\Roman@bar}[1]{% #1 is in Roman, i.e. MMXII
%% (or any other text, "Caesar" would work, too).
%% similar to code of  by Prof. Enrico Gregorio (egreg) at
%% http://tex.stackexchange.com/questions/24065/roman-numerals-formatting/24084#24084
\@bsphack%
\edef\romanbartmp{#1}%
%% height:
\sbox0{\textsf{\romanbartmp}}%
%% top line:
%% + top of top line:
\@tempdima=\ht0%
\advance\@tempdima+0.05ex%0.03
\dimen1=\@tempdima%
%% + bottom of top line:
\@tempdima=-\ht0%
\advance\@tempdima+0.05ex%0.07
\dimen2=\@tempdima%
%% width:
\sbox0{\textsf{\expandafter\r@iseL\romanbartmp @}}%
\@tempdima=\wd0%
\advance\@tempdima+1pt%
\dimen0=\@tempdima%
\@esphack%
\makebox[\dimen0]{%
\rlap{\vrule width\dimen0 height\dimen1 depth\dimen2}% top    line
\rlap{\vrule width\dimen0 height 0.06ex depth 0.03ex}% bottom line
\kern0.5pt\textsf{\expandafter\r@iseL\romanbartmp @}\kern0.5pt}%
}

%    \end{macrocode}
%
% While it is not needed for the bars, it is sometimes asked for a command
% to convert an Arabic number into a Roman one. |\roman{...}| can only be applied
% to a counter, i.\,e. |\roman{12}| does not work, but
%\begin{verbatim}
%\newcounter{examplecounter}
%\setcounter{examplecounter}{12}
%\roman{examplecounter}
%\end{verbatim}
% would be necessary. Further non-negative values would not work.
% |\romannum| accepts any number as argument. (If the argument is no number,
% an error message is given and the argument printed (executed)
%  without applying anything to it.)
%
%    \begin{macrocode}
\DeclareRobustCommand{\romannum}[1]{%
\ifnumeric{#1}{% is numeric
\@tempcnta=#1\relax%
%% similar to \xroman from the http://ctan.org/pkg/pageslts package
\ifnum\@tempcnta<1%
  \ifnum #1>0%
    \relax \roman{#1}%
  \else%
    \ifnum #1<0%
     -\romannumeral\number-#1@%
    \else%
      0%
    \fi%
  \fi%
\else\@roman\@tempcnta%
\fi%
}{% is not numeric
\PackageError{romanbar}{%
Argument of \string\romannum\space is not a number}{%
The command \string\romannum\space converts an Arabic number into a%
 lower-case Roman one,\MessageBreak%
but the used argument of \string\romannum\space is not an Arabic number%
 but\MessageBreak%
`#1`,\MessageBreak%
which will now be printed unchanged.}%
#1%
}%
}

%    \end{macrocode}
%
% The same for upper-case Roman numbers, |\Romannum| instead of |\Roman|:
%
%    \begin{macrocode}
\DeclareRobustCommand{\Romannum}[1]{%
\ifnumeric{#1}{% is numeric
\@tempcnta=#1\relax%
%% similar to \XRoman from the http://ctan.org/pkg/pageslts package
\ifnum\@tempcnta<1%
  \ifnum #1>0%
    \relax \Roman{#1}%
  \else%
    \ifnum #1<0%
     -\expandafter\@slowromancap\romannumeral\number-#1@%
    \else%
      0%
    \fi%
  \fi%
\else\@Roman\@tempcnta%
\fi%
}{% is not numeric
\PackageError{romanbar}{%
Argument of \string\Romannum\space is not a number}{%
The command \string\Romannum\space converts an Arabic number into an%
 upper-case Roman one,\MessageBreak%
but the used argument of \string\Romannum\space is not an Arabic number%
 but\MessageBreak%
`#1`.}%
#1%
}%
}

%    \end{macrocode}
%
%    \begin{macrocode}
%</package>
%    \end{macrocode}
%
% \newpage
%
% \section{Installation}
%
% \subsection{Downloads\label{ss:Downloads}}
%
% Everything should be available on \CTAN{}, \url{http://www.ctan.org/tex-archive/},
% but may need additional packages themselves.\\
%
% \DescribeMacro{romanbar.dtx}
% For unpacking the |romanbar.dtx| file and constructing the documentation
% it is required:
% \begin{description}
% \item[-] \TeX{} Format \LaTeXe{}, \url{http://www.CTAN.org/}
%
% \item[-] document class \xpackage{ltxdoc}, 2007/11/11, v2.0u,
%           \url{http://ctan.org/pkg/ltxdoc}
%
% \item[-] package \xpackage{holtxdoc}, 2011/02/04, v0.21,
%           \url{http://ctan.org/pkg/holtxdoc}
%
% \item[-] package \xpackage{hypdoc}, 2010/03/26, v1.9,
%           \url{http://ctan.org/pkg/hypdoc}
% \end{description}
%
% \DescribeMacro{romanbar.sty}
% The \texttt{romanbar.sty} for \LaTeXe{} (i.\,e.~each document using
% the \xpackage{romanbar} package) requires:
% \begin{description}
% \item[-] \TeX Format \LaTeXe{}, \url{http://www.CTAN.org/}
% \end{description}
%
% \DescribeMacro{romanbar-example.tex}
% The \texttt{romanbar-example.tex} requires the same files as all
% documents using the \xpackage{romanbar} package and additionally:
% \begin{description}
% \item[-] class \xpackage{article}, 2007/10/19, v1.4h, from \xpackage{classes.dtx}:\\
%           \CTAN{macros/latex/base/classes.dtx}
%
% \item[-] package \xpackage{romanbar}, 2012/01/01, v1.0f,\\
%           %%\url{http://ctan.org/pkg/romanbar}\\
%           \CTAN{macros/latex/contrib/romanbar/}
%   (Well, it is the example file for this package, and because you are reading the
%    documentation for the \xpackage{romanbar} package, it can be assumed that you already
%    have some version of it -- is it the current one?)
% \end{description}
%
% \DescribeMacro{Alternative}
% As possible alternative package in section \ref{sec:Alternatives} there is listed
% \begin{description}
% \item[-] none.
% \end{description}
%
% \DescribeMacro{Oberdiek}
% \DescribeMacro{holtxdoc}
% All packages of \textsc{Heiko Oberdiek's} bundle `oberdiek'
% (especially \xpackage{holtxdoc}) are also available in a TDS compliant ZIP archive:\\
% \CTAN{install/macros/latex/contrib/oberdiek.tds.zip}.\\
% It is probably best to download and use this, because the packages in there
% are quite probably both recent and compatible among themselves.\\
%
% \DescribeMacro{hyperref}
% \noindent \xpackage{hyperref} is not included in that bundle and needs to be
% downloaded separately,\\
% \url{http://mirror.ctan.org/install/macros/latex/contrib/hyperref.tds.zip}.\\
%
% \DescribeMacro{M\"{u}nch}
% A hyperlinked list of my (other) packages can be found at
% \url{http://www.Uni-Bonn.de/~uzs5pv/LaTeX.html}.\\
%
% \pagebreak
%
% \subsection{Package, unpacking TDS}
% \paragraph{Package.} This package should become available on \CTAN{} soon:
% \begin{description}
% \item[\CTAN{macros/latex/contrib/romanbar/romanbar.dtx}]\hspace*{0.1cm} \\
%       The source file.
% \item[\CTAN{macros/latex/contrib/romanbar/romanbar.pdf}]\hspace*{0.1cm} \\
%       The documentation.
% \item[\CTAN{macros/latex/contrib/romanbar/romanbar-example.pdf}]\hspace*{0.1cm} \\
%       The compiled example file, as it should look like.
% \item[\CTAN{macros/latex/contrib/romanbar/README}]\hspace*{0.1cm} \\
%       The README file.
% \end{description}
% There is also a romanbar.tds.zip available:
% \begin{description}
% \item[\CTAN{install/macros/latex/contrib/romanbar.tds.zip}]\hspace*{0.1cm} \\
%       Everything in \xfile{TDS} compliant, compiled format.
% \end{description}
% which additionally contains\\
% \begin{tabular}{ll}
% romanbar.ins & The installation file.\\
% romanbar.drv & The driver to generate the documentation.\\
% romanbar.sty & The \xext{sty}le file.\\
% romanbar-example.tex & The example file.
% \end{tabular}
%
% \bigskip
%
% \noindent For required other packages, please see the preceding subsection.
%
% \paragraph{Unpacking.} The \xfile{.dtx} file is a self-extracting
% \docstrip{} archive. The files are extracted by running the
% \xext{.dtx} through \plainTeX{}:
% \begin{quote}
%   \verb|tex romanbar.dtx|
% \end{quote}
%
% About generating the documentation see paragraph~\ref{GenDoc} below.\\
%
% \paragraph{TDS.} Now the different files must be moved into
% the different directories in your installation TDS tree
% (also known as \xfile{texmf} tree):
% \begin{quote}
% \def\t{^^A
% \begin{tabular}{@{}>{\ttfamily}l@{ $\rightarrow$ }>{\ttfamily}l@{}}
%   romanbar.sty & tex/latex/romanbar/romanbar.sty\\
%   romanbar.pdf & doc/latex/romanbar/romanbar.pdf\\
%   romanbar-example.tex & doc/latex/romanbar/romanbar-example.tex\\
%   romanbar-example.pdf & doc/latex/romanbar/romanbar-example.pdf\\
%   romanbar.dtx & source/latex/romanbar/romanbar.dtx\\
% \end{tabular}^^A
% }^^A
% \sbox0{\t}^^A
% \ifdim\wd0>\linewidth
%   \begingroup
%     \advance\linewidth by\leftmargin
%     \advance\linewidth by\rightmargin
%   \edef\x{\endgroup
%     \def\noexpand\lw{\the\linewidth}^^A
%   }\x
%   \def\lwbox{^^A
%     \leavevmode
%     \hbox to \linewidth{^^A
%       \kern-\leftmargin\relax
%       \hss
%       \usebox0
%       \hss
%       \kern-\rightmargin\relax
%     }^^A
%   }^^A
%   \ifdim\wd0>\lw
%     \sbox0{\small\t}^^A
%     \ifdim\wd0>\linewidth
%       \ifdim\wd0>\lw
%         \sbox0{\footnotesize\t}^^A
%         \ifdim\wd0>\linewidth
%           \ifdim\wd0>\lw
%             \sbox0{\scriptsize\t}^^A
%             \ifdim\wd0>\linewidth
%               \ifdim\wd0>\lw
%                 \sbox0{\tiny\t}^^A
%                 \ifdim\wd0>\linewidth
%                   \lwbox
%                 \else
%                   \usebox0
%                 \fi
%               \else
%                 \lwbox
%               \fi
%             \else
%               \usebox0
%             \fi
%           \else
%             \lwbox
%           \fi
%         \else
%           \usebox0
%         \fi
%       \else
%         \lwbox
%       \fi
%     \else
%       \usebox0
%     \fi
%   \else
%     \lwbox
%   \fi
% \else
%   \usebox0
% \fi
% \end{quote}
% If you have a \xfile{docstrip.cfg} that configures and enables \docstrip{}'s
% \xfile{TDS} installing feature, then some files can already be in the right
% place, see the documentation of \docstrip{}.
%
% \subsection{Refresh file name databases}
%
% If your \TeX{}~distribution (\teTeX{}, \mikTeX{},\dots{}) relies on file name
% databases, you must refresh these. For example, \teTeX{} users run
% \verb|texhash| or \verb|mktexlsr|.
%
% \subsection{Some details for the interested}
%
% \paragraph{Unpacking with \LaTeX{}.}
% The \xfile{.dtx} chooses its action depending on the format:
% \begin{description}
% \item[\plainTeX:] Run \docstrip{} and extract the files.
% \item[\LaTeX:] Generate the documentation.
% \end{description}
% If you insist on using \LaTeX{} for \docstrip{} (really,
% \docstrip{} does not need \LaTeX{}), then inform the autodetect routine
% about your intention:
% \begin{quote}
%   \verb|latex \let\install=y% \iffalse meta-comment
%
% File: romanbar.dtx
% Version: 2012/01/01 v1.0f
%
% Copyright (C) 2011 by
%    H.-Martin M"unch <Martin dot Muench at Uni-Bonn dot de>
%
% The original code for the bars was written by Prof. Enrico Gregorio
% (http://profs.sci.univr.it/~gregorio/), and published at
% http://tex.stackexchange.com/questions/24065/roman-numerals-formatting/24084#24084
% I (H.-Martin Münch) submitted this package to CTAN
% (after diverse updates of the code and writing a documentation).
%
% This work may be distributed and/or modified under the
% conditions of the LaTeX Project Public License, either
% version 1.3c of this license or (at your option) any later
% version. This version of this license is in
%    http://www.latex-project.org/lppl/lppl-1-3c.txt
% and the latest version of this license is in
%    http://www.latex-project.org/lppl.txt
% and version 1.3c or later is part of all distributions of
% LaTeX version 2005/12/01 or later.
%
% This work has the LPPL maintenance status "maintained".
%
% The Current Maintainer of this work is H.-Martin Muench.
%
% This work consists of the main source file romanbar.dtx,
% the README, and the derived files
%    romanbar.sty, romanbar.pdf,
%    romanbar.ins, romanbar.drv,
%    romanbar-example.tex, romanbar-example.pdf.
%
% Distribution:
%    CTAN:macros/latex/contrib/romanbar/romanbar.dtx
%    CTAN:macros/latex/contrib/romanbar/romanbar.pdf
%    CTAN:install/macros/latex/contrib/romanbar.tds.zip
%
% Cf. http://ctan.org/pkg/romanbar and
% http://www.tex.ac.uk/tex-archive/help/Catalogue/entries/romanbar.html
% (It might take some time for the entries to be created.)
%
% Unpacking:
%    (a) If romanbar.ins is present:
%           tex romanbar.ins
%    (b) Without romanbar.ins:
%           tex romanbar.dtx
%    (c) If you insist on using LaTeX
%           latex \let\install=y\input{romanbar.dtx}
%        (quote the arguments according to the demands of your shell)
%
% Documentation:
%    (a) If romanbar.drv is present:
%           (pdf)latex romanbar.drv
%           makeindex -s gind.ist romanbar.idx
%           (pdf)latex romanbar.drv
%           makeindex -s gind.ist romanbar.idx
%           (pdf)latex romanbar.drv
%    (b) Without romanbar.drv:
%           (pdf)latex romanbar.dtx
%           makeindex -s gind.ist romanbar.idx
%           (pdf)latex romanbar.dtx
%           makeindex -s gind.ist romanbar.idx
%           (pdf)latex romanbar.dtx
%
%    The class ltxdoc loads the configuration file ltxdoc.cfg
%    if available. Here you can specify further options, e.g.
%    use DIN A4 as paper format:
%       \PassOptionsToClass{a4paper}{article}
%
% Installation:
%    TDS:tex/latex/romanbar/romanbar.sty
%    TDS:doc/latex/romanbar/romanbar.pdf
%    TDS:doc/latex/romanbar/romanbar-example.tex
%    TDS:doc/latex/romanbar/romanbar-example.pdf
%    TDS:source/latex/romanbar/romanbar.dtx
%
%<*ignore>
\begingroup
  \catcode123=1 %
  \catcode125=2 %
  \def\x{LaTeX2e}%
\expandafter\endgroup
\ifcase 0\ifx\install y1\fi\expandafter
         \ifx\csname processbatchFile\endcsname\relax\else1\fi
         \ifx\fmtname\x\else 1\fi\relax
\else\csname fi\endcsname
%</ignore>
%<*install>
\input docstrip.tex
\Msg{********************************************************************}
\Msg{* Installation                                                     *}
\Msg{* Package: romanbar 2012/01/01 v1.0f Roman numbers with bars (HMM) *}
\Msg{********************************************************************}

\keepsilent
\askforoverwritefalse

\let\MetaPrefix\relax
\preamble

This is a generated file.

Project: romanbar
Version: 2012/01/01 v1.0f

Copyright (C) 2011 by
    H.-Martin M"unch <Martin dot Muench at Uni-Bonn dot de>

The usual disclaimer applies:
If it doesn't work right that's your problem.
(Nevertheless, send an e-mail to the maintainer
 when you find an error in this package.)

This work may be distributed and/or modified under the
conditions of the LaTeX Project Public License, either
version 1.3c of this license or (at your option) any later
version. This version of this license is in
   http://www.latex-project.org/lppl/lppl-1-3c.txt
and the latest version of this license is in
   http://www.latex-project.org/lppl.txt
and version 1.3c or later is part of all distributions of
LaTeX version 2005/12/01 or later.

This work has the LPPL maintenance status "maintained".

The Current Maintainer of this work is H.-Martin Muench.

This work consists of the main source file romanbar.dtx,
the README, and the derived files
   romanbar.sty, romanbar.pdf,
   romanbar.ins, romanbar.drv,
   romanbar-example.tex, romanbar-example.pdf.

\endpreamble
\let\MetaPrefix\DoubleperCent

\generate{%
  \file{romanbar.ins}{\from{romanbar.dtx}{install}}%
  \file{romanbar.drv}{\from{romanbar.dtx}{driver}}%
  \usedir{tex/latex/romanbar}%
  \file{romanbar.sty}{\from{romanbar.dtx}{package}}%
  \usedir{doc/latex/romanbar}%
  \file{romanbar-example.tex}{\from{romanbar.dtx}{example}}%
}

\catcode32=13\relax% active space
\let =\space%
\Msg{************************************************************************}
\Msg{*}
\Msg{* To finish the installation you have to move the following}
\Msg{* file into a directory searched by TeX:}
\Msg{*}
\Msg{*     romanbar.sty}
\Msg{*}
\Msg{* To produce the documentation run the file `romanbar.drv'}
\Msg{* through (pdf)LaTeX, e.g.}
\Msg{*  pdflatex romanbar.drv}
\Msg{*  makeindex -s gind.ist romanbar.idx}
\Msg{*  pdflatex romanbar.drv}
\Msg{*  makeindex -s gind.ist romanbar.idx}
\Msg{*  pdflatex romanbar.drv}
\Msg{*}
\Msg{* At least three runs are necessary e.g. to get the}
\Msg{*  references right!}
\Msg{*}
\Msg{* Happy TeXing!}
\Msg{*}
\Msg{************************************************************************}

\endbatchfile
%</install>
%<*ignore>
\fi
%</ignore>
%
% \section{The documentation driver file}
%
% The next bit of code contains the documentation driver file for
% \TeX{}, i.\,e., the file that will produce the documentation you
% are currently reading. It will be extracted from this file by the
% \texttt{docstrip} programme. That is, run \LaTeX{} on \texttt{docstrip}
% and specify the \texttt{driver} option when \texttt{docstrip}
% asks for options.
%
%    \begin{macrocode}
%<*driver>
\NeedsTeXFormat{LaTeX2e}[2011/06/27]
\ProvidesFile{romanbar.drv}%
  [2012/01/01 v1.0f Roman numbers with bars (HMM)]
\documentclass{ltxdoc}[2007/11/11]% v2.0u
\usepackage{holtxdoc}[2011/02/04]%  v0.21
%% romanbar may work with earlier versions of LaTeX2e and those
%% class and package, but this was not tested.
%% Please consider updating your LaTeX, class, and package
%% to the most recent version (if they are not already the most
%% recent version).
\hypersetup{%
 pdfsubject={Writing Roman numbers with bars (HMM)},%
 pdfkeywords={LaTeX, romanbar, roman, Roman, bars, H.-Martin Muench},%
 pdfencoding=auto,%
 pdflang={en},%
 breaklinks=true,%
 linktoc=all,%
 pdfstartview=FitH,%
 pdfpagelayout=OneColumn,%
 bookmarksnumbered=true,%
 bookmarksopen=true,%
 bookmarksopenlevel=2,%
 pdfmenubar=true,%
 pdftoolbar=true,%
 pdfwindowui=true,%
 pdfnewwindow=true%
}
\CodelineIndex
\hyphenation{docu-ment ana-lysed}
\gdef\unit#1{\mathord{\thinspace\mathrm{#1}}}%
\makeatletter
\@ifundefined{eTeX}{\gdef\eTeX{$\m@th \varepsilon $-\TeX }}{% else \relax
}
\makeatother
\begin{document}
  \DocInput{romanbar.dtx}%
\end{document}
%</driver>
%    \end{macrocode}
%
% \fi
%
% \CheckSum{179}
%
% \CharacterTable
%  {Upper-case    \A\B\C\D\E\F\G\H\I\J\K\L\M\N\O\P\Q\R\S\T\U\V\W\X\Y\Z
%   Lower-case    \a\b\c\d\e\f\g\h\i\j\k\l\m\n\o\p\q\r\s\t\u\v\w\x\y\z
%   Digits        \0\1\2\3\4\5\6\7\8\9
%   Exclamation   \!     Double quote  \"     Hash (number) \#
%   Dollar        \$     Percent       \%     Ampersand     \&
%   Acute accent  \'     Left paren    \(     Right paren   \)
%   Asterisk      \*     Plus          \+     Comma         \,
%   Minus         \-     Point         \.     Solidus       \/
%   Colon         \:     Semicolon     \;     Less than     \<
%   Equals        \=     Greater than  \>     Question mark \?
%   Commercial at \@     Left bracket  \[     Backslash     \\
%   Right bracket \]     Circumflex    \^     Underscore    \_
%   Grave accent  \`     Left brace    \{     Vertical bar  \|
%   Right brace   \}     Tilde         \~}
%
% \GetFileInfo{romanbar.drv}
%
% \begingroup
%   \def\x{\#,\$,\^,\_,\~,\ ,\&,\{,\},\%}%
%   \makeatletter
%   \@onelevel@sanitize\x
% \expandafter\endgroup
% \expandafter\DoNotIndex\expandafter{\x}
% \expandafter\DoNotIndex\expandafter{\string\ }
% \begingroup
%   \makeatletter
%     \lccode`9=32\relax
%     \lowercase{%^^A
%       \edef\x{\noexpand\DoNotIndex{\@backslashchar9}}%^^A
%     }%^^A
%   \expandafter\endgroup\x
%
% \DoNotIndex{\\,\,}
% \DoNotIndex{\documentclass,\usepackage,\ProvidesPackage,\begin,\end}
% \DoNotIndex{\MessageBreak}
% \DoNotIndex{\NeedsTeXFormat,\DoNotIndex,\verb}
% \DoNotIndex{\def,\edef,\gdef,\global}
% \DoNotIndex{\ifx,\listfiles,\mathord,\mathrm}
% \DoNotIndex{\bigskip,\space,\thinspace,\ldots}
% \DoNotIndex{\indent,\noindent,\newline,\linebreak,\pagebreak,\newpage}
% \DoNotIndex{\textbf,\textit,\textsf,\texttt,\textquotedblleft,\textquotedblright}
% \DoNotIndex{\plainTeX,\TeX,\LaTeX,\pdfLaTeX}
% \DoNotIndex{\item}
%
% \title{The \xpackage{romanbar} package}
% \date{2012/01/01 v1.0f}
% \author{H.-Martin M\"{u}nch\\\xemail{Martin.Muench at Uni-Bonn.de}}
%
% \maketitle
%
% \begin{abstract}
%  This \LaTeX{} package allows to write Roman numbers
%  (or any other text) with bars. (Additionally, commands for converting Arabic numbers
%  into Roman ones are provided and an |\ifnumeric| test function.)
% \end{abstract}
%
% \bigskip
%
% \noindent Disclaimer for web links: The author is not responsible for any contents
% referred to in this work unless he has full knowledge of illegal contents.
% If any damage occurs by the use of information presented there, only the
% author of the respective pages might be liable, not the one who has referred
% to these pages.
%
% \bigskip
%
% \noindent {\color{green} Save per page about $200\unit{ml}$ water,
% $2\unit{g}$ CO$_{2}$ and $2\unit{g}$ wood:\\
% Therefore please print only if this is really necessary.}
%
% \pagebreak
%
% \tableofcontents
%
% \pagebreak
%
% \section{Introduction}
%
% \indent This \LaTeX{} package allows to write Roman numbers
% (or any other text) with bars. (Additionally, command for converting Arabic numbers
% into Roman ones are provided and an |\ifnumeric| test function.)
%
% \bigskip
%
% \section{Usage}
%
% \indent Just load the package placing
% \begin{quote}
%   |\usepackage{romanbar}|
% \end{quote}
% \noindent in the preamble of your \LaTeXe{} source file.\\
% |\Romanbar{...}| then produces a Roman number with bars (please see the example file).
% |\ifnumeric{test}{true}{false}| tests for |test| being numeric.
% |\romannum{...}| turns an Arabic number into a lowercase Roman one, and
% |\Romannum{...}| turns an Arabic number into an uppercase Roman one.
%
% \subsection*{Options}
% \DescribeMacro{options}
% \indent The \xpackage{romanbar} package takes no options.
%
% \section{Alternatives\label{sec:Alternatives}}
%
% \begin{description}
% \item[-] the original code written by Prof. Enrico Gregorio\\
%   (\url{http://profs.sci.univr.it/~gregorio/}),\\
%   \url{http://tex.stackexchange.com/questions/24065/roman-numerals-formatting/24084#24084}
%\begin{verbatim}
%\def\barroman#1{\sbox0{#1}\dimen0=\dimexpr\wd0+1pt\relax
%  \makebox[\dimen0]{\rlap{\vrule width\dimen0 height 0.06ex depth 0.06ex}%
%    \rlap{\vrule width\dimen0 height\dimexpr\ht0+0.03ex\relax
%            depth\dimexpr-\ht0+0.09ex\relax}%
%    \kern.5pt#1\kern.5pt}}
%
%\barroman{I} \barroman{XI}
%\end{verbatim}
%
% \item[-]  Some fonts provide single characters for Roman numerals,
%   cf.~e.\,g.
% \url{http://tex.stackexchange.com/questions/38695/using-unicode-roman-numerals-in-xetex}.
% \end{description}
%
% \noindent (You programmed or found another alternative,
%  which is available at \CTAN{}?\\
%  OK, send an e-mail to me with the name, location at \CTAN{},
%  and a short notice, and I will probably include it in
%  the list above.)
%
% \newpage
%
% \section{Example}
%
%    \begin{macrocode}
%<*example>
\documentclass[british]{article}[2007/10/19]% v1.4h
%%%%%%%%%%%%%%%%%%%%%%%%%%%%%%%%%%%%%%%%%%%%%%%%%%%%%%%%%%%%%%%%%%%%%
\usepackage[extension=pdf,%
 plainpages=false,%
 pdfpagelabels=true,%
 hyperindex=false,%
 pdflang={en},%
 pdftitle={romanbar package example},%
 pdfauthor={H.-Martin Muench},%
 pdfsubject={Example for the romanbar package},%
 pdfkeywords={LaTeX, romanbar, roman, Roman, bars, H.-Martin Muench},%
 pdfview={XYZ null null 1},%
 pdfstartview={XYZ null null 1},%
 pdfpagelayout=SinglePage]{hyperref}[2011/12/04]% v6.82m
\usepackage{romanbar}[2012/01/01]% v1.0f
\gdef\unit#1{\mathord{\thinspace\mathrm{#1}}}%
\listfiles
\begin{document}
\pagenumbering{arabic}
\section*{Example for romanbar}

This example demonstrates the use of package\newline
\textsf{romanbar}, v1.0f as of 2012/01/01 (HMM).\newline
There are no options to be used.\newline

\noindent For more details please see the documentation!\newline

\noindent Save per page about $200\unit{ml}$ water,
$2\unit{g}$ CO$_{2}$ and $2\unit{g}$ wood:\newline
Therefore please print only if this is really necessary.\newline

\noindent This package provides the command \verb|\Romanbar|
to print bars below and over the following:

\begin{description}
\item[-] Roman numbers: \verb|\Romanbar{MMXII}| prints \Romanbar{MMXII}

\item[-] Arabic numbers turned into upper-case Roman numbers:\newline
           \verb|\Romanbar{2012}| prints \Romanbar{2012}

\item[-] negative Arabic numbers turned into upper-case Roman numbers
           \newline
           (although historically there were no negative Roman numbers):
           \newline
           \verb|\Romanbar{-12}| prints \Romanbar{-12}

\item[-] zero Arabic number ($0$; although historically
           there was no Roman zero):\newline
           \verb|\Romanbar{0}| prints \Romanbar{0}

\item[-] some arbitrary text:
           \verb|\Romanbar{Caesar}| prints \Romanbar{Caesar}\newline
           (with descenders: \Romanbar{AgjpqyW})


\newcounter{example}
\setcounter{example}{21}

\item[-] some counter's value:
           \verb|\Romanbar{\theexample}| prints \Romanbar{\theexample}
           \newline
           (where the value of \texttt{example} is \theexample)

\item[-] Arabic numbers, without turning them into upper-case Roman
           numbers:\newline
           \verb|\Romanbar{\relax 2012}| prints \Romanbar{\relax  2012}
\end{description}

Special care was taken for "L" (50), e.\,g. in 555/DLV: \Romanbar{555}.\\


\verb|\romannum{...}| converts an Arabic number into a lower-case Roman one,
e.\,g. \verb|\romannum{2012}| prints \romannum{2012}, and
\verb|\Romannum{...}| converts an Arabic number into an upper-case Roman one,
e.\,g. \verb|\Romannum{2012}| prints \Romannum{2012}.

\end{document}
%</example>
%    \end{macrocode}
%
% \StopEventually{}
%
% \newpage
%
% \section{The implementation}
%
% We start off by checking that we are loading into \LaTeXe{} and
% announcing the name and version of this package.
%
%    \begin{macrocode}
%<*package>
%    \end{macrocode}
%
%    \begin{macrocode}
\NeedsTeXFormat{LaTeX2e}[2011/06/27]
\ProvidesPackage{romanbar}[2012/01/01 v1.0f
              Roman numbers with bars (HMM)]

%    \end{macrocode}
%
% A short description of the \xpackage{romanbar} package:
%
%    \begin{macrocode}
%% Allows to write Roman numbers (or any other text) with bars
%% and to convert Arabic numbers into Roman ones.
%    \end{macrocode}
%
% A last information for the user:
%
%    \begin{macrocode}
%% romanbar may work with earlier versions of LaTeX,
%% but this was not tested. Please consider updating
%% your LaTeX to the most recent version
%% (if it is not already the most recent version).

%    \end{macrocode}
%
% \indent See subsection~\ref{ss:Downloads} about how to get it.\\
%
% There are no options to be processed.
%
% A command to gobble a possible minus sign is needed and
% |\gobbleminus| from \url{http://www.tex.ac.uk/cgi-bin/texfaq2html?label=isitanum}
% is used.
%
%    \begin{macrocode}
\def\gobbleminus#1{\ifx-#1\else#1\fi}
%% from http://www.tex.ac.uk/cgi-bin/texfaq2html?label=isitanum

%    \end{macrocode}
%
% We want to test whether the argument passed to |\Romanbar| is numeric
% and therefore define |\ifnumeric{test}{true}{false}|,
% where |test| is to be analysed, |true| is the code to be executed,
% if |test| is numeric, and |false| is the code to be executed,
% if |test| is not numeric. This is done similar to\\
% \url{http://tex.stackexchange.com/a/17119}.
%
%    \begin{macrocode}
\newcommand{\ifnumeric}[3]{%
%% similar to http://tex.stackexchange.com/a/17119
\sbox\z@{\@tempcnta=0\gobbleminus#1\relax}%
\ifdim\wd0>\z@\relax#3% is not numeric
\else#2% is numeric
\fi%
}

%    \end{macrocode}
%
% We define |\r@iseL| to raise any L (50), otherwise a good deal
% of the horizontal part of the letter would be swallowed by the lower bar
% under the \textquotedblleft{}number\textquotedblright{}.
%
%    \begin{macrocode}
\def\r@iseL#1{\ifx @#1% then terminate
\else%
\if L#1\raisebox{0.05ex}{L}\else #1\fi%
\expandafter\r@iseL%
\fi%
}

%    \end{macrocode}
%
% \pagebreak
%
% We define the |\Romanbar| command.
%
%    \begin{macrocode}
\DeclareRobustCommand{\Romanbar}[1]{%
\ifnumeric{#1}{% is numeric
%    \end{macrocode}
%
% If its parameter is numeric, the temporary counter |\@tempcnta| is set to the value.
%
%    \begin{macrocode}
\@tempcnta=#1\relax%
%    \end{macrocode}
%
% Normally there are no non-positive Roman numbers, therefore we do some trick
% similar to |\XRoman| from the \url{http://ctan.org/pkg/pageslts} package.
%
%    \begin{macrocode}
%% similar to \XRoman from the http://ctan.org/pkg/pageslts package
\ifnum\@tempcnta<1%
  \ifnum #1>0%
    \relax \Roman@bar{\Roman{#1}}%
  \else%
    \ifnum #1<0%
     -\Roman@bar{\expandafter\@slowromancap\romannumeral\number-#1@%
     }%
    \else%
      \Roman@bar{0}%
    \fi%
  \fi%
\else\Roman@bar{\@Roman\@tempcnta}%
\fi%
}{% is not numeric
\Roman@bar{#1}%
}%
}

%    \end{macrocode}
%
% Whether it is numeric or not, in the end |\Roman@bar| is applied to it.
%
%    \begin{macrocode}
\DeclareRobustCommand{\Roman@bar}[1]{% #1 is in Roman, i.e. MMXII
%% (or any other text, "Caesar" would work, too).
%% similar to code of  by Prof. Enrico Gregorio (egreg) at
%% http://tex.stackexchange.com/questions/24065/roman-numerals-formatting/24084#24084
\@bsphack%
\edef\romanbartmp{#1}%
%% height:
\sbox0{\textsf{\romanbartmp}}%
%% top line:
%% + top of top line:
\@tempdima=\ht0%
\advance\@tempdima+0.05ex%0.03
\dimen1=\@tempdima%
%% + bottom of top line:
\@tempdima=-\ht0%
\advance\@tempdima+0.05ex%0.07
\dimen2=\@tempdima%
%% width:
\sbox0{\textsf{\expandafter\r@iseL\romanbartmp @}}%
\@tempdima=\wd0%
\advance\@tempdima+1pt%
\dimen0=\@tempdima%
\@esphack%
\makebox[\dimen0]{%
\rlap{\vrule width\dimen0 height\dimen1 depth\dimen2}% top    line
\rlap{\vrule width\dimen0 height 0.06ex depth 0.03ex}% bottom line
\kern0.5pt\textsf{\expandafter\r@iseL\romanbartmp @}\kern0.5pt}%
}

%    \end{macrocode}
%
% While it is not needed for the bars, it is sometimes asked for a command
% to convert an Arabic number into a Roman one. |\roman{...}| can only be applied
% to a counter, i.\,e. |\roman{12}| does not work, but
%\begin{verbatim}
%\newcounter{examplecounter}
%\setcounter{examplecounter}{12}
%\roman{examplecounter}
%\end{verbatim}
% would be necessary. Further non-negative values would not work.
% |\romannum| accepts any number as argument. (If the argument is no number,
% an error message is given and the argument printed (executed)
%  without applying anything to it.)
%
%    \begin{macrocode}
\DeclareRobustCommand{\romannum}[1]{%
\ifnumeric{#1}{% is numeric
\@tempcnta=#1\relax%
%% similar to \xroman from the http://ctan.org/pkg/pageslts package
\ifnum\@tempcnta<1%
  \ifnum #1>0%
    \relax \roman{#1}%
  \else%
    \ifnum #1<0%
     -\romannumeral\number-#1@%
    \else%
      0%
    \fi%
  \fi%
\else\@roman\@tempcnta%
\fi%
}{% is not numeric
\PackageError{romanbar}{%
Argument of \string\romannum\space is not a number}{%
The command \string\romannum\space converts an Arabic number into a%
 lower-case Roman one,\MessageBreak%
but the used argument of \string\romannum\space is not an Arabic number%
 but\MessageBreak%
`#1`,\MessageBreak%
which will now be printed unchanged.}%
#1%
}%
}

%    \end{macrocode}
%
% The same for upper-case Roman numbers, |\Romannum| instead of |\Roman|:
%
%    \begin{macrocode}
\DeclareRobustCommand{\Romannum}[1]{%
\ifnumeric{#1}{% is numeric
\@tempcnta=#1\relax%
%% similar to \XRoman from the http://ctan.org/pkg/pageslts package
\ifnum\@tempcnta<1%
  \ifnum #1>0%
    \relax \Roman{#1}%
  \else%
    \ifnum #1<0%
     -\expandafter\@slowromancap\romannumeral\number-#1@%
    \else%
      0%
    \fi%
  \fi%
\else\@Roman\@tempcnta%
\fi%
}{% is not numeric
\PackageError{romanbar}{%
Argument of \string\Romannum\space is not a number}{%
The command \string\Romannum\space converts an Arabic number into an%
 upper-case Roman one,\MessageBreak%
but the used argument of \string\Romannum\space is not an Arabic number%
 but\MessageBreak%
`#1`.}%
#1%
}%
}

%    \end{macrocode}
%
%    \begin{macrocode}
%</package>
%    \end{macrocode}
%
% \newpage
%
% \section{Installation}
%
% \subsection{Downloads\label{ss:Downloads}}
%
% Everything should be available on \CTAN{}, \url{http://www.ctan.org/tex-archive/},
% but may need additional packages themselves.\\
%
% \DescribeMacro{romanbar.dtx}
% For unpacking the |romanbar.dtx| file and constructing the documentation
% it is required:
% \begin{description}
% \item[-] \TeX{} Format \LaTeXe{}, \url{http://www.CTAN.org/}
%
% \item[-] document class \xpackage{ltxdoc}, 2007/11/11, v2.0u,
%           \url{http://ctan.org/pkg/ltxdoc}
%
% \item[-] package \xpackage{holtxdoc}, 2011/02/04, v0.21,
%           \url{http://ctan.org/pkg/holtxdoc}
%
% \item[-] package \xpackage{hypdoc}, 2010/03/26, v1.9,
%           \url{http://ctan.org/pkg/hypdoc}
% \end{description}
%
% \DescribeMacro{romanbar.sty}
% The \texttt{romanbar.sty} for \LaTeXe{} (i.\,e.~each document using
% the \xpackage{romanbar} package) requires:
% \begin{description}
% \item[-] \TeX Format \LaTeXe{}, \url{http://www.CTAN.org/}
% \end{description}
%
% \DescribeMacro{romanbar-example.tex}
% The \texttt{romanbar-example.tex} requires the same files as all
% documents using the \xpackage{romanbar} package and additionally:
% \begin{description}
% \item[-] class \xpackage{article}, 2007/10/19, v1.4h, from \xpackage{classes.dtx}:\\
%           \CTAN{macros/latex/base/classes.dtx}
%
% \item[-] package \xpackage{romanbar}, 2012/01/01, v1.0f,\\
%           %%\url{http://ctan.org/pkg/romanbar}\\
%           \CTAN{macros/latex/contrib/romanbar/}
%   (Well, it is the example file for this package, and because you are reading the
%    documentation for the \xpackage{romanbar} package, it can be assumed that you already
%    have some version of it -- is it the current one?)
% \end{description}
%
% \DescribeMacro{Alternative}
% As possible alternative package in section \ref{sec:Alternatives} there is listed
% \begin{description}
% \item[-] none.
% \end{description}
%
% \DescribeMacro{Oberdiek}
% \DescribeMacro{holtxdoc}
% All packages of \textsc{Heiko Oberdiek's} bundle `oberdiek'
% (especially \xpackage{holtxdoc}) are also available in a TDS compliant ZIP archive:\\
% \CTAN{install/macros/latex/contrib/oberdiek.tds.zip}.\\
% It is probably best to download and use this, because the packages in there
% are quite probably both recent and compatible among themselves.\\
%
% \DescribeMacro{hyperref}
% \noindent \xpackage{hyperref} is not included in that bundle and needs to be
% downloaded separately,\\
% \url{http://mirror.ctan.org/install/macros/latex/contrib/hyperref.tds.zip}.\\
%
% \DescribeMacro{M\"{u}nch}
% A hyperlinked list of my (other) packages can be found at
% \url{http://www.Uni-Bonn.de/~uzs5pv/LaTeX.html}.\\
%
% \pagebreak
%
% \subsection{Package, unpacking TDS}
% \paragraph{Package.} This package should become available on \CTAN{} soon:
% \begin{description}
% \item[\CTAN{macros/latex/contrib/romanbar/romanbar.dtx}]\hspace*{0.1cm} \\
%       The source file.
% \item[\CTAN{macros/latex/contrib/romanbar/romanbar.pdf}]\hspace*{0.1cm} \\
%       The documentation.
% \item[\CTAN{macros/latex/contrib/romanbar/romanbar-example.pdf}]\hspace*{0.1cm} \\
%       The compiled example file, as it should look like.
% \item[\CTAN{macros/latex/contrib/romanbar/README}]\hspace*{0.1cm} \\
%       The README file.
% \end{description}
% There is also a romanbar.tds.zip available:
% \begin{description}
% \item[\CTAN{install/macros/latex/contrib/romanbar.tds.zip}]\hspace*{0.1cm} \\
%       Everything in \xfile{TDS} compliant, compiled format.
% \end{description}
% which additionally contains\\
% \begin{tabular}{ll}
% romanbar.ins & The installation file.\\
% romanbar.drv & The driver to generate the documentation.\\
% romanbar.sty & The \xext{sty}le file.\\
% romanbar-example.tex & The example file.
% \end{tabular}
%
% \bigskip
%
% \noindent For required other packages, please see the preceding subsection.
%
% \paragraph{Unpacking.} The \xfile{.dtx} file is a self-extracting
% \docstrip{} archive. The files are extracted by running the
% \xext{.dtx} through \plainTeX{}:
% \begin{quote}
%   \verb|tex romanbar.dtx|
% \end{quote}
%
% About generating the documentation see paragraph~\ref{GenDoc} below.\\
%
% \paragraph{TDS.} Now the different files must be moved into
% the different directories in your installation TDS tree
% (also known as \xfile{texmf} tree):
% \begin{quote}
% \def\t{^^A
% \begin{tabular}{@{}>{\ttfamily}l@{ $\rightarrow$ }>{\ttfamily}l@{}}
%   romanbar.sty & tex/latex/romanbar/romanbar.sty\\
%   romanbar.pdf & doc/latex/romanbar/romanbar.pdf\\
%   romanbar-example.tex & doc/latex/romanbar/romanbar-example.tex\\
%   romanbar-example.pdf & doc/latex/romanbar/romanbar-example.pdf\\
%   romanbar.dtx & source/latex/romanbar/romanbar.dtx\\
% \end{tabular}^^A
% }^^A
% \sbox0{\t}^^A
% \ifdim\wd0>\linewidth
%   \begingroup
%     \advance\linewidth by\leftmargin
%     \advance\linewidth by\rightmargin
%   \edef\x{\endgroup
%     \def\noexpand\lw{\the\linewidth}^^A
%   }\x
%   \def\lwbox{^^A
%     \leavevmode
%     \hbox to \linewidth{^^A
%       \kern-\leftmargin\relax
%       \hss
%       \usebox0
%       \hss
%       \kern-\rightmargin\relax
%     }^^A
%   }^^A
%   \ifdim\wd0>\lw
%     \sbox0{\small\t}^^A
%     \ifdim\wd0>\linewidth
%       \ifdim\wd0>\lw
%         \sbox0{\footnotesize\t}^^A
%         \ifdim\wd0>\linewidth
%           \ifdim\wd0>\lw
%             \sbox0{\scriptsize\t}^^A
%             \ifdim\wd0>\linewidth
%               \ifdim\wd0>\lw
%                 \sbox0{\tiny\t}^^A
%                 \ifdim\wd0>\linewidth
%                   \lwbox
%                 \else
%                   \usebox0
%                 \fi
%               \else
%                 \lwbox
%               \fi
%             \else
%               \usebox0
%             \fi
%           \else
%             \lwbox
%           \fi
%         \else
%           \usebox0
%         \fi
%       \else
%         \lwbox
%       \fi
%     \else
%       \usebox0
%     \fi
%   \else
%     \lwbox
%   \fi
% \else
%   \usebox0
% \fi
% \end{quote}
% If you have a \xfile{docstrip.cfg} that configures and enables \docstrip{}'s
% \xfile{TDS} installing feature, then some files can already be in the right
% place, see the documentation of \docstrip{}.
%
% \subsection{Refresh file name databases}
%
% If your \TeX{}~distribution (\teTeX{}, \mikTeX{},\dots{}) relies on file name
% databases, you must refresh these. For example, \teTeX{} users run
% \verb|texhash| or \verb|mktexlsr|.
%
% \subsection{Some details for the interested}
%
% \paragraph{Unpacking with \LaTeX{}.}
% The \xfile{.dtx} chooses its action depending on the format:
% \begin{description}
% \item[\plainTeX:] Run \docstrip{} and extract the files.
% \item[\LaTeX:] Generate the documentation.
% \end{description}
% If you insist on using \LaTeX{} for \docstrip{} (really,
% \docstrip{} does not need \LaTeX{}), then inform the autodetect routine
% about your intention:
% \begin{quote}
%   \verb|latex \let\install=y\input{romanbar.dtx}|
% \end{quote}
% Do not forget to quote the argument according to the demands
% of your shell.
%
% \paragraph{Generating the documentation.\label{GenDoc}}
% You can use both the \xfile{.dtx} or the \xfile{.drv} to generate
% the documentation. The process can be configured by a
% configuration file \xfile{ltxdoc.cfg}. For instance, put the following
% line into this file, if you want to have A4 as paper format:
% \begin{quote}
%   \verb|\PassOptionsToClass{a4paper}{article}|
% \end{quote}
%
% \noindent An example follows how to generate the
% documentation with \pdfLaTeX{}:
%
% \begin{quote}
%\begin{verbatim}
%pdflatex romanbar.dtx
%makeindex -s gind.ist romanbar.idx
%pdflatex romanbar.dtx
%makeindex -s gind.ist romanbar.idx
%pdflatex romanbar.dtx
%\end{verbatim}
% \end{quote}
%
% \subsection{Compiling the example}
%
% The example file, \textsf{romanbar-example.tex}, can be compiled via\\
% |(pdf)(la)tex romanbar-example.tex|.
%
% \section{Acknowledgements}
%
% I would like to thank \textsc{Heiko Oberdiek} for providing
% the \xpackage{hyperref} as well as a~lot~(!) of other useful packages
% (from which I also got everything I know about creating a file in \xext{dtx}
% format, ok, say it: copying),
% Prof.~\textsc{Enrico Gregorio} (egreg) for his answer
% \url{http://tex.stackexchange.com/questions/24065/roman-numerals-formatting/24084#24084},
% the \Newsgroup{comp.text.tex} and \Newsgroup{de.comp.text.tex}
% newsgroups as well as \url{http://tex.stackexchange.com}
% for their help in all things \TeX{}.
%
% \pagebreak
%
% \phantomsection
% \begin{History}\label{History}
%   \begin{Version}{2011/07/25 v1.0a}
%     \item First code by Prof.~\textsc{Enrico Gregorio} (egreg) at
%             \url{http://tex.stackexchange.com/questions/24065/roman-numerals-formatting/24084#24084}
%   \end{Version}
%   \begin{Version}{2011/07/26 v1.0b}
%     \item Packed into a \xext{sty} file.
%     \item Removed the requirement of \eTeX{} for the package.
%   \end{Version}
%   \begin{Version}{2011/08/25 v1.0c}
%     \item Renamed \xpackage{romanbar} package.
%     \item Minor details.
%   \end{Version}
%   \begin{Version}{2011/09/16 v1.0d}
%     \item Made commands robust.
%     \item Minor details.
%   \end{Version}
%   \begin{Version}{2011/12/16 v1.0e}
%     \item Additionally accepts now Arabic numbers and converts them to Roman ones with bars.
%     \item Also non-positive numbers are now accepted.
%   \end{Version}
%   \begin{Version}{2012/01/01 v1.0f}
%     \item Added |\romannum| and |\Romannum| to convert Arabic to Roman numbers (without bars).
%     \item |\r@iseL|.
%     \item Created a \xfile{dtx}, example, and README.
%     \item Upload to \CTAN{}.
%   \end{Version}
% \end{History}
%
% \bigskip
%
% When you find a mistake or have a suggestion for an improvement of this package,
% please send an e-mail to the maintainer, thanks! (Please see BUG REPORTS in the README.)
%
% \pagebreak
%
% \PrintIndex
%
% \Finale
\endinput|
% \end{quote}
% Do not forget to quote the argument according to the demands
% of your shell.
%
% \paragraph{Generating the documentation.\label{GenDoc}}
% You can use both the \xfile{.dtx} or the \xfile{.drv} to generate
% the documentation. The process can be configured by a
% configuration file \xfile{ltxdoc.cfg}. For instance, put the following
% line into this file, if you want to have A4 as paper format:
% \begin{quote}
%   \verb|\PassOptionsToClass{a4paper}{article}|
% \end{quote}
%
% \noindent An example follows how to generate the
% documentation with \pdfLaTeX{}:
%
% \begin{quote}
%\begin{verbatim}
%pdflatex romanbar.dtx
%makeindex -s gind.ist romanbar.idx
%pdflatex romanbar.dtx
%makeindex -s gind.ist romanbar.idx
%pdflatex romanbar.dtx
%\end{verbatim}
% \end{quote}
%
% \subsection{Compiling the example}
%
% The example file, \textsf{romanbar-example.tex}, can be compiled via\\
% |(pdf)(la)tex romanbar-example.tex|.
%
% \section{Acknowledgements}
%
% I would like to thank \textsc{Heiko Oberdiek} for providing
% the \xpackage{hyperref} as well as a~lot~(!) of other useful packages
% (from which I also got everything I know about creating a file in \xext{dtx}
% format, ok, say it: copying),
% Prof.~\textsc{Enrico Gregorio} (egreg) for his answer
% \url{http://tex.stackexchange.com/questions/24065/roman-numerals-formatting/24084#24084},
% the \Newsgroup{comp.text.tex} and \Newsgroup{de.comp.text.tex}
% newsgroups as well as \url{http://tex.stackexchange.com}
% for their help in all things \TeX{}.
%
% \pagebreak
%
% \phantomsection
% \begin{History}\label{History}
%   \begin{Version}{2011/07/25 v1.0a}
%     \item First code by Prof.~\textsc{Enrico Gregorio} (egreg) at
%             \url{http://tex.stackexchange.com/questions/24065/roman-numerals-formatting/24084#24084}
%   \end{Version}
%   \begin{Version}{2011/07/26 v1.0b}
%     \item Packed into a \xext{sty} file.
%     \item Removed the requirement of \eTeX{} for the package.
%   \end{Version}
%   \begin{Version}{2011/08/25 v1.0c}
%     \item Renamed \xpackage{romanbar} package.
%     \item Minor details.
%   \end{Version}
%   \begin{Version}{2011/09/16 v1.0d}
%     \item Made commands robust.
%     \item Minor details.
%   \end{Version}
%   \begin{Version}{2011/12/16 v1.0e}
%     \item Additionally accepts now Arabic numbers and converts them to Roman ones with bars.
%     \item Also non-positive numbers are now accepted.
%   \end{Version}
%   \begin{Version}{2012/01/01 v1.0f}
%     \item Added |\romannum| and |\Romannum| to convert Arabic to Roman numbers (without bars).
%     \item |\r@iseL|.
%     \item Created a \xfile{dtx}, example, and README.
%     \item Upload to \CTAN{}.
%   \end{Version}
% \end{History}
%
% \bigskip
%
% When you find a mistake or have a suggestion for an improvement of this package,
% please send an e-mail to the maintainer, thanks! (Please see BUG REPORTS in the README.)
%
% \pagebreak
%
% \PrintIndex
%
% \Finale
\endinput
%        (quote the arguments according to the demands of your shell)
%
% Documentation:
%    (a) If romanbar.drv is present:
%           (pdf)latex romanbar.drv
%           makeindex -s gind.ist romanbar.idx
%           (pdf)latex romanbar.drv
%           makeindex -s gind.ist romanbar.idx
%           (pdf)latex romanbar.drv
%    (b) Without romanbar.drv:
%           (pdf)latex romanbar.dtx
%           makeindex -s gind.ist romanbar.idx
%           (pdf)latex romanbar.dtx
%           makeindex -s gind.ist romanbar.idx
%           (pdf)latex romanbar.dtx
%
%    The class ltxdoc loads the configuration file ltxdoc.cfg
%    if available. Here you can specify further options, e.g.
%    use DIN A4 as paper format:
%       \PassOptionsToClass{a4paper}{article}
%
% Installation:
%    TDS:tex/latex/romanbar/romanbar.sty
%    TDS:doc/latex/romanbar/romanbar.pdf
%    TDS:doc/latex/romanbar/romanbar-example.tex
%    TDS:doc/latex/romanbar/romanbar-example.pdf
%    TDS:source/latex/romanbar/romanbar.dtx
%
%<*ignore>
\begingroup
  \catcode123=1 %
  \catcode125=2 %
  \def\x{LaTeX2e}%
\expandafter\endgroup
\ifcase 0\ifx\install y1\fi\expandafter
         \ifx\csname processbatchFile\endcsname\relax\else1\fi
         \ifx\fmtname\x\else 1\fi\relax
\else\csname fi\endcsname
%</ignore>
%<*install>
\input docstrip.tex
\Msg{********************************************************************}
\Msg{* Installation                                                     *}
\Msg{* Package: romanbar 2012/01/01 v1.0f Roman numbers with bars (HMM) *}
\Msg{********************************************************************}

\keepsilent
\askforoverwritefalse

\let\MetaPrefix\relax
\preamble

This is a generated file.

Project: romanbar
Version: 2012/01/01 v1.0f

Copyright (C) 2011 by
    H.-Martin M"unch <Martin dot Muench at Uni-Bonn dot de>

The usual disclaimer applies:
If it doesn't work right that's your problem.
(Nevertheless, send an e-mail to the maintainer
 when you find an error in this package.)

This work may be distributed and/or modified under the
conditions of the LaTeX Project Public License, either
version 1.3c of this license or (at your option) any later
version. This version of this license is in
   http://www.latex-project.org/lppl/lppl-1-3c.txt
and the latest version of this license is in
   http://www.latex-project.org/lppl.txt
and version 1.3c or later is part of all distributions of
LaTeX version 2005/12/01 or later.

This work has the LPPL maintenance status "maintained".

The Current Maintainer of this work is H.-Martin Muench.

This work consists of the main source file romanbar.dtx,
the README, and the derived files
   romanbar.sty, romanbar.pdf,
   romanbar.ins, romanbar.drv,
   romanbar-example.tex, romanbar-example.pdf.

\endpreamble
\let\MetaPrefix\DoubleperCent

\generate{%
  \file{romanbar.ins}{\from{romanbar.dtx}{install}}%
  \file{romanbar.drv}{\from{romanbar.dtx}{driver}}%
  \usedir{tex/latex/romanbar}%
  \file{romanbar.sty}{\from{romanbar.dtx}{package}}%
  \usedir{doc/latex/romanbar}%
  \file{romanbar-example.tex}{\from{romanbar.dtx}{example}}%
}

\catcode32=13\relax% active space
\let =\space%
\Msg{************************************************************************}
\Msg{*}
\Msg{* To finish the installation you have to move the following}
\Msg{* file into a directory searched by TeX:}
\Msg{*}
\Msg{*     romanbar.sty}
\Msg{*}
\Msg{* To produce the documentation run the file `romanbar.drv'}
\Msg{* through (pdf)LaTeX, e.g.}
\Msg{*  pdflatex romanbar.drv}
\Msg{*  makeindex -s gind.ist romanbar.idx}
\Msg{*  pdflatex romanbar.drv}
\Msg{*  makeindex -s gind.ist romanbar.idx}
\Msg{*  pdflatex romanbar.drv}
\Msg{*}
\Msg{* At least three runs are necessary e.g. to get the}
\Msg{*  references right!}
\Msg{*}
\Msg{* Happy TeXing!}
\Msg{*}
\Msg{************************************************************************}

\endbatchfile
%</install>
%<*ignore>
\fi
%</ignore>
%
% \section{The documentation driver file}
%
% The next bit of code contains the documentation driver file for
% \TeX{}, i.\,e., the file that will produce the documentation you
% are currently reading. It will be extracted from this file by the
% \texttt{docstrip} programme. That is, run \LaTeX{} on \texttt{docstrip}
% and specify the \texttt{driver} option when \texttt{docstrip}
% asks for options.
%
%    \begin{macrocode}
%<*driver>
\NeedsTeXFormat{LaTeX2e}[2011/06/27]
\ProvidesFile{romanbar.drv}%
  [2012/01/01 v1.0f Roman numbers with bars (HMM)]
\documentclass{ltxdoc}[2007/11/11]% v2.0u
\usepackage{holtxdoc}[2011/02/04]%  v0.21
%% romanbar may work with earlier versions of LaTeX2e and those
%% class and package, but this was not tested.
%% Please consider updating your LaTeX, class, and package
%% to the most recent version (if they are not already the most
%% recent version).
\hypersetup{%
 pdfsubject={Writing Roman numbers with bars (HMM)},%
 pdfkeywords={LaTeX, romanbar, roman, Roman, bars, H.-Martin Muench},%
 pdfencoding=auto,%
 pdflang={en},%
 breaklinks=true,%
 linktoc=all,%
 pdfstartview=FitH,%
 pdfpagelayout=OneColumn,%
 bookmarksnumbered=true,%
 bookmarksopen=true,%
 bookmarksopenlevel=2,%
 pdfmenubar=true,%
 pdftoolbar=true,%
 pdfwindowui=true,%
 pdfnewwindow=true%
}
\CodelineIndex
\hyphenation{docu-ment ana-lysed}
\gdef\unit#1{\mathord{\thinspace\mathrm{#1}}}%
\makeatletter
\@ifundefined{eTeX}{\gdef\eTeX{$\m@th \varepsilon $-\TeX }}{% else \relax
}
\makeatother
\begin{document}
  \DocInput{romanbar.dtx}%
\end{document}
%</driver>
%    \end{macrocode}
%
% \fi
%
% \CheckSum{179}
%
% \CharacterTable
%  {Upper-case    \A\B\C\D\E\F\G\H\I\J\K\L\M\N\O\P\Q\R\S\T\U\V\W\X\Y\Z
%   Lower-case    \a\b\c\d\e\f\g\h\i\j\k\l\m\n\o\p\q\r\s\t\u\v\w\x\y\z
%   Digits        \0\1\2\3\4\5\6\7\8\9
%   Exclamation   \!     Double quote  \"     Hash (number) \#
%   Dollar        \$     Percent       \%     Ampersand     \&
%   Acute accent  \'     Left paren    \(     Right paren   \)
%   Asterisk      \*     Plus          \+     Comma         \,
%   Minus         \-     Point         \.     Solidus       \/
%   Colon         \:     Semicolon     \;     Less than     \<
%   Equals        \=     Greater than  \>     Question mark \?
%   Commercial at \@     Left bracket  \[     Backslash     \\
%   Right bracket \]     Circumflex    \^     Underscore    \_
%   Grave accent  \`     Left brace    \{     Vertical bar  \|
%   Right brace   \}     Tilde         \~}
%
% \GetFileInfo{romanbar.drv}
%
% \begingroup
%   \def\x{\#,\$,\^,\_,\~,\ ,\&,\{,\},\%}%
%   \makeatletter
%   \@onelevel@sanitize\x
% \expandafter\endgroup
% \expandafter\DoNotIndex\expandafter{\x}
% \expandafter\DoNotIndex\expandafter{\string\ }
% \begingroup
%   \makeatletter
%     \lccode`9=32\relax
%     \lowercase{%^^A
%       \edef\x{\noexpand\DoNotIndex{\@backslashchar9}}%^^A
%     }%^^A
%   \expandafter\endgroup\x
%
% \DoNotIndex{\\,\,}
% \DoNotIndex{\documentclass,\usepackage,\ProvidesPackage,\begin,\end}
% \DoNotIndex{\MessageBreak}
% \DoNotIndex{\NeedsTeXFormat,\DoNotIndex,\verb}
% \DoNotIndex{\def,\edef,\gdef,\global}
% \DoNotIndex{\ifx,\listfiles,\mathord,\mathrm}
% \DoNotIndex{\bigskip,\space,\thinspace,\ldots}
% \DoNotIndex{\indent,\noindent,\newline,\linebreak,\pagebreak,\newpage}
% \DoNotIndex{\textbf,\textit,\textsf,\texttt,\textquotedblleft,\textquotedblright}
% \DoNotIndex{\plainTeX,\TeX,\LaTeX,\pdfLaTeX}
% \DoNotIndex{\item}
%
% \title{The \xpackage{romanbar} package}
% \date{2012/01/01 v1.0f}
% \author{H.-Martin M\"{u}nch\\\xemail{Martin.Muench at Uni-Bonn.de}}
%
% \maketitle
%
% \begin{abstract}
%  This \LaTeX{} package allows to write Roman numbers
%  (or any other text) with bars. (Additionally, commands for converting Arabic numbers
%  into Roman ones are provided and an |\ifnumeric| test function.)
% \end{abstract}
%
% \bigskip
%
% \noindent Disclaimer for web links: The author is not responsible for any contents
% referred to in this work unless he has full knowledge of illegal contents.
% If any damage occurs by the use of information presented there, only the
% author of the respective pages might be liable, not the one who has referred
% to these pages.
%
% \bigskip
%
% \noindent {\color{green} Save per page about $200\unit{ml}$ water,
% $2\unit{g}$ CO$_{2}$ and $2\unit{g}$ wood:\\
% Therefore please print only if this is really necessary.}
%
% \pagebreak
%
% \tableofcontents
%
% \pagebreak
%
% \section{Introduction}
%
% \indent This \LaTeX{} package allows to write Roman numbers
% (or any other text) with bars. (Additionally, command for converting Arabic numbers
% into Roman ones are provided and an |\ifnumeric| test function.)
%
% \bigskip
%
% \section{Usage}
%
% \indent Just load the package placing
% \begin{quote}
%   |\usepackage{romanbar}|
% \end{quote}
% \noindent in the preamble of your \LaTeXe{} source file.\\
% |\Romanbar{...}| then produces a Roman number with bars (please see the example file).
% |\ifnumeric{test}{true}{false}| tests for |test| being numeric.
% |\romannum{...}| turns an Arabic number into a lowercase Roman one, and
% |\Romannum{...}| turns an Arabic number into an uppercase Roman one.
%
% \subsection*{Options}
% \DescribeMacro{options}
% \indent The \xpackage{romanbar} package takes no options.
%
% \section{Alternatives\label{sec:Alternatives}}
%
% \begin{description}
% \item[-] the original code written by Prof. Enrico Gregorio\\
%   (\url{http://profs.sci.univr.it/~gregorio/}),\\
%   \url{http://tex.stackexchange.com/questions/24065/roman-numerals-formatting/24084#24084}
%\begin{verbatim}
%\def\barroman#1{\sbox0{#1}\dimen0=\dimexpr\wd0+1pt\relax
%  \makebox[\dimen0]{\rlap{\vrule width\dimen0 height 0.06ex depth 0.06ex}%
%    \rlap{\vrule width\dimen0 height\dimexpr\ht0+0.03ex\relax
%            depth\dimexpr-\ht0+0.09ex\relax}%
%    \kern.5pt#1\kern.5pt}}
%
%\barroman{I} \barroman{XI}
%\end{verbatim}
%
% \item[-]  Some fonts provide single characters for Roman numerals,
%   cf.~e.\,g.
% \url{http://tex.stackexchange.com/questions/38695/using-unicode-roman-numerals-in-xetex}.
% \end{description}
%
% \noindent (You programmed or found another alternative,
%  which is available at \CTAN{}?\\
%  OK, send an e-mail to me with the name, location at \CTAN{},
%  and a short notice, and I will probably include it in
%  the list above.)
%
% \newpage
%
% \section{Example}
%
%    \begin{macrocode}
%<*example>
\documentclass[british]{article}[2007/10/19]% v1.4h
%%%%%%%%%%%%%%%%%%%%%%%%%%%%%%%%%%%%%%%%%%%%%%%%%%%%%%%%%%%%%%%%%%%%%
\usepackage[extension=pdf,%
 plainpages=false,%
 pdfpagelabels=true,%
 hyperindex=false,%
 pdflang={en},%
 pdftitle={romanbar package example},%
 pdfauthor={H.-Martin Muench},%
 pdfsubject={Example for the romanbar package},%
 pdfkeywords={LaTeX, romanbar, roman, Roman, bars, H.-Martin Muench},%
 pdfview={XYZ null null 1},%
 pdfstartview={XYZ null null 1},%
 pdfpagelayout=SinglePage]{hyperref}[2011/12/04]% v6.82m
\usepackage{romanbar}[2012/01/01]% v1.0f
\gdef\unit#1{\mathord{\thinspace\mathrm{#1}}}%
\listfiles
\begin{document}
\pagenumbering{arabic}
\section*{Example for romanbar}

This example demonstrates the use of package\newline
\textsf{romanbar}, v1.0f as of 2012/01/01 (HMM).\newline
There are no options to be used.\newline

\noindent For more details please see the documentation!\newline

\noindent Save per page about $200\unit{ml}$ water,
$2\unit{g}$ CO$_{2}$ and $2\unit{g}$ wood:\newline
Therefore please print only if this is really necessary.\newline

\noindent This package provides the command \verb|\Romanbar|
to print bars below and over the following:

\begin{description}
\item[-] Roman numbers: \verb|\Romanbar{MMXII}| prints \Romanbar{MMXII}

\item[-] Arabic numbers turned into upper-case Roman numbers:\newline
           \verb|\Romanbar{2012}| prints \Romanbar{2012}

\item[-] negative Arabic numbers turned into upper-case Roman numbers
           \newline
           (although historically there were no negative Roman numbers):
           \newline
           \verb|\Romanbar{-12}| prints \Romanbar{-12}

\item[-] zero Arabic number ($0$; although historically
           there was no Roman zero):\newline
           \verb|\Romanbar{0}| prints \Romanbar{0}

\item[-] some arbitrary text:
           \verb|\Romanbar{Caesar}| prints \Romanbar{Caesar}\newline
           (with descenders: \Romanbar{AgjpqyW})


\newcounter{example}
\setcounter{example}{21}

\item[-] some counter's value:
           \verb|\Romanbar{\theexample}| prints \Romanbar{\theexample}
           \newline
           (where the value of \texttt{example} is \theexample)

\item[-] Arabic numbers, without turning them into upper-case Roman
           numbers:\newline
           \verb|\Romanbar{\relax 2012}| prints \Romanbar{\relax  2012}
\end{description}

Special care was taken for "L" (50), e.\,g. in 555/DLV: \Romanbar{555}.\\


\verb|\romannum{...}| converts an Arabic number into a lower-case Roman one,
e.\,g. \verb|\romannum{2012}| prints \romannum{2012}, and
\verb|\Romannum{...}| converts an Arabic number into an upper-case Roman one,
e.\,g. \verb|\Romannum{2012}| prints \Romannum{2012}.

\end{document}
%</example>
%    \end{macrocode}
%
% \StopEventually{}
%
% \newpage
%
% \section{The implementation}
%
% We start off by checking that we are loading into \LaTeXe{} and
% announcing the name and version of this package.
%
%    \begin{macrocode}
%<*package>
%    \end{macrocode}
%
%    \begin{macrocode}
\NeedsTeXFormat{LaTeX2e}[2011/06/27]
\ProvidesPackage{romanbar}[2012/01/01 v1.0f
              Roman numbers with bars (HMM)]

%    \end{macrocode}
%
% A short description of the \xpackage{romanbar} package:
%
%    \begin{macrocode}
%% Allows to write Roman numbers (or any other text) with bars
%% and to convert Arabic numbers into Roman ones.
%    \end{macrocode}
%
% A last information for the user:
%
%    \begin{macrocode}
%% romanbar may work with earlier versions of LaTeX,
%% but this was not tested. Please consider updating
%% your LaTeX to the most recent version
%% (if it is not already the most recent version).

%    \end{macrocode}
%
% \indent See subsection~\ref{ss:Downloads} about how to get it.\\
%
% There are no options to be processed.
%
% A command to gobble a possible minus sign is needed and
% |\gobbleminus| from \url{http://www.tex.ac.uk/cgi-bin/texfaq2html?label=isitanum}
% is used.
%
%    \begin{macrocode}
\def\gobbleminus#1{\ifx-#1\else#1\fi}
%% from http://www.tex.ac.uk/cgi-bin/texfaq2html?label=isitanum

%    \end{macrocode}
%
% We want to test whether the argument passed to |\Romanbar| is numeric
% and therefore define |\ifnumeric{test}{true}{false}|,
% where |test| is to be analysed, |true| is the code to be executed,
% if |test| is numeric, and |false| is the code to be executed,
% if |test| is not numeric. This is done similar to\\
% \url{http://tex.stackexchange.com/a/17119}.
%
%    \begin{macrocode}
\newcommand{\ifnumeric}[3]{%
%% similar to http://tex.stackexchange.com/a/17119
\sbox\z@{\@tempcnta=0\gobbleminus#1\relax}%
\ifdim\wd0>\z@\relax#3% is not numeric
\else#2% is numeric
\fi%
}

%    \end{macrocode}
%
% We define |\r@iseL| to raise any L (50), otherwise a good deal
% of the horizontal part of the letter would be swallowed by the lower bar
% under the \textquotedblleft{}number\textquotedblright{}.
%
%    \begin{macrocode}
\def\r@iseL#1{\ifx @#1% then terminate
\else%
\if L#1\raisebox{0.05ex}{L}\else #1\fi%
\expandafter\r@iseL%
\fi%
}

%    \end{macrocode}
%
% \pagebreak
%
% We define the |\Romanbar| command.
%
%    \begin{macrocode}
\DeclareRobustCommand{\Romanbar}[1]{%
\ifnumeric{#1}{% is numeric
%    \end{macrocode}
%
% If its parameter is numeric, the temporary counter |\@tempcnta| is set to the value.
%
%    \begin{macrocode}
\@tempcnta=#1\relax%
%    \end{macrocode}
%
% Normally there are no non-positive Roman numbers, therefore we do some trick
% similar to |\XRoman| from the \url{http://ctan.org/pkg/pageslts} package.
%
%    \begin{macrocode}
%% similar to \XRoman from the http://ctan.org/pkg/pageslts package
\ifnum\@tempcnta<1%
  \ifnum #1>0%
    \relax \Roman@bar{\Roman{#1}}%
  \else%
    \ifnum #1<0%
     -\Roman@bar{\expandafter\@slowromancap\romannumeral\number-#1@%
     }%
    \else%
      \Roman@bar{0}%
    \fi%
  \fi%
\else\Roman@bar{\@Roman\@tempcnta}%
\fi%
}{% is not numeric
\Roman@bar{#1}%
}%
}

%    \end{macrocode}
%
% Whether it is numeric or not, in the end |\Roman@bar| is applied to it.
%
%    \begin{macrocode}
\DeclareRobustCommand{\Roman@bar}[1]{% #1 is in Roman, i.e. MMXII
%% (or any other text, "Caesar" would work, too).
%% similar to code of  by Prof. Enrico Gregorio (egreg) at
%% http://tex.stackexchange.com/questions/24065/roman-numerals-formatting/24084#24084
\@bsphack%
\edef\romanbartmp{#1}%
%% height:
\sbox0{\textsf{\romanbartmp}}%
%% top line:
%% + top of top line:
\@tempdima=\ht0%
\advance\@tempdima+0.05ex%0.03
\dimen1=\@tempdima%
%% + bottom of top line:
\@tempdima=-\ht0%
\advance\@tempdima+0.05ex%0.07
\dimen2=\@tempdima%
%% width:
\sbox0{\textsf{\expandafter\r@iseL\romanbartmp @}}%
\@tempdima=\wd0%
\advance\@tempdima+1pt%
\dimen0=\@tempdima%
\@esphack%
\makebox[\dimen0]{%
\rlap{\vrule width\dimen0 height\dimen1 depth\dimen2}% top    line
\rlap{\vrule width\dimen0 height 0.06ex depth 0.03ex}% bottom line
\kern0.5pt\textsf{\expandafter\r@iseL\romanbartmp @}\kern0.5pt}%
}

%    \end{macrocode}
%
% While it is not needed for the bars, it is sometimes asked for a command
% to convert an Arabic number into a Roman one. |\roman{...}| can only be applied
% to a counter, i.\,e. |\roman{12}| does not work, but
%\begin{verbatim}
%\newcounter{examplecounter}
%\setcounter{examplecounter}{12}
%\roman{examplecounter}
%\end{verbatim}
% would be necessary. Further non-negative values would not work.
% |\romannum| accepts any number as argument. (If the argument is no number,
% an error message is given and the argument printed (executed)
%  without applying anything to it.)
%
%    \begin{macrocode}
\DeclareRobustCommand{\romannum}[1]{%
\ifnumeric{#1}{% is numeric
\@tempcnta=#1\relax%
%% similar to \xroman from the http://ctan.org/pkg/pageslts package
\ifnum\@tempcnta<1%
  \ifnum #1>0%
    \relax \roman{#1}%
  \else%
    \ifnum #1<0%
     -\romannumeral\number-#1@%
    \else%
      0%
    \fi%
  \fi%
\else\@roman\@tempcnta%
\fi%
}{% is not numeric
\PackageError{romanbar}{%
Argument of \string\romannum\space is not a number}{%
The command \string\romannum\space converts an Arabic number into a%
 lower-case Roman one,\MessageBreak%
but the used argument of \string\romannum\space is not an Arabic number%
 but\MessageBreak%
`#1`,\MessageBreak%
which will now be printed unchanged.}%
#1%
}%
}

%    \end{macrocode}
%
% The same for upper-case Roman numbers, |\Romannum| instead of |\Roman|:
%
%    \begin{macrocode}
\DeclareRobustCommand{\Romannum}[1]{%
\ifnumeric{#1}{% is numeric
\@tempcnta=#1\relax%
%% similar to \XRoman from the http://ctan.org/pkg/pageslts package
\ifnum\@tempcnta<1%
  \ifnum #1>0%
    \relax \Roman{#1}%
  \else%
    \ifnum #1<0%
     -\expandafter\@slowromancap\romannumeral\number-#1@%
    \else%
      0%
    \fi%
  \fi%
\else\@Roman\@tempcnta%
\fi%
}{% is not numeric
\PackageError{romanbar}{%
Argument of \string\Romannum\space is not a number}{%
The command \string\Romannum\space converts an Arabic number into an%
 upper-case Roman one,\MessageBreak%
but the used argument of \string\Romannum\space is not an Arabic number%
 but\MessageBreak%
`#1`.}%
#1%
}%
}

%    \end{macrocode}
%
%    \begin{macrocode}
%</package>
%    \end{macrocode}
%
% \newpage
%
% \section{Installation}
%
% \subsection{Downloads\label{ss:Downloads}}
%
% Everything should be available on \CTAN{}, \url{http://www.ctan.org/tex-archive/},
% but may need additional packages themselves.\\
%
% \DescribeMacro{romanbar.dtx}
% For unpacking the |romanbar.dtx| file and constructing the documentation
% it is required:
% \begin{description}
% \item[-] \TeX{} Format \LaTeXe{}, \url{http://www.CTAN.org/}
%
% \item[-] document class \xpackage{ltxdoc}, 2007/11/11, v2.0u,
%           \url{http://ctan.org/pkg/ltxdoc}
%
% \item[-] package \xpackage{holtxdoc}, 2011/02/04, v0.21,
%           \url{http://ctan.org/pkg/holtxdoc}
%
% \item[-] package \xpackage{hypdoc}, 2010/03/26, v1.9,
%           \url{http://ctan.org/pkg/hypdoc}
% \end{description}
%
% \DescribeMacro{romanbar.sty}
% The \texttt{romanbar.sty} for \LaTeXe{} (i.\,e.~each document using
% the \xpackage{romanbar} package) requires:
% \begin{description}
% \item[-] \TeX Format \LaTeXe{}, \url{http://www.CTAN.org/}
% \end{description}
%
% \DescribeMacro{romanbar-example.tex}
% The \texttt{romanbar-example.tex} requires the same files as all
% documents using the \xpackage{romanbar} package and additionally:
% \begin{description}
% \item[-] class \xpackage{article}, 2007/10/19, v1.4h, from \xpackage{classes.dtx}:\\
%           \CTAN{macros/latex/base/classes.dtx}
%
% \item[-] package \xpackage{romanbar}, 2012/01/01, v1.0f,\\
%           %%\url{http://ctan.org/pkg/romanbar}\\
%           \CTAN{macros/latex/contrib/romanbar/}
%   (Well, it is the example file for this package, and because you are reading the
%    documentation for the \xpackage{romanbar} package, it can be assumed that you already
%    have some version of it -- is it the current one?)
% \end{description}
%
% \DescribeMacro{Alternative}
% As possible alternative package in section \ref{sec:Alternatives} there is listed
% \begin{description}
% \item[-] none.
% \end{description}
%
% \DescribeMacro{Oberdiek}
% \DescribeMacro{holtxdoc}
% All packages of \textsc{Heiko Oberdiek's} bundle `oberdiek'
% (especially \xpackage{holtxdoc}) are also available in a TDS compliant ZIP archive:\\
% \CTAN{install/macros/latex/contrib/oberdiek.tds.zip}.\\
% It is probably best to download and use this, because the packages in there
% are quite probably both recent and compatible among themselves.\\
%
% \DescribeMacro{hyperref}
% \noindent \xpackage{hyperref} is not included in that bundle and needs to be
% downloaded separately,\\
% \url{http://mirror.ctan.org/install/macros/latex/contrib/hyperref.tds.zip}.\\
%
% \DescribeMacro{M\"{u}nch}
% A hyperlinked list of my (other) packages can be found at
% \url{http://www.Uni-Bonn.de/~uzs5pv/LaTeX.html}.\\
%
% \pagebreak
%
% \subsection{Package, unpacking TDS}
% \paragraph{Package.} This package should become available on \CTAN{} soon:
% \begin{description}
% \item[\CTAN{macros/latex/contrib/romanbar/romanbar.dtx}]\hspace*{0.1cm} \\
%       The source file.
% \item[\CTAN{macros/latex/contrib/romanbar/romanbar.pdf}]\hspace*{0.1cm} \\
%       The documentation.
% \item[\CTAN{macros/latex/contrib/romanbar/romanbar-example.pdf}]\hspace*{0.1cm} \\
%       The compiled example file, as it should look like.
% \item[\CTAN{macros/latex/contrib/romanbar/README}]\hspace*{0.1cm} \\
%       The README file.
% \end{description}
% There is also a romanbar.tds.zip available:
% \begin{description}
% \item[\CTAN{install/macros/latex/contrib/romanbar.tds.zip}]\hspace*{0.1cm} \\
%       Everything in \xfile{TDS} compliant, compiled format.
% \end{description}
% which additionally contains\\
% \begin{tabular}{ll}
% romanbar.ins & The installation file.\\
% romanbar.drv & The driver to generate the documentation.\\
% romanbar.sty & The \xext{sty}le file.\\
% romanbar-example.tex & The example file.
% \end{tabular}
%
% \bigskip
%
% \noindent For required other packages, please see the preceding subsection.
%
% \paragraph{Unpacking.} The \xfile{.dtx} file is a self-extracting
% \docstrip{} archive. The files are extracted by running the
% \xext{.dtx} through \plainTeX{}:
% \begin{quote}
%   \verb|tex romanbar.dtx|
% \end{quote}
%
% About generating the documentation see paragraph~\ref{GenDoc} below.\\
%
% \paragraph{TDS.} Now the different files must be moved into
% the different directories in your installation TDS tree
% (also known as \xfile{texmf} tree):
% \begin{quote}
% \def\t{^^A
% \begin{tabular}{@{}>{\ttfamily}l@{ $\rightarrow$ }>{\ttfamily}l@{}}
%   romanbar.sty & tex/latex/romanbar/romanbar.sty\\
%   romanbar.pdf & doc/latex/romanbar/romanbar.pdf\\
%   romanbar-example.tex & doc/latex/romanbar/romanbar-example.tex\\
%   romanbar-example.pdf & doc/latex/romanbar/romanbar-example.pdf\\
%   romanbar.dtx & source/latex/romanbar/romanbar.dtx\\
% \end{tabular}^^A
% }^^A
% \sbox0{\t}^^A
% \ifdim\wd0>\linewidth
%   \begingroup
%     \advance\linewidth by\leftmargin
%     \advance\linewidth by\rightmargin
%   \edef\x{\endgroup
%     \def\noexpand\lw{\the\linewidth}^^A
%   }\x
%   \def\lwbox{^^A
%     \leavevmode
%     \hbox to \linewidth{^^A
%       \kern-\leftmargin\relax
%       \hss
%       \usebox0
%       \hss
%       \kern-\rightmargin\relax
%     }^^A
%   }^^A
%   \ifdim\wd0>\lw
%     \sbox0{\small\t}^^A
%     \ifdim\wd0>\linewidth
%       \ifdim\wd0>\lw
%         \sbox0{\footnotesize\t}^^A
%         \ifdim\wd0>\linewidth
%           \ifdim\wd0>\lw
%             \sbox0{\scriptsize\t}^^A
%             \ifdim\wd0>\linewidth
%               \ifdim\wd0>\lw
%                 \sbox0{\tiny\t}^^A
%                 \ifdim\wd0>\linewidth
%                   \lwbox
%                 \else
%                   \usebox0
%                 \fi
%               \else
%                 \lwbox
%               \fi
%             \else
%               \usebox0
%             \fi
%           \else
%             \lwbox
%           \fi
%         \else
%           \usebox0
%         \fi
%       \else
%         \lwbox
%       \fi
%     \else
%       \usebox0
%     \fi
%   \else
%     \lwbox
%   \fi
% \else
%   \usebox0
% \fi
% \end{quote}
% If you have a \xfile{docstrip.cfg} that configures and enables \docstrip{}'s
% \xfile{TDS} installing feature, then some files can already be in the right
% place, see the documentation of \docstrip{}.
%
% \subsection{Refresh file name databases}
%
% If your \TeX{}~distribution (\teTeX{}, \mikTeX{},\dots{}) relies on file name
% databases, you must refresh these. For example, \teTeX{} users run
% \verb|texhash| or \verb|mktexlsr|.
%
% \subsection{Some details for the interested}
%
% \paragraph{Unpacking with \LaTeX{}.}
% The \xfile{.dtx} chooses its action depending on the format:
% \begin{description}
% \item[\plainTeX:] Run \docstrip{} and extract the files.
% \item[\LaTeX:] Generate the documentation.
% \end{description}
% If you insist on using \LaTeX{} for \docstrip{} (really,
% \docstrip{} does not need \LaTeX{}), then inform the autodetect routine
% about your intention:
% \begin{quote}
%   \verb|latex \let\install=y% \iffalse meta-comment
%
% File: romanbar.dtx
% Version: 2012/01/01 v1.0f
%
% Copyright (C) 2011 by
%    H.-Martin M"unch <Martin dot Muench at Uni-Bonn dot de>
%
% The original code for the bars was written by Prof. Enrico Gregorio
% (http://profs.sci.univr.it/~gregorio/), and published at
% http://tex.stackexchange.com/questions/24065/roman-numerals-formatting/24084#24084
% I (H.-Martin Münch) submitted this package to CTAN
% (after diverse updates of the code and writing a documentation).
%
% This work may be distributed and/or modified under the
% conditions of the LaTeX Project Public License, either
% version 1.3c of this license or (at your option) any later
% version. This version of this license is in
%    http://www.latex-project.org/lppl/lppl-1-3c.txt
% and the latest version of this license is in
%    http://www.latex-project.org/lppl.txt
% and version 1.3c or later is part of all distributions of
% LaTeX version 2005/12/01 or later.
%
% This work has the LPPL maintenance status "maintained".
%
% The Current Maintainer of this work is H.-Martin Muench.
%
% This work consists of the main source file romanbar.dtx,
% the README, and the derived files
%    romanbar.sty, romanbar.pdf,
%    romanbar.ins, romanbar.drv,
%    romanbar-example.tex, romanbar-example.pdf.
%
% Distribution:
%    CTAN:macros/latex/contrib/romanbar/romanbar.dtx
%    CTAN:macros/latex/contrib/romanbar/romanbar.pdf
%    CTAN:install/macros/latex/contrib/romanbar.tds.zip
%
% Cf. http://ctan.org/pkg/romanbar and
% http://www.tex.ac.uk/tex-archive/help/Catalogue/entries/romanbar.html
% (It might take some time for the entries to be created.)
%
% Unpacking:
%    (a) If romanbar.ins is present:
%           tex romanbar.ins
%    (b) Without romanbar.ins:
%           tex romanbar.dtx
%    (c) If you insist on using LaTeX
%           latex \let\install=y% \iffalse meta-comment
%
% File: romanbar.dtx
% Version: 2012/01/01 v1.0f
%
% Copyright (C) 2011 by
%    H.-Martin M"unch <Martin dot Muench at Uni-Bonn dot de>
%
% The original code for the bars was written by Prof. Enrico Gregorio
% (http://profs.sci.univr.it/~gregorio/), and published at
% http://tex.stackexchange.com/questions/24065/roman-numerals-formatting/24084#24084
% I (H.-Martin Münch) submitted this package to CTAN
% (after diverse updates of the code and writing a documentation).
%
% This work may be distributed and/or modified under the
% conditions of the LaTeX Project Public License, either
% version 1.3c of this license or (at your option) any later
% version. This version of this license is in
%    http://www.latex-project.org/lppl/lppl-1-3c.txt
% and the latest version of this license is in
%    http://www.latex-project.org/lppl.txt
% and version 1.3c or later is part of all distributions of
% LaTeX version 2005/12/01 or later.
%
% This work has the LPPL maintenance status "maintained".
%
% The Current Maintainer of this work is H.-Martin Muench.
%
% This work consists of the main source file romanbar.dtx,
% the README, and the derived files
%    romanbar.sty, romanbar.pdf,
%    romanbar.ins, romanbar.drv,
%    romanbar-example.tex, romanbar-example.pdf.
%
% Distribution:
%    CTAN:macros/latex/contrib/romanbar/romanbar.dtx
%    CTAN:macros/latex/contrib/romanbar/romanbar.pdf
%    CTAN:install/macros/latex/contrib/romanbar.tds.zip
%
% Cf. http://ctan.org/pkg/romanbar and
% http://www.tex.ac.uk/tex-archive/help/Catalogue/entries/romanbar.html
% (It might take some time for the entries to be created.)
%
% Unpacking:
%    (a) If romanbar.ins is present:
%           tex romanbar.ins
%    (b) Without romanbar.ins:
%           tex romanbar.dtx
%    (c) If you insist on using LaTeX
%           latex \let\install=y\input{romanbar.dtx}
%        (quote the arguments according to the demands of your shell)
%
% Documentation:
%    (a) If romanbar.drv is present:
%           (pdf)latex romanbar.drv
%           makeindex -s gind.ist romanbar.idx
%           (pdf)latex romanbar.drv
%           makeindex -s gind.ist romanbar.idx
%           (pdf)latex romanbar.drv
%    (b) Without romanbar.drv:
%           (pdf)latex romanbar.dtx
%           makeindex -s gind.ist romanbar.idx
%           (pdf)latex romanbar.dtx
%           makeindex -s gind.ist romanbar.idx
%           (pdf)latex romanbar.dtx
%
%    The class ltxdoc loads the configuration file ltxdoc.cfg
%    if available. Here you can specify further options, e.g.
%    use DIN A4 as paper format:
%       \PassOptionsToClass{a4paper}{article}
%
% Installation:
%    TDS:tex/latex/romanbar/romanbar.sty
%    TDS:doc/latex/romanbar/romanbar.pdf
%    TDS:doc/latex/romanbar/romanbar-example.tex
%    TDS:doc/latex/romanbar/romanbar-example.pdf
%    TDS:source/latex/romanbar/romanbar.dtx
%
%<*ignore>
\begingroup
  \catcode123=1 %
  \catcode125=2 %
  \def\x{LaTeX2e}%
\expandafter\endgroup
\ifcase 0\ifx\install y1\fi\expandafter
         \ifx\csname processbatchFile\endcsname\relax\else1\fi
         \ifx\fmtname\x\else 1\fi\relax
\else\csname fi\endcsname
%</ignore>
%<*install>
\input docstrip.tex
\Msg{********************************************************************}
\Msg{* Installation                                                     *}
\Msg{* Package: romanbar 2012/01/01 v1.0f Roman numbers with bars (HMM) *}
\Msg{********************************************************************}

\keepsilent
\askforoverwritefalse

\let\MetaPrefix\relax
\preamble

This is a generated file.

Project: romanbar
Version: 2012/01/01 v1.0f

Copyright (C) 2011 by
    H.-Martin M"unch <Martin dot Muench at Uni-Bonn dot de>

The usual disclaimer applies:
If it doesn't work right that's your problem.
(Nevertheless, send an e-mail to the maintainer
 when you find an error in this package.)

This work may be distributed and/or modified under the
conditions of the LaTeX Project Public License, either
version 1.3c of this license or (at your option) any later
version. This version of this license is in
   http://www.latex-project.org/lppl/lppl-1-3c.txt
and the latest version of this license is in
   http://www.latex-project.org/lppl.txt
and version 1.3c or later is part of all distributions of
LaTeX version 2005/12/01 or later.

This work has the LPPL maintenance status "maintained".

The Current Maintainer of this work is H.-Martin Muench.

This work consists of the main source file romanbar.dtx,
the README, and the derived files
   romanbar.sty, romanbar.pdf,
   romanbar.ins, romanbar.drv,
   romanbar-example.tex, romanbar-example.pdf.

\endpreamble
\let\MetaPrefix\DoubleperCent

\generate{%
  \file{romanbar.ins}{\from{romanbar.dtx}{install}}%
  \file{romanbar.drv}{\from{romanbar.dtx}{driver}}%
  \usedir{tex/latex/romanbar}%
  \file{romanbar.sty}{\from{romanbar.dtx}{package}}%
  \usedir{doc/latex/romanbar}%
  \file{romanbar-example.tex}{\from{romanbar.dtx}{example}}%
}

\catcode32=13\relax% active space
\let =\space%
\Msg{************************************************************************}
\Msg{*}
\Msg{* To finish the installation you have to move the following}
\Msg{* file into a directory searched by TeX:}
\Msg{*}
\Msg{*     romanbar.sty}
\Msg{*}
\Msg{* To produce the documentation run the file `romanbar.drv'}
\Msg{* through (pdf)LaTeX, e.g.}
\Msg{*  pdflatex romanbar.drv}
\Msg{*  makeindex -s gind.ist romanbar.idx}
\Msg{*  pdflatex romanbar.drv}
\Msg{*  makeindex -s gind.ist romanbar.idx}
\Msg{*  pdflatex romanbar.drv}
\Msg{*}
\Msg{* At least three runs are necessary e.g. to get the}
\Msg{*  references right!}
\Msg{*}
\Msg{* Happy TeXing!}
\Msg{*}
\Msg{************************************************************************}

\endbatchfile
%</install>
%<*ignore>
\fi
%</ignore>
%
% \section{The documentation driver file}
%
% The next bit of code contains the documentation driver file for
% \TeX{}, i.\,e., the file that will produce the documentation you
% are currently reading. It will be extracted from this file by the
% \texttt{docstrip} programme. That is, run \LaTeX{} on \texttt{docstrip}
% and specify the \texttt{driver} option when \texttt{docstrip}
% asks for options.
%
%    \begin{macrocode}
%<*driver>
\NeedsTeXFormat{LaTeX2e}[2011/06/27]
\ProvidesFile{romanbar.drv}%
  [2012/01/01 v1.0f Roman numbers with bars (HMM)]
\documentclass{ltxdoc}[2007/11/11]% v2.0u
\usepackage{holtxdoc}[2011/02/04]%  v0.21
%% romanbar may work with earlier versions of LaTeX2e and those
%% class and package, but this was not tested.
%% Please consider updating your LaTeX, class, and package
%% to the most recent version (if they are not already the most
%% recent version).
\hypersetup{%
 pdfsubject={Writing Roman numbers with bars (HMM)},%
 pdfkeywords={LaTeX, romanbar, roman, Roman, bars, H.-Martin Muench},%
 pdfencoding=auto,%
 pdflang={en},%
 breaklinks=true,%
 linktoc=all,%
 pdfstartview=FitH,%
 pdfpagelayout=OneColumn,%
 bookmarksnumbered=true,%
 bookmarksopen=true,%
 bookmarksopenlevel=2,%
 pdfmenubar=true,%
 pdftoolbar=true,%
 pdfwindowui=true,%
 pdfnewwindow=true%
}
\CodelineIndex
\hyphenation{docu-ment ana-lysed}
\gdef\unit#1{\mathord{\thinspace\mathrm{#1}}}%
\makeatletter
\@ifundefined{eTeX}{\gdef\eTeX{$\m@th \varepsilon $-\TeX }}{% else \relax
}
\makeatother
\begin{document}
  \DocInput{romanbar.dtx}%
\end{document}
%</driver>
%    \end{macrocode}
%
% \fi
%
% \CheckSum{179}
%
% \CharacterTable
%  {Upper-case    \A\B\C\D\E\F\G\H\I\J\K\L\M\N\O\P\Q\R\S\T\U\V\W\X\Y\Z
%   Lower-case    \a\b\c\d\e\f\g\h\i\j\k\l\m\n\o\p\q\r\s\t\u\v\w\x\y\z
%   Digits        \0\1\2\3\4\5\6\7\8\9
%   Exclamation   \!     Double quote  \"     Hash (number) \#
%   Dollar        \$     Percent       \%     Ampersand     \&
%   Acute accent  \'     Left paren    \(     Right paren   \)
%   Asterisk      \*     Plus          \+     Comma         \,
%   Minus         \-     Point         \.     Solidus       \/
%   Colon         \:     Semicolon     \;     Less than     \<
%   Equals        \=     Greater than  \>     Question mark \?
%   Commercial at \@     Left bracket  \[     Backslash     \\
%   Right bracket \]     Circumflex    \^     Underscore    \_
%   Grave accent  \`     Left brace    \{     Vertical bar  \|
%   Right brace   \}     Tilde         \~}
%
% \GetFileInfo{romanbar.drv}
%
% \begingroup
%   \def\x{\#,\$,\^,\_,\~,\ ,\&,\{,\},\%}%
%   \makeatletter
%   \@onelevel@sanitize\x
% \expandafter\endgroup
% \expandafter\DoNotIndex\expandafter{\x}
% \expandafter\DoNotIndex\expandafter{\string\ }
% \begingroup
%   \makeatletter
%     \lccode`9=32\relax
%     \lowercase{%^^A
%       \edef\x{\noexpand\DoNotIndex{\@backslashchar9}}%^^A
%     }%^^A
%   \expandafter\endgroup\x
%
% \DoNotIndex{\\,\,}
% \DoNotIndex{\documentclass,\usepackage,\ProvidesPackage,\begin,\end}
% \DoNotIndex{\MessageBreak}
% \DoNotIndex{\NeedsTeXFormat,\DoNotIndex,\verb}
% \DoNotIndex{\def,\edef,\gdef,\global}
% \DoNotIndex{\ifx,\listfiles,\mathord,\mathrm}
% \DoNotIndex{\bigskip,\space,\thinspace,\ldots}
% \DoNotIndex{\indent,\noindent,\newline,\linebreak,\pagebreak,\newpage}
% \DoNotIndex{\textbf,\textit,\textsf,\texttt,\textquotedblleft,\textquotedblright}
% \DoNotIndex{\plainTeX,\TeX,\LaTeX,\pdfLaTeX}
% \DoNotIndex{\item}
%
% \title{The \xpackage{romanbar} package}
% \date{2012/01/01 v1.0f}
% \author{H.-Martin M\"{u}nch\\\xemail{Martin.Muench at Uni-Bonn.de}}
%
% \maketitle
%
% \begin{abstract}
%  This \LaTeX{} package allows to write Roman numbers
%  (or any other text) with bars. (Additionally, commands for converting Arabic numbers
%  into Roman ones are provided and an |\ifnumeric| test function.)
% \end{abstract}
%
% \bigskip
%
% \noindent Disclaimer for web links: The author is not responsible for any contents
% referred to in this work unless he has full knowledge of illegal contents.
% If any damage occurs by the use of information presented there, only the
% author of the respective pages might be liable, not the one who has referred
% to these pages.
%
% \bigskip
%
% \noindent {\color{green} Save per page about $200\unit{ml}$ water,
% $2\unit{g}$ CO$_{2}$ and $2\unit{g}$ wood:\\
% Therefore please print only if this is really necessary.}
%
% \pagebreak
%
% \tableofcontents
%
% \pagebreak
%
% \section{Introduction}
%
% \indent This \LaTeX{} package allows to write Roman numbers
% (or any other text) with bars. (Additionally, command for converting Arabic numbers
% into Roman ones are provided and an |\ifnumeric| test function.)
%
% \bigskip
%
% \section{Usage}
%
% \indent Just load the package placing
% \begin{quote}
%   |\usepackage{romanbar}|
% \end{quote}
% \noindent in the preamble of your \LaTeXe{} source file.\\
% |\Romanbar{...}| then produces a Roman number with bars (please see the example file).
% |\ifnumeric{test}{true}{false}| tests for |test| being numeric.
% |\romannum{...}| turns an Arabic number into a lowercase Roman one, and
% |\Romannum{...}| turns an Arabic number into an uppercase Roman one.
%
% \subsection*{Options}
% \DescribeMacro{options}
% \indent The \xpackage{romanbar} package takes no options.
%
% \section{Alternatives\label{sec:Alternatives}}
%
% \begin{description}
% \item[-] the original code written by Prof. Enrico Gregorio\\
%   (\url{http://profs.sci.univr.it/~gregorio/}),\\
%   \url{http://tex.stackexchange.com/questions/24065/roman-numerals-formatting/24084#24084}
%\begin{verbatim}
%\def\barroman#1{\sbox0{#1}\dimen0=\dimexpr\wd0+1pt\relax
%  \makebox[\dimen0]{\rlap{\vrule width\dimen0 height 0.06ex depth 0.06ex}%
%    \rlap{\vrule width\dimen0 height\dimexpr\ht0+0.03ex\relax
%            depth\dimexpr-\ht0+0.09ex\relax}%
%    \kern.5pt#1\kern.5pt}}
%
%\barroman{I} \barroman{XI}
%\end{verbatim}
%
% \item[-]  Some fonts provide single characters for Roman numerals,
%   cf.~e.\,g.
% \url{http://tex.stackexchange.com/questions/38695/using-unicode-roman-numerals-in-xetex}.
% \end{description}
%
% \noindent (You programmed or found another alternative,
%  which is available at \CTAN{}?\\
%  OK, send an e-mail to me with the name, location at \CTAN{},
%  and a short notice, and I will probably include it in
%  the list above.)
%
% \newpage
%
% \section{Example}
%
%    \begin{macrocode}
%<*example>
\documentclass[british]{article}[2007/10/19]% v1.4h
%%%%%%%%%%%%%%%%%%%%%%%%%%%%%%%%%%%%%%%%%%%%%%%%%%%%%%%%%%%%%%%%%%%%%
\usepackage[extension=pdf,%
 plainpages=false,%
 pdfpagelabels=true,%
 hyperindex=false,%
 pdflang={en},%
 pdftitle={romanbar package example},%
 pdfauthor={H.-Martin Muench},%
 pdfsubject={Example for the romanbar package},%
 pdfkeywords={LaTeX, romanbar, roman, Roman, bars, H.-Martin Muench},%
 pdfview={XYZ null null 1},%
 pdfstartview={XYZ null null 1},%
 pdfpagelayout=SinglePage]{hyperref}[2011/12/04]% v6.82m
\usepackage{romanbar}[2012/01/01]% v1.0f
\gdef\unit#1{\mathord{\thinspace\mathrm{#1}}}%
\listfiles
\begin{document}
\pagenumbering{arabic}
\section*{Example for romanbar}

This example demonstrates the use of package\newline
\textsf{romanbar}, v1.0f as of 2012/01/01 (HMM).\newline
There are no options to be used.\newline

\noindent For more details please see the documentation!\newline

\noindent Save per page about $200\unit{ml}$ water,
$2\unit{g}$ CO$_{2}$ and $2\unit{g}$ wood:\newline
Therefore please print only if this is really necessary.\newline

\noindent This package provides the command \verb|\Romanbar|
to print bars below and over the following:

\begin{description}
\item[-] Roman numbers: \verb|\Romanbar{MMXII}| prints \Romanbar{MMXII}

\item[-] Arabic numbers turned into upper-case Roman numbers:\newline
           \verb|\Romanbar{2012}| prints \Romanbar{2012}

\item[-] negative Arabic numbers turned into upper-case Roman numbers
           \newline
           (although historically there were no negative Roman numbers):
           \newline
           \verb|\Romanbar{-12}| prints \Romanbar{-12}

\item[-] zero Arabic number ($0$; although historically
           there was no Roman zero):\newline
           \verb|\Romanbar{0}| prints \Romanbar{0}

\item[-] some arbitrary text:
           \verb|\Romanbar{Caesar}| prints \Romanbar{Caesar}\newline
           (with descenders: \Romanbar{AgjpqyW})


\newcounter{example}
\setcounter{example}{21}

\item[-] some counter's value:
           \verb|\Romanbar{\theexample}| prints \Romanbar{\theexample}
           \newline
           (where the value of \texttt{example} is \theexample)

\item[-] Arabic numbers, without turning them into upper-case Roman
           numbers:\newline
           \verb|\Romanbar{\relax 2012}| prints \Romanbar{\relax  2012}
\end{description}

Special care was taken for "L" (50), e.\,g. in 555/DLV: \Romanbar{555}.\\


\verb|\romannum{...}| converts an Arabic number into a lower-case Roman one,
e.\,g. \verb|\romannum{2012}| prints \romannum{2012}, and
\verb|\Romannum{...}| converts an Arabic number into an upper-case Roman one,
e.\,g. \verb|\Romannum{2012}| prints \Romannum{2012}.

\end{document}
%</example>
%    \end{macrocode}
%
% \StopEventually{}
%
% \newpage
%
% \section{The implementation}
%
% We start off by checking that we are loading into \LaTeXe{} and
% announcing the name and version of this package.
%
%    \begin{macrocode}
%<*package>
%    \end{macrocode}
%
%    \begin{macrocode}
\NeedsTeXFormat{LaTeX2e}[2011/06/27]
\ProvidesPackage{romanbar}[2012/01/01 v1.0f
              Roman numbers with bars (HMM)]

%    \end{macrocode}
%
% A short description of the \xpackage{romanbar} package:
%
%    \begin{macrocode}
%% Allows to write Roman numbers (or any other text) with bars
%% and to convert Arabic numbers into Roman ones.
%    \end{macrocode}
%
% A last information for the user:
%
%    \begin{macrocode}
%% romanbar may work with earlier versions of LaTeX,
%% but this was not tested. Please consider updating
%% your LaTeX to the most recent version
%% (if it is not already the most recent version).

%    \end{macrocode}
%
% \indent See subsection~\ref{ss:Downloads} about how to get it.\\
%
% There are no options to be processed.
%
% A command to gobble a possible minus sign is needed and
% |\gobbleminus| from \url{http://www.tex.ac.uk/cgi-bin/texfaq2html?label=isitanum}
% is used.
%
%    \begin{macrocode}
\def\gobbleminus#1{\ifx-#1\else#1\fi}
%% from http://www.tex.ac.uk/cgi-bin/texfaq2html?label=isitanum

%    \end{macrocode}
%
% We want to test whether the argument passed to |\Romanbar| is numeric
% and therefore define |\ifnumeric{test}{true}{false}|,
% where |test| is to be analysed, |true| is the code to be executed,
% if |test| is numeric, and |false| is the code to be executed,
% if |test| is not numeric. This is done similar to\\
% \url{http://tex.stackexchange.com/a/17119}.
%
%    \begin{macrocode}
\newcommand{\ifnumeric}[3]{%
%% similar to http://tex.stackexchange.com/a/17119
\sbox\z@{\@tempcnta=0\gobbleminus#1\relax}%
\ifdim\wd0>\z@\relax#3% is not numeric
\else#2% is numeric
\fi%
}

%    \end{macrocode}
%
% We define |\r@iseL| to raise any L (50), otherwise a good deal
% of the horizontal part of the letter would be swallowed by the lower bar
% under the \textquotedblleft{}number\textquotedblright{}.
%
%    \begin{macrocode}
\def\r@iseL#1{\ifx @#1% then terminate
\else%
\if L#1\raisebox{0.05ex}{L}\else #1\fi%
\expandafter\r@iseL%
\fi%
}

%    \end{macrocode}
%
% \pagebreak
%
% We define the |\Romanbar| command.
%
%    \begin{macrocode}
\DeclareRobustCommand{\Romanbar}[1]{%
\ifnumeric{#1}{% is numeric
%    \end{macrocode}
%
% If its parameter is numeric, the temporary counter |\@tempcnta| is set to the value.
%
%    \begin{macrocode}
\@tempcnta=#1\relax%
%    \end{macrocode}
%
% Normally there are no non-positive Roman numbers, therefore we do some trick
% similar to |\XRoman| from the \url{http://ctan.org/pkg/pageslts} package.
%
%    \begin{macrocode}
%% similar to \XRoman from the http://ctan.org/pkg/pageslts package
\ifnum\@tempcnta<1%
  \ifnum #1>0%
    \relax \Roman@bar{\Roman{#1}}%
  \else%
    \ifnum #1<0%
     -\Roman@bar{\expandafter\@slowromancap\romannumeral\number-#1@%
     }%
    \else%
      \Roman@bar{0}%
    \fi%
  \fi%
\else\Roman@bar{\@Roman\@tempcnta}%
\fi%
}{% is not numeric
\Roman@bar{#1}%
}%
}

%    \end{macrocode}
%
% Whether it is numeric or not, in the end |\Roman@bar| is applied to it.
%
%    \begin{macrocode}
\DeclareRobustCommand{\Roman@bar}[1]{% #1 is in Roman, i.e. MMXII
%% (or any other text, "Caesar" would work, too).
%% similar to code of  by Prof. Enrico Gregorio (egreg) at
%% http://tex.stackexchange.com/questions/24065/roman-numerals-formatting/24084#24084
\@bsphack%
\edef\romanbartmp{#1}%
%% height:
\sbox0{\textsf{\romanbartmp}}%
%% top line:
%% + top of top line:
\@tempdima=\ht0%
\advance\@tempdima+0.05ex%0.03
\dimen1=\@tempdima%
%% + bottom of top line:
\@tempdima=-\ht0%
\advance\@tempdima+0.05ex%0.07
\dimen2=\@tempdima%
%% width:
\sbox0{\textsf{\expandafter\r@iseL\romanbartmp @}}%
\@tempdima=\wd0%
\advance\@tempdima+1pt%
\dimen0=\@tempdima%
\@esphack%
\makebox[\dimen0]{%
\rlap{\vrule width\dimen0 height\dimen1 depth\dimen2}% top    line
\rlap{\vrule width\dimen0 height 0.06ex depth 0.03ex}% bottom line
\kern0.5pt\textsf{\expandafter\r@iseL\romanbartmp @}\kern0.5pt}%
}

%    \end{macrocode}
%
% While it is not needed for the bars, it is sometimes asked for a command
% to convert an Arabic number into a Roman one. |\roman{...}| can only be applied
% to a counter, i.\,e. |\roman{12}| does not work, but
%\begin{verbatim}
%\newcounter{examplecounter}
%\setcounter{examplecounter}{12}
%\roman{examplecounter}
%\end{verbatim}
% would be necessary. Further non-negative values would not work.
% |\romannum| accepts any number as argument. (If the argument is no number,
% an error message is given and the argument printed (executed)
%  without applying anything to it.)
%
%    \begin{macrocode}
\DeclareRobustCommand{\romannum}[1]{%
\ifnumeric{#1}{% is numeric
\@tempcnta=#1\relax%
%% similar to \xroman from the http://ctan.org/pkg/pageslts package
\ifnum\@tempcnta<1%
  \ifnum #1>0%
    \relax \roman{#1}%
  \else%
    \ifnum #1<0%
     -\romannumeral\number-#1@%
    \else%
      0%
    \fi%
  \fi%
\else\@roman\@tempcnta%
\fi%
}{% is not numeric
\PackageError{romanbar}{%
Argument of \string\romannum\space is not a number}{%
The command \string\romannum\space converts an Arabic number into a%
 lower-case Roman one,\MessageBreak%
but the used argument of \string\romannum\space is not an Arabic number%
 but\MessageBreak%
`#1`,\MessageBreak%
which will now be printed unchanged.}%
#1%
}%
}

%    \end{macrocode}
%
% The same for upper-case Roman numbers, |\Romannum| instead of |\Roman|:
%
%    \begin{macrocode}
\DeclareRobustCommand{\Romannum}[1]{%
\ifnumeric{#1}{% is numeric
\@tempcnta=#1\relax%
%% similar to \XRoman from the http://ctan.org/pkg/pageslts package
\ifnum\@tempcnta<1%
  \ifnum #1>0%
    \relax \Roman{#1}%
  \else%
    \ifnum #1<0%
     -\expandafter\@slowromancap\romannumeral\number-#1@%
    \else%
      0%
    \fi%
  \fi%
\else\@Roman\@tempcnta%
\fi%
}{% is not numeric
\PackageError{romanbar}{%
Argument of \string\Romannum\space is not a number}{%
The command \string\Romannum\space converts an Arabic number into an%
 upper-case Roman one,\MessageBreak%
but the used argument of \string\Romannum\space is not an Arabic number%
 but\MessageBreak%
`#1`.}%
#1%
}%
}

%    \end{macrocode}
%
%    \begin{macrocode}
%</package>
%    \end{macrocode}
%
% \newpage
%
% \section{Installation}
%
% \subsection{Downloads\label{ss:Downloads}}
%
% Everything should be available on \CTAN{}, \url{http://www.ctan.org/tex-archive/},
% but may need additional packages themselves.\\
%
% \DescribeMacro{romanbar.dtx}
% For unpacking the |romanbar.dtx| file and constructing the documentation
% it is required:
% \begin{description}
% \item[-] \TeX{} Format \LaTeXe{}, \url{http://www.CTAN.org/}
%
% \item[-] document class \xpackage{ltxdoc}, 2007/11/11, v2.0u,
%           \url{http://ctan.org/pkg/ltxdoc}
%
% \item[-] package \xpackage{holtxdoc}, 2011/02/04, v0.21,
%           \url{http://ctan.org/pkg/holtxdoc}
%
% \item[-] package \xpackage{hypdoc}, 2010/03/26, v1.9,
%           \url{http://ctan.org/pkg/hypdoc}
% \end{description}
%
% \DescribeMacro{romanbar.sty}
% The \texttt{romanbar.sty} for \LaTeXe{} (i.\,e.~each document using
% the \xpackage{romanbar} package) requires:
% \begin{description}
% \item[-] \TeX Format \LaTeXe{}, \url{http://www.CTAN.org/}
% \end{description}
%
% \DescribeMacro{romanbar-example.tex}
% The \texttt{romanbar-example.tex} requires the same files as all
% documents using the \xpackage{romanbar} package and additionally:
% \begin{description}
% \item[-] class \xpackage{article}, 2007/10/19, v1.4h, from \xpackage{classes.dtx}:\\
%           \CTAN{macros/latex/base/classes.dtx}
%
% \item[-] package \xpackage{romanbar}, 2012/01/01, v1.0f,\\
%           %%\url{http://ctan.org/pkg/romanbar}\\
%           \CTAN{macros/latex/contrib/romanbar/}
%   (Well, it is the example file for this package, and because you are reading the
%    documentation for the \xpackage{romanbar} package, it can be assumed that you already
%    have some version of it -- is it the current one?)
% \end{description}
%
% \DescribeMacro{Alternative}
% As possible alternative package in section \ref{sec:Alternatives} there is listed
% \begin{description}
% \item[-] none.
% \end{description}
%
% \DescribeMacro{Oberdiek}
% \DescribeMacro{holtxdoc}
% All packages of \textsc{Heiko Oberdiek's} bundle `oberdiek'
% (especially \xpackage{holtxdoc}) are also available in a TDS compliant ZIP archive:\\
% \CTAN{install/macros/latex/contrib/oberdiek.tds.zip}.\\
% It is probably best to download and use this, because the packages in there
% are quite probably both recent and compatible among themselves.\\
%
% \DescribeMacro{hyperref}
% \noindent \xpackage{hyperref} is not included in that bundle and needs to be
% downloaded separately,\\
% \url{http://mirror.ctan.org/install/macros/latex/contrib/hyperref.tds.zip}.\\
%
% \DescribeMacro{M\"{u}nch}
% A hyperlinked list of my (other) packages can be found at
% \url{http://www.Uni-Bonn.de/~uzs5pv/LaTeX.html}.\\
%
% \pagebreak
%
% \subsection{Package, unpacking TDS}
% \paragraph{Package.} This package should become available on \CTAN{} soon:
% \begin{description}
% \item[\CTAN{macros/latex/contrib/romanbar/romanbar.dtx}]\hspace*{0.1cm} \\
%       The source file.
% \item[\CTAN{macros/latex/contrib/romanbar/romanbar.pdf}]\hspace*{0.1cm} \\
%       The documentation.
% \item[\CTAN{macros/latex/contrib/romanbar/romanbar-example.pdf}]\hspace*{0.1cm} \\
%       The compiled example file, as it should look like.
% \item[\CTAN{macros/latex/contrib/romanbar/README}]\hspace*{0.1cm} \\
%       The README file.
% \end{description}
% There is also a romanbar.tds.zip available:
% \begin{description}
% \item[\CTAN{install/macros/latex/contrib/romanbar.tds.zip}]\hspace*{0.1cm} \\
%       Everything in \xfile{TDS} compliant, compiled format.
% \end{description}
% which additionally contains\\
% \begin{tabular}{ll}
% romanbar.ins & The installation file.\\
% romanbar.drv & The driver to generate the documentation.\\
% romanbar.sty & The \xext{sty}le file.\\
% romanbar-example.tex & The example file.
% \end{tabular}
%
% \bigskip
%
% \noindent For required other packages, please see the preceding subsection.
%
% \paragraph{Unpacking.} The \xfile{.dtx} file is a self-extracting
% \docstrip{} archive. The files are extracted by running the
% \xext{.dtx} through \plainTeX{}:
% \begin{quote}
%   \verb|tex romanbar.dtx|
% \end{quote}
%
% About generating the documentation see paragraph~\ref{GenDoc} below.\\
%
% \paragraph{TDS.} Now the different files must be moved into
% the different directories in your installation TDS tree
% (also known as \xfile{texmf} tree):
% \begin{quote}
% \def\t{^^A
% \begin{tabular}{@{}>{\ttfamily}l@{ $\rightarrow$ }>{\ttfamily}l@{}}
%   romanbar.sty & tex/latex/romanbar/romanbar.sty\\
%   romanbar.pdf & doc/latex/romanbar/romanbar.pdf\\
%   romanbar-example.tex & doc/latex/romanbar/romanbar-example.tex\\
%   romanbar-example.pdf & doc/latex/romanbar/romanbar-example.pdf\\
%   romanbar.dtx & source/latex/romanbar/romanbar.dtx\\
% \end{tabular}^^A
% }^^A
% \sbox0{\t}^^A
% \ifdim\wd0>\linewidth
%   \begingroup
%     \advance\linewidth by\leftmargin
%     \advance\linewidth by\rightmargin
%   \edef\x{\endgroup
%     \def\noexpand\lw{\the\linewidth}^^A
%   }\x
%   \def\lwbox{^^A
%     \leavevmode
%     \hbox to \linewidth{^^A
%       \kern-\leftmargin\relax
%       \hss
%       \usebox0
%       \hss
%       \kern-\rightmargin\relax
%     }^^A
%   }^^A
%   \ifdim\wd0>\lw
%     \sbox0{\small\t}^^A
%     \ifdim\wd0>\linewidth
%       \ifdim\wd0>\lw
%         \sbox0{\footnotesize\t}^^A
%         \ifdim\wd0>\linewidth
%           \ifdim\wd0>\lw
%             \sbox0{\scriptsize\t}^^A
%             \ifdim\wd0>\linewidth
%               \ifdim\wd0>\lw
%                 \sbox0{\tiny\t}^^A
%                 \ifdim\wd0>\linewidth
%                   \lwbox
%                 \else
%                   \usebox0
%                 \fi
%               \else
%                 \lwbox
%               \fi
%             \else
%               \usebox0
%             \fi
%           \else
%             \lwbox
%           \fi
%         \else
%           \usebox0
%         \fi
%       \else
%         \lwbox
%       \fi
%     \else
%       \usebox0
%     \fi
%   \else
%     \lwbox
%   \fi
% \else
%   \usebox0
% \fi
% \end{quote}
% If you have a \xfile{docstrip.cfg} that configures and enables \docstrip{}'s
% \xfile{TDS} installing feature, then some files can already be in the right
% place, see the documentation of \docstrip{}.
%
% \subsection{Refresh file name databases}
%
% If your \TeX{}~distribution (\teTeX{}, \mikTeX{},\dots{}) relies on file name
% databases, you must refresh these. For example, \teTeX{} users run
% \verb|texhash| or \verb|mktexlsr|.
%
% \subsection{Some details for the interested}
%
% \paragraph{Unpacking with \LaTeX{}.}
% The \xfile{.dtx} chooses its action depending on the format:
% \begin{description}
% \item[\plainTeX:] Run \docstrip{} and extract the files.
% \item[\LaTeX:] Generate the documentation.
% \end{description}
% If you insist on using \LaTeX{} for \docstrip{} (really,
% \docstrip{} does not need \LaTeX{}), then inform the autodetect routine
% about your intention:
% \begin{quote}
%   \verb|latex \let\install=y\input{romanbar.dtx}|
% \end{quote}
% Do not forget to quote the argument according to the demands
% of your shell.
%
% \paragraph{Generating the documentation.\label{GenDoc}}
% You can use both the \xfile{.dtx} or the \xfile{.drv} to generate
% the documentation. The process can be configured by a
% configuration file \xfile{ltxdoc.cfg}. For instance, put the following
% line into this file, if you want to have A4 as paper format:
% \begin{quote}
%   \verb|\PassOptionsToClass{a4paper}{article}|
% \end{quote}
%
% \noindent An example follows how to generate the
% documentation with \pdfLaTeX{}:
%
% \begin{quote}
%\begin{verbatim}
%pdflatex romanbar.dtx
%makeindex -s gind.ist romanbar.idx
%pdflatex romanbar.dtx
%makeindex -s gind.ist romanbar.idx
%pdflatex romanbar.dtx
%\end{verbatim}
% \end{quote}
%
% \subsection{Compiling the example}
%
% The example file, \textsf{romanbar-example.tex}, can be compiled via\\
% |(pdf)(la)tex romanbar-example.tex|.
%
% \section{Acknowledgements}
%
% I would like to thank \textsc{Heiko Oberdiek} for providing
% the \xpackage{hyperref} as well as a~lot~(!) of other useful packages
% (from which I also got everything I know about creating a file in \xext{dtx}
% format, ok, say it: copying),
% Prof.~\textsc{Enrico Gregorio} (egreg) for his answer
% \url{http://tex.stackexchange.com/questions/24065/roman-numerals-formatting/24084#24084},
% the \Newsgroup{comp.text.tex} and \Newsgroup{de.comp.text.tex}
% newsgroups as well as \url{http://tex.stackexchange.com}
% for their help in all things \TeX{}.
%
% \pagebreak
%
% \phantomsection
% \begin{History}\label{History}
%   \begin{Version}{2011/07/25 v1.0a}
%     \item First code by Prof.~\textsc{Enrico Gregorio} (egreg) at
%             \url{http://tex.stackexchange.com/questions/24065/roman-numerals-formatting/24084#24084}
%   \end{Version}
%   \begin{Version}{2011/07/26 v1.0b}
%     \item Packed into a \xext{sty} file.
%     \item Removed the requirement of \eTeX{} for the package.
%   \end{Version}
%   \begin{Version}{2011/08/25 v1.0c}
%     \item Renamed \xpackage{romanbar} package.
%     \item Minor details.
%   \end{Version}
%   \begin{Version}{2011/09/16 v1.0d}
%     \item Made commands robust.
%     \item Minor details.
%   \end{Version}
%   \begin{Version}{2011/12/16 v1.0e}
%     \item Additionally accepts now Arabic numbers and converts them to Roman ones with bars.
%     \item Also non-positive numbers are now accepted.
%   \end{Version}
%   \begin{Version}{2012/01/01 v1.0f}
%     \item Added |\romannum| and |\Romannum| to convert Arabic to Roman numbers (without bars).
%     \item |\r@iseL|.
%     \item Created a \xfile{dtx}, example, and README.
%     \item Upload to \CTAN{}.
%   \end{Version}
% \end{History}
%
% \bigskip
%
% When you find a mistake or have a suggestion for an improvement of this package,
% please send an e-mail to the maintainer, thanks! (Please see BUG REPORTS in the README.)
%
% \pagebreak
%
% \PrintIndex
%
% \Finale
\endinput
%        (quote the arguments according to the demands of your shell)
%
% Documentation:
%    (a) If romanbar.drv is present:
%           (pdf)latex romanbar.drv
%           makeindex -s gind.ist romanbar.idx
%           (pdf)latex romanbar.drv
%           makeindex -s gind.ist romanbar.idx
%           (pdf)latex romanbar.drv
%    (b) Without romanbar.drv:
%           (pdf)latex romanbar.dtx
%           makeindex -s gind.ist romanbar.idx
%           (pdf)latex romanbar.dtx
%           makeindex -s gind.ist romanbar.idx
%           (pdf)latex romanbar.dtx
%
%    The class ltxdoc loads the configuration file ltxdoc.cfg
%    if available. Here you can specify further options, e.g.
%    use DIN A4 as paper format:
%       \PassOptionsToClass{a4paper}{article}
%
% Installation:
%    TDS:tex/latex/romanbar/romanbar.sty
%    TDS:doc/latex/romanbar/romanbar.pdf
%    TDS:doc/latex/romanbar/romanbar-example.tex
%    TDS:doc/latex/romanbar/romanbar-example.pdf
%    TDS:source/latex/romanbar/romanbar.dtx
%
%<*ignore>
\begingroup
  \catcode123=1 %
  \catcode125=2 %
  \def\x{LaTeX2e}%
\expandafter\endgroup
\ifcase 0\ifx\install y1\fi\expandafter
         \ifx\csname processbatchFile\endcsname\relax\else1\fi
         \ifx\fmtname\x\else 1\fi\relax
\else\csname fi\endcsname
%</ignore>
%<*install>
\input docstrip.tex
\Msg{********************************************************************}
\Msg{* Installation                                                     *}
\Msg{* Package: romanbar 2012/01/01 v1.0f Roman numbers with bars (HMM) *}
\Msg{********************************************************************}

\keepsilent
\askforoverwritefalse

\let\MetaPrefix\relax
\preamble

This is a generated file.

Project: romanbar
Version: 2012/01/01 v1.0f

Copyright (C) 2011 by
    H.-Martin M"unch <Martin dot Muench at Uni-Bonn dot de>

The usual disclaimer applies:
If it doesn't work right that's your problem.
(Nevertheless, send an e-mail to the maintainer
 when you find an error in this package.)

This work may be distributed and/or modified under the
conditions of the LaTeX Project Public License, either
version 1.3c of this license or (at your option) any later
version. This version of this license is in
   http://www.latex-project.org/lppl/lppl-1-3c.txt
and the latest version of this license is in
   http://www.latex-project.org/lppl.txt
and version 1.3c or later is part of all distributions of
LaTeX version 2005/12/01 or later.

This work has the LPPL maintenance status "maintained".

The Current Maintainer of this work is H.-Martin Muench.

This work consists of the main source file romanbar.dtx,
the README, and the derived files
   romanbar.sty, romanbar.pdf,
   romanbar.ins, romanbar.drv,
   romanbar-example.tex, romanbar-example.pdf.

\endpreamble
\let\MetaPrefix\DoubleperCent

\generate{%
  \file{romanbar.ins}{\from{romanbar.dtx}{install}}%
  \file{romanbar.drv}{\from{romanbar.dtx}{driver}}%
  \usedir{tex/latex/romanbar}%
  \file{romanbar.sty}{\from{romanbar.dtx}{package}}%
  \usedir{doc/latex/romanbar}%
  \file{romanbar-example.tex}{\from{romanbar.dtx}{example}}%
}

\catcode32=13\relax% active space
\let =\space%
\Msg{************************************************************************}
\Msg{*}
\Msg{* To finish the installation you have to move the following}
\Msg{* file into a directory searched by TeX:}
\Msg{*}
\Msg{*     romanbar.sty}
\Msg{*}
\Msg{* To produce the documentation run the file `romanbar.drv'}
\Msg{* through (pdf)LaTeX, e.g.}
\Msg{*  pdflatex romanbar.drv}
\Msg{*  makeindex -s gind.ist romanbar.idx}
\Msg{*  pdflatex romanbar.drv}
\Msg{*  makeindex -s gind.ist romanbar.idx}
\Msg{*  pdflatex romanbar.drv}
\Msg{*}
\Msg{* At least three runs are necessary e.g. to get the}
\Msg{*  references right!}
\Msg{*}
\Msg{* Happy TeXing!}
\Msg{*}
\Msg{************************************************************************}

\endbatchfile
%</install>
%<*ignore>
\fi
%</ignore>
%
% \section{The documentation driver file}
%
% The next bit of code contains the documentation driver file for
% \TeX{}, i.\,e., the file that will produce the documentation you
% are currently reading. It will be extracted from this file by the
% \texttt{docstrip} programme. That is, run \LaTeX{} on \texttt{docstrip}
% and specify the \texttt{driver} option when \texttt{docstrip}
% asks for options.
%
%    \begin{macrocode}
%<*driver>
\NeedsTeXFormat{LaTeX2e}[2011/06/27]
\ProvidesFile{romanbar.drv}%
  [2012/01/01 v1.0f Roman numbers with bars (HMM)]
\documentclass{ltxdoc}[2007/11/11]% v2.0u
\usepackage{holtxdoc}[2011/02/04]%  v0.21
%% romanbar may work with earlier versions of LaTeX2e and those
%% class and package, but this was not tested.
%% Please consider updating your LaTeX, class, and package
%% to the most recent version (if they are not already the most
%% recent version).
\hypersetup{%
 pdfsubject={Writing Roman numbers with bars (HMM)},%
 pdfkeywords={LaTeX, romanbar, roman, Roman, bars, H.-Martin Muench},%
 pdfencoding=auto,%
 pdflang={en},%
 breaklinks=true,%
 linktoc=all,%
 pdfstartview=FitH,%
 pdfpagelayout=OneColumn,%
 bookmarksnumbered=true,%
 bookmarksopen=true,%
 bookmarksopenlevel=2,%
 pdfmenubar=true,%
 pdftoolbar=true,%
 pdfwindowui=true,%
 pdfnewwindow=true%
}
\CodelineIndex
\hyphenation{docu-ment ana-lysed}
\gdef\unit#1{\mathord{\thinspace\mathrm{#1}}}%
\makeatletter
\@ifundefined{eTeX}{\gdef\eTeX{$\m@th \varepsilon $-\TeX }}{% else \relax
}
\makeatother
\begin{document}
  \DocInput{romanbar.dtx}%
\end{document}
%</driver>
%    \end{macrocode}
%
% \fi
%
% \CheckSum{179}
%
% \CharacterTable
%  {Upper-case    \A\B\C\D\E\F\G\H\I\J\K\L\M\N\O\P\Q\R\S\T\U\V\W\X\Y\Z
%   Lower-case    \a\b\c\d\e\f\g\h\i\j\k\l\m\n\o\p\q\r\s\t\u\v\w\x\y\z
%   Digits        \0\1\2\3\4\5\6\7\8\9
%   Exclamation   \!     Double quote  \"     Hash (number) \#
%   Dollar        \$     Percent       \%     Ampersand     \&
%   Acute accent  \'     Left paren    \(     Right paren   \)
%   Asterisk      \*     Plus          \+     Comma         \,
%   Minus         \-     Point         \.     Solidus       \/
%   Colon         \:     Semicolon     \;     Less than     \<
%   Equals        \=     Greater than  \>     Question mark \?
%   Commercial at \@     Left bracket  \[     Backslash     \\
%   Right bracket \]     Circumflex    \^     Underscore    \_
%   Grave accent  \`     Left brace    \{     Vertical bar  \|
%   Right brace   \}     Tilde         \~}
%
% \GetFileInfo{romanbar.drv}
%
% \begingroup
%   \def\x{\#,\$,\^,\_,\~,\ ,\&,\{,\},\%}%
%   \makeatletter
%   \@onelevel@sanitize\x
% \expandafter\endgroup
% \expandafter\DoNotIndex\expandafter{\x}
% \expandafter\DoNotIndex\expandafter{\string\ }
% \begingroup
%   \makeatletter
%     \lccode`9=32\relax
%     \lowercase{%^^A
%       \edef\x{\noexpand\DoNotIndex{\@backslashchar9}}%^^A
%     }%^^A
%   \expandafter\endgroup\x
%
% \DoNotIndex{\\,\,}
% \DoNotIndex{\documentclass,\usepackage,\ProvidesPackage,\begin,\end}
% \DoNotIndex{\MessageBreak}
% \DoNotIndex{\NeedsTeXFormat,\DoNotIndex,\verb}
% \DoNotIndex{\def,\edef,\gdef,\global}
% \DoNotIndex{\ifx,\listfiles,\mathord,\mathrm}
% \DoNotIndex{\bigskip,\space,\thinspace,\ldots}
% \DoNotIndex{\indent,\noindent,\newline,\linebreak,\pagebreak,\newpage}
% \DoNotIndex{\textbf,\textit,\textsf,\texttt,\textquotedblleft,\textquotedblright}
% \DoNotIndex{\plainTeX,\TeX,\LaTeX,\pdfLaTeX}
% \DoNotIndex{\item}
%
% \title{The \xpackage{romanbar} package}
% \date{2012/01/01 v1.0f}
% \author{H.-Martin M\"{u}nch\\\xemail{Martin.Muench at Uni-Bonn.de}}
%
% \maketitle
%
% \begin{abstract}
%  This \LaTeX{} package allows to write Roman numbers
%  (or any other text) with bars. (Additionally, commands for converting Arabic numbers
%  into Roman ones are provided and an |\ifnumeric| test function.)
% \end{abstract}
%
% \bigskip
%
% \noindent Disclaimer for web links: The author is not responsible for any contents
% referred to in this work unless he has full knowledge of illegal contents.
% If any damage occurs by the use of information presented there, only the
% author of the respective pages might be liable, not the one who has referred
% to these pages.
%
% \bigskip
%
% \noindent {\color{green} Save per page about $200\unit{ml}$ water,
% $2\unit{g}$ CO$_{2}$ and $2\unit{g}$ wood:\\
% Therefore please print only if this is really necessary.}
%
% \pagebreak
%
% \tableofcontents
%
% \pagebreak
%
% \section{Introduction}
%
% \indent This \LaTeX{} package allows to write Roman numbers
% (or any other text) with bars. (Additionally, command for converting Arabic numbers
% into Roman ones are provided and an |\ifnumeric| test function.)
%
% \bigskip
%
% \section{Usage}
%
% \indent Just load the package placing
% \begin{quote}
%   |\usepackage{romanbar}|
% \end{quote}
% \noindent in the preamble of your \LaTeXe{} source file.\\
% |\Romanbar{...}| then produces a Roman number with bars (please see the example file).
% |\ifnumeric{test}{true}{false}| tests for |test| being numeric.
% |\romannum{...}| turns an Arabic number into a lowercase Roman one, and
% |\Romannum{...}| turns an Arabic number into an uppercase Roman one.
%
% \subsection*{Options}
% \DescribeMacro{options}
% \indent The \xpackage{romanbar} package takes no options.
%
% \section{Alternatives\label{sec:Alternatives}}
%
% \begin{description}
% \item[-] the original code written by Prof. Enrico Gregorio\\
%   (\url{http://profs.sci.univr.it/~gregorio/}),\\
%   \url{http://tex.stackexchange.com/questions/24065/roman-numerals-formatting/24084#24084}
%\begin{verbatim}
%\def\barroman#1{\sbox0{#1}\dimen0=\dimexpr\wd0+1pt\relax
%  \makebox[\dimen0]{\rlap{\vrule width\dimen0 height 0.06ex depth 0.06ex}%
%    \rlap{\vrule width\dimen0 height\dimexpr\ht0+0.03ex\relax
%            depth\dimexpr-\ht0+0.09ex\relax}%
%    \kern.5pt#1\kern.5pt}}
%
%\barroman{I} \barroman{XI}
%\end{verbatim}
%
% \item[-]  Some fonts provide single characters for Roman numerals,
%   cf.~e.\,g.
% \url{http://tex.stackexchange.com/questions/38695/using-unicode-roman-numerals-in-xetex}.
% \end{description}
%
% \noindent (You programmed or found another alternative,
%  which is available at \CTAN{}?\\
%  OK, send an e-mail to me with the name, location at \CTAN{},
%  and a short notice, and I will probably include it in
%  the list above.)
%
% \newpage
%
% \section{Example}
%
%    \begin{macrocode}
%<*example>
\documentclass[british]{article}[2007/10/19]% v1.4h
%%%%%%%%%%%%%%%%%%%%%%%%%%%%%%%%%%%%%%%%%%%%%%%%%%%%%%%%%%%%%%%%%%%%%
\usepackage[extension=pdf,%
 plainpages=false,%
 pdfpagelabels=true,%
 hyperindex=false,%
 pdflang={en},%
 pdftitle={romanbar package example},%
 pdfauthor={H.-Martin Muench},%
 pdfsubject={Example for the romanbar package},%
 pdfkeywords={LaTeX, romanbar, roman, Roman, bars, H.-Martin Muench},%
 pdfview={XYZ null null 1},%
 pdfstartview={XYZ null null 1},%
 pdfpagelayout=SinglePage]{hyperref}[2011/12/04]% v6.82m
\usepackage{romanbar}[2012/01/01]% v1.0f
\gdef\unit#1{\mathord{\thinspace\mathrm{#1}}}%
\listfiles
\begin{document}
\pagenumbering{arabic}
\section*{Example for romanbar}

This example demonstrates the use of package\newline
\textsf{romanbar}, v1.0f as of 2012/01/01 (HMM).\newline
There are no options to be used.\newline

\noindent For more details please see the documentation!\newline

\noindent Save per page about $200\unit{ml}$ water,
$2\unit{g}$ CO$_{2}$ and $2\unit{g}$ wood:\newline
Therefore please print only if this is really necessary.\newline

\noindent This package provides the command \verb|\Romanbar|
to print bars below and over the following:

\begin{description}
\item[-] Roman numbers: \verb|\Romanbar{MMXII}| prints \Romanbar{MMXII}

\item[-] Arabic numbers turned into upper-case Roman numbers:\newline
           \verb|\Romanbar{2012}| prints \Romanbar{2012}

\item[-] negative Arabic numbers turned into upper-case Roman numbers
           \newline
           (although historically there were no negative Roman numbers):
           \newline
           \verb|\Romanbar{-12}| prints \Romanbar{-12}

\item[-] zero Arabic number ($0$; although historically
           there was no Roman zero):\newline
           \verb|\Romanbar{0}| prints \Romanbar{0}

\item[-] some arbitrary text:
           \verb|\Romanbar{Caesar}| prints \Romanbar{Caesar}\newline
           (with descenders: \Romanbar{AgjpqyW})


\newcounter{example}
\setcounter{example}{21}

\item[-] some counter's value:
           \verb|\Romanbar{\theexample}| prints \Romanbar{\theexample}
           \newline
           (where the value of \texttt{example} is \theexample)

\item[-] Arabic numbers, without turning them into upper-case Roman
           numbers:\newline
           \verb|\Romanbar{\relax 2012}| prints \Romanbar{\relax  2012}
\end{description}

Special care was taken for "L" (50), e.\,g. in 555/DLV: \Romanbar{555}.\\


\verb|\romannum{...}| converts an Arabic number into a lower-case Roman one,
e.\,g. \verb|\romannum{2012}| prints \romannum{2012}, and
\verb|\Romannum{...}| converts an Arabic number into an upper-case Roman one,
e.\,g. \verb|\Romannum{2012}| prints \Romannum{2012}.

\end{document}
%</example>
%    \end{macrocode}
%
% \StopEventually{}
%
% \newpage
%
% \section{The implementation}
%
% We start off by checking that we are loading into \LaTeXe{} and
% announcing the name and version of this package.
%
%    \begin{macrocode}
%<*package>
%    \end{macrocode}
%
%    \begin{macrocode}
\NeedsTeXFormat{LaTeX2e}[2011/06/27]
\ProvidesPackage{romanbar}[2012/01/01 v1.0f
              Roman numbers with bars (HMM)]

%    \end{macrocode}
%
% A short description of the \xpackage{romanbar} package:
%
%    \begin{macrocode}
%% Allows to write Roman numbers (or any other text) with bars
%% and to convert Arabic numbers into Roman ones.
%    \end{macrocode}
%
% A last information for the user:
%
%    \begin{macrocode}
%% romanbar may work with earlier versions of LaTeX,
%% but this was not tested. Please consider updating
%% your LaTeX to the most recent version
%% (if it is not already the most recent version).

%    \end{macrocode}
%
% \indent See subsection~\ref{ss:Downloads} about how to get it.\\
%
% There are no options to be processed.
%
% A command to gobble a possible minus sign is needed and
% |\gobbleminus| from \url{http://www.tex.ac.uk/cgi-bin/texfaq2html?label=isitanum}
% is used.
%
%    \begin{macrocode}
\def\gobbleminus#1{\ifx-#1\else#1\fi}
%% from http://www.tex.ac.uk/cgi-bin/texfaq2html?label=isitanum

%    \end{macrocode}
%
% We want to test whether the argument passed to |\Romanbar| is numeric
% and therefore define |\ifnumeric{test}{true}{false}|,
% where |test| is to be analysed, |true| is the code to be executed,
% if |test| is numeric, and |false| is the code to be executed,
% if |test| is not numeric. This is done similar to\\
% \url{http://tex.stackexchange.com/a/17119}.
%
%    \begin{macrocode}
\newcommand{\ifnumeric}[3]{%
%% similar to http://tex.stackexchange.com/a/17119
\sbox\z@{\@tempcnta=0\gobbleminus#1\relax}%
\ifdim\wd0>\z@\relax#3% is not numeric
\else#2% is numeric
\fi%
}

%    \end{macrocode}
%
% We define |\r@iseL| to raise any L (50), otherwise a good deal
% of the horizontal part of the letter would be swallowed by the lower bar
% under the \textquotedblleft{}number\textquotedblright{}.
%
%    \begin{macrocode}
\def\r@iseL#1{\ifx @#1% then terminate
\else%
\if L#1\raisebox{0.05ex}{L}\else #1\fi%
\expandafter\r@iseL%
\fi%
}

%    \end{macrocode}
%
% \pagebreak
%
% We define the |\Romanbar| command.
%
%    \begin{macrocode}
\DeclareRobustCommand{\Romanbar}[1]{%
\ifnumeric{#1}{% is numeric
%    \end{macrocode}
%
% If its parameter is numeric, the temporary counter |\@tempcnta| is set to the value.
%
%    \begin{macrocode}
\@tempcnta=#1\relax%
%    \end{macrocode}
%
% Normally there are no non-positive Roman numbers, therefore we do some trick
% similar to |\XRoman| from the \url{http://ctan.org/pkg/pageslts} package.
%
%    \begin{macrocode}
%% similar to \XRoman from the http://ctan.org/pkg/pageslts package
\ifnum\@tempcnta<1%
  \ifnum #1>0%
    \relax \Roman@bar{\Roman{#1}}%
  \else%
    \ifnum #1<0%
     -\Roman@bar{\expandafter\@slowromancap\romannumeral\number-#1@%
     }%
    \else%
      \Roman@bar{0}%
    \fi%
  \fi%
\else\Roman@bar{\@Roman\@tempcnta}%
\fi%
}{% is not numeric
\Roman@bar{#1}%
}%
}

%    \end{macrocode}
%
% Whether it is numeric or not, in the end |\Roman@bar| is applied to it.
%
%    \begin{macrocode}
\DeclareRobustCommand{\Roman@bar}[1]{% #1 is in Roman, i.e. MMXII
%% (or any other text, "Caesar" would work, too).
%% similar to code of  by Prof. Enrico Gregorio (egreg) at
%% http://tex.stackexchange.com/questions/24065/roman-numerals-formatting/24084#24084
\@bsphack%
\edef\romanbartmp{#1}%
%% height:
\sbox0{\textsf{\romanbartmp}}%
%% top line:
%% + top of top line:
\@tempdima=\ht0%
\advance\@tempdima+0.05ex%0.03
\dimen1=\@tempdima%
%% + bottom of top line:
\@tempdima=-\ht0%
\advance\@tempdima+0.05ex%0.07
\dimen2=\@tempdima%
%% width:
\sbox0{\textsf{\expandafter\r@iseL\romanbartmp @}}%
\@tempdima=\wd0%
\advance\@tempdima+1pt%
\dimen0=\@tempdima%
\@esphack%
\makebox[\dimen0]{%
\rlap{\vrule width\dimen0 height\dimen1 depth\dimen2}% top    line
\rlap{\vrule width\dimen0 height 0.06ex depth 0.03ex}% bottom line
\kern0.5pt\textsf{\expandafter\r@iseL\romanbartmp @}\kern0.5pt}%
}

%    \end{macrocode}
%
% While it is not needed for the bars, it is sometimes asked for a command
% to convert an Arabic number into a Roman one. |\roman{...}| can only be applied
% to a counter, i.\,e. |\roman{12}| does not work, but
%\begin{verbatim}
%\newcounter{examplecounter}
%\setcounter{examplecounter}{12}
%\roman{examplecounter}
%\end{verbatim}
% would be necessary. Further non-negative values would not work.
% |\romannum| accepts any number as argument. (If the argument is no number,
% an error message is given and the argument printed (executed)
%  without applying anything to it.)
%
%    \begin{macrocode}
\DeclareRobustCommand{\romannum}[1]{%
\ifnumeric{#1}{% is numeric
\@tempcnta=#1\relax%
%% similar to \xroman from the http://ctan.org/pkg/pageslts package
\ifnum\@tempcnta<1%
  \ifnum #1>0%
    \relax \roman{#1}%
  \else%
    \ifnum #1<0%
     -\romannumeral\number-#1@%
    \else%
      0%
    \fi%
  \fi%
\else\@roman\@tempcnta%
\fi%
}{% is not numeric
\PackageError{romanbar}{%
Argument of \string\romannum\space is not a number}{%
The command \string\romannum\space converts an Arabic number into a%
 lower-case Roman one,\MessageBreak%
but the used argument of \string\romannum\space is not an Arabic number%
 but\MessageBreak%
`#1`,\MessageBreak%
which will now be printed unchanged.}%
#1%
}%
}

%    \end{macrocode}
%
% The same for upper-case Roman numbers, |\Romannum| instead of |\Roman|:
%
%    \begin{macrocode}
\DeclareRobustCommand{\Romannum}[1]{%
\ifnumeric{#1}{% is numeric
\@tempcnta=#1\relax%
%% similar to \XRoman from the http://ctan.org/pkg/pageslts package
\ifnum\@tempcnta<1%
  \ifnum #1>0%
    \relax \Roman{#1}%
  \else%
    \ifnum #1<0%
     -\expandafter\@slowromancap\romannumeral\number-#1@%
    \else%
      0%
    \fi%
  \fi%
\else\@Roman\@tempcnta%
\fi%
}{% is not numeric
\PackageError{romanbar}{%
Argument of \string\Romannum\space is not a number}{%
The command \string\Romannum\space converts an Arabic number into an%
 upper-case Roman one,\MessageBreak%
but the used argument of \string\Romannum\space is not an Arabic number%
 but\MessageBreak%
`#1`.}%
#1%
}%
}

%    \end{macrocode}
%
%    \begin{macrocode}
%</package>
%    \end{macrocode}
%
% \newpage
%
% \section{Installation}
%
% \subsection{Downloads\label{ss:Downloads}}
%
% Everything should be available on \CTAN{}, \url{http://www.ctan.org/tex-archive/},
% but may need additional packages themselves.\\
%
% \DescribeMacro{romanbar.dtx}
% For unpacking the |romanbar.dtx| file and constructing the documentation
% it is required:
% \begin{description}
% \item[-] \TeX{} Format \LaTeXe{}, \url{http://www.CTAN.org/}
%
% \item[-] document class \xpackage{ltxdoc}, 2007/11/11, v2.0u,
%           \url{http://ctan.org/pkg/ltxdoc}
%
% \item[-] package \xpackage{holtxdoc}, 2011/02/04, v0.21,
%           \url{http://ctan.org/pkg/holtxdoc}
%
% \item[-] package \xpackage{hypdoc}, 2010/03/26, v1.9,
%           \url{http://ctan.org/pkg/hypdoc}
% \end{description}
%
% \DescribeMacro{romanbar.sty}
% The \texttt{romanbar.sty} for \LaTeXe{} (i.\,e.~each document using
% the \xpackage{romanbar} package) requires:
% \begin{description}
% \item[-] \TeX Format \LaTeXe{}, \url{http://www.CTAN.org/}
% \end{description}
%
% \DescribeMacro{romanbar-example.tex}
% The \texttt{romanbar-example.tex} requires the same files as all
% documents using the \xpackage{romanbar} package and additionally:
% \begin{description}
% \item[-] class \xpackage{article}, 2007/10/19, v1.4h, from \xpackage{classes.dtx}:\\
%           \CTAN{macros/latex/base/classes.dtx}
%
% \item[-] package \xpackage{romanbar}, 2012/01/01, v1.0f,\\
%           %%\url{http://ctan.org/pkg/romanbar}\\
%           \CTAN{macros/latex/contrib/romanbar/}
%   (Well, it is the example file for this package, and because you are reading the
%    documentation for the \xpackage{romanbar} package, it can be assumed that you already
%    have some version of it -- is it the current one?)
% \end{description}
%
% \DescribeMacro{Alternative}
% As possible alternative package in section \ref{sec:Alternatives} there is listed
% \begin{description}
% \item[-] none.
% \end{description}
%
% \DescribeMacro{Oberdiek}
% \DescribeMacro{holtxdoc}
% All packages of \textsc{Heiko Oberdiek's} bundle `oberdiek'
% (especially \xpackage{holtxdoc}) are also available in a TDS compliant ZIP archive:\\
% \CTAN{install/macros/latex/contrib/oberdiek.tds.zip}.\\
% It is probably best to download and use this, because the packages in there
% are quite probably both recent and compatible among themselves.\\
%
% \DescribeMacro{hyperref}
% \noindent \xpackage{hyperref} is not included in that bundle and needs to be
% downloaded separately,\\
% \url{http://mirror.ctan.org/install/macros/latex/contrib/hyperref.tds.zip}.\\
%
% \DescribeMacro{M\"{u}nch}
% A hyperlinked list of my (other) packages can be found at
% \url{http://www.Uni-Bonn.de/~uzs5pv/LaTeX.html}.\\
%
% \pagebreak
%
% \subsection{Package, unpacking TDS}
% \paragraph{Package.} This package should become available on \CTAN{} soon:
% \begin{description}
% \item[\CTAN{macros/latex/contrib/romanbar/romanbar.dtx}]\hspace*{0.1cm} \\
%       The source file.
% \item[\CTAN{macros/latex/contrib/romanbar/romanbar.pdf}]\hspace*{0.1cm} \\
%       The documentation.
% \item[\CTAN{macros/latex/contrib/romanbar/romanbar-example.pdf}]\hspace*{0.1cm} \\
%       The compiled example file, as it should look like.
% \item[\CTAN{macros/latex/contrib/romanbar/README}]\hspace*{0.1cm} \\
%       The README file.
% \end{description}
% There is also a romanbar.tds.zip available:
% \begin{description}
% \item[\CTAN{install/macros/latex/contrib/romanbar.tds.zip}]\hspace*{0.1cm} \\
%       Everything in \xfile{TDS} compliant, compiled format.
% \end{description}
% which additionally contains\\
% \begin{tabular}{ll}
% romanbar.ins & The installation file.\\
% romanbar.drv & The driver to generate the documentation.\\
% romanbar.sty & The \xext{sty}le file.\\
% romanbar-example.tex & The example file.
% \end{tabular}
%
% \bigskip
%
% \noindent For required other packages, please see the preceding subsection.
%
% \paragraph{Unpacking.} The \xfile{.dtx} file is a self-extracting
% \docstrip{} archive. The files are extracted by running the
% \xext{.dtx} through \plainTeX{}:
% \begin{quote}
%   \verb|tex romanbar.dtx|
% \end{quote}
%
% About generating the documentation see paragraph~\ref{GenDoc} below.\\
%
% \paragraph{TDS.} Now the different files must be moved into
% the different directories in your installation TDS tree
% (also known as \xfile{texmf} tree):
% \begin{quote}
% \def\t{^^A
% \begin{tabular}{@{}>{\ttfamily}l@{ $\rightarrow$ }>{\ttfamily}l@{}}
%   romanbar.sty & tex/latex/romanbar/romanbar.sty\\
%   romanbar.pdf & doc/latex/romanbar/romanbar.pdf\\
%   romanbar-example.tex & doc/latex/romanbar/romanbar-example.tex\\
%   romanbar-example.pdf & doc/latex/romanbar/romanbar-example.pdf\\
%   romanbar.dtx & source/latex/romanbar/romanbar.dtx\\
% \end{tabular}^^A
% }^^A
% \sbox0{\t}^^A
% \ifdim\wd0>\linewidth
%   \begingroup
%     \advance\linewidth by\leftmargin
%     \advance\linewidth by\rightmargin
%   \edef\x{\endgroup
%     \def\noexpand\lw{\the\linewidth}^^A
%   }\x
%   \def\lwbox{^^A
%     \leavevmode
%     \hbox to \linewidth{^^A
%       \kern-\leftmargin\relax
%       \hss
%       \usebox0
%       \hss
%       \kern-\rightmargin\relax
%     }^^A
%   }^^A
%   \ifdim\wd0>\lw
%     \sbox0{\small\t}^^A
%     \ifdim\wd0>\linewidth
%       \ifdim\wd0>\lw
%         \sbox0{\footnotesize\t}^^A
%         \ifdim\wd0>\linewidth
%           \ifdim\wd0>\lw
%             \sbox0{\scriptsize\t}^^A
%             \ifdim\wd0>\linewidth
%               \ifdim\wd0>\lw
%                 \sbox0{\tiny\t}^^A
%                 \ifdim\wd0>\linewidth
%                   \lwbox
%                 \else
%                   \usebox0
%                 \fi
%               \else
%                 \lwbox
%               \fi
%             \else
%               \usebox0
%             \fi
%           \else
%             \lwbox
%           \fi
%         \else
%           \usebox0
%         \fi
%       \else
%         \lwbox
%       \fi
%     \else
%       \usebox0
%     \fi
%   \else
%     \lwbox
%   \fi
% \else
%   \usebox0
% \fi
% \end{quote}
% If you have a \xfile{docstrip.cfg} that configures and enables \docstrip{}'s
% \xfile{TDS} installing feature, then some files can already be in the right
% place, see the documentation of \docstrip{}.
%
% \subsection{Refresh file name databases}
%
% If your \TeX{}~distribution (\teTeX{}, \mikTeX{},\dots{}) relies on file name
% databases, you must refresh these. For example, \teTeX{} users run
% \verb|texhash| or \verb|mktexlsr|.
%
% \subsection{Some details for the interested}
%
% \paragraph{Unpacking with \LaTeX{}.}
% The \xfile{.dtx} chooses its action depending on the format:
% \begin{description}
% \item[\plainTeX:] Run \docstrip{} and extract the files.
% \item[\LaTeX:] Generate the documentation.
% \end{description}
% If you insist on using \LaTeX{} for \docstrip{} (really,
% \docstrip{} does not need \LaTeX{}), then inform the autodetect routine
% about your intention:
% \begin{quote}
%   \verb|latex \let\install=y% \iffalse meta-comment
%
% File: romanbar.dtx
% Version: 2012/01/01 v1.0f
%
% Copyright (C) 2011 by
%    H.-Martin M"unch <Martin dot Muench at Uni-Bonn dot de>
%
% The original code for the bars was written by Prof. Enrico Gregorio
% (http://profs.sci.univr.it/~gregorio/), and published at
% http://tex.stackexchange.com/questions/24065/roman-numerals-formatting/24084#24084
% I (H.-Martin Münch) submitted this package to CTAN
% (after diverse updates of the code and writing a documentation).
%
% This work may be distributed and/or modified under the
% conditions of the LaTeX Project Public License, either
% version 1.3c of this license or (at your option) any later
% version. This version of this license is in
%    http://www.latex-project.org/lppl/lppl-1-3c.txt
% and the latest version of this license is in
%    http://www.latex-project.org/lppl.txt
% and version 1.3c or later is part of all distributions of
% LaTeX version 2005/12/01 or later.
%
% This work has the LPPL maintenance status "maintained".
%
% The Current Maintainer of this work is H.-Martin Muench.
%
% This work consists of the main source file romanbar.dtx,
% the README, and the derived files
%    romanbar.sty, romanbar.pdf,
%    romanbar.ins, romanbar.drv,
%    romanbar-example.tex, romanbar-example.pdf.
%
% Distribution:
%    CTAN:macros/latex/contrib/romanbar/romanbar.dtx
%    CTAN:macros/latex/contrib/romanbar/romanbar.pdf
%    CTAN:install/macros/latex/contrib/romanbar.tds.zip
%
% Cf. http://ctan.org/pkg/romanbar and
% http://www.tex.ac.uk/tex-archive/help/Catalogue/entries/romanbar.html
% (It might take some time for the entries to be created.)
%
% Unpacking:
%    (a) If romanbar.ins is present:
%           tex romanbar.ins
%    (b) Without romanbar.ins:
%           tex romanbar.dtx
%    (c) If you insist on using LaTeX
%           latex \let\install=y\input{romanbar.dtx}
%        (quote the arguments according to the demands of your shell)
%
% Documentation:
%    (a) If romanbar.drv is present:
%           (pdf)latex romanbar.drv
%           makeindex -s gind.ist romanbar.idx
%           (pdf)latex romanbar.drv
%           makeindex -s gind.ist romanbar.idx
%           (pdf)latex romanbar.drv
%    (b) Without romanbar.drv:
%           (pdf)latex romanbar.dtx
%           makeindex -s gind.ist romanbar.idx
%           (pdf)latex romanbar.dtx
%           makeindex -s gind.ist romanbar.idx
%           (pdf)latex romanbar.dtx
%
%    The class ltxdoc loads the configuration file ltxdoc.cfg
%    if available. Here you can specify further options, e.g.
%    use DIN A4 as paper format:
%       \PassOptionsToClass{a4paper}{article}
%
% Installation:
%    TDS:tex/latex/romanbar/romanbar.sty
%    TDS:doc/latex/romanbar/romanbar.pdf
%    TDS:doc/latex/romanbar/romanbar-example.tex
%    TDS:doc/latex/romanbar/romanbar-example.pdf
%    TDS:source/latex/romanbar/romanbar.dtx
%
%<*ignore>
\begingroup
  \catcode123=1 %
  \catcode125=2 %
  \def\x{LaTeX2e}%
\expandafter\endgroup
\ifcase 0\ifx\install y1\fi\expandafter
         \ifx\csname processbatchFile\endcsname\relax\else1\fi
         \ifx\fmtname\x\else 1\fi\relax
\else\csname fi\endcsname
%</ignore>
%<*install>
\input docstrip.tex
\Msg{********************************************************************}
\Msg{* Installation                                                     *}
\Msg{* Package: romanbar 2012/01/01 v1.0f Roman numbers with bars (HMM) *}
\Msg{********************************************************************}

\keepsilent
\askforoverwritefalse

\let\MetaPrefix\relax
\preamble

This is a generated file.

Project: romanbar
Version: 2012/01/01 v1.0f

Copyright (C) 2011 by
    H.-Martin M"unch <Martin dot Muench at Uni-Bonn dot de>

The usual disclaimer applies:
If it doesn't work right that's your problem.
(Nevertheless, send an e-mail to the maintainer
 when you find an error in this package.)

This work may be distributed and/or modified under the
conditions of the LaTeX Project Public License, either
version 1.3c of this license or (at your option) any later
version. This version of this license is in
   http://www.latex-project.org/lppl/lppl-1-3c.txt
and the latest version of this license is in
   http://www.latex-project.org/lppl.txt
and version 1.3c or later is part of all distributions of
LaTeX version 2005/12/01 or later.

This work has the LPPL maintenance status "maintained".

The Current Maintainer of this work is H.-Martin Muench.

This work consists of the main source file romanbar.dtx,
the README, and the derived files
   romanbar.sty, romanbar.pdf,
   romanbar.ins, romanbar.drv,
   romanbar-example.tex, romanbar-example.pdf.

\endpreamble
\let\MetaPrefix\DoubleperCent

\generate{%
  \file{romanbar.ins}{\from{romanbar.dtx}{install}}%
  \file{romanbar.drv}{\from{romanbar.dtx}{driver}}%
  \usedir{tex/latex/romanbar}%
  \file{romanbar.sty}{\from{romanbar.dtx}{package}}%
  \usedir{doc/latex/romanbar}%
  \file{romanbar-example.tex}{\from{romanbar.dtx}{example}}%
}

\catcode32=13\relax% active space
\let =\space%
\Msg{************************************************************************}
\Msg{*}
\Msg{* To finish the installation you have to move the following}
\Msg{* file into a directory searched by TeX:}
\Msg{*}
\Msg{*     romanbar.sty}
\Msg{*}
\Msg{* To produce the documentation run the file `romanbar.drv'}
\Msg{* through (pdf)LaTeX, e.g.}
\Msg{*  pdflatex romanbar.drv}
\Msg{*  makeindex -s gind.ist romanbar.idx}
\Msg{*  pdflatex romanbar.drv}
\Msg{*  makeindex -s gind.ist romanbar.idx}
\Msg{*  pdflatex romanbar.drv}
\Msg{*}
\Msg{* At least three runs are necessary e.g. to get the}
\Msg{*  references right!}
\Msg{*}
\Msg{* Happy TeXing!}
\Msg{*}
\Msg{************************************************************************}

\endbatchfile
%</install>
%<*ignore>
\fi
%</ignore>
%
% \section{The documentation driver file}
%
% The next bit of code contains the documentation driver file for
% \TeX{}, i.\,e., the file that will produce the documentation you
% are currently reading. It will be extracted from this file by the
% \texttt{docstrip} programme. That is, run \LaTeX{} on \texttt{docstrip}
% and specify the \texttt{driver} option when \texttt{docstrip}
% asks for options.
%
%    \begin{macrocode}
%<*driver>
\NeedsTeXFormat{LaTeX2e}[2011/06/27]
\ProvidesFile{romanbar.drv}%
  [2012/01/01 v1.0f Roman numbers with bars (HMM)]
\documentclass{ltxdoc}[2007/11/11]% v2.0u
\usepackage{holtxdoc}[2011/02/04]%  v0.21
%% romanbar may work with earlier versions of LaTeX2e and those
%% class and package, but this was not tested.
%% Please consider updating your LaTeX, class, and package
%% to the most recent version (if they are not already the most
%% recent version).
\hypersetup{%
 pdfsubject={Writing Roman numbers with bars (HMM)},%
 pdfkeywords={LaTeX, romanbar, roman, Roman, bars, H.-Martin Muench},%
 pdfencoding=auto,%
 pdflang={en},%
 breaklinks=true,%
 linktoc=all,%
 pdfstartview=FitH,%
 pdfpagelayout=OneColumn,%
 bookmarksnumbered=true,%
 bookmarksopen=true,%
 bookmarksopenlevel=2,%
 pdfmenubar=true,%
 pdftoolbar=true,%
 pdfwindowui=true,%
 pdfnewwindow=true%
}
\CodelineIndex
\hyphenation{docu-ment ana-lysed}
\gdef\unit#1{\mathord{\thinspace\mathrm{#1}}}%
\makeatletter
\@ifundefined{eTeX}{\gdef\eTeX{$\m@th \varepsilon $-\TeX }}{% else \relax
}
\makeatother
\begin{document}
  \DocInput{romanbar.dtx}%
\end{document}
%</driver>
%    \end{macrocode}
%
% \fi
%
% \CheckSum{179}
%
% \CharacterTable
%  {Upper-case    \A\B\C\D\E\F\G\H\I\J\K\L\M\N\O\P\Q\R\S\T\U\V\W\X\Y\Z
%   Lower-case    \a\b\c\d\e\f\g\h\i\j\k\l\m\n\o\p\q\r\s\t\u\v\w\x\y\z
%   Digits        \0\1\2\3\4\5\6\7\8\9
%   Exclamation   \!     Double quote  \"     Hash (number) \#
%   Dollar        \$     Percent       \%     Ampersand     \&
%   Acute accent  \'     Left paren    \(     Right paren   \)
%   Asterisk      \*     Plus          \+     Comma         \,
%   Minus         \-     Point         \.     Solidus       \/
%   Colon         \:     Semicolon     \;     Less than     \<
%   Equals        \=     Greater than  \>     Question mark \?
%   Commercial at \@     Left bracket  \[     Backslash     \\
%   Right bracket \]     Circumflex    \^     Underscore    \_
%   Grave accent  \`     Left brace    \{     Vertical bar  \|
%   Right brace   \}     Tilde         \~}
%
% \GetFileInfo{romanbar.drv}
%
% \begingroup
%   \def\x{\#,\$,\^,\_,\~,\ ,\&,\{,\},\%}%
%   \makeatletter
%   \@onelevel@sanitize\x
% \expandafter\endgroup
% \expandafter\DoNotIndex\expandafter{\x}
% \expandafter\DoNotIndex\expandafter{\string\ }
% \begingroup
%   \makeatletter
%     \lccode`9=32\relax
%     \lowercase{%^^A
%       \edef\x{\noexpand\DoNotIndex{\@backslashchar9}}%^^A
%     }%^^A
%   \expandafter\endgroup\x
%
% \DoNotIndex{\\,\,}
% \DoNotIndex{\documentclass,\usepackage,\ProvidesPackage,\begin,\end}
% \DoNotIndex{\MessageBreak}
% \DoNotIndex{\NeedsTeXFormat,\DoNotIndex,\verb}
% \DoNotIndex{\def,\edef,\gdef,\global}
% \DoNotIndex{\ifx,\listfiles,\mathord,\mathrm}
% \DoNotIndex{\bigskip,\space,\thinspace,\ldots}
% \DoNotIndex{\indent,\noindent,\newline,\linebreak,\pagebreak,\newpage}
% \DoNotIndex{\textbf,\textit,\textsf,\texttt,\textquotedblleft,\textquotedblright}
% \DoNotIndex{\plainTeX,\TeX,\LaTeX,\pdfLaTeX}
% \DoNotIndex{\item}
%
% \title{The \xpackage{romanbar} package}
% \date{2012/01/01 v1.0f}
% \author{H.-Martin M\"{u}nch\\\xemail{Martin.Muench at Uni-Bonn.de}}
%
% \maketitle
%
% \begin{abstract}
%  This \LaTeX{} package allows to write Roman numbers
%  (or any other text) with bars. (Additionally, commands for converting Arabic numbers
%  into Roman ones are provided and an |\ifnumeric| test function.)
% \end{abstract}
%
% \bigskip
%
% \noindent Disclaimer for web links: The author is not responsible for any contents
% referred to in this work unless he has full knowledge of illegal contents.
% If any damage occurs by the use of information presented there, only the
% author of the respective pages might be liable, not the one who has referred
% to these pages.
%
% \bigskip
%
% \noindent {\color{green} Save per page about $200\unit{ml}$ water,
% $2\unit{g}$ CO$_{2}$ and $2\unit{g}$ wood:\\
% Therefore please print only if this is really necessary.}
%
% \pagebreak
%
% \tableofcontents
%
% \pagebreak
%
% \section{Introduction}
%
% \indent This \LaTeX{} package allows to write Roman numbers
% (or any other text) with bars. (Additionally, command for converting Arabic numbers
% into Roman ones are provided and an |\ifnumeric| test function.)
%
% \bigskip
%
% \section{Usage}
%
% \indent Just load the package placing
% \begin{quote}
%   |\usepackage{romanbar}|
% \end{quote}
% \noindent in the preamble of your \LaTeXe{} source file.\\
% |\Romanbar{...}| then produces a Roman number with bars (please see the example file).
% |\ifnumeric{test}{true}{false}| tests for |test| being numeric.
% |\romannum{...}| turns an Arabic number into a lowercase Roman one, and
% |\Romannum{...}| turns an Arabic number into an uppercase Roman one.
%
% \subsection*{Options}
% \DescribeMacro{options}
% \indent The \xpackage{romanbar} package takes no options.
%
% \section{Alternatives\label{sec:Alternatives}}
%
% \begin{description}
% \item[-] the original code written by Prof. Enrico Gregorio\\
%   (\url{http://profs.sci.univr.it/~gregorio/}),\\
%   \url{http://tex.stackexchange.com/questions/24065/roman-numerals-formatting/24084#24084}
%\begin{verbatim}
%\def\barroman#1{\sbox0{#1}\dimen0=\dimexpr\wd0+1pt\relax
%  \makebox[\dimen0]{\rlap{\vrule width\dimen0 height 0.06ex depth 0.06ex}%
%    \rlap{\vrule width\dimen0 height\dimexpr\ht0+0.03ex\relax
%            depth\dimexpr-\ht0+0.09ex\relax}%
%    \kern.5pt#1\kern.5pt}}
%
%\barroman{I} \barroman{XI}
%\end{verbatim}
%
% \item[-]  Some fonts provide single characters for Roman numerals,
%   cf.~e.\,g.
% \url{http://tex.stackexchange.com/questions/38695/using-unicode-roman-numerals-in-xetex}.
% \end{description}
%
% \noindent (You programmed or found another alternative,
%  which is available at \CTAN{}?\\
%  OK, send an e-mail to me with the name, location at \CTAN{},
%  and a short notice, and I will probably include it in
%  the list above.)
%
% \newpage
%
% \section{Example}
%
%    \begin{macrocode}
%<*example>
\documentclass[british]{article}[2007/10/19]% v1.4h
%%%%%%%%%%%%%%%%%%%%%%%%%%%%%%%%%%%%%%%%%%%%%%%%%%%%%%%%%%%%%%%%%%%%%
\usepackage[extension=pdf,%
 plainpages=false,%
 pdfpagelabels=true,%
 hyperindex=false,%
 pdflang={en},%
 pdftitle={romanbar package example},%
 pdfauthor={H.-Martin Muench},%
 pdfsubject={Example for the romanbar package},%
 pdfkeywords={LaTeX, romanbar, roman, Roman, bars, H.-Martin Muench},%
 pdfview={XYZ null null 1},%
 pdfstartview={XYZ null null 1},%
 pdfpagelayout=SinglePage]{hyperref}[2011/12/04]% v6.82m
\usepackage{romanbar}[2012/01/01]% v1.0f
\gdef\unit#1{\mathord{\thinspace\mathrm{#1}}}%
\listfiles
\begin{document}
\pagenumbering{arabic}
\section*{Example for romanbar}

This example demonstrates the use of package\newline
\textsf{romanbar}, v1.0f as of 2012/01/01 (HMM).\newline
There are no options to be used.\newline

\noindent For more details please see the documentation!\newline

\noindent Save per page about $200\unit{ml}$ water,
$2\unit{g}$ CO$_{2}$ and $2\unit{g}$ wood:\newline
Therefore please print only if this is really necessary.\newline

\noindent This package provides the command \verb|\Romanbar|
to print bars below and over the following:

\begin{description}
\item[-] Roman numbers: \verb|\Romanbar{MMXII}| prints \Romanbar{MMXII}

\item[-] Arabic numbers turned into upper-case Roman numbers:\newline
           \verb|\Romanbar{2012}| prints \Romanbar{2012}

\item[-] negative Arabic numbers turned into upper-case Roman numbers
           \newline
           (although historically there were no negative Roman numbers):
           \newline
           \verb|\Romanbar{-12}| prints \Romanbar{-12}

\item[-] zero Arabic number ($0$; although historically
           there was no Roman zero):\newline
           \verb|\Romanbar{0}| prints \Romanbar{0}

\item[-] some arbitrary text:
           \verb|\Romanbar{Caesar}| prints \Romanbar{Caesar}\newline
           (with descenders: \Romanbar{AgjpqyW})


\newcounter{example}
\setcounter{example}{21}

\item[-] some counter's value:
           \verb|\Romanbar{\theexample}| prints \Romanbar{\theexample}
           \newline
           (where the value of \texttt{example} is \theexample)

\item[-] Arabic numbers, without turning them into upper-case Roman
           numbers:\newline
           \verb|\Romanbar{\relax 2012}| prints \Romanbar{\relax  2012}
\end{description}

Special care was taken for "L" (50), e.\,g. in 555/DLV: \Romanbar{555}.\\


\verb|\romannum{...}| converts an Arabic number into a lower-case Roman one,
e.\,g. \verb|\romannum{2012}| prints \romannum{2012}, and
\verb|\Romannum{...}| converts an Arabic number into an upper-case Roman one,
e.\,g. \verb|\Romannum{2012}| prints \Romannum{2012}.

\end{document}
%</example>
%    \end{macrocode}
%
% \StopEventually{}
%
% \newpage
%
% \section{The implementation}
%
% We start off by checking that we are loading into \LaTeXe{} and
% announcing the name and version of this package.
%
%    \begin{macrocode}
%<*package>
%    \end{macrocode}
%
%    \begin{macrocode}
\NeedsTeXFormat{LaTeX2e}[2011/06/27]
\ProvidesPackage{romanbar}[2012/01/01 v1.0f
              Roman numbers with bars (HMM)]

%    \end{macrocode}
%
% A short description of the \xpackage{romanbar} package:
%
%    \begin{macrocode}
%% Allows to write Roman numbers (or any other text) with bars
%% and to convert Arabic numbers into Roman ones.
%    \end{macrocode}
%
% A last information for the user:
%
%    \begin{macrocode}
%% romanbar may work with earlier versions of LaTeX,
%% but this was not tested. Please consider updating
%% your LaTeX to the most recent version
%% (if it is not already the most recent version).

%    \end{macrocode}
%
% \indent See subsection~\ref{ss:Downloads} about how to get it.\\
%
% There are no options to be processed.
%
% A command to gobble a possible minus sign is needed and
% |\gobbleminus| from \url{http://www.tex.ac.uk/cgi-bin/texfaq2html?label=isitanum}
% is used.
%
%    \begin{macrocode}
\def\gobbleminus#1{\ifx-#1\else#1\fi}
%% from http://www.tex.ac.uk/cgi-bin/texfaq2html?label=isitanum

%    \end{macrocode}
%
% We want to test whether the argument passed to |\Romanbar| is numeric
% and therefore define |\ifnumeric{test}{true}{false}|,
% where |test| is to be analysed, |true| is the code to be executed,
% if |test| is numeric, and |false| is the code to be executed,
% if |test| is not numeric. This is done similar to\\
% \url{http://tex.stackexchange.com/a/17119}.
%
%    \begin{macrocode}
\newcommand{\ifnumeric}[3]{%
%% similar to http://tex.stackexchange.com/a/17119
\sbox\z@{\@tempcnta=0\gobbleminus#1\relax}%
\ifdim\wd0>\z@\relax#3% is not numeric
\else#2% is numeric
\fi%
}

%    \end{macrocode}
%
% We define |\r@iseL| to raise any L (50), otherwise a good deal
% of the horizontal part of the letter would be swallowed by the lower bar
% under the \textquotedblleft{}number\textquotedblright{}.
%
%    \begin{macrocode}
\def\r@iseL#1{\ifx @#1% then terminate
\else%
\if L#1\raisebox{0.05ex}{L}\else #1\fi%
\expandafter\r@iseL%
\fi%
}

%    \end{macrocode}
%
% \pagebreak
%
% We define the |\Romanbar| command.
%
%    \begin{macrocode}
\DeclareRobustCommand{\Romanbar}[1]{%
\ifnumeric{#1}{% is numeric
%    \end{macrocode}
%
% If its parameter is numeric, the temporary counter |\@tempcnta| is set to the value.
%
%    \begin{macrocode}
\@tempcnta=#1\relax%
%    \end{macrocode}
%
% Normally there are no non-positive Roman numbers, therefore we do some trick
% similar to |\XRoman| from the \url{http://ctan.org/pkg/pageslts} package.
%
%    \begin{macrocode}
%% similar to \XRoman from the http://ctan.org/pkg/pageslts package
\ifnum\@tempcnta<1%
  \ifnum #1>0%
    \relax \Roman@bar{\Roman{#1}}%
  \else%
    \ifnum #1<0%
     -\Roman@bar{\expandafter\@slowromancap\romannumeral\number-#1@%
     }%
    \else%
      \Roman@bar{0}%
    \fi%
  \fi%
\else\Roman@bar{\@Roman\@tempcnta}%
\fi%
}{% is not numeric
\Roman@bar{#1}%
}%
}

%    \end{macrocode}
%
% Whether it is numeric or not, in the end |\Roman@bar| is applied to it.
%
%    \begin{macrocode}
\DeclareRobustCommand{\Roman@bar}[1]{% #1 is in Roman, i.e. MMXII
%% (or any other text, "Caesar" would work, too).
%% similar to code of  by Prof. Enrico Gregorio (egreg) at
%% http://tex.stackexchange.com/questions/24065/roman-numerals-formatting/24084#24084
\@bsphack%
\edef\romanbartmp{#1}%
%% height:
\sbox0{\textsf{\romanbartmp}}%
%% top line:
%% + top of top line:
\@tempdima=\ht0%
\advance\@tempdima+0.05ex%0.03
\dimen1=\@tempdima%
%% + bottom of top line:
\@tempdima=-\ht0%
\advance\@tempdima+0.05ex%0.07
\dimen2=\@tempdima%
%% width:
\sbox0{\textsf{\expandafter\r@iseL\romanbartmp @}}%
\@tempdima=\wd0%
\advance\@tempdima+1pt%
\dimen0=\@tempdima%
\@esphack%
\makebox[\dimen0]{%
\rlap{\vrule width\dimen0 height\dimen1 depth\dimen2}% top    line
\rlap{\vrule width\dimen0 height 0.06ex depth 0.03ex}% bottom line
\kern0.5pt\textsf{\expandafter\r@iseL\romanbartmp @}\kern0.5pt}%
}

%    \end{macrocode}
%
% While it is not needed for the bars, it is sometimes asked for a command
% to convert an Arabic number into a Roman one. |\roman{...}| can only be applied
% to a counter, i.\,e. |\roman{12}| does not work, but
%\begin{verbatim}
%\newcounter{examplecounter}
%\setcounter{examplecounter}{12}
%\roman{examplecounter}
%\end{verbatim}
% would be necessary. Further non-negative values would not work.
% |\romannum| accepts any number as argument. (If the argument is no number,
% an error message is given and the argument printed (executed)
%  without applying anything to it.)
%
%    \begin{macrocode}
\DeclareRobustCommand{\romannum}[1]{%
\ifnumeric{#1}{% is numeric
\@tempcnta=#1\relax%
%% similar to \xroman from the http://ctan.org/pkg/pageslts package
\ifnum\@tempcnta<1%
  \ifnum #1>0%
    \relax \roman{#1}%
  \else%
    \ifnum #1<0%
     -\romannumeral\number-#1@%
    \else%
      0%
    \fi%
  \fi%
\else\@roman\@tempcnta%
\fi%
}{% is not numeric
\PackageError{romanbar}{%
Argument of \string\romannum\space is not a number}{%
The command \string\romannum\space converts an Arabic number into a%
 lower-case Roman one,\MessageBreak%
but the used argument of \string\romannum\space is not an Arabic number%
 but\MessageBreak%
`#1`,\MessageBreak%
which will now be printed unchanged.}%
#1%
}%
}

%    \end{macrocode}
%
% The same for upper-case Roman numbers, |\Romannum| instead of |\Roman|:
%
%    \begin{macrocode}
\DeclareRobustCommand{\Romannum}[1]{%
\ifnumeric{#1}{% is numeric
\@tempcnta=#1\relax%
%% similar to \XRoman from the http://ctan.org/pkg/pageslts package
\ifnum\@tempcnta<1%
  \ifnum #1>0%
    \relax \Roman{#1}%
  \else%
    \ifnum #1<0%
     -\expandafter\@slowromancap\romannumeral\number-#1@%
    \else%
      0%
    \fi%
  \fi%
\else\@Roman\@tempcnta%
\fi%
}{% is not numeric
\PackageError{romanbar}{%
Argument of \string\Romannum\space is not a number}{%
The command \string\Romannum\space converts an Arabic number into an%
 upper-case Roman one,\MessageBreak%
but the used argument of \string\Romannum\space is not an Arabic number%
 but\MessageBreak%
`#1`.}%
#1%
}%
}

%    \end{macrocode}
%
%    \begin{macrocode}
%</package>
%    \end{macrocode}
%
% \newpage
%
% \section{Installation}
%
% \subsection{Downloads\label{ss:Downloads}}
%
% Everything should be available on \CTAN{}, \url{http://www.ctan.org/tex-archive/},
% but may need additional packages themselves.\\
%
% \DescribeMacro{romanbar.dtx}
% For unpacking the |romanbar.dtx| file and constructing the documentation
% it is required:
% \begin{description}
% \item[-] \TeX{} Format \LaTeXe{}, \url{http://www.CTAN.org/}
%
% \item[-] document class \xpackage{ltxdoc}, 2007/11/11, v2.0u,
%           \url{http://ctan.org/pkg/ltxdoc}
%
% \item[-] package \xpackage{holtxdoc}, 2011/02/04, v0.21,
%           \url{http://ctan.org/pkg/holtxdoc}
%
% \item[-] package \xpackage{hypdoc}, 2010/03/26, v1.9,
%           \url{http://ctan.org/pkg/hypdoc}
% \end{description}
%
% \DescribeMacro{romanbar.sty}
% The \texttt{romanbar.sty} for \LaTeXe{} (i.\,e.~each document using
% the \xpackage{romanbar} package) requires:
% \begin{description}
% \item[-] \TeX Format \LaTeXe{}, \url{http://www.CTAN.org/}
% \end{description}
%
% \DescribeMacro{romanbar-example.tex}
% The \texttt{romanbar-example.tex} requires the same files as all
% documents using the \xpackage{romanbar} package and additionally:
% \begin{description}
% \item[-] class \xpackage{article}, 2007/10/19, v1.4h, from \xpackage{classes.dtx}:\\
%           \CTAN{macros/latex/base/classes.dtx}
%
% \item[-] package \xpackage{romanbar}, 2012/01/01, v1.0f,\\
%           %%\url{http://ctan.org/pkg/romanbar}\\
%           \CTAN{macros/latex/contrib/romanbar/}
%   (Well, it is the example file for this package, and because you are reading the
%    documentation for the \xpackage{romanbar} package, it can be assumed that you already
%    have some version of it -- is it the current one?)
% \end{description}
%
% \DescribeMacro{Alternative}
% As possible alternative package in section \ref{sec:Alternatives} there is listed
% \begin{description}
% \item[-] none.
% \end{description}
%
% \DescribeMacro{Oberdiek}
% \DescribeMacro{holtxdoc}
% All packages of \textsc{Heiko Oberdiek's} bundle `oberdiek'
% (especially \xpackage{holtxdoc}) are also available in a TDS compliant ZIP archive:\\
% \CTAN{install/macros/latex/contrib/oberdiek.tds.zip}.\\
% It is probably best to download and use this, because the packages in there
% are quite probably both recent and compatible among themselves.\\
%
% \DescribeMacro{hyperref}
% \noindent \xpackage{hyperref} is not included in that bundle and needs to be
% downloaded separately,\\
% \url{http://mirror.ctan.org/install/macros/latex/contrib/hyperref.tds.zip}.\\
%
% \DescribeMacro{M\"{u}nch}
% A hyperlinked list of my (other) packages can be found at
% \url{http://www.Uni-Bonn.de/~uzs5pv/LaTeX.html}.\\
%
% \pagebreak
%
% \subsection{Package, unpacking TDS}
% \paragraph{Package.} This package should become available on \CTAN{} soon:
% \begin{description}
% \item[\CTAN{macros/latex/contrib/romanbar/romanbar.dtx}]\hspace*{0.1cm} \\
%       The source file.
% \item[\CTAN{macros/latex/contrib/romanbar/romanbar.pdf}]\hspace*{0.1cm} \\
%       The documentation.
% \item[\CTAN{macros/latex/contrib/romanbar/romanbar-example.pdf}]\hspace*{0.1cm} \\
%       The compiled example file, as it should look like.
% \item[\CTAN{macros/latex/contrib/romanbar/README}]\hspace*{0.1cm} \\
%       The README file.
% \end{description}
% There is also a romanbar.tds.zip available:
% \begin{description}
% \item[\CTAN{install/macros/latex/contrib/romanbar.tds.zip}]\hspace*{0.1cm} \\
%       Everything in \xfile{TDS} compliant, compiled format.
% \end{description}
% which additionally contains\\
% \begin{tabular}{ll}
% romanbar.ins & The installation file.\\
% romanbar.drv & The driver to generate the documentation.\\
% romanbar.sty & The \xext{sty}le file.\\
% romanbar-example.tex & The example file.
% \end{tabular}
%
% \bigskip
%
% \noindent For required other packages, please see the preceding subsection.
%
% \paragraph{Unpacking.} The \xfile{.dtx} file is a self-extracting
% \docstrip{} archive. The files are extracted by running the
% \xext{.dtx} through \plainTeX{}:
% \begin{quote}
%   \verb|tex romanbar.dtx|
% \end{quote}
%
% About generating the documentation see paragraph~\ref{GenDoc} below.\\
%
% \paragraph{TDS.} Now the different files must be moved into
% the different directories in your installation TDS tree
% (also known as \xfile{texmf} tree):
% \begin{quote}
% \def\t{^^A
% \begin{tabular}{@{}>{\ttfamily}l@{ $\rightarrow$ }>{\ttfamily}l@{}}
%   romanbar.sty & tex/latex/romanbar/romanbar.sty\\
%   romanbar.pdf & doc/latex/romanbar/romanbar.pdf\\
%   romanbar-example.tex & doc/latex/romanbar/romanbar-example.tex\\
%   romanbar-example.pdf & doc/latex/romanbar/romanbar-example.pdf\\
%   romanbar.dtx & source/latex/romanbar/romanbar.dtx\\
% \end{tabular}^^A
% }^^A
% \sbox0{\t}^^A
% \ifdim\wd0>\linewidth
%   \begingroup
%     \advance\linewidth by\leftmargin
%     \advance\linewidth by\rightmargin
%   \edef\x{\endgroup
%     \def\noexpand\lw{\the\linewidth}^^A
%   }\x
%   \def\lwbox{^^A
%     \leavevmode
%     \hbox to \linewidth{^^A
%       \kern-\leftmargin\relax
%       \hss
%       \usebox0
%       \hss
%       \kern-\rightmargin\relax
%     }^^A
%   }^^A
%   \ifdim\wd0>\lw
%     \sbox0{\small\t}^^A
%     \ifdim\wd0>\linewidth
%       \ifdim\wd0>\lw
%         \sbox0{\footnotesize\t}^^A
%         \ifdim\wd0>\linewidth
%           \ifdim\wd0>\lw
%             \sbox0{\scriptsize\t}^^A
%             \ifdim\wd0>\linewidth
%               \ifdim\wd0>\lw
%                 \sbox0{\tiny\t}^^A
%                 \ifdim\wd0>\linewidth
%                   \lwbox
%                 \else
%                   \usebox0
%                 \fi
%               \else
%                 \lwbox
%               \fi
%             \else
%               \usebox0
%             \fi
%           \else
%             \lwbox
%           \fi
%         \else
%           \usebox0
%         \fi
%       \else
%         \lwbox
%       \fi
%     \else
%       \usebox0
%     \fi
%   \else
%     \lwbox
%   \fi
% \else
%   \usebox0
% \fi
% \end{quote}
% If you have a \xfile{docstrip.cfg} that configures and enables \docstrip{}'s
% \xfile{TDS} installing feature, then some files can already be in the right
% place, see the documentation of \docstrip{}.
%
% \subsection{Refresh file name databases}
%
% If your \TeX{}~distribution (\teTeX{}, \mikTeX{},\dots{}) relies on file name
% databases, you must refresh these. For example, \teTeX{} users run
% \verb|texhash| or \verb|mktexlsr|.
%
% \subsection{Some details for the interested}
%
% \paragraph{Unpacking with \LaTeX{}.}
% The \xfile{.dtx} chooses its action depending on the format:
% \begin{description}
% \item[\plainTeX:] Run \docstrip{} and extract the files.
% \item[\LaTeX:] Generate the documentation.
% \end{description}
% If you insist on using \LaTeX{} for \docstrip{} (really,
% \docstrip{} does not need \LaTeX{}), then inform the autodetect routine
% about your intention:
% \begin{quote}
%   \verb|latex \let\install=y\input{romanbar.dtx}|
% \end{quote}
% Do not forget to quote the argument according to the demands
% of your shell.
%
% \paragraph{Generating the documentation.\label{GenDoc}}
% You can use both the \xfile{.dtx} or the \xfile{.drv} to generate
% the documentation. The process can be configured by a
% configuration file \xfile{ltxdoc.cfg}. For instance, put the following
% line into this file, if you want to have A4 as paper format:
% \begin{quote}
%   \verb|\PassOptionsToClass{a4paper}{article}|
% \end{quote}
%
% \noindent An example follows how to generate the
% documentation with \pdfLaTeX{}:
%
% \begin{quote}
%\begin{verbatim}
%pdflatex romanbar.dtx
%makeindex -s gind.ist romanbar.idx
%pdflatex romanbar.dtx
%makeindex -s gind.ist romanbar.idx
%pdflatex romanbar.dtx
%\end{verbatim}
% \end{quote}
%
% \subsection{Compiling the example}
%
% The example file, \textsf{romanbar-example.tex}, can be compiled via\\
% |(pdf)(la)tex romanbar-example.tex|.
%
% \section{Acknowledgements}
%
% I would like to thank \textsc{Heiko Oberdiek} for providing
% the \xpackage{hyperref} as well as a~lot~(!) of other useful packages
% (from which I also got everything I know about creating a file in \xext{dtx}
% format, ok, say it: copying),
% Prof.~\textsc{Enrico Gregorio} (egreg) for his answer
% \url{http://tex.stackexchange.com/questions/24065/roman-numerals-formatting/24084#24084},
% the \Newsgroup{comp.text.tex} and \Newsgroup{de.comp.text.tex}
% newsgroups as well as \url{http://tex.stackexchange.com}
% for their help in all things \TeX{}.
%
% \pagebreak
%
% \phantomsection
% \begin{History}\label{History}
%   \begin{Version}{2011/07/25 v1.0a}
%     \item First code by Prof.~\textsc{Enrico Gregorio} (egreg) at
%             \url{http://tex.stackexchange.com/questions/24065/roman-numerals-formatting/24084#24084}
%   \end{Version}
%   \begin{Version}{2011/07/26 v1.0b}
%     \item Packed into a \xext{sty} file.
%     \item Removed the requirement of \eTeX{} for the package.
%   \end{Version}
%   \begin{Version}{2011/08/25 v1.0c}
%     \item Renamed \xpackage{romanbar} package.
%     \item Minor details.
%   \end{Version}
%   \begin{Version}{2011/09/16 v1.0d}
%     \item Made commands robust.
%     \item Minor details.
%   \end{Version}
%   \begin{Version}{2011/12/16 v1.0e}
%     \item Additionally accepts now Arabic numbers and converts them to Roman ones with bars.
%     \item Also non-positive numbers are now accepted.
%   \end{Version}
%   \begin{Version}{2012/01/01 v1.0f}
%     \item Added |\romannum| and |\Romannum| to convert Arabic to Roman numbers (without bars).
%     \item |\r@iseL|.
%     \item Created a \xfile{dtx}, example, and README.
%     \item Upload to \CTAN{}.
%   \end{Version}
% \end{History}
%
% \bigskip
%
% When you find a mistake or have a suggestion for an improvement of this package,
% please send an e-mail to the maintainer, thanks! (Please see BUG REPORTS in the README.)
%
% \pagebreak
%
% \PrintIndex
%
% \Finale
\endinput|
% \end{quote}
% Do not forget to quote the argument according to the demands
% of your shell.
%
% \paragraph{Generating the documentation.\label{GenDoc}}
% You can use both the \xfile{.dtx} or the \xfile{.drv} to generate
% the documentation. The process can be configured by a
% configuration file \xfile{ltxdoc.cfg}. For instance, put the following
% line into this file, if you want to have A4 as paper format:
% \begin{quote}
%   \verb|\PassOptionsToClass{a4paper}{article}|
% \end{quote}
%
% \noindent An example follows how to generate the
% documentation with \pdfLaTeX{}:
%
% \begin{quote}
%\begin{verbatim}
%pdflatex romanbar.dtx
%makeindex -s gind.ist romanbar.idx
%pdflatex romanbar.dtx
%makeindex -s gind.ist romanbar.idx
%pdflatex romanbar.dtx
%\end{verbatim}
% \end{quote}
%
% \subsection{Compiling the example}
%
% The example file, \textsf{romanbar-example.tex}, can be compiled via\\
% |(pdf)(la)tex romanbar-example.tex|.
%
% \section{Acknowledgements}
%
% I would like to thank \textsc{Heiko Oberdiek} for providing
% the \xpackage{hyperref} as well as a~lot~(!) of other useful packages
% (from which I also got everything I know about creating a file in \xext{dtx}
% format, ok, say it: copying),
% Prof.~\textsc{Enrico Gregorio} (egreg) for his answer
% \url{http://tex.stackexchange.com/questions/24065/roman-numerals-formatting/24084#24084},
% the \Newsgroup{comp.text.tex} and \Newsgroup{de.comp.text.tex}
% newsgroups as well as \url{http://tex.stackexchange.com}
% for their help in all things \TeX{}.
%
% \pagebreak
%
% \phantomsection
% \begin{History}\label{History}
%   \begin{Version}{2011/07/25 v1.0a}
%     \item First code by Prof.~\textsc{Enrico Gregorio} (egreg) at
%             \url{http://tex.stackexchange.com/questions/24065/roman-numerals-formatting/24084#24084}
%   \end{Version}
%   \begin{Version}{2011/07/26 v1.0b}
%     \item Packed into a \xext{sty} file.
%     \item Removed the requirement of \eTeX{} for the package.
%   \end{Version}
%   \begin{Version}{2011/08/25 v1.0c}
%     \item Renamed \xpackage{romanbar} package.
%     \item Minor details.
%   \end{Version}
%   \begin{Version}{2011/09/16 v1.0d}
%     \item Made commands robust.
%     \item Minor details.
%   \end{Version}
%   \begin{Version}{2011/12/16 v1.0e}
%     \item Additionally accepts now Arabic numbers and converts them to Roman ones with bars.
%     \item Also non-positive numbers are now accepted.
%   \end{Version}
%   \begin{Version}{2012/01/01 v1.0f}
%     \item Added |\romannum| and |\Romannum| to convert Arabic to Roman numbers (without bars).
%     \item |\r@iseL|.
%     \item Created a \xfile{dtx}, example, and README.
%     \item Upload to \CTAN{}.
%   \end{Version}
% \end{History}
%
% \bigskip
%
% When you find a mistake or have a suggestion for an improvement of this package,
% please send an e-mail to the maintainer, thanks! (Please see BUG REPORTS in the README.)
%
% \pagebreak
%
% \PrintIndex
%
% \Finale
\endinput|
% \end{quote}
% Do not forget to quote the argument according to the demands
% of your shell.
%
% \paragraph{Generating the documentation.\label{GenDoc}}
% You can use both the \xfile{.dtx} or the \xfile{.drv} to generate
% the documentation. The process can be configured by a
% configuration file \xfile{ltxdoc.cfg}. For instance, put the following
% line into this file, if you want to have A4 as paper format:
% \begin{quote}
%   \verb|\PassOptionsToClass{a4paper}{article}|
% \end{quote}
%
% \noindent An example follows how to generate the
% documentation with \pdfLaTeX{}:
%
% \begin{quote}
%\begin{verbatim}
%pdflatex romanbar.dtx
%makeindex -s gind.ist romanbar.idx
%pdflatex romanbar.dtx
%makeindex -s gind.ist romanbar.idx
%pdflatex romanbar.dtx
%\end{verbatim}
% \end{quote}
%
% \subsection{Compiling the example}
%
% The example file, \textsf{romanbar-example.tex}, can be compiled via\\
% |(pdf)(la)tex romanbar-example.tex|.
%
% \section{Acknowledgements}
%
% I would like to thank \textsc{Heiko Oberdiek} for providing
% the \xpackage{hyperref} as well as a~lot~(!) of other useful packages
% (from which I also got everything I know about creating a file in \xext{dtx}
% format, ok, say it: copying),
% Prof.~\textsc{Enrico Gregorio} (egreg) for his answer
% \url{http://tex.stackexchange.com/questions/24065/roman-numerals-formatting/24084#24084},
% the \Newsgroup{comp.text.tex} and \Newsgroup{de.comp.text.tex}
% newsgroups as well as \url{http://tex.stackexchange.com}
% for their help in all things \TeX{}.
%
% \pagebreak
%
% \phantomsection
% \begin{History}\label{History}
%   \begin{Version}{2011/07/25 v1.0a}
%     \item First code by Prof.~\textsc{Enrico Gregorio} (egreg) at
%             \url{http://tex.stackexchange.com/questions/24065/roman-numerals-formatting/24084#24084}
%   \end{Version}
%   \begin{Version}{2011/07/26 v1.0b}
%     \item Packed into a \xext{sty} file.
%     \item Removed the requirement of \eTeX{} for the package.
%   \end{Version}
%   \begin{Version}{2011/08/25 v1.0c}
%     \item Renamed \xpackage{romanbar} package.
%     \item Minor details.
%   \end{Version}
%   \begin{Version}{2011/09/16 v1.0d}
%     \item Made commands robust.
%     \item Minor details.
%   \end{Version}
%   \begin{Version}{2011/12/16 v1.0e}
%     \item Additionally accepts now Arabic numbers and converts them to Roman ones with bars.
%     \item Also non-positive numbers are now accepted.
%   \end{Version}
%   \begin{Version}{2012/01/01 v1.0f}
%     \item Added |\romannum| and |\Romannum| to convert Arabic to Roman numbers (without bars).
%     \item |\r@iseL|.
%     \item Created a \xfile{dtx}, example, and README.
%     \item Upload to \CTAN{}.
%   \end{Version}
% \end{History}
%
% \bigskip
%
% When you find a mistake or have a suggestion for an improvement of this package,
% please send an e-mail to the maintainer, thanks! (Please see BUG REPORTS in the README.)
%
% \pagebreak
%
% \PrintIndex
%
% \Finale
\endinput|
% \end{quote}
% Do not forget to quote the argument according to the demands
% of your shell.
%
% \paragraph{Generating the documentation.\label{GenDoc}}
% You can use both the \xfile{.dtx} or the \xfile{.drv} to generate
% the documentation. The process can be configured by a
% configuration file \xfile{ltxdoc.cfg}. For instance, put the following
% line into this file, if you want to have A4 as paper format:
% \begin{quote}
%   \verb|\PassOptionsToClass{a4paper}{article}|
% \end{quote}
%
% \noindent An example follows how to generate the
% documentation with \pdfLaTeX{}:
%
% \begin{quote}
%\begin{verbatim}
%pdflatex romanbar.dtx
%makeindex -s gind.ist romanbar.idx
%pdflatex romanbar.dtx
%makeindex -s gind.ist romanbar.idx
%pdflatex romanbar.dtx
%\end{verbatim}
% \end{quote}
%
% \subsection{Compiling the example}
%
% The example file, \textsf{romanbar-example.tex}, can be compiled via\\
% |(pdf)(la)tex romanbar-example.tex|.
%
% \section{Acknowledgements}
%
% I would like to thank \textsc{Heiko Oberdiek} for providing
% the \xpackage{hyperref} as well as a~lot~(!) of other useful packages
% (from which I also got everything I know about creating a file in \xext{dtx}
% format, ok, say it: copying),
% Prof.~\textsc{Enrico Gregorio} (egreg) for his answer
% \url{http://tex.stackexchange.com/questions/24065/roman-numerals-formatting/24084#24084},
% the \Newsgroup{comp.text.tex} and \Newsgroup{de.comp.text.tex}
% newsgroups as well as \url{http://tex.stackexchange.com}
% for their help in all things \TeX{}.
%
% \pagebreak
%
% \phantomsection
% \begin{History}\label{History}
%   \begin{Version}{2011/07/25 v1.0a}
%     \item First code by Prof.~\textsc{Enrico Gregorio} (egreg) at
%             \url{http://tex.stackexchange.com/questions/24065/roman-numerals-formatting/24084#24084}
%   \end{Version}
%   \begin{Version}{2011/07/26 v1.0b}
%     \item Packed into a \xext{sty} file.
%     \item Removed the requirement of \eTeX{} for the package.
%   \end{Version}
%   \begin{Version}{2011/08/25 v1.0c}
%     \item Renamed \xpackage{romanbar} package.
%     \item Minor details.
%   \end{Version}
%   \begin{Version}{2011/09/16 v1.0d}
%     \item Made commands robust.
%     \item Minor details.
%   \end{Version}
%   \begin{Version}{2011/12/16 v1.0e}
%     \item Additionally accepts now Arabic numbers and converts them to Roman ones with bars.
%     \item Also non-positive numbers are now accepted.
%   \end{Version}
%   \begin{Version}{2012/01/01 v1.0f}
%     \item Added |\romannum| and |\Romannum| to convert Arabic to Roman numbers (without bars).
%     \item |\r@iseL|.
%     \item Created a \xfile{dtx}, example, and README.
%     \item Upload to \CTAN{}.
%   \end{Version}
% \end{History}
%
% \bigskip
%
% When you find a mistake or have a suggestion for an improvement of this package,
% please send an e-mail to the maintainer, thanks! (Please see BUG REPORTS in the README.)
%
% \pagebreak
%
% \PrintIndex
%
% \Finale
\endinput