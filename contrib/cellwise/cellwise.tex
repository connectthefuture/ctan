
\input cellwise.sty
\cellwise
\def\jailcmds{\ifnum \the\jails=1 \rm \else\fi}
\def\cellblockcmds{\ifnum \the\cellblocks=1 \sl \else\fi}
\def\cellfloorcmds{\ifnum \the\cellfloors=1 \it \else\fi}
\def\cellcmds{\ifnum\the\cellfloors=1 \ifnum \the\cells=3 \tt  \else  \fi\fi}
\def\cellno{\the\jails.\the\cellblocks.\the\cellfloors.\the\cells}
\parindent=0pt
This is a pure \TeX{} macro. The macro enables you to define jails (= tables),
cellblocks(= tablesegments),cellfloors (=tablelines) and cells (= single
cells) and to pass commands to them.Columns are represented by a series of
cells(= all cells having the No. 3 make a column). Includes a simple notes function to place notes in a cell. Examples
below. The code has not more than 38 lines, I think there is no need to
comment it. \vskip\baselineskip

\vbox{
Example 1: The command $\backslash jailcmds \{\backslash ifnum \backslash
the \backslash jails=2  \backslash  rm  \backslash  else  \backslash  fi\}$
tells \TeX{} to type the whole jail (=table) rm as long as no other
command is given at a lower level.}
\vskip\baselineskip
\vbox{Example 2: The command $\backslash cellblockcmds \{\backslash
ifnum \backslash the \backslash cellblocks=2  \backslash sl  \backslash  else  \backslash  fi\}$
tells \TeX{} to type the whole cellblock (=tablesegment) slanted as long as no other
command is given at a lower level.}
\vskip\baselineskip

\vbox{Example 3: The command $\backslash cellfloorcmds \{\backslash ifnum \backslash
the \backslash cellfloors=2  \backslash it  \backslash  else  \backslash  fi\}$
tells \TeX{} to type the whole cellfloor (=tableline) italic as long as no
other command is given at a lower level.}
\vbox{Example 4: The command $\backslash cellcmds \{\backslash ifnum \backslash
the \backslash cells=2  \backslash  tt  \backslash  else  \backslash  fi\}$
tells \TeX{} to type the second cell in every jail/block/floor with boldface.}
Commands maybe combined by the $\backslash if \backslash fi$ routines.
Next feature will be, to let a text flow through the jails, blocks, floors and
cells, but this is not worked out yet. The command will be defined
as $\backslash$ cellchain.
\cellrag{\raggedright}
\def\cellnoshow{}
\jail{\jailname{\centerline{Example 5: JAILHOUSE ROCK}}% 
\cellblock{%
\cellfloor{%
\cell{Spider Murphy\cellnote{Tenor Saxophone}}
\cell{Little Joe \cellnote{Slide Trombone}}
\cell{Escaped \cellnote{none}}
}%
\cellfloor{%
\cell{Purple Gang \cellnote{Rhythm Section}}
\cell{Escaped \cellnote{none}}
\cell{Escaped \cellnote{none}}
}%
\cellfloor{%
\cell{No.47}
\cell{No.3}
\cell{Escaped \cellnote{none}}
}%
\cellfloor{%
\cell{Shifty Henry \cellnote{Wants to make a break}}
\cell{Bugs\cellnote{Says:``Nix Nix''}}
\cell{Escaped \cellnote{none}}
}%
}%
}%

\cellblock{
\cellfloor{\cell{}}
\cellfloor{
\cell{}\cell{}
}}
{\thecellnotes[Instruments]}%
\bye
