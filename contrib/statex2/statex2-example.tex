\documentclass[dvipsnames,usenames]{report}
%\documentclass[dvipsnames,usenames,autobold]{report}
\usepackage{statex2}
\usepackage{shortvrb}
\MakeShortVerb{@}
% Examples
\begin{document}

Many accents have been re-defined

@ c \c{c} \pi \cpi@ $$ c \c{c} \pi \cpi$$ %upright constants like the speed of light and 3.14159...

@int \e{\im x} \d{x}@ $$\int \e{\im x} \d{x}$$ %\d{x}; also note new commands \e and \im

@\^{\beta_1}=b_1@ $$\^{\beta_1}=b_1$$

@\=x=\frac{1}{n}\sum x_i@ $$\=x=\frac{1}{n}\sum x_i$$  %also, \b{x}, but see \ol{x} below

@\b{x} = \frac{1}{n} \wrap[()]{x_1 +\.+ x_n}@ $$\b{x} = \frac{1}{n} \wrap[()]{x_1 +\.+ x_n}$$

Sometimes overline is better:  @\b{x} \vs \ol{x}@ $$\b{x} \vs \ol{x}$$

And, underlines are nice too: @\ul{x}@ $$\ul{x}$$

Derivatives and partial derivatives:

@\deriv{x}{x^2+y^2}@ $$\deriv{x}{x^2+y^2}$$
@\pderiv{x}{x^2+y^2}@ $$\pderiv{x}{x^2+y^2}$$

Or, rather, in the order of @\frac@:

@\derivf{x^2+y^2}{x}@ $$\derivf{x^2+y^2}{x}$$
@\pderivf{x^2+y^2}{x}@ $$\pderivf{x^2+y^2}{x}$$

A few other nice-to-haves:

@\chisq@ $$\chisq$$

@\Gamma[n+1]=n!@ $$\Gamma[n+1]=n!$$

@\binom{n}{x}@ $$\binom{n}{x}$$ %provided by amsmath package

@\e{x}@  $$\e{x}$$

@\H_0: \mu=0 \vs \H_1: \mu \neq 0 (\neg \H_0) @ $$\H_0: \mu=0 \vs \H_1: \mu \neq 0 (\neg \H_0) $$ 

@\logit \wrap{p} = \log \wrap{\frac{p}{1-p}}@ $$\logit \wrap{p} = \log \wrap{\frac{p}{1-p}}$$
\pagebreak
Common distributions along with other features follows:

Normal Distribution

@Z ~ \N{0}{1}, \where \E{Z}=0 \and \V{Z}=1@ $$Z ~ \N{0}{1}, \where \E{Z}=0 \and \V{Z}=1$$

@\P{|Z|>z_\ha}=\alpha@ $$\P{|Z|>z_\ha}=\alpha$$

@\pN[z]{0}{1}@ $$\pN[z]{0}{1}$$ 

or, in general

@\pN[z]{\mu}{\sd^2}@ $$\pN[z]{\mu}{\sd^2}$$

Sometimes, we subscript the following operations:

@\E[z]{Z}=0, \V[z]{Z}=1, \and \P[z]{|Z|>z_\ha}=\alpha@ 
$$\E[z]{Z}=0, \V[z]{Z}=1, \and \P[z]{|Z|>z_\ha}=\alpha$$

Multivariate Normal Distribution

@\bm{X} ~ \N[p]{\bm{\mu}}{\sfsl{\Sigma}}@ $$\bm{X} ~ \N[p]{\bm{\mu}}{\sfsl{\Sigma}}$$ 
%\bm provided by the bm package 

Chi-square Distribution

@Z_i \iid \N{0}{1}, \where i=1 ,\., n@ $$Z_i \iid \N{0}{1}, \where i=1 ,\., n$$

@\chisq = \sum_i Z_i^2 ~ \Chi{n}@ $$\chisq = \sum_i Z_i^2 ~ \Chi{n}$$

@\pChi[z]{n}@ $$\pChi[z]{n}$$

t Distribution

@\frac{\N{0}{1}}{\sqrt{\frac{\Chisq{n}}{n}}} ~ \t{n}@ 
$$\frac{\N{0}{1}}{\sqrt{\frac{\Chisq{n}}{n}}} ~ \t{n}$$
\pagebreak
F Distribution
    
@X_i, Y_{\~i} \iid \N{0}{1} \where i=1 ,\., n; \~i=1 ,\., m \and \V{X_i, Y_{\~i}}=\sd_{xy}=0@ 
$$X_i, Y_{\~i} \iid \N{0}{1} \where i=1 ,\., n; \~i=1 ,\., m \and \V{X_i, Y_{\~i}}=\sd_{xy}=0$$

@\chisq_x = \sum_i X_i^2 ~ \Chi{n}@ $$\chisq_x = \sum_i X_i^2 ~ \Chi{n}$$

@\chisq_y = \sum_{\~i} Y_{\~i}^2 ~ \Chi{m}@ $$\chisq_y = \sum_{\~i} Y_{\~i}^2 ~ \Chi{m}$$

@\frac{\chisq_x}{\chisq_y} ~ \F{n}{m}@ $$\frac{\chisq_x}{\chisq_y} ~ \F{n}{m}$$

Beta Distribution

@B=\frac{\frac{n}{m}F}{1+\frac{n}{m}F} ~ \Bet{\frac{n}{2}}{\frac{m}{2}}@ 
$$B=\frac{\frac{n}{m}F}{1+\frac{n}{m}F} ~ \Bet{\frac{n}{2}}{\frac{m}{2}}$$

@\pBet{\alpha}{\beta}@ $$\pBet{\alpha}{\beta}$$ 

Gamma Distribution

@G ~ \Gam{\alpha}{\beta}@ $$G ~ \Gam{\alpha}{\beta}$$

@\pGam{\alpha}{\beta}@ $$\pGam{\alpha}{\beta}$$

Cauchy Distribution

@C ~ \Cau{\theta}{\nu}@ $$C ~ \Cau{\theta}{\nu}$$

@\pCau{\theta}{\nu}@ $$\pCau{\theta}{\nu}$$

Uniform Distribution

@X ~ \U{0, 1}@ $$X ~ \U{0, 1}$$

@\pU{0}{1}@ $$\pU{0}{1}$$

or, in general

@\pU{a}{b}@ $$\pU{a}{b}$$

Exponential Distribution

@X ~ \Exp{\lambda}@ $$X ~ \Exp{\lambda}$$

@\pExp{\lambda}@ $$\pExp{\lambda}$$

Hotelling's $T^2$ Distribution

@X ~ \Tsq{\nu_1}{\nu_2}@ $$X ~ \Tsq{\nu_1}{\nu_2}$$

Inverse Chi-square Distribution

@X ~ \IC{\nu}@ $$X ~ \IC{\nu}$$

Inverse Gamma Distribution

@X ~ \IG{\alpha}{\beta}@ $$X ~ \IG{\alpha}{\beta}$$

Pareto Distribution

@X ~ \Par{\alpha}{\beta}@ $$X ~ \Par{\alpha}{\beta}$$

@\pPar{\alpha}{\beta}@ $$\pPar{\alpha}{\beta}$$

Wishart Distribution

@\sfsl{X} ~ \W{\nu}{\sfsl{S}}@ $$\sfsl{X} ~ \W{\nu}{\sfsl{S}}$$

Inverse Wishart Distribution

@\sfsl{X} ~ \IW{\nu}{\sfsl{S^{-1}}}@ $$\sfsl{X} ~ \IW{\nu}{\sfsl{S^{-1}}}$$

Binomial Distribution

@X ~ \Bin{n}{p}@ $$X ~ \Bin{n}{p}$$

%@\pBin{n}{p}@ $$\pBin{n}{p}$$

Bernoulli Distribution

@X ~ \B{p}@ $$X ~ \B{p}$$

Beta-Binomial Distribution

@X ~ \BB{p}@ $$X ~ \BB{p}$$

%@\pBB{n}{\alpha}{\beta}@ $$\pBB{n}{\alpha}{\beta}$$

Negative-Binomial Distribution

@X ~ \NB{n}{p}@ $$X ~ \NB{n}{p}$$

Hypergeometric Distribution

@X ~ \HG{n}{M}{N}@ $$X ~ \HG{n}{M}{N}$$

Poisson Distribution

@X ~ \Poi{\mu}@ $$X ~ \Poi{\mu}$$

%@\pPoi{\mu}@ $$\pPoi{\mu}$$

Dirichlet Distribution

@\bm{X} ~ \Dir{\alpha_1 \. \alpha_k}@ $$\bm{X} ~ \Dir{\alpha_1 \. \alpha_k}$$

Multinomial Distribution

@\bm{X} ~ \M{n}{\alpha_1 \. \alpha_k}@ $$\bm{X} ~ \M{n}{\alpha_1 \. \alpha_k}$$

\pagebreak

To compute critical values for the Normal distribution, create the
NCRIT program for your TI-83 (or equivalent) calculator.  At each step, the 
calculator display is shown, followed by what you should do (\Rect\ is the 
cursor):\\
\Rect\\
\Prgm\to@NEW@\to@1:Create New@\\
@Name=@\Rect\\
NCRIT\Enter\\
@:@\Rect\\
\Prgm\to@I/O@\to@2:Prompt@\\
@:Prompt@ \Rect\\
\Alpha[A],\Alpha[T]\Enter\\
@:@\Rect\\
\Distr\to@DISTR@\to@3:invNorm(@\\
@:invNorm(@\Rect\\
1-(\Alpha[A]$\div$\Alpha[T]))\Sto\Alpha[C]\Enter\\
@:@\Rect\\
\Prgm\to@I/O@\to@3:Disp@\\
@:Disp@ \Rect\\
\Alpha[C]\Enter\\
@:@\Rect\\
\Quit\\

Suppose @A@ is $\alpha$ and @T@ is the number of tails.  To run the program:\\
\Rect\\
\Prgm\to@EXEC@\to@NCRIT@\\
@prgmNCRIT@\Rect\\
\Enter\\
@A=?@\Rect\\
0.05\Enter\\
@T=?@\Rect\\
2\Enter\\
@1.959963986@
\end{document}

