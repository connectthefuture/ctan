\documentclass{article}
\usepackage[T1]{fontenc}
\usepackage[ansinew]{inputenc}
\usepackage[english]{babel}
\usepackage{csquotes}
\usepackage[section,toc1,hyperref]{javadoc}

\hypersetup{colorlinks,citecolor=black,filecolor=black,linkcolor=black,urlcolor=black}

\author{Jolle\footnote{Comments, Help, Questions, Critics to \mbox{joerman.lieder@gmx.net}}  }
\date{\today}
\title{Documentation of TexGen-Doclet}

\begin{document}
\maketitle

\abstract{This documentation describes the use of the TexGen-Doclet. A doclet is a class derived from com.sun.javadoc.Doclet that can be used to generate a documentation with the javadoc-tool out of commented java-source-code. By default javadoc generates a HTML documentation, TexGen generates \TeX-files according to the javadoc-package. The doclet is under GNU GENERAL PUBLIC LICENSE\footnote{www.gnu.org}}

\tableofcontents

\section{Usage}
\subsection{Calling javadoc}
To use the doclet it has to be a parameter for the javadoc programm. The javadoc syntax is:

javadoc [options] [packagenames] [sourcefiles] [@files]

Possible Options are -doclet and -docletpath. For TexGen write \enquote{TeXGen} after -doclet and after -docletpath the Path to the TexGen.jar jararchive.

\subsection{Option for TexGen}
TexGen provides an own option. You can use -dest and the path to the outputfolder, where the generated files will be located. If you don't use this option, an folder named \enquote{texgendoc} will be created.

The final syntax is:

javadoc -doclet TexGen -docletpath <...texgen.jar> -dest <ausgabepath>  ...

\subsection{Using Eclipse}
With Eclipse Javadoc can be called with an plugin. You have the possibility to choose a custom doclet. Use TexGen for the docletname and the path to the jararchive for the docletpath. The next dialog provides the inputfield for extra javadoc options. Here you can add -dest <outputpath>.

\section{Known and Open Issues}
\begin{itemize}
	\item Special character in the documented text are converted not to interprete them as Tex-control character. These are:
\verb+{ } _ ^ & # [ ]+. All other special characters, that might build a tex-control-character should be avoided.
	\item Generic data types aren't supported.
	\item Duplicate Classnames (e.g. in different packages) aren't supported in linking.
\end{itemize}

\section{Source Code documentation}

\begin{jdclass}[class]{TexGen}
\begin{jdclassheader}

\jdpublic 
\jdinherits{\jdtypesimple{Object}\jdinh \jdtypesimple{Doclet}}
\JDtext{The Doclet writes the javadoc-content into Tex-files referring to the javadoc-Package.
 
 Mandatory CommandLineOption is:  -dest destinationpath for the outputfile(s)}
\JDauthor{Jolle}
\JDsince{13.05.2008}
\JDversion{1.0}
\end{jdclassheader}
\begin{jdinheritancetable} \jdInhEntry{\jdtypesimple{LanguageVersion} languageVersion(  )}{Doclet}
 \jdInhEntry{\jdtypesimple{int} optionLength( \jdtypesimple{String} )}{Doclet}
 \jdInhEntry{\jdtypesimple{boolean} start( \jdtypesimple{RootDoc} )}{Doclet}
 \jdInhEntry{\jdtypesimple{boolean} validOptions( \jdtypearray{String}{\lbrack{}\rbrack{}\lbrack{}\rbrack{}}, \jdtypesimple{DocErrorReporter} )}{Doclet}
 \jdInhEntry{\jdtypesimple{Object} clone(  )}{Object}
 \jdInhEntry{\jdtypesimple{boolean} equals( \jdtypesimple{Object} )}{Object}
 \jdInhEntry{\jdtypesimple{void} finalize(  )}{Object}
 \jdInhEntry{\jdtypesimple{Class} getClass(  )}{Object}
 \jdInhEntry{\jdtypesimple{int} hashCode(  )}{Object}
 \jdInhEntry{\jdtypesimple{void} notify(  )}{Object}
 \jdInhEntry{\jdtypesimple{void} notifyAll(  )}{Object}
 \jdInhEntry{\jdtypesimple{String} toString(  )}{Object}
 \jdInhEntry{\jdtypesimple{void} wait( \jdtypesimple{long} )}{Object}
 \jdInhEntry{\jdtypesimple{void} wait( \jdtypesimple{long}, \jdtypesimple{int} )}{Object}
 \jdInhEntry{\jdtypesimple{void} wait(  )}{Object}
\end{jdinheritancetable}
\begin{jdfield}{destpath}
\jdprivate \jdstatic 
\jdtype{\jdtypesimple{String}}
\JDtext{In this variable the destinationpath is storaged
  
 The variable is set during validation of the commandlineparameter and read to write
 the output files}
\JDauthor{Jolle}
\JDsince{version 1.0 from 13.05.2008}
\end{jdfield}
\begin{jdconstructor}
\jdpublic 
\end{jdconstructor}
\begin{jdmethod}{start}
\jdpublic \jdstatic 
\jdtype{\jdtypesimple{boolean}}
\JDpara{\jdtypesimple{RootDoc}}{root}{the parsed element with ALL information}
\JDtext{method that is called from the javadoc-programm after parsing the inputfiles}
\JDauthor{Jolle}
\JDreturn{true, if process successful; false if not}
\JDsince{version 1.0 from 13.05.2008}
\end{jdmethod}
\begin{jdmethod}{optionLength}
\jdpublic \jdstatic 
\jdtype{\jdtypesimple{int}}
\JDpara{\jdtypesimple{String}}{option}{optionname (starting with "-")}
\JDtext{Method, with number of arguments to an commandlineparameter.
 Returns 2 for -dest}
\JDauthor{Jolle}
\JDreturn{0, if option doesn't exist, otherwise a value > 1}
\JDsince{version 1.0 from 13.05.2008}
\end{jdmethod}
\begin{jdmethod}{validOptions}
\jdpublic \jdstatic 
\jdtype{\jdtypesimple{boolean}}
\JDpara{\jdtypearray{String}{\lbrack{}\rbrack{}\lbrack{}\rbrack{}}}{options}{array with all options and their arguments}
\JDpara{\jdtypesimple{DocErrorReporter}}{reporter}{an object to report the process}
\JDtext{Method validating the commandline call.}
\JDauthor{Jolle}
\JDreturn{validation result}
\JDsince{version 1.0 from 13.05.2008}
\end{jdmethod}
\end{jdclass}

\begin{jdclass}[class]{ClassWriter}
\begin{jdclassheader}

\jdpublic 
\jdinherits{\jdtypesimple{Object}}
\JDtext{Class that collection the javadoc information for one class and writes them in to the destination file.}
\JDauthor{Jolle}
\JDsince{13.05.2008}
\JDversion{1.0}
\end{jdclassheader}
\begin{jdinheritancetable} \jdInhEntry{\jdtypesimple{Object} clone(  )}{Object}
 \jdInhEntry{\jdtypesimple{boolean} equals( \jdtypesimple{Object} )}{Object}
 \jdInhEntry{\jdtypesimple{void} finalize(  )}{Object}
 \jdInhEntry{\jdtypesimple{Class} getClass(  )}{Object}
 \jdInhEntry{\jdtypesimple{int} hashCode(  )}{Object}
 \jdInhEntry{\jdtypesimple{void} notify(  )}{Object}
 \jdInhEntry{\jdtypesimple{void} notifyAll(  )}{Object}
 \jdInhEntry{\jdtypesimple{String} toString(  )}{Object}
 \jdInhEntry{\jdtypesimple{void} wait( \jdtypesimple{long} )}{Object}
 \jdInhEntry{\jdtypesimple{void} wait( \jdtypesimple{long}, \jdtypesimple{int} )}{Object}
 \jdInhEntry{\jdtypesimple{void} wait(  )}{Object}
\end{jdinheritancetable}
\begin{jdfield}{JDclass}
\jdprivate \jdfinal \jdstatic 
\jdtype{\jdtypesimple{String}}
\end{jdfield}
\begin{jdfield}{JDheader}
\jdprivate \jdfinal \jdstatic 
\jdtype{\jdtypesimple{String}}
\end{jdfield}
\begin{jdfield}{JDmethod}
\jdprivate \jdfinal \jdstatic 
\jdtype{\jdtypesimple{String}}
\end{jdfield}
\begin{jdfield}{JDconstructor}
\jdprivate \jdfinal \jdstatic 
\jdtype{\jdtypesimple{String}}
\end{jdfield}
\begin{jdfield}{JDfield}
\jdprivate \jdfinal \jdstatic 
\jdtype{\jdtypesimple{String}}
\end{jdfield}
\begin{jdfield}{JDinterfaceOpt}
\jdprivate \jdfinal \jdstatic 
\jdtype{\jdtypesimple{String}}
\end{jdfield}
\begin{jdfield}{JDclassOpt}
\jdprivate \jdfinal \jdstatic 
\jdtype{\jdtypesimple{String}}
\end{jdfield}
\begin{jdfield}{JDCpublic}
\jdprivate \jdfinal \jdstatic 
\jdtype{\jdtypesimple{String}}
\end{jdfield}
\begin{jdfield}{JDCprivate}
\jdprivate \jdfinal \jdstatic 
\jdtype{\jdtypesimple{String}}
\end{jdfield}
\begin{jdfield}{JDCprotected}
\jdprivate \jdfinal \jdstatic 
\jdtype{\jdtypesimple{String}}
\end{jdfield}
\begin{jdfield}{JDCfinal}
\jdprivate \jdfinal \jdstatic 
\jdtype{\jdtypesimple{String}}
\end{jdfield}
\begin{jdfield}{JDCstatic}
\jdprivate \jdfinal \jdstatic 
\jdtype{\jdtypesimple{String}}
\end{jdfield}
\begin{jdfield}{JDCtransient}
\jdprivate \jdfinal \jdstatic 
\jdtype{\jdtypesimple{String}}
\end{jdfield}
\begin{jdfield}{JDCvolatile}
\jdprivate \jdfinal \jdstatic 
\jdtype{\jdtypesimple{String}}
\end{jdfield}
\begin{jdfield}{JDCabstract}
\jdprivate \jdfinal \jdstatic 
\jdtype{\jdtypesimple{String}}
\end{jdfield}
\begin{jdfield}{JDCpackage}
\jdprivate \jdfinal \jdstatic 
\jdtype{\jdtypesimple{String}}
\end{jdfield}
\begin{jdfield}{JDCinherits}
\jdprivate \jdfinal \jdstatic 
\jdtype{\jdtypesimple{String}}
\end{jdfield}
\begin{jdfield}{JDCinhArrow}
\jdprivate \jdfinal \jdstatic 
\jdtype{\jdtypesimple{String}}
\end{jdfield}
\begin{jdfield}{JDCimplements}
\jdprivate \jdfinal \jdstatic 
\jdtype{\jdtypesimple{String}}
\end{jdfield}
\begin{jdfield}{JDCouterclass}
\jdprivate \jdfinal \jdstatic 
\jdtype{\jdtypesimple{String}}
\end{jdfield}
\begin{jdfield}{JDCtype}
\jdprivate \jdfinal \jdstatic 
\jdtype{\jdtypesimple{String}}
\end{jdfield}
\begin{jdfield}{JDcategory}
\jdprivate \jdfinal \jdstatic 
\jdtype{\jdtypesimple{String}}
\end{jdfield}
\begin{jdfield}{JDdeprecated}
\jdprivate \jdfinal \jdstatic 
\jdtype{\jdtypesimple{String}}
\end{jdfield}
\begin{jdfield}{JDsee}
\jdprivate \jdfinal \jdstatic 
\jdtype{\jdtypesimple{String}}
\end{jdfield}
\begin{jdfield}{JDserial}
\jdprivate \jdfinal \jdstatic 
\jdtype{\jdtypesimple{String}}
\end{jdfield}
\begin{jdfield}{JDserialData}
\jdprivate \jdfinal \jdstatic 
\jdtype{\jdtypesimple{String}}
\end{jdfield}
\begin{jdfield}{JDserialField}
\jdprivate \jdfinal \jdstatic 
\jdtype{\jdtypesimple{String}}
\end{jdfield}
\begin{jdfield}{JDsince}
\jdprivate \jdfinal \jdstatic 
\jdtype{\jdtypesimple{String}}
\end{jdfield}
\begin{jdfield}{JDtext}
\jdprivate \jdfinal \jdstatic 
\jdtype{\jdtypesimple{String}}
\end{jdfield}
\begin{jdfield}{JDversion}
\jdprivate \jdfinal \jdstatic 
\jdtype{\jdtypesimple{String}}
\end{jdfield}
\begin{jdfield}{JDreturn}
\jdprivate \jdfinal \jdstatic 
\jdtype{\jdtypesimple{String}}
\end{jdfield}
\begin{jdfield}{JDauthor}
\jdprivate \jdfinal \jdstatic 
\jdtype{\jdtypesimple{String}}
\end{jdfield}
\begin{jdfield}{JDpara}
\jdprivate \jdfinal \jdstatic 
\jdtype{\jdtypesimple{String}}
\end{jdfield}
\begin{jdfield}{JDthrows}
\jdprivate \jdfinal \jdstatic 
\jdtype{\jdtypesimple{String}}
\end{jdfield}
\begin{jdfield}{JDinhtable}
\jdprivate \jdfinal \jdstatic 
\jdtype{\jdtypesimple{String}}
\end{jdfield}
\begin{jdfield}{JDClinksimple}
\jdprivate \jdfinal \jdstatic 
\jdtype{\jdtypesimple{String}}
\end{jdfield}
\begin{jdfield}{JDClinkarray}
\jdprivate \jdfinal \jdstatic 
\jdtype{\jdtypesimple{String}}
\end{jdfield}
\begin{jdfield}{ps\_dateiausgabe}
\jdprivate 
\jdtype{\jdtypesimple{TexPrintStream}}
\JDtext{Stream, the output is written to.}
\JDauthor{Jolle}
\JDsince{version 1.0 from 13.05.2008}
\end{jdfield}
\begin{jdconstructor}
\jdpublic 
\JDpara{\jdtypesimple{String}}{s\_zielpath}{path with the destination file}
\JDthrows{IOException}{If the file cannot be created.}
\JDthrows{FileNotFoundException}{If the file isn't available after creation process}
\JDtext{Initializes the ClassWriter, creating a new file with a stream into}
\JDauthor{Jolle}
\JDsince{version 1.0 of 13.05.2008}
\end{jdconstructor}
\begin{jdmethod}{print}
\jdpublic 
\jdtype{\jdtypesimple{void}}
\JDpara{\jdtypesimple{ClassDoc}}{cd}{class-object}
\JDtext{Writes the information of this file}
\JDsince{version 1.0 from 13.05.2008}
\JDauthor{Jolle}
\end{jdmethod}
\begin{jdmethod}{print}
\jdprivate 
\jdtype{\jdtypesimple{void}}
\JDpara{\jdtypesimple{FieldDoc}}{fd}{field-object}
\JDtext{Writes the information of a field}
\JDsince{version 1.0 from 13.05.2008}
\JDauthor{Jolle}
\end{jdmethod}
\begin{jdmethod}{print}
\jdprivate 
\jdtype{\jdtypesimple{void}}
\JDpara{\jdtypesimple{ConstructorDoc}}{cd}{constructor-object}
\JDtext{Writes the informationen of a constructor}
\JDsince{version 1.0 from 13.05.2008}
\JDauthor{Jolle}
\end{jdmethod}
\begin{jdmethod}{print}
\jdprivate 
\jdtype{\jdtypesimple{void}}
\JDpara{\jdtypesimple{MethodDoc}}{md}{method-object}
\JDtext{Writes the information of a method}
\JDsince{version 1.0 from 13.05.2008}
\JDauthor{Jolle}
\end{jdmethod}
\begin{jdmethod}{printTags}
\jdprivate 
\jdtype{\jdtypesimple{void}}
\JDpara{\jdtypesimple{Doc}}{d}{Doc-Type with the javadoc-information}
\JDtext{Writes all primitv javadoc attributes}
\JDsince{version 1.0 from 13.05.2008}
\JDauthor{Jolle}
\end{jdmethod}
\begin{jdmethod}{printClassInfo}
\jdprivate 
\jdtype{\jdtypesimple{InhTable}}
\JDpara{\jdtypesimple{ClassDoc}}{cd}{Class-object}
\JDtext{Writes the classheader}
\JDsince{version 1.0 from 13.05.2008}
\JDauthor{Jolle}
\JDreturn{The tableOfInheritance is created in this method and returned for later use.}
\end{jdmethod}
\begin{jdmethod}{getLinks}
\jdpublic \jdstatic 
\jdtype{\jdtypesimple{String}}
\JDpara{\jdtypesimple{Type}}{t}{Type-Object}
\JDtext{Returns the type (Array or Simple) of a type}
\JDsince{version 1.0 from 13.05.2008}
\JDauthor{Jolle}
\JDreturn{string with the corresponding texcommand}
\end{jdmethod}
\begin{jdmethod}{printLinks}
\jdprivate 
\jdtype{\jdtypesimple{void}}
\JDpara{\jdtypesimple{Type}}{t}{Type-Object}
\JDtext{Writes the linked type}
\JDauthor{Jolle}
\JDsince{version 1.0 from 13.05.2008}
\end{jdmethod}
\begin{jdmethod}{printInhTable}
\jdprivate 
\jdtype{\jdtypesimple{void}}
\JDpara{\jdtypesimple{InhTable}}{it}{the object with an (unsorted) table}
\JDtext{Writes the tableOfInheritance and all entries}
\JDsince{version 1.0 from 13.05.2008}
\JDauthor{Jolle}
\end{jdmethod}
\begin{jdmethod}{print}
\jdprivate 
\jdtype{\jdtypesimple{void}}
\JDtext{Writes an linebreak}
\JDsince{version 1.0 from 13.05.2008}
\JDauthor{Jolle}
\end{jdmethod}
\begin{jdmethod}{printCommand}
\jdprivate 
\jdtype{\jdtypesimple{void}}
\JDpara{\jdtypesimple{String}}{befehl}{name of the command}
\JDtext{Writes a tex-command}
\JDsince{version 1.0 from 13.05.2008}
\JDauthor{Jolle}
\end{jdmethod}
\begin{jdmethod}{printOpt}
\jdprivate 
\jdtype{\jdtypesimple{void}}
\JDpara{\jdtypesimple{String}}{option}{option name}
\JDtext{Writes a tex-option}
\JDsince{version 1.0 from 13.05.2008}
\JDauthor{Jolle}
\end{jdmethod}
\begin{jdmethod}{printArgument}
\jdprivate 
\jdtype{\jdtypesimple{void}}
\JDpara{\jdtypesimple{String}}{arg}{argumentname}
\JDtext{Writes a tex-argument}
\JDsince{version 1.0 from 13.05.2008}
\JDauthor{Jolle}
\end{jdmethod}
\begin{jdmethod}{printBegin}
\jdprivate 
\jdtype{\jdtypesimple{void}}
\JDpara{\jdtypesimple{String}}{umgebung}{environmentname}
\JDtext{Writes the beginning of an environment}
\JDsince{version 1.0 from 13.05.2008}
\JDauthor{Jolle}
\end{jdmethod}
\begin{jdmethod}{printEnd}
\jdprivate 
\jdtype{\jdtypesimple{void}}
\JDpara{\jdtypesimple{String}}{umgebung}{environmentname}
\JDtext{Writes the ending of an environment}
\JDsince{version 1.0 from 13.05.2008}
\JDauthor{Jolle}
\end{jdmethod}
\end{jdclass}

\begin{jdclass}[class]{TexPrintStream}
\begin{jdclassheader}

\jdpublic 
\jdinherits{\jdtypesimple{Object}\jdinh \jdtypesimple{OutputStream}\jdinh \jdtypesimple{FilterOutputStream}\jdinh \jdtypesimple{PrintStream}}
\JDtext{Conversion of special characters into tex-format}
\JDauthor{Jolle}
\JDsince{13.05.2008}
\JDversion{1.0}
\end{jdclassheader}
\begin{jdinheritancetable} \jdInhEntry{\jdtypesimple{OutputStream} out}{FilterOutputStream}
 \jdInhEntry{\jdtypesimple{PrintStream} append( \jdtypesimple{CharSequence} )}{PrintStream}
 \jdInhEntry{\jdtypesimple{PrintStream} append( \jdtypesimple{CharSequence}, \jdtypesimple{int}, \jdtypesimple{int} )}{PrintStream}
 \jdInhEntry{\jdtypesimple{PrintStream} append( \jdtypesimple{char} )}{PrintStream}
 \jdInhEntry{\jdtypesimple{Appendable} append( \jdtypesimple{char} )}{PrintStream}
 \jdInhEntry{\jdtypesimple{Appendable} append( \jdtypesimple{CharSequence}, \jdtypesimple{int}, \jdtypesimple{int} )}{PrintStream}
 \jdInhEntry{\jdtypesimple{Appendable} append( \jdtypesimple{CharSequence} )}{PrintStream}
 \jdInhEntry{\jdtypesimple{boolean} checkError(  )}{PrintStream}
 \jdInhEntry{\jdtypesimple{void} clearError(  )}{PrintStream}
 \jdInhEntry{\jdtypesimple{void} close(  )}{PrintStream}
 \jdInhEntry{\jdtypesimple{void} flush(  )}{PrintStream}
 \jdInhEntry{\jdtypesimple{PrintStream} format( \jdtypesimple{String}, \jdtypearray{Object}{\lbrack{}\rbrack{}} )}{PrintStream}
 \jdInhEntry{\jdtypesimple{PrintStream} format( \jdtypesimple{Locale}, \jdtypesimple{String}, \jdtypearray{Object}{\lbrack{}\rbrack{}} )}{PrintStream}
 \jdInhEntry{\jdtypesimple{void} print( \jdtypesimple{boolean} )}{PrintStream}
 \jdInhEntry{\jdtypesimple{void} print( \jdtypesimple{char} )}{PrintStream}
 \jdInhEntry{\jdtypesimple{void} print( \jdtypesimple{int} )}{PrintStream}
 \jdInhEntry{\jdtypesimple{void} print( \jdtypesimple{long} )}{PrintStream}
 \jdInhEntry{\jdtypesimple{void} print( \jdtypesimple{float} )}{PrintStream}
 \jdInhEntry{\jdtypesimple{void} print( \jdtypesimple{double} )}{PrintStream}
 \jdInhEntry{\jdtypesimple{void} print( \jdtypearray{char}{\lbrack{}\rbrack{}} )}{PrintStream}
 \jdInhEntry{\jdtypesimple{void} print( \jdtypesimple{String} )}{PrintStream}
 \jdInhEntry{\jdtypesimple{void} print( \jdtypesimple{Object} )}{PrintStream}
 \jdInhEntry{\jdtypesimple{PrintStream} printf( \jdtypesimple{String}, \jdtypearray{Object}{\lbrack{}\rbrack{}} )}{PrintStream}
 \jdInhEntry{\jdtypesimple{PrintStream} printf( \jdtypesimple{Locale}, \jdtypesimple{String}, \jdtypearray{Object}{\lbrack{}\rbrack{}} )}{PrintStream}
 \jdInhEntry{\jdtypesimple{void} println(  )}{PrintStream}
 \jdInhEntry{\jdtypesimple{void} println( \jdtypesimple{boolean} )}{PrintStream}
 \jdInhEntry{\jdtypesimple{void} println( \jdtypesimple{char} )}{PrintStream}
 \jdInhEntry{\jdtypesimple{void} println( \jdtypesimple{int} )}{PrintStream}
 \jdInhEntry{\jdtypesimple{void} println( \jdtypesimple{long} )}{PrintStream}
 \jdInhEntry{\jdtypesimple{void} println( \jdtypesimple{float} )}{PrintStream}
 \jdInhEntry{\jdtypesimple{void} println( \jdtypesimple{double} )}{PrintStream}
 \jdInhEntry{\jdtypesimple{void} println( \jdtypearray{char}{\lbrack{}\rbrack{}} )}{PrintStream}
 \jdInhEntry{\jdtypesimple{void} println( \jdtypesimple{String} )}{PrintStream}
 \jdInhEntry{\jdtypesimple{void} println( \jdtypesimple{Object} )}{PrintStream}
 \jdInhEntry{\jdtypesimple{void} setError(  )}{PrintStream}
 \jdInhEntry{\jdtypesimple{void} write( \jdtypesimple{int} )}{PrintStream}
 \jdInhEntry{\jdtypesimple{void} write( \jdtypearray{byte}{\lbrack{}\rbrack{}}, \jdtypesimple{int}, \jdtypesimple{int} )}{PrintStream}
 \jdInhEntry{\jdtypesimple{void} close(  )}{FilterOutputStream}
 \jdInhEntry{\jdtypesimple{void} flush(  )}{FilterOutputStream}
 \jdInhEntry{\jdtypesimple{void} write( \jdtypesimple{int} )}{FilterOutputStream}
 \jdInhEntry{\jdtypesimple{void} write( \jdtypearray{byte}{\lbrack{}\rbrack{}} )}{FilterOutputStream}
 \jdInhEntry{\jdtypesimple{void} write( \jdtypearray{byte}{\lbrack{}\rbrack{}}, \jdtypesimple{int}, \jdtypesimple{int} )}{FilterOutputStream}
 \jdInhEntry{\jdtypesimple{void} close(  )}{OutputStream}
 \jdInhEntry{\jdtypesimple{void} flush(  )}{OutputStream}
 \jdInhEntry{\jdtypesimple{void} write( \jdtypesimple{int} )}{OutputStream}
 \jdInhEntry{\jdtypesimple{void} write( \jdtypearray{byte}{\lbrack{}\rbrack{}} )}{OutputStream}
 \jdInhEntry{\jdtypesimple{void} write( \jdtypearray{byte}{\lbrack{}\rbrack{}}, \jdtypesimple{int}, \jdtypesimple{int} )}{OutputStream}
 \jdInhEntry{\jdtypesimple{Object} clone(  )}{Object}
 \jdInhEntry{\jdtypesimple{boolean} equals( \jdtypesimple{Object} )}{Object}
 \jdInhEntry{\jdtypesimple{void} finalize(  )}{Object}
 \jdInhEntry{\jdtypesimple{Class} getClass(  )}{Object}
 \jdInhEntry{\jdtypesimple{int} hashCode(  )}{Object}
 \jdInhEntry{\jdtypesimple{void} notify(  )}{Object}
 \jdInhEntry{\jdtypesimple{void} notifyAll(  )}{Object}
 \jdInhEntry{\jdtypesimple{String} toString(  )}{Object}
 \jdInhEntry{\jdtypesimple{void} wait( \jdtypesimple{long} )}{Object}
 \jdInhEntry{\jdtypesimple{void} wait( \jdtypesimple{long}, \jdtypesimple{int} )}{Object}
 \jdInhEntry{\jdtypesimple{void} wait(  )}{Object}
\end{jdinheritancetable}
\begin{jdconstructor}
\jdpublic 
\JDpara{\jdtypesimple{File}}{f}{File to write}
\JDthrows{FileNotFoundException}{if the file doesn't exist}
\JDtext{Construktor, initialising the stream into the given file}
\JDsince{Version 1.0 from 13.05.2008}
\JDauthor{Jolle}
\end{jdconstructor}
\begin{jdmethod}{printTex}
\jdpublic 
\jdtype{\jdtypesimple{void}}
\JDpara{\jdtypesimple{String}}{ausgabe}{unformatted string}
\JDtext{Converts an String and writes it to the stream}
\JDsince{Version 1.0 from 13.05.2008}
\JDauthor{Jolle}
\end{jdmethod}
\begin{jdmethod}{umwandlung}
\jdpublic \jdstatic 
\jdtype{\jdtypesimple{String}}
\JDpara{\jdtypesimple{String}}{unformatted}{unformatted string}
\JDtext{Converts the special characters to \TeX-format
 
 Characters, that are converted: \{ \} \_ \^{} \&{} \#{} \lbrack{} \rbrack{}}
\JDreturn{formatted string}
\JDsince{Version 1.0 from 13.05.2008}
\JDauthor{Jolle}
\end{jdmethod}
\end{jdclass}

\begin{jdclass}[class]{InhTable}
\begin{jdclassheader}

\jdpublic 
\jdinherits{\jdtypesimple{Object}}
\JDtext{Table of Inheritance}
\JDauthor{Jolle}
\JDsince{13.05.2008}
\JDversion{1.0}
\end{jdclassheader}
\begin{jdinheritancetable} \jdInhEntry{\jdtypesimple{Object} clone(  )}{Object}
 \jdInhEntry{\jdtypesimple{boolean} equals( \jdtypesimple{Object} )}{Object}
 \jdInhEntry{\jdtypesimple{void} finalize(  )}{Object}
 \jdInhEntry{\jdtypesimple{Class} getClass(  )}{Object}
 \jdInhEntry{\jdtypesimple{int} hashCode(  )}{Object}
 \jdInhEntry{\jdtypesimple{void} notify(  )}{Object}
 \jdInhEntry{\jdtypesimple{void} notifyAll(  )}{Object}
 \jdInhEntry{\jdtypesimple{String} toString(  )}{Object}
 \jdInhEntry{\jdtypesimple{void} wait( \jdtypesimple{long} )}{Object}
 \jdInhEntry{\jdtypesimple{void} wait( \jdtypesimple{long}, \jdtypesimple{int} )}{Object}
 \jdInhEntry{\jdtypesimple{void} wait(  )}{Object}
\end{jdinheritancetable}
\begin{jdfield}{tabelle}
\jdprivate 
\jdtype{\jdtypesimple{ArrayList}}
\JDtext{List containing all entries}
\JDauthor{Jolle}
\JDsince{version 1.0 from 13.05.2008}
\end{jdfield}
\begin{jdconstructor}
\jdpublic 
\end{jdconstructor}
\begin{jdmethod}{addEntries}
\jdpublic 
\jdtype{\jdtypesimple{void}}
\JDpara{\jdtypesimple{ClassDoc}}{parent}{die �bergeordenete Klasse}
\JDtext{Adds the fields and methods of the parent-class to the table}
\JDauthor{Jolle}
\JDsince{version 1.0 from 13.05.2008}
\end{jdmethod}
\begin{jdmethod}{sortTable}
\jdpublic 
\jdtype{\jdtypesimple{void}}
\JDtext{Sorts the table to field/methods, then inheriting class, than alphanumeric}
\JDauthor{Jolle}
\JDsince{version 1.0 from 13.05.2008}
\end{jdmethod}
\begin{jdmethod}{getTexTableEntries}
\jdpublic 
\jdtype{\jdtypesimple{String}}
\JDtext{Returns an string containing all entries in Tex-format}
\JDauthor{Jolle}
\JDsince{Version 1.0 from 13.05.2008}
\JDreturn{the hole table content as one string.}
\end{jdmethod}
\end{jdclass}

\begin{jdclass}[class]{InhEntry}
\begin{jdclassheader}

\jdpublic 
\jdimplements{Comparable}
\jdinherits{\jdtypesimple{Object}}
\JDtext{One Entry of the table.}
\JDauthor{Jolle}
\JDversion{1.0}
\JDsince{13.05.2008}
\end{jdclassheader}
\begin{jdinheritancetable} \jdInhEntry{\jdtypesimple{Object} clone(  )}{Object}
 \jdInhEntry{\jdtypesimple{boolean} equals( \jdtypesimple{Object} )}{Object}
 \jdInhEntry{\jdtypesimple{void} finalize(  )}{Object}
 \jdInhEntry{\jdtypesimple{Class} getClass(  )}{Object}
 \jdInhEntry{\jdtypesimple{int} hashCode(  )}{Object}
 \jdInhEntry{\jdtypesimple{void} notify(  )}{Object}
 \jdInhEntry{\jdtypesimple{void} notifyAll(  )}{Object}
 \jdInhEntry{\jdtypesimple{String} toString(  )}{Object}
 \jdInhEntry{\jdtypesimple{void} wait( \jdtypesimple{long} )}{Object}
 \jdInhEntry{\jdtypesimple{void} wait( \jdtypesimple{long}, \jdtypesimple{int} )}{Object}
 \jdInhEntry{\jdtypesimple{void} wait(  )}{Object}
\end{jdinheritancetable}
\begin{jdfield}{JDInhEntry}
\jdprivate \jdfinal \jdstatic 
\jdtype{\jdtypesimple{String}}
\JDtext{Tex-command for an entry}
\JDauthor{Jolle}
\JDsince{version 1.0 from 13.05.2008}
\end{jdfield}
\begin{jdfield}{eintrag}
\jdprivate 
\jdtype{\jdtypesimple{Doc}}
\JDtext{the element of an entry}
\JDauthor{Jolle}
\JDsince{version 1.0 from 13.05.2008}
\end{jdfield}
\begin{jdfield}{parent}
\jdprivate 
\jdtype{\jdtypesimple{ClassDoc}}
\JDtext{The parent class of an entry}
\JDauthor{Jolle}
\JDsince{version 1.0 from 13.05.2008}
\end{jdfield}
\begin{jdconstructor}
\jdpublic 
\JDpara{\jdtypesimple{Doc}}{d}{element}
\JDpara{\jdtypesimple{ClassDoc}}{parent}{the inheriter}
\JDtext{Creates the entry with the element and the parentclass}
\JDauthor{Jolle}
\JDsince{version 1.0 from 13.05.2008}
\end{jdconstructor}
\begin{jdmethod}{getTexTableEntry}
\jdpublic 
\jdtype{\jdtypesimple{String}}
\JDtext{Creates an entry-line in tex-format}
\JDauthor{Jolle}
\JDsince{version 1.0 from 13.05.2008}
\JDreturn{the formatted tex-line}
\end{jdmethod}
\begin{jdmethod}{compareTo}
\jdpublic 
\jdtype{\jdtypesimple{int}}
\JDpara{\jdtypesimple{InhEntry}}{name}{the object to compare to}
\JDtext{Overrides the compartTo-Method of Comparable}
\JDauthor{Jolle}
\JDsince{Version 1.0 from 13.05.2008}
\JDreturn{0, when equal; 1, if the object is an field and the other one a method, or - when equal- , the parent is higher than the other parent or -when equal- the alphanumeric comparison of the names.
 otherwise -1}
\end{jdmethod}
\begin{jdmethod}{compareInheritation}
\jdprivate 
\jdtype{\jdtypesimple{int}}
\JDpara{\jdtypesimple{ClassDoc}}{cd}{the other parent}
\JDtext{Compares to parent to hierarchy}
\JDauthor{Jolle}
\JDsince{Version 1.0 from 13.05.2008}
\JDreturn{0, when equal, 1 when the own parent is higher, otherwise -1}
\end{jdmethod}
\begin{jdmethod}{compareType}
\jdprivate 
\jdtype{\jdtypesimple{int}}
\JDpara{\jdtypesimple{Doc}}{externDoc}{the other element}
\JDtext{Compares two element to type}
\JDauthor{Jolle}
\JDsince{Version 1.0 from 13.05.2008}
\JDreturn{0, if both method or both field, 1 if the own one is field and the other method, otherwise -1}
\end{jdmethod}
\end{jdclass}



\end{document}