\ProvidesFile{longnamefilelist.tex}[2012/03/14 documenting longnamefilelist.sty]
\title{\textsf{\Huge                            %% \Huge 2012/03/14
       longnamefilelist.sty}\\---\\\cs{listfiles} 
       when some File Names\\Consist of quite a Lot of 
       Characters, such as\\\relax
       [\,to be continued in next package update\,]\thanks{This
       document describes version
       \textcolor{blue}{\UseVersionOf{\jobname.sty}}
       of \textsf{\jobname.sty} as of \UseDateOf{\jobname.sty}.}}
{ \RequirePackage{makedoc} \ProcessLineMessage{}
  \MakeJobDoc{19}
  {\SectionLevelTwoParseInput}  }
\documentclass[fleqn]{article}%% TODO paper dimensions!?
\ProvidesFile{makedoc.cfg}[2011/06/27 documentation settings] 

\author{Uwe L\"uck\thanks{\url{http://contact-ednotes.sty.de.vu}}}
% \author{Uwe L\"uck---{\tt http://contact-ednotes.sty.de.vu}}

%% hyperref:
\RequirePackage{ifpdf}
\usepackage[%
  \ifpdf
%     bookmarks=false,          %% 2010/12/22
%     bookmarksnumbered,
    bookmarksopen,              %% 2011/01/24!?
    bookmarksopenlevel=2,       %% 2011/01/23
%     pdfpagemode=UseNone,
%     pdfstartpage=10,
%     pdfstartview=FitH,
    citebordercolor={ .6 1    .6},
    filebordercolor={1    .6 1},
    linkbordercolor={1    .9  .7},
     urlbordercolor={ .7 1   1},   %% playing 2011/01/24
  \else
    draft
  \fi
]{hyperref}

\RequirePackage{niceverb}[2011/01/24] 
\RequirePackage{readprov}               %% 2010/12/08
\RequirePackage{hypertoc}               %% 2011/01/23
\RequirePackage{texlinks}               %% 2011/01/24
\makeatletter
  \@ifundefined{strong} 
               {\let\strong\textbf}     %% 2011/01/24
               {} 
  \@ifundefined{file} 
               {\let\file\texttt}       %% 2011/05/23
               {} 
\makeatother

\errorcontextlines=4
\pagestyle{headings}

\endinput

 %% shared formatting settings
\ReadPackageInfos{longnamefilelist}
\sloppy
\begin{document}
\maketitle
\begin{abstract}\noindent
'longnamefilelist.sty' equips \LaTeX's `\listfiles' with an optional 
argument for the number of characters in the longest base filename. 
This way you get a neatly aligned file list even when it contains 
files whose base names have more than 8 characters.
\end{abstract}
\tableofcontents

%   \newpage
% \section{Features and Usage}
\section{Installing and Calling}
The file 'longnamefilelist.sty' is provided ready, installation only requires
putting it somewhere where \TeX\ finds it
(which may need updating the filename data
 base).\urlfoot{ukfaqref}{inst-wlcf}           %% corr. 2011/02/08

%% extended 2011/01/14:
Below the `\documentclass' line(s) and above `\begin{document}',
you load 'longnamefilelist.sty' (as usually) by
\begin{verbatim}
  \usepackage{longnamefilelist}
\end{verbatim}
Alternatively---e.g., for use with \ctanpkgref{myfilist} from the 
\ctanpkgref{fileinfo} bundle, see~Sec.~\ref{sec:myfilist}, 
or in order to include the `.cls' file in the list---you may load it by 
\begin{verbatim}
  \RequirePackage{longnamefilelist}
\end{verbatim}
before `\documentclass' or when you don't use `\documentclass'. 

% \section{Example}

%   \pagebreak
% \section{Implementation}
\section{Package File Header (Legalese)} %% ize -> ese 2012/09/30
\ProvidesFile{longnamefilelist.tex}[2012/03/14 documenting longnamefilelist.sty]
\title{\textsf{\Huge                            %% \Huge 2012/03/14
       longnamefilelist.sty}\\---\\\cs{listfiles} 
       when some File Names\\Consist of quite a Lot of 
       Characters, such as\\\relax
       [\,to be continued in next package update\,]\thanks{This
       document describes version
       \textcolor{blue}{\UseVersionOf{\jobname.sty}}
       of \textsf{\jobname.sty} as of \UseDateOf{\jobname.sty}.}}
{ \RequirePackage{makedoc} \ProcessLineMessage{}
  \MakeJobDoc{19}
  {\SectionLevelTwoParseInput}  }
\documentclass[fleqn]{article}%% TODO paper dimensions!?
\ProvidesFile{makedoc.cfg}[2011/06/27 documentation settings] 

\author{Uwe L\"uck\thanks{\url{http://contact-ednotes.sty.de.vu}}}
% \author{Uwe L\"uck---{\tt http://contact-ednotes.sty.de.vu}}

%% hyperref:
\RequirePackage{ifpdf}
\usepackage[%
  \ifpdf
%     bookmarks=false,          %% 2010/12/22
%     bookmarksnumbered,
    bookmarksopen,              %% 2011/01/24!?
    bookmarksopenlevel=2,       %% 2011/01/23
%     pdfpagemode=UseNone,
%     pdfstartpage=10,
%     pdfstartview=FitH,
    citebordercolor={ .6 1    .6},
    filebordercolor={1    .6 1},
    linkbordercolor={1    .9  .7},
     urlbordercolor={ .7 1   1},   %% playing 2011/01/24
  \else
    draft
  \fi
]{hyperref}

\RequirePackage{niceverb}[2011/01/24] 
\RequirePackage{readprov}               %% 2010/12/08
\RequirePackage{hypertoc}               %% 2011/01/23
\RequirePackage{texlinks}               %% 2011/01/24
\makeatletter
  \@ifundefined{strong} 
               {\let\strong\textbf}     %% 2011/01/24
               {} 
  \@ifundefined{file} 
               {\let\file\texttt}       %% 2011/05/23
               {} 
\makeatother

\errorcontextlines=4
\pagestyle{headings}

\endinput

 %% shared formatting settings
\ReadPackageInfos{longnamefilelist}
\sloppy
\begin{document}
\maketitle
\begin{abstract}\noindent
'longnamefilelist.sty' equips \LaTeX's `\listfiles' with an optional 
argument for the number of characters in the longest base filename. 
This way you get a neatly aligned file list even when it contains 
files whose base names have more than 8 characters.
\end{abstract}
\tableofcontents

%   \newpage
% \section{Features and Usage}
\section{Installing and Calling}
The file 'longnamefilelist.sty' is provided ready, installation only requires
putting it somewhere where \TeX\ finds it
(which may need updating the filename data
 base).\urlfoot{ukfaqref}{inst-wlcf}           %% corr. 2011/02/08

%% extended 2011/01/14:
Below the `\documentclass' line(s) and above `\begin{document}',
you load 'longnamefilelist.sty' (as usually) by
\begin{verbatim}
  \usepackage{longnamefilelist}
\end{verbatim}
Alternatively---e.g., for use with \ctanpkgref{myfilist} from the 
\ctanpkgref{fileinfo} bundle, see~Sec.~\ref{sec:myfilist}, 
or in order to include the `.cls' file in the list---you may load it by 
\begin{verbatim}
  \RequirePackage{longnamefilelist}
\end{verbatim}
before `\documentclass' or when you don't use `\documentclass'. 

% \section{Example}

%   \pagebreak
% \section{Implementation}
\section{Package File Header (Legalese)} %% ize -> ese 2012/09/30
\ProvidesFile{longnamefilelist.tex}[2012/03/14 documenting longnamefilelist.sty]
\title{\textsf{\Huge                            %% \Huge 2012/03/14
       longnamefilelist.sty}\\---\\\cs{listfiles} 
       when some File Names\\Consist of quite a Lot of 
       Characters, such as\\\relax
       [\,to be continued in next package update\,]\thanks{This
       document describes version
       \textcolor{blue}{\UseVersionOf{\jobname.sty}}
       of \textsf{\jobname.sty} as of \UseDateOf{\jobname.sty}.}}
{ \RequirePackage{makedoc} \ProcessLineMessage{}
  \MakeJobDoc{19}
  {\SectionLevelTwoParseInput}  }
\documentclass[fleqn]{article}%% TODO paper dimensions!?
\ProvidesFile{makedoc.cfg}[2011/06/27 documentation settings] 

\author{Uwe L\"uck\thanks{\url{http://contact-ednotes.sty.de.vu}}}
% \author{Uwe L\"uck---{\tt http://contact-ednotes.sty.de.vu}}

%% hyperref:
\RequirePackage{ifpdf}
\usepackage[%
  \ifpdf
%     bookmarks=false,          %% 2010/12/22
%     bookmarksnumbered,
    bookmarksopen,              %% 2011/01/24!?
    bookmarksopenlevel=2,       %% 2011/01/23
%     pdfpagemode=UseNone,
%     pdfstartpage=10,
%     pdfstartview=FitH,
    citebordercolor={ .6 1    .6},
    filebordercolor={1    .6 1},
    linkbordercolor={1    .9  .7},
     urlbordercolor={ .7 1   1},   %% playing 2011/01/24
  \else
    draft
  \fi
]{hyperref}

\RequirePackage{niceverb}[2011/01/24] 
\RequirePackage{readprov}               %% 2010/12/08
\RequirePackage{hypertoc}               %% 2011/01/23
\RequirePackage{texlinks}               %% 2011/01/24
\makeatletter
  \@ifundefined{strong} 
               {\let\strong\textbf}     %% 2011/01/24
               {} 
  \@ifundefined{file} 
               {\let\file\texttt}       %% 2011/05/23
               {} 
\makeatother

\errorcontextlines=4
\pagestyle{headings}

\endinput

 %% shared formatting settings
\ReadPackageInfos{longnamefilelist}
\sloppy
\begin{document}
\maketitle
\begin{abstract}\noindent
'longnamefilelist.sty' equips \LaTeX's `\listfiles' with an optional 
argument for the number of characters in the longest base filename. 
This way you get a neatly aligned file list even when it contains 
files whose base names have more than 8 characters.
\end{abstract}
\tableofcontents

%   \newpage
% \section{Features and Usage}
\section{Installing and Calling}
The file 'longnamefilelist.sty' is provided ready, installation only requires
putting it somewhere where \TeX\ finds it
(which may need updating the filename data
 base).\urlfoot{ukfaqref}{inst-wlcf}           %% corr. 2011/02/08

%% extended 2011/01/14:
Below the `\documentclass' line(s) and above `\begin{document}',
you load 'longnamefilelist.sty' (as usually) by
\begin{verbatim}
  \usepackage{longnamefilelist}
\end{verbatim}
Alternatively---e.g., for use with \ctanpkgref{myfilist} from the 
\ctanpkgref{fileinfo} bundle, see~Sec.~\ref{sec:myfilist}, 
or in order to include the `.cls' file in the list---you may load it by 
\begin{verbatim}
  \RequirePackage{longnamefilelist}
\end{verbatim}
before `\documentclass' or when you don't use `\documentclass'. 

% \section{Example}

%   \pagebreak
% \section{Implementation}
\section{Package File Header (Legalese)} %% ize -> ese 2012/09/30
\ProvidesFile{longnamefilelist.tex}[2012/03/14 documenting longnamefilelist.sty]
\title{\textsf{\Huge                            %% \Huge 2012/03/14
       longnamefilelist.sty}\\---\\\cs{listfiles} 
       when some File Names\\Consist of quite a Lot of 
       Characters, such as\\\relax
       [\,to be continued in next package update\,]\thanks{This
       document describes version
       \textcolor{blue}{\UseVersionOf{\jobname.sty}}
       of \textsf{\jobname.sty} as of \UseDateOf{\jobname.sty}.}}
{ \RequirePackage{makedoc} \ProcessLineMessage{}
  \MakeJobDoc{19}
  {\SectionLevelTwoParseInput}  }
\documentclass[fleqn]{article}%% TODO paper dimensions!?
\input{makedoc.cfg} %% shared formatting settings
\ReadPackageInfos{longnamefilelist}
\sloppy
\begin{document}
\maketitle
\begin{abstract}\noindent
'longnamefilelist.sty' equips \LaTeX's `\listfiles' with an optional 
argument for the number of characters in the longest base filename. 
This way you get a neatly aligned file list even when it contains 
files whose base names have more than 8 characters.
\end{abstract}
\tableofcontents

%   \newpage
% \section{Features and Usage}
\section{Installing and Calling}
The file 'longnamefilelist.sty' is provided ready, installation only requires
putting it somewhere where \TeX\ finds it
(which may need updating the filename data
 base).\urlfoot{ukfaqref}{inst-wlcf}           %% corr. 2011/02/08

%% extended 2011/01/14:
Below the `\documentclass' line(s) and above `\begin{document}',
you load 'longnamefilelist.sty' (as usually) by
\begin{verbatim}
  \usepackage{longnamefilelist}
\end{verbatim}
Alternatively---e.g., for use with \ctanpkgref{myfilist} from the 
\ctanpkgref{fileinfo} bundle, see~Sec.~\ref{sec:myfilist}, 
or in order to include the `.cls' file in the list---you may load it by 
\begin{verbatim}
  \RequirePackage{longnamefilelist}
\end{verbatim}
before `\documentclass' or when you don't use `\documentclass'. 

% \section{Example}

%   \pagebreak
% \section{Implementation}
\section{Package File Header (Legalese)} %% ize -> ese 2012/09/30
\input{longnamefilelist.doc}
\end{document}

VERSION HISTORY

2012/03/11  for v0.1    started
2012/03/12              completed 
2012/03/14      v0.1b   name in title \Huge
2012/09/30  for v0.2    ize -> ese

\end{document}

VERSION HISTORY

2012/03/11  for v0.1    started
2012/03/12              completed 
2012/03/14      v0.1b   name in title \Huge
2012/09/30  for v0.2    ize -> ese

\end{document}

VERSION HISTORY

2012/03/11  for v0.1    started
2012/03/12              completed 
2012/03/14      v0.1b   name in title \Huge
2012/09/30  for v0.2    ize -> ese

\end{document}

VERSION HISTORY

2012/03/11  for v0.1    started
2012/03/12              completed 
2012/03/14      v0.1b   name in title \Huge
2012/09/30  for v0.2    ize -> ese
