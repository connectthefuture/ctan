%^^A manfnt.dtx --- doc source file for manfnt.sty  -*-LaTeX-*-
% \iffalse meta-comment
%% Copyright (C) 1998 - 99 by Axel Kielhorn, all rights reserved
%% Copyright (C) 1999 by Denis Kosygin, all rights reserved.
%% 
%% This program can be redistributed and/or modified under the terms
%% of the LaTeX Project Public License Distributed from CTAN
%% archives in directory macros/latex/base/lppl.txt; either
%% version 1 of the License, or any later version.
% \fi
% \CheckSum{118}
% \iffalse
% Below we define the release date and version of manfnt for which
% this documentation is written.
% \fi
\def\fileversion{0.2}
\def\filedate{1999/07/01}
% \iffalse
% Keep the definitions above uncommented and before the rest of the
% package code so that the docstrip sticks them first in manfnt.sty.
% Use \def to avoid conflicts with definitions \fileversion and
% \filedate in other packages.
% Keep them near the beginning and only in one place so that it is
% easy to find them and modify for new releases.
% Do not hide them from doc module, so that the numbers in
% documentation are consistent with the numbers in manfnt.sty.
% Do not enclose them in macrocode environment, as it inserts them in
% the documentation too early.  The beginning of section `Code' is
% hacked to reflect their presence in manfnt.sty
% \fi
% \newcommand{\fixme}[1][FIXME!]{\texttt{#1}}
% \newcommand{\docdate}{1999/07/01}
% \iffalse
% \docdate definition is hidden from docstrip because it is relevant
% only in the documentation.   
% Keep it before \maketitle, so that the documentation date is correct.
% If this package is ever archived with version control, perhaps it is
% worthwhile to hack the archiving program to automatically update
% \docdate whenever the current file changes.
% \fi
%
% \changes{0.1}{\fixme}{Initial version}
% \changes{0.2}{1997/06/10}{More symbol names added.  Commands
%   \cmd{\manfntsymbol} and \cmd{\texdbend} (with variations).}
%
% \title{The \texttt{Manfnt} package\\
%  A quick way to access the symbols in manfnt\thanks{This manual
%    documents version \fileversion{} released on \filedate{}.}}
% \author{Axel Kielhorn\thanks{With additions by Denis Kosygin.}}
% \renewcommand{\today}{\docdate}
% \maketitle
%
% \iffalse 
% Below we hide the driver from doc module.
% We still follow the doc conventions in case we decide later to
% include the driver in the manual or to extract it into a separate
% file with docstrip.
% \fi
% \iffalse
% \section{Driver for \texttt{manfnt}}
%    \begin{macrocode}
%<*driver>
\documentclass{ltxdoc} 
\usepackage{manfnt} 
\begin{document}
\DocInput{manfnt.dtx}
\end{document}
%</driver>
%    \end{macrocode}
% \fi
%
% \section{Introduction}
% The \TeX{} and metafont manuals use some special symbols not found in
% the normal CM-fonts. Most of these symbols will be of little use for
% the average author, but some, like the ``Dangerous Bend'' sign may be
% approriate for some textbooks.
%
% Since there is no easy way to access these symbols I wrote a small
% package which I distributed with my \texttt{refman} package. I didn't
% expect much response and was quite surprised to get a mail from Denis
% Kosygin who suggested some improvements. He also suggested releasing
% it a stand-alone package, thus forcing me to write some
% documentation\texttt{:-)}
%
% \subsection*{A word of caution}
% Please use the symbols from this package sparingly.
% Their unusual shapes attract attention, which means also
% that they distract the reader from the main text.  For any advice
% how to write well there is a brilliant example, where it was not
% followed.  Still, in general, restraining means of expression
% improves the overall aesthetic quality.
%
% This point of view is reflected in the design of \texttt{manfnt}.
% In particular, that was the reason, why there are no switches to
% manfnt similar to \cmd{\textrm} or \cmd{\rmfamily}.
%
% \section{The symbols}
%
% \DescribeMacro{\manfntsymbol}
% To access a symbol in manfnt by its
% code say \cmd{\manfntsymbol}\marg{code}.  Symbol codes are shown in
% a table at the end of this manual.  Recall that in \TeX{} octal
% codes begin with |'| and hexadecimal codes with |"|.
%
% Some symbols with ``interesting'' shapes are named to alleviate the
% burden of looking up their codes.  Their names are listed at the end
% of this manual.
%
% \subsection*{The dangerous bends}
% \newcommand{\danger}{\marginpar[\hfill\dbend]{\dbend\hfill}}
% \newenvironment{example}{\smallskip\begin{trivlist}\item[]}
%    {\end{trivlist}\smallskip}
% As shown below, if \danger\cmd{\dbend} is used directly in a text,
% \LaTeX{} will place the center of the sign plate on the baseline.  As
% the result, the sign pole may overlap with the contents of the next
% line\footnote{The reason for such design can be seen in the \TeX{}book
%   by D.~Knuth \cite{book:knuth}.  There in ``dangerous bend''
%   paragraphs the ``ground level'' of the sign is the base of the
%   next line.  Special arrangments (which are beyond the scope of
%   this manual) are needed to produce similar effect in a \LaTeX{}
%   document.}.
% \begin{example}
%   \newlength{\myminipagesep}
%   \newlength{\myminipagewidth}
%   \setlength{\myminipagewidth}{.5\textwidth}
%   \addtolength{\myminipagewidth}{-2.5em}
%   \begin{minipage}{\myminipagewidth}
%     Here is an example of an unfortunate use of \dbend.  The text on
%     the next line is partially obscured by the sign pole.
%   \end{minipage}
%   \hspace{1em}^^A
%   \addtolength{\myminipagewidth}{3.5em}^^A
%   \begin{minipage}{\myminipagewidth}
%\begin{verbatim}
%Here is an example of an unfortunate 
%use of \dbend.  The text on the next
%line is partially obscured by
%the sign pole.
%\end{verbatim}
%   \end{minipage}
% \end{example}
% A better approach is illustrated in the beginning of this paragraph,
% where the dangerous bend sign is placed on the text margins.  This was
% achieved with the help of the command~\cmd{\danger}\footnote{In
%   general you should design your own commands for using this symbol,
%   which reflect the meaning you assign to it.   Book~\cite{book:lamport} on
%   \LaTeX{} by~L.~Lamport explains how to place text on margins.},
%   defined in \texttt{manfnt.dtx}:
%\begin{verbatim}
%\newcommand{\danger}{\marginpar[\hfill\dbend]{\dbend\hfill}}
%\end{verbatim}
% The first sentence of the current paragraph was entered into the
% source as
% \begin{example}
%\begin{verbatim}
% As shown below, if \danger\cmd{\dbend} is used directly in a text,
% \LaTeX{} will place the center of the sign plate on the baseline.
%\end{verbatim}
% \end{example}
%
% \DescribeMacro{\textdbend}
% \DescribeMacro{\textlhdbend}
% \DescribeMacro{\textreversedvideodbend} 
% For rare occasions, when you do need to include \textdbend{} into a
% line of text, \texttt{manfnt} provides command~\cmd{\textdbend},
% which raises the ``ground level'' of the sign to the base line making
% it suitable for inclusion.  For example, the dangerous bend sign in
% the previous sentence was produced by~\cmd{\textdbend}.  Commands
% \cmd{\textlhdbend} and \cmd{\textreversedvideodbend} act similarly.
% \iffalse
% We delay the printing of bibliography and tables until the end of
% the document. 
% Below we define few commands which are used only in tables.
% We also make | an ordinary character again, since it has a special
% meaning in tables. 
% \fi
% \renewcommand{\caption}[1]{\textbf{\large #1}\bigskip\bigskip}
% \newcommand{\subcaptionsep}{1ex}
% \newcommand{\subtablesep}{2.5ex}
% \newcommand{\subcaption}[1]{\textbf{#1}}
% \DeleteShortVerb{\|}
% \StopEventually{
% \bibliographystyle{plain}
% \begin{thebibliography}{1}
%    \bibitem{book:knuth} D.\ E.\ Knuth.
%      \newblock {\em The \TeX{}book}.
%      \newblock Addison-Wesley, Reading, Massachusetts, 1994.
%    \bibitem{book:lamport} L.\ Lamport.
%      \newblock {\em \LaTeX: a document preparation system}, -- 2nd ed.
%      \newblock Addison-Wesley, Reading, Massachusetts, 1994.
% \end{thebibliography}
% \begin{table}[p]
%   \begin{center}
%     \caption{Symbols in manfnt}
%     \begin{tabular}{c|c|c|c|c|c|c|c|c|c}
%       \textit{x}
%       & \textit{'0} & \textit{'1} & \textit{'2} & \textit{'3}
%       & \textit{'4} & \textit{'5} & \textit{'6} & \textit{'7}
%       &  \\ \hline
%       \textit{'00x} &
%       \manfntsymbol{'000} & \manfntsymbol{'001} &
%       \manfntsymbol{'002} & \manfntsymbol{'003} &
%       \manfntsymbol{'004} & \manfntsymbol{'005} &
%       \manfntsymbol{'006} & \manfntsymbol{'007} & \rule[-3ex]{0pt}{5ex}
%       \texttt{"0x} \\ \cline{1-9}
%       \textit{'01x} &
%       \manfntsymbol{'010} & \manfntsymbol{'011} &
%       \manfntsymbol{'012} & \manfntsymbol{'013} &
%       \manfntsymbol{'014} & \manfntsymbol{'015} &
%       \manfntsymbol{'016} & \manfntsymbol{'017} & 
%       \texttt{"0y} \\ \hline
%       \textit{'02x} &
%       \manfntsymbol{'020} & \manfntsymbol{'021} &
%       \manfntsymbol{'022} & \manfntsymbol{'023} &
%       \manfntsymbol{'024} & \manfntsymbol{'025} &
%       \manfntsymbol{'026} & \manfntsymbol{'027} & 
%       \texttt{"1x} \\ \cline{1-9}
%       \textit{'03x} &
%       \manfntsymbol{'030} & \manfntsymbol{'031} &
%       \manfntsymbol{'032} & \manfntsymbol{'033} &
%       \manfntsymbol{'034} & \manfntsymbol{'035} &
%       \manfntsymbol{'036} & \manfntsymbol{'037} & 
%       \texttt{"1y} \\ \hline
%       \textit{'04x} &
%       \manfntsymbol{'040} & \manfntsymbol{'041} &
%       \manfntsymbol{'042} & \manfntsymbol{'043} &
%       \manfntsymbol{'044} & \manfntsymbol{'045} &
%       \manfntsymbol{'046} & \manfntsymbol{'047} & 
%       \texttt{"2x} \\ \cline{1-9}
%       \textit{'05x} &
%       \manfntsymbol{'050} & \manfntsymbol{'051} &
%       \manfntsymbol{'052} & \manfntsymbol{'053} &
%       \manfntsymbol{'054} & \manfntsymbol{'055} &
%       \manfntsymbol{'056} & \manfntsymbol{'057} & 
%       \texttt{"2y} \\ \hline
%       \textit{'06x} &
%       \manfntsymbol{'060} & \manfntsymbol{'061} &
%       \manfntsymbol{'062} & \manfntsymbol{'063} &
%       \manfntsymbol{'064} & \manfntsymbol{'065} &
%       \manfntsymbol{'066} & \manfntsymbol{'067} & 
%       \texttt{"3x} \\ \cline{1-9}
%       \textit{'07x} &
%       \manfntsymbol{'070} & \manfntsymbol{'071} &
%       \manfntsymbol{'072} & \manfntsymbol{'073} &
%       \manfntsymbol{'074} & \manfntsymbol{'075} &
%       \manfntsymbol{'076} & \manfntsymbol{'077} & 
%       \texttt{"3y} \\ \hline
%       \textit{'10x} &
%       \manfntsymbol{'100} & \manfntsymbol{'101} &
%       \manfntsymbol{'102} & \manfntsymbol{'103} &
%       \manfntsymbol{'104} & \manfntsymbol{'105} &
%       \manfntsymbol{'106} & \manfntsymbol{'107} & 
%       \texttt{"4x} \\ \cline{1-9}
%       \textit{'11x} &
%       \manfntsymbol{'110} & \manfntsymbol{'111} &
%       \manfntsymbol{'112} & \manfntsymbol{'113} &
%       \manfntsymbol{'114} & \manfntsymbol{'115} &
%       \manfntsymbol{'116} & \manfntsymbol{'117} & 
%       \texttt{"4y} \\ \hline
%       \textit{'12x} &
%       \manfntsymbol{'120} & \manfntsymbol{'121} &
%       \manfntsymbol{'122} & \manfntsymbol{'123} &
%       \manfntsymbol{'124} & \manfntsymbol{'125} &
%       \manfntsymbol{'126} & \manfntsymbol{'127} & 
%       \texttt{"5x} \\ \cline{1-9}
%       \textit{'13x} &
%       \manfntsymbol{'130} & \manfntsymbol{'131} &
%       \manfntsymbol{'132} & \manfntsymbol{'133} &
%       \manfntsymbol{'134} & \manfntsymbol{'135} &
%       \manfntsymbol{'136} & \manfntsymbol{'137} & 
%       \texttt{"5y} \\ \hline
%       \textit{'14x} &
%       \manfntsymbol{'140} & \manfntsymbol{'141} &
%       \manfntsymbol{'142} & \manfntsymbol{'143} &
%       \manfntsymbol{'144} & \manfntsymbol{'145} &
%       \manfntsymbol{'146} & \manfntsymbol{'147} & 
%       \texttt{"6x} \\ \cline{1-9}
%       \textit{'15x} &
%       \manfntsymbol{'150} & \manfntsymbol{'151} &
%       \manfntsymbol{'152} & \manfntsymbol{'153} &
%       \manfntsymbol{'154} & \manfntsymbol{'155} &
%       \manfntsymbol{'156} & \manfntsymbol{'157} & 
%       \texttt{"6y} \\ \hline
%       \textit{'16x} &
%       \manfntsymbol{'160} & \manfntsymbol{'161} &
%       \manfntsymbol{'162} & \manfntsymbol{'163} &
%       \manfntsymbol{'164} & \manfntsymbol{'165} &
%       \manfntsymbol{'166} & \manfntsymbol{'167} & 
%       \texttt{"7x} \\ \cline{1-9}
%       \textit{'17x} &
%       \manfntsymbol{'170} & \manfntsymbol{'171} &
%       \manfntsymbol{'172} & \manfntsymbol{'173} &
%       \manfntsymbol{'174} & \manfntsymbol{'175} &
%       \manfntsymbol{'176} & \manfntsymbol{'177} & \rule[-3ex]{0pt}{5ex}
%       \texttt{"7y}\\\hline
%       & \texttt{"8} & \texttt{"9} & \texttt{"A} & \texttt{"B}
%       & \texttt{"C} & \texttt{"D} & \texttt{"E} & \texttt{"F}
%       & \texttt{y}
%     \end{tabular}
%   \end{center}
% \end{table}
% \begin{table}[p]
%   \begin{center}
%     \caption{Symbol names in \texttt{manfnt}}
%     \begin{tabular}{clcl}
%       \multicolumn{4}{l}{\subcaption{Pen nibs}}\\[\subcaptionsep]
%       \manhpennib & \cmd{\manhpennib} &
%       \mantiltpennib & \cmd{\mantiltpennib} \\
%       \manvpennib & \cmd{\manvpennib} & & \\[\subtablesep]
%       \multicolumn{4}{l}{\subcaption{Triangles}} \\[\subcaptionsep]
%       \mantriangleup & \cmd{\mantriangleup} &
%       \mantriangleright & \cmd{\mantriangleright}\\
%       \mantriangledown & \cmd{\mantriangledown} & & \\[\subtablesep]
%       \multicolumn{4}{l}{\subcaption{Kidney beans}} \\[\subcaptionsep]
%       \mankidney & \cmd{\mankidney} &
%       \manpenkidney & \cmd{\manpenkidney} \\
%       \manboldkidney & \cmd{\manboldkidney} &
%       \manlhpenkidney & \cmd{\manlhpenkidney} \\[\subtablesep]
%       \multicolumn{4}{l}{\subcaption{Circle variations}} \\[\subcaptionsep]
%       \manquartercircle & \cmd{\manquartercircle} &
%       \manfilledquartercircle & \cmd{\manfilledquartercircle} \\
%       \manrotatedquartercircle & \cmd{\manrotatedquartercircle} &
%       \mancone & \cmd{\mancone} \\
%       \manconcentriccircles & \cmd{\manconcentriccircles} &
%       \manconcentricdiamond & \cmd{\manconcentricdiamond} \\[\subtablesep]
%       \multicolumn{4}{l}{\subcaption{Cubes}}\\[\subcaptionsep]
%       \mancube & \cmd{\mancube} &
%       \manimpossiblecube & \cmd{\manimpossiblecube}\\[\subtablesep]
%       \multicolumn{4}{l}{\subcaption{Quadrifoliums}}\\[\subcaptionsep]
%       \manquadrifolium & \cmd{\manquadrifolium} &
%       \manrotatedquadrifolium & \cmd{\manrotatedquadrifolium}\\[\subtablesep]
%       \multicolumn{4}{l}{\subcaption{Miscellaneous symbols}}
%       \\[\subcaptionsep]
%       \manstar & \cmd{\manstar} &
%       \manerrarrow & \cmd{\manerrarrow} \\[\subtablesep]
%       \multicolumn{4}{l}{\subcaption{Dangerous bend signs}}\\[\subcaptionsep]
%       \dbend & \cmd{\dbend} &
%       \reversedvideodbend & \cmd{\reversedvideodbend} \\[2ex]
%       \lhdbend & \cmd{\lhdbend} & \\[2ex]
%     \end{tabular}
%   \end{center}
% \end{table}
%}
%\MakeShortVerb{\|} ^^A End of \StopEventually. Restore the settings.
%
% \section{Code}
% \subsection{Identification part}
% \iffalse
% In principle, definitions of \fileversion and \filedate ought to be
% here.  Since we have moved them to the beginning of the file for
% practical reasons, we need to hack the beginning of this section to
% show that they are in manfnt.sty and to fix the line numbering of code.
% The block below fakes the definitions of \fileversion and \filedate.
% To trick docstrip we bury them in <false> tags and hide the tags from
% the doc module in meta-comments.
% \fi
% \iffalse
%<*false>
%\fi
% \begin{macro}{\fileversion}
% \begin{macro}{\filedate}
% For technical reasons the definitions of \cmd{\fileversion} and
% \cmd{\filedate} are left blank in the documentation.  In the actual
% \texttt{manfnt.sty} they are filled with current values (\fileversion{} and
% \filedate{} respectively).
% They are also shown on the title page of this manual. 
%    \begin{macrocode}
\def\fileversion{}
\def\filedate{}
%    \end{macrocode}
% \end{macro}
% \end{macro}
% \iffalse
%</false>
% \fi
%    \begin{macrocode}
\NeedsTeXFormat{LaTeX2e} 
\ProvidesPackage{manfnt}[\filedate \fileversion LaTeX2e manfnt package]
%    \end{macrocode}
%
% \subsection{Options}
% Currently no options are supported by this package.
%
% \subsection{Main code}
% \subsubsection{Declaration of fonts}
%    \begin{macrocode}
\DeclareFontFamily{U}{manual}{} 
\DeclareFontShape{U}{manual}{m}{n}{ <->  manfnt }{}
%    \end{macrocode}
% \begin{macro}{\manfntsymbol}
% Generally speaking \cmd{\manfntsymbol} must select font shape and
% series too, in order to work correctly, since only normal shape and
% medium series are provided.  Instead we rely implicitely on default
% behaviour of \LaTeX{} font substitution mechanism.
%    \begin{macrocode}
\newcommand{\manfntsymbol}[1]{%
    {\fontencoding{U}\fontfamily{manual}\selectfont\symbol{#1}}}
%    \end{macrocode}
% \end{macro}
%
% \subsubsection{Symbol names}
% The names with few exceptions are derived from the corresponding
% definitions in \texttt{manfnt.mf}.
%    \begin{macrocode}
\newcommand{\manhpennib}{\manfntsymbol{21}}
\newcommand{\mantiltpennib}{\manfntsymbol{22}}
\newcommand{\manvpennib}{\manfntsymbol{23}}
\newcommand{\mankidney}{\manfntsymbol{17}}
\newcommand{\manboldkidney}{\manfntsymbol{18}}
\newcommand{\manpenkidney}{\manfntsymbol{19}}
\newcommand{\manlhpenkidney}{\manfntsymbol{20}}
\newcommand{\manquartercircle}{\manfntsymbol{32}}
\newcommand{\manfilledquartercircle}{\manfntsymbol{33}}
\newcommand{\manrotatedquartercircle}{\manfntsymbol{34}}
\newcommand{\mancone}{\manfntsymbol{35}}
\newcommand{\manconcentriccircles}{\manfntsymbol{36}}
\newcommand{\manconcentricdiamond}{\manfntsymbol{37}}
\newcommand{\mantriangleright}{\manfntsymbol{120}}% Triangle for exercises
\newcommand{\mantriangleup}{% Upper triangle for Addison-Wesley logo
  \manfntsymbol{54}}
\newcommand{\mantriangledown}{% Lower triangle for Addison-Wesley logo
  \manfntsymbol{55}}
\newcommand{\mancube}{\manfntsymbol{28}}
\newcommand{\manimpossiblecube}{\manfntsymbol{29}}
\newcommand{\manquadrifolium}{\manfntsymbol{38}}% \fouru
\newcommand{\manrotatedquadrifolium}{\manfntsymbol{39}}% \fourc
\newcommand{\manstar}{\manfntsymbol{30}}% Bicentennial star
\newcommand{\manerrarrow}{\manfntsymbol{121}}% Arrow for errata lists
\newcommand{\dbend}{\manfntsymbol{127}}% Z-shaped
\newcommand{\lhdbend}{\manfntsymbol{126}}% Lefthanded (S-shaped)
\newcommand{\reversedvideodbend}{\manfntsymbol{0}}% Reversed video
%    \end{macrocode}
%
% \subsubsection{Other commands}
% \begin{macro}{\textdbend}
% \begin{macro}{\textlhdbend}
% \begin{macro}{\textreversedvideodbend}
%   User level commands \cmd{\textdbend}, \cmd{\textlhdbend}
%   and \cmd{\textreversedvideodbend} just provide the appropriate
%   argument to~\cmd{\text@dbend} which does the actual work.
%    \begin{macrocode}
\newcommand{\textdbend}{\text@dbend{\dbend}}
\newcommand{\textlhdbend}{\text@dbend{\lhdbend}}
\newcommand{\textreversedvideodbend}{\text@dbend{\reversedvideodbend}}
%    \end{macrocode}
% \end{macro}
% \end{macro}
% \end{macro}
%
% \begin{macro}{\test@dbend} 
% The source in \texttt{manfnt.mf} specifies \textdbend{}  in cmr10
% settings as a box 500/36pt~wide, 270/36pt~high and 0pt~deep.
% The actual image sticks out to~20/36pt${}+\epsilon$ above and
% to~11pt below the box. 
% At 10pt size |\dbend| produces a box with height 7.5pt and zero depth.
% The sign pole extends to 11pt below the base line\footnote{The
%   ``dangerous bend'' paragraphs in the \TeX{}book~\cite{book:knuth} are
%   typeset in 9pt size fonts with \cmd{\baselineskip} set to 11pt.}.
% All this means that we need to raise the sign by 22/15 of its height.
%    \begin{macrocode}
\newlength{\dbend@height}
\newcommand{\text@dbend}[1]{%
  \settoheight{\dbend@height}{#1}%
  \divide\dbend@height by 15%
  \multiply\dbend@height by 22%
  \raisebox{\dbend@height}{#1}}
%    \end{macrocode}
% This definition is clumsy but has an advantage of working.   
% Question by Denis Kosygin:
%   \begin{quote}\small
%     Can anyone explain why does
%\begin{verbatim}
%\newcommand{\text@dbend}[1]{\raisebox{22\height/15}{#1}}
%\end{verbatim}
%     break?  Sometimes it does what I mean, and sometimes it typesets
%     the argument raised to to 22\cmd{\height} followed by /15 on the
%     baseline.
%   \end{quote}
% \end{macro}
%
% We end the file with an explicit |\endinput| which prevents the
% docstrip program from putting character table into the generated
% files.
%    \begin{macrocode}
\endinput
%    \end{macrocode}
%% \CharacterTable
%%  {Upper-case    \A\B\C\D\E\F\G\H\I\J\K\L\M\N\O\P\Q\R\S\T\U\V\W\X\Y\Z
%%   Lower-case    \a\b\c\d\e\f\g\h\i\j\k\l\m\n\o\p\q\r\s\t\u\v\w\x\y\z
%%   Digits        \0\1\2\3\4\5\6\7\8\9
%%   Exclamation   \!     Double quote  \"     Hash (number) \#
%%   Dollar        \$     Percent       \%     Ampersand     \&
%%   Acute accent  \'     Left paren    \(     Right paren   \)
%%   Asterisk      \*     Plus          \+     Comma         \,
%%   Minus         \-     Point         \.     Solidus       \/
%%   Colon         \:     Semicolon     \;     Less than     \<
%%   Equals        \=     Greater than  \>     Question mark \?
%%   Commercial at \@     Left bracket  \[     Backslash     \\
%%   Right bracket \]     Circumflex    \^     Underscore    \_
%%   Grave accent  \`     Left brace    \{     Vertical bar  \|
%%   Right brace   \}     Tilde         \~}
%%
% \DeleteShortVerb{\|} ^^A For table printing
% \Finale
%^^A End of file "manfnt.dtx"

