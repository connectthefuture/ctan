%%
%% rubikexamples.tex 
%% March 03, 2017
%% Requires  rubik bundle v4.0 and also the tikz package
%% Part of the Rubik bundle (www.ctan.org/pkg/rubik)
%% authors: RWD Nickalls & A Syropoulos
%%--------NOTE-------------------------------------------
%% USAGE:   (pdf)latex  --shell-escape   rubikexamples.tex
%% run three times to get all labels and fig nos correct
%%-------------------------------------------------------
\documentclass[a4paper]{article}
%------rubik----------
\usepackage{tikz}   %% load tikz BEFORE rubikcube
\usepackage{rubikcube}
\usepackage{rubikrotation} 
\usepackage{rubikpatterns} 
%------LUA-----------
% LuaTeX is  a Unicode engine (as is XeTeX)
\usepackage{ifluatex}
\ifluatex
  \usepackage{fontspec}
\fi
%--------------------
\usepackage{url}
\usepackage{ifpdf}
\usepackage{lmodern}
%---------------------
\ifpdf
  \usepackage[verbose]{microtype}
  \usepackage{cmap}
  \usepackage[pdfencoding=auto]{hyperref}

  \hypersetup{%
     pdftitle={rubikexamples.pdf},
     pdfsubject={Rubik cube, Rubik bundle},
     pdfkeywords={Rubik cube LaTeX}
     }
\fi
%%-------------------


%-------------
\pagestyle{myheadings}
\markright{\texttt{rubikexamples.pdf} \ \ 
(Rubik bundle v4.0,  March 03, 2017) \hspace{0.5cm}\texttt{www.ctan.org/pkg/rubik}}
%-----------
%% brace and bracket 
\newcommand{\Rubikbracket}[1]{$\left(\mbox{#1}\right)$}
\newcommand{\Rubikbrace}[1]{$\left\{\mbox{#1}\right\}$}
\newcommand{\cubenumber}[1]{\strut\raisebox{1cm}{#1}}
\newcommand{\leftface}{\textsc{left}\ face}
\newcommand{\rightface}{\textsc{right}\ face}
\newcommand{\upface}{\textsc{up}\ face}
\newcommand{\downface}{\textsc{down}\ face}
\newcommand{\frontface}{\textsc{front}\ face}
\newcommand{\backface}{\textsc{back}\ face}
%---------------------------------
%% increase text height and width  
\addtolength{\oddsidemargin}{-1.5cm}
\addtolength{\textwidth}{3cm}
\addtolength{\textheight}{1cm}
%----------------------
\begin{document}

\ifpdf\pdfbookmark[1]{Title}{Title}\fi
\title{Rubik examples\\ 
\smallskip\normalsize\texttt{www.ctan.org/tex-archives/macros/latex/contrib/rubik/rubikexamples.pdf}
\,\footnote{This file is part of the Rubik bundle. To generate this file, 
use the following command: \newline \texttt{\$ pdflatex --shell-escape  rubikexamples.tex}}}
\author{RWD Nickalls\,\footnote{email: \textsf{dick@nickalls.org}}%
  \ \ \& A Syropoulos\,\footnote{email: \textsf{asyropoulos@yahoo.com}}}
\date{03 March 2017 (v4.0)}
\maketitle


%%------------------
\section{Preliminaries}


These  examples  were generated using  the \TeX\ 
Rubik bundle\,\footnote{\url{http://www.ctan.org/pkg/rubik}} v4.0. They assume some  
familiarity with the three complementary packages \textsc{rubikcube}, 
\textsc{rubikrotation} and \textsc{rubikpatterns}. For~documentation see the files:

\begin{quote}
\begin{verbatim}
rubikcube.pdf
rubikrotation.pdf
rubikpatterns.pdf
rubikpatternsLIST.pdf
\end{verbatim}
\end{quote}


This file requires the following packages: \texttt{tikz, rubikcube, rubikrotation,
rubikpatterns}; note that the \texttt{tikz} package must be loaded \textit{before} 
the \texttt{rubikcube} package.

This file needs to be run using  the \verb!--shell-escape! command-line option; 
for example:
\begin{quote}
\begin{verbatim}
pdflatex  --shell-escape  rubikexample.tex
\end{verbatim}
\end{quote}
This is because nearly all the examples  make use of the  \verb!\RubikRotation! 
command, which  calls the Perl script  \texttt{rubikrotation.pl}.  
If you do forget to use the command-line switch, the file will still run, but 
all the cubes will remain in the initial unprocessed configuration. 


\subsection{Environments}

When using the Rubik bundle one sometimes needs to be mindful of the various 
\LaTeX\ environments in which Rubik commands are placed (e.g.,~the figure, 
minipage and TikZ picture environments), since these environments restrict 
the actions of commands they contain to the particular environment. 
The \verb!\ShowCube! command is also relevant here, since it is a 
minipage-wrapper for the TikZ picture environment. Only Rubik  \verb!\Draw..! 
commands actually need to be inside a TikZ picture environment.

This issue arises because the Rubik bundle allows you to create figures 
showing different stages  during a sequence of rotations. Consequently 
the effects of  commands executed inside an environment (especially 
commands which determine the colour-state or rotations), may not  be 
apparent to subsequent commands outside that particular environment. 
See Example~3  for an illustration of how to handle environments.

\pagebreak

%%------------------
\section{Examples}

\subsection{Sixspot}
\typeout{---Example (sixspot)}

In Figure~\ref{fig:cubeincube} we show  the so-called ``sixspot'' configuration,
 generated from a solved cube using the rotation sequence 
\rr{U}, \rr{Dp}, \rr{R}, \rr{Lp}, \rr{F}, \rr{Bp}, \rr{U}, \rr{Dp}.

%%--------FIGURE 1 -----------
\begin{figure}[hbt]
\centering
\RubikCubeSolved
\RubikRotation{\sixspot}
\ShowCube{7cm}{0.7}{\DrawRubikCubeSF}
\caption{\label{fig:cubeincube}The `sixspot' configuration.}
\end{figure}
%----------------

{\noindent}Creating a macro to hold  a rotation sequence greatly facilitates 
their use, as follows:
\begin{quote}
\begin{verbatim}
\newcommand{\sixspot}{[sixspot],U,Dp,R,Lp,F,Bp,U,Dp}
\end{verbatim}
\end{quote}
We can now process this sequence  using its macro name as an argument for the 
\verb!\RubikRotation! command. The code used for the above Figure  uses the 
\verb!\ShowCube{}{}{}! command for which \verb!#1! is the minipage width, \verb!#2! 
is the tikzpicture scale factor (0--1), and \verb!#3!  can include  \textsc{rubikcube}
package  \verb!\Draw..! commands,  and any commands which are valid for use in a 
TikZ \texttt{tikzpicture} environment. The code for the above figure is as follows:
\begin{quote}
\begin{verbatim}
\begin{figure}[hbt]
   \centering
   \RubikCubeSolved
   \RubikRotation{\sixspot}
   \ShowCube{7cm}{0.7}{\DrawRubikCubeSF}
\caption{...}
\end{figure}
\end{verbatim}
\end{quote}

Note that the sixspot sequence is a so-called `order~3' sequence, which means that 
running the `sixspot' sequence 3 times returns the cube to its original `solved' 
state. The~command for processing it three times is  \verb!\RubikRotation[3]{\sixspot}!.

Note that the semi-flat form of the cube here is generated by the
 \verb!\DrawRubikCubeSF! command, where the terminal SF denotes the Semi-Flat form.



\subsubsection{Log-file extract}

Users may find it instructive to inspect the the log-file and follow the dynamic 
interaction between \LaTeX\  and the Perl script. 
This is easy to follow, since output by \texttt{rubikrotation.sty} is prefixed 
by 3 dashes (\verb!---!), while output by the Perl script is prefixed by 
3 dots (\verb!...!). Search for the keyword `Example', as this is written to 
the log-file at the start of each example. 

The following is the log-file extract associated with Example~1 (from the 
author's Debian Linux platform).

\begin{quote}
\begin{verbatim} 
---Example (sixspot)
LaTeX Font Info:    Try loading font information for T1+cmss on input line 134.

(/usr/local/texlive/2016/texmf-dist/tex/latex/base/t1cmss.fd
File: t1cmss.fd 2014/09/29 v2.5h Standard LaTeX font definitions
)
---TeX process (rubikrotation.sty)-------
---NEW rotation command------------------
---command = RubikRotation[1]{[SixSpot],U,Dp,R,Lp,F,Bp,U,Dp,<(8q*, 8f*)>}
---writing current Rubik state to file rubikstate.dat
\openout7 = `rubikstate.dat'.

\ourRRcounter=\count121
---CALLing Perl script (rubikrotation.pl)
runsystem(perl rubikrotation.pl -i rubikstate.dat -o rubikstateNEW.dat)...execu
ted.

---inputting NEW datafile (data written by Perl script)
(./rubikstateNEW.dat

...PERL process..................................
...script = rubikrotation.pl v4.0 (03 March 2017)
...reading the current rubik state (from File: rubikstate.dat)
...up,W,W,W,W,W,W,W,W,W
...down,Y,Y,Y,Y,Y,Y,Y,Y,Y
...left,B,B,B,B,B,B,B,B,B
...right,G,G,G,G,G,G,G,G,G
...front,O,O,O,O,O,O,O,O,O
...back,R,R,R,R,R,R,R,R,R
...
...command=checkstate
...checking state of cube
...cubiesum = 54 (Red=9, Or=9, Ye=9, Gr=9, Bl=9, Wh=9, X=0)
...
...command=rotation,[SixSpot],U,Dp,R,Lp,F,Bp,U,Dp,<(8q*, 8f*)>
...dataline = rotation,[SixSpot],U,Dp,R,Lp,F,Bp,U,Dp,<(8q*; 8f*)>
...[SixSpot] is a label OK
...rotation U, OK
...rotation Dp, OK 
...rotation R, OK
...rotation Lp, OK
...rotation F, OK
...rotation Bp, OK
...rotation U, OK
...rotation Dp, OK 
...writing new Rubik state to file rubikstateNEW.dat
...SequenceName = SixSpot
...SequenceInfo = (8q*; 8f*)
...SequenceShort = [SixSpot],U,Dp,R,Lp,F,Bp,U,Dp
...SequenceLong = U,Dp,R,Lp,F,Bp,U,Dp
)
\end{verbatim}
\end{quote}



\pagebreak
%%-------------------------

\subsection{ShowRubikErrors}
\typeout{---Example (ShowRubikErrors)}


 In this example we demonstrate  the use of the  \verb!\ShowRubikErrors! command, 
 which  places a copy of  the Perl output file \verb!rubikstateERRORS.dat! 
 underneath the graphic so you can see a list of all the  errors, if any. 
 Note that this example is similar to the previous one except that we have  
 introduced several  errors---e.g.,~bad minipage width,  typos, as well as 
 some animals---into  the rotation sequence). 
 It is important to note that  the  \verb!\ShowRubikErrors! command must be 
 placed \textit{after}  the TikZ picture environment (i.e.,~in this case after 
the \verb!\ShowCube! command), or even at the end of the document. 
Note that full details of all errors are also included in the .log file. 

%%--------FIGURE 2 -----------

\RubikCubeSolved
\RubikRotation{[sixspot],U,Dp,R,Lp,F,Bp,U,Dpppp,cat,dog}

\begin{figure}[hbt]
  \centering
  \ShowCubeF{4.5cm}{0.5}{\DrawRubikCubeF}
  \ShowRubikErrors
\parbox{0.8\textwidth}{%
\caption{\label{fig:cubeincubeE}The same `sixspot' sequence of rotations as shown 
in Example~1,  but now with some   errors (wrong minipage width, typos and some 
animals\,!) in the rotation sequence (it \textit{should} be just 
\texttt{U,Dp,R,Lp,F,Bp,U,Dp}).}}
\end{figure}

{\noindent}In this example we have used the F version of the \verb!\ShowCube! 
command (\verb!\ShowCubeF!)  which places an fbox around the image so you can see the 
extent of any white space etc. This reveals that the set minipage-width ($4.5$cm) in 
the \verb!\ShowCubeF! command---see code below---is too narrow: it should be 
5cm ($10 \times 0.5$) to just include the whole image (i.e.,~$10 \times$ the TikZ 
scale-factor in this case). 
Once fixed, we can remove the F from the \verb!\ShowCubeF! command.
Note also  that only `\verb!\Draw...!' commands really need to be inside the TikZ 
picture environment (i.e.,~inside the \verb!\ShowCube! command). The above figure 
was generated by the following code.
%-----------------
\begin{quote}
\begin{verbatim}
\RubikCubeSolved
\RubikRotation{[sixspot],U,Dp,R,Lp,F,Bp,U,Dpppp,cat,dog}
\begin{figure}[hbt]
  \centering
  \ShowCubeF{4.5cm}{0.7}{\DrawRubikCubeF}
  \ShowRubikErrors
\caption{....}
\end{figure}
\end{verbatim}
\end{quote}
Even if the \verb!\ShowRubikErrors! command is not used, it is always a good idea 
to check the file \verb!rubikstateERRORS.dat! after a \LaTeX\ run, since this 
file will also reveal any  errors.

Note that the completely flat form of the cube here is generated by the
 \verb!\DrawRubikCubeF! command, where the terminal F denotes the Flat form.



\pagebreak
%----------------------

\subsection{Environments}
\typeout{---Example (Environments)}


In this example we highlight the fact that Rubik commands used inside a \LaTeX\ 
environment remain local to that environment, and how this can sometimes  be 
problematic.  Commands whose reach is meant to be  more global need to be 
executed outside such environments, where they can implement  global colour 
settings, which will then be accessible to Rubik \verb!\Draw..! commands inside 
subsequent environments.

Since we are drawing images, this is primarily an issue 
with the   the \verb!\minipage!, \verb!\figure!, and TikZ picture environments.
Consequently, it is generally best when drawing a sequence of  cubes  to reserve 
the TikZ picture environment  only  for Rubik \verb!\Draw..! commands and TikZ 
commands. Importantly, this  also applies to the commonly used  \verb!\ShowCube! 
command, since this is a minipage-wrapper for the TikZ  picture environment 
(see the \textsc{rubikcube} package documentation).

In this example the first cube (9a) uses a \verb!\RubikCubeSolved! command 
\textit{inside} a \verb!\ShowCube! environment. However, if we now perform the 
rotation  \textRubik{R}\ (using the command \verb!\RubikRotation{R}!) this results 
in  a quite unexpected effect on cube (9b) (and obviously not correct). This is because 
 the effect of the initial \verb!\RubikCubeSolved! command 
 (setting a new colour-state) is not  visible outside its \verb!\ShowCube! 
 environment, and hence the subsequent \verb!\RubikRotation{R}! command is unaware 
 of this recent attempt to update  the global colour-state information. 
 It turns out that this was actually last updated following the action of the \textbackslash\texttt{RubikRotation\{[sixspot],....\}} command used in 
Example~2, being the last colour-state command executed \textit{outside} an 
environment (a \texttt{figure}  environment in that example). Consequently, 
the strange form of cube (9b) is not what we expected.


\medskip
\noindent%
\cubenumber{9a}%
\ShowCube{2cm}{0.5}{%
    \RubikCubeSolved%
    \DrawRubikCubeRU}%
\quad\Rubik{R}%    
\RubikRotation{R}%
\cubenumber{9b}%
\ShowCube{2cm}{0.5}{\DrawRubikCubeRU}%
%---------------
\hspace{5mm}
\begin{minipage}{0.5\textwidth}
\begin{verbatim}
\usepackage{tikz,rubikcube,rubikrotation}
\newcommand{\cubenumber}[1]{\strut\raisebox{1cm}{#1}}
...
\cubenumber{9a}%
\ShowCube{2cm}{0.5}{%
    \RubikCubeSolved%
    \DrawRubikCubeRU%
     }%
\quad\Rubik{R}%
\RubikRotation{R}%
\cubenumber{9b}%
\ShowCube{2cm}{0.5}{\DrawRubikCubeRU}%
\end{verbatim}
\end{minipage}

\bigskip

If we now bring the \verb!\RubikCubeSolved! command out and place it before 
the \verb!\ShowCube! command then its  `state' information  becomes globally 
accessible (i.e.,~colour-state gets updated), and  therefore  gets used by the  
\verb!\RubikRotation{R}! command, and  hence cube (9d)  is rendered correctly.

\bigskip

\noindent%
\RubikCubeSolved%
\cubenumber{9c}%
\ShowCube{2cm}{0.5}{\DrawRubikCubeRU}%
\quad\Rubik{R}%
\RubikRotation{R}%
\cubenumber{9d}%
\ShowCube{2cm}{0.5}{\DrawRubikCubeRU}%
%---------------
\hspace{5mm}
\begin{minipage}{0.5\textwidth}
\begin{verbatim}
\RubikCubeSolved%
\cubenumber{9c}%
\ShowCube{2cm}{0.5}{\DrawRubikCubeRU}%
\quad\Rubik{R}%
\RubikRotation{R}%
\cubenumber{9d}%
\ShowCube{2cm}{0.5}{\DrawRubikCubeRU}%
\end{minipage}
\end{verbatim}
\end{minipage}




\pagebreak
%----------------------
\subsection{Coordinates}
\typeout{---Example (coordinates)}


For all cubes the origin of coordinates is defined as the bottom left corner of 
the \frontface. Consequently, it is easy to determine the coordinates of points 
and hence draw lines, circles, and place lettering or other objects using the 
standard TikZ \verb!\draw..! and \verb!\node..! commands. Note that for 
convenience  point $P$ is designed to be $(-1,-1)$ on the 2D view.
(The~following diagram  is Fig~1 from the \textsc{rubikcube} package documentation).


\begin{figure}[hbt]
\centering
\RubikFaceFrontAll{X}   % X = default = grey
\ShowCube{5.6cm}{0.7}{%
  \DrawFlatFront

\draw[line join=round,line cap=round,ultra thick] (0,0) -- (0,4); % Y
\draw[line join=round,line cap=round,ultra thick] (0,0) -- (4,0); % X

%draw labels
\node (Ylabel) at (-0.35,  3.8)  {$Y$};
\node (Xlabel) at ( 3.8,  -0.4)  {$X$};


%%----------draw the Rubik faces--------
%% outline Front face
\draw[line join=round,line cap=round,ultra thick]%
     (0,3) -- (-1,2) -- (-1,-1) -- (2,-1) -- (3,0);
%%
%% outline  Down face
\draw[line join=round,line cap=round,ultra thick]%
     (-1,-1) -- (0, 0);
%%
%%
%% draw labels
\node (Y0) at (-0.4, 0)  [blue]{$0$};
\node (Y1) at (-0.4, 1)  [blue]{$1$};
\node (Y2) at (-0.4, 2)  [blue]{$2$};
\node (Y3) at (-0.4, 3)  [blue]{$3$};
%%
\node (X0) at (0, -0.5)  [blue]{$0$};
\node (X1) at (1, -0.5)  [blue]{$1$};
\node (X2) at (2, -0.5)  [blue]{$2$};
\node (X3) at (3, -0.5)  [blue]{$3$};
%%
\node (P) at (-2.4, -1)  [blue]{$P\,(-1,-1)$};
}
\end{figure}



{\noindent}The code for the figure is given below.

We  draw everything in the  \verb!\ShowCube! environment; the \frontface\  
in grey (colour code = \texttt{X}) using the Rubik command 
\verb!\DrawFlatFront{X}!, and then draw all the lines and text using standard 
TikZ commands. The correct  minipage-width argument (5.6cm) for the 
\verb!\ShowCube! command is determined by trial-and-error, using the `fbox' 
form of the command (\verb!\ShowCubeF!), and then the `F' is removed 
($\rightarrow$ \verb!\ShowCube!). In order to avoid confusion, all Rubik commands 
start with a capital letter (e.g.,~\verb!\Draw..!), while all TikZ commands 
start with a lower-case letter (e.g.,~\verb!\draw..!).

\begin{quote}
\begin{verbatim}
\begin{figure}[hbt]
\centering
\RubikFaceFrontAll{X}% X = default non-colour (grey)
\ShowCube{5.6cm}{0.7}{%
   \DrawFlatFront
   \draw[line join=round,line cap=round,ultra thick] (0,0) -- (0,4);% Yaxis
   \draw[line join=round,line cap=round,ultra thick] (0,0) -- (4,0);% Xaxis
   \node (Ylabel) at (-0.35,  3.8)  {$Y$};
   \node (Xlabel) at ( 3.8,  -0.4)  {$X$};
   %% outline Left and Down faces
   \draw[line join=round,line cap=round,ultra thick]%
        (0,3) -- (-1,2) -- (-1,-1) -- (2,-1) -- (3,0);
   \draw[line join=round,line cap=round,ultra thick]%
        (-1,-1) -- (0, 0);
   \node (Y0) at (-0.4, 0)  [blue]{$0$};
   \node (Y1) at (-0.4, 1)  [blue]{$1$};
   \node (Y2) at (-0.4, 2)  [blue]{$2$};
   \node (Y3) at (-0.4, 3)  [blue]{$3$};
   \node (X0) at (0, -0.5)  [blue]{$0$};
   \node (X1) at (1, -0.5)  [blue]{$1$};
   \node (X2) at (2, -0.5)  [blue]{$2$};
   \node (X3) at (3, -0.5)  [blue]{$3$};
   \node (P) at (-2.4, -1)  [blue]{$P\,(-1,-1)$};
   }
\end{figure}
\end{verbatim} 
\end{quote}




\pagebreak
%----------------------
\subsection{Face notation}
\typeout{---Example (face notation)}


%% = my Fig A (rubikcube.dtx)

The following diagram  is Fig~2 from the \textsc{rubikcube} package documentation.

\begin{figure}[htb]
\centering%
\RubikFaceUpAll{Y}
\RubikFaceFrontAll{G}
\RubikFaceRightAll{O}
\ShowCubeF{5cm}{0.6}{%
\DrawRubikCubeRU%
%%-----------------

%% Right
\draw[line width=2pt,color=blue,->] (3.5,2) -- (5.3, 2);
\node (R) at (4.6, 2.5)  [blue]{\textbf{\textsf{R}}};
\node (x) at (5.8, 2)  [blue]{\textbf{\textsf{x}}};

%%Left
\draw[line width=2pt,color=blue] (-0.2,2) -- (-1.5, 2);
\node (L) at (-1.1, 2.5)  [blue]{\textbf{\textsf{L}}};

%%Up
\draw[line width=2pt,color=blue,->] (2, 3.5) -- (2, 5.5);
\node (U) at (1.4, 4.7)  [blue]{\textbf{\textsf{U}}};
\node (y) at (2, 6.1)  [blue]{\textbf{\textsf{y}}};

%%Down
\draw[line width=2pt,color=blue] (2, -0.2) -- (2, -1.5);
\node (D) at (2.6, -1.1)  [blue]{\textbf{\textsf{D}}};


%%Front
\draw[line width=2pt,color=blue,->] (1.5, 1.5) -- (0, -1);
\node (F) at (0.7, -0.7)  [blue]{\textbf{\textsf{F}}};
\node (z) at (-0.3, -1.4)  [blue]{\textbf{\textsf{z}}};

%%Back
\draw[line width=2pt,color=blue] (3.2, 4.2) -- (4, 5.5);
\node (B) at (4.4, 5)  [blue]{\textbf{\textsf{B}}};
%%
}
\end{figure}

{\noindent}The code for the figure is as follows 
(the origin $(0,0)$ is at the bottom left corner of the \frontface).


\begin{quote}
\begin{verbatim}
\begin{figure}[htb]
\centering%
\RubikFaceUpAll{Y}
\RubikFaceFrontAll{G}
\RubikFaceRightAll{O}
\ShowCubeF{5cm}{0.6}{%
  \DrawRubikCubeRU%
  %% Right
  \draw[line width=2pt,color=blue,->] (3.5,2) -- (5.3, 2);
  \node (R) at (4.6, 2.5)  [blue]{\textbf{\textsf{R}}};
  \node (x) at (5.8, 2)  [blue]{\textbf{\textsf{x}}};
  %%Left
  \draw[line width=2pt,color=blue] (-0.2,2) -- (-1.5, 2);
  \node (L) at (-1.1, 2.5)  [blue]{\textbf{\textsf{L}}};
  %%Up
  \draw[line width=2pt,color=blue,->] (2, 3.5) -- (2, 5.5);
  \node (U) at (1.4, 4.7)  [blue]{\textbf{\textsf{U}}};
  \node (y) at (2, 6.1)  [blue]{\textbf{\textsf{y}}};
  %%Down
  \draw[line width=2pt,color=blue] (2, -0.2) -- (2, -1.5);
  \node (D) at (2.6, -1.1)  [blue]{\textbf{\textsf{D}}};
  %%Front
  \draw[line width=2pt,color=blue,->] (1.5, 1.5) -- (0, -1);
  \node (F) at (0.7, -0.7)  [blue]{\textbf{\textsf{F}}};
  \node (z) at (-0.3, -1.4)  [blue]{\textbf{\textsf{z}}};
  %%Back
  \draw[line width=2pt,color=blue] (3.2, 4.2) -- (4, 5.5);
  \node (B) at (4.4, 5)  [blue]{\textbf{\textsf{B}}};
  }
\end{figure} 
\end{verbatim}
\end{quote}





\pagebreak
%----------------------
\subsection{Grey cube}
\label{sec:greycube}
\typeout{---Example (grey cube)}


When explaining elementary layer~1 moves, it can be useful to use the `grey cube' 
(\verb!\RubikCubeGrey!), as this  sets up only the central cubie on each face; 
we have shown the first cube in Semi-Flat (SF) mode simply to show  how the grey 
cube is configured  (note that an `all-grey' cube is also available: 
\verb!\RubikCubeAllGrey!). Both of these grey cube commands will also  accept 
the word `gray' (to be consistent with TikZ).

In this  example, we show how to position a single  `flipped'  white/orange edge 
cubie in the top  layer. 
  

\begin{figure}[hbt]

\centering%
\RubikCubeGrey%
\RubikFaceUp{X}{X}{X}%
            {X}{W}{O}%
            {X}{X}{X}%
               
\RubikFaceRight{X}{W}{X}
               {X}{G}{X}
               {X}{X}{X}
%
\cubenumber{1}%
\ShowCube{5cm}{0.5}{\DrawRubikCubeSF}%
%------------------
\quad\Rubik{Rp}\RubikRotation{Rp}
\cubenumber{2}%
\ShowCube{2cm}{0.5}{\DrawRubikCube}%
%-----------------------
\quad\Rubik{Fp}\RubikRotation{Fp}
\cubenumber{3}%
\ShowCube{2cm}{0.5}{\DrawRubikCube}%
%-----------------------
\end{figure}



{\noindent}The code for the figure is given below. 
After setting up the first cube, we then just use the \verb!\RubikRotation! 
command to generate the  remaining cubes.
The colours are coded as follows:  R~(red), O~(orange), Y~(yellow), 
G~(green), B~(blue),  W~(white),  and   X~(grey).

\begin{quote}
\begin{verbatim}
\usepackage{tikz,rubikcube,rubikrotation}
\newcommand{\cubenumber}[1]{\strut\raisebox{1cm}{#1}}
...
\begin{figure}[hbt]
\centering
% set up the first cube
\RubikCubeGrey%
\RubikFaceUp{X}{X}{X}%
            {X}{W}{O}%
            {X}{X}{X}%

\RubikFaceRight{X}{W}{X}
               {X}{G}{X}
               {X}{X}{X}
\cubenumber{1}%
\ShowCube{5cm}{0.5}{\DrawRubikCubeSF}%
%
\quad\Rubik{Rp}\RubikRotation{Rp}
\cubenumber{2}%
\ShowCube{2cm}{0.5}{\DrawRubikCube}%
%
\quad\Rubik{Fp}\RubikRotation{Fp}
\cubenumber{3}%
\ShowCube{2cm}{0.5}{\DrawRubikCube}%
\end{figure} 
\end{verbatim}
\end{quote}




\pagebreak

%-------------------------------
\subsection{Scramble a cube}
\label{sec:scrambleacube}
\typeout{---Example (scramble a cube)}


In this example we use the \verb!\RubikRotation! command to scramble a `solved' 
Rubik cube via  a sequence of 120 random rotations, using the following  command 
(the details of the process can be seen in the \texttt{.log} file):
\begin{quote}
\begin{verbatim}
\RubikRotation{random,120}
\end{verbatim}
\end{quote}
On this occasion we draw the cube Flat (F) using the command \verb!\DrawRubikCubeF!.
In this example, we also make use of the \verb!\SaveRubikState{}! command to  save 
the final  configuration (state) to a file (\texttt{rubikexampfig4.tex}) using 
\verb!\SaveRubikState{rubikexampfig4.tex}!,
so we can  display the  same cube configuration later  but in  a different format
(we show it again in the following example (Example~\ref{sec:saverubikstate}).
Note that since we are using a random sequence, it follows that each 
time this file is run not only will a visually different  cube  be generated,  
but the same state  will be shown in both 
here and in Example~\ref{sec:saverubikstate}.
%%--------FIGURE 3 -----------
\begin{figure}[hbt]
  \centering
  \RubikCubeSolved
  \RubikRotation{random,120}
  \SaveRubikState{rubikexampfig4.tex}  
  \ShowCubeF{6cm}{0.5}{\DrawRubikCubeF}
\parbox{0.7\textwidth}{%
\caption{\label{fig:randomone}This shows a cube  generated by 120 random rotations}}
\end{figure}
%----------------
\begin{quote}
\begin{verbatim}
\usepackage{tikz,rubikcube,rubikrotation}
...
\begin{figure}[hbt]
  \centering
  \RubikCubeSolved
  \RubikRotation{random,120}
  \SaveRubikState{rubikexampfig4.tex}  
  \ShowCubeF{6cm}{0.5}{\DrawRubikCubeF}
\caption{....}
\end{figure}
\end{verbatim}
\end{quote}

{\noindent}Q: How do we determine the minipage-width ($7.2$cm) in the 
\verb!\ShowCube! command?

\smallskip

{\noindent}A: The object is 12~cubie squares  wide. Since the TikZ scale-factor 
argument of the  \verb!\ShowCube! command (cms/unit length; default = 1) in this 
case is set to~$0.5$, then the true width of  the image will 
be $12 \times 0.5 = 6$\,cm. Note that here we have used the \verb!\ShowCubeF!
command and so we can see that this is correct. Changing the scale-factor will 
change the image size and hence a new width argument will be required to just fit 
the image.  

Note that in this particular case (where there is only a single image in the 
`figure' environment), since the \verb!\ShowCube! command places the image 
(in a TikZ picture environment) centrally  inside a minipage, the image will 
in fact be centrally placed in  the \verb!\textwidth! provided the image 
is \textit{smaller} than the fbox---i.e.,~if we used instead a minipage-width of, 
say, 12\,cm  the image would still  appear centred in the \verb!\textwidth! in 
this case. However, when there are several images in the `figure', then the spacing 
may appear strange unless each image closely fits its own minipage-width etc.
It is often useful, therefore, to check the size of the fbox (using the 
\verb!\ShowCubeF! command) as we have done here.
   
  
 
\pagebreak

%----------------------
\subsection{SaveRubikState}
\label{sec:saverubikstate}
\typeout{---Example (SaveRubikState)}

In this example we display a cube having the same state as that shown in 
the previous example (Example~\ref{sec:scrambleacube}). The configuration state 
was saved from Figure~\ref{fig:randomone} using the  command 
\verb!\SaveRubikState{rubikexampfig4.tex}!, and then input here  
using \verb!%% filename: rubikexampfig4.tex%
\RubikFaceUp{G}{B}{R}{W}{O}{Y}{Y}{O}{R}%
\RubikFaceDown{O}{B}{Y}{R}{R}{R}{W}{O}{W}%
\RubikFaceLeft{Y}{G}{R}{B}{B}{O}{O}{Y}{B}%
\RubikFaceRight{W}{G}{Y}{W}{G}{G}{O}{W}{R}%
\RubikFaceFront{G}{G}{B}{Y}{Y}{B}{W}{R}{B}%
\RubikFaceBack{B}{O}{O}{R}{W}{Y}{G}{W}{G}%
!. These commands therefore allow  the 
state of a previous cube to be saved to a file, and then  displayed again 
later in  a different format.
%%-----------FIGURE 4 -----------
\begin{figure}[hbt]
  \centering
  %% filename: rubikexampfig4.tex%
\RubikFaceUp{G}{B}{R}{W}{O}{Y}{Y}{O}{R}%
\RubikFaceDown{O}{B}{Y}{R}{R}{R}{W}{O}{W}%
\RubikFaceLeft{Y}{G}{R}{B}{B}{O}{O}{Y}{B}%
\RubikFaceRight{W}{G}{Y}{W}{G}{G}{O}{W}{R}%
\RubikFaceFront{G}{G}{B}{Y}{Y}{B}{W}{R}{B}%
\RubikFaceBack{B}{O}{O}{R}{W}{Y}{G}{W}{G}%

  \ShowCube{4cm}{0.8}{\DrawRubikCubeLD}
\parbox{0.7\textwidth}{%
\caption{\label{fig:randomtwo}This shows a Rubik cube in exactly the  same 
state as  the one shown in Figure~\ref{fig:randomone}} }
\end{figure}
%%---------------------------------
\begin{quote}
\begin{verbatim}
\usepackage{tikz,rubikcube,rubikrotation}
...
\begin{figure}[hbt]
  \centering
  %% filename: rubikexampfig4.tex%
\RubikFaceUp{G}{B}{R}{W}{O}{Y}{Y}{O}{R}%
\RubikFaceDown{O}{B}{Y}{R}{R}{R}{W}{O}{W}%
\RubikFaceLeft{Y}{G}{R}{B}{B}{O}{O}{Y}{B}%
\RubikFaceRight{W}{G}{Y}{W}{G}{G}{O}{W}{R}%
\RubikFaceFront{G}{G}{B}{Y}{Y}{B}{W}{R}{B}%
\RubikFaceBack{B}{O}{O}{R}{W}{Y}{G}{W}{G}%

  \Showcube{4cm}{0.8}{\DrawRubikCubeLD}
\caption{....} 
\end{figure}
\end{verbatim}
\end{quote}




\pagebreak
%----------------------
\subsection{Series of cubes}
\typeout{---Example (series of cubes)}

Here we show a convenient way of displaying a series of small cubes showing 
a sequence of rotations (\rr{U}, \rr{R}, \rr{F}). 


\begin{figure}[hbt]
\centering
\RubikCubeSolved 
\ShowCube{2cm}{0.5}{\DrawRubikCubeRU}%
\quad\Rubik{U}\quad%
\RubikRotation{U}\ShowCube{2cm}{0.5}{\DrawRubikCubeRU}%
\quad\Rubik{R}\quad%
\RubikRotation{R}\ShowCube{2cm}{0.5}{\DrawRubikCubeRU}%
\quad\Rubik{F}\quad%
\RubikRotation{F}\ShowCube{2cm}{0.5}{\DrawRubikCubeRU}%
\caption{The rotations \rr{U}, \rr{R}, \rr{F}\ on a solved cube.} 
\end{figure}

{\noindent}The code for the above sequence is as follows:
\begin{quote}
\begin{verbatim}
\usepackage{tikz,rubikcube,rubikrotation}
...
\begin{figure}[hbt]
  \centering%
  \RubikCubeSolved%
  \ShowCube{2cm}{0.5}{\DrawRubikCubeRU}%
  \quad\Rubik{U}\quad%
  \RubikRotation{U}\ShowCube{2cm}{0.5}{\DrawRubikCubeRU}% 
  \quad\Rubik{R}\quad%
  \RubikRotation{R}\ShowCube{2cm}{0.5}{\DrawRubikCubeRU}%
  \quad\Rubik{F}\quad%
  \RubikRotation{F}\ShowCube{2cm}{0.5}{\DrawRubikCubeRU}%
\caption{The rotations \rr{U}, \rr{R}, \rr{F}\ on a solved cube.} 
\end{figure}
\end{verbatim}
\end{quote}

{\noindent}Note that we are starting with the default white-opposite-yellow (WY) 
solved cube, using the command  \verb!\RubikCubeSolved!, which is functionally  
the same as the more explicit \verb!\RubikCubeSolvedWY! (if you forget the 
terminal two letters then at least you will get a `solved' cube). 
A white-opposite-blue (WB) solved cube is available as \verb!\RubikCubeSolvedWB!.




\pagebreak
%----------------------
\subsection{Rotation sequence}
\typeout{---Example (rotation sequence)}

We now explore using the named Rubik cube rotation sequences and associated 
patterns available in the  \textsc{rubikpatterns} package---a small macro database 
(see the \textsc{rubikpatterns} documentation, and also its companion file \texttt{rubikpatternsLIST.pdf}). Having  the sequences available as  macros is 
very convenient since (a)~it avoids making errors when typing them out, 
and (b)~allows the easy application of software tools.
 
 A Rubik pattern  is the configuration generated  by a sequence of rotations 
 (or `moves') from some initial starting configuration  (typically a `solved' 
 configuration).  For example, FourSpot is a well known pattern which we can  
 generate  from a solved Rubik cube  using the macro \verb!\FourSpot!;  it is 
 defined as follows:
\begin{quote}
\begin{verbatim}
\newcommand{\FourSpot}{[FourSpot],F2,B2,U,Dp,R2,L2,U,Dp,<(12q*, 8f*)>}
\newcommand{\fourspot}{\FourSpot}
\end{verbatim}
\end{quote}
 Note that for convenience the macros names in the \textsc{rubikpatterns} package 
 are defined in  both upper and lower-case (i.e.,~the macros \verb!\FourSpot! 
 and \verb!\fourspot! are identical).  The following figure shows the FourSpot 
 sequence and pattern. 

\bigskip

\noindent%
\RubikCubeSolvedWY
\ShowCube{2cm}{0.5}{\DrawRubikCubeRD}
\RubikRotation{\FourSpot}
\quad\SequenceBraceA{FourSpot}{%
                     \ShowSequence{}{\Rubik}{\SequenceLong}%
                     }
\quad\ShowCube{2cm}{0.5}{\DrawRubikCubeSF}

\bigskip

{\noindent}The code for the above figure is as follows:


\begin{quote}
\begin{verbatim}
\usepackage{tikz,rubikcube,rubikrotation,rubikpatterns}
...
\noindent%
\RubikCubeSolvedWY
\ShowCube{2cm}{0.5}{\DrawRubikCubeRD}
\RubikRotation{\FourSpot}
\quad
\SequenceBraceA{FourSpot}{%
                         \ShowSequence{}{\Rubik}{\SequenceLong}%
                        }
\quad
\ShowCube{2cm}{0.5}{\DrawRubikCubeSF}
\end{verbatim} 
\end{quote}
Note that we have spread  the code slightly here in order to emphasise that 
the \verb!\ShowSequence! command is being used as an argument for the 
\verb!\SequenceBraceA! command (the `A' in the command \verb!\SequenceBraceA! 
denotes that the annotation is placed Above the sequence.)
We have used a solved cube with the WY (White opposite Yellow) 
configuration (\verb!\RubikCubeSolvedWY!).
The first cube is drawn from the  RD (Right-Down) viewpoint (\verb!\DrawRubikCubeRD!).
The second cube is drawn from the SF (Semi-Flat)  viewpoint (\verb!\DrawRubikCubeSF!) 
so we can see all the faces.

\pagebreak 

%----------------------
\subsection{SixTs}
\typeout{---Example (SixTs)}

A  more interesting cube pattern  is the SixTs configuration (from the 
\textsc{rubikpatterns} package),  which we  now show in a slightly 
different way (adding an extra face), as follows:

\bigskip

\noindent\hfil%
\RubikCubeSolvedWB%
\ShowCube{1.6cm}{0.4}{\DrawRubikCubeRU\DrawFlatDown{0}{-3.3}}%
\RubikRotation{\sixts}
\quad\SequenceBraceA{SixTs}{\ShowSequence{,\ }{\texttt}{\SequenceLong}}
\quad\ShowCube{2cm}{0.5}{\DrawRubikCubeRU}%
\quad\ShowCube{2cm}{0.5}{\DrawFaceFrontSide}%
\hfil

\bigskip

This time we have started with a solved cube having the WB  configuration 
(White opposite Blue), which is generated  using the command  
\verb!\RubikCubeSolvedWB! 
(we have added the \downface\ (blue) below to reveal the colour of this 
face---see note below). 

The rotation  sequence is   in `long-format' (expanded into 
separate rotations), comma-separated and space,  typewriter font,  
using the command \verb!\ShowSequence{,\ }{\texttt}{\SequenceLong}!.

The final image shows just the \frontface\  together with all the 
side-bars indicating the colours of the  adjacent facelets, 
using the \verb!\DrawFaceFrontSide! command. The images are 
separated using \verb!\quad!.
The code for the above figure is as follows:

\begin{quote}
\begin{verbatim}
\usepackage{tikz,rubikcube,rubikrotation,rubikpatterns}
...
\noindent\hfil
\RubikCubeSolvedWB
\ShowCube{1.6cm}{0.4}{%
                      \DrawRubikCubeRU%
                      \DrawFlatDown{0}{-3.3}%
                      }
\RubikRotation{\SixTs}
\quad\SequenceBraceA{SixTs}{%
                      \ShowSequence{,\ }{\texttt}{\SequenceLong}%
                      }
\quad\ShowCube{2cm}{0.5}{\DrawRubikCubeRU}
\quad\ShowCube{2cm}{0.5}{\DrawFaceFrontSide}
\hfil
\end{verbatim} 
\end{quote}



\subsubsection*{Notes}

\begin{enumerate} 

 \item   We have drawn the  \downface\ of the first cube using the command
\verb!\DrawFlatDown{0}{-3.3}! where the two arguments are  the $x$ and 
$y$ coordinates of the \textit{bottom left} corner of the \downface\  (blue). 
Note that the grid origin of  all cube images  coincides with the 
\textit{bottom left} corner of the \frontface\ (green in this case).

\item   The first image is really just 4 units wide. This is because the 2D width 
of the  \textit{side}  face (red) is designed to measure  1~unit wide in the 
oblique view (similarly, the 2D height of the  \textit{top} face also measures 
just 1~unit). Consequently, since the TikZ scale factor used is $0.4$, then 
the (minimum) width argument for its \verb!\ShowCube{}{}{}! command is 
$4\times 0.4 = 1.6$cm., hence we have \verb!\ShowCube{1.6cm}{0.4}{...}!.

\end{enumerate}



\pagebreak
%----------------------
\subsection{Three-edge cycle}
\typeout{---Example (three-edge cycle)}


The following example shows a sequence often used in solving the final layer.
 The black (no flip) and magenta (flip) arrows indicate  the \upface\  
cubie movement associated with the `three-edge cycle sequence 
\ShowSequence{,}{\rr}{F,R,U,Rp,Up,Fp}. The blue arrows indicate so-called 
collateral damage (two pairs of corner cubies swap positions) which can be fixed 
at a later stage. (This diagram  is from Section~13 in the 
\textsc{rubikcube} package documentation).

 \bigskip%
 \noindent\hfil%
 \RubikCubeSolved%
 \ShowCube{1.6cm}{0.4}{\DrawRubikCubeRU}%
 \quad\ShowCube{1.6cm}{0.4}{%
    \DrawFlatUpSide%
    \draw[thick,->,color=magenta] (1.5,0.5) -- (2.4, 1.4);
    \draw[thick,->] (2.5,1.5) -- (1.6, 2.4);
    \draw[thick,->,color=magenta] (1.3, 2.3) -- (1.3, 0.5);
    \draw[thick,<->,  color=blue] (0.5,2.6) -- (2.5, 2.6);
    \draw[thick,<->,  color=blue] (0.5,0.3) -- (2.5, 0.3);
    }%
 \RubikRotation{F,R,U,Rp,Up,Fp}%
 \quad\ShowSequence{}{\Rubik}{\SequenceLong}\quad$\longrightarrow$\quad%
 \ShowCube{1.6cm}{0.4}{\DrawFlatUpSide}%
 \hfil%
 
 \bigskip%


{\noindent}The code for the figure is as follows 


\begin{verbatim}
 \bigskip%
 \noindent\hfil%
 \RubikCubeSolved%
 \ShowCube{1.6cm}{0.4}{\DrawRubikCubeRU}%
 \quad\ShowCube{1.6cm}{0.4}{%
    \DrawFlatUpSide%
    \draw[thick,->,color=magenta] (1.5,0.5) -- (2.4, 1.4);
    \draw[thick,->] (2.5,1.5) -- (1.6, 2.4);
    \draw[thick,->,color=magenta] (1.3, 2.3) -- (1.3, 0.5);
    \draw[thick,<->,  color=blue] (0.5,2.6) -- (2.5, 2.6);
    \draw[thick,<->,  color=blue] (0.5,0.3) -- (2.5, 0.3);
    }%
 \RubikRotation{F,R,U,Rp,Up,Fp}%
 \quad\ShowSequence{}{\Rubik}{\SequenceLong}\quad$\longrightarrow$\quad%
 \ShowCube{1.6cm}{0.4}{\DrawFlatUpSide}%
 \hfil%
 
 \bigskip
\end{verbatim}





\pagebreak
%----------------------
\subsection{Superflip}
\typeout{---Example (subperflip)}


Once you can solve Rubik's cube, then an interesting exercise  is to generate  
the so-called `superflip' configuration, in which all the 
corners are correctly solved, while all the edges are 
flipped\,\footnote{See the `superflip' entry in \textit{Wikipedia}, and also 
the Kociemba website (\url{www.kociemba.org/cube.htm}); particularly the page 
\url{http://kociemba.org/math/oh.htm}}.

%--------draw the images-------------
\begin{figure}[hbt]
  \centering
  \RubikCubeSolved
  \ShowCube{2cm}{0.5}{\DrawRubikCubeRU}
  \quad$\longrightarrow$\quad%
  \RubikRotation{\superflip}%
  \ShowCube{2cm}{0.5}{\DrawRubikCubeLD}
  \quad\quad%
  \ShowCube{5cm}{0.5}{\DrawRubikCubeSF}
\caption{Two representations of the superflip configuration.}
\end{figure}
%----------------------

{\noindent}A superflip sequence converts the solved cube on the left 
into the form on the right, using the command \verb!\RubikRotation{\superflip}!.
The full code for the above figure is as follows:
\begin{verbatim}
\usepackage{tikz,rubikcube,rubikrotation,rubikpatterns}
...
\begin{figure}[hbt]
  \centering
  \RubikCubeSolved
  \ShowCube{2cm}{0.5}{\DrawRubikCubeRU}
  \quad$\longrightarrow$\quad%
  \RubikRotation{\superflip}%
  \ShowCube{2cm}{0.5}{\DrawRubikCubeLD}
  \quad\quad%
  \ShowCube{5cm}{0.5}{\DrawRubikCubeSF}
\caption{...}
\end{figure}
\end{verbatim}


The following superflip sequence\,\footnote{This particular  superflip sequence 
(in the \textsc{rubikpatterns} package) is due to Reid (1995); 
for details see the \textsc{rubikpatterns} package 
documentation, and also  \url{http://kociemba.org/math/oh.htm}. 
Another 20-move superflip sequence (due to H Kociemba),  is designated as  K32466  
  in  \url{www.nickalls.org/dick/papers/tex/RUBIK20moves.zip}.} 
has just 20  HTM rotations (Half Turn Metric: counting 180~degree turns as just 
one `rotation'). Note that the \textsc{rubikpatterns} package contains  this particular  
superflip sequence  as the macro \verb!\superflip!
(see \texttt{rubikpatterns.pdf}). Consequently  the  code 
\verb!\ShowSequence{,}{\large\texttt}{\superflip}!, will typeset the sequence
as follows:

\begin{center}
%% show the Short code
\ShowSequence{,}{\texttt}{\superflip}
\end{center}
Note that for convenience, the \textsc{rubikpatterns} package includes the sequence 
name (in square brackets)  as the first element of the associated  macro. 
This is possible since the contents of a comma-separated  square bracket  is not 
actioned as a rotation when it appears as part of the argument of  
the \verb!\RubikRotation! command.   

\pagebreak

Next  we present the sequence without commas in the form of hieroglyphs using 
the \verb!\Rubik! font, for which we require the `expanded' \verb!\SequenceLong! 
form (since the `short form' includes trailing digits---see \S\,9 in  
\texttt{rubikcube.pdf}), using the code

\begin{quote}
\begin{verbatim}
\usepackage{tikz,rubikcube,rubikrotation,rubikpatterns}
...
\RubikCubeSolved
\RubikRotation{\superflip}
\noindent\strut\hspace{-8mm}\ShowSequence{}{\Rubik}{\SequenceLong}
\end{verbatim}
\end{quote}
which gives


\bigskip

\RubikCubeSolved
\RubikRotation{\superflip}
\noindent\strut\hspace{-8mm}\ShowSequence{}{\Rubik}{\SequenceLong}


%%---------------------------------
\bigskip

\subsubsection*{Equivalent sequence}


Interestingly,  the superflip sequence is actually equivalent to
\Rubikbrace{\Rubikbracket{\Rubik{Rmp}\Rubik{Up}}4, \Rubik{yp}, \Rubik{x}}3.
Furthermore we can readily demonstrate this, as we can process this novel form 
of the sequence using  some useful features of the \verb!\RubikRotation! 
command, as follows:

\begin{quote}
\begin{verbatim}
\RubikCubeSolved%
\RubikRotation[3]{[superflip],(Rmp,Up)4,yp,x}%
\ShowCube{4cm}{0.4}{\DrawRubikCubeSF}%
\end{verbatim}
\end{quote}
which generates the following 

%-------------------
\begin{figure}[hbt]
  \centering
  \RubikCubeSolved%
  \RubikRotation[3]{[shortsuper],(Rmp,Up)4,yp,x}%
  \ShowCube{5cm}{0.5}{\DrawRubikCubeSF}
\end{figure}
%----------------------
{\noindent}which is exactly the same configuration as before.
Note that to do this we had to make  use of  the `repeat'  option \verb![3]!  
as well as  the \verb!(Rmp,Up)4! `repeat-block' in 
the argument of  the \verb!\RubikRotation! command above. 




\pagebreak
%----------------------
\subsection{Inverse sequence}
\typeout{---Example (inverse sequence)}


Generating the inverse of a Rubik sequence involves (a)~reversing the order of 
the sequence, and (b)~inverting each rotation in the sequence 
(see Sections~5.1 and 5.11 in the \textsc{rubikrotation} package documentation).


From  the grey-cube example~(\ref{sec:greycube}) we saw that the sixspot sequence  is: \verb!U,Dp,R,Lp,F,Bp,U,Dp!;
its inverse is therefore readily determined as \verb!D,Up,B,Fp,L,Rp,D,Up!. 
Note that this is easy to check since the sequence generated by the  
\verb!\RubikRotation! command is held by the macro \verb!\SequenceLong!. 
For~example, the output of the following commands
\begin{quote}
\begin{verbatim}
\fbox{\strut\ %
The inverse of the sixspot sequence is:
\RubikRotation{\sixspot,<inverse>}
\ShowSequence{,}{\texttt}{\SequenceLong}.
}
\end{verbatim}
\end{quote}
is \quad\fbox{\strut\ %
The inverse of the sixspot sequence is: 
\RubikRotation{\sixspot,<inverse>}
\ShowSequence{,}{\texttt}{\SequenceLong}.
}

\bigskip

A sequence and its inverse will annihilate each other when applied consecutively.  
For example, in the following figure we start with a solved cube and 
apply the sixspot sequence. Applying the  inverse of the sixspot sequence then 
results in the solved cube configuration again.

\newcommand{\sixspotINV}{[sixspotINV],D,Up,B,Fp,L,Rp,D,Up}
\newcommand{\sixspotarrow}{$\quad\overrightarrow{\strut\textsc{sixspot}}\quad$}
\newcommand{\sixspotINVarrow}{$\quad\overrightarrow{\strut\textsc{sixspotINV}}\quad$}

\begin{figure}[hbt]
\centering
\RubikCubeSolved%  
   \ShowCube{2cm}{0.5}{\DrawRubikCubeRU}\sixspotarrow% 
\RubikRotation{\sixspot}%   
   \ShowCube{2cm}{0.5}{\DrawRubikCubeRU}\sixspotINVarrow%
\RubikRotation{\sixspotINV}%  
   \ShowCube{2cm}{0.5}{\DrawRubikCubeRU}
\end{figure}

{\noindent}The code for the above figure is as follows:

\begin{quote}
\begin{verbatim}
\usepackage{tikz,rubikcube,rubikrotation,rubikpatterns}
...
\newcommand{\sixspotINV}{[sixspotINV],D,Up,B,Fp,L,Rp,D,Up}
\newcommand{\sixspotarrow}{$\quad\overrightarrow{\strut\textsc{sixspot}}\quad$}
\newcommand{\sixspotINVarrow}{$\quad\overrightarrow{\strut\textsc{sixspotINV}}\quad$}

\begin{figure}[hbt]
\centering
\RubikCubeSolved%  
   \ShowCube{2cm}{0.5}{\DrawRubikCubeRU}\sixspotarrow% 
\RubikRotation{\sixspot}%   
   \ShowCube{2cm}{0.5}{\DrawRubikCubeRU}\sixspotINVarrow%
\RubikRotation{\sixspotINV}%  
   \ShowCube{2cm}{0.5}{\DrawRubikCubeRU}
\end{figure}
\end{verbatim}
\end{quote}

\pagebreak

A significant property of the superflip configuration is that it is its own inverse.
Consequently we can achieve  a similar result simply by applying the 
superflip sequence \textit{twice in succession}, as follows:

\newcommand{\superfliparrow}{$\quad\overrightarrow{\strut\textsc{superflip}}\quad$}


\begin{figure}[hbt]
\centering
\RubikCubeSolved%  
   \ShowCube{2cm}{0.5}{\DrawRubikCubeRU}\superfliparrow% 
\RubikRotation{\superflip}%   
   \ShowCube{2cm}{0.5}{\DrawRubikCubeRU}\superfliparrow%
\RubikRotation{\superflip}%  
   \ShowCube{2cm}{0.5}{\DrawRubikCubeRU}
\end{figure}




\begin{center}
------ END ------
\end{center}

\end{document}