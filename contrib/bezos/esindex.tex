% +--------------------------------------------------+
% | Typeset this file to get the documentation.      |
% +--------------------------------------------------+
%
%% Copyright (C) 1998-2004, 2006, 2008 Javier Bezos
%% All Rights Reserved
%% http://www.tex-tipografia.com
%%
%% This work may be distributed and/or modified under the conditions
%% of the LaTeX Project Public License, either version 1.3 of this
%% license or (at your option) any later version.  The latest version
%% of this license is in
%%     http://www.latex-project.org/lppl.txt
%% and version 1.3 or later is part of all distributions of LaTeX
%% version 2003/12/01 or later.
%%
%% This work has the LPPL maintenance status "maintained".
%%
%% This Current Maintainer of this work is Javier Bezos.
%%
%% This work consists of the files esindex.tex and esindex.sty.

\documentclass{article}
\usepackage[spanish,activeacute]{babel}
\spanishdatedel
\usepackage[cp1252]{inputenc}

\title{Paquete \textsf{esindex}\footnote{Este
     archivo est\'a actualmente en la versi\'on 1.4
     con fecha 2008-09-28.
     Esta copia del manual se compuso el~\today.}}

 \author{Javier Bezos\footnote{Para comentarios y sugerencias: 
\texttt{http://www.tex-tipografia.com}.}}
   
\raggedright
\parskip=1ex

\date{2008-09-28}

\begin{document}
	
\maketitle

This package defines the command \verb|\esindex| which
easies writing Spanish index entries:
\begin{verbatim}
\esindex{ca��n}
\end{verbatim}
is equivalent to
\begin{verbatim}
\index{can^^ffon@ca��n}
\end{verbatim}
Since it is an especifically Spanish tool, the documentation
is in Spanish.

\vspace*{1pc}

Este paquete ha sido dise'nado para facilitar la escritura
de 'indices correctamente alfabetizados en castellano. Su principal orden es
\verb|\esindex| que convierte a una forma adecuada su argumento.
As'i por ejemplo,
\begin{verbatim}
\esindex{ca��n}
\end{verbatim}
equivale a
\begin{verbatim}
\index{can^^ffon@ca��n}
\end{verbatim}
No es necesario usar \textsf{babel} salvo, l'ogicamente, si los acentos
est'an escritos en forma de abreviaciones (\verb|'a|, \verb|'e|, etc.)
en lugar de con los caracteres reales.  En este 'ultimo caso, el paquete
utiliza ciertas 'ordenes internas de \textsf{babel} por lo que no
puedo garantizar su funcionamiento correcto con versiones distintas a
las 3.6 a 3.8.  En caso de que \textsf{esindex} sea incompatible con
futuras versiones de \textsf{babel} intentar'e adaptarlo en el menor
tiempo posible.

Salvo el car'acter \verb|actual| (normalmente \verb|@|) se pueden usar
todos los caracteres especiales de \textit{MakeIndex}.  Se pueden usar
convenciones diferentes a las normales, pero en este caso hacen falta
ajustes adicionales en caso de que los modificados sean \verb|actual|,
\verb|encap|, \verb|level| o \verb|quote|.  En ese caso basta con
indicar los caracteres que hay que usar como opciones de paquete.  Por
ejemplo, si para \verb|quote| decidimos usar \verb|$| en nuestro
archivo \verb|.ist| particular, tendr'iamos que llamar al paquete del
siguiente modo:
\begin{verbatim}
\usepackage[quote=$]{esindex}
\end{verbatim}

Es importante observar que, a diferencia de la opci'on para alem'an de
\textit{MakeIndex}, el uso de \verb|"| en abreviaciones como \verb|"u|
es completamente leg'itimo, ya que el paquete reconoce tal
combinaci'on y la trata aparte.  Lo mismo vale para \verb|'| o
\verb|~| en caso de que se usaran como car'acter especial.  Es decir
\begin{verbatim}
\esindex{{"!`}Cig"ue'nas{"!}|textbf}
\end{verbatim}
equivale a
\begin{verbatim}
\index{{"!`}Ciguen^^ffas{"!}@{"!`}Cig\"ue\~nas{"!}|textbf}
\end{verbatim}

Sin embargo, el uso del car'acter \verb|quote| ante \verb|encap| o
\verb|level| no se detecta a menos que el grupo est'e encerrado entre
llaves.  Por ejemplo, en lugar de \verb/\esindex{Pleca: "|}/ debe
escribirse \verb/\esindex{Pleca: {"|}}/.  (En realidad en este caso
podr'ia haberse usado \verb|\index|.  Es tan s'olo un ejemplo.)

Aunque el hecho de que \verb|@| no se pueda usar en \verb|\esindex|
hace que todav'ia algunas entradas se tengan que hacer <<a mano>>,
la mayor parte del trabajo se ve considerablemente simplificado.

Finalmente, hay que se'nalar que con este paquete no se crea en el 
'indice una entrada propia para la palabras que empiezan por e'ne,
sino que tan s'olo se a'naden al final de la ene. En el rar'isimo
caso de que hubiera palabras que empiezan por e'ne habria que
modificar el archivo \verb|.ind| a mano.

% La versi�n 1.2 corrige la supresi�n incorrecta de acentos en los
% primeros niveles si hab�a m�s de uno.

La versi�n 1.3 elimina una incompatibilidad con recientes versiones
de \LaTeX{} y a�ade nuevas funciones:
\begin{itemize}
\item Opci�n de paquete \verb|ignorespaces|: al formar la clave de 
ordenaci�n se suprimen los espacios, de forma que:
$$\mbox{adentro} < \mbox{a donde} < \mbox{adonde}.$$
\item Opci�n de paquete \verb|replaceindex|: el comportamiento de
\verb|\index| se reemplaza por el de \verb|\esindex|, aunque en este
caso no es posible introducir entradas que no se adapten a lo
requerido por \verb|\esindex|.
\item La orden \verb|\ignorewords| da
una lista de palabras separadas por comas que no se cuentan en la
ordenaci�n.  Por ejemplo, con \verb|\ignorewords{de}| tendr�amos:
$$\mbox{pino albar} < \mbox{pino laricio} < \mbox{pino de monta�a}.$$
Distingue la caja, por lo que las formas con may�sculas hay que darlas
expl�citamente, si hicieran falta.
\end{itemize}

La versi�n 1.4 elimina un peque�o bicho con \verb|"U| y a�ade nuevas 
funciones:
\begin{itemize}
\item La lista de t�kenes \verb|\everyesindex| permite dar definiciones 
locales para establecer el comportamiento de otras �rdenes. Por 
ejemplo:
\begin{verbatim}
\everyesindex{\renewcommand\emph[1]{#1}}
\end{verbatim}
elimina de la clave esa orden. Con:
\begin{verbatim}
\everyesindex{\renewcommand\emph[1]{#1'}}
\end{verbatim}
la entrada en cursiva ir�a detr�s de la redonda, si la hubiera. Es 
una t�cnica que se puede emplear en otros casos para reajustar el
orden de entradas id�nticas.
\item La orden \verb|\esindexsort| permite predefinir claves asociadas
a entradas concretas, para ajustar su ordenaci�n (lo que normalmente
se consigue a�adiendo texto adicional para que makeindex lo tenga en
cuenta).  Estas correspondencias deben darse antes de la aparici�n del
primer \verb|\esindex| con ese t�rmino, y las claves se procesan 
posteriormente con \verb|ignorespaces|, \verb|\ignorewords| y 
\verb|\everyesindex|, si est�n activadas.
Por ejemplo:
\begin{verbatim}
\esindexsort{adonde}{adonde'1}
\esindexsort{ad�nde}{adonde'2}
\esindexsort{a donde}{a donde'7}
\esindexsort{a d�nde}{a donde'8}
\end{verbatim}
dar�a el orden \emph{adonde, ad�nde, a donde, a d�nde}, con
\verb|ignorespaces|, o bien \emph{a donde, a d�nde,} [probablemente
otros t�rminos], \emph{adonde, ad�nde}, sin \verb|ignorespaces|.
\end{itemize}



\end{document}
