%\iffalse % MetaComment
%%
%% + - - - - - - - - - - - - - - - - - - - - - - - - - - - - - - +
%% |            Copyright(C) 1997-2010 by F. Bosisio             |
%% |                                                             |
%% | This program can be redistributed and/or modified under	 |
%% | the terms of the LaTeX Project Public License, either       |
%% | version 1.3 of this license or (at your option) any later   |
%% | version. The latest version of this license is in           |
%% |   http://www.latex-project.org/lppl.txt                     |
%% | and version 1.3 or later is part of all LaTeX distributions |
%% | version 2005/12/01 or later.                                |
%% |                                                             |
%% | This work has the LPPL maintenance status `maintained'.     |
%% | The Current Maintainer of this work is F. Bosisio.          |
%% |                                                             |
%% | This work consists of files accenti.dtx and accenti.html    |
%% | and of the derived files accenti.sty and accenti.pdf.       |
%% |                                                             |
%% | E-mail:   fbosisio@bigfoot.com                              |
%% | CTAN location: macros/latex/contrib/bosisio/                |
%% + - - - - - - - - - - - - - - - - - - - - - - - - - - - - - - +
%%
%%	If you make any improvment, find any bug or have
%%	any suggestion, please let me know about it.
%%
%<*package,driver>
%\fi
%
\def\FileName{accenti}
%\iffalse % MetaComment
%</package,driver>
%<*package>
%\fi
\def\fileversion{2.3}
\def\filedate{2005/04/02}
\def\docdate{2005/04/09}
\def\filedescr{Commands to make accents (FB)}
%
%\iffalse % MetaComment
%</package>
%<*dtx>
%- - - - - - - - - - - - - - - - - - - - - - - - - - - - - - - - - - - -
%		This section is the installation driver
%
\def\batchfile{\FileName.dtx}
%
\input docstrip
%
\keepsilent
% \askforoverwritefalse
%
\generateFile{\FileName.sty}{f}{\from{\FileName.dtx}{package}}
%
\generateFile{\FileName.drv}{f}{\from{\FileName.dtx}{driver}}
%
\Msg{******************************************************}
\Msg{*}
\Msg{* To produce the documentation run the}
\Msg{* file `\FileName.drv' through LaTeX.}
\Msg{*}
\Msg{******************************************************}
%
\endbatchfile
%
%		End of the installation driver
%- - - - - - - - - - - - - - - - - - - - - - - - - - - - - - - - - - - -
%</dtx>
%
%		This section is the documentation driver
%
%<+driver>\documentclass[12pt,a4paper]{ltxdoc}
%<+driver>  \EnableCrossrefs
%<+driver>  \CodelineIndex
%<+driver>  \RecordChanges
%<+driver>  %\OnlyDescription	% Uncomment not to see the implementation
%<+driver>\begin{document}
%<+driver>  \DocInput{\FileName.dtx}
%<+driver>  \PrintIndex
%<+driver>  \PrintChanges
%<+driver>\end{document}
%
%		End of the documentation driver
%- - - - - - - - - - - - - - - - - - - - - - - - - - - - - - - - - - - -
%<*package>
%\fi
%
% \changes{v0.1}{4 May 1997}{First release (basic accents-command)}
% \changes{v0.2}{20 August 1997}{Added extended-chars handling}
% \changes{v1.0}{5 September 1997}{Documentation added}
% \changes{v1.1}{7 November 1997}{Fixed a bug in the options}
% \changes{v2.0}{20 December 1997}{Separated from package ``quotes''}
% \changes{v2.1}{5 March 1999}{Added copyright notice and changed addresses}
% \changes{v2.2}{2 April 2005}{All commands made robust}
%\iffalse % MetaComment
% The previous change was requested by Robin Fairbains (robin.fairbains@cl.cam.ac.uk)
%\fi
% \changes{v2.3}{9 April 2005}{Usage of the double-quote character (") avoided}
%
% \MakeShortVerb{\|}
%
% \title{Package \texttt{\FileName}\thanks{This is version \fileversion,
% last revised \filedate; documentation date \docdate}}
% \author{F. Bosisio\\\normalsize E-mail: \texttt{fbosisio@bigfoot.com}}
% \date{\filedate}
% \maketitle
%
% \begin{abstract}
%       Documentation for the package \texttt{\FileName}.
% \end{abstract}
%
% \section{Introduction}
%       This package provides a shorter version of some accents-making
%	commands, particularly suited for italian language.
%
%       In particular some commands are redefined, so care should be
%       taken, expecially when including this package in an already
%       existent \LaTeX{} file.
%
%       \noindent
%       The redefined commands are:\\
%	``|\a|'', ``|\i|'', ``|\o|'' and ``|\u|'',\\
%	whose job can now be done by the commands\\
%	``|\oFinnick|'' (finnick ``o'' bar),
%	``|\DotlessI|'' (dotless ``i'', for accents),
%	``|\aAccent|'' (for accents in |tabbing| environment)
%	and ``|\uAccent|'' (upsidedown ``hat'' accent).
%
% \section{Obsolete package}
%
%	This package is \emph{obsolete}, as it has been superseeded by
%	the standard ``|inputenc|'' package: indeed, the declaration\\
%	\mbox{}\qquad |\usepackage[latin1]{inputenc}|\\
%	directly allows the use of the characters ``\`a'', ``\`e'',
%	``\`\i'', ``\`o'', ``\`u'' and ``\'e'' in the document, which
%	is extremly useful with an italian keyboard.
%
% \section{Required packages}
%
%	This package requires the ``|\xspace|'' command provided by
%	the ``|xspace|'' package.
%
% \section{Options}
%
%	This package provides the options ``|ExtdChar|'' and
%	``|ExtdCharOnly|'' which allow the use of the special
%	characters ``\`a'', ``\`e'', ``\`\i'', ``\`o'', ``\`u'' and ``\'e''
%	as accents-making commands.
%	The second option also inhibits the definitions of
%	``|\a|'', ``|\e|'', ``|\i|'', ``|\o|'', ``|\u|'', ``|\ee|''
%	and ``|\che|'' as short forms of the accent-commands
%	(it is particularly useful for non-Italian documents).
%
% \section{Accents}
%
%	Unless you specify the ``|ExtdCharOnly|'' option, the
%	commands ``|\a|'', ``|\e|'', ``|\i|'', ``|\o|'', ``|\u|'' and ``|\E|''
%       prints the corresponding letter with a grave accent;
%       the commands ``|\ee|'' and ``|\che|'' print the letter ``|e|''
%       or the three letters ``|che|'' with an acute accent (i.e. ``\'e''
%	and ``ch\'e'').
%
%	Moreover, if the ``|ExtdChar|'' or ``|ExtdCharOnly|'' option was
%	used, the chars ``\`a'', ``\`e'', ``\`\i'', ``\`o'', ``\`u'' and ``\'e''
%	are equivalent to the commands ``|\`a|'', ``|\`e|'', ``|\`\i|'',
%	``|\`o|'', ``|\`u|'' and ``|\'e|''.
%
% \StopEventually{}
% \newpage
% \section{Implementation}
%
%    \begin{macrocode}
%%
\NeedsTeXFormat{LaTeX2e}[1995/12/01]
\ProvidesPackage{\FileName}[\filedate\space v\fileversion\space\filedescr]
\RequirePackage{xspace}[1996/12/06 v1.05]
%%
\newif\if@ExtendedAccChar\@ExtendedAccCharfalse
\DeclareOption{ExtdChar}{\@ExtendedAccChartrue}
%%
\newif\if@NotOnlyExtended@\@NotOnlyExtended@true
\DeclareOption{ExtdCharOnly}{\@NotOnlyExtended@false}
%%
\ProcessOptions
%%
%    \end{macrocode}
%
%	\begin{macro}{\a,\e,\i,\o,\u,\ee,\che,\E}
%
%	The original commands ``|\a|'', ``|\i|'', ``|\o|'' and ``|\u|'' are
%	saved before being redefined.
%	The combined use of an intermediate internal command and of
%	``|\DeclareRobustCommand*|'', allow for all the commands of the
%	package to be robust (even the saved original ones).
%
%    \begin{macrocode}
%%
\let\a@RIGINAL=\a
\let\i@RIGINAL=\i
\let\o@RIGINAL=\o
\let\u@RIGINAL=\u
%%
\DeclareRobustCommand*\aAccent{\a@RIGINAL}
\DeclareRobustCommand*\DotlessI{\i@RIGINAL}
\DeclareRobustCommand*\oFinnick{\o@RIGINAL}
\DeclareRobustCommand*\uAccent{\u@RIGINAL}
%    \end{macrocode}
%
%	The commands ``|\a|'', ``|\e|'', ``|\i|'', ``|\o|'', ``|\u|'' and ``|\E|''
%	prints the corresponding letter with a grave accent;
%	the commands ``|\ee|'' and ``|\che|'' print the letter ``|e|''
%	or the three letters ``|che|'' with an acute accent (i.e. ``\'e''
%	and ``ch\'e'').
%	The definitions are deferred at the ``|\begin{document}|''
%	in order to avoid conflicts with other packages.
%
%    \begin{macrocode}
%%
\if@NotOnlyExtended@	%-\-\-\-\-\-\-\-\-\-\-\-\-\-\-\-\-\-\-\-\-\-\-\-\-\
  \AtBeginDocument{%
    \renewcommand*\a{\`a\xspace}%
    \newcommand*\e{\`e\xspace}%
    \renewcommand*\i{\`\DotlessI\xspace}%
    \renewcommand*\o{\`o\xspace}%
    \renewcommand*\u{\`u\xspace}%
    \newcommand*\E{\`E\xspace}%
    \newcommand*\ee{\'e\xspace}%
    \newcommand*\che{ch\'e\xspace}%
  }
\fi	%-/-/-/-/-/-/-/-/-/-/-/-/-/-/-/-/-/-/-/-/-/-/-/-/-/-/-/-/-/-/-/-/-/
%    \end{macrocode}
%	\end{macro}
%
%	\begin{macro}{Accents}
%	A trick adapted from the |doc| package (which, perhaps, may be
%	done in a better way) is used (if the ``|ExtdChar|'' option was
%	selected) to associate the extended-ASCII chars which represents
%	accents with the corresponding accent-making commands.
%
%    \begin{macrocode}
%%
\if@ExtendedAccChar	%-\-\-\-\-\-\-\-\-\-\-\-\-\-\-\-\-\-\-\-\-\-\-\-\-\
%%
%%	%%%%%%%%%%
%%	% � = \a %
%%	%%%%%%%%%%
\begingroup
  \catcode`\~\active  \lccode`\~`\�%
  \lowercase{%
  \global\expandafter\let
     \csname ac\string\�\endcsname~%
  \gdef~{\`a}}%
\endgroup
\global\catcode`\�\active
%%
%%	%%%%%%%%%%
%%	% � = \e %
%%	%%%%%%%%%%
\begingroup
  \catcode`\~\active  \lccode`\~`\�%
  \lowercase{%
  \global\expandafter\let
     \csname ac\string\�\endcsname~%
  \gdef~{\`e}}%
\endgroup
\global\catcode`\�\active
%%
%%	%%%%%%%%%%%
%%	% � = \ee %
%%	%%%%%%%%%%%
\begingroup
  \catcode`\~\active  \lccode`\~`\�%
  \lowercase{%
  \global\expandafter\let
     \csname ac\string\�\endcsname~%
  \gdef~{\'e}}%
\endgroup
\global\catcode`\�\active
%%
%%	%%%%%%%%%%
%%	% � = \i %
%%	%%%%%%%%%%
\begingroup
  \catcode`\~\active  \lccode`\~`\�%
  \lowercase{%
  \global\expandafter\let
     \csname ac\string\�\endcsname~%
  \gdef~{\`\DotlessI}}%
\endgroup
\global\catcode`\�\active
%%
%%	%%%%%%%%%%
%%	% � = \o %
%%	%%%%%%%%%%
\begingroup
  \catcode`\~\active  \lccode`\~`\�%
  \lowercase{%
  \global\expandafter\let
     \csname ac\string\�\endcsname~%
  \gdef~{\`o}}%
\endgroup
\global\catcode`\�\active
%%
%%	%%%%%%%%%%
%%	% � = \u %
%%	%%%%%%%%%%
\begingroup
  \catcode`\~\active  \lccode`\~`\�%
  \lowercase{%
  \global\expandafter\let
     \csname ac\string\�\endcsname~%
  \gdef~{\`u}}%
\endgroup
\global\catcode`\�\active
%%
\fi	%-/-/-/-/-/-/-/-/-/-/-/-/-/-/-/-/-/-/-/-/-/-/-/-/-/-/-/-/-/-/-/-/-/
%    \end{macrocode}
%	\end{macro}
%
%\iffalse % MetaComment
%<*package>
%\fi
%
% \CheckSum{271}
% \Finale
%
\endinput
