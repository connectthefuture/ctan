%% LyX 2.2.0 created this file.  For more info, see http://www.lyx.org/.
%% Do not edit unless you really know what you are doing.
\documentclass[twoside,english]{article}
\usepackage{lmodern}
\renewcommand{\sfdefault}{lmss}
\renewcommand{\ttdefault}{lmtt}
\usepackage[T1]{fontenc}
\usepackage[latin9]{inputenc}
\usepackage{geometry}
\geometry{verbose,lmargin=4cm,rmargin=3.5cm}
\setcounter{secnumdepth}{2}
\setcounter{tocdepth}{1}
\usepackage{wrapfig}
\usepackage{booktabs}
\usepackage{amstext}
\usepackage{esint}

\makeatletter

%%%%%%%%%%%%%%%%%%%%%%%%%%%%%% LyX specific LaTeX commands.
%% Because html converters don't know tabularnewline
\providecommand{\tabularnewline}{\\}

%%%%%%%%%%%%%%%%%%%%%%%%%%%%%% Textclass specific LaTeX commands.
 \newenvironment{example}{\begin{center}\ttfamily}{\end{center}}
\newenvironment{lyxcode}
{\par\begin{list}{}{
\setlength{\rightmargin}{\leftmargin}
\setlength{\listparindent}{0pt}% needed for AMS classes
\raggedright
\setlength{\itemsep}{0pt}
\setlength{\parsep}{0pt}
\normalfont\ttfamily}%
 \item[]}
{\end{list}}
\newcommand{\strong}[1]{\textbf{#1}}

%%%%%%%%%%%%%%%%%%%%%%%%%%%%%% User specified LaTeX commands.
\usepackage{diffcoeffx}

\@ifundefined{showcaptionsetup}{}{%
 \PassOptionsToPackage{caption=false}{subfig}}
\usepackage{subfig}
\makeatother

\usepackage{babel}
\begin{document}

\title{\texttt{diffcoeffx}~\\
extending the \texttt{diffcoeff} package}

\author{Andrew Parsloe\\
{\small{}(aparsloe@clear.net.nz)}}
\maketitle
\begin{abstract}
\noindent \texttt{diffcoeffx.sty} is \texttt{diffcoeff.sty} `on steroids'.
It provides additional functionality for the trailing optional argument
and extends the algorithm used to calculate the overall order of differentiation
of mixed partial derivatives. That now accepts order-of-differentiation
specifications that include powers of numbers and variables, subscripts
on variables, and (possibly nested) parentheses with numerical coefficients.
The enhancements come under the category of `gilding the lily'. 
\end{abstract}

\section{The \texttt{diffcoeffx} package}

The \texttt{diffcoeffx} package is \texttt{diffcoeff} `on steroids',
providing exactly the same commands but with some extra functionality.\texttt{ }It
is called in the usual way in the LaTeX preamble:
\begin{lyxcode}
\textbackslash{}usepackage\{diffcoeffx\}
\end{lyxcode}
It is assumed that you are familiar with the \texttt{diffcoeff} package
and its manual.\texttt{ }There are two enhancements to that package: \texttt{diffcoeffx}
takes the calculation of the overall order of mixed partial derivatives
deep into `overkill' territory, accepting single-token powers of numbers
and variables, single-token subscripts on variables, and possibly nested
parentheses with numerical coefficients. The \texttt{\textbackslash{}times}
token ($\times$) can also be used in an order specification. The other
enhancement is an extension to the capabilities of the trailing optional
argument. 

\subsection[Exploiting the final argument]{Exploiting the trailing optional argument}

For \texttt{diffcoeff }there was an attempt to give a `natural feel'
to the design choices made and their use. By comparison the additional
functionality that the trailing optional argument acquires in \texttt{diffcoeffx.sty}
is in the nature of a \emph{hack}. It works, but I'm not sure that it should
be encouraged. 

In \texttt{diffcoeff} if you write \texttt{\textbackslash{}diff yx\{\}}
the trailing but \emph{empty} optional argument is ignored. Not so in \texttt{diffcoeffx}:
\begin{example}
\textbackslash{}diffp yx\{\}${\displaystyle \Longrightarrow\quad\diffp yx{}}$
\end{example}
The parentheses are inserted without a subscript. Thus we can write (for
instance) Lagrange's equations of motion in analytical mechanics in the
manner:
\begin{example}
\textbackslash{}diffp L\{q\_k\}-\textbackslash{}diff{*}\{\textbackslash{}diffp
L\{\textbackslash{}dot\{q\}\_k\}\{\}\}t = 0 $\Longrightarrow\quad{\displaystyle \diffp L{q_{k}}-\diff*{\diffp L{\dot{q}_{k}}{}}t}=0$,
\end{example}
without having to bother with inserting \texttt{\textbackslash{}left(}
and \texttt{\textbackslash{}right}).\texttt{ }The empty trailing optional
argument and the default delimiters for partial derivatives do the job
for us. 

There are many other places in analytical mechanics where using an empty
trailing optional argument is a similarly convenient way of writing large
parentheses, for instance, 
\begin{example}
\textbackslash{}dot\{q\_k\}=\textbackslash{}diffp H\{\textbackslash{}diffp
S\{q\_k\}\{\}\} ${\displaystyle \Longrightarrow\quad\dot{q_{k}}=\diffp H{\diffp S{q_{k}}{}}}\!.$ 
\end{example}
An application of Lagrange's equations (to a one-dimensional elastic solid)
gives rise to a Langrangian density function,
\begin{example}
\textbackslash{}frac 12\textbackslash{}left\textbackslash{}\{ \textbackslash{}rho\textbackslash{}dot\{\textbackslash{}eta\}\textasciicircum{}2-E\textbackslash{}diff\textbackslash{}eta
x\{;2;()\}\textbackslash{}right \textbackslash{}\} $\Longrightarrow\quad{\displaystyle \frac{1}{2}\left\{ \rho\dot{\eta}^{2}-E\diff\eta x{;2;()}\right\} }.$
\end{example}
Another application of those equations (the acoustic approximation to the
irrotational motion of a compressible non-viscous fluid) produces a Lagrangian
density
\begin{example}
\textbackslash{}frac 12\textbackslash{}rho\textbackslash{}left\textbackslash{}\{(\textbackslash{}nabla\textbackslash{}psi)\textasciicircum{}2-\textbackslash{}frac
1\{c\textasciicircum{}2\}\textbackslash{}diff\textbackslash{}psi t\{;2;()\}\textbackslash{}right\textbackslash{}\}
$\Longrightarrow\quad{\displaystyle \frac{1}{2}\rho\left\{ (\nabla\psi)^{2}-\frac{1}{c^{2}}\diff\psi t{;2;()}\right\} }.$
\end{example}
In both examples, the trailing optional argument of the \emph{ordinary}
derivative has been filled by a semicolon-delimited list: \texttt{\{;2;()\}}.
The initial slot where a subscript is specified is empty but the semicolon
is necessarily included. The second spot specifies a \emph{superscript}
and the third slot the delimiters to use. Since parentheses are not the
default delimiters for an ordinary derivative we needed to specify them
explicitly here. However, this does not change the default delimiters which
remain \texttt{.|} for an ordinary derivative and can only be changed by
means of the \texttt{\textbackslash{}diffset} command.

Both subscript and superscript can be used at the same time. In a text
on ordinary differential equations, an example employing Green's functions
gives rise to
\begin{example}
\textbackslash{}diff{[}n-1{]}Gx\{\textbackslash{}xi-\textbackslash{}epsilon;\textbackslash{}xi+\textbackslash{}epsilon;{[}{]}\}
$\Longrightarrow\quad{\displaystyle \diff[n-1]Gx{\xi-\epsilon;\xi+\epsilon;[]}}$
\end{example}
the derivative being evaluated at both superscript and subscript values
and the difference taken. Here the trailing optional argument has its first
three slots filled, with square brackets explicitly specified. The same
book includes the example
\begin{example}
\textbackslash{}diff{*}{[}p-1{]}\{x\textasciicircum{}\textbackslash{}alpha\}\textbackslash{}alpha\{\textbackslash{}alpha=a;;\textbackslash{}\{\textbackslash{}\}\}
$\Longrightarrow\quad{\displaystyle \diff*[p-1]{x^{\alpha}}\alpha{\alpha=a;;\{\}}}$
\end{example}
where, this time braces are specified in the trailing optional argument.\footnote{For LyX users, the braces \textbackslash{}\{ and \textbackslash{}\} are
inserted into a formula in the maths editor simply by typing the braces
without the backslashes. LyX takes care of the latter.} 

This argument can be used to form the absolute value of a derivative,
\begin{example}
\textbackslash{}diff yx\{;;||\} ${\displaystyle \Longrightarrow\quad\diff yx{;;||}}$ 
\end{example}
where both initial slots, subscript and superscript, are empty and two
semicolons necessarily included in the trailing optional argument: \texttt{\{;;||\}}.
It also provides an alternative way, indeed \emph{two} alternative ways,
of forming a quotient of derivatives:
\begin{example}
\textbackslash{}diff yx\{;;./\}\textbackslash{}diff xy=\textbackslash{}diff
yx\textbackslash{}diff xy\{;;/.\} ${\displaystyle {\displaystyle \Longrightarrow\quad\diff yx{;;./}}\diff xy={\displaystyle \diff yx}\diff xy{;;/.}}$
\end{example}
where the delimiter specification \texttt{./} on the left has been changed
to \texttt{/.} on the right. The spacing in the two quotients is not quite
identical, which might be relevant in some contexts. As a more realistic
example of use of the same construct, if $F(x,t)$ is a function of $x$
and $t$ and $x=x(t)$, then if $\diff Ft=0$,
\begin{example}
\textbackslash{}diff xt=-\textbackslash{}diffp Ft\{;;./\}\textbackslash{}diffp
Fx $\Longrightarrow\quad{\displaystyle \diff xt=-\diffp Ft{;;./}\diffp Fx}$
\end{example}
For an inline use, you may prefer to use the slash form of the derivative
$\diff y/z{0;;();-1}$. In this case a \emph{fourth} slot in the trailing
optional argument has been filled, the \texttt{nudge override} slot, since
the default nudge is designed to position the subscript relative to the
\emph{displaystyle} delimiters.

The complete specification of what is available in the trailing optional
argument is:
\begin{example}
\{ subscript; superscript; delimiters; nudge override \}
\end{example}
\begin{itemize}
\item In `normal' use, the \texttt{subscript} is the point of evaluation (ordinary
derivatives), or list of variables held constant (partial derivatives).
Since the list of variables held constant is likely to be comma-separated,
so we have the need for semicolons to separate items in the larger list. 
\item The \texttt{superscript} is generally a power to which the derivative is
raised but, as instanced by the Green's function example, it can also be
another point of evaluation of the derivative.
\item The \texttt{delimiters} are, by default, \texttt{.|} for ordinary derivatives
and \texttt{()} for partial derivatives. These are not always the right
ones for a particular task. Rather than changing them \emph{globally} as
the use of \texttt{\textbackslash{}diffset} entails, they can be changed
\emph{locally} for the particular instance by specifying them in this slot.
The global choices are unaffected. 
\item If the built-in placement of sub- or superscript relative to the right
delimiter is unsatisfactory, a value specified in the \texttt{nudge override}
slot\texttt{ }overrides the default value locally. The value is a pure
number which \texttt{diffcoeffx} treats as that number of mu (1/18 of an
em). (For comparison, a thin space \textbackslash{}, and a negative thin
space \textbackslash{}! are 3/18 of an em.) The default nudges are shown
in Table~\ref{tab:Default-nudges}. They are intended for displaystyle
presentation, and are not affected by any value included in this slot.
\end{itemize}
\noindent\begin{minipage}[t]{1\columnwidth}%
\begin{wraptable}[10]{o}{0.35\columnwidth}%
\centering{}\caption{\label{tab:Default-nudges}Default nudges}
\begin{tabular}{|c|c|}
\hline 
right delimiter & nudge\tabularnewline
\hline 
\hline 
), > & -6\tabularnewline
\hline 
\textbackslash{}\} & -4\tabularnewline
\hline 
|, {]} & 0\tabularnewline
\hline 
other & 0\tabularnewline
\hline 
\end{tabular}\end{wraptable}%
Note that if one wants to use the nudge override with the default delimiters,
it is necessary to indicate all preceding slots, even if they are empty,
e.g., \texttt{\{;;;-3\}}. Similarly, to change the delimiters, to parentheses
say, without sub- or superscript, it is necessary to indicate all preceding
empty slots, but the following one does not need to be indicated: \texttt{\{;;()\}}.
If one wants to specify a superscript, 2 say, but leave all else unchanged,
it is only necessary to specify the one preceding empty slot: \texttt{\{;2\}}.
Trailing empty slots can be omitted, which is why, if one wants to use
the trailing empty argument `as nature intended', i.e., to specify a
point of evaluation or variables held constant, one can close one's mind
to the other potential slots and simply write (for instance) \texttt{\{0\}}
or \texttt{\{x=1\}}.%
\end{minipage}

\subsection{The enhanced mixed partial derivative algorithm\label{subsec:The-enhanced-mixed}}

In the documentation for \texttt{diffcoeff.sty}  I discussed the transition
table, Table~\ref{tab:Input-output-states}, in which signed \strong{s},
numeric \strong{n}, or algebraic \strong{a} states changed to one of
the others, or not, depending on the nature of the current token: sign,
digit or variable. Signs and digits were explicitly defined; anything and
everything else was called a (prime) variable. (Not quite true: in fact
\texttt{diffcoeff.sty} checked for \texttt{(}, \texttt{\textasciicircum{}}
and \texttt{\_} and raised an error if they were encountered.)\texttt{ }

\begin{table}[h]
\noindent \begin{centering}
\caption{A first enhancement}
\subfloat[\label{tab:Input-output-states}State transitions]{\noindent \centering{}\medskip{}
\begin{tabular}{ccccc}
\cmidrule{2-5} 
 & Curr. state & Curr. token & Action & Next state\tabularnewline
\cmidrule{2-5} 
1 & \strong{s} & $s$ & $Ts\to s'$; $T=s'$ & \strong{s}\tabularnewline
\cmidrule{2-5} 
2 & \strong{s} & $d$ & $Td$ & \strong{n}\tabularnewline
\cmidrule{2-5} 
3 & \strong{s} & $v$ & $Vv$; $T1v$ & \strong{a}\tabularnewline
\cmidrule{2-5} 
4 & \strong{n} & $s$ & $\mathbf{N}T$; $T=s$ & \strong{s}\tabularnewline
\cmidrule{2-5} 
5 & \strong{n} & $d$ & $Td$ & \strong{n}\tabularnewline
\cmidrule{2-5} 
6 & \strong{n} & $v$ & $Vv$; $Tv$ & \strong{a}\tabularnewline
\cmidrule{2-5} 
7 & \strong{a} & $s$ & $\mathbf{V}V,$; $V=\textrm{�}$; $\mathbf{A}T$; $T=s$ & \strong{s}\tabularnewline
\cmidrule{2-5} 
8 & \strong{a} & $d$ & error & \strong{!!}\tabularnewline
\cmidrule{2-5} 
9 & \strong{a} & $v$ & $Vv$; $Tv$ & \strong{a}\tabularnewline
\cmidrule{2-5} 
\end{tabular}}
\par\end{centering}
\noindent \centering{}\subfloat[\label{tab:Allowing-powers-variables}Allowing powers of variables]{\centering{}%
\begin{tabular}{ccccc}
\cmidrule{2-5} 
 & Curr. state & Curr. token & Action & Next state\tabularnewline
\cmidrule{2-5} 
8 & \strong{a} & $d$ & $Vd$; $Td$  & \strong{a}\tabularnewline
\cmidrule{2-5} 
\end{tabular}}
\end{table}
There is a certain inner logic at play here. Multi-token variables like
$kmn$ are included in the above scheme. But having accommodated $mn$,
surely one should be able to handle $mm$, i.e. $m^{2}$? And if $m^{2}$,
then why not $m^{n}$? In fact it is easy to do so. Since the superscript
token \textasciicircum{} is neither sign nor digit, no longer raise an
error if it is encountered but treat it, among the `everything else'
tokens, as a variable. If we change row 8 of the table as in Table~\ref{tab:Allowing-powers-variables}
we have enlarged our scheme to include powers of variables \textendash{}
not only numerical powers (row 8) but also algebraic powers (row 9). As
a side-effect, if we also suppress the raising of an error when the subscript
token \texttt{\_} is encountered, it too will be classified as a variable
and allow numeric and algebraic subscripts on variables: things like $k_{2}$
or $k_{n}$.

Implicit in this discussion is the understanding that exponents and subscripts
are restricted to \emph{single tokens}. Coping with multi-token quantities
in those positions would entail changes to other parts of the code, which
I have chosen not to do. 

This is a simple way of enlarging the range of tokens acceptable to the
overall-order algorithm, but it does assume that the user does \emph{not}
include a sign as a superscript or subscript. If they do, then when the
algorithm meets the sign it arrives at row 7 of the table and stores what
is clearly an unintended variable, something like \texttt{k\textasciicircum{}}
or \texttt{k\_}. So, we need to check when a sign is met whether the previous
token was one of \texttt{\textasciicircum{}} or \texttt{\_} and raise an
error if it was. But then the thought arises: if we are going to the trouble
of checking for sub- or superscript tokens, why just raise an error? Why
not incorporate signs in sub- or superscript positions into the scheme?

To this end, we might introduce a fourth state, the \emph{script} state,
denoted by \strong{p}. A script token, denoted $p$, is one of \texttt{\textasciicircum{}}
or \texttt{\_}. There is only one way to enter a script state, and that
is by appending a script token to a \emph{variable}. Appending a script
token to a sign or number (or, indeed, another script token) raises an
error. Table~\ref{tab:Enlarged-scheme-transitios} is the result. In this
scheme, signs can be used as sub- or superscripts to variables, but not
to numbers. We might console ourselves with the thought that this is, in
any case, a limitation of the calculational engine used to evaluate our
integer expressions. The \texttt{l3int} module of the LaTeX3 bundle \texttt{l3kernel}
cannot handle powers of integers.

\begin{table}[h]
\caption{\label{tab:Enlarged-scheme-transitios}Transition states for an enlarged
scheme}

\noindent \centering{}%
\begin{tabular}{ccccc}
\cmidrule{2-5} 
 & Curr. state & Curr. token & Action & Next state\tabularnewline
\cmidrule{2-5} 
1 & \strong{s} & $s$ & $Ts\to s'$; $T=s'$ & \strong{s}\tabularnewline
\cmidrule{2-5} 
2 & \strong{s} & $d$ & $Td$ & \strong{n}\tabularnewline
\cmidrule{2-5} 
3 & \strong{s} & $v$ & $Vv$; $T1v$ & \strong{a}\tabularnewline
\cmidrule{2-5} 
4 & \strong{s} & $p$ & error & \strong{!!}\tabularnewline
\cmidrule{2-5} 
5 & \strong{n} & $s$ & $\mathbf{N}T$; $T=s$ & \strong{s}\tabularnewline
\cmidrule{2-5} 
6 & \strong{n} & $d$ & $Td$ & \strong{n}\tabularnewline
\cmidrule{2-5} 
7 & \strong{n} & $v$ & $Vv$; $Tv$ & \strong{a}\tabularnewline
\cmidrule{2-5} 
8 & \strong{n} & $p$ & error & \strong{!!}\tabularnewline
\cmidrule{2-5} 
9 & \strong{a} & $s$ & $\mathbf{V}V,$; $V=\textrm{�}$; $\mathbf{A}T$; $T=s$ & \strong{s}\tabularnewline
\cmidrule{2-5} 
10 & \strong{a} & $x\in\{dv\}$  & $Vx$; $Tx$ & \strong{a}\tabularnewline
\cmidrule{2-5} 
11 & \strong{a} & $p$ & $Vp$; $Tp$ & \strong{p}\tabularnewline
\cmidrule{2-5} 
12 & \strong{p} & $x\in\{sdv\}$  & $Vx$; $Tx$ & \strong{a}\tabularnewline
\cmidrule{2-5} 
13 & \strong{p} & $p$ & error & \strong{!!}\tabularnewline
\cmidrule{2-5} 
\end{tabular}
\end{table}
Yet this still leaves an unfinished feeling. While attaching a script token
to a sign or other script token is a nonmathematical usage, attaching a
superscript token to a number is a basic mathematical use, and so two of
the errors raised can really be ignored. For the other, the question nags:
why should we have to remember that although variables can be raised to
powers, numbers cannot be? The urge to enlarge the scheme again is irresistible.
Exponents on numbers should be accepted; but subscripts should not. The
latter is a nonmathematical usage or is used only in special contexts remote
from the present one. But that means we need to distinguish sub- and superscript
tokens. We can't lump them together as `script tokens'.

\subsubsection{Raising numbers to powers: new states}

So a first step is to enlarge the number of states. We need an \emph{exponent}
state \strong{e} when we encounter the token \textasciicircum{} and a
\emph{subscript }state \strong{b} when we encounter the token \_. That
allows us to distinguish acceptable forms like \texttt{2\textasciicircum{}3}
($2^{3}$) from unacceptable ones like \texttt{2\_3} ($2_{3}$). But how
do we know which state to transition to when we meet the \texttt{3} in
\texttt{2\textasciicircum{}3}? The current state is the exponent one \strong{e}
and the \texttt{3} could be decorating either a variable or a number. We
need to know the \emph{previous} state as well as the current one. If the
previous state is numeric we transition to a numeric state; if it is algebraic,
we transition to an algebraic state.

But that also introduces a problem. It is perfectly acceptable to add a
digit to a term in a numeric state. Normally, this is how a multi-digit
number is accumulated: \texttt{23}4, two hundred and thirty four. That
is a very different meaning from \texttt{2\textasciicircum{}34} which means
$2^{3}4$ to us (rather than $2^{34}$ since we accept only single-token
superscripts). At this point, the syntax required by the underlying engine
used for evaluating numerical expressions comes into play. For all numerical
evaluations except those involving exponents, \texttt{l3int} of the LaTeX3
kernel is used; for expressions involving exponents, \texttt{l3fp} is used.
To \texttt{l3fp}, \texttt{2\textasciicircum{}34} is read as $2^{34}$.
We need to insert a multiplication token between the \texttt{3} and \texttt{4},
which for \texttt{l3fp} is the asterisk, \texttt{{*}}. Considering the
different tokens that might follow \emph{that}, we are forced to introduce
a third new state, the \emph{multiplicative} state, \strong{m}. So, to
introduce powers of numbers means considering three new states and reference
to the previous state.

That, of course, is \emph{numeric} powers of numbers. To also allow algebraic
powers, forms like $2^{n}$, introduces further complication. These can't
be evaluated numerically, so presumably they are to be classified as variables.
We need to consider terms like $+2^{n}$, $3*2^{n}$, $3^{m}2^{n}$, and
$2^{n}m$. The problem here is that we have something that looks as if
it is going to be a number (the digit 2) but then transforms into a variable,
$2^{n}$. Do we need a \emph{fourth} new state, the entangled state \strong{q}
(the `q' as in `quantum entanglement')?

\begin{table}
\noindent \centering{}\caption{\label{tab:Revised-input-output}State transitions of the full scheme}
\medskip{}
\begin{tabular}{cccccc}
\cmidrule{2-6} 
 & $S_{-}$ & $S$ & $t\in\left\{ sdv\text{\textasciicircum\_}*\right\} $ & Action & $S_{+}$\tabularnewline
\cmidrule{2-6} 
1 &  & \strong{s} & $s$ & $Ts\to s'$; $T=s'$ & \strong{s}\tabularnewline
\cmidrule{2-6} 
2 &  & \strong{s} & $d$ & $Qd$; $Td$ & \strong{n}\tabularnewline
\cmidrule{2-6} 
3 &  & \strong{s} & $v$ & $Vv$; $T1v$ & \strong{a}\tabularnewline
\cmidrule{2-6} 
4 &  & \strong{n} & $s$ & $Q=\textrm{�}$; $\mathbf{N}T$; $T=s$ & \strong{s}\tabularnewline
\cmidrule{2-6} 
5 &  & \strong{n} & $d$  & $Qd$; $Td$  & \strong{n}\tabularnewline
\cmidrule{2-6} 
6 &  & \strong{n} & $v$ & $Q=\textrm{�}$; $Vv$; $Tv$ & \strong{a}\tabularnewline
\cmidrule{2-6} 
7 &  & \strong{n} & $\text{\textasciicircum}$ & $Q\text{\textasciicircum}$; $T\text{\textasciicircum}$ & \strong{e}\tabularnewline
\cmidrule{2-6} 
8 &  & \strong{n} & $*$ & $Q=\textrm{�}$; $T*$ & \strong{m}\tabularnewline
\cmidrule{2-6} 
9 &  & \strong{a} & $s$ & $\mathbf{V}V,$; $V=\textrm{�}$; $\mathbf{A}T$; $T=s$ & \strong{s}\tabularnewline
\cmidrule{2-6} 
10 & \strong{e} & \strong{a} & $d$ & $Vd$; $Td$ & \strong{a}\tabularnewline
\cmidrule{2-6} 
11 &  & \strong{a} & $v$ & $Vv$; $Tv$ & \strong{a}\tabularnewline
\cmidrule{2-6} 
12 &  & \strong{a} & $\text{\textasciicircum}$ & $V\text{\textasciicircum}$; $T\text{\textasciicircum}$ & \strong{e}\tabularnewline
\cmidrule{2-6} 
13 &  & \strong{a} & $\text{\_}$ & $V\text{\_}$; $T\text{\_}$ & \strong{b}\tabularnewline
\cmidrule{2-6} 
14 & \strong{a} & \strong{e} & $t\in\left\{ sdv\right\} $ & $Vt$; $Tt$  & \strong{a}\tabularnewline
\cmidrule{2-6} 
15 & \strong{n} & \strong{e} & $d$ & $Q=\textrm{�}$; $Td*$ & \strong{m}\tabularnewline
\cmidrule{2-6} 
16 & \strong{n} & \strong{e} & $v$ & $Qv$; $Tv$; $V=Q$; $Q=\textrm{�}$ & \strong{a}\tabularnewline
\cmidrule{2-6} 
17 & \strong{a} & \strong{b} & $t\in\left\{ sdv\right\} $ & $Vt$; $Tt$  & \strong{a}\tabularnewline
\cmidrule{2-6} 
18 & \strong{e} & \strong{m} & $s$ & $T1$; $\mathbf{N}T$; $T=s$ & \strong{s}\tabularnewline
\cmidrule{2-6} 
19 &  & \strong{m} & $d$ & $Qd$; $Td$  & \strong{n}\tabularnewline
\cmidrule{2-6} 
20 &  & \strong{m} & $v$ & $Vv$; $T1v$ & \strong{a}\tabularnewline
\cmidrule{2-6} 
21 &  & \strong{m} & $*$ &  & \strong{m}\tabularnewline
\cmidrule{2-6} 
\end{tabular}
\end{table}
In fact I find that these problems can all be dealt with not by creating
another state but by including another accumulator for \emph{potential}
variables. I'll call it $Q$ (from the quantum suggestion, or perhaps quasi-variable,
or even acqumulator). It stores numbers whose status has not been determined
yet: they might yet be followed by a superscript token which might in turn
be followed by a variable. Once resolved, $Q$ either transfers its contents
to $V$, the variable accumulator, and is emptied (row 16), or is emptied
forthwith (rows 4, 6, 8, 15).

Table~\ref{tab:Revised-input-output} lists the transitions. I've denoted
the previous state by $S_{-}$, the present state by $S$, and the next
state by $S_{+}$. The final row of the table is intended: do nothing if
we meet a multiplicative token when in a multiplicative state. The first
scan through an order specification (to split it into numeric and algebraic
parts) may introduce a {*} token (rows 8 and 15). We don't want to introduce
a second such token in the recursive determination of the coefficients
of variables. Hence row 21: do nothing. Also, if in the order specification
we have something like $2^{3}*3^{2}$ (since \texttt{2\textasciicircum{}33\textasciicircum{}2}
looks weird), we don't want the manually inserted {*} to cause an error
because of the automatically inserted one (row 15).

Possibilities not explicitly present in the table generally raise an error,
e.g. current state \strong{s} and current token $\textnormal{\textasciicircum}$,
or previous state \strong{n}, current state \strong{e} and current token
$s$ ($+$ or $-$), and so on. I have omitted them from the table in the
interests of space. The table is big enough already.

With this table of transitions it is now possible to handle order specifications
that include components like $n^{2}$ or $n^{m}$ or $k^{+}$ or $k_{2}$
or $k_{n}$ or $2^{2}$ or $2^{3}3^{2}$ or $2\times3^{n}$ or $2^{2}3^{n}$
or \ldots{} 

Note that the \texttt{\textbackslash{}times} token is converted internally
by \texttt{diffcoeffx.sty} to the asterisk. They can be used interchangeably
but it certainly looks more elegant. 

So, what could be better on a cool winter's evening, snug before the warmth
of the fire, a glass of sustaining liquid to hand, than to do a few mixed
partial derivatives? Like this,
\begin{example}
\textbackslash{}diffp{[}3\textasciicircum{}22\textasciicircum{}22\textasciicircum{}n+m,12\textasciicircum{}n-3m+2\textasciicircum{}3k,5m+2\textbackslash{}times2\textasciicircum{}32\textasciicircum{}n{]}\{F(x,y,z)\}\{x,y,z\}$\Longrightarrow\quad{\displaystyle \diffp[3^{2}2^{2}2^{n}+m,12^{n}-3m+2^{3}k,5m+2\times2^{3}2^{n}]{F(x,y,z)}{x,y,z}}$
\end{example}
or like this,
\begin{example}
\textbackslash{}diffp{[}k\textasciicircum{}+k\_-+1,2\textbackslash{}times
k\_-,3\textasciicircum{}2k\_-,3k\textasciicircum{}+{]}\{F(x,y,z,w)\}\{x,y,z,w\}
${\displaystyle \Longrightarrow\quad\diffp[k^{+}k_{-}+1,2\times k_{-},3^{2}k_{-},3k^{+}]{F(x,y,z,w)}{x,y,z,w}}$
\end{example}
In the first example the \texttt{\textbackslash{}times} symbol is inserted
by \texttt{diffcoeffx} in the overall order of differentiation in the numerator
so as to prevent the formation $522^{n}$ which would be read as 522 raised
to the power $n$ \textendash{} and for a similar reason it was used in
specifying the order of differentiation of the variable $z$ in the denominator
in the first example, but could and should have been deleted from the order
of differentiation of the variable $y$ in the second example. 

\subsubsection{Parentheses}

The other major shortcoming of the basic scheme outlined in \texttt{diffcoeff.sty}
was the inability to handle even the simplest instance of parentheses in
an order specification \textendash{} something like\texttt{ {[}m-(n-1),m+(n-1){]}}
which might well arise in a Taylor expansion. Indeed, there is more reason
for including these in our scheme than exponents of numbers or $+$ or
$-$ as sub- or superscripts. 

How might we fit parentheses to the scheme? We are not seeking a general
treatment. Rather we wish to be able to handle order specifications a little
more complicated (but only a little) than the one just given, say something
like \texttt{{[}m+2(n-1),m-(n-1){]}}, perhaps with nesting. In that case
the following stipulations meet our needs:
\begin{itemize}
\item a left parenthesis, (, either starts an item in the comma list, or is preceded
by a sign or a number or $*$ or (, but \emph{not} by a variable or \textasciicircum{}
or \_ or );
\item a right parenthesis, ), either concludes an item in the comma list, or
is followed by a sign or ), but \emph{not} by a number or a variable or
\textasciicircum{} or \_ or $*$ or (.
\end{itemize}
These limitations allow nesting of parentheses but not products of parentheses.
The main limitation they impose is that a variable lie \emph{within} parentheses
but not adjoining-outside. They enable us to get away with the following
`cheap and cheerful' scheme. It means we do not need to add parenthesis
states to our scheme. The particular point to note are the $+0$ insertions.
When we start parsing an expression from the left we do not know what it
contains. In particular when we meet a left parenthesis, we have no foreknowledge
of whether the parenthesised expression will be numeric, algebraic or a
mix of both. We need to prepare for both by inserting a left parenthesis
to both numeric and algebraic parts. But that brings us up against a quirk
of \texttt{l3int}, the `engine' behind the numerical evaluations performed
in \texttt{diffcoeff} and \texttt{diffcoeffx}. \texttt{l3int} objects to
an empty pair of parentheses, \texttt{()}, which we would have should either
numeric or algebraic parts be missing from the parenthesised expression.
To avoid this we insert $+0$ and \texttt{l3int} is happy.\texttt{ }

\begin{table}
\centering{}\caption{\label{tab:Parentheses}Parentheses}
\begin{tabular}{ccccc}
\cmidrule{2-5} 
 & Curr. state & Curr. token & Action & Next state\tabularnewline
\cmidrule{2-5} 
1 & \strong{s} & ( & $T\text{1*(}$; $\mathbf{N}T$; $\mathbf{A}T$; $T=+$ & \strong{s}\tabularnewline
\cmidrule{2-5} 
2 & \strong{s} & ) & $\mathbf{N})$; $\mathbf{A})$ & \strong{s}\tabularnewline
\cmidrule{2-5} 
3 & \strong{n} & ( & $T*($; $\mathbf{N}T$; $\mathbf{A}T$; $T=+$ & \strong{s}\tabularnewline
\cmidrule{2-5} 
4 & \strong{n} & ) & $T)$; $\mathbf{N}T$; $\mathbf{A}\text{+0)}$; $T=+$ & \strong{s}\tabularnewline
\cmidrule{2-5} 
5 & \strong{a} & ) & $\mathbf{V}V,$; $V=\textrm{�}$; $\mathbf{N}\text{+0)}$; $T)$; $\mathbf{A}T$
; $T=+$ & \strong{s}\tabularnewline
\cmidrule{2-5} 
6 & \strong{m} & ( & $T($; $\mathbf{N}T$; $\mathbf{A}T$; $T=+$ & \strong{s}\tabularnewline
\cmidrule{2-5} 
\end{tabular}
\end{table}

\begin{itemize}
\item Row 1. Quirks of the \texttt{l3int} module of the \LaTeX{}3 kernel mean
we need to insert \strong{1{*}} before the left parenthesis.\footnote{Specifically, \texttt{\textbackslash{}int\_eval\{}\textbf{ }\texttt{\textendash (}
or \texttt{\textbackslash{}int\_eval\{}\textbf{ }\texttt{+(} throw errors.} Note that we add $T$ to \emph{both} the numeric and algebraic parts of
the expression. We are working through our expression $\mathbf{E}$ from
the left, token by token, and have no foreknowledge of what the parenthesised
expression contains, whether algebraic terms only or numeric terms only
or some combination of both. Hence the need to prepare for both. The system
shifts to a signed state \strong{s} with $T=+$, exactly the same as when
beginning to scan $\mathbf{E}$. After all, the parenthesised expression
is an expression in itself.
\item Row 2. This is to allow nested parentheses like )). It shouldn't arise
otherwise. Because of rows 4 and 5, the first right parenthesis puts the
system into a signed state. The current term will be $T=+$, but we ignore
it and store only a right parenthesis in both numeric and algebraic parts.
\item Row 3. We already have a number present in $T$; only the asterisk needs
inserting before the parenthesis. Again we add $T$ to \emph{both} the
numeric and algebraic parts of the expression, initialise $T$ to $+$
and change the state to a signed one.
\item Row 4. We are in a numeric state. We append ) to the current term and the
current term to the numeric part of the expression. We append $+0)$ to
the algebraic part, and shift to a signed state \strong{s} with $T=+$,
as at the outset. The $+0)$ in the algebraic part is necessary to prevent
an empty parenthesis pair in $\mathbf{A}$ should the parenthesised expression
have contained \emph{no} algebraic term.
\item Row 5. We are in an algebraic state. We append ) to the current term and
the current term to the algebraic part of the expression. We append $+0)$
to the numeric part and shift to the initial signed state again. The $+0)$
in the numeric part is necessary to prevent an empty parenthesis pair in
$\mathbf{N}$ should the parenthesised expression have contained \emph{no}
numeric term.
\item Row 6. We are in the new state, the multiplicative state, and the current
token is a left parenthesis. We have already met and inserted an asterisk
(row 5); we don't need to insert another. We append ( to $T$, $T$ to
both numeric and algebraic parts and shift to the initial signed state
again. The use of the multiplicative state prevents a string of asterisks
arising (but we have been able to avoid introducing new states for left
and right parentheses).
\end{itemize}
To work through an example, suppose we have an order specificiation \strong{{[}m+(n\textendash (k\textendash 1)),m\textendash{} 2(n+(k\textendash 1)),k{]}}.
(Digit \strong{1} rather than a lower-case letter \strong{l} within the
parentheses!) Concatenating, with linking + signs gives \strong{m+(n\textendash (k-1))+m\textendash 2(n+(k\textendash 1))+k}.
Splitting into numeric and algebraic parts now results in \strong{+1{*}(\textendash 1{*}(\textendash 1))\textendash 2{*}(+1{*}(\textendash 1))}
for the numeric part, evaluating to \strong{3}, and \strong{+1m+1{*}(+1n\textendash 1{*}(+1k+0))+1m\textendash 2{*}(+1n+1{*}(+1k+0))+1k}
for the algebraic part. 

Removing \strong{m} from the latter and splitting into numeric and algebraic
parts gives \strong{+1+ 1{*}(\textendash 1{*}(+0))+1\textendash 2{*}(+1{*}(+0))}
for the numeric part, evaluating to \strong{2} which is the overall coefficient
of \strong{m}, and \strong{+1{*}(+1n\textendash 1{*}(+1k+0))\textendash 2{*}(+1n+1{*}(+1k+0))+1k}
for the algebraic part.

Now remove \strong{n} from this resulting algebraic part and again split
into parts. The result is \strong{+1{*}(+1-1{*}(+0))\textendash 2{*}(+1+1{*}(+0))}
for the numeric part, evaluating to \strong{\textendash 1} which is the
overall coefficient of \strong{n}, and \strong{+1{*}(\textendash 1{*}(+1k+0))\textendash 2{*}(+1{*}(+1k+0))+1k}
for the algebraic part. 

Removing \strong{k} from this and splitting gives \strong{+1{*}(\textendash 1{*}(+1+0))\textendash 2{*}(+1{*}(+1+0))+1}
for the numeric part, evaluating to \strong{\textendash 2} which is the
overall coefficient of \strong{k}, and \strong{+1{*}(\textendash 1{*}(+0)) \textendash 2{*}(+1{*}(+0))}
for the algebraic part. But we have run out of variables and so the process
stops at this point:
\begin{example}
\textbackslash{}diffp{[}m+(n-(k-1)),m-2(n+(k-1)),k{]}F\{x,y,z\}$\Longrightarrow{\displaystyle \hspace*{1em}\diffp[m+(n-(k-1)),m-2(n+(k-1)),k]F{x,y,z}}$
\end{example}

\end{document}
