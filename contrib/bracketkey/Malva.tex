%% Malva.tex
%
% An example for the bracketkey package (C. Heibl)
%
\documentclass[11pt, spanish]{article}
\usepackage[a5paper]{geometry} 
\usepackage{bracketkey} % load bracketkey.sty
%
%%% BEGIN DOCUMENT
\begin{document}
%
\pagestyle{empty} 
%
% header of key (alternatively you could use: \keytitle{Malva})
\vspace{3ex}
\noindent\textbf{Clave para las especies chilenas de \textit{Malva}}\footnote{A. Marticorena, Malvaceae. In: C. Marticorena \& R. Rodr\'iguez (eds.). Flora de Chile 2(3): 22-105, Editorial Universidad de Concepci\'on, Chile.} 
\vspace{1ex}
%
% the key environment:
\begin{key}{M.~} 
\leadONE{}{Br\'acteas del cal\'iculo unidas}
\name{}{2}{}
\leadTWO{Br\'acteas del cal\'iculo libres}
\name{}{3}{}
\leadONE{}{Br\'acteas lanceoladas, m\'as cortas que el c\'aliz; flores en general solitarias, raro 2-3}
\name{}{assurgentiflora}{(Kellogg) M.F.Ray}
\leadTWO{Br\'acteas del cal\'iculo anchas, ovadas, excediendo el c\'aliz; flores usualmente agrupadas}
\name{}{dendromorpha}{M.F.Ray}
\leadONE{1}{Hojas superiores laciniadas}
\name{}{moschata}{L.}
\leadTWO{Hojas superiores lobadas}
\name{}{4}{}
\leadONE{}{Br\'acteas del cal\'iculo lineares}
\name{}{5}{}
\leadTWO{Br\'acteas del cal\'iculo ovadas}
\name{}{6}{}
\leadONE{}{Mericarpos lisos}
\name{}{neglecta}{Wallr.}
\leadTWO{Mericarpos rugosos }
\name{}{parviflora}{L.}
\leadONE{4}{P\'etalos de 5-15 mm de largo, pedicelos de casi el largo del c\'aliz }
\name{}{nicaeensis}{All.}
\leadTWO{P\'etalos de 16-25 mm de largo, pedicelos 3-4 veces m\'as largos que el c\'aliz}
\name{}{sylvestris}{L.}
\end{key}
\end{document}