% \iffalse meta-comment
%
% b1encoding.dtx
%
%  Author: Peter Wilson (Herries Press) herries dot press at earthlink dot net
%  Copyright 2005 Peter R. Wilson
%
%  This work may be distributed and/or modified under the
%  conditions of the Latex Project Public License, either
%  version 1.3 of this license or (at your option) any
%  later version.
%  The latest version of the license is in
%    http://www.latex-project.org/lppl.txt
%  and version 1.3 or later is part of all distributions of
%  LaTeX version 2003/06/01 or later.
%
%  This work has the LPPL maintenance status "author-maintained".
%
%  This work consists of the files listed in the README file.
%
%
%<*driver>
\documentclass[twoside]{ltxdoc}
\usepackage{url}
\usepackage[draft=false,
            plainpages=false,
            pdfpagelabels,
            bookmarksnumbered,
            hyperindex=false
           ]{hyperref}
\providecommand{\phantomsection}{}
%%\OnlyDescription %% comment this out for the full glory
\EnableCrossrefs
\CodelineIndex
\setcounter{StandardModuleDepth}{1}
\makeatletter
  \@mparswitchfalse
\makeatother
\renewcommand{\MakeUppercase}[1]{#1}
\pagestyle{headings}
\newenvironment{addtomargins}[1]{%
  \begin{list}{}{%
    \topsep 0pt%
    \addtolength{\leftmargin}{#1}%
    \addtolength{\rightmargin}{#1}%
    \listparindent \parindent
    \itemindent \parindent
    \parsep \parskip}%
  \item[]}{\end{list}}
\begin{document}
  \raggedbottom
  \DocInput{b1encoding.dtx}
\end{document}
%</driver>
%
% \fi
%
% \CheckSum{461}
%
% \DoNotIndex{\',\.,\@M,\@@input,\@addtoreset,\@arabic,\@badmath}
% \DoNotIndex{\@centercr,\@cite}
% \DoNotIndex{\@dotsep,\@empty,\@float,\@gobble,\@gobbletwo,\@ignoretrue}
% \DoNotIndex{\@input,\@ixpt,\@m}
% \DoNotIndex{\@minus,\@mkboth,\@ne,\@nil,\@nomath,\@plus,\@set@topoint}
% \DoNotIndex{\@tempboxa,\@tempcnta,\@tempdima,\@tempdimb}
% \DoNotIndex{\@tempswafalse,\@tempswatrue,\@viipt,\@viiipt,\@vipt}
% \DoNotIndex{\@vpt,\@warning,\@xiipt,\@xipt,\@xivpt,\@xpt,\@xviipt}
% \DoNotIndex{\@xxpt,\@xxvpt,\\,\ ,\addpenalty,\addtolength,\addvspace}
% \DoNotIndex{\advance,\Alph,\alph}
% \DoNotIndex{\arabic,\ast,\begin,\begingroup,\bfseries,\bgroup,\box}
% \DoNotIndex{\bullet}
% \DoNotIndex{\cdot,\cite,\CodelineIndex,\cr,\day,\DeclareOption}
% \DoNotIndex{\def,\DisableCrossrefs,\divide,\DocInput,\documentclass}
% \DoNotIndex{\DoNotIndex,\egroup,\ifdim,\else,\fi,\em,\endtrivlist}
% \DoNotIndex{\EnableCrossrefs,\end,\end@dblfloat,\end@float,\endgroup}
% \DoNotIndex{\endlist,\everycr,\everypar,\ExecuteOptions,\expandafter}
% \DoNotIndex{\fbox}
% \DoNotIndex{\filedate,\filename,\fileversion,\fontsize,\framebox,\gdef}
% \DoNotIndex{\global,\halign,\hangindent,\hbox,\hfil,\hfill,\hrule}
% \DoNotIndex{\hsize,\hskip,\hspace,\hss,\if@tempswa,\ifcase,\or,\fi,\fi}
% \DoNotIndex{\ifhmode,\ifvmode,\ifnum,\iftrue,\ifx,\fi,\fi,\fi,\fi,\fi}
% \DoNotIndex{\input}
% \DoNotIndex{\jobname,\kern,\leavevmode,\let,\leftmark}
% \DoNotIndex{\list,\llap,\long,\m@ne,\m@th,\mark,\markboth,\markright}
% \DoNotIndex{\month,\newcommand,\newcounter,\newenvironment}
% \DoNotIndex{\NeedsTeXFormat,\newdimen}
% \DoNotIndex{\newlength,\newpage,\nobreak,\noindent,\null,\number}
% \DoNotIndex{\numberline,\OldMakeindex,\OnlyDescription,\p@}
% \DoNotIndex{\pagestyle,\par,\paragraph,\paragraphmark,\parfillskip}
% \DoNotIndex{\penalty,\PrintChanges,\PrintIndex,\ProcessOptions}
% \DoNotIndex{\protect,\ProvidesClass,\raggedbottom,\raggedright}
% \DoNotIndex{\refstepcounter,\relax,\renewcommand,\reset@font}
% \DoNotIndex{\rightmargin,\rightmark,\rightskip,\rlap,\rmfamily,\roman}
% \DoNotIndex{\roman,\secdef,\selectfont,\setbox,\setcounter,\setlength}
% \DoNotIndex{\settowidth,\sfcode,\skip,\sloppy,\slshape,\space}
% \DoNotIndex{\symbol,\the,\trivlist,\typeout,\tw@,\undefined,\uppercase}
% \DoNotIndex{\usecounter,\usefont,\usepackage,\vfil,\vfill,\viiipt}
% \DoNotIndex{\viipt,\vipt,\vskip,\vspace}
% \DoNotIndex{\wd,\xiipt,\year,\z@}
%
% \changes{v1.0}{2005/11/27}{First public release}
%
% \def\fileversion{v1.0} \def\filedate{2005/11/27}
% \newcommand*{\Lpack}[1]{\textsf {#1}}           ^^A typeset a package
% \newcommand*{\Lopt}[1]{\textsf {#1}}            ^^A typeset an option
% \newcommand*{\file}[1]{\texttt {#1}}            ^^A typeset a file
% \newcommand*{\Lcount}[1]{\textsl {\small#1}}    ^^A typeset a counter
% \newcommand*{\pstyle}[1]{\textsl {#1}}          ^^A typeset a pagestyle
% \newcommand*{\Lenv}[1]{\texttt {#1}}            ^^A typeset an environment
% \newcommand*{\AD}{\textsc{ad}}
% \newcommand*{\thisfont}{OandS}
%
% \title{The \Lpack{B1} encoding files\thanks{This
%        file has version number \fileversion, last revised
%        \filedate.}}
%
% \author{%
% Peter Wilson\thanks{\texttt{herries dot press at earthlink dot net}}\\
% Herries Press
% }
% \date{\filedate}
% \maketitle
% \begin{abstract}
%    A set of encoding files are provided for the B1 encoding for the
% \Lpack{bookhands} fonts.
% \end{abstract}
% \tableofcontents
%
% \section{Introduction}
%
%    The \Lpack{bookhands} fonts have several glyphs that are not in modern
% fonts while also excluding many of the modern glyphs. I have created
% a new encoding for the \Lpack{bookhands} which I have called B1, or 
% TeXBookHands1 (TeXBH1).
%
%   To use the B1 encoding you need to have at least the files \file{b1enc.def}
% (similar to \file{t1enc.def}) and \file{b1cmr.fd} (similar to
% \file{t1cmr.fd}) installed on your LaTeX system.
%
%   To use a B1 encoded font in a document the preamble must have at least: \\
% \verb?\usepackage[...,B1,...]{fontenc}?
%
% This manual is typeset according to the conventions of the
% \LaTeX{} \textsc{docstrip} utility which enables the automatic
% extraction of the \LaTeX{} macro source files~\cite{COMPANION}.
%
%
%
% \StopEventually{
% \bibliographystyle{alpha}
% \renewcommand{\refname}{Bibliography}
% \begin{thebibliography}{GMS94}
% \addcontentsline{toc}{section}{\refname}
%
% \bibitem[MG04]{COMPANION}
% Frank Mittelbach and Michel Goossens.
% \newblock \emph{The LaTeX Companion}.
% \newblock Second edition.
% \newblock Addison-Wesley Publishing Company, 2004.
%
%
% \end{thebibliography}
%
% \PrintIndex
%
% }
%
%
%
% \section{The code} \label{sec:mf}
%
%
% \subsection{The \file{b1enc.def} file}
%
%    The \file{b1enc.def} file is the \Lpack{B1} encoding's equivalent to
% the \Lpack{T1} encoding's \file{t1enc.def}, the original of which is
% maintained in the kernel's \file{ltoutenc.dtx} file.
%
%    This file should be put somewhere where LaTeX will look for *.def files.
% For example: \url{/usr/TeX/texmf-local/tex/latex/bookhands/b1enc.def}.
% 
%    \begin{macrocode}
%<*B1>
\ProvidesFile{b1enc.def}
 [2005/08/09 v0.1
         Definitions for the TeXBookhands1 encoding]
\DeclareFontEncoding{B1}{}{}
%    \end{macrocode}
% The majority of the accents. I keep slots (o000/h00/d0), (o012/h0A/d10)
% and (o015/h0D/d13) empty; glyphs that would normally be in those slots are
% moved further down.
%    \begin{macrocode}
%%\DeclareTextAccent{\`}{B1}{0}% see 23
\DeclareTextAccent{\'}{B1}{1}
\DeclareTextAccent{\^}{B1}{2}
\DeclareTextAccent{\~}{B1}{3}
\DeclareTextAccent{\"}{B1}{4}
\DeclareTextAccent{\H}{B1}{5}
\DeclareTextAccent{\r}{B1}{6}
\DeclareTextAccent{\v}{B1}{7}
\DeclareTextAccent{\u}{B1}{8}
\DeclareTextAccent{\=}{B1}{9}
%%\DeclareTextAccent{\.}{B1}{10}% see 24
%    \end{macrocode}
% Some accents are built by hand.
%    \begin{macrocode}
%%% barunder (\b) 9
\DeclareTextCommand{\b}{B1}[1]
   {\hmode@bgroup\o@lign{\relax#1\crcr\hidewidth\sh@ft{29}%
     \vbox to.2ex{\hbox{\char9}\vss}\hidewidth}\egroup}
%%% cedilla (\c) 11
\DeclareTextCommand{\c}{B1}[1]
   {\leavevmode\setbox\z@\hbox{#1}\ifdim\ht\z@=1ex\accent11 #1%
     \else{\ooalign{\hidewidth\char11\hidewidth
        \crcr\unhbox\z@}}\fi}
%%% dotunder (\d)
\DeclareTextCommand{\d}{B1}[1]
   {\hmode@bgroup
    \o@lign{\relax#1\crcr\hidewidth\sh@ft{10}.\hidewidth}\egroup}
%%% ogonek 12
\DeclareTextCommand{\k}{B1}[1]
   {\oalign{\null#1\crcr\hidewidth\char12}}
%%\DeclareTextCommand{\textperthousand}{B1}
%%   {\%\char 24 }          % space or `relax as delimiter?
%%\DeclareTextCommand{\textpertenthousand}{B1}
%%   {\%\char 24\char 24 }  % space or `relax as delimiter?
%    \end{macrocode}
% Punctuation marks.
%    \begin{macrocode}
%%\DeclareTextSymbol{\quotesinglbase}{B1}{13}
\DeclareTextSymbol{\guilsinglleft}{B1}{14}
\DeclareTextSymbol{\guilsinglright}{B1}{15}
\DeclareTextSymbol{\textquotedblleft}{B1}{16}
\DeclareTextSymbol{\textquotedblright}{B1}{17}
\DeclareTextSymbol{\quotedblbase}{B1}{18}
\DeclareTextSymbol{\guillemotleft}{B1}{19}
\DeclareTextSymbol{\guillemotright}{B1}{20}
\DeclareTextSymbol{\textendash}{B1}{21}
\DeclareTextSymbol{\textemdash}{B1}{22}
\DeclareTextAccent{\`}{B1}{23}
%%%\DeclareTextSymbol{\textcompwordmark}{B1}{23}
\DeclareTextAccent{\.}{B1}{24}
\DeclareTextSymbol{\i}{B1}{25}
\DeclareTextSymbol{\j}{B1}{26}
%    \end{macrocode}
% The `f' ligatures, which TeX takes care of.
% \begin{verbatim}
%%% ff 27, fi 28, fl 29, ffi 30, ffl 31
% \end{verbatim}
% A few other symbols, but on the whole we are into ASCII territory.
%    \begin{macrocode}
\DeclareTextSymbol{\textvisiblespace}{B1}{32}
\DeclareTextSymbol{\textquotedbl}{B1}{`\"}% 34
%%% # 35
\DeclareTextSymbol{\textdollar}{B1}{`\$}% 36
%%% % 37
%%% & 38
\DeclareTextSymbol{\textquoteright}{B1}{`\'}% 39
%    \end{macrocode}
% \begin{verbatim}
%%% ( 40, ) 41, * 42, + 43, , 44, - 45, . 46, / 47, 
%%% 0 48, 1 49, 2 50, 3 51, 4 52, 5 53, 6 54, 7 55, 8 56, 9 57,
%%% : 58, ; 59,
% \end{verbatim}
% A couple more symbols.
%    \begin{macrocode}
\DeclareTextSymbol{\textless}{B1}{`\<}% 60
%%% = 61
\DeclareTextSymbol{\textgreater}{B1}{`\>}% 62
%    \end{macrocode}
% Back into normal ASCII mode.
% \begin{verbatim}
%%% ? 63, @ 64,
%%% A 65, B 66, C 67, D 68, E 69, F 70, G 71, H 72, I 73, J 74,
%%% K 75, L 76, M 77, N 78, O 79, P 80, Q 81, R 82, S 83, T 84,
%%% U 85, V 86, W 87, X 88, Y 89, Z 90,
%%% [ 91, \ 92,
% \end{verbatim}
% Another little symbol group.
%    \begin{macrocode}
\DeclareTextSymbol{\textbackslash}{B1}{`\\}% 92
%%% ] 93
\DeclareTextSymbol{\textasciicircum}{B1}{`\^}% 94
\DeclareTextSymbol{\textunderscore}{B1}{95}
\DeclareTextSymbol{\textquoteleft}{B1}{`\`}% 96
%    \end{macrocode}
% \begin{verbatim}
%%% a 97,  b 98,  c 99,  d 100, e 101, f 102, g 103, h 104, i 105, j 106,
%%% k 107, l 108, m 109, n 110, o 111, p 112, q 113, r 114, s 115, t 116,
%%% u 117, v 118, w 119, x 120, y 121, z 122,
% \end{verbatim}
% Now we are into the post-ASCII realm. Most of the remainder are 
% accented characters,
%    \begin{macrocode}
\DeclareTextComposite{\.}{B1}{i}{`\i}% 
\DeclareTextComposite{\.}{B1}{\i}{`\i}% 
\DeclareTextSymbol{\textbraceleft}{B1}{`\{}% 123
\DeclareTextSymbol{\textbar}{B1}{`\|}% 124
\DeclareTextSymbol{\textbraceright}{B1}{`\}}% 125
\DeclareTextSymbol{\textasciitilde}{B1}{`\~}% 126
%    \end{macrocode}
% Replace T1 \u{A} by \cs{quotesinglbase}.
%    \begin{macrocode}
%%\DeclareTextComposite{\u}{B1}{A}{128}
\DeclareTextSymbol{\quotesinglbase}{B1}{128}
\DeclareTextComposite{\k}{B1}{A}{129}
\DeclareTextComposite{\'}{B1}{C}{130}
\DeclareTextComposite{\v}{B1}{C}{131}
\DeclareTextComposite{\v}{B1}{D}{132}
\DeclareTextComposite{\v}{B1}{E}{133}
\DeclareTextComposite{\k}{B1}{E}{134}
%    \end{macrocode}
% Replace T1 \u{G} by \cs{textparagraph}.
%    \begin{macrocode}
%%\DeclareTextComposite{\u}{B1}{G}{135}
\DeclareTextSymbol{\textparagraph}{B1}{135}
\DeclareTextComposite{\'}{B1}{L}{136}
\DeclareTextComposite{\v}{B1}{L}{137}
\DeclareTextSymbol{\L}{B1}{138}
\DeclareTextComposite{\'}{B1}{N}{139}
\DeclareTextComposite{\v}{B1}{N}{140}
\DeclareTextSymbol{\NG}{B1}{141}
\DeclareTextComposite{\H}{B1}{O}{142}
\DeclareTextComposite{\'}{B1}{R}{143}
\DeclareTextComposite{\v}{B1}{R}{144}
\DeclareTextComposite{\'}{B1}{S}{145}
\DeclareTextComposite{\v}{B1}{S}{146}
%    \end{macrocode}
% Replace T1 \c{S} by \cs{textslongt} --- the (long) s-t ligature.
%    \begin{macrocode}
%%\DeclareTextComposite{\c}{B1}{S}{147}
%% longs-t ligature {147}
\DeclareTextSymbol{\textslongt}{B1}{147}
\DeclareTextComposite{\v}{B1}{T}{148}
%    \end{macrocode}
% Replace T1 \c{T} by \cs{texthalfr} --- a half r glyph
%    \begin{macrocode}
%%\DeclareTextComposite{\c}{B1}{T}{149}
%% half-r {149}
\DeclareTextSymbol{\textrhalf}{B1}{149}
\DeclareTextComposite{\H}{B1}{U}{150}
\DeclareTextComposite{\r}{B1}{U}{151}
\DeclareTextComposite{\"}{B1}{Y}{152}
\DeclareTextComposite{\'}{B1}{Z}{153}
\DeclareTextComposite{\v}{B1}{Z}{154}
\DeclareTextComposite{\.}{B1}{Z}{155}
\DeclareTextComposite{\.}{B1}{I}{157}
\DeclareTextSymbol{\dj}{B1}{158}
\DeclareTextSymbol{\textsection}{B1}{159}
%    \end{macrocode}
% Replace T1 \u{a} by \cs{textslong} --- the long s glyph
%    \begin{macrocode}
%%\DeclareTextComposite{\u}{B1}{a}{160}
\DeclareTextSymbol{\textslong}{B1}{160}
\DeclareTextComposite{\k}{B1}{a}{161}
\DeclareTextComposite{\'}{B1}{c}{162}
\DeclareTextComposite{\v}{B1}{c}{163}
\DeclareTextComposite{\v}{B1}{d}{164}
\DeclareTextComposite{\v}{B1}{e}{165}
\DeclareTextComposite{\k}{B1}{e}{166}
%    \end{macrocode}
% Replace T1 \u{g} by \cs{textet} --- the e-t ligature (this is not the \&).
%    \begin{macrocode}
%%\DeclareTextComposite{\u}{B1}{g}{167}
%% e-t ligature {167}
\DeclareTextSymbol{\textet}{B1}{167}
\DeclareTextComposite{\'}{B1}{l}{168}
\DeclareTextComposite{\v}{B1}{l}{169}
\DeclareTextSymbol{\l}{B1}{170}
\DeclareTextComposite{\'}{B1}{n}{171}
\DeclareTextComposite{\v}{B1}{n}{172}
\DeclareTextSymbol{\ng}{B1}{173}
\DeclareTextComposite{\H}{B1}{o}{174}
\DeclareTextComposite{\'}{B1}{r}{175}
\DeclareTextComposite{\v}{B1}{r}{176}
\DeclareTextComposite{\'}{B1}{s}{177}
\DeclareTextComposite{\v}{B1}{s}{178}
%    \end{macrocode}
% Replace the T1 \c{s} by \cs{textst} --- the (short) s-t ligature
%    \begin{macrocode}
%%\DeclareTextComposite{\c}{B1}{s}{179}
%% s-t ligature {179}
\DeclareTextSymbol{\textst}{B1}{179}
\DeclareTextComposite{\v}{B1}{t}{180}
%    \end{macrocode}
% Replace the T1 \c{t} by \cs{textct} --- the c-t ligature
%    \begin{macrocode}
%%\DeclareTextComposite{\c}{B1}{t}{181}
%% c-t ligature {181}
\DeclareTextSymbol{\textct}{B1}{181}
\DeclareTextComposite{\H}{B1}{u}{182}
\DeclareTextComposite{\r}{B1}{u}{183}
\DeclareTextComposite{\"}{B1}{y}{184}
\DeclareTextComposite{\'}{B1}{z}{185}
\DeclareTextComposite{\v}{B1}{z}{186}
\DeclareTextComposite{\.}{B1}{z}{187}
\DeclareTextSymbol{\textexclamdown}{B1}{189}
\DeclareTextSymbol{\textquestiondown}{B1}{190}
\DeclareTextSymbol{\textsterling}{B1}{191}
\DeclareTextComposite{\`}{B1}{A}{192}
\DeclareTextComposite{\'}{B1}{A}{193}
\DeclareTextComposite{\^}{B1}{A}{194}
\DeclareTextComposite{\~}{B1}{A}{195}
\DeclareTextComposite{\"}{B1}{A}{196}
\DeclareTextComposite{\r}{B1}{A}{197}
\DeclareTextSymbol{\AE}{B1}{198}
\DeclareTextComposite{\c}{B1}{C}{199}
\DeclareTextComposite{\`}{B1}{E}{200}
\DeclareTextComposite{\'}{B1}{E}{201}
\DeclareTextComposite{\^}{B1}{E}{202}
\DeclareTextComposite{\"}{B1}{E}{203}
\DeclareTextComposite{\`}{B1}{I}{204}
\DeclareTextComposite{\'}{B1}{I}{205}
\DeclareTextComposite{\^}{B1}{I}{206}
\DeclareTextComposite{\"}{B1}{I}{207}
\DeclareTextSymbol{\DH}{B1}{208}
\DeclareTextSymbol{\DJ}{B1}{208}
\DeclareTextComposite{\~}{B1}{N}{209}
\DeclareTextComposite{\`}{B1}{O}{210}
\DeclareTextComposite{\'}{B1}{O}{211}
\DeclareTextComposite{\^}{B1}{O}{212}
\DeclareTextComposite{\~}{B1}{O}{213}
\DeclareTextComposite{\"}{B1}{O}{214}
\DeclareTextSymbol{\OE}{B1}{215}
\DeclareTextSymbol{\O}{B1}{216}
\DeclareTextComposite{\`}{B1}{U}{217}
\DeclareTextComposite{\'}{B1}{U}{218}
\DeclareTextComposite{\^}{B1}{U}{219}
\DeclareTextComposite{\"}{B1}{U}{220}
\DeclareTextComposite{\'}{B1}{Y}{221}
\DeclareTextSymbol{\TH}{B1}{222}
\DeclareTextSymbol{\SS}{B1}{223}
\DeclareTextComposite{\`}{B1}{a}{224}
\DeclareTextComposite{\'}{B1}{a}{225}
\DeclareTextComposite{\^}{B1}{a}{226}
\DeclareTextComposite{\~}{B1}{a}{227}
\DeclareTextComposite{\"}{B1}{a}{228}
\DeclareTextComposite{\r}{B1}{a}{229}
\DeclareTextSymbol{\ae}{B1}{230}
\DeclareTextComposite{\c}{B1}{c}{231}
\DeclareTextComposite{\`}{B1}{e}{232}
\DeclareTextComposite{\'}{B1}{e}{233}
\DeclareTextComposite{\^}{B1}{e}{234}
\DeclareTextComposite{\"}{B1}{e}{235}
\DeclareTextComposite{\`}{B1}{i}{236}
\DeclareTextComposite{\`}{B1}{\i}{236}
\DeclareTextComposite{\'}{B1}{i}{237}
\DeclareTextComposite{\'}{B1}{\i}{237}
\DeclareTextComposite{\^}{B1}{i}{238}
\DeclareTextComposite{\^}{B1}{\i}{238}
\DeclareTextComposite{\"}{B1}{i}{239}
\DeclareTextComposite{\"}{B1}{\i}{239}
\DeclareTextSymbol{\dh}{B1}{240}
\DeclareTextComposite{\~}{B1}{n}{241}
\DeclareTextComposite{\`}{B1}{o}{242}
\DeclareTextComposite{\'}{B1}{o}{243}
\DeclareTextComposite{\^}{B1}{o}{244}
\DeclareTextComposite{\~}{B1}{o}{245}
\DeclareTextComposite{\"}{B1}{o}{246}
\DeclareTextSymbol{\oe}{B1}{247}
\DeclareTextSymbol{\o}{B1}{248}
\DeclareTextComposite{\`}{B1}{u}{249}
\DeclareTextComposite{\'}{B1}{u}{250}
\DeclareTextComposite{\^}{B1}{u}{251}
\DeclareTextComposite{\"}{B1}{u}{252}
\DeclareTextComposite{\'}{B1}{y}{253}
\DeclareTextSymbol{\th}{B1}{254}
\DeclareTextSymbol{\ss}{B1}{255}

%</B1>
%    \end{macrocode}
%
% \subsection{The \file{TeXB1.enc} file}
%
% The \file{TeXB1.enc} file is the \Lpack{B1} encoding's version of
% the \Lpack{T1} encoding's \file{8b.enc} file. It provides the PostScript 
% encoding vector for Type 1 versions of \Lpack{B1} fonts. Slots 
% (o000/h00/d0), (o012/h0A/d10) and (o015/h0D/d13) are empty.
%
%    This file should be put somewhere where LaTeX will look for *.enc files.
% For example: \url{/usr/TeX/texmf-local/tex/latex/bookhands/TeXB1.enc}.
% 
%    \begin{macrocode}
%<*enc>
/TeXB1 [
%%%% Comments are the 8b.enc entries
 /.notdef        % 0
 /acute          % 1 /dotaccent    % 1
 /circumflex     % 2 /fi           % 2
 /tilde          % 3 /fl           % 3
 /dieresis       % 4 /fraction     % 4
 /hungarumlaut   % 5
 /ring           % 6 /Lslash       % 6
 /caron          % 7 /lslash       % 7
 /breve          % 8 /ogonek       % 8
 /macron         % 9 /ring          % 9
 /.notdef        % 10
 /cedilla        % 11 /breve        % 11
 /ogonek         % 12 /minus        % 12
 /.notdef        % 13
 /guilsinglleft  % 14 /Zcaron       % 14
 /guilsinglright % 15 /zcaron       % 15
% 0x10
 /quotedblleft   % 16 /caron         % 16
 /quotedblright  % 17 /dotlessi      % 17
 /quotedblbase   % 18 /dotlessj      % 18
 /guillemotleft  % 19 /ff            % 19
 /guillemotright % 20 /ffi           % 20
 /endash         % 21 /ffl           % 21
 /emdash         % 22 /.notdef       % 22
 /grave          % 23 /.notdef       % 23
 /dotaccent      % 24 /.notdef       % 24
 /dotlessi       % 25 /.notdef       % 25
 /dotlessj       % 26 /.notdef       % 26
 /ff             % 27 /.notdef       % 27
 /fi             % 28 /.notdef       % 28
 /fl             % 29 /.notdef       % 29
 /ffi            % 30 /grave         % 30
 /ffl            % 31 /quotesingle   % 31
% 0x20 (ASCII begins)
 /space          % 32
 /exclam         % 33
 /quotedbl       % 34
 /numbersign     % 35
 /dollar         % 36
 /percent        % 37
 /ampersand      % 38
 /quoteright     % 39
 /parenleft      % 40
 /parenright     % 41
 /asterisk       % 42
 /plus           % 43
 /comma          % 44
 /hyphen         % 45
 /period         % 46
 /slash          % 47
% 0x30
 /zero           % 48
 /one            % 49
 /two            % 50
 /three          % 51
 /four           % 52
 /five           % 53
 /six            % 54
 /seven          % 55
 /eight          % 56
 /nine           % 57
 /colon          % 58
 /semicolon      % 59
 /less           % 60
 /equal          % 61
 /greater        % 62
 /question       % 63
% 0x40
 /at             % 64
 /A              % 65
 /B              % 66
 /C              % 67
 /D              % 68
 /E              % 69
 /F              % 70
 /G              % 71
 /H              % 72
 /I              % 73
 /J              % 74
 /K              % 75
 /L              % 76
 /M              % 77
 /N              % 78
 /O              % 79
% 0x50
 /P              % 80
 /Q              % 81
 /R              % 82
 /S              % 83
 /T              % 84
 /U              % 85
 /V              % 86
 /W              % 87
 /X              % 88
 /Y              % 89
 /Z              % 90
 /bracketleft    % 91
 /backslash      % 92
 /bracketright   % 93
 /asciicircum    % 94
 /underscore     % 95
% 0x60
 /quoteleft      % 96
 /a              % 97
 /b              % 98
 /c              % 99
 /d              % 100
 /e              % 101
 /f              % 102
 /g              % 103
 /h              % 104
 /i              % 105
 /j              % 106
 /k              % 107
 /l              % 108
 /m              % 109
 /n              % 110
 /o              % 111
% 0x70
 /p              % 112
 /q              % 113
 /r              % 114
 /s              % 115
 /t              % 116
 /u              % 117
 /v              % 118
 /w              % 119
 /x              % 120
 /y              % 121
 /z              % 122
 /braceleft      % 123
 /bar            % 124
 /braceright     % 125
 /asciitilde     % 126
 /hyphen         % 127 /.notdef % rubout; ASCII ends         % 127
% 0x80
 /quotesinglbase % 128  /Euro            % 128
 /Aogonek        % 129 /.notdef         % 129
 /Cacute         % 130 /quotesinglbase  % 130
 /Ccaron         % 131 /florin          % 131
 /Dcaron         % 132 /quotedblbase    % 132
 /Ecaron         % 133 /ellipsis        % 133
 /Eogonek        % 134 /dagger          % 134
 /paragraph      % 135 /daggerdbl       % 135
 /Lacute         % 136 /circumflex      % 136
 /Lcaron         % 137 perthousand     % 137
 /Lslash         % 138 /Scaron          % 138
 /Nacute         % 139 /guilsinglleft   % 139
 /Ncaron         % 140 /OE              % 140
 /Eng            % 141 /.notdef         % 141
 /Ohungarumlaut  % 142 /.notdef         % 142
 /Racute         % 143 /.notdef         % 143
% 0x90
 /Rcaron         % 144 /.notdef         % 144
 /Sacute         % 145 /.notdef         % 145
 /Scaron         % 146 /.notdef         % 146
%    \end{macrocode}
% \verb?/slong_t? is my name for the long s-t ligauture
%    \begin{macrocode}
 /slong_t        % 147 /quotedblleft    % 147
 /Tcaron         % 148 /quotedblright   % 148
%    \end{macrocode}
% \verb?/r.half? is my name for the half r glyph
%    \begin{macrocode}
 /r.half         % 149 /bullet          % 149
 /Uhungarumlaut  % 150 /endash          % 150
 /Uring          % 151 /emdash          % 151
 /Ydieresis      % 152 /tilde           % 152
 /Zacute         % 153 /trademark       % 153
 /Zcaron         % 154 /scaron          % 154
 /Zdot           % 155 /guilsinglright  % 155
 /IJ             % 156 /oe              % 156
 /Idot           % 157 /.notdef         % 157
 /dcroat % (dyet)% 158 /.notdef         % 158
 /section        % 159 /Ydieresis       % 159
% 0xA0
 /slong          % 160 /.notdef         % 160
 % nobreakspace
 /aogonek        % 161 /exclamdown      % 161
 /cacute         % 162 /cent            % 162
 /ccaron         % 163 /sterling        % 163
 /dcaron         % 164 /currency        % 164
 /ecaron         % 165 /yen             % 165
 /eogonek        % 166 /brokenbar       % 166
%    \end{macrocode}
% \verb?/e_t? is my name for the e-t ligature (this is not an \&).
%    \begin{macrocode}
 /e_t            % 167 /section         % 167
 /lacute         % 168 /dieresis        % 168
 /lcaron         % 169 /copyright      % 169
 /lslash         % 170 /ordfeminine     % 170
 /nacute         % 171 /guillemotleft   % 171
 /ncaron         % 172 /logicalnot      % 172
 /eng            % 173 /hyphen % Y&Y (also at 45); Windows' softhyphen % 173
 /ohungarumlaut  % 174 /registered      % 174
 /racute         % 175 /macron          % 175
% 0xD0
 /rcaron         % 176 /degree          % 176
 /sacute         % 177 /plusminus       % 177
 /scaron         % 178 /twosuperior     % 178
%    \end{macrocode}
% \verb?/s_t? is my name for the (short) s-t ligature.
%    \begin{macrocode}
 /s_t            % 179 /threesuperior   % 179
 /tcaron         % 180 /acute           % 180
%    \end{macrocode}
% \verb?/c_t? is my name for the c-t ligature.
%    \begin{macrocode}
 /c_t            % 181 /mu              % 181
 /uhungarumlaut  % 182 /paragraph       % 182
 /uring          % 183 /periodcentered  % 183
 /ydieresis      % 184 cedilla         % 184
 /zacute         % 185 /onesuperior     % 185
 /zcaron         % 186 /ordmasculine    % 186
 /zdot           % 187 /guillemotright  % 187
 /ij             % 188 /onequarter      % 188
 /exclamdown     % 189 /onehalf         % 189
 /questiondown   % 190 /threequarters   % 190
 /sterling       % 191 /questiondown    % 191
% 0xC0
 /Agrave         % 192
 /Aacute         % 193
 /Acircumflex    % 194
 /Atilde         % 195
 /Adieresis      % 196
 /Aring          % 197
 /AE             % 198
 /Ccedilla       % 199
 /Egrave         % 200
 /Eacute         % 201
 /Ecircumflex    % 202
 /Edieresis      % 203
 /Igrave         % 204
 /Iacute         % 205
 /Icircumflex    % 206
 /Idieresis      % 207
% 0xD0
 /Eth            % 208
 /Ntilde         % 209
 /Ograve         % 210
 /Oacute         % 211
 /Ocircumflex    % 212
 /Otilde         % 213
 /Odieresis      % 214
 /OE             % 215 /multiply        % 215
 /Oslash         % 216
 /Ugrave         % 217
 /Uacute         % 218
 /Ucircumflex    % 219
 /Udieresis      % 220
 /Yacute         % 221
 /Thorn          % 222
 /Germandbls     % 223
% 0xE0
 /agrave         % 224
 /aacute         % 225
 /acircumflex    % 226
 /atilde         % 227
 /adieresis      % 228
 /aring          % 229
 /ae             % 230
 /ccedilla       % 231
 /egrave         % 232
 /eacute         % 233
 /ecircumflex    % 234
 /edieresis      % 235
 /igrave         % 236
 /iacute         % 237
 /icircumflex    % 238
 /idieresis      % 239
% 0xF0
 /eth            % 240
 /ntilde         % 241
 /ograve         % 242
 /oacute         % 243
 /ocircumflex    % 244
 /otilde         % 245
 /odieresis      % 246
 /oe             % 247 /divide          % 247
 /oslash         % 248
 /ugrave         % 249
 /uacute         % 250
 /ucircumflex    % 251
 /udieresis      % 252
 /yacute         % 253
 /thorn          % 254
 /germandbls     % 255  /ydieresis       % 255
] def

%</enc>
%    \end{macrocode}
%
%
% \subsection{The \file{b1cmr.fd} files}
%
%    At a minimum CMR must be provided as a B1 font, otherwise the NFSS system
% objects. I have not done a real B1 encoding for CMR, but the following
% at least keeps NFSS quiet. It is \file{t1cmr.fd} with the `t1' changed to
% `b1' throughout.
%
%    This file should be put somewhere where LaTeX will look for *.fd files.
% For example: \url{/usr/TeX/texmf-local/tex/latex/bookhands/b1cmr.fd}.
% 
%    \begin{macrocode}
%<*b1cmr>
%% This is file b1cmr.fd based on
%%         file `t1cmr.fd',
\ProvidesFile{b1cmr.fd}
        [2005/11/27 v1.0 bookhand font definitions]
\providecommand{\EC@family}[5]{%
  \DeclareFontShape{#1}{#2}{#3}{#4}%
  {<5><6><7><8><9><10><10.95><12><14.4>%
   <17.28><20.74><24.88><29.86><35.83>genb*#5}{}}
\DeclareFontFamily{B1}{cmr}{}
\EC@family{B1}{cmr}{m}{n}{ecrm}
\EC@family{B1}{cmr}{m}{sl}{ecsl}
\EC@family{B1}{cmr}{m}{it}{ecti}
\EC@family{B1}{cmr}{m}{sc}{eccc}
\EC@family{B1}{cmr}{bx}{n}{ecbx}
\EC@family{B1}{cmr}{b}{n}{ecrb}
\EC@family{B1}{cmr}{bx}{it}{ecbi}
\EC@family{B1}{cmr}{bx}{sl}{ecbl}
\EC@family{B1}{cmr}{bx}{sc}{ecxc}
\EC@family{B1}{cmr}{m}{ui}{ecui}

%</b1cmr>
%    \end{macrocode}
%
% \Finale
\endinput

%% \CharacterTable
%%  {Upper-case    \A\B\C\D\E\F\G\H\I\J\K\L\M\N\O\P\Q\R\S\T\U\V\W\X\Y\Z
%%   Lower-case    \a\b\c\d\e\f\g\h\i\j\k\l\m\n\o\p\q\r\s\t\u\v\w\x\y\z
%%   Digits        \0\1\2\3\4\5\6\7\8\9
%%   Exclamation   \!     Double quote  \"     Hash (number) \#
%%   Dollar        \$     Percent       \%     Ampersand     \&
%%   Acute accent  \'     Left paren    \(     Right paren   \)
%%   Asterisk      \*     Plus          \+     Comma         \,
%%   Minus         \-     Point         \.     Solidus       \/
%%   Colon         \:     Semicolon     \;     Less than     \<
%%   Equals        \=     Greater than  \>     Question mark \?
%%   Commercial at \@     Left bracket  \[     Backslash     \\
%%   Right bracket \]     Circumflex    \^     Underscore    \_
%%   Grave accent  \`     Left brace    \{     Vertical bar  \|
%%   Right brace   \}     Tilde         \~}


