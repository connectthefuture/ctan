% \iffalse meta-comment
% !TEX TS-program = dtxmk 
%
% Copyright (C) 2015 by John Burt <burt@brandeis.edu>
% --------------------------------------------------
% 
% This file may be distributed and/or modified under the
% conditions of the LaTeX Project Public License, either version 1.3
% of this license or (at your option) any later version.
% The latest version of this license is in:
%
%    http://www.latex-project.org/lppl.txt
%
% and version 1.2 or later is part of all distributions of LaTeX 
% version 1999/12/01 or later.
%
% \fi
%
% \iffalse
%<package>\NeedsTeXFormat{LaTeX2e}[1997/12/01]
%<package>\ProvidesPackage{poemscol}
%<package>   [2015/6/18 v2.72 poemscol file] 
%
%<*driver>
\documentclass{ltxdoc}
\usepackage{makeidx,multicol,keyval,ifthen}
\EnableCrossrefs         
\CodelineIndex
\RecordChanges
\begin{document}
  \DocInput{poemscol.dtx}
\end{document}
%</driver>
% \fi
%
% \CheckSum{7758}
%
% \CharacterTable
%  {Upper-case    \A\B\C\D\E\F\G\H\I\J\K\L\M\N\O\P\Q\R\S\T\U\V\W\X\Y\Z
%   Lower-case    \a\b\c\d\e\f\g\h\i\j\k\l\m\n\o\p\q\r\s\t\u\v\w\x\y\z
%   Digits        \0\1\2\3\4\5\6\7\8\9
%   Exclamation   \!     Double quote  \"     Hash (number) \#
%   Dollar        \$     Percent       \%     Ampersand     \&
%   Acute accent  \'     Left paren    \(     Right paren   \)
%   Asterisk      \*     Plus          \+     Comma         \,
%   Minus         \-     Point         \.     Solidus       \/
%   Colon         \:     Semicolon     \;     Less than     \<
%   Equals        \=     Greater than  \>     Question mark \?
%   Commercial at \@     Left bracket  \[     Backslash     \\
%   Right bracket \]     Circumflex    \^     Underscore    \_
%   Grave accent  \`     Left brace    \{     Vertical bar  \|
%   Right brace   \}     Tilde         \~}
%
% 
% \changes{v1.0}{2002/06/05}{Initial version}
% \changes{v1.01}{2002/06/25}{Added macros for titles of poems without
% formal titles.  Corrected typographical errors in manual.}
% \changes{v1.1}{2003/04/18}{Added commands to change mark in cases
% where page break fall (or does not fall) on stanza break.  Added
% command to change the amount of indentation in runover lines in
% verse.  Removed special geometry to make the program more generic.}
% \changes{v1.11}{2005/02/12}{Corrected bugs in headers for
% explanatory notes and emendations.}
% \changes{v1.2}{2005/2/19}{Accidentals and typescript variants test
% to see whether textual notes are being made.  Corrected a bug when
% quoting verse in note sections.  Added a command to change the page
% number of the table of contents.  Added a command to input the table
% of contents file.  Added commands to input the external files for
% emendations, explanatory notes, textual notes, and the index of
% titles and first lines.  Added a command to test whether there is an
% .aux file from a previous run and to prevent typesetting of the
% external files for emendations, explanatory notes, textual notes if
% there is not.  Added a command to restore the main page style after
% the table of contents.} \changes{v1.7}{2005/2/22}{Changed how labels
% are made for use in contents and notes sections.  Control sequences
% in titles (such as for italics and so on) no longer require special
% commands, although those commands are retained for backward
% compatibility.} \changes{v2.01}{2005/2/22}{Rewrote all of the title
% macros in order to make them easier to maintain.  Added environment
% for flushing runover lines in verse to the right margin.  Added
% information on how to customize titles.}
% \changes{v2.02}{2005/2/28}{Corrected an error in setting poems
% without formal titles} \changes{v2.12}{2005/3/19}{Notes sections can
% take ranges of lines.} \changes{v2.13}{2005/3/20}{tsentry no longer
% needs an unskip if the entry starts with a comma.}
% \changes{v2.2}{2005/3/23}{The user can create new kinds of endnote.}
% \changes{v2.21}{2005/3/24}{Cross references can refer to line
% numbers} \changes{v2.22}{2005/3/25}{Removed dependence upon
% marn.sty.  Enabled multiple levels of block-formatted footnotes}
% \changes{v2.24}{2005/4/02}{Macros to aid typesetting parallel
% texts.} \changes{v2.25}{2005/5/06}{Bug fix in running headers to
% notes sections} \changes{v2.26}{2005/5/09}{Fix no unskip in vertical
% mode error} \changes{v2.27}{2005/5/09}{Changed name of linelabel
% command, to prepare for making poemscol and lineno compatible}
% \changes{v2.30}{2005/5/16}{Prose sections such as Authors'
% Introductions are line numbered by lineno.sty.  Textnotes,
% emendations, and explanatory notes for prose sections may be either
% footnotes or endnotes, and the latter are sent to the same place as
% their verse equivalents.  Good solution to ``missing number''
% problem on first runs.} \changes{v2.301}{2005/5/31}{Corrected small
% bug in table of contents macros, rewrote manual for clarity}
% \changes{v2.31}{2005/6/3}{Ranges of line for notes sections (in
% verse or prose sections) need no longer involve counting of lines.
% Lemmas may be nested, may overlap, and may cross stanza boundaries.}
% \changes{v2.32}{2005/6/29}{Simpler configuration of marginal line
% numbers, revision of manual} \changes{v2.33}{2005/7/03}{Fixed a bug
% in index generation, compensated for odd interaction between
% fancyhdr and reversemarginpar} \changes{v2.34}{2005/7/17}{Easier
% modification of format of tables of contents}
% \changes{v2.35}{2005/7/18}{poemscol can break the lines for long
% titles in the table of contents on its own without making a mess of
% it.  poemscol can break the lines for long titles on its own without
% making a mess of it.  Modified macros for volume titles to use the
% facilities for poem titles.} \changes{v2.36}{2005/8/16}{Versions of
% note macros for typescript variants and accidentals for annotated
% prose.  Macros for marking up and annotating prose by paragraph and
% sentence number.  Macros for cross references by sentence and
% paragraph number.} \changes{v2.37}{2005/8/30}{In cases of broken
% lines or runover lines you can choose whether to set the line number
% with the beginning of the line or with the end of the line.  Also,
% you can choose to have it set the number always at the beginning if
% the line number is at the left, and always at the end if it is at
% the right.} \changes{v2.38}{2005/9/04}{Checked cases where macros
% might introduce extra white space.  Made sure that line numbers for
% verse, prose, and paragraph-sentence prose line up.  Fixed
% overlapping of paragraph and line numbers for paragraph-sentence
% mode.  Up to four sentences can appear on a line in paragraph-line
% mode without having their marginal line numbers overlap.  Fixed a
% bug that added an extra colon in footnotes in paragraph-sentence
% format.  Changed ``gutter'' and ``outer'' line numbering in verse
% and paragraph-sentence prose to use Peter Wilson's sidepars rather
% than marginpars.} \changes{v2.39}{2005/9/07}{Left a DeclareOption
% hanging in version 2.38.  Fixed that.}
% \changes{v2.391}{2005/10/06}{Corrections to user's guide.}
% \changes{v2.392}{2005/10/13}{More corrections to user's guide.}
% \changes{v2.394}{2005/12/10}{More corrections to user's guide.}
% \changes{v2.4}{2006/02/25}{Commands to make new kinds of endnotes
% also make notes for line numbered prose sections and for prose
% sections with paragraph and sentence numbering.}
% \changes{v2.41}{2006/03/02}{Fixed a bug in line numbers of new kinds
% of prose note.} \changes{v2.42}{2006/03/04}{Fixed error in handling
% conditionals when setting prose numbered by sentence and paragraph.}
% \changes{v2.43}{2006/03/06}{Notes in prose numbered by sentence and
% paragraph number can refer only to the sentence number even when
% paragraphs are also being counted.}
% \changes{v2.44}{2006/03/07}{Fixed a bug which caused new kinds of
% endnotes in prose sections to add unwanted white space to the
% output.} \changes{v2.45}{2006/04/08}{Use of sidepar in v2.38
% introduced a problem, causing line numbers sometimes to drift up a
% point.  Changed commands for putting verse line numbers, but not
% paragraph and sentence numbers, to use marginpars again.}
% \changes{v2.46}{2006/04/10}{Changed definition of sidepar so that it
% uses struts to find the depth of the line, restored line number
% commands to use sidepar and its variants.}
% \changes{v2.47}{2006/04/12}{Changed name of sidepar to pcmclsidepar
% to avoid conflicts with the sidepar command from the memoir class,
% since they are no longer exactly the same.  Added commands to put a
% marginal reference to the notes section in the margins of poem
% titles.} \changes{v2.48}{2006/04/13}{Added more general form of the
% commands to put a marginal reference to the notes section in the
% margins of poem titles.} \changes{v2.49}{2006/04/25}{Changed
% marginal reference command to use marginpar rather than pmclsidepar
% (alas, it still adds vertical white space after the title with
% pmclsidepar).  Added marginal reference commands for sequences and
% sequence sections.  Fixed errors in manual.  }
% \changes{v2.50}{2006/05/13}{Added marginal reference commands for
% poem sections, poem subtitles, epigraphs, and dedications.  These
% commands no longer require an additional label command to set the
% label to which they refer.  Added commands to make marginal
% references to the pages in notes sections where comments on
% individual lines may be found, for textual notes, explanatory notes,
% emendations, and user-defined notes.  Added commands for changing
% the names of the Table of Contents, the various endnotes sections,
% and the running headers for the contents and endnotes sections, in
% order to facilitate editions of works in languages other than
% English.} \changes{v2.51}{2006/05/22}{Changed prose section
% environment so that it does not always assume that the modulo for
% line numbers in prose is the same as the number in verse.  Corrected
% a problem with alignment of line numbers in the prosesectionnoreset
% environment.  Added commands to support generation of multiple
% indices using the splitindex package.  Fixed a variety of small
% bugs, including a newly introduced one in the correction to textual
% notes for quoted verse.} \changes{v2.52}{2007/07/10}{Changed
% dedication to poemdedication to avoid a conflict with a command in
% koma-script.  dedication command is provided using providecommand
% for compatibility} \changes{v2.53}{2007/07/14}{Changed the name of
% an internal token register to avoid a conflict with babel}
% \changes{v2.531}{2011/08/16}{Fixed a typo in example code.}
% \GetFileInfo{poemscol.sty} \changes{v2.6}{2012/08/31}{Added an
% optional argument to the poem environment to center the text block
% of the poem on the longest line.  Added a switch to enable titles to
% be centered on the page.} \changes{v2.61}{2013/02/06}{Added a
% boolean to switch notes to one note per paragraph form.  Boolean to
% switch off poem titles in notes. Two column formatting for notes sections.}
% \changes{v2.62}{2013/05/22}{You can adjust the margins of 
% epigraphs, dedications, and similar things. Made the appearance of 
% entries in the table of contents for endnotes sections 
% customizable. Fixed a bug which caused user-defined  
% endnotes to put extra white space in prose passages.}
% \changes{v2.63}{2013/06/01}{Revisions to manual.}
% \changes{v2.64}{2013/08/02}{Bug fix: remove extra white space in 
% custom endnotes for prose sections when notes are in single paragraph mode. Fix 
% appearance of page number in endnote sections.}
% \changes{v2.65}{2013/12/20}{Modifications to parallel text macros, 
% corrections to manual.}
% \changes{v2.66}{2015/5/14}{Change to page styles}
% \changes{v2.7}{2015/6/18}{Improvements to parallel texts.}
% \changes{v2.72}{2015/6/25}{Improvements to parallel texts. Generic note types for parallel texts.
% corrections to manual.}
%
% \DoNotIndex{\newcommand,\newenvironment,\def,\edef,\else,\renewcommand,\makeatletter,\makeatother}
% \DoNotIndex{\abovedisplayskip,\belowdisplayskip,\abovedisplayshortskip,\belowdisplayshortskip}
% \DoNotIndex{\newcommand, \section, \subsection, \subsubsection,
% \fileversion,} \DoNotIndex{\filedate, \docdate, \leavevmode,
% \vspace, \small, \normalsize, \fontsize,} \DoNotIndex{\newcommand,
% \section, \subsection, \subsubsection, \fileversion}
% \DoNotIndex{\filedate, \docdate, \leavevmode, \vspace, \small,
% \normalsize, \fontsize} \DoNotIndex{\selectfont, \abovedisplayskip,
% \belowdisplayskip, \setlength, \item} \DoNotIndex{\global, \input,
% \def, \edef, \texttt, \textsf, \setcounter}
% \DoNotIndex{\pagenumbering, \oddsidemargin, \evensidemargin,
% \newdimen, \renewcommand} \DoNotIndex{\makeatletter, \let, \@listi,
% \@listl, \z@, \@xpt, \@xipt, \p@ , \@plus2} \DoNotIndex{\@minus5,
% \@minus3, \@plus3, \makeatother, \@minus4, \relax, \pagestyle}
% \DoNotIndex{\fancyhead, \fancyfoot, \thepage, \@ixp}
% \DoNotIndex{\begin \botmark, \botmark, \catcode, \croppadbot,
% \croppadtop} \DoNotIndex{\croppagewidth, \endverse, \equal,
% \fancyfoot, \fancyhead, \fancyhf} \DoNotIndex{\fancypagestyle,
% \firstmark, \footrulewidth, \hbox, \hbox, \headrulewidth}
% \DoNotIndex{\hfil, \hspace, \ifthenelse, \immediate, \itemindent,
% \listparindent} \DoNotIndex{\makeatletter, \nolinebreak, \openout,
% \raggedbottom, \setcounter, \textrm} \DoNotIndex{\value, \vspace,
% \write} \DoNotIndex{\afterpoemtitleskip, \addtocounter, \advance,
% \begin, \bf, \bigskip} \DoNotIndex{\bigskipamount, \fi, \@fulltitle,
% \label, \large, \leftheader} \DoNotIndex{\leftheadervalue,
% \leftmargin, \leftmargini} \DoNotIndex{\medskip, \newcounter,
% \nobreak, \par, \textbf}
% \DoNotIndex{\closeout,\em,\emph,\expandafter,\Huge,\it,\jobname,\kern,\newif,
% \newwrite,\p@,\pageref,\renewenvironment,\rm,\scriptsize,\string,
% \textsc, \unskip} 
% \setcounter{IndexColumns}{2} 
% \MakeShortVerb{|}
% \def\fileversion{2.72} 
% \def\filedate{29 June 2015} 
% \def\docdate{29 June 2015}
%  \title{Typesetting Poetry Collections with
% \textsf{poemscol}\thanks{This document corresponds to
% \textsf{poemscol}~\fileversion, dated \filedate.}} 
% \author{John Burt
% \\ \texttt{burt@brandeis.edu}} 
% \date{29 June 2015} 
% \maketitle
% \begin{abstract} \textsf{poemscol} provides macros for \LaTeX\ for
% setting collections of poetry.  It is especially suited for setting
% collections of poetry in which several volumes are combined, such as
% in a critical edition of a poet's Collected Poems.  It provides the
% structures required to produce a critical edition of the kind
% specified by the Modern Language Association's Committee on
% Scholarly Editions, and it automatically marks every occasion where
% a stanza break falls on a page break.  \end{abstract}
% \tableofcontents \hfuzz=50pt \section{Introduction}
% 
% \textsf{poemscol} provides the structures necessary for editing a
% critical edition of a volume of poems or of a collection of a poet's
% works.  \textsf{poemscol} numbers the lines, and produces separate,
% formatted endnote sections for emendations, textual collations, and
% explanatory notes (or, optionally, multi-layered footnotes), tying
% each note to the number of the line upon which it is a comment.  It
% also automatically generates a table of contents, an index of titles
% and first lines, and divider pages for the sections of the volumes.
% It produces running headers of the form ``Emendations to pp.~xx-yy''
% for the note sections.  And it marks occasions when a stanza break
% falls at a page break.
% 
% Producing line numbers for verse is something for which \LaTeX\
% would seem to be especially suited.  A line of verse, after all, is
% not a carriage return but a logical unit, an element of
% versification more than of typesetting, perhaps running over several
% physical lines, or perhaps split into half-lines as speakers or
% subjects change.  With \textsf{poemscol} you mark lines, stanzas,
% and entire poems up as logical units, and \LaTeX\ does the
% formatting and counting.  Once you have marked out the logical units
% of the poem, \textsf{poemscol} will automatically mark every textual
% note, emendation, or explanatory note with the range of line 
% numbers to which it applies. 
%
% In poetry which does not have a regular stanzaic form, it is useful
% to be able to mark automatically occasions where there is a stanza
% break at the bottom of a page which the reader might not notice. 
% Doing this by hand is not only tedious and easy to get wrong, but
% also a process you will have to start over again if anything about
% your volume changes, if you add a poem, say, or even decide to break
% a title across two lines.  \textsf{poemscol} takes care of this
% process, so that the editor need never worry about it, automatically
% marking cases where the page break coincides with a stanza break
% with a symbol. 
% 
% Editions of Collected Poetry might also require special structures
% to reflect the fact that they are made up of the contents of several
% volumes of poetry.  In particular, such editions require special
% structures for setting up specially formatted divider pages between
% volumes. They also require tables of contents and other front matter. 
% \textsf{poemscol} provides these structures.
% 
% The best features of \textsf{poemscol} are of course simply that it 
% is \TeX: it uses \TeX's automatic kerning and setting of ligatures, 
% its algorithm for justifying lines (in prose sections), and \LaTeX's 
% way of setting verse.
% 
% Using \LaTeX\ to typeset critical editions offers more advantages
% than simply the ability to automate tedious and easy to fumble
% tasks.  Modern \TeX\ compilers have the ability to produce
% output in Adobe pdf format.  Adobe pdf output can be used as camera
% ready copy, saving your publisher time and expense, and perhaps
% making a marginally economic critical edition a bit easier to bring
% to press.  Furthermore, since typesetting the edition yourself in
% \LaTeX\ obviates the publisher's own typesetting of your text, it
% removes another possible source of new errors.  (You should expect
% to work with your publisher on the final design of the book, but
% \LaTeX\ is a flexible language which will enable you to reproduce
% most book designs.) 
%
% Editions made in the formats of proprietary software such as
% QuarkXPress\copyright\ or InDesign\copyright\ will become not only obsolete
% but unreadable if those programs pass from use.  Since your \LaTeX\
% sources for your edition are in ASCII, or perhaps Unicode, they provide a permanent
% record of your local intentions at every point in your edition,
% whether or not those who wish to consult your files have access to a
% \LaTeX\ compiler, or indeed whether or not they can read \LaTeX\
% code.
%
% Finally, although \LaTeX\ is a typesetting language, not a content
% markup language, the ability to create new commands which the
% language offers comes very close to enabling one to realize the
% ideal of completely separating content markup from formatting.  One
% advantage of this kind of markup is that even if the appearance of
% the poem on the page may be ambiguous, the editor's intentions about
% the logical structure of the poem will be preserved in the \LaTeX\
% source. Should you wish later to produce an electronic edition of
% your work, either using XML or the SGML markup approved by the Text
% Encoding Initiative, transforming your texts from \LaTeX\ to XML
% would largely (although not entirely) be a matter of performing a
% series of global search-and-replaces, and could conceivably be done
% with a perl script.
% 
% \textsf{poemscol} is also suited for verse drama, and the package
% \textsf{dramatist} has been modified to work with \textsf{poemscol}.
% \textsf{poemscol} can provide line numbers and notes for prose
% sections such as the author's introduction or prose poetry, and it
% can also handle inset prose passages in the midst of verse.
% \textsf{poemscol} is not, however, designed for typesetting large
% scale prose works, although very possibly it could be tweaked to 
% work.  For critical editions of prose works, several
% other packages are available, including the \texttt{EDMAC} format, a
% \TeX\ format analogous to but distinct from \LaTeX, by John
% Lavagnino and Dominik Wujastyk, or \textsf{ledmac}, a port of
% \texttt{EDMAC} into \LaTeX\  originally by Peter Wilson and
% now maintained  as \textsf{eledmac} by Ma\"{\i}eul Rouquette, or \textsf{ednotes}, a
% completely independent set of macros for critical editions which
% builds on \textsf{manypar} and \textsf{lineno}, by Uwe L\"{u}ck.
% 
% \section{Dependencies and compatibility with other packages}
%
%\subsection{~General}
% \textsf{poemscol} depends upon several other packages, which you
% should be sure you have in your preamble and search path: 
% \textsf{fancyhdr} for managing the running headers, \textsf{makeidx}
% and \textsf{multicol} for managing the index, \textsf{geometry}, and
% \textsf{ifthen} and \textsf{keyval} to simplify page geometry. 
%
%  If
% you are using the |\JHpoemtitle| alternate form of the |\poemtitle|
% command, you need to add \textsf{mparhack} to your package list, to
% make sure that the marginal references |\JHpoemtitle| adds appear on
% the correct side of the page.  You should make sure that
% \textsf{poemscol} is the very last package you load in your
% preamble, or at least is loaded after those packages upon which it
% depends.
% 
% If you are planning to generate multiple indices, substitute 
% \textsf{splitindex} for \textsf{makeidx} in your list of packages.
% 
%
% \subsection{~Prose sections}
% If you plan to number lines in prose contexts, and to make textual
% notes, emendations, or explanatory notes in prose contexts, you will
% need to add the \textsf{lineno} package to your preamble.  I use the
% ``right'' and ``modulo'' options with this package, but you can set
% the options however you wish to make how you number the lines of
% prose sections consistent with how you number the lines of verse
% sections. 
%
% \subsection{~Multi-layer footnotes}  If you plan to use
% paragraph-formatted footnotes rather than endnotes, you should load
% \textsf{manyfoot}.  You should load it with the ``ruled,'' and
% ``para'' options.  (\textsf{poemscol's} footnote commands are just a
% wrapper around \textsf{manyfoot}.)  \textsf{manyfoot} inherited some
% of the limitations of \LaTeX\ in dealing with long inserts at the
% end of the page, so you may have to do some fiddling.  As fixes to
% these problems emerge, I will incorporate them into
% \textsf{poemscol}.  \textsf{manyfoot} does not allow multiple
% paragraph footnotes when typesetting in ``para'' mode.  This is
% probably a feature, not a bug, since one would want such notes to be
% set in several paragraphs, rather than running them together into
% one, as \textsf{manyfoot} would naturally want to do in para mode.
% For that reason, I have defined the two traditional classes of note
% in which multiple paragraph notes are likely to appear, ``sources,''
% and ``explanatory notes,'' to be typeset in ``plain'' mode, which
% opens a new paragraph with each note.  If your textual notes or
% emendations sections also have multiple paragraph notes, you should
% re-define them in order to produce footnotes in plain rather than in
% para mode, if you wish to produce footnotes rather than endnotes.
%
% \textsf{manyfoot} sometimes gets into conflicts with other packages
% over the number of |\counts| it uses, causing \LaTeX\ to complain
% that there is no room for a |\newcount|. If this happens, 
% issue |\usepackage{etex}| and |\reserveinserts|.
%
% \subsection{~Verse drama and \textsf{dramatist}}
% Massimiliano Dominici has made his \textsf{dramatist} package
% compatible with \textsf{poemscol}, for which I am very grateful. 
% Versions 1.2a or later of \textsf{dramatist} are compatible with 
% versions 2.3 or later of \textsf{poemscol}. 
%
%
% \subsection{~Parallel Text Editions} If you are using the |\keeptitlestraight|
% hack described below  at \ref{parallelhack}, you should add \textsf{etoolbox}
% to your package list. If you are using the six pre-defined endnote sections
% for parallel text editions (|\rectotexnote|, |\versotextnote|, |\rectoemendation|, |\versoemendation|,
% |\rectoexplanatory|, |\versoexplanatory| and their prose equivalents |\rectoprosetextnote|
% and so on) you do not need to use this hack. You only need to use this hack if 
% you are defining your own endnote sections using |\definenewnotetype|. 
% 
% \subsection{~Crop marks}
% \textsf{poemscol} no longer provides cropmarks, but is compatible with 
% the style packages that do. I use \textsf{crop}, with the 
% letter,pdftex,cam, and center options.
%
%
% \subsection{~Incompatibility with \textsf{memoir} and \textsf{verse}}
% Unfortunately, \textsf{poemscol} is not compatible with the
% \textsf{memoir} class and the \textsf{verse} package, because they
% share some command names. I am preparing a workaround for this 
% problem. 
%
% \subsection{~Conflict with \textsf{BibLaTeX} }
% \BibTeX\ works with \textsf{poemscol} with no problem. But a bug I haven't
% yet tracked down  (probably involving |\makeatletter|)
% complicates the use of \textsf{BibLaTeX}. To use \textsf{BibLaTeX}
% you will need to open up the .bbl file. You will find near the top
% the following lines:\\
%  |\makeatletter|\\
%  |\@ifundefined{ver@biblatex.sty}|\\
%  | {\@latex@error|\\
%     |{Missing 'biblatex' package}|\\
%    | {The bibliography requires the 'biblatex' package.}|\\
%     | \aftergroup\endinput}|\\
%   |{}|\\
%  Just comment out these lines and you will be able to use the .bbl file as usual.
% 
%
% \subsection{~Compatibility with earlier versions of
% \textsf{poemscol}} Making \textsf{poemscol} compatible with
% \textsf{lineno} required me also to change the commands for turning
% verse line numbering on and off, which means that version 2.3 of
% \textsf{poemscol} is incompatible with prior versions.  The old
% command |\makelinenumbers| has been replaced with
% |\makeverselinenumbers| to distinguish it from a command from the
% \textsf{lineno} world.  (|\makelinenumbers| will still work, but I
% have deprecated it.)  To turn on verse line numbering, issue
% |\makeverselinenumbers|.  Because the counter ``linenumber'' has
% been changed to ``verselinenumber,'' you must use
% |\global\verselinenumbersfalse| to turn off line numbering, rather
% than |\global\linenumbersfalse| as before.  
% 
% Up to version 2.44 \textsf{poemscol} borrowed the code for
% |\sidepar| non-floating marginal paragraphs from the memoir class, 
% and also made |\leftsidepar| and |\rightsidepar| commands on their 
% model. Version 2.46 modified Peter Wilson's original code for 
% |\sidepar|, so therefore I have changed the names of |\sidepar|, 
% |\leftsidepar|, and |\rightsidepar| to |\pmclsidepar|, 
% |\pmclleftsidepar| and |\pmclrightsidepar| respectively. If you 
% used any of these commands in your files, you will need to change 
% their names.
% 
% Up to version 2.5 the |\definenewnotetype| command took four 
% arguments, using the third argument for both the running header for 
% the notes section and for the entry in the table of contents. In 
% order to make it easier to handle titles in languages other than 
% English I have found it necessary to separate these two uses, so 
% |\definenewnotetype| now takes five arguments. You will need to 
% retrofit files that use the old version of this command. You can 
% see the details at section \ref{definingnewnotes}, page 
% \pageref{definingnewnotes}.
%
% \section{Marking up individual poems}
% First, a word about the command names. The command names may seem 
% ugly and long. And there are separate commands for many tasks that 
% seem closely related, such as a command to mark the title of a 
% section of a poetic sequence, and a separate command to mark a 
% subsection. But the names do describe pretty much what
% each macro does, and they do specify exactly what the object they 
% mark is supposed to be.  The markup is designed to look like content
% markup, marking objects as a poem title, as a stanza, as a line, and so on. 
% \textsf{poemscol} gives all of these content terms typographical
% meaning.  
% 
% There are also many commands for special purposes whose necessity
% may not seem clear until the editor finds him or herself in the jam
% the command was designed for.  For instance, \textsf{poemscol}
% normally encourages a page break before the title of a poem, or
% before the title of a section of a poetic sequence, to discourage
% page breaks between the title and the poem. But for the first
% section of a poetic sequence, or the first section of a poem in
% sections, one does not want to encourage \LaTeX\ to break the page 
% before the title, since that would leave the title of the sequence
% or the title of the larger poem as a widow on the previous page.
% |\sequencefirstsectiontitle| and its siblings are designed for this 
% situation. Although all commands like |\sequencefirstsectiontitle| 
% discourage page breaks before the title (and all of the commands
% discourage page breaks after the title), they will still sometimes 
% happen, particularly if you have a multi-line sequence title
% followed by a multi-line section title. For these cases, page
% breaks will have to be explicitly issued.
% 
% \subsection{~The title of the poem}
% 
% \DescribeMacro{\poemtitle}|\poemtitle{}|, as its name implies, sets
% the title of the poem. 
%The macro takes one
% argument --- the title, of course.  The macro typesets the title in
% the body of the volume (testing first to see whether there is enough
% space at the bottom of the page to get the title and a couple of
% lines of the poem in), typesets the title in the table of contents
% (with the page number), typesets the title in the textual notes
% (adding the page number, and checking to see whether there is room
% enough on the page), and prepares similar entries in the lists of
% emendations and in the explanatory notes (if you need them).
% \textsf{poemscol} gives default values for
% such things as the font size, the separation between the top of the
% title and the bottom of the previous poem, the separation between
% the bottom of the title and the first line, and so on.  It also sets
% penalties in order to encourage page breaks just before a title, and
% to discourage page breaks between a title and a poem. 
% 
% You can
% change all these values in your header file by changing the value of
% the parameters listed below in the implementation section using
% either |\setlength| or |\renewcommand| (depending upon what you are
% changing).  |\poemtitlefont| globally sets the font size (and
% leading) for all poem titles in the main text. 
% |\contentspoemtitlefont| globally sets the font size (and leading)
% for poem titles in the table of contents.  |\afterpoemtitleskip|
% sets the vertical separation between a poem and its title. 
% |\afterpoemskip| sets the vertical separation between the end of a
% poem and the title of the next poem.  |\poemtitlepenalty| encourages
% but does not require a page break just before a poem title.  
% If you don't want the entry in the notes section for this poem to include
% the page number of the poem, issue |\putpagenumberinnotesfalse|. A
% complete list of all of the special commands for setting font sizes,
% skips, and penalties, is in section \ref{changingfontsandleading} below.
% 

% Multiple line titles (if it matters how they are broken up on page), 
% titles of subsections of poems, titles of
% sequences of poems, and titles of the elements of a sequence of
% poems are all special cases, with special commands, which will be
% dealt with below at sections \ref{multilinetitles} and \ref{poeticsequences}. 
%You can also create new varieties of title. For instructions 
% on how to do this, see section \ref{customizing titles} below, page \pageref{customizing titles}.
% 
% \DescribeMacro{\poemsubtitle} Subtitles of poems should be issued as 
% the arguments to the |\poemsubtitle{}| macro.
% 
% \DescribeMacro{\titleindent} For indented parts of titles. In the 
% macros for setting the sections and subsections of poetic sequences, 
% \textsf{poemscol} uses |\titleindent| internally to set up a 
% hierarchy of indentations. (A section of a sequence is indented one 
% |\titleindent|. A subsection is indented two. And so on.) 
% |\titleindent| is also used internally in the commands for multiple 
% line titles. Instructions about changing the value of |\titleindent|
% can be found in section \ref{changingfontsandleading}.
% 
% \subsection{~The body of the poem}
% 
% \DescribeEnv{poem} The body of every poem should be placed in a
% |poem| environment.  Putting the body of the poem between
% |\begin{poem}| and |\end{poem}| resets the line counter to 1, and
% puts the poem in a |verse| environment (to handle run over lines
% automatically).  \textsf{poemscol} slightly modifies the |verse|
% environment from the standard \LaTeX\ definition, increasing the
% indentation used for run over lines, in order to make the difference
% between the indented run over portion of a long line, on one hand,
% and an explicitly indented second line, more obvious in the output.
%
% The |poem| environment takes an optional argument. If your poem has 
% short lines, and you wish to center it on the page, rather than set 
% it against the left margin of the normal |poem| environment, you can 
% give the width of the line you want to use to center the poem as the 
% argument. Alternatively, the length |\versewidth| can be used as the 
% argument. You can set the length of |\versewidth| to the proper 
% amount using |\settowidth| as follows:
% \DescribeMacro{\versewidth}
% |\settowidth{\versewidth}{Text of line to use for centering}|. To use |\versewidth|
% to center the poem on a sample line, enter the |poem| environment this way:
% |\begin{poem}[\versewidth]|.
% 
% \DescribeMacro{\centertitles}
% If you center the poem on a sample 
% line, you probably also want to center the title of the poem too, 
% which you can do by issuing |\centertitles|.
% 
% \textsf{poemscol} turns off automatic hyphenation in poetry
% environments.  The idea here is that every hyphen in the printed
% poem is authorial, obviating the need for you to compile a
% hyphenated-lines list to distinguish between authorial hyphens and
% hyphens you added for lineation purposes.  Line-ending hyphens
% should not be a feature of verse anyway, I think.  (You may wish to
% change this for your own edition, in order to improve its look; but
% if you do so you must keep track of added hyphens yourself.  This
% list will be easy to compile, however, because only authorial
% hyphens will appear in your source code.  Automatically added
% hyphens will appear only in the output.  You might even modify the
% output routine so that automatically added hyphens have a different
% look.  That would be non-trivial, but Donald Knuth has an exercise
% about doing just that in \emph{The \TeX book}.)  \textsf{poemscol}
% turns automatic hyphenation back on in prose contexts, so if you
% wish to keep a hyphenation list for such things as authorial
% prefaces and so on, you must do so yourself manually.
% (Alternatively, you can turn automatic hyphenation off in those
% contexts as well, by setting the |\language| to 255.  For an example
% of how to do this, look at the definition of the |\poem| environment
% in the implementation section below.  If you do turn automatic
% hyphenation off, it would be wise to restrict the change to some
% particular environment, rather than changing the |\language|
% globally.  You might wish, for instance, to turn automatic
% hyphenation off in the |\prosesection| environment.)
% 
% \DescribeEnv{stanza} Every stanza should be placed in its own stanza
% environment.  Every poem should have at least one stanza.  Marking
% the beginning and end of every stanza (with |\begin{stanza}| and
% |\end{stanza}|) provides \textsf{poemscol} with a way of detecting
% cases in which a page boundary falls on a stanza break, since in
% those cases a page turn happens when one is \emph{inside} a poem
% environment but \emph{not} inside a stanza environment.  Further,
% marking the beginning and end of every stanza makes the logical
% structure of the poem (and the editor's intentions about it) clear
% to readers of your source code.  
%
% \textsf{poemscol} adds a little bit
% more white space between stanzas than the standard \LaTeX\ |verse|
% environment does.  (I found that the standard stanza breaks did not
% leap out on the page as stanza breaks.)
% 
% \DescribeMacro{\stanzaatbottom}
% \DescribeMacro{\nostanzaatbottom}
% If you wish to change the symbol used to mark cases in which a 
% break falls on a page turn, issue the command |\stanzaatbottom{}|, 
% using as the argument whatever you wish to use to mark such cases. 
% The default is *, which is suitable for a published volume. If you 
% are preparing a typescript for submission, you may wish to set 
% |\stanzaatbottom| to ``stanza break.'' 
% 
% If you wish also to mark all cases where there is NOT a stanza break 
% at the bottom of the page, issue the command |\nostanzaatbottom{}|, 
% using as the argument whatever you wish to mark such cases. The 
% default is |\relax| (which is to say, ``don't do anything'').
% 
% \DescribeMacro{\verseline} |\verseline| should mark the end of every
% line, except the last line of every stanza (which should be marked
% with |\end{stanza}|).  |\verseline| marks the end of every line as a
% prosodic unit (since a line of verse is not simply a carriage
% return), and advances the line counter.  There are commands below
% for changing the line counter's horizontal placement (how close it
% is to the left or right margin of the line) and appearance. (See
% section \ref{marginalgeometry} for these.)
% 
% 
% \DescribeMacro{\setverselinemodulo} |\setverselinemodulo| sets how
% often a marginal line number appears.  To print a line number every
% five lines, issue |\setverselinemodulo{5}|.  The default value is
% 10.
% 
% 
% \DescribeMacro{\makeverselinenumbers}  
% Marginal verse line numbering is on by default.  To turn line
% numbering off, issue |\global\verselinenumbersfalse|.  To turn it
% back on, issue |\makeverselinenumbers|.
%
% \DescribeMacro{\verselinenumberstoright}
% \DescribeMacro{\verselinenumberstoleft}
% \DescribeMacro{\verselinenumbersswitch} 
% \DescribeMacro{\verselinenumbersgutter}
% \DescribeMacro{\verselinenumbersouter}
% To put all of the line numbers to the left, issue
% |\verselinenumberstoleft| in your preamble.
% |\verselinenumberstoright| puts the line numbers in the right
% margin.  (These two commands use non-floating marginal note macros
% derived from the \textsf{memoir} class, |\pmclleftsidepar| and
% |\pmclrightsidepar|, respectively.)  If you wish them to appear in
% the outer margins, issue |\verselinenumbersouter| in your preamble.
% If you want the line numbers to appear in the inner margins, issue
% |\verselinenumbersgutter| in your preamble.  (These two commands use
% |\pmclsidepar|, derived from |\sidepar| from the \textsf{memoir}
% class, but modified by Dan Leucking.)  You can adjust the distance
% between the marginal line number and the text with
% |\setlength{\marginparsep}{}|.The default value for |\marginparsep| 
% is 18pt, and the default value for the width of the box in which 
% the line numbers are set, |\marginparwidth|, is 10 pt.
% 
% The line numbers should print on the baseline. You can adjust them 
% to move them up or down by changing the value of a length called
% |\pmclsideparvshift|. The default value is
% |\setlength{\pmclsideparvshift}{0ex}|. 
%
% \subsection{~Special line markup}
% \DescribeMacro{\linebend} \textsf{poemscol} automatically runs over
% long lines, indenting the run over portion on the next physical
% line.  If you are unhappy with where \textsf{poemscol} has run over
% a particular line, you can ``bend'' that line by issuing |\linebend|
% at the point where you wish it to run over.  The run over portion of
% the line will be indented just as if \textsf{poemscol} had ``bent''
% the line at your selected point.  This command only works if you
% have chosen to bend the line at some point earlier than
% \textsf{poemscol} would have chosen on its own.  If you really do
% want to extend a line further into the right margin, you can
% probably do so by using a combination of |\nobreak| and |\hbox{}|,
% or by turning all of the spaces in that line into unbreakable
% spaces, marked with |~| in your source.  But \textsf{poemscol} will
% complain if you do this, and rightly so, since the result is likely
% to be ugly.
% 
% You may also wish to use |\linebend| to reproduce how your author
% broke up long lines on the page (if you know that your author cared
% about such things and did not leave them up to the typesetter).
% 
% |\linebend| should only be used for managing run over lines, not for
% cases in which a line is to be broken into separate half-lines.  For
% cases in which a line is to be broken into half-lines, use the
% |\brokenline| macro.  The two macros do similar (but not identical)
% things.  But a ``linebend'' is a feature of typesetting, and a
% ``broken line'' is a feature of versification, and it seems best to
% distinguish them logically. (|\linebend|, like |\brokenline|, issues 
% a carriage return without incrementing the line number, but 
% |\linebend| adds indentation to the next line.)
% 
% \DescribeMacro{\runoverindent} To change the indentation for the run
% over portion of verse lines globally, issue the amount of
% indentation as the argument to |\runoverindent{}|.
% |\runoverindent{}| is currently set to 6 em, rather more than the
% standard amount in the definition of the \LaTeX\ |verse|
% environment. By default, the runover portion of long lines is 
% indented. You may wish to flush the runover portion to the right 
% margin instead. For instructions about how to do this, see section 
% \ref{rightflushverse}, page \pageref{rightflushverse}.
% 
% \DescribeMacro{\verseindent} The |\verseindent| command is to be
% used for formally indented lines. It should not be used to indent 
% the run over portion of long lines (which \textsf{poemscol} handles 
% automatically anyway).  |\verseindent| indents the line a bit less
% than the run over portion of long ``bent'' lines are indented, so 
% that the two cases can be visually distinguished. By default, the 
% length of |\verseindent| is 2 em. To change it, say to 3 em, issue 
% |\renewcommand{\verseindent}{\hspace{3em}}|
%  
% 
% \DescribeMacro{\brokenline}
% \DescribeMacro{\versephantom}|\brokenline| should be used where you
% wish to break the line without incrementing the line counter. 
% Authors often break a line into two half-lines, setting the
% beginning of the second half-line flush with the end of the first
% half-line.  (Sometimes these broken lines indicate a change of
% speaker.  Sometimes they just indicate a change of subject, usually
% an abrupt one, a change which calls attention to itself by
% interrupting a line rather than waiting for a line break or stanza
% break).  |\brokenline| is normally used with |\versephantom{}|,
% which adds white space exactly as long as its argument would have 
% been had it been set in type. |\versephantom{}| thus provides an easy 
% way of setting the beginning of the second half-line flush with the 
% end of the first, whatever the font size or special formatting of the 
% first line.
% 
% 
% The sestet of Yeats's sonnet ``Leda and the Swan,'' has such a broken 
% line: 
% \begin{verse}
% A shudder in the loins engenders there\\
% The broken wall, the burning roof and tower\\
% And Agamemnon dead.\\
% $\phantom{And Agamemnon dead.~}$ Being so caught up,\\
% So mastered by the brute blood of the air,\\
% Did she put on his knowledge with his power\\
% Before the indifferent beak could let her drop?\end{verse}
% 
% To set the broken line properly, issue:
% \begin{verse}
% 	And Agamemnon dead.|\brokenline|\\
% 	|\versephantom{And Agamemnon dead.}| Being so caught up,|\verseline|
% 	\end{verse}
%
% \DescribeMacro{\tweakbrokenline}
% By default, \textsf{poemscol} sets the line numbers level with the 
% ends of broken lines. If the line numbers are set to the right of 
% the text, this seems the obvious choice. Conceivably you might want 
% to set the line number level with the beginning of a broken line 
% when the line numbers are set to the left. To ensure that the line 
% numbers of broken lines are set level with the beginning of a 
% broken line when the line number is on the left, issue 
% |\tweakbrokenline| in your preamble. 
%
% You should issue |\tweakbrokenline| with caution, because although 
% it changes how |\brokenline| is handled, it does not change the 
% handling of long lines with runover. Runover lines will still set 
% the line number level with the end of the line. You can have 
% runover lines set their line number level with the beginning of the 
% line by issuing |\startverseline| at the beginning of the line.
%
% \DescribeMacro{\brokenlineatbeginning} Conceivably you might want 
% |\brokenline| to set the line number level with the beginning of 
% the line in every case. To do this, issue |\brokenlineatbeginning| 
% in your preamble. If you do this, you should be prepared to mark 
% the beginning of runover lines with |\startverseline|.
%
% \DescribeMacro{\startverseline}
% If you begin a verse line with |\startverseline| the line number 
% will be set level with the beginning of the line rather than with 
% the end. This is useful, for instance, if you want to ensure that 
% the line number of a line with runover is set level with the 
% beginning of the line. You can, if you don't want to keep track of 
% where runover lines are, start every line with |\startverseline|. 
%
% \DescribeMacro{\tweakstartverseline} 
% Rather than setting the line number of every runover line level with
% the beginning of the line, it may make more sense for you to set the
% line number level with the beginning in cases where the line number
% is to the left, and level with the end in cases where the line
% number is to the right.  You might want to do this, for instance, if
% you have decided to set all the verse lines at the outer margin (by
% issuing |\verselinenumbersouter| in your preamble), or have decided to
% set all the verse lines at the inner margin (by issuing
% |\verselinenumbersgutter| in your preamble).  To ensure that
% |\startverseline| does this, issue |\tweakstartverseline| in your
% preamble.
% 	
% \DescribeMacro{\stanzalinestraddle}
% Some poets (such as Robert Penn Warren) occasionally introduce a 
% stanza break in the middle of a broken line, considering the line to 
% be a single metrical unit despite the fact that it straddles a stanza 
% break. To record these cases, mark the end of the first half-line 
% with |\end{stanza}| as usual. But instead of opening the next stanza 
% with |\begin{stanza}| issue |\stanzalinestraddle| instead. This will 
% make sure that the line counter counts the straddling line as only one 
% line, despite the stanza break. |\stanzalinestraddle| is usually used 
% with |\versephantom{}|.
% 
% \DescribeEnv{rightflushverse}
% \DescribeMacro{\rightversebegin}\label{rightflushverse} By default,
% \textsf{poemscol} indents the runover portion of long lines by the
% amount |\runoverindent| (by default 6 em).  It is possible to set
% verse so that the runover portion is flushed to the right margin.
% To do this, you should put the |\rightflushverse| environment inside
% the |\poem| environment.  And you must mark the beginning of each
% verse line with |\rightversebegin|.  Christian Ebert has reminded me
% that the |\rightflushverse| environment must be issued \emph{inside}
% the poem environment.  Otherwise it simply flushes every line to the
% right.
%
% It is rather clumsy to need a special environment for verse with 
% right flushed runover. I plan to provide a command for switching 
% between a verse environment that indents runover lines and a verse 
% environment that flushes runover lines to the right. I have seen 
% examples of this kind of environment from Markus Kohm and Christian 
% Ebert, but so far I haven't been able to persuade them to play well with \textsf{poemscol}
% 
% \DescribeMacro{\poemlinelabel} \label{poemlinelabel}
% |\poemlinelabel{}| enables crossreferences by line number.  To make
% a line label, issue |\poemlinelabel{text of label}| right after the
% line to which you wish to refer, where ``text of label'' is some
% distinctive label you can use for a reference elsewhere.  (You
% should issue the command \emph{after} the |\verseline| or
% |\end{stanza}| command, to make sure that the line number will be
% correct.)  To produce the reference, just issue |\ref{text of label}| as usual.  
% (|\poemlinelabel| is just a crudely hacked
% version of |\label| from \LaTeX.) |\pageref{text of label}| will set
% the page number of the page on which the label appears.
% 
% \section{Making a table of contents}
% \subsection{~Setup}
% \DescribeMacro{\makepoemcontents} 
% I found the normal \LaTeX\ macros for making tables of contents
% unsuited to poetry, so I have provided my own.  To make a table of
% contents for your poems, issue |\makepoemcontents|.  It might seem
% as though you would want to issue this command in your preamble.  In
% fact it is better, for reasons explained below, to issue it
% immediately after issuing the |\putpoemcontents|, the command which
% inputs and typesets a table of contents from a prior run.
% Information for your table of contents will be written to an
% external file with the extension .ctn.  You may need to run your
% book through \LaTeX\ twice in order to generate a table of contents,
% the first time to generate the correct page numbers, and the second
% time to set the table of contents.  (Under some circumstances --- if
% you have an index, for instance --- you may need three passes rather
% than two.)  |\makepoemcontents| takes an optional argument, which
% sets the page number for the table of contents.  If your publisher
% wishes to add some front matter so that the contents will appear on
% page vii, you should issue |\makepoemcontents{7}|.

% All of the commands that generate lines in the table of contents are
% somewhat messy and somewhat rigid, but you can modify them if you
% wish.  Lines in the table of contents usually include an indent (how
% much of one depends upon what the entry is --- whether it is a
% volume title, a poem title, or something else), then the title
% itself.  If the entry is for a poem, for a section of a poem, or for
% a section of a poetic sequence, a slash follows the title, followed
% by the page number.  If you wish to give \emph{all} your entries a
% different look altogether, modify the |\c@ntentsinfo| macro below
% (see section \ref{poemtitlecode}, page \pageref{poemtitlecode}).  To
% change the appearance of a \emph{particular} kind of entry (for all
% sequences of poems, for instance), go to the definition of the title
% of that kind of entry (e.g. |\sequencetitle|) below, find the line
% where it calls the |\c@ntentsinfo| macro, and change the arguments
% to that macro to suit you.
%
%\DescribeMacro{\setcontentsleaders} By default the title and the page 
% are separated by a slash (actually by |~/~|). If you wish to change 
% this, say to put dot leaders in instead, issue what you wish to use 
% as the argument to |\setcontentsleaders|, as for instance 
% |\setcontentsleaders{\poemdotfill}|, which will flush the number to 
% the right margin, and insert leading dots up to the number.
% 
% \DescribeMacro{\putpoemcontents}\label{putpoemcontents} 
% |\putpoemcontents| will input the table of contents file if it
% exists, and complain on your log file if it does not.  Because the
% table of contents uses a special page style and special fonts
% |\putpoemcontents| restores the default fancy page style once it has
% input the contents, and restores |\normalfont| as well.  Be sure to
% reset the style of page numbering (from |\pagenumbering{roman}| to
% |\pagenumbering{arabic}|) after issuing |\putpoemcontents|, since
% when you read in the external file created by |\makepoemcontents|
% the page numbering will be set to roman.  The command
% |\resetpagestyle| (which is called by |\putpoemcontents|) will reset
% the fancy page style to that used in the main body of the volume,
% but it won't change the page numbering to arabic, in case there is
% some frontmatter between the contents and the main text, so you
% should be sure to do so yourself.  If you neglect to issue
% |\pagenumbering{arabic}| before you begin to get to your poems your
% page numbering will remain in roman numerals.  The notes sections
% use the page numbers from the poem to construct running headers of
% the form ``Notes to pp.~xx--yy.''  If those numbers are roman
% numerals rather than arabic ones, \textsf{poemscol} will not
% recognize them as numbers, and you will get the infamous ``Missing
% Number'' error that puzzles so many users of \LaTeX.
% 
% Because |\putpoemcontents| uses the table of contents file from a
% prior run, you should issue |\putpoemcontents| \emph{before} issuing
% |\makepoemcontents|.  That sounds counterintuitive.  But remember
% that |\putpoemcontents| will read in the contents file you made on a
% previous run, and |\makepoemcontents| will over-write that file and
% start a new one.  You should issue |\makepoemcontents| just
% \emph{after} issuing |\putpoemcontents|.
% 
% \subsection{~Contents entries for notes sections and the index}
%
% \begin{sloppypar}\DescribeMacro{\puttextnotes}
% \DescribeMacro{\putemendations}
% \DescribeMacro{\putexplanatory} 
% |\puttextnotes|, and its siblings |\putemendations| and
% |\putexplanatory|, linput the formatted notes
% sections, and put entries for them into your table of contents.  
% You will have to run \LaTeX\ again to set the
% table of contents properly.  The first run sets a label to find the
% page number to add to the table of contents, and the second run
% actually sets the table of contents.  (These commands are described
% below, in section \ref{putexternalfiles}.)  All of these commands
% also restore normal hyphenation.  \end{sloppypar}
%
% \DescribeMacro{\putpoemindex}
% |\putpoemindex| inputs the formatted index, and adds an entry 
% for it into the table of contents.
% Remember that if you are making an index of titles and first lines
% you may need to run your file through \LaTeX\ after running
% MakeIndex, to put the entry for the index into your table of
% contents.  
%
% \DescribeMacro{\putmultiplepoemindex}
%Use |\putmultiplepoemindex| instead if 
% you are making several indices. For instructions about 
% |\putmultiplepoemindex| see section \ref{multipleindex}, 
% p.~\pageref{multipleindex}.
%
% \DescribeMacro{\finish} Rather than issuing |\putpoemcontents| in
% the proper place, and |\puttextnotes| and its siblings in their
% proper places, you can issue the |\finish| command just after your
% last poem.  |\finish| turns automatic hyphenation back on, closes
% and reads back in the external files for the emendations, the
% textual collations, and the explanatory notes, sets the index of
% first lines and titles, puts the information about these sections in
% the table of contents, closes the external file for the contents,
% and reads it back in.  One disadvantage of this method is that
% you print the book with the contents at the end, so when you print
% it you must retrieve the contents pages and put them in the proper
% place.  I found the convenience worth the hassle, but you might not.
% This command is deprecated now. But I have retained it, since the 
% few users who gave me their opinions of the matter told me that 
% they preferred to use |\finish| than |\puttextnotes|.
% 
% 
% 
% \section{Making an index of titles and first lines}
% \subsection{~Setup}
% \DescribeMacro{\indexingontrue} To set up the Index section of your
% volume, issue |\indexingontrue| in your preamble.  This will create
% the external file for your index information and send that file the
% typesetting information for its title.  You must process this
% external file (the .idx file) with MakeIndex and run your book
% through \LaTeX\ again in order to set a properly organized and
% formatted index.  MakeIndex sorts the various multi-level ``index
% cards'' generated by all of the |\index| commands in your source
% files.  Once you have processed your .idx file with MakeIndex, and
% generated the sorted and formatted .ind file for your book, the
% |\putpoemindex| command will insert the typeset index where you
% issue it, and put an entry for your index in your table of contents.
% You will have to run your file through \LaTeX\ again to make the
% page number for that entry correct.
% 
% \textsf{poemscol} sets up a two-column index with a simple running
% header.  (Setting up a two-column index, however, requires that you
% include the \textsf{multicol} package in your |\usepackage{}| list.)
% To turn indexing off, issue |\global\indexingonfalse|.  
% 
% Enter the individual index entries in the poems as you normally
% would for use by MakeIndex (say, after each title and first line).  One of
% the nice features of MakeIndex is that every entry has two parts,
% one for specifying the sort key of the entry (or subentry), the other
% for specifying the font and typesetting information of the entry (or
% subentry).  This enables you to drop all of the initial uses of
% ``the'' in titles for sorting purposes, while still including the
% ``the'' in the title.  It also enables you to distinguish titles
% from first lines (by italicizing them), and to include typesetting
% information (such as italicization) in titles or first lines without
% messing up their sort order. 
% 
% \subsection{~Making Multiple Indices}\label{multipleindex}
% \textsf{poemscol} can work with \textsf{splitindex} to generate 
% more than one index. You might, for instance, wish a separate index 
% of titles and index of first lines, or an index of proper names. To 
% do this, call \textsf{splitindex} in your list of packages. Then, 
% in your preamble, declare the indices you are planning to use, e.g.
% |\newindex[Index of Titles]{idx}|
% |\newindex[Index of First Lines]{fir}| 
% |\newindex[Onomastic Index]{ono}|. The second argument is the 
% ``shortcut'' defined by \textsf{splitindex}, and will be used to 
% mark index entries in your text (see below), and as the extension 
% for the external index files used by MakeIndex.
% 
% Then mark up individual entries. Some entries (in the example 
% above, the ``Index of Titles'' examples) can be marked up with 
% |\index| just as if you were using only one index. Entries for 
% other indices should be marked using the |\sindex| command from 
% \textsf{splitindex}, as follows
% \begin{verbatim} 
% \sindex[fir]{Of Man's first disobedience and the fruit} 
% \end{verbatim}
% where the 
% first argument is the ``shortcut'' and the second is the index 
% entry. If you are loading \textsf{splitindex} with the 
% \textsf{split} option, the different types of index entry will be 
% sent to separate output streams, which each must be processed separately 
% with MakeIndex. \TeX\ has only a limited number of such output 
% streams, however, so it is probably wiser not to choose that 
% option, but to use splitindex.tex, splitindex.pl, or one of the 
% other means provided with the \textsf{splitindex} package to break 
% up the index into the separate parts, each of which must be 
% processed with MakeIndex separately.
% 
% To include the indices, and to add entries for them to the table of 
% contents, issue |\putmultiplepoemindex| for each one. 
% |\putmultiplepoemindex| takes four arguments, which are: 1. The 
% three letter ``shortcut'' used by |\splitindex|, 2. The title of 
% the Index on its first page, 3. The running header for that index 
% section, and 4. The name of the section as you wish it to appear in 
% the Table of Contents. So, for instance
% \begin{verbatim}
% \putmultiplepoemindex{fir}{INDEX OF FIRST LINES}{Index of First 
% Lines}{First Line Index} \end{verbatim}
% includes an index with shortcut ``fir,'' title ``INDEX OF FIRST 
% LINES,'' running header ``Index of First Lines,'' and contents 
% entry ``First Line Index.''
% \section{Collections with multiple volumes}
% \DescribeEnv{volumetitlepage} \DescribeEnv{maintitlepage}
% The |\volumetitlepage| environment is an
% environment for divider pages in collections made up of several
% volumes.  Volume title pages will always appear on recto pages. 
% \textsf{poemscol} will automatically create a blank verso page
% preceding the volume title page if it necessary.  The
% |\maintitlepage| environment is for the title page of the whole
% book.  The main title page will also automatically always be on a
% recto page. These divider pages have special
% page styles, with no page numbers and no running headers.
% 
% \begin{sloppypar}
% \DescribeMacro{\wholebooktitle} \DescribeMacro{\volumetitle}
% \DescribeMacro{\volumetitlefirstline}
% \DescribeMacro{\volumetitlemiddleline}
% \DescribeMacro{\volumetitlelastline} 
% \DescribeMacro{\volumesubtitle}
% \DescribeMacro{\volumesectiontitle} 
% |\wholebooktitle{}| is for the title of the entire collection.  
% |\volumetitle{}| is for the title of an individual volume within a
% Collected Poems.  The |\volumetitlefirstline{}|,
% |\volumetitlemiddleline{}|, and |\volumetitlelastline{}| macros are
% for multi-line titles of volumes gathered into the collection.  (Use
% these last commands if you want to specify how a long title is
% broken up both in the text and in the table of contents.
% \textsf{poemscol} will break up long titles on its own anyway
% if you don't specify how to break them up, but
% you might not be happy with how it breaks up the lines.)
% |\volumesubtitle{}| is for setting the subtitle to a volume.
% |\volumesectiontitle{}| is for setting the title of a section of a
% volume.  In addition to setting the text in the body of your
% edition, these commands send the title (and the formatting
% information about it) to the table of contents file and to the
% textnotes file.\end{sloppypar}
%
% \DescribeMacro{\volumeepigraph} \DescribeMacro{\volumeattribution}
% \DescribeMacro{\volumeheader} \DescribeMacro{\leftheader}
% \DescribeMacro{\rightheader} |\volumeepigraph{}| and
% |\volumeattribution{}| are for epigraphs and attributions on the
% divider pages.  |\volumeheader{}| or |\rightheader{}| is the right
% running header for that volume (except for divider pages, the table
% of contents, the notes sections, and other special cases).  Normally
% that is just the name of the particular volume.  |\leftheader{}| is
% the left running header for the entire book (except for divider
% pages, the table of contents, the notes sections, and other special
% cases).  Normally that is the name of the entire book.
% 
% \section{Recording textual notes}\label{recordingtextualnotes}
% \subsection{~Setup for sections for
% textual notes, emendations, and explanatory notes.}
% \DescribeMacro{\makeemendations} \DescribeMacro{\maketextnotes}
% \DescribeMacro{\makeexplanatorynotes} 
% To collect emendations, issue |\makeemendations| in your preamble. 
% To collect textual collations, issue |\maketextnotes| in your
% preamble.  To collect explanatory notes, issue
% |\makeexplanatorynotes| in your preamble.
% If you issue these commands,
% \textsf{poemscol} puts all of the emendations, textual collations,
% and explanatory notes in the backmatter of your volume.
% \textsf{poemscol} opens external files for emendations, textual
% collations, and explanatory notes (with extensions .emd, .ent, and
% .enx respectively), and reads them back in at the place you specify
% with |\puttextnotes|, |\putexplanatory|, or |\putemendations|, or at
% the end (if you use the |\finish| command).
%  These commands  also send to the external files the commands for typesetting the
% title of each section (e.g. Explanatory Notes) and for modifying the
% running headers of the notes sections to xx--yy format (e.g.
% ``Explanatory Notes to pp.~55--7'').
% 
% If you wish to use multi-level footnotes rather than endnotes, you 
% do not need to issue |\maketextnotes| or its relatives (but you 
% should of course issue |\textnotesatfoot| or its relatives). You can mix 
% footnotes or endnotes as you wish. You may, for instance, wish to 
% set explanatory notes as footnotes, and textual notes and 
% emendations as endnotes. Instructions for setting up 
% \textsf{poemscol} to make footnotes are in section 
% \ref{makingfootnotes} below.
% 
% In end note sections \textsf{poemscol} opens an entry (headed with
% the title and page of the poem in the text) in the textual notes
% section for every poem (since, minimally, you want to include
% information about the variants and copy text of the poem), but only
% opens entries in the emendations or explanatory notes sections if
% you actually have emendations or explanatory notes on that poem.
%
% The notes for each poem are set in a block paragraph, with a new
% paragraph for every |\poemtitle| command, and the line counter for
% the notes is reset to 1 with every |\begin{poem}|.  (For long poems
% in books or cantos, it would be wise to use |\poemtitle| for the
% canto names or numbers, setting the poem title in some other way,
% perhaps with |\volumetitle| described above.)
% 
% \begin{sloppypar}
% \DescribeMacro{\iftextnotessinglepar} 
% \DescribeMacro{\ifemendationssingplepar}
% \DescribeMacro{\ifexplanationssinglepar}
% You may, 
% particularly if you are annotating long poems, wish each 
% note to appear in its own paragraph rather than in a block paragraph 
% with the other notes to that poem. To do this, you need to set 
% the relevant boolean. Issue |\textnotessinglepartrue| 
% |\emendationssingplepartrue| or |\explanationssinglepartrue| to givve
% each such note a separate paragraph.
% \end{sloppypar}
%
% \DescribeMacro{\iftextnotestwocol} 
% \DescribeMacro{\ifemendationstwocol}
% \DescribeMacro{\ifexplanationstwocol}
% By default, notes are set in single columns. To set them in two column mode, 
% set the relevant booleans to true, e.g. |\textnotestwocoltrue|.
%
% \DescribeMacro{\putpagenumberinnotesfalse}
% \DescribeMacro{\puttitleinnotesfalse}
% You may not wish to have the poem's page number and title appear in
% the notes.  To turn off the appearance of the page number in the
% notes, issue |\putpagenumberinnotesfalse|.  You may also not wish to
% put the poem's title in the notes.  To turn off the appearance of the
% poem's title in the notes, issue |\puttitleinnotesfalse|.
%
% \DescribeMacro{\contentsendnotesindent}
% \DescribeMacro{\contentsendnotesfont}
% You can control the appearance of the entry for endnote sections in 
% the table of contents by changing the definition of 
% |\contentsendnotesindent| (set by default to |\contentsindentone|) 
% or |\contentsendnotesfont| (set by default to 
% |\contentspoemtitlefont|.
%
% \subsection{~Textual notes of various kinds}
% \DescribeMacro{\sources} To record information about the copy text,
% editions and publication history of individual poems, or any
% information not tied to specific lines in the poem, you should place
% that information in the argument to the |\sources{}| macro.
% Typically, you should issue this macro after you have issued
% |\poemtitle{}| and before you issue |\begin{poem}|.  If you wish to
% send information to the textual notes file (such as to force a page
% break), you can do so by using |\sources{}|.  You can send
% typesetting information to other sections by using
% |\literalemend{}|, |\literalexplain{}|, or |\literalcontents{}|.
% There is also a |\literaltextnote{}| command, which is equivalent to
% |\sources{}|.
% 
% \DescribeMacro{\textnote}
% \DescribeMacro{\emendation}
% \DescribeMacro{\explanatory}
% |\textnote{}| is used to capture variants and tie them to the
% correct line number.  You don't have to count line numbers. 
% \textsf{poemscol} keeps track of them for you.  Issue |\textnote{}|
% immediately after the |\verseline| command which marks the ending of
% the line you wish to comment upon.  Put the text of your note (which
% may be simply the recording of a variant in standard notation) into
% the argument of the macro.  You should put both the lemma and the
% variants or comment in the argument to the |\textnote{}| macro. 
% For emendations, use |\emendation{}|, and for explanatory notes,
% use |\explanatory{}|. These notes go to their own endnote sections. The
% only difference is that each poem opens a new paragraph for its own 
% textual notes, but paragraphs for explanatory notes or emendations are only 
% opened if there actually are explanatory notes or emendations for that poem.
%
% \DescribeMacro{\sameword}
% \DescribeMacro{\missingpunct}
%  To
% put the $\sim$ glyph in your note (used for recording places where
% the variant and the copy text have the same word, as for instance
% when recording a variation of punctuation) use |\sameword|.  To put
% the ${}_{\wedge}$ glyph into your text (used for recording places
% where a punctuation mark is missing in a variant), use
% |\missingpunct|. 
%
% For notes that refer to ranges of lines, |\textnote|, |\emendation|,
% and |\explanatory| have an optional argument, which records how many
% lines the note refers to.  There are two ways of using this
% argument.  The grubby but simple way is simply to put into the
% argument the number of lines it refers to.  If, for instance, you
% issue |\textnote[4]{lemma] rest of note}|, \textsf{poemscol}, rather
% than printing merely the number of the line which closed just before
% you issued |\textnote|, will print a line range, beginning at that
% line, and ending at the fourth line after it.  If that |\textnote|
% were issued after line 214, this is to say, the output in the
% textual notes would be: 214--218: lemma] rest of note.
% 
% \begin{sloppypar} It seems clumsy to count the lines referred to by
% a note, especially since you don't have to count lines to set where
% the note begins.  A more sophisticated way, and one which involves
% no counting, is to make a unique poem line label for the last line
% of the range by issuing |\poemlinelabel{some unique label}| just
% after the |\verseline| or |\end{stanza}| that ends the range.  (To
% use |\poemlinelabel| see section \ref{poemlinelabel}.)  Then, back
% after the |\verseline| or |\end{stanza}| command that closes the
% first line of the range, put the text of the label into the optional
% argument of |\textnote| as follows: |\textnote[some unique label]{lemma]rest of note}|.  
%This is the preferred method.  The
% line-counting method is deprecated, but retained for backwards
% compatibility with earlier versions.  If you use the label method,
% you will have to run your file through \LaTeX\ twice, once to define
% the labels, and once to set the text.\end{sloppypar}
%
% Here is a complete example:
%\begin{verbatim}
%Yes, then I met the beast. Well, better, indeed,\verseline
%\textnote[had it been the]{better, indeed, / Had it been the] better if 
%it / Had been the 1953TS (revised in black pencil 
%on 1953TS to 1953)}
%Had it been the manifest beast and the circumstantial\verseline
%\poemlinelabel{had it been the}
%Avatar of destruction. But no beast then: the towering\verseline
%Definition, angelic, arrogant, abstract,\verseline
%Greaved in glory, thewed with light, the bright\verseline
%Brow tall as dawn. I could not see the eyes.\end{stanza}
% \end{verbatim}
% 
% Overlapping lemmas, or lemmas that 
% include stanza breaks, do not pose any difficulties for |poemscol|. 
% 
% |\emendation{}| and |\explanatory{}| are used exactly as |\textnote{}| 
% is. Issue the emendation or the explanatory note as the argument to 
% the command. Place the command immediately after the |\verseline| that 
% concludes the line to which it is a comment. These commands also can 
% take optional arguments to specify line ranges.
% 
% If you quote verse \emph{within} a textual note, an explanatory
% note, or an emendation, you may want to use the normal \LaTeX\
% markup for verse for that quotation, rather than the special markup
% for \textsf{poemscol}.  \textsf{poemscol} will not complain about
% this.
% 
% If a textual note \emph{ends} with a quotation in verse, you should
% follow the |\textnote{}| entry with
% |\quotedversecorrectiontextnote|.  I found that if I quoted verse in
% the note sections a dirty trick I used to send the notes to the
% external files would cause the following notes on that poem in the
% notes sections to be set in a new paragraph.  (This is of course not
% a problem if the quoted verse is the last thing in the notes to that
% poem, since the next poem will open a new paragraph in the notes
% anyway.  It is also not a problem if more of the individual textnote
% follows the verse.  It's only a problem if the last thing in the
% note is quoted verse, and the note is not the last one on that
% poem.)  The command should set it flush as it is, but I have found
% odd cases I can't understand in which that doesn't work, so the
% command has an optional argument, which should be the additional
% length you need to adjust the next line by to make it flush.  The
% default is 0pt.  |\quotedversecorrectionexplanatory| and
% |\quotedversecorrectionemendation| should follow explanatory notes
% or emendations that end with snippets of verse.  These three
% commands are an embarrassing kludge on my part.  They simply send
% commands to the relevant note sections that prohibit opening a new
% paragraph.  You should issue this command in the main text right
% after the close of the textnote (not in the text of the textnote
% itself).
% 
% \DescribeMacro{\accidental} |\accidental{}| behaves exactly like 
% |\textnote{}|. If you wish to distinguish between accidentals and 
% substantives, this command provides a way of doing so. If you wish to 
% exclude 
% accidentals from your textual notes, issue 
% |\global||\includeaccidentalsfalse| in your preamble. Accidentals 
% are included by default, but if you issue |\includeaccidentalstrue| 
% in your preamble, you will remind yourself of that fact. 
% 
% \textsf{poemscol} does not provide for a separate backmatter section
% for accidentals, but it would be trivial to construct one, using 
% |\definenewnotetype|, described in section \ref{definingnewnotes} 
% below.  Use |\definenewnotetype| to create a command called 
% |\newaccidental|, then issue| \let\accidental=\newaccidental|.
% 
% If you wish to exclude accidentals from your printed output, but to
% mark them in your source files, so that your published collation
% consists only of substantives, issue
% |\global\includeaccidentalsfalse| in your preamble.  Many publishers
% are reluctant to publish accidentals, believing that they are, well,
% less substantive than substantives.  Using the |\accidental{}|
% command allows you to exclude accidentals from the published version
% should your publisher insist, while preserving the information about
% them should the publisher's mind change.  In the very worst case, if
% you have marked all the accidentals in this way you can still
% produce a list of accidentals for later use, and other scholars can
% search for accidentals in your source files simply by searching for
% the string |\accidental|.
% 
% 
% \DescribeMacro{\tsvariant} \DescribeMacro{\tsentry} Some publishers
% may turn up their noses at typescript or manuscript variants in just
% the way they turn up their noses at accidentals.  If you wish to
% distinguish between published variants and typescript,
% man\-u\-script, or galley variants, |\tsvariant| (for a whole note)
% and |\tsentry| (for a typescript entry in a list of variants in a
% textual note that includes other kinds of variant) provide ways of
% doing so.  By default, typescript variants are included.  To exclude
% typescript variants, issue |\global\includetypescriptsfalse| in your
% preamble.  You can issue |\global||\includetypescriptstrue| in your
% preamble, to remind yourself that you are including typescript
% variants.  In that (default) case |\tsvariant{}| will behave exactly
% like |\textnote{}|, and entries in other notes marked with
% |\tsentry| will be included in your output.
% 
% Here is how to use |\tsentry|:
% if you wish to include typescript entries in a single note including
% those entries in a list with variants from other published versions
% (as for instance when a comma appears in a typescript but only in
% the second edition of the published poem), simply issue
% |\textnote{}| as usual, marking the relevant variant in the list of
% variants with the |\tsentry{}| macro. If typescript variants
% are excluded, the typescript entry will also be excluded. 
%
% Here is a typical use of the |\tsentry{}| command:
% \begin{verbatim}
% Of moonlit desert. A stallion, white and flashing, slips,\verseline
% \textnote{Of moonlit] Of the moonlit {\em NY\/}\tsentry{, SP85TS 
% (revised in black pen to SP85)}}
% \end{verbatim}
%  
% Notice that since the |\tsentry{}| comes in the middle of the list, 
% it begins with a comma. You might expect an |\unskip| command 
% before the comma, to prevent 
% \LaTeX\ from adding white space before the comma. \textsf{poemscol} 
% automatically provides the |\unskip| if the entry starts with a 
% comma or a semicolon.
%
% You can
% mark individual variants with |\tsentry{}| in the arguments to the
% |\explanatory{}| and |\emendation{}| commands as well, and 
% |\tsentry| will also behave normally in any endnote or footnote section 
% you define using |\definenewnotetype|.
% 
% \textsf{poemscol} does not provide for a separate backmatter section
% for typescript variants, but it would be trivial to construct one,
% again using |\definenewnotetype|. To do this, you would also have to 
% copy all of the emendations, explanatory notes, and textual notes 
% which have a |\tsentry| in their arguments, and add
% |\tsvariant{}| commands in those places. For information about 
% |\definenewnotetype| see section \ref{definingnewnotes}.
%
% \section{Creating new kinds of endnote} \label{definingnewnotes}
% \DescribeMacro{\definenewnotetype} The |\definenewnotetype| macro
% can be used to create new kinds of endnote, in case the kinds
% \textsf{poemscol} defined are not enough.  |\definenewnotetype|
% takes five arguments.  They are \begin{enumerate} \item A base name
% used to construct \begin{itemize} \item some |\if| variables (e.g.
% |\ifmynoteson| to see whether you are collecting ``mynotes'',
% |\ifnomynoteyet| to see whether you have written a ``mynote'' for a
% particular poem yet, |\ifmynotestwocol| to set the notes in two
% column mode, and |\ifmynotesinglepar| to set the notes in
% one paragraph per note form, rather than in block paragraphs).  
% \textsf{poemscol} will only open an entry for a poem in the
% ``mynotes'' section if there actually is a ``mynote'' on that poem.
% Note that every  |\if| that refers to a section is plural
% (|\ifmynoteson|, |\ifmynotestwocol|, |\ifmynotessinglepar|) and 
% that every |\if| that refers to a singular note
% is singular (|\ifnomynoteyet|).  \item The name for the output
% channel for writing the notes to the appropriate external file.
% This name will always be plural.  \item The name for the ``literal''
% command for writing literal, unexpanded text on the file (as
% |\literaltextnote{}| does on the textnotes file), and for the note
% command that sends the actual textual note (line number and all) to
% the section (e.g. |\mynote{}|), also the command to write notes in
% line-numbered prose (e.g. |\prosemynote{}{}|) or prose numbered by
% paragraph and sentence (e.g. |\pmmynote{}|).  These commands will
% always be singular.  \item The name of the |\put| command (e.g.
% |\putmynotes|) which closes the external file for the notes section,
% reads it back in to your document, and adds an entry in the table of
% contents for that notes section.  This command will always be
% plural.  \item The name of the |\margref| command for putting a
% marginal mark next to a line giving the page number in the apparatus
% where a note is to be found (e.g. |\margrefmynote|) \end{itemize}
% \item The extension for the external file which will hold the notes
% as you generate them.  ``Emendations'' for instance, are written to
% an external file called |\jobname.emd| The file name will always be
% |\jobname|, and the different species of endnotes will all have
% distinctive extensions.  (|\jobname| is a macro that always expands
% to the name of your top level source file.  If, for instance, I am
% setting running a file called ``myfile.tex'' through \LaTeX\,
% |\jobname.emd| translates to ``myfile.emd.'') \item The title of the
% notes section as you wish it to appear in the running header (e.g.
% ``Emendations'').  \item The title of the notes section as you wish
% it to appear on the first page of the notes section (e.g.
% ``EMENDATIONS'').  \item The title of the notes section as you wish
% it to appear in the table of contents.  \end{enumerate}
% 
% \sloppypar So, for instance,\begin{verbatim} 
% \definenewnotetype{xyznote}{xyz}{Xyznotes to}{XYZNOTES}{Xyznotes} 
% \end{verbatim}
% creates the booleans |\ifxyznoteson|,  |\ifnoxyznoteyet|, and 
% |\ifxyznotessinglepar|,  the
% commands |\xyznote|, |\literalxyznote|, |\prosexyznote|,
% |\pmxyznote|, and |\putxyznotes| (and, although you don't ever have
% to use it explicitly, |\firstxyznote|) the output channel xyznotes,
% the external file |\jobname.xyz|, an entry in the contents for
% ``Xyznotes'', and an endnotes section with a running header of the
% form ``Xyznotes to pp.~xx--yy'' and a title XYZNOTES. |\xyznote|
% takes the same arguments that |\explanatory| does (which is to say,
% an optional argument for a label for entering line ranges, and an
% argument for the lemma and the comment).  There won't be an entry
% for a poem in the ``Xyznotes'' endnotes section unless there
% actually is an xyznote for the poem.  |\prosexyznote| takes the same
% arguments |\proseexplanatory| does (which is to say, an optional
% argument for a label for entering line ranges, an argument for a
% label to capture the line number of the first line, and an argument
% for the lemma and the comment).  |\pmxyznote| behaves exactly like
% |\pmexplanatory|.
% 
% \section{Multi-level footnotes} \label{makingfootnotes}
% \textsf{poemscol} will set multiple 
% levels of footnotes in block paragraphs, separating each level with 
% a rule. To enable this feature, call the \textsf{manyfoot} package 
% as follows in your preamble: |\usepackage[ruled,para*]{manyfoot}|.
% 
% To set your textual notes as footnotes, issue |\textnotesatfoot| in
% your preamble.  If you do this, \textsf{poemscol} will actually
% create two sets of footnotes for textnotes, one for those that are
% called with |\textnote|, which will produce an unindented block of
% notes, and one for those that are called with |\sources|, which will
% produce individual indented paragraphs, opening a new paragraph with
% every note.  |\sources| notes often have several paragraphs, so they
% are inappropriate for block-paragraph or ``para'' mode, and because
% those paragraphs are often in fact prose paragraphs, rather than
% blocks of notations, they are indented, so that they will be read as
% prose.  To set your emendations as footnotes, issue
% |\emendationsatfoot| in your preamble.  These notes will be set by
% default as block paragraphs.  If you have multiple-paragraph notes
% in your emendations, you should redefine these notes to set plain
% rather than para style notes. (See the next paragraph for how to do 
% this.) To set your explanatory notes as
% footnotes, issue |\explanationsatfoot|.  These notes will also, by
% default, be plain style notes, with one indented paragraph per note.
% You do not need to issue |\maketextnotes|, |\makeemendations| or
% |\makeexplanatorynotes| if you set those notes as footnotes.  If you
% use |\literaltextnote| anywhere in your file, however, you will lose
% the text it would have put in your textnotes, since |\literaltext|
% just writes its argument onto the external file for textual notes.
% 
% 
% To change the style of your footnotes between plain and
% para modes, globally change the values some of the following
% booleans.  |\ifsourcesfootnotespara|, obviously, is true if
% footnotes from the |\sources| command are set in ``para'' mode
% (default is false).  |\iftextfootnotespara| is true if footnotes
% from |\textnote| are set in ``para'' mode (default is true).
% |\ifemendationfootnotespara|, or |\ifexplanfootnotespara| are the
% booleans which set the style of footnotes from |\emendation| and
% |\explanatory| respectively.  By default, footnotes from
% |\emendation| are set in ``para'' mode, and footnotes from
% |\explanatory| are set in ``plain'' mode.  g explanatory
% footnotes from plain to para mode, for instance, issue
% |\global\explanfootnotesparatrue|.  Counterintuitive as it sounds,
% you must issue this \emph{before} issuing |\textnotesatfoot| (or the
% command for whichever other kind of note you are wishing to change
% the appearance of).
% 
% When |\sources| is defined to send endnotes to the textual notes 
% section, they usually appear immediately underneath the title of 
% the volume or poem they comment upon. If you are having |\sources| 
% output footnotes, it would be a good idea to open the note with the 
% title of the volume or poem it is a comment upon, in case there is 
% more than one poem on the page.
% 
% When note classes are set as footnotes, the symbol which marks 
% cases in which a page turn falls on a stanza break will be set 
% \emph{beneath} the footnotes. This is a consequence of the way 
% \textsf{poemscol} uses the |\mark| mechanism to keep track of these 
% cases, so I have not figured out how to change it. But it may well 
% be more of an idiosyncrasy than a bug, particularly if the symbol 
% is changed from * to something like ``stanza break at page turn.''
%  
% To create a new level of footnote, you must first declare it to 
% make it known to \textsf{manyfoot}, which requires each level of 
% footnote to be given a unique suffix (usually a capital letter of 
% the alphabet). You can declare a new footnote level called 
% |\footnoteE{}| for instance, by issuing 
% |\DeclareNewFootnote[para]{E}|. You will want notes which don't put 
% a mark in the body of the text, and begin with the line number or 
% range of line numbers of the lemma. To define a footnote called 
% |\mynote{}|, first declare new footnote |\footnoteE| (A, B, C and D 
% are reserved for sources, textual notes, emendations, and explanatory notes, 
% respectively). Then issue
%  |\newcommand{\mynote}[2][0]{\setlemmarange{#1}|\\
% |\FootnotetextE{\citerange}{#2}|
% | \resetlemmacounters}| (|\setlemmarange|, |\citerange|, and 
% |\resetlemmacounters| enable your footnote to handle line ranges in 
% the lemma.) If a |\mynote| refers to a range of lines, you can issue 
% the same kinds of optional argument that are used with |\textnote|.
% 
% \section{Pulling it all together} 
% \DescribeMacro{\testforauxonfirstrun} If you don't want to listen to
% \LaTeX\ complain about missing references during your first run,
% issue |\testforauxonfirstrun|, which won't let |\finish|,
% |\putexplanatory|, |\puttextnotes|, or |\putemendations| run unless
% there is an .aux file for a previous run.  \emph{Be sure to issue
% this command before issuing} |\begin{document}|, \emph{since that
% reads in the existing .aux file, and overwrites it with a new one.}
% 
% \label{putexternalfiles}
% \DescribeMacro{\putemendations} This macro closes the external file
% for emendations and reads it in.  It also puts an entry for the
% emendations section into your table of contents.  After
% |\putemendations| has read in the external file for emendations, you
% will need to run your file through \LaTeX\ again, to put the correct
% page number in the table of contents file.
% 
% \DescribeMacro{\puttextnotes}
% \DescribeMacro{\putexplanatory} |\puttextnotes| and 
% |\putexplanatory| do the same thing for the textual 
% notes and for the explanatory notes.
% 
% All of these last three commands restore normal hyphenation. Since 
% all three are printed with the |\small| font size, the hyphen 
% character is defined only for that size. This should not create 
% other issues, however, because the commands that prohibit 
% hyphenation in the |poem| environment (and in the |contentsentry| 
% environment) are local to those environments, so the restoration of 
% the hyphen character may be redundant.
%
% \DescribeMacro{\putpoemindex} |\putpoemindex| reads in the index at
% this point, and adds an entry for the index to external file for the
% table of contents.
% 
% \DescribeMacro{\putmultiplepoemindex} If you are generating 
% several indices using \textsf{splitindex} |\putmultiplepoemindex| 
% reads in one of them at this point. For instructions about this 
% command, see section \ref{multipleindex}, p.~\pageref{multipleindex}.
% 
% \DescribeMacro{\finish} The
% |\finish| macro is deprecated but still available.  It is designed
% to be placed just before |\end{document}| (which, in fact, it
% replaces --- but you should still put an |\end{document}| at the end
% of your document for form's sake).  |\finish| closes the external
% files for the three notes sections and reads them back into the
% document (turning the automatic hyphenation back on for these
% sections).  It adds entries to the table of contents file for the
% notes sections.  It reads in the formatted index file (which you
% have made from your raw .idx file using MakeIndex between \LaTeX\
% runs) and adds an entry for the index to the table of contents.
% Then it closes the external file for the table of contents and reads
% that in (putting it at the end of the document, as I explained
% before).  Then it adds a page explaining that the symbol (by default
% *) at the bottom of certain pages marks places where a stanza break
% and a page break coincide.  If you have an editor's introduction
% preceding your notes section, or a list of abbreviations, you may
% wish to consult the |\finish| macro for suggestions about how to add
% the lines to read those files in and add entries for them in the
% table of contents.
%
% 
% \section{Special cases}
% \subsection{~Epigraphs, attributions etc.}
% \DescribeMacro{\epigraph} \DescribeMacro{\headnote}
% \DescribeMacro{\attribution} \DescribeMacro{\poemdedication} 
% |\epigraph|, |\headnote|, |\attribution|, and |\poemdedication| set
% their arguments as epigraphs, headnotes, attributions, or
% dedications.  If you have dedications, epigraphs, or attributions on
% the divider pages of editions that combine several volumes of
% poetry, use |\volumededication{}|, |\volumeepigraph{}|, and
% |\volumeattribution{}|. You can adjust the left and right margins 
% of epigraphs by changing the values of |\epigraphquoteleftmargin| 
% or |\epigraphquoterightmargin|
% 
% \subsection{~Inset prose quotations}
% \DescribeMacro{\saveverselinenumber}
% \DescribeMacro{\restoreverselinenumber} Sometimes you will have inset prose 
% passages, but you will not wish to start the line numbering over 
% after the passage. The way to set this is to issue |\end{stanza}| 
% at the end of the stanza before the prose insert. Then issue 
% |\saveverselinenumber|. Then issue |\end{poem}|. Insert the prose 
% quotation (perhaps followed by |\attribution{}|). Then issue 
% |\begin{poem}| followed by |\restoreverselinenumber| and |\begin{stanza}|. 
% Then begin entering the rest of the poem as usual. Longer, independent prose 
% sections (such as the author's preface or prose poems) require 
% different handling. To provide line numbers and notes for such 
% prose sections, see section \ref{prosesections}.
% 
% \subsection{~Multiline poem titles}\label{multilinetitles}
% \DescribeMacro{\poemtitlefirstline}
% \DescribeMacro{\poemtitlemiddleline}
% \DescribeMacro{\poemtitlelastline}
% \DescribeMacro{\poemtitlenocontents}
% \DescribeMacro{\poemtitlebaretitle}
%  Very long titles present special problems.  In the first place, they
% require special provisions to make sure that page breaks don't
% happen in the middle of a title, or immediately after a title.  In
% the second place, they raise questions about how the author wished
% to break them up over several lines.  If you think it might matter,
% these macros make it possible for you to ensure that long titles are
% broken in exactly the same way in your text and in your table of
% contents.  (I didn't pursue this question for titles in notes
% sections, and in notes sections these commands simply concatenate a
% long title.) 
% 
% If you wish to set the titles differently in the text and in the
% contents, you can set the title in the text and notes with
% |\poemtitlenocontents{}| and send a differently formatted entry to
% the table of contents file using |\literalcontents{}|. (Remember 
% that you will have to send not only the title of the poem to the 
% contents, but also the other formatting information, such as the 
% indentation, the slash, and the label for the page number.)
% |\poemtitlebaretitle{}| simply sets the title in the text, enabling
% you to use |\literalcontents{}| and |literaltextnotes{}| to produce
% whatever effect you wish.
% 
% Alternatively, you can just let \textsf{poemscol} handle the long 
% titles by itself. If you are pretty sure that the author did not 
% care how long titles were broken up, this would probably be the 
% best thing to do. Remember also that if \textsf{poemscol} breaks a 
% title across two lines, you can always force it not break the line 
% by using enclosing the argument to |\poemtitle{}| in |\hbox{}|. You 
% may get a complaint on your log file about an overfull |\hbox| if 
% you do this, however.
%
% \subsection{~Italicized poem titles and empty poem titles} Titles
% with italics or other special formatting posed special problems in
% earlier versions (before version 1.7) since the typesetting commands
% gummed up how I passed parameters around that I was unable to solve
% using |\protect|.  All of these commands are obsolete now, but are
% retained for backward compatibility.
% 
% \DescribeMacro{\poemtitleitalic{}{}} |\poemtitleitalic{}{}| takes
% two arguments.  The first is the title of the poem with the italics
% (or small capitals or international characters or what have you),
% for typesetting, and the second is the title without the commands,
% for cross reference purposes.  (Actually, now that \textsf{poemscol}
% no longer needs to treat italics in titles in a different way, it can
% be anything you want, since the second argument is discarded.)
% 
% \DescribeMacro{\poemtitlenotitle{}}
% |\poemtitlenotitle{}|, with its paradoxical name, is for poems 
% without formal titles. Usually such poems are referred to by their 
% first lines. This macro does not produce a title in the text, but 
% sets its argument in the table of contents and in the notes. Usually 
% you use the first line as the moral equivalent of a title.
% 
%  \DescribeMacro{\poemtitlenotitleitalic{}{}}
%  |\poemtitlenotitleitalic{}{}|, with its even more paradoxical name, 
% is for poems without formal titles but with italicized words in their 
% first lines.  |\poemtitlenotitleitalic{}{}| takes two arguments. The 
% first is the first line with the italics (for typesetting 
% purposes), and the second is the 
% first line without the italics (for crossreference purposes). This command
% is obsolete now that italics in titles no longer require special 
% handing, but is retained for backwards compatibility.
% 
% \subsection{~Poems in sections}
% 
% \begin{sloppypar}
% \DescribeMacro{\poemsectiontitle} 
% |\poemsectiontitle{}| and  |\poemsubsectiontitle{}|, as their 
% names imply, are for the titles of sections or subsections of poems
% that have them.  |\poemfirstsectiontitle| is for the first section
% of a poem with many sections.  (It differs from |\poemsectiontitle|
% in that it tries to make it harder for a page break to separate the
% title of a poem from the title of its first section.)
% |\poemsectiontitlefirstline{}|, |\poemsectiontitlemiddleline|,
% |\poemsectiontitlelastline{}|, |\poemsectiontitlenocontents{}|, and
% |\poemsectiontitlebaretitle{}|), as their titles imply, are used for
% special cases such as long section titles, if you wish to specify
% explicitly how the title is to be broken up in the text and in the
% table of contents.  You can construct variants of these macros for
% dealing with special cases I didn't anticipate by using
% |\poemsectiontitlebaretitle{}| and |\literalcontents{}| and its
% siblings.\end{sloppypar}
% 
% \subsection{~Poetic sequences}\label{poeticsequences}
% 
% \DescribeMacro{\sequencetitle} \DescribeMacro{\sequencesectiontitle}
% \DescribeMacro{\sequencefirstsectiontitle} Sequences of related
% poems, perhaps with an over-title, demand special handling in the
% text, in the notes sections, and in the table of contents.  (A
% moment's thought will show that they are different in some ways from
% poems in sections.  For one thing, a sequence manifests a different
% relationship between part and whole than a poem in sections does,
% and often provides stronger experiences of momentary closure between
% sections.) |\sequencetitle{}| sets the title of the sequence, and 
% |\sequencesectiontitle{}| sets the titles of the sections of the 
% sequence. |\sequencefirstsectiontitle{}| gives a little more
% vertical space between the main title and the first section title
% than |\sequencesectiontitle{}| gives between adjacent sections later
% on in the sequence (since the over-title for the sequence is in
% larger type) and adjusts the page breaking penalties to reflect the
% fact that such titles should not occur near the bottom of a page,
% since there should be no page break between the sequence title and
% the title of the first section of the sequence (although page breaks
% are permissible, indeed even slightly favored, between later
% sections).  There should also never be a page break between a
% section title and the first or second line of the section.  The
% penalties \textsf{poemscol} sets should take care of these
% automatically, but you may still have cases which require you to
% specify a page break with |\pagebreak| or |\newpage|.  There are a
% host of other commands, all with self-explanatory titles, that deal
% with some special situations that arise in setting the elements of
% poetic sequences, such as cases where you wish to specify how long 
% titles will be broken up into lines, or where you wish no entry to 
% be made for the section in the table of contents or in the notes 
% sections.   These macros don't of course exhaust the dizzying
% possibilities, but from them you can construct whatever other macros
% you may need. 
%
% Here is a list of all of the defined title commands. 
% 
% \begin{itemize}
%     \item Macros for Poem titles
%     \begin{itemize}
% \item |\poemtitle|
% \item |\poemtitlefirstline|
% \item |\poemtitlemiddleline|
% \item |\poemtitlelastline|
% \item |\poemtitlenotitle|
% \item |\poemtitlenocontents|
% \item |\poemtitlebaretitle|
% \item |\poemsubtitle|	
% 	\end{itemize}
%      \item Poems in Sections
%      \begin{itemize}
% \item |\poemsectiontitle|
% \item |\poemsectiontitlefirstline|
% \item |\poemsectiontitlemiddleline|
% \item |\poemsectiontitlelastline|
% \item |\poemsectiontitlenocontents|
% \item |\poemsectiontitlebaretitle|
% \item |\poemfirstsectiontitle|
% \item |\poemfirstsectiontitlebaretitle|
% \item |\poemsubsectiontitle|
% 	 \end{itemize}
%       \item Sequence Titles
%       \begin{itemize}
% \item |\sequencetitle|
% \item |\sequencetitlefirstline|
% \item |\sequencetitlemiddleline|
% \item |\sequencetitlelastline|
% \item |\sequencetitlenonotes| 
% 	  \end{itemize}
% 	\item Sequence Sections
%        \begin{itemize}
% \item |\sequencesectiontitle|
% \item |\sequencesectiontitlenocontents|
% \item |\sequencesectiontitlenonotes|
% \item |\sequencesectiontitlebaretitle|
% \item |\sequencesectiontitlefirstline|
% \item |\sequencesectiontitlemiddleline|
% \item |\sequencesectiontitlelastline|
% \item |\sequencefirstsectiontitle|
% \item |\sequencefirstsectiontitlenocontents|
% \item |\sequencefirstsectiontitlenonotes|
% \item |\sequencefirstsectiontitlefirstline|
% \item |\sequencefirstsectiontitlemiddleline|
% \item |\sequencefirstsectiontitlelastline|
% \item |\sequencesectionsubtitle|
% \end{itemize}
%          \item Sequence Subsections
%          \begin{itemize}
% \item |\sequencesubsectiontitle|
% \item |\sequencesubsectiontitlenocontents|
% \item |\sequencefirstsubsectiontitle|
% \item |\sequencefirstsubsectiontitlenocontents|
% \item |\sequencesubsectiontitlefirstline|
% \item |\sequencesubsectiontitlemiddleline|
% \item |\sequencesubsectiontitlelastline|
% \item |\sequencefirstsubsectiontitlefirstline|
% \item |\sequencefirstsubsectiontitlemiddleline|
% \item |\sequencefirstsubsectiontitlelastline|
% 
% \item |\sequencesubsubsectiontitle|
% \item |\sequencesubsubsectiontitlenocontents|
% \item |\sequencefirstsubsubsectiontitlenocontents|
% \end{itemize}
% \end{itemize}
% 	
% 	
% \subsection{~Customizing titles} \label{customizing titles} \textsf{poemscol} provides a large
% number of ways of setting titles, but conceivably you may wish to
% define your own.  Here is a template for designing a new kind of
% title.  Suppose you wish to create a kind of title called
% |\mytitle{}|.  It would take one argument, which would be the title
% itself.  The macro defining the |\mytitle| command should call three macros,
% |\set@p@emtitle|, to set the title in the main text and make the
% label for use by the contents and notes sections, |\c@ntentsinfo| to
% set the title in the table of contents, and |\t@xtnotesinfo|, which
% sets the title in the notes sections.  (You should enclose the whole thing
% between |\makeatletter| and |\makeatother|.) 

% |\set@p@emtitle|
% takes six arguments, which are, in order, (1) the title (which, if
% you are calling this command inside of the definition of |\mytitle|
% should just be |#1|), (2) commands to be issued before setting the
% title (minimally, a penalty, such as |\poemtitlepenalty|), (3) the
% font size in which the title is to be set in the text (such as
% |\poemtitlefont| or |\large|), (4) any indentation of the title
% (|\relax| for |\poemtitle|, or |\titleindent| for
% |\sequencesectiontitle|), (5) commands to be issued after the
% title is set (such as |\afterpoemtitleskip|) and (6) the amount 
% overrun lines in the title should be indented.  

%|\c@ntentsinfo| takes
% four arguments, which are, in order, (1) the title, (2) the
% indentation of the title in the table of contents (such as
% |\contentsindentone|), (3), the font size in which the title
% should be set (such as |\contentspoemtitlefont|) and (4) the amount 
% overrun lines in the table of contents should be indented (such as 
% |\contentsindenttwoamount|). 

%  |\t@xtnotesinfo|
% takes one argument, the title. 

%  So, for example, to create a new kind of title called |\mytitle| which looks
% just like |\poemtitle|, issue
% \begin{verbatim}
% \makeatletter
% \newcommand{\mytitle}[1]{%
% \set@p@emtitle{#1}{\poemtitlepenalty}{\poemtitlefont}%
% {\relax}{\nobreak\par\nobreak\afterpoemtitleskip\nobreak}{\z@}%
% \c@ntentsinfo{#1}{\contentsindentone}{\contentspoemtitlefont}%
% {\contentsindenttwoamount}%
% \t@xtnotesinfo{#1}%
% }
% \makeatother
% \end{verbatim}
%
% (The |\z@| which is the sixth argument to |\set@p@emtitle| above is
% a constant set to zero. You will also notice that I became completely hysterical
% on the subject of preventing a page break after a poem title.)

% By default, \textsf{poemscol} 
% does not indent the overrun portion of titles relative to the first 
% line of the title, but if that first line is indented will set the 
% overrun portions to be indented the same amount. Overrun portions 
% of titles in the table of contents will be indented by one 
% |\contentsindentamount| more than the first line is indented. Both 
% of these behaviors can be changed by changing the arguments to 
% |\set@p@emtitle| or |\c@ntentsinfo|. 
% 
% If you wish to design a title which might have multiple lines, 
% and you wish to specify how those lines are broken up, you
% will need to write three macros: one for the first line, one for
% middle lines, and one for the last lines. 
% (This only matters if you care how the title is broken up. You 
% can choose to let \LaTeX\ do the breaking if you wish.) Before issuing
% |\set@p@emtitle| issue |\firstlinesettings|, |\middlelinesettings|,
% or |\lastlinesettings| (as the case may require).  And issue
% |\restoresinglelinesettings| last of all in each of your three
% macros.  The lines will be broken in the table of contents and in
% the body of the text in the same way, but will be run in in the
% notes sections.
% 
% \DescribeMacro{\JHpoemtitle}
% In his edition of the poetry of Jos\'e Maria Heredia, Tilmann 
% Altenberg placed a marker in the margin next to poem titles with a
% reference to where in the apparatus notes on that poem could be
% found.  To do this, use the |\JHpoemtitle| command.  This command
% takes two arguments: first, the title of the poem, second, the label
% used by the |\label| command on the page of the notes section to
% which the title refers.  So, you would issue the poem title command
% this way: |\JHpoemtitle{Los desvelados}{EP_LosDesvelados_AC}|.  Up to
% version 2.49 of \textsf{poemscol} you had to be sure to define the
% label in the notes sections by issuing
% |\sources{\label{EP_LosDesvelados_AC}}|.  This is no longer
% necessary, and |\JHpoemtitle| defines both the |\label| and the
% |\pageref|.  Instructions are given in section \ref{JHpoemtitle},
% p.~\pageref{JHpoemtitle} for how to construct similar commands for
% sequences, sequence sections, and so on.
% 
% The following related commands are defined already 
% \begin{itemize}
% \item |\JHsequencetitle|, \item |\JHsequencefirstsectiontitle|, 
% \item |\JHsequencesectiontitle|, \item |\JHsequencesubsectiontitle|,  
% \item |\JHpoemsectiontitle|,   \item |\JHpoemfirstsectiontitle|, 
% \item |\JHpoemsubtitle|,  \item |\JHepigraph|,\item  |\JHprosesectiontitle|, and 
% \item |\JHdedication|.\end{itemize}
% 
% If you are using |\JHpoemtitle| and its siblings, you need to be
% sure to add \textsf{mparhack} to your package list.  You can prevent
% the marginal reference from switching sides on alternate pages by issuing 
% |\@mparswitchfalse| (be sure to put |\makeatletter| and 
% |\makeatother| around this command). You can control the side on 
% which the marginal reference appears with |\normalmarginpar| (to 
% place them on the right) or |\reversemarginpar| (to place them on 
% the left). |\JHpoemtitle| and its siblings place the marginal 
% reference using |\marginpar|.
% 
% \DescribeMacro{\setmpargpoemtitle}
% |\setmargpoemtitle{}{}| is a more general version of 
% |\JHpoemtitle{}{}|, with the second argument being any commands you 
% want to be issued while the title is being processed.
% 
% \DescribeMacro{\margreftextnote}
% Another feature of Altenberg's Heredia edition is a marginal mark 
% attached to lines to which text notes refer showing where in the 
% apparatus a comment upon that line is to be found. 
% |\margreftextnote| sets a label in the textual notes section, and 
% set a marginal note with a page reference to that label. Issue 
% |\margreftextnote| \emph{before} issuing the |\verseline| or 
% |\end{stanza}| that closes the line to which it refers. (The textual 
% note itself can be an ordinary |\textnote|, and should be place, as 
% usual, \emph{after} the |\verseline| or |\end{stanza}|.
% 
% |\margreftextnote| generates its own label, so there is no need to 
% make up one yourself. 
% 
% |\margreftextnote| uses |\marginpar| rather 
% than |\pmclsidepar| to set the note, because I had to use 
% |\marginpar| for |\JHpoemtitle| and its siblings, and I wanted the 
% whole family of markers to be controlled by the same settings. You 
% use the settings that control the behavior of |\marginpar| (e.g. 
% |\mparswitch| and |\reversemarginpar|) to control the behavior of 
% these marks. 
% 
% \begin{sloppypar}
% If you define a new endnote type using |\definenewnotetype|,
% |\definenewnotetype| will also define a new |\margref| command to go
% with it.\end{sloppypar}
% 
% \DescribeMacro{\setmargrefmarker} Altenberg's edition used a ding 
% from the pifonts collection in the marginal marks made by 
% |\margreftextnote| and its siblings. I have set the default to 
% |\dag|, so as not to depend upon yet another style package. You can 
% choose another mark using |\setmargrefmarker| (e.g. 
% |\setmargrefmarker{\ding{228}}|).
% 
% \DescribeMacro{\margrefemendation} |\margrefemendation| is a 
% version of |\margreftextnote| for emendations.
% 
% \DescribeMacro{\margrefexplanatory}|\margrefexplanatory| is a 
% version of |\margreftextnote| for explanatory notes.
%
% \subsection{~Quoted Verse}
% Sometimes you wish to quote verse within verse (as when a speaker 
% in a verse play quotes a poem). \DescribeMacro{\quotedverse} The 
% |quotedverse| environment is desiged for this contingency.
% 	
% \subsection{~Visual formatting}
% \DescribeMacro{\longpage}
% \DescribeMacro{\shortpage}
% The ideal in \LaTeX\ is to separate 
% content markup and formatting as completely as possible. In the text 
% itself you simply mark the logical units (a poem, a title, a stanza, 
% a textual note), and a header program like this interprets the 
% meaning of those terms typographically. However, one still winds up 
% having to do some visual formatting. Especially in prose sections, 
% such as in the textual notes, you will sometimes have to enlarge or 
% shrink a page by one line in order to avoid an intractable widow or 
% club line, and when one does so one must be careful to do the same to 
% the other member of the two-page spread of which that page is a part
% so that the spread will be properly balanced. 
% These commands enable one to wiggle pagelength enough to handle these 
% problems. Since verse sections should be set with |\raggedbottom| 
% in effect, the commands are not necessary in verse sections.
%
% |\longpage| and |\shortpage| are cumulative: issuing |\longpage| twice lengthens the
% page by two lines, and issuing  |\shortpage| twice shortens the page by two lines.
% 
% \section{Customizing page geometry}
% 
% 
% \subsection{~Font sizes, skips, sinks, indents, and penalties}
% 
% \label{changingfontsandleading}
% \textsf{poemscol} sets the font sizes, but not the font family, of
% fonts for many common tasks, in order to create a hierarchy of
% sizes.  They can be changed with a |\renewcommand| of the following
% form, where the first number is the font size in points and the
% second is the leading:\\
% |\renewcommand{\volumetitlefont}{\fontsize{18}{23}\selectfont}|
% 
% 
% The following table gives the default values for the fonts. The titles of 
% the fonts give their uses. Backmatterheaderfont is for setting the 
% titles of back matter sections. Sequence section titles use 
% poemtitlefont. Backmattertext font is |\small|, backmatterintrofont 
% is |\normalsize|.
% \bigskip
% 
% \begin{tabular}{l l l}
% \textbf{Font} & \textbf{Size} & \textbf{Leading} \\
% volumetitlefont & 18 & 23 \\
% volumesubtitlefont & 14 & 18 \\
% poemtitlefont & 14 & 18 \\
% sequencetitlefont & 16 & 20 \\
% subsectiontitlefont & 12 & 14.5 \\
% backmatterheaderfont & 16 & 20 \\
% backmattervolumefont & 12 & 14.5 \\
% backmattervolumesubtitlefont & 11 & 13 \\
% contentsvolumefont & 13 & 20 \\
% contentsvolumesubtitlefont & 12 & 14.5 \\
% contentssequencetitlefont & 10 & 14 \\
% contentspoemtitlefont & 10 & 14 \\
% \end{tabular}
% 
% \bigskip
% 
% The following table gives various skips, sinks, penalties, and indentations. The 
% titles of these should be self-explanatory. These should be changed 
% with |\renewcommand| as follows:\\ 
% |\newcommand{\volumetitlesink}{\leavevmode\vspace{23.5pt}}|
% \bigskip
% 
% \begin{tabular}{l l l}
%     \textbf{Name} & \textbf{Argument} & \textbf{Comment} \\
%     backmattersink & 21.5pt & \relax \\
%     backmatterafterheadersink & 11pt& \relax \\
%     afterpoemtitleskip & |\smallskip| & \relax \\
%     afterpoemskip & |\bigskip| & \relax \\
%     aftersequencetitleskip & |\medskip| & \relax \\
%     stanzaskip & |\medskip| & Extra skip between stanzas \\
%     titleindent & |\titleindentamount| (1pc) & use |\setlength| to 
%     change \\
%     parindent & titleindentamount & \relax \\
%     volumetitleshift & volumetitleshiftamount & use |\setlength| to 
%     change \\
%     contentsindentone & 24pt & \relax \\
%     contentsindenttwo & 42pt & \relax \\
%     contentsindentthree & 60pt & \relax \\
%     contentsindentfour & 72pt & \relax \\
%     runoverindentvalue & 6em & for runover lines in verse\\
%     verseindent & 2em & for formally indented lines\\
%     notesclubpoempenalty & -300 & \relax \\
%     notessequenceclubpenalty & -1500 & \relax \\
%     notesvolumetitlepenalty & -1500 & \relax \\
%     epigraphquoteleftmargin & 1.5em & for epigraphs, attributions, etc.\\
%     epigraphquoterightmargin & 1.5em & \relax \\
%     \end{tabular}
% 
% \subsection{~Page styles}
% \textsf{poemscol} also sets several default page styles, defined 
% below. The main page style is ``fancy'' (defined below), used for 
% prose frontmatter (such as introductions), the main body of the 
% poetry, and the endnotes sections. ``Volumefirststyle'' is for the 
% first page of a volume of poems. ``Volumetitlestyle'' is for the 
% title page of a volume of poems. The external file for the table of 
% contents redefines ``fancy,'' so you should issue |\resetpagestyle| 
% after issuing |\putpoemcontents|. 
% 
% \subsection{~Page geometry and type leading}
% 
% You have to give geometry the values that suit you.  I include the
% values I used in my own book, just to give you a starting place.
%  The default values for marginparsep and
% marginparwidth are different now (18pt and 10pt respectively).  With
% that warning, here is a list of the values I used:
% 
% \bigskip
% 
% \begin{tabular}{l l}
%     textheight&6.75in\\
%     marginparsep&0pc\\
%     marginparwidth&0pc\\
%     headheight& 14pt\\
% paperwidth&6in\\
% paperheight& 9in\\
% inner&4.5pc\\
% outer&4.5pc\\
%     footskip&15pt\\
%     \end{tabular}
%     
%\bigskip
% 
%% 
%% In my own book I also, strictly to save money, and at the price of
%% some ugliness, changed the definition of |\normalsize| and |\small|,
%% so that the former was 10 pt type on 11 pt leading (rather than 12
%% pt), and that the latter was 9 pt type on 10 pt leading.  You can
%% reproduce this setting by issuing |\tightleading|, although I would
%% advise not doing so unless you are under financial constraints.
% 
% \section{Prose Sections with Line Numbers} \label{prosesections} \textsf{poemscol}
% uses the \textsf{lineno} package to make marginal line numbers for
% prose sections such as the Author's Introduction, Afterword,
% Author's Notes, or such prose interludes or prose poems the volume
% may include.  To control the appearance, placement, and frequency of
% marginal line numbers in prose sections, you may wish to consult the
% manual to \textsf{lineno}.  I have added a minimal user interface to
% make \textsf{lineno} cohere with poemscol.  You may have to fiddle
% with the parameters of both \textsf{poemscol} and \textsf{lineno} to
% persuade them to set marginal line numbers that look the same way in
% prose and verse contexts.  The default values should work as is.
%
% Line numbers in prose sections are placed using a different
% |\length| from that used in verse sections.  Essentially, verse
% sections, and prose sections which use paragraph and sentence
% numbering --- see below, in section \ref{paragraphsentence} ---
% place the marginal line numbers using |\marginparsep|.  Prose
% sections which are numbered by line use |\linenumbersep| instead,
% which is a length defined in the \textsf{lineno} package. I have
% given values for both lengths which seem to place line numbers on
% the left and on the right side in prose sections roughly where they
% are placed in verse sections.  You can change these values with
% |\setcounter{countername}{length}|.
%
% |\verselinenumberstoleft| and |\verselinenumberstoright| also place
% the line numbers to the left or to the right, respectively, in prose
% sections.  If you wish the numbering to appear on a different side
% in prose contexts you can set the side by issuing |\leftlinenumbers|
% or |\rightlinenumbers|.  These last are commands from the
% \textsf{lineno} package.  The two forms in which line numbers switch
% sides in prose contexts can't at the moment be used, because of a
% clash between how \textsf{lineno} handles line numbers that switch
% sides and the way I pass the line numbers to notes sections.  If you
% have set the verse line numbers to switch sides, prose line numbers
% will appear on the right side.  I hope soon to find a workaround for
% this problem.
%
% By default, the line modulo for prose sections is the same as that for 
% verse sections; if you have a line number every 5 lines in verse, 
% you will have the same in prose. You can set the prose line modulo 
% independently with |\setprosemodulo|.
% 
% \textsf{poemscol} may deal with rather long stretches of prose, but
% it is not designed to be used for critical editions of extended
% prose texts, as \texttt{EDMAC}, \textsf{ledmac}, or \textsf{ednotes}
% are.  The method of entering line ranges into notes in prose
% contexts is something of a hack, and rather a worse one than the one
% employed in verse contexts, since it involves using two labels for
% each note, one to capture the line number of the beginning, the
% other to capture the line number of the end.  Other limitations of
% this method may show up with longer prose extracts.
% 
% \DescribeMacro{\prosesectiontitle} Prose sections may be titled with
% |\prosesectiontitle{}| which is essentially just |\poemtitle|, or
% |\prosesectiontitlenotitle|, which is essentially just
% |\poemtitlenotitle|.  (Remember that if you use the latter command,
% you must specify a ``dummy'' title to use in the note and contents
% sections.)  They will be formatted in the notes and contents
% sections as poems are.  Alternatively, you can use
% |\literalcontents| and |\literaltextnotes| to set such titles as you
% please.
% 
%\DescribeMacro{prosesection}
% Prose sections should be enclosed in a |prosesection| environment
% (between |\begin{prosesection}| and |\end{prosesection}|).  Entering
% this environment sets the modulo repeat for line numbering to the
% same value that prevails in verse contexts, and resets the prose
% line number to zero.  If for some reason you do not wish to reset
% the line number but continue on from where it left off at the end of
% the last prose section, use the |prosesectionnoreset| environment
% instead.
% 
% \DescribeMacro{\setprosemodulo}
% If you wish to set the modulo repeat for line numbering in prose to 
% some value different for that from verse contexts, issue the new 
% modulo as the argument to |\setprosemodulo{}|.
% 
% 
% It is possible, if a prose section is interpolated in the midst of
% verse, to number the prose and verse continuously.  To move from a
% verse passage to a prose passage, just before the |\end{poem}| at
% the end of the first verse passage issue |\saveverselinenumber| to
% save the number of the verse passage.  After you issue
% |\begin{prosesection}| issue
% |\resetlinenumber[\theverselineumberscratch]| to apply that number
% to the prose section.  To number continuously while moving from
% prose to verse is more tricky.  One way might be to mark a label at
% the end of the prose passage with |\proselinelabel{prose end label}|
% (or some other suitably unique label).  This label will record the
% line number of the last prose line.  Pass that value to
% |\restoreverselinenumber| by issuing
% |\setcounterfromref{verselinenumberscratch}{prose end label}|.  Then
% issue |\restoreverselinenumber| just after issuing the
% |\begin{poem}| that restarts the verse again.  I would recommend
% against numbering verse and prose continuously, however, since a
% verse line and a prose line are different categories of thing, the
% one being an element of prosody, the other being only an element of
% typography.
% 
%
% \DescribeMacro{\proselinelabel}
% |\proselinelabel| is simply |\linelabel| from \textsf{lineno}. I 
% have provided the new name to make it easy to distinguish from 
% |\poemlinelabel|. 
%
% 
%
% \DescribeMacro{\prosetextnote}
% \DescribeMacro{\prosemendation}
% \DescribeMacro{\proseexplanatory}
% Special versions of |\textnote|, |\emendation| and |\explanatory|
% are available for use in prose sections.  |\prosetextnote|,
% |\proseemendation|, and |\proseexplanatory| differ from the verse
% versions in that they each require an extra argument, a unique label
% to be used in locating the line number for use by the notes
% commands.  If |\maketextnotes|, |\makeemendations|, or
% |\makeexplanatorynotes| have been issued, the endnotes will be sent
% to the usual sections, where all the notes from that prose section
% will be grouped into a paragraph, with the entry headed by the title
% given by |\prosesectiontitle{}|, just as if that were the title of a
% poem.  |\textnotesatfoot|, |\emendationsatfoot|, and
% |\explanationsatfoot| will send the notes from prose contexts to the
% usual footnote levels.  Notes commands in prose contexts take three
% arguments.  As in verse sections, in prose sections textnotes,
% explanatory notes, and emendations have an optional first argument
% for recording line ranges.  As in verse sections, you can record
% these line ranges either by simply putting in the argument how many
% lines the lemma will refer to (a deprecated method retained for
% backward compatibility reasons), or the argument of a
% |\proselinelabel| command on the last line of the lemma.  (The only
% difference from the verse version here is that you should make the
% label using the |\proselinelabel| command, rather than with the
% |\poemlinelabel| command.)  The second argument is a label to
% identify each note uniquely, which is used to find the line number 
% at which the note begins.  You will need a different label for
% every note.  The third argument is the text of the note, which
% should include both the lemma and the rest of the note, as in the 
% notes for verse sections.
% |\proseemendation[]{}{}| and
% |\proseexplanatory[]{}{}| are coded the same way. So, if you wished 
% to make a textual note on a passage that begins with ``Call me 
% Ishmael.'' and ends ``a way of driving away the spleen,'' you would 
% issue at the point where the note is to 
% occur (at the beginning of the passage):
% \begin{verbatim}
% \prosetextnote[a way of driving away the spleen] {Call me Ishmael}
% {Call me Ishmael \ldots\ a way of driving away the spleen] text of
% textual note} \end{verbatim} On the
% last line of the passage to which the note refers you would 
% issue: \\ |\proselinelabel{a way of driving away the spleen}|.
%
% \DescribeMacro{\prosetsvariant}
% \DescribeMacro{\proseaccidental}
% \DescribeMacro{\prosetsaccidental}
% You can issue |\sources| and |\tsentry| in prose contexts as usual. 
% For typescript variants and accidental variants you need to issue 
% |\prosetsvariant[]{}{}|, |\proseaccidental{}{}|, and 
% |\prosetsaccidental{}{}|. |\prosetsvariant| is entered just as 
% |\prosetextnote[]{}{}| is, with an optional argument (for use with 
% line ranges, as usual), and two mandatory arguments (the first a 
% label to capture the line number, the second the lemma and comment).
%
% |\proseaccidental| and |\prosetsaccidental| take only the two
% mandatory arguments, since by definition an accidental variant is
% confined to one line.
% 
% One caveat: because the line numbers for notes in prose contexts
% are generated by a label/ref mechanism, each note will result in a
% complaint about an undefined reference the first time the file is run through
% \LaTeX.  Either run the first pass in quiet mode, or be prepared to
% ignore a lot of griping until
% \LaTeX\ has generated the .aux file at the end of the first run.
%
% \section{Paragraph and Sentence Annotation}\label{paragraphsentence}
% \subsection{~Setup}
% Annotating a prose section by sentence number, or by paragraph number and 
% sentence number, has some advantages over annotating it by line 
% number. The principal advantage is that the annotation and reference 
% system can remain stable over multiple editions, even if they are set 
% in different formats. The most familiar of such systems is the 
% chapter and verse system in the Bible, introduced by Archbishop 
% Stephen Langton in the thirteenth century (who divided the Vulgate 
% edition into numbered chapters) and by Robert Estienne in the 
% sixteenth century (who added the verse numbers). Such a form of 
% annotation might be especially useful for medieval prosimetra, a form 
% combining verse and prose. (Hence the commands that follow often 
% begin with |pm|.)
% 
% \DescribeMacro{pmsection}
% You should put prose marked up for sentence and paragraph 
% annotation in a |pmsection| environment.
% You do not need to put prose marked up for sentence and paragraph 
% annotation in a |prosesection| environment. That environment is 
% only needed if you are line-numbering prose. You also do not need 
% to load the \textsf{lineno} package if you are using the sentence 
% and paragraph system for annotating prose. You might, however, 
% still wish to set the title with |\prosesectiontitle|.
% 
% \DescribeMacro{\setprosebysentence} 
% |\setprosebysentence| sets \textsf{poemscol} to number prose by
% sentence rather than by line number.  This command is mostly a
% placeholder now, used only for setting the value of |\marginparsep|
% to something better for prose.  (|\marginparsep| sets how far the
% line numbers are from the text block.)  |\verselinenumberstoleft|,
% |\verselinenumberstoright|, and |\verselinenumbersswitch| each pick
% convenient values for |\marginparsep|.  By default these values all
% line up on the page, but if you change any of them, you may have to
% use reset |\marginparsep| to some value you will have to find by
% trial and error, since they each use |\marginparsep| in slightly
% different ways.  You can change |\marginparsep| either by using
% |\setlength{\marginparsep}{length}| or, better, by using
% |\renewcommand| to change the setting of |\marginparsep| in
% |\verselinenumberstoleft| or whatever else you intend to use.
%
% \DescribeMacro{\pmsentence} Each sentence must begin with
% |\pmsentence|.  |\pmsentence| increments the sentence counter,
% optionally puts the sentence number in the running text, and
% optionally puts the sentence number in the margin.  
%
% |\runningsentencenumberstrue|
% sets sentence
% numbers in the running text.
% To set sentence numbers in the margin, issue
% |\marginsentencenumberstrue|. You can issue both, either, or neither of 
% these commands. 
%
% \DescribeMacro{\setpmmodulo} If you are putting sentence numbers in
% the margin, you need not number every sentence.  Indeed, if there is
% more than one sentence on a line, the numbers will overwrite each
% other (unless you use the marn.sty package, and substitute |\marn|
% for |\pmclrightsidepar|, |\pmclleftsidepar| or |\pmclsidepar| in the
% code for |\pmsentence|).  You can also avoid overlapping sentence
% numbers by using |\pmsentencetwo| and its siblings, defined below.
% You can set the modulo for sentence marginal numbers by issuing
% |\setpmmodulo{}|.
%
% \DescribeMacro{\pmsentencetwo}
% \DescribeMacro{\pmsentencethree}
% \DescribeMacro{\pmsentencefour} If you know, from a prior run, 
% that there will be a second or third sentence on some particular 
% line, marking them with these commmands will ensure that the 
% marginal sentence numbers do not overwrite each other. These macros 
% are something of a kludge, but they will do until I figure out how 
% to do this better.
%
% \DescribeMacro{\pmnumberstoright}
% \DescribeMacro{\pmnumberstoleft}
% \DescribeMacro{\pmnumbersgutter}
% \DescribeMacro{\pmnumbersouter}
% If you wish to control the
% margin in which sentence and paragraph numbers appear, issue
% |\pmnumbersright|, |\pmnumbersleft|, |\pmnumbersgutter|, or
% |\pmnumbersouter|.  By default, these numbers will line up with the
% line numbers in verse sections.  If you change the value of
% |\marginparsep| for prose sections from those used by their verse
% equivalents, it is a good idea to issue the verse equivalent command
% (|\versenumberstoright| or whatever) again when you move from a
% prose to a verse environment.
% 
%
% |\runningsentencenumberformat| sets the format of running sentence
% numbers.  Currently it is |\textsuperscript| and |\tiny|.  You can
% change this with |\renewcommand|.  The format of marginal sentence
% numbers is set by |\marginsentencenumberformat|.  Currently it is
% |\tiny|.  You can change this with |\renewcommmand|.
% 
% \DescribeMacro{\pmpara}
% \DescribeMacro{\pmsentencebypara}
% \DescribeMacro{\pmsparainmartrue}
% \DescribeMacro{\pmpararunningtrue}
% You can also number paragraphs.  Each paragraph must begin with
% |\pmpara|.  |\pmpara| increments the paragraph counter, optionally
% puts the paragraph number in the running text, and optionally puts
% the paragraph number in the margin.  To reset the sentence counter
% at the beginning of each paragraph, issue |\pmsentencebyparatrue| in
% your preamble.  To set paragraph numbers in the margin, issue
% |\pmparainmartrue|.  To set paragraph numbers in the running text,
% issue |\pmpararunningtrue|. You can issue both, either, or neither of 
% these last two commands. Which margin these numbers are set in
% is controlled by the same commands that control placement of
% marginal sentence numbers.
% 
% If you begin a paragraph with |\pmpara| \textsf{poemscol} will 
% assume that you intend to include the paragraph number in your 
% notes (in the form \emph{paragraph: sentence:} for simple 
% references, or \emph{firstparagraph: firstsentence: -- 
% lastparagraph: lastentence:} for ranges of sentences) even if you do not reset 
% the sentence number at the beginning of each paragraph. Normally 
% you would not issue |\pmpara| if you are not resetting the sentence 
% number at the beginning of each paragraph, since under those 
% circumstances you don't really need to keep track of the paragraph 
% numbers, in which case each note would simply include the sentence 
% number (just as in verse it would include the line number). In the 
% unusual event in which you wish to put the paragraph number in the 
% margin but refer only to sentence numbers in the notes, simply 
% follow every use of |\pmpara| with |\pmparasfalse|.
% 
% \DescribeMacro{\pmpararunningformat}\DescribeMacro{\pmparmarformat}
% The format of running paragraph numbers is set 
% by |\pmparrunningformat|. Currently it is 
% |\textbf| and |\small|.
% The format of marginal paragraph numbers is set by 
% |\pmparmarformat|. Currently it is 
% |\textbf| and |\small|. Either can 
% be changed with |\renewcommand|.
% 
% \subsection{~Cross Reference by Sentence and Paragraph}
% \DescribeMacro{\sentencelabel}
% To make labels that refer to sentences or paragraphs, issue
% |\sentencelabel{}|.  The label will be either both the paragraph
% number and the sentence number (if you are resetting sentence numbers
% by paragraph), or only the sentence number (if you are not resetting
% sentence numbers by paragraph).
% 
% \subsection{~End notes by Sentence and Paragraph}
% 
% \begin{sloppypar}
% \DescribeMacro{\pmtextnote}
% \DescribeMacro{\pmexplanatory}
% \DescribeMacro{\pmemendation}
% \DescribeMacro{\pmaccidental}
% \DescribeMacro{\pmtsvariant}
% \DescribeMacro{\pmtsaccidental}
% |\pmtextnote|, |\pmexplanatory|, |\pmemendation|, |\pmaccidental|,
% |\pmtsvariant| and |\pmtsaccidental| are special versions of the
% usual endnote commands for use with annotations by sentence and
% paragraph.  If you have not issued |\pmsentencebypara| these
% commands will write the sentence number at the head of each note,
% where the line number would usually go.  If you have issued
% |\pmsentencebypara|, these commands will issue both the paragraph
% and sentence number, in the format \textless paragraph
% number\textgreater:\textless sentence number\textgreater.  You can
% refer to ranges of sentences (or paragraphs and sentences) in a way
% very similar to how you do so in verse sections.  At the end of the
% range, issue a |\sentencelabel| command, with a unique label.  At
% the beginning of the range, issue your |\pmtextnote| command with the
% optional argument in square brackets before the main argument,
% giving your unique label as the value of that argument, e.g.
% |\pmtextnote[some label]{text of lemma] comment}|.  At the end of 
% your range, issue |\sentencelabel{some label}|, using the same 
% label you put in the square brackets for the |\pmtextnote|. If you have not
% issued |\pmsentencebypara|, the range will simply be the range of
% sentence numbers.  If you have issued |\pmsentencebypara|, the range
% will include paragraph and sentence numbers, like this:
% 4:12--6:13.\end{sloppypar}
% 
% |\sources| needs no special handling if you are annotating by 
% sentence and paragraph.
% 
% \subsection{~Footnotes by Sentence and Paragraph}
% |\textnoteatfoot| shifts |\pmtextnotes| to the foot, as with 
% |\prosetextnote|. |\emendationsatfoot| and |\explanationsatfoot| 
% work the same way.
% \subsection{~Bible Chapter and Verse}
% Since |\pmsentence| and |\pmpar| are the moral equivalents of bible 
% verse and chapter, |\biblechapter| and |\bibleverse| are provided 
% for convenience.

% \section{Parallel Text Editions} \label{parallelhack}
% \subsection{~Setup for Parallel Texts in Verse}
% \textsf{poemscol} can make parallel
% text editions in which the contrasting versions are set on facing
% verso and recto pages.  \textsf{poemscol} does not automatically provide the page
% breaks for parallel editions, and even if it did, one would still
% probably have to specify page breaks every two or three pages to keep
% the texts in register anyway, so you will have to use |\newpage|
% to mark every page break.  If you are willing to set the page breaks
% yourself, however, \textsf{poemscol} does provide macros for keeping the line count
% correct on successive recto or verso pages, and for keeping track of
% whether or not one was in the middle of a stanza when one closed the
% previous verso or recto page (as the case may be).  And it provides, in
% the |\synchrolabel| and |\synchroref| commands, a way of marking which
% lines on the recto side correspond to particular lines on the verso
% side.  |\synchroref| will even tag the marginal line number it makes
% with a marker to specify the text to which it refers.
% 
% \DescribeMacro{\startparalleltexts}
% \DescribeMacro{\finishparalleltexts}
% \DescribeEnv{parallelverse}
% You must begin a region of parallel text on a \emph{verso} page. 
%  A parallel text region must begin with 
% |\startparalleltexts| and end with |\finishparalleltexts|. 
% Setting parallel texts requires one to redefine the poem and stanza 
% environments, and |\startparalleltexts| redefines those environments.
% You can make those changes local to the parallel text region by using 
% |\begin{parallelverse}|  and |\end{parallelverse}| rather than |\startparalleltexts|
% and |\finishparalleltexts|.
% |\startparalleltexts| and |\finishparalleltexts| can be issued on their own
% but I have combined them to make the |parallelverse| environment. 
%
% Although usually with \textsf{poemscol} it is up to you where to put 
% the marginal line numbers, when you issue |\startparalleltexts| the 
% positions are fixed as follows: marginal line numbers are in the left 
% margin, line cross references (marking which lines in the recto  text 
% correspond to which lines in the verso text, which 
% are made with |\synchroref|) are in the right margin.
% 
%
% \DescribeMacro{\startrectopage}
% \DescribeMacro{\finishrectopage}
% \DescribeMacro{\startversopage}
% \DescribeMacro{\finishversopage}
% Verso pages 
% should begin with |\startversopage| and end with |\finishversopage|.
% Recto pages should begin with |\startrectopage| and end with 
% |\finishrectopage|.  You have to declare the page breaks manually, 
% so be sure you are putting the |\finishversopage|  or |\finishrectopage| commands 
% in places which 
% will work. |\finishparalleltexts| just makes certain that stanzas, and poems
% that you opened in a region of parallel texts are closed when you leave it.
% 
% \DescribeMacro{\versopoemtitle}
% \DescribeMacro{\rectopoemtitle}
% \DescribeMacro{\versopoemtitlenocontents}
% \DescribeMacro{\rectopoemtitlenocontents}
% Use |\versopoemtitle| and |\rectopoemtitle| 
% rather than |\poemtitle| for the title of the poem (or prose passage)
% on the verso or recto pages.
% if you are making a parallel text of Wordsworth's 1805 and 1850
% versions of \emph{The Prelude}, you might want to use something
% like |\versopoemtitle{The Prelude (1805 Version)}| and 
% |\rectopoemtitle{The Prelude{1850 Version)}|.
% I have also made versions of these commands that do not make
% entries in the table of contents, should the need arise for them. 
% If you wish to make titles for other things,
% such as poetic sequences or sequence sections, you can use the techniques
% described in section \ref{customizing titles} above, page \pageref{customizing titles}, 
% except substitute the macro |\versoset@p@emtitle| 
% or |\rectoset@p@emtitle| for |\set@p@emtitle|, using
% all the same arguments you would have used with |\set@p@emtitle|.
%
%\subsection{~Parallel Texts and on the fly note types}
% If you are making a parallel text edition, you may need to 
% use |\definenewnotetype| to make the different kinds of endnote 
% for the parallel texts. For instance, 
% if you are making a parallel text of Wordsworth's 1805 and 1850
% versions of \emph{The Prelude}, you may use something like |\fiveexplanatory|
% and |\fiftyexplanatory| for explanatory notes for the poems on the verso 
% and recto pages. 
%
%
%  \DescribeMacro{\keeptitlestraight}
% Without special care, \textsf{poemscol} may become
% confused about whether it has seen, for example, a |\fiveexplanatory| for 
% a particular book of \emph{The Prelude} or not, and, if |\settitleinnotes| is set, 
% may neglect to put the 
% title of that book of \emph{The Prelude} in the notes 
% for the first note for that book. To remedy this, |\versopoemtitle|
% and |\rectopoemtitle| set up a way of keeping track whether each  species 
% of note has appeared already or not. 
%  While setting parallel texts \textsf{poemscol} checks at the beginning of every
%  verso or recto page whether the the different species of endnote for the poem on
%  that page has appeared before. On rare occasions this turns out not to be frequently
% enough in cases
% where you are defining your own kinds of endnote. 
% In this case, the following hack, |\keeptitlesstraight|, will check whether the verso or recto
% poem has set that species of note each time it is issued. Suppose, for instance,
% going back to our Wordsworth example, 
% that you have defined, using |\definenewnotetype|, the following six
% kinds of endnotes: |\fivetextnote| |\fiveemendation|, |\fiveexplanatory|,
% |\fiftytextnote|, |\fiftyemendation|, and |\fiftyexplanatory|. You should
% define a new command called |\keeptitlesstraight| to keep track of these six
% kinds of note. Here is  how it might look:
%
% \medskip
%
% \noindent|\newcommand{\keeptitlesstraight}{|\\
% |\pretocmd{\fivetextnote}{\versotitleinnotescheck}{\relax}{\relax}|
% |\pretocmd{\fiveemendation}{\versotitleinnotescheck}{\relax}{\relax}|
% |\pretocmd{\fiveexplanatory}{\versotitleinnotescheck}{\relax}{\relax}|
% |\pretocmd{\fiftytextnote}{\rectotitleinnotescheck}{\relax}{\relax}|
% |\pretocmd{\fiftyemendation}{\rectotitleinnotescheck}{\relax}{\relax}|
% |\pretocmd{\fiftyexplanatory}{\rectotitleinnotescheck}{\relax}{\relax}|
% |}|
%
% \medskip
%
% All those fields filled with |\relax| are for reporting the success of failure
% |\pretocmd| has had in appending |\versotitleinnotescheck| (or |\rectotitleinnotescheck|)
% to the different endnotes commands.  You may wish to replace
% the |\relax| with something like |\typeout{fiftytextnote modified}| or 
% |\typeout{fiftytextnote not modified}|. 
%
%If you are using the |\keeptitlestraight|
% hack described here, you should add \textsf{etoolbox}
% to your package list.
%
%  \DescribeMacro{\resettitleinnotes}
% If you are defining new kinds of note, and a section of 
% non-parallel texts follow a section of parallel texts, you
% will need to issue the following hack after the parallel texts close
% (again I am using the Wordsworth example):\\
% |\newcommand{\resettitleinnotes}{|\\
% |\nofiveemendationyettrue%|\\
% |\nofiveexplanatoryyettrue%|\\
% |\nofivetextnoteyettrue%|\\
% |\nofiftyemendationyettrue%|\\
% |\nofiftyexplanatoryyettrue%|\\
% |\nofiftytextnoteyettrue%|\\
% |}|
%
% \subsection{~Parallel Texts and Generic Note types}
%  \DescribeMacro{\rectotexnote}
%  \DescribeMacro{\rectoemendation}
%  \DescribeMacro{\rectoexplanatory}
%  \DescribeMacro{\versotexnote}
%  \DescribeMacro{\versoemendation}
%  \DescribeMacro{\versoexplanatory}
% In case you find using these hacks a bit burdensome, 
%  I have defined for your convenience six generic kinds of endnote for use in parallel text
%  editions: |\rectotextnote|, |\rectoeemendation|, |\rectoexplanatory|,
%  |\versotextnote|, |\versoemendation|, and |\versoexplanatory,|  as well as |\rectoprosetexnote|,
% |\versoprosetextnote,| |\rectoproseemendation,| |\versoproseemendation|, 
% |\rectoproseexplanatory|, and |\versoproseexplanatory.| (I have also defined
% |\literalrectotextnote|, |\literalversotextnote|, |\literalrectoemendation|, |\literalversoemendation|,
% |\literalrectoexplain|, and |\literalversoexplain| for sending material other than formal
% endnotes to these endnote sections.
%  You can use all of these without issuing the moral equivalent of |\keeptitlesstraight| and 
% |\resettitleinnotes|.
%
%  \DescribeMacro{\makerectotexnote}
%  \DescribeMacro{\makerectoemendation}
%  \DescribeMacro{\makerectoexplanatory}
%To use these
% kind of note, issue |\makeversotextnotes|, |\makeversoemendations|, 
% |\makeversoexplanatorynotes| or their recto equivalents in your preamble. You can mark
% up your texts with these notes 
% just as you would do with  |\textnote|,  |\emendation| or |\explanatory| (or |\literalversotextnote|
% |\literalversoemend| and |\literalversoexplain|), without any of the special hacks
% I have just described.
%
% \DescribeMacro{\changerectotextnotesname}
% \DescribeMacro{\changerectotextnotesheader}
% \DescribeMacro{\rectotextnotessinglepar}
% \DescribeMacro{\rectotextnotessingletwocol}
% I have given the endnotes sections generated by these notes rather
% generic names (|\rectotextnote| and |\versotextnote| for instance, both open endnote sections
% with the title TEXTUAL NOTES, with a running header of the form ``Textual Notes to pp.~xx--yy''
% an a contents entry of Textual Notes).  You will probably need to change
% those to something more descriptive. You can change any of these with 
% |\changeversotextnotesname{Your Title}|, |\changeversotextnotesheader{Your running header}|,
% and |\changeversotextnotescontentsname{Your contents entry}|. 
% You can change the appearance of these notes
% by changing the values of the booleans |\versotextnotessinglepar| and |\versotextnotestwocol|.
% Similar sets of commands are available for |\versoemendation| and |\versoexplanatory|
% and their recto equivalents.
%
% \DescribeMacro{\putrectotextnotes}
% To include these note sections in your document, issue |\putversotextnotes|, |\putrectotextnotes|,
% |\putversoemendations|,  |\putrectoemendations|, |\putversoexplanatorynotes|, or 
% |\putrectoexplanatorynotes| at the point where you wish them to appear. These commands
% behave exactly like |\puttextnotes|.
%
% \subsection{~Parallel Text Labels}
% \DescribeMacro{\synchrolabel} \DescribeMacro{\synchroref}
% \DescribeMacro{\setsynchroflag} You may want to mark cross references
% to make it clear to the reader which line of the poem on the recto side
% of the spread corresponds to a particular line on the verso side.
% To mark these cross references, you
% need to make a label on the verso side (using |\syncrolabel|), and a corresponding
% reference on the recto side (using |\synchroref|).  You may also wish to add a mark to
% that reference (using |\setsynchroflag|) to specify to the reader what edition it refers to.  So, for
% instance, if your verso text is the 1805 version of Wordsworth's
% \emph{The Prelude}, and your recto text is the 1850 version, you may
% wish to specify |\setsynchroflag{1805: }|, which will preface the
% references put out by the |\synchroref| command with~`1805:~'.
% Suppose you wished to tie these corresponding passages from the ``Boy of Winander''
% episode in Book 3.   To make the label, issue the
% |\synchrolabel| command on the verso side (the 1805 side) just after the
% |\verseline| or |\end{stanza}| which closes the line you wish to
% label:
% 
% \begin{verbatim}
%This boy was taken from his mates, and died \verseline
%In childhood ere he was full ten years old \verseline
% \synchrolabel{full ten years old}
% \end{verbatim}
% You don't have to quote the whole line in the label, but it should be 
% long enough to uniquely pick out that line in the poem. (If the line 
% is repeated in the poem, you will have to exercise your ingenuity in 
% making up a unique label.)
% 
% Over on the recto side (the 1850 side), you will want to use |\synchroref| in order 
% to print the marginal cross reference. The |\synchroref| command 
% should come \emph{before} the |\verseline| that closes the line it is 
% tied to, as follows:
% 
% \begin{verbatim}
%This Boy was taken from his mates, and died\verseline
%In childhood, ere he was full twelve years old.
% \synchroref{full ten years old}}\verseline
% \end{verbatim}
%
% \subsection{~Parallel Prose Passages}
% \DescribeMacro{parallelprose}
% \DescribeMacro{\startparalleltextprose}
% \DescribeMacro{\finishparalleltextprose}
% \textsf{poemscol} is not designed to manage long parallel texts in prose,
% but for shorter texts up to a couple of dozen pages, such as 
% two versions of an Author's Preface, say, \textsf{poemscol} has the |parallelprose|
% environment. (You can also use |\startparalleltextprose| and |\finishparalleltextprose|.)
% Like passages of parallel text in verse, parallel prose passages must start
% on a verso page.
%
% \DescribeMacro{\startversoprosepage}
% \DescribeMacro{\finishversoprosepage}
% \DescribeMacro{\startrectoprosepage}
% \DescribeMacro{\finishrectoprosepage}
% Each verso page must start with |\startversoprosepage| and finish with 
% |\finishversoprosepage|, and each recto page must start with |\startrectoprosepage|
% and finish with |\finishrectoprosepage|. If a page continues a paragraph from the previous
% page, say, from a previous recto page, put the beginning of the 
% text on the second page right after the |\startrectopage| command with 
% no blank line. (Alternatively, you could just
% begin that page with |\noindent|.) 
%
% \DescribeMacro{\parastart}
% \DescribeMacro{\paraend}
% You must mark the beginning and end of every paragraph in parallel prose passages
% with |\parastart| and |\paraend|.
%
% You will have to pick the page breaks manually, using |\newpage|, once you find where
% \LaTeX\ would naturally break the page. Because how \LaTeX\ justifies text is a little
% unpredictable, however, you may find that the line before the page break, which filled 
% the line before you added the page break, is actually a little shorter, and thus is 
% no longer justified, once you add the page break. 
% To fix this, add an |\hfil| between each word of the last line.
%
% \section{Using \textsf{poemscol} in a Multilingual World}
% 
% \begin{sloppypar}
% \DescribeMacro{\changecontentsname}
% By default, \textsf{poemscol} uses English language titles for such
% things as the Table of Contents, the Explanatory Notes, and so on,
% and uses English language running headers.  As of version 2.5 these
% are easily changed.  To change the name of the Table of Contents
% (the default is CONTENTS) to Inhaltsverzeichnis, issue
% |\changecontentsname{Inhaltsverzeichnis}|.  To change the running
% header on contents pages, issue 
% |\changecontentsheader{your header}|.\end{sloppypar}
% 
% \begin{sloppypar}
% \DescribeMacro{\changenotesname} To change the name for the endnote 
% section for textual notes (the default is TEXTUAL NOTES), issue 
% |\changenotesname{your version}|. To change the running header (by 
% default ``Textual Notes to'') issue 
% |\changenotesheader{your version}|.
% To change the name for 
% the textual notes in the table of contents issue 
% |\changetextnotescontentsname{your version}|.
% \DescribeMacro{\changesinglepageabbrev}
% \DescribeMacro{\changemultiplepageabbrev}
% Running headers refer to single pages with the English 
% abbreviation ``p.'' and to multiple pages with the English 
% abbreviation ``pp.'' To change these issue 
% |\changesinglepageabbrev{new abbrev}| or 
% |\changemultiplepageabbrev{new abbrev}| respectively. The ultimate 
% outcome is to translate a header of the form ``Textual Notes to 
% pp.~xx-yy'' to the form appropriate for your language.\end{sloppypar}
% 
% \begin{sloppypar}
% \DescribeMacro{\changeemendationsname} To change the name for the endnote 
% section for emendations (the default is EMENDATIONS), issue 
% |\changeemendationsname{your version}|. To change the running header (by 
% default ``Emendations to'') issue 
% |\changeemendationsheader{your version}|. To change the name for 
% the emendations in the table of contents issue 
% |\changeemendationscontentsname{your version}.|\end{sloppypar}
%
%\begin{sloppypar}
% \DescribeMacro{\changeexplanationsname} To change the name for the endnote 
% section for explanatory notes (the default is EXPLANATORY NOTES), issue 
% |\changeexplanationsname{your version}|. To change the running header (by 
% default ``Explanatory Notes to'') issue 
% |\changeexplanationsheader{your version}|.
% To change the name for 
% the explanatory notes in the table of contents issue 
% |\changeexplanationscontentsname{your version}|.\end{sloppypar}
%
% \begin{sloppypar} \DescribeMacro{\changepoemindexname} To change the name
% for the index (the default is ``Index of Titles and First Lines'', issue
% |\changepoemindexname{your version}|.  To change the running
% header (the default is ``Index of Titles and First Lines'') issue
% |\changepoemindexheader{your version}|. To change the contents 
% entry for the index (the default is ``Index of Titles and First Lines'') 
% issue 
%  |\changepoemindexcontentsname{your version}|.\end{sloppypar}
%
%\section{A sample driver file}
% 
% If you are compiling a collected poems, you are probably preparing
% your volume from many files.  You may (or may not) wish to subdivide
% your edition by the original volumes.  Or you may subdivide by year,
% or by the significant phases of the poet's career.  One way to keep
% in control of the process is to create a driver file, which calls
% all the appropriate\ .sty files and sets the variables globally, and
% then read in the sections of the edition using |\input|.  Here is a
% version of the driver file I used for
% \emph{The Collected Poems of Robert Penn Warren}:
% \begin{verbatim}
% \documentclass[10pt,twoside]{article}
% \usepackage{fancyhdr,kdgreek,makeidx,times,multicol,geometry,}
% \usepackage{keyval,ifthen,mparhack,manyfoot,poemscol}
% \begin{document}
% \leftheader{The Collected Poems of Robert Penn Warren}
% \makeexplanatorynotes
% \makeemendations 
% \maketextnotes
% \makepoemcontents
% \makeverselinenumbers
% \global\indexingontrue
% \global\includeaccidentalstrue
% \global\includetypescriptstrue
% \input maintitle.tex
% \input poems1922_1943.tex
% \input thirtysix.tex
% \input eleven.tex
% \input sp43.tex
% \input promises1.tex
% \input promises2.tex
% \input promises3.tex
% \input yeo1.tex
% \input yeo2.tex
% \input yeo3.tex
% \input tale1.tex
% \input tale2.tex
% \input incarnations.tex
% \input audubon.tex
% \input orelse1.tex
% \input orelse2.tex
% \input arcturus.tex
% \input nowthen.tex
% \input beinghere.tex
% \input rumor1.tex
% \input rumor2.tex
% \input cj1.tex
% \input cj2.tex
% \input altitudes.tex
% \input poems1943_1989.tex
% \input introduction.tex
% \finish
% \end{verbatim}
% 
% \section{A sample poem markup}
% Here is a poem Robert Penn Warren wrote in his first volume, 
% \emph{Thirty-Six Poems} (1935). The poem is the first poem of his 
% sequence ``Kentucky Mountain Farm.'' I include the over-title, to 
% demonstrate how to mark up such cases. 
% \begin{verbatim}
% 	\sequencetitle{Kentucky Mountain Farm}
% \index{Kentucky Mountain Farm@{\em Kentucky Mountain Farm\/}}
% \sources{Text: TSP. Variants: SP43, SP66 (Deletes ``The Cardinal,'' ``The
% Jay,'' and ``Watershed''), SP75 (Same sections as SP66), SP85 (Restores
% ``Watershed''), {\em Helsinki\/} (includes only ``Rebuke of 
% the Rocks'' and 
% ``At the Hour of the Breaking of the Rocks''). 
% ``The Owl'' (above) was marked as a section of ``Kentucky
% Mountain Farm'' when it first appeared in {\em Poetry,\/} but it was never
% included in any book version of the entire sequence. The sequence in {\em
% Poetry\/} included, in this order, ``The Owl,'' ``The Cardinal,'' and
% ``Watershed.'' TSP uses lower case Roman numerals in the section 
% titles. The typescript drafts in the Beinecke Library do not seem to 
% be setting copies.}
% \sequencefirstsectiontitle{I. Rebuke of the Rocks}
% \index{Rebuke of the Rocks@{\em Rebuke of the Rocks\/}}
% \sources{Text: TSP. Variants: {\em Nation\/}, 11 Jan.\ 1928, 
% p.~47, {\em Literary
% Digest,\/} 28 Jan.\ 1928, p.~32, {\em Vanderbilt Masquerader,\/} 10
% (Dec.\ 1933), p.~16, SP43, SP66, SP75, SP85, {\em Helsinki\/}, 
% Broadside: The Press at Colorado College, printed on paper handmade by 
% Thomas Leech for the American Poetry Society, April~26, 1985. This 
% poem was not included in SP85 until the second set of galleys, in 
% which a photocopy of the SP75 text is a stapled insert.} 
% \begin{poem}
% \begin{stanza}
% Now on you is the hungry equinox,\verseline
% \index{Now on you is the hungry equinox}
% O little stubborn people of the hill,\verseline
% \accidental{hill,] \sameword--- {\em Nation,\/} {\em Literary Digest\/}
% \sameword, {\em Vanderbilt\/} (I include the reading 
% from {\em Vanderbilt\/}
% even though it is the same as in TSP, because 
% {\em Vanderbilt\/} was published
% after the other magazine versions but before TSP.)}
% The season of the obscene moon whose pull\verseline
% Disturbs the sod, the rabbit, the lank fox,\verseline
% Moving the waters, the boar's dull blood,\verseline
% And the acrid sap of the ironwood.\end{stanza}
% 
% \begin{stanza}
% But breed no tender thing among the rocks.\verseline
% Rocks are too old under the mad moon,\verseline
% \accidental{old\missingpunct] \sameword, {\em Vanderbilt\/}}
% Renouncing passion by the strength that locks\verseline
% The eternal agony of fire in stone.\end{stanza}
% 
% \begin{stanza}
% Then quit yourselves as stone and cease\verseline
% \accidental{stone\missingpunct] \sameword, {\em Vanderbilt\/}}
% To break the weary stubble-field for seed;\verseline
% Let not the naked cattle bear increase,\verseline
% Let barley wither and the bright milkweed.\verseline
% \accidental{milkweed.] milk-weed. {\em Vanderbilt\/}}
% Instruct the heart, lean men, of a rocky place\verseline
% That even the little flesh and fevered bone\verseline
% May keep the sweet sterility of stone.\end{stanza}
% \end{poem}
% \end{verbatim}

% \goodbreak
% \section{A sample divider page}
% \begin{verbatim}
% \begin{volumetitlepage}
% 
% \volumetitle{Being Here}
% \volumeheader{Being Here}
% \volumesubtitle{Poetry 1977--1980}
% \volumededication{To Gabriel Thomas Penn\\
% (1836--1920)}
% \bigskip
% 
% \volumeepigraph{OLD MAN: You get old and you can't do anybody
% any good any more.\\
% BOY: You do me some good, Grandpa. You
% tell me things.}
% 
% 
% \bigskip
% 
% \volumeepigraph{There is in short no absolute time standard.}
% \volumeattribution{{\em Van Nostrand's Scientific Encyclopedia,\/}\\
% Fifth Edition, p. 2203}
% 
% \bigskip
% 
% 
% \volumeepigraph{I thirst to know the power and nature of Time\ldots.}
% \volumeattribution{St.\ Augustine: {\em Confessions,\/}\\
% Book {\em XI,\/} Chapter {\em XXIII\/}\\
% Translated by Albert C. Outler}
% 
% 
% \bigskip
% 
% \volumeepigraph{Time is the dimension in which God 
% strives to define His own Being.}
% 
% \sources{The typescript material in the Beinecke Library 
% indicates that this 
% book went through many drafts, and some rethinking about what to 
% include and how to order the poems. Until a late stage of 
% production---very late, since the typescript includes a designed 
% mock-up of the proposed title page---the title of this volume was {\em 
% Life is a Fable.\/} Indeed, as late as the time the book was set in 
% galleys it was called {\em Life is a Fable\/}, although the title was 
% corrected on the galleys. Like {\em Now and Then\/}, the typescript 
% was assembled from photocopies of re-typed copies of poems made after 
% their magazine appearance, with annotations giving the name of the 
% magazine in which they occurred. Unlike {\em Now and Then\/}, the 
% typescript seems to have been typed in one piece, and the annotations 
% added for the convenience of the permissions department at 
% Random House. The page numbers have been added---presumably 
% later---in blue pen. The copy is marked up in Warren's hand in black 
% pencil, and in another hand in red pencil. (Unless marked otherwise, 
% all of the mark-up described here is in black pencil.) \tsentry{Like most 
% Warren typescripts BHTS is single-spaced. The dedication to Gabriel 
% Thomas Penn and the dialogue between the old man and the boy appeared 
% on a separate page, and the three epigraphs appeared on another 
% separate page. BHTS gives 1838 as the year of Gabriel 
% Thomas Penn's 
% birth (the error is corrected on the galleys). In the quotation from 
% Augustine, the second word is corrected to ``thirst'' from ``yearn,'' 
% and the translator's name is added in pencil, not in Warren's 
% handwriting.} In one of Warren's own 
% copies at Western Kentucky University, he has corrected the date of 
% Gabriel Thomas Penn's birth to 1837 (copy 310). Some revisions in this book 
% may have come very late 
% indeed. A letter to Erskine of January 28, 1980, refers to changes 
% made in a bound manuscript that Warren kept after returning the 
% copy-edited typescripts. There are revisions not only on the galleys, 
% but also on the page proofs. There are even some corrections on the 
% ``blues'' for this volume, including some new copy on a card referred 
% to in a note on the cover of the ``blues'' but now lost. The Secker 
% and Warburg London edition is, except for the title page, printed from 
% the American plates. There is a copy of this book, marked up by 
% Warren with 
% proposed revisions for Stuart Wright's use on May 25, 1987, in the 
% Special Collections 
% Department of the Emory University Library. I quote Warren's 
% notes by permission. 
% Warren has also marked up, slightly, a copy of this book now in the 
% Robert Penn Warren Room of the Library of the Kentucky Museum at 
% Western Kentucky University. I quote that volume too by permission. 
% The setting typescript for the section of SP85 drawn from this volume 
% consisted of taped-in photocopies of passages from BH. Because the 
% revisions for SP85 are sometimes a vexed question, I have indicated 
% where the SP85 revisions are made on SP85TS.}
% 
% 
% \sources{Title Page] SP85 omits the dialogue between the old man and 
% the young boy. SP85 capitalizes ``time standard'' in the quotation 
% from Van Nostrand's Scientific Encyclopedia, and does not capitalize 
% ``Time'' in the quotation from Augustine.}
%  
% 
% 
% \end{volumetitlepage}
% \end{verbatim}
% 
% 
% 
% 
% \StopEventually{\PrintChanges\PrintIndex}
% \section{Implementation}
% \subsection{~Page geometry and crop marks}


%    \begin{macro}{\tightgeometry} There's no reason why you should use the
% page geometry I specify here. If you issue |\tightgeometry| in your 
% preamble, you will have the geometry I have used. If you issue 
% |\tightleading|  you will 
% redefine |\normalsize| and |\small| to use tighter leading. 
%  Normal
% 10 point \LaTeX\ uses 12 point leading.  Strictly to save money, I
% redefined |\normalsize| to use 11 point leading.  
% I use the \textsf{geometry} package,
% since it's simpler to figure out.  I give my page geometry here
% chiefly to remind the reader of some important variables that should
% be kept track of, such as marginparsep, which controls the
% separation of the marginal line numbers from the verse line,
% marginparwidth, which controls how wide the block for the marginal
% line numbers will be, headsep, which controls how far the text will
% be from the running headers, and footskip, which controls how far
% the last verse line will be from the marker which tells the reader
% that there is a stanza break at the end of the page (if there is
% one).  I've also set the odd side margin and the even side margin
% for two-sided printing.  Finally, I have set the default handling
% of page endings to |\raggedbottom|.  Verse should always be set with
% a ragged bottom margin, otherwise \LaTeX\ will stretch the spaces
% between stanzas to make the bottom perfectly flush.  When you enter
% prose sections, such as the editor's introduction, you should enter
% the command |\flushbottom| to make sure that the prose sections have
% flush page bottoms. 
%    \begin{macrocode}
 \newcommand{\tightgeometry}{\geometry{verbose, tmargin= 33pt,
 textwidth=318.21pt,textheight=550pt,
 marginparsep=7pt,marginparwidth=10pt,
 headheight= 15pt,
 headsep=19pt,
 footskip=15pt,
 lmargin=132pt}
\setlength{\oddsidemargin}{60pt}
\setlength{\evensidemargin}{65.41pt}}
\newcommand{\tightleading}{%
\makeatletter
\renewcommand\normalsize{%
   \@setfontsize\normalsize\@xpt\@xipt
   \abovedisplayskip 10\p@ \@plus2\p@ \@minus5\p@
   \abovedisplayshortskip \z@ \@plus3\p@
   \belowdisplayshortskip 6\p@ \@plus3\p@ \@minus3\p@
   \belowdisplayskip \abovedisplayskip
   \let\@listi\@listI}
\makeatother
\normalsize
\makeatletter
\renewcommand\small{%
   \@setfontsize\small\@ixpt{10}%
   \abovedisplayskip 8.5\p@ \@plus3\p@ \@minus4\p@
   \abovedisplayshortskip \z@ \@plus2\p@
   \belowdisplayshortskip 4\p@ \@plus2\p@ \@minus2\p@
   \def\@listi{\leftmargin\leftmargini
               \topsep 4\p@ \@plus2\p@ \@minus2\p@
               \parsep 2\p@ \@plus\p@ \@minus\p@
               \itemsep \parsep}%
   \belowdisplayskip \abovedisplayskip
}
\makeatother}


%    \end{macrocode}
%    \end{macro}
% 

% \subsubsection{Internal font size commands} You will probably never need 
% to use these font commands explicitly. I defined these for internal 
% use within \textsf{poemscol}, so that a command will always call the 
% same fonts.
% 
% As a way of making global changes to repeated features I have
% defined various special fonts for special purposes.  For instance,
% in a Collected Poems in which the poems are arranged by volume, with
% divider pages for the beginning of each volume, the first command
% sets the font for the volume title on the divider page, using the
% default font family, setting it in 18 point type on 23 point
% leading.  The second command, as the name of the command indicates,
% sets the font for the subtitle of the volume to be 14 point type on
% 18 point leading.  You can change these either by changing the
% values in your (renamed) style file, or by using |\renewcommand|. 
% For instance, you can change the font parameters for the font for
% setting poem titles from 14 on 18 to 12 on 14 by issuing
% |\renewcommand{\poemtitlefont}{\fontsize{12}{14}\selectfont}| in
% your preamble.  Notice that there are special fonts for titles in
% the back matter and in the table of contents as well.

%    \begin{macrocode}
\newcommand{\volumetitlefont}{\fontsize{18}{23}\selectfont}
\newcommand{\volumesubtitlefont}{\fontsize{14}{18}\selectfont}
\newcommand{\poemtitlefont}{\fontsize{14}{18}\selectfont}
\newcommand{\sequencetitlefont}{\fontsize{16}{20}\selectfont}
\newcommand{\subsectiontitlefont}{\fontsize{12}{14.5}\selectfont}
\newcommand{\backmatterheaderfont}{\fontsize{16}{20}\selectfont}
\newcommand{\volumetitlesink}{\leavevmode\vspace{23.5pt}}
\newcommand{\backmattersink}{\leavevmode\vspace{21.5pt}}
\newcommand{\backmatterafterheadersink}{\leavevmode\vspace{11pt}}
\newcommand{\backmattertextfont}{\small}
\newcommand{\backmatterintrofont}{\normalsize}
\newcommand{\backmattervolumefont}{\fontsize{12}{14.5}\selectfont}
\newcommand{\backmattervolumesubtitlefont}{\fontsize{11}{13}\selectfont}
\newcommand{\contentsvolumefont}{\fontsize{14}{20}\selectfont}
\newcommand{\contentsvolumesubtitlefont}{\fontsize{12}{14.5}\selectfont}
\newcommand{\contentssequencetitlefont}{\fontsize{10}{14}\selectfont}
\newcommand{\contentspoemtitlefont}{\fontsize{10}{14}\selectfont}
%    \end{macrocode}

% \subsection{~Miscellaneous dimensions and constants}
% \textsf{poemscol} sets relatively large penalties for widows and
% clubs, both in the main text, and in the notes sections.  There are
% also special penalties after volume titles in the notes sections,
% since one does not want to leave a volume title in the notes
% sections stranded at the bottom of a page.  (Setting penalties may
% not automatically solve all the problems, however.  You can fix an
% intractable widow problem in the contents or in the notes sections
% by issuing |\literalcontents{\pagebreak}| or
% |\literalemend{\pagebreak}| just before you issue the
% |\poemtitle{}| or |\volumetitle{}| that resulted in the problematic
% title line.)  \textsf{poemscol} also manipulates penalties for places like
% the space between a poem's title and its first line.  It defines
% special values for the white space before and after poems and for
% the amount of a stanza break.  |\contentsindentone|,
% |\contentsindenttwo|, and so on are amounts to indent different
% hierarchical levels, say of poetic sequences and subsequences, in
% the table of contents. 
%    \begin{macrocode}
\widowpenalty=300
\clubpenalty=300
\newcommand{\notespoemclubpenalty}{\penalty-300}
\newcommand{\notessequenceclubpenalty}{\penalty-1500}
\newcommand{\notesvolumetitlepenalty}{\penalty-1500}
%    \end{macrocode}
% Since the right margin of verse is ragged anyway, I allowed some 
% flexibility for slightly longer lines.
%    \begin{macrocode}
\hfuzz=2pt
%    \end{macrocode}
%    For explicitly declared indented lines:
%    \begin{macrocode}
\newcommand{\verseindent}{\hspace{2em}}
%    \end{macrocode}
%    For multicolumn notes:
%    \begin{macrocode}
\setlength{\columnsep}{15pt}
%    \end{macrocode}
%    A hierarchy of indentations for sequence section titles and 
% subsection titles.
%    \begin{macrocode}
\newdimen\titleindentamount \titleindentamount=1pc
\newdimen\titleindenttwoamount \titleindenttwoamount=2pc
\newdimen\titleindentthreeamount \titleindentthreeamount=3pc
\newcommand{\titleindent}{\hspace{\titleindentamount}}
\setlength{\parindent}{\titleindentamount}
%    \end{macrocode}
%    White space after poem titles, after poems, and after sequence 
% titles.
%    \begin{macrocode}
\newcommand{\afterpoemtitleskip}{\smallskip}
\newcommand{\afterpoemskip}{\bigskip}
\newcommand{\aftersequencetitleskip}{\medskip}
%    \end{macrocode}
%    A little extra white space between stanzas.
%    \begin{macrocode}
\newcommand{\stanzaskip}{\medskip}
%    \end{macrocode}
%    A hierarchy of indentations for the table of contents:
%    \begin{macrocode}
\newdimen\volumetitleshiftamount \volumetitleshiftamount=1pc
\newcommand{\volumetitleshift}{\hbox{\hspace{\volumetitleshiftamount}}}
\newdimen\voladditionalamount \voladditionalamount=2pc
\newdimen\contentsindentoneamount \contentsindentoneamount=24pt
\newcommand{\contentsindentone}{\hspace{\contentsindentoneamount}}
\newdimen\contentsindenttwoamount \contentsindenttwoamount=42pt
\newcommand{\contentsindenttwo}{\hspace{\contentsindenttwoamount}}
\newdimen\contentsindentthreeamount \contentsindentthreeamount=60pt
\newcommand{\contentsindentthree}{\hspace{\contentsindentthreeamount}}
\newdimen\contentsindentfouramount \contentsindentfouramount=72pt
\newcommand{\contentsindentfour}{\hspace{\contentsindentfouramount}}
\newdimen\contentsindentfiveamount \contentsindentfiveamount=80pt
\newcommand{\contentsindentfive}{\hspace{\contentsindentfiveamount}}
%    \end{macrocode}
%    Penalties to encourage page breaks before poem titles, before 
% multi-line poem titles, and before sequence titles.
%    \begin{macrocode}
\newcommand{\poemtitlepenalty}{\penalty-1000}
\newcommand{\multilinetitlepenalty}{\penalty-3000}
\newcommand{\sequencetitlepenalty}{\penalty-3000}
\newcommand{\multilinesequencepenalty}{\penalty-3500}
%    \end{macrocode}

% \subsection{~Verse} The verse environment here is taken from \LaTeX, 
% slightly modified to indent run over lines more. The idea is to 
% distinguish the automatic indentation of run over lines from the 
% explicitly declared indentation of indented lines. You can change 
% the amount of indentation of runover lines with |\runoverindent{}|.
%    \begin{macrocode}
\def\makeatletter{\catcode`\@=11}
\def\makeatother{\catcode`\@=12 }
\newcommand{\runoverindent}[1]{\global\edef\runoverindentvalue{#1}
}
\runoverindent{6em}
% pmclverse (runover indented  more than it is in the normal verse 
% environment)
% \newenvironment{pmclverse}{%
% \let\\=\@centercr%  
% \list{}{\itemsep\z@ \itemindent -\runoverindentvalue\listparindent \itemindent 
%           \rightmargin 1sp\leftmargin 1sp\advance\leftmargin 
%           \runoverindentvalue}\item[]}
% {\endlist}
\newenvironment{pmclverse}[1][\linewidth]{%
\let\\=\@centercr%
\list{}{\itemsep\z@ 
\itemindent -\runoverindentvalue
\listparindent \itemindent
         \ifdim #1 < \linewidth
            \rightmargin        \z@
            \setlength{\leftmargin}{\linewidth}%
            \addtolength{\leftmargin}{-#1}%
            \addtolength{\leftmargin}{-0.5\leftmargin}%
          \else
            \rightmargin        \leftmargin
          \fi
\advance\leftmargin\runoverindentvalue}
\item[]}
{\endlist}
\newif\ifinquotedverse
 \newenvironment{quotedverse}{%
 \inquotedversetrue
 \let\\=\@centercr%  
 \list{}{\itemsep\z@ \itemindent -\runoverindentvalue\listparindent \itemindent 
           \rightmargin 0sp\leftmargin 0sp\advance\leftmargin 
           \runoverindentvalue}\item[]}
 {\endlist\inquotedversefalse}
% rfoverse (flushes runover to right) (Just a placeholder now.)
% \newif\ifrfv
% \rfvfalse
% \newcommand{\rightflushrunoververse}{\global\rfvtrue}
% \newcommand{\indentrunoververse}{\global\rfvfalse}
% \newenvironment{rfoverse}{\begin{pmclverse}}{\end{pmclverse}}
% % verse environment with a switch
% \renewenvironment{verse}{\ifrfv\begin{rfoverse}\else\begin{pmclverse}\fi}
% {\ifrfv\end{rfoverse}\else\end{pmclverse}\fi}
%    \end{macrocode}

% \subsection{~Miscellaneous internal counters}
% |\verselinenumber| is the line number, obviously. |\printlineindex| 
% records how many lines have passed since the last marginal line 
% number was output. |\lineindexrepeat| governs how often the marginal 
% line numbers appear. |\lemmalines| and |\lemmaend| are for setting 
% number ranges for multiline lemmas.
%    \begin{macrocode}
\newcounter{verselinenumber}
\newcounter{printlineindex}
\newcounter{verselinenumberscratch}
\newcounter{printlineindexscratch}
\newcounter{lineindexrepeat} \setcounter{lineindexrepeat}{9}
\newcounter{lemmalines}
\setcounter{lemmalines}{0}
\newcounter{lemmaend}
\setcounter{lemmaend}{0}
%    \end{macrocode}
% This next counter is used to make labels for each poem for the contents and notes sections.
%    \begin{macrocode}
\newcounter{poemnumber}
\setcounter{poemnumber}{1}
%    \end{macrocode}
%    These next two are used to set the page numbers in the running 
% headers of the various notes sections, which are of the form 
% ``Textual notes to pp.~xx--yy.''
%    \begin{macrocode}
\newcounter{notepageholdertitle} \setcounter{notepageholdertitle}{1}
\newcounter{notepageholdernote}  \setcounter{notepageholdernote}{1}
%    \end{macrocode}
%    \subsection{~Miscellaneous token lists}
% Since the explanatory notes section and the emendations section should
% only open an entry for poems for which there actually are explanatory notes or
% emendations, the current are held in token lists in case they are needed.|\titlesofar| 
% and |\titleincrement| are used to concatenate multi-line titles.
%    \begin{macrocode}
\newtoks{\fulltitleholder}
\newtoks{\titlesofar}
\newtoks{\titleincrement}
%    \end{macrocode}    
%    \subsection{~Miscellaneous booleans}
%    Most of these have self-explanatory names.
%    \begin{macrocode}
\newif\ifinstanza
\newif\ifinpoem
\newif\ifpoemcontentson
\newif\iftextnoteson
\newif\ifverselinenumbers
\newif\ifexplanon
\newif\ifemendationson
 \global\verselinenumberstrue
\newif\ifindexingon
\global\indexingontrue
\ifindexingon
  \makeindex
\fi
\newif\ifincludeaccidentals
\global\includeaccidentalstrue
\newif\ifincludetypescripts
\global\includetypescriptstrue
\newif\ifredundantemendations
\global\redundantemendationsfalse
\newif\ifnoemendyet
\newif\ifnoexplainyet
\global\noemendyettrue
\global\noexplainyettrue
\newif{\ifmiddlecontentsline}
\middlecontentslinefalse
\newif{\iflastcontentsline}
\lastcontentslinefalse
\newif{\ifputpagenumberincontents}
\putpagenumberincontentstrue
\newif{\ifputpagenumberinnotes}
\putpagenumberinnotestrue
\newif{\ifsinglelinetitle}
\singlelinetitletrue
\newif{\iftitlefirstline}
\titlefirstlinefalse
\newif{\iftitlemiddleline}
\titlemiddlelinefalse
\newif{\iftitlelastline}
\titlelastlinefalse
\newif{\ifverserightflush}
\verserightflushfalse
\newif\ifrangelemma
\rangelemmafalse
\newcommand{\makeverselinenumbers}{\global\verselinenumberstrue}
\let\makelinenumbers=\makeverselinenumbers% for compatibility
\newif\iftextnotesatend
\textnotesatendtrue
\newif\ifemendationsatend
\emendationsatendtrue
\newif\ifexplanatend
\explanatendtrue
\newif\iftextfootnotespara
\textfootnotesparatrue
\newif\ifsourcesfootnotespara
\sourcesfootnotesparafalse
\newif\ifemendationfootnotespara
\emendationfootnotesparatrue
\newif\ifexplanfootnotespara
\explanfootnotesparafalse
%    \end{macrocode}

% \subsection{~Page styles} These define several special page styles. 
% To use them you must have the \textsf{fancyhdr} package installed. 
% First, the default running headers: the right side (|\volumeheader|)
% is the title of the volume in a multivolume edition, and the left
% (|\leftheader|) is the title of the general collection. 
%    \begin{macro}{\volumeheader} 
%    \begin{macrocode}
\newcommand{\volumeheader}[1]{\global\edef\volumeheadervalue{#1}
}
%    \end{macrocode}
%    \end{macro}
%    \begin{macro}{\leftheader}
%    \begin{macrocode}
\newcommand{\leftheader}[1]{\global\edef\leftheadervalue{#1}
}
\leftheader{\relax}
\volumeheader{\relax}
\newcommand{\rightheader}[1]{\volumeheader{#1}}
%    \end{macrocode}
%    \end{macro}
% The default page style. Note the use of the mark mechanism to keep 
% track of cases where the stanza breaks and page breaks overlap.
%    \begin{macro}{\pagestyle{fancy}}
%    \begin{macrocode}
\newlength{\headoffsetlength}
\setlength{\headoffsetlength}{-1sp}
%\addtolength{\headoffsetlength}{\parindent}
\raggedbottom
\pagestyle{fancy}
\fancyhead{}
\fancyfoot{}
\fancyhfoffset[RO,LE]{\headoffsetlength}
\fancyhead[RO]{{\small\textrm{\thepage}}}
\fancyhead[LE]{{\small\textrm{\thepage}}}
\fancyhead[CO]{\hfil{\small{\em \volumeheadervalue}}}
\fancyhead[CE]{{\small{\em \leftheadervalue}}\hfil}
\fancyfoot[CE]{{\small \botmark}\hfil}
\fancyfoot[CO]{\hfil{\small \botmark}}
\renewcommand{\headrulewidth}{0pt}
\renewcommand{\footrulewidth}{0pt}
\fancypagestyle{main}{%
\fancyhead{}
\fancyfoot{}
\fancyhfoffset[RO,LE]{\headoffsetlength}
\fancyhead[RO]{{\small\textrm{\thepage}}}
\fancyhead[LE]{{\small\textrm{\thepage}}}
\fancyhead[CO]{\hfil{\small{\em \volumeheadervalue}}}
\fancyhead[CE]{{\small{\em \leftheadervalue}}\hfil}
\fancyfoot[CE]{{\small \botmark}\hfil}
\fancyfoot[CO]{\hfil{\small \botmark}}
\renewcommand{\headrulewidth}{0pt}
\renewcommand{\footrulewidth}{0pt}
}
%    \end{macrocode}
%    \end{macro}

% Style for the first page of every volume in a Collected Poems. You 
% don't have to call this style explicitly. It is called by the |\volumetitlepage| 
% environment. But of course if you want to change the style of those 
% pages, you should do it here.
%    \begin{macrocode}
\fancypagestyle{volumefirststyle}{%
\fancyhf{}
\fancyfoot[C]{\hfil{\small \botmark}\hfil}
\renewcommand{\headrulewidth}{0pt}
\renewcommand{\footrulewidth}{0pt}
}
%    \end{macrocode}
% Style for plain pages.
%    \begin{macrocode}
\fancypagestyle{plain}{%
\fancyhf{}
\fancyhead[RO,LE]{{\small\textrm{\thepage}}}
\renewcommand{\headrulewidth}{0pt}
\renewcommand{\footrulewidth}{0pt}
}
%    \end{macrocode}
% Style for divider pages between volumes
%    \begin{macrocode}
\fancypagestyle{volumetitlestyle}{%
\fancyhf{}
\renewcommand{\headrulewidth}{0pt}
\renewcommand{\footrulewidth}{0pt}
}
%    \end{macrocode}

%    \begin{macro}{\resetpagestyle}
% |\resetpagestyle| restores the fancy page style of the main sections of the
% book. |\putpoemcontents| changes the fancy page style, and |\resetpagestyle|,
% which is called by |\putpoemcontents| as it finishes, restores the original
% style
%    \begin{macrocode}
 \newcommand{\resetpagestyle}{\pagestyle{main}
% \fancyhead{}
% \fancyfoot{}
% \fancyhead[RO,LE]{{\small\textrm{\thepage}}}
% \fancyhead[CO]{\hfil{\small{\em \volumeheadervalue}}}
% \fancyhead[CE]{{\small{\em \leftheadervalue}\hfil}}
% \fancyfoot[CE]{{\small \botmark}\hfil}
% \fancyfoot[CO]{\hfil{\small \botmark}}
% \renewcommand{\headrulewidth}{0pt}
% \renewcommand{\footrulewidth}{0pt}
}
%    \end{macrocode}
%    \end{macro}
%    \begin{macro}{\clearemptydoublepage}
%    This macro is useful for making sure that the verso of a title
%    page is blank.
%    \begin{macrocode}
 \newcommand{\clearemptydoublepage}{\newpage{\pagestyle{empty}\cleardoublepage}}
% 
%    \end{macrocode}
%    \end{macro}
%    \begin{macro}{\mymarks}
% Here the marks mechanism keeps track of pages from which notes 
% come. Marks are used different in the notes sections and in the body 
% of the volume. In the body, they are used to keep track of cases 
% where a stanza break falls on a page break. In the notes sections, 
% they set the running headers.
%    \begin{macrocode}
\newtoks{\singlepageabbrev}
\singlepageabbrev={p.}
\newcommand{\changesinglepageabbrev}[1]{\singlepageabbrev={#1}}
\newtoks{\multiplepageabbrev}
\newcommand{\changemultiplepageabbrev}[1]{\multiplepageabbrev={#1}}
\multiplepageabbrev{pp.}
\newcommand{\mymarks}{%
   \ifthenelse{\equal{\firstmark}{\botmark}}%
        {\unskip ~\the\singlepageabbrev~\firstmark}% if equal
        {\unskip ~\the\multiplepageabbrev~\firstmark--\botmark}}%if not equal
\newcommand{\poemdotfill}{\dotfill}
%    \end{macrocode}
%    \end{macro}
%    \subsection{~Environments: poem, and stanza} Here are definitions 
% of the poem and stanza environments, and of the |\verseline|  and 
% |\stanzalinestraddle| macros.
%    \begin{macro}{stanza}
% Page breaks are slightly encouraged in stanza breaks.  Notice that
% entering a stanza sets the mark to |\relax| and leaving it sets the
% mark to * (the default value of |\stanzaatbottom|).  If you want to
% use some other symbol for this purpose, use |\stanzaatbottom{}| or
% |\nostanzaatbottom{}| (if you also want to mark cases where there
% is \emph{no} stanza break at the bottom of the page) to change it.
%    \begin{macrocode}
\newenvironment{stanza}{\penalty-100\instanzatrue\mark{\nostanzaatbottomvalue\ }}
 {\nobreak\mark{\stanzaatbottomvalue\ }\nobreak\ifspeciallinelock%
\relax\else%
\nolinebreak\incrementverselinenumber%
\fi%
\speciallinelockfalse%
\nobreak
  \instanzafalse

  \stanzaskip
}

\newenvironment{marginenvironment}{}{}
\newcommand{\stanzalinestraddle}
    {\begin{stanza}\addtocounter{verselinenumber}{-1}%
    \addtocounter{printlineindex}{-1}%
    }
%    \end{macrocode}
% The next two environments are for contents entries and titles. The 
% idea is to handle overrun titles the way overrun lines are handled 
% in verse. Contents entries indent the overrun slightly. Titles do 
% not indent the overrun relative to the first line, but preserve the 
% indentation of that line.
%    \begin{macrocode}   
\newlength{\contentsentryoverrun}
\setlength{\contentsentryoverrun}{\contentsindentoneamount}
\newenvironment{contentsentryenvironment}{%
\language=255%no hyphenation in contents
\let\\=\@centercr%  
\list{}{\topsep\z@%
\partopsep\z@%
\itemsep\z@ \itemindent -\contentsentryoverrun\listparindent \itemindent 
          \rightmargin 1sp\leftmargin 1sp\advance\leftmargin 
          \contentsentryoverrun}\item[]}{\endlist}
\newlength{\titleentryoverrun}
\setlength{\titleentryoverrun}{\z@}
\newenvironment{titleentryenvironment}{%
\language=255%no hyphenation in poem titles
\let\\=\@centercr%  
\list{}{\topsep\z@%
\partopsep\z@%
\itemsep\z@ \itemindent -\titleentryoverrun\listparindent \itemindent 
          \rightmargin 1sp\leftmargin 1sp\advance\leftmargin 
          \titleentryoverrun}\item[]}{\endlist}	  
%    \end{macrocode}
%    \end{macro}
% The poem environment.  The little trick about setting the language
% number is designed to suppress automatic hyphenation in the poem
% environment.  The booleans tell the |\emendation{}| and
% |\explanatory{}| macros that there have not yet been emendations or
% explanatory notes for this poem.  The little trick with |\mark| sets
% the value of |\mark| to the current value of |\stanzaatbottom| (by
% default *) if the current location is in a poem but not in a stanza
% (which is to say, in a stanza break).  The |\pagestyle| for poetry
% sets the value of |\mark| in the footer, which is * if the page break
% happens at a stanza break, and |\relax| if the page break happens
% anywhere else.
%    \begin{macro}{poem}
%    \begin{macrocode}
% \newenvironment{poem}{\raggedright%
%      \language=255%no hyphenation in verse
%      \noemendyettrue%
%      \noexplainyettrue%
%      \setcounter{verselinenumber}{0}\setcounter{printlineindex}{0}%
%      \nobreak\begin{verse}%   
%      \inpoemtrue\nobreak\mark{\relax}%
%      }{\end{verse}%
%      \inpoemfalse\mark{\relax}%
%      \goodbreak\afterpoemskip%\bigskip
%      }
\newlength{\versewidth}
\newenvironment{poem}[1][\linewidth]
{\raggedright%
     \language=255%no hyphenation in verse
     \noemendyettrue%
     \noexplainyettrue%
     \setcounter{verselinenumber}{0}\setcounter{printlineindex}{0}%
     \nobreak\begin{pmclverse}[#1]%
     \inpoemtrue\nobreak\mark{\relax}%
     }{\end{pmclverse}%
     \inpoemfalse\mark{\relax}%
     \goodbreak\afterpoemskip%\bigskip
     } 
%    \end{macrocode}
%    \end{macro}
%    \begin{macro}{rightflushverse}
% Sets verse with the runover portions of long lines flushed to the right. Each line
% must begin with |\rightversebegin|.This code was suggested to me by Mark Wooding
% on comp.text.tex, who describes the code as ``very unpleasant.''
%    \begin{macrocode}
\newenvironment{rightflushverse}{\let\oldleftskip=\leftskip%
\leftskip=1 \leftskip\advance\leftskip by 0pt plus 1fill}%
{\let\leftskip=\oldleftskip}
%    \end{macrocode}
%    \end{macro}
%    \subsection{~Environments: main title page, divider pages}
% 	 Environments for the main title page, and for the divider pages for 
% 	 individual volumes in a Collected Poems.
%    \begin{macro}{volumetitlepage}
%    \begin{macro}{maintitlepage}
%    \begin{macro}{booksectionpage}
%    \begin{macrocode}
\newenvironment{volumetitlepage}
{\par\clearpage{\pagestyle{volumetitlestyle}\cleardoublepage}
\thispagestyle{volumetitlestyle}}
    {\newpage\thispagestyle{volumetitlestyle}
	\cleardoublepage\thispagestyle{volumefirststyle}}
\newenvironment{maintitlepage}
{\par\cleardoublepage\thispagestyle{volumetitlestyle}
}{\newpage\thispagestyle{volumetitlestyle}\cleardoublepage}
\newenvironment{booksectionpage}
{\par\newpage\thispagestyle{volumetitlestyle}}{\newpage}
%    \end{macrocode}
%    \end{macro}
%    \end{macro}
%    \end{macro}
% \subsection{~Marginal line numbers, verse lines, line cross
% references, etc.}\label{marginalgeometry} The default is that the
% line numbers are on the right side, and that the |\marginparsep| is
% 18pt.  If you use |\pmclleftsidepar| the placement of the number is
% controlled by |\marginparsep| and |\marginparwidth|.  You should
% experiment with different values until you find a set that suits
% you.  A rule of thumb for me is that the |\marginparsep| if the
% numbers is on the right should be the negative of the value if it is
% on the left.  First, some lengths and ifs:
%    \begin{macrocode}
 \setlength{\marginparsep}{18pt}
 \setlength{\marginparwidth}{10pt}
\newif\ifnumbersswitch \numbersswitchfalse
\newif\ifnumbersright \numbersrighttrue
%    \end{macrocode}
%   Next, code for |\pmclsidepar|, taken from the memoir class, by 
% Peter Wilson. Modified by Dan Luecking.
%    \begin{macro}{\pmclsidepar}
%    \begin{macrocode}
\newcommand{\pmclsidepar}{\@dblarg{\@pmclsidepar}}
%\long\def\@pmclsidepar[#1]#2{\@bsphack\leavevmode\vadjust{%
\long\def\@pmclsidepar[#1]#2{\@bsphack\strut\vadjust{% added
\checkoddpage
  \ifpmclsideparswitch
    \ifpmclreversesidepar
      \ifoddpage
        \oddpagefalse
      \else
        \oddpagetrue
      \fi
    \fi
  \else
    \oddpagetrue
    \ifpmclreversesidepar
      \oddpagefalse
    \fi
  \fi
  \rlap{\kern-\parindent
    \if@twocolumn
      \if@firstcolumn     % put at left
        \kern -\marginparsep \kern -\marginparwidth % at left
      \else               % put at right
        \kern \columnwidth \kern \marginparsep % at right
      \fi
    \else
      \ifoddpage            % put at right
        \kern \textwidth \kern \marginparsep        % at right
      \else                 % put at left
        \kern -\marginparsep \kern -\marginparwidth % at left
      \fi
    \fi
%     \vbox to 0pt{%
%      \kern \pmclsideparvshift%           % vertical shift to align top text lines
%      \begin{minipage}{\marginparwidth}%
    \setbox0=\vtop to 0pt{% added
     \begin{minipage}[t]{\marginparwidth}% added
     \ifoddpage #2\else #1\fi%
      \end{minipage}%
%   \vss}}}\@esphack}
\vss}% added
 \vtop to 0pt{\kern\pmclsideparvshift % default should be 0pt 
  \kern-\dp\strutbox \kern-\ht0 \box0 \vss}}}\@esphack}% added 


%    \end{macrocode}
%    \end{macro}
%    \begin{macro}{\thepmemc@@page}
% The value of the page counter.
%    \begin{macrocode}
\gdef\thepmemc@@page{\the\c@page}

%    \end{macrocode}
%    \end{macro}
%    \begin{macro}{\pmemlabel}
% A version of |\label| that uses |\pmemprotected@write|.
%    \begin{macrocode}
\providecommand{\pmemlabel}[1]{\@bsphack
  \pmemprotected@write\@auxout{}%
    {\string\newpmemlabel{#1}{\thepmemc@@page}}%
  \@esphack}
%    \end{macrocode}
%    \end{macro}
%    \begin{macro}{\pmemprotected@write}
% |\pmemprotected@write| is a modified version of the kernel's
% |\protected@write|.
%
%    \begin{macrocode}
\long\def\pmemprotected@write#1#2#3{%
  \begingroup
  \let\thepmemc@@page\relax
  #2%
  \let\protect\@unexpandable@protect
  \edef\reserved@a{\write#1{#3}}%
  \reserved@a
  \endgroup
  \if@nobreak\ifvmode\nobreak\fi\fi}

%    \end{macrocode}
%    \end{macro}
%    \begin{macro}{\newpmemlabel}
%    \begin{macro}{\pmemlabelref}
% We have to be able to cope with a particular label
% not (yet) being in the the aux file when we come to check 
% the page number.
%    \begin{macrocode}
\providecommand{\newpmemlabel}[2]{{\global\@namedef{m@#1}{#2}}}
\providecommand{\pmemlabelref}[1]{%
  \expandafter\ifx\csname m@#1\endcsname\relax
    0%
  \else
    \csname m@#1\endcsname
  \fi}

%    \end{macrocode}
%    \end{macro}
%    \end{macro}
%    \begin{macro}{\@memcnta}
% We need a scratch count register.
%    \begin{macrocode}
\newcount\@memcnta
%    \end{macrocode}
%    \end{macro}
%    \begin{macro}{\checkoddpage}
% This is the user level command to check for odd/even page. It does
% a robust check for |\strictpagecheck| otherwise the simple minded
% check. It sets |\ifoddpage| to TRUE if the page is odd, otherwise
% it sets it to FALSE.
%
%    This is now fixed so that it should work for non-arabic page numbering.
% It uses a new label/pageref variant based on the page counter value, not
% its printed representation. This also gets rid of worrying about hyperref!!
% The problem was discovered by Bastiaan Niels Veelo
%
%    \begin{macrocode}
\DeclareRobustCommand{\checkoddpage}{%
  \oddpagefalse%
  \ifstrictpagecheck%
    \stepcounter{cp@cntr}\pmemlabel{\cplabel\thecp@cntr}%
    \@memcnta=\pmemlabelref{\cplabel\thecp@cntr}\relax
    \ifodd\@memcnta\oddpagetrue\fi
  \else
    \ifodd\c@page\oddpagetrue\fi
  \fi}

%    \end{macrocode}
%    \end{macro} 
%    \begin{macro}{\ifpmclreversesidepar}
%    \begin{macro}{\ifpmclsideparswitch}
% Analogues of |\marginpar| controls.
%    \begin{macrocode}
\newif\ifpmclreversesidepar
%  \pmclreversesideparfalse
  \pmclreversesidepartrue
\newif\ifpmclsideparswitch
  \pmclsideparswitchfalse
\if@twoside \pmclsideparswitchtrue \fi

%    \end{macrocode}
%    \end{macro}
%    \end{macro}
%    \begin{macro}{\ifoddpage}
%    \begin{macro}{\ifstrictpagecheck}
%    \begin{macro}{\c@cp@cntr}
%    \begin{macro}{\cplabel}
% The boolean |\ifoddpage| is TRUE if the checked page is odd. The
% boolean |\ifstrictpagecheck| is for turning on (TRUE) and off (FALSE)
% the strictest method of page checking. The counter |cp@cntr| is used
% to make unique labels, which start with |\cplabel|.
%    \begin{macrocode}
\newif\ifoddpage
\newif\ifstrictpagecheck
  \strictpagecheckfalse
\newcounter{cp@cntr}
\newcommand{\cplabel}{^_}
%    \end{macrocode}
%    \end{macro}
%    \end{macro}
%    \end{macro}
%    \end{macro}
%    Two-column and one-column printing is again realized via a switch
%    which is defined in the kernel. The default is single column printing.
%
%
%    \begin{macro}{\pmclsideparvshift}
% Vertical shift for sidepar to align with text line
%    \begin{macrocode}
\newlength{\pmclsideparvshift}
%  \setlength{\pmclsideparvshift}{-2.08ex}% seems to work for all font sizes
%\setlength{\pmclsideparvshift}{-2.58ex}
\setlength{\pmclsideparvshift}{0ex}
%    \end{macrocode}
%    \end{macro}
%   My code resumes here
%    \begin{macro}{\verselinenumberstoleft}
%    \begin{macro}{\verselinenumberstoright}
%    \begin{macrocode}
\newcommand{\verselinenumberstoleft}{%
   \setlength{\marginparsep}{18pt}%
   \setlength{\marginparwidth}{10pt}%
   \numbersswitchfalse\numbersrightfalse}
\newcommand{\verselinenumberstoright}{%
   \setlength{\marginparsep}{18pt}%
   \setlength{\marginparwidth}{10pt}%
   \numbersswitchfalse\numbersrighttrue}
%    \end{macrocode}
%    \end{macro}
%    \end{macro}
%    \begin{macro}{\verselinenumbersswitch}
%    \begin{macro}{\verselinenumbersgutter}
%    \begin{macro}{\verselinenumbersouter}
%    \begin{macrocode}
\newcommand{\verselinenumbersswitch}{%
   \strictpagechecktrue%
   \setlength{\marginparsep}{18pt}%
   \setlength{\marginparwidth}{10pt}%
   \numbersswitchtrue\numbersrightfalse\pmclsideparswitchtrue}
\newif\ifnumbersgutter \numbersgutterfalse%
\newcommand{\verselinenumbersgutter}{\verselinenumbersswitch%
\pmclreversesideparfalse%
\numbersguttertrue%
}
\newcommand{\verselinenumbersouter}{\verselinenumbersswitch%
\pmclreversesidepartrue% 
\numbersgutterfalse}
%    \end{macrocode}
%    \end{macro}
%    \end{macro}
%    \end{macro}
%    \begin{macro}{\putverselinenumber}
%    \begin{macrocode}
\newcommand{\putverselinenumber}{\nolinebreak\begin{marginenvironment}%
  \nolinebreak%
% \ifnumbersswitch\marginpar{\hfil\textrm{\scriptsize\theverselinenumber}}%
%     \else% 
%     \ifnumbersright\marginpar{\hfil\textrm{\scriptsize\theverselinenumber}}%
%       \else%
%       \marginpar{\textrm{\scriptsize\theverselinenumber}\hfil}%
%       \fi%
%     \fi              
  \ifnumbersswitch\pmclsidepar{\hfil\textrm{\scriptsize\theverselinenumber}}%
    \else% 
    \ifnumbersright\pmclrightsidepar{\hfil\textrm{\scriptsize\theverselinenumber}}%
      \else%
      \pmclleftsidepar
{\textrm{\scriptsize\theverselinenumber}\hfil}%
      \fi%
    \fi%
%  \marn{\hfil\textrm{\theverselinenumber}}
%  \marginpar{\hfil\textrm{\scriptsize\theverselinenumber}}
%   \linenumberside{\textrm{\scriptsize\theverselinenumber}\hfil}
% \pmclrightsidepar{\textrm{\scriptsize\theverselinenumber}}  
  \nolinebreak%
  \end{marginenvironment}%
  \nolinebreak%
}
%    \end{macrocode}
%    \end{macro}
%    \begin{macro}{\verseline}
%    \begin{macrocode}
\newif\ifspeciallinelock
\speciallinelockfalse
\newcommand{\verseline}{\ifspeciallinelock%
\relax\else%
\nolinebreak\incrementverselinenumber%
\fi%
\speciallinelockfalse%
\nolinebreak\\
}
\newcommand{\incrementverselinenumber}{%
\nolinebreak\addtocounter{verselinenumber}{1}%
  \addtocounter{printlineindex}{1}%
  \ifnum\theprintlineindex>\thelineindexrepeat%
    \ifverselinenumbers\nobreak%
      \putverselinenumber\nobreak%
    \fi\nobreak%
    \setcounter{printlineindex}{0}%
   \fi%
}
\newcommand{\setverselinemodulo}[1]{%
\setcounter{lineindexrepeat}{#1}%
\addtocounter{lineindexrepeat}{-1}%
}
%    \end{macrocode}
%    \end{macro} 
%    \begin{macro}{\linebend}
% A linebend is a a normal linebreak from \LaTeX, with no incrementing 
% of the line number counter, but the indentation makes the line look 
% like a run over line. If you change the value of the run over in the 
% definition of the |\verse| environment, be sure to change it here 
% too. |\linebend| should be used for tweaking how \textsf{poemscol} 
% runs over long lines.
%
%    \begin{macrocode}
\newcommand{\linebend}{\\
\hspace{\runoverindentvalue}%
 }
%    \end{macrocode}
%    \end{macro}
%  |\saveverselinenumber| and |\restoreverselinenumber| are useful for setting 
% prose interjections in the midst of poems.
%    \begin{macro}{\saveverselinenumber}
%    \begin{macro}{\restoreverselinenumber}
%    \begin{macrocode} 
\newcommand{\saveverselinenumber}{\setcounter{verselinenumberscratch}%
    {\value{verselinenumber}}%
   \setcounter{printlineindexscratch}{\value{printlineindex}}}
\newcommand{\restoreverselinenumber}{\setcounter{verselinenumber}%
     {\value{verselinenumberscratch}}%
    \setcounter{printlineindex}{\value{printlineindexscratch}}}
%    \end{macrocode}
% 	\end{macro}
% 	\end{macro}
%    \begin{macro}{\rightversebegin}
% By default, \textsf{poemscol} indents the runover portion of long lines by the
% amount |\runoverindent| (by default 6 em).  It is possible to set verse so that
% the runover portion is flushed to the right margin.  To do this, you should
% enclose the |\poem| environment in the |\rightflushverse| environment.  And you
% must mark the beginning of each verse line with |\rightversebegin|
%    \begin{macrocode}
\newcommand{\rightversebegin}
{\leavevmode\nobreak\hskip0pt plus -1fill\ignorespaces}
%    \end{macrocode}
%    \end{macro}
%    \begin{macro}{\hour}
% |\hour{}| is useful for typesetting the AM and the PM in the time.
%    \begin{macrocode}	
\newcommand{\hour}[1]{{{\textsc{#1}}\ $\!\!\!$}
 }
%    \end{macrocode}
%    \end{macro}
%  \begin{macro}{\brokenline}
%  \begin{macro}{\versephantom}
%  |\brokenline| and |\versephantom{}| are a pair, useful for setting 
%  cases where there is a line break in the midst of a single metrical 
%  line. |\tweakbrokenline| ensures that the line number will mark 
%  the first partial line if the line numbers are on the left, the 
%  last partial line if the line numbers are on the right. If you 
%  have issued |\tweakbrokenline| in your preamble, the setting of 
%  runover lines may not be consistent with the setting of broken 
%  lines, because the line number will be set on the level of the end 
%  of the line. This is not a problem if the line numbers are set on 
%  the right, but it may look odd if the numbers are set on the left.
%  |\startverseline| will ensure that the line number marks the 
%  beginning of any line it begins. If you put |\startverseline| at 
%  the beginning of the runover lines, it will ensure that the line 
%  number marks the beginning, not the end, of the line. If you wish 
%  |\startverseline| to align the number with the beginning of 
%  runover lines when the line number is on the left, and with the 
%  end of runover lines when the line number is on the right, issue 
%  |\tweakstartverseline| in your preamble.
%    \begin{macrocode}
\newcommand{\brokenline}{\\}
\newcommand{\versephantom}[1]{\leavevmode\phantom{#1}}
%    \end{macrocode}
%    \end{macro}
%    \end{macro}
%    \begin{macro}{\tweakbrokenline}
% Causes |\brokenline| to set the line number level with the 
% beginning of the line if the number is on the left, level with the 
% end of the line if the number is on the right. Issue it in the 
% preamble.
%    \begin{macrocode}
\newcommand{\tweakbrokenline}{%
  \renewcommand{\brokenline}{\ifnumbersswitch\relax%
  \ifnumbersgutter% shift on odd pages for gutter numbers
  \ifodd\c@page \relax \else \speciallinelocktrue\incrementverselinenumber\fi
  \else % shift on even pages for outer numbers
  \ifodd\c@page \speciallinelocktrue\incrementverselinenumber \fi
  \fi
     \else \ifnumbersright \relax%
       \else \speciallinelocktrue\incrementverselinenumber%
       \fi%
     \fi%
\\}
}
%    \end{macrocode}
%    \end{macro}
%    \begin{macro}{\brokenlineatbeginning}
% Issue |\brokenlineatbeginning| if you want broken line to set the 
% line number level with the beginning of the line, regardless of 
% whether the line number is at the right or at the left
%    \begin{macrocode}
\newcommand{\brokenlineatbeginning}{%
  \renewcommand{\brokenline}{%
  \speciallinelocktrue\incrementverselinenumber%
  \\ 
  }
}  
%    \end{macrocode}
%    \end{macro}
%    \begin{macro}{\startverseline}
% Put |\startverseline| at the beginning of every line where you wish 
% to ensure that the line number is level with the beginning of the 
% line. This is particularly useful for overrun lines when 
% |\tweakbrokenline| or |\brokenlineatbeginning| is in effect.
%    \begin{macrocode}
\newcommand{\startverseline}{%
\speciallinelocktrue%
\leavevmode%
%\kern-3pt
\incrementverselinenumber%
}
%    \end{macrocode}
%    \end{macro}
%    \begin{macro}{\tweakstartverseline}
% You may want |\startverseline| to align the line number with the 
% beginning of the line if the line number is on the left, and with 
% the end of the line if the line number is on the right. To do this, 
% issue |\tweakstartline| in your preamble.
%    \begin{macrocode}
\newcommand{\tweakstartverseline}{%
\renewcommand{\startverseline}{%
\leavevmode%
\ifnumbersswitch%
% gutter case: shift on odd pages to top of runover line
  \ifnumbersgutter%
      \ifodd\c@page \relax \else \speciallinelocktrue\incrementverselinenumber\fi
% outer case: shift on even pages to top of runover line
  \else
      \ifodd\c@page \speciallinelocktrue\incrementverselinenumber \fi
  \fi% end of ifnumbersgutter
\else \ifnumbersright \relax% numbers don't switch. Are they to the right?
       \else \speciallinelocktrue\incrementverselinenumber% or to the left?
       \fi% end of ifnumbersright
\fi% end of ifnumbersswitch
}% end of renewcommand
}% end of tweakverseline
%    \end{macrocode}
%    \end{macro}
%  \begin{macro}{\poemlinelabel}
%  |\poemlinelabel| is a hacked version of |\label| for crossreferences by 
%  line.
%    \begin{macrocode}
\def\poemlinelabel#1{\@bsphack
 \protected@write\@auxout{}%
	 {\string\newlabel{#1}{{\theverselinenumber}{\thepage}}}%
\@esphack
}
%    \end{macrocode}
%    \end{macro}
% \subsection{~Setup for contents, textual notes, emendations, and explanatory notes}
%  First, the table of contents:
%  \begin{macro}{\makepoemcontents}
%  An external file is created with the extension .ctn, and the 
% filename of your driver file. The 
% channel to that file is called |\poemcontents|. The rest of the 
% command is the page style for the contents section. You will notice 
% the running head Contents, and the title of the Contents section. I am 
% assuming that the table of contents starts on page 5, but it may be 
% different in different books. 
% If you want to change that page 
% number (say, to page 3) without messing with this file, issue 
% |\literalcontents{\setcounter{page}{3}}| just after you issue 
% |\makepoemcontents| Alternatively, you can issue |\makepoemcontents{}| with its
% optional argument, putting the page number in the argument.
%    \begin{macrocode}
\newtoks{\pmclcontentsname}
\pmclcontentsname={CONTENTS}
\newtoks{\pmclcontentsheader}
\pmclcontentsheader={Contents}
\newcommand{\changecontentsname}[1]{%
\pmclcontentsname={#1}}
\newcommand{\changecontentsheader}[1]{%
\pmclcontentsheader={#1}}  
\newcommand{\makepoemcontents}[1][5]{\global\poemcontentsontrue
  \newwrite\poemcontents
  \immediate\openout\poemcontents=\jobname.ctn
  \literalcontents{\flushbottom\normalfont
{\par\clearpage{\pagestyle{volumetitlestyle}\cleardoublepage}
\pagestyle{fancy}\thispagestyle{volumetitlestyle}}
\pagenumbering{roman}
\setcounter{page}{#1}
\fancyhead[RO,LE]{{\small\textrm{\thepage}}}
\fancyhead[CO]{{\small{\em \the\pmclcontentsheader}}}
\fancyhead[CE]{{\small{\em \the\pmclcontentsheader}}}
\fancyfoot{}
\backmattersink
\begin{center}{\normalfont \backmatterheaderfont
\the\pmclcontentsname}\end{center}
\lefthyphenmin=2\backmatterafterheadersink\tolerance=500\language=0}
\literalcontents{  \relax}}
%    \end{macrocode}
%    \end{macro}

% Next, the textual collations:
%    \begin{macro}{\maketextnotes} Like the |\makecontents| macro, 
% |\maketextnotes| opens an external file using |\jobname|, this time with the extension .ent. It 
% makes sure that line numbering is on (since it makes no sense to make 
% notes without numbering the lines, particularly since the 
% |\textnote{}| macro uses the line number). And it makes sure that the 
% notes section is set with a flush bottom line, rather than with the 
% ragged bottom line used for the verse sections. Also, it sets a 
% label for sending the page number of the textual notes to the table of 
% contents. 
%
% First, a few commands for use by all of the endnotes 
% sections. |\setendnotessectiontitledefaults| sets the title of the endnotes section
% at the top of the first page of that species of note, and takes six arguments,
% which are, in order, (1) the title  of the endnote section, (2) 
% commands to be issued before setting the
% title (minimally, a penalty, such as |\poemtitlepenalty|), (3) the
% font size in which the title is to be set in the text (such as
% |\poemtitlefont| or |\large|), (4) any indentation of the title
% (|\relax| to put the title where |\poemtitle| puts it, or |\titleindent| to put it where
% |\sequencesectiontitle| puts it), (5) commands to be issued after the
% title is set (such as |\afterpoemtitleskip|) and (6) the label to 
% be used for this section, to be used to set the page number in the 
% table of contents.
%
% |\setendnotessectiontitle| is a simpler version of |\setendnotessectiontitledefaults|
% with all the defaults filled in, leaving only two arguments, the 
% title of the section, and the label for use by the table of 
% contents.
%
% |\contentsendnotesdefaults| sets the defaults for the table of 
% contents entries for endnotes sections. It takes four arguments
% which are, in order, (1) the title of the endnote section, (2) the
% indentation of the title in the table of contents (such as
% |\contentsindentone|), (3), the font size in which the title
% should be set (such as |\contentspoemtitlefont|) and (4) the amount 
% overrun lines in the table of contents should be indented (such as 
% |\contentsindenttwoamount|).
%
% |\contentsendnotesindent| sets the default indent for table of 
% contents entries for endnotes sections, and |\contentsendnotesfont| 
% sets the default font for the.
%
% |\contentsendnotestitle| is a simpler command for sending 
% information about endnote sections to the table of contents, taking 
% only two arguments, the title and the label used for page reference 
% purposes.
%    \begin{macrocode}
% Generic command for titles of endnote sections
\newcommand{\setendnotessectiontitledefaults}[6]{%
#2%
#3 \begin{titleentryenvironment}%
{\ifcentertitleson\begin{center}#4 #1\end{center}\else #4 #1\fi}%
\end{titleentryenvironment}%
#5%
\label{#6}
}
% Simpler version with only two parameters, the title and the label 
% for page references
\newcommand{\setendnotessectiontitle}[2]{
\setendnotessectiontitledefaults{#1}{\poemtitlepenalty\backmattersink\begin{center}}{\backmatterheaderfont}
{\relax}{\end{center}\lefthyphenmin=2\backmatterafterheadersink}{#2}}
% Generic command for sending info about endnotes sections to table 
% of contents
\newcommand{\contentsendnotesdefaults}[4]{\ifpoemcontentson
\immediate\write\poemcontents{\string\begin\string{contentsentryenvironment\string}}%
\literalcontents{#2{#3#1}}%
  \ifputpagenumberincontents
\immediate\write\poemcontents{\string\contentsleaders}%
\immediate\write\poemcontents{\string\pageref
\string{#4\string}}%
  \else
  \literalcontents{\relax}%
  \fi
\immediate\write\poemcontents{\string\end\string{contentsentryenvironment\string}}%
\immediate\write\poemcontents{\string\par}%
\fi
}
% default styles for contents entries for endnotes sections
\newcommand{\contentsendnotesindent}{\contentsindentone}
\newcommand{\contentsendnotesfont}{\contentspoemtitlefont}
% simpler command to set up contents entry with only tw parameters, 
% the title and the label for page references
\newcommand{\contentsendnotestitle}[2]{%
\contentsendnotesdefaults{#1}{\contentsendnotesindent}{\contentsendnotesfont}{#2}%
}
% tokens to use to change the names of the sections
\newtoks{\pmclnotesname}
\pmclnotesname={TEXTUAL NOTES}
\newcommand{\changenotesname}[1]{\pmclnotesname={#1}}
\newtoks{\notesheadername}
\notesheadername={Textual Notes to}
\newcommand{\changenotesheader}[1]{\notesheadername={#1}}
\newtoks{\textnotescontentsname}
\textnotescontentsname={Textual Notes}
\newcommand{\changetextnotescontentsname}[1]{\textnotescontentsname={#1}}
% A boolean to change to two column notes.
\newif\iftextnotestwocol
\textnotestwocolfalse
% Setup for textual notes
\newcommand{\maketextnotes}{\global\textnotesontrue
  \global\verselinenumberstrue
  \newwrite\textnotes
  \immediate\openout\textnotes=\jobname.ent
  \literaltextnote{\flushbottom{\par\clearpage
  {\pagestyle{volumetitlestyle}\cleardoublepage}
\thispagestyle{volumetitlestyle}}
\fancyhead{}
\fancyfoot{}
\fancyhead[RO,LE]{{\small\textrm{\thepage}}}
\fancyhead[CO]{{\small{\em \the\notesheadername~\mymarks}}}
\fancyhead[CE]{{\small{\em \the\notesheadername~\mymarks}}}
\fancyfoot[C]{}
\mark{3}
\setendnotessectiontitle{\the\pmclnotesname}{textualnotes}
\tolerance=500\language=0
\normalfont \backmattertextfont
\iftextnotestwocol
\begin{multicols}{2}
\fi
\setlength{\parindent}{-10pt}
}
}

%    \end{macrocode}
%    \end{macro}

%    \begin{macro}{\makeemendations}
% Next, the emendations section:
%    \begin{macrocode}
% tokens for changing the names of these notes
\newtoks{\emendationsname}
\emendationsname={EMENDATIONS}
\newcommand{\changeemendationsname}[1]{\emendationsname={#1}}
\newtoks{\emendationsheadername}
\emendationsheadername={Emendations to}
\newcommand{\changeemendationsheader}[1]{\emendationsheadername={#1}}
\newtoks{\emendationscontentsname}
\emendationscontentsname={Emendations}
\newcommand{\changeemendationscontentsname}[1]{%
\emendationscontentsname={#1}}
% two column notes
\newif\ifemendationstwocol
\emendationstwocolfalse
% Emendations section setup
\newcommand{\makeemendations}{\global\emendationsontrue 
  \newwrite\emendations
  \global\verselinenumberstrue
   \immediate\openout\emendations=\jobname.emd
   \literalemend{\flushbottom{\par\clearpage
   {\pagestyle{volumetitlestyle}\cleardoublepage}
\thispagestyle{volumetitlestyle}}
\fancyhead{}
\fancyhead[RO,LE]{{\small\textrm{\thepage}}}
\fancyhead[CO]{{\small{\em \the\emendationsheadername~\mymarks}}}
\fancyhead[CE]{{\small{\em \the\emendationsheadername~\mymarks}}}
\fancyfoot{}
\mark{3}
\setendnotessectiontitle{\the\emendationsname}{emendationnotes}
\tolerance=500\language=0
\normalfont \backmattertextfont
\ifemendationstwocol
\begin{multicols}{2}
\fi
\setlength{\parindent}{-10pt}
}
}
%    \end{macrocode}
%    \end{macro}

%    \begin{macro}{\makeexplanatorynotes}
% And the explanatory notes:
%    \begin{macrocode}
% Tokens to change names of these notes
\newtoks{\explanationsname}
\explanationsname={EXPLANATORY NOTES}
\newcommand{\changeexplanationsname}[1]{\explanationsname={#1}}
\newtoks{\explanationsheadername}
\explanationsheadername={Explanatory Notes to}
\newcommand{\changeexplanationsheader}[1]{\explanationsheadername={#1}}
\newtoks{\explanationscontentsname}
\explanationscontentsname={Explanatory Notes}
\newcommand{\changeexplanationscontentsname}[1]{%
\explanationscontentsname={#1}}
% Two column notes
\newif\ifexplanationstwocol
\explanationstwocolfalse
% Explanatory notes setup
\newcommand{\makeexplanatorynotes}{\global\explanontrue
     \global\verselinenumberstrue
     \newwrite\explanations
     \immediate\openout\explanations=\jobname.enx
     \literalexplain{\flushbottom{\par\clearpage
	 {\pagestyle{volumetitlestyle}\cleardoublepage}
\thispagestyle{volumetitlestyle}}
\fancyhead[RO,LE]{{\small\textrm{\thepage}}}
\fancyhead[CO]{{\small{\em \the\explanationsheadername~\mymarks}}}
\fancyhead[CE]{{\small{\em \the\explanationsheadername~\mymarks}}}
\fancyfoot{}
\mark{3}
\setendnotessectiontitle{\the\explanationsname}{explanatorynotes}
\tolerance=500\language=0
\normalfont \backmattertextfont 
\ifexplanationstwocol
\begin{multicols}{2}
\fi
\setlength{\parindent}{-10pt}
}
}

%    \end{macrocode}
%    \end{macro}
%
% Auxiliary commands for note and title sections
%    \begin{macrocode}
%    
\newcommand{\@pagemarktotextnotes}[1]{%
\immediate\write\textnotes{\string\setcounterfrompageref\string{notepageholdertitle\string}%
\string{#1\string}\string\unskip}%
\immediate\write\textnotes{\string\mark%
\string{\string\thenotepageholdertitle\string}}%
}
\newcommand{\@poempagetotextnotes}[1]{%
\immediate\write\textnotes{\string\par\string\argpageref\string{#1\string}\string\unskip}%
\@pagemarktotextnotes{#1}}     
\newcommand{\@poemtitletotextnotes}[1]{\literaltextnote{\textbf{#1}}}
%    \end{macrocode}
%    

% \subsection{~Book, volume, and volume section titles}
%    \begin{macro}{\wholebooktitle}
%    \begin{macro}{\booksection}
%    \begin{macrocode}
\newcommand{\wholebooktitle}[1]{{\Huge\begin{center} 
\hbox{#1}\end{center}}
}
\newcommand{\booksection}[1]{\volumesection{#1}}
%    \end{macrocode}
%    \end{macro}
%    \end{macro}
%    \begin{macro}{\volumetitle{}}
%    \begin{macrocode}
\newcommand{\volumetitle}[1]{%
  \set@p@emtitle{#1}{\volumetitlesink\par}{\volumetitlefont}
  {\relax}{\nobreak\par\nobreak}{\z@}
  \putpagenumberincontentsfalse
  \putpagenumberinnotesfalse
  \literalcontents{\goodbreak}
  \c@ntentsinfo{#1}{\relax}{\contentsvolumefont}
  {\contentsindentoneamount}
  \literalcontents{\vspace{5pt}\par\nobreak}
  \literaltextnote{\notesvolumetitlepenalty\bigskip\goodbreak\par}
  \t@xtnotesinfo{\backmattervolumefont #1 \backmattertextfont\nobreak}
  \putpagenumberincontentstrue
  \putpagenumberinnotestrue}
%    \end{macrocode}
%    \end{macro} 
%    \begin{macro}{\volumetitlefirstline}
%    \begin{macro}{\volumetitlelastline}
%    \begin{macro}{\volumesubtitle} 
%    \begin{macro}{\volumesectiontitle}
%    \begin{macrocode}
\newcommand{\volumetitlefirstline}[1]{\firstlinesettings%
  \set@p@emtitle{#1}{\volumetitlesink\par}
  {\volumetitlefont}
  {\relax}{\nobreak\par\nobreak}{\z@}
  \putpagenumberincontentsfalse
  \putpagenumberinnotesfalse
  \c@ntentsinfo{#1}{\relax}{\contentsvolumefont}
  {\contentsindentoneamount}
%  \literalcontents{\vspace{5pt}\par\nobreak}
%  \literaltextnote{\notesvolumetitlepenalty\bigskip\par}
  \t@xtnotesinfo{\backmattervolumefont #1 \backmattertextfont\nobreak}
  \putpagenumberincontentstrue
  \putpagenumberinnotestrue}
\newcommand{\volumetitlemiddleline}[1]{\middlelinesettings%
  \set@p@emtitle{#1}{\relax}
  {\volumetitlefont}
  {\relax}{\nobreak\par\nobreak}{\z@}
  \putpagenumberincontentsfalse
  \putpagenumberinnotesfalse
  \c@ntentsinfo{#1}{\contentsindentone}{\contentsvolumefont}
  {\contentsindenttwoamount}
  \t@xtnotesinfo{\backmattervolumefont #1 \backmattertextfont\nobreak}
  \putpagenumberincontentstrue
  \putpagenumberinnotestrue}
\newcommand{\volumetitlelastline}[1]{\lastlinesettings%
  \set@p@emtitle{#1}{\relax}
  {\volumetitlefont}
  {\relax}{\nobreak\par\nobreak}{\z@}
  \putpagenumberincontentsfalse
  \putpagenumberinnotesfalse
  \c@ntentsinfo{#1}{\contentsindentone}{\contentsvolumefont}
  {\contentsindenttwoamount}
  \literalcontents{\vspace{5pt}\par\nobreak}
  \t@xtnotesinfo{\backmattervolumefont #1 \backmattertextfont\nobreak}
  \putpagenumberincontentstrue
  \putpagenumberinnotestrue}
\newcommand{\volumesubtitle}[1]{{\volumesubtitlefont 
\volumetitleshift #1 \par}
  \ifpoemcontentson
    \literalcontents{{\contentsvolumesubtitlefont 
#1\vspace{3pt}\par\nobreak}}
  \fi
  \iftextnoteson\iftextnotesatend
     \immediate\write\textnotes{\string\backmattervolumesubtitlefont\ #1}
     \immediate\write\textnotes{\string\par \string\backmattertextfont} 
     \immediate\write\textnotes{\string\nobreak}
  \fi\fi
}
\newcommand{\volumesectiontitle}[1]
{\clearpage\thispagestyle{volumefirststyle}
\begin{center} {\volumesubtitlefont\textbf{#1}}
\end{center} \bigskip\bigskip
  \ifpoemcontentson
  \literalcontents{\goodbreak{\contentsvolumesubtitlefont #1\par\nobreak}}
  \fi
  \iftextnoteson\iftextnotesatend
     \immediate\write\textnotes{\string\notesvolumetitlepenalty}
     \immediate\write\textnotes{\string\par \string\begin{bf}}
     \immediate\write\textnotes{#1 \string\end{bf}}
     \immediate\write\textnotes{\string\par}
     \immediate\write\textnotes{\string\nobreak}
  \fi\fi
}
%    \end{macrocode}
%    \end{macro}
%    \end{macro}
%    \end{macro}
%    \end{macro}

% \subsection{~Commands for setting titles of poems and sequences}
% First, commands for sending labels to the table of contents and to 
% the notes sections. Since these are internal commands, not intended 
% to be issued in the body of the text, they might have been marked out 
% with ``at'' signs to prevent accidental redefinition. But since they 
% are repeatedly written out to the external files, it seemed more 
% economical to run the risk of redefinition than to write out 
% |\makeatletter| and |\makeatother| with every note.
%    \begin{macrocode}
\newcommand{\makepoemlabel}[1]{\label{#1}}
\newcommand{\argpageref}[1]{\pageref{#1}}
%    \end{macrocode}
%  \subsubsection{Poem Titles} 
%  \begin{macro}{\poemtitle}\label{poemtitlecode}
% All of the other commands in this section are variations on this
% one, changing the penalties, the skips, the indentations, and the
% fonts.  The command breaks into three sections.  First, the title
% is set in the text.  |\poemtitlepenalty| tests whether there is
% room for the title and the first couple of lines.  |\poemtitlefont|
% globally sets the fonts for all poem titles.  You can put the
% argument in an |\hbox{}| to make sure that the poem title is not
% broken across two lines, and the |\label| records the page for the
% contents and notes sections.  |\afterpoemtitleskip| is the amount
% of white space after a poem title.  |\fulltitleholder| and
% |\@compoundlabelscratch| record the title for concatenation later.
% All of these commands can manage titles with control sequences in
% them, such as for italics or international characters.  In an
% earlier version there were special commands for such titles, using
% two arguments (one with the control sequence, one without it).
% Those commands are no longer necessary, but have been retained for
% backward compatibility with earlier versions.
% 
% Second, the poem is entered in the external file for the table of 
% contents. |\contentsindentone| sets it in one level in the hierarchy 
% of indentations, and |\contentspoemtitlefont| is the type size used 
% for poems in the table of contents. The |\pageref{}| holds the page 
% number of the poem. 
% 
% Third, a new paragraph is opened in the textual notes for textual 
% collations. |\@poempagetotextnotes{}| sends the page number to the 
% textual notes. |\@poemtitletotextnotes{}| sends the title.
%    \begin{macrocode}


\newcommand{\m@kep@emlabel}{\ifsinglelinetitle%
  \addtocounter{poemnumber}{1}%
  \makepoemlabel{poem\thepoemnumber}%
 \else\iftitlefirstline%
     \addtocounter{poemnumber}{1}%
     \makepoemlabel{poem\thepoemnumber}%
     \else \relax
\fi
\fi
}

\newcommand{\m@ken@teholder}[1]{%
\ifsinglelinetitle
\fulltitleholder={#1}%
\global\edef\@compoundlabelscratch{poem\thepoemnumber}%
\else\iftitlefirstline
       \fulltitleholder={#1}%
       \global\edef\@compoundlabelscratch{poem\thepoemnumber}%
     \else\iftitlemiddleline
	     \titlesofar=\fulltitleholder
	     \titleincrement={~#1}%
	     \global\edef\titleconcat{\the\titlesofar \the\titleincrement}%
	     \fulltitleholder=\expandafter{\titleconcat}%
	     \global\edef\@compoundlabelscratch{poem\thepoemnumber}%
	  \else
	      \titlesofar=\fulltitleholder
	      \titleincrement={~#1}%
	      \global\edef\titleconcat{\the\titlesofar \the\titleincrement}%
	      \fulltitleholder=\expandafter{\titleconcat}%
	      \global\edef\@compoundlabelscratch{poem\thepoemnumber}%
	   \fi
       \fi
  \fi
}
\newif\ifcentertitleson
\newcommand{\centertitles}{\centertitlesontrue}
\makeatletter
\newcommand{\set@p@emtitle}[6]{\setlength{\titleentryoverrun}{#6}%
{#2}%
{#3 \begin{titleentryenvironment}%
{\ifcentertitleson\begin{center}#4 #1\end{center}\else #4 #1\fi}%
\end{titleentryenvironment}}%
{#5}%
\m@kep@emlabel
\m@ken@teholder{#1}%
\setlength{\titleentryoverrun}{\z@}}

% \newcommand{\set@p@emtitle}[6]{\setlength{\titleentryoverrun}{#6}%
% {#2}%
% {#3 \begin{titleentryenvironment}{#4 #1}\end{titleentryenvironment}}%
% {#5}%
% \m@kep@emlabel
% \m@ken@teholder{#1}%
% \setlength{\titleentryoverrun}{\z@}}

\newcommand{\setcontentsleaders}[1]{%
   \def\contentsleaders{#1\nobreak%
}}
\setcontentsleaders{~/~}
\newcommand{\c@ntentsinfo}[4]{\ifpoemcontentson
\immediate\write\poemcontents{\string\setlength\string\contentsentryoverrun\string{#4\string}}%
\immediate\write\poemcontents{\string\begin\string{contentsentryenvironment\string}}%
\literalcontents{#2{#3#1}}%
  \ifputpagenumberincontents
\immediate\write\poemcontents{\string\contentsleaders}%
\immediate\write\poemcontents{\string\pageref
\string{poem\thepoemnumber\string}}%
  \else
  \literalcontents{\relax}%
  \fi
\immediate\write\poemcontents{\string\end\string{contentsentryenvironment\string}}%
\immediate\write\poemcontents{\string\par}%
\immediate\write\poemcontents{\string\setlength\string\contentsentryoverrun
\string{\string\contentsindentoneamount\string}}%
\fi
}
\newif\ifputtitleinnotes
\puttitleinnotestrue
\newcommand{\t@xtnotesinfo}[1]{\iftextnoteson
\iftextnotesatend
 \immediate\write\textnotes{\string\notespoemclubpenalty}%
  \ifputpagenumberinnotes 
   \@poempagetotextnotes{poem\thepoemnumber}%
  \fi 
\ifputtitleinnotes
   \@poemtitletotextnotes{#1}%
\fi
%    \else
%   \relax
   \fi\fi
}
\newcommand{\poemtitle}[1]{%
\set@p@emtitle{#1}{\poemtitlepenalty}{\poemtitlefont}%
{\relax}{\nobreak\par\nobreak\afterpoemtitleskip\nobreak}{\z@}%
\c@ntentsinfo{#1}{\contentsindentone}{\contentspoemtitlefont}%
{\contentsindenttwoamount}%
\t@xtnotesinfo{#1}%
}
\newcommand{\poemtitlenocontents}[1]{%
\set@p@emtitle{#1}{\poemtitlepenalty}{\poemtitlefont}%
{\relax}{\nobreak\par\nobreak\afterpoemtitleskip\nobreak}{\z@}%
\t@xtnotesinfo{#1}%
}
\newcommand{\poemtitlebaretitle}[1]{%
\set@p@emtitle{#1}{\poemtitlepenalty}{\poemtitlefont}%
{\relax}{\nobreak\par\nobreak\afterpoemtitleskip\nobreak}{\z@}%
}
%    \end{macrocode}
%    \end{macro}
%    \begin{macro}{\poemtitleitalic} Poems with italic titles or other 
% formatting (such as |\hour{}|) take two arguments. The second 
% argument is discarded. This command is to maintain backward compatibility
% with earlier versions.
%    \begin{macrocode}
\newcommand{\poemtitleitalic}[2]{%
\poemtitle{#1}}
%    \end{macrocode}
%    \end{macro}
%    For poem subtitles, see the entry on |\sequencesectionsubtitle{}|.
%    
%    \begin{macro}{\poemtitlefirstline}
%    \begin{macro}{\poemtitlemiddleline}
%    \begin{macro}{\poemtitlelastline}
% Next, macros for setting multiline poem titles. Notice the special larger 
% penalty for page breaks in the middle of  multiple line title. 
%    \begin{macrocode}
\newcommand{\firstlinesettings}{\singlelinetitlefalse
\titlefirstlinetrue
\putpagenumberincontentsfalse
\putpagenumberinnotestrue}

\newcommand{\middlelinesettings}{\singlelinetitlefalse
\titlemiddlelinetrue
\putpagenumberincontentsfalse
\putpagenumberinnotesfalse}

\newcommand{\lastlinesettings}{\singlelinetitlefalse
\titlelastlinetrue
\putpagenumberincontentstrue
\putpagenumberinnotesfalse}

\newcommand{\restoresinglelinesettings}{\putpagenumberinnotestrue
\putpagenumberincontentstrue
\titlefirstlinefalse
\titlemiddlelinefalse
\titlelastlinefalse
\singlelinetitletrue}


\newcommand{\poemtitlefirstline}[1]{
\firstlinesettings
\set@p@emtitle{#1}{\multilinetitlepenalty}{\poemtitlefont}%
{\relax}{\nobreak\par\nobreak}{\z@}%
\c@ntentsinfo{#1}{\contentsindentone}{\contentspoemtitlefont}%
{\contentsindenttwoamount}%
\t@xtnotesinfo{#1}%
\restoresinglelinesettings
}
%    \end{macrocode}

% Notice the use of token registers and |\edef| here to concatenate the first and second 
% lines.
%    \begin{macrocode}
\newcommand{\poemtitlemiddleline}[1]{%
\middlelinesettings
\set@p@emtitle{#1}{\par\nobreak}{\poemtitlefont}%
{\relax}{\nobreak\par\nobreak}{\z@}%
\c@ntentsinfo{#1}{\contentsindenttwo}{\contentspoemtitlefont}%
{\contentsindentthreeamount}%
\t@xtnotesinfo{#1}%
\restoresinglelinesettings
}
\newcommand{\poemtitlelastline}[1]{%
\lastlinesettings
\set@p@emtitle{#1}{\par\nobreak}{\poemtitlefont}%
{\relax}{\nobreak\par\afterpoemtitleskip\nobreak}{\z@}%
\c@ntentsinfo{#1}{\contentsindenttwo}{\contentspoemtitlefont}%
{\contentsindentthreeamount}%
\t@xtnotesinfo{#1}%
\restoresinglelinesettings
}
%    \end{macrocode}
%    \end{macro}
%    \end{macro}
%    \end{macro}
% Titles for sections of poems, and multi-line titles of sections 
% of poems
%    \begin{macro}{\poemsectiontitle}
%    \begin{macro}{\poemsectiontitlefirstline}
%    \begin{macro}{\poemsectiontitlemiddleline}
%    \begin{macro}{\poemsectiontitlelastline}
%    \begin{macrocode}
\newcommand{\poemsectiontitle}[1]{%
\set@p@emtitle{#1}{\poemtitlepenalty}{\subsectiontitlefont}%
{\titleindent}{\nobreak\par\nobreak\afterpoemtitleskip\nobreak}%
{\titleindentamount}%
\c@ntentsinfo{#1}{\contentsindenttwo}{\contentspoemtitlefont}%
{\contentsindentthreeamount}%
\t@xtnotesinfo{#1}%
}
\newcommand{\poemsectiontitlefirstline}[1]{%
\firstlinesettings
\set@p@emtitle{#1}{\multilinetitlepenalty}{\subsectiontitlefont}%
{\titleindent}{\nobreak\par\nobreak}{\titleindentamount}%
\c@ntentsinfo{#1}{\contentsindenttwo}{\contentspoemtitlefont}%
{\contentsindentthreeamount}%
\t@xtnotesinfo{#1}%
\restoresinglelinesettings
}
\newcommand{\poemsectiontitlemiddleline}[1]{%
\middlelinesettings
\set@p@emtitle{#1}{\relax}{\subsectiontitlefont}%
{\titleindent}{\nobreak\par\nobreak}{\titleindentamount}%
\c@ntentsinfo{#1}{\contentsindentthree}{\contentspoemtitlefont}%
{\contentsindentfouramount}%
\t@xtnotesinfo{#1}%
\restoresinglelinesettings
}
\newcommand{\poemsectiontitlelastline}[1]{%
\lastlinesettings
\set@p@emtitle{#1}{\par\nobreak}{\subsectiontitlefont}%
{\titleindent}{\nobreak\par\afterpoemtitleskip\nobreak}%
{\titleindentamount}%
\c@ntentsinfo{#1}{\contentsindentthree}{\contentspoemtitlefont}%
{\contentsindentfouramount}%
\t@xtnotesinfo{#1}%
\restoresinglelinesettings
}
%    \end{macrocode}
%    \end{macro}
%    \end{macro}
%    \end{macro}
%    \end{macro}
%    \begin{macro}{\poemsectiontitlenocontents}
%    \begin{macro}{\poemsectionbaretitle}
%    \begin{macro}{\poemsubsectiontitle}
%    \begin{macro}{\poemtitlenotitle}
%    \begin{macro}{\poemtitlenotitleitalic}
%    \begin{macrocode}
\newcommand{\poemsectiontitlenocontents}[1]{%
  \sequencesubsectiontitlenocontents{#1}}%
\newcommand{\poemfirstsectiontitle}[1]{%
\set@p@emtitle{#1}{\par\nobreak}{\subsectiontitlefont}%
{\titleindent}{\nobreak\par\nobreak\afterpoemtitleskip\nobreak}%
{\titleindentamount}%
\c@ntentsinfo{#1}{\contentsindenttwo}{\contentspoemtitlefont}%
{\contentsindentthreeamount}%
\t@xtnotesinfo{#1}%                                                        
}
\newcommand{\poemfirstsectiontitlebaretitle}[1]{%
\set@p@emtitle{#1}{\par\nobreak}{\subsectiontitlefont}%
{\titleindent}{\nobreak\par\nobreak\afterpoemtitleskip\nobreak}%
{\titleindentamount}}
\newcommand{\poemsectiontitlebaretitle}[1]{%
\set@p@emtitle{#1}{\poemtitlepenalty}{\subsectiontitlefont}%
{\titleindent}{\nobreak\par\nobreak\afterpoemtitleskip\nobreak}%
{\titleindentamount}%
}
\newcommand{\poemsubsectiontitle}[1]{%
\set@p@emtitle{#1}{\poemtitlepenalty}{\subsectiontitlefont}%
{\titleindent\titleindent}{\nobreak\par\nobreak\afterpoemtitleskip\nobreak}%
{\titleindenttwoamount}%
\c@ntentsinfo{#1}{\contentsindentthree}{\contentspoemtitlefont}%
{\contentsindentfouramount}%
\t@xtnotesinfo{#1}%
}
\newcommand{\poemtitlenotitle}[1]{%
\m@kep@emlabel
\m@ken@teholder{#1}%
\c@ntentsinfo{#1}{\contentsindentone}{\contentspoemtitlefont}%
{\contentsindenttwoamount}%
\t@xtnotesinfo{#1}%
}
\newcommand{\poemtitlenotitleitalic}[2]{%
\poemtitlenotitle{1}}
%    \end{macrocode}
%    \end{macro}
%    \end{macro}
%    \end{macro}
%    \end{macro}
%    \end{macro}  
%    
%    \subsubsection{Poetic Sequences: Setting the Main Title}
%    \begin{macro}{\sequencetitle}
%    Variants here for multiple line titles and titles with font 
% commands. Also some all-purpose kludges to work around other problems.
%    \begin{macrocode}
\newcommand{\sequencetitle}[1]{
\set@p@emtitle{#1}{\sequencetitlepenalty}{\sequencetitlefont}
{\relax}{\nobreak\par\nobreak\aftersequencetitleskip\nobreak}{\z@}
\putpagenumberincontentsfalse
\c@ntentsinfo{#1}{\contentsindentone}{\contentssequencetitlefont}
{\contentsindenttwoamount}
\putpagenumberincontentstrue
\t@xtnotesinfo{#1}
}
%    \end{macrocode}
%    \end{macro}

%    \begin{macro}{\sequencetitlefirstline}
%    \begin{macro}{\sequencetitlemiddleline}
%    \begin{macro}{\sequencetitlelastline}
%  To set the title of a poetic sequence if it requires several lines 
% to do so.
%    \begin{macrocode}
\newcommand{\sequencetitlefirstline}[1]{
\firstlinesettings
\set@p@emtitle{#1}{\multilinesequencepenalty}{\sequencetitlefont}
{\relax}{\nobreak\par\nobreak}{\z@}
\c@ntentsinfo{#1}{\contentsindentone}{\contentssequencetitlefont}
{\contentsindenttwoamount}
\t@xtnotesinfo{#1}
\restoresinglelinesettings
}
\newcommand{\sequencetitlemiddleline}[1]{
\middlelinesettings
\set@p@emtitle{#1}{\par\nobreak}{\sequencetitlefont}
{\relax}{\nobreak\par\nobreak}{\z@}
\c@ntentsinfo{#1}{\contentsindenttwo}{\contentssequencetitlefont}
{\contentsindentthreeamount}
\t@xtnotesinfo{#1}
\restoresinglelinesettings
}
\newcommand{\sequencetitlelastline}[1]{
\lastlinesettings
\set@p@emtitle{#1}{\par\nobreak}{\sequencetitlefont}
{\relax}{\nobreak\par\aftersequencetitleskip\nobreak}{\z@}
\c@ntentsinfo{#1}{\contentsindenttwo}{\contentssequencetitlefont}
{\contentsindentthreeamount}
\t@xtnotesinfo{#1}
\restoresinglelinesettings
}
%    \end{macrocode}
%    \end{macro}
%    \end{macro}
%    \end{macro}
%    \subsubsection{Sections of poetic sequences}
%    \begin{macro}{\sequencesectiontitle}
% First, the easy case: a simple section of a poetic sequence.
%    \begin{macrocode}
\newcommand{\sequencesectiontitle}[1]{
\set@p@emtitle{#1}{\poemtitlepenalty}{\poemtitlefont}
{\titleindent}{\nobreak\par\nobreak\afterpoemtitleskip\nobreak}
{\titleindentamount}
\c@ntentsinfo{#1}{\contentsindenttwo}{\contentspoemtitlefont}
{\contentsindentthreeamount}
\t@xtnotesinfo{#1}
}
%    \end{macrocode}
%    \end{macro}

%    \begin{macro}{\sequencefirstsectiontitle}
% The first poem in a sequence is a special case. 
%    \begin{macrocode}
\newcommand{\sequencefirstsectiontitle}[1]{
\set@p@emtitle{#1}{\par\nobreak\afterpoemtitleskip}{\poemtitlefont}
{\titleindent}{\nobreak\par\nobreak\afterpoemtitleskip\nobreak}
{\titleindentamount}
\c@ntentsinfo{#1}{\contentsindenttwo}{\contentspoemtitlefont}
{\contentsindentthreeamount}
\t@xtnotesinfo{#1}
}
%    \end{macrocode}
%    \end{macro}
%    \begin{macro}{\sequencesectiontitlenonotes}
% If you don't have any notes on a sequence section, it seems a shame 
% to open an entry in the textual notes for that section:
%    \begin{macrocode}
\newcommand{\sequencetitlenonotes}[1]{
\set@p@emtitle{#1}{\par\nobreak}{\sequencetitlefont}
{\relax}{\nobreak\par\aftersequencetitleskip\nobreak}{\z@}
\c@ntentsinfo{#1}{\contentsindenttwo}{\contentssequencetitlefont}
{\contentsindentthreeamount}
%\t@xtnotesinfo{#1}
}
\newcommand{\sequencesectiontitlenonotes}[1]{
\set@p@emtitle{#1}{\poemtitlepenalty}{\poemtitlefont}
{\titleindent}{\nobreak\par\nobreak\afterpoemtitleskip\nobreak}
{\titleindentamount}
\c@ntentsinfo{#1}{\contentsindenttwo}{\contentspoemtitlefont}
{\contentsindentthreeamount}
%\t@xtnotesinfo{#1}
}
%    \end{macrocode}
%    \end{macro}
% For multiple line titles of sequence sections
%    \begin{macrocode}
\newcommand{\sequencesectiontitlefirstline}[1]{
\firstlinesettings
\set@p@emtitle{#1}{\multilinetitlepenalty}{\poemtitlefont}
{\titleindent}{\nobreak\par\nobreak}{\titleindentamount}
\c@ntentsinfo{#1}{\contentsindenttwo}{\contentspoemtitlefont}
{\contentsindentthreeamount}
\t@xtnotesinfo{#1}
\restoresinglelinesettings
}
\newcommand{\sequencesectiontitlemiddleline}[1]{
\middlelinesettings
\set@p@emtitle{#1}{\par\nobreak}{\poemtitlefont}
{\titleindent}{\nobreak\par\nobreak}{\titleindentamount}
\c@ntentsinfo{#1}{\contentsindentthree}{\contentspoemtitlefont}
{\contentsindentfouramount}
\t@xtnotesinfo{#1}
\restoresinglelinesettings
}
\newcommand{\sequencesectiontitlelastline}[1]{
\lastlinesettings
\set@p@emtitle{#1}{\par\nobreak}{\poemtitlefont}
{\titleindent}{\nobreak\par\afterpoemtitleskip\nobreak}{\titleindentamount}
\c@ntentsinfo{#1}{\contentsindentthree}{\contentspoemtitlefont}
{\contentsindentfouramount}
\t@xtnotesinfo{#1}
\restoresinglelinesettings
}
%    \end{macrocode}
% The following macros are
% to kludge around situations where the section title is just a number.  This
% macro has two arguments.  Originally, the first was the title of the section.
% The second is the title of the sequence as a whole. I've retained
% the second argument for backward compatibility with earlier versions, but 
% you can set it to anything, since it's discarded now.
%    \begin{macrocode}
\newcommand{\sequencefirstsectiontitlenocontents}[2]{
\set@p@emtitle{#1}{\par\nobreak\afterpoemtitleskip}{\poemtitlefont}
{\titleindent}{\nobreak\par\nobreak\afterpoemtitleskip\nobreak}
{\titleindentamount}
%\c@ntentsinfo{#1}{\contentsindenttwo}{\contentspoemtitlefont}
\t@xtnotesinfo{#1}
}
\newcommand{\sequencesectiontitlenocontents}[2]{
\set@p@emtitle{#1}{\poemtitlepenalty}{\poemtitlefont}
{\titleindent}{\nobreak\par\nobreak\afterpoemtitleskip\nobreak}
{\titleindentamount}
%\c@ntentsinfo{#1}{\contentsindenttwo}{\contentspoemtitlefont}
\t@xtnotesinfo{#1}
}
%    \end{macrocode}
% If the first section of a sequence is a multiple line title, the 
% following macros can handle the special problems that situation poses.
%    \begin{macrocode}
\newcommand{\sequencefirstsectiontitlefirstline}[1]{
\firstlinesettings
\set@p@emtitle{#1}{\par\nobreak\afterpoemtitleskip}{\poemtitlefont}
{\titleindent}{\nobreak\par\nobreak\afterpoemtitleskip\nobreak}
{\titleindentamount}
\c@ntentsinfo{#1}{\contentsindenttwo}{\contentspoemtitlefont}
{\contentsindentthreeamount}
\t@xtnotesinfo{#1}
\restoresinglelinesettings
}
\newcommand{\sequencefirstsectiontitlemiddleline}[1]{
\sequencesectiontitlemiddleline{#1}}

\newcommand{\sequencefirstsectiontitlelastline}[1]
{\sequencesectiontitlelastline{#1}}

\newcommand{\sequencefirstsectiontitlenonotes}[1]{
\set@p@emtitle{#1}{\par\nobreak\afterpoemtitleskip}{\poemtitlefont}
{\titleindent}{\nobreak\par\nobreak\afterpoemtitleskip\nobreak}
{\titleindentamount}
\c@ntentsinfo{#1}{\contentsindenttwo}{\contentspoemtitlefont}
{\contentsindentthreeamount}
%\t@xtnotesinfo{#1}
}
%    \end{macrocode}

%    \begin{macro}{\sequencesectionsubtitle}
%    \begin{macro}{\poemsubtitle}
%    \begin{macrocode}
\newcommand{\sequencesectionsubtitle}[1]{{\subsectiontitlefont \hbox{\titleindent #1}}
  \nobreak\par\nobreak\afterpoemtitleskip\nobreak
}
\newcommand{\poemsubtitle}[1]{\sequencesectionsubtitle{#1}
}
%    \end{macrocode}
%    \end{macro}
%    \end{macro}
% Another (former) kludge for special situations:
%    \begin{macrocode}
\newcommand{\sequencesectiontitleitalicnonotes}[2]
{\sequencesectiontitlenonotes{#1}
}
%    \end{macrocode}
% Italic titles used to require two arguments. These commands
% are retained only for backward compatibility.
% 
%    \begin{macro}{\sequencesectiontitleitalic}
%    \begin{macrocode}
\newcommand{\sequencesectiontitleitalic}[2]{
\sequencesectiontitle{#1}}
\newcommand{\sequencesectiontitlefirstlineitalic}[2]
{\sequencesectiontitlefirstline{#1}}
\newcommand{\sequencesectiontitlebaretitle}[1]{
\set@p@emtitle{#1}{\poemtitlepenalty}{\poemtitlefont}
{\titleindent}{\nobreak\par\nobreak\afterpoemtitleskip\nobreak}{\titleindentamount}
}
%    \end{macrocode}
%    \end{macro}



% \subsubsection{Subsections of sequences} 
%    \begin{macro}{\sequencesubsectiontitle}

% 	Continuing several levels down: macros for setting subsections and 
% subsubsections  of sequences. 
%    \begin{macrocode}
\newcommand{\sequencesubsectiontitle}[1]{
\set@p@emtitle{#1}{\poemtitlepenalty}{\subsectiontitlefont}
{\titleindent\titleindent}{\nobreak\par\nobreak\afterpoemtitleskip\nobreak}
{\titleindenttwoamount}
\c@ntentsinfo{#1}{\contentsindentthree}{\contentspoemtitlefont}
{\contentsindentfouramount}
\t@xtnotesinfo{#1}
}
\newcommand{\sequencesubsectiontitlenocontents}[1]{
\set@p@emtitle{#1}{\poemtitlepenalty}{\subsectiontitlefont}
{\titleindent\titleindent}{\nobreak\par\nobreak\afterpoemtitleskip\nobreak}
{\titleindenttwoamount}
%\c@ntentsinfo{#1}{\contentsindentthree}{\contentspoemtitlefont}{\contentsindentfouramount}
\t@xtnotesinfo{#1}
}
\newcommand{\sequencefirstsubsectiontitle}[1]
{\set@p@emtitle{#1}{\par\nobreak}{\subsectiontitlefont}
{\titleindent\titleindent}{\nobreak\par\nobreak\afterpoemtitleskip\nobreak}
{\titleindenttwoamount}
\c@ntentsinfo{#1}{\contentsindentthree}{\contentspoemtitlefont}
{\contentsindentfouramount}
\t@xtnotesinfo{#1}
}
\newcommand{\sequencefirstsubsectiontitlenocontents}[1]
{\set@p@emtitle{#1}{\par\nobreak}{\subsectiontitlefont}
{\titleindent\titleindent}{\nobreak\par\nobreak\afterpoemtitleskip\nobreak}
{\titleindenttwoamount}
%\c@ntentsinfo{#1}{\contentsindentthree}{\contentspoemtitlefont}
%{\contentsindentfouramount}
\t@xtnotesinfo{#1}
}
\newcommand{\sequencesubsectiontitlefirstline}[1]{
\firstlinesettings
\set@p@emtitle{#1}{\multilinetitlepenalty}{\subsectiontitlefont}
{\titleindent\titleindent}{\nobreak\par\nobreak}{\titleindenttwoamount}
\c@ntentsinfo{#1}{\contentsindentthree}{\contentspoemtitlefont}
{\contentsindentfouramount}
\t@xtnotesinfo{#1}
\restoresinglelinesettings
}
\newcommand{\sequencesubsectiontitlemiddleline}[1]{
\middlelinesettings
\set@p@emtitle{#1}{\nobreak}{\subsectiontitlefont}
{\titleindent\titleindent}{\nobreak\par\nobreak}{\titleindenttwoamount}
\c@ntentsinfo{#1}{\contentsindentfour}{\contentspoemtitlefont}
{\contentsindentfiveamount}
\t@xtnotesinfo{#1}
\restoresinglelinesettings
}
\newcommand{\sequencesubsectiontitlelastline}[1]{
\lastlinesettings
\set@p@emtitle{#1}{\nobreak}{\subsectiontitlefont}
{\titleindent\titleindent}{\nobreak\par\afterpoemtitleskip\nobreak}
{\titleindenttwoamount}
\c@ntentsinfo{#1}{\contentsindentfour}{\contentspoemtitlefont}
{\contentsindentfiveamount}
\t@xtnotesinfo{#1}
\restoresinglelinesettings
}
\newcommand{\sequencesubsubsectiontitle}[1]{
\set@p@emtitle{#1}{\poemtitlepenalty}{\subsectiontitlefont}
{\titleindent\titleindent\titleindent}
{\nobreak\par\nobreak\afterpoemtitleskip\nobreak}
{\titleindentthreeamount}
\c@ntentsinfo{#1}{\contentsindentfour}{\contentspoemtitlefont}
{\contentsindentfiveamount}
\t@xtnotesinfo{#1}
}
\newcommand{\sequencesubsubsectiontitlenocontents}[1]{
\set@p@emtitle{#1}{\poemtitlepenalty}{\subsectiontitlefont}
{\titleindent\titleindent\titleindent}{\nobreak\par\nobreak\afterpoemtitleskip\nobreak}
{\titleindentthreeamount}
%\c@ntentsinfo{#1}{\contentsindentfour}{\contentspoemtitlefont}
%{\contentsindentfiveamount}
\t@xtnotesinfo{#1}
}
\newcommand{\sequencefirstsubsubsectiontitlenocontents}[1]{
\set@p@emtitle{#1}{\nobreak\par\nobreak}{\subsectiontitlefont}
{\titleindent\titleindent\titleindent}{\nobreak\par\nobreak\afterpoemtitleskip\nobreak}
{\titleindentthreeamount}
%\c@ntentsinfo{#1}{\contentsindentfour}{\contentspoemtitlefont}{\contentsindentfiveamount}
\t@xtnotesinfo{#1}
}


\newcommand{\sequencefirstsubsectiontitlefirstline}[1]{
\firstlinesettings
\set@p@emtitle{#1}{\par\nobreak}{\subsectiontitlefont}
{\titleindent\titleindent}{\nobreak\par\nobreak}{\titleindenttwoamount}
\c@ntentsinfo{#1}{\contentsindentthree}{\contentspoemtitlefont}
{\contentsindentfouramount}
\t@xtnotesinfo{#1}
\restoresinglelinesettings
}
\newcommand{\sequencefirstsubsectiontitlemiddleline}[1]{
\sequencesubsectiontitlemiddleline{#1}}
\newcommand{\sequencefirstsubsectiontitlelastline}[1]{
\sequencesubsectiontitlelastline{#1}}
%    \end{macrocode}
%    \end{macro}

% Interjections in sequences, as in Robert Penn Warren's \emph{Or Else}.
%    \begin{macrocode}
\newcommand{\interjectiontitlefirstline}[1]{
\sequencefirstsubsectiontitlefirstline{#1}}
\newcommand{\interjectiontitlelastline}[1]{
\sequencefirstsubsectiontitlelastline{#1}
}
\newcommand{\interjectiontitlemiddleline}[1]{
\sequencefirstsubsectionmiddleline{#1}
}
%    \end{macrocode}
%    \begin{macro}{\setmargpoemtitle}
% Titles with a marginal reference to the notes giving the page in 
% the notes where a comment on the poem appears, as in Altenberg's 
% Heredia edition.\label{JHpoemtitle} |\setmargpoemtitle| takes as 
% its second argument anything you want to add to the line while the 
% title is being processed. |\JHpoemtitle| is just 
% |\setmargpoemtitle| with the second argument passed to |\JHlabel|, 
% which actually makes the marginal mark.
%    \begin{macrocode}
\newcommand{\setmargpoemtitle}[2]{%
\set@margp@emtitle{#1}{\poemtitlepenalty}{\poemtitlefont}%
{\relax}{\nobreak\par\nobreak\afterpoemtitleskip\nobreak}{\z@}{#2}%
\c@ntentsinfo{#1}{\contentsindentone}{\contentspoemtitlefont}%
{\contentsindenttwoamount}%
\t@xtnotesinfo{#1}%
\sources{\label{#2}}%                  
}
\newcommand{\set@margp@emtitle}[7]{\setlength{\titleentryoverrun}{#6}%
{#2}%
{#3 \begin{titleentryenvironment}%
{#4 #1 #7}\end{titleentryenvironment}}%
{#5}%
\m@kep@emlabel%
\m@ken@teholder{#1}%
\setlength{\titleentryoverrun}{\z@}}
%    \end{macrocode}
%    \end{macro}
%    \begin{macro}{\JHpoemtitle} Macros to set marginal line 
%    references.
%    \begin{macrocode}
% \newcommand{\JHpoemtitle}[2]{%
% \setmargpoemtitle{#1}{\JHlabel{#2}}%
% }
\newcommand{\JHpoemtitle}[2]{%
\set@margp@emtitle{#1}{\poemtitlepenalty}{\poemtitlefont}%
{\relax}{\nobreak\par\nobreak\afterpoemtitleskip\nobreak}{\z@}%
{\JHlabel{#2}}%
\c@ntentsinfo{#1}{\contentsindentone}{\contentspoemtitlefont}%
{\contentsindenttwoamount}%
\t@xtnotesinfo{#1}%
\sources{\label{#2}}%                  
}
\newcommand{\JHprosesectiontitle}[2]{%
\set@margp@emtitle{#1}{\poemtitlepenalty}{\poemtitlefont}%
{\relax}{\nobreak\par\nobreak\afterpoemtitleskip\nobreak}{\z@}%
{\JHlabel{#2}}%
\c@ntentsinfo{#1}{\contentsindentone}{\contentspoemtitlefont}%
{\contentsindenttwoamount}%
\t@xtnotesinfo{#1}%
\sources{\label{#2}}%                  
}  
\newcommand{\JHlabel}[1]{%
\marginpar{{\scriptsize{\the\margrefmarker~p.~\pageref{#1}}}}}
%    \end{macrocode}
%    \end{macro}
%    \begin{macro}{\JHsequencetitle} Macros to set marginal page 
%    references for sequences, sequence sections, poem sections, poem 
%    subtitles, dedications, and epigraphs.
%    \begin{macrocode}
\newcommand{\JHsequencetitle}[2]{%
\set@margp@emtitle{#1}{\sequencetitlepenalty}{\sequencetitlefont}%
{\relax}{\nobreak\par\nobreak\aftersequencetitleskip\nobreak}{\z@}%
{\JHlabel{#2}}%
\putpagenumberincontentsfalse%
\c@ntentsinfo{#1}{\contentsindentone}{\contentssequencetitlefont}%
{\contentsindenttwoamount}%
\putpagenumberincontentstrue%
\t@xtnotesinfo{#1}%
\sources{\label{#2}}% 
}
\newcommand{\JHsequencesectiontitle}[2]{%
\set@margp@emtitle{#1}{\poemtitlepenalty}{\poemtitlefont}%
{\titleindent}{\nobreak\par\nobreak\afterpoemtitleskip\nobreak}%
{\titleindentamount}{\JHlabel{#2}}%
\c@ntentsinfo{#1}{\contentsindenttwo}{\contentspoemtitlefont}%
{\contentsindentthreeamount}%
\t@xtnotesinfo{#1}%
\sources{\label{#2}}%                   
}
\newcommand{\JHsequencefirstsectiontitle}[2]{%
\set@margp@emtitle{#1}{\par\nobreak\afterpoemtitleskip}{\poemtitlefont}%
{\titleindent}{\nobreak\par\nobreak\afterpoemtitleskip\nobreak}%
{\titleindentamount}{\JHlabel{#2}}%
\c@ntentsinfo{#1}{\contentsindenttwo}{\contentspoemtitlefont}%
{\contentsindentthreeamount}%
\t@xtnotesinfo{#1}%
\sources{\label{#2}}% 
}
\newcommand{\JHpoemsectiontitle}[2]{%
\set@margp@emtitle{#1}{\poemtitlepenalty}{\subsectiontitlefont}%
{\titleindent}{\nobreak\par\nobreak\afterpoemtitleskip\nobreak}%
{\titleindentamount}{\JHlabel{#2}}%
\c@ntentsinfo{#1}{\contentsindenttwo}{\contentspoemtitlefont}%
{\contentsindentthreeamount}%
\t@xtnotesinfo{#1}%
\sources{\label{#2}}% 
}
\newcommand{\JHpoemfirstsectiontitle}[2]{%
\set@margp@emtitle{#1}{\par\nobreak}{\subsectiontitlefont}%
{\titleindent}{\nobreak\par\nobreak\afterpoemtitleskip\nobreak}%
{\titleindentamount}{\JHlabel{#2}}%
\c@ntentsinfo{#1}{\contentsindenttwo}{\contentspoemtitlefont}%
{\contentsindentthreeamount}%
\t@xtnotesinfo{#1}%                                                        
\sources{\label{#2}}% 
}
\newcommand{\JHsequencesectionsubtitle}[2]{%
\set@margp@emtitle{#1}{\par\nobreak}{\subsectiontitlefont}%
{\titleindent}{\nobreak\par\nobreak\afterpoemtitleskip\nobreak}%
{\titleindentamount}{\JHlabel{#2}}%
\sources{\label{#2}}% 
}
\newcommand{\JHpoemsubtitle}[2]{\JHsequencesectionsubtitle{#1}{#2}
}
\newcommand{\JHdedication}[2]{\nopagebreak\afterpoemtitleskip\begin{epigraphquote}%
\emph{#1}\JHlabel{#2}\end{epigraphquote}%
\sources{\label{#2}}\afterpoemtitleskip\nopagebreak}
\newcommand{\JHepigraph}[2]{\nopagebreak\afterpoemtitleskip\begin{epigraphquote}%
\emph{#1}\JHlabel{#2}\end{epigraphquote}%
\sources{\label{#2}}\afterpoemtitleskip\nopagebreak}
%    \end{macrocode}
%    \end{macro}
%    \begin{macro}{\backmattersectiontitle} Macros to set the titles of back matter sections
%    such as textual notes.
%     
%    \begin{macrocode}
\newcommand{\backmattersectiontitle}[1]{%
\set@p@emtitle{#1}{\poemtitlepenalty}{\volumetitlefont}%
{\relax}{\backmattersink}{\z@}%
\c@ntentsinfo{#1}{contentsindentone}{\contentspoemtitlefont}%
{\contentsindenttwoamount}%
\t@xtnotesinfo{#1}
}
%    \end{macrocode}
%    \end{macro}
%    \begin{macro}{\margreftextnote} Macros to set marginal page 
% references showing where in the apparatus a comment upon the 
% marked line is to be found. |\margrefspecial{}| is for occasions
% when you might need to set the label by hand. For instance, suppose
% you have several textnotes on a line, and the note you wish to draw
% attention to is on a different page from the other notes.
% |\margreftextnote| under those circumstances will point only to the
% first note. Use |\margrefspecial| to point to the other note,
% giving |\margrefspecial| a unique argument, and opening that note
% with a \label{} command using the same argument.   
%     
%    \begin{macrocode}
\newcounter{margrefnumber}
\setcounter{margrefnumber}{1}
\newtoks{\margrefmarker}
\margrefmarker={\dag}
\newcommand{\setmargrefmarker}[1]{\margrefmarker={#1}}
\newcommand{\makemargreflabel}{\string\label\string{margref\themargrefnumber\string}}
\newcommand{\setmargref}
{\marginpar{\scriptsize {\the\margrefmarker~p.~\pageref{margref\themargrefnumber}}}
}
\newcommand{\margreftextnote}{%
\iftextnoteson\addtocounter{margrefnumber}{1}%
\immediate\write\textnotes{\makemargreflabel}%
\setmargref\else\relax\fi%                                             
}
\newcommand{\margrefexplanatory}{%
\ifexplanon\addtocounter{margrefnumber}{1}%
\immediate\write\explanations{\makemargreflabel}%
\setmargref\else\relax\fi%                                             
}
\newcommand{\margrefemendation}{%
\ifemendationson\addtocounter{margrefnumber}{1}%
\immediate\write\emendations{\makemargreflabel}%
\setmargref\else\relax\fi%                                             
}
\newcommand{\margrefspecial}[1]{%
\marginpar{\scriptsize {\the\margrefmarker~p.~\pageref{#1}}}
}
%    \end{macrocode}
%    \end{macro}
%    To change other title making commands to produce titles which add
%    something while processing the title, you should: 1.  add one to
%    the number of arguments it uses 2.  replace the use of
%    |\set@p@emtitle| with |\set@margp@emtitle| 3.  add |{#2}| or 
%    |\JHlabel{#2}| or whatever as the
%    last (seventh) argument to |\set@margp@emtitle|.  So, to change
%    |\sequencetitle{}| to |\JHsequencetitle{}{}| start with:
% \begin{verbatim}
% \newcommand{\sequencetitle}[1]{
% \set@p@emtitle{#1}{\sequencetitlepenalty}{\sequencetitlefont}
% {\relax}{\nobreak\par\nobreak\aftersequencetitleskip\nobreak}{\z@}
% \putpagenumberincontentsfalse
% \c@ntentsinfo{#1}{\contentsindentone}{\contentssequencetitlefont}
% {\contentsindenttwoamount}
% \putpagenumberincontentstrue
% \t@xtnotesinfo{#1}
% }
% \end{verbatim}
%  and change it to
% \begin{verbatim}
% \catcode`\@=11
% \newcommand{\JHsequencetitle}[2]{
% \set@margp@emtitle{#1}{\sequencetitlepenalty}{\sequencetitlefont}
% {\relax}{\nobreak\par\nobreak\aftersequencetitleskip\nobreak}{\z@}
% {\JHlabel{#2}}\putpagenumberincontentsfalse
% \c@ntentsinfo{#1}{\contentsindentone}{\contentssequencetitlefont}
% {\contentsindenttwoamount}
% \putpagenumberincontentstrue
% \t@xtnotesinfo{#1}
% }
% \catcode`\@=12
% \end{verbatim}
% \subsection{~Epigraphs, headnotes, attributions, dedications}
% Notice that epigraphs to volumes are handled differently from 
% epigraphs to poems.
%    \begin{macro}{\epigraph}
%    \begin{macro}{\headnote}
%    \begin{macro}{\attribution}
%    \begin{macro}{\poemdedication}
%    \begin{macro}{\volumededication}
%    \begin{macro}{\volumeepigraph}
%    \begin{macro}{\volumeattribution}
%    \begin{macrocode}
\newcommand{\epigraph}[1]{\nopagebreak\afterpoemtitleskip\begin{epigraphquote}
  \emph{#1}\end{epigraphquote}\afterpoemtitleskip\nopagebreak}
\newcommand{\headnote}[1]{\epigraph{#1}}
\newcommand{\attribution}[1]{\nopagebreak\afterpoemtitleskip\begin{epigraphquote}
 {\small\emph{#1}}\end{epigraphquote}\afterpoemtitleskip\nopagebreak}
\newcommand{\poemdedication}[1]{\nopagebreak\afterpoemtitleskip\begin{epigraphquote}
 \emph{#1}\end{epigraphquote}\afterpoemtitleskip\nopagebreak}
\providecommand{\dedication}[1]{\poemdedication{#1}}
\newcommand{\volumededication}[1]{\par\bigskip\begin{volumetitlepagequote}
  \emph{#1}\end{volumetitlepagequote}}
\newcommand{\volumeepigraph}[1]{\par\bigskip\begin{volumetitlepagequote}
  \emph{#1}\end{volumetitlepagequote}}
\newcommand{\volumeattribution}[1]{\par\smallskip\begin{volumetitlepagequote}
 {\small \emph{#1}}\end{volumetitlepagequote}}
%    \end{macrocode}
%    \end{macro}
%    \end{macro}
%    \end{macro}
%    \end{macro}
%    \end{macro}
%    \end{macro}
%    \end{macro}
%    \subsection{~Tools used for making note sections} The main tool is 
% a dirty trick borrowed from John Lavagnino's package for endnotes, 
% \textsf{endnotes}, which allows for writing out literal characters to an 
% external file. The trick involves redefining the space character as 
% the newline character and treating the text of the note as the body of 
% a macro that |\meaning| specifies. The result is that the text is 
% written to the external file in a long ribbon one word wide.
%    \begin{macrocode} 
\def\strip#1>{}
\newcommand{\literaltextnote}[1]{\iftextnotesatend
     \begingroup
        \def\next{#1}%
        \newlinechar='40
        \immediate\write\textnotes{\expandafter\strip\meaning\next}%
     \endgroup\fi
 }
\newcommand{\literalemend}[1]{\ifemendationsatend
     \begingroup
        \def\next{#1}%
        \newlinechar='40
        \immediate\write\emendations{\expandafter\strip\meaning\next}%
     \endgroup\fi
}
\newcommand{\literalexplain}[1]{\ifexplanatend
     \begingroup
        \def\next{#1}%
        \newlinechar='40
        \immediate\write\explanations{\expandafter\strip\meaning\next}%
     \endgroup\fi
}
\newcommand{\literalcontents}[1]{\ifpoemcontentson
     \begingroup
        \def\next{#1}%
        \newlinechar='40
        \immediate\write\poemcontents{\expandafter\strip\meaning\next}%
     \endgroup\fi
}
\newcommand{\literaltextnoteshort}[1]{
\iftextnotesatend
        \begingroup
        \def\next{#1}%
        \immediate\write\textnotes{\expandafter\strip\meaning\next}%
     \endgroup\fi}
\newcommand{\literalemendshort}[1]{\ifemendationsatend
        \begingroup
        \def\next{#1}%
        \immediate\write\emendations{\expandafter\strip\meaning\next}%
     \endgroup\fi}
\newcommand{\literalexplainshort}[1]{\ifexplanatend
        \begingroup
        \def\next{#1}%
        \immediate\write\explanations{\expandafter\strip\meaning\next}%
     \endgroup\fi}
\newcommand{\literalcontentsshort}[1]{\ifpoemcontentson
        \begingroup
        \immediate\write\poemcontents{\expandafter\strip\meaning\next}%
     \endgroup\fi}
\newcommand{\sameword}{$\sim$}
\newcommand{\missingpunct}{${}_{\wedge}$}
%    \end{macrocode}
%    \begin{macro}{\pmccheckifinteger}
% |\pmccheckifinteger{|\meta{num}|}| checks if \meta{num} is an integer.
% If it is, then |\ifinteger| is set TRUE, otherwise it is set FALSE.
% (Code taken from memoir class, and based on Donald Arseneau's 
% |\Lpack{cite}| package).
%    \begin{macrocode}
\newcommand{\pmccheckifinteger}[1]{%
  \protected@edef\@vsa{#1}%
  \ifcat _\ifnum9<1\pmcgobm{#1} _\else A\fi
    \integertrue%
  \else
    \integerfalse%
  \fi%
}
\newif\ifinteger
%    \begin{macro}{\pmcgobm}
% |\pmcgobm{|\meta{num}|}| is defined as \meta{num}. It could be defined as: \\
% |\newcommand{\pmcgobm}[1]{\ifx-#1\expandafter\gobm\else#1\fi}| \\
% which would remove a leading minus sign (hyphen) from its argument
% (|gobm| = gobble minus sign).
% (Code from memoir class and a posting to comp.text.tex by Donald Arseneau on 1997/07/21).
%    \begin{macrocode}
\newcommand{\pmcgobm}[1]{#1}
%    \end{macrocode}
%    \end{macro}
% \subsection{~Commands to make notes and send info to contents} 
%    \begin{macro}{\setlemmarange}
% |\setlemmarange| calculates the range of line numbers for multi-line 
% lemmas.
%    \begin{macrocode}
% \newcommand{\setlemmarange}[1]{
% \setcounter{lemmalines}{#1}
% \ifthenelse{\equal{\value{lemmalines}}{0}}{\rangelemmafalse}{\rangelemmatrue}
% \setcounter{lemmaend}{\theverselinenumber}
% \addtocounter{lemmaend}{\thelemmalines}
% }
\newcommand{\setlemmarange}[1]{%
\pmccheckifinteger{#1}%
\ifinteger
\setcounter{lemmalines}{#1}%
\ifthenelse{\equal{\value{lemmalines}}{0}}{\rangelemmafalse}{\rangelemmatrue}%
\setcounter{lemmaend}{\theverselinenumber}%
\addtocounter{lemmaend}{\thelemmalines}%
\else
\setcounterfromref{lemmaend}{#1}\rangelemmatrue
\fi
\makeatother}
%    \end{macrocode}
%    \end{macro}
%    \begin{macro}{\citerange}
% |\citerange| sets the range of line numbers for multi-line lemmas
%    \begin{macrocode}
\newcommand{\citerange}{%
\ifrangelemma
\theverselinenumber --\thelemmaend
\else
\theverselinenumber
\fi}
%    \end{macrocode}
%    \end{macro}
%    \begin{macro}{\resetlemmacounters}
% |\resetlemmacounters| resets the counts for multiline lemmas. 
%    \begin{macrocode}
\newcommand{\resetlemmacounters}{\rangelemmafalse
\setcounter{lemmalines}{0}%
\setcounter{lemmaend}{0}%
}
%    \end{macrocode}
%    \end{macro}
%    \begin{macro}{\checknoteheaders}
% |\checknoteheaders| Checks and corrects the values in the running 
% headers of notes sections. The running headers are of the form 
% ``Notes to pages xx--yy.'' Every title and every note checks 
% whether the values in the running header should be changed. This 
% command essentially gives a list of strings to be written on the 
% external file for the note section. When these strings are read 
% back in when the file is set, they make a little program that 
% recalculates the values in the headers during the processing of 
% every note. Also typesets the line number or line number range in 
% notes sections.
%    \begin{macrocode}
\newcommand{\checknoteheaders}{\string\setcounter\string{notepageholdernote\string}%
\string{\thepage\string}\string\ifhmode\string\unskip\string\fi%
\string\ifthenelse\string{\string\value\string{notepageholdernote\string}>%
\string\value\string{notepageholdertitle\string}\string}%
\string{\string\mark\string{\string\thenotepageholdernote\string}\string}%
\string{\string\mark\string{\string\thenotepageholdertitle\string}\string}\string\unskip
\string\unskip\string\relax\ \citerange :\string~\string\nolinebreak}%

%    \end{macrocode}
%    \end{macro}
%    \begin{macro}{\textnote}
% The 
% basic |\textnote{}| command, on which everything else is based, has 
% several parts. First, it writes out a little program 
% on the external endnotes file which, when it is read back in when 
% the endnotes are typeset, sets a variable with the value of 
% the page of the line the note is commenting upon and checks to see 
% whether that value is equal to or greater than that of previous notes 
% on that page of notes. Depending upon the outcome of that test, it 
% sets the value of |\mark| with the correct numbers to print the proper 
% running header of the form ``Textual Notes to pp.xx--yy.'' Second, it 
% sends the line number of the line it is a comment upon and the text of 
% the note to the notes section. The optional first argument is the 
% number of lines covered by a multiline lemma. If there is no 
% optional first argument specified, the default is 0. The second 
% argument is the text of the note, which includes the rest of the 
% lemma, other than the line number. It is up to you to suitably 
% abbreviate long lemmas.
% 
%
% 
%    \begin{macrocode}
\newif\iftextnotessinglepar
\newcommand{\textnote}[2][0]{%
\iftextnoteson
    \setlemmarange{#1}%
\immediate\write\textnotes{\checknoteheaders}%
    \literaltextnote{#2}%
    \iftextnotessinglepar%
    \literaltextnote{\par}%
    \fi%
    \fi 
\resetlemmacounters
}
%    \end{macrocode}
%    \end{macro}
%    \begin{macro}{\accidental}
% Accidentals: As it is, the command just tests to see whether 
% accidentals are being included or not, and sets the accidental as a 
% textnote if accidentals are being included. It would not be hard to 
% divert accidentals to another external file in order to compile a 
% separate list of accidentals. Accidentals, by their nature, cannot have multiline 
% lemmas.
%    \begin{macrocode}
\newcommand{\accidental}[1]{%
\iftextnoteson%
 \ifincludeaccidentals%
   \iftextnotesatend\immediate\write\textnotes{\string\nobreak}\fi%
\textnote{#1}%
\iftextnotesatend\immediate\write\textnotes{\string\nobreak}\fi%
    \iftextnotessinglepar%
    \literaltextnote{\par}%
    \fi%
 \fi\fi
}
%    \end{macrocode}
%    \end{macro}
%    \begin{macro}{\tsvariant}
%    \begin{macro}{\tsaccidental}
%    Typescript variants. Treated like accidentals. If they are being 
% collected, they are sent to the textual notes. Again, it would not be 
% hard to collect them separately. Typescript variants can have 
% multiline lemmas.
%    \begin{macrocode}
\newcommand{\tsvariant}[2][0]{%
\iftextnoteson
 \ifincludetypescripts%
 \iftextnotesatend\immediate\write\textnotes{\string\nobreak}\fi%
\textnote[#1]{#2}%
\iftextnotesatend\immediate\write\textnotes{\string\nobreak}\fi%
  \fi\fi
}
\newcommand{\tsaccidental}[1]{\iftextnoteson\ifincludetypescripts\accidental{#1}\fi\fi}
%    \end{macrocode}
%    \end{macro}
%    \end{macro}
%    \begin{macro}{\tsentry}
% Typescript entries. Allows one to to selectively include or exclude 
% typescript entries from lists of variants. If an  entry begins with 
% a comma (as it commonly will, since it will typically be a member 
% of a list of entries, you need not put |\unskip| before the comma, 
% since \textsf{poemscol} will do it for you. Thanks to Donald 
% Arseneau for suggesting how to do this.
%    \begin{macrocode}
\newcommand{\tsentry}[1]{\iftextnoteson\ifincludetypescripts{\@ifnextchar,{\unskip}{% 
�\@ifnextchar;{\unskip}{}% 
}#1}\fi\fi\unskip}
%    \end{macrocode}
%    \end{macro}
%    \begin{macro}{\sources}
%    |\sources{}| is essentially a |\textnote{}| with no line number 
% and no page checking. Useful for typesetting the sources of the text 
% (hence the name) and for sending literal characters to the endnotes 
% file for other purposes.
%    \begin{macrocode}
\newcommand{\sources}[1]{%
\iftextnoteson
    \begingroup
        \newlinechar='40
        \def\next{#1}%
        \immediate\write\textnotes{\expandafter\strip\meaning\next}%
     \endgroup
    \iftextnotessinglepar%
    \literaltextnote{\par}%
    \fi%
 \fi
}
%    \end{macrocode}
%    \end{macro}
%    
%    \subsection{~Emendations and explanatory notes}
%    The difference between these kinds of notes and ordinary 
% |\textnotes{}| is that they must first test to see whether there have 
% been any emendations or explanatory notes for the poem in question. 
% If there have not been prior notes, then a new paragraph must be 
% opened and the page number and title of the poem set in the note 
% section.
%    \begin{macro}{\titletoothernotes}
% |\titletoothernotes| sends the page and title information to the 
% Emendations or Explanatory Notes sections. It is called only for 
% the first emendation or explanatory note for a poem. It is designed 
% so that if the user wishes to define another category of notes in 
% which might not apply to all of the poems to be set --- not all 
% poems, for instance, have emendations --- this command can be used 
% for those notes as well.
%    \begin{macrocode}
\newcommand{\titletoothernotes}{\string\par
\ifputpagenumberinnotes%
\string\argpageref\string{\@compoundlabelscratch\string}%
\string\setcounterfrompageref\string{notepageholdertitle\string}%
\string{\@compoundlabelscratch\string}%
\fi%
\string\textbf\string{\ \the\fulltitleholder\string}}%
%    \end{macrocode}
%    \end{macro}
%    \begin{macro}{\emendation}
%    \begin{macrocode}
\newif\ifemendationssinglepar
\newcommand{\emendation}[2][0]{
\ifemendationson
    \setlemmarange{#1}%
    \ifredundantemendations
       \iftextnoteson
	 \ifrangelemma
	     \textnote[#1]{#2}%
	 \else
	     \textnote{#2}%
	 \fi % ifrangelemma}
    \iftextnotessinglepar%
    \literaltextnote{\par}%
    \fi% iftextnotesinglepar
       \fi %iftextnoteson
   \fi % ifredundantemendations    
    \ifnoemendyet % firstemendation
     \firstemendation
      \fi %noemendyet
      \global\noemendyetfalse
\immediate\write\emendations{\checknoteheaders}%
    \literalemend{#2}%
    \ifemendationssinglepar%
    \literalemend{\par}%
    \fi% ifemendationssinglepar    
\fi % ifemendationson 
\resetlemmacounters
}
%    \end{macrocode}
%    \end{macro}
%  This next is an internal command, 
% called by |\emendation|. There's no need to use it explicitly.
%    \begin{macrocode}
\newcommand{\firstemendation}{\ifemendationsatend
\immediate\write\emendations{\titletoothernotes}\fi
}
%    \end{macrocode}
%    \begin{macro}{\explanatory}
%    Again, |\firstexplanatory| is internal, called by |\explanatory|.
%    \begin{macrocode}
\newif\ifexplanationssinglepar
\newcommand{\explanatory}[2][0]{%
\ifexplanon
    \setlemmarange{#1}%
    \ifnoexplainyet % first explanation
    \firstexplanatory
      \fi %noexplainyet
      \global\noexplainyetfalse
\immediate\write\explanations{\checknoteheaders}%
    \literalexplain{#2}%
    \ifexplanationssinglepar%
    \literalexplain{\par}%
    \fi%
    \fi % ifexplanon 
\resetlemmacounters
}

\newcommand{\firstexplanatory}{\ifexplanatend
\immediate\write\explanations{\titletoothernotes}\fi
}

%    \end{macrocode}
%    \end{macro}
% 
%    \subsection{~Making new notes sections}
%    
%    \begin{macro}{\definenewnotetype} This monster macro has nine 
% sections, and does all of the things required to set up a new end 
% notes section. If you decide to mark a new kind of note in your 
% text with, say |\mynote|, but decide at the last moment against 
% including those notes in your edition, simply don't issue |\putmynotes| at 
% the point it might have appeared, and neither the notes section nor 
% the contents entry for it will be included.
%  
%    \begin{macrocode}
\makeatletter
\long\def\appendtomacro#1#2{%
\begingroup
    \toks@\expandafter{#1#2}%
    \xdef#1{\the\toks@}%
  \endgroup}
\makeatother
\newcommand{\definenewnotetype}[5]{%
% 1.  make an if for this kind of note and set it to true by default;
% make an if for the first instance of the command per poem and set it
% to false; make an if for setting the notes in single note 
% paragraphs and set it to false; make an if for setting
% these notes in two column mode and set it to false
%
% Thanks to Igor Pechtchanski and Ulrich Schwarz.
\expandafter\newif\csname if#1son\endcsname
\global\csname #1sontrue\endcsname
\expandafter\newif\csname ifno#1yet\endcsname
\csname no#1yettrue\endcsname
\expandafter\newif\csname if#1ssinglepar\endcsname
\csname #1singleparfalse\endcsname
\expandafter\newif\csname if#1stwocol\endcsname
\csname #1stwocolfalse\endcsname
% 2. open a new output stream
\expandafter\newwrite\csname #1s\endcsname
% 3. open a file to associate with the stream
\immediate\expandafter\openout\csname #1s\endcsname=\expandafter\jobname .#2
% 4. set up a literalwrite command
 \expandafter\newcommand\csname literal#1\endcsname[1]{
\begingroup
\def\next{##1}%
\newlinechar='40
\immediate\write\expandafter\csname #1s\endcsname{\expandafter\strip\meaning\next}%
\endgroup}
% 5. send heading information to output file
% first, use literaltext to send commands without parameters
\csname literal#1\endcsname{\flushbottom{\par\clearpage
  {\pagestyle{volumetitlestyle}\cleardoublepage}
\thispagestyle{volumetitlestyle}}
\fancyhead{}
\fancyfoot{}
\fancyhead[RO,LE]{{\small\textrm{\thepage}}}}
% then immediate write to send the parameter for the marks in the 
% headers
\immediate\write\csname #1s\endcsname{\string\fancyhead[CO]\string{\string{\string\small
\string{\string\em\ #3\string~\string\mymarks\string}\string}\string}}
\immediate\write\csname #1s\endcsname{\string\fancyhead[CE]\string{\string{\string\small
\string{\string\em\ #3\string~\string\mymarks\string}\string}\string}}
% then literal to finish the headers
\csname literal#1\endcsname{\fancyfoot[C]{}
\mark{3}
\backmattersink}
% then immediate write to send the title of the section to print at 
% the top of the page
\immediate\write\csname #1s\endcsname{
\string\setendnotessectiontitle\string{#4\string}\string{#1notes\string}}
% literal write to turn hyphenation on, set the font for the note 
% section, and turn on two column mode (if the boolean is set)
\csname literal#1\endcsname{\lefthyphenmin=2\backmatterafterheadersink
\tolerance=500\language=0
\normalfont \backmattertextfont}
\immediate\write\csname #1s\endcsname{\string\if#1stwocol}
\csname 
literal#1\endcsname{\begin{multicols}{2}\fi\setlength{\parindent}{-10pt}}%
% \csname literal#1\endcsname{\setlength}{\parindent}{2}}% problem here
% 6. set up a command to write the page and title of the poem for the first 
% instance of this kind of note in a poem
\expandafter\newcommand\csname first#1\endcsname{%
\immediate\write\expandafter\csname #1s\endcsname{\titletoothernotes}%
\immediate\write\csname #1s\endcsname{\string\if#1ssinglepar}%
\immediate\write\csname #1s\endcsname{\string\par}%
\immediate\write\csname #1s\endcsname{\string\fi}%
}
% 7. set up a command to write a note, complete with line numbers
\expandafter\newcommand\csname #1\endcsname[2][0]{%
\setlemmarange{##1}%
\csname ifno#1yet\endcsname%
 \csname first#1\endcsname%
 \fi%
\global\csname no#1yetfalse\endcsname%
\immediate\write\expandafter\csname 
#1s\endcsname{\string\nobreak\string\nobreak\checknoteheaders}%
\csname literal#1\endcsname{##2}%
\csname if#1ssinglepar\endcsname%
\csname literal#1\endcsname{\par}%
\fi%
\resetlemmacounters%
}
% prose  version
\expandafter\newcommand\csname prose#1\endcsname[3][0]{%	
\unskip\proselinelabel{##2}%
\setproselemmarange{##2}{##1}%
\csname ifno#1yet\endcsname%
\csname first#1\endcsname%
\fi%
\global\csname no#1yetfalse\endcsname%
\immediate\write\expandafter\csname #1s\endcsname{\checkprosenoteheaders}%
\csname literal#1\endcsname{##3}%
\csname if#1ssinglepar\endcsname%
\csname literal#1\endcsname{\par}%
\unskip%
\fi%
\resetlemmacounters\unskip%
}
% pm version
\expandafter\newcommand\csname pm#1\endcsname[2][0]{%                                                         	      
     \setpmlemmarange{##1}%
     \csname ifno#1yet\endcsname%
     \csname first#1\endcsname%
     \fi%
\global\csname no#1yetfalse\endcsname%
\immediate\write\expandafter\csname #1s\endcsname{\pmchecknoteheaders\string~}%
\immediate\write\expandafter\csname #1s\endcsname{\pmciterange}%
\csname literal#1\endcsname{##2}%
\csname if#1ssinglepar\endcsname%
\csname literal#1\endcsname{\par}%
\fi%
\pmresetlemmacounters\unskip%
} 
% 8.  modify the end of the poem environment to reset the first
% command per poem boolean (so that the page and title will be set in
% the notes section when the first note is called for a poem)
%  thanks to Heiko Oberdiek
\makeatletter
\appendtomacro{\endpoem}{\global\expandafter\csname no#1yettrue\endcsname}
\makeatother
\makeatletter
\appendtomacro{\endprosesection}{\global\expandafter\csname no#1yettrue\endcsname}
\makeatother
\makeatletter
\appendtomacro{\endpmsection}{\global\expandafter\csname no#1yettrue\endcsname}
\makeatother

% 9. command for closing the output file
%  command for reading in and printing the output file
\expandafter\newcommand\csname put#1s\endcsname{
\newpage\hyphenationforsmall
\csname if#1stwocol\endcsname%
\csname literal#1\endcsname{\end{multicols}}%
\fi %\iftwocol
\immediate\expandafter\closeout\csname #1s\endcsname
\expandafter\input \jobname.#2
\ifpoemcontentson
 \contentsendnotestitle{#5}{#1notes}
\fi
  }
% 10. command for making a marginal mark at a line, giving the page 
% number in the apparatus where a note is to be found
\expandafter\newcommand\csname margref#1\endcsname{%
\addtocounter{margrefnumber}{1}
\immediate\write\expandafter\csname #1s\endcsname{\makemargreflabel}%
\setmargref                                             
}
}
%    \end{macrocode}
%    \end{macro}
% \subsection{~Prose sections}\textsf{poemscol} calls on
% \textsf{lineno} to do all the heavy lifting for making line numbers
% in prose sections and for making endnotes of various kinds for prose
% sections.  First, set up a counter for the line number and for the
% modulo line number for prose sections, and set it to equal that used
% in verse sections.  Then set the font for marginal line numbers to
% the size used in marginal line numbers in verse sections.
%    \begin{macrocode}
\newcounter{proselinenumber}
\setcounter{proselinenumber}{1}
\newcounter{prosemodulo}
\setcounter{prosemodulo}{\value{lineindexrepeat}}
\addtocounter{prosemodulo}{1}
\newcommand{\setprosemodulo}[1]{\modulolinenumbers[#1]}
%    \end{macrocode}
% Then commands for setting titles of prose sections. These are just 
% versions of |\poemtitle|. Fancier versions will follow if there is 
% need.
%    \begin{macrocode}
\newcommand{\prosesectiontitle}[1]{\poemtitle{#1}}
\newcommand{\prosesectiontitlenotitle}[1]{\poemtitlenotitle{#1}}
%    \end{macrocode}
% Next, environments for prose sections. The environment resets (or 
% doesn't, in the second case) the marginal line number. The 
% default modulo line number is that prevailing in verse sections, 
% but you can change it at will.
%    \begin{macrocode}
\setcounter{prosemodulo}{\value{lineindexrepeat}}%
\addtocounter{prosemodulo}{1}%
\ifx\modulolinenumbers\undefined%
 \relax\else                               
\modulolinenumbers[\value{prosemodulo}]% if lineno is called
\fi
\newif\ifinprosesection
\newenvironment{prosesection}
{\inprosesectiontrue\resetlinenumber\begin{linenumbers}%
\renewcommand{\linenumberfont}{\scriptsize}%
\ifnumbersright\rightlinenumbers*\setlength{\linenumbersep}{-\marginparsep}%
%\else\ifnumbersswitch\switchlinenumbers\setlength{\linenumbersep}{\marginparsep}%
\else\ifnumbersswitch\rightlinenumbers*\setlength{\linenumbersep}{-\marginparsep}%
\else\leftlinenumbers*\setlength{\linenumbersep}{\marginparsep}%
\addtolength{\linenumbersep}{5pt}% seems about right
\fi\fi
\noemendyettrue%
\noexplainyettrue%
}
{\end{linenumbers}\inprosesectionfalse}
\newenvironment{prosesectionnoreset}
{\begin{linenumbers}%
\renewcommand{\linenumberfont}{\scriptsize}
\ifnumbersright\rightlinenumbers*\setlength{\linenumbersep}{-\marginparsep}%
%\else\ifnumbersswitch\switchlinenumbers\setlength{\linenumbersep}{\marginparsep}%
\else\ifnumbersswitch\rightlinenumbers*\setlength{\linenumbersep}{-\marginparsep}%
\else\leftlinenumbers*\setlength{\linenumbersep}{\marginparsep}%
\addtolength{\linenumbersep}{5pt}% seems about right
\fi\fi
% \noemendyettrue%
% \noexplainyettrue%
}{\end{linenumbers}}
%    \end{macrocode}
% 
% Set up commands for notes sections.
%    \begin{macro}{\setproselemmastart}
% Finds the line number at the beginning of a lemma. Thanks to Robin 
% Fairbairns and Heiko Oberdiek for |\ifrefundefined|.
% Fairbairns version
%    \begin{macrocode}
% \newcommand*\ifrefundefined[1] {% 
% � �\expandafter\ifx\csname r@#1\endcsname\relax 
% }
% Oberdiek version
\makeatletter 
\newcommand*{\ifrefundefined}[1]{%
  \expandafter\ifx\csname r@#1\endcsname\relax
    \expandafter\@firstoftwo
  \else
    \expandafter\@secondoftwo
  \fi
}
\newcommand*{\@extract@ref}[2]{%
  \expandafter\expandafter\expandafter#1\csname
r@#2\endcsname{}{}\@nil
}

\newcommand*{\@extractref}{%
  \@extract@ref\@car
}

\newcommand*{\@extractpageref}{%
  \@extract@ref\@secondcar
}

\long\def\@secondcar#1#2#3\@nil{#2}

\newcommand*{\setcounterfromref}[2]{%
  \ifrefundefined{#2}{%
    \protect\G@refundefinedtrue
    \@latex@warning{Reference `#2' on page \thepage \space
        undefined}%
    \setcounter{#1}{0}%
  }{%
    \setcounter{#1}{\@extractref{#2}}%
  }%
}
\newcommand*{\setcounterfrompageref}[2]{%
  \ifrefundefined{#2}{%
    \protect\G@refundefinedtrue
    \@latex@warning{Reference `#2' on page \thepage \space
        undefined}%
    \setcounter{#1}{0}%
  }{%
    \setcounter{#1}{\@extractpageref{#2}}%
  }%
}
% 
\makeatother 
\newcommand{\setproselemmastart}[1]{%
\unskip
\setcounterfromref{proselinenumber}{#1}}
%    \end{macrocode}
%    \end{macro}
%    \begin{macro}{\setproselemmarange}
% Finds the end of a multi-line lemma.
%    \begin{macrocode}
% \newcommand{\setproselemmarange}[2]{%
% \setproselemmastart{#1}%
% \setcounter{lemmalines}{#2}%
% \ifthenelse{\equal{\value{lemmalines}}{0}}{\rangelemmafalse}{\rangelemmatrue}%
% \setcounter{lemmaend}{\theproselinenumber}%
% \addtocounter{lemmaend}{\thelemmalines}\unskip%
% }
\newcommand{\setproselemmarange}[2]{%
\setproselemmastart{#1}%
\pmccheckifinteger{#2}%
\ifinteger
\setcounter{lemmalines}{#2}%
\ifthenelse{\equal{\value{lemmalines}}{0}}{\rangelemmafalse}{\rangelemmatrue}%
\setcounter{lemmaend}{\theproselinenumber}%
\addtocounter{lemmaend}{\thelemmalines}\unskip%
\else
% \ref{#2}
\setcounterfromref{lemmaend}{#2}\rangelemmatrue
\fi\unskip
}
%    \end{macrocode}
%    \end{macro}
%    \begin{macro}{\proseciterange}
% Sets the range note for in a multi-line lemma.
%    \begin{macrocode}
\newcommand{\proseciterange}{%
\ifrangelemma%
\theproselinenumber --\thelemmaend%
\else%
\theproselinenumber%
\fi}
%    \end{macrocode}
%    \end{macro}
%    \begin{macro}{\checkprosenoteheaders}
% Updates running header for note sections. Typesets line number in 
% notes.
%    \begin{macrocode}
\newcommand{\checkprosenoteheaders}{\string\setcounter\string{notepageholdernote\string}%
\string{\thepage\string}\string\ifhmode\string\unskip\string\fi
\string\ifthenelse\string{\string\value\string{notepageholdernote\string}>
\string\value\string{notepageholdertitle\string}\string}%
\string{\string\mark\string{\string\thenotepageholdernote\string}\string}%
\string{\string\mark\string{\string\thenotepageholdertitle\string}\string}\string\unskip
\string\unskip\string\relax\ \proseciterange :\string~\string\nolinebreak}%
%    \end{macrocode}
%    \end{macro}
%    \begin{macro}{\prosetextnote}
%    \begin{macrocode}
\newcommand{\proselinelabel}[1]{\ifhmode\unskip\fi\linelabel{#1}}
\newcommand{\prosetextnote}[3][0]{%
\iftextnoteson%
\unskip\proselinelabel{#2}%
\setproselemmarange{#2}{#1}%
\immediate\write\textnotes{\checkprosenoteheaders}%
\literaltextnote{#3}%
\iftextnotessinglepar%
\literaltextnote{\par}%
\fi%
\resetlemmacounters%
\fi\unskip%
}
%    \end{macrocode}
%    \end{macro}
%    \begin{macro}{\proseemendation}
% Notice that it calls |\firstemendation| as usual.
%    \begin{macrocode}
\newcommand{\proseemendation}[3][0]{%
\ifemendationson%
\unskip\proselinelabel{#2}%
\setproselemmarange{#2}{#1}%
    \ifredundantemendations
       \iftextnoteson
	 \ifrangelemma
	     \prosetextnote[#1]{#2}{#3}%
	 \else
	     \prosetextnote{#2}{#3}%
	 \fi % ifrangelemma}
       \fi %iftextnoteson
\iftextnotessinglepar%
\literaltextnote{\par}%
\fi%  iftextnotessinglepar     
   \fi % ifredundantemendations    
    \ifnoemendyet % firstemendation
     \firstemendation
      \fi%noemendyet
      \global\noemendyetfalse
\immediate\write\emendations{\checkprosenoteheaders}%
    \literalemend{#3}%
\ifemendationssinglepar%
\literalemend{\par}%
\fi% ifemendationssinglep
    \resetlemmacounters%
\fi\unskip%
}
%    \end{macrocode}
%    \end{macro}
%    \begin{macro}{\proseexplanatory}
%    \begin{macrocode}
\newcommand{\proseexplanatory}[3][0]{%
\ifexplanon%
\unskip\proselinelabel{#2}%
\setproselemmarange{#2}{#1}%  
    \ifnoexplainyet% firstexplanation
     \firstexplanatory%
      \fi%noexplainyet
      \global\noexplainyetfalse%
\immediate\write\explanations{\checkprosenoteheaders}%
    \literalexplain{#3}%
\ifexplanationssinglepar%
    \literalexplain{\par}%
    \fi% ifexplanationssinglepar
    \resetlemmacounters%
\fi\unskip%
}
%    \end{macrocode}
%    \end{macro}
%    \begin{macro}{\proseaccidental}
%    \begin{macrocode}
\newcommand{\proseaccidental}[2]{
\iftextnoteson
 \ifincludeaccidentals
    \immediate\write\textnotes{\string\nobreak}%
\prosetextnote{#1}{#2}%
\immediate\write\textnotes{\string\nobreak}%
    \iftextnotessinglepar%
    \literaltextnote{\par}%
    \fi%
\fi\fi
}
\newcommand{\prosetsvariant}[3][0]{\iftextnoteson
 \ifincludetypescripts
   \immediate\write
\textnotes{\string\nobreak}%
\prosetextnote[#1]{#2}{#3}%
\immediate\write\textnotes{\string\nobreak}%
    \iftextnotessinglepar%
    \literaltextnote{\par}%
    \fi%
\fi\fi
}
\newcommand{\prosetsaccidental}[2]{\iftextnoteson
\ifincludetypescripts\proseaccidental{#1}{#2}\fi\fi}
%    \end{macrocode}
%    \end{macro}
%    \subsection{~Annotation by Sentence and Paragraph Number}
% Marking sentences. First, some |\if|s, token lists, and counters.
%    \begin{macrocode}
\newif{\ifprosebysentence}
\newif{\ifrunningsentencenumbers}\runningsentencenumberstrue
\newif{\ifmarginsentencenumbers}\marginsentencenumberstrue

\newtoks{\pmnoteheader}

\newcounter{pmsentencenumber}\setcounter{pmsentencenumber}{0}
\newcounter{pmmodulo}\setcounter{pmmodulo}{3}
\newcounter{pmindexcount}\setcounter{pmindexcount}{0}
%    \end{macrocode}
% A starter. Mostly a placeholder. For changes to a whole section.
%    \begin{macrocode}
\newcommand{\setprosebysentence}{\prosebysentencetrue%
}
\newenvironment{pmsection}{
\noemendyettrue%
\noexplainyettrue%
}{}
%    \end{macrocode}
% Formatting for sentence numbers in the running text and in the 
% margin.
%    \begin{macrocode}
\newcommand{\runningsentencenumberformat}[1]{\nobreak%
\textsuperscript{\scriptsize{#1}}\unskip\nobreak}
\newcommand{\marginsentencenumberformat}[1]{\scriptsize{#1}}
\newcommand{\setpmmodulo}[1]{\setcounter{pmmodulo}{#1}%
\addtocounter{pmmodulo}{-1}%
}
\newcommand{\pmnumberstoleft}{%\catcode`\@=11%            
\numbersswitchfalse\numbersrightfalse%
\setlength{\marginparsep}{18pt}%
}
\newcommand{\pmnumberstoright}{%\catcode`\@=11%
\numbersswitchfalse\numbersrighttrue%
\setlength{\marginparsep}{-18pt}
% \setlength{\marginparsep}{18pt}
}
\newcommand{\pmnumbersswitch}{\strictpagechecktrue\numbersswitchtrue\numbersrightfalse%
\setlength{\marginparsep}{18pt}%
}
\newcommand{\pmnumbersgutter}{%\catcode`\@=11%
\pmnumbersswitch%
\pmclreversesideparfalse\numbersguttertrue%
\setlength{\marginparsep}{18pt}%
}
\newcommand{\pmnumbersouter}{%\catcode`\@=11%
\pmnumbersswitch%
\pmclreversesidepartrue\numbersgutterfalse%
\setlength{\marginparsep}{18pt}%
}
%    \end{macrocode}
% Macros for actually writing the numbers in the margin.
%    \begin{macrocode}
\newcommand{\putpmsentencenumber}{%
\putpmmarginnumber{\thepmsentencenumber}%
{\marginsentencenumberformat}}
\newcommand{\putpmmarginnumber}[2]{%
\nolinebreak\begin{marginenvironment}%
  \nolinebreak%
  \ifnumbersswitch%\marginpar{\hfil #2{#1}}
  \pmclsidepar{\hfil #2{#1}}%
    \else% 
    \ifnumbersright%\marginpar{\hfil #2{#1}}
    \pmclrightsidepar{\hfil #2{#1}}%
      \else% \marginpar{#2{#1}\hfil}
      \pmclleftsidepar
{#2{#1}\hfil}%
      \fi%
    \fi%
  \nolinebreak%
  \end{marginenvironment}\nolinebreak%
}
%    \end{macrocode}
%    \begin{macro}{\pmsentence}
% Put this before every sentence. (This is the opposite of the 
% convention in verse sections, where |\verseline| is at the end of 
% each line.
%    \begin{macrocode}
\newcommand{\pmsentence}{%
% increment sentence number at beginning of sentence
\addtocounter{pmsentencenumber}{1}%
\ifrunningsentencenumbers% add running number
\runningsentencenumberformat{\thepmsentencenumber}%
\fi%
\ifmarginsentencenumbers% add marginal number
  \addtocounter{pmindexcount}{1}%
  \leavevmode%
\ifthenelse{\value{pmindexcount}>\value{pmmodulo}}%
{     \putpmsentencenumber\nobreak%
    \setcounter{pmindexcount}{0}%
}{\relax}%             
\fi%
% create header for notes if any are to occur
\ifpmparas%
\global\edef\noteheaderconcat{\thepmparagraph:\thepmsentencenumber}%
\pmnoteheader=\expandafter{\noteheaderconcat}%
\else%
%\global\edef\noteheaderconcat{\thepmsentencenumber:}%
\global\edef\noteheaderconcat{\thepmsentencenumber}                                                    
\pmnoteheader=\expandafter{\noteheaderconcat}%
\fi%
}
%    \end{macrocode}
%    \end{macro}
%    \begin{macro}{\pmsentencetwo}
% Kludge for the second sentence in a line.
%    \begin{macrocode}
\newcommand{\pmsentencetwo}{%
% increment sentence number at beginning of sentence
\addtocounter{pmsentencenumber}{1}%
\ifrunningsentencenumbers% add running number
\runningsentencenumberformat{\thepmsentencenumber}%
\fi%
\ifmarginsentencenumbers% add marginal number
  \addtocounter{pmindexcount}{1}%
  \leavevmode%
\ifthenelse{\value{pmindexcount}>\value{pmmodulo}}%
{     \putpmsentencenumbertwo\nobreak%
    \setcounter{pmindexcount}{0}%
}{\relax}%              
%   \ifnum\thepmindexcount>\thepmmodulo%
%       \putpmsentencenumbertwo\nobreak%
%     \setcounter{pmindexcount}{0}%
%    \fi%
\fi%
% create header for notes if any are to occur
\ifpmparas%
\global\edef\noteheaderconcat{\thepmparagraph:\thepmsentencenumber}%
\pmnoteheader=\expandafter{\noteheaderconcat}%
\else%
\global\edef\noteheaderconcat{\thepmsentencenumber}%
\pmnoteheader=\expandafter{\noteheaderconcat}%
\fi%
}

\newcommand{\putpmsentencenumbertwo}{%
\putpmmarginnumber{\rlap{\phantom{1},~\thepmsentencenumber}}%
{\marginsentencenumberformat}}
%    \end{macrocode}
%    \end{macro}
%    \begin{macro}{\pmsentencethree}
% Kludge for the third sentence in a line.
%    \begin{macrocode}
\newcommand{\pmsentencethree}{%
% increment sentence number at beginning of sentence
\addtocounter{pmsentencenumber}{1}%
\ifrunningsentencenumbers% add running number
\runningsentencenumberformat{\thepmsentencenumber}%
\fi%
\ifmarginsentencenumbers% add marginal number
  \addtocounter{pmindexcount}{1}%
  \leavevmode%
\ifthenelse{\value{pmindexcount}>\value{pmmodulo}}%
{     \putpmsentencenumberthree\nobreak%
    \setcounter{pmindexcount}{0}%
}{\relax}%              
%   \ifnum\thepmindexcount>\thepmmodulo%
%       \putpmsentencenumberthree\nobreak%
%     \setcounter{pmindexcount}{0}%
%    \fi%
\fi%
% create header for notes if any are to occur
\ifpmparas%
\global\edef\noteheaderconcat{\thepmparagraph:\thepmsentencenumber}%
\pmnoteheader=\expandafter{\noteheaderconcat}%
\else%
\global\edef\noteheaderconcat{\thepmsentencenumber}%
\pmnoteheader=\expandafter{\noteheaderconcat}%
\fi%
}

\newcommand{\putpmsentencenumberthree}{%
\putpmmarginnumber{\rlap{\phantom{1,~2},~\thepmsentencenumber}}%
{\marginsentencenumberformat}}
%    \end{macrocode}
%    \end{macro}
%    \begin{macro}{\pmsentencefour}
%    \begin{macrocode}
\newcommand{\pmsentencefour}{%
% increment sentence number at beginning of sentence
\addtocounter{pmsentencenumber}{1}%
\ifrunningsentencenumbers% add running number
\runningsentencenumberformat{\thepmsentencenumber}%
\fi%
\ifmarginsentencenumbers% add marginal number
  \addtocounter{pmindexcount}{1}%
  \leavevmode%
\ifthenelse{\value{pmindexcount}>\value{pmmodulo}}%
{     \putpmsentencenumberfour\nobreak%
    \setcounter{pmindexcount}{0}%
}{\relax}%              
\fi%
% create header for notes if any are to occur
\ifpmparas%
\global\edef\noteheaderconcat{\thepmparagraph:\thepmsentencenumber}%
\pmnoteheader=\expandafter{\noteheaderconcat}%
\else%
\global\edef\noteheaderconcat{\thepmsentencenumber}%
\pmnoteheader=\expandafter{\noteheaderconcat}%
\fi%
}

\newcommand{\putpmsentencenumberfour}{%
\putpmmarginnumber{\rlap{\phantom{1,~2,~3},~\thepmsentencenumber}}%
{\marginsentencenumberformat}}

%    \end{macrocode}
%    \end{macro}
% Macros for numbering paragraphs
%    \begin{macrocode}
\newcounter{pmparagraph}
\setcounter{pmparagraph}{0}
\newif{\ifpmparas}\pmparasfalse
\newif{\ifpmparainmar}\pmparainmartrue
\newif{\ifpmpararunning}\pmpararunningfalse
\newif{\ifpmsentencebypara}% reset sentence counter every paragraph
\pmsentencebyparatrue
%    \end{macrocode}
% Formatting marginal and running paragraph numbers, printing them.
%    \begin{macrocode}
\newcommand{\pmparmarformat}[1]{\textbf{{\small #1}}}
\newcommand{\pmpararunningformat}[1]{\textbf{\small #1}}
\newcommand{\putpmmarparanumber}{%
\putpmmarginnumber{\llap{\thepmparagraph~~}}{\pmparmarformat}}
%    \end{macrocode}
%    \begin{macro}{\pmpara}
% Mark the beginning of each paragraph with |\pmpara|
%    \begin{macrocode}
\newcommand{\pmpara}{\leavevmode%
\pmparastrue\addtocounter{pmparagraph}{1}%
\ifpmparainmar% put in marginal paragraph number
   \putpmmarparanumber
\fi 
\ifpmpararunning% put in running number
  \pmpararunningformat{\thepmparagraph}%
\fi
\ifpmsentencebypara% reset sentence counter
   \setcounter{pmsentencenumber}{0}%
   \setcounter{pmindexcount}{0}%
\fi}
%    \end{macrocode}
%    \end{macro}
%    \begin{macro}{\sentencelabel}
% Makes a label either by sentence number or by paragraph and 
% sentence number, for use by |\ref| and by the various notes commands.
%    \begin{macrocode}
\makeatletter
\def\sentencelabel#1{\@bsphack%
\ifpmparas%
\protected@write\@auxout{}%
	 {\string\newlabel{#1}{{\thepmparagraph :\thepmsentencenumber}{\thepage}}}%
\else%
\protected@write\@auxout{}%
	 {\string\newlabel{#1}{{\thepmsentencenumber}{\thepage}}}%
\fi%
\@esphack%
}
\makeatother
%    \end{macrocode}
%    \end{macro}
% Macros to set up endnote commands for paragraph and sentence
% annotation.
%    \begin{macrocode}
\newtoks{\pmrangeend} % for notes that refer to ranges
% setting the range for the lemma:
\newcommand{\setpmlemmarange}[1]{%
\pmccheckifinteger{#1}%
\ifinteger%
\setcounter{lemmalines}{#1}%
\ifthenelse{\equal{\value{lemmalines}}{0}}{\rangelemmafalse}{\rangelemmatrue}%
\setcounter{lemmaend}{\thepmsentencenumber}%
\addtocounter{lemmaend}{\thelemmalines}%
\pmrangeend={\thelemmaend}%
\else%
\pmrangeend={\ref{#1}}\rangelemmatrue%
\fi%
}
% Setting the string to include ranges in notes:
\newcommand{\pmciterange}{%
\ifrangelemma{\the\pmnoteheader --\the\pmrangeend:}%
\else%
\the\pmnoteheader:%
\fi}
% For running headers
\newcommand{\pmchecknoteheaders}{\string\setcounter\string{notepageholdernote\string}%
\string{\thepage\string}\string\ifhmode\string\unskip\string\fi
\string\ifthenelse\string{\string\value\string{notepageholdernote\string}>
\string\value\string{notepageholdertitle\string}\string}%
\string{\string\mark\string{\string\thenotepageholdernote\string}\string}%
\string{\string\mark\string{\string\thenotepageholdertitle\string}\string}\string\unskip}
% to reset lemma counters at end of note
\newcommand{\pmresetlemmacounters}{\rangelemmafalse%
\setcounter{lemmalines}{0}%
\setcounter{lemmaend}{0}%
\pmrangeend={\relax}%
}
%    \end{macrocode}
%    \begin{macro}{\pmtextnote}
%    \begin{macrocode}
\newcommand{\pmtextnote}[2][0]{% 
\iftextnoteson%
\setpmlemmarange{#1}%
\immediate\write\textnotes{\pmchecknoteheaders\string~}%
    \immediate\write\textnotes{\pmciterange}%
    \literaltextnote{#2}%
    \iftextnotessinglepar%
    \literaltextnote{\par}%
    \fi%
    \fi%
\pmresetlemmacounters%
\unskip}
%    \end{macrocode}
%    \end{macro}
%    \begin{macro}{\pmexplanatory}
%    \begin{macrocode}
\newcommand{\pmexplanatory}[2][0]{%
\ifexplanon%
\setpmlemmarange{#1}%
    \ifnoexplainyet % first explanation
    \firstexplanatory%
      \fi %noexplainyet
      \global\noexplainyetfalse%
\immediate\write\explanations{\pmchecknoteheaders\string~}%
\immediate\write\explanations{\pmciterange}%
    \literalexplain{#2}%
    \ifexplanationssinglepar%
    \literalexplain{\par}%
    \fi %ifexplanationssinglepar
    \fi % ifexplanon 
\pmresetlemmacounters%
\unskip%
}
%    \end{macrocode}
%    \end{macro}
%    \begin{macro}{\pmemendation}
%    \begin{macrocode}
\newcommand{\pmemendation}[2][0]{%
\ifemendationson%
    \setpmlemmarange{#1}%
    \ifredundantemendations%
       \iftextnoteson%
	 \ifrangelemma%
	     \pmtextnote[#1]{#2}%
	 \else%
	     \pmtextnote{#2}%
	 \fi % ifrangelemma}
       \fi %iftextnoteson
   \fi % ifredundantemendations    
    \ifnoemendyet % firstemendation
     \firstemendation%
      \fi %noemendyet
      \global\noemendyetfalse%
\immediate\write\emendations{\pmchecknoteheaders\string~}%
\immediate\write\emendations{\pmciterange}%
    \literalemend{#2}%
\ifemendationssinglepar%
    \literalemend{\par}%
    \fi% ifemendationssinglepar 
    \fi % ifemendationson 
\pmresetlemmacounters%
}
%    \end{macrocode}
%    \end{macro}
%    \begin{macro}{\pmaccidental}
%    \begin{macrocode}
\newcommand{\pmaccidental}[1]{%
\iftextnoteson%
 \ifincludeaccidentals%
    \immediate\write\textnotes{\string\nobreak}%
\pmtextnote{#1}%
\immediate\write\textnotes{\string\nobreak}%
    \iftextnotessinglepar%
    \literaltextnote{\par}%
    \fi%
\fi\fi%
}
%    \end{macrocode}
%    \end{macro}
%    \begin{macro}{\pmtsvariant}
%    \begin{macrocode}
\newcommand{\pmtsvariant}[2][0]{\iftextnoteson%
 \ifincludetypescripts%
   \immediate\write\textnotes{\string\nobreak}%
\pmtextnote[#1]{#2}%
\immediate\write\textnotes{\string\nobreak}%
    \iftextnotessinglepar%
    \literaltextnote{\par}%
    \fi%
\fi\fi%
}
\newcommand{\pmtsaccidental}[1]{\iftextnoteson\ifincludetypescripts\pmaccidental{#1}\fi\fi}
%    \end{macrocode}
%    \end{macro}
%\subsection{~Using Footnotes} |\texnotesatfoot| has to redefine 
%    |\sources| and a few others as well. 
%    \begin{macro}{\textnotesatfoot}
%    \begin{macrocode}
\newcommand{\textnotesatfoot}{% first redefine textnotes
\global\textnotesontrue%
\global\textnotesatendfalse%
\ifsourcesfootnotespara%
\DeclareNewFootnote[para]{A}%
\else%
\DeclareNewFootnote[plain]{A}%
\fi%
\iftextfootnotespara%
\DeclareNewFootnote[para]{B}%
\else%
\DeclareNewFootnote[plain]{B}%
\fi%
\renewcommand{\textnote}[2][0]{\setlemmarange{##1}%
\FootnotetextB{\relax}{\citerange:~##2}%
\resetlemmacounters}% 
% then redefine sources
\renewcommand{\sources}[1]{\FootnotetextA{\relax}{##1}}%
% then prose textnote
\renewcommand{\prosetextnote}[3][0]{%
\unskip\proselinelabel{##2}%
\setproselemmarange{##2}{##1}%
\FootnotetextB{\relax}{\proseciterange:~##3}\resetlemmacounters}%
% then pmtextnote
\renewcommand{\pmtextnote}[2][0]{\setpmlemmarange{##1}%
\FootnotetextB{\relax}{\pmciterange~##2}\pmresetlemmacounters}
}
%    \end{macrocode}
%    \end{macro}
%    \begin{macro}{\emendationsatfoot}
%    \begin{macrocode}
\newcommand{\emendationsatfoot}{%
\global\emendationsontrue%
\global\emendationsatendfalse%
\ifemendationfootnotespara%
\DeclareNewFootnote[para]{C}%
\else%
\DeclareNewFootnote[plain]{C}%
\fi%
\renewcommand{\emendation}[2][0]{\setlemmarange{##1}%
\FootnotetextC{\relax}{\citerange:~##2}%
\resetlemmacounters}%
\renewcommand{\proseemendation}[3][0]{\unskip\proselinelabel{##2}%
\setproselemmarange{##2}{##1}%
\FootnotetextC{\relax}{\proseciterange:~##3}\resetlemmacounters}%
\renewcommand{\pmemendation}[2][0]{\setpmlemmarange{##1}%
\FootnotetextC{\relax}{\pmciterange~##2}\pmresetlemmacounters}
}
%    \end{macrocode}
%    \end{macro}
%    \begin{macro}{\explanationsatfoot}
%    \begin{macrocode}
\newcommand{\explanationsatfoot}{%
\global\explanontrue%
\global\explanatendfalse%
\ifexplanfootnotespara%
\DeclareNewFootnote[para]{D}%
\else%
\DeclareNewFootnote[plain]{D}%
\fi%
\renewcommand{\explanatory}[2][0]{\setlemmarange{##1}%
\FootnotetextD{\relax}{\citerange:~##2}%
\resetlemmacounters}%
\renewcommand{\proseexplanatory}[3][0]{\unskip\proselinelabel{##2}%
\setproselemmarange{##2}{##1}%
\FootnotetextD{\relax}{\proseciterange:~##3}}
\renewcommand{\pmexplanatory}[2][0]{\setpmlemmarange{##1}%
\FootnotetextD{\relax}{\pmciterange~##2}\pmresetlemmacounters}%
}
%    \end{macrocode}
%    \end{macro}
%    \begin{macro}{\bibleverse}
%    \begin{macrocode}
\newcommand{\bibleverse}{\pmsentence}
\newcommand{\biblechapter}{\pmpara}
%    \end{macrocode}
%    \end{macro}
%    \subsection{~Embarrassing kludges} Because the dirty trick 
% mentioned above writes out textnotes as a ribbon one word wide, 
% quoted poetry in the notes sections will leave a blank line at the 
% end, which will be interpreted as a paragraph break when the file is 
% read back in by \LaTeX. These kludges correct that problem.
%    \begin{macrocode}
\newcommand{\quotedversecorrectiontextnote}[1][0pt]
{\iftextnoteson%
\immediate\write\textnotes{\string\noindent\string\kern-#1}%
\fi}
\newcommand{\quotedversecorrectionemendation}[1][0pt]
{\ifemendationson%
\immediate\write\emendations{\string\noindent\string\kern-#1}%
\fi}
\newcommand{\quotedversecorrectionexplanatory}[1][0pt]
{\ifexplanon%
\immediate\write\explanations{\string\noindent\string\kern-#1}%
\fi}
%    \end{macrocode}
%    \subsection{~Marking stanza breaks on page turns} Use the macro 
% below to change the symbol used to mark cases in which there is or is 
% not a stanza break at a page turn.
%    \begin{macrocode}
\newcommand{\stanzaatbottom}[1]{\global\edef\stanzaatbottomvalue{#1}}
\stanzaatbottom{*}
\newcommand{\nostanzaatbottom}[1]{\global\edef\nostanzaatbottomvalue{#1}}
\nostanzaatbottom{\relax}
%    \end{macrocode}
% \subsection{~Parallel Texts}
% First, three utility macros from the memoir class:
%  \begin{macro}{\cleartorecto}
% A repeat of |\cleardoublepage|; clears to a recto (odd-numbered) page.
%    \begin{macrocode}
\def\cleartorecto{\clearpage\if@twoside \ifodd\c@page\else
  \hbox{}\thispagestyle{empty}%
  \newpage\if@twocolumn\hbox{}\newpage\fi\fi\fi}
%    \end{macrocode}
%    \end{macro}
%
%    \begin{macro}{\cleartoverso}
% Clears to a verso (even-numbered) page.
%    \begin{macrocode}
\def\cleartoverso{\clearpage\if@twoside 
  \ifodd\c@page\hbox{}\thispagestyle{empty}%
  \newpage\if@twocolumn\hbox{}\newpage\fi\fi\fi}
%    \end{macrocode}
%    \end{macro}
%
%   
%    \begin{macro}{\ifenv}
%  Macro for testing whether one is in a particular environment.
%    \begin{macrocode}
\makeatletter
\def\ifenv#1{%
      \def\reserved@a{#1}% 
      \ifx\reserved@a\@currenvir 
        \expandafter\@firstoftwo 
      \else 
        \expandafter\@secondoftwo 
      \fi 
} 
\makeatother
%    \end{macrocode}
%    \end{macro}
% Macros for forcing the marginal line numbers to the left side or 
% the right side. These marginal line numbers, unlike those made with 
% |\marginpar|, do not float. This code is modeled after code from 
% the memoir class, by Peter Wilson.
% 
%    \begin{macro}{\pmclleftsidepar}
% As in the memoir class, these sidepars do not float.
%    \begin{macrocode}
\newdimen\pmclsavsk
\newcount\pmclsavsf
\def\pmclbsphack{%
  \relax
  \ifhmode
    \pmclsavsk\lastskip
    \pmclsavsf\spacefactor
  \fi}
\def\pmclesphack{%
  \relax
  \ifhmode
    \spacefactor\pmclsavsf
    \ifdim\pmclsavsk>0pt
      \ignorespaces
    \fi
  \fi}
% \newcommand{\pmclleftsidepar}[1]{%
% \pmclbsphack\leavevmode\vadjust{%
%     \rlap{\kern-\parindent%
% 	  \kern -\marginparsep \kern -\marginparwidth % at left
%       \vbox to 0pt{%
%        \kern \pmclsideparvshift%           % vertical shift to align top text lines
%        \begin{minipage}{\marginparwidth}%
% 	#1%
% 	\end{minipage}%
%     \vss}}}\pmclesphack%
% } % version 2.44
\newcommand{\pmclleftsidepar}[1]{%
\pmclbsphack\strut\vadjust{% changed
\rlap{\kern-\parindent%
\kern -\marginparsep \kern -\marginparwidth% at left
%       \vbox to 0pt{%
%        \kern \pmclsideparvshift%           % vertical shift to align top text lines
%        \begin{minipage}{\marginparwidth}%
\setbox0=\vtop to 0pt{% added
\begin{minipage}[t]{\marginparwidth}% added
#1%
\end{minipage}%
%     \vss}}}\pmclesphack%
\vss}% changed
\vtop to 0pt{\kern\pmclsideparvshift % default should be 0pt 
\kern-\dp\strutbox \kern-\ht0 \box0 \vss}}}\pmclesphack}% changed

\newcommand{\pmclrightsidepar}[1]{%
% \pmclbsphack\leavevmode\vadjust{%
\pmclbsphack\strut\vadjust{%
\rlap{\kern-\parindent%
% 	  \kern \textwidth \kern -\marginparsep        % at right
\kern\textwidth\kern\marginparsep% at right
%       \vbox to 0pt{%
%        \kern \pmclsideparvshift%           % vertical shift to align top text lines
%        \begin{minipage}{\marginparwidth}%
\setbox0=\vtop to 0pt{% added
\begin{minipage}[t]{\marginparwidth}% added
#1%
\end{minipage}%
%     \vss}}}\pmclesphack%
\vss}% changed
\vtop to 0pt{\kern\pmclsideparvshift% default should be 0pt 
\kern-\dp\strutbox \kern-\ht0 \box0 \vss}}}\pmclesphack}% changed

%    \end{macrocode}
%    \end{macro}
  
% Booleans for parallel texts
%    \begin{macrocode}
\newif\ifparalleltexts
\paralleltextsfalse
\newif\ifrecto
\rectofalse
\newif\ifverso
\versofalse
%    \end{macrocode}
% Booleans are needed to describe situations in which poems and stanzas 
% are left open at the page break. |\finishrectopage| and 
% |\finishversopage| will close poems and stanzas (so as not to 
% generate errors because of environments left open), although as far 
% as the poem is concerned the stanza or poem may still be open. (I 
% will say that the stanza or poem in such a case is ``artificially 
% closed.'') There are two of them for each case of artificial closure. 
% The first boolean (for each stanza or poem artificially closed) 
% must reflect whether the stanza or poem has been (artificially) 
% closed, to reopen on the next page of the same kind (recto or verso). 
% If a poem or stanza has been (artificially) closed, the |\mark| will 
% not be changed (so that it will still reflect whether the reader 
% should understand there to be a stanza break at the bottom of the 
% page or not). The second boolean for each case of artificial closure 
% reflects whether the |\mark| has been changed or not.
%    \begin{macrocode}
\newif\ifrectopoempending
\rectopoempendingfalse
\newif\ifversopoempending
\versopoempendingfalse
\newif\ifrectostanzapending
\rectostanzapendingfalse
\newif\ifversostanzapending
\versostanzapendingfalse
\newif\ifrectostanzastillopen
\rectostanzastillopenfalse
\newif\ifversostanzastillopen
\versostanzastillopenfalse
\newif\ifrectopoemstillopen
\rectopoemstillopenfalse
\newif\ifversopoemstillopen
\versopoemstillopenfalse
% the same for prose sections
\newif\ifrectoprosesectionpending
\rectoprosesectionpendingfalse
\newif\ifversoprosesectionpending
\versoprosesectionpendingfalse
\newif\ifrectoprosesectionstillopen
\rectoprosesectionstillopenfalse
\newif\ifversoprosesectionstillopen
\versoprosesectionstillopenfalse
% the same for quoted verse
\newif\ifrectoquotedversepending
\rectoquotedversependingfalse
\newif\ifversoquotedversepending
\versoquotedversependingfalse
\newif\ifrectoquotedversestillopen
\rectoquotedversestillopenfalse
\newif\ifversoquotedversestillopen
\versoquotedversestillopenfalse
%    \end{macrocode}
%    A further set of booleans is required for emendations and 
% explanatory notes. Again the idea is that if a poem is artificially 
% closed, but has already had its first emendation or explanatory note 
% written to the external notes files, \textsf{poemscol} will remember that 
% the next emendation or explanatory note it sees is not the first one 
% for that poem and thus won't set the heading information for that 
% poem in the notes section all over again.
%    \begin{macrocode}
\newif\ifrectoemendationspending
\rectoemendationspendingfalse
\newif\ifversoemendationspending
\versoemendationspendingfalse
\newif\ifrectoexplanationspending
\rectoexplanationspendingfalse
\newif\ifversoexplanationspending
\versoexplanationspendingfalse
%    \end{macrocode}
%    Counters for parallel texts: New counters are needed, not only for the recto and verso lines, but 
% also for the scratch value which determines whether or not it is time to 
% output a marginal line number. These are saved at the end of every 
% page, and restored at the beginning of every page of the same kind.
%    \begin{macrocode}
\newcounter{rectolinecounter}
\setcounter{rectolinecounter}{1}
\newcounter{rectoindexscratch}
\setcounter{rectoindexscratch}{1}
\newcounter{versolinecounter}
\setcounter{versolinecounter}{1}
\newcounter{versoindexscratch}
\setcounter{versoindexscratch}{1}
%    \end{macrocode}
%    \begin{macro}{\startparalleltexts}
% Clears to a verso (even-numbered) page, fixes where to put line 
% numbers, modifies poem and stanza environments, changes where 
% marginal line numbers are set, sets the booleans.   
%    \begin{macrocode}
\newcommand{\startparalleltexts}{%
\leavevmode
\cleartoverso
\paralleltextstrue 
\renewenvironment{poem}{\raggedright
     \language=255%no hyphenation in verse
     \noemendyettrue
     \noexplainyettrue
     \setcounter{verselinenumber}{0}\setcounter{printlineindex}{0}
     \nobreak\begin{pmclverse}   
     \inpoemtrue\nobreak\mark{\relax}}
     % must end poem if parallel text. Don't worry, 
     % begin recto page and begin verso page will 
     % reopen the poem and reset the line counter to the right value
     % if it was finish recto page or finish verso page
     % that closed the poem rather than your own declaration.
{\ifversopoemstillopen 
       \end{pmclverse}
	\inpoemfalse\mark{\relax}
   \else
      \ifrectopoemstillopen
	  \end{pmclverse}
	  \inpoemfalse\mark{\relax}
      \else
\end{pmclverse}
     \inpoemfalse\mark{\relax}
     \goodbreak\afterpoemskip%\bigskip
\fi\fi     }
\renewenvironment{stanza}{\penalty-100\instanzatrue\mark{\nostanzaatbottomvalue\ }}
% must end stanza if parallel text. Don't worry, 
% begin recto page and begin verso page will 
% reopen the stanza and reset the line counter to the right value
% if it was finish recto page or finish verso page
% that closed the stanza rather than your own declaration. 
{\ifversostanzastillopen
       \relax
   \else
      \ifrectostanzastillopen
	  \relax
      \else
	 \nobreak\mark{\stanzaatbottomvalue\ }\nobreak
\ifspeciallinelock%
\relax\else%
\nolinebreak\incrementverselinenumber%
\fi%
\speciallinelockfalse%	
	 \nobreak
	 \instanzafalse

	 \stanzaskip
\fi\fi
}
\renewcommand{\putverselinenumber}{\nolinebreak\begin{marginenvironment}%
  \nolinebreak
%  \marn{\hfil\textrm{\theverselinenumber}}
%  \marginpar{\hfil\textrm{\scriptsize\theverselinenumber}}
  \pmclleftsidepar{\textrm{\scriptsize\theverselinenumber}}%  
  \nolinebreak
  \end{marginenvironment}%
  \nolinebreak
}
}
\newenvironment{parallelverse}{\startparalleltexts}
{\finishparalleltexts}
\newtoks{\versotitleholder}
\newtoks{\rectotitleholder}
\newcommand{\makeversotitleholder}[1]{%
\versotitleholder={#1}%
}
\newcommand{\makerectotitleholder}[1]{%
\rectotitleholder={#1}%
}
%    \end{macrocode}
%    \end{macro}
%    \begin{macro}{\versopoemtitle}
%    \begin{macrocode}
\makeatletter
\newcommand{\versopoemtitle}[1]{%
\versoset@p@emtitle{#1}{\poemtitlepenalty}{\poemtitlefont}%
{\relax}{\nobreak\par\nobreak\afterpoemtitleskip\nobreak}{\z@}%
\c@ntentsinfo{#1}{\contentsindentone}{\contentspoemtitlefont}%
{\contentsindenttwoamount}%
}
\newcommand{\versoset@p@emtitle}[6]{\pmclversoresetsettitleinnotes%
\setlength{\titleentryoverrun}{#6}%
{#2}%
{#3 \begin{titleentryenvironment}%
{\ifcentertitleson\begin{center}#4 #1\end{center}\else #4 #1\fi}%
\end{titleentryenvironment}}%
{#5}%
\m@kep@emlabel
\versom@ken@teholder{#1}%
\setlength{\titleentryoverrun}{\z@}}
\newcommand{\versom@ken@teholder}[1]{%
\ifsinglelinetitle
\versotitleholder={#1}%
\global\edef\@compoundlabelscratch{poem\thepoemnumber}%
\else\iftitlefirstline
       \versotitleholder={#1}%
       \global\edef\@compoundlabelscratch{poem\thepoemnumber}%
     \else\iftitlemiddleline
	     \titlesofar=\versotitleholder
	     \titleincrement={~#1}%
	     \global\edef\titleconcat{\the\titlesofar \the\titleincrement}%
	     \versotitleholder=\expandafter{\titleconcat}%
	     \global\edef\@compoundlabelscratch{poem\thepoemnumber}%
	  \else
	      \titlesofar=\versotitleholder
	      \titleincrement={~#1}%
	      \global\edef\titleconcat{\the\titlesofar \the\titleincrement}%
	      \versotitleholder=\expandafter{\titleconcat}%
	      \global\edef\@compoundlabelscratch{poem\thepoemnumber}%
	   \fi
       \fi
  \fi
}
\newcommand{\versotitletoothernotes}{\string\par
\ifputpagenumberinnotes%
\string\argpageref\string{\@compoundlabelscratch\string}%
\string\setcounterfrompageref\string{notepageholdertitle\string}%
\string{\@compoundlabelscratch\string}%
\fi%
\string\textbf\string{\ \the\versotitleholder\string}
}%
\newcommand{\versopoemtitlenocontents}[1]{%
\versoset@p@emtitle{#1}{\poemtitlepenalty}{\poemtitlefont}%
{\relax}{\nobreak\par\nobreak\afterpoemtitleskip\nobreak}{\z@}%
}
\newcommand{\versotitleinnotescheck}{\ifparalleltexts\let\titletoothernotes=\versotitletoothernotes\fi}
\makeatother
%    \end{macrocode}
%    \end{macro}
%    \begin{macro}{\rectopoemtitle}
%    \begin{macrocode}
\makeatletter
\newcommand{\rectopoemtitle}[1]{%
\rectoset@p@emtitle{#1}{\poemtitlepenalty}{\poemtitlefont}%
{\relax}{\nobreak\par\nobreak\afterpoemtitleskip\nobreak}{\z@}%
\c@ntentsinfo{#1}{\contentsindentone}{\contentspoemtitlefont}%
{\contentsindenttwoamount}%
}
\newcommand{\rectoset@p@emtitle}[6]{\pmclrectoresetsettitleinnotes
\setlength{\titleentryoverrun}{#6}%
{#2}%
{#3 \begin{titleentryenvironment}%
{\ifcentertitleson\begin{center}#4 #1\end{center}\else #4 #1\fi}%
\end{titleentryenvironment}}%
{#5}%
\m@kep@emlabel
\rectom@ken@teholder{#1}%
\setlength{\titleentryoverrun}{\z@}}
\newcommand{\rectom@ken@teholder}[1]{%
\ifsinglelinetitle
\rectotitleholder={#1}%
\global\edef\@compoundlabelscratch{poem\thepoemnumber}%
\else\iftitlefirstline
       \rectotitleholder={#1}%
       \global\edef\@compoundlabelscratch{poem\thepoemnumber}%
     \else\iftitlemiddleline
	     \titlesofar=\rectotitleholder
	     \titleincrement={~#1}%
	     \global\edef\titleconcat{\the\titlesofar \the\titleincrement}%
	     \rectotitleholder=\expandafter{\titleconcat}%
	     \global\edef\@compoundlabelscratch{poem\thepoemnumber}%
	  \else
	      \titlesofar=\rectotitleholder
	      \titleincrement={~#1}%
	      \global\edef\titleconcat{\the\titlesofar \the\titleincrement}%
	      \rectotitleholder=\expandafter{\titleconcat}%
	      \global\edef\@compoundlabelscratch{poem\thepoemnumber}%
	   \fi
       \fi
  \fi
}
\newcommand{\rectotitletoothernotes}{\string\par
\ifputpagenumberinnotes%
\string\argpageref\string{\@compoundlabelscratch\string}%
\string\setcounterfrompageref\string{notepageholdertitle\string}%
\string{\@compoundlabelscratch\string}%
\fi%
\string\textbf\string{\ \the\rectotitleholder\string}
}%
\newcommand{\rectopoemtitlenocontents}[1]{%
\rectoset@p@emtitle{#1}{\poemtitlepenalty}{\poemtitlefont}%
{\relax}{\nobreak\par\nobreak\afterpoemtitleskip\nobreak}{\z@}%
%\c@ntentsinfo{#1}{\contentsindentone}{\contentspoemtitlefont}%
%{\contentsindenttwoamount}%
%\t@xtnotesinfo{#1}%
}
\newcommand{\rectotitleinnotescheck}{\ifparalleltexts\let\titletoothernotes=\rectotitletoothernotes\fi}
\makeatother
%    \end{macrocode}
%    \end{macro}
%    \begin{macro}{\saveversoline}
%    Macros for saving and restoring line numbers:
%    \begin{macrocode}
\newcommand{\saveversoline}{\setcounter{versolinecounter}%
 {\value{verselinenumber}}%
\setcounter{versoindexscratch}{\value{printlineindex}}}   
%    \end{macrocode}
%    \end{macro}
%    \begin{macro}{\restoreversoline}
%    \begin{macrocode}
\newcommand{\restoreversoline}{\setcounter{verselinenumber}%
  {\value{versolinecounter}}%
\setcounter{printlineindex}{\value{versoindexscratch}}}
%    \end{macrocode}
%    \end{macro}
%    \begin{macro}{\saverectoline}
%    \begin{macrocode}
\newcommand{\saverectoline}{\setcounter{rectolinecounter}%
 {\value{verselinenumber}}%
\setcounter{rectoindexscratch}{\value{printlineindex}}}
%    \end{macrocode}
%    \end{macro}
%    \begin{macro}{\restorerectoline}
%    \begin{macrocode}
\newcommand{\restorerectoline}{\setcounter{verselinenumber}%
  {\value{rectolinecounter}}%
\setcounter{printlineindex}{\value{rectoindexscratch}}}
%    \end{macrocode}
%    \end{macro}
% Macros for line cross references:
% 
% You will probably want to use line cross references to show which
% lines on the recto page correspond to which lines on the verso page.
% When you do this, you will set, in the margin of a line on the recto
% page, the line number of the corresponding line on the verso side, in
% square brackets.  You may mark that line number with a distinctive
% identifier.  |\synchroflag| is a token register to hold that
% identifier, and |\setsynchroflag| is a command to put the identifier
% in the register.  For instance, if you are setting the 1805 version of
% Wordsworth's \emph{The Prelude} on the verso, and the 1850 version on
% the recto, you might wish to set the ``synchroflag'' to ``1805'' so
% that the line cross reference will look something like ``[1805:
% 147]''.
%    \begin{macro}{\setsynchroflag}
%    \begin{macrocode}
\newtoks{\synchroflag}
\newcommand{\setsynchroflag}[1]{\synchroflag={#1}}
\setsynchroflag{\relax}
%    \end{macrocode}
%    \end{macro}
% Here are commands for defining and calling ``synchrolabels'', line 
% cross references that tie verso and recto pages:
%    \begin{macro}{\synchrolabel}
%    \begin{macrocode}
\newcommand{\synchrolabel}[1]{\poemlinelabel{#1}}
\newcommand{\hfilll}{\hskip 0pt plus 1 filll}
\newcommand{\synchroref}[1]{%
\pmclrightsidepar{{\textrm{\scriptsize[\the\synchroflag\ref{#1}]}}}%
%  \hfill\rlap{{\textrm{\scriptsize[\the\synchroflag\ref{#1}]}}}                           
%  \ifnum\theprintlineindex=\thelineindexrepeat
%  \hspace{-7pt}
% \fi
 }
%    \end{macrocode}
%    \end{macro}
%    Finally, the |\startversopage|, |\finishversopage|, |\startrectopage| 
% |\finishrectopage| and |\finishparalleltexts| macros:
%    \begin{macro}{\startversopage}
%    \begin{macrocode}
\newcommand{\startversopage}{
\ifrectostanzastillopen
  \end{stanza}
  \rectostanzastillopenfalse
  \end{poem} 
   \rectopoemstillopenfalse
\fi
\ifrectopoemstillopen % from previous page
  \end{poem} 
  \rectopoemstillopenfalse
\fi 
\ifversopoempending % from previous verso
   \begin{poem}
  \global\versopoempendingfalse
\fi
\ifversostanzapending % from previous verso
  \begin{stanza}
  \global\versostanzapendingfalse
\fi
\ifversoemendationspending
  \noemendyetfalse
  \global\versoemendationspendingfalse
\fi
\ifversoexplanationspending
  \noexplainyetfalse
  \global\versoexplanationspendingfalse
\fi
\rectofalse\versotrue\restoreversoline
\let\titletoothernotes=\versotitletoothernotes}
%    \end{macrocode}
%    \end{macro}
%    \begin{macro}{\finishversopage}
%    \begin{macrocode}
\newcommand{\finishversopage}{
\saveversoline\versofalse\
\ifinpoem
  \global\versopoempendingtrue
  \versopoemstillopentrue
\fi
\ifinstanza
   \global\versostanzapendingtrue 
   \versostanzastillopentrue 
   \global\versopoempendingtrue
     \versopoemstillopentrue 
\fi
\ifnoemendyet
 \relax
 \else
   \global\versoemendationspendingtrue
    \noemendyetfalse
\fi
\ifnoexplainyet
   \relax
    \else
       \global\versoexplanationspendingtrue
       \noexplainyetfalse 
\fi
% \clearpage % keep in comment until footnotes problems are worked out
}
%    \end{macrocode}
%    \end{macro}
%    \begin{macro}{\startrectopage}
%    \begin{macrocode}
\newcommand{\startrectopage}{
\ifversostanzastillopen
  \end{stanza} 
  \versostanzastillopenfalse
  \end{poem} 
  \versopoemstillopenfalse
\fi 
\ifversopoemstillopen
  \end{poem} 
  \versopoemstillopenfalse
\fi
\ifrectopoempending
  \begin{poem}
  \global\rectopoempendingfalse
\fi
\ifrectostanzapending
  \begin{stanza} 
  \global\rectostanzapendingfalse
\fi
\ifrectoemendationspending
  \noemendyetfalse
  \global\rectoemendationspendingfalse
\fi
\ifrectoexplanationspending
  \noexplainyetfalse
  \global\rectoexplanationspendingfalse
\fi
\versofalse\rectotrue\restorerectoline
\let\titletoothernotes=\rectotitletoothernotes}
%    \end{macrocode}
%    \end{macro}
%    \begin{macro}{\finishrectopage}
%    \begin{macrocode}
\newcommand{\finishrectopage}{\saverectoline\rectofalse
\ifinpoem
  \global\rectopoempendingtrue
  \rectopoemstillopentrue 
\fi
\ifinstanza
   \global\rectostanzapendingtrue   
   \rectostanzastillopentrue 
   \global\rectopoempendingtrue      
     \rectopoemstillopentrue 
\fi
\ifnoemendyet
  \relax
  \else
    \global\rectoemendationspendingtrue
    \noemendyetfalse
\fi
\ifnoexplainyet
  \relax
  \else
    \global\versoexplanationspendingtrue
    \noexplainyetfalse 
\fi
% \clearpage % until footnote problem is solved
}
%    \end{macrocode}
%    \end{macro}
%    \begin{macro}{\finishparalleltexts}
%    \begin{macrocode}
\newcommand{\finishparalleltexts}{
\ifinstanza
\end{stanza}
\fi
\ifinpoem
\end{poem}
\fi
\ifinprosesection
\end{prosesection}
\fi
\ifinquotedverse
\end{quotedverse}
\fi
\rectofalse\versofalse\paralleltextsfalse
\makeatletter\pmclresetsettitleinnotes\makeatother
}
%    \end{macrocode}
%    \end{macro}
%    \begin{macro}{\makeversotextnotes}
% Next, the versotextnotes section:
%    \begin{macrocode}
% tokens for changing the names of these notes
\newtoks{\versotextnotesname}
\versotextnotesname={TEXTUAL NOTES}
\newcommand{\changeversotextnotesname}[1]{\versotextnotesname={#1}}
\newtoks{\versotextnotesheadername}
\versotextnotesheadername={Textual Notes to}
\newcommand{\changeversotextnotesheader}[1]{\versotextnotesheadername={#1}}
\newtoks{\versotextnotescontentsname}
\versotextnotescontentsname={Textual Notes}
\newcommand{\changeversotextnotescontentsname}[1]{%
\versotextnotescontentsname={#1}}
% two column notes
\newif\ifversotextnotestwocol
\versotextnotestwocolfalse
% versotextnotes section setup
\newcommand{\makeversotextnotes}{\global\textnotesontrue 
  \newwrite\versotextnotes
  \global\verselinenumberstrue
   \immediate\openout\versotextnotes=\jobname.vtn
   \literalversotextnote{\flushbottom{\par\clearpage
   {\pagestyle{volumetitlestyle}\cleardoublepage}
\thispagestyle{volumetitlestyle}}
\fancyhead{}
\fancyhead[RO,LE]{{\small\textrm{\thepage}}}
\fancyhead[CO]{{\small{\em \the\versotextnotesheadername~\mymarks}}}
\fancyhead[CE]{{\small{\em \the\versotextnotesheadername~\mymarks}}}
\fancyfoot{}
\mark{3}
\setendnotessectiontitle{\the\versotextnotesname}{versotextnotes}
\tolerance=500\language=0
\normalfont \backmattertextfont
\ifversotextnotestwocol
\begin{multicols}{2}
\fi
\setlength{\parindent}{-10pt}
}
}
%    \end{macrocode}
%    \end{macro}
% \begin{macro}{\versotextnote}
%    \begin{macrocode}
\newif\ifversotextnotessinglepar
\newif\ifnoversotextnoteyet
\newcommand{\literalversotextnote}[1]{\begingroup
        \def\next{#1}%
        \newlinechar='40
        \immediate\write\versotextnotes{\expandafter\strip\meaning\next}%
     \endgroup
}
\newcommand{\firstversotextnote}{\immediate\write\versotextnotes{\versotitletoothernotes}
    \ifversotextnotessinglepar%
    \literalversotextnote{\par}%
    \fi% ifversotextnotessinglepa
 }
\newcommand{\versotextnote}[2][0]{\versotitleinnotescheck
    \setlemmarange{#1}%  
    \ifnoversotextnoteyet % firstversotextnote
     \firstversotextnote
      \fi %noversotextnoteyet
      \global\noversotextnoteyetfalse
\immediate\write\versotextnotes{\checknoteheaders}%
    \literalversotextnote{#2}%
    \ifversotextnotessinglepar%
    \literalversotextnote{\par}%
    \fi% ifversotextnotessinglepar    
\resetlemmacounters
}
%    \end{macrocode}
%    \end{macro}
% \begin{macro}{\makerectotextnotes}
% Next, the rectotextnotes section:
%    \begin{macrocode}
% tokens for changing the names of these notes
\newtoks{\rectotextnotesname}
\rectotextnotesname={TEXTUAL NOTES}
\newcommand{\changerectotextnotesname}[1]{\rectotextnotesname={#1}}
\newtoks{\rectotextnotesheadername}
\rectotextnotesheadername={Textual Notes to}
\newcommand{\changerectotextnotesheader}[1]{\rectotextnotesheadername={#1}}
\newtoks{\rectotextnotescontentsname}
\rectotextnotescontentsname={Textual Notes}
\newcommand{\changerectotextnotescontentsname}[1]{%
\rectotextnotescontentsname={#1}}
% two column notes
\newif\ifrectotextnotestwocol
\rectotextnotestwocolfalse
% rectotextnotes section setup
\newcommand{\makerectotextnotes}{\global\textnotesontrue 
  \newwrite\rectotextnotes
  \global\verselinenumberstrue
   \immediate\openout\rectotextnotes=\jobname.rtn
   \literalrectotextnote{\flushbottom{\par\clearpage
   {\pagestyle{volumetitlestyle}\cleardoublepage}
\thispagestyle{volumetitlestyle}}
\fancyhead{}
\fancyhead[RO,LE]{{\small\textrm{\thepage}}}
\fancyhead[CO]{{\small{\em \the\rectotextnotesheadername~\mymarks}}}
\fancyhead[CE]{{\small{\em \the\rectotextnotesheadername~\mymarks}}}
\fancyfoot{}
\mark{3}
\setendnotessectiontitle{\the\rectotextnotesname}{rectotextnotes}
\tolerance=500\language=0
\normalfont \backmattertextfont
\ifrectotextnotestwocol
\begin{multicols}{2}
\fi
\setlength{\parindent}{-10pt}
}
}
%    \end{macrocode}
%    \end{macro}
% \begin{macro}{\rectotextnote}
%    \begin{macrocode}
\newif\ifrectotextnotessinglepar
\newif\ifnorectotextnoteyet
\newcommand{\literalrectotextnote}[1]{\begingroup
        \def\next{#1}%
        \newlinechar='40
        \immediate\write\rectotextnotes{\expandafter\strip\meaning\next}%
     \endgroup
}
\newcommand{\firstrectotextnote}{\immediate\write\rectotextnotes{\rectotitletoothernotes}
    \ifrectotextnotessinglepar%
    \literalrectotextnote{\par}%
    \fi% ifrectotextnotessinglepar  
}
\newcommand{\rectotextnote}[2][0]{\rectotitleinnotescheck
    \setlemmarange{#1}%  
    \ifnorectotextnoteyet % firstrectotextnote
     \firstrectotextnote
      \fi %norectotextnoteyet
      \global\norectotextnoteyetfalse
\immediate\write\rectotextnotes{\checknoteheaders}%
    \literalrectotextnote{#2}%
    \ifrectotextnotessinglepar%
    \literalrectotextnote{\par}%
    \fi% ifrectotextnotessinglepar    
\resetlemmacounters
}
%    \end{macrocode}
%     \end{macro}
% \begin{macro}{\makeversoemendations}
% Next, the versoemendations section:
%    \begin{macrocode}
% tokens for changing the names of these notes
\newtoks{\versoemendationsname}
\versoemendationsname={EMENDATIONS}
\newcommand{\changeversoemendationsname}[1]{\versoemendationsname={#1}}
\newtoks{\versoemendationsheadername}
\versoemendationsheadername={Emendations to}
\newcommand{\changeversoemendationsheader}[1]{\versoemendationsheadername={#1}}
\newtoks{\versoemendationscontentsname}
\versoemendationscontentsname={Emendations}
\newcommand{\changeversoemendationscontentsname}[1]{%
\versoemendationscontentsname={#1}}
% two column notes
\newif\ifversoemendationstwocol
\versoemendationstwocolfalse
% versoemendations section setup
\newcommand{\makeversoemendations}{\global\emendationsontrue 
  \newwrite\versoemendations
  \global\verselinenumberstrue
   \immediate\openout\versoemendations=\jobname.vem
   \literalversoemendation{\flushbottom{\par\clearpage
   {\pagestyle{volumetitlestyle}\cleardoublepage}
\thispagestyle{volumetitlestyle}}
\fancyhead{}
\fancyhead[RO,LE]{{\small\textrm{\thepage}}}
\fancyhead[CO]{{\small{\em \the\versoemendationsheadername~\mymarks}}}
\fancyhead[CE]{{\small{\em \the\versoemendationsheadername~\mymarks}}}
\fancyfoot{}
\mark{3}
\setendnotessectiontitle{\the\versoemendationsname}{versoemendations}
\tolerance=500\language=0
\normalfont \backmattertextfont
\ifversoemendationstwocol
\begin{multicols}{2}
\fi
\setlength{\parindent}{-10pt}
}
}
%    \end{macrocode}
%    \end{macro}
% \begin{macro}{\versoemendation}
%    \begin{macrocode}
\newif\ifversoemendationssinglepar
\newif\ifnoversoemendationyet
\newcommand{\literalversoemendation}[1]{\begingroup
        \def\next{#1}%
        \newlinechar='40
        \immediate\write\versoemendations{\expandafter\strip\meaning\next}%
     \endgroup
}
\newcommand{\firstversoemendation}{\immediate\write\versoemendations{\versotitletoothernotes}
    \ifversoemendationssinglepar%
    \literalversoemendation{\par}%
    \fi% ifversoemendationssinglepar  
}
\newcommand{\versoemendation}[2][0]{\versotitleinnotescheck%
    \setlemmarange{#1}%  
    \ifnoversoemendationyet % firstversoemendation
     \firstversoemendation
      \fi %noversoemendationyet
      \global\noversoemendationyetfalse
\immediate\write\versoemendations{\checknoteheaders}%
    \literalversoemendation{#2}%
    \ifversoemendationssinglepar%
    \literalversoemendation{\par}%
    \fi% ifversoemendationssinglepar    
\resetlemmacounters
}
%    \end{macrocode}
%    \end{macro}
% \begin{macro}{\makerectoemendations}
% Next, the rectoemendations section:
%    \begin{macrocode}
% tokens for changing the names of these notes
\newtoks{\rectoemendationsname}
\rectoemendationsname={EMENDATIONS}
\newcommand{\changerectoemendationsname}[1]{\rectoemendationsname={#1}}
\newtoks{\rectoemendationsheadername}
\rectoemendationsheadername={Emendations to}
\newcommand{\changerectoemendationsheader}[1]{\rectoemendationsheadername={#1}}
\newtoks{\rectoemendationscontentsname}
\rectoemendationscontentsname={Emendations}
\newcommand{\changerectoemendationscontentsname}[1]{%
\rectoemendationscontentsname={#1}}
% two column notes
\newif\ifrectoemendationstwocol
\rectoemendationstwocolfalse
% rectoemendations section setup
\newcommand{\makerectoemendations}{\global\emendationsontrue 
  \newwrite\rectoemendations
  \global\verselinenumberstrue
   \immediate\openout\rectoemendations=\jobname.rem
   \literalrectoemendation{\flushbottom{\par\clearpage
   {\pagestyle{volumetitlestyle}\cleardoublepage}
\thispagestyle{volumetitlestyle}}
\fancyhead{}
\fancyhead[RO,LE]{{\small\textrm{\thepage}}}
\fancyhead[CO]{{\small{\em \the\rectoemendationsheadername~\mymarks}}}
\fancyhead[CE]{{\small{\em \the\rectoemendationsheadername~\mymarks}}}
\fancyfoot{}
\mark{3}
\setendnotessectiontitle{\the\rectoemendationsname}{rectoemendations}
\tolerance=500\language=0
\normalfont \backmattertextfont
\ifrectoemendationstwocol
\begin{multicols}{2}
\fi
\setlength{\parindent}{-10pt}
}
}
%    \end{macrocode}
%    \end{macro}
% \begin{macro}{\rectoemendation}
%    \begin{macrocode}
\newif\ifrectoemendationssinglepar
\newif\ifnorectoemendationyet
\newcommand{\literalrectoemendation}[1]{\begingroup
        \def\next{#1}%
        \newlinechar='40
        \immediate\write\rectoemendations{\expandafter\strip\meaning\next}%
     \endgroup
}
\newcommand{\firstrectoemendation}{\immediate\write\rectoemendations{\rectotitletoothernotes}
    \ifrectoemendationssinglepar%
    \literalrectoemendation{\par}%
    \fi% ifrectoemendationssinglepar   
}
\newcommand{\rectoemendation}[2][0]{\rectotitleinnotescheck
    \setlemmarange{#1}%  
    \ifnorectoemendationyet % firstrectoemendation
     \firstrectoemendation
      \fi %norectoemendationyet
      \global\norectoemendationyetfalse
\immediate\write\rectoemendations{\checknoteheaders}%
    \literalrectoemendation{#2}%
    \ifrectoemendationssinglepar%
    \literalrectoemendation{\par}%
    \fi% ifrectoemendationssinglepar    
\resetlemmacounters
}
%    \end{macrocode}
%    \end{macro}
% \begin{macro}{\makeversoexplanatorynotes}
% And the verso explanatory notes:
%    \begin{macrocode}
% Tokens to change names of these notes
\newtoks{\versoexplanationsname}
\versoexplanationsname={EXPLANATORY NOTES}
\newcommand{\changeversoexplanationsname}[1]{\versoexplanationsname={#1}}
\newtoks{\versoexplanationsheadername}
\versoexplanationsheadername={Explanatory Notes to}
\newcommand{\changeversoexplanationsheader}[1]{\versoexplanationsheadername={#1}}
\newtoks{\versoexplanationscontentsname}
\versoexplanationscontentsname={Explanatory Notes}
\newcommand{\changeversoexplanationscontentsname}[1]{%
\versoexplanationscontentsname={#1}}
% Two column notes
\newif\ifversoexplanationstwocol
\versoexplanationstwocolfalse
% versoexplanatory notes setup
\newcommand{\makeversoexplanatorynotes}{\global\explanontrue
     \global\verselinenumberstrue
     \newwrite\versoexplanations
     \immediate\openout\versoexplanations=\jobname.vex
     \literalversoexplain{\flushbottom{\par\clearpage
	 {\pagestyle{volumetitlestyle}\cleardoublepage}
\thispagestyle{volumetitlestyle}}
\fancyhead[RO,LE]{{\small\textrm{\thepage}}}
\fancyhead[CO]{{\small{\em \the\versoexplanationsheadername~\mymarks}}}
\fancyhead[CE]{{\small{\em \the\versoexplanationsheadername~\mymarks}}}
\fancyfoot{}
\mark{3}
\setendnotessectiontitle{\the\versoexplanationsname}{versoexplanatorynotes}
\tolerance=500\language=0
\normalfont \backmattertextfont 
\ifversoexplanationstwocol
\begin{multicols}{2}
\fi
\setlength{\parindent}{-10pt}
}
}
%    \begin{macro}{\versoexplanatory}
%    Again, |\firstversoexplanatory| is internal, called by |\versoexplanatory|.
%    \begin{macrocode}
\newcommand{\literalversoexplain}[1]{
     \begingroup
        \def\next{#1}%
        \newlinechar='40
        \immediate\write\versoexplanations{\expandafter\strip\meaning\next}%
     \endgroup
}
\newif\ifversoexplanationssinglepar
\newcommand{\firstversoexplanatory}{\immediate\write\versoexplanations{\versotitletoothernotes}
    \ifversoexplanationssinglepar%
    \literalversoexplain{\par}%
\fi}
\newif\ifnoversoexplainyet
\newcommand{\versoexplanatory}[2][0]{\versotitleinnotescheck%
\ifexplanon
    \setlemmarange{#1}%
    \ifnoversoexplainyet % first versoexplanation
    \firstversoexplanatory
      \fi %noversoexplainyet
      \global\noversoexplainyetfalse
\immediate\write\versoexplanations{\checknoteheaders}%
    \literalversoexplain{#2}%
    \ifversoexplanationssinglepar%
    \literalversoexplain{\par}%
    \fi%
    \fi % ifexplanon 
\resetlemmacounters
}
%    \end{macrocode}
%    \end{macro}
% \begin{macro}{\makerectoexplanatorynotes}
% And the recto explanatory notes:
%    \begin{macrocode}
% Tokens to change names of these notes
\newtoks{\rectoexplanationsname}
\rectoexplanationsname={EXPLANATORY NOTES}
\newcommand{\changerectoexplanationsname}[1]{\rectoexplanationsname={#1}}
\newtoks{\rectoexplanationsheadername}
\rectoexplanationsheadername={Explanatory Notes to}
\newcommand{\changerectoexplanationsheader}[1]{\rectoexplanationsheadername={#1}}
\newtoks{\rectoexplanationscontentsname}
\rectoexplanationscontentsname={Explanatory Notes}
\newcommand{\changerectoexplanationscontentsname}[1]{%
\rectoexplanationscontentsname={#1}}
% Two column notes
\newif\ifrectoexplanationstwocol
\rectoexplanationstwocolfalse
% rectoexplanatory notes setup
\newcommand{\makerectoexplanatorynotes}{\global\explanontrue
     \global\verselinenumberstrue
     \newwrite\rectoexplanations
     \immediate\openout\rectoexplanations=\jobname.rex
     \literalrectoexplain{\flushbottom{\par\clearpage
	 {\pagestyle{volumetitlestyle}\cleardoublepage}
\thispagestyle{volumetitlestyle}}
\fancyhead[RO,LE]{{\small\textrm{\thepage}}}
\fancyhead[CO]{{\small{\em \the\rectoexplanationsheadername~\mymarks}}}
\fancyhead[CE]{{\small{\em \the\rectoexplanationsheadername~\mymarks}}}
\fancyfoot{}
\mark{3}
\setendnotessectiontitle{\the\rectoexplanationsname}{rectoexplanatorynotes}
\tolerance=500\language=0
\normalfont \backmattertextfont 
\ifrectoexplanationstwocol
\begin{multicols}{2}
\fi
\setlength{\parindent}{-10pt}
}
}
%    \begin{macro}{\rectoexplanatory}
%    Again, |\firstrectoexplanatory| is internal, called by |\rectoexplanatory|.
%    \begin{macrocode}
\newcommand{\literalrectoexplain}[1]{
     \begingroup
        \def\next{#1}%
        \newlinechar='40
        \immediate\write\rectoexplanations{\expandafter\strip\meaning\next}%
     \endgroup
}
\newif\ifrectoexplanationssinglepar
\newcommand{\firstrectoexplanatory}{\immediate\write\rectoexplanations{\rectotitletoothernotes}
    \ifrectoexplanationssinglepar%
    \literalrectoexplain{\par}%
    \fi%
}
\newif\ifnorectoexplainyet
\newcommand{\rectoexplanatory}[2][0]{\rectotitleinnotescheck%
\ifexplanon
    \setlemmarange{#1}%
    \ifnorectoexplainyet % first rectoexplanation
    \firstrectoexplanatory
      \fi %norectoexplainyet
      \global\norectoexplainyetfalse
\immediate\write\rectoexplanations{\checknoteheaders}%
    \literalrectoexplain{#2}%
    \ifrectoexplanationssinglepar%
    \literalrectoexplain{\par}%
    \fi%
    \fi % ifexplanon 
\resetlemmacounters
}
%    \end{macrocode}
%    \end{macro}
%    \begin{macro}{\versoprosetextnote}
%    \begin{macrocode}
\newcommand{\versoprosetextnote}[3][0]{%
\iftextnoteson%
\versotitleinnotescheck
\unskip\proselinelabel{#2}%
\setproselemmarange{#2}{#1}%
   \ifnoversotextnoteyet % firstversotextnote
     \firstversotextnote
      \fi %noversotextnoteyet
      \global\noversotextnoteyetfalse
\immediate\write\versotextnotes{\checkprosenoteheaders}%
\literalversotextnote{#3}%
\ifversotextnotessinglepar%
\literalversotextnote{\par}%
\fi%
\resetlemmacounters%
\fi\unskip%
}
%    \end{macrocode}
%    \end{macro}
%    \begin{macro}{\versoproseemendation}
% Notice that it calls |\firstemendation| as usual.
%    \begin{macrocode}
\newcommand{\versoproseemendation}[3][0]{%
\ifemendationson%
\versotitleinnotescheck
\unskip\proselinelabel{#2}%
\setproselemmarange{#2}{#1}%
    \ifredundantemendations
       \iftextnoteson
	 \ifrangelemma
	     \versoprosetextnote[#1]{#2}{#3}%
	 \else
	     \versoprosetextnote{#2}{#3}%
	 \fi % ifrangelemma}
       \fi %iftextnoteson
\ifversotextnotessinglepar%
\literalversotextnote{\par}%
\fi%  iftextnotessinglepar     
   \fi % ifredundantemendations    
    \ifnoversoemendationyet % firstemendation
     \firstversoemendation
      \fi%noemendyet
      \global\noversoemendationyetfalse
\immediate\write\versoemendations{\checkprosenoteheaders}%
    \literalversoemendation{#3}%
\ifversoemendationssinglepar%
\literalversoemendation{\par}%
\fi% ifemendationssinglep
    \resetlemmacounters%
\fi\unskip%
}
%    \end{macrocode}
%    \end{macro}
%    \begin{macro}{\versoproseexplanatory}
%    \begin{macrocode}
\newcommand{\versoproseexplanatory}[3][0]{%
\ifexplanon%
\versotitleinnotescheck
\unskip\proselinelabel{#2}%
\setproselemmarange{#2}{#1}%  
    \ifnoversoexplainyet% firstexplanation
     \firstversoexplanatory%
      \fi%noexplainyet
      \global\noversoexplainyetfalse%
\immediate\write\versoexplanations{\checkprosenoteheaders}%
    \literalversoexplain{#3}%
\ifversoexplanationssinglepar%
    \literalversoexplain{\par}%
    \fi% ifexplanationssinglepar
    \resetlemmacounters%
\fi\unskip%
}
%    \end{macrocode}
%    \end{macro}
%    \begin{macro}{\rectoprosetextnote}
%    \begin{macrocode}
\newcommand{\rectoprosetextnote}[3][0]{%
\iftextnoteson%
\rectotitleinnotescheck
\unskip\proselinelabel{#2}%
\setproselemmarange{#2}{#1}%
  \ifnorectotextnoteyet % firstrectotextnote
     \firstrectotextnote
      \fi %norectotextnoteyet
      \global\norectotextnoteyetfalse
\immediate\write\rectotextnotes{\checkprosenoteheaders}%
\literalrectotextnote{#3}%
\ifrectotextnotessinglepar%
\literalrectotextnote{\par}%
\fi%
\resetlemmacounters%
\fi\unskip%
}
%    \end{macrocode}
%    \end{macro}
%    \begin{macro}{\rectoproseemendation}
% Notice that it calls |\firstemendation| as usual.
%    \begin{macrocode}
\newcommand{\rectoproseemendation}[3][0]{%
\ifemendationson%
\rectotitleinnotescheck
\unskip\proselinelabel{#2}%
\setproselemmarange{#2}{#1}%
    \ifredundantemendations
       \iftextnoteson
	 \ifrangelemma
	     \rectoprosetextnote[#1]{#2}{#3}%
	 \else
	     \rectoprosetextnote{#2}{#3}%
	 \fi % ifrangelemma}
       \fi %iftextnoteson
\ifrectotextnotessinglepar%
\literalrectotextnote{\par}%
\fi%  iftextnotessinglepar     
   \fi % ifredundantemendations    
    \ifnorectoemendationyet % firstemendation
     \firstrectoemendation
      \fi%noemendyet
      \global\norectoemendationyetfalse
\immediate\write\rectoemendations{\checkprosenoteheaders}%
    \literalrectoemendation{#3}%
\ifrectoemendationssinglepar%
\literalrectoemendation{\par}%
\fi% ifemendationssinglep
    \resetlemmacounters%
\fi\unskip%
}
%    \end{macrocode}
%    \end{macro}
%    \begin{macro}{\rectoproseexplanatory}
%    \begin{macrocode}
\newcommand{\rectoproseexplanatory}[3][0]{%
\ifexplanon%
\rectotitleinnotescheck
\unskip\proselinelabel{#2}%
\setproselemmarange{#2}{#1}%  
    \ifnorectoexplainyet% firstexplanation
     \firstrectoexplanatory%
      \fi%noexplainyet
      \global\norectoexplainyetfalse%
\immediate\write\rectoexplanations{\checkprosenoteheaders}%
    \literalrectoexplain{#3}%
\ifrectoexplanationssinglepar%
    \literalrectoexplain{\par}%
    \fi% ifexplanationssinglepar
    \resetlemmacounters%
\fi\unskip%
}
%    \end{macrocode}
%    \end{macro}
%    \begin{macro}{\pmclresetsettitleinnotes}
%    \begin{macrocode}
\newcommand{\pmclresetsettitleinnotes}{%
\noversoemendationyettrue%
\noversoexplainyettrue%
\noversotextnoteyettrue%
\norectoemendationyettrue%
\norectoexplainyettrue%
\norectotextnoteyettrue%
}
%    \end{macrocode}
%    \end{macro}
%    \begin{macro}{\pmclversoresetsettitleinnotes}
%    \begin{macrocode}
\newcommand{\pmclversoresetsettitleinnotes}{%
\noversoemendationyettrue%
\noversoexplainyettrue%
\noversotextnoteyettrue%
}
%    \end{macrocode}
%    \end{macro}
%    \begin{macro}{\pmclrectoresetsettitleinnotes}
%    \begin{macrocode}
\newcommand{\pmclrectoresetsettitleinnotes}{%
\norectoemendationyettrue%
\norectoexplainyettrue%
\norectotextnoteyettrue%
}
%    \end{macrocode}
%    \end{macro}
%    \begin{macro}{\literalexplain}
%    \begin{macrocode}
\let\literalexplanatory=\literalexplain
\let\literalversoexplanatory=\literalversoexplain
\let\literalrectoexplanatory=\literalrectoexplain
%    \end{macrocode}
%    \end{macro}
%    \begin{macro}{\putversotextnotes} Does the same for textual notes.
% 
%    \begin{macrocode}
\newcommand{\putversotextnotes}{
\iftextnoteson
\ifversotextnotestwocol
\literalversotextnote{\end{multicols}}
\fi %\iftextnotetwocol
   \newpage
   \hyphenationforsmall
   \immediate\closeout\versotextnotes
   \input \jobname.vtn
\contentsendnotestitle{\the\versotextnotescontentsname}{versotextnotes}
\else
 \relax
\fi}
%    \end{macrocode}
%    \end{macro}
%    \begin{macro}{\putrectotextnotes} Does the same for recto textual notes.
% 
%    \begin{macrocode}
\newcommand{\putrectotextnotes}{
\iftextnoteson
\ifrectotextnotestwocol
\literalrectotextnote{\end{multicols}}
\fi %\iftextnotetwocol
   \newpage
   \hyphenationforsmall
   \immediate\closeout\rectotextnotes
   \input \jobname.rtn
\contentsendnotestitle{\the\rectotextnotescontentsname}{rectotextnotes}
\else
 \relax
\fi}
%    \end{macrocode}
%    \end{macro}
%    \begin{macro}{\putversoemendations} This macro closes external file 
% for verso emendations, 
% reads it in, and puts an entry for the emendations into the table of contents
% file. The macro tests to see whether there is an .aux file from a previous run 
% before doing this, and complains in the .log file if the .aux file does not
% exist.
% 
%    \begin{macrocode}
\newcommand{\putversoemendations}{
\ifemendationson
\ifversoemendationstwocol
\literalversoemendation{\end{multicols}}
\fi %\iftextnotetwocol
   \newpage
   \hyphenationforsmall
   \immediate\closeout\versoemendations
   \input \jobname.vem
\contentsendnotestitle{\the\versoemendationscontentsname}{versoemendations}
\else
 \relax
\fi}
%    \end{macrocode}
%    \end{macro}
%    \begin{macro}{\putrectoemendations} This macro closes external file 
% for recto emendations, 
% reads it in, and puts an entry for the emendations into the table of contents
% file. The macro tests to see whether there is an .aux file from a previous run 
% before doing this, and complains in the .log file if the .aux file does not
% exist.
% 
%    \begin{macrocode}
\newcommand{\putrectoemendations}{
\ifemendationson
\ifrectoemendationstwocol
\literalrectoemendation{\end{multicols}}
\fi %\iftextnotetwocol
   \newpage
   \hyphenationforsmall
   \immediate\closeout\rectoemendations
   \input \jobname.rem
\contentsendnotestitle{\the\rectoemendationscontentsname}{rectoemendations}
\else
 \relax
\fi}
%    \end{macrocode}
%    \end{macro}
%    \begin{macro}{\putversoexplanatorynotes} Does the same for verso explanatory notes. 
%    \begin{macrocode}
\newcommand{\putversoexplanatorynotes}{
\ifexplanon
\ifversoexplanationstwocol
\literalversoexplain{\end{multicols}}
\fi %\iftextnotetwocol
   \newpage
   \hyphenationforsmall
   \immediate\closeout\versoexplanations
   \input \jobname.vex
\ifpoemcontentson
 \ifexplanon
\contentsendnotestitle{\the\versoexplanationscontentsname}{versoexplanatorynotes}
  \fi
\fi
\else
 \relax
\fi}%    \end{macrocode}
%    \end{macro}
%    \begin{macro}{\putrectoexplanatorynotes} Does the same for recto explanatory notes. 
%    \begin{macrocode}
\newcommand{\putrectoexplanatorynotes}{
\ifexplanon
\ifrectoexplanationstwocol
\literalrectoexplain{\end{multicols}}
\fi %\iftextnotetwocol
   \newpage
   \hyphenationforsmall
   \immediate\closeout\rectoexplanations
   \input \jobname.rex
\ifpoemcontentson
 \ifexplanon
\contentsendnotestitle{\the\rectoexplanationscontentsname}{rectoexplanatorynotes}
  \fi
\fi
\else
 \relax
\fi}
%    \end{macrocode}
%    \end{macro}
% \subsection{~Parallel Texts in Prose}
% First, some booleans.
%    \begin{macro}{\ifversopassagestillopen}
%    \begin{macrocode}
\newif\ifversopassagestillopen
\versopassagestillopenfalse
\newif\ifrectopassagestillopen
\rectopassagestillopenfalse
%    \end{macrocode}
%    \end{macro}
% When the last recto page ended, was the last paragraph finished?
% True means the next recto should continue the paragraph from the last recto
%    \begin{macro}{\ifinpara}
%    \begin{macrocode}
\newif\ifinpara
\inparafalse
\newif\ifrectoparapending
\rectoparapendingfalse
\newif\ifrectoparaopen
\rectoparaopenfalse
\newif\ifversoparapending
\versoparapendingfalse
\newif\ifversoparaopen
\versoparaopenfalse
%    \end{macrocode}
%    \end{macro}
% Counters: prosepage versoproselinecounter, etc., and commands to set them.
%    \begin{macro}{\makeprosepagelabel}
%    \begin{macrocode}
\newcounter{prosepage}
\setcounter{prosepage}{1}
\newcommand{\makeprosepagelabel}{\proselinelabel{prosepageno\theprosepage}}
\newcounter{versoproselinecounter}
\newcommand{\saveversoproseline}{%
\refstepcounter{prosepage}%
\makeprosepagelabel
\setcounterfromref{versoproselinecounter}{prosepageno\theprosepage}
}
\newcommand{\restoreversoproseline}{\setcounter{linenumber}{\value{versoproselinecounter}}}%
\newcounter{rectoproselinecounter}
\newcommand{\saverectoproseline}{%
\refstepcounter{prosepage}%
\makeprosepagelabel
\setcounterfromref{rectoproselinecounter}{prosepageno\theprosepage}
}
\newcommand{\restorerectoproseline}{\setcounter{linenumber}%
 {\value{rectoproselinecounter}}}%
%    \end{macrocode}
%    \end{macro}
% The |\parastart| and |\paraend| macros, to mark the beginning and end
% of paragraphs
%    \begin{macro}{\parastart}
%    \begin{macrocode}
\newcommand{\parastart}{\inparatrue}
%    \end{macrocode}
%    \end{macro}
%    \begin{macro}{\paraend}
%    \begin{macrocode}
\newcommand{\paraend}{\inparafalse}
%    \end{macrocode}
%    \end{macro}
% |\startparalleltextsprose| and |\finishparalleltextsprose|
%    \begin{macro}{\startparalleltextsprose}
%    \begin{macrocode}
\newcommand{\startparalleltextsprose}{%
\leavevmode
\cleartoverso
\paralleltextstrue 
\inparafalse
}
%    \end{macrocode}
%    \end{macro}
%    \begin{macro}{\finishparalleltextsprose}
%    \begin{macrocode}
\newcommand{\finishparalleltextsprose}{%
\paralleltextsfalse}
%    \end{macrocode}
%    \end{macro}
%    \begin{macro}{parallelprose}
%    \begin{macrocode}
\newenvironment{parallelprose}{\startparalleltextprose} {\finishparalleltextsprose}
%    \end{macrocode}
%    \end{macro}
% |\startversoprosepage|
%    \begin{macro}{\startversoprosepage}
%    \begin{macrocode}
\newcommand{\startversoprosepage}{%
\let\titletoothernotes=\versotitletoothernotes
\restoreversoproseline%
\ifversoparapending{\inparatrue}\else{\inparafalse}\fi%
\ifinpara{\noindent}\else{\relax}\fi%
}
%    \end{macrocode}
%    \end{macro}
% |\finishversoprosepage|
%    \begin{macro}{\finishversoprosepage}
%    \begin{macrocode}
\newcommand{\finishversoprosepage}{%
\ifinpara{\versoparapendingtrue}\else{\versoparapendingfalse}\fi%
\saveversoproseline%
}
%    \end{macrocode}
%    \end{macro}
% |\startrectoprosepage|
%    \begin{macro}{\startrectoprosepage}
%    \begin{macrocode}
\newcommand{\startrectoprosepage}{%
\let\titletoothernotes=\rectotitletoothernotes
\restorerectoproseline%
\ifrectoparapending{\inparatrue}\else{\inparafalse}\fi%
\ifinpara{\noindent}\else{\relax}\fi%
}
%    \end{macrocode}
%    \end{macro}
%    \begin{macro}{\finishrectoprosepage}
%    \begin{macrocode}
\newcommand{\finishrectoprosepage}{%
\ifinpara{\rectoparapendingtrue}\else{\rectoparapendingfalse}\fi%
\saverectoproseline
}
%    \end{macrocode}
%    \end{macro}
% \subsection{~Finishing up} 
%    \begin{macro}{\finish}
% The  |\finish| macro  reads in the notes, 
% the index (previously created from your .aux file by MakeIndex), and 
% the contents sections and sets them. The notes sections send page 
% information to the contents file as they are set. Then the page 
% information for the index is sent to the contents file. Finally the 
% contents file is closed and read in. If you have an editor's 
% introduction or a list of abbreviations, copy this macro, uncomment out the commented out lines, 
% and issue it all as a |\renewcommand| for |\finish|. Be sure to 
% have the command |\label{editorsintroduction}| on the first page of 
% your introduction.

% It usually takes three 
% passes to get the table of contents and the index files correct. 
% 
% The consequence of doing using |\finish| is that the table of
% contents comes last.  There are ways around this that are described
% earlier.  The notes sections are all typeset in the |\small| size.
% 
%    \begin{macrocode}
\newcommand{\finish}{
\small\bf \hyphenchar\font=45
\small\it \hyphenchar\font=45
\small\rm \hyphenchar\font=45

\ifemendationson
   \newpage
   \immediate\closeout\emendations
   \input \jobname.emd 
\fi
\iftextnoteson
   \newpage
   \immediate\closeout\textnotes
   \input \jobname.ent
   \newpage
 \fi
 \ifexplanon
      \immediate\closeout\explanations
      \input \jobname.enx
      \newpage
 \fi
 \ifindexingon
 \printindex
 \fi
 \ifpoemcontentson
%      \immediate\write\poemcontents{\string\vspace{10pt}\string\par}
%      \immediate\write
% \poemcontents{\string\contentspoemtitlefont\ Introduction to the Notes}
%      \immediate\write
% \poemcontents{\string~ /  \string~\ \string\pageref{editorsintroduction} \string\par}
%      \immediate\write\poemcontents{\string\par \string\smallskip}
%      \immediate\write\poemcontents{\string\contentspoemtitlefont\ Abbreviations}
%      \immediate\write
% \poemcontents{\string~ /  \string~\ \string\pageref{abbreviations} \string\par}
%      \immediate\write\poemcontents{\string\par \string\smallskip}
     \ifemendationson
      \immediate\write\poemcontents{\string\contentspoemtitlefont\ 
      \the\emendationscontentsname}
      \immediate\write
\poemcontents{\string~ /  \string~\ \string\pageref{emendationnotes} \string\par}
      \immediate\write\poemcontents{\string\par \string\smallskip}
     \fi
     \iftextnoteson
      \immediate\write\poemcontents{\string\contentspoemtitlefont\ 
      \the\textnotescontentsname}
      \immediate\write
\poemcontents{\string~ /  \string~\ \string\pageref{textualnotes} \string\par}
      \immediate\write\poemcontents{\string\par \string\smallskip}
     \fi
     \ifexplanon 
       \immediate\write
\poemcontents{\string\contentspoemtitlefont\ \the\explanationscontentsname}
       \immediate\write
\poemcontents{\string~ /  \string~\ \string\pageref{explanatorynotes}}
     \fi
     \ifindexingon
        \immediate\write\poemcontents{\string\par \string\smallskip}
        \immediate\write
\poemcontents{\string\contentspoemtitlefont\ \the\poemindexcontentsname}
        \immediate\write
\poemcontents{\string~ /  \string~\ \string\pageref{indexpage}\string\par}
     \fi
    \immediate\closeout\poemcontents
    \input \jobname.ctn
 \fi
 \clearpage
\thispagestyle{volumetitlestyle}
\vspace{106pt}
\begin{center}
{\poemtitlefont Note}
\end{center}

\noindent The symbol \stanzaatbottomvalue\ is used to indicate 
a space between sections of a poem 
wherever such spaces are lost in pagination.
\cleardoublepage
}
%    \end{macrocode}
%    \end{macro}
%    \begin{macro}{\testforauxonfirstrun} Put this in the preamble. It suppresses
% |\finish|, |\puttextnotes|, |\putemendations|, and |\putexplanatory| unless
% there is an .aux file from a previous run, this avoiding some error messages. 
%  
%    \begin{macrocode}
\newcommand{\testforauxonfirstrun}{\IfFileExists{\jobname.aux}{\relax}
{\let\finish=\relax 
\let\puttextnotes=\relax
\let\putexplanatory=\relax
\let\putemendations=\relax
\typeout{Run again to read in endnotes files}}}
%    \end{macrocode}
%    \end{macro}
%    \begin{macro}{\putpoemcontents}
% You may wish to do the tasks done by |\finish| separately. This will give you
% more flexibility. |\putpoemcontents| inserts the table of contents file from a 
% prior run. Be sure to issue it \emph{before} you issue |\makepoemcontents.|

%    \begin{macrocode}
\newcommand{\putpoemcontents}{
\IfFileExists{\jobname.ctn}{\input \jobname.ctn 
\clearpage \resetpagestyle\normalsize}
{\typeout{Run again to input contents file here}}
}
%    \end{macrocode}
%    \end{macro}
%    \begin{macro}{\hyphenationforsmall} This command restores hyphenation for the
% backmatter sections of the book. Hyphenation is turned off in the main body of 
% the book, because verse should not be hyphenated. This macro turns hyphenation 
% on, but only for small roman, italic, and bold fonts. Backmatter sections are
% all set in |\small|. |\putemendations|, |\puttextnotes|, and
% |\putexplanatory| all call |\hyphenationforsmall|.
% 
%    \begin{macrocode}
\newcommand{\hyphenationforsmall}{\small\bf \hyphenchar\font=45
\small\it \hyphenchar\font=45
\small\rm \hyphenchar\font=45
}
%    \end{macrocode}
%    \end{macro}
%    \begin{macro}{\putemendations} This macro closes external file for emendations, 
% reads it in, and puts an entry for the emendations into the table of contents
% file. The macro tests to see whether there is an .aux file from a previous run 
% before doing this, and complains in the .log file if the .aux file does not
% exist.
% 
%    \begin{macrocode}
\newcommand{\putemendations}{
\ifemendationson
\ifemendationstwocol
\literalemend{\end{multicols}}
\fi %\iftextnotetwocol
   \newpage
   \hyphenationforsmall
   \immediate\closeout\emendations
   \input \jobname.emd
\contentsendnotestitle{\the\emendationscontentsname}{emendationnotes}
\else
 \relax
\fi}
%    \end{macrocode}
%    \end{macro}
%    \begin{macro}{\putexplanatory} Does the same for explanatory notes.
% 
%    \begin{macrocode}
\newcommand{\putexplanatory}{
\ifexplanon
\ifexplanationstwocol
\literalexplain{\end{multicols}}
\fi %\iftextnotetwocol
   \newpage
   \hyphenationforsmall
   \immediate\closeout\explanations
   \input \jobname.enx
\ifpoemcontentson
 \ifexplanon
\contentsendnotestitle{\the\explanationscontentsname}{explanatorynotes}
  \fi
\fi
\else
 \relax
\fi}
%    \end{macrocode}
%    \end{macro}
%    \begin{macro}{\puttextnotes} Does the same for textual notes.
% 
%    \begin{macrocode}
\newcommand{\puttextnotes}{
\iftextnoteson
\iftextnotestwocol
\literaltextnote{\end{multicols}}
\fi %\iftextnotetwocol
   \newpage
   \hyphenationforsmall
   \immediate\closeout\textnotes
   \input \jobname.ent
\contentsendnotestitle{\the\textnotescontentsname}{textualnotes}
\else
 \relax
\fi}
%    \end{macrocode}
%    \end{macro}
%    \begin{macro}{\putpoemindex} Reads in the index of titles and first lines
%  and adds an entry for the index 
% into the table of contents.
% 
%    \begin{macrocode}
\newtoks{\poemindexname}
\poemindexname={Index of Titles and First Lines}
\newcommand{\changepoemindexname}[1]{\poemindexname={#1}}
\newtoks{\poemindexheadername}
\poemindexheadername={Index of Titles and First Lines}
\newcommand{\changepoemindexheader}[1]{\poemindexheadername={#1}}
\newtoks{\poemindexcontentsname}
\poemindexcontentsname={Index of Titles and First Lines}
\newcommand{\changepoemindexcontentsname}[1]{\poemindexcontentsname={#1}}
\newcommand{\putpoemindex}{\ifindexingon
 \printindex
 \fi
 \ifpoemcontentson\ifindexingon
     \immediate\write\poemcontents{\string\par \string\smallskip}
     \immediate\write
\poemcontents{\string\contentspoemtitlefont\ \the\poemindexcontentsname}
     \immediate\write
\poemcontents{\string~\string\contentsleaders \string~\ \string\pageref{indexpage}\string\par}
  \fi\fi}
\newcommand{\putmultiplepoemindex}[4]{%
\ifindexingon
\poemindexname={#2}
\poemindexheadername={#3}
\poemindexcontentsname={#4}
\poemindexlabelname={#4}
\printindex[#1][#2]                                      
\fi
\ifpoemcontentson\ifindexingon
     \immediate\write\poemcontents{\string\par \string\smallskip}
     \immediate\write
\poemcontents{\string\contentspoemtitlefont\ \the\poemindexcontentsname}
     \immediate\write
\poemcontents{\string~ \string\contentsleaders \string~\ 
\string\pageref{\the\poemindexlabelname}\string\par}
  \fi\fi
} 
\newcommand{\noteaboutstanzamarkpage}{\begin{volumetitlepage}
%\thispagestyle{volumetitlestyle}
\vspace{106pt}
\begin{center}
{\poemtitlefont Note}
\end{center}

\noindent The symbol \stanzaatbottomvalue\ is used to indicate 
a space between sections of a poem 
wherever such spaces are lost in pagination.
\end{volumetitlepage}}
%    \end{macrocode}
%    \end{macro}
% \subsection{~Visual formatting} The ideal in \LaTeX\ is to separate 
% content markup and formatting as completely as possible. In the text 
% itself you simply mark the logical units (a poem, a title, a stanza, 
% a textual note), and a header program like this interprets the 
% meaning of those terms typographically. However, one still winds up 
% having to do some visual formatting. Sometimes you have to enlarge or 
% shrink a page by one line in order to avoid an intractable widow or 
% club line, and when one does so one must be careful to do the same to 
% the other member of the two-page spread of which that page is a part. 
% These commands enable one to wiggle pagelength enough to handle these 
% problems.
%    \begin{macrocode}
\newcommand{\longpage}{\enlargethispage{\baselineskip}}
\newcommand{\shortpage}{\enlargethispage{-\baselineskip}}
%    \end{macrocode}
%    

%  \subsection{~Special environments: cjquotation and theindex}  
% An environment for special prose quotations within verse
% as in ``Chief Joseph of the Nez Perce''
% 
% \goodbreak
% 
%    \begin{macro}{cjquotation}
%    \begin{macrocode}
\newenvironment{cjquotation}
                {\tolerance=2000%
                \vspace{-\bigskipamount}
               \restoreverselinenumber
               \language=255
               \noindent\relax
               \list{}{\listparindent 0em%
                        \itemindent 0em%
                        \rightmargin 2em%
                        \leftmargin 2em%
                        \topsep 0em%
                        \parskip 0em%
                        \itemindent    \listparindent
                        \rightmargin   \leftmargin
                        \parsep        \z@ \@plus\p@}%
                \item[]
                }
               {\endlist}
%    \end{macrocode}
%    \end{macro}
%    \begin{macro}{theindex}
% And the environment for the index:
%    \begin{macrocode}
\newtoks{\poemindexlabelname}
\poemindexlabelname={indexpage}
\newcommand{\changepoemindexlabelname}[1]{\poemindexlabelname={#1}}
\renewenvironment{theindex}
     {{\par\clearpage{\pagestyle{volumetitlestyle}\cleardoublepage}
     \thispagestyle{volumetitlestyle}}
     \makeatletter
     \def\pmclidxitem{\par\hangindent 1em}
\fancyhead[RO,LE]{{\small\textrm{\thepage}}}
\fancyhead[CO]{{\small{\em \the\poemindexheadername}}}
\fancyhead[CE]{{\small{\em \the\poemindexheadername}}}
\fancyfoot{}
     \label{\the\poemindexlabelname}
     \lefthyphenmin=2\tolerance=500\language=0
     \normalfont \backmattertextfont
               \raggedright
                \begin{multicols}
{2}[\backmattersink\begin{center}{\normalfont 
\backmatterheaderfont \the\poemindexname}\end{center}
\backmatterafterheadersink]%
                \let\item\pmclidxitem}
               {\end{multicols}\clearpage}
\makeatother
%    \end{macrocode}
%    \end{macro}
% Finally, some other miscellaneous environments.
%    \begin{macrocode}
\newlength{\epigraphquoteleftmargin}
\newlength{\epigraphquoterightmargin}
\setlength{\epigraphquoteleftmargin}{1.5em}
\setlength{\epigraphquoterightmargin}{1.5em}
\newenvironment{epigraphquote}{\list{}{\rightmargin\epigraphquoterightmargin%
                                       \leftmargin\epigraphquoteleftmargin%
                                       \topsep 0pt%
                                       \partopsep 0pt%
                                       \listparindent 0pt%
                                       }%
                \item\relax}
               {\endlist}
\newenvironment{volumetitlepagequote}{\list{}{\rightmargin \voladditionalamount%
                                              \leftmargin \voladditionalamount%                                             
                                       \topsep 0pt%
                                       \partopsep 0pt%
                                       \listparindent 0pt
                                       }%
                \item\relax}
               {\endlist}
%    \end{macrocode}
% \clearpage
% \Finale
\endinput			   
--------------060108090809060900030409--
