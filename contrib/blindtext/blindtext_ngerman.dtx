%
%
% ^^A This document is generated by mk_blindtext_texts.rb
%
%
% \subsection{German -- New Orthography (babel: ngerman)}
% \changes{V1.9e}{2011-12-09}{Correction NGerman}
% Thanks to Felix Lehmann for corrections.
%
%    \begin{macro}{\blindtext@ngerman}
%    Define flag, so we can check if language is defined.
%    \begin{macrocode}
\def\blindtext@ngerman{}
%    \end{macrocode}
%    \end{macro}
%
% Define the default blind text for Ngerman.
%    \begin{macrocode}
\blind@addtext{ngerman}{%
  \def\blindtext@text{%
    Dies hier ist ein Blindtext zum Testen von
    Textausgaben\blindtext@endsentence Wer diesen Text liest, ist selbst
    schuld\blindtext@endsentence Der Text gibt lediglich den Grauwert
    der Schrift an\blindtext@endsentence Ist das wirklich so? Ist es
    gleich\-g\"ul\-tig, ob ich schreibe: \glqq Dies ist ein
    Blindtext\grqq\ oder \glqq Huardest gefburn\grqq ? Kjift --
    mitnichten! Ein Blindtext bietet mir wichtige
    Informationen\blindtext@endsentence An ihm messe ich die
    \blindmarkup{Lesbarkeit einer Schrift}, ihre Anmutung, wie
    harmonisch die Figuren zueinander stehen und pr\"u\-fe, wie breit
    oder schmal sie l\"auft\blindtext@endsentence Ein Blindtext sollte
    m\"og\-lichst viele \blindmarkup{verschiedene Buchstaben} enthalten
    und in der Originalsprache gesetzt sein\blindtext@endsentence Er
    muss keinen Sinn ergeben, sollte aber lesbar
    sein\blindtext@endsentence Fremdsprachige Texte wie \glqq Lorem
    ipsum\grqq\ dienen nicht dem eigentlichen Zweck, da sie eine falsche
    Anmutung vermitteln\blindtext@endsentence%
  }% \blindtext@text
}
%    \end{macrocode}
%
%
% \changes{V1.9e}{2011-12-09}{Default paragraph start for Ngerman}
% Define different paragraph starts for second and later paragraphs.
% The first paragraph gets no special start.
%    \begin{macrocode}
\blind@addtext{ngerman}{%
  \def\blindtext@parstart{%
      \ifcase\value{blind@countparstart}\or
Das hier ist der zweite Absatz.\or
Und nun folgt -- ob man es glaubt oder nicht --  der dritte Absatz.\or
Nach diesem vierten Absatz beginnen wir eine neue Z\"ahlung.\or
        \setcounter{blind@countparstart}{0}
      \fi
      \stepcounter{blind@countparstart}
  }% \blindtext@parstart
}
%    \end{macrocode}
%
% Define counters for list environments.
%    \begin{macrocode}
\blind@addtext{ngerman}{%
  \def\blindtext@count{%
    \ifcase\value{blind@listcount}\or
      Erster\or Zweiter\or Dritter\or Vierter\or F{\"u}nfter\or
      Sechster\or Siebter\or Achter\or Neunter\or Zehnter\or
      Elfter\or Zw{\"o}lfter\or Dreizehnter\or Vierzehnter%
    \else
      Noch ein%
    \fi
  }% \blindtext@count
  \def\blindtext@item{Listenpunkt, Stufe~\arabic{blind@levelcount}}%
}%\addto\extrasngerman
%    \end{macrocode}
%
% Define title lines for Ngerman.
%    \begin{macrocode}
\blind@addtext{ngerman}{%
  \def\blindtext@heading{{\"U}berschrift auf Ebene\xspace}%
  \def\blindtext@list{Listen}%
  \def\blindtext@listEx{Beispiel einer Liste\xspace}%
}%\addto\extrasngerman
%    \end{macrocode}
%
% Add the title for |\blindmathpaper|.
%    \begin{macrocode}
\blind@addtext{ngerman}{%
    \def\blindtext@blindmath{Blindtext mit mathematischen Formeln}%
}%\addto\extrasngerman
%    \end{macrocode}
%
%
% Define the bible-option text for ngerman.
%    \begin{macrocode}
\ifblindbible
\blind@addtext{ngerman}{%
  \def\blindtext@text{%
    Da sprach Gott der Herr zu der Schlange: Weil du solches getan hast,
    seist du verflucht vor allem Vieh und vor allen Tieren auf dem
    Felde. Auf deinem Bauche sollst du gehen und Erde essen dein Leben
    lang.
    Gott sprach zu Mose: \glqq Ich werde sein, der Ich sein werde.\grqq\
    Und sprach: Also sollst du den Kindern Israel sagen: \glqq Ich werde
    sein\grqq\ hat mich zu euch gesandt\ldots
    und er soll davon opfern ein Opfer dem Herrn, n\"amlich das Fett,
    welches die Eingeweide bedeckt, und alles Fett am Eingeweide,\ldots
    Und der HERR redete mit Mose in der W\"uste Sinai und sprach:
    Jair, der Sohn Manasses, nahm die ganze Gegend Argob bis an die
    Grenze der Gessuriter und Maachathiter und hiess das Basan nach
    seinem Namen D\"orfer Jairs bis auf den heutigen Tag.%
  }% \blindtext@text
  \def\blindtext@parstart{}%no change for bible option
}
\fi %\ifbible
%    \end{macrocode}
%
% Define the random-option text for ngerman.
%    \begin{macrocode}
\ifblindrandom
  \blind@addtext{ngerman}{%
      \setcounter{blindtext}{17}
      \def\blindtext@text{%
      \blind@countxx=1 %
      \loop  
        \ifcase\value{blind@randomcount}%
Dies hier ist ein Blindtext zum Testen von
Textausgaben\blindtext@endsentence
\or Gerne werden Pangramme als Blindtexte
verwendet\blindtext@endsentence
\or Das griechische Wort Pangramm (oder holoalphabetischer Satz)
bezeichnet einen Satz, der alle Buchstaben des Alphabets
enth\"alt\blindtext@endsentence
\or Wobei man \glqq alle Buchstaben\grqq\ mit und ohne Umlaute z\"ahlen
kann\blindtext@endsentence
\or Aber das soll uns hier nicht k\"ummern, eigentlich wollen wir doch
eine Geschichte erz\"ahlen\blindtext@endsentence
\or Aber wozu wollen wir eine Geschichte erz\"ahlen?\xspace
\or Ach ja, wir brauchen Text um das Layout dieses Textes zu p\"ufen --
dazu nimmt man meist einen Blindtext\blindtext@endsentence%
      \setcounter{blind@randomcount}{-1}%
      \fi%
      \refstepcounter{blind@randomcount}%
    \ifnum\blind@countxx<\value{blind@randommax}\advance\blind@countxx by 1 %
    \repeat%
    \setcounter{blind@randommax}{\value{blindtext}}%
    }% \blindtext@text  
    \def\blindtext@parstart{}%no change for random option
  }
\fi %option random
%    \end{macrocode}
%
% Define the pangram-option text for ngerman.
%    \begin{macrocode}
\ifblindpangram
\blind@addtext{ngerman}{%
    \setcounter{blindtext}{5}
    \def\blindtext@text{%
    \blind@countxx=1 %
    \loop  
      \ifcase\value{blind@pangramcount}%
Franz jagt im komplett verwahrlosten Taxi quer durch
Bayern\blindtext@endsentence
\or Zw\"olf Boxk\"ampfer jagen Viktor quer \"uber den gro{\ss}en Sylter
Deich\blindtext@endsentence
\or Vogel Quax zwickt Johnys Pferd Bim\blindtext@endsentence
\or Sylvia wagt quick den Jux bei Pforzheim\blindtext@endsentence
\or Prall vom Whisky flog Quax den Jet zu Bruch\blindtext@endsentence
\or Jeder wackere Bayer vertilgt bequem zwo Pfund
Kalbshaxen\blindtext@endsentence
\or Stanleys Expeditionszug quer durch Afrika wird von jedermann
bewundert\blindtext@endsentence%
    \setcounter{blind@pangramcount}{-1}%
    \fi%
    \refstepcounter{blind@pangramcount}%
  \ifnum\blind@countxx<\value{blind@pangrammax}\advance\blind@countxx by 1 %
  \repeat%
  \setcounter{blind@pangrammax}{\value{blindtext}}%
  }% \blindtext@text  
  \def\blindtext@parstart{}%no change for pangram option
}
\fi %option pangram
%    \end{macrocode}
%
% If the package \Lpack{ngerman} is loaded, select the language.
%    \begin{macrocode}
\@ifpackageloaded{ngerman}{\selectlanguage{ngerman}}{}
%    \end{macrocode}
%
% ^^A %%%%%%%%%% End German -- New Orthography %%%%%%%%%%%%%%%%
%
