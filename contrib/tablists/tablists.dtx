% \iffalse
%
%    tablists.dtx - tabulated list of short items.
%    (c) 2007, 2008 Olga Lapko (Lapko.O@g23.relcom.ru)
%
%    This program is provided under the terms of the
%    LaTeX Project Public License distributed from CTAN
%    archives in directory macros/latex/base/lppl.txt.
%
% \fi
%
% \iffalse
%<*driver>
\ProvidesFile{tablists.dtx}
%</driver>
%<package>\NeedsTeXFormat{LaTeX2e}[1999/12/01]
%<package>\ProvidesPackage{tablists}
%<*package>
    [2009/08/04 v0.0e (beta) Tabulated list of short items]
%</package>
%
%<*batchfile>
\begingroup

\input docstrip.tex

\keepsilent
\askforoverwritefalse

\generate{\file{tablists.sty}{\from{tablists.dtx}{package}}}

\endgroup
%</batchfile>
%
%<*driver>
\documentclass{ltxdoc}

\usepackage{ifpdf}
\ifpdf
  \usepackage{mathptmx,courier}
  \usepackage[scaled=0.90]{helvet}
  \addtolength\marginparwidth{15pt}
\fi

\usepackage{paralist}
\usepackage{makecell}
\usepackage{amsthm}
\newtheorem{theorem}{Theorem}
\usepackage{tablists}
\IfFileExists{hyperref.sty}{\usepackage{hyperref}}{}
\IfFileExists{listpen.sty}{\usepackage{listpen}}{}
\providecommand\RestoreSpaces{\bigskip}
\EnableCrossrefs
\CodelineIndex
\RecordChanges
\makeatletter
\@beginparpenalty10000
\widowpenalty10000
\clubpenalty10000
\makeatother
\providecommand*{\file}[1]{\texttt{#1}}
\providecommand*{\package}[1]{\textsf{#1}}
\providecommand*{\cls}[1]{\textsf{#1}}
\providecommand*{\env}[1]{\texttt{#1}}

\OnlyDescription
\begin{document}
  \DocInput{tablists.dtx}
  \PrintChanges
\end{document}
%</driver>
% \fi
%
% \CheckSum{493}
%
% \GetFileInfo{tablists.dtx}
%
% \DoNotIndex{\newcommand,\newenvironment}
%
%
% \title{The \textsf{tablists} package\thanks{This
%          file has version number \fileversion,
%          last revised \filedate.}}
%   \author{%
%   Olga Lapko\\
%   {\tt Lapko.O@g23.relcom.ru} }
%   \date{\filedate}
%   \maketitle
%   \begin{abstract}
% This package gives variant of environment for creating of list of short
% items in the way of tabular alignment. That could be useful for list of
% exercises in math educational literature.  It allows to build one-level
% and two-level tabulated lists.
%   \end{abstract}
%
% \tableofcontents
%
%\newpage
% \section{Building Commands}
%
% \DescribeMacro{tabenum}
% \DescribeMacro{\tabenumitem}
% \DescribeMacro{\item}
% The \env{tabenum} environment prints short items aligned in columns.
%
% Each item starts with |\tabenumitem| command.
% First example of \env{tabenum} list---please note that you may set optional
% argument in the same way as in modified environments from \package{enumerate}
% or \package{paralist} packages:
%\par\begin{small}%
%\begin{verbatim}
%\begin{tabenum}[\bfseries1)]%
%\tabenumitem
%$z=\displaystyle\frac xy$;
%\tabenumitem
%$2^x=9;$\cr
%
%\tabenumitem
%$3^{2x+3}=16 $;
%\tabenumitem
%$z=2x^2+4y^2$;\par
%\tabenumitem
%$u=\sqrt{x^2+y^2+z^2}$;
%\tabenumitem
%$v=gt+\displaystyle\frac{g}{4}t$;\\[1ex]
%\tabenumitem
%$u=2^{5x-3y+z}$;
%\tabenumitem
%$w=(v+7)^2+(u-3)^2$;
%
%
%\tabenumitem
%$5^x=\displaystyle\frac{4}{3} ;$
%\tabenumitem
%$z=(x+1)^2+y^2$;\\*
%\tabenumitem
%$2+5+8+ \ldots +(3n+2)=155$, $n\in \mathrm{N};$
%\tabenumitem
%$t=5u^2+8v^2$;
%\end{tabenum}
%\end{verbatim}
%\end{small}
% \begin{tabenum}[\bfseries1)]%
% \tabenumitem
% $z=\displaystyle\frac xy$;
% \tabenumitem
% $2^x=9;$\cr
%
% \tabenumitem
% $3^{2x+3}=4 $;
% \tabenumitem
% $z=2x^2+4y^2$;\par
% \tabenumitem
% $u=\sqrt{x^2+y^2+z^2}$;
% \tabenumitem
% $v=gt+\displaystyle\frac{g}{4}t$;\\[1ex]
% \tabenumitem
% $u=2^{5x-3y+z}$;
% \tabenumitem
% $w=(v+7)^2+(u-3)^2$;
%
%
% \tabenumitem
% $5^x=\displaystyle\frac{4}{3} ;$
% \tabenumitem
% $z=(x+1)^2+y^2$;\\*
% \tabenumitem
% $2+5+8+ \ldots +(3n+2)=155$, $n\in \mathrm{N};$
% \tabenumitem
% $t=5u^2+8v^2$;
% \end{tabenum}
%
% You may see that empty line (or |\par| command), |\\| command and |\cr|
% do the same: start a new row. The |\\| macro works like in array environment
% and allows optional argument with vertical correction.
% (Also the |\\*| command suppresses breaking between pages, see next example.)
%
% Please note that two or more |\par|'s
% |\cr|'s  or |\\|'s (and also any combination of these three commands)
% create additional empty lines.
%
% You may use the |\item|
% macro instead of |\tabenumitem| in this \env{tabenum} environments (see next examples).
%
% \DescribeMacro{\notabenumitem}
% \DescribeMacro{\noitem}
% \DescribeMacro{\skipitem}
% On the next step you can wish to do the list like before more compact. The
% |\multicolumn| does not work here correctly\footnote{
%   Also the plain \TeX's commands like \cs{omit} and \cs{span}/\cs{multispan} commands,
%   I think, need too complex code.}. At first you may use plain \TeX's
% command |\hidewidth| to hide width of widest column entry; then you may use
% other variant of |\tabenumitem|(|\item|) command---|\notabenumitem|(|\noitem|)---%^^A
% this command increases list entry, but does not create a new column;
% third command |\skipitem| allows you to skip
% one \env{tabenum} column\footnote{The \env{tabenum} column includes two columns:
% it works like \texttt{rl} columns in \env{tabular} environment. Thus, \cs{skipitem}
% equals to \texttt{\&\&} combination.}.
% Next example show usage of these three commands:
%\par\begin{small}%
%\begin{verbatim}
%\begin{tabenum}[\bfseries1)]%
%\item
%$z=\displaystyle\frac xy$;
%\noitem
%$2^x=9;$
%\item
%$3^{2x+3}=4 $.
%\item
%$z=2x^2+4y^2$;\nopagebreak
%
%\item
%$u=\sqrt{x^2+y^2+z^2}$;
%\item
%$v=gt+\displaystyle\frac{g}{4}t$,
%\item
%$u=2^{5x-3y+z}$.\cr
%\item
%$w=(v+7)^2+(u-3)^2$;
%\item
%$5^x=\displaystyle\frac{4}{3} ;$
%\item
%$z=(x+1)^2+y^2$;\\*
%\item
%$2+5+8+ \ldots +(3n+2)=155$,
%   $n\in \mathrm{N};$\hidewidth\skipitem
%\item
%$t=5u^2+8v^2$;
%\end{tabenum}
%\end{verbatim}
%\end{small}%
% \begin{tabenum}[\bfseries1)]%
% \item\label{tabenum:I:1}
% $z=\displaystyle\frac xy$;
% \noitem\label{tabenum:I:2}
% $2^x=9;$
% \item
% $3^{2x+3}=4 $.
% \item
% $z=2x^2+4y^2$;\nopagebreak
%
% \item
% $u=\sqrt{x^2+y^2+z^2}$;
% \item
% $v=gt+\displaystyle\frac{g}{4}t$,
% \item
% $u=2^{5x-3y+z}$.\cr
% \item
% $w=(v+7)^2+(u-3)^2$;
% \item
% $5^x=\displaystyle\frac{4}{3} ;$
% \item
% $z=(x+1)^2+y^2$;\\*
% \item\label{tabenum:I:11}
% $2+5+8+ \ldots +(3n+2)=155$, $n\in \mathrm{N};$\hidewidth\skipitem
% \item
% $t=5u^2+8v^2$;
% \end{tabenum}
% The items \ref{tabenum:I:1} and \ref{tabenum:I:2} were joined in one column: the item \ref{tabenum:I:2}
% used |\noitem| (|\notabenumitem|) command. The item \ref{tabenum:I:11} occupies two columns,
% so it uses the |\hidewidth| (hides item width) and |\skipitem| (skips column of next item) commands.
%
%
% \subsection{Spacing}
%
% Vertical spaces around tabulated list are equal to the list ones |\topsep+\partopsep|.
%
% Space between rows depends on the |\jot| value, like in \package{amsmath} environments
% like \env{align}, \package{gather} etc.
%
% \DescribeMacro{\tabenumsep}
% The horizontal spacing between items of list. It defined like:
% \begin{quote}
% |\renewcommand\tabenumsep{\hskip1em}|
% \end{quote}
% The |\labelsep| parameter is used after item number.
%
% \DescribeMacro{\tabenumindent}
% The |\tabenumindent| command sets left margin of \env{tabenum} environment.
% It can be defined like horizontal space/skip or text,
% for example:
% \begin{quote}
% |\renewcommand\tabenumindent{\hskip\parindent}|
% \end{quote}
% or
% \begin{quote}
% |\renewcommand\tabenumindent{Word }|
% \end{quote}
% That could be useful also
% inside, e.g., a proper list environments like \env{enumerate} or \env{itemise}
% for corrections of left margins.
%
% \subsection{The \env{tabenum} environment inside a proper list}
%
% The version 0.0e of \package{tablists} package enlarges left margin
% of \env{tabenum}, when environment appears inside proper list,
% by the value of left margin of this list. Sometimes left edges
% of \env{tabenum} labels and list labels of the same level are not aligned
% correctly (it happens when |\labelwidth| value is not equal to the natural
% width of labels). In these cases you may define the correction skip
% with the |\tabenumindent| command.
%
% Also the version 0.0e loads labels of empty items of upper/outer list
% at the left side of the first item of \env{tabenum} automatically.  It follows
% mechanism similar to building of empty items in list environment.
% (Empty items of \LaTeX\ list are consecutively stored to special box register
% and printed out at the beginning when the list text appears;
% the same happens at the beginning of \env{tabenum} environment:
% this box, if not empty, is printed at the left of first \env{tabenum} item.)
%\par\begin{small}%
%\begin{verbatim}
%\begin{enumerate}[\bfseries1)]%
%\item
%\begin{tabenum}[a)]%
%...
%\end{verbatim}%
%\end{small}%
% \begin{enumerate}[\bfseries1)]%
% \item
% \begin{tabenum}[a)]%
% \item
% {$z=\displaystyle\frac xy$};
% \noitem
% {$2^x=9;$}
% \item
% {$3^{2x+3}=4 $}.
% \item
% {$z=2x^2+4y^2$};\nopagebreak
%
% \item
% $u=\sqrt{x^2+y^2+z^2}$;
% \item
% $v=gt+\displaystyle\frac{g}{4}t$,
% \item
% $u=2^{5x-3y+z}$.\\
% \item
% $w=(v+7)^2+(u-3)^2$;
% \item
% $5^x=\displaystyle\frac{4}{3} ;$
% \item
% $z=(x+1)^2+y^2$;\cr
% \item
% $2+5+8+ \ldots +(3n+2)=155$, $n\in \mathrm{N};$\hidewidth\skipitem
% \item
% $t=5u^2+8v^2$.%
% \end{tabenum}%
% \end{enumerate}%
%\par\kern-\baselineskip
% \emph{Please note} that for the case of \env{tabenum} environment inside list,
% there is not compatibility between version 0.0e and earlier
% versions---left margin of \env{tabenum} is changed (mechanism
% of placing empty list items on one baseline using compensating vertical
% spaces could work but its calculation was changed, see also next paragraph).\smallskip
%
% \begin{small}%
% \DescribeMacro{\liststrut}
% In versions until 0.0d there was offered the |\liststrut| command in the
% cases when needed to put empty list item on the same line with first item of
% \env{tabenum}. The command raises first line of next text block at the baseline of
% previous. (Without any option it puts negative
% baselineskip. If there is a high element (any math sentence) in the first
% line of next text, it must be written in option without |$|'s,
% the height of this argument is used for compensating skips argument.)
% Since this command must take into account various vertical skips, which
% appear at the start of new item and also height of printing text,
% the adjusting of baselines does not always work correctly.\par
% \end{small}%
% \RestoreSpaces
% \subsection{The \env{subtabenum} environment: second level, variant I}
%
% \DescribeMacro{subtabenum}
% The second level of equations/exersizes list \env{subtabenum} based on \env{tabular}
% environment.
%
%\begin{small}%
%\begin{verbatim}
%\begin{tabenum}[\bfseries 1)]%
%\item
%\begin{subtabenum}[a)]%
%\item
%$z=\displaystyle\frac xy$;
%\noitem
%$2^x=9;$
%\item
%$3^{2x+3}=4 $.
%\item
%$z=2x^2+4y^2$;
%\end{subtabenum}
%
%\item
%\begin{subtabenum}[a)]%
%\item
%$u=\sqrt{x^2+y^2+z^2}$;
%\item
%...
%\end{subtabenum}
%\end{tabenum}
%\end{verbatim}%
%\end{small}%
% \begin{tabenum}[\bfseries 1)]%
% \item
% \begin{subtabenum}[a)]%
% \item
% $z=\displaystyle\frac xy$;
% \noitem
% $2^x=9;$
% \item
% $3^{2x+3}=4 $.
% \item
% $z=2x^2+4y^2$;
% \end{subtabenum}\nopagebreak
%
% \item
% \begin{subtabenum}[a)]
% \item
% $u=\sqrt{x^2+y^2+z^2}$;
% \item
% $v=gt+\gape[b]{\displaystyle\frac{g}{4}}t$,
% \item
% $u=2^{5x-3y+z}$.\\
% \item
% $w=(v+7)^2+(u-3)^2$;
% \item
% $5^x=\gape[b]{\displaystyle\frac{4}{3}} ;$
% \item
% $z=(x+1)^2+y^2$;\\
% \item
% $2+5+8+ \ldots +(3n+2)=155$, $n\in \mathrm{N};$\hidewidth\strut\skipitem
% \item
% $t=5u^2+8v^2$;
% \end{subtabenum}
% \end{tabenum}
% Please note that |\hidewidth| skip is followed by the |\strut| command inside
% \env{subtabenum} environment:
% the skips at the ``edges'' of \env{tabular} columns or before |\\| command doesn't work.
%
% \subsection{The \cs{subtabenumitem}/\cs{subitem} macros: second level, variant II}
%
% The previous example shows that columns were destroyed from one
% \env{subtabenum} environment to another. Also the rows of sublist cannot break between pages.
%
% \DescribeMacro{\subtabenumitem}
% \DescribeMacro{\subitem}
% There is another variant for two-level tabulated list. If you use second option
% in \env{tabenum} environment, you may use |\subtabenumitem|/|\subitem| commands
% for the second level\footnote{If there are not second optional argument, these commands
% work like \cs{tabenumitem}/\cs{item}.}.
%
% If you put |\subtabenumitem| (|\subitem|)  after |\tabenumitem| (|\item|), you get extra space between
% two numbers, created by |\tabenumsep| skip. The |\negtabenumsep| command
% cancels this skip: the version 0.0c included this command inside the |\tabenumitem| definition.
%\par\begin{small}\openup-1pt\enlargethispage{2\baselineskip}
%\begin{verbatim}
%\def\tabenumsep{\qquad}
%\begin{tabenum}[\bfseries 1)][a)]%
%\item
%\subitem
%$z=\displaystyle\frac xy$;
%\nosubitem
%$2^x=9;$
%\subitem
%$3^{2x+3}=4 $.
%\subitem
%$z=2x^2+4y^2$;\\
%\startnumber{4}
%\item
%\subitem
%$u=\sqrt{x^2+y^2+z^2}$;
%\subitem
%$v=gt+\displaystyle\frac{g}{4}t$,
%\subitem
%$u=2^{5x-3y+z}$.\\
%\startsubnumber{7}\subtabrow
%\subitem
%$w=(v+7)^2+(u-3)^2$;
%\subitem
%$5^x=\displaystyle\frac{4}{3} ;$
%\subitem
%$z=(x+1)^2+y^2$;\\
%\subtabrow
%\subitem
%$2+5+8+ \ldots +(3n+2)=155$, $n\in \mathrm{N};$\hidewidth\skipitem
%\subitem
%$t=5u^2+8v^2$;
%\end{tabenum}
%\end{verbatim}%
%\end{small}
% \begin{tabenum}[\bfseries 1)][a)]%
% \item
% \subitem
% $z=\displaystyle\frac xy$;
% \nosubitem
% $2^x=9;$
% \subitem
% $3^{2x+3}=4 $.
% \subitem
% $z=2x^2+4y^2$;\\
% \startnumber{4}\relax
% \item
% \subitem
% $u=\sqrt{x^2+y^2+z^2}$;
% \subitem
% $v=gt+\displaystyle\frac{g}{4}t$,
% \subitem
% $u=2^{5x-3y+z}$.\\
%\startsubnumber{7}\subtabrow
% \subitem
% $w=(v+7)^2+(u-3)^2$;
% \subitem
% $5^x=\displaystyle\frac{4}{3} ;$
% \subitem
% $z=(x+1)^2+y^2$;\\
%\subtabrow
% \subitem
% $2+5+8+ \ldots +(3n+2)=155$, $n\in \mathrm{N};$\hidewidth\skipitem
% \subitem
% $t=5u^2+8v^2$;
% \end{tabenum}
% There were used two commands |\startnumber| and |\startsubnumber| which set
% next start numbers for items of each of two levels\footnote{The \cs{startnumber}
%   command can be used inside any list
%   environment.}. The |\startnumber|
% allows you to divide \env{tabenum} environment and restart with necessary counter.
% You may still use the traditional |\setcounter{enum..}{..}| combination,
% if you know the level of your list and \env{tabenum}/\allowbreak\env{subtabenum} environments.
% For start of the new row from subitem, you need to use~\nobreak\quad1)\nobreak\enskip
% the |\skipitem| command to skip
% column, occupied by the ``parent'' label (|\tabenumitem|) and~\nobreak\quad2)\nobreak\enskip
% the |\negtabenumsep| command to undo column separation.
% These two commands abbreviated by |\subtabrow| command.
%
% \subsection{Placing the QED sign at the end of \env{tabenum} environment}
%
% When the \env{tabenum} environment is used inside \env{proof} environment (the
% \texttt{amsthm} package),
% the better way is to put QED at the end of last \env{tabenum} line. You may use
% the |\tabqedhere| command:
%
% \begin{theorem}%
% You may put the QED sign inside the \env{tabenum} environment.
% \end{theorem}%
% \begin{proof}%
% Let's put our favorite \env{tabenum} environment inside the  \env{proof} one,
% and at the end of it write the |\tabqedhere| (or |\qedhere|) command:
%\begin{verbatim}
%\begin{theorem}
% You may put the QED sign inside the \env{tabenum} environment.
%\end{theorem}
%\begin{proof}
% Let's put...
%\begin{tabenum}[\bfseries 1)][a)]%
%...
%\subitem
%$t=5u^2+8v^2$;\qedhere
%\end{tabenum}
%\end{proof}
%\end{verbatim}%
% \begin{tabenum}[\bfseries 1)][a)]%
% \item
% \subitem
% $z=\displaystyle\frac xy$;
% \nosubitem
% $2^x=9;$
% \subitem
% $3^{2x+3}=4 $.
% \subitem
% $z=2x^2+4y^2$;\\
% \startnumber{4}\relax
% \item
% \subitem
% $u=\sqrt{x^2+y^2+z^2}$;
% \subitem
% $v=gt+\displaystyle\frac{g}{4}t$,
% \subitem
% $u=2^{5x-3y+z}$.\\ \subtabrow
% \subitem
% $w=(v+7)^2+(u-3)^2$;
% \subitem
% $5^x=\displaystyle\frac{4}{3} ;$
% \subitem
% $z=(x+1)^2+y^2$;\\ \subtabrow
% \subitem
% $2+5+8+ \ldots +(3n+2)=155$, $n\in \mathrm{N};$\hidewidth\skipitem
% \subitem
% $t=5u^2+8v^2$;\qedhere
% \end{tabenum}
% \end{proof}%
%
% \subsection{Restoring of \cs{item} as command from list environments}
%
% \DescribeMacro{\restorelistitem}
% For the cases when standard lists appear inside \env{tablist}, you may
% restore original |\item| meaning.
%
% \StopEventually{}\clearpage
%
% \section{Code of package}
%
%  First goes request of \package{makecell} package.
%    \begin{macrocode}
\RequirePackage{makecell}
\@ifpackagelater{makecell}{2007/05/24 V0.1d}{}{\PackageError
    {tablists}{For a successful cooperation we need at least\MessageBreak
    version `2007/05/24 V0.1d' of package makecell,\MessageBreak
        but only version\MessageBreak
          `\csname ver@makecell.\@pkgextension\endcsname'\MessageBreak
        is available}\@eha\endinput}
%    \end{macrocode}
%
% The horizontal separation space between items/columns.
%    \begin{macrocode}
\newcommand\tabenumsep{\hskip1em}
\newcommand\negtabenumsep{%
    \settowidth\@tempdima{\tabenumsep}\hspace*{-\@tempdima}}
%    \end{macrocode}
%
% \begin{macro}{\tabenumitem}
% \begin{macro}{\subtabenumitem}
% Definition of aligned items of `tabular'ed list
%    \begin{macrocode}
\newcommand\tabenumitem{\ifvoid\@labels\else\llap{\box\@labels\hspace{-\@totalleftmargin}}\fi
    &\@ifnextchar[%
    {\TeXr@@tabenumitem{&}}%
    {\TeXr@tabenumitem{&}{\@enumctr}\setcounter{\@subenumctr}\z@}}
%    \end{macrocode}
% The macro for second level of tabulated items.
%    \begin{macrocode}
\newcommand\subtabenumitem{\ifnum\value{\@subenumctr}=\z@\negtabenumsep
    \let\negtabenumsep\relax\fi&
    \@ifnextchar[%]
      {\TeXr@@tabenumitem{&}}%
      {\TeXr@tabenumitem{&}{\@subenumctr}}}
%    \end{macrocode}
% The macros for list and optional numeration. For a successful usage of |\label|--|\ref| mechanism
% the |\refstepcounter| contains were divided in two parts: the |\stepcounter| command was placed in the
% first column, before printing of label, on the second column was placed mechanism of creation
% labels.
% \changes{0.0b}{2007/09/07}{The \cmd{\refstepcounter} command divided into two parts to allow
%       labels for tabular items.}
%    \begin{macrocode}
\newcommand\TeXr@tabenumitem[2]{%
    \stepcounter{#2}{\@nameuse{label#2}}#1
    \protected@edef
      \@currentlabel{\csname p@#2\endcsname\csname the#2\endcsname}}
\@ifdefinable\TeXr@@tabenumitem{}
\def\TeXr@@tabenumitem#1[#2]{{#2}#1}
%    \end{macrocode}
% \end{macro}%
% \end{macro}%
%
% \begin{macro}{\notabenumitem}
% \begin{macro}{\nosubtabenumitem}
% Definition of not aligned items with emulation of spaces around.
% The starred form of |\notabenumitem| skips space before number.
%    \begin{macrocode}
\newcommand\notabenumitem{\@ifstar
    {\TeXr@notabenumitem{\@enumctr}}%
    {\tabenumsep\TeXr@notabenumitem{\@enumctr}}}
%    \end{macrocode}
% The macro for second level of tabulated items.
%    \begin{macrocode}
\newcommand\nosubtabenumitem{\@ifstar
    {\TeXr@notabenumitem{\@subenumctr}}%
    {\tabenumsep\TeXr@notabenumitem{\@subenumctr}}}
\newcommand\TeXr@notabenumitem[1]{\@ifnextchar[%]
    {\TeXr@@tabenumitem{\hskip\labelsep}}%
    {\TeXr@tabenumitem{\hskip\labelsep}{#1}}}
%    \end{macrocode}
% \end{macro}%
% \end{macro}%
%
% \begin{macro}{\skipitem}
% Command allows to skip column of items in tabulated list.
% One item column occupies two tabular ones and needs |&&| for skipping
%    \begin{macrocode}
\newcommand\skipitem[1][1]{\count@=#1\multiply\count@2\advance\count@\@ne
    \eline{\count@}}
%    \end{macrocode}
% \end{macro}%
%
% \begin{macro}{\subtabrow}
% Command allows to skip column of items in tabulated list.
% One item column occupies two tabular ones and needs |&&| for skipping
%    \begin{macrocode}
\newcommand\subtabrow{\skipitem\negtabenumsep}
%    \end{macrocode}
% \end{macro}%
%
% \begin{macro}{\tabenumindent}
% If this command empty there is no any indentation. To set necessary indentation
% you may define this command as any text or horizontal skip.
%    \begin{macrocode}
\newcommand\tabenumindent{}
%    \end{macrocode}
% \end{macro}%
%
% \begin{environment}{tabenum}
% \begin{environment}{rtabenum}
% Start of definition of tabulated lists themselves---\env{tabenum}
% and \env{rtabenum}.
% They allow usage of optional argument in the same way as in
% \env{enumerate} environment from \package{paralist} package.
% The second definition, \env{rtabenum}, replaces |\Alph|/|\alph|
% numeration for \texttt{A}/texttt{a} entries by Russian
% |\Asbuk|/|\asbuk| numeration.
%    \begin{macrocode}
\newenvironment{rtabenum}{\let\Alph\Asbuk\let\alph\asbuk
    \tabenum}{\endtabenum}
%    \end{macrocode}
%
% These environments use mechanism of optional argument from |\@enumlabel@|
% command.
%    \begin{macrocode}
\newcommand\tabenum{\@ifnextchar[{\TeXr@tabenum@}{\TeXr@tabenum@[]}}
\@ifdefinable\TeXr@tabenum@{}
\def\TeXr@tabenum@[#1]{\@ifnextchar[%]
    {\TeXr@tabenum@@[#1]}{\TeXr@tabenum@@[#1][]}}
%    \end{macrocode}
%
% These environments allow two options for item numeration and subitem numeration.
%    \begin{macrocode}
\@ifdefinable\TeXr@tabenum@@{}
\long\def\TeXr@tabenum@@[#1][#2]{\par
    \addpenalty\@beginparpenalty
%    \end{macrocode}
% Vertical spaces around \env{tabenum}  similar to spaces in list environments.
% The space above reduced by |\jot| to make it equal to the space below.
%    \begin{macrocode}
    \@topsepadd\topsep\advance\@topsepadd\partopsep
    \advance\@topsepadd-\jot
    \ifdim\@topsepadd<\z@\@topsepadd=\z@\fi
    \if@nobreak\vspace{-\jot}\else\addvspace\@topsepadd\fi
%    \end{macrocode}
% Interline spacing enlarged by |\jot| as in \package{amsmath}'s environments like
% \env{align} etc.
%    \begin{macrocode}
    \openup\jot
%    \end{macrocode}
% Check for list depth and settings for first level.
% If it exceeded, the error message pops out.
%    \begin{macrocode}
    \ifnum\@enumdepth>\thr@@
      \@toodeep
    \else
      \advance\@enumdepth\@ne
      \edef\@enumctr{enum\romannumeral\the\@enumdepth}%
      \setcounter{\@enumctr}\z@
      \let\item\tabenumitem\let\noitem\notabenumitem
%    \end{macrocode}
% Check for list depth and settings for second level.
% If it exceeded, one level used.
%    \begin{macrocode}
      \ifnum\@enumdepth>\thr@@
        \PackageWarning{tablists}{tablist has one level}{}%
        \let\subitem\tabenumitem\let\nosubitem\notabenumitem
      \else
        \let\subitem\subtabenumitem\let\nosubitem\nosubtabenumitem
        \advance\@enumdepth\@ne
        \edef\@subenumctr{enum\romannumeral\the\@enumdepth}%
        \setcounter{\@subenumctr}\z@
      \fi
    \fi
%    \end{macrocode}
% Checks for first option. If option is absent used default enumerating.
%    \begin{macrocode}
    \def\tempa{#1}\ifx\tempa\empty\else
      \TeXr@enumlabel@{#1}{\@enumctr}\fi
%    \end{macrocode}
% Checks for second option. If option is absent used default enumerating.
%    \begin{macrocode}
    \def\tempa{#2}\ifx\tempa\empty\else
      \TeXr@enumlabel@{#2}{\@subenumctr}\fi
    \@TeXr@tabenum
    }
%    \end{macrocode}
% There building macro for item/subitem labels.
%    \begin{macrocode}
\newcommand\TeXr@enumlabel@[2]{%
    \@tempcnta0
    \TeXr@lab{}\def\TeXr@label{\TeXr@label@{#2}}%
    \expandafter\TeXr@loop\@gobble{}#1\@@@
    \ifnum\@tempcnta=1\else
      \PackageWarning{tablists}{Incorrect label; no or multiple
        counters.\MessageBreak The label is: \@gobble#1}%
    \fi
    \expandafter\edef\csname label#2\endcsname{\the\TeXr@lab}%
    \expandafter\let\csname the#2\endcsname\TeXr@the%
}
%    \end{macrocode}
%
% For column alignment is used |\halign| command of plain \TeX.
% In this case \env{tabenum}/\env{rtabenum} environments can run through a few pages.
%    \begin{macrocode}
\newcommand{\@TeXr@tabenum}{\begingroup
    \let\\\TeXr@arraycr\TeXr@tlist@arrayjot\let\par=\cr
    \settowidth\@tempdima{\tabenumindent}\gdef\tabenumindent{}%
    \settowidth\@tempdimb{\tabenumsep}%
    \ifnum\@listdepth>\z@\advance\@tempdima\@totalleftmargin\fi
    \let\qedhere\tabqedhere
    \halign\bgroup
    \strut##\hskip\@tempdima
        \hskip-\@tempdimb&&\hskip\@tempdimb\hfil##\hskip\labelsep
        &##\hfil\cr}
%    \end{macrocode}
%
% Finish of \env{tabenum} environment.
%    \begin{macrocode}
\def\endtabenum{\crcr\egroup\endgroup\par
    \addpenalty\@endparpenalty
    \advance\@topsepadd\jot
    \addvspace\@topsepadd}
%    \end{macrocode}
% \end{environment}%
% \end{environment}%
%
% \begin{environment}{subtabenum}
% \begin{environment}{rsubtabenum}
% The tabulated list of second level. This list uses a real \env{tabular}
% environment. In this case it is placed in one page. This list as its
% parent environment use optional argument in the way of \package{paralist}'s
% \env{enumerate}.
%    \begin{macrocode}
\newcommand\subtabenum[1][]{\TeXr@tlist@arrayjot
    \setcounter{\@subenumctr}\z@
    \def\tempa{#1}\ifx\tempa\empty\else
      \TeXr@enumlabel@{#1}{\@subenumctr}\fi
    \let\item\subtabenumitem\let\noitem\nosubtabenumitem
    \let\tabenumitem\subtabenumitem\let\notabenumitem\nosubtabenumitem
    \let\subitem\subtabenumitem\let\nosubitem\nosubtabenumitem
    \@TeXr@subtabenum
    }
\newcommand{\@TeXr@subtabenum}{\let\negtabenumsep\relax
    \hbox{\ifnum0=`}\fi
    \begin{tabular}[t]{@{\tabenumindent}l@{}
                        *{12}{r@{\hskip\labelsep}l@{\tabenumsep}}@{}}%
    }
\def\endsubtabenum{\end{tabular}%
    \ifnum0=`{\fi}\gdef\tabenumindent{}}
%    \end{macrocode}
%
% Russified environment with |\Asbuk|/|\asbuk| counters instead of
% |\Alph|/|\alph|.
%    \begin{macrocode}
\newenvironment{rsubtabenum}{\let\Alph\Asbuk\let\alph\asbuk%
   \subtabenum}{\endsubtabenum}
%    \end{macrocode}
% \end{environment}%
% \end{environment}%
%
% \begin{macro}{\arrayjot}
% The redefinition of |\\| command inside tabular to enlarge interline spacing
% by |\jot| value---to create analogous spacing to parent, \env{tabenum}, environment.
%    \begin{macrocode}
\@ifdefinable\TeXr@rixarraycr{}
\let\TeXr@rixarraycr\TeXr@xarraycr
\newcommand\TeXr@Xarraycr{\@ifnextchar[%]
     \TeXr@Xargarraycr {\@argarraycr[\jot]}}
\@ifdefinable\TeXr@Xargarraycr{}
\def\TeXr@Xargarraycr[#1]{\@tempdima\jot\advance\@tempdima#1%
                      \@argarraycr[\@tempdima]}
\@ifdefinable\TeXr@arrayjot{}
\def\TeXr@arrayjot   [#1]{\let\@xarraycr\TeXr@Xarraycr\jot#1}
\def\noarrayjot{\let\@xarraycr\TeXr@rixarraycr}
\newcommand\TeXr@tlist@arrayjot {\@ifnextchar[%]
    \TeXr@arrayjot{\TeXr@arrayjot[\jot]}}
\AtBeginDocument{\let\arrayjot\TeXr@tlist@arrayjot}
\def\TeXr@arraycr{%
  \relax\iffalse{\fi\ifnum 0=`}\fi
  \@ifstar{\vadjust{\penalty\@M}\@xarraycr}\@xarraycr}
%    \end{macrocode}
% \end{macro}
%
% \begin{macro}{\startnumber}
% The variant of |\setcounter{enum..}{..}| sentence.
% This command sets start number of current list (you must put this command
% inside of list or \env{tabenum} environment).
% The |\setcounter{enum..}{..}| sentence needs previous number,
% but  |\startnumber| needs a real start number of a list.
%    \begin{macrocode}
\newcommand\startnumber[1]{\count@#1\advance\count@-1%
    \global\expandafter\csname c@\@enumctr\endcsname\count@}
\newcommand\startsubnumber[1]{\count@#1\advance\count@-1%
    \global\expandafter\csname c@\@subenumctr\endcsname\count@}
%    \end{macrocode}
% \end{macro}
%
% \begin{macro}{\liststrut}
% For the cases of \env{tablist} appears just after empty
% \env{enumerate} (or analogous) list entry.
% This command inserts negative |\baselineskip| and put start of
% `tabular'ed  list on the baseline of list number.
%
% Optional argument allows to align baselines in the case of high
% elements in first line of \env{tablist}. In the version 0.0e the calculation
% of compensating space was changed.
%    \begin{macrocode}
\newcommand\liststrut[1][]{\def\tempa{#1}\ifx\tempa\empty
    \strut\@tempskipa\z@
  \else\settoheight\@tempdima{{$#1$}}\vrule height\@tempdima depth\dp\strutbox width\z@
     \advance\@tempdima-\ht\strutbox
     \ifdim\@tempdima>\lineskip\advance\@tempdima\lineskip\fi
     \@tempskipa\@tempdima
  \fi
  \advance\@tempskipa\baselineskip
%  \advance\@tempskipa-\@outerparskip
  \par\nobreak\vskip-\parskip
%  \advance\@tempskipa\lastskip
   \nobreak\kern-\@tempskipa\@nobreaktrue}
%    \end{macrocode}
% \end{macro}
%
% \begin{macro}{\tabqedhere}
% Places the QED sign at the end of \env{tabenum} line inside the \env{proof} environment.
%    \begin{macrocode}
\newcommand\tabqedhere{\cr
  \noalign{\@par\@tempdima\prevdepth
    \kern-\@tempdima\kern-\baselineskip\kern\lineskip
    {\strut\popQED}\@par\kern\@tempdima}}
%    \end{macrocode}
% \end{macro}
%
% \begin{macro}{\restorelistitem}
% For the cases when standard lists appear inside \env{tablist}.
%    \begin{macrocode}
\let\TeXr@orilistitem\item
\newcommand\restorelistitem{\let\item\TeXr@orilistitem}
%    \end{macrocode}
% \end{macro}
%
% \Finale
\endinput
%%
%% End of file tablists.dtx 