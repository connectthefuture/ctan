\documentclass[DIV=8, pagesize=auto]{scrartcl}

\usepackage{fixltx2e}
\usepackage{etex}
\usepackage{lmodern}
\usepackage[T1]{fontenc}
\usepackage{textcomp}
\usepackage{microtype}
\usepackage{hyperref}

\newcommand*{\pkg}[1]{\textsf{#1}}
\newcommand*{\cs}[1]{\texttt{\textbackslash#1}}
\makeatletter
\newcommand*{\cmd}[1]{\cs{\expandafter\@gobble\string#1}}
\makeatother
\newcommand*{\env}[1]{\texttt{#1}}
\newcommand*{\meta}[1]{\textlangle\textsl{#1}\textrangle}
\newcommand*{\marg}[1]{\texttt{\{}\meta{#1}\texttt{\}}}
\newcommand*{\oarg}[1]{\texttt{[}\meta{#1}\texttt{]}}

\addtokomafont{title}{\rmfamily}

\title{The \pkg{dichokey} package}
\author{Nico Dam}
\date{17 dec 1999}


\begin{document}

\maketitle

\noindent
This \LaTeXe\ style file defines the environment \env{Key} in which
dichotomous identification keys can be constructed.
Numbering of alternatives and indentation are taken of care of
automatically.
The key should be \emph{strictly} dichotomous.


\minisec{Directions for use:}

The key should be constructed within the environment \env{Key}. The
environment has one parameter, that is used as first part of the name
when a species is keyed out (usually an abbreviation of the genus
name).

Within the key, every alternative should begin with the command \cmd{\alter}.
If a species is keyed out, use the command \cmd{\name}\oarg{option}\marg{name}
(otherwise just continue with the next alternative).
The optional argument will be used instead of the parameter provided
on entering the environment \env{Key}.


\minisec{To do:}

provide example file (e.\,g.\ \texttt{rhodocyb.tex})

\end{document}

