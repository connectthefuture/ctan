%
% Modified by Sameer Vijay
% Last Change: Wed Jul 27 2005 13:00 CEST
%
%%%%%%%%%%%%%%%%%%%%%%%%%%%%%%%%%%%%%%%%%%%%%%%%%%%%%%%%%%%%%%%%%%%%%%%%
%
% Sample Notre Dame Thesis/Dissertation
% Using Donald Peterson's ndthesis classfile
%
% Written by Jeff Squyres and Don Peterson
%
% Provided by the Information Technology Committee of
%   the Graduate Student Union
%   http://www.gsu.nd.edu/
%
% Nothing in this document is serious except the format.  :-)
%
% If you have any suggestions, comments, questions, please send e-mail
% to: ndthesis@gsu.nd.edu
%
%%%%%%%%%%%%%%%%%%%%%%%%%%%%%%%%%%%%%%%%%%%%%%%%%%%%%%%%%%%%%%%%%%%%%%%%

%
% Chapter 2
%

\chapter{GNU THINGS ARE GOOD THINGS FOR ALL GRADUATE STUDENTS OR SO IT SEEMS}
\label{chap:golfing}

\section{Gnu See, Gnu Do, Gnu Goes Golfing with Green Golf Genes and
  Gesticulates Grapes}

So why do gnus do what they do?  This is a perennial question that has
yet to be answered definitively by scientists.  Is their future
somehow tied inexplicably with that of humans?  Hard to say, but we do
feed them a lot.  It has even been theorized that rotundness is a
symbol of status or class within the Gnus; those who are more
productive (i.e., cute, furry, friendly) will be fed more than those
who are less so.  So the more rotund, the higher status one has in the
Gnu society.

One could extrapolate this to mean that there is a super-Gnu out there
somewhere; the biggest, routundest Gnu that you've ever seen, probably
of epic proportions!  This would have to be the Leader of Gnus, or LoG
for short.  But the LoG would definitely have to be the cutest,
furriest, and most friendly Gnu that you've ever seen.

\subsection{The LoG}

So how does the LoG get chosen?  Ultimately by humans.  So we can say
that the Gnu society is perhaps the truest democracy that has ever
existed; the leader is chosen by merit, and chosen by complete
outsiders.  As such, the LoG must truly epitomize all that Gnus stand
for: opposedness to overmanagement, cuteness, friendliness, and
furriness~\citep{gloonson98:_gnuly_discov_gnus}.  The gnus themselves
vote at an anual election, based upon these attributes (campagaining
is an anethema to Gnus; see Section~\ref{sec:groovin-gnus}).

\subsubsection{Election Data}
\label{sec:data}

Table~\ref{tbl:votes} shows the latest electoral college voting by the
LoG for the year 2000.  Each Gnu is scored on a scale of one to ten on
the attributes described above.  The results shown in the table are
average scores in each category for all votes; the Gnu's final score
is shown in the final column.

%
% Be aware that page-spanning tables a Very Odd Creatures.  The
% "longtable" environment in LaTeX does some deep Voodoo to make
% everything work out properly.  One of its deep incantations is to
% make the table appear as though it is double spaced.  You can fix
% this by trailing each line with "\\[-6em]" instead of just "\\".
%

\begin{center}
  \begin{longtable}{lccccc}
    \caption{ELECTORAL COLLEGE RESULTS FOR THE LoG ELECTION IN THE YEAR
2000\label{tbl:votes}\/}\\
        \toprule
        Candidate\footnote{note all names begin with G} & Anti-management & Cuteness & Friendliness & Furriness & Aggregate \\
        \midrule
\endfirsthead % Everything above goes at the top of the 1st page only
% As with the first header, we don't want obscene amounts of space for
% subsequent headings either, and eliminate an em of whitespace.
  \caption[]{{\em Continued}}\\
  \midrule
  Candidate & Anti-management & Cuteness & Friendliness & Furriness & Aggregate \\
  \midrule
\endhead % Everything above here (and below the \endfirsthead) goes at the top
         % of continuation pages.  The [] argument prevents a duplicate
         % entry from appearing in the table of contents.
% The following 3 lines are provided as an example only -- per ND
% guidelines, the footer at the bottom of a page for a longtable
% should not have a bottom line.  Only the absolute bottom of the
% table should have a final \bottomline

%  \midline
%  \multicolumn{6}{|r|}{\textit{continued}\ldots} \\
%  \bottomrule
\endfoot % The above section goes at the bottom of continuation pages
  \bottomrule
\endlastfoot % The very last bottom of the table
    Glen & 6.2 & 7.0 & 6.1 & 9.8 & 7.2 \\
    Goober & 6.9 & 2.1 & 5.7 & 4.1 & 4.6 \\
    Genevra & 2.2 & 2.0 & 1.1 & 1.1 & 1.6 \\
    Greg & 8.3 & 0.4 & 1.1 & 9.5 & 4.8 \\
    Gina & 6.0 & 7.8 & 6.4 & 4.9 & 6.2 \\
    Geof & 1.1 & 8.7 & 3.7 & 7.3 & 5.2 \\
    Grendel & 2.8 & 1.7 & 3.4 & 3.2 & 2.7 \\
    Geronimo & 1.2 & 1.2 & 8.8 & 2.2 & 3.3 \\
    Gabrielle & 4.7 & 3.6 & 0.8 & 2.0 & 2.7 \\
    Giovani & 8.4 & 5.8 & 3.4 & 7.4 & 6.2 \\
    Graham & 4.7 & 5.8 & 5.3 & 0 & 3.9 \\
    Gil & 5.9 & 4.0 & 5.5 & 7.6 & 5.7 \\
    Gerald & 2.0 & 3.7 & 8.0 & 4.3 & 4.5 \\
    Guilani & 7.7 & 3.9 & 2.7 & 6.4 & 5.1 \\
    Guido & 7.6 & 4.3 & 6.5 & 1.0 & 4.8 \\
    Godzilla & 5.1 & 2.2 & 5.3 & 6.9 & 4.8 \\
    Gail & 5.7 & 7.9 & 4.1 & 1.0 & 4.6 \\
    Garth & 4.7 & 7.1 & 2.5 & 3.0 & 4.3 \\
    Gavin & 1.1 & 9.5 & 0.4 & 8.0 & 4.7 \\
    George & 9.5 & 4.5 & 9.1 & 7.5 & 7.6 \\
    Gunnar & 1.4 & 5.8 & 4.8 & 6.2 & 4.5 \\
    Gillian & 7.6 & 9.0 & 6.4 & 4.6 & 6.9 \\
    Greta & 1.5 & 0.5 & 0.9 & 7.7 & 2.6 \\
    Gabby & 1.2 & 3.3 & 7.0 & 2.1 & 3.4 \\
    Gaetena & 6.8 & 1.9 & 4.1 & 8.3 & 5.2 \\
    Ganet & 2.3 & 1.1 & 8.5 & 7.3 & 4.8 \\
    Gardenia & 1.8 & 9.5 & 9.9 & 3.0 & 6.0 \\
    Genna & 5.2 & 3.7 & 3.4 & 3.8 & 4.0 \\
    Genesis & 1.7 & 8.3 & 6.7 & 4.9 & 5.4 \\
    Genaveve & 4.7 & 8.9 & 3.4 & 9.2 & 6.5 \\
    Gene & 3.3 & 6.9 & 0.6 & 5.5 & 4.0 \\
    Gilda & 5.2 & 4.6 & 9.9 & 1.4 & 5.2 \\
    Goldie & 8.9 & 9.1 & 2.0 & 8.2 & 7.0 \\
    Grace & 5.9 & 3.2 & 3.1 & 4.3 & 4.1 \\
    Gretchen & 4.5 & 6.5 & 1.6 & 1.3 & 3.4 \\
    Garrick & 4.8 & 5.7 & 9.4 & 5.1 & 6.2 \\
    Gallagher & 7.4 & 0.4 & 7.6 & 0.4 & 3.9 \\
    Gerry & 1.4 & 8.8 & 4.7 & 0.5 & 3.8 \\
    Gertrude & 9.1 & 8.3 & 0.4 & 5.5 & 5.8 \\
    Gehosephet & 6.6 & 2.9 & 8.3 & 4.4 & 5.5 \\
    Gohn & 8.7 & 2.6 & 7.4 & 2.3 & 5.2 \\
    Gibby & 8.7 & 6.9 & 4.7 & 7.2 & 6.9 \\
  \end{longtable}
\end{center}

As you can see from Table~\ref{tbl:votes}, George (my favorite Gnu)
won for the year 2000, with an aggregate score of 7.6.

% % uncomment the following lines,
% if using chapter-wise bibliography
%
% \bibliographystyle{ndnatbib}
% \bibliography{example}
