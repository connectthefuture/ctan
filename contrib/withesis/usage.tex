% usage.tex
%
% This file explains how to use the withesis style
%   it is heavily modelled after a similar chapter by McCauley
%   for the Purdue Thesis style
%
% Eric Benedict, May 2000
%
% It is provided without warranty on an AS IS basis.


\chapter{Using the {\tt withesis} Style}

You can get a copy of the \LaTeX{} style for creating a University
of Wisconsin--Madison thesis or dissertation from:

{\tt http://www.cae.wisc.edu/\verb+~+benedict/LaTeX.html}

After somehow unpacking it, you will have the style files ({\tt withesis.sty}
{\tt withe10.sty}, and {\tt withe12.sty}) as well the files used to create
this document.  The files used for this document can be copied and used as a
template for your own thesis or dissertation.

The final printed form of this document is useful, but the
combination of the source code and final copy form a much more valuable
reference.  Keeping a working copy of the this document can be helpful
when you are later working on your thesis or disseration and want to know
how to do something.  If you find a similar example in this document,
then you can simply look at the corresponding source code and add it to
your document.    Because many parts of this document were written by
different people, the styles and techniques are also different and provide
different ways of achieving the same or similar results.

Because of the typical size of theses, it makes sense to break the document
up into several smaller files.  Usually this is done at the chapter level.
These files can then be {\tt \verb|\include|}d in a {\em root} file.  It is
the {\em root} file that you will run \LaTeX{} on.  For this manual, the
root file is called {\tt main.tex}.

\section{The Root File and the Preamble}
The {\tt \verb|\documentclass|} command is used to tell \LaTeX{} that you will
be using the {\tt withesis} document class and it is the first command in your
root file.  Class options such as {\tt 10pt}, {\tt 12pt}, {\tt msthesis} or
{\tt margincheck} are specified here:

{\tt \verb|\documentclass[12pt,msthesis]{withesis}|}

The class option {\tt msthesis} sets the margins to be appropriate for depositing
with the UW library, namely a 1.25 inch left margin with the remaining margins 1 inch.
The defaults for the title page are also defined for a thesis and for a Master of
Science degree.

The class option {\tt margincheck} will place a small black square at the end of
each line which exceeds the margins.\footnote{In reality, the square is
placed at the end of lines which exceed their {\tt \char92hbox}.  This usually
(but not always) indicates a  margin violation on the right margin.  Left
margin violations aren't indicated and if the margin violation is large enough,
there isn't room for the black box to be visiable.}  This is visible both in the {\tt .dvi} file
as well as in the {\tt .ps} file.

The area immediately following this command is called the {\em preamble} and is
used for things like including different style packages,
defining new macros and declaring the page style.

The style packages can be used to easily change the thesis font.  For example,
this document is set in Times Roman instead of the \LaTeX default of Computer
Modern.  This change was performed by including the {\tt times} package:

{\tt\verb|\usepackage{times}|}\footnote{In this document, the typewriter font
{\tt $\backslash$tt} was redefined to use the Computer Modern font with the command
{\tt $\backslash$renewcommand\{$\backslash$ttdefault\}\{cmtt\}}.  
For more information, see~\cite{goossens}.}

Remember that if you change the fonts from the default Computer Modern to
PostScript ({\em e.g.} Times Roman) then in order to correctly see the
document, you will need to convert the {\tt *.dvi} output into a {\tt *.ps}
file and view the document with a PostScript viewer. This is required since 
most {\tt *.dvi} previewer programs cannot 
display PostScript fonts.  Usually, the previewer will substitute
default fonts so the document may be viewed; however, since the alternate
fonts may not be the same size, the formatting of the document may appear
to be incorrect.

The style package for including Postscript figures, {\tt epsfig}, is included with

{\tt\verb|\usepackage{epsfig}|}

If multiple style packages are required, then they can be combined into one statement
as follows:

{\tt\verb|\usepackage{epsfig,times}|}

Many different style packages are available.  For more information, see~\cite{goossens}.

The page styles are defined using a similar method.
A special style is defined for the {\tt withesis} style:

{\tt\verb|\pagestyle{thesisdraft}|}

This style causes the footer text to become:

{\verb| DRAFT: Do Not Distribute        <time><Date>        <input file name>|}

This appears at the bottom of every page.

In addition to the page style command, the {\tt withesis} has defined several useful
commands which are specified in the preamble.  They include {\tt \verb| \draftmargin|},
{\tt \verb|\draftscreen|}, {\tt \verb|\noappendixtables|}, and
{\tt \verb|\noappendixfigures|}.

The command  {\tt \verb|\draftmargin|} draws a PostScript box with the dimensions of
the margins.  This makes it easy to check that the margins are correct and to see if
any of the text or figures are outside of the required margins.  This box is only visible
in the {\tt .ps} file since it is a PostScript special.


The command  {\tt \verb|\draftscreen|} draws a PostScript screen with the word {\em DRAFT}
in light grey and diagonally across the page.  This screen is only visible
in the {\tt .ps} file since it is a PostScript special.

The commands {\tt \verb|\noappendixtables|} and/or {\tt \verb|\noappendixfigures|} should
be used if the appendix does not have either tables or figures respectively.  These commands
inhibit the Appendix Table or Appendix Figure titles in the List of Tables or List of
Figures.\label{usage:noapp}


If you have specified the {\tt psfig} or {\tt epsfig} document style package, then a useful
command is {\tt \verb|\psdraft|}.  This command will show the bounding box that the figure
would occupy (instead of actually including the figure).  This speeds up the draft copy
printing, reduces toner usage and the drawn box is visible in the {\tt .dvi} file.

The next usual command is {\tt \verb|\begin{document}|}.  The following example is part
of the root file used for this manual.

\begin{quote} \singlespace\footnotesize\tt
\begin{verbatim}
\bibliographystyle{plain}
% prelude.tex
%   - titlepage
%   - dedication
%   - acknowledgments
%   - table of contents, list of tables and list of figures
%   - nomenclature
%   - abstract
%============================================================================


\clearpage\pagenumbering{roman}  % This makes the page numbers Roman (i, ii, etc)



% TITLE PAGE
%   - define \title{} \author{} \date{}
\title{How to \LaTeX\ a Thesis}
\author{Eric R. L. Benedict}
\date{2000}
%   - The default degree is ``Doctor of Philosophy''
%     (unless the document style msthesis is specified
%      and then the default degree is ``Master of Science'')
%     Degree can be changed using the command \degree{}
\degree{Master \TeX nician}
%   - The default is dissertation, unless the document style
%     msthesis was specified in which case it becomes thesis.
%     If msthesis is specified for the MS margins, you can
%     still have a dissertation if you specify \disseration
%\disseration
%   - for a masters project report, specify \project
%\project
%   - for a preliminary report, specify \prelim
\prelim
%   - for a masters thesis, specify \thesis
%\thesis
%   - The default department is ``Electrical Engineering''
%     The department can be changed using the command \department{}
%\department{New Department}
%   - once the above are defined, use \maketitle to generate the titlepage
\maketitle

% COPYRIGHT PAGE
%   - To include a copyright page use \copyrightpage
\copyrightpage

% DEDICATION
\begin{dedication}
To my pet rock, Skippy.
\end{dedication}

% ACKNOWLEDGMENTS
\begin{acknowledgments}
I thank the many people who have done lots of nice things for me.
\end{acknowledgments}

% CONTENTS, TABLES, FIGURES
\tableofcontents
\listoftables
\listoffigures

% NOMENCLATURE
\begin{nomenclature}
\begin{description}
\item{\makebox[0.75in][l]{\TeX}}
       \parbox[t]{5in}{a typesetting system by Donald Knuth~\cite{knuth}.  It
       also refers to the ``plain'' format.  The proper pronounciation
       rhymes with ``heck'' and ``peck'' and does not sound like
       ``hex'' or ``Rex.''\\}

\item{\makebox[0.75in][l]{\LaTeX}}  
        \parbox[t]{5in}{a set of \TeX{} macros originally written by Leslie 
        Lamport~\cite{lamport}.  The proper pronunciation is 
        {\tt l\={a}$\cdot$tek'} and not {\tt l\={a}'$\cdot$teks} (see above).\\}

\item{\makebox[0.75in][l]{{\sc Bib}\TeX}} 
         \parbox[t]{5in}{a bibliography generation program by Oren 
                Patashnik~\cite{lamport}
                that can be used with either plain \TeX{} or \LaTeX{}.\\}

\item{\makebox[0.75in][l]{$C_1$}} Constant 1

\item{\makebox[0.75in][l]{$V$}}    Voltage 

\item{\makebox[0.75in][l]{\$}}     US Dollars
\end{description}
\end{nomenclature}


\advisorname{Bucky J. Badger}
\advisortitle{Assistant Professor}
% ABSTRACT
\begin{umiabstract}
  % !TEX root = main.tex
% !TEX encoding = Windows Latin 1
% !TEX TS-program = pdflatex
% 
% Archivo: abstract.tex (en ingles)


\chapter{Abstract} % No cambiar el titulo
\selectlanguage{english}
\noindent
Duis tristique sollicitudin leo nec consequat. Praesent et dui convallis velit tincidunt fermentum. Mauris cursus purus at sem viverra sed imperdiet sapien imperdiet. Aliquam mattis, elit eget rutrum vulputate, tortor sem pulvinar justo, sit amet mollis felis sem at nibh. Donec malesuada, neque id interdum eleifend, arcu augue porta elit, nec tristique libero metus at massa. Fusce fringilla laoreet rhoncus. Suspendisse potenti. Phasellus dignissim sodales mauris at pharetra. Donec gravida fringilla velit ac rutrum.

Curabitur ornare lectus id diam molestie eu imperdiet nulla tempus. Maecenas vestibulum enim et dui ornare blandit. Vivamus fermentum faucibus viverra. Maecenas at justo sapien. Aenean rhoncus augue mattis purus rhoncus venenatis. Suspendisse metus felis, porttitor in varius in, vulputate at tortor. Aliquam molestie, turpis et malesuada porta, tortor sapien pharetra sapien, ac rhoncus quam dolor a sapien. Pellentesque varius laoreet enim ut auctor. Nullam nec ultricies nisi. Nullam porta lectus et ante consectetur posuere.

Duis tristique sollicitudin leo nec consequat. Praesent et dui convallis velit tincidunt fermentum. Mauris cursus purus at sem viverra sed imperdiet sapien imperdiet. Aliquam mattis, elit eget rutrum vulputate, tortor sem pulvinar justo, sit amet mollis felis sem at nibh. Donec malesuada, neque id interdum eleifend, arcu augue porta elit, nec tristique libero metus at massa. Fusce fringilla laoreet rhoncus. Suspendisse potenti. Phasellus dignissim sodales mauris at pharetra. Donec gravida fringilla velit ac rutrum.

Duis tristique sollicitudin leo nec consequat. Praesent et dui convallis velit tincidunt fermentum. Mauris cursus purus at sem viverra sed imperdiet sapien imperdiet. Aliquam mattis, elit eget rutrum vulputate, tortor sem pulvinar justo, sit amet mollis felis sem at nibh. Donec malesuada, neque id interdum eleifend, arcu augue porta elit, nec tristique libero metus at massa. Fusce fringilla laoreet rhoncus. Suspendisse potenti. Phasellus dignissim sodales mauris at pharetra. Donec gravida fringilla velit ac rutrum.

Curabitur ornare lectus id diam molestie eu imperdiet nulla tempus. Maecenas vestibulum enim et dui ornare blandit. Vivamus fermentum faucibus viverra. Maecenas at justo sapien. Aenean rhoncus augue mattis purus rhoncus venenatis. Suspendisse metus felis, porttitor in varius in, vulputate at tortor. Aliquam molestie, turpis et malesuada porta, tortor sapien pharetra sapien, ac rhoncus quam dolor a sapien. Pellentesque varius laoreet enim ut auctor. Nullam nec ultricies nisi. Nullam porta lectus et ante consectetur posuere.

Duis tristique sollicitudin leo nec consequat. Praesent et dui convallis velit tincidunt fermentum. Mauris cursus purus at sem viverra sed imperdiet sapien imperdiet. Aliquam mattis, elit eget rutrum vulputate, tortor sem pulvinar justo, sit amet mollis felis sem at nibh. Donec malesuada, neque id interdum eleifend, arcu augue porta elit, nec tristique libero metus at massa. Fusce fringilla laoreet rhoncus. Suspendisse potenti. Phasellus dignissim sodales mauris at pharetra. Donec gravida fringilla velit ac rutrum.

\bigskip
\noindent
\textit{Key words:} first word; second word; third word.
% Separar palabras con punto-y-comas.

\checklanguage
% Fin archivo abstract.tex
\endinput 
\end{umiabstract}

\begin{abstract}
  % !TEX root = main.tex
% !TEX encoding = Windows Latin 1
% !TEX TS-program = pdflatex
% 
% Archivo: abstract.tex (en ingles)


\chapter{Abstract} % No cambiar el titulo
\selectlanguage{english}
\noindent
Duis tristique sollicitudin leo nec consequat. Praesent et dui convallis velit tincidunt fermentum. Mauris cursus purus at sem viverra sed imperdiet sapien imperdiet. Aliquam mattis, elit eget rutrum vulputate, tortor sem pulvinar justo, sit amet mollis felis sem at nibh. Donec malesuada, neque id interdum eleifend, arcu augue porta elit, nec tristique libero metus at massa. Fusce fringilla laoreet rhoncus. Suspendisse potenti. Phasellus dignissim sodales mauris at pharetra. Donec gravida fringilla velit ac rutrum.

Curabitur ornare lectus id diam molestie eu imperdiet nulla tempus. Maecenas vestibulum enim et dui ornare blandit. Vivamus fermentum faucibus viverra. Maecenas at justo sapien. Aenean rhoncus augue mattis purus rhoncus venenatis. Suspendisse metus felis, porttitor in varius in, vulputate at tortor. Aliquam molestie, turpis et malesuada porta, tortor sapien pharetra sapien, ac rhoncus quam dolor a sapien. Pellentesque varius laoreet enim ut auctor. Nullam nec ultricies nisi. Nullam porta lectus et ante consectetur posuere.

Duis tristique sollicitudin leo nec consequat. Praesent et dui convallis velit tincidunt fermentum. Mauris cursus purus at sem viverra sed imperdiet sapien imperdiet. Aliquam mattis, elit eget rutrum vulputate, tortor sem pulvinar justo, sit amet mollis felis sem at nibh. Donec malesuada, neque id interdum eleifend, arcu augue porta elit, nec tristique libero metus at massa. Fusce fringilla laoreet rhoncus. Suspendisse potenti. Phasellus dignissim sodales mauris at pharetra. Donec gravida fringilla velit ac rutrum.

Duis tristique sollicitudin leo nec consequat. Praesent et dui convallis velit tincidunt fermentum. Mauris cursus purus at sem viverra sed imperdiet sapien imperdiet. Aliquam mattis, elit eget rutrum vulputate, tortor sem pulvinar justo, sit amet mollis felis sem at nibh. Donec malesuada, neque id interdum eleifend, arcu augue porta elit, nec tristique libero metus at massa. Fusce fringilla laoreet rhoncus. Suspendisse potenti. Phasellus dignissim sodales mauris at pharetra. Donec gravida fringilla velit ac rutrum.

Curabitur ornare lectus id diam molestie eu imperdiet nulla tempus. Maecenas vestibulum enim et dui ornare blandit. Vivamus fermentum faucibus viverra. Maecenas at justo sapien. Aenean rhoncus augue mattis purus rhoncus venenatis. Suspendisse metus felis, porttitor in varius in, vulputate at tortor. Aliquam molestie, turpis et malesuada porta, tortor sapien pharetra sapien, ac rhoncus quam dolor a sapien. Pellentesque varius laoreet enim ut auctor. Nullam nec ultricies nisi. Nullam porta lectus et ante consectetur posuere.

Duis tristique sollicitudin leo nec consequat. Praesent et dui convallis velit tincidunt fermentum. Mauris cursus purus at sem viverra sed imperdiet sapien imperdiet. Aliquam mattis, elit eget rutrum vulputate, tortor sem pulvinar justo, sit amet mollis felis sem at nibh. Donec malesuada, neque id interdum eleifend, arcu augue porta elit, nec tristique libero metus at massa. Fusce fringilla laoreet rhoncus. Suspendisse potenti. Phasellus dignissim sodales mauris at pharetra. Donec gravida fringilla velit ac rutrum.

\bigskip
\noindent
\textit{Key words:} first word; second word; third word.
% Separar palabras con punto-y-comas.

\checklanguage
% Fin archivo abstract.tex
\endinput 
\end{abstract}


\clearpage\pagenumbering{arabic} % This makes the page numbers Arabic (1, 2, etc)
        % Title page, abstract, table of contents, etc
\intro

%
% Используемые далее команды определяются в файле common.tex.
%

% Актуальность работы
\actualitysection
\actualitytext

% Степень разработанности темы исследования
\developmentsection
\developmenttext

% Цели и задачи диссертационной работы
\objectivesection
\objectivetext

% Научная новизна
\noveltysection
\noveltytext

% Теоретическая и практическая значимость
\valuesection
\valuetext

% Методология и методы исследования
\methodssection
\methodstext

% Результаты и положения, выносимые на защиту
\resultssection
\resultstext

% Степень достоверности и апробация результатов
\approbationsection
\approbationtext

% Публикации
\pubsection
\pubtext

% Личный вклад автора
\contribsection
\contribtext

% Структура и объем диссертации
\structsection
\structtext
          % Chapter 1
% Essential LaTeX - Jon Warbrick 02/88
%   - Edited May, July 2000 -E. Benedict


% Copyright (C) Jon Warbrick and Plymouth Polytechnic 1989
% Permission is granted to reproduce the document in any way providing
% that it is distributed for free, except for any reasonable charges for
% printing, distribution, staff time, etc.  Direct commercial
% exploitation is not permitted.  Extracts may be made from this
% document providing an acknowledgment of the original source is
% maintained.

% NOTICE: This document has been edited for use in the UW-Madison
% Example Thesis file.


% counters used for the sample file example
\newcounter{savesection}
\newcounter{savesubsection}


% commands to do 'LaTeX Manual-like' examples

\newlength{\egwidth}\setlength{\egwidth}{0.42\textwidth}

\newenvironment{eg}{\begin{list}{}{\setlength{\leftmargin}%
{0.05\textwidth}\setlength{\rightmargin}{\leftmargin}}%
\item[]\footnotesize}{\end{list}}

\newenvironment{egbox}{\begin{minipage}[t]{\egwidth}}{\end{minipage}}

\newcommand{\egstart}{\begin{eg}\begin{egbox}}
\newcommand{\egmid}{\end{egbox}\hfill\begin{egbox}}
\newcommand{\egend}{\end{egbox}\end{eg}}

% one or two other commands
\newcommand{\fn}[1]{\hbox{\tt #1}}
\newcommand{\llo}[1]{(see line #1)}
\newcommand{\lls}[1]{(see lines #1)}
\newcommand{\bs}{$\backslash$}


\chapter{Essential \LaTeX{}}

This chapter introduces some key ideas behind \LaTeX{} and give you the ``essential''
items of information.  This chapter is an edited form of the paper
``Essential \LaTeX{}'' by Jon Warbrick, Plymouth Polytechnic.

\section{Introduction}
This document is an attempt to give you all the essential
information that you will need in order to use the \LaTeX{} Document
Preparation System.  Only very basic features are covered, and a
vast amount of detail has been omitted.  In a document of this size
it is not possible to include everything that you might need to know,
and if you intend to make extensive use of the program you should
refer to a more complete reference.  Attempting to produce complex
documents using only the information found below will require
much more work than it should, and will probably produce a less
than satisfactory result.

The main reference for \LaTeX{} is {\em The \LaTeX{} User's guide and
Reference Manual\/} by Leslie Lamport.  This contains most of the
information that you will ever need to know about the program, and
you will need access to a copy if you are to use \LaTeX{} seriously.
You should also consider getting a copy of {\em The \LaTeX{}
Companion\/} 

\section{How does \LaTeX{} work?}

In order to use \LaTeX{} you generate a file containing
both the text that you wish to print and instructions to tell \LaTeX{}
how you want it to appear.  You will normally create
this file using your system's text editor.  You can give the file any name you
like, but it should end ``\fn{.TEX}'' to identify the file's contents.
You then get \LaTeX{} to process the file, and it creates a
new file of typesetting commands; this has the same name as your file but
the ``\fn{.TEX}'' ending is replaced by ``\fn{.DVI}''.  This stands for
`{\it D\/}e{\it v\/}ice {\it I\/}ndependent' and, as the name implies, this file
can be used to create output on a range of printing devices.
Your {\em local guide\/} will go into more detail.

Rather than encourage you to dictate exactly how your document
should be laid out, \LaTeX{} instructions allow you describe its
{\em logical structure\/}.  For example, you can think of a quotation
embedded within your text as an element of this logical structure: you would
normally expect a quotation to be displayed in a recognisable style to set it
off from the rest of the text.
A human typesetter would recognise the quotation and handle
it accordingly, but since \LaTeX{} is only a computer program it requires
your help.  There are therefore \LaTeX{} commands that allow you to
identify quotations and as a result allow \LaTeX{} to typeset them correctly.

Fundamental to \LaTeX{} is the idea of a {\em document style\/} that
determines exactly how a document will be formatted.  \LaTeX{} provides
standard document styles that describe how standard logical structures
(such as quotations) should be formatted.  You may have to supplement
these styles by specifying the formatting of logical structures
peculiar to your document, such as mathematical formulae.  You can
also modify the standard document styles or even create an entirely
new one, though you should know the basic principles of typographical
design before creating a radically new style.

There are a number of good reasons for concentrating on the logical
structure rather than on the appearance of a document.  It prevents
you from making elementary typographical errors in the mistaken
idea that they improve the aesthetics of a document---you should
remember that the primary function of document design is to make
documents easier to read, not prettier.  It is more flexible, since
you only need to alter the definition of the quotation style
to change the appearance of all the quotations in a document.  Most
important of all, logical design encourages better writing.
A visual system makes it easier to create visual effects rather than
a coherent structure; logical design encourages you to concentrate on
your writing and makes it harder to use formatting as a substitute
for good writing.

\section{A Sample \LaTeX{} file}


\begin{figure} %---------------------------------------------------------------
{\singlespace\tt\footnotesize\begin{verbatim}
 1: % SMALL.TEX -- Released 5 July 1985
 2: % USE THIS FILE AS A MODEL FOR MAKING YOUR OWN LaTeX INPUT FILE.
 3: % EVERYTHING TO THE RIGHT OF A  %  IS A REMARK TO YOU AND IS IGNORED
 4: % BY LaTeX.
 5: %
 6: % WARNING!  DO NOT TYPE ANY OF THE FOLLOWING 10 CHARACTERS EXCEPT AS
 7: % DIRECTED:        &   $   #   %   _   {   }   ^   ~   \
 8:
 9: \documentclass[11pt,a4]{article}  % YOUR INPUT FILE MUST CONTAIN THESE
10: \begin{document}                  % TWO LINES PLUS THE \end COMMAND AT
11:                                   % THE END
12:
13: \section{Simple Text}          % THIS COMMAND MAKES A SECTION TITLE.
14:
15: Words are separated by one or    more      spaces.  Paragraphs are
16:     separated by one or more blank lines.  The output is not affected
17: by adding extra spaces or extra blank lines to the input file.
18:
19:
20: Double quotes are typed like this: ``quoted text''.
21: Single quotes are typed like this: `single-quoted text'.
22:
23: Long dashes are typed as three dash characters---like this.
24:
25: Italic text is typed like this: {\em this is italic text}.
26: Bold   text is typed like this: {\bf this is  bold  text}.
27:
28: \subsection{A Warning or Two}        % THIS MAKES A SUBSECTION TITLE.
29:
30: If you get too much space after a mid-sentence period---abbreviations
31: like etc.\ are the common culprits)---then type a backslash followed by
32: a space after the period, as in this sentence.
33:
34: Remember, don't type the 10 special characters (such as dollar sign and
35: backslash) except as directed!  The following seven are printed by
36: typing a backslash in front of them:  \$  \&  \#  \%  \_  \{  and  \}.
37: The manual tells how to make other symbols.
38:
39: \end{document}                    % THE INPUT FILE ENDS LIKE THIS
\end{verbatim}  }

\caption{A Sample \LaTeX{} File}\label{fig:sample}

\end{figure} %-----------------------------------------------------------------



Have a look at the example \LaTeX{} file in Figure~\ref{fig:sample}.  It
is a slightly modified copy of the standard \LaTeX{} example file
\fn{SMALL.TEX}.  The line numbers down the left-hand side
are not part of the file, but have been added to make it easier to
identify various portions.

Try entering this file (without the line numbers), save the text as \fn{small.tex},
next run \LaTeX{} on it, and then view the output:

{\tt \singlespace\begin{verbatim}
% latex small
% xdvi small               # displays the output on the screen
% dvips -o small.ps small  # to create a PostScript file, small.ps
% lp -d<printer> small.ps  # to print
\end{verbatim}}

\subsection{Running Text}

Most documents consist almost entirely of running text---words formed
into sentences, which are in turn formed into paragraphs---and the example file
is no exception. Describing running text poses no problems, you just type
it in naturally. In the output that it produces, \LaTeX{} will fill
lines and adjust the
spacing between words to give tidy left and right margins.
The spacing and distribution of the words in your input
file will have no effect at all on the eventual output.
Any number of spaces in your input file
are treated as a single space by \LaTeX{}, it also regards the
end of each line as a space between words \lls{15--17}.
A new paragraph is
indicated by a blank line in your input file, so don't leave
any blank lines unless you really wish to start a paragraph.

\LaTeX{} reserves a number of the less common keyboard characters for its
own use. The ten characters
\begin{quote}\begin{verbatim}
#  $  %  &  ~  _  ^  \  {  }
\end{verbatim}\end{quote}
should not appear as part of your text, because if they do
\LaTeX{} will get confused.

\subsection{\LaTeX{} Commands}

There are a number of words in the file that start `\verb|\|' \lls{9,
10 and 13}.  These are \LaTeX{} {\em commands\/} and they describe
the structure of your document. There are a number of things that you
should realize about these commands:
\begin{itemize}

\item All \LaTeX{} commands consist of a `\verb|\|' followed by one or more
characters.

\item \LaTeX{} commands should be typed using the correct mixture of upper- and
lower-case letters.  \verb|\BEGIN| is {\em not\/} the same as \verb|\begin|.

\item Some commands are placed within your text.  These are used to
switch things, like different typestyles, on and off. The \verb|\em|
command is used like this to emphasize text, normally by changing to
an {\it italic\/} typestyle \llo{25}.  The command and the text are
always enclosed between `\verb|{|' and `\verb|}|'---the `\verb|{\em|'
turns the effect on and and the `\verb|}|' turns it off.

\item There are other commands that look like
\begin{quote}\begin{verbatim}
\command{text}
\end{verbatim}\end{quote}
In this case the text is called the ``argument'' of the command.  The
\verb|\section| command is like this \llo{13}.
Sometimes you have to use curly brackets `\verb|{}|' to enclose the argument,
sometimes square brackets `\verb|[]|', and sometimes both at once.
There is method behind this apparent madness, but for the
time being you should be sure to copy the commands exactly as given.

\item When a command's name is made up entirely of letters, you must make sure
that the end of the command is marked by something that isn't a letter.
This is usually either the opening bracket around the command's argument, or
it's a space.  When it's a space, that space is always ignored by \LaTeX. We
will see later that this can sometimes be a problem.

\end{itemize}

\subsection{Overall structure}

There are some \LaTeX{} commands that must appear in every document.
The actual text of the document always starts with a
\verb|\begin{document}| command and ends with an \verb|\end{document}|
command \lls{10 and 39}.  Anything that comes after the \break
\verb|\end{document}| command is ignored.  Everything that comes
before the \break\verb|\begin{document}| command is called the
{\em preamble\/}. The preamble can only contain \LaTeX{} commands
to describe the document's style.

One command that must appear in the preamble is the
\verb|\documentclass| command \llo{9}.  This command specifies the
overall style for the document.  Our example file is a simple
technical document, and uses the {\tt article\/} class.  The document
you are reading was produced with the {\tt withesis\/} class. There
are other classes that you can use, as you will find out later on in
this document.

\subsection{Other Things to Look At}

\LaTeX{} can print both opening and closing quote characters, and can manage
either of these either single or double.  To do this it uses the two quote
characters from your keyboard: {\tt `} and {\tt '}. You will probably think of
{\tt '} as the ordinary single quote character which probably looks like
{\tt\symbol{'23}} or {\tt\symbol{'15}} on your keyboard,

and {\tt `} as a ``funny'' character that probably appears as
{\tt\symbol{'22}}. You type these characters once for single quote
\llo{21},  and twice for double quotes \llo{20}. The double quote
character {\tt "} itself is almost never used and should not be used
unless you want your text to look "funny" (compare the quote in the
previous sentence).

\LaTeX{} can produce three different kinds of dashes.
A long dash, for use as a punctuation symbol, as is typed as three dash
characters in a row, like this `\verb|---|' \llo{23}.  A shorter dash,
used between numbers as in `10--20', is typed as two dash
characters in a row, while a single dash character is used as a hyphen.

From time to time you will need to include one or more of the \LaTeX{}
special symbols in your text.  Seven of them can be printed by
making them into commands by proceeding them by backslash
\llo{36}.  The remaining three symbols can be produced by more
advanced commands, as can symbols that do not appear on your keyboard
such as \dag, \ddag, \S, \pounds, \copyright, $\sharp$ and $\clubsuit$.

It is sometimes useful to include comments in a \LaTeX{} file, to remind
you of what you have done or why you did it.  Everything to the
right of a \verb|%| sign is ignored by \LaTeX{}, and so it can
be used to introduce a comment.

\section{Document Classes and Class Options}\label{sec:styles}

There are four standard document classes available in \LaTeX:
\nobreak

\begin{description}

\item[{\tt article}]  intended for short documents and articles for publication.
Articles do not have chapters, and when \verb|\maketitle| is used to generate

a title (see Section~\ref{sec:title}) it appears at the top of the first page

rather than on a page of its own.

\item[{\tt report}] intended for longer technical documents.
It is similar to
{\tt article}, except that it contains chapters and the title appears on a page
of its own.

\item[{\tt book}] intended as a basis for book publication.  Page layout is
adjusted assuming that the output will eventually be used to print on
both sides of the paper.

\item[{\tt letter}]  intended for producing personal letters.  This style
will allow you to produce all the elements of a well laid out letter:
addresses, date, signature, etc.
\end{description}

An additional document class, the one used for this document and for
University of Wisconsin--Madison theses, is \fn{withesis}.


These standard classes can be modified by a number of {\em class
options\/}. They appear in square brackets after the
\verb|\documentclass| command. Only one class can ever be used but
you can have more than one class option, in which case their names
should be separated by commas.  The standard style options are:
\begin{description}

\item[{\tt 11pt}]  prints the document using eleven-point type for the running
 text
rather that the ten-point type normally used. Eleven-point type is about
ten percent larger than ten-point.

\item[{\tt 12pt}]  prints the document using twelve-point type for the running
 text
rather than the ten-point type normally used. Twelve-point type is about
twenty percent larger than ten-point.

\item[{\tt twoside}]  causes documents in the article or report styles to be
formatted for printing on both sides of the paper.  This is the default for the
book style.

\item[{\tt twocolumn}] produces two column on each page.

\item[{\tt titlepage}]  causes the \verb|\maketitle| command to generate a
title on a separate page for documents in the \fn{article} style.
A separate page is always used in both the \fn{report} and \fn{book} styles.

\end{description}

The University of Wisconsin--Madison thesis style, \fn{withesis} also
has some class options defined.  These class options are for
ten-point type (\fn{10pt}), tweleve-point type (\fn{12pt}), two-sided
printing (\fn{twoside}), Master Thesis margins (\fn{msthesis}) and an
option to print a small black box on lines which exceed the margins
(\fn{margincheck}).

\section{Environments}

We mentioned earlier the idea of identifying a quotation to \LaTeX{} so that
it could arrange to typeset it correctly. To do this you enclose the
quotation between the commands \verb|\begin{quotation}| and
\verb|\end{quotation}|.
This is an example of a \LaTeX{} construction called an {\em environment\/}.
A number of
special effects are obtained by putting text into particular environments.

\subsection{Quotations}

There are two environments for quotations: \fn{quote} and \fn{quotation}.
\fn{quote} is used either for a short quotation or for a sequence of
short quotations separated by blank lines:
\egstart\singlespace
\begin{verbatim}
US presidents ... remarks:
\begin{quote}
The buck stops here.

I am not a crook.
\end{quote}
\end{verbatim}
\egmid%
US presidents have been known for their pithy remarks:
\begin{quote}
The buck stops here.

I am not a crook.
\end{quote}
\egend

Use the \fn{quotation} environment for quotations that consist of more
than one paragraph.  Paragraphs in the input are separated by blank
lines as usual:
\egstart\singlespace
\begin{verbatim}

Here is some advice to remember:
\begin{quotation}
Environments for making
...other things as well.

Many problems
...environments.
\end{quotation}
\end{verbatim}
\egmid%
Here is some advice to remember:
\begin{quotation}
Environments for making quotations
can be used for other things as well.

Many problems can be solved by
novel applications of existing
environments.
\end{quotation}
\egend

\subsection{Centering and Flushing}

Text can be centered on the page by putting it within the \fn{center}
environment, and it will appear flush against the left or right margins if it
is placed within the \fn{flushleft} or \fn{flushright} environments.

Text within these environments will be formatted in the normal way, in
{\samepage
particular the ends of the lines that you type are just regarded as spaces.  To
indicate a ``newline'' you need to type the \verb|\\| command.  For example:
\egstart\singlespace
\begin{verbatim}
\begin{center}
one
two
three \\
four \\
five
\end{center}

\end{verbatim}
\egmid%
\begin{center}

one
two
three \\
four \\

five
\end{center}
\egend
}

\subsection{Lists}

There are three environments for constructing lists.  In each one each new
item is begun with an \verb|\item| command.  In the \fn{itemize} environment
the start of each item is given a marker, in the \fn{enumerate}
environment each item is marked by a number.  These environments can be nested
within each other in which case the amount of indentation used
is adjusted accordingly:
\egstart\singlespace

\begin{verbatim}
\begin{itemize}
\item Itemized lists are handy.
\item However, don't forget
  \begin{enumerate}
  \item The `item' command.
  \item The `end' command.
  \end{enumerate}
\end{itemize}
\end{verbatim}
\egmid%
\begin{itemize}
\item Itemized lists are handy.
\item However, don't forget
  \begin{enumerate}
  \item The `item' command.
  \item The `end' command.
  \end{enumerate}
\end{itemize}
\egend


The third list making environment is \fn{description}.  In a description you
specify the item labels inside square brackets after the \verb|\item| command.
For example:
\egstart\singlespace
\begin{verbatim}
Three animals that you should
know about are:
\begin{description}
  \item[gnat] A small
            animal...
  \item[gnu] A large
           animal...
  \item[armadillo] A ...
\end{description}
\end{verbatim}
\egmid%
Three animals that you should
know about are:
\begin{description}
  \item[gnat] A small animal that causes no end of trouble.
  \item[gnu] A large animal that causes no end of trouble.
  \item[armadillo] A medium-sized animal.
\end{description}
\egend

\subsection{Tables}

Because \LaTeX{} will almost always convert a sequence of spaces
into a single space, it can be rather difficult to lay out tables.
See what happens in this example
 \nolinebreak
\begin{eg}
\begin{minipage}[t]{0.55\textwidth} \singlespace
\begin{verbatim}
\begin{flushleft}
Income  Expenditure Result   \\
20s 0d  19s 11d     happiness \\
20s 0d  20s 1d      misery  \\
\end{flushleft}
\end{verbatim}
\end{minipage}
\begin{minipage}[t]{0.3\textwidth}
\begin{flushleft}
Income  Expenditure Result   \\
20s 0d  19s 11d     happiness \\
20s 0d  20s 1d      misery  \\
\end{flushleft}
\end{minipage}
\end{eg}

The \fn{tabbing} environment overcomes this problem. Within it you
set tabstops and tab to them much like you do on a typewriter.
Tabstops are set with the \verb|\=| command, and the \verb|\>|
command moves to the next stop.  The \verb|\\| command is used to
separate each line.  A line that ends \verb|\kill| produces no
output, and can be used to set tabstops:
\nolinebreak
\begin{eg}
\begin{minipage}[t]{0.6\textwidth}
\singlespace
\begin{verbatim}
\begin{tabbing}
Income \=Expenditure \=    \kill
Income \>Expenditure \>Result \\
20s 0d \>19s 11d \>Happiness \\
20s 0d \>20s 1d  \>Misery    \\
\end{tabbing}
\end{verbatim}
\end{minipage}
\vspace{1ex}
\begin{minipage}[t]{0.35\textwidth}
\begin{tabbing}
\singlespace
Income \=Expenditure \=    \kill
Income \>Expenditure \>Result \\
20s 0d \>19s 11d \>Happiness \\
20s 0d \>20s 1d  \>Misery    \\
\end{tabbing}
\end{minipage}
\end{eg}

Unlike a typewriter's tab key, the \verb|\>| command always moves to the next
tabstop in sequence, even if this means moving to the left.  This can cause
text to be overwritten if the gap between two tabstops is too small.

\subsection{Verbatim Output}

Sometimes you will want to include text exactly as it appears on a terminal
screen.  For example, you might want to include part of a computer program.
Not only do you want \LaTeX{} to stop playing around with the layout of your
text, you also want to be able to type all the characters on your keyboard
without confusing \LaTeX. The \fn{verbatim} environment has this effect:

\egstart

\begin{flushleft}\singlespace
\verb|The section of program in|  \\
 \verb|question is :|\\
 \verb|\begin{verbatim}|           \\
\verb|{ this finds %a & %b }|     \\[2ex]

\verb|for i := 1 to 27 do|        \\
\ \ \ \verb|begin|                \\
\ \ \ \verb|table[i] := fn(i);|   \\

\ \ \ \verb|process(i)|           \\
\ \ \ \verb|end;|                 \\
\verb|\end{verbatim}|
\end{flushleft}
\egmid%
The section of program in
question is:
\begin{verbatim}
{ this finds %a & %b }

for i := 1 to 27 do
   begin
   table[i] := fn(i);
   process(i)
   end;

\end{verbatim}
\egend

The \fn{withesis} document style also provides the command {\tt \verb|\verbatimfile{foo.fe}|}
which will read in the file {\tt foo.fe} into the document in \fn{verbatim} format with
the font \verb|\tt|.  See Appendix~\ref{matlab} for an example.

\section{Type Styles}

We have already come across the \verb|\em| command for changing
typeface.  Here is a full list of the available typefaces:
\begin{quote}\singlespace\begin{tabbing}
\verb|\sc|~~ \= \sc Small Caps~~~ \= \verb|\sc|~~ \= \sc Small Caps~~~
                                  \= \verb|\sc|~~ \=                   \kill
\verb|\rm|   \> \rm Roman         \> \verb|\it|   \> \it Italic
                                  \> \verb|\sc|   \> \sc Small Caps    \\
\verb|\em|   \> \em Emphatic      \> \verb|\sl| \> \sl Slanted
                                  \> \verb|\tt|   \> \tt Typewriter     \\
\verb|\bf|   \> \bf Boldface      \> \verb|\sf| \> \sf Sans Serif
\end{tabbing}\end{quote}

Remember that these commands are used {\em inside\/} a pair of braces to limit
the amount of text that they effect.  In addition to the eight typeface
commands, there are a set of commands that alter the size of the type.  These
commands are:
\begin{quotation}\singlespace\begin{tabbing}
\verb|\footnotesize|~~ \= \verb|\footnotesize|~~ \= \verb|\footnotesize| \=
 \kill
\verb|\tiny|           \> \verb|\small|          \> \verb|\large|        \>
\verb|\huge|  \\
\verb|\scriptsize|     \> \verb|\normalsize|     \> \verb|\Large|        \>
\verb|\Huge|  \\
\verb|\footnotesize|   \>                        \> \verb|\LARGE|
\end{tabbing}\end{quotation}

\section{Sectioning Commands and Tables of Contents}
\label{ess:sectioning}

Technical documents, like this one, are often divided into sections.
Each section has a heading containing a title and a number for easy
reference.  \LaTeX{} has a series of commands that will allow you to identify
different sorts of sections.  Once you have done this \LaTeX{} takes on the
responsibility of laying out the title and of providing the numbers.

The commands that you can use are:
\begin{quote}\singlespace\begin{tabbing}
\verb|\subsubsection| \= \verb|\subsubsection|~~~~~~~~~~ \=           \kill
\verb|\chapter|       \> \verb|\subsection|    \> \verb|\paragraph|    \\
\verb|\section|       \> \verb|\subsubsection| \> \verb|\subparagraph| \\
\end{tabbing}\end{quote}
The naming of these last two is unfortunate, since they do not really have
anything to do with `paragraphs' in the normal sense of the word; they are just
lower levels of section.  In most document styles, headings made with
\verb|\paragraph| and \verb|\subparagraph| are not numbered.  \verb|\chapter|
is not available in document style \fn{article}.  The commands should be used
in the order given, since sections are numbered within chapters, subsections
within sections, etc.

A seventh sectioning command, \verb|\part|, is also available.  Its use is
always optional, and it is used to divide a large document into series of
parts.  It does not alter the numbering used for any of the other commands.

Including the command \verb|\tableofcontents| in you document will cause a
contents list to be included, containing information collected from the various
sectioning commands.  You will notice that each time your document is run
through \LaTeX{} the table of contents is always made up of the headings from
the previous version of the document.  This is because \LaTeX{} collects
information for the table as it processes the document, and then includes it
the next time it is run.  This can sometimes mean that the document has to be
processed through \LaTeX{} twice to get a correct table of contents.

\section{Producing Special Symbols}

You can include in you \LaTeX{} document a wide range of symbols that do not
appear on you your keyboard. For a start, you can add an accent to any letter:
\begin{quote}\singlespace\begin{tabbing}

\t{oo} \= \verb|\t{oo}|~~~ \=
\t{oo} \= \verb|\t{oo}|~~~ \=
\t{oo} \= \verb|\t{oo}|~~~ \=
\t{oo} \= \verb|\t{oo}|~~~ \=
\t{oo} \= \verb|\t{oo}|~~~ \=
\t{oo} \=                       \kill

\a`{o} \> \verb|\`{o}|  \> \~{o}  \> \verb|\~{o}|  \> \v{o}  \> \verb|\v{o}| \>
\c{o}  \> \verb|\c{o}|  \> \a'{o} \> \verb|\'{o}|  \\
\a={o} \> \verb|\={o}|  \> \H{o}  \> \verb|\H{o}|  \> \d{o}  \> \verb|\d{o}| \>
\^{o}  \> \verb|\^{o}|  \> \.{o}  \> \verb|\.{o}|  \\
\t{oo} \> \verb|\t{oo}| \> \b{o}  \> \verb|\b{o}|  \\  \"{o} \> \verb|\"{o}| \>
\u{o}  \> \verb|\u{o}|  \\
\end{tabbing}\end{quote}

A number of other symbols are available, and can be used by including the
following commands:
\begin{quote}\singlespace\begin{tabbing}

\LaTeX~\= \verb|\copyright|~~~~ \= \LaTeX~\= \verb|\copyright|~~~~ \=
\LaTeX~\=  \kill

\dag       \> \verb|\dag|       \> \S     \> \verb|\S|     \>
\copyright \> \verb|\copyright| \\
\ddag      \> \verb|\ddag|      \> \P     \> \verb|\P|     \>
\pounds    \> \verb|\pounds|    \\
\oe        \> \verb|\oe|        \> \OE    \> \verb|\OE|    \>
\ae        \> \verb|\AE|        \\
\AE        \> \verb|\AE|        \> \aa    \> \verb|\aa|    \>
\AA        \> \verb|\AA|        \\
\o         \> \verb|\o|         \> \O     \> \verb|\O|     \>
\l         \> \verb|\l|         \\
\L         \> \verb|\E|         \> \ss    \> \verb|\ss|    \>
?`         \> \verb|?`|         \\
!`         \> \verb|!`|         \> \ldots \> \verb|\ldots| \>
\LaTeX     \> \verb|\LaTeX|     \\
\end{tabbing}\end{quote}
There is also a \verb|\today| command that prints the current date. When you
use these commands remember that \LaTeX{} will ignore any spaces that
follow them, so that you can type `\verb|\pounds 20|' to get `\pounds 20'.
However, if you type `\verb|LaTeX is wonderful|' you will get `\LaTeX is
wonderful'---notice the lack of space after \LaTeX.
To overcome this problem you can follow any of these commands by a
pair of empty brackets and then any spaces that you wish to include,
and you will see that
\verb|\LaTeX{} really is wonderful!| (\LaTeX{} really is wonderful!).

\section{Titles}\label{sec:title}

Most documents have a title.  To title a \LaTeX{} document, you include the
following commands in your document, usually just after
\verb|begin{document}|.
\begin{quote}\singlespace\begin{verbatim}
\title{required title}
\author{required author}
\date{required date}
\maketitle
\end{verbatim}\end{quote}
If there are several authors, then their names should be separated by
\verb|\and|; they can also be separated by \verb|\\| if you want them to be
centred on different lines.  If the \verb|\date| command is left out, then the
current date will be printed.
\egstart
\singlespace
\begin{verbatim}
\title{Essential \LaTeX}
\author{J Warbrick \and An Other}
\date{14th February 1988}
\maketitle
\end{verbatim}
\egmid
\begin{center}
{\normalsize Essential \LaTeX}\\[4ex]
J Warbrick\hspace{1em}A N Other\\[2ex]
14th February 1988
\end{center}
\egend

The exact appearance of the title varies depending on
the document style.  In styles \fn{report} and \fn{book} the title appears on a
page of its own. In the \fn{article} style it normally appears at the top
of the first page, the style option \fn{titlepage} will alter this (see
Section~\ref{sec:styles}).  In the \fn{withesis} style, the title is created on a
seperate page in the format appropriate to a UW-Madison thesis or dissertation.

\section{Errors}

When you create a new input file for \LaTeX{} you will probably make mistakes.
Everybody does, and it's nothing to be worried about.  As with most computer
programs, there are two sorts of mistake that you can make: those that \LaTeX{}
notices and those that it doesn't.  To take a rather silly example, since
\LaTeX{} doesn't understand what you are saying it isn't going to be worried if
you mis-spell some of the words in your text.  You will just have to accurately
proof-read your printed output.  On the other hand, if you mis-spell one of
the environment names in your file then \LaTeX won't know what you want it
to do.

When this sort of thing happens, \LaTeX{} prints an error message on your
terminal screen and then stops and waits for you to take some action.
Unfortunately, the error messages that it produces are rather user-unfriendly
and a little frightening.  Nevertheless, if you know where to look they
will probably tell you where the error is and went wrong.

Consider what would happen if you mistyped \verb|\begin{itemize}| so that it
became \break\verb|\begin{itemie}|.  When \LaTeX{} processes this instruction, it
displays the following on your terminal:
\begin{quote}\singlespace\begin{verbatim}
LaTeX error.  See LaTeX manual for explanation.
              Type  H <return>  for immediate help.
! Environment itemie undefined.
\@latexerr ...for immediate help.}\errmessage {#1}
                                                  \endgroup
l.140 \begin{itemie}

?
\end{verbatim}\end{quote}
After typing the `?' \LaTeX{} stops and waits for you to tell it what to do.

The first two lines of the message just tell you that the error was detected by
\LaTeX{}. The third line, the one that starts `!' is the {\em error indicator}.
 It
tells you what the problem is, though until you have had some experience of
\LaTeX{} this may not mean a lot to you.  In this case it is just telling you
that it doesn't recognise an environment called \fn{itemie}.
The next two lines tell you what
\LaTeX{} was doing when it found the error, they are irrelevant at the moment
and can be ignored. The final line is called the {\em error locator}, and is
a copy of the line from your file that caused the problem.
It start with a line number to help you to find it in your file, and
if the error was in the middle of a line it will be shown
broken at the point where \LaTeX{} realised that there was an error.  \LaTeX{}
can sometimes pass the point where the real error is before discovering that
something is wrong, but it doesn't usually get very far.

At this point you could do several things.  If you knew enough about \LaTeX{}
you might be able to fix the problem, or you could type `X' and press the
return key to stop \LaTeX{} running while you go and correct the error.  The
best thing to do, however, is just to press the return key.  This will allow
\LaTeX{} to go on running as if nothing had happened.  If you have made one
mistake, then you have probably made several and you may as well try to find
them all in one go.  It's much more efficient to do it this way than to run
\LaTeX{} over and over again fixing one error at a time. Don't worry about
remembering what the errors were---a copy of all the error messages is being
saved in a {\em log\/} file so that you can look at them afterwards.

If you look at the line that caused the error it's normally obvious what the
problem was.  If you can't work out what you problem is look at the hints
below, and if they don't help consult Chapter~6 of the manual~\cite{lamport}.
  It contains a
list of all of the error messages that you are likely to encounter together with
some hints as to what may have caused them.

Some of the most common mistakes that cause errors are
\begin{itemize}
\item A mispelled command or environment name.
\item Improperly matched `\verb|{|' and `\verb|}|'---remember that they should
 always
come in pairs.
\item Trying to use one of the ten special characters \verb|# $ % & _ { } ~ ^|
and \verb|\| as an ordinary printing symbol.
\item A missing \verb|\end| command.
\item A missing command argument (that's the bit enclosed in '\verb|{|' and
`\verb|}|').
\end{itemize}

One error can get \LaTeX{} so confused that it reports a series of spurious
errors as a result.  If you have an error that you understand, followed by a
series that you don't, then try correcting the first error---the rest
may vanish as if by magic.
Sometimes \LaTeX{} may write a {\tt *} and stop without an error message.  This
is normally caused by a missing \verb|\end{document}| command, but other errors
can cause it.  If this happens type \verb|\stop| and press the return key.

Finally, \LaTeX{} will sometimes print {\em warning\/} messages.  They report
problems that were not bad enough to cause \LaTeX{} to stop processing, but
nevertheless may require investigation.  The most common problems are
`overfull' and `underfull' lines of text.  A message like:
\begin{quote}\footnotesize\begin{verbatim}
Overfull \hbox (10.58649pt too wide) in paragraph at lines 172--175
[]\tenrm Mathematical for-mu-las may be dis-played. A dis-played
\end{verbatim}\end{quote}
indicates that \LaTeX{} could not find a good place to break a line when laying
out a paragraph.  As a result, it was forced to let the line stick out into the
right-hand margin, in this case by 10.6 points.  Since a point is about 1/72nd
of an inch this may be rather hard to see, but it will be there none the less.

This particular problem happens because \LaTeX{} is rather fussy about line
breaking, and it would rather generate a line that is too long than generate a
paragraph that doesn't meet its high standards.  The simplest way around the
problem is to enclose the entire offending paragraph between
\verb|\begin{sloppypar}| and \verb|\end{sloppypar}| commands.  This tells
\LaTeX{} that you are happy for it to break its own rules while it is working on
that particular bit of text.

Alternatively, messages about ``Underfull \verb|\hbox'es''| may appear.
These are lines that had to have more space inserted between
words than \LaTeX{} would have liked.  In general there is not much that you
can do about these.  Your output will look fine, even if the line looks a bit
stretched.  About the only thing you could do is to re-write the offending
paragraph!

\section{A Final Reminder}

You now know enough \LaTeX{} to produce a wide range of documents.  But this
document has only scratched the surface of the
things that \LaTeX{} can do.  This entire document was itself produced with
\LaTeX{} (with no sticking things in or clever use of a photocopier) and even
it hasn't used all the features that it could.  From this you may get some
feeling for the power that \LaTeX{} puts at your disposal.

Please remember what was said in the introduction: if you {\bf do} have a
complex document to produce then {\bf go and read the manual}.  You will be
wasting your time if you rely only on what you have read here.
     % Edited ``Essential LaTeX'' by Jon Warbrick
\chapter{Figures and Tables}\label{quad}
This chapter\footnote{Most of the text in this chapter's introduction is from {\em How to
\TeX{} a Thesis: The Purdue Thesis Styles}} shows some example ways of incorporating tables and figures into \LaTeX{}.
Special environments exist for tables and figures and are special because they are
allowed to {\em float}---that is, \LaTeX{} doesn't always put them in the exact place
that they occur in your input file.  An algorithm is used to place the floating environments,
or floats, at locations which are typographically correct.  This may cause endless frustration
if you want to have a figure or table occur at a specific location.  There are a few
methods for solving this.

You can exert some influence on \LaTeX{}'s float placement algorithm by using
{\em float position specifiers}.  These specifiers, listed below, tell \LaTeX{}
what you prefer.
\begin{tabbing}
{\tt hhhhhh} \= ``bottom'' \=  \kill
{\tt h}\> ``here'' \> do not move this object \\
{\tt p}\> ``page'' \> put this object on a page of floats \\
{\tt b}\> ``bottom'' \> put this object at the bottom of a page\\
{\tt t}\> ``top'' \> put this object at the top of a page\\
\end{tabbing}

Any combination of these can be used:
\begin{quote}\tt\singlespace\begin{verbatim}
\begin{figure}[htbp]
 ...
\caption{A Figure!}
\end{figure}
\end{verbatim}\end{quote}

In this example, we asked \LaTeX{} to ``put the figure `here' if possible.  If it
is not possible (according to the rule encoded in the float algorithm), put it on the
next float page.  A float page is a page which contains nothing but floating objects,
{\em e.g.} a page of nothing but figures or tables.  If this isn't possible, try to put it
at the `top' of a page.  The last thing to try is to put the figure at the `bottom' of
a page.''

The remainder of this chapter deals with some examples of what to put into the figure,
the ellipsis (\ldots ) in the example above.

\section{Tables}
Table~\ref{pde.tab1} is an example table from the UW Math Department.
\begin{table}[htbp]
\centering
\caption{PDE solve times, $15^3+1$
equations.\label{pde.tab1}}
\begin{tabular}{||l|l|l|l|l|l||}\hline
Precond. & Time & Nonlinear & Krylov
& Function & Precond. \\
 & & Iterations & Iterations & calls & solves \\ \hline
None & 1260.9u & 3 & 26 & 30 & 0  \\
 &(21:09) & & & &  \\ \hline
FFT  & 983.4u & 2  & 5  & 8  & 7 \\
&(16:31) & & & & \\ \hline
\end{tabular}
\end{table}
The code to generate it is as follows:
\begin{quote}\tt\singlespace\begin{verbatim}
\begin{table}[htbp]
\centering
\caption{PDE solve times, $15^3+1$
equations.\label{pde.tab1}}
\begin{tabular}{||l|l|l|l|l|l||}\hline
Precond. & Time & Nonlinear & Krylov
& Function & Precond. \\
 & & Iterations & Iterations & calls & solves \\ \hline
None & 1260.9u & 3 & 26 & 30 & 0  \\
 &(21:09) & & & &  \\ \hline
FFT  & 983.4u & 2  & 5  & 8  & 7 \\
&(16:31) & & & & \\ \hline
\end{tabular}
\end{table}
\end{verbatim}\end{quote}

\section{Figures}
There are many different ways to incorporate figures into a \LaTeX{}
document.  \LaTeX{} has an internal {\tt picture} environment and
some programs will generate files which are in this format and can
be simply {\tt include}d.  In addition to \LaTeX{} native {\tt picture}
format, additional packages can be loaded in the {\tt\verb|\documentstyle|}
command (or using the {\tt input} command) to allow \LaTeX{} to process
non-native formats such as PostScript.

\subsection{\tt gnuplot}
The graph of Figure~\ref{gelfand.fig2}
 was created by gnuplot. For simple graphs this is a
 great utility.  For example, if you want a sin curve in your thesis
 try the following:
\begin{quote}\tt\singlespace\begin{verbatim}
 (terminal window): gnuplot
 (in gnuplot):
                 set terminal latex
                 set output "foo.tex"
                 plot sin(x)
                 quit
\end{verbatim}\end{quote}
This will generate a file called {\tt foo.tex} which can be read in
with the following statements.
\begin{figure}[htbp]
\centering
% GNUPLOT: LaTeX picture
\setlength{\unitlength}{0.240900pt}
\ifx\plotpoint\undefined\newsavebox{\plotpoint}\fi
\sbox{\plotpoint}{\rule[-0.175pt]{0.350pt}{0.350pt}}%
\begin{picture}(1500,900)(0,0)
%\tenrm
\sbox{\plotpoint}{\rule[-0.175pt]{0.350pt}{0.350pt}}%
\put(264,158){\rule[-0.175pt]{282.335pt}{0.350pt}}
\put(264,158){\rule[-0.175pt]{0.350pt}{151.526pt}}
\put(264,158){\rule[-0.175pt]{4.818pt}{0.350pt}}
%\put(242,158){\makebox(0,0)[r]{0}}
\put(1416,158){\rule[-0.175pt]{4.818pt}{0.350pt}}
\put(264,284){\rule[-0.175pt]{4.818pt}{0.350pt}}
%\put(242,284){\makebox(0,0)[r]{2}}
\put(1416,284){\rule[-0.175pt]{4.818pt}{0.350pt}}
\put(264,410){\rule[-0.175pt]{4.818pt}{0.350pt}}
%\put(242,410){\makebox(0,0)[r]{4}}
\put(1416,410){\rule[-0.175pt]{4.818pt}{0.350pt}}
\put(264,535){\rule[-0.175pt]{4.818pt}{0.350pt}}
%\put(242,535){\makebox(0,0)[r]{6}}
\put(1416,535){\rule[-0.175pt]{4.818pt}{0.350pt}}
\put(264,661){\rule[-0.175pt]{4.818pt}{0.350pt}}
%\put(242,661){\makebox(0,0)[r]{8}}
\put(1416,661){\rule[-0.175pt]{4.818pt}{0.350pt}}
\put(264,787){\rule[-0.175pt]{4.818pt}{0.350pt}}
%\put(242,787){\makebox(0,0)[r]{10}}
\put(1416,787){\rule[-0.175pt]{4.818pt}{0.350pt}}
\put(264,158){\rule[-0.175pt]{0.350pt}{4.818pt}}
%\put(264,113){\makebox(0,0){0}}
\put(264,767){\rule[-0.175pt]{0.350pt}{4.818pt}}
\put(411,158){\rule[-0.175pt]{0.350pt}{4.818pt}}
%\put(411,113){\makebox(0,0){0.5}}
\put(411,767){\rule[-0.175pt]{0.350pt}{4.818pt}}
\put(557,158){\rule[-0.175pt]{0.350pt}{4.818pt}}
%\put(557,113){\makebox(0,0){1}}
\put(557,767){\rule[-0.175pt]{0.350pt}{4.818pt}}
\put(704,158){\rule[-0.175pt]{0.350pt}{4.818pt}}
%\put(704,113){\makebox(0,0){1.5}}
\put(704,767){\rule[-0.175pt]{0.350pt}{4.818pt}}
\put(850,158){\rule[-0.175pt]{0.350pt}{4.818pt}}
%\put(850,113){\makebox(0,0){2}}
\put(850,767){\rule[-0.175pt]{0.350pt}{4.818pt}}
\put(997,158){\rule[-0.175pt]{0.350pt}{4.818pt}}
%\put(997,113){\makebox(0,0){2.5}}
\put(997,767){\rule[-0.175pt]{0.350pt}{4.818pt}}
\put(1143,158){\rule[-0.175pt]{0.350pt}{4.818pt}}
%\put(1143,113){\makebox(0,0){3}}
\put(1143,767){\rule[-0.175pt]{0.350pt}{4.818pt}}
\put(1290,158){\rule[-0.175pt]{0.350pt}{4.818pt}}
%\put(1290,113){\makebox(0,0){3.5}}
\put(1290,767){\rule[-0.175pt]{0.350pt}{4.818pt}}
\put(1436,158){\rule[-0.175pt]{0.350pt}{4.818pt}}
%\put(1436,113){\makebox(0,0){4}}
\put(1436,767){\rule[-0.175pt]{0.350pt}{4.818pt}}
\put(264,158){\rule[-0.175pt]{282.335pt}{0.350pt}}
\put(1436,158){\rule[-0.175pt]{0.350pt}{151.526pt}}
\put(264,787){\rule[-0.175pt]{282.335pt}{0.350pt}}
\put(100,472){\makebox(0,0)[l]{\shortstack{$\| u\|$}}}
\put(850,68){\makebox(0,0){$\lambda$}}
%\put(850,832){\makebox(0,0){plot}}
\put(264,158){\rule[-0.175pt]{0.350pt}{151.526pt}}
%\put(1306,722){\makebox(0,0)[r]{}}
%\put(1328,722){\rule[-0.175pt]{15.899pt}{0.350pt}}
\put(264,158){\usebox{\plotpoint}}
\put(264,158){\rule[-0.175pt]{6.304pt}{0.350pt}}
\put(290,159){\rule[-0.175pt]{6.304pt}{0.350pt}}
\put(316,160){\rule[-0.175pt]{6.304pt}{0.350pt}}
\put(342,161){\rule[-0.175pt]{6.304pt}{0.350pt}}
\put(368,162){\rule[-0.175pt]{6.304pt}{0.350pt}}
\put(394,163){\rule[-0.175pt]{6.304pt}{0.350pt}}
\put(420,164){\rule[-0.175pt]{5.644pt}{0.350pt}}
\put(444,165){\rule[-0.175pt]{5.644pt}{0.350pt}}
\put(467,166){\rule[-0.175pt]{5.644pt}{0.350pt}}
\put(491,167){\rule[-0.175pt]{5.644pt}{0.350pt}}
\put(514,168){\rule[-0.175pt]{5.644pt}{0.350pt}}
\put(538,169){\rule[-0.175pt]{5.644pt}{0.350pt}}
\put(561,170){\rule[-0.175pt]{5.644pt}{0.350pt}}
\put(585,171){\rule[-0.175pt]{6.384pt}{0.350pt}}
\put(611,172){\rule[-0.175pt]{6.384pt}{0.350pt}}
\put(638,173){\rule[-0.175pt]{6.384pt}{0.350pt}}
\put(664,174){\rule[-0.175pt]{6.384pt}{0.350pt}}
\put(691,175){\rule[-0.175pt]{6.384pt}{0.350pt}}
\put(717,176){\rule[-0.175pt]{6.384pt}{0.350pt}}
\put(744,177){\rule[-0.175pt]{5.862pt}{0.350pt}}
\put(768,178){\rule[-0.175pt]{5.862pt}{0.350pt}}
\put(792,179){\rule[-0.175pt]{5.862pt}{0.350pt}}
\put(816,180){\rule[-0.175pt]{5.862pt}{0.350pt}}
\put(841,181){\rule[-0.175pt]{5.862pt}{0.350pt}}
\put(865,182){\rule[-0.175pt]{5.862pt}{0.350pt}}
\put(889,183){\rule[-0.175pt]{4.371pt}{0.350pt}}
\put(908,184){\rule[-0.175pt]{4.371pt}{0.350pt}}
\put(926,185){\rule[-0.175pt]{4.371pt}{0.350pt}}
\put(944,186){\rule[-0.175pt]{4.371pt}{0.350pt}}
\put(962,187){\rule[-0.175pt]{4.371pt}{0.350pt}}
\put(980,188){\rule[-0.175pt]{4.371pt}{0.350pt}}
\put(998,189){\rule[-0.175pt]{4.371pt}{0.350pt}}
\put(1017,190){\rule[-0.175pt]{4.216pt}{0.350pt}}
\put(1034,191){\rule[-0.175pt]{4.216pt}{0.350pt}}
\put(1052,192){\rule[-0.175pt]{4.216pt}{0.350pt}}
\put(1069,193){\rule[-0.175pt]{4.216pt}{0.350pt}}
\put(1087,194){\rule[-0.175pt]{4.216pt}{0.350pt}}
\put(1104,195){\rule[-0.175pt]{4.216pt}{0.350pt}}
\put(1122,196){\rule[-0.175pt]{3.172pt}{0.350pt}}
\put(1135,197){\rule[-0.175pt]{3.172pt}{0.350pt}}
\put(1148,198){\rule[-0.175pt]{3.172pt}{0.350pt}}
\put(1161,199){\rule[-0.175pt]{3.172pt}{0.350pt}}
\put(1174,200){\rule[-0.175pt]{3.172pt}{0.350pt}}
\put(1187,201){\rule[-0.175pt]{3.172pt}{0.350pt}}
\put(1200,202){\rule[-0.175pt]{1.893pt}{0.350pt}}
\put(1208,203){\rule[-0.175pt]{1.893pt}{0.350pt}}
\put(1216,204){\rule[-0.175pt]{1.893pt}{0.350pt}}
\put(1224,205){\rule[-0.175pt]{1.893pt}{0.350pt}}
\put(1232,206){\rule[-0.175pt]{1.893pt}{0.350pt}}
\put(1240,207){\rule[-0.175pt]{1.893pt}{0.350pt}}
\put(1248,208){\rule[-0.175pt]{1.893pt}{0.350pt}}
\put(1256,209){\rule[-0.175pt]{1.245pt}{0.350pt}}
\put(1261,210){\rule[-0.175pt]{1.245pt}{0.350pt}}
\put(1266,211){\rule[-0.175pt]{1.245pt}{0.350pt}}
\put(1271,212){\rule[-0.175pt]{1.245pt}{0.350pt}}
\put(1276,213){\rule[-0.175pt]{1.245pt}{0.350pt}}
\put(1281,214){\rule[-0.175pt]{1.245pt}{0.350pt}}
\put(1286,215){\usebox{\plotpoint}}
\put(1288,216){\usebox{\plotpoint}}
\put(1289,217){\usebox{\plotpoint}}
\put(1291,218){\usebox{\plotpoint}}
\put(1292,219){\usebox{\plotpoint}}
\put(1294,220){\usebox{\plotpoint}}
\put(1295,221){\usebox{\plotpoint}}
\put(1295,222){\rule[-0.175pt]{0.361pt}{0.350pt}}
\put(1294,223){\rule[-0.175pt]{0.361pt}{0.350pt}}
\put(1292,224){\rule[-0.175pt]{0.361pt}{0.350pt}}
\put(1291,225){\rule[-0.175pt]{0.361pt}{0.350pt}}
\put(1289,226){\rule[-0.175pt]{0.361pt}{0.350pt}}
\put(1288,227){\rule[-0.175pt]{0.361pt}{0.350pt}}
\put(1284,228){\rule[-0.175pt]{0.964pt}{0.350pt}}
\put(1280,229){\rule[-0.175pt]{0.964pt}{0.350pt}}
\put(1276,230){\rule[-0.175pt]{0.964pt}{0.350pt}}
\put(1272,231){\rule[-0.175pt]{0.964pt}{0.350pt}}
\put(1268,232){\rule[-0.175pt]{0.964pt}{0.350pt}}
\put(1264,233){\rule[-0.175pt]{0.964pt}{0.350pt}}
\put(1258,234){\rule[-0.175pt]{1.273pt}{0.350pt}}
\put(1253,235){\rule[-0.175pt]{1.273pt}{0.350pt}}
\put(1248,236){\rule[-0.175pt]{1.273pt}{0.350pt}}
\put(1242,237){\rule[-0.175pt]{1.273pt}{0.350pt}}
\put(1237,238){\rule[-0.175pt]{1.273pt}{0.350pt}}
\put(1232,239){\rule[-0.175pt]{1.273pt}{0.350pt}}
\put(1227,240){\rule[-0.175pt]{1.273pt}{0.350pt}}
\put(1219,241){\rule[-0.175pt]{1.847pt}{0.350pt}}
\put(1211,242){\rule[-0.175pt]{1.847pt}{0.350pt}}
\put(1204,243){\rule[-0.175pt]{1.847pt}{0.350pt}}
\put(1196,244){\rule[-0.175pt]{1.847pt}{0.350pt}}
\put(1188,245){\rule[-0.175pt]{1.847pt}{0.350pt}}
\put(1181,246){\rule[-0.175pt]{1.847pt}{0.350pt}}
\put(1172,247){\rule[-0.175pt]{2.128pt}{0.350pt}}
\put(1163,248){\rule[-0.175pt]{2.128pt}{0.350pt}}
\put(1154,249){\rule[-0.175pt]{2.128pt}{0.350pt}}
\put(1145,250){\rule[-0.175pt]{2.128pt}{0.350pt}}
\put(1136,251){\rule[-0.175pt]{2.128pt}{0.350pt}}
\put(1128,252){\rule[-0.175pt]{2.128pt}{0.350pt}}
\put(1120,253){\rule[-0.175pt]{1.893pt}{0.350pt}}
\put(1112,254){\rule[-0.175pt]{1.893pt}{0.350pt}}
\put(1104,255){\rule[-0.175pt]{1.893pt}{0.350pt}}
\put(1096,256){\rule[-0.175pt]{1.893pt}{0.350pt}}
\put(1088,257){\rule[-0.175pt]{1.893pt}{0.350pt}}
\put(1080,258){\rule[-0.175pt]{1.893pt}{0.350pt}}
\put(1073,259){\rule[-0.175pt]{1.893pt}{0.350pt}}
\put(1063,260){\rule[-0.175pt]{2.208pt}{0.350pt}}
\put(1054,261){\rule[-0.175pt]{2.208pt}{0.350pt}}
\put(1045,262){\rule[-0.175pt]{2.208pt}{0.350pt}}
\put(1036,263){\rule[-0.175pt]{2.208pt}{0.350pt}}
\put(1027,264){\rule[-0.175pt]{2.208pt}{0.350pt}}
\put(1018,265){\rule[-0.175pt]{2.208pt}{0.350pt}}
\put(1009,266){\rule[-0.175pt]{2.168pt}{0.350pt}}
\put(1000,267){\rule[-0.175pt]{2.168pt}{0.350pt}}
\put(991,268){\rule[-0.175pt]{2.168pt}{0.350pt}}
\put(982,269){\rule[-0.175pt]{2.168pt}{0.350pt}}
\put(973,270){\rule[-0.175pt]{2.168pt}{0.350pt}}
\put(964,271){\rule[-0.175pt]{2.168pt}{0.350pt}}
\put(957,272){\rule[-0.175pt]{1.686pt}{0.350pt}}
\put(950,273){\rule[-0.175pt]{1.686pt}{0.350pt}}
\put(943,274){\rule[-0.175pt]{1.686pt}{0.350pt}}
\put(936,275){\rule[-0.175pt]{1.686pt}{0.350pt}}
\put(929,276){\rule[-0.175pt]{1.686pt}{0.350pt}}
\put(922,277){\rule[-0.175pt]{1.686pt}{0.350pt}}
\put(915,278){\rule[-0.175pt]{1.686pt}{0.350pt}}
\put(907,279){\rule[-0.175pt]{1.767pt}{0.350pt}}
\put(900,280){\rule[-0.175pt]{1.767pt}{0.350pt}}
\put(893,281){\rule[-0.175pt]{1.767pt}{0.350pt}}
\put(885,282){\rule[-0.175pt]{1.767pt}{0.350pt}}
\put(878,283){\rule[-0.175pt]{1.767pt}{0.350pt}}
\put(871,284){\rule[-0.175pt]{1.767pt}{0.350pt}}
\put(864,285){\rule[-0.175pt]{1.486pt}{0.350pt}}
\put(858,286){\rule[-0.175pt]{1.486pt}{0.350pt}}
\put(852,287){\rule[-0.175pt]{1.486pt}{0.350pt}}
\put(846,288){\rule[-0.175pt]{1.486pt}{0.350pt}}
\put(840,289){\rule[-0.175pt]{1.486pt}{0.350pt}}
\put(834,290){\rule[-0.175pt]{1.486pt}{0.350pt}}
\put(829,291){\rule[-0.175pt]{0.998pt}{0.350pt}}
\put(825,292){\rule[-0.175pt]{0.998pt}{0.350pt}}
\put(821,293){\rule[-0.175pt]{0.998pt}{0.350pt}}
\put(817,294){\rule[-0.175pt]{0.998pt}{0.350pt}}
\put(813,295){\rule[-0.175pt]{0.998pt}{0.350pt}}
\put(809,296){\rule[-0.175pt]{0.998pt}{0.350pt}}
\put(805,297){\rule[-0.175pt]{0.998pt}{0.350pt}}
\put(801,298){\rule[-0.175pt]{0.883pt}{0.350pt}}
\put(797,299){\rule[-0.175pt]{0.883pt}{0.350pt}}
\put(793,300){\rule[-0.175pt]{0.883pt}{0.350pt}}
\put(790,301){\rule[-0.175pt]{0.883pt}{0.350pt}}
\put(786,302){\rule[-0.175pt]{0.883pt}{0.350pt}}
\put(783,303){\rule[-0.175pt]{0.883pt}{0.350pt}}
\put(780,304){\rule[-0.175pt]{0.522pt}{0.350pt}}
\put(778,305){\rule[-0.175pt]{0.522pt}{0.350pt}}
\put(776,306){\rule[-0.175pt]{0.522pt}{0.350pt}}
\put(774,307){\rule[-0.175pt]{0.522pt}{0.350pt}}
\put(772,308){\rule[-0.175pt]{0.522pt}{0.350pt}}
\put(770,309){\rule[-0.175pt]{0.522pt}{0.350pt}}
\put(770,310){\usebox{\plotpoint}}
\put(769,311){\usebox{\plotpoint}}
\put(768,312){\usebox{\plotpoint}}
\put(767,314){\usebox{\plotpoint}}
\put(766,315){\usebox{\plotpoint}}
\put(765,316){\rule[-0.175pt]{0.350pt}{0.723pt}}
\put(766,320){\rule[-0.175pt]{0.350pt}{0.723pt}}
\put(767,323){\usebox{\plotpoint}}
\put(768,324){\usebox{\plotpoint}}
\put(769,325){\usebox{\plotpoint}}
\put(771,326){\usebox{\plotpoint}}
\put(772,327){\usebox{\plotpoint}}
\put(774,328){\usebox{\plotpoint}}
\put(775,329){\usebox{\plotpoint}}
\put(777,330){\rule[-0.175pt]{0.602pt}{0.350pt}}
\put(779,331){\rule[-0.175pt]{0.602pt}{0.350pt}}
\put(782,332){\rule[-0.175pt]{0.602pt}{0.350pt}}
\put(784,333){\rule[-0.175pt]{0.602pt}{0.350pt}}
\put(787,334){\rule[-0.175pt]{0.602pt}{0.350pt}}
\put(789,335){\rule[-0.175pt]{0.602pt}{0.350pt}}
\put(792,336){\rule[-0.175pt]{0.843pt}{0.350pt}}
\put(795,337){\rule[-0.175pt]{0.843pt}{0.350pt}}
\put(799,338){\rule[-0.175pt]{0.843pt}{0.350pt}}
\put(802,339){\rule[-0.175pt]{0.843pt}{0.350pt}}
\put(806,340){\rule[-0.175pt]{0.843pt}{0.350pt}}
\put(809,341){\rule[-0.175pt]{0.843pt}{0.350pt}}
\put(813,342){\rule[-0.175pt]{0.826pt}{0.350pt}}
\put(816,343){\rule[-0.175pt]{0.826pt}{0.350pt}}
\put(819,344){\rule[-0.175pt]{0.826pt}{0.350pt}}
\put(823,345){\rule[-0.175pt]{0.826pt}{0.350pt}}
\put(826,346){\rule[-0.175pt]{0.826pt}{0.350pt}}
\put(830,347){\rule[-0.175pt]{0.826pt}{0.350pt}}
\put(833,348){\rule[-0.175pt]{0.826pt}{0.350pt}}
\put(837,349){\rule[-0.175pt]{1.084pt}{0.350pt}}
\put(841,350){\rule[-0.175pt]{1.084pt}{0.350pt}}
\put(846,351){\rule[-0.175pt]{1.084pt}{0.350pt}}
\put(850,352){\rule[-0.175pt]{1.084pt}{0.350pt}}
\put(855,353){\rule[-0.175pt]{1.084pt}{0.350pt}}
\put(859,354){\rule[-0.175pt]{1.084pt}{0.350pt}}
\put(864,355){\rule[-0.175pt]{1.164pt}{0.350pt}}
\put(868,356){\rule[-0.175pt]{1.164pt}{0.350pt}}
\put(873,357){\rule[-0.175pt]{1.164pt}{0.350pt}}
\put(878,358){\rule[-0.175pt]{1.164pt}{0.350pt}}
\put(883,359){\rule[-0.175pt]{1.164pt}{0.350pt}}
\put(888,360){\rule[-0.175pt]{1.164pt}{0.350pt}}
\put(892,361){\rule[-0.175pt]{1.032pt}{0.350pt}}
\put(897,362){\rule[-0.175pt]{1.032pt}{0.350pt}}
\put(901,363){\rule[-0.175pt]{1.032pt}{0.350pt}}
\put(905,364){\rule[-0.175pt]{1.032pt}{0.350pt}}
\put(910,365){\rule[-0.175pt]{1.032pt}{0.350pt}}
\put(914,366){\rule[-0.175pt]{1.032pt}{0.350pt}}
\put(918,367){\rule[-0.175pt]{1.032pt}{0.350pt}}
\put(922,368){\rule[-0.175pt]{1.205pt}{0.350pt}}
\put(928,369){\rule[-0.175pt]{1.204pt}{0.350pt}}
\put(933,370){\rule[-0.175pt]{1.204pt}{0.350pt}}
\put(938,371){\rule[-0.175pt]{1.204pt}{0.350pt}}
\put(943,372){\rule[-0.175pt]{1.204pt}{0.350pt}}
\put(948,373){\rule[-0.175pt]{1.204pt}{0.350pt}}
\put(953,374){\rule[-0.175pt]{1.124pt}{0.350pt}}
\put(957,375){\rule[-0.175pt]{1.124pt}{0.350pt}}
\put(962,376){\rule[-0.175pt]{1.124pt}{0.350pt}}
\put(967,377){\rule[-0.175pt]{1.124pt}{0.350pt}}
\put(971,378){\rule[-0.175pt]{1.124pt}{0.350pt}}
\put(976,379){\rule[-0.175pt]{1.124pt}{0.350pt}}
\put(981,380){\rule[-0.175pt]{0.929pt}{0.350pt}}
\put(984,381){\rule[-0.175pt]{0.929pt}{0.350pt}}
\put(988,382){\rule[-0.175pt]{0.929pt}{0.350pt}}
\put(992,383){\rule[-0.175pt]{0.929pt}{0.350pt}}
\put(996,384){\rule[-0.175pt]{0.929pt}{0.350pt}}
\put(1000,385){\rule[-0.175pt]{0.929pt}{0.350pt}}
\put(1004,386){\rule[-0.175pt]{0.929pt}{0.350pt}}
\put(1007,387){\rule[-0.175pt]{0.923pt}{0.350pt}}
\put(1011,388){\rule[-0.175pt]{0.923pt}{0.350pt}}
\put(1015,389){\rule[-0.175pt]{0.923pt}{0.350pt}}
\put(1019,390){\rule[-0.175pt]{0.923pt}{0.350pt}}
\put(1023,391){\rule[-0.175pt]{0.923pt}{0.350pt}}
\put(1027,392){\rule[-0.175pt]{0.923pt}{0.350pt}}
\put(1031,393){\rule[-0.175pt]{0.843pt}{0.350pt}}
\put(1034,394){\rule[-0.175pt]{0.843pt}{0.350pt}}
\put(1038,395){\rule[-0.175pt]{0.843pt}{0.350pt}}
\put(1041,396){\rule[-0.175pt]{0.843pt}{0.350pt}}
\put(1045,397){\rule[-0.175pt]{0.843pt}{0.350pt}}
\put(1048,398){\rule[-0.175pt]{0.843pt}{0.350pt}}
\put(1052,399){\rule[-0.175pt]{0.585pt}{0.350pt}}
\put(1054,400){\rule[-0.175pt]{0.585pt}{0.350pt}}
\put(1056,401){\rule[-0.175pt]{0.585pt}{0.350pt}}
\put(1059,402){\rule[-0.175pt]{0.585pt}{0.350pt}}
\put(1061,403){\rule[-0.175pt]{0.585pt}{0.350pt}}
\put(1064,404){\rule[-0.175pt]{0.585pt}{0.350pt}}
\put(1066,405){\rule[-0.175pt]{0.585pt}{0.350pt}}
\put(1069,406){\rule[-0.175pt]{0.522pt}{0.350pt}}
\put(1071,407){\rule[-0.175pt]{0.522pt}{0.350pt}}
\put(1073,408){\rule[-0.175pt]{0.522pt}{0.350pt}}
\put(1075,409){\rule[-0.175pt]{0.522pt}{0.350pt}}
\put(1077,410){\rule[-0.175pt]{0.522pt}{0.350pt}}
\put(1079,411){\rule[-0.175pt]{0.522pt}{0.350pt}}
\put(1081,412){\rule[-0.175pt]{0.402pt}{0.350pt}}
\put(1083,413){\rule[-0.175pt]{0.401pt}{0.350pt}}
\put(1085,414){\rule[-0.175pt]{0.401pt}{0.350pt}}
\put(1086,415){\rule[-0.175pt]{0.401pt}{0.350pt}}
\put(1088,416){\rule[-0.175pt]{0.401pt}{0.350pt}}
\put(1090,417){\rule[-0.175pt]{0.401pt}{0.350pt}}
\put(1091,418){\usebox{\plotpoint}}
\put(1092,418){\usebox{\plotpoint}}
\put(1093,419){\usebox{\plotpoint}}
\put(1094,420){\usebox{\plotpoint}}
\put(1095,422){\usebox{\plotpoint}}
\put(1096,423){\usebox{\plotpoint}}
\put(1097,424){\rule[-0.175pt]{0.350pt}{0.723pt}}
\put(1098,428){\rule[-0.175pt]{0.350pt}{0.723pt}}
\put(1099,431){\rule[-0.175pt]{0.350pt}{1.686pt}}
\put(1098,438){\usebox{\plotpoint}}
\put(1097,439){\usebox{\plotpoint}}
\put(1096,440){\usebox{\plotpoint}}
\put(1095,441){\usebox{\plotpoint}}
\put(1094,442){\usebox{\plotpoint}}
\put(1091,444){\usebox{\plotpoint}}
\put(1090,445){\usebox{\plotpoint}}
\put(1089,446){\usebox{\plotpoint}}
\put(1088,447){\usebox{\plotpoint}}
\put(1087,448){\usebox{\plotpoint}}
\put(1086,449){\usebox{\plotpoint}}
\put(1084,450){\usebox{\plotpoint}}
\put(1083,451){\usebox{\plotpoint}}
\put(1081,452){\usebox{\plotpoint}}
\put(1080,453){\usebox{\plotpoint}}
\put(1078,454){\usebox{\plotpoint}}
\put(1077,455){\usebox{\plotpoint}}
\put(1076,456){\usebox{\plotpoint}}
\put(1074,457){\rule[-0.175pt]{0.442pt}{0.350pt}}
\put(1072,458){\rule[-0.175pt]{0.442pt}{0.350pt}}
\put(1070,459){\rule[-0.175pt]{0.442pt}{0.350pt}}
\put(1068,460){\rule[-0.175pt]{0.442pt}{0.350pt}}
\put(1066,461){\rule[-0.175pt]{0.442pt}{0.350pt}}
\put(1065,462){\rule[-0.175pt]{0.442pt}{0.350pt}}
\put(1063,463){\rule[-0.175pt]{0.482pt}{0.350pt}}
\put(1061,464){\rule[-0.175pt]{0.482pt}{0.350pt}}
\put(1059,465){\rule[-0.175pt]{0.482pt}{0.350pt}}
\put(1057,466){\rule[-0.175pt]{0.482pt}{0.350pt}}
\put(1055,467){\rule[-0.175pt]{0.482pt}{0.350pt}}
\put(1053,468){\rule[-0.175pt]{0.482pt}{0.350pt}}
\put(1051,469){\rule[-0.175pt]{0.482pt}{0.350pt}}
\put(1049,470){\rule[-0.175pt]{0.482pt}{0.350pt}}
\put(1047,471){\rule[-0.175pt]{0.482pt}{0.350pt}}
\put(1045,472){\rule[-0.175pt]{0.482pt}{0.350pt}}
\put(1043,473){\rule[-0.175pt]{0.482pt}{0.350pt}}
\put(1041,474){\rule[-0.175pt]{0.482pt}{0.350pt}}
\put(1039,475){\rule[-0.175pt]{0.482pt}{0.350pt}}
\put(1036,476){\rule[-0.175pt]{0.522pt}{0.350pt}}
\put(1034,477){\rule[-0.175pt]{0.522pt}{0.350pt}}
\put(1032,478){\rule[-0.175pt]{0.522pt}{0.350pt}}
\put(1030,479){\rule[-0.175pt]{0.522pt}{0.350pt}}
\put(1028,480){\rule[-0.175pt]{0.522pt}{0.350pt}}
\put(1026,481){\rule[-0.175pt]{0.522pt}{0.350pt}}
\put(1023,482){\rule[-0.175pt]{0.522pt}{0.350pt}}
\put(1021,483){\rule[-0.175pt]{0.522pt}{0.350pt}}
\put(1019,484){\rule[-0.175pt]{0.522pt}{0.350pt}}
\put(1017,485){\rule[-0.175pt]{0.522pt}{0.350pt}}
\put(1015,486){\rule[-0.175pt]{0.522pt}{0.350pt}}
\put(1013,487){\rule[-0.175pt]{0.522pt}{0.350pt}}
\put(1011,488){\rule[-0.175pt]{0.447pt}{0.350pt}}
\put(1009,489){\rule[-0.175pt]{0.447pt}{0.350pt}}
\put(1007,490){\rule[-0.175pt]{0.447pt}{0.350pt}}
\put(1005,491){\rule[-0.175pt]{0.447pt}{0.350pt}}
\put(1003,492){\rule[-0.175pt]{0.447pt}{0.350pt}}
\put(1001,493){\rule[-0.175pt]{0.447pt}{0.350pt}}
\put(1000,494){\rule[-0.175pt]{0.447pt}{0.350pt}}
\put(998,495){\rule[-0.175pt]{0.442pt}{0.350pt}}
\put(996,496){\rule[-0.175pt]{0.442pt}{0.350pt}}
\put(994,497){\rule[-0.175pt]{0.442pt}{0.350pt}}
\put(992,498){\rule[-0.175pt]{0.442pt}{0.350pt}}
\put(990,499){\rule[-0.175pt]{0.442pt}{0.350pt}}
\put(989,500){\rule[-0.175pt]{0.442pt}{0.350pt}}
\put(987,501){\rule[-0.175pt]{0.442pt}{0.350pt}}
\put(985,502){\rule[-0.175pt]{0.442pt}{0.350pt}}
\put(983,503){\rule[-0.175pt]{0.442pt}{0.350pt}}
\put(981,504){\rule[-0.175pt]{0.442pt}{0.350pt}}
\put(979,505){\rule[-0.175pt]{0.442pt}{0.350pt}}
\put(978,506){\rule[-0.175pt]{0.442pt}{0.350pt}}
\put(976,507){\usebox{\plotpoint}}
\put(975,508){\usebox{\plotpoint}}
\put(974,509){\usebox{\plotpoint}}
\put(972,510){\usebox{\plotpoint}}
\put(971,511){\usebox{\plotpoint}}
\put(970,512){\usebox{\plotpoint}}
\put(969,513){\usebox{\plotpoint}}
\put(967,514){\usebox{\plotpoint}}
\put(966,515){\usebox{\plotpoint}}
\put(965,516){\usebox{\plotpoint}}
\put(964,517){\usebox{\plotpoint}}
\put(963,518){\usebox{\plotpoint}}
\put(962,519){\usebox{\plotpoint}}
\put(962,520){\usebox{\plotpoint}}
\put(961,521){\usebox{\plotpoint}}
\put(960,522){\usebox{\plotpoint}}
\put(959,524){\usebox{\plotpoint}}
\put(958,525){\usebox{\plotpoint}}
\put(957,527){\rule[-0.175pt]{0.350pt}{0.361pt}}
\put(956,528){\rule[-0.175pt]{0.350pt}{0.361pt}}
\put(955,530){\rule[-0.175pt]{0.350pt}{0.361pt}}
\put(954,531){\rule[-0.175pt]{0.350pt}{0.361pt}}
\put(953,533){\rule[-0.175pt]{0.350pt}{0.723pt}}
\put(952,536){\rule[-0.175pt]{0.350pt}{0.723pt}}
\put(951,539){\rule[-0.175pt]{0.350pt}{1.686pt}}
\put(950,546){\rule[-0.175pt]{0.350pt}{1.445pt}}
\put(951,552){\rule[-0.175pt]{0.350pt}{0.482pt}}
\put(952,554){\rule[-0.175pt]{0.350pt}{0.482pt}}
\put(953,556){\rule[-0.175pt]{0.350pt}{0.482pt}}
\put(954,558){\rule[-0.175pt]{0.350pt}{0.562pt}}
\put(955,560){\rule[-0.175pt]{0.350pt}{0.562pt}}
\put(956,562){\rule[-0.175pt]{0.350pt}{0.562pt}}
\put(957,564){\usebox{\plotpoint}}
\put(958,566){\usebox{\plotpoint}}
\put(959,567){\usebox{\plotpoint}}
\put(960,568){\usebox{\plotpoint}}
\put(961,569){\usebox{\plotpoint}}
\put(962,571){\usebox{\plotpoint}}
\put(963,572){\usebox{\plotpoint}}
\put(964,573){\usebox{\plotpoint}}
\put(965,574){\usebox{\plotpoint}}
\put(966,575){\usebox{\plotpoint}}
\put(967,577){\usebox{\plotpoint}}
\put(968,578){\usebox{\plotpoint}}
\put(969,579){\usebox{\plotpoint}}
\put(970,580){\usebox{\plotpoint}}
\put(971,581){\usebox{\plotpoint}}
\put(972,582){\usebox{\plotpoint}}
\put(973,584){\usebox{\plotpoint}}
\put(974,585){\usebox{\plotpoint}}
\put(975,586){\usebox{\plotpoint}}
\put(976,587){\usebox{\plotpoint}}
\put(977,588){\usebox{\plotpoint}}
\put(978,589){\usebox{\plotpoint}}
\put(979,590){\usebox{\plotpoint}}
\put(980,591){\usebox{\plotpoint}}
\put(981,592){\usebox{\plotpoint}}
\put(982,593){\usebox{\plotpoint}}
\put(983,594){\usebox{\plotpoint}}
\put(984,595){\usebox{\plotpoint}}
\put(985,596){\usebox{\plotpoint}}
\put(986,597){\usebox{\plotpoint}}
\put(987,598){\usebox{\plotpoint}}
\put(988,600){\usebox{\plotpoint}}
\put(989,601){\usebox{\plotpoint}}
\put(990,603){\usebox{\plotpoint}}
\put(991,604){\usebox{\plotpoint}}
\put(992,605){\usebox{\plotpoint}}
\put(993,606){\usebox{\plotpoint}}
\put(994,607){\usebox{\plotpoint}}
\put(995,608){\usebox{\plotpoint}}
\put(996,609){\usebox{\plotpoint}}
\put(997,610){\usebox{\plotpoint}}
\put(998,611){\usebox{\plotpoint}}
\put(999,612){\usebox{\plotpoint}}
\put(1000,613){\usebox{\plotpoint}}
\put(1001,615){\usebox{\plotpoint}}
\put(1002,616){\usebox{\plotpoint}}
\put(1003,617){\usebox{\plotpoint}}
\put(1004,619){\usebox{\plotpoint}}
\put(1005,620){\usebox{\plotpoint}}
\put(1006,622){\rule[-0.175pt]{0.350pt}{0.361pt}}
\put(1007,623){\rule[-0.175pt]{0.350pt}{0.361pt}}
\put(1008,625){\rule[-0.175pt]{0.350pt}{0.361pt}}
\put(1009,626){\rule[-0.175pt]{0.350pt}{0.361pt}}
\put(1010,628){\rule[-0.175pt]{0.350pt}{0.562pt}}
\put(1011,630){\rule[-0.175pt]{0.350pt}{0.562pt}}
\put(1012,632){\rule[-0.175pt]{0.350pt}{0.562pt}}
\put(1013,634){\rule[-0.175pt]{0.350pt}{0.723pt}}
\put(1014,638){\rule[-0.175pt]{0.350pt}{0.723pt}}
\put(1015,641){\rule[-0.175pt]{0.350pt}{1.445pt}}
\put(1016,647){\rule[-0.175pt]{0.350pt}{1.686pt}}
\put(1017,654){\rule[-0.175pt]{0.350pt}{3.734pt}}
\put(1016,669){\rule[-0.175pt]{0.350pt}{0.843pt}}
\put(1015,673){\rule[-0.175pt]{0.350pt}{1.445pt}}
\put(1014,679){\rule[-0.175pt]{0.350pt}{0.723pt}}
\put(1013,682){\rule[-0.175pt]{0.350pt}{0.723pt}}
\put(1012,685){\rule[-0.175pt]{0.350pt}{0.562pt}}
\put(1011,687){\rule[-0.175pt]{0.350pt}{0.562pt}}
\put(1010,689){\rule[-0.175pt]{0.350pt}{0.562pt}}
\put(1009,691){\rule[-0.175pt]{0.350pt}{0.723pt}}
\put(1008,695){\rule[-0.175pt]{0.350pt}{0.723pt}}
\put(1007,698){\rule[-0.175pt]{0.350pt}{0.482pt}}
\put(1006,700){\rule[-0.175pt]{0.350pt}{0.482pt}}
\put(1005,702){\rule[-0.175pt]{0.350pt}{0.482pt}}
\put(1004,704){\rule[-0.175pt]{0.350pt}{0.562pt}}
\put(1003,706){\rule[-0.175pt]{0.350pt}{0.562pt}}
\put(1002,708){\rule[-0.175pt]{0.350pt}{0.562pt}}
\put(1001,710){\rule[-0.175pt]{0.350pt}{0.723pt}}
\put(1000,714){\rule[-0.175pt]{0.350pt}{0.723pt}}
\put(999,717){\rule[-0.175pt]{0.350pt}{0.482pt}}
\put(998,719){\rule[-0.175pt]{0.350pt}{0.482pt}}
\put(997,721){\rule[-0.175pt]{0.350pt}{0.482pt}}
\put(996,723){\rule[-0.175pt]{0.350pt}{0.843pt}}
\put(995,726){\rule[-0.175pt]{0.350pt}{0.843pt}}
\put(994,730){\rule[-0.175pt]{0.350pt}{0.723pt}}
\put(993,733){\rule[-0.175pt]{0.350pt}{0.723pt}}
\put(992,736){\rule[-0.175pt]{0.350pt}{0.843pt}}
\put(991,739){\rule[-0.175pt]{0.350pt}{0.843pt}}
\put(990,743){\rule[-0.175pt]{0.350pt}{1.445pt}}
\put(989,749){\rule[-0.175pt]{0.350pt}{1.445pt}}
\put(988,755){\rule[-0.175pt]{0.350pt}{1.686pt}}
\put(987,762){\rule[-0.175pt]{0.350pt}{4.577pt}}
\put(988,781){\rule[-0.175pt]{0.350pt}{1.445pt}}
\end{picture}

\caption{Gelfand equation on the ball, $3\leq n \leq 9$.
\label{gelfand.fig2}}
\end{figure}
\begin{quote}\tt\singlespace\begin{verbatim}
\begin{figure}[htbp]
\centering
% GNUPLOT: LaTeX picture
\setlength{\unitlength}{0.240900pt}
\ifx\plotpoint\undefined\newsavebox{\plotpoint}\fi
\sbox{\plotpoint}{\rule[-0.175pt]{0.350pt}{0.350pt}}%
\begin{picture}(1500,900)(0,0)
%\tenrm
\sbox{\plotpoint}{\rule[-0.175pt]{0.350pt}{0.350pt}}%
\put(264,158){\rule[-0.175pt]{282.335pt}{0.350pt}}
\put(264,158){\rule[-0.175pt]{0.350pt}{151.526pt}}
\put(264,158){\rule[-0.175pt]{4.818pt}{0.350pt}}
%\put(242,158){\makebox(0,0)[r]{0}}
\put(1416,158){\rule[-0.175pt]{4.818pt}{0.350pt}}
\put(264,284){\rule[-0.175pt]{4.818pt}{0.350pt}}
%\put(242,284){\makebox(0,0)[r]{2}}
\put(1416,284){\rule[-0.175pt]{4.818pt}{0.350pt}}
\put(264,410){\rule[-0.175pt]{4.818pt}{0.350pt}}
%\put(242,410){\makebox(0,0)[r]{4}}
\put(1416,410){\rule[-0.175pt]{4.818pt}{0.350pt}}
\put(264,535){\rule[-0.175pt]{4.818pt}{0.350pt}}
%\put(242,535){\makebox(0,0)[r]{6}}
\put(1416,535){\rule[-0.175pt]{4.818pt}{0.350pt}}
\put(264,661){\rule[-0.175pt]{4.818pt}{0.350pt}}
%\put(242,661){\makebox(0,0)[r]{8}}
\put(1416,661){\rule[-0.175pt]{4.818pt}{0.350pt}}
\put(264,787){\rule[-0.175pt]{4.818pt}{0.350pt}}
%\put(242,787){\makebox(0,0)[r]{10}}
\put(1416,787){\rule[-0.175pt]{4.818pt}{0.350pt}}
\put(264,158){\rule[-0.175pt]{0.350pt}{4.818pt}}
%\put(264,113){\makebox(0,0){0}}
\put(264,767){\rule[-0.175pt]{0.350pt}{4.818pt}}
\put(411,158){\rule[-0.175pt]{0.350pt}{4.818pt}}
%\put(411,113){\makebox(0,0){0.5}}
\put(411,767){\rule[-0.175pt]{0.350pt}{4.818pt}}
\put(557,158){\rule[-0.175pt]{0.350pt}{4.818pt}}
%\put(557,113){\makebox(0,0){1}}
\put(557,767){\rule[-0.175pt]{0.350pt}{4.818pt}}
\put(704,158){\rule[-0.175pt]{0.350pt}{4.818pt}}
%\put(704,113){\makebox(0,0){1.5}}
\put(704,767){\rule[-0.175pt]{0.350pt}{4.818pt}}
\put(850,158){\rule[-0.175pt]{0.350pt}{4.818pt}}
%\put(850,113){\makebox(0,0){2}}
\put(850,767){\rule[-0.175pt]{0.350pt}{4.818pt}}
\put(997,158){\rule[-0.175pt]{0.350pt}{4.818pt}}
%\put(997,113){\makebox(0,0){2.5}}
\put(997,767){\rule[-0.175pt]{0.350pt}{4.818pt}}
\put(1143,158){\rule[-0.175pt]{0.350pt}{4.818pt}}
%\put(1143,113){\makebox(0,0){3}}
\put(1143,767){\rule[-0.175pt]{0.350pt}{4.818pt}}
\put(1290,158){\rule[-0.175pt]{0.350pt}{4.818pt}}
%\put(1290,113){\makebox(0,0){3.5}}
\put(1290,767){\rule[-0.175pt]{0.350pt}{4.818pt}}
\put(1436,158){\rule[-0.175pt]{0.350pt}{4.818pt}}
%\put(1436,113){\makebox(0,0){4}}
\put(1436,767){\rule[-0.175pt]{0.350pt}{4.818pt}}
\put(264,158){\rule[-0.175pt]{282.335pt}{0.350pt}}
\put(1436,158){\rule[-0.175pt]{0.350pt}{151.526pt}}
\put(264,787){\rule[-0.175pt]{282.335pt}{0.350pt}}
\put(100,472){\makebox(0,0)[l]{\shortstack{$\| u\|$}}}
\put(850,68){\makebox(0,0){$\lambda$}}
%\put(850,832){\makebox(0,0){plot}}
\put(264,158){\rule[-0.175pt]{0.350pt}{151.526pt}}
%\put(1306,722){\makebox(0,0)[r]{}}
%\put(1328,722){\rule[-0.175pt]{15.899pt}{0.350pt}}
\put(264,158){\usebox{\plotpoint}}
\put(264,158){\rule[-0.175pt]{6.304pt}{0.350pt}}
\put(290,159){\rule[-0.175pt]{6.304pt}{0.350pt}}
\put(316,160){\rule[-0.175pt]{6.304pt}{0.350pt}}
\put(342,161){\rule[-0.175pt]{6.304pt}{0.350pt}}
\put(368,162){\rule[-0.175pt]{6.304pt}{0.350pt}}
\put(394,163){\rule[-0.175pt]{6.304pt}{0.350pt}}
\put(420,164){\rule[-0.175pt]{5.644pt}{0.350pt}}
\put(444,165){\rule[-0.175pt]{5.644pt}{0.350pt}}
\put(467,166){\rule[-0.175pt]{5.644pt}{0.350pt}}
\put(491,167){\rule[-0.175pt]{5.644pt}{0.350pt}}
\put(514,168){\rule[-0.175pt]{5.644pt}{0.350pt}}
\put(538,169){\rule[-0.175pt]{5.644pt}{0.350pt}}
\put(561,170){\rule[-0.175pt]{5.644pt}{0.350pt}}
\put(585,171){\rule[-0.175pt]{6.384pt}{0.350pt}}
\put(611,172){\rule[-0.175pt]{6.384pt}{0.350pt}}
\put(638,173){\rule[-0.175pt]{6.384pt}{0.350pt}}
\put(664,174){\rule[-0.175pt]{6.384pt}{0.350pt}}
\put(691,175){\rule[-0.175pt]{6.384pt}{0.350pt}}
\put(717,176){\rule[-0.175pt]{6.384pt}{0.350pt}}
\put(744,177){\rule[-0.175pt]{5.862pt}{0.350pt}}
\put(768,178){\rule[-0.175pt]{5.862pt}{0.350pt}}
\put(792,179){\rule[-0.175pt]{5.862pt}{0.350pt}}
\put(816,180){\rule[-0.175pt]{5.862pt}{0.350pt}}
\put(841,181){\rule[-0.175pt]{5.862pt}{0.350pt}}
\put(865,182){\rule[-0.175pt]{5.862pt}{0.350pt}}
\put(889,183){\rule[-0.175pt]{4.371pt}{0.350pt}}
\put(908,184){\rule[-0.175pt]{4.371pt}{0.350pt}}
\put(926,185){\rule[-0.175pt]{4.371pt}{0.350pt}}
\put(944,186){\rule[-0.175pt]{4.371pt}{0.350pt}}
\put(962,187){\rule[-0.175pt]{4.371pt}{0.350pt}}
\put(980,188){\rule[-0.175pt]{4.371pt}{0.350pt}}
\put(998,189){\rule[-0.175pt]{4.371pt}{0.350pt}}
\put(1017,190){\rule[-0.175pt]{4.216pt}{0.350pt}}
\put(1034,191){\rule[-0.175pt]{4.216pt}{0.350pt}}
\put(1052,192){\rule[-0.175pt]{4.216pt}{0.350pt}}
\put(1069,193){\rule[-0.175pt]{4.216pt}{0.350pt}}
\put(1087,194){\rule[-0.175pt]{4.216pt}{0.350pt}}
\put(1104,195){\rule[-0.175pt]{4.216pt}{0.350pt}}
\put(1122,196){\rule[-0.175pt]{3.172pt}{0.350pt}}
\put(1135,197){\rule[-0.175pt]{3.172pt}{0.350pt}}
\put(1148,198){\rule[-0.175pt]{3.172pt}{0.350pt}}
\put(1161,199){\rule[-0.175pt]{3.172pt}{0.350pt}}
\put(1174,200){\rule[-0.175pt]{3.172pt}{0.350pt}}
\put(1187,201){\rule[-0.175pt]{3.172pt}{0.350pt}}
\put(1200,202){\rule[-0.175pt]{1.893pt}{0.350pt}}
\put(1208,203){\rule[-0.175pt]{1.893pt}{0.350pt}}
\put(1216,204){\rule[-0.175pt]{1.893pt}{0.350pt}}
\put(1224,205){\rule[-0.175pt]{1.893pt}{0.350pt}}
\put(1232,206){\rule[-0.175pt]{1.893pt}{0.350pt}}
\put(1240,207){\rule[-0.175pt]{1.893pt}{0.350pt}}
\put(1248,208){\rule[-0.175pt]{1.893pt}{0.350pt}}
\put(1256,209){\rule[-0.175pt]{1.245pt}{0.350pt}}
\put(1261,210){\rule[-0.175pt]{1.245pt}{0.350pt}}
\put(1266,211){\rule[-0.175pt]{1.245pt}{0.350pt}}
\put(1271,212){\rule[-0.175pt]{1.245pt}{0.350pt}}
\put(1276,213){\rule[-0.175pt]{1.245pt}{0.350pt}}
\put(1281,214){\rule[-0.175pt]{1.245pt}{0.350pt}}
\put(1286,215){\usebox{\plotpoint}}
\put(1288,216){\usebox{\plotpoint}}
\put(1289,217){\usebox{\plotpoint}}
\put(1291,218){\usebox{\plotpoint}}
\put(1292,219){\usebox{\plotpoint}}
\put(1294,220){\usebox{\plotpoint}}
\put(1295,221){\usebox{\plotpoint}}
\put(1295,222){\rule[-0.175pt]{0.361pt}{0.350pt}}
\put(1294,223){\rule[-0.175pt]{0.361pt}{0.350pt}}
\put(1292,224){\rule[-0.175pt]{0.361pt}{0.350pt}}
\put(1291,225){\rule[-0.175pt]{0.361pt}{0.350pt}}
\put(1289,226){\rule[-0.175pt]{0.361pt}{0.350pt}}
\put(1288,227){\rule[-0.175pt]{0.361pt}{0.350pt}}
\put(1284,228){\rule[-0.175pt]{0.964pt}{0.350pt}}
\put(1280,229){\rule[-0.175pt]{0.964pt}{0.350pt}}
\put(1276,230){\rule[-0.175pt]{0.964pt}{0.350pt}}
\put(1272,231){\rule[-0.175pt]{0.964pt}{0.350pt}}
\put(1268,232){\rule[-0.175pt]{0.964pt}{0.350pt}}
\put(1264,233){\rule[-0.175pt]{0.964pt}{0.350pt}}
\put(1258,234){\rule[-0.175pt]{1.273pt}{0.350pt}}
\put(1253,235){\rule[-0.175pt]{1.273pt}{0.350pt}}
\put(1248,236){\rule[-0.175pt]{1.273pt}{0.350pt}}
\put(1242,237){\rule[-0.175pt]{1.273pt}{0.350pt}}
\put(1237,238){\rule[-0.175pt]{1.273pt}{0.350pt}}
\put(1232,239){\rule[-0.175pt]{1.273pt}{0.350pt}}
\put(1227,240){\rule[-0.175pt]{1.273pt}{0.350pt}}
\put(1219,241){\rule[-0.175pt]{1.847pt}{0.350pt}}
\put(1211,242){\rule[-0.175pt]{1.847pt}{0.350pt}}
\put(1204,243){\rule[-0.175pt]{1.847pt}{0.350pt}}
\put(1196,244){\rule[-0.175pt]{1.847pt}{0.350pt}}
\put(1188,245){\rule[-0.175pt]{1.847pt}{0.350pt}}
\put(1181,246){\rule[-0.175pt]{1.847pt}{0.350pt}}
\put(1172,247){\rule[-0.175pt]{2.128pt}{0.350pt}}
\put(1163,248){\rule[-0.175pt]{2.128pt}{0.350pt}}
\put(1154,249){\rule[-0.175pt]{2.128pt}{0.350pt}}
\put(1145,250){\rule[-0.175pt]{2.128pt}{0.350pt}}
\put(1136,251){\rule[-0.175pt]{2.128pt}{0.350pt}}
\put(1128,252){\rule[-0.175pt]{2.128pt}{0.350pt}}
\put(1120,253){\rule[-0.175pt]{1.893pt}{0.350pt}}
\put(1112,254){\rule[-0.175pt]{1.893pt}{0.350pt}}
\put(1104,255){\rule[-0.175pt]{1.893pt}{0.350pt}}
\put(1096,256){\rule[-0.175pt]{1.893pt}{0.350pt}}
\put(1088,257){\rule[-0.175pt]{1.893pt}{0.350pt}}
\put(1080,258){\rule[-0.175pt]{1.893pt}{0.350pt}}
\put(1073,259){\rule[-0.175pt]{1.893pt}{0.350pt}}
\put(1063,260){\rule[-0.175pt]{2.208pt}{0.350pt}}
\put(1054,261){\rule[-0.175pt]{2.208pt}{0.350pt}}
\put(1045,262){\rule[-0.175pt]{2.208pt}{0.350pt}}
\put(1036,263){\rule[-0.175pt]{2.208pt}{0.350pt}}
\put(1027,264){\rule[-0.175pt]{2.208pt}{0.350pt}}
\put(1018,265){\rule[-0.175pt]{2.208pt}{0.350pt}}
\put(1009,266){\rule[-0.175pt]{2.168pt}{0.350pt}}
\put(1000,267){\rule[-0.175pt]{2.168pt}{0.350pt}}
\put(991,268){\rule[-0.175pt]{2.168pt}{0.350pt}}
\put(982,269){\rule[-0.175pt]{2.168pt}{0.350pt}}
\put(973,270){\rule[-0.175pt]{2.168pt}{0.350pt}}
\put(964,271){\rule[-0.175pt]{2.168pt}{0.350pt}}
\put(957,272){\rule[-0.175pt]{1.686pt}{0.350pt}}
\put(950,273){\rule[-0.175pt]{1.686pt}{0.350pt}}
\put(943,274){\rule[-0.175pt]{1.686pt}{0.350pt}}
\put(936,275){\rule[-0.175pt]{1.686pt}{0.350pt}}
\put(929,276){\rule[-0.175pt]{1.686pt}{0.350pt}}
\put(922,277){\rule[-0.175pt]{1.686pt}{0.350pt}}
\put(915,278){\rule[-0.175pt]{1.686pt}{0.350pt}}
\put(907,279){\rule[-0.175pt]{1.767pt}{0.350pt}}
\put(900,280){\rule[-0.175pt]{1.767pt}{0.350pt}}
\put(893,281){\rule[-0.175pt]{1.767pt}{0.350pt}}
\put(885,282){\rule[-0.175pt]{1.767pt}{0.350pt}}
\put(878,283){\rule[-0.175pt]{1.767pt}{0.350pt}}
\put(871,284){\rule[-0.175pt]{1.767pt}{0.350pt}}
\put(864,285){\rule[-0.175pt]{1.486pt}{0.350pt}}
\put(858,286){\rule[-0.175pt]{1.486pt}{0.350pt}}
\put(852,287){\rule[-0.175pt]{1.486pt}{0.350pt}}
\put(846,288){\rule[-0.175pt]{1.486pt}{0.350pt}}
\put(840,289){\rule[-0.175pt]{1.486pt}{0.350pt}}
\put(834,290){\rule[-0.175pt]{1.486pt}{0.350pt}}
\put(829,291){\rule[-0.175pt]{0.998pt}{0.350pt}}
\put(825,292){\rule[-0.175pt]{0.998pt}{0.350pt}}
\put(821,293){\rule[-0.175pt]{0.998pt}{0.350pt}}
\put(817,294){\rule[-0.175pt]{0.998pt}{0.350pt}}
\put(813,295){\rule[-0.175pt]{0.998pt}{0.350pt}}
\put(809,296){\rule[-0.175pt]{0.998pt}{0.350pt}}
\put(805,297){\rule[-0.175pt]{0.998pt}{0.350pt}}
\put(801,298){\rule[-0.175pt]{0.883pt}{0.350pt}}
\put(797,299){\rule[-0.175pt]{0.883pt}{0.350pt}}
\put(793,300){\rule[-0.175pt]{0.883pt}{0.350pt}}
\put(790,301){\rule[-0.175pt]{0.883pt}{0.350pt}}
\put(786,302){\rule[-0.175pt]{0.883pt}{0.350pt}}
\put(783,303){\rule[-0.175pt]{0.883pt}{0.350pt}}
\put(780,304){\rule[-0.175pt]{0.522pt}{0.350pt}}
\put(778,305){\rule[-0.175pt]{0.522pt}{0.350pt}}
\put(776,306){\rule[-0.175pt]{0.522pt}{0.350pt}}
\put(774,307){\rule[-0.175pt]{0.522pt}{0.350pt}}
\put(772,308){\rule[-0.175pt]{0.522pt}{0.350pt}}
\put(770,309){\rule[-0.175pt]{0.522pt}{0.350pt}}
\put(770,310){\usebox{\plotpoint}}
\put(769,311){\usebox{\plotpoint}}
\put(768,312){\usebox{\plotpoint}}
\put(767,314){\usebox{\plotpoint}}
\put(766,315){\usebox{\plotpoint}}
\put(765,316){\rule[-0.175pt]{0.350pt}{0.723pt}}
\put(766,320){\rule[-0.175pt]{0.350pt}{0.723pt}}
\put(767,323){\usebox{\plotpoint}}
\put(768,324){\usebox{\plotpoint}}
\put(769,325){\usebox{\plotpoint}}
\put(771,326){\usebox{\plotpoint}}
\put(772,327){\usebox{\plotpoint}}
\put(774,328){\usebox{\plotpoint}}
\put(775,329){\usebox{\plotpoint}}
\put(777,330){\rule[-0.175pt]{0.602pt}{0.350pt}}
\put(779,331){\rule[-0.175pt]{0.602pt}{0.350pt}}
\put(782,332){\rule[-0.175pt]{0.602pt}{0.350pt}}
\put(784,333){\rule[-0.175pt]{0.602pt}{0.350pt}}
\put(787,334){\rule[-0.175pt]{0.602pt}{0.350pt}}
\put(789,335){\rule[-0.175pt]{0.602pt}{0.350pt}}
\put(792,336){\rule[-0.175pt]{0.843pt}{0.350pt}}
\put(795,337){\rule[-0.175pt]{0.843pt}{0.350pt}}
\put(799,338){\rule[-0.175pt]{0.843pt}{0.350pt}}
\put(802,339){\rule[-0.175pt]{0.843pt}{0.350pt}}
\put(806,340){\rule[-0.175pt]{0.843pt}{0.350pt}}
\put(809,341){\rule[-0.175pt]{0.843pt}{0.350pt}}
\put(813,342){\rule[-0.175pt]{0.826pt}{0.350pt}}
\put(816,343){\rule[-0.175pt]{0.826pt}{0.350pt}}
\put(819,344){\rule[-0.175pt]{0.826pt}{0.350pt}}
\put(823,345){\rule[-0.175pt]{0.826pt}{0.350pt}}
\put(826,346){\rule[-0.175pt]{0.826pt}{0.350pt}}
\put(830,347){\rule[-0.175pt]{0.826pt}{0.350pt}}
\put(833,348){\rule[-0.175pt]{0.826pt}{0.350pt}}
\put(837,349){\rule[-0.175pt]{1.084pt}{0.350pt}}
\put(841,350){\rule[-0.175pt]{1.084pt}{0.350pt}}
\put(846,351){\rule[-0.175pt]{1.084pt}{0.350pt}}
\put(850,352){\rule[-0.175pt]{1.084pt}{0.350pt}}
\put(855,353){\rule[-0.175pt]{1.084pt}{0.350pt}}
\put(859,354){\rule[-0.175pt]{1.084pt}{0.350pt}}
\put(864,355){\rule[-0.175pt]{1.164pt}{0.350pt}}
\put(868,356){\rule[-0.175pt]{1.164pt}{0.350pt}}
\put(873,357){\rule[-0.175pt]{1.164pt}{0.350pt}}
\put(878,358){\rule[-0.175pt]{1.164pt}{0.350pt}}
\put(883,359){\rule[-0.175pt]{1.164pt}{0.350pt}}
\put(888,360){\rule[-0.175pt]{1.164pt}{0.350pt}}
\put(892,361){\rule[-0.175pt]{1.032pt}{0.350pt}}
\put(897,362){\rule[-0.175pt]{1.032pt}{0.350pt}}
\put(901,363){\rule[-0.175pt]{1.032pt}{0.350pt}}
\put(905,364){\rule[-0.175pt]{1.032pt}{0.350pt}}
\put(910,365){\rule[-0.175pt]{1.032pt}{0.350pt}}
\put(914,366){\rule[-0.175pt]{1.032pt}{0.350pt}}
\put(918,367){\rule[-0.175pt]{1.032pt}{0.350pt}}
\put(922,368){\rule[-0.175pt]{1.205pt}{0.350pt}}
\put(928,369){\rule[-0.175pt]{1.204pt}{0.350pt}}
\put(933,370){\rule[-0.175pt]{1.204pt}{0.350pt}}
\put(938,371){\rule[-0.175pt]{1.204pt}{0.350pt}}
\put(943,372){\rule[-0.175pt]{1.204pt}{0.350pt}}
\put(948,373){\rule[-0.175pt]{1.204pt}{0.350pt}}
\put(953,374){\rule[-0.175pt]{1.124pt}{0.350pt}}
\put(957,375){\rule[-0.175pt]{1.124pt}{0.350pt}}
\put(962,376){\rule[-0.175pt]{1.124pt}{0.350pt}}
\put(967,377){\rule[-0.175pt]{1.124pt}{0.350pt}}
\put(971,378){\rule[-0.175pt]{1.124pt}{0.350pt}}
\put(976,379){\rule[-0.175pt]{1.124pt}{0.350pt}}
\put(981,380){\rule[-0.175pt]{0.929pt}{0.350pt}}
\put(984,381){\rule[-0.175pt]{0.929pt}{0.350pt}}
\put(988,382){\rule[-0.175pt]{0.929pt}{0.350pt}}
\put(992,383){\rule[-0.175pt]{0.929pt}{0.350pt}}
\put(996,384){\rule[-0.175pt]{0.929pt}{0.350pt}}
\put(1000,385){\rule[-0.175pt]{0.929pt}{0.350pt}}
\put(1004,386){\rule[-0.175pt]{0.929pt}{0.350pt}}
\put(1007,387){\rule[-0.175pt]{0.923pt}{0.350pt}}
\put(1011,388){\rule[-0.175pt]{0.923pt}{0.350pt}}
\put(1015,389){\rule[-0.175pt]{0.923pt}{0.350pt}}
\put(1019,390){\rule[-0.175pt]{0.923pt}{0.350pt}}
\put(1023,391){\rule[-0.175pt]{0.923pt}{0.350pt}}
\put(1027,392){\rule[-0.175pt]{0.923pt}{0.350pt}}
\put(1031,393){\rule[-0.175pt]{0.843pt}{0.350pt}}
\put(1034,394){\rule[-0.175pt]{0.843pt}{0.350pt}}
\put(1038,395){\rule[-0.175pt]{0.843pt}{0.350pt}}
\put(1041,396){\rule[-0.175pt]{0.843pt}{0.350pt}}
\put(1045,397){\rule[-0.175pt]{0.843pt}{0.350pt}}
\put(1048,398){\rule[-0.175pt]{0.843pt}{0.350pt}}
\put(1052,399){\rule[-0.175pt]{0.585pt}{0.350pt}}
\put(1054,400){\rule[-0.175pt]{0.585pt}{0.350pt}}
\put(1056,401){\rule[-0.175pt]{0.585pt}{0.350pt}}
\put(1059,402){\rule[-0.175pt]{0.585pt}{0.350pt}}
\put(1061,403){\rule[-0.175pt]{0.585pt}{0.350pt}}
\put(1064,404){\rule[-0.175pt]{0.585pt}{0.350pt}}
\put(1066,405){\rule[-0.175pt]{0.585pt}{0.350pt}}
\put(1069,406){\rule[-0.175pt]{0.522pt}{0.350pt}}
\put(1071,407){\rule[-0.175pt]{0.522pt}{0.350pt}}
\put(1073,408){\rule[-0.175pt]{0.522pt}{0.350pt}}
\put(1075,409){\rule[-0.175pt]{0.522pt}{0.350pt}}
\put(1077,410){\rule[-0.175pt]{0.522pt}{0.350pt}}
\put(1079,411){\rule[-0.175pt]{0.522pt}{0.350pt}}
\put(1081,412){\rule[-0.175pt]{0.402pt}{0.350pt}}
\put(1083,413){\rule[-0.175pt]{0.401pt}{0.350pt}}
\put(1085,414){\rule[-0.175pt]{0.401pt}{0.350pt}}
\put(1086,415){\rule[-0.175pt]{0.401pt}{0.350pt}}
\put(1088,416){\rule[-0.175pt]{0.401pt}{0.350pt}}
\put(1090,417){\rule[-0.175pt]{0.401pt}{0.350pt}}
\put(1091,418){\usebox{\plotpoint}}
\put(1092,418){\usebox{\plotpoint}}
\put(1093,419){\usebox{\plotpoint}}
\put(1094,420){\usebox{\plotpoint}}
\put(1095,422){\usebox{\plotpoint}}
\put(1096,423){\usebox{\plotpoint}}
\put(1097,424){\rule[-0.175pt]{0.350pt}{0.723pt}}
\put(1098,428){\rule[-0.175pt]{0.350pt}{0.723pt}}
\put(1099,431){\rule[-0.175pt]{0.350pt}{1.686pt}}
\put(1098,438){\usebox{\plotpoint}}
\put(1097,439){\usebox{\plotpoint}}
\put(1096,440){\usebox{\plotpoint}}
\put(1095,441){\usebox{\plotpoint}}
\put(1094,442){\usebox{\plotpoint}}
\put(1091,444){\usebox{\plotpoint}}
\put(1090,445){\usebox{\plotpoint}}
\put(1089,446){\usebox{\plotpoint}}
\put(1088,447){\usebox{\plotpoint}}
\put(1087,448){\usebox{\plotpoint}}
\put(1086,449){\usebox{\plotpoint}}
\put(1084,450){\usebox{\plotpoint}}
\put(1083,451){\usebox{\plotpoint}}
\put(1081,452){\usebox{\plotpoint}}
\put(1080,453){\usebox{\plotpoint}}
\put(1078,454){\usebox{\plotpoint}}
\put(1077,455){\usebox{\plotpoint}}
\put(1076,456){\usebox{\plotpoint}}
\put(1074,457){\rule[-0.175pt]{0.442pt}{0.350pt}}
\put(1072,458){\rule[-0.175pt]{0.442pt}{0.350pt}}
\put(1070,459){\rule[-0.175pt]{0.442pt}{0.350pt}}
\put(1068,460){\rule[-0.175pt]{0.442pt}{0.350pt}}
\put(1066,461){\rule[-0.175pt]{0.442pt}{0.350pt}}
\put(1065,462){\rule[-0.175pt]{0.442pt}{0.350pt}}
\put(1063,463){\rule[-0.175pt]{0.482pt}{0.350pt}}
\put(1061,464){\rule[-0.175pt]{0.482pt}{0.350pt}}
\put(1059,465){\rule[-0.175pt]{0.482pt}{0.350pt}}
\put(1057,466){\rule[-0.175pt]{0.482pt}{0.350pt}}
\put(1055,467){\rule[-0.175pt]{0.482pt}{0.350pt}}
\put(1053,468){\rule[-0.175pt]{0.482pt}{0.350pt}}
\put(1051,469){\rule[-0.175pt]{0.482pt}{0.350pt}}
\put(1049,470){\rule[-0.175pt]{0.482pt}{0.350pt}}
\put(1047,471){\rule[-0.175pt]{0.482pt}{0.350pt}}
\put(1045,472){\rule[-0.175pt]{0.482pt}{0.350pt}}
\put(1043,473){\rule[-0.175pt]{0.482pt}{0.350pt}}
\put(1041,474){\rule[-0.175pt]{0.482pt}{0.350pt}}
\put(1039,475){\rule[-0.175pt]{0.482pt}{0.350pt}}
\put(1036,476){\rule[-0.175pt]{0.522pt}{0.350pt}}
\put(1034,477){\rule[-0.175pt]{0.522pt}{0.350pt}}
\put(1032,478){\rule[-0.175pt]{0.522pt}{0.350pt}}
\put(1030,479){\rule[-0.175pt]{0.522pt}{0.350pt}}
\put(1028,480){\rule[-0.175pt]{0.522pt}{0.350pt}}
\put(1026,481){\rule[-0.175pt]{0.522pt}{0.350pt}}
\put(1023,482){\rule[-0.175pt]{0.522pt}{0.350pt}}
\put(1021,483){\rule[-0.175pt]{0.522pt}{0.350pt}}
\put(1019,484){\rule[-0.175pt]{0.522pt}{0.350pt}}
\put(1017,485){\rule[-0.175pt]{0.522pt}{0.350pt}}
\put(1015,486){\rule[-0.175pt]{0.522pt}{0.350pt}}
\put(1013,487){\rule[-0.175pt]{0.522pt}{0.350pt}}
\put(1011,488){\rule[-0.175pt]{0.447pt}{0.350pt}}
\put(1009,489){\rule[-0.175pt]{0.447pt}{0.350pt}}
\put(1007,490){\rule[-0.175pt]{0.447pt}{0.350pt}}
\put(1005,491){\rule[-0.175pt]{0.447pt}{0.350pt}}
\put(1003,492){\rule[-0.175pt]{0.447pt}{0.350pt}}
\put(1001,493){\rule[-0.175pt]{0.447pt}{0.350pt}}
\put(1000,494){\rule[-0.175pt]{0.447pt}{0.350pt}}
\put(998,495){\rule[-0.175pt]{0.442pt}{0.350pt}}
\put(996,496){\rule[-0.175pt]{0.442pt}{0.350pt}}
\put(994,497){\rule[-0.175pt]{0.442pt}{0.350pt}}
\put(992,498){\rule[-0.175pt]{0.442pt}{0.350pt}}
\put(990,499){\rule[-0.175pt]{0.442pt}{0.350pt}}
\put(989,500){\rule[-0.175pt]{0.442pt}{0.350pt}}
\put(987,501){\rule[-0.175pt]{0.442pt}{0.350pt}}
\put(985,502){\rule[-0.175pt]{0.442pt}{0.350pt}}
\put(983,503){\rule[-0.175pt]{0.442pt}{0.350pt}}
\put(981,504){\rule[-0.175pt]{0.442pt}{0.350pt}}
\put(979,505){\rule[-0.175pt]{0.442pt}{0.350pt}}
\put(978,506){\rule[-0.175pt]{0.442pt}{0.350pt}}
\put(976,507){\usebox{\plotpoint}}
\put(975,508){\usebox{\plotpoint}}
\put(974,509){\usebox{\plotpoint}}
\put(972,510){\usebox{\plotpoint}}
\put(971,511){\usebox{\plotpoint}}
\put(970,512){\usebox{\plotpoint}}
\put(969,513){\usebox{\plotpoint}}
\put(967,514){\usebox{\plotpoint}}
\put(966,515){\usebox{\plotpoint}}
\put(965,516){\usebox{\plotpoint}}
\put(964,517){\usebox{\plotpoint}}
\put(963,518){\usebox{\plotpoint}}
\put(962,519){\usebox{\plotpoint}}
\put(962,520){\usebox{\plotpoint}}
\put(961,521){\usebox{\plotpoint}}
\put(960,522){\usebox{\plotpoint}}
\put(959,524){\usebox{\plotpoint}}
\put(958,525){\usebox{\plotpoint}}
\put(957,527){\rule[-0.175pt]{0.350pt}{0.361pt}}
\put(956,528){\rule[-0.175pt]{0.350pt}{0.361pt}}
\put(955,530){\rule[-0.175pt]{0.350pt}{0.361pt}}
\put(954,531){\rule[-0.175pt]{0.350pt}{0.361pt}}
\put(953,533){\rule[-0.175pt]{0.350pt}{0.723pt}}
\put(952,536){\rule[-0.175pt]{0.350pt}{0.723pt}}
\put(951,539){\rule[-0.175pt]{0.350pt}{1.686pt}}
\put(950,546){\rule[-0.175pt]{0.350pt}{1.445pt}}
\put(951,552){\rule[-0.175pt]{0.350pt}{0.482pt}}
\put(952,554){\rule[-0.175pt]{0.350pt}{0.482pt}}
\put(953,556){\rule[-0.175pt]{0.350pt}{0.482pt}}
\put(954,558){\rule[-0.175pt]{0.350pt}{0.562pt}}
\put(955,560){\rule[-0.175pt]{0.350pt}{0.562pt}}
\put(956,562){\rule[-0.175pt]{0.350pt}{0.562pt}}
\put(957,564){\usebox{\plotpoint}}
\put(958,566){\usebox{\plotpoint}}
\put(959,567){\usebox{\plotpoint}}
\put(960,568){\usebox{\plotpoint}}
\put(961,569){\usebox{\plotpoint}}
\put(962,571){\usebox{\plotpoint}}
\put(963,572){\usebox{\plotpoint}}
\put(964,573){\usebox{\plotpoint}}
\put(965,574){\usebox{\plotpoint}}
\put(966,575){\usebox{\plotpoint}}
\put(967,577){\usebox{\plotpoint}}
\put(968,578){\usebox{\plotpoint}}
\put(969,579){\usebox{\plotpoint}}
\put(970,580){\usebox{\plotpoint}}
\put(971,581){\usebox{\plotpoint}}
\put(972,582){\usebox{\plotpoint}}
\put(973,584){\usebox{\plotpoint}}
\put(974,585){\usebox{\plotpoint}}
\put(975,586){\usebox{\plotpoint}}
\put(976,587){\usebox{\plotpoint}}
\put(977,588){\usebox{\plotpoint}}
\put(978,589){\usebox{\plotpoint}}
\put(979,590){\usebox{\plotpoint}}
\put(980,591){\usebox{\plotpoint}}
\put(981,592){\usebox{\plotpoint}}
\put(982,593){\usebox{\plotpoint}}
\put(983,594){\usebox{\plotpoint}}
\put(984,595){\usebox{\plotpoint}}
\put(985,596){\usebox{\plotpoint}}
\put(986,597){\usebox{\plotpoint}}
\put(987,598){\usebox{\plotpoint}}
\put(988,600){\usebox{\plotpoint}}
\put(989,601){\usebox{\plotpoint}}
\put(990,603){\usebox{\plotpoint}}
\put(991,604){\usebox{\plotpoint}}
\put(992,605){\usebox{\plotpoint}}
\put(993,606){\usebox{\plotpoint}}
\put(994,607){\usebox{\plotpoint}}
\put(995,608){\usebox{\plotpoint}}
\put(996,609){\usebox{\plotpoint}}
\put(997,610){\usebox{\plotpoint}}
\put(998,611){\usebox{\plotpoint}}
\put(999,612){\usebox{\plotpoint}}
\put(1000,613){\usebox{\plotpoint}}
\put(1001,615){\usebox{\plotpoint}}
\put(1002,616){\usebox{\plotpoint}}
\put(1003,617){\usebox{\plotpoint}}
\put(1004,619){\usebox{\plotpoint}}
\put(1005,620){\usebox{\plotpoint}}
\put(1006,622){\rule[-0.175pt]{0.350pt}{0.361pt}}
\put(1007,623){\rule[-0.175pt]{0.350pt}{0.361pt}}
\put(1008,625){\rule[-0.175pt]{0.350pt}{0.361pt}}
\put(1009,626){\rule[-0.175pt]{0.350pt}{0.361pt}}
\put(1010,628){\rule[-0.175pt]{0.350pt}{0.562pt}}
\put(1011,630){\rule[-0.175pt]{0.350pt}{0.562pt}}
\put(1012,632){\rule[-0.175pt]{0.350pt}{0.562pt}}
\put(1013,634){\rule[-0.175pt]{0.350pt}{0.723pt}}
\put(1014,638){\rule[-0.175pt]{0.350pt}{0.723pt}}
\put(1015,641){\rule[-0.175pt]{0.350pt}{1.445pt}}
\put(1016,647){\rule[-0.175pt]{0.350pt}{1.686pt}}
\put(1017,654){\rule[-0.175pt]{0.350pt}{3.734pt}}
\put(1016,669){\rule[-0.175pt]{0.350pt}{0.843pt}}
\put(1015,673){\rule[-0.175pt]{0.350pt}{1.445pt}}
\put(1014,679){\rule[-0.175pt]{0.350pt}{0.723pt}}
\put(1013,682){\rule[-0.175pt]{0.350pt}{0.723pt}}
\put(1012,685){\rule[-0.175pt]{0.350pt}{0.562pt}}
\put(1011,687){\rule[-0.175pt]{0.350pt}{0.562pt}}
\put(1010,689){\rule[-0.175pt]{0.350pt}{0.562pt}}
\put(1009,691){\rule[-0.175pt]{0.350pt}{0.723pt}}
\put(1008,695){\rule[-0.175pt]{0.350pt}{0.723pt}}
\put(1007,698){\rule[-0.175pt]{0.350pt}{0.482pt}}
\put(1006,700){\rule[-0.175pt]{0.350pt}{0.482pt}}
\put(1005,702){\rule[-0.175pt]{0.350pt}{0.482pt}}
\put(1004,704){\rule[-0.175pt]{0.350pt}{0.562pt}}
\put(1003,706){\rule[-0.175pt]{0.350pt}{0.562pt}}
\put(1002,708){\rule[-0.175pt]{0.350pt}{0.562pt}}
\put(1001,710){\rule[-0.175pt]{0.350pt}{0.723pt}}
\put(1000,714){\rule[-0.175pt]{0.350pt}{0.723pt}}
\put(999,717){\rule[-0.175pt]{0.350pt}{0.482pt}}
\put(998,719){\rule[-0.175pt]{0.350pt}{0.482pt}}
\put(997,721){\rule[-0.175pt]{0.350pt}{0.482pt}}
\put(996,723){\rule[-0.175pt]{0.350pt}{0.843pt}}
\put(995,726){\rule[-0.175pt]{0.350pt}{0.843pt}}
\put(994,730){\rule[-0.175pt]{0.350pt}{0.723pt}}
\put(993,733){\rule[-0.175pt]{0.350pt}{0.723pt}}
\put(992,736){\rule[-0.175pt]{0.350pt}{0.843pt}}
\put(991,739){\rule[-0.175pt]{0.350pt}{0.843pt}}
\put(990,743){\rule[-0.175pt]{0.350pt}{1.445pt}}
\put(989,749){\rule[-0.175pt]{0.350pt}{1.445pt}}
\put(988,755){\rule[-0.175pt]{0.350pt}{1.686pt}}
\put(987,762){\rule[-0.175pt]{0.350pt}{4.577pt}}
\put(988,781){\rule[-0.175pt]{0.350pt}{1.445pt}}
\end{picture}

\caption{Gelfand equation on the ball, $3\leq n \leq 9$.
\label{gelfand.fig2}}
\end{figure}
\end{verbatim}\end{quote}
One advantage to using the native \LaTeX{} {\tt picture} environment
is that the fonts will be assured to agree and the pictures can be viewed
in the {\tt .dvi} viewer.

\subsection{PostScript}
Many drawing applications now allow the export of a graphic to the
{\em Encapsulated PostScript} format.  These files have a suffix of
{\tt .EPS} or {\tt .EPSF} and are similar to a regular PostScript
file except that they contain a {\em bounding box} which describes
the dimensions of the figure.

In order to include PostScript figures, the {\tt epsfig} (or {\tt psfig}
depending on the system you are using) style file must be included in either
the {\tt\verb|\documentstyle|} command or the preamble using the {\tt input} command.

Figure~\ref{vwcontr} is a plot from Matlab.
\begin{figure}[htbp]
\centerline{
\psfig{figure=vwcontr.eps,width=5in,angle=0}
           }
\caption{$\sigma$ as a Function of Voltage and Speed, $\alpha = 20$}
\label{vwcontr}
\end{figure}
The commands to include this figure are
\begin{quote}\tt\singlespace\begin{verbatim}
\begin{figure}[htbp]
\centerline{
\psfig{figure=vwcontr.ps,width=5in,angle=0}
           }
\caption{$\sigma$ as a Function of Voltage and Speed, $\alpha = 20$}
\label{vwcontr}
\end{figure}
\end{verbatim}\end{quote}

Observe that the {\tt \verb|\psfig|} command allows the scaling of the figure
by setting either the {\tt width} or {\tt height} of the figure.  If only one
dimension is specified, the other is computed to keep the same aspect ratio.
The figure can also be rotated by setting {\tt angle} to the desired value in
degrees.
           % Chapter 3 Edited from UW Math Dept's Sample Thesis
% bibs.tex
%
% This chapter briefly talks about BibTex and is mostly
% copied from a similar chapter from "How to TeX a Thesis:
% The Purdue Thesis Styles" by James Darrell McCauley and
% Scott Hucker
%

\newcommand{\BibTeX}{{\sc Bib}\TeX}

\chapter{Citations and Bibliographies}
This chapter is an edited form of the same chapter from {\em How to 
\TeX{} a Thesis: The Purdue Thesis Styles} by James Darrell McCauley and
Scott Hucker.

The task of compiling and formatting the sources cited in papers can
be quite tedious, especially for large documents like theses.  A program
separate from \LaTeX{}, called ``\BibTeX{},''can be used to automate this task~\cite{lamport}.

\section{The Citation Command}
When referring to the work of someone else, the {\tt \verb|\cite|} command is used.
This generates the citation in the text for you.  In the above paragraph, the command
{\tt \verb|\cite{lamport}|} was used after the word ``task.''  The formatting of your
citation is handled by either the document style or a style option.  The default citation
style uses the number system (a number in square brackets).  Other citation styles
may use the author-date system, (Lamport, 1986) or the superscript$^3$ system.

\section{Bibliography Styles}
The way that a reference is formatted in your bibliography depends on the bibliography
style, which is specified near the beginning of your document with the\break
{\tt \verb|\bibliographstyle{file}|} command.  The file {\tt file.bst} is the name of the 
bibliography style file.  Standard \BibTeX{} bibliography style files include {\tt plain},
{\tt unsrt}, {\tt alpha}, and {\tt abbrev}.  The bibliography style governs whether or not
references are sorted, whether first names or initials are used for authors, whether or 
not last names are listed first, the location of the year in the references (after the
author or at the end of the reference), {\em etc.}.  You may be required by your
department or major professor to follow as style for a particular journal.  If so, then you
will need to find a \BibTeX{} style file to suit your needs.  Most major journals have
style files.  If you cannot locate an appropriate \BibTeX{} style file, then choose the
one which is closest and then edit the {\tt .bbl} file by hand.  See Section~\ref{BBL}
for a brief discussion on the {\tt .bbl} file.  Some common, but non-standard \BibTeX{}
styles include
\begin{tabbing}
{\tt jacs-new.bstxxxx}\= {\em Journal of the American Chemical Society}\kill
{\tt acm.bst}\>The Association for Computing Machinery\\
{\tt ieeetr.bst}\> The {\em IEEE Transactions} style\\
{\tt jacs-new.bst}\> {\em Journal of the American Chemical Society}
\end{tabbing}

\section{The Database}
The  {\tt \verb|\bibliography{file}|} command is placed in your input file at the location
where the ``List of References'' section\footnote{or ``Bibliography'' 
if {\tt \char92 altbibtitle } has been specified in the preamble.} would be.  It specifies the name (or names) of
your bibliographic data base, {\tt file.bib}.  An example entry in a \BibTeX{}
database is:
\begin{quote}\singlespace\tt\begin{verbatim}
@book{ lamport86 ,
     author =    "Leslie Lamport" ,
     title =     "\LaTeX: A Document Preparation System" ,
     publisher = "Addison--Wesley Pub.\ Co." ,
     year =      "1986" ,
     address =   "Reading, MA" 
}
\end{verbatim}\end{quote}

The citation key is the first field in this entry--- citing this book in a \LaTeX{}
file would look like
\begin{quote}\singlespace\tt\begin{verbatim}
According to Lamport~\cite{lamport86} ...
\end{verbatim}\end{quote}
The tilde ({\tt \verb|~|}) is used to tie the word ``Lamport'' to the citation
generated.  The space between these words is then unbreakable---the word ``Lamport''
and the citation \cite{lamport} will not be split across two lines if they happen to occur
near the end of a line.

A listing of all entry types with their required and optional fields is given in 
Appendix~\ref{bibrefs}. There are several tools which exist to help in editing a \BibTeX{}
file, however, their use is beyond the scope of this manual and can be found by searching
the net.  You can simply use a plain text editor like {\tt vi} or {\tt WordPad} to edit
and create the database files.

There are several rules which you must follow when creating your database.  Authors are
always listed by their full names, first name first, and multiple authors are separated
by {\tt and}.  For example
\begin{quote}\singlespace\tt\begin{verbatim}
author = "John Jay Park and Frederick Gene Watson and
          Michelle Catherine Smith",
\end{verbatim}\end{quote}
If you were using {\tt abbrv} as your {\tt bibliographystyle}, a reference for these
authors may look like:
\begin{quote}
J.J. Park, F.G. Watson, and M.C. Smith \ldots
\end{quote}

Some styles only capitalize the first word of the title.  If you use any acronyms or
other words that should always be capitalized in titles, then they should be 
enclosed in {\tt \{\}}'s ({\em e.g.}, {\tt \{Fortran\}}, {\tt \{N\}ewton}).
This protects the case of these characters.

There are several other rules for \BibTeX{} listed in~\cite{lamport} which should be
referred to because they are not discussed here.

\section{Putting It All Together}
\label{BBL}
To aid the reader in understanding how all of this works together, the following 
excerpt was taken from Lamport~\cite{lamport}:
\begin{quotation}\singlespace
When you ran \LaTeX{} with the input file {\tt sample.tex}, you may have
noticed that \LaTeX{} created a file named {\tt sample.aux}.  This file,
called an {\em auxiliary} file, contains cross-referencing information.  Since
{\tt sample.tex} contains no cross-referencing commands, the auxiliary file it
produces has no information.  However, suppose that \LaTeX{} is run with an
input file named {\tt myfile.tex} that has citations and bibliography-making
[or referencing] commands.  The auxiliary file {\tt myfile.aux} that it produces
will contain all of the citation keys and the arguments of the {\tt \verb|\bibliography|}
and {\tt\verb|\bibliographystyle|} commands.  When \BibTeX{} is run, it reads
this information from the auxiliary file and produces a file named {\tt myfile.bbl}
containing \LaTeX{} commands to produce the source list \ldots The next time
\LaTeX{} is run on {\tt myfile.tex}, the {\tt \verb|\bibliography|} command reads
the {\tt bbl} file ({\tt myfile.bbl}), which generates the source list.
\end{quotation}

Thus, the command sequence for a source file called {\tt main.tex} which is going to
use \BibTeX{} would be:
\begin{quote}\singlespace\tt\begin{verbatim}
latex main.tex
bibtex main
latex main
latex main
\end{verbatim}\end{quote}
The first \LaTeX{} is to collect all of the citations for \BibTeX{}.  Then
\BibTeX{} is run to generate the bibliography.  \LaTeX{} is run again to
incorporate the bibliography into the document and the run the last time to
update any references (like pages in the Table of Contents) which changed when
the bibliography was included.
           % Chapter 4 From PU Thesis styles, by J.D. McCauley
% usage.tex
%
% This file explains how to use the withesis style
%   it is heavily modelled after a similar chapter by McCauley
%   for the Purdue Thesis style
%
% Eric Benedict, May 2000
%
% It is provided without warranty on an AS IS basis.


\chapter{Using the {\tt withesis} Style}

You can get a copy of the \LaTeX{} style for creating a University
of Wisconsin--Madison thesis or dissertation from:

{\tt http://www.cae.wisc.edu/\verb+~+benedict/LaTeX.html}

After somehow unpacking it, you will have the style files ({\tt withesis.sty}
{\tt withe10.sty}, and {\tt withe12.sty}) as well the files used to create
this document.  The files used for this document can be copied and used as a
template for your own thesis or dissertation.

The final printed form of this document is useful, but the
combination of the source code and final copy form a much more valuable
reference.  Keeping a working copy of the this document can be helpful
when you are later working on your thesis or disseration and want to know
how to do something.  If you find a similar example in this document,
then you can simply look at the corresponding source code and add it to
your document.    Because many parts of this document were written by
different people, the styles and techniques are also different and provide
different ways of achieving the same or similar results.

Because of the typical size of theses, it makes sense to break the document
up into several smaller files.  Usually this is done at the chapter level.
These files can then be {\tt \verb|\include|}d in a {\em root} file.  It is
the {\em root} file that you will run \LaTeX{} on.  For this manual, the
root file is called {\tt main.tex}.

\section{The Root File and the Preamble}
The {\tt \verb|\documentclass|} command is used to tell \LaTeX{} that you will
be using the {\tt withesis} document class and it is the first command in your
root file.  Class options such as {\tt 10pt}, {\tt 12pt}, {\tt msthesis} or
{\tt margincheck} are specified here:

{\tt \verb|\documentclass[12pt,msthesis]{withesis}|}

The class option {\tt msthesis} sets the margins to be appropriate for depositing
with the UW library, namely a 1.25 inch left margin with the remaining margins 1 inch.
The defaults for the title page are also defined for a thesis and for a Master of
Science degree.

The class option {\tt margincheck} will place a small black square at the end of
each line which exceeds the margins.\footnote{In reality, the square is
placed at the end of lines which exceed their {\tt \char92hbox}.  This usually
(but not always) indicates a  margin violation on the right margin.  Left
margin violations aren't indicated and if the margin violation is large enough,
there isn't room for the black box to be visiable.}  This is visible both in the {\tt .dvi} file
as well as in the {\tt .ps} file.

The area immediately following this command is called the {\em preamble} and is
used for things like including different style packages,
defining new macros and declaring the page style.

The style packages can be used to easily change the thesis font.  For example,
this document is set in Times Roman instead of the \LaTeX default of Computer
Modern.  This change was performed by including the {\tt times} package:

{\tt\verb|\usepackage{times}|}\footnote{In this document, the typewriter font
{\tt $\backslash$tt} was redefined to use the Computer Modern font with the command
{\tt $\backslash$renewcommand\{$\backslash$ttdefault\}\{cmtt\}}.  
For more information, see~\cite{goossens}.}

Remember that if you change the fonts from the default Computer Modern to
PostScript ({\em e.g.} Times Roman) then in order to correctly see the
document, you will need to convert the {\tt *.dvi} output into a {\tt *.ps}
file and view the document with a PostScript viewer. This is required since 
most {\tt *.dvi} previewer programs cannot 
display PostScript fonts.  Usually, the previewer will substitute
default fonts so the document may be viewed; however, since the alternate
fonts may not be the same size, the formatting of the document may appear
to be incorrect.

The style package for including Postscript figures, {\tt epsfig}, is included with

{\tt\verb|\usepackage{epsfig}|}

If multiple style packages are required, then they can be combined into one statement
as follows:

{\tt\verb|\usepackage{epsfig,times}|}

Many different style packages are available.  For more information, see~\cite{goossens}.

The page styles are defined using a similar method.
A special style is defined for the {\tt withesis} style:

{\tt\verb|\pagestyle{thesisdraft}|}

This style causes the footer text to become:

{\verb| DRAFT: Do Not Distribute        <time><Date>        <input file name>|}

This appears at the bottom of every page.

In addition to the page style command, the {\tt withesis} has defined several useful
commands which are specified in the preamble.  They include {\tt \verb| \draftmargin|},
{\tt \verb|\draftscreen|}, {\tt \verb|\noappendixtables|}, and
{\tt \verb|\noappendixfigures|}.

The command  {\tt \verb|\draftmargin|} draws a PostScript box with the dimensions of
the margins.  This makes it easy to check that the margins are correct and to see if
any of the text or figures are outside of the required margins.  This box is only visible
in the {\tt .ps} file since it is a PostScript special.


The command  {\tt \verb|\draftscreen|} draws a PostScript screen with the word {\em DRAFT}
in light grey and diagonally across the page.  This screen is only visible
in the {\tt .ps} file since it is a PostScript special.

The commands {\tt \verb|\noappendixtables|} and/or {\tt \verb|\noappendixfigures|} should
be used if the appendix does not have either tables or figures respectively.  These commands
inhibit the Appendix Table or Appendix Figure titles in the List of Tables or List of
Figures.\label{usage:noapp}


If you have specified the {\tt psfig} or {\tt epsfig} document style package, then a useful
command is {\tt \verb|\psdraft|}.  This command will show the bounding box that the figure
would occupy (instead of actually including the figure).  This speeds up the draft copy
printing, reduces toner usage and the drawn box is visible in the {\tt .dvi} file.

The next usual command is {\tt \verb|\begin{document}|}.  The following example is part
of the root file used for this manual.

\begin{quote} \singlespace\footnotesize\tt
\begin{verbatim}
\bibliographystyle{plain}
% prelude.tex
%   - titlepage
%   - dedication
%   - acknowledgments
%   - table of contents, list of tables and list of figures
%   - nomenclature
%   - abstract
%============================================================================


\clearpage\pagenumbering{roman}  % This makes the page numbers Roman (i, ii, etc)



% TITLE PAGE
%   - define \title{} \author{} \date{}
\title{How to \LaTeX\ a Thesis}
\author{Eric R. L. Benedict}
\date{2000}
%   - The default degree is ``Doctor of Philosophy''
%     (unless the document style msthesis is specified
%      and then the default degree is ``Master of Science'')
%     Degree can be changed using the command \degree{}
\degree{Master \TeX nician}
%   - The default is dissertation, unless the document style
%     msthesis was specified in which case it becomes thesis.
%     If msthesis is specified for the MS margins, you can
%     still have a dissertation if you specify \disseration
%\disseration
%   - for a masters project report, specify \project
%\project
%   - for a preliminary report, specify \prelim
\prelim
%   - for a masters thesis, specify \thesis
%\thesis
%   - The default department is ``Electrical Engineering''
%     The department can be changed using the command \department{}
%\department{New Department}
%   - once the above are defined, use \maketitle to generate the titlepage
\maketitle

% COPYRIGHT PAGE
%   - To include a copyright page use \copyrightpage
\copyrightpage

% DEDICATION
\begin{dedication}
To my pet rock, Skippy.
\end{dedication}

% ACKNOWLEDGMENTS
\begin{acknowledgments}
I thank the many people who have done lots of nice things for me.
\end{acknowledgments}

% CONTENTS, TABLES, FIGURES
\tableofcontents
\listoftables
\listoffigures

% NOMENCLATURE
\begin{nomenclature}
\begin{description}
\item{\makebox[0.75in][l]{\TeX}}
       \parbox[t]{5in}{a typesetting system by Donald Knuth~\cite{knuth}.  It
       also refers to the ``plain'' format.  The proper pronounciation
       rhymes with ``heck'' and ``peck'' and does not sound like
       ``hex'' or ``Rex.''\\}

\item{\makebox[0.75in][l]{\LaTeX}}  
        \parbox[t]{5in}{a set of \TeX{} macros originally written by Leslie 
        Lamport~\cite{lamport}.  The proper pronunciation is 
        {\tt l\={a}$\cdot$tek'} and not {\tt l\={a}'$\cdot$teks} (see above).\\}

\item{\makebox[0.75in][l]{{\sc Bib}\TeX}} 
         \parbox[t]{5in}{a bibliography generation program by Oren 
                Patashnik~\cite{lamport}
                that can be used with either plain \TeX{} or \LaTeX{}.\\}

\item{\makebox[0.75in][l]{$C_1$}} Constant 1

\item{\makebox[0.75in][l]{$V$}}    Voltage 

\item{\makebox[0.75in][l]{\$}}     US Dollars
\end{description}
\end{nomenclature}


\advisorname{Bucky J. Badger}
\advisortitle{Assistant Professor}
% ABSTRACT
\begin{umiabstract}
  % !TEX root = main.tex
% !TEX encoding = Windows Latin 1
% !TEX TS-program = pdflatex
% 
% Archivo: abstract.tex (en ingles)


\chapter{Abstract} % No cambiar el titulo
\selectlanguage{english}
\noindent
Duis tristique sollicitudin leo nec consequat. Praesent et dui convallis velit tincidunt fermentum. Mauris cursus purus at sem viverra sed imperdiet sapien imperdiet. Aliquam mattis, elit eget rutrum vulputate, tortor sem pulvinar justo, sit amet mollis felis sem at nibh. Donec malesuada, neque id interdum eleifend, arcu augue porta elit, nec tristique libero metus at massa. Fusce fringilla laoreet rhoncus. Suspendisse potenti. Phasellus dignissim sodales mauris at pharetra. Donec gravida fringilla velit ac rutrum.

Curabitur ornare lectus id diam molestie eu imperdiet nulla tempus. Maecenas vestibulum enim et dui ornare blandit. Vivamus fermentum faucibus viverra. Maecenas at justo sapien. Aenean rhoncus augue mattis purus rhoncus venenatis. Suspendisse metus felis, porttitor in varius in, vulputate at tortor. Aliquam molestie, turpis et malesuada porta, tortor sapien pharetra sapien, ac rhoncus quam dolor a sapien. Pellentesque varius laoreet enim ut auctor. Nullam nec ultricies nisi. Nullam porta lectus et ante consectetur posuere.

Duis tristique sollicitudin leo nec consequat. Praesent et dui convallis velit tincidunt fermentum. Mauris cursus purus at sem viverra sed imperdiet sapien imperdiet. Aliquam mattis, elit eget rutrum vulputate, tortor sem pulvinar justo, sit amet mollis felis sem at nibh. Donec malesuada, neque id interdum eleifend, arcu augue porta elit, nec tristique libero metus at massa. Fusce fringilla laoreet rhoncus. Suspendisse potenti. Phasellus dignissim sodales mauris at pharetra. Donec gravida fringilla velit ac rutrum.

Duis tristique sollicitudin leo nec consequat. Praesent et dui convallis velit tincidunt fermentum. Mauris cursus purus at sem viverra sed imperdiet sapien imperdiet. Aliquam mattis, elit eget rutrum vulputate, tortor sem pulvinar justo, sit amet mollis felis sem at nibh. Donec malesuada, neque id interdum eleifend, arcu augue porta elit, nec tristique libero metus at massa. Fusce fringilla laoreet rhoncus. Suspendisse potenti. Phasellus dignissim sodales mauris at pharetra. Donec gravida fringilla velit ac rutrum.

Curabitur ornare lectus id diam molestie eu imperdiet nulla tempus. Maecenas vestibulum enim et dui ornare blandit. Vivamus fermentum faucibus viverra. Maecenas at justo sapien. Aenean rhoncus augue mattis purus rhoncus venenatis. Suspendisse metus felis, porttitor in varius in, vulputate at tortor. Aliquam molestie, turpis et malesuada porta, tortor sapien pharetra sapien, ac rhoncus quam dolor a sapien. Pellentesque varius laoreet enim ut auctor. Nullam nec ultricies nisi. Nullam porta lectus et ante consectetur posuere.

Duis tristique sollicitudin leo nec consequat. Praesent et dui convallis velit tincidunt fermentum. Mauris cursus purus at sem viverra sed imperdiet sapien imperdiet. Aliquam mattis, elit eget rutrum vulputate, tortor sem pulvinar justo, sit amet mollis felis sem at nibh. Donec malesuada, neque id interdum eleifend, arcu augue porta elit, nec tristique libero metus at massa. Fusce fringilla laoreet rhoncus. Suspendisse potenti. Phasellus dignissim sodales mauris at pharetra. Donec gravida fringilla velit ac rutrum.

\bigskip
\noindent
\textit{Key words:} first word; second word; third word.
% Separar palabras con punto-y-comas.

\checklanguage
% Fin archivo abstract.tex
\endinput 
\end{umiabstract}

\begin{abstract}
  % !TEX root = main.tex
% !TEX encoding = Windows Latin 1
% !TEX TS-program = pdflatex
% 
% Archivo: abstract.tex (en ingles)


\chapter{Abstract} % No cambiar el titulo
\selectlanguage{english}
\noindent
Duis tristique sollicitudin leo nec consequat. Praesent et dui convallis velit tincidunt fermentum. Mauris cursus purus at sem viverra sed imperdiet sapien imperdiet. Aliquam mattis, elit eget rutrum vulputate, tortor sem pulvinar justo, sit amet mollis felis sem at nibh. Donec malesuada, neque id interdum eleifend, arcu augue porta elit, nec tristique libero metus at massa. Fusce fringilla laoreet rhoncus. Suspendisse potenti. Phasellus dignissim sodales mauris at pharetra. Donec gravida fringilla velit ac rutrum.

Curabitur ornare lectus id diam molestie eu imperdiet nulla tempus. Maecenas vestibulum enim et dui ornare blandit. Vivamus fermentum faucibus viverra. Maecenas at justo sapien. Aenean rhoncus augue mattis purus rhoncus venenatis. Suspendisse metus felis, porttitor in varius in, vulputate at tortor. Aliquam molestie, turpis et malesuada porta, tortor sapien pharetra sapien, ac rhoncus quam dolor a sapien. Pellentesque varius laoreet enim ut auctor. Nullam nec ultricies nisi. Nullam porta lectus et ante consectetur posuere.

Duis tristique sollicitudin leo nec consequat. Praesent et dui convallis velit tincidunt fermentum. Mauris cursus purus at sem viverra sed imperdiet sapien imperdiet. Aliquam mattis, elit eget rutrum vulputate, tortor sem pulvinar justo, sit amet mollis felis sem at nibh. Donec malesuada, neque id interdum eleifend, arcu augue porta elit, nec tristique libero metus at massa. Fusce fringilla laoreet rhoncus. Suspendisse potenti. Phasellus dignissim sodales mauris at pharetra. Donec gravida fringilla velit ac rutrum.

Duis tristique sollicitudin leo nec consequat. Praesent et dui convallis velit tincidunt fermentum. Mauris cursus purus at sem viverra sed imperdiet sapien imperdiet. Aliquam mattis, elit eget rutrum vulputate, tortor sem pulvinar justo, sit amet mollis felis sem at nibh. Donec malesuada, neque id interdum eleifend, arcu augue porta elit, nec tristique libero metus at massa. Fusce fringilla laoreet rhoncus. Suspendisse potenti. Phasellus dignissim sodales mauris at pharetra. Donec gravida fringilla velit ac rutrum.

Curabitur ornare lectus id diam molestie eu imperdiet nulla tempus. Maecenas vestibulum enim et dui ornare blandit. Vivamus fermentum faucibus viverra. Maecenas at justo sapien. Aenean rhoncus augue mattis purus rhoncus venenatis. Suspendisse metus felis, porttitor in varius in, vulputate at tortor. Aliquam molestie, turpis et malesuada porta, tortor sapien pharetra sapien, ac rhoncus quam dolor a sapien. Pellentesque varius laoreet enim ut auctor. Nullam nec ultricies nisi. Nullam porta lectus et ante consectetur posuere.

Duis tristique sollicitudin leo nec consequat. Praesent et dui convallis velit tincidunt fermentum. Mauris cursus purus at sem viverra sed imperdiet sapien imperdiet. Aliquam mattis, elit eget rutrum vulputate, tortor sem pulvinar justo, sit amet mollis felis sem at nibh. Donec malesuada, neque id interdum eleifend, arcu augue porta elit, nec tristique libero metus at massa. Fusce fringilla laoreet rhoncus. Suspendisse potenti. Phasellus dignissim sodales mauris at pharetra. Donec gravida fringilla velit ac rutrum.

\bigskip
\noindent
\textit{Key words:} first word; second word; third word.
% Separar palabras con punto-y-comas.

\checklanguage
% Fin archivo abstract.tex
\endinput 
\end{abstract}


\clearpage\pagenumbering{arabic} % This makes the page numbers Arabic (1, 2, etc)
        % Title page, abstract, table of contents, etc
\intro

%
% Используемые далее команды определяются в файле common.tex.
%

% Актуальность работы
\actualitysection
\actualitytext

% Степень разработанности темы исследования
\developmentsection
\developmenttext

% Цели и задачи диссертационной работы
\objectivesection
\objectivetext

% Научная новизна
\noveltysection
\noveltytext

% Теоретическая и практическая значимость
\valuesection
\valuetext

% Методология и методы исследования
\methodssection
\methodstext

% Результаты и положения, выносимые на защиту
\resultssection
\resultstext

% Степень достоверности и апробация результатов
\approbationsection
\approbationtext

% Публикации
\pubsection
\pubtext

% Личный вклад автора
\contribsection
\contribtext

% Структура и объем диссертации
\structsection
\structtext
          % Chapter 1
% Essential LaTeX - Jon Warbrick 02/88
%   - Edited May, July 2000 -E. Benedict


% Copyright (C) Jon Warbrick and Plymouth Polytechnic 1989
% Permission is granted to reproduce the document in any way providing
% that it is distributed for free, except for any reasonable charges for
% printing, distribution, staff time, etc.  Direct commercial
% exploitation is not permitted.  Extracts may be made from this
% document providing an acknowledgment of the original source is
% maintained.

% NOTICE: This document has been edited for use in the UW-Madison
% Example Thesis file.


% counters used for the sample file example
\newcounter{savesection}
\newcounter{savesubsection}


% commands to do 'LaTeX Manual-like' examples

\newlength{\egwidth}\setlength{\egwidth}{0.42\textwidth}

\newenvironment{eg}{\begin{list}{}{\setlength{\leftmargin}%
{0.05\textwidth}\setlength{\rightmargin}{\leftmargin}}%
\item[]\footnotesize}{\end{list}}

\newenvironment{egbox}{\begin{minipage}[t]{\egwidth}}{\end{minipage}}

\newcommand{\egstart}{\begin{eg}\begin{egbox}}
\newcommand{\egmid}{\end{egbox}\hfill\begin{egbox}}
\newcommand{\egend}{\end{egbox}\end{eg}}

% one or two other commands
\newcommand{\fn}[1]{\hbox{\tt #1}}
\newcommand{\llo}[1]{(see line #1)}
\newcommand{\lls}[1]{(see lines #1)}
\newcommand{\bs}{$\backslash$}


\chapter{Essential \LaTeX{}}

This chapter introduces some key ideas behind \LaTeX{} and give you the ``essential''
items of information.  This chapter is an edited form of the paper
``Essential \LaTeX{}'' by Jon Warbrick, Plymouth Polytechnic.

\section{Introduction}
This document is an attempt to give you all the essential
information that you will need in order to use the \LaTeX{} Document
Preparation System.  Only very basic features are covered, and a
vast amount of detail has been omitted.  In a document of this size
it is not possible to include everything that you might need to know,
and if you intend to make extensive use of the program you should
refer to a more complete reference.  Attempting to produce complex
documents using only the information found below will require
much more work than it should, and will probably produce a less
than satisfactory result.

The main reference for \LaTeX{} is {\em The \LaTeX{} User's guide and
Reference Manual\/} by Leslie Lamport.  This contains most of the
information that you will ever need to know about the program, and
you will need access to a copy if you are to use \LaTeX{} seriously.
You should also consider getting a copy of {\em The \LaTeX{}
Companion\/} 

\section{How does \LaTeX{} work?}

In order to use \LaTeX{} you generate a file containing
both the text that you wish to print and instructions to tell \LaTeX{}
how you want it to appear.  You will normally create
this file using your system's text editor.  You can give the file any name you
like, but it should end ``\fn{.TEX}'' to identify the file's contents.
You then get \LaTeX{} to process the file, and it creates a
new file of typesetting commands; this has the same name as your file but
the ``\fn{.TEX}'' ending is replaced by ``\fn{.DVI}''.  This stands for
`{\it D\/}e{\it v\/}ice {\it I\/}ndependent' and, as the name implies, this file
can be used to create output on a range of printing devices.
Your {\em local guide\/} will go into more detail.

Rather than encourage you to dictate exactly how your document
should be laid out, \LaTeX{} instructions allow you describe its
{\em logical structure\/}.  For example, you can think of a quotation
embedded within your text as an element of this logical structure: you would
normally expect a quotation to be displayed in a recognisable style to set it
off from the rest of the text.
A human typesetter would recognise the quotation and handle
it accordingly, but since \LaTeX{} is only a computer program it requires
your help.  There are therefore \LaTeX{} commands that allow you to
identify quotations and as a result allow \LaTeX{} to typeset them correctly.

Fundamental to \LaTeX{} is the idea of a {\em document style\/} that
determines exactly how a document will be formatted.  \LaTeX{} provides
standard document styles that describe how standard logical structures
(such as quotations) should be formatted.  You may have to supplement
these styles by specifying the formatting of logical structures
peculiar to your document, such as mathematical formulae.  You can
also modify the standard document styles or even create an entirely
new one, though you should know the basic principles of typographical
design before creating a radically new style.

There are a number of good reasons for concentrating on the logical
structure rather than on the appearance of a document.  It prevents
you from making elementary typographical errors in the mistaken
idea that they improve the aesthetics of a document---you should
remember that the primary function of document design is to make
documents easier to read, not prettier.  It is more flexible, since
you only need to alter the definition of the quotation style
to change the appearance of all the quotations in a document.  Most
important of all, logical design encourages better writing.
A visual system makes it easier to create visual effects rather than
a coherent structure; logical design encourages you to concentrate on
your writing and makes it harder to use formatting as a substitute
for good writing.

\section{A Sample \LaTeX{} file}


\begin{figure} %---------------------------------------------------------------
{\singlespace\tt\footnotesize\begin{verbatim}
 1: % SMALL.TEX -- Released 5 July 1985
 2: % USE THIS FILE AS A MODEL FOR MAKING YOUR OWN LaTeX INPUT FILE.
 3: % EVERYTHING TO THE RIGHT OF A  %  IS A REMARK TO YOU AND IS IGNORED
 4: % BY LaTeX.
 5: %
 6: % WARNING!  DO NOT TYPE ANY OF THE FOLLOWING 10 CHARACTERS EXCEPT AS
 7: % DIRECTED:        &   $   #   %   _   {   }   ^   ~   \
 8:
 9: \documentclass[11pt,a4]{article}  % YOUR INPUT FILE MUST CONTAIN THESE
10: \begin{document}                  % TWO LINES PLUS THE \end COMMAND AT
11:                                   % THE END
12:
13: \section{Simple Text}          % THIS COMMAND MAKES A SECTION TITLE.
14:
15: Words are separated by one or    more      spaces.  Paragraphs are
16:     separated by one or more blank lines.  The output is not affected
17: by adding extra spaces or extra blank lines to the input file.
18:
19:
20: Double quotes are typed like this: ``quoted text''.
21: Single quotes are typed like this: `single-quoted text'.
22:
23: Long dashes are typed as three dash characters---like this.
24:
25: Italic text is typed like this: {\em this is italic text}.
26: Bold   text is typed like this: {\bf this is  bold  text}.
27:
28: \subsection{A Warning or Two}        % THIS MAKES A SUBSECTION TITLE.
29:
30: If you get too much space after a mid-sentence period---abbreviations
31: like etc.\ are the common culprits)---then type a backslash followed by
32: a space after the period, as in this sentence.
33:
34: Remember, don't type the 10 special characters (such as dollar sign and
35: backslash) except as directed!  The following seven are printed by
36: typing a backslash in front of them:  \$  \&  \#  \%  \_  \{  and  \}.
37: The manual tells how to make other symbols.
38:
39: \end{document}                    % THE INPUT FILE ENDS LIKE THIS
\end{verbatim}  }

\caption{A Sample \LaTeX{} File}\label{fig:sample}

\end{figure} %-----------------------------------------------------------------



Have a look at the example \LaTeX{} file in Figure~\ref{fig:sample}.  It
is a slightly modified copy of the standard \LaTeX{} example file
\fn{SMALL.TEX}.  The line numbers down the left-hand side
are not part of the file, but have been added to make it easier to
identify various portions.

Try entering this file (without the line numbers), save the text as \fn{small.tex},
next run \LaTeX{} on it, and then view the output:

{\tt \singlespace\begin{verbatim}
% latex small
% xdvi small               # displays the output on the screen
% dvips -o small.ps small  # to create a PostScript file, small.ps
% lp -d<printer> small.ps  # to print
\end{verbatim}}

\subsection{Running Text}

Most documents consist almost entirely of running text---words formed
into sentences, which are in turn formed into paragraphs---and the example file
is no exception. Describing running text poses no problems, you just type
it in naturally. In the output that it produces, \LaTeX{} will fill
lines and adjust the
spacing between words to give tidy left and right margins.
The spacing and distribution of the words in your input
file will have no effect at all on the eventual output.
Any number of spaces in your input file
are treated as a single space by \LaTeX{}, it also regards the
end of each line as a space between words \lls{15--17}.
A new paragraph is
indicated by a blank line in your input file, so don't leave
any blank lines unless you really wish to start a paragraph.

\LaTeX{} reserves a number of the less common keyboard characters for its
own use. The ten characters
\begin{quote}\begin{verbatim}
#  $  %  &  ~  _  ^  \  {  }
\end{verbatim}\end{quote}
should not appear as part of your text, because if they do
\LaTeX{} will get confused.

\subsection{\LaTeX{} Commands}

There are a number of words in the file that start `\verb|\|' \lls{9,
10 and 13}.  These are \LaTeX{} {\em commands\/} and they describe
the structure of your document. There are a number of things that you
should realize about these commands:
\begin{itemize}

\item All \LaTeX{} commands consist of a `\verb|\|' followed by one or more
characters.

\item \LaTeX{} commands should be typed using the correct mixture of upper- and
lower-case letters.  \verb|\BEGIN| is {\em not\/} the same as \verb|\begin|.

\item Some commands are placed within your text.  These are used to
switch things, like different typestyles, on and off. The \verb|\em|
command is used like this to emphasize text, normally by changing to
an {\it italic\/} typestyle \llo{25}.  The command and the text are
always enclosed between `\verb|{|' and `\verb|}|'---the `\verb|{\em|'
turns the effect on and and the `\verb|}|' turns it off.

\item There are other commands that look like
\begin{quote}\begin{verbatim}
\command{text}
\end{verbatim}\end{quote}
In this case the text is called the ``argument'' of the command.  The
\verb|\section| command is like this \llo{13}.
Sometimes you have to use curly brackets `\verb|{}|' to enclose the argument,
sometimes square brackets `\verb|[]|', and sometimes both at once.
There is method behind this apparent madness, but for the
time being you should be sure to copy the commands exactly as given.

\item When a command's name is made up entirely of letters, you must make sure
that the end of the command is marked by something that isn't a letter.
This is usually either the opening bracket around the command's argument, or
it's a space.  When it's a space, that space is always ignored by \LaTeX. We
will see later that this can sometimes be a problem.

\end{itemize}

\subsection{Overall structure}

There are some \LaTeX{} commands that must appear in every document.
The actual text of the document always starts with a
\verb|\begin{document}| command and ends with an \verb|\end{document}|
command \lls{10 and 39}.  Anything that comes after the \break
\verb|\end{document}| command is ignored.  Everything that comes
before the \break\verb|\begin{document}| command is called the
{\em preamble\/}. The preamble can only contain \LaTeX{} commands
to describe the document's style.

One command that must appear in the preamble is the
\verb|\documentclass| command \llo{9}.  This command specifies the
overall style for the document.  Our example file is a simple
technical document, and uses the {\tt article\/} class.  The document
you are reading was produced with the {\tt withesis\/} class. There
are other classes that you can use, as you will find out later on in
this document.

\subsection{Other Things to Look At}

\LaTeX{} can print both opening and closing quote characters, and can manage
either of these either single or double.  To do this it uses the two quote
characters from your keyboard: {\tt `} and {\tt '}. You will probably think of
{\tt '} as the ordinary single quote character which probably looks like
{\tt\symbol{'23}} or {\tt\symbol{'15}} on your keyboard,

and {\tt `} as a ``funny'' character that probably appears as
{\tt\symbol{'22}}. You type these characters once for single quote
\llo{21},  and twice for double quotes \llo{20}. The double quote
character {\tt "} itself is almost never used and should not be used
unless you want your text to look "funny" (compare the quote in the
previous sentence).

\LaTeX{} can produce three different kinds of dashes.
A long dash, for use as a punctuation symbol, as is typed as three dash
characters in a row, like this `\verb|---|' \llo{23}.  A shorter dash,
used between numbers as in `10--20', is typed as two dash
characters in a row, while a single dash character is used as a hyphen.

From time to time you will need to include one or more of the \LaTeX{}
special symbols in your text.  Seven of them can be printed by
making them into commands by proceeding them by backslash
\llo{36}.  The remaining three symbols can be produced by more
advanced commands, as can symbols that do not appear on your keyboard
such as \dag, \ddag, \S, \pounds, \copyright, $\sharp$ and $\clubsuit$.

It is sometimes useful to include comments in a \LaTeX{} file, to remind
you of what you have done or why you did it.  Everything to the
right of a \verb|%| sign is ignored by \LaTeX{}, and so it can
be used to introduce a comment.

\section{Document Classes and Class Options}\label{sec:styles}

There are four standard document classes available in \LaTeX:
\nobreak

\begin{description}

\item[{\tt article}]  intended for short documents and articles for publication.
Articles do not have chapters, and when \verb|\maketitle| is used to generate

a title (see Section~\ref{sec:title}) it appears at the top of the first page

rather than on a page of its own.

\item[{\tt report}] intended for longer technical documents.
It is similar to
{\tt article}, except that it contains chapters and the title appears on a page
of its own.

\item[{\tt book}] intended as a basis for book publication.  Page layout is
adjusted assuming that the output will eventually be used to print on
both sides of the paper.

\item[{\tt letter}]  intended for producing personal letters.  This style
will allow you to produce all the elements of a well laid out letter:
addresses, date, signature, etc.
\end{description}

An additional document class, the one used for this document and for
University of Wisconsin--Madison theses, is \fn{withesis}.


These standard classes can be modified by a number of {\em class
options\/}. They appear in square brackets after the
\verb|\documentclass| command. Only one class can ever be used but
you can have more than one class option, in which case their names
should be separated by commas.  The standard style options are:
\begin{description}

\item[{\tt 11pt}]  prints the document using eleven-point type for the running
 text
rather that the ten-point type normally used. Eleven-point type is about
ten percent larger than ten-point.

\item[{\tt 12pt}]  prints the document using twelve-point type for the running
 text
rather than the ten-point type normally used. Twelve-point type is about
twenty percent larger than ten-point.

\item[{\tt twoside}]  causes documents in the article or report styles to be
formatted for printing on both sides of the paper.  This is the default for the
book style.

\item[{\tt twocolumn}] produces two column on each page.

\item[{\tt titlepage}]  causes the \verb|\maketitle| command to generate a
title on a separate page for documents in the \fn{article} style.
A separate page is always used in both the \fn{report} and \fn{book} styles.

\end{description}

The University of Wisconsin--Madison thesis style, \fn{withesis} also
has some class options defined.  These class options are for
ten-point type (\fn{10pt}), tweleve-point type (\fn{12pt}), two-sided
printing (\fn{twoside}), Master Thesis margins (\fn{msthesis}) and an
option to print a small black box on lines which exceed the margins
(\fn{margincheck}).

\section{Environments}

We mentioned earlier the idea of identifying a quotation to \LaTeX{} so that
it could arrange to typeset it correctly. To do this you enclose the
quotation between the commands \verb|\begin{quotation}| and
\verb|\end{quotation}|.
This is an example of a \LaTeX{} construction called an {\em environment\/}.
A number of
special effects are obtained by putting text into particular environments.

\subsection{Quotations}

There are two environments for quotations: \fn{quote} and \fn{quotation}.
\fn{quote} is used either for a short quotation or for a sequence of
short quotations separated by blank lines:
\egstart\singlespace
\begin{verbatim}
US presidents ... remarks:
\begin{quote}
The buck stops here.

I am not a crook.
\end{quote}
\end{verbatim}
\egmid%
US presidents have been known for their pithy remarks:
\begin{quote}
The buck stops here.

I am not a crook.
\end{quote}
\egend

Use the \fn{quotation} environment for quotations that consist of more
than one paragraph.  Paragraphs in the input are separated by blank
lines as usual:
\egstart\singlespace
\begin{verbatim}

Here is some advice to remember:
\begin{quotation}
Environments for making
...other things as well.

Many problems
...environments.
\end{quotation}
\end{verbatim}
\egmid%
Here is some advice to remember:
\begin{quotation}
Environments for making quotations
can be used for other things as well.

Many problems can be solved by
novel applications of existing
environments.
\end{quotation}
\egend

\subsection{Centering and Flushing}

Text can be centered on the page by putting it within the \fn{center}
environment, and it will appear flush against the left or right margins if it
is placed within the \fn{flushleft} or \fn{flushright} environments.

Text within these environments will be formatted in the normal way, in
{\samepage
particular the ends of the lines that you type are just regarded as spaces.  To
indicate a ``newline'' you need to type the \verb|\\| command.  For example:
\egstart\singlespace
\begin{verbatim}
\begin{center}
one
two
three \\
four \\
five
\end{center}

\end{verbatim}
\egmid%
\begin{center}

one
two
three \\
four \\

five
\end{center}
\egend
}

\subsection{Lists}

There are three environments for constructing lists.  In each one each new
item is begun with an \verb|\item| command.  In the \fn{itemize} environment
the start of each item is given a marker, in the \fn{enumerate}
environment each item is marked by a number.  These environments can be nested
within each other in which case the amount of indentation used
is adjusted accordingly:
\egstart\singlespace

\begin{verbatim}
\begin{itemize}
\item Itemized lists are handy.
\item However, don't forget
  \begin{enumerate}
  \item The `item' command.
  \item The `end' command.
  \end{enumerate}
\end{itemize}
\end{verbatim}
\egmid%
\begin{itemize}
\item Itemized lists are handy.
\item However, don't forget
  \begin{enumerate}
  \item The `item' command.
  \item The `end' command.
  \end{enumerate}
\end{itemize}
\egend


The third list making environment is \fn{description}.  In a description you
specify the item labels inside square brackets after the \verb|\item| command.
For example:
\egstart\singlespace
\begin{verbatim}
Three animals that you should
know about are:
\begin{description}
  \item[gnat] A small
            animal...
  \item[gnu] A large
           animal...
  \item[armadillo] A ...
\end{description}
\end{verbatim}
\egmid%
Three animals that you should
know about are:
\begin{description}
  \item[gnat] A small animal that causes no end of trouble.
  \item[gnu] A large animal that causes no end of trouble.
  \item[armadillo] A medium-sized animal.
\end{description}
\egend

\subsection{Tables}

Because \LaTeX{} will almost always convert a sequence of spaces
into a single space, it can be rather difficult to lay out tables.
See what happens in this example
 \nolinebreak
\begin{eg}
\begin{minipage}[t]{0.55\textwidth} \singlespace
\begin{verbatim}
\begin{flushleft}
Income  Expenditure Result   \\
20s 0d  19s 11d     happiness \\
20s 0d  20s 1d      misery  \\
\end{flushleft}
\end{verbatim}
\end{minipage}
\begin{minipage}[t]{0.3\textwidth}
\begin{flushleft}
Income  Expenditure Result   \\
20s 0d  19s 11d     happiness \\
20s 0d  20s 1d      misery  \\
\end{flushleft}
\end{minipage}
\end{eg}

The \fn{tabbing} environment overcomes this problem. Within it you
set tabstops and tab to them much like you do on a typewriter.
Tabstops are set with the \verb|\=| command, and the \verb|\>|
command moves to the next stop.  The \verb|\\| command is used to
separate each line.  A line that ends \verb|\kill| produces no
output, and can be used to set tabstops:
\nolinebreak
\begin{eg}
\begin{minipage}[t]{0.6\textwidth}
\singlespace
\begin{verbatim}
\begin{tabbing}
Income \=Expenditure \=    \kill
Income \>Expenditure \>Result \\
20s 0d \>19s 11d \>Happiness \\
20s 0d \>20s 1d  \>Misery    \\
\end{tabbing}
\end{verbatim}
\end{minipage}
\vspace{1ex}
\begin{minipage}[t]{0.35\textwidth}
\begin{tabbing}
\singlespace
Income \=Expenditure \=    \kill
Income \>Expenditure \>Result \\
20s 0d \>19s 11d \>Happiness \\
20s 0d \>20s 1d  \>Misery    \\
\end{tabbing}
\end{minipage}
\end{eg}

Unlike a typewriter's tab key, the \verb|\>| command always moves to the next
tabstop in sequence, even if this means moving to the left.  This can cause
text to be overwritten if the gap between two tabstops is too small.

\subsection{Verbatim Output}

Sometimes you will want to include text exactly as it appears on a terminal
screen.  For example, you might want to include part of a computer program.
Not only do you want \LaTeX{} to stop playing around with the layout of your
text, you also want to be able to type all the characters on your keyboard
without confusing \LaTeX. The \fn{verbatim} environment has this effect:

\egstart

\begin{flushleft}\singlespace
\verb|The section of program in|  \\
 \verb|question is :|\\
 \verb|\begin{verbatim}|           \\
\verb|{ this finds %a & %b }|     \\[2ex]

\verb|for i := 1 to 27 do|        \\
\ \ \ \verb|begin|                \\
\ \ \ \verb|table[i] := fn(i);|   \\

\ \ \ \verb|process(i)|           \\
\ \ \ \verb|end;|                 \\
\verb|\end{verbatim}|
\end{flushleft}
\egmid%
The section of program in
question is:
\begin{verbatim}
{ this finds %a & %b }

for i := 1 to 27 do
   begin
   table[i] := fn(i);
   process(i)
   end;

\end{verbatim}
\egend

The \fn{withesis} document style also provides the command {\tt \verb|\verbatimfile{foo.fe}|}
which will read in the file {\tt foo.fe} into the document in \fn{verbatim} format with
the font \verb|\tt|.  See Appendix~\ref{matlab} for an example.

\section{Type Styles}

We have already come across the \verb|\em| command for changing
typeface.  Here is a full list of the available typefaces:
\begin{quote}\singlespace\begin{tabbing}
\verb|\sc|~~ \= \sc Small Caps~~~ \= \verb|\sc|~~ \= \sc Small Caps~~~
                                  \= \verb|\sc|~~ \=                   \kill
\verb|\rm|   \> \rm Roman         \> \verb|\it|   \> \it Italic
                                  \> \verb|\sc|   \> \sc Small Caps    \\
\verb|\em|   \> \em Emphatic      \> \verb|\sl| \> \sl Slanted
                                  \> \verb|\tt|   \> \tt Typewriter     \\
\verb|\bf|   \> \bf Boldface      \> \verb|\sf| \> \sf Sans Serif
\end{tabbing}\end{quote}

Remember that these commands are used {\em inside\/} a pair of braces to limit
the amount of text that they effect.  In addition to the eight typeface
commands, there are a set of commands that alter the size of the type.  These
commands are:
\begin{quotation}\singlespace\begin{tabbing}
\verb|\footnotesize|~~ \= \verb|\footnotesize|~~ \= \verb|\footnotesize| \=
 \kill
\verb|\tiny|           \> \verb|\small|          \> \verb|\large|        \>
\verb|\huge|  \\
\verb|\scriptsize|     \> \verb|\normalsize|     \> \verb|\Large|        \>
\verb|\Huge|  \\
\verb|\footnotesize|   \>                        \> \verb|\LARGE|
\end{tabbing}\end{quotation}

\section{Sectioning Commands and Tables of Contents}
\label{ess:sectioning}

Technical documents, like this one, are often divided into sections.
Each section has a heading containing a title and a number for easy
reference.  \LaTeX{} has a series of commands that will allow you to identify
different sorts of sections.  Once you have done this \LaTeX{} takes on the
responsibility of laying out the title and of providing the numbers.

The commands that you can use are:
\begin{quote}\singlespace\begin{tabbing}
\verb|\subsubsection| \= \verb|\subsubsection|~~~~~~~~~~ \=           \kill
\verb|\chapter|       \> \verb|\subsection|    \> \verb|\paragraph|    \\
\verb|\section|       \> \verb|\subsubsection| \> \verb|\subparagraph| \\
\end{tabbing}\end{quote}
The naming of these last two is unfortunate, since they do not really have
anything to do with `paragraphs' in the normal sense of the word; they are just
lower levels of section.  In most document styles, headings made with
\verb|\paragraph| and \verb|\subparagraph| are not numbered.  \verb|\chapter|
is not available in document style \fn{article}.  The commands should be used
in the order given, since sections are numbered within chapters, subsections
within sections, etc.

A seventh sectioning command, \verb|\part|, is also available.  Its use is
always optional, and it is used to divide a large document into series of
parts.  It does not alter the numbering used for any of the other commands.

Including the command \verb|\tableofcontents| in you document will cause a
contents list to be included, containing information collected from the various
sectioning commands.  You will notice that each time your document is run
through \LaTeX{} the table of contents is always made up of the headings from
the previous version of the document.  This is because \LaTeX{} collects
information for the table as it processes the document, and then includes it
the next time it is run.  This can sometimes mean that the document has to be
processed through \LaTeX{} twice to get a correct table of contents.

\section{Producing Special Symbols}

You can include in you \LaTeX{} document a wide range of symbols that do not
appear on you your keyboard. For a start, you can add an accent to any letter:
\begin{quote}\singlespace\begin{tabbing}

\t{oo} \= \verb|\t{oo}|~~~ \=
\t{oo} \= \verb|\t{oo}|~~~ \=
\t{oo} \= \verb|\t{oo}|~~~ \=
\t{oo} \= \verb|\t{oo}|~~~ \=
\t{oo} \= \verb|\t{oo}|~~~ \=
\t{oo} \=                       \kill

\a`{o} \> \verb|\`{o}|  \> \~{o}  \> \verb|\~{o}|  \> \v{o}  \> \verb|\v{o}| \>
\c{o}  \> \verb|\c{o}|  \> \a'{o} \> \verb|\'{o}|  \\
\a={o} \> \verb|\={o}|  \> \H{o}  \> \verb|\H{o}|  \> \d{o}  \> \verb|\d{o}| \>
\^{o}  \> \verb|\^{o}|  \> \.{o}  \> \verb|\.{o}|  \\
\t{oo} \> \verb|\t{oo}| \> \b{o}  \> \verb|\b{o}|  \\  \"{o} \> \verb|\"{o}| \>
\u{o}  \> \verb|\u{o}|  \\
\end{tabbing}\end{quote}

A number of other symbols are available, and can be used by including the
following commands:
\begin{quote}\singlespace\begin{tabbing}

\LaTeX~\= \verb|\copyright|~~~~ \= \LaTeX~\= \verb|\copyright|~~~~ \=
\LaTeX~\=  \kill

\dag       \> \verb|\dag|       \> \S     \> \verb|\S|     \>
\copyright \> \verb|\copyright| \\
\ddag      \> \verb|\ddag|      \> \P     \> \verb|\P|     \>
\pounds    \> \verb|\pounds|    \\
\oe        \> \verb|\oe|        \> \OE    \> \verb|\OE|    \>
\ae        \> \verb|\AE|        \\
\AE        \> \verb|\AE|        \> \aa    \> \verb|\aa|    \>
\AA        \> \verb|\AA|        \\
\o         \> \verb|\o|         \> \O     \> \verb|\O|     \>
\l         \> \verb|\l|         \\
\L         \> \verb|\E|         \> \ss    \> \verb|\ss|    \>
?`         \> \verb|?`|         \\
!`         \> \verb|!`|         \> \ldots \> \verb|\ldots| \>
\LaTeX     \> \verb|\LaTeX|     \\
\end{tabbing}\end{quote}
There is also a \verb|\today| command that prints the current date. When you
use these commands remember that \LaTeX{} will ignore any spaces that
follow them, so that you can type `\verb|\pounds 20|' to get `\pounds 20'.
However, if you type `\verb|LaTeX is wonderful|' you will get `\LaTeX is
wonderful'---notice the lack of space after \LaTeX.
To overcome this problem you can follow any of these commands by a
pair of empty brackets and then any spaces that you wish to include,
and you will see that
\verb|\LaTeX{} really is wonderful!| (\LaTeX{} really is wonderful!).

\section{Titles}\label{sec:title}

Most documents have a title.  To title a \LaTeX{} document, you include the
following commands in your document, usually just after
\verb|begin{document}|.
\begin{quote}\singlespace\begin{verbatim}
\title{required title}
\author{required author}
\date{required date}
\maketitle
\end{verbatim}\end{quote}
If there are several authors, then their names should be separated by
\verb|\and|; they can also be separated by \verb|\\| if you want them to be
centred on different lines.  If the \verb|\date| command is left out, then the
current date will be printed.
\egstart
\singlespace
\begin{verbatim}
\title{Essential \LaTeX}
\author{J Warbrick \and An Other}
\date{14th February 1988}
\maketitle
\end{verbatim}
\egmid
\begin{center}
{\normalsize Essential \LaTeX}\\[4ex]
J Warbrick\hspace{1em}A N Other\\[2ex]
14th February 1988
\end{center}
\egend

The exact appearance of the title varies depending on
the document style.  In styles \fn{report} and \fn{book} the title appears on a
page of its own. In the \fn{article} style it normally appears at the top
of the first page, the style option \fn{titlepage} will alter this (see
Section~\ref{sec:styles}).  In the \fn{withesis} style, the title is created on a
seperate page in the format appropriate to a UW-Madison thesis or dissertation.

\section{Errors}

When you create a new input file for \LaTeX{} you will probably make mistakes.
Everybody does, and it's nothing to be worried about.  As with most computer
programs, there are two sorts of mistake that you can make: those that \LaTeX{}
notices and those that it doesn't.  To take a rather silly example, since
\LaTeX{} doesn't understand what you are saying it isn't going to be worried if
you mis-spell some of the words in your text.  You will just have to accurately
proof-read your printed output.  On the other hand, if you mis-spell one of
the environment names in your file then \LaTeX won't know what you want it
to do.

When this sort of thing happens, \LaTeX{} prints an error message on your
terminal screen and then stops and waits for you to take some action.
Unfortunately, the error messages that it produces are rather user-unfriendly
and a little frightening.  Nevertheless, if you know where to look they
will probably tell you where the error is and went wrong.

Consider what would happen if you mistyped \verb|\begin{itemize}| so that it
became \break\verb|\begin{itemie}|.  When \LaTeX{} processes this instruction, it
displays the following on your terminal:
\begin{quote}\singlespace\begin{verbatim}
LaTeX error.  See LaTeX manual for explanation.
              Type  H <return>  for immediate help.
! Environment itemie undefined.
\@latexerr ...for immediate help.}\errmessage {#1}
                                                  \endgroup
l.140 \begin{itemie}

?
\end{verbatim}\end{quote}
After typing the `?' \LaTeX{} stops and waits for you to tell it what to do.

The first two lines of the message just tell you that the error was detected by
\LaTeX{}. The third line, the one that starts `!' is the {\em error indicator}.
 It
tells you what the problem is, though until you have had some experience of
\LaTeX{} this may not mean a lot to you.  In this case it is just telling you
that it doesn't recognise an environment called \fn{itemie}.
The next two lines tell you what
\LaTeX{} was doing when it found the error, they are irrelevant at the moment
and can be ignored. The final line is called the {\em error locator}, and is
a copy of the line from your file that caused the problem.
It start with a line number to help you to find it in your file, and
if the error was in the middle of a line it will be shown
broken at the point where \LaTeX{} realised that there was an error.  \LaTeX{}
can sometimes pass the point where the real error is before discovering that
something is wrong, but it doesn't usually get very far.

At this point you could do several things.  If you knew enough about \LaTeX{}
you might be able to fix the problem, or you could type `X' and press the
return key to stop \LaTeX{} running while you go and correct the error.  The
best thing to do, however, is just to press the return key.  This will allow
\LaTeX{} to go on running as if nothing had happened.  If you have made one
mistake, then you have probably made several and you may as well try to find
them all in one go.  It's much more efficient to do it this way than to run
\LaTeX{} over and over again fixing one error at a time. Don't worry about
remembering what the errors were---a copy of all the error messages is being
saved in a {\em log\/} file so that you can look at them afterwards.

If you look at the line that caused the error it's normally obvious what the
problem was.  If you can't work out what you problem is look at the hints
below, and if they don't help consult Chapter~6 of the manual~\cite{lamport}.
  It contains a
list of all of the error messages that you are likely to encounter together with
some hints as to what may have caused them.

Some of the most common mistakes that cause errors are
\begin{itemize}
\item A mispelled command or environment name.
\item Improperly matched `\verb|{|' and `\verb|}|'---remember that they should
 always
come in pairs.
\item Trying to use one of the ten special characters \verb|# $ % & _ { } ~ ^|
and \verb|\| as an ordinary printing symbol.
\item A missing \verb|\end| command.
\item A missing command argument (that's the bit enclosed in '\verb|{|' and
`\verb|}|').
\end{itemize}

One error can get \LaTeX{} so confused that it reports a series of spurious
errors as a result.  If you have an error that you understand, followed by a
series that you don't, then try correcting the first error---the rest
may vanish as if by magic.
Sometimes \LaTeX{} may write a {\tt *} and stop without an error message.  This
is normally caused by a missing \verb|\end{document}| command, but other errors
can cause it.  If this happens type \verb|\stop| and press the return key.

Finally, \LaTeX{} will sometimes print {\em warning\/} messages.  They report
problems that were not bad enough to cause \LaTeX{} to stop processing, but
nevertheless may require investigation.  The most common problems are
`overfull' and `underfull' lines of text.  A message like:
\begin{quote}\footnotesize\begin{verbatim}
Overfull \hbox (10.58649pt too wide) in paragraph at lines 172--175
[]\tenrm Mathematical for-mu-las may be dis-played. A dis-played
\end{verbatim}\end{quote}
indicates that \LaTeX{} could not find a good place to break a line when laying
out a paragraph.  As a result, it was forced to let the line stick out into the
right-hand margin, in this case by 10.6 points.  Since a point is about 1/72nd
of an inch this may be rather hard to see, but it will be there none the less.

This particular problem happens because \LaTeX{} is rather fussy about line
breaking, and it would rather generate a line that is too long than generate a
paragraph that doesn't meet its high standards.  The simplest way around the
problem is to enclose the entire offending paragraph between
\verb|\begin{sloppypar}| and \verb|\end{sloppypar}| commands.  This tells
\LaTeX{} that you are happy for it to break its own rules while it is working on
that particular bit of text.

Alternatively, messages about ``Underfull \verb|\hbox'es''| may appear.
These are lines that had to have more space inserted between
words than \LaTeX{} would have liked.  In general there is not much that you
can do about these.  Your output will look fine, even if the line looks a bit
stretched.  About the only thing you could do is to re-write the offending
paragraph!

\section{A Final Reminder}

You now know enough \LaTeX{} to produce a wide range of documents.  But this
document has only scratched the surface of the
things that \LaTeX{} can do.  This entire document was itself produced with
\LaTeX{} (with no sticking things in or clever use of a photocopier) and even
it hasn't used all the features that it could.  From this you may get some
feeling for the power that \LaTeX{} puts at your disposal.

Please remember what was said in the introduction: if you {\bf do} have a
complex document to produce then {\bf go and read the manual}.  You will be
wasting your time if you rely only on what you have read here.
     % Edited ``Essential LaTeX'' by Jon Warbrick
\chapter{Figures and Tables}\label{quad}
This chapter\footnote{Most of the text in this chapter's introduction is from {\em How to
\TeX{} a Thesis: The Purdue Thesis Styles}} shows some example ways of incorporating tables and figures into \LaTeX{}.
Special environments exist for tables and figures and are special because they are
allowed to {\em float}---that is, \LaTeX{} doesn't always put them in the exact place
that they occur in your input file.  An algorithm is used to place the floating environments,
or floats, at locations which are typographically correct.  This may cause endless frustration
if you want to have a figure or table occur at a specific location.  There are a few
methods for solving this.

You can exert some influence on \LaTeX{}'s float placement algorithm by using
{\em float position specifiers}.  These specifiers, listed below, tell \LaTeX{}
what you prefer.
\begin{tabbing}
{\tt hhhhhh} \= ``bottom'' \=  \kill
{\tt h}\> ``here'' \> do not move this object \\
{\tt p}\> ``page'' \> put this object on a page of floats \\
{\tt b}\> ``bottom'' \> put this object at the bottom of a page\\
{\tt t}\> ``top'' \> put this object at the top of a page\\
\end{tabbing}

Any combination of these can be used:
\begin{quote}\tt\singlespace\begin{verbatim}
\begin{figure}[htbp]
 ...
\caption{A Figure!}
\end{figure}
\end{verbatim}\end{quote}

In this example, we asked \LaTeX{} to ``put the figure `here' if possible.  If it
is not possible (according to the rule encoded in the float algorithm), put it on the
next float page.  A float page is a page which contains nothing but floating objects,
{\em e.g.} a page of nothing but figures or tables.  If this isn't possible, try to put it
at the `top' of a page.  The last thing to try is to put the figure at the `bottom' of
a page.''

The remainder of this chapter deals with some examples of what to put into the figure,
the ellipsis (\ldots ) in the example above.

\section{Tables}
Table~\ref{pde.tab1} is an example table from the UW Math Department.
\begin{table}[htbp]
\centering
\caption{PDE solve times, $15^3+1$
equations.\label{pde.tab1}}
\begin{tabular}{||l|l|l|l|l|l||}\hline
Precond. & Time & Nonlinear & Krylov
& Function & Precond. \\
 & & Iterations & Iterations & calls & solves \\ \hline
None & 1260.9u & 3 & 26 & 30 & 0  \\
 &(21:09) & & & &  \\ \hline
FFT  & 983.4u & 2  & 5  & 8  & 7 \\
&(16:31) & & & & \\ \hline
\end{tabular}
\end{table}
The code to generate it is as follows:
\begin{quote}\tt\singlespace\begin{verbatim}
\begin{table}[htbp]
\centering
\caption{PDE solve times, $15^3+1$
equations.\label{pde.tab1}}
\begin{tabular}{||l|l|l|l|l|l||}\hline
Precond. & Time & Nonlinear & Krylov
& Function & Precond. \\
 & & Iterations & Iterations & calls & solves \\ \hline
None & 1260.9u & 3 & 26 & 30 & 0  \\
 &(21:09) & & & &  \\ \hline
FFT  & 983.4u & 2  & 5  & 8  & 7 \\
&(16:31) & & & & \\ \hline
\end{tabular}
\end{table}
\end{verbatim}\end{quote}

\section{Figures}
There are many different ways to incorporate figures into a \LaTeX{}
document.  \LaTeX{} has an internal {\tt picture} environment and
some programs will generate files which are in this format and can
be simply {\tt include}d.  In addition to \LaTeX{} native {\tt picture}
format, additional packages can be loaded in the {\tt\verb|\documentstyle|}
command (or using the {\tt input} command) to allow \LaTeX{} to process
non-native formats such as PostScript.

\subsection{\tt gnuplot}
The graph of Figure~\ref{gelfand.fig2}
 was created by gnuplot. For simple graphs this is a
 great utility.  For example, if you want a sin curve in your thesis
 try the following:
\begin{quote}\tt\singlespace\begin{verbatim}
 (terminal window): gnuplot
 (in gnuplot):
                 set terminal latex
                 set output "foo.tex"
                 plot sin(x)
                 quit
\end{verbatim}\end{quote}
This will generate a file called {\tt foo.tex} which can be read in
with the following statements.
\begin{figure}[htbp]
\centering
% GNUPLOT: LaTeX picture
\setlength{\unitlength}{0.240900pt}
\ifx\plotpoint\undefined\newsavebox{\plotpoint}\fi
\sbox{\plotpoint}{\rule[-0.175pt]{0.350pt}{0.350pt}}%
\begin{picture}(1500,900)(0,0)
%\tenrm
\sbox{\plotpoint}{\rule[-0.175pt]{0.350pt}{0.350pt}}%
\put(264,158){\rule[-0.175pt]{282.335pt}{0.350pt}}
\put(264,158){\rule[-0.175pt]{0.350pt}{151.526pt}}
\put(264,158){\rule[-0.175pt]{4.818pt}{0.350pt}}
%\put(242,158){\makebox(0,0)[r]{0}}
\put(1416,158){\rule[-0.175pt]{4.818pt}{0.350pt}}
\put(264,284){\rule[-0.175pt]{4.818pt}{0.350pt}}
%\put(242,284){\makebox(0,0)[r]{2}}
\put(1416,284){\rule[-0.175pt]{4.818pt}{0.350pt}}
\put(264,410){\rule[-0.175pt]{4.818pt}{0.350pt}}
%\put(242,410){\makebox(0,0)[r]{4}}
\put(1416,410){\rule[-0.175pt]{4.818pt}{0.350pt}}
\put(264,535){\rule[-0.175pt]{4.818pt}{0.350pt}}
%\put(242,535){\makebox(0,0)[r]{6}}
\put(1416,535){\rule[-0.175pt]{4.818pt}{0.350pt}}
\put(264,661){\rule[-0.175pt]{4.818pt}{0.350pt}}
%\put(242,661){\makebox(0,0)[r]{8}}
\put(1416,661){\rule[-0.175pt]{4.818pt}{0.350pt}}
\put(264,787){\rule[-0.175pt]{4.818pt}{0.350pt}}
%\put(242,787){\makebox(0,0)[r]{10}}
\put(1416,787){\rule[-0.175pt]{4.818pt}{0.350pt}}
\put(264,158){\rule[-0.175pt]{0.350pt}{4.818pt}}
%\put(264,113){\makebox(0,0){0}}
\put(264,767){\rule[-0.175pt]{0.350pt}{4.818pt}}
\put(411,158){\rule[-0.175pt]{0.350pt}{4.818pt}}
%\put(411,113){\makebox(0,0){0.5}}
\put(411,767){\rule[-0.175pt]{0.350pt}{4.818pt}}
\put(557,158){\rule[-0.175pt]{0.350pt}{4.818pt}}
%\put(557,113){\makebox(0,0){1}}
\put(557,767){\rule[-0.175pt]{0.350pt}{4.818pt}}
\put(704,158){\rule[-0.175pt]{0.350pt}{4.818pt}}
%\put(704,113){\makebox(0,0){1.5}}
\put(704,767){\rule[-0.175pt]{0.350pt}{4.818pt}}
\put(850,158){\rule[-0.175pt]{0.350pt}{4.818pt}}
%\put(850,113){\makebox(0,0){2}}
\put(850,767){\rule[-0.175pt]{0.350pt}{4.818pt}}
\put(997,158){\rule[-0.175pt]{0.350pt}{4.818pt}}
%\put(997,113){\makebox(0,0){2.5}}
\put(997,767){\rule[-0.175pt]{0.350pt}{4.818pt}}
\put(1143,158){\rule[-0.175pt]{0.350pt}{4.818pt}}
%\put(1143,113){\makebox(0,0){3}}
\put(1143,767){\rule[-0.175pt]{0.350pt}{4.818pt}}
\put(1290,158){\rule[-0.175pt]{0.350pt}{4.818pt}}
%\put(1290,113){\makebox(0,0){3.5}}
\put(1290,767){\rule[-0.175pt]{0.350pt}{4.818pt}}
\put(1436,158){\rule[-0.175pt]{0.350pt}{4.818pt}}
%\put(1436,113){\makebox(0,0){4}}
\put(1436,767){\rule[-0.175pt]{0.350pt}{4.818pt}}
\put(264,158){\rule[-0.175pt]{282.335pt}{0.350pt}}
\put(1436,158){\rule[-0.175pt]{0.350pt}{151.526pt}}
\put(264,787){\rule[-0.175pt]{282.335pt}{0.350pt}}
\put(100,472){\makebox(0,0)[l]{\shortstack{$\| u\|$}}}
\put(850,68){\makebox(0,0){$\lambda$}}
%\put(850,832){\makebox(0,0){plot}}
\put(264,158){\rule[-0.175pt]{0.350pt}{151.526pt}}
%\put(1306,722){\makebox(0,0)[r]{}}
%\put(1328,722){\rule[-0.175pt]{15.899pt}{0.350pt}}
\put(264,158){\usebox{\plotpoint}}
\put(264,158){\rule[-0.175pt]{6.304pt}{0.350pt}}
\put(290,159){\rule[-0.175pt]{6.304pt}{0.350pt}}
\put(316,160){\rule[-0.175pt]{6.304pt}{0.350pt}}
\put(342,161){\rule[-0.175pt]{6.304pt}{0.350pt}}
\put(368,162){\rule[-0.175pt]{6.304pt}{0.350pt}}
\put(394,163){\rule[-0.175pt]{6.304pt}{0.350pt}}
\put(420,164){\rule[-0.175pt]{5.644pt}{0.350pt}}
\put(444,165){\rule[-0.175pt]{5.644pt}{0.350pt}}
\put(467,166){\rule[-0.175pt]{5.644pt}{0.350pt}}
\put(491,167){\rule[-0.175pt]{5.644pt}{0.350pt}}
\put(514,168){\rule[-0.175pt]{5.644pt}{0.350pt}}
\put(538,169){\rule[-0.175pt]{5.644pt}{0.350pt}}
\put(561,170){\rule[-0.175pt]{5.644pt}{0.350pt}}
\put(585,171){\rule[-0.175pt]{6.384pt}{0.350pt}}
\put(611,172){\rule[-0.175pt]{6.384pt}{0.350pt}}
\put(638,173){\rule[-0.175pt]{6.384pt}{0.350pt}}
\put(664,174){\rule[-0.175pt]{6.384pt}{0.350pt}}
\put(691,175){\rule[-0.175pt]{6.384pt}{0.350pt}}
\put(717,176){\rule[-0.175pt]{6.384pt}{0.350pt}}
\put(744,177){\rule[-0.175pt]{5.862pt}{0.350pt}}
\put(768,178){\rule[-0.175pt]{5.862pt}{0.350pt}}
\put(792,179){\rule[-0.175pt]{5.862pt}{0.350pt}}
\put(816,180){\rule[-0.175pt]{5.862pt}{0.350pt}}
\put(841,181){\rule[-0.175pt]{5.862pt}{0.350pt}}
\put(865,182){\rule[-0.175pt]{5.862pt}{0.350pt}}
\put(889,183){\rule[-0.175pt]{4.371pt}{0.350pt}}
\put(908,184){\rule[-0.175pt]{4.371pt}{0.350pt}}
\put(926,185){\rule[-0.175pt]{4.371pt}{0.350pt}}
\put(944,186){\rule[-0.175pt]{4.371pt}{0.350pt}}
\put(962,187){\rule[-0.175pt]{4.371pt}{0.350pt}}
\put(980,188){\rule[-0.175pt]{4.371pt}{0.350pt}}
\put(998,189){\rule[-0.175pt]{4.371pt}{0.350pt}}
\put(1017,190){\rule[-0.175pt]{4.216pt}{0.350pt}}
\put(1034,191){\rule[-0.175pt]{4.216pt}{0.350pt}}
\put(1052,192){\rule[-0.175pt]{4.216pt}{0.350pt}}
\put(1069,193){\rule[-0.175pt]{4.216pt}{0.350pt}}
\put(1087,194){\rule[-0.175pt]{4.216pt}{0.350pt}}
\put(1104,195){\rule[-0.175pt]{4.216pt}{0.350pt}}
\put(1122,196){\rule[-0.175pt]{3.172pt}{0.350pt}}
\put(1135,197){\rule[-0.175pt]{3.172pt}{0.350pt}}
\put(1148,198){\rule[-0.175pt]{3.172pt}{0.350pt}}
\put(1161,199){\rule[-0.175pt]{3.172pt}{0.350pt}}
\put(1174,200){\rule[-0.175pt]{3.172pt}{0.350pt}}
\put(1187,201){\rule[-0.175pt]{3.172pt}{0.350pt}}
\put(1200,202){\rule[-0.175pt]{1.893pt}{0.350pt}}
\put(1208,203){\rule[-0.175pt]{1.893pt}{0.350pt}}
\put(1216,204){\rule[-0.175pt]{1.893pt}{0.350pt}}
\put(1224,205){\rule[-0.175pt]{1.893pt}{0.350pt}}
\put(1232,206){\rule[-0.175pt]{1.893pt}{0.350pt}}
\put(1240,207){\rule[-0.175pt]{1.893pt}{0.350pt}}
\put(1248,208){\rule[-0.175pt]{1.893pt}{0.350pt}}
\put(1256,209){\rule[-0.175pt]{1.245pt}{0.350pt}}
\put(1261,210){\rule[-0.175pt]{1.245pt}{0.350pt}}
\put(1266,211){\rule[-0.175pt]{1.245pt}{0.350pt}}
\put(1271,212){\rule[-0.175pt]{1.245pt}{0.350pt}}
\put(1276,213){\rule[-0.175pt]{1.245pt}{0.350pt}}
\put(1281,214){\rule[-0.175pt]{1.245pt}{0.350pt}}
\put(1286,215){\usebox{\plotpoint}}
\put(1288,216){\usebox{\plotpoint}}
\put(1289,217){\usebox{\plotpoint}}
\put(1291,218){\usebox{\plotpoint}}
\put(1292,219){\usebox{\plotpoint}}
\put(1294,220){\usebox{\plotpoint}}
\put(1295,221){\usebox{\plotpoint}}
\put(1295,222){\rule[-0.175pt]{0.361pt}{0.350pt}}
\put(1294,223){\rule[-0.175pt]{0.361pt}{0.350pt}}
\put(1292,224){\rule[-0.175pt]{0.361pt}{0.350pt}}
\put(1291,225){\rule[-0.175pt]{0.361pt}{0.350pt}}
\put(1289,226){\rule[-0.175pt]{0.361pt}{0.350pt}}
\put(1288,227){\rule[-0.175pt]{0.361pt}{0.350pt}}
\put(1284,228){\rule[-0.175pt]{0.964pt}{0.350pt}}
\put(1280,229){\rule[-0.175pt]{0.964pt}{0.350pt}}
\put(1276,230){\rule[-0.175pt]{0.964pt}{0.350pt}}
\put(1272,231){\rule[-0.175pt]{0.964pt}{0.350pt}}
\put(1268,232){\rule[-0.175pt]{0.964pt}{0.350pt}}
\put(1264,233){\rule[-0.175pt]{0.964pt}{0.350pt}}
\put(1258,234){\rule[-0.175pt]{1.273pt}{0.350pt}}
\put(1253,235){\rule[-0.175pt]{1.273pt}{0.350pt}}
\put(1248,236){\rule[-0.175pt]{1.273pt}{0.350pt}}
\put(1242,237){\rule[-0.175pt]{1.273pt}{0.350pt}}
\put(1237,238){\rule[-0.175pt]{1.273pt}{0.350pt}}
\put(1232,239){\rule[-0.175pt]{1.273pt}{0.350pt}}
\put(1227,240){\rule[-0.175pt]{1.273pt}{0.350pt}}
\put(1219,241){\rule[-0.175pt]{1.847pt}{0.350pt}}
\put(1211,242){\rule[-0.175pt]{1.847pt}{0.350pt}}
\put(1204,243){\rule[-0.175pt]{1.847pt}{0.350pt}}
\put(1196,244){\rule[-0.175pt]{1.847pt}{0.350pt}}
\put(1188,245){\rule[-0.175pt]{1.847pt}{0.350pt}}
\put(1181,246){\rule[-0.175pt]{1.847pt}{0.350pt}}
\put(1172,247){\rule[-0.175pt]{2.128pt}{0.350pt}}
\put(1163,248){\rule[-0.175pt]{2.128pt}{0.350pt}}
\put(1154,249){\rule[-0.175pt]{2.128pt}{0.350pt}}
\put(1145,250){\rule[-0.175pt]{2.128pt}{0.350pt}}
\put(1136,251){\rule[-0.175pt]{2.128pt}{0.350pt}}
\put(1128,252){\rule[-0.175pt]{2.128pt}{0.350pt}}
\put(1120,253){\rule[-0.175pt]{1.893pt}{0.350pt}}
\put(1112,254){\rule[-0.175pt]{1.893pt}{0.350pt}}
\put(1104,255){\rule[-0.175pt]{1.893pt}{0.350pt}}
\put(1096,256){\rule[-0.175pt]{1.893pt}{0.350pt}}
\put(1088,257){\rule[-0.175pt]{1.893pt}{0.350pt}}
\put(1080,258){\rule[-0.175pt]{1.893pt}{0.350pt}}
\put(1073,259){\rule[-0.175pt]{1.893pt}{0.350pt}}
\put(1063,260){\rule[-0.175pt]{2.208pt}{0.350pt}}
\put(1054,261){\rule[-0.175pt]{2.208pt}{0.350pt}}
\put(1045,262){\rule[-0.175pt]{2.208pt}{0.350pt}}
\put(1036,263){\rule[-0.175pt]{2.208pt}{0.350pt}}
\put(1027,264){\rule[-0.175pt]{2.208pt}{0.350pt}}
\put(1018,265){\rule[-0.175pt]{2.208pt}{0.350pt}}
\put(1009,266){\rule[-0.175pt]{2.168pt}{0.350pt}}
\put(1000,267){\rule[-0.175pt]{2.168pt}{0.350pt}}
\put(991,268){\rule[-0.175pt]{2.168pt}{0.350pt}}
\put(982,269){\rule[-0.175pt]{2.168pt}{0.350pt}}
\put(973,270){\rule[-0.175pt]{2.168pt}{0.350pt}}
\put(964,271){\rule[-0.175pt]{2.168pt}{0.350pt}}
\put(957,272){\rule[-0.175pt]{1.686pt}{0.350pt}}
\put(950,273){\rule[-0.175pt]{1.686pt}{0.350pt}}
\put(943,274){\rule[-0.175pt]{1.686pt}{0.350pt}}
\put(936,275){\rule[-0.175pt]{1.686pt}{0.350pt}}
\put(929,276){\rule[-0.175pt]{1.686pt}{0.350pt}}
\put(922,277){\rule[-0.175pt]{1.686pt}{0.350pt}}
\put(915,278){\rule[-0.175pt]{1.686pt}{0.350pt}}
\put(907,279){\rule[-0.175pt]{1.767pt}{0.350pt}}
\put(900,280){\rule[-0.175pt]{1.767pt}{0.350pt}}
\put(893,281){\rule[-0.175pt]{1.767pt}{0.350pt}}
\put(885,282){\rule[-0.175pt]{1.767pt}{0.350pt}}
\put(878,283){\rule[-0.175pt]{1.767pt}{0.350pt}}
\put(871,284){\rule[-0.175pt]{1.767pt}{0.350pt}}
\put(864,285){\rule[-0.175pt]{1.486pt}{0.350pt}}
\put(858,286){\rule[-0.175pt]{1.486pt}{0.350pt}}
\put(852,287){\rule[-0.175pt]{1.486pt}{0.350pt}}
\put(846,288){\rule[-0.175pt]{1.486pt}{0.350pt}}
\put(840,289){\rule[-0.175pt]{1.486pt}{0.350pt}}
\put(834,290){\rule[-0.175pt]{1.486pt}{0.350pt}}
\put(829,291){\rule[-0.175pt]{0.998pt}{0.350pt}}
\put(825,292){\rule[-0.175pt]{0.998pt}{0.350pt}}
\put(821,293){\rule[-0.175pt]{0.998pt}{0.350pt}}
\put(817,294){\rule[-0.175pt]{0.998pt}{0.350pt}}
\put(813,295){\rule[-0.175pt]{0.998pt}{0.350pt}}
\put(809,296){\rule[-0.175pt]{0.998pt}{0.350pt}}
\put(805,297){\rule[-0.175pt]{0.998pt}{0.350pt}}
\put(801,298){\rule[-0.175pt]{0.883pt}{0.350pt}}
\put(797,299){\rule[-0.175pt]{0.883pt}{0.350pt}}
\put(793,300){\rule[-0.175pt]{0.883pt}{0.350pt}}
\put(790,301){\rule[-0.175pt]{0.883pt}{0.350pt}}
\put(786,302){\rule[-0.175pt]{0.883pt}{0.350pt}}
\put(783,303){\rule[-0.175pt]{0.883pt}{0.350pt}}
\put(780,304){\rule[-0.175pt]{0.522pt}{0.350pt}}
\put(778,305){\rule[-0.175pt]{0.522pt}{0.350pt}}
\put(776,306){\rule[-0.175pt]{0.522pt}{0.350pt}}
\put(774,307){\rule[-0.175pt]{0.522pt}{0.350pt}}
\put(772,308){\rule[-0.175pt]{0.522pt}{0.350pt}}
\put(770,309){\rule[-0.175pt]{0.522pt}{0.350pt}}
\put(770,310){\usebox{\plotpoint}}
\put(769,311){\usebox{\plotpoint}}
\put(768,312){\usebox{\plotpoint}}
\put(767,314){\usebox{\plotpoint}}
\put(766,315){\usebox{\plotpoint}}
\put(765,316){\rule[-0.175pt]{0.350pt}{0.723pt}}
\put(766,320){\rule[-0.175pt]{0.350pt}{0.723pt}}
\put(767,323){\usebox{\plotpoint}}
\put(768,324){\usebox{\plotpoint}}
\put(769,325){\usebox{\plotpoint}}
\put(771,326){\usebox{\plotpoint}}
\put(772,327){\usebox{\plotpoint}}
\put(774,328){\usebox{\plotpoint}}
\put(775,329){\usebox{\plotpoint}}
\put(777,330){\rule[-0.175pt]{0.602pt}{0.350pt}}
\put(779,331){\rule[-0.175pt]{0.602pt}{0.350pt}}
\put(782,332){\rule[-0.175pt]{0.602pt}{0.350pt}}
\put(784,333){\rule[-0.175pt]{0.602pt}{0.350pt}}
\put(787,334){\rule[-0.175pt]{0.602pt}{0.350pt}}
\put(789,335){\rule[-0.175pt]{0.602pt}{0.350pt}}
\put(792,336){\rule[-0.175pt]{0.843pt}{0.350pt}}
\put(795,337){\rule[-0.175pt]{0.843pt}{0.350pt}}
\put(799,338){\rule[-0.175pt]{0.843pt}{0.350pt}}
\put(802,339){\rule[-0.175pt]{0.843pt}{0.350pt}}
\put(806,340){\rule[-0.175pt]{0.843pt}{0.350pt}}
\put(809,341){\rule[-0.175pt]{0.843pt}{0.350pt}}
\put(813,342){\rule[-0.175pt]{0.826pt}{0.350pt}}
\put(816,343){\rule[-0.175pt]{0.826pt}{0.350pt}}
\put(819,344){\rule[-0.175pt]{0.826pt}{0.350pt}}
\put(823,345){\rule[-0.175pt]{0.826pt}{0.350pt}}
\put(826,346){\rule[-0.175pt]{0.826pt}{0.350pt}}
\put(830,347){\rule[-0.175pt]{0.826pt}{0.350pt}}
\put(833,348){\rule[-0.175pt]{0.826pt}{0.350pt}}
\put(837,349){\rule[-0.175pt]{1.084pt}{0.350pt}}
\put(841,350){\rule[-0.175pt]{1.084pt}{0.350pt}}
\put(846,351){\rule[-0.175pt]{1.084pt}{0.350pt}}
\put(850,352){\rule[-0.175pt]{1.084pt}{0.350pt}}
\put(855,353){\rule[-0.175pt]{1.084pt}{0.350pt}}
\put(859,354){\rule[-0.175pt]{1.084pt}{0.350pt}}
\put(864,355){\rule[-0.175pt]{1.164pt}{0.350pt}}
\put(868,356){\rule[-0.175pt]{1.164pt}{0.350pt}}
\put(873,357){\rule[-0.175pt]{1.164pt}{0.350pt}}
\put(878,358){\rule[-0.175pt]{1.164pt}{0.350pt}}
\put(883,359){\rule[-0.175pt]{1.164pt}{0.350pt}}
\put(888,360){\rule[-0.175pt]{1.164pt}{0.350pt}}
\put(892,361){\rule[-0.175pt]{1.032pt}{0.350pt}}
\put(897,362){\rule[-0.175pt]{1.032pt}{0.350pt}}
\put(901,363){\rule[-0.175pt]{1.032pt}{0.350pt}}
\put(905,364){\rule[-0.175pt]{1.032pt}{0.350pt}}
\put(910,365){\rule[-0.175pt]{1.032pt}{0.350pt}}
\put(914,366){\rule[-0.175pt]{1.032pt}{0.350pt}}
\put(918,367){\rule[-0.175pt]{1.032pt}{0.350pt}}
\put(922,368){\rule[-0.175pt]{1.205pt}{0.350pt}}
\put(928,369){\rule[-0.175pt]{1.204pt}{0.350pt}}
\put(933,370){\rule[-0.175pt]{1.204pt}{0.350pt}}
\put(938,371){\rule[-0.175pt]{1.204pt}{0.350pt}}
\put(943,372){\rule[-0.175pt]{1.204pt}{0.350pt}}
\put(948,373){\rule[-0.175pt]{1.204pt}{0.350pt}}
\put(953,374){\rule[-0.175pt]{1.124pt}{0.350pt}}
\put(957,375){\rule[-0.175pt]{1.124pt}{0.350pt}}
\put(962,376){\rule[-0.175pt]{1.124pt}{0.350pt}}
\put(967,377){\rule[-0.175pt]{1.124pt}{0.350pt}}
\put(971,378){\rule[-0.175pt]{1.124pt}{0.350pt}}
\put(976,379){\rule[-0.175pt]{1.124pt}{0.350pt}}
\put(981,380){\rule[-0.175pt]{0.929pt}{0.350pt}}
\put(984,381){\rule[-0.175pt]{0.929pt}{0.350pt}}
\put(988,382){\rule[-0.175pt]{0.929pt}{0.350pt}}
\put(992,383){\rule[-0.175pt]{0.929pt}{0.350pt}}
\put(996,384){\rule[-0.175pt]{0.929pt}{0.350pt}}
\put(1000,385){\rule[-0.175pt]{0.929pt}{0.350pt}}
\put(1004,386){\rule[-0.175pt]{0.929pt}{0.350pt}}
\put(1007,387){\rule[-0.175pt]{0.923pt}{0.350pt}}
\put(1011,388){\rule[-0.175pt]{0.923pt}{0.350pt}}
\put(1015,389){\rule[-0.175pt]{0.923pt}{0.350pt}}
\put(1019,390){\rule[-0.175pt]{0.923pt}{0.350pt}}
\put(1023,391){\rule[-0.175pt]{0.923pt}{0.350pt}}
\put(1027,392){\rule[-0.175pt]{0.923pt}{0.350pt}}
\put(1031,393){\rule[-0.175pt]{0.843pt}{0.350pt}}
\put(1034,394){\rule[-0.175pt]{0.843pt}{0.350pt}}
\put(1038,395){\rule[-0.175pt]{0.843pt}{0.350pt}}
\put(1041,396){\rule[-0.175pt]{0.843pt}{0.350pt}}
\put(1045,397){\rule[-0.175pt]{0.843pt}{0.350pt}}
\put(1048,398){\rule[-0.175pt]{0.843pt}{0.350pt}}
\put(1052,399){\rule[-0.175pt]{0.585pt}{0.350pt}}
\put(1054,400){\rule[-0.175pt]{0.585pt}{0.350pt}}
\put(1056,401){\rule[-0.175pt]{0.585pt}{0.350pt}}
\put(1059,402){\rule[-0.175pt]{0.585pt}{0.350pt}}
\put(1061,403){\rule[-0.175pt]{0.585pt}{0.350pt}}
\put(1064,404){\rule[-0.175pt]{0.585pt}{0.350pt}}
\put(1066,405){\rule[-0.175pt]{0.585pt}{0.350pt}}
\put(1069,406){\rule[-0.175pt]{0.522pt}{0.350pt}}
\put(1071,407){\rule[-0.175pt]{0.522pt}{0.350pt}}
\put(1073,408){\rule[-0.175pt]{0.522pt}{0.350pt}}
\put(1075,409){\rule[-0.175pt]{0.522pt}{0.350pt}}
\put(1077,410){\rule[-0.175pt]{0.522pt}{0.350pt}}
\put(1079,411){\rule[-0.175pt]{0.522pt}{0.350pt}}
\put(1081,412){\rule[-0.175pt]{0.402pt}{0.350pt}}
\put(1083,413){\rule[-0.175pt]{0.401pt}{0.350pt}}
\put(1085,414){\rule[-0.175pt]{0.401pt}{0.350pt}}
\put(1086,415){\rule[-0.175pt]{0.401pt}{0.350pt}}
\put(1088,416){\rule[-0.175pt]{0.401pt}{0.350pt}}
\put(1090,417){\rule[-0.175pt]{0.401pt}{0.350pt}}
\put(1091,418){\usebox{\plotpoint}}
\put(1092,418){\usebox{\plotpoint}}
\put(1093,419){\usebox{\plotpoint}}
\put(1094,420){\usebox{\plotpoint}}
\put(1095,422){\usebox{\plotpoint}}
\put(1096,423){\usebox{\plotpoint}}
\put(1097,424){\rule[-0.175pt]{0.350pt}{0.723pt}}
\put(1098,428){\rule[-0.175pt]{0.350pt}{0.723pt}}
\put(1099,431){\rule[-0.175pt]{0.350pt}{1.686pt}}
\put(1098,438){\usebox{\plotpoint}}
\put(1097,439){\usebox{\plotpoint}}
\put(1096,440){\usebox{\plotpoint}}
\put(1095,441){\usebox{\plotpoint}}
\put(1094,442){\usebox{\plotpoint}}
\put(1091,444){\usebox{\plotpoint}}
\put(1090,445){\usebox{\plotpoint}}
\put(1089,446){\usebox{\plotpoint}}
\put(1088,447){\usebox{\plotpoint}}
\put(1087,448){\usebox{\plotpoint}}
\put(1086,449){\usebox{\plotpoint}}
\put(1084,450){\usebox{\plotpoint}}
\put(1083,451){\usebox{\plotpoint}}
\put(1081,452){\usebox{\plotpoint}}
\put(1080,453){\usebox{\plotpoint}}
\put(1078,454){\usebox{\plotpoint}}
\put(1077,455){\usebox{\plotpoint}}
\put(1076,456){\usebox{\plotpoint}}
\put(1074,457){\rule[-0.175pt]{0.442pt}{0.350pt}}
\put(1072,458){\rule[-0.175pt]{0.442pt}{0.350pt}}
\put(1070,459){\rule[-0.175pt]{0.442pt}{0.350pt}}
\put(1068,460){\rule[-0.175pt]{0.442pt}{0.350pt}}
\put(1066,461){\rule[-0.175pt]{0.442pt}{0.350pt}}
\put(1065,462){\rule[-0.175pt]{0.442pt}{0.350pt}}
\put(1063,463){\rule[-0.175pt]{0.482pt}{0.350pt}}
\put(1061,464){\rule[-0.175pt]{0.482pt}{0.350pt}}
\put(1059,465){\rule[-0.175pt]{0.482pt}{0.350pt}}
\put(1057,466){\rule[-0.175pt]{0.482pt}{0.350pt}}
\put(1055,467){\rule[-0.175pt]{0.482pt}{0.350pt}}
\put(1053,468){\rule[-0.175pt]{0.482pt}{0.350pt}}
\put(1051,469){\rule[-0.175pt]{0.482pt}{0.350pt}}
\put(1049,470){\rule[-0.175pt]{0.482pt}{0.350pt}}
\put(1047,471){\rule[-0.175pt]{0.482pt}{0.350pt}}
\put(1045,472){\rule[-0.175pt]{0.482pt}{0.350pt}}
\put(1043,473){\rule[-0.175pt]{0.482pt}{0.350pt}}
\put(1041,474){\rule[-0.175pt]{0.482pt}{0.350pt}}
\put(1039,475){\rule[-0.175pt]{0.482pt}{0.350pt}}
\put(1036,476){\rule[-0.175pt]{0.522pt}{0.350pt}}
\put(1034,477){\rule[-0.175pt]{0.522pt}{0.350pt}}
\put(1032,478){\rule[-0.175pt]{0.522pt}{0.350pt}}
\put(1030,479){\rule[-0.175pt]{0.522pt}{0.350pt}}
\put(1028,480){\rule[-0.175pt]{0.522pt}{0.350pt}}
\put(1026,481){\rule[-0.175pt]{0.522pt}{0.350pt}}
\put(1023,482){\rule[-0.175pt]{0.522pt}{0.350pt}}
\put(1021,483){\rule[-0.175pt]{0.522pt}{0.350pt}}
\put(1019,484){\rule[-0.175pt]{0.522pt}{0.350pt}}
\put(1017,485){\rule[-0.175pt]{0.522pt}{0.350pt}}
\put(1015,486){\rule[-0.175pt]{0.522pt}{0.350pt}}
\put(1013,487){\rule[-0.175pt]{0.522pt}{0.350pt}}
\put(1011,488){\rule[-0.175pt]{0.447pt}{0.350pt}}
\put(1009,489){\rule[-0.175pt]{0.447pt}{0.350pt}}
\put(1007,490){\rule[-0.175pt]{0.447pt}{0.350pt}}
\put(1005,491){\rule[-0.175pt]{0.447pt}{0.350pt}}
\put(1003,492){\rule[-0.175pt]{0.447pt}{0.350pt}}
\put(1001,493){\rule[-0.175pt]{0.447pt}{0.350pt}}
\put(1000,494){\rule[-0.175pt]{0.447pt}{0.350pt}}
\put(998,495){\rule[-0.175pt]{0.442pt}{0.350pt}}
\put(996,496){\rule[-0.175pt]{0.442pt}{0.350pt}}
\put(994,497){\rule[-0.175pt]{0.442pt}{0.350pt}}
\put(992,498){\rule[-0.175pt]{0.442pt}{0.350pt}}
\put(990,499){\rule[-0.175pt]{0.442pt}{0.350pt}}
\put(989,500){\rule[-0.175pt]{0.442pt}{0.350pt}}
\put(987,501){\rule[-0.175pt]{0.442pt}{0.350pt}}
\put(985,502){\rule[-0.175pt]{0.442pt}{0.350pt}}
\put(983,503){\rule[-0.175pt]{0.442pt}{0.350pt}}
\put(981,504){\rule[-0.175pt]{0.442pt}{0.350pt}}
\put(979,505){\rule[-0.175pt]{0.442pt}{0.350pt}}
\put(978,506){\rule[-0.175pt]{0.442pt}{0.350pt}}
\put(976,507){\usebox{\plotpoint}}
\put(975,508){\usebox{\plotpoint}}
\put(974,509){\usebox{\plotpoint}}
\put(972,510){\usebox{\plotpoint}}
\put(971,511){\usebox{\plotpoint}}
\put(970,512){\usebox{\plotpoint}}
\put(969,513){\usebox{\plotpoint}}
\put(967,514){\usebox{\plotpoint}}
\put(966,515){\usebox{\plotpoint}}
\put(965,516){\usebox{\plotpoint}}
\put(964,517){\usebox{\plotpoint}}
\put(963,518){\usebox{\plotpoint}}
\put(962,519){\usebox{\plotpoint}}
\put(962,520){\usebox{\plotpoint}}
\put(961,521){\usebox{\plotpoint}}
\put(960,522){\usebox{\plotpoint}}
\put(959,524){\usebox{\plotpoint}}
\put(958,525){\usebox{\plotpoint}}
\put(957,527){\rule[-0.175pt]{0.350pt}{0.361pt}}
\put(956,528){\rule[-0.175pt]{0.350pt}{0.361pt}}
\put(955,530){\rule[-0.175pt]{0.350pt}{0.361pt}}
\put(954,531){\rule[-0.175pt]{0.350pt}{0.361pt}}
\put(953,533){\rule[-0.175pt]{0.350pt}{0.723pt}}
\put(952,536){\rule[-0.175pt]{0.350pt}{0.723pt}}
\put(951,539){\rule[-0.175pt]{0.350pt}{1.686pt}}
\put(950,546){\rule[-0.175pt]{0.350pt}{1.445pt}}
\put(951,552){\rule[-0.175pt]{0.350pt}{0.482pt}}
\put(952,554){\rule[-0.175pt]{0.350pt}{0.482pt}}
\put(953,556){\rule[-0.175pt]{0.350pt}{0.482pt}}
\put(954,558){\rule[-0.175pt]{0.350pt}{0.562pt}}
\put(955,560){\rule[-0.175pt]{0.350pt}{0.562pt}}
\put(956,562){\rule[-0.175pt]{0.350pt}{0.562pt}}
\put(957,564){\usebox{\plotpoint}}
\put(958,566){\usebox{\plotpoint}}
\put(959,567){\usebox{\plotpoint}}
\put(960,568){\usebox{\plotpoint}}
\put(961,569){\usebox{\plotpoint}}
\put(962,571){\usebox{\plotpoint}}
\put(963,572){\usebox{\plotpoint}}
\put(964,573){\usebox{\plotpoint}}
\put(965,574){\usebox{\plotpoint}}
\put(966,575){\usebox{\plotpoint}}
\put(967,577){\usebox{\plotpoint}}
\put(968,578){\usebox{\plotpoint}}
\put(969,579){\usebox{\plotpoint}}
\put(970,580){\usebox{\plotpoint}}
\put(971,581){\usebox{\plotpoint}}
\put(972,582){\usebox{\plotpoint}}
\put(973,584){\usebox{\plotpoint}}
\put(974,585){\usebox{\plotpoint}}
\put(975,586){\usebox{\plotpoint}}
\put(976,587){\usebox{\plotpoint}}
\put(977,588){\usebox{\plotpoint}}
\put(978,589){\usebox{\plotpoint}}
\put(979,590){\usebox{\plotpoint}}
\put(980,591){\usebox{\plotpoint}}
\put(981,592){\usebox{\plotpoint}}
\put(982,593){\usebox{\plotpoint}}
\put(983,594){\usebox{\plotpoint}}
\put(984,595){\usebox{\plotpoint}}
\put(985,596){\usebox{\plotpoint}}
\put(986,597){\usebox{\plotpoint}}
\put(987,598){\usebox{\plotpoint}}
\put(988,600){\usebox{\plotpoint}}
\put(989,601){\usebox{\plotpoint}}
\put(990,603){\usebox{\plotpoint}}
\put(991,604){\usebox{\plotpoint}}
\put(992,605){\usebox{\plotpoint}}
\put(993,606){\usebox{\plotpoint}}
\put(994,607){\usebox{\plotpoint}}
\put(995,608){\usebox{\plotpoint}}
\put(996,609){\usebox{\plotpoint}}
\put(997,610){\usebox{\plotpoint}}
\put(998,611){\usebox{\plotpoint}}
\put(999,612){\usebox{\plotpoint}}
\put(1000,613){\usebox{\plotpoint}}
\put(1001,615){\usebox{\plotpoint}}
\put(1002,616){\usebox{\plotpoint}}
\put(1003,617){\usebox{\plotpoint}}
\put(1004,619){\usebox{\plotpoint}}
\put(1005,620){\usebox{\plotpoint}}
\put(1006,622){\rule[-0.175pt]{0.350pt}{0.361pt}}
\put(1007,623){\rule[-0.175pt]{0.350pt}{0.361pt}}
\put(1008,625){\rule[-0.175pt]{0.350pt}{0.361pt}}
\put(1009,626){\rule[-0.175pt]{0.350pt}{0.361pt}}
\put(1010,628){\rule[-0.175pt]{0.350pt}{0.562pt}}
\put(1011,630){\rule[-0.175pt]{0.350pt}{0.562pt}}
\put(1012,632){\rule[-0.175pt]{0.350pt}{0.562pt}}
\put(1013,634){\rule[-0.175pt]{0.350pt}{0.723pt}}
\put(1014,638){\rule[-0.175pt]{0.350pt}{0.723pt}}
\put(1015,641){\rule[-0.175pt]{0.350pt}{1.445pt}}
\put(1016,647){\rule[-0.175pt]{0.350pt}{1.686pt}}
\put(1017,654){\rule[-0.175pt]{0.350pt}{3.734pt}}
\put(1016,669){\rule[-0.175pt]{0.350pt}{0.843pt}}
\put(1015,673){\rule[-0.175pt]{0.350pt}{1.445pt}}
\put(1014,679){\rule[-0.175pt]{0.350pt}{0.723pt}}
\put(1013,682){\rule[-0.175pt]{0.350pt}{0.723pt}}
\put(1012,685){\rule[-0.175pt]{0.350pt}{0.562pt}}
\put(1011,687){\rule[-0.175pt]{0.350pt}{0.562pt}}
\put(1010,689){\rule[-0.175pt]{0.350pt}{0.562pt}}
\put(1009,691){\rule[-0.175pt]{0.350pt}{0.723pt}}
\put(1008,695){\rule[-0.175pt]{0.350pt}{0.723pt}}
\put(1007,698){\rule[-0.175pt]{0.350pt}{0.482pt}}
\put(1006,700){\rule[-0.175pt]{0.350pt}{0.482pt}}
\put(1005,702){\rule[-0.175pt]{0.350pt}{0.482pt}}
\put(1004,704){\rule[-0.175pt]{0.350pt}{0.562pt}}
\put(1003,706){\rule[-0.175pt]{0.350pt}{0.562pt}}
\put(1002,708){\rule[-0.175pt]{0.350pt}{0.562pt}}
\put(1001,710){\rule[-0.175pt]{0.350pt}{0.723pt}}
\put(1000,714){\rule[-0.175pt]{0.350pt}{0.723pt}}
\put(999,717){\rule[-0.175pt]{0.350pt}{0.482pt}}
\put(998,719){\rule[-0.175pt]{0.350pt}{0.482pt}}
\put(997,721){\rule[-0.175pt]{0.350pt}{0.482pt}}
\put(996,723){\rule[-0.175pt]{0.350pt}{0.843pt}}
\put(995,726){\rule[-0.175pt]{0.350pt}{0.843pt}}
\put(994,730){\rule[-0.175pt]{0.350pt}{0.723pt}}
\put(993,733){\rule[-0.175pt]{0.350pt}{0.723pt}}
\put(992,736){\rule[-0.175pt]{0.350pt}{0.843pt}}
\put(991,739){\rule[-0.175pt]{0.350pt}{0.843pt}}
\put(990,743){\rule[-0.175pt]{0.350pt}{1.445pt}}
\put(989,749){\rule[-0.175pt]{0.350pt}{1.445pt}}
\put(988,755){\rule[-0.175pt]{0.350pt}{1.686pt}}
\put(987,762){\rule[-0.175pt]{0.350pt}{4.577pt}}
\put(988,781){\rule[-0.175pt]{0.350pt}{1.445pt}}
\end{picture}

\caption{Gelfand equation on the ball, $3\leq n \leq 9$.
\label{gelfand.fig2}}
\end{figure}
\begin{quote}\tt\singlespace\begin{verbatim}
\begin{figure}[htbp]
\centering
% GNUPLOT: LaTeX picture
\setlength{\unitlength}{0.240900pt}
\ifx\plotpoint\undefined\newsavebox{\plotpoint}\fi
\sbox{\plotpoint}{\rule[-0.175pt]{0.350pt}{0.350pt}}%
\begin{picture}(1500,900)(0,0)
%\tenrm
\sbox{\plotpoint}{\rule[-0.175pt]{0.350pt}{0.350pt}}%
\put(264,158){\rule[-0.175pt]{282.335pt}{0.350pt}}
\put(264,158){\rule[-0.175pt]{0.350pt}{151.526pt}}
\put(264,158){\rule[-0.175pt]{4.818pt}{0.350pt}}
%\put(242,158){\makebox(0,0)[r]{0}}
\put(1416,158){\rule[-0.175pt]{4.818pt}{0.350pt}}
\put(264,284){\rule[-0.175pt]{4.818pt}{0.350pt}}
%\put(242,284){\makebox(0,0)[r]{2}}
\put(1416,284){\rule[-0.175pt]{4.818pt}{0.350pt}}
\put(264,410){\rule[-0.175pt]{4.818pt}{0.350pt}}
%\put(242,410){\makebox(0,0)[r]{4}}
\put(1416,410){\rule[-0.175pt]{4.818pt}{0.350pt}}
\put(264,535){\rule[-0.175pt]{4.818pt}{0.350pt}}
%\put(242,535){\makebox(0,0)[r]{6}}
\put(1416,535){\rule[-0.175pt]{4.818pt}{0.350pt}}
\put(264,661){\rule[-0.175pt]{4.818pt}{0.350pt}}
%\put(242,661){\makebox(0,0)[r]{8}}
\put(1416,661){\rule[-0.175pt]{4.818pt}{0.350pt}}
\put(264,787){\rule[-0.175pt]{4.818pt}{0.350pt}}
%\put(242,787){\makebox(0,0)[r]{10}}
\put(1416,787){\rule[-0.175pt]{4.818pt}{0.350pt}}
\put(264,158){\rule[-0.175pt]{0.350pt}{4.818pt}}
%\put(264,113){\makebox(0,0){0}}
\put(264,767){\rule[-0.175pt]{0.350pt}{4.818pt}}
\put(411,158){\rule[-0.175pt]{0.350pt}{4.818pt}}
%\put(411,113){\makebox(0,0){0.5}}
\put(411,767){\rule[-0.175pt]{0.350pt}{4.818pt}}
\put(557,158){\rule[-0.175pt]{0.350pt}{4.818pt}}
%\put(557,113){\makebox(0,0){1}}
\put(557,767){\rule[-0.175pt]{0.350pt}{4.818pt}}
\put(704,158){\rule[-0.175pt]{0.350pt}{4.818pt}}
%\put(704,113){\makebox(0,0){1.5}}
\put(704,767){\rule[-0.175pt]{0.350pt}{4.818pt}}
\put(850,158){\rule[-0.175pt]{0.350pt}{4.818pt}}
%\put(850,113){\makebox(0,0){2}}
\put(850,767){\rule[-0.175pt]{0.350pt}{4.818pt}}
\put(997,158){\rule[-0.175pt]{0.350pt}{4.818pt}}
%\put(997,113){\makebox(0,0){2.5}}
\put(997,767){\rule[-0.175pt]{0.350pt}{4.818pt}}
\put(1143,158){\rule[-0.175pt]{0.350pt}{4.818pt}}
%\put(1143,113){\makebox(0,0){3}}
\put(1143,767){\rule[-0.175pt]{0.350pt}{4.818pt}}
\put(1290,158){\rule[-0.175pt]{0.350pt}{4.818pt}}
%\put(1290,113){\makebox(0,0){3.5}}
\put(1290,767){\rule[-0.175pt]{0.350pt}{4.818pt}}
\put(1436,158){\rule[-0.175pt]{0.350pt}{4.818pt}}
%\put(1436,113){\makebox(0,0){4}}
\put(1436,767){\rule[-0.175pt]{0.350pt}{4.818pt}}
\put(264,158){\rule[-0.175pt]{282.335pt}{0.350pt}}
\put(1436,158){\rule[-0.175pt]{0.350pt}{151.526pt}}
\put(264,787){\rule[-0.175pt]{282.335pt}{0.350pt}}
\put(100,472){\makebox(0,0)[l]{\shortstack{$\| u\|$}}}
\put(850,68){\makebox(0,0){$\lambda$}}
%\put(850,832){\makebox(0,0){plot}}
\put(264,158){\rule[-0.175pt]{0.350pt}{151.526pt}}
%\put(1306,722){\makebox(0,0)[r]{}}
%\put(1328,722){\rule[-0.175pt]{15.899pt}{0.350pt}}
\put(264,158){\usebox{\plotpoint}}
\put(264,158){\rule[-0.175pt]{6.304pt}{0.350pt}}
\put(290,159){\rule[-0.175pt]{6.304pt}{0.350pt}}
\put(316,160){\rule[-0.175pt]{6.304pt}{0.350pt}}
\put(342,161){\rule[-0.175pt]{6.304pt}{0.350pt}}
\put(368,162){\rule[-0.175pt]{6.304pt}{0.350pt}}
\put(394,163){\rule[-0.175pt]{6.304pt}{0.350pt}}
\put(420,164){\rule[-0.175pt]{5.644pt}{0.350pt}}
\put(444,165){\rule[-0.175pt]{5.644pt}{0.350pt}}
\put(467,166){\rule[-0.175pt]{5.644pt}{0.350pt}}
\put(491,167){\rule[-0.175pt]{5.644pt}{0.350pt}}
\put(514,168){\rule[-0.175pt]{5.644pt}{0.350pt}}
\put(538,169){\rule[-0.175pt]{5.644pt}{0.350pt}}
\put(561,170){\rule[-0.175pt]{5.644pt}{0.350pt}}
\put(585,171){\rule[-0.175pt]{6.384pt}{0.350pt}}
\put(611,172){\rule[-0.175pt]{6.384pt}{0.350pt}}
\put(638,173){\rule[-0.175pt]{6.384pt}{0.350pt}}
\put(664,174){\rule[-0.175pt]{6.384pt}{0.350pt}}
\put(691,175){\rule[-0.175pt]{6.384pt}{0.350pt}}
\put(717,176){\rule[-0.175pt]{6.384pt}{0.350pt}}
\put(744,177){\rule[-0.175pt]{5.862pt}{0.350pt}}
\put(768,178){\rule[-0.175pt]{5.862pt}{0.350pt}}
\put(792,179){\rule[-0.175pt]{5.862pt}{0.350pt}}
\put(816,180){\rule[-0.175pt]{5.862pt}{0.350pt}}
\put(841,181){\rule[-0.175pt]{5.862pt}{0.350pt}}
\put(865,182){\rule[-0.175pt]{5.862pt}{0.350pt}}
\put(889,183){\rule[-0.175pt]{4.371pt}{0.350pt}}
\put(908,184){\rule[-0.175pt]{4.371pt}{0.350pt}}
\put(926,185){\rule[-0.175pt]{4.371pt}{0.350pt}}
\put(944,186){\rule[-0.175pt]{4.371pt}{0.350pt}}
\put(962,187){\rule[-0.175pt]{4.371pt}{0.350pt}}
\put(980,188){\rule[-0.175pt]{4.371pt}{0.350pt}}
\put(998,189){\rule[-0.175pt]{4.371pt}{0.350pt}}
\put(1017,190){\rule[-0.175pt]{4.216pt}{0.350pt}}
\put(1034,191){\rule[-0.175pt]{4.216pt}{0.350pt}}
\put(1052,192){\rule[-0.175pt]{4.216pt}{0.350pt}}
\put(1069,193){\rule[-0.175pt]{4.216pt}{0.350pt}}
\put(1087,194){\rule[-0.175pt]{4.216pt}{0.350pt}}
\put(1104,195){\rule[-0.175pt]{4.216pt}{0.350pt}}
\put(1122,196){\rule[-0.175pt]{3.172pt}{0.350pt}}
\put(1135,197){\rule[-0.175pt]{3.172pt}{0.350pt}}
\put(1148,198){\rule[-0.175pt]{3.172pt}{0.350pt}}
\put(1161,199){\rule[-0.175pt]{3.172pt}{0.350pt}}
\put(1174,200){\rule[-0.175pt]{3.172pt}{0.350pt}}
\put(1187,201){\rule[-0.175pt]{3.172pt}{0.350pt}}
\put(1200,202){\rule[-0.175pt]{1.893pt}{0.350pt}}
\put(1208,203){\rule[-0.175pt]{1.893pt}{0.350pt}}
\put(1216,204){\rule[-0.175pt]{1.893pt}{0.350pt}}
\put(1224,205){\rule[-0.175pt]{1.893pt}{0.350pt}}
\put(1232,206){\rule[-0.175pt]{1.893pt}{0.350pt}}
\put(1240,207){\rule[-0.175pt]{1.893pt}{0.350pt}}
\put(1248,208){\rule[-0.175pt]{1.893pt}{0.350pt}}
\put(1256,209){\rule[-0.175pt]{1.245pt}{0.350pt}}
\put(1261,210){\rule[-0.175pt]{1.245pt}{0.350pt}}
\put(1266,211){\rule[-0.175pt]{1.245pt}{0.350pt}}
\put(1271,212){\rule[-0.175pt]{1.245pt}{0.350pt}}
\put(1276,213){\rule[-0.175pt]{1.245pt}{0.350pt}}
\put(1281,214){\rule[-0.175pt]{1.245pt}{0.350pt}}
\put(1286,215){\usebox{\plotpoint}}
\put(1288,216){\usebox{\plotpoint}}
\put(1289,217){\usebox{\plotpoint}}
\put(1291,218){\usebox{\plotpoint}}
\put(1292,219){\usebox{\plotpoint}}
\put(1294,220){\usebox{\plotpoint}}
\put(1295,221){\usebox{\plotpoint}}
\put(1295,222){\rule[-0.175pt]{0.361pt}{0.350pt}}
\put(1294,223){\rule[-0.175pt]{0.361pt}{0.350pt}}
\put(1292,224){\rule[-0.175pt]{0.361pt}{0.350pt}}
\put(1291,225){\rule[-0.175pt]{0.361pt}{0.350pt}}
\put(1289,226){\rule[-0.175pt]{0.361pt}{0.350pt}}
\put(1288,227){\rule[-0.175pt]{0.361pt}{0.350pt}}
\put(1284,228){\rule[-0.175pt]{0.964pt}{0.350pt}}
\put(1280,229){\rule[-0.175pt]{0.964pt}{0.350pt}}
\put(1276,230){\rule[-0.175pt]{0.964pt}{0.350pt}}
\put(1272,231){\rule[-0.175pt]{0.964pt}{0.350pt}}
\put(1268,232){\rule[-0.175pt]{0.964pt}{0.350pt}}
\put(1264,233){\rule[-0.175pt]{0.964pt}{0.350pt}}
\put(1258,234){\rule[-0.175pt]{1.273pt}{0.350pt}}
\put(1253,235){\rule[-0.175pt]{1.273pt}{0.350pt}}
\put(1248,236){\rule[-0.175pt]{1.273pt}{0.350pt}}
\put(1242,237){\rule[-0.175pt]{1.273pt}{0.350pt}}
\put(1237,238){\rule[-0.175pt]{1.273pt}{0.350pt}}
\put(1232,239){\rule[-0.175pt]{1.273pt}{0.350pt}}
\put(1227,240){\rule[-0.175pt]{1.273pt}{0.350pt}}
\put(1219,241){\rule[-0.175pt]{1.847pt}{0.350pt}}
\put(1211,242){\rule[-0.175pt]{1.847pt}{0.350pt}}
\put(1204,243){\rule[-0.175pt]{1.847pt}{0.350pt}}
\put(1196,244){\rule[-0.175pt]{1.847pt}{0.350pt}}
\put(1188,245){\rule[-0.175pt]{1.847pt}{0.350pt}}
\put(1181,246){\rule[-0.175pt]{1.847pt}{0.350pt}}
\put(1172,247){\rule[-0.175pt]{2.128pt}{0.350pt}}
\put(1163,248){\rule[-0.175pt]{2.128pt}{0.350pt}}
\put(1154,249){\rule[-0.175pt]{2.128pt}{0.350pt}}
\put(1145,250){\rule[-0.175pt]{2.128pt}{0.350pt}}
\put(1136,251){\rule[-0.175pt]{2.128pt}{0.350pt}}
\put(1128,252){\rule[-0.175pt]{2.128pt}{0.350pt}}
\put(1120,253){\rule[-0.175pt]{1.893pt}{0.350pt}}
\put(1112,254){\rule[-0.175pt]{1.893pt}{0.350pt}}
\put(1104,255){\rule[-0.175pt]{1.893pt}{0.350pt}}
\put(1096,256){\rule[-0.175pt]{1.893pt}{0.350pt}}
\put(1088,257){\rule[-0.175pt]{1.893pt}{0.350pt}}
\put(1080,258){\rule[-0.175pt]{1.893pt}{0.350pt}}
\put(1073,259){\rule[-0.175pt]{1.893pt}{0.350pt}}
\put(1063,260){\rule[-0.175pt]{2.208pt}{0.350pt}}
\put(1054,261){\rule[-0.175pt]{2.208pt}{0.350pt}}
\put(1045,262){\rule[-0.175pt]{2.208pt}{0.350pt}}
\put(1036,263){\rule[-0.175pt]{2.208pt}{0.350pt}}
\put(1027,264){\rule[-0.175pt]{2.208pt}{0.350pt}}
\put(1018,265){\rule[-0.175pt]{2.208pt}{0.350pt}}
\put(1009,266){\rule[-0.175pt]{2.168pt}{0.350pt}}
\put(1000,267){\rule[-0.175pt]{2.168pt}{0.350pt}}
\put(991,268){\rule[-0.175pt]{2.168pt}{0.350pt}}
\put(982,269){\rule[-0.175pt]{2.168pt}{0.350pt}}
\put(973,270){\rule[-0.175pt]{2.168pt}{0.350pt}}
\put(964,271){\rule[-0.175pt]{2.168pt}{0.350pt}}
\put(957,272){\rule[-0.175pt]{1.686pt}{0.350pt}}
\put(950,273){\rule[-0.175pt]{1.686pt}{0.350pt}}
\put(943,274){\rule[-0.175pt]{1.686pt}{0.350pt}}
\put(936,275){\rule[-0.175pt]{1.686pt}{0.350pt}}
\put(929,276){\rule[-0.175pt]{1.686pt}{0.350pt}}
\put(922,277){\rule[-0.175pt]{1.686pt}{0.350pt}}
\put(915,278){\rule[-0.175pt]{1.686pt}{0.350pt}}
\put(907,279){\rule[-0.175pt]{1.767pt}{0.350pt}}
\put(900,280){\rule[-0.175pt]{1.767pt}{0.350pt}}
\put(893,281){\rule[-0.175pt]{1.767pt}{0.350pt}}
\put(885,282){\rule[-0.175pt]{1.767pt}{0.350pt}}
\put(878,283){\rule[-0.175pt]{1.767pt}{0.350pt}}
\put(871,284){\rule[-0.175pt]{1.767pt}{0.350pt}}
\put(864,285){\rule[-0.175pt]{1.486pt}{0.350pt}}
\put(858,286){\rule[-0.175pt]{1.486pt}{0.350pt}}
\put(852,287){\rule[-0.175pt]{1.486pt}{0.350pt}}
\put(846,288){\rule[-0.175pt]{1.486pt}{0.350pt}}
\put(840,289){\rule[-0.175pt]{1.486pt}{0.350pt}}
\put(834,290){\rule[-0.175pt]{1.486pt}{0.350pt}}
\put(829,291){\rule[-0.175pt]{0.998pt}{0.350pt}}
\put(825,292){\rule[-0.175pt]{0.998pt}{0.350pt}}
\put(821,293){\rule[-0.175pt]{0.998pt}{0.350pt}}
\put(817,294){\rule[-0.175pt]{0.998pt}{0.350pt}}
\put(813,295){\rule[-0.175pt]{0.998pt}{0.350pt}}
\put(809,296){\rule[-0.175pt]{0.998pt}{0.350pt}}
\put(805,297){\rule[-0.175pt]{0.998pt}{0.350pt}}
\put(801,298){\rule[-0.175pt]{0.883pt}{0.350pt}}
\put(797,299){\rule[-0.175pt]{0.883pt}{0.350pt}}
\put(793,300){\rule[-0.175pt]{0.883pt}{0.350pt}}
\put(790,301){\rule[-0.175pt]{0.883pt}{0.350pt}}
\put(786,302){\rule[-0.175pt]{0.883pt}{0.350pt}}
\put(783,303){\rule[-0.175pt]{0.883pt}{0.350pt}}
\put(780,304){\rule[-0.175pt]{0.522pt}{0.350pt}}
\put(778,305){\rule[-0.175pt]{0.522pt}{0.350pt}}
\put(776,306){\rule[-0.175pt]{0.522pt}{0.350pt}}
\put(774,307){\rule[-0.175pt]{0.522pt}{0.350pt}}
\put(772,308){\rule[-0.175pt]{0.522pt}{0.350pt}}
\put(770,309){\rule[-0.175pt]{0.522pt}{0.350pt}}
\put(770,310){\usebox{\plotpoint}}
\put(769,311){\usebox{\plotpoint}}
\put(768,312){\usebox{\plotpoint}}
\put(767,314){\usebox{\plotpoint}}
\put(766,315){\usebox{\plotpoint}}
\put(765,316){\rule[-0.175pt]{0.350pt}{0.723pt}}
\put(766,320){\rule[-0.175pt]{0.350pt}{0.723pt}}
\put(767,323){\usebox{\plotpoint}}
\put(768,324){\usebox{\plotpoint}}
\put(769,325){\usebox{\plotpoint}}
\put(771,326){\usebox{\plotpoint}}
\put(772,327){\usebox{\plotpoint}}
\put(774,328){\usebox{\plotpoint}}
\put(775,329){\usebox{\plotpoint}}
\put(777,330){\rule[-0.175pt]{0.602pt}{0.350pt}}
\put(779,331){\rule[-0.175pt]{0.602pt}{0.350pt}}
\put(782,332){\rule[-0.175pt]{0.602pt}{0.350pt}}
\put(784,333){\rule[-0.175pt]{0.602pt}{0.350pt}}
\put(787,334){\rule[-0.175pt]{0.602pt}{0.350pt}}
\put(789,335){\rule[-0.175pt]{0.602pt}{0.350pt}}
\put(792,336){\rule[-0.175pt]{0.843pt}{0.350pt}}
\put(795,337){\rule[-0.175pt]{0.843pt}{0.350pt}}
\put(799,338){\rule[-0.175pt]{0.843pt}{0.350pt}}
\put(802,339){\rule[-0.175pt]{0.843pt}{0.350pt}}
\put(806,340){\rule[-0.175pt]{0.843pt}{0.350pt}}
\put(809,341){\rule[-0.175pt]{0.843pt}{0.350pt}}
\put(813,342){\rule[-0.175pt]{0.826pt}{0.350pt}}
\put(816,343){\rule[-0.175pt]{0.826pt}{0.350pt}}
\put(819,344){\rule[-0.175pt]{0.826pt}{0.350pt}}
\put(823,345){\rule[-0.175pt]{0.826pt}{0.350pt}}
\put(826,346){\rule[-0.175pt]{0.826pt}{0.350pt}}
\put(830,347){\rule[-0.175pt]{0.826pt}{0.350pt}}
\put(833,348){\rule[-0.175pt]{0.826pt}{0.350pt}}
\put(837,349){\rule[-0.175pt]{1.084pt}{0.350pt}}
\put(841,350){\rule[-0.175pt]{1.084pt}{0.350pt}}
\put(846,351){\rule[-0.175pt]{1.084pt}{0.350pt}}
\put(850,352){\rule[-0.175pt]{1.084pt}{0.350pt}}
\put(855,353){\rule[-0.175pt]{1.084pt}{0.350pt}}
\put(859,354){\rule[-0.175pt]{1.084pt}{0.350pt}}
\put(864,355){\rule[-0.175pt]{1.164pt}{0.350pt}}
\put(868,356){\rule[-0.175pt]{1.164pt}{0.350pt}}
\put(873,357){\rule[-0.175pt]{1.164pt}{0.350pt}}
\put(878,358){\rule[-0.175pt]{1.164pt}{0.350pt}}
\put(883,359){\rule[-0.175pt]{1.164pt}{0.350pt}}
\put(888,360){\rule[-0.175pt]{1.164pt}{0.350pt}}
\put(892,361){\rule[-0.175pt]{1.032pt}{0.350pt}}
\put(897,362){\rule[-0.175pt]{1.032pt}{0.350pt}}
\put(901,363){\rule[-0.175pt]{1.032pt}{0.350pt}}
\put(905,364){\rule[-0.175pt]{1.032pt}{0.350pt}}
\put(910,365){\rule[-0.175pt]{1.032pt}{0.350pt}}
\put(914,366){\rule[-0.175pt]{1.032pt}{0.350pt}}
\put(918,367){\rule[-0.175pt]{1.032pt}{0.350pt}}
\put(922,368){\rule[-0.175pt]{1.205pt}{0.350pt}}
\put(928,369){\rule[-0.175pt]{1.204pt}{0.350pt}}
\put(933,370){\rule[-0.175pt]{1.204pt}{0.350pt}}
\put(938,371){\rule[-0.175pt]{1.204pt}{0.350pt}}
\put(943,372){\rule[-0.175pt]{1.204pt}{0.350pt}}
\put(948,373){\rule[-0.175pt]{1.204pt}{0.350pt}}
\put(953,374){\rule[-0.175pt]{1.124pt}{0.350pt}}
\put(957,375){\rule[-0.175pt]{1.124pt}{0.350pt}}
\put(962,376){\rule[-0.175pt]{1.124pt}{0.350pt}}
\put(967,377){\rule[-0.175pt]{1.124pt}{0.350pt}}
\put(971,378){\rule[-0.175pt]{1.124pt}{0.350pt}}
\put(976,379){\rule[-0.175pt]{1.124pt}{0.350pt}}
\put(981,380){\rule[-0.175pt]{0.929pt}{0.350pt}}
\put(984,381){\rule[-0.175pt]{0.929pt}{0.350pt}}
\put(988,382){\rule[-0.175pt]{0.929pt}{0.350pt}}
\put(992,383){\rule[-0.175pt]{0.929pt}{0.350pt}}
\put(996,384){\rule[-0.175pt]{0.929pt}{0.350pt}}
\put(1000,385){\rule[-0.175pt]{0.929pt}{0.350pt}}
\put(1004,386){\rule[-0.175pt]{0.929pt}{0.350pt}}
\put(1007,387){\rule[-0.175pt]{0.923pt}{0.350pt}}
\put(1011,388){\rule[-0.175pt]{0.923pt}{0.350pt}}
\put(1015,389){\rule[-0.175pt]{0.923pt}{0.350pt}}
\put(1019,390){\rule[-0.175pt]{0.923pt}{0.350pt}}
\put(1023,391){\rule[-0.175pt]{0.923pt}{0.350pt}}
\put(1027,392){\rule[-0.175pt]{0.923pt}{0.350pt}}
\put(1031,393){\rule[-0.175pt]{0.843pt}{0.350pt}}
\put(1034,394){\rule[-0.175pt]{0.843pt}{0.350pt}}
\put(1038,395){\rule[-0.175pt]{0.843pt}{0.350pt}}
\put(1041,396){\rule[-0.175pt]{0.843pt}{0.350pt}}
\put(1045,397){\rule[-0.175pt]{0.843pt}{0.350pt}}
\put(1048,398){\rule[-0.175pt]{0.843pt}{0.350pt}}
\put(1052,399){\rule[-0.175pt]{0.585pt}{0.350pt}}
\put(1054,400){\rule[-0.175pt]{0.585pt}{0.350pt}}
\put(1056,401){\rule[-0.175pt]{0.585pt}{0.350pt}}
\put(1059,402){\rule[-0.175pt]{0.585pt}{0.350pt}}
\put(1061,403){\rule[-0.175pt]{0.585pt}{0.350pt}}
\put(1064,404){\rule[-0.175pt]{0.585pt}{0.350pt}}
\put(1066,405){\rule[-0.175pt]{0.585pt}{0.350pt}}
\put(1069,406){\rule[-0.175pt]{0.522pt}{0.350pt}}
\put(1071,407){\rule[-0.175pt]{0.522pt}{0.350pt}}
\put(1073,408){\rule[-0.175pt]{0.522pt}{0.350pt}}
\put(1075,409){\rule[-0.175pt]{0.522pt}{0.350pt}}
\put(1077,410){\rule[-0.175pt]{0.522pt}{0.350pt}}
\put(1079,411){\rule[-0.175pt]{0.522pt}{0.350pt}}
\put(1081,412){\rule[-0.175pt]{0.402pt}{0.350pt}}
\put(1083,413){\rule[-0.175pt]{0.401pt}{0.350pt}}
\put(1085,414){\rule[-0.175pt]{0.401pt}{0.350pt}}
\put(1086,415){\rule[-0.175pt]{0.401pt}{0.350pt}}
\put(1088,416){\rule[-0.175pt]{0.401pt}{0.350pt}}
\put(1090,417){\rule[-0.175pt]{0.401pt}{0.350pt}}
\put(1091,418){\usebox{\plotpoint}}
\put(1092,418){\usebox{\plotpoint}}
\put(1093,419){\usebox{\plotpoint}}
\put(1094,420){\usebox{\plotpoint}}
\put(1095,422){\usebox{\plotpoint}}
\put(1096,423){\usebox{\plotpoint}}
\put(1097,424){\rule[-0.175pt]{0.350pt}{0.723pt}}
\put(1098,428){\rule[-0.175pt]{0.350pt}{0.723pt}}
\put(1099,431){\rule[-0.175pt]{0.350pt}{1.686pt}}
\put(1098,438){\usebox{\plotpoint}}
\put(1097,439){\usebox{\plotpoint}}
\put(1096,440){\usebox{\plotpoint}}
\put(1095,441){\usebox{\plotpoint}}
\put(1094,442){\usebox{\plotpoint}}
\put(1091,444){\usebox{\plotpoint}}
\put(1090,445){\usebox{\plotpoint}}
\put(1089,446){\usebox{\plotpoint}}
\put(1088,447){\usebox{\plotpoint}}
\put(1087,448){\usebox{\plotpoint}}
\put(1086,449){\usebox{\plotpoint}}
\put(1084,450){\usebox{\plotpoint}}
\put(1083,451){\usebox{\plotpoint}}
\put(1081,452){\usebox{\plotpoint}}
\put(1080,453){\usebox{\plotpoint}}
\put(1078,454){\usebox{\plotpoint}}
\put(1077,455){\usebox{\plotpoint}}
\put(1076,456){\usebox{\plotpoint}}
\put(1074,457){\rule[-0.175pt]{0.442pt}{0.350pt}}
\put(1072,458){\rule[-0.175pt]{0.442pt}{0.350pt}}
\put(1070,459){\rule[-0.175pt]{0.442pt}{0.350pt}}
\put(1068,460){\rule[-0.175pt]{0.442pt}{0.350pt}}
\put(1066,461){\rule[-0.175pt]{0.442pt}{0.350pt}}
\put(1065,462){\rule[-0.175pt]{0.442pt}{0.350pt}}
\put(1063,463){\rule[-0.175pt]{0.482pt}{0.350pt}}
\put(1061,464){\rule[-0.175pt]{0.482pt}{0.350pt}}
\put(1059,465){\rule[-0.175pt]{0.482pt}{0.350pt}}
\put(1057,466){\rule[-0.175pt]{0.482pt}{0.350pt}}
\put(1055,467){\rule[-0.175pt]{0.482pt}{0.350pt}}
\put(1053,468){\rule[-0.175pt]{0.482pt}{0.350pt}}
\put(1051,469){\rule[-0.175pt]{0.482pt}{0.350pt}}
\put(1049,470){\rule[-0.175pt]{0.482pt}{0.350pt}}
\put(1047,471){\rule[-0.175pt]{0.482pt}{0.350pt}}
\put(1045,472){\rule[-0.175pt]{0.482pt}{0.350pt}}
\put(1043,473){\rule[-0.175pt]{0.482pt}{0.350pt}}
\put(1041,474){\rule[-0.175pt]{0.482pt}{0.350pt}}
\put(1039,475){\rule[-0.175pt]{0.482pt}{0.350pt}}
\put(1036,476){\rule[-0.175pt]{0.522pt}{0.350pt}}
\put(1034,477){\rule[-0.175pt]{0.522pt}{0.350pt}}
\put(1032,478){\rule[-0.175pt]{0.522pt}{0.350pt}}
\put(1030,479){\rule[-0.175pt]{0.522pt}{0.350pt}}
\put(1028,480){\rule[-0.175pt]{0.522pt}{0.350pt}}
\put(1026,481){\rule[-0.175pt]{0.522pt}{0.350pt}}
\put(1023,482){\rule[-0.175pt]{0.522pt}{0.350pt}}
\put(1021,483){\rule[-0.175pt]{0.522pt}{0.350pt}}
\put(1019,484){\rule[-0.175pt]{0.522pt}{0.350pt}}
\put(1017,485){\rule[-0.175pt]{0.522pt}{0.350pt}}
\put(1015,486){\rule[-0.175pt]{0.522pt}{0.350pt}}
\put(1013,487){\rule[-0.175pt]{0.522pt}{0.350pt}}
\put(1011,488){\rule[-0.175pt]{0.447pt}{0.350pt}}
\put(1009,489){\rule[-0.175pt]{0.447pt}{0.350pt}}
\put(1007,490){\rule[-0.175pt]{0.447pt}{0.350pt}}
\put(1005,491){\rule[-0.175pt]{0.447pt}{0.350pt}}
\put(1003,492){\rule[-0.175pt]{0.447pt}{0.350pt}}
\put(1001,493){\rule[-0.175pt]{0.447pt}{0.350pt}}
\put(1000,494){\rule[-0.175pt]{0.447pt}{0.350pt}}
\put(998,495){\rule[-0.175pt]{0.442pt}{0.350pt}}
\put(996,496){\rule[-0.175pt]{0.442pt}{0.350pt}}
\put(994,497){\rule[-0.175pt]{0.442pt}{0.350pt}}
\put(992,498){\rule[-0.175pt]{0.442pt}{0.350pt}}
\put(990,499){\rule[-0.175pt]{0.442pt}{0.350pt}}
\put(989,500){\rule[-0.175pt]{0.442pt}{0.350pt}}
\put(987,501){\rule[-0.175pt]{0.442pt}{0.350pt}}
\put(985,502){\rule[-0.175pt]{0.442pt}{0.350pt}}
\put(983,503){\rule[-0.175pt]{0.442pt}{0.350pt}}
\put(981,504){\rule[-0.175pt]{0.442pt}{0.350pt}}
\put(979,505){\rule[-0.175pt]{0.442pt}{0.350pt}}
\put(978,506){\rule[-0.175pt]{0.442pt}{0.350pt}}
\put(976,507){\usebox{\plotpoint}}
\put(975,508){\usebox{\plotpoint}}
\put(974,509){\usebox{\plotpoint}}
\put(972,510){\usebox{\plotpoint}}
\put(971,511){\usebox{\plotpoint}}
\put(970,512){\usebox{\plotpoint}}
\put(969,513){\usebox{\plotpoint}}
\put(967,514){\usebox{\plotpoint}}
\put(966,515){\usebox{\plotpoint}}
\put(965,516){\usebox{\plotpoint}}
\put(964,517){\usebox{\plotpoint}}
\put(963,518){\usebox{\plotpoint}}
\put(962,519){\usebox{\plotpoint}}
\put(962,520){\usebox{\plotpoint}}
\put(961,521){\usebox{\plotpoint}}
\put(960,522){\usebox{\plotpoint}}
\put(959,524){\usebox{\plotpoint}}
\put(958,525){\usebox{\plotpoint}}
\put(957,527){\rule[-0.175pt]{0.350pt}{0.361pt}}
\put(956,528){\rule[-0.175pt]{0.350pt}{0.361pt}}
\put(955,530){\rule[-0.175pt]{0.350pt}{0.361pt}}
\put(954,531){\rule[-0.175pt]{0.350pt}{0.361pt}}
\put(953,533){\rule[-0.175pt]{0.350pt}{0.723pt}}
\put(952,536){\rule[-0.175pt]{0.350pt}{0.723pt}}
\put(951,539){\rule[-0.175pt]{0.350pt}{1.686pt}}
\put(950,546){\rule[-0.175pt]{0.350pt}{1.445pt}}
\put(951,552){\rule[-0.175pt]{0.350pt}{0.482pt}}
\put(952,554){\rule[-0.175pt]{0.350pt}{0.482pt}}
\put(953,556){\rule[-0.175pt]{0.350pt}{0.482pt}}
\put(954,558){\rule[-0.175pt]{0.350pt}{0.562pt}}
\put(955,560){\rule[-0.175pt]{0.350pt}{0.562pt}}
\put(956,562){\rule[-0.175pt]{0.350pt}{0.562pt}}
\put(957,564){\usebox{\plotpoint}}
\put(958,566){\usebox{\plotpoint}}
\put(959,567){\usebox{\plotpoint}}
\put(960,568){\usebox{\plotpoint}}
\put(961,569){\usebox{\plotpoint}}
\put(962,571){\usebox{\plotpoint}}
\put(963,572){\usebox{\plotpoint}}
\put(964,573){\usebox{\plotpoint}}
\put(965,574){\usebox{\plotpoint}}
\put(966,575){\usebox{\plotpoint}}
\put(967,577){\usebox{\plotpoint}}
\put(968,578){\usebox{\plotpoint}}
\put(969,579){\usebox{\plotpoint}}
\put(970,580){\usebox{\plotpoint}}
\put(971,581){\usebox{\plotpoint}}
\put(972,582){\usebox{\plotpoint}}
\put(973,584){\usebox{\plotpoint}}
\put(974,585){\usebox{\plotpoint}}
\put(975,586){\usebox{\plotpoint}}
\put(976,587){\usebox{\plotpoint}}
\put(977,588){\usebox{\plotpoint}}
\put(978,589){\usebox{\plotpoint}}
\put(979,590){\usebox{\plotpoint}}
\put(980,591){\usebox{\plotpoint}}
\put(981,592){\usebox{\plotpoint}}
\put(982,593){\usebox{\plotpoint}}
\put(983,594){\usebox{\plotpoint}}
\put(984,595){\usebox{\plotpoint}}
\put(985,596){\usebox{\plotpoint}}
\put(986,597){\usebox{\plotpoint}}
\put(987,598){\usebox{\plotpoint}}
\put(988,600){\usebox{\plotpoint}}
\put(989,601){\usebox{\plotpoint}}
\put(990,603){\usebox{\plotpoint}}
\put(991,604){\usebox{\plotpoint}}
\put(992,605){\usebox{\plotpoint}}
\put(993,606){\usebox{\plotpoint}}
\put(994,607){\usebox{\plotpoint}}
\put(995,608){\usebox{\plotpoint}}
\put(996,609){\usebox{\plotpoint}}
\put(997,610){\usebox{\plotpoint}}
\put(998,611){\usebox{\plotpoint}}
\put(999,612){\usebox{\plotpoint}}
\put(1000,613){\usebox{\plotpoint}}
\put(1001,615){\usebox{\plotpoint}}
\put(1002,616){\usebox{\plotpoint}}
\put(1003,617){\usebox{\plotpoint}}
\put(1004,619){\usebox{\plotpoint}}
\put(1005,620){\usebox{\plotpoint}}
\put(1006,622){\rule[-0.175pt]{0.350pt}{0.361pt}}
\put(1007,623){\rule[-0.175pt]{0.350pt}{0.361pt}}
\put(1008,625){\rule[-0.175pt]{0.350pt}{0.361pt}}
\put(1009,626){\rule[-0.175pt]{0.350pt}{0.361pt}}
\put(1010,628){\rule[-0.175pt]{0.350pt}{0.562pt}}
\put(1011,630){\rule[-0.175pt]{0.350pt}{0.562pt}}
\put(1012,632){\rule[-0.175pt]{0.350pt}{0.562pt}}
\put(1013,634){\rule[-0.175pt]{0.350pt}{0.723pt}}
\put(1014,638){\rule[-0.175pt]{0.350pt}{0.723pt}}
\put(1015,641){\rule[-0.175pt]{0.350pt}{1.445pt}}
\put(1016,647){\rule[-0.175pt]{0.350pt}{1.686pt}}
\put(1017,654){\rule[-0.175pt]{0.350pt}{3.734pt}}
\put(1016,669){\rule[-0.175pt]{0.350pt}{0.843pt}}
\put(1015,673){\rule[-0.175pt]{0.350pt}{1.445pt}}
\put(1014,679){\rule[-0.175pt]{0.350pt}{0.723pt}}
\put(1013,682){\rule[-0.175pt]{0.350pt}{0.723pt}}
\put(1012,685){\rule[-0.175pt]{0.350pt}{0.562pt}}
\put(1011,687){\rule[-0.175pt]{0.350pt}{0.562pt}}
\put(1010,689){\rule[-0.175pt]{0.350pt}{0.562pt}}
\put(1009,691){\rule[-0.175pt]{0.350pt}{0.723pt}}
\put(1008,695){\rule[-0.175pt]{0.350pt}{0.723pt}}
\put(1007,698){\rule[-0.175pt]{0.350pt}{0.482pt}}
\put(1006,700){\rule[-0.175pt]{0.350pt}{0.482pt}}
\put(1005,702){\rule[-0.175pt]{0.350pt}{0.482pt}}
\put(1004,704){\rule[-0.175pt]{0.350pt}{0.562pt}}
\put(1003,706){\rule[-0.175pt]{0.350pt}{0.562pt}}
\put(1002,708){\rule[-0.175pt]{0.350pt}{0.562pt}}
\put(1001,710){\rule[-0.175pt]{0.350pt}{0.723pt}}
\put(1000,714){\rule[-0.175pt]{0.350pt}{0.723pt}}
\put(999,717){\rule[-0.175pt]{0.350pt}{0.482pt}}
\put(998,719){\rule[-0.175pt]{0.350pt}{0.482pt}}
\put(997,721){\rule[-0.175pt]{0.350pt}{0.482pt}}
\put(996,723){\rule[-0.175pt]{0.350pt}{0.843pt}}
\put(995,726){\rule[-0.175pt]{0.350pt}{0.843pt}}
\put(994,730){\rule[-0.175pt]{0.350pt}{0.723pt}}
\put(993,733){\rule[-0.175pt]{0.350pt}{0.723pt}}
\put(992,736){\rule[-0.175pt]{0.350pt}{0.843pt}}
\put(991,739){\rule[-0.175pt]{0.350pt}{0.843pt}}
\put(990,743){\rule[-0.175pt]{0.350pt}{1.445pt}}
\put(989,749){\rule[-0.175pt]{0.350pt}{1.445pt}}
\put(988,755){\rule[-0.175pt]{0.350pt}{1.686pt}}
\put(987,762){\rule[-0.175pt]{0.350pt}{4.577pt}}
\put(988,781){\rule[-0.175pt]{0.350pt}{1.445pt}}
\end{picture}

\caption{Gelfand equation on the ball, $3\leq n \leq 9$.
\label{gelfand.fig2}}
\end{figure}
\end{verbatim}\end{quote}
One advantage to using the native \LaTeX{} {\tt picture} environment
is that the fonts will be assured to agree and the pictures can be viewed
in the {\tt .dvi} viewer.

\subsection{PostScript}
Many drawing applications now allow the export of a graphic to the
{\em Encapsulated PostScript} format.  These files have a suffix of
{\tt .EPS} or {\tt .EPSF} and are similar to a regular PostScript
file except that they contain a {\em bounding box} which describes
the dimensions of the figure.

In order to include PostScript figures, the {\tt epsfig} (or {\tt psfig}
depending on the system you are using) style file must be included in either
the {\tt\verb|\documentstyle|} command or the preamble using the {\tt input} command.

Figure~\ref{vwcontr} is a plot from Matlab.
\begin{figure}[htbp]
\centerline{
\psfig{figure=vwcontr.eps,width=5in,angle=0}
           }
\caption{$\sigma$ as a Function of Voltage and Speed, $\alpha = 20$}
\label{vwcontr}
\end{figure}
The commands to include this figure are
\begin{quote}\tt\singlespace\begin{verbatim}
\begin{figure}[htbp]
\centerline{
\psfig{figure=vwcontr.ps,width=5in,angle=0}
           }
\caption{$\sigma$ as a Function of Voltage and Speed, $\alpha = 20$}
\label{vwcontr}
\end{figure}
\end{verbatim}\end{quote}

Observe that the {\tt \verb|\psfig|} command allows the scaling of the figure
by setting either the {\tt width} or {\tt height} of the figure.  If only one
dimension is specified, the other is computed to keep the same aspect ratio.
The figure can also be rotated by setting {\tt angle} to the desired value in
degrees.
           % Chapter 3 Edited from UW Math Dept's Sample Thesis
% bibs.tex
%
% This chapter briefly talks about BibTex and is mostly
% copied from a similar chapter from "How to TeX a Thesis:
% The Purdue Thesis Styles" by James Darrell McCauley and
% Scott Hucker
%

\newcommand{\BibTeX}{{\sc Bib}\TeX}

\chapter{Citations and Bibliographies}
This chapter is an edited form of the same chapter from {\em How to 
\TeX{} a Thesis: The Purdue Thesis Styles} by James Darrell McCauley and
Scott Hucker.

The task of compiling and formatting the sources cited in papers can
be quite tedious, especially for large documents like theses.  A program
separate from \LaTeX{}, called ``\BibTeX{},''can be used to automate this task~\cite{lamport}.

\section{The Citation Command}
When referring to the work of someone else, the {\tt \verb|\cite|} command is used.
This generates the citation in the text for you.  In the above paragraph, the command
{\tt \verb|\cite{lamport}|} was used after the word ``task.''  The formatting of your
citation is handled by either the document style or a style option.  The default citation
style uses the number system (a number in square brackets).  Other citation styles
may use the author-date system, (Lamport, 1986) or the superscript$^3$ system.

\section{Bibliography Styles}
The way that a reference is formatted in your bibliography depends on the bibliography
style, which is specified near the beginning of your document with the\break
{\tt \verb|\bibliographstyle{file}|} command.  The file {\tt file.bst} is the name of the 
bibliography style file.  Standard \BibTeX{} bibliography style files include {\tt plain},
{\tt unsrt}, {\tt alpha}, and {\tt abbrev}.  The bibliography style governs whether or not
references are sorted, whether first names or initials are used for authors, whether or 
not last names are listed first, the location of the year in the references (after the
author or at the end of the reference), {\em etc.}.  You may be required by your
department or major professor to follow as style for a particular journal.  If so, then you
will need to find a \BibTeX{} style file to suit your needs.  Most major journals have
style files.  If you cannot locate an appropriate \BibTeX{} style file, then choose the
one which is closest and then edit the {\tt .bbl} file by hand.  See Section~\ref{BBL}
for a brief discussion on the {\tt .bbl} file.  Some common, but non-standard \BibTeX{}
styles include
\begin{tabbing}
{\tt jacs-new.bstxxxx}\= {\em Journal of the American Chemical Society}\kill
{\tt acm.bst}\>The Association for Computing Machinery\\
{\tt ieeetr.bst}\> The {\em IEEE Transactions} style\\
{\tt jacs-new.bst}\> {\em Journal of the American Chemical Society}
\end{tabbing}

\section{The Database}
The  {\tt \verb|\bibliography{file}|} command is placed in your input file at the location
where the ``List of References'' section\footnote{or ``Bibliography'' 
if {\tt \char92 altbibtitle } has been specified in the preamble.} would be.  It specifies the name (or names) of
your bibliographic data base, {\tt file.bib}.  An example entry in a \BibTeX{}
database is:
\begin{quote}\singlespace\tt\begin{verbatim}
@book{ lamport86 ,
     author =    "Leslie Lamport" ,
     title =     "\LaTeX: A Document Preparation System" ,
     publisher = "Addison--Wesley Pub.\ Co." ,
     year =      "1986" ,
     address =   "Reading, MA" 
}
\end{verbatim}\end{quote}

The citation key is the first field in this entry--- citing this book in a \LaTeX{}
file would look like
\begin{quote}\singlespace\tt\begin{verbatim}
According to Lamport~\cite{lamport86} ...
\end{verbatim}\end{quote}
The tilde ({\tt \verb|~|}) is used to tie the word ``Lamport'' to the citation
generated.  The space between these words is then unbreakable---the word ``Lamport''
and the citation \cite{lamport} will not be split across two lines if they happen to occur
near the end of a line.

A listing of all entry types with their required and optional fields is given in 
Appendix~\ref{bibrefs}. There are several tools which exist to help in editing a \BibTeX{}
file, however, their use is beyond the scope of this manual and can be found by searching
the net.  You can simply use a plain text editor like {\tt vi} or {\tt WordPad} to edit
and create the database files.

There are several rules which you must follow when creating your database.  Authors are
always listed by their full names, first name first, and multiple authors are separated
by {\tt and}.  For example
\begin{quote}\singlespace\tt\begin{verbatim}
author = "John Jay Park and Frederick Gene Watson and
          Michelle Catherine Smith",
\end{verbatim}\end{quote}
If you were using {\tt abbrv} as your {\tt bibliographystyle}, a reference for these
authors may look like:
\begin{quote}
J.J. Park, F.G. Watson, and M.C. Smith \ldots
\end{quote}

Some styles only capitalize the first word of the title.  If you use any acronyms or
other words that should always be capitalized in titles, then they should be 
enclosed in {\tt \{\}}'s ({\em e.g.}, {\tt \{Fortran\}}, {\tt \{N\}ewton}).
This protects the case of these characters.

There are several other rules for \BibTeX{} listed in~\cite{lamport} which should be
referred to because they are not discussed here.

\section{Putting It All Together}
\label{BBL}
To aid the reader in understanding how all of this works together, the following 
excerpt was taken from Lamport~\cite{lamport}:
\begin{quotation}\singlespace
When you ran \LaTeX{} with the input file {\tt sample.tex}, you may have
noticed that \LaTeX{} created a file named {\tt sample.aux}.  This file,
called an {\em auxiliary} file, contains cross-referencing information.  Since
{\tt sample.tex} contains no cross-referencing commands, the auxiliary file it
produces has no information.  However, suppose that \LaTeX{} is run with an
input file named {\tt myfile.tex} that has citations and bibliography-making
[or referencing] commands.  The auxiliary file {\tt myfile.aux} that it produces
will contain all of the citation keys and the arguments of the {\tt \verb|\bibliography|}
and {\tt\verb|\bibliographystyle|} commands.  When \BibTeX{} is run, it reads
this information from the auxiliary file and produces a file named {\tt myfile.bbl}
containing \LaTeX{} commands to produce the source list \ldots The next time
\LaTeX{} is run on {\tt myfile.tex}, the {\tt \verb|\bibliography|} command reads
the {\tt bbl} file ({\tt myfile.bbl}), which generates the source list.
\end{quotation}

Thus, the command sequence for a source file called {\tt main.tex} which is going to
use \BibTeX{} would be:
\begin{quote}\singlespace\tt\begin{verbatim}
latex main.tex
bibtex main
latex main
latex main
\end{verbatim}\end{quote}
The first \LaTeX{} is to collect all of the citations for \BibTeX{}.  Then
\BibTeX{} is run to generate the bibliography.  \LaTeX{} is run again to
incorporate the bibliography into the document and the run the last time to
update any references (like pages in the Table of Contents) which changed when
the bibliography was included.
           % Chapter 4 From PU Thesis styles, by J.D. McCauley
% usage.tex
%
% This file explains how to use the withesis style
%   it is heavily modelled after a similar chapter by McCauley
%   for the Purdue Thesis style
%
% Eric Benedict, May 2000
%
% It is provided without warranty on an AS IS basis.


\chapter{Using the {\tt withesis} Style}

You can get a copy of the \LaTeX{} style for creating a University
of Wisconsin--Madison thesis or dissertation from:

{\tt http://www.cae.wisc.edu/\verb+~+benedict/LaTeX.html}

After somehow unpacking it, you will have the style files ({\tt withesis.sty}
{\tt withe10.sty}, and {\tt withe12.sty}) as well the files used to create
this document.  The files used for this document can be copied and used as a
template for your own thesis or dissertation.

The final printed form of this document is useful, but the
combination of the source code and final copy form a much more valuable
reference.  Keeping a working copy of the this document can be helpful
when you are later working on your thesis or disseration and want to know
how to do something.  If you find a similar example in this document,
then you can simply look at the corresponding source code and add it to
your document.    Because many parts of this document were written by
different people, the styles and techniques are also different and provide
different ways of achieving the same or similar results.

Because of the typical size of theses, it makes sense to break the document
up into several smaller files.  Usually this is done at the chapter level.
These files can then be {\tt \verb|\include|}d in a {\em root} file.  It is
the {\em root} file that you will run \LaTeX{} on.  For this manual, the
root file is called {\tt main.tex}.

\section{The Root File and the Preamble}
The {\tt \verb|\documentclass|} command is used to tell \LaTeX{} that you will
be using the {\tt withesis} document class and it is the first command in your
root file.  Class options such as {\tt 10pt}, {\tt 12pt}, {\tt msthesis} or
{\tt margincheck} are specified here:

{\tt \verb|\documentclass[12pt,msthesis]{withesis}|}

The class option {\tt msthesis} sets the margins to be appropriate for depositing
with the UW library, namely a 1.25 inch left margin with the remaining margins 1 inch.
The defaults for the title page are also defined for a thesis and for a Master of
Science degree.

The class option {\tt margincheck} will place a small black square at the end of
each line which exceeds the margins.\footnote{In reality, the square is
placed at the end of lines which exceed their {\tt \char92hbox}.  This usually
(but not always) indicates a  margin violation on the right margin.  Left
margin violations aren't indicated and if the margin violation is large enough,
there isn't room for the black box to be visiable.}  This is visible both in the {\tt .dvi} file
as well as in the {\tt .ps} file.

The area immediately following this command is called the {\em preamble} and is
used for things like including different style packages,
defining new macros and declaring the page style.

The style packages can be used to easily change the thesis font.  For example,
this document is set in Times Roman instead of the \LaTeX default of Computer
Modern.  This change was performed by including the {\tt times} package:

{\tt\verb|\usepackage{times}|}\footnote{In this document, the typewriter font
{\tt $\backslash$tt} was redefined to use the Computer Modern font with the command
{\tt $\backslash$renewcommand\{$\backslash$ttdefault\}\{cmtt\}}.  
For more information, see~\cite{goossens}.}

Remember that if you change the fonts from the default Computer Modern to
PostScript ({\em e.g.} Times Roman) then in order to correctly see the
document, you will need to convert the {\tt *.dvi} output into a {\tt *.ps}
file and view the document with a PostScript viewer. This is required since 
most {\tt *.dvi} previewer programs cannot 
display PostScript fonts.  Usually, the previewer will substitute
default fonts so the document may be viewed; however, since the alternate
fonts may not be the same size, the formatting of the document may appear
to be incorrect.

The style package for including Postscript figures, {\tt epsfig}, is included with

{\tt\verb|\usepackage{epsfig}|}

If multiple style packages are required, then they can be combined into one statement
as follows:

{\tt\verb|\usepackage{epsfig,times}|}

Many different style packages are available.  For more information, see~\cite{goossens}.

The page styles are defined using a similar method.
A special style is defined for the {\tt withesis} style:

{\tt\verb|\pagestyle{thesisdraft}|}

This style causes the footer text to become:

{\verb| DRAFT: Do Not Distribute        <time><Date>        <input file name>|}

This appears at the bottom of every page.

In addition to the page style command, the {\tt withesis} has defined several useful
commands which are specified in the preamble.  They include {\tt \verb| \draftmargin|},
{\tt \verb|\draftscreen|}, {\tt \verb|\noappendixtables|}, and
{\tt \verb|\noappendixfigures|}.

The command  {\tt \verb|\draftmargin|} draws a PostScript box with the dimensions of
the margins.  This makes it easy to check that the margins are correct and to see if
any of the text or figures are outside of the required margins.  This box is only visible
in the {\tt .ps} file since it is a PostScript special.


The command  {\tt \verb|\draftscreen|} draws a PostScript screen with the word {\em DRAFT}
in light grey and diagonally across the page.  This screen is only visible
in the {\tt .ps} file since it is a PostScript special.

The commands {\tt \verb|\noappendixtables|} and/or {\tt \verb|\noappendixfigures|} should
be used if the appendix does not have either tables or figures respectively.  These commands
inhibit the Appendix Table or Appendix Figure titles in the List of Tables or List of
Figures.\label{usage:noapp}


If you have specified the {\tt psfig} or {\tt epsfig} document style package, then a useful
command is {\tt \verb|\psdraft|}.  This command will show the bounding box that the figure
would occupy (instead of actually including the figure).  This speeds up the draft copy
printing, reduces toner usage and the drawn box is visible in the {\tt .dvi} file.

The next usual command is {\tt \verb|\begin{document}|}.  The following example is part
of the root file used for this manual.

\begin{quote} \singlespace\footnotesize\tt
\begin{verbatim}
\bibliographystyle{plain}
% prelude.tex
%   - titlepage
%   - dedication
%   - acknowledgments
%   - table of contents, list of tables and list of figures
%   - nomenclature
%   - abstract
%============================================================================


\clearpage\pagenumbering{roman}  % This makes the page numbers Roman (i, ii, etc)



% TITLE PAGE
%   - define \title{} \author{} \date{}
\title{How to \LaTeX\ a Thesis}
\author{Eric R. L. Benedict}
\date{2000}
%   - The default degree is ``Doctor of Philosophy''
%     (unless the document style msthesis is specified
%      and then the default degree is ``Master of Science'')
%     Degree can be changed using the command \degree{}
\degree{Master \TeX nician}
%   - The default is dissertation, unless the document style
%     msthesis was specified in which case it becomes thesis.
%     If msthesis is specified for the MS margins, you can
%     still have a dissertation if you specify \disseration
%\disseration
%   - for a masters project report, specify \project
%\project
%   - for a preliminary report, specify \prelim
\prelim
%   - for a masters thesis, specify \thesis
%\thesis
%   - The default department is ``Electrical Engineering''
%     The department can be changed using the command \department{}
%\department{New Department}
%   - once the above are defined, use \maketitle to generate the titlepage
\maketitle

% COPYRIGHT PAGE
%   - To include a copyright page use \copyrightpage
\copyrightpage

% DEDICATION
\begin{dedication}
To my pet rock, Skippy.
\end{dedication}

% ACKNOWLEDGMENTS
\begin{acknowledgments}
I thank the many people who have done lots of nice things for me.
\end{acknowledgments}

% CONTENTS, TABLES, FIGURES
\tableofcontents
\listoftables
\listoffigures

% NOMENCLATURE
\begin{nomenclature}
\begin{description}
\item{\makebox[0.75in][l]{\TeX}}
       \parbox[t]{5in}{a typesetting system by Donald Knuth~\cite{knuth}.  It
       also refers to the ``plain'' format.  The proper pronounciation
       rhymes with ``heck'' and ``peck'' and does not sound like
       ``hex'' or ``Rex.''\\}

\item{\makebox[0.75in][l]{\LaTeX}}  
        \parbox[t]{5in}{a set of \TeX{} macros originally written by Leslie 
        Lamport~\cite{lamport}.  The proper pronunciation is 
        {\tt l\={a}$\cdot$tek'} and not {\tt l\={a}'$\cdot$teks} (see above).\\}

\item{\makebox[0.75in][l]{{\sc Bib}\TeX}} 
         \parbox[t]{5in}{a bibliography generation program by Oren 
                Patashnik~\cite{lamport}
                that can be used with either plain \TeX{} or \LaTeX{}.\\}

\item{\makebox[0.75in][l]{$C_1$}} Constant 1

\item{\makebox[0.75in][l]{$V$}}    Voltage 

\item{\makebox[0.75in][l]{\$}}     US Dollars
\end{description}
\end{nomenclature}


\advisorname{Bucky J. Badger}
\advisortitle{Assistant Professor}
% ABSTRACT
\begin{umiabstract}
  \input{abstract}
\end{umiabstract}

\begin{abstract}
  \input{abstract}
\end{abstract}


\clearpage\pagenumbering{arabic} % This makes the page numbers Arabic (1, 2, etc)
        % Title page, abstract, table of contents, etc
\intro

%
% Используемые далее команды определяются в файле common.tex.
%

% Актуальность работы
\actualitysection
\actualitytext

% Степень разработанности темы исследования
\developmentsection
\developmenttext

% Цели и задачи диссертационной работы
\objectivesection
\objectivetext

% Научная новизна
\noveltysection
\noveltytext

% Теоретическая и практическая значимость
\valuesection
\valuetext

% Методология и методы исследования
\methodssection
\methodstext

% Результаты и положения, выносимые на защиту
\resultssection
\resultstext

% Степень достоверности и апробация результатов
\approbationsection
\approbationtext

% Публикации
\pubsection
\pubtext

% Личный вклад автора
\contribsection
\contribtext

% Структура и объем диссертации
\structsection
\structtext
          % Chapter 1
% Essential LaTeX - Jon Warbrick 02/88
%   - Edited May, July 2000 -E. Benedict


% Copyright (C) Jon Warbrick and Plymouth Polytechnic 1989
% Permission is granted to reproduce the document in any way providing
% that it is distributed for free, except for any reasonable charges for
% printing, distribution, staff time, etc.  Direct commercial
% exploitation is not permitted.  Extracts may be made from this
% document providing an acknowledgment of the original source is
% maintained.

% NOTICE: This document has been edited for use in the UW-Madison
% Example Thesis file.


% counters used for the sample file example
\newcounter{savesection}
\newcounter{savesubsection}


% commands to do 'LaTeX Manual-like' examples

\newlength{\egwidth}\setlength{\egwidth}{0.42\textwidth}

\newenvironment{eg}{\begin{list}{}{\setlength{\leftmargin}%
{0.05\textwidth}\setlength{\rightmargin}{\leftmargin}}%
\item[]\footnotesize}{\end{list}}

\newenvironment{egbox}{\begin{minipage}[t]{\egwidth}}{\end{minipage}}

\newcommand{\egstart}{\begin{eg}\begin{egbox}}
\newcommand{\egmid}{\end{egbox}\hfill\begin{egbox}}
\newcommand{\egend}{\end{egbox}\end{eg}}

% one or two other commands
\newcommand{\fn}[1]{\hbox{\tt #1}}
\newcommand{\llo}[1]{(see line #1)}
\newcommand{\lls}[1]{(see lines #1)}
\newcommand{\bs}{$\backslash$}


\chapter{Essential \LaTeX{}}

This chapter introduces some key ideas behind \LaTeX{} and give you the ``essential''
items of information.  This chapter is an edited form of the paper
``Essential \LaTeX{}'' by Jon Warbrick, Plymouth Polytechnic.

\section{Introduction}
This document is an attempt to give you all the essential
information that you will need in order to use the \LaTeX{} Document
Preparation System.  Only very basic features are covered, and a
vast amount of detail has been omitted.  In a document of this size
it is not possible to include everything that you might need to know,
and if you intend to make extensive use of the program you should
refer to a more complete reference.  Attempting to produce complex
documents using only the information found below will require
much more work than it should, and will probably produce a less
than satisfactory result.

The main reference for \LaTeX{} is {\em The \LaTeX{} User's guide and
Reference Manual\/} by Leslie Lamport.  This contains most of the
information that you will ever need to know about the program, and
you will need access to a copy if you are to use \LaTeX{} seriously.
You should also consider getting a copy of {\em The \LaTeX{}
Companion\/} 

\section{How does \LaTeX{} work?}

In order to use \LaTeX{} you generate a file containing
both the text that you wish to print and instructions to tell \LaTeX{}
how you want it to appear.  You will normally create
this file using your system's text editor.  You can give the file any name you
like, but it should end ``\fn{.TEX}'' to identify the file's contents.
You then get \LaTeX{} to process the file, and it creates a
new file of typesetting commands; this has the same name as your file but
the ``\fn{.TEX}'' ending is replaced by ``\fn{.DVI}''.  This stands for
`{\it D\/}e{\it v\/}ice {\it I\/}ndependent' and, as the name implies, this file
can be used to create output on a range of printing devices.
Your {\em local guide\/} will go into more detail.

Rather than encourage you to dictate exactly how your document
should be laid out, \LaTeX{} instructions allow you describe its
{\em logical structure\/}.  For example, you can think of a quotation
embedded within your text as an element of this logical structure: you would
normally expect a quotation to be displayed in a recognisable style to set it
off from the rest of the text.
A human typesetter would recognise the quotation and handle
it accordingly, but since \LaTeX{} is only a computer program it requires
your help.  There are therefore \LaTeX{} commands that allow you to
identify quotations and as a result allow \LaTeX{} to typeset them correctly.

Fundamental to \LaTeX{} is the idea of a {\em document style\/} that
determines exactly how a document will be formatted.  \LaTeX{} provides
standard document styles that describe how standard logical structures
(such as quotations) should be formatted.  You may have to supplement
these styles by specifying the formatting of logical structures
peculiar to your document, such as mathematical formulae.  You can
also modify the standard document styles or even create an entirely
new one, though you should know the basic principles of typographical
design before creating a radically new style.

There are a number of good reasons for concentrating on the logical
structure rather than on the appearance of a document.  It prevents
you from making elementary typographical errors in the mistaken
idea that they improve the aesthetics of a document---you should
remember that the primary function of document design is to make
documents easier to read, not prettier.  It is more flexible, since
you only need to alter the definition of the quotation style
to change the appearance of all the quotations in a document.  Most
important of all, logical design encourages better writing.
A visual system makes it easier to create visual effects rather than
a coherent structure; logical design encourages you to concentrate on
your writing and makes it harder to use formatting as a substitute
for good writing.

\section{A Sample \LaTeX{} file}


\begin{figure} %---------------------------------------------------------------
{\singlespace\tt\footnotesize\begin{verbatim}
 1: % SMALL.TEX -- Released 5 July 1985
 2: % USE THIS FILE AS A MODEL FOR MAKING YOUR OWN LaTeX INPUT FILE.
 3: % EVERYTHING TO THE RIGHT OF A  %  IS A REMARK TO YOU AND IS IGNORED
 4: % BY LaTeX.
 5: %
 6: % WARNING!  DO NOT TYPE ANY OF THE FOLLOWING 10 CHARACTERS EXCEPT AS
 7: % DIRECTED:        &   $   #   %   _   {   }   ^   ~   \
 8:
 9: \documentclass[11pt,a4]{article}  % YOUR INPUT FILE MUST CONTAIN THESE
10: \begin{document}                  % TWO LINES PLUS THE \end COMMAND AT
11:                                   % THE END
12:
13: \section{Simple Text}          % THIS COMMAND MAKES A SECTION TITLE.
14:
15: Words are separated by one or    more      spaces.  Paragraphs are
16:     separated by one or more blank lines.  The output is not affected
17: by adding extra spaces or extra blank lines to the input file.
18:
19:
20: Double quotes are typed like this: ``quoted text''.
21: Single quotes are typed like this: `single-quoted text'.
22:
23: Long dashes are typed as three dash characters---like this.
24:
25: Italic text is typed like this: {\em this is italic text}.
26: Bold   text is typed like this: {\bf this is  bold  text}.
27:
28: \subsection{A Warning or Two}        % THIS MAKES A SUBSECTION TITLE.
29:
30: If you get too much space after a mid-sentence period---abbreviations
31: like etc.\ are the common culprits)---then type a backslash followed by
32: a space after the period, as in this sentence.
33:
34: Remember, don't type the 10 special characters (such as dollar sign and
35: backslash) except as directed!  The following seven are printed by
36: typing a backslash in front of them:  \$  \&  \#  \%  \_  \{  and  \}.
37: The manual tells how to make other symbols.
38:
39: \end{document}                    % THE INPUT FILE ENDS LIKE THIS
\end{verbatim}  }

\caption{A Sample \LaTeX{} File}\label{fig:sample}

\end{figure} %-----------------------------------------------------------------



Have a look at the example \LaTeX{} file in Figure~\ref{fig:sample}.  It
is a slightly modified copy of the standard \LaTeX{} example file
\fn{SMALL.TEX}.  The line numbers down the left-hand side
are not part of the file, but have been added to make it easier to
identify various portions.

Try entering this file (without the line numbers), save the text as \fn{small.tex},
next run \LaTeX{} on it, and then view the output:

{\tt \singlespace\begin{verbatim}
% latex small
% xdvi small               # displays the output on the screen
% dvips -o small.ps small  # to create a PostScript file, small.ps
% lp -d<printer> small.ps  # to print
\end{verbatim}}

\subsection{Running Text}

Most documents consist almost entirely of running text---words formed
into sentences, which are in turn formed into paragraphs---and the example file
is no exception. Describing running text poses no problems, you just type
it in naturally. In the output that it produces, \LaTeX{} will fill
lines and adjust the
spacing between words to give tidy left and right margins.
The spacing and distribution of the words in your input
file will have no effect at all on the eventual output.
Any number of spaces in your input file
are treated as a single space by \LaTeX{}, it also regards the
end of each line as a space between words \lls{15--17}.
A new paragraph is
indicated by a blank line in your input file, so don't leave
any blank lines unless you really wish to start a paragraph.

\LaTeX{} reserves a number of the less common keyboard characters for its
own use. The ten characters
\begin{quote}\begin{verbatim}
#  $  %  &  ~  _  ^  \  {  }
\end{verbatim}\end{quote}
should not appear as part of your text, because if they do
\LaTeX{} will get confused.

\subsection{\LaTeX{} Commands}

There are a number of words in the file that start `\verb|\|' \lls{9,
10 and 13}.  These are \LaTeX{} {\em commands\/} and they describe
the structure of your document. There are a number of things that you
should realize about these commands:
\begin{itemize}

\item All \LaTeX{} commands consist of a `\verb|\|' followed by one or more
characters.

\item \LaTeX{} commands should be typed using the correct mixture of upper- and
lower-case letters.  \verb|\BEGIN| is {\em not\/} the same as \verb|\begin|.

\item Some commands are placed within your text.  These are used to
switch things, like different typestyles, on and off. The \verb|\em|
command is used like this to emphasize text, normally by changing to
an {\it italic\/} typestyle \llo{25}.  The command and the text are
always enclosed between `\verb|{|' and `\verb|}|'---the `\verb|{\em|'
turns the effect on and and the `\verb|}|' turns it off.

\item There are other commands that look like
\begin{quote}\begin{verbatim}
\command{text}
\end{verbatim}\end{quote}
In this case the text is called the ``argument'' of the command.  The
\verb|\section| command is like this \llo{13}.
Sometimes you have to use curly brackets `\verb|{}|' to enclose the argument,
sometimes square brackets `\verb|[]|', and sometimes both at once.
There is method behind this apparent madness, but for the
time being you should be sure to copy the commands exactly as given.

\item When a command's name is made up entirely of letters, you must make sure
that the end of the command is marked by something that isn't a letter.
This is usually either the opening bracket around the command's argument, or
it's a space.  When it's a space, that space is always ignored by \LaTeX. We
will see later that this can sometimes be a problem.

\end{itemize}

\subsection{Overall structure}

There are some \LaTeX{} commands that must appear in every document.
The actual text of the document always starts with a
\verb|\begin{document}| command and ends with an \verb|\end{document}|
command \lls{10 and 39}.  Anything that comes after the \break
\verb|\end{document}| command is ignored.  Everything that comes
before the \break\verb|\begin{document}| command is called the
{\em preamble\/}. The preamble can only contain \LaTeX{} commands
to describe the document's style.

One command that must appear in the preamble is the
\verb|\documentclass| command \llo{9}.  This command specifies the
overall style for the document.  Our example file is a simple
technical document, and uses the {\tt article\/} class.  The document
you are reading was produced with the {\tt withesis\/} class. There
are other classes that you can use, as you will find out later on in
this document.

\subsection{Other Things to Look At}

\LaTeX{} can print both opening and closing quote characters, and can manage
either of these either single or double.  To do this it uses the two quote
characters from your keyboard: {\tt `} and {\tt '}. You will probably think of
{\tt '} as the ordinary single quote character which probably looks like
{\tt\symbol{'23}} or {\tt\symbol{'15}} on your keyboard,

and {\tt `} as a ``funny'' character that probably appears as
{\tt\symbol{'22}}. You type these characters once for single quote
\llo{21},  and twice for double quotes \llo{20}. The double quote
character {\tt "} itself is almost never used and should not be used
unless you want your text to look "funny" (compare the quote in the
previous sentence).

\LaTeX{} can produce three different kinds of dashes.
A long dash, for use as a punctuation symbol, as is typed as three dash
characters in a row, like this `\verb|---|' \llo{23}.  A shorter dash,
used between numbers as in `10--20', is typed as two dash
characters in a row, while a single dash character is used as a hyphen.

From time to time you will need to include one or more of the \LaTeX{}
special symbols in your text.  Seven of them can be printed by
making them into commands by proceeding them by backslash
\llo{36}.  The remaining three symbols can be produced by more
advanced commands, as can symbols that do not appear on your keyboard
such as \dag, \ddag, \S, \pounds, \copyright, $\sharp$ and $\clubsuit$.

It is sometimes useful to include comments in a \LaTeX{} file, to remind
you of what you have done or why you did it.  Everything to the
right of a \verb|%| sign is ignored by \LaTeX{}, and so it can
be used to introduce a comment.

\section{Document Classes and Class Options}\label{sec:styles}

There are four standard document classes available in \LaTeX:
\nobreak

\begin{description}

\item[{\tt article}]  intended for short documents and articles for publication.
Articles do not have chapters, and when \verb|\maketitle| is used to generate

a title (see Section~\ref{sec:title}) it appears at the top of the first page

rather than on a page of its own.

\item[{\tt report}] intended for longer technical documents.
It is similar to
{\tt article}, except that it contains chapters and the title appears on a page
of its own.

\item[{\tt book}] intended as a basis for book publication.  Page layout is
adjusted assuming that the output will eventually be used to print on
both sides of the paper.

\item[{\tt letter}]  intended for producing personal letters.  This style
will allow you to produce all the elements of a well laid out letter:
addresses, date, signature, etc.
\end{description}

An additional document class, the one used for this document and for
University of Wisconsin--Madison theses, is \fn{withesis}.


These standard classes can be modified by a number of {\em class
options\/}. They appear in square brackets after the
\verb|\documentclass| command. Only one class can ever be used but
you can have more than one class option, in which case their names
should be separated by commas.  The standard style options are:
\begin{description}

\item[{\tt 11pt}]  prints the document using eleven-point type for the running
 text
rather that the ten-point type normally used. Eleven-point type is about
ten percent larger than ten-point.

\item[{\tt 12pt}]  prints the document using twelve-point type for the running
 text
rather than the ten-point type normally used. Twelve-point type is about
twenty percent larger than ten-point.

\item[{\tt twoside}]  causes documents in the article or report styles to be
formatted for printing on both sides of the paper.  This is the default for the
book style.

\item[{\tt twocolumn}] produces two column on each page.

\item[{\tt titlepage}]  causes the \verb|\maketitle| command to generate a
title on a separate page for documents in the \fn{article} style.
A separate page is always used in both the \fn{report} and \fn{book} styles.

\end{description}

The University of Wisconsin--Madison thesis style, \fn{withesis} also
has some class options defined.  These class options are for
ten-point type (\fn{10pt}), tweleve-point type (\fn{12pt}), two-sided
printing (\fn{twoside}), Master Thesis margins (\fn{msthesis}) and an
option to print a small black box on lines which exceed the margins
(\fn{margincheck}).

\section{Environments}

We mentioned earlier the idea of identifying a quotation to \LaTeX{} so that
it could arrange to typeset it correctly. To do this you enclose the
quotation between the commands \verb|\begin{quotation}| and
\verb|\end{quotation}|.
This is an example of a \LaTeX{} construction called an {\em environment\/}.
A number of
special effects are obtained by putting text into particular environments.

\subsection{Quotations}

There are two environments for quotations: \fn{quote} and \fn{quotation}.
\fn{quote} is used either for a short quotation or for a sequence of
short quotations separated by blank lines:
\egstart\singlespace
\begin{verbatim}
US presidents ... remarks:
\begin{quote}
The buck stops here.

I am not a crook.
\end{quote}
\end{verbatim}
\egmid%
US presidents have been known for their pithy remarks:
\begin{quote}
The buck stops here.

I am not a crook.
\end{quote}
\egend

Use the \fn{quotation} environment for quotations that consist of more
than one paragraph.  Paragraphs in the input are separated by blank
lines as usual:
\egstart\singlespace
\begin{verbatim}

Here is some advice to remember:
\begin{quotation}
Environments for making
...other things as well.

Many problems
...environments.
\end{quotation}
\end{verbatim}
\egmid%
Here is some advice to remember:
\begin{quotation}
Environments for making quotations
can be used for other things as well.

Many problems can be solved by
novel applications of existing
environments.
\end{quotation}
\egend

\subsection{Centering and Flushing}

Text can be centered on the page by putting it within the \fn{center}
environment, and it will appear flush against the left or right margins if it
is placed within the \fn{flushleft} or \fn{flushright} environments.

Text within these environments will be formatted in the normal way, in
{\samepage
particular the ends of the lines that you type are just regarded as spaces.  To
indicate a ``newline'' you need to type the \verb|\\| command.  For example:
\egstart\singlespace
\begin{verbatim}
\begin{center}
one
two
three \\
four \\
five
\end{center}

\end{verbatim}
\egmid%
\begin{center}

one
two
three \\
four \\

five
\end{center}
\egend
}

\subsection{Lists}

There are three environments for constructing lists.  In each one each new
item is begun with an \verb|\item| command.  In the \fn{itemize} environment
the start of each item is given a marker, in the \fn{enumerate}
environment each item is marked by a number.  These environments can be nested
within each other in which case the amount of indentation used
is adjusted accordingly:
\egstart\singlespace

\begin{verbatim}
\begin{itemize}
\item Itemized lists are handy.
\item However, don't forget
  \begin{enumerate}
  \item The `item' command.
  \item The `end' command.
  \end{enumerate}
\end{itemize}
\end{verbatim}
\egmid%
\begin{itemize}
\item Itemized lists are handy.
\item However, don't forget
  \begin{enumerate}
  \item The `item' command.
  \item The `end' command.
  \end{enumerate}
\end{itemize}
\egend


The third list making environment is \fn{description}.  In a description you
specify the item labels inside square brackets after the \verb|\item| command.
For example:
\egstart\singlespace
\begin{verbatim}
Three animals that you should
know about are:
\begin{description}
  \item[gnat] A small
            animal...
  \item[gnu] A large
           animal...
  \item[armadillo] A ...
\end{description}
\end{verbatim}
\egmid%
Three animals that you should
know about are:
\begin{description}
  \item[gnat] A small animal that causes no end of trouble.
  \item[gnu] A large animal that causes no end of trouble.
  \item[armadillo] A medium-sized animal.
\end{description}
\egend

\subsection{Tables}

Because \LaTeX{} will almost always convert a sequence of spaces
into a single space, it can be rather difficult to lay out tables.
See what happens in this example
 \nolinebreak
\begin{eg}
\begin{minipage}[t]{0.55\textwidth} \singlespace
\begin{verbatim}
\begin{flushleft}
Income  Expenditure Result   \\
20s 0d  19s 11d     happiness \\
20s 0d  20s 1d      misery  \\
\end{flushleft}
\end{verbatim}
\end{minipage}
\begin{minipage}[t]{0.3\textwidth}
\begin{flushleft}
Income  Expenditure Result   \\
20s 0d  19s 11d     happiness \\
20s 0d  20s 1d      misery  \\
\end{flushleft}
\end{minipage}
\end{eg}

The \fn{tabbing} environment overcomes this problem. Within it you
set tabstops and tab to them much like you do on a typewriter.
Tabstops are set with the \verb|\=| command, and the \verb|\>|
command moves to the next stop.  The \verb|\\| command is used to
separate each line.  A line that ends \verb|\kill| produces no
output, and can be used to set tabstops:
\nolinebreak
\begin{eg}
\begin{minipage}[t]{0.6\textwidth}
\singlespace
\begin{verbatim}
\begin{tabbing}
Income \=Expenditure \=    \kill
Income \>Expenditure \>Result \\
20s 0d \>19s 11d \>Happiness \\
20s 0d \>20s 1d  \>Misery    \\
\end{tabbing}
\end{verbatim}
\end{minipage}
\vspace{1ex}
\begin{minipage}[t]{0.35\textwidth}
\begin{tabbing}
\singlespace
Income \=Expenditure \=    \kill
Income \>Expenditure \>Result \\
20s 0d \>19s 11d \>Happiness \\
20s 0d \>20s 1d  \>Misery    \\
\end{tabbing}
\end{minipage}
\end{eg}

Unlike a typewriter's tab key, the \verb|\>| command always moves to the next
tabstop in sequence, even if this means moving to the left.  This can cause
text to be overwritten if the gap between two tabstops is too small.

\subsection{Verbatim Output}

Sometimes you will want to include text exactly as it appears on a terminal
screen.  For example, you might want to include part of a computer program.
Not only do you want \LaTeX{} to stop playing around with the layout of your
text, you also want to be able to type all the characters on your keyboard
without confusing \LaTeX. The \fn{verbatim} environment has this effect:

\egstart

\begin{flushleft}\singlespace
\verb|The section of program in|  \\
 \verb|question is :|\\
 \verb|\begin{verbatim}|           \\
\verb|{ this finds %a & %b }|     \\[2ex]

\verb|for i := 1 to 27 do|        \\
\ \ \ \verb|begin|                \\
\ \ \ \verb|table[i] := fn(i);|   \\

\ \ \ \verb|process(i)|           \\
\ \ \ \verb|end;|                 \\
\verb|\end{verbatim}|
\end{flushleft}
\egmid%
The section of program in
question is:
\begin{verbatim}
{ this finds %a & %b }

for i := 1 to 27 do
   begin
   table[i] := fn(i);
   process(i)
   end;

\end{verbatim}
\egend

The \fn{withesis} document style also provides the command {\tt \verb|\verbatimfile{foo.fe}|}
which will read in the file {\tt foo.fe} into the document in \fn{verbatim} format with
the font \verb|\tt|.  See Appendix~\ref{matlab} for an example.

\section{Type Styles}

We have already come across the \verb|\em| command for changing
typeface.  Here is a full list of the available typefaces:
\begin{quote}\singlespace\begin{tabbing}
\verb|\sc|~~ \= \sc Small Caps~~~ \= \verb|\sc|~~ \= \sc Small Caps~~~
                                  \= \verb|\sc|~~ \=                   \kill
\verb|\rm|   \> \rm Roman         \> \verb|\it|   \> \it Italic
                                  \> \verb|\sc|   \> \sc Small Caps    \\
\verb|\em|   \> \em Emphatic      \> \verb|\sl| \> \sl Slanted
                                  \> \verb|\tt|   \> \tt Typewriter     \\
\verb|\bf|   \> \bf Boldface      \> \verb|\sf| \> \sf Sans Serif
\end{tabbing}\end{quote}

Remember that these commands are used {\em inside\/} a pair of braces to limit
the amount of text that they effect.  In addition to the eight typeface
commands, there are a set of commands that alter the size of the type.  These
commands are:
\begin{quotation}\singlespace\begin{tabbing}
\verb|\footnotesize|~~ \= \verb|\footnotesize|~~ \= \verb|\footnotesize| \=
 \kill
\verb|\tiny|           \> \verb|\small|          \> \verb|\large|        \>
\verb|\huge|  \\
\verb|\scriptsize|     \> \verb|\normalsize|     \> \verb|\Large|        \>
\verb|\Huge|  \\
\verb|\footnotesize|   \>                        \> \verb|\LARGE|
\end{tabbing}\end{quotation}

\section{Sectioning Commands and Tables of Contents}
\label{ess:sectioning}

Technical documents, like this one, are often divided into sections.
Each section has a heading containing a title and a number for easy
reference.  \LaTeX{} has a series of commands that will allow you to identify
different sorts of sections.  Once you have done this \LaTeX{} takes on the
responsibility of laying out the title and of providing the numbers.

The commands that you can use are:
\begin{quote}\singlespace\begin{tabbing}
\verb|\subsubsection| \= \verb|\subsubsection|~~~~~~~~~~ \=           \kill
\verb|\chapter|       \> \verb|\subsection|    \> \verb|\paragraph|    \\
\verb|\section|       \> \verb|\subsubsection| \> \verb|\subparagraph| \\
\end{tabbing}\end{quote}
The naming of these last two is unfortunate, since they do not really have
anything to do with `paragraphs' in the normal sense of the word; they are just
lower levels of section.  In most document styles, headings made with
\verb|\paragraph| and \verb|\subparagraph| are not numbered.  \verb|\chapter|
is not available in document style \fn{article}.  The commands should be used
in the order given, since sections are numbered within chapters, subsections
within sections, etc.

A seventh sectioning command, \verb|\part|, is also available.  Its use is
always optional, and it is used to divide a large document into series of
parts.  It does not alter the numbering used for any of the other commands.

Including the command \verb|\tableofcontents| in you document will cause a
contents list to be included, containing information collected from the various
sectioning commands.  You will notice that each time your document is run
through \LaTeX{} the table of contents is always made up of the headings from
the previous version of the document.  This is because \LaTeX{} collects
information for the table as it processes the document, and then includes it
the next time it is run.  This can sometimes mean that the document has to be
processed through \LaTeX{} twice to get a correct table of contents.

\section{Producing Special Symbols}

You can include in you \LaTeX{} document a wide range of symbols that do not
appear on you your keyboard. For a start, you can add an accent to any letter:
\begin{quote}\singlespace\begin{tabbing}

\t{oo} \= \verb|\t{oo}|~~~ \=
\t{oo} \= \verb|\t{oo}|~~~ \=
\t{oo} \= \verb|\t{oo}|~~~ \=
\t{oo} \= \verb|\t{oo}|~~~ \=
\t{oo} \= \verb|\t{oo}|~~~ \=
\t{oo} \=                       \kill

\a`{o} \> \verb|\`{o}|  \> \~{o}  \> \verb|\~{o}|  \> \v{o}  \> \verb|\v{o}| \>
\c{o}  \> \verb|\c{o}|  \> \a'{o} \> \verb|\'{o}|  \\
\a={o} \> \verb|\={o}|  \> \H{o}  \> \verb|\H{o}|  \> \d{o}  \> \verb|\d{o}| \>
\^{o}  \> \verb|\^{o}|  \> \.{o}  \> \verb|\.{o}|  \\
\t{oo} \> \verb|\t{oo}| \> \b{o}  \> \verb|\b{o}|  \\  \"{o} \> \verb|\"{o}| \>
\u{o}  \> \verb|\u{o}|  \\
\end{tabbing}\end{quote}

A number of other symbols are available, and can be used by including the
following commands:
\begin{quote}\singlespace\begin{tabbing}

\LaTeX~\= \verb|\copyright|~~~~ \= \LaTeX~\= \verb|\copyright|~~~~ \=
\LaTeX~\=  \kill

\dag       \> \verb|\dag|       \> \S     \> \verb|\S|     \>
\copyright \> \verb|\copyright| \\
\ddag      \> \verb|\ddag|      \> \P     \> \verb|\P|     \>
\pounds    \> \verb|\pounds|    \\
\oe        \> \verb|\oe|        \> \OE    \> \verb|\OE|    \>
\ae        \> \verb|\AE|        \\
\AE        \> \verb|\AE|        \> \aa    \> \verb|\aa|    \>
\AA        \> \verb|\AA|        \\
\o         \> \verb|\o|         \> \O     \> \verb|\O|     \>
\l         \> \verb|\l|         \\
\L         \> \verb|\E|         \> \ss    \> \verb|\ss|    \>
?`         \> \verb|?`|         \\
!`         \> \verb|!`|         \> \ldots \> \verb|\ldots| \>
\LaTeX     \> \verb|\LaTeX|     \\
\end{tabbing}\end{quote}
There is also a \verb|\today| command that prints the current date. When you
use these commands remember that \LaTeX{} will ignore any spaces that
follow them, so that you can type `\verb|\pounds 20|' to get `\pounds 20'.
However, if you type `\verb|LaTeX is wonderful|' you will get `\LaTeX is
wonderful'---notice the lack of space after \LaTeX.
To overcome this problem you can follow any of these commands by a
pair of empty brackets and then any spaces that you wish to include,
and you will see that
\verb|\LaTeX{} really is wonderful!| (\LaTeX{} really is wonderful!).

\section{Titles}\label{sec:title}

Most documents have a title.  To title a \LaTeX{} document, you include the
following commands in your document, usually just after
\verb|begin{document}|.
\begin{quote}\singlespace\begin{verbatim}
\title{required title}
\author{required author}
\date{required date}
\maketitle
\end{verbatim}\end{quote}
If there are several authors, then their names should be separated by
\verb|\and|; they can also be separated by \verb|\\| if you want them to be
centred on different lines.  If the \verb|\date| command is left out, then the
current date will be printed.
\egstart
\singlespace
\begin{verbatim}
\title{Essential \LaTeX}
\author{J Warbrick \and An Other}
\date{14th February 1988}
\maketitle
\end{verbatim}
\egmid
\begin{center}
{\normalsize Essential \LaTeX}\\[4ex]
J Warbrick\hspace{1em}A N Other\\[2ex]
14th February 1988
\end{center}
\egend

The exact appearance of the title varies depending on
the document style.  In styles \fn{report} and \fn{book} the title appears on a
page of its own. In the \fn{article} style it normally appears at the top
of the first page, the style option \fn{titlepage} will alter this (see
Section~\ref{sec:styles}).  In the \fn{withesis} style, the title is created on a
seperate page in the format appropriate to a UW-Madison thesis or dissertation.

\section{Errors}

When you create a new input file for \LaTeX{} you will probably make mistakes.
Everybody does, and it's nothing to be worried about.  As with most computer
programs, there are two sorts of mistake that you can make: those that \LaTeX{}
notices and those that it doesn't.  To take a rather silly example, since
\LaTeX{} doesn't understand what you are saying it isn't going to be worried if
you mis-spell some of the words in your text.  You will just have to accurately
proof-read your printed output.  On the other hand, if you mis-spell one of
the environment names in your file then \LaTeX won't know what you want it
to do.

When this sort of thing happens, \LaTeX{} prints an error message on your
terminal screen and then stops and waits for you to take some action.
Unfortunately, the error messages that it produces are rather user-unfriendly
and a little frightening.  Nevertheless, if you know where to look they
will probably tell you where the error is and went wrong.

Consider what would happen if you mistyped \verb|\begin{itemize}| so that it
became \break\verb|\begin{itemie}|.  When \LaTeX{} processes this instruction, it
displays the following on your terminal:
\begin{quote}\singlespace\begin{verbatim}
LaTeX error.  See LaTeX manual for explanation.
              Type  H <return>  for immediate help.
! Environment itemie undefined.
\@latexerr ...for immediate help.}\errmessage {#1}
                                                  \endgroup
l.140 \begin{itemie}

?
\end{verbatim}\end{quote}
After typing the `?' \LaTeX{} stops and waits for you to tell it what to do.

The first two lines of the message just tell you that the error was detected by
\LaTeX{}. The third line, the one that starts `!' is the {\em error indicator}.
 It
tells you what the problem is, though until you have had some experience of
\LaTeX{} this may not mean a lot to you.  In this case it is just telling you
that it doesn't recognise an environment called \fn{itemie}.
The next two lines tell you what
\LaTeX{} was doing when it found the error, they are irrelevant at the moment
and can be ignored. The final line is called the {\em error locator}, and is
a copy of the line from your file that caused the problem.
It start with a line number to help you to find it in your file, and
if the error was in the middle of a line it will be shown
broken at the point where \LaTeX{} realised that there was an error.  \LaTeX{}
can sometimes pass the point where the real error is before discovering that
something is wrong, but it doesn't usually get very far.

At this point you could do several things.  If you knew enough about \LaTeX{}
you might be able to fix the problem, or you could type `X' and press the
return key to stop \LaTeX{} running while you go and correct the error.  The
best thing to do, however, is just to press the return key.  This will allow
\LaTeX{} to go on running as if nothing had happened.  If you have made one
mistake, then you have probably made several and you may as well try to find
them all in one go.  It's much more efficient to do it this way than to run
\LaTeX{} over and over again fixing one error at a time. Don't worry about
remembering what the errors were---a copy of all the error messages is being
saved in a {\em log\/} file so that you can look at them afterwards.

If you look at the line that caused the error it's normally obvious what the
problem was.  If you can't work out what you problem is look at the hints
below, and if they don't help consult Chapter~6 of the manual~\cite{lamport}.
  It contains a
list of all of the error messages that you are likely to encounter together with
some hints as to what may have caused them.

Some of the most common mistakes that cause errors are
\begin{itemize}
\item A mispelled command or environment name.
\item Improperly matched `\verb|{|' and `\verb|}|'---remember that they should
 always
come in pairs.
\item Trying to use one of the ten special characters \verb|# $ % & _ { } ~ ^|
and \verb|\| as an ordinary printing symbol.
\item A missing \verb|\end| command.
\item A missing command argument (that's the bit enclosed in '\verb|{|' and
`\verb|}|').
\end{itemize}

One error can get \LaTeX{} so confused that it reports a series of spurious
errors as a result.  If you have an error that you understand, followed by a
series that you don't, then try correcting the first error---the rest
may vanish as if by magic.
Sometimes \LaTeX{} may write a {\tt *} and stop without an error message.  This
is normally caused by a missing \verb|\end{document}| command, but other errors
can cause it.  If this happens type \verb|\stop| and press the return key.

Finally, \LaTeX{} will sometimes print {\em warning\/} messages.  They report
problems that were not bad enough to cause \LaTeX{} to stop processing, but
nevertheless may require investigation.  The most common problems are
`overfull' and `underfull' lines of text.  A message like:
\begin{quote}\footnotesize\begin{verbatim}
Overfull \hbox (10.58649pt too wide) in paragraph at lines 172--175
[]\tenrm Mathematical for-mu-las may be dis-played. A dis-played
\end{verbatim}\end{quote}
indicates that \LaTeX{} could not find a good place to break a line when laying
out a paragraph.  As a result, it was forced to let the line stick out into the
right-hand margin, in this case by 10.6 points.  Since a point is about 1/72nd
of an inch this may be rather hard to see, but it will be there none the less.

This particular problem happens because \LaTeX{} is rather fussy about line
breaking, and it would rather generate a line that is too long than generate a
paragraph that doesn't meet its high standards.  The simplest way around the
problem is to enclose the entire offending paragraph between
\verb|\begin{sloppypar}| and \verb|\end{sloppypar}| commands.  This tells
\LaTeX{} that you are happy for it to break its own rules while it is working on
that particular bit of text.

Alternatively, messages about ``Underfull \verb|\hbox'es''| may appear.
These are lines that had to have more space inserted between
words than \LaTeX{} would have liked.  In general there is not much that you
can do about these.  Your output will look fine, even if the line looks a bit
stretched.  About the only thing you could do is to re-write the offending
paragraph!

\section{A Final Reminder}

You now know enough \LaTeX{} to produce a wide range of documents.  But this
document has only scratched the surface of the
things that \LaTeX{} can do.  This entire document was itself produced with
\LaTeX{} (with no sticking things in or clever use of a photocopier) and even
it hasn't used all the features that it could.  From this you may get some
feeling for the power that \LaTeX{} puts at your disposal.

Please remember what was said in the introduction: if you {\bf do} have a
complex document to produce then {\bf go and read the manual}.  You will be
wasting your time if you rely only on what you have read here.
     % Edited ``Essential LaTeX'' by Jon Warbrick
\chapter{Figures and Tables}\label{quad}
This chapter\footnote{Most of the text in this chapter's introduction is from {\em How to
\TeX{} a Thesis: The Purdue Thesis Styles}} shows some example ways of incorporating tables and figures into \LaTeX{}.
Special environments exist for tables and figures and are special because they are
allowed to {\em float}---that is, \LaTeX{} doesn't always put them in the exact place
that they occur in your input file.  An algorithm is used to place the floating environments,
or floats, at locations which are typographically correct.  This may cause endless frustration
if you want to have a figure or table occur at a specific location.  There are a few
methods for solving this.

You can exert some influence on \LaTeX{}'s float placement algorithm by using
{\em float position specifiers}.  These specifiers, listed below, tell \LaTeX{}
what you prefer.
\begin{tabbing}
{\tt hhhhhh} \= ``bottom'' \=  \kill
{\tt h}\> ``here'' \> do not move this object \\
{\tt p}\> ``page'' \> put this object on a page of floats \\
{\tt b}\> ``bottom'' \> put this object at the bottom of a page\\
{\tt t}\> ``top'' \> put this object at the top of a page\\
\end{tabbing}

Any combination of these can be used:
\begin{quote}\tt\singlespace\begin{verbatim}
\begin{figure}[htbp]
 ...
\caption{A Figure!}
\end{figure}
\end{verbatim}\end{quote}

In this example, we asked \LaTeX{} to ``put the figure `here' if possible.  If it
is not possible (according to the rule encoded in the float algorithm), put it on the
next float page.  A float page is a page which contains nothing but floating objects,
{\em e.g.} a page of nothing but figures or tables.  If this isn't possible, try to put it
at the `top' of a page.  The last thing to try is to put the figure at the `bottom' of
a page.''

The remainder of this chapter deals with some examples of what to put into the figure,
the ellipsis (\ldots ) in the example above.

\section{Tables}
Table~\ref{pde.tab1} is an example table from the UW Math Department.
\begin{table}[htbp]
\centering
\caption{PDE solve times, $15^3+1$
equations.\label{pde.tab1}}
\begin{tabular}{||l|l|l|l|l|l||}\hline
Precond. & Time & Nonlinear & Krylov
& Function & Precond. \\
 & & Iterations & Iterations & calls & solves \\ \hline
None & 1260.9u & 3 & 26 & 30 & 0  \\
 &(21:09) & & & &  \\ \hline
FFT  & 983.4u & 2  & 5  & 8  & 7 \\
&(16:31) & & & & \\ \hline
\end{tabular}
\end{table}
The code to generate it is as follows:
\begin{quote}\tt\singlespace\begin{verbatim}
\begin{table}[htbp]
\centering
\caption{PDE solve times, $15^3+1$
equations.\label{pde.tab1}}
\begin{tabular}{||l|l|l|l|l|l||}\hline
Precond. & Time & Nonlinear & Krylov
& Function & Precond. \\
 & & Iterations & Iterations & calls & solves \\ \hline
None & 1260.9u & 3 & 26 & 30 & 0  \\
 &(21:09) & & & &  \\ \hline
FFT  & 983.4u & 2  & 5  & 8  & 7 \\
&(16:31) & & & & \\ \hline
\end{tabular}
\end{table}
\end{verbatim}\end{quote}

\section{Figures}
There are many different ways to incorporate figures into a \LaTeX{}
document.  \LaTeX{} has an internal {\tt picture} environment and
some programs will generate files which are in this format and can
be simply {\tt include}d.  In addition to \LaTeX{} native {\tt picture}
format, additional packages can be loaded in the {\tt\verb|\documentstyle|}
command (or using the {\tt input} command) to allow \LaTeX{} to process
non-native formats such as PostScript.

\subsection{\tt gnuplot}
The graph of Figure~\ref{gelfand.fig2}
 was created by gnuplot. For simple graphs this is a
 great utility.  For example, if you want a sin curve in your thesis
 try the following:
\begin{quote}\tt\singlespace\begin{verbatim}
 (terminal window): gnuplot
 (in gnuplot):
                 set terminal latex
                 set output "foo.tex"
                 plot sin(x)
                 quit
\end{verbatim}\end{quote}
This will generate a file called {\tt foo.tex} which can be read in
with the following statements.
\begin{figure}[htbp]
\centering
\input{fig2.tex}
\caption{Gelfand equation on the ball, $3\leq n \leq 9$.
\label{gelfand.fig2}}
\end{figure}
\begin{quote}\tt\singlespace\begin{verbatim}
\begin{figure}[htbp]
\centering
\input{fig2.tex}
\caption{Gelfand equation on the ball, $3\leq n \leq 9$.
\label{gelfand.fig2}}
\end{figure}
\end{verbatim}\end{quote}
One advantage to using the native \LaTeX{} {\tt picture} environment
is that the fonts will be assured to agree and the pictures can be viewed
in the {\tt .dvi} viewer.

\subsection{PostScript}
Many drawing applications now allow the export of a graphic to the
{\em Encapsulated PostScript} format.  These files have a suffix of
{\tt .EPS} or {\tt .EPSF} and are similar to a regular PostScript
file except that they contain a {\em bounding box} which describes
the dimensions of the figure.

In order to include PostScript figures, the {\tt epsfig} (or {\tt psfig}
depending on the system you are using) style file must be included in either
the {\tt\verb|\documentstyle|} command or the preamble using the {\tt input} command.

Figure~\ref{vwcontr} is a plot from Matlab.
\begin{figure}[htbp]
\centerline{
\psfig{figure=vwcontr.eps,width=5in,angle=0}
           }
\caption{$\sigma$ as a Function of Voltage and Speed, $\alpha = 20$}
\label{vwcontr}
\end{figure}
The commands to include this figure are
\begin{quote}\tt\singlespace\begin{verbatim}
\begin{figure}[htbp]
\centerline{
\psfig{figure=vwcontr.ps,width=5in,angle=0}
           }
\caption{$\sigma$ as a Function of Voltage and Speed, $\alpha = 20$}
\label{vwcontr}
\end{figure}
\end{verbatim}\end{quote}

Observe that the {\tt \verb|\psfig|} command allows the scaling of the figure
by setting either the {\tt width} or {\tt height} of the figure.  If only one
dimension is specified, the other is computed to keep the same aspect ratio.
The figure can also be rotated by setting {\tt angle} to the desired value in
degrees.
           % Chapter 3 Edited from UW Math Dept's Sample Thesis
% bibs.tex
%
% This chapter briefly talks about BibTex and is mostly
% copied from a similar chapter from "How to TeX a Thesis:
% The Purdue Thesis Styles" by James Darrell McCauley and
% Scott Hucker
%

\newcommand{\BibTeX}{{\sc Bib}\TeX}

\chapter{Citations and Bibliographies}
This chapter is an edited form of the same chapter from {\em How to 
\TeX{} a Thesis: The Purdue Thesis Styles} by James Darrell McCauley and
Scott Hucker.

The task of compiling and formatting the sources cited in papers can
be quite tedious, especially for large documents like theses.  A program
separate from \LaTeX{}, called ``\BibTeX{},''can be used to automate this task~\cite{lamport}.

\section{The Citation Command}
When referring to the work of someone else, the {\tt \verb|\cite|} command is used.
This generates the citation in the text for you.  In the above paragraph, the command
{\tt \verb|\cite{lamport}|} was used after the word ``task.''  The formatting of your
citation is handled by either the document style or a style option.  The default citation
style uses the number system (a number in square brackets).  Other citation styles
may use the author-date system, (Lamport, 1986) or the superscript$^3$ system.

\section{Bibliography Styles}
The way that a reference is formatted in your bibliography depends on the bibliography
style, which is specified near the beginning of your document with the\break
{\tt \verb|\bibliographstyle{file}|} command.  The file {\tt file.bst} is the name of the 
bibliography style file.  Standard \BibTeX{} bibliography style files include {\tt plain},
{\tt unsrt}, {\tt alpha}, and {\tt abbrev}.  The bibliography style governs whether or not
references are sorted, whether first names or initials are used for authors, whether or 
not last names are listed first, the location of the year in the references (after the
author or at the end of the reference), {\em etc.}.  You may be required by your
department or major professor to follow as style for a particular journal.  If so, then you
will need to find a \BibTeX{} style file to suit your needs.  Most major journals have
style files.  If you cannot locate an appropriate \BibTeX{} style file, then choose the
one which is closest and then edit the {\tt .bbl} file by hand.  See Section~\ref{BBL}
for a brief discussion on the {\tt .bbl} file.  Some common, but non-standard \BibTeX{}
styles include
\begin{tabbing}
{\tt jacs-new.bstxxxx}\= {\em Journal of the American Chemical Society}\kill
{\tt acm.bst}\>The Association for Computing Machinery\\
{\tt ieeetr.bst}\> The {\em IEEE Transactions} style\\
{\tt jacs-new.bst}\> {\em Journal of the American Chemical Society}
\end{tabbing}

\section{The Database}
The  {\tt \verb|\bibliography{file}|} command is placed in your input file at the location
where the ``List of References'' section\footnote{or ``Bibliography'' 
if {\tt \char92 altbibtitle } has been specified in the preamble.} would be.  It specifies the name (or names) of
your bibliographic data base, {\tt file.bib}.  An example entry in a \BibTeX{}
database is:
\begin{quote}\singlespace\tt\begin{verbatim}
@book{ lamport86 ,
     author =    "Leslie Lamport" ,
     title =     "\LaTeX: A Document Preparation System" ,
     publisher = "Addison--Wesley Pub.\ Co." ,
     year =      "1986" ,
     address =   "Reading, MA" 
}
\end{verbatim}\end{quote}

The citation key is the first field in this entry--- citing this book in a \LaTeX{}
file would look like
\begin{quote}\singlespace\tt\begin{verbatim}
According to Lamport~\cite{lamport86} ...
\end{verbatim}\end{quote}
The tilde ({\tt \verb|~|}) is used to tie the word ``Lamport'' to the citation
generated.  The space between these words is then unbreakable---the word ``Lamport''
and the citation \cite{lamport} will not be split across two lines if they happen to occur
near the end of a line.

A listing of all entry types with their required and optional fields is given in 
Appendix~\ref{bibrefs}. There are several tools which exist to help in editing a \BibTeX{}
file, however, their use is beyond the scope of this manual and can be found by searching
the net.  You can simply use a plain text editor like {\tt vi} or {\tt WordPad} to edit
and create the database files.

There are several rules which you must follow when creating your database.  Authors are
always listed by their full names, first name first, and multiple authors are separated
by {\tt and}.  For example
\begin{quote}\singlespace\tt\begin{verbatim}
author = "John Jay Park and Frederick Gene Watson and
          Michelle Catherine Smith",
\end{verbatim}\end{quote}
If you were using {\tt abbrv} as your {\tt bibliographystyle}, a reference for these
authors may look like:
\begin{quote}
J.J. Park, F.G. Watson, and M.C. Smith \ldots
\end{quote}

Some styles only capitalize the first word of the title.  If you use any acronyms or
other words that should always be capitalized in titles, then they should be 
enclosed in {\tt \{\}}'s ({\em e.g.}, {\tt \{Fortran\}}, {\tt \{N\}ewton}).
This protects the case of these characters.

There are several other rules for \BibTeX{} listed in~\cite{lamport} which should be
referred to because they are not discussed here.

\section{Putting It All Together}
\label{BBL}
To aid the reader in understanding how all of this works together, the following 
excerpt was taken from Lamport~\cite{lamport}:
\begin{quotation}\singlespace
When you ran \LaTeX{} with the input file {\tt sample.tex}, you may have
noticed that \LaTeX{} created a file named {\tt sample.aux}.  This file,
called an {\em auxiliary} file, contains cross-referencing information.  Since
{\tt sample.tex} contains no cross-referencing commands, the auxiliary file it
produces has no information.  However, suppose that \LaTeX{} is run with an
input file named {\tt myfile.tex} that has citations and bibliography-making
[or referencing] commands.  The auxiliary file {\tt myfile.aux} that it produces
will contain all of the citation keys and the arguments of the {\tt \verb|\bibliography|}
and {\tt\verb|\bibliographystyle|} commands.  When \BibTeX{} is run, it reads
this information from the auxiliary file and produces a file named {\tt myfile.bbl}
containing \LaTeX{} commands to produce the source list \ldots The next time
\LaTeX{} is run on {\tt myfile.tex}, the {\tt \verb|\bibliography|} command reads
the {\tt bbl} file ({\tt myfile.bbl}), which generates the source list.
\end{quotation}

Thus, the command sequence for a source file called {\tt main.tex} which is going to
use \BibTeX{} would be:
\begin{quote}\singlespace\tt\begin{verbatim}
latex main.tex
bibtex main
latex main
latex main
\end{verbatim}\end{quote}
The first \LaTeX{} is to collect all of the citations for \BibTeX{}.  Then
\BibTeX{} is run to generate the bibliography.  \LaTeX{} is run again to
incorporate the bibliography into the document and the run the last time to
update any references (like pages in the Table of Contents) which changed when
the bibliography was included.
           % Chapter 4 From PU Thesis styles, by J.D. McCauley
% usage.tex
%
% This file explains how to use the withesis style
%   it is heavily modelled after a similar chapter by McCauley
%   for the Purdue Thesis style
%
% Eric Benedict, May 2000
%
% It is provided without warranty on an AS IS basis.


\chapter{Using the {\tt withesis} Style}

You can get a copy of the \LaTeX{} style for creating a University
of Wisconsin--Madison thesis or dissertation from:

{\tt http://www.cae.wisc.edu/\verb+~+benedict/LaTeX.html}

After somehow unpacking it, you will have the style files ({\tt withesis.sty}
{\tt withe10.sty}, and {\tt withe12.sty}) as well the files used to create
this document.  The files used for this document can be copied and used as a
template for your own thesis or dissertation.

The final printed form of this document is useful, but the
combination of the source code and final copy form a much more valuable
reference.  Keeping a working copy of the this document can be helpful
when you are later working on your thesis or disseration and want to know
how to do something.  If you find a similar example in this document,
then you can simply look at the corresponding source code and add it to
your document.    Because many parts of this document were written by
different people, the styles and techniques are also different and provide
different ways of achieving the same or similar results.

Because of the typical size of theses, it makes sense to break the document
up into several smaller files.  Usually this is done at the chapter level.
These files can then be {\tt \verb|\include|}d in a {\em root} file.  It is
the {\em root} file that you will run \LaTeX{} on.  For this manual, the
root file is called {\tt main.tex}.

\section{The Root File and the Preamble}
The {\tt \verb|\documentclass|} command is used to tell \LaTeX{} that you will
be using the {\tt withesis} document class and it is the first command in your
root file.  Class options such as {\tt 10pt}, {\tt 12pt}, {\tt msthesis} or
{\tt margincheck} are specified here:

{\tt \verb|\documentclass[12pt,msthesis]{withesis}|}

The class option {\tt msthesis} sets the margins to be appropriate for depositing
with the UW library, namely a 1.25 inch left margin with the remaining margins 1 inch.
The defaults for the title page are also defined for a thesis and for a Master of
Science degree.

The class option {\tt margincheck} will place a small black square at the end of
each line which exceeds the margins.\footnote{In reality, the square is
placed at the end of lines which exceed their {\tt \char92hbox}.  This usually
(but not always) indicates a  margin violation on the right margin.  Left
margin violations aren't indicated and if the margin violation is large enough,
there isn't room for the black box to be visiable.}  This is visible both in the {\tt .dvi} file
as well as in the {\tt .ps} file.

The area immediately following this command is called the {\em preamble} and is
used for things like including different style packages,
defining new macros and declaring the page style.

The style packages can be used to easily change the thesis font.  For example,
this document is set in Times Roman instead of the \LaTeX default of Computer
Modern.  This change was performed by including the {\tt times} package:

{\tt\verb|\usepackage{times}|}\footnote{In this document, the typewriter font
{\tt $\backslash$tt} was redefined to use the Computer Modern font with the command
{\tt $\backslash$renewcommand\{$\backslash$ttdefault\}\{cmtt\}}.  
For more information, see~\cite{goossens}.}

Remember that if you change the fonts from the default Computer Modern to
PostScript ({\em e.g.} Times Roman) then in order to correctly see the
document, you will need to convert the {\tt *.dvi} output into a {\tt *.ps}
file and view the document with a PostScript viewer. This is required since 
most {\tt *.dvi} previewer programs cannot 
display PostScript fonts.  Usually, the previewer will substitute
default fonts so the document may be viewed; however, since the alternate
fonts may not be the same size, the formatting of the document may appear
to be incorrect.

The style package for including Postscript figures, {\tt epsfig}, is included with

{\tt\verb|\usepackage{epsfig}|}

If multiple style packages are required, then they can be combined into one statement
as follows:

{\tt\verb|\usepackage{epsfig,times}|}

Many different style packages are available.  For more information, see~\cite{goossens}.

The page styles are defined using a similar method.
A special style is defined for the {\tt withesis} style:

{\tt\verb|\pagestyle{thesisdraft}|}

This style causes the footer text to become:

{\verb| DRAFT: Do Not Distribute        <time><Date>        <input file name>|}

This appears at the bottom of every page.

In addition to the page style command, the {\tt withesis} has defined several useful
commands which are specified in the preamble.  They include {\tt \verb| \draftmargin|},
{\tt \verb|\draftscreen|}, {\tt \verb|\noappendixtables|}, and
{\tt \verb|\noappendixfigures|}.

The command  {\tt \verb|\draftmargin|} draws a PostScript box with the dimensions of
the margins.  This makes it easy to check that the margins are correct and to see if
any of the text or figures are outside of the required margins.  This box is only visible
in the {\tt .ps} file since it is a PostScript special.


The command  {\tt \verb|\draftscreen|} draws a PostScript screen with the word {\em DRAFT}
in light grey and diagonally across the page.  This screen is only visible
in the {\tt .ps} file since it is a PostScript special.

The commands {\tt \verb|\noappendixtables|} and/or {\tt \verb|\noappendixfigures|} should
be used if the appendix does not have either tables or figures respectively.  These commands
inhibit the Appendix Table or Appendix Figure titles in the List of Tables or List of
Figures.\label{usage:noapp}


If you have specified the {\tt psfig} or {\tt epsfig} document style package, then a useful
command is {\tt \verb|\psdraft|}.  This command will show the bounding box that the figure
would occupy (instead of actually including the figure).  This speeds up the draft copy
printing, reduces toner usage and the drawn box is visible in the {\tt .dvi} file.

The next usual command is {\tt \verb|\begin{document}|}.  The following example is part
of the root file used for this manual.

\begin{quote} \singlespace\footnotesize\tt
\begin{verbatim}
\bibliographystyle{plain}
\include{prelude}        % Title page, abstract, table of contents, etc
\include{intro}          % Chapter 1
\include{essentials}     % Edited ``Essential LaTeX'' by Jon Warbrick
\include{figs}           % Chapter 3 Edited from UW Math Dept's Sample Thesis
\include{bibs}           % Chapter 4 From PU Thesis styles, by J.D. McCauley
\include{usage}          % Chapter 5 Strongly based on similar by J.D. McCauley
\bibliography{refs}      % Make the bibliography
\begin{appendices}       % Start of the Appendix Chapters.  If there is only
                         % one Appendix Chapter, then use \begin{appendix}
\include{code}         % Including computer code listings
\include{bibref}         % a BibTeX reference
\include{math}           % Complex Equations from the UW Math Department
\include{acro}           % A discussion on generating PDF files.
\end{appendices}         % End of the Appendix Chapters.  ibid on \end{appendix}
%\include{vita}          % Optional Vita, use \begin{vita} vita text \end{vita}
\end{document}
\end{verbatim}
\end{quote}

\section{Prelude}
After the {\tt \verb|\begin{document}|} comes the preliminary information found in
theses.  In this manual, the information is kept in the file {\tt prelude.tex} (see
above).  These pages will need to be numbered with roman numerals, so use
\begin{quote}\tt\singlespace\begin{verbatim}
\clearpage\pagenumbering{roman}
\end{verbatim}\end{quote}

Next, comes your thesis or dissertation title, your name, date of graduation, department
and degree.
\begin{quote}\tt\singlespace\begin{verbatim}
\title{How to \LaTeX\ a Thesis}
\author{Eric R. Benedict}
\date{2000}
%   - The default degree is ``Doctor of Philosophy''
%     Degree can be changed using the command \degree{}
%\degree{New Degree}
%   - for a PhD dissertation (default), specify \dissertation
%\dissertation
%   - for a masters project report, specify \project
%\project
%   - for a preliminary report, specify \prelim
%\prelim
%   - for a masters thesis, specify \thesis
%\thesis
%   - The default department is ``Electrical Engineering''
%     The department can be changed using the command \department{}
%\department{New Department}
\end{verbatim}\end{quote}

If you specified the class option {\tt msthesis}, then the degree is changed to
{\em Master \break of Science} and the {\tt \verb|\thesis|} option is specified.  If you
want to have the masters margins with another document, then the {\tt \verb|\degree|}
and {\tt \verb|\dissertation|},  {\tt \verb|\project|}, {\em etc.\/} can be specified
as needed.

Once the
above are all defined, use  {\tt \verb|\maketitle|} to generate the title page.
\begin{quote}\tt\singlespace\begin{verbatim}
\maketitle
\end{verbatim}\end{quote}

If you wish to include a copyright page (see Section~\ref{copyright} for
information on registering the copyright.), then add the command
\begin{quote}\tt\singlespace\begin{verbatim}
\copyrightpage
\end{verbatim}\end{quote}
This will generate the proper copyright page and will use the name and date specified
in {\tt \verb|\author{}|} and {\tt \verb|\date{}|}.

Next are the dedications and acknowledgements:
\begin{quote}\tt\singlespace\begin{verbatim}
\begin{dedication}
To my pet rock, Skippy.
\end{dedication}

\begin{acknowledgments}
I thank the many people who have done lots of nice things for me.
\end{acknowledgments}
\end{verbatim}\end{quote}

You must tell \LaTeX{} to generate a table of contents, a list of tables and a list of
figures:
\begin{quote}\tt\singlespace\begin{verbatim}
\tableofcontents
\listoftables
\listoffigures
\end{verbatim}\end{quote}

If you wish to have a nomenclature, list of symbols or glossary it can go here.
\begin{quote}\tt\singlespace\begin{verbatim}
\begin{nomenclature}
%\begin{listofsymbols}
%\begin{glossary}
\begin{tabular}{ll}
$C_1$ & Constant 1\\
\ldots
\end{tabular}
%\end{glossary}
%\end{listofsymbols}
\end{nomenclature}
\end{verbatim}\end{quote}

If your abstract will be microfilmed by Bell and Howell (formerly UMI), then you
will need to generate an abstract of less than 350 words.  This abstract can be created
using the {\tt umiabstract} environment.  This environment requires that you define your
advisor and your advisor's title using {\tt \verb|\advisorname{}|} and
{\tt \verb|\advisortitle{}|}.
\begin{quote}\tt\singlespace\begin{verbatim}
\advisorname{Bucky J. Badger}
\advisortitle{Assistant Professor}
% ABSTRACT
\begin{umiabstract}
\noindent       % Don't indent first paragraph.
This explains the basics for using \LaTeX\ to typeset a
dissertation, thesis or project report for the University of
Wisconsin-Madison.

...

\end{umiabstract}
\end{verbatim}\end{quote}
This will place your name, title and required text at the top of the page and follow the
abstract text with your advisor's name at the bottom for your advisor's signature.  This
page is not numbered and would be submitted separately.

If you will have an abstract as part of your document, then the {\tt abstract} environment
should be used.
\begin{quote}\tt\singlespace\begin{verbatim}
\begin{abstract}
\noindent       % Don't indent first paragraph.
This explains the basics for using \LaTeX\ to typeset a
dissertation, thesis or project report for the University of
Wisconsin-Madison.

...

\end{abstract}
\end{verbatim}\end{quote}
This will generate a page number and it will be included in the Table
of Contents.  

If you will have both the UMI and regular abstracts like this document, then
you will probably want to write the abstract once and save it in a seperate
file such as {\tt abstract.tex}.  Then, you can use the same abstract for
both purposes.

\begin{quote}\begin{verbatim}
\begin{umiabstract}
  \input{abstract}
\end{umiabstract}

\begin{abstract}
  \input{abstract}
\end{abstract}
\end{verbatim}\end{quote}

Finally, the page numbers must be changed to arabic numbers to conclude the preliminary
portion of the document.
\begin{quote}\tt\singlespace\begin{verbatim}
\clearpage\pagenumbering{arabic}
\end{verbatim}\end{quote}

\section{The Body}
At the beginning of {\tt intro.tex} there is the following command:
\begin{quote}\tt\singlespace\begin{verbatim}
\chapter{Introducing the {\tt withesis} \LaTeX{} Style Guide}
\end{verbatim}\end{quote}
Following that is the text of the chapter.  The body of your thesis is separated by
sectioning commands like {\tt \verb|\chapter{}|}.  For more information on the sectioning
commands, see Section~\ref{ess:sectioning}.

Remember the basic rule of outlining you learned in grammar school:
\begin{quote}
You cannot have an `A' if you do not have a `B'
\end{quote}
Take care to have at least two {\tt \verb|\section|}s if you use the command; have
two {\tt \verb|\subsection|}s, {\em etc}.



\section{Additional Theorem Like Environments}
The {\tt withesis} style adds numerous additional theorem like environments.  These
environments were included to allow compatibility with the University of Wisconsin's
Math Department's style file.  These environments are
{\tt theorem}, {\tt assertion}, {\tt claim}, {\tt conjecture}, {\tt corollary},
{\tt definition}, {\tt example}, {\tt figger}, {\tt lemma}, {\tt prop} and {\tt remark}.

As an example, consider the following.
\begin{lemma}
Assuming that $\partial\Omega_2 = \emptyset$ and that $h(t) = 1$, we
have $$
\begin{array}{lr}
\Delta u = f, &  x\in\Omega ,\\[2pt]
u =  g_1, &  x\in\partial\Omega .
\end{array}
$$
\end{lemma}
which was produced with the following:
\begin{quote}\tt\singlespace\begin{verbatim}
\begin{lemma}
Assuming that $\partial\Omega_2 = \emptyset$ and that $h(t) = 1$, we
have $$
\begin{array}{lr}
\Delta u = f, &  x\in\Omega ,\\[2pt]
 u =  g_1, &  x\in\partial\Omega .
\end{array}
$$
\end{lemma}
\end{verbatim}\end{quote}

\section{Bibliography or References}
As a final note, the default title for the references chapter is ``LIST OF REFERENCES.''
Since some people may prefer ``BIBLIOGRAPHY'', the command
\break{\tt \verb|\altbibtitle|}
has been added to change the chapter title.

\section{Appendices}
There are two commands which are available to suppress the writing of the auxiliary information
(to the {\tt .lot} and {\tt .lof} files).  They are:
\begin{quote}\tt\singlespace\begin{verbatim}
\noappendixtables                % Don't have appendix tables
\noappendixfigures               % Don't have appendix figures
\end{verbatim}\end{quote}
These commands should be in the preamble.  See Section~\ref{usage:noapp}.

There are two environments for doing the appendix chapter: {\tt appendix} and
\break {\tt appendices}.  If you have only one chapter in the appendix, use the {\tt appendix}
environment.  If you have more than one chapter, like this manual, use the
{\tt appendices} environment.
\begin{quote}\tt\singlespace\footnotesize\begin{verbatim}
\begin{appendices}  % Start of the Appendix Chapters.  If there is only
                    % one Appendix Chapter, then use \begin{appendix}
\include{code}      % Including computer code listings
\include{bibref}    % a BibTeX reference
\include{math}      % Complex Equations from the UW Math Department

\end{appendices}    % End of the Appendix Chapters. ibid on \end{appendix}
\end{verbatim}\end{quote}
The difference between these two environments is the way that the chapter header is
created and how this is listed in the table of contents.
          % Chapter 5 Strongly based on similar by J.D. McCauley
\bibliography{refs}      % Make the bibliography
\begin{appendices}       % Start of the Appendix Chapters.  If there is only
                         % one Appendix Chapter, then use \begin{appendix}
% code.tex
% this file is part of the example UW-Madison Thesis document
% It demonstrates one method for incorporating program listings
% into a document.

\chapter{Matlab Code} \label{matlab}
This is an example of a Matlab m-file.
\verbatimfile{derivs.m}
         % Including computer code listings
\chapter{Bib\TeX\ Entries}
\label{bibrefs}
The following shows the fields required in all types of Bib\TeX\ entries.
Fields with {\tt OPT} prefixed are optional (the three letters {\tt OPT} should 
not be used).  If an optional field is not used, then the entire field can be deleted.

{\tt
\singlespace
\begin{verbatim}

@Unpublished{,                            @Manual{,
  author =      "",                         title =           "",
  title =       "",                         OPTauthor =       "",
  note =        "",                         OPTorganization = "",
  OPTyear =     "",                         OPTaddress =      "",
  OPTmonth =    ""                          OPTedition =      "",
}                                           OPTyear =         "",
                                            OPTmonth =        "",
@TechReport{,                               OPTnote =         "" 
  author =      "",                       }
  title =       "",
  institution = "",                       @InProceedings{,
  year =        "",                         author =          "",
  OPTtype =     "",                         title =           "",
  OPTnumber =   "",                         booktitle =       "",
  OPTaddress =  "",                         year =            "",
  OPTmonth =    "",                         OPTeditor =       "",
  OPTnote =     ""                          OPTpages =        "",
}                                           OPTorganization = "",
                                            OPTpublisher =    "",
@Proceedings{,                              OPTaddress =      "",
  title =           "",                     OPTmonth =        "",
  year =            "",                     OPTnote =         "" 
  OPTeditor =       "",                   }
  OPTpublisher =    "",
  OPTorganization = "",
  OPTaddress =      "",
  OPTmonth =        "",
  OPTnote =         "" 
}



@PhDThesis{,                              @InCollection{,
  author =      "",                         author =          "",
  title =       "",                         title =           "",
  school =      "",                         booktitle =       "",
  year =        "",                         publisher =       "",
  OPTaddress =  "",                         year =            "",
  OPTmonth =    "",                         OPTeditor =       "",
  OPTnote =     ""                          OPTchapter =      "",
}                                           OPTpages =        "",
                                            OPTaddress =      "",
                                            OPTmonth =        "",
                                            OPTnote =         ""
                                          }

 
@Misc{,                                   @InCollection{,
  OPTauthor =       "",                     author =          "",
  OPTtitle =        "",                     title =           "",
  OPThowpublished = "",                     chapter =         "",
  OPTyear =         "",                     publisher =       "",
  OPTmonth =        "",                     year =            "",
  OPTnote =         ""                      OPTeditor =       "",
}                                           OPTpages =        "",
}                                           OPTvolume =       "",
                                            OPTseries =       "",
                                            OPTaddress =      "",
                                            OPTedition =      "",
                                            OPTmonth =        "",
                                            OPTnote =         ""
                                          }

@MastersThesis{,                          @Article{,
  author =      "",                         author =          "",
  title =       "",                         title =           "",
  school =      "",                         journal =         "",
  year =        "",                         year =            "",
  OPTaddress =  "",                         OPTvolume =       "",
  OPTmonth =    "",                         OPTnumber =       "",
  OPTnote =     ""                          OPTpages =        "",
}                                           OPTmonth =        "",
                                            OPTnote =         ""
                                           }\end{verbatim} }
         % a BibTeX reference
\chapter{Mathematics Examples}
This appendix provides an example of \LaTeX's typesetting
capabilities.  Most of text was obtained from the University of
Wisconsin-Madison Math Department's example thesis file.

\section{Matrices}
The equations for the {\em dq}-model of an induction machine in the
synchronous reference frame are
\begin{eqnarray}
 \left[\begin{array}{c} v_{qs}^e\\v_{ds}^e\\v_{qr}^e\\v_{dr}^e  \end{array}\right]                                                                                                                                                                                                                                                                                                                                                                                                                                                                                                              
 &=& \left[ \begin{array}{cccc}
 r_s + x_s\frac{\rho}{\omega_b} & \frac{\omega_e}{\omega_b}x_s & x_m\frac{\rho}{\omega_b} & \frac{\omega_e}{\omega_b}x_m \\
 -\frac{\omega_e}{\omega_b}x_s & r_s + x_s\frac{\rho}{\omega_b} & -\frac{\omega_e}{\omega_b}x_m & x_m\frac{\rho}{\omega_b} \\
 x_m\frac{\rho}{\omega_b} & \frac{\omega_e -\omega_r}{\omega_b}x_m & r_r'+x_r'\frac{\rho}{\omega_b} & \frac{\omega_e - \omega_r}{\omega_b}x_r' \\
 -\frac{\omega_e -\omega_r}{\omega_b}x_m & x_m\frac{\rho}{\omega_b} & -\frac{\omega_e - \omega_r}{\omega_b}x_r' & r_r' + x_r'\frac{\rho}{\omega_b}
 \end{array} \right]
 \left[\begin{array}{c} i_{qs}^e\\i_{ds}^e\\i_{qr}^e\\i_{dr}^e\end{array} \right] \label{volteq}\\
 T_e&=&\frac{3}{2}\frac{P}{2}\frac{x_m}{\omega_b}\left(i_{qs}^ei_{dr}^e - i_{ds}^ei_{qr}^e\right) \label{torqueeq}\\
 T_e-T_l&=&\frac{2J\omega_b}{P}\frac{d}{dt}\left(\frac{\omega_r}{\omega_b}\right) \label{mecheq}.
\end{eqnarray}

\section{Multi-line Equations}

\LaTeX{} has a built-in equation array feature, however the
equation numbers must be on the same line as an equation.  For example:
\begin{eqnarray}
\Delta u + \lambda e^u &= 0&u\in \Omega,  \nonumber \\
u&=0&u\in\partial\Omega.
\end{eqnarray}

Alternatively, the number can be centered in the equation using the
following method.
%
% The equation-array feature in LaTeX is a bad idea.  For centered
% numbers you should set your own equations and arrays as follows:
%
\def\dd{\displaystyle}
\begin{equation}\label{gelfand}
\begin{array}{rl}
\dd \Delta u + \lambda e^u = 0, &
\dd u\in \Omega,\\[8pt] % add 8pt extra vertical space. 1 line=23pt
\dd u=0, & \dd u\in\partial\Omega.
\end{array}
\end{equation}
The previous equation had a label.  It may be referenced as
equation~(\ref{gelfand}).


\section{More Complicated Equations}
\section*{Rellich's identity}\label{rellich.section}
\setcounter{theorem}{0}
%
%

Standard developments of Pohozaev's identity used an identity by
Rellich~\cite{rellich:der40}, reproduced here.

\begin{lemma}[Rellich]
Given $L$ in divergence form and $a,d$ defined above, $u\in C^2
(\Omega )$, we have
\begin{equation}\label{rellich}
\int_{\Omega}(-Lu)\nabla u\cdot (x-\overline{x})\, dx
= (1-\frac{n}{2}) \int_{\Omega} a(\nabla u,\nabla u) \, dx
-
\frac{1}{2} \int_{\Omega}
d(\nabla u, \nabla u) \, dx
\end{equation}
$$
+
\frac{1}{2} \int_{\partial\Omega} a(\nabla u,\nabla u)(x-\overline{x})
\cdot \nu  \, dS
-
\int_{\partial\Omega}
a(\nabla u,\nu )\nabla u\cdot (x-\overline{x}) \, dS.
$$
\end{lemma}
{\bf Proof:}\\
There is no loss in generality to take $\overline{x} = 0$. First
rewrite $L$:
$$Lu = \frac{1}{2}\left[ \sum_{i}\sum_{j}
\frac{\partial}{\partial x_i}
\left( a_{ij} \frac{\partial u}{\partial x_j} \right) +
\sum_{i}\sum_{j}
\frac{\partial}{\partial x_i}
\left( a_{ij} \frac{\partial u}{\partial x_j} \right)
\right]$$
Switching the order of summation on the second term and relabeling
subscripts, $j \rightarrow i$ and $i \rightarrow j$, then using the fact
that $a_{ij}(x)$ is a symmetric matrix,
gives the symmetric form needed to derive Rellich's identity.
\begin{equation}
Lu = \frac{1}{2} \sum_{i,j}\left[
\frac{\partial}{\partial x_i}
\left( a_{ij} \frac{\partial u}{\partial x_j} \right) +
\frac{\partial}{\partial x_j}
\left( a_{ij} \frac{\partial u}{\partial x_i} \right)
\right].
\end{equation}

Multiplying $-Lu$ by $\frac{\partial u}{\partial x_k} x_k$ and integrating
over $\Omega$, yields
$$\int_{\Omega}(-Lu)\frac{\partial u}{\partial x_k} x_k \, dx=
-\frac{1}{2} \int_{\Omega}
\sum_{i,j}\left[
\frac{\partial}{\partial x_i}
\left( a_{ij} \frac{\partial u}{\partial x_j} \right) +
\frac{\partial}{\partial x_j}
\left( a_{ij} \frac{\partial u}{\partial x_i} \right)
\right]
\frac{\partial u}{\partial x_k} x_k \, dx$$
Integrating by parts (for integral theorems see~\cite[p. 20]
{zeidler:nfa88IIa})
gives
$$= \frac{1}{2} \int_{\Omega}
\sum_{i,j} a_{ij} \left[
\frac{\partial u}{\partial x_j}
\frac{\partial^2 u}{\partial x_k\partial x_i} +
\frac{\partial u}{\partial x_i}
\frac{\partial^2 u}{\partial x_k\partial x_j}
\right] x_k \, dx
$$
$$
+
\frac{1}{2} \int_{\Omega}
\sum_{i,j} a_{ij} \left[
\frac{\partial u}{\partial x_j} \delta_{ik} +
\frac{\partial u}{\partial x_i} \delta_{jk}
\right] \frac{\partial u}{\partial x_k} \, dx
$$
$$- \frac{1}{2} \int_{\partial\Omega}
\sum_{i,j} a_{ij} \left[
\frac{\partial u}{\partial x_j} \nu_{i} +
\frac{\partial u}{\partial x_i} \nu_{j}
\right] \frac{\partial u}{\partial x_k} x_k \, dx
$$
= $I_1 + I_2 + I_3$, where the unit normal vector is $\nu$.
One may rewrite $I_1$ as
$$I_1 = \frac{1}{2} \int_{\Omega}
\sum_{i,j} a_{ij} \frac{\partial}{\partial x_k}\left(
\frac{\partial u}{\partial x_i}
\frac{\partial u}{\partial x_j}
\right) x_k \, dx
$$
Integrating the first term by parts again yields
$$I_1 = -\frac{1}{2} \int_{\Omega}
\sum_{i,j} a_{ij} \left(
\frac{\partial u}{\partial x_i}
\frac{\partial u}{\partial x_j}
\right) \, dx
+
\frac{1}{2} \int_{\partial\Omega}
\sum_{i,j} a_{ij} \left(
\frac{\partial u}{\partial x_i}
\frac{\partial u}{\partial x_j}
\right) x_k \nu_k \, dS
$$
$$
-
\frac{1}{2} \int_{\Omega}
\sum_{i,j} \left(
\frac{\partial u}{\partial x_i}
\frac{\partial u}{\partial x_j}
\right) x_k \frac{\partial a_{ij}}{\partial x_k}\, dx.
$$
Summing over $k$ gives
$$\int_{\Omega}(-Lu)(\nabla u\cdot x)\, dx =
-\frac{n}{2} \int_{\Omega}
\sum_{i,j} a_{ij} \left(
\frac{\partial u}{\partial x_i}
\frac{\partial u}{\partial x_j}
\right) \, dx
$$
$$
+
\frac{1}{2} \int_{\partial\Omega}
\sum_{i,j} a_{ij} \left(
\frac{\partial u}{\partial x_i}
\frac{\partial u}{\partial x_j}
\right) (x\cdot \nu ) \, dS
-
\frac{1}{2} \int_{\Omega}
\sum_{i,j} \left(
\frac{\partial u}{\partial x_i}
\frac{\partial u}{\partial x_j}
\right) (x\cdot  \nabla a_{ij}) \, dx
$$
$$
+
\frac{1}{2} \int_{\Omega}
\sum_{i,j,k} a_{ij} \left[
\frac{\partial u}{\partial x_j}
\frac{\partial u}{\partial x_k} \delta_{ik} +
\frac{\partial u}{\partial x_i}
\frac{\partial u}{\partial x_k} \delta_{jk}
\right] \, dx
$$
$$- \frac{1}{2} \int_{\partial\Omega}
\sum_{i,j} a_{ij} \left[
\frac{\partial u}{\partial x_j} \nu_{i} +
\frac{\partial u}{\partial x_i} \nu_{j}
\right] (\nabla u\cdot x) \, dS.
$$
Combining the first and fourth term on the right-hand side
simplifies the expression
$$\int_{\Omega}(-Lu)(\nabla u\cdot x)\, dx
=
(1-\frac{n}{2}) \int_{\Omega}
\sum_{i,j} a_{ij} \left(
\frac{\partial u}{\partial x_i}
\frac{\partial u}{\partial x_j}
\right) \, dx
$$
$$
+
\frac{1}{2} \int_{\partial\Omega}
\sum_{i,j} a_{ij} \left(
\frac{\partial u}{\partial x_i}
\frac{\partial u}{\partial x_j}
\right) (x\cdot \nu ) \, dS
-
\frac{1}{2} \int_{\Omega}
\sum_{i,j} \left(
\frac{\partial u}{\partial x_i}
\frac{\partial u}{\partial x_j}
\right) (x\cdot  \nabla a_{ij}) \, dx
$$
$$
-
\frac{1}{2} \int_{\partial\Omega}
\sum_{i,j} a_{ij} \left[
\frac{\partial u}{\partial x_j} \nu_{i} +
\frac{\partial u}{\partial x_i} \nu_{j}
\right] (\nabla u\cdot x) \, dS.
$$
Using the notation defined above, the result follows.


           % Complex Equations from the UW Math Department
% --------------------------------------------------------------------------
% the ACRO package
% 
%   Typeset Acronyms
% 
% --------------------------------------------------------------------------
% Clemens Niederberger
% Web:    https://bitbucket.org/cgnieder/acro/
% E-Mail: contact@mychemistry.eu
% --------------------------------------------------------------------------
% Copyright 2011-2017 Clemens Niederberger
% 
% This work may be distributed and/or modified under the
% conditions of the LaTeX Project Public License, either version 1.3
% of this license or (at your option) any later version.
% The latest version of this license is in
%   http://www.latex-project.org/lppl.txt
% and version 1.3 or later is part of all distributions of LaTeX
% version 2005/12/01 or later.
% 
% This work has the LPPL maintenance status `maintained'.
% 
% The Current Maintainer of this work is Clemens Niederberger.
% --------------------------------------------------------------------------
% The acro package consists of the files
%  - acro.sty, acro_en.tex, acro_en.pdf, README
% --------------------------------------------------------------------------
% If you have any ideas, questions, suggestions or bugs to report, please
% feel free to contact me.
% --------------------------------------------------------------------------
\RequirePackage{expl3,xparse,l3keys2e,xtemplate,etoolbox}
\ProvidesExplPackage
  {acro}
  {2017/01/22}
  {2.7a}
  {Typeset Acronyms}

% --------------------------------------------------------------------------
% warning and error messages:
\msg_new:nnn {acro} {undefined}
  {
    You've~ requested~ acronym~ `#1'~ \msg_line_context: \ but~ you~
    apparently~ haven't~ defined~ it,~ yet! \\
    Maybe~ you've~ misspelled~ `#1'?
  }

\msg_new:nnn {acro} {macro}
  {
    A~ macro~ with~ the~ csname~ `#1'~ already~ exists! \\
    Unless~ you~ set~ acro's~ option~ `strict'~ I~ won't~ redefine~ it~
    \msg_line_context: .
  } 

\msg_new:nnn {acro} {replaced}
  {
    The~ #1~ `#2' ~you ~used~ \msg_line_context: \ is~ deprecated~ and~ has~
    been~ replaced~ by~ `#3'. ~Since~ I~ will~ not~ guarantee~ that~ #1~ `#2'~
    will~ be~ kept~ forever~ I~ strongly~ encourage~ you~ to~ switch!
  }

\msg_new:nnn {acro} {deprecated}
  {
    The~ #1~ `#2'~ you~ used~ \msg_line_context: \ is~ deprecated~and~ there~
    is~ no~ replacement.~ Since~ I~ will~ not~ guarantee~ that~ #1~ `#2'~
    will~ be~ kept~ forever~ I~ strongly~ encourage~ you~ to~ remove~ it~
    from~ your~ document.
  }

\msg_new:nnn {acro} {substitute-short}
  {
    There~ is~ no~ short~ form~ set~ for~ acronym~ `#1'! \\
    I~ am~ setting~ the~ short~ form~ equal~ to~ the~ ID~ `#1'. \\
    If~ that~ is~ not~ what~ you~ want~ make~ sure~ to~ add~ an~ explicit~
    short~ form.
  }
\msg_new:nnn {acro} {ending-exists}
  {
    An~ ending~ with~ the~ name~ `#1'~ already~ exists! \\ \\
    I~ am~ overwriting~ the~ defaults.
  }

\msg_new:nnn {acro} {ending-before-acronyms}
  {
    You~ are~ using~ \token_to_str:N \ProvideAcroEnding \ after~ you've~
    declared~ at~ least~ one~ acronym.~ This~ will~ lead~ to~ trouble! \\
    Make~ sure~ to~ define~ endings~ before~ *any*~ acronym~ declarations!
  }

\msg_new:nnn {acro} {no-alternative}
  {
    There~ is~ no~ alternative~ form~ for~ acronym~ `#1'! \\ \\
    I~ am~ using~ the~ short~ form~ instead.
  }

\msg_new:nnn {acro} {unknown}
  {
    You're~ trying~ to~ use~ the~ #1~ `#2'~ \msg_line_context: . \\
    However,~ I~ do~ not~ know~ #1~ `#2'! \\
    If~ this~ is~ no~ typo~ please~ contact~ the~ package~ author. \\ \\
    I~ am~ going~ to~ use~ the~ #1~ `#3'~ instead.
  }
\msg_new:nnn {acro} {definitions-missing}
  {
    I~ cannot~ find~ the~ file~ \c_acro_definition_file_name_tl
    .\c_acro_definition_file_extension_tl !~ This~ file~ contains~ all~
    essential~ user~ commands~ of~ acro~ and~ is~ a~ crucial~ part~ of~ the~
    package!~ Please~ check~ your~ installation.
  }

% --------------------------------------------------------------------------
% logging:
\cs_new:Npn \acro_if_log:T #1 { \use:n {#1} }

\bool_new:N \l__acro_log_acronyms_bool
\bool_new:N \l__acro_log_acronyms_verbose_bool

\keys_define:nn {acro}
  {
    log           .choice: ,
    log / true    .code:n    =
      \bool_set_true:N \l__acro_log_acronyms_bool
      \bool_set_false:N \l__acro_log_acronyms_verbose_bool ,
    log / silent  .meta:n    = { log = true } ,
    log / verbose .code:n    =
      \bool_set_true:N \l__acro_log_acronyms_bool
      \bool_set_true:N \l__acro_log_acronyms_verbose_bool ,
    log / false   .code:n    =
      \bool_set_false:N \l__acro_log_acronyms_bool
      \bool_set_false:N \l__acro_log_acronyms_verbose_bool ,
    log           .default:n = true ,
    log           .initial:n = false
  }

\cs_new:Npn \__acro_write_log:nn #1#2 { \ \ \ #1 ~ = ~ {#2} }
\cs_new:Npn \__acro_write_log_property:nnn #1#2#3
  { \__acro_write_log:nn {#2} { \__acro_get_property:nn {#3} {#1} } }

\cs_new:Npn \__acro_ending_log_entry:nn #1#2
  {
    | \\
    | \__acro_write_log_property:nnn {#1} {short-#2} {short_#2} \\
    | \__acro_write_log_property:nnn {#1} {short-#2-form} {short_#2_form} \\
    | \__acro_write_log_property:nnn {#1} {long-#2} {long_#2} \\
    | \__acro_write_log_property:nnn {#1} {long-#2-form} {long_#2_form} \\
    | \__acro_write_log_property:nnn {#1} {alt-#2} {alt_#2} \\
    | \__acro_write_log_property:nnn {#1} {alt-#2-form} {alt_#2_form} \\
  }
  
\msg_new:nnn {acro} {log-acronym-verbose}
  {
    ================================================= \\
    | ~ \msg_info_text:n {acro}~ --~ defining~ new~ acronym: \\
    | \__acro_write_log:nn {ID} {#1} \\
    | \__acro_write_log_property:nnn {#1} {short} {short} \\
    | \__acro_write_log_property:nnn {#1}{long} {long} \\
    | \__acro_write_log_property:nnn {#1}{alt} {alt} \\
    | \__acro_write_log_property:nnn {#1}{sort} {sort} \\
    | \__acro_write_log_property:nnn {#1}{class} {class} \\
    | \__acro_write_log_property:nnn {#1} {list} {list} \\
    | \__acro_write_log_property:nnn {#1} {extra} {extra} \\
    | \__acro_write_log_property:nnn {#1} {foreign} {foreign} \\
    | \__acro_write_log_property:nnn {#1} {single} {single} \\
    | \__acro_write_log_property:nnn {#1} {pdfstring} {pdfstring} \\
    | \__acro_write_log_property:nnn {#1} {accsupp} {accsupp} \\
    | \__acro_write_log_property:nnn {#1} {tooltip} {tooltip} \\
    | \\
    | \__acro_write_log_property:nnn {#1} {short-indefinite} {short_indefinite} \\
    | \__acro_write_log_property:nnn {#1} {long-indefinite} {long_indefinite} \\
    | \__acro_write_log_property:nnn {#1} {alt-indefinite} {alt_indefinite} \\
    \seq_map_function:NN \l__acro_endings_seq \__acro_ending_log_entry:n
    | \\
    | \__acro_write_log_property:nnn {#1} {short-format} {short_format} \\
    | \__acro_write_log_property:nnn {#1} {long-format} {long_format} \\
    | \__acro_write_log_property:nnn {#1} {first-long-format} {first_long_format} \\
    | \__acro_write_log_property:nnn {#1} {single-format} {single_format} \\
    | \__acro_write_log_property:nnn {#1} {foreign-lang} {foreign_lang} \\    
    | \\
    | \__acro_write_log_property:nnn {#1} {cite} {citation} \\
    | \__acro_write_log_property:nnn {#1} {index} {index} \\
    | \__acro_write_log_property:nnn {#1} {index-sort} {index_sort} \\
    | \\
    | \__acro_write_log_property:nnn {#1} {long-pre} {long_pre} \\
    | \__acro_write_log_property:nnn {#1} {long-post} {long_post} \\    
    | \__acro_write_log_property:nnn {#1} {index-cmd} {index_cmd} \\
    | \__acro_write_log_property:nnn {#1} {first-style} {first_style} \\
    =================================================
  }

\msg_new:nnn {acro} {log-acronym-silent}
  {
    ================================================= \\
    | ~ \msg_info_text:n {acro}~ --~ defining~ new~ acronym: \\
    | \__acro_write_log:nn {ID} {#1} \\
    | \__acro_write_log_property:nnn {#1} {short} {short} \\
    | \__acro_write_log_property:nnn {#1} {long} {long} \\
    | \__acro_write_log_property:nnn {#1} {alt} {alt} \\
    | \__acro_write_log_property:nnn {#1} {sort} {sort} \\
    | \__acro_write_log_property:nnn {#1} {class} {class} \\
    | \__acro_write_log_property:nnn {#1} {list} {list} \\
    | \__acro_write_log_property:nnn {#1} {extra} {extra} \\
    | \__acro_write_log_property:nnn {#1} {foreign} {foreign} \\
    | \__acro_write_log_property:nnn {#1} {pdfstring} {pdfstring} \\
    | \__acro_write_log_property:nnn {#1} {cite} {citation} \\
    =================================================
  }
  
\cs_new_protected:Npn \__acro_log_acronym:n #1
  {
    \bool_if:NT \l__acro_log_acronyms_bool
      {
        \cs_set:Npn \__acro_ending_log_entry:n ##1
          { \__acro_ending_log_entry:nn {#1} {##1} }
        \bool_if:NTF \l__acro_log_acronyms_verbose_bool
          { \msg_log:nnn {acro} {log-acronym-verbose} {#1} }
          { \msg_log:nnn {acro} {log-acronym-silent} {#1} }
      }
  }   

% --------------------------------------------------------------------------
% message macros:
\cs_new:Npn \__acro_remove_backslash:N #1
  { \exp_after:wN \use_none:n \token_to_str:N #1 }

\cs_new_protected:Npn \acro_new_message_commands:Nnnn #1#2#3#4
  {
    \clist_map_inline:nn {#2}
      {
        \cs_new_protected:cpn { \__acro_remove_backslash:N #1 ##1 }
          {
            \bool_if:NTF \l__acro_silence_bool
              { \use:c { \__acro_remove_backslash:N #3 n##1 } {acro} }
              { \use:c { \__acro_remove_backslash:N #4 n##1 } {acro} }
          }
      }
  }

\acro_new_message_commands:Nnnn \acro_serious_message: {n,nn,nnn}
  { \msg_warning: }
  { \msg_error: }

\acro_new_message_commands:Nnnn \acro_harmless_message: {n,nn,nnn,nnnn}
  { \msg_info: }
  { \msg_warning: }

\cs_new_protected:Npn \acro_option_deprecated:nn #1#2
  {
    \tl_if_blank:nTF {#2}
      { \acro_harmless_message:nnn  {deprecated} {option} {#1} }
      { \acro_harmless_message:nnnn {replaced}   {option} {#1} {#2} }
  }
\cs_new_protected:Npn \acro_option_deprecated:n #1
  { \acro_option_deprecated:nn {#1} {} }

\cs_new_protected:Npn \acro_command_deprecated:NN #1#2
  {
    \tl_if_blank:nTF {#2}
      {
        \acro_harmless_message:nnn {deprecated} {command}
          { \token_to_str:N #1 }
      }
      {
        \acro_harmless_message:nnnn {replaced} {command}
          { \token_to_str:N #1 }
          { \token_to_str:N #2 }
      }
  }

% --------------------------------------------------------------------------
% temporary variables
\tl_new:N   \l__acro_tmpa_tl
\tl_new:N   \l__acro_tmpb_tl
\tl_new:N   \l__acro_tmpc_tl
\prop_new:N \l__acro_tmpa_prop
\prop_new:N \l__acro_tmpb_prop
\seq_new:N  \l__acro_tmpa_seq
\seq_new:N  \l__acro_tmpb_seq
\int_new:N  \l__acro_tmpa_int
\int_new:N  \l__acro_tmpb_int
\int_new:N  \l__acro_tmpc_int
\int_new:N  \l__acro_tmpd_int

% --------------------------------------------------------------------------
% variants of kernel commands
\cs_generate_variant:Nn \quark_if_no_value:nTF  { V }
\cs_generate_variant:Nn \tl_put_right:Nn        { NV, Nv }
\cs_generate_variant:Nn \tl_if_eq:nnT           { V }
\cs_generate_variant:Nn \tl_if_eq:nnF           { V }
\cs_generate_variant:Nn \seq_use:Nnnn           { c }
\cs_generate_variant:Nn \seq_gset_split:Nnn     { c }
\cs_generate_variant:Nn \seq_set_split:Nnn      { NnV }
\cs_generate_variant:Nn \seq_if_in:NnT          { NV }
\cs_generate_variant:Nn \prop_put:Nnn           { NnV, cnx, cnv }
\cs_generate_variant:Nn \prop_get:NnNTF         { cnc }
\cs_generate_variant:Nn \prop_get:NnNF          { cn, cnc }
\cs_generate_variant:Nn \prop_get:NnN           { cnc }
\cs_generate_variant:Nn \cs_generate_variant:Nn { c }
\cs_generate_variant:Nn \str_case:nn            { V }

% --------------------------------------------------------------------------
% variables:
\bool_new:N      \l__acro_silence_bool
\bool_new:N      \l__acro_mark_as_used_bool
\bool_new:N      \g__acro_mark_first_as_used_bool
\bool_new:N      \l__acro_use_single_bool
\bool_new:N      \l__acro_print_only_used_bool
\bool_set_true:N \l__acro_print_only_used_bool
\bool_new:N      \l__acro_hyperref_loaded_bool
\bool_new:N      \l__acro_use_hyperref_bool
\bool_new:N      \l__acro_xspace_bool
\bool_new:N      \l__acro_custom_format_bool
\bool_new:N      \l__acro_strict_bool
\bool_new:N      \l__acro_create_macros_bool
\bool_new:N      \l__acro_first_upper_bool
\bool_new:N      \l__acro_indefinite_bool
\bool_new:N      \l__acro_upper_indefinite_bool
\bool_new:N      \l__acro_foreign_bool
\bool_set_true:N \l__acro_foreign_bool
\bool_new:N      \l__acro_sort_bool
\bool_set_true:N \l__acro_sort_bool
\bool_new:N      \l__acro_capitalize_list_bool
\bool_new:N      \l__acro_citation_all_bool
\bool_new:N      \l__acro_citation_first_bool
\bool_set_true:N \l__acro_citation_first_bool
\bool_new:N      \l__acro_group_citation_bool
\bool_new:N      \l__acro_acc_supp_bool
\bool_new:N      \l__acro_tooltip_bool
\bool_new:N      \l__acro_inside_tooltip_bool
\bool_new:N      \l__acro_following_page_bool
\bool_new:N      \l__acro_following_pages_bool
\bool_new:N      \l__acro_addto_index_bool
\bool_new:N      \l__acro_is_excluded_bool
\bool_new:N      \l__acro_is_included_bool
\bool_new:N      \l__acro_page_punct_bool
\bool_new:N      \l__acro_page_brackets_bool
\bool_new:N      \l__acro_page_display_bool
\bool_new:N      \l__acro_new_page_numbering_bool
\bool_new:N      \l__acro_first_use_brackets_bool
\bool_new:N      \l__acro_first_only_short_bool
\bool_new:N      \l__acro_first_only_long_bool
\bool_new:N      \l__acro_first_reversed_bool
\bool_new:N      \l__acro_first_switched_bool
\bool_new:N      \l__acro_use_note_bool
\bool_new:N      \l__acro_extra_punct_bool
\bool_new:N      \l__acro_extra_use_brackets_bool
\bool_new:N      \l__acro_in_list_bool
\bool_new:N      \l__acro_place_label_bool
\bool_new:N      \l__acro_list_all_pages_bool
\bool_set_true:N \l__acro_list_all_pages_bool
\bool_new:N      \g__acro_first_acronym_declared_bool
\bool_new:N      \l__acro_include_endings_format_bool
\bool_new:N      \l__acro_list_reverse_long_extra_bool
\bool_new:N      \l__acro_use_acronyms_bool
\bool_set_true:N \l__acro_use_acronyms_bool

\tl_new:N   \l__acro_ignore_tl
\tl_new:N   \l__acro_default_indefinite_tl
\tl_set:Nn  \l__acro_default_indefinite_tl {a}
\tl_new:N   \l__acro_foreign_sep_tl
\tl_new:N   \l__acro_extra_instance_tl
\tl_set:Nn  \l__acro_extra_instance_tl {default}
\tl_new:N   \l__acro_page_instance_tl
\tl_set:Nn  \l__acro_page_instance_tl  {none}
\tl_new:N   \l__acro_page_name_tl
\tl_new:N   \l__acro_pages_name_tl
\tl_new:N   \l__acro_next_page_tl
\tl_new:N   \l__acro_next_pages_tl
\tl_new:N   \l__acro_list_instance_tl
\tl_set:Nn  \l__acro_list_instance_tl  {description}
\tl_new:N   \l__acro_list_type_tl
% \tl_new:N   \l__acro_list_tl
\tl_new:N   \l__acro_list_heading_cmd_tl
\tl_set:Nn  \l__acro_list_heading_cmd_tl {section*}
\tl_new:N   \l__acro_list_name_tl
\tl_new:N   \l__acro_list_before_tl
\tl_new:N   \l__acro_list_after_tl
\tl_new:N   \l__acro_custom_format_tl
\tl_new:N   \l__acro_first_between_tl
\tl_new:N   \l__acro_citation_connect_tl
\tl_new:N   \l__acro_between_group_connect_citation_tl
\tl_new:N   \l__acro_extra_brackets_tl
\tl_new:N   \l__acro_extra_punct_tl
\tl_new:N   \l__acro_first_brackets_tl
\tl_new:N   \l__acro_page_punct_tl
\tl_new:N   \l__acro_page_brackets_tl
\tl_new:N   \l__acro_last_page_tl
\tl_new:N   \l__acro_current_page_tl
\tl_new:N   \l__acro_list_table_tl
\tl_new:N   \l__acro_list_table_spec_tl
\tl_new:N   \l__acro_acc_supp_tl
\tl_new:N   \l__acro_acc_supp_options_tl
\tl_new:N   \l__acro_label_prefix_tl
\tl_set:Nn  \l__acro_label_prefix_tl { ac: }
\tl_new:N   \l__acro_index_short_tl
\tl_new:N   \l__acro_first_instance_tl
\tl_set:Nn  \l__acro_first_instance_tl {default}

\tl_new:N   \l__acro_short_tl
\tl_new:N   \l__acro_short_format_tl
\tl_new:N   \l__acro_list_short_format_tl

\tl_new:N   \l__acro_alt_tl
\tl_new:N   \l__acro_alt_format_tl

\tl_new:N   \l__acro_long_tl
\tl_new:N   \l__acro_list_long_format_tl

\tl_new:N   \l__acro_single_form_tl
\tl_set:Nn  \l__acro_single_form_tl {long}

\tl_new:N   \l__acro_extra_format_tl

\tl_new:N   \l__acro_foreign_format_tl
\tl_new:N   \l__acro_foreign_list_format_tl
\tl_set:Nn  \l__acro_foreign_list_format_tl { \acroenparen }

\tl_new:N   \l__acro_index_format_tl


\skip_new:N  \l__acro_page_space_skip

\dim_new:N  \l__acro_short_width_dim
\dim_set:Nn \l__acro_short_width_dim {3em}

\prop_new:N \l__acro_list_styles_prop
\prop_new:N \l__acro_list_headings_prop
\prop_new:N \l__acro_first_styles_prop
\prop_new:N \l__acro_extra_styles_prop
\prop_new:N \l__acro_page_styles_prop

% --------------------------------------------------------------------------
% small commands for use at various places
\cs_new:Npn \acro_no_break: { \tex_penalty:D \c_ten_thousand }

\cs_new_protected:Npn \__acro_first_upper_case:n #1
  { \tl_upper_case:n { \tl_head:n {#1} } \tl_tail:n {#1} }
\cs_generate_variant:Nn \__acro_first_upper_case:n { x }
\cs_generate_variant:Nn \tl_mixed_case:n { x }

\cs_new_eq:NN \acro_first_upper_case:n \__acro_first_upper_case:n

\NewDocumentCommand \acfirstupper { m }
  { \acro_first_upper_case:n {#1} }

% --------------------------------------------------------------------------
% options:
\keys_define:nn {acro}
  {
    messages          .choice: ,
    messages / silent .code:n     =
      \bool_set_true:N \l__acro_silence_bool ,
    messages / loud .code:n       =
      \bool_set_false:N \l__acro_silence_bool ,
    messages          .value_required:n = true ,
    accsupp           .bool_set:N = \l__acro_acc_supp_bool ,
    accsupp-options   .tl_set:N   = \l__acro_acc_supp_options_tl ,
    tooltip           .bool_set:N = \l__acro_tooltip_bool ,
    tooltip-cmd       .code:n     = \cs_set:Npn \__acro_tooltip_cmd:nn {#1} ,
    tooltip-cmd       .value_required:n = true ,
    macros            .bool_set:N = \l__acro_create_macros_bool ,
    xspace            .bool_set:N = \l__acro_xspace_bool ,
    % xspace            .code:n     = \acro_option_deprecated:nn {xspace} {} ,
    strict            .bool_set:N = \l__acro_strict_bool ,
    sort              .bool_set:N = \l__acro_sort_bool ,
    short-format      .code:n     =
      \tl_set:Nn \l__acro_short_format_tl {#1}
      \tl_set_eq:NN \l__acro_alt_format_tl \l__acro_short_format_tl
      \tl_set:Nn \l__acro_list_short_format_tl {#1} ,
    short-format      .value_required:n = true ,
    long-format       .code:n     =
      \tl_set:Nn \l__acro_long_format_tl {#1}
      \tl_set:Nn \l__acro_first_long_format_tl {#1}
      \tl_set:Nn \l__acro_list_long_format_tl {#1} ,
    long-format       .value_required:n = true ,
    first-long-format .code:n     =
      \tl_set:Nn \l__acro_first_long_format_tl {#1} ,
    first-long-format .value_required:n = true ,
    single-format     .tl_set:N   = \l__acro_single_format_tl ,
    single-format     .value_required:n = true ,
    format-include-endings .bool_set:N = \l__acro_include_endings_format_bool ,
    display-foreign   .bool_set:N = \l__acro_foreign_bool ,
    foreign-format    .tl_set:N   = \l__acro_foreign_format_tl ,
    foreign-format    .value_required:n = true ,
    list-short-format .tl_set:N   = \l__acro_list_short_format_tl ,
    list-short-format .value_required:n = true ,
    list-short-width  .dim_set:N  = \l__acro_short_width_dim ,
    list-short-width  .value_required:n = true ,
    list-long-format  .tl_set:N   = \l__acro_list_long_format_tl ,
    list-long-format  .value_required:n = true ,
    list-foreign-format .tl_set:N = \l__acro_foreign_list_format_tl ,
    list-foreign-format .value_required:n = true ,
    extra-format      .tl_set:N   = \l__acro_extra_format_tl ,
    extra-format      .value_required:n = true ,
    single            .bool_set:N = \l__acro_use_single_bool ,
    single-form       .tl_set:N   = \l__acro_single_form_tl ,
    single-form       .value_required:n = true ,
    first-style       .code:n     = \acro_set_first_style:n {#1} ,
    first-style       .value_required:n = true ,
    extra-style       .code:n     = \acro_set_extra_style:n {#1} ,
    extra-style       .value_required:n = true ,
    label             .bool_set:N = \l__acro_place_label_bool ,
    label-prefix      .tl_set:N   = \l__acro_label_prefix_tl ,
    label-prefix      .value_required:n = true ,
    pages             .choice: ,
    pages / all       .code:n     =
      \bool_set_true:N \l__acro_list_all_pages_bool ,
    pages / first     .code:n     =
      \bool_set_true:N \l__acro_place_label_bool
      \bool_set_false:N \l__acro_list_all_pages_bool ,
    pages             .value_required:n = true ,
    page-ref          .code:n     =
      \acro_option_deprecated:nn {page-ref} {page-style}
      \acro_set_page_style:n {#1} ,
    page-style        .code:n     = \acro_set_page_style:n {#1} ,
    page-style        .value_required:n = true ,
    page-name         .tl_set:N   = \l__acro_page_name_tl ,
    page-name         .value_required:n = true ,
    pages-name        .tl_set:N   = \l__acro_pages_name_tl ,
    pages-name        .value_required:n = true ,
    following-page    .bool_set:N = \l__acro_following_page_bool ,
    following-pages   .bool_set:N = \l__acro_following_pages_bool ,
    following-pages*  .meta:n     =
      { following-page = #1 , following-pages = #1 } ,
    following-pages*  .default:n  = true ,
    next-page         .tl_set:N   = \l__acro_next_page_tl ,
    next-page         .value_required:n = true ,
    next-pages        .tl_set:N   = \l__acro_next_pages_tl ,
    next-pages        .value_required:n = true ,
    list-style        .code:n     = \acro_set_list_style:n {#1} ,
    list-style        .value_required:n = true ,
    list-heading      .code:n     = \acro_set_list_heading:n {#1} ,
    list-heading      .value_required:n = true ,
    list-name         .tl_set:N   = \l__acro_list_name_tl ,
    list-name         .value_required:n = true ,
    hyperref          .bool_set:N = \l__acro_use_hyperref_bool ,
    only-used         .bool_set:N = \l__acro_print_only_used_bool ,
    mark-as-used      .choice: ,
    mark-as-used / first .code:n  =
      \bool_gset_true:N \g__acro_mark_first_as_used_bool ,
    mark-as-used / any   .code:n  =
      \bool_gset_false:N \g__acro_mark_first_as_used_bool ,
    mark-as-used      .default:n  = any ,
    list-caps         .bool_set:N = \l__acro_capitalize_list_bool ,
    cite              .choice: ,
    cite / all        .code:n     =
      \bool_set_true:N \l__acro_citation_all_bool
      \bool_set_true:N \l__acro_citation_first_bool ,
    cite / none       .code:n     =
      \bool_set_false:N \l__acro_citation_all_bool
      \bool_set_false:N \l__acro_citation_first_bool ,
    cite / first      .code:n     =
      \bool_set_false:N \l__acro_citation_all_bool
      \bool_set_true:N  \l__acro_citation_first_bool ,
    cite              .default:n  = all ,
    cite-cmd          .code:n     =
      \cs_set:Npn \__acro_citation_cmd:w {#1} ,
    cite-cmd          .value_required:n = true ,
    group-cite-cmd    .code:n     =
      \cs_set:Npn \__acro_group_citation_cmd:w {#1} ,
    group-cite-cmd    .value_required:n = true ,
    group-citation    .bool_set:N = \l__acro_group_citation_bool ,
    cite-connect      .tl_set:N   = \l__acro_citation_connect_tl ,
    cite-connect      .initial:n  = \nobreakspace ,
    cite-connect      .value_required:n = true ,
    group-cite-connect .tl_set:N = \l__acro_between_group_connect_citation_tl ,
    group-cite-connect .initial:n = {,\nobreakspace} ,
    group-cite-connect .value_required:n = true ,
    index             .bool_set:N = \l__acro_addto_index_bool ,
    index-cmd         .code:n     =
      \cs_set:Npn \__acro_index_cmd:n {#1} ,
    index-cmd         .value_required:n = true ,
    uc-cmd            .code:n     =
      \cs_set_eq:NN \__acro_first_upper_case:n #1 ,
    uc-cmd            .value_required:n = true
  }

\AtBeginDocument
  {
    \bool_if:NTF \l__acro_xspace_bool
      {
        \@ifpackageloaded {xspace}
          { }
          { \RequirePackage {xspace} }
        \cs_new_eq:NN \acro_xspace: \xspace
      }
      { \cs_new:Npn \acro_xspace: {} }
  }

% --------------------------------------------------------------------------
% setup command:
\NewDocumentCommand \acsetup { m }
  { \keys_set:nn {acro} {#1} \ignorespaces }

% --------------------------------------------------------------------------
% we use xtemplate for different object types and with a different number of
% arguments; let's declare functions for usage later so we don't have to
% bother

% objects with one argument:
\cs_new_protected:Npn \acro_page_number_instance:nn #1#2
  { \UseInstance {acro-page-number} {#1} {#2} }
\cs_generate_variant:Nn \acro_page_number_instance:nn {V}

\cs_new_protected:Npn \acro_extra_instance:nn #1#2
  { \UseInstance {acro-extra} {#1} {#2} }
\cs_generate_variant:Nn \acro_extra_instance:nn {VV}

\cs_new_protected:Npn \acro_title_instance:nn #1#2
  { \UseInstance {acro-title} {#1} {#2} }
\cs_generate_variant:Nn \acro_title_instance:nn {VV}

% objects with two arguments:
\cs_new_protected:Npn \acro_list_instance:nnn #1#2#3
  { \UseInstance {acro-list} {#1} {#2} {#3} }
\cs_generate_variant:Nn \acro_list_instance:nnn {VVV}

\cs_new_protected:Npn \acro_first_instance:nn #1#2
  {
    \tl_if_blank:VF \l__acro_first_style_tl
      {
        \tl_set_eq:NN
          \l__acro_first_instance_tl
          \l__acro_first_style_tl
      }
    \acro_if_defined:nT {#1}
      {
        \use:x {
          \UseInstance {acro-first}
            { \exp_not:V \l__acro_first_instance_tl }
            { \exp_not:n {#1} }
            { \exp_not:n {#2} }
          }
      }
  }
\cs_generate_variant:Nn \acro_first_instance:nn {nV}
  
% --------------------------------------------------------------------------
% hyperref support
\cs_new_eq:NN \acro_hyper_target:nn \use_ii:nn
\cs_new_eq:NN \acro_hyper_link:nn   \use_ii:nn

\cs_new_protected:Npn \acro_activate_hyperref_support:
  {
    \bool_if:nT { \l__acro_hyperref_loaded_bool && \l__acro_use_hyperref_bool }
      {
        \cs_set_eq:NN \acro_hyper_link:nn \hyperlink
        \cs_set:Npn \acro_hyper_target:nn ##1##2
          { \raisebox { 3ex } [ 0pt ] { \hypertarget {##1} { } } ##2 }
      }
  }

% #1: tl var
% #2: id
% #3: text
\cs_new_protected:Npn \__acro_make_link:Nnn #1#2#3
  {
    \bool_if:nTF
      { \l__acro_use_hyperref_bool && \l__acro_hyperref_loaded_bool }
      {
        \tl_set:Nn #1
           {
             \acro_hyper_link:nn {#2} { \phantom {#3} }
             \acro_if_is_single:nTF {#2}
               { \hbox_overlap_left:n {#3} }
               { \acro_color_link:n { \hbox_overlap_left:n {#3} } }
           }
       }
       { \tl_set:Nn #1 {#3} }
  }
\cs_generate_variant:Nn \__acro_make_link:Nnn {NnV}

\cs_new:Npn \acro_color_link:n #1
  {
    \cs_if_exist:NTF \hypersetup
      {
        \ifHy@colorlinks
          \exp_after:wN \use_i:nn
        \else
          \ifHy@ocgcolorlinks
            \exp_after:wN \use_i:nn
          \else
            \exp_after:wN \exp_after:wN \exp_after:wN \use_ii:nn
          \fi
        \fi
        { \textcolor { \@linkcolor } {#1} }
        {#1}
      }
      {#1}
  }

\AtBeginDocument{
  \cs_if_exist:NF \textcolor { \cs_new_eq:NN \textcolor \use_ii:nn }
}

% --------------------------------------------------------------------------
% output style of the first time an acronym is used

% helper macros for the styles
% #1: short|long
% #2: id
% #3: long
\cs_new_protected:Npn \__acro_print_form_and_indefinite:nnn #1#2#3
  {
    \group_begin:
      \acro_for_all_trailing_tokens_do:n
        { \acro_deactivate_trailing_action:n {##1} }
      \str_case:nn {#1}
        {
          {long} {
            \bool_if:nT
              {
                \l__acro_first_only_long_bool ||
                !\l__acro_first_only_short_bool
              }
              {
                \acro_write_indefinite:nn {#2} {long}
                \acro_write_expanded:nnn {#2} {first-long} {#3}
              }
          }
          {short} {
            \bool_if:nT
              {
                !\l__acro_first_only_long_bool ||
                \l__acro_first_only_short_bool
              }
              {
                \acro_write_indefinite:nn {#2} {short}
                \acro_write_compact:nn {#2} {short}
              }
          }
        }
    \group_end:
  }

\cs_new_protected:Npn \__acro_open_bracket:
  {
    \bool_if:nT
      {
        !\l__acro_first_only_long_bool &&
        !\l__acro_first_only_short_bool
      }
      {
        \acro_space:
        \tl_if_blank:VF \l__acro_first_between_tl
          {
            \tl_use:N \l__acro_first_between_tl
            \acro_space:
          }
        \bool_if:NT \l__acro_first_use_brackets_bool
          { \tl_head:N \l__acro_first_brackets_tl }
      }
  }

\cs_new_protected:Npn \__acro_close_bracket:
  {
    \bool_if:nT
      {
        \l__acro_first_use_brackets_bool &&
        !\l__acro_first_only_short_bool &&
        !\l__acro_first_only_long_bool
      }
      { \tl_tail:N \l__acro_first_brackets_tl }
  }
  
% #1: short|long
% #2: id
% #3: long
\cs_new_protected:Npn \__acro_print_form:nnn #1#2#3
  {
    \str_case:nn {#1}
      {
        {long} {
          \bool_if:nT
            {
              \l__acro_first_only_long_bool ||
              !\l__acro_first_only_short_bool
            }
            { \acro_write_expanded:nnn {#2} {first-long} {#3} }
        }
        {short} {
          \bool_if:nT
            {
              !\l__acro_first_only_long_bool ||
              \l__acro_first_only_short_bool
            }
            { \acro_write_compact:nn {#2} {short} }
        }
      }
  }

% #1: id
\cs_new_protected:Npn \__acro_foreign_sep:n #1
  {
    \bool_if:nT
      {
         \l__acro_foreign_bool &&
        !\l__acro_first_only_short_bool &&
        !\l__acro_first_only_long_bool
      }
      { \acro_if_foreign:nT {#1} { \tl_use:N \l__acro_foreign_sep_tl } }
  }
  
% #1: id
\cs_new_protected:Npn \__acro_print_foreign:n #1
  {
    \bool_if:nT
      {
         \l__acro_foreign_bool &&
        !\l__acro_first_only_short_bool &&
        !\l__acro_first_only_long_bool
      }
      { \acro_write_foreign:n {#1} }
  }

\cs_new_protected:Npn \__acro_print_citation:n #1
  {
    \bool_if:NT \l__acro_group_citation_bool
      { \acro_group_cite:n {#1} }
  }

\cs_new_protected:Npn \__acro_finalize_first:n #1
  {
    \bool_if:NF \l__acro_group_citation_bool
      { \acro_cite_if:nn { \l__acro_citation_first_bool } {#1} }
    \acro_index_if:nn { \l__acro_addto_index_bool } {#1}
  }

% --------------------------------------------------------------------------
% the `acro-first' object, templates, instances:
% #1: id
% #2: long
\DeclareObjectType {acro-first} {2}

% template for inline appearance:
\DeclareTemplateInterface {acro-first} {inline} {2}
  {
    brackets      : boolean   = true  ,
    brackets-type : tokenlist = ()    ,
    only-short    : boolean   = false ,
    only-long     : boolean   = false ,
    reversed      : boolean   = false ,
    between       : tokenlist         ,
    foreign-sep   : tokenlist = {,~}
  }
\DeclareTemplateCode {acro-first} {inline} {2}
  {
    brackets      = \l__acro_first_use_brackets_bool ,
    brackets-type = \l__acro_first_brackets_tl       ,
    only-short    = \l__acro_first_only_short_bool   ,
    only-long     = \l__acro_first_only_long_bool    ,
    reversed      = \l__acro_first_reversed_bool     ,
    between       = \l__acro_first_between_tl        ,
    foreign-sep   = \l__acro_foreign_sep_tl
  }
  {
    \AssignTemplateKeys
    \bool_if:NTF \l__acro_first_reversed_bool
      { % zuerst kurze Form, dann lange Form:
        \__acro_print_form_and_indefinite:nnn {short} {#1} {#2}
        \__acro_open_bracket:
        \__acro_print_foreign:n {#1}
        \__acro_foreign_sep:n {#1}
        \__acro_print_form:nnn {long} {#1} {#2}
        \__acro_print_citation:n {#1}
        \__acro_close_bracket:
        \__acro_finalize_first:n {#1}
      }
      { % zuerst lange Form, dann kurze Form:
        \__acro_print_form_and_indefinite:nnn {long} {#1} {#2}
        \__acro_open_bracket:
        \__acro_print_foreign:n {#1}
        \__acro_foreign_sep:n {#1}
        \__acro_print_form:nnn {short} {#1} {#2}
        \__acro_print_citation:n {#1}
        \__acro_close_bracket:
        \__acro_finalize_first:n {#1}
      }
  }

% template for footnotes, sidenotes, ...
\cs_new:Npn \__acro_note_command:n #1 {#1}
\DeclareTemplateInterface {acro-first} {note} {2}
  {
    use-note     : boolean    = true ,
    note-command : function 1 = \footnote {#1} ,
    foreign-sep  : tokenlist  = {,~} ,
    reversed      : boolean   = false ,
  }

\DeclareTemplateCode {acro-first} {note} {2}
  {
    use-note     = \l__acro_use_note_bool  ,
    note-command = \__acro_note_command:n  ,
    foreign-sep  = \l__acro_foreign_sep_tl ,
    reversed     = \l__acro_first_reversed_bool
  }
  {
    \AssignTemplateKeys
    \bool_if:NTF \l__acro_first_reversed_bool
      { % long in text and short in note
        \__acro_print_form_and_indefinite:nnn {long} {#1} {#2}
        \bool_if:NT \l__acro_use_note_bool
          {
            \__acro_note_command:n
              {
                \__acro_print_foreign:n {#1}
                \__acro_foreign_sep:n {#1}
                \__acro_print_form:nnn {short} {#1} {#2}
                \__acro_print_citation:n {#1}
                \__acro_finalize_first:n {#1}
              }
          }
      }
      { % short in text and long in note
        \__acro_print_form_and_indefinite:nnn {short} {#1} {#2}
        \bool_if:NT \l__acro_use_note_bool
          {
            \__acro_note_command:n
              {
                \__acro_print_foreign:n {#1}
                \__acro_foreign_sep:n {#1}
                \__acro_print_form:nnn {long} {#1} {#2}
                \__acro_print_citation:n {#1}
                \__acro_finalize_first:n {#1}
              }
          }
      }
  }

% --------------------------------------------------------------------------
% declare new first styles:
\cs_new_protected:Npn \acro_declare_first_style:nnn #1#2#3
  {
    \DeclareInstance {acro-first} {#1} {#2} {#3}
    \prop_put:Nnn \l__acro_first_styles_prop  {#1} {#2}
  }

% #1: name
% #2: template
% #3: settings
\NewDocumentCommand \DeclareAcroFirstStyle {mmm}
  { \acro_declare_first_style:nnn {#1} {#2} {#3} }

% set a list style
\cs_new_protected:Npn \acro_set_first_style:n #1
  {
    \prop_if_in:NnTF \l__acro_first_styles_prop {#1}
      { \__acro_set_first_style:n {#1} }
      {
        \msg_warning:nnnnn {acro} {unknown}
          {first~ style}
          {#1}
          {default}
        \__acro_set_first_style:n {default}
      }
  }

\cs_new_protected:Npn \__acro_set_first_style:n #1
  {
    \tl_set:Nn \l__acro_first_instance_tl {#1}
    \prop_get:NnN \l__acro_first_styles_prop {#1} \l__acro_tmpa_tl
  }

% --------------------------------------------------------------------------
% formatting the extras information:
\DeclareObjectType {acro-extra} {1}

\DeclareTemplateInterface {acro-extra} {inline} {1}
  {
    punct         : boolean   = false ,
    punct-symbol  : tokenlist = {,}   ,
    brackets      : boolean   = true  ,
    brackets-type : tokenlist = ()
  }

\DeclareTemplateCode {acro-extra} {inline} {1}
  {
    punct         = \l__acro_extra_punct_bool        ,
    punct-symbol  = \l__acro_extra_punct_tl          ,
    brackets      = \l__acro_extra_use_brackets_bool ,
    brackets-type = \l__acro_extra_brackets_tl
  }
  {
    \AssignTemplateKeys
    \bool_if:NT \l__acro_extra_punct_bool
      { \tl_use:N \l__acro_extra_punct_tl \tl_use:N \c_space_tl }
    \bool_if:NT \l__acro_extra_use_brackets_bool
      { \tl_head:N \l__acro_extra_brackets_tl }
    \acro_write_long:Vn \l__acro_extra_format_tl {#1}
    \bool_if:NT \l__acro_extra_use_brackets_bool
      { \tl_tail:N \l__acro_extra_brackets_tl }
  }

% declare new extra styles:
\cs_new_protected:Npn \acro_declare_etxra_style:nnn #1#2#3
  {
    \DeclareInstance {acro-etxra} {#1} {#2} {#3}
    \prop_put:Nnn \l__acro_etxra_styles_prop  {#1} {#2}
  }

% #1: name
% #2: template
% #3: settings
\NewDocumentCommand \DeclareAcroExtraStyle {mmm}
  { \acro_declare_extra_style:nnn {#1} {#2} {#3} }

% set an extra style
\cs_new_protected:Npn \acro_set_extra_style:n #1
  {
    \prop_if_in:NnTF \l__acro_extra_styles_prop {#1}
      { \__acro_set_extra_style:n {#1} }
      {
        \msg_warning:nnnnn {acro} {unknown}
          {extra~ style}
          {#1}
          {default}
        \__acro_set_extra_style:n {default}
      }
  }

\cs_new_protected:Npn \__acro_set_extra_style:n #1
  {
    \tl_set:Nn \l__acro_extra_instance_tl {#1}
    \prop_get:NnN \l__acro_extra_styles_prop {#1} \l__acro_tmpa_tl
  }

\cs_new_protected:Npn \acro_declare_extra_style:nnn #1#2#3
  {
    \DeclareInstance {acro-extra} {#1} {#2} {#3}
    \prop_put:Nnn \l__acro_extra_styles_prop  {#1} {#2}
  }

% --------------------------------------------------------------------------
% outputting the page numbers:
\RequirePackage {zref-abspage}

\cs_new_protected:Npn \__acro_create_page_records:n #1
  {
    \seq_new:c { g__acro_#1_pages_seq }
    \tl_new:c  { g__acro_#1_recorded_pages_tl }
  }

\cs_new_protected:Npn \acro_hyper_page:n #1 { \use:n {#1} }

\cs_new:Npn \acro_get_thepage:nnn #1#2#3 { \acro_hyper_page:n {#1} }
\cs_new:Npn \acro_get_thepage_from:N #1
  { \exp_after:wN \acro_get_thepage:nnn #1 }

\cs_new:Npn \acro_get_page_number:nnn #1#2#3 {#2}
\cs_new:Npn \acro_get_page_number_from:N #1
  { \exp_after:wN \acro_get_page_number:nnn #1 }

\cs_new:Npn \acro_get_abspage:nnn #1#2#3 {#3}
\cs_new:Npn \acro_get_abspage_from:N #1
  { \exp_after:wN \acro_get_abspage:nnn #1 }

\cs_new:Npn \acro_page_range_comma: {}

\cs_new_protected:Npn \acro_print_page_numbers:n #1
  {
    \seq_if_empty:cF {g__acro_#1_pages_seq}
      {
        \bool_if:NTF \l__acro_list_all_pages_bool
          {
            % have the numbers changed?
            \tl_set:Nx \l__acro_tmpa_tl
              { \seq_use:cn {g__acro_#1_pages_seq} {|} }
            \tl_if_eq:cNF {g__acro_#1_recorded_pages_tl} \l__acro_tmpa_tl
              {
                \@latex@warning@no@line
                  {Rerun~to~get~page~numbers~of~acronym~#1~in~acronym~list~right}
              }
            \tl_clear:N \l__acro_write_pages_tl
            \tl_clear:N \l__acro_last_page_tl
            \tl_clear:N \l__acro_current_page_tl
            \seq_set_eq:Nc \l__acro_tmpb_seq { g__acro_#1_pages_seq }
            \seq_remove_duplicates:N \l__acro_tmpb_seq
            \seq_clear:N \l__acro_tmpa_seq
            \cs_set_protected:Npn \acro_page_range_comma:
              { \cs_set:Npn \acro_page_range_comma: { ,~ } }
            % get the numbers:
            \int_compare:nNnTF { \seq_count:N \l__acro_tmpb_seq } = { 1 }
              {
                \tl_use:N \l__acro_page_name_tl
                \seq_get_right:cN { g__acro_#1_pages_seq } \l__acro_tmpa_tl
                \acro_get_thepage_from:N \l__acro_tmpa_tl
              }
              {
                \tl_use:N \l__acro_pages_name_tl
                \seq_map_inline:cn { g__acro_#1_pages_seq }
                  {
                    \tl_if_blank:VTF \l__acro_last_page_tl
                      {% we're at the beginning
                        \seq_put_right:Nn \l__acro_tmpa_seq {##1}
                        \tl_set:Nn \l__acro_last_page_tl {##1}
                      }
                      {% we'at least at the second page
                         % current page:
                         \tl_set:Nn  \l__acro_current_page_tl {##1}
                         % last page:
                         \seq_get_right:NN \l__acro_tmpa_seq \l__acro_last_page_tl
                         \tl_if_eq:NNTF \l__acro_current_page_tl \l__acro_last_page_tl
                           {% there were more than one appearance on the current page
                             \seq_put_right:Nn \l__acro_tmpa_seq {##1}
                           }
                           {% new page
                             \acro_determine_page_ranges:NNn
                               \l__acro_tmpa_seq
                               \l__acro_write_pages_tl
                               {##1}
                           }
                      }
                  }
                \seq_if_empty:NF \l__acro_tmpa_seq
                  {
                    \acro_determine_page_ranges:NNV
                      \l__acro_tmpa_seq
                      \l__acro_write_pages_tl
                      \l__acro_current_page_tl
                  }
                \tl_use:N \l__acro_write_pages_tl
                \tl_clear:N \l__acro_write_pages_tl
              }
          }
          {
            \tl_use:N \l__acro_page_name_tl
            \pageref{\l__acro_label_prefix_tl #1}
          }
      }
    \seq_clear:N \l__acro_tmpa_seq
    \seq_clear:N \l__acro_tmpb_seq
  }

\cs_new:Npn \acro_determine_page_ranges:NNn #1#2#3
  {
    \seq_remove_duplicates:N #1
    % current page:
    \int_set:Nn \l__acro_tmpa_int { \acro_get_abspage:nnn #3 }
    \int_set:Nn \l__acro_tmpb_int { \acro_get_page_number:nnn #3 }
    % last page:
    \seq_get_right:NN #1 \l__acro_last_page_tl
    \int_set:Nn \l__acro_tmpc_int
      { \acro_get_abspage_from:N \l__acro_last_page_tl }
    \int_set:Nn \l__acro_tmpd_int
      { \acro_get_page_number_from:N \l__acro_last_page_tl }
    \bool_if:nTF
      {
        \int_compare_p:nNn
          { \l__acro_tmpa_int - \l__acro_tmpc_int }
           =
          { \l__acro_tmpb_int - \l__acro_tmpd_int }
        &&
        \int_compare_p:nNn
        { \l__acro_tmpb_int - \l__acro_tmpd_int } = {1}
      }
      {% same kind of page numbering, one page ahead
       % => possible range
         \seq_put_right:Nn #1 {#3}
      }
      {% any possible range ended
        \tl_put_right:Nn #2 { \acro_page_range_comma: }
        \int_compare:nNnTF
          { \seq_count:N #1 } > {2}
          {% real range
            \seq_get_left:NN #1 \l__acro_tmpa_tl
            \tl_put_right:Nx #2 { \acro_get_thepage_from:N \l__acro_tmpa_tl }
            \bool_if:NTF \l__acro_following_pages_bool
              { \tl_put_right:Nn #2 { \l__acro_next_pages_tl } }
              {
                \tl_put_right:Nn #2 { -- }
                \seq_get_right:NN #1 \l__acro_tmpa_tl
                \tl_put_right:Nx #2 { \acro_get_thepage_from:N \l__acro_tmpa_tl }
              }
          }
          {
            \int_compare:nNnTF
              { \seq_count:N #1 } = {2}
              {% range of two pages
                \seq_get_left:NN #1 \l__acro_tmpa_tl
                \tl_put_right:Nx #2 { \acro_get_thepage_from:N \l__acro_tmpa_tl }
                \bool_if:NTF \l__acro_following_page_bool
                  { \tl_put_right:Nn #2 { \l__acro_next_page_tl } }
                  {
                    \tl_put_right:Nn #2 { ,~ }
                    \seq_get_right:NN #1 \l__acro_tmpa_tl
                    \tl_put_right:Nx #2 { \acro_get_thepage_from:N \l__acro_tmpa_tl }
                  }
              }
              {% no range at all
                \seq_get_right:NN #1 \l__acro_tmpa_tl
                \tl_put_right:Nx #2 { \acro_get_thepage_from:N \l__acro_tmpa_tl }
              }
          }
        \seq_clear:N #1
        \seq_put_right:Nn #1 {#3}
      }
  }
\cs_generate_variant:Nn \acro_determine_page_ranges:NNn { NNV }

% --------------------------------------------------------------------------
\DeclareObjectType {acro-page-number} {1}

\DeclareTemplateInterface {acro-page-number} {inline} {1}
  {
    display       : boolean   = true  ,
    punct         : boolean   = false ,
    punct-symbol  : tokenlist = {,}   ,
    brackets      : boolean   = false ,
    brackets-type : tokenlist = ()    ,
    space         : skip      = .333333em plus .166666em minus .111111em
  }

\DeclareTemplateCode {acro-page-number} {inline} {1}
  {
    display       = \l__acro_page_display_bool  ,
    punct         = \l__acro_page_punct_bool    ,
    punct-symbol  = \l__acro_page_punct_tl      ,
    brackets      = \l__acro_page_brackets_bool ,
    brackets-type = \l__acro_page_brackets_tl   ,
    space         = \l__acro_page_space_skip
  }
  {
    \AssignTemplateKeys
    \bool_if:NT \l__acro_page_display_bool
      {
        \bool_if:NT \l__acro_page_punct_bool
          { \tl_use:N \l__acro_page_punct_tl }
        % \tl_use:N \c_space_tl
        \dim_compare:nNnF { \l__acro_page_space_skip } = { 0pt }
          { \skip_horizontal:N \l__acro_page_space_skip }
        \bool_if:NT \l__acro_page_brackets_bool
          { \tl_head:N \l__acro_page_brackets_tl }
        \acro_print_page_numbers:n {#1}
        \bool_if:NT \l__acro_page_brackets_bool
          { \tl_tail:N \l__acro_page_brackets_tl }
      }
  }

% declare new page styles:
\cs_new_protected:Npn \acro_declare_page_style:nnn #1#2#3
  {
    \DeclareInstance {acro-page-number} {#1} {#2} {#3}
    \prop_put:Nnn \l__acro_page_styles_prop  {#1} {#2}
  }

% #1: name
% #2: template
% #3: settings
\NewDocumentCommand \DeclareAcroPageStyle {mmm}
  { \acro_declare_page_style:nnn {#1} {#2} {#3} }

% set a page style
\cs_new_protected:Npn \acro_set_page_style:n #1
  {
    \prop_if_in:NnTF \l__acro_page_styles_prop {#1}
      { \__acro_set_page_style:n {#1} }
      {
        \msg_warning:nnnnn {acro} {unknown}
          {page~ style}
          {#1}
          {none}
        \__acro_set_page_style:n {none}
      }
  }

\cs_new_protected:Npn \__acro_set_page_style:n #1
  {
    \tl_set:Nn \l__acro_page_instance_tl {#1}
    \prop_get:NnN \l__acro_page_styles_prop {#1} \l__acro_tmpa_tl
  }

% --------------------------------------------------------------------------
% the title of the list:
\cs_new:Npn \acro_list_title_format:n #1 {#1}

\DeclareObjectType {acro-title} {1}

\DeclareTemplateInterface {acro-title} {sectioning} {1}
  { name-format : function 1 = #1 }

\DeclareTemplateCode {acro-title} {sectioning} {1}
  { name-format = \acro_list_title_format:n }
  {
    \AssignTemplateKeys
    \acro_list_title_format:n {#1}
  }

% set a list heading:
\cs_new_protected:Npn \acro_set_list_heading:n #1
  {
    \prop_if_in:NnTF \l__acro_list_headings_prop {#1}
      { \__acro_set_list_heading:n {#1} }
      {
        \msg_warning:nnnnn {acro} {unknown}
          {list~ heading}
          {#1}
          {section*}
        \__acro_set_list_heading:n {section*}
      }
  }

\cs_new_protected:Npn \__acro_set_list_heading:n #1
  {
    \tl_set:Nn \l__acro_list_heading_cmd_tl {#1}
    % \prop_get:NnN \l__acro_list_headings_prop
    %   {#1}
    %   \l__acro_list_heading_cmd_tl
  }
  
\cs_new_protected:Npn \acro_declare_list_heading:nn #1#2
  {
    \prop_put:Nnn \l__acro_list_headings_prop {#1} {#2}
    \DeclareInstance {acro-title} {#1} {sectioning}
      { name-format = #2 {##1} }
  }

\NewDocumentCommand \DeclareAcroListHeading {mm}
  { \acro_declare_list_heading:nn {#1} {#2} }

% --------------------------------------------------------------------------
% typesetting the acronym list
\DeclareObjectType {acro-list} {2}

% #1: id
% #2: excluded classes
\prg_new_protected_conditional:Npnn \acro_if_is_excluded:nn #1#2 {T,F,TF}
  {
    \bool_set_false:N \l__acro_is_excluded_bool
    \tl_if_blank:nF {#2}
      {
        \clist_map_inline:nn {#2}
          {
            \prop_get:NnNT \l__acro_class_prop {#1} \l__acro_tmpa_tl
              {
                \seq_set_split:NnV \l__acro_tmpa_seq {,} \l__acro_tmpa_tl
                \seq_if_in:NnT \l__acro_tmpa_seq {##1}
                  { \bool_set_true:N \l__acro_is_excluded_bool }
              }
          }
      }
    \bool_if:NTF \l__acro_is_excluded_bool
      { \prg_return_true: }
      { \prg_return_false: }
  }

% #1: id
% #2: included classes
\prg_new_protected_conditional:Npnn \acro_if_is_included:nn #1#2 {T,F,TF}
  {
    \bool_set_false:N \l__acro_is_included_bool
    \tl_if_blank:nTF {#2}
      { \bool_set_true:N \l__acro_is_included_bool }
      {
        \clist_map_inline:nn {#2}
          {
            \prop_get:NnNT \l__acro_class_prop {#1} \l__acro_tmpa_tl
              {
                \seq_set_split:NnV \l__acro_tmpa_seq {,} \l__acro_tmpa_tl
                \seq_if_in:NnT \l__acro_tmpa_seq {##1}
                  { \bool_set_true:N \l__acro_is_included_bool }
              }
          }
      }
    \bool_if:NTF \l__acro_is_included_bool
      { \prg_return_true: }
      { \prg_return_false: }
  }

% #1: id
\cs_new_protected:Npn \__acro_list_entry_short:n #1
  {
    \acro_hyper_target:nn
      {#1}
      {
        \acro_acc_supp:nn
          {#1}
          {
            \acro_write_short:nn {#1}
              {
                \l__acro_list_short_format_tl
                { \__acro_get_property:nn {short} {#1} }
              }
          }
      }
  }

% #1: id
\cs_new_protected:Npn \__acro_list_entry_long:n #1
  {
    \group_begin:
      \bool_if:NT \l__acro_capitalize_list_bool
        { \bool_set_true:N \l__acro_first_upper_bool }
      \acro_write_long:Vf \l__acro_list_long_format_tl
        {
          \prop_if_in:NnTF \l__acro_list_prop {#1}
            { \__acro_get_property:nn {list} {#1} }
            { \__acro_get_property:nn {long} {#1} }
        }
    \group_end:
    \bool_if:NT \l__acro_foreign_bool
      { \acro_get_foreign:n {#1} }
    \acro_cite_if:nn { \l__acro_citation_all_bool } {#1}
  }

% #1: id
\cs_new_protected:Npn \__acro_list_entry_extra:n #1
  {
    \prop_get:NnNT \l__acro_extra_prop {#1} \l__acro_tmpa_tl
      {
        \acro_extra_instance:VV
          \l__acro_extra_instance_tl
          \l__acro_tmpa_tl
      }
  }

% #1: id
\cs_new_protected:Npn \__acro_list_entry_page:n #1
  {
    \bool_if:nT { \cs_if_exist_p:c { acro@#1@once } }
      {
        \acro_page_number_instance:Vn
          \l__acro_page_instance_tl
          {#1}
      }
  }
  
% macro for retrieval of items in the list:
% #1: property
% #2: id
\cs_new_protected:Npn \acro_list_entry:nn #1#2
  {
    \str_case:nnF {#1}
      {
        {short} { \__acro_list_entry_short:n {#2} }
        {long}  { \__acro_list_entry_long:n {#2} }
        {extra} { \__acro_list_entry_extra:n {#2} }
        {page}  { \__acro_list_entry_page:n {#2} }
      }
      { \__acro_get_property:nn {#1} {#2} }
  }

% this macro may/should be redefined in templates:
% #1: short
% #2: long
% #3: extra
% #4: page number(s)
\cs_new_protected:Npn \acro_print_list_entry:nnnn #1#2#3#4
  { #1 #2 #3 #4 }

\cs_new_protected:Npn \acro_for_all_acronyms_do:n #1
  { \prop_map_inline:Nn \l__acro_short_prop {#1} }

% test, if acronyms should be printed or not; needs testing for in/excluded
% classes and options `only-used' and `single' -- this macro should be used in
% the template code for retrieving the list
  
% #1: id
% #2: included classes
% #3: excluded classes
\prg_new_protected_conditional:Npnn \acro_if_entry:nnn #1#2#3 {T,F,TF}
  {
    \bool_if:nTF
      {
        \bool_if_p:c { g__acro_#1_in_list_bool } &&
        (
          ( \l__acro_use_single_bool && \cs_if_exist_p:c { acro@#1@twice } )
          ||
          (
            !\l__acro_use_single_bool &&
            \cs_if_exist_p:c { acro@#1@once } &&
            \l__acro_print_only_used_bool
          )
        )
        ||
        ( !\l__acro_use_single_bool && !\l__acro_print_only_used_bool )
      }
      {
        \acro_if_is_excluded:nnTF {#1} {#3}
          { \prg_return_false: }
          {
            \acro_if_is_included:nnTF {#1} {#2}
              {
                \bool_if:nTF
                  { \g__acro_use_barriers_bool && \l__acro_use_barrier_bool }
                  {
                    \acro_if_in_barrier:nxTF {#1}
                      { \int_use:N \g__acro_barrier_int }
                      { \prg_return_true: }
                      { \prg_return_false: }
                  }
                  { \prg_return_true: }
              }
              { \prg_return_false: }
          }
      }
      { \prg_return_false: }
  }

\tl_new:N \l__acro_list_entries_tl

% this macro is used in templates for fetching all items to be printed; it
% collects all entries in a tl which then is used where needed
%
% #1: tl containing the entries
% #2: included classes
% #3: excluded classes
\cs_new_protected:Npn \acro_build_list_entries:Nnn #1#2#3
  {
    \tl_clear:N #1
    \acro_for_all_acronyms_do:n
      {% ##1: id; ##2: short form
        \acro_get:n {##1}
        \acro_if_entry:nnnT {##1} {#2} {#3}
          {
            \tl_put_right:Nn #1
              {
                \acro_print_list_entry:nnnn
                  { \acro_list_entry:nn {short} {##1} }
                  { \acro_list_entry:nn {long} {##1} }
                  { \acro_list_entry:nn {extra} {##1} }
                  { \acro_list_entry:nn {page} {##1} }
              }
          }
      }
  }

% this macro is used in templates for fetching all items to be printed:
\cs_new_protected:Npn \acro_list_items:nn #1#2
  {
    \acro_build_list_entries:Nnn \l__acro_list_entries_tl {#1} {#2}
    \tl_use:N \l__acro_list_entries_tl
  }
  
% --------------------------------------------------------------------------
% declare templates for the list:
% `list' template:
\DeclareTemplateInterface {acro-list} {list} {2}
  {
    foreign-sep : tokenlist = {~} ,
    list        : tokenlist = {description} ,
    reverse     : boolean   = false ,
    before      : tokenlist = ,
    after       : tokenlist =
  }

\DeclareTemplateCode {acro-list} {list} {2}
  {
    foreign-sep = \l__acro_foreign_sep_tl ,
    list        = \l__acro_list_tl ,
    reverse     = \l__acro_list_reverse_long_extra_bool ,
    before      = \l__acro_list_before_tl ,
    after       = \l__acro_list_after_tl
  }
  {
    \AssignTemplateKeys
    \bool_set_true:N \l__acro_in_list_bool
    \acro_activate_hyperref_support:
    \bool_if:NTF \l__acro_list_reverse_long_extra_bool
      {
        \cs_set_protected:Npn \acro_print_list_entry:nnnn ##1##2##3##4
          { \item [##1] ##3 ##2 ##4 }
      }
      {
        \cs_set_protected:Npn \acro_print_list_entry:nnnn ##1##2##3##4
          { \item [##1] ##2 ##3 ##4 }
      }
    \use:x
      {
        \exp_not:V \l__acro_list_before_tl
        \exp_not:N \begin { \exp_not:V \l__acro_list_tl }
          \exp_not:n { \acro_list_items:nn {#1} {#2} }
        \exp_not:N \end { \exp_not:V \l__acro_list_tl }
        \exp_not:V \l__acro_list_after_tl
      }
  }

% `list-of' template:
\DeclareTemplateInterface {acro-list} {list-of} {2}
  {
    foreign-sep : tokenlist = {~} ,
    style       : tokenlist = {toc} ,
    reverse     : boolean   = false ,
    before      : tokenlist = ,
    after       : tokenlist =
  }

\DeclareTemplateCode {acro-list} {list-of} {2}
  {
    foreign-sep = \l__acro_foreign_sep_tl ,
    style       = \l__acro_list_of_style ,
    reverse     = \l__acro_list_reverse_long_extra_bool ,
    before      = \l__acro_list_before_tl ,
    after       = \l__acro_list_after_tl
  }
  {
    \AssignTemplateKeys
    \bool_set_true:N \l__acro_in_list_bool
    \tl_if_eq:VnT \l__acro_page_instance_tl {none}
      { \tl_set:Nn \l__acro_page_instance_tl {plain} }
    \tl_set:Nn \l__acro_page_name_tl {}
    \tl_set:Nn \l__acro_pages_name_tl {}
    \acro_activate_hyperref_support:
    \str_case:Vn \l__acro_list_of_style
      {
        {toc}
        { % similar to the table of contents
          \bool_if:NTF \l__acro_list_reverse_long_extra_bool
            {
              \cs_if_exist:NTF \chapter
                {
                  \cs_set_protected:Npn \acro_print_list_entry:nnnn ##1##2##3##4
                    {
                      \contentsline{chapter}{##1}{}{}
                      \contentsline{section}{##3##2}{##4}{}
                    } 
                }
                {
                  \cs_set_protected:Npn \acro_print_list_entry:nnnn ##1##2##3##4
                    {
                      \contentsline{section}{##1}{}{}
                      \contentsline{subsection}{##3##2}{##4}{}
                    }
                }
            }
            {
              \cs_if_exist:NTF \chapter
                {
                  \cs_set_protected:Npn \acro_print_list_entry:nnnn ##1##2##3##4
                    {
                      \contentsline{chapter}{##1}{}{}
                      \contentsline{section}{##2##3}{##4}{}
                    } 
                }
                {
                  \cs_set_protected:Npn \acro_print_list_entry:nnnn ##1##2##3##4
                    {
                      \contentsline{section}{##1}{}{}
                      \contentsline{subsection}{##2##3}{##4}{}
                    }
                }
            }
        }
        {lof}
        { % similar to the list of figures
          \cs_set_protected:Npn \l@acro
            { \@dottedtocline {1} {1.5em} {\l__acro_short_width_dim} }
          \bool_if:NTF \l__acro_list_reverse_long_extra_bool
            {
              \cs_set_protected:Npn \acro_print_list_entry:nnnn ##1##2##3##4
                { \contentsline{acro}{\numberline{##1}{##3##2}}{##4}{} }
            }
            {
              \cs_set_protected:Npn \acro_print_list_entry:nnnn ##1##2##3##4
                { \contentsline{acro}{\numberline{##1}{##2##3}}{##4}{} }
            }
        }
      }
    \use:x
      {
        \exp_not:V \l__acro_list_before_tl
        \exp_not:n { \acro_list_items:nn {#1} {#2} }
        \exp_not:V \l__acro_list_before_tl
      }
  }
  
% `table' template:
\DeclareTemplateInterface {acro-list} {table} {2}
  {
    table       : tokenlist = tabular ,
    table-spec  : tokenlist = lp{.7\linewidth} ,
    foreign-sep : tokenlist = {~} ,
    reverse     : boolean   = false ,
    before      : tokenlist = ,
    after       : tokenlist = 
  }

\DeclareTemplateCode {acro-list} {table} {2}
  {
    table       = \l__acro_list_table_tl      ,
    table-spec  = \l__acro_list_table_spec_tl ,
    foreign-sep = \l__acro_foreign_sep_tl ,
    reverse     = \l__acro_list_reverse_long_extra_bool ,
    before      = \l__acro_list_before_tl ,
    after       = \l__acro_list_after_tl
  }
  {
    \AssignTemplateKeys
    \acro_activate_hyperref_support:
    \bool_if:NTF \l__acro_list_reverse_long_extra_bool
      {
        \cs_set_protected:Npn \acro_print_list_entry:nnnn ##1##2##3##4
          { ##1 & ##3 ##2 ##4 \tabularnewline }
      }
      {
        \cs_set_protected:Npn \acro_print_list_entry:nnnn ##1##2##3##4
          { ##1 & ##2 ##3 ##4 \tabularnewline }
      }
    \acro_build_list_entries:Nnn \l__acro_list_entries_tl {#1} {#2}
    \use:x
      {
        \exp_not:V \l__acro_list_before_tl
        \exp_not:N \begin { \exp_not:V \l__acro_list_table_tl }
          { \exp_not:V \l__acro_list_table_spec_tl }
        \exp_not:V \l__acro_list_entries_tl
        \exp_not:N \end { \exp_not:V \l__acro_list_table_tl }
        \exp_not:V \l__acro_list_after_tl
      }
  }

% `extra-table' template:
\DeclareTemplateInterface {acro-list} {extra-table} {2}
  {
    table       : tokenlist = tabular ,
    table-spec  : tokenlist = llll ,
    foreign-sep : tokenlist = {~} ,
    reverse     : boolean   = false ,
    before      : tokenlist = ,
    after       : tokenlist = 
  }

\DeclareTemplateCode {acro-list} {extra-table} {2}
  {
    table       = \l__acro_list_table_tl      ,
    table-spec  = \l__acro_list_table_spec_tl ,
    foreign-sep = \l__acro_foreign_sep_tl ,
    reverse     = \l__acro_list_reverse_long_extra_bool ,
    before      = \l__acro_list_before_tl ,
    after       = \l__acro_list_after_tl
  }
  {
    \AssignTemplateKeys
    \acro_activate_hyperref_support:
    \bool_if:NTF \l__acro_list_reverse_long_extra_bool
      {
        \cs_set_protected:Npn \acro_print_list_entry:nnnn ##1##2##3##4
          { ##1 & ##3 & ##2 & ##4 \tabularnewline }
      }
      {
        \cs_set_protected:Npn \acro_print_list_entry:nnnn ##1##2##3##4
          { ##1 & ##2 & ##3 & ##4 \tabularnewline }
      }
    \acro_build_list_entries:Nnn \l__acro_list_entries_tl {#1} {#2}
    \use:x
      {
        \exp_not:V \l__acro_list_before_tl
        \exp_not:N \begin { \exp_not:V \l__acro_list_table_tl }
          { \exp_not:V \l__acro_list_table_spec_tl }
        \exp_not:V \l__acro_list_entries_tl
        \exp_not:N \end { \exp_not:V \l__acro_list_table_tl }
        \exp_not:V \l__acro_list_after_tl
      }
  }

% --------------------------------------------------------------------------
% declare new list styles:
\cs_new_protected:Npn \acro_declare_list_style:nnn #1#2#3
  {
    \DeclareInstance {acro-list} {#1} {#2} {#3}
    \prop_put:Nnn \l__acro_list_styles_prop  {#1} {#2}
  }

% #1: name
% #2: template
% #3: settings
\NewDocumentCommand \DeclareAcroListStyle {mmm}
  { \acro_declare_list_style:nnn {#1} {#2} {#3} }

% set a list style
\cs_new_protected:Npn \acro_set_list_style:n #1
  {
    \prop_if_in:NnTF \l__acro_list_styles_prop {#1}
      { \__acro_set_list_style:n {#1} }
      {
        \msg_warning:nnnnn {acro} {unknown}
          {list~ style}
          {#1}
          {description}
        \__acro_set_list_style:n {description}
      }
  }

\cs_new_protected:Npn \__acro_set_list_style:n #1
  {
    \tl_set:Nn \l__acro_list_instance_tl {#1}
    \prop_get:NnN \l__acro_list_styles_prop {#1} \l__acro_list_type_tl
  }

% --------------------------------------------------------------------------
% automatic typesetting, the internals of \ac:
% #1: id
  
\cs_new_protected:Npn \acro_use:n #1
  {
    % get the acronym and the plural settings:
    \acro_get:n {#1}
    \acro_is_used:nTF {#1}
      {
        % this is not the first time
        \acro_write_indefinite:nn {#1} {short}
        \acro_write_compact:nn {#1} {short}
        \acro_after:n {#1}
      }
      {
        % this is the first time
        \bool_gset_true:c { g__acro_#1_first_use_bool }
        \acro_if_is_single:nTF {#1}
          { \acro_single:n {#1} }
          { \acro_first_instance:nV {#1} \l__acro_long_tl }
      }
  }

% single appearances:
\cs_new_protected:Npn \acro_single:n #1
  {
    \acro_cite:
    \acro_single_form:nV {#1} \l__acro_single_form_tl
    \acro_after:n {#1}
  }
  
% #1: ID
% #2: long|first|<other>
\cs_new_protected:Npn \acro_single_form:nn #1#2
  {
    \acro_write_indefinite:nn {#1} {#2}
    \str_case:nnF {#2}
      {
        {long} {
          \tl_if_blank:VT \l__acro_single_format_tl
            {
              \bool_if:NTF \l__acro_custom_long_format_bool
                {
                  \tl_set_eq:NN
                    \l__acro_single_format_tl
                    \l__acro_custom_long_format_tl
                }
                {
                  \tl_set_eq:NN
                    \l__acro_single_format_tl
                    \l__acro_long_format_tl
                }
            }
          \tl_if_blank:VT \l__acro_single_tl
            { \tl_set_eq:NN \l__acro_single_tl \l__acro_long_tl }
          \acro_write_long:VV \l__acro_single_format_tl \l__acro_single_tl
        }
        {first} {
          \tl_if_blank:VF \l__acro_single_format_tl
            {
              \tl_set_eq:NN
                \l__acro_first_long_format_tl
                \l__acro_single_format_tl
            }
          \tl_if_blank:VT \l__acro_single_tl
            { \tl_set_eq:NN \l__acro_single_tl \l__acro_long_tl }
          \acro_first_instance:nV {#1} \l__acro_single_tl
        }
      }
      { % other (e.g. short)
        \tl_if_blank:VF \l__acro_single_tl
          { \tl_set_eq:cN {l__acro_#2_tl} \l__acro_single_tl }
        \tl_if_blank:VF \l__acro_single_format_tl
          { \tl_set_eq:cN {l__acro_#2_format_tl} \l__acro_single_format_tl }
        \acro_write_compact:nn {#1} {#2}
      }
  }
\cs_generate_variant:Nn \acro_single_form:nn {nV}

\prg_new_conditional:Npnn \acro_if_is_single:n #1 { p,T,TF }
  {
    \bool_if:nTF
      { !\l__acro_use_single_bool || \cs_if_exist_p:c { acro@#1@twice } }
      { \prg_return_false: }
      { \prg_return_true: }
  }

\cs_new_protected:Npn \acro_use_acronym:n #1
  { \use:c {bool_set_#1:N} \l__acro_mark_as_used_bool }

% --------------------------------------------------------------------------
% some helpers we'll need more often:
\seq_new:N \g__acro_declared_acronyms_seq

\prg_new_conditional:Npnn \acro_if_defined:n #1 {p,T,F,TF}
  {
    \seq_if_in:NnTF \g__acro_declared_acronyms_seq {#1}
      { \prg_return_true: }
      { \prg_return_false: }
  }

\cs_new_protected:Npn \acro_defined:n #1
  {
    \acro_if_defined:nF {#1}
      { \acro_serious_message:nn {undefined} {#1} }
  }

% expandably gets property but doesn't transform property name -- internal
% name is needed
% #1: property
% #2: id
\cs_new:Npn \__acro_get_property:nn #1#2
  { \prop_item:cn {l__acro_#1_prop} {#2} }

% #1: id
% #2: property
% #3: set case
% #4: not set case
\prg_new_protected_conditional:Npnn \acro_get_property:nn #1#2 {T,F,TF}
  {
    \tl_set:Nn \l__acro_tmpa_tl {#2}
    \tl_replace_all:Nnn \l__acro_tmpa_tl {-} {_}
    \prop_get:cncTF
      {l__acro_ \l__acro_tmpa_tl _prop}
      {#1}
      {l__acro_ \l__acro_tmpa_tl _tl}
      { \prg_return_true: }
      { \prg_return_false: }
  }

\cs_new_protected:Npn \acro_get_property:nn #1#2
  { \acro_get_property:nnTF {#1} {#2} {} {} }
\cs_generate_variant:Nn \acro_get_property:nn {V}

% #1: id
% #2: property
% #3: set case
% #4: not set case
\prg_new_protected_conditional:Npnn \acro_if_property:nn #1#2 {T,F,TF}
  {
    \tl_set:Nn \l__acro_tmpa_tl {#2}
    \tl_replace_all:Nnn \l__acro_tmpa_tl {-} {_}
    \prop_if_in:cnTF
      {l__acro_ \l__acro_tmpa_tl _prop}
      {#1}
      { \prg_return_true: }
      { \prg_return_false: }
  }

\seq_new:N \l__acro_actions_seq

% within this command one can refer to the current id with `#1'
\cs_new_protected:Npn \acro_add_action:n #1
  { \seq_put_right:Nn \l__acro_actions_seq {#1} }

\tl_new:N \l_acro_current_id_tl
\cs_new_protected:Npn \__acro_get_actions:n #1
  {
    \seq_map_inline:Nn \l__acro_actions_seq
      {
        \cs_set:Npn \__acro_action:n ####1 {##1}
        \__acro_action:n {#1}
      }
  }

\cs_new_protected:Npn \acro_get:n #1
  {
    \bool_if:NF \l__acro_in_list_bool { \leavevmode }
    \acro_activate_hyperref_support:
    % short:
    \prop_get:NnNF \l__acro_short_prop {#1} \l__acro_tmpa_tl {}
    \__acro_make_link:NnV \l__acro_short_tl {#1} \l__acro_tmpa_tl
    % \acro_get_property:nn {#1} {short-format}
     % alt:
    \prop_get:NnNTF \l__acro_alt_prop {#1} \l__acro_tmpa_tl
      { \__acro_make_link:NnV \l__acro_alt_tl {#1} \l__acro_tmpa_tl }
      { \tl_set_eq:NN \l__acro_alt_tl \l__acro_short_tl }
    % long:
    \acro_get_property:nn {#1} {long}
    % \acro_get_property:nn {#1} {long-format}
    % foreign:
    \acro_get_property:nn {#1} {foreign}
    % foreign-lang:
    \acro_get_property:nn {#1} {foreign-lang}
    % extra:
    \acro_get_property:nn {#1} {extra}
    % \acro_get_property:nn {#1} {extra-format}
    % single:
    \acro_get_property:nn {#1} {single}
    % \acro_get_property:nn {#1} {single-format}
    % first-style:
    \acro_get_property:nn {#1} {first-style}
    % formatting
    \prop_get:NnNTF \l__acro_long_format_prop {#1}
      \l__acro_custom_long_format_tl
      { \bool_set_true:N  \l__acro_custom_long_format_bool }
      { \bool_set_false:N \l__acro_custom_long_format_bool }
    \acro_get_property:nn {#1} {first-long-format}
    \prop_get:NnNTF \l__acro_format_prop {#1} \l__acro_custom_format_tl
      { \bool_set_true:N \l__acro_custom_format_bool }
      { \bool_set_false:N \l__acro_custom_format_bool }
    \acro_get_property:nn {#1} {single-format}
    \acro_for_endings_do:n
      {
        \bool_if:cT {l__acro_##1_bool}
          { \__acro_set_ending_for:nnn {##1} {#1} {long} }
      }
    \acro_get_property:nnF {#1} {long-post}
      { \tl_clear:N \l__acro_long_post_tl }
    \acro_get_property:nnT {#1} {long-pre}
      { \tl_put_left:NV \l__acro_long_tl \l__acro_long_pre_tl }
    \__acro_get_actions:n {#1}
  }

% --------------------------------------------------------------------------
% plural endings and similar concepts:
\seq_new:N \l__acro_endings_seq

\cs_new_protected:Npn \acro_for_endings_do:n #1
  { \seq_map_inline:Nn \l__acro_endings_seq {#1} }

% #1: ending
% #2: ID
\cs_new_protected:Npn \__acro_set_ending:nn #1#2
  {
    \bool_if:cT {l__acro_#1_bool}
      {
        \__acro_set_ending_for:nnn {#1} {#2} {short}
        \__acro_set_ending_for:nnn {#1} {#2} {alt}
        \__acro_set_ending_for:nnn {#1} {#2} {long}
      }
  }

\tl_new:N \l__acro_endings_tl

\bool_new:N \l__acro_use_ending_form_bool

% this does nothing if a non-existent ending (#1) or non-existent form (#3) is
% input
% #1: ending
% #2: id
% #3: short|alt|long
\cs_new_protected:Npn \__acro_set_ending_for:nnn #1#2#3
  {
    \acro_if_ending_form_exist:nnT {#1} {#3}
      {
        \bool_if:nTF { \prop_item:cn {l__acro_#3_#1_form_prop} {#2} }
          { \prop_get:cnc {l__acro_#3_#1_prop} {#2} {l__acro_#3_tl}  }
          { \prop_get:cnc {l__acro_#3_#1_prop} {#2} {l__acro_#3_#1_tl} }
      }
  }

\cs_new_protected:Npn \__acro_set_endings:n #1
  {
    \acro_for_endings_do:n
      { \__acro_set_ending:nn {##1} {#1} }
  }

% #1: id
% #2: short|alt|…
\cs_new_protected:Npn \acro_get_ending_form:nn #1#2
  {
    \acro_for_endings_do:n
      {
        \acro_if_ending_form_exist:nnT {##1} {#2}
          {
            \bool_if:nT
              {
                \prop_item:cn {l__acro_#2_##1_form_prop} {#1}
                &&
                \use:c {l__acro_##1_bool}
              }
              { \prop_get:cncF {l__acro_#2_##1_prop} {#1} {l__acro_#2_tl} {} }
          }
      }
  }

% #1: id
% #2: short|alt|…
\cs_new_protected:Npn \acro_endings:nn #1#2
  {
    \group_begin:
      \bool_if:NTF \l__acro_include_endings_format_bool
        {
          \bool_if:NTF \l__acro_custom_format_bool
            { \l__acro_custom_format_tl }
            { \tl_use:c {l__acro_#2_format_tl} }
        }
        { \use:n }
        {
          \acro_for_endings_do:n
            {
              \__acro_set_ending_for:nnn {##1} {#1} {#2}
              \bool_if:cT {l__acro_##1_bool}
                { \tl_use:c {l__acro_#2_##1_tl} }
            }
        }
    \group_end:
  }

\prg_new_conditional:Npnn \acro_if_ending_exist:n #1 {p,T,F,TF}
  {
    \seq_if_in:NnTF \l__acro_endings_seq {#1}
      { \prg_return_true: }
      { \prg_return_false: }
  }

% #1: ending
% #2: short|alt|…
\prg_new_conditional:Npnn \acro_if_ending_form_exist:nn #1#2 {p,T,F,TF}
  {
    \cs_if_exist:cTF {l__acro_#2_#1_prop}
      { \prg_return_true: }
      { \prg_return_false: }
  }
  
% #1: name
% #2: default short
% #3: default long
\cs_new_protected:Npn \acro_provide_ending:nnn #1#2#3
  {
    \acro_if_ending_exist:nTF {#1}
      {
        \acro_harmless_message:nn {ending-exists} {#1}
        % short variables
        \acro_set_ending_variables:nnn {short} {#1} {#2}
        % alt variables
        \acro_set_ending_variables:nnn {alt} {#1} {#2}
        % long variables
        \acro_set_ending_variables:nnn {long} {#1} {#3}
      }
      {
        % registering:
        \bool_if:NT \g__acro_first_acronym_declared_bool
          { \acro_serious_message:n {ending-before-acronyms} }
        \seq_put_right:Nn \l__acro_endings_seq {#1}
        \bool_new:c {l__acro_#1_bool}
        % short variables
        \acro_define_and_set_ending_variables:nnn {short} {#1} {#2}
        % alt variables
        \acro_define_and_set_ending_variables:nnn {alt} {#1} {#2}
        % long variables
        \acro_define_and_set_ending_variables:nnn {long} {#1} {#3}
        % define setup command:
        \tl_set:Nn \l__acro_tmpa_tl {#1}
        \tl_replace_all:Nnn \l__acro_tmpa_tl {-} {_}
        \cs_new_protected:cpn {acro_ \l__acro_tmpa_tl :}
          { \bool_set_true:c {l__acro_#1_bool} }
        % acronym properties:
        % short-<ending>:
        \acro_declare_property:nnn {short_#1} {short-#1}
          {
            \prop_put:cnn {l__acro_short_#1_form_prop} {##1} { \c_false_bool }
            \prop_put:cnx {l__acro_pdfstring_short_#1_prop}
              {##1} { \prop_item:Nn \l__acro_short_prop {##1} \exp_not:n {##2} }
          }
        % short-<ending>-form:
        \acro_declare_property_generic:nnn {short_#1_form} {short-#1-form}
          {
            \__acro_property_check:nn {##1} {short-#1-form}
            \prop_put:cnn {l__acro_short_#1_form_prop} {##1} { \c_true_bool }
            \prop_put:cnn {l__acro_short_#1_prop} {##1} {##2}
            \prop_put:cnn {l__acro_pdfstring_short_#1_prop} {##1} {##2}
          }
        % alt-<ending>:
        \acro_declare_property:nnn {alt_#1} {alt-#1}
          {
            \prop_put:cnn {l__acro_alt_#1_form_prop} {##1} { \c_false_bool }
            \prop_put:cnx {l__acro_pdfstring_alt_#1_prop}
              {##1} { \prop_item:Nn \l__acro_alt_prop {##1} \exp_not:n {##2} }
          }
        % alt-<ending>-form:
        \acro_declare_property_generic:nnn {alt_#1_form} {alt-#1-form}
          {
            \__acro_property_check:nn {##1} {alt-#1-form}
            \prop_put:cnn {l__acro_alt_#1_form_prop} {##1} { \c_true_bool }
            \prop_put:cnn {l__acro_alt_#1_prop} {##1} {##2}
            \prop_put:cnn {l__acro_pdfstring_alt_#1_prop} {##1} {##2}
          }
        % long-<ending>:
        \acro_declare_property:nnn {long_#1} {long-#1}
          { \prop_put:cnn {l__acro_long_#1_form_prop} {##1} { \c_false_bool } }
        % long-<ending>-form:
        \acro_declare_property_generic:nnn {long_#1_form} {long-#1-form}
          {
            \__acro_property_check:nn {##1} {long-#1-form}
            \prop_put:cnn {l__acro_long_#1_form_prop} {##1} { \c_true_bool }
            \prop_put:cnn {l__acro_long_#1_prop} {##1} {##2}
          }
        % options:
        %   short-<ending>-ending
        %   alt-<ending>-ending
        %   long-<ending>-ending
        %   <ending>-ending
        \keys_define:nn {acro}
          {
            short-#1-ending .code:n =
              \bool_if:NT \g__acro_first_acronym_declared_bool
                { \acro_serious_message:n {ending-before-acronyms} }
              \tl_set:cn {l__acro_default_short_#1_tl} {##1} ,
            alt-#1-ending   .code:n =
              \bool_if:NT \g__acro_first_acronym_declared_bool
                { \acro_serious_message:n {ending-before-acronyms} }
              \tl_set:cn {l__acro_default_alt_#1_tl} {##1} ,
            long-#1-ending  .code:n =
              \bool_if:NT \g__acro_first_acronym_declared_bool
                { \acro_serious_message:n {ending-before-acronyms} }
              \tl_set:cn {l__acro_default_long_#1_tl} {##1},
            #1-ending       .code:n   =
              \bool_if:NT \g__acro_first_acronym_declared_bool
                { \acro_serious_message:n {ending-before-acronyms} }
              \__acro_read_ending_settings:nww {#1} ##1// \acro_stop:
          }
        % pdfstrings:
        % TODO: add long forms:
        \prop_new:c {l__acro_pdfstring_short_#1_prop}
        \cs_new:cpn {acro_pdf_string_short_#1:n} ##1
          {
            \acro_if_star_gobble:nTF {##1}
              { \prop_item:cn {l__acro_pdfstring_short_#1_prop} }
              { \prop_item:cn {l__acro_pdfstring_short_#1_prop} {##1} }
          }
        \cs_new:cpn {acpdfstring#1} { \use:c {acro_pdf_string_short_#1:n} }
        \prop_new:c {l__acro_pdfstring_alt_#1_prop}
        \cs_new:cpn {acro_pdf_string_alt_#1:n} ##1
          {
            \acro_if_star_gobble:nTF {##1}
              { \prop_item:cn {l__acro_pdfstring_alt_#1_prop} }
              { \prop_item:cn {l__acro_pdfstring_alt_#1_prop} {##1} }
          }
        \cs_new:cpn {acpdfstringalt#1} { \use:c {acro_pdf_string_alt_#1:n} }
      }
  }

% #1: short|alt|long
% #2: ending name
% #3: default ending
\cs_new_protected:Npn \acro_define_and_set_ending_variables:nnn #1#2#3
  {
    \acro_define_ending_variables:nn {#1} {#2}
    \acro_set_ending_variables:nnn {#1} {#2} {#3}
  }

% #1: short|alt|long
% #2: ending name
\cs_new_protected:Npn \acro_define_ending_variables:nn #1#2
  {
    \prop_new:c {l__acro_#1_#2_prop}
    \prop_new:c {l__acro_#1_#2_form_prop}
    \tl_new:c   {l__acro_#1_#2_tl}
    \tl_new:c   {l__acro_default_#1_#2_tl}
  }

% #1: short|alt|long
% #2: ending name
% #3: default ending
\cs_new_protected:Npn \acro_set_ending_variables:nnn #1#2#3
  { \tl_set:cn  {l__acro_default_#1_#2_tl} {#3} }

% #1: ending name
% #2: short (and long if #4 is blank)
% #3: long
\cs_new_protected:Npn \__acro_read_ending_settings:nww #1#2/#3/#4 \acro_stop:
  {
    \acro_set_ending_variables:nnn {short} {#1} {#2}
    \acro_set_ending_variables:nnn {alt} {#1} {#2}
    \tl_if_blank:nTF {#4}
      { \acro_set_ending_variables:nnn {long} {#1} {#3} }
      { \acro_set_ending_variables:nnn {long} {#1} {#2} }
  }

\NewDocumentCommand \ProvideAcroEnding {mmm}
  { \acro_provide_ending:nnn {#1} {#2} {#3} }

% --------------------------------------------------------------------------
% enable us to know if the acronym is used only once and provide a different
% style for that:
\prg_new_protected_conditional:Npnn \acro_is_used:n #1 { T,F,TF }
  {
    \acro_record_barrier:n {#1}
    \bool_if:nTF
      {
        \bool_if_p:c { g__acro_#1_used_bool } &&
        (
          (
            \bool_if_p:c { g__acro_#1_first_use_bool } &&
            \g__acro_mark_first_as_used_bool
          )
          ||
          ! \g__acro_mark_first_as_used_bool
        )
      }
      {
        \bool_if:NTF \l__acro_mark_as_used_bool
          {
            \__acro_aux_file:Nxxxx \acro@used@twice
              {#1}
              { \thepage }
              { \arabic {page} }
              { \arabic {abspage} }
          }
          { \__acro_aux_file:Nxxxx \acro@used@twice {#1} {} {} {} }
        \prg_return_true:
      }
      {
        \bool_if:NTF \l__acro_mark_as_used_bool
          {
            \__acro_aux_file:Nxxxx \acro@used@once
              {#1}
              { \thepage }
              { \arabic {page} }
              { \arabic {abspage} }
            \bool_if:nT
              {
                !\bool_if_p:c { g__acro_#1_label_bool } &&
                \l__acro_place_label_bool
              }
              {
                \bool_gset_true:c { g__acro_#1_label_bool }
                \label{\l__acro_label_prefix_tl #1}
              }
            \bool_gset_true:c { g__acro_#1_used_bool }
          }
          { \__acro_aux_file:Nxxxx \acro@used@once {#1} {} {} {} }
        \prg_return_false:
      }
  }

\cs_new:Npn \acro_is_used:n #1
  { \acro_is_used:nTF {#1} { } { } }

\cs_new_protected:Npn \__acro_aux_file:Nnnnn #1#2#3#4#5
  { \iow_shipout:Nn \@auxout { #1 {#2} {#3} {#4} {#5} } }
\cs_generate_variant:Nn \__acro_aux_file:Nnnnn { Nxxxx }
  
\cs_new_protected:Npn \__acro_aux_file_now:n #1
  { \iow_now:Nn \@auxout {#1} }
\cs_generate_variant:Nn \__acro_aux_file_now:n { x }

% --------------------------------------------------------------------------
% the commands for the auxiliary file:
\cs_new_protected:Npn \acro@used@once #1#2#3#4
  {
    \cs_gset_nopar:cpn {acro@#1@once} {#1}
    \bool_gset_true:c {g__acro_#1_in_list_bool}
    \tl_if_empty:nF {#2#3#4}
      {
        % \bool_gset_true:c { g__acro_#1_used_bool }
        \seq_gput_right:cn {g__acro_#1_pages_seq} { {#2}{#3}{#4} }
      }
  }
\cs_new_protected:Npn \acro@used@twice #1#2#3#4
  {
    \cs_gset_nopar:cpn {acro@#1@twice} {#1}
    \tl_if_empty:nF {#2#3#4}
      { \seq_gput_right:cn {g__acro_#1_pages_seq} { {#2}{#3}{#4} } }
  }

\cs_new_protected:Npn \acro@pages #1#2
  { \tl_gset:cn {g__acro_#1_recorded_pages_tl} {#2} }

\bool_new:N \g__acro_rerun_bool

\cs_new_protected:Npn \acro@rerun@check
  {
    \bool_if:NT \g__acro_rerun_bool
      {
        \@latex@warning@no@line
          {Acronyms~ may~ have~ changed.~ Please~ rerun~ LaTeX}
      }
  }

\AtEndDocument
  {
    \bool_gset_false:N \g__acro_rerun_bool
    \cs_gset_protected:Npn \acro@used@once #1#2#3#4
      {
        \tl_set:Nn \l__acro_tmpa_tl {#1}
        \tl_if_eq:cNF {acro@#1@once} \l__acro_tmpa_tl
          { \bool_gset_true:N \g__acro_rerun_bool }
      }
    \cs_gset_protected:Npn \acro@used@twice #1#2#3#4
      {
        \tl_set:Nn \l__acro_tmpa_tl {#1}
        \tl_if_eq:cNF {acro@#1@twice} \l__acro_tmpa_tl
          { \bool_gset_true:N \g__acro_rerun_bool }
      }
    \acro_for_all_acronyms_do:n
      {
        \seq_if_empty:cF {g__acro_#1_pages_seq}
          {
            \__acro_aux_file_now:x
              {
                \token_to_str:N \acro@pages {#1}
                  { \seq_use:cn {g__acro_#1_pages_seq} {|} } ^^J
                \token_to_str:N \acro@barriers {#1}
                  { \seq_use:cn {g__acro_#1_barriers_seq} {,} }
              }
          }
        \acro_check_barriers:n {#1}
      }
    \__acro_aux_file_now:n { \acro@rerun@check }
  }

% if `acro' is deactivated prevent unnecessary errors from aux file:
\if@filesw
\AtBeginDocument
  {
    \__acro_aux_file_now:n
      {
        \providecommand \acro@used@once [4] {} ^^J
        \providecommand \acro@used@twice [4] {} ^^J
        \providecommand \acro@pages [2] {} ^^J
        \providecommand \acro@rerun@check {} ^^J
        \providecommand \acro@print@list {} ^^J
        \providecommand \acro@barriers [2] {}
      }
  }
\fi

% --------------------------------------------------------------------------
% typeset the short form:
% #1: ID
% #2: short form
\cs_new_protected:Npn \acro_write_short:nn #1#2
  {
    \mode_if_horizontal:F { \leavevmode }
    \group_begin:
      \bool_if:NTF \l__acro_custom_format_bool
        { \l__acro_custom_format_tl }
        { \l__acro_short_format_tl }
      {#2}
    \group_end:
  }
\cs_generate_variant:Nn \acro_write_short:nn { nV , nv }

% typeset the alternative form:
% #1: ID
% #2: alt form
\cs_new_protected:Npn \acro_write_alt:nn #1#2
  {
    \mode_if_horizontal:F { \leavevmode }
    \group_begin:
      \bool_if:NTF \l__acro_custom_format_bool
        { \l__acro_custom_format_tl }
        { \l__acro_alt_format_tl }
      {#2}
    \group_end:
  }
\cs_generate_variant:Nn \acro_write_alt:nn { nV , nv }

% typeset a long form:
%   TODO: rethink the formatting mechanism
%   right now a custom format gets applied additionally to the global one
%   although before it
% #1: format
% #2: long form
\cs_new_protected:Npn \acro_write_long:nn #1#2
  {
    \mode_if_horizontal:F { \leavevmode }
    \group_begin:
      \bool_if:NTF \l__acro_custom_long_format_bool
        { \l__acro_custom_long_format_tl }
        { \use:n }
      {
        \use:x
          {
            \exp_not:n {#1}
            {
              \bool_if:NTF \l__acro_first_upper_bool
                { \exp_not:N \__acro_first_upper_case:n { \exp_not:n {#2} } }
                { \exp_not:n {#2} }
            }
          }
      }
    \group_end:
  }
\cs_generate_variant:Nn \acro_write_long:nn { VV,Vo,Vf,V,v,vv }

\prg_new_conditional:Npnn \acro_if_foreign:n #1 {T,F,TF}
  {
    \bool_if:nTF
      {
        \l__acro_foreign_bool
        &&
        \prop_if_in_p:Nn \l__acro_foreign_prop {#1}
      }
      { \prg_return_true: }
      { \prg_return_false: }
  }

\cs_new_protected:Npn \acro_foreign_language:nn #1#2 {}
\AtBeginDocument{
  \cs_if_exist:NTF \foreignlanguage
    {
      \cs_set_protected:Npn \acro_foreign_language:nn #1#2
        {
          \tl_if_blank:nTF {#1}
            {#2}
            { \foreignlanguage {#1} {#2} }
        }
    }
    {
      \cs_set_protected:Npn \acro_foreign_language:nn #1#2
        { \use_ii:nn {#1} {#2} }
    }
}
\cs_generate_variant:Nn \acro_foreign_language:nn {VV}

\cs_new_protected:Npn \acro_write_foreign:n #1
  {
    \acro_if_foreign:nT {#1}
      {
        \prop_get:NnNT \l__acro_foreign_prop {#1} \l__acro_foreign_tl
          {
            \group_begin:
              \tl_use:N \l__acro_foreign_format_tl
              {
                \acro_foreign_language:VV
                  \l__acro_foreign_lang_tl
                  \l__acro_foreign_tl
              }
            \group_end:
          }
      }
  }

\cs_new:Npn \acroenparen #1 { ( #1 ) }

\cs_new_protected:Npn \acro_get_foreign:n #1
  {
    \prop_get:NnNT \l__acro_foreign_prop {#1} \l__acro_foreign_tl
      {
        \tl_use:N \l__acro_foreign_sep_tl
        \group_begin:
          \tl_use:N \l__acro_foreign_list_format_tl
          {
            \acro_foreign_language:VV
              \l__acro_foreign_lang_tl
              \l__acro_foreign_tl
          }
        \group_end:
      }
  }

% --------------------------------------------------------------------------
% #1: id
% #2: short|alt
\cs_set_protected:Npn \acro_write_compact:nn #1#2
  {
    \acro_get_ending_form:nn {#1} {#2}
    \acro_acc_supp:nn
      {#1}
      {
        \acro_write_tooltip:nnV
          {#1}
          {
            \use:c {acro_write_#2:nv} {#1} {l__acro_#2_tl}
            \acro_endings:nn {#1} {#2}
          }
          \l__acro_long_tl
      }
  }

% TODO: get rid of argument #3?
% #1: ID
% #2: long|first-long|list-long|extra
% #3: long form
\cs_new_protected:Npn \acro_write_expanded:nnn #1#2#3
  {
    \tl_set:Nn \l__acro_tmpa_tl {#2}
    \tl_replace_all:Nnn \l__acro_tmpa_tl {-} {_}
    \acro_write_long:vn {l__acro_ \l__acro_tmpa_tl _format_tl} {#3}
    \acro_endings:nn {#1} {long}
    \tl_if_in:nnT {#2} {long}
      { \l__acro_long_post_tl }
  }
\cs_generate_variant:Nn \acro_write_expanded:nnn { nnV }

% #1: ID
% #2: long|first-long|list-long|extra
\cs_new_protected:Npn \acro_write_expanded:nn #1#2
  {
    \tl_set:Nn \l__acro_tmpa_tl {#2}
    \tl_replace_all:Nnn \l__acro_tmpa_tl {-} {_}
    \acro_write_long:vv
      {l__acro_ \l__acro_tmpa_tl _format_tl}
      {l__acro_ \l__acro_tmpa_tl _tl}
    \acro_endings:nn {#1} {long}
    \tl_if_in:nnT {#2} {long}
      { \l__acro_long_post_tl }
  }

% #1: id
\cs_new:Npn \acro_after:n #1
  {
    \acro_cite_if:nn { \l__acro_citation_all_bool } {#1}
    \acro_index_if:nn { \l__acro_addto_index_bool } {#1}
  }

\cs_new_protected:Npn \acro_check_single:n #1
  {
    \acro_if_is_single:nT {#1}
      { \cs_set_eq:NN \acro_hyper_link:nn \use_ii:nn }
  }

% --------------------------------------------------------------------------
% #1: id
\cs_new_protected:Npn \acro_before:n #1
  {
    \acro_get:n {#1}
    \acro_is_used:n {#1}
    \acro_check_single:n {#1}
  }

% the standard internals:
% #1: id
\cs_new_protected:Npn \acro_short:n #1
  {
    \acro_before:n {#1}
    \acro_write_indefinite:nn {#1} {short}
    \acro_write_compact:nn {#1} {short}
    \acro_after:n {#1}
  }

% get alternative entry:
% #1: id
\cs_new_protected:Npn \acro_alt:n #1
  {
    \acro_before:n {#1}
    \acro_alt_error:n {#1}
    \acro_write_indefinite:nn {#1} {alt}
    \acro_write_compact:nn {#1} {alt}
    \acro_after:n {#1}
  }

% get long entry:
% #1: id
\cs_new_protected:Npn \acro_long:n #1
  {
    \acro_before:n {#1}
    \acro_write_indefinite:nn {#1} {long}
    \acro_write_expanded:nn {#1} {long}
    \acro_after:n {#1}
  }

% get foreign entry:
% #1: id
\cs_new_protected:Npn \acro_foreign:n #1
  {
    \acro_get:n {#1}
    \tl_if_blank:VF \l__acro_foreign_tl
      {
        \acro_is_used:n {#1}
        \acro_check_single:n {#1}
        \acro_write_long:VV \l__acro_foreign_format_tl \l__acro_foreign_tl
        \acro_after:n {#1}
      }
  }

% get extra entry:
% #1: id
\cs_new_protected:Npn \acro_extra:n #1
  {
    \acro_get:n {#1}
    \tl_if_blank:VF \l__acro_extra_tl
      {
        \acro_is_used:n {#1}
        \acro_check_single:n {#1}
        \acro_write_long:VV \l__acro_extra_format_tl \l__acro_extra_tl
        \acro_after:n {#1}
      }
  }

% output like the first time:
% #1: id
\cs_new_protected:Npn \acro_first:n #1
  {
    \bool_gset_true:c {g__acro_#1_first_use_bool}
    \acro_before:n {#1}
    \acro_first_instance:nV {#1} \l__acro_long_tl
  }

% output like the first time with own long version:
% #1: id
% #2: instead of long entry
\cs_new_protected:Npn \acro_first_like:nn #1#2
  {
    \bool_gset_true:c {g__acro_#1_first_use_bool}
    \acro_before:n {#1}
    \acro_first_instance:nn {#1} {#2}
  }

% ----------------------------------------------------------------------------
% citations:
\cs_new:Npn \__acro_citation_cmd:w { \cite } %{}
\cs_new:Npn \__acro_group_citation_cmd:w { \cite } %{}

% #1 pre
% #2 post
% #3 key
\cs_new:Npn \__acro_cite:nnn #1#2#3
  {
    \quark_if_no_value:nTF {#1}
      { \__acro_citation_cmd:w {#3} }
      {
        \quark_if_no_value:nTF {#2}
          { \__acro_citation_cmd:w [ #1 ] {#3} }
          { \__acro_citation_cmd:w [ #1 ] [ #2 ] {#3} }
      }
  }
\cs_generate_variant:Nn \__acro_cite:nnn { VVV }

\cs_new_protected:Npn \acro_cite:n #1
  {
    \prop_get:NnNT \l__acro_citation_prop {#1} \l__acro_tmpa_tl
      {
        \prop_get:NnN \l__acro_citation_pre_prop {#1} \l__acro_tmpb_tl
        \prop_get:NnN \l__acro_citation_post_prop {#1} \l__acro_tmpc_tl
        \acro_no_break:
        \tl_use:N \l__acro_citation_connect_tl
        \__acro_cite:VVV
          \l__acro_tmpb_tl
          \l__acro_tmpc_tl
          \l__acro_tmpa_tl
      }
  }

\cs_new_protected:Npn \acro_group_cite:n #1
  {
    \group_begin:
      \cs_set_eq:NN \__acro_citation_cmd:w \__acro_group_citation_cmd:w
      \tl_set_eq:NN
        \l__acro_citation_connect_tl
        \l__acro_between_group_connect_citation_tl
      \acro_cite_if:nn { \l__acro_citation_first_bool } {#1}
    \group_end:
  }

\cs_new_protected:Npn \acro_cite_if:nn #1#2
  { \bool_if:nT {#1} { \acro_cite:n {#2} } }

% ----------------------------------------------------------------------------
% indexing:
\cs_new_protected:Npn \acro_index_if:nn #1#2
  {
    \bool_if:nT { (#1) && \l__acro_mark_as_used_bool }
      {
        \prop_get:NnN \l__acro_index_cmd_prop  {#2} \l__acro_tmpa_tl
        \prop_get:NnN \l__acro_index_sort_prop {#2} \l__acro_tmpb_tl
        \prop_get:NnN \l__acro_index_prop      {#2} \l__acro_tmpc_tl
        \__acro_index:VnVV
          \l__acro_tmpa_tl
          {#2}
          \l__acro_tmpb_tl
          \l__acro_tmpc_tl
      }
  }

\cs_new:Npn \__acro_index_cmd:n { \index }

% #1: cmd
% #2: key
% #3: sort
% #4: replace
\cs_new_protected:Npn \__acro_index:nnnn #1#2#3#4
  {
    \prop_get:NnNF \l__acro_short_prop  {#2} \l__acro_index_short_tl {}
    \prop_get:NnNF \l__acro_format_prop {#2} \l__acro_index_format_tl {}
    \quark_if_no_value:VTF \l__acro_index_format_tl
      { \tl_set:Nn \l__acro_tmpa_tl { \l__acro_short_format_tl \l__acro_index_short_tl } }
      { \tl_set:Nn \l__acro_tmpa_tl { \l__acro_index_format_tl \l__acro_index_short_tl } }
    \quark_if_no_value:nF {#1}
      { \cs_set:Npn \__acro_index_cmd:n {#1} }
    \quark_if_no_value:nTF {#4}
      {
        \quark_if_no_value:nTF {#3}
          { \__acro_index_cmd:n { #2 @ { \l__acro_tmpa_tl } } }
          { \__acro_index_cmd:n { #3 @ { \l__acro_tmpa_tl } } }
      }
      { \__acro_index_cmd:n {#4} }
  }
\cs_generate_variant:Nn \__acro_index:nnnn { VnVV }

% ----------------------------------------------------------------------------
% accessability support
\cs_new_eq:NN \acro_acc_supp:nn \use_ii:nn

\cs_new_protected:Npn \acro_get_acc_supp:nn #1#2
  {
    \prop_get:NnNF \l__acro_acc_supp_prop {#1} \l__acro_acc_supp_tl
      { \prop_get:NnNF \l__acro_short_prop {#1} \l__acro_acc_supp_tl {} }
    \acro_for_endings_do:n
      {
        \bool_if:cT {l__acro_##1_bool}
          {
            \tl_put_right:Nv
              \l__acro_acc_supp_tl
              {l__acro_short_##1_tl}
          }
      }
    \acro_do_acc_supp:VVn
      \l__acro_acc_supp_tl
      \l__acro_acc_supp_options_tl
      {#2}
  }

\cs_new:Npn \acro_do_acc_supp:nnn #1#2#3
  {
    \BeginAccSupp { ActualText = #1 , #2 }
      #3
    \EndAccSupp { }
  }
\cs_generate_variant:Nn \acro_do_acc_supp:nnn { VV }

\AtEndPreamble
  {
    \bool_if:NT \l__acro_acc_supp_bool
      {
        \RequirePackage {accsupp}
        \cs_set_eq:NN \acro_acc_supp:nn \acro_get_acc_supp:nn
      }
    \bool_if:NT \l__acro_tooltip_bool
      {
        \RequirePackage {pdfcomment}
        \cs_if_eq:NNT \__acro_tooltip_cmd:nn \use_i:nn
          { \cs_set:Npn \__acro_tooltip_cmd:nn { \pdftooltip } }
      }
  }

% --------------------------------------------------------------------------
% tooltips for acronyms

% #1: id
% #2: printed text
% #3: tool tip text
\cs_new_protected:Npn \acro_write_tooltip:nnn #1#2#3
  {
    \prop_get:NnNTF \l__acro_tooltip_prop {#1} \l__acro_tmpa_tl
      { \__acro_check_tooltip:nV {#2} \l__acro_tmpa_tl }
      { \__acro_check_tooltip:nn {#2} {#3} }
  }
\cs_generate_variant:Nn \acro_write_tooltip:nnn { nnV }

% #1: printed text
% #2: tool tip text
\cs_new_protected:Npn \__acro_check_tooltip:nn #1#2
  {
    \bool_if:NTF \l__acro_inside_tooltip_bool
      {#1}
      {
        \bool_set_true:N \l__acro_inside_tooltip_bool
        \__acro_tooltip_cmd:nn {#1} {#2}
      }
  }
\cs_generate_variant:Nn \__acro_check_tooltip:nn { nV }

% use whatever command you like for creating tooltips here:
% #1: printed text
% #2: tool tip text
\cs_new_eq:NN \__acro_tooltip_cmd:nn \use_i:nn
  
% --------------------------------------------------------------------------
% indefinite articles:

% #1: ID
% #2: short|long|alt
\cs_new_protected:Npn \acro_write_indefinite:nn #1#2
  {
    \bool_if:NT \l__acro_indefinite_bool
      { \prop_item:cn { l__acro_#2_indefinite_prop } {#1} ~ }
    \bool_if:NT \l__acro_upper_indefinite_bool
      { %  \bool_set_true:N \l__acro_first_upper_bool
         \__acro_first_upper_case:x
           { \prop_item:cn { l__acro_#2_indefinite_prop } {#1} } ~
      }
  }

% --------------------------------------------------------------------------
% experimental sorting feature:

% the following code is an adaption of expl3 code used for \str_if_eq:NN(TF)
\sys_if_engine_luatex:TF
  {
    \tl_set:Nn \l__acro_tmpa_tl
      {
        acro ~ = ~ acro ~ or ~ { ~ } ~
        function ~ acro.strcmp ~ (A, B) ~
          if ~ A ~ == ~ B ~ then ~
            tex.write ("0") ~
          elseif ~ A ~ < ~ B ~ then ~
            tex.write ("-1") ~
          else ~
            tex.write ("1") ~
          end ~
        end
      }
    \luatex_directlua:D { \l__acro_tmpa_tl }
    \cs_new_protected:Npn \acro_strcmp:nn #1#2
      {
        \luatex_directlua:D
          {
            acro.strcmp
              (
                " \__acro_escape_x:n {#1} " ,
                " \__acro_escape_x:n {#2} "
              )
          }
      }
    \cs_new:Npn \__acro_escape_x:n #1
      {
        \luatex_luaescapestring:D
          { \etex_detokenize:D \exp_after:wN { \luatex_expanded:D {#1} } }
      }
  }
  { \cs_new_eq:NN \acro_strcmp:nn \pdftex_strcmp:D }

\AtBeginDocument
  {
    \bool_if:NT \l__acro_sort_bool
      {
        \cs_new_protected:Npn \acro_sort_prop:NN #1#2
          {
            \seq_clear:N  \l__acro_tmpa_seq
            \prop_clear:N \l__acro_tmpa_prop
            \prop_clear:N \l__acro_tmpb_prop
            \prop_map_inline:Nn #2
              {
                \seq_put_right:Nn \l__acro_tmpa_seq {##2}
                \prop_put:Nnn \l__acro_tmpa_prop {##1} {##2}
              }
            \seq_sort:Nn \l__acro_tmpa_seq
              {
                \int_compare:nTF
                  {
                    \acro_strcmp:nn
                      { \str_fold_case:n {##1} }
                      { \str_fold_case:n {##2} }
                        = \c_minus_one
                  }
                  { \sort_return_same: }
                  { \sort_return_swapped: }
              }
            \seq_map_inline:Nn \l__acro_tmpa_seq
              {
                \prop_map_inline:Nn \l__acro_tmpa_prop
                  {
                    \str_if_eq:nnT {##1} {####2}
                      {
                        \prop_get:NnN #1 {####1} \l__acro_tmpa_tl
                        \prop_put:NnV \l__acro_tmpb_prop {####1}
                          \l__acro_tmpa_tl
                      }
                  }
              }
            \prop_set_eq:NN #1 \l__acro_tmpb_prop
          }
      }
  }

% --------------------------------------------------------------------------
% regarding list printing:
% this command ensures that a rerun warning is given when \printacronyms
% is set the first time. This mechanism doesn't make very much sense,
% should be replaced by a different and more efficient one
%
\cs_new_protected:Npn \acro@print@list
  { \cs_if_exist:NF \acro@printed@list { \cs_new:Npn \acro@printed@list { printed } } }

% --------------------------------------------------------------------------
% trailing tokens and what to do when present
\prop_new:N \l__acro_trailing_tokens_prop
\prop_new:N \l__acro_trailing_actions_prop
\bool_new:N \l__acro_trailing_tokens_bool
\tl_new:N   \l__acro_trailing_tokens_tl

\cs_new_protected:Npn \acro_new_trailing_token:n #1
  { \bool_new:c {l__acro_trailing_#1_bool} }
\cs_new_protected:Npn \acro_activate_trailing_action:n #1
  { \bool_set_true:c {l__acro_trailing_#1_bool} }
\cs_new_protected:Npn \acro_deactivate_trailing_action:n #1
  { \bool_set_false:c {l__acro_trailing_#1_bool} }

% register a new token but don't activate its action:
% #1: token
% #2: name
\cs_new_protected:Npn \acro_register_trailing_token:Nn #1#2
  {
    \prop_put:Nnn \l__acro_trailing_tokens_prop {#2} {#1}
    \prop_put:Nnn \l__acro_trailing_actions_prop {#1}
      { \acro_activate_trailing_action:n {#2} }
    \acro_new_trailing_token:n {#2}
  }
  
\NewDocumentCommand \AcroRegisterTrailing {mm}
  { \acro_register_trailing_token:Nn #1 {#2} }

\cs_new_protected:Npn \acro_for_all_trailing_tokens_do:n #1
  { \prop_map_inline:Nn \l__acro_trailing_tokens_prop {#1} }

% activate a token:
\cs_new_protected:Npn \acro_activate_trailing_token:n #1
  {
    \prop_get:NnN \l__acro_trailing_tokens_prop {#1} \l__acro_tmpa_tl
    \tl_put_right:NV \l__acro_trailing_tokens_tl \l__acro_tmpa_tl
  }

% deactivate a token:
\cs_new_protected:Npn \acro_deactivate_trailing_token:n #1
  {
    \prop_get:NnN \l__acro_trailing_tokens_prop {#1} \l__acro_tmpa_tl
    \tl_remove_all:NV \l__acro_trailing_tokens_tl \l__acro_tmpa_tl
  }

% #1: name
\prg_new_conditional:Npnn \acro_if_trailing_token:n #1 {p,T,F,TF}
  {
    \bool_if:cTF {l__acro_trailing_#1_bool}
      { \prg_return_true: }
      { \prg_return_false: }
  }

% #1: csv list of names
\prg_new_protected_conditional:Npnn \acro_if_trailing_tokens:n #1 {T,F,TF}
  {
    \bool_set_false:N \l__acro_trailing_tokens_bool
    \clist_map_inline:nn {#1}
      {
        \bool_if:cT {l__acro_trailing_##1_bool}
          {
            \bool_set_true:N \l__acro_trailing_tokens_bool
            \clist_map_break:
          }
      }
    \bool_if:NTF \l__acro_trailing_tokens_bool
      { \prg_return_true: }
      { \prg_return_false: }
  }

\cs_new_protected:Npn \aciftrailing { \acro_if_trailing_tokens:nTF }

\cs_new_protected:Npn \__acro_check_trail:N #1
  {
    \tl_map_inline:Nn \l__acro_trailing_tokens_tl
      {
        \token_if_eq_meaning:NNT #1 ##1
          { \prop_item:Nn \l__acro_trailing_actions_prop {##1} }
      }
  }

% options for activating actions:
\keys_define:nn {acro}
  {
    activate-trailing-tokens   .code:n =
      \clist_map_inline:nn {#1} { \acro_activate_trailing_token:n {##1} } ,
    activate-trailing-tokens   .initial:n = dot ,
    deactivate-trailing-tokens .code:n =
      \clist_map_inline:nn {#1} { \acro_deactivate_trailing_token:n {##1} }
  }

% some user macros:
\cs_new_protected:Npn \acro_dot:
  { \acro_if_trailing_tokens:nF {dot} {.\@} }

\cs_new_protected:Npn \acro_space:
  { \acro_if_trailing_tokens:nF {dash,babel-hyphen} { \c_space_tl } }

\NewDocumentCommand \acdot   {} { \acro_dot: }
\NewDocumentCommand \acspace {} { \acro_space: }
  
% ---------------------------------------------------------------------------
% reset outputs, they'll behave like the first time again (!not like the _only_
% time!):
\cs_new_protected:Npn \acro_reset:n #1
  {
    \bool_gset_false:c { g__acro_#1_used_bool }
    \bool_gset_false:c { g__acro_#1_first_use_bool }
  }

\cs_new_protected:Npn \acro_mark_as_used:n #1
  {
    \bool_gset_true:c { g__acro_#1_used_bool }
    \bool_gset_true:c { g__acro_#1_first_use_bool }
    \bool_gset_true:c { g__acro_#1_in_list_bool }
    \if@filesw
      \__acro_aux_file_now:n { \acro@used@once {#1} {} {} {} }
      \__acro_aux_file_now:n { \acro@used@twice {#1} {} {} {} }
    \fi
  }

\cs_new_protected:Npn \acro_reset_all:
  { \acro_for_all_acronyms_do:n { \acro_reset:n {##1} } }

% make sure that no acronym is used at the beginning of the document, see
% issue #81 for reasons why this may be necessary:
\AfterEndPreamble { \acro_reset_all: }
  
\cs_new_protected:Npn \acro_mark_all_as_used:
  { \acro_for_all_acronyms_do:n { \acro_mark_as_used:n {##1} } }

\DeclareExpandableDocumentCommand \acifused { m }
  { \acro_if_acronym_used:nTF {#1} }

\prg_new_conditional:Npnn \acro_if_acronym_used:n #1 { p,T,F,TF }
  {
    \bool_if:nTF
      {
        \bool_if_p:c { g__acro_#1_used_bool } &&
        ( !\acro_if_is_single_p:n {#1} )
      }
      { \prg_return_true: }
      { \prg_return_false: }
  }

\NewDocumentCommand \acresetall {}
  { \acro_reset_all: }

\NewDocumentCommand \acuseall {}
  { \acro_mark_all_as_used: }

\NewDocumentCommand \acreset { > { \SplitList { , } } m }
  { \ProcessList {#1} { \acro_reset:n } \ignorespaces }

\NewDocumentCommand \acuse { > { \SplitList { , } } m }
  { \ProcessList {#1} { \acro_mark_as_used:n } \ignorespaces }

% --------------------------------------------------------------------------
% acronym barriers: allow local lists of only those acronyms used between two
% barriers

\int_new:N  \g__acro_barrier_int
\bool_new:N \g__acro_use_barriers_bool
\bool_new:N \g__acro_reset_at_barrier_bool
\bool_new:N \l__acro_use_barrier_bool

\keys_define:nn {acro}
  {
    use-barriers      .bool_gset:N = \g__acro_use_barriers_bool ,
    use-barriers      .initial:n   = false ,
    reset-at-barriers .bool_gset:N = \g__acro_reset_at_barrier_bool ,
    reset-at-barriers .initial:n   = false
  }

\cs_new_protected:Npn \acro_barrier:
  {
    \int_gincr:N \g__acro_barrier_int
    \bool_if:NT \g__acro_reset_at_barrier_bool
      { \acro_reset_all: }
  }

\NewDocumentCommand \acbarrier {}
  { \acro_barrier: }

\cs_new_protected:Npn \acro_check_barriers:n #1
  {
    \bool_if:NT \g__acro_use_barriers_bool
      {
        \tl_set:Nx \l__acro_tmpa_tl
          { \seq_use:cn {g__acro_#1_barriers_seq} {} }
        \tl_set:Nx \l__acro_tmpb_tl
          { \seq_use:cn {g__acro_#1_recorded_barriers_seq} {} }
        \tl_if_eq:NNF \l__acro_tmpa_tl \l__acro_tmpb_tl
          {
            \@latex@warning@no@line
              {Rerun~to~get~barriers~of~acronym~#1~right}
          }
      }
  }

\cs_new_protected:Npn \acro_record_barrier:n #1
  {
    \bool_if:NT \g__acro_use_barriers_bool
      {
        \seq_if_in:cxF {g__acro_#1_barriers_seq}
          { \int_use:N \g__acro_barrier_int }
          {
            \seq_gput_right:cx  {g__acro_#1_barriers_seq}
              { \int_use:N \g__acro_barrier_int }
          }
      }
  }

% #1: id
% #2: barrier number
\prg_new_protected_conditional:Npnn \acro_if_in_barrier:nn #1#2 {T,F,TF}
  {
    \seq_if_in:cnTF {g__acro_#1_recorded_barriers_seq} {#2}
      { \prg_return_true: }
      { \prg_return_false: }
  }
\cs_generate_variant:Nn \acro_if_in_barrier:nnTF {nx}

\cs_new:Npn \acro@barriers #1#2
  { \seq_gset_split:cnn {g__acro_#1_recorded_barriers_seq} {,} {#2} }

% --------------------------------------------------------------------------
% the user commands -- preparation:
\cs_new_protected:Npn \acro_begin:
  {
    \group_begin:
    \__acro_check_after_end:w
  }

\cs_new_protected:Npn \__acro_check_after_end:w #1 \acro_end:
  {
    \cs_set:Npn \__acro_execute:
      {
        \__acro_check_trail:N \l_peek_token
        #1
        \acro_end: % this will end the group opened by \acro_begin:
      }
    \peek_after:Nw \__acro_execute:
  }

\cs_new_protected:Npn \acro_end: { \group_end: }

\cs_new_protected:Npn \acro_reset_specials:
  {
    \bool_set_false:N \l__acro_indefinite_bool
    \bool_set_false:N \l__acro_first_upper_bool
    \bool_set_false:N \l__acro_upper_indefinite_bool
    % \bool_set_false:N \l__acro_citation_all_bool
    % \bool_set_false:N \l__acro_citation_first_bool
    % \bool_set_false:N \l__acro_addto_index_bool
    \acro_for_endings_do:n { \bool_set_false:c {l__acro_##1_bool} }
  }

% #1: ID
% #2: true|false
\cs_new_protected:Npn \acro_check_acronym:nn #1#2
  {
    \acro_defined:n {#1}
    \acro_use_acronym:n {#2}
  }

% #1: boolean
% #2: ID
\cs_new_protected:Npn \acro_check_and_mark_if:nn #1#2
  {
    \bool_if:nTF
      { (#1) || !\l__acro_use_acronyms_bool }
      { \acro_check_acronym:nn {#2} {false} }
      { \acro_check_acronym:nn {#2} {true} }
  }

\cs_new_protected:Npn \acro_switch_off:
  { \bool_set_false:N \l__acro_use_acronyms_bool }

\cs_new_protected:Npn \acro_switch_on:
  { \bool_set_true:N \l__acro_use_acronyms_bool }

\NewDocumentCommand \acswitchoff {}
  { \acro_switch_off: }

\NewDocumentCommand \acswitchon {}
  { \acro_switch_on: }

% commands for (re)defining \ac-like macros:
\cs_new_protected:Npn \acro_define_new_acro_command:NN #1#2
  {
    % #1: csname
    % #2: definition where `#1' refers to the ID
    \cs_new_protected:Npn #1 ##1##2
      {
        \cs_set:Npn \__acro_tmp_command:n ####1 {##2}
        \exp_args:NNnx #2 ##1 {sO{}m}
          {
            \acro_begin:
              \acro_reset_specials:
              \keys_set:nn {acro} {########2}
              \acro_check_and_mark_if:nn {########1} {########3}
              \exp_not:o { \__acro_tmp_command:n {####3} }
            \acro_end:
          }
      }
  }
\cs_generate_variant:Nn \acro_define_new_acro_command:NN {cc}

% commands for (re)defining \acflike-like macros:
\cs_new_protected:Npn \acro_define_new_acro_pseudo_command:NN #1#2
  {
    % #1: csname
    % #2: definition where `#1' refers to the ID and `#2' to the pseudo long form
    \cs_new_protected:Npn #1 ##1##2
      {
        \cs_set:Npn \__acro_tmp_command:nn ####1####2 {##2}
        \exp_args:NNnx #2 ##1 {smm}
          {
            \acro_begin:
              \acro_reset_specials:
              \acro_check_and_mark_if:nn {########1} {########2}
              \exp_not:o { \__acro_tmp_command:nn {####2} {####3} }
            \acro_end:
          }
      }
  }
\cs_generate_variant:Nn \acro_define_new_acro_pseudo_command:NN {cc}

\clist_map_inline:nn {New,Renew,Declare,Provide}
  {
    \acro_define_new_acro_command:cc
      {#1AcroCommand}
      {#1DocumentCommand}
    \acro_define_new_acro_pseudo_command:cc
      {#1PseudoAcroCommand}
      {#1DocumentCommand}
  }

% --------------------------------------------------------------------------
% user commands -- facilities
\cs_new_protected:Npn \acro_first_upper:
  {
    \bool_if:NTF \l__acro_indefinite_bool
      {
        \bool_set_false:N \l__acro_indefinite_bool
        \bool_set_true:N \l__acro_upper_indefinite_bool
      }
      { \bool_set_true:N \l__acro_first_upper_bool }
  }

\cs_new_protected:Npn \acro_indefinite:
  {
    \bool_if:NTF \l__acro_first_upper_bool
      {
        \bool_set_true:N \l__acro_upper_indefinite_bool
        \bool_set_false:N \l__acro_first_upper_bool
      }
      { \bool_set_true:N \l__acro_indefinite_bool }
  }

\cs_new_protected:Npn \acro_cite:
  {
    \bool_set_true:N \l__acro_citation_all_bool
    \bool_set_true:N \l__acro_citation_first_bool
  }

\cs_new_protected:Npn \acro_no_cite:
  {
    \bool_set_false:N \l__acro_citation_all_bool
    \bool_set_false:N \l__acro_citation_first_bool
  }

\cs_new_protected:Npn \acro_index:
  { \bool_set_true:N \l__acro_addto_index_bool }

% similar macros \acro_<ending>: are defined by \acro_provide_ending:nnn

% ---------------------------------------------------------------------------
% process options:
\ProcessKeysPackageOptions {acro}

% ---------------------------------------------------------------------------
% PDF bookmark support
\cs_new:Npn \acpdfstring
  { \acro_pdf_string_short:n }

\cs_new:Npn \acpdfstringalt
  { \acro_pdf_string_alt:n }

\cs_new:Npn \acpdfstringlong
  { \acro_pdf_string_long:n }

\cs_new:Npn \acpdfstringfirst #1
  { \acpdfstringlong {#1} ~ ( \acpdfstring {#1} ) }

% TODO: place this somewhere where endings are defined:
\cs_new:Npn \acpdfstringlongplural
  { \acro_pdf_string_long_plural:n }

\prg_new_conditional:Npnn \acro_if_star_gobble:n #1 {TF}
  {
    \if_meaning:w *#1
      \prg_return_true:
    \else:
      \prg_return_false:
    \fi:
  }

\cs_new:Npn \acro_expandable_long:n #1
  { \prop_item:Nn \l__acro_long_prop {#1} }

\cs_new:Npn \acro_expandable_long_plural:n #1
  {
    \bool_if:nTF
      { \prop_item:Nn \l__acro_long_plural_form_prop {#1} }
      { \prop_item:Nn \l__acro_long_plural_prop {#1} }
      {
        \prop_item:Nn \l__acro_long_prop {#1}
        \prop_item:Nn \l__acro_long_plural_prop {#1}
      }
  }

\cs_new:Npn \acro_pdf_string_long:n #1
  {
    \acro_if_star_gobble:nTF {#1}
      { \acro_expandable_long:n }
      { \acro_expandable_long:n {#1} }
  }

% TODO: place this somewhere where endings are defined:
\cs_new:Npn \acro_pdf_string_long_plural:n #1
  {
    \acro_if_star_gobble:nTF {#1}
      { \acro_expandable_long_plural:n }
      { \acro_expandable_long_plural:n {#1} }
  }
  
\cs_new:Npn \acro_pdf_string_short:n #1
  {
    \acro_if_star_gobble:nTF {#1}
      { \prop_item:Nn \l__acro_pdfstring_short_prop }
      { \prop_item:Nn \l__acro_pdfstring_short_prop {#1} }
  }
  
\cs_new:Npn \acro_pdf_string_alt:n #1
  {
    \acro_if_star_gobble:nTF {#1}
      { \prop_item:Nn \l__acro_pdfstring_alt_prop }
      { \prop_item:Nn \l__acro_pdfstring_alt_prop {#1} }
  }

\AtBeginDocument
  {
    \@ifpackageloaded {hyperref}
      {
        \bool_set_true:N \l__acro_hyperref_loaded_bool
        \pdfstringdefDisableCommands
          {
            \cs_set_eq:NN \ac   \acpdfstring
            \cs_set_eq:NN \Ac   \acpdfstring
            \cs_set_eq:NN \acs  \acpdfstring
            \cs_set_eq:NN \acl  \acpdfstringlong
            \cs_set_eq:NN \Acl  \acpdfstringlong
            \cs_set_eq:NN \acf  \acpdfstringfirst
            \cs_set_eq:NN \Acf  \acpdfstringfirst
            \cs_set_eq:NN \aca  \acpdfstringalt
            \cs_set_eq:NN \acp  \acpdfstringplural
            \cs_set_eq:NN \Acp  \acpdfstringplural
            \cs_set_eq:NN \acsp \acpdfstringplural
            \cs_set_eq:NN \aclp \acpdfstringlongplural
            \cs_set_eq:NN \Aclp \acpdfstringlongplural
            \cs_set_eq:NN \acfp \acpdfstringplural
            \cs_set_eq:NN \Acfp \acpdfstringplural
            \cs_set_eq:NN \acap \acpdfstringaltplural
          }
        \cs_set_protected:Npn \acro_hyper_page:n #1 { \hyperpage {#1} }
      } {}
  }

% --------------------------------------------------------------------------
% additional variables:
\tl_new:N \l__acro_current_property_tl

% --------------------------------------------------------------------------
% key and order checking
\msg_new:nnn {acro} {no-id}
  {
    Something~ has~ gone~ wrong,~ you've~ probably~ forgotten~ to~ set~ the~
    acronym~ ID.
  }

\msg_new:nnn {acro} {before-short}
  {
    You've~ set~ the~ property~ `#2'~ before~ the~ `short'~ property~ for~
    acronym~ `#1'~ but~ it~ needs~ to~ be~ set~ after~ it.
  }

\msg_new:nnn {acro} {missing}
  { The~ `#2'~ property~ for~ acronym~ `#1'~ is~ missing. }

\msg_new:nnn {acro} {doubled-property}
  {
    It~ seems~ to~ me~ you~ have~ used~ the~ `#2'~ property~ twice~ in~ the~
    declaration~ of~ acronym~ `#1'.~ If~ you~ haven't~ there's~
    something~ different~ wrong~ and~ I'm~ lost.~ You~'re~ on~ your~ own~
    then.
  }

\cs_new_protected:Npn \__acro_property_check:nn #1#2
  {
    \tl_if_blank:VT \l__acro_current_property_tl
      { \acro_serious_message:n {no-id} }
    \bool_if:cF { l__acro_#1_short_set_bool }
      {
        \keys_set:nn { acro / declare-acronym } { short = {#1} }
        \acro_harmless_message:nn {substitute-short} {#1}
      }
    \bool_new:c { l__acro_#1_#2_set_bool }
    \bool_set_true:c { l__acro_#1_#2_set_bool }
  }

\cs_new_protected:Npn \__acro_first_property_check:nn #1#2
  {
    \cs_if_exist:cTF { l__acro_#1_short_set_bool }
      {
         \bool_if:cT { l__acro_#1_short_set_bool }
           { \acro_serious_message:nnn {doubled-property} {#1} {#2} }
      }
      {
        \bool_new:c { l__acro_#1_short_set_bool }
        \bool_set_true:c { l__acro_#1_short_set_bool }
      }
  }

% --------------------------------------------------------------------------
% the internal property selection functions for \DeclareAcronym:

% #1: name in associated cs
% #2: property name
% #3: action
\cs_new_protected:Npn \acro_declare_property_generic:nnn #1#2#3
  {
    \prop_clear_new:c {l__acro_#1_prop}
    \cs_new_protected:cpn   {__acro_declare_#1:nn} ##1##2 {#3}
    \cs_generate_variant:cn {__acro_declare_#1:nn} {V}
    \keys_define:nn {acro/declare-acronym}
      {
        #2 .code:n =
          \use:c {__acro_declare_#1:Vn} \l__acro_current_property_tl {##1}
      }
  }

% #1: name in associated cs
% #2: property name
% #3: action
\cs_new_protected:Npn \acro_declare_property:nnn #1#2#3
  {
    \acro_declare_property_generic:nnn {#1} {#2}
      {
        \__acro_property_check:nn {##1} {#2}
        \prop_put:cnn {l__acro_#1_prop} {##1} {##2}
        #3
      }
  }

% #1: name in associated cs
% #2: property name
\cs_new_protected:Npn \acro_declare_property:nn #1#2
  { \acro_declare_property:nnn {#1} {#2} {} }
\cs_generate_variant:Nn \acro_declare_property:nn { V }

\cs_new_protected:Npn \acro_declare_simple_property:n #1
  {
    \tl_set:Nn \l__acro_tmpa_tl {#1}
    \tl_replace_all:Nnn \l__acro_tmpa_tl {-} {_}
    \tl_clear_new:c  {l__acro_ \l__acro_tmpa_tl _tl}
    \acro_declare_property:Vn \l__acro_tmpa_tl {#1}
  }

% #1: new alias property
% #2: old property
\cs_new_protected:Npn \acro_declare_property_alias:nn #1#2
  {
    \keys_define:nn {acro/declare-acronym}
      { #1 .meta:n = { #2 = {##1} } }
  }

% --------------------------------------------------------------------------
% declare the properties for \DeclareAcronym:
% short:
\acro_declare_property_generic:nnn {short} {short}
  {
    \__acro_first_property_check:nn {#1} {short}
    \prop_put:Nnn \l__acro_short_prop      {#1} {#2}
    \prop_put:Nnn \l__acro_sort_prop       {#1} {#1}
    \prop_put:Nnn \l__acro_index_sort_prop {#1} {#1}
    \prop_put:Nnn \l__acro_alt_prop        {#1} {#2}
    \prop_put:Nnn \l__acro_pdfstring_short_prop {#1} {#2}
    \prop_put:Nnn \l__acro_pdfstring_alt_prop {#1} {#2}
    \acro_for_endings_do:n
      {
        \prop_put:cnv {l__acro_short_##1_prop}
          {#1} {l__acro_default_short_##1_tl}
        \prop_put:cnx {l__acro_pdfstring_short_##1_prop}
          {#1} { \exp_not:n {#2} \exp_not:v {l__acro_default_short_##1_tl} }
        \prop_put:cnn {l__acro_short_##1_form_prop} {#1} { \c_false_bool }
        \prop_put:cnv {l__acro_alt_##1_prop}
          {#1} {l__acro_default_alt_##1_tl}
        \prop_put:cnx {l__acro_pdfstring_alt_##1_prop}
          {#1} { \exp_not:n {#2} \exp_not:v {l__acro_default_short_##1_tl} }
        \prop_put:cnn {l__acro_alt_##1_form_prop} {#1} { \c_false_bool }
      }
    \prop_put:NnV \l__acro_short_indefinite_prop
      {#1} \l__acro_default_indefinite_tl
    \prop_put:NnV \l__acro_alt_indefinite_prop
      {#1} \l__acro_default_indefinite_tl
  }

% long:
\acro_declare_property:nnn {long} {long}
  {
    \acro_for_endings_do:n
      { \prop_put:cnn {l__acro_long_##1_form_prop} {#1} { \c_false_bool } }
    \prop_put:NnV \l__acro_long_indefinite_prop
      {#1}
      \l__acro_default_indefinite_tl
    \acro_for_endings_do:n
      {
        \bool_if:cF {l__acro_#1_long-##1_set_bool}
          { \prop_put:cnv {l__acro_long_##1_prop} {#1} {l__acro_default_long_##1_tl} }
      }
  }

\acro_declare_simple_property:n {first-style}

% list:
\acro_declare_simple_property:n {list}

% defines `short-plural', `long-plural' and `long-plural-form' as well as the
% options `plural-ending', `short-plural-ending' and `long-plural-ending':
% \ProvideAcroEnding {plural} {s} {s}

% short indefinite article:
\acro_declare_simple_property:n {short-indefinite}

% long indefinite article:
\acro_declare_simple_property:n {long-indefinite}

% pre long:
\acro_declare_simple_property:n {long-pre}

% post long:
\acro_declare_simple_property:n {long-post}

% sort:
\acro_declare_property:nnn {sort} {sort}
  {
    \bool_if:cF { l__acro_#1_index-sort_set_bool }
      { \prop_put:Nnn \l__acro_index_sort_prop {#1} {#2} }
  }

% alternative:
\acro_declare_property:nnn {alt} {alt}
  {
    \prop_put:Nnn \l__acro_pdfstring_alt_prop {#1} {#2}
    \prop_put:NnV \l__acro_alt_indefinite_prop
      {#1} \l__acro_default_indefinite_tl
  }

\cs_set_protected:Npn \acro_alt_error:n #1
  {
    \bool_if:cF {l__acro_#1_alt_set_bool} 
      { \acro_harmless_message:nn {no-alternative} {#1} }
  }

% alt. indefinite article:
\acro_declare_simple_property:n {alt-indefinite}

% foreign:
\acro_declare_simple_property:n {foreign}

% foreign:
\acro_declare_simple_property:n {foreign-lang}

% format:
\acro_declare_simple_property:n {format}

% short format:
\acro_declare_property_alias:nn {short-format} {format}

% long format:
\acro_declare_simple_property:n {long-format}

% first long format:
\acro_declare_simple_property:n {first-long-format}

% pdfstring -- currently needs to be done `by hand':
\prop_new:N \l__acro_pdfstring_short_prop
\cs_new_protected:Npn \__acro_declare_pdfstring:nw #1#2/#3/#4 \acro_stop:
  {
    \__acro_property_check:nn {#1} {pdfstring}
    \prop_put:Nnx \l__acro_pdfstring_short_prop {#1} {#2}
    \acro_for_endings_do:n
      {
        \tl_if_blank:nTF {#4}
          {
            \prop_put:cnx {l__acro_pdfstring_short_##1_prop}
              {#1} { \exp_not:n {#2} \exp_not:v {l__acro_default_short_##1_tl} }
          }
          {
            \prop_put:cnn {l__acro_pdfstring_short_##1_prop}
              {#1} {#2#3}
          }
      }
  }
\cs_generate_variant:Nn \__acro_declare_pdfstring:nw { V }
\keys_define:nn { acro / declare-acronym }
  {
    pdfstring    .code:n =
      \__acro_declare_pdfstring:Vw \l__acro_current_property_tl #1 // \acro_stop: ,
  }

\prop_new:N \l__acro_pdfstring_alt_prop
\cs_new_protected:Npn \__acro_declare_pdfstring_alt:nw #1#2/#3/#4 \acro_stop:
  {
    \__acro_property_check:nn {#1} { pdfstring-alt }
    \prop_put:Nnn \l__acro_pdfstring_alt_prop {#1} {#2}
    \acro_for_endings_do:n
      {
        \tl_if_empty:nTF {#3}
          {
            \prop_put:cnx {l__acro_pdfstring_alt_##1_prop}
              {#1} { \exp_not:n {#2} \exp_not:v {l__acro_default_alt_##1_tl} }
          }
          { \prop_put:cnn {l__acro_pdfstring_alt_##1_prop} {#1} {#2#3} }
      }
  }
\cs_generate_variant:Nn \__acro_declare_pdfstring_alt:nw { V }
\keys_define:nn { acro / declare-acronym }
  {
    pdfstring-alt .code:n =
      \__acro_declare_pdfstring_alt:Vw \l__acro_current_property_tl #1 // \acro_stop: ,
  }
  
% class:
\acro_declare_simple_property:n {class}

% extra information:
\acro_declare_simple_property:n {extra}

% single appearances:
\acro_declare_simple_property:n {single}

% single format:
\acro_declare_simple_property:n {single-format}

% acc supp:
\acro_declare_property:nn {acc_supp} {accsupp}

% tooltip:
\acro_declare_simple_property:n {tooltip}

% citation -- currently needs to be done `by hand':
\prop_new:N \l__acro_citation_prop
\prop_new:N \l__acro_citation_pre_prop
\prop_new:N \l__acro_citation_post_prop
\cs_new_protected:Npn \__acro_declare_citation:nw #1#2[#3]#4[#5]#6#7 \acro_stop:
  {
    % no options: #1: ID, #2: property, #3 is blank
    % 1 option:   #1: ID, #4: property, #3: option, #5 is blank
    % 2 options:  #1: ID: #6: property, #3: first option, #5: second option
    \tl_if_blank:nF { #2#4#6 }
      {
        \tl_if_empty:nTF {#3}
          { \__acro_declare_citation_aux:nnnn {#1} { } { } {#2} }
          {
            \tl_if_empty:nTF {#5}
              { \__acro_declare_citation_aux:nnnn {#1} {#3} {  } {#4} }
              { \__acro_declare_citation_aux:nnnn {#1} {#3} {#5} {#6} }
          }
      }
  }
\cs_generate_variant:Nn \__acro_declare_citation:nw { V }

\keys_define:nn { acro / declare-acronym }
  {
    cite .code:n =
      \__acro_declare_citation:Vw
        \l__acro_current_property_tl #1 [][] \scan_stop: \acro_stop:
  }

\cs_new_protected:Npn \__acro_declare_citation_aux:nnnn #1#2#3#4
  {
    \__acro_property_check:nn {#1} {cite}
    \prop_put:Nnn \l__acro_citation_prop {#1} {#4}
    \tl_if_empty:nF {#2}
      { \prop_put:Nnn \l__acro_citation_pre_prop {#1} {#2} }
    \tl_if_empty:nF {#3}
      { \prop_put:Nnn \l__acro_citation_post_prop {#1} {#3} }
  }

% TODO:
% add index entries, by default \index{<sort>@<short>}
% index: overwrite default <sort>@<short> entry completely
% index-sort: overwrite the <sort> part of <sort>@<short> entry

% need to take care of custom index cmd, at least
%  - \index{}
%  - \index[]{}
% question is, though, if it should be the same one for all acronyms?
% I go for yes but would also add a `post' property that allows to add arbitrary
% TeX code after an acronym is typeset

% index:
\acro_declare_simple_property:n {index}

% index-sort:
\acro_declare_simple_property:n {index-sort}

% index-cmd:
\acro_declare_simple_property:n {index-cmd}

% --------------------------------------------------------------------------
% acronym macros:
\cs_new_protected:Npn \__acro_define_acronym_macro:n #1
  {
    \bool_if:NT \l__acro_create_macros_bool
      {
        \cs_if_exist:cTF {#1}
          {
            \bool_if:NTF \l__acro_strict_bool
              { \cs_set:cpn {#1} { \ac {#1} \acro_xspace: } }
              { \acro_serious_message:nn {macro} {#1} }
          }
          { \cs_new:cpn {#1} { \ac {#1} \acro_xspace: } }
      }
  }

% --------------------------------------------------------------------------
% internal acronym declaring function:
\cs_new_protected:Npn \acro_declare_acronym:nn #1#2
  {
    \seq_gput_right:Nn \g__acro_declared_acronyms_seq {#1}
    \bool_gset_true:N \g__acro_first_acronym_declared_bool
    \tl_set:Nn \l__acro_current_property_tl {#1}
    \keys_set:nn { acro / declare-acronym } {#2}
    \bool_new:c {g__acro_#1_first_use_bool}
    \bool_new:c {g__acro_#1_used_bool}
    \bool_new:c {g__acro_#1_label_bool}
    \bool_new:c {g__acro_#1_in_list_bool}
    \seq_new:c  {g__acro_#1_barriers_seq}
    \seq_new:c  {g__acro_#1_recorded_barriers_seq}
    \bool_if:NF \l__acro_print_only_used_bool
      { \bool_gset_true:c {g__acro_#1_in_list_bool} }
    \__acro_create_page_records:n {#1}
    \__acro_define_acronym_macro:n {#1}
    \tl_clear:N \l__acro_current_property_tl
    \bool_if:cF {l__acro_#1_short_set_bool}
      { \acro_serious_message:nnn {missing} {#1} {short} }
    \bool_if:cF {l__acro_#1_long_set_bool}
      { \acro_serious_message:nnn {missing} {#1} {long} }
    \__acro_log_acronym:n {#1}
  }

% --------------------------------------------------------------------------
% the user command:
\NewDocumentCommand \DeclareAcronym {mm}
  { \acro_declare_acronym:nn {#1} {#2} }

% --------------------------------------------------------------------------
% print the list:
% #1: list of classes
% #2: list of excluded classes
\tl_new:N \l__acro_included_classes_tl
\tl_new:N \l__acro_excluded_classes_tl

\keys_define:nn { acro / print-acronyms }
  {
    include-classes   .tl_set:N   = \l__acro_included_classes_tl ,
    exclude-classes   .tl_set:N   = \l__acro_excluded_classes_tl ,
    name              .tl_set:N   = \l__acro_list_name_tl ,
    heading           .code:n     = \__acro_set_list_heading:n {#1} ,
    sort              .bool_set:N = \l__acro_sort_bool ,
    local-to-barriers .bool_set:N = \l__acro_use_barrier_bool
  }

\cs_new_protected:Npn \acro_print_acronyms:n #1
  {
    \group_begin:
      % this is a cheap trick to prevent the \@noitemerr
      % if one forgot to delete either the aux file or
      % remove \printacronyms -- but it's local:
      \cs_set:Npn \@noitemerr {}
      \tl_clear:N \l__acro_included_classes_tl
      \tl_clear:N \l__acro_excluded_classes_tl
      \keys_set:nn { acro / print-acronyms } {#1}
      \__acro_aux_file_now:n { \acro@print@list }
      \bool_if:NT \l__acro_sort_bool
        { \acro_sort_prop:NN \l__acro_short_prop \l__acro_sort_prop }
      \acro_title_instance:VV
        \l__acro_list_heading_cmd_tl
        \l__acro_list_name_tl
      \cs_if_exist:NTF \acro@printed@list
        {
          \acro_list_instance:VVV
            \l__acro_list_instance_tl
            \l__acro_included_classes_tl
            \l__acro_excluded_classes_tl
        }
        { \@latex@warning@no@line {Rerun~to~get~acronym~list~right} }
    \group_end:
  }

\NewDocumentCommand \printacronyms { O{} }
  { \acro_print_acronyms:n {#1} }

% --------------------------------------------------------------------------
% language support
\RequirePackage {translations}

\cs_new_protected:Npn \__acro_declare_translation:www #1 \q_mark #2=#3 \q_stop
  {
    \tl_set:Nx \l__acro_tmpa_tl { \tl_trim_spaces:n {#1} }
    \tl_set:Nx \l__acro_tmpb_tl { \tl_trim_spaces:n {#2} }
    \tl_if_in:nnT {#3} {=}
      {} % TODO: misplaced equal sign
    \tl_set:Nx \l__acro_tmpc_tl { \tl_trim_spaces:n {#3} }
    \__acro_declare_translation:VVV
      \l__acro_tmpb_tl
      \l__acro_tmpa_tl
      \l__acro_tmpc_tl
  }

% #1: key
% #2: lang
% #3: translation
\cs_new_protected:Npn \__acro_declare_translation:nnn #1#2#3
  { \DeclareTranslation {#1} {#2} {#3} }
\cs_generate_variant:Nn \__acro_declare_translation:nnn {VVV}

% #1: key
% #2: csv list: { <lang1> = <translation1> , <lang2> = <translation2> }
\cs_new_protected:Npn \acro_declare_translation:nn #1#2
  {
    \clist_map_inline:nn {#2}
      {
        \tl_if_blank:nF {##1}
          { \__acro_declare_translation:www #1 \q_mark ##1 \q_stop }
      }
  }

\NewDocumentCommand \DeclareAcroTranslation {mm}
  { \acro_declare_translation:nn {#1} {#2} }
\@onlypreamble \DeclareAcroTranslation

% tokenlists using the translations:
\tl_set:Nn \l__acro_list_name_tl  { \GetTranslation {acronym-list-name} }
\tl_set:Nn \l__acro_page_name_tl  { \GetTranslation {acronym-page-name}\@\, }
\tl_set:Nn \l__acro_pages_name_tl { \GetTranslation {acronym-pages-name}\@\, }
\tl_set:Nn \l__acro_next_page_tl  { \,\GetTranslation {acronym-next-page}\@ }
\tl_set:Nn \l__acro_next_pages_tl { \,\GetTranslation {acronym-next-pages}\@ }

% --------------------------------------------------------------------------
% definition file:
% \tl_const:Nn \c_acro_definition_file_name_tl      {acro}
% \tl_const:Nn \c_acro_definition_file_extension_tl {def}

% \file_if_exist:nTF
%   { \c_acro_definition_file_name_tl .\c_acro_definition_file_extension_tl }
%   {
%     \@onefilewithoptions
%       {\c_acro_definition_file_name_tl} [] []
%       \c_acro_definition_file_extension_tl
%   }
%   { \acro_serious_message:n {definitions-missing} }


% --------------------------------------------------------------------------
% first appearance styles:
\DeclareAcroFirstStyle {default} {inline}
  { }

\DeclareAcroFirstStyle {reversed} {inline}
  { reversed = true }

\DeclareAcroFirstStyle {short} {inline}
  {
    only-short = true ,
    brackets = false
  }

\DeclareAcroFirstStyle {long} {inline}
  {
    only-long = true ,
    brackets = false
  }

\DeclareAcroFirstStyle {square} {inline}
  { brackets-type = [] }

\DeclareAcroFirstStyle {plain} {inline}
  {
    brackets = false ,
    between = --
  }

\DeclareAcroFirstStyle {plain-reversed} {inline}
  { 
    brackets = false ,
    between = -- ,
    reversed = true
  }

\DeclareAcroFirstStyle {footnote} {note}
  { }

\DeclareAcroFirstStyle {footnote-reversed} {note}
  { reversed = true }

\DeclareAcroFirstStyle {sidenote} {note}
  { note-command = \sidenote {#1} }

\DeclareAcroFirstStyle {sidenote-reversed} {note}
  {
    note-command = \sidenote {#1} ,
    reversed     = true
  }

\DeclareAcroFirstStyle {empty} {note}
  { use-note = false }

% --------------------------------------------------------------------------
% extra info appearance styles:
\DeclareAcroExtraStyle {default} {inline}
  {
    brackets     = false ,
    punct        = true ,
    punct-symbol = .
  }

\DeclareAcroExtraStyle {plain} {inline}
  {
    brackets     = false ,
    punct        = true ,
    punct-symbol =
  }

\DeclareAcroExtraStyle {paren} {inline}
  {
    brackets     = true ,
    punct        = true ,
    punct-symbol =
  }

\DeclareAcroExtraStyle {bracket} {inline}
  {
    brackets      = true ,
    punct         = true ,
    punct-symbol  = ,
    brackets-type = []
  }

\DeclareAcroExtraStyle {comma} {inline}
  {
    punct         = true,
    punct-symbol  = {,} ,
    brackets      = false
  }

% --------------------------------------------------------------------------
% page number appearance styles:
\DeclareAcroPageStyle {default} {inline}
  {
    punct = true ,
    punct-symbol = .
  }
  
\DeclareAcroPageStyle {plain}   {inline}
  { punct = false }

\DeclareAcroPageStyle {comma}   {inline}
  { punct = true }

\DeclareAcroPageStyle {paren}   {inline}
  {
    brackets=true ,
    punct-symbol = ~
  }

\DeclareAcroPageStyle {none}    {inline}
  { display = false }

% --------------------------------------------------------------------------
% list heading styles:
\DeclareAcroListHeading {part}           {\part}
\DeclareAcroListHeading {part*}          {\part*}
\DeclareAcroListHeading {chapter}        {\chapter}
\DeclareAcroListHeading {chapter*}       {\chapter*}
\DeclareAcroListHeading {addchap}        {\addchap}
\DeclareAcroListHeading {section}        {\section}
\DeclareAcroListHeading {section*}       {\section*}
\DeclareAcroListHeading {addsec}         {\addsec}
\DeclareAcroListHeading {subsection}     {\subsection}
\DeclareAcroListHeading {subsection*}    {\subsection*}
\DeclareAcroListHeading {subsubsection}  {\subsubsection}
\DeclareAcroListHeading {subsubsection*} {\subsubsection*}
\DeclareAcroListHeading {none}           {\use_none:n}

% --------------------------------------------------------------------------
% list styles:
\DeclareAcroListStyle {description} {list}
  { }

\DeclareAcroListStyle {toc} {list-of}
  { }

\DeclareAcroListStyle {lof} {list-of}
  { style = lof }

\DeclareAcroListStyle {tabular} {table}
  { table = tabular }

\DeclareAcroListStyle {longtable} {table}
  { table = longtable }

\DeclareAcroListStyle {extra-tabular} {extra-table}
  { table = tabular }

\DeclareAcroListStyle {extra-longtable} {extra-table}
  { table = longtable }

\DeclareAcroListStyle {extra-tabular-rev} {extra-table}
  {
    table   = tabular ,
    reverse = true
  }

\DeclareAcroListStyle {extra-longtable-rev} {extra-table}
  {
    table   = longtable ,
    reverse = true
  }

% --------------------------------------------------------------------------
% register some tokens to be checked for:
\AcroRegisterTrailing . {dot}
\AcroRegisterTrailing - {dash}
\AcroRegisterTrailing \babelhyphen {babel-hyphen}

% --------------------------------------------------------------------------
% the user commands
% automatic:
\NewAcroCommand \ac
  { \acro_use:n {#1} }

\NewAcroCommand \iac
  {
    \acro_indefinite:
    \acro_use:n {#1}
  }

\NewAcroCommand \Iac
  {
    \acro_first_upper:
    \acro_indefinite:
    \acro_use:n {#1}
  }

\NewAcroCommand \Ac
  {
    \acro_first_upper:
    \acro_use:n {#1}
  }

\NewAcroCommand \acp
  {
    \acro_plural:
    \acro_use:n {#1}
  }

\NewAcroCommand \Acp
  {
    \acro_plural:
    \acro_first_upper:
    \acro_use:n {#1}
  }

\NewAcroCommand \acsingle
  {
    \acro_get:n {#1}
    \acro_single:n {#1}
  }

\NewAcroCommand \Acsingle
  {
    \acro_get:n {#1}
    \acro_first_upper:
    \acro_single:n {#1}
  }

% short:
\NewAcroCommand \acs
  { \acro_short:n {#1} }

\NewAcroCommand \iacs
  {
    \acro_indefinite:
    \acro_short:n {#1}
  }

\NewAcroCommand \Iacs
  {
    \acro_first_upper:
    \acro_indefinite:
    \acro_short:n {#1}
  }

\NewAcroCommand \acsp
  {
    \acro_plural:
    \acro_short:n {#1}
  }

% alt:
\NewAcroCommand \aca
  { \acro_alt:n {#1} }

\NewAcroCommand \Aca
  {
    \acro_first_upper:
    \acro_alt:n {#1}
  }
  
\NewAcroCommand \iaca
  {
    \acro_indefinite:
    \acro_alt:n {#1}
  }

\NewAcroCommand \Iaca
  {
    \acro_first_upper:
    \acro_indefinite:
    \acro_alt:n {#1}
  }

\NewAcroCommand \acap
  {
    \acro_plural:
    \acro_alt:n {#1}
  }

% long:
\NewAcroCommand \acl
  { \acro_long:n {#1} }

\NewAcroCommand \iacl
  {
    \acro_indefinite:
    \acro_long:n {#1}
  }

\NewAcroCommand \Iacl
  {
    \acro_first_upper:
    \acro_indefinite:
    \acro_long:n {#1}
  }

\NewAcroCommand \Acl
  {
    \acro_first_upper:
    \acro_long:n {#1}
  }

\NewAcroCommand \aclp
  {
    \acro_plural:
    \acro_long:n {#1}
  }

\NewAcroCommand \Aclp
  {
    \acro_plural:
    \acro_first_upper:
    \acro_long:n {#1}
  }

% first:
\NewAcroCommand \acf
  { \acro_first:n {#1} }

\NewAcroCommand \iacf
  {
    \acro_indefinite:
    \acro_first:n {#1}
  }

\NewAcroCommand \Iacf
  {
    \acro_first_upper:
    \acro_indefinite:
    \acro_first:n {#1}
  }

\NewAcroCommand \Acf
  {
    \acro_first_upper:
    \acro_first:n {#1}
  }

\NewAcroCommand \acfp
  {
    \acro_plural:
    \acro_first:n {#1}
  }

\NewAcroCommand \Acfp
  {
    \acro_plural:
    \acro_first_upper:
    \acro_first:n {#1}
  }

% first-like:
\NewPseudoAcroCommand \acflike
  { \acro_first_like:nn {#1} {#2} }

\NewPseudoAcroCommand \iacflike
  {
    \acro_indefinite:
    \acro_first_like:nn {#1} {#2}
  }

\NewPseudoAcroCommand \Iacflike
  {
    \acro_first_upper:
    \acro_indefinite:
    \acro_first_like:nn {#1} {#2}
  }

\NewPseudoAcroCommand \acfplike
  {
    \acro_plural:
    \acro_first_like:nn {#1} {#2}
  }

% --------------------------------------------------------------------------
% endings:
\ProvideAcroEnding {plural} {s} {s}

% --------------------------------------------------------------------------
% translations:
% list name
\DeclareAcroTranslation {acronym-list-name}
  {
    Fallback   = Acronyms ,
    English    = Acronyms ,
    French     = Acronymes ,
    German     = Abk\"urzungen ,
    Italian    = Acronimi ,
    Portuguese = Acr\'onimos ,
    Spanish    = Siglas ,
    Catalan    = Sigles ,
    Turkish    = K\i saltmalar
  }

% page name
\DeclareAcroTranslation {acronym-page-name}
  {
    Fallback   = p. ,
    English    = p. ,
    German     = S. ,
    Portuguese = p.
  }

% pages name
\DeclareAcroTranslation {acronym-pages-name}
  {
    Fallback   = pp. ,
    English    = pp. ,
    German     = S. ,
    Portuguese = pp.
  }

% following page
\DeclareAcroTranslation {acronym-next-page}
  {
    Fallback   = f. ,
    English    = f. ,
    German     = f. ,
    Portuguese = s.
  }

% following pages
\DeclareAcroTranslation {acronym-next-pages}
  {
    Fallback   = ff. ,
    English    = ff. ,
    German     = ff. ,
    Portuguese = ss.
  }

% --------------------------------------------------------------------------
% allow for a configuration file:
\tl_const:Nn \c_acro_config_file_name_tl      {acro}
\tl_const:Nn \c_acro_config_file_extension_tl {cfg}

\file_if_exist:nT
  { \c_acro_config_file_name_tl .\c_acro_config_file_extension_tl }
  {
    \@onefilewithoptions
      {\c_acro_config_file_name_tl} [] []
      \c_acro_config_file_extension_tl
  }

\tex_endinput:D
% --------------------------------------------------------------------------
% HISTORY:
2012/06/22 v0.1  - first public release
2012/06/23 v0.1a - bug fix, added `strict' and `macros' option and creation of
                   shortcut macros
                 - added capitalized version of long forms
                 - added `sort' option
2012/06/24 v0.1b - added \Acf and \Acfp, added option `plural-ending'
2012/06/24 v0.1c - added excluded argument to \printacronyms
2012/06/24 v0.2  - renamed \NewAcronym => \DeclareAcronym
                   \AcronymFormat => \DeclareAcronymFormat
2012/06/25 v0.2a - new first-style's: `short' and `reversed'
2012/06/25 v0.3  - new list formats: extra-tabular, extra-longtable,
                   extra-tabular-rev, extra-longtable-rev
                 - extra precaution when using \printacronyms to avoid errors.
2012/06/27 v0.3a - new option `list-caps', \Acp added
2012/06/29 v0.3b - extended the `text' template to the `acro-first' object
                 - added `acro-first' instances `plain' and `plain-reversed'
2012/07/16 v0.3c - small adjustments to the documentation
2012/07/23 v0.3d - first CTAN version
2012/07/24 v0.3e - adapted to updated l3kernel
2012/09/28 v0.4  - added means to add citations to acronyms
2012/10/07 v0.4a - new options: "uc-cmd", "list-long-format"
                 - preliminary language support, needs package `translations'
2012/11/30 v0.5  - added starred variants of the commands that won't mark an
                   acronym as used
                 - added \acreset{<id>}
                 - added preliminary support for pdf strings: in pdf strings
                   always the singular lowercase short version is inserted
                   (the equivalent of \acs)
2012/12/14 v0.6  - bug with not-colored links resolved
                 - bug introduced with the last update (full expansion of the
                   short entry) resolved
                 - option `xspace' added
2013/01/02 v0.6a - \acuseall
2013/01/16 v1.0  - new syntax of \DeclareAcronym
                 - new option `version'
                 - new `accsupp' acronym property
                 - new `sort' acronym property
                 - new syntax of \printacronyms
                 - new default: `sort=true' 
                 - new options `page-ranges', `next-page', `next-pages',
                   `pages-name', `record-pages'
                 - no automatic label placement for page number referencing
                   any more
2013/01/26 v1.1  - bug fix in the plural detection
                 - new keys `long-pre' and `long-post'
                 - new keys `index', `index-sort' and `index-cmd'
                 - new options `index' and `index-cmd'
2013/01/29 v1.1a - added `long-format' key
                 - renamed `format' key into `short-format', kept `format' for
                   compatibility reasons
2013/02/09 v1.2  - error message instead of hanging when an undefined acronym
                   is used
                 - added `first-long-format' key and `first-long-format' option
                 - added \acflike and \acfplike
                 - improvements and bug fixes to the page recording mechanism
                 - new option `mark-as-used'
                 - new keys: `short-indefinite', `alt-indefinite' and
                   `long-indefinite'
                 - new commands: \iac, \Iac, \iacs, \Iacs, \iaca, \Iaca, \iacl,
                   \Iacl, \iacf, \Iacf, \iacflike and \Iacflike
2013/04/04 v1.2a - added Portuguese translations
2013/05/06 v1.3  - protected internal commands where appropriate
                 - new option `sort' to \printacronyms
                 - renamed options `print-acronyms/header' and `list-header'
                   into `print-acronyms/heading' and `list-heading'
                 - fix: added missing group to \printacronyms
                 - add key `foreign'
                 - rewritten page-recording:
                   * most importantly: record them at shipout; this is done
                     when \acro@used@once or \acro@used@twice are written to
                     the aux file
                   * no restrictions regarding \pagenumbering
                   * options `page-ranges' and `record-pages' are deprecated
                   * new options `following-page' and `following-pages'
                 - disable \@noitemerr in the list of acronyms: we don't need
                   it there but there are occasions when it is annoying
                 - cleaned the sty file, added a few more comments
2013/05/09 v1.3a - Bug fix: corrected wrong argument checking in \Ac, thanks
                   to Michel Voßkuhle
2013/05/30 v1.3b - obey \if@filesw
2013/06/16 v1.3c - added \leavevmode to \acro_get:n
2013/07/08 v1.3d - corrected wrong call of \leavevmode in the list
                   (list-type=list)
2013/08/07 v1.3e - bug fix in the list when testing for used acronyms
                 - new commands \acifused, \acfirstupper
2013/08/27 v1.4  - new property `list'
2013/09/02 v1.4a - bug fix: used acronyms are added to the list when the list
                   is printed before the use
                 - \DeclareAcronym may now be used after \begin{document}
2013/09/24 v1.4b - bug fix: only-used=false works again for only declared but
                   unused acronyms (only if option single is not used)
2013/11/04 v1.4c - remove \hbox from the written short form
                 - changed \__acro_make_link:nNN in a way that it doesn't box
                   its when links are deactivated
2013/11/22 v1.4d - require `l3sort' independently from the `sort' option
                   instead of at begin document in order to avoid conflicts
                   with `babel' and `french'
2013/12/18 v1.5  - new option `label=true|false' that
                   places \label{<prefix>:<id>} the first time an acronym is
                   used
                 - new option `pages=first|all' that determines if in the list
                   of acronyms all appearances are listed or only the first
                   time; implicitly sets `label=true'
2015/02/26 v1.6  - new `acro-title' instance `none'
                 - change of expl3's tl uppercasing function (adapt to updates
                   of l3kernel and friends
                 - new package option `messages=silent|loud'
                 - fix issue https://bitbucket.org/cgnieder/acro/issue/23/
                 - fix issue https://bitbucket.org/cgnieder/acro/issue/24/
                 - fix issue https://bitbucket.org/cgnieder/acro/issue/28/
                 - drop support for version 0
2015/04/08 v1.6a - more generalized user command definitions, see
                   http://tex.stackexchange.com/q/236362/ for an application
2015/05/10 v1.6b - \ProcessKeysPackageOptions ,
                 - correct bug http://tex.stackexchange.com/q/236860/ :
                   option `pages = first' works again
2015/08/16 v2.0  - fix https://bitbucket.org/cgnieder/acro/issue/36
                 - implement https://bitbucket.org/cgnieder/acro/issue/39
                 - implement https://bitbucket.org/cgnieder/acro/issue/40
                   (=> new option `group-cite-cmd')
                 - add ideas for https://bitbucket.org/cgnieder/acro/issue/41
                 - implement https://bitbucket.org/cgnieder/acro/issue/18
                 - implement https://bitbucket.org/cgnieder/acro/issue/43
                 - further generalization for defining user commands:
                   \NewAcroCommand, \NewPseudoAcroCommand and siblings
                 - bug fix in indefinite versions with first-upper
                 - add `short-<ending>-form' equivalent to
                   `long-<ending>-form'
                   (https://bitbucket.org/cgnieder/acro/issue/44)
                 - implement https://bitbucket.org/cgnieder/acro/issue/35
                 - new option `single-form'
2015/08/25 v2.0a - fix https://bitbucket.org/cgnieder/acro/issue/38 and
                   https://bitbucket.org/cgnieder/acro/issue/49
2015/08/29 v2.0b - fix https://bitbucket.org/cgnieder/acro/issue/44
                 - fix https://bitbucket.org/cgnieder/acro/issue/45
                 - implement https://bitbucket.org/cgnieder/acro/issue/42
2015/09/05 v2.1  - add list object type `list-of' that prints the list like a
                   toc or lof, new option `list-short-width',
                 - correct bug in the `plain' extra style
                 - implemented `tooltip' property
                 - remove \tl_to_lowercase:n
2015/10/03 v2.2  - fix https://bitbucket.org/cgnieder/acro/issue/52
                 - fix typo in `list-of' object
                 - \DeclareAcroListStyle
                 - \DeclareAcroListHeading
                 - input `acro.cfg' if present
                 - all acro commands have an optional argument: \ac*[]{}
2016/01/07 v2.2a - \prop_get:Nn => \prop_item:Nn
2016/01/21 v2.2b - fix issue #59
2016/02/02 v2.2c - fix issue #60
2016/03/14 v2.3  - foreign-format may be a macro taking an argument
                 - \Aca, \Acsingle
                 - properties `single' and `single-format', option
                   `single-format' => issue #62
                 - property `first-style' => issue #61
                 - fix issue #64: long-post entry is now added *after* the
                   endings
                 - property `foreign-lang'
                 - fix issue #65
2016/03/25 v2.4  - extend class mechanism: classes can be used like tags
                 - add idea of `barriers' and list local to those barriers
                   => new option `reset-at-barriers'
                   => new option `local-to-barriers' for \printacronyms
                   => new command \acbarrier
2016/04/14 v2.4a - if undefined acronym is used and `messages = silent' is
                   active don't through error
2016/05/03 v2.4b - expand `pdfstring' property before saving => issue #69
                 - \ProvideAcroEnding can be used twice – it then just sets
                   the defaults
                 - the option <ending>-ending has a new syntax:
                   * <ending>-ending = <val> sets all endings to <val>
                   * <ending>-ending = <val1>/<val2> sets short endings to
                     <val1> and long endings to <val2>
                 - a single appearance should be treated like a first
                   appearance as far as citations are concerned
2016/05/25 v2.5  - some of the entries added to the aux file need to be
                   written \immediate in order to avoid this trap:
                   http://tex.stackexchange.com/q/116001/
                 - cleaner interface for first-style template definitions
                 - new `acro-first' instances `footnote-reversed' and
                   `sidenote-reversed'
                 - new commands \DeclareAcroFirstStyle, \DeclareAcroExtraStyle
                   and \DeclareAcroPageStyle
                 - add warning `ending-before-acronyms' to options setting the
                   defaults of an ending; this should avoid confusion
                 - property declaration for acronyms should be handled by
                   internal functions
                 - improvements in the list template code
                 - logging info when an acronym is declared
                 - remove deprecated options
                 - new option `use-barriers'
                 - new option `following-pages*'
                 - option `page-ref' replaced by option `page-style'
2016/05/26 v2.5a - bug fix: remove erroneous group in `<ending>-ending' option
2016/05/30 v2.5b - fix issue #72
2016/07/20 v2.6  - \l__acro_use_acronyms_bool can be used to disable \ac
                   e.g. in the trial phase of a table like `tabu'; interface:
                   \acro_switch_off: and \acswitchoff
                 - fix issue #79
                 - fix issue #74
                 - fix error: acronyms with same sort entry are not
                   overwritten any more in the list of acronyms
2016/08/13 v2.6a - fix issues #80 and #81
2016/08/13 v2.6b - version stepped by accident
2016/08/16 v2.6c - really fixes issue #81
2016/08/30 v2.6d - fix issue #82
2016/09/04 v2.6e - fix issue in http://tex.stackexchange.com/q/270034/
2017/01/22 v2.7  - rename \acro_get_property:nn into \__acro_get_property:nn
                 - \acro_get_property:nn, \acro_get_property:nnTF,
                   \acro_if_property:nnTF, retrieve property without error if
                   not set
                 - make \__acro_declare_property functions public
                 - \acro_add_action:n (adds code to \acro_get:n)
2017/02/09 v2.7a - adapt to integration of l3sort into l3kernel

% --------------------------------------------------------------------------
% TODO:
- extend option `macros' to also define uppercase macros, possibly as a choice
- add \ACF, \ACFP, \ACL and \ACLP that will print all words of the long form
  capitalized
           % A discussion on generating PDF files.
\end{appendices}         % End of the Appendix Chapters.  ibid on \end{appendix}
%\NeedsTeXFormat{LaTeX2e}
\ProvidesClass{vita}[1996/10/09
                     class file ``vita'' to create Curriculum Vitae]
%%%%%%%%%%%%%%%%%%%%%%%%%%%%%%%%%%%%%%%%%%%%%%%%%%%%%%%%%%%%%%%%%%%%%%%
%%
%% (C) Copyright 1995, Andrej Brodnik, ABrodnik@UWaterloo.CA. All
%% rights reserved.
%%
%% This is a generated file. Permission is granted to to customize the 
%% declarations in this file to serve the needs of your installation. 
%% However, no permission is granted to distribute a modified version of 
%% this file under its original name. 
%% 
%% \CharacterTable
%%  {Upper-case    \A\B\C\D\E\F\G\H\I\J\K\L\M\N\O\P\Q\R\S\T\U\V\W\X\Y\Z
%%   Lower-case    \a\b\c\d\e\f\g\h\i\j\k\l\m\n\o\p\q\r\s\t\u\v\w\x\y\z
%%   Digits        \0\1\2\3\4\5\6\7\8\9
%%   Exclamation   \!     Double quote  \"     Hash (number) \#
%%   Dollar        \$     Percent       \%     Ampersand     \&
%%   Acute accent  \'     Left paren    \(     Right paren   \)
%%   Asterisk      \*     Plus          \+     Comma         \,
%%   Minus         \-     Point         \.     Solidus       \/
%%   Colon         \:     Semicolon     \;     Less than     \<
%%   Equals        \=     Greater than  \>     Question mark \?
%%   Commercial at \@     Left bracket  \[     Backslash     \\
%%   Right bracket \]     Circumflex    \^     Underscore    \_
%%   Grave accent  \`     Left brace    \{     Vertical bar  \|
%%   Right brace   \}     Tilde         \~}
%%
%%---

%%%%%%%%%%%%%%%%%%%%%%%%%%%%%%%%%%%%%%%%%%%%%%%%%%%%%%%%%%%%%%%%%%%%%%%
%
%    - based on vita.sty by kcb@hss.caltech.edu
%    - 1995/02/07: the first version
%    - 1996/10/09: if there is no business address the field is
%                  left out
%
% User documentation: This class file only provides basic definitions
% =================== of environments, which are then used in class
% option files to instantiate entries for different disciplines. Thus,
% create your document as follows:
%
%    \documentclass[<discipline>]{vita}
%    \begin{document}
%      \name{Andrej Brodnik}
%      \businessAddress{First line \\ second line of bussines address}
%      \homeAddress{Again \\ multiline address \\ perhaps with phone number}
%    \begin{vita}
%      % here comes a real Curriculum Vitae for particular <discipline>
%    \end{vita}
%    \end{document}
%
% where it is assumed that file ``vita<discipline>.clo'' exists and defines
% proper categories used in given discipline. For detail explanation on
% categories in different disciplines see individual ``.clo'' files.
%
% The output will have format:
%
%   o on the first page will appear a title ``Curriculum Vitae'' (to
%     change it, see below under i18n notes -- internationalization)
%   o below will be your name
%   o below, side by side, your business and home address headed
%     by strings ``Business address'' and ``Home address''
%     respectively (to change these strings see below in i18n notes).
%   o then will follow the rest of CV as defined by ``<discipline>.clo''
%     file.
%   o the header of each but first page will include your name and the
%     page number.
%   o on the last page in the bottom right you will have the current
%     date, that is month and year (to change this, see below under
%     i18n notes).
%
%------
%
% i18n NOTES: If you are making CV for some other language, you have to
% =========== redefine:
%   - title:
%       o use command:   ``\title{<new title>}''
%       o default value: ``Curriculum Vitae''
%   - date:   
%       o use command:   ``\today{<date})''
%       o default value: ``<current month>, <current year>'' (in English)
%   - addresses headers:
%       o use command:   ``\HeaderBusiness{<new header>}''
%                        ``\HeaderHome{<new header>}''
%       o default value: ``Business address''
%                        ``Home address''
%
%------
%
% System documentation: class ``vita'' is based on the class
% ===================== ``article''. It changes the title into
% <default value> (see i18n notes) and the name becomes an
% author. Individual categories, publications and references are
% implemented using ``description'' environment. 
%
%----------------------------------------

%%%%
%
% Process options and load class article:
%---
\let\@optionsToInput=\@empty
\DeclareOption*{
  \IfFileExists{vita\CurrentOption.clo}%
    {\edef\@optionToInput{vita\CurrentOption.clo}}%
    {\PassOptionsToClass{\CurrentOption}{article}}
}
\ProcessOptions
\LoadClass{article}

%%%%
%
% First all i18n definitions:
%---
\title{Curriculum Vitae}
\renewcommand{\today}{
  \ifcase\month\or
    January\or February\or March\or April\or May\or June\or
    July\or August\or September\or October\or November\or December\fi,
  \space\number\year}
\newcommand\HeaderBusiness[1]{\def\@businessAddressHeader{#1}}
  \HeaderBusiness{Business Address}
\newcommand\HeaderHome[1]{\def\@homeAddressHeader{#1}}
  \HeaderHome{Home Address}

%%%%
%
% Next, header definitions:
%---
\date{\relax}
\newcommand{\name}[1]{
  \renewcommand{\@author}{#1} \markright{\protect\small\@author}
}
\newcommand{\businessAddress}[1]{\def\@businessAddress{#1}}
  \businessAddress{}
\newcommand{\homeAddress}[1]{\def\@homeAddress{#1}}
  \homeAddress{}

%%%%
%
% \maketitle command, which prints out the title and the name of person
%---
\renewcommand{\maketitle}{\newpage
  \global\@topnum\z@   % Prevents figures from going at top of page.
  \begin{center}
    {\LARGE \@title}

    \medskip

    {\large \@author}
  \end{center}

  \bigskip

  \thispagestyle{plain}

  \gdef\@author{}\gdef\@title{}
}

%%%%
%
% ``vita'' environment:
%---
\pagestyle{empty}
\newenvironment{vita}{
     % first page is empty style though the following pages have on the
     % right side written the name from the \name command
  \ifx\@author\@empty\@warning{Missing name command}\fi
     % next we start to layout information. First the title and the
     % name,

  \maketitle
     % followed by both addresses,
  \begin{tabular*}{\textwidth}{@{\extracolsep{\fill}}ll@{}}
    \begin{tabular}[t]{@{}l@{}}
    \ifx\@businessAddress\@empty\mbox{}\else
       {\small \@businessAddressHeader:}
    \\ \@businessAddress
    \fi
    \end{tabular}
  &
    \ifx\@homeAddress\@empty\@warning{Missing home address}%
    \else
      \begin{tabular}[t]{@{}l@{}}
         {\small \@homeAddressHeader:}
      \\ \@homeAddress
      \end{tabular}
    \fi
  \end{tabular*}

  \bigskip

  \thispagestyle{empty}
}{   % quite at the bottom of last page we have a date
  \par\nopagebreak\vfill\hfill \today
}%end vita environment

%%%%
%
% Curriculum vitae consists of categories which we create using
% command:
%
%      \newcategory[The name]{The label}
%
% where <The name> is written in bold character as a small title of
% category. It appears at the left margine of a page. If <The name>
% parameter is missing, it takes the same value as <The label>, which,
% in turn is used to refer to individual category. For example
% commands:
%
%      \newcategory{Name of category}
%      \newcategory[Name of category]{Name of category}
%
% have the same result. Now, to use category:
%
%      \newcategory[Some category]{some other name}
%
% the input has form:
%
% \begin{some other name}
%   \item The first item
%   \item The second one etc.
% \end{some other name}
%
% and the category will have on the output title ``Some category''.
% Entries in each category are preceded by \item.
%
%-----
% i18n NOTE: One can use as the names of categories strings in
% ========== different languages, but the labels can be the same in
% the same language, which is useful if you have a single CV and you
% want outputs in different languages.
%---
\def\@newCategory[#1]#2{%
  \newenvironment{#2}{\medskip\pagebreak[2]\par
    \textbf{\small #1}\nopagebreak
    \begin{description}}{\end{description}\par}
}
\def\@noNameCategory#1{\@newCategory[#1]{#1}}
\def\newcategory{\@ifnextchar[{\@newCategory}{\@noNameCategory}}

%%%%
%
% Inside categories we have different ``kinds'' (such as different
% publications), which we create using command \newkind. It has the
% same parameters as \newcategory and all comments at command
% newcategory are also valid here.
%---
\def\@newKind[#1]#2{%
  \newenvironment{#2}{
    \pagebreak[2]
    \item \textbf{\small #1}\nopagebreak
      \begin{description}
  }{  \end{description}\par }
}
\def\@noNameKind#1{\@newKind[#1]{#1}}
\def\newkind{\@ifnextchar[{\@newKind}{\@noNameKind}}

%%%%
%
% There is a special category ``plaincategory'' which entries are
% simply listed without any indentation, and in particular, multiple
% references are separated by \and command. It can be used for
% references.
%---
\def\@newPlainCategory[#1]#2{%
  \newenvironment{#2}{
    \medskip\pagebreak[2]\par
    \textbf{\small #1}\nopagebreak
    \renewcommand{\and}{
             \end{tabular}
      \item[]\begin{tabular}[t]{l}
    }
    \begin{description}
    \item[] \begin{tabular}[t]{l}
  }{        \end{tabular}
    \end{description}\par
  }
}
\def\@noNamePlainCategory#1{\@newPlainCategory[#1]{#1}}
\def\newplaincategory{\@ifnextchar[{\@newPlainCategory}{\@noNamePlainCategory}}

%%%%
%
% Finally, formatting parameters and the possible option to input:
%---
\pagestyle{myheadings}
\parindent 0pt
\nofiles

\ifx\@optionToInput\@empty\relax
\else \input \@optionToInput
\fi
          % Optional Vita, use \begin{vita} vita text \end{vita}
\end{document}
\end{verbatim}
\end{quote}

\section{Prelude}
After the {\tt \verb|\begin{document}|} comes the preliminary information found in
theses.  In this manual, the information is kept in the file {\tt prelude.tex} (see
above).  These pages will need to be numbered with roman numerals, so use
\begin{quote}\tt\singlespace\begin{verbatim}
\clearpage\pagenumbering{roman}
\end{verbatim}\end{quote}

Next, comes your thesis or dissertation title, your name, date of graduation, department
and degree.
\begin{quote}\tt\singlespace\begin{verbatim}
\title{How to \LaTeX\ a Thesis}
\author{Eric R. Benedict}
\date{2000}
%   - The default degree is ``Doctor of Philosophy''
%     Degree can be changed using the command \degree{}
%\degree{New Degree}
%   - for a PhD dissertation (default), specify \dissertation
%\dissertation
%   - for a masters project report, specify \project
%\project
%   - for a preliminary report, specify \prelim
%\prelim
%   - for a masters thesis, specify \thesis
%\thesis
%   - The default department is ``Electrical Engineering''
%     The department can be changed using the command \department{}
%\department{New Department}
\end{verbatim}\end{quote}

If you specified the class option {\tt msthesis}, then the degree is changed to
{\em Master \break of Science} and the {\tt \verb|\thesis|} option is specified.  If you
want to have the masters margins with another document, then the {\tt \verb|\degree|}
and {\tt \verb|\dissertation|},  {\tt \verb|\project|}, {\em etc.\/} can be specified
as needed.

Once the
above are all defined, use  {\tt \verb|\maketitle|} to generate the title page.
\begin{quote}\tt\singlespace\begin{verbatim}
\maketitle
\end{verbatim}\end{quote}

If you wish to include a copyright page (see Section~\ref{copyright} for
information on registering the copyright.), then add the command
\begin{quote}\tt\singlespace\begin{verbatim}
\copyrightpage
\end{verbatim}\end{quote}
This will generate the proper copyright page and will use the name and date specified
in {\tt \verb|\author{}|} and {\tt \verb|\date{}|}.

Next are the dedications and acknowledgements:
\begin{quote}\tt\singlespace\begin{verbatim}
\begin{dedication}
To my pet rock, Skippy.
\end{dedication}

\begin{acknowledgments}
I thank the many people who have done lots of nice things for me.
\end{acknowledgments}
\end{verbatim}\end{quote}

You must tell \LaTeX{} to generate a table of contents, a list of tables and a list of
figures:
\begin{quote}\tt\singlespace\begin{verbatim}
\tableofcontents
\listoftables
\listoffigures
\end{verbatim}\end{quote}

If you wish to have a nomenclature, list of symbols or glossary it can go here.
\begin{quote}\tt\singlespace\begin{verbatim}
\begin{nomenclature}
%\begin{listofsymbols}
%\begin{glossary}
\begin{tabular}{ll}
$C_1$ & Constant 1\\
\ldots
\end{tabular}
%\end{glossary}
%\end{listofsymbols}
\end{nomenclature}
\end{verbatim}\end{quote}

If your abstract will be microfilmed by Bell and Howell (formerly UMI), then you
will need to generate an abstract of less than 350 words.  This abstract can be created
using the {\tt umiabstract} environment.  This environment requires that you define your
advisor and your advisor's title using {\tt \verb|\advisorname{}|} and
{\tt \verb|\advisortitle{}|}.
\begin{quote}\tt\singlespace\begin{verbatim}
\advisorname{Bucky J. Badger}
\advisortitle{Assistant Professor}
% ABSTRACT
\begin{umiabstract}
\noindent       % Don't indent first paragraph.
This explains the basics for using \LaTeX\ to typeset a
dissertation, thesis or project report for the University of
Wisconsin-Madison.

...

\end{umiabstract}
\end{verbatim}\end{quote}
This will place your name, title and required text at the top of the page and follow the
abstract text with your advisor's name at the bottom for your advisor's signature.  This
page is not numbered and would be submitted separately.

If you will have an abstract as part of your document, then the {\tt abstract} environment
should be used.
\begin{quote}\tt\singlespace\begin{verbatim}
\begin{abstract}
\noindent       % Don't indent first paragraph.
This explains the basics for using \LaTeX\ to typeset a
dissertation, thesis or project report for the University of
Wisconsin-Madison.

...

\end{abstract}
\end{verbatim}\end{quote}
This will generate a page number and it will be included in the Table
of Contents.  

If you will have both the UMI and regular abstracts like this document, then
you will probably want to write the abstract once and save it in a seperate
file such as {\tt abstract.tex}.  Then, you can use the same abstract for
both purposes.

\begin{quote}\begin{verbatim}
\begin{umiabstract}
  % !TEX root = main.tex
% !TEX encoding = Windows Latin 1
% !TEX TS-program = pdflatex
% 
% Archivo: abstract.tex (en ingles)


\chapter{Abstract} % No cambiar el titulo
\selectlanguage{english}
\noindent
Duis tristique sollicitudin leo nec consequat. Praesent et dui convallis velit tincidunt fermentum. Mauris cursus purus at sem viverra sed imperdiet sapien imperdiet. Aliquam mattis, elit eget rutrum vulputate, tortor sem pulvinar justo, sit amet mollis felis sem at nibh. Donec malesuada, neque id interdum eleifend, arcu augue porta elit, nec tristique libero metus at massa. Fusce fringilla laoreet rhoncus. Suspendisse potenti. Phasellus dignissim sodales mauris at pharetra. Donec gravida fringilla velit ac rutrum.

Curabitur ornare lectus id diam molestie eu imperdiet nulla tempus. Maecenas vestibulum enim et dui ornare blandit. Vivamus fermentum faucibus viverra. Maecenas at justo sapien. Aenean rhoncus augue mattis purus rhoncus venenatis. Suspendisse metus felis, porttitor in varius in, vulputate at tortor. Aliquam molestie, turpis et malesuada porta, tortor sapien pharetra sapien, ac rhoncus quam dolor a sapien. Pellentesque varius laoreet enim ut auctor. Nullam nec ultricies nisi. Nullam porta lectus et ante consectetur posuere.

Duis tristique sollicitudin leo nec consequat. Praesent et dui convallis velit tincidunt fermentum. Mauris cursus purus at sem viverra sed imperdiet sapien imperdiet. Aliquam mattis, elit eget rutrum vulputate, tortor sem pulvinar justo, sit amet mollis felis sem at nibh. Donec malesuada, neque id interdum eleifend, arcu augue porta elit, nec tristique libero metus at massa. Fusce fringilla laoreet rhoncus. Suspendisse potenti. Phasellus dignissim sodales mauris at pharetra. Donec gravida fringilla velit ac rutrum.

Duis tristique sollicitudin leo nec consequat. Praesent et dui convallis velit tincidunt fermentum. Mauris cursus purus at sem viverra sed imperdiet sapien imperdiet. Aliquam mattis, elit eget rutrum vulputate, tortor sem pulvinar justo, sit amet mollis felis sem at nibh. Donec malesuada, neque id interdum eleifend, arcu augue porta elit, nec tristique libero metus at massa. Fusce fringilla laoreet rhoncus. Suspendisse potenti. Phasellus dignissim sodales mauris at pharetra. Donec gravida fringilla velit ac rutrum.

Curabitur ornare lectus id diam molestie eu imperdiet nulla tempus. Maecenas vestibulum enim et dui ornare blandit. Vivamus fermentum faucibus viverra. Maecenas at justo sapien. Aenean rhoncus augue mattis purus rhoncus venenatis. Suspendisse metus felis, porttitor in varius in, vulputate at tortor. Aliquam molestie, turpis et malesuada porta, tortor sapien pharetra sapien, ac rhoncus quam dolor a sapien. Pellentesque varius laoreet enim ut auctor. Nullam nec ultricies nisi. Nullam porta lectus et ante consectetur posuere.

Duis tristique sollicitudin leo nec consequat. Praesent et dui convallis velit tincidunt fermentum. Mauris cursus purus at sem viverra sed imperdiet sapien imperdiet. Aliquam mattis, elit eget rutrum vulputate, tortor sem pulvinar justo, sit amet mollis felis sem at nibh. Donec malesuada, neque id interdum eleifend, arcu augue porta elit, nec tristique libero metus at massa. Fusce fringilla laoreet rhoncus. Suspendisse potenti. Phasellus dignissim sodales mauris at pharetra. Donec gravida fringilla velit ac rutrum.

\bigskip
\noindent
\textit{Key words:} first word; second word; third word.
% Separar palabras con punto-y-comas.

\checklanguage
% Fin archivo abstract.tex
\endinput 
\end{umiabstract}

\begin{abstract}
  % !TEX root = main.tex
% !TEX encoding = Windows Latin 1
% !TEX TS-program = pdflatex
% 
% Archivo: abstract.tex (en ingles)


\chapter{Abstract} % No cambiar el titulo
\selectlanguage{english}
\noindent
Duis tristique sollicitudin leo nec consequat. Praesent et dui convallis velit tincidunt fermentum. Mauris cursus purus at sem viverra sed imperdiet sapien imperdiet. Aliquam mattis, elit eget rutrum vulputate, tortor sem pulvinar justo, sit amet mollis felis sem at nibh. Donec malesuada, neque id interdum eleifend, arcu augue porta elit, nec tristique libero metus at massa. Fusce fringilla laoreet rhoncus. Suspendisse potenti. Phasellus dignissim sodales mauris at pharetra. Donec gravida fringilla velit ac rutrum.

Curabitur ornare lectus id diam molestie eu imperdiet nulla tempus. Maecenas vestibulum enim et dui ornare blandit. Vivamus fermentum faucibus viverra. Maecenas at justo sapien. Aenean rhoncus augue mattis purus rhoncus venenatis. Suspendisse metus felis, porttitor in varius in, vulputate at tortor. Aliquam molestie, turpis et malesuada porta, tortor sapien pharetra sapien, ac rhoncus quam dolor a sapien. Pellentesque varius laoreet enim ut auctor. Nullam nec ultricies nisi. Nullam porta lectus et ante consectetur posuere.

Duis tristique sollicitudin leo nec consequat. Praesent et dui convallis velit tincidunt fermentum. Mauris cursus purus at sem viverra sed imperdiet sapien imperdiet. Aliquam mattis, elit eget rutrum vulputate, tortor sem pulvinar justo, sit amet mollis felis sem at nibh. Donec malesuada, neque id interdum eleifend, arcu augue porta elit, nec tristique libero metus at massa. Fusce fringilla laoreet rhoncus. Suspendisse potenti. Phasellus dignissim sodales mauris at pharetra. Donec gravida fringilla velit ac rutrum.

Duis tristique sollicitudin leo nec consequat. Praesent et dui convallis velit tincidunt fermentum. Mauris cursus purus at sem viverra sed imperdiet sapien imperdiet. Aliquam mattis, elit eget rutrum vulputate, tortor sem pulvinar justo, sit amet mollis felis sem at nibh. Donec malesuada, neque id interdum eleifend, arcu augue porta elit, nec tristique libero metus at massa. Fusce fringilla laoreet rhoncus. Suspendisse potenti. Phasellus dignissim sodales mauris at pharetra. Donec gravida fringilla velit ac rutrum.

Curabitur ornare lectus id diam molestie eu imperdiet nulla tempus. Maecenas vestibulum enim et dui ornare blandit. Vivamus fermentum faucibus viverra. Maecenas at justo sapien. Aenean rhoncus augue mattis purus rhoncus venenatis. Suspendisse metus felis, porttitor in varius in, vulputate at tortor. Aliquam molestie, turpis et malesuada porta, tortor sapien pharetra sapien, ac rhoncus quam dolor a sapien. Pellentesque varius laoreet enim ut auctor. Nullam nec ultricies nisi. Nullam porta lectus et ante consectetur posuere.

Duis tristique sollicitudin leo nec consequat. Praesent et dui convallis velit tincidunt fermentum. Mauris cursus purus at sem viverra sed imperdiet sapien imperdiet. Aliquam mattis, elit eget rutrum vulputate, tortor sem pulvinar justo, sit amet mollis felis sem at nibh. Donec malesuada, neque id interdum eleifend, arcu augue porta elit, nec tristique libero metus at massa. Fusce fringilla laoreet rhoncus. Suspendisse potenti. Phasellus dignissim sodales mauris at pharetra. Donec gravida fringilla velit ac rutrum.

\bigskip
\noindent
\textit{Key words:} first word; second word; third word.
% Separar palabras con punto-y-comas.

\checklanguage
% Fin archivo abstract.tex
\endinput 
\end{abstract}
\end{verbatim}\end{quote}

Finally, the page numbers must be changed to arabic numbers to conclude the preliminary
portion of the document.
\begin{quote}\tt\singlespace\begin{verbatim}
\clearpage\pagenumbering{arabic}
\end{verbatim}\end{quote}

\section{The Body}
At the beginning of {\tt intro.tex} there is the following command:
\begin{quote}\tt\singlespace\begin{verbatim}
\chapter{Introducing the {\tt withesis} \LaTeX{} Style Guide}
\end{verbatim}\end{quote}
Following that is the text of the chapter.  The body of your thesis is separated by
sectioning commands like {\tt \verb|\chapter{}|}.  For more information on the sectioning
commands, see Section~\ref{ess:sectioning}.

Remember the basic rule of outlining you learned in grammar school:
\begin{quote}
You cannot have an `A' if you do not have a `B'
\end{quote}
Take care to have at least two {\tt \verb|\section|}s if you use the command; have
two {\tt \verb|\subsection|}s, {\em etc}.



\section{Additional Theorem Like Environments}
The {\tt withesis} style adds numerous additional theorem like environments.  These
environments were included to allow compatibility with the University of Wisconsin's
Math Department's style file.  These environments are
{\tt theorem}, {\tt assertion}, {\tt claim}, {\tt conjecture}, {\tt corollary},
{\tt definition}, {\tt example}, {\tt figger}, {\tt lemma}, {\tt prop} and {\tt remark}.

As an example, consider the following.
\begin{lemma}
Assuming that $\partial\Omega_2 = \emptyset$ and that $h(t) = 1$, we
have $$
\begin{array}{lr}
\Delta u = f, &  x\in\Omega ,\\[2pt]
u =  g_1, &  x\in\partial\Omega .
\end{array}
$$
\end{lemma}
which was produced with the following:
\begin{quote}\tt\singlespace\begin{verbatim}
\begin{lemma}
Assuming that $\partial\Omega_2 = \emptyset$ and that $h(t) = 1$, we
have $$
\begin{array}{lr}
\Delta u = f, &  x\in\Omega ,\\[2pt]
 u =  g_1, &  x\in\partial\Omega .
\end{array}
$$
\end{lemma}
\end{verbatim}\end{quote}

\section{Bibliography or References}
As a final note, the default title for the references chapter is ``LIST OF REFERENCES.''
Since some people may prefer ``BIBLIOGRAPHY'', the command
\break{\tt \verb|\altbibtitle|}
has been added to change the chapter title.

\section{Appendices}
There are two commands which are available to suppress the writing of the auxiliary information
(to the {\tt .lot} and {\tt .lof} files).  They are:
\begin{quote}\tt\singlespace\begin{verbatim}
\noappendixtables                % Don't have appendix tables
\noappendixfigures               % Don't have appendix figures
\end{verbatim}\end{quote}
These commands should be in the preamble.  See Section~\ref{usage:noapp}.

There are two environments for doing the appendix chapter: {\tt appendix} and
\break {\tt appendices}.  If you have only one chapter in the appendix, use the {\tt appendix}
environment.  If you have more than one chapter, like this manual, use the
{\tt appendices} environment.
\begin{quote}\tt\singlespace\footnotesize\begin{verbatim}
\begin{appendices}  % Start of the Appendix Chapters.  If there is only
                    % one Appendix Chapter, then use \begin{appendix}
% code.tex
% this file is part of the example UW-Madison Thesis document
% It demonstrates one method for incorporating program listings
% into a document.

\chapter{Matlab Code} \label{matlab}
This is an example of a Matlab m-file.
\verbatimfile{derivs.m}
      % Including computer code listings
\chapter{Bib\TeX\ Entries}
\label{bibrefs}
The following shows the fields required in all types of Bib\TeX\ entries.
Fields with {\tt OPT} prefixed are optional (the three letters {\tt OPT} should 
not be used).  If an optional field is not used, then the entire field can be deleted.

{\tt
\singlespace
\begin{verbatim}

@Unpublished{,                            @Manual{,
  author =      "",                         title =           "",
  title =       "",                         OPTauthor =       "",
  note =        "",                         OPTorganization = "",
  OPTyear =     "",                         OPTaddress =      "",
  OPTmonth =    ""                          OPTedition =      "",
}                                           OPTyear =         "",
                                            OPTmonth =        "",
@TechReport{,                               OPTnote =         "" 
  author =      "",                       }
  title =       "",
  institution = "",                       @InProceedings{,
  year =        "",                         author =          "",
  OPTtype =     "",                         title =           "",
  OPTnumber =   "",                         booktitle =       "",
  OPTaddress =  "",                         year =            "",
  OPTmonth =    "",                         OPTeditor =       "",
  OPTnote =     ""                          OPTpages =        "",
}                                           OPTorganization = "",
                                            OPTpublisher =    "",
@Proceedings{,                              OPTaddress =      "",
  title =           "",                     OPTmonth =        "",
  year =            "",                     OPTnote =         "" 
  OPTeditor =       "",                   }
  OPTpublisher =    "",
  OPTorganization = "",
  OPTaddress =      "",
  OPTmonth =        "",
  OPTnote =         "" 
}



@PhDThesis{,                              @InCollection{,
  author =      "",                         author =          "",
  title =       "",                         title =           "",
  school =      "",                         booktitle =       "",
  year =        "",                         publisher =       "",
  OPTaddress =  "",                         year =            "",
  OPTmonth =    "",                         OPTeditor =       "",
  OPTnote =     ""                          OPTchapter =      "",
}                                           OPTpages =        "",
                                            OPTaddress =      "",
                                            OPTmonth =        "",
                                            OPTnote =         ""
                                          }

 
@Misc{,                                   @InCollection{,
  OPTauthor =       "",                     author =          "",
  OPTtitle =        "",                     title =           "",
  OPThowpublished = "",                     chapter =         "",
  OPTyear =         "",                     publisher =       "",
  OPTmonth =        "",                     year =            "",
  OPTnote =         ""                      OPTeditor =       "",
}                                           OPTpages =        "",
}                                           OPTvolume =       "",
                                            OPTseries =       "",
                                            OPTaddress =      "",
                                            OPTedition =      "",
                                            OPTmonth =        "",
                                            OPTnote =         ""
                                          }

@MastersThesis{,                          @Article{,
  author =      "",                         author =          "",
  title =       "",                         title =           "",
  school =      "",                         journal =         "",
  year =        "",                         year =            "",
  OPTaddress =  "",                         OPTvolume =       "",
  OPTmonth =    "",                         OPTnumber =       "",
  OPTnote =     ""                          OPTpages =        "",
}                                           OPTmonth =        "",
                                            OPTnote =         ""
                                           }\end{verbatim} }
    % a BibTeX reference
\chapter{Mathematics Examples}
This appendix provides an example of \LaTeX's typesetting
capabilities.  Most of text was obtained from the University of
Wisconsin-Madison Math Department's example thesis file.

\section{Matrices}
The equations for the {\em dq}-model of an induction machine in the
synchronous reference frame are
\begin{eqnarray}
 \left[\begin{array}{c} v_{qs}^e\\v_{ds}^e\\v_{qr}^e\\v_{dr}^e  \end{array}\right]                                                                                                                                                                                                                                                                                                                                                                                                                                                                                                              
 &=& \left[ \begin{array}{cccc}
 r_s + x_s\frac{\rho}{\omega_b} & \frac{\omega_e}{\omega_b}x_s & x_m\frac{\rho}{\omega_b} & \frac{\omega_e}{\omega_b}x_m \\
 -\frac{\omega_e}{\omega_b}x_s & r_s + x_s\frac{\rho}{\omega_b} & -\frac{\omega_e}{\omega_b}x_m & x_m\frac{\rho}{\omega_b} \\
 x_m\frac{\rho}{\omega_b} & \frac{\omega_e -\omega_r}{\omega_b}x_m & r_r'+x_r'\frac{\rho}{\omega_b} & \frac{\omega_e - \omega_r}{\omega_b}x_r' \\
 -\frac{\omega_e -\omega_r}{\omega_b}x_m & x_m\frac{\rho}{\omega_b} & -\frac{\omega_e - \omega_r}{\omega_b}x_r' & r_r' + x_r'\frac{\rho}{\omega_b}
 \end{array} \right]
 \left[\begin{array}{c} i_{qs}^e\\i_{ds}^e\\i_{qr}^e\\i_{dr}^e\end{array} \right] \label{volteq}\\
 T_e&=&\frac{3}{2}\frac{P}{2}\frac{x_m}{\omega_b}\left(i_{qs}^ei_{dr}^e - i_{ds}^ei_{qr}^e\right) \label{torqueeq}\\
 T_e-T_l&=&\frac{2J\omega_b}{P}\frac{d}{dt}\left(\frac{\omega_r}{\omega_b}\right) \label{mecheq}.
\end{eqnarray}

\section{Multi-line Equations}

\LaTeX{} has a built-in equation array feature, however the
equation numbers must be on the same line as an equation.  For example:
\begin{eqnarray}
\Delta u + \lambda e^u &= 0&u\in \Omega,  \nonumber \\
u&=0&u\in\partial\Omega.
\end{eqnarray}

Alternatively, the number can be centered in the equation using the
following method.
%
% The equation-array feature in LaTeX is a bad idea.  For centered
% numbers you should set your own equations and arrays as follows:
%
\def\dd{\displaystyle}
\begin{equation}\label{gelfand}
\begin{array}{rl}
\dd \Delta u + \lambda e^u = 0, &
\dd u\in \Omega,\\[8pt] % add 8pt extra vertical space. 1 line=23pt
\dd u=0, & \dd u\in\partial\Omega.
\end{array}
\end{equation}
The previous equation had a label.  It may be referenced as
equation~(\ref{gelfand}).


\section{More Complicated Equations}
\section*{Rellich's identity}\label{rellich.section}
\setcounter{theorem}{0}
%
%

Standard developments of Pohozaev's identity used an identity by
Rellich~\cite{rellich:der40}, reproduced here.

\begin{lemma}[Rellich]
Given $L$ in divergence form and $a,d$ defined above, $u\in C^2
(\Omega )$, we have
\begin{equation}\label{rellich}
\int_{\Omega}(-Lu)\nabla u\cdot (x-\overline{x})\, dx
= (1-\frac{n}{2}) \int_{\Omega} a(\nabla u,\nabla u) \, dx
-
\frac{1}{2} \int_{\Omega}
d(\nabla u, \nabla u) \, dx
\end{equation}
$$
+
\frac{1}{2} \int_{\partial\Omega} a(\nabla u,\nabla u)(x-\overline{x})
\cdot \nu  \, dS
-
\int_{\partial\Omega}
a(\nabla u,\nu )\nabla u\cdot (x-\overline{x}) \, dS.
$$
\end{lemma}
{\bf Proof:}\\
There is no loss in generality to take $\overline{x} = 0$. First
rewrite $L$:
$$Lu = \frac{1}{2}\left[ \sum_{i}\sum_{j}
\frac{\partial}{\partial x_i}
\left( a_{ij} \frac{\partial u}{\partial x_j} \right) +
\sum_{i}\sum_{j}
\frac{\partial}{\partial x_i}
\left( a_{ij} \frac{\partial u}{\partial x_j} \right)
\right]$$
Switching the order of summation on the second term and relabeling
subscripts, $j \rightarrow i$ and $i \rightarrow j$, then using the fact
that $a_{ij}(x)$ is a symmetric matrix,
gives the symmetric form needed to derive Rellich's identity.
\begin{equation}
Lu = \frac{1}{2} \sum_{i,j}\left[
\frac{\partial}{\partial x_i}
\left( a_{ij} \frac{\partial u}{\partial x_j} \right) +
\frac{\partial}{\partial x_j}
\left( a_{ij} \frac{\partial u}{\partial x_i} \right)
\right].
\end{equation}

Multiplying $-Lu$ by $\frac{\partial u}{\partial x_k} x_k$ and integrating
over $\Omega$, yields
$$\int_{\Omega}(-Lu)\frac{\partial u}{\partial x_k} x_k \, dx=
-\frac{1}{2} \int_{\Omega}
\sum_{i,j}\left[
\frac{\partial}{\partial x_i}
\left( a_{ij} \frac{\partial u}{\partial x_j} \right) +
\frac{\partial}{\partial x_j}
\left( a_{ij} \frac{\partial u}{\partial x_i} \right)
\right]
\frac{\partial u}{\partial x_k} x_k \, dx$$
Integrating by parts (for integral theorems see~\cite[p. 20]
{zeidler:nfa88IIa})
gives
$$= \frac{1}{2} \int_{\Omega}
\sum_{i,j} a_{ij} \left[
\frac{\partial u}{\partial x_j}
\frac{\partial^2 u}{\partial x_k\partial x_i} +
\frac{\partial u}{\partial x_i}
\frac{\partial^2 u}{\partial x_k\partial x_j}
\right] x_k \, dx
$$
$$
+
\frac{1}{2} \int_{\Omega}
\sum_{i,j} a_{ij} \left[
\frac{\partial u}{\partial x_j} \delta_{ik} +
\frac{\partial u}{\partial x_i} \delta_{jk}
\right] \frac{\partial u}{\partial x_k} \, dx
$$
$$- \frac{1}{2} \int_{\partial\Omega}
\sum_{i,j} a_{ij} \left[
\frac{\partial u}{\partial x_j} \nu_{i} +
\frac{\partial u}{\partial x_i} \nu_{j}
\right] \frac{\partial u}{\partial x_k} x_k \, dx
$$
= $I_1 + I_2 + I_3$, where the unit normal vector is $\nu$.
One may rewrite $I_1$ as
$$I_1 = \frac{1}{2} \int_{\Omega}
\sum_{i,j} a_{ij} \frac{\partial}{\partial x_k}\left(
\frac{\partial u}{\partial x_i}
\frac{\partial u}{\partial x_j}
\right) x_k \, dx
$$
Integrating the first term by parts again yields
$$I_1 = -\frac{1}{2} \int_{\Omega}
\sum_{i,j} a_{ij} \left(
\frac{\partial u}{\partial x_i}
\frac{\partial u}{\partial x_j}
\right) \, dx
+
\frac{1}{2} \int_{\partial\Omega}
\sum_{i,j} a_{ij} \left(
\frac{\partial u}{\partial x_i}
\frac{\partial u}{\partial x_j}
\right) x_k \nu_k \, dS
$$
$$
-
\frac{1}{2} \int_{\Omega}
\sum_{i,j} \left(
\frac{\partial u}{\partial x_i}
\frac{\partial u}{\partial x_j}
\right) x_k \frac{\partial a_{ij}}{\partial x_k}\, dx.
$$
Summing over $k$ gives
$$\int_{\Omega}(-Lu)(\nabla u\cdot x)\, dx =
-\frac{n}{2} \int_{\Omega}
\sum_{i,j} a_{ij} \left(
\frac{\partial u}{\partial x_i}
\frac{\partial u}{\partial x_j}
\right) \, dx
$$
$$
+
\frac{1}{2} \int_{\partial\Omega}
\sum_{i,j} a_{ij} \left(
\frac{\partial u}{\partial x_i}
\frac{\partial u}{\partial x_j}
\right) (x\cdot \nu ) \, dS
-
\frac{1}{2} \int_{\Omega}
\sum_{i,j} \left(
\frac{\partial u}{\partial x_i}
\frac{\partial u}{\partial x_j}
\right) (x\cdot  \nabla a_{ij}) \, dx
$$
$$
+
\frac{1}{2} \int_{\Omega}
\sum_{i,j,k} a_{ij} \left[
\frac{\partial u}{\partial x_j}
\frac{\partial u}{\partial x_k} \delta_{ik} +
\frac{\partial u}{\partial x_i}
\frac{\partial u}{\partial x_k} \delta_{jk}
\right] \, dx
$$
$$- \frac{1}{2} \int_{\partial\Omega}
\sum_{i,j} a_{ij} \left[
\frac{\partial u}{\partial x_j} \nu_{i} +
\frac{\partial u}{\partial x_i} \nu_{j}
\right] (\nabla u\cdot x) \, dS.
$$
Combining the first and fourth term on the right-hand side
simplifies the expression
$$\int_{\Omega}(-Lu)(\nabla u\cdot x)\, dx
=
(1-\frac{n}{2}) \int_{\Omega}
\sum_{i,j} a_{ij} \left(
\frac{\partial u}{\partial x_i}
\frac{\partial u}{\partial x_j}
\right) \, dx
$$
$$
+
\frac{1}{2} \int_{\partial\Omega}
\sum_{i,j} a_{ij} \left(
\frac{\partial u}{\partial x_i}
\frac{\partial u}{\partial x_j}
\right) (x\cdot \nu ) \, dS
-
\frac{1}{2} \int_{\Omega}
\sum_{i,j} \left(
\frac{\partial u}{\partial x_i}
\frac{\partial u}{\partial x_j}
\right) (x\cdot  \nabla a_{ij}) \, dx
$$
$$
-
\frac{1}{2} \int_{\partial\Omega}
\sum_{i,j} a_{ij} \left[
\frac{\partial u}{\partial x_j} \nu_{i} +
\frac{\partial u}{\partial x_i} \nu_{j}
\right] (\nabla u\cdot x) \, dS.
$$
Using the notation defined above, the result follows.


      % Complex Equations from the UW Math Department

\end{appendices}    % End of the Appendix Chapters. ibid on \end{appendix}
\end{verbatim}\end{quote}
The difference between these two environments is the way that the chapter header is
created and how this is listed in the table of contents.
          % Chapter 5 Strongly based on similar by J.D. McCauley
\bibliography{refs}      % Make the bibliography
\begin{appendices}       % Start of the Appendix Chapters.  If there is only
                         % one Appendix Chapter, then use \begin{appendix}
% code.tex
% this file is part of the example UW-Madison Thesis document
% It demonstrates one method for incorporating program listings
% into a document.

\chapter{Matlab Code} \label{matlab}
This is an example of a Matlab m-file.
\verbatimfile{derivs.m}
         % Including computer code listings
\chapter{Bib\TeX\ Entries}
\label{bibrefs}
The following shows the fields required in all types of Bib\TeX\ entries.
Fields with {\tt OPT} prefixed are optional (the three letters {\tt OPT} should 
not be used).  If an optional field is not used, then the entire field can be deleted.

{\tt
\singlespace
\begin{verbatim}

@Unpublished{,                            @Manual{,
  author =      "",                         title =           "",
  title =       "",                         OPTauthor =       "",
  note =        "",                         OPTorganization = "",
  OPTyear =     "",                         OPTaddress =      "",
  OPTmonth =    ""                          OPTedition =      "",
}                                           OPTyear =         "",
                                            OPTmonth =        "",
@TechReport{,                               OPTnote =         "" 
  author =      "",                       }
  title =       "",
  institution = "",                       @InProceedings{,
  year =        "",                         author =          "",
  OPTtype =     "",                         title =           "",
  OPTnumber =   "",                         booktitle =       "",
  OPTaddress =  "",                         year =            "",
  OPTmonth =    "",                         OPTeditor =       "",
  OPTnote =     ""                          OPTpages =        "",
}                                           OPTorganization = "",
                                            OPTpublisher =    "",
@Proceedings{,                              OPTaddress =      "",
  title =           "",                     OPTmonth =        "",
  year =            "",                     OPTnote =         "" 
  OPTeditor =       "",                   }
  OPTpublisher =    "",
  OPTorganization = "",
  OPTaddress =      "",
  OPTmonth =        "",
  OPTnote =         "" 
}



@PhDThesis{,                              @InCollection{,
  author =      "",                         author =          "",
  title =       "",                         title =           "",
  school =      "",                         booktitle =       "",
  year =        "",                         publisher =       "",
  OPTaddress =  "",                         year =            "",
  OPTmonth =    "",                         OPTeditor =       "",
  OPTnote =     ""                          OPTchapter =      "",
}                                           OPTpages =        "",
                                            OPTaddress =      "",
                                            OPTmonth =        "",
                                            OPTnote =         ""
                                          }

 
@Misc{,                                   @InCollection{,
  OPTauthor =       "",                     author =          "",
  OPTtitle =        "",                     title =           "",
  OPThowpublished = "",                     chapter =         "",
  OPTyear =         "",                     publisher =       "",
  OPTmonth =        "",                     year =            "",
  OPTnote =         ""                      OPTeditor =       "",
}                                           OPTpages =        "",
}                                           OPTvolume =       "",
                                            OPTseries =       "",
                                            OPTaddress =      "",
                                            OPTedition =      "",
                                            OPTmonth =        "",
                                            OPTnote =         ""
                                          }

@MastersThesis{,                          @Article{,
  author =      "",                         author =          "",
  title =       "",                         title =           "",
  school =      "",                         journal =         "",
  year =        "",                         year =            "",
  OPTaddress =  "",                         OPTvolume =       "",
  OPTmonth =    "",                         OPTnumber =       "",
  OPTnote =     ""                          OPTpages =        "",
}                                           OPTmonth =        "",
                                            OPTnote =         ""
                                           }\end{verbatim} }
         % a BibTeX reference
\chapter{Mathematics Examples}
This appendix provides an example of \LaTeX's typesetting
capabilities.  Most of text was obtained from the University of
Wisconsin-Madison Math Department's example thesis file.

\section{Matrices}
The equations for the {\em dq}-model of an induction machine in the
synchronous reference frame are
\begin{eqnarray}
 \left[\begin{array}{c} v_{qs}^e\\v_{ds}^e\\v_{qr}^e\\v_{dr}^e  \end{array}\right]                                                                                                                                                                                                                                                                                                                                                                                                                                                                                                              
 &=& \left[ \begin{array}{cccc}
 r_s + x_s\frac{\rho}{\omega_b} & \frac{\omega_e}{\omega_b}x_s & x_m\frac{\rho}{\omega_b} & \frac{\omega_e}{\omega_b}x_m \\
 -\frac{\omega_e}{\omega_b}x_s & r_s + x_s\frac{\rho}{\omega_b} & -\frac{\omega_e}{\omega_b}x_m & x_m\frac{\rho}{\omega_b} \\
 x_m\frac{\rho}{\omega_b} & \frac{\omega_e -\omega_r}{\omega_b}x_m & r_r'+x_r'\frac{\rho}{\omega_b} & \frac{\omega_e - \omega_r}{\omega_b}x_r' \\
 -\frac{\omega_e -\omega_r}{\omega_b}x_m & x_m\frac{\rho}{\omega_b} & -\frac{\omega_e - \omega_r}{\omega_b}x_r' & r_r' + x_r'\frac{\rho}{\omega_b}
 \end{array} \right]
 \left[\begin{array}{c} i_{qs}^e\\i_{ds}^e\\i_{qr}^e\\i_{dr}^e\end{array} \right] \label{volteq}\\
 T_e&=&\frac{3}{2}\frac{P}{2}\frac{x_m}{\omega_b}\left(i_{qs}^ei_{dr}^e - i_{ds}^ei_{qr}^e\right) \label{torqueeq}\\
 T_e-T_l&=&\frac{2J\omega_b}{P}\frac{d}{dt}\left(\frac{\omega_r}{\omega_b}\right) \label{mecheq}.
\end{eqnarray}

\section{Multi-line Equations}

\LaTeX{} has a built-in equation array feature, however the
equation numbers must be on the same line as an equation.  For example:
\begin{eqnarray}
\Delta u + \lambda e^u &= 0&u\in \Omega,  \nonumber \\
u&=0&u\in\partial\Omega.
\end{eqnarray}

Alternatively, the number can be centered in the equation using the
following method.
%
% The equation-array feature in LaTeX is a bad idea.  For centered
% numbers you should set your own equations and arrays as follows:
%
\def\dd{\displaystyle}
\begin{equation}\label{gelfand}
\begin{array}{rl}
\dd \Delta u + \lambda e^u = 0, &
\dd u\in \Omega,\\[8pt] % add 8pt extra vertical space. 1 line=23pt
\dd u=0, & \dd u\in\partial\Omega.
\end{array}
\end{equation}
The previous equation had a label.  It may be referenced as
equation~(\ref{gelfand}).


\section{More Complicated Equations}
\section*{Rellich's identity}\label{rellich.section}
\setcounter{theorem}{0}
%
%

Standard developments of Pohozaev's identity used an identity by
Rellich~\cite{rellich:der40}, reproduced here.

\begin{lemma}[Rellich]
Given $L$ in divergence form and $a,d$ defined above, $u\in C^2
(\Omega )$, we have
\begin{equation}\label{rellich}
\int_{\Omega}(-Lu)\nabla u\cdot (x-\overline{x})\, dx
= (1-\frac{n}{2}) \int_{\Omega} a(\nabla u,\nabla u) \, dx
-
\frac{1}{2} \int_{\Omega}
d(\nabla u, \nabla u) \, dx
\end{equation}
$$
+
\frac{1}{2} \int_{\partial\Omega} a(\nabla u,\nabla u)(x-\overline{x})
\cdot \nu  \, dS
-
\int_{\partial\Omega}
a(\nabla u,\nu )\nabla u\cdot (x-\overline{x}) \, dS.
$$
\end{lemma}
{\bf Proof:}\\
There is no loss in generality to take $\overline{x} = 0$. First
rewrite $L$:
$$Lu = \frac{1}{2}\left[ \sum_{i}\sum_{j}
\frac{\partial}{\partial x_i}
\left( a_{ij} \frac{\partial u}{\partial x_j} \right) +
\sum_{i}\sum_{j}
\frac{\partial}{\partial x_i}
\left( a_{ij} \frac{\partial u}{\partial x_j} \right)
\right]$$
Switching the order of summation on the second term and relabeling
subscripts, $j \rightarrow i$ and $i \rightarrow j$, then using the fact
that $a_{ij}(x)$ is a symmetric matrix,
gives the symmetric form needed to derive Rellich's identity.
\begin{equation}
Lu = \frac{1}{2} \sum_{i,j}\left[
\frac{\partial}{\partial x_i}
\left( a_{ij} \frac{\partial u}{\partial x_j} \right) +
\frac{\partial}{\partial x_j}
\left( a_{ij} \frac{\partial u}{\partial x_i} \right)
\right].
\end{equation}

Multiplying $-Lu$ by $\frac{\partial u}{\partial x_k} x_k$ and integrating
over $\Omega$, yields
$$\int_{\Omega}(-Lu)\frac{\partial u}{\partial x_k} x_k \, dx=
-\frac{1}{2} \int_{\Omega}
\sum_{i,j}\left[
\frac{\partial}{\partial x_i}
\left( a_{ij} \frac{\partial u}{\partial x_j} \right) +
\frac{\partial}{\partial x_j}
\left( a_{ij} \frac{\partial u}{\partial x_i} \right)
\right]
\frac{\partial u}{\partial x_k} x_k \, dx$$
Integrating by parts (for integral theorems see~\cite[p. 20]
{zeidler:nfa88IIa})
gives
$$= \frac{1}{2} \int_{\Omega}
\sum_{i,j} a_{ij} \left[
\frac{\partial u}{\partial x_j}
\frac{\partial^2 u}{\partial x_k\partial x_i} +
\frac{\partial u}{\partial x_i}
\frac{\partial^2 u}{\partial x_k\partial x_j}
\right] x_k \, dx
$$
$$
+
\frac{1}{2} \int_{\Omega}
\sum_{i,j} a_{ij} \left[
\frac{\partial u}{\partial x_j} \delta_{ik} +
\frac{\partial u}{\partial x_i} \delta_{jk}
\right] \frac{\partial u}{\partial x_k} \, dx
$$
$$- \frac{1}{2} \int_{\partial\Omega}
\sum_{i,j} a_{ij} \left[
\frac{\partial u}{\partial x_j} \nu_{i} +
\frac{\partial u}{\partial x_i} \nu_{j}
\right] \frac{\partial u}{\partial x_k} x_k \, dx
$$
= $I_1 + I_2 + I_3$, where the unit normal vector is $\nu$.
One may rewrite $I_1$ as
$$I_1 = \frac{1}{2} \int_{\Omega}
\sum_{i,j} a_{ij} \frac{\partial}{\partial x_k}\left(
\frac{\partial u}{\partial x_i}
\frac{\partial u}{\partial x_j}
\right) x_k \, dx
$$
Integrating the first term by parts again yields
$$I_1 = -\frac{1}{2} \int_{\Omega}
\sum_{i,j} a_{ij} \left(
\frac{\partial u}{\partial x_i}
\frac{\partial u}{\partial x_j}
\right) \, dx
+
\frac{1}{2} \int_{\partial\Omega}
\sum_{i,j} a_{ij} \left(
\frac{\partial u}{\partial x_i}
\frac{\partial u}{\partial x_j}
\right) x_k \nu_k \, dS
$$
$$
-
\frac{1}{2} \int_{\Omega}
\sum_{i,j} \left(
\frac{\partial u}{\partial x_i}
\frac{\partial u}{\partial x_j}
\right) x_k \frac{\partial a_{ij}}{\partial x_k}\, dx.
$$
Summing over $k$ gives
$$\int_{\Omega}(-Lu)(\nabla u\cdot x)\, dx =
-\frac{n}{2} \int_{\Omega}
\sum_{i,j} a_{ij} \left(
\frac{\partial u}{\partial x_i}
\frac{\partial u}{\partial x_j}
\right) \, dx
$$
$$
+
\frac{1}{2} \int_{\partial\Omega}
\sum_{i,j} a_{ij} \left(
\frac{\partial u}{\partial x_i}
\frac{\partial u}{\partial x_j}
\right) (x\cdot \nu ) \, dS
-
\frac{1}{2} \int_{\Omega}
\sum_{i,j} \left(
\frac{\partial u}{\partial x_i}
\frac{\partial u}{\partial x_j}
\right) (x\cdot  \nabla a_{ij}) \, dx
$$
$$
+
\frac{1}{2} \int_{\Omega}
\sum_{i,j,k} a_{ij} \left[
\frac{\partial u}{\partial x_j}
\frac{\partial u}{\partial x_k} \delta_{ik} +
\frac{\partial u}{\partial x_i}
\frac{\partial u}{\partial x_k} \delta_{jk}
\right] \, dx
$$
$$- \frac{1}{2} \int_{\partial\Omega}
\sum_{i,j} a_{ij} \left[
\frac{\partial u}{\partial x_j} \nu_{i} +
\frac{\partial u}{\partial x_i} \nu_{j}
\right] (\nabla u\cdot x) \, dS.
$$
Combining the first and fourth term on the right-hand side
simplifies the expression
$$\int_{\Omega}(-Lu)(\nabla u\cdot x)\, dx
=
(1-\frac{n}{2}) \int_{\Omega}
\sum_{i,j} a_{ij} \left(
\frac{\partial u}{\partial x_i}
\frac{\partial u}{\partial x_j}
\right) \, dx
$$
$$
+
\frac{1}{2} \int_{\partial\Omega}
\sum_{i,j} a_{ij} \left(
\frac{\partial u}{\partial x_i}
\frac{\partial u}{\partial x_j}
\right) (x\cdot \nu ) \, dS
-
\frac{1}{2} \int_{\Omega}
\sum_{i,j} \left(
\frac{\partial u}{\partial x_i}
\frac{\partial u}{\partial x_j}
\right) (x\cdot  \nabla a_{ij}) \, dx
$$
$$
-
\frac{1}{2} \int_{\partial\Omega}
\sum_{i,j} a_{ij} \left[
\frac{\partial u}{\partial x_j} \nu_{i} +
\frac{\partial u}{\partial x_i} \nu_{j}
\right] (\nabla u\cdot x) \, dS.
$$
Using the notation defined above, the result follows.


           % Complex Equations from the UW Math Department
% --------------------------------------------------------------------------
% the ACRO package
% 
%   Typeset Acronyms
% 
% --------------------------------------------------------------------------
% Clemens Niederberger
% Web:    https://bitbucket.org/cgnieder/acro/
% E-Mail: contact@mychemistry.eu
% --------------------------------------------------------------------------
% Copyright 2011-2017 Clemens Niederberger
% 
% This work may be distributed and/or modified under the
% conditions of the LaTeX Project Public License, either version 1.3
% of this license or (at your option) any later version.
% The latest version of this license is in
%   http://www.latex-project.org/lppl.txt
% and version 1.3 or later is part of all distributions of LaTeX
% version 2005/12/01 or later.
% 
% This work has the LPPL maintenance status `maintained'.
% 
% The Current Maintainer of this work is Clemens Niederberger.
% --------------------------------------------------------------------------
% The acro package consists of the files
%  - acro.sty, acro_en.tex, acro_en.pdf, README
% --------------------------------------------------------------------------
% If you have any ideas, questions, suggestions or bugs to report, please
% feel free to contact me.
% --------------------------------------------------------------------------
\RequirePackage{expl3,xparse,l3keys2e,xtemplate,etoolbox}
\ProvidesExplPackage
  {acro}
  {2017/01/22}
  {2.7a}
  {Typeset Acronyms}

% --------------------------------------------------------------------------
% warning and error messages:
\msg_new:nnn {acro} {undefined}
  {
    You've~ requested~ acronym~ `#1'~ \msg_line_context: \ but~ you~
    apparently~ haven't~ defined~ it,~ yet! \\
    Maybe~ you've~ misspelled~ `#1'?
  }

\msg_new:nnn {acro} {macro}
  {
    A~ macro~ with~ the~ csname~ `#1'~ already~ exists! \\
    Unless~ you~ set~ acro's~ option~ `strict'~ I~ won't~ redefine~ it~
    \msg_line_context: .
  } 

\msg_new:nnn {acro} {replaced}
  {
    The~ #1~ `#2' ~you ~used~ \msg_line_context: \ is~ deprecated~ and~ has~
    been~ replaced~ by~ `#3'. ~Since~ I~ will~ not~ guarantee~ that~ #1~ `#2'~
    will~ be~ kept~ forever~ I~ strongly~ encourage~ you~ to~ switch!
  }

\msg_new:nnn {acro} {deprecated}
  {
    The~ #1~ `#2'~ you~ used~ \msg_line_context: \ is~ deprecated~and~ there~
    is~ no~ replacement.~ Since~ I~ will~ not~ guarantee~ that~ #1~ `#2'~
    will~ be~ kept~ forever~ I~ strongly~ encourage~ you~ to~ remove~ it~
    from~ your~ document.
  }

\msg_new:nnn {acro} {substitute-short}
  {
    There~ is~ no~ short~ form~ set~ for~ acronym~ `#1'! \\
    I~ am~ setting~ the~ short~ form~ equal~ to~ the~ ID~ `#1'. \\
    If~ that~ is~ not~ what~ you~ want~ make~ sure~ to~ add~ an~ explicit~
    short~ form.
  }
\msg_new:nnn {acro} {ending-exists}
  {
    An~ ending~ with~ the~ name~ `#1'~ already~ exists! \\ \\
    I~ am~ overwriting~ the~ defaults.
  }

\msg_new:nnn {acro} {ending-before-acronyms}
  {
    You~ are~ using~ \token_to_str:N \ProvideAcroEnding \ after~ you've~
    declared~ at~ least~ one~ acronym.~ This~ will~ lead~ to~ trouble! \\
    Make~ sure~ to~ define~ endings~ before~ *any*~ acronym~ declarations!
  }

\msg_new:nnn {acro} {no-alternative}
  {
    There~ is~ no~ alternative~ form~ for~ acronym~ `#1'! \\ \\
    I~ am~ using~ the~ short~ form~ instead.
  }

\msg_new:nnn {acro} {unknown}
  {
    You're~ trying~ to~ use~ the~ #1~ `#2'~ \msg_line_context: . \\
    However,~ I~ do~ not~ know~ #1~ `#2'! \\
    If~ this~ is~ no~ typo~ please~ contact~ the~ package~ author. \\ \\
    I~ am~ going~ to~ use~ the~ #1~ `#3'~ instead.
  }
\msg_new:nnn {acro} {definitions-missing}
  {
    I~ cannot~ find~ the~ file~ \c_acro_definition_file_name_tl
    .\c_acro_definition_file_extension_tl !~ This~ file~ contains~ all~
    essential~ user~ commands~ of~ acro~ and~ is~ a~ crucial~ part~ of~ the~
    package!~ Please~ check~ your~ installation.
  }

% --------------------------------------------------------------------------
% logging:
\cs_new:Npn \acro_if_log:T #1 { \use:n {#1} }

\bool_new:N \l__acro_log_acronyms_bool
\bool_new:N \l__acro_log_acronyms_verbose_bool

\keys_define:nn {acro}
  {
    log           .choice: ,
    log / true    .code:n    =
      \bool_set_true:N \l__acro_log_acronyms_bool
      \bool_set_false:N \l__acro_log_acronyms_verbose_bool ,
    log / silent  .meta:n    = { log = true } ,
    log / verbose .code:n    =
      \bool_set_true:N \l__acro_log_acronyms_bool
      \bool_set_true:N \l__acro_log_acronyms_verbose_bool ,
    log / false   .code:n    =
      \bool_set_false:N \l__acro_log_acronyms_bool
      \bool_set_false:N \l__acro_log_acronyms_verbose_bool ,
    log           .default:n = true ,
    log           .initial:n = false
  }

\cs_new:Npn \__acro_write_log:nn #1#2 { \ \ \ #1 ~ = ~ {#2} }
\cs_new:Npn \__acro_write_log_property:nnn #1#2#3
  { \__acro_write_log:nn {#2} { \__acro_get_property:nn {#3} {#1} } }

\cs_new:Npn \__acro_ending_log_entry:nn #1#2
  {
    | \\
    | \__acro_write_log_property:nnn {#1} {short-#2} {short_#2} \\
    | \__acro_write_log_property:nnn {#1} {short-#2-form} {short_#2_form} \\
    | \__acro_write_log_property:nnn {#1} {long-#2} {long_#2} \\
    | \__acro_write_log_property:nnn {#1} {long-#2-form} {long_#2_form} \\
    | \__acro_write_log_property:nnn {#1} {alt-#2} {alt_#2} \\
    | \__acro_write_log_property:nnn {#1} {alt-#2-form} {alt_#2_form} \\
  }
  
\msg_new:nnn {acro} {log-acronym-verbose}
  {
    ================================================= \\
    | ~ \msg_info_text:n {acro}~ --~ defining~ new~ acronym: \\
    | \__acro_write_log:nn {ID} {#1} \\
    | \__acro_write_log_property:nnn {#1} {short} {short} \\
    | \__acro_write_log_property:nnn {#1}{long} {long} \\
    | \__acro_write_log_property:nnn {#1}{alt} {alt} \\
    | \__acro_write_log_property:nnn {#1}{sort} {sort} \\
    | \__acro_write_log_property:nnn {#1}{class} {class} \\
    | \__acro_write_log_property:nnn {#1} {list} {list} \\
    | \__acro_write_log_property:nnn {#1} {extra} {extra} \\
    | \__acro_write_log_property:nnn {#1} {foreign} {foreign} \\
    | \__acro_write_log_property:nnn {#1} {single} {single} \\
    | \__acro_write_log_property:nnn {#1} {pdfstring} {pdfstring} \\
    | \__acro_write_log_property:nnn {#1} {accsupp} {accsupp} \\
    | \__acro_write_log_property:nnn {#1} {tooltip} {tooltip} \\
    | \\
    | \__acro_write_log_property:nnn {#1} {short-indefinite} {short_indefinite} \\
    | \__acro_write_log_property:nnn {#1} {long-indefinite} {long_indefinite} \\
    | \__acro_write_log_property:nnn {#1} {alt-indefinite} {alt_indefinite} \\
    \seq_map_function:NN \l__acro_endings_seq \__acro_ending_log_entry:n
    | \\
    | \__acro_write_log_property:nnn {#1} {short-format} {short_format} \\
    | \__acro_write_log_property:nnn {#1} {long-format} {long_format} \\
    | \__acro_write_log_property:nnn {#1} {first-long-format} {first_long_format} \\
    | \__acro_write_log_property:nnn {#1} {single-format} {single_format} \\
    | \__acro_write_log_property:nnn {#1} {foreign-lang} {foreign_lang} \\    
    | \\
    | \__acro_write_log_property:nnn {#1} {cite} {citation} \\
    | \__acro_write_log_property:nnn {#1} {index} {index} \\
    | \__acro_write_log_property:nnn {#1} {index-sort} {index_sort} \\
    | \\
    | \__acro_write_log_property:nnn {#1} {long-pre} {long_pre} \\
    | \__acro_write_log_property:nnn {#1} {long-post} {long_post} \\    
    | \__acro_write_log_property:nnn {#1} {index-cmd} {index_cmd} \\
    | \__acro_write_log_property:nnn {#1} {first-style} {first_style} \\
    =================================================
  }

\msg_new:nnn {acro} {log-acronym-silent}
  {
    ================================================= \\
    | ~ \msg_info_text:n {acro}~ --~ defining~ new~ acronym: \\
    | \__acro_write_log:nn {ID} {#1} \\
    | \__acro_write_log_property:nnn {#1} {short} {short} \\
    | \__acro_write_log_property:nnn {#1} {long} {long} \\
    | \__acro_write_log_property:nnn {#1} {alt} {alt} \\
    | \__acro_write_log_property:nnn {#1} {sort} {sort} \\
    | \__acro_write_log_property:nnn {#1} {class} {class} \\
    | \__acro_write_log_property:nnn {#1} {list} {list} \\
    | \__acro_write_log_property:nnn {#1} {extra} {extra} \\
    | \__acro_write_log_property:nnn {#1} {foreign} {foreign} \\
    | \__acro_write_log_property:nnn {#1} {pdfstring} {pdfstring} \\
    | \__acro_write_log_property:nnn {#1} {cite} {citation} \\
    =================================================
  }
  
\cs_new_protected:Npn \__acro_log_acronym:n #1
  {
    \bool_if:NT \l__acro_log_acronyms_bool
      {
        \cs_set:Npn \__acro_ending_log_entry:n ##1
          { \__acro_ending_log_entry:nn {#1} {##1} }
        \bool_if:NTF \l__acro_log_acronyms_verbose_bool
          { \msg_log:nnn {acro} {log-acronym-verbose} {#1} }
          { \msg_log:nnn {acro} {log-acronym-silent} {#1} }
      }
  }   

% --------------------------------------------------------------------------
% message macros:
\cs_new:Npn \__acro_remove_backslash:N #1
  { \exp_after:wN \use_none:n \token_to_str:N #1 }

\cs_new_protected:Npn \acro_new_message_commands:Nnnn #1#2#3#4
  {
    \clist_map_inline:nn {#2}
      {
        \cs_new_protected:cpn { \__acro_remove_backslash:N #1 ##1 }
          {
            \bool_if:NTF \l__acro_silence_bool
              { \use:c { \__acro_remove_backslash:N #3 n##1 } {acro} }
              { \use:c { \__acro_remove_backslash:N #4 n##1 } {acro} }
          }
      }
  }

\acro_new_message_commands:Nnnn \acro_serious_message: {n,nn,nnn}
  { \msg_warning: }
  { \msg_error: }

\acro_new_message_commands:Nnnn \acro_harmless_message: {n,nn,nnn,nnnn}
  { \msg_info: }
  { \msg_warning: }

\cs_new_protected:Npn \acro_option_deprecated:nn #1#2
  {
    \tl_if_blank:nTF {#2}
      { \acro_harmless_message:nnn  {deprecated} {option} {#1} }
      { \acro_harmless_message:nnnn {replaced}   {option} {#1} {#2} }
  }
\cs_new_protected:Npn \acro_option_deprecated:n #1
  { \acro_option_deprecated:nn {#1} {} }

\cs_new_protected:Npn \acro_command_deprecated:NN #1#2
  {
    \tl_if_blank:nTF {#2}
      {
        \acro_harmless_message:nnn {deprecated} {command}
          { \token_to_str:N #1 }
      }
      {
        \acro_harmless_message:nnnn {replaced} {command}
          { \token_to_str:N #1 }
          { \token_to_str:N #2 }
      }
  }

% --------------------------------------------------------------------------
% temporary variables
\tl_new:N   \l__acro_tmpa_tl
\tl_new:N   \l__acro_tmpb_tl
\tl_new:N   \l__acro_tmpc_tl
\prop_new:N \l__acro_tmpa_prop
\prop_new:N \l__acro_tmpb_prop
\seq_new:N  \l__acro_tmpa_seq
\seq_new:N  \l__acro_tmpb_seq
\int_new:N  \l__acro_tmpa_int
\int_new:N  \l__acro_tmpb_int
\int_new:N  \l__acro_tmpc_int
\int_new:N  \l__acro_tmpd_int

% --------------------------------------------------------------------------
% variants of kernel commands
\cs_generate_variant:Nn \quark_if_no_value:nTF  { V }
\cs_generate_variant:Nn \tl_put_right:Nn        { NV, Nv }
\cs_generate_variant:Nn \tl_if_eq:nnT           { V }
\cs_generate_variant:Nn \tl_if_eq:nnF           { V }
\cs_generate_variant:Nn \seq_use:Nnnn           { c }
\cs_generate_variant:Nn \seq_gset_split:Nnn     { c }
\cs_generate_variant:Nn \seq_set_split:Nnn      { NnV }
\cs_generate_variant:Nn \seq_if_in:NnT          { NV }
\cs_generate_variant:Nn \prop_put:Nnn           { NnV, cnx, cnv }
\cs_generate_variant:Nn \prop_get:NnNTF         { cnc }
\cs_generate_variant:Nn \prop_get:NnNF          { cn, cnc }
\cs_generate_variant:Nn \prop_get:NnN           { cnc }
\cs_generate_variant:Nn \cs_generate_variant:Nn { c }
\cs_generate_variant:Nn \str_case:nn            { V }

% --------------------------------------------------------------------------
% variables:
\bool_new:N      \l__acro_silence_bool
\bool_new:N      \l__acro_mark_as_used_bool
\bool_new:N      \g__acro_mark_first_as_used_bool
\bool_new:N      \l__acro_use_single_bool
\bool_new:N      \l__acro_print_only_used_bool
\bool_set_true:N \l__acro_print_only_used_bool
\bool_new:N      \l__acro_hyperref_loaded_bool
\bool_new:N      \l__acro_use_hyperref_bool
\bool_new:N      \l__acro_xspace_bool
\bool_new:N      \l__acro_custom_format_bool
\bool_new:N      \l__acro_strict_bool
\bool_new:N      \l__acro_create_macros_bool
\bool_new:N      \l__acro_first_upper_bool
\bool_new:N      \l__acro_indefinite_bool
\bool_new:N      \l__acro_upper_indefinite_bool
\bool_new:N      \l__acro_foreign_bool
\bool_set_true:N \l__acro_foreign_bool
\bool_new:N      \l__acro_sort_bool
\bool_set_true:N \l__acro_sort_bool
\bool_new:N      \l__acro_capitalize_list_bool
\bool_new:N      \l__acro_citation_all_bool
\bool_new:N      \l__acro_citation_first_bool
\bool_set_true:N \l__acro_citation_first_bool
\bool_new:N      \l__acro_group_citation_bool
\bool_new:N      \l__acro_acc_supp_bool
\bool_new:N      \l__acro_tooltip_bool
\bool_new:N      \l__acro_inside_tooltip_bool
\bool_new:N      \l__acro_following_page_bool
\bool_new:N      \l__acro_following_pages_bool
\bool_new:N      \l__acro_addto_index_bool
\bool_new:N      \l__acro_is_excluded_bool
\bool_new:N      \l__acro_is_included_bool
\bool_new:N      \l__acro_page_punct_bool
\bool_new:N      \l__acro_page_brackets_bool
\bool_new:N      \l__acro_page_display_bool
\bool_new:N      \l__acro_new_page_numbering_bool
\bool_new:N      \l__acro_first_use_brackets_bool
\bool_new:N      \l__acro_first_only_short_bool
\bool_new:N      \l__acro_first_only_long_bool
\bool_new:N      \l__acro_first_reversed_bool
\bool_new:N      \l__acro_first_switched_bool
\bool_new:N      \l__acro_use_note_bool
\bool_new:N      \l__acro_extra_punct_bool
\bool_new:N      \l__acro_extra_use_brackets_bool
\bool_new:N      \l__acro_in_list_bool
\bool_new:N      \l__acro_place_label_bool
\bool_new:N      \l__acro_list_all_pages_bool
\bool_set_true:N \l__acro_list_all_pages_bool
\bool_new:N      \g__acro_first_acronym_declared_bool
\bool_new:N      \l__acro_include_endings_format_bool
\bool_new:N      \l__acro_list_reverse_long_extra_bool
\bool_new:N      \l__acro_use_acronyms_bool
\bool_set_true:N \l__acro_use_acronyms_bool

\tl_new:N   \l__acro_ignore_tl
\tl_new:N   \l__acro_default_indefinite_tl
\tl_set:Nn  \l__acro_default_indefinite_tl {a}
\tl_new:N   \l__acro_foreign_sep_tl
\tl_new:N   \l__acro_extra_instance_tl
\tl_set:Nn  \l__acro_extra_instance_tl {default}
\tl_new:N   \l__acro_page_instance_tl
\tl_set:Nn  \l__acro_page_instance_tl  {none}
\tl_new:N   \l__acro_page_name_tl
\tl_new:N   \l__acro_pages_name_tl
\tl_new:N   \l__acro_next_page_tl
\tl_new:N   \l__acro_next_pages_tl
\tl_new:N   \l__acro_list_instance_tl
\tl_set:Nn  \l__acro_list_instance_tl  {description}
\tl_new:N   \l__acro_list_type_tl
% \tl_new:N   \l__acro_list_tl
\tl_new:N   \l__acro_list_heading_cmd_tl
\tl_set:Nn  \l__acro_list_heading_cmd_tl {section*}
\tl_new:N   \l__acro_list_name_tl
\tl_new:N   \l__acro_list_before_tl
\tl_new:N   \l__acro_list_after_tl
\tl_new:N   \l__acro_custom_format_tl
\tl_new:N   \l__acro_first_between_tl
\tl_new:N   \l__acro_citation_connect_tl
\tl_new:N   \l__acro_between_group_connect_citation_tl
\tl_new:N   \l__acro_extra_brackets_tl
\tl_new:N   \l__acro_extra_punct_tl
\tl_new:N   \l__acro_first_brackets_tl
\tl_new:N   \l__acro_page_punct_tl
\tl_new:N   \l__acro_page_brackets_tl
\tl_new:N   \l__acro_last_page_tl
\tl_new:N   \l__acro_current_page_tl
\tl_new:N   \l__acro_list_table_tl
\tl_new:N   \l__acro_list_table_spec_tl
\tl_new:N   \l__acro_acc_supp_tl
\tl_new:N   \l__acro_acc_supp_options_tl
\tl_new:N   \l__acro_label_prefix_tl
\tl_set:Nn  \l__acro_label_prefix_tl { ac: }
\tl_new:N   \l__acro_index_short_tl
\tl_new:N   \l__acro_first_instance_tl
\tl_set:Nn  \l__acro_first_instance_tl {default}

\tl_new:N   \l__acro_short_tl
\tl_new:N   \l__acro_short_format_tl
\tl_new:N   \l__acro_list_short_format_tl

\tl_new:N   \l__acro_alt_tl
\tl_new:N   \l__acro_alt_format_tl

\tl_new:N   \l__acro_long_tl
\tl_new:N   \l__acro_list_long_format_tl

\tl_new:N   \l__acro_single_form_tl
\tl_set:Nn  \l__acro_single_form_tl {long}

\tl_new:N   \l__acro_extra_format_tl

\tl_new:N   \l__acro_foreign_format_tl
\tl_new:N   \l__acro_foreign_list_format_tl
\tl_set:Nn  \l__acro_foreign_list_format_tl { \acroenparen }

\tl_new:N   \l__acro_index_format_tl


\skip_new:N  \l__acro_page_space_skip

\dim_new:N  \l__acro_short_width_dim
\dim_set:Nn \l__acro_short_width_dim {3em}

\prop_new:N \l__acro_list_styles_prop
\prop_new:N \l__acro_list_headings_prop
\prop_new:N \l__acro_first_styles_prop
\prop_new:N \l__acro_extra_styles_prop
\prop_new:N \l__acro_page_styles_prop

% --------------------------------------------------------------------------
% small commands for use at various places
\cs_new:Npn \acro_no_break: { \tex_penalty:D \c_ten_thousand }

\cs_new_protected:Npn \__acro_first_upper_case:n #1
  { \tl_upper_case:n { \tl_head:n {#1} } \tl_tail:n {#1} }
\cs_generate_variant:Nn \__acro_first_upper_case:n { x }
\cs_generate_variant:Nn \tl_mixed_case:n { x }

\cs_new_eq:NN \acro_first_upper_case:n \__acro_first_upper_case:n

\NewDocumentCommand \acfirstupper { m }
  { \acro_first_upper_case:n {#1} }

% --------------------------------------------------------------------------
% options:
\keys_define:nn {acro}
  {
    messages          .choice: ,
    messages / silent .code:n     =
      \bool_set_true:N \l__acro_silence_bool ,
    messages / loud .code:n       =
      \bool_set_false:N \l__acro_silence_bool ,
    messages          .value_required:n = true ,
    accsupp           .bool_set:N = \l__acro_acc_supp_bool ,
    accsupp-options   .tl_set:N   = \l__acro_acc_supp_options_tl ,
    tooltip           .bool_set:N = \l__acro_tooltip_bool ,
    tooltip-cmd       .code:n     = \cs_set:Npn \__acro_tooltip_cmd:nn {#1} ,
    tooltip-cmd       .value_required:n = true ,
    macros            .bool_set:N = \l__acro_create_macros_bool ,
    xspace            .bool_set:N = \l__acro_xspace_bool ,
    % xspace            .code:n     = \acro_option_deprecated:nn {xspace} {} ,
    strict            .bool_set:N = \l__acro_strict_bool ,
    sort              .bool_set:N = \l__acro_sort_bool ,
    short-format      .code:n     =
      \tl_set:Nn \l__acro_short_format_tl {#1}
      \tl_set_eq:NN \l__acro_alt_format_tl \l__acro_short_format_tl
      \tl_set:Nn \l__acro_list_short_format_tl {#1} ,
    short-format      .value_required:n = true ,
    long-format       .code:n     =
      \tl_set:Nn \l__acro_long_format_tl {#1}
      \tl_set:Nn \l__acro_first_long_format_tl {#1}
      \tl_set:Nn \l__acro_list_long_format_tl {#1} ,
    long-format       .value_required:n = true ,
    first-long-format .code:n     =
      \tl_set:Nn \l__acro_first_long_format_tl {#1} ,
    first-long-format .value_required:n = true ,
    single-format     .tl_set:N   = \l__acro_single_format_tl ,
    single-format     .value_required:n = true ,
    format-include-endings .bool_set:N = \l__acro_include_endings_format_bool ,
    display-foreign   .bool_set:N = \l__acro_foreign_bool ,
    foreign-format    .tl_set:N   = \l__acro_foreign_format_tl ,
    foreign-format    .value_required:n = true ,
    list-short-format .tl_set:N   = \l__acro_list_short_format_tl ,
    list-short-format .value_required:n = true ,
    list-short-width  .dim_set:N  = \l__acro_short_width_dim ,
    list-short-width  .value_required:n = true ,
    list-long-format  .tl_set:N   = \l__acro_list_long_format_tl ,
    list-long-format  .value_required:n = true ,
    list-foreign-format .tl_set:N = \l__acro_foreign_list_format_tl ,
    list-foreign-format .value_required:n = true ,
    extra-format      .tl_set:N   = \l__acro_extra_format_tl ,
    extra-format      .value_required:n = true ,
    single            .bool_set:N = \l__acro_use_single_bool ,
    single-form       .tl_set:N   = \l__acro_single_form_tl ,
    single-form       .value_required:n = true ,
    first-style       .code:n     = \acro_set_first_style:n {#1} ,
    first-style       .value_required:n = true ,
    extra-style       .code:n     = \acro_set_extra_style:n {#1} ,
    extra-style       .value_required:n = true ,
    label             .bool_set:N = \l__acro_place_label_bool ,
    label-prefix      .tl_set:N   = \l__acro_label_prefix_tl ,
    label-prefix      .value_required:n = true ,
    pages             .choice: ,
    pages / all       .code:n     =
      \bool_set_true:N \l__acro_list_all_pages_bool ,
    pages / first     .code:n     =
      \bool_set_true:N \l__acro_place_label_bool
      \bool_set_false:N \l__acro_list_all_pages_bool ,
    pages             .value_required:n = true ,
    page-ref          .code:n     =
      \acro_option_deprecated:nn {page-ref} {page-style}
      \acro_set_page_style:n {#1} ,
    page-style        .code:n     = \acro_set_page_style:n {#1} ,
    page-style        .value_required:n = true ,
    page-name         .tl_set:N   = \l__acro_page_name_tl ,
    page-name         .value_required:n = true ,
    pages-name        .tl_set:N   = \l__acro_pages_name_tl ,
    pages-name        .value_required:n = true ,
    following-page    .bool_set:N = \l__acro_following_page_bool ,
    following-pages   .bool_set:N = \l__acro_following_pages_bool ,
    following-pages*  .meta:n     =
      { following-page = #1 , following-pages = #1 } ,
    following-pages*  .default:n  = true ,
    next-page         .tl_set:N   = \l__acro_next_page_tl ,
    next-page         .value_required:n = true ,
    next-pages        .tl_set:N   = \l__acro_next_pages_tl ,
    next-pages        .value_required:n = true ,
    list-style        .code:n     = \acro_set_list_style:n {#1} ,
    list-style        .value_required:n = true ,
    list-heading      .code:n     = \acro_set_list_heading:n {#1} ,
    list-heading      .value_required:n = true ,
    list-name         .tl_set:N   = \l__acro_list_name_tl ,
    list-name         .value_required:n = true ,
    hyperref          .bool_set:N = \l__acro_use_hyperref_bool ,
    only-used         .bool_set:N = \l__acro_print_only_used_bool ,
    mark-as-used      .choice: ,
    mark-as-used / first .code:n  =
      \bool_gset_true:N \g__acro_mark_first_as_used_bool ,
    mark-as-used / any   .code:n  =
      \bool_gset_false:N \g__acro_mark_first_as_used_bool ,
    mark-as-used      .default:n  = any ,
    list-caps         .bool_set:N = \l__acro_capitalize_list_bool ,
    cite              .choice: ,
    cite / all        .code:n     =
      \bool_set_true:N \l__acro_citation_all_bool
      \bool_set_true:N \l__acro_citation_first_bool ,
    cite / none       .code:n     =
      \bool_set_false:N \l__acro_citation_all_bool
      \bool_set_false:N \l__acro_citation_first_bool ,
    cite / first      .code:n     =
      \bool_set_false:N \l__acro_citation_all_bool
      \bool_set_true:N  \l__acro_citation_first_bool ,
    cite              .default:n  = all ,
    cite-cmd          .code:n     =
      \cs_set:Npn \__acro_citation_cmd:w {#1} ,
    cite-cmd          .value_required:n = true ,
    group-cite-cmd    .code:n     =
      \cs_set:Npn \__acro_group_citation_cmd:w {#1} ,
    group-cite-cmd    .value_required:n = true ,
    group-citation    .bool_set:N = \l__acro_group_citation_bool ,
    cite-connect      .tl_set:N   = \l__acro_citation_connect_tl ,
    cite-connect      .initial:n  = \nobreakspace ,
    cite-connect      .value_required:n = true ,
    group-cite-connect .tl_set:N = \l__acro_between_group_connect_citation_tl ,
    group-cite-connect .initial:n = {,\nobreakspace} ,
    group-cite-connect .value_required:n = true ,
    index             .bool_set:N = \l__acro_addto_index_bool ,
    index-cmd         .code:n     =
      \cs_set:Npn \__acro_index_cmd:n {#1} ,
    index-cmd         .value_required:n = true ,
    uc-cmd            .code:n     =
      \cs_set_eq:NN \__acro_first_upper_case:n #1 ,
    uc-cmd            .value_required:n = true
  }

\AtBeginDocument
  {
    \bool_if:NTF \l__acro_xspace_bool
      {
        \@ifpackageloaded {xspace}
          { }
          { \RequirePackage {xspace} }
        \cs_new_eq:NN \acro_xspace: \xspace
      }
      { \cs_new:Npn \acro_xspace: {} }
  }

% --------------------------------------------------------------------------
% setup command:
\NewDocumentCommand \acsetup { m }
  { \keys_set:nn {acro} {#1} \ignorespaces }

% --------------------------------------------------------------------------
% we use xtemplate for different object types and with a different number of
% arguments; let's declare functions for usage later so we don't have to
% bother

% objects with one argument:
\cs_new_protected:Npn \acro_page_number_instance:nn #1#2
  { \UseInstance {acro-page-number} {#1} {#2} }
\cs_generate_variant:Nn \acro_page_number_instance:nn {V}

\cs_new_protected:Npn \acro_extra_instance:nn #1#2
  { \UseInstance {acro-extra} {#1} {#2} }
\cs_generate_variant:Nn \acro_extra_instance:nn {VV}

\cs_new_protected:Npn \acro_title_instance:nn #1#2
  { \UseInstance {acro-title} {#1} {#2} }
\cs_generate_variant:Nn \acro_title_instance:nn {VV}

% objects with two arguments:
\cs_new_protected:Npn \acro_list_instance:nnn #1#2#3
  { \UseInstance {acro-list} {#1} {#2} {#3} }
\cs_generate_variant:Nn \acro_list_instance:nnn {VVV}

\cs_new_protected:Npn \acro_first_instance:nn #1#2
  {
    \tl_if_blank:VF \l__acro_first_style_tl
      {
        \tl_set_eq:NN
          \l__acro_first_instance_tl
          \l__acro_first_style_tl
      }
    \acro_if_defined:nT {#1}
      {
        \use:x {
          \UseInstance {acro-first}
            { \exp_not:V \l__acro_first_instance_tl }
            { \exp_not:n {#1} }
            { \exp_not:n {#2} }
          }
      }
  }
\cs_generate_variant:Nn \acro_first_instance:nn {nV}
  
% --------------------------------------------------------------------------
% hyperref support
\cs_new_eq:NN \acro_hyper_target:nn \use_ii:nn
\cs_new_eq:NN \acro_hyper_link:nn   \use_ii:nn

\cs_new_protected:Npn \acro_activate_hyperref_support:
  {
    \bool_if:nT { \l__acro_hyperref_loaded_bool && \l__acro_use_hyperref_bool }
      {
        \cs_set_eq:NN \acro_hyper_link:nn \hyperlink
        \cs_set:Npn \acro_hyper_target:nn ##1##2
          { \raisebox { 3ex } [ 0pt ] { \hypertarget {##1} { } } ##2 }
      }
  }

% #1: tl var
% #2: id
% #3: text
\cs_new_protected:Npn \__acro_make_link:Nnn #1#2#3
  {
    \bool_if:nTF
      { \l__acro_use_hyperref_bool && \l__acro_hyperref_loaded_bool }
      {
        \tl_set:Nn #1
           {
             \acro_hyper_link:nn {#2} { \phantom {#3} }
             \acro_if_is_single:nTF {#2}
               { \hbox_overlap_left:n {#3} }
               { \acro_color_link:n { \hbox_overlap_left:n {#3} } }
           }
       }
       { \tl_set:Nn #1 {#3} }
  }
\cs_generate_variant:Nn \__acro_make_link:Nnn {NnV}

\cs_new:Npn \acro_color_link:n #1
  {
    \cs_if_exist:NTF \hypersetup
      {
        \ifHy@colorlinks
          \exp_after:wN \use_i:nn
        \else
          \ifHy@ocgcolorlinks
            \exp_after:wN \use_i:nn
          \else
            \exp_after:wN \exp_after:wN \exp_after:wN \use_ii:nn
          \fi
        \fi
        { \textcolor { \@linkcolor } {#1} }
        {#1}
      }
      {#1}
  }

\AtBeginDocument{
  \cs_if_exist:NF \textcolor { \cs_new_eq:NN \textcolor \use_ii:nn }
}

% --------------------------------------------------------------------------
% output style of the first time an acronym is used

% helper macros for the styles
% #1: short|long
% #2: id
% #3: long
\cs_new_protected:Npn \__acro_print_form_and_indefinite:nnn #1#2#3
  {
    \group_begin:
      \acro_for_all_trailing_tokens_do:n
        { \acro_deactivate_trailing_action:n {##1} }
      \str_case:nn {#1}
        {
          {long} {
            \bool_if:nT
              {
                \l__acro_first_only_long_bool ||
                !\l__acro_first_only_short_bool
              }
              {
                \acro_write_indefinite:nn {#2} {long}
                \acro_write_expanded:nnn {#2} {first-long} {#3}
              }
          }
          {short} {
            \bool_if:nT
              {
                !\l__acro_first_only_long_bool ||
                \l__acro_first_only_short_bool
              }
              {
                \acro_write_indefinite:nn {#2} {short}
                \acro_write_compact:nn {#2} {short}
              }
          }
        }
    \group_end:
  }

\cs_new_protected:Npn \__acro_open_bracket:
  {
    \bool_if:nT
      {
        !\l__acro_first_only_long_bool &&
        !\l__acro_first_only_short_bool
      }
      {
        \acro_space:
        \tl_if_blank:VF \l__acro_first_between_tl
          {
            \tl_use:N \l__acro_first_between_tl
            \acro_space:
          }
        \bool_if:NT \l__acro_first_use_brackets_bool
          { \tl_head:N \l__acro_first_brackets_tl }
      }
  }

\cs_new_protected:Npn \__acro_close_bracket:
  {
    \bool_if:nT
      {
        \l__acro_first_use_brackets_bool &&
        !\l__acro_first_only_short_bool &&
        !\l__acro_first_only_long_bool
      }
      { \tl_tail:N \l__acro_first_brackets_tl }
  }
  
% #1: short|long
% #2: id
% #3: long
\cs_new_protected:Npn \__acro_print_form:nnn #1#2#3
  {
    \str_case:nn {#1}
      {
        {long} {
          \bool_if:nT
            {
              \l__acro_first_only_long_bool ||
              !\l__acro_first_only_short_bool
            }
            { \acro_write_expanded:nnn {#2} {first-long} {#3} }
        }
        {short} {
          \bool_if:nT
            {
              !\l__acro_first_only_long_bool ||
              \l__acro_first_only_short_bool
            }
            { \acro_write_compact:nn {#2} {short} }
        }
      }
  }

% #1: id
\cs_new_protected:Npn \__acro_foreign_sep:n #1
  {
    \bool_if:nT
      {
         \l__acro_foreign_bool &&
        !\l__acro_first_only_short_bool &&
        !\l__acro_first_only_long_bool
      }
      { \acro_if_foreign:nT {#1} { \tl_use:N \l__acro_foreign_sep_tl } }
  }
  
% #1: id
\cs_new_protected:Npn \__acro_print_foreign:n #1
  {
    \bool_if:nT
      {
         \l__acro_foreign_bool &&
        !\l__acro_first_only_short_bool &&
        !\l__acro_first_only_long_bool
      }
      { \acro_write_foreign:n {#1} }
  }

\cs_new_protected:Npn \__acro_print_citation:n #1
  {
    \bool_if:NT \l__acro_group_citation_bool
      { \acro_group_cite:n {#1} }
  }

\cs_new_protected:Npn \__acro_finalize_first:n #1
  {
    \bool_if:NF \l__acro_group_citation_bool
      { \acro_cite_if:nn { \l__acro_citation_first_bool } {#1} }
    \acro_index_if:nn { \l__acro_addto_index_bool } {#1}
  }

% --------------------------------------------------------------------------
% the `acro-first' object, templates, instances:
% #1: id
% #2: long
\DeclareObjectType {acro-first} {2}

% template for inline appearance:
\DeclareTemplateInterface {acro-first} {inline} {2}
  {
    brackets      : boolean   = true  ,
    brackets-type : tokenlist = ()    ,
    only-short    : boolean   = false ,
    only-long     : boolean   = false ,
    reversed      : boolean   = false ,
    between       : tokenlist         ,
    foreign-sep   : tokenlist = {,~}
  }
\DeclareTemplateCode {acro-first} {inline} {2}
  {
    brackets      = \l__acro_first_use_brackets_bool ,
    brackets-type = \l__acro_first_brackets_tl       ,
    only-short    = \l__acro_first_only_short_bool   ,
    only-long     = \l__acro_first_only_long_bool    ,
    reversed      = \l__acro_first_reversed_bool     ,
    between       = \l__acro_first_between_tl        ,
    foreign-sep   = \l__acro_foreign_sep_tl
  }
  {
    \AssignTemplateKeys
    \bool_if:NTF \l__acro_first_reversed_bool
      { % zuerst kurze Form, dann lange Form:
        \__acro_print_form_and_indefinite:nnn {short} {#1} {#2}
        \__acro_open_bracket:
        \__acro_print_foreign:n {#1}
        \__acro_foreign_sep:n {#1}
        \__acro_print_form:nnn {long} {#1} {#2}
        \__acro_print_citation:n {#1}
        \__acro_close_bracket:
        \__acro_finalize_first:n {#1}
      }
      { % zuerst lange Form, dann kurze Form:
        \__acro_print_form_and_indefinite:nnn {long} {#1} {#2}
        \__acro_open_bracket:
        \__acro_print_foreign:n {#1}
        \__acro_foreign_sep:n {#1}
        \__acro_print_form:nnn {short} {#1} {#2}
        \__acro_print_citation:n {#1}
        \__acro_close_bracket:
        \__acro_finalize_first:n {#1}
      }
  }

% template for footnotes, sidenotes, ...
\cs_new:Npn \__acro_note_command:n #1 {#1}
\DeclareTemplateInterface {acro-first} {note} {2}
  {
    use-note     : boolean    = true ,
    note-command : function 1 = \footnote {#1} ,
    foreign-sep  : tokenlist  = {,~} ,
    reversed      : boolean   = false ,
  }

\DeclareTemplateCode {acro-first} {note} {2}
  {
    use-note     = \l__acro_use_note_bool  ,
    note-command = \__acro_note_command:n  ,
    foreign-sep  = \l__acro_foreign_sep_tl ,
    reversed     = \l__acro_first_reversed_bool
  }
  {
    \AssignTemplateKeys
    \bool_if:NTF \l__acro_first_reversed_bool
      { % long in text and short in note
        \__acro_print_form_and_indefinite:nnn {long} {#1} {#2}
        \bool_if:NT \l__acro_use_note_bool
          {
            \__acro_note_command:n
              {
                \__acro_print_foreign:n {#1}
                \__acro_foreign_sep:n {#1}
                \__acro_print_form:nnn {short} {#1} {#2}
                \__acro_print_citation:n {#1}
                \__acro_finalize_first:n {#1}
              }
          }
      }
      { % short in text and long in note
        \__acro_print_form_and_indefinite:nnn {short} {#1} {#2}
        \bool_if:NT \l__acro_use_note_bool
          {
            \__acro_note_command:n
              {
                \__acro_print_foreign:n {#1}
                \__acro_foreign_sep:n {#1}
                \__acro_print_form:nnn {long} {#1} {#2}
                \__acro_print_citation:n {#1}
                \__acro_finalize_first:n {#1}
              }
          }
      }
  }

% --------------------------------------------------------------------------
% declare new first styles:
\cs_new_protected:Npn \acro_declare_first_style:nnn #1#2#3
  {
    \DeclareInstance {acro-first} {#1} {#2} {#3}
    \prop_put:Nnn \l__acro_first_styles_prop  {#1} {#2}
  }

% #1: name
% #2: template
% #3: settings
\NewDocumentCommand \DeclareAcroFirstStyle {mmm}
  { \acro_declare_first_style:nnn {#1} {#2} {#3} }

% set a list style
\cs_new_protected:Npn \acro_set_first_style:n #1
  {
    \prop_if_in:NnTF \l__acro_first_styles_prop {#1}
      { \__acro_set_first_style:n {#1} }
      {
        \msg_warning:nnnnn {acro} {unknown}
          {first~ style}
          {#1}
          {default}
        \__acro_set_first_style:n {default}
      }
  }

\cs_new_protected:Npn \__acro_set_first_style:n #1
  {
    \tl_set:Nn \l__acro_first_instance_tl {#1}
    \prop_get:NnN \l__acro_first_styles_prop {#1} \l__acro_tmpa_tl
  }

% --------------------------------------------------------------------------
% formatting the extras information:
\DeclareObjectType {acro-extra} {1}

\DeclareTemplateInterface {acro-extra} {inline} {1}
  {
    punct         : boolean   = false ,
    punct-symbol  : tokenlist = {,}   ,
    brackets      : boolean   = true  ,
    brackets-type : tokenlist = ()
  }

\DeclareTemplateCode {acro-extra} {inline} {1}
  {
    punct         = \l__acro_extra_punct_bool        ,
    punct-symbol  = \l__acro_extra_punct_tl          ,
    brackets      = \l__acro_extra_use_brackets_bool ,
    brackets-type = \l__acro_extra_brackets_tl
  }
  {
    \AssignTemplateKeys
    \bool_if:NT \l__acro_extra_punct_bool
      { \tl_use:N \l__acro_extra_punct_tl \tl_use:N \c_space_tl }
    \bool_if:NT \l__acro_extra_use_brackets_bool
      { \tl_head:N \l__acro_extra_brackets_tl }
    \acro_write_long:Vn \l__acro_extra_format_tl {#1}
    \bool_if:NT \l__acro_extra_use_brackets_bool
      { \tl_tail:N \l__acro_extra_brackets_tl }
  }

% declare new extra styles:
\cs_new_protected:Npn \acro_declare_etxra_style:nnn #1#2#3
  {
    \DeclareInstance {acro-etxra} {#1} {#2} {#3}
    \prop_put:Nnn \l__acro_etxra_styles_prop  {#1} {#2}
  }

% #1: name
% #2: template
% #3: settings
\NewDocumentCommand \DeclareAcroExtraStyle {mmm}
  { \acro_declare_extra_style:nnn {#1} {#2} {#3} }

% set an extra style
\cs_new_protected:Npn \acro_set_extra_style:n #1
  {
    \prop_if_in:NnTF \l__acro_extra_styles_prop {#1}
      { \__acro_set_extra_style:n {#1} }
      {
        \msg_warning:nnnnn {acro} {unknown}
          {extra~ style}
          {#1}
          {default}
        \__acro_set_extra_style:n {default}
      }
  }

\cs_new_protected:Npn \__acro_set_extra_style:n #1
  {
    \tl_set:Nn \l__acro_extra_instance_tl {#1}
    \prop_get:NnN \l__acro_extra_styles_prop {#1} \l__acro_tmpa_tl
  }

\cs_new_protected:Npn \acro_declare_extra_style:nnn #1#2#3
  {
    \DeclareInstance {acro-extra} {#1} {#2} {#3}
    \prop_put:Nnn \l__acro_extra_styles_prop  {#1} {#2}
  }

% --------------------------------------------------------------------------
% outputting the page numbers:
\RequirePackage {zref-abspage}

\cs_new_protected:Npn \__acro_create_page_records:n #1
  {
    \seq_new:c { g__acro_#1_pages_seq }
    \tl_new:c  { g__acro_#1_recorded_pages_tl }
  }

\cs_new_protected:Npn \acro_hyper_page:n #1 { \use:n {#1} }

\cs_new:Npn \acro_get_thepage:nnn #1#2#3 { \acro_hyper_page:n {#1} }
\cs_new:Npn \acro_get_thepage_from:N #1
  { \exp_after:wN \acro_get_thepage:nnn #1 }

\cs_new:Npn \acro_get_page_number:nnn #1#2#3 {#2}
\cs_new:Npn \acro_get_page_number_from:N #1
  { \exp_after:wN \acro_get_page_number:nnn #1 }

\cs_new:Npn \acro_get_abspage:nnn #1#2#3 {#3}
\cs_new:Npn \acro_get_abspage_from:N #1
  { \exp_after:wN \acro_get_abspage:nnn #1 }

\cs_new:Npn \acro_page_range_comma: {}

\cs_new_protected:Npn \acro_print_page_numbers:n #1
  {
    \seq_if_empty:cF {g__acro_#1_pages_seq}
      {
        \bool_if:NTF \l__acro_list_all_pages_bool
          {
            % have the numbers changed?
            \tl_set:Nx \l__acro_tmpa_tl
              { \seq_use:cn {g__acro_#1_pages_seq} {|} }
            \tl_if_eq:cNF {g__acro_#1_recorded_pages_tl} \l__acro_tmpa_tl
              {
                \@latex@warning@no@line
                  {Rerun~to~get~page~numbers~of~acronym~#1~in~acronym~list~right}
              }
            \tl_clear:N \l__acro_write_pages_tl
            \tl_clear:N \l__acro_last_page_tl
            \tl_clear:N \l__acro_current_page_tl
            \seq_set_eq:Nc \l__acro_tmpb_seq { g__acro_#1_pages_seq }
            \seq_remove_duplicates:N \l__acro_tmpb_seq
            \seq_clear:N \l__acro_tmpa_seq
            \cs_set_protected:Npn \acro_page_range_comma:
              { \cs_set:Npn \acro_page_range_comma: { ,~ } }
            % get the numbers:
            \int_compare:nNnTF { \seq_count:N \l__acro_tmpb_seq } = { 1 }
              {
                \tl_use:N \l__acro_page_name_tl
                \seq_get_right:cN { g__acro_#1_pages_seq } \l__acro_tmpa_tl
                \acro_get_thepage_from:N \l__acro_tmpa_tl
              }
              {
                \tl_use:N \l__acro_pages_name_tl
                \seq_map_inline:cn { g__acro_#1_pages_seq }
                  {
                    \tl_if_blank:VTF \l__acro_last_page_tl
                      {% we're at the beginning
                        \seq_put_right:Nn \l__acro_tmpa_seq {##1}
                        \tl_set:Nn \l__acro_last_page_tl {##1}
                      }
                      {% we'at least at the second page
                         % current page:
                         \tl_set:Nn  \l__acro_current_page_tl {##1}
                         % last page:
                         \seq_get_right:NN \l__acro_tmpa_seq \l__acro_last_page_tl
                         \tl_if_eq:NNTF \l__acro_current_page_tl \l__acro_last_page_tl
                           {% there were more than one appearance on the current page
                             \seq_put_right:Nn \l__acro_tmpa_seq {##1}
                           }
                           {% new page
                             \acro_determine_page_ranges:NNn
                               \l__acro_tmpa_seq
                               \l__acro_write_pages_tl
                               {##1}
                           }
                      }
                  }
                \seq_if_empty:NF \l__acro_tmpa_seq
                  {
                    \acro_determine_page_ranges:NNV
                      \l__acro_tmpa_seq
                      \l__acro_write_pages_tl
                      \l__acro_current_page_tl
                  }
                \tl_use:N \l__acro_write_pages_tl
                \tl_clear:N \l__acro_write_pages_tl
              }
          }
          {
            \tl_use:N \l__acro_page_name_tl
            \pageref{\l__acro_label_prefix_tl #1}
          }
      }
    \seq_clear:N \l__acro_tmpa_seq
    \seq_clear:N \l__acro_tmpb_seq
  }

\cs_new:Npn \acro_determine_page_ranges:NNn #1#2#3
  {
    \seq_remove_duplicates:N #1
    % current page:
    \int_set:Nn \l__acro_tmpa_int { \acro_get_abspage:nnn #3 }
    \int_set:Nn \l__acro_tmpb_int { \acro_get_page_number:nnn #3 }
    % last page:
    \seq_get_right:NN #1 \l__acro_last_page_tl
    \int_set:Nn \l__acro_tmpc_int
      { \acro_get_abspage_from:N \l__acro_last_page_tl }
    \int_set:Nn \l__acro_tmpd_int
      { \acro_get_page_number_from:N \l__acro_last_page_tl }
    \bool_if:nTF
      {
        \int_compare_p:nNn
          { \l__acro_tmpa_int - \l__acro_tmpc_int }
           =
          { \l__acro_tmpb_int - \l__acro_tmpd_int }
        &&
        \int_compare_p:nNn
        { \l__acro_tmpb_int - \l__acro_tmpd_int } = {1}
      }
      {% same kind of page numbering, one page ahead
       % => possible range
         \seq_put_right:Nn #1 {#3}
      }
      {% any possible range ended
        \tl_put_right:Nn #2 { \acro_page_range_comma: }
        \int_compare:nNnTF
          { \seq_count:N #1 } > {2}
          {% real range
            \seq_get_left:NN #1 \l__acro_tmpa_tl
            \tl_put_right:Nx #2 { \acro_get_thepage_from:N \l__acro_tmpa_tl }
            \bool_if:NTF \l__acro_following_pages_bool
              { \tl_put_right:Nn #2 { \l__acro_next_pages_tl } }
              {
                \tl_put_right:Nn #2 { -- }
                \seq_get_right:NN #1 \l__acro_tmpa_tl
                \tl_put_right:Nx #2 { \acro_get_thepage_from:N \l__acro_tmpa_tl }
              }
          }
          {
            \int_compare:nNnTF
              { \seq_count:N #1 } = {2}
              {% range of two pages
                \seq_get_left:NN #1 \l__acro_tmpa_tl
                \tl_put_right:Nx #2 { \acro_get_thepage_from:N \l__acro_tmpa_tl }
                \bool_if:NTF \l__acro_following_page_bool
                  { \tl_put_right:Nn #2 { \l__acro_next_page_tl } }
                  {
                    \tl_put_right:Nn #2 { ,~ }
                    \seq_get_right:NN #1 \l__acro_tmpa_tl
                    \tl_put_right:Nx #2 { \acro_get_thepage_from:N \l__acro_tmpa_tl }
                  }
              }
              {% no range at all
                \seq_get_right:NN #1 \l__acro_tmpa_tl
                \tl_put_right:Nx #2 { \acro_get_thepage_from:N \l__acro_tmpa_tl }
              }
          }
        \seq_clear:N #1
        \seq_put_right:Nn #1 {#3}
      }
  }
\cs_generate_variant:Nn \acro_determine_page_ranges:NNn { NNV }

% --------------------------------------------------------------------------
\DeclareObjectType {acro-page-number} {1}

\DeclareTemplateInterface {acro-page-number} {inline} {1}
  {
    display       : boolean   = true  ,
    punct         : boolean   = false ,
    punct-symbol  : tokenlist = {,}   ,
    brackets      : boolean   = false ,
    brackets-type : tokenlist = ()    ,
    space         : skip      = .333333em plus .166666em minus .111111em
  }

\DeclareTemplateCode {acro-page-number} {inline} {1}
  {
    display       = \l__acro_page_display_bool  ,
    punct         = \l__acro_page_punct_bool    ,
    punct-symbol  = \l__acro_page_punct_tl      ,
    brackets      = \l__acro_page_brackets_bool ,
    brackets-type = \l__acro_page_brackets_tl   ,
    space         = \l__acro_page_space_skip
  }
  {
    \AssignTemplateKeys
    \bool_if:NT \l__acro_page_display_bool
      {
        \bool_if:NT \l__acro_page_punct_bool
          { \tl_use:N \l__acro_page_punct_tl }
        % \tl_use:N \c_space_tl
        \dim_compare:nNnF { \l__acro_page_space_skip } = { 0pt }
          { \skip_horizontal:N \l__acro_page_space_skip }
        \bool_if:NT \l__acro_page_brackets_bool
          { \tl_head:N \l__acro_page_brackets_tl }
        \acro_print_page_numbers:n {#1}
        \bool_if:NT \l__acro_page_brackets_bool
          { \tl_tail:N \l__acro_page_brackets_tl }
      }
  }

% declare new page styles:
\cs_new_protected:Npn \acro_declare_page_style:nnn #1#2#3
  {
    \DeclareInstance {acro-page-number} {#1} {#2} {#3}
    \prop_put:Nnn \l__acro_page_styles_prop  {#1} {#2}
  }

% #1: name
% #2: template
% #3: settings
\NewDocumentCommand \DeclareAcroPageStyle {mmm}
  { \acro_declare_page_style:nnn {#1} {#2} {#3} }

% set a page style
\cs_new_protected:Npn \acro_set_page_style:n #1
  {
    \prop_if_in:NnTF \l__acro_page_styles_prop {#1}
      { \__acro_set_page_style:n {#1} }
      {
        \msg_warning:nnnnn {acro} {unknown}
          {page~ style}
          {#1}
          {none}
        \__acro_set_page_style:n {none}
      }
  }

\cs_new_protected:Npn \__acro_set_page_style:n #1
  {
    \tl_set:Nn \l__acro_page_instance_tl {#1}
    \prop_get:NnN \l__acro_page_styles_prop {#1} \l__acro_tmpa_tl
  }

% --------------------------------------------------------------------------
% the title of the list:
\cs_new:Npn \acro_list_title_format:n #1 {#1}

\DeclareObjectType {acro-title} {1}

\DeclareTemplateInterface {acro-title} {sectioning} {1}
  { name-format : function 1 = #1 }

\DeclareTemplateCode {acro-title} {sectioning} {1}
  { name-format = \acro_list_title_format:n }
  {
    \AssignTemplateKeys
    \acro_list_title_format:n {#1}
  }

% set a list heading:
\cs_new_protected:Npn \acro_set_list_heading:n #1
  {
    \prop_if_in:NnTF \l__acro_list_headings_prop {#1}
      { \__acro_set_list_heading:n {#1} }
      {
        \msg_warning:nnnnn {acro} {unknown}
          {list~ heading}
          {#1}
          {section*}
        \__acro_set_list_heading:n {section*}
      }
  }

\cs_new_protected:Npn \__acro_set_list_heading:n #1
  {
    \tl_set:Nn \l__acro_list_heading_cmd_tl {#1}
    % \prop_get:NnN \l__acro_list_headings_prop
    %   {#1}
    %   \l__acro_list_heading_cmd_tl
  }
  
\cs_new_protected:Npn \acro_declare_list_heading:nn #1#2
  {
    \prop_put:Nnn \l__acro_list_headings_prop {#1} {#2}
    \DeclareInstance {acro-title} {#1} {sectioning}
      { name-format = #2 {##1} }
  }

\NewDocumentCommand \DeclareAcroListHeading {mm}
  { \acro_declare_list_heading:nn {#1} {#2} }

% --------------------------------------------------------------------------
% typesetting the acronym list
\DeclareObjectType {acro-list} {2}

% #1: id
% #2: excluded classes
\prg_new_protected_conditional:Npnn \acro_if_is_excluded:nn #1#2 {T,F,TF}
  {
    \bool_set_false:N \l__acro_is_excluded_bool
    \tl_if_blank:nF {#2}
      {
        \clist_map_inline:nn {#2}
          {
            \prop_get:NnNT \l__acro_class_prop {#1} \l__acro_tmpa_tl
              {
                \seq_set_split:NnV \l__acro_tmpa_seq {,} \l__acro_tmpa_tl
                \seq_if_in:NnT \l__acro_tmpa_seq {##1}
                  { \bool_set_true:N \l__acro_is_excluded_bool }
              }
          }
      }
    \bool_if:NTF \l__acro_is_excluded_bool
      { \prg_return_true: }
      { \prg_return_false: }
  }

% #1: id
% #2: included classes
\prg_new_protected_conditional:Npnn \acro_if_is_included:nn #1#2 {T,F,TF}
  {
    \bool_set_false:N \l__acro_is_included_bool
    \tl_if_blank:nTF {#2}
      { \bool_set_true:N \l__acro_is_included_bool }
      {
        \clist_map_inline:nn {#2}
          {
            \prop_get:NnNT \l__acro_class_prop {#1} \l__acro_tmpa_tl
              {
                \seq_set_split:NnV \l__acro_tmpa_seq {,} \l__acro_tmpa_tl
                \seq_if_in:NnT \l__acro_tmpa_seq {##1}
                  { \bool_set_true:N \l__acro_is_included_bool }
              }
          }
      }
    \bool_if:NTF \l__acro_is_included_bool
      { \prg_return_true: }
      { \prg_return_false: }
  }

% #1: id
\cs_new_protected:Npn \__acro_list_entry_short:n #1
  {
    \acro_hyper_target:nn
      {#1}
      {
        \acro_acc_supp:nn
          {#1}
          {
            \acro_write_short:nn {#1}
              {
                \l__acro_list_short_format_tl
                { \__acro_get_property:nn {short} {#1} }
              }
          }
      }
  }

% #1: id
\cs_new_protected:Npn \__acro_list_entry_long:n #1
  {
    \group_begin:
      \bool_if:NT \l__acro_capitalize_list_bool
        { \bool_set_true:N \l__acro_first_upper_bool }
      \acro_write_long:Vf \l__acro_list_long_format_tl
        {
          \prop_if_in:NnTF \l__acro_list_prop {#1}
            { \__acro_get_property:nn {list} {#1} }
            { \__acro_get_property:nn {long} {#1} }
        }
    \group_end:
    \bool_if:NT \l__acro_foreign_bool
      { \acro_get_foreign:n {#1} }
    \acro_cite_if:nn { \l__acro_citation_all_bool } {#1}
  }

% #1: id
\cs_new_protected:Npn \__acro_list_entry_extra:n #1
  {
    \prop_get:NnNT \l__acro_extra_prop {#1} \l__acro_tmpa_tl
      {
        \acro_extra_instance:VV
          \l__acro_extra_instance_tl
          \l__acro_tmpa_tl
      }
  }

% #1: id
\cs_new_protected:Npn \__acro_list_entry_page:n #1
  {
    \bool_if:nT { \cs_if_exist_p:c { acro@#1@once } }
      {
        \acro_page_number_instance:Vn
          \l__acro_page_instance_tl
          {#1}
      }
  }
  
% macro for retrieval of items in the list:
% #1: property
% #2: id
\cs_new_protected:Npn \acro_list_entry:nn #1#2
  {
    \str_case:nnF {#1}
      {
        {short} { \__acro_list_entry_short:n {#2} }
        {long}  { \__acro_list_entry_long:n {#2} }
        {extra} { \__acro_list_entry_extra:n {#2} }
        {page}  { \__acro_list_entry_page:n {#2} }
      }
      { \__acro_get_property:nn {#1} {#2} }
  }

% this macro may/should be redefined in templates:
% #1: short
% #2: long
% #3: extra
% #4: page number(s)
\cs_new_protected:Npn \acro_print_list_entry:nnnn #1#2#3#4
  { #1 #2 #3 #4 }

\cs_new_protected:Npn \acro_for_all_acronyms_do:n #1
  { \prop_map_inline:Nn \l__acro_short_prop {#1} }

% test, if acronyms should be printed or not; needs testing for in/excluded
% classes and options `only-used' and `single' -- this macro should be used in
% the template code for retrieving the list
  
% #1: id
% #2: included classes
% #3: excluded classes
\prg_new_protected_conditional:Npnn \acro_if_entry:nnn #1#2#3 {T,F,TF}
  {
    \bool_if:nTF
      {
        \bool_if_p:c { g__acro_#1_in_list_bool } &&
        (
          ( \l__acro_use_single_bool && \cs_if_exist_p:c { acro@#1@twice } )
          ||
          (
            !\l__acro_use_single_bool &&
            \cs_if_exist_p:c { acro@#1@once } &&
            \l__acro_print_only_used_bool
          )
        )
        ||
        ( !\l__acro_use_single_bool && !\l__acro_print_only_used_bool )
      }
      {
        \acro_if_is_excluded:nnTF {#1} {#3}
          { \prg_return_false: }
          {
            \acro_if_is_included:nnTF {#1} {#2}
              {
                \bool_if:nTF
                  { \g__acro_use_barriers_bool && \l__acro_use_barrier_bool }
                  {
                    \acro_if_in_barrier:nxTF {#1}
                      { \int_use:N \g__acro_barrier_int }
                      { \prg_return_true: }
                      { \prg_return_false: }
                  }
                  { \prg_return_true: }
              }
              { \prg_return_false: }
          }
      }
      { \prg_return_false: }
  }

\tl_new:N \l__acro_list_entries_tl

% this macro is used in templates for fetching all items to be printed; it
% collects all entries in a tl which then is used where needed
%
% #1: tl containing the entries
% #2: included classes
% #3: excluded classes
\cs_new_protected:Npn \acro_build_list_entries:Nnn #1#2#3
  {
    \tl_clear:N #1
    \acro_for_all_acronyms_do:n
      {% ##1: id; ##2: short form
        \acro_get:n {##1}
        \acro_if_entry:nnnT {##1} {#2} {#3}
          {
            \tl_put_right:Nn #1
              {
                \acro_print_list_entry:nnnn
                  { \acro_list_entry:nn {short} {##1} }
                  { \acro_list_entry:nn {long} {##1} }
                  { \acro_list_entry:nn {extra} {##1} }
                  { \acro_list_entry:nn {page} {##1} }
              }
          }
      }
  }

% this macro is used in templates for fetching all items to be printed:
\cs_new_protected:Npn \acro_list_items:nn #1#2
  {
    \acro_build_list_entries:Nnn \l__acro_list_entries_tl {#1} {#2}
    \tl_use:N \l__acro_list_entries_tl
  }
  
% --------------------------------------------------------------------------
% declare templates for the list:
% `list' template:
\DeclareTemplateInterface {acro-list} {list} {2}
  {
    foreign-sep : tokenlist = {~} ,
    list        : tokenlist = {description} ,
    reverse     : boolean   = false ,
    before      : tokenlist = ,
    after       : tokenlist =
  }

\DeclareTemplateCode {acro-list} {list} {2}
  {
    foreign-sep = \l__acro_foreign_sep_tl ,
    list        = \l__acro_list_tl ,
    reverse     = \l__acro_list_reverse_long_extra_bool ,
    before      = \l__acro_list_before_tl ,
    after       = \l__acro_list_after_tl
  }
  {
    \AssignTemplateKeys
    \bool_set_true:N \l__acro_in_list_bool
    \acro_activate_hyperref_support:
    \bool_if:NTF \l__acro_list_reverse_long_extra_bool
      {
        \cs_set_protected:Npn \acro_print_list_entry:nnnn ##1##2##3##4
          { \item [##1] ##3 ##2 ##4 }
      }
      {
        \cs_set_protected:Npn \acro_print_list_entry:nnnn ##1##2##3##4
          { \item [##1] ##2 ##3 ##4 }
      }
    \use:x
      {
        \exp_not:V \l__acro_list_before_tl
        \exp_not:N \begin { \exp_not:V \l__acro_list_tl }
          \exp_not:n { \acro_list_items:nn {#1} {#2} }
        \exp_not:N \end { \exp_not:V \l__acro_list_tl }
        \exp_not:V \l__acro_list_after_tl
      }
  }

% `list-of' template:
\DeclareTemplateInterface {acro-list} {list-of} {2}
  {
    foreign-sep : tokenlist = {~} ,
    style       : tokenlist = {toc} ,
    reverse     : boolean   = false ,
    before      : tokenlist = ,
    after       : tokenlist =
  }

\DeclareTemplateCode {acro-list} {list-of} {2}
  {
    foreign-sep = \l__acro_foreign_sep_tl ,
    style       = \l__acro_list_of_style ,
    reverse     = \l__acro_list_reverse_long_extra_bool ,
    before      = \l__acro_list_before_tl ,
    after       = \l__acro_list_after_tl
  }
  {
    \AssignTemplateKeys
    \bool_set_true:N \l__acro_in_list_bool
    \tl_if_eq:VnT \l__acro_page_instance_tl {none}
      { \tl_set:Nn \l__acro_page_instance_tl {plain} }
    \tl_set:Nn \l__acro_page_name_tl {}
    \tl_set:Nn \l__acro_pages_name_tl {}
    \acro_activate_hyperref_support:
    \str_case:Vn \l__acro_list_of_style
      {
        {toc}
        { % similar to the table of contents
          \bool_if:NTF \l__acro_list_reverse_long_extra_bool
            {
              \cs_if_exist:NTF \chapter
                {
                  \cs_set_protected:Npn \acro_print_list_entry:nnnn ##1##2##3##4
                    {
                      \contentsline{chapter}{##1}{}{}
                      \contentsline{section}{##3##2}{##4}{}
                    } 
                }
                {
                  \cs_set_protected:Npn \acro_print_list_entry:nnnn ##1##2##3##4
                    {
                      \contentsline{section}{##1}{}{}
                      \contentsline{subsection}{##3##2}{##4}{}
                    }
                }
            }
            {
              \cs_if_exist:NTF \chapter
                {
                  \cs_set_protected:Npn \acro_print_list_entry:nnnn ##1##2##3##4
                    {
                      \contentsline{chapter}{##1}{}{}
                      \contentsline{section}{##2##3}{##4}{}
                    } 
                }
                {
                  \cs_set_protected:Npn \acro_print_list_entry:nnnn ##1##2##3##4
                    {
                      \contentsline{section}{##1}{}{}
                      \contentsline{subsection}{##2##3}{##4}{}
                    }
                }
            }
        }
        {lof}
        { % similar to the list of figures
          \cs_set_protected:Npn \l@acro
            { \@dottedtocline {1} {1.5em} {\l__acro_short_width_dim} }
          \bool_if:NTF \l__acro_list_reverse_long_extra_bool
            {
              \cs_set_protected:Npn \acro_print_list_entry:nnnn ##1##2##3##4
                { \contentsline{acro}{\numberline{##1}{##3##2}}{##4}{} }
            }
            {
              \cs_set_protected:Npn \acro_print_list_entry:nnnn ##1##2##3##4
                { \contentsline{acro}{\numberline{##1}{##2##3}}{##4}{} }
            }
        }
      }
    \use:x
      {
        \exp_not:V \l__acro_list_before_tl
        \exp_not:n { \acro_list_items:nn {#1} {#2} }
        \exp_not:V \l__acro_list_before_tl
      }
  }
  
% `table' template:
\DeclareTemplateInterface {acro-list} {table} {2}
  {
    table       : tokenlist = tabular ,
    table-spec  : tokenlist = lp{.7\linewidth} ,
    foreign-sep : tokenlist = {~} ,
    reverse     : boolean   = false ,
    before      : tokenlist = ,
    after       : tokenlist = 
  }

\DeclareTemplateCode {acro-list} {table} {2}
  {
    table       = \l__acro_list_table_tl      ,
    table-spec  = \l__acro_list_table_spec_tl ,
    foreign-sep = \l__acro_foreign_sep_tl ,
    reverse     = \l__acro_list_reverse_long_extra_bool ,
    before      = \l__acro_list_before_tl ,
    after       = \l__acro_list_after_tl
  }
  {
    \AssignTemplateKeys
    \acro_activate_hyperref_support:
    \bool_if:NTF \l__acro_list_reverse_long_extra_bool
      {
        \cs_set_protected:Npn \acro_print_list_entry:nnnn ##1##2##3##4
          { ##1 & ##3 ##2 ##4 \tabularnewline }
      }
      {
        \cs_set_protected:Npn \acro_print_list_entry:nnnn ##1##2##3##4
          { ##1 & ##2 ##3 ##4 \tabularnewline }
      }
    \acro_build_list_entries:Nnn \l__acro_list_entries_tl {#1} {#2}
    \use:x
      {
        \exp_not:V \l__acro_list_before_tl
        \exp_not:N \begin { \exp_not:V \l__acro_list_table_tl }
          { \exp_not:V \l__acro_list_table_spec_tl }
        \exp_not:V \l__acro_list_entries_tl
        \exp_not:N \end { \exp_not:V \l__acro_list_table_tl }
        \exp_not:V \l__acro_list_after_tl
      }
  }

% `extra-table' template:
\DeclareTemplateInterface {acro-list} {extra-table} {2}
  {
    table       : tokenlist = tabular ,
    table-spec  : tokenlist = llll ,
    foreign-sep : tokenlist = {~} ,
    reverse     : boolean   = false ,
    before      : tokenlist = ,
    after       : tokenlist = 
  }

\DeclareTemplateCode {acro-list} {extra-table} {2}
  {
    table       = \l__acro_list_table_tl      ,
    table-spec  = \l__acro_list_table_spec_tl ,
    foreign-sep = \l__acro_foreign_sep_tl ,
    reverse     = \l__acro_list_reverse_long_extra_bool ,
    before      = \l__acro_list_before_tl ,
    after       = \l__acro_list_after_tl
  }
  {
    \AssignTemplateKeys
    \acro_activate_hyperref_support:
    \bool_if:NTF \l__acro_list_reverse_long_extra_bool
      {
        \cs_set_protected:Npn \acro_print_list_entry:nnnn ##1##2##3##4
          { ##1 & ##3 & ##2 & ##4 \tabularnewline }
      }
      {
        \cs_set_protected:Npn \acro_print_list_entry:nnnn ##1##2##3##4
          { ##1 & ##2 & ##3 & ##4 \tabularnewline }
      }
    \acro_build_list_entries:Nnn \l__acro_list_entries_tl {#1} {#2}
    \use:x
      {
        \exp_not:V \l__acro_list_before_tl
        \exp_not:N \begin { \exp_not:V \l__acro_list_table_tl }
          { \exp_not:V \l__acro_list_table_spec_tl }
        \exp_not:V \l__acro_list_entries_tl
        \exp_not:N \end { \exp_not:V \l__acro_list_table_tl }
        \exp_not:V \l__acro_list_after_tl
      }
  }

% --------------------------------------------------------------------------
% declare new list styles:
\cs_new_protected:Npn \acro_declare_list_style:nnn #1#2#3
  {
    \DeclareInstance {acro-list} {#1} {#2} {#3}
    \prop_put:Nnn \l__acro_list_styles_prop  {#1} {#2}
  }

% #1: name
% #2: template
% #3: settings
\NewDocumentCommand \DeclareAcroListStyle {mmm}
  { \acro_declare_list_style:nnn {#1} {#2} {#3} }

% set a list style
\cs_new_protected:Npn \acro_set_list_style:n #1
  {
    \prop_if_in:NnTF \l__acro_list_styles_prop {#1}
      { \__acro_set_list_style:n {#1} }
      {
        \msg_warning:nnnnn {acro} {unknown}
          {list~ style}
          {#1}
          {description}
        \__acro_set_list_style:n {description}
      }
  }

\cs_new_protected:Npn \__acro_set_list_style:n #1
  {
    \tl_set:Nn \l__acro_list_instance_tl {#1}
    \prop_get:NnN \l__acro_list_styles_prop {#1} \l__acro_list_type_tl
  }

% --------------------------------------------------------------------------
% automatic typesetting, the internals of \ac:
% #1: id
  
\cs_new_protected:Npn \acro_use:n #1
  {
    % get the acronym and the plural settings:
    \acro_get:n {#1}
    \acro_is_used:nTF {#1}
      {
        % this is not the first time
        \acro_write_indefinite:nn {#1} {short}
        \acro_write_compact:nn {#1} {short}
        \acro_after:n {#1}
      }
      {
        % this is the first time
        \bool_gset_true:c { g__acro_#1_first_use_bool }
        \acro_if_is_single:nTF {#1}
          { \acro_single:n {#1} }
          { \acro_first_instance:nV {#1} \l__acro_long_tl }
      }
  }

% single appearances:
\cs_new_protected:Npn \acro_single:n #1
  {
    \acro_cite:
    \acro_single_form:nV {#1} \l__acro_single_form_tl
    \acro_after:n {#1}
  }
  
% #1: ID
% #2: long|first|<other>
\cs_new_protected:Npn \acro_single_form:nn #1#2
  {
    \acro_write_indefinite:nn {#1} {#2}
    \str_case:nnF {#2}
      {
        {long} {
          \tl_if_blank:VT \l__acro_single_format_tl
            {
              \bool_if:NTF \l__acro_custom_long_format_bool
                {
                  \tl_set_eq:NN
                    \l__acro_single_format_tl
                    \l__acro_custom_long_format_tl
                }
                {
                  \tl_set_eq:NN
                    \l__acro_single_format_tl
                    \l__acro_long_format_tl
                }
            }
          \tl_if_blank:VT \l__acro_single_tl
            { \tl_set_eq:NN \l__acro_single_tl \l__acro_long_tl }
          \acro_write_long:VV \l__acro_single_format_tl \l__acro_single_tl
        }
        {first} {
          \tl_if_blank:VF \l__acro_single_format_tl
            {
              \tl_set_eq:NN
                \l__acro_first_long_format_tl
                \l__acro_single_format_tl
            }
          \tl_if_blank:VT \l__acro_single_tl
            { \tl_set_eq:NN \l__acro_single_tl \l__acro_long_tl }
          \acro_first_instance:nV {#1} \l__acro_single_tl
        }
      }
      { % other (e.g. short)
        \tl_if_blank:VF \l__acro_single_tl
          { \tl_set_eq:cN {l__acro_#2_tl} \l__acro_single_tl }
        \tl_if_blank:VF \l__acro_single_format_tl
          { \tl_set_eq:cN {l__acro_#2_format_tl} \l__acro_single_format_tl }
        \acro_write_compact:nn {#1} {#2}
      }
  }
\cs_generate_variant:Nn \acro_single_form:nn {nV}

\prg_new_conditional:Npnn \acro_if_is_single:n #1 { p,T,TF }
  {
    \bool_if:nTF
      { !\l__acro_use_single_bool || \cs_if_exist_p:c { acro@#1@twice } }
      { \prg_return_false: }
      { \prg_return_true: }
  }

\cs_new_protected:Npn \acro_use_acronym:n #1
  { \use:c {bool_set_#1:N} \l__acro_mark_as_used_bool }

% --------------------------------------------------------------------------
% some helpers we'll need more often:
\seq_new:N \g__acro_declared_acronyms_seq

\prg_new_conditional:Npnn \acro_if_defined:n #1 {p,T,F,TF}
  {
    \seq_if_in:NnTF \g__acro_declared_acronyms_seq {#1}
      { \prg_return_true: }
      { \prg_return_false: }
  }

\cs_new_protected:Npn \acro_defined:n #1
  {
    \acro_if_defined:nF {#1}
      { \acro_serious_message:nn {undefined} {#1} }
  }

% expandably gets property but doesn't transform property name -- internal
% name is needed
% #1: property
% #2: id
\cs_new:Npn \__acro_get_property:nn #1#2
  { \prop_item:cn {l__acro_#1_prop} {#2} }

% #1: id
% #2: property
% #3: set case
% #4: not set case
\prg_new_protected_conditional:Npnn \acro_get_property:nn #1#2 {T,F,TF}
  {
    \tl_set:Nn \l__acro_tmpa_tl {#2}
    \tl_replace_all:Nnn \l__acro_tmpa_tl {-} {_}
    \prop_get:cncTF
      {l__acro_ \l__acro_tmpa_tl _prop}
      {#1}
      {l__acro_ \l__acro_tmpa_tl _tl}
      { \prg_return_true: }
      { \prg_return_false: }
  }

\cs_new_protected:Npn \acro_get_property:nn #1#2
  { \acro_get_property:nnTF {#1} {#2} {} {} }
\cs_generate_variant:Nn \acro_get_property:nn {V}

% #1: id
% #2: property
% #3: set case
% #4: not set case
\prg_new_protected_conditional:Npnn \acro_if_property:nn #1#2 {T,F,TF}
  {
    \tl_set:Nn \l__acro_tmpa_tl {#2}
    \tl_replace_all:Nnn \l__acro_tmpa_tl {-} {_}
    \prop_if_in:cnTF
      {l__acro_ \l__acro_tmpa_tl _prop}
      {#1}
      { \prg_return_true: }
      { \prg_return_false: }
  }

\seq_new:N \l__acro_actions_seq

% within this command one can refer to the current id with `#1'
\cs_new_protected:Npn \acro_add_action:n #1
  { \seq_put_right:Nn \l__acro_actions_seq {#1} }

\tl_new:N \l_acro_current_id_tl
\cs_new_protected:Npn \__acro_get_actions:n #1
  {
    \seq_map_inline:Nn \l__acro_actions_seq
      {
        \cs_set:Npn \__acro_action:n ####1 {##1}
        \__acro_action:n {#1}
      }
  }

\cs_new_protected:Npn \acro_get:n #1
  {
    \bool_if:NF \l__acro_in_list_bool { \leavevmode }
    \acro_activate_hyperref_support:
    % short:
    \prop_get:NnNF \l__acro_short_prop {#1} \l__acro_tmpa_tl {}
    \__acro_make_link:NnV \l__acro_short_tl {#1} \l__acro_tmpa_tl
    % \acro_get_property:nn {#1} {short-format}
     % alt:
    \prop_get:NnNTF \l__acro_alt_prop {#1} \l__acro_tmpa_tl
      { \__acro_make_link:NnV \l__acro_alt_tl {#1} \l__acro_tmpa_tl }
      { \tl_set_eq:NN \l__acro_alt_tl \l__acro_short_tl }
    % long:
    \acro_get_property:nn {#1} {long}
    % \acro_get_property:nn {#1} {long-format}
    % foreign:
    \acro_get_property:nn {#1} {foreign}
    % foreign-lang:
    \acro_get_property:nn {#1} {foreign-lang}
    % extra:
    \acro_get_property:nn {#1} {extra}
    % \acro_get_property:nn {#1} {extra-format}
    % single:
    \acro_get_property:nn {#1} {single}
    % \acro_get_property:nn {#1} {single-format}
    % first-style:
    \acro_get_property:nn {#1} {first-style}
    % formatting
    \prop_get:NnNTF \l__acro_long_format_prop {#1}
      \l__acro_custom_long_format_tl
      { \bool_set_true:N  \l__acro_custom_long_format_bool }
      { \bool_set_false:N \l__acro_custom_long_format_bool }
    \acro_get_property:nn {#1} {first-long-format}
    \prop_get:NnNTF \l__acro_format_prop {#1} \l__acro_custom_format_tl
      { \bool_set_true:N \l__acro_custom_format_bool }
      { \bool_set_false:N \l__acro_custom_format_bool }
    \acro_get_property:nn {#1} {single-format}
    \acro_for_endings_do:n
      {
        \bool_if:cT {l__acro_##1_bool}
          { \__acro_set_ending_for:nnn {##1} {#1} {long} }
      }
    \acro_get_property:nnF {#1} {long-post}
      { \tl_clear:N \l__acro_long_post_tl }
    \acro_get_property:nnT {#1} {long-pre}
      { \tl_put_left:NV \l__acro_long_tl \l__acro_long_pre_tl }
    \__acro_get_actions:n {#1}
  }

% --------------------------------------------------------------------------
% plural endings and similar concepts:
\seq_new:N \l__acro_endings_seq

\cs_new_protected:Npn \acro_for_endings_do:n #1
  { \seq_map_inline:Nn \l__acro_endings_seq {#1} }

% #1: ending
% #2: ID
\cs_new_protected:Npn \__acro_set_ending:nn #1#2
  {
    \bool_if:cT {l__acro_#1_bool}
      {
        \__acro_set_ending_for:nnn {#1} {#2} {short}
        \__acro_set_ending_for:nnn {#1} {#2} {alt}
        \__acro_set_ending_for:nnn {#1} {#2} {long}
      }
  }

\tl_new:N \l__acro_endings_tl

\bool_new:N \l__acro_use_ending_form_bool

% this does nothing if a non-existent ending (#1) or non-existent form (#3) is
% input
% #1: ending
% #2: id
% #3: short|alt|long
\cs_new_protected:Npn \__acro_set_ending_for:nnn #1#2#3
  {
    \acro_if_ending_form_exist:nnT {#1} {#3}
      {
        \bool_if:nTF { \prop_item:cn {l__acro_#3_#1_form_prop} {#2} }
          { \prop_get:cnc {l__acro_#3_#1_prop} {#2} {l__acro_#3_tl}  }
          { \prop_get:cnc {l__acro_#3_#1_prop} {#2} {l__acro_#3_#1_tl} }
      }
  }

\cs_new_protected:Npn \__acro_set_endings:n #1
  {
    \acro_for_endings_do:n
      { \__acro_set_ending:nn {##1} {#1} }
  }

% #1: id
% #2: short|alt|…
\cs_new_protected:Npn \acro_get_ending_form:nn #1#2
  {
    \acro_for_endings_do:n
      {
        \acro_if_ending_form_exist:nnT {##1} {#2}
          {
            \bool_if:nT
              {
                \prop_item:cn {l__acro_#2_##1_form_prop} {#1}
                &&
                \use:c {l__acro_##1_bool}
              }
              { \prop_get:cncF {l__acro_#2_##1_prop} {#1} {l__acro_#2_tl} {} }
          }
      }
  }

% #1: id
% #2: short|alt|…
\cs_new_protected:Npn \acro_endings:nn #1#2
  {
    \group_begin:
      \bool_if:NTF \l__acro_include_endings_format_bool
        {
          \bool_if:NTF \l__acro_custom_format_bool
            { \l__acro_custom_format_tl }
            { \tl_use:c {l__acro_#2_format_tl} }
        }
        { \use:n }
        {
          \acro_for_endings_do:n
            {
              \__acro_set_ending_for:nnn {##1} {#1} {#2}
              \bool_if:cT {l__acro_##1_bool}
                { \tl_use:c {l__acro_#2_##1_tl} }
            }
        }
    \group_end:
  }

\prg_new_conditional:Npnn \acro_if_ending_exist:n #1 {p,T,F,TF}
  {
    \seq_if_in:NnTF \l__acro_endings_seq {#1}
      { \prg_return_true: }
      { \prg_return_false: }
  }

% #1: ending
% #2: short|alt|…
\prg_new_conditional:Npnn \acro_if_ending_form_exist:nn #1#2 {p,T,F,TF}
  {
    \cs_if_exist:cTF {l__acro_#2_#1_prop}
      { \prg_return_true: }
      { \prg_return_false: }
  }
  
% #1: name
% #2: default short
% #3: default long
\cs_new_protected:Npn \acro_provide_ending:nnn #1#2#3
  {
    \acro_if_ending_exist:nTF {#1}
      {
        \acro_harmless_message:nn {ending-exists} {#1}
        % short variables
        \acro_set_ending_variables:nnn {short} {#1} {#2}
        % alt variables
        \acro_set_ending_variables:nnn {alt} {#1} {#2}
        % long variables
        \acro_set_ending_variables:nnn {long} {#1} {#3}
      }
      {
        % registering:
        \bool_if:NT \g__acro_first_acronym_declared_bool
          { \acro_serious_message:n {ending-before-acronyms} }
        \seq_put_right:Nn \l__acro_endings_seq {#1}
        \bool_new:c {l__acro_#1_bool}
        % short variables
        \acro_define_and_set_ending_variables:nnn {short} {#1} {#2}
        % alt variables
        \acro_define_and_set_ending_variables:nnn {alt} {#1} {#2}
        % long variables
        \acro_define_and_set_ending_variables:nnn {long} {#1} {#3}
        % define setup command:
        \tl_set:Nn \l__acro_tmpa_tl {#1}
        \tl_replace_all:Nnn \l__acro_tmpa_tl {-} {_}
        \cs_new_protected:cpn {acro_ \l__acro_tmpa_tl :}
          { \bool_set_true:c {l__acro_#1_bool} }
        % acronym properties:
        % short-<ending>:
        \acro_declare_property:nnn {short_#1} {short-#1}
          {
            \prop_put:cnn {l__acro_short_#1_form_prop} {##1} { \c_false_bool }
            \prop_put:cnx {l__acro_pdfstring_short_#1_prop}
              {##1} { \prop_item:Nn \l__acro_short_prop {##1} \exp_not:n {##2} }
          }
        % short-<ending>-form:
        \acro_declare_property_generic:nnn {short_#1_form} {short-#1-form}
          {
            \__acro_property_check:nn {##1} {short-#1-form}
            \prop_put:cnn {l__acro_short_#1_form_prop} {##1} { \c_true_bool }
            \prop_put:cnn {l__acro_short_#1_prop} {##1} {##2}
            \prop_put:cnn {l__acro_pdfstring_short_#1_prop} {##1} {##2}
          }
        % alt-<ending>:
        \acro_declare_property:nnn {alt_#1} {alt-#1}
          {
            \prop_put:cnn {l__acro_alt_#1_form_prop} {##1} { \c_false_bool }
            \prop_put:cnx {l__acro_pdfstring_alt_#1_prop}
              {##1} { \prop_item:Nn \l__acro_alt_prop {##1} \exp_not:n {##2} }
          }
        % alt-<ending>-form:
        \acro_declare_property_generic:nnn {alt_#1_form} {alt-#1-form}
          {
            \__acro_property_check:nn {##1} {alt-#1-form}
            \prop_put:cnn {l__acro_alt_#1_form_prop} {##1} { \c_true_bool }
            \prop_put:cnn {l__acro_alt_#1_prop} {##1} {##2}
            \prop_put:cnn {l__acro_pdfstring_alt_#1_prop} {##1} {##2}
          }
        % long-<ending>:
        \acro_declare_property:nnn {long_#1} {long-#1}
          { \prop_put:cnn {l__acro_long_#1_form_prop} {##1} { \c_false_bool } }
        % long-<ending>-form:
        \acro_declare_property_generic:nnn {long_#1_form} {long-#1-form}
          {
            \__acro_property_check:nn {##1} {long-#1-form}
            \prop_put:cnn {l__acro_long_#1_form_prop} {##1} { \c_true_bool }
            \prop_put:cnn {l__acro_long_#1_prop} {##1} {##2}
          }
        % options:
        %   short-<ending>-ending
        %   alt-<ending>-ending
        %   long-<ending>-ending
        %   <ending>-ending
        \keys_define:nn {acro}
          {
            short-#1-ending .code:n =
              \bool_if:NT \g__acro_first_acronym_declared_bool
                { \acro_serious_message:n {ending-before-acronyms} }
              \tl_set:cn {l__acro_default_short_#1_tl} {##1} ,
            alt-#1-ending   .code:n =
              \bool_if:NT \g__acro_first_acronym_declared_bool
                { \acro_serious_message:n {ending-before-acronyms} }
              \tl_set:cn {l__acro_default_alt_#1_tl} {##1} ,
            long-#1-ending  .code:n =
              \bool_if:NT \g__acro_first_acronym_declared_bool
                { \acro_serious_message:n {ending-before-acronyms} }
              \tl_set:cn {l__acro_default_long_#1_tl} {##1},
            #1-ending       .code:n   =
              \bool_if:NT \g__acro_first_acronym_declared_bool
                { \acro_serious_message:n {ending-before-acronyms} }
              \__acro_read_ending_settings:nww {#1} ##1// \acro_stop:
          }
        % pdfstrings:
        % TODO: add long forms:
        \prop_new:c {l__acro_pdfstring_short_#1_prop}
        \cs_new:cpn {acro_pdf_string_short_#1:n} ##1
          {
            \acro_if_star_gobble:nTF {##1}
              { \prop_item:cn {l__acro_pdfstring_short_#1_prop} }
              { \prop_item:cn {l__acro_pdfstring_short_#1_prop} {##1} }
          }
        \cs_new:cpn {acpdfstring#1} { \use:c {acro_pdf_string_short_#1:n} }
        \prop_new:c {l__acro_pdfstring_alt_#1_prop}
        \cs_new:cpn {acro_pdf_string_alt_#1:n} ##1
          {
            \acro_if_star_gobble:nTF {##1}
              { \prop_item:cn {l__acro_pdfstring_alt_#1_prop} }
              { \prop_item:cn {l__acro_pdfstring_alt_#1_prop} {##1} }
          }
        \cs_new:cpn {acpdfstringalt#1} { \use:c {acro_pdf_string_alt_#1:n} }
      }
  }

% #1: short|alt|long
% #2: ending name
% #3: default ending
\cs_new_protected:Npn \acro_define_and_set_ending_variables:nnn #1#2#3
  {
    \acro_define_ending_variables:nn {#1} {#2}
    \acro_set_ending_variables:nnn {#1} {#2} {#3}
  }

% #1: short|alt|long
% #2: ending name
\cs_new_protected:Npn \acro_define_ending_variables:nn #1#2
  {
    \prop_new:c {l__acro_#1_#2_prop}
    \prop_new:c {l__acro_#1_#2_form_prop}
    \tl_new:c   {l__acro_#1_#2_tl}
    \tl_new:c   {l__acro_default_#1_#2_tl}
  }

% #1: short|alt|long
% #2: ending name
% #3: default ending
\cs_new_protected:Npn \acro_set_ending_variables:nnn #1#2#3
  { \tl_set:cn  {l__acro_default_#1_#2_tl} {#3} }

% #1: ending name
% #2: short (and long if #4 is blank)
% #3: long
\cs_new_protected:Npn \__acro_read_ending_settings:nww #1#2/#3/#4 \acro_stop:
  {
    \acro_set_ending_variables:nnn {short} {#1} {#2}
    \acro_set_ending_variables:nnn {alt} {#1} {#2}
    \tl_if_blank:nTF {#4}
      { \acro_set_ending_variables:nnn {long} {#1} {#3} }
      { \acro_set_ending_variables:nnn {long} {#1} {#2} }
  }

\NewDocumentCommand \ProvideAcroEnding {mmm}
  { \acro_provide_ending:nnn {#1} {#2} {#3} }

% --------------------------------------------------------------------------
% enable us to know if the acronym is used only once and provide a different
% style for that:
\prg_new_protected_conditional:Npnn \acro_is_used:n #1 { T,F,TF }
  {
    \acro_record_barrier:n {#1}
    \bool_if:nTF
      {
        \bool_if_p:c { g__acro_#1_used_bool } &&
        (
          (
            \bool_if_p:c { g__acro_#1_first_use_bool } &&
            \g__acro_mark_first_as_used_bool
          )
          ||
          ! \g__acro_mark_first_as_used_bool
        )
      }
      {
        \bool_if:NTF \l__acro_mark_as_used_bool
          {
            \__acro_aux_file:Nxxxx \acro@used@twice
              {#1}
              { \thepage }
              { \arabic {page} }
              { \arabic {abspage} }
          }
          { \__acro_aux_file:Nxxxx \acro@used@twice {#1} {} {} {} }
        \prg_return_true:
      }
      {
        \bool_if:NTF \l__acro_mark_as_used_bool
          {
            \__acro_aux_file:Nxxxx \acro@used@once
              {#1}
              { \thepage }
              { \arabic {page} }
              { \arabic {abspage} }
            \bool_if:nT
              {
                !\bool_if_p:c { g__acro_#1_label_bool } &&
                \l__acro_place_label_bool
              }
              {
                \bool_gset_true:c { g__acro_#1_label_bool }
                \label{\l__acro_label_prefix_tl #1}
              }
            \bool_gset_true:c { g__acro_#1_used_bool }
          }
          { \__acro_aux_file:Nxxxx \acro@used@once {#1} {} {} {} }
        \prg_return_false:
      }
  }

\cs_new:Npn \acro_is_used:n #1
  { \acro_is_used:nTF {#1} { } { } }

\cs_new_protected:Npn \__acro_aux_file:Nnnnn #1#2#3#4#5
  { \iow_shipout:Nn \@auxout { #1 {#2} {#3} {#4} {#5} } }
\cs_generate_variant:Nn \__acro_aux_file:Nnnnn { Nxxxx }
  
\cs_new_protected:Npn \__acro_aux_file_now:n #1
  { \iow_now:Nn \@auxout {#1} }
\cs_generate_variant:Nn \__acro_aux_file_now:n { x }

% --------------------------------------------------------------------------
% the commands for the auxiliary file:
\cs_new_protected:Npn \acro@used@once #1#2#3#4
  {
    \cs_gset_nopar:cpn {acro@#1@once} {#1}
    \bool_gset_true:c {g__acro_#1_in_list_bool}
    \tl_if_empty:nF {#2#3#4}
      {
        % \bool_gset_true:c { g__acro_#1_used_bool }
        \seq_gput_right:cn {g__acro_#1_pages_seq} { {#2}{#3}{#4} }
      }
  }
\cs_new_protected:Npn \acro@used@twice #1#2#3#4
  {
    \cs_gset_nopar:cpn {acro@#1@twice} {#1}
    \tl_if_empty:nF {#2#3#4}
      { \seq_gput_right:cn {g__acro_#1_pages_seq} { {#2}{#3}{#4} } }
  }

\cs_new_protected:Npn \acro@pages #1#2
  { \tl_gset:cn {g__acro_#1_recorded_pages_tl} {#2} }

\bool_new:N \g__acro_rerun_bool

\cs_new_protected:Npn \acro@rerun@check
  {
    \bool_if:NT \g__acro_rerun_bool
      {
        \@latex@warning@no@line
          {Acronyms~ may~ have~ changed.~ Please~ rerun~ LaTeX}
      }
  }

\AtEndDocument
  {
    \bool_gset_false:N \g__acro_rerun_bool
    \cs_gset_protected:Npn \acro@used@once #1#2#3#4
      {
        \tl_set:Nn \l__acro_tmpa_tl {#1}
        \tl_if_eq:cNF {acro@#1@once} \l__acro_tmpa_tl
          { \bool_gset_true:N \g__acro_rerun_bool }
      }
    \cs_gset_protected:Npn \acro@used@twice #1#2#3#4
      {
        \tl_set:Nn \l__acro_tmpa_tl {#1}
        \tl_if_eq:cNF {acro@#1@twice} \l__acro_tmpa_tl
          { \bool_gset_true:N \g__acro_rerun_bool }
      }
    \acro_for_all_acronyms_do:n
      {
        \seq_if_empty:cF {g__acro_#1_pages_seq}
          {
            \__acro_aux_file_now:x
              {
                \token_to_str:N \acro@pages {#1}
                  { \seq_use:cn {g__acro_#1_pages_seq} {|} } ^^J
                \token_to_str:N \acro@barriers {#1}
                  { \seq_use:cn {g__acro_#1_barriers_seq} {,} }
              }
          }
        \acro_check_barriers:n {#1}
      }
    \__acro_aux_file_now:n { \acro@rerun@check }
  }

% if `acro' is deactivated prevent unnecessary errors from aux file:
\if@filesw
\AtBeginDocument
  {
    \__acro_aux_file_now:n
      {
        \providecommand \acro@used@once [4] {} ^^J
        \providecommand \acro@used@twice [4] {} ^^J
        \providecommand \acro@pages [2] {} ^^J
        \providecommand \acro@rerun@check {} ^^J
        \providecommand \acro@print@list {} ^^J
        \providecommand \acro@barriers [2] {}
      }
  }
\fi

% --------------------------------------------------------------------------
% typeset the short form:
% #1: ID
% #2: short form
\cs_new_protected:Npn \acro_write_short:nn #1#2
  {
    \mode_if_horizontal:F { \leavevmode }
    \group_begin:
      \bool_if:NTF \l__acro_custom_format_bool
        { \l__acro_custom_format_tl }
        { \l__acro_short_format_tl }
      {#2}
    \group_end:
  }
\cs_generate_variant:Nn \acro_write_short:nn { nV , nv }

% typeset the alternative form:
% #1: ID
% #2: alt form
\cs_new_protected:Npn \acro_write_alt:nn #1#2
  {
    \mode_if_horizontal:F { \leavevmode }
    \group_begin:
      \bool_if:NTF \l__acro_custom_format_bool
        { \l__acro_custom_format_tl }
        { \l__acro_alt_format_tl }
      {#2}
    \group_end:
  }
\cs_generate_variant:Nn \acro_write_alt:nn { nV , nv }

% typeset a long form:
%   TODO: rethink the formatting mechanism
%   right now a custom format gets applied additionally to the global one
%   although before it
% #1: format
% #2: long form
\cs_new_protected:Npn \acro_write_long:nn #1#2
  {
    \mode_if_horizontal:F { \leavevmode }
    \group_begin:
      \bool_if:NTF \l__acro_custom_long_format_bool
        { \l__acro_custom_long_format_tl }
        { \use:n }
      {
        \use:x
          {
            \exp_not:n {#1}
            {
              \bool_if:NTF \l__acro_first_upper_bool
                { \exp_not:N \__acro_first_upper_case:n { \exp_not:n {#2} } }
                { \exp_not:n {#2} }
            }
          }
      }
    \group_end:
  }
\cs_generate_variant:Nn \acro_write_long:nn { VV,Vo,Vf,V,v,vv }

\prg_new_conditional:Npnn \acro_if_foreign:n #1 {T,F,TF}
  {
    \bool_if:nTF
      {
        \l__acro_foreign_bool
        &&
        \prop_if_in_p:Nn \l__acro_foreign_prop {#1}
      }
      { \prg_return_true: }
      { \prg_return_false: }
  }

\cs_new_protected:Npn \acro_foreign_language:nn #1#2 {}
\AtBeginDocument{
  \cs_if_exist:NTF \foreignlanguage
    {
      \cs_set_protected:Npn \acro_foreign_language:nn #1#2
        {
          \tl_if_blank:nTF {#1}
            {#2}
            { \foreignlanguage {#1} {#2} }
        }
    }
    {
      \cs_set_protected:Npn \acro_foreign_language:nn #1#2
        { \use_ii:nn {#1} {#2} }
    }
}
\cs_generate_variant:Nn \acro_foreign_language:nn {VV}

\cs_new_protected:Npn \acro_write_foreign:n #1
  {
    \acro_if_foreign:nT {#1}
      {
        \prop_get:NnNT \l__acro_foreign_prop {#1} \l__acro_foreign_tl
          {
            \group_begin:
              \tl_use:N \l__acro_foreign_format_tl
              {
                \acro_foreign_language:VV
                  \l__acro_foreign_lang_tl
                  \l__acro_foreign_tl
              }
            \group_end:
          }
      }
  }

\cs_new:Npn \acroenparen #1 { ( #1 ) }

\cs_new_protected:Npn \acro_get_foreign:n #1
  {
    \prop_get:NnNT \l__acro_foreign_prop {#1} \l__acro_foreign_tl
      {
        \tl_use:N \l__acro_foreign_sep_tl
        \group_begin:
          \tl_use:N \l__acro_foreign_list_format_tl
          {
            \acro_foreign_language:VV
              \l__acro_foreign_lang_tl
              \l__acro_foreign_tl
          }
        \group_end:
      }
  }

% --------------------------------------------------------------------------
% #1: id
% #2: short|alt
\cs_set_protected:Npn \acro_write_compact:nn #1#2
  {
    \acro_get_ending_form:nn {#1} {#2}
    \acro_acc_supp:nn
      {#1}
      {
        \acro_write_tooltip:nnV
          {#1}
          {
            \use:c {acro_write_#2:nv} {#1} {l__acro_#2_tl}
            \acro_endings:nn {#1} {#2}
          }
          \l__acro_long_tl
      }
  }

% TODO: get rid of argument #3?
% #1: ID
% #2: long|first-long|list-long|extra
% #3: long form
\cs_new_protected:Npn \acro_write_expanded:nnn #1#2#3
  {
    \tl_set:Nn \l__acro_tmpa_tl {#2}
    \tl_replace_all:Nnn \l__acro_tmpa_tl {-} {_}
    \acro_write_long:vn {l__acro_ \l__acro_tmpa_tl _format_tl} {#3}
    \acro_endings:nn {#1} {long}
    \tl_if_in:nnT {#2} {long}
      { \l__acro_long_post_tl }
  }
\cs_generate_variant:Nn \acro_write_expanded:nnn { nnV }

% #1: ID
% #2: long|first-long|list-long|extra
\cs_new_protected:Npn \acro_write_expanded:nn #1#2
  {
    \tl_set:Nn \l__acro_tmpa_tl {#2}
    \tl_replace_all:Nnn \l__acro_tmpa_tl {-} {_}
    \acro_write_long:vv
      {l__acro_ \l__acro_tmpa_tl _format_tl}
      {l__acro_ \l__acro_tmpa_tl _tl}
    \acro_endings:nn {#1} {long}
    \tl_if_in:nnT {#2} {long}
      { \l__acro_long_post_tl }
  }

% #1: id
\cs_new:Npn \acro_after:n #1
  {
    \acro_cite_if:nn { \l__acro_citation_all_bool } {#1}
    \acro_index_if:nn { \l__acro_addto_index_bool } {#1}
  }

\cs_new_protected:Npn \acro_check_single:n #1
  {
    \acro_if_is_single:nT {#1}
      { \cs_set_eq:NN \acro_hyper_link:nn \use_ii:nn }
  }

% --------------------------------------------------------------------------
% #1: id
\cs_new_protected:Npn \acro_before:n #1
  {
    \acro_get:n {#1}
    \acro_is_used:n {#1}
    \acro_check_single:n {#1}
  }

% the standard internals:
% #1: id
\cs_new_protected:Npn \acro_short:n #1
  {
    \acro_before:n {#1}
    \acro_write_indefinite:nn {#1} {short}
    \acro_write_compact:nn {#1} {short}
    \acro_after:n {#1}
  }

% get alternative entry:
% #1: id
\cs_new_protected:Npn \acro_alt:n #1
  {
    \acro_before:n {#1}
    \acro_alt_error:n {#1}
    \acro_write_indefinite:nn {#1} {alt}
    \acro_write_compact:nn {#1} {alt}
    \acro_after:n {#1}
  }

% get long entry:
% #1: id
\cs_new_protected:Npn \acro_long:n #1
  {
    \acro_before:n {#1}
    \acro_write_indefinite:nn {#1} {long}
    \acro_write_expanded:nn {#1} {long}
    \acro_after:n {#1}
  }

% get foreign entry:
% #1: id
\cs_new_protected:Npn \acro_foreign:n #1
  {
    \acro_get:n {#1}
    \tl_if_blank:VF \l__acro_foreign_tl
      {
        \acro_is_used:n {#1}
        \acro_check_single:n {#1}
        \acro_write_long:VV \l__acro_foreign_format_tl \l__acro_foreign_tl
        \acro_after:n {#1}
      }
  }

% get extra entry:
% #1: id
\cs_new_protected:Npn \acro_extra:n #1
  {
    \acro_get:n {#1}
    \tl_if_blank:VF \l__acro_extra_tl
      {
        \acro_is_used:n {#1}
        \acro_check_single:n {#1}
        \acro_write_long:VV \l__acro_extra_format_tl \l__acro_extra_tl
        \acro_after:n {#1}
      }
  }

% output like the first time:
% #1: id
\cs_new_protected:Npn \acro_first:n #1
  {
    \bool_gset_true:c {g__acro_#1_first_use_bool}
    \acro_before:n {#1}
    \acro_first_instance:nV {#1} \l__acro_long_tl
  }

% output like the first time with own long version:
% #1: id
% #2: instead of long entry
\cs_new_protected:Npn \acro_first_like:nn #1#2
  {
    \bool_gset_true:c {g__acro_#1_first_use_bool}
    \acro_before:n {#1}
    \acro_first_instance:nn {#1} {#2}
  }

% ----------------------------------------------------------------------------
% citations:
\cs_new:Npn \__acro_citation_cmd:w { \cite } %{}
\cs_new:Npn \__acro_group_citation_cmd:w { \cite } %{}

% #1 pre
% #2 post
% #3 key
\cs_new:Npn \__acro_cite:nnn #1#2#3
  {
    \quark_if_no_value:nTF {#1}
      { \__acro_citation_cmd:w {#3} }
      {
        \quark_if_no_value:nTF {#2}
          { \__acro_citation_cmd:w [ #1 ] {#3} }
          { \__acro_citation_cmd:w [ #1 ] [ #2 ] {#3} }
      }
  }
\cs_generate_variant:Nn \__acro_cite:nnn { VVV }

\cs_new_protected:Npn \acro_cite:n #1
  {
    \prop_get:NnNT \l__acro_citation_prop {#1} \l__acro_tmpa_tl
      {
        \prop_get:NnN \l__acro_citation_pre_prop {#1} \l__acro_tmpb_tl
        \prop_get:NnN \l__acro_citation_post_prop {#1} \l__acro_tmpc_tl
        \acro_no_break:
        \tl_use:N \l__acro_citation_connect_tl
        \__acro_cite:VVV
          \l__acro_tmpb_tl
          \l__acro_tmpc_tl
          \l__acro_tmpa_tl
      }
  }

\cs_new_protected:Npn \acro_group_cite:n #1
  {
    \group_begin:
      \cs_set_eq:NN \__acro_citation_cmd:w \__acro_group_citation_cmd:w
      \tl_set_eq:NN
        \l__acro_citation_connect_tl
        \l__acro_between_group_connect_citation_tl
      \acro_cite_if:nn { \l__acro_citation_first_bool } {#1}
    \group_end:
  }

\cs_new_protected:Npn \acro_cite_if:nn #1#2
  { \bool_if:nT {#1} { \acro_cite:n {#2} } }

% ----------------------------------------------------------------------------
% indexing:
\cs_new_protected:Npn \acro_index_if:nn #1#2
  {
    \bool_if:nT { (#1) && \l__acro_mark_as_used_bool }
      {
        \prop_get:NnN \l__acro_index_cmd_prop  {#2} \l__acro_tmpa_tl
        \prop_get:NnN \l__acro_index_sort_prop {#2} \l__acro_tmpb_tl
        \prop_get:NnN \l__acro_index_prop      {#2} \l__acro_tmpc_tl
        \__acro_index:VnVV
          \l__acro_tmpa_tl
          {#2}
          \l__acro_tmpb_tl
          \l__acro_tmpc_tl
      }
  }

\cs_new:Npn \__acro_index_cmd:n { \index }

% #1: cmd
% #2: key
% #3: sort
% #4: replace
\cs_new_protected:Npn \__acro_index:nnnn #1#2#3#4
  {
    \prop_get:NnNF \l__acro_short_prop  {#2} \l__acro_index_short_tl {}
    \prop_get:NnNF \l__acro_format_prop {#2} \l__acro_index_format_tl {}
    \quark_if_no_value:VTF \l__acro_index_format_tl
      { \tl_set:Nn \l__acro_tmpa_tl { \l__acro_short_format_tl \l__acro_index_short_tl } }
      { \tl_set:Nn \l__acro_tmpa_tl { \l__acro_index_format_tl \l__acro_index_short_tl } }
    \quark_if_no_value:nF {#1}
      { \cs_set:Npn \__acro_index_cmd:n {#1} }
    \quark_if_no_value:nTF {#4}
      {
        \quark_if_no_value:nTF {#3}
          { \__acro_index_cmd:n { #2 @ { \l__acro_tmpa_tl } } }
          { \__acro_index_cmd:n { #3 @ { \l__acro_tmpa_tl } } }
      }
      { \__acro_index_cmd:n {#4} }
  }
\cs_generate_variant:Nn \__acro_index:nnnn { VnVV }

% ----------------------------------------------------------------------------
% accessability support
\cs_new_eq:NN \acro_acc_supp:nn \use_ii:nn

\cs_new_protected:Npn \acro_get_acc_supp:nn #1#2
  {
    \prop_get:NnNF \l__acro_acc_supp_prop {#1} \l__acro_acc_supp_tl
      { \prop_get:NnNF \l__acro_short_prop {#1} \l__acro_acc_supp_tl {} }
    \acro_for_endings_do:n
      {
        \bool_if:cT {l__acro_##1_bool}
          {
            \tl_put_right:Nv
              \l__acro_acc_supp_tl
              {l__acro_short_##1_tl}
          }
      }
    \acro_do_acc_supp:VVn
      \l__acro_acc_supp_tl
      \l__acro_acc_supp_options_tl
      {#2}
  }

\cs_new:Npn \acro_do_acc_supp:nnn #1#2#3
  {
    \BeginAccSupp { ActualText = #1 , #2 }
      #3
    \EndAccSupp { }
  }
\cs_generate_variant:Nn \acro_do_acc_supp:nnn { VV }

\AtEndPreamble
  {
    \bool_if:NT \l__acro_acc_supp_bool
      {
        \RequirePackage {accsupp}
        \cs_set_eq:NN \acro_acc_supp:nn \acro_get_acc_supp:nn
      }
    \bool_if:NT \l__acro_tooltip_bool
      {
        \RequirePackage {pdfcomment}
        \cs_if_eq:NNT \__acro_tooltip_cmd:nn \use_i:nn
          { \cs_set:Npn \__acro_tooltip_cmd:nn { \pdftooltip } }
      }
  }

% --------------------------------------------------------------------------
% tooltips for acronyms

% #1: id
% #2: printed text
% #3: tool tip text
\cs_new_protected:Npn \acro_write_tooltip:nnn #1#2#3
  {
    \prop_get:NnNTF \l__acro_tooltip_prop {#1} \l__acro_tmpa_tl
      { \__acro_check_tooltip:nV {#2} \l__acro_tmpa_tl }
      { \__acro_check_tooltip:nn {#2} {#3} }
  }
\cs_generate_variant:Nn \acro_write_tooltip:nnn { nnV }

% #1: printed text
% #2: tool tip text
\cs_new_protected:Npn \__acro_check_tooltip:nn #1#2
  {
    \bool_if:NTF \l__acro_inside_tooltip_bool
      {#1}
      {
        \bool_set_true:N \l__acro_inside_tooltip_bool
        \__acro_tooltip_cmd:nn {#1} {#2}
      }
  }
\cs_generate_variant:Nn \__acro_check_tooltip:nn { nV }

% use whatever command you like for creating tooltips here:
% #1: printed text
% #2: tool tip text
\cs_new_eq:NN \__acro_tooltip_cmd:nn \use_i:nn
  
% --------------------------------------------------------------------------
% indefinite articles:

% #1: ID
% #2: short|long|alt
\cs_new_protected:Npn \acro_write_indefinite:nn #1#2
  {
    \bool_if:NT \l__acro_indefinite_bool
      { \prop_item:cn { l__acro_#2_indefinite_prop } {#1} ~ }
    \bool_if:NT \l__acro_upper_indefinite_bool
      { %  \bool_set_true:N \l__acro_first_upper_bool
         \__acro_first_upper_case:x
           { \prop_item:cn { l__acro_#2_indefinite_prop } {#1} } ~
      }
  }

% --------------------------------------------------------------------------
% experimental sorting feature:

% the following code is an adaption of expl3 code used for \str_if_eq:NN(TF)
\sys_if_engine_luatex:TF
  {
    \tl_set:Nn \l__acro_tmpa_tl
      {
        acro ~ = ~ acro ~ or ~ { ~ } ~
        function ~ acro.strcmp ~ (A, B) ~
          if ~ A ~ == ~ B ~ then ~
            tex.write ("0") ~
          elseif ~ A ~ < ~ B ~ then ~
            tex.write ("-1") ~
          else ~
            tex.write ("1") ~
          end ~
        end
      }
    \luatex_directlua:D { \l__acro_tmpa_tl }
    \cs_new_protected:Npn \acro_strcmp:nn #1#2
      {
        \luatex_directlua:D
          {
            acro.strcmp
              (
                " \__acro_escape_x:n {#1} " ,
                " \__acro_escape_x:n {#2} "
              )
          }
      }
    \cs_new:Npn \__acro_escape_x:n #1
      {
        \luatex_luaescapestring:D
          { \etex_detokenize:D \exp_after:wN { \luatex_expanded:D {#1} } }
      }
  }
  { \cs_new_eq:NN \acro_strcmp:nn \pdftex_strcmp:D }

\AtBeginDocument
  {
    \bool_if:NT \l__acro_sort_bool
      {
        \cs_new_protected:Npn \acro_sort_prop:NN #1#2
          {
            \seq_clear:N  \l__acro_tmpa_seq
            \prop_clear:N \l__acro_tmpa_prop
            \prop_clear:N \l__acro_tmpb_prop
            \prop_map_inline:Nn #2
              {
                \seq_put_right:Nn \l__acro_tmpa_seq {##2}
                \prop_put:Nnn \l__acro_tmpa_prop {##1} {##2}
              }
            \seq_sort:Nn \l__acro_tmpa_seq
              {
                \int_compare:nTF
                  {
                    \acro_strcmp:nn
                      { \str_fold_case:n {##1} }
                      { \str_fold_case:n {##2} }
                        = \c_minus_one
                  }
                  { \sort_return_same: }
                  { \sort_return_swapped: }
              }
            \seq_map_inline:Nn \l__acro_tmpa_seq
              {
                \prop_map_inline:Nn \l__acro_tmpa_prop
                  {
                    \str_if_eq:nnT {##1} {####2}
                      {
                        \prop_get:NnN #1 {####1} \l__acro_tmpa_tl
                        \prop_put:NnV \l__acro_tmpb_prop {####1}
                          \l__acro_tmpa_tl
                      }
                  }
              }
            \prop_set_eq:NN #1 \l__acro_tmpb_prop
          }
      }
  }

% --------------------------------------------------------------------------
% regarding list printing:
% this command ensures that a rerun warning is given when \printacronyms
% is set the first time. This mechanism doesn't make very much sense,
% should be replaced by a different and more efficient one
%
\cs_new_protected:Npn \acro@print@list
  { \cs_if_exist:NF \acro@printed@list { \cs_new:Npn \acro@printed@list { printed } } }

% --------------------------------------------------------------------------
% trailing tokens and what to do when present
\prop_new:N \l__acro_trailing_tokens_prop
\prop_new:N \l__acro_trailing_actions_prop
\bool_new:N \l__acro_trailing_tokens_bool
\tl_new:N   \l__acro_trailing_tokens_tl

\cs_new_protected:Npn \acro_new_trailing_token:n #1
  { \bool_new:c {l__acro_trailing_#1_bool} }
\cs_new_protected:Npn \acro_activate_trailing_action:n #1
  { \bool_set_true:c {l__acro_trailing_#1_bool} }
\cs_new_protected:Npn \acro_deactivate_trailing_action:n #1
  { \bool_set_false:c {l__acro_trailing_#1_bool} }

% register a new token but don't activate its action:
% #1: token
% #2: name
\cs_new_protected:Npn \acro_register_trailing_token:Nn #1#2
  {
    \prop_put:Nnn \l__acro_trailing_tokens_prop {#2} {#1}
    \prop_put:Nnn \l__acro_trailing_actions_prop {#1}
      { \acro_activate_trailing_action:n {#2} }
    \acro_new_trailing_token:n {#2}
  }
  
\NewDocumentCommand \AcroRegisterTrailing {mm}
  { \acro_register_trailing_token:Nn #1 {#2} }

\cs_new_protected:Npn \acro_for_all_trailing_tokens_do:n #1
  { \prop_map_inline:Nn \l__acro_trailing_tokens_prop {#1} }

% activate a token:
\cs_new_protected:Npn \acro_activate_trailing_token:n #1
  {
    \prop_get:NnN \l__acro_trailing_tokens_prop {#1} \l__acro_tmpa_tl
    \tl_put_right:NV \l__acro_trailing_tokens_tl \l__acro_tmpa_tl
  }

% deactivate a token:
\cs_new_protected:Npn \acro_deactivate_trailing_token:n #1
  {
    \prop_get:NnN \l__acro_trailing_tokens_prop {#1} \l__acro_tmpa_tl
    \tl_remove_all:NV \l__acro_trailing_tokens_tl \l__acro_tmpa_tl
  }

% #1: name
\prg_new_conditional:Npnn \acro_if_trailing_token:n #1 {p,T,F,TF}
  {
    \bool_if:cTF {l__acro_trailing_#1_bool}
      { \prg_return_true: }
      { \prg_return_false: }
  }

% #1: csv list of names
\prg_new_protected_conditional:Npnn \acro_if_trailing_tokens:n #1 {T,F,TF}
  {
    \bool_set_false:N \l__acro_trailing_tokens_bool
    \clist_map_inline:nn {#1}
      {
        \bool_if:cT {l__acro_trailing_##1_bool}
          {
            \bool_set_true:N \l__acro_trailing_tokens_bool
            \clist_map_break:
          }
      }
    \bool_if:NTF \l__acro_trailing_tokens_bool
      { \prg_return_true: }
      { \prg_return_false: }
  }

\cs_new_protected:Npn \aciftrailing { \acro_if_trailing_tokens:nTF }

\cs_new_protected:Npn \__acro_check_trail:N #1
  {
    \tl_map_inline:Nn \l__acro_trailing_tokens_tl
      {
        \token_if_eq_meaning:NNT #1 ##1
          { \prop_item:Nn \l__acro_trailing_actions_prop {##1} }
      }
  }

% options for activating actions:
\keys_define:nn {acro}
  {
    activate-trailing-tokens   .code:n =
      \clist_map_inline:nn {#1} { \acro_activate_trailing_token:n {##1} } ,
    activate-trailing-tokens   .initial:n = dot ,
    deactivate-trailing-tokens .code:n =
      \clist_map_inline:nn {#1} { \acro_deactivate_trailing_token:n {##1} }
  }

% some user macros:
\cs_new_protected:Npn \acro_dot:
  { \acro_if_trailing_tokens:nF {dot} {.\@} }

\cs_new_protected:Npn \acro_space:
  { \acro_if_trailing_tokens:nF {dash,babel-hyphen} { \c_space_tl } }

\NewDocumentCommand \acdot   {} { \acro_dot: }
\NewDocumentCommand \acspace {} { \acro_space: }
  
% ---------------------------------------------------------------------------
% reset outputs, they'll behave like the first time again (!not like the _only_
% time!):
\cs_new_protected:Npn \acro_reset:n #1
  {
    \bool_gset_false:c { g__acro_#1_used_bool }
    \bool_gset_false:c { g__acro_#1_first_use_bool }
  }

\cs_new_protected:Npn \acro_mark_as_used:n #1
  {
    \bool_gset_true:c { g__acro_#1_used_bool }
    \bool_gset_true:c { g__acro_#1_first_use_bool }
    \bool_gset_true:c { g__acro_#1_in_list_bool }
    \if@filesw
      \__acro_aux_file_now:n { \acro@used@once {#1} {} {} {} }
      \__acro_aux_file_now:n { \acro@used@twice {#1} {} {} {} }
    \fi
  }

\cs_new_protected:Npn \acro_reset_all:
  { \acro_for_all_acronyms_do:n { \acro_reset:n {##1} } }

% make sure that no acronym is used at the beginning of the document, see
% issue #81 for reasons why this may be necessary:
\AfterEndPreamble { \acro_reset_all: }
  
\cs_new_protected:Npn \acro_mark_all_as_used:
  { \acro_for_all_acronyms_do:n { \acro_mark_as_used:n {##1} } }

\DeclareExpandableDocumentCommand \acifused { m }
  { \acro_if_acronym_used:nTF {#1} }

\prg_new_conditional:Npnn \acro_if_acronym_used:n #1 { p,T,F,TF }
  {
    \bool_if:nTF
      {
        \bool_if_p:c { g__acro_#1_used_bool } &&
        ( !\acro_if_is_single_p:n {#1} )
      }
      { \prg_return_true: }
      { \prg_return_false: }
  }

\NewDocumentCommand \acresetall {}
  { \acro_reset_all: }

\NewDocumentCommand \acuseall {}
  { \acro_mark_all_as_used: }

\NewDocumentCommand \acreset { > { \SplitList { , } } m }
  { \ProcessList {#1} { \acro_reset:n } \ignorespaces }

\NewDocumentCommand \acuse { > { \SplitList { , } } m }
  { \ProcessList {#1} { \acro_mark_as_used:n } \ignorespaces }

% --------------------------------------------------------------------------
% acronym barriers: allow local lists of only those acronyms used between two
% barriers

\int_new:N  \g__acro_barrier_int
\bool_new:N \g__acro_use_barriers_bool
\bool_new:N \g__acro_reset_at_barrier_bool
\bool_new:N \l__acro_use_barrier_bool

\keys_define:nn {acro}
  {
    use-barriers      .bool_gset:N = \g__acro_use_barriers_bool ,
    use-barriers      .initial:n   = false ,
    reset-at-barriers .bool_gset:N = \g__acro_reset_at_barrier_bool ,
    reset-at-barriers .initial:n   = false
  }

\cs_new_protected:Npn \acro_barrier:
  {
    \int_gincr:N \g__acro_barrier_int
    \bool_if:NT \g__acro_reset_at_barrier_bool
      { \acro_reset_all: }
  }

\NewDocumentCommand \acbarrier {}
  { \acro_barrier: }

\cs_new_protected:Npn \acro_check_barriers:n #1
  {
    \bool_if:NT \g__acro_use_barriers_bool
      {
        \tl_set:Nx \l__acro_tmpa_tl
          { \seq_use:cn {g__acro_#1_barriers_seq} {} }
        \tl_set:Nx \l__acro_tmpb_tl
          { \seq_use:cn {g__acro_#1_recorded_barriers_seq} {} }
        \tl_if_eq:NNF \l__acro_tmpa_tl \l__acro_tmpb_tl
          {
            \@latex@warning@no@line
              {Rerun~to~get~barriers~of~acronym~#1~right}
          }
      }
  }

\cs_new_protected:Npn \acro_record_barrier:n #1
  {
    \bool_if:NT \g__acro_use_barriers_bool
      {
        \seq_if_in:cxF {g__acro_#1_barriers_seq}
          { \int_use:N \g__acro_barrier_int }
          {
            \seq_gput_right:cx  {g__acro_#1_barriers_seq}
              { \int_use:N \g__acro_barrier_int }
          }
      }
  }

% #1: id
% #2: barrier number
\prg_new_protected_conditional:Npnn \acro_if_in_barrier:nn #1#2 {T,F,TF}
  {
    \seq_if_in:cnTF {g__acro_#1_recorded_barriers_seq} {#2}
      { \prg_return_true: }
      { \prg_return_false: }
  }
\cs_generate_variant:Nn \acro_if_in_barrier:nnTF {nx}

\cs_new:Npn \acro@barriers #1#2
  { \seq_gset_split:cnn {g__acro_#1_recorded_barriers_seq} {,} {#2} }

% --------------------------------------------------------------------------
% the user commands -- preparation:
\cs_new_protected:Npn \acro_begin:
  {
    \group_begin:
    \__acro_check_after_end:w
  }

\cs_new_protected:Npn \__acro_check_after_end:w #1 \acro_end:
  {
    \cs_set:Npn \__acro_execute:
      {
        \__acro_check_trail:N \l_peek_token
        #1
        \acro_end: % this will end the group opened by \acro_begin:
      }
    \peek_after:Nw \__acro_execute:
  }

\cs_new_protected:Npn \acro_end: { \group_end: }

\cs_new_protected:Npn \acro_reset_specials:
  {
    \bool_set_false:N \l__acro_indefinite_bool
    \bool_set_false:N \l__acro_first_upper_bool
    \bool_set_false:N \l__acro_upper_indefinite_bool
    % \bool_set_false:N \l__acro_citation_all_bool
    % \bool_set_false:N \l__acro_citation_first_bool
    % \bool_set_false:N \l__acro_addto_index_bool
    \acro_for_endings_do:n { \bool_set_false:c {l__acro_##1_bool} }
  }

% #1: ID
% #2: true|false
\cs_new_protected:Npn \acro_check_acronym:nn #1#2
  {
    \acro_defined:n {#1}
    \acro_use_acronym:n {#2}
  }

% #1: boolean
% #2: ID
\cs_new_protected:Npn \acro_check_and_mark_if:nn #1#2
  {
    \bool_if:nTF
      { (#1) || !\l__acro_use_acronyms_bool }
      { \acro_check_acronym:nn {#2} {false} }
      { \acro_check_acronym:nn {#2} {true} }
  }

\cs_new_protected:Npn \acro_switch_off:
  { \bool_set_false:N \l__acro_use_acronyms_bool }

\cs_new_protected:Npn \acro_switch_on:
  { \bool_set_true:N \l__acro_use_acronyms_bool }

\NewDocumentCommand \acswitchoff {}
  { \acro_switch_off: }

\NewDocumentCommand \acswitchon {}
  { \acro_switch_on: }

% commands for (re)defining \ac-like macros:
\cs_new_protected:Npn \acro_define_new_acro_command:NN #1#2
  {
    % #1: csname
    % #2: definition where `#1' refers to the ID
    \cs_new_protected:Npn #1 ##1##2
      {
        \cs_set:Npn \__acro_tmp_command:n ####1 {##2}
        \exp_args:NNnx #2 ##1 {sO{}m}
          {
            \acro_begin:
              \acro_reset_specials:
              \keys_set:nn {acro} {########2}
              \acro_check_and_mark_if:nn {########1} {########3}
              \exp_not:o { \__acro_tmp_command:n {####3} }
            \acro_end:
          }
      }
  }
\cs_generate_variant:Nn \acro_define_new_acro_command:NN {cc}

% commands for (re)defining \acflike-like macros:
\cs_new_protected:Npn \acro_define_new_acro_pseudo_command:NN #1#2
  {
    % #1: csname
    % #2: definition where `#1' refers to the ID and `#2' to the pseudo long form
    \cs_new_protected:Npn #1 ##1##2
      {
        \cs_set:Npn \__acro_tmp_command:nn ####1####2 {##2}
        \exp_args:NNnx #2 ##1 {smm}
          {
            \acro_begin:
              \acro_reset_specials:
              \acro_check_and_mark_if:nn {########1} {########2}
              \exp_not:o { \__acro_tmp_command:nn {####2} {####3} }
            \acro_end:
          }
      }
  }
\cs_generate_variant:Nn \acro_define_new_acro_pseudo_command:NN {cc}

\clist_map_inline:nn {New,Renew,Declare,Provide}
  {
    \acro_define_new_acro_command:cc
      {#1AcroCommand}
      {#1DocumentCommand}
    \acro_define_new_acro_pseudo_command:cc
      {#1PseudoAcroCommand}
      {#1DocumentCommand}
  }

% --------------------------------------------------------------------------
% user commands -- facilities
\cs_new_protected:Npn \acro_first_upper:
  {
    \bool_if:NTF \l__acro_indefinite_bool
      {
        \bool_set_false:N \l__acro_indefinite_bool
        \bool_set_true:N \l__acro_upper_indefinite_bool
      }
      { \bool_set_true:N \l__acro_first_upper_bool }
  }

\cs_new_protected:Npn \acro_indefinite:
  {
    \bool_if:NTF \l__acro_first_upper_bool
      {
        \bool_set_true:N \l__acro_upper_indefinite_bool
        \bool_set_false:N \l__acro_first_upper_bool
      }
      { \bool_set_true:N \l__acro_indefinite_bool }
  }

\cs_new_protected:Npn \acro_cite:
  {
    \bool_set_true:N \l__acro_citation_all_bool
    \bool_set_true:N \l__acro_citation_first_bool
  }

\cs_new_protected:Npn \acro_no_cite:
  {
    \bool_set_false:N \l__acro_citation_all_bool
    \bool_set_false:N \l__acro_citation_first_bool
  }

\cs_new_protected:Npn \acro_index:
  { \bool_set_true:N \l__acro_addto_index_bool }

% similar macros \acro_<ending>: are defined by \acro_provide_ending:nnn

% ---------------------------------------------------------------------------
% process options:
\ProcessKeysPackageOptions {acro}

% ---------------------------------------------------------------------------
% PDF bookmark support
\cs_new:Npn \acpdfstring
  { \acro_pdf_string_short:n }

\cs_new:Npn \acpdfstringalt
  { \acro_pdf_string_alt:n }

\cs_new:Npn \acpdfstringlong
  { \acro_pdf_string_long:n }

\cs_new:Npn \acpdfstringfirst #1
  { \acpdfstringlong {#1} ~ ( \acpdfstring {#1} ) }

% TODO: place this somewhere where endings are defined:
\cs_new:Npn \acpdfstringlongplural
  { \acro_pdf_string_long_plural:n }

\prg_new_conditional:Npnn \acro_if_star_gobble:n #1 {TF}
  {
    \if_meaning:w *#1
      \prg_return_true:
    \else:
      \prg_return_false:
    \fi:
  }

\cs_new:Npn \acro_expandable_long:n #1
  { \prop_item:Nn \l__acro_long_prop {#1} }

\cs_new:Npn \acro_expandable_long_plural:n #1
  {
    \bool_if:nTF
      { \prop_item:Nn \l__acro_long_plural_form_prop {#1} }
      { \prop_item:Nn \l__acro_long_plural_prop {#1} }
      {
        \prop_item:Nn \l__acro_long_prop {#1}
        \prop_item:Nn \l__acro_long_plural_prop {#1}
      }
  }

\cs_new:Npn \acro_pdf_string_long:n #1
  {
    \acro_if_star_gobble:nTF {#1}
      { \acro_expandable_long:n }
      { \acro_expandable_long:n {#1} }
  }

% TODO: place this somewhere where endings are defined:
\cs_new:Npn \acro_pdf_string_long_plural:n #1
  {
    \acro_if_star_gobble:nTF {#1}
      { \acro_expandable_long_plural:n }
      { \acro_expandable_long_plural:n {#1} }
  }
  
\cs_new:Npn \acro_pdf_string_short:n #1
  {
    \acro_if_star_gobble:nTF {#1}
      { \prop_item:Nn \l__acro_pdfstring_short_prop }
      { \prop_item:Nn \l__acro_pdfstring_short_prop {#1} }
  }
  
\cs_new:Npn \acro_pdf_string_alt:n #1
  {
    \acro_if_star_gobble:nTF {#1}
      { \prop_item:Nn \l__acro_pdfstring_alt_prop }
      { \prop_item:Nn \l__acro_pdfstring_alt_prop {#1} }
  }

\AtBeginDocument
  {
    \@ifpackageloaded {hyperref}
      {
        \bool_set_true:N \l__acro_hyperref_loaded_bool
        \pdfstringdefDisableCommands
          {
            \cs_set_eq:NN \ac   \acpdfstring
            \cs_set_eq:NN \Ac   \acpdfstring
            \cs_set_eq:NN \acs  \acpdfstring
            \cs_set_eq:NN \acl  \acpdfstringlong
            \cs_set_eq:NN \Acl  \acpdfstringlong
            \cs_set_eq:NN \acf  \acpdfstringfirst
            \cs_set_eq:NN \Acf  \acpdfstringfirst
            \cs_set_eq:NN \aca  \acpdfstringalt
            \cs_set_eq:NN \acp  \acpdfstringplural
            \cs_set_eq:NN \Acp  \acpdfstringplural
            \cs_set_eq:NN \acsp \acpdfstringplural
            \cs_set_eq:NN \aclp \acpdfstringlongplural
            \cs_set_eq:NN \Aclp \acpdfstringlongplural
            \cs_set_eq:NN \acfp \acpdfstringplural
            \cs_set_eq:NN \Acfp \acpdfstringplural
            \cs_set_eq:NN \acap \acpdfstringaltplural
          }
        \cs_set_protected:Npn \acro_hyper_page:n #1 { \hyperpage {#1} }
      } {}
  }

% --------------------------------------------------------------------------
% additional variables:
\tl_new:N \l__acro_current_property_tl

% --------------------------------------------------------------------------
% key and order checking
\msg_new:nnn {acro} {no-id}
  {
    Something~ has~ gone~ wrong,~ you've~ probably~ forgotten~ to~ set~ the~
    acronym~ ID.
  }

\msg_new:nnn {acro} {before-short}
  {
    You've~ set~ the~ property~ `#2'~ before~ the~ `short'~ property~ for~
    acronym~ `#1'~ but~ it~ needs~ to~ be~ set~ after~ it.
  }

\msg_new:nnn {acro} {missing}
  { The~ `#2'~ property~ for~ acronym~ `#1'~ is~ missing. }

\msg_new:nnn {acro} {doubled-property}
  {
    It~ seems~ to~ me~ you~ have~ used~ the~ `#2'~ property~ twice~ in~ the~
    declaration~ of~ acronym~ `#1'.~ If~ you~ haven't~ there's~
    something~ different~ wrong~ and~ I'm~ lost.~ You~'re~ on~ your~ own~
    then.
  }

\cs_new_protected:Npn \__acro_property_check:nn #1#2
  {
    \tl_if_blank:VT \l__acro_current_property_tl
      { \acro_serious_message:n {no-id} }
    \bool_if:cF { l__acro_#1_short_set_bool }
      {
        \keys_set:nn { acro / declare-acronym } { short = {#1} }
        \acro_harmless_message:nn {substitute-short} {#1}
      }
    \bool_new:c { l__acro_#1_#2_set_bool }
    \bool_set_true:c { l__acro_#1_#2_set_bool }
  }

\cs_new_protected:Npn \__acro_first_property_check:nn #1#2
  {
    \cs_if_exist:cTF { l__acro_#1_short_set_bool }
      {
         \bool_if:cT { l__acro_#1_short_set_bool }
           { \acro_serious_message:nnn {doubled-property} {#1} {#2} }
      }
      {
        \bool_new:c { l__acro_#1_short_set_bool }
        \bool_set_true:c { l__acro_#1_short_set_bool }
      }
  }

% --------------------------------------------------------------------------
% the internal property selection functions for \DeclareAcronym:

% #1: name in associated cs
% #2: property name
% #3: action
\cs_new_protected:Npn \acro_declare_property_generic:nnn #1#2#3
  {
    \prop_clear_new:c {l__acro_#1_prop}
    \cs_new_protected:cpn   {__acro_declare_#1:nn} ##1##2 {#3}
    \cs_generate_variant:cn {__acro_declare_#1:nn} {V}
    \keys_define:nn {acro/declare-acronym}
      {
        #2 .code:n =
          \use:c {__acro_declare_#1:Vn} \l__acro_current_property_tl {##1}
      }
  }

% #1: name in associated cs
% #2: property name
% #3: action
\cs_new_protected:Npn \acro_declare_property:nnn #1#2#3
  {
    \acro_declare_property_generic:nnn {#1} {#2}
      {
        \__acro_property_check:nn {##1} {#2}
        \prop_put:cnn {l__acro_#1_prop} {##1} {##2}
        #3
      }
  }

% #1: name in associated cs
% #2: property name
\cs_new_protected:Npn \acro_declare_property:nn #1#2
  { \acro_declare_property:nnn {#1} {#2} {} }
\cs_generate_variant:Nn \acro_declare_property:nn { V }

\cs_new_protected:Npn \acro_declare_simple_property:n #1
  {
    \tl_set:Nn \l__acro_tmpa_tl {#1}
    \tl_replace_all:Nnn \l__acro_tmpa_tl {-} {_}
    \tl_clear_new:c  {l__acro_ \l__acro_tmpa_tl _tl}
    \acro_declare_property:Vn \l__acro_tmpa_tl {#1}
  }

% #1: new alias property
% #2: old property
\cs_new_protected:Npn \acro_declare_property_alias:nn #1#2
  {
    \keys_define:nn {acro/declare-acronym}
      { #1 .meta:n = { #2 = {##1} } }
  }

% --------------------------------------------------------------------------
% declare the properties for \DeclareAcronym:
% short:
\acro_declare_property_generic:nnn {short} {short}
  {
    \__acro_first_property_check:nn {#1} {short}
    \prop_put:Nnn \l__acro_short_prop      {#1} {#2}
    \prop_put:Nnn \l__acro_sort_prop       {#1} {#1}
    \prop_put:Nnn \l__acro_index_sort_prop {#1} {#1}
    \prop_put:Nnn \l__acro_alt_prop        {#1} {#2}
    \prop_put:Nnn \l__acro_pdfstring_short_prop {#1} {#2}
    \prop_put:Nnn \l__acro_pdfstring_alt_prop {#1} {#2}
    \acro_for_endings_do:n
      {
        \prop_put:cnv {l__acro_short_##1_prop}
          {#1} {l__acro_default_short_##1_tl}
        \prop_put:cnx {l__acro_pdfstring_short_##1_prop}
          {#1} { \exp_not:n {#2} \exp_not:v {l__acro_default_short_##1_tl} }
        \prop_put:cnn {l__acro_short_##1_form_prop} {#1} { \c_false_bool }
        \prop_put:cnv {l__acro_alt_##1_prop}
          {#1} {l__acro_default_alt_##1_tl}
        \prop_put:cnx {l__acro_pdfstring_alt_##1_prop}
          {#1} { \exp_not:n {#2} \exp_not:v {l__acro_default_short_##1_tl} }
        \prop_put:cnn {l__acro_alt_##1_form_prop} {#1} { \c_false_bool }
      }
    \prop_put:NnV \l__acro_short_indefinite_prop
      {#1} \l__acro_default_indefinite_tl
    \prop_put:NnV \l__acro_alt_indefinite_prop
      {#1} \l__acro_default_indefinite_tl
  }

% long:
\acro_declare_property:nnn {long} {long}
  {
    \acro_for_endings_do:n
      { \prop_put:cnn {l__acro_long_##1_form_prop} {#1} { \c_false_bool } }
    \prop_put:NnV \l__acro_long_indefinite_prop
      {#1}
      \l__acro_default_indefinite_tl
    \acro_for_endings_do:n
      {
        \bool_if:cF {l__acro_#1_long-##1_set_bool}
          { \prop_put:cnv {l__acro_long_##1_prop} {#1} {l__acro_default_long_##1_tl} }
      }
  }

\acro_declare_simple_property:n {first-style}

% list:
\acro_declare_simple_property:n {list}

% defines `short-plural', `long-plural' and `long-plural-form' as well as the
% options `plural-ending', `short-plural-ending' and `long-plural-ending':
% \ProvideAcroEnding {plural} {s} {s}

% short indefinite article:
\acro_declare_simple_property:n {short-indefinite}

% long indefinite article:
\acro_declare_simple_property:n {long-indefinite}

% pre long:
\acro_declare_simple_property:n {long-pre}

% post long:
\acro_declare_simple_property:n {long-post}

% sort:
\acro_declare_property:nnn {sort} {sort}
  {
    \bool_if:cF { l__acro_#1_index-sort_set_bool }
      { \prop_put:Nnn \l__acro_index_sort_prop {#1} {#2} }
  }

% alternative:
\acro_declare_property:nnn {alt} {alt}
  {
    \prop_put:Nnn \l__acro_pdfstring_alt_prop {#1} {#2}
    \prop_put:NnV \l__acro_alt_indefinite_prop
      {#1} \l__acro_default_indefinite_tl
  }

\cs_set_protected:Npn \acro_alt_error:n #1
  {
    \bool_if:cF {l__acro_#1_alt_set_bool} 
      { \acro_harmless_message:nn {no-alternative} {#1} }
  }

% alt. indefinite article:
\acro_declare_simple_property:n {alt-indefinite}

% foreign:
\acro_declare_simple_property:n {foreign}

% foreign:
\acro_declare_simple_property:n {foreign-lang}

% format:
\acro_declare_simple_property:n {format}

% short format:
\acro_declare_property_alias:nn {short-format} {format}

% long format:
\acro_declare_simple_property:n {long-format}

% first long format:
\acro_declare_simple_property:n {first-long-format}

% pdfstring -- currently needs to be done `by hand':
\prop_new:N \l__acro_pdfstring_short_prop
\cs_new_protected:Npn \__acro_declare_pdfstring:nw #1#2/#3/#4 \acro_stop:
  {
    \__acro_property_check:nn {#1} {pdfstring}
    \prop_put:Nnx \l__acro_pdfstring_short_prop {#1} {#2}
    \acro_for_endings_do:n
      {
        \tl_if_blank:nTF {#4}
          {
            \prop_put:cnx {l__acro_pdfstring_short_##1_prop}
              {#1} { \exp_not:n {#2} \exp_not:v {l__acro_default_short_##1_tl} }
          }
          {
            \prop_put:cnn {l__acro_pdfstring_short_##1_prop}
              {#1} {#2#3}
          }
      }
  }
\cs_generate_variant:Nn \__acro_declare_pdfstring:nw { V }
\keys_define:nn { acro / declare-acronym }
  {
    pdfstring    .code:n =
      \__acro_declare_pdfstring:Vw \l__acro_current_property_tl #1 // \acro_stop: ,
  }

\prop_new:N \l__acro_pdfstring_alt_prop
\cs_new_protected:Npn \__acro_declare_pdfstring_alt:nw #1#2/#3/#4 \acro_stop:
  {
    \__acro_property_check:nn {#1} { pdfstring-alt }
    \prop_put:Nnn \l__acro_pdfstring_alt_prop {#1} {#2}
    \acro_for_endings_do:n
      {
        \tl_if_empty:nTF {#3}
          {
            \prop_put:cnx {l__acro_pdfstring_alt_##1_prop}
              {#1} { \exp_not:n {#2} \exp_not:v {l__acro_default_alt_##1_tl} }
          }
          { \prop_put:cnn {l__acro_pdfstring_alt_##1_prop} {#1} {#2#3} }
      }
  }
\cs_generate_variant:Nn \__acro_declare_pdfstring_alt:nw { V }
\keys_define:nn { acro / declare-acronym }
  {
    pdfstring-alt .code:n =
      \__acro_declare_pdfstring_alt:Vw \l__acro_current_property_tl #1 // \acro_stop: ,
  }
  
% class:
\acro_declare_simple_property:n {class}

% extra information:
\acro_declare_simple_property:n {extra}

% single appearances:
\acro_declare_simple_property:n {single}

% single format:
\acro_declare_simple_property:n {single-format}

% acc supp:
\acro_declare_property:nn {acc_supp} {accsupp}

% tooltip:
\acro_declare_simple_property:n {tooltip}

% citation -- currently needs to be done `by hand':
\prop_new:N \l__acro_citation_prop
\prop_new:N \l__acro_citation_pre_prop
\prop_new:N \l__acro_citation_post_prop
\cs_new_protected:Npn \__acro_declare_citation:nw #1#2[#3]#4[#5]#6#7 \acro_stop:
  {
    % no options: #1: ID, #2: property, #3 is blank
    % 1 option:   #1: ID, #4: property, #3: option, #5 is blank
    % 2 options:  #1: ID: #6: property, #3: first option, #5: second option
    \tl_if_blank:nF { #2#4#6 }
      {
        \tl_if_empty:nTF {#3}
          { \__acro_declare_citation_aux:nnnn {#1} { } { } {#2} }
          {
            \tl_if_empty:nTF {#5}
              { \__acro_declare_citation_aux:nnnn {#1} {#3} {  } {#4} }
              { \__acro_declare_citation_aux:nnnn {#1} {#3} {#5} {#6} }
          }
      }
  }
\cs_generate_variant:Nn \__acro_declare_citation:nw { V }

\keys_define:nn { acro / declare-acronym }
  {
    cite .code:n =
      \__acro_declare_citation:Vw
        \l__acro_current_property_tl #1 [][] \scan_stop: \acro_stop:
  }

\cs_new_protected:Npn \__acro_declare_citation_aux:nnnn #1#2#3#4
  {
    \__acro_property_check:nn {#1} {cite}
    \prop_put:Nnn \l__acro_citation_prop {#1} {#4}
    \tl_if_empty:nF {#2}
      { \prop_put:Nnn \l__acro_citation_pre_prop {#1} {#2} }
    \tl_if_empty:nF {#3}
      { \prop_put:Nnn \l__acro_citation_post_prop {#1} {#3} }
  }

% TODO:
% add index entries, by default \index{<sort>@<short>}
% index: overwrite default <sort>@<short> entry completely
% index-sort: overwrite the <sort> part of <sort>@<short> entry

% need to take care of custom index cmd, at least
%  - \index{}
%  - \index[]{}
% question is, though, if it should be the same one for all acronyms?
% I go for yes but would also add a `post' property that allows to add arbitrary
% TeX code after an acronym is typeset

% index:
\acro_declare_simple_property:n {index}

% index-sort:
\acro_declare_simple_property:n {index-sort}

% index-cmd:
\acro_declare_simple_property:n {index-cmd}

% --------------------------------------------------------------------------
% acronym macros:
\cs_new_protected:Npn \__acro_define_acronym_macro:n #1
  {
    \bool_if:NT \l__acro_create_macros_bool
      {
        \cs_if_exist:cTF {#1}
          {
            \bool_if:NTF \l__acro_strict_bool
              { \cs_set:cpn {#1} { \ac {#1} \acro_xspace: } }
              { \acro_serious_message:nn {macro} {#1} }
          }
          { \cs_new:cpn {#1} { \ac {#1} \acro_xspace: } }
      }
  }

% --------------------------------------------------------------------------
% internal acronym declaring function:
\cs_new_protected:Npn \acro_declare_acronym:nn #1#2
  {
    \seq_gput_right:Nn \g__acro_declared_acronyms_seq {#1}
    \bool_gset_true:N \g__acro_first_acronym_declared_bool
    \tl_set:Nn \l__acro_current_property_tl {#1}
    \keys_set:nn { acro / declare-acronym } {#2}
    \bool_new:c {g__acro_#1_first_use_bool}
    \bool_new:c {g__acro_#1_used_bool}
    \bool_new:c {g__acro_#1_label_bool}
    \bool_new:c {g__acro_#1_in_list_bool}
    \seq_new:c  {g__acro_#1_barriers_seq}
    \seq_new:c  {g__acro_#1_recorded_barriers_seq}
    \bool_if:NF \l__acro_print_only_used_bool
      { \bool_gset_true:c {g__acro_#1_in_list_bool} }
    \__acro_create_page_records:n {#1}
    \__acro_define_acronym_macro:n {#1}
    \tl_clear:N \l__acro_current_property_tl
    \bool_if:cF {l__acro_#1_short_set_bool}
      { \acro_serious_message:nnn {missing} {#1} {short} }
    \bool_if:cF {l__acro_#1_long_set_bool}
      { \acro_serious_message:nnn {missing} {#1} {long} }
    \__acro_log_acronym:n {#1}
  }

% --------------------------------------------------------------------------
% the user command:
\NewDocumentCommand \DeclareAcronym {mm}
  { \acro_declare_acronym:nn {#1} {#2} }

% --------------------------------------------------------------------------
% print the list:
% #1: list of classes
% #2: list of excluded classes
\tl_new:N \l__acro_included_classes_tl
\tl_new:N \l__acro_excluded_classes_tl

\keys_define:nn { acro / print-acronyms }
  {
    include-classes   .tl_set:N   = \l__acro_included_classes_tl ,
    exclude-classes   .tl_set:N   = \l__acro_excluded_classes_tl ,
    name              .tl_set:N   = \l__acro_list_name_tl ,
    heading           .code:n     = \__acro_set_list_heading:n {#1} ,
    sort              .bool_set:N = \l__acro_sort_bool ,
    local-to-barriers .bool_set:N = \l__acro_use_barrier_bool
  }

\cs_new_protected:Npn \acro_print_acronyms:n #1
  {
    \group_begin:
      % this is a cheap trick to prevent the \@noitemerr
      % if one forgot to delete either the aux file or
      % remove \printacronyms -- but it's local:
      \cs_set:Npn \@noitemerr {}
      \tl_clear:N \l__acro_included_classes_tl
      \tl_clear:N \l__acro_excluded_classes_tl
      \keys_set:nn { acro / print-acronyms } {#1}
      \__acro_aux_file_now:n { \acro@print@list }
      \bool_if:NT \l__acro_sort_bool
        { \acro_sort_prop:NN \l__acro_short_prop \l__acro_sort_prop }
      \acro_title_instance:VV
        \l__acro_list_heading_cmd_tl
        \l__acro_list_name_tl
      \cs_if_exist:NTF \acro@printed@list
        {
          \acro_list_instance:VVV
            \l__acro_list_instance_tl
            \l__acro_included_classes_tl
            \l__acro_excluded_classes_tl
        }
        { \@latex@warning@no@line {Rerun~to~get~acronym~list~right} }
    \group_end:
  }

\NewDocumentCommand \printacronyms { O{} }
  { \acro_print_acronyms:n {#1} }

% --------------------------------------------------------------------------
% language support
\RequirePackage {translations}

\cs_new_protected:Npn \__acro_declare_translation:www #1 \q_mark #2=#3 \q_stop
  {
    \tl_set:Nx \l__acro_tmpa_tl { \tl_trim_spaces:n {#1} }
    \tl_set:Nx \l__acro_tmpb_tl { \tl_trim_spaces:n {#2} }
    \tl_if_in:nnT {#3} {=}
      {} % TODO: misplaced equal sign
    \tl_set:Nx \l__acro_tmpc_tl { \tl_trim_spaces:n {#3} }
    \__acro_declare_translation:VVV
      \l__acro_tmpb_tl
      \l__acro_tmpa_tl
      \l__acro_tmpc_tl
  }

% #1: key
% #2: lang
% #3: translation
\cs_new_protected:Npn \__acro_declare_translation:nnn #1#2#3
  { \DeclareTranslation {#1} {#2} {#3} }
\cs_generate_variant:Nn \__acro_declare_translation:nnn {VVV}

% #1: key
% #2: csv list: { <lang1> = <translation1> , <lang2> = <translation2> }
\cs_new_protected:Npn \acro_declare_translation:nn #1#2
  {
    \clist_map_inline:nn {#2}
      {
        \tl_if_blank:nF {##1}
          { \__acro_declare_translation:www #1 \q_mark ##1 \q_stop }
      }
  }

\NewDocumentCommand \DeclareAcroTranslation {mm}
  { \acro_declare_translation:nn {#1} {#2} }
\@onlypreamble \DeclareAcroTranslation

% tokenlists using the translations:
\tl_set:Nn \l__acro_list_name_tl  { \GetTranslation {acronym-list-name} }
\tl_set:Nn \l__acro_page_name_tl  { \GetTranslation {acronym-page-name}\@\, }
\tl_set:Nn \l__acro_pages_name_tl { \GetTranslation {acronym-pages-name}\@\, }
\tl_set:Nn \l__acro_next_page_tl  { \,\GetTranslation {acronym-next-page}\@ }
\tl_set:Nn \l__acro_next_pages_tl { \,\GetTranslation {acronym-next-pages}\@ }

% --------------------------------------------------------------------------
% definition file:
% \tl_const:Nn \c_acro_definition_file_name_tl      {acro}
% \tl_const:Nn \c_acro_definition_file_extension_tl {def}

% \file_if_exist:nTF
%   { \c_acro_definition_file_name_tl .\c_acro_definition_file_extension_tl }
%   {
%     \@onefilewithoptions
%       {\c_acro_definition_file_name_tl} [] []
%       \c_acro_definition_file_extension_tl
%   }
%   { \acro_serious_message:n {definitions-missing} }


% --------------------------------------------------------------------------
% first appearance styles:
\DeclareAcroFirstStyle {default} {inline}
  { }

\DeclareAcroFirstStyle {reversed} {inline}
  { reversed = true }

\DeclareAcroFirstStyle {short} {inline}
  {
    only-short = true ,
    brackets = false
  }

\DeclareAcroFirstStyle {long} {inline}
  {
    only-long = true ,
    brackets = false
  }

\DeclareAcroFirstStyle {square} {inline}
  { brackets-type = [] }

\DeclareAcroFirstStyle {plain} {inline}
  {
    brackets = false ,
    between = --
  }

\DeclareAcroFirstStyle {plain-reversed} {inline}
  { 
    brackets = false ,
    between = -- ,
    reversed = true
  }

\DeclareAcroFirstStyle {footnote} {note}
  { }

\DeclareAcroFirstStyle {footnote-reversed} {note}
  { reversed = true }

\DeclareAcroFirstStyle {sidenote} {note}
  { note-command = \sidenote {#1} }

\DeclareAcroFirstStyle {sidenote-reversed} {note}
  {
    note-command = \sidenote {#1} ,
    reversed     = true
  }

\DeclareAcroFirstStyle {empty} {note}
  { use-note = false }

% --------------------------------------------------------------------------
% extra info appearance styles:
\DeclareAcroExtraStyle {default} {inline}
  {
    brackets     = false ,
    punct        = true ,
    punct-symbol = .
  }

\DeclareAcroExtraStyle {plain} {inline}
  {
    brackets     = false ,
    punct        = true ,
    punct-symbol =
  }

\DeclareAcroExtraStyle {paren} {inline}
  {
    brackets     = true ,
    punct        = true ,
    punct-symbol =
  }

\DeclareAcroExtraStyle {bracket} {inline}
  {
    brackets      = true ,
    punct         = true ,
    punct-symbol  = ,
    brackets-type = []
  }

\DeclareAcroExtraStyle {comma} {inline}
  {
    punct         = true,
    punct-symbol  = {,} ,
    brackets      = false
  }

% --------------------------------------------------------------------------
% page number appearance styles:
\DeclareAcroPageStyle {default} {inline}
  {
    punct = true ,
    punct-symbol = .
  }
  
\DeclareAcroPageStyle {plain}   {inline}
  { punct = false }

\DeclareAcroPageStyle {comma}   {inline}
  { punct = true }

\DeclareAcroPageStyle {paren}   {inline}
  {
    brackets=true ,
    punct-symbol = ~
  }

\DeclareAcroPageStyle {none}    {inline}
  { display = false }

% --------------------------------------------------------------------------
% list heading styles:
\DeclareAcroListHeading {part}           {\part}
\DeclareAcroListHeading {part*}          {\part*}
\DeclareAcroListHeading {chapter}        {\chapter}
\DeclareAcroListHeading {chapter*}       {\chapter*}
\DeclareAcroListHeading {addchap}        {\addchap}
\DeclareAcroListHeading {section}        {\section}
\DeclareAcroListHeading {section*}       {\section*}
\DeclareAcroListHeading {addsec}         {\addsec}
\DeclareAcroListHeading {subsection}     {\subsection}
\DeclareAcroListHeading {subsection*}    {\subsection*}
\DeclareAcroListHeading {subsubsection}  {\subsubsection}
\DeclareAcroListHeading {subsubsection*} {\subsubsection*}
\DeclareAcroListHeading {none}           {\use_none:n}

% --------------------------------------------------------------------------
% list styles:
\DeclareAcroListStyle {description} {list}
  { }

\DeclareAcroListStyle {toc} {list-of}
  { }

\DeclareAcroListStyle {lof} {list-of}
  { style = lof }

\DeclareAcroListStyle {tabular} {table}
  { table = tabular }

\DeclareAcroListStyle {longtable} {table}
  { table = longtable }

\DeclareAcroListStyle {extra-tabular} {extra-table}
  { table = tabular }

\DeclareAcroListStyle {extra-longtable} {extra-table}
  { table = longtable }

\DeclareAcroListStyle {extra-tabular-rev} {extra-table}
  {
    table   = tabular ,
    reverse = true
  }

\DeclareAcroListStyle {extra-longtable-rev} {extra-table}
  {
    table   = longtable ,
    reverse = true
  }

% --------------------------------------------------------------------------
% register some tokens to be checked for:
\AcroRegisterTrailing . {dot}
\AcroRegisterTrailing - {dash}
\AcroRegisterTrailing \babelhyphen {babel-hyphen}

% --------------------------------------------------------------------------
% the user commands
% automatic:
\NewAcroCommand \ac
  { \acro_use:n {#1} }

\NewAcroCommand \iac
  {
    \acro_indefinite:
    \acro_use:n {#1}
  }

\NewAcroCommand \Iac
  {
    \acro_first_upper:
    \acro_indefinite:
    \acro_use:n {#1}
  }

\NewAcroCommand \Ac
  {
    \acro_first_upper:
    \acro_use:n {#1}
  }

\NewAcroCommand \acp
  {
    \acro_plural:
    \acro_use:n {#1}
  }

\NewAcroCommand \Acp
  {
    \acro_plural:
    \acro_first_upper:
    \acro_use:n {#1}
  }

\NewAcroCommand \acsingle
  {
    \acro_get:n {#1}
    \acro_single:n {#1}
  }

\NewAcroCommand \Acsingle
  {
    \acro_get:n {#1}
    \acro_first_upper:
    \acro_single:n {#1}
  }

% short:
\NewAcroCommand \acs
  { \acro_short:n {#1} }

\NewAcroCommand \iacs
  {
    \acro_indefinite:
    \acro_short:n {#1}
  }

\NewAcroCommand \Iacs
  {
    \acro_first_upper:
    \acro_indefinite:
    \acro_short:n {#1}
  }

\NewAcroCommand \acsp
  {
    \acro_plural:
    \acro_short:n {#1}
  }

% alt:
\NewAcroCommand \aca
  { \acro_alt:n {#1} }

\NewAcroCommand \Aca
  {
    \acro_first_upper:
    \acro_alt:n {#1}
  }
  
\NewAcroCommand \iaca
  {
    \acro_indefinite:
    \acro_alt:n {#1}
  }

\NewAcroCommand \Iaca
  {
    \acro_first_upper:
    \acro_indefinite:
    \acro_alt:n {#1}
  }

\NewAcroCommand \acap
  {
    \acro_plural:
    \acro_alt:n {#1}
  }

% long:
\NewAcroCommand \acl
  { \acro_long:n {#1} }

\NewAcroCommand \iacl
  {
    \acro_indefinite:
    \acro_long:n {#1}
  }

\NewAcroCommand \Iacl
  {
    \acro_first_upper:
    \acro_indefinite:
    \acro_long:n {#1}
  }

\NewAcroCommand \Acl
  {
    \acro_first_upper:
    \acro_long:n {#1}
  }

\NewAcroCommand \aclp
  {
    \acro_plural:
    \acro_long:n {#1}
  }

\NewAcroCommand \Aclp
  {
    \acro_plural:
    \acro_first_upper:
    \acro_long:n {#1}
  }

% first:
\NewAcroCommand \acf
  { \acro_first:n {#1} }

\NewAcroCommand \iacf
  {
    \acro_indefinite:
    \acro_first:n {#1}
  }

\NewAcroCommand \Iacf
  {
    \acro_first_upper:
    \acro_indefinite:
    \acro_first:n {#1}
  }

\NewAcroCommand \Acf
  {
    \acro_first_upper:
    \acro_first:n {#1}
  }

\NewAcroCommand \acfp
  {
    \acro_plural:
    \acro_first:n {#1}
  }

\NewAcroCommand \Acfp
  {
    \acro_plural:
    \acro_first_upper:
    \acro_first:n {#1}
  }

% first-like:
\NewPseudoAcroCommand \acflike
  { \acro_first_like:nn {#1} {#2} }

\NewPseudoAcroCommand \iacflike
  {
    \acro_indefinite:
    \acro_first_like:nn {#1} {#2}
  }

\NewPseudoAcroCommand \Iacflike
  {
    \acro_first_upper:
    \acro_indefinite:
    \acro_first_like:nn {#1} {#2}
  }

\NewPseudoAcroCommand \acfplike
  {
    \acro_plural:
    \acro_first_like:nn {#1} {#2}
  }

% --------------------------------------------------------------------------
% endings:
\ProvideAcroEnding {plural} {s} {s}

% --------------------------------------------------------------------------
% translations:
% list name
\DeclareAcroTranslation {acronym-list-name}
  {
    Fallback   = Acronyms ,
    English    = Acronyms ,
    French     = Acronymes ,
    German     = Abk\"urzungen ,
    Italian    = Acronimi ,
    Portuguese = Acr\'onimos ,
    Spanish    = Siglas ,
    Catalan    = Sigles ,
    Turkish    = K\i saltmalar
  }

% page name
\DeclareAcroTranslation {acronym-page-name}
  {
    Fallback   = p. ,
    English    = p. ,
    German     = S. ,
    Portuguese = p.
  }

% pages name
\DeclareAcroTranslation {acronym-pages-name}
  {
    Fallback   = pp. ,
    English    = pp. ,
    German     = S. ,
    Portuguese = pp.
  }

% following page
\DeclareAcroTranslation {acronym-next-page}
  {
    Fallback   = f. ,
    English    = f. ,
    German     = f. ,
    Portuguese = s.
  }

% following pages
\DeclareAcroTranslation {acronym-next-pages}
  {
    Fallback   = ff. ,
    English    = ff. ,
    German     = ff. ,
    Portuguese = ss.
  }

% --------------------------------------------------------------------------
% allow for a configuration file:
\tl_const:Nn \c_acro_config_file_name_tl      {acro}
\tl_const:Nn \c_acro_config_file_extension_tl {cfg}

\file_if_exist:nT
  { \c_acro_config_file_name_tl .\c_acro_config_file_extension_tl }
  {
    \@onefilewithoptions
      {\c_acro_config_file_name_tl} [] []
      \c_acro_config_file_extension_tl
  }

\tex_endinput:D
% --------------------------------------------------------------------------
% HISTORY:
2012/06/22 v0.1  - first public release
2012/06/23 v0.1a - bug fix, added `strict' and `macros' option and creation of
                   shortcut macros
                 - added capitalized version of long forms
                 - added `sort' option
2012/06/24 v0.1b - added \Acf and \Acfp, added option `plural-ending'
2012/06/24 v0.1c - added excluded argument to \printacronyms
2012/06/24 v0.2  - renamed \NewAcronym => \DeclareAcronym
                   \AcronymFormat => \DeclareAcronymFormat
2012/06/25 v0.2a - new first-style's: `short' and `reversed'
2012/06/25 v0.3  - new list formats: extra-tabular, extra-longtable,
                   extra-tabular-rev, extra-longtable-rev
                 - extra precaution when using \printacronyms to avoid errors.
2012/06/27 v0.3a - new option `list-caps', \Acp added
2012/06/29 v0.3b - extended the `text' template to the `acro-first' object
                 - added `acro-first' instances `plain' and `plain-reversed'
2012/07/16 v0.3c - small adjustments to the documentation
2012/07/23 v0.3d - first CTAN version
2012/07/24 v0.3e - adapted to updated l3kernel
2012/09/28 v0.4  - added means to add citations to acronyms
2012/10/07 v0.4a - new options: "uc-cmd", "list-long-format"
                 - preliminary language support, needs package `translations'
2012/11/30 v0.5  - added starred variants of the commands that won't mark an
                   acronym as used
                 - added \acreset{<id>}
                 - added preliminary support for pdf strings: in pdf strings
                   always the singular lowercase short version is inserted
                   (the equivalent of \acs)
2012/12/14 v0.6  - bug with not-colored links resolved
                 - bug introduced with the last update (full expansion of the
                   short entry) resolved
                 - option `xspace' added
2013/01/02 v0.6a - \acuseall
2013/01/16 v1.0  - new syntax of \DeclareAcronym
                 - new option `version'
                 - new `accsupp' acronym property
                 - new `sort' acronym property
                 - new syntax of \printacronyms
                 - new default: `sort=true' 
                 - new options `page-ranges', `next-page', `next-pages',
                   `pages-name', `record-pages'
                 - no automatic label placement for page number referencing
                   any more
2013/01/26 v1.1  - bug fix in the plural detection
                 - new keys `long-pre' and `long-post'
                 - new keys `index', `index-sort' and `index-cmd'
                 - new options `index' and `index-cmd'
2013/01/29 v1.1a - added `long-format' key
                 - renamed `format' key into `short-format', kept `format' for
                   compatibility reasons
2013/02/09 v1.2  - error message instead of hanging when an undefined acronym
                   is used
                 - added `first-long-format' key and `first-long-format' option
                 - added \acflike and \acfplike
                 - improvements and bug fixes to the page recording mechanism
                 - new option `mark-as-used'
                 - new keys: `short-indefinite', `alt-indefinite' and
                   `long-indefinite'
                 - new commands: \iac, \Iac, \iacs, \Iacs, \iaca, \Iaca, \iacl,
                   \Iacl, \iacf, \Iacf, \iacflike and \Iacflike
2013/04/04 v1.2a - added Portuguese translations
2013/05/06 v1.3  - protected internal commands where appropriate
                 - new option `sort' to \printacronyms
                 - renamed options `print-acronyms/header' and `list-header'
                   into `print-acronyms/heading' and `list-heading'
                 - fix: added missing group to \printacronyms
                 - add key `foreign'
                 - rewritten page-recording:
                   * most importantly: record them at shipout; this is done
                     when \acro@used@once or \acro@used@twice are written to
                     the aux file
                   * no restrictions regarding \pagenumbering
                   * options `page-ranges' and `record-pages' are deprecated
                   * new options `following-page' and `following-pages'
                 - disable \@noitemerr in the list of acronyms: we don't need
                   it there but there are occasions when it is annoying
                 - cleaned the sty file, added a few more comments
2013/05/09 v1.3a - Bug fix: corrected wrong argument checking in \Ac, thanks
                   to Michel Voßkuhle
2013/05/30 v1.3b - obey \if@filesw
2013/06/16 v1.3c - added \leavevmode to \acro_get:n
2013/07/08 v1.3d - corrected wrong call of \leavevmode in the list
                   (list-type=list)
2013/08/07 v1.3e - bug fix in the list when testing for used acronyms
                 - new commands \acifused, \acfirstupper
2013/08/27 v1.4  - new property `list'
2013/09/02 v1.4a - bug fix: used acronyms are added to the list when the list
                   is printed before the use
                 - \DeclareAcronym may now be used after \begin{document}
2013/09/24 v1.4b - bug fix: only-used=false works again for only declared but
                   unused acronyms (only if option single is not used)
2013/11/04 v1.4c - remove \hbox from the written short form
                 - changed \__acro_make_link:nNN in a way that it doesn't box
                   its when links are deactivated
2013/11/22 v1.4d - require `l3sort' independently from the `sort' option
                   instead of at begin document in order to avoid conflicts
                   with `babel' and `french'
2013/12/18 v1.5  - new option `label=true|false' that
                   places \label{<prefix>:<id>} the first time an acronym is
                   used
                 - new option `pages=first|all' that determines if in the list
                   of acronyms all appearances are listed or only the first
                   time; implicitly sets `label=true'
2015/02/26 v1.6  - new `acro-title' instance `none'
                 - change of expl3's tl uppercasing function (adapt to updates
                   of l3kernel and friends
                 - new package option `messages=silent|loud'
                 - fix issue https://bitbucket.org/cgnieder/acro/issue/23/
                 - fix issue https://bitbucket.org/cgnieder/acro/issue/24/
                 - fix issue https://bitbucket.org/cgnieder/acro/issue/28/
                 - drop support for version 0
2015/04/08 v1.6a - more generalized user command definitions, see
                   http://tex.stackexchange.com/q/236362/ for an application
2015/05/10 v1.6b - \ProcessKeysPackageOptions ,
                 - correct bug http://tex.stackexchange.com/q/236860/ :
                   option `pages = first' works again
2015/08/16 v2.0  - fix https://bitbucket.org/cgnieder/acro/issue/36
                 - implement https://bitbucket.org/cgnieder/acro/issue/39
                 - implement https://bitbucket.org/cgnieder/acro/issue/40
                   (=> new option `group-cite-cmd')
                 - add ideas for https://bitbucket.org/cgnieder/acro/issue/41
                 - implement https://bitbucket.org/cgnieder/acro/issue/18
                 - implement https://bitbucket.org/cgnieder/acro/issue/43
                 - further generalization for defining user commands:
                   \NewAcroCommand, \NewPseudoAcroCommand and siblings
                 - bug fix in indefinite versions with first-upper
                 - add `short-<ending>-form' equivalent to
                   `long-<ending>-form'
                   (https://bitbucket.org/cgnieder/acro/issue/44)
                 - implement https://bitbucket.org/cgnieder/acro/issue/35
                 - new option `single-form'
2015/08/25 v2.0a - fix https://bitbucket.org/cgnieder/acro/issue/38 and
                   https://bitbucket.org/cgnieder/acro/issue/49
2015/08/29 v2.0b - fix https://bitbucket.org/cgnieder/acro/issue/44
                 - fix https://bitbucket.org/cgnieder/acro/issue/45
                 - implement https://bitbucket.org/cgnieder/acro/issue/42
2015/09/05 v2.1  - add list object type `list-of' that prints the list like a
                   toc or lof, new option `list-short-width',
                 - correct bug in the `plain' extra style
                 - implemented `tooltip' property
                 - remove \tl_to_lowercase:n
2015/10/03 v2.2  - fix https://bitbucket.org/cgnieder/acro/issue/52
                 - fix typo in `list-of' object
                 - \DeclareAcroListStyle
                 - \DeclareAcroListHeading
                 - input `acro.cfg' if present
                 - all acro commands have an optional argument: \ac*[]{}
2016/01/07 v2.2a - \prop_get:Nn => \prop_item:Nn
2016/01/21 v2.2b - fix issue #59
2016/02/02 v2.2c - fix issue #60
2016/03/14 v2.3  - foreign-format may be a macro taking an argument
                 - \Aca, \Acsingle
                 - properties `single' and `single-format', option
                   `single-format' => issue #62
                 - property `first-style' => issue #61
                 - fix issue #64: long-post entry is now added *after* the
                   endings
                 - property `foreign-lang'
                 - fix issue #65
2016/03/25 v2.4  - extend class mechanism: classes can be used like tags
                 - add idea of `barriers' and list local to those barriers
                   => new option `reset-at-barriers'
                   => new option `local-to-barriers' for \printacronyms
                   => new command \acbarrier
2016/04/14 v2.4a - if undefined acronym is used and `messages = silent' is
                   active don't through error
2016/05/03 v2.4b - expand `pdfstring' property before saving => issue #69
                 - \ProvideAcroEnding can be used twice – it then just sets
                   the defaults
                 - the option <ending>-ending has a new syntax:
                   * <ending>-ending = <val> sets all endings to <val>
                   * <ending>-ending = <val1>/<val2> sets short endings to
                     <val1> and long endings to <val2>
                 - a single appearance should be treated like a first
                   appearance as far as citations are concerned
2016/05/25 v2.5  - some of the entries added to the aux file need to be
                   written \immediate in order to avoid this trap:
                   http://tex.stackexchange.com/q/116001/
                 - cleaner interface for first-style template definitions
                 - new `acro-first' instances `footnote-reversed' and
                   `sidenote-reversed'
                 - new commands \DeclareAcroFirstStyle, \DeclareAcroExtraStyle
                   and \DeclareAcroPageStyle
                 - add warning `ending-before-acronyms' to options setting the
                   defaults of an ending; this should avoid confusion
                 - property declaration for acronyms should be handled by
                   internal functions
                 - improvements in the list template code
                 - logging info when an acronym is declared
                 - remove deprecated options
                 - new option `use-barriers'
                 - new option `following-pages*'
                 - option `page-ref' replaced by option `page-style'
2016/05/26 v2.5a - bug fix: remove erroneous group in `<ending>-ending' option
2016/05/30 v2.5b - fix issue #72
2016/07/20 v2.6  - \l__acro_use_acronyms_bool can be used to disable \ac
                   e.g. in the trial phase of a table like `tabu'; interface:
                   \acro_switch_off: and \acswitchoff
                 - fix issue #79
                 - fix issue #74
                 - fix error: acronyms with same sort entry are not
                   overwritten any more in the list of acronyms
2016/08/13 v2.6a - fix issues #80 and #81
2016/08/13 v2.6b - version stepped by accident
2016/08/16 v2.6c - really fixes issue #81
2016/08/30 v2.6d - fix issue #82
2016/09/04 v2.6e - fix issue in http://tex.stackexchange.com/q/270034/
2017/01/22 v2.7  - rename \acro_get_property:nn into \__acro_get_property:nn
                 - \acro_get_property:nn, \acro_get_property:nnTF,
                   \acro_if_property:nnTF, retrieve property without error if
                   not set
                 - make \__acro_declare_property functions public
                 - \acro_add_action:n (adds code to \acro_get:n)
2017/02/09 v2.7a - adapt to integration of l3sort into l3kernel

% --------------------------------------------------------------------------
% TODO:
- extend option `macros' to also define uppercase macros, possibly as a choice
- add \ACF, \ACFP, \ACL and \ACLP that will print all words of the long form
  capitalized
           % A discussion on generating PDF files.
\end{appendices}         % End of the Appendix Chapters.  ibid on \end{appendix}
%\NeedsTeXFormat{LaTeX2e}
\ProvidesClass{vita}[1996/10/09
                     class file ``vita'' to create Curriculum Vitae]
%%%%%%%%%%%%%%%%%%%%%%%%%%%%%%%%%%%%%%%%%%%%%%%%%%%%%%%%%%%%%%%%%%%%%%%
%%
%% (C) Copyright 1995, Andrej Brodnik, ABrodnik@UWaterloo.CA. All
%% rights reserved.
%%
%% This is a generated file. Permission is granted to to customize the 
%% declarations in this file to serve the needs of your installation. 
%% However, no permission is granted to distribute a modified version of 
%% this file under its original name. 
%% 
%% \CharacterTable
%%  {Upper-case    \A\B\C\D\E\F\G\H\I\J\K\L\M\N\O\P\Q\R\S\T\U\V\W\X\Y\Z
%%   Lower-case    \a\b\c\d\e\f\g\h\i\j\k\l\m\n\o\p\q\r\s\t\u\v\w\x\y\z
%%   Digits        \0\1\2\3\4\5\6\7\8\9
%%   Exclamation   \!     Double quote  \"     Hash (number) \#
%%   Dollar        \$     Percent       \%     Ampersand     \&
%%   Acute accent  \'     Left paren    \(     Right paren   \)
%%   Asterisk      \*     Plus          \+     Comma         \,
%%   Minus         \-     Point         \.     Solidus       \/
%%   Colon         \:     Semicolon     \;     Less than     \<
%%   Equals        \=     Greater than  \>     Question mark \?
%%   Commercial at \@     Left bracket  \[     Backslash     \\
%%   Right bracket \]     Circumflex    \^     Underscore    \_
%%   Grave accent  \`     Left brace    \{     Vertical bar  \|
%%   Right brace   \}     Tilde         \~}
%%
%%---

%%%%%%%%%%%%%%%%%%%%%%%%%%%%%%%%%%%%%%%%%%%%%%%%%%%%%%%%%%%%%%%%%%%%%%%
%
%    - based on vita.sty by kcb@hss.caltech.edu
%    - 1995/02/07: the first version
%    - 1996/10/09: if there is no business address the field is
%                  left out
%
% User documentation: This class file only provides basic definitions
% =================== of environments, which are then used in class
% option files to instantiate entries for different disciplines. Thus,
% create your document as follows:
%
%    \documentclass[<discipline>]{vita}
%    \begin{document}
%      \name{Andrej Brodnik}
%      \businessAddress{First line \\ second line of bussines address}
%      \homeAddress{Again \\ multiline address \\ perhaps with phone number}
%    \begin{vita}
%      % here comes a real Curriculum Vitae for particular <discipline>
%    \end{vita}
%    \end{document}
%
% where it is assumed that file ``vita<discipline>.clo'' exists and defines
% proper categories used in given discipline. For detail explanation on
% categories in different disciplines see individual ``.clo'' files.
%
% The output will have format:
%
%   o on the first page will appear a title ``Curriculum Vitae'' (to
%     change it, see below under i18n notes -- internationalization)
%   o below will be your name
%   o below, side by side, your business and home address headed
%     by strings ``Business address'' and ``Home address''
%     respectively (to change these strings see below in i18n notes).
%   o then will follow the rest of CV as defined by ``<discipline>.clo''
%     file.
%   o the header of each but first page will include your name and the
%     page number.
%   o on the last page in the bottom right you will have the current
%     date, that is month and year (to change this, see below under
%     i18n notes).
%
%------
%
% i18n NOTES: If you are making CV for some other language, you have to
% =========== redefine:
%   - title:
%       o use command:   ``\title{<new title>}''
%       o default value: ``Curriculum Vitae''
%   - date:   
%       o use command:   ``\today{<date})''
%       o default value: ``<current month>, <current year>'' (in English)
%   - addresses headers:
%       o use command:   ``\HeaderBusiness{<new header>}''
%                        ``\HeaderHome{<new header>}''
%       o default value: ``Business address''
%                        ``Home address''
%
%------
%
% System documentation: class ``vita'' is based on the class
% ===================== ``article''. It changes the title into
% <default value> (see i18n notes) and the name becomes an
% author. Individual categories, publications and references are
% implemented using ``description'' environment. 
%
%----------------------------------------

%%%%
%
% Process options and load class article:
%---
\let\@optionsToInput=\@empty
\DeclareOption*{
  \IfFileExists{vita\CurrentOption.clo}%
    {\edef\@optionToInput{vita\CurrentOption.clo}}%
    {\PassOptionsToClass{\CurrentOption}{article}}
}
\ProcessOptions
\LoadClass{article}

%%%%
%
% First all i18n definitions:
%---
\title{Curriculum Vitae}
\renewcommand{\today}{
  \ifcase\month\or
    January\or February\or March\or April\or May\or June\or
    July\or August\or September\or October\or November\or December\fi,
  \space\number\year}
\newcommand\HeaderBusiness[1]{\def\@businessAddressHeader{#1}}
  \HeaderBusiness{Business Address}
\newcommand\HeaderHome[1]{\def\@homeAddressHeader{#1}}
  \HeaderHome{Home Address}

%%%%
%
% Next, header definitions:
%---
\date{\relax}
\newcommand{\name}[1]{
  \renewcommand{\@author}{#1} \markright{\protect\small\@author}
}
\newcommand{\businessAddress}[1]{\def\@businessAddress{#1}}
  \businessAddress{}
\newcommand{\homeAddress}[1]{\def\@homeAddress{#1}}
  \homeAddress{}

%%%%
%
% \maketitle command, which prints out the title and the name of person
%---
\renewcommand{\maketitle}{\newpage
  \global\@topnum\z@   % Prevents figures from going at top of page.
  \begin{center}
    {\LARGE \@title}

    \medskip

    {\large \@author}
  \end{center}

  \bigskip

  \thispagestyle{plain}

  \gdef\@author{}\gdef\@title{}
}

%%%%
%
% ``vita'' environment:
%---
\pagestyle{empty}
\newenvironment{vita}{
     % first page is empty style though the following pages have on the
     % right side written the name from the \name command
  \ifx\@author\@empty\@warning{Missing name command}\fi
     % next we start to layout information. First the title and the
     % name,

  \maketitle
     % followed by both addresses,
  \begin{tabular*}{\textwidth}{@{\extracolsep{\fill}}ll@{}}
    \begin{tabular}[t]{@{}l@{}}
    \ifx\@businessAddress\@empty\mbox{}\else
       {\small \@businessAddressHeader:}
    \\ \@businessAddress
    \fi
    \end{tabular}
  &
    \ifx\@homeAddress\@empty\@warning{Missing home address}%
    \else
      \begin{tabular}[t]{@{}l@{}}
         {\small \@homeAddressHeader:}
      \\ \@homeAddress
      \end{tabular}
    \fi
  \end{tabular*}

  \bigskip

  \thispagestyle{empty}
}{   % quite at the bottom of last page we have a date
  \par\nopagebreak\vfill\hfill \today
}%end vita environment

%%%%
%
% Curriculum vitae consists of categories which we create using
% command:
%
%      \newcategory[The name]{The label}
%
% where <The name> is written in bold character as a small title of
% category. It appears at the left margine of a page. If <The name>
% parameter is missing, it takes the same value as <The label>, which,
% in turn is used to refer to individual category. For example
% commands:
%
%      \newcategory{Name of category}
%      \newcategory[Name of category]{Name of category}
%
% have the same result. Now, to use category:
%
%      \newcategory[Some category]{some other name}
%
% the input has form:
%
% \begin{some other name}
%   \item The first item
%   \item The second one etc.
% \end{some other name}
%
% and the category will have on the output title ``Some category''.
% Entries in each category are preceded by \item.
%
%-----
% i18n NOTE: One can use as the names of categories strings in
% ========== different languages, but the labels can be the same in
% the same language, which is useful if you have a single CV and you
% want outputs in different languages.
%---
\def\@newCategory[#1]#2{%
  \newenvironment{#2}{\medskip\pagebreak[2]\par
    \textbf{\small #1}\nopagebreak
    \begin{description}}{\end{description}\par}
}
\def\@noNameCategory#1{\@newCategory[#1]{#1}}
\def\newcategory{\@ifnextchar[{\@newCategory}{\@noNameCategory}}

%%%%
%
% Inside categories we have different ``kinds'' (such as different
% publications), which we create using command \newkind. It has the
% same parameters as \newcategory and all comments at command
% newcategory are also valid here.
%---
\def\@newKind[#1]#2{%
  \newenvironment{#2}{
    \pagebreak[2]
    \item \textbf{\small #1}\nopagebreak
      \begin{description}
  }{  \end{description}\par }
}
\def\@noNameKind#1{\@newKind[#1]{#1}}
\def\newkind{\@ifnextchar[{\@newKind}{\@noNameKind}}

%%%%
%
% There is a special category ``plaincategory'' which entries are
% simply listed without any indentation, and in particular, multiple
% references are separated by \and command. It can be used for
% references.
%---
\def\@newPlainCategory[#1]#2{%
  \newenvironment{#2}{
    \medskip\pagebreak[2]\par
    \textbf{\small #1}\nopagebreak
    \renewcommand{\and}{
             \end{tabular}
      \item[]\begin{tabular}[t]{l}
    }
    \begin{description}
    \item[] \begin{tabular}[t]{l}
  }{        \end{tabular}
    \end{description}\par
  }
}
\def\@noNamePlainCategory#1{\@newPlainCategory[#1]{#1}}
\def\newplaincategory{\@ifnextchar[{\@newPlainCategory}{\@noNamePlainCategory}}

%%%%
%
% Finally, formatting parameters and the possible option to input:
%---
\pagestyle{myheadings}
\parindent 0pt
\nofiles

\ifx\@optionToInput\@empty\relax
\else \input \@optionToInput
\fi
          % Optional Vita, use \begin{vita} vita text \end{vita}
\end{document}
\end{verbatim}
\end{quote}

\section{Prelude}
After the {\tt \verb|\begin{document}|} comes the preliminary information found in
theses.  In this manual, the information is kept in the file {\tt prelude.tex} (see
above).  These pages will need to be numbered with roman numerals, so use
\begin{quote}\tt\singlespace\begin{verbatim}
\clearpage\pagenumbering{roman}
\end{verbatim}\end{quote}

Next, comes your thesis or dissertation title, your name, date of graduation, department
and degree.
\begin{quote}\tt\singlespace\begin{verbatim}
\title{How to \LaTeX\ a Thesis}
\author{Eric R. Benedict}
\date{2000}
%   - The default degree is ``Doctor of Philosophy''
%     Degree can be changed using the command \degree{}
%\degree{New Degree}
%   - for a PhD dissertation (default), specify \dissertation
%\dissertation
%   - for a masters project report, specify \project
%\project
%   - for a preliminary report, specify \prelim
%\prelim
%   - for a masters thesis, specify \thesis
%\thesis
%   - The default department is ``Electrical Engineering''
%     The department can be changed using the command \department{}
%\department{New Department}
\end{verbatim}\end{quote}

If you specified the class option {\tt msthesis}, then the degree is changed to
{\em Master \break of Science} and the {\tt \verb|\thesis|} option is specified.  If you
want to have the masters margins with another document, then the {\tt \verb|\degree|}
and {\tt \verb|\dissertation|},  {\tt \verb|\project|}, {\em etc.\/} can be specified
as needed.

Once the
above are all defined, use  {\tt \verb|\maketitle|} to generate the title page.
\begin{quote}\tt\singlespace\begin{verbatim}
\maketitle
\end{verbatim}\end{quote}

If you wish to include a copyright page (see Section~\ref{copyright} for
information on registering the copyright.), then add the command
\begin{quote}\tt\singlespace\begin{verbatim}
\copyrightpage
\end{verbatim}\end{quote}
This will generate the proper copyright page and will use the name and date specified
in {\tt \verb|\author{}|} and {\tt \verb|\date{}|}.

Next are the dedications and acknowledgements:
\begin{quote}\tt\singlespace\begin{verbatim}
\begin{dedication}
To my pet rock, Skippy.
\end{dedication}

\begin{acknowledgments}
I thank the many people who have done lots of nice things for me.
\end{acknowledgments}
\end{verbatim}\end{quote}

You must tell \LaTeX{} to generate a table of contents, a list of tables and a list of
figures:
\begin{quote}\tt\singlespace\begin{verbatim}
\tableofcontents
\listoftables
\listoffigures
\end{verbatim}\end{quote}

If you wish to have a nomenclature, list of symbols or glossary it can go here.
\begin{quote}\tt\singlespace\begin{verbatim}
\begin{nomenclature}
%\begin{listofsymbols}
%\begin{glossary}
\begin{tabular}{ll}
$C_1$ & Constant 1\\
\ldots
\end{tabular}
%\end{glossary}
%\end{listofsymbols}
\end{nomenclature}
\end{verbatim}\end{quote}

If your abstract will be microfilmed by Bell and Howell (formerly UMI), then you
will need to generate an abstract of less than 350 words.  This abstract can be created
using the {\tt umiabstract} environment.  This environment requires that you define your
advisor and your advisor's title using {\tt \verb|\advisorname{}|} and
{\tt \verb|\advisortitle{}|}.
\begin{quote}\tt\singlespace\begin{verbatim}
\advisorname{Bucky J. Badger}
\advisortitle{Assistant Professor}
% ABSTRACT
\begin{umiabstract}
\noindent       % Don't indent first paragraph.
This explains the basics for using \LaTeX\ to typeset a
dissertation, thesis or project report for the University of
Wisconsin-Madison.

...

\end{umiabstract}
\end{verbatim}\end{quote}
This will place your name, title and required text at the top of the page and follow the
abstract text with your advisor's name at the bottom for your advisor's signature.  This
page is not numbered and would be submitted separately.

If you will have an abstract as part of your document, then the {\tt abstract} environment
should be used.
\begin{quote}\tt\singlespace\begin{verbatim}
\begin{abstract}
\noindent       % Don't indent first paragraph.
This explains the basics for using \LaTeX\ to typeset a
dissertation, thesis or project report for the University of
Wisconsin-Madison.

...

\end{abstract}
\end{verbatim}\end{quote}
This will generate a page number and it will be included in the Table
of Contents.  

If you will have both the UMI and regular abstracts like this document, then
you will probably want to write the abstract once and save it in a seperate
file such as {\tt abstract.tex}.  Then, you can use the same abstract for
both purposes.

\begin{quote}\begin{verbatim}
\begin{umiabstract}
  % !TEX root = main.tex
% !TEX encoding = Windows Latin 1
% !TEX TS-program = pdflatex
% 
% Archivo: abstract.tex (en ingles)


\chapter{Abstract} % No cambiar el titulo
\selectlanguage{english}
\noindent
Duis tristique sollicitudin leo nec consequat. Praesent et dui convallis velit tincidunt fermentum. Mauris cursus purus at sem viverra sed imperdiet sapien imperdiet. Aliquam mattis, elit eget rutrum vulputate, tortor sem pulvinar justo, sit amet mollis felis sem at nibh. Donec malesuada, neque id interdum eleifend, arcu augue porta elit, nec tristique libero metus at massa. Fusce fringilla laoreet rhoncus. Suspendisse potenti. Phasellus dignissim sodales mauris at pharetra. Donec gravida fringilla velit ac rutrum.

Curabitur ornare lectus id diam molestie eu imperdiet nulla tempus. Maecenas vestibulum enim et dui ornare blandit. Vivamus fermentum faucibus viverra. Maecenas at justo sapien. Aenean rhoncus augue mattis purus rhoncus venenatis. Suspendisse metus felis, porttitor in varius in, vulputate at tortor. Aliquam molestie, turpis et malesuada porta, tortor sapien pharetra sapien, ac rhoncus quam dolor a sapien. Pellentesque varius laoreet enim ut auctor. Nullam nec ultricies nisi. Nullam porta lectus et ante consectetur posuere.

Duis tristique sollicitudin leo nec consequat. Praesent et dui convallis velit tincidunt fermentum. Mauris cursus purus at sem viverra sed imperdiet sapien imperdiet. Aliquam mattis, elit eget rutrum vulputate, tortor sem pulvinar justo, sit amet mollis felis sem at nibh. Donec malesuada, neque id interdum eleifend, arcu augue porta elit, nec tristique libero metus at massa. Fusce fringilla laoreet rhoncus. Suspendisse potenti. Phasellus dignissim sodales mauris at pharetra. Donec gravida fringilla velit ac rutrum.

Duis tristique sollicitudin leo nec consequat. Praesent et dui convallis velit tincidunt fermentum. Mauris cursus purus at sem viverra sed imperdiet sapien imperdiet. Aliquam mattis, elit eget rutrum vulputate, tortor sem pulvinar justo, sit amet mollis felis sem at nibh. Donec malesuada, neque id interdum eleifend, arcu augue porta elit, nec tristique libero metus at massa. Fusce fringilla laoreet rhoncus. Suspendisse potenti. Phasellus dignissim sodales mauris at pharetra. Donec gravida fringilla velit ac rutrum.

Curabitur ornare lectus id diam molestie eu imperdiet nulla tempus. Maecenas vestibulum enim et dui ornare blandit. Vivamus fermentum faucibus viverra. Maecenas at justo sapien. Aenean rhoncus augue mattis purus rhoncus venenatis. Suspendisse metus felis, porttitor in varius in, vulputate at tortor. Aliquam molestie, turpis et malesuada porta, tortor sapien pharetra sapien, ac rhoncus quam dolor a sapien. Pellentesque varius laoreet enim ut auctor. Nullam nec ultricies nisi. Nullam porta lectus et ante consectetur posuere.

Duis tristique sollicitudin leo nec consequat. Praesent et dui convallis velit tincidunt fermentum. Mauris cursus purus at sem viverra sed imperdiet sapien imperdiet. Aliquam mattis, elit eget rutrum vulputate, tortor sem pulvinar justo, sit amet mollis felis sem at nibh. Donec malesuada, neque id interdum eleifend, arcu augue porta elit, nec tristique libero metus at massa. Fusce fringilla laoreet rhoncus. Suspendisse potenti. Phasellus dignissim sodales mauris at pharetra. Donec gravida fringilla velit ac rutrum.

\bigskip
\noindent
\textit{Key words:} first word; second word; third word.
% Separar palabras con punto-y-comas.

\checklanguage
% Fin archivo abstract.tex
\endinput 
\end{umiabstract}

\begin{abstract}
  % !TEX root = main.tex
% !TEX encoding = Windows Latin 1
% !TEX TS-program = pdflatex
% 
% Archivo: abstract.tex (en ingles)


\chapter{Abstract} % No cambiar el titulo
\selectlanguage{english}
\noindent
Duis tristique sollicitudin leo nec consequat. Praesent et dui convallis velit tincidunt fermentum. Mauris cursus purus at sem viverra sed imperdiet sapien imperdiet. Aliquam mattis, elit eget rutrum vulputate, tortor sem pulvinar justo, sit amet mollis felis sem at nibh. Donec malesuada, neque id interdum eleifend, arcu augue porta elit, nec tristique libero metus at massa. Fusce fringilla laoreet rhoncus. Suspendisse potenti. Phasellus dignissim sodales mauris at pharetra. Donec gravida fringilla velit ac rutrum.

Curabitur ornare lectus id diam molestie eu imperdiet nulla tempus. Maecenas vestibulum enim et dui ornare blandit. Vivamus fermentum faucibus viverra. Maecenas at justo sapien. Aenean rhoncus augue mattis purus rhoncus venenatis. Suspendisse metus felis, porttitor in varius in, vulputate at tortor. Aliquam molestie, turpis et malesuada porta, tortor sapien pharetra sapien, ac rhoncus quam dolor a sapien. Pellentesque varius laoreet enim ut auctor. Nullam nec ultricies nisi. Nullam porta lectus et ante consectetur posuere.

Duis tristique sollicitudin leo nec consequat. Praesent et dui convallis velit tincidunt fermentum. Mauris cursus purus at sem viverra sed imperdiet sapien imperdiet. Aliquam mattis, elit eget rutrum vulputate, tortor sem pulvinar justo, sit amet mollis felis sem at nibh. Donec malesuada, neque id interdum eleifend, arcu augue porta elit, nec tristique libero metus at massa. Fusce fringilla laoreet rhoncus. Suspendisse potenti. Phasellus dignissim sodales mauris at pharetra. Donec gravida fringilla velit ac rutrum.

Duis tristique sollicitudin leo nec consequat. Praesent et dui convallis velit tincidunt fermentum. Mauris cursus purus at sem viverra sed imperdiet sapien imperdiet. Aliquam mattis, elit eget rutrum vulputate, tortor sem pulvinar justo, sit amet mollis felis sem at nibh. Donec malesuada, neque id interdum eleifend, arcu augue porta elit, nec tristique libero metus at massa. Fusce fringilla laoreet rhoncus. Suspendisse potenti. Phasellus dignissim sodales mauris at pharetra. Donec gravida fringilla velit ac rutrum.

Curabitur ornare lectus id diam molestie eu imperdiet nulla tempus. Maecenas vestibulum enim et dui ornare blandit. Vivamus fermentum faucibus viverra. Maecenas at justo sapien. Aenean rhoncus augue mattis purus rhoncus venenatis. Suspendisse metus felis, porttitor in varius in, vulputate at tortor. Aliquam molestie, turpis et malesuada porta, tortor sapien pharetra sapien, ac rhoncus quam dolor a sapien. Pellentesque varius laoreet enim ut auctor. Nullam nec ultricies nisi. Nullam porta lectus et ante consectetur posuere.

Duis tristique sollicitudin leo nec consequat. Praesent et dui convallis velit tincidunt fermentum. Mauris cursus purus at sem viverra sed imperdiet sapien imperdiet. Aliquam mattis, elit eget rutrum vulputate, tortor sem pulvinar justo, sit amet mollis felis sem at nibh. Donec malesuada, neque id interdum eleifend, arcu augue porta elit, nec tristique libero metus at massa. Fusce fringilla laoreet rhoncus. Suspendisse potenti. Phasellus dignissim sodales mauris at pharetra. Donec gravida fringilla velit ac rutrum.

\bigskip
\noindent
\textit{Key words:} first word; second word; third word.
% Separar palabras con punto-y-comas.

\checklanguage
% Fin archivo abstract.tex
\endinput 
\end{abstract}
\end{verbatim}\end{quote}

Finally, the page numbers must be changed to arabic numbers to conclude the preliminary
portion of the document.
\begin{quote}\tt\singlespace\begin{verbatim}
\clearpage\pagenumbering{arabic}
\end{verbatim}\end{quote}

\section{The Body}
At the beginning of {\tt intro.tex} there is the following command:
\begin{quote}\tt\singlespace\begin{verbatim}
\chapter{Introducing the {\tt withesis} \LaTeX{} Style Guide}
\end{verbatim}\end{quote}
Following that is the text of the chapter.  The body of your thesis is separated by
sectioning commands like {\tt \verb|\chapter{}|}.  For more information on the sectioning
commands, see Section~\ref{ess:sectioning}.

Remember the basic rule of outlining you learned in grammar school:
\begin{quote}
You cannot have an `A' if you do not have a `B'
\end{quote}
Take care to have at least two {\tt \verb|\section|}s if you use the command; have
two {\tt \verb|\subsection|}s, {\em etc}.



\section{Additional Theorem Like Environments}
The {\tt withesis} style adds numerous additional theorem like environments.  These
environments were included to allow compatibility with the University of Wisconsin's
Math Department's style file.  These environments are
{\tt theorem}, {\tt assertion}, {\tt claim}, {\tt conjecture}, {\tt corollary},
{\tt definition}, {\tt example}, {\tt figger}, {\tt lemma}, {\tt prop} and {\tt remark}.

As an example, consider the following.
\begin{lemma}
Assuming that $\partial\Omega_2 = \emptyset$ and that $h(t) = 1$, we
have $$
\begin{array}{lr}
\Delta u = f, &  x\in\Omega ,\\[2pt]
u =  g_1, &  x\in\partial\Omega .
\end{array}
$$
\end{lemma}
which was produced with the following:
\begin{quote}\tt\singlespace\begin{verbatim}
\begin{lemma}
Assuming that $\partial\Omega_2 = \emptyset$ and that $h(t) = 1$, we
have $$
\begin{array}{lr}
\Delta u = f, &  x\in\Omega ,\\[2pt]
 u =  g_1, &  x\in\partial\Omega .
\end{array}
$$
\end{lemma}
\end{verbatim}\end{quote}

\section{Bibliography or References}
As a final note, the default title for the references chapter is ``LIST OF REFERENCES.''
Since some people may prefer ``BIBLIOGRAPHY'', the command
\break{\tt \verb|\altbibtitle|}
has been added to change the chapter title.

\section{Appendices}
There are two commands which are available to suppress the writing of the auxiliary information
(to the {\tt .lot} and {\tt .lof} files).  They are:
\begin{quote}\tt\singlespace\begin{verbatim}
\noappendixtables                % Don't have appendix tables
\noappendixfigures               % Don't have appendix figures
\end{verbatim}\end{quote}
These commands should be in the preamble.  See Section~\ref{usage:noapp}.

There are two environments for doing the appendix chapter: {\tt appendix} and
\break {\tt appendices}.  If you have only one chapter in the appendix, use the {\tt appendix}
environment.  If you have more than one chapter, like this manual, use the
{\tt appendices} environment.
\begin{quote}\tt\singlespace\footnotesize\begin{verbatim}
\begin{appendices}  % Start of the Appendix Chapters.  If there is only
                    % one Appendix Chapter, then use \begin{appendix}
% code.tex
% this file is part of the example UW-Madison Thesis document
% It demonstrates one method for incorporating program listings
% into a document.

\chapter{Matlab Code} \label{matlab}
This is an example of a Matlab m-file.
\verbatimfile{derivs.m}
      % Including computer code listings
\chapter{Bib\TeX\ Entries}
\label{bibrefs}
The following shows the fields required in all types of Bib\TeX\ entries.
Fields with {\tt OPT} prefixed are optional (the three letters {\tt OPT} should 
not be used).  If an optional field is not used, then the entire field can be deleted.

{\tt
\singlespace
\begin{verbatim}

@Unpublished{,                            @Manual{,
  author =      "",                         title =           "",
  title =       "",                         OPTauthor =       "",
  note =        "",                         OPTorganization = "",
  OPTyear =     "",                         OPTaddress =      "",
  OPTmonth =    ""                          OPTedition =      "",
}                                           OPTyear =         "",
                                            OPTmonth =        "",
@TechReport{,                               OPTnote =         "" 
  author =      "",                       }
  title =       "",
  institution = "",                       @InProceedings{,
  year =        "",                         author =          "",
  OPTtype =     "",                         title =           "",
  OPTnumber =   "",                         booktitle =       "",
  OPTaddress =  "",                         year =            "",
  OPTmonth =    "",                         OPTeditor =       "",
  OPTnote =     ""                          OPTpages =        "",
}                                           OPTorganization = "",
                                            OPTpublisher =    "",
@Proceedings{,                              OPTaddress =      "",
  title =           "",                     OPTmonth =        "",
  year =            "",                     OPTnote =         "" 
  OPTeditor =       "",                   }
  OPTpublisher =    "",
  OPTorganization = "",
  OPTaddress =      "",
  OPTmonth =        "",
  OPTnote =         "" 
}



@PhDThesis{,                              @InCollection{,
  author =      "",                         author =          "",
  title =       "",                         title =           "",
  school =      "",                         booktitle =       "",
  year =        "",                         publisher =       "",
  OPTaddress =  "",                         year =            "",
  OPTmonth =    "",                         OPTeditor =       "",
  OPTnote =     ""                          OPTchapter =      "",
}                                           OPTpages =        "",
                                            OPTaddress =      "",
                                            OPTmonth =        "",
                                            OPTnote =         ""
                                          }

 
@Misc{,                                   @InCollection{,
  OPTauthor =       "",                     author =          "",
  OPTtitle =        "",                     title =           "",
  OPThowpublished = "",                     chapter =         "",
  OPTyear =         "",                     publisher =       "",
  OPTmonth =        "",                     year =            "",
  OPTnote =         ""                      OPTeditor =       "",
}                                           OPTpages =        "",
}                                           OPTvolume =       "",
                                            OPTseries =       "",
                                            OPTaddress =      "",
                                            OPTedition =      "",
                                            OPTmonth =        "",
                                            OPTnote =         ""
                                          }

@MastersThesis{,                          @Article{,
  author =      "",                         author =          "",
  title =       "",                         title =           "",
  school =      "",                         journal =         "",
  year =        "",                         year =            "",
  OPTaddress =  "",                         OPTvolume =       "",
  OPTmonth =    "",                         OPTnumber =       "",
  OPTnote =     ""                          OPTpages =        "",
}                                           OPTmonth =        "",
                                            OPTnote =         ""
                                           }\end{verbatim} }
    % a BibTeX reference
\chapter{Mathematics Examples}
This appendix provides an example of \LaTeX's typesetting
capabilities.  Most of text was obtained from the University of
Wisconsin-Madison Math Department's example thesis file.

\section{Matrices}
The equations for the {\em dq}-model of an induction machine in the
synchronous reference frame are
\begin{eqnarray}
 \left[\begin{array}{c} v_{qs}^e\\v_{ds}^e\\v_{qr}^e\\v_{dr}^e  \end{array}\right]                                                                                                                                                                                                                                                                                                                                                                                                                                                                                                              
 &=& \left[ \begin{array}{cccc}
 r_s + x_s\frac{\rho}{\omega_b} & \frac{\omega_e}{\omega_b}x_s & x_m\frac{\rho}{\omega_b} & \frac{\omega_e}{\omega_b}x_m \\
 -\frac{\omega_e}{\omega_b}x_s & r_s + x_s\frac{\rho}{\omega_b} & -\frac{\omega_e}{\omega_b}x_m & x_m\frac{\rho}{\omega_b} \\
 x_m\frac{\rho}{\omega_b} & \frac{\omega_e -\omega_r}{\omega_b}x_m & r_r'+x_r'\frac{\rho}{\omega_b} & \frac{\omega_e - \omega_r}{\omega_b}x_r' \\
 -\frac{\omega_e -\omega_r}{\omega_b}x_m & x_m\frac{\rho}{\omega_b} & -\frac{\omega_e - \omega_r}{\omega_b}x_r' & r_r' + x_r'\frac{\rho}{\omega_b}
 \end{array} \right]
 \left[\begin{array}{c} i_{qs}^e\\i_{ds}^e\\i_{qr}^e\\i_{dr}^e\end{array} \right] \label{volteq}\\
 T_e&=&\frac{3}{2}\frac{P}{2}\frac{x_m}{\omega_b}\left(i_{qs}^ei_{dr}^e - i_{ds}^ei_{qr}^e\right) \label{torqueeq}\\
 T_e-T_l&=&\frac{2J\omega_b}{P}\frac{d}{dt}\left(\frac{\omega_r}{\omega_b}\right) \label{mecheq}.
\end{eqnarray}

\section{Multi-line Equations}

\LaTeX{} has a built-in equation array feature, however the
equation numbers must be on the same line as an equation.  For example:
\begin{eqnarray}
\Delta u + \lambda e^u &= 0&u\in \Omega,  \nonumber \\
u&=0&u\in\partial\Omega.
\end{eqnarray}

Alternatively, the number can be centered in the equation using the
following method.
%
% The equation-array feature in LaTeX is a bad idea.  For centered
% numbers you should set your own equations and arrays as follows:
%
\def\dd{\displaystyle}
\begin{equation}\label{gelfand}
\begin{array}{rl}
\dd \Delta u + \lambda e^u = 0, &
\dd u\in \Omega,\\[8pt] % add 8pt extra vertical space. 1 line=23pt
\dd u=0, & \dd u\in\partial\Omega.
\end{array}
\end{equation}
The previous equation had a label.  It may be referenced as
equation~(\ref{gelfand}).


\section{More Complicated Equations}
\section*{Rellich's identity}\label{rellich.section}
\setcounter{theorem}{0}
%
%

Standard developments of Pohozaev's identity used an identity by
Rellich~\cite{rellich:der40}, reproduced here.

\begin{lemma}[Rellich]
Given $L$ in divergence form and $a,d$ defined above, $u\in C^2
(\Omega )$, we have
\begin{equation}\label{rellich}
\int_{\Omega}(-Lu)\nabla u\cdot (x-\overline{x})\, dx
= (1-\frac{n}{2}) \int_{\Omega} a(\nabla u,\nabla u) \, dx
-
\frac{1}{2} \int_{\Omega}
d(\nabla u, \nabla u) \, dx
\end{equation}
$$
+
\frac{1}{2} \int_{\partial\Omega} a(\nabla u,\nabla u)(x-\overline{x})
\cdot \nu  \, dS
-
\int_{\partial\Omega}
a(\nabla u,\nu )\nabla u\cdot (x-\overline{x}) \, dS.
$$
\end{lemma}
{\bf Proof:}\\
There is no loss in generality to take $\overline{x} = 0$. First
rewrite $L$:
$$Lu = \frac{1}{2}\left[ \sum_{i}\sum_{j}
\frac{\partial}{\partial x_i}
\left( a_{ij} \frac{\partial u}{\partial x_j} \right) +
\sum_{i}\sum_{j}
\frac{\partial}{\partial x_i}
\left( a_{ij} \frac{\partial u}{\partial x_j} \right)
\right]$$
Switching the order of summation on the second term and relabeling
subscripts, $j \rightarrow i$ and $i \rightarrow j$, then using the fact
that $a_{ij}(x)$ is a symmetric matrix,
gives the symmetric form needed to derive Rellich's identity.
\begin{equation}
Lu = \frac{1}{2} \sum_{i,j}\left[
\frac{\partial}{\partial x_i}
\left( a_{ij} \frac{\partial u}{\partial x_j} \right) +
\frac{\partial}{\partial x_j}
\left( a_{ij} \frac{\partial u}{\partial x_i} \right)
\right].
\end{equation}

Multiplying $-Lu$ by $\frac{\partial u}{\partial x_k} x_k$ and integrating
over $\Omega$, yields
$$\int_{\Omega}(-Lu)\frac{\partial u}{\partial x_k} x_k \, dx=
-\frac{1}{2} \int_{\Omega}
\sum_{i,j}\left[
\frac{\partial}{\partial x_i}
\left( a_{ij} \frac{\partial u}{\partial x_j} \right) +
\frac{\partial}{\partial x_j}
\left( a_{ij} \frac{\partial u}{\partial x_i} \right)
\right]
\frac{\partial u}{\partial x_k} x_k \, dx$$
Integrating by parts (for integral theorems see~\cite[p. 20]
{zeidler:nfa88IIa})
gives
$$= \frac{1}{2} \int_{\Omega}
\sum_{i,j} a_{ij} \left[
\frac{\partial u}{\partial x_j}
\frac{\partial^2 u}{\partial x_k\partial x_i} +
\frac{\partial u}{\partial x_i}
\frac{\partial^2 u}{\partial x_k\partial x_j}
\right] x_k \, dx
$$
$$
+
\frac{1}{2} \int_{\Omega}
\sum_{i,j} a_{ij} \left[
\frac{\partial u}{\partial x_j} \delta_{ik} +
\frac{\partial u}{\partial x_i} \delta_{jk}
\right] \frac{\partial u}{\partial x_k} \, dx
$$
$$- \frac{1}{2} \int_{\partial\Omega}
\sum_{i,j} a_{ij} \left[
\frac{\partial u}{\partial x_j} \nu_{i} +
\frac{\partial u}{\partial x_i} \nu_{j}
\right] \frac{\partial u}{\partial x_k} x_k \, dx
$$
= $I_1 + I_2 + I_3$, where the unit normal vector is $\nu$.
One may rewrite $I_1$ as
$$I_1 = \frac{1}{2} \int_{\Omega}
\sum_{i,j} a_{ij} \frac{\partial}{\partial x_k}\left(
\frac{\partial u}{\partial x_i}
\frac{\partial u}{\partial x_j}
\right) x_k \, dx
$$
Integrating the first term by parts again yields
$$I_1 = -\frac{1}{2} \int_{\Omega}
\sum_{i,j} a_{ij} \left(
\frac{\partial u}{\partial x_i}
\frac{\partial u}{\partial x_j}
\right) \, dx
+
\frac{1}{2} \int_{\partial\Omega}
\sum_{i,j} a_{ij} \left(
\frac{\partial u}{\partial x_i}
\frac{\partial u}{\partial x_j}
\right) x_k \nu_k \, dS
$$
$$
-
\frac{1}{2} \int_{\Omega}
\sum_{i,j} \left(
\frac{\partial u}{\partial x_i}
\frac{\partial u}{\partial x_j}
\right) x_k \frac{\partial a_{ij}}{\partial x_k}\, dx.
$$
Summing over $k$ gives
$$\int_{\Omega}(-Lu)(\nabla u\cdot x)\, dx =
-\frac{n}{2} \int_{\Omega}
\sum_{i,j} a_{ij} \left(
\frac{\partial u}{\partial x_i}
\frac{\partial u}{\partial x_j}
\right) \, dx
$$
$$
+
\frac{1}{2} \int_{\partial\Omega}
\sum_{i,j} a_{ij} \left(
\frac{\partial u}{\partial x_i}
\frac{\partial u}{\partial x_j}
\right) (x\cdot \nu ) \, dS
-
\frac{1}{2} \int_{\Omega}
\sum_{i,j} \left(
\frac{\partial u}{\partial x_i}
\frac{\partial u}{\partial x_j}
\right) (x\cdot  \nabla a_{ij}) \, dx
$$
$$
+
\frac{1}{2} \int_{\Omega}
\sum_{i,j,k} a_{ij} \left[
\frac{\partial u}{\partial x_j}
\frac{\partial u}{\partial x_k} \delta_{ik} +
\frac{\partial u}{\partial x_i}
\frac{\partial u}{\partial x_k} \delta_{jk}
\right] \, dx
$$
$$- \frac{1}{2} \int_{\partial\Omega}
\sum_{i,j} a_{ij} \left[
\frac{\partial u}{\partial x_j} \nu_{i} +
\frac{\partial u}{\partial x_i} \nu_{j}
\right] (\nabla u\cdot x) \, dS.
$$
Combining the first and fourth term on the right-hand side
simplifies the expression
$$\int_{\Omega}(-Lu)(\nabla u\cdot x)\, dx
=
(1-\frac{n}{2}) \int_{\Omega}
\sum_{i,j} a_{ij} \left(
\frac{\partial u}{\partial x_i}
\frac{\partial u}{\partial x_j}
\right) \, dx
$$
$$
+
\frac{1}{2} \int_{\partial\Omega}
\sum_{i,j} a_{ij} \left(
\frac{\partial u}{\partial x_i}
\frac{\partial u}{\partial x_j}
\right) (x\cdot \nu ) \, dS
-
\frac{1}{2} \int_{\Omega}
\sum_{i,j} \left(
\frac{\partial u}{\partial x_i}
\frac{\partial u}{\partial x_j}
\right) (x\cdot  \nabla a_{ij}) \, dx
$$
$$
-
\frac{1}{2} \int_{\partial\Omega}
\sum_{i,j} a_{ij} \left[
\frac{\partial u}{\partial x_j} \nu_{i} +
\frac{\partial u}{\partial x_i} \nu_{j}
\right] (\nabla u\cdot x) \, dS.
$$
Using the notation defined above, the result follows.


      % Complex Equations from the UW Math Department

\end{appendices}    % End of the Appendix Chapters. ibid on \end{appendix}
\end{verbatim}\end{quote}
The difference between these two environments is the way that the chapter header is
created and how this is listed in the table of contents.
          % Chapter 5 Strongly based on similar by J.D. McCauley
\bibliography{refs}      % Make the bibliography
\begin{appendices}       % Start of the Appendix Chapters.  If there is only
                         % one Appendix Chapter, then use \begin{appendix}
% code.tex
% this file is part of the example UW-Madison Thesis document
% It demonstrates one method for incorporating program listings
% into a document.

\chapter{Matlab Code} \label{matlab}
This is an example of a Matlab m-file.
\verbatimfile{derivs.m}
         % Including computer code listings
\chapter{Bib\TeX\ Entries}
\label{bibrefs}
The following shows the fields required in all types of Bib\TeX\ entries.
Fields with {\tt OPT} prefixed are optional (the three letters {\tt OPT} should 
not be used).  If an optional field is not used, then the entire field can be deleted.

{\tt
\singlespace
\begin{verbatim}

@Unpublished{,                            @Manual{,
  author =      "",                         title =           "",
  title =       "",                         OPTauthor =       "",
  note =        "",                         OPTorganization = "",
  OPTyear =     "",                         OPTaddress =      "",
  OPTmonth =    ""                          OPTedition =      "",
}                                           OPTyear =         "",
                                            OPTmonth =        "",
@TechReport{,                               OPTnote =         "" 
  author =      "",                       }
  title =       "",
  institution = "",                       @InProceedings{,
  year =        "",                         author =          "",
  OPTtype =     "",                         title =           "",
  OPTnumber =   "",                         booktitle =       "",
  OPTaddress =  "",                         year =            "",
  OPTmonth =    "",                         OPTeditor =       "",
  OPTnote =     ""                          OPTpages =        "",
}                                           OPTorganization = "",
                                            OPTpublisher =    "",
@Proceedings{,                              OPTaddress =      "",
  title =           "",                     OPTmonth =        "",
  year =            "",                     OPTnote =         "" 
  OPTeditor =       "",                   }
  OPTpublisher =    "",
  OPTorganization = "",
  OPTaddress =      "",
  OPTmonth =        "",
  OPTnote =         "" 
}



@PhDThesis{,                              @InCollection{,
  author =      "",                         author =          "",
  title =       "",                         title =           "",
  school =      "",                         booktitle =       "",
  year =        "",                         publisher =       "",
  OPTaddress =  "",                         year =            "",
  OPTmonth =    "",                         OPTeditor =       "",
  OPTnote =     ""                          OPTchapter =      "",
}                                           OPTpages =        "",
                                            OPTaddress =      "",
                                            OPTmonth =        "",
                                            OPTnote =         ""
                                          }

 
@Misc{,                                   @InCollection{,
  OPTauthor =       "",                     author =          "",
  OPTtitle =        "",                     title =           "",
  OPThowpublished = "",                     chapter =         "",
  OPTyear =         "",                     publisher =       "",
  OPTmonth =        "",                     year =            "",
  OPTnote =         ""                      OPTeditor =       "",
}                                           OPTpages =        "",
}                                           OPTvolume =       "",
                                            OPTseries =       "",
                                            OPTaddress =      "",
                                            OPTedition =      "",
                                            OPTmonth =        "",
                                            OPTnote =         ""
                                          }

@MastersThesis{,                          @Article{,
  author =      "",                         author =          "",
  title =       "",                         title =           "",
  school =      "",                         journal =         "",
  year =        "",                         year =            "",
  OPTaddress =  "",                         OPTvolume =       "",
  OPTmonth =    "",                         OPTnumber =       "",
  OPTnote =     ""                          OPTpages =        "",
}                                           OPTmonth =        "",
                                            OPTnote =         ""
                                           }\end{verbatim} }
         % a BibTeX reference
\chapter{Mathematics Examples}
This appendix provides an example of \LaTeX's typesetting
capabilities.  Most of text was obtained from the University of
Wisconsin-Madison Math Department's example thesis file.

\section{Matrices}
The equations for the {\em dq}-model of an induction machine in the
synchronous reference frame are
\begin{eqnarray}
 \left[\begin{array}{c} v_{qs}^e\\v_{ds}^e\\v_{qr}^e\\v_{dr}^e  \end{array}\right]                                                                                                                                                                                                                                                                                                                                                                                                                                                                                                              
 &=& \left[ \begin{array}{cccc}
 r_s + x_s\frac{\rho}{\omega_b} & \frac{\omega_e}{\omega_b}x_s & x_m\frac{\rho}{\omega_b} & \frac{\omega_e}{\omega_b}x_m \\
 -\frac{\omega_e}{\omega_b}x_s & r_s + x_s\frac{\rho}{\omega_b} & -\frac{\omega_e}{\omega_b}x_m & x_m\frac{\rho}{\omega_b} \\
 x_m\frac{\rho}{\omega_b} & \frac{\omega_e -\omega_r}{\omega_b}x_m & r_r'+x_r'\frac{\rho}{\omega_b} & \frac{\omega_e - \omega_r}{\omega_b}x_r' \\
 -\frac{\omega_e -\omega_r}{\omega_b}x_m & x_m\frac{\rho}{\omega_b} & -\frac{\omega_e - \omega_r}{\omega_b}x_r' & r_r' + x_r'\frac{\rho}{\omega_b}
 \end{array} \right]
 \left[\begin{array}{c} i_{qs}^e\\i_{ds}^e\\i_{qr}^e\\i_{dr}^e\end{array} \right] \label{volteq}\\
 T_e&=&\frac{3}{2}\frac{P}{2}\frac{x_m}{\omega_b}\left(i_{qs}^ei_{dr}^e - i_{ds}^ei_{qr}^e\right) \label{torqueeq}\\
 T_e-T_l&=&\frac{2J\omega_b}{P}\frac{d}{dt}\left(\frac{\omega_r}{\omega_b}\right) \label{mecheq}.
\end{eqnarray}

\section{Multi-line Equations}

\LaTeX{} has a built-in equation array feature, however the
equation numbers must be on the same line as an equation.  For example:
\begin{eqnarray}
\Delta u + \lambda e^u &= 0&u\in \Omega,  \nonumber \\
u&=0&u\in\partial\Omega.
\end{eqnarray}

Alternatively, the number can be centered in the equation using the
following method.
%
% The equation-array feature in LaTeX is a bad idea.  For centered
% numbers you should set your own equations and arrays as follows:
%
\def\dd{\displaystyle}
\begin{equation}\label{gelfand}
\begin{array}{rl}
\dd \Delta u + \lambda e^u = 0, &
\dd u\in \Omega,\\[8pt] % add 8pt extra vertical space. 1 line=23pt
\dd u=0, & \dd u\in\partial\Omega.
\end{array}
\end{equation}
The previous equation had a label.  It may be referenced as
equation~(\ref{gelfand}).


\section{More Complicated Equations}
\section*{Rellich's identity}\label{rellich.section}
\setcounter{theorem}{0}
%
%

Standard developments of Pohozaev's identity used an identity by
Rellich~\cite{rellich:der40}, reproduced here.

\begin{lemma}[Rellich]
Given $L$ in divergence form and $a,d$ defined above, $u\in C^2
(\Omega )$, we have
\begin{equation}\label{rellich}
\int_{\Omega}(-Lu)\nabla u\cdot (x-\overline{x})\, dx
= (1-\frac{n}{2}) \int_{\Omega} a(\nabla u,\nabla u) \, dx
-
\frac{1}{2} \int_{\Omega}
d(\nabla u, \nabla u) \, dx
\end{equation}
$$
+
\frac{1}{2} \int_{\partial\Omega} a(\nabla u,\nabla u)(x-\overline{x})
\cdot \nu  \, dS
-
\int_{\partial\Omega}
a(\nabla u,\nu )\nabla u\cdot (x-\overline{x}) \, dS.
$$
\end{lemma}
{\bf Proof:}\\
There is no loss in generality to take $\overline{x} = 0$. First
rewrite $L$:
$$Lu = \frac{1}{2}\left[ \sum_{i}\sum_{j}
\frac{\partial}{\partial x_i}
\left( a_{ij} \frac{\partial u}{\partial x_j} \right) +
\sum_{i}\sum_{j}
\frac{\partial}{\partial x_i}
\left( a_{ij} \frac{\partial u}{\partial x_j} \right)
\right]$$
Switching the order of summation on the second term and relabeling
subscripts, $j \rightarrow i$ and $i \rightarrow j$, then using the fact
that $a_{ij}(x)$ is a symmetric matrix,
gives the symmetric form needed to derive Rellich's identity.
\begin{equation}
Lu = \frac{1}{2} \sum_{i,j}\left[
\frac{\partial}{\partial x_i}
\left( a_{ij} \frac{\partial u}{\partial x_j} \right) +
\frac{\partial}{\partial x_j}
\left( a_{ij} \frac{\partial u}{\partial x_i} \right)
\right].
\end{equation}

Multiplying $-Lu$ by $\frac{\partial u}{\partial x_k} x_k$ and integrating
over $\Omega$, yields
$$\int_{\Omega}(-Lu)\frac{\partial u}{\partial x_k} x_k \, dx=
-\frac{1}{2} \int_{\Omega}
\sum_{i,j}\left[
\frac{\partial}{\partial x_i}
\left( a_{ij} \frac{\partial u}{\partial x_j} \right) +
\frac{\partial}{\partial x_j}
\left( a_{ij} \frac{\partial u}{\partial x_i} \right)
\right]
\frac{\partial u}{\partial x_k} x_k \, dx$$
Integrating by parts (for integral theorems see~\cite[p. 20]
{zeidler:nfa88IIa})
gives
$$= \frac{1}{2} \int_{\Omega}
\sum_{i,j} a_{ij} \left[
\frac{\partial u}{\partial x_j}
\frac{\partial^2 u}{\partial x_k\partial x_i} +
\frac{\partial u}{\partial x_i}
\frac{\partial^2 u}{\partial x_k\partial x_j}
\right] x_k \, dx
$$
$$
+
\frac{1}{2} \int_{\Omega}
\sum_{i,j} a_{ij} \left[
\frac{\partial u}{\partial x_j} \delta_{ik} +
\frac{\partial u}{\partial x_i} \delta_{jk}
\right] \frac{\partial u}{\partial x_k} \, dx
$$
$$- \frac{1}{2} \int_{\partial\Omega}
\sum_{i,j} a_{ij} \left[
\frac{\partial u}{\partial x_j} \nu_{i} +
\frac{\partial u}{\partial x_i} \nu_{j}
\right] \frac{\partial u}{\partial x_k} x_k \, dx
$$
= $I_1 + I_2 + I_3$, where the unit normal vector is $\nu$.
One may rewrite $I_1$ as
$$I_1 = \frac{1}{2} \int_{\Omega}
\sum_{i,j} a_{ij} \frac{\partial}{\partial x_k}\left(
\frac{\partial u}{\partial x_i}
\frac{\partial u}{\partial x_j}
\right) x_k \, dx
$$
Integrating the first term by parts again yields
$$I_1 = -\frac{1}{2} \int_{\Omega}
\sum_{i,j} a_{ij} \left(
\frac{\partial u}{\partial x_i}
\frac{\partial u}{\partial x_j}
\right) \, dx
+
\frac{1}{2} \int_{\partial\Omega}
\sum_{i,j} a_{ij} \left(
\frac{\partial u}{\partial x_i}
\frac{\partial u}{\partial x_j}
\right) x_k \nu_k \, dS
$$
$$
-
\frac{1}{2} \int_{\Omega}
\sum_{i,j} \left(
\frac{\partial u}{\partial x_i}
\frac{\partial u}{\partial x_j}
\right) x_k \frac{\partial a_{ij}}{\partial x_k}\, dx.
$$
Summing over $k$ gives
$$\int_{\Omega}(-Lu)(\nabla u\cdot x)\, dx =
-\frac{n}{2} \int_{\Omega}
\sum_{i,j} a_{ij} \left(
\frac{\partial u}{\partial x_i}
\frac{\partial u}{\partial x_j}
\right) \, dx
$$
$$
+
\frac{1}{2} \int_{\partial\Omega}
\sum_{i,j} a_{ij} \left(
\frac{\partial u}{\partial x_i}
\frac{\partial u}{\partial x_j}
\right) (x\cdot \nu ) \, dS
-
\frac{1}{2} \int_{\Omega}
\sum_{i,j} \left(
\frac{\partial u}{\partial x_i}
\frac{\partial u}{\partial x_j}
\right) (x\cdot  \nabla a_{ij}) \, dx
$$
$$
+
\frac{1}{2} \int_{\Omega}
\sum_{i,j,k} a_{ij} \left[
\frac{\partial u}{\partial x_j}
\frac{\partial u}{\partial x_k} \delta_{ik} +
\frac{\partial u}{\partial x_i}
\frac{\partial u}{\partial x_k} \delta_{jk}
\right] \, dx
$$
$$- \frac{1}{2} \int_{\partial\Omega}
\sum_{i,j} a_{ij} \left[
\frac{\partial u}{\partial x_j} \nu_{i} +
\frac{\partial u}{\partial x_i} \nu_{j}
\right] (\nabla u\cdot x) \, dS.
$$
Combining the first and fourth term on the right-hand side
simplifies the expression
$$\int_{\Omega}(-Lu)(\nabla u\cdot x)\, dx
=
(1-\frac{n}{2}) \int_{\Omega}
\sum_{i,j} a_{ij} \left(
\frac{\partial u}{\partial x_i}
\frac{\partial u}{\partial x_j}
\right) \, dx
$$
$$
+
\frac{1}{2} \int_{\partial\Omega}
\sum_{i,j} a_{ij} \left(
\frac{\partial u}{\partial x_i}
\frac{\partial u}{\partial x_j}
\right) (x\cdot \nu ) \, dS
-
\frac{1}{2} \int_{\Omega}
\sum_{i,j} \left(
\frac{\partial u}{\partial x_i}
\frac{\partial u}{\partial x_j}
\right) (x\cdot  \nabla a_{ij}) \, dx
$$
$$
-
\frac{1}{2} \int_{\partial\Omega}
\sum_{i,j} a_{ij} \left[
\frac{\partial u}{\partial x_j} \nu_{i} +
\frac{\partial u}{\partial x_i} \nu_{j}
\right] (\nabla u\cdot x) \, dS.
$$
Using the notation defined above, the result follows.


           % Complex Equations from the UW Math Department
% --------------------------------------------------------------------------
% the ACRO package
% 
%   Typeset Acronyms
% 
% --------------------------------------------------------------------------
% Clemens Niederberger
% Web:    https://bitbucket.org/cgnieder/acro/
% E-Mail: contact@mychemistry.eu
% --------------------------------------------------------------------------
% Copyright 2011-2017 Clemens Niederberger
% 
% This work may be distributed and/or modified under the
% conditions of the LaTeX Project Public License, either version 1.3
% of this license or (at your option) any later version.
% The latest version of this license is in
%   http://www.latex-project.org/lppl.txt
% and version 1.3 or later is part of all distributions of LaTeX
% version 2005/12/01 or later.
% 
% This work has the LPPL maintenance status `maintained'.
% 
% The Current Maintainer of this work is Clemens Niederberger.
% --------------------------------------------------------------------------
% The acro package consists of the files
%  - acro.sty, acro_en.tex, acro_en.pdf, README
% --------------------------------------------------------------------------
% If you have any ideas, questions, suggestions or bugs to report, please
% feel free to contact me.
% --------------------------------------------------------------------------
\RequirePackage{expl3,xparse,l3keys2e,xtemplate,etoolbox}
\ProvidesExplPackage
  {acro}
  {2017/01/22}
  {2.7a}
  {Typeset Acronyms}

% --------------------------------------------------------------------------
% warning and error messages:
\msg_new:nnn {acro} {undefined}
  {
    You've~ requested~ acronym~ `#1'~ \msg_line_context: \ but~ you~
    apparently~ haven't~ defined~ it,~ yet! \\
    Maybe~ you've~ misspelled~ `#1'?
  }

\msg_new:nnn {acro} {macro}
  {
    A~ macro~ with~ the~ csname~ `#1'~ already~ exists! \\
    Unless~ you~ set~ acro's~ option~ `strict'~ I~ won't~ redefine~ it~
    \msg_line_context: .
  } 

\msg_new:nnn {acro} {replaced}
  {
    The~ #1~ `#2' ~you ~used~ \msg_line_context: \ is~ deprecated~ and~ has~
    been~ replaced~ by~ `#3'. ~Since~ I~ will~ not~ guarantee~ that~ #1~ `#2'~
    will~ be~ kept~ forever~ I~ strongly~ encourage~ you~ to~ switch!
  }

\msg_new:nnn {acro} {deprecated}
  {
    The~ #1~ `#2'~ you~ used~ \msg_line_context: \ is~ deprecated~and~ there~
    is~ no~ replacement.~ Since~ I~ will~ not~ guarantee~ that~ #1~ `#2'~
    will~ be~ kept~ forever~ I~ strongly~ encourage~ you~ to~ remove~ it~
    from~ your~ document.
  }

\msg_new:nnn {acro} {substitute-short}
  {
    There~ is~ no~ short~ form~ set~ for~ acronym~ `#1'! \\
    I~ am~ setting~ the~ short~ form~ equal~ to~ the~ ID~ `#1'. \\
    If~ that~ is~ not~ what~ you~ want~ make~ sure~ to~ add~ an~ explicit~
    short~ form.
  }
\msg_new:nnn {acro} {ending-exists}
  {
    An~ ending~ with~ the~ name~ `#1'~ already~ exists! \\ \\
    I~ am~ overwriting~ the~ defaults.
  }

\msg_new:nnn {acro} {ending-before-acronyms}
  {
    You~ are~ using~ \token_to_str:N \ProvideAcroEnding \ after~ you've~
    declared~ at~ least~ one~ acronym.~ This~ will~ lead~ to~ trouble! \\
    Make~ sure~ to~ define~ endings~ before~ *any*~ acronym~ declarations!
  }

\msg_new:nnn {acro} {no-alternative}
  {
    There~ is~ no~ alternative~ form~ for~ acronym~ `#1'! \\ \\
    I~ am~ using~ the~ short~ form~ instead.
  }

\msg_new:nnn {acro} {unknown}
  {
    You're~ trying~ to~ use~ the~ #1~ `#2'~ \msg_line_context: . \\
    However,~ I~ do~ not~ know~ #1~ `#2'! \\
    If~ this~ is~ no~ typo~ please~ contact~ the~ package~ author. \\ \\
    I~ am~ going~ to~ use~ the~ #1~ `#3'~ instead.
  }
\msg_new:nnn {acro} {definitions-missing}
  {
    I~ cannot~ find~ the~ file~ \c_acro_definition_file_name_tl
    .\c_acro_definition_file_extension_tl !~ This~ file~ contains~ all~
    essential~ user~ commands~ of~ acro~ and~ is~ a~ crucial~ part~ of~ the~
    package!~ Please~ check~ your~ installation.
  }

% --------------------------------------------------------------------------
% logging:
\cs_new:Npn \acro_if_log:T #1 { \use:n {#1} }

\bool_new:N \l__acro_log_acronyms_bool
\bool_new:N \l__acro_log_acronyms_verbose_bool

\keys_define:nn {acro}
  {
    log           .choice: ,
    log / true    .code:n    =
      \bool_set_true:N \l__acro_log_acronyms_bool
      \bool_set_false:N \l__acro_log_acronyms_verbose_bool ,
    log / silent  .meta:n    = { log = true } ,
    log / verbose .code:n    =
      \bool_set_true:N \l__acro_log_acronyms_bool
      \bool_set_true:N \l__acro_log_acronyms_verbose_bool ,
    log / false   .code:n    =
      \bool_set_false:N \l__acro_log_acronyms_bool
      \bool_set_false:N \l__acro_log_acronyms_verbose_bool ,
    log           .default:n = true ,
    log           .initial:n = false
  }

\cs_new:Npn \__acro_write_log:nn #1#2 { \ \ \ #1 ~ = ~ {#2} }
\cs_new:Npn \__acro_write_log_property:nnn #1#2#3
  { \__acro_write_log:nn {#2} { \__acro_get_property:nn {#3} {#1} } }

\cs_new:Npn \__acro_ending_log_entry:nn #1#2
  {
    | \\
    | \__acro_write_log_property:nnn {#1} {short-#2} {short_#2} \\
    | \__acro_write_log_property:nnn {#1} {short-#2-form} {short_#2_form} \\
    | \__acro_write_log_property:nnn {#1} {long-#2} {long_#2} \\
    | \__acro_write_log_property:nnn {#1} {long-#2-form} {long_#2_form} \\
    | \__acro_write_log_property:nnn {#1} {alt-#2} {alt_#2} \\
    | \__acro_write_log_property:nnn {#1} {alt-#2-form} {alt_#2_form} \\
  }
  
\msg_new:nnn {acro} {log-acronym-verbose}
  {
    ================================================= \\
    | ~ \msg_info_text:n {acro}~ --~ defining~ new~ acronym: \\
    | \__acro_write_log:nn {ID} {#1} \\
    | \__acro_write_log_property:nnn {#1} {short} {short} \\
    | \__acro_write_log_property:nnn {#1}{long} {long} \\
    | \__acro_write_log_property:nnn {#1}{alt} {alt} \\
    | \__acro_write_log_property:nnn {#1}{sort} {sort} \\
    | \__acro_write_log_property:nnn {#1}{class} {class} \\
    | \__acro_write_log_property:nnn {#1} {list} {list} \\
    | \__acro_write_log_property:nnn {#1} {extra} {extra} \\
    | \__acro_write_log_property:nnn {#1} {foreign} {foreign} \\
    | \__acro_write_log_property:nnn {#1} {single} {single} \\
    | \__acro_write_log_property:nnn {#1} {pdfstring} {pdfstring} \\
    | \__acro_write_log_property:nnn {#1} {accsupp} {accsupp} \\
    | \__acro_write_log_property:nnn {#1} {tooltip} {tooltip} \\
    | \\
    | \__acro_write_log_property:nnn {#1} {short-indefinite} {short_indefinite} \\
    | \__acro_write_log_property:nnn {#1} {long-indefinite} {long_indefinite} \\
    | \__acro_write_log_property:nnn {#1} {alt-indefinite} {alt_indefinite} \\
    \seq_map_function:NN \l__acro_endings_seq \__acro_ending_log_entry:n
    | \\
    | \__acro_write_log_property:nnn {#1} {short-format} {short_format} \\
    | \__acro_write_log_property:nnn {#1} {long-format} {long_format} \\
    | \__acro_write_log_property:nnn {#1} {first-long-format} {first_long_format} \\
    | \__acro_write_log_property:nnn {#1} {single-format} {single_format} \\
    | \__acro_write_log_property:nnn {#1} {foreign-lang} {foreign_lang} \\    
    | \\
    | \__acro_write_log_property:nnn {#1} {cite} {citation} \\
    | \__acro_write_log_property:nnn {#1} {index} {index} \\
    | \__acro_write_log_property:nnn {#1} {index-sort} {index_sort} \\
    | \\
    | \__acro_write_log_property:nnn {#1} {long-pre} {long_pre} \\
    | \__acro_write_log_property:nnn {#1} {long-post} {long_post} \\    
    | \__acro_write_log_property:nnn {#1} {index-cmd} {index_cmd} \\
    | \__acro_write_log_property:nnn {#1} {first-style} {first_style} \\
    =================================================
  }

\msg_new:nnn {acro} {log-acronym-silent}
  {
    ================================================= \\
    | ~ \msg_info_text:n {acro}~ --~ defining~ new~ acronym: \\
    | \__acro_write_log:nn {ID} {#1} \\
    | \__acro_write_log_property:nnn {#1} {short} {short} \\
    | \__acro_write_log_property:nnn {#1} {long} {long} \\
    | \__acro_write_log_property:nnn {#1} {alt} {alt} \\
    | \__acro_write_log_property:nnn {#1} {sort} {sort} \\
    | \__acro_write_log_property:nnn {#1} {class} {class} \\
    | \__acro_write_log_property:nnn {#1} {list} {list} \\
    | \__acro_write_log_property:nnn {#1} {extra} {extra} \\
    | \__acro_write_log_property:nnn {#1} {foreign} {foreign} \\
    | \__acro_write_log_property:nnn {#1} {pdfstring} {pdfstring} \\
    | \__acro_write_log_property:nnn {#1} {cite} {citation} \\
    =================================================
  }
  
\cs_new_protected:Npn \__acro_log_acronym:n #1
  {
    \bool_if:NT \l__acro_log_acronyms_bool
      {
        \cs_set:Npn \__acro_ending_log_entry:n ##1
          { \__acro_ending_log_entry:nn {#1} {##1} }
        \bool_if:NTF \l__acro_log_acronyms_verbose_bool
          { \msg_log:nnn {acro} {log-acronym-verbose} {#1} }
          { \msg_log:nnn {acro} {log-acronym-silent} {#1} }
      }
  }   

% --------------------------------------------------------------------------
% message macros:
\cs_new:Npn \__acro_remove_backslash:N #1
  { \exp_after:wN \use_none:n \token_to_str:N #1 }

\cs_new_protected:Npn \acro_new_message_commands:Nnnn #1#2#3#4
  {
    \clist_map_inline:nn {#2}
      {
        \cs_new_protected:cpn { \__acro_remove_backslash:N #1 ##1 }
          {
            \bool_if:NTF \l__acro_silence_bool
              { \use:c { \__acro_remove_backslash:N #3 n##1 } {acro} }
              { \use:c { \__acro_remove_backslash:N #4 n##1 } {acro} }
          }
      }
  }

\acro_new_message_commands:Nnnn \acro_serious_message: {n,nn,nnn}
  { \msg_warning: }
  { \msg_error: }

\acro_new_message_commands:Nnnn \acro_harmless_message: {n,nn,nnn,nnnn}
  { \msg_info: }
  { \msg_warning: }

\cs_new_protected:Npn \acro_option_deprecated:nn #1#2
  {
    \tl_if_blank:nTF {#2}
      { \acro_harmless_message:nnn  {deprecated} {option} {#1} }
      { \acro_harmless_message:nnnn {replaced}   {option} {#1} {#2} }
  }
\cs_new_protected:Npn \acro_option_deprecated:n #1
  { \acro_option_deprecated:nn {#1} {} }

\cs_new_protected:Npn \acro_command_deprecated:NN #1#2
  {
    \tl_if_blank:nTF {#2}
      {
        \acro_harmless_message:nnn {deprecated} {command}
          { \token_to_str:N #1 }
      }
      {
        \acro_harmless_message:nnnn {replaced} {command}
          { \token_to_str:N #1 }
          { \token_to_str:N #2 }
      }
  }

% --------------------------------------------------------------------------
% temporary variables
\tl_new:N   \l__acro_tmpa_tl
\tl_new:N   \l__acro_tmpb_tl
\tl_new:N   \l__acro_tmpc_tl
\prop_new:N \l__acro_tmpa_prop
\prop_new:N \l__acro_tmpb_prop
\seq_new:N  \l__acro_tmpa_seq
\seq_new:N  \l__acro_tmpb_seq
\int_new:N  \l__acro_tmpa_int
\int_new:N  \l__acro_tmpb_int
\int_new:N  \l__acro_tmpc_int
\int_new:N  \l__acro_tmpd_int

% --------------------------------------------------------------------------
% variants of kernel commands
\cs_generate_variant:Nn \quark_if_no_value:nTF  { V }
\cs_generate_variant:Nn \tl_put_right:Nn        { NV, Nv }
\cs_generate_variant:Nn \tl_if_eq:nnT           { V }
\cs_generate_variant:Nn \tl_if_eq:nnF           { V }
\cs_generate_variant:Nn \seq_use:Nnnn           { c }
\cs_generate_variant:Nn \seq_gset_split:Nnn     { c }
\cs_generate_variant:Nn \seq_set_split:Nnn      { NnV }
\cs_generate_variant:Nn \seq_if_in:NnT          { NV }
\cs_generate_variant:Nn \prop_put:Nnn           { NnV, cnx, cnv }
\cs_generate_variant:Nn \prop_get:NnNTF         { cnc }
\cs_generate_variant:Nn \prop_get:NnNF          { cn, cnc }
\cs_generate_variant:Nn \prop_get:NnN           { cnc }
\cs_generate_variant:Nn \cs_generate_variant:Nn { c }
\cs_generate_variant:Nn \str_case:nn            { V }

% --------------------------------------------------------------------------
% variables:
\bool_new:N      \l__acro_silence_bool
\bool_new:N      \l__acro_mark_as_used_bool
\bool_new:N      \g__acro_mark_first_as_used_bool
\bool_new:N      \l__acro_use_single_bool
\bool_new:N      \l__acro_print_only_used_bool
\bool_set_true:N \l__acro_print_only_used_bool
\bool_new:N      \l__acro_hyperref_loaded_bool
\bool_new:N      \l__acro_use_hyperref_bool
\bool_new:N      \l__acro_xspace_bool
\bool_new:N      \l__acro_custom_format_bool
\bool_new:N      \l__acro_strict_bool
\bool_new:N      \l__acro_create_macros_bool
\bool_new:N      \l__acro_first_upper_bool
\bool_new:N      \l__acro_indefinite_bool
\bool_new:N      \l__acro_upper_indefinite_bool
\bool_new:N      \l__acro_foreign_bool
\bool_set_true:N \l__acro_foreign_bool
\bool_new:N      \l__acro_sort_bool
\bool_set_true:N \l__acro_sort_bool
\bool_new:N      \l__acro_capitalize_list_bool
\bool_new:N      \l__acro_citation_all_bool
\bool_new:N      \l__acro_citation_first_bool
\bool_set_true:N \l__acro_citation_first_bool
\bool_new:N      \l__acro_group_citation_bool
\bool_new:N      \l__acro_acc_supp_bool
\bool_new:N      \l__acro_tooltip_bool
\bool_new:N      \l__acro_inside_tooltip_bool
\bool_new:N      \l__acro_following_page_bool
\bool_new:N      \l__acro_following_pages_bool
\bool_new:N      \l__acro_addto_index_bool
\bool_new:N      \l__acro_is_excluded_bool
\bool_new:N      \l__acro_is_included_bool
\bool_new:N      \l__acro_page_punct_bool
\bool_new:N      \l__acro_page_brackets_bool
\bool_new:N      \l__acro_page_display_bool
\bool_new:N      \l__acro_new_page_numbering_bool
\bool_new:N      \l__acro_first_use_brackets_bool
\bool_new:N      \l__acro_first_only_short_bool
\bool_new:N      \l__acro_first_only_long_bool
\bool_new:N      \l__acro_first_reversed_bool
\bool_new:N      \l__acro_first_switched_bool
\bool_new:N      \l__acro_use_note_bool
\bool_new:N      \l__acro_extra_punct_bool
\bool_new:N      \l__acro_extra_use_brackets_bool
\bool_new:N      \l__acro_in_list_bool
\bool_new:N      \l__acro_place_label_bool
\bool_new:N      \l__acro_list_all_pages_bool
\bool_set_true:N \l__acro_list_all_pages_bool
\bool_new:N      \g__acro_first_acronym_declared_bool
\bool_new:N      \l__acro_include_endings_format_bool
\bool_new:N      \l__acro_list_reverse_long_extra_bool
\bool_new:N      \l__acro_use_acronyms_bool
\bool_set_true:N \l__acro_use_acronyms_bool

\tl_new:N   \l__acro_ignore_tl
\tl_new:N   \l__acro_default_indefinite_tl
\tl_set:Nn  \l__acro_default_indefinite_tl {a}
\tl_new:N   \l__acro_foreign_sep_tl
\tl_new:N   \l__acro_extra_instance_tl
\tl_set:Nn  \l__acro_extra_instance_tl {default}
\tl_new:N   \l__acro_page_instance_tl
\tl_set:Nn  \l__acro_page_instance_tl  {none}
\tl_new:N   \l__acro_page_name_tl
\tl_new:N   \l__acro_pages_name_tl
\tl_new:N   \l__acro_next_page_tl
\tl_new:N   \l__acro_next_pages_tl
\tl_new:N   \l__acro_list_instance_tl
\tl_set:Nn  \l__acro_list_instance_tl  {description}
\tl_new:N   \l__acro_list_type_tl
% \tl_new:N   \l__acro_list_tl
\tl_new:N   \l__acro_list_heading_cmd_tl
\tl_set:Nn  \l__acro_list_heading_cmd_tl {section*}
\tl_new:N   \l__acro_list_name_tl
\tl_new:N   \l__acro_list_before_tl
\tl_new:N   \l__acro_list_after_tl
\tl_new:N   \l__acro_custom_format_tl
\tl_new:N   \l__acro_first_between_tl
\tl_new:N   \l__acro_citation_connect_tl
\tl_new:N   \l__acro_between_group_connect_citation_tl
\tl_new:N   \l__acro_extra_brackets_tl
\tl_new:N   \l__acro_extra_punct_tl
\tl_new:N   \l__acro_first_brackets_tl
\tl_new:N   \l__acro_page_punct_tl
\tl_new:N   \l__acro_page_brackets_tl
\tl_new:N   \l__acro_last_page_tl
\tl_new:N   \l__acro_current_page_tl
\tl_new:N   \l__acro_list_table_tl
\tl_new:N   \l__acro_list_table_spec_tl
\tl_new:N   \l__acro_acc_supp_tl
\tl_new:N   \l__acro_acc_supp_options_tl
\tl_new:N   \l__acro_label_prefix_tl
\tl_set:Nn  \l__acro_label_prefix_tl { ac: }
\tl_new:N   \l__acro_index_short_tl
\tl_new:N   \l__acro_first_instance_tl
\tl_set:Nn  \l__acro_first_instance_tl {default}

\tl_new:N   \l__acro_short_tl
\tl_new:N   \l__acro_short_format_tl
\tl_new:N   \l__acro_list_short_format_tl

\tl_new:N   \l__acro_alt_tl
\tl_new:N   \l__acro_alt_format_tl

\tl_new:N   \l__acro_long_tl
\tl_new:N   \l__acro_list_long_format_tl

\tl_new:N   \l__acro_single_form_tl
\tl_set:Nn  \l__acro_single_form_tl {long}

\tl_new:N   \l__acro_extra_format_tl

\tl_new:N   \l__acro_foreign_format_tl
\tl_new:N   \l__acro_foreign_list_format_tl
\tl_set:Nn  \l__acro_foreign_list_format_tl { \acroenparen }

\tl_new:N   \l__acro_index_format_tl


\skip_new:N  \l__acro_page_space_skip

\dim_new:N  \l__acro_short_width_dim
\dim_set:Nn \l__acro_short_width_dim {3em}

\prop_new:N \l__acro_list_styles_prop
\prop_new:N \l__acro_list_headings_prop
\prop_new:N \l__acro_first_styles_prop
\prop_new:N \l__acro_extra_styles_prop
\prop_new:N \l__acro_page_styles_prop

% --------------------------------------------------------------------------
% small commands for use at various places
\cs_new:Npn \acro_no_break: { \tex_penalty:D \c_ten_thousand }

\cs_new_protected:Npn \__acro_first_upper_case:n #1
  { \tl_upper_case:n { \tl_head:n {#1} } \tl_tail:n {#1} }
\cs_generate_variant:Nn \__acro_first_upper_case:n { x }
\cs_generate_variant:Nn \tl_mixed_case:n { x }

\cs_new_eq:NN \acro_first_upper_case:n \__acro_first_upper_case:n

\NewDocumentCommand \acfirstupper { m }
  { \acro_first_upper_case:n {#1} }

% --------------------------------------------------------------------------
% options:
\keys_define:nn {acro}
  {
    messages          .choice: ,
    messages / silent .code:n     =
      \bool_set_true:N \l__acro_silence_bool ,
    messages / loud .code:n       =
      \bool_set_false:N \l__acro_silence_bool ,
    messages          .value_required:n = true ,
    accsupp           .bool_set:N = \l__acro_acc_supp_bool ,
    accsupp-options   .tl_set:N   = \l__acro_acc_supp_options_tl ,
    tooltip           .bool_set:N = \l__acro_tooltip_bool ,
    tooltip-cmd       .code:n     = \cs_set:Npn \__acro_tooltip_cmd:nn {#1} ,
    tooltip-cmd       .value_required:n = true ,
    macros            .bool_set:N = \l__acro_create_macros_bool ,
    xspace            .bool_set:N = \l__acro_xspace_bool ,
    % xspace            .code:n     = \acro_option_deprecated:nn {xspace} {} ,
    strict            .bool_set:N = \l__acro_strict_bool ,
    sort              .bool_set:N = \l__acro_sort_bool ,
    short-format      .code:n     =
      \tl_set:Nn \l__acro_short_format_tl {#1}
      \tl_set_eq:NN \l__acro_alt_format_tl \l__acro_short_format_tl
      \tl_set:Nn \l__acro_list_short_format_tl {#1} ,
    short-format      .value_required:n = true ,
    long-format       .code:n     =
      \tl_set:Nn \l__acro_long_format_tl {#1}
      \tl_set:Nn \l__acro_first_long_format_tl {#1}
      \tl_set:Nn \l__acro_list_long_format_tl {#1} ,
    long-format       .value_required:n = true ,
    first-long-format .code:n     =
      \tl_set:Nn \l__acro_first_long_format_tl {#1} ,
    first-long-format .value_required:n = true ,
    single-format     .tl_set:N   = \l__acro_single_format_tl ,
    single-format     .value_required:n = true ,
    format-include-endings .bool_set:N = \l__acro_include_endings_format_bool ,
    display-foreign   .bool_set:N = \l__acro_foreign_bool ,
    foreign-format    .tl_set:N   = \l__acro_foreign_format_tl ,
    foreign-format    .value_required:n = true ,
    list-short-format .tl_set:N   = \l__acro_list_short_format_tl ,
    list-short-format .value_required:n = true ,
    list-short-width  .dim_set:N  = \l__acro_short_width_dim ,
    list-short-width  .value_required:n = true ,
    list-long-format  .tl_set:N   = \l__acro_list_long_format_tl ,
    list-long-format  .value_required:n = true ,
    list-foreign-format .tl_set:N = \l__acro_foreign_list_format_tl ,
    list-foreign-format .value_required:n = true ,
    extra-format      .tl_set:N   = \l__acro_extra_format_tl ,
    extra-format      .value_required:n = true ,
    single            .bool_set:N = \l__acro_use_single_bool ,
    single-form       .tl_set:N   = \l__acro_single_form_tl ,
    single-form       .value_required:n = true ,
    first-style       .code:n     = \acro_set_first_style:n {#1} ,
    first-style       .value_required:n = true ,
    extra-style       .code:n     = \acro_set_extra_style:n {#1} ,
    extra-style       .value_required:n = true ,
    label             .bool_set:N = \l__acro_place_label_bool ,
    label-prefix      .tl_set:N   = \l__acro_label_prefix_tl ,
    label-prefix      .value_required:n = true ,
    pages             .choice: ,
    pages / all       .code:n     =
      \bool_set_true:N \l__acro_list_all_pages_bool ,
    pages / first     .code:n     =
      \bool_set_true:N \l__acro_place_label_bool
      \bool_set_false:N \l__acro_list_all_pages_bool ,
    pages             .value_required:n = true ,
    page-ref          .code:n     =
      \acro_option_deprecated:nn {page-ref} {page-style}
      \acro_set_page_style:n {#1} ,
    page-style        .code:n     = \acro_set_page_style:n {#1} ,
    page-style        .value_required:n = true ,
    page-name         .tl_set:N   = \l__acro_page_name_tl ,
    page-name         .value_required:n = true ,
    pages-name        .tl_set:N   = \l__acro_pages_name_tl ,
    pages-name        .value_required:n = true ,
    following-page    .bool_set:N = \l__acro_following_page_bool ,
    following-pages   .bool_set:N = \l__acro_following_pages_bool ,
    following-pages*  .meta:n     =
      { following-page = #1 , following-pages = #1 } ,
    following-pages*  .default:n  = true ,
    next-page         .tl_set:N   = \l__acro_next_page_tl ,
    next-page         .value_required:n = true ,
    next-pages        .tl_set:N   = \l__acro_next_pages_tl ,
    next-pages        .value_required:n = true ,
    list-style        .code:n     = \acro_set_list_style:n {#1} ,
    list-style        .value_required:n = true ,
    list-heading      .code:n     = \acro_set_list_heading:n {#1} ,
    list-heading      .value_required:n = true ,
    list-name         .tl_set:N   = \l__acro_list_name_tl ,
    list-name         .value_required:n = true ,
    hyperref          .bool_set:N = \l__acro_use_hyperref_bool ,
    only-used         .bool_set:N = \l__acro_print_only_used_bool ,
    mark-as-used      .choice: ,
    mark-as-used / first .code:n  =
      \bool_gset_true:N \g__acro_mark_first_as_used_bool ,
    mark-as-used / any   .code:n  =
      \bool_gset_false:N \g__acro_mark_first_as_used_bool ,
    mark-as-used      .default:n  = any ,
    list-caps         .bool_set:N = \l__acro_capitalize_list_bool ,
    cite              .choice: ,
    cite / all        .code:n     =
      \bool_set_true:N \l__acro_citation_all_bool
      \bool_set_true:N \l__acro_citation_first_bool ,
    cite / none       .code:n     =
      \bool_set_false:N \l__acro_citation_all_bool
      \bool_set_false:N \l__acro_citation_first_bool ,
    cite / first      .code:n     =
      \bool_set_false:N \l__acro_citation_all_bool
      \bool_set_true:N  \l__acro_citation_first_bool ,
    cite              .default:n  = all ,
    cite-cmd          .code:n     =
      \cs_set:Npn \__acro_citation_cmd:w {#1} ,
    cite-cmd          .value_required:n = true ,
    group-cite-cmd    .code:n     =
      \cs_set:Npn \__acro_group_citation_cmd:w {#1} ,
    group-cite-cmd    .value_required:n = true ,
    group-citation    .bool_set:N = \l__acro_group_citation_bool ,
    cite-connect      .tl_set:N   = \l__acro_citation_connect_tl ,
    cite-connect      .initial:n  = \nobreakspace ,
    cite-connect      .value_required:n = true ,
    group-cite-connect .tl_set:N = \l__acro_between_group_connect_citation_tl ,
    group-cite-connect .initial:n = {,\nobreakspace} ,
    group-cite-connect .value_required:n = true ,
    index             .bool_set:N = \l__acro_addto_index_bool ,
    index-cmd         .code:n     =
      \cs_set:Npn \__acro_index_cmd:n {#1} ,
    index-cmd         .value_required:n = true ,
    uc-cmd            .code:n     =
      \cs_set_eq:NN \__acro_first_upper_case:n #1 ,
    uc-cmd            .value_required:n = true
  }

\AtBeginDocument
  {
    \bool_if:NTF \l__acro_xspace_bool
      {
        \@ifpackageloaded {xspace}
          { }
          { \RequirePackage {xspace} }
        \cs_new_eq:NN \acro_xspace: \xspace
      }
      { \cs_new:Npn \acro_xspace: {} }
  }

% --------------------------------------------------------------------------
% setup command:
\NewDocumentCommand \acsetup { m }
  { \keys_set:nn {acro} {#1} \ignorespaces }

% --------------------------------------------------------------------------
% we use xtemplate for different object types and with a different number of
% arguments; let's declare functions for usage later so we don't have to
% bother

% objects with one argument:
\cs_new_protected:Npn \acro_page_number_instance:nn #1#2
  { \UseInstance {acro-page-number} {#1} {#2} }
\cs_generate_variant:Nn \acro_page_number_instance:nn {V}

\cs_new_protected:Npn \acro_extra_instance:nn #1#2
  { \UseInstance {acro-extra} {#1} {#2} }
\cs_generate_variant:Nn \acro_extra_instance:nn {VV}

\cs_new_protected:Npn \acro_title_instance:nn #1#2
  { \UseInstance {acro-title} {#1} {#2} }
\cs_generate_variant:Nn \acro_title_instance:nn {VV}

% objects with two arguments:
\cs_new_protected:Npn \acro_list_instance:nnn #1#2#3
  { \UseInstance {acro-list} {#1} {#2} {#3} }
\cs_generate_variant:Nn \acro_list_instance:nnn {VVV}

\cs_new_protected:Npn \acro_first_instance:nn #1#2
  {
    \tl_if_blank:VF \l__acro_first_style_tl
      {
        \tl_set_eq:NN
          \l__acro_first_instance_tl
          \l__acro_first_style_tl
      }
    \acro_if_defined:nT {#1}
      {
        \use:x {
          \UseInstance {acro-first}
            { \exp_not:V \l__acro_first_instance_tl }
            { \exp_not:n {#1} }
            { \exp_not:n {#2} }
          }
      }
  }
\cs_generate_variant:Nn \acro_first_instance:nn {nV}
  
% --------------------------------------------------------------------------
% hyperref support
\cs_new_eq:NN \acro_hyper_target:nn \use_ii:nn
\cs_new_eq:NN \acro_hyper_link:nn   \use_ii:nn

\cs_new_protected:Npn \acro_activate_hyperref_support:
  {
    \bool_if:nT { \l__acro_hyperref_loaded_bool && \l__acro_use_hyperref_bool }
      {
        \cs_set_eq:NN \acro_hyper_link:nn \hyperlink
        \cs_set:Npn \acro_hyper_target:nn ##1##2
          { \raisebox { 3ex } [ 0pt ] { \hypertarget {##1} { } } ##2 }
      }
  }

% #1: tl var
% #2: id
% #3: text
\cs_new_protected:Npn \__acro_make_link:Nnn #1#2#3
  {
    \bool_if:nTF
      { \l__acro_use_hyperref_bool && \l__acro_hyperref_loaded_bool }
      {
        \tl_set:Nn #1
           {
             \acro_hyper_link:nn {#2} { \phantom {#3} }
             \acro_if_is_single:nTF {#2}
               { \hbox_overlap_left:n {#3} }
               { \acro_color_link:n { \hbox_overlap_left:n {#3} } }
           }
       }
       { \tl_set:Nn #1 {#3} }
  }
\cs_generate_variant:Nn \__acro_make_link:Nnn {NnV}

\cs_new:Npn \acro_color_link:n #1
  {
    \cs_if_exist:NTF \hypersetup
      {
        \ifHy@colorlinks
          \exp_after:wN \use_i:nn
        \else
          \ifHy@ocgcolorlinks
            \exp_after:wN \use_i:nn
          \else
            \exp_after:wN \exp_after:wN \exp_after:wN \use_ii:nn
          \fi
        \fi
        { \textcolor { \@linkcolor } {#1} }
        {#1}
      }
      {#1}
  }

\AtBeginDocument{
  \cs_if_exist:NF \textcolor { \cs_new_eq:NN \textcolor \use_ii:nn }
}

% --------------------------------------------------------------------------
% output style of the first time an acronym is used

% helper macros for the styles
% #1: short|long
% #2: id
% #3: long
\cs_new_protected:Npn \__acro_print_form_and_indefinite:nnn #1#2#3
  {
    \group_begin:
      \acro_for_all_trailing_tokens_do:n
        { \acro_deactivate_trailing_action:n {##1} }
      \str_case:nn {#1}
        {
          {long} {
            \bool_if:nT
              {
                \l__acro_first_only_long_bool ||
                !\l__acro_first_only_short_bool
              }
              {
                \acro_write_indefinite:nn {#2} {long}
                \acro_write_expanded:nnn {#2} {first-long} {#3}
              }
          }
          {short} {
            \bool_if:nT
              {
                !\l__acro_first_only_long_bool ||
                \l__acro_first_only_short_bool
              }
              {
                \acro_write_indefinite:nn {#2} {short}
                \acro_write_compact:nn {#2} {short}
              }
          }
        }
    \group_end:
  }

\cs_new_protected:Npn \__acro_open_bracket:
  {
    \bool_if:nT
      {
        !\l__acro_first_only_long_bool &&
        !\l__acro_first_only_short_bool
      }
      {
        \acro_space:
        \tl_if_blank:VF \l__acro_first_between_tl
          {
            \tl_use:N \l__acro_first_between_tl
            \acro_space:
          }
        \bool_if:NT \l__acro_first_use_brackets_bool
          { \tl_head:N \l__acro_first_brackets_tl }
      }
  }

\cs_new_protected:Npn \__acro_close_bracket:
  {
    \bool_if:nT
      {
        \l__acro_first_use_brackets_bool &&
        !\l__acro_first_only_short_bool &&
        !\l__acro_first_only_long_bool
      }
      { \tl_tail:N \l__acro_first_brackets_tl }
  }
  
% #1: short|long
% #2: id
% #3: long
\cs_new_protected:Npn \__acro_print_form:nnn #1#2#3
  {
    \str_case:nn {#1}
      {
        {long} {
          \bool_if:nT
            {
              \l__acro_first_only_long_bool ||
              !\l__acro_first_only_short_bool
            }
            { \acro_write_expanded:nnn {#2} {first-long} {#3} }
        }
        {short} {
          \bool_if:nT
            {
              !\l__acro_first_only_long_bool ||
              \l__acro_first_only_short_bool
            }
            { \acro_write_compact:nn {#2} {short} }
        }
      }
  }

% #1: id
\cs_new_protected:Npn \__acro_foreign_sep:n #1
  {
    \bool_if:nT
      {
         \l__acro_foreign_bool &&
        !\l__acro_first_only_short_bool &&
        !\l__acro_first_only_long_bool
      }
      { \acro_if_foreign:nT {#1} { \tl_use:N \l__acro_foreign_sep_tl } }
  }
  
% #1: id
\cs_new_protected:Npn \__acro_print_foreign:n #1
  {
    \bool_if:nT
      {
         \l__acro_foreign_bool &&
        !\l__acro_first_only_short_bool &&
        !\l__acro_first_only_long_bool
      }
      { \acro_write_foreign:n {#1} }
  }

\cs_new_protected:Npn \__acro_print_citation:n #1
  {
    \bool_if:NT \l__acro_group_citation_bool
      { \acro_group_cite:n {#1} }
  }

\cs_new_protected:Npn \__acro_finalize_first:n #1
  {
    \bool_if:NF \l__acro_group_citation_bool
      { \acro_cite_if:nn { \l__acro_citation_first_bool } {#1} }
    \acro_index_if:nn { \l__acro_addto_index_bool } {#1}
  }

% --------------------------------------------------------------------------
% the `acro-first' object, templates, instances:
% #1: id
% #2: long
\DeclareObjectType {acro-first} {2}

% template for inline appearance:
\DeclareTemplateInterface {acro-first} {inline} {2}
  {
    brackets      : boolean   = true  ,
    brackets-type : tokenlist = ()    ,
    only-short    : boolean   = false ,
    only-long     : boolean   = false ,
    reversed      : boolean   = false ,
    between       : tokenlist         ,
    foreign-sep   : tokenlist = {,~}
  }
\DeclareTemplateCode {acro-first} {inline} {2}
  {
    brackets      = \l__acro_first_use_brackets_bool ,
    brackets-type = \l__acro_first_brackets_tl       ,
    only-short    = \l__acro_first_only_short_bool   ,
    only-long     = \l__acro_first_only_long_bool    ,
    reversed      = \l__acro_first_reversed_bool     ,
    between       = \l__acro_first_between_tl        ,
    foreign-sep   = \l__acro_foreign_sep_tl
  }
  {
    \AssignTemplateKeys
    \bool_if:NTF \l__acro_first_reversed_bool
      { % zuerst kurze Form, dann lange Form:
        \__acro_print_form_and_indefinite:nnn {short} {#1} {#2}
        \__acro_open_bracket:
        \__acro_print_foreign:n {#1}
        \__acro_foreign_sep:n {#1}
        \__acro_print_form:nnn {long} {#1} {#2}
        \__acro_print_citation:n {#1}
        \__acro_close_bracket:
        \__acro_finalize_first:n {#1}
      }
      { % zuerst lange Form, dann kurze Form:
        \__acro_print_form_and_indefinite:nnn {long} {#1} {#2}
        \__acro_open_bracket:
        \__acro_print_foreign:n {#1}
        \__acro_foreign_sep:n {#1}
        \__acro_print_form:nnn {short} {#1} {#2}
        \__acro_print_citation:n {#1}
        \__acro_close_bracket:
        \__acro_finalize_first:n {#1}
      }
  }

% template for footnotes, sidenotes, ...
\cs_new:Npn \__acro_note_command:n #1 {#1}
\DeclareTemplateInterface {acro-first} {note} {2}
  {
    use-note     : boolean    = true ,
    note-command : function 1 = \footnote {#1} ,
    foreign-sep  : tokenlist  = {,~} ,
    reversed      : boolean   = false ,
  }

\DeclareTemplateCode {acro-first} {note} {2}
  {
    use-note     = \l__acro_use_note_bool  ,
    note-command = \__acro_note_command:n  ,
    foreign-sep  = \l__acro_foreign_sep_tl ,
    reversed     = \l__acro_first_reversed_bool
  }
  {
    \AssignTemplateKeys
    \bool_if:NTF \l__acro_first_reversed_bool
      { % long in text and short in note
        \__acro_print_form_and_indefinite:nnn {long} {#1} {#2}
        \bool_if:NT \l__acro_use_note_bool
          {
            \__acro_note_command:n
              {
                \__acro_print_foreign:n {#1}
                \__acro_foreign_sep:n {#1}
                \__acro_print_form:nnn {short} {#1} {#2}
                \__acro_print_citation:n {#1}
                \__acro_finalize_first:n {#1}
              }
          }
      }
      { % short in text and long in note
        \__acro_print_form_and_indefinite:nnn {short} {#1} {#2}
        \bool_if:NT \l__acro_use_note_bool
          {
            \__acro_note_command:n
              {
                \__acro_print_foreign:n {#1}
                \__acro_foreign_sep:n {#1}
                \__acro_print_form:nnn {long} {#1} {#2}
                \__acro_print_citation:n {#1}
                \__acro_finalize_first:n {#1}
              }
          }
      }
  }

% --------------------------------------------------------------------------
% declare new first styles:
\cs_new_protected:Npn \acro_declare_first_style:nnn #1#2#3
  {
    \DeclareInstance {acro-first} {#1} {#2} {#3}
    \prop_put:Nnn \l__acro_first_styles_prop  {#1} {#2}
  }

% #1: name
% #2: template
% #3: settings
\NewDocumentCommand \DeclareAcroFirstStyle {mmm}
  { \acro_declare_first_style:nnn {#1} {#2} {#3} }

% set a list style
\cs_new_protected:Npn \acro_set_first_style:n #1
  {
    \prop_if_in:NnTF \l__acro_first_styles_prop {#1}
      { \__acro_set_first_style:n {#1} }
      {
        \msg_warning:nnnnn {acro} {unknown}
          {first~ style}
          {#1}
          {default}
        \__acro_set_first_style:n {default}
      }
  }

\cs_new_protected:Npn \__acro_set_first_style:n #1
  {
    \tl_set:Nn \l__acro_first_instance_tl {#1}
    \prop_get:NnN \l__acro_first_styles_prop {#1} \l__acro_tmpa_tl
  }

% --------------------------------------------------------------------------
% formatting the extras information:
\DeclareObjectType {acro-extra} {1}

\DeclareTemplateInterface {acro-extra} {inline} {1}
  {
    punct         : boolean   = false ,
    punct-symbol  : tokenlist = {,}   ,
    brackets      : boolean   = true  ,
    brackets-type : tokenlist = ()
  }

\DeclareTemplateCode {acro-extra} {inline} {1}
  {
    punct         = \l__acro_extra_punct_bool        ,
    punct-symbol  = \l__acro_extra_punct_tl          ,
    brackets      = \l__acro_extra_use_brackets_bool ,
    brackets-type = \l__acro_extra_brackets_tl
  }
  {
    \AssignTemplateKeys
    \bool_if:NT \l__acro_extra_punct_bool
      { \tl_use:N \l__acro_extra_punct_tl \tl_use:N \c_space_tl }
    \bool_if:NT \l__acro_extra_use_brackets_bool
      { \tl_head:N \l__acro_extra_brackets_tl }
    \acro_write_long:Vn \l__acro_extra_format_tl {#1}
    \bool_if:NT \l__acro_extra_use_brackets_bool
      { \tl_tail:N \l__acro_extra_brackets_tl }
  }

% declare new extra styles:
\cs_new_protected:Npn \acro_declare_etxra_style:nnn #1#2#3
  {
    \DeclareInstance {acro-etxra} {#1} {#2} {#3}
    \prop_put:Nnn \l__acro_etxra_styles_prop  {#1} {#2}
  }

% #1: name
% #2: template
% #3: settings
\NewDocumentCommand \DeclareAcroExtraStyle {mmm}
  { \acro_declare_extra_style:nnn {#1} {#2} {#3} }

% set an extra style
\cs_new_protected:Npn \acro_set_extra_style:n #1
  {
    \prop_if_in:NnTF \l__acro_extra_styles_prop {#1}
      { \__acro_set_extra_style:n {#1} }
      {
        \msg_warning:nnnnn {acro} {unknown}
          {extra~ style}
          {#1}
          {default}
        \__acro_set_extra_style:n {default}
      }
  }

\cs_new_protected:Npn \__acro_set_extra_style:n #1
  {
    \tl_set:Nn \l__acro_extra_instance_tl {#1}
    \prop_get:NnN \l__acro_extra_styles_prop {#1} \l__acro_tmpa_tl
  }

\cs_new_protected:Npn \acro_declare_extra_style:nnn #1#2#3
  {
    \DeclareInstance {acro-extra} {#1} {#2} {#3}
    \prop_put:Nnn \l__acro_extra_styles_prop  {#1} {#2}
  }

% --------------------------------------------------------------------------
% outputting the page numbers:
\RequirePackage {zref-abspage}

\cs_new_protected:Npn \__acro_create_page_records:n #1
  {
    \seq_new:c { g__acro_#1_pages_seq }
    \tl_new:c  { g__acro_#1_recorded_pages_tl }
  }

\cs_new_protected:Npn \acro_hyper_page:n #1 { \use:n {#1} }

\cs_new:Npn \acro_get_thepage:nnn #1#2#3 { \acro_hyper_page:n {#1} }
\cs_new:Npn \acro_get_thepage_from:N #1
  { \exp_after:wN \acro_get_thepage:nnn #1 }

\cs_new:Npn \acro_get_page_number:nnn #1#2#3 {#2}
\cs_new:Npn \acro_get_page_number_from:N #1
  { \exp_after:wN \acro_get_page_number:nnn #1 }

\cs_new:Npn \acro_get_abspage:nnn #1#2#3 {#3}
\cs_new:Npn \acro_get_abspage_from:N #1
  { \exp_after:wN \acro_get_abspage:nnn #1 }

\cs_new:Npn \acro_page_range_comma: {}

\cs_new_protected:Npn \acro_print_page_numbers:n #1
  {
    \seq_if_empty:cF {g__acro_#1_pages_seq}
      {
        \bool_if:NTF \l__acro_list_all_pages_bool
          {
            % have the numbers changed?
            \tl_set:Nx \l__acro_tmpa_tl
              { \seq_use:cn {g__acro_#1_pages_seq} {|} }
            \tl_if_eq:cNF {g__acro_#1_recorded_pages_tl} \l__acro_tmpa_tl
              {
                \@latex@warning@no@line
                  {Rerun~to~get~page~numbers~of~acronym~#1~in~acronym~list~right}
              }
            \tl_clear:N \l__acro_write_pages_tl
            \tl_clear:N \l__acro_last_page_tl
            \tl_clear:N \l__acro_current_page_tl
            \seq_set_eq:Nc \l__acro_tmpb_seq { g__acro_#1_pages_seq }
            \seq_remove_duplicates:N \l__acro_tmpb_seq
            \seq_clear:N \l__acro_tmpa_seq
            \cs_set_protected:Npn \acro_page_range_comma:
              { \cs_set:Npn \acro_page_range_comma: { ,~ } }
            % get the numbers:
            \int_compare:nNnTF { \seq_count:N \l__acro_tmpb_seq } = { 1 }
              {
                \tl_use:N \l__acro_page_name_tl
                \seq_get_right:cN { g__acro_#1_pages_seq } \l__acro_tmpa_tl
                \acro_get_thepage_from:N \l__acro_tmpa_tl
              }
              {
                \tl_use:N \l__acro_pages_name_tl
                \seq_map_inline:cn { g__acro_#1_pages_seq }
                  {
                    \tl_if_blank:VTF \l__acro_last_page_tl
                      {% we're at the beginning
                        \seq_put_right:Nn \l__acro_tmpa_seq {##1}
                        \tl_set:Nn \l__acro_last_page_tl {##1}
                      }
                      {% we'at least at the second page
                         % current page:
                         \tl_set:Nn  \l__acro_current_page_tl {##1}
                         % last page:
                         \seq_get_right:NN \l__acro_tmpa_seq \l__acro_last_page_tl
                         \tl_if_eq:NNTF \l__acro_current_page_tl \l__acro_last_page_tl
                           {% there were more than one appearance on the current page
                             \seq_put_right:Nn \l__acro_tmpa_seq {##1}
                           }
                           {% new page
                             \acro_determine_page_ranges:NNn
                               \l__acro_tmpa_seq
                               \l__acro_write_pages_tl
                               {##1}
                           }
                      }
                  }
                \seq_if_empty:NF \l__acro_tmpa_seq
                  {
                    \acro_determine_page_ranges:NNV
                      \l__acro_tmpa_seq
                      \l__acro_write_pages_tl
                      \l__acro_current_page_tl
                  }
                \tl_use:N \l__acro_write_pages_tl
                \tl_clear:N \l__acro_write_pages_tl
              }
          }
          {
            \tl_use:N \l__acro_page_name_tl
            \pageref{\l__acro_label_prefix_tl #1}
          }
      }
    \seq_clear:N \l__acro_tmpa_seq
    \seq_clear:N \l__acro_tmpb_seq
  }

\cs_new:Npn \acro_determine_page_ranges:NNn #1#2#3
  {
    \seq_remove_duplicates:N #1
    % current page:
    \int_set:Nn \l__acro_tmpa_int { \acro_get_abspage:nnn #3 }
    \int_set:Nn \l__acro_tmpb_int { \acro_get_page_number:nnn #3 }
    % last page:
    \seq_get_right:NN #1 \l__acro_last_page_tl
    \int_set:Nn \l__acro_tmpc_int
      { \acro_get_abspage_from:N \l__acro_last_page_tl }
    \int_set:Nn \l__acro_tmpd_int
      { \acro_get_page_number_from:N \l__acro_last_page_tl }
    \bool_if:nTF
      {
        \int_compare_p:nNn
          { \l__acro_tmpa_int - \l__acro_tmpc_int }
           =
          { \l__acro_tmpb_int - \l__acro_tmpd_int }
        &&
        \int_compare_p:nNn
        { \l__acro_tmpb_int - \l__acro_tmpd_int } = {1}
      }
      {% same kind of page numbering, one page ahead
       % => possible range
         \seq_put_right:Nn #1 {#3}
      }
      {% any possible range ended
        \tl_put_right:Nn #2 { \acro_page_range_comma: }
        \int_compare:nNnTF
          { \seq_count:N #1 } > {2}
          {% real range
            \seq_get_left:NN #1 \l__acro_tmpa_tl
            \tl_put_right:Nx #2 { \acro_get_thepage_from:N \l__acro_tmpa_tl }
            \bool_if:NTF \l__acro_following_pages_bool
              { \tl_put_right:Nn #2 { \l__acro_next_pages_tl } }
              {
                \tl_put_right:Nn #2 { -- }
                \seq_get_right:NN #1 \l__acro_tmpa_tl
                \tl_put_right:Nx #2 { \acro_get_thepage_from:N \l__acro_tmpa_tl }
              }
          }
          {
            \int_compare:nNnTF
              { \seq_count:N #1 } = {2}
              {% range of two pages
                \seq_get_left:NN #1 \l__acro_tmpa_tl
                \tl_put_right:Nx #2 { \acro_get_thepage_from:N \l__acro_tmpa_tl }
                \bool_if:NTF \l__acro_following_page_bool
                  { \tl_put_right:Nn #2 { \l__acro_next_page_tl } }
                  {
                    \tl_put_right:Nn #2 { ,~ }
                    \seq_get_right:NN #1 \l__acro_tmpa_tl
                    \tl_put_right:Nx #2 { \acro_get_thepage_from:N \l__acro_tmpa_tl }
                  }
              }
              {% no range at all
                \seq_get_right:NN #1 \l__acro_tmpa_tl
                \tl_put_right:Nx #2 { \acro_get_thepage_from:N \l__acro_tmpa_tl }
              }
          }
        \seq_clear:N #1
        \seq_put_right:Nn #1 {#3}
      }
  }
\cs_generate_variant:Nn \acro_determine_page_ranges:NNn { NNV }

% --------------------------------------------------------------------------
\DeclareObjectType {acro-page-number} {1}

\DeclareTemplateInterface {acro-page-number} {inline} {1}
  {
    display       : boolean   = true  ,
    punct         : boolean   = false ,
    punct-symbol  : tokenlist = {,}   ,
    brackets      : boolean   = false ,
    brackets-type : tokenlist = ()    ,
    space         : skip      = .333333em plus .166666em minus .111111em
  }

\DeclareTemplateCode {acro-page-number} {inline} {1}
  {
    display       = \l__acro_page_display_bool  ,
    punct         = \l__acro_page_punct_bool    ,
    punct-symbol  = \l__acro_page_punct_tl      ,
    brackets      = \l__acro_page_brackets_bool ,
    brackets-type = \l__acro_page_brackets_tl   ,
    space         = \l__acro_page_space_skip
  }
  {
    \AssignTemplateKeys
    \bool_if:NT \l__acro_page_display_bool
      {
        \bool_if:NT \l__acro_page_punct_bool
          { \tl_use:N \l__acro_page_punct_tl }
        % \tl_use:N \c_space_tl
        \dim_compare:nNnF { \l__acro_page_space_skip } = { 0pt }
          { \skip_horizontal:N \l__acro_page_space_skip }
        \bool_if:NT \l__acro_page_brackets_bool
          { \tl_head:N \l__acro_page_brackets_tl }
        \acro_print_page_numbers:n {#1}
        \bool_if:NT \l__acro_page_brackets_bool
          { \tl_tail:N \l__acro_page_brackets_tl }
      }
  }

% declare new page styles:
\cs_new_protected:Npn \acro_declare_page_style:nnn #1#2#3
  {
    \DeclareInstance {acro-page-number} {#1} {#2} {#3}
    \prop_put:Nnn \l__acro_page_styles_prop  {#1} {#2}
  }

% #1: name
% #2: template
% #3: settings
\NewDocumentCommand \DeclareAcroPageStyle {mmm}
  { \acro_declare_page_style:nnn {#1} {#2} {#3} }

% set a page style
\cs_new_protected:Npn \acro_set_page_style:n #1
  {
    \prop_if_in:NnTF \l__acro_page_styles_prop {#1}
      { \__acro_set_page_style:n {#1} }
      {
        \msg_warning:nnnnn {acro} {unknown}
          {page~ style}
          {#1}
          {none}
        \__acro_set_page_style:n {none}
      }
  }

\cs_new_protected:Npn \__acro_set_page_style:n #1
  {
    \tl_set:Nn \l__acro_page_instance_tl {#1}
    \prop_get:NnN \l__acro_page_styles_prop {#1} \l__acro_tmpa_tl
  }

% --------------------------------------------------------------------------
% the title of the list:
\cs_new:Npn \acro_list_title_format:n #1 {#1}

\DeclareObjectType {acro-title} {1}

\DeclareTemplateInterface {acro-title} {sectioning} {1}
  { name-format : function 1 = #1 }

\DeclareTemplateCode {acro-title} {sectioning} {1}
  { name-format = \acro_list_title_format:n }
  {
    \AssignTemplateKeys
    \acro_list_title_format:n {#1}
  }

% set a list heading:
\cs_new_protected:Npn \acro_set_list_heading:n #1
  {
    \prop_if_in:NnTF \l__acro_list_headings_prop {#1}
      { \__acro_set_list_heading:n {#1} }
      {
        \msg_warning:nnnnn {acro} {unknown}
          {list~ heading}
          {#1}
          {section*}
        \__acro_set_list_heading:n {section*}
      }
  }

\cs_new_protected:Npn \__acro_set_list_heading:n #1
  {
    \tl_set:Nn \l__acro_list_heading_cmd_tl {#1}
    % \prop_get:NnN \l__acro_list_headings_prop
    %   {#1}
    %   \l__acro_list_heading_cmd_tl
  }
  
\cs_new_protected:Npn \acro_declare_list_heading:nn #1#2
  {
    \prop_put:Nnn \l__acro_list_headings_prop {#1} {#2}
    \DeclareInstance {acro-title} {#1} {sectioning}
      { name-format = #2 {##1} }
  }

\NewDocumentCommand \DeclareAcroListHeading {mm}
  { \acro_declare_list_heading:nn {#1} {#2} }

% --------------------------------------------------------------------------
% typesetting the acronym list
\DeclareObjectType {acro-list} {2}

% #1: id
% #2: excluded classes
\prg_new_protected_conditional:Npnn \acro_if_is_excluded:nn #1#2 {T,F,TF}
  {
    \bool_set_false:N \l__acro_is_excluded_bool
    \tl_if_blank:nF {#2}
      {
        \clist_map_inline:nn {#2}
          {
            \prop_get:NnNT \l__acro_class_prop {#1} \l__acro_tmpa_tl
              {
                \seq_set_split:NnV \l__acro_tmpa_seq {,} \l__acro_tmpa_tl
                \seq_if_in:NnT \l__acro_tmpa_seq {##1}
                  { \bool_set_true:N \l__acro_is_excluded_bool }
              }
          }
      }
    \bool_if:NTF \l__acro_is_excluded_bool
      { \prg_return_true: }
      { \prg_return_false: }
  }

% #1: id
% #2: included classes
\prg_new_protected_conditional:Npnn \acro_if_is_included:nn #1#2 {T,F,TF}
  {
    \bool_set_false:N \l__acro_is_included_bool
    \tl_if_blank:nTF {#2}
      { \bool_set_true:N \l__acro_is_included_bool }
      {
        \clist_map_inline:nn {#2}
          {
            \prop_get:NnNT \l__acro_class_prop {#1} \l__acro_tmpa_tl
              {
                \seq_set_split:NnV \l__acro_tmpa_seq {,} \l__acro_tmpa_tl
                \seq_if_in:NnT \l__acro_tmpa_seq {##1}
                  { \bool_set_true:N \l__acro_is_included_bool }
              }
          }
      }
    \bool_if:NTF \l__acro_is_included_bool
      { \prg_return_true: }
      { \prg_return_false: }
  }

% #1: id
\cs_new_protected:Npn \__acro_list_entry_short:n #1
  {
    \acro_hyper_target:nn
      {#1}
      {
        \acro_acc_supp:nn
          {#1}
          {
            \acro_write_short:nn {#1}
              {
                \l__acro_list_short_format_tl
                { \__acro_get_property:nn {short} {#1} }
              }
          }
      }
  }

% #1: id
\cs_new_protected:Npn \__acro_list_entry_long:n #1
  {
    \group_begin:
      \bool_if:NT \l__acro_capitalize_list_bool
        { \bool_set_true:N \l__acro_first_upper_bool }
      \acro_write_long:Vf \l__acro_list_long_format_tl
        {
          \prop_if_in:NnTF \l__acro_list_prop {#1}
            { \__acro_get_property:nn {list} {#1} }
            { \__acro_get_property:nn {long} {#1} }
        }
    \group_end:
    \bool_if:NT \l__acro_foreign_bool
      { \acro_get_foreign:n {#1} }
    \acro_cite_if:nn { \l__acro_citation_all_bool } {#1}
  }

% #1: id
\cs_new_protected:Npn \__acro_list_entry_extra:n #1
  {
    \prop_get:NnNT \l__acro_extra_prop {#1} \l__acro_tmpa_tl
      {
        \acro_extra_instance:VV
          \l__acro_extra_instance_tl
          \l__acro_tmpa_tl
      }
  }

% #1: id
\cs_new_protected:Npn \__acro_list_entry_page:n #1
  {
    \bool_if:nT { \cs_if_exist_p:c { acro@#1@once } }
      {
        \acro_page_number_instance:Vn
          \l__acro_page_instance_tl
          {#1}
      }
  }
  
% macro for retrieval of items in the list:
% #1: property
% #2: id
\cs_new_protected:Npn \acro_list_entry:nn #1#2
  {
    \str_case:nnF {#1}
      {
        {short} { \__acro_list_entry_short:n {#2} }
        {long}  { \__acro_list_entry_long:n {#2} }
        {extra} { \__acro_list_entry_extra:n {#2} }
        {page}  { \__acro_list_entry_page:n {#2} }
      }
      { \__acro_get_property:nn {#1} {#2} }
  }

% this macro may/should be redefined in templates:
% #1: short
% #2: long
% #3: extra
% #4: page number(s)
\cs_new_protected:Npn \acro_print_list_entry:nnnn #1#2#3#4
  { #1 #2 #3 #4 }

\cs_new_protected:Npn \acro_for_all_acronyms_do:n #1
  { \prop_map_inline:Nn \l__acro_short_prop {#1} }

% test, if acronyms should be printed or not; needs testing for in/excluded
% classes and options `only-used' and `single' -- this macro should be used in
% the template code for retrieving the list
  
% #1: id
% #2: included classes
% #3: excluded classes
\prg_new_protected_conditional:Npnn \acro_if_entry:nnn #1#2#3 {T,F,TF}
  {
    \bool_if:nTF
      {
        \bool_if_p:c { g__acro_#1_in_list_bool } &&
        (
          ( \l__acro_use_single_bool && \cs_if_exist_p:c { acro@#1@twice } )
          ||
          (
            !\l__acro_use_single_bool &&
            \cs_if_exist_p:c { acro@#1@once } &&
            \l__acro_print_only_used_bool
          )
        )
        ||
        ( !\l__acro_use_single_bool && !\l__acro_print_only_used_bool )
      }
      {
        \acro_if_is_excluded:nnTF {#1} {#3}
          { \prg_return_false: }
          {
            \acro_if_is_included:nnTF {#1} {#2}
              {
                \bool_if:nTF
                  { \g__acro_use_barriers_bool && \l__acro_use_barrier_bool }
                  {
                    \acro_if_in_barrier:nxTF {#1}
                      { \int_use:N \g__acro_barrier_int }
                      { \prg_return_true: }
                      { \prg_return_false: }
                  }
                  { \prg_return_true: }
              }
              { \prg_return_false: }
          }
      }
      { \prg_return_false: }
  }

\tl_new:N \l__acro_list_entries_tl

% this macro is used in templates for fetching all items to be printed; it
% collects all entries in a tl which then is used where needed
%
% #1: tl containing the entries
% #2: included classes
% #3: excluded classes
\cs_new_protected:Npn \acro_build_list_entries:Nnn #1#2#3
  {
    \tl_clear:N #1
    \acro_for_all_acronyms_do:n
      {% ##1: id; ##2: short form
        \acro_get:n {##1}
        \acro_if_entry:nnnT {##1} {#2} {#3}
          {
            \tl_put_right:Nn #1
              {
                \acro_print_list_entry:nnnn
                  { \acro_list_entry:nn {short} {##1} }
                  { \acro_list_entry:nn {long} {##1} }
                  { \acro_list_entry:nn {extra} {##1} }
                  { \acro_list_entry:nn {page} {##1} }
              }
          }
      }
  }

% this macro is used in templates for fetching all items to be printed:
\cs_new_protected:Npn \acro_list_items:nn #1#2
  {
    \acro_build_list_entries:Nnn \l__acro_list_entries_tl {#1} {#2}
    \tl_use:N \l__acro_list_entries_tl
  }
  
% --------------------------------------------------------------------------
% declare templates for the list:
% `list' template:
\DeclareTemplateInterface {acro-list} {list} {2}
  {
    foreign-sep : tokenlist = {~} ,
    list        : tokenlist = {description} ,
    reverse     : boolean   = false ,
    before      : tokenlist = ,
    after       : tokenlist =
  }

\DeclareTemplateCode {acro-list} {list} {2}
  {
    foreign-sep = \l__acro_foreign_sep_tl ,
    list        = \l__acro_list_tl ,
    reverse     = \l__acro_list_reverse_long_extra_bool ,
    before      = \l__acro_list_before_tl ,
    after       = \l__acro_list_after_tl
  }
  {
    \AssignTemplateKeys
    \bool_set_true:N \l__acro_in_list_bool
    \acro_activate_hyperref_support:
    \bool_if:NTF \l__acro_list_reverse_long_extra_bool
      {
        \cs_set_protected:Npn \acro_print_list_entry:nnnn ##1##2##3##4
          { \item [##1] ##3 ##2 ##4 }
      }
      {
        \cs_set_protected:Npn \acro_print_list_entry:nnnn ##1##2##3##4
          { \item [##1] ##2 ##3 ##4 }
      }
    \use:x
      {
        \exp_not:V \l__acro_list_before_tl
        \exp_not:N \begin { \exp_not:V \l__acro_list_tl }
          \exp_not:n { \acro_list_items:nn {#1} {#2} }
        \exp_not:N \end { \exp_not:V \l__acro_list_tl }
        \exp_not:V \l__acro_list_after_tl
      }
  }

% `list-of' template:
\DeclareTemplateInterface {acro-list} {list-of} {2}
  {
    foreign-sep : tokenlist = {~} ,
    style       : tokenlist = {toc} ,
    reverse     : boolean   = false ,
    before      : tokenlist = ,
    after       : tokenlist =
  }

\DeclareTemplateCode {acro-list} {list-of} {2}
  {
    foreign-sep = \l__acro_foreign_sep_tl ,
    style       = \l__acro_list_of_style ,
    reverse     = \l__acro_list_reverse_long_extra_bool ,
    before      = \l__acro_list_before_tl ,
    after       = \l__acro_list_after_tl
  }
  {
    \AssignTemplateKeys
    \bool_set_true:N \l__acro_in_list_bool
    \tl_if_eq:VnT \l__acro_page_instance_tl {none}
      { \tl_set:Nn \l__acro_page_instance_tl {plain} }
    \tl_set:Nn \l__acro_page_name_tl {}
    \tl_set:Nn \l__acro_pages_name_tl {}
    \acro_activate_hyperref_support:
    \str_case:Vn \l__acro_list_of_style
      {
        {toc}
        { % similar to the table of contents
          \bool_if:NTF \l__acro_list_reverse_long_extra_bool
            {
              \cs_if_exist:NTF \chapter
                {
                  \cs_set_protected:Npn \acro_print_list_entry:nnnn ##1##2##3##4
                    {
                      \contentsline{chapter}{##1}{}{}
                      \contentsline{section}{##3##2}{##4}{}
                    } 
                }
                {
                  \cs_set_protected:Npn \acro_print_list_entry:nnnn ##1##2##3##4
                    {
                      \contentsline{section}{##1}{}{}
                      \contentsline{subsection}{##3##2}{##4}{}
                    }
                }
            }
            {
              \cs_if_exist:NTF \chapter
                {
                  \cs_set_protected:Npn \acro_print_list_entry:nnnn ##1##2##3##4
                    {
                      \contentsline{chapter}{##1}{}{}
                      \contentsline{section}{##2##3}{##4}{}
                    } 
                }
                {
                  \cs_set_protected:Npn \acro_print_list_entry:nnnn ##1##2##3##4
                    {
                      \contentsline{section}{##1}{}{}
                      \contentsline{subsection}{##2##3}{##4}{}
                    }
                }
            }
        }
        {lof}
        { % similar to the list of figures
          \cs_set_protected:Npn \l@acro
            { \@dottedtocline {1} {1.5em} {\l__acro_short_width_dim} }
          \bool_if:NTF \l__acro_list_reverse_long_extra_bool
            {
              \cs_set_protected:Npn \acro_print_list_entry:nnnn ##1##2##3##4
                { \contentsline{acro}{\numberline{##1}{##3##2}}{##4}{} }
            }
            {
              \cs_set_protected:Npn \acro_print_list_entry:nnnn ##1##2##3##4
                { \contentsline{acro}{\numberline{##1}{##2##3}}{##4}{} }
            }
        }
      }
    \use:x
      {
        \exp_not:V \l__acro_list_before_tl
        \exp_not:n { \acro_list_items:nn {#1} {#2} }
        \exp_not:V \l__acro_list_before_tl
      }
  }
  
% `table' template:
\DeclareTemplateInterface {acro-list} {table} {2}
  {
    table       : tokenlist = tabular ,
    table-spec  : tokenlist = lp{.7\linewidth} ,
    foreign-sep : tokenlist = {~} ,
    reverse     : boolean   = false ,
    before      : tokenlist = ,
    after       : tokenlist = 
  }

\DeclareTemplateCode {acro-list} {table} {2}
  {
    table       = \l__acro_list_table_tl      ,
    table-spec  = \l__acro_list_table_spec_tl ,
    foreign-sep = \l__acro_foreign_sep_tl ,
    reverse     = \l__acro_list_reverse_long_extra_bool ,
    before      = \l__acro_list_before_tl ,
    after       = \l__acro_list_after_tl
  }
  {
    \AssignTemplateKeys
    \acro_activate_hyperref_support:
    \bool_if:NTF \l__acro_list_reverse_long_extra_bool
      {
        \cs_set_protected:Npn \acro_print_list_entry:nnnn ##1##2##3##4
          { ##1 & ##3 ##2 ##4 \tabularnewline }
      }
      {
        \cs_set_protected:Npn \acro_print_list_entry:nnnn ##1##2##3##4
          { ##1 & ##2 ##3 ##4 \tabularnewline }
      }
    \acro_build_list_entries:Nnn \l__acro_list_entries_tl {#1} {#2}
    \use:x
      {
        \exp_not:V \l__acro_list_before_tl
        \exp_not:N \begin { \exp_not:V \l__acro_list_table_tl }
          { \exp_not:V \l__acro_list_table_spec_tl }
        \exp_not:V \l__acro_list_entries_tl
        \exp_not:N \end { \exp_not:V \l__acro_list_table_tl }
        \exp_not:V \l__acro_list_after_tl
      }
  }

% `extra-table' template:
\DeclareTemplateInterface {acro-list} {extra-table} {2}
  {
    table       : tokenlist = tabular ,
    table-spec  : tokenlist = llll ,
    foreign-sep : tokenlist = {~} ,
    reverse     : boolean   = false ,
    before      : tokenlist = ,
    after       : tokenlist = 
  }

\DeclareTemplateCode {acro-list} {extra-table} {2}
  {
    table       = \l__acro_list_table_tl      ,
    table-spec  = \l__acro_list_table_spec_tl ,
    foreign-sep = \l__acro_foreign_sep_tl ,
    reverse     = \l__acro_list_reverse_long_extra_bool ,
    before      = \l__acro_list_before_tl ,
    after       = \l__acro_list_after_tl
  }
  {
    \AssignTemplateKeys
    \acro_activate_hyperref_support:
    \bool_if:NTF \l__acro_list_reverse_long_extra_bool
      {
        \cs_set_protected:Npn \acro_print_list_entry:nnnn ##1##2##3##4
          { ##1 & ##3 & ##2 & ##4 \tabularnewline }
      }
      {
        \cs_set_protected:Npn \acro_print_list_entry:nnnn ##1##2##3##4
          { ##1 & ##2 & ##3 & ##4 \tabularnewline }
      }
    \acro_build_list_entries:Nnn \l__acro_list_entries_tl {#1} {#2}
    \use:x
      {
        \exp_not:V \l__acro_list_before_tl
        \exp_not:N \begin { \exp_not:V \l__acro_list_table_tl }
          { \exp_not:V \l__acro_list_table_spec_tl }
        \exp_not:V \l__acro_list_entries_tl
        \exp_not:N \end { \exp_not:V \l__acro_list_table_tl }
        \exp_not:V \l__acro_list_after_tl
      }
  }

% --------------------------------------------------------------------------
% declare new list styles:
\cs_new_protected:Npn \acro_declare_list_style:nnn #1#2#3
  {
    \DeclareInstance {acro-list} {#1} {#2} {#3}
    \prop_put:Nnn \l__acro_list_styles_prop  {#1} {#2}
  }

% #1: name
% #2: template
% #3: settings
\NewDocumentCommand \DeclareAcroListStyle {mmm}
  { \acro_declare_list_style:nnn {#1} {#2} {#3} }

% set a list style
\cs_new_protected:Npn \acro_set_list_style:n #1
  {
    \prop_if_in:NnTF \l__acro_list_styles_prop {#1}
      { \__acro_set_list_style:n {#1} }
      {
        \msg_warning:nnnnn {acro} {unknown}
          {list~ style}
          {#1}
          {description}
        \__acro_set_list_style:n {description}
      }
  }

\cs_new_protected:Npn \__acro_set_list_style:n #1
  {
    \tl_set:Nn \l__acro_list_instance_tl {#1}
    \prop_get:NnN \l__acro_list_styles_prop {#1} \l__acro_list_type_tl
  }

% --------------------------------------------------------------------------
% automatic typesetting, the internals of \ac:
% #1: id
  
\cs_new_protected:Npn \acro_use:n #1
  {
    % get the acronym and the plural settings:
    \acro_get:n {#1}
    \acro_is_used:nTF {#1}
      {
        % this is not the first time
        \acro_write_indefinite:nn {#1} {short}
        \acro_write_compact:nn {#1} {short}
        \acro_after:n {#1}
      }
      {
        % this is the first time
        \bool_gset_true:c { g__acro_#1_first_use_bool }
        \acro_if_is_single:nTF {#1}
          { \acro_single:n {#1} }
          { \acro_first_instance:nV {#1} \l__acro_long_tl }
      }
  }

% single appearances:
\cs_new_protected:Npn \acro_single:n #1
  {
    \acro_cite:
    \acro_single_form:nV {#1} \l__acro_single_form_tl
    \acro_after:n {#1}
  }
  
% #1: ID
% #2: long|first|<other>
\cs_new_protected:Npn \acro_single_form:nn #1#2
  {
    \acro_write_indefinite:nn {#1} {#2}
    \str_case:nnF {#2}
      {
        {long} {
          \tl_if_blank:VT \l__acro_single_format_tl
            {
              \bool_if:NTF \l__acro_custom_long_format_bool
                {
                  \tl_set_eq:NN
                    \l__acro_single_format_tl
                    \l__acro_custom_long_format_tl
                }
                {
                  \tl_set_eq:NN
                    \l__acro_single_format_tl
                    \l__acro_long_format_tl
                }
            }
          \tl_if_blank:VT \l__acro_single_tl
            { \tl_set_eq:NN \l__acro_single_tl \l__acro_long_tl }
          \acro_write_long:VV \l__acro_single_format_tl \l__acro_single_tl
        }
        {first} {
          \tl_if_blank:VF \l__acro_single_format_tl
            {
              \tl_set_eq:NN
                \l__acro_first_long_format_tl
                \l__acro_single_format_tl
            }
          \tl_if_blank:VT \l__acro_single_tl
            { \tl_set_eq:NN \l__acro_single_tl \l__acro_long_tl }
          \acro_first_instance:nV {#1} \l__acro_single_tl
        }
      }
      { % other (e.g. short)
        \tl_if_blank:VF \l__acro_single_tl
          { \tl_set_eq:cN {l__acro_#2_tl} \l__acro_single_tl }
        \tl_if_blank:VF \l__acro_single_format_tl
          { \tl_set_eq:cN {l__acro_#2_format_tl} \l__acro_single_format_tl }
        \acro_write_compact:nn {#1} {#2}
      }
  }
\cs_generate_variant:Nn \acro_single_form:nn {nV}

\prg_new_conditional:Npnn \acro_if_is_single:n #1 { p,T,TF }
  {
    \bool_if:nTF
      { !\l__acro_use_single_bool || \cs_if_exist_p:c { acro@#1@twice } }
      { \prg_return_false: }
      { \prg_return_true: }
  }

\cs_new_protected:Npn \acro_use_acronym:n #1
  { \use:c {bool_set_#1:N} \l__acro_mark_as_used_bool }

% --------------------------------------------------------------------------
% some helpers we'll need more often:
\seq_new:N \g__acro_declared_acronyms_seq

\prg_new_conditional:Npnn \acro_if_defined:n #1 {p,T,F,TF}
  {
    \seq_if_in:NnTF \g__acro_declared_acronyms_seq {#1}
      { \prg_return_true: }
      { \prg_return_false: }
  }

\cs_new_protected:Npn \acro_defined:n #1
  {
    \acro_if_defined:nF {#1}
      { \acro_serious_message:nn {undefined} {#1} }
  }

% expandably gets property but doesn't transform property name -- internal
% name is needed
% #1: property
% #2: id
\cs_new:Npn \__acro_get_property:nn #1#2
  { \prop_item:cn {l__acro_#1_prop} {#2} }

% #1: id
% #2: property
% #3: set case
% #4: not set case
\prg_new_protected_conditional:Npnn \acro_get_property:nn #1#2 {T,F,TF}
  {
    \tl_set:Nn \l__acro_tmpa_tl {#2}
    \tl_replace_all:Nnn \l__acro_tmpa_tl {-} {_}
    \prop_get:cncTF
      {l__acro_ \l__acro_tmpa_tl _prop}
      {#1}
      {l__acro_ \l__acro_tmpa_tl _tl}
      { \prg_return_true: }
      { \prg_return_false: }
  }

\cs_new_protected:Npn \acro_get_property:nn #1#2
  { \acro_get_property:nnTF {#1} {#2} {} {} }
\cs_generate_variant:Nn \acro_get_property:nn {V}

% #1: id
% #2: property
% #3: set case
% #4: not set case
\prg_new_protected_conditional:Npnn \acro_if_property:nn #1#2 {T,F,TF}
  {
    \tl_set:Nn \l__acro_tmpa_tl {#2}
    \tl_replace_all:Nnn \l__acro_tmpa_tl {-} {_}
    \prop_if_in:cnTF
      {l__acro_ \l__acro_tmpa_tl _prop}
      {#1}
      { \prg_return_true: }
      { \prg_return_false: }
  }

\seq_new:N \l__acro_actions_seq

% within this command one can refer to the current id with `#1'
\cs_new_protected:Npn \acro_add_action:n #1
  { \seq_put_right:Nn \l__acro_actions_seq {#1} }

\tl_new:N \l_acro_current_id_tl
\cs_new_protected:Npn \__acro_get_actions:n #1
  {
    \seq_map_inline:Nn \l__acro_actions_seq
      {
        \cs_set:Npn \__acro_action:n ####1 {##1}
        \__acro_action:n {#1}
      }
  }

\cs_new_protected:Npn \acro_get:n #1
  {
    \bool_if:NF \l__acro_in_list_bool { \leavevmode }
    \acro_activate_hyperref_support:
    % short:
    \prop_get:NnNF \l__acro_short_prop {#1} \l__acro_tmpa_tl {}
    \__acro_make_link:NnV \l__acro_short_tl {#1} \l__acro_tmpa_tl
    % \acro_get_property:nn {#1} {short-format}
     % alt:
    \prop_get:NnNTF \l__acro_alt_prop {#1} \l__acro_tmpa_tl
      { \__acro_make_link:NnV \l__acro_alt_tl {#1} \l__acro_tmpa_tl }
      { \tl_set_eq:NN \l__acro_alt_tl \l__acro_short_tl }
    % long:
    \acro_get_property:nn {#1} {long}
    % \acro_get_property:nn {#1} {long-format}
    % foreign:
    \acro_get_property:nn {#1} {foreign}
    % foreign-lang:
    \acro_get_property:nn {#1} {foreign-lang}
    % extra:
    \acro_get_property:nn {#1} {extra}
    % \acro_get_property:nn {#1} {extra-format}
    % single:
    \acro_get_property:nn {#1} {single}
    % \acro_get_property:nn {#1} {single-format}
    % first-style:
    \acro_get_property:nn {#1} {first-style}
    % formatting
    \prop_get:NnNTF \l__acro_long_format_prop {#1}
      \l__acro_custom_long_format_tl
      { \bool_set_true:N  \l__acro_custom_long_format_bool }
      { \bool_set_false:N \l__acro_custom_long_format_bool }
    \acro_get_property:nn {#1} {first-long-format}
    \prop_get:NnNTF \l__acro_format_prop {#1} \l__acro_custom_format_tl
      { \bool_set_true:N \l__acro_custom_format_bool }
      { \bool_set_false:N \l__acro_custom_format_bool }
    \acro_get_property:nn {#1} {single-format}
    \acro_for_endings_do:n
      {
        \bool_if:cT {l__acro_##1_bool}
          { \__acro_set_ending_for:nnn {##1} {#1} {long} }
      }
    \acro_get_property:nnF {#1} {long-post}
      { \tl_clear:N \l__acro_long_post_tl }
    \acro_get_property:nnT {#1} {long-pre}
      { \tl_put_left:NV \l__acro_long_tl \l__acro_long_pre_tl }
    \__acro_get_actions:n {#1}
  }

% --------------------------------------------------------------------------
% plural endings and similar concepts:
\seq_new:N \l__acro_endings_seq

\cs_new_protected:Npn \acro_for_endings_do:n #1
  { \seq_map_inline:Nn \l__acro_endings_seq {#1} }

% #1: ending
% #2: ID
\cs_new_protected:Npn \__acro_set_ending:nn #1#2
  {
    \bool_if:cT {l__acro_#1_bool}
      {
        \__acro_set_ending_for:nnn {#1} {#2} {short}
        \__acro_set_ending_for:nnn {#1} {#2} {alt}
        \__acro_set_ending_for:nnn {#1} {#2} {long}
      }
  }

\tl_new:N \l__acro_endings_tl

\bool_new:N \l__acro_use_ending_form_bool

% this does nothing if a non-existent ending (#1) or non-existent form (#3) is
% input
% #1: ending
% #2: id
% #3: short|alt|long
\cs_new_protected:Npn \__acro_set_ending_for:nnn #1#2#3
  {
    \acro_if_ending_form_exist:nnT {#1} {#3}
      {
        \bool_if:nTF { \prop_item:cn {l__acro_#3_#1_form_prop} {#2} }
          { \prop_get:cnc {l__acro_#3_#1_prop} {#2} {l__acro_#3_tl}  }
          { \prop_get:cnc {l__acro_#3_#1_prop} {#2} {l__acro_#3_#1_tl} }
      }
  }

\cs_new_protected:Npn \__acro_set_endings:n #1
  {
    \acro_for_endings_do:n
      { \__acro_set_ending:nn {##1} {#1} }
  }

% #1: id
% #2: short|alt|…
\cs_new_protected:Npn \acro_get_ending_form:nn #1#2
  {
    \acro_for_endings_do:n
      {
        \acro_if_ending_form_exist:nnT {##1} {#2}
          {
            \bool_if:nT
              {
                \prop_item:cn {l__acro_#2_##1_form_prop} {#1}
                &&
                \use:c {l__acro_##1_bool}
              }
              { \prop_get:cncF {l__acro_#2_##1_prop} {#1} {l__acro_#2_tl} {} }
          }
      }
  }

% #1: id
% #2: short|alt|…
\cs_new_protected:Npn \acro_endings:nn #1#2
  {
    \group_begin:
      \bool_if:NTF \l__acro_include_endings_format_bool
        {
          \bool_if:NTF \l__acro_custom_format_bool
            { \l__acro_custom_format_tl }
            { \tl_use:c {l__acro_#2_format_tl} }
        }
        { \use:n }
        {
          \acro_for_endings_do:n
            {
              \__acro_set_ending_for:nnn {##1} {#1} {#2}
              \bool_if:cT {l__acro_##1_bool}
                { \tl_use:c {l__acro_#2_##1_tl} }
            }
        }
    \group_end:
  }

\prg_new_conditional:Npnn \acro_if_ending_exist:n #1 {p,T,F,TF}
  {
    \seq_if_in:NnTF \l__acro_endings_seq {#1}
      { \prg_return_true: }
      { \prg_return_false: }
  }

% #1: ending
% #2: short|alt|…
\prg_new_conditional:Npnn \acro_if_ending_form_exist:nn #1#2 {p,T,F,TF}
  {
    \cs_if_exist:cTF {l__acro_#2_#1_prop}
      { \prg_return_true: }
      { \prg_return_false: }
  }
  
% #1: name
% #2: default short
% #3: default long
\cs_new_protected:Npn \acro_provide_ending:nnn #1#2#3
  {
    \acro_if_ending_exist:nTF {#1}
      {
        \acro_harmless_message:nn {ending-exists} {#1}
        % short variables
        \acro_set_ending_variables:nnn {short} {#1} {#2}
        % alt variables
        \acro_set_ending_variables:nnn {alt} {#1} {#2}
        % long variables
        \acro_set_ending_variables:nnn {long} {#1} {#3}
      }
      {
        % registering:
        \bool_if:NT \g__acro_first_acronym_declared_bool
          { \acro_serious_message:n {ending-before-acronyms} }
        \seq_put_right:Nn \l__acro_endings_seq {#1}
        \bool_new:c {l__acro_#1_bool}
        % short variables
        \acro_define_and_set_ending_variables:nnn {short} {#1} {#2}
        % alt variables
        \acro_define_and_set_ending_variables:nnn {alt} {#1} {#2}
        % long variables
        \acro_define_and_set_ending_variables:nnn {long} {#1} {#3}
        % define setup command:
        \tl_set:Nn \l__acro_tmpa_tl {#1}
        \tl_replace_all:Nnn \l__acro_tmpa_tl {-} {_}
        \cs_new_protected:cpn {acro_ \l__acro_tmpa_tl :}
          { \bool_set_true:c {l__acro_#1_bool} }
        % acronym properties:
        % short-<ending>:
        \acro_declare_property:nnn {short_#1} {short-#1}
          {
            \prop_put:cnn {l__acro_short_#1_form_prop} {##1} { \c_false_bool }
            \prop_put:cnx {l__acro_pdfstring_short_#1_prop}
              {##1} { \prop_item:Nn \l__acro_short_prop {##1} \exp_not:n {##2} }
          }
        % short-<ending>-form:
        \acro_declare_property_generic:nnn {short_#1_form} {short-#1-form}
          {
            \__acro_property_check:nn {##1} {short-#1-form}
            \prop_put:cnn {l__acro_short_#1_form_prop} {##1} { \c_true_bool }
            \prop_put:cnn {l__acro_short_#1_prop} {##1} {##2}
            \prop_put:cnn {l__acro_pdfstring_short_#1_prop} {##1} {##2}
          }
        % alt-<ending>:
        \acro_declare_property:nnn {alt_#1} {alt-#1}
          {
            \prop_put:cnn {l__acro_alt_#1_form_prop} {##1} { \c_false_bool }
            \prop_put:cnx {l__acro_pdfstring_alt_#1_prop}
              {##1} { \prop_item:Nn \l__acro_alt_prop {##1} \exp_not:n {##2} }
          }
        % alt-<ending>-form:
        \acro_declare_property_generic:nnn {alt_#1_form} {alt-#1-form}
          {
            \__acro_property_check:nn {##1} {alt-#1-form}
            \prop_put:cnn {l__acro_alt_#1_form_prop} {##1} { \c_true_bool }
            \prop_put:cnn {l__acro_alt_#1_prop} {##1} {##2}
            \prop_put:cnn {l__acro_pdfstring_alt_#1_prop} {##1} {##2}
          }
        % long-<ending>:
        \acro_declare_property:nnn {long_#1} {long-#1}
          { \prop_put:cnn {l__acro_long_#1_form_prop} {##1} { \c_false_bool } }
        % long-<ending>-form:
        \acro_declare_property_generic:nnn {long_#1_form} {long-#1-form}
          {
            \__acro_property_check:nn {##1} {long-#1-form}
            \prop_put:cnn {l__acro_long_#1_form_prop} {##1} { \c_true_bool }
            \prop_put:cnn {l__acro_long_#1_prop} {##1} {##2}
          }
        % options:
        %   short-<ending>-ending
        %   alt-<ending>-ending
        %   long-<ending>-ending
        %   <ending>-ending
        \keys_define:nn {acro}
          {
            short-#1-ending .code:n =
              \bool_if:NT \g__acro_first_acronym_declared_bool
                { \acro_serious_message:n {ending-before-acronyms} }
              \tl_set:cn {l__acro_default_short_#1_tl} {##1} ,
            alt-#1-ending   .code:n =
              \bool_if:NT \g__acro_first_acronym_declared_bool
                { \acro_serious_message:n {ending-before-acronyms} }
              \tl_set:cn {l__acro_default_alt_#1_tl} {##1} ,
            long-#1-ending  .code:n =
              \bool_if:NT \g__acro_first_acronym_declared_bool
                { \acro_serious_message:n {ending-before-acronyms} }
              \tl_set:cn {l__acro_default_long_#1_tl} {##1},
            #1-ending       .code:n   =
              \bool_if:NT \g__acro_first_acronym_declared_bool
                { \acro_serious_message:n {ending-before-acronyms} }
              \__acro_read_ending_settings:nww {#1} ##1// \acro_stop:
          }
        % pdfstrings:
        % TODO: add long forms:
        \prop_new:c {l__acro_pdfstring_short_#1_prop}
        \cs_new:cpn {acro_pdf_string_short_#1:n} ##1
          {
            \acro_if_star_gobble:nTF {##1}
              { \prop_item:cn {l__acro_pdfstring_short_#1_prop} }
              { \prop_item:cn {l__acro_pdfstring_short_#1_prop} {##1} }
          }
        \cs_new:cpn {acpdfstring#1} { \use:c {acro_pdf_string_short_#1:n} }
        \prop_new:c {l__acro_pdfstring_alt_#1_prop}
        \cs_new:cpn {acro_pdf_string_alt_#1:n} ##1
          {
            \acro_if_star_gobble:nTF {##1}
              { \prop_item:cn {l__acro_pdfstring_alt_#1_prop} }
              { \prop_item:cn {l__acro_pdfstring_alt_#1_prop} {##1} }
          }
        \cs_new:cpn {acpdfstringalt#1} { \use:c {acro_pdf_string_alt_#1:n} }
      }
  }

% #1: short|alt|long
% #2: ending name
% #3: default ending
\cs_new_protected:Npn \acro_define_and_set_ending_variables:nnn #1#2#3
  {
    \acro_define_ending_variables:nn {#1} {#2}
    \acro_set_ending_variables:nnn {#1} {#2} {#3}
  }

% #1: short|alt|long
% #2: ending name
\cs_new_protected:Npn \acro_define_ending_variables:nn #1#2
  {
    \prop_new:c {l__acro_#1_#2_prop}
    \prop_new:c {l__acro_#1_#2_form_prop}
    \tl_new:c   {l__acro_#1_#2_tl}
    \tl_new:c   {l__acro_default_#1_#2_tl}
  }

% #1: short|alt|long
% #2: ending name
% #3: default ending
\cs_new_protected:Npn \acro_set_ending_variables:nnn #1#2#3
  { \tl_set:cn  {l__acro_default_#1_#2_tl} {#3} }

% #1: ending name
% #2: short (and long if #4 is blank)
% #3: long
\cs_new_protected:Npn \__acro_read_ending_settings:nww #1#2/#3/#4 \acro_stop:
  {
    \acro_set_ending_variables:nnn {short} {#1} {#2}
    \acro_set_ending_variables:nnn {alt} {#1} {#2}
    \tl_if_blank:nTF {#4}
      { \acro_set_ending_variables:nnn {long} {#1} {#3} }
      { \acro_set_ending_variables:nnn {long} {#1} {#2} }
  }

\NewDocumentCommand \ProvideAcroEnding {mmm}
  { \acro_provide_ending:nnn {#1} {#2} {#3} }

% --------------------------------------------------------------------------
% enable us to know if the acronym is used only once and provide a different
% style for that:
\prg_new_protected_conditional:Npnn \acro_is_used:n #1 { T,F,TF }
  {
    \acro_record_barrier:n {#1}
    \bool_if:nTF
      {
        \bool_if_p:c { g__acro_#1_used_bool } &&
        (
          (
            \bool_if_p:c { g__acro_#1_first_use_bool } &&
            \g__acro_mark_first_as_used_bool
          )
          ||
          ! \g__acro_mark_first_as_used_bool
        )
      }
      {
        \bool_if:NTF \l__acro_mark_as_used_bool
          {
            \__acro_aux_file:Nxxxx \acro@used@twice
              {#1}
              { \thepage }
              { \arabic {page} }
              { \arabic {abspage} }
          }
          { \__acro_aux_file:Nxxxx \acro@used@twice {#1} {} {} {} }
        \prg_return_true:
      }
      {
        \bool_if:NTF \l__acro_mark_as_used_bool
          {
            \__acro_aux_file:Nxxxx \acro@used@once
              {#1}
              { \thepage }
              { \arabic {page} }
              { \arabic {abspage} }
            \bool_if:nT
              {
                !\bool_if_p:c { g__acro_#1_label_bool } &&
                \l__acro_place_label_bool
              }
              {
                \bool_gset_true:c { g__acro_#1_label_bool }
                \label{\l__acro_label_prefix_tl #1}
              }
            \bool_gset_true:c { g__acro_#1_used_bool }
          }
          { \__acro_aux_file:Nxxxx \acro@used@once {#1} {} {} {} }
        \prg_return_false:
      }
  }

\cs_new:Npn \acro_is_used:n #1
  { \acro_is_used:nTF {#1} { } { } }

\cs_new_protected:Npn \__acro_aux_file:Nnnnn #1#2#3#4#5
  { \iow_shipout:Nn \@auxout { #1 {#2} {#3} {#4} {#5} } }
\cs_generate_variant:Nn \__acro_aux_file:Nnnnn { Nxxxx }
  
\cs_new_protected:Npn \__acro_aux_file_now:n #1
  { \iow_now:Nn \@auxout {#1} }
\cs_generate_variant:Nn \__acro_aux_file_now:n { x }

% --------------------------------------------------------------------------
% the commands for the auxiliary file:
\cs_new_protected:Npn \acro@used@once #1#2#3#4
  {
    \cs_gset_nopar:cpn {acro@#1@once} {#1}
    \bool_gset_true:c {g__acro_#1_in_list_bool}
    \tl_if_empty:nF {#2#3#4}
      {
        % \bool_gset_true:c { g__acro_#1_used_bool }
        \seq_gput_right:cn {g__acro_#1_pages_seq} { {#2}{#3}{#4} }
      }
  }
\cs_new_protected:Npn \acro@used@twice #1#2#3#4
  {
    \cs_gset_nopar:cpn {acro@#1@twice} {#1}
    \tl_if_empty:nF {#2#3#4}
      { \seq_gput_right:cn {g__acro_#1_pages_seq} { {#2}{#3}{#4} } }
  }

\cs_new_protected:Npn \acro@pages #1#2
  { \tl_gset:cn {g__acro_#1_recorded_pages_tl} {#2} }

\bool_new:N \g__acro_rerun_bool

\cs_new_protected:Npn \acro@rerun@check
  {
    \bool_if:NT \g__acro_rerun_bool
      {
        \@latex@warning@no@line
          {Acronyms~ may~ have~ changed.~ Please~ rerun~ LaTeX}
      }
  }

\AtEndDocument
  {
    \bool_gset_false:N \g__acro_rerun_bool
    \cs_gset_protected:Npn \acro@used@once #1#2#3#4
      {
        \tl_set:Nn \l__acro_tmpa_tl {#1}
        \tl_if_eq:cNF {acro@#1@once} \l__acro_tmpa_tl
          { \bool_gset_true:N \g__acro_rerun_bool }
      }
    \cs_gset_protected:Npn \acro@used@twice #1#2#3#4
      {
        \tl_set:Nn \l__acro_tmpa_tl {#1}
        \tl_if_eq:cNF {acro@#1@twice} \l__acro_tmpa_tl
          { \bool_gset_true:N \g__acro_rerun_bool }
      }
    \acro_for_all_acronyms_do:n
      {
        \seq_if_empty:cF {g__acro_#1_pages_seq}
          {
            \__acro_aux_file_now:x
              {
                \token_to_str:N \acro@pages {#1}
                  { \seq_use:cn {g__acro_#1_pages_seq} {|} } ^^J
                \token_to_str:N \acro@barriers {#1}
                  { \seq_use:cn {g__acro_#1_barriers_seq} {,} }
              }
          }
        \acro_check_barriers:n {#1}
      }
    \__acro_aux_file_now:n { \acro@rerun@check }
  }

% if `acro' is deactivated prevent unnecessary errors from aux file:
\if@filesw
\AtBeginDocument
  {
    \__acro_aux_file_now:n
      {
        \providecommand \acro@used@once [4] {} ^^J
        \providecommand \acro@used@twice [4] {} ^^J
        \providecommand \acro@pages [2] {} ^^J
        \providecommand \acro@rerun@check {} ^^J
        \providecommand \acro@print@list {} ^^J
        \providecommand \acro@barriers [2] {}
      }
  }
\fi

% --------------------------------------------------------------------------
% typeset the short form:
% #1: ID
% #2: short form
\cs_new_protected:Npn \acro_write_short:nn #1#2
  {
    \mode_if_horizontal:F { \leavevmode }
    \group_begin:
      \bool_if:NTF \l__acro_custom_format_bool
        { \l__acro_custom_format_tl }
        { \l__acro_short_format_tl }
      {#2}
    \group_end:
  }
\cs_generate_variant:Nn \acro_write_short:nn { nV , nv }

% typeset the alternative form:
% #1: ID
% #2: alt form
\cs_new_protected:Npn \acro_write_alt:nn #1#2
  {
    \mode_if_horizontal:F { \leavevmode }
    \group_begin:
      \bool_if:NTF \l__acro_custom_format_bool
        { \l__acro_custom_format_tl }
        { \l__acro_alt_format_tl }
      {#2}
    \group_end:
  }
\cs_generate_variant:Nn \acro_write_alt:nn { nV , nv }

% typeset a long form:
%   TODO: rethink the formatting mechanism
%   right now a custom format gets applied additionally to the global one
%   although before it
% #1: format
% #2: long form
\cs_new_protected:Npn \acro_write_long:nn #1#2
  {
    \mode_if_horizontal:F { \leavevmode }
    \group_begin:
      \bool_if:NTF \l__acro_custom_long_format_bool
        { \l__acro_custom_long_format_tl }
        { \use:n }
      {
        \use:x
          {
            \exp_not:n {#1}
            {
              \bool_if:NTF \l__acro_first_upper_bool
                { \exp_not:N \__acro_first_upper_case:n { \exp_not:n {#2} } }
                { \exp_not:n {#2} }
            }
          }
      }
    \group_end:
  }
\cs_generate_variant:Nn \acro_write_long:nn { VV,Vo,Vf,V,v,vv }

\prg_new_conditional:Npnn \acro_if_foreign:n #1 {T,F,TF}
  {
    \bool_if:nTF
      {
        \l__acro_foreign_bool
        &&
        \prop_if_in_p:Nn \l__acro_foreign_prop {#1}
      }
      { \prg_return_true: }
      { \prg_return_false: }
  }

\cs_new_protected:Npn \acro_foreign_language:nn #1#2 {}
\AtBeginDocument{
  \cs_if_exist:NTF \foreignlanguage
    {
      \cs_set_protected:Npn \acro_foreign_language:nn #1#2
        {
          \tl_if_blank:nTF {#1}
            {#2}
            { \foreignlanguage {#1} {#2} }
        }
    }
    {
      \cs_set_protected:Npn \acro_foreign_language:nn #1#2
        { \use_ii:nn {#1} {#2} }
    }
}
\cs_generate_variant:Nn \acro_foreign_language:nn {VV}

\cs_new_protected:Npn \acro_write_foreign:n #1
  {
    \acro_if_foreign:nT {#1}
      {
        \prop_get:NnNT \l__acro_foreign_prop {#1} \l__acro_foreign_tl
          {
            \group_begin:
              \tl_use:N \l__acro_foreign_format_tl
              {
                \acro_foreign_language:VV
                  \l__acro_foreign_lang_tl
                  \l__acro_foreign_tl
              }
            \group_end:
          }
      }
  }

\cs_new:Npn \acroenparen #1 { ( #1 ) }

\cs_new_protected:Npn \acro_get_foreign:n #1
  {
    \prop_get:NnNT \l__acro_foreign_prop {#1} \l__acro_foreign_tl
      {
        \tl_use:N \l__acro_foreign_sep_tl
        \group_begin:
          \tl_use:N \l__acro_foreign_list_format_tl
          {
            \acro_foreign_language:VV
              \l__acro_foreign_lang_tl
              \l__acro_foreign_tl
          }
        \group_end:
      }
  }

% --------------------------------------------------------------------------
% #1: id
% #2: short|alt
\cs_set_protected:Npn \acro_write_compact:nn #1#2
  {
    \acro_get_ending_form:nn {#1} {#2}
    \acro_acc_supp:nn
      {#1}
      {
        \acro_write_tooltip:nnV
          {#1}
          {
            \use:c {acro_write_#2:nv} {#1} {l__acro_#2_tl}
            \acro_endings:nn {#1} {#2}
          }
          \l__acro_long_tl
      }
  }

% TODO: get rid of argument #3?
% #1: ID
% #2: long|first-long|list-long|extra
% #3: long form
\cs_new_protected:Npn \acro_write_expanded:nnn #1#2#3
  {
    \tl_set:Nn \l__acro_tmpa_tl {#2}
    \tl_replace_all:Nnn \l__acro_tmpa_tl {-} {_}
    \acro_write_long:vn {l__acro_ \l__acro_tmpa_tl _format_tl} {#3}
    \acro_endings:nn {#1} {long}
    \tl_if_in:nnT {#2} {long}
      { \l__acro_long_post_tl }
  }
\cs_generate_variant:Nn \acro_write_expanded:nnn { nnV }

% #1: ID
% #2: long|first-long|list-long|extra
\cs_new_protected:Npn \acro_write_expanded:nn #1#2
  {
    \tl_set:Nn \l__acro_tmpa_tl {#2}
    \tl_replace_all:Nnn \l__acro_tmpa_tl {-} {_}
    \acro_write_long:vv
      {l__acro_ \l__acro_tmpa_tl _format_tl}
      {l__acro_ \l__acro_tmpa_tl _tl}
    \acro_endings:nn {#1} {long}
    \tl_if_in:nnT {#2} {long}
      { \l__acro_long_post_tl }
  }

% #1: id
\cs_new:Npn \acro_after:n #1
  {
    \acro_cite_if:nn { \l__acro_citation_all_bool } {#1}
    \acro_index_if:nn { \l__acro_addto_index_bool } {#1}
  }

\cs_new_protected:Npn \acro_check_single:n #1
  {
    \acro_if_is_single:nT {#1}
      { \cs_set_eq:NN \acro_hyper_link:nn \use_ii:nn }
  }

% --------------------------------------------------------------------------
% #1: id
\cs_new_protected:Npn \acro_before:n #1
  {
    \acro_get:n {#1}
    \acro_is_used:n {#1}
    \acro_check_single:n {#1}
  }

% the standard internals:
% #1: id
\cs_new_protected:Npn \acro_short:n #1
  {
    \acro_before:n {#1}
    \acro_write_indefinite:nn {#1} {short}
    \acro_write_compact:nn {#1} {short}
    \acro_after:n {#1}
  }

% get alternative entry:
% #1: id
\cs_new_protected:Npn \acro_alt:n #1
  {
    \acro_before:n {#1}
    \acro_alt_error:n {#1}
    \acro_write_indefinite:nn {#1} {alt}
    \acro_write_compact:nn {#1} {alt}
    \acro_after:n {#1}
  }

% get long entry:
% #1: id
\cs_new_protected:Npn \acro_long:n #1
  {
    \acro_before:n {#1}
    \acro_write_indefinite:nn {#1} {long}
    \acro_write_expanded:nn {#1} {long}
    \acro_after:n {#1}
  }

% get foreign entry:
% #1: id
\cs_new_protected:Npn \acro_foreign:n #1
  {
    \acro_get:n {#1}
    \tl_if_blank:VF \l__acro_foreign_tl
      {
        \acro_is_used:n {#1}
        \acro_check_single:n {#1}
        \acro_write_long:VV \l__acro_foreign_format_tl \l__acro_foreign_tl
        \acro_after:n {#1}
      }
  }

% get extra entry:
% #1: id
\cs_new_protected:Npn \acro_extra:n #1
  {
    \acro_get:n {#1}
    \tl_if_blank:VF \l__acro_extra_tl
      {
        \acro_is_used:n {#1}
        \acro_check_single:n {#1}
        \acro_write_long:VV \l__acro_extra_format_tl \l__acro_extra_tl
        \acro_after:n {#1}
      }
  }

% output like the first time:
% #1: id
\cs_new_protected:Npn \acro_first:n #1
  {
    \bool_gset_true:c {g__acro_#1_first_use_bool}
    \acro_before:n {#1}
    \acro_first_instance:nV {#1} \l__acro_long_tl
  }

% output like the first time with own long version:
% #1: id
% #2: instead of long entry
\cs_new_protected:Npn \acro_first_like:nn #1#2
  {
    \bool_gset_true:c {g__acro_#1_first_use_bool}
    \acro_before:n {#1}
    \acro_first_instance:nn {#1} {#2}
  }

% ----------------------------------------------------------------------------
% citations:
\cs_new:Npn \__acro_citation_cmd:w { \cite } %{}
\cs_new:Npn \__acro_group_citation_cmd:w { \cite } %{}

% #1 pre
% #2 post
% #3 key
\cs_new:Npn \__acro_cite:nnn #1#2#3
  {
    \quark_if_no_value:nTF {#1}
      { \__acro_citation_cmd:w {#3} }
      {
        \quark_if_no_value:nTF {#2}
          { \__acro_citation_cmd:w [ #1 ] {#3} }
          { \__acro_citation_cmd:w [ #1 ] [ #2 ] {#3} }
      }
  }
\cs_generate_variant:Nn \__acro_cite:nnn { VVV }

\cs_new_protected:Npn \acro_cite:n #1
  {
    \prop_get:NnNT \l__acro_citation_prop {#1} \l__acro_tmpa_tl
      {
        \prop_get:NnN \l__acro_citation_pre_prop {#1} \l__acro_tmpb_tl
        \prop_get:NnN \l__acro_citation_post_prop {#1} \l__acro_tmpc_tl
        \acro_no_break:
        \tl_use:N \l__acro_citation_connect_tl
        \__acro_cite:VVV
          \l__acro_tmpb_tl
          \l__acro_tmpc_tl
          \l__acro_tmpa_tl
      }
  }

\cs_new_protected:Npn \acro_group_cite:n #1
  {
    \group_begin:
      \cs_set_eq:NN \__acro_citation_cmd:w \__acro_group_citation_cmd:w
      \tl_set_eq:NN
        \l__acro_citation_connect_tl
        \l__acro_between_group_connect_citation_tl
      \acro_cite_if:nn { \l__acro_citation_first_bool } {#1}
    \group_end:
  }

\cs_new_protected:Npn \acro_cite_if:nn #1#2
  { \bool_if:nT {#1} { \acro_cite:n {#2} } }

% ----------------------------------------------------------------------------
% indexing:
\cs_new_protected:Npn \acro_index_if:nn #1#2
  {
    \bool_if:nT { (#1) && \l__acro_mark_as_used_bool }
      {
        \prop_get:NnN \l__acro_index_cmd_prop  {#2} \l__acro_tmpa_tl
        \prop_get:NnN \l__acro_index_sort_prop {#2} \l__acro_tmpb_tl
        \prop_get:NnN \l__acro_index_prop      {#2} \l__acro_tmpc_tl
        \__acro_index:VnVV
          \l__acro_tmpa_tl
          {#2}
          \l__acro_tmpb_tl
          \l__acro_tmpc_tl
      }
  }

\cs_new:Npn \__acro_index_cmd:n { \index }

% #1: cmd
% #2: key
% #3: sort
% #4: replace
\cs_new_protected:Npn \__acro_index:nnnn #1#2#3#4
  {
    \prop_get:NnNF \l__acro_short_prop  {#2} \l__acro_index_short_tl {}
    \prop_get:NnNF \l__acro_format_prop {#2} \l__acro_index_format_tl {}
    \quark_if_no_value:VTF \l__acro_index_format_tl
      { \tl_set:Nn \l__acro_tmpa_tl { \l__acro_short_format_tl \l__acro_index_short_tl } }
      { \tl_set:Nn \l__acro_tmpa_tl { \l__acro_index_format_tl \l__acro_index_short_tl } }
    \quark_if_no_value:nF {#1}
      { \cs_set:Npn \__acro_index_cmd:n {#1} }
    \quark_if_no_value:nTF {#4}
      {
        \quark_if_no_value:nTF {#3}
          { \__acro_index_cmd:n { #2 @ { \l__acro_tmpa_tl } } }
          { \__acro_index_cmd:n { #3 @ { \l__acro_tmpa_tl } } }
      }
      { \__acro_index_cmd:n {#4} }
  }
\cs_generate_variant:Nn \__acro_index:nnnn { VnVV }

% ----------------------------------------------------------------------------
% accessability support
\cs_new_eq:NN \acro_acc_supp:nn \use_ii:nn

\cs_new_protected:Npn \acro_get_acc_supp:nn #1#2
  {
    \prop_get:NnNF \l__acro_acc_supp_prop {#1} \l__acro_acc_supp_tl
      { \prop_get:NnNF \l__acro_short_prop {#1} \l__acro_acc_supp_tl {} }
    \acro_for_endings_do:n
      {
        \bool_if:cT {l__acro_##1_bool}
          {
            \tl_put_right:Nv
              \l__acro_acc_supp_tl
              {l__acro_short_##1_tl}
          }
      }
    \acro_do_acc_supp:VVn
      \l__acro_acc_supp_tl
      \l__acro_acc_supp_options_tl
      {#2}
  }

\cs_new:Npn \acro_do_acc_supp:nnn #1#2#3
  {
    \BeginAccSupp { ActualText = #1 , #2 }
      #3
    \EndAccSupp { }
  }
\cs_generate_variant:Nn \acro_do_acc_supp:nnn { VV }

\AtEndPreamble
  {
    \bool_if:NT \l__acro_acc_supp_bool
      {
        \RequirePackage {accsupp}
        \cs_set_eq:NN \acro_acc_supp:nn \acro_get_acc_supp:nn
      }
    \bool_if:NT \l__acro_tooltip_bool
      {
        \RequirePackage {pdfcomment}
        \cs_if_eq:NNT \__acro_tooltip_cmd:nn \use_i:nn
          { \cs_set:Npn \__acro_tooltip_cmd:nn { \pdftooltip } }
      }
  }

% --------------------------------------------------------------------------
% tooltips for acronyms

% #1: id
% #2: printed text
% #3: tool tip text
\cs_new_protected:Npn \acro_write_tooltip:nnn #1#2#3
  {
    \prop_get:NnNTF \l__acro_tooltip_prop {#1} \l__acro_tmpa_tl
      { \__acro_check_tooltip:nV {#2} \l__acro_tmpa_tl }
      { \__acro_check_tooltip:nn {#2} {#3} }
  }
\cs_generate_variant:Nn \acro_write_tooltip:nnn { nnV }

% #1: printed text
% #2: tool tip text
\cs_new_protected:Npn \__acro_check_tooltip:nn #1#2
  {
    \bool_if:NTF \l__acro_inside_tooltip_bool
      {#1}
      {
        \bool_set_true:N \l__acro_inside_tooltip_bool
        \__acro_tooltip_cmd:nn {#1} {#2}
      }
  }
\cs_generate_variant:Nn \__acro_check_tooltip:nn { nV }

% use whatever command you like for creating tooltips here:
% #1: printed text
% #2: tool tip text
\cs_new_eq:NN \__acro_tooltip_cmd:nn \use_i:nn
  
% --------------------------------------------------------------------------
% indefinite articles:

% #1: ID
% #2: short|long|alt
\cs_new_protected:Npn \acro_write_indefinite:nn #1#2
  {
    \bool_if:NT \l__acro_indefinite_bool
      { \prop_item:cn { l__acro_#2_indefinite_prop } {#1} ~ }
    \bool_if:NT \l__acro_upper_indefinite_bool
      { %  \bool_set_true:N \l__acro_first_upper_bool
         \__acro_first_upper_case:x
           { \prop_item:cn { l__acro_#2_indefinite_prop } {#1} } ~
      }
  }

% --------------------------------------------------------------------------
% experimental sorting feature:

% the following code is an adaption of expl3 code used for \str_if_eq:NN(TF)
\sys_if_engine_luatex:TF
  {
    \tl_set:Nn \l__acro_tmpa_tl
      {
        acro ~ = ~ acro ~ or ~ { ~ } ~
        function ~ acro.strcmp ~ (A, B) ~
          if ~ A ~ == ~ B ~ then ~
            tex.write ("0") ~
          elseif ~ A ~ < ~ B ~ then ~
            tex.write ("-1") ~
          else ~
            tex.write ("1") ~
          end ~
        end
      }
    \luatex_directlua:D { \l__acro_tmpa_tl }
    \cs_new_protected:Npn \acro_strcmp:nn #1#2
      {
        \luatex_directlua:D
          {
            acro.strcmp
              (
                " \__acro_escape_x:n {#1} " ,
                " \__acro_escape_x:n {#2} "
              )
          }
      }
    \cs_new:Npn \__acro_escape_x:n #1
      {
        \luatex_luaescapestring:D
          { \etex_detokenize:D \exp_after:wN { \luatex_expanded:D {#1} } }
      }
  }
  { \cs_new_eq:NN \acro_strcmp:nn \pdftex_strcmp:D }

\AtBeginDocument
  {
    \bool_if:NT \l__acro_sort_bool
      {
        \cs_new_protected:Npn \acro_sort_prop:NN #1#2
          {
            \seq_clear:N  \l__acro_tmpa_seq
            \prop_clear:N \l__acro_tmpa_prop
            \prop_clear:N \l__acro_tmpb_prop
            \prop_map_inline:Nn #2
              {
                \seq_put_right:Nn \l__acro_tmpa_seq {##2}
                \prop_put:Nnn \l__acro_tmpa_prop {##1} {##2}
              }
            \seq_sort:Nn \l__acro_tmpa_seq
              {
                \int_compare:nTF
                  {
                    \acro_strcmp:nn
                      { \str_fold_case:n {##1} }
                      { \str_fold_case:n {##2} }
                        = \c_minus_one
                  }
                  { \sort_return_same: }
                  { \sort_return_swapped: }
              }
            \seq_map_inline:Nn \l__acro_tmpa_seq
              {
                \prop_map_inline:Nn \l__acro_tmpa_prop
                  {
                    \str_if_eq:nnT {##1} {####2}
                      {
                        \prop_get:NnN #1 {####1} \l__acro_tmpa_tl
                        \prop_put:NnV \l__acro_tmpb_prop {####1}
                          \l__acro_tmpa_tl
                      }
                  }
              }
            \prop_set_eq:NN #1 \l__acro_tmpb_prop
          }
      }
  }

% --------------------------------------------------------------------------
% regarding list printing:
% this command ensures that a rerun warning is given when \printacronyms
% is set the first time. This mechanism doesn't make very much sense,
% should be replaced by a different and more efficient one
%
\cs_new_protected:Npn \acro@print@list
  { \cs_if_exist:NF \acro@printed@list { \cs_new:Npn \acro@printed@list { printed } } }

% --------------------------------------------------------------------------
% trailing tokens and what to do when present
\prop_new:N \l__acro_trailing_tokens_prop
\prop_new:N \l__acro_trailing_actions_prop
\bool_new:N \l__acro_trailing_tokens_bool
\tl_new:N   \l__acro_trailing_tokens_tl

\cs_new_protected:Npn \acro_new_trailing_token:n #1
  { \bool_new:c {l__acro_trailing_#1_bool} }
\cs_new_protected:Npn \acro_activate_trailing_action:n #1
  { \bool_set_true:c {l__acro_trailing_#1_bool} }
\cs_new_protected:Npn \acro_deactivate_trailing_action:n #1
  { \bool_set_false:c {l__acro_trailing_#1_bool} }

% register a new token but don't activate its action:
% #1: token
% #2: name
\cs_new_protected:Npn \acro_register_trailing_token:Nn #1#2
  {
    \prop_put:Nnn \l__acro_trailing_tokens_prop {#2} {#1}
    \prop_put:Nnn \l__acro_trailing_actions_prop {#1}
      { \acro_activate_trailing_action:n {#2} }
    \acro_new_trailing_token:n {#2}
  }
  
\NewDocumentCommand \AcroRegisterTrailing {mm}
  { \acro_register_trailing_token:Nn #1 {#2} }

\cs_new_protected:Npn \acro_for_all_trailing_tokens_do:n #1
  { \prop_map_inline:Nn \l__acro_trailing_tokens_prop {#1} }

% activate a token:
\cs_new_protected:Npn \acro_activate_trailing_token:n #1
  {
    \prop_get:NnN \l__acro_trailing_tokens_prop {#1} \l__acro_tmpa_tl
    \tl_put_right:NV \l__acro_trailing_tokens_tl \l__acro_tmpa_tl
  }

% deactivate a token:
\cs_new_protected:Npn \acro_deactivate_trailing_token:n #1
  {
    \prop_get:NnN \l__acro_trailing_tokens_prop {#1} \l__acro_tmpa_tl
    \tl_remove_all:NV \l__acro_trailing_tokens_tl \l__acro_tmpa_tl
  }

% #1: name
\prg_new_conditional:Npnn \acro_if_trailing_token:n #1 {p,T,F,TF}
  {
    \bool_if:cTF {l__acro_trailing_#1_bool}
      { \prg_return_true: }
      { \prg_return_false: }
  }

% #1: csv list of names
\prg_new_protected_conditional:Npnn \acro_if_trailing_tokens:n #1 {T,F,TF}
  {
    \bool_set_false:N \l__acro_trailing_tokens_bool
    \clist_map_inline:nn {#1}
      {
        \bool_if:cT {l__acro_trailing_##1_bool}
          {
            \bool_set_true:N \l__acro_trailing_tokens_bool
            \clist_map_break:
          }
      }
    \bool_if:NTF \l__acro_trailing_tokens_bool
      { \prg_return_true: }
      { \prg_return_false: }
  }

\cs_new_protected:Npn \aciftrailing { \acro_if_trailing_tokens:nTF }

\cs_new_protected:Npn \__acro_check_trail:N #1
  {
    \tl_map_inline:Nn \l__acro_trailing_tokens_tl
      {
        \token_if_eq_meaning:NNT #1 ##1
          { \prop_item:Nn \l__acro_trailing_actions_prop {##1} }
      }
  }

% options for activating actions:
\keys_define:nn {acro}
  {
    activate-trailing-tokens   .code:n =
      \clist_map_inline:nn {#1} { \acro_activate_trailing_token:n {##1} } ,
    activate-trailing-tokens   .initial:n = dot ,
    deactivate-trailing-tokens .code:n =
      \clist_map_inline:nn {#1} { \acro_deactivate_trailing_token:n {##1} }
  }

% some user macros:
\cs_new_protected:Npn \acro_dot:
  { \acro_if_trailing_tokens:nF {dot} {.\@} }

\cs_new_protected:Npn \acro_space:
  { \acro_if_trailing_tokens:nF {dash,babel-hyphen} { \c_space_tl } }

\NewDocumentCommand \acdot   {} { \acro_dot: }
\NewDocumentCommand \acspace {} { \acro_space: }
  
% ---------------------------------------------------------------------------
% reset outputs, they'll behave like the first time again (!not like the _only_
% time!):
\cs_new_protected:Npn \acro_reset:n #1
  {
    \bool_gset_false:c { g__acro_#1_used_bool }
    \bool_gset_false:c { g__acro_#1_first_use_bool }
  }

\cs_new_protected:Npn \acro_mark_as_used:n #1
  {
    \bool_gset_true:c { g__acro_#1_used_bool }
    \bool_gset_true:c { g__acro_#1_first_use_bool }
    \bool_gset_true:c { g__acro_#1_in_list_bool }
    \if@filesw
      \__acro_aux_file_now:n { \acro@used@once {#1} {} {} {} }
      \__acro_aux_file_now:n { \acro@used@twice {#1} {} {} {} }
    \fi
  }

\cs_new_protected:Npn \acro_reset_all:
  { \acro_for_all_acronyms_do:n { \acro_reset:n {##1} } }

% make sure that no acronym is used at the beginning of the document, see
% issue #81 for reasons why this may be necessary:
\AfterEndPreamble { \acro_reset_all: }
  
\cs_new_protected:Npn \acro_mark_all_as_used:
  { \acro_for_all_acronyms_do:n { \acro_mark_as_used:n {##1} } }

\DeclareExpandableDocumentCommand \acifused { m }
  { \acro_if_acronym_used:nTF {#1} }

\prg_new_conditional:Npnn \acro_if_acronym_used:n #1 { p,T,F,TF }
  {
    \bool_if:nTF
      {
        \bool_if_p:c { g__acro_#1_used_bool } &&
        ( !\acro_if_is_single_p:n {#1} )
      }
      { \prg_return_true: }
      { \prg_return_false: }
  }

\NewDocumentCommand \acresetall {}
  { \acro_reset_all: }

\NewDocumentCommand \acuseall {}
  { \acro_mark_all_as_used: }

\NewDocumentCommand \acreset { > { \SplitList { , } } m }
  { \ProcessList {#1} { \acro_reset:n } \ignorespaces }

\NewDocumentCommand \acuse { > { \SplitList { , } } m }
  { \ProcessList {#1} { \acro_mark_as_used:n } \ignorespaces }

% --------------------------------------------------------------------------
% acronym barriers: allow local lists of only those acronyms used between two
% barriers

\int_new:N  \g__acro_barrier_int
\bool_new:N \g__acro_use_barriers_bool
\bool_new:N \g__acro_reset_at_barrier_bool
\bool_new:N \l__acro_use_barrier_bool

\keys_define:nn {acro}
  {
    use-barriers      .bool_gset:N = \g__acro_use_barriers_bool ,
    use-barriers      .initial:n   = false ,
    reset-at-barriers .bool_gset:N = \g__acro_reset_at_barrier_bool ,
    reset-at-barriers .initial:n   = false
  }

\cs_new_protected:Npn \acro_barrier:
  {
    \int_gincr:N \g__acro_barrier_int
    \bool_if:NT \g__acro_reset_at_barrier_bool
      { \acro_reset_all: }
  }

\NewDocumentCommand \acbarrier {}
  { \acro_barrier: }

\cs_new_protected:Npn \acro_check_barriers:n #1
  {
    \bool_if:NT \g__acro_use_barriers_bool
      {
        \tl_set:Nx \l__acro_tmpa_tl
          { \seq_use:cn {g__acro_#1_barriers_seq} {} }
        \tl_set:Nx \l__acro_tmpb_tl
          { \seq_use:cn {g__acro_#1_recorded_barriers_seq} {} }
        \tl_if_eq:NNF \l__acro_tmpa_tl \l__acro_tmpb_tl
          {
            \@latex@warning@no@line
              {Rerun~to~get~barriers~of~acronym~#1~right}
          }
      }
  }

\cs_new_protected:Npn \acro_record_barrier:n #1
  {
    \bool_if:NT \g__acro_use_barriers_bool
      {
        \seq_if_in:cxF {g__acro_#1_barriers_seq}
          { \int_use:N \g__acro_barrier_int }
          {
            \seq_gput_right:cx  {g__acro_#1_barriers_seq}
              { \int_use:N \g__acro_barrier_int }
          }
      }
  }

% #1: id
% #2: barrier number
\prg_new_protected_conditional:Npnn \acro_if_in_barrier:nn #1#2 {T,F,TF}
  {
    \seq_if_in:cnTF {g__acro_#1_recorded_barriers_seq} {#2}
      { \prg_return_true: }
      { \prg_return_false: }
  }
\cs_generate_variant:Nn \acro_if_in_barrier:nnTF {nx}

\cs_new:Npn \acro@barriers #1#2
  { \seq_gset_split:cnn {g__acro_#1_recorded_barriers_seq} {,} {#2} }

% --------------------------------------------------------------------------
% the user commands -- preparation:
\cs_new_protected:Npn \acro_begin:
  {
    \group_begin:
    \__acro_check_after_end:w
  }

\cs_new_protected:Npn \__acro_check_after_end:w #1 \acro_end:
  {
    \cs_set:Npn \__acro_execute:
      {
        \__acro_check_trail:N \l_peek_token
        #1
        \acro_end: % this will end the group opened by \acro_begin:
      }
    \peek_after:Nw \__acro_execute:
  }

\cs_new_protected:Npn \acro_end: { \group_end: }

\cs_new_protected:Npn \acro_reset_specials:
  {
    \bool_set_false:N \l__acro_indefinite_bool
    \bool_set_false:N \l__acro_first_upper_bool
    \bool_set_false:N \l__acro_upper_indefinite_bool
    % \bool_set_false:N \l__acro_citation_all_bool
    % \bool_set_false:N \l__acro_citation_first_bool
    % \bool_set_false:N \l__acro_addto_index_bool
    \acro_for_endings_do:n { \bool_set_false:c {l__acro_##1_bool} }
  }

% #1: ID
% #2: true|false
\cs_new_protected:Npn \acro_check_acronym:nn #1#2
  {
    \acro_defined:n {#1}
    \acro_use_acronym:n {#2}
  }

% #1: boolean
% #2: ID
\cs_new_protected:Npn \acro_check_and_mark_if:nn #1#2
  {
    \bool_if:nTF
      { (#1) || !\l__acro_use_acronyms_bool }
      { \acro_check_acronym:nn {#2} {false} }
      { \acro_check_acronym:nn {#2} {true} }
  }

\cs_new_protected:Npn \acro_switch_off:
  { \bool_set_false:N \l__acro_use_acronyms_bool }

\cs_new_protected:Npn \acro_switch_on:
  { \bool_set_true:N \l__acro_use_acronyms_bool }

\NewDocumentCommand \acswitchoff {}
  { \acro_switch_off: }

\NewDocumentCommand \acswitchon {}
  { \acro_switch_on: }

% commands for (re)defining \ac-like macros:
\cs_new_protected:Npn \acro_define_new_acro_command:NN #1#2
  {
    % #1: csname
    % #2: definition where `#1' refers to the ID
    \cs_new_protected:Npn #1 ##1##2
      {
        \cs_set:Npn \__acro_tmp_command:n ####1 {##2}
        \exp_args:NNnx #2 ##1 {sO{}m}
          {
            \acro_begin:
              \acro_reset_specials:
              \keys_set:nn {acro} {########2}
              \acro_check_and_mark_if:nn {########1} {########3}
              \exp_not:o { \__acro_tmp_command:n {####3} }
            \acro_end:
          }
      }
  }
\cs_generate_variant:Nn \acro_define_new_acro_command:NN {cc}

% commands for (re)defining \acflike-like macros:
\cs_new_protected:Npn \acro_define_new_acro_pseudo_command:NN #1#2
  {
    % #1: csname
    % #2: definition where `#1' refers to the ID and `#2' to the pseudo long form
    \cs_new_protected:Npn #1 ##1##2
      {
        \cs_set:Npn \__acro_tmp_command:nn ####1####2 {##2}
        \exp_args:NNnx #2 ##1 {smm}
          {
            \acro_begin:
              \acro_reset_specials:
              \acro_check_and_mark_if:nn {########1} {########2}
              \exp_not:o { \__acro_tmp_command:nn {####2} {####3} }
            \acro_end:
          }
      }
  }
\cs_generate_variant:Nn \acro_define_new_acro_pseudo_command:NN {cc}

\clist_map_inline:nn {New,Renew,Declare,Provide}
  {
    \acro_define_new_acro_command:cc
      {#1AcroCommand}
      {#1DocumentCommand}
    \acro_define_new_acro_pseudo_command:cc
      {#1PseudoAcroCommand}
      {#1DocumentCommand}
  }

% --------------------------------------------------------------------------
% user commands -- facilities
\cs_new_protected:Npn \acro_first_upper:
  {
    \bool_if:NTF \l__acro_indefinite_bool
      {
        \bool_set_false:N \l__acro_indefinite_bool
        \bool_set_true:N \l__acro_upper_indefinite_bool
      }
      { \bool_set_true:N \l__acro_first_upper_bool }
  }

\cs_new_protected:Npn \acro_indefinite:
  {
    \bool_if:NTF \l__acro_first_upper_bool
      {
        \bool_set_true:N \l__acro_upper_indefinite_bool
        \bool_set_false:N \l__acro_first_upper_bool
      }
      { \bool_set_true:N \l__acro_indefinite_bool }
  }

\cs_new_protected:Npn \acro_cite:
  {
    \bool_set_true:N \l__acro_citation_all_bool
    \bool_set_true:N \l__acro_citation_first_bool
  }

\cs_new_protected:Npn \acro_no_cite:
  {
    \bool_set_false:N \l__acro_citation_all_bool
    \bool_set_false:N \l__acro_citation_first_bool
  }

\cs_new_protected:Npn \acro_index:
  { \bool_set_true:N \l__acro_addto_index_bool }

% similar macros \acro_<ending>: are defined by \acro_provide_ending:nnn

% ---------------------------------------------------------------------------
% process options:
\ProcessKeysPackageOptions {acro}

% ---------------------------------------------------------------------------
% PDF bookmark support
\cs_new:Npn \acpdfstring
  { \acro_pdf_string_short:n }

\cs_new:Npn \acpdfstringalt
  { \acro_pdf_string_alt:n }

\cs_new:Npn \acpdfstringlong
  { \acro_pdf_string_long:n }

\cs_new:Npn \acpdfstringfirst #1
  { \acpdfstringlong {#1} ~ ( \acpdfstring {#1} ) }

% TODO: place this somewhere where endings are defined:
\cs_new:Npn \acpdfstringlongplural
  { \acro_pdf_string_long_plural:n }

\prg_new_conditional:Npnn \acro_if_star_gobble:n #1 {TF}
  {
    \if_meaning:w *#1
      \prg_return_true:
    \else:
      \prg_return_false:
    \fi:
  }

\cs_new:Npn \acro_expandable_long:n #1
  { \prop_item:Nn \l__acro_long_prop {#1} }

\cs_new:Npn \acro_expandable_long_plural:n #1
  {
    \bool_if:nTF
      { \prop_item:Nn \l__acro_long_plural_form_prop {#1} }
      { \prop_item:Nn \l__acro_long_plural_prop {#1} }
      {
        \prop_item:Nn \l__acro_long_prop {#1}
        \prop_item:Nn \l__acro_long_plural_prop {#1}
      }
  }

\cs_new:Npn \acro_pdf_string_long:n #1
  {
    \acro_if_star_gobble:nTF {#1}
      { \acro_expandable_long:n }
      { \acro_expandable_long:n {#1} }
  }

% TODO: place this somewhere where endings are defined:
\cs_new:Npn \acro_pdf_string_long_plural:n #1
  {
    \acro_if_star_gobble:nTF {#1}
      { \acro_expandable_long_plural:n }
      { \acro_expandable_long_plural:n {#1} }
  }
  
\cs_new:Npn \acro_pdf_string_short:n #1
  {
    \acro_if_star_gobble:nTF {#1}
      { \prop_item:Nn \l__acro_pdfstring_short_prop }
      { \prop_item:Nn \l__acro_pdfstring_short_prop {#1} }
  }
  
\cs_new:Npn \acro_pdf_string_alt:n #1
  {
    \acro_if_star_gobble:nTF {#1}
      { \prop_item:Nn \l__acro_pdfstring_alt_prop }
      { \prop_item:Nn \l__acro_pdfstring_alt_prop {#1} }
  }

\AtBeginDocument
  {
    \@ifpackageloaded {hyperref}
      {
        \bool_set_true:N \l__acro_hyperref_loaded_bool
        \pdfstringdefDisableCommands
          {
            \cs_set_eq:NN \ac   \acpdfstring
            \cs_set_eq:NN \Ac   \acpdfstring
            \cs_set_eq:NN \acs  \acpdfstring
            \cs_set_eq:NN \acl  \acpdfstringlong
            \cs_set_eq:NN \Acl  \acpdfstringlong
            \cs_set_eq:NN \acf  \acpdfstringfirst
            \cs_set_eq:NN \Acf  \acpdfstringfirst
            \cs_set_eq:NN \aca  \acpdfstringalt
            \cs_set_eq:NN \acp  \acpdfstringplural
            \cs_set_eq:NN \Acp  \acpdfstringplural
            \cs_set_eq:NN \acsp \acpdfstringplural
            \cs_set_eq:NN \aclp \acpdfstringlongplural
            \cs_set_eq:NN \Aclp \acpdfstringlongplural
            \cs_set_eq:NN \acfp \acpdfstringplural
            \cs_set_eq:NN \Acfp \acpdfstringplural
            \cs_set_eq:NN \acap \acpdfstringaltplural
          }
        \cs_set_protected:Npn \acro_hyper_page:n #1 { \hyperpage {#1} }
      } {}
  }

% --------------------------------------------------------------------------
% additional variables:
\tl_new:N \l__acro_current_property_tl

% --------------------------------------------------------------------------
% key and order checking
\msg_new:nnn {acro} {no-id}
  {
    Something~ has~ gone~ wrong,~ you've~ probably~ forgotten~ to~ set~ the~
    acronym~ ID.
  }

\msg_new:nnn {acro} {before-short}
  {
    You've~ set~ the~ property~ `#2'~ before~ the~ `short'~ property~ for~
    acronym~ `#1'~ but~ it~ needs~ to~ be~ set~ after~ it.
  }

\msg_new:nnn {acro} {missing}
  { The~ `#2'~ property~ for~ acronym~ `#1'~ is~ missing. }

\msg_new:nnn {acro} {doubled-property}
  {
    It~ seems~ to~ me~ you~ have~ used~ the~ `#2'~ property~ twice~ in~ the~
    declaration~ of~ acronym~ `#1'.~ If~ you~ haven't~ there's~
    something~ different~ wrong~ and~ I'm~ lost.~ You~'re~ on~ your~ own~
    then.
  }

\cs_new_protected:Npn \__acro_property_check:nn #1#2
  {
    \tl_if_blank:VT \l__acro_current_property_tl
      { \acro_serious_message:n {no-id} }
    \bool_if:cF { l__acro_#1_short_set_bool }
      {
        \keys_set:nn { acro / declare-acronym } { short = {#1} }
        \acro_harmless_message:nn {substitute-short} {#1}
      }
    \bool_new:c { l__acro_#1_#2_set_bool }
    \bool_set_true:c { l__acro_#1_#2_set_bool }
  }

\cs_new_protected:Npn \__acro_first_property_check:nn #1#2
  {
    \cs_if_exist:cTF { l__acro_#1_short_set_bool }
      {
         \bool_if:cT { l__acro_#1_short_set_bool }
           { \acro_serious_message:nnn {doubled-property} {#1} {#2} }
      }
      {
        \bool_new:c { l__acro_#1_short_set_bool }
        \bool_set_true:c { l__acro_#1_short_set_bool }
      }
  }

% --------------------------------------------------------------------------
% the internal property selection functions for \DeclareAcronym:

% #1: name in associated cs
% #2: property name
% #3: action
\cs_new_protected:Npn \acro_declare_property_generic:nnn #1#2#3
  {
    \prop_clear_new:c {l__acro_#1_prop}
    \cs_new_protected:cpn   {__acro_declare_#1:nn} ##1##2 {#3}
    \cs_generate_variant:cn {__acro_declare_#1:nn} {V}
    \keys_define:nn {acro/declare-acronym}
      {
        #2 .code:n =
          \use:c {__acro_declare_#1:Vn} \l__acro_current_property_tl {##1}
      }
  }

% #1: name in associated cs
% #2: property name
% #3: action
\cs_new_protected:Npn \acro_declare_property:nnn #1#2#3
  {
    \acro_declare_property_generic:nnn {#1} {#2}
      {
        \__acro_property_check:nn {##1} {#2}
        \prop_put:cnn {l__acro_#1_prop} {##1} {##2}
        #3
      }
  }

% #1: name in associated cs
% #2: property name
\cs_new_protected:Npn \acro_declare_property:nn #1#2
  { \acro_declare_property:nnn {#1} {#2} {} }
\cs_generate_variant:Nn \acro_declare_property:nn { V }

\cs_new_protected:Npn \acro_declare_simple_property:n #1
  {
    \tl_set:Nn \l__acro_tmpa_tl {#1}
    \tl_replace_all:Nnn \l__acro_tmpa_tl {-} {_}
    \tl_clear_new:c  {l__acro_ \l__acro_tmpa_tl _tl}
    \acro_declare_property:Vn \l__acro_tmpa_tl {#1}
  }

% #1: new alias property
% #2: old property
\cs_new_protected:Npn \acro_declare_property_alias:nn #1#2
  {
    \keys_define:nn {acro/declare-acronym}
      { #1 .meta:n = { #2 = {##1} } }
  }

% --------------------------------------------------------------------------
% declare the properties for \DeclareAcronym:
% short:
\acro_declare_property_generic:nnn {short} {short}
  {
    \__acro_first_property_check:nn {#1} {short}
    \prop_put:Nnn \l__acro_short_prop      {#1} {#2}
    \prop_put:Nnn \l__acro_sort_prop       {#1} {#1}
    \prop_put:Nnn \l__acro_index_sort_prop {#1} {#1}
    \prop_put:Nnn \l__acro_alt_prop        {#1} {#2}
    \prop_put:Nnn \l__acro_pdfstring_short_prop {#1} {#2}
    \prop_put:Nnn \l__acro_pdfstring_alt_prop {#1} {#2}
    \acro_for_endings_do:n
      {
        \prop_put:cnv {l__acro_short_##1_prop}
          {#1} {l__acro_default_short_##1_tl}
        \prop_put:cnx {l__acro_pdfstring_short_##1_prop}
          {#1} { \exp_not:n {#2} \exp_not:v {l__acro_default_short_##1_tl} }
        \prop_put:cnn {l__acro_short_##1_form_prop} {#1} { \c_false_bool }
        \prop_put:cnv {l__acro_alt_##1_prop}
          {#1} {l__acro_default_alt_##1_tl}
        \prop_put:cnx {l__acro_pdfstring_alt_##1_prop}
          {#1} { \exp_not:n {#2} \exp_not:v {l__acro_default_short_##1_tl} }
        \prop_put:cnn {l__acro_alt_##1_form_prop} {#1} { \c_false_bool }
      }
    \prop_put:NnV \l__acro_short_indefinite_prop
      {#1} \l__acro_default_indefinite_tl
    \prop_put:NnV \l__acro_alt_indefinite_prop
      {#1} \l__acro_default_indefinite_tl
  }

% long:
\acro_declare_property:nnn {long} {long}
  {
    \acro_for_endings_do:n
      { \prop_put:cnn {l__acro_long_##1_form_prop} {#1} { \c_false_bool } }
    \prop_put:NnV \l__acro_long_indefinite_prop
      {#1}
      \l__acro_default_indefinite_tl
    \acro_for_endings_do:n
      {
        \bool_if:cF {l__acro_#1_long-##1_set_bool}
          { \prop_put:cnv {l__acro_long_##1_prop} {#1} {l__acro_default_long_##1_tl} }
      }
  }

\acro_declare_simple_property:n {first-style}

% list:
\acro_declare_simple_property:n {list}

% defines `short-plural', `long-plural' and `long-plural-form' as well as the
% options `plural-ending', `short-plural-ending' and `long-plural-ending':
% \ProvideAcroEnding {plural} {s} {s}

% short indefinite article:
\acro_declare_simple_property:n {short-indefinite}

% long indefinite article:
\acro_declare_simple_property:n {long-indefinite}

% pre long:
\acro_declare_simple_property:n {long-pre}

% post long:
\acro_declare_simple_property:n {long-post}

% sort:
\acro_declare_property:nnn {sort} {sort}
  {
    \bool_if:cF { l__acro_#1_index-sort_set_bool }
      { \prop_put:Nnn \l__acro_index_sort_prop {#1} {#2} }
  }

% alternative:
\acro_declare_property:nnn {alt} {alt}
  {
    \prop_put:Nnn \l__acro_pdfstring_alt_prop {#1} {#2}
    \prop_put:NnV \l__acro_alt_indefinite_prop
      {#1} \l__acro_default_indefinite_tl
  }

\cs_set_protected:Npn \acro_alt_error:n #1
  {
    \bool_if:cF {l__acro_#1_alt_set_bool} 
      { \acro_harmless_message:nn {no-alternative} {#1} }
  }

% alt. indefinite article:
\acro_declare_simple_property:n {alt-indefinite}

% foreign:
\acro_declare_simple_property:n {foreign}

% foreign:
\acro_declare_simple_property:n {foreign-lang}

% format:
\acro_declare_simple_property:n {format}

% short format:
\acro_declare_property_alias:nn {short-format} {format}

% long format:
\acro_declare_simple_property:n {long-format}

% first long format:
\acro_declare_simple_property:n {first-long-format}

% pdfstring -- currently needs to be done `by hand':
\prop_new:N \l__acro_pdfstring_short_prop
\cs_new_protected:Npn \__acro_declare_pdfstring:nw #1#2/#3/#4 \acro_stop:
  {
    \__acro_property_check:nn {#1} {pdfstring}
    \prop_put:Nnx \l__acro_pdfstring_short_prop {#1} {#2}
    \acro_for_endings_do:n
      {
        \tl_if_blank:nTF {#4}
          {
            \prop_put:cnx {l__acro_pdfstring_short_##1_prop}
              {#1} { \exp_not:n {#2} \exp_not:v {l__acro_default_short_##1_tl} }
          }
          {
            \prop_put:cnn {l__acro_pdfstring_short_##1_prop}
              {#1} {#2#3}
          }
      }
  }
\cs_generate_variant:Nn \__acro_declare_pdfstring:nw { V }
\keys_define:nn { acro / declare-acronym }
  {
    pdfstring    .code:n =
      \__acro_declare_pdfstring:Vw \l__acro_current_property_tl #1 // \acro_stop: ,
  }

\prop_new:N \l__acro_pdfstring_alt_prop
\cs_new_protected:Npn \__acro_declare_pdfstring_alt:nw #1#2/#3/#4 \acro_stop:
  {
    \__acro_property_check:nn {#1} { pdfstring-alt }
    \prop_put:Nnn \l__acro_pdfstring_alt_prop {#1} {#2}
    \acro_for_endings_do:n
      {
        \tl_if_empty:nTF {#3}
          {
            \prop_put:cnx {l__acro_pdfstring_alt_##1_prop}
              {#1} { \exp_not:n {#2} \exp_not:v {l__acro_default_alt_##1_tl} }
          }
          { \prop_put:cnn {l__acro_pdfstring_alt_##1_prop} {#1} {#2#3} }
      }
  }
\cs_generate_variant:Nn \__acro_declare_pdfstring_alt:nw { V }
\keys_define:nn { acro / declare-acronym }
  {
    pdfstring-alt .code:n =
      \__acro_declare_pdfstring_alt:Vw \l__acro_current_property_tl #1 // \acro_stop: ,
  }
  
% class:
\acro_declare_simple_property:n {class}

% extra information:
\acro_declare_simple_property:n {extra}

% single appearances:
\acro_declare_simple_property:n {single}

% single format:
\acro_declare_simple_property:n {single-format}

% acc supp:
\acro_declare_property:nn {acc_supp} {accsupp}

% tooltip:
\acro_declare_simple_property:n {tooltip}

% citation -- currently needs to be done `by hand':
\prop_new:N \l__acro_citation_prop
\prop_new:N \l__acro_citation_pre_prop
\prop_new:N \l__acro_citation_post_prop
\cs_new_protected:Npn \__acro_declare_citation:nw #1#2[#3]#4[#5]#6#7 \acro_stop:
  {
    % no options: #1: ID, #2: property, #3 is blank
    % 1 option:   #1: ID, #4: property, #3: option, #5 is blank
    % 2 options:  #1: ID: #6: property, #3: first option, #5: second option
    \tl_if_blank:nF { #2#4#6 }
      {
        \tl_if_empty:nTF {#3}
          { \__acro_declare_citation_aux:nnnn {#1} { } { } {#2} }
          {
            \tl_if_empty:nTF {#5}
              { \__acro_declare_citation_aux:nnnn {#1} {#3} {  } {#4} }
              { \__acro_declare_citation_aux:nnnn {#1} {#3} {#5} {#6} }
          }
      }
  }
\cs_generate_variant:Nn \__acro_declare_citation:nw { V }

\keys_define:nn { acro / declare-acronym }
  {
    cite .code:n =
      \__acro_declare_citation:Vw
        \l__acro_current_property_tl #1 [][] \scan_stop: \acro_stop:
  }

\cs_new_protected:Npn \__acro_declare_citation_aux:nnnn #1#2#3#4
  {
    \__acro_property_check:nn {#1} {cite}
    \prop_put:Nnn \l__acro_citation_prop {#1} {#4}
    \tl_if_empty:nF {#2}
      { \prop_put:Nnn \l__acro_citation_pre_prop {#1} {#2} }
    \tl_if_empty:nF {#3}
      { \prop_put:Nnn \l__acro_citation_post_prop {#1} {#3} }
  }

% TODO:
% add index entries, by default \index{<sort>@<short>}
% index: overwrite default <sort>@<short> entry completely
% index-sort: overwrite the <sort> part of <sort>@<short> entry

% need to take care of custom index cmd, at least
%  - \index{}
%  - \index[]{}
% question is, though, if it should be the same one for all acronyms?
% I go for yes but would also add a `post' property that allows to add arbitrary
% TeX code after an acronym is typeset

% index:
\acro_declare_simple_property:n {index}

% index-sort:
\acro_declare_simple_property:n {index-sort}

% index-cmd:
\acro_declare_simple_property:n {index-cmd}

% --------------------------------------------------------------------------
% acronym macros:
\cs_new_protected:Npn \__acro_define_acronym_macro:n #1
  {
    \bool_if:NT \l__acro_create_macros_bool
      {
        \cs_if_exist:cTF {#1}
          {
            \bool_if:NTF \l__acro_strict_bool
              { \cs_set:cpn {#1} { \ac {#1} \acro_xspace: } }
              { \acro_serious_message:nn {macro} {#1} }
          }
          { \cs_new:cpn {#1} { \ac {#1} \acro_xspace: } }
      }
  }

% --------------------------------------------------------------------------
% internal acronym declaring function:
\cs_new_protected:Npn \acro_declare_acronym:nn #1#2
  {
    \seq_gput_right:Nn \g__acro_declared_acronyms_seq {#1}
    \bool_gset_true:N \g__acro_first_acronym_declared_bool
    \tl_set:Nn \l__acro_current_property_tl {#1}
    \keys_set:nn { acro / declare-acronym } {#2}
    \bool_new:c {g__acro_#1_first_use_bool}
    \bool_new:c {g__acro_#1_used_bool}
    \bool_new:c {g__acro_#1_label_bool}
    \bool_new:c {g__acro_#1_in_list_bool}
    \seq_new:c  {g__acro_#1_barriers_seq}
    \seq_new:c  {g__acro_#1_recorded_barriers_seq}
    \bool_if:NF \l__acro_print_only_used_bool
      { \bool_gset_true:c {g__acro_#1_in_list_bool} }
    \__acro_create_page_records:n {#1}
    \__acro_define_acronym_macro:n {#1}
    \tl_clear:N \l__acro_current_property_tl
    \bool_if:cF {l__acro_#1_short_set_bool}
      { \acro_serious_message:nnn {missing} {#1} {short} }
    \bool_if:cF {l__acro_#1_long_set_bool}
      { \acro_serious_message:nnn {missing} {#1} {long} }
    \__acro_log_acronym:n {#1}
  }

% --------------------------------------------------------------------------
% the user command:
\NewDocumentCommand \DeclareAcronym {mm}
  { \acro_declare_acronym:nn {#1} {#2} }

% --------------------------------------------------------------------------
% print the list:
% #1: list of classes
% #2: list of excluded classes
\tl_new:N \l__acro_included_classes_tl
\tl_new:N \l__acro_excluded_classes_tl

\keys_define:nn { acro / print-acronyms }
  {
    include-classes   .tl_set:N   = \l__acro_included_classes_tl ,
    exclude-classes   .tl_set:N   = \l__acro_excluded_classes_tl ,
    name              .tl_set:N   = \l__acro_list_name_tl ,
    heading           .code:n     = \__acro_set_list_heading:n {#1} ,
    sort              .bool_set:N = \l__acro_sort_bool ,
    local-to-barriers .bool_set:N = \l__acro_use_barrier_bool
  }

\cs_new_protected:Npn \acro_print_acronyms:n #1
  {
    \group_begin:
      % this is a cheap trick to prevent the \@noitemerr
      % if one forgot to delete either the aux file or
      % remove \printacronyms -- but it's local:
      \cs_set:Npn \@noitemerr {}
      \tl_clear:N \l__acro_included_classes_tl
      \tl_clear:N \l__acro_excluded_classes_tl
      \keys_set:nn { acro / print-acronyms } {#1}
      \__acro_aux_file_now:n { \acro@print@list }
      \bool_if:NT \l__acro_sort_bool
        { \acro_sort_prop:NN \l__acro_short_prop \l__acro_sort_prop }
      \acro_title_instance:VV
        \l__acro_list_heading_cmd_tl
        \l__acro_list_name_tl
      \cs_if_exist:NTF \acro@printed@list
        {
          \acro_list_instance:VVV
            \l__acro_list_instance_tl
            \l__acro_included_classes_tl
            \l__acro_excluded_classes_tl
        }
        { \@latex@warning@no@line {Rerun~to~get~acronym~list~right} }
    \group_end:
  }

\NewDocumentCommand \printacronyms { O{} }
  { \acro_print_acronyms:n {#1} }

% --------------------------------------------------------------------------
% language support
\RequirePackage {translations}

\cs_new_protected:Npn \__acro_declare_translation:www #1 \q_mark #2=#3 \q_stop
  {
    \tl_set:Nx \l__acro_tmpa_tl { \tl_trim_spaces:n {#1} }
    \tl_set:Nx \l__acro_tmpb_tl { \tl_trim_spaces:n {#2} }
    \tl_if_in:nnT {#3} {=}
      {} % TODO: misplaced equal sign
    \tl_set:Nx \l__acro_tmpc_tl { \tl_trim_spaces:n {#3} }
    \__acro_declare_translation:VVV
      \l__acro_tmpb_tl
      \l__acro_tmpa_tl
      \l__acro_tmpc_tl
  }

% #1: key
% #2: lang
% #3: translation
\cs_new_protected:Npn \__acro_declare_translation:nnn #1#2#3
  { \DeclareTranslation {#1} {#2} {#3} }
\cs_generate_variant:Nn \__acro_declare_translation:nnn {VVV}

% #1: key
% #2: csv list: { <lang1> = <translation1> , <lang2> = <translation2> }
\cs_new_protected:Npn \acro_declare_translation:nn #1#2
  {
    \clist_map_inline:nn {#2}
      {
        \tl_if_blank:nF {##1}
          { \__acro_declare_translation:www #1 \q_mark ##1 \q_stop }
      }
  }

\NewDocumentCommand \DeclareAcroTranslation {mm}
  { \acro_declare_translation:nn {#1} {#2} }
\@onlypreamble \DeclareAcroTranslation

% tokenlists using the translations:
\tl_set:Nn \l__acro_list_name_tl  { \GetTranslation {acronym-list-name} }
\tl_set:Nn \l__acro_page_name_tl  { \GetTranslation {acronym-page-name}\@\, }
\tl_set:Nn \l__acro_pages_name_tl { \GetTranslation {acronym-pages-name}\@\, }
\tl_set:Nn \l__acro_next_page_tl  { \,\GetTranslation {acronym-next-page}\@ }
\tl_set:Nn \l__acro_next_pages_tl { \,\GetTranslation {acronym-next-pages}\@ }

% --------------------------------------------------------------------------
% definition file:
% \tl_const:Nn \c_acro_definition_file_name_tl      {acro}
% \tl_const:Nn \c_acro_definition_file_extension_tl {def}

% \file_if_exist:nTF
%   { \c_acro_definition_file_name_tl .\c_acro_definition_file_extension_tl }
%   {
%     \@onefilewithoptions
%       {\c_acro_definition_file_name_tl} [] []
%       \c_acro_definition_file_extension_tl
%   }
%   { \acro_serious_message:n {definitions-missing} }


% --------------------------------------------------------------------------
% first appearance styles:
\DeclareAcroFirstStyle {default} {inline}
  { }

\DeclareAcroFirstStyle {reversed} {inline}
  { reversed = true }

\DeclareAcroFirstStyle {short} {inline}
  {
    only-short = true ,
    brackets = false
  }

\DeclareAcroFirstStyle {long} {inline}
  {
    only-long = true ,
    brackets = false
  }

\DeclareAcroFirstStyle {square} {inline}
  { brackets-type = [] }

\DeclareAcroFirstStyle {plain} {inline}
  {
    brackets = false ,
    between = --
  }

\DeclareAcroFirstStyle {plain-reversed} {inline}
  { 
    brackets = false ,
    between = -- ,
    reversed = true
  }

\DeclareAcroFirstStyle {footnote} {note}
  { }

\DeclareAcroFirstStyle {footnote-reversed} {note}
  { reversed = true }

\DeclareAcroFirstStyle {sidenote} {note}
  { note-command = \sidenote {#1} }

\DeclareAcroFirstStyle {sidenote-reversed} {note}
  {
    note-command = \sidenote {#1} ,
    reversed     = true
  }

\DeclareAcroFirstStyle {empty} {note}
  { use-note = false }

% --------------------------------------------------------------------------
% extra info appearance styles:
\DeclareAcroExtraStyle {default} {inline}
  {
    brackets     = false ,
    punct        = true ,
    punct-symbol = .
  }

\DeclareAcroExtraStyle {plain} {inline}
  {
    brackets     = false ,
    punct        = true ,
    punct-symbol =
  }

\DeclareAcroExtraStyle {paren} {inline}
  {
    brackets     = true ,
    punct        = true ,
    punct-symbol =
  }

\DeclareAcroExtraStyle {bracket} {inline}
  {
    brackets      = true ,
    punct         = true ,
    punct-symbol  = ,
    brackets-type = []
  }

\DeclareAcroExtraStyle {comma} {inline}
  {
    punct         = true,
    punct-symbol  = {,} ,
    brackets      = false
  }

% --------------------------------------------------------------------------
% page number appearance styles:
\DeclareAcroPageStyle {default} {inline}
  {
    punct = true ,
    punct-symbol = .
  }
  
\DeclareAcroPageStyle {plain}   {inline}
  { punct = false }

\DeclareAcroPageStyle {comma}   {inline}
  { punct = true }

\DeclareAcroPageStyle {paren}   {inline}
  {
    brackets=true ,
    punct-symbol = ~
  }

\DeclareAcroPageStyle {none}    {inline}
  { display = false }

% --------------------------------------------------------------------------
% list heading styles:
\DeclareAcroListHeading {part}           {\part}
\DeclareAcroListHeading {part*}          {\part*}
\DeclareAcroListHeading {chapter}        {\chapter}
\DeclareAcroListHeading {chapter*}       {\chapter*}
\DeclareAcroListHeading {addchap}        {\addchap}
\DeclareAcroListHeading {section}        {\section}
\DeclareAcroListHeading {section*}       {\section*}
\DeclareAcroListHeading {addsec}         {\addsec}
\DeclareAcroListHeading {subsection}     {\subsection}
\DeclareAcroListHeading {subsection*}    {\subsection*}
\DeclareAcroListHeading {subsubsection}  {\subsubsection}
\DeclareAcroListHeading {subsubsection*} {\subsubsection*}
\DeclareAcroListHeading {none}           {\use_none:n}

% --------------------------------------------------------------------------
% list styles:
\DeclareAcroListStyle {description} {list}
  { }

\DeclareAcroListStyle {toc} {list-of}
  { }

\DeclareAcroListStyle {lof} {list-of}
  { style = lof }

\DeclareAcroListStyle {tabular} {table}
  { table = tabular }

\DeclareAcroListStyle {longtable} {table}
  { table = longtable }

\DeclareAcroListStyle {extra-tabular} {extra-table}
  { table = tabular }

\DeclareAcroListStyle {extra-longtable} {extra-table}
  { table = longtable }

\DeclareAcroListStyle {extra-tabular-rev} {extra-table}
  {
    table   = tabular ,
    reverse = true
  }

\DeclareAcroListStyle {extra-longtable-rev} {extra-table}
  {
    table   = longtable ,
    reverse = true
  }

% --------------------------------------------------------------------------
% register some tokens to be checked for:
\AcroRegisterTrailing . {dot}
\AcroRegisterTrailing - {dash}
\AcroRegisterTrailing \babelhyphen {babel-hyphen}

% --------------------------------------------------------------------------
% the user commands
% automatic:
\NewAcroCommand \ac
  { \acro_use:n {#1} }

\NewAcroCommand \iac
  {
    \acro_indefinite:
    \acro_use:n {#1}
  }

\NewAcroCommand \Iac
  {
    \acro_first_upper:
    \acro_indefinite:
    \acro_use:n {#1}
  }

\NewAcroCommand \Ac
  {
    \acro_first_upper:
    \acro_use:n {#1}
  }

\NewAcroCommand \acp
  {
    \acro_plural:
    \acro_use:n {#1}
  }

\NewAcroCommand \Acp
  {
    \acro_plural:
    \acro_first_upper:
    \acro_use:n {#1}
  }

\NewAcroCommand \acsingle
  {
    \acro_get:n {#1}
    \acro_single:n {#1}
  }

\NewAcroCommand \Acsingle
  {
    \acro_get:n {#1}
    \acro_first_upper:
    \acro_single:n {#1}
  }

% short:
\NewAcroCommand \acs
  { \acro_short:n {#1} }

\NewAcroCommand \iacs
  {
    \acro_indefinite:
    \acro_short:n {#1}
  }

\NewAcroCommand \Iacs
  {
    \acro_first_upper:
    \acro_indefinite:
    \acro_short:n {#1}
  }

\NewAcroCommand \acsp
  {
    \acro_plural:
    \acro_short:n {#1}
  }

% alt:
\NewAcroCommand \aca
  { \acro_alt:n {#1} }

\NewAcroCommand \Aca
  {
    \acro_first_upper:
    \acro_alt:n {#1}
  }
  
\NewAcroCommand \iaca
  {
    \acro_indefinite:
    \acro_alt:n {#1}
  }

\NewAcroCommand \Iaca
  {
    \acro_first_upper:
    \acro_indefinite:
    \acro_alt:n {#1}
  }

\NewAcroCommand \acap
  {
    \acro_plural:
    \acro_alt:n {#1}
  }

% long:
\NewAcroCommand \acl
  { \acro_long:n {#1} }

\NewAcroCommand \iacl
  {
    \acro_indefinite:
    \acro_long:n {#1}
  }

\NewAcroCommand \Iacl
  {
    \acro_first_upper:
    \acro_indefinite:
    \acro_long:n {#1}
  }

\NewAcroCommand \Acl
  {
    \acro_first_upper:
    \acro_long:n {#1}
  }

\NewAcroCommand \aclp
  {
    \acro_plural:
    \acro_long:n {#1}
  }

\NewAcroCommand \Aclp
  {
    \acro_plural:
    \acro_first_upper:
    \acro_long:n {#1}
  }

% first:
\NewAcroCommand \acf
  { \acro_first:n {#1} }

\NewAcroCommand \iacf
  {
    \acro_indefinite:
    \acro_first:n {#1}
  }

\NewAcroCommand \Iacf
  {
    \acro_first_upper:
    \acro_indefinite:
    \acro_first:n {#1}
  }

\NewAcroCommand \Acf
  {
    \acro_first_upper:
    \acro_first:n {#1}
  }

\NewAcroCommand \acfp
  {
    \acro_plural:
    \acro_first:n {#1}
  }

\NewAcroCommand \Acfp
  {
    \acro_plural:
    \acro_first_upper:
    \acro_first:n {#1}
  }

% first-like:
\NewPseudoAcroCommand \acflike
  { \acro_first_like:nn {#1} {#2} }

\NewPseudoAcroCommand \iacflike
  {
    \acro_indefinite:
    \acro_first_like:nn {#1} {#2}
  }

\NewPseudoAcroCommand \Iacflike
  {
    \acro_first_upper:
    \acro_indefinite:
    \acro_first_like:nn {#1} {#2}
  }

\NewPseudoAcroCommand \acfplike
  {
    \acro_plural:
    \acro_first_like:nn {#1} {#2}
  }

% --------------------------------------------------------------------------
% endings:
\ProvideAcroEnding {plural} {s} {s}

% --------------------------------------------------------------------------
% translations:
% list name
\DeclareAcroTranslation {acronym-list-name}
  {
    Fallback   = Acronyms ,
    English    = Acronyms ,
    French     = Acronymes ,
    German     = Abk\"urzungen ,
    Italian    = Acronimi ,
    Portuguese = Acr\'onimos ,
    Spanish    = Siglas ,
    Catalan    = Sigles ,
    Turkish    = K\i saltmalar
  }

% page name
\DeclareAcroTranslation {acronym-page-name}
  {
    Fallback   = p. ,
    English    = p. ,
    German     = S. ,
    Portuguese = p.
  }

% pages name
\DeclareAcroTranslation {acronym-pages-name}
  {
    Fallback   = pp. ,
    English    = pp. ,
    German     = S. ,
    Portuguese = pp.
  }

% following page
\DeclareAcroTranslation {acronym-next-page}
  {
    Fallback   = f. ,
    English    = f. ,
    German     = f. ,
    Portuguese = s.
  }

% following pages
\DeclareAcroTranslation {acronym-next-pages}
  {
    Fallback   = ff. ,
    English    = ff. ,
    German     = ff. ,
    Portuguese = ss.
  }

% --------------------------------------------------------------------------
% allow for a configuration file:
\tl_const:Nn \c_acro_config_file_name_tl      {acro}
\tl_const:Nn \c_acro_config_file_extension_tl {cfg}

\file_if_exist:nT
  { \c_acro_config_file_name_tl .\c_acro_config_file_extension_tl }
  {
    \@onefilewithoptions
      {\c_acro_config_file_name_tl} [] []
      \c_acro_config_file_extension_tl
  }

\tex_endinput:D
% --------------------------------------------------------------------------
% HISTORY:
2012/06/22 v0.1  - first public release
2012/06/23 v0.1a - bug fix, added `strict' and `macros' option and creation of
                   shortcut macros
                 - added capitalized version of long forms
                 - added `sort' option
2012/06/24 v0.1b - added \Acf and \Acfp, added option `plural-ending'
2012/06/24 v0.1c - added excluded argument to \printacronyms
2012/06/24 v0.2  - renamed \NewAcronym => \DeclareAcronym
                   \AcronymFormat => \DeclareAcronymFormat
2012/06/25 v0.2a - new first-style's: `short' and `reversed'
2012/06/25 v0.3  - new list formats: extra-tabular, extra-longtable,
                   extra-tabular-rev, extra-longtable-rev
                 - extra precaution when using \printacronyms to avoid errors.
2012/06/27 v0.3a - new option `list-caps', \Acp added
2012/06/29 v0.3b - extended the `text' template to the `acro-first' object
                 - added `acro-first' instances `plain' and `plain-reversed'
2012/07/16 v0.3c - small adjustments to the documentation
2012/07/23 v0.3d - first CTAN version
2012/07/24 v0.3e - adapted to updated l3kernel
2012/09/28 v0.4  - added means to add citations to acronyms
2012/10/07 v0.4a - new options: "uc-cmd", "list-long-format"
                 - preliminary language support, needs package `translations'
2012/11/30 v0.5  - added starred variants of the commands that won't mark an
                   acronym as used
                 - added \acreset{<id>}
                 - added preliminary support for pdf strings: in pdf strings
                   always the singular lowercase short version is inserted
                   (the equivalent of \acs)
2012/12/14 v0.6  - bug with not-colored links resolved
                 - bug introduced with the last update (full expansion of the
                   short entry) resolved
                 - option `xspace' added
2013/01/02 v0.6a - \acuseall
2013/01/16 v1.0  - new syntax of \DeclareAcronym
                 - new option `version'
                 - new `accsupp' acronym property
                 - new `sort' acronym property
                 - new syntax of \printacronyms
                 - new default: `sort=true' 
                 - new options `page-ranges', `next-page', `next-pages',
                   `pages-name', `record-pages'
                 - no automatic label placement for page number referencing
                   any more
2013/01/26 v1.1  - bug fix in the plural detection
                 - new keys `long-pre' and `long-post'
                 - new keys `index', `index-sort' and `index-cmd'
                 - new options `index' and `index-cmd'
2013/01/29 v1.1a - added `long-format' key
                 - renamed `format' key into `short-format', kept `format' for
                   compatibility reasons
2013/02/09 v1.2  - error message instead of hanging when an undefined acronym
                   is used
                 - added `first-long-format' key and `first-long-format' option
                 - added \acflike and \acfplike
                 - improvements and bug fixes to the page recording mechanism
                 - new option `mark-as-used'
                 - new keys: `short-indefinite', `alt-indefinite' and
                   `long-indefinite'
                 - new commands: \iac, \Iac, \iacs, \Iacs, \iaca, \Iaca, \iacl,
                   \Iacl, \iacf, \Iacf, \iacflike and \Iacflike
2013/04/04 v1.2a - added Portuguese translations
2013/05/06 v1.3  - protected internal commands where appropriate
                 - new option `sort' to \printacronyms
                 - renamed options `print-acronyms/header' and `list-header'
                   into `print-acronyms/heading' and `list-heading'
                 - fix: added missing group to \printacronyms
                 - add key `foreign'
                 - rewritten page-recording:
                   * most importantly: record them at shipout; this is done
                     when \acro@used@once or \acro@used@twice are written to
                     the aux file
                   * no restrictions regarding \pagenumbering
                   * options `page-ranges' and `record-pages' are deprecated
                   * new options `following-page' and `following-pages'
                 - disable \@noitemerr in the list of acronyms: we don't need
                   it there but there are occasions when it is annoying
                 - cleaned the sty file, added a few more comments
2013/05/09 v1.3a - Bug fix: corrected wrong argument checking in \Ac, thanks
                   to Michel Voßkuhle
2013/05/30 v1.3b - obey \if@filesw
2013/06/16 v1.3c - added \leavevmode to \acro_get:n
2013/07/08 v1.3d - corrected wrong call of \leavevmode in the list
                   (list-type=list)
2013/08/07 v1.3e - bug fix in the list when testing for used acronyms
                 - new commands \acifused, \acfirstupper
2013/08/27 v1.4  - new property `list'
2013/09/02 v1.4a - bug fix: used acronyms are added to the list when the list
                   is printed before the use
                 - \DeclareAcronym may now be used after \begin{document}
2013/09/24 v1.4b - bug fix: only-used=false works again for only declared but
                   unused acronyms (only if option single is not used)
2013/11/04 v1.4c - remove \hbox from the written short form
                 - changed \__acro_make_link:nNN in a way that it doesn't box
                   its when links are deactivated
2013/11/22 v1.4d - require `l3sort' independently from the `sort' option
                   instead of at begin document in order to avoid conflicts
                   with `babel' and `french'
2013/12/18 v1.5  - new option `label=true|false' that
                   places \label{<prefix>:<id>} the first time an acronym is
                   used
                 - new option `pages=first|all' that determines if in the list
                   of acronyms all appearances are listed or only the first
                   time; implicitly sets `label=true'
2015/02/26 v1.6  - new `acro-title' instance `none'
                 - change of expl3's tl uppercasing function (adapt to updates
                   of l3kernel and friends
                 - new package option `messages=silent|loud'
                 - fix issue https://bitbucket.org/cgnieder/acro/issue/23/
                 - fix issue https://bitbucket.org/cgnieder/acro/issue/24/
                 - fix issue https://bitbucket.org/cgnieder/acro/issue/28/
                 - drop support for version 0
2015/04/08 v1.6a - more generalized user command definitions, see
                   http://tex.stackexchange.com/q/236362/ for an application
2015/05/10 v1.6b - \ProcessKeysPackageOptions ,
                 - correct bug http://tex.stackexchange.com/q/236860/ :
                   option `pages = first' works again
2015/08/16 v2.0  - fix https://bitbucket.org/cgnieder/acro/issue/36
                 - implement https://bitbucket.org/cgnieder/acro/issue/39
                 - implement https://bitbucket.org/cgnieder/acro/issue/40
                   (=> new option `group-cite-cmd')
                 - add ideas for https://bitbucket.org/cgnieder/acro/issue/41
                 - implement https://bitbucket.org/cgnieder/acro/issue/18
                 - implement https://bitbucket.org/cgnieder/acro/issue/43
                 - further generalization for defining user commands:
                   \NewAcroCommand, \NewPseudoAcroCommand and siblings
                 - bug fix in indefinite versions with first-upper
                 - add `short-<ending>-form' equivalent to
                   `long-<ending>-form'
                   (https://bitbucket.org/cgnieder/acro/issue/44)
                 - implement https://bitbucket.org/cgnieder/acro/issue/35
                 - new option `single-form'
2015/08/25 v2.0a - fix https://bitbucket.org/cgnieder/acro/issue/38 and
                   https://bitbucket.org/cgnieder/acro/issue/49
2015/08/29 v2.0b - fix https://bitbucket.org/cgnieder/acro/issue/44
                 - fix https://bitbucket.org/cgnieder/acro/issue/45
                 - implement https://bitbucket.org/cgnieder/acro/issue/42
2015/09/05 v2.1  - add list object type `list-of' that prints the list like a
                   toc or lof, new option `list-short-width',
                 - correct bug in the `plain' extra style
                 - implemented `tooltip' property
                 - remove \tl_to_lowercase:n
2015/10/03 v2.2  - fix https://bitbucket.org/cgnieder/acro/issue/52
                 - fix typo in `list-of' object
                 - \DeclareAcroListStyle
                 - \DeclareAcroListHeading
                 - input `acro.cfg' if present
                 - all acro commands have an optional argument: \ac*[]{}
2016/01/07 v2.2a - \prop_get:Nn => \prop_item:Nn
2016/01/21 v2.2b - fix issue #59
2016/02/02 v2.2c - fix issue #60
2016/03/14 v2.3  - foreign-format may be a macro taking an argument
                 - \Aca, \Acsingle
                 - properties `single' and `single-format', option
                   `single-format' => issue #62
                 - property `first-style' => issue #61
                 - fix issue #64: long-post entry is now added *after* the
                   endings
                 - property `foreign-lang'
                 - fix issue #65
2016/03/25 v2.4  - extend class mechanism: classes can be used like tags
                 - add idea of `barriers' and list local to those barriers
                   => new option `reset-at-barriers'
                   => new option `local-to-barriers' for \printacronyms
                   => new command \acbarrier
2016/04/14 v2.4a - if undefined acronym is used and `messages = silent' is
                   active don't through error
2016/05/03 v2.4b - expand `pdfstring' property before saving => issue #69
                 - \ProvideAcroEnding can be used twice – it then just sets
                   the defaults
                 - the option <ending>-ending has a new syntax:
                   * <ending>-ending = <val> sets all endings to <val>
                   * <ending>-ending = <val1>/<val2> sets short endings to
                     <val1> and long endings to <val2>
                 - a single appearance should be treated like a first
                   appearance as far as citations are concerned
2016/05/25 v2.5  - some of the entries added to the aux file need to be
                   written \immediate in order to avoid this trap:
                   http://tex.stackexchange.com/q/116001/
                 - cleaner interface for first-style template definitions
                 - new `acro-first' instances `footnote-reversed' and
                   `sidenote-reversed'
                 - new commands \DeclareAcroFirstStyle, \DeclareAcroExtraStyle
                   and \DeclareAcroPageStyle
                 - add warning `ending-before-acronyms' to options setting the
                   defaults of an ending; this should avoid confusion
                 - property declaration for acronyms should be handled by
                   internal functions
                 - improvements in the list template code
                 - logging info when an acronym is declared
                 - remove deprecated options
                 - new option `use-barriers'
                 - new option `following-pages*'
                 - option `page-ref' replaced by option `page-style'
2016/05/26 v2.5a - bug fix: remove erroneous group in `<ending>-ending' option
2016/05/30 v2.5b - fix issue #72
2016/07/20 v2.6  - \l__acro_use_acronyms_bool can be used to disable \ac
                   e.g. in the trial phase of a table like `tabu'; interface:
                   \acro_switch_off: and \acswitchoff
                 - fix issue #79
                 - fix issue #74
                 - fix error: acronyms with same sort entry are not
                   overwritten any more in the list of acronyms
2016/08/13 v2.6a - fix issues #80 and #81
2016/08/13 v2.6b - version stepped by accident
2016/08/16 v2.6c - really fixes issue #81
2016/08/30 v2.6d - fix issue #82
2016/09/04 v2.6e - fix issue in http://tex.stackexchange.com/q/270034/
2017/01/22 v2.7  - rename \acro_get_property:nn into \__acro_get_property:nn
                 - \acro_get_property:nn, \acro_get_property:nnTF,
                   \acro_if_property:nnTF, retrieve property without error if
                   not set
                 - make \__acro_declare_property functions public
                 - \acro_add_action:n (adds code to \acro_get:n)
2017/02/09 v2.7a - adapt to integration of l3sort into l3kernel

% --------------------------------------------------------------------------
% TODO:
- extend option `macros' to also define uppercase macros, possibly as a choice
- add \ACF, \ACFP, \ACL and \ACLP that will print all words of the long form
  capitalized
           % A discussion on generating PDF files.
\end{appendices}         % End of the Appendix Chapters.  ibid on \end{appendix}
%\NeedsTeXFormat{LaTeX2e}
\ProvidesClass{vita}[1996/10/09
                     class file ``vita'' to create Curriculum Vitae]
%%%%%%%%%%%%%%%%%%%%%%%%%%%%%%%%%%%%%%%%%%%%%%%%%%%%%%%%%%%%%%%%%%%%%%%
%%
%% (C) Copyright 1995, Andrej Brodnik, ABrodnik@UWaterloo.CA. All
%% rights reserved.
%%
%% This is a generated file. Permission is granted to to customize the 
%% declarations in this file to serve the needs of your installation. 
%% However, no permission is granted to distribute a modified version of 
%% this file under its original name. 
%% 
%% \CharacterTable
%%  {Upper-case    \A\B\C\D\E\F\G\H\I\J\K\L\M\N\O\P\Q\R\S\T\U\V\W\X\Y\Z
%%   Lower-case    \a\b\c\d\e\f\g\h\i\j\k\l\m\n\o\p\q\r\s\t\u\v\w\x\y\z
%%   Digits        \0\1\2\3\4\5\6\7\8\9
%%   Exclamation   \!     Double quote  \"     Hash (number) \#
%%   Dollar        \$     Percent       \%     Ampersand     \&
%%   Acute accent  \'     Left paren    \(     Right paren   \)
%%   Asterisk      \*     Plus          \+     Comma         \,
%%   Minus         \-     Point         \.     Solidus       \/
%%   Colon         \:     Semicolon     \;     Less than     \<
%%   Equals        \=     Greater than  \>     Question mark \?
%%   Commercial at \@     Left bracket  \[     Backslash     \\
%%   Right bracket \]     Circumflex    \^     Underscore    \_
%%   Grave accent  \`     Left brace    \{     Vertical bar  \|
%%   Right brace   \}     Tilde         \~}
%%
%%---

%%%%%%%%%%%%%%%%%%%%%%%%%%%%%%%%%%%%%%%%%%%%%%%%%%%%%%%%%%%%%%%%%%%%%%%
%
%    - based on vita.sty by kcb@hss.caltech.edu
%    - 1995/02/07: the first version
%    - 1996/10/09: if there is no business address the field is
%                  left out
%
% User documentation: This class file only provides basic definitions
% =================== of environments, which are then used in class
% option files to instantiate entries for different disciplines. Thus,
% create your document as follows:
%
%    \documentclass[<discipline>]{vita}
%    \begin{document}
%      \name{Andrej Brodnik}
%      \businessAddress{First line \\ second line of bussines address}
%      \homeAddress{Again \\ multiline address \\ perhaps with phone number}
%    \begin{vita}
%      % here comes a real Curriculum Vitae for particular <discipline>
%    \end{vita}
%    \end{document}
%
% where it is assumed that file ``vita<discipline>.clo'' exists and defines
% proper categories used in given discipline. For detail explanation on
% categories in different disciplines see individual ``.clo'' files.
%
% The output will have format:
%
%   o on the first page will appear a title ``Curriculum Vitae'' (to
%     change it, see below under i18n notes -- internationalization)
%   o below will be your name
%   o below, side by side, your business and home address headed
%     by strings ``Business address'' and ``Home address''
%     respectively (to change these strings see below in i18n notes).
%   o then will follow the rest of CV as defined by ``<discipline>.clo''
%     file.
%   o the header of each but first page will include your name and the
%     page number.
%   o on the last page in the bottom right you will have the current
%     date, that is month and year (to change this, see below under
%     i18n notes).
%
%------
%
% i18n NOTES: If you are making CV for some other language, you have to
% =========== redefine:
%   - title:
%       o use command:   ``\title{<new title>}''
%       o default value: ``Curriculum Vitae''
%   - date:   
%       o use command:   ``\today{<date})''
%       o default value: ``<current month>, <current year>'' (in English)
%   - addresses headers:
%       o use command:   ``\HeaderBusiness{<new header>}''
%                        ``\HeaderHome{<new header>}''
%       o default value: ``Business address''
%                        ``Home address''
%
%------
%
% System documentation: class ``vita'' is based on the class
% ===================== ``article''. It changes the title into
% <default value> (see i18n notes) and the name becomes an
% author. Individual categories, publications and references are
% implemented using ``description'' environment. 
%
%----------------------------------------

%%%%
%
% Process options and load class article:
%---
\let\@optionsToInput=\@empty
\DeclareOption*{
  \IfFileExists{vita\CurrentOption.clo}%
    {\edef\@optionToInput{vita\CurrentOption.clo}}%
    {\PassOptionsToClass{\CurrentOption}{article}}
}
\ProcessOptions
\LoadClass{article}

%%%%
%
% First all i18n definitions:
%---
\title{Curriculum Vitae}
\renewcommand{\today}{
  \ifcase\month\or
    January\or February\or March\or April\or May\or June\or
    July\or August\or September\or October\or November\or December\fi,
  \space\number\year}
\newcommand\HeaderBusiness[1]{\def\@businessAddressHeader{#1}}
  \HeaderBusiness{Business Address}
\newcommand\HeaderHome[1]{\def\@homeAddressHeader{#1}}
  \HeaderHome{Home Address}

%%%%
%
% Next, header definitions:
%---
\date{\relax}
\newcommand{\name}[1]{
  \renewcommand{\@author}{#1} \markright{\protect\small\@author}
}
\newcommand{\businessAddress}[1]{\def\@businessAddress{#1}}
  \businessAddress{}
\newcommand{\homeAddress}[1]{\def\@homeAddress{#1}}
  \homeAddress{}

%%%%
%
% \maketitle command, which prints out the title and the name of person
%---
\renewcommand{\maketitle}{\newpage
  \global\@topnum\z@   % Prevents figures from going at top of page.
  \begin{center}
    {\LARGE \@title}

    \medskip

    {\large \@author}
  \end{center}

  \bigskip

  \thispagestyle{plain}

  \gdef\@author{}\gdef\@title{}
}

%%%%
%
% ``vita'' environment:
%---
\pagestyle{empty}
\newenvironment{vita}{
     % first page is empty style though the following pages have on the
     % right side written the name from the \name command
  \ifx\@author\@empty\@warning{Missing name command}\fi
     % next we start to layout information. First the title and the
     % name,

  \maketitle
     % followed by both addresses,
  \begin{tabular*}{\textwidth}{@{\extracolsep{\fill}}ll@{}}
    \begin{tabular}[t]{@{}l@{}}
    \ifx\@businessAddress\@empty\mbox{}\else
       {\small \@businessAddressHeader:}
    \\ \@businessAddress
    \fi
    \end{tabular}
  &
    \ifx\@homeAddress\@empty\@warning{Missing home address}%
    \else
      \begin{tabular}[t]{@{}l@{}}
         {\small \@homeAddressHeader:}
      \\ \@homeAddress
      \end{tabular}
    \fi
  \end{tabular*}

  \bigskip

  \thispagestyle{empty}
}{   % quite at the bottom of last page we have a date
  \par\nopagebreak\vfill\hfill \today
}%end vita environment

%%%%
%
% Curriculum vitae consists of categories which we create using
% command:
%
%      \newcategory[The name]{The label}
%
% where <The name> is written in bold character as a small title of
% category. It appears at the left margine of a page. If <The name>
% parameter is missing, it takes the same value as <The label>, which,
% in turn is used to refer to individual category. For example
% commands:
%
%      \newcategory{Name of category}
%      \newcategory[Name of category]{Name of category}
%
% have the same result. Now, to use category:
%
%      \newcategory[Some category]{some other name}
%
% the input has form:
%
% \begin{some other name}
%   \item The first item
%   \item The second one etc.
% \end{some other name}
%
% and the category will have on the output title ``Some category''.
% Entries in each category are preceded by \item.
%
%-----
% i18n NOTE: One can use as the names of categories strings in
% ========== different languages, but the labels can be the same in
% the same language, which is useful if you have a single CV and you
% want outputs in different languages.
%---
\def\@newCategory[#1]#2{%
  \newenvironment{#2}{\medskip\pagebreak[2]\par
    \textbf{\small #1}\nopagebreak
    \begin{description}}{\end{description}\par}
}
\def\@noNameCategory#1{\@newCategory[#1]{#1}}
\def\newcategory{\@ifnextchar[{\@newCategory}{\@noNameCategory}}

%%%%
%
% Inside categories we have different ``kinds'' (such as different
% publications), which we create using command \newkind. It has the
% same parameters as \newcategory and all comments at command
% newcategory are also valid here.
%---
\def\@newKind[#1]#2{%
  \newenvironment{#2}{
    \pagebreak[2]
    \item \textbf{\small #1}\nopagebreak
      \begin{description}
  }{  \end{description}\par }
}
\def\@noNameKind#1{\@newKind[#1]{#1}}
\def\newkind{\@ifnextchar[{\@newKind}{\@noNameKind}}

%%%%
%
% There is a special category ``plaincategory'' which entries are
% simply listed without any indentation, and in particular, multiple
% references are separated by \and command. It can be used for
% references.
%---
\def\@newPlainCategory[#1]#2{%
  \newenvironment{#2}{
    \medskip\pagebreak[2]\par
    \textbf{\small #1}\nopagebreak
    \renewcommand{\and}{
             \end{tabular}
      \item[]\begin{tabular}[t]{l}
    }
    \begin{description}
    \item[] \begin{tabular}[t]{l}
  }{        \end{tabular}
    \end{description}\par
  }
}
\def\@noNamePlainCategory#1{\@newPlainCategory[#1]{#1}}
\def\newplaincategory{\@ifnextchar[{\@newPlainCategory}{\@noNamePlainCategory}}

%%%%
%
% Finally, formatting parameters and the possible option to input:
%---
\pagestyle{myheadings}
\parindent 0pt
\nofiles

\ifx\@optionToInput\@empty\relax
\else \input \@optionToInput
\fi
          % Optional Vita, use \begin{vita} vita text \end{vita}
\end{document}
\end{verbatim}
\end{quote}

\section{Prelude}
After the {\tt \verb|\begin{document}|} comes the preliminary information found in
theses.  In this manual, the information is kept in the file {\tt prelude.tex} (see
above).  These pages will need to be numbered with roman numerals, so use
\begin{quote}\tt\singlespace\begin{verbatim}
\clearpage\pagenumbering{roman}
\end{verbatim}\end{quote}

Next, comes your thesis or dissertation title, your name, date of graduation, department
and degree.
\begin{quote}\tt\singlespace\begin{verbatim}
\title{How to \LaTeX\ a Thesis}
\author{Eric R. Benedict}
\date{2000}
%   - The default degree is ``Doctor of Philosophy''
%     Degree can be changed using the command \degree{}
%\degree{New Degree}
%   - for a PhD dissertation (default), specify \dissertation
%\dissertation
%   - for a masters project report, specify \project
%\project
%   - for a preliminary report, specify \prelim
%\prelim
%   - for a masters thesis, specify \thesis
%\thesis
%   - The default department is ``Electrical Engineering''
%     The department can be changed using the command \department{}
%\department{New Department}
\end{verbatim}\end{quote}

If you specified the class option {\tt msthesis}, then the degree is changed to
{\em Master \break of Science} and the {\tt \verb|\thesis|} option is specified.  If you
want to have the masters margins with another document, then the {\tt \verb|\degree|}
and {\tt \verb|\dissertation|},  {\tt \verb|\project|}, {\em etc.\/} can be specified
as needed.

Once the
above are all defined, use  {\tt \verb|\maketitle|} to generate the title page.
\begin{quote}\tt\singlespace\begin{verbatim}
\maketitle
\end{verbatim}\end{quote}

If you wish to include a copyright page (see Section~\ref{copyright} for
information on registering the copyright.), then add the command
\begin{quote}\tt\singlespace\begin{verbatim}
\copyrightpage
\end{verbatim}\end{quote}
This will generate the proper copyright page and will use the name and date specified
in {\tt \verb|\author{}|} and {\tt \verb|\date{}|}.

Next are the dedications and acknowledgements:
\begin{quote}\tt\singlespace\begin{verbatim}
\begin{dedication}
To my pet rock, Skippy.
\end{dedication}

\begin{acknowledgments}
I thank the many people who have done lots of nice things for me.
\end{acknowledgments}
\end{verbatim}\end{quote}

You must tell \LaTeX{} to generate a table of contents, a list of tables and a list of
figures:
\begin{quote}\tt\singlespace\begin{verbatim}
\tableofcontents
\listoftables
\listoffigures
\end{verbatim}\end{quote}

If you wish to have a nomenclature, list of symbols or glossary it can go here.
\begin{quote}\tt\singlespace\begin{verbatim}
\begin{nomenclature}
%\begin{listofsymbols}
%\begin{glossary}
\begin{tabular}{ll}
$C_1$ & Constant 1\\
\ldots
\end{tabular}
%\end{glossary}
%\end{listofsymbols}
\end{nomenclature}
\end{verbatim}\end{quote}

If your abstract will be microfilmed by Bell and Howell (formerly UMI), then you
will need to generate an abstract of less than 350 words.  This abstract can be created
using the {\tt umiabstract} environment.  This environment requires that you define your
advisor and your advisor's title using {\tt \verb|\advisorname{}|} and
{\tt \verb|\advisortitle{}|}.
\begin{quote}\tt\singlespace\begin{verbatim}
\advisorname{Bucky J. Badger}
\advisortitle{Assistant Professor}
% ABSTRACT
\begin{umiabstract}
\noindent       % Don't indent first paragraph.
This explains the basics for using \LaTeX\ to typeset a
dissertation, thesis or project report for the University of
Wisconsin-Madison.

...

\end{umiabstract}
\end{verbatim}\end{quote}
This will place your name, title and required text at the top of the page and follow the
abstract text with your advisor's name at the bottom for your advisor's signature.  This
page is not numbered and would be submitted separately.

If you will have an abstract as part of your document, then the {\tt abstract} environment
should be used.
\begin{quote}\tt\singlespace\begin{verbatim}
\begin{abstract}
\noindent       % Don't indent first paragraph.
This explains the basics for using \LaTeX\ to typeset a
dissertation, thesis or project report for the University of
Wisconsin-Madison.

...

\end{abstract}
\end{verbatim}\end{quote}
This will generate a page number and it will be included in the Table
of Contents.  

If you will have both the UMI and regular abstracts like this document, then
you will probably want to write the abstract once and save it in a seperate
file such as {\tt abstract.tex}.  Then, you can use the same abstract for
both purposes.

\begin{quote}\begin{verbatim}
\begin{umiabstract}
  % !TEX root = main.tex
% !TEX encoding = Windows Latin 1
% !TEX TS-program = pdflatex
% 
% Archivo: abstract.tex (en ingles)


\chapter{Abstract} % No cambiar el titulo
\selectlanguage{english}
\noindent
Duis tristique sollicitudin leo nec consequat. Praesent et dui convallis velit tincidunt fermentum. Mauris cursus purus at sem viverra sed imperdiet sapien imperdiet. Aliquam mattis, elit eget rutrum vulputate, tortor sem pulvinar justo, sit amet mollis felis sem at nibh. Donec malesuada, neque id interdum eleifend, arcu augue porta elit, nec tristique libero metus at massa. Fusce fringilla laoreet rhoncus. Suspendisse potenti. Phasellus dignissim sodales mauris at pharetra. Donec gravida fringilla velit ac rutrum.

Curabitur ornare lectus id diam molestie eu imperdiet nulla tempus. Maecenas vestibulum enim et dui ornare blandit. Vivamus fermentum faucibus viverra. Maecenas at justo sapien. Aenean rhoncus augue mattis purus rhoncus venenatis. Suspendisse metus felis, porttitor in varius in, vulputate at tortor. Aliquam molestie, turpis et malesuada porta, tortor sapien pharetra sapien, ac rhoncus quam dolor a sapien. Pellentesque varius laoreet enim ut auctor. Nullam nec ultricies nisi. Nullam porta lectus et ante consectetur posuere.

Duis tristique sollicitudin leo nec consequat. Praesent et dui convallis velit tincidunt fermentum. Mauris cursus purus at sem viverra sed imperdiet sapien imperdiet. Aliquam mattis, elit eget rutrum vulputate, tortor sem pulvinar justo, sit amet mollis felis sem at nibh. Donec malesuada, neque id interdum eleifend, arcu augue porta elit, nec tristique libero metus at massa. Fusce fringilla laoreet rhoncus. Suspendisse potenti. Phasellus dignissim sodales mauris at pharetra. Donec gravida fringilla velit ac rutrum.

Duis tristique sollicitudin leo nec consequat. Praesent et dui convallis velit tincidunt fermentum. Mauris cursus purus at sem viverra sed imperdiet sapien imperdiet. Aliquam mattis, elit eget rutrum vulputate, tortor sem pulvinar justo, sit amet mollis felis sem at nibh. Donec malesuada, neque id interdum eleifend, arcu augue porta elit, nec tristique libero metus at massa. Fusce fringilla laoreet rhoncus. Suspendisse potenti. Phasellus dignissim sodales mauris at pharetra. Donec gravida fringilla velit ac rutrum.

Curabitur ornare lectus id diam molestie eu imperdiet nulla tempus. Maecenas vestibulum enim et dui ornare blandit. Vivamus fermentum faucibus viverra. Maecenas at justo sapien. Aenean rhoncus augue mattis purus rhoncus venenatis. Suspendisse metus felis, porttitor in varius in, vulputate at tortor. Aliquam molestie, turpis et malesuada porta, tortor sapien pharetra sapien, ac rhoncus quam dolor a sapien. Pellentesque varius laoreet enim ut auctor. Nullam nec ultricies nisi. Nullam porta lectus et ante consectetur posuere.

Duis tristique sollicitudin leo nec consequat. Praesent et dui convallis velit tincidunt fermentum. Mauris cursus purus at sem viverra sed imperdiet sapien imperdiet. Aliquam mattis, elit eget rutrum vulputate, tortor sem pulvinar justo, sit amet mollis felis sem at nibh. Donec malesuada, neque id interdum eleifend, arcu augue porta elit, nec tristique libero metus at massa. Fusce fringilla laoreet rhoncus. Suspendisse potenti. Phasellus dignissim sodales mauris at pharetra. Donec gravida fringilla velit ac rutrum.

\bigskip
\noindent
\textit{Key words:} first word; second word; third word.
% Separar palabras con punto-y-comas.

\checklanguage
% Fin archivo abstract.tex
\endinput 
\end{umiabstract}

\begin{abstract}
  % !TEX root = main.tex
% !TEX encoding = Windows Latin 1
% !TEX TS-program = pdflatex
% 
% Archivo: abstract.tex (en ingles)


\chapter{Abstract} % No cambiar el titulo
\selectlanguage{english}
\noindent
Duis tristique sollicitudin leo nec consequat. Praesent et dui convallis velit tincidunt fermentum. Mauris cursus purus at sem viverra sed imperdiet sapien imperdiet. Aliquam mattis, elit eget rutrum vulputate, tortor sem pulvinar justo, sit amet mollis felis sem at nibh. Donec malesuada, neque id interdum eleifend, arcu augue porta elit, nec tristique libero metus at massa. Fusce fringilla laoreet rhoncus. Suspendisse potenti. Phasellus dignissim sodales mauris at pharetra. Donec gravida fringilla velit ac rutrum.

Curabitur ornare lectus id diam molestie eu imperdiet nulla tempus. Maecenas vestibulum enim et dui ornare blandit. Vivamus fermentum faucibus viverra. Maecenas at justo sapien. Aenean rhoncus augue mattis purus rhoncus venenatis. Suspendisse metus felis, porttitor in varius in, vulputate at tortor. Aliquam molestie, turpis et malesuada porta, tortor sapien pharetra sapien, ac rhoncus quam dolor a sapien. Pellentesque varius laoreet enim ut auctor. Nullam nec ultricies nisi. Nullam porta lectus et ante consectetur posuere.

Duis tristique sollicitudin leo nec consequat. Praesent et dui convallis velit tincidunt fermentum. Mauris cursus purus at sem viverra sed imperdiet sapien imperdiet. Aliquam mattis, elit eget rutrum vulputate, tortor sem pulvinar justo, sit amet mollis felis sem at nibh. Donec malesuada, neque id interdum eleifend, arcu augue porta elit, nec tristique libero metus at massa. Fusce fringilla laoreet rhoncus. Suspendisse potenti. Phasellus dignissim sodales mauris at pharetra. Donec gravida fringilla velit ac rutrum.

Duis tristique sollicitudin leo nec consequat. Praesent et dui convallis velit tincidunt fermentum. Mauris cursus purus at sem viverra sed imperdiet sapien imperdiet. Aliquam mattis, elit eget rutrum vulputate, tortor sem pulvinar justo, sit amet mollis felis sem at nibh. Donec malesuada, neque id interdum eleifend, arcu augue porta elit, nec tristique libero metus at massa. Fusce fringilla laoreet rhoncus. Suspendisse potenti. Phasellus dignissim sodales mauris at pharetra. Donec gravida fringilla velit ac rutrum.

Curabitur ornare lectus id diam molestie eu imperdiet nulla tempus. Maecenas vestibulum enim et dui ornare blandit. Vivamus fermentum faucibus viverra. Maecenas at justo sapien. Aenean rhoncus augue mattis purus rhoncus venenatis. Suspendisse metus felis, porttitor in varius in, vulputate at tortor. Aliquam molestie, turpis et malesuada porta, tortor sapien pharetra sapien, ac rhoncus quam dolor a sapien. Pellentesque varius laoreet enim ut auctor. Nullam nec ultricies nisi. Nullam porta lectus et ante consectetur posuere.

Duis tristique sollicitudin leo nec consequat. Praesent et dui convallis velit tincidunt fermentum. Mauris cursus purus at sem viverra sed imperdiet sapien imperdiet. Aliquam mattis, elit eget rutrum vulputate, tortor sem pulvinar justo, sit amet mollis felis sem at nibh. Donec malesuada, neque id interdum eleifend, arcu augue porta elit, nec tristique libero metus at massa. Fusce fringilla laoreet rhoncus. Suspendisse potenti. Phasellus dignissim sodales mauris at pharetra. Donec gravida fringilla velit ac rutrum.

\bigskip
\noindent
\textit{Key words:} first word; second word; third word.
% Separar palabras con punto-y-comas.

\checklanguage
% Fin archivo abstract.tex
\endinput 
\end{abstract}
\end{verbatim}\end{quote}

Finally, the page numbers must be changed to arabic numbers to conclude the preliminary
portion of the document.
\begin{quote}\tt\singlespace\begin{verbatim}
\clearpage\pagenumbering{arabic}
\end{verbatim}\end{quote}

\section{The Body}
At the beginning of {\tt intro.tex} there is the following command:
\begin{quote}\tt\singlespace\begin{verbatim}
\chapter{Introducing the {\tt withesis} \LaTeX{} Style Guide}
\end{verbatim}\end{quote}
Following that is the text of the chapter.  The body of your thesis is separated by
sectioning commands like {\tt \verb|\chapter{}|}.  For more information on the sectioning
commands, see Section~\ref{ess:sectioning}.

Remember the basic rule of outlining you learned in grammar school:
\begin{quote}
You cannot have an `A' if you do not have a `B'
\end{quote}
Take care to have at least two {\tt \verb|\section|}s if you use the command; have
two {\tt \verb|\subsection|}s, {\em etc}.



\section{Additional Theorem Like Environments}
The {\tt withesis} style adds numerous additional theorem like environments.  These
environments were included to allow compatibility with the University of Wisconsin's
Math Department's style file.  These environments are
{\tt theorem}, {\tt assertion}, {\tt claim}, {\tt conjecture}, {\tt corollary},
{\tt definition}, {\tt example}, {\tt figger}, {\tt lemma}, {\tt prop} and {\tt remark}.

As an example, consider the following.
\begin{lemma}
Assuming that $\partial\Omega_2 = \emptyset$ and that $h(t) = 1$, we
have $$
\begin{array}{lr}
\Delta u = f, &  x\in\Omega ,\\[2pt]
u =  g_1, &  x\in\partial\Omega .
\end{array}
$$
\end{lemma}
which was produced with the following:
\begin{quote}\tt\singlespace\begin{verbatim}
\begin{lemma}
Assuming that $\partial\Omega_2 = \emptyset$ and that $h(t) = 1$, we
have $$
\begin{array}{lr}
\Delta u = f, &  x\in\Omega ,\\[2pt]
 u =  g_1, &  x\in\partial\Omega .
\end{array}
$$
\end{lemma}
\end{verbatim}\end{quote}

\section{Bibliography or References}
As a final note, the default title for the references chapter is ``LIST OF REFERENCES.''
Since some people may prefer ``BIBLIOGRAPHY'', the command
\break{\tt \verb|\altbibtitle|}
has been added to change the chapter title.

\section{Appendices}
There are two commands which are available to suppress the writing of the auxiliary information
(to the {\tt .lot} and {\tt .lof} files).  They are:
\begin{quote}\tt\singlespace\begin{verbatim}
\noappendixtables                % Don't have appendix tables
\noappendixfigures               % Don't have appendix figures
\end{verbatim}\end{quote}
These commands should be in the preamble.  See Section~\ref{usage:noapp}.

There are two environments for doing the appendix chapter: {\tt appendix} and
\break {\tt appendices}.  If you have only one chapter in the appendix, use the {\tt appendix}
environment.  If you have more than one chapter, like this manual, use the
{\tt appendices} environment.
\begin{quote}\tt\singlespace\footnotesize\begin{verbatim}
\begin{appendices}  % Start of the Appendix Chapters.  If there is only
                    % one Appendix Chapter, then use \begin{appendix}
% code.tex
% this file is part of the example UW-Madison Thesis document
% It demonstrates one method for incorporating program listings
% into a document.

\chapter{Matlab Code} \label{matlab}
This is an example of a Matlab m-file.
\verbatimfile{derivs.m}
      % Including computer code listings
\chapter{Bib\TeX\ Entries}
\label{bibrefs}
The following shows the fields required in all types of Bib\TeX\ entries.
Fields with {\tt OPT} prefixed are optional (the three letters {\tt OPT} should 
not be used).  If an optional field is not used, then the entire field can be deleted.

{\tt
\singlespace
\begin{verbatim}

@Unpublished{,                            @Manual{,
  author =      "",                         title =           "",
  title =       "",                         OPTauthor =       "",
  note =        "",                         OPTorganization = "",
  OPTyear =     "",                         OPTaddress =      "",
  OPTmonth =    ""                          OPTedition =      "",
}                                           OPTyear =         "",
                                            OPTmonth =        "",
@TechReport{,                               OPTnote =         "" 
  author =      "",                       }
  title =       "",
  institution = "",                       @InProceedings{,
  year =        "",                         author =          "",
  OPTtype =     "",                         title =           "",
  OPTnumber =   "",                         booktitle =       "",
  OPTaddress =  "",                         year =            "",
  OPTmonth =    "",                         OPTeditor =       "",
  OPTnote =     ""                          OPTpages =        "",
}                                           OPTorganization = "",
                                            OPTpublisher =    "",
@Proceedings{,                              OPTaddress =      "",
  title =           "",                     OPTmonth =        "",
  year =            "",                     OPTnote =         "" 
  OPTeditor =       "",                   }
  OPTpublisher =    "",
  OPTorganization = "",
  OPTaddress =      "",
  OPTmonth =        "",
  OPTnote =         "" 
}



@PhDThesis{,                              @InCollection{,
  author =      "",                         author =          "",
  title =       "",                         title =           "",
  school =      "",                         booktitle =       "",
  year =        "",                         publisher =       "",
  OPTaddress =  "",                         year =            "",
  OPTmonth =    "",                         OPTeditor =       "",
  OPTnote =     ""                          OPTchapter =      "",
}                                           OPTpages =        "",
                                            OPTaddress =      "",
                                            OPTmonth =        "",
                                            OPTnote =         ""
                                          }

 
@Misc{,                                   @InCollection{,
  OPTauthor =       "",                     author =          "",
  OPTtitle =        "",                     title =           "",
  OPThowpublished = "",                     chapter =         "",
  OPTyear =         "",                     publisher =       "",
  OPTmonth =        "",                     year =            "",
  OPTnote =         ""                      OPTeditor =       "",
}                                           OPTpages =        "",
}                                           OPTvolume =       "",
                                            OPTseries =       "",
                                            OPTaddress =      "",
                                            OPTedition =      "",
                                            OPTmonth =        "",
                                            OPTnote =         ""
                                          }

@MastersThesis{,                          @Article{,
  author =      "",                         author =          "",
  title =       "",                         title =           "",
  school =      "",                         journal =         "",
  year =        "",                         year =            "",
  OPTaddress =  "",                         OPTvolume =       "",
  OPTmonth =    "",                         OPTnumber =       "",
  OPTnote =     ""                          OPTpages =        "",
}                                           OPTmonth =        "",
                                            OPTnote =         ""
                                           }\end{verbatim} }
    % a BibTeX reference
\chapter{Mathematics Examples}
This appendix provides an example of \LaTeX's typesetting
capabilities.  Most of text was obtained from the University of
Wisconsin-Madison Math Department's example thesis file.

\section{Matrices}
The equations for the {\em dq}-model of an induction machine in the
synchronous reference frame are
\begin{eqnarray}
 \left[\begin{array}{c} v_{qs}^e\\v_{ds}^e\\v_{qr}^e\\v_{dr}^e  \end{array}\right]                                                                                                                                                                                                                                                                                                                                                                                                                                                                                                              
 &=& \left[ \begin{array}{cccc}
 r_s + x_s\frac{\rho}{\omega_b} & \frac{\omega_e}{\omega_b}x_s & x_m\frac{\rho}{\omega_b} & \frac{\omega_e}{\omega_b}x_m \\
 -\frac{\omega_e}{\omega_b}x_s & r_s + x_s\frac{\rho}{\omega_b} & -\frac{\omega_e}{\omega_b}x_m & x_m\frac{\rho}{\omega_b} \\
 x_m\frac{\rho}{\omega_b} & \frac{\omega_e -\omega_r}{\omega_b}x_m & r_r'+x_r'\frac{\rho}{\omega_b} & \frac{\omega_e - \omega_r}{\omega_b}x_r' \\
 -\frac{\omega_e -\omega_r}{\omega_b}x_m & x_m\frac{\rho}{\omega_b} & -\frac{\omega_e - \omega_r}{\omega_b}x_r' & r_r' + x_r'\frac{\rho}{\omega_b}
 \end{array} \right]
 \left[\begin{array}{c} i_{qs}^e\\i_{ds}^e\\i_{qr}^e\\i_{dr}^e\end{array} \right] \label{volteq}\\
 T_e&=&\frac{3}{2}\frac{P}{2}\frac{x_m}{\omega_b}\left(i_{qs}^ei_{dr}^e - i_{ds}^ei_{qr}^e\right) \label{torqueeq}\\
 T_e-T_l&=&\frac{2J\omega_b}{P}\frac{d}{dt}\left(\frac{\omega_r}{\omega_b}\right) \label{mecheq}.
\end{eqnarray}

\section{Multi-line Equations}

\LaTeX{} has a built-in equation array feature, however the
equation numbers must be on the same line as an equation.  For example:
\begin{eqnarray}
\Delta u + \lambda e^u &= 0&u\in \Omega,  \nonumber \\
u&=0&u\in\partial\Omega.
\end{eqnarray}

Alternatively, the number can be centered in the equation using the
following method.
%
% The equation-array feature in LaTeX is a bad idea.  For centered
% numbers you should set your own equations and arrays as follows:
%
\def\dd{\displaystyle}
\begin{equation}\label{gelfand}
\begin{array}{rl}
\dd \Delta u + \lambda e^u = 0, &
\dd u\in \Omega,\\[8pt] % add 8pt extra vertical space. 1 line=23pt
\dd u=0, & \dd u\in\partial\Omega.
\end{array}
\end{equation}
The previous equation had a label.  It may be referenced as
equation~(\ref{gelfand}).


\section{More Complicated Equations}
\section*{Rellich's identity}\label{rellich.section}
\setcounter{theorem}{0}
%
%

Standard developments of Pohozaev's identity used an identity by
Rellich~\cite{rellich:der40}, reproduced here.

\begin{lemma}[Rellich]
Given $L$ in divergence form and $a,d$ defined above, $u\in C^2
(\Omega )$, we have
\begin{equation}\label{rellich}
\int_{\Omega}(-Lu)\nabla u\cdot (x-\overline{x})\, dx
= (1-\frac{n}{2}) \int_{\Omega} a(\nabla u,\nabla u) \, dx
-
\frac{1}{2} \int_{\Omega}
d(\nabla u, \nabla u) \, dx
\end{equation}
$$
+
\frac{1}{2} \int_{\partial\Omega} a(\nabla u,\nabla u)(x-\overline{x})
\cdot \nu  \, dS
-
\int_{\partial\Omega}
a(\nabla u,\nu )\nabla u\cdot (x-\overline{x}) \, dS.
$$
\end{lemma}
{\bf Proof:}\\
There is no loss in generality to take $\overline{x} = 0$. First
rewrite $L$:
$$Lu = \frac{1}{2}\left[ \sum_{i}\sum_{j}
\frac{\partial}{\partial x_i}
\left( a_{ij} \frac{\partial u}{\partial x_j} \right) +
\sum_{i}\sum_{j}
\frac{\partial}{\partial x_i}
\left( a_{ij} \frac{\partial u}{\partial x_j} \right)
\right]$$
Switching the order of summation on the second term and relabeling
subscripts, $j \rightarrow i$ and $i \rightarrow j$, then using the fact
that $a_{ij}(x)$ is a symmetric matrix,
gives the symmetric form needed to derive Rellich's identity.
\begin{equation}
Lu = \frac{1}{2} \sum_{i,j}\left[
\frac{\partial}{\partial x_i}
\left( a_{ij} \frac{\partial u}{\partial x_j} \right) +
\frac{\partial}{\partial x_j}
\left( a_{ij} \frac{\partial u}{\partial x_i} \right)
\right].
\end{equation}

Multiplying $-Lu$ by $\frac{\partial u}{\partial x_k} x_k$ and integrating
over $\Omega$, yields
$$\int_{\Omega}(-Lu)\frac{\partial u}{\partial x_k} x_k \, dx=
-\frac{1}{2} \int_{\Omega}
\sum_{i,j}\left[
\frac{\partial}{\partial x_i}
\left( a_{ij} \frac{\partial u}{\partial x_j} \right) +
\frac{\partial}{\partial x_j}
\left( a_{ij} \frac{\partial u}{\partial x_i} \right)
\right]
\frac{\partial u}{\partial x_k} x_k \, dx$$
Integrating by parts (for integral theorems see~\cite[p. 20]
{zeidler:nfa88IIa})
gives
$$= \frac{1}{2} \int_{\Omega}
\sum_{i,j} a_{ij} \left[
\frac{\partial u}{\partial x_j}
\frac{\partial^2 u}{\partial x_k\partial x_i} +
\frac{\partial u}{\partial x_i}
\frac{\partial^2 u}{\partial x_k\partial x_j}
\right] x_k \, dx
$$
$$
+
\frac{1}{2} \int_{\Omega}
\sum_{i,j} a_{ij} \left[
\frac{\partial u}{\partial x_j} \delta_{ik} +
\frac{\partial u}{\partial x_i} \delta_{jk}
\right] \frac{\partial u}{\partial x_k} \, dx
$$
$$- \frac{1}{2} \int_{\partial\Omega}
\sum_{i,j} a_{ij} \left[
\frac{\partial u}{\partial x_j} \nu_{i} +
\frac{\partial u}{\partial x_i} \nu_{j}
\right] \frac{\partial u}{\partial x_k} x_k \, dx
$$
= $I_1 + I_2 + I_3$, where the unit normal vector is $\nu$.
One may rewrite $I_1$ as
$$I_1 = \frac{1}{2} \int_{\Omega}
\sum_{i,j} a_{ij} \frac{\partial}{\partial x_k}\left(
\frac{\partial u}{\partial x_i}
\frac{\partial u}{\partial x_j}
\right) x_k \, dx
$$
Integrating the first term by parts again yields
$$I_1 = -\frac{1}{2} \int_{\Omega}
\sum_{i,j} a_{ij} \left(
\frac{\partial u}{\partial x_i}
\frac{\partial u}{\partial x_j}
\right) \, dx
+
\frac{1}{2} \int_{\partial\Omega}
\sum_{i,j} a_{ij} \left(
\frac{\partial u}{\partial x_i}
\frac{\partial u}{\partial x_j}
\right) x_k \nu_k \, dS
$$
$$
-
\frac{1}{2} \int_{\Omega}
\sum_{i,j} \left(
\frac{\partial u}{\partial x_i}
\frac{\partial u}{\partial x_j}
\right) x_k \frac{\partial a_{ij}}{\partial x_k}\, dx.
$$
Summing over $k$ gives
$$\int_{\Omega}(-Lu)(\nabla u\cdot x)\, dx =
-\frac{n}{2} \int_{\Omega}
\sum_{i,j} a_{ij} \left(
\frac{\partial u}{\partial x_i}
\frac{\partial u}{\partial x_j}
\right) \, dx
$$
$$
+
\frac{1}{2} \int_{\partial\Omega}
\sum_{i,j} a_{ij} \left(
\frac{\partial u}{\partial x_i}
\frac{\partial u}{\partial x_j}
\right) (x\cdot \nu ) \, dS
-
\frac{1}{2} \int_{\Omega}
\sum_{i,j} \left(
\frac{\partial u}{\partial x_i}
\frac{\partial u}{\partial x_j}
\right) (x\cdot  \nabla a_{ij}) \, dx
$$
$$
+
\frac{1}{2} \int_{\Omega}
\sum_{i,j,k} a_{ij} \left[
\frac{\partial u}{\partial x_j}
\frac{\partial u}{\partial x_k} \delta_{ik} +
\frac{\partial u}{\partial x_i}
\frac{\partial u}{\partial x_k} \delta_{jk}
\right] \, dx
$$
$$- \frac{1}{2} \int_{\partial\Omega}
\sum_{i,j} a_{ij} \left[
\frac{\partial u}{\partial x_j} \nu_{i} +
\frac{\partial u}{\partial x_i} \nu_{j}
\right] (\nabla u\cdot x) \, dS.
$$
Combining the first and fourth term on the right-hand side
simplifies the expression
$$\int_{\Omega}(-Lu)(\nabla u\cdot x)\, dx
=
(1-\frac{n}{2}) \int_{\Omega}
\sum_{i,j} a_{ij} \left(
\frac{\partial u}{\partial x_i}
\frac{\partial u}{\partial x_j}
\right) \, dx
$$
$$
+
\frac{1}{2} \int_{\partial\Omega}
\sum_{i,j} a_{ij} \left(
\frac{\partial u}{\partial x_i}
\frac{\partial u}{\partial x_j}
\right) (x\cdot \nu ) \, dS
-
\frac{1}{2} \int_{\Omega}
\sum_{i,j} \left(
\frac{\partial u}{\partial x_i}
\frac{\partial u}{\partial x_j}
\right) (x\cdot  \nabla a_{ij}) \, dx
$$
$$
-
\frac{1}{2} \int_{\partial\Omega}
\sum_{i,j} a_{ij} \left[
\frac{\partial u}{\partial x_j} \nu_{i} +
\frac{\partial u}{\partial x_i} \nu_{j}
\right] (\nabla u\cdot x) \, dS.
$$
Using the notation defined above, the result follows.


      % Complex Equations from the UW Math Department

\end{appendices}    % End of the Appendix Chapters. ibid on \end{appendix}
\end{verbatim}\end{quote}
The difference between these two environments is the way that the chapter header is
created and how this is listed in the table of contents.
