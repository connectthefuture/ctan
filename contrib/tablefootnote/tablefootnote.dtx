% \iffalse meta-comment
%
% File: tablefootnote.dtx
% Version: 2014/01/26 v1.1c
%
% Copyright (C) 2011 - 2014 by
%    H.-Martin M"unch <Martin dot Muench at Uni-Bonn dot de>
%
% This work may be distributed and/or modified under the
% conditions of the LaTeX Project Public License, either
% version 1.3c of this license or (at your option) any later
% version. This version of this license is in
%    http://www.latex-project.org/lppl/lppl-1-3c.txt
% and the latest version of this license is in
%    http://www.latex-project.org/lppl.txt
% and version 1.3c or later is part of all distributions of
% LaTeX version 2005/12/01 or later.
%
% This work has the LPPL maintenance status "maintained".
%
% The Current Maintainer of this work is H.-Martin Muench.
%
% This work consists of the main source file tablefootnote.dtx,
% the README, and the derived files
%    tablefootnote.sty, tablefootnote.pdf,
%    tablefootnote.ins, tablefootnote.drv,
%    tablefootnote-example.tex, tablefootnote-example.pdf.
%
% Distribution:
%    http://mirror.ctan.org/macros/latex/contrib/tablefootnote/tablefootnote.dtx
%    http://mirror.ctan.org/macros/latex/contrib/tablefootnote/tablefootnote.pdf
%    http://mirror.ctan.org/install/macros/latex/contrib/tablefootnote.tds.zip
%
% See http://www.ctan.org/pkg/tablefootnote (Catalogue)
%
% Unpacking:
%    (a) If tablefootnote.ins is present:
%           tex tablefootnote.ins
%    (b) Without tablefootnote.ins:
%           tex tablefootnote.dtx
%    (c) If you insist on using LaTeX
%           latex \let\install=y% \iffalse meta-comment
%
% File: tablefootnote.dtx
% Version: 2014/01/26 v1.1c
%
% Copyright (C) 2011 - 2014 by
%    H.-Martin M"unch <Martin dot Muench at Uni-Bonn dot de>
%
% This work may be distributed and/or modified under the
% conditions of the LaTeX Project Public License, either
% version 1.3c of this license or (at your option) any later
% version. This version of this license is in
%    http://www.latex-project.org/lppl/lppl-1-3c.txt
% and the latest version of this license is in
%    http://www.latex-project.org/lppl.txt
% and version 1.3c or later is part of all distributions of
% LaTeX version 2005/12/01 or later.
%
% This work has the LPPL maintenance status "maintained".
%
% The Current Maintainer of this work is H.-Martin Muench.
%
% This work consists of the main source file tablefootnote.dtx,
% the README, and the derived files
%    tablefootnote.sty, tablefootnote.pdf,
%    tablefootnote.ins, tablefootnote.drv,
%    tablefootnote-example.tex, tablefootnote-example.pdf.
%
% Distribution:
%    http://mirror.ctan.org/macros/latex/contrib/tablefootnote/tablefootnote.dtx
%    http://mirror.ctan.org/macros/latex/contrib/tablefootnote/tablefootnote.pdf
%    http://mirror.ctan.org/install/macros/latex/contrib/tablefootnote.tds.zip
%
% See http://www.ctan.org/pkg/tablefootnote (Catalogue)
%
% Unpacking:
%    (a) If tablefootnote.ins is present:
%           tex tablefootnote.ins
%    (b) Without tablefootnote.ins:
%           tex tablefootnote.dtx
%    (c) If you insist on using LaTeX
%           latex \let\install=y% \iffalse meta-comment
%
% File: tablefootnote.dtx
% Version: 2014/01/26 v1.1c
%
% Copyright (C) 2011 - 2014 by
%    H.-Martin M"unch <Martin dot Muench at Uni-Bonn dot de>
%
% This work may be distributed and/or modified under the
% conditions of the LaTeX Project Public License, either
% version 1.3c of this license or (at your option) any later
% version. This version of this license is in
%    http://www.latex-project.org/lppl/lppl-1-3c.txt
% and the latest version of this license is in
%    http://www.latex-project.org/lppl.txt
% and version 1.3c or later is part of all distributions of
% LaTeX version 2005/12/01 or later.
%
% This work has the LPPL maintenance status "maintained".
%
% The Current Maintainer of this work is H.-Martin Muench.
%
% This work consists of the main source file tablefootnote.dtx,
% the README, and the derived files
%    tablefootnote.sty, tablefootnote.pdf,
%    tablefootnote.ins, tablefootnote.drv,
%    tablefootnote-example.tex, tablefootnote-example.pdf.
%
% Distribution:
%    http://mirror.ctan.org/macros/latex/contrib/tablefootnote/tablefootnote.dtx
%    http://mirror.ctan.org/macros/latex/contrib/tablefootnote/tablefootnote.pdf
%    http://mirror.ctan.org/install/macros/latex/contrib/tablefootnote.tds.zip
%
% See http://www.ctan.org/pkg/tablefootnote (Catalogue)
%
% Unpacking:
%    (a) If tablefootnote.ins is present:
%           tex tablefootnote.ins
%    (b) Without tablefootnote.ins:
%           tex tablefootnote.dtx
%    (c) If you insist on using LaTeX
%           latex \let\install=y% \iffalse meta-comment
%
% File: tablefootnote.dtx
% Version: 2014/01/26 v1.1c
%
% Copyright (C) 2011 - 2014 by
%    H.-Martin M"unch <Martin dot Muench at Uni-Bonn dot de>
%
% This work may be distributed and/or modified under the
% conditions of the LaTeX Project Public License, either
% version 1.3c of this license or (at your option) any later
% version. This version of this license is in
%    http://www.latex-project.org/lppl/lppl-1-3c.txt
% and the latest version of this license is in
%    http://www.latex-project.org/lppl.txt
% and version 1.3c or later is part of all distributions of
% LaTeX version 2005/12/01 or later.
%
% This work has the LPPL maintenance status "maintained".
%
% The Current Maintainer of this work is H.-Martin Muench.
%
% This work consists of the main source file tablefootnote.dtx,
% the README, and the derived files
%    tablefootnote.sty, tablefootnote.pdf,
%    tablefootnote.ins, tablefootnote.drv,
%    tablefootnote-example.tex, tablefootnote-example.pdf.
%
% Distribution:
%    http://mirror.ctan.org/macros/latex/contrib/tablefootnote/tablefootnote.dtx
%    http://mirror.ctan.org/macros/latex/contrib/tablefootnote/tablefootnote.pdf
%    http://mirror.ctan.org/install/macros/latex/contrib/tablefootnote.tds.zip
%
% See http://www.ctan.org/pkg/tablefootnote (Catalogue)
%
% Unpacking:
%    (a) If tablefootnote.ins is present:
%           tex tablefootnote.ins
%    (b) Without tablefootnote.ins:
%           tex tablefootnote.dtx
%    (c) If you insist on using LaTeX
%           latex \let\install=y\input{tablefootnote.dtx}
%        (quote the arguments according to the demands of your shell)
%
% Documentation:
%    (a) If tablefootnote.drv is present:
%           (pdf)latex tablefootnote.drv
%           makeindex -s gind.ist tablefootnote.idx
%           (pdf)latex tablefootnote.drv
%           makeindex -s gind.ist tablefootnote.idx
%           (pdf)latex tablefootnote.drv
%    (b) Without tablefootnote.drv:
%           (pdf)latex tablefootnote.dtx
%           makeindex -s gind.ist tablefootnote.idx
%           (pdf)latex tablefootnote.dtx
%           makeindex -s gind.ist tablefootnote.idx
%           (pdf)latex tablefootnote.dtx
%
%    The class ltxdoc loads the configuration file ltxdoc.cfg
%    if available. Here you can specify further options, e.g.
%    use DIN A4 as paper format:
%       \PassOptionsToClass{a4paper}{article}
%
% Installation:
%    TDS:tex/latex/tablefootnote/tablefootnote.sty
%    TDS:doc/latex/tablefootnote/tablefootnote.pdf
%    TDS:doc/latex/tablefootnote/tablefootnote-example.tex
%    TDS:doc/latex/tablefootnote/tablefootnote-example.pdf
%    TDS:source/latex/tablefootnote/tablefootnote.dtx
%
%<*ignore>
\begingroup
  \catcode123=1 %
  \catcode125=2 %
  \def\x{LaTeX2e}%
\expandafter\endgroup
\ifcase 0\ifx\install y1\fi\expandafter
         \ifx\csname processbatchFile\endcsname\relax\else1\fi
         \ifx\fmtname\x\else 1\fi\relax
\else\csname fi\endcsname
%</ignore>
%<*install>
\input docstrip.tex
\Msg{******************************************************************}
\Msg{* Installation                                                   *}
\Msg{* Package: tablefootnote 2014/01/26 v1.1c Table foot notes (HMM) *}
\Msg{******************************************************************}

\keepsilent
\askforoverwritefalse

\let\MetaPrefix\relax
\preamble

This is a generated file.

Project: tablefootnote
Version: 2014/01/26 v1.1c

Copyright (C) 2011 - 2014 by
    H.-Martin M"unch <Martin dot Muench at Uni-Bonn dot de>

The usual disclaimer applies:
If it doesn't work right that's your problem.
(Nevertheless, send an e-mail to the maintainer
 when you find an error in this package.)

This work may be distributed and/or modified under the
conditions of the LaTeX Project Public License, either
version 1.3c of this license or (at your option) any later
version. This version of this license is in
   http://www.latex-project.org/lppl/lppl-1-3c.txt
and the latest version of this license is in
   http://www.latex-project.org/lppl.txt
and version 1.3c or later is part of all distributions of
LaTeX version 2005/12/01 or later.

This work has the LPPL maintenance status "maintained".

The Current Maintainer of this work is H.-Martin Muench.

This work consists of the main source file tablefootnote.dtx,
the README, and the derived files
   tablefootnote.sty, tablefootnote.pdf,
   tablefootnote.ins, tablefootnote.drv,
   tablefootnote-example.tex, tablefootnote-example.pdf.

In memoriam Tommy Muench + 2014/01/02.

\endpreamble
\let\MetaPrefix\DoubleperCent

\generate{%
  \file{tablefootnote.ins}{\from{tablefootnote.dtx}{install}}%
  \file{tablefootnote.drv}{\from{tablefootnote.dtx}{driver}}%
  \usedir{tex/latex/tablefootnote}%
  \file{tablefootnote.sty}{\from{tablefootnote.dtx}{package}}%
  \usedir{doc/latex/tablefootnote}%
  \file{tablefootnote-example.tex}{\from{tablefootnote.dtx}{example}}%
}

\catcode32=13\relax% active space
\let =\space%
\Msg{************************************************************************}
\Msg{*}
\Msg{* To finish the installation you have to move the following}
\Msg{* file into a directory searched by TeX:}
\Msg{*}
\Msg{*     tablefootnote.sty}
\Msg{*}
\Msg{* To produce the documentation run the file `tablefootnote.drv'}
\Msg{* through (pdf)LaTeX, e.g.}
\Msg{*  pdflatex tablefootnote.drv}
\Msg{*  makeindex -s gind.ist tablefootnote.idx}
\Msg{*  pdflatex tablefootnote.drv}
\Msg{*  makeindex -s gind.ist tablefootnote.idx}
\Msg{*  pdflatex tablefootnote.drv}
\Msg{*}
\Msg{* At least three runs are necessary e.g. to get the}
\Msg{*  references right!}
\Msg{*}
\Msg{* Happy TeXing!}
\Msg{*}
\Msg{************************************************************************}

\endbatchfile
%</install>
%<*ignore>
\fi
%</ignore>
%
% \section{The documentation driver file}
%
% The next bit of code contains the documentation driver file for
% \TeX{}, i.\,e., the file that will produce the documentation you
% are currently reading. It will be extracted from this file by the
% \texttt{docstrip} programme. That is, run \LaTeX{} on \texttt{docstrip}
% and specify the \texttt{driver} option when \texttt{docstrip}
% asks for options.
%
%    \begin{macrocode}
%<*driver>
\NeedsTeXFormat{LaTeX2e}[2011/06/27]
\ProvidesFile{tablefootnote.drv}%
  [2014/01/26 v1.1c Table foot notes (HMM)]
\documentclass{ltxdoc}[2007/11/11]% v2.0u
\usepackage{holtxdoc}[2012/03/21]%  v0.24
%% tablefootnote may work with earlier versions of LaTeX2e and those
%% class and package, but this was not tested.
%% Please consider updating your LaTeX, class, and package
%% to the most recent version (if they are not already the most
%% recent version).
\hypersetup{%
 pdfsubject={Footnotes in Tables (HMM)},%
 pdfkeywords={LaTeX, tablefootnote, footnote, table, H.-Martin Muench},%
 pdfencoding=auto,%
 pdflang={en},%
 breaklinks=true,%
 linktoc=all,%
 pdfstartview=FitH,%
 pdfpagelayout=OneColumn,%
 bookmarksnumbered=true,%
 bookmarksopen=true,%
 bookmarksopenlevel=3,%
 pdfmenubar=true,%
 pdftoolbar=true,%
 pdfwindowui=true,%
 pdfnewwindow=true%
}
\CodelineIndex
\hyphenation{printing docu-ment}
\gdef\unit#1{\mathord{\thinspace\mathrm{#1}}}%
\makeatletter
\@ifundefined{eTeX}{\gdef\eTeX{$\m@th \varepsilon $-\TeX }}{% else \relax
}
\makeatother
\begin{document}
  \DocInput{tablefootnote.dtx}%
\end{document}
%</driver>
%    \end{macrocode}
%
% \fi
%
% \CheckSum{584}
%
% \CharacterTable
%  {Upper-case    \A\B\C\D\E\F\G\H\I\J\K\L\M\N\O\P\Q\R\S\T\U\V\W\X\Y\Z
%   Lower-case    \a\b\c\d\e\f\g\h\i\j\k\l\m\n\o\p\q\r\s\t\u\v\w\x\y\z
%   Digits        \0\1\2\3\4\5\6\7\8\9
%   Exclamation   \!     Double quote  \"     Hash (number) \#
%   Dollar        \$     Percent       \%     Ampersand     \&
%   Acute accent  \'     Left paren    \(     Right paren   \)
%   Asterisk      \*     Plus          \+     Comma         \,
%   Minus         \-     Point         \.     Solidus       \/
%   Colon         \:     Semicolon     \;     Less than     \<
%   Equals        \=     Greater than  \>     Question mark \?
%   Commercial at \@     Left bracket  \[     Backslash     \\
%   Right bracket \]     Circumflex    \^     Underscore    \_
%   Grave accent  \`     Left brace    \{     Vertical bar  \|
%   Right brace   \}     Tilde         \~}
%
% \GetFileInfo{tablefootnote.drv}
%
% \begingroup
%   \def\x{\#,\$,\^,\_,\~,\ ,\&,\{,\},\%}%
%   \makeatletter
%   \@onelevel@sanitize\x
% \expandafter\endgroup
% \expandafter\DoNotIndex\expandafter{\x}
% \expandafter\DoNotIndex\expandafter{\string\ }
% \begingroup
%   \makeatletter
%     \lccode`9=32\relax
%     \lowercase{%^^A
%       \edef\x{\noexpand\DoNotIndex{\@backslashchar9}}%^^A
%     }%^^A
%   \expandafter\endgroup\x
%
% \DoNotIndex{\\,\,}
% \DoNotIndex{\documentclass,\usepackage,\ProvidesPackage,\begin,\end}
% \DoNotIndex{\MessageBreak,\message}
% \DoNotIndex{\NeedsTeXFormat,\DoNotIndex,\verb}
% \DoNotIndex{\def,\edef,\gdef,\global}
% \DoNotIndex{\ifx,\listfiles,\mathord,\mathrm}
% \DoNotIndex{\kvoptions,\SetupKeyvalOptions,\ProcessKeyvalOptions}
% \DoNotIndex{\bigskip,\space,\thinspace,\ldots}
% \DoNotIndex{\indent,\noindent,\newline,\linebreak,\pagebreak}
% \DoNotIndex{\textbf,\textit,\textsf,\texttt,\textsc,\textquotedblleft,\textquotedblright}
% \DoNotIndex{\plainTeX,\TeX,\LaTeX,\pdfLaTeX}
% \DoNotIndex{\chapter,\section,\the,\csname,\endscname}
% \DoNotIndex{\value,\arabic,\hline,\text,\caption,\centering,\clear,\url}
%
% \title{The \xpackage{tablefootnote} package}
% \date{2014/01/26 v1.1c}
% \author{H.-Martin M\"{u}nch\\\xemail{Martin.Muench at Uni-Bonn.de}}
%
% \maketitle
%
% \begin{abstract}
% This \LaTeX{} package provides the command |\tablefootnote| to be used
% in a \texttt{table} or \texttt{sidewaystable} environment,
% where |\footnote| will not work and when using |\footnotemark| and
% |\footnotetext| and adjusting the counters (including \texttt{Hfootnote})
% manually is either too much work or would not even work (\texttt{sidewaystable}).
% \end{abstract}
%
% \bigskip
%
% \noindent Disclaimer for web links: The author is not responsible for any contents
% referred to in this work unless he has full knowledge of illegal contents.
% If any damage occurs by the use of information presented there, only the
% author of the respective pages might be liable, not the one who has referred
% to these pages.
%
% \bigskip
%
% \noindent {\color{green} Save per page about $200\unit{ml}$ water,
% $2\unit{g}$ CO$_{2}$ and $2\unit{g}$ wood:\\
% Therefore please print only if this is really necessary.}
%
% \pagebreak
%
% \tableofcontents
%
% \pagebreak
%
% \section{Introduction\label{sec:Introduction}}
%
% \indent This \LaTeX{} package provides the command |\tablefootnote{...}| to be
% used in a \texttt{table} or \texttt{sidewaystable} environment. In those
% environments |\footnote{...}| would not work and would need to be replaced
% by |\footnotemark| in the \texttt{(sideways)\allowbreak table} and
% |\footnotetext{...}| after the \texttt{(sideways)table} environment ended.
% Additionally the counters for \texttt{footnote} and
% (when the \xpackage{hyperref} package is used) \texttt{Hfootnote}
% would need to be adjusted manually. Then still the hyperlinks
% in a \texttt{(sideways)table} would not work.
% When the \xpackage{footnotebackref} package is used,
% |\footnotemark| and |\footnotetext{...}| would need to be redefined.
% Just using |\tablefootnote{...}| (with optional argument for custom foot note
% marks) in the (sideways)table does all this automatically.
% (Compatibility with the \xpackage{footnotebackref} package
% was achieved without redefining |\footnotemark| or |\footnotetext|.)
% Redefining |\footnote| was \emph{not} done as to not create conflicts with
% other packages.
%
% \section{Usage}
%
% \indent Just load the package placing
% \begin{quote}
%   |\usepackage{tablefootnote}|
% \end{quote}
% \noindent in the preamble of your \LaTeXe{} source file and use\newline
% |\tablefootnote{...}| instead of |\footnote{...}| and\newline
% |\tablefootnote[...]{...}| instead of |\footnote[...]{...}|
% in (sideways)\allowbreak tables.\\
% When the \xpackage{footnotebackref} package is used,
% it must be loaded before the \xpackage{tablefootnote} package
% and the \xpackage{hyperrefpackage} with option |hyperfootnotes=true|
% must be loaded, too. (Backreference{-}links without links do not make sense.)\\
% When the \xpackage{rotating} package (for |sidewaystable|s) is used,
% it must be loaded before the \xpackage{tablefootnote} package and it must be
% a recent version (v2.16a, 2009/03/28, or newer).\\
% If |\tablefootnote{...}| is used in a tabular environment, which is inside
% of a (sideways)table environment, everything is fine, but when the tabular
% environment is not inside of a (sideways)table, the \textbf{table}footnote
% will not work.\\
%
% Pages with tables with footnotes, which are rotated with the \xpackage{lscape}-
% or with the \xpackage{pdflscape}-package (|\begin{landscape}|, table, |\end{landscape}|),
% are possible. But when the \xpackage{footmisc}-package is used, \xpackage{(pdf)lscape}
% should be loaded before \xpackage{footmisc}.\\
%
% When the \xpackage{footmisc}-package is used with option \texttt{para},
% this is regarded with the exception of a \texttt{sidewaystable}. There the footnotes
% are printed one below the other. (Well, this is better than no footnote in a
% \texttt{sidewaystable} at all, isn't it?)\\
%
% When the (sideways)table floats, the footnotes are set and (when \xpackage{hyperref}
% is used) hyperlinked, but they are not automatically adapted when the table floats
% over/under another footnote. Thus either do not use a footnote between original and
% float position of the (sideways)table, or place the (sideways)table in
% \textquotedblleft here\textquotedblright\ position. |\clear(double)page|, |h(!)|,
% |H|~from the \xpackage{float} package (\url{http://www.ctan.org/pkg/float}),
% or |\FloatBarrier| from the \xpackage{picins} package
% (\url{http://www.ctan.org/pkg/picins}) might help, too.
% (Or after finishing the document, move the (sideways)table in the source code
% near the position where it floats to or use the optional footnote marks.)
%
% \pagebreak
%
% \section{Alternatives\label{sec:Alternatives}}
%
% \begin{description}
% \item[-] The \xpackage{longtable} package provides the \texttt{longtable} environment
%   as replacement for the combined \texttt{table} and \texttt{tabular} environments.
%   Footnotes are real footnotes (not just tablenotes), are continuously numbered and
%   hyperlinked (when using the \xpackage{hyperref} package), and the hyperlinks really
%   work. As drawback the appearance of the caption changes slightly (e.\,g.~distance
%   to the table, width of the caption), which can probably be changed back manually.
%   Furthermore, longtables are meaned to break over more than one page. If that is not
%   wished, it must be prevented by |\nopagebreak|-commands and by ending the longtable
%   lines with |\\*| instead of |\\|. longtables do not float. (Therefore using the
%   \xpackage{tablefootnote} package and |\FloatBarrier| from the \xpackage{picins} package
%   before and after the table environment is similar~- but \xpackage{tablefootnote} does not
%   change the table-caption!) \texttt{sidewaystable} does not work with it.\\
%   \url{http://www.ctan.org/pkg/longtable}
%
% \item[-] The \xpackage{supertabular} package provides the \texttt{mpsupertabular}
%   environment as replacement for the combined \texttt{table} and \texttt{tabular}
%   environments. Footnotes are just tablenotes (with working hyperlinks when using the
%   \xpackage{hyperref} package), i.\,e. numbered a, b, c and placed below the table
%   and not at the end of the page. Therefore there is no float problem (because the
%   tablenotes numbering is not included in the continuous numbering of the footnotes).
%   Placing the \texttt{supertabular} inside of a \texttt{sidewaystable} breaks the
%   hyperlinks to the tablenotes.\\
%   \url{http://www.ctan.org/pkg/supertabular}
%
% \item[-] The \xpackage{ctable} package has its very own notation for defining tables.
%   It can create tablenotes and sideways-tables. The tablenotes are not automatically
%   hyperlinked. The \texttt{ctable}s float. Because the tablenotes numbering is not
%   included in the continuous numbering of the footnotes there is no float problem.\\
%   \url{http://www.ctan.org/pkg/ctable}
%
% \item[-] The \xpackage{footnote} package provides |\makesavenoteenv{table}|. After
%   loading the package and using that command in the preamble, in tables |\footnote{...}|
%   can be used. Using |\makesavenoteenv{tabular}| and \\
%   |\makesavenoteenv{sidewaystable}|
%   is possible, but it neither solves the float problem, nor do the created hyperlinks
%   work (i.\,e.~they aim at wrong locations). The \xpackage{mdwtab} from the same bundle
%   is incompatible with other table-related packages (e.\,g. \xpackage{supertabular},
%   \xpackage{array}) and not $100\,\%$ compatible with the \texttt{tabular} environment.\\
%   \url{http://www.ctan.org/pkg/footnote}\\
%   \url{http://www.ctan.org/pkg/mdwtab}
%
% \item[-] The \xpackage{tabularx} package does produce footnotes for sidewaystables,
%   but uses a, b, c instead of 1, 2, 3. The hyperlinks to the footnotes do not work.
%   Because the footnotes numbering is not included in the continuous numbering of
%   the other footnotes there is no float problem.\\
%   \url{http://www.ctan.org/pkg/tabularx}
%
% \item[-] Placing a tabular inside a minipage inside a table produces tablenotes.
%   Therefore there is no float problem (because the footnotes are not continuously
%   numbered). The hyperlinks to the table notes indeed work.
%
% \pagebreak
%
% \item[-] The \xpackage{threeparttable} package creates tablenotes again. Therefore
%   there is no float problem (because the tablenotes are not continuously numbered with
%   the footnotes). There are no hyperlinks to the table notes (at least not automatically).
%   Using |sidewaystable| (with table notes) works.\\
%   \url{http://www.ctan.org/pkg/threeparttable}
%
% \item[-] The \xpackage{threeparttablex} package creates tablenotes again. Therefore there
%   is no float problem (because the tablenotes are not continuously numbered with the
%   footnotes). With option \texttt{referable} the tablenotes are hyperlinked. Use of
%   a |sidewaystable| (with table notes) did not work for me. When using the
%   \texttt{referable} option according to the example in the \xpackage{threeparttablex}
%   manual the \xpackage{longtable} package is used, therefore that package could be used
%   directly without \xpackage{threeparttablex} (see above).\\
%   \url{http://www.ctan.org/pkg/threeparttablex}
%
% \item[-] One can manually use |\footnotemark| in the table and |\footnotetext{...}|
%  after the table environment ended and manually change the |footnote| and
%  (when the \xpackage{hyperref} package is used) |Hfootnote| counters and needs
%  to insert
%  \begin{verbatim}
%     \makeatletter
%     \global\let\Hy@saved@currentHref\@currentHref
%     \hyper@makecurrent{Hfootnote}%
%     \global\let\Hy@footnote@currentHref\@currentHref
%     \global\let\@currentHref\Hy@saved@currentHref
%     \makeatother
%  \end{verbatim}
%  before each |\footnotetext{...}|, but with more than very few footnotes this can become
%  much work (and is prone to produce errors). And this does not even work with
%  sidewaystables. (And it does not work with the \xpackage{footnotebackref} package
%  without redefining |\footnotemark| and |\footnotetext{...}|.)
% \end{description}
%
% \bigskip
%
% \noindent (You programmed or found another alternative,
%  which is available at \CTAN{}?\\
%  OK, send an e-mail to me with the name, location at \CTAN{},
%  and a short notice, and I will probably include it in
%  the list above.)
%
% \newpage
%
% \section{Example}
%
%    \begin{macrocode}
%<*example>
\documentclass[british]{article}[2007/10/19]% v1.4h
%%%%%%%%%%%%%%%%%%%%%%%%%%%%%%%%%%%%%%%%%%%%%%%%%%%%%%%%%%%%%%%%%%%%%
\usepackage{float}[2001/11/08]%    v1.3d
\usepackage{placeins}[2005/04/18]% v2.2  ; for \FloatBarrier
\usepackage{rotating}[2009/03/28]% v2.16a; for sidewaystable-environment
\usepackage[%
 hyperfootnotes=true,%
 extension=pdf,%
 plainpages=false,%
 pdfpagelabels=true,%
 hyperindex=false,%
 pdflang={en},%
 pdftitle={tablefootnote package example},%
 pdfauthor={H.-Martin Muench},%
 pdfsubject={Example for the tablefootnote package},%
 pdfkeywords={LaTeX, tablefootnote, footnote, table, H.-Martin Muench},%
 % pdfview=FitH and FitBH do not work: hyperlinks in sidewaystables
 % do no lead to the footnotes, due to a bug in pdfTeX,
 % computing wrong anchor coordinates (Heiko Oberdiek, 29. October 2011)
 % pdfview=Fit, FitV, FitR, FitB, FitBV work
 % print is OK for all those options
 pdfstartview=FitH,%
 pdfpagelayout=OneColumn%
]{hyperref}[2012/11/06]% v6.83m
 % Due to the urls used in the example, either the hyperref or the url
 % package are needed (or the urls must be removed before compiling).

\usepackage{footnotebackref}[2012/07/01]% v1.0
\usepackage{tablefootnote}[2014/01/26]%   v1.1c

\gdef\unit#1{\mathord{\thinspace\mathrm{#1}}}%
\listfiles
\begin{document}
\pagenumbering{arabic}
\section*{Example for tablefootnote}

This example demonstrates the use of package\newline
\textsf{tablefootnote}, v1.1c as of 2014/01/26 (HMM).\newline
There were no options used. (The package provides no options.)\newline

\textbf{The \texttt{tablefootnote-example.tex} needs to be compiled
at least twice to get the references right!}\newline

If the etoolbox-package is found, it is automatically used.\newline

For more details please see the documentation!\newline

\noindent Save per page about $200\unit{ml}$ water,
$2\unit{g}$ CO$_{2}$ and $2\unit{g}$ wood:\newline
Therefore please print only if this is really necessary.\newline

Here is some text.\footnote{And this is a text footnote.}\newline

Tables \ref{tab.symbol}, \ref{tab.normal}, \ref{tab.another} and
\ref{tab.floatH} show normal tables, table~\ref{tab.sideways} depicts
a sidewaystable. Table~\ref{tab.floatH} uses the float
specifier~\texttt{H} from the float package.\newline

\texttt{Hyperref} option \verb|pdfview=FitH| and \verb|FitBH| do
not work due to a bug in pdf\TeX{}, computing wrong
anchor coordinates (\textsc{Heiko Oberdiek}, 29.~October 2011).
Depending on used pdf-viewer, hyperlinks in sidewaystables lead
e.\,g.~at the end of the document, not at the footnote.
\verb|pdfview=Fit|, \verb|FitV|, \verb|FitR|, \verb|FitB|,
\verb|FitBV| work, print is OK for all those options.

\bigskip

\listoftables

\pagebreak

\renewcommand{\thefootnote}{\fnsymbol{footnote}}
\verb|\renewcommand{\thefootnote}{\fnsymbol{footnote}}|
causes footnote{-}symbol{-}footnotes, which are possible
(see Table~\ref{tab.symbol}).

\begin{table}
\centering
\begin{tabular}{ccc}
Another\tablefootnote{A table footnote.} & %
text\tablefootnote{Another table footnote.} & %
in a table\tablefootnote{A third table footnote.}
\end{tabular}
\caption[A footnotesymbol table]{%
A table with footnote-symbol-footnotes.\label{tab.symbol}}
\end{table}

Some text.\footnote{A text footnote.}

\renewcommand{\thefootnote}{\arabic{footnote}}
\verb|\renewcommand{\thefootnote}{\arabic{footnote}}|
switches back to normal footnote numbers again.

\pagebreak

\begin{table}
\centering
\begin{tabular}{ccc}
Some\tablefootnote{A table footnote.} & %
text\tablefootnote[99]{A table footnote with custom footnote number.} & %
in a table\tablefootnote{A third table footnote.}
\end{tabular}
\caption[A table]{A normal table.\label{tab.normal}}
\end{table}

Some text.\footnote{Another text footnote.}

\pagebreak

%    \end{macrocode}
%
% \pagebreak
%
%    \begin{macrocode}
More text.\footnote{And yet another text footnote.}

\begin{table}[t]
\centering
\begin{tabular}{|c|c|c|}
\hline
Another\tablefootnote{A $3^{rd}$ table footnote.} & %
text\tablefootnote{Another $3^{rd}$ table footnote.} & %
in a table\tablefootnote{A $3^{rd}$ third table footnote.}\\ \hline
\end{tabular}
\caption[Another table]{Another table (third one)\label{tab.another}}
\end{table}

Please note that Table~\ref{tab.another} floated to the top of the page.
While the footnotes are set and (when hyperref is used) hyperlinked,
they are not automatically adapted. Thus either do not use a footnote
at the same page before the table, or place the table in
\textquotedblleft here\textquotedblright\ or
\textquotedblleft bottom\textquotedblright\ position.
\verb|\clear(double)page|, \verb|h(!)|, \verb|H|~from the
\texttt{float} package, or \verb|\FloatBarrier| from the
\texttt{picins} package might help, too. (Or move the table in the
source code near the position where it floats to
or use the optional footnote marks.)

Table~\ref{tab.floatH} (page~\pageref{tab.floatH}) uses float specifier
\texttt{H} from the float package and does not float.

Some text.\footnote{This is just another text footnote.}

\pagebreak

\FloatBarrier

\begin{sidewaystable}
\centering%
\begin{tabular}{ccc}
Text\tablefootnote{Please rotate the view for testing the %
hyperlinks.} & %
in a\tablefootnote[98]{Another sidewaystable footnote %
with optional footnote mark.} & %
sidewaystable%
\tablefootnote{Sidewaystable-environment provided by %
\url{http://www.ctan.org/pkg/rotating} package.}
\end{tabular}
\caption[A sideways table]{A table in the \texttt{sideways} %
environment\label{tab.sideways}}%
\end{sidewaystable}%

\FloatBarrier

\pagebreak

%    \end{macrocode}
%
% \pagebreak
%
%    \begin{macrocode}
A last table, here with float specifier \texttt{H} from the
float\footnote{\url{http://www.ctan.org/pkg/float}} package.

\begin{table}[H]
\centering
\begin{tabular}{ccc}
Another\tablefootnote{A $5^{th}$ table footnote.} & %
text\tablefootnote{Another $5^{th}$ table footnote.} & %
in a table\tablefootnote{A $5^{th}$ third table footnote.}
\end{tabular}
\caption[A last table]{A very last table\label{tab.floatH}}
\end{table}

Some text.\footnote{This is just another text footnote.}

\pagebreak

End of the example for the
tablefootnote\footnote{\url{http://www.ctan.org/pkg/tablefootnote}}
package.

\end{document}
%</example>
%    \end{macrocode}
%
% \StopEventually{}
%
% \newpage
%
% \section{The implementation}
%
% We start off by checking that we are loading into \LaTeXe{} and
% announcing the name and version of this package.
%
%    \begin{macrocode}
%<*package>
%    \end{macrocode}
%
%    \begin{macrocode}
\NeedsTeXFormat{LaTeX2e}[2011/06/27]
\ProvidesPackage{tablefootnote}[2014/01/26 v1.1c
              Table foot notes (HMM)]
%    \end{macrocode}
%
% A short description of the \xpackage{tablefootnote} package:
%
%    \begin{macrocode}
%% Provides the \tablefootnote[...]{...} command
%% for footnotes in (sideways)tables.

%    \end{macrocode}
%
% We need the \xpackage{ltxcmds} and \xpackage{letltxmacro} packages
% by \textsc{Heiko Oberdiek}:
%
%    \begin{macrocode}
\RequirePackage{ltxcmds}[2011/11/09]%     v1.22
\RequirePackage{letltxmacro}[2010/09/02]% v1.4
\RequirePackage{xifthen}[2009/04/17]%     v1.3
%    \end{macrocode}
%
% A last information for the user:
%
%    \begin{macrocode}
%% tablefootnote may work with earlier versions of LaTeX and those
%% packages, but this was not tested. Please consider updating
%% your LaTeX and packages to the most recent version
%% (if they are not already the most recent version).

%    \end{macrocode}
%
% \indent See subsection~\ref{ss:Downloads} about how to get them.\\
%
% When the \xpackage{rotating} package (for |sidewaystable|s) is used,
% it must be loaded before the \xpackage{tablefootnote} package and it must be
% a recent version:
%
%    \begin{macrocode}
\@ifpackageloaded{rotating}{%
  \gdef\tfn@rotating{1}
  \@ifpackagelater{rotating}{2009/03/28}% v2.16a
   {% >= 2009/03/28, OK
   }{% else
     \PackageWarningNoLine{tablefootnote}{%
       It is required version\MessageBreak%
       2009/03/28 v2.16a (or later) of package rotating,\MessageBreak%
       but only version\MessageBreak%
      `\csname ver@rotating.sty\endcsname'\MessageBreak%
      is available}%
   }%
 }{% else
   \gdef\tfn@rotating{0}%
 }

%    \end{macrocode}
%
% There are no options to be processed.\\
%
% We need some definitions:
%
%    \begin{macrocode}
\def\tfn@footnotetablecount{0}
\def\tfn@footnotetableprint{0}
\def\tfn@footnotezero{0}
\def\tfn@fnt{0}
\def\tfn@swt{0}%       _s_ide_w_ays_t_able
\def\tfn@fmpw{\relax}% f_oot_m_isc_p_ara_w_arn-switch
\def\tfn@fnbr{0}%      f_oot_n_ote_b_ack_r_ef(erence package)

%    \end{macrocode}
%
% \begin{macro}{\tfn@hyperfootnotes}
% We need a wrapper for the used |\ifHy@hyperfootnotes|,
% which is not defined when the \xpackage{hyperref} package has not been loaded:
%
%    \begin{macrocode}
\DeclareRobustCommand{\tfn@hyperfootnotes}[2]{%
  \ifHy@hyperfootnotes%
    #1%
  \else%
    #2%
  \fi%
  }

%    \end{macrocode}
% \end{macro}
%
% \begin{macro}{\tablemakefntext}
% |\tablemakefntext| makes the footnotetext for a footnote in a table.
%
%    \begin{macrocode}
\ltx@ifpackageloaded{footnotebackref}{%
  \gdef\tfn@fnbr{1}%
  \ifx\FootnoteBackref@symbol\empty%
    \ifFootnoteBackref@numberlinked%
      \newcommand\tablemakefntext[1]{%
        \noindent\makebox[1.634em][r]{%
         \mbox{\textsuperscript{\normalfont%
           \hyperlink{tfn:m\theHfootnote}{\@thefnmark}%
        }}}%
        \ #1%
      }
    \else
      \newcommand\tablemakefntext[1]{%
        \noindent\makebox[1.8em][r]{%
          \mbox{\textsuperscript{\normalfont%
            \@thefnmark%
        }}}%
        #1%
      }
    \fi
  \else
    \newcommand\tablemakefntext[1]{%
      \noindent\makebox[1.8em][r]{%
        \mbox{\textsuperscript{\normalfont%
          \ifFootnoteBackref@numberlinked%
            \hyperlink{tfn:m\theHfootnote}{\@thefnmark}\ %
          \else%
            \@thefnmark%
          \fi%
      }}}%
      \hyperlink{tfn:m\theHfootnote}{\FootnoteBackref@symbol}%
      #1%
    }
  \fi
}{%
  \newcommand\tablemakefntext[1]{%
    \noindent\makebox[1.8em][r]{%
      \mbox{\textsuperscript{\normalfont%
        \@thefnmark%
    }}}%
    #1%
  }
}

%    \end{macrocode}
%
% \end{macro}
%
% \pagebreak
%
% \begin{macro}{\swtablemakefntext}
%    \begin{macrocode}
\newcommand\swtablemakefntext[1]{\noindent\ #1}

%    \end{macrocode}
% \end{macro}
%
% \begin{macro}{\tfn@footnotemarks}
% |\tfn@footnotemarks| uses the |\footnotemark| command to place a foot note mark
% and decreases the |\tfn@footnotetablecount| (i.\,e.~increases the negative value)
% of the number of footnote(marks) in the table. If the \xpackage{footnotebackref}
% package is used, it is also necessary to create an according |\hypertarget| for
% the back reference.
%
%    \begin{macrocode}
\newcommand{\tfn@footnotemarks}[1][]{%
  \hbox{%
    \ifthenelse{\isempty{#1}}%
      {\footnotemark%
       \ltx@ifpackageloaded{footnotebackref}%
         {\raisebox{\ht\strutbox}{\hypertarget{tfn:m\theHfootnote}{}}}%
         {% footnotebackref not loaded
         }%
      }%
      {\ltx@ifpackageloaded{hyperref}%
        {\tfn@hyperfootnotes{%
            \hyperlink{tfn:#1}{\footnotemark[#1]}%
          }{\footnotemark[#1]}}%
        {\footnotemark[#1]}%
       \ltx@ifpackageloaded{footnotebackref}%
        {\raisebox{\ht\strutbox}{\hypertarget{tfn:m#1}{}}}{}%
      }%
   }%
  \@tempcnta=\tfn@footnotetablecount\relax%
  \advance\@tempcnta by -1\relax%
  \xdef\tfn@footnotetablecount{\the\@tempcnta}%
}

%    \end{macrocode}
% \end{macro}
%
% \begin{macro}{\tfn@footmiscparawarn}
% When the \xpackage{footmisc} package is used with option |para|,
% we need to issue a warning. (Any idea how to resolve this issue?)\\
% It must be given inside of it's own command, because |\ifFN@para| is not
% defined when \xpackage{footmisc} has not been loaded, and then the
% |\fi| would be a |\fi| without |\if| (and \TeX{} does not like single
% |\fi|s).
%
%    \begin{macrocode}
\newcommand{\tfn@footmiscparawarn}{%
  \ifFN@para%
    \PackageWarning{tablefootnote}{%
      Package footmisc with option para used.\MessageBreak%
      Unfortunately package tablefootnote is not yet\MessageBreak%
      able to regard this in sidewaystables.\MessageBreak%
      Here the footnotes are printed one below\MessageBreak%
      the other. Sorry!\MessageBreak%
     }%
  \fi%
  }

%    \end{macrocode}
% \end{macro}
%
% \begin{macro}{\tfn@updateHyper}
% |\tfn@updateHyper| updates some \xpackage{hyperref} internals.
%
%    \begin{macrocode}
\newcommand{\tfn@updateHyper}{%
  % The code of this command is partly from the
  % hyperref package by Heiko Oberdiek, 2011/10/01, v6.82j.
  % Newer versions of that package are available at CTAN.org.
  \global\let\Hy@saved@currentHref\@currentHref%
  \tfn@hyperfootnotes{\hyper@makecurrent{Hfootnote}}{}%
  \global\let\Hy@footnote@currentHref\@currentHref%
  \global\let\@currentHref\Hy@saved@currentHref%
  }

%    \end{macrocode}
% \end{macro}
%
% \begin{macro}{\tfn@footnotetext}
% |\tfn@footnotetext| contains the |\footnotetext[...]{...}| command and handles
% the numbering. |\tfn@footnotetext| will be executed after the table.
% The foot note counter has been increased automatically with each |\footnotemark|,
% therefore it must be reset to its value before the table began.
%
%    \begin{macrocode}
\newcommand{\tfn@footnotetext}[2][]{%
  \ifx\tfn@footnotetableprint\tfn@footnotezero%
    % This is the first footnotetext after the (sideways)table.
    \addtocounter{footnote}{\tfn@footnotetablecount}%
%    \end{macrocode}
%
% When the \xpackage{hyperref} package is used, also its |Hfootnote| counter must
% be reset (when the footnotes are hyperlinked).
%
%    \begin{macrocode}
    \ltx@ifpackageloaded{hyperref}{%
      \tfn@hyperfootnotes{%
        \addtocounter{Hfootnote}{\tfn@footnotetablecount}}{}%
      \tfn@updateHyper%
     }{}%
  \fi%
%    \end{macrocode}
%
% For issuing the text for the next footnote the footnote-counter
% (and |Hfootnote| counter in case of links) are increased by one.
%
%    \begin{macrocode}
  \addtocounter{footnote}{+1}%
  \ltx@ifpackageloaded{hyperref}{%
    \tfn@hyperfootnotes{\addtocounter{Hfootnote}{+1}}{}%
%    \end{macrocode}
%
% The \xpackage{hyperref} package was not expecting this kind of change of the
% |(H)footnote| numbers, therefore some code-lines are missing here which can be
% found at other places in that same package and are called by this command.
%
%    \begin{macrocode}
    \tfn@updateHyper%
%    \end{macrocode}
%
% When \xpackage{hyperref} is not loaded, nothing is to be done here.
%
%    \begin{macrocode}
   }{}%
%    \end{macrocode}
%
% The |\footnotetext[...]{...}| places the foot note text as usual.
%
%    \begin{macrocode}
  \ifx\tfn@swt\tfn@footnotezero%
    \ifthenelse{\isempty{#1}}%
      {\footnotetext{#2}}%
      {\ltx@ifpackageloaded{hyperref}%
         {\tfn@hyperfootnotes{%
             \footnotetext[#1]{%
               \raisebox{\ht\strutbox}{\hypertarget{tfn:#1}{}}#2}%
           }{\footnotetext[#1]{#2}%
           }%
         }{\footnotetext[#1]{#2}}%
      }%
  \else%
%    \end{macrocode}
%
% If there is more than one tablefootnote in the |sidewaystable|, \\
% the |\tfn@footmiscparawarn| shall be given,
%
%    \begin{macrocode}
    \ifx\tfn@footnotetableprint\tfn@footnotezero%
      \gdef\tfn@fmpw{\tfn@footmiscparawarn}%
    \else%
      \ltx@ifpackageloaded{footmisc}{\tfn@fmpw}{\relax}%
%    \end{macrocode}
%
% but we do not want to repeat the message for a third(, forth,\ldots)
% tablefootnote in the |sidewaystable|.
%
%    \begin{macrocode}
      \gdef\tfn@fmpw{\relax}%
    \fi%
%    \end{macrocode}
%
% |\@makefnmark| is \\
% |\hbox {\@textsuperscript {\normalfont \@thefnmark }}|, \\
% but according to |\show\@thefnmark| here |\@thefnmark| is
% |{\protect \itshape}| instead of |\thefootnote| (Why?!).\\
% If the \xpackage{footnotebackref} package is used,
% back references must be created for the table footnotes, too.
%
%    \begin{macrocode}
    \ltx@ifpackageloaded{footnotebackref}{%
      \LetLtxMacro{\tfncurrentmft}{\@makefntext}%
      \LetLtxMacro{\@makefntext}{\swtablemakefntext}%
      \ifthenelse{\isempty{#1}}%
        {\footnotetext{%
           \raisebox{\ht\strutbox}{\hypertarget{tfn:\theHfootnote}{}}%
           \hbox{\@textsuperscript{\normalfont%
              \hyperlink{tfn:m\theHfootnote}{\thefootnote}}}\thinspace%
         \ifx\FootnoteBackref@symbol\empty%
         \else\hyperlink{tfn:m\theHfootnote}{\FootnoteBackref@symbol}%
         \fi%
         #2}%
       }{\edef\tfn@tfn{\arabic{footnote}}%
         \edef\tfn@thn{\arabic{Hfootnote}}%
         \setcounter{footnote}{#1}%
         \setcounter{Hfootnote}{#1}%
         \tfn@updateHyper%
         \footnotetext{%
           \raisebox{\ht\strutbox}{\hypertarget{tfn:#1}{}}%
           \hbox{\@textsuperscript{\normalfont%
                  \hyperlink{tfn:m#1}{#1}}}%
           \thinspace%
           \ifx\FootnoteBackref@symbol\empty%
           \else%
             \raisebox{\ht\strutbox}{%
               \hyperlink{tfn:m#1}{\FootnoteBackref@symbol}}%
           \fi%
           #2%
          }%
         \setcounter{footnote}{\tfn@tfn}%
         \setcounter{Hfootnote}{\tfn@thn}%
         \tfn@updateHyper%
       }%
      \LetLtxMacro{\@makefntext}{\tfncurrentmft}%
     }{% footnotebackref not loaded
       \ifthenelse{\isempty{#1}}%
         {\footnotetext{%
            \hbox{\@textsuperscript{\normalfont\thefootnote}}%
            \ltx@ifpackageloaded{hyperref}%
              {\tfn@hyperfootnotes{%
                 \raisebox{\ht\strutbox}{%
                   \hypertarget{tfn:\theHfootnote}{}}}{}%
              }{}%
            \thinspace%
            #2}%
        }{%
%    \end{macrocode}
%
% \pagebreak
%
%    \begin{macrocode}
          \edef\tfn@tfn{\arabic{footnote}}%
          \ltx@ifpackageloaded{hyperref}{%
            \tfn@hyperfootnotes{\xdef\tfn@thn{\arabic{Hfootnote}}}{}%
           }{}%
          \setcounter{footnote}{#1}
          \ltx@ifpackageloaded{hyperref}{%
            \tfn@hyperfootnotes{\setcounter{Hfootnote}{#1}}{}%
            \tfn@updateHyper%
           }{}%
          \footnotetext{%
            \hbox{\@textsuperscript{\normalfont #1}}%
            \ltx@ifpackageloaded{hyperref}{%
              \tfn@hyperfootnotes{\raisebox{\ht\strutbox}{%
                                         \hypertarget{tfn:#1}{}}}{}%
             }{}%
            \thinspace%
            #2}%
          \setcounter{footnote}{\tfn@tfn}%
          \ltx@ifpackageloaded{hyperref}{%
            \tfn@hyperfootnotes{\setcounter{Hfootnote}{\tfn@thn}}{}%
            \tfn@updateHyper%
           }{}%
        }%
     }%
  \fi%
%    \end{macrocode}
%
% For further footnotes in the table the \\
% |\ifx\tfn@footnotetableprint\tfn@footnotezero|\ldots|\fi| \\
% from the beginning of this |\tfn@footnotetext| macro must not be repeated,
% therefore we deactivate this by defining |\tfn@footnotetableprint| to
% something other than |0| (here |1| is used).
%
%    \begin{macrocode}
  \gdef\tfn@footnotetableprint{1}%
%    \end{macrocode}
%
% |\tfn@footnotetablecount| contains the negative number of footnote texts
% to be printed, therefore it is increase by one (i.\,e.~the negative number
% decreases, i.\,e.~$-3$ becomes~$-2$).
%
%    \begin{macrocode}
  \@tempcnta=\tfn@footnotetablecount\relax%
  \advance\@tempcnta by +1\relax%
  \xdef\tfn@footnotetablecount{\the\@tempcnta}%
%    \end{macrocode}
%
% When it is zero, all footnote texts have been placed, and\\
% |\tfn@footnotetableprint| can be reset to zero and\\
% |\tfn@tablefootnoteprintout| can be cleared.
%
%    \begin{macrocode}
  \ifx\tfn@footnotetablecount\tfn@footnotezero% all footnotes placed
    \gdef\tfn@footnotetableprint{0}%
    \global\let\tfn@tablefootnoteprintout\relax% brute force
 %\else there is at least one \footnotemark which needs \footnotetext
  \fi
}

%    \end{macrocode}
% \end{macro}
%
% \begin{macro}{\tfn@tablefootnoteprintout}
% When something shall be cleared or redefined, it is a good idea
% to define it. It~will be later filled with the foot note text to be printed.
%
%    \begin{macrocode}
\newcommand{\tfn@tablefootnoteprintout}{\relax}

%    \end{macrocode}
% \end{macro}
%
% \pagebreak
%
% \begin{macro}{\tablefootnote}
% |\tablefootnote[...]{...}| is the command to be used by the user instead of
% |\footnote[...]{...}| in the (sideways)table. It places a foot note mark via
% |\tfn@footnotemarks| and appends a |\tfn@footnotetext{#1}| to the\\
% |\tfn@tablefootnoteprintout| macro, and remembers that table foot notes
% have been used.
%
%    \begin{macrocode}
\newcommand{\tablefootnote}[2][]{%
  \tfn@footnotemarks[#1]%
  \ifthenelse{\isempty{#1}}{\relax}{%
    \ltx@GlobalPrependToMacro{\tfn@tablefootnoteprintout}{%
      \addtocounter{footnote}{+1}%
      \ltx@ifpackageloaded{hyperref}{%
        \tfn@hyperfootnotes{\addtocounter{Hfootnote}{+1}}{}%
        \tfn@updateHyper%
       }{}%
     }%
   }%
  \ltx@GlobalAppendToMacro{\tfn@tablefootnoteprintout}{%
    \tfn@footnotetext[#1]{#2}%
    \ifthenelse{\isempty{#1}}{\relax}{%
      \addtocounter{footnote}{-1}%
      \ltx@ifpackageloaded{hyperref}{%
        \tfn@hyperfootnotes{\addtocounter{Hfootnote}{-1}}{}%
        \tfn@updateHyper%
       }{}%
     }%
   }%
  \gdef\tfn@fnt{1}%
  }

%    \end{macrocode}
% \end{macro}
%
% \begin{macro}{\tfn@swtbox}
% For |sidewaystable|s the |\tfn@tablefootnoteprintout| is placed in a \\
% |\parbox|, which can be placed at the end of the |sidewaystable|:
%
%    \begin{macrocode}
\newcommand{\tfn@swtbox}{%
\ifvoid \footins%
\vskip 2\baselineskip%
\else%
\vskip \skip\footins%
\fi%
\parbox[b][][t]{\linewidth}{\raggedright\tfn@tablefootnoteprintout}%
}

%    \end{macrocode}
% \end{macro}
%
% |\tfn@tablefootnoteprintout| shall be executed after the end of the table environment.
% The \xpackage{etoolbox} package provides very easy syntax for this:
%
%    \begin{macrocode}
\IfFileExists{etoolbox.sty}{%
  \RequirePackage{etoolbox}[2011/01/03]%   v2.1
  \AfterEndEnvironment{table}{%
    \ltx@ifpackageloaded{footnotebackref}{%
      \LetLtxMacro{\tfncurrentmft}{\@makefntext}%
      \LetLtxMacro{\@makefntext}{\tablemakefntext}%
       \tfn@tablefootnoteprintout%
      \LetLtxMacro{\@makefntext}{\tfncurrentmft}%
     }{% footnotebackref not loaded
      \tfn@tablefootnoteprintout%
     }%
      \gdef\tfn@fnt{0}%
   }%
%    \end{macrocode}
%
% When the \xpackage{rotating} package has been loaded, we need to detect
% whether the |\tablefootnote| is placed inside of a |sidewaystable|.
%
%    \begin{macrocode}
  \ltx@ifpackageloaded{rotating}{%
    \AtBeginEnvironment{sidewaystable}{%
      \gdef\tfn@swt{1}%
    }
%    \end{macrocode}
%
% At the end of |sidewaystable| environments |\tfn@swtbox| shall be placed
% (if~there was any |\tablefootnote| in that |sidewaystable| at all).
% And |\tfn@swt| as well as |\tfn@fnt| must be reset to |0|.
%
%    \begin{macrocode}
    \AtEndEnvironment{sidewaystable}{%
      \ifx\tfn@fnt\tfn@footnotezero% \relax
      \else%
        \ltx@ifpackageloaded{footnotebackref}{%
          \LetLtxMacro{\tfncurrentmft}{\@makefntext}%
          \LetLtxMacro{\@makefntext}{\tablemakefntext}%
           \tfn@swtbox%
          \LetLtxMacro{\@makefntext}{\tfncurrentmft}%
         }{% footnotebackref not loaded
          \tfn@swtbox%
          }%
      \fi%
      \gdef\tfn@swt{0}%
      \gdef\tfn@fnt{0}%
     }%
   }{}% else \relax
}{%
%    \end{macrocode}
%
% When the \xpackage{etoolbox} package is not available (e.\,g. because of missing
% \eTeX{}) we~do a workaround:
%
%    \begin{macrocode}
  \PackageWarning{tablefootnote}{%
    Package etoolbox not available, therefore\MessageBreak%
    substituting commands \string\AtEndEnvironment\space %
    and\MessageBreak%
    \string\AfterEndEnvironment\space for tablefootnotes%
    \MessageBreak%
   }%
  \@ifpackagelater{letltxmacro}{2010/09/02}%
   {% >= 2010/09/02, OK
    \GlobalLetLtxMacro{\tfnorigtable}{\table}
    \GlobalLetLtxMacro{\tfnendorigtable}{\endtable}
%    \end{macrocode}
%
% |\GlobalLetLtxMacro| was introduced in version 2010/09/02 v1.4 of the
% \xpackage{letltxmacro} package.
%
%    \begin{macrocode}
   }{% else
     \PackageError{tablefootnote}{%
       Outdated version of letltxmacro package used%
      }{Version 2010/09/02 v1.4 or newer needed but not found!%
        \MessageBreak%
        Update the used letltxmacro package.}
        \LetLtxMacro{\tfnorigtable}{\table}
        \LetLtxMacro{\tfnendorigtable}{\endtable}
    }%
%    \end{macrocode}
%
% \pagebreak
%
%    \begin{macrocode}
  \ltx@ifpackageloaded{float}{%
    \@ifpackagelater{float}{2001/11/08}%   v1.3d
     {% >= 2001/11/08, OK
     }{% else
       \PackageWarningNoLine{tablefootnote}{%
         It is required version\MessageBreak
         2001/11/08 v1.3d (or later) of package float,\MessageBreak
         but only version\MessageBreak
        `\csname ver@float.sty\endcsname'\MessageBreak
        is available}%
     }%
   }{\relax}
  \renewenvironment{table}{%
    \ltx@ifpackageloaded{float}{%
%    \end{macrocode}
%
% When \xpackage{etoolbox.sty} was not found, and the \xpackage{float} package
% has been loaded, and the float specifier is \texttt{H}, then |\endtable| was
% let to |\float@endH| by the \xpackage{float} package (line~79 of
% \xpackage{float.sty}, 2001/11/08 v1.3d:\\
% |\expandafter\let\csname end#1\endcsname\float@endH|), i.\,e.~appending\\
% |\tfn@tablefootnoteprintout \gdef\tfn@fnt{0}| to |\endtable| is useless.
% Instead of this, it must be appended to |\float@endH|.\\
% (Why is this not necessary when \xpackage{etoolbox} is used?
% That is a good question, answers are welcome.)
%
%    \begin{macrocode}
      \ltx@LocalAppendToMacro{\float@endH}{%
        \ltx@ifpackageloaded{footnotebackref}{%
          \LetLtxMacro{\tfncurrentmft}{\@makefntext}%
          \LetLtxMacro{\@makefntext}{\tablemakefntext}%
           \tfn@tablefootnoteprintout%
          \LetLtxMacro{\@makefntext}{\tfncurrentmft}%
         }{% footnotebackref not loaded
          \tfn@tablefootnoteprintout%
         }%
          \gdef\tfn@fnt{0}%
       }%
     }{\relax}
    \tfnorigtable%
   }{%
     \tfnendorigtable%
     \ltx@ifpackageloaded{footnotebackref}{%
       \LetLtxMacro{\tfncurrentmft}{\@makefntext}%
       \LetLtxMacro{\@makefntext}{\tablemakefntext}%
        \tfn@tablefootnoteprintout%
       \LetLtxMacro{\@makefntext}{\tfncurrentmft}%
      }{% footnotebackref not loaded
       \tfn@tablefootnoteprintout%
      }%
     \gdef\tfn@fnt{0}%
    }%
%    \end{macrocode}
%
% \pagebreak
%
% The |sidewaystable|-environment must be treated separately:
%
%    \begin{macrocode}
  \ltx@ifpackageloaded{rotating}{%
    \let\tfnorigsidewaystable\sidewaystable
    \let\tfnendorigsidewaystable\endsidewaystable
    \renewenvironment{sidewaystable}{%
      \gdef\tfn@swt{1}%
      \tfnorigsidewaystable%
    }{%
      \ifx\tfn@fnt\tfn@footnotezero% \relax
      \else%
        \ltx@ifpackageloaded{footnotebackref}{%
          \LetLtxMacro{\tfncurrentmft}{\@makefntext}%
          \LetLtxMacro{\@makefntext}{\tablemakefntext}%
           \tfn@swtbox%
          \LetLtxMacro{\@makefntext}{\tfncurrentmft}%
         }{% footnotebackref not loaded
          \tfn@swtbox%
         }%
      \fi%
      \gdef\tfn@swt{0}%
      \gdef\tfn@fnt{0}%
      \tfnendorigsidewaystable%
     }%
   }{}%
}

%    \end{macrocode}
%
% \DescribeMacro{\AtBeginDocument}
% When the \xpackage{rotating} (for |sidewaystable|s) or the
% \xpackage{footnotebackref} (for back references for the footnotes)
% package is used, it must be loaded before the \xpackage{tablefootnote} package.
% Using both packages is also possible, in which case both must be loaded before
% the \xpackage{tablefootnote} package. If any one of those packages is loaded
% |\AtBeginDocument| after this check, the according error message will not be
% given, but packages really should be loaded before |\AtBeginDocument|.
%
%    \begin{macrocode}
\AtBeginDocument{%
  \ltx@ifpackageloaded{footnotebackref}{%
    \ifx\tfn@fnbr\tfn@footnotezero%
      \PackageError{tablefootnote}{%
        Package footnotebackref loaded after tablefootnote}{%
        When using both the footnotebackref and the tablefootnote %
        package,\MessageBreak%
        the footnotebackref package MUST be loaded before the %
        tablefootnote package!\MessageBreak%
       }%
    \fi%
   }{% else \relax
   }
  \ltx@ifpackageloaded{rotating}{%
    \ifx\tfn@rotating\tfn@footnotezero%
      \PackageError{tablefootnote}{%
        Package rotating loaded after tablefootnote}{%
        When using both the rotating and the tablefootnote package,%
        \MessageBreak%
        the rotating package MUST be loaded before the tablefootnote%
        \space package!\MessageBreak%
       }%
    \fi%
   }{% else \relax
   }
%    \end{macrocode}
%
% \pagebreak
%
% When the document is compiled with \LuaLaTeX{}, hyperlinks in rotated content
% will be misplaced, regardless of the use of the \xpackage{tablefootnote} package
% (or of tables or footnotes). The text to be printed will be OK.
% This problem of \LuaLaTeX{} (not \xpackage{tablefootnote}) was found and reported
% by \textsc{Arno Trautmann}, 2011/11/15 -- thanks!
%
%    \begin{macrocode}
  \ltx@ifpackageloaded{hyperref}{%
    \tfn@hyperfootnotes{
      \ifluatex%
        \PackageWarningNoLine{tablefootnote}{%
          LuaTeX and hyperref package used:\MessageBreak%
          Hyperreferences in rotated content\MessageBreak%
          will be misplaced\MessageBreak%
          regardless of tablefootnote package.\MessageBreak%
          The text to be printed will be OK%
         }%
      \fi%
     }{% else \relax
     }%
   }{% else \relax
   }%
%    \end{macrocode}
%
% When the \xpackage{footnotebackref} package is used but the footnotes are not
% hyperlinked, neither the back references nor the tablefootnotes will work.
%
%    \begin{macrocode}
  \ltx@ifpackageloaded{footnotebackref}{%
     \ltx@ifpackageloaded{hyperref}{%
        \tfn@hyperfootnotes{\relax}{% else
          \PackageError{tablefootnote}{%
             footnotebackref but not hyperfootnotes}{%
             The footnotebackref package is used\MessageBreak%
             but the hyperfootnotes option of the hyperref package%
             \MessageBreak%
             was set to false.\MessageBreak%
             Either enable hyperfootnotes or do not use footnotebackref.%
             \MessageBreak%
             Loading of the tablefootnote package will be aborted.%
           }%
          \endinput%
      }%
    }{\PackageError{tablefootnote}{footnotebackref but not hyperref}{%
        The footnotebackref package is used\MessageBreak%
        but not the hyperref package.\MessageBreak%
        Either load hyperref or do not use footnotebackref.%
        \MessageBreak%
        Loading of the tablefootnote package will be aborted.%
        }%
      \endinput%
    }%
  }{}
}

%    \end{macrocode}
%
%    \begin{macrocode}
%</package>
%    \end{macrocode}
%
% \pagebreak
%
% \section{Installation}
%
% \subsection{Downloads\label{ss:Downloads}}
%
% Everything is available on \CTAN{}, \url{http://www.ctan.org/},
% but may need additional packages themselves.\\
%
% \DescribeMacro{tablefootnote.dtx}
% For unpacking the |tablefootnote.dtx| file and constructing the documentation
% it is required:
% \begin{description}
% \item[-] \TeX{} Format \LaTeXe{}: \url{http://www.CTAN.org/}
%
% \item[-] document class \xpackage{ltxdoc}, 2007/11/11, v2.0u,
%           \url{http://www.ctan.org/pkg/ltxdoc}
%
% \item[-] package \xpackage{holtxdoc}, 2012/03/21, v0.24,
%           \url{http://www.ctan.org/pkg/holtxdoc}
%
% \item[-] package \xpackage{hypdoc}, 2011/08/19, v1.11,
%           \url{http://www.ctan.org/pkg/hypdoc}
% \end{description}
%
% \DescribeMacro{tablefootnote.sty}
% The |tablefootnote.sty| for \LaTeXe{} (i.\,e.~each document using
% the \xpackage{tablefootnote} package) requires:
% \begin{description}
% \item[-] \TeX{} Format \LaTeXe{}, \url{http://www.CTAN.org/}
%
% \item[-] package \xpackage{letltxmacro}, 2010/09/02, v1.4,
%           \url{http://www.ctan.org/pkg/letltxmacro}.
%
% \item[-] package \xpackage{ltxcmds}, 2011/11/09, v1.22,
%           \url{http://www.ctan.org/pkg/ltxcmds}
% \end{description}
%
% When the \xpackage{hyperref} package is used, also the \xpackage{ifluatex} package
% is needed, but it is already loaded automatically by the \xpackage{hyperref}
% package.\\
% When the \xpackage{etoolbox} package is available, it is used:
% \begin{description}
% \item[-] package \xpackage{etoolbox}, 2011/01/03, v2.1,
%           \url{http://www.ctan.org/pkg/etoolbox}.
% \end{description}
%
% \DescribeMacro{tablefootnote-example.tex}
% The |tablefootnote-example.tex| requires the same files as all
% documents using the \xpackage{tablefootnote} package and additionally:
% \begin{description}
% \item[-] class \xpackage{article}, 2007/10/19, v1.4h, from \xpackage{classes.dtx}:\\
%           \CTAN{macros/latex/base/classes.dtx}
%
% \item[-] package \xpackage{rotating}, 2009/03/28, v2.16a,
%           \url{http://www.ctan.org/pkg/rotating}
%
% \item[-] package \xpackage{float}, 2001/11/08, v1.3d,
%           \url{http://www.ctan.org/pkg/float}
%
% \item[-] package \xpackage{placeins}, 2005/04/18, v2.2,
%           \url{http://www.ctan.org/pkg/placeins}\\
%           for |\FloatBarrier|
%
% \item[-] package \xpackage{hyperref}, 2012/11/06, v6.83m,
%           \url{http://www.ctan.org/pkg/hyperref}
%
% \item[-] package \xpackage{tablefootnote}, 2014/01/26, v1.1c,\\
%           \url{http://www.ctan.org/pkg/tablefootnote}\\
%   (Well, it is the example file for this package, and because you
%    are reading the documentation for the \xpackage{tablefootnote} package,
%    it can be assumed that you already have some version of it~--
%    is it the current one?)
% \end{description}
%
% \DescribeMacro{Oberdiek}
% \DescribeMacro{holtxdoc}
% \DescribeMacro{ifluatex}
% \DescribeMacro{letltxmacro}
% \DescribeMacro{ltxcmds}
% All packages of \textsc{Heiko Oberdiek}'s bundle `oberdiek'
% (especially \xpackage{holtxdoc}, \xpackage{ifluatex}, \xpackage{letltxmacro},
% and \xpackage{ltxcmds}) are also available in a TDS compliant ZIP archive:\\
% \CTAN{install/macros/latex/contrib/oberdiek.tds.zip}.\\
% It is probably best to download and use this, because the packages in there
% are quite probably both recent and compatible among themselves.\\
%
% \pagebreak
%
% \DescribeMacro{hyperref}
% \noindent \xpackage{hyperref} is not included in that bundle and needs to be
% downloaded separately,\\
% \url{http://www.ctan.org/install/macros/latex/contrib/hyperref.tds.zip}.\\
%
% \DescribeMacro{M\"{u}nch}
% A hyperlinked list of my (other) packages can be found at
% \url{http://www.ctan.org/author/muench-hm}.\\
%
% \subsection{Package, unpacking TDS}
% \paragraph{Package.} This package is available on \CTAN{}
% \begin{description}
% \item[\CTAN{macros/latex/contrib/tablefootnote/tablefootnote.dtx}]\hspace*{0.1cm} \\
%       The source file.
% \item[\CTAN{macros/latex/contrib/tablefootnote/tablefootnote.pdf}]\hspace*{0.1cm} \\
%       The documentation.
% \item[\CTAN{macros/latex/contrib/tablefootnote/tablefootnote-example.pdf}]\hspace*{0.1cm} \\
%       The compiled example file, as it should look like.
% \item[\CTAN{macros/latex/contrib/tablefootnote/README}]\hspace*{0.1cm} \\
%       The README file.
% \end{description}
% There is also a tablefootnote.tds.zip available:
% \begin{description}
% \item[\CTAN{install/macros/latex/contrib/tablefootnote.tds.zip}]\hspace*{0.1cm} \\
%       Everything in \xfile{TDS} compliant, compiled format.
% \end{description}
% which additionally contains\\
% \begin{tabular}{ll}
% tablefootnote.ins & The installation file.\\
% tablefootnote.drv & The driver to generate the documentation.\\
% tablefootnote.sty & The \xext{sty}le file.\\
% tablefootnote-example.tex & The example file.
% \end{tabular}
%
% \bigskip
%
% \noindent For required other packages, please see the preceding subsection.
%
% \paragraph{Unpacking.} The \xfile{.dtx} file is a self-extracting
% \docstrip{} archive. The files are extracted by running the
% \xext{dtx} through \plainTeX{}:
% \begin{quote}
%   \verb|tex tablefootnote.dtx|
% \end{quote}
%
% \noindent About generating the documentation see paragraph~\ref{GenDoc} below.\\
%
% \paragraph{TDS.} Now the different files must be moved into
% the different directories in your installation TDS tree
% (also known as \xfile{texmf} tree):
% \begin{quote}
% \def\t{^^A
% \begin{tabular}{@{}>{\ttfamily}l@{ $\rightarrow$ }>{\ttfamily}l@{}}
%   tablefootnote.sty & tex/latex/tablefootnote/tablefootnote.sty\\
%   tablefootnote.pdf & doc/latex/tablefootnote/tablefootnote.pdf\\
%   tablefootnote-example.tex & doc/latex/tablefootnote/tablefootnote-example.tex\\
%   tablefootnote-example.pdf & doc/latex/tablefootnote/tablefootnote-example.pdf\\
%   tablefootnote.dtx & source/latex/tablefootnote/tablefootnote.dtx\\
% \end{tabular}^^A
% }^^A
% \sbox0{\t}^^A
% \ifdim\wd0>\linewidth
%   \begingroup
%     \advance\linewidth by\leftmargin
%     \advance\linewidth by\rightmargin
%   \edef\x{\endgroup
%     \def\noexpand\lw{\the\linewidth}^^A
%   }\x
%   \def\lwbox{^^A
%     \leavevmode
%     \hbox to \linewidth{^^A
%       \kern-\leftmargin\relax
%       \hss
%       \usebox0
%       \hss
%       \kern-\rightmargin\relax
%     }^^A
%   }^^A
%   \ifdim\wd0>\lw
%     \sbox0{\small\t}^^A
%     \ifdim\wd0>\linewidth
%       \ifdim\wd0>\lw
%         \sbox0{\footnotesize\t}^^A
%         \ifdim\wd0>\linewidth
%           \ifdim\wd0>\lw
%             \sbox0{\scriptsize\t}^^A
%             \ifdim\wd0>\linewidth
%               \ifdim\wd0>\lw
%                 \sbox0{\tiny\t}^^A
%                 \ifdim\wd0>\linewidth
%                   \lwbox
%                 \else
%                   \usebox0
%                 \fi
%               \else
%                 \lwbox
%               \fi
%             \else
%               \usebox0
%             \fi
%           \else
%             \lwbox
%           \fi
%         \else
%           \usebox0
%         \fi
%       \else
%         \lwbox
%       \fi
%     \else
%       \usebox0
%     \fi
%   \else
%     \lwbox
%   \fi
% \else
%   \usebox0
% \fi
% \end{quote}
% If you have a \xfile{docstrip.cfg} that configures and enables \docstrip{}'s
% \xfile{TDS} installing feature, then some files can already be in the right
% place, see the documentation of \docstrip{}.
%
% \subsection{Refresh file name databases}
%
% If your \TeX{}~distribution (\TeX live, \mikTeX{}, \dots{}) relies on file name
% databases, you must refresh these. For example, \teTeX{} users run
% \verb|texhash| or \verb|mktexlsr|.
%
% \subsection{Some details for the interested}
%
% \paragraph{Unpacking with \LaTeX{}.}
% The \xfile{.dtx} chooses its action depending on the format:
% \begin{description}
% \item[\plainTeX:] Run \docstrip{} and extract the files.
% \item[\LaTeX:] Generate the documentation.
% \end{description}
% If you insist on using \LaTeX{} for \docstrip{} (really,
% \docstrip{} does not need \LaTeX{}), then inform the autodetect routine
% about your intention:
% \begin{quote}
%   \verb|latex \let\install=y\input{tablefootnote.dtx}|
% \end{quote}
% Do not forget to quote the argument according to the demands
% of your shell.
%
% \paragraph{Generating the documentation.\label{GenDoc}}
% You can use both the \xfile{.dtx} or the \xfile{.drv} to generate
% the documentation. The process can be configured by a
% configuration file \xfile{ltxdoc.cfg}. For instance, put the following
% line into this file, if you want to have A4 as paper format:
% \begin{quote}
%   \verb|\PassOptionsToClass{a4paper}{article}|
% \end{quote}
%
% \noindent An example follows how to generate the
% documentation with \pdfLaTeX{}:
%
% \begin{quote}
%\begin{verbatim}
%pdflatex tablefootnote.dtx
%makeindex -s gind.ist tablefootnote.idx
%pdflatex tablefootnote.dtx
%makeindex -s gind.ist tablefootnote.idx
%pdflatex tablefootnote.dtx
%\end{verbatim}
% \end{quote}
%
% \subsection{Compiling the example}
%
% The example file, \textsf{tablefootnote-example.tex}, can be compiled via\\
% \indent |latex tablefootnote-example.tex|,\\
% \indent |lualatex tablefootnote-example.tex|,\\
% or (recommended)\\
% \indent |pdflatex thumbs-example.tex|\\
% and will need at least two compiler runs to get everything right.
%
% \bigskip
%
% \section{Acknowledgements}
%
% I would like to thank \textsc{Heiko Oberdiek} for providing
% the \xpackage{hyperref}, \xpackage{ifluatex}, \xpackage{letltxmacro},
% \xpackage{ltxcmds}, as well as a~lot~(!) of other useful packages
% (from which I also got everything I know about creating a file in
% \xext{dtx} format, OK, say it: copying),
% and for information about the \pdfTeX{}-bug,
% \textsc{Arno Trautmann} for reporting the bug of misplaced hyperreferences in
% |sidewaystable|s using \LuaTeX{}, independently of the \xpackage{tablefootnote}
% package or footnotes at all,
% \textsc{Thomas~V.} for reporting the bug of not handling the case of
% \xpackage{hyperref} option |hyperfootnotes=false|,
% \textsc{Sverre Stausland Johnsen} for reporting the bug in case of \emph{not}
% using \xpackage{hyperref},
% and the \Newsgroup{comp.text.tex} and \Newsgroup{de.comp.text.tex} newsgroups
% as well as everybody at \url{http://tex.stackexchange.com}
% for their help in all things \TeX{}.
%
% \pagebreak
%
% \phantomsection
% \begin{History}\label{History}
%   \begin{Version}{2011/10/26 v1.0a, tabfootn}
%     \item Upload to \CTAN{macros/latex/contrib/tablefootnote/}.
%   \end{Version}
%   \begin{Version}{2011/10/29 v1.0b}
%     \item Renamed to \xpackage{tablefootnote}.
%     \item Added support for the |sidewaystable|-environment of the
%             \xpackage{rotating} package.
%     \item Diverse changes in documentation and README.
%   \end{Version}
%   \begin{Version}{2011/11/06 v1.0c}
%     \item Replaced |^{\text{\thefootnote}}| by |{}^{\text{\textup{\thefootnote}}}|.
%     \item Bug fix: When \xpackage{etoolbox.sty} was not found, and the \xpackage{float}
%             package had been loaded, and the float specifier was \texttt{H},
%             then the |\tablefootnote|s were not printed for that table. Fixed.
%   \end{Version}
%   \begin{Version}{2011/11/19 v1.0d}
%     \item Replaced |{}^{\text{\textup{\thefootnote}}}| by
%             |\textsuperscript{\thefootnote}|, therefore the \xpackage{amstext}
%             package is no longer required by the \xpackage{tablefootnote} package.
%     \item \xpackage{lscape} and \xpackage{pdflscape} package now work with
%             \xpackage{tablefootnote} (except for using a \texttt{sidewaystable} on
%             a \texttt{landscape} page).
%     \item Added information about footnotes in \texttt{sidewaystable} being printed
%             one below the other, even if \xpackage{footmisc} package with option
%             \texttt{para} is used.
%     \item Redefined |\tfn@swtbox| from
%  \begin{verbatim}
%  \vspace{0.8cm}%
%  \begin{spacing}{0.1}%
%  \subcaptionbox*{}[\linewidth][l]{\tfn@tablefootnoteprintout}%
%  \end{spacing}%
%  \end{verbatim}
%             to
%  \begin{verbatim}
%  \vspace{2\baselineskip}%
%  \parbox[b][][t]{\linewidth}{\raggedright\tfn@tablefootnoteprintout}%
%  \end{verbatim}
%             , therefore the \xpackage{setspace} and \xpackage{subcaption} packages
%             are no longer required by the \xpackage{tablefootnote} package.
%     \item No longer uses |\numexpr|, therefore also usable without \eTeX{}.
%     \item Replaced |\textsuperscript{\thefootnote}| by
%             |\hbox {\@textsuperscript {\normalfont \thefootnote }}|,
%             which is next to the original definition of |\@makefnmark|
%             (|\hbox {\@textsuperscript {\normalfont \@thefnmark }}|,
%             but according to |\show\@thefnmark| here |\@thefnmark| is
%             |{\protect \itshape}| instead of |\thefootnote|).
%     \item Added an error message, when \LuaTeX{}, \xpackage{rotating} package, and
%            \xpackage{hyperref} package are used together. Hyperreferences in
%            |sidewaystable|s are misplaced, independently of the
%            \xpackage{tablefootnote} package or footnotes at all. (Bug reported
%            by \textsc{Arno Trautmann}, 2011/11/15 -- thanks!)
%     \item Added footnotesymbol-footnote-numbers in the example.
%   \end{Version}
%   \begin{Version}{2011/11/26 v1.0e}
%     \item Check for \xpackage{hyperref} is only done once.
%     \item Message regarding the misplacement of hyperlinks by \LuaLaTeX{}
%             changed form error to warning.
%     \item Replaced |\vspace{2\baselineskip}| by
%  \begin{verbatim}
% \ifvoid \footins%
% \vskip 2\baselineskip%
% \else%
% \vskip \skip\footins%
% \fi%
%  \end{verbatim}
%             between the |sidewaystable|s and their footnotes.
%     \item The warning about un-regarded |para|-option of \xpackage{footmoisc}
%             package is only given at the second |tablefootnote| in a |sidewaystable|,
%             not repeatedly for every |tablefootnote| in a |sidewaystable|.
%   \end{Version}
%   \begin{Version}{2012/01/01 v1.0f}
%     \item Bugfix: Wrong installation path given in the documentation, fixed.
%     \item Update of documentation, README, and \xfile{dtx} internals.
%   \end{Version}
%   \begin{Version}{2012/01/14 v1.0g}
%     \item Bugfix: There was a |relax| instead of a |\relax|, fixed.
%   \end{Version}
%   \begin{Version}{2012/07/29 v1.0h}
%     \item \xpackage{tablefootnote} is now compatible to the new
%             \xpackage{footnotebackref} package, 2012/07/01, v1.0,
%             \url{http://www.ctan.org/pkg/footnotebackref}.
%   \end{Version}
%   \begin{Version}{2014/01/08 v1.1a}
%     \item Tablefootnotes now regard optional footnote marks:
%             |\tablefootnote[| custom mark here! |]{...}|
%     \item Bug fix: the case of \xpackage{hyperref} option |hyperfootnotes=false|
%             was not handeled at all. (Bug reported by \textsc{Thomas~V.}, %
%             2014/01/02 -- thanks!)
%     \item A lot of details.
%   \end{Version}
%   \begin{Version}{2014/01/20 v1.1b}
%     \item Bug fix: Package was broken when \xpackage{hyperref} was \emph{not} used.
%            (Bug reported by \textsc{Sverre Stausland Johnsen} -- thanks!)
%   \end{Version}
%   \begin{Version}{2014/01/26 v1.1c}
%     \item The emergency bug fix of version 2014/01/20 v1.1b was replaced by using
%             the new robust macro |\tfn@hyperfootnotes|, which only \emph{internally}
%             uses |\ifHy@hyperfootnotes|. If the \xpackage{hyperref} package is not used,
%             \LaTeX{} now does not see any confusing
%             |\ifHy@hyperfootnotes|\ldots|\else|\ldots|\fi| at all.
%     \item URLs and README updated.
%   \end{Version}
% \end{History}
%
% \bigskip
%
% When you find a mistake or have a suggestion for an improvement of this package,
% please send an e-mail to the maintainer, thanks! (Please see BUG REPORTS in the README.)
%
% \bigskip
%
% \PrintIndex
%
% \Finale
\endinput
%        (quote the arguments according to the demands of your shell)
%
% Documentation:
%    (a) If tablefootnote.drv is present:
%           (pdf)latex tablefootnote.drv
%           makeindex -s gind.ist tablefootnote.idx
%           (pdf)latex tablefootnote.drv
%           makeindex -s gind.ist tablefootnote.idx
%           (pdf)latex tablefootnote.drv
%    (b) Without tablefootnote.drv:
%           (pdf)latex tablefootnote.dtx
%           makeindex -s gind.ist tablefootnote.idx
%           (pdf)latex tablefootnote.dtx
%           makeindex -s gind.ist tablefootnote.idx
%           (pdf)latex tablefootnote.dtx
%
%    The class ltxdoc loads the configuration file ltxdoc.cfg
%    if available. Here you can specify further options, e.g.
%    use DIN A4 as paper format:
%       \PassOptionsToClass{a4paper}{article}
%
% Installation:
%    TDS:tex/latex/tablefootnote/tablefootnote.sty
%    TDS:doc/latex/tablefootnote/tablefootnote.pdf
%    TDS:doc/latex/tablefootnote/tablefootnote-example.tex
%    TDS:doc/latex/tablefootnote/tablefootnote-example.pdf
%    TDS:source/latex/tablefootnote/tablefootnote.dtx
%
%<*ignore>
\begingroup
  \catcode123=1 %
  \catcode125=2 %
  \def\x{LaTeX2e}%
\expandafter\endgroup
\ifcase 0\ifx\install y1\fi\expandafter
         \ifx\csname processbatchFile\endcsname\relax\else1\fi
         \ifx\fmtname\x\else 1\fi\relax
\else\csname fi\endcsname
%</ignore>
%<*install>
\input docstrip.tex
\Msg{******************************************************************}
\Msg{* Installation                                                   *}
\Msg{* Package: tablefootnote 2014/01/26 v1.1c Table foot notes (HMM) *}
\Msg{******************************************************************}

\keepsilent
\askforoverwritefalse

\let\MetaPrefix\relax
\preamble

This is a generated file.

Project: tablefootnote
Version: 2014/01/26 v1.1c

Copyright (C) 2011 - 2014 by
    H.-Martin M"unch <Martin dot Muench at Uni-Bonn dot de>

The usual disclaimer applies:
If it doesn't work right that's your problem.
(Nevertheless, send an e-mail to the maintainer
 when you find an error in this package.)

This work may be distributed and/or modified under the
conditions of the LaTeX Project Public License, either
version 1.3c of this license or (at your option) any later
version. This version of this license is in
   http://www.latex-project.org/lppl/lppl-1-3c.txt
and the latest version of this license is in
   http://www.latex-project.org/lppl.txt
and version 1.3c or later is part of all distributions of
LaTeX version 2005/12/01 or later.

This work has the LPPL maintenance status "maintained".

The Current Maintainer of this work is H.-Martin Muench.

This work consists of the main source file tablefootnote.dtx,
the README, and the derived files
   tablefootnote.sty, tablefootnote.pdf,
   tablefootnote.ins, tablefootnote.drv,
   tablefootnote-example.tex, tablefootnote-example.pdf.

In memoriam Tommy Muench + 2014/01/02.

\endpreamble
\let\MetaPrefix\DoubleperCent

\generate{%
  \file{tablefootnote.ins}{\from{tablefootnote.dtx}{install}}%
  \file{tablefootnote.drv}{\from{tablefootnote.dtx}{driver}}%
  \usedir{tex/latex/tablefootnote}%
  \file{tablefootnote.sty}{\from{tablefootnote.dtx}{package}}%
  \usedir{doc/latex/tablefootnote}%
  \file{tablefootnote-example.tex}{\from{tablefootnote.dtx}{example}}%
}

\catcode32=13\relax% active space
\let =\space%
\Msg{************************************************************************}
\Msg{*}
\Msg{* To finish the installation you have to move the following}
\Msg{* file into a directory searched by TeX:}
\Msg{*}
\Msg{*     tablefootnote.sty}
\Msg{*}
\Msg{* To produce the documentation run the file `tablefootnote.drv'}
\Msg{* through (pdf)LaTeX, e.g.}
\Msg{*  pdflatex tablefootnote.drv}
\Msg{*  makeindex -s gind.ist tablefootnote.idx}
\Msg{*  pdflatex tablefootnote.drv}
\Msg{*  makeindex -s gind.ist tablefootnote.idx}
\Msg{*  pdflatex tablefootnote.drv}
\Msg{*}
\Msg{* At least three runs are necessary e.g. to get the}
\Msg{*  references right!}
\Msg{*}
\Msg{* Happy TeXing!}
\Msg{*}
\Msg{************************************************************************}

\endbatchfile
%</install>
%<*ignore>
\fi
%</ignore>
%
% \section{The documentation driver file}
%
% The next bit of code contains the documentation driver file for
% \TeX{}, i.\,e., the file that will produce the documentation you
% are currently reading. It will be extracted from this file by the
% \texttt{docstrip} programme. That is, run \LaTeX{} on \texttt{docstrip}
% and specify the \texttt{driver} option when \texttt{docstrip}
% asks for options.
%
%    \begin{macrocode}
%<*driver>
\NeedsTeXFormat{LaTeX2e}[2011/06/27]
\ProvidesFile{tablefootnote.drv}%
  [2014/01/26 v1.1c Table foot notes (HMM)]
\documentclass{ltxdoc}[2007/11/11]% v2.0u
\usepackage{holtxdoc}[2012/03/21]%  v0.24
%% tablefootnote may work with earlier versions of LaTeX2e and those
%% class and package, but this was not tested.
%% Please consider updating your LaTeX, class, and package
%% to the most recent version (if they are not already the most
%% recent version).
\hypersetup{%
 pdfsubject={Footnotes in Tables (HMM)},%
 pdfkeywords={LaTeX, tablefootnote, footnote, table, H.-Martin Muench},%
 pdfencoding=auto,%
 pdflang={en},%
 breaklinks=true,%
 linktoc=all,%
 pdfstartview=FitH,%
 pdfpagelayout=OneColumn,%
 bookmarksnumbered=true,%
 bookmarksopen=true,%
 bookmarksopenlevel=3,%
 pdfmenubar=true,%
 pdftoolbar=true,%
 pdfwindowui=true,%
 pdfnewwindow=true%
}
\CodelineIndex
\hyphenation{printing docu-ment}
\gdef\unit#1{\mathord{\thinspace\mathrm{#1}}}%
\makeatletter
\@ifundefined{eTeX}{\gdef\eTeX{$\m@th \varepsilon $-\TeX }}{% else \relax
}
\makeatother
\begin{document}
  \DocInput{tablefootnote.dtx}%
\end{document}
%</driver>
%    \end{macrocode}
%
% \fi
%
% \CheckSum{584}
%
% \CharacterTable
%  {Upper-case    \A\B\C\D\E\F\G\H\I\J\K\L\M\N\O\P\Q\R\S\T\U\V\W\X\Y\Z
%   Lower-case    \a\b\c\d\e\f\g\h\i\j\k\l\m\n\o\p\q\r\s\t\u\v\w\x\y\z
%   Digits        \0\1\2\3\4\5\6\7\8\9
%   Exclamation   \!     Double quote  \"     Hash (number) \#
%   Dollar        \$     Percent       \%     Ampersand     \&
%   Acute accent  \'     Left paren    \(     Right paren   \)
%   Asterisk      \*     Plus          \+     Comma         \,
%   Minus         \-     Point         \.     Solidus       \/
%   Colon         \:     Semicolon     \;     Less than     \<
%   Equals        \=     Greater than  \>     Question mark \?
%   Commercial at \@     Left bracket  \[     Backslash     \\
%   Right bracket \]     Circumflex    \^     Underscore    \_
%   Grave accent  \`     Left brace    \{     Vertical bar  \|
%   Right brace   \}     Tilde         \~}
%
% \GetFileInfo{tablefootnote.drv}
%
% \begingroup
%   \def\x{\#,\$,\^,\_,\~,\ ,\&,\{,\},\%}%
%   \makeatletter
%   \@onelevel@sanitize\x
% \expandafter\endgroup
% \expandafter\DoNotIndex\expandafter{\x}
% \expandafter\DoNotIndex\expandafter{\string\ }
% \begingroup
%   \makeatletter
%     \lccode`9=32\relax
%     \lowercase{%^^A
%       \edef\x{\noexpand\DoNotIndex{\@backslashchar9}}%^^A
%     }%^^A
%   \expandafter\endgroup\x
%
% \DoNotIndex{\\,\,}
% \DoNotIndex{\documentclass,\usepackage,\ProvidesPackage,\begin,\end}
% \DoNotIndex{\MessageBreak,\message}
% \DoNotIndex{\NeedsTeXFormat,\DoNotIndex,\verb}
% \DoNotIndex{\def,\edef,\gdef,\global}
% \DoNotIndex{\ifx,\listfiles,\mathord,\mathrm}
% \DoNotIndex{\kvoptions,\SetupKeyvalOptions,\ProcessKeyvalOptions}
% \DoNotIndex{\bigskip,\space,\thinspace,\ldots}
% \DoNotIndex{\indent,\noindent,\newline,\linebreak,\pagebreak}
% \DoNotIndex{\textbf,\textit,\textsf,\texttt,\textsc,\textquotedblleft,\textquotedblright}
% \DoNotIndex{\plainTeX,\TeX,\LaTeX,\pdfLaTeX}
% \DoNotIndex{\chapter,\section,\the,\csname,\endscname}
% \DoNotIndex{\value,\arabic,\hline,\text,\caption,\centering,\clear,\url}
%
% \title{The \xpackage{tablefootnote} package}
% \date{2014/01/26 v1.1c}
% \author{H.-Martin M\"{u}nch\\\xemail{Martin.Muench at Uni-Bonn.de}}
%
% \maketitle
%
% \begin{abstract}
% This \LaTeX{} package provides the command |\tablefootnote| to be used
% in a \texttt{table} or \texttt{sidewaystable} environment,
% where |\footnote| will not work and when using |\footnotemark| and
% |\footnotetext| and adjusting the counters (including \texttt{Hfootnote})
% manually is either too much work or would not even work (\texttt{sidewaystable}).
% \end{abstract}
%
% \bigskip
%
% \noindent Disclaimer for web links: The author is not responsible for any contents
% referred to in this work unless he has full knowledge of illegal contents.
% If any damage occurs by the use of information presented there, only the
% author of the respective pages might be liable, not the one who has referred
% to these pages.
%
% \bigskip
%
% \noindent {\color{green} Save per page about $200\unit{ml}$ water,
% $2\unit{g}$ CO$_{2}$ and $2\unit{g}$ wood:\\
% Therefore please print only if this is really necessary.}
%
% \pagebreak
%
% \tableofcontents
%
% \pagebreak
%
% \section{Introduction\label{sec:Introduction}}
%
% \indent This \LaTeX{} package provides the command |\tablefootnote{...}| to be
% used in a \texttt{table} or \texttt{sidewaystable} environment. In those
% environments |\footnote{...}| would not work and would need to be replaced
% by |\footnotemark| in the \texttt{(sideways)\allowbreak table} and
% |\footnotetext{...}| after the \texttt{(sideways)table} environment ended.
% Additionally the counters for \texttt{footnote} and
% (when the \xpackage{hyperref} package is used) \texttt{Hfootnote}
% would need to be adjusted manually. Then still the hyperlinks
% in a \texttt{(sideways)table} would not work.
% When the \xpackage{footnotebackref} package is used,
% |\footnotemark| and |\footnotetext{...}| would need to be redefined.
% Just using |\tablefootnote{...}| (with optional argument for custom foot note
% marks) in the (sideways)table does all this automatically.
% (Compatibility with the \xpackage{footnotebackref} package
% was achieved without redefining |\footnotemark| or |\footnotetext|.)
% Redefining |\footnote| was \emph{not} done as to not create conflicts with
% other packages.
%
% \section{Usage}
%
% \indent Just load the package placing
% \begin{quote}
%   |\usepackage{tablefootnote}|
% \end{quote}
% \noindent in the preamble of your \LaTeXe{} source file and use\newline
% |\tablefootnote{...}| instead of |\footnote{...}| and\newline
% |\tablefootnote[...]{...}| instead of |\footnote[...]{...}|
% in (sideways)\allowbreak tables.\\
% When the \xpackage{footnotebackref} package is used,
% it must be loaded before the \xpackage{tablefootnote} package
% and the \xpackage{hyperrefpackage} with option |hyperfootnotes=true|
% must be loaded, too. (Backreference{-}links without links do not make sense.)\\
% When the \xpackage{rotating} package (for |sidewaystable|s) is used,
% it must be loaded before the \xpackage{tablefootnote} package and it must be
% a recent version (v2.16a, 2009/03/28, or newer).\\
% If |\tablefootnote{...}| is used in a tabular environment, which is inside
% of a (sideways)table environment, everything is fine, but when the tabular
% environment is not inside of a (sideways)table, the \textbf{table}footnote
% will not work.\\
%
% Pages with tables with footnotes, which are rotated with the \xpackage{lscape}-
% or with the \xpackage{pdflscape}-package (|\begin{landscape}|, table, |\end{landscape}|),
% are possible. But when the \xpackage{footmisc}-package is used, \xpackage{(pdf)lscape}
% should be loaded before \xpackage{footmisc}.\\
%
% When the \xpackage{footmisc}-package is used with option \texttt{para},
% this is regarded with the exception of a \texttt{sidewaystable}. There the footnotes
% are printed one below the other. (Well, this is better than no footnote in a
% \texttt{sidewaystable} at all, isn't it?)\\
%
% When the (sideways)table floats, the footnotes are set and (when \xpackage{hyperref}
% is used) hyperlinked, but they are not automatically adapted when the table floats
% over/under another footnote. Thus either do not use a footnote between original and
% float position of the (sideways)table, or place the (sideways)table in
% \textquotedblleft here\textquotedblright\ position. |\clear(double)page|, |h(!)|,
% |H|~from the \xpackage{float} package (\url{http://www.ctan.org/pkg/float}),
% or |\FloatBarrier| from the \xpackage{picins} package
% (\url{http://www.ctan.org/pkg/picins}) might help, too.
% (Or after finishing the document, move the (sideways)table in the source code
% near the position where it floats to or use the optional footnote marks.)
%
% \pagebreak
%
% \section{Alternatives\label{sec:Alternatives}}
%
% \begin{description}
% \item[-] The \xpackage{longtable} package provides the \texttt{longtable} environment
%   as replacement for the combined \texttt{table} and \texttt{tabular} environments.
%   Footnotes are real footnotes (not just tablenotes), are continuously numbered and
%   hyperlinked (when using the \xpackage{hyperref} package), and the hyperlinks really
%   work. As drawback the appearance of the caption changes slightly (e.\,g.~distance
%   to the table, width of the caption), which can probably be changed back manually.
%   Furthermore, longtables are meaned to break over more than one page. If that is not
%   wished, it must be prevented by |\nopagebreak|-commands and by ending the longtable
%   lines with |\\*| instead of |\\|. longtables do not float. (Therefore using the
%   \xpackage{tablefootnote} package and |\FloatBarrier| from the \xpackage{picins} package
%   before and after the table environment is similar~- but \xpackage{tablefootnote} does not
%   change the table-caption!) \texttt{sidewaystable} does not work with it.\\
%   \url{http://www.ctan.org/pkg/longtable}
%
% \item[-] The \xpackage{supertabular} package provides the \texttt{mpsupertabular}
%   environment as replacement for the combined \texttt{table} and \texttt{tabular}
%   environments. Footnotes are just tablenotes (with working hyperlinks when using the
%   \xpackage{hyperref} package), i.\,e. numbered a, b, c and placed below the table
%   and not at the end of the page. Therefore there is no float problem (because the
%   tablenotes numbering is not included in the continuous numbering of the footnotes).
%   Placing the \texttt{supertabular} inside of a \texttt{sidewaystable} breaks the
%   hyperlinks to the tablenotes.\\
%   \url{http://www.ctan.org/pkg/supertabular}
%
% \item[-] The \xpackage{ctable} package has its very own notation for defining tables.
%   It can create tablenotes and sideways-tables. The tablenotes are not automatically
%   hyperlinked. The \texttt{ctable}s float. Because the tablenotes numbering is not
%   included in the continuous numbering of the footnotes there is no float problem.\\
%   \url{http://www.ctan.org/pkg/ctable}
%
% \item[-] The \xpackage{footnote} package provides |\makesavenoteenv{table}|. After
%   loading the package and using that command in the preamble, in tables |\footnote{...}|
%   can be used. Using |\makesavenoteenv{tabular}| and \\
%   |\makesavenoteenv{sidewaystable}|
%   is possible, but it neither solves the float problem, nor do the created hyperlinks
%   work (i.\,e.~they aim at wrong locations). The \xpackage{mdwtab} from the same bundle
%   is incompatible with other table-related packages (e.\,g. \xpackage{supertabular},
%   \xpackage{array}) and not $100\,\%$ compatible with the \texttt{tabular} environment.\\
%   \url{http://www.ctan.org/pkg/footnote}\\
%   \url{http://www.ctan.org/pkg/mdwtab}
%
% \item[-] The \xpackage{tabularx} package does produce footnotes for sidewaystables,
%   but uses a, b, c instead of 1, 2, 3. The hyperlinks to the footnotes do not work.
%   Because the footnotes numbering is not included in the continuous numbering of
%   the other footnotes there is no float problem.\\
%   \url{http://www.ctan.org/pkg/tabularx}
%
% \item[-] Placing a tabular inside a minipage inside a table produces tablenotes.
%   Therefore there is no float problem (because the footnotes are not continuously
%   numbered). The hyperlinks to the table notes indeed work.
%
% \pagebreak
%
% \item[-] The \xpackage{threeparttable} package creates tablenotes again. Therefore
%   there is no float problem (because the tablenotes are not continuously numbered with
%   the footnotes). There are no hyperlinks to the table notes (at least not automatically).
%   Using |sidewaystable| (with table notes) works.\\
%   \url{http://www.ctan.org/pkg/threeparttable}
%
% \item[-] The \xpackage{threeparttablex} package creates tablenotes again. Therefore there
%   is no float problem (because the tablenotes are not continuously numbered with the
%   footnotes). With option \texttt{referable} the tablenotes are hyperlinked. Use of
%   a |sidewaystable| (with table notes) did not work for me. When using the
%   \texttt{referable} option according to the example in the \xpackage{threeparttablex}
%   manual the \xpackage{longtable} package is used, therefore that package could be used
%   directly without \xpackage{threeparttablex} (see above).\\
%   \url{http://www.ctan.org/pkg/threeparttablex}
%
% \item[-] One can manually use |\footnotemark| in the table and |\footnotetext{...}|
%  after the table environment ended and manually change the |footnote| and
%  (when the \xpackage{hyperref} package is used) |Hfootnote| counters and needs
%  to insert
%  \begin{verbatim}
%     \makeatletter
%     \global\let\Hy@saved@currentHref\@currentHref
%     \hyper@makecurrent{Hfootnote}%
%     \global\let\Hy@footnote@currentHref\@currentHref
%     \global\let\@currentHref\Hy@saved@currentHref
%     \makeatother
%  \end{verbatim}
%  before each |\footnotetext{...}|, but with more than very few footnotes this can become
%  much work (and is prone to produce errors). And this does not even work with
%  sidewaystables. (And it does not work with the \xpackage{footnotebackref} package
%  without redefining |\footnotemark| and |\footnotetext{...}|.)
% \end{description}
%
% \bigskip
%
% \noindent (You programmed or found another alternative,
%  which is available at \CTAN{}?\\
%  OK, send an e-mail to me with the name, location at \CTAN{},
%  and a short notice, and I will probably include it in
%  the list above.)
%
% \newpage
%
% \section{Example}
%
%    \begin{macrocode}
%<*example>
\documentclass[british]{article}[2007/10/19]% v1.4h
%%%%%%%%%%%%%%%%%%%%%%%%%%%%%%%%%%%%%%%%%%%%%%%%%%%%%%%%%%%%%%%%%%%%%
\usepackage{float}[2001/11/08]%    v1.3d
\usepackage{placeins}[2005/04/18]% v2.2  ; for \FloatBarrier
\usepackage{rotating}[2009/03/28]% v2.16a; for sidewaystable-environment
\usepackage[%
 hyperfootnotes=true,%
 extension=pdf,%
 plainpages=false,%
 pdfpagelabels=true,%
 hyperindex=false,%
 pdflang={en},%
 pdftitle={tablefootnote package example},%
 pdfauthor={H.-Martin Muench},%
 pdfsubject={Example for the tablefootnote package},%
 pdfkeywords={LaTeX, tablefootnote, footnote, table, H.-Martin Muench},%
 % pdfview=FitH and FitBH do not work: hyperlinks in sidewaystables
 % do no lead to the footnotes, due to a bug in pdfTeX,
 % computing wrong anchor coordinates (Heiko Oberdiek, 29. October 2011)
 % pdfview=Fit, FitV, FitR, FitB, FitBV work
 % print is OK for all those options
 pdfstartview=FitH,%
 pdfpagelayout=OneColumn%
]{hyperref}[2012/11/06]% v6.83m
 % Due to the urls used in the example, either the hyperref or the url
 % package are needed (or the urls must be removed before compiling).

\usepackage{footnotebackref}[2012/07/01]% v1.0
\usepackage{tablefootnote}[2014/01/26]%   v1.1c

\gdef\unit#1{\mathord{\thinspace\mathrm{#1}}}%
\listfiles
\begin{document}
\pagenumbering{arabic}
\section*{Example for tablefootnote}

This example demonstrates the use of package\newline
\textsf{tablefootnote}, v1.1c as of 2014/01/26 (HMM).\newline
There were no options used. (The package provides no options.)\newline

\textbf{The \texttt{tablefootnote-example.tex} needs to be compiled
at least twice to get the references right!}\newline

If the etoolbox-package is found, it is automatically used.\newline

For more details please see the documentation!\newline

\noindent Save per page about $200\unit{ml}$ water,
$2\unit{g}$ CO$_{2}$ and $2\unit{g}$ wood:\newline
Therefore please print only if this is really necessary.\newline

Here is some text.\footnote{And this is a text footnote.}\newline

Tables \ref{tab.symbol}, \ref{tab.normal}, \ref{tab.another} and
\ref{tab.floatH} show normal tables, table~\ref{tab.sideways} depicts
a sidewaystable. Table~\ref{tab.floatH} uses the float
specifier~\texttt{H} from the float package.\newline

\texttt{Hyperref} option \verb|pdfview=FitH| and \verb|FitBH| do
not work due to a bug in pdf\TeX{}, computing wrong
anchor coordinates (\textsc{Heiko Oberdiek}, 29.~October 2011).
Depending on used pdf-viewer, hyperlinks in sidewaystables lead
e.\,g.~at the end of the document, not at the footnote.
\verb|pdfview=Fit|, \verb|FitV|, \verb|FitR|, \verb|FitB|,
\verb|FitBV| work, print is OK for all those options.

\bigskip

\listoftables

\pagebreak

\renewcommand{\thefootnote}{\fnsymbol{footnote}}
\verb|\renewcommand{\thefootnote}{\fnsymbol{footnote}}|
causes footnote{-}symbol{-}footnotes, which are possible
(see Table~\ref{tab.symbol}).

\begin{table}
\centering
\begin{tabular}{ccc}
Another\tablefootnote{A table footnote.} & %
text\tablefootnote{Another table footnote.} & %
in a table\tablefootnote{A third table footnote.}
\end{tabular}
\caption[A footnotesymbol table]{%
A table with footnote-symbol-footnotes.\label{tab.symbol}}
\end{table}

Some text.\footnote{A text footnote.}

\renewcommand{\thefootnote}{\arabic{footnote}}
\verb|\renewcommand{\thefootnote}{\arabic{footnote}}|
switches back to normal footnote numbers again.

\pagebreak

\begin{table}
\centering
\begin{tabular}{ccc}
Some\tablefootnote{A table footnote.} & %
text\tablefootnote[99]{A table footnote with custom footnote number.} & %
in a table\tablefootnote{A third table footnote.}
\end{tabular}
\caption[A table]{A normal table.\label{tab.normal}}
\end{table}

Some text.\footnote{Another text footnote.}

\pagebreak

%    \end{macrocode}
%
% \pagebreak
%
%    \begin{macrocode}
More text.\footnote{And yet another text footnote.}

\begin{table}[t]
\centering
\begin{tabular}{|c|c|c|}
\hline
Another\tablefootnote{A $3^{rd}$ table footnote.} & %
text\tablefootnote{Another $3^{rd}$ table footnote.} & %
in a table\tablefootnote{A $3^{rd}$ third table footnote.}\\ \hline
\end{tabular}
\caption[Another table]{Another table (third one)\label{tab.another}}
\end{table}

Please note that Table~\ref{tab.another} floated to the top of the page.
While the footnotes are set and (when hyperref is used) hyperlinked,
they are not automatically adapted. Thus either do not use a footnote
at the same page before the table, or place the table in
\textquotedblleft here\textquotedblright\ or
\textquotedblleft bottom\textquotedblright\ position.
\verb|\clear(double)page|, \verb|h(!)|, \verb|H|~from the
\texttt{float} package, or \verb|\FloatBarrier| from the
\texttt{picins} package might help, too. (Or move the table in the
source code near the position where it floats to
or use the optional footnote marks.)

Table~\ref{tab.floatH} (page~\pageref{tab.floatH}) uses float specifier
\texttt{H} from the float package and does not float.

Some text.\footnote{This is just another text footnote.}

\pagebreak

\FloatBarrier

\begin{sidewaystable}
\centering%
\begin{tabular}{ccc}
Text\tablefootnote{Please rotate the view for testing the %
hyperlinks.} & %
in a\tablefootnote[98]{Another sidewaystable footnote %
with optional footnote mark.} & %
sidewaystable%
\tablefootnote{Sidewaystable-environment provided by %
\url{http://www.ctan.org/pkg/rotating} package.}
\end{tabular}
\caption[A sideways table]{A table in the \texttt{sideways} %
environment\label{tab.sideways}}%
\end{sidewaystable}%

\FloatBarrier

\pagebreak

%    \end{macrocode}
%
% \pagebreak
%
%    \begin{macrocode}
A last table, here with float specifier \texttt{H} from the
float\footnote{\url{http://www.ctan.org/pkg/float}} package.

\begin{table}[H]
\centering
\begin{tabular}{ccc}
Another\tablefootnote{A $5^{th}$ table footnote.} & %
text\tablefootnote{Another $5^{th}$ table footnote.} & %
in a table\tablefootnote{A $5^{th}$ third table footnote.}
\end{tabular}
\caption[A last table]{A very last table\label{tab.floatH}}
\end{table}

Some text.\footnote{This is just another text footnote.}

\pagebreak

End of the example for the
tablefootnote\footnote{\url{http://www.ctan.org/pkg/tablefootnote}}
package.

\end{document}
%</example>
%    \end{macrocode}
%
% \StopEventually{}
%
% \newpage
%
% \section{The implementation}
%
% We start off by checking that we are loading into \LaTeXe{} and
% announcing the name and version of this package.
%
%    \begin{macrocode}
%<*package>
%    \end{macrocode}
%
%    \begin{macrocode}
\NeedsTeXFormat{LaTeX2e}[2011/06/27]
\ProvidesPackage{tablefootnote}[2014/01/26 v1.1c
              Table foot notes (HMM)]
%    \end{macrocode}
%
% A short description of the \xpackage{tablefootnote} package:
%
%    \begin{macrocode}
%% Provides the \tablefootnote[...]{...} command
%% for footnotes in (sideways)tables.

%    \end{macrocode}
%
% We need the \xpackage{ltxcmds} and \xpackage{letltxmacro} packages
% by \textsc{Heiko Oberdiek}:
%
%    \begin{macrocode}
\RequirePackage{ltxcmds}[2011/11/09]%     v1.22
\RequirePackage{letltxmacro}[2010/09/02]% v1.4
\RequirePackage{xifthen}[2009/04/17]%     v1.3
%    \end{macrocode}
%
% A last information for the user:
%
%    \begin{macrocode}
%% tablefootnote may work with earlier versions of LaTeX and those
%% packages, but this was not tested. Please consider updating
%% your LaTeX and packages to the most recent version
%% (if they are not already the most recent version).

%    \end{macrocode}
%
% \indent See subsection~\ref{ss:Downloads} about how to get them.\\
%
% When the \xpackage{rotating} package (for |sidewaystable|s) is used,
% it must be loaded before the \xpackage{tablefootnote} package and it must be
% a recent version:
%
%    \begin{macrocode}
\@ifpackageloaded{rotating}{%
  \gdef\tfn@rotating{1}
  \@ifpackagelater{rotating}{2009/03/28}% v2.16a
   {% >= 2009/03/28, OK
   }{% else
     \PackageWarningNoLine{tablefootnote}{%
       It is required version\MessageBreak%
       2009/03/28 v2.16a (or later) of package rotating,\MessageBreak%
       but only version\MessageBreak%
      `\csname ver@rotating.sty\endcsname'\MessageBreak%
      is available}%
   }%
 }{% else
   \gdef\tfn@rotating{0}%
 }

%    \end{macrocode}
%
% There are no options to be processed.\\
%
% We need some definitions:
%
%    \begin{macrocode}
\def\tfn@footnotetablecount{0}
\def\tfn@footnotetableprint{0}
\def\tfn@footnotezero{0}
\def\tfn@fnt{0}
\def\tfn@swt{0}%       _s_ide_w_ays_t_able
\def\tfn@fmpw{\relax}% f_oot_m_isc_p_ara_w_arn-switch
\def\tfn@fnbr{0}%      f_oot_n_ote_b_ack_r_ef(erence package)

%    \end{macrocode}
%
% \begin{macro}{\tfn@hyperfootnotes}
% We need a wrapper for the used |\ifHy@hyperfootnotes|,
% which is not defined when the \xpackage{hyperref} package has not been loaded:
%
%    \begin{macrocode}
\DeclareRobustCommand{\tfn@hyperfootnotes}[2]{%
  \ifHy@hyperfootnotes%
    #1%
  \else%
    #2%
  \fi%
  }

%    \end{macrocode}
% \end{macro}
%
% \begin{macro}{\tablemakefntext}
% |\tablemakefntext| makes the footnotetext for a footnote in a table.
%
%    \begin{macrocode}
\ltx@ifpackageloaded{footnotebackref}{%
  \gdef\tfn@fnbr{1}%
  \ifx\FootnoteBackref@symbol\empty%
    \ifFootnoteBackref@numberlinked%
      \newcommand\tablemakefntext[1]{%
        \noindent\makebox[1.634em][r]{%
         \mbox{\textsuperscript{\normalfont%
           \hyperlink{tfn:m\theHfootnote}{\@thefnmark}%
        }}}%
        \ #1%
      }
    \else
      \newcommand\tablemakefntext[1]{%
        \noindent\makebox[1.8em][r]{%
          \mbox{\textsuperscript{\normalfont%
            \@thefnmark%
        }}}%
        #1%
      }
    \fi
  \else
    \newcommand\tablemakefntext[1]{%
      \noindent\makebox[1.8em][r]{%
        \mbox{\textsuperscript{\normalfont%
          \ifFootnoteBackref@numberlinked%
            \hyperlink{tfn:m\theHfootnote}{\@thefnmark}\ %
          \else%
            \@thefnmark%
          \fi%
      }}}%
      \hyperlink{tfn:m\theHfootnote}{\FootnoteBackref@symbol}%
      #1%
    }
  \fi
}{%
  \newcommand\tablemakefntext[1]{%
    \noindent\makebox[1.8em][r]{%
      \mbox{\textsuperscript{\normalfont%
        \@thefnmark%
    }}}%
    #1%
  }
}

%    \end{macrocode}
%
% \end{macro}
%
% \pagebreak
%
% \begin{macro}{\swtablemakefntext}
%    \begin{macrocode}
\newcommand\swtablemakefntext[1]{\noindent\ #1}

%    \end{macrocode}
% \end{macro}
%
% \begin{macro}{\tfn@footnotemarks}
% |\tfn@footnotemarks| uses the |\footnotemark| command to place a foot note mark
% and decreases the |\tfn@footnotetablecount| (i.\,e.~increases the negative value)
% of the number of footnote(marks) in the table. If the \xpackage{footnotebackref}
% package is used, it is also necessary to create an according |\hypertarget| for
% the back reference.
%
%    \begin{macrocode}
\newcommand{\tfn@footnotemarks}[1][]{%
  \hbox{%
    \ifthenelse{\isempty{#1}}%
      {\footnotemark%
       \ltx@ifpackageloaded{footnotebackref}%
         {\raisebox{\ht\strutbox}{\hypertarget{tfn:m\theHfootnote}{}}}%
         {% footnotebackref not loaded
         }%
      }%
      {\ltx@ifpackageloaded{hyperref}%
        {\tfn@hyperfootnotes{%
            \hyperlink{tfn:#1}{\footnotemark[#1]}%
          }{\footnotemark[#1]}}%
        {\footnotemark[#1]}%
       \ltx@ifpackageloaded{footnotebackref}%
        {\raisebox{\ht\strutbox}{\hypertarget{tfn:m#1}{}}}{}%
      }%
   }%
  \@tempcnta=\tfn@footnotetablecount\relax%
  \advance\@tempcnta by -1\relax%
  \xdef\tfn@footnotetablecount{\the\@tempcnta}%
}

%    \end{macrocode}
% \end{macro}
%
% \begin{macro}{\tfn@footmiscparawarn}
% When the \xpackage{footmisc} package is used with option |para|,
% we need to issue a warning. (Any idea how to resolve this issue?)\\
% It must be given inside of it's own command, because |\ifFN@para| is not
% defined when \xpackage{footmisc} has not been loaded, and then the
% |\fi| would be a |\fi| without |\if| (and \TeX{} does not like single
% |\fi|s).
%
%    \begin{macrocode}
\newcommand{\tfn@footmiscparawarn}{%
  \ifFN@para%
    \PackageWarning{tablefootnote}{%
      Package footmisc with option para used.\MessageBreak%
      Unfortunately package tablefootnote is not yet\MessageBreak%
      able to regard this in sidewaystables.\MessageBreak%
      Here the footnotes are printed one below\MessageBreak%
      the other. Sorry!\MessageBreak%
     }%
  \fi%
  }

%    \end{macrocode}
% \end{macro}
%
% \begin{macro}{\tfn@updateHyper}
% |\tfn@updateHyper| updates some \xpackage{hyperref} internals.
%
%    \begin{macrocode}
\newcommand{\tfn@updateHyper}{%
  % The code of this command is partly from the
  % hyperref package by Heiko Oberdiek, 2011/10/01, v6.82j.
  % Newer versions of that package are available at CTAN.org.
  \global\let\Hy@saved@currentHref\@currentHref%
  \tfn@hyperfootnotes{\hyper@makecurrent{Hfootnote}}{}%
  \global\let\Hy@footnote@currentHref\@currentHref%
  \global\let\@currentHref\Hy@saved@currentHref%
  }

%    \end{macrocode}
% \end{macro}
%
% \begin{macro}{\tfn@footnotetext}
% |\tfn@footnotetext| contains the |\footnotetext[...]{...}| command and handles
% the numbering. |\tfn@footnotetext| will be executed after the table.
% The foot note counter has been increased automatically with each |\footnotemark|,
% therefore it must be reset to its value before the table began.
%
%    \begin{macrocode}
\newcommand{\tfn@footnotetext}[2][]{%
  \ifx\tfn@footnotetableprint\tfn@footnotezero%
    % This is the first footnotetext after the (sideways)table.
    \addtocounter{footnote}{\tfn@footnotetablecount}%
%    \end{macrocode}
%
% When the \xpackage{hyperref} package is used, also its |Hfootnote| counter must
% be reset (when the footnotes are hyperlinked).
%
%    \begin{macrocode}
    \ltx@ifpackageloaded{hyperref}{%
      \tfn@hyperfootnotes{%
        \addtocounter{Hfootnote}{\tfn@footnotetablecount}}{}%
      \tfn@updateHyper%
     }{}%
  \fi%
%    \end{macrocode}
%
% For issuing the text for the next footnote the footnote-counter
% (and |Hfootnote| counter in case of links) are increased by one.
%
%    \begin{macrocode}
  \addtocounter{footnote}{+1}%
  \ltx@ifpackageloaded{hyperref}{%
    \tfn@hyperfootnotes{\addtocounter{Hfootnote}{+1}}{}%
%    \end{macrocode}
%
% The \xpackage{hyperref} package was not expecting this kind of change of the
% |(H)footnote| numbers, therefore some code-lines are missing here which can be
% found at other places in that same package and are called by this command.
%
%    \begin{macrocode}
    \tfn@updateHyper%
%    \end{macrocode}
%
% When \xpackage{hyperref} is not loaded, nothing is to be done here.
%
%    \begin{macrocode}
   }{}%
%    \end{macrocode}
%
% The |\footnotetext[...]{...}| places the foot note text as usual.
%
%    \begin{macrocode}
  \ifx\tfn@swt\tfn@footnotezero%
    \ifthenelse{\isempty{#1}}%
      {\footnotetext{#2}}%
      {\ltx@ifpackageloaded{hyperref}%
         {\tfn@hyperfootnotes{%
             \footnotetext[#1]{%
               \raisebox{\ht\strutbox}{\hypertarget{tfn:#1}{}}#2}%
           }{\footnotetext[#1]{#2}%
           }%
         }{\footnotetext[#1]{#2}}%
      }%
  \else%
%    \end{macrocode}
%
% If there is more than one tablefootnote in the |sidewaystable|, \\
% the |\tfn@footmiscparawarn| shall be given,
%
%    \begin{macrocode}
    \ifx\tfn@footnotetableprint\tfn@footnotezero%
      \gdef\tfn@fmpw{\tfn@footmiscparawarn}%
    \else%
      \ltx@ifpackageloaded{footmisc}{\tfn@fmpw}{\relax}%
%    \end{macrocode}
%
% but we do not want to repeat the message for a third(, forth,\ldots)
% tablefootnote in the |sidewaystable|.
%
%    \begin{macrocode}
      \gdef\tfn@fmpw{\relax}%
    \fi%
%    \end{macrocode}
%
% |\@makefnmark| is \\
% |\hbox {\@textsuperscript {\normalfont \@thefnmark }}|, \\
% but according to |\show\@thefnmark| here |\@thefnmark| is
% |{\protect \itshape}| instead of |\thefootnote| (Why?!).\\
% If the \xpackage{footnotebackref} package is used,
% back references must be created for the table footnotes, too.
%
%    \begin{macrocode}
    \ltx@ifpackageloaded{footnotebackref}{%
      \LetLtxMacro{\tfncurrentmft}{\@makefntext}%
      \LetLtxMacro{\@makefntext}{\swtablemakefntext}%
      \ifthenelse{\isempty{#1}}%
        {\footnotetext{%
           \raisebox{\ht\strutbox}{\hypertarget{tfn:\theHfootnote}{}}%
           \hbox{\@textsuperscript{\normalfont%
              \hyperlink{tfn:m\theHfootnote}{\thefootnote}}}\thinspace%
         \ifx\FootnoteBackref@symbol\empty%
         \else\hyperlink{tfn:m\theHfootnote}{\FootnoteBackref@symbol}%
         \fi%
         #2}%
       }{\edef\tfn@tfn{\arabic{footnote}}%
         \edef\tfn@thn{\arabic{Hfootnote}}%
         \setcounter{footnote}{#1}%
         \setcounter{Hfootnote}{#1}%
         \tfn@updateHyper%
         \footnotetext{%
           \raisebox{\ht\strutbox}{\hypertarget{tfn:#1}{}}%
           \hbox{\@textsuperscript{\normalfont%
                  \hyperlink{tfn:m#1}{#1}}}%
           \thinspace%
           \ifx\FootnoteBackref@symbol\empty%
           \else%
             \raisebox{\ht\strutbox}{%
               \hyperlink{tfn:m#1}{\FootnoteBackref@symbol}}%
           \fi%
           #2%
          }%
         \setcounter{footnote}{\tfn@tfn}%
         \setcounter{Hfootnote}{\tfn@thn}%
         \tfn@updateHyper%
       }%
      \LetLtxMacro{\@makefntext}{\tfncurrentmft}%
     }{% footnotebackref not loaded
       \ifthenelse{\isempty{#1}}%
         {\footnotetext{%
            \hbox{\@textsuperscript{\normalfont\thefootnote}}%
            \ltx@ifpackageloaded{hyperref}%
              {\tfn@hyperfootnotes{%
                 \raisebox{\ht\strutbox}{%
                   \hypertarget{tfn:\theHfootnote}{}}}{}%
              }{}%
            \thinspace%
            #2}%
        }{%
%    \end{macrocode}
%
% \pagebreak
%
%    \begin{macrocode}
          \edef\tfn@tfn{\arabic{footnote}}%
          \ltx@ifpackageloaded{hyperref}{%
            \tfn@hyperfootnotes{\xdef\tfn@thn{\arabic{Hfootnote}}}{}%
           }{}%
          \setcounter{footnote}{#1}
          \ltx@ifpackageloaded{hyperref}{%
            \tfn@hyperfootnotes{\setcounter{Hfootnote}{#1}}{}%
            \tfn@updateHyper%
           }{}%
          \footnotetext{%
            \hbox{\@textsuperscript{\normalfont #1}}%
            \ltx@ifpackageloaded{hyperref}{%
              \tfn@hyperfootnotes{\raisebox{\ht\strutbox}{%
                                         \hypertarget{tfn:#1}{}}}{}%
             }{}%
            \thinspace%
            #2}%
          \setcounter{footnote}{\tfn@tfn}%
          \ltx@ifpackageloaded{hyperref}{%
            \tfn@hyperfootnotes{\setcounter{Hfootnote}{\tfn@thn}}{}%
            \tfn@updateHyper%
           }{}%
        }%
     }%
  \fi%
%    \end{macrocode}
%
% For further footnotes in the table the \\
% |\ifx\tfn@footnotetableprint\tfn@footnotezero|\ldots|\fi| \\
% from the beginning of this |\tfn@footnotetext| macro must not be repeated,
% therefore we deactivate this by defining |\tfn@footnotetableprint| to
% something other than |0| (here |1| is used).
%
%    \begin{macrocode}
  \gdef\tfn@footnotetableprint{1}%
%    \end{macrocode}
%
% |\tfn@footnotetablecount| contains the negative number of footnote texts
% to be printed, therefore it is increase by one (i.\,e.~the negative number
% decreases, i.\,e.~$-3$ becomes~$-2$).
%
%    \begin{macrocode}
  \@tempcnta=\tfn@footnotetablecount\relax%
  \advance\@tempcnta by +1\relax%
  \xdef\tfn@footnotetablecount{\the\@tempcnta}%
%    \end{macrocode}
%
% When it is zero, all footnote texts have been placed, and\\
% |\tfn@footnotetableprint| can be reset to zero and\\
% |\tfn@tablefootnoteprintout| can be cleared.
%
%    \begin{macrocode}
  \ifx\tfn@footnotetablecount\tfn@footnotezero% all footnotes placed
    \gdef\tfn@footnotetableprint{0}%
    \global\let\tfn@tablefootnoteprintout\relax% brute force
 %\else there is at least one \footnotemark which needs \footnotetext
  \fi
}

%    \end{macrocode}
% \end{macro}
%
% \begin{macro}{\tfn@tablefootnoteprintout}
% When something shall be cleared or redefined, it is a good idea
% to define it. It~will be later filled with the foot note text to be printed.
%
%    \begin{macrocode}
\newcommand{\tfn@tablefootnoteprintout}{\relax}

%    \end{macrocode}
% \end{macro}
%
% \pagebreak
%
% \begin{macro}{\tablefootnote}
% |\tablefootnote[...]{...}| is the command to be used by the user instead of
% |\footnote[...]{...}| in the (sideways)table. It places a foot note mark via
% |\tfn@footnotemarks| and appends a |\tfn@footnotetext{#1}| to the\\
% |\tfn@tablefootnoteprintout| macro, and remembers that table foot notes
% have been used.
%
%    \begin{macrocode}
\newcommand{\tablefootnote}[2][]{%
  \tfn@footnotemarks[#1]%
  \ifthenelse{\isempty{#1}}{\relax}{%
    \ltx@GlobalPrependToMacro{\tfn@tablefootnoteprintout}{%
      \addtocounter{footnote}{+1}%
      \ltx@ifpackageloaded{hyperref}{%
        \tfn@hyperfootnotes{\addtocounter{Hfootnote}{+1}}{}%
        \tfn@updateHyper%
       }{}%
     }%
   }%
  \ltx@GlobalAppendToMacro{\tfn@tablefootnoteprintout}{%
    \tfn@footnotetext[#1]{#2}%
    \ifthenelse{\isempty{#1}}{\relax}{%
      \addtocounter{footnote}{-1}%
      \ltx@ifpackageloaded{hyperref}{%
        \tfn@hyperfootnotes{\addtocounter{Hfootnote}{-1}}{}%
        \tfn@updateHyper%
       }{}%
     }%
   }%
  \gdef\tfn@fnt{1}%
  }

%    \end{macrocode}
% \end{macro}
%
% \begin{macro}{\tfn@swtbox}
% For |sidewaystable|s the |\tfn@tablefootnoteprintout| is placed in a \\
% |\parbox|, which can be placed at the end of the |sidewaystable|:
%
%    \begin{macrocode}
\newcommand{\tfn@swtbox}{%
\ifvoid \footins%
\vskip 2\baselineskip%
\else%
\vskip \skip\footins%
\fi%
\parbox[b][][t]{\linewidth}{\raggedright\tfn@tablefootnoteprintout}%
}

%    \end{macrocode}
% \end{macro}
%
% |\tfn@tablefootnoteprintout| shall be executed after the end of the table environment.
% The \xpackage{etoolbox} package provides very easy syntax for this:
%
%    \begin{macrocode}
\IfFileExists{etoolbox.sty}{%
  \RequirePackage{etoolbox}[2011/01/03]%   v2.1
  \AfterEndEnvironment{table}{%
    \ltx@ifpackageloaded{footnotebackref}{%
      \LetLtxMacro{\tfncurrentmft}{\@makefntext}%
      \LetLtxMacro{\@makefntext}{\tablemakefntext}%
       \tfn@tablefootnoteprintout%
      \LetLtxMacro{\@makefntext}{\tfncurrentmft}%
     }{% footnotebackref not loaded
      \tfn@tablefootnoteprintout%
     }%
      \gdef\tfn@fnt{0}%
   }%
%    \end{macrocode}
%
% When the \xpackage{rotating} package has been loaded, we need to detect
% whether the |\tablefootnote| is placed inside of a |sidewaystable|.
%
%    \begin{macrocode}
  \ltx@ifpackageloaded{rotating}{%
    \AtBeginEnvironment{sidewaystable}{%
      \gdef\tfn@swt{1}%
    }
%    \end{macrocode}
%
% At the end of |sidewaystable| environments |\tfn@swtbox| shall be placed
% (if~there was any |\tablefootnote| in that |sidewaystable| at all).
% And |\tfn@swt| as well as |\tfn@fnt| must be reset to |0|.
%
%    \begin{macrocode}
    \AtEndEnvironment{sidewaystable}{%
      \ifx\tfn@fnt\tfn@footnotezero% \relax
      \else%
        \ltx@ifpackageloaded{footnotebackref}{%
          \LetLtxMacro{\tfncurrentmft}{\@makefntext}%
          \LetLtxMacro{\@makefntext}{\tablemakefntext}%
           \tfn@swtbox%
          \LetLtxMacro{\@makefntext}{\tfncurrentmft}%
         }{% footnotebackref not loaded
          \tfn@swtbox%
          }%
      \fi%
      \gdef\tfn@swt{0}%
      \gdef\tfn@fnt{0}%
     }%
   }{}% else \relax
}{%
%    \end{macrocode}
%
% When the \xpackage{etoolbox} package is not available (e.\,g. because of missing
% \eTeX{}) we~do a workaround:
%
%    \begin{macrocode}
  \PackageWarning{tablefootnote}{%
    Package etoolbox not available, therefore\MessageBreak%
    substituting commands \string\AtEndEnvironment\space %
    and\MessageBreak%
    \string\AfterEndEnvironment\space for tablefootnotes%
    \MessageBreak%
   }%
  \@ifpackagelater{letltxmacro}{2010/09/02}%
   {% >= 2010/09/02, OK
    \GlobalLetLtxMacro{\tfnorigtable}{\table}
    \GlobalLetLtxMacro{\tfnendorigtable}{\endtable}
%    \end{macrocode}
%
% |\GlobalLetLtxMacro| was introduced in version 2010/09/02 v1.4 of the
% \xpackage{letltxmacro} package.
%
%    \begin{macrocode}
   }{% else
     \PackageError{tablefootnote}{%
       Outdated version of letltxmacro package used%
      }{Version 2010/09/02 v1.4 or newer needed but not found!%
        \MessageBreak%
        Update the used letltxmacro package.}
        \LetLtxMacro{\tfnorigtable}{\table}
        \LetLtxMacro{\tfnendorigtable}{\endtable}
    }%
%    \end{macrocode}
%
% \pagebreak
%
%    \begin{macrocode}
  \ltx@ifpackageloaded{float}{%
    \@ifpackagelater{float}{2001/11/08}%   v1.3d
     {% >= 2001/11/08, OK
     }{% else
       \PackageWarningNoLine{tablefootnote}{%
         It is required version\MessageBreak
         2001/11/08 v1.3d (or later) of package float,\MessageBreak
         but only version\MessageBreak
        `\csname ver@float.sty\endcsname'\MessageBreak
        is available}%
     }%
   }{\relax}
  \renewenvironment{table}{%
    \ltx@ifpackageloaded{float}{%
%    \end{macrocode}
%
% When \xpackage{etoolbox.sty} was not found, and the \xpackage{float} package
% has been loaded, and the float specifier is \texttt{H}, then |\endtable| was
% let to |\float@endH| by the \xpackage{float} package (line~79 of
% \xpackage{float.sty}, 2001/11/08 v1.3d:\\
% |\expandafter\let\csname end#1\endcsname\float@endH|), i.\,e.~appending\\
% |\tfn@tablefootnoteprintout \gdef\tfn@fnt{0}| to |\endtable| is useless.
% Instead of this, it must be appended to |\float@endH|.\\
% (Why is this not necessary when \xpackage{etoolbox} is used?
% That is a good question, answers are welcome.)
%
%    \begin{macrocode}
      \ltx@LocalAppendToMacro{\float@endH}{%
        \ltx@ifpackageloaded{footnotebackref}{%
          \LetLtxMacro{\tfncurrentmft}{\@makefntext}%
          \LetLtxMacro{\@makefntext}{\tablemakefntext}%
           \tfn@tablefootnoteprintout%
          \LetLtxMacro{\@makefntext}{\tfncurrentmft}%
         }{% footnotebackref not loaded
          \tfn@tablefootnoteprintout%
         }%
          \gdef\tfn@fnt{0}%
       }%
     }{\relax}
    \tfnorigtable%
   }{%
     \tfnendorigtable%
     \ltx@ifpackageloaded{footnotebackref}{%
       \LetLtxMacro{\tfncurrentmft}{\@makefntext}%
       \LetLtxMacro{\@makefntext}{\tablemakefntext}%
        \tfn@tablefootnoteprintout%
       \LetLtxMacro{\@makefntext}{\tfncurrentmft}%
      }{% footnotebackref not loaded
       \tfn@tablefootnoteprintout%
      }%
     \gdef\tfn@fnt{0}%
    }%
%    \end{macrocode}
%
% \pagebreak
%
% The |sidewaystable|-environment must be treated separately:
%
%    \begin{macrocode}
  \ltx@ifpackageloaded{rotating}{%
    \let\tfnorigsidewaystable\sidewaystable
    \let\tfnendorigsidewaystable\endsidewaystable
    \renewenvironment{sidewaystable}{%
      \gdef\tfn@swt{1}%
      \tfnorigsidewaystable%
    }{%
      \ifx\tfn@fnt\tfn@footnotezero% \relax
      \else%
        \ltx@ifpackageloaded{footnotebackref}{%
          \LetLtxMacro{\tfncurrentmft}{\@makefntext}%
          \LetLtxMacro{\@makefntext}{\tablemakefntext}%
           \tfn@swtbox%
          \LetLtxMacro{\@makefntext}{\tfncurrentmft}%
         }{% footnotebackref not loaded
          \tfn@swtbox%
         }%
      \fi%
      \gdef\tfn@swt{0}%
      \gdef\tfn@fnt{0}%
      \tfnendorigsidewaystable%
     }%
   }{}%
}

%    \end{macrocode}
%
% \DescribeMacro{\AtBeginDocument}
% When the \xpackage{rotating} (for |sidewaystable|s) or the
% \xpackage{footnotebackref} (for back references for the footnotes)
% package is used, it must be loaded before the \xpackage{tablefootnote} package.
% Using both packages is also possible, in which case both must be loaded before
% the \xpackage{tablefootnote} package. If any one of those packages is loaded
% |\AtBeginDocument| after this check, the according error message will not be
% given, but packages really should be loaded before |\AtBeginDocument|.
%
%    \begin{macrocode}
\AtBeginDocument{%
  \ltx@ifpackageloaded{footnotebackref}{%
    \ifx\tfn@fnbr\tfn@footnotezero%
      \PackageError{tablefootnote}{%
        Package footnotebackref loaded after tablefootnote}{%
        When using both the footnotebackref and the tablefootnote %
        package,\MessageBreak%
        the footnotebackref package MUST be loaded before the %
        tablefootnote package!\MessageBreak%
       }%
    \fi%
   }{% else \relax
   }
  \ltx@ifpackageloaded{rotating}{%
    \ifx\tfn@rotating\tfn@footnotezero%
      \PackageError{tablefootnote}{%
        Package rotating loaded after tablefootnote}{%
        When using both the rotating and the tablefootnote package,%
        \MessageBreak%
        the rotating package MUST be loaded before the tablefootnote%
        \space package!\MessageBreak%
       }%
    \fi%
   }{% else \relax
   }
%    \end{macrocode}
%
% \pagebreak
%
% When the document is compiled with \LuaLaTeX{}, hyperlinks in rotated content
% will be misplaced, regardless of the use of the \xpackage{tablefootnote} package
% (or of tables or footnotes). The text to be printed will be OK.
% This problem of \LuaLaTeX{} (not \xpackage{tablefootnote}) was found and reported
% by \textsc{Arno Trautmann}, 2011/11/15 -- thanks!
%
%    \begin{macrocode}
  \ltx@ifpackageloaded{hyperref}{%
    \tfn@hyperfootnotes{
      \ifluatex%
        \PackageWarningNoLine{tablefootnote}{%
          LuaTeX and hyperref package used:\MessageBreak%
          Hyperreferences in rotated content\MessageBreak%
          will be misplaced\MessageBreak%
          regardless of tablefootnote package.\MessageBreak%
          The text to be printed will be OK%
         }%
      \fi%
     }{% else \relax
     }%
   }{% else \relax
   }%
%    \end{macrocode}
%
% When the \xpackage{footnotebackref} package is used but the footnotes are not
% hyperlinked, neither the back references nor the tablefootnotes will work.
%
%    \begin{macrocode}
  \ltx@ifpackageloaded{footnotebackref}{%
     \ltx@ifpackageloaded{hyperref}{%
        \tfn@hyperfootnotes{\relax}{% else
          \PackageError{tablefootnote}{%
             footnotebackref but not hyperfootnotes}{%
             The footnotebackref package is used\MessageBreak%
             but the hyperfootnotes option of the hyperref package%
             \MessageBreak%
             was set to false.\MessageBreak%
             Either enable hyperfootnotes or do not use footnotebackref.%
             \MessageBreak%
             Loading of the tablefootnote package will be aborted.%
           }%
          \endinput%
      }%
    }{\PackageError{tablefootnote}{footnotebackref but not hyperref}{%
        The footnotebackref package is used\MessageBreak%
        but not the hyperref package.\MessageBreak%
        Either load hyperref or do not use footnotebackref.%
        \MessageBreak%
        Loading of the tablefootnote package will be aborted.%
        }%
      \endinput%
    }%
  }{}
}

%    \end{macrocode}
%
%    \begin{macrocode}
%</package>
%    \end{macrocode}
%
% \pagebreak
%
% \section{Installation}
%
% \subsection{Downloads\label{ss:Downloads}}
%
% Everything is available on \CTAN{}, \url{http://www.ctan.org/},
% but may need additional packages themselves.\\
%
% \DescribeMacro{tablefootnote.dtx}
% For unpacking the |tablefootnote.dtx| file and constructing the documentation
% it is required:
% \begin{description}
% \item[-] \TeX{} Format \LaTeXe{}: \url{http://www.CTAN.org/}
%
% \item[-] document class \xpackage{ltxdoc}, 2007/11/11, v2.0u,
%           \url{http://www.ctan.org/pkg/ltxdoc}
%
% \item[-] package \xpackage{holtxdoc}, 2012/03/21, v0.24,
%           \url{http://www.ctan.org/pkg/holtxdoc}
%
% \item[-] package \xpackage{hypdoc}, 2011/08/19, v1.11,
%           \url{http://www.ctan.org/pkg/hypdoc}
% \end{description}
%
% \DescribeMacro{tablefootnote.sty}
% The |tablefootnote.sty| for \LaTeXe{} (i.\,e.~each document using
% the \xpackage{tablefootnote} package) requires:
% \begin{description}
% \item[-] \TeX{} Format \LaTeXe{}, \url{http://www.CTAN.org/}
%
% \item[-] package \xpackage{letltxmacro}, 2010/09/02, v1.4,
%           \url{http://www.ctan.org/pkg/letltxmacro}.
%
% \item[-] package \xpackage{ltxcmds}, 2011/11/09, v1.22,
%           \url{http://www.ctan.org/pkg/ltxcmds}
% \end{description}
%
% When the \xpackage{hyperref} package is used, also the \xpackage{ifluatex} package
% is needed, but it is already loaded automatically by the \xpackage{hyperref}
% package.\\
% When the \xpackage{etoolbox} package is available, it is used:
% \begin{description}
% \item[-] package \xpackage{etoolbox}, 2011/01/03, v2.1,
%           \url{http://www.ctan.org/pkg/etoolbox}.
% \end{description}
%
% \DescribeMacro{tablefootnote-example.tex}
% The |tablefootnote-example.tex| requires the same files as all
% documents using the \xpackage{tablefootnote} package and additionally:
% \begin{description}
% \item[-] class \xpackage{article}, 2007/10/19, v1.4h, from \xpackage{classes.dtx}:\\
%           \CTAN{macros/latex/base/classes.dtx}
%
% \item[-] package \xpackage{rotating}, 2009/03/28, v2.16a,
%           \url{http://www.ctan.org/pkg/rotating}
%
% \item[-] package \xpackage{float}, 2001/11/08, v1.3d,
%           \url{http://www.ctan.org/pkg/float}
%
% \item[-] package \xpackage{placeins}, 2005/04/18, v2.2,
%           \url{http://www.ctan.org/pkg/placeins}\\
%           for |\FloatBarrier|
%
% \item[-] package \xpackage{hyperref}, 2012/11/06, v6.83m,
%           \url{http://www.ctan.org/pkg/hyperref}
%
% \item[-] package \xpackage{tablefootnote}, 2014/01/26, v1.1c,\\
%           \url{http://www.ctan.org/pkg/tablefootnote}\\
%   (Well, it is the example file for this package, and because you
%    are reading the documentation for the \xpackage{tablefootnote} package,
%    it can be assumed that you already have some version of it~--
%    is it the current one?)
% \end{description}
%
% \DescribeMacro{Oberdiek}
% \DescribeMacro{holtxdoc}
% \DescribeMacro{ifluatex}
% \DescribeMacro{letltxmacro}
% \DescribeMacro{ltxcmds}
% All packages of \textsc{Heiko Oberdiek}'s bundle `oberdiek'
% (especially \xpackage{holtxdoc}, \xpackage{ifluatex}, \xpackage{letltxmacro},
% and \xpackage{ltxcmds}) are also available in a TDS compliant ZIP archive:\\
% \CTAN{install/macros/latex/contrib/oberdiek.tds.zip}.\\
% It is probably best to download and use this, because the packages in there
% are quite probably both recent and compatible among themselves.\\
%
% \pagebreak
%
% \DescribeMacro{hyperref}
% \noindent \xpackage{hyperref} is not included in that bundle and needs to be
% downloaded separately,\\
% \url{http://www.ctan.org/install/macros/latex/contrib/hyperref.tds.zip}.\\
%
% \DescribeMacro{M\"{u}nch}
% A hyperlinked list of my (other) packages can be found at
% \url{http://www.ctan.org/author/muench-hm}.\\
%
% \subsection{Package, unpacking TDS}
% \paragraph{Package.} This package is available on \CTAN{}
% \begin{description}
% \item[\CTAN{macros/latex/contrib/tablefootnote/tablefootnote.dtx}]\hspace*{0.1cm} \\
%       The source file.
% \item[\CTAN{macros/latex/contrib/tablefootnote/tablefootnote.pdf}]\hspace*{0.1cm} \\
%       The documentation.
% \item[\CTAN{macros/latex/contrib/tablefootnote/tablefootnote-example.pdf}]\hspace*{0.1cm} \\
%       The compiled example file, as it should look like.
% \item[\CTAN{macros/latex/contrib/tablefootnote/README}]\hspace*{0.1cm} \\
%       The README file.
% \end{description}
% There is also a tablefootnote.tds.zip available:
% \begin{description}
% \item[\CTAN{install/macros/latex/contrib/tablefootnote.tds.zip}]\hspace*{0.1cm} \\
%       Everything in \xfile{TDS} compliant, compiled format.
% \end{description}
% which additionally contains\\
% \begin{tabular}{ll}
% tablefootnote.ins & The installation file.\\
% tablefootnote.drv & The driver to generate the documentation.\\
% tablefootnote.sty & The \xext{sty}le file.\\
% tablefootnote-example.tex & The example file.
% \end{tabular}
%
% \bigskip
%
% \noindent For required other packages, please see the preceding subsection.
%
% \paragraph{Unpacking.} The \xfile{.dtx} file is a self-extracting
% \docstrip{} archive. The files are extracted by running the
% \xext{dtx} through \plainTeX{}:
% \begin{quote}
%   \verb|tex tablefootnote.dtx|
% \end{quote}
%
% \noindent About generating the documentation see paragraph~\ref{GenDoc} below.\\
%
% \paragraph{TDS.} Now the different files must be moved into
% the different directories in your installation TDS tree
% (also known as \xfile{texmf} tree):
% \begin{quote}
% \def\t{^^A
% \begin{tabular}{@{}>{\ttfamily}l@{ $\rightarrow$ }>{\ttfamily}l@{}}
%   tablefootnote.sty & tex/latex/tablefootnote/tablefootnote.sty\\
%   tablefootnote.pdf & doc/latex/tablefootnote/tablefootnote.pdf\\
%   tablefootnote-example.tex & doc/latex/tablefootnote/tablefootnote-example.tex\\
%   tablefootnote-example.pdf & doc/latex/tablefootnote/tablefootnote-example.pdf\\
%   tablefootnote.dtx & source/latex/tablefootnote/tablefootnote.dtx\\
% \end{tabular}^^A
% }^^A
% \sbox0{\t}^^A
% \ifdim\wd0>\linewidth
%   \begingroup
%     \advance\linewidth by\leftmargin
%     \advance\linewidth by\rightmargin
%   \edef\x{\endgroup
%     \def\noexpand\lw{\the\linewidth}^^A
%   }\x
%   \def\lwbox{^^A
%     \leavevmode
%     \hbox to \linewidth{^^A
%       \kern-\leftmargin\relax
%       \hss
%       \usebox0
%       \hss
%       \kern-\rightmargin\relax
%     }^^A
%   }^^A
%   \ifdim\wd0>\lw
%     \sbox0{\small\t}^^A
%     \ifdim\wd0>\linewidth
%       \ifdim\wd0>\lw
%         \sbox0{\footnotesize\t}^^A
%         \ifdim\wd0>\linewidth
%           \ifdim\wd0>\lw
%             \sbox0{\scriptsize\t}^^A
%             \ifdim\wd0>\linewidth
%               \ifdim\wd0>\lw
%                 \sbox0{\tiny\t}^^A
%                 \ifdim\wd0>\linewidth
%                   \lwbox
%                 \else
%                   \usebox0
%                 \fi
%               \else
%                 \lwbox
%               \fi
%             \else
%               \usebox0
%             \fi
%           \else
%             \lwbox
%           \fi
%         \else
%           \usebox0
%         \fi
%       \else
%         \lwbox
%       \fi
%     \else
%       \usebox0
%     \fi
%   \else
%     \lwbox
%   \fi
% \else
%   \usebox0
% \fi
% \end{quote}
% If you have a \xfile{docstrip.cfg} that configures and enables \docstrip{}'s
% \xfile{TDS} installing feature, then some files can already be in the right
% place, see the documentation of \docstrip{}.
%
% \subsection{Refresh file name databases}
%
% If your \TeX{}~distribution (\TeX live, \mikTeX{}, \dots{}) relies on file name
% databases, you must refresh these. For example, \teTeX{} users run
% \verb|texhash| or \verb|mktexlsr|.
%
% \subsection{Some details for the interested}
%
% \paragraph{Unpacking with \LaTeX{}.}
% The \xfile{.dtx} chooses its action depending on the format:
% \begin{description}
% \item[\plainTeX:] Run \docstrip{} and extract the files.
% \item[\LaTeX:] Generate the documentation.
% \end{description}
% If you insist on using \LaTeX{} for \docstrip{} (really,
% \docstrip{} does not need \LaTeX{}), then inform the autodetect routine
% about your intention:
% \begin{quote}
%   \verb|latex \let\install=y% \iffalse meta-comment
%
% File: tablefootnote.dtx
% Version: 2014/01/26 v1.1c
%
% Copyright (C) 2011 - 2014 by
%    H.-Martin M"unch <Martin dot Muench at Uni-Bonn dot de>
%
% This work may be distributed and/or modified under the
% conditions of the LaTeX Project Public License, either
% version 1.3c of this license or (at your option) any later
% version. This version of this license is in
%    http://www.latex-project.org/lppl/lppl-1-3c.txt
% and the latest version of this license is in
%    http://www.latex-project.org/lppl.txt
% and version 1.3c or later is part of all distributions of
% LaTeX version 2005/12/01 or later.
%
% This work has the LPPL maintenance status "maintained".
%
% The Current Maintainer of this work is H.-Martin Muench.
%
% This work consists of the main source file tablefootnote.dtx,
% the README, and the derived files
%    tablefootnote.sty, tablefootnote.pdf,
%    tablefootnote.ins, tablefootnote.drv,
%    tablefootnote-example.tex, tablefootnote-example.pdf.
%
% Distribution:
%    http://mirror.ctan.org/macros/latex/contrib/tablefootnote/tablefootnote.dtx
%    http://mirror.ctan.org/macros/latex/contrib/tablefootnote/tablefootnote.pdf
%    http://mirror.ctan.org/install/macros/latex/contrib/tablefootnote.tds.zip
%
% See http://www.ctan.org/pkg/tablefootnote (Catalogue)
%
% Unpacking:
%    (a) If tablefootnote.ins is present:
%           tex tablefootnote.ins
%    (b) Without tablefootnote.ins:
%           tex tablefootnote.dtx
%    (c) If you insist on using LaTeX
%           latex \let\install=y\input{tablefootnote.dtx}
%        (quote the arguments according to the demands of your shell)
%
% Documentation:
%    (a) If tablefootnote.drv is present:
%           (pdf)latex tablefootnote.drv
%           makeindex -s gind.ist tablefootnote.idx
%           (pdf)latex tablefootnote.drv
%           makeindex -s gind.ist tablefootnote.idx
%           (pdf)latex tablefootnote.drv
%    (b) Without tablefootnote.drv:
%           (pdf)latex tablefootnote.dtx
%           makeindex -s gind.ist tablefootnote.idx
%           (pdf)latex tablefootnote.dtx
%           makeindex -s gind.ist tablefootnote.idx
%           (pdf)latex tablefootnote.dtx
%
%    The class ltxdoc loads the configuration file ltxdoc.cfg
%    if available. Here you can specify further options, e.g.
%    use DIN A4 as paper format:
%       \PassOptionsToClass{a4paper}{article}
%
% Installation:
%    TDS:tex/latex/tablefootnote/tablefootnote.sty
%    TDS:doc/latex/tablefootnote/tablefootnote.pdf
%    TDS:doc/latex/tablefootnote/tablefootnote-example.tex
%    TDS:doc/latex/tablefootnote/tablefootnote-example.pdf
%    TDS:source/latex/tablefootnote/tablefootnote.dtx
%
%<*ignore>
\begingroup
  \catcode123=1 %
  \catcode125=2 %
  \def\x{LaTeX2e}%
\expandafter\endgroup
\ifcase 0\ifx\install y1\fi\expandafter
         \ifx\csname processbatchFile\endcsname\relax\else1\fi
         \ifx\fmtname\x\else 1\fi\relax
\else\csname fi\endcsname
%</ignore>
%<*install>
\input docstrip.tex
\Msg{******************************************************************}
\Msg{* Installation                                                   *}
\Msg{* Package: tablefootnote 2014/01/26 v1.1c Table foot notes (HMM) *}
\Msg{******************************************************************}

\keepsilent
\askforoverwritefalse

\let\MetaPrefix\relax
\preamble

This is a generated file.

Project: tablefootnote
Version: 2014/01/26 v1.1c

Copyright (C) 2011 - 2014 by
    H.-Martin M"unch <Martin dot Muench at Uni-Bonn dot de>

The usual disclaimer applies:
If it doesn't work right that's your problem.
(Nevertheless, send an e-mail to the maintainer
 when you find an error in this package.)

This work may be distributed and/or modified under the
conditions of the LaTeX Project Public License, either
version 1.3c of this license or (at your option) any later
version. This version of this license is in
   http://www.latex-project.org/lppl/lppl-1-3c.txt
and the latest version of this license is in
   http://www.latex-project.org/lppl.txt
and version 1.3c or later is part of all distributions of
LaTeX version 2005/12/01 or later.

This work has the LPPL maintenance status "maintained".

The Current Maintainer of this work is H.-Martin Muench.

This work consists of the main source file tablefootnote.dtx,
the README, and the derived files
   tablefootnote.sty, tablefootnote.pdf,
   tablefootnote.ins, tablefootnote.drv,
   tablefootnote-example.tex, tablefootnote-example.pdf.

In memoriam Tommy Muench + 2014/01/02.

\endpreamble
\let\MetaPrefix\DoubleperCent

\generate{%
  \file{tablefootnote.ins}{\from{tablefootnote.dtx}{install}}%
  \file{tablefootnote.drv}{\from{tablefootnote.dtx}{driver}}%
  \usedir{tex/latex/tablefootnote}%
  \file{tablefootnote.sty}{\from{tablefootnote.dtx}{package}}%
  \usedir{doc/latex/tablefootnote}%
  \file{tablefootnote-example.tex}{\from{tablefootnote.dtx}{example}}%
}

\catcode32=13\relax% active space
\let =\space%
\Msg{************************************************************************}
\Msg{*}
\Msg{* To finish the installation you have to move the following}
\Msg{* file into a directory searched by TeX:}
\Msg{*}
\Msg{*     tablefootnote.sty}
\Msg{*}
\Msg{* To produce the documentation run the file `tablefootnote.drv'}
\Msg{* through (pdf)LaTeX, e.g.}
\Msg{*  pdflatex tablefootnote.drv}
\Msg{*  makeindex -s gind.ist tablefootnote.idx}
\Msg{*  pdflatex tablefootnote.drv}
\Msg{*  makeindex -s gind.ist tablefootnote.idx}
\Msg{*  pdflatex tablefootnote.drv}
\Msg{*}
\Msg{* At least three runs are necessary e.g. to get the}
\Msg{*  references right!}
\Msg{*}
\Msg{* Happy TeXing!}
\Msg{*}
\Msg{************************************************************************}

\endbatchfile
%</install>
%<*ignore>
\fi
%</ignore>
%
% \section{The documentation driver file}
%
% The next bit of code contains the documentation driver file for
% \TeX{}, i.\,e., the file that will produce the documentation you
% are currently reading. It will be extracted from this file by the
% \texttt{docstrip} programme. That is, run \LaTeX{} on \texttt{docstrip}
% and specify the \texttt{driver} option when \texttt{docstrip}
% asks for options.
%
%    \begin{macrocode}
%<*driver>
\NeedsTeXFormat{LaTeX2e}[2011/06/27]
\ProvidesFile{tablefootnote.drv}%
  [2014/01/26 v1.1c Table foot notes (HMM)]
\documentclass{ltxdoc}[2007/11/11]% v2.0u
\usepackage{holtxdoc}[2012/03/21]%  v0.24
%% tablefootnote may work with earlier versions of LaTeX2e and those
%% class and package, but this was not tested.
%% Please consider updating your LaTeX, class, and package
%% to the most recent version (if they are not already the most
%% recent version).
\hypersetup{%
 pdfsubject={Footnotes in Tables (HMM)},%
 pdfkeywords={LaTeX, tablefootnote, footnote, table, H.-Martin Muench},%
 pdfencoding=auto,%
 pdflang={en},%
 breaklinks=true,%
 linktoc=all,%
 pdfstartview=FitH,%
 pdfpagelayout=OneColumn,%
 bookmarksnumbered=true,%
 bookmarksopen=true,%
 bookmarksopenlevel=3,%
 pdfmenubar=true,%
 pdftoolbar=true,%
 pdfwindowui=true,%
 pdfnewwindow=true%
}
\CodelineIndex
\hyphenation{printing docu-ment}
\gdef\unit#1{\mathord{\thinspace\mathrm{#1}}}%
\makeatletter
\@ifundefined{eTeX}{\gdef\eTeX{$\m@th \varepsilon $-\TeX }}{% else \relax
}
\makeatother
\begin{document}
  \DocInput{tablefootnote.dtx}%
\end{document}
%</driver>
%    \end{macrocode}
%
% \fi
%
% \CheckSum{584}
%
% \CharacterTable
%  {Upper-case    \A\B\C\D\E\F\G\H\I\J\K\L\M\N\O\P\Q\R\S\T\U\V\W\X\Y\Z
%   Lower-case    \a\b\c\d\e\f\g\h\i\j\k\l\m\n\o\p\q\r\s\t\u\v\w\x\y\z
%   Digits        \0\1\2\3\4\5\6\7\8\9
%   Exclamation   \!     Double quote  \"     Hash (number) \#
%   Dollar        \$     Percent       \%     Ampersand     \&
%   Acute accent  \'     Left paren    \(     Right paren   \)
%   Asterisk      \*     Plus          \+     Comma         \,
%   Minus         \-     Point         \.     Solidus       \/
%   Colon         \:     Semicolon     \;     Less than     \<
%   Equals        \=     Greater than  \>     Question mark \?
%   Commercial at \@     Left bracket  \[     Backslash     \\
%   Right bracket \]     Circumflex    \^     Underscore    \_
%   Grave accent  \`     Left brace    \{     Vertical bar  \|
%   Right brace   \}     Tilde         \~}
%
% \GetFileInfo{tablefootnote.drv}
%
% \begingroup
%   \def\x{\#,\$,\^,\_,\~,\ ,\&,\{,\},\%}%
%   \makeatletter
%   \@onelevel@sanitize\x
% \expandafter\endgroup
% \expandafter\DoNotIndex\expandafter{\x}
% \expandafter\DoNotIndex\expandafter{\string\ }
% \begingroup
%   \makeatletter
%     \lccode`9=32\relax
%     \lowercase{%^^A
%       \edef\x{\noexpand\DoNotIndex{\@backslashchar9}}%^^A
%     }%^^A
%   \expandafter\endgroup\x
%
% \DoNotIndex{\\,\,}
% \DoNotIndex{\documentclass,\usepackage,\ProvidesPackage,\begin,\end}
% \DoNotIndex{\MessageBreak,\message}
% \DoNotIndex{\NeedsTeXFormat,\DoNotIndex,\verb}
% \DoNotIndex{\def,\edef,\gdef,\global}
% \DoNotIndex{\ifx,\listfiles,\mathord,\mathrm}
% \DoNotIndex{\kvoptions,\SetupKeyvalOptions,\ProcessKeyvalOptions}
% \DoNotIndex{\bigskip,\space,\thinspace,\ldots}
% \DoNotIndex{\indent,\noindent,\newline,\linebreak,\pagebreak}
% \DoNotIndex{\textbf,\textit,\textsf,\texttt,\textsc,\textquotedblleft,\textquotedblright}
% \DoNotIndex{\plainTeX,\TeX,\LaTeX,\pdfLaTeX}
% \DoNotIndex{\chapter,\section,\the,\csname,\endscname}
% \DoNotIndex{\value,\arabic,\hline,\text,\caption,\centering,\clear,\url}
%
% \title{The \xpackage{tablefootnote} package}
% \date{2014/01/26 v1.1c}
% \author{H.-Martin M\"{u}nch\\\xemail{Martin.Muench at Uni-Bonn.de}}
%
% \maketitle
%
% \begin{abstract}
% This \LaTeX{} package provides the command |\tablefootnote| to be used
% in a \texttt{table} or \texttt{sidewaystable} environment,
% where |\footnote| will not work and when using |\footnotemark| and
% |\footnotetext| and adjusting the counters (including \texttt{Hfootnote})
% manually is either too much work or would not even work (\texttt{sidewaystable}).
% \end{abstract}
%
% \bigskip
%
% \noindent Disclaimer for web links: The author is not responsible for any contents
% referred to in this work unless he has full knowledge of illegal contents.
% If any damage occurs by the use of information presented there, only the
% author of the respective pages might be liable, not the one who has referred
% to these pages.
%
% \bigskip
%
% \noindent {\color{green} Save per page about $200\unit{ml}$ water,
% $2\unit{g}$ CO$_{2}$ and $2\unit{g}$ wood:\\
% Therefore please print only if this is really necessary.}
%
% \pagebreak
%
% \tableofcontents
%
% \pagebreak
%
% \section{Introduction\label{sec:Introduction}}
%
% \indent This \LaTeX{} package provides the command |\tablefootnote{...}| to be
% used in a \texttt{table} or \texttt{sidewaystable} environment. In those
% environments |\footnote{...}| would not work and would need to be replaced
% by |\footnotemark| in the \texttt{(sideways)\allowbreak table} and
% |\footnotetext{...}| after the \texttt{(sideways)table} environment ended.
% Additionally the counters for \texttt{footnote} and
% (when the \xpackage{hyperref} package is used) \texttt{Hfootnote}
% would need to be adjusted manually. Then still the hyperlinks
% in a \texttt{(sideways)table} would not work.
% When the \xpackage{footnotebackref} package is used,
% |\footnotemark| and |\footnotetext{...}| would need to be redefined.
% Just using |\tablefootnote{...}| (with optional argument for custom foot note
% marks) in the (sideways)table does all this automatically.
% (Compatibility with the \xpackage{footnotebackref} package
% was achieved without redefining |\footnotemark| or |\footnotetext|.)
% Redefining |\footnote| was \emph{not} done as to not create conflicts with
% other packages.
%
% \section{Usage}
%
% \indent Just load the package placing
% \begin{quote}
%   |\usepackage{tablefootnote}|
% \end{quote}
% \noindent in the preamble of your \LaTeXe{} source file and use\newline
% |\tablefootnote{...}| instead of |\footnote{...}| and\newline
% |\tablefootnote[...]{...}| instead of |\footnote[...]{...}|
% in (sideways)\allowbreak tables.\\
% When the \xpackage{footnotebackref} package is used,
% it must be loaded before the \xpackage{tablefootnote} package
% and the \xpackage{hyperrefpackage} with option |hyperfootnotes=true|
% must be loaded, too. (Backreference{-}links without links do not make sense.)\\
% When the \xpackage{rotating} package (for |sidewaystable|s) is used,
% it must be loaded before the \xpackage{tablefootnote} package and it must be
% a recent version (v2.16a, 2009/03/28, or newer).\\
% If |\tablefootnote{...}| is used in a tabular environment, which is inside
% of a (sideways)table environment, everything is fine, but when the tabular
% environment is not inside of a (sideways)table, the \textbf{table}footnote
% will not work.\\
%
% Pages with tables with footnotes, which are rotated with the \xpackage{lscape}-
% or with the \xpackage{pdflscape}-package (|\begin{landscape}|, table, |\end{landscape}|),
% are possible. But when the \xpackage{footmisc}-package is used, \xpackage{(pdf)lscape}
% should be loaded before \xpackage{footmisc}.\\
%
% When the \xpackage{footmisc}-package is used with option \texttt{para},
% this is regarded with the exception of a \texttt{sidewaystable}. There the footnotes
% are printed one below the other. (Well, this is better than no footnote in a
% \texttt{sidewaystable} at all, isn't it?)\\
%
% When the (sideways)table floats, the footnotes are set and (when \xpackage{hyperref}
% is used) hyperlinked, but they are not automatically adapted when the table floats
% over/under another footnote. Thus either do not use a footnote between original and
% float position of the (sideways)table, or place the (sideways)table in
% \textquotedblleft here\textquotedblright\ position. |\clear(double)page|, |h(!)|,
% |H|~from the \xpackage{float} package (\url{http://www.ctan.org/pkg/float}),
% or |\FloatBarrier| from the \xpackage{picins} package
% (\url{http://www.ctan.org/pkg/picins}) might help, too.
% (Or after finishing the document, move the (sideways)table in the source code
% near the position where it floats to or use the optional footnote marks.)
%
% \pagebreak
%
% \section{Alternatives\label{sec:Alternatives}}
%
% \begin{description}
% \item[-] The \xpackage{longtable} package provides the \texttt{longtable} environment
%   as replacement for the combined \texttt{table} and \texttt{tabular} environments.
%   Footnotes are real footnotes (not just tablenotes), are continuously numbered and
%   hyperlinked (when using the \xpackage{hyperref} package), and the hyperlinks really
%   work. As drawback the appearance of the caption changes slightly (e.\,g.~distance
%   to the table, width of the caption), which can probably be changed back manually.
%   Furthermore, longtables are meaned to break over more than one page. If that is not
%   wished, it must be prevented by |\nopagebreak|-commands and by ending the longtable
%   lines with |\\*| instead of |\\|. longtables do not float. (Therefore using the
%   \xpackage{tablefootnote} package and |\FloatBarrier| from the \xpackage{picins} package
%   before and after the table environment is similar~- but \xpackage{tablefootnote} does not
%   change the table-caption!) \texttt{sidewaystable} does not work with it.\\
%   \url{http://www.ctan.org/pkg/longtable}
%
% \item[-] The \xpackage{supertabular} package provides the \texttt{mpsupertabular}
%   environment as replacement for the combined \texttt{table} and \texttt{tabular}
%   environments. Footnotes are just tablenotes (with working hyperlinks when using the
%   \xpackage{hyperref} package), i.\,e. numbered a, b, c and placed below the table
%   and not at the end of the page. Therefore there is no float problem (because the
%   tablenotes numbering is not included in the continuous numbering of the footnotes).
%   Placing the \texttt{supertabular} inside of a \texttt{sidewaystable} breaks the
%   hyperlinks to the tablenotes.\\
%   \url{http://www.ctan.org/pkg/supertabular}
%
% \item[-] The \xpackage{ctable} package has its very own notation for defining tables.
%   It can create tablenotes and sideways-tables. The tablenotes are not automatically
%   hyperlinked. The \texttt{ctable}s float. Because the tablenotes numbering is not
%   included in the continuous numbering of the footnotes there is no float problem.\\
%   \url{http://www.ctan.org/pkg/ctable}
%
% \item[-] The \xpackage{footnote} package provides |\makesavenoteenv{table}|. After
%   loading the package and using that command in the preamble, in tables |\footnote{...}|
%   can be used. Using |\makesavenoteenv{tabular}| and \\
%   |\makesavenoteenv{sidewaystable}|
%   is possible, but it neither solves the float problem, nor do the created hyperlinks
%   work (i.\,e.~they aim at wrong locations). The \xpackage{mdwtab} from the same bundle
%   is incompatible with other table-related packages (e.\,g. \xpackage{supertabular},
%   \xpackage{array}) and not $100\,\%$ compatible with the \texttt{tabular} environment.\\
%   \url{http://www.ctan.org/pkg/footnote}\\
%   \url{http://www.ctan.org/pkg/mdwtab}
%
% \item[-] The \xpackage{tabularx} package does produce footnotes for sidewaystables,
%   but uses a, b, c instead of 1, 2, 3. The hyperlinks to the footnotes do not work.
%   Because the footnotes numbering is not included in the continuous numbering of
%   the other footnotes there is no float problem.\\
%   \url{http://www.ctan.org/pkg/tabularx}
%
% \item[-] Placing a tabular inside a minipage inside a table produces tablenotes.
%   Therefore there is no float problem (because the footnotes are not continuously
%   numbered). The hyperlinks to the table notes indeed work.
%
% \pagebreak
%
% \item[-] The \xpackage{threeparttable} package creates tablenotes again. Therefore
%   there is no float problem (because the tablenotes are not continuously numbered with
%   the footnotes). There are no hyperlinks to the table notes (at least not automatically).
%   Using |sidewaystable| (with table notes) works.\\
%   \url{http://www.ctan.org/pkg/threeparttable}
%
% \item[-] The \xpackage{threeparttablex} package creates tablenotes again. Therefore there
%   is no float problem (because the tablenotes are not continuously numbered with the
%   footnotes). With option \texttt{referable} the tablenotes are hyperlinked. Use of
%   a |sidewaystable| (with table notes) did not work for me. When using the
%   \texttt{referable} option according to the example in the \xpackage{threeparttablex}
%   manual the \xpackage{longtable} package is used, therefore that package could be used
%   directly without \xpackage{threeparttablex} (see above).\\
%   \url{http://www.ctan.org/pkg/threeparttablex}
%
% \item[-] One can manually use |\footnotemark| in the table and |\footnotetext{...}|
%  after the table environment ended and manually change the |footnote| and
%  (when the \xpackage{hyperref} package is used) |Hfootnote| counters and needs
%  to insert
%  \begin{verbatim}
%     \makeatletter
%     \global\let\Hy@saved@currentHref\@currentHref
%     \hyper@makecurrent{Hfootnote}%
%     \global\let\Hy@footnote@currentHref\@currentHref
%     \global\let\@currentHref\Hy@saved@currentHref
%     \makeatother
%  \end{verbatim}
%  before each |\footnotetext{...}|, but with more than very few footnotes this can become
%  much work (and is prone to produce errors). And this does not even work with
%  sidewaystables. (And it does not work with the \xpackage{footnotebackref} package
%  without redefining |\footnotemark| and |\footnotetext{...}|.)
% \end{description}
%
% \bigskip
%
% \noindent (You programmed or found another alternative,
%  which is available at \CTAN{}?\\
%  OK, send an e-mail to me with the name, location at \CTAN{},
%  and a short notice, and I will probably include it in
%  the list above.)
%
% \newpage
%
% \section{Example}
%
%    \begin{macrocode}
%<*example>
\documentclass[british]{article}[2007/10/19]% v1.4h
%%%%%%%%%%%%%%%%%%%%%%%%%%%%%%%%%%%%%%%%%%%%%%%%%%%%%%%%%%%%%%%%%%%%%
\usepackage{float}[2001/11/08]%    v1.3d
\usepackage{placeins}[2005/04/18]% v2.2  ; for \FloatBarrier
\usepackage{rotating}[2009/03/28]% v2.16a; for sidewaystable-environment
\usepackage[%
 hyperfootnotes=true,%
 extension=pdf,%
 plainpages=false,%
 pdfpagelabels=true,%
 hyperindex=false,%
 pdflang={en},%
 pdftitle={tablefootnote package example},%
 pdfauthor={H.-Martin Muench},%
 pdfsubject={Example for the tablefootnote package},%
 pdfkeywords={LaTeX, tablefootnote, footnote, table, H.-Martin Muench},%
 % pdfview=FitH and FitBH do not work: hyperlinks in sidewaystables
 % do no lead to the footnotes, due to a bug in pdfTeX,
 % computing wrong anchor coordinates (Heiko Oberdiek, 29. October 2011)
 % pdfview=Fit, FitV, FitR, FitB, FitBV work
 % print is OK for all those options
 pdfstartview=FitH,%
 pdfpagelayout=OneColumn%
]{hyperref}[2012/11/06]% v6.83m
 % Due to the urls used in the example, either the hyperref or the url
 % package are needed (or the urls must be removed before compiling).

\usepackage{footnotebackref}[2012/07/01]% v1.0
\usepackage{tablefootnote}[2014/01/26]%   v1.1c

\gdef\unit#1{\mathord{\thinspace\mathrm{#1}}}%
\listfiles
\begin{document}
\pagenumbering{arabic}
\section*{Example for tablefootnote}

This example demonstrates the use of package\newline
\textsf{tablefootnote}, v1.1c as of 2014/01/26 (HMM).\newline
There were no options used. (The package provides no options.)\newline

\textbf{The \texttt{tablefootnote-example.tex} needs to be compiled
at least twice to get the references right!}\newline

If the etoolbox-package is found, it is automatically used.\newline

For more details please see the documentation!\newline

\noindent Save per page about $200\unit{ml}$ water,
$2\unit{g}$ CO$_{2}$ and $2\unit{g}$ wood:\newline
Therefore please print only if this is really necessary.\newline

Here is some text.\footnote{And this is a text footnote.}\newline

Tables \ref{tab.symbol}, \ref{tab.normal}, \ref{tab.another} and
\ref{tab.floatH} show normal tables, table~\ref{tab.sideways} depicts
a sidewaystable. Table~\ref{tab.floatH} uses the float
specifier~\texttt{H} from the float package.\newline

\texttt{Hyperref} option \verb|pdfview=FitH| and \verb|FitBH| do
not work due to a bug in pdf\TeX{}, computing wrong
anchor coordinates (\textsc{Heiko Oberdiek}, 29.~October 2011).
Depending on used pdf-viewer, hyperlinks in sidewaystables lead
e.\,g.~at the end of the document, not at the footnote.
\verb|pdfview=Fit|, \verb|FitV|, \verb|FitR|, \verb|FitB|,
\verb|FitBV| work, print is OK for all those options.

\bigskip

\listoftables

\pagebreak

\renewcommand{\thefootnote}{\fnsymbol{footnote}}
\verb|\renewcommand{\thefootnote}{\fnsymbol{footnote}}|
causes footnote{-}symbol{-}footnotes, which are possible
(see Table~\ref{tab.symbol}).

\begin{table}
\centering
\begin{tabular}{ccc}
Another\tablefootnote{A table footnote.} & %
text\tablefootnote{Another table footnote.} & %
in a table\tablefootnote{A third table footnote.}
\end{tabular}
\caption[A footnotesymbol table]{%
A table with footnote-symbol-footnotes.\label{tab.symbol}}
\end{table}

Some text.\footnote{A text footnote.}

\renewcommand{\thefootnote}{\arabic{footnote}}
\verb|\renewcommand{\thefootnote}{\arabic{footnote}}|
switches back to normal footnote numbers again.

\pagebreak

\begin{table}
\centering
\begin{tabular}{ccc}
Some\tablefootnote{A table footnote.} & %
text\tablefootnote[99]{A table footnote with custom footnote number.} & %
in a table\tablefootnote{A third table footnote.}
\end{tabular}
\caption[A table]{A normal table.\label{tab.normal}}
\end{table}

Some text.\footnote{Another text footnote.}

\pagebreak

%    \end{macrocode}
%
% \pagebreak
%
%    \begin{macrocode}
More text.\footnote{And yet another text footnote.}

\begin{table}[t]
\centering
\begin{tabular}{|c|c|c|}
\hline
Another\tablefootnote{A $3^{rd}$ table footnote.} & %
text\tablefootnote{Another $3^{rd}$ table footnote.} & %
in a table\tablefootnote{A $3^{rd}$ third table footnote.}\\ \hline
\end{tabular}
\caption[Another table]{Another table (third one)\label{tab.another}}
\end{table}

Please note that Table~\ref{tab.another} floated to the top of the page.
While the footnotes are set and (when hyperref is used) hyperlinked,
they are not automatically adapted. Thus either do not use a footnote
at the same page before the table, or place the table in
\textquotedblleft here\textquotedblright\ or
\textquotedblleft bottom\textquotedblright\ position.
\verb|\clear(double)page|, \verb|h(!)|, \verb|H|~from the
\texttt{float} package, or \verb|\FloatBarrier| from the
\texttt{picins} package might help, too. (Or move the table in the
source code near the position where it floats to
or use the optional footnote marks.)

Table~\ref{tab.floatH} (page~\pageref{tab.floatH}) uses float specifier
\texttt{H} from the float package and does not float.

Some text.\footnote{This is just another text footnote.}

\pagebreak

\FloatBarrier

\begin{sidewaystable}
\centering%
\begin{tabular}{ccc}
Text\tablefootnote{Please rotate the view for testing the %
hyperlinks.} & %
in a\tablefootnote[98]{Another sidewaystable footnote %
with optional footnote mark.} & %
sidewaystable%
\tablefootnote{Sidewaystable-environment provided by %
\url{http://www.ctan.org/pkg/rotating} package.}
\end{tabular}
\caption[A sideways table]{A table in the \texttt{sideways} %
environment\label{tab.sideways}}%
\end{sidewaystable}%

\FloatBarrier

\pagebreak

%    \end{macrocode}
%
% \pagebreak
%
%    \begin{macrocode}
A last table, here with float specifier \texttt{H} from the
float\footnote{\url{http://www.ctan.org/pkg/float}} package.

\begin{table}[H]
\centering
\begin{tabular}{ccc}
Another\tablefootnote{A $5^{th}$ table footnote.} & %
text\tablefootnote{Another $5^{th}$ table footnote.} & %
in a table\tablefootnote{A $5^{th}$ third table footnote.}
\end{tabular}
\caption[A last table]{A very last table\label{tab.floatH}}
\end{table}

Some text.\footnote{This is just another text footnote.}

\pagebreak

End of the example for the
tablefootnote\footnote{\url{http://www.ctan.org/pkg/tablefootnote}}
package.

\end{document}
%</example>
%    \end{macrocode}
%
% \StopEventually{}
%
% \newpage
%
% \section{The implementation}
%
% We start off by checking that we are loading into \LaTeXe{} and
% announcing the name and version of this package.
%
%    \begin{macrocode}
%<*package>
%    \end{macrocode}
%
%    \begin{macrocode}
\NeedsTeXFormat{LaTeX2e}[2011/06/27]
\ProvidesPackage{tablefootnote}[2014/01/26 v1.1c
              Table foot notes (HMM)]
%    \end{macrocode}
%
% A short description of the \xpackage{tablefootnote} package:
%
%    \begin{macrocode}
%% Provides the \tablefootnote[...]{...} command
%% for footnotes in (sideways)tables.

%    \end{macrocode}
%
% We need the \xpackage{ltxcmds} and \xpackage{letltxmacro} packages
% by \textsc{Heiko Oberdiek}:
%
%    \begin{macrocode}
\RequirePackage{ltxcmds}[2011/11/09]%     v1.22
\RequirePackage{letltxmacro}[2010/09/02]% v1.4
\RequirePackage{xifthen}[2009/04/17]%     v1.3
%    \end{macrocode}
%
% A last information for the user:
%
%    \begin{macrocode}
%% tablefootnote may work with earlier versions of LaTeX and those
%% packages, but this was not tested. Please consider updating
%% your LaTeX and packages to the most recent version
%% (if they are not already the most recent version).

%    \end{macrocode}
%
% \indent See subsection~\ref{ss:Downloads} about how to get them.\\
%
% When the \xpackage{rotating} package (for |sidewaystable|s) is used,
% it must be loaded before the \xpackage{tablefootnote} package and it must be
% a recent version:
%
%    \begin{macrocode}
\@ifpackageloaded{rotating}{%
  \gdef\tfn@rotating{1}
  \@ifpackagelater{rotating}{2009/03/28}% v2.16a
   {% >= 2009/03/28, OK
   }{% else
     \PackageWarningNoLine{tablefootnote}{%
       It is required version\MessageBreak%
       2009/03/28 v2.16a (or later) of package rotating,\MessageBreak%
       but only version\MessageBreak%
      `\csname ver@rotating.sty\endcsname'\MessageBreak%
      is available}%
   }%
 }{% else
   \gdef\tfn@rotating{0}%
 }

%    \end{macrocode}
%
% There are no options to be processed.\\
%
% We need some definitions:
%
%    \begin{macrocode}
\def\tfn@footnotetablecount{0}
\def\tfn@footnotetableprint{0}
\def\tfn@footnotezero{0}
\def\tfn@fnt{0}
\def\tfn@swt{0}%       _s_ide_w_ays_t_able
\def\tfn@fmpw{\relax}% f_oot_m_isc_p_ara_w_arn-switch
\def\tfn@fnbr{0}%      f_oot_n_ote_b_ack_r_ef(erence package)

%    \end{macrocode}
%
% \begin{macro}{\tfn@hyperfootnotes}
% We need a wrapper for the used |\ifHy@hyperfootnotes|,
% which is not defined when the \xpackage{hyperref} package has not been loaded:
%
%    \begin{macrocode}
\DeclareRobustCommand{\tfn@hyperfootnotes}[2]{%
  \ifHy@hyperfootnotes%
    #1%
  \else%
    #2%
  \fi%
  }

%    \end{macrocode}
% \end{macro}
%
% \begin{macro}{\tablemakefntext}
% |\tablemakefntext| makes the footnotetext for a footnote in a table.
%
%    \begin{macrocode}
\ltx@ifpackageloaded{footnotebackref}{%
  \gdef\tfn@fnbr{1}%
  \ifx\FootnoteBackref@symbol\empty%
    \ifFootnoteBackref@numberlinked%
      \newcommand\tablemakefntext[1]{%
        \noindent\makebox[1.634em][r]{%
         \mbox{\textsuperscript{\normalfont%
           \hyperlink{tfn:m\theHfootnote}{\@thefnmark}%
        }}}%
        \ #1%
      }
    \else
      \newcommand\tablemakefntext[1]{%
        \noindent\makebox[1.8em][r]{%
          \mbox{\textsuperscript{\normalfont%
            \@thefnmark%
        }}}%
        #1%
      }
    \fi
  \else
    \newcommand\tablemakefntext[1]{%
      \noindent\makebox[1.8em][r]{%
        \mbox{\textsuperscript{\normalfont%
          \ifFootnoteBackref@numberlinked%
            \hyperlink{tfn:m\theHfootnote}{\@thefnmark}\ %
          \else%
            \@thefnmark%
          \fi%
      }}}%
      \hyperlink{tfn:m\theHfootnote}{\FootnoteBackref@symbol}%
      #1%
    }
  \fi
}{%
  \newcommand\tablemakefntext[1]{%
    \noindent\makebox[1.8em][r]{%
      \mbox{\textsuperscript{\normalfont%
        \@thefnmark%
    }}}%
    #1%
  }
}

%    \end{macrocode}
%
% \end{macro}
%
% \pagebreak
%
% \begin{macro}{\swtablemakefntext}
%    \begin{macrocode}
\newcommand\swtablemakefntext[1]{\noindent\ #1}

%    \end{macrocode}
% \end{macro}
%
% \begin{macro}{\tfn@footnotemarks}
% |\tfn@footnotemarks| uses the |\footnotemark| command to place a foot note mark
% and decreases the |\tfn@footnotetablecount| (i.\,e.~increases the negative value)
% of the number of footnote(marks) in the table. If the \xpackage{footnotebackref}
% package is used, it is also necessary to create an according |\hypertarget| for
% the back reference.
%
%    \begin{macrocode}
\newcommand{\tfn@footnotemarks}[1][]{%
  \hbox{%
    \ifthenelse{\isempty{#1}}%
      {\footnotemark%
       \ltx@ifpackageloaded{footnotebackref}%
         {\raisebox{\ht\strutbox}{\hypertarget{tfn:m\theHfootnote}{}}}%
         {% footnotebackref not loaded
         }%
      }%
      {\ltx@ifpackageloaded{hyperref}%
        {\tfn@hyperfootnotes{%
            \hyperlink{tfn:#1}{\footnotemark[#1]}%
          }{\footnotemark[#1]}}%
        {\footnotemark[#1]}%
       \ltx@ifpackageloaded{footnotebackref}%
        {\raisebox{\ht\strutbox}{\hypertarget{tfn:m#1}{}}}{}%
      }%
   }%
  \@tempcnta=\tfn@footnotetablecount\relax%
  \advance\@tempcnta by -1\relax%
  \xdef\tfn@footnotetablecount{\the\@tempcnta}%
}

%    \end{macrocode}
% \end{macro}
%
% \begin{macro}{\tfn@footmiscparawarn}
% When the \xpackage{footmisc} package is used with option |para|,
% we need to issue a warning. (Any idea how to resolve this issue?)\\
% It must be given inside of it's own command, because |\ifFN@para| is not
% defined when \xpackage{footmisc} has not been loaded, and then the
% |\fi| would be a |\fi| without |\if| (and \TeX{} does not like single
% |\fi|s).
%
%    \begin{macrocode}
\newcommand{\tfn@footmiscparawarn}{%
  \ifFN@para%
    \PackageWarning{tablefootnote}{%
      Package footmisc with option para used.\MessageBreak%
      Unfortunately package tablefootnote is not yet\MessageBreak%
      able to regard this in sidewaystables.\MessageBreak%
      Here the footnotes are printed one below\MessageBreak%
      the other. Sorry!\MessageBreak%
     }%
  \fi%
  }

%    \end{macrocode}
% \end{macro}
%
% \begin{macro}{\tfn@updateHyper}
% |\tfn@updateHyper| updates some \xpackage{hyperref} internals.
%
%    \begin{macrocode}
\newcommand{\tfn@updateHyper}{%
  % The code of this command is partly from the
  % hyperref package by Heiko Oberdiek, 2011/10/01, v6.82j.
  % Newer versions of that package are available at CTAN.org.
  \global\let\Hy@saved@currentHref\@currentHref%
  \tfn@hyperfootnotes{\hyper@makecurrent{Hfootnote}}{}%
  \global\let\Hy@footnote@currentHref\@currentHref%
  \global\let\@currentHref\Hy@saved@currentHref%
  }

%    \end{macrocode}
% \end{macro}
%
% \begin{macro}{\tfn@footnotetext}
% |\tfn@footnotetext| contains the |\footnotetext[...]{...}| command and handles
% the numbering. |\tfn@footnotetext| will be executed after the table.
% The foot note counter has been increased automatically with each |\footnotemark|,
% therefore it must be reset to its value before the table began.
%
%    \begin{macrocode}
\newcommand{\tfn@footnotetext}[2][]{%
  \ifx\tfn@footnotetableprint\tfn@footnotezero%
    % This is the first footnotetext after the (sideways)table.
    \addtocounter{footnote}{\tfn@footnotetablecount}%
%    \end{macrocode}
%
% When the \xpackage{hyperref} package is used, also its |Hfootnote| counter must
% be reset (when the footnotes are hyperlinked).
%
%    \begin{macrocode}
    \ltx@ifpackageloaded{hyperref}{%
      \tfn@hyperfootnotes{%
        \addtocounter{Hfootnote}{\tfn@footnotetablecount}}{}%
      \tfn@updateHyper%
     }{}%
  \fi%
%    \end{macrocode}
%
% For issuing the text for the next footnote the footnote-counter
% (and |Hfootnote| counter in case of links) are increased by one.
%
%    \begin{macrocode}
  \addtocounter{footnote}{+1}%
  \ltx@ifpackageloaded{hyperref}{%
    \tfn@hyperfootnotes{\addtocounter{Hfootnote}{+1}}{}%
%    \end{macrocode}
%
% The \xpackage{hyperref} package was not expecting this kind of change of the
% |(H)footnote| numbers, therefore some code-lines are missing here which can be
% found at other places in that same package and are called by this command.
%
%    \begin{macrocode}
    \tfn@updateHyper%
%    \end{macrocode}
%
% When \xpackage{hyperref} is not loaded, nothing is to be done here.
%
%    \begin{macrocode}
   }{}%
%    \end{macrocode}
%
% The |\footnotetext[...]{...}| places the foot note text as usual.
%
%    \begin{macrocode}
  \ifx\tfn@swt\tfn@footnotezero%
    \ifthenelse{\isempty{#1}}%
      {\footnotetext{#2}}%
      {\ltx@ifpackageloaded{hyperref}%
         {\tfn@hyperfootnotes{%
             \footnotetext[#1]{%
               \raisebox{\ht\strutbox}{\hypertarget{tfn:#1}{}}#2}%
           }{\footnotetext[#1]{#2}%
           }%
         }{\footnotetext[#1]{#2}}%
      }%
  \else%
%    \end{macrocode}
%
% If there is more than one tablefootnote in the |sidewaystable|, \\
% the |\tfn@footmiscparawarn| shall be given,
%
%    \begin{macrocode}
    \ifx\tfn@footnotetableprint\tfn@footnotezero%
      \gdef\tfn@fmpw{\tfn@footmiscparawarn}%
    \else%
      \ltx@ifpackageloaded{footmisc}{\tfn@fmpw}{\relax}%
%    \end{macrocode}
%
% but we do not want to repeat the message for a third(, forth,\ldots)
% tablefootnote in the |sidewaystable|.
%
%    \begin{macrocode}
      \gdef\tfn@fmpw{\relax}%
    \fi%
%    \end{macrocode}
%
% |\@makefnmark| is \\
% |\hbox {\@textsuperscript {\normalfont \@thefnmark }}|, \\
% but according to |\show\@thefnmark| here |\@thefnmark| is
% |{\protect \itshape}| instead of |\thefootnote| (Why?!).\\
% If the \xpackage{footnotebackref} package is used,
% back references must be created for the table footnotes, too.
%
%    \begin{macrocode}
    \ltx@ifpackageloaded{footnotebackref}{%
      \LetLtxMacro{\tfncurrentmft}{\@makefntext}%
      \LetLtxMacro{\@makefntext}{\swtablemakefntext}%
      \ifthenelse{\isempty{#1}}%
        {\footnotetext{%
           \raisebox{\ht\strutbox}{\hypertarget{tfn:\theHfootnote}{}}%
           \hbox{\@textsuperscript{\normalfont%
              \hyperlink{tfn:m\theHfootnote}{\thefootnote}}}\thinspace%
         \ifx\FootnoteBackref@symbol\empty%
         \else\hyperlink{tfn:m\theHfootnote}{\FootnoteBackref@symbol}%
         \fi%
         #2}%
       }{\edef\tfn@tfn{\arabic{footnote}}%
         \edef\tfn@thn{\arabic{Hfootnote}}%
         \setcounter{footnote}{#1}%
         \setcounter{Hfootnote}{#1}%
         \tfn@updateHyper%
         \footnotetext{%
           \raisebox{\ht\strutbox}{\hypertarget{tfn:#1}{}}%
           \hbox{\@textsuperscript{\normalfont%
                  \hyperlink{tfn:m#1}{#1}}}%
           \thinspace%
           \ifx\FootnoteBackref@symbol\empty%
           \else%
             \raisebox{\ht\strutbox}{%
               \hyperlink{tfn:m#1}{\FootnoteBackref@symbol}}%
           \fi%
           #2%
          }%
         \setcounter{footnote}{\tfn@tfn}%
         \setcounter{Hfootnote}{\tfn@thn}%
         \tfn@updateHyper%
       }%
      \LetLtxMacro{\@makefntext}{\tfncurrentmft}%
     }{% footnotebackref not loaded
       \ifthenelse{\isempty{#1}}%
         {\footnotetext{%
            \hbox{\@textsuperscript{\normalfont\thefootnote}}%
            \ltx@ifpackageloaded{hyperref}%
              {\tfn@hyperfootnotes{%
                 \raisebox{\ht\strutbox}{%
                   \hypertarget{tfn:\theHfootnote}{}}}{}%
              }{}%
            \thinspace%
            #2}%
        }{%
%    \end{macrocode}
%
% \pagebreak
%
%    \begin{macrocode}
          \edef\tfn@tfn{\arabic{footnote}}%
          \ltx@ifpackageloaded{hyperref}{%
            \tfn@hyperfootnotes{\xdef\tfn@thn{\arabic{Hfootnote}}}{}%
           }{}%
          \setcounter{footnote}{#1}
          \ltx@ifpackageloaded{hyperref}{%
            \tfn@hyperfootnotes{\setcounter{Hfootnote}{#1}}{}%
            \tfn@updateHyper%
           }{}%
          \footnotetext{%
            \hbox{\@textsuperscript{\normalfont #1}}%
            \ltx@ifpackageloaded{hyperref}{%
              \tfn@hyperfootnotes{\raisebox{\ht\strutbox}{%
                                         \hypertarget{tfn:#1}{}}}{}%
             }{}%
            \thinspace%
            #2}%
          \setcounter{footnote}{\tfn@tfn}%
          \ltx@ifpackageloaded{hyperref}{%
            \tfn@hyperfootnotes{\setcounter{Hfootnote}{\tfn@thn}}{}%
            \tfn@updateHyper%
           }{}%
        }%
     }%
  \fi%
%    \end{macrocode}
%
% For further footnotes in the table the \\
% |\ifx\tfn@footnotetableprint\tfn@footnotezero|\ldots|\fi| \\
% from the beginning of this |\tfn@footnotetext| macro must not be repeated,
% therefore we deactivate this by defining |\tfn@footnotetableprint| to
% something other than |0| (here |1| is used).
%
%    \begin{macrocode}
  \gdef\tfn@footnotetableprint{1}%
%    \end{macrocode}
%
% |\tfn@footnotetablecount| contains the negative number of footnote texts
% to be printed, therefore it is increase by one (i.\,e.~the negative number
% decreases, i.\,e.~$-3$ becomes~$-2$).
%
%    \begin{macrocode}
  \@tempcnta=\tfn@footnotetablecount\relax%
  \advance\@tempcnta by +1\relax%
  \xdef\tfn@footnotetablecount{\the\@tempcnta}%
%    \end{macrocode}
%
% When it is zero, all footnote texts have been placed, and\\
% |\tfn@footnotetableprint| can be reset to zero and\\
% |\tfn@tablefootnoteprintout| can be cleared.
%
%    \begin{macrocode}
  \ifx\tfn@footnotetablecount\tfn@footnotezero% all footnotes placed
    \gdef\tfn@footnotetableprint{0}%
    \global\let\tfn@tablefootnoteprintout\relax% brute force
 %\else there is at least one \footnotemark which needs \footnotetext
  \fi
}

%    \end{macrocode}
% \end{macro}
%
% \begin{macro}{\tfn@tablefootnoteprintout}
% When something shall be cleared or redefined, it is a good idea
% to define it. It~will be later filled with the foot note text to be printed.
%
%    \begin{macrocode}
\newcommand{\tfn@tablefootnoteprintout}{\relax}

%    \end{macrocode}
% \end{macro}
%
% \pagebreak
%
% \begin{macro}{\tablefootnote}
% |\tablefootnote[...]{...}| is the command to be used by the user instead of
% |\footnote[...]{...}| in the (sideways)table. It places a foot note mark via
% |\tfn@footnotemarks| and appends a |\tfn@footnotetext{#1}| to the\\
% |\tfn@tablefootnoteprintout| macro, and remembers that table foot notes
% have been used.
%
%    \begin{macrocode}
\newcommand{\tablefootnote}[2][]{%
  \tfn@footnotemarks[#1]%
  \ifthenelse{\isempty{#1}}{\relax}{%
    \ltx@GlobalPrependToMacro{\tfn@tablefootnoteprintout}{%
      \addtocounter{footnote}{+1}%
      \ltx@ifpackageloaded{hyperref}{%
        \tfn@hyperfootnotes{\addtocounter{Hfootnote}{+1}}{}%
        \tfn@updateHyper%
       }{}%
     }%
   }%
  \ltx@GlobalAppendToMacro{\tfn@tablefootnoteprintout}{%
    \tfn@footnotetext[#1]{#2}%
    \ifthenelse{\isempty{#1}}{\relax}{%
      \addtocounter{footnote}{-1}%
      \ltx@ifpackageloaded{hyperref}{%
        \tfn@hyperfootnotes{\addtocounter{Hfootnote}{-1}}{}%
        \tfn@updateHyper%
       }{}%
     }%
   }%
  \gdef\tfn@fnt{1}%
  }

%    \end{macrocode}
% \end{macro}
%
% \begin{macro}{\tfn@swtbox}
% For |sidewaystable|s the |\tfn@tablefootnoteprintout| is placed in a \\
% |\parbox|, which can be placed at the end of the |sidewaystable|:
%
%    \begin{macrocode}
\newcommand{\tfn@swtbox}{%
\ifvoid \footins%
\vskip 2\baselineskip%
\else%
\vskip \skip\footins%
\fi%
\parbox[b][][t]{\linewidth}{\raggedright\tfn@tablefootnoteprintout}%
}

%    \end{macrocode}
% \end{macro}
%
% |\tfn@tablefootnoteprintout| shall be executed after the end of the table environment.
% The \xpackage{etoolbox} package provides very easy syntax for this:
%
%    \begin{macrocode}
\IfFileExists{etoolbox.sty}{%
  \RequirePackage{etoolbox}[2011/01/03]%   v2.1
  \AfterEndEnvironment{table}{%
    \ltx@ifpackageloaded{footnotebackref}{%
      \LetLtxMacro{\tfncurrentmft}{\@makefntext}%
      \LetLtxMacro{\@makefntext}{\tablemakefntext}%
       \tfn@tablefootnoteprintout%
      \LetLtxMacro{\@makefntext}{\tfncurrentmft}%
     }{% footnotebackref not loaded
      \tfn@tablefootnoteprintout%
     }%
      \gdef\tfn@fnt{0}%
   }%
%    \end{macrocode}
%
% When the \xpackage{rotating} package has been loaded, we need to detect
% whether the |\tablefootnote| is placed inside of a |sidewaystable|.
%
%    \begin{macrocode}
  \ltx@ifpackageloaded{rotating}{%
    \AtBeginEnvironment{sidewaystable}{%
      \gdef\tfn@swt{1}%
    }
%    \end{macrocode}
%
% At the end of |sidewaystable| environments |\tfn@swtbox| shall be placed
% (if~there was any |\tablefootnote| in that |sidewaystable| at all).
% And |\tfn@swt| as well as |\tfn@fnt| must be reset to |0|.
%
%    \begin{macrocode}
    \AtEndEnvironment{sidewaystable}{%
      \ifx\tfn@fnt\tfn@footnotezero% \relax
      \else%
        \ltx@ifpackageloaded{footnotebackref}{%
          \LetLtxMacro{\tfncurrentmft}{\@makefntext}%
          \LetLtxMacro{\@makefntext}{\tablemakefntext}%
           \tfn@swtbox%
          \LetLtxMacro{\@makefntext}{\tfncurrentmft}%
         }{% footnotebackref not loaded
          \tfn@swtbox%
          }%
      \fi%
      \gdef\tfn@swt{0}%
      \gdef\tfn@fnt{0}%
     }%
   }{}% else \relax
}{%
%    \end{macrocode}
%
% When the \xpackage{etoolbox} package is not available (e.\,g. because of missing
% \eTeX{}) we~do a workaround:
%
%    \begin{macrocode}
  \PackageWarning{tablefootnote}{%
    Package etoolbox not available, therefore\MessageBreak%
    substituting commands \string\AtEndEnvironment\space %
    and\MessageBreak%
    \string\AfterEndEnvironment\space for tablefootnotes%
    \MessageBreak%
   }%
  \@ifpackagelater{letltxmacro}{2010/09/02}%
   {% >= 2010/09/02, OK
    \GlobalLetLtxMacro{\tfnorigtable}{\table}
    \GlobalLetLtxMacro{\tfnendorigtable}{\endtable}
%    \end{macrocode}
%
% |\GlobalLetLtxMacro| was introduced in version 2010/09/02 v1.4 of the
% \xpackage{letltxmacro} package.
%
%    \begin{macrocode}
   }{% else
     \PackageError{tablefootnote}{%
       Outdated version of letltxmacro package used%
      }{Version 2010/09/02 v1.4 or newer needed but not found!%
        \MessageBreak%
        Update the used letltxmacro package.}
        \LetLtxMacro{\tfnorigtable}{\table}
        \LetLtxMacro{\tfnendorigtable}{\endtable}
    }%
%    \end{macrocode}
%
% \pagebreak
%
%    \begin{macrocode}
  \ltx@ifpackageloaded{float}{%
    \@ifpackagelater{float}{2001/11/08}%   v1.3d
     {% >= 2001/11/08, OK
     }{% else
       \PackageWarningNoLine{tablefootnote}{%
         It is required version\MessageBreak
         2001/11/08 v1.3d (or later) of package float,\MessageBreak
         but only version\MessageBreak
        `\csname ver@float.sty\endcsname'\MessageBreak
        is available}%
     }%
   }{\relax}
  \renewenvironment{table}{%
    \ltx@ifpackageloaded{float}{%
%    \end{macrocode}
%
% When \xpackage{etoolbox.sty} was not found, and the \xpackage{float} package
% has been loaded, and the float specifier is \texttt{H}, then |\endtable| was
% let to |\float@endH| by the \xpackage{float} package (line~79 of
% \xpackage{float.sty}, 2001/11/08 v1.3d:\\
% |\expandafter\let\csname end#1\endcsname\float@endH|), i.\,e.~appending\\
% |\tfn@tablefootnoteprintout \gdef\tfn@fnt{0}| to |\endtable| is useless.
% Instead of this, it must be appended to |\float@endH|.\\
% (Why is this not necessary when \xpackage{etoolbox} is used?
% That is a good question, answers are welcome.)
%
%    \begin{macrocode}
      \ltx@LocalAppendToMacro{\float@endH}{%
        \ltx@ifpackageloaded{footnotebackref}{%
          \LetLtxMacro{\tfncurrentmft}{\@makefntext}%
          \LetLtxMacro{\@makefntext}{\tablemakefntext}%
           \tfn@tablefootnoteprintout%
          \LetLtxMacro{\@makefntext}{\tfncurrentmft}%
         }{% footnotebackref not loaded
          \tfn@tablefootnoteprintout%
         }%
          \gdef\tfn@fnt{0}%
       }%
     }{\relax}
    \tfnorigtable%
   }{%
     \tfnendorigtable%
     \ltx@ifpackageloaded{footnotebackref}{%
       \LetLtxMacro{\tfncurrentmft}{\@makefntext}%
       \LetLtxMacro{\@makefntext}{\tablemakefntext}%
        \tfn@tablefootnoteprintout%
       \LetLtxMacro{\@makefntext}{\tfncurrentmft}%
      }{% footnotebackref not loaded
       \tfn@tablefootnoteprintout%
      }%
     \gdef\tfn@fnt{0}%
    }%
%    \end{macrocode}
%
% \pagebreak
%
% The |sidewaystable|-environment must be treated separately:
%
%    \begin{macrocode}
  \ltx@ifpackageloaded{rotating}{%
    \let\tfnorigsidewaystable\sidewaystable
    \let\tfnendorigsidewaystable\endsidewaystable
    \renewenvironment{sidewaystable}{%
      \gdef\tfn@swt{1}%
      \tfnorigsidewaystable%
    }{%
      \ifx\tfn@fnt\tfn@footnotezero% \relax
      \else%
        \ltx@ifpackageloaded{footnotebackref}{%
          \LetLtxMacro{\tfncurrentmft}{\@makefntext}%
          \LetLtxMacro{\@makefntext}{\tablemakefntext}%
           \tfn@swtbox%
          \LetLtxMacro{\@makefntext}{\tfncurrentmft}%
         }{% footnotebackref not loaded
          \tfn@swtbox%
         }%
      \fi%
      \gdef\tfn@swt{0}%
      \gdef\tfn@fnt{0}%
      \tfnendorigsidewaystable%
     }%
   }{}%
}

%    \end{macrocode}
%
% \DescribeMacro{\AtBeginDocument}
% When the \xpackage{rotating} (for |sidewaystable|s) or the
% \xpackage{footnotebackref} (for back references for the footnotes)
% package is used, it must be loaded before the \xpackage{tablefootnote} package.
% Using both packages is also possible, in which case both must be loaded before
% the \xpackage{tablefootnote} package. If any one of those packages is loaded
% |\AtBeginDocument| after this check, the according error message will not be
% given, but packages really should be loaded before |\AtBeginDocument|.
%
%    \begin{macrocode}
\AtBeginDocument{%
  \ltx@ifpackageloaded{footnotebackref}{%
    \ifx\tfn@fnbr\tfn@footnotezero%
      \PackageError{tablefootnote}{%
        Package footnotebackref loaded after tablefootnote}{%
        When using both the footnotebackref and the tablefootnote %
        package,\MessageBreak%
        the footnotebackref package MUST be loaded before the %
        tablefootnote package!\MessageBreak%
       }%
    \fi%
   }{% else \relax
   }
  \ltx@ifpackageloaded{rotating}{%
    \ifx\tfn@rotating\tfn@footnotezero%
      \PackageError{tablefootnote}{%
        Package rotating loaded after tablefootnote}{%
        When using both the rotating and the tablefootnote package,%
        \MessageBreak%
        the rotating package MUST be loaded before the tablefootnote%
        \space package!\MessageBreak%
       }%
    \fi%
   }{% else \relax
   }
%    \end{macrocode}
%
% \pagebreak
%
% When the document is compiled with \LuaLaTeX{}, hyperlinks in rotated content
% will be misplaced, regardless of the use of the \xpackage{tablefootnote} package
% (or of tables or footnotes). The text to be printed will be OK.
% This problem of \LuaLaTeX{} (not \xpackage{tablefootnote}) was found and reported
% by \textsc{Arno Trautmann}, 2011/11/15 -- thanks!
%
%    \begin{macrocode}
  \ltx@ifpackageloaded{hyperref}{%
    \tfn@hyperfootnotes{
      \ifluatex%
        \PackageWarningNoLine{tablefootnote}{%
          LuaTeX and hyperref package used:\MessageBreak%
          Hyperreferences in rotated content\MessageBreak%
          will be misplaced\MessageBreak%
          regardless of tablefootnote package.\MessageBreak%
          The text to be printed will be OK%
         }%
      \fi%
     }{% else \relax
     }%
   }{% else \relax
   }%
%    \end{macrocode}
%
% When the \xpackage{footnotebackref} package is used but the footnotes are not
% hyperlinked, neither the back references nor the tablefootnotes will work.
%
%    \begin{macrocode}
  \ltx@ifpackageloaded{footnotebackref}{%
     \ltx@ifpackageloaded{hyperref}{%
        \tfn@hyperfootnotes{\relax}{% else
          \PackageError{tablefootnote}{%
             footnotebackref but not hyperfootnotes}{%
             The footnotebackref package is used\MessageBreak%
             but the hyperfootnotes option of the hyperref package%
             \MessageBreak%
             was set to false.\MessageBreak%
             Either enable hyperfootnotes or do not use footnotebackref.%
             \MessageBreak%
             Loading of the tablefootnote package will be aborted.%
           }%
          \endinput%
      }%
    }{\PackageError{tablefootnote}{footnotebackref but not hyperref}{%
        The footnotebackref package is used\MessageBreak%
        but not the hyperref package.\MessageBreak%
        Either load hyperref or do not use footnotebackref.%
        \MessageBreak%
        Loading of the tablefootnote package will be aborted.%
        }%
      \endinput%
    }%
  }{}
}

%    \end{macrocode}
%
%    \begin{macrocode}
%</package>
%    \end{macrocode}
%
% \pagebreak
%
% \section{Installation}
%
% \subsection{Downloads\label{ss:Downloads}}
%
% Everything is available on \CTAN{}, \url{http://www.ctan.org/},
% but may need additional packages themselves.\\
%
% \DescribeMacro{tablefootnote.dtx}
% For unpacking the |tablefootnote.dtx| file and constructing the documentation
% it is required:
% \begin{description}
% \item[-] \TeX{} Format \LaTeXe{}: \url{http://www.CTAN.org/}
%
% \item[-] document class \xpackage{ltxdoc}, 2007/11/11, v2.0u,
%           \url{http://www.ctan.org/pkg/ltxdoc}
%
% \item[-] package \xpackage{holtxdoc}, 2012/03/21, v0.24,
%           \url{http://www.ctan.org/pkg/holtxdoc}
%
% \item[-] package \xpackage{hypdoc}, 2011/08/19, v1.11,
%           \url{http://www.ctan.org/pkg/hypdoc}
% \end{description}
%
% \DescribeMacro{tablefootnote.sty}
% The |tablefootnote.sty| for \LaTeXe{} (i.\,e.~each document using
% the \xpackage{tablefootnote} package) requires:
% \begin{description}
% \item[-] \TeX{} Format \LaTeXe{}, \url{http://www.CTAN.org/}
%
% \item[-] package \xpackage{letltxmacro}, 2010/09/02, v1.4,
%           \url{http://www.ctan.org/pkg/letltxmacro}.
%
% \item[-] package \xpackage{ltxcmds}, 2011/11/09, v1.22,
%           \url{http://www.ctan.org/pkg/ltxcmds}
% \end{description}
%
% When the \xpackage{hyperref} package is used, also the \xpackage{ifluatex} package
% is needed, but it is already loaded automatically by the \xpackage{hyperref}
% package.\\
% When the \xpackage{etoolbox} package is available, it is used:
% \begin{description}
% \item[-] package \xpackage{etoolbox}, 2011/01/03, v2.1,
%           \url{http://www.ctan.org/pkg/etoolbox}.
% \end{description}
%
% \DescribeMacro{tablefootnote-example.tex}
% The |tablefootnote-example.tex| requires the same files as all
% documents using the \xpackage{tablefootnote} package and additionally:
% \begin{description}
% \item[-] class \xpackage{article}, 2007/10/19, v1.4h, from \xpackage{classes.dtx}:\\
%           \CTAN{macros/latex/base/classes.dtx}
%
% \item[-] package \xpackage{rotating}, 2009/03/28, v2.16a,
%           \url{http://www.ctan.org/pkg/rotating}
%
% \item[-] package \xpackage{float}, 2001/11/08, v1.3d,
%           \url{http://www.ctan.org/pkg/float}
%
% \item[-] package \xpackage{placeins}, 2005/04/18, v2.2,
%           \url{http://www.ctan.org/pkg/placeins}\\
%           for |\FloatBarrier|
%
% \item[-] package \xpackage{hyperref}, 2012/11/06, v6.83m,
%           \url{http://www.ctan.org/pkg/hyperref}
%
% \item[-] package \xpackage{tablefootnote}, 2014/01/26, v1.1c,\\
%           \url{http://www.ctan.org/pkg/tablefootnote}\\
%   (Well, it is the example file for this package, and because you
%    are reading the documentation for the \xpackage{tablefootnote} package,
%    it can be assumed that you already have some version of it~--
%    is it the current one?)
% \end{description}
%
% \DescribeMacro{Oberdiek}
% \DescribeMacro{holtxdoc}
% \DescribeMacro{ifluatex}
% \DescribeMacro{letltxmacro}
% \DescribeMacro{ltxcmds}
% All packages of \textsc{Heiko Oberdiek}'s bundle `oberdiek'
% (especially \xpackage{holtxdoc}, \xpackage{ifluatex}, \xpackage{letltxmacro},
% and \xpackage{ltxcmds}) are also available in a TDS compliant ZIP archive:\\
% \CTAN{install/macros/latex/contrib/oberdiek.tds.zip}.\\
% It is probably best to download and use this, because the packages in there
% are quite probably both recent and compatible among themselves.\\
%
% \pagebreak
%
% \DescribeMacro{hyperref}
% \noindent \xpackage{hyperref} is not included in that bundle and needs to be
% downloaded separately,\\
% \url{http://www.ctan.org/install/macros/latex/contrib/hyperref.tds.zip}.\\
%
% \DescribeMacro{M\"{u}nch}
% A hyperlinked list of my (other) packages can be found at
% \url{http://www.ctan.org/author/muench-hm}.\\
%
% \subsection{Package, unpacking TDS}
% \paragraph{Package.} This package is available on \CTAN{}
% \begin{description}
% \item[\CTAN{macros/latex/contrib/tablefootnote/tablefootnote.dtx}]\hspace*{0.1cm} \\
%       The source file.
% \item[\CTAN{macros/latex/contrib/tablefootnote/tablefootnote.pdf}]\hspace*{0.1cm} \\
%       The documentation.
% \item[\CTAN{macros/latex/contrib/tablefootnote/tablefootnote-example.pdf}]\hspace*{0.1cm} \\
%       The compiled example file, as it should look like.
% \item[\CTAN{macros/latex/contrib/tablefootnote/README}]\hspace*{0.1cm} \\
%       The README file.
% \end{description}
% There is also a tablefootnote.tds.zip available:
% \begin{description}
% \item[\CTAN{install/macros/latex/contrib/tablefootnote.tds.zip}]\hspace*{0.1cm} \\
%       Everything in \xfile{TDS} compliant, compiled format.
% \end{description}
% which additionally contains\\
% \begin{tabular}{ll}
% tablefootnote.ins & The installation file.\\
% tablefootnote.drv & The driver to generate the documentation.\\
% tablefootnote.sty & The \xext{sty}le file.\\
% tablefootnote-example.tex & The example file.
% \end{tabular}
%
% \bigskip
%
% \noindent For required other packages, please see the preceding subsection.
%
% \paragraph{Unpacking.} The \xfile{.dtx} file is a self-extracting
% \docstrip{} archive. The files are extracted by running the
% \xext{dtx} through \plainTeX{}:
% \begin{quote}
%   \verb|tex tablefootnote.dtx|
% \end{quote}
%
% \noindent About generating the documentation see paragraph~\ref{GenDoc} below.\\
%
% \paragraph{TDS.} Now the different files must be moved into
% the different directories in your installation TDS tree
% (also known as \xfile{texmf} tree):
% \begin{quote}
% \def\t{^^A
% \begin{tabular}{@{}>{\ttfamily}l@{ $\rightarrow$ }>{\ttfamily}l@{}}
%   tablefootnote.sty & tex/latex/tablefootnote/tablefootnote.sty\\
%   tablefootnote.pdf & doc/latex/tablefootnote/tablefootnote.pdf\\
%   tablefootnote-example.tex & doc/latex/tablefootnote/tablefootnote-example.tex\\
%   tablefootnote-example.pdf & doc/latex/tablefootnote/tablefootnote-example.pdf\\
%   tablefootnote.dtx & source/latex/tablefootnote/tablefootnote.dtx\\
% \end{tabular}^^A
% }^^A
% \sbox0{\t}^^A
% \ifdim\wd0>\linewidth
%   \begingroup
%     \advance\linewidth by\leftmargin
%     \advance\linewidth by\rightmargin
%   \edef\x{\endgroup
%     \def\noexpand\lw{\the\linewidth}^^A
%   }\x
%   \def\lwbox{^^A
%     \leavevmode
%     \hbox to \linewidth{^^A
%       \kern-\leftmargin\relax
%       \hss
%       \usebox0
%       \hss
%       \kern-\rightmargin\relax
%     }^^A
%   }^^A
%   \ifdim\wd0>\lw
%     \sbox0{\small\t}^^A
%     \ifdim\wd0>\linewidth
%       \ifdim\wd0>\lw
%         \sbox0{\footnotesize\t}^^A
%         \ifdim\wd0>\linewidth
%           \ifdim\wd0>\lw
%             \sbox0{\scriptsize\t}^^A
%             \ifdim\wd0>\linewidth
%               \ifdim\wd0>\lw
%                 \sbox0{\tiny\t}^^A
%                 \ifdim\wd0>\linewidth
%                   \lwbox
%                 \else
%                   \usebox0
%                 \fi
%               \else
%                 \lwbox
%               \fi
%             \else
%               \usebox0
%             \fi
%           \else
%             \lwbox
%           \fi
%         \else
%           \usebox0
%         \fi
%       \else
%         \lwbox
%       \fi
%     \else
%       \usebox0
%     \fi
%   \else
%     \lwbox
%   \fi
% \else
%   \usebox0
% \fi
% \end{quote}
% If you have a \xfile{docstrip.cfg} that configures and enables \docstrip{}'s
% \xfile{TDS} installing feature, then some files can already be in the right
% place, see the documentation of \docstrip{}.
%
% \subsection{Refresh file name databases}
%
% If your \TeX{}~distribution (\TeX live, \mikTeX{}, \dots{}) relies on file name
% databases, you must refresh these. For example, \teTeX{} users run
% \verb|texhash| or \verb|mktexlsr|.
%
% \subsection{Some details for the interested}
%
% \paragraph{Unpacking with \LaTeX{}.}
% The \xfile{.dtx} chooses its action depending on the format:
% \begin{description}
% \item[\plainTeX:] Run \docstrip{} and extract the files.
% \item[\LaTeX:] Generate the documentation.
% \end{description}
% If you insist on using \LaTeX{} for \docstrip{} (really,
% \docstrip{} does not need \LaTeX{}), then inform the autodetect routine
% about your intention:
% \begin{quote}
%   \verb|latex \let\install=y\input{tablefootnote.dtx}|
% \end{quote}
% Do not forget to quote the argument according to the demands
% of your shell.
%
% \paragraph{Generating the documentation.\label{GenDoc}}
% You can use both the \xfile{.dtx} or the \xfile{.drv} to generate
% the documentation. The process can be configured by a
% configuration file \xfile{ltxdoc.cfg}. For instance, put the following
% line into this file, if you want to have A4 as paper format:
% \begin{quote}
%   \verb|\PassOptionsToClass{a4paper}{article}|
% \end{quote}
%
% \noindent An example follows how to generate the
% documentation with \pdfLaTeX{}:
%
% \begin{quote}
%\begin{verbatim}
%pdflatex tablefootnote.dtx
%makeindex -s gind.ist tablefootnote.idx
%pdflatex tablefootnote.dtx
%makeindex -s gind.ist tablefootnote.idx
%pdflatex tablefootnote.dtx
%\end{verbatim}
% \end{quote}
%
% \subsection{Compiling the example}
%
% The example file, \textsf{tablefootnote-example.tex}, can be compiled via\\
% \indent |latex tablefootnote-example.tex|,\\
% \indent |lualatex tablefootnote-example.tex|,\\
% or (recommended)\\
% \indent |pdflatex thumbs-example.tex|\\
% and will need at least two compiler runs to get everything right.
%
% \bigskip
%
% \section{Acknowledgements}
%
% I would like to thank \textsc{Heiko Oberdiek} for providing
% the \xpackage{hyperref}, \xpackage{ifluatex}, \xpackage{letltxmacro},
% \xpackage{ltxcmds}, as well as a~lot~(!) of other useful packages
% (from which I also got everything I know about creating a file in
% \xext{dtx} format, OK, say it: copying),
% and for information about the \pdfTeX{}-bug,
% \textsc{Arno Trautmann} for reporting the bug of misplaced hyperreferences in
% |sidewaystable|s using \LuaTeX{}, independently of the \xpackage{tablefootnote}
% package or footnotes at all,
% \textsc{Thomas~V.} for reporting the bug of not handling the case of
% \xpackage{hyperref} option |hyperfootnotes=false|,
% \textsc{Sverre Stausland Johnsen} for reporting the bug in case of \emph{not}
% using \xpackage{hyperref},
% and the \Newsgroup{comp.text.tex} and \Newsgroup{de.comp.text.tex} newsgroups
% as well as everybody at \url{http://tex.stackexchange.com}
% for their help in all things \TeX{}.
%
% \pagebreak
%
% \phantomsection
% \begin{History}\label{History}
%   \begin{Version}{2011/10/26 v1.0a, tabfootn}
%     \item Upload to \CTAN{macros/latex/contrib/tablefootnote/}.
%   \end{Version}
%   \begin{Version}{2011/10/29 v1.0b}
%     \item Renamed to \xpackage{tablefootnote}.
%     \item Added support for the |sidewaystable|-environment of the
%             \xpackage{rotating} package.
%     \item Diverse changes in documentation and README.
%   \end{Version}
%   \begin{Version}{2011/11/06 v1.0c}
%     \item Replaced |^{\text{\thefootnote}}| by |{}^{\text{\textup{\thefootnote}}}|.
%     \item Bug fix: When \xpackage{etoolbox.sty} was not found, and the \xpackage{float}
%             package had been loaded, and the float specifier was \texttt{H},
%             then the |\tablefootnote|s were not printed for that table. Fixed.
%   \end{Version}
%   \begin{Version}{2011/11/19 v1.0d}
%     \item Replaced |{}^{\text{\textup{\thefootnote}}}| by
%             |\textsuperscript{\thefootnote}|, therefore the \xpackage{amstext}
%             package is no longer required by the \xpackage{tablefootnote} package.
%     \item \xpackage{lscape} and \xpackage{pdflscape} package now work with
%             \xpackage{tablefootnote} (except for using a \texttt{sidewaystable} on
%             a \texttt{landscape} page).
%     \item Added information about footnotes in \texttt{sidewaystable} being printed
%             one below the other, even if \xpackage{footmisc} package with option
%             \texttt{para} is used.
%     \item Redefined |\tfn@swtbox| from
%  \begin{verbatim}
%  \vspace{0.8cm}%
%  \begin{spacing}{0.1}%
%  \subcaptionbox*{}[\linewidth][l]{\tfn@tablefootnoteprintout}%
%  \end{spacing}%
%  \end{verbatim}
%             to
%  \begin{verbatim}
%  \vspace{2\baselineskip}%
%  \parbox[b][][t]{\linewidth}{\raggedright\tfn@tablefootnoteprintout}%
%  \end{verbatim}
%             , therefore the \xpackage{setspace} and \xpackage{subcaption} packages
%             are no longer required by the \xpackage{tablefootnote} package.
%     \item No longer uses |\numexpr|, therefore also usable without \eTeX{}.
%     \item Replaced |\textsuperscript{\thefootnote}| by
%             |\hbox {\@textsuperscript {\normalfont \thefootnote }}|,
%             which is next to the original definition of |\@makefnmark|
%             (|\hbox {\@textsuperscript {\normalfont \@thefnmark }}|,
%             but according to |\show\@thefnmark| here |\@thefnmark| is
%             |{\protect \itshape}| instead of |\thefootnote|).
%     \item Added an error message, when \LuaTeX{}, \xpackage{rotating} package, and
%            \xpackage{hyperref} package are used together. Hyperreferences in
%            |sidewaystable|s are misplaced, independently of the
%            \xpackage{tablefootnote} package or footnotes at all. (Bug reported
%            by \textsc{Arno Trautmann}, 2011/11/15 -- thanks!)
%     \item Added footnotesymbol-footnote-numbers in the example.
%   \end{Version}
%   \begin{Version}{2011/11/26 v1.0e}
%     \item Check for \xpackage{hyperref} is only done once.
%     \item Message regarding the misplacement of hyperlinks by \LuaLaTeX{}
%             changed form error to warning.
%     \item Replaced |\vspace{2\baselineskip}| by
%  \begin{verbatim}
% \ifvoid \footins%
% \vskip 2\baselineskip%
% \else%
% \vskip \skip\footins%
% \fi%
%  \end{verbatim}
%             between the |sidewaystable|s and their footnotes.
%     \item The warning about un-regarded |para|-option of \xpackage{footmoisc}
%             package is only given at the second |tablefootnote| in a |sidewaystable|,
%             not repeatedly for every |tablefootnote| in a |sidewaystable|.
%   \end{Version}
%   \begin{Version}{2012/01/01 v1.0f}
%     \item Bugfix: Wrong installation path given in the documentation, fixed.
%     \item Update of documentation, README, and \xfile{dtx} internals.
%   \end{Version}
%   \begin{Version}{2012/01/14 v1.0g}
%     \item Bugfix: There was a |relax| instead of a |\relax|, fixed.
%   \end{Version}
%   \begin{Version}{2012/07/29 v1.0h}
%     \item \xpackage{tablefootnote} is now compatible to the new
%             \xpackage{footnotebackref} package, 2012/07/01, v1.0,
%             \url{http://www.ctan.org/pkg/footnotebackref}.
%   \end{Version}
%   \begin{Version}{2014/01/08 v1.1a}
%     \item Tablefootnotes now regard optional footnote marks:
%             |\tablefootnote[| custom mark here! |]{...}|
%     \item Bug fix: the case of \xpackage{hyperref} option |hyperfootnotes=false|
%             was not handeled at all. (Bug reported by \textsc{Thomas~V.}, %
%             2014/01/02 -- thanks!)
%     \item A lot of details.
%   \end{Version}
%   \begin{Version}{2014/01/20 v1.1b}
%     \item Bug fix: Package was broken when \xpackage{hyperref} was \emph{not} used.
%            (Bug reported by \textsc{Sverre Stausland Johnsen} -- thanks!)
%   \end{Version}
%   \begin{Version}{2014/01/26 v1.1c}
%     \item The emergency bug fix of version 2014/01/20 v1.1b was replaced by using
%             the new robust macro |\tfn@hyperfootnotes|, which only \emph{internally}
%             uses |\ifHy@hyperfootnotes|. If the \xpackage{hyperref} package is not used,
%             \LaTeX{} now does not see any confusing
%             |\ifHy@hyperfootnotes|\ldots|\else|\ldots|\fi| at all.
%     \item URLs and README updated.
%   \end{Version}
% \end{History}
%
% \bigskip
%
% When you find a mistake or have a suggestion for an improvement of this package,
% please send an e-mail to the maintainer, thanks! (Please see BUG REPORTS in the README.)
%
% \bigskip
%
% \PrintIndex
%
% \Finale
\endinput|
% \end{quote}
% Do not forget to quote the argument according to the demands
% of your shell.
%
% \paragraph{Generating the documentation.\label{GenDoc}}
% You can use both the \xfile{.dtx} or the \xfile{.drv} to generate
% the documentation. The process can be configured by a
% configuration file \xfile{ltxdoc.cfg}. For instance, put the following
% line into this file, if you want to have A4 as paper format:
% \begin{quote}
%   \verb|\PassOptionsToClass{a4paper}{article}|
% \end{quote}
%
% \noindent An example follows how to generate the
% documentation with \pdfLaTeX{}:
%
% \begin{quote}
%\begin{verbatim}
%pdflatex tablefootnote.dtx
%makeindex -s gind.ist tablefootnote.idx
%pdflatex tablefootnote.dtx
%makeindex -s gind.ist tablefootnote.idx
%pdflatex tablefootnote.dtx
%\end{verbatim}
% \end{quote}
%
% \subsection{Compiling the example}
%
% The example file, \textsf{tablefootnote-example.tex}, can be compiled via\\
% \indent |latex tablefootnote-example.tex|,\\
% \indent |lualatex tablefootnote-example.tex|,\\
% or (recommended)\\
% \indent |pdflatex thumbs-example.tex|\\
% and will need at least two compiler runs to get everything right.
%
% \bigskip
%
% \section{Acknowledgements}
%
% I would like to thank \textsc{Heiko Oberdiek} for providing
% the \xpackage{hyperref}, \xpackage{ifluatex}, \xpackage{letltxmacro},
% \xpackage{ltxcmds}, as well as a~lot~(!) of other useful packages
% (from which I also got everything I know about creating a file in
% \xext{dtx} format, OK, say it: copying),
% and for information about the \pdfTeX{}-bug,
% \textsc{Arno Trautmann} for reporting the bug of misplaced hyperreferences in
% |sidewaystable|s using \LuaTeX{}, independently of the \xpackage{tablefootnote}
% package or footnotes at all,
% \textsc{Thomas~V.} for reporting the bug of not handling the case of
% \xpackage{hyperref} option |hyperfootnotes=false|,
% \textsc{Sverre Stausland Johnsen} for reporting the bug in case of \emph{not}
% using \xpackage{hyperref},
% and the \Newsgroup{comp.text.tex} and \Newsgroup{de.comp.text.tex} newsgroups
% as well as everybody at \url{http://tex.stackexchange.com}
% for their help in all things \TeX{}.
%
% \pagebreak
%
% \phantomsection
% \begin{History}\label{History}
%   \begin{Version}{2011/10/26 v1.0a, tabfootn}
%     \item Upload to \CTAN{macros/latex/contrib/tablefootnote/}.
%   \end{Version}
%   \begin{Version}{2011/10/29 v1.0b}
%     \item Renamed to \xpackage{tablefootnote}.
%     \item Added support for the |sidewaystable|-environment of the
%             \xpackage{rotating} package.
%     \item Diverse changes in documentation and README.
%   \end{Version}
%   \begin{Version}{2011/11/06 v1.0c}
%     \item Replaced |^{\text{\thefootnote}}| by |{}^{\text{\textup{\thefootnote}}}|.
%     \item Bug fix: When \xpackage{etoolbox.sty} was not found, and the \xpackage{float}
%             package had been loaded, and the float specifier was \texttt{H},
%             then the |\tablefootnote|s were not printed for that table. Fixed.
%   \end{Version}
%   \begin{Version}{2011/11/19 v1.0d}
%     \item Replaced |{}^{\text{\textup{\thefootnote}}}| by
%             |\textsuperscript{\thefootnote}|, therefore the \xpackage{amstext}
%             package is no longer required by the \xpackage{tablefootnote} package.
%     \item \xpackage{lscape} and \xpackage{pdflscape} package now work with
%             \xpackage{tablefootnote} (except for using a \texttt{sidewaystable} on
%             a \texttt{landscape} page).
%     \item Added information about footnotes in \texttt{sidewaystable} being printed
%             one below the other, even if \xpackage{footmisc} package with option
%             \texttt{para} is used.
%     \item Redefined |\tfn@swtbox| from
%  \begin{verbatim}
%  \vspace{0.8cm}%
%  \begin{spacing}{0.1}%
%  \subcaptionbox*{}[\linewidth][l]{\tfn@tablefootnoteprintout}%
%  \end{spacing}%
%  \end{verbatim}
%             to
%  \begin{verbatim}
%  \vspace{2\baselineskip}%
%  \parbox[b][][t]{\linewidth}{\raggedright\tfn@tablefootnoteprintout}%
%  \end{verbatim}
%             , therefore the \xpackage{setspace} and \xpackage{subcaption} packages
%             are no longer required by the \xpackage{tablefootnote} package.
%     \item No longer uses |\numexpr|, therefore also usable without \eTeX{}.
%     \item Replaced |\textsuperscript{\thefootnote}| by
%             |\hbox {\@textsuperscript {\normalfont \thefootnote }}|,
%             which is next to the original definition of |\@makefnmark|
%             (|\hbox {\@textsuperscript {\normalfont \@thefnmark }}|,
%             but according to |\show\@thefnmark| here |\@thefnmark| is
%             |{\protect \itshape}| instead of |\thefootnote|).
%     \item Added an error message, when \LuaTeX{}, \xpackage{rotating} package, and
%            \xpackage{hyperref} package are used together. Hyperreferences in
%            |sidewaystable|s are misplaced, independently of the
%            \xpackage{tablefootnote} package or footnotes at all. (Bug reported
%            by \textsc{Arno Trautmann}, 2011/11/15 -- thanks!)
%     \item Added footnotesymbol-footnote-numbers in the example.
%   \end{Version}
%   \begin{Version}{2011/11/26 v1.0e}
%     \item Check for \xpackage{hyperref} is only done once.
%     \item Message regarding the misplacement of hyperlinks by \LuaLaTeX{}
%             changed form error to warning.
%     \item Replaced |\vspace{2\baselineskip}| by
%  \begin{verbatim}
% \ifvoid \footins%
% \vskip 2\baselineskip%
% \else%
% \vskip \skip\footins%
% \fi%
%  \end{verbatim}
%             between the |sidewaystable|s and their footnotes.
%     \item The warning about un-regarded |para|-option of \xpackage{footmoisc}
%             package is only given at the second |tablefootnote| in a |sidewaystable|,
%             not repeatedly for every |tablefootnote| in a |sidewaystable|.
%   \end{Version}
%   \begin{Version}{2012/01/01 v1.0f}
%     \item Bugfix: Wrong installation path given in the documentation, fixed.
%     \item Update of documentation, README, and \xfile{dtx} internals.
%   \end{Version}
%   \begin{Version}{2012/01/14 v1.0g}
%     \item Bugfix: There was a |relax| instead of a |\relax|, fixed.
%   \end{Version}
%   \begin{Version}{2012/07/29 v1.0h}
%     \item \xpackage{tablefootnote} is now compatible to the new
%             \xpackage{footnotebackref} package, 2012/07/01, v1.0,
%             \url{http://www.ctan.org/pkg/footnotebackref}.
%   \end{Version}
%   \begin{Version}{2014/01/08 v1.1a}
%     \item Tablefootnotes now regard optional footnote marks:
%             |\tablefootnote[| custom mark here! |]{...}|
%     \item Bug fix: the case of \xpackage{hyperref} option |hyperfootnotes=false|
%             was not handeled at all. (Bug reported by \textsc{Thomas~V.}, %
%             2014/01/02 -- thanks!)
%     \item A lot of details.
%   \end{Version}
%   \begin{Version}{2014/01/20 v1.1b}
%     \item Bug fix: Package was broken when \xpackage{hyperref} was \emph{not} used.
%            (Bug reported by \textsc{Sverre Stausland Johnsen} -- thanks!)
%   \end{Version}
%   \begin{Version}{2014/01/26 v1.1c}
%     \item The emergency bug fix of version 2014/01/20 v1.1b was replaced by using
%             the new robust macro |\tfn@hyperfootnotes|, which only \emph{internally}
%             uses |\ifHy@hyperfootnotes|. If the \xpackage{hyperref} package is not used,
%             \LaTeX{} now does not see any confusing
%             |\ifHy@hyperfootnotes|\ldots|\else|\ldots|\fi| at all.
%     \item URLs and README updated.
%   \end{Version}
% \end{History}
%
% \bigskip
%
% When you find a mistake or have a suggestion for an improvement of this package,
% please send an e-mail to the maintainer, thanks! (Please see BUG REPORTS in the README.)
%
% \bigskip
%
% \PrintIndex
%
% \Finale
\endinput
%        (quote the arguments according to the demands of your shell)
%
% Documentation:
%    (a) If tablefootnote.drv is present:
%           (pdf)latex tablefootnote.drv
%           makeindex -s gind.ist tablefootnote.idx
%           (pdf)latex tablefootnote.drv
%           makeindex -s gind.ist tablefootnote.idx
%           (pdf)latex tablefootnote.drv
%    (b) Without tablefootnote.drv:
%           (pdf)latex tablefootnote.dtx
%           makeindex -s gind.ist tablefootnote.idx
%           (pdf)latex tablefootnote.dtx
%           makeindex -s gind.ist tablefootnote.idx
%           (pdf)latex tablefootnote.dtx
%
%    The class ltxdoc loads the configuration file ltxdoc.cfg
%    if available. Here you can specify further options, e.g.
%    use DIN A4 as paper format:
%       \PassOptionsToClass{a4paper}{article}
%
% Installation:
%    TDS:tex/latex/tablefootnote/tablefootnote.sty
%    TDS:doc/latex/tablefootnote/tablefootnote.pdf
%    TDS:doc/latex/tablefootnote/tablefootnote-example.tex
%    TDS:doc/latex/tablefootnote/tablefootnote-example.pdf
%    TDS:source/latex/tablefootnote/tablefootnote.dtx
%
%<*ignore>
\begingroup
  \catcode123=1 %
  \catcode125=2 %
  \def\x{LaTeX2e}%
\expandafter\endgroup
\ifcase 0\ifx\install y1\fi\expandafter
         \ifx\csname processbatchFile\endcsname\relax\else1\fi
         \ifx\fmtname\x\else 1\fi\relax
\else\csname fi\endcsname
%</ignore>
%<*install>
\input docstrip.tex
\Msg{******************************************************************}
\Msg{* Installation                                                   *}
\Msg{* Package: tablefootnote 2014/01/26 v1.1c Table foot notes (HMM) *}
\Msg{******************************************************************}

\keepsilent
\askforoverwritefalse

\let\MetaPrefix\relax
\preamble

This is a generated file.

Project: tablefootnote
Version: 2014/01/26 v1.1c

Copyright (C) 2011 - 2014 by
    H.-Martin M"unch <Martin dot Muench at Uni-Bonn dot de>

The usual disclaimer applies:
If it doesn't work right that's your problem.
(Nevertheless, send an e-mail to the maintainer
 when you find an error in this package.)

This work may be distributed and/or modified under the
conditions of the LaTeX Project Public License, either
version 1.3c of this license or (at your option) any later
version. This version of this license is in
   http://www.latex-project.org/lppl/lppl-1-3c.txt
and the latest version of this license is in
   http://www.latex-project.org/lppl.txt
and version 1.3c or later is part of all distributions of
LaTeX version 2005/12/01 or later.

This work has the LPPL maintenance status "maintained".

The Current Maintainer of this work is H.-Martin Muench.

This work consists of the main source file tablefootnote.dtx,
the README, and the derived files
   tablefootnote.sty, tablefootnote.pdf,
   tablefootnote.ins, tablefootnote.drv,
   tablefootnote-example.tex, tablefootnote-example.pdf.

In memoriam Tommy Muench + 2014/01/02.

\endpreamble
\let\MetaPrefix\DoubleperCent

\generate{%
  \file{tablefootnote.ins}{\from{tablefootnote.dtx}{install}}%
  \file{tablefootnote.drv}{\from{tablefootnote.dtx}{driver}}%
  \usedir{tex/latex/tablefootnote}%
  \file{tablefootnote.sty}{\from{tablefootnote.dtx}{package}}%
  \usedir{doc/latex/tablefootnote}%
  \file{tablefootnote-example.tex}{\from{tablefootnote.dtx}{example}}%
}

\catcode32=13\relax% active space
\let =\space%
\Msg{************************************************************************}
\Msg{*}
\Msg{* To finish the installation you have to move the following}
\Msg{* file into a directory searched by TeX:}
\Msg{*}
\Msg{*     tablefootnote.sty}
\Msg{*}
\Msg{* To produce the documentation run the file `tablefootnote.drv'}
\Msg{* through (pdf)LaTeX, e.g.}
\Msg{*  pdflatex tablefootnote.drv}
\Msg{*  makeindex -s gind.ist tablefootnote.idx}
\Msg{*  pdflatex tablefootnote.drv}
\Msg{*  makeindex -s gind.ist tablefootnote.idx}
\Msg{*  pdflatex tablefootnote.drv}
\Msg{*}
\Msg{* At least three runs are necessary e.g. to get the}
\Msg{*  references right!}
\Msg{*}
\Msg{* Happy TeXing!}
\Msg{*}
\Msg{************************************************************************}

\endbatchfile
%</install>
%<*ignore>
\fi
%</ignore>
%
% \section{The documentation driver file}
%
% The next bit of code contains the documentation driver file for
% \TeX{}, i.\,e., the file that will produce the documentation you
% are currently reading. It will be extracted from this file by the
% \texttt{docstrip} programme. That is, run \LaTeX{} on \texttt{docstrip}
% and specify the \texttt{driver} option when \texttt{docstrip}
% asks for options.
%
%    \begin{macrocode}
%<*driver>
\NeedsTeXFormat{LaTeX2e}[2011/06/27]
\ProvidesFile{tablefootnote.drv}%
  [2014/01/26 v1.1c Table foot notes (HMM)]
\documentclass{ltxdoc}[2007/11/11]% v2.0u
\usepackage{holtxdoc}[2012/03/21]%  v0.24
%% tablefootnote may work with earlier versions of LaTeX2e and those
%% class and package, but this was not tested.
%% Please consider updating your LaTeX, class, and package
%% to the most recent version (if they are not already the most
%% recent version).
\hypersetup{%
 pdfsubject={Footnotes in Tables (HMM)},%
 pdfkeywords={LaTeX, tablefootnote, footnote, table, H.-Martin Muench},%
 pdfencoding=auto,%
 pdflang={en},%
 breaklinks=true,%
 linktoc=all,%
 pdfstartview=FitH,%
 pdfpagelayout=OneColumn,%
 bookmarksnumbered=true,%
 bookmarksopen=true,%
 bookmarksopenlevel=3,%
 pdfmenubar=true,%
 pdftoolbar=true,%
 pdfwindowui=true,%
 pdfnewwindow=true%
}
\CodelineIndex
\hyphenation{printing docu-ment}
\gdef\unit#1{\mathord{\thinspace\mathrm{#1}}}%
\makeatletter
\@ifundefined{eTeX}{\gdef\eTeX{$\m@th \varepsilon $-\TeX }}{% else \relax
}
\makeatother
\begin{document}
  \DocInput{tablefootnote.dtx}%
\end{document}
%</driver>
%    \end{macrocode}
%
% \fi
%
% \CheckSum{584}
%
% \CharacterTable
%  {Upper-case    \A\B\C\D\E\F\G\H\I\J\K\L\M\N\O\P\Q\R\S\T\U\V\W\X\Y\Z
%   Lower-case    \a\b\c\d\e\f\g\h\i\j\k\l\m\n\o\p\q\r\s\t\u\v\w\x\y\z
%   Digits        \0\1\2\3\4\5\6\7\8\9
%   Exclamation   \!     Double quote  \"     Hash (number) \#
%   Dollar        \$     Percent       \%     Ampersand     \&
%   Acute accent  \'     Left paren    \(     Right paren   \)
%   Asterisk      \*     Plus          \+     Comma         \,
%   Minus         \-     Point         \.     Solidus       \/
%   Colon         \:     Semicolon     \;     Less than     \<
%   Equals        \=     Greater than  \>     Question mark \?
%   Commercial at \@     Left bracket  \[     Backslash     \\
%   Right bracket \]     Circumflex    \^     Underscore    \_
%   Grave accent  \`     Left brace    \{     Vertical bar  \|
%   Right brace   \}     Tilde         \~}
%
% \GetFileInfo{tablefootnote.drv}
%
% \begingroup
%   \def\x{\#,\$,\^,\_,\~,\ ,\&,\{,\},\%}%
%   \makeatletter
%   \@onelevel@sanitize\x
% \expandafter\endgroup
% \expandafter\DoNotIndex\expandafter{\x}
% \expandafter\DoNotIndex\expandafter{\string\ }
% \begingroup
%   \makeatletter
%     \lccode`9=32\relax
%     \lowercase{%^^A
%       \edef\x{\noexpand\DoNotIndex{\@backslashchar9}}%^^A
%     }%^^A
%   \expandafter\endgroup\x
%
% \DoNotIndex{\\,\,}
% \DoNotIndex{\documentclass,\usepackage,\ProvidesPackage,\begin,\end}
% \DoNotIndex{\MessageBreak,\message}
% \DoNotIndex{\NeedsTeXFormat,\DoNotIndex,\verb}
% \DoNotIndex{\def,\edef,\gdef,\global}
% \DoNotIndex{\ifx,\listfiles,\mathord,\mathrm}
% \DoNotIndex{\kvoptions,\SetupKeyvalOptions,\ProcessKeyvalOptions}
% \DoNotIndex{\bigskip,\space,\thinspace,\ldots}
% \DoNotIndex{\indent,\noindent,\newline,\linebreak,\pagebreak}
% \DoNotIndex{\textbf,\textit,\textsf,\texttt,\textsc,\textquotedblleft,\textquotedblright}
% \DoNotIndex{\plainTeX,\TeX,\LaTeX,\pdfLaTeX}
% \DoNotIndex{\chapter,\section,\the,\csname,\endscname}
% \DoNotIndex{\value,\arabic,\hline,\text,\caption,\centering,\clear,\url}
%
% \title{The \xpackage{tablefootnote} package}
% \date{2014/01/26 v1.1c}
% \author{H.-Martin M\"{u}nch\\\xemail{Martin.Muench at Uni-Bonn.de}}
%
% \maketitle
%
% \begin{abstract}
% This \LaTeX{} package provides the command |\tablefootnote| to be used
% in a \texttt{table} or \texttt{sidewaystable} environment,
% where |\footnote| will not work and when using |\footnotemark| and
% |\footnotetext| and adjusting the counters (including \texttt{Hfootnote})
% manually is either too much work or would not even work (\texttt{sidewaystable}).
% \end{abstract}
%
% \bigskip
%
% \noindent Disclaimer for web links: The author is not responsible for any contents
% referred to in this work unless he has full knowledge of illegal contents.
% If any damage occurs by the use of information presented there, only the
% author of the respective pages might be liable, not the one who has referred
% to these pages.
%
% \bigskip
%
% \noindent {\color{green} Save per page about $200\unit{ml}$ water,
% $2\unit{g}$ CO$_{2}$ and $2\unit{g}$ wood:\\
% Therefore please print only if this is really necessary.}
%
% \pagebreak
%
% \tableofcontents
%
% \pagebreak
%
% \section{Introduction\label{sec:Introduction}}
%
% \indent This \LaTeX{} package provides the command |\tablefootnote{...}| to be
% used in a \texttt{table} or \texttt{sidewaystable} environment. In those
% environments |\footnote{...}| would not work and would need to be replaced
% by |\footnotemark| in the \texttt{(sideways)\allowbreak table} and
% |\footnotetext{...}| after the \texttt{(sideways)table} environment ended.
% Additionally the counters for \texttt{footnote} and
% (when the \xpackage{hyperref} package is used) \texttt{Hfootnote}
% would need to be adjusted manually. Then still the hyperlinks
% in a \texttt{(sideways)table} would not work.
% When the \xpackage{footnotebackref} package is used,
% |\footnotemark| and |\footnotetext{...}| would need to be redefined.
% Just using |\tablefootnote{...}| (with optional argument for custom foot note
% marks) in the (sideways)table does all this automatically.
% (Compatibility with the \xpackage{footnotebackref} package
% was achieved without redefining |\footnotemark| or |\footnotetext|.)
% Redefining |\footnote| was \emph{not} done as to not create conflicts with
% other packages.
%
% \section{Usage}
%
% \indent Just load the package placing
% \begin{quote}
%   |\usepackage{tablefootnote}|
% \end{quote}
% \noindent in the preamble of your \LaTeXe{} source file and use\newline
% |\tablefootnote{...}| instead of |\footnote{...}| and\newline
% |\tablefootnote[...]{...}| instead of |\footnote[...]{...}|
% in (sideways)\allowbreak tables.\\
% When the \xpackage{footnotebackref} package is used,
% it must be loaded before the \xpackage{tablefootnote} package
% and the \xpackage{hyperrefpackage} with option |hyperfootnotes=true|
% must be loaded, too. (Backreference{-}links without links do not make sense.)\\
% When the \xpackage{rotating} package (for |sidewaystable|s) is used,
% it must be loaded before the \xpackage{tablefootnote} package and it must be
% a recent version (v2.16a, 2009/03/28, or newer).\\
% If |\tablefootnote{...}| is used in a tabular environment, which is inside
% of a (sideways)table environment, everything is fine, but when the tabular
% environment is not inside of a (sideways)table, the \textbf{table}footnote
% will not work.\\
%
% Pages with tables with footnotes, which are rotated with the \xpackage{lscape}-
% or with the \xpackage{pdflscape}-package (|\begin{landscape}|, table, |\end{landscape}|),
% are possible. But when the \xpackage{footmisc}-package is used, \xpackage{(pdf)lscape}
% should be loaded before \xpackage{footmisc}.\\
%
% When the \xpackage{footmisc}-package is used with option \texttt{para},
% this is regarded with the exception of a \texttt{sidewaystable}. There the footnotes
% are printed one below the other. (Well, this is better than no footnote in a
% \texttt{sidewaystable} at all, isn't it?)\\
%
% When the (sideways)table floats, the footnotes are set and (when \xpackage{hyperref}
% is used) hyperlinked, but they are not automatically adapted when the table floats
% over/under another footnote. Thus either do not use a footnote between original and
% float position of the (sideways)table, or place the (sideways)table in
% \textquotedblleft here\textquotedblright\ position. |\clear(double)page|, |h(!)|,
% |H|~from the \xpackage{float} package (\url{http://www.ctan.org/pkg/float}),
% or |\FloatBarrier| from the \xpackage{picins} package
% (\url{http://www.ctan.org/pkg/picins}) might help, too.
% (Or after finishing the document, move the (sideways)table in the source code
% near the position where it floats to or use the optional footnote marks.)
%
% \pagebreak
%
% \section{Alternatives\label{sec:Alternatives}}
%
% \begin{description}
% \item[-] The \xpackage{longtable} package provides the \texttt{longtable} environment
%   as replacement for the combined \texttt{table} and \texttt{tabular} environments.
%   Footnotes are real footnotes (not just tablenotes), are continuously numbered and
%   hyperlinked (when using the \xpackage{hyperref} package), and the hyperlinks really
%   work. As drawback the appearance of the caption changes slightly (e.\,g.~distance
%   to the table, width of the caption), which can probably be changed back manually.
%   Furthermore, longtables are meaned to break over more than one page. If that is not
%   wished, it must be prevented by |\nopagebreak|-commands and by ending the longtable
%   lines with |\\*| instead of |\\|. longtables do not float. (Therefore using the
%   \xpackage{tablefootnote} package and |\FloatBarrier| from the \xpackage{picins} package
%   before and after the table environment is similar~- but \xpackage{tablefootnote} does not
%   change the table-caption!) \texttt{sidewaystable} does not work with it.\\
%   \url{http://www.ctan.org/pkg/longtable}
%
% \item[-] The \xpackage{supertabular} package provides the \texttt{mpsupertabular}
%   environment as replacement for the combined \texttt{table} and \texttt{tabular}
%   environments. Footnotes are just tablenotes (with working hyperlinks when using the
%   \xpackage{hyperref} package), i.\,e. numbered a, b, c and placed below the table
%   and not at the end of the page. Therefore there is no float problem (because the
%   tablenotes numbering is not included in the continuous numbering of the footnotes).
%   Placing the \texttt{supertabular} inside of a \texttt{sidewaystable} breaks the
%   hyperlinks to the tablenotes.\\
%   \url{http://www.ctan.org/pkg/supertabular}
%
% \item[-] The \xpackage{ctable} package has its very own notation for defining tables.
%   It can create tablenotes and sideways-tables. The tablenotes are not automatically
%   hyperlinked. The \texttt{ctable}s float. Because the tablenotes numbering is not
%   included in the continuous numbering of the footnotes there is no float problem.\\
%   \url{http://www.ctan.org/pkg/ctable}
%
% \item[-] The \xpackage{footnote} package provides |\makesavenoteenv{table}|. After
%   loading the package and using that command in the preamble, in tables |\footnote{...}|
%   can be used. Using |\makesavenoteenv{tabular}| and \\
%   |\makesavenoteenv{sidewaystable}|
%   is possible, but it neither solves the float problem, nor do the created hyperlinks
%   work (i.\,e.~they aim at wrong locations). The \xpackage{mdwtab} from the same bundle
%   is incompatible with other table-related packages (e.\,g. \xpackage{supertabular},
%   \xpackage{array}) and not $100\,\%$ compatible with the \texttt{tabular} environment.\\
%   \url{http://www.ctan.org/pkg/footnote}\\
%   \url{http://www.ctan.org/pkg/mdwtab}
%
% \item[-] The \xpackage{tabularx} package does produce footnotes for sidewaystables,
%   but uses a, b, c instead of 1, 2, 3. The hyperlinks to the footnotes do not work.
%   Because the footnotes numbering is not included in the continuous numbering of
%   the other footnotes there is no float problem.\\
%   \url{http://www.ctan.org/pkg/tabularx}
%
% \item[-] Placing a tabular inside a minipage inside a table produces tablenotes.
%   Therefore there is no float problem (because the footnotes are not continuously
%   numbered). The hyperlinks to the table notes indeed work.
%
% \pagebreak
%
% \item[-] The \xpackage{threeparttable} package creates tablenotes again. Therefore
%   there is no float problem (because the tablenotes are not continuously numbered with
%   the footnotes). There are no hyperlinks to the table notes (at least not automatically).
%   Using |sidewaystable| (with table notes) works.\\
%   \url{http://www.ctan.org/pkg/threeparttable}
%
% \item[-] The \xpackage{threeparttablex} package creates tablenotes again. Therefore there
%   is no float problem (because the tablenotes are not continuously numbered with the
%   footnotes). With option \texttt{referable} the tablenotes are hyperlinked. Use of
%   a |sidewaystable| (with table notes) did not work for me. When using the
%   \texttt{referable} option according to the example in the \xpackage{threeparttablex}
%   manual the \xpackage{longtable} package is used, therefore that package could be used
%   directly without \xpackage{threeparttablex} (see above).\\
%   \url{http://www.ctan.org/pkg/threeparttablex}
%
% \item[-] One can manually use |\footnotemark| in the table and |\footnotetext{...}|
%  after the table environment ended and manually change the |footnote| and
%  (when the \xpackage{hyperref} package is used) |Hfootnote| counters and needs
%  to insert
%  \begin{verbatim}
%     \makeatletter
%     \global\let\Hy@saved@currentHref\@currentHref
%     \hyper@makecurrent{Hfootnote}%
%     \global\let\Hy@footnote@currentHref\@currentHref
%     \global\let\@currentHref\Hy@saved@currentHref
%     \makeatother
%  \end{verbatim}
%  before each |\footnotetext{...}|, but with more than very few footnotes this can become
%  much work (and is prone to produce errors). And this does not even work with
%  sidewaystables. (And it does not work with the \xpackage{footnotebackref} package
%  without redefining |\footnotemark| and |\footnotetext{...}|.)
% \end{description}
%
% \bigskip
%
% \noindent (You programmed or found another alternative,
%  which is available at \CTAN{}?\\
%  OK, send an e-mail to me with the name, location at \CTAN{},
%  and a short notice, and I will probably include it in
%  the list above.)
%
% \newpage
%
% \section{Example}
%
%    \begin{macrocode}
%<*example>
\documentclass[british]{article}[2007/10/19]% v1.4h
%%%%%%%%%%%%%%%%%%%%%%%%%%%%%%%%%%%%%%%%%%%%%%%%%%%%%%%%%%%%%%%%%%%%%
\usepackage{float}[2001/11/08]%    v1.3d
\usepackage{placeins}[2005/04/18]% v2.2  ; for \FloatBarrier
\usepackage{rotating}[2009/03/28]% v2.16a; for sidewaystable-environment
\usepackage[%
 hyperfootnotes=true,%
 extension=pdf,%
 plainpages=false,%
 pdfpagelabels=true,%
 hyperindex=false,%
 pdflang={en},%
 pdftitle={tablefootnote package example},%
 pdfauthor={H.-Martin Muench},%
 pdfsubject={Example for the tablefootnote package},%
 pdfkeywords={LaTeX, tablefootnote, footnote, table, H.-Martin Muench},%
 % pdfview=FitH and FitBH do not work: hyperlinks in sidewaystables
 % do no lead to the footnotes, due to a bug in pdfTeX,
 % computing wrong anchor coordinates (Heiko Oberdiek, 29. October 2011)
 % pdfview=Fit, FitV, FitR, FitB, FitBV work
 % print is OK for all those options
 pdfstartview=FitH,%
 pdfpagelayout=OneColumn%
]{hyperref}[2012/11/06]% v6.83m
 % Due to the urls used in the example, either the hyperref or the url
 % package are needed (or the urls must be removed before compiling).

\usepackage{footnotebackref}[2012/07/01]% v1.0
\usepackage{tablefootnote}[2014/01/26]%   v1.1c

\gdef\unit#1{\mathord{\thinspace\mathrm{#1}}}%
\listfiles
\begin{document}
\pagenumbering{arabic}
\section*{Example for tablefootnote}

This example demonstrates the use of package\newline
\textsf{tablefootnote}, v1.1c as of 2014/01/26 (HMM).\newline
There were no options used. (The package provides no options.)\newline

\textbf{The \texttt{tablefootnote-example.tex} needs to be compiled
at least twice to get the references right!}\newline

If the etoolbox-package is found, it is automatically used.\newline

For more details please see the documentation!\newline

\noindent Save per page about $200\unit{ml}$ water,
$2\unit{g}$ CO$_{2}$ and $2\unit{g}$ wood:\newline
Therefore please print only if this is really necessary.\newline

Here is some text.\footnote{And this is a text footnote.}\newline

Tables \ref{tab.symbol}, \ref{tab.normal}, \ref{tab.another} and
\ref{tab.floatH} show normal tables, table~\ref{tab.sideways} depicts
a sidewaystable. Table~\ref{tab.floatH} uses the float
specifier~\texttt{H} from the float package.\newline

\texttt{Hyperref} option \verb|pdfview=FitH| and \verb|FitBH| do
not work due to a bug in pdf\TeX{}, computing wrong
anchor coordinates (\textsc{Heiko Oberdiek}, 29.~October 2011).
Depending on used pdf-viewer, hyperlinks in sidewaystables lead
e.\,g.~at the end of the document, not at the footnote.
\verb|pdfview=Fit|, \verb|FitV|, \verb|FitR|, \verb|FitB|,
\verb|FitBV| work, print is OK for all those options.

\bigskip

\listoftables

\pagebreak

\renewcommand{\thefootnote}{\fnsymbol{footnote}}
\verb|\renewcommand{\thefootnote}{\fnsymbol{footnote}}|
causes footnote{-}symbol{-}footnotes, which are possible
(see Table~\ref{tab.symbol}).

\begin{table}
\centering
\begin{tabular}{ccc}
Another\tablefootnote{A table footnote.} & %
text\tablefootnote{Another table footnote.} & %
in a table\tablefootnote{A third table footnote.}
\end{tabular}
\caption[A footnotesymbol table]{%
A table with footnote-symbol-footnotes.\label{tab.symbol}}
\end{table}

Some text.\footnote{A text footnote.}

\renewcommand{\thefootnote}{\arabic{footnote}}
\verb|\renewcommand{\thefootnote}{\arabic{footnote}}|
switches back to normal footnote numbers again.

\pagebreak

\begin{table}
\centering
\begin{tabular}{ccc}
Some\tablefootnote{A table footnote.} & %
text\tablefootnote[99]{A table footnote with custom footnote number.} & %
in a table\tablefootnote{A third table footnote.}
\end{tabular}
\caption[A table]{A normal table.\label{tab.normal}}
\end{table}

Some text.\footnote{Another text footnote.}

\pagebreak

%    \end{macrocode}
%
% \pagebreak
%
%    \begin{macrocode}
More text.\footnote{And yet another text footnote.}

\begin{table}[t]
\centering
\begin{tabular}{|c|c|c|}
\hline
Another\tablefootnote{A $3^{rd}$ table footnote.} & %
text\tablefootnote{Another $3^{rd}$ table footnote.} & %
in a table\tablefootnote{A $3^{rd}$ third table footnote.}\\ \hline
\end{tabular}
\caption[Another table]{Another table (third one)\label{tab.another}}
\end{table}

Please note that Table~\ref{tab.another} floated to the top of the page.
While the footnotes are set and (when hyperref is used) hyperlinked,
they are not automatically adapted. Thus either do not use a footnote
at the same page before the table, or place the table in
\textquotedblleft here\textquotedblright\ or
\textquotedblleft bottom\textquotedblright\ position.
\verb|\clear(double)page|, \verb|h(!)|, \verb|H|~from the
\texttt{float} package, or \verb|\FloatBarrier| from the
\texttt{picins} package might help, too. (Or move the table in the
source code near the position where it floats to
or use the optional footnote marks.)

Table~\ref{tab.floatH} (page~\pageref{tab.floatH}) uses float specifier
\texttt{H} from the float package and does not float.

Some text.\footnote{This is just another text footnote.}

\pagebreak

\FloatBarrier

\begin{sidewaystable}
\centering%
\begin{tabular}{ccc}
Text\tablefootnote{Please rotate the view for testing the %
hyperlinks.} & %
in a\tablefootnote[98]{Another sidewaystable footnote %
with optional footnote mark.} & %
sidewaystable%
\tablefootnote{Sidewaystable-environment provided by %
\url{http://www.ctan.org/pkg/rotating} package.}
\end{tabular}
\caption[A sideways table]{A table in the \texttt{sideways} %
environment\label{tab.sideways}}%
\end{sidewaystable}%

\FloatBarrier

\pagebreak

%    \end{macrocode}
%
% \pagebreak
%
%    \begin{macrocode}
A last table, here with float specifier \texttt{H} from the
float\footnote{\url{http://www.ctan.org/pkg/float}} package.

\begin{table}[H]
\centering
\begin{tabular}{ccc}
Another\tablefootnote{A $5^{th}$ table footnote.} & %
text\tablefootnote{Another $5^{th}$ table footnote.} & %
in a table\tablefootnote{A $5^{th}$ third table footnote.}
\end{tabular}
\caption[A last table]{A very last table\label{tab.floatH}}
\end{table}

Some text.\footnote{This is just another text footnote.}

\pagebreak

End of the example for the
tablefootnote\footnote{\url{http://www.ctan.org/pkg/tablefootnote}}
package.

\end{document}
%</example>
%    \end{macrocode}
%
% \StopEventually{}
%
% \newpage
%
% \section{The implementation}
%
% We start off by checking that we are loading into \LaTeXe{} and
% announcing the name and version of this package.
%
%    \begin{macrocode}
%<*package>
%    \end{macrocode}
%
%    \begin{macrocode}
\NeedsTeXFormat{LaTeX2e}[2011/06/27]
\ProvidesPackage{tablefootnote}[2014/01/26 v1.1c
              Table foot notes (HMM)]
%    \end{macrocode}
%
% A short description of the \xpackage{tablefootnote} package:
%
%    \begin{macrocode}
%% Provides the \tablefootnote[...]{...} command
%% for footnotes in (sideways)tables.

%    \end{macrocode}
%
% We need the \xpackage{ltxcmds} and \xpackage{letltxmacro} packages
% by \textsc{Heiko Oberdiek}:
%
%    \begin{macrocode}
\RequirePackage{ltxcmds}[2011/11/09]%     v1.22
\RequirePackage{letltxmacro}[2010/09/02]% v1.4
\RequirePackage{xifthen}[2009/04/17]%     v1.3
%    \end{macrocode}
%
% A last information for the user:
%
%    \begin{macrocode}
%% tablefootnote may work with earlier versions of LaTeX and those
%% packages, but this was not tested. Please consider updating
%% your LaTeX and packages to the most recent version
%% (if they are not already the most recent version).

%    \end{macrocode}
%
% \indent See subsection~\ref{ss:Downloads} about how to get them.\\
%
% When the \xpackage{rotating} package (for |sidewaystable|s) is used,
% it must be loaded before the \xpackage{tablefootnote} package and it must be
% a recent version:
%
%    \begin{macrocode}
\@ifpackageloaded{rotating}{%
  \gdef\tfn@rotating{1}
  \@ifpackagelater{rotating}{2009/03/28}% v2.16a
   {% >= 2009/03/28, OK
   }{% else
     \PackageWarningNoLine{tablefootnote}{%
       It is required version\MessageBreak%
       2009/03/28 v2.16a (or later) of package rotating,\MessageBreak%
       but only version\MessageBreak%
      `\csname ver@rotating.sty\endcsname'\MessageBreak%
      is available}%
   }%
 }{% else
   \gdef\tfn@rotating{0}%
 }

%    \end{macrocode}
%
% There are no options to be processed.\\
%
% We need some definitions:
%
%    \begin{macrocode}
\def\tfn@footnotetablecount{0}
\def\tfn@footnotetableprint{0}
\def\tfn@footnotezero{0}
\def\tfn@fnt{0}
\def\tfn@swt{0}%       _s_ide_w_ays_t_able
\def\tfn@fmpw{\relax}% f_oot_m_isc_p_ara_w_arn-switch
\def\tfn@fnbr{0}%      f_oot_n_ote_b_ack_r_ef(erence package)

%    \end{macrocode}
%
% \begin{macro}{\tfn@hyperfootnotes}
% We need a wrapper for the used |\ifHy@hyperfootnotes|,
% which is not defined when the \xpackage{hyperref} package has not been loaded:
%
%    \begin{macrocode}
\DeclareRobustCommand{\tfn@hyperfootnotes}[2]{%
  \ifHy@hyperfootnotes%
    #1%
  \else%
    #2%
  \fi%
  }

%    \end{macrocode}
% \end{macro}
%
% \begin{macro}{\tablemakefntext}
% |\tablemakefntext| makes the footnotetext for a footnote in a table.
%
%    \begin{macrocode}
\ltx@ifpackageloaded{footnotebackref}{%
  \gdef\tfn@fnbr{1}%
  \ifx\FootnoteBackref@symbol\empty%
    \ifFootnoteBackref@numberlinked%
      \newcommand\tablemakefntext[1]{%
        \noindent\makebox[1.634em][r]{%
         \mbox{\textsuperscript{\normalfont%
           \hyperlink{tfn:m\theHfootnote}{\@thefnmark}%
        }}}%
        \ #1%
      }
    \else
      \newcommand\tablemakefntext[1]{%
        \noindent\makebox[1.8em][r]{%
          \mbox{\textsuperscript{\normalfont%
            \@thefnmark%
        }}}%
        #1%
      }
    \fi
  \else
    \newcommand\tablemakefntext[1]{%
      \noindent\makebox[1.8em][r]{%
        \mbox{\textsuperscript{\normalfont%
          \ifFootnoteBackref@numberlinked%
            \hyperlink{tfn:m\theHfootnote}{\@thefnmark}\ %
          \else%
            \@thefnmark%
          \fi%
      }}}%
      \hyperlink{tfn:m\theHfootnote}{\FootnoteBackref@symbol}%
      #1%
    }
  \fi
}{%
  \newcommand\tablemakefntext[1]{%
    \noindent\makebox[1.8em][r]{%
      \mbox{\textsuperscript{\normalfont%
        \@thefnmark%
    }}}%
    #1%
  }
}

%    \end{macrocode}
%
% \end{macro}
%
% \pagebreak
%
% \begin{macro}{\swtablemakefntext}
%    \begin{macrocode}
\newcommand\swtablemakefntext[1]{\noindent\ #1}

%    \end{macrocode}
% \end{macro}
%
% \begin{macro}{\tfn@footnotemarks}
% |\tfn@footnotemarks| uses the |\footnotemark| command to place a foot note mark
% and decreases the |\tfn@footnotetablecount| (i.\,e.~increases the negative value)
% of the number of footnote(marks) in the table. If the \xpackage{footnotebackref}
% package is used, it is also necessary to create an according |\hypertarget| for
% the back reference.
%
%    \begin{macrocode}
\newcommand{\tfn@footnotemarks}[1][]{%
  \hbox{%
    \ifthenelse{\isempty{#1}}%
      {\footnotemark%
       \ltx@ifpackageloaded{footnotebackref}%
         {\raisebox{\ht\strutbox}{\hypertarget{tfn:m\theHfootnote}{}}}%
         {% footnotebackref not loaded
         }%
      }%
      {\ltx@ifpackageloaded{hyperref}%
        {\tfn@hyperfootnotes{%
            \hyperlink{tfn:#1}{\footnotemark[#1]}%
          }{\footnotemark[#1]}}%
        {\footnotemark[#1]}%
       \ltx@ifpackageloaded{footnotebackref}%
        {\raisebox{\ht\strutbox}{\hypertarget{tfn:m#1}{}}}{}%
      }%
   }%
  \@tempcnta=\tfn@footnotetablecount\relax%
  \advance\@tempcnta by -1\relax%
  \xdef\tfn@footnotetablecount{\the\@tempcnta}%
}

%    \end{macrocode}
% \end{macro}
%
% \begin{macro}{\tfn@footmiscparawarn}
% When the \xpackage{footmisc} package is used with option |para|,
% we need to issue a warning. (Any idea how to resolve this issue?)\\
% It must be given inside of it's own command, because |\ifFN@para| is not
% defined when \xpackage{footmisc} has not been loaded, and then the
% |\fi| would be a |\fi| without |\if| (and \TeX{} does not like single
% |\fi|s).
%
%    \begin{macrocode}
\newcommand{\tfn@footmiscparawarn}{%
  \ifFN@para%
    \PackageWarning{tablefootnote}{%
      Package footmisc with option para used.\MessageBreak%
      Unfortunately package tablefootnote is not yet\MessageBreak%
      able to regard this in sidewaystables.\MessageBreak%
      Here the footnotes are printed one below\MessageBreak%
      the other. Sorry!\MessageBreak%
     }%
  \fi%
  }

%    \end{macrocode}
% \end{macro}
%
% \begin{macro}{\tfn@updateHyper}
% |\tfn@updateHyper| updates some \xpackage{hyperref} internals.
%
%    \begin{macrocode}
\newcommand{\tfn@updateHyper}{%
  % The code of this command is partly from the
  % hyperref package by Heiko Oberdiek, 2011/10/01, v6.82j.
  % Newer versions of that package are available at CTAN.org.
  \global\let\Hy@saved@currentHref\@currentHref%
  \tfn@hyperfootnotes{\hyper@makecurrent{Hfootnote}}{}%
  \global\let\Hy@footnote@currentHref\@currentHref%
  \global\let\@currentHref\Hy@saved@currentHref%
  }

%    \end{macrocode}
% \end{macro}
%
% \begin{macro}{\tfn@footnotetext}
% |\tfn@footnotetext| contains the |\footnotetext[...]{...}| command and handles
% the numbering. |\tfn@footnotetext| will be executed after the table.
% The foot note counter has been increased automatically with each |\footnotemark|,
% therefore it must be reset to its value before the table began.
%
%    \begin{macrocode}
\newcommand{\tfn@footnotetext}[2][]{%
  \ifx\tfn@footnotetableprint\tfn@footnotezero%
    % This is the first footnotetext after the (sideways)table.
    \addtocounter{footnote}{\tfn@footnotetablecount}%
%    \end{macrocode}
%
% When the \xpackage{hyperref} package is used, also its |Hfootnote| counter must
% be reset (when the footnotes are hyperlinked).
%
%    \begin{macrocode}
    \ltx@ifpackageloaded{hyperref}{%
      \tfn@hyperfootnotes{%
        \addtocounter{Hfootnote}{\tfn@footnotetablecount}}{}%
      \tfn@updateHyper%
     }{}%
  \fi%
%    \end{macrocode}
%
% For issuing the text for the next footnote the footnote-counter
% (and |Hfootnote| counter in case of links) are increased by one.
%
%    \begin{macrocode}
  \addtocounter{footnote}{+1}%
  \ltx@ifpackageloaded{hyperref}{%
    \tfn@hyperfootnotes{\addtocounter{Hfootnote}{+1}}{}%
%    \end{macrocode}
%
% The \xpackage{hyperref} package was not expecting this kind of change of the
% |(H)footnote| numbers, therefore some code-lines are missing here which can be
% found at other places in that same package and are called by this command.
%
%    \begin{macrocode}
    \tfn@updateHyper%
%    \end{macrocode}
%
% When \xpackage{hyperref} is not loaded, nothing is to be done here.
%
%    \begin{macrocode}
   }{}%
%    \end{macrocode}
%
% The |\footnotetext[...]{...}| places the foot note text as usual.
%
%    \begin{macrocode}
  \ifx\tfn@swt\tfn@footnotezero%
    \ifthenelse{\isempty{#1}}%
      {\footnotetext{#2}}%
      {\ltx@ifpackageloaded{hyperref}%
         {\tfn@hyperfootnotes{%
             \footnotetext[#1]{%
               \raisebox{\ht\strutbox}{\hypertarget{tfn:#1}{}}#2}%
           }{\footnotetext[#1]{#2}%
           }%
         }{\footnotetext[#1]{#2}}%
      }%
  \else%
%    \end{macrocode}
%
% If there is more than one tablefootnote in the |sidewaystable|, \\
% the |\tfn@footmiscparawarn| shall be given,
%
%    \begin{macrocode}
    \ifx\tfn@footnotetableprint\tfn@footnotezero%
      \gdef\tfn@fmpw{\tfn@footmiscparawarn}%
    \else%
      \ltx@ifpackageloaded{footmisc}{\tfn@fmpw}{\relax}%
%    \end{macrocode}
%
% but we do not want to repeat the message for a third(, forth,\ldots)
% tablefootnote in the |sidewaystable|.
%
%    \begin{macrocode}
      \gdef\tfn@fmpw{\relax}%
    \fi%
%    \end{macrocode}
%
% |\@makefnmark| is \\
% |\hbox {\@textsuperscript {\normalfont \@thefnmark }}|, \\
% but according to |\show\@thefnmark| here |\@thefnmark| is
% |{\protect \itshape}| instead of |\thefootnote| (Why?!).\\
% If the \xpackage{footnotebackref} package is used,
% back references must be created for the table footnotes, too.
%
%    \begin{macrocode}
    \ltx@ifpackageloaded{footnotebackref}{%
      \LetLtxMacro{\tfncurrentmft}{\@makefntext}%
      \LetLtxMacro{\@makefntext}{\swtablemakefntext}%
      \ifthenelse{\isempty{#1}}%
        {\footnotetext{%
           \raisebox{\ht\strutbox}{\hypertarget{tfn:\theHfootnote}{}}%
           \hbox{\@textsuperscript{\normalfont%
              \hyperlink{tfn:m\theHfootnote}{\thefootnote}}}\thinspace%
         \ifx\FootnoteBackref@symbol\empty%
         \else\hyperlink{tfn:m\theHfootnote}{\FootnoteBackref@symbol}%
         \fi%
         #2}%
       }{\edef\tfn@tfn{\arabic{footnote}}%
         \edef\tfn@thn{\arabic{Hfootnote}}%
         \setcounter{footnote}{#1}%
         \setcounter{Hfootnote}{#1}%
         \tfn@updateHyper%
         \footnotetext{%
           \raisebox{\ht\strutbox}{\hypertarget{tfn:#1}{}}%
           \hbox{\@textsuperscript{\normalfont%
                  \hyperlink{tfn:m#1}{#1}}}%
           \thinspace%
           \ifx\FootnoteBackref@symbol\empty%
           \else%
             \raisebox{\ht\strutbox}{%
               \hyperlink{tfn:m#1}{\FootnoteBackref@symbol}}%
           \fi%
           #2%
          }%
         \setcounter{footnote}{\tfn@tfn}%
         \setcounter{Hfootnote}{\tfn@thn}%
         \tfn@updateHyper%
       }%
      \LetLtxMacro{\@makefntext}{\tfncurrentmft}%
     }{% footnotebackref not loaded
       \ifthenelse{\isempty{#1}}%
         {\footnotetext{%
            \hbox{\@textsuperscript{\normalfont\thefootnote}}%
            \ltx@ifpackageloaded{hyperref}%
              {\tfn@hyperfootnotes{%
                 \raisebox{\ht\strutbox}{%
                   \hypertarget{tfn:\theHfootnote}{}}}{}%
              }{}%
            \thinspace%
            #2}%
        }{%
%    \end{macrocode}
%
% \pagebreak
%
%    \begin{macrocode}
          \edef\tfn@tfn{\arabic{footnote}}%
          \ltx@ifpackageloaded{hyperref}{%
            \tfn@hyperfootnotes{\xdef\tfn@thn{\arabic{Hfootnote}}}{}%
           }{}%
          \setcounter{footnote}{#1}
          \ltx@ifpackageloaded{hyperref}{%
            \tfn@hyperfootnotes{\setcounter{Hfootnote}{#1}}{}%
            \tfn@updateHyper%
           }{}%
          \footnotetext{%
            \hbox{\@textsuperscript{\normalfont #1}}%
            \ltx@ifpackageloaded{hyperref}{%
              \tfn@hyperfootnotes{\raisebox{\ht\strutbox}{%
                                         \hypertarget{tfn:#1}{}}}{}%
             }{}%
            \thinspace%
            #2}%
          \setcounter{footnote}{\tfn@tfn}%
          \ltx@ifpackageloaded{hyperref}{%
            \tfn@hyperfootnotes{\setcounter{Hfootnote}{\tfn@thn}}{}%
            \tfn@updateHyper%
           }{}%
        }%
     }%
  \fi%
%    \end{macrocode}
%
% For further footnotes in the table the \\
% |\ifx\tfn@footnotetableprint\tfn@footnotezero|\ldots|\fi| \\
% from the beginning of this |\tfn@footnotetext| macro must not be repeated,
% therefore we deactivate this by defining |\tfn@footnotetableprint| to
% something other than |0| (here |1| is used).
%
%    \begin{macrocode}
  \gdef\tfn@footnotetableprint{1}%
%    \end{macrocode}
%
% |\tfn@footnotetablecount| contains the negative number of footnote texts
% to be printed, therefore it is increase by one (i.\,e.~the negative number
% decreases, i.\,e.~$-3$ becomes~$-2$).
%
%    \begin{macrocode}
  \@tempcnta=\tfn@footnotetablecount\relax%
  \advance\@tempcnta by +1\relax%
  \xdef\tfn@footnotetablecount{\the\@tempcnta}%
%    \end{macrocode}
%
% When it is zero, all footnote texts have been placed, and\\
% |\tfn@footnotetableprint| can be reset to zero and\\
% |\tfn@tablefootnoteprintout| can be cleared.
%
%    \begin{macrocode}
  \ifx\tfn@footnotetablecount\tfn@footnotezero% all footnotes placed
    \gdef\tfn@footnotetableprint{0}%
    \global\let\tfn@tablefootnoteprintout\relax% brute force
 %\else there is at least one \footnotemark which needs \footnotetext
  \fi
}

%    \end{macrocode}
% \end{macro}
%
% \begin{macro}{\tfn@tablefootnoteprintout}
% When something shall be cleared or redefined, it is a good idea
% to define it. It~will be later filled with the foot note text to be printed.
%
%    \begin{macrocode}
\newcommand{\tfn@tablefootnoteprintout}{\relax}

%    \end{macrocode}
% \end{macro}
%
% \pagebreak
%
% \begin{macro}{\tablefootnote}
% |\tablefootnote[...]{...}| is the command to be used by the user instead of
% |\footnote[...]{...}| in the (sideways)table. It places a foot note mark via
% |\tfn@footnotemarks| and appends a |\tfn@footnotetext{#1}| to the\\
% |\tfn@tablefootnoteprintout| macro, and remembers that table foot notes
% have been used.
%
%    \begin{macrocode}
\newcommand{\tablefootnote}[2][]{%
  \tfn@footnotemarks[#1]%
  \ifthenelse{\isempty{#1}}{\relax}{%
    \ltx@GlobalPrependToMacro{\tfn@tablefootnoteprintout}{%
      \addtocounter{footnote}{+1}%
      \ltx@ifpackageloaded{hyperref}{%
        \tfn@hyperfootnotes{\addtocounter{Hfootnote}{+1}}{}%
        \tfn@updateHyper%
       }{}%
     }%
   }%
  \ltx@GlobalAppendToMacro{\tfn@tablefootnoteprintout}{%
    \tfn@footnotetext[#1]{#2}%
    \ifthenelse{\isempty{#1}}{\relax}{%
      \addtocounter{footnote}{-1}%
      \ltx@ifpackageloaded{hyperref}{%
        \tfn@hyperfootnotes{\addtocounter{Hfootnote}{-1}}{}%
        \tfn@updateHyper%
       }{}%
     }%
   }%
  \gdef\tfn@fnt{1}%
  }

%    \end{macrocode}
% \end{macro}
%
% \begin{macro}{\tfn@swtbox}
% For |sidewaystable|s the |\tfn@tablefootnoteprintout| is placed in a \\
% |\parbox|, which can be placed at the end of the |sidewaystable|:
%
%    \begin{macrocode}
\newcommand{\tfn@swtbox}{%
\ifvoid \footins%
\vskip 2\baselineskip%
\else%
\vskip \skip\footins%
\fi%
\parbox[b][][t]{\linewidth}{\raggedright\tfn@tablefootnoteprintout}%
}

%    \end{macrocode}
% \end{macro}
%
% |\tfn@tablefootnoteprintout| shall be executed after the end of the table environment.
% The \xpackage{etoolbox} package provides very easy syntax for this:
%
%    \begin{macrocode}
\IfFileExists{etoolbox.sty}{%
  \RequirePackage{etoolbox}[2011/01/03]%   v2.1
  \AfterEndEnvironment{table}{%
    \ltx@ifpackageloaded{footnotebackref}{%
      \LetLtxMacro{\tfncurrentmft}{\@makefntext}%
      \LetLtxMacro{\@makefntext}{\tablemakefntext}%
       \tfn@tablefootnoteprintout%
      \LetLtxMacro{\@makefntext}{\tfncurrentmft}%
     }{% footnotebackref not loaded
      \tfn@tablefootnoteprintout%
     }%
      \gdef\tfn@fnt{0}%
   }%
%    \end{macrocode}
%
% When the \xpackage{rotating} package has been loaded, we need to detect
% whether the |\tablefootnote| is placed inside of a |sidewaystable|.
%
%    \begin{macrocode}
  \ltx@ifpackageloaded{rotating}{%
    \AtBeginEnvironment{sidewaystable}{%
      \gdef\tfn@swt{1}%
    }
%    \end{macrocode}
%
% At the end of |sidewaystable| environments |\tfn@swtbox| shall be placed
% (if~there was any |\tablefootnote| in that |sidewaystable| at all).
% And |\tfn@swt| as well as |\tfn@fnt| must be reset to |0|.
%
%    \begin{macrocode}
    \AtEndEnvironment{sidewaystable}{%
      \ifx\tfn@fnt\tfn@footnotezero% \relax
      \else%
        \ltx@ifpackageloaded{footnotebackref}{%
          \LetLtxMacro{\tfncurrentmft}{\@makefntext}%
          \LetLtxMacro{\@makefntext}{\tablemakefntext}%
           \tfn@swtbox%
          \LetLtxMacro{\@makefntext}{\tfncurrentmft}%
         }{% footnotebackref not loaded
          \tfn@swtbox%
          }%
      \fi%
      \gdef\tfn@swt{0}%
      \gdef\tfn@fnt{0}%
     }%
   }{}% else \relax
}{%
%    \end{macrocode}
%
% When the \xpackage{etoolbox} package is not available (e.\,g. because of missing
% \eTeX{}) we~do a workaround:
%
%    \begin{macrocode}
  \PackageWarning{tablefootnote}{%
    Package etoolbox not available, therefore\MessageBreak%
    substituting commands \string\AtEndEnvironment\space %
    and\MessageBreak%
    \string\AfterEndEnvironment\space for tablefootnotes%
    \MessageBreak%
   }%
  \@ifpackagelater{letltxmacro}{2010/09/02}%
   {% >= 2010/09/02, OK
    \GlobalLetLtxMacro{\tfnorigtable}{\table}
    \GlobalLetLtxMacro{\tfnendorigtable}{\endtable}
%    \end{macrocode}
%
% |\GlobalLetLtxMacro| was introduced in version 2010/09/02 v1.4 of the
% \xpackage{letltxmacro} package.
%
%    \begin{macrocode}
   }{% else
     \PackageError{tablefootnote}{%
       Outdated version of letltxmacro package used%
      }{Version 2010/09/02 v1.4 or newer needed but not found!%
        \MessageBreak%
        Update the used letltxmacro package.}
        \LetLtxMacro{\tfnorigtable}{\table}
        \LetLtxMacro{\tfnendorigtable}{\endtable}
    }%
%    \end{macrocode}
%
% \pagebreak
%
%    \begin{macrocode}
  \ltx@ifpackageloaded{float}{%
    \@ifpackagelater{float}{2001/11/08}%   v1.3d
     {% >= 2001/11/08, OK
     }{% else
       \PackageWarningNoLine{tablefootnote}{%
         It is required version\MessageBreak
         2001/11/08 v1.3d (or later) of package float,\MessageBreak
         but only version\MessageBreak
        `\csname ver@float.sty\endcsname'\MessageBreak
        is available}%
     }%
   }{\relax}
  \renewenvironment{table}{%
    \ltx@ifpackageloaded{float}{%
%    \end{macrocode}
%
% When \xpackage{etoolbox.sty} was not found, and the \xpackage{float} package
% has been loaded, and the float specifier is \texttt{H}, then |\endtable| was
% let to |\float@endH| by the \xpackage{float} package (line~79 of
% \xpackage{float.sty}, 2001/11/08 v1.3d:\\
% |\expandafter\let\csname end#1\endcsname\float@endH|), i.\,e.~appending\\
% |\tfn@tablefootnoteprintout \gdef\tfn@fnt{0}| to |\endtable| is useless.
% Instead of this, it must be appended to |\float@endH|.\\
% (Why is this not necessary when \xpackage{etoolbox} is used?
% That is a good question, answers are welcome.)
%
%    \begin{macrocode}
      \ltx@LocalAppendToMacro{\float@endH}{%
        \ltx@ifpackageloaded{footnotebackref}{%
          \LetLtxMacro{\tfncurrentmft}{\@makefntext}%
          \LetLtxMacro{\@makefntext}{\tablemakefntext}%
           \tfn@tablefootnoteprintout%
          \LetLtxMacro{\@makefntext}{\tfncurrentmft}%
         }{% footnotebackref not loaded
          \tfn@tablefootnoteprintout%
         }%
          \gdef\tfn@fnt{0}%
       }%
     }{\relax}
    \tfnorigtable%
   }{%
     \tfnendorigtable%
     \ltx@ifpackageloaded{footnotebackref}{%
       \LetLtxMacro{\tfncurrentmft}{\@makefntext}%
       \LetLtxMacro{\@makefntext}{\tablemakefntext}%
        \tfn@tablefootnoteprintout%
       \LetLtxMacro{\@makefntext}{\tfncurrentmft}%
      }{% footnotebackref not loaded
       \tfn@tablefootnoteprintout%
      }%
     \gdef\tfn@fnt{0}%
    }%
%    \end{macrocode}
%
% \pagebreak
%
% The |sidewaystable|-environment must be treated separately:
%
%    \begin{macrocode}
  \ltx@ifpackageloaded{rotating}{%
    \let\tfnorigsidewaystable\sidewaystable
    \let\tfnendorigsidewaystable\endsidewaystable
    \renewenvironment{sidewaystable}{%
      \gdef\tfn@swt{1}%
      \tfnorigsidewaystable%
    }{%
      \ifx\tfn@fnt\tfn@footnotezero% \relax
      \else%
        \ltx@ifpackageloaded{footnotebackref}{%
          \LetLtxMacro{\tfncurrentmft}{\@makefntext}%
          \LetLtxMacro{\@makefntext}{\tablemakefntext}%
           \tfn@swtbox%
          \LetLtxMacro{\@makefntext}{\tfncurrentmft}%
         }{% footnotebackref not loaded
          \tfn@swtbox%
         }%
      \fi%
      \gdef\tfn@swt{0}%
      \gdef\tfn@fnt{0}%
      \tfnendorigsidewaystable%
     }%
   }{}%
}

%    \end{macrocode}
%
% \DescribeMacro{\AtBeginDocument}
% When the \xpackage{rotating} (for |sidewaystable|s) or the
% \xpackage{footnotebackref} (for back references for the footnotes)
% package is used, it must be loaded before the \xpackage{tablefootnote} package.
% Using both packages is also possible, in which case both must be loaded before
% the \xpackage{tablefootnote} package. If any one of those packages is loaded
% |\AtBeginDocument| after this check, the according error message will not be
% given, but packages really should be loaded before |\AtBeginDocument|.
%
%    \begin{macrocode}
\AtBeginDocument{%
  \ltx@ifpackageloaded{footnotebackref}{%
    \ifx\tfn@fnbr\tfn@footnotezero%
      \PackageError{tablefootnote}{%
        Package footnotebackref loaded after tablefootnote}{%
        When using both the footnotebackref and the tablefootnote %
        package,\MessageBreak%
        the footnotebackref package MUST be loaded before the %
        tablefootnote package!\MessageBreak%
       }%
    \fi%
   }{% else \relax
   }
  \ltx@ifpackageloaded{rotating}{%
    \ifx\tfn@rotating\tfn@footnotezero%
      \PackageError{tablefootnote}{%
        Package rotating loaded after tablefootnote}{%
        When using both the rotating and the tablefootnote package,%
        \MessageBreak%
        the rotating package MUST be loaded before the tablefootnote%
        \space package!\MessageBreak%
       }%
    \fi%
   }{% else \relax
   }
%    \end{macrocode}
%
% \pagebreak
%
% When the document is compiled with \LuaLaTeX{}, hyperlinks in rotated content
% will be misplaced, regardless of the use of the \xpackage{tablefootnote} package
% (or of tables or footnotes). The text to be printed will be OK.
% This problem of \LuaLaTeX{} (not \xpackage{tablefootnote}) was found and reported
% by \textsc{Arno Trautmann}, 2011/11/15 -- thanks!
%
%    \begin{macrocode}
  \ltx@ifpackageloaded{hyperref}{%
    \tfn@hyperfootnotes{
      \ifluatex%
        \PackageWarningNoLine{tablefootnote}{%
          LuaTeX and hyperref package used:\MessageBreak%
          Hyperreferences in rotated content\MessageBreak%
          will be misplaced\MessageBreak%
          regardless of tablefootnote package.\MessageBreak%
          The text to be printed will be OK%
         }%
      \fi%
     }{% else \relax
     }%
   }{% else \relax
   }%
%    \end{macrocode}
%
% When the \xpackage{footnotebackref} package is used but the footnotes are not
% hyperlinked, neither the back references nor the tablefootnotes will work.
%
%    \begin{macrocode}
  \ltx@ifpackageloaded{footnotebackref}{%
     \ltx@ifpackageloaded{hyperref}{%
        \tfn@hyperfootnotes{\relax}{% else
          \PackageError{tablefootnote}{%
             footnotebackref but not hyperfootnotes}{%
             The footnotebackref package is used\MessageBreak%
             but the hyperfootnotes option of the hyperref package%
             \MessageBreak%
             was set to false.\MessageBreak%
             Either enable hyperfootnotes or do not use footnotebackref.%
             \MessageBreak%
             Loading of the tablefootnote package will be aborted.%
           }%
          \endinput%
      }%
    }{\PackageError{tablefootnote}{footnotebackref but not hyperref}{%
        The footnotebackref package is used\MessageBreak%
        but not the hyperref package.\MessageBreak%
        Either load hyperref or do not use footnotebackref.%
        \MessageBreak%
        Loading of the tablefootnote package will be aborted.%
        }%
      \endinput%
    }%
  }{}
}

%    \end{macrocode}
%
%    \begin{macrocode}
%</package>
%    \end{macrocode}
%
% \pagebreak
%
% \section{Installation}
%
% \subsection{Downloads\label{ss:Downloads}}
%
% Everything is available on \CTAN{}, \url{http://www.ctan.org/},
% but may need additional packages themselves.\\
%
% \DescribeMacro{tablefootnote.dtx}
% For unpacking the |tablefootnote.dtx| file and constructing the documentation
% it is required:
% \begin{description}
% \item[-] \TeX{} Format \LaTeXe{}: \url{http://www.CTAN.org/}
%
% \item[-] document class \xpackage{ltxdoc}, 2007/11/11, v2.0u,
%           \url{http://www.ctan.org/pkg/ltxdoc}
%
% \item[-] package \xpackage{holtxdoc}, 2012/03/21, v0.24,
%           \url{http://www.ctan.org/pkg/holtxdoc}
%
% \item[-] package \xpackage{hypdoc}, 2011/08/19, v1.11,
%           \url{http://www.ctan.org/pkg/hypdoc}
% \end{description}
%
% \DescribeMacro{tablefootnote.sty}
% The |tablefootnote.sty| for \LaTeXe{} (i.\,e.~each document using
% the \xpackage{tablefootnote} package) requires:
% \begin{description}
% \item[-] \TeX{} Format \LaTeXe{}, \url{http://www.CTAN.org/}
%
% \item[-] package \xpackage{letltxmacro}, 2010/09/02, v1.4,
%           \url{http://www.ctan.org/pkg/letltxmacro}.
%
% \item[-] package \xpackage{ltxcmds}, 2011/11/09, v1.22,
%           \url{http://www.ctan.org/pkg/ltxcmds}
% \end{description}
%
% When the \xpackage{hyperref} package is used, also the \xpackage{ifluatex} package
% is needed, but it is already loaded automatically by the \xpackage{hyperref}
% package.\\
% When the \xpackage{etoolbox} package is available, it is used:
% \begin{description}
% \item[-] package \xpackage{etoolbox}, 2011/01/03, v2.1,
%           \url{http://www.ctan.org/pkg/etoolbox}.
% \end{description}
%
% \DescribeMacro{tablefootnote-example.tex}
% The |tablefootnote-example.tex| requires the same files as all
% documents using the \xpackage{tablefootnote} package and additionally:
% \begin{description}
% \item[-] class \xpackage{article}, 2007/10/19, v1.4h, from \xpackage{classes.dtx}:\\
%           \CTAN{macros/latex/base/classes.dtx}
%
% \item[-] package \xpackage{rotating}, 2009/03/28, v2.16a,
%           \url{http://www.ctan.org/pkg/rotating}
%
% \item[-] package \xpackage{float}, 2001/11/08, v1.3d,
%           \url{http://www.ctan.org/pkg/float}
%
% \item[-] package \xpackage{placeins}, 2005/04/18, v2.2,
%           \url{http://www.ctan.org/pkg/placeins}\\
%           for |\FloatBarrier|
%
% \item[-] package \xpackage{hyperref}, 2012/11/06, v6.83m,
%           \url{http://www.ctan.org/pkg/hyperref}
%
% \item[-] package \xpackage{tablefootnote}, 2014/01/26, v1.1c,\\
%           \url{http://www.ctan.org/pkg/tablefootnote}\\
%   (Well, it is the example file for this package, and because you
%    are reading the documentation for the \xpackage{tablefootnote} package,
%    it can be assumed that you already have some version of it~--
%    is it the current one?)
% \end{description}
%
% \DescribeMacro{Oberdiek}
% \DescribeMacro{holtxdoc}
% \DescribeMacro{ifluatex}
% \DescribeMacro{letltxmacro}
% \DescribeMacro{ltxcmds}
% All packages of \textsc{Heiko Oberdiek}'s bundle `oberdiek'
% (especially \xpackage{holtxdoc}, \xpackage{ifluatex}, \xpackage{letltxmacro},
% and \xpackage{ltxcmds}) are also available in a TDS compliant ZIP archive:\\
% \CTAN{install/macros/latex/contrib/oberdiek.tds.zip}.\\
% It is probably best to download and use this, because the packages in there
% are quite probably both recent and compatible among themselves.\\
%
% \pagebreak
%
% \DescribeMacro{hyperref}
% \noindent \xpackage{hyperref} is not included in that bundle and needs to be
% downloaded separately,\\
% \url{http://www.ctan.org/install/macros/latex/contrib/hyperref.tds.zip}.\\
%
% \DescribeMacro{M\"{u}nch}
% A hyperlinked list of my (other) packages can be found at
% \url{http://www.ctan.org/author/muench-hm}.\\
%
% \subsection{Package, unpacking TDS}
% \paragraph{Package.} This package is available on \CTAN{}
% \begin{description}
% \item[\CTAN{macros/latex/contrib/tablefootnote/tablefootnote.dtx}]\hspace*{0.1cm} \\
%       The source file.
% \item[\CTAN{macros/latex/contrib/tablefootnote/tablefootnote.pdf}]\hspace*{0.1cm} \\
%       The documentation.
% \item[\CTAN{macros/latex/contrib/tablefootnote/tablefootnote-example.pdf}]\hspace*{0.1cm} \\
%       The compiled example file, as it should look like.
% \item[\CTAN{macros/latex/contrib/tablefootnote/README}]\hspace*{0.1cm} \\
%       The README file.
% \end{description}
% There is also a tablefootnote.tds.zip available:
% \begin{description}
% \item[\CTAN{install/macros/latex/contrib/tablefootnote.tds.zip}]\hspace*{0.1cm} \\
%       Everything in \xfile{TDS} compliant, compiled format.
% \end{description}
% which additionally contains\\
% \begin{tabular}{ll}
% tablefootnote.ins & The installation file.\\
% tablefootnote.drv & The driver to generate the documentation.\\
% tablefootnote.sty & The \xext{sty}le file.\\
% tablefootnote-example.tex & The example file.
% \end{tabular}
%
% \bigskip
%
% \noindent For required other packages, please see the preceding subsection.
%
% \paragraph{Unpacking.} The \xfile{.dtx} file is a self-extracting
% \docstrip{} archive. The files are extracted by running the
% \xext{dtx} through \plainTeX{}:
% \begin{quote}
%   \verb|tex tablefootnote.dtx|
% \end{quote}
%
% \noindent About generating the documentation see paragraph~\ref{GenDoc} below.\\
%
% \paragraph{TDS.} Now the different files must be moved into
% the different directories in your installation TDS tree
% (also known as \xfile{texmf} tree):
% \begin{quote}
% \def\t{^^A
% \begin{tabular}{@{}>{\ttfamily}l@{ $\rightarrow$ }>{\ttfamily}l@{}}
%   tablefootnote.sty & tex/latex/tablefootnote/tablefootnote.sty\\
%   tablefootnote.pdf & doc/latex/tablefootnote/tablefootnote.pdf\\
%   tablefootnote-example.tex & doc/latex/tablefootnote/tablefootnote-example.tex\\
%   tablefootnote-example.pdf & doc/latex/tablefootnote/tablefootnote-example.pdf\\
%   tablefootnote.dtx & source/latex/tablefootnote/tablefootnote.dtx\\
% \end{tabular}^^A
% }^^A
% \sbox0{\t}^^A
% \ifdim\wd0>\linewidth
%   \begingroup
%     \advance\linewidth by\leftmargin
%     \advance\linewidth by\rightmargin
%   \edef\x{\endgroup
%     \def\noexpand\lw{\the\linewidth}^^A
%   }\x
%   \def\lwbox{^^A
%     \leavevmode
%     \hbox to \linewidth{^^A
%       \kern-\leftmargin\relax
%       \hss
%       \usebox0
%       \hss
%       \kern-\rightmargin\relax
%     }^^A
%   }^^A
%   \ifdim\wd0>\lw
%     \sbox0{\small\t}^^A
%     \ifdim\wd0>\linewidth
%       \ifdim\wd0>\lw
%         \sbox0{\footnotesize\t}^^A
%         \ifdim\wd0>\linewidth
%           \ifdim\wd0>\lw
%             \sbox0{\scriptsize\t}^^A
%             \ifdim\wd0>\linewidth
%               \ifdim\wd0>\lw
%                 \sbox0{\tiny\t}^^A
%                 \ifdim\wd0>\linewidth
%                   \lwbox
%                 \else
%                   \usebox0
%                 \fi
%               \else
%                 \lwbox
%               \fi
%             \else
%               \usebox0
%             \fi
%           \else
%             \lwbox
%           \fi
%         \else
%           \usebox0
%         \fi
%       \else
%         \lwbox
%       \fi
%     \else
%       \usebox0
%     \fi
%   \else
%     \lwbox
%   \fi
% \else
%   \usebox0
% \fi
% \end{quote}
% If you have a \xfile{docstrip.cfg} that configures and enables \docstrip{}'s
% \xfile{TDS} installing feature, then some files can already be in the right
% place, see the documentation of \docstrip{}.
%
% \subsection{Refresh file name databases}
%
% If your \TeX{}~distribution (\TeX live, \mikTeX{}, \dots{}) relies on file name
% databases, you must refresh these. For example, \teTeX{} users run
% \verb|texhash| or \verb|mktexlsr|.
%
% \subsection{Some details for the interested}
%
% \paragraph{Unpacking with \LaTeX{}.}
% The \xfile{.dtx} chooses its action depending on the format:
% \begin{description}
% \item[\plainTeX:] Run \docstrip{} and extract the files.
% \item[\LaTeX:] Generate the documentation.
% \end{description}
% If you insist on using \LaTeX{} for \docstrip{} (really,
% \docstrip{} does not need \LaTeX{}), then inform the autodetect routine
% about your intention:
% \begin{quote}
%   \verb|latex \let\install=y% \iffalse meta-comment
%
% File: tablefootnote.dtx
% Version: 2014/01/26 v1.1c
%
% Copyright (C) 2011 - 2014 by
%    H.-Martin M"unch <Martin dot Muench at Uni-Bonn dot de>
%
% This work may be distributed and/or modified under the
% conditions of the LaTeX Project Public License, either
% version 1.3c of this license or (at your option) any later
% version. This version of this license is in
%    http://www.latex-project.org/lppl/lppl-1-3c.txt
% and the latest version of this license is in
%    http://www.latex-project.org/lppl.txt
% and version 1.3c or later is part of all distributions of
% LaTeX version 2005/12/01 or later.
%
% This work has the LPPL maintenance status "maintained".
%
% The Current Maintainer of this work is H.-Martin Muench.
%
% This work consists of the main source file tablefootnote.dtx,
% the README, and the derived files
%    tablefootnote.sty, tablefootnote.pdf,
%    tablefootnote.ins, tablefootnote.drv,
%    tablefootnote-example.tex, tablefootnote-example.pdf.
%
% Distribution:
%    http://mirror.ctan.org/macros/latex/contrib/tablefootnote/tablefootnote.dtx
%    http://mirror.ctan.org/macros/latex/contrib/tablefootnote/tablefootnote.pdf
%    http://mirror.ctan.org/install/macros/latex/contrib/tablefootnote.tds.zip
%
% See http://www.ctan.org/pkg/tablefootnote (Catalogue)
%
% Unpacking:
%    (a) If tablefootnote.ins is present:
%           tex tablefootnote.ins
%    (b) Without tablefootnote.ins:
%           tex tablefootnote.dtx
%    (c) If you insist on using LaTeX
%           latex \let\install=y% \iffalse meta-comment
%
% File: tablefootnote.dtx
% Version: 2014/01/26 v1.1c
%
% Copyright (C) 2011 - 2014 by
%    H.-Martin M"unch <Martin dot Muench at Uni-Bonn dot de>
%
% This work may be distributed and/or modified under the
% conditions of the LaTeX Project Public License, either
% version 1.3c of this license or (at your option) any later
% version. This version of this license is in
%    http://www.latex-project.org/lppl/lppl-1-3c.txt
% and the latest version of this license is in
%    http://www.latex-project.org/lppl.txt
% and version 1.3c or later is part of all distributions of
% LaTeX version 2005/12/01 or later.
%
% This work has the LPPL maintenance status "maintained".
%
% The Current Maintainer of this work is H.-Martin Muench.
%
% This work consists of the main source file tablefootnote.dtx,
% the README, and the derived files
%    tablefootnote.sty, tablefootnote.pdf,
%    tablefootnote.ins, tablefootnote.drv,
%    tablefootnote-example.tex, tablefootnote-example.pdf.
%
% Distribution:
%    http://mirror.ctan.org/macros/latex/contrib/tablefootnote/tablefootnote.dtx
%    http://mirror.ctan.org/macros/latex/contrib/tablefootnote/tablefootnote.pdf
%    http://mirror.ctan.org/install/macros/latex/contrib/tablefootnote.tds.zip
%
% See http://www.ctan.org/pkg/tablefootnote (Catalogue)
%
% Unpacking:
%    (a) If tablefootnote.ins is present:
%           tex tablefootnote.ins
%    (b) Without tablefootnote.ins:
%           tex tablefootnote.dtx
%    (c) If you insist on using LaTeX
%           latex \let\install=y\input{tablefootnote.dtx}
%        (quote the arguments according to the demands of your shell)
%
% Documentation:
%    (a) If tablefootnote.drv is present:
%           (pdf)latex tablefootnote.drv
%           makeindex -s gind.ist tablefootnote.idx
%           (pdf)latex tablefootnote.drv
%           makeindex -s gind.ist tablefootnote.idx
%           (pdf)latex tablefootnote.drv
%    (b) Without tablefootnote.drv:
%           (pdf)latex tablefootnote.dtx
%           makeindex -s gind.ist tablefootnote.idx
%           (pdf)latex tablefootnote.dtx
%           makeindex -s gind.ist tablefootnote.idx
%           (pdf)latex tablefootnote.dtx
%
%    The class ltxdoc loads the configuration file ltxdoc.cfg
%    if available. Here you can specify further options, e.g.
%    use DIN A4 as paper format:
%       \PassOptionsToClass{a4paper}{article}
%
% Installation:
%    TDS:tex/latex/tablefootnote/tablefootnote.sty
%    TDS:doc/latex/tablefootnote/tablefootnote.pdf
%    TDS:doc/latex/tablefootnote/tablefootnote-example.tex
%    TDS:doc/latex/tablefootnote/tablefootnote-example.pdf
%    TDS:source/latex/tablefootnote/tablefootnote.dtx
%
%<*ignore>
\begingroup
  \catcode123=1 %
  \catcode125=2 %
  \def\x{LaTeX2e}%
\expandafter\endgroup
\ifcase 0\ifx\install y1\fi\expandafter
         \ifx\csname processbatchFile\endcsname\relax\else1\fi
         \ifx\fmtname\x\else 1\fi\relax
\else\csname fi\endcsname
%</ignore>
%<*install>
\input docstrip.tex
\Msg{******************************************************************}
\Msg{* Installation                                                   *}
\Msg{* Package: tablefootnote 2014/01/26 v1.1c Table foot notes (HMM) *}
\Msg{******************************************************************}

\keepsilent
\askforoverwritefalse

\let\MetaPrefix\relax
\preamble

This is a generated file.

Project: tablefootnote
Version: 2014/01/26 v1.1c

Copyright (C) 2011 - 2014 by
    H.-Martin M"unch <Martin dot Muench at Uni-Bonn dot de>

The usual disclaimer applies:
If it doesn't work right that's your problem.
(Nevertheless, send an e-mail to the maintainer
 when you find an error in this package.)

This work may be distributed and/or modified under the
conditions of the LaTeX Project Public License, either
version 1.3c of this license or (at your option) any later
version. This version of this license is in
   http://www.latex-project.org/lppl/lppl-1-3c.txt
and the latest version of this license is in
   http://www.latex-project.org/lppl.txt
and version 1.3c or later is part of all distributions of
LaTeX version 2005/12/01 or later.

This work has the LPPL maintenance status "maintained".

The Current Maintainer of this work is H.-Martin Muench.

This work consists of the main source file tablefootnote.dtx,
the README, and the derived files
   tablefootnote.sty, tablefootnote.pdf,
   tablefootnote.ins, tablefootnote.drv,
   tablefootnote-example.tex, tablefootnote-example.pdf.

In memoriam Tommy Muench + 2014/01/02.

\endpreamble
\let\MetaPrefix\DoubleperCent

\generate{%
  \file{tablefootnote.ins}{\from{tablefootnote.dtx}{install}}%
  \file{tablefootnote.drv}{\from{tablefootnote.dtx}{driver}}%
  \usedir{tex/latex/tablefootnote}%
  \file{tablefootnote.sty}{\from{tablefootnote.dtx}{package}}%
  \usedir{doc/latex/tablefootnote}%
  \file{tablefootnote-example.tex}{\from{tablefootnote.dtx}{example}}%
}

\catcode32=13\relax% active space
\let =\space%
\Msg{************************************************************************}
\Msg{*}
\Msg{* To finish the installation you have to move the following}
\Msg{* file into a directory searched by TeX:}
\Msg{*}
\Msg{*     tablefootnote.sty}
\Msg{*}
\Msg{* To produce the documentation run the file `tablefootnote.drv'}
\Msg{* through (pdf)LaTeX, e.g.}
\Msg{*  pdflatex tablefootnote.drv}
\Msg{*  makeindex -s gind.ist tablefootnote.idx}
\Msg{*  pdflatex tablefootnote.drv}
\Msg{*  makeindex -s gind.ist tablefootnote.idx}
\Msg{*  pdflatex tablefootnote.drv}
\Msg{*}
\Msg{* At least three runs are necessary e.g. to get the}
\Msg{*  references right!}
\Msg{*}
\Msg{* Happy TeXing!}
\Msg{*}
\Msg{************************************************************************}

\endbatchfile
%</install>
%<*ignore>
\fi
%</ignore>
%
% \section{The documentation driver file}
%
% The next bit of code contains the documentation driver file for
% \TeX{}, i.\,e., the file that will produce the documentation you
% are currently reading. It will be extracted from this file by the
% \texttt{docstrip} programme. That is, run \LaTeX{} on \texttt{docstrip}
% and specify the \texttt{driver} option when \texttt{docstrip}
% asks for options.
%
%    \begin{macrocode}
%<*driver>
\NeedsTeXFormat{LaTeX2e}[2011/06/27]
\ProvidesFile{tablefootnote.drv}%
  [2014/01/26 v1.1c Table foot notes (HMM)]
\documentclass{ltxdoc}[2007/11/11]% v2.0u
\usepackage{holtxdoc}[2012/03/21]%  v0.24
%% tablefootnote may work with earlier versions of LaTeX2e and those
%% class and package, but this was not tested.
%% Please consider updating your LaTeX, class, and package
%% to the most recent version (if they are not already the most
%% recent version).
\hypersetup{%
 pdfsubject={Footnotes in Tables (HMM)},%
 pdfkeywords={LaTeX, tablefootnote, footnote, table, H.-Martin Muench},%
 pdfencoding=auto,%
 pdflang={en},%
 breaklinks=true,%
 linktoc=all,%
 pdfstartview=FitH,%
 pdfpagelayout=OneColumn,%
 bookmarksnumbered=true,%
 bookmarksopen=true,%
 bookmarksopenlevel=3,%
 pdfmenubar=true,%
 pdftoolbar=true,%
 pdfwindowui=true,%
 pdfnewwindow=true%
}
\CodelineIndex
\hyphenation{printing docu-ment}
\gdef\unit#1{\mathord{\thinspace\mathrm{#1}}}%
\makeatletter
\@ifundefined{eTeX}{\gdef\eTeX{$\m@th \varepsilon $-\TeX }}{% else \relax
}
\makeatother
\begin{document}
  \DocInput{tablefootnote.dtx}%
\end{document}
%</driver>
%    \end{macrocode}
%
% \fi
%
% \CheckSum{584}
%
% \CharacterTable
%  {Upper-case    \A\B\C\D\E\F\G\H\I\J\K\L\M\N\O\P\Q\R\S\T\U\V\W\X\Y\Z
%   Lower-case    \a\b\c\d\e\f\g\h\i\j\k\l\m\n\o\p\q\r\s\t\u\v\w\x\y\z
%   Digits        \0\1\2\3\4\5\6\7\8\9
%   Exclamation   \!     Double quote  \"     Hash (number) \#
%   Dollar        \$     Percent       \%     Ampersand     \&
%   Acute accent  \'     Left paren    \(     Right paren   \)
%   Asterisk      \*     Plus          \+     Comma         \,
%   Minus         \-     Point         \.     Solidus       \/
%   Colon         \:     Semicolon     \;     Less than     \<
%   Equals        \=     Greater than  \>     Question mark \?
%   Commercial at \@     Left bracket  \[     Backslash     \\
%   Right bracket \]     Circumflex    \^     Underscore    \_
%   Grave accent  \`     Left brace    \{     Vertical bar  \|
%   Right brace   \}     Tilde         \~}
%
% \GetFileInfo{tablefootnote.drv}
%
% \begingroup
%   \def\x{\#,\$,\^,\_,\~,\ ,\&,\{,\},\%}%
%   \makeatletter
%   \@onelevel@sanitize\x
% \expandafter\endgroup
% \expandafter\DoNotIndex\expandafter{\x}
% \expandafter\DoNotIndex\expandafter{\string\ }
% \begingroup
%   \makeatletter
%     \lccode`9=32\relax
%     \lowercase{%^^A
%       \edef\x{\noexpand\DoNotIndex{\@backslashchar9}}%^^A
%     }%^^A
%   \expandafter\endgroup\x
%
% \DoNotIndex{\\,\,}
% \DoNotIndex{\documentclass,\usepackage,\ProvidesPackage,\begin,\end}
% \DoNotIndex{\MessageBreak,\message}
% \DoNotIndex{\NeedsTeXFormat,\DoNotIndex,\verb}
% \DoNotIndex{\def,\edef,\gdef,\global}
% \DoNotIndex{\ifx,\listfiles,\mathord,\mathrm}
% \DoNotIndex{\kvoptions,\SetupKeyvalOptions,\ProcessKeyvalOptions}
% \DoNotIndex{\bigskip,\space,\thinspace,\ldots}
% \DoNotIndex{\indent,\noindent,\newline,\linebreak,\pagebreak}
% \DoNotIndex{\textbf,\textit,\textsf,\texttt,\textsc,\textquotedblleft,\textquotedblright}
% \DoNotIndex{\plainTeX,\TeX,\LaTeX,\pdfLaTeX}
% \DoNotIndex{\chapter,\section,\the,\csname,\endscname}
% \DoNotIndex{\value,\arabic,\hline,\text,\caption,\centering,\clear,\url}
%
% \title{The \xpackage{tablefootnote} package}
% \date{2014/01/26 v1.1c}
% \author{H.-Martin M\"{u}nch\\\xemail{Martin.Muench at Uni-Bonn.de}}
%
% \maketitle
%
% \begin{abstract}
% This \LaTeX{} package provides the command |\tablefootnote| to be used
% in a \texttt{table} or \texttt{sidewaystable} environment,
% where |\footnote| will not work and when using |\footnotemark| and
% |\footnotetext| and adjusting the counters (including \texttt{Hfootnote})
% manually is either too much work or would not even work (\texttt{sidewaystable}).
% \end{abstract}
%
% \bigskip
%
% \noindent Disclaimer for web links: The author is not responsible for any contents
% referred to in this work unless he has full knowledge of illegal contents.
% If any damage occurs by the use of information presented there, only the
% author of the respective pages might be liable, not the one who has referred
% to these pages.
%
% \bigskip
%
% \noindent {\color{green} Save per page about $200\unit{ml}$ water,
% $2\unit{g}$ CO$_{2}$ and $2\unit{g}$ wood:\\
% Therefore please print only if this is really necessary.}
%
% \pagebreak
%
% \tableofcontents
%
% \pagebreak
%
% \section{Introduction\label{sec:Introduction}}
%
% \indent This \LaTeX{} package provides the command |\tablefootnote{...}| to be
% used in a \texttt{table} or \texttt{sidewaystable} environment. In those
% environments |\footnote{...}| would not work and would need to be replaced
% by |\footnotemark| in the \texttt{(sideways)\allowbreak table} and
% |\footnotetext{...}| after the \texttt{(sideways)table} environment ended.
% Additionally the counters for \texttt{footnote} and
% (when the \xpackage{hyperref} package is used) \texttt{Hfootnote}
% would need to be adjusted manually. Then still the hyperlinks
% in a \texttt{(sideways)table} would not work.
% When the \xpackage{footnotebackref} package is used,
% |\footnotemark| and |\footnotetext{...}| would need to be redefined.
% Just using |\tablefootnote{...}| (with optional argument for custom foot note
% marks) in the (sideways)table does all this automatically.
% (Compatibility with the \xpackage{footnotebackref} package
% was achieved without redefining |\footnotemark| or |\footnotetext|.)
% Redefining |\footnote| was \emph{not} done as to not create conflicts with
% other packages.
%
% \section{Usage}
%
% \indent Just load the package placing
% \begin{quote}
%   |\usepackage{tablefootnote}|
% \end{quote}
% \noindent in the preamble of your \LaTeXe{} source file and use\newline
% |\tablefootnote{...}| instead of |\footnote{...}| and\newline
% |\tablefootnote[...]{...}| instead of |\footnote[...]{...}|
% in (sideways)\allowbreak tables.\\
% When the \xpackage{footnotebackref} package is used,
% it must be loaded before the \xpackage{tablefootnote} package
% and the \xpackage{hyperrefpackage} with option |hyperfootnotes=true|
% must be loaded, too. (Backreference{-}links without links do not make sense.)\\
% When the \xpackage{rotating} package (for |sidewaystable|s) is used,
% it must be loaded before the \xpackage{tablefootnote} package and it must be
% a recent version (v2.16a, 2009/03/28, or newer).\\
% If |\tablefootnote{...}| is used in a tabular environment, which is inside
% of a (sideways)table environment, everything is fine, but when the tabular
% environment is not inside of a (sideways)table, the \textbf{table}footnote
% will not work.\\
%
% Pages with tables with footnotes, which are rotated with the \xpackage{lscape}-
% or with the \xpackage{pdflscape}-package (|\begin{landscape}|, table, |\end{landscape}|),
% are possible. But when the \xpackage{footmisc}-package is used, \xpackage{(pdf)lscape}
% should be loaded before \xpackage{footmisc}.\\
%
% When the \xpackage{footmisc}-package is used with option \texttt{para},
% this is regarded with the exception of a \texttt{sidewaystable}. There the footnotes
% are printed one below the other. (Well, this is better than no footnote in a
% \texttt{sidewaystable} at all, isn't it?)\\
%
% When the (sideways)table floats, the footnotes are set and (when \xpackage{hyperref}
% is used) hyperlinked, but they are not automatically adapted when the table floats
% over/under another footnote. Thus either do not use a footnote between original and
% float position of the (sideways)table, or place the (sideways)table in
% \textquotedblleft here\textquotedblright\ position. |\clear(double)page|, |h(!)|,
% |H|~from the \xpackage{float} package (\url{http://www.ctan.org/pkg/float}),
% or |\FloatBarrier| from the \xpackage{picins} package
% (\url{http://www.ctan.org/pkg/picins}) might help, too.
% (Or after finishing the document, move the (sideways)table in the source code
% near the position where it floats to or use the optional footnote marks.)
%
% \pagebreak
%
% \section{Alternatives\label{sec:Alternatives}}
%
% \begin{description}
% \item[-] The \xpackage{longtable} package provides the \texttt{longtable} environment
%   as replacement for the combined \texttt{table} and \texttt{tabular} environments.
%   Footnotes are real footnotes (not just tablenotes), are continuously numbered and
%   hyperlinked (when using the \xpackage{hyperref} package), and the hyperlinks really
%   work. As drawback the appearance of the caption changes slightly (e.\,g.~distance
%   to the table, width of the caption), which can probably be changed back manually.
%   Furthermore, longtables are meaned to break over more than one page. If that is not
%   wished, it must be prevented by |\nopagebreak|-commands and by ending the longtable
%   lines with |\\*| instead of |\\|. longtables do not float. (Therefore using the
%   \xpackage{tablefootnote} package and |\FloatBarrier| from the \xpackage{picins} package
%   before and after the table environment is similar~- but \xpackage{tablefootnote} does not
%   change the table-caption!) \texttt{sidewaystable} does not work with it.\\
%   \url{http://www.ctan.org/pkg/longtable}
%
% \item[-] The \xpackage{supertabular} package provides the \texttt{mpsupertabular}
%   environment as replacement for the combined \texttt{table} and \texttt{tabular}
%   environments. Footnotes are just tablenotes (with working hyperlinks when using the
%   \xpackage{hyperref} package), i.\,e. numbered a, b, c and placed below the table
%   and not at the end of the page. Therefore there is no float problem (because the
%   tablenotes numbering is not included in the continuous numbering of the footnotes).
%   Placing the \texttt{supertabular} inside of a \texttt{sidewaystable} breaks the
%   hyperlinks to the tablenotes.\\
%   \url{http://www.ctan.org/pkg/supertabular}
%
% \item[-] The \xpackage{ctable} package has its very own notation for defining tables.
%   It can create tablenotes and sideways-tables. The tablenotes are not automatically
%   hyperlinked. The \texttt{ctable}s float. Because the tablenotes numbering is not
%   included in the continuous numbering of the footnotes there is no float problem.\\
%   \url{http://www.ctan.org/pkg/ctable}
%
% \item[-] The \xpackage{footnote} package provides |\makesavenoteenv{table}|. After
%   loading the package and using that command in the preamble, in tables |\footnote{...}|
%   can be used. Using |\makesavenoteenv{tabular}| and \\
%   |\makesavenoteenv{sidewaystable}|
%   is possible, but it neither solves the float problem, nor do the created hyperlinks
%   work (i.\,e.~they aim at wrong locations). The \xpackage{mdwtab} from the same bundle
%   is incompatible with other table-related packages (e.\,g. \xpackage{supertabular},
%   \xpackage{array}) and not $100\,\%$ compatible with the \texttt{tabular} environment.\\
%   \url{http://www.ctan.org/pkg/footnote}\\
%   \url{http://www.ctan.org/pkg/mdwtab}
%
% \item[-] The \xpackage{tabularx} package does produce footnotes for sidewaystables,
%   but uses a, b, c instead of 1, 2, 3. The hyperlinks to the footnotes do not work.
%   Because the footnotes numbering is not included in the continuous numbering of
%   the other footnotes there is no float problem.\\
%   \url{http://www.ctan.org/pkg/tabularx}
%
% \item[-] Placing a tabular inside a minipage inside a table produces tablenotes.
%   Therefore there is no float problem (because the footnotes are not continuously
%   numbered). The hyperlinks to the table notes indeed work.
%
% \pagebreak
%
% \item[-] The \xpackage{threeparttable} package creates tablenotes again. Therefore
%   there is no float problem (because the tablenotes are not continuously numbered with
%   the footnotes). There are no hyperlinks to the table notes (at least not automatically).
%   Using |sidewaystable| (with table notes) works.\\
%   \url{http://www.ctan.org/pkg/threeparttable}
%
% \item[-] The \xpackage{threeparttablex} package creates tablenotes again. Therefore there
%   is no float problem (because the tablenotes are not continuously numbered with the
%   footnotes). With option \texttt{referable} the tablenotes are hyperlinked. Use of
%   a |sidewaystable| (with table notes) did not work for me. When using the
%   \texttt{referable} option according to the example in the \xpackage{threeparttablex}
%   manual the \xpackage{longtable} package is used, therefore that package could be used
%   directly without \xpackage{threeparttablex} (see above).\\
%   \url{http://www.ctan.org/pkg/threeparttablex}
%
% \item[-] One can manually use |\footnotemark| in the table and |\footnotetext{...}|
%  after the table environment ended and manually change the |footnote| and
%  (when the \xpackage{hyperref} package is used) |Hfootnote| counters and needs
%  to insert
%  \begin{verbatim}
%     \makeatletter
%     \global\let\Hy@saved@currentHref\@currentHref
%     \hyper@makecurrent{Hfootnote}%
%     \global\let\Hy@footnote@currentHref\@currentHref
%     \global\let\@currentHref\Hy@saved@currentHref
%     \makeatother
%  \end{verbatim}
%  before each |\footnotetext{...}|, but with more than very few footnotes this can become
%  much work (and is prone to produce errors). And this does not even work with
%  sidewaystables. (And it does not work with the \xpackage{footnotebackref} package
%  without redefining |\footnotemark| and |\footnotetext{...}|.)
% \end{description}
%
% \bigskip
%
% \noindent (You programmed or found another alternative,
%  which is available at \CTAN{}?\\
%  OK, send an e-mail to me with the name, location at \CTAN{},
%  and a short notice, and I will probably include it in
%  the list above.)
%
% \newpage
%
% \section{Example}
%
%    \begin{macrocode}
%<*example>
\documentclass[british]{article}[2007/10/19]% v1.4h
%%%%%%%%%%%%%%%%%%%%%%%%%%%%%%%%%%%%%%%%%%%%%%%%%%%%%%%%%%%%%%%%%%%%%
\usepackage{float}[2001/11/08]%    v1.3d
\usepackage{placeins}[2005/04/18]% v2.2  ; for \FloatBarrier
\usepackage{rotating}[2009/03/28]% v2.16a; for sidewaystable-environment
\usepackage[%
 hyperfootnotes=true,%
 extension=pdf,%
 plainpages=false,%
 pdfpagelabels=true,%
 hyperindex=false,%
 pdflang={en},%
 pdftitle={tablefootnote package example},%
 pdfauthor={H.-Martin Muench},%
 pdfsubject={Example for the tablefootnote package},%
 pdfkeywords={LaTeX, tablefootnote, footnote, table, H.-Martin Muench},%
 % pdfview=FitH and FitBH do not work: hyperlinks in sidewaystables
 % do no lead to the footnotes, due to a bug in pdfTeX,
 % computing wrong anchor coordinates (Heiko Oberdiek, 29. October 2011)
 % pdfview=Fit, FitV, FitR, FitB, FitBV work
 % print is OK for all those options
 pdfstartview=FitH,%
 pdfpagelayout=OneColumn%
]{hyperref}[2012/11/06]% v6.83m
 % Due to the urls used in the example, either the hyperref or the url
 % package are needed (or the urls must be removed before compiling).

\usepackage{footnotebackref}[2012/07/01]% v1.0
\usepackage{tablefootnote}[2014/01/26]%   v1.1c

\gdef\unit#1{\mathord{\thinspace\mathrm{#1}}}%
\listfiles
\begin{document}
\pagenumbering{arabic}
\section*{Example for tablefootnote}

This example demonstrates the use of package\newline
\textsf{tablefootnote}, v1.1c as of 2014/01/26 (HMM).\newline
There were no options used. (The package provides no options.)\newline

\textbf{The \texttt{tablefootnote-example.tex} needs to be compiled
at least twice to get the references right!}\newline

If the etoolbox-package is found, it is automatically used.\newline

For more details please see the documentation!\newline

\noindent Save per page about $200\unit{ml}$ water,
$2\unit{g}$ CO$_{2}$ and $2\unit{g}$ wood:\newline
Therefore please print only if this is really necessary.\newline

Here is some text.\footnote{And this is a text footnote.}\newline

Tables \ref{tab.symbol}, \ref{tab.normal}, \ref{tab.another} and
\ref{tab.floatH} show normal tables, table~\ref{tab.sideways} depicts
a sidewaystable. Table~\ref{tab.floatH} uses the float
specifier~\texttt{H} from the float package.\newline

\texttt{Hyperref} option \verb|pdfview=FitH| and \verb|FitBH| do
not work due to a bug in pdf\TeX{}, computing wrong
anchor coordinates (\textsc{Heiko Oberdiek}, 29.~October 2011).
Depending on used pdf-viewer, hyperlinks in sidewaystables lead
e.\,g.~at the end of the document, not at the footnote.
\verb|pdfview=Fit|, \verb|FitV|, \verb|FitR|, \verb|FitB|,
\verb|FitBV| work, print is OK for all those options.

\bigskip

\listoftables

\pagebreak

\renewcommand{\thefootnote}{\fnsymbol{footnote}}
\verb|\renewcommand{\thefootnote}{\fnsymbol{footnote}}|
causes footnote{-}symbol{-}footnotes, which are possible
(see Table~\ref{tab.symbol}).

\begin{table}
\centering
\begin{tabular}{ccc}
Another\tablefootnote{A table footnote.} & %
text\tablefootnote{Another table footnote.} & %
in a table\tablefootnote{A third table footnote.}
\end{tabular}
\caption[A footnotesymbol table]{%
A table with footnote-symbol-footnotes.\label{tab.symbol}}
\end{table}

Some text.\footnote{A text footnote.}

\renewcommand{\thefootnote}{\arabic{footnote}}
\verb|\renewcommand{\thefootnote}{\arabic{footnote}}|
switches back to normal footnote numbers again.

\pagebreak

\begin{table}
\centering
\begin{tabular}{ccc}
Some\tablefootnote{A table footnote.} & %
text\tablefootnote[99]{A table footnote with custom footnote number.} & %
in a table\tablefootnote{A third table footnote.}
\end{tabular}
\caption[A table]{A normal table.\label{tab.normal}}
\end{table}

Some text.\footnote{Another text footnote.}

\pagebreak

%    \end{macrocode}
%
% \pagebreak
%
%    \begin{macrocode}
More text.\footnote{And yet another text footnote.}

\begin{table}[t]
\centering
\begin{tabular}{|c|c|c|}
\hline
Another\tablefootnote{A $3^{rd}$ table footnote.} & %
text\tablefootnote{Another $3^{rd}$ table footnote.} & %
in a table\tablefootnote{A $3^{rd}$ third table footnote.}\\ \hline
\end{tabular}
\caption[Another table]{Another table (third one)\label{tab.another}}
\end{table}

Please note that Table~\ref{tab.another} floated to the top of the page.
While the footnotes are set and (when hyperref is used) hyperlinked,
they are not automatically adapted. Thus either do not use a footnote
at the same page before the table, or place the table in
\textquotedblleft here\textquotedblright\ or
\textquotedblleft bottom\textquotedblright\ position.
\verb|\clear(double)page|, \verb|h(!)|, \verb|H|~from the
\texttt{float} package, or \verb|\FloatBarrier| from the
\texttt{picins} package might help, too. (Or move the table in the
source code near the position where it floats to
or use the optional footnote marks.)

Table~\ref{tab.floatH} (page~\pageref{tab.floatH}) uses float specifier
\texttt{H} from the float package and does not float.

Some text.\footnote{This is just another text footnote.}

\pagebreak

\FloatBarrier

\begin{sidewaystable}
\centering%
\begin{tabular}{ccc}
Text\tablefootnote{Please rotate the view for testing the %
hyperlinks.} & %
in a\tablefootnote[98]{Another sidewaystable footnote %
with optional footnote mark.} & %
sidewaystable%
\tablefootnote{Sidewaystable-environment provided by %
\url{http://www.ctan.org/pkg/rotating} package.}
\end{tabular}
\caption[A sideways table]{A table in the \texttt{sideways} %
environment\label{tab.sideways}}%
\end{sidewaystable}%

\FloatBarrier

\pagebreak

%    \end{macrocode}
%
% \pagebreak
%
%    \begin{macrocode}
A last table, here with float specifier \texttt{H} from the
float\footnote{\url{http://www.ctan.org/pkg/float}} package.

\begin{table}[H]
\centering
\begin{tabular}{ccc}
Another\tablefootnote{A $5^{th}$ table footnote.} & %
text\tablefootnote{Another $5^{th}$ table footnote.} & %
in a table\tablefootnote{A $5^{th}$ third table footnote.}
\end{tabular}
\caption[A last table]{A very last table\label{tab.floatH}}
\end{table}

Some text.\footnote{This is just another text footnote.}

\pagebreak

End of the example for the
tablefootnote\footnote{\url{http://www.ctan.org/pkg/tablefootnote}}
package.

\end{document}
%</example>
%    \end{macrocode}
%
% \StopEventually{}
%
% \newpage
%
% \section{The implementation}
%
% We start off by checking that we are loading into \LaTeXe{} and
% announcing the name and version of this package.
%
%    \begin{macrocode}
%<*package>
%    \end{macrocode}
%
%    \begin{macrocode}
\NeedsTeXFormat{LaTeX2e}[2011/06/27]
\ProvidesPackage{tablefootnote}[2014/01/26 v1.1c
              Table foot notes (HMM)]
%    \end{macrocode}
%
% A short description of the \xpackage{tablefootnote} package:
%
%    \begin{macrocode}
%% Provides the \tablefootnote[...]{...} command
%% for footnotes in (sideways)tables.

%    \end{macrocode}
%
% We need the \xpackage{ltxcmds} and \xpackage{letltxmacro} packages
% by \textsc{Heiko Oberdiek}:
%
%    \begin{macrocode}
\RequirePackage{ltxcmds}[2011/11/09]%     v1.22
\RequirePackage{letltxmacro}[2010/09/02]% v1.4
\RequirePackage{xifthen}[2009/04/17]%     v1.3
%    \end{macrocode}
%
% A last information for the user:
%
%    \begin{macrocode}
%% tablefootnote may work with earlier versions of LaTeX and those
%% packages, but this was not tested. Please consider updating
%% your LaTeX and packages to the most recent version
%% (if they are not already the most recent version).

%    \end{macrocode}
%
% \indent See subsection~\ref{ss:Downloads} about how to get them.\\
%
% When the \xpackage{rotating} package (for |sidewaystable|s) is used,
% it must be loaded before the \xpackage{tablefootnote} package and it must be
% a recent version:
%
%    \begin{macrocode}
\@ifpackageloaded{rotating}{%
  \gdef\tfn@rotating{1}
  \@ifpackagelater{rotating}{2009/03/28}% v2.16a
   {% >= 2009/03/28, OK
   }{% else
     \PackageWarningNoLine{tablefootnote}{%
       It is required version\MessageBreak%
       2009/03/28 v2.16a (or later) of package rotating,\MessageBreak%
       but only version\MessageBreak%
      `\csname ver@rotating.sty\endcsname'\MessageBreak%
      is available}%
   }%
 }{% else
   \gdef\tfn@rotating{0}%
 }

%    \end{macrocode}
%
% There are no options to be processed.\\
%
% We need some definitions:
%
%    \begin{macrocode}
\def\tfn@footnotetablecount{0}
\def\tfn@footnotetableprint{0}
\def\tfn@footnotezero{0}
\def\tfn@fnt{0}
\def\tfn@swt{0}%       _s_ide_w_ays_t_able
\def\tfn@fmpw{\relax}% f_oot_m_isc_p_ara_w_arn-switch
\def\tfn@fnbr{0}%      f_oot_n_ote_b_ack_r_ef(erence package)

%    \end{macrocode}
%
% \begin{macro}{\tfn@hyperfootnotes}
% We need a wrapper for the used |\ifHy@hyperfootnotes|,
% which is not defined when the \xpackage{hyperref} package has not been loaded:
%
%    \begin{macrocode}
\DeclareRobustCommand{\tfn@hyperfootnotes}[2]{%
  \ifHy@hyperfootnotes%
    #1%
  \else%
    #2%
  \fi%
  }

%    \end{macrocode}
% \end{macro}
%
% \begin{macro}{\tablemakefntext}
% |\tablemakefntext| makes the footnotetext for a footnote in a table.
%
%    \begin{macrocode}
\ltx@ifpackageloaded{footnotebackref}{%
  \gdef\tfn@fnbr{1}%
  \ifx\FootnoteBackref@symbol\empty%
    \ifFootnoteBackref@numberlinked%
      \newcommand\tablemakefntext[1]{%
        \noindent\makebox[1.634em][r]{%
         \mbox{\textsuperscript{\normalfont%
           \hyperlink{tfn:m\theHfootnote}{\@thefnmark}%
        }}}%
        \ #1%
      }
    \else
      \newcommand\tablemakefntext[1]{%
        \noindent\makebox[1.8em][r]{%
          \mbox{\textsuperscript{\normalfont%
            \@thefnmark%
        }}}%
        #1%
      }
    \fi
  \else
    \newcommand\tablemakefntext[1]{%
      \noindent\makebox[1.8em][r]{%
        \mbox{\textsuperscript{\normalfont%
          \ifFootnoteBackref@numberlinked%
            \hyperlink{tfn:m\theHfootnote}{\@thefnmark}\ %
          \else%
            \@thefnmark%
          \fi%
      }}}%
      \hyperlink{tfn:m\theHfootnote}{\FootnoteBackref@symbol}%
      #1%
    }
  \fi
}{%
  \newcommand\tablemakefntext[1]{%
    \noindent\makebox[1.8em][r]{%
      \mbox{\textsuperscript{\normalfont%
        \@thefnmark%
    }}}%
    #1%
  }
}

%    \end{macrocode}
%
% \end{macro}
%
% \pagebreak
%
% \begin{macro}{\swtablemakefntext}
%    \begin{macrocode}
\newcommand\swtablemakefntext[1]{\noindent\ #1}

%    \end{macrocode}
% \end{macro}
%
% \begin{macro}{\tfn@footnotemarks}
% |\tfn@footnotemarks| uses the |\footnotemark| command to place a foot note mark
% and decreases the |\tfn@footnotetablecount| (i.\,e.~increases the negative value)
% of the number of footnote(marks) in the table. If the \xpackage{footnotebackref}
% package is used, it is also necessary to create an according |\hypertarget| for
% the back reference.
%
%    \begin{macrocode}
\newcommand{\tfn@footnotemarks}[1][]{%
  \hbox{%
    \ifthenelse{\isempty{#1}}%
      {\footnotemark%
       \ltx@ifpackageloaded{footnotebackref}%
         {\raisebox{\ht\strutbox}{\hypertarget{tfn:m\theHfootnote}{}}}%
         {% footnotebackref not loaded
         }%
      }%
      {\ltx@ifpackageloaded{hyperref}%
        {\tfn@hyperfootnotes{%
            \hyperlink{tfn:#1}{\footnotemark[#1]}%
          }{\footnotemark[#1]}}%
        {\footnotemark[#1]}%
       \ltx@ifpackageloaded{footnotebackref}%
        {\raisebox{\ht\strutbox}{\hypertarget{tfn:m#1}{}}}{}%
      }%
   }%
  \@tempcnta=\tfn@footnotetablecount\relax%
  \advance\@tempcnta by -1\relax%
  \xdef\tfn@footnotetablecount{\the\@tempcnta}%
}

%    \end{macrocode}
% \end{macro}
%
% \begin{macro}{\tfn@footmiscparawarn}
% When the \xpackage{footmisc} package is used with option |para|,
% we need to issue a warning. (Any idea how to resolve this issue?)\\
% It must be given inside of it's own command, because |\ifFN@para| is not
% defined when \xpackage{footmisc} has not been loaded, and then the
% |\fi| would be a |\fi| without |\if| (and \TeX{} does not like single
% |\fi|s).
%
%    \begin{macrocode}
\newcommand{\tfn@footmiscparawarn}{%
  \ifFN@para%
    \PackageWarning{tablefootnote}{%
      Package footmisc with option para used.\MessageBreak%
      Unfortunately package tablefootnote is not yet\MessageBreak%
      able to regard this in sidewaystables.\MessageBreak%
      Here the footnotes are printed one below\MessageBreak%
      the other. Sorry!\MessageBreak%
     }%
  \fi%
  }

%    \end{macrocode}
% \end{macro}
%
% \begin{macro}{\tfn@updateHyper}
% |\tfn@updateHyper| updates some \xpackage{hyperref} internals.
%
%    \begin{macrocode}
\newcommand{\tfn@updateHyper}{%
  % The code of this command is partly from the
  % hyperref package by Heiko Oberdiek, 2011/10/01, v6.82j.
  % Newer versions of that package are available at CTAN.org.
  \global\let\Hy@saved@currentHref\@currentHref%
  \tfn@hyperfootnotes{\hyper@makecurrent{Hfootnote}}{}%
  \global\let\Hy@footnote@currentHref\@currentHref%
  \global\let\@currentHref\Hy@saved@currentHref%
  }

%    \end{macrocode}
% \end{macro}
%
% \begin{macro}{\tfn@footnotetext}
% |\tfn@footnotetext| contains the |\footnotetext[...]{...}| command and handles
% the numbering. |\tfn@footnotetext| will be executed after the table.
% The foot note counter has been increased automatically with each |\footnotemark|,
% therefore it must be reset to its value before the table began.
%
%    \begin{macrocode}
\newcommand{\tfn@footnotetext}[2][]{%
  \ifx\tfn@footnotetableprint\tfn@footnotezero%
    % This is the first footnotetext after the (sideways)table.
    \addtocounter{footnote}{\tfn@footnotetablecount}%
%    \end{macrocode}
%
% When the \xpackage{hyperref} package is used, also its |Hfootnote| counter must
% be reset (when the footnotes are hyperlinked).
%
%    \begin{macrocode}
    \ltx@ifpackageloaded{hyperref}{%
      \tfn@hyperfootnotes{%
        \addtocounter{Hfootnote}{\tfn@footnotetablecount}}{}%
      \tfn@updateHyper%
     }{}%
  \fi%
%    \end{macrocode}
%
% For issuing the text for the next footnote the footnote-counter
% (and |Hfootnote| counter in case of links) are increased by one.
%
%    \begin{macrocode}
  \addtocounter{footnote}{+1}%
  \ltx@ifpackageloaded{hyperref}{%
    \tfn@hyperfootnotes{\addtocounter{Hfootnote}{+1}}{}%
%    \end{macrocode}
%
% The \xpackage{hyperref} package was not expecting this kind of change of the
% |(H)footnote| numbers, therefore some code-lines are missing here which can be
% found at other places in that same package and are called by this command.
%
%    \begin{macrocode}
    \tfn@updateHyper%
%    \end{macrocode}
%
% When \xpackage{hyperref} is not loaded, nothing is to be done here.
%
%    \begin{macrocode}
   }{}%
%    \end{macrocode}
%
% The |\footnotetext[...]{...}| places the foot note text as usual.
%
%    \begin{macrocode}
  \ifx\tfn@swt\tfn@footnotezero%
    \ifthenelse{\isempty{#1}}%
      {\footnotetext{#2}}%
      {\ltx@ifpackageloaded{hyperref}%
         {\tfn@hyperfootnotes{%
             \footnotetext[#1]{%
               \raisebox{\ht\strutbox}{\hypertarget{tfn:#1}{}}#2}%
           }{\footnotetext[#1]{#2}%
           }%
         }{\footnotetext[#1]{#2}}%
      }%
  \else%
%    \end{macrocode}
%
% If there is more than one tablefootnote in the |sidewaystable|, \\
% the |\tfn@footmiscparawarn| shall be given,
%
%    \begin{macrocode}
    \ifx\tfn@footnotetableprint\tfn@footnotezero%
      \gdef\tfn@fmpw{\tfn@footmiscparawarn}%
    \else%
      \ltx@ifpackageloaded{footmisc}{\tfn@fmpw}{\relax}%
%    \end{macrocode}
%
% but we do not want to repeat the message for a third(, forth,\ldots)
% tablefootnote in the |sidewaystable|.
%
%    \begin{macrocode}
      \gdef\tfn@fmpw{\relax}%
    \fi%
%    \end{macrocode}
%
% |\@makefnmark| is \\
% |\hbox {\@textsuperscript {\normalfont \@thefnmark }}|, \\
% but according to |\show\@thefnmark| here |\@thefnmark| is
% |{\protect \itshape}| instead of |\thefootnote| (Why?!).\\
% If the \xpackage{footnotebackref} package is used,
% back references must be created for the table footnotes, too.
%
%    \begin{macrocode}
    \ltx@ifpackageloaded{footnotebackref}{%
      \LetLtxMacro{\tfncurrentmft}{\@makefntext}%
      \LetLtxMacro{\@makefntext}{\swtablemakefntext}%
      \ifthenelse{\isempty{#1}}%
        {\footnotetext{%
           \raisebox{\ht\strutbox}{\hypertarget{tfn:\theHfootnote}{}}%
           \hbox{\@textsuperscript{\normalfont%
              \hyperlink{tfn:m\theHfootnote}{\thefootnote}}}\thinspace%
         \ifx\FootnoteBackref@symbol\empty%
         \else\hyperlink{tfn:m\theHfootnote}{\FootnoteBackref@symbol}%
         \fi%
         #2}%
       }{\edef\tfn@tfn{\arabic{footnote}}%
         \edef\tfn@thn{\arabic{Hfootnote}}%
         \setcounter{footnote}{#1}%
         \setcounter{Hfootnote}{#1}%
         \tfn@updateHyper%
         \footnotetext{%
           \raisebox{\ht\strutbox}{\hypertarget{tfn:#1}{}}%
           \hbox{\@textsuperscript{\normalfont%
                  \hyperlink{tfn:m#1}{#1}}}%
           \thinspace%
           \ifx\FootnoteBackref@symbol\empty%
           \else%
             \raisebox{\ht\strutbox}{%
               \hyperlink{tfn:m#1}{\FootnoteBackref@symbol}}%
           \fi%
           #2%
          }%
         \setcounter{footnote}{\tfn@tfn}%
         \setcounter{Hfootnote}{\tfn@thn}%
         \tfn@updateHyper%
       }%
      \LetLtxMacro{\@makefntext}{\tfncurrentmft}%
     }{% footnotebackref not loaded
       \ifthenelse{\isempty{#1}}%
         {\footnotetext{%
            \hbox{\@textsuperscript{\normalfont\thefootnote}}%
            \ltx@ifpackageloaded{hyperref}%
              {\tfn@hyperfootnotes{%
                 \raisebox{\ht\strutbox}{%
                   \hypertarget{tfn:\theHfootnote}{}}}{}%
              }{}%
            \thinspace%
            #2}%
        }{%
%    \end{macrocode}
%
% \pagebreak
%
%    \begin{macrocode}
          \edef\tfn@tfn{\arabic{footnote}}%
          \ltx@ifpackageloaded{hyperref}{%
            \tfn@hyperfootnotes{\xdef\tfn@thn{\arabic{Hfootnote}}}{}%
           }{}%
          \setcounter{footnote}{#1}
          \ltx@ifpackageloaded{hyperref}{%
            \tfn@hyperfootnotes{\setcounter{Hfootnote}{#1}}{}%
            \tfn@updateHyper%
           }{}%
          \footnotetext{%
            \hbox{\@textsuperscript{\normalfont #1}}%
            \ltx@ifpackageloaded{hyperref}{%
              \tfn@hyperfootnotes{\raisebox{\ht\strutbox}{%
                                         \hypertarget{tfn:#1}{}}}{}%
             }{}%
            \thinspace%
            #2}%
          \setcounter{footnote}{\tfn@tfn}%
          \ltx@ifpackageloaded{hyperref}{%
            \tfn@hyperfootnotes{\setcounter{Hfootnote}{\tfn@thn}}{}%
            \tfn@updateHyper%
           }{}%
        }%
     }%
  \fi%
%    \end{macrocode}
%
% For further footnotes in the table the \\
% |\ifx\tfn@footnotetableprint\tfn@footnotezero|\ldots|\fi| \\
% from the beginning of this |\tfn@footnotetext| macro must not be repeated,
% therefore we deactivate this by defining |\tfn@footnotetableprint| to
% something other than |0| (here |1| is used).
%
%    \begin{macrocode}
  \gdef\tfn@footnotetableprint{1}%
%    \end{macrocode}
%
% |\tfn@footnotetablecount| contains the negative number of footnote texts
% to be printed, therefore it is increase by one (i.\,e.~the negative number
% decreases, i.\,e.~$-3$ becomes~$-2$).
%
%    \begin{macrocode}
  \@tempcnta=\tfn@footnotetablecount\relax%
  \advance\@tempcnta by +1\relax%
  \xdef\tfn@footnotetablecount{\the\@tempcnta}%
%    \end{macrocode}
%
% When it is zero, all footnote texts have been placed, and\\
% |\tfn@footnotetableprint| can be reset to zero and\\
% |\tfn@tablefootnoteprintout| can be cleared.
%
%    \begin{macrocode}
  \ifx\tfn@footnotetablecount\tfn@footnotezero% all footnotes placed
    \gdef\tfn@footnotetableprint{0}%
    \global\let\tfn@tablefootnoteprintout\relax% brute force
 %\else there is at least one \footnotemark which needs \footnotetext
  \fi
}

%    \end{macrocode}
% \end{macro}
%
% \begin{macro}{\tfn@tablefootnoteprintout}
% When something shall be cleared or redefined, it is a good idea
% to define it. It~will be later filled with the foot note text to be printed.
%
%    \begin{macrocode}
\newcommand{\tfn@tablefootnoteprintout}{\relax}

%    \end{macrocode}
% \end{macro}
%
% \pagebreak
%
% \begin{macro}{\tablefootnote}
% |\tablefootnote[...]{...}| is the command to be used by the user instead of
% |\footnote[...]{...}| in the (sideways)table. It places a foot note mark via
% |\tfn@footnotemarks| and appends a |\tfn@footnotetext{#1}| to the\\
% |\tfn@tablefootnoteprintout| macro, and remembers that table foot notes
% have been used.
%
%    \begin{macrocode}
\newcommand{\tablefootnote}[2][]{%
  \tfn@footnotemarks[#1]%
  \ifthenelse{\isempty{#1}}{\relax}{%
    \ltx@GlobalPrependToMacro{\tfn@tablefootnoteprintout}{%
      \addtocounter{footnote}{+1}%
      \ltx@ifpackageloaded{hyperref}{%
        \tfn@hyperfootnotes{\addtocounter{Hfootnote}{+1}}{}%
        \tfn@updateHyper%
       }{}%
     }%
   }%
  \ltx@GlobalAppendToMacro{\tfn@tablefootnoteprintout}{%
    \tfn@footnotetext[#1]{#2}%
    \ifthenelse{\isempty{#1}}{\relax}{%
      \addtocounter{footnote}{-1}%
      \ltx@ifpackageloaded{hyperref}{%
        \tfn@hyperfootnotes{\addtocounter{Hfootnote}{-1}}{}%
        \tfn@updateHyper%
       }{}%
     }%
   }%
  \gdef\tfn@fnt{1}%
  }

%    \end{macrocode}
% \end{macro}
%
% \begin{macro}{\tfn@swtbox}
% For |sidewaystable|s the |\tfn@tablefootnoteprintout| is placed in a \\
% |\parbox|, which can be placed at the end of the |sidewaystable|:
%
%    \begin{macrocode}
\newcommand{\tfn@swtbox}{%
\ifvoid \footins%
\vskip 2\baselineskip%
\else%
\vskip \skip\footins%
\fi%
\parbox[b][][t]{\linewidth}{\raggedright\tfn@tablefootnoteprintout}%
}

%    \end{macrocode}
% \end{macro}
%
% |\tfn@tablefootnoteprintout| shall be executed after the end of the table environment.
% The \xpackage{etoolbox} package provides very easy syntax for this:
%
%    \begin{macrocode}
\IfFileExists{etoolbox.sty}{%
  \RequirePackage{etoolbox}[2011/01/03]%   v2.1
  \AfterEndEnvironment{table}{%
    \ltx@ifpackageloaded{footnotebackref}{%
      \LetLtxMacro{\tfncurrentmft}{\@makefntext}%
      \LetLtxMacro{\@makefntext}{\tablemakefntext}%
       \tfn@tablefootnoteprintout%
      \LetLtxMacro{\@makefntext}{\tfncurrentmft}%
     }{% footnotebackref not loaded
      \tfn@tablefootnoteprintout%
     }%
      \gdef\tfn@fnt{0}%
   }%
%    \end{macrocode}
%
% When the \xpackage{rotating} package has been loaded, we need to detect
% whether the |\tablefootnote| is placed inside of a |sidewaystable|.
%
%    \begin{macrocode}
  \ltx@ifpackageloaded{rotating}{%
    \AtBeginEnvironment{sidewaystable}{%
      \gdef\tfn@swt{1}%
    }
%    \end{macrocode}
%
% At the end of |sidewaystable| environments |\tfn@swtbox| shall be placed
% (if~there was any |\tablefootnote| in that |sidewaystable| at all).
% And |\tfn@swt| as well as |\tfn@fnt| must be reset to |0|.
%
%    \begin{macrocode}
    \AtEndEnvironment{sidewaystable}{%
      \ifx\tfn@fnt\tfn@footnotezero% \relax
      \else%
        \ltx@ifpackageloaded{footnotebackref}{%
          \LetLtxMacro{\tfncurrentmft}{\@makefntext}%
          \LetLtxMacro{\@makefntext}{\tablemakefntext}%
           \tfn@swtbox%
          \LetLtxMacro{\@makefntext}{\tfncurrentmft}%
         }{% footnotebackref not loaded
          \tfn@swtbox%
          }%
      \fi%
      \gdef\tfn@swt{0}%
      \gdef\tfn@fnt{0}%
     }%
   }{}% else \relax
}{%
%    \end{macrocode}
%
% When the \xpackage{etoolbox} package is not available (e.\,g. because of missing
% \eTeX{}) we~do a workaround:
%
%    \begin{macrocode}
  \PackageWarning{tablefootnote}{%
    Package etoolbox not available, therefore\MessageBreak%
    substituting commands \string\AtEndEnvironment\space %
    and\MessageBreak%
    \string\AfterEndEnvironment\space for tablefootnotes%
    \MessageBreak%
   }%
  \@ifpackagelater{letltxmacro}{2010/09/02}%
   {% >= 2010/09/02, OK
    \GlobalLetLtxMacro{\tfnorigtable}{\table}
    \GlobalLetLtxMacro{\tfnendorigtable}{\endtable}
%    \end{macrocode}
%
% |\GlobalLetLtxMacro| was introduced in version 2010/09/02 v1.4 of the
% \xpackage{letltxmacro} package.
%
%    \begin{macrocode}
   }{% else
     \PackageError{tablefootnote}{%
       Outdated version of letltxmacro package used%
      }{Version 2010/09/02 v1.4 or newer needed but not found!%
        \MessageBreak%
        Update the used letltxmacro package.}
        \LetLtxMacro{\tfnorigtable}{\table}
        \LetLtxMacro{\tfnendorigtable}{\endtable}
    }%
%    \end{macrocode}
%
% \pagebreak
%
%    \begin{macrocode}
  \ltx@ifpackageloaded{float}{%
    \@ifpackagelater{float}{2001/11/08}%   v1.3d
     {% >= 2001/11/08, OK
     }{% else
       \PackageWarningNoLine{tablefootnote}{%
         It is required version\MessageBreak
         2001/11/08 v1.3d (or later) of package float,\MessageBreak
         but only version\MessageBreak
        `\csname ver@float.sty\endcsname'\MessageBreak
        is available}%
     }%
   }{\relax}
  \renewenvironment{table}{%
    \ltx@ifpackageloaded{float}{%
%    \end{macrocode}
%
% When \xpackage{etoolbox.sty} was not found, and the \xpackage{float} package
% has been loaded, and the float specifier is \texttt{H}, then |\endtable| was
% let to |\float@endH| by the \xpackage{float} package (line~79 of
% \xpackage{float.sty}, 2001/11/08 v1.3d:\\
% |\expandafter\let\csname end#1\endcsname\float@endH|), i.\,e.~appending\\
% |\tfn@tablefootnoteprintout \gdef\tfn@fnt{0}| to |\endtable| is useless.
% Instead of this, it must be appended to |\float@endH|.\\
% (Why is this not necessary when \xpackage{etoolbox} is used?
% That is a good question, answers are welcome.)
%
%    \begin{macrocode}
      \ltx@LocalAppendToMacro{\float@endH}{%
        \ltx@ifpackageloaded{footnotebackref}{%
          \LetLtxMacro{\tfncurrentmft}{\@makefntext}%
          \LetLtxMacro{\@makefntext}{\tablemakefntext}%
           \tfn@tablefootnoteprintout%
          \LetLtxMacro{\@makefntext}{\tfncurrentmft}%
         }{% footnotebackref not loaded
          \tfn@tablefootnoteprintout%
         }%
          \gdef\tfn@fnt{0}%
       }%
     }{\relax}
    \tfnorigtable%
   }{%
     \tfnendorigtable%
     \ltx@ifpackageloaded{footnotebackref}{%
       \LetLtxMacro{\tfncurrentmft}{\@makefntext}%
       \LetLtxMacro{\@makefntext}{\tablemakefntext}%
        \tfn@tablefootnoteprintout%
       \LetLtxMacro{\@makefntext}{\tfncurrentmft}%
      }{% footnotebackref not loaded
       \tfn@tablefootnoteprintout%
      }%
     \gdef\tfn@fnt{0}%
    }%
%    \end{macrocode}
%
% \pagebreak
%
% The |sidewaystable|-environment must be treated separately:
%
%    \begin{macrocode}
  \ltx@ifpackageloaded{rotating}{%
    \let\tfnorigsidewaystable\sidewaystable
    \let\tfnendorigsidewaystable\endsidewaystable
    \renewenvironment{sidewaystable}{%
      \gdef\tfn@swt{1}%
      \tfnorigsidewaystable%
    }{%
      \ifx\tfn@fnt\tfn@footnotezero% \relax
      \else%
        \ltx@ifpackageloaded{footnotebackref}{%
          \LetLtxMacro{\tfncurrentmft}{\@makefntext}%
          \LetLtxMacro{\@makefntext}{\tablemakefntext}%
           \tfn@swtbox%
          \LetLtxMacro{\@makefntext}{\tfncurrentmft}%
         }{% footnotebackref not loaded
          \tfn@swtbox%
         }%
      \fi%
      \gdef\tfn@swt{0}%
      \gdef\tfn@fnt{0}%
      \tfnendorigsidewaystable%
     }%
   }{}%
}

%    \end{macrocode}
%
% \DescribeMacro{\AtBeginDocument}
% When the \xpackage{rotating} (for |sidewaystable|s) or the
% \xpackage{footnotebackref} (for back references for the footnotes)
% package is used, it must be loaded before the \xpackage{tablefootnote} package.
% Using both packages is also possible, in which case both must be loaded before
% the \xpackage{tablefootnote} package. If any one of those packages is loaded
% |\AtBeginDocument| after this check, the according error message will not be
% given, but packages really should be loaded before |\AtBeginDocument|.
%
%    \begin{macrocode}
\AtBeginDocument{%
  \ltx@ifpackageloaded{footnotebackref}{%
    \ifx\tfn@fnbr\tfn@footnotezero%
      \PackageError{tablefootnote}{%
        Package footnotebackref loaded after tablefootnote}{%
        When using both the footnotebackref and the tablefootnote %
        package,\MessageBreak%
        the footnotebackref package MUST be loaded before the %
        tablefootnote package!\MessageBreak%
       }%
    \fi%
   }{% else \relax
   }
  \ltx@ifpackageloaded{rotating}{%
    \ifx\tfn@rotating\tfn@footnotezero%
      \PackageError{tablefootnote}{%
        Package rotating loaded after tablefootnote}{%
        When using both the rotating and the tablefootnote package,%
        \MessageBreak%
        the rotating package MUST be loaded before the tablefootnote%
        \space package!\MessageBreak%
       }%
    \fi%
   }{% else \relax
   }
%    \end{macrocode}
%
% \pagebreak
%
% When the document is compiled with \LuaLaTeX{}, hyperlinks in rotated content
% will be misplaced, regardless of the use of the \xpackage{tablefootnote} package
% (or of tables or footnotes). The text to be printed will be OK.
% This problem of \LuaLaTeX{} (not \xpackage{tablefootnote}) was found and reported
% by \textsc{Arno Trautmann}, 2011/11/15 -- thanks!
%
%    \begin{macrocode}
  \ltx@ifpackageloaded{hyperref}{%
    \tfn@hyperfootnotes{
      \ifluatex%
        \PackageWarningNoLine{tablefootnote}{%
          LuaTeX and hyperref package used:\MessageBreak%
          Hyperreferences in rotated content\MessageBreak%
          will be misplaced\MessageBreak%
          regardless of tablefootnote package.\MessageBreak%
          The text to be printed will be OK%
         }%
      \fi%
     }{% else \relax
     }%
   }{% else \relax
   }%
%    \end{macrocode}
%
% When the \xpackage{footnotebackref} package is used but the footnotes are not
% hyperlinked, neither the back references nor the tablefootnotes will work.
%
%    \begin{macrocode}
  \ltx@ifpackageloaded{footnotebackref}{%
     \ltx@ifpackageloaded{hyperref}{%
        \tfn@hyperfootnotes{\relax}{% else
          \PackageError{tablefootnote}{%
             footnotebackref but not hyperfootnotes}{%
             The footnotebackref package is used\MessageBreak%
             but the hyperfootnotes option of the hyperref package%
             \MessageBreak%
             was set to false.\MessageBreak%
             Either enable hyperfootnotes or do not use footnotebackref.%
             \MessageBreak%
             Loading of the tablefootnote package will be aborted.%
           }%
          \endinput%
      }%
    }{\PackageError{tablefootnote}{footnotebackref but not hyperref}{%
        The footnotebackref package is used\MessageBreak%
        but not the hyperref package.\MessageBreak%
        Either load hyperref or do not use footnotebackref.%
        \MessageBreak%
        Loading of the tablefootnote package will be aborted.%
        }%
      \endinput%
    }%
  }{}
}

%    \end{macrocode}
%
%    \begin{macrocode}
%</package>
%    \end{macrocode}
%
% \pagebreak
%
% \section{Installation}
%
% \subsection{Downloads\label{ss:Downloads}}
%
% Everything is available on \CTAN{}, \url{http://www.ctan.org/},
% but may need additional packages themselves.\\
%
% \DescribeMacro{tablefootnote.dtx}
% For unpacking the |tablefootnote.dtx| file and constructing the documentation
% it is required:
% \begin{description}
% \item[-] \TeX{} Format \LaTeXe{}: \url{http://www.CTAN.org/}
%
% \item[-] document class \xpackage{ltxdoc}, 2007/11/11, v2.0u,
%           \url{http://www.ctan.org/pkg/ltxdoc}
%
% \item[-] package \xpackage{holtxdoc}, 2012/03/21, v0.24,
%           \url{http://www.ctan.org/pkg/holtxdoc}
%
% \item[-] package \xpackage{hypdoc}, 2011/08/19, v1.11,
%           \url{http://www.ctan.org/pkg/hypdoc}
% \end{description}
%
% \DescribeMacro{tablefootnote.sty}
% The |tablefootnote.sty| for \LaTeXe{} (i.\,e.~each document using
% the \xpackage{tablefootnote} package) requires:
% \begin{description}
% \item[-] \TeX{} Format \LaTeXe{}, \url{http://www.CTAN.org/}
%
% \item[-] package \xpackage{letltxmacro}, 2010/09/02, v1.4,
%           \url{http://www.ctan.org/pkg/letltxmacro}.
%
% \item[-] package \xpackage{ltxcmds}, 2011/11/09, v1.22,
%           \url{http://www.ctan.org/pkg/ltxcmds}
% \end{description}
%
% When the \xpackage{hyperref} package is used, also the \xpackage{ifluatex} package
% is needed, but it is already loaded automatically by the \xpackage{hyperref}
% package.\\
% When the \xpackage{etoolbox} package is available, it is used:
% \begin{description}
% \item[-] package \xpackage{etoolbox}, 2011/01/03, v2.1,
%           \url{http://www.ctan.org/pkg/etoolbox}.
% \end{description}
%
% \DescribeMacro{tablefootnote-example.tex}
% The |tablefootnote-example.tex| requires the same files as all
% documents using the \xpackage{tablefootnote} package and additionally:
% \begin{description}
% \item[-] class \xpackage{article}, 2007/10/19, v1.4h, from \xpackage{classes.dtx}:\\
%           \CTAN{macros/latex/base/classes.dtx}
%
% \item[-] package \xpackage{rotating}, 2009/03/28, v2.16a,
%           \url{http://www.ctan.org/pkg/rotating}
%
% \item[-] package \xpackage{float}, 2001/11/08, v1.3d,
%           \url{http://www.ctan.org/pkg/float}
%
% \item[-] package \xpackage{placeins}, 2005/04/18, v2.2,
%           \url{http://www.ctan.org/pkg/placeins}\\
%           for |\FloatBarrier|
%
% \item[-] package \xpackage{hyperref}, 2012/11/06, v6.83m,
%           \url{http://www.ctan.org/pkg/hyperref}
%
% \item[-] package \xpackage{tablefootnote}, 2014/01/26, v1.1c,\\
%           \url{http://www.ctan.org/pkg/tablefootnote}\\
%   (Well, it is the example file for this package, and because you
%    are reading the documentation for the \xpackage{tablefootnote} package,
%    it can be assumed that you already have some version of it~--
%    is it the current one?)
% \end{description}
%
% \DescribeMacro{Oberdiek}
% \DescribeMacro{holtxdoc}
% \DescribeMacro{ifluatex}
% \DescribeMacro{letltxmacro}
% \DescribeMacro{ltxcmds}
% All packages of \textsc{Heiko Oberdiek}'s bundle `oberdiek'
% (especially \xpackage{holtxdoc}, \xpackage{ifluatex}, \xpackage{letltxmacro},
% and \xpackage{ltxcmds}) are also available in a TDS compliant ZIP archive:\\
% \CTAN{install/macros/latex/contrib/oberdiek.tds.zip}.\\
% It is probably best to download and use this, because the packages in there
% are quite probably both recent and compatible among themselves.\\
%
% \pagebreak
%
% \DescribeMacro{hyperref}
% \noindent \xpackage{hyperref} is not included in that bundle and needs to be
% downloaded separately,\\
% \url{http://www.ctan.org/install/macros/latex/contrib/hyperref.tds.zip}.\\
%
% \DescribeMacro{M\"{u}nch}
% A hyperlinked list of my (other) packages can be found at
% \url{http://www.ctan.org/author/muench-hm}.\\
%
% \subsection{Package, unpacking TDS}
% \paragraph{Package.} This package is available on \CTAN{}
% \begin{description}
% \item[\CTAN{macros/latex/contrib/tablefootnote/tablefootnote.dtx}]\hspace*{0.1cm} \\
%       The source file.
% \item[\CTAN{macros/latex/contrib/tablefootnote/tablefootnote.pdf}]\hspace*{0.1cm} \\
%       The documentation.
% \item[\CTAN{macros/latex/contrib/tablefootnote/tablefootnote-example.pdf}]\hspace*{0.1cm} \\
%       The compiled example file, as it should look like.
% \item[\CTAN{macros/latex/contrib/tablefootnote/README}]\hspace*{0.1cm} \\
%       The README file.
% \end{description}
% There is also a tablefootnote.tds.zip available:
% \begin{description}
% \item[\CTAN{install/macros/latex/contrib/tablefootnote.tds.zip}]\hspace*{0.1cm} \\
%       Everything in \xfile{TDS} compliant, compiled format.
% \end{description}
% which additionally contains\\
% \begin{tabular}{ll}
% tablefootnote.ins & The installation file.\\
% tablefootnote.drv & The driver to generate the documentation.\\
% tablefootnote.sty & The \xext{sty}le file.\\
% tablefootnote-example.tex & The example file.
% \end{tabular}
%
% \bigskip
%
% \noindent For required other packages, please see the preceding subsection.
%
% \paragraph{Unpacking.} The \xfile{.dtx} file is a self-extracting
% \docstrip{} archive. The files are extracted by running the
% \xext{dtx} through \plainTeX{}:
% \begin{quote}
%   \verb|tex tablefootnote.dtx|
% \end{quote}
%
% \noindent About generating the documentation see paragraph~\ref{GenDoc} below.\\
%
% \paragraph{TDS.} Now the different files must be moved into
% the different directories in your installation TDS tree
% (also known as \xfile{texmf} tree):
% \begin{quote}
% \def\t{^^A
% \begin{tabular}{@{}>{\ttfamily}l@{ $\rightarrow$ }>{\ttfamily}l@{}}
%   tablefootnote.sty & tex/latex/tablefootnote/tablefootnote.sty\\
%   tablefootnote.pdf & doc/latex/tablefootnote/tablefootnote.pdf\\
%   tablefootnote-example.tex & doc/latex/tablefootnote/tablefootnote-example.tex\\
%   tablefootnote-example.pdf & doc/latex/tablefootnote/tablefootnote-example.pdf\\
%   tablefootnote.dtx & source/latex/tablefootnote/tablefootnote.dtx\\
% \end{tabular}^^A
% }^^A
% \sbox0{\t}^^A
% \ifdim\wd0>\linewidth
%   \begingroup
%     \advance\linewidth by\leftmargin
%     \advance\linewidth by\rightmargin
%   \edef\x{\endgroup
%     \def\noexpand\lw{\the\linewidth}^^A
%   }\x
%   \def\lwbox{^^A
%     \leavevmode
%     \hbox to \linewidth{^^A
%       \kern-\leftmargin\relax
%       \hss
%       \usebox0
%       \hss
%       \kern-\rightmargin\relax
%     }^^A
%   }^^A
%   \ifdim\wd0>\lw
%     \sbox0{\small\t}^^A
%     \ifdim\wd0>\linewidth
%       \ifdim\wd0>\lw
%         \sbox0{\footnotesize\t}^^A
%         \ifdim\wd0>\linewidth
%           \ifdim\wd0>\lw
%             \sbox0{\scriptsize\t}^^A
%             \ifdim\wd0>\linewidth
%               \ifdim\wd0>\lw
%                 \sbox0{\tiny\t}^^A
%                 \ifdim\wd0>\linewidth
%                   \lwbox
%                 \else
%                   \usebox0
%                 \fi
%               \else
%                 \lwbox
%               \fi
%             \else
%               \usebox0
%             \fi
%           \else
%             \lwbox
%           \fi
%         \else
%           \usebox0
%         \fi
%       \else
%         \lwbox
%       \fi
%     \else
%       \usebox0
%     \fi
%   \else
%     \lwbox
%   \fi
% \else
%   \usebox0
% \fi
% \end{quote}
% If you have a \xfile{docstrip.cfg} that configures and enables \docstrip{}'s
% \xfile{TDS} installing feature, then some files can already be in the right
% place, see the documentation of \docstrip{}.
%
% \subsection{Refresh file name databases}
%
% If your \TeX{}~distribution (\TeX live, \mikTeX{}, \dots{}) relies on file name
% databases, you must refresh these. For example, \teTeX{} users run
% \verb|texhash| or \verb|mktexlsr|.
%
% \subsection{Some details for the interested}
%
% \paragraph{Unpacking with \LaTeX{}.}
% The \xfile{.dtx} chooses its action depending on the format:
% \begin{description}
% \item[\plainTeX:] Run \docstrip{} and extract the files.
% \item[\LaTeX:] Generate the documentation.
% \end{description}
% If you insist on using \LaTeX{} for \docstrip{} (really,
% \docstrip{} does not need \LaTeX{}), then inform the autodetect routine
% about your intention:
% \begin{quote}
%   \verb|latex \let\install=y\input{tablefootnote.dtx}|
% \end{quote}
% Do not forget to quote the argument according to the demands
% of your shell.
%
% \paragraph{Generating the documentation.\label{GenDoc}}
% You can use both the \xfile{.dtx} or the \xfile{.drv} to generate
% the documentation. The process can be configured by a
% configuration file \xfile{ltxdoc.cfg}. For instance, put the following
% line into this file, if you want to have A4 as paper format:
% \begin{quote}
%   \verb|\PassOptionsToClass{a4paper}{article}|
% \end{quote}
%
% \noindent An example follows how to generate the
% documentation with \pdfLaTeX{}:
%
% \begin{quote}
%\begin{verbatim}
%pdflatex tablefootnote.dtx
%makeindex -s gind.ist tablefootnote.idx
%pdflatex tablefootnote.dtx
%makeindex -s gind.ist tablefootnote.idx
%pdflatex tablefootnote.dtx
%\end{verbatim}
% \end{quote}
%
% \subsection{Compiling the example}
%
% The example file, \textsf{tablefootnote-example.tex}, can be compiled via\\
% \indent |latex tablefootnote-example.tex|,\\
% \indent |lualatex tablefootnote-example.tex|,\\
% or (recommended)\\
% \indent |pdflatex thumbs-example.tex|\\
% and will need at least two compiler runs to get everything right.
%
% \bigskip
%
% \section{Acknowledgements}
%
% I would like to thank \textsc{Heiko Oberdiek} for providing
% the \xpackage{hyperref}, \xpackage{ifluatex}, \xpackage{letltxmacro},
% \xpackage{ltxcmds}, as well as a~lot~(!) of other useful packages
% (from which I also got everything I know about creating a file in
% \xext{dtx} format, OK, say it: copying),
% and for information about the \pdfTeX{}-bug,
% \textsc{Arno Trautmann} for reporting the bug of misplaced hyperreferences in
% |sidewaystable|s using \LuaTeX{}, independently of the \xpackage{tablefootnote}
% package or footnotes at all,
% \textsc{Thomas~V.} for reporting the bug of not handling the case of
% \xpackage{hyperref} option |hyperfootnotes=false|,
% \textsc{Sverre Stausland Johnsen} for reporting the bug in case of \emph{not}
% using \xpackage{hyperref},
% and the \Newsgroup{comp.text.tex} and \Newsgroup{de.comp.text.tex} newsgroups
% as well as everybody at \url{http://tex.stackexchange.com}
% for their help in all things \TeX{}.
%
% \pagebreak
%
% \phantomsection
% \begin{History}\label{History}
%   \begin{Version}{2011/10/26 v1.0a, tabfootn}
%     \item Upload to \CTAN{macros/latex/contrib/tablefootnote/}.
%   \end{Version}
%   \begin{Version}{2011/10/29 v1.0b}
%     \item Renamed to \xpackage{tablefootnote}.
%     \item Added support for the |sidewaystable|-environment of the
%             \xpackage{rotating} package.
%     \item Diverse changes in documentation and README.
%   \end{Version}
%   \begin{Version}{2011/11/06 v1.0c}
%     \item Replaced |^{\text{\thefootnote}}| by |{}^{\text{\textup{\thefootnote}}}|.
%     \item Bug fix: When \xpackage{etoolbox.sty} was not found, and the \xpackage{float}
%             package had been loaded, and the float specifier was \texttt{H},
%             then the |\tablefootnote|s were not printed for that table. Fixed.
%   \end{Version}
%   \begin{Version}{2011/11/19 v1.0d}
%     \item Replaced |{}^{\text{\textup{\thefootnote}}}| by
%             |\textsuperscript{\thefootnote}|, therefore the \xpackage{amstext}
%             package is no longer required by the \xpackage{tablefootnote} package.
%     \item \xpackage{lscape} and \xpackage{pdflscape} package now work with
%             \xpackage{tablefootnote} (except for using a \texttt{sidewaystable} on
%             a \texttt{landscape} page).
%     \item Added information about footnotes in \texttt{sidewaystable} being printed
%             one below the other, even if \xpackage{footmisc} package with option
%             \texttt{para} is used.
%     \item Redefined |\tfn@swtbox| from
%  \begin{verbatim}
%  \vspace{0.8cm}%
%  \begin{spacing}{0.1}%
%  \subcaptionbox*{}[\linewidth][l]{\tfn@tablefootnoteprintout}%
%  \end{spacing}%
%  \end{verbatim}
%             to
%  \begin{verbatim}
%  \vspace{2\baselineskip}%
%  \parbox[b][][t]{\linewidth}{\raggedright\tfn@tablefootnoteprintout}%
%  \end{verbatim}
%             , therefore the \xpackage{setspace} and \xpackage{subcaption} packages
%             are no longer required by the \xpackage{tablefootnote} package.
%     \item No longer uses |\numexpr|, therefore also usable without \eTeX{}.
%     \item Replaced |\textsuperscript{\thefootnote}| by
%             |\hbox {\@textsuperscript {\normalfont \thefootnote }}|,
%             which is next to the original definition of |\@makefnmark|
%             (|\hbox {\@textsuperscript {\normalfont \@thefnmark }}|,
%             but according to |\show\@thefnmark| here |\@thefnmark| is
%             |{\protect \itshape}| instead of |\thefootnote|).
%     \item Added an error message, when \LuaTeX{}, \xpackage{rotating} package, and
%            \xpackage{hyperref} package are used together. Hyperreferences in
%            |sidewaystable|s are misplaced, independently of the
%            \xpackage{tablefootnote} package or footnotes at all. (Bug reported
%            by \textsc{Arno Trautmann}, 2011/11/15 -- thanks!)
%     \item Added footnotesymbol-footnote-numbers in the example.
%   \end{Version}
%   \begin{Version}{2011/11/26 v1.0e}
%     \item Check for \xpackage{hyperref} is only done once.
%     \item Message regarding the misplacement of hyperlinks by \LuaLaTeX{}
%             changed form error to warning.
%     \item Replaced |\vspace{2\baselineskip}| by
%  \begin{verbatim}
% \ifvoid \footins%
% \vskip 2\baselineskip%
% \else%
% \vskip \skip\footins%
% \fi%
%  \end{verbatim}
%             between the |sidewaystable|s and their footnotes.
%     \item The warning about un-regarded |para|-option of \xpackage{footmoisc}
%             package is only given at the second |tablefootnote| in a |sidewaystable|,
%             not repeatedly for every |tablefootnote| in a |sidewaystable|.
%   \end{Version}
%   \begin{Version}{2012/01/01 v1.0f}
%     \item Bugfix: Wrong installation path given in the documentation, fixed.
%     \item Update of documentation, README, and \xfile{dtx} internals.
%   \end{Version}
%   \begin{Version}{2012/01/14 v1.0g}
%     \item Bugfix: There was a |relax| instead of a |\relax|, fixed.
%   \end{Version}
%   \begin{Version}{2012/07/29 v1.0h}
%     \item \xpackage{tablefootnote} is now compatible to the new
%             \xpackage{footnotebackref} package, 2012/07/01, v1.0,
%             \url{http://www.ctan.org/pkg/footnotebackref}.
%   \end{Version}
%   \begin{Version}{2014/01/08 v1.1a}
%     \item Tablefootnotes now regard optional footnote marks:
%             |\tablefootnote[| custom mark here! |]{...}|
%     \item Bug fix: the case of \xpackage{hyperref} option |hyperfootnotes=false|
%             was not handeled at all. (Bug reported by \textsc{Thomas~V.}, %
%             2014/01/02 -- thanks!)
%     \item A lot of details.
%   \end{Version}
%   \begin{Version}{2014/01/20 v1.1b}
%     \item Bug fix: Package was broken when \xpackage{hyperref} was \emph{not} used.
%            (Bug reported by \textsc{Sverre Stausland Johnsen} -- thanks!)
%   \end{Version}
%   \begin{Version}{2014/01/26 v1.1c}
%     \item The emergency bug fix of version 2014/01/20 v1.1b was replaced by using
%             the new robust macro |\tfn@hyperfootnotes|, which only \emph{internally}
%             uses |\ifHy@hyperfootnotes|. If the \xpackage{hyperref} package is not used,
%             \LaTeX{} now does not see any confusing
%             |\ifHy@hyperfootnotes|\ldots|\else|\ldots|\fi| at all.
%     \item URLs and README updated.
%   \end{Version}
% \end{History}
%
% \bigskip
%
% When you find a mistake or have a suggestion for an improvement of this package,
% please send an e-mail to the maintainer, thanks! (Please see BUG REPORTS in the README.)
%
% \bigskip
%
% \PrintIndex
%
% \Finale
\endinput
%        (quote the arguments according to the demands of your shell)
%
% Documentation:
%    (a) If tablefootnote.drv is present:
%           (pdf)latex tablefootnote.drv
%           makeindex -s gind.ist tablefootnote.idx
%           (pdf)latex tablefootnote.drv
%           makeindex -s gind.ist tablefootnote.idx
%           (pdf)latex tablefootnote.drv
%    (b) Without tablefootnote.drv:
%           (pdf)latex tablefootnote.dtx
%           makeindex -s gind.ist tablefootnote.idx
%           (pdf)latex tablefootnote.dtx
%           makeindex -s gind.ist tablefootnote.idx
%           (pdf)latex tablefootnote.dtx
%
%    The class ltxdoc loads the configuration file ltxdoc.cfg
%    if available. Here you can specify further options, e.g.
%    use DIN A4 as paper format:
%       \PassOptionsToClass{a4paper}{article}
%
% Installation:
%    TDS:tex/latex/tablefootnote/tablefootnote.sty
%    TDS:doc/latex/tablefootnote/tablefootnote.pdf
%    TDS:doc/latex/tablefootnote/tablefootnote-example.tex
%    TDS:doc/latex/tablefootnote/tablefootnote-example.pdf
%    TDS:source/latex/tablefootnote/tablefootnote.dtx
%
%<*ignore>
\begingroup
  \catcode123=1 %
  \catcode125=2 %
  \def\x{LaTeX2e}%
\expandafter\endgroup
\ifcase 0\ifx\install y1\fi\expandafter
         \ifx\csname processbatchFile\endcsname\relax\else1\fi
         \ifx\fmtname\x\else 1\fi\relax
\else\csname fi\endcsname
%</ignore>
%<*install>
\input docstrip.tex
\Msg{******************************************************************}
\Msg{* Installation                                                   *}
\Msg{* Package: tablefootnote 2014/01/26 v1.1c Table foot notes (HMM) *}
\Msg{******************************************************************}

\keepsilent
\askforoverwritefalse

\let\MetaPrefix\relax
\preamble

This is a generated file.

Project: tablefootnote
Version: 2014/01/26 v1.1c

Copyright (C) 2011 - 2014 by
    H.-Martin M"unch <Martin dot Muench at Uni-Bonn dot de>

The usual disclaimer applies:
If it doesn't work right that's your problem.
(Nevertheless, send an e-mail to the maintainer
 when you find an error in this package.)

This work may be distributed and/or modified under the
conditions of the LaTeX Project Public License, either
version 1.3c of this license or (at your option) any later
version. This version of this license is in
   http://www.latex-project.org/lppl/lppl-1-3c.txt
and the latest version of this license is in
   http://www.latex-project.org/lppl.txt
and version 1.3c or later is part of all distributions of
LaTeX version 2005/12/01 or later.

This work has the LPPL maintenance status "maintained".

The Current Maintainer of this work is H.-Martin Muench.

This work consists of the main source file tablefootnote.dtx,
the README, and the derived files
   tablefootnote.sty, tablefootnote.pdf,
   tablefootnote.ins, tablefootnote.drv,
   tablefootnote-example.tex, tablefootnote-example.pdf.

In memoriam Tommy Muench + 2014/01/02.

\endpreamble
\let\MetaPrefix\DoubleperCent

\generate{%
  \file{tablefootnote.ins}{\from{tablefootnote.dtx}{install}}%
  \file{tablefootnote.drv}{\from{tablefootnote.dtx}{driver}}%
  \usedir{tex/latex/tablefootnote}%
  \file{tablefootnote.sty}{\from{tablefootnote.dtx}{package}}%
  \usedir{doc/latex/tablefootnote}%
  \file{tablefootnote-example.tex}{\from{tablefootnote.dtx}{example}}%
}

\catcode32=13\relax% active space
\let =\space%
\Msg{************************************************************************}
\Msg{*}
\Msg{* To finish the installation you have to move the following}
\Msg{* file into a directory searched by TeX:}
\Msg{*}
\Msg{*     tablefootnote.sty}
\Msg{*}
\Msg{* To produce the documentation run the file `tablefootnote.drv'}
\Msg{* through (pdf)LaTeX, e.g.}
\Msg{*  pdflatex tablefootnote.drv}
\Msg{*  makeindex -s gind.ist tablefootnote.idx}
\Msg{*  pdflatex tablefootnote.drv}
\Msg{*  makeindex -s gind.ist tablefootnote.idx}
\Msg{*  pdflatex tablefootnote.drv}
\Msg{*}
\Msg{* At least three runs are necessary e.g. to get the}
\Msg{*  references right!}
\Msg{*}
\Msg{* Happy TeXing!}
\Msg{*}
\Msg{************************************************************************}

\endbatchfile
%</install>
%<*ignore>
\fi
%</ignore>
%
% \section{The documentation driver file}
%
% The next bit of code contains the documentation driver file for
% \TeX{}, i.\,e., the file that will produce the documentation you
% are currently reading. It will be extracted from this file by the
% \texttt{docstrip} programme. That is, run \LaTeX{} on \texttt{docstrip}
% and specify the \texttt{driver} option when \texttt{docstrip}
% asks for options.
%
%    \begin{macrocode}
%<*driver>
\NeedsTeXFormat{LaTeX2e}[2011/06/27]
\ProvidesFile{tablefootnote.drv}%
  [2014/01/26 v1.1c Table foot notes (HMM)]
\documentclass{ltxdoc}[2007/11/11]% v2.0u
\usepackage{holtxdoc}[2012/03/21]%  v0.24
%% tablefootnote may work with earlier versions of LaTeX2e and those
%% class and package, but this was not tested.
%% Please consider updating your LaTeX, class, and package
%% to the most recent version (if they are not already the most
%% recent version).
\hypersetup{%
 pdfsubject={Footnotes in Tables (HMM)},%
 pdfkeywords={LaTeX, tablefootnote, footnote, table, H.-Martin Muench},%
 pdfencoding=auto,%
 pdflang={en},%
 breaklinks=true,%
 linktoc=all,%
 pdfstartview=FitH,%
 pdfpagelayout=OneColumn,%
 bookmarksnumbered=true,%
 bookmarksopen=true,%
 bookmarksopenlevel=3,%
 pdfmenubar=true,%
 pdftoolbar=true,%
 pdfwindowui=true,%
 pdfnewwindow=true%
}
\CodelineIndex
\hyphenation{printing docu-ment}
\gdef\unit#1{\mathord{\thinspace\mathrm{#1}}}%
\makeatletter
\@ifundefined{eTeX}{\gdef\eTeX{$\m@th \varepsilon $-\TeX }}{% else \relax
}
\makeatother
\begin{document}
  \DocInput{tablefootnote.dtx}%
\end{document}
%</driver>
%    \end{macrocode}
%
% \fi
%
% \CheckSum{584}
%
% \CharacterTable
%  {Upper-case    \A\B\C\D\E\F\G\H\I\J\K\L\M\N\O\P\Q\R\S\T\U\V\W\X\Y\Z
%   Lower-case    \a\b\c\d\e\f\g\h\i\j\k\l\m\n\o\p\q\r\s\t\u\v\w\x\y\z
%   Digits        \0\1\2\3\4\5\6\7\8\9
%   Exclamation   \!     Double quote  \"     Hash (number) \#
%   Dollar        \$     Percent       \%     Ampersand     \&
%   Acute accent  \'     Left paren    \(     Right paren   \)
%   Asterisk      \*     Plus          \+     Comma         \,
%   Minus         \-     Point         \.     Solidus       \/
%   Colon         \:     Semicolon     \;     Less than     \<
%   Equals        \=     Greater than  \>     Question mark \?
%   Commercial at \@     Left bracket  \[     Backslash     \\
%   Right bracket \]     Circumflex    \^     Underscore    \_
%   Grave accent  \`     Left brace    \{     Vertical bar  \|
%   Right brace   \}     Tilde         \~}
%
% \GetFileInfo{tablefootnote.drv}
%
% \begingroup
%   \def\x{\#,\$,\^,\_,\~,\ ,\&,\{,\},\%}%
%   \makeatletter
%   \@onelevel@sanitize\x
% \expandafter\endgroup
% \expandafter\DoNotIndex\expandafter{\x}
% \expandafter\DoNotIndex\expandafter{\string\ }
% \begingroup
%   \makeatletter
%     \lccode`9=32\relax
%     \lowercase{%^^A
%       \edef\x{\noexpand\DoNotIndex{\@backslashchar9}}%^^A
%     }%^^A
%   \expandafter\endgroup\x
%
% \DoNotIndex{\\,\,}
% \DoNotIndex{\documentclass,\usepackage,\ProvidesPackage,\begin,\end}
% \DoNotIndex{\MessageBreak,\message}
% \DoNotIndex{\NeedsTeXFormat,\DoNotIndex,\verb}
% \DoNotIndex{\def,\edef,\gdef,\global}
% \DoNotIndex{\ifx,\listfiles,\mathord,\mathrm}
% \DoNotIndex{\kvoptions,\SetupKeyvalOptions,\ProcessKeyvalOptions}
% \DoNotIndex{\bigskip,\space,\thinspace,\ldots}
% \DoNotIndex{\indent,\noindent,\newline,\linebreak,\pagebreak}
% \DoNotIndex{\textbf,\textit,\textsf,\texttt,\textsc,\textquotedblleft,\textquotedblright}
% \DoNotIndex{\plainTeX,\TeX,\LaTeX,\pdfLaTeX}
% \DoNotIndex{\chapter,\section,\the,\csname,\endscname}
% \DoNotIndex{\value,\arabic,\hline,\text,\caption,\centering,\clear,\url}
%
% \title{The \xpackage{tablefootnote} package}
% \date{2014/01/26 v1.1c}
% \author{H.-Martin M\"{u}nch\\\xemail{Martin.Muench at Uni-Bonn.de}}
%
% \maketitle
%
% \begin{abstract}
% This \LaTeX{} package provides the command |\tablefootnote| to be used
% in a \texttt{table} or \texttt{sidewaystable} environment,
% where |\footnote| will not work and when using |\footnotemark| and
% |\footnotetext| and adjusting the counters (including \texttt{Hfootnote})
% manually is either too much work or would not even work (\texttt{sidewaystable}).
% \end{abstract}
%
% \bigskip
%
% \noindent Disclaimer for web links: The author is not responsible for any contents
% referred to in this work unless he has full knowledge of illegal contents.
% If any damage occurs by the use of information presented there, only the
% author of the respective pages might be liable, not the one who has referred
% to these pages.
%
% \bigskip
%
% \noindent {\color{green} Save per page about $200\unit{ml}$ water,
% $2\unit{g}$ CO$_{2}$ and $2\unit{g}$ wood:\\
% Therefore please print only if this is really necessary.}
%
% \pagebreak
%
% \tableofcontents
%
% \pagebreak
%
% \section{Introduction\label{sec:Introduction}}
%
% \indent This \LaTeX{} package provides the command |\tablefootnote{...}| to be
% used in a \texttt{table} or \texttt{sidewaystable} environment. In those
% environments |\footnote{...}| would not work and would need to be replaced
% by |\footnotemark| in the \texttt{(sideways)\allowbreak table} and
% |\footnotetext{...}| after the \texttt{(sideways)table} environment ended.
% Additionally the counters for \texttt{footnote} and
% (when the \xpackage{hyperref} package is used) \texttt{Hfootnote}
% would need to be adjusted manually. Then still the hyperlinks
% in a \texttt{(sideways)table} would not work.
% When the \xpackage{footnotebackref} package is used,
% |\footnotemark| and |\footnotetext{...}| would need to be redefined.
% Just using |\tablefootnote{...}| (with optional argument for custom foot note
% marks) in the (sideways)table does all this automatically.
% (Compatibility with the \xpackage{footnotebackref} package
% was achieved without redefining |\footnotemark| or |\footnotetext|.)
% Redefining |\footnote| was \emph{not} done as to not create conflicts with
% other packages.
%
% \section{Usage}
%
% \indent Just load the package placing
% \begin{quote}
%   |\usepackage{tablefootnote}|
% \end{quote}
% \noindent in the preamble of your \LaTeXe{} source file and use\newline
% |\tablefootnote{...}| instead of |\footnote{...}| and\newline
% |\tablefootnote[...]{...}| instead of |\footnote[...]{...}|
% in (sideways)\allowbreak tables.\\
% When the \xpackage{footnotebackref} package is used,
% it must be loaded before the \xpackage{tablefootnote} package
% and the \xpackage{hyperrefpackage} with option |hyperfootnotes=true|
% must be loaded, too. (Backreference{-}links without links do not make sense.)\\
% When the \xpackage{rotating} package (for |sidewaystable|s) is used,
% it must be loaded before the \xpackage{tablefootnote} package and it must be
% a recent version (v2.16a, 2009/03/28, or newer).\\
% If |\tablefootnote{...}| is used in a tabular environment, which is inside
% of a (sideways)table environment, everything is fine, but when the tabular
% environment is not inside of a (sideways)table, the \textbf{table}footnote
% will not work.\\
%
% Pages with tables with footnotes, which are rotated with the \xpackage{lscape}-
% or with the \xpackage{pdflscape}-package (|\begin{landscape}|, table, |\end{landscape}|),
% are possible. But when the \xpackage{footmisc}-package is used, \xpackage{(pdf)lscape}
% should be loaded before \xpackage{footmisc}.\\
%
% When the \xpackage{footmisc}-package is used with option \texttt{para},
% this is regarded with the exception of a \texttt{sidewaystable}. There the footnotes
% are printed one below the other. (Well, this is better than no footnote in a
% \texttt{sidewaystable} at all, isn't it?)\\
%
% When the (sideways)table floats, the footnotes are set and (when \xpackage{hyperref}
% is used) hyperlinked, but they are not automatically adapted when the table floats
% over/under another footnote. Thus either do not use a footnote between original and
% float position of the (sideways)table, or place the (sideways)table in
% \textquotedblleft here\textquotedblright\ position. |\clear(double)page|, |h(!)|,
% |H|~from the \xpackage{float} package (\url{http://www.ctan.org/pkg/float}),
% or |\FloatBarrier| from the \xpackage{picins} package
% (\url{http://www.ctan.org/pkg/picins}) might help, too.
% (Or after finishing the document, move the (sideways)table in the source code
% near the position where it floats to or use the optional footnote marks.)
%
% \pagebreak
%
% \section{Alternatives\label{sec:Alternatives}}
%
% \begin{description}
% \item[-] The \xpackage{longtable} package provides the \texttt{longtable} environment
%   as replacement for the combined \texttt{table} and \texttt{tabular} environments.
%   Footnotes are real footnotes (not just tablenotes), are continuously numbered and
%   hyperlinked (when using the \xpackage{hyperref} package), and the hyperlinks really
%   work. As drawback the appearance of the caption changes slightly (e.\,g.~distance
%   to the table, width of the caption), which can probably be changed back manually.
%   Furthermore, longtables are meaned to break over more than one page. If that is not
%   wished, it must be prevented by |\nopagebreak|-commands and by ending the longtable
%   lines with |\\*| instead of |\\|. longtables do not float. (Therefore using the
%   \xpackage{tablefootnote} package and |\FloatBarrier| from the \xpackage{picins} package
%   before and after the table environment is similar~- but \xpackage{tablefootnote} does not
%   change the table-caption!) \texttt{sidewaystable} does not work with it.\\
%   \url{http://www.ctan.org/pkg/longtable}
%
% \item[-] The \xpackage{supertabular} package provides the \texttt{mpsupertabular}
%   environment as replacement for the combined \texttt{table} and \texttt{tabular}
%   environments. Footnotes are just tablenotes (with working hyperlinks when using the
%   \xpackage{hyperref} package), i.\,e. numbered a, b, c and placed below the table
%   and not at the end of the page. Therefore there is no float problem (because the
%   tablenotes numbering is not included in the continuous numbering of the footnotes).
%   Placing the \texttt{supertabular} inside of a \texttt{sidewaystable} breaks the
%   hyperlinks to the tablenotes.\\
%   \url{http://www.ctan.org/pkg/supertabular}
%
% \item[-] The \xpackage{ctable} package has its very own notation for defining tables.
%   It can create tablenotes and sideways-tables. The tablenotes are not automatically
%   hyperlinked. The \texttt{ctable}s float. Because the tablenotes numbering is not
%   included in the continuous numbering of the footnotes there is no float problem.\\
%   \url{http://www.ctan.org/pkg/ctable}
%
% \item[-] The \xpackage{footnote} package provides |\makesavenoteenv{table}|. After
%   loading the package and using that command in the preamble, in tables |\footnote{...}|
%   can be used. Using |\makesavenoteenv{tabular}| and \\
%   |\makesavenoteenv{sidewaystable}|
%   is possible, but it neither solves the float problem, nor do the created hyperlinks
%   work (i.\,e.~they aim at wrong locations). The \xpackage{mdwtab} from the same bundle
%   is incompatible with other table-related packages (e.\,g. \xpackage{supertabular},
%   \xpackage{array}) and not $100\,\%$ compatible with the \texttt{tabular} environment.\\
%   \url{http://www.ctan.org/pkg/footnote}\\
%   \url{http://www.ctan.org/pkg/mdwtab}
%
% \item[-] The \xpackage{tabularx} package does produce footnotes for sidewaystables,
%   but uses a, b, c instead of 1, 2, 3. The hyperlinks to the footnotes do not work.
%   Because the footnotes numbering is not included in the continuous numbering of
%   the other footnotes there is no float problem.\\
%   \url{http://www.ctan.org/pkg/tabularx}
%
% \item[-] Placing a tabular inside a minipage inside a table produces tablenotes.
%   Therefore there is no float problem (because the footnotes are not continuously
%   numbered). The hyperlinks to the table notes indeed work.
%
% \pagebreak
%
% \item[-] The \xpackage{threeparttable} package creates tablenotes again. Therefore
%   there is no float problem (because the tablenotes are not continuously numbered with
%   the footnotes). There are no hyperlinks to the table notes (at least not automatically).
%   Using |sidewaystable| (with table notes) works.\\
%   \url{http://www.ctan.org/pkg/threeparttable}
%
% \item[-] The \xpackage{threeparttablex} package creates tablenotes again. Therefore there
%   is no float problem (because the tablenotes are not continuously numbered with the
%   footnotes). With option \texttt{referable} the tablenotes are hyperlinked. Use of
%   a |sidewaystable| (with table notes) did not work for me. When using the
%   \texttt{referable} option according to the example in the \xpackage{threeparttablex}
%   manual the \xpackage{longtable} package is used, therefore that package could be used
%   directly without \xpackage{threeparttablex} (see above).\\
%   \url{http://www.ctan.org/pkg/threeparttablex}
%
% \item[-] One can manually use |\footnotemark| in the table and |\footnotetext{...}|
%  after the table environment ended and manually change the |footnote| and
%  (when the \xpackage{hyperref} package is used) |Hfootnote| counters and needs
%  to insert
%  \begin{verbatim}
%     \makeatletter
%     \global\let\Hy@saved@currentHref\@currentHref
%     \hyper@makecurrent{Hfootnote}%
%     \global\let\Hy@footnote@currentHref\@currentHref
%     \global\let\@currentHref\Hy@saved@currentHref
%     \makeatother
%  \end{verbatim}
%  before each |\footnotetext{...}|, but with more than very few footnotes this can become
%  much work (and is prone to produce errors). And this does not even work with
%  sidewaystables. (And it does not work with the \xpackage{footnotebackref} package
%  without redefining |\footnotemark| and |\footnotetext{...}|.)
% \end{description}
%
% \bigskip
%
% \noindent (You programmed or found another alternative,
%  which is available at \CTAN{}?\\
%  OK, send an e-mail to me with the name, location at \CTAN{},
%  and a short notice, and I will probably include it in
%  the list above.)
%
% \newpage
%
% \section{Example}
%
%    \begin{macrocode}
%<*example>
\documentclass[british]{article}[2007/10/19]% v1.4h
%%%%%%%%%%%%%%%%%%%%%%%%%%%%%%%%%%%%%%%%%%%%%%%%%%%%%%%%%%%%%%%%%%%%%
\usepackage{float}[2001/11/08]%    v1.3d
\usepackage{placeins}[2005/04/18]% v2.2  ; for \FloatBarrier
\usepackage{rotating}[2009/03/28]% v2.16a; for sidewaystable-environment
\usepackage[%
 hyperfootnotes=true,%
 extension=pdf,%
 plainpages=false,%
 pdfpagelabels=true,%
 hyperindex=false,%
 pdflang={en},%
 pdftitle={tablefootnote package example},%
 pdfauthor={H.-Martin Muench},%
 pdfsubject={Example for the tablefootnote package},%
 pdfkeywords={LaTeX, tablefootnote, footnote, table, H.-Martin Muench},%
 % pdfview=FitH and FitBH do not work: hyperlinks in sidewaystables
 % do no lead to the footnotes, due to a bug in pdfTeX,
 % computing wrong anchor coordinates (Heiko Oberdiek, 29. October 2011)
 % pdfview=Fit, FitV, FitR, FitB, FitBV work
 % print is OK for all those options
 pdfstartview=FitH,%
 pdfpagelayout=OneColumn%
]{hyperref}[2012/11/06]% v6.83m
 % Due to the urls used in the example, either the hyperref or the url
 % package are needed (or the urls must be removed before compiling).

\usepackage{footnotebackref}[2012/07/01]% v1.0
\usepackage{tablefootnote}[2014/01/26]%   v1.1c

\gdef\unit#1{\mathord{\thinspace\mathrm{#1}}}%
\listfiles
\begin{document}
\pagenumbering{arabic}
\section*{Example for tablefootnote}

This example demonstrates the use of package\newline
\textsf{tablefootnote}, v1.1c as of 2014/01/26 (HMM).\newline
There were no options used. (The package provides no options.)\newline

\textbf{The \texttt{tablefootnote-example.tex} needs to be compiled
at least twice to get the references right!}\newline

If the etoolbox-package is found, it is automatically used.\newline

For more details please see the documentation!\newline

\noindent Save per page about $200\unit{ml}$ water,
$2\unit{g}$ CO$_{2}$ and $2\unit{g}$ wood:\newline
Therefore please print only if this is really necessary.\newline

Here is some text.\footnote{And this is a text footnote.}\newline

Tables \ref{tab.symbol}, \ref{tab.normal}, \ref{tab.another} and
\ref{tab.floatH} show normal tables, table~\ref{tab.sideways} depicts
a sidewaystable. Table~\ref{tab.floatH} uses the float
specifier~\texttt{H} from the float package.\newline

\texttt{Hyperref} option \verb|pdfview=FitH| and \verb|FitBH| do
not work due to a bug in pdf\TeX{}, computing wrong
anchor coordinates (\textsc{Heiko Oberdiek}, 29.~October 2011).
Depending on used pdf-viewer, hyperlinks in sidewaystables lead
e.\,g.~at the end of the document, not at the footnote.
\verb|pdfview=Fit|, \verb|FitV|, \verb|FitR|, \verb|FitB|,
\verb|FitBV| work, print is OK for all those options.

\bigskip

\listoftables

\pagebreak

\renewcommand{\thefootnote}{\fnsymbol{footnote}}
\verb|\renewcommand{\thefootnote}{\fnsymbol{footnote}}|
causes footnote{-}symbol{-}footnotes, which are possible
(see Table~\ref{tab.symbol}).

\begin{table}
\centering
\begin{tabular}{ccc}
Another\tablefootnote{A table footnote.} & %
text\tablefootnote{Another table footnote.} & %
in a table\tablefootnote{A third table footnote.}
\end{tabular}
\caption[A footnotesymbol table]{%
A table with footnote-symbol-footnotes.\label{tab.symbol}}
\end{table}

Some text.\footnote{A text footnote.}

\renewcommand{\thefootnote}{\arabic{footnote}}
\verb|\renewcommand{\thefootnote}{\arabic{footnote}}|
switches back to normal footnote numbers again.

\pagebreak

\begin{table}
\centering
\begin{tabular}{ccc}
Some\tablefootnote{A table footnote.} & %
text\tablefootnote[99]{A table footnote with custom footnote number.} & %
in a table\tablefootnote{A third table footnote.}
\end{tabular}
\caption[A table]{A normal table.\label{tab.normal}}
\end{table}

Some text.\footnote{Another text footnote.}

\pagebreak

%    \end{macrocode}
%
% \pagebreak
%
%    \begin{macrocode}
More text.\footnote{And yet another text footnote.}

\begin{table}[t]
\centering
\begin{tabular}{|c|c|c|}
\hline
Another\tablefootnote{A $3^{rd}$ table footnote.} & %
text\tablefootnote{Another $3^{rd}$ table footnote.} & %
in a table\tablefootnote{A $3^{rd}$ third table footnote.}\\ \hline
\end{tabular}
\caption[Another table]{Another table (third one)\label{tab.another}}
\end{table}

Please note that Table~\ref{tab.another} floated to the top of the page.
While the footnotes are set and (when hyperref is used) hyperlinked,
they are not automatically adapted. Thus either do not use a footnote
at the same page before the table, or place the table in
\textquotedblleft here\textquotedblright\ or
\textquotedblleft bottom\textquotedblright\ position.
\verb|\clear(double)page|, \verb|h(!)|, \verb|H|~from the
\texttt{float} package, or \verb|\FloatBarrier| from the
\texttt{picins} package might help, too. (Or move the table in the
source code near the position where it floats to
or use the optional footnote marks.)

Table~\ref{tab.floatH} (page~\pageref{tab.floatH}) uses float specifier
\texttt{H} from the float package and does not float.

Some text.\footnote{This is just another text footnote.}

\pagebreak

\FloatBarrier

\begin{sidewaystable}
\centering%
\begin{tabular}{ccc}
Text\tablefootnote{Please rotate the view for testing the %
hyperlinks.} & %
in a\tablefootnote[98]{Another sidewaystable footnote %
with optional footnote mark.} & %
sidewaystable%
\tablefootnote{Sidewaystable-environment provided by %
\url{http://www.ctan.org/pkg/rotating} package.}
\end{tabular}
\caption[A sideways table]{A table in the \texttt{sideways} %
environment\label{tab.sideways}}%
\end{sidewaystable}%

\FloatBarrier

\pagebreak

%    \end{macrocode}
%
% \pagebreak
%
%    \begin{macrocode}
A last table, here with float specifier \texttt{H} from the
float\footnote{\url{http://www.ctan.org/pkg/float}} package.

\begin{table}[H]
\centering
\begin{tabular}{ccc}
Another\tablefootnote{A $5^{th}$ table footnote.} & %
text\tablefootnote{Another $5^{th}$ table footnote.} & %
in a table\tablefootnote{A $5^{th}$ third table footnote.}
\end{tabular}
\caption[A last table]{A very last table\label{tab.floatH}}
\end{table}

Some text.\footnote{This is just another text footnote.}

\pagebreak

End of the example for the
tablefootnote\footnote{\url{http://www.ctan.org/pkg/tablefootnote}}
package.

\end{document}
%</example>
%    \end{macrocode}
%
% \StopEventually{}
%
% \newpage
%
% \section{The implementation}
%
% We start off by checking that we are loading into \LaTeXe{} and
% announcing the name and version of this package.
%
%    \begin{macrocode}
%<*package>
%    \end{macrocode}
%
%    \begin{macrocode}
\NeedsTeXFormat{LaTeX2e}[2011/06/27]
\ProvidesPackage{tablefootnote}[2014/01/26 v1.1c
              Table foot notes (HMM)]
%    \end{macrocode}
%
% A short description of the \xpackage{tablefootnote} package:
%
%    \begin{macrocode}
%% Provides the \tablefootnote[...]{...} command
%% for footnotes in (sideways)tables.

%    \end{macrocode}
%
% We need the \xpackage{ltxcmds} and \xpackage{letltxmacro} packages
% by \textsc{Heiko Oberdiek}:
%
%    \begin{macrocode}
\RequirePackage{ltxcmds}[2011/11/09]%     v1.22
\RequirePackage{letltxmacro}[2010/09/02]% v1.4
\RequirePackage{xifthen}[2009/04/17]%     v1.3
%    \end{macrocode}
%
% A last information for the user:
%
%    \begin{macrocode}
%% tablefootnote may work with earlier versions of LaTeX and those
%% packages, but this was not tested. Please consider updating
%% your LaTeX and packages to the most recent version
%% (if they are not already the most recent version).

%    \end{macrocode}
%
% \indent See subsection~\ref{ss:Downloads} about how to get them.\\
%
% When the \xpackage{rotating} package (for |sidewaystable|s) is used,
% it must be loaded before the \xpackage{tablefootnote} package and it must be
% a recent version:
%
%    \begin{macrocode}
\@ifpackageloaded{rotating}{%
  \gdef\tfn@rotating{1}
  \@ifpackagelater{rotating}{2009/03/28}% v2.16a
   {% >= 2009/03/28, OK
   }{% else
     \PackageWarningNoLine{tablefootnote}{%
       It is required version\MessageBreak%
       2009/03/28 v2.16a (or later) of package rotating,\MessageBreak%
       but only version\MessageBreak%
      `\csname ver@rotating.sty\endcsname'\MessageBreak%
      is available}%
   }%
 }{% else
   \gdef\tfn@rotating{0}%
 }

%    \end{macrocode}
%
% There are no options to be processed.\\
%
% We need some definitions:
%
%    \begin{macrocode}
\def\tfn@footnotetablecount{0}
\def\tfn@footnotetableprint{0}
\def\tfn@footnotezero{0}
\def\tfn@fnt{0}
\def\tfn@swt{0}%       _s_ide_w_ays_t_able
\def\tfn@fmpw{\relax}% f_oot_m_isc_p_ara_w_arn-switch
\def\tfn@fnbr{0}%      f_oot_n_ote_b_ack_r_ef(erence package)

%    \end{macrocode}
%
% \begin{macro}{\tfn@hyperfootnotes}
% We need a wrapper for the used |\ifHy@hyperfootnotes|,
% which is not defined when the \xpackage{hyperref} package has not been loaded:
%
%    \begin{macrocode}
\DeclareRobustCommand{\tfn@hyperfootnotes}[2]{%
  \ifHy@hyperfootnotes%
    #1%
  \else%
    #2%
  \fi%
  }

%    \end{macrocode}
% \end{macro}
%
% \begin{macro}{\tablemakefntext}
% |\tablemakefntext| makes the footnotetext for a footnote in a table.
%
%    \begin{macrocode}
\ltx@ifpackageloaded{footnotebackref}{%
  \gdef\tfn@fnbr{1}%
  \ifx\FootnoteBackref@symbol\empty%
    \ifFootnoteBackref@numberlinked%
      \newcommand\tablemakefntext[1]{%
        \noindent\makebox[1.634em][r]{%
         \mbox{\textsuperscript{\normalfont%
           \hyperlink{tfn:m\theHfootnote}{\@thefnmark}%
        }}}%
        \ #1%
      }
    \else
      \newcommand\tablemakefntext[1]{%
        \noindent\makebox[1.8em][r]{%
          \mbox{\textsuperscript{\normalfont%
            \@thefnmark%
        }}}%
        #1%
      }
    \fi
  \else
    \newcommand\tablemakefntext[1]{%
      \noindent\makebox[1.8em][r]{%
        \mbox{\textsuperscript{\normalfont%
          \ifFootnoteBackref@numberlinked%
            \hyperlink{tfn:m\theHfootnote}{\@thefnmark}\ %
          \else%
            \@thefnmark%
          \fi%
      }}}%
      \hyperlink{tfn:m\theHfootnote}{\FootnoteBackref@symbol}%
      #1%
    }
  \fi
}{%
  \newcommand\tablemakefntext[1]{%
    \noindent\makebox[1.8em][r]{%
      \mbox{\textsuperscript{\normalfont%
        \@thefnmark%
    }}}%
    #1%
  }
}

%    \end{macrocode}
%
% \end{macro}
%
% \pagebreak
%
% \begin{macro}{\swtablemakefntext}
%    \begin{macrocode}
\newcommand\swtablemakefntext[1]{\noindent\ #1}

%    \end{macrocode}
% \end{macro}
%
% \begin{macro}{\tfn@footnotemarks}
% |\tfn@footnotemarks| uses the |\footnotemark| command to place a foot note mark
% and decreases the |\tfn@footnotetablecount| (i.\,e.~increases the negative value)
% of the number of footnote(marks) in the table. If the \xpackage{footnotebackref}
% package is used, it is also necessary to create an according |\hypertarget| for
% the back reference.
%
%    \begin{macrocode}
\newcommand{\tfn@footnotemarks}[1][]{%
  \hbox{%
    \ifthenelse{\isempty{#1}}%
      {\footnotemark%
       \ltx@ifpackageloaded{footnotebackref}%
         {\raisebox{\ht\strutbox}{\hypertarget{tfn:m\theHfootnote}{}}}%
         {% footnotebackref not loaded
         }%
      }%
      {\ltx@ifpackageloaded{hyperref}%
        {\tfn@hyperfootnotes{%
            \hyperlink{tfn:#1}{\footnotemark[#1]}%
          }{\footnotemark[#1]}}%
        {\footnotemark[#1]}%
       \ltx@ifpackageloaded{footnotebackref}%
        {\raisebox{\ht\strutbox}{\hypertarget{tfn:m#1}{}}}{}%
      }%
   }%
  \@tempcnta=\tfn@footnotetablecount\relax%
  \advance\@tempcnta by -1\relax%
  \xdef\tfn@footnotetablecount{\the\@tempcnta}%
}

%    \end{macrocode}
% \end{macro}
%
% \begin{macro}{\tfn@footmiscparawarn}
% When the \xpackage{footmisc} package is used with option |para|,
% we need to issue a warning. (Any idea how to resolve this issue?)\\
% It must be given inside of it's own command, because |\ifFN@para| is not
% defined when \xpackage{footmisc} has not been loaded, and then the
% |\fi| would be a |\fi| without |\if| (and \TeX{} does not like single
% |\fi|s).
%
%    \begin{macrocode}
\newcommand{\tfn@footmiscparawarn}{%
  \ifFN@para%
    \PackageWarning{tablefootnote}{%
      Package footmisc with option para used.\MessageBreak%
      Unfortunately package tablefootnote is not yet\MessageBreak%
      able to regard this in sidewaystables.\MessageBreak%
      Here the footnotes are printed one below\MessageBreak%
      the other. Sorry!\MessageBreak%
     }%
  \fi%
  }

%    \end{macrocode}
% \end{macro}
%
% \begin{macro}{\tfn@updateHyper}
% |\tfn@updateHyper| updates some \xpackage{hyperref} internals.
%
%    \begin{macrocode}
\newcommand{\tfn@updateHyper}{%
  % The code of this command is partly from the
  % hyperref package by Heiko Oberdiek, 2011/10/01, v6.82j.
  % Newer versions of that package are available at CTAN.org.
  \global\let\Hy@saved@currentHref\@currentHref%
  \tfn@hyperfootnotes{\hyper@makecurrent{Hfootnote}}{}%
  \global\let\Hy@footnote@currentHref\@currentHref%
  \global\let\@currentHref\Hy@saved@currentHref%
  }

%    \end{macrocode}
% \end{macro}
%
% \begin{macro}{\tfn@footnotetext}
% |\tfn@footnotetext| contains the |\footnotetext[...]{...}| command and handles
% the numbering. |\tfn@footnotetext| will be executed after the table.
% The foot note counter has been increased automatically with each |\footnotemark|,
% therefore it must be reset to its value before the table began.
%
%    \begin{macrocode}
\newcommand{\tfn@footnotetext}[2][]{%
  \ifx\tfn@footnotetableprint\tfn@footnotezero%
    % This is the first footnotetext after the (sideways)table.
    \addtocounter{footnote}{\tfn@footnotetablecount}%
%    \end{macrocode}
%
% When the \xpackage{hyperref} package is used, also its |Hfootnote| counter must
% be reset (when the footnotes are hyperlinked).
%
%    \begin{macrocode}
    \ltx@ifpackageloaded{hyperref}{%
      \tfn@hyperfootnotes{%
        \addtocounter{Hfootnote}{\tfn@footnotetablecount}}{}%
      \tfn@updateHyper%
     }{}%
  \fi%
%    \end{macrocode}
%
% For issuing the text for the next footnote the footnote-counter
% (and |Hfootnote| counter in case of links) are increased by one.
%
%    \begin{macrocode}
  \addtocounter{footnote}{+1}%
  \ltx@ifpackageloaded{hyperref}{%
    \tfn@hyperfootnotes{\addtocounter{Hfootnote}{+1}}{}%
%    \end{macrocode}
%
% The \xpackage{hyperref} package was not expecting this kind of change of the
% |(H)footnote| numbers, therefore some code-lines are missing here which can be
% found at other places in that same package and are called by this command.
%
%    \begin{macrocode}
    \tfn@updateHyper%
%    \end{macrocode}
%
% When \xpackage{hyperref} is not loaded, nothing is to be done here.
%
%    \begin{macrocode}
   }{}%
%    \end{macrocode}
%
% The |\footnotetext[...]{...}| places the foot note text as usual.
%
%    \begin{macrocode}
  \ifx\tfn@swt\tfn@footnotezero%
    \ifthenelse{\isempty{#1}}%
      {\footnotetext{#2}}%
      {\ltx@ifpackageloaded{hyperref}%
         {\tfn@hyperfootnotes{%
             \footnotetext[#1]{%
               \raisebox{\ht\strutbox}{\hypertarget{tfn:#1}{}}#2}%
           }{\footnotetext[#1]{#2}%
           }%
         }{\footnotetext[#1]{#2}}%
      }%
  \else%
%    \end{macrocode}
%
% If there is more than one tablefootnote in the |sidewaystable|, \\
% the |\tfn@footmiscparawarn| shall be given,
%
%    \begin{macrocode}
    \ifx\tfn@footnotetableprint\tfn@footnotezero%
      \gdef\tfn@fmpw{\tfn@footmiscparawarn}%
    \else%
      \ltx@ifpackageloaded{footmisc}{\tfn@fmpw}{\relax}%
%    \end{macrocode}
%
% but we do not want to repeat the message for a third(, forth,\ldots)
% tablefootnote in the |sidewaystable|.
%
%    \begin{macrocode}
      \gdef\tfn@fmpw{\relax}%
    \fi%
%    \end{macrocode}
%
% |\@makefnmark| is \\
% |\hbox {\@textsuperscript {\normalfont \@thefnmark }}|, \\
% but according to |\show\@thefnmark| here |\@thefnmark| is
% |{\protect \itshape}| instead of |\thefootnote| (Why?!).\\
% If the \xpackage{footnotebackref} package is used,
% back references must be created for the table footnotes, too.
%
%    \begin{macrocode}
    \ltx@ifpackageloaded{footnotebackref}{%
      \LetLtxMacro{\tfncurrentmft}{\@makefntext}%
      \LetLtxMacro{\@makefntext}{\swtablemakefntext}%
      \ifthenelse{\isempty{#1}}%
        {\footnotetext{%
           \raisebox{\ht\strutbox}{\hypertarget{tfn:\theHfootnote}{}}%
           \hbox{\@textsuperscript{\normalfont%
              \hyperlink{tfn:m\theHfootnote}{\thefootnote}}}\thinspace%
         \ifx\FootnoteBackref@symbol\empty%
         \else\hyperlink{tfn:m\theHfootnote}{\FootnoteBackref@symbol}%
         \fi%
         #2}%
       }{\edef\tfn@tfn{\arabic{footnote}}%
         \edef\tfn@thn{\arabic{Hfootnote}}%
         \setcounter{footnote}{#1}%
         \setcounter{Hfootnote}{#1}%
         \tfn@updateHyper%
         \footnotetext{%
           \raisebox{\ht\strutbox}{\hypertarget{tfn:#1}{}}%
           \hbox{\@textsuperscript{\normalfont%
                  \hyperlink{tfn:m#1}{#1}}}%
           \thinspace%
           \ifx\FootnoteBackref@symbol\empty%
           \else%
             \raisebox{\ht\strutbox}{%
               \hyperlink{tfn:m#1}{\FootnoteBackref@symbol}}%
           \fi%
           #2%
          }%
         \setcounter{footnote}{\tfn@tfn}%
         \setcounter{Hfootnote}{\tfn@thn}%
         \tfn@updateHyper%
       }%
      \LetLtxMacro{\@makefntext}{\tfncurrentmft}%
     }{% footnotebackref not loaded
       \ifthenelse{\isempty{#1}}%
         {\footnotetext{%
            \hbox{\@textsuperscript{\normalfont\thefootnote}}%
            \ltx@ifpackageloaded{hyperref}%
              {\tfn@hyperfootnotes{%
                 \raisebox{\ht\strutbox}{%
                   \hypertarget{tfn:\theHfootnote}{}}}{}%
              }{}%
            \thinspace%
            #2}%
        }{%
%    \end{macrocode}
%
% \pagebreak
%
%    \begin{macrocode}
          \edef\tfn@tfn{\arabic{footnote}}%
          \ltx@ifpackageloaded{hyperref}{%
            \tfn@hyperfootnotes{\xdef\tfn@thn{\arabic{Hfootnote}}}{}%
           }{}%
          \setcounter{footnote}{#1}
          \ltx@ifpackageloaded{hyperref}{%
            \tfn@hyperfootnotes{\setcounter{Hfootnote}{#1}}{}%
            \tfn@updateHyper%
           }{}%
          \footnotetext{%
            \hbox{\@textsuperscript{\normalfont #1}}%
            \ltx@ifpackageloaded{hyperref}{%
              \tfn@hyperfootnotes{\raisebox{\ht\strutbox}{%
                                         \hypertarget{tfn:#1}{}}}{}%
             }{}%
            \thinspace%
            #2}%
          \setcounter{footnote}{\tfn@tfn}%
          \ltx@ifpackageloaded{hyperref}{%
            \tfn@hyperfootnotes{\setcounter{Hfootnote}{\tfn@thn}}{}%
            \tfn@updateHyper%
           }{}%
        }%
     }%
  \fi%
%    \end{macrocode}
%
% For further footnotes in the table the \\
% |\ifx\tfn@footnotetableprint\tfn@footnotezero|\ldots|\fi| \\
% from the beginning of this |\tfn@footnotetext| macro must not be repeated,
% therefore we deactivate this by defining |\tfn@footnotetableprint| to
% something other than |0| (here |1| is used).
%
%    \begin{macrocode}
  \gdef\tfn@footnotetableprint{1}%
%    \end{macrocode}
%
% |\tfn@footnotetablecount| contains the negative number of footnote texts
% to be printed, therefore it is increase by one (i.\,e.~the negative number
% decreases, i.\,e.~$-3$ becomes~$-2$).
%
%    \begin{macrocode}
  \@tempcnta=\tfn@footnotetablecount\relax%
  \advance\@tempcnta by +1\relax%
  \xdef\tfn@footnotetablecount{\the\@tempcnta}%
%    \end{macrocode}
%
% When it is zero, all footnote texts have been placed, and\\
% |\tfn@footnotetableprint| can be reset to zero and\\
% |\tfn@tablefootnoteprintout| can be cleared.
%
%    \begin{macrocode}
  \ifx\tfn@footnotetablecount\tfn@footnotezero% all footnotes placed
    \gdef\tfn@footnotetableprint{0}%
    \global\let\tfn@tablefootnoteprintout\relax% brute force
 %\else there is at least one \footnotemark which needs \footnotetext
  \fi
}

%    \end{macrocode}
% \end{macro}
%
% \begin{macro}{\tfn@tablefootnoteprintout}
% When something shall be cleared or redefined, it is a good idea
% to define it. It~will be later filled with the foot note text to be printed.
%
%    \begin{macrocode}
\newcommand{\tfn@tablefootnoteprintout}{\relax}

%    \end{macrocode}
% \end{macro}
%
% \pagebreak
%
% \begin{macro}{\tablefootnote}
% |\tablefootnote[...]{...}| is the command to be used by the user instead of
% |\footnote[...]{...}| in the (sideways)table. It places a foot note mark via
% |\tfn@footnotemarks| and appends a |\tfn@footnotetext{#1}| to the\\
% |\tfn@tablefootnoteprintout| macro, and remembers that table foot notes
% have been used.
%
%    \begin{macrocode}
\newcommand{\tablefootnote}[2][]{%
  \tfn@footnotemarks[#1]%
  \ifthenelse{\isempty{#1}}{\relax}{%
    \ltx@GlobalPrependToMacro{\tfn@tablefootnoteprintout}{%
      \addtocounter{footnote}{+1}%
      \ltx@ifpackageloaded{hyperref}{%
        \tfn@hyperfootnotes{\addtocounter{Hfootnote}{+1}}{}%
        \tfn@updateHyper%
       }{}%
     }%
   }%
  \ltx@GlobalAppendToMacro{\tfn@tablefootnoteprintout}{%
    \tfn@footnotetext[#1]{#2}%
    \ifthenelse{\isempty{#1}}{\relax}{%
      \addtocounter{footnote}{-1}%
      \ltx@ifpackageloaded{hyperref}{%
        \tfn@hyperfootnotes{\addtocounter{Hfootnote}{-1}}{}%
        \tfn@updateHyper%
       }{}%
     }%
   }%
  \gdef\tfn@fnt{1}%
  }

%    \end{macrocode}
% \end{macro}
%
% \begin{macro}{\tfn@swtbox}
% For |sidewaystable|s the |\tfn@tablefootnoteprintout| is placed in a \\
% |\parbox|, which can be placed at the end of the |sidewaystable|:
%
%    \begin{macrocode}
\newcommand{\tfn@swtbox}{%
\ifvoid \footins%
\vskip 2\baselineskip%
\else%
\vskip \skip\footins%
\fi%
\parbox[b][][t]{\linewidth}{\raggedright\tfn@tablefootnoteprintout}%
}

%    \end{macrocode}
% \end{macro}
%
% |\tfn@tablefootnoteprintout| shall be executed after the end of the table environment.
% The \xpackage{etoolbox} package provides very easy syntax for this:
%
%    \begin{macrocode}
\IfFileExists{etoolbox.sty}{%
  \RequirePackage{etoolbox}[2011/01/03]%   v2.1
  \AfterEndEnvironment{table}{%
    \ltx@ifpackageloaded{footnotebackref}{%
      \LetLtxMacro{\tfncurrentmft}{\@makefntext}%
      \LetLtxMacro{\@makefntext}{\tablemakefntext}%
       \tfn@tablefootnoteprintout%
      \LetLtxMacro{\@makefntext}{\tfncurrentmft}%
     }{% footnotebackref not loaded
      \tfn@tablefootnoteprintout%
     }%
      \gdef\tfn@fnt{0}%
   }%
%    \end{macrocode}
%
% When the \xpackage{rotating} package has been loaded, we need to detect
% whether the |\tablefootnote| is placed inside of a |sidewaystable|.
%
%    \begin{macrocode}
  \ltx@ifpackageloaded{rotating}{%
    \AtBeginEnvironment{sidewaystable}{%
      \gdef\tfn@swt{1}%
    }
%    \end{macrocode}
%
% At the end of |sidewaystable| environments |\tfn@swtbox| shall be placed
% (if~there was any |\tablefootnote| in that |sidewaystable| at all).
% And |\tfn@swt| as well as |\tfn@fnt| must be reset to |0|.
%
%    \begin{macrocode}
    \AtEndEnvironment{sidewaystable}{%
      \ifx\tfn@fnt\tfn@footnotezero% \relax
      \else%
        \ltx@ifpackageloaded{footnotebackref}{%
          \LetLtxMacro{\tfncurrentmft}{\@makefntext}%
          \LetLtxMacro{\@makefntext}{\tablemakefntext}%
           \tfn@swtbox%
          \LetLtxMacro{\@makefntext}{\tfncurrentmft}%
         }{% footnotebackref not loaded
          \tfn@swtbox%
          }%
      \fi%
      \gdef\tfn@swt{0}%
      \gdef\tfn@fnt{0}%
     }%
   }{}% else \relax
}{%
%    \end{macrocode}
%
% When the \xpackage{etoolbox} package is not available (e.\,g. because of missing
% \eTeX{}) we~do a workaround:
%
%    \begin{macrocode}
  \PackageWarning{tablefootnote}{%
    Package etoolbox not available, therefore\MessageBreak%
    substituting commands \string\AtEndEnvironment\space %
    and\MessageBreak%
    \string\AfterEndEnvironment\space for tablefootnotes%
    \MessageBreak%
   }%
  \@ifpackagelater{letltxmacro}{2010/09/02}%
   {% >= 2010/09/02, OK
    \GlobalLetLtxMacro{\tfnorigtable}{\table}
    \GlobalLetLtxMacro{\tfnendorigtable}{\endtable}
%    \end{macrocode}
%
% |\GlobalLetLtxMacro| was introduced in version 2010/09/02 v1.4 of the
% \xpackage{letltxmacro} package.
%
%    \begin{macrocode}
   }{% else
     \PackageError{tablefootnote}{%
       Outdated version of letltxmacro package used%
      }{Version 2010/09/02 v1.4 or newer needed but not found!%
        \MessageBreak%
        Update the used letltxmacro package.}
        \LetLtxMacro{\tfnorigtable}{\table}
        \LetLtxMacro{\tfnendorigtable}{\endtable}
    }%
%    \end{macrocode}
%
% \pagebreak
%
%    \begin{macrocode}
  \ltx@ifpackageloaded{float}{%
    \@ifpackagelater{float}{2001/11/08}%   v1.3d
     {% >= 2001/11/08, OK
     }{% else
       \PackageWarningNoLine{tablefootnote}{%
         It is required version\MessageBreak
         2001/11/08 v1.3d (or later) of package float,\MessageBreak
         but only version\MessageBreak
        `\csname ver@float.sty\endcsname'\MessageBreak
        is available}%
     }%
   }{\relax}
  \renewenvironment{table}{%
    \ltx@ifpackageloaded{float}{%
%    \end{macrocode}
%
% When \xpackage{etoolbox.sty} was not found, and the \xpackage{float} package
% has been loaded, and the float specifier is \texttt{H}, then |\endtable| was
% let to |\float@endH| by the \xpackage{float} package (line~79 of
% \xpackage{float.sty}, 2001/11/08 v1.3d:\\
% |\expandafter\let\csname end#1\endcsname\float@endH|), i.\,e.~appending\\
% |\tfn@tablefootnoteprintout \gdef\tfn@fnt{0}| to |\endtable| is useless.
% Instead of this, it must be appended to |\float@endH|.\\
% (Why is this not necessary when \xpackage{etoolbox} is used?
% That is a good question, answers are welcome.)
%
%    \begin{macrocode}
      \ltx@LocalAppendToMacro{\float@endH}{%
        \ltx@ifpackageloaded{footnotebackref}{%
          \LetLtxMacro{\tfncurrentmft}{\@makefntext}%
          \LetLtxMacro{\@makefntext}{\tablemakefntext}%
           \tfn@tablefootnoteprintout%
          \LetLtxMacro{\@makefntext}{\tfncurrentmft}%
         }{% footnotebackref not loaded
          \tfn@tablefootnoteprintout%
         }%
          \gdef\tfn@fnt{0}%
       }%
     }{\relax}
    \tfnorigtable%
   }{%
     \tfnendorigtable%
     \ltx@ifpackageloaded{footnotebackref}{%
       \LetLtxMacro{\tfncurrentmft}{\@makefntext}%
       \LetLtxMacro{\@makefntext}{\tablemakefntext}%
        \tfn@tablefootnoteprintout%
       \LetLtxMacro{\@makefntext}{\tfncurrentmft}%
      }{% footnotebackref not loaded
       \tfn@tablefootnoteprintout%
      }%
     \gdef\tfn@fnt{0}%
    }%
%    \end{macrocode}
%
% \pagebreak
%
% The |sidewaystable|-environment must be treated separately:
%
%    \begin{macrocode}
  \ltx@ifpackageloaded{rotating}{%
    \let\tfnorigsidewaystable\sidewaystable
    \let\tfnendorigsidewaystable\endsidewaystable
    \renewenvironment{sidewaystable}{%
      \gdef\tfn@swt{1}%
      \tfnorigsidewaystable%
    }{%
      \ifx\tfn@fnt\tfn@footnotezero% \relax
      \else%
        \ltx@ifpackageloaded{footnotebackref}{%
          \LetLtxMacro{\tfncurrentmft}{\@makefntext}%
          \LetLtxMacro{\@makefntext}{\tablemakefntext}%
           \tfn@swtbox%
          \LetLtxMacro{\@makefntext}{\tfncurrentmft}%
         }{% footnotebackref not loaded
          \tfn@swtbox%
         }%
      \fi%
      \gdef\tfn@swt{0}%
      \gdef\tfn@fnt{0}%
      \tfnendorigsidewaystable%
     }%
   }{}%
}

%    \end{macrocode}
%
% \DescribeMacro{\AtBeginDocument}
% When the \xpackage{rotating} (for |sidewaystable|s) or the
% \xpackage{footnotebackref} (for back references for the footnotes)
% package is used, it must be loaded before the \xpackage{tablefootnote} package.
% Using both packages is also possible, in which case both must be loaded before
% the \xpackage{tablefootnote} package. If any one of those packages is loaded
% |\AtBeginDocument| after this check, the according error message will not be
% given, but packages really should be loaded before |\AtBeginDocument|.
%
%    \begin{macrocode}
\AtBeginDocument{%
  \ltx@ifpackageloaded{footnotebackref}{%
    \ifx\tfn@fnbr\tfn@footnotezero%
      \PackageError{tablefootnote}{%
        Package footnotebackref loaded after tablefootnote}{%
        When using both the footnotebackref and the tablefootnote %
        package,\MessageBreak%
        the footnotebackref package MUST be loaded before the %
        tablefootnote package!\MessageBreak%
       }%
    \fi%
   }{% else \relax
   }
  \ltx@ifpackageloaded{rotating}{%
    \ifx\tfn@rotating\tfn@footnotezero%
      \PackageError{tablefootnote}{%
        Package rotating loaded after tablefootnote}{%
        When using both the rotating and the tablefootnote package,%
        \MessageBreak%
        the rotating package MUST be loaded before the tablefootnote%
        \space package!\MessageBreak%
       }%
    \fi%
   }{% else \relax
   }
%    \end{macrocode}
%
% \pagebreak
%
% When the document is compiled with \LuaLaTeX{}, hyperlinks in rotated content
% will be misplaced, regardless of the use of the \xpackage{tablefootnote} package
% (or of tables or footnotes). The text to be printed will be OK.
% This problem of \LuaLaTeX{} (not \xpackage{tablefootnote}) was found and reported
% by \textsc{Arno Trautmann}, 2011/11/15 -- thanks!
%
%    \begin{macrocode}
  \ltx@ifpackageloaded{hyperref}{%
    \tfn@hyperfootnotes{
      \ifluatex%
        \PackageWarningNoLine{tablefootnote}{%
          LuaTeX and hyperref package used:\MessageBreak%
          Hyperreferences in rotated content\MessageBreak%
          will be misplaced\MessageBreak%
          regardless of tablefootnote package.\MessageBreak%
          The text to be printed will be OK%
         }%
      \fi%
     }{% else \relax
     }%
   }{% else \relax
   }%
%    \end{macrocode}
%
% When the \xpackage{footnotebackref} package is used but the footnotes are not
% hyperlinked, neither the back references nor the tablefootnotes will work.
%
%    \begin{macrocode}
  \ltx@ifpackageloaded{footnotebackref}{%
     \ltx@ifpackageloaded{hyperref}{%
        \tfn@hyperfootnotes{\relax}{% else
          \PackageError{tablefootnote}{%
             footnotebackref but not hyperfootnotes}{%
             The footnotebackref package is used\MessageBreak%
             but the hyperfootnotes option of the hyperref package%
             \MessageBreak%
             was set to false.\MessageBreak%
             Either enable hyperfootnotes or do not use footnotebackref.%
             \MessageBreak%
             Loading of the tablefootnote package will be aborted.%
           }%
          \endinput%
      }%
    }{\PackageError{tablefootnote}{footnotebackref but not hyperref}{%
        The footnotebackref package is used\MessageBreak%
        but not the hyperref package.\MessageBreak%
        Either load hyperref or do not use footnotebackref.%
        \MessageBreak%
        Loading of the tablefootnote package will be aborted.%
        }%
      \endinput%
    }%
  }{}
}

%    \end{macrocode}
%
%    \begin{macrocode}
%</package>
%    \end{macrocode}
%
% \pagebreak
%
% \section{Installation}
%
% \subsection{Downloads\label{ss:Downloads}}
%
% Everything is available on \CTAN{}, \url{http://www.ctan.org/},
% but may need additional packages themselves.\\
%
% \DescribeMacro{tablefootnote.dtx}
% For unpacking the |tablefootnote.dtx| file and constructing the documentation
% it is required:
% \begin{description}
% \item[-] \TeX{} Format \LaTeXe{}: \url{http://www.CTAN.org/}
%
% \item[-] document class \xpackage{ltxdoc}, 2007/11/11, v2.0u,
%           \url{http://www.ctan.org/pkg/ltxdoc}
%
% \item[-] package \xpackage{holtxdoc}, 2012/03/21, v0.24,
%           \url{http://www.ctan.org/pkg/holtxdoc}
%
% \item[-] package \xpackage{hypdoc}, 2011/08/19, v1.11,
%           \url{http://www.ctan.org/pkg/hypdoc}
% \end{description}
%
% \DescribeMacro{tablefootnote.sty}
% The |tablefootnote.sty| for \LaTeXe{} (i.\,e.~each document using
% the \xpackage{tablefootnote} package) requires:
% \begin{description}
% \item[-] \TeX{} Format \LaTeXe{}, \url{http://www.CTAN.org/}
%
% \item[-] package \xpackage{letltxmacro}, 2010/09/02, v1.4,
%           \url{http://www.ctan.org/pkg/letltxmacro}.
%
% \item[-] package \xpackage{ltxcmds}, 2011/11/09, v1.22,
%           \url{http://www.ctan.org/pkg/ltxcmds}
% \end{description}
%
% When the \xpackage{hyperref} package is used, also the \xpackage{ifluatex} package
% is needed, but it is already loaded automatically by the \xpackage{hyperref}
% package.\\
% When the \xpackage{etoolbox} package is available, it is used:
% \begin{description}
% \item[-] package \xpackage{etoolbox}, 2011/01/03, v2.1,
%           \url{http://www.ctan.org/pkg/etoolbox}.
% \end{description}
%
% \DescribeMacro{tablefootnote-example.tex}
% The |tablefootnote-example.tex| requires the same files as all
% documents using the \xpackage{tablefootnote} package and additionally:
% \begin{description}
% \item[-] class \xpackage{article}, 2007/10/19, v1.4h, from \xpackage{classes.dtx}:\\
%           \CTAN{macros/latex/base/classes.dtx}
%
% \item[-] package \xpackage{rotating}, 2009/03/28, v2.16a,
%           \url{http://www.ctan.org/pkg/rotating}
%
% \item[-] package \xpackage{float}, 2001/11/08, v1.3d,
%           \url{http://www.ctan.org/pkg/float}
%
% \item[-] package \xpackage{placeins}, 2005/04/18, v2.2,
%           \url{http://www.ctan.org/pkg/placeins}\\
%           for |\FloatBarrier|
%
% \item[-] package \xpackage{hyperref}, 2012/11/06, v6.83m,
%           \url{http://www.ctan.org/pkg/hyperref}
%
% \item[-] package \xpackage{tablefootnote}, 2014/01/26, v1.1c,\\
%           \url{http://www.ctan.org/pkg/tablefootnote}\\
%   (Well, it is the example file for this package, and because you
%    are reading the documentation for the \xpackage{tablefootnote} package,
%    it can be assumed that you already have some version of it~--
%    is it the current one?)
% \end{description}
%
% \DescribeMacro{Oberdiek}
% \DescribeMacro{holtxdoc}
% \DescribeMacro{ifluatex}
% \DescribeMacro{letltxmacro}
% \DescribeMacro{ltxcmds}
% All packages of \textsc{Heiko Oberdiek}'s bundle `oberdiek'
% (especially \xpackage{holtxdoc}, \xpackage{ifluatex}, \xpackage{letltxmacro},
% and \xpackage{ltxcmds}) are also available in a TDS compliant ZIP archive:\\
% \CTAN{install/macros/latex/contrib/oberdiek.tds.zip}.\\
% It is probably best to download and use this, because the packages in there
% are quite probably both recent and compatible among themselves.\\
%
% \pagebreak
%
% \DescribeMacro{hyperref}
% \noindent \xpackage{hyperref} is not included in that bundle and needs to be
% downloaded separately,\\
% \url{http://www.ctan.org/install/macros/latex/contrib/hyperref.tds.zip}.\\
%
% \DescribeMacro{M\"{u}nch}
% A hyperlinked list of my (other) packages can be found at
% \url{http://www.ctan.org/author/muench-hm}.\\
%
% \subsection{Package, unpacking TDS}
% \paragraph{Package.} This package is available on \CTAN{}
% \begin{description}
% \item[\CTAN{macros/latex/contrib/tablefootnote/tablefootnote.dtx}]\hspace*{0.1cm} \\
%       The source file.
% \item[\CTAN{macros/latex/contrib/tablefootnote/tablefootnote.pdf}]\hspace*{0.1cm} \\
%       The documentation.
% \item[\CTAN{macros/latex/contrib/tablefootnote/tablefootnote-example.pdf}]\hspace*{0.1cm} \\
%       The compiled example file, as it should look like.
% \item[\CTAN{macros/latex/contrib/tablefootnote/README}]\hspace*{0.1cm} \\
%       The README file.
% \end{description}
% There is also a tablefootnote.tds.zip available:
% \begin{description}
% \item[\CTAN{install/macros/latex/contrib/tablefootnote.tds.zip}]\hspace*{0.1cm} \\
%       Everything in \xfile{TDS} compliant, compiled format.
% \end{description}
% which additionally contains\\
% \begin{tabular}{ll}
% tablefootnote.ins & The installation file.\\
% tablefootnote.drv & The driver to generate the documentation.\\
% tablefootnote.sty & The \xext{sty}le file.\\
% tablefootnote-example.tex & The example file.
% \end{tabular}
%
% \bigskip
%
% \noindent For required other packages, please see the preceding subsection.
%
% \paragraph{Unpacking.} The \xfile{.dtx} file is a self-extracting
% \docstrip{} archive. The files are extracted by running the
% \xext{dtx} through \plainTeX{}:
% \begin{quote}
%   \verb|tex tablefootnote.dtx|
% \end{quote}
%
% \noindent About generating the documentation see paragraph~\ref{GenDoc} below.\\
%
% \paragraph{TDS.} Now the different files must be moved into
% the different directories in your installation TDS tree
% (also known as \xfile{texmf} tree):
% \begin{quote}
% \def\t{^^A
% \begin{tabular}{@{}>{\ttfamily}l@{ $\rightarrow$ }>{\ttfamily}l@{}}
%   tablefootnote.sty & tex/latex/tablefootnote/tablefootnote.sty\\
%   tablefootnote.pdf & doc/latex/tablefootnote/tablefootnote.pdf\\
%   tablefootnote-example.tex & doc/latex/tablefootnote/tablefootnote-example.tex\\
%   tablefootnote-example.pdf & doc/latex/tablefootnote/tablefootnote-example.pdf\\
%   tablefootnote.dtx & source/latex/tablefootnote/tablefootnote.dtx\\
% \end{tabular}^^A
% }^^A
% \sbox0{\t}^^A
% \ifdim\wd0>\linewidth
%   \begingroup
%     \advance\linewidth by\leftmargin
%     \advance\linewidth by\rightmargin
%   \edef\x{\endgroup
%     \def\noexpand\lw{\the\linewidth}^^A
%   }\x
%   \def\lwbox{^^A
%     \leavevmode
%     \hbox to \linewidth{^^A
%       \kern-\leftmargin\relax
%       \hss
%       \usebox0
%       \hss
%       \kern-\rightmargin\relax
%     }^^A
%   }^^A
%   \ifdim\wd0>\lw
%     \sbox0{\small\t}^^A
%     \ifdim\wd0>\linewidth
%       \ifdim\wd0>\lw
%         \sbox0{\footnotesize\t}^^A
%         \ifdim\wd0>\linewidth
%           \ifdim\wd0>\lw
%             \sbox0{\scriptsize\t}^^A
%             \ifdim\wd0>\linewidth
%               \ifdim\wd0>\lw
%                 \sbox0{\tiny\t}^^A
%                 \ifdim\wd0>\linewidth
%                   \lwbox
%                 \else
%                   \usebox0
%                 \fi
%               \else
%                 \lwbox
%               \fi
%             \else
%               \usebox0
%             \fi
%           \else
%             \lwbox
%           \fi
%         \else
%           \usebox0
%         \fi
%       \else
%         \lwbox
%       \fi
%     \else
%       \usebox0
%     \fi
%   \else
%     \lwbox
%   \fi
% \else
%   \usebox0
% \fi
% \end{quote}
% If you have a \xfile{docstrip.cfg} that configures and enables \docstrip{}'s
% \xfile{TDS} installing feature, then some files can already be in the right
% place, see the documentation of \docstrip{}.
%
% \subsection{Refresh file name databases}
%
% If your \TeX{}~distribution (\TeX live, \mikTeX{}, \dots{}) relies on file name
% databases, you must refresh these. For example, \teTeX{} users run
% \verb|texhash| or \verb|mktexlsr|.
%
% \subsection{Some details for the interested}
%
% \paragraph{Unpacking with \LaTeX{}.}
% The \xfile{.dtx} chooses its action depending on the format:
% \begin{description}
% \item[\plainTeX:] Run \docstrip{} and extract the files.
% \item[\LaTeX:] Generate the documentation.
% \end{description}
% If you insist on using \LaTeX{} for \docstrip{} (really,
% \docstrip{} does not need \LaTeX{}), then inform the autodetect routine
% about your intention:
% \begin{quote}
%   \verb|latex \let\install=y% \iffalse meta-comment
%
% File: tablefootnote.dtx
% Version: 2014/01/26 v1.1c
%
% Copyright (C) 2011 - 2014 by
%    H.-Martin M"unch <Martin dot Muench at Uni-Bonn dot de>
%
% This work may be distributed and/or modified under the
% conditions of the LaTeX Project Public License, either
% version 1.3c of this license or (at your option) any later
% version. This version of this license is in
%    http://www.latex-project.org/lppl/lppl-1-3c.txt
% and the latest version of this license is in
%    http://www.latex-project.org/lppl.txt
% and version 1.3c or later is part of all distributions of
% LaTeX version 2005/12/01 or later.
%
% This work has the LPPL maintenance status "maintained".
%
% The Current Maintainer of this work is H.-Martin Muench.
%
% This work consists of the main source file tablefootnote.dtx,
% the README, and the derived files
%    tablefootnote.sty, tablefootnote.pdf,
%    tablefootnote.ins, tablefootnote.drv,
%    tablefootnote-example.tex, tablefootnote-example.pdf.
%
% Distribution:
%    http://mirror.ctan.org/macros/latex/contrib/tablefootnote/tablefootnote.dtx
%    http://mirror.ctan.org/macros/latex/contrib/tablefootnote/tablefootnote.pdf
%    http://mirror.ctan.org/install/macros/latex/contrib/tablefootnote.tds.zip
%
% See http://www.ctan.org/pkg/tablefootnote (Catalogue)
%
% Unpacking:
%    (a) If tablefootnote.ins is present:
%           tex tablefootnote.ins
%    (b) Without tablefootnote.ins:
%           tex tablefootnote.dtx
%    (c) If you insist on using LaTeX
%           latex \let\install=y\input{tablefootnote.dtx}
%        (quote the arguments according to the demands of your shell)
%
% Documentation:
%    (a) If tablefootnote.drv is present:
%           (pdf)latex tablefootnote.drv
%           makeindex -s gind.ist tablefootnote.idx
%           (pdf)latex tablefootnote.drv
%           makeindex -s gind.ist tablefootnote.idx
%           (pdf)latex tablefootnote.drv
%    (b) Without tablefootnote.drv:
%           (pdf)latex tablefootnote.dtx
%           makeindex -s gind.ist tablefootnote.idx
%           (pdf)latex tablefootnote.dtx
%           makeindex -s gind.ist tablefootnote.idx
%           (pdf)latex tablefootnote.dtx
%
%    The class ltxdoc loads the configuration file ltxdoc.cfg
%    if available. Here you can specify further options, e.g.
%    use DIN A4 as paper format:
%       \PassOptionsToClass{a4paper}{article}
%
% Installation:
%    TDS:tex/latex/tablefootnote/tablefootnote.sty
%    TDS:doc/latex/tablefootnote/tablefootnote.pdf
%    TDS:doc/latex/tablefootnote/tablefootnote-example.tex
%    TDS:doc/latex/tablefootnote/tablefootnote-example.pdf
%    TDS:source/latex/tablefootnote/tablefootnote.dtx
%
%<*ignore>
\begingroup
  \catcode123=1 %
  \catcode125=2 %
  \def\x{LaTeX2e}%
\expandafter\endgroup
\ifcase 0\ifx\install y1\fi\expandafter
         \ifx\csname processbatchFile\endcsname\relax\else1\fi
         \ifx\fmtname\x\else 1\fi\relax
\else\csname fi\endcsname
%</ignore>
%<*install>
\input docstrip.tex
\Msg{******************************************************************}
\Msg{* Installation                                                   *}
\Msg{* Package: tablefootnote 2014/01/26 v1.1c Table foot notes (HMM) *}
\Msg{******************************************************************}

\keepsilent
\askforoverwritefalse

\let\MetaPrefix\relax
\preamble

This is a generated file.

Project: tablefootnote
Version: 2014/01/26 v1.1c

Copyright (C) 2011 - 2014 by
    H.-Martin M"unch <Martin dot Muench at Uni-Bonn dot de>

The usual disclaimer applies:
If it doesn't work right that's your problem.
(Nevertheless, send an e-mail to the maintainer
 when you find an error in this package.)

This work may be distributed and/or modified under the
conditions of the LaTeX Project Public License, either
version 1.3c of this license or (at your option) any later
version. This version of this license is in
   http://www.latex-project.org/lppl/lppl-1-3c.txt
and the latest version of this license is in
   http://www.latex-project.org/lppl.txt
and version 1.3c or later is part of all distributions of
LaTeX version 2005/12/01 or later.

This work has the LPPL maintenance status "maintained".

The Current Maintainer of this work is H.-Martin Muench.

This work consists of the main source file tablefootnote.dtx,
the README, and the derived files
   tablefootnote.sty, tablefootnote.pdf,
   tablefootnote.ins, tablefootnote.drv,
   tablefootnote-example.tex, tablefootnote-example.pdf.

In memoriam Tommy Muench + 2014/01/02.

\endpreamble
\let\MetaPrefix\DoubleperCent

\generate{%
  \file{tablefootnote.ins}{\from{tablefootnote.dtx}{install}}%
  \file{tablefootnote.drv}{\from{tablefootnote.dtx}{driver}}%
  \usedir{tex/latex/tablefootnote}%
  \file{tablefootnote.sty}{\from{tablefootnote.dtx}{package}}%
  \usedir{doc/latex/tablefootnote}%
  \file{tablefootnote-example.tex}{\from{tablefootnote.dtx}{example}}%
}

\catcode32=13\relax% active space
\let =\space%
\Msg{************************************************************************}
\Msg{*}
\Msg{* To finish the installation you have to move the following}
\Msg{* file into a directory searched by TeX:}
\Msg{*}
\Msg{*     tablefootnote.sty}
\Msg{*}
\Msg{* To produce the documentation run the file `tablefootnote.drv'}
\Msg{* through (pdf)LaTeX, e.g.}
\Msg{*  pdflatex tablefootnote.drv}
\Msg{*  makeindex -s gind.ist tablefootnote.idx}
\Msg{*  pdflatex tablefootnote.drv}
\Msg{*  makeindex -s gind.ist tablefootnote.idx}
\Msg{*  pdflatex tablefootnote.drv}
\Msg{*}
\Msg{* At least three runs are necessary e.g. to get the}
\Msg{*  references right!}
\Msg{*}
\Msg{* Happy TeXing!}
\Msg{*}
\Msg{************************************************************************}

\endbatchfile
%</install>
%<*ignore>
\fi
%</ignore>
%
% \section{The documentation driver file}
%
% The next bit of code contains the documentation driver file for
% \TeX{}, i.\,e., the file that will produce the documentation you
% are currently reading. It will be extracted from this file by the
% \texttt{docstrip} programme. That is, run \LaTeX{} on \texttt{docstrip}
% and specify the \texttt{driver} option when \texttt{docstrip}
% asks for options.
%
%    \begin{macrocode}
%<*driver>
\NeedsTeXFormat{LaTeX2e}[2011/06/27]
\ProvidesFile{tablefootnote.drv}%
  [2014/01/26 v1.1c Table foot notes (HMM)]
\documentclass{ltxdoc}[2007/11/11]% v2.0u
\usepackage{holtxdoc}[2012/03/21]%  v0.24
%% tablefootnote may work with earlier versions of LaTeX2e and those
%% class and package, but this was not tested.
%% Please consider updating your LaTeX, class, and package
%% to the most recent version (if they are not already the most
%% recent version).
\hypersetup{%
 pdfsubject={Footnotes in Tables (HMM)},%
 pdfkeywords={LaTeX, tablefootnote, footnote, table, H.-Martin Muench},%
 pdfencoding=auto,%
 pdflang={en},%
 breaklinks=true,%
 linktoc=all,%
 pdfstartview=FitH,%
 pdfpagelayout=OneColumn,%
 bookmarksnumbered=true,%
 bookmarksopen=true,%
 bookmarksopenlevel=3,%
 pdfmenubar=true,%
 pdftoolbar=true,%
 pdfwindowui=true,%
 pdfnewwindow=true%
}
\CodelineIndex
\hyphenation{printing docu-ment}
\gdef\unit#1{\mathord{\thinspace\mathrm{#1}}}%
\makeatletter
\@ifundefined{eTeX}{\gdef\eTeX{$\m@th \varepsilon $-\TeX }}{% else \relax
}
\makeatother
\begin{document}
  \DocInput{tablefootnote.dtx}%
\end{document}
%</driver>
%    \end{macrocode}
%
% \fi
%
% \CheckSum{584}
%
% \CharacterTable
%  {Upper-case    \A\B\C\D\E\F\G\H\I\J\K\L\M\N\O\P\Q\R\S\T\U\V\W\X\Y\Z
%   Lower-case    \a\b\c\d\e\f\g\h\i\j\k\l\m\n\o\p\q\r\s\t\u\v\w\x\y\z
%   Digits        \0\1\2\3\4\5\6\7\8\9
%   Exclamation   \!     Double quote  \"     Hash (number) \#
%   Dollar        \$     Percent       \%     Ampersand     \&
%   Acute accent  \'     Left paren    \(     Right paren   \)
%   Asterisk      \*     Plus          \+     Comma         \,
%   Minus         \-     Point         \.     Solidus       \/
%   Colon         \:     Semicolon     \;     Less than     \<
%   Equals        \=     Greater than  \>     Question mark \?
%   Commercial at \@     Left bracket  \[     Backslash     \\
%   Right bracket \]     Circumflex    \^     Underscore    \_
%   Grave accent  \`     Left brace    \{     Vertical bar  \|
%   Right brace   \}     Tilde         \~}
%
% \GetFileInfo{tablefootnote.drv}
%
% \begingroup
%   \def\x{\#,\$,\^,\_,\~,\ ,\&,\{,\},\%}%
%   \makeatletter
%   \@onelevel@sanitize\x
% \expandafter\endgroup
% \expandafter\DoNotIndex\expandafter{\x}
% \expandafter\DoNotIndex\expandafter{\string\ }
% \begingroup
%   \makeatletter
%     \lccode`9=32\relax
%     \lowercase{%^^A
%       \edef\x{\noexpand\DoNotIndex{\@backslashchar9}}%^^A
%     }%^^A
%   \expandafter\endgroup\x
%
% \DoNotIndex{\\,\,}
% \DoNotIndex{\documentclass,\usepackage,\ProvidesPackage,\begin,\end}
% \DoNotIndex{\MessageBreak,\message}
% \DoNotIndex{\NeedsTeXFormat,\DoNotIndex,\verb}
% \DoNotIndex{\def,\edef,\gdef,\global}
% \DoNotIndex{\ifx,\listfiles,\mathord,\mathrm}
% \DoNotIndex{\kvoptions,\SetupKeyvalOptions,\ProcessKeyvalOptions}
% \DoNotIndex{\bigskip,\space,\thinspace,\ldots}
% \DoNotIndex{\indent,\noindent,\newline,\linebreak,\pagebreak}
% \DoNotIndex{\textbf,\textit,\textsf,\texttt,\textsc,\textquotedblleft,\textquotedblright}
% \DoNotIndex{\plainTeX,\TeX,\LaTeX,\pdfLaTeX}
% \DoNotIndex{\chapter,\section,\the,\csname,\endscname}
% \DoNotIndex{\value,\arabic,\hline,\text,\caption,\centering,\clear,\url}
%
% \title{The \xpackage{tablefootnote} package}
% \date{2014/01/26 v1.1c}
% \author{H.-Martin M\"{u}nch\\\xemail{Martin.Muench at Uni-Bonn.de}}
%
% \maketitle
%
% \begin{abstract}
% This \LaTeX{} package provides the command |\tablefootnote| to be used
% in a \texttt{table} or \texttt{sidewaystable} environment,
% where |\footnote| will not work and when using |\footnotemark| and
% |\footnotetext| and adjusting the counters (including \texttt{Hfootnote})
% manually is either too much work or would not even work (\texttt{sidewaystable}).
% \end{abstract}
%
% \bigskip
%
% \noindent Disclaimer for web links: The author is not responsible for any contents
% referred to in this work unless he has full knowledge of illegal contents.
% If any damage occurs by the use of information presented there, only the
% author of the respective pages might be liable, not the one who has referred
% to these pages.
%
% \bigskip
%
% \noindent {\color{green} Save per page about $200\unit{ml}$ water,
% $2\unit{g}$ CO$_{2}$ and $2\unit{g}$ wood:\\
% Therefore please print only if this is really necessary.}
%
% \pagebreak
%
% \tableofcontents
%
% \pagebreak
%
% \section{Introduction\label{sec:Introduction}}
%
% \indent This \LaTeX{} package provides the command |\tablefootnote{...}| to be
% used in a \texttt{table} or \texttt{sidewaystable} environment. In those
% environments |\footnote{...}| would not work and would need to be replaced
% by |\footnotemark| in the \texttt{(sideways)\allowbreak table} and
% |\footnotetext{...}| after the \texttt{(sideways)table} environment ended.
% Additionally the counters for \texttt{footnote} and
% (when the \xpackage{hyperref} package is used) \texttt{Hfootnote}
% would need to be adjusted manually. Then still the hyperlinks
% in a \texttt{(sideways)table} would not work.
% When the \xpackage{footnotebackref} package is used,
% |\footnotemark| and |\footnotetext{...}| would need to be redefined.
% Just using |\tablefootnote{...}| (with optional argument for custom foot note
% marks) in the (sideways)table does all this automatically.
% (Compatibility with the \xpackage{footnotebackref} package
% was achieved without redefining |\footnotemark| or |\footnotetext|.)
% Redefining |\footnote| was \emph{not} done as to not create conflicts with
% other packages.
%
% \section{Usage}
%
% \indent Just load the package placing
% \begin{quote}
%   |\usepackage{tablefootnote}|
% \end{quote}
% \noindent in the preamble of your \LaTeXe{} source file and use\newline
% |\tablefootnote{...}| instead of |\footnote{...}| and\newline
% |\tablefootnote[...]{...}| instead of |\footnote[...]{...}|
% in (sideways)\allowbreak tables.\\
% When the \xpackage{footnotebackref} package is used,
% it must be loaded before the \xpackage{tablefootnote} package
% and the \xpackage{hyperrefpackage} with option |hyperfootnotes=true|
% must be loaded, too. (Backreference{-}links without links do not make sense.)\\
% When the \xpackage{rotating} package (for |sidewaystable|s) is used,
% it must be loaded before the \xpackage{tablefootnote} package and it must be
% a recent version (v2.16a, 2009/03/28, or newer).\\
% If |\tablefootnote{...}| is used in a tabular environment, which is inside
% of a (sideways)table environment, everything is fine, but when the tabular
% environment is not inside of a (sideways)table, the \textbf{table}footnote
% will not work.\\
%
% Pages with tables with footnotes, which are rotated with the \xpackage{lscape}-
% or with the \xpackage{pdflscape}-package (|\begin{landscape}|, table, |\end{landscape}|),
% are possible. But when the \xpackage{footmisc}-package is used, \xpackage{(pdf)lscape}
% should be loaded before \xpackage{footmisc}.\\
%
% When the \xpackage{footmisc}-package is used with option \texttt{para},
% this is regarded with the exception of a \texttt{sidewaystable}. There the footnotes
% are printed one below the other. (Well, this is better than no footnote in a
% \texttt{sidewaystable} at all, isn't it?)\\
%
% When the (sideways)table floats, the footnotes are set and (when \xpackage{hyperref}
% is used) hyperlinked, but they are not automatically adapted when the table floats
% over/under another footnote. Thus either do not use a footnote between original and
% float position of the (sideways)table, or place the (sideways)table in
% \textquotedblleft here\textquotedblright\ position. |\clear(double)page|, |h(!)|,
% |H|~from the \xpackage{float} package (\url{http://www.ctan.org/pkg/float}),
% or |\FloatBarrier| from the \xpackage{picins} package
% (\url{http://www.ctan.org/pkg/picins}) might help, too.
% (Or after finishing the document, move the (sideways)table in the source code
% near the position where it floats to or use the optional footnote marks.)
%
% \pagebreak
%
% \section{Alternatives\label{sec:Alternatives}}
%
% \begin{description}
% \item[-] The \xpackage{longtable} package provides the \texttt{longtable} environment
%   as replacement for the combined \texttt{table} and \texttt{tabular} environments.
%   Footnotes are real footnotes (not just tablenotes), are continuously numbered and
%   hyperlinked (when using the \xpackage{hyperref} package), and the hyperlinks really
%   work. As drawback the appearance of the caption changes slightly (e.\,g.~distance
%   to the table, width of the caption), which can probably be changed back manually.
%   Furthermore, longtables are meaned to break over more than one page. If that is not
%   wished, it must be prevented by |\nopagebreak|-commands and by ending the longtable
%   lines with |\\*| instead of |\\|. longtables do not float. (Therefore using the
%   \xpackage{tablefootnote} package and |\FloatBarrier| from the \xpackage{picins} package
%   before and after the table environment is similar~- but \xpackage{tablefootnote} does not
%   change the table-caption!) \texttt{sidewaystable} does not work with it.\\
%   \url{http://www.ctan.org/pkg/longtable}
%
% \item[-] The \xpackage{supertabular} package provides the \texttt{mpsupertabular}
%   environment as replacement for the combined \texttt{table} and \texttt{tabular}
%   environments. Footnotes are just tablenotes (with working hyperlinks when using the
%   \xpackage{hyperref} package), i.\,e. numbered a, b, c and placed below the table
%   and not at the end of the page. Therefore there is no float problem (because the
%   tablenotes numbering is not included in the continuous numbering of the footnotes).
%   Placing the \texttt{supertabular} inside of a \texttt{sidewaystable} breaks the
%   hyperlinks to the tablenotes.\\
%   \url{http://www.ctan.org/pkg/supertabular}
%
% \item[-] The \xpackage{ctable} package has its very own notation for defining tables.
%   It can create tablenotes and sideways-tables. The tablenotes are not automatically
%   hyperlinked. The \texttt{ctable}s float. Because the tablenotes numbering is not
%   included in the continuous numbering of the footnotes there is no float problem.\\
%   \url{http://www.ctan.org/pkg/ctable}
%
% \item[-] The \xpackage{footnote} package provides |\makesavenoteenv{table}|. After
%   loading the package and using that command in the preamble, in tables |\footnote{...}|
%   can be used. Using |\makesavenoteenv{tabular}| and \\
%   |\makesavenoteenv{sidewaystable}|
%   is possible, but it neither solves the float problem, nor do the created hyperlinks
%   work (i.\,e.~they aim at wrong locations). The \xpackage{mdwtab} from the same bundle
%   is incompatible with other table-related packages (e.\,g. \xpackage{supertabular},
%   \xpackage{array}) and not $100\,\%$ compatible with the \texttt{tabular} environment.\\
%   \url{http://www.ctan.org/pkg/footnote}\\
%   \url{http://www.ctan.org/pkg/mdwtab}
%
% \item[-] The \xpackage{tabularx} package does produce footnotes for sidewaystables,
%   but uses a, b, c instead of 1, 2, 3. The hyperlinks to the footnotes do not work.
%   Because the footnotes numbering is not included in the continuous numbering of
%   the other footnotes there is no float problem.\\
%   \url{http://www.ctan.org/pkg/tabularx}
%
% \item[-] Placing a tabular inside a minipage inside a table produces tablenotes.
%   Therefore there is no float problem (because the footnotes are not continuously
%   numbered). The hyperlinks to the table notes indeed work.
%
% \pagebreak
%
% \item[-] The \xpackage{threeparttable} package creates tablenotes again. Therefore
%   there is no float problem (because the tablenotes are not continuously numbered with
%   the footnotes). There are no hyperlinks to the table notes (at least not automatically).
%   Using |sidewaystable| (with table notes) works.\\
%   \url{http://www.ctan.org/pkg/threeparttable}
%
% \item[-] The \xpackage{threeparttablex} package creates tablenotes again. Therefore there
%   is no float problem (because the tablenotes are not continuously numbered with the
%   footnotes). With option \texttt{referable} the tablenotes are hyperlinked. Use of
%   a |sidewaystable| (with table notes) did not work for me. When using the
%   \texttt{referable} option according to the example in the \xpackage{threeparttablex}
%   manual the \xpackage{longtable} package is used, therefore that package could be used
%   directly without \xpackage{threeparttablex} (see above).\\
%   \url{http://www.ctan.org/pkg/threeparttablex}
%
% \item[-] One can manually use |\footnotemark| in the table and |\footnotetext{...}|
%  after the table environment ended and manually change the |footnote| and
%  (when the \xpackage{hyperref} package is used) |Hfootnote| counters and needs
%  to insert
%  \begin{verbatim}
%     \makeatletter
%     \global\let\Hy@saved@currentHref\@currentHref
%     \hyper@makecurrent{Hfootnote}%
%     \global\let\Hy@footnote@currentHref\@currentHref
%     \global\let\@currentHref\Hy@saved@currentHref
%     \makeatother
%  \end{verbatim}
%  before each |\footnotetext{...}|, but with more than very few footnotes this can become
%  much work (and is prone to produce errors). And this does not even work with
%  sidewaystables. (And it does not work with the \xpackage{footnotebackref} package
%  without redefining |\footnotemark| and |\footnotetext{...}|.)
% \end{description}
%
% \bigskip
%
% \noindent (You programmed or found another alternative,
%  which is available at \CTAN{}?\\
%  OK, send an e-mail to me with the name, location at \CTAN{},
%  and a short notice, and I will probably include it in
%  the list above.)
%
% \newpage
%
% \section{Example}
%
%    \begin{macrocode}
%<*example>
\documentclass[british]{article}[2007/10/19]% v1.4h
%%%%%%%%%%%%%%%%%%%%%%%%%%%%%%%%%%%%%%%%%%%%%%%%%%%%%%%%%%%%%%%%%%%%%
\usepackage{float}[2001/11/08]%    v1.3d
\usepackage{placeins}[2005/04/18]% v2.2  ; for \FloatBarrier
\usepackage{rotating}[2009/03/28]% v2.16a; for sidewaystable-environment
\usepackage[%
 hyperfootnotes=true,%
 extension=pdf,%
 plainpages=false,%
 pdfpagelabels=true,%
 hyperindex=false,%
 pdflang={en},%
 pdftitle={tablefootnote package example},%
 pdfauthor={H.-Martin Muench},%
 pdfsubject={Example for the tablefootnote package},%
 pdfkeywords={LaTeX, tablefootnote, footnote, table, H.-Martin Muench},%
 % pdfview=FitH and FitBH do not work: hyperlinks in sidewaystables
 % do no lead to the footnotes, due to a bug in pdfTeX,
 % computing wrong anchor coordinates (Heiko Oberdiek, 29. October 2011)
 % pdfview=Fit, FitV, FitR, FitB, FitBV work
 % print is OK for all those options
 pdfstartview=FitH,%
 pdfpagelayout=OneColumn%
]{hyperref}[2012/11/06]% v6.83m
 % Due to the urls used in the example, either the hyperref or the url
 % package are needed (or the urls must be removed before compiling).

\usepackage{footnotebackref}[2012/07/01]% v1.0
\usepackage{tablefootnote}[2014/01/26]%   v1.1c

\gdef\unit#1{\mathord{\thinspace\mathrm{#1}}}%
\listfiles
\begin{document}
\pagenumbering{arabic}
\section*{Example for tablefootnote}

This example demonstrates the use of package\newline
\textsf{tablefootnote}, v1.1c as of 2014/01/26 (HMM).\newline
There were no options used. (The package provides no options.)\newline

\textbf{The \texttt{tablefootnote-example.tex} needs to be compiled
at least twice to get the references right!}\newline

If the etoolbox-package is found, it is automatically used.\newline

For more details please see the documentation!\newline

\noindent Save per page about $200\unit{ml}$ water,
$2\unit{g}$ CO$_{2}$ and $2\unit{g}$ wood:\newline
Therefore please print only if this is really necessary.\newline

Here is some text.\footnote{And this is a text footnote.}\newline

Tables \ref{tab.symbol}, \ref{tab.normal}, \ref{tab.another} and
\ref{tab.floatH} show normal tables, table~\ref{tab.sideways} depicts
a sidewaystable. Table~\ref{tab.floatH} uses the float
specifier~\texttt{H} from the float package.\newline

\texttt{Hyperref} option \verb|pdfview=FitH| and \verb|FitBH| do
not work due to a bug in pdf\TeX{}, computing wrong
anchor coordinates (\textsc{Heiko Oberdiek}, 29.~October 2011).
Depending on used pdf-viewer, hyperlinks in sidewaystables lead
e.\,g.~at the end of the document, not at the footnote.
\verb|pdfview=Fit|, \verb|FitV|, \verb|FitR|, \verb|FitB|,
\verb|FitBV| work, print is OK for all those options.

\bigskip

\listoftables

\pagebreak

\renewcommand{\thefootnote}{\fnsymbol{footnote}}
\verb|\renewcommand{\thefootnote}{\fnsymbol{footnote}}|
causes footnote{-}symbol{-}footnotes, which are possible
(see Table~\ref{tab.symbol}).

\begin{table}
\centering
\begin{tabular}{ccc}
Another\tablefootnote{A table footnote.} & %
text\tablefootnote{Another table footnote.} & %
in a table\tablefootnote{A third table footnote.}
\end{tabular}
\caption[A footnotesymbol table]{%
A table with footnote-symbol-footnotes.\label{tab.symbol}}
\end{table}

Some text.\footnote{A text footnote.}

\renewcommand{\thefootnote}{\arabic{footnote}}
\verb|\renewcommand{\thefootnote}{\arabic{footnote}}|
switches back to normal footnote numbers again.

\pagebreak

\begin{table}
\centering
\begin{tabular}{ccc}
Some\tablefootnote{A table footnote.} & %
text\tablefootnote[99]{A table footnote with custom footnote number.} & %
in a table\tablefootnote{A third table footnote.}
\end{tabular}
\caption[A table]{A normal table.\label{tab.normal}}
\end{table}

Some text.\footnote{Another text footnote.}

\pagebreak

%    \end{macrocode}
%
% \pagebreak
%
%    \begin{macrocode}
More text.\footnote{And yet another text footnote.}

\begin{table}[t]
\centering
\begin{tabular}{|c|c|c|}
\hline
Another\tablefootnote{A $3^{rd}$ table footnote.} & %
text\tablefootnote{Another $3^{rd}$ table footnote.} & %
in a table\tablefootnote{A $3^{rd}$ third table footnote.}\\ \hline
\end{tabular}
\caption[Another table]{Another table (third one)\label{tab.another}}
\end{table}

Please note that Table~\ref{tab.another} floated to the top of the page.
While the footnotes are set and (when hyperref is used) hyperlinked,
they are not automatically adapted. Thus either do not use a footnote
at the same page before the table, or place the table in
\textquotedblleft here\textquotedblright\ or
\textquotedblleft bottom\textquotedblright\ position.
\verb|\clear(double)page|, \verb|h(!)|, \verb|H|~from the
\texttt{float} package, or \verb|\FloatBarrier| from the
\texttt{picins} package might help, too. (Or move the table in the
source code near the position where it floats to
or use the optional footnote marks.)

Table~\ref{tab.floatH} (page~\pageref{tab.floatH}) uses float specifier
\texttt{H} from the float package and does not float.

Some text.\footnote{This is just another text footnote.}

\pagebreak

\FloatBarrier

\begin{sidewaystable}
\centering%
\begin{tabular}{ccc}
Text\tablefootnote{Please rotate the view for testing the %
hyperlinks.} & %
in a\tablefootnote[98]{Another sidewaystable footnote %
with optional footnote mark.} & %
sidewaystable%
\tablefootnote{Sidewaystable-environment provided by %
\url{http://www.ctan.org/pkg/rotating} package.}
\end{tabular}
\caption[A sideways table]{A table in the \texttt{sideways} %
environment\label{tab.sideways}}%
\end{sidewaystable}%

\FloatBarrier

\pagebreak

%    \end{macrocode}
%
% \pagebreak
%
%    \begin{macrocode}
A last table, here with float specifier \texttt{H} from the
float\footnote{\url{http://www.ctan.org/pkg/float}} package.

\begin{table}[H]
\centering
\begin{tabular}{ccc}
Another\tablefootnote{A $5^{th}$ table footnote.} & %
text\tablefootnote{Another $5^{th}$ table footnote.} & %
in a table\tablefootnote{A $5^{th}$ third table footnote.}
\end{tabular}
\caption[A last table]{A very last table\label{tab.floatH}}
\end{table}

Some text.\footnote{This is just another text footnote.}

\pagebreak

End of the example for the
tablefootnote\footnote{\url{http://www.ctan.org/pkg/tablefootnote}}
package.

\end{document}
%</example>
%    \end{macrocode}
%
% \StopEventually{}
%
% \newpage
%
% \section{The implementation}
%
% We start off by checking that we are loading into \LaTeXe{} and
% announcing the name and version of this package.
%
%    \begin{macrocode}
%<*package>
%    \end{macrocode}
%
%    \begin{macrocode}
\NeedsTeXFormat{LaTeX2e}[2011/06/27]
\ProvidesPackage{tablefootnote}[2014/01/26 v1.1c
              Table foot notes (HMM)]
%    \end{macrocode}
%
% A short description of the \xpackage{tablefootnote} package:
%
%    \begin{macrocode}
%% Provides the \tablefootnote[...]{...} command
%% for footnotes in (sideways)tables.

%    \end{macrocode}
%
% We need the \xpackage{ltxcmds} and \xpackage{letltxmacro} packages
% by \textsc{Heiko Oberdiek}:
%
%    \begin{macrocode}
\RequirePackage{ltxcmds}[2011/11/09]%     v1.22
\RequirePackage{letltxmacro}[2010/09/02]% v1.4
\RequirePackage{xifthen}[2009/04/17]%     v1.3
%    \end{macrocode}
%
% A last information for the user:
%
%    \begin{macrocode}
%% tablefootnote may work with earlier versions of LaTeX and those
%% packages, but this was not tested. Please consider updating
%% your LaTeX and packages to the most recent version
%% (if they are not already the most recent version).

%    \end{macrocode}
%
% \indent See subsection~\ref{ss:Downloads} about how to get them.\\
%
% When the \xpackage{rotating} package (for |sidewaystable|s) is used,
% it must be loaded before the \xpackage{tablefootnote} package and it must be
% a recent version:
%
%    \begin{macrocode}
\@ifpackageloaded{rotating}{%
  \gdef\tfn@rotating{1}
  \@ifpackagelater{rotating}{2009/03/28}% v2.16a
   {% >= 2009/03/28, OK
   }{% else
     \PackageWarningNoLine{tablefootnote}{%
       It is required version\MessageBreak%
       2009/03/28 v2.16a (or later) of package rotating,\MessageBreak%
       but only version\MessageBreak%
      `\csname ver@rotating.sty\endcsname'\MessageBreak%
      is available}%
   }%
 }{% else
   \gdef\tfn@rotating{0}%
 }

%    \end{macrocode}
%
% There are no options to be processed.\\
%
% We need some definitions:
%
%    \begin{macrocode}
\def\tfn@footnotetablecount{0}
\def\tfn@footnotetableprint{0}
\def\tfn@footnotezero{0}
\def\tfn@fnt{0}
\def\tfn@swt{0}%       _s_ide_w_ays_t_able
\def\tfn@fmpw{\relax}% f_oot_m_isc_p_ara_w_arn-switch
\def\tfn@fnbr{0}%      f_oot_n_ote_b_ack_r_ef(erence package)

%    \end{macrocode}
%
% \begin{macro}{\tfn@hyperfootnotes}
% We need a wrapper for the used |\ifHy@hyperfootnotes|,
% which is not defined when the \xpackage{hyperref} package has not been loaded:
%
%    \begin{macrocode}
\DeclareRobustCommand{\tfn@hyperfootnotes}[2]{%
  \ifHy@hyperfootnotes%
    #1%
  \else%
    #2%
  \fi%
  }

%    \end{macrocode}
% \end{macro}
%
% \begin{macro}{\tablemakefntext}
% |\tablemakefntext| makes the footnotetext for a footnote in a table.
%
%    \begin{macrocode}
\ltx@ifpackageloaded{footnotebackref}{%
  \gdef\tfn@fnbr{1}%
  \ifx\FootnoteBackref@symbol\empty%
    \ifFootnoteBackref@numberlinked%
      \newcommand\tablemakefntext[1]{%
        \noindent\makebox[1.634em][r]{%
         \mbox{\textsuperscript{\normalfont%
           \hyperlink{tfn:m\theHfootnote}{\@thefnmark}%
        }}}%
        \ #1%
      }
    \else
      \newcommand\tablemakefntext[1]{%
        \noindent\makebox[1.8em][r]{%
          \mbox{\textsuperscript{\normalfont%
            \@thefnmark%
        }}}%
        #1%
      }
    \fi
  \else
    \newcommand\tablemakefntext[1]{%
      \noindent\makebox[1.8em][r]{%
        \mbox{\textsuperscript{\normalfont%
          \ifFootnoteBackref@numberlinked%
            \hyperlink{tfn:m\theHfootnote}{\@thefnmark}\ %
          \else%
            \@thefnmark%
          \fi%
      }}}%
      \hyperlink{tfn:m\theHfootnote}{\FootnoteBackref@symbol}%
      #1%
    }
  \fi
}{%
  \newcommand\tablemakefntext[1]{%
    \noindent\makebox[1.8em][r]{%
      \mbox{\textsuperscript{\normalfont%
        \@thefnmark%
    }}}%
    #1%
  }
}

%    \end{macrocode}
%
% \end{macro}
%
% \pagebreak
%
% \begin{macro}{\swtablemakefntext}
%    \begin{macrocode}
\newcommand\swtablemakefntext[1]{\noindent\ #1}

%    \end{macrocode}
% \end{macro}
%
% \begin{macro}{\tfn@footnotemarks}
% |\tfn@footnotemarks| uses the |\footnotemark| command to place a foot note mark
% and decreases the |\tfn@footnotetablecount| (i.\,e.~increases the negative value)
% of the number of footnote(marks) in the table. If the \xpackage{footnotebackref}
% package is used, it is also necessary to create an according |\hypertarget| for
% the back reference.
%
%    \begin{macrocode}
\newcommand{\tfn@footnotemarks}[1][]{%
  \hbox{%
    \ifthenelse{\isempty{#1}}%
      {\footnotemark%
       \ltx@ifpackageloaded{footnotebackref}%
         {\raisebox{\ht\strutbox}{\hypertarget{tfn:m\theHfootnote}{}}}%
         {% footnotebackref not loaded
         }%
      }%
      {\ltx@ifpackageloaded{hyperref}%
        {\tfn@hyperfootnotes{%
            \hyperlink{tfn:#1}{\footnotemark[#1]}%
          }{\footnotemark[#1]}}%
        {\footnotemark[#1]}%
       \ltx@ifpackageloaded{footnotebackref}%
        {\raisebox{\ht\strutbox}{\hypertarget{tfn:m#1}{}}}{}%
      }%
   }%
  \@tempcnta=\tfn@footnotetablecount\relax%
  \advance\@tempcnta by -1\relax%
  \xdef\tfn@footnotetablecount{\the\@tempcnta}%
}

%    \end{macrocode}
% \end{macro}
%
% \begin{macro}{\tfn@footmiscparawarn}
% When the \xpackage{footmisc} package is used with option |para|,
% we need to issue a warning. (Any idea how to resolve this issue?)\\
% It must be given inside of it's own command, because |\ifFN@para| is not
% defined when \xpackage{footmisc} has not been loaded, and then the
% |\fi| would be a |\fi| without |\if| (and \TeX{} does not like single
% |\fi|s).
%
%    \begin{macrocode}
\newcommand{\tfn@footmiscparawarn}{%
  \ifFN@para%
    \PackageWarning{tablefootnote}{%
      Package footmisc with option para used.\MessageBreak%
      Unfortunately package tablefootnote is not yet\MessageBreak%
      able to regard this in sidewaystables.\MessageBreak%
      Here the footnotes are printed one below\MessageBreak%
      the other. Sorry!\MessageBreak%
     }%
  \fi%
  }

%    \end{macrocode}
% \end{macro}
%
% \begin{macro}{\tfn@updateHyper}
% |\tfn@updateHyper| updates some \xpackage{hyperref} internals.
%
%    \begin{macrocode}
\newcommand{\tfn@updateHyper}{%
  % The code of this command is partly from the
  % hyperref package by Heiko Oberdiek, 2011/10/01, v6.82j.
  % Newer versions of that package are available at CTAN.org.
  \global\let\Hy@saved@currentHref\@currentHref%
  \tfn@hyperfootnotes{\hyper@makecurrent{Hfootnote}}{}%
  \global\let\Hy@footnote@currentHref\@currentHref%
  \global\let\@currentHref\Hy@saved@currentHref%
  }

%    \end{macrocode}
% \end{macro}
%
% \begin{macro}{\tfn@footnotetext}
% |\tfn@footnotetext| contains the |\footnotetext[...]{...}| command and handles
% the numbering. |\tfn@footnotetext| will be executed after the table.
% The foot note counter has been increased automatically with each |\footnotemark|,
% therefore it must be reset to its value before the table began.
%
%    \begin{macrocode}
\newcommand{\tfn@footnotetext}[2][]{%
  \ifx\tfn@footnotetableprint\tfn@footnotezero%
    % This is the first footnotetext after the (sideways)table.
    \addtocounter{footnote}{\tfn@footnotetablecount}%
%    \end{macrocode}
%
% When the \xpackage{hyperref} package is used, also its |Hfootnote| counter must
% be reset (when the footnotes are hyperlinked).
%
%    \begin{macrocode}
    \ltx@ifpackageloaded{hyperref}{%
      \tfn@hyperfootnotes{%
        \addtocounter{Hfootnote}{\tfn@footnotetablecount}}{}%
      \tfn@updateHyper%
     }{}%
  \fi%
%    \end{macrocode}
%
% For issuing the text for the next footnote the footnote-counter
% (and |Hfootnote| counter in case of links) are increased by one.
%
%    \begin{macrocode}
  \addtocounter{footnote}{+1}%
  \ltx@ifpackageloaded{hyperref}{%
    \tfn@hyperfootnotes{\addtocounter{Hfootnote}{+1}}{}%
%    \end{macrocode}
%
% The \xpackage{hyperref} package was not expecting this kind of change of the
% |(H)footnote| numbers, therefore some code-lines are missing here which can be
% found at other places in that same package and are called by this command.
%
%    \begin{macrocode}
    \tfn@updateHyper%
%    \end{macrocode}
%
% When \xpackage{hyperref} is not loaded, nothing is to be done here.
%
%    \begin{macrocode}
   }{}%
%    \end{macrocode}
%
% The |\footnotetext[...]{...}| places the foot note text as usual.
%
%    \begin{macrocode}
  \ifx\tfn@swt\tfn@footnotezero%
    \ifthenelse{\isempty{#1}}%
      {\footnotetext{#2}}%
      {\ltx@ifpackageloaded{hyperref}%
         {\tfn@hyperfootnotes{%
             \footnotetext[#1]{%
               \raisebox{\ht\strutbox}{\hypertarget{tfn:#1}{}}#2}%
           }{\footnotetext[#1]{#2}%
           }%
         }{\footnotetext[#1]{#2}}%
      }%
  \else%
%    \end{macrocode}
%
% If there is more than one tablefootnote in the |sidewaystable|, \\
% the |\tfn@footmiscparawarn| shall be given,
%
%    \begin{macrocode}
    \ifx\tfn@footnotetableprint\tfn@footnotezero%
      \gdef\tfn@fmpw{\tfn@footmiscparawarn}%
    \else%
      \ltx@ifpackageloaded{footmisc}{\tfn@fmpw}{\relax}%
%    \end{macrocode}
%
% but we do not want to repeat the message for a third(, forth,\ldots)
% tablefootnote in the |sidewaystable|.
%
%    \begin{macrocode}
      \gdef\tfn@fmpw{\relax}%
    \fi%
%    \end{macrocode}
%
% |\@makefnmark| is \\
% |\hbox {\@textsuperscript {\normalfont \@thefnmark }}|, \\
% but according to |\show\@thefnmark| here |\@thefnmark| is
% |{\protect \itshape}| instead of |\thefootnote| (Why?!).\\
% If the \xpackage{footnotebackref} package is used,
% back references must be created for the table footnotes, too.
%
%    \begin{macrocode}
    \ltx@ifpackageloaded{footnotebackref}{%
      \LetLtxMacro{\tfncurrentmft}{\@makefntext}%
      \LetLtxMacro{\@makefntext}{\swtablemakefntext}%
      \ifthenelse{\isempty{#1}}%
        {\footnotetext{%
           \raisebox{\ht\strutbox}{\hypertarget{tfn:\theHfootnote}{}}%
           \hbox{\@textsuperscript{\normalfont%
              \hyperlink{tfn:m\theHfootnote}{\thefootnote}}}\thinspace%
         \ifx\FootnoteBackref@symbol\empty%
         \else\hyperlink{tfn:m\theHfootnote}{\FootnoteBackref@symbol}%
         \fi%
         #2}%
       }{\edef\tfn@tfn{\arabic{footnote}}%
         \edef\tfn@thn{\arabic{Hfootnote}}%
         \setcounter{footnote}{#1}%
         \setcounter{Hfootnote}{#1}%
         \tfn@updateHyper%
         \footnotetext{%
           \raisebox{\ht\strutbox}{\hypertarget{tfn:#1}{}}%
           \hbox{\@textsuperscript{\normalfont%
                  \hyperlink{tfn:m#1}{#1}}}%
           \thinspace%
           \ifx\FootnoteBackref@symbol\empty%
           \else%
             \raisebox{\ht\strutbox}{%
               \hyperlink{tfn:m#1}{\FootnoteBackref@symbol}}%
           \fi%
           #2%
          }%
         \setcounter{footnote}{\tfn@tfn}%
         \setcounter{Hfootnote}{\tfn@thn}%
         \tfn@updateHyper%
       }%
      \LetLtxMacro{\@makefntext}{\tfncurrentmft}%
     }{% footnotebackref not loaded
       \ifthenelse{\isempty{#1}}%
         {\footnotetext{%
            \hbox{\@textsuperscript{\normalfont\thefootnote}}%
            \ltx@ifpackageloaded{hyperref}%
              {\tfn@hyperfootnotes{%
                 \raisebox{\ht\strutbox}{%
                   \hypertarget{tfn:\theHfootnote}{}}}{}%
              }{}%
            \thinspace%
            #2}%
        }{%
%    \end{macrocode}
%
% \pagebreak
%
%    \begin{macrocode}
          \edef\tfn@tfn{\arabic{footnote}}%
          \ltx@ifpackageloaded{hyperref}{%
            \tfn@hyperfootnotes{\xdef\tfn@thn{\arabic{Hfootnote}}}{}%
           }{}%
          \setcounter{footnote}{#1}
          \ltx@ifpackageloaded{hyperref}{%
            \tfn@hyperfootnotes{\setcounter{Hfootnote}{#1}}{}%
            \tfn@updateHyper%
           }{}%
          \footnotetext{%
            \hbox{\@textsuperscript{\normalfont #1}}%
            \ltx@ifpackageloaded{hyperref}{%
              \tfn@hyperfootnotes{\raisebox{\ht\strutbox}{%
                                         \hypertarget{tfn:#1}{}}}{}%
             }{}%
            \thinspace%
            #2}%
          \setcounter{footnote}{\tfn@tfn}%
          \ltx@ifpackageloaded{hyperref}{%
            \tfn@hyperfootnotes{\setcounter{Hfootnote}{\tfn@thn}}{}%
            \tfn@updateHyper%
           }{}%
        }%
     }%
  \fi%
%    \end{macrocode}
%
% For further footnotes in the table the \\
% |\ifx\tfn@footnotetableprint\tfn@footnotezero|\ldots|\fi| \\
% from the beginning of this |\tfn@footnotetext| macro must not be repeated,
% therefore we deactivate this by defining |\tfn@footnotetableprint| to
% something other than |0| (here |1| is used).
%
%    \begin{macrocode}
  \gdef\tfn@footnotetableprint{1}%
%    \end{macrocode}
%
% |\tfn@footnotetablecount| contains the negative number of footnote texts
% to be printed, therefore it is increase by one (i.\,e.~the negative number
% decreases, i.\,e.~$-3$ becomes~$-2$).
%
%    \begin{macrocode}
  \@tempcnta=\tfn@footnotetablecount\relax%
  \advance\@tempcnta by +1\relax%
  \xdef\tfn@footnotetablecount{\the\@tempcnta}%
%    \end{macrocode}
%
% When it is zero, all footnote texts have been placed, and\\
% |\tfn@footnotetableprint| can be reset to zero and\\
% |\tfn@tablefootnoteprintout| can be cleared.
%
%    \begin{macrocode}
  \ifx\tfn@footnotetablecount\tfn@footnotezero% all footnotes placed
    \gdef\tfn@footnotetableprint{0}%
    \global\let\tfn@tablefootnoteprintout\relax% brute force
 %\else there is at least one \footnotemark which needs \footnotetext
  \fi
}

%    \end{macrocode}
% \end{macro}
%
% \begin{macro}{\tfn@tablefootnoteprintout}
% When something shall be cleared or redefined, it is a good idea
% to define it. It~will be later filled with the foot note text to be printed.
%
%    \begin{macrocode}
\newcommand{\tfn@tablefootnoteprintout}{\relax}

%    \end{macrocode}
% \end{macro}
%
% \pagebreak
%
% \begin{macro}{\tablefootnote}
% |\tablefootnote[...]{...}| is the command to be used by the user instead of
% |\footnote[...]{...}| in the (sideways)table. It places a foot note mark via
% |\tfn@footnotemarks| and appends a |\tfn@footnotetext{#1}| to the\\
% |\tfn@tablefootnoteprintout| macro, and remembers that table foot notes
% have been used.
%
%    \begin{macrocode}
\newcommand{\tablefootnote}[2][]{%
  \tfn@footnotemarks[#1]%
  \ifthenelse{\isempty{#1}}{\relax}{%
    \ltx@GlobalPrependToMacro{\tfn@tablefootnoteprintout}{%
      \addtocounter{footnote}{+1}%
      \ltx@ifpackageloaded{hyperref}{%
        \tfn@hyperfootnotes{\addtocounter{Hfootnote}{+1}}{}%
        \tfn@updateHyper%
       }{}%
     }%
   }%
  \ltx@GlobalAppendToMacro{\tfn@tablefootnoteprintout}{%
    \tfn@footnotetext[#1]{#2}%
    \ifthenelse{\isempty{#1}}{\relax}{%
      \addtocounter{footnote}{-1}%
      \ltx@ifpackageloaded{hyperref}{%
        \tfn@hyperfootnotes{\addtocounter{Hfootnote}{-1}}{}%
        \tfn@updateHyper%
       }{}%
     }%
   }%
  \gdef\tfn@fnt{1}%
  }

%    \end{macrocode}
% \end{macro}
%
% \begin{macro}{\tfn@swtbox}
% For |sidewaystable|s the |\tfn@tablefootnoteprintout| is placed in a \\
% |\parbox|, which can be placed at the end of the |sidewaystable|:
%
%    \begin{macrocode}
\newcommand{\tfn@swtbox}{%
\ifvoid \footins%
\vskip 2\baselineskip%
\else%
\vskip \skip\footins%
\fi%
\parbox[b][][t]{\linewidth}{\raggedright\tfn@tablefootnoteprintout}%
}

%    \end{macrocode}
% \end{macro}
%
% |\tfn@tablefootnoteprintout| shall be executed after the end of the table environment.
% The \xpackage{etoolbox} package provides very easy syntax for this:
%
%    \begin{macrocode}
\IfFileExists{etoolbox.sty}{%
  \RequirePackage{etoolbox}[2011/01/03]%   v2.1
  \AfterEndEnvironment{table}{%
    \ltx@ifpackageloaded{footnotebackref}{%
      \LetLtxMacro{\tfncurrentmft}{\@makefntext}%
      \LetLtxMacro{\@makefntext}{\tablemakefntext}%
       \tfn@tablefootnoteprintout%
      \LetLtxMacro{\@makefntext}{\tfncurrentmft}%
     }{% footnotebackref not loaded
      \tfn@tablefootnoteprintout%
     }%
      \gdef\tfn@fnt{0}%
   }%
%    \end{macrocode}
%
% When the \xpackage{rotating} package has been loaded, we need to detect
% whether the |\tablefootnote| is placed inside of a |sidewaystable|.
%
%    \begin{macrocode}
  \ltx@ifpackageloaded{rotating}{%
    \AtBeginEnvironment{sidewaystable}{%
      \gdef\tfn@swt{1}%
    }
%    \end{macrocode}
%
% At the end of |sidewaystable| environments |\tfn@swtbox| shall be placed
% (if~there was any |\tablefootnote| in that |sidewaystable| at all).
% And |\tfn@swt| as well as |\tfn@fnt| must be reset to |0|.
%
%    \begin{macrocode}
    \AtEndEnvironment{sidewaystable}{%
      \ifx\tfn@fnt\tfn@footnotezero% \relax
      \else%
        \ltx@ifpackageloaded{footnotebackref}{%
          \LetLtxMacro{\tfncurrentmft}{\@makefntext}%
          \LetLtxMacro{\@makefntext}{\tablemakefntext}%
           \tfn@swtbox%
          \LetLtxMacro{\@makefntext}{\tfncurrentmft}%
         }{% footnotebackref not loaded
          \tfn@swtbox%
          }%
      \fi%
      \gdef\tfn@swt{0}%
      \gdef\tfn@fnt{0}%
     }%
   }{}% else \relax
}{%
%    \end{macrocode}
%
% When the \xpackage{etoolbox} package is not available (e.\,g. because of missing
% \eTeX{}) we~do a workaround:
%
%    \begin{macrocode}
  \PackageWarning{tablefootnote}{%
    Package etoolbox not available, therefore\MessageBreak%
    substituting commands \string\AtEndEnvironment\space %
    and\MessageBreak%
    \string\AfterEndEnvironment\space for tablefootnotes%
    \MessageBreak%
   }%
  \@ifpackagelater{letltxmacro}{2010/09/02}%
   {% >= 2010/09/02, OK
    \GlobalLetLtxMacro{\tfnorigtable}{\table}
    \GlobalLetLtxMacro{\tfnendorigtable}{\endtable}
%    \end{macrocode}
%
% |\GlobalLetLtxMacro| was introduced in version 2010/09/02 v1.4 of the
% \xpackage{letltxmacro} package.
%
%    \begin{macrocode}
   }{% else
     \PackageError{tablefootnote}{%
       Outdated version of letltxmacro package used%
      }{Version 2010/09/02 v1.4 or newer needed but not found!%
        \MessageBreak%
        Update the used letltxmacro package.}
        \LetLtxMacro{\tfnorigtable}{\table}
        \LetLtxMacro{\tfnendorigtable}{\endtable}
    }%
%    \end{macrocode}
%
% \pagebreak
%
%    \begin{macrocode}
  \ltx@ifpackageloaded{float}{%
    \@ifpackagelater{float}{2001/11/08}%   v1.3d
     {% >= 2001/11/08, OK
     }{% else
       \PackageWarningNoLine{tablefootnote}{%
         It is required version\MessageBreak
         2001/11/08 v1.3d (or later) of package float,\MessageBreak
         but only version\MessageBreak
        `\csname ver@float.sty\endcsname'\MessageBreak
        is available}%
     }%
   }{\relax}
  \renewenvironment{table}{%
    \ltx@ifpackageloaded{float}{%
%    \end{macrocode}
%
% When \xpackage{etoolbox.sty} was not found, and the \xpackage{float} package
% has been loaded, and the float specifier is \texttt{H}, then |\endtable| was
% let to |\float@endH| by the \xpackage{float} package (line~79 of
% \xpackage{float.sty}, 2001/11/08 v1.3d:\\
% |\expandafter\let\csname end#1\endcsname\float@endH|), i.\,e.~appending\\
% |\tfn@tablefootnoteprintout \gdef\tfn@fnt{0}| to |\endtable| is useless.
% Instead of this, it must be appended to |\float@endH|.\\
% (Why is this not necessary when \xpackage{etoolbox} is used?
% That is a good question, answers are welcome.)
%
%    \begin{macrocode}
      \ltx@LocalAppendToMacro{\float@endH}{%
        \ltx@ifpackageloaded{footnotebackref}{%
          \LetLtxMacro{\tfncurrentmft}{\@makefntext}%
          \LetLtxMacro{\@makefntext}{\tablemakefntext}%
           \tfn@tablefootnoteprintout%
          \LetLtxMacro{\@makefntext}{\tfncurrentmft}%
         }{% footnotebackref not loaded
          \tfn@tablefootnoteprintout%
         }%
          \gdef\tfn@fnt{0}%
       }%
     }{\relax}
    \tfnorigtable%
   }{%
     \tfnendorigtable%
     \ltx@ifpackageloaded{footnotebackref}{%
       \LetLtxMacro{\tfncurrentmft}{\@makefntext}%
       \LetLtxMacro{\@makefntext}{\tablemakefntext}%
        \tfn@tablefootnoteprintout%
       \LetLtxMacro{\@makefntext}{\tfncurrentmft}%
      }{% footnotebackref not loaded
       \tfn@tablefootnoteprintout%
      }%
     \gdef\tfn@fnt{0}%
    }%
%    \end{macrocode}
%
% \pagebreak
%
% The |sidewaystable|-environment must be treated separately:
%
%    \begin{macrocode}
  \ltx@ifpackageloaded{rotating}{%
    \let\tfnorigsidewaystable\sidewaystable
    \let\tfnendorigsidewaystable\endsidewaystable
    \renewenvironment{sidewaystable}{%
      \gdef\tfn@swt{1}%
      \tfnorigsidewaystable%
    }{%
      \ifx\tfn@fnt\tfn@footnotezero% \relax
      \else%
        \ltx@ifpackageloaded{footnotebackref}{%
          \LetLtxMacro{\tfncurrentmft}{\@makefntext}%
          \LetLtxMacro{\@makefntext}{\tablemakefntext}%
           \tfn@swtbox%
          \LetLtxMacro{\@makefntext}{\tfncurrentmft}%
         }{% footnotebackref not loaded
          \tfn@swtbox%
         }%
      \fi%
      \gdef\tfn@swt{0}%
      \gdef\tfn@fnt{0}%
      \tfnendorigsidewaystable%
     }%
   }{}%
}

%    \end{macrocode}
%
% \DescribeMacro{\AtBeginDocument}
% When the \xpackage{rotating} (for |sidewaystable|s) or the
% \xpackage{footnotebackref} (for back references for the footnotes)
% package is used, it must be loaded before the \xpackage{tablefootnote} package.
% Using both packages is also possible, in which case both must be loaded before
% the \xpackage{tablefootnote} package. If any one of those packages is loaded
% |\AtBeginDocument| after this check, the according error message will not be
% given, but packages really should be loaded before |\AtBeginDocument|.
%
%    \begin{macrocode}
\AtBeginDocument{%
  \ltx@ifpackageloaded{footnotebackref}{%
    \ifx\tfn@fnbr\tfn@footnotezero%
      \PackageError{tablefootnote}{%
        Package footnotebackref loaded after tablefootnote}{%
        When using both the footnotebackref and the tablefootnote %
        package,\MessageBreak%
        the footnotebackref package MUST be loaded before the %
        tablefootnote package!\MessageBreak%
       }%
    \fi%
   }{% else \relax
   }
  \ltx@ifpackageloaded{rotating}{%
    \ifx\tfn@rotating\tfn@footnotezero%
      \PackageError{tablefootnote}{%
        Package rotating loaded after tablefootnote}{%
        When using both the rotating and the tablefootnote package,%
        \MessageBreak%
        the rotating package MUST be loaded before the tablefootnote%
        \space package!\MessageBreak%
       }%
    \fi%
   }{% else \relax
   }
%    \end{macrocode}
%
% \pagebreak
%
% When the document is compiled with \LuaLaTeX{}, hyperlinks in rotated content
% will be misplaced, regardless of the use of the \xpackage{tablefootnote} package
% (or of tables or footnotes). The text to be printed will be OK.
% This problem of \LuaLaTeX{} (not \xpackage{tablefootnote}) was found and reported
% by \textsc{Arno Trautmann}, 2011/11/15 -- thanks!
%
%    \begin{macrocode}
  \ltx@ifpackageloaded{hyperref}{%
    \tfn@hyperfootnotes{
      \ifluatex%
        \PackageWarningNoLine{tablefootnote}{%
          LuaTeX and hyperref package used:\MessageBreak%
          Hyperreferences in rotated content\MessageBreak%
          will be misplaced\MessageBreak%
          regardless of tablefootnote package.\MessageBreak%
          The text to be printed will be OK%
         }%
      \fi%
     }{% else \relax
     }%
   }{% else \relax
   }%
%    \end{macrocode}
%
% When the \xpackage{footnotebackref} package is used but the footnotes are not
% hyperlinked, neither the back references nor the tablefootnotes will work.
%
%    \begin{macrocode}
  \ltx@ifpackageloaded{footnotebackref}{%
     \ltx@ifpackageloaded{hyperref}{%
        \tfn@hyperfootnotes{\relax}{% else
          \PackageError{tablefootnote}{%
             footnotebackref but not hyperfootnotes}{%
             The footnotebackref package is used\MessageBreak%
             but the hyperfootnotes option of the hyperref package%
             \MessageBreak%
             was set to false.\MessageBreak%
             Either enable hyperfootnotes or do not use footnotebackref.%
             \MessageBreak%
             Loading of the tablefootnote package will be aborted.%
           }%
          \endinput%
      }%
    }{\PackageError{tablefootnote}{footnotebackref but not hyperref}{%
        The footnotebackref package is used\MessageBreak%
        but not the hyperref package.\MessageBreak%
        Either load hyperref or do not use footnotebackref.%
        \MessageBreak%
        Loading of the tablefootnote package will be aborted.%
        }%
      \endinput%
    }%
  }{}
}

%    \end{macrocode}
%
%    \begin{macrocode}
%</package>
%    \end{macrocode}
%
% \pagebreak
%
% \section{Installation}
%
% \subsection{Downloads\label{ss:Downloads}}
%
% Everything is available on \CTAN{}, \url{http://www.ctan.org/},
% but may need additional packages themselves.\\
%
% \DescribeMacro{tablefootnote.dtx}
% For unpacking the |tablefootnote.dtx| file and constructing the documentation
% it is required:
% \begin{description}
% \item[-] \TeX{} Format \LaTeXe{}: \url{http://www.CTAN.org/}
%
% \item[-] document class \xpackage{ltxdoc}, 2007/11/11, v2.0u,
%           \url{http://www.ctan.org/pkg/ltxdoc}
%
% \item[-] package \xpackage{holtxdoc}, 2012/03/21, v0.24,
%           \url{http://www.ctan.org/pkg/holtxdoc}
%
% \item[-] package \xpackage{hypdoc}, 2011/08/19, v1.11,
%           \url{http://www.ctan.org/pkg/hypdoc}
% \end{description}
%
% \DescribeMacro{tablefootnote.sty}
% The |tablefootnote.sty| for \LaTeXe{} (i.\,e.~each document using
% the \xpackage{tablefootnote} package) requires:
% \begin{description}
% \item[-] \TeX{} Format \LaTeXe{}, \url{http://www.CTAN.org/}
%
% \item[-] package \xpackage{letltxmacro}, 2010/09/02, v1.4,
%           \url{http://www.ctan.org/pkg/letltxmacro}.
%
% \item[-] package \xpackage{ltxcmds}, 2011/11/09, v1.22,
%           \url{http://www.ctan.org/pkg/ltxcmds}
% \end{description}
%
% When the \xpackage{hyperref} package is used, also the \xpackage{ifluatex} package
% is needed, but it is already loaded automatically by the \xpackage{hyperref}
% package.\\
% When the \xpackage{etoolbox} package is available, it is used:
% \begin{description}
% \item[-] package \xpackage{etoolbox}, 2011/01/03, v2.1,
%           \url{http://www.ctan.org/pkg/etoolbox}.
% \end{description}
%
% \DescribeMacro{tablefootnote-example.tex}
% The |tablefootnote-example.tex| requires the same files as all
% documents using the \xpackage{tablefootnote} package and additionally:
% \begin{description}
% \item[-] class \xpackage{article}, 2007/10/19, v1.4h, from \xpackage{classes.dtx}:\\
%           \CTAN{macros/latex/base/classes.dtx}
%
% \item[-] package \xpackage{rotating}, 2009/03/28, v2.16a,
%           \url{http://www.ctan.org/pkg/rotating}
%
% \item[-] package \xpackage{float}, 2001/11/08, v1.3d,
%           \url{http://www.ctan.org/pkg/float}
%
% \item[-] package \xpackage{placeins}, 2005/04/18, v2.2,
%           \url{http://www.ctan.org/pkg/placeins}\\
%           for |\FloatBarrier|
%
% \item[-] package \xpackage{hyperref}, 2012/11/06, v6.83m,
%           \url{http://www.ctan.org/pkg/hyperref}
%
% \item[-] package \xpackage{tablefootnote}, 2014/01/26, v1.1c,\\
%           \url{http://www.ctan.org/pkg/tablefootnote}\\
%   (Well, it is the example file for this package, and because you
%    are reading the documentation for the \xpackage{tablefootnote} package,
%    it can be assumed that you already have some version of it~--
%    is it the current one?)
% \end{description}
%
% \DescribeMacro{Oberdiek}
% \DescribeMacro{holtxdoc}
% \DescribeMacro{ifluatex}
% \DescribeMacro{letltxmacro}
% \DescribeMacro{ltxcmds}
% All packages of \textsc{Heiko Oberdiek}'s bundle `oberdiek'
% (especially \xpackage{holtxdoc}, \xpackage{ifluatex}, \xpackage{letltxmacro},
% and \xpackage{ltxcmds}) are also available in a TDS compliant ZIP archive:\\
% \CTAN{install/macros/latex/contrib/oberdiek.tds.zip}.\\
% It is probably best to download and use this, because the packages in there
% are quite probably both recent and compatible among themselves.\\
%
% \pagebreak
%
% \DescribeMacro{hyperref}
% \noindent \xpackage{hyperref} is not included in that bundle and needs to be
% downloaded separately,\\
% \url{http://www.ctan.org/install/macros/latex/contrib/hyperref.tds.zip}.\\
%
% \DescribeMacro{M\"{u}nch}
% A hyperlinked list of my (other) packages can be found at
% \url{http://www.ctan.org/author/muench-hm}.\\
%
% \subsection{Package, unpacking TDS}
% \paragraph{Package.} This package is available on \CTAN{}
% \begin{description}
% \item[\CTAN{macros/latex/contrib/tablefootnote/tablefootnote.dtx}]\hspace*{0.1cm} \\
%       The source file.
% \item[\CTAN{macros/latex/contrib/tablefootnote/tablefootnote.pdf}]\hspace*{0.1cm} \\
%       The documentation.
% \item[\CTAN{macros/latex/contrib/tablefootnote/tablefootnote-example.pdf}]\hspace*{0.1cm} \\
%       The compiled example file, as it should look like.
% \item[\CTAN{macros/latex/contrib/tablefootnote/README}]\hspace*{0.1cm} \\
%       The README file.
% \end{description}
% There is also a tablefootnote.tds.zip available:
% \begin{description}
% \item[\CTAN{install/macros/latex/contrib/tablefootnote.tds.zip}]\hspace*{0.1cm} \\
%       Everything in \xfile{TDS} compliant, compiled format.
% \end{description}
% which additionally contains\\
% \begin{tabular}{ll}
% tablefootnote.ins & The installation file.\\
% tablefootnote.drv & The driver to generate the documentation.\\
% tablefootnote.sty & The \xext{sty}le file.\\
% tablefootnote-example.tex & The example file.
% \end{tabular}
%
% \bigskip
%
% \noindent For required other packages, please see the preceding subsection.
%
% \paragraph{Unpacking.} The \xfile{.dtx} file is a self-extracting
% \docstrip{} archive. The files are extracted by running the
% \xext{dtx} through \plainTeX{}:
% \begin{quote}
%   \verb|tex tablefootnote.dtx|
% \end{quote}
%
% \noindent About generating the documentation see paragraph~\ref{GenDoc} below.\\
%
% \paragraph{TDS.} Now the different files must be moved into
% the different directories in your installation TDS tree
% (also known as \xfile{texmf} tree):
% \begin{quote}
% \def\t{^^A
% \begin{tabular}{@{}>{\ttfamily}l@{ $\rightarrow$ }>{\ttfamily}l@{}}
%   tablefootnote.sty & tex/latex/tablefootnote/tablefootnote.sty\\
%   tablefootnote.pdf & doc/latex/tablefootnote/tablefootnote.pdf\\
%   tablefootnote-example.tex & doc/latex/tablefootnote/tablefootnote-example.tex\\
%   tablefootnote-example.pdf & doc/latex/tablefootnote/tablefootnote-example.pdf\\
%   tablefootnote.dtx & source/latex/tablefootnote/tablefootnote.dtx\\
% \end{tabular}^^A
% }^^A
% \sbox0{\t}^^A
% \ifdim\wd0>\linewidth
%   \begingroup
%     \advance\linewidth by\leftmargin
%     \advance\linewidth by\rightmargin
%   \edef\x{\endgroup
%     \def\noexpand\lw{\the\linewidth}^^A
%   }\x
%   \def\lwbox{^^A
%     \leavevmode
%     \hbox to \linewidth{^^A
%       \kern-\leftmargin\relax
%       \hss
%       \usebox0
%       \hss
%       \kern-\rightmargin\relax
%     }^^A
%   }^^A
%   \ifdim\wd0>\lw
%     \sbox0{\small\t}^^A
%     \ifdim\wd0>\linewidth
%       \ifdim\wd0>\lw
%         \sbox0{\footnotesize\t}^^A
%         \ifdim\wd0>\linewidth
%           \ifdim\wd0>\lw
%             \sbox0{\scriptsize\t}^^A
%             \ifdim\wd0>\linewidth
%               \ifdim\wd0>\lw
%                 \sbox0{\tiny\t}^^A
%                 \ifdim\wd0>\linewidth
%                   \lwbox
%                 \else
%                   \usebox0
%                 \fi
%               \else
%                 \lwbox
%               \fi
%             \else
%               \usebox0
%             \fi
%           \else
%             \lwbox
%           \fi
%         \else
%           \usebox0
%         \fi
%       \else
%         \lwbox
%       \fi
%     \else
%       \usebox0
%     \fi
%   \else
%     \lwbox
%   \fi
% \else
%   \usebox0
% \fi
% \end{quote}
% If you have a \xfile{docstrip.cfg} that configures and enables \docstrip{}'s
% \xfile{TDS} installing feature, then some files can already be in the right
% place, see the documentation of \docstrip{}.
%
% \subsection{Refresh file name databases}
%
% If your \TeX{}~distribution (\TeX live, \mikTeX{}, \dots{}) relies on file name
% databases, you must refresh these. For example, \teTeX{} users run
% \verb|texhash| or \verb|mktexlsr|.
%
% \subsection{Some details for the interested}
%
% \paragraph{Unpacking with \LaTeX{}.}
% The \xfile{.dtx} chooses its action depending on the format:
% \begin{description}
% \item[\plainTeX:] Run \docstrip{} and extract the files.
% \item[\LaTeX:] Generate the documentation.
% \end{description}
% If you insist on using \LaTeX{} for \docstrip{} (really,
% \docstrip{} does not need \LaTeX{}), then inform the autodetect routine
% about your intention:
% \begin{quote}
%   \verb|latex \let\install=y\input{tablefootnote.dtx}|
% \end{quote}
% Do not forget to quote the argument according to the demands
% of your shell.
%
% \paragraph{Generating the documentation.\label{GenDoc}}
% You can use both the \xfile{.dtx} or the \xfile{.drv} to generate
% the documentation. The process can be configured by a
% configuration file \xfile{ltxdoc.cfg}. For instance, put the following
% line into this file, if you want to have A4 as paper format:
% \begin{quote}
%   \verb|\PassOptionsToClass{a4paper}{article}|
% \end{quote}
%
% \noindent An example follows how to generate the
% documentation with \pdfLaTeX{}:
%
% \begin{quote}
%\begin{verbatim}
%pdflatex tablefootnote.dtx
%makeindex -s gind.ist tablefootnote.idx
%pdflatex tablefootnote.dtx
%makeindex -s gind.ist tablefootnote.idx
%pdflatex tablefootnote.dtx
%\end{verbatim}
% \end{quote}
%
% \subsection{Compiling the example}
%
% The example file, \textsf{tablefootnote-example.tex}, can be compiled via\\
% \indent |latex tablefootnote-example.tex|,\\
% \indent |lualatex tablefootnote-example.tex|,\\
% or (recommended)\\
% \indent |pdflatex thumbs-example.tex|\\
% and will need at least two compiler runs to get everything right.
%
% \bigskip
%
% \section{Acknowledgements}
%
% I would like to thank \textsc{Heiko Oberdiek} for providing
% the \xpackage{hyperref}, \xpackage{ifluatex}, \xpackage{letltxmacro},
% \xpackage{ltxcmds}, as well as a~lot~(!) of other useful packages
% (from which I also got everything I know about creating a file in
% \xext{dtx} format, OK, say it: copying),
% and for information about the \pdfTeX{}-bug,
% \textsc{Arno Trautmann} for reporting the bug of misplaced hyperreferences in
% |sidewaystable|s using \LuaTeX{}, independently of the \xpackage{tablefootnote}
% package or footnotes at all,
% \textsc{Thomas~V.} for reporting the bug of not handling the case of
% \xpackage{hyperref} option |hyperfootnotes=false|,
% \textsc{Sverre Stausland Johnsen} for reporting the bug in case of \emph{not}
% using \xpackage{hyperref},
% and the \Newsgroup{comp.text.tex} and \Newsgroup{de.comp.text.tex} newsgroups
% as well as everybody at \url{http://tex.stackexchange.com}
% for their help in all things \TeX{}.
%
% \pagebreak
%
% \phantomsection
% \begin{History}\label{History}
%   \begin{Version}{2011/10/26 v1.0a, tabfootn}
%     \item Upload to \CTAN{macros/latex/contrib/tablefootnote/}.
%   \end{Version}
%   \begin{Version}{2011/10/29 v1.0b}
%     \item Renamed to \xpackage{tablefootnote}.
%     \item Added support for the |sidewaystable|-environment of the
%             \xpackage{rotating} package.
%     \item Diverse changes in documentation and README.
%   \end{Version}
%   \begin{Version}{2011/11/06 v1.0c}
%     \item Replaced |^{\text{\thefootnote}}| by |{}^{\text{\textup{\thefootnote}}}|.
%     \item Bug fix: When \xpackage{etoolbox.sty} was not found, and the \xpackage{float}
%             package had been loaded, and the float specifier was \texttt{H},
%             then the |\tablefootnote|s were not printed for that table. Fixed.
%   \end{Version}
%   \begin{Version}{2011/11/19 v1.0d}
%     \item Replaced |{}^{\text{\textup{\thefootnote}}}| by
%             |\textsuperscript{\thefootnote}|, therefore the \xpackage{amstext}
%             package is no longer required by the \xpackage{tablefootnote} package.
%     \item \xpackage{lscape} and \xpackage{pdflscape} package now work with
%             \xpackage{tablefootnote} (except for using a \texttt{sidewaystable} on
%             a \texttt{landscape} page).
%     \item Added information about footnotes in \texttt{sidewaystable} being printed
%             one below the other, even if \xpackage{footmisc} package with option
%             \texttt{para} is used.
%     \item Redefined |\tfn@swtbox| from
%  \begin{verbatim}
%  \vspace{0.8cm}%
%  \begin{spacing}{0.1}%
%  \subcaptionbox*{}[\linewidth][l]{\tfn@tablefootnoteprintout}%
%  \end{spacing}%
%  \end{verbatim}
%             to
%  \begin{verbatim}
%  \vspace{2\baselineskip}%
%  \parbox[b][][t]{\linewidth}{\raggedright\tfn@tablefootnoteprintout}%
%  \end{verbatim}
%             , therefore the \xpackage{setspace} and \xpackage{subcaption} packages
%             are no longer required by the \xpackage{tablefootnote} package.
%     \item No longer uses |\numexpr|, therefore also usable without \eTeX{}.
%     \item Replaced |\textsuperscript{\thefootnote}| by
%             |\hbox {\@textsuperscript {\normalfont \thefootnote }}|,
%             which is next to the original definition of |\@makefnmark|
%             (|\hbox {\@textsuperscript {\normalfont \@thefnmark }}|,
%             but according to |\show\@thefnmark| here |\@thefnmark| is
%             |{\protect \itshape}| instead of |\thefootnote|).
%     \item Added an error message, when \LuaTeX{}, \xpackage{rotating} package, and
%            \xpackage{hyperref} package are used together. Hyperreferences in
%            |sidewaystable|s are misplaced, independently of the
%            \xpackage{tablefootnote} package or footnotes at all. (Bug reported
%            by \textsc{Arno Trautmann}, 2011/11/15 -- thanks!)
%     \item Added footnotesymbol-footnote-numbers in the example.
%   \end{Version}
%   \begin{Version}{2011/11/26 v1.0e}
%     \item Check for \xpackage{hyperref} is only done once.
%     \item Message regarding the misplacement of hyperlinks by \LuaLaTeX{}
%             changed form error to warning.
%     \item Replaced |\vspace{2\baselineskip}| by
%  \begin{verbatim}
% \ifvoid \footins%
% \vskip 2\baselineskip%
% \else%
% \vskip \skip\footins%
% \fi%
%  \end{verbatim}
%             between the |sidewaystable|s and their footnotes.
%     \item The warning about un-regarded |para|-option of \xpackage{footmoisc}
%             package is only given at the second |tablefootnote| in a |sidewaystable|,
%             not repeatedly for every |tablefootnote| in a |sidewaystable|.
%   \end{Version}
%   \begin{Version}{2012/01/01 v1.0f}
%     \item Bugfix: Wrong installation path given in the documentation, fixed.
%     \item Update of documentation, README, and \xfile{dtx} internals.
%   \end{Version}
%   \begin{Version}{2012/01/14 v1.0g}
%     \item Bugfix: There was a |relax| instead of a |\relax|, fixed.
%   \end{Version}
%   \begin{Version}{2012/07/29 v1.0h}
%     \item \xpackage{tablefootnote} is now compatible to the new
%             \xpackage{footnotebackref} package, 2012/07/01, v1.0,
%             \url{http://www.ctan.org/pkg/footnotebackref}.
%   \end{Version}
%   \begin{Version}{2014/01/08 v1.1a}
%     \item Tablefootnotes now regard optional footnote marks:
%             |\tablefootnote[| custom mark here! |]{...}|
%     \item Bug fix: the case of \xpackage{hyperref} option |hyperfootnotes=false|
%             was not handeled at all. (Bug reported by \textsc{Thomas~V.}, %
%             2014/01/02 -- thanks!)
%     \item A lot of details.
%   \end{Version}
%   \begin{Version}{2014/01/20 v1.1b}
%     \item Bug fix: Package was broken when \xpackage{hyperref} was \emph{not} used.
%            (Bug reported by \textsc{Sverre Stausland Johnsen} -- thanks!)
%   \end{Version}
%   \begin{Version}{2014/01/26 v1.1c}
%     \item The emergency bug fix of version 2014/01/20 v1.1b was replaced by using
%             the new robust macro |\tfn@hyperfootnotes|, which only \emph{internally}
%             uses |\ifHy@hyperfootnotes|. If the \xpackage{hyperref} package is not used,
%             \LaTeX{} now does not see any confusing
%             |\ifHy@hyperfootnotes|\ldots|\else|\ldots|\fi| at all.
%     \item URLs and README updated.
%   \end{Version}
% \end{History}
%
% \bigskip
%
% When you find a mistake or have a suggestion for an improvement of this package,
% please send an e-mail to the maintainer, thanks! (Please see BUG REPORTS in the README.)
%
% \bigskip
%
% \PrintIndex
%
% \Finale
\endinput|
% \end{quote}
% Do not forget to quote the argument according to the demands
% of your shell.
%
% \paragraph{Generating the documentation.\label{GenDoc}}
% You can use both the \xfile{.dtx} or the \xfile{.drv} to generate
% the documentation. The process can be configured by a
% configuration file \xfile{ltxdoc.cfg}. For instance, put the following
% line into this file, if you want to have A4 as paper format:
% \begin{quote}
%   \verb|\PassOptionsToClass{a4paper}{article}|
% \end{quote}
%
% \noindent An example follows how to generate the
% documentation with \pdfLaTeX{}:
%
% \begin{quote}
%\begin{verbatim}
%pdflatex tablefootnote.dtx
%makeindex -s gind.ist tablefootnote.idx
%pdflatex tablefootnote.dtx
%makeindex -s gind.ist tablefootnote.idx
%pdflatex tablefootnote.dtx
%\end{verbatim}
% \end{quote}
%
% \subsection{Compiling the example}
%
% The example file, \textsf{tablefootnote-example.tex}, can be compiled via\\
% \indent |latex tablefootnote-example.tex|,\\
% \indent |lualatex tablefootnote-example.tex|,\\
% or (recommended)\\
% \indent |pdflatex thumbs-example.tex|\\
% and will need at least two compiler runs to get everything right.
%
% \bigskip
%
% \section{Acknowledgements}
%
% I would like to thank \textsc{Heiko Oberdiek} for providing
% the \xpackage{hyperref}, \xpackage{ifluatex}, \xpackage{letltxmacro},
% \xpackage{ltxcmds}, as well as a~lot~(!) of other useful packages
% (from which I also got everything I know about creating a file in
% \xext{dtx} format, OK, say it: copying),
% and for information about the \pdfTeX{}-bug,
% \textsc{Arno Trautmann} for reporting the bug of misplaced hyperreferences in
% |sidewaystable|s using \LuaTeX{}, independently of the \xpackage{tablefootnote}
% package or footnotes at all,
% \textsc{Thomas~V.} for reporting the bug of not handling the case of
% \xpackage{hyperref} option |hyperfootnotes=false|,
% \textsc{Sverre Stausland Johnsen} for reporting the bug in case of \emph{not}
% using \xpackage{hyperref},
% and the \Newsgroup{comp.text.tex} and \Newsgroup{de.comp.text.tex} newsgroups
% as well as everybody at \url{http://tex.stackexchange.com}
% for their help in all things \TeX{}.
%
% \pagebreak
%
% \phantomsection
% \begin{History}\label{History}
%   \begin{Version}{2011/10/26 v1.0a, tabfootn}
%     \item Upload to \CTAN{macros/latex/contrib/tablefootnote/}.
%   \end{Version}
%   \begin{Version}{2011/10/29 v1.0b}
%     \item Renamed to \xpackage{tablefootnote}.
%     \item Added support for the |sidewaystable|-environment of the
%             \xpackage{rotating} package.
%     \item Diverse changes in documentation and README.
%   \end{Version}
%   \begin{Version}{2011/11/06 v1.0c}
%     \item Replaced |^{\text{\thefootnote}}| by |{}^{\text{\textup{\thefootnote}}}|.
%     \item Bug fix: When \xpackage{etoolbox.sty} was not found, and the \xpackage{float}
%             package had been loaded, and the float specifier was \texttt{H},
%             then the |\tablefootnote|s were not printed for that table. Fixed.
%   \end{Version}
%   \begin{Version}{2011/11/19 v1.0d}
%     \item Replaced |{}^{\text{\textup{\thefootnote}}}| by
%             |\textsuperscript{\thefootnote}|, therefore the \xpackage{amstext}
%             package is no longer required by the \xpackage{tablefootnote} package.
%     \item \xpackage{lscape} and \xpackage{pdflscape} package now work with
%             \xpackage{tablefootnote} (except for using a \texttt{sidewaystable} on
%             a \texttt{landscape} page).
%     \item Added information about footnotes in \texttt{sidewaystable} being printed
%             one below the other, even if \xpackage{footmisc} package with option
%             \texttt{para} is used.
%     \item Redefined |\tfn@swtbox| from
%  \begin{verbatim}
%  \vspace{0.8cm}%
%  \begin{spacing}{0.1}%
%  \subcaptionbox*{}[\linewidth][l]{\tfn@tablefootnoteprintout}%
%  \end{spacing}%
%  \end{verbatim}
%             to
%  \begin{verbatim}
%  \vspace{2\baselineskip}%
%  \parbox[b][][t]{\linewidth}{\raggedright\tfn@tablefootnoteprintout}%
%  \end{verbatim}
%             , therefore the \xpackage{setspace} and \xpackage{subcaption} packages
%             are no longer required by the \xpackage{tablefootnote} package.
%     \item No longer uses |\numexpr|, therefore also usable without \eTeX{}.
%     \item Replaced |\textsuperscript{\thefootnote}| by
%             |\hbox {\@textsuperscript {\normalfont \thefootnote }}|,
%             which is next to the original definition of |\@makefnmark|
%             (|\hbox {\@textsuperscript {\normalfont \@thefnmark }}|,
%             but according to |\show\@thefnmark| here |\@thefnmark| is
%             |{\protect \itshape}| instead of |\thefootnote|).
%     \item Added an error message, when \LuaTeX{}, \xpackage{rotating} package, and
%            \xpackage{hyperref} package are used together. Hyperreferences in
%            |sidewaystable|s are misplaced, independently of the
%            \xpackage{tablefootnote} package or footnotes at all. (Bug reported
%            by \textsc{Arno Trautmann}, 2011/11/15 -- thanks!)
%     \item Added footnotesymbol-footnote-numbers in the example.
%   \end{Version}
%   \begin{Version}{2011/11/26 v1.0e}
%     \item Check for \xpackage{hyperref} is only done once.
%     \item Message regarding the misplacement of hyperlinks by \LuaLaTeX{}
%             changed form error to warning.
%     \item Replaced |\vspace{2\baselineskip}| by
%  \begin{verbatim}
% \ifvoid \footins%
% \vskip 2\baselineskip%
% \else%
% \vskip \skip\footins%
% \fi%
%  \end{verbatim}
%             between the |sidewaystable|s and their footnotes.
%     \item The warning about un-regarded |para|-option of \xpackage{footmoisc}
%             package is only given at the second |tablefootnote| in a |sidewaystable|,
%             not repeatedly for every |tablefootnote| in a |sidewaystable|.
%   \end{Version}
%   \begin{Version}{2012/01/01 v1.0f}
%     \item Bugfix: Wrong installation path given in the documentation, fixed.
%     \item Update of documentation, README, and \xfile{dtx} internals.
%   \end{Version}
%   \begin{Version}{2012/01/14 v1.0g}
%     \item Bugfix: There was a |relax| instead of a |\relax|, fixed.
%   \end{Version}
%   \begin{Version}{2012/07/29 v1.0h}
%     \item \xpackage{tablefootnote} is now compatible to the new
%             \xpackage{footnotebackref} package, 2012/07/01, v1.0,
%             \url{http://www.ctan.org/pkg/footnotebackref}.
%   \end{Version}
%   \begin{Version}{2014/01/08 v1.1a}
%     \item Tablefootnotes now regard optional footnote marks:
%             |\tablefootnote[| custom mark here! |]{...}|
%     \item Bug fix: the case of \xpackage{hyperref} option |hyperfootnotes=false|
%             was not handeled at all. (Bug reported by \textsc{Thomas~V.}, %
%             2014/01/02 -- thanks!)
%     \item A lot of details.
%   \end{Version}
%   \begin{Version}{2014/01/20 v1.1b}
%     \item Bug fix: Package was broken when \xpackage{hyperref} was \emph{not} used.
%            (Bug reported by \textsc{Sverre Stausland Johnsen} -- thanks!)
%   \end{Version}
%   \begin{Version}{2014/01/26 v1.1c}
%     \item The emergency bug fix of version 2014/01/20 v1.1b was replaced by using
%             the new robust macro |\tfn@hyperfootnotes|, which only \emph{internally}
%             uses |\ifHy@hyperfootnotes|. If the \xpackage{hyperref} package is not used,
%             \LaTeX{} now does not see any confusing
%             |\ifHy@hyperfootnotes|\ldots|\else|\ldots|\fi| at all.
%     \item URLs and README updated.
%   \end{Version}
% \end{History}
%
% \bigskip
%
% When you find a mistake or have a suggestion for an improvement of this package,
% please send an e-mail to the maintainer, thanks! (Please see BUG REPORTS in the README.)
%
% \bigskip
%
% \PrintIndex
%
% \Finale
\endinput|
% \end{quote}
% Do not forget to quote the argument according to the demands
% of your shell.
%
% \paragraph{Generating the documentation.\label{GenDoc}}
% You can use both the \xfile{.dtx} or the \xfile{.drv} to generate
% the documentation. The process can be configured by a
% configuration file \xfile{ltxdoc.cfg}. For instance, put the following
% line into this file, if you want to have A4 as paper format:
% \begin{quote}
%   \verb|\PassOptionsToClass{a4paper}{article}|
% \end{quote}
%
% \noindent An example follows how to generate the
% documentation with \pdfLaTeX{}:
%
% \begin{quote}
%\begin{verbatim}
%pdflatex tablefootnote.dtx
%makeindex -s gind.ist tablefootnote.idx
%pdflatex tablefootnote.dtx
%makeindex -s gind.ist tablefootnote.idx
%pdflatex tablefootnote.dtx
%\end{verbatim}
% \end{quote}
%
% \subsection{Compiling the example}
%
% The example file, \textsf{tablefootnote-example.tex}, can be compiled via\\
% \indent |latex tablefootnote-example.tex|,\\
% \indent |lualatex tablefootnote-example.tex|,\\
% or (recommended)\\
% \indent |pdflatex thumbs-example.tex|\\
% and will need at least two compiler runs to get everything right.
%
% \bigskip
%
% \section{Acknowledgements}
%
% I would like to thank \textsc{Heiko Oberdiek} for providing
% the \xpackage{hyperref}, \xpackage{ifluatex}, \xpackage{letltxmacro},
% \xpackage{ltxcmds}, as well as a~lot~(!) of other useful packages
% (from which I also got everything I know about creating a file in
% \xext{dtx} format, OK, say it: copying),
% and for information about the \pdfTeX{}-bug,
% \textsc{Arno Trautmann} for reporting the bug of misplaced hyperreferences in
% |sidewaystable|s using \LuaTeX{}, independently of the \xpackage{tablefootnote}
% package or footnotes at all,
% \textsc{Thomas~V.} for reporting the bug of not handling the case of
% \xpackage{hyperref} option |hyperfootnotes=false|,
% \textsc{Sverre Stausland Johnsen} for reporting the bug in case of \emph{not}
% using \xpackage{hyperref},
% and the \Newsgroup{comp.text.tex} and \Newsgroup{de.comp.text.tex} newsgroups
% as well as everybody at \url{http://tex.stackexchange.com}
% for their help in all things \TeX{}.
%
% \pagebreak
%
% \phantomsection
% \begin{History}\label{History}
%   \begin{Version}{2011/10/26 v1.0a, tabfootn}
%     \item Upload to \CTAN{macros/latex/contrib/tablefootnote/}.
%   \end{Version}
%   \begin{Version}{2011/10/29 v1.0b}
%     \item Renamed to \xpackage{tablefootnote}.
%     \item Added support for the |sidewaystable|-environment of the
%             \xpackage{rotating} package.
%     \item Diverse changes in documentation and README.
%   \end{Version}
%   \begin{Version}{2011/11/06 v1.0c}
%     \item Replaced |^{\text{\thefootnote}}| by |{}^{\text{\textup{\thefootnote}}}|.
%     \item Bug fix: When \xpackage{etoolbox.sty} was not found, and the \xpackage{float}
%             package had been loaded, and the float specifier was \texttt{H},
%             then the |\tablefootnote|s were not printed for that table. Fixed.
%   \end{Version}
%   \begin{Version}{2011/11/19 v1.0d}
%     \item Replaced |{}^{\text{\textup{\thefootnote}}}| by
%             |\textsuperscript{\thefootnote}|, therefore the \xpackage{amstext}
%             package is no longer required by the \xpackage{tablefootnote} package.
%     \item \xpackage{lscape} and \xpackage{pdflscape} package now work with
%             \xpackage{tablefootnote} (except for using a \texttt{sidewaystable} on
%             a \texttt{landscape} page).
%     \item Added information about footnotes in \texttt{sidewaystable} being printed
%             one below the other, even if \xpackage{footmisc} package with option
%             \texttt{para} is used.
%     \item Redefined |\tfn@swtbox| from
%  \begin{verbatim}
%  \vspace{0.8cm}%
%  \begin{spacing}{0.1}%
%  \subcaptionbox*{}[\linewidth][l]{\tfn@tablefootnoteprintout}%
%  \end{spacing}%
%  \end{verbatim}
%             to
%  \begin{verbatim}
%  \vspace{2\baselineskip}%
%  \parbox[b][][t]{\linewidth}{\raggedright\tfn@tablefootnoteprintout}%
%  \end{verbatim}
%             , therefore the \xpackage{setspace} and \xpackage{subcaption} packages
%             are no longer required by the \xpackage{tablefootnote} package.
%     \item No longer uses |\numexpr|, therefore also usable without \eTeX{}.
%     \item Replaced |\textsuperscript{\thefootnote}| by
%             |\hbox {\@textsuperscript {\normalfont \thefootnote }}|,
%             which is next to the original definition of |\@makefnmark|
%             (|\hbox {\@textsuperscript {\normalfont \@thefnmark }}|,
%             but according to |\show\@thefnmark| here |\@thefnmark| is
%             |{\protect \itshape}| instead of |\thefootnote|).
%     \item Added an error message, when \LuaTeX{}, \xpackage{rotating} package, and
%            \xpackage{hyperref} package are used together. Hyperreferences in
%            |sidewaystable|s are misplaced, independently of the
%            \xpackage{tablefootnote} package or footnotes at all. (Bug reported
%            by \textsc{Arno Trautmann}, 2011/11/15 -- thanks!)
%     \item Added footnotesymbol-footnote-numbers in the example.
%   \end{Version}
%   \begin{Version}{2011/11/26 v1.0e}
%     \item Check for \xpackage{hyperref} is only done once.
%     \item Message regarding the misplacement of hyperlinks by \LuaLaTeX{}
%             changed form error to warning.
%     \item Replaced |\vspace{2\baselineskip}| by
%  \begin{verbatim}
% \ifvoid \footins%
% \vskip 2\baselineskip%
% \else%
% \vskip \skip\footins%
% \fi%
%  \end{verbatim}
%             between the |sidewaystable|s and their footnotes.
%     \item The warning about un-regarded |para|-option of \xpackage{footmoisc}
%             package is only given at the second |tablefootnote| in a |sidewaystable|,
%             not repeatedly for every |tablefootnote| in a |sidewaystable|.
%   \end{Version}
%   \begin{Version}{2012/01/01 v1.0f}
%     \item Bugfix: Wrong installation path given in the documentation, fixed.
%     \item Update of documentation, README, and \xfile{dtx} internals.
%   \end{Version}
%   \begin{Version}{2012/01/14 v1.0g}
%     \item Bugfix: There was a |relax| instead of a |\relax|, fixed.
%   \end{Version}
%   \begin{Version}{2012/07/29 v1.0h}
%     \item \xpackage{tablefootnote} is now compatible to the new
%             \xpackage{footnotebackref} package, 2012/07/01, v1.0,
%             \url{http://www.ctan.org/pkg/footnotebackref}.
%   \end{Version}
%   \begin{Version}{2014/01/08 v1.1a}
%     \item Tablefootnotes now regard optional footnote marks:
%             |\tablefootnote[| custom mark here! |]{...}|
%     \item Bug fix: the case of \xpackage{hyperref} option |hyperfootnotes=false|
%             was not handeled at all. (Bug reported by \textsc{Thomas~V.}, %
%             2014/01/02 -- thanks!)
%     \item A lot of details.
%   \end{Version}
%   \begin{Version}{2014/01/20 v1.1b}
%     \item Bug fix: Package was broken when \xpackage{hyperref} was \emph{not} used.
%            (Bug reported by \textsc{Sverre Stausland Johnsen} -- thanks!)
%   \end{Version}
%   \begin{Version}{2014/01/26 v1.1c}
%     \item The emergency bug fix of version 2014/01/20 v1.1b was replaced by using
%             the new robust macro |\tfn@hyperfootnotes|, which only \emph{internally}
%             uses |\ifHy@hyperfootnotes|. If the \xpackage{hyperref} package is not used,
%             \LaTeX{} now does not see any confusing
%             |\ifHy@hyperfootnotes|\ldots|\else|\ldots|\fi| at all.
%     \item URLs and README updated.
%   \end{Version}
% \end{History}
%
% \bigskip
%
% When you find a mistake or have a suggestion for an improvement of this package,
% please send an e-mail to the maintainer, thanks! (Please see BUG REPORTS in the README.)
%
% \bigskip
%
% \PrintIndex
%
% \Finale
\endinput
%        (quote the arguments according to the demands of your shell)
%
% Documentation:
%    (a) If tablefootnote.drv is present:
%           (pdf)latex tablefootnote.drv
%           makeindex -s gind.ist tablefootnote.idx
%           (pdf)latex tablefootnote.drv
%           makeindex -s gind.ist tablefootnote.idx
%           (pdf)latex tablefootnote.drv
%    (b) Without tablefootnote.drv:
%           (pdf)latex tablefootnote.dtx
%           makeindex -s gind.ist tablefootnote.idx
%           (pdf)latex tablefootnote.dtx
%           makeindex -s gind.ist tablefootnote.idx
%           (pdf)latex tablefootnote.dtx
%
%    The class ltxdoc loads the configuration file ltxdoc.cfg
%    if available. Here you can specify further options, e.g.
%    use DIN A4 as paper format:
%       \PassOptionsToClass{a4paper}{article}
%
% Installation:
%    TDS:tex/latex/tablefootnote/tablefootnote.sty
%    TDS:doc/latex/tablefootnote/tablefootnote.pdf
%    TDS:doc/latex/tablefootnote/tablefootnote-example.tex
%    TDS:doc/latex/tablefootnote/tablefootnote-example.pdf
%    TDS:source/latex/tablefootnote/tablefootnote.dtx
%
%<*ignore>
\begingroup
  \catcode123=1 %
  \catcode125=2 %
  \def\x{LaTeX2e}%
\expandafter\endgroup
\ifcase 0\ifx\install y1\fi\expandafter
         \ifx\csname processbatchFile\endcsname\relax\else1\fi
         \ifx\fmtname\x\else 1\fi\relax
\else\csname fi\endcsname
%</ignore>
%<*install>
\input docstrip.tex
\Msg{******************************************************************}
\Msg{* Installation                                                   *}
\Msg{* Package: tablefootnote 2014/01/26 v1.1c Table foot notes (HMM) *}
\Msg{******************************************************************}

\keepsilent
\askforoverwritefalse

\let\MetaPrefix\relax
\preamble

This is a generated file.

Project: tablefootnote
Version: 2014/01/26 v1.1c

Copyright (C) 2011 - 2014 by
    H.-Martin M"unch <Martin dot Muench at Uni-Bonn dot de>

The usual disclaimer applies:
If it doesn't work right that's your problem.
(Nevertheless, send an e-mail to the maintainer
 when you find an error in this package.)

This work may be distributed and/or modified under the
conditions of the LaTeX Project Public License, either
version 1.3c of this license or (at your option) any later
version. This version of this license is in
   http://www.latex-project.org/lppl/lppl-1-3c.txt
and the latest version of this license is in
   http://www.latex-project.org/lppl.txt
and version 1.3c or later is part of all distributions of
LaTeX version 2005/12/01 or later.

This work has the LPPL maintenance status "maintained".

The Current Maintainer of this work is H.-Martin Muench.

This work consists of the main source file tablefootnote.dtx,
the README, and the derived files
   tablefootnote.sty, tablefootnote.pdf,
   tablefootnote.ins, tablefootnote.drv,
   tablefootnote-example.tex, tablefootnote-example.pdf.

In memoriam Tommy Muench + 2014/01/02.

\endpreamble
\let\MetaPrefix\DoubleperCent

\generate{%
  \file{tablefootnote.ins}{\from{tablefootnote.dtx}{install}}%
  \file{tablefootnote.drv}{\from{tablefootnote.dtx}{driver}}%
  \usedir{tex/latex/tablefootnote}%
  \file{tablefootnote.sty}{\from{tablefootnote.dtx}{package}}%
  \usedir{doc/latex/tablefootnote}%
  \file{tablefootnote-example.tex}{\from{tablefootnote.dtx}{example}}%
}

\catcode32=13\relax% active space
\let =\space%
\Msg{************************************************************************}
\Msg{*}
\Msg{* To finish the installation you have to move the following}
\Msg{* file into a directory searched by TeX:}
\Msg{*}
\Msg{*     tablefootnote.sty}
\Msg{*}
\Msg{* To produce the documentation run the file `tablefootnote.drv'}
\Msg{* through (pdf)LaTeX, e.g.}
\Msg{*  pdflatex tablefootnote.drv}
\Msg{*  makeindex -s gind.ist tablefootnote.idx}
\Msg{*  pdflatex tablefootnote.drv}
\Msg{*  makeindex -s gind.ist tablefootnote.idx}
\Msg{*  pdflatex tablefootnote.drv}
\Msg{*}
\Msg{* At least three runs are necessary e.g. to get the}
\Msg{*  references right!}
\Msg{*}
\Msg{* Happy TeXing!}
\Msg{*}
\Msg{************************************************************************}

\endbatchfile
%</install>
%<*ignore>
\fi
%</ignore>
%
% \section{The documentation driver file}
%
% The next bit of code contains the documentation driver file for
% \TeX{}, i.\,e., the file that will produce the documentation you
% are currently reading. It will be extracted from this file by the
% \texttt{docstrip} programme. That is, run \LaTeX{} on \texttt{docstrip}
% and specify the \texttt{driver} option when \texttt{docstrip}
% asks for options.
%
%    \begin{macrocode}
%<*driver>
\NeedsTeXFormat{LaTeX2e}[2011/06/27]
\ProvidesFile{tablefootnote.drv}%
  [2014/01/26 v1.1c Table foot notes (HMM)]
\documentclass{ltxdoc}[2007/11/11]% v2.0u
\usepackage{holtxdoc}[2012/03/21]%  v0.24
%% tablefootnote may work with earlier versions of LaTeX2e and those
%% class and package, but this was not tested.
%% Please consider updating your LaTeX, class, and package
%% to the most recent version (if they are not already the most
%% recent version).
\hypersetup{%
 pdfsubject={Footnotes in Tables (HMM)},%
 pdfkeywords={LaTeX, tablefootnote, footnote, table, H.-Martin Muench},%
 pdfencoding=auto,%
 pdflang={en},%
 breaklinks=true,%
 linktoc=all,%
 pdfstartview=FitH,%
 pdfpagelayout=OneColumn,%
 bookmarksnumbered=true,%
 bookmarksopen=true,%
 bookmarksopenlevel=3,%
 pdfmenubar=true,%
 pdftoolbar=true,%
 pdfwindowui=true,%
 pdfnewwindow=true%
}
\CodelineIndex
\hyphenation{printing docu-ment}
\gdef\unit#1{\mathord{\thinspace\mathrm{#1}}}%
\makeatletter
\@ifundefined{eTeX}{\gdef\eTeX{$\m@th \varepsilon $-\TeX }}{% else \relax
}
\makeatother
\begin{document}
  \DocInput{tablefootnote.dtx}%
\end{document}
%</driver>
%    \end{macrocode}
%
% \fi
%
% \CheckSum{584}
%
% \CharacterTable
%  {Upper-case    \A\B\C\D\E\F\G\H\I\J\K\L\M\N\O\P\Q\R\S\T\U\V\W\X\Y\Z
%   Lower-case    \a\b\c\d\e\f\g\h\i\j\k\l\m\n\o\p\q\r\s\t\u\v\w\x\y\z
%   Digits        \0\1\2\3\4\5\6\7\8\9
%   Exclamation   \!     Double quote  \"     Hash (number) \#
%   Dollar        \$     Percent       \%     Ampersand     \&
%   Acute accent  \'     Left paren    \(     Right paren   \)
%   Asterisk      \*     Plus          \+     Comma         \,
%   Minus         \-     Point         \.     Solidus       \/
%   Colon         \:     Semicolon     \;     Less than     \<
%   Equals        \=     Greater than  \>     Question mark \?
%   Commercial at \@     Left bracket  \[     Backslash     \\
%   Right bracket \]     Circumflex    \^     Underscore    \_
%   Grave accent  \`     Left brace    \{     Vertical bar  \|
%   Right brace   \}     Tilde         \~}
%
% \GetFileInfo{tablefootnote.drv}
%
% \begingroup
%   \def\x{\#,\$,\^,\_,\~,\ ,\&,\{,\},\%}%
%   \makeatletter
%   \@onelevel@sanitize\x
% \expandafter\endgroup
% \expandafter\DoNotIndex\expandafter{\x}
% \expandafter\DoNotIndex\expandafter{\string\ }
% \begingroup
%   \makeatletter
%     \lccode`9=32\relax
%     \lowercase{%^^A
%       \edef\x{\noexpand\DoNotIndex{\@backslashchar9}}%^^A
%     }%^^A
%   \expandafter\endgroup\x
%
% \DoNotIndex{\\,\,}
% \DoNotIndex{\documentclass,\usepackage,\ProvidesPackage,\begin,\end}
% \DoNotIndex{\MessageBreak,\message}
% \DoNotIndex{\NeedsTeXFormat,\DoNotIndex,\verb}
% \DoNotIndex{\def,\edef,\gdef,\global}
% \DoNotIndex{\ifx,\listfiles,\mathord,\mathrm}
% \DoNotIndex{\kvoptions,\SetupKeyvalOptions,\ProcessKeyvalOptions}
% \DoNotIndex{\bigskip,\space,\thinspace,\ldots}
% \DoNotIndex{\indent,\noindent,\newline,\linebreak,\pagebreak}
% \DoNotIndex{\textbf,\textit,\textsf,\texttt,\textsc,\textquotedblleft,\textquotedblright}
% \DoNotIndex{\plainTeX,\TeX,\LaTeX,\pdfLaTeX}
% \DoNotIndex{\chapter,\section,\the,\csname,\endscname}
% \DoNotIndex{\value,\arabic,\hline,\text,\caption,\centering,\clear,\url}
%
% \title{The \xpackage{tablefootnote} package}
% \date{2014/01/26 v1.1c}
% \author{H.-Martin M\"{u}nch\\\xemail{Martin.Muench at Uni-Bonn.de}}
%
% \maketitle
%
% \begin{abstract}
% This \LaTeX{} package provides the command |\tablefootnote| to be used
% in a \texttt{table} or \texttt{sidewaystable} environment,
% where |\footnote| will not work and when using |\footnotemark| and
% |\footnotetext| and adjusting the counters (including \texttt{Hfootnote})
% manually is either too much work or would not even work (\texttt{sidewaystable}).
% \end{abstract}
%
% \bigskip
%
% \noindent Disclaimer for web links: The author is not responsible for any contents
% referred to in this work unless he has full knowledge of illegal contents.
% If any damage occurs by the use of information presented there, only the
% author of the respective pages might be liable, not the one who has referred
% to these pages.
%
% \bigskip
%
% \noindent {\color{green} Save per page about $200\unit{ml}$ water,
% $2\unit{g}$ CO$_{2}$ and $2\unit{g}$ wood:\\
% Therefore please print only if this is really necessary.}
%
% \pagebreak
%
% \tableofcontents
%
% \pagebreak
%
% \section{Introduction\label{sec:Introduction}}
%
% \indent This \LaTeX{} package provides the command |\tablefootnote{...}| to be
% used in a \texttt{table} or \texttt{sidewaystable} environment. In those
% environments |\footnote{...}| would not work and would need to be replaced
% by |\footnotemark| in the \texttt{(sideways)\allowbreak table} and
% |\footnotetext{...}| after the \texttt{(sideways)table} environment ended.
% Additionally the counters for \texttt{footnote} and
% (when the \xpackage{hyperref} package is used) \texttt{Hfootnote}
% would need to be adjusted manually. Then still the hyperlinks
% in a \texttt{(sideways)table} would not work.
% When the \xpackage{footnotebackref} package is used,
% |\footnotemark| and |\footnotetext{...}| would need to be redefined.
% Just using |\tablefootnote{...}| (with optional argument for custom foot note
% marks) in the (sideways)table does all this automatically.
% (Compatibility with the \xpackage{footnotebackref} package
% was achieved without redefining |\footnotemark| or |\footnotetext|.)
% Redefining |\footnote| was \emph{not} done as to not create conflicts with
% other packages.
%
% \section{Usage}
%
% \indent Just load the package placing
% \begin{quote}
%   |\usepackage{tablefootnote}|
% \end{quote}
% \noindent in the preamble of your \LaTeXe{} source file and use\newline
% |\tablefootnote{...}| instead of |\footnote{...}| and\newline
% |\tablefootnote[...]{...}| instead of |\footnote[...]{...}|
% in (sideways)\allowbreak tables.\\
% When the \xpackage{footnotebackref} package is used,
% it must be loaded before the \xpackage{tablefootnote} package
% and the \xpackage{hyperrefpackage} with option |hyperfootnotes=true|
% must be loaded, too. (Backreference{-}links without links do not make sense.)\\
% When the \xpackage{rotating} package (for |sidewaystable|s) is used,
% it must be loaded before the \xpackage{tablefootnote} package and it must be
% a recent version (v2.16a, 2009/03/28, or newer).\\
% If |\tablefootnote{...}| is used in a tabular environment, which is inside
% of a (sideways)table environment, everything is fine, but when the tabular
% environment is not inside of a (sideways)table, the \textbf{table}footnote
% will not work.\\
%
% Pages with tables with footnotes, which are rotated with the \xpackage{lscape}-
% or with the \xpackage{pdflscape}-package (|\begin{landscape}|, table, |\end{landscape}|),
% are possible. But when the \xpackage{footmisc}-package is used, \xpackage{(pdf)lscape}
% should be loaded before \xpackage{footmisc}.\\
%
% When the \xpackage{footmisc}-package is used with option \texttt{para},
% this is regarded with the exception of a \texttt{sidewaystable}. There the footnotes
% are printed one below the other. (Well, this is better than no footnote in a
% \texttt{sidewaystable} at all, isn't it?)\\
%
% When the (sideways)table floats, the footnotes are set and (when \xpackage{hyperref}
% is used) hyperlinked, but they are not automatically adapted when the table floats
% over/under another footnote. Thus either do not use a footnote between original and
% float position of the (sideways)table, or place the (sideways)table in
% \textquotedblleft here\textquotedblright\ position. |\clear(double)page|, |h(!)|,
% |H|~from the \xpackage{float} package (\url{http://www.ctan.org/pkg/float}),
% or |\FloatBarrier| from the \xpackage{picins} package
% (\url{http://www.ctan.org/pkg/picins}) might help, too.
% (Or after finishing the document, move the (sideways)table in the source code
% near the position where it floats to or use the optional footnote marks.)
%
% \pagebreak
%
% \section{Alternatives\label{sec:Alternatives}}
%
% \begin{description}
% \item[-] The \xpackage{longtable} package provides the \texttt{longtable} environment
%   as replacement for the combined \texttt{table} and \texttt{tabular} environments.
%   Footnotes are real footnotes (not just tablenotes), are continuously numbered and
%   hyperlinked (when using the \xpackage{hyperref} package), and the hyperlinks really
%   work. As drawback the appearance of the caption changes slightly (e.\,g.~distance
%   to the table, width of the caption), which can probably be changed back manually.
%   Furthermore, longtables are meaned to break over more than one page. If that is not
%   wished, it must be prevented by |\nopagebreak|-commands and by ending the longtable
%   lines with |\\*| instead of |\\|. longtables do not float. (Therefore using the
%   \xpackage{tablefootnote} package and |\FloatBarrier| from the \xpackage{picins} package
%   before and after the table environment is similar~- but \xpackage{tablefootnote} does not
%   change the table-caption!) \texttt{sidewaystable} does not work with it.\\
%   \url{http://www.ctan.org/pkg/longtable}
%
% \item[-] The \xpackage{supertabular} package provides the \texttt{mpsupertabular}
%   environment as replacement for the combined \texttt{table} and \texttt{tabular}
%   environments. Footnotes are just tablenotes (with working hyperlinks when using the
%   \xpackage{hyperref} package), i.\,e. numbered a, b, c and placed below the table
%   and not at the end of the page. Therefore there is no float problem (because the
%   tablenotes numbering is not included in the continuous numbering of the footnotes).
%   Placing the \texttt{supertabular} inside of a \texttt{sidewaystable} breaks the
%   hyperlinks to the tablenotes.\\
%   \url{http://www.ctan.org/pkg/supertabular}
%
% \item[-] The \xpackage{ctable} package has its very own notation for defining tables.
%   It can create tablenotes and sideways-tables. The tablenotes are not automatically
%   hyperlinked. The \texttt{ctable}s float. Because the tablenotes numbering is not
%   included in the continuous numbering of the footnotes there is no float problem.\\
%   \url{http://www.ctan.org/pkg/ctable}
%
% \item[-] The \xpackage{footnote} package provides |\makesavenoteenv{table}|. After
%   loading the package and using that command in the preamble, in tables |\footnote{...}|
%   can be used. Using |\makesavenoteenv{tabular}| and \\
%   |\makesavenoteenv{sidewaystable}|
%   is possible, but it neither solves the float problem, nor do the created hyperlinks
%   work (i.\,e.~they aim at wrong locations). The \xpackage{mdwtab} from the same bundle
%   is incompatible with other table-related packages (e.\,g. \xpackage{supertabular},
%   \xpackage{array}) and not $100\,\%$ compatible with the \texttt{tabular} environment.\\
%   \url{http://www.ctan.org/pkg/footnote}\\
%   \url{http://www.ctan.org/pkg/mdwtab}
%
% \item[-] The \xpackage{tabularx} package does produce footnotes for sidewaystables,
%   but uses a, b, c instead of 1, 2, 3. The hyperlinks to the footnotes do not work.
%   Because the footnotes numbering is not included in the continuous numbering of
%   the other footnotes there is no float problem.\\
%   \url{http://www.ctan.org/pkg/tabularx}
%
% \item[-] Placing a tabular inside a minipage inside a table produces tablenotes.
%   Therefore there is no float problem (because the footnotes are not continuously
%   numbered). The hyperlinks to the table notes indeed work.
%
% \pagebreak
%
% \item[-] The \xpackage{threeparttable} package creates tablenotes again. Therefore
%   there is no float problem (because the tablenotes are not continuously numbered with
%   the footnotes). There are no hyperlinks to the table notes (at least not automatically).
%   Using |sidewaystable| (with table notes) works.\\
%   \url{http://www.ctan.org/pkg/threeparttable}
%
% \item[-] The \xpackage{threeparttablex} package creates tablenotes again. Therefore there
%   is no float problem (because the tablenotes are not continuously numbered with the
%   footnotes). With option \texttt{referable} the tablenotes are hyperlinked. Use of
%   a |sidewaystable| (with table notes) did not work for me. When using the
%   \texttt{referable} option according to the example in the \xpackage{threeparttablex}
%   manual the \xpackage{longtable} package is used, therefore that package could be used
%   directly without \xpackage{threeparttablex} (see above).\\
%   \url{http://www.ctan.org/pkg/threeparttablex}
%
% \item[-] One can manually use |\footnotemark| in the table and |\footnotetext{...}|
%  after the table environment ended and manually change the |footnote| and
%  (when the \xpackage{hyperref} package is used) |Hfootnote| counters and needs
%  to insert
%  \begin{verbatim}
%     \makeatletter
%     \global\let\Hy@saved@currentHref\@currentHref
%     \hyper@makecurrent{Hfootnote}%
%     \global\let\Hy@footnote@currentHref\@currentHref
%     \global\let\@currentHref\Hy@saved@currentHref
%     \makeatother
%  \end{verbatim}
%  before each |\footnotetext{...}|, but with more than very few footnotes this can become
%  much work (and is prone to produce errors). And this does not even work with
%  sidewaystables. (And it does not work with the \xpackage{footnotebackref} package
%  without redefining |\footnotemark| and |\footnotetext{...}|.)
% \end{description}
%
% \bigskip
%
% \noindent (You programmed or found another alternative,
%  which is available at \CTAN{}?\\
%  OK, send an e-mail to me with the name, location at \CTAN{},
%  and a short notice, and I will probably include it in
%  the list above.)
%
% \newpage
%
% \section{Example}
%
%    \begin{macrocode}
%<*example>
\documentclass[british]{article}[2007/10/19]% v1.4h
%%%%%%%%%%%%%%%%%%%%%%%%%%%%%%%%%%%%%%%%%%%%%%%%%%%%%%%%%%%%%%%%%%%%%
\usepackage{float}[2001/11/08]%    v1.3d
\usepackage{placeins}[2005/04/18]% v2.2  ; for \FloatBarrier
\usepackage{rotating}[2009/03/28]% v2.16a; for sidewaystable-environment
\usepackage[%
 hyperfootnotes=true,%
 extension=pdf,%
 plainpages=false,%
 pdfpagelabels=true,%
 hyperindex=false,%
 pdflang={en},%
 pdftitle={tablefootnote package example},%
 pdfauthor={H.-Martin Muench},%
 pdfsubject={Example for the tablefootnote package},%
 pdfkeywords={LaTeX, tablefootnote, footnote, table, H.-Martin Muench},%
 % pdfview=FitH and FitBH do not work: hyperlinks in sidewaystables
 % do no lead to the footnotes, due to a bug in pdfTeX,
 % computing wrong anchor coordinates (Heiko Oberdiek, 29. October 2011)
 % pdfview=Fit, FitV, FitR, FitB, FitBV work
 % print is OK for all those options
 pdfstartview=FitH,%
 pdfpagelayout=OneColumn%
]{hyperref}[2012/11/06]% v6.83m
 % Due to the urls used in the example, either the hyperref or the url
 % package are needed (or the urls must be removed before compiling).

\usepackage{footnotebackref}[2012/07/01]% v1.0
\usepackage{tablefootnote}[2014/01/26]%   v1.1c

\gdef\unit#1{\mathord{\thinspace\mathrm{#1}}}%
\listfiles
\begin{document}
\pagenumbering{arabic}
\section*{Example for tablefootnote}

This example demonstrates the use of package\newline
\textsf{tablefootnote}, v1.1c as of 2014/01/26 (HMM).\newline
There were no options used. (The package provides no options.)\newline

\textbf{The \texttt{tablefootnote-example.tex} needs to be compiled
at least twice to get the references right!}\newline

If the etoolbox-package is found, it is automatically used.\newline

For more details please see the documentation!\newline

\noindent Save per page about $200\unit{ml}$ water,
$2\unit{g}$ CO$_{2}$ and $2\unit{g}$ wood:\newline
Therefore please print only if this is really necessary.\newline

Here is some text.\footnote{And this is a text footnote.}\newline

Tables \ref{tab.symbol}, \ref{tab.normal}, \ref{tab.another} and
\ref{tab.floatH} show normal tables, table~\ref{tab.sideways} depicts
a sidewaystable. Table~\ref{tab.floatH} uses the float
specifier~\texttt{H} from the float package.\newline

\texttt{Hyperref} option \verb|pdfview=FitH| and \verb|FitBH| do
not work due to a bug in pdf\TeX{}, computing wrong
anchor coordinates (\textsc{Heiko Oberdiek}, 29.~October 2011).
Depending on used pdf-viewer, hyperlinks in sidewaystables lead
e.\,g.~at the end of the document, not at the footnote.
\verb|pdfview=Fit|, \verb|FitV|, \verb|FitR|, \verb|FitB|,
\verb|FitBV| work, print is OK for all those options.

\bigskip

\listoftables

\pagebreak

\renewcommand{\thefootnote}{\fnsymbol{footnote}}
\verb|\renewcommand{\thefootnote}{\fnsymbol{footnote}}|
causes footnote{-}symbol{-}footnotes, which are possible
(see Table~\ref{tab.symbol}).

\begin{table}
\centering
\begin{tabular}{ccc}
Another\tablefootnote{A table footnote.} & %
text\tablefootnote{Another table footnote.} & %
in a table\tablefootnote{A third table footnote.}
\end{tabular}
\caption[A footnotesymbol table]{%
A table with footnote-symbol-footnotes.\label{tab.symbol}}
\end{table}

Some text.\footnote{A text footnote.}

\renewcommand{\thefootnote}{\arabic{footnote}}
\verb|\renewcommand{\thefootnote}{\arabic{footnote}}|
switches back to normal footnote numbers again.

\pagebreak

\begin{table}
\centering
\begin{tabular}{ccc}
Some\tablefootnote{A table footnote.} & %
text\tablefootnote[99]{A table footnote with custom footnote number.} & %
in a table\tablefootnote{A third table footnote.}
\end{tabular}
\caption[A table]{A normal table.\label{tab.normal}}
\end{table}

Some text.\footnote{Another text footnote.}

\pagebreak

%    \end{macrocode}
%
% \pagebreak
%
%    \begin{macrocode}
More text.\footnote{And yet another text footnote.}

\begin{table}[t]
\centering
\begin{tabular}{|c|c|c|}
\hline
Another\tablefootnote{A $3^{rd}$ table footnote.} & %
text\tablefootnote{Another $3^{rd}$ table footnote.} & %
in a table\tablefootnote{A $3^{rd}$ third table footnote.}\\ \hline
\end{tabular}
\caption[Another table]{Another table (third one)\label{tab.another}}
\end{table}

Please note that Table~\ref{tab.another} floated to the top of the page.
While the footnotes are set and (when hyperref is used) hyperlinked,
they are not automatically adapted. Thus either do not use a footnote
at the same page before the table, or place the table in
\textquotedblleft here\textquotedblright\ or
\textquotedblleft bottom\textquotedblright\ position.
\verb|\clear(double)page|, \verb|h(!)|, \verb|H|~from the
\texttt{float} package, or \verb|\FloatBarrier| from the
\texttt{picins} package might help, too. (Or move the table in the
source code near the position where it floats to
or use the optional footnote marks.)

Table~\ref{tab.floatH} (page~\pageref{tab.floatH}) uses float specifier
\texttt{H} from the float package and does not float.

Some text.\footnote{This is just another text footnote.}

\pagebreak

\FloatBarrier

\begin{sidewaystable}
\centering%
\begin{tabular}{ccc}
Text\tablefootnote{Please rotate the view for testing the %
hyperlinks.} & %
in a\tablefootnote[98]{Another sidewaystable footnote %
with optional footnote mark.} & %
sidewaystable%
\tablefootnote{Sidewaystable-environment provided by %
\url{http://www.ctan.org/pkg/rotating} package.}
\end{tabular}
\caption[A sideways table]{A table in the \texttt{sideways} %
environment\label{tab.sideways}}%
\end{sidewaystable}%

\FloatBarrier

\pagebreak

%    \end{macrocode}
%
% \pagebreak
%
%    \begin{macrocode}
A last table, here with float specifier \texttt{H} from the
float\footnote{\url{http://www.ctan.org/pkg/float}} package.

\begin{table}[H]
\centering
\begin{tabular}{ccc}
Another\tablefootnote{A $5^{th}$ table footnote.} & %
text\tablefootnote{Another $5^{th}$ table footnote.} & %
in a table\tablefootnote{A $5^{th}$ third table footnote.}
\end{tabular}
\caption[A last table]{A very last table\label{tab.floatH}}
\end{table}

Some text.\footnote{This is just another text footnote.}

\pagebreak

End of the example for the
tablefootnote\footnote{\url{http://www.ctan.org/pkg/tablefootnote}}
package.

\end{document}
%</example>
%    \end{macrocode}
%
% \StopEventually{}
%
% \newpage
%
% \section{The implementation}
%
% We start off by checking that we are loading into \LaTeXe{} and
% announcing the name and version of this package.
%
%    \begin{macrocode}
%<*package>
%    \end{macrocode}
%
%    \begin{macrocode}
\NeedsTeXFormat{LaTeX2e}[2011/06/27]
\ProvidesPackage{tablefootnote}[2014/01/26 v1.1c
              Table foot notes (HMM)]
%    \end{macrocode}
%
% A short description of the \xpackage{tablefootnote} package:
%
%    \begin{macrocode}
%% Provides the \tablefootnote[...]{...} command
%% for footnotes in (sideways)tables.

%    \end{macrocode}
%
% We need the \xpackage{ltxcmds} and \xpackage{letltxmacro} packages
% by \textsc{Heiko Oberdiek}:
%
%    \begin{macrocode}
\RequirePackage{ltxcmds}[2011/11/09]%     v1.22
\RequirePackage{letltxmacro}[2010/09/02]% v1.4
\RequirePackage{xifthen}[2009/04/17]%     v1.3
%    \end{macrocode}
%
% A last information for the user:
%
%    \begin{macrocode}
%% tablefootnote may work with earlier versions of LaTeX and those
%% packages, but this was not tested. Please consider updating
%% your LaTeX and packages to the most recent version
%% (if they are not already the most recent version).

%    \end{macrocode}
%
% \indent See subsection~\ref{ss:Downloads} about how to get them.\\
%
% When the \xpackage{rotating} package (for |sidewaystable|s) is used,
% it must be loaded before the \xpackage{tablefootnote} package and it must be
% a recent version:
%
%    \begin{macrocode}
\@ifpackageloaded{rotating}{%
  \gdef\tfn@rotating{1}
  \@ifpackagelater{rotating}{2009/03/28}% v2.16a
   {% >= 2009/03/28, OK
   }{% else
     \PackageWarningNoLine{tablefootnote}{%
       It is required version\MessageBreak%
       2009/03/28 v2.16a (or later) of package rotating,\MessageBreak%
       but only version\MessageBreak%
      `\csname ver@rotating.sty\endcsname'\MessageBreak%
      is available}%
   }%
 }{% else
   \gdef\tfn@rotating{0}%
 }

%    \end{macrocode}
%
% There are no options to be processed.\\
%
% We need some definitions:
%
%    \begin{macrocode}
\def\tfn@footnotetablecount{0}
\def\tfn@footnotetableprint{0}
\def\tfn@footnotezero{0}
\def\tfn@fnt{0}
\def\tfn@swt{0}%       _s_ide_w_ays_t_able
\def\tfn@fmpw{\relax}% f_oot_m_isc_p_ara_w_arn-switch
\def\tfn@fnbr{0}%      f_oot_n_ote_b_ack_r_ef(erence package)

%    \end{macrocode}
%
% \begin{macro}{\tfn@hyperfootnotes}
% We need a wrapper for the used |\ifHy@hyperfootnotes|,
% which is not defined when the \xpackage{hyperref} package has not been loaded:
%
%    \begin{macrocode}
\DeclareRobustCommand{\tfn@hyperfootnotes}[2]{%
  \ifHy@hyperfootnotes%
    #1%
  \else%
    #2%
  \fi%
  }

%    \end{macrocode}
% \end{macro}
%
% \begin{macro}{\tablemakefntext}
% |\tablemakefntext| makes the footnotetext for a footnote in a table.
%
%    \begin{macrocode}
\ltx@ifpackageloaded{footnotebackref}{%
  \gdef\tfn@fnbr{1}%
  \ifx\FootnoteBackref@symbol\empty%
    \ifFootnoteBackref@numberlinked%
      \newcommand\tablemakefntext[1]{%
        \noindent\makebox[1.634em][r]{%
         \mbox{\textsuperscript{\normalfont%
           \hyperlink{tfn:m\theHfootnote}{\@thefnmark}%
        }}}%
        \ #1%
      }
    \else
      \newcommand\tablemakefntext[1]{%
        \noindent\makebox[1.8em][r]{%
          \mbox{\textsuperscript{\normalfont%
            \@thefnmark%
        }}}%
        #1%
      }
    \fi
  \else
    \newcommand\tablemakefntext[1]{%
      \noindent\makebox[1.8em][r]{%
        \mbox{\textsuperscript{\normalfont%
          \ifFootnoteBackref@numberlinked%
            \hyperlink{tfn:m\theHfootnote}{\@thefnmark}\ %
          \else%
            \@thefnmark%
          \fi%
      }}}%
      \hyperlink{tfn:m\theHfootnote}{\FootnoteBackref@symbol}%
      #1%
    }
  \fi
}{%
  \newcommand\tablemakefntext[1]{%
    \noindent\makebox[1.8em][r]{%
      \mbox{\textsuperscript{\normalfont%
        \@thefnmark%
    }}}%
    #1%
  }
}

%    \end{macrocode}
%
% \end{macro}
%
% \pagebreak
%
% \begin{macro}{\swtablemakefntext}
%    \begin{macrocode}
\newcommand\swtablemakefntext[1]{\noindent\ #1}

%    \end{macrocode}
% \end{macro}
%
% \begin{macro}{\tfn@footnotemarks}
% |\tfn@footnotemarks| uses the |\footnotemark| command to place a foot note mark
% and decreases the |\tfn@footnotetablecount| (i.\,e.~increases the negative value)
% of the number of footnote(marks) in the table. If the \xpackage{footnotebackref}
% package is used, it is also necessary to create an according |\hypertarget| for
% the back reference.
%
%    \begin{macrocode}
\newcommand{\tfn@footnotemarks}[1][]{%
  \hbox{%
    \ifthenelse{\isempty{#1}}%
      {\footnotemark%
       \ltx@ifpackageloaded{footnotebackref}%
         {\raisebox{\ht\strutbox}{\hypertarget{tfn:m\theHfootnote}{}}}%
         {% footnotebackref not loaded
         }%
      }%
      {\ltx@ifpackageloaded{hyperref}%
        {\tfn@hyperfootnotes{%
            \hyperlink{tfn:#1}{\footnotemark[#1]}%
          }{\footnotemark[#1]}}%
        {\footnotemark[#1]}%
       \ltx@ifpackageloaded{footnotebackref}%
        {\raisebox{\ht\strutbox}{\hypertarget{tfn:m#1}{}}}{}%
      }%
   }%
  \@tempcnta=\tfn@footnotetablecount\relax%
  \advance\@tempcnta by -1\relax%
  \xdef\tfn@footnotetablecount{\the\@tempcnta}%
}

%    \end{macrocode}
% \end{macro}
%
% \begin{macro}{\tfn@footmiscparawarn}
% When the \xpackage{footmisc} package is used with option |para|,
% we need to issue a warning. (Any idea how to resolve this issue?)\\
% It must be given inside of it's own command, because |\ifFN@para| is not
% defined when \xpackage{footmisc} has not been loaded, and then the
% |\fi| would be a |\fi| without |\if| (and \TeX{} does not like single
% |\fi|s).
%
%    \begin{macrocode}
\newcommand{\tfn@footmiscparawarn}{%
  \ifFN@para%
    \PackageWarning{tablefootnote}{%
      Package footmisc with option para used.\MessageBreak%
      Unfortunately package tablefootnote is not yet\MessageBreak%
      able to regard this in sidewaystables.\MessageBreak%
      Here the footnotes are printed one below\MessageBreak%
      the other. Sorry!\MessageBreak%
     }%
  \fi%
  }

%    \end{macrocode}
% \end{macro}
%
% \begin{macro}{\tfn@updateHyper}
% |\tfn@updateHyper| updates some \xpackage{hyperref} internals.
%
%    \begin{macrocode}
\newcommand{\tfn@updateHyper}{%
  % The code of this command is partly from the
  % hyperref package by Heiko Oberdiek, 2011/10/01, v6.82j.
  % Newer versions of that package are available at CTAN.org.
  \global\let\Hy@saved@currentHref\@currentHref%
  \tfn@hyperfootnotes{\hyper@makecurrent{Hfootnote}}{}%
  \global\let\Hy@footnote@currentHref\@currentHref%
  \global\let\@currentHref\Hy@saved@currentHref%
  }

%    \end{macrocode}
% \end{macro}
%
% \begin{macro}{\tfn@footnotetext}
% |\tfn@footnotetext| contains the |\footnotetext[...]{...}| command and handles
% the numbering. |\tfn@footnotetext| will be executed after the table.
% The foot note counter has been increased automatically with each |\footnotemark|,
% therefore it must be reset to its value before the table began.
%
%    \begin{macrocode}
\newcommand{\tfn@footnotetext}[2][]{%
  \ifx\tfn@footnotetableprint\tfn@footnotezero%
    % This is the first footnotetext after the (sideways)table.
    \addtocounter{footnote}{\tfn@footnotetablecount}%
%    \end{macrocode}
%
% When the \xpackage{hyperref} package is used, also its |Hfootnote| counter must
% be reset (when the footnotes are hyperlinked).
%
%    \begin{macrocode}
    \ltx@ifpackageloaded{hyperref}{%
      \tfn@hyperfootnotes{%
        \addtocounter{Hfootnote}{\tfn@footnotetablecount}}{}%
      \tfn@updateHyper%
     }{}%
  \fi%
%    \end{macrocode}
%
% For issuing the text for the next footnote the footnote-counter
% (and |Hfootnote| counter in case of links) are increased by one.
%
%    \begin{macrocode}
  \addtocounter{footnote}{+1}%
  \ltx@ifpackageloaded{hyperref}{%
    \tfn@hyperfootnotes{\addtocounter{Hfootnote}{+1}}{}%
%    \end{macrocode}
%
% The \xpackage{hyperref} package was not expecting this kind of change of the
% |(H)footnote| numbers, therefore some code-lines are missing here which can be
% found at other places in that same package and are called by this command.
%
%    \begin{macrocode}
    \tfn@updateHyper%
%    \end{macrocode}
%
% When \xpackage{hyperref} is not loaded, nothing is to be done here.
%
%    \begin{macrocode}
   }{}%
%    \end{macrocode}
%
% The |\footnotetext[...]{...}| places the foot note text as usual.
%
%    \begin{macrocode}
  \ifx\tfn@swt\tfn@footnotezero%
    \ifthenelse{\isempty{#1}}%
      {\footnotetext{#2}}%
      {\ltx@ifpackageloaded{hyperref}%
         {\tfn@hyperfootnotes{%
             \footnotetext[#1]{%
               \raisebox{\ht\strutbox}{\hypertarget{tfn:#1}{}}#2}%
           }{\footnotetext[#1]{#2}%
           }%
         }{\footnotetext[#1]{#2}}%
      }%
  \else%
%    \end{macrocode}
%
% If there is more than one tablefootnote in the |sidewaystable|, \\
% the |\tfn@footmiscparawarn| shall be given,
%
%    \begin{macrocode}
    \ifx\tfn@footnotetableprint\tfn@footnotezero%
      \gdef\tfn@fmpw{\tfn@footmiscparawarn}%
    \else%
      \ltx@ifpackageloaded{footmisc}{\tfn@fmpw}{\relax}%
%    \end{macrocode}
%
% but we do not want to repeat the message for a third(, forth,\ldots)
% tablefootnote in the |sidewaystable|.
%
%    \begin{macrocode}
      \gdef\tfn@fmpw{\relax}%
    \fi%
%    \end{macrocode}
%
% |\@makefnmark| is \\
% |\hbox {\@textsuperscript {\normalfont \@thefnmark }}|, \\
% but according to |\show\@thefnmark| here |\@thefnmark| is
% |{\protect \itshape}| instead of |\thefootnote| (Why?!).\\
% If the \xpackage{footnotebackref} package is used,
% back references must be created for the table footnotes, too.
%
%    \begin{macrocode}
    \ltx@ifpackageloaded{footnotebackref}{%
      \LetLtxMacro{\tfncurrentmft}{\@makefntext}%
      \LetLtxMacro{\@makefntext}{\swtablemakefntext}%
      \ifthenelse{\isempty{#1}}%
        {\footnotetext{%
           \raisebox{\ht\strutbox}{\hypertarget{tfn:\theHfootnote}{}}%
           \hbox{\@textsuperscript{\normalfont%
              \hyperlink{tfn:m\theHfootnote}{\thefootnote}}}\thinspace%
         \ifx\FootnoteBackref@symbol\empty%
         \else\hyperlink{tfn:m\theHfootnote}{\FootnoteBackref@symbol}%
         \fi%
         #2}%
       }{\edef\tfn@tfn{\arabic{footnote}}%
         \edef\tfn@thn{\arabic{Hfootnote}}%
         \setcounter{footnote}{#1}%
         \setcounter{Hfootnote}{#1}%
         \tfn@updateHyper%
         \footnotetext{%
           \raisebox{\ht\strutbox}{\hypertarget{tfn:#1}{}}%
           \hbox{\@textsuperscript{\normalfont%
                  \hyperlink{tfn:m#1}{#1}}}%
           \thinspace%
           \ifx\FootnoteBackref@symbol\empty%
           \else%
             \raisebox{\ht\strutbox}{%
               \hyperlink{tfn:m#1}{\FootnoteBackref@symbol}}%
           \fi%
           #2%
          }%
         \setcounter{footnote}{\tfn@tfn}%
         \setcounter{Hfootnote}{\tfn@thn}%
         \tfn@updateHyper%
       }%
      \LetLtxMacro{\@makefntext}{\tfncurrentmft}%
     }{% footnotebackref not loaded
       \ifthenelse{\isempty{#1}}%
         {\footnotetext{%
            \hbox{\@textsuperscript{\normalfont\thefootnote}}%
            \ltx@ifpackageloaded{hyperref}%
              {\tfn@hyperfootnotes{%
                 \raisebox{\ht\strutbox}{%
                   \hypertarget{tfn:\theHfootnote}{}}}{}%
              }{}%
            \thinspace%
            #2}%
        }{%
%    \end{macrocode}
%
% \pagebreak
%
%    \begin{macrocode}
          \edef\tfn@tfn{\arabic{footnote}}%
          \ltx@ifpackageloaded{hyperref}{%
            \tfn@hyperfootnotes{\xdef\tfn@thn{\arabic{Hfootnote}}}{}%
           }{}%
          \setcounter{footnote}{#1}
          \ltx@ifpackageloaded{hyperref}{%
            \tfn@hyperfootnotes{\setcounter{Hfootnote}{#1}}{}%
            \tfn@updateHyper%
           }{}%
          \footnotetext{%
            \hbox{\@textsuperscript{\normalfont #1}}%
            \ltx@ifpackageloaded{hyperref}{%
              \tfn@hyperfootnotes{\raisebox{\ht\strutbox}{%
                                         \hypertarget{tfn:#1}{}}}{}%
             }{}%
            \thinspace%
            #2}%
          \setcounter{footnote}{\tfn@tfn}%
          \ltx@ifpackageloaded{hyperref}{%
            \tfn@hyperfootnotes{\setcounter{Hfootnote}{\tfn@thn}}{}%
            \tfn@updateHyper%
           }{}%
        }%
     }%
  \fi%
%    \end{macrocode}
%
% For further footnotes in the table the \\
% |\ifx\tfn@footnotetableprint\tfn@footnotezero|\ldots|\fi| \\
% from the beginning of this |\tfn@footnotetext| macro must not be repeated,
% therefore we deactivate this by defining |\tfn@footnotetableprint| to
% something other than |0| (here |1| is used).
%
%    \begin{macrocode}
  \gdef\tfn@footnotetableprint{1}%
%    \end{macrocode}
%
% |\tfn@footnotetablecount| contains the negative number of footnote texts
% to be printed, therefore it is increase by one (i.\,e.~the negative number
% decreases, i.\,e.~$-3$ becomes~$-2$).
%
%    \begin{macrocode}
  \@tempcnta=\tfn@footnotetablecount\relax%
  \advance\@tempcnta by +1\relax%
  \xdef\tfn@footnotetablecount{\the\@tempcnta}%
%    \end{macrocode}
%
% When it is zero, all footnote texts have been placed, and\\
% |\tfn@footnotetableprint| can be reset to zero and\\
% |\tfn@tablefootnoteprintout| can be cleared.
%
%    \begin{macrocode}
  \ifx\tfn@footnotetablecount\tfn@footnotezero% all footnotes placed
    \gdef\tfn@footnotetableprint{0}%
    \global\let\tfn@tablefootnoteprintout\relax% brute force
 %\else there is at least one \footnotemark which needs \footnotetext
  \fi
}

%    \end{macrocode}
% \end{macro}
%
% \begin{macro}{\tfn@tablefootnoteprintout}
% When something shall be cleared or redefined, it is a good idea
% to define it. It~will be later filled with the foot note text to be printed.
%
%    \begin{macrocode}
\newcommand{\tfn@tablefootnoteprintout}{\relax}

%    \end{macrocode}
% \end{macro}
%
% \pagebreak
%
% \begin{macro}{\tablefootnote}
% |\tablefootnote[...]{...}| is the command to be used by the user instead of
% |\footnote[...]{...}| in the (sideways)table. It places a foot note mark via
% |\tfn@footnotemarks| and appends a |\tfn@footnotetext{#1}| to the\\
% |\tfn@tablefootnoteprintout| macro, and remembers that table foot notes
% have been used.
%
%    \begin{macrocode}
\newcommand{\tablefootnote}[2][]{%
  \tfn@footnotemarks[#1]%
  \ifthenelse{\isempty{#1}}{\relax}{%
    \ltx@GlobalPrependToMacro{\tfn@tablefootnoteprintout}{%
      \addtocounter{footnote}{+1}%
      \ltx@ifpackageloaded{hyperref}{%
        \tfn@hyperfootnotes{\addtocounter{Hfootnote}{+1}}{}%
        \tfn@updateHyper%
       }{}%
     }%
   }%
  \ltx@GlobalAppendToMacro{\tfn@tablefootnoteprintout}{%
    \tfn@footnotetext[#1]{#2}%
    \ifthenelse{\isempty{#1}}{\relax}{%
      \addtocounter{footnote}{-1}%
      \ltx@ifpackageloaded{hyperref}{%
        \tfn@hyperfootnotes{\addtocounter{Hfootnote}{-1}}{}%
        \tfn@updateHyper%
       }{}%
     }%
   }%
  \gdef\tfn@fnt{1}%
  }

%    \end{macrocode}
% \end{macro}
%
% \begin{macro}{\tfn@swtbox}
% For |sidewaystable|s the |\tfn@tablefootnoteprintout| is placed in a \\
% |\parbox|, which can be placed at the end of the |sidewaystable|:
%
%    \begin{macrocode}
\newcommand{\tfn@swtbox}{%
\ifvoid \footins%
\vskip 2\baselineskip%
\else%
\vskip \skip\footins%
\fi%
\parbox[b][][t]{\linewidth}{\raggedright\tfn@tablefootnoteprintout}%
}

%    \end{macrocode}
% \end{macro}
%
% |\tfn@tablefootnoteprintout| shall be executed after the end of the table environment.
% The \xpackage{etoolbox} package provides very easy syntax for this:
%
%    \begin{macrocode}
\IfFileExists{etoolbox.sty}{%
  \RequirePackage{etoolbox}[2011/01/03]%   v2.1
  \AfterEndEnvironment{table}{%
    \ltx@ifpackageloaded{footnotebackref}{%
      \LetLtxMacro{\tfncurrentmft}{\@makefntext}%
      \LetLtxMacro{\@makefntext}{\tablemakefntext}%
       \tfn@tablefootnoteprintout%
      \LetLtxMacro{\@makefntext}{\tfncurrentmft}%
     }{% footnotebackref not loaded
      \tfn@tablefootnoteprintout%
     }%
      \gdef\tfn@fnt{0}%
   }%
%    \end{macrocode}
%
% When the \xpackage{rotating} package has been loaded, we need to detect
% whether the |\tablefootnote| is placed inside of a |sidewaystable|.
%
%    \begin{macrocode}
  \ltx@ifpackageloaded{rotating}{%
    \AtBeginEnvironment{sidewaystable}{%
      \gdef\tfn@swt{1}%
    }
%    \end{macrocode}
%
% At the end of |sidewaystable| environments |\tfn@swtbox| shall be placed
% (if~there was any |\tablefootnote| in that |sidewaystable| at all).
% And |\tfn@swt| as well as |\tfn@fnt| must be reset to |0|.
%
%    \begin{macrocode}
    \AtEndEnvironment{sidewaystable}{%
      \ifx\tfn@fnt\tfn@footnotezero% \relax
      \else%
        \ltx@ifpackageloaded{footnotebackref}{%
          \LetLtxMacro{\tfncurrentmft}{\@makefntext}%
          \LetLtxMacro{\@makefntext}{\tablemakefntext}%
           \tfn@swtbox%
          \LetLtxMacro{\@makefntext}{\tfncurrentmft}%
         }{% footnotebackref not loaded
          \tfn@swtbox%
          }%
      \fi%
      \gdef\tfn@swt{0}%
      \gdef\tfn@fnt{0}%
     }%
   }{}% else \relax
}{%
%    \end{macrocode}
%
% When the \xpackage{etoolbox} package is not available (e.\,g. because of missing
% \eTeX{}) we~do a workaround:
%
%    \begin{macrocode}
  \PackageWarning{tablefootnote}{%
    Package etoolbox not available, therefore\MessageBreak%
    substituting commands \string\AtEndEnvironment\space %
    and\MessageBreak%
    \string\AfterEndEnvironment\space for tablefootnotes%
    \MessageBreak%
   }%
  \@ifpackagelater{letltxmacro}{2010/09/02}%
   {% >= 2010/09/02, OK
    \GlobalLetLtxMacro{\tfnorigtable}{\table}
    \GlobalLetLtxMacro{\tfnendorigtable}{\endtable}
%    \end{macrocode}
%
% |\GlobalLetLtxMacro| was introduced in version 2010/09/02 v1.4 of the
% \xpackage{letltxmacro} package.
%
%    \begin{macrocode}
   }{% else
     \PackageError{tablefootnote}{%
       Outdated version of letltxmacro package used%
      }{Version 2010/09/02 v1.4 or newer needed but not found!%
        \MessageBreak%
        Update the used letltxmacro package.}
        \LetLtxMacro{\tfnorigtable}{\table}
        \LetLtxMacro{\tfnendorigtable}{\endtable}
    }%
%    \end{macrocode}
%
% \pagebreak
%
%    \begin{macrocode}
  \ltx@ifpackageloaded{float}{%
    \@ifpackagelater{float}{2001/11/08}%   v1.3d
     {% >= 2001/11/08, OK
     }{% else
       \PackageWarningNoLine{tablefootnote}{%
         It is required version\MessageBreak
         2001/11/08 v1.3d (or later) of package float,\MessageBreak
         but only version\MessageBreak
        `\csname ver@float.sty\endcsname'\MessageBreak
        is available}%
     }%
   }{\relax}
  \renewenvironment{table}{%
    \ltx@ifpackageloaded{float}{%
%    \end{macrocode}
%
% When \xpackage{etoolbox.sty} was not found, and the \xpackage{float} package
% has been loaded, and the float specifier is \texttt{H}, then |\endtable| was
% let to |\float@endH| by the \xpackage{float} package (line~79 of
% \xpackage{float.sty}, 2001/11/08 v1.3d:\\
% |\expandafter\let\csname end#1\endcsname\float@endH|), i.\,e.~appending\\
% |\tfn@tablefootnoteprintout \gdef\tfn@fnt{0}| to |\endtable| is useless.
% Instead of this, it must be appended to |\float@endH|.\\
% (Why is this not necessary when \xpackage{etoolbox} is used?
% That is a good question, answers are welcome.)
%
%    \begin{macrocode}
      \ltx@LocalAppendToMacro{\float@endH}{%
        \ltx@ifpackageloaded{footnotebackref}{%
          \LetLtxMacro{\tfncurrentmft}{\@makefntext}%
          \LetLtxMacro{\@makefntext}{\tablemakefntext}%
           \tfn@tablefootnoteprintout%
          \LetLtxMacro{\@makefntext}{\tfncurrentmft}%
         }{% footnotebackref not loaded
          \tfn@tablefootnoteprintout%
         }%
          \gdef\tfn@fnt{0}%
       }%
     }{\relax}
    \tfnorigtable%
   }{%
     \tfnendorigtable%
     \ltx@ifpackageloaded{footnotebackref}{%
       \LetLtxMacro{\tfncurrentmft}{\@makefntext}%
       \LetLtxMacro{\@makefntext}{\tablemakefntext}%
        \tfn@tablefootnoteprintout%
       \LetLtxMacro{\@makefntext}{\tfncurrentmft}%
      }{% footnotebackref not loaded
       \tfn@tablefootnoteprintout%
      }%
     \gdef\tfn@fnt{0}%
    }%
%    \end{macrocode}
%
% \pagebreak
%
% The |sidewaystable|-environment must be treated separately:
%
%    \begin{macrocode}
  \ltx@ifpackageloaded{rotating}{%
    \let\tfnorigsidewaystable\sidewaystable
    \let\tfnendorigsidewaystable\endsidewaystable
    \renewenvironment{sidewaystable}{%
      \gdef\tfn@swt{1}%
      \tfnorigsidewaystable%
    }{%
      \ifx\tfn@fnt\tfn@footnotezero% \relax
      \else%
        \ltx@ifpackageloaded{footnotebackref}{%
          \LetLtxMacro{\tfncurrentmft}{\@makefntext}%
          \LetLtxMacro{\@makefntext}{\tablemakefntext}%
           \tfn@swtbox%
          \LetLtxMacro{\@makefntext}{\tfncurrentmft}%
         }{% footnotebackref not loaded
          \tfn@swtbox%
         }%
      \fi%
      \gdef\tfn@swt{0}%
      \gdef\tfn@fnt{0}%
      \tfnendorigsidewaystable%
     }%
   }{}%
}

%    \end{macrocode}
%
% \DescribeMacro{\AtBeginDocument}
% When the \xpackage{rotating} (for |sidewaystable|s) or the
% \xpackage{footnotebackref} (for back references for the footnotes)
% package is used, it must be loaded before the \xpackage{tablefootnote} package.
% Using both packages is also possible, in which case both must be loaded before
% the \xpackage{tablefootnote} package. If any one of those packages is loaded
% |\AtBeginDocument| after this check, the according error message will not be
% given, but packages really should be loaded before |\AtBeginDocument|.
%
%    \begin{macrocode}
\AtBeginDocument{%
  \ltx@ifpackageloaded{footnotebackref}{%
    \ifx\tfn@fnbr\tfn@footnotezero%
      \PackageError{tablefootnote}{%
        Package footnotebackref loaded after tablefootnote}{%
        When using both the footnotebackref and the tablefootnote %
        package,\MessageBreak%
        the footnotebackref package MUST be loaded before the %
        tablefootnote package!\MessageBreak%
       }%
    \fi%
   }{% else \relax
   }
  \ltx@ifpackageloaded{rotating}{%
    \ifx\tfn@rotating\tfn@footnotezero%
      \PackageError{tablefootnote}{%
        Package rotating loaded after tablefootnote}{%
        When using both the rotating and the tablefootnote package,%
        \MessageBreak%
        the rotating package MUST be loaded before the tablefootnote%
        \space package!\MessageBreak%
       }%
    \fi%
   }{% else \relax
   }
%    \end{macrocode}
%
% \pagebreak
%
% When the document is compiled with \LuaLaTeX{}, hyperlinks in rotated content
% will be misplaced, regardless of the use of the \xpackage{tablefootnote} package
% (or of tables or footnotes). The text to be printed will be OK.
% This problem of \LuaLaTeX{} (not \xpackage{tablefootnote}) was found and reported
% by \textsc{Arno Trautmann}, 2011/11/15 -- thanks!
%
%    \begin{macrocode}
  \ltx@ifpackageloaded{hyperref}{%
    \tfn@hyperfootnotes{
      \ifluatex%
        \PackageWarningNoLine{tablefootnote}{%
          LuaTeX and hyperref package used:\MessageBreak%
          Hyperreferences in rotated content\MessageBreak%
          will be misplaced\MessageBreak%
          regardless of tablefootnote package.\MessageBreak%
          The text to be printed will be OK%
         }%
      \fi%
     }{% else \relax
     }%
   }{% else \relax
   }%
%    \end{macrocode}
%
% When the \xpackage{footnotebackref} package is used but the footnotes are not
% hyperlinked, neither the back references nor the tablefootnotes will work.
%
%    \begin{macrocode}
  \ltx@ifpackageloaded{footnotebackref}{%
     \ltx@ifpackageloaded{hyperref}{%
        \tfn@hyperfootnotes{\relax}{% else
          \PackageError{tablefootnote}{%
             footnotebackref but not hyperfootnotes}{%
             The footnotebackref package is used\MessageBreak%
             but the hyperfootnotes option of the hyperref package%
             \MessageBreak%
             was set to false.\MessageBreak%
             Either enable hyperfootnotes or do not use footnotebackref.%
             \MessageBreak%
             Loading of the tablefootnote package will be aborted.%
           }%
          \endinput%
      }%
    }{\PackageError{tablefootnote}{footnotebackref but not hyperref}{%
        The footnotebackref package is used\MessageBreak%
        but not the hyperref package.\MessageBreak%
        Either load hyperref or do not use footnotebackref.%
        \MessageBreak%
        Loading of the tablefootnote package will be aborted.%
        }%
      \endinput%
    }%
  }{}
}

%    \end{macrocode}
%
%    \begin{macrocode}
%</package>
%    \end{macrocode}
%
% \pagebreak
%
% \section{Installation}
%
% \subsection{Downloads\label{ss:Downloads}}
%
% Everything is available on \CTAN{}, \url{http://www.ctan.org/},
% but may need additional packages themselves.\\
%
% \DescribeMacro{tablefootnote.dtx}
% For unpacking the |tablefootnote.dtx| file and constructing the documentation
% it is required:
% \begin{description}
% \item[-] \TeX{} Format \LaTeXe{}: \url{http://www.CTAN.org/}
%
% \item[-] document class \xpackage{ltxdoc}, 2007/11/11, v2.0u,
%           \url{http://www.ctan.org/pkg/ltxdoc}
%
% \item[-] package \xpackage{holtxdoc}, 2012/03/21, v0.24,
%           \url{http://www.ctan.org/pkg/holtxdoc}
%
% \item[-] package \xpackage{hypdoc}, 2011/08/19, v1.11,
%           \url{http://www.ctan.org/pkg/hypdoc}
% \end{description}
%
% \DescribeMacro{tablefootnote.sty}
% The |tablefootnote.sty| for \LaTeXe{} (i.\,e.~each document using
% the \xpackage{tablefootnote} package) requires:
% \begin{description}
% \item[-] \TeX{} Format \LaTeXe{}, \url{http://www.CTAN.org/}
%
% \item[-] package \xpackage{letltxmacro}, 2010/09/02, v1.4,
%           \url{http://www.ctan.org/pkg/letltxmacro}.
%
% \item[-] package \xpackage{ltxcmds}, 2011/11/09, v1.22,
%           \url{http://www.ctan.org/pkg/ltxcmds}
% \end{description}
%
% When the \xpackage{hyperref} package is used, also the \xpackage{ifluatex} package
% is needed, but it is already loaded automatically by the \xpackage{hyperref}
% package.\\
% When the \xpackage{etoolbox} package is available, it is used:
% \begin{description}
% \item[-] package \xpackage{etoolbox}, 2011/01/03, v2.1,
%           \url{http://www.ctan.org/pkg/etoolbox}.
% \end{description}
%
% \DescribeMacro{tablefootnote-example.tex}
% The |tablefootnote-example.tex| requires the same files as all
% documents using the \xpackage{tablefootnote} package and additionally:
% \begin{description}
% \item[-] class \xpackage{article}, 2007/10/19, v1.4h, from \xpackage{classes.dtx}:\\
%           \CTAN{macros/latex/base/classes.dtx}
%
% \item[-] package \xpackage{rotating}, 2009/03/28, v2.16a,
%           \url{http://www.ctan.org/pkg/rotating}
%
% \item[-] package \xpackage{float}, 2001/11/08, v1.3d,
%           \url{http://www.ctan.org/pkg/float}
%
% \item[-] package \xpackage{placeins}, 2005/04/18, v2.2,
%           \url{http://www.ctan.org/pkg/placeins}\\
%           for |\FloatBarrier|
%
% \item[-] package \xpackage{hyperref}, 2012/11/06, v6.83m,
%           \url{http://www.ctan.org/pkg/hyperref}
%
% \item[-] package \xpackage{tablefootnote}, 2014/01/26, v1.1c,\\
%           \url{http://www.ctan.org/pkg/tablefootnote}\\
%   (Well, it is the example file for this package, and because you
%    are reading the documentation for the \xpackage{tablefootnote} package,
%    it can be assumed that you already have some version of it~--
%    is it the current one?)
% \end{description}
%
% \DescribeMacro{Oberdiek}
% \DescribeMacro{holtxdoc}
% \DescribeMacro{ifluatex}
% \DescribeMacro{letltxmacro}
% \DescribeMacro{ltxcmds}
% All packages of \textsc{Heiko Oberdiek}'s bundle `oberdiek'
% (especially \xpackage{holtxdoc}, \xpackage{ifluatex}, \xpackage{letltxmacro},
% and \xpackage{ltxcmds}) are also available in a TDS compliant ZIP archive:\\
% \CTAN{install/macros/latex/contrib/oberdiek.tds.zip}.\\
% It is probably best to download and use this, because the packages in there
% are quite probably both recent and compatible among themselves.\\
%
% \pagebreak
%
% \DescribeMacro{hyperref}
% \noindent \xpackage{hyperref} is not included in that bundle and needs to be
% downloaded separately,\\
% \url{http://www.ctan.org/install/macros/latex/contrib/hyperref.tds.zip}.\\
%
% \DescribeMacro{M\"{u}nch}
% A hyperlinked list of my (other) packages can be found at
% \url{http://www.ctan.org/author/muench-hm}.\\
%
% \subsection{Package, unpacking TDS}
% \paragraph{Package.} This package is available on \CTAN{}
% \begin{description}
% \item[\CTAN{macros/latex/contrib/tablefootnote/tablefootnote.dtx}]\hspace*{0.1cm} \\
%       The source file.
% \item[\CTAN{macros/latex/contrib/tablefootnote/tablefootnote.pdf}]\hspace*{0.1cm} \\
%       The documentation.
% \item[\CTAN{macros/latex/contrib/tablefootnote/tablefootnote-example.pdf}]\hspace*{0.1cm} \\
%       The compiled example file, as it should look like.
% \item[\CTAN{macros/latex/contrib/tablefootnote/README}]\hspace*{0.1cm} \\
%       The README file.
% \end{description}
% There is also a tablefootnote.tds.zip available:
% \begin{description}
% \item[\CTAN{install/macros/latex/contrib/tablefootnote.tds.zip}]\hspace*{0.1cm} \\
%       Everything in \xfile{TDS} compliant, compiled format.
% \end{description}
% which additionally contains\\
% \begin{tabular}{ll}
% tablefootnote.ins & The installation file.\\
% tablefootnote.drv & The driver to generate the documentation.\\
% tablefootnote.sty & The \xext{sty}le file.\\
% tablefootnote-example.tex & The example file.
% \end{tabular}
%
% \bigskip
%
% \noindent For required other packages, please see the preceding subsection.
%
% \paragraph{Unpacking.} The \xfile{.dtx} file is a self-extracting
% \docstrip{} archive. The files are extracted by running the
% \xext{dtx} through \plainTeX{}:
% \begin{quote}
%   \verb|tex tablefootnote.dtx|
% \end{quote}
%
% \noindent About generating the documentation see paragraph~\ref{GenDoc} below.\\
%
% \paragraph{TDS.} Now the different files must be moved into
% the different directories in your installation TDS tree
% (also known as \xfile{texmf} tree):
% \begin{quote}
% \def\t{^^A
% \begin{tabular}{@{}>{\ttfamily}l@{ $\rightarrow$ }>{\ttfamily}l@{}}
%   tablefootnote.sty & tex/latex/tablefootnote/tablefootnote.sty\\
%   tablefootnote.pdf & doc/latex/tablefootnote/tablefootnote.pdf\\
%   tablefootnote-example.tex & doc/latex/tablefootnote/tablefootnote-example.tex\\
%   tablefootnote-example.pdf & doc/latex/tablefootnote/tablefootnote-example.pdf\\
%   tablefootnote.dtx & source/latex/tablefootnote/tablefootnote.dtx\\
% \end{tabular}^^A
% }^^A
% \sbox0{\t}^^A
% \ifdim\wd0>\linewidth
%   \begingroup
%     \advance\linewidth by\leftmargin
%     \advance\linewidth by\rightmargin
%   \edef\x{\endgroup
%     \def\noexpand\lw{\the\linewidth}^^A
%   }\x
%   \def\lwbox{^^A
%     \leavevmode
%     \hbox to \linewidth{^^A
%       \kern-\leftmargin\relax
%       \hss
%       \usebox0
%       \hss
%       \kern-\rightmargin\relax
%     }^^A
%   }^^A
%   \ifdim\wd0>\lw
%     \sbox0{\small\t}^^A
%     \ifdim\wd0>\linewidth
%       \ifdim\wd0>\lw
%         \sbox0{\footnotesize\t}^^A
%         \ifdim\wd0>\linewidth
%           \ifdim\wd0>\lw
%             \sbox0{\scriptsize\t}^^A
%             \ifdim\wd0>\linewidth
%               \ifdim\wd0>\lw
%                 \sbox0{\tiny\t}^^A
%                 \ifdim\wd0>\linewidth
%                   \lwbox
%                 \else
%                   \usebox0
%                 \fi
%               \else
%                 \lwbox
%               \fi
%             \else
%               \usebox0
%             \fi
%           \else
%             \lwbox
%           \fi
%         \else
%           \usebox0
%         \fi
%       \else
%         \lwbox
%       \fi
%     \else
%       \usebox0
%     \fi
%   \else
%     \lwbox
%   \fi
% \else
%   \usebox0
% \fi
% \end{quote}
% If you have a \xfile{docstrip.cfg} that configures and enables \docstrip{}'s
% \xfile{TDS} installing feature, then some files can already be in the right
% place, see the documentation of \docstrip{}.
%
% \subsection{Refresh file name databases}
%
% If your \TeX{}~distribution (\TeX live, \mikTeX{}, \dots{}) relies on file name
% databases, you must refresh these. For example, \teTeX{} users run
% \verb|texhash| or \verb|mktexlsr|.
%
% \subsection{Some details for the interested}
%
% \paragraph{Unpacking with \LaTeX{}.}
% The \xfile{.dtx} chooses its action depending on the format:
% \begin{description}
% \item[\plainTeX:] Run \docstrip{} and extract the files.
% \item[\LaTeX:] Generate the documentation.
% \end{description}
% If you insist on using \LaTeX{} for \docstrip{} (really,
% \docstrip{} does not need \LaTeX{}), then inform the autodetect routine
% about your intention:
% \begin{quote}
%   \verb|latex \let\install=y% \iffalse meta-comment
%
% File: tablefootnote.dtx
% Version: 2014/01/26 v1.1c
%
% Copyright (C) 2011 - 2014 by
%    H.-Martin M"unch <Martin dot Muench at Uni-Bonn dot de>
%
% This work may be distributed and/or modified under the
% conditions of the LaTeX Project Public License, either
% version 1.3c of this license or (at your option) any later
% version. This version of this license is in
%    http://www.latex-project.org/lppl/lppl-1-3c.txt
% and the latest version of this license is in
%    http://www.latex-project.org/lppl.txt
% and version 1.3c or later is part of all distributions of
% LaTeX version 2005/12/01 or later.
%
% This work has the LPPL maintenance status "maintained".
%
% The Current Maintainer of this work is H.-Martin Muench.
%
% This work consists of the main source file tablefootnote.dtx,
% the README, and the derived files
%    tablefootnote.sty, tablefootnote.pdf,
%    tablefootnote.ins, tablefootnote.drv,
%    tablefootnote-example.tex, tablefootnote-example.pdf.
%
% Distribution:
%    http://mirror.ctan.org/macros/latex/contrib/tablefootnote/tablefootnote.dtx
%    http://mirror.ctan.org/macros/latex/contrib/tablefootnote/tablefootnote.pdf
%    http://mirror.ctan.org/install/macros/latex/contrib/tablefootnote.tds.zip
%
% See http://www.ctan.org/pkg/tablefootnote (Catalogue)
%
% Unpacking:
%    (a) If tablefootnote.ins is present:
%           tex tablefootnote.ins
%    (b) Without tablefootnote.ins:
%           tex tablefootnote.dtx
%    (c) If you insist on using LaTeX
%           latex \let\install=y% \iffalse meta-comment
%
% File: tablefootnote.dtx
% Version: 2014/01/26 v1.1c
%
% Copyright (C) 2011 - 2014 by
%    H.-Martin M"unch <Martin dot Muench at Uni-Bonn dot de>
%
% This work may be distributed and/or modified under the
% conditions of the LaTeX Project Public License, either
% version 1.3c of this license or (at your option) any later
% version. This version of this license is in
%    http://www.latex-project.org/lppl/lppl-1-3c.txt
% and the latest version of this license is in
%    http://www.latex-project.org/lppl.txt
% and version 1.3c or later is part of all distributions of
% LaTeX version 2005/12/01 or later.
%
% This work has the LPPL maintenance status "maintained".
%
% The Current Maintainer of this work is H.-Martin Muench.
%
% This work consists of the main source file tablefootnote.dtx,
% the README, and the derived files
%    tablefootnote.sty, tablefootnote.pdf,
%    tablefootnote.ins, tablefootnote.drv,
%    tablefootnote-example.tex, tablefootnote-example.pdf.
%
% Distribution:
%    http://mirror.ctan.org/macros/latex/contrib/tablefootnote/tablefootnote.dtx
%    http://mirror.ctan.org/macros/latex/contrib/tablefootnote/tablefootnote.pdf
%    http://mirror.ctan.org/install/macros/latex/contrib/tablefootnote.tds.zip
%
% See http://www.ctan.org/pkg/tablefootnote (Catalogue)
%
% Unpacking:
%    (a) If tablefootnote.ins is present:
%           tex tablefootnote.ins
%    (b) Without tablefootnote.ins:
%           tex tablefootnote.dtx
%    (c) If you insist on using LaTeX
%           latex \let\install=y% \iffalse meta-comment
%
% File: tablefootnote.dtx
% Version: 2014/01/26 v1.1c
%
% Copyright (C) 2011 - 2014 by
%    H.-Martin M"unch <Martin dot Muench at Uni-Bonn dot de>
%
% This work may be distributed and/or modified under the
% conditions of the LaTeX Project Public License, either
% version 1.3c of this license or (at your option) any later
% version. This version of this license is in
%    http://www.latex-project.org/lppl/lppl-1-3c.txt
% and the latest version of this license is in
%    http://www.latex-project.org/lppl.txt
% and version 1.3c or later is part of all distributions of
% LaTeX version 2005/12/01 or later.
%
% This work has the LPPL maintenance status "maintained".
%
% The Current Maintainer of this work is H.-Martin Muench.
%
% This work consists of the main source file tablefootnote.dtx,
% the README, and the derived files
%    tablefootnote.sty, tablefootnote.pdf,
%    tablefootnote.ins, tablefootnote.drv,
%    tablefootnote-example.tex, tablefootnote-example.pdf.
%
% Distribution:
%    http://mirror.ctan.org/macros/latex/contrib/tablefootnote/tablefootnote.dtx
%    http://mirror.ctan.org/macros/latex/contrib/tablefootnote/tablefootnote.pdf
%    http://mirror.ctan.org/install/macros/latex/contrib/tablefootnote.tds.zip
%
% See http://www.ctan.org/pkg/tablefootnote (Catalogue)
%
% Unpacking:
%    (a) If tablefootnote.ins is present:
%           tex tablefootnote.ins
%    (b) Without tablefootnote.ins:
%           tex tablefootnote.dtx
%    (c) If you insist on using LaTeX
%           latex \let\install=y\input{tablefootnote.dtx}
%        (quote the arguments according to the demands of your shell)
%
% Documentation:
%    (a) If tablefootnote.drv is present:
%           (pdf)latex tablefootnote.drv
%           makeindex -s gind.ist tablefootnote.idx
%           (pdf)latex tablefootnote.drv
%           makeindex -s gind.ist tablefootnote.idx
%           (pdf)latex tablefootnote.drv
%    (b) Without tablefootnote.drv:
%           (pdf)latex tablefootnote.dtx
%           makeindex -s gind.ist tablefootnote.idx
%           (pdf)latex tablefootnote.dtx
%           makeindex -s gind.ist tablefootnote.idx
%           (pdf)latex tablefootnote.dtx
%
%    The class ltxdoc loads the configuration file ltxdoc.cfg
%    if available. Here you can specify further options, e.g.
%    use DIN A4 as paper format:
%       \PassOptionsToClass{a4paper}{article}
%
% Installation:
%    TDS:tex/latex/tablefootnote/tablefootnote.sty
%    TDS:doc/latex/tablefootnote/tablefootnote.pdf
%    TDS:doc/latex/tablefootnote/tablefootnote-example.tex
%    TDS:doc/latex/tablefootnote/tablefootnote-example.pdf
%    TDS:source/latex/tablefootnote/tablefootnote.dtx
%
%<*ignore>
\begingroup
  \catcode123=1 %
  \catcode125=2 %
  \def\x{LaTeX2e}%
\expandafter\endgroup
\ifcase 0\ifx\install y1\fi\expandafter
         \ifx\csname processbatchFile\endcsname\relax\else1\fi
         \ifx\fmtname\x\else 1\fi\relax
\else\csname fi\endcsname
%</ignore>
%<*install>
\input docstrip.tex
\Msg{******************************************************************}
\Msg{* Installation                                                   *}
\Msg{* Package: tablefootnote 2014/01/26 v1.1c Table foot notes (HMM) *}
\Msg{******************************************************************}

\keepsilent
\askforoverwritefalse

\let\MetaPrefix\relax
\preamble

This is a generated file.

Project: tablefootnote
Version: 2014/01/26 v1.1c

Copyright (C) 2011 - 2014 by
    H.-Martin M"unch <Martin dot Muench at Uni-Bonn dot de>

The usual disclaimer applies:
If it doesn't work right that's your problem.
(Nevertheless, send an e-mail to the maintainer
 when you find an error in this package.)

This work may be distributed and/or modified under the
conditions of the LaTeX Project Public License, either
version 1.3c of this license or (at your option) any later
version. This version of this license is in
   http://www.latex-project.org/lppl/lppl-1-3c.txt
and the latest version of this license is in
   http://www.latex-project.org/lppl.txt
and version 1.3c or later is part of all distributions of
LaTeX version 2005/12/01 or later.

This work has the LPPL maintenance status "maintained".

The Current Maintainer of this work is H.-Martin Muench.

This work consists of the main source file tablefootnote.dtx,
the README, and the derived files
   tablefootnote.sty, tablefootnote.pdf,
   tablefootnote.ins, tablefootnote.drv,
   tablefootnote-example.tex, tablefootnote-example.pdf.

In memoriam Tommy Muench + 2014/01/02.

\endpreamble
\let\MetaPrefix\DoubleperCent

\generate{%
  \file{tablefootnote.ins}{\from{tablefootnote.dtx}{install}}%
  \file{tablefootnote.drv}{\from{tablefootnote.dtx}{driver}}%
  \usedir{tex/latex/tablefootnote}%
  \file{tablefootnote.sty}{\from{tablefootnote.dtx}{package}}%
  \usedir{doc/latex/tablefootnote}%
  \file{tablefootnote-example.tex}{\from{tablefootnote.dtx}{example}}%
}

\catcode32=13\relax% active space
\let =\space%
\Msg{************************************************************************}
\Msg{*}
\Msg{* To finish the installation you have to move the following}
\Msg{* file into a directory searched by TeX:}
\Msg{*}
\Msg{*     tablefootnote.sty}
\Msg{*}
\Msg{* To produce the documentation run the file `tablefootnote.drv'}
\Msg{* through (pdf)LaTeX, e.g.}
\Msg{*  pdflatex tablefootnote.drv}
\Msg{*  makeindex -s gind.ist tablefootnote.idx}
\Msg{*  pdflatex tablefootnote.drv}
\Msg{*  makeindex -s gind.ist tablefootnote.idx}
\Msg{*  pdflatex tablefootnote.drv}
\Msg{*}
\Msg{* At least three runs are necessary e.g. to get the}
\Msg{*  references right!}
\Msg{*}
\Msg{* Happy TeXing!}
\Msg{*}
\Msg{************************************************************************}

\endbatchfile
%</install>
%<*ignore>
\fi
%</ignore>
%
% \section{The documentation driver file}
%
% The next bit of code contains the documentation driver file for
% \TeX{}, i.\,e., the file that will produce the documentation you
% are currently reading. It will be extracted from this file by the
% \texttt{docstrip} programme. That is, run \LaTeX{} on \texttt{docstrip}
% and specify the \texttt{driver} option when \texttt{docstrip}
% asks for options.
%
%    \begin{macrocode}
%<*driver>
\NeedsTeXFormat{LaTeX2e}[2011/06/27]
\ProvidesFile{tablefootnote.drv}%
  [2014/01/26 v1.1c Table foot notes (HMM)]
\documentclass{ltxdoc}[2007/11/11]% v2.0u
\usepackage{holtxdoc}[2012/03/21]%  v0.24
%% tablefootnote may work with earlier versions of LaTeX2e and those
%% class and package, but this was not tested.
%% Please consider updating your LaTeX, class, and package
%% to the most recent version (if they are not already the most
%% recent version).
\hypersetup{%
 pdfsubject={Footnotes in Tables (HMM)},%
 pdfkeywords={LaTeX, tablefootnote, footnote, table, H.-Martin Muench},%
 pdfencoding=auto,%
 pdflang={en},%
 breaklinks=true,%
 linktoc=all,%
 pdfstartview=FitH,%
 pdfpagelayout=OneColumn,%
 bookmarksnumbered=true,%
 bookmarksopen=true,%
 bookmarksopenlevel=3,%
 pdfmenubar=true,%
 pdftoolbar=true,%
 pdfwindowui=true,%
 pdfnewwindow=true%
}
\CodelineIndex
\hyphenation{printing docu-ment}
\gdef\unit#1{\mathord{\thinspace\mathrm{#1}}}%
\makeatletter
\@ifundefined{eTeX}{\gdef\eTeX{$\m@th \varepsilon $-\TeX }}{% else \relax
}
\makeatother
\begin{document}
  \DocInput{tablefootnote.dtx}%
\end{document}
%</driver>
%    \end{macrocode}
%
% \fi
%
% \CheckSum{584}
%
% \CharacterTable
%  {Upper-case    \A\B\C\D\E\F\G\H\I\J\K\L\M\N\O\P\Q\R\S\T\U\V\W\X\Y\Z
%   Lower-case    \a\b\c\d\e\f\g\h\i\j\k\l\m\n\o\p\q\r\s\t\u\v\w\x\y\z
%   Digits        \0\1\2\3\4\5\6\7\8\9
%   Exclamation   \!     Double quote  \"     Hash (number) \#
%   Dollar        \$     Percent       \%     Ampersand     \&
%   Acute accent  \'     Left paren    \(     Right paren   \)
%   Asterisk      \*     Plus          \+     Comma         \,
%   Minus         \-     Point         \.     Solidus       \/
%   Colon         \:     Semicolon     \;     Less than     \<
%   Equals        \=     Greater than  \>     Question mark \?
%   Commercial at \@     Left bracket  \[     Backslash     \\
%   Right bracket \]     Circumflex    \^     Underscore    \_
%   Grave accent  \`     Left brace    \{     Vertical bar  \|
%   Right brace   \}     Tilde         \~}
%
% \GetFileInfo{tablefootnote.drv}
%
% \begingroup
%   \def\x{\#,\$,\^,\_,\~,\ ,\&,\{,\},\%}%
%   \makeatletter
%   \@onelevel@sanitize\x
% \expandafter\endgroup
% \expandafter\DoNotIndex\expandafter{\x}
% \expandafter\DoNotIndex\expandafter{\string\ }
% \begingroup
%   \makeatletter
%     \lccode`9=32\relax
%     \lowercase{%^^A
%       \edef\x{\noexpand\DoNotIndex{\@backslashchar9}}%^^A
%     }%^^A
%   \expandafter\endgroup\x
%
% \DoNotIndex{\\,\,}
% \DoNotIndex{\documentclass,\usepackage,\ProvidesPackage,\begin,\end}
% \DoNotIndex{\MessageBreak,\message}
% \DoNotIndex{\NeedsTeXFormat,\DoNotIndex,\verb}
% \DoNotIndex{\def,\edef,\gdef,\global}
% \DoNotIndex{\ifx,\listfiles,\mathord,\mathrm}
% \DoNotIndex{\kvoptions,\SetupKeyvalOptions,\ProcessKeyvalOptions}
% \DoNotIndex{\bigskip,\space,\thinspace,\ldots}
% \DoNotIndex{\indent,\noindent,\newline,\linebreak,\pagebreak}
% \DoNotIndex{\textbf,\textit,\textsf,\texttt,\textsc,\textquotedblleft,\textquotedblright}
% \DoNotIndex{\plainTeX,\TeX,\LaTeX,\pdfLaTeX}
% \DoNotIndex{\chapter,\section,\the,\csname,\endscname}
% \DoNotIndex{\value,\arabic,\hline,\text,\caption,\centering,\clear,\url}
%
% \title{The \xpackage{tablefootnote} package}
% \date{2014/01/26 v1.1c}
% \author{H.-Martin M\"{u}nch\\\xemail{Martin.Muench at Uni-Bonn.de}}
%
% \maketitle
%
% \begin{abstract}
% This \LaTeX{} package provides the command |\tablefootnote| to be used
% in a \texttt{table} or \texttt{sidewaystable} environment,
% where |\footnote| will not work and when using |\footnotemark| and
% |\footnotetext| and adjusting the counters (including \texttt{Hfootnote})
% manually is either too much work or would not even work (\texttt{sidewaystable}).
% \end{abstract}
%
% \bigskip
%
% \noindent Disclaimer for web links: The author is not responsible for any contents
% referred to in this work unless he has full knowledge of illegal contents.
% If any damage occurs by the use of information presented there, only the
% author of the respective pages might be liable, not the one who has referred
% to these pages.
%
% \bigskip
%
% \noindent {\color{green} Save per page about $200\unit{ml}$ water,
% $2\unit{g}$ CO$_{2}$ and $2\unit{g}$ wood:\\
% Therefore please print only if this is really necessary.}
%
% \pagebreak
%
% \tableofcontents
%
% \pagebreak
%
% \section{Introduction\label{sec:Introduction}}
%
% \indent This \LaTeX{} package provides the command |\tablefootnote{...}| to be
% used in a \texttt{table} or \texttt{sidewaystable} environment. In those
% environments |\footnote{...}| would not work and would need to be replaced
% by |\footnotemark| in the \texttt{(sideways)\allowbreak table} and
% |\footnotetext{...}| after the \texttt{(sideways)table} environment ended.
% Additionally the counters for \texttt{footnote} and
% (when the \xpackage{hyperref} package is used) \texttt{Hfootnote}
% would need to be adjusted manually. Then still the hyperlinks
% in a \texttt{(sideways)table} would not work.
% When the \xpackage{footnotebackref} package is used,
% |\footnotemark| and |\footnotetext{...}| would need to be redefined.
% Just using |\tablefootnote{...}| (with optional argument for custom foot note
% marks) in the (sideways)table does all this automatically.
% (Compatibility with the \xpackage{footnotebackref} package
% was achieved without redefining |\footnotemark| or |\footnotetext|.)
% Redefining |\footnote| was \emph{not} done as to not create conflicts with
% other packages.
%
% \section{Usage}
%
% \indent Just load the package placing
% \begin{quote}
%   |\usepackage{tablefootnote}|
% \end{quote}
% \noindent in the preamble of your \LaTeXe{} source file and use\newline
% |\tablefootnote{...}| instead of |\footnote{...}| and\newline
% |\tablefootnote[...]{...}| instead of |\footnote[...]{...}|
% in (sideways)\allowbreak tables.\\
% When the \xpackage{footnotebackref} package is used,
% it must be loaded before the \xpackage{tablefootnote} package
% and the \xpackage{hyperrefpackage} with option |hyperfootnotes=true|
% must be loaded, too. (Backreference{-}links without links do not make sense.)\\
% When the \xpackage{rotating} package (for |sidewaystable|s) is used,
% it must be loaded before the \xpackage{tablefootnote} package and it must be
% a recent version (v2.16a, 2009/03/28, or newer).\\
% If |\tablefootnote{...}| is used in a tabular environment, which is inside
% of a (sideways)table environment, everything is fine, but when the tabular
% environment is not inside of a (sideways)table, the \textbf{table}footnote
% will not work.\\
%
% Pages with tables with footnotes, which are rotated with the \xpackage{lscape}-
% or with the \xpackage{pdflscape}-package (|\begin{landscape}|, table, |\end{landscape}|),
% are possible. But when the \xpackage{footmisc}-package is used, \xpackage{(pdf)lscape}
% should be loaded before \xpackage{footmisc}.\\
%
% When the \xpackage{footmisc}-package is used with option \texttt{para},
% this is regarded with the exception of a \texttt{sidewaystable}. There the footnotes
% are printed one below the other. (Well, this is better than no footnote in a
% \texttt{sidewaystable} at all, isn't it?)\\
%
% When the (sideways)table floats, the footnotes are set and (when \xpackage{hyperref}
% is used) hyperlinked, but they are not automatically adapted when the table floats
% over/under another footnote. Thus either do not use a footnote between original and
% float position of the (sideways)table, or place the (sideways)table in
% \textquotedblleft here\textquotedblright\ position. |\clear(double)page|, |h(!)|,
% |H|~from the \xpackage{float} package (\url{http://www.ctan.org/pkg/float}),
% or |\FloatBarrier| from the \xpackage{picins} package
% (\url{http://www.ctan.org/pkg/picins}) might help, too.
% (Or after finishing the document, move the (sideways)table in the source code
% near the position where it floats to or use the optional footnote marks.)
%
% \pagebreak
%
% \section{Alternatives\label{sec:Alternatives}}
%
% \begin{description}
% \item[-] The \xpackage{longtable} package provides the \texttt{longtable} environment
%   as replacement for the combined \texttt{table} and \texttt{tabular} environments.
%   Footnotes are real footnotes (not just tablenotes), are continuously numbered and
%   hyperlinked (when using the \xpackage{hyperref} package), and the hyperlinks really
%   work. As drawback the appearance of the caption changes slightly (e.\,g.~distance
%   to the table, width of the caption), which can probably be changed back manually.
%   Furthermore, longtables are meaned to break over more than one page. If that is not
%   wished, it must be prevented by |\nopagebreak|-commands and by ending the longtable
%   lines with |\\*| instead of |\\|. longtables do not float. (Therefore using the
%   \xpackage{tablefootnote} package and |\FloatBarrier| from the \xpackage{picins} package
%   before and after the table environment is similar~- but \xpackage{tablefootnote} does not
%   change the table-caption!) \texttt{sidewaystable} does not work with it.\\
%   \url{http://www.ctan.org/pkg/longtable}
%
% \item[-] The \xpackage{supertabular} package provides the \texttt{mpsupertabular}
%   environment as replacement for the combined \texttt{table} and \texttt{tabular}
%   environments. Footnotes are just tablenotes (with working hyperlinks when using the
%   \xpackage{hyperref} package), i.\,e. numbered a, b, c and placed below the table
%   and not at the end of the page. Therefore there is no float problem (because the
%   tablenotes numbering is not included in the continuous numbering of the footnotes).
%   Placing the \texttt{supertabular} inside of a \texttt{sidewaystable} breaks the
%   hyperlinks to the tablenotes.\\
%   \url{http://www.ctan.org/pkg/supertabular}
%
% \item[-] The \xpackage{ctable} package has its very own notation for defining tables.
%   It can create tablenotes and sideways-tables. The tablenotes are not automatically
%   hyperlinked. The \texttt{ctable}s float. Because the tablenotes numbering is not
%   included in the continuous numbering of the footnotes there is no float problem.\\
%   \url{http://www.ctan.org/pkg/ctable}
%
% \item[-] The \xpackage{footnote} package provides |\makesavenoteenv{table}|. After
%   loading the package and using that command in the preamble, in tables |\footnote{...}|
%   can be used. Using |\makesavenoteenv{tabular}| and \\
%   |\makesavenoteenv{sidewaystable}|
%   is possible, but it neither solves the float problem, nor do the created hyperlinks
%   work (i.\,e.~they aim at wrong locations). The \xpackage{mdwtab} from the same bundle
%   is incompatible with other table-related packages (e.\,g. \xpackage{supertabular},
%   \xpackage{array}) and not $100\,\%$ compatible with the \texttt{tabular} environment.\\
%   \url{http://www.ctan.org/pkg/footnote}\\
%   \url{http://www.ctan.org/pkg/mdwtab}
%
% \item[-] The \xpackage{tabularx} package does produce footnotes for sidewaystables,
%   but uses a, b, c instead of 1, 2, 3. The hyperlinks to the footnotes do not work.
%   Because the footnotes numbering is not included in the continuous numbering of
%   the other footnotes there is no float problem.\\
%   \url{http://www.ctan.org/pkg/tabularx}
%
% \item[-] Placing a tabular inside a minipage inside a table produces tablenotes.
%   Therefore there is no float problem (because the footnotes are not continuously
%   numbered). The hyperlinks to the table notes indeed work.
%
% \pagebreak
%
% \item[-] The \xpackage{threeparttable} package creates tablenotes again. Therefore
%   there is no float problem (because the tablenotes are not continuously numbered with
%   the footnotes). There are no hyperlinks to the table notes (at least not automatically).
%   Using |sidewaystable| (with table notes) works.\\
%   \url{http://www.ctan.org/pkg/threeparttable}
%
% \item[-] The \xpackage{threeparttablex} package creates tablenotes again. Therefore there
%   is no float problem (because the tablenotes are not continuously numbered with the
%   footnotes). With option \texttt{referable} the tablenotes are hyperlinked. Use of
%   a |sidewaystable| (with table notes) did not work for me. When using the
%   \texttt{referable} option according to the example in the \xpackage{threeparttablex}
%   manual the \xpackage{longtable} package is used, therefore that package could be used
%   directly without \xpackage{threeparttablex} (see above).\\
%   \url{http://www.ctan.org/pkg/threeparttablex}
%
% \item[-] One can manually use |\footnotemark| in the table and |\footnotetext{...}|
%  after the table environment ended and manually change the |footnote| and
%  (when the \xpackage{hyperref} package is used) |Hfootnote| counters and needs
%  to insert
%  \begin{verbatim}
%     \makeatletter
%     \global\let\Hy@saved@currentHref\@currentHref
%     \hyper@makecurrent{Hfootnote}%
%     \global\let\Hy@footnote@currentHref\@currentHref
%     \global\let\@currentHref\Hy@saved@currentHref
%     \makeatother
%  \end{verbatim}
%  before each |\footnotetext{...}|, but with more than very few footnotes this can become
%  much work (and is prone to produce errors). And this does not even work with
%  sidewaystables. (And it does not work with the \xpackage{footnotebackref} package
%  without redefining |\footnotemark| and |\footnotetext{...}|.)
% \end{description}
%
% \bigskip
%
% \noindent (You programmed or found another alternative,
%  which is available at \CTAN{}?\\
%  OK, send an e-mail to me with the name, location at \CTAN{},
%  and a short notice, and I will probably include it in
%  the list above.)
%
% \newpage
%
% \section{Example}
%
%    \begin{macrocode}
%<*example>
\documentclass[british]{article}[2007/10/19]% v1.4h
%%%%%%%%%%%%%%%%%%%%%%%%%%%%%%%%%%%%%%%%%%%%%%%%%%%%%%%%%%%%%%%%%%%%%
\usepackage{float}[2001/11/08]%    v1.3d
\usepackage{placeins}[2005/04/18]% v2.2  ; for \FloatBarrier
\usepackage{rotating}[2009/03/28]% v2.16a; for sidewaystable-environment
\usepackage[%
 hyperfootnotes=true,%
 extension=pdf,%
 plainpages=false,%
 pdfpagelabels=true,%
 hyperindex=false,%
 pdflang={en},%
 pdftitle={tablefootnote package example},%
 pdfauthor={H.-Martin Muench},%
 pdfsubject={Example for the tablefootnote package},%
 pdfkeywords={LaTeX, tablefootnote, footnote, table, H.-Martin Muench},%
 % pdfview=FitH and FitBH do not work: hyperlinks in sidewaystables
 % do no lead to the footnotes, due to a bug in pdfTeX,
 % computing wrong anchor coordinates (Heiko Oberdiek, 29. October 2011)
 % pdfview=Fit, FitV, FitR, FitB, FitBV work
 % print is OK for all those options
 pdfstartview=FitH,%
 pdfpagelayout=OneColumn%
]{hyperref}[2012/11/06]% v6.83m
 % Due to the urls used in the example, either the hyperref or the url
 % package are needed (or the urls must be removed before compiling).

\usepackage{footnotebackref}[2012/07/01]% v1.0
\usepackage{tablefootnote}[2014/01/26]%   v1.1c

\gdef\unit#1{\mathord{\thinspace\mathrm{#1}}}%
\listfiles
\begin{document}
\pagenumbering{arabic}
\section*{Example for tablefootnote}

This example demonstrates the use of package\newline
\textsf{tablefootnote}, v1.1c as of 2014/01/26 (HMM).\newline
There were no options used. (The package provides no options.)\newline

\textbf{The \texttt{tablefootnote-example.tex} needs to be compiled
at least twice to get the references right!}\newline

If the etoolbox-package is found, it is automatically used.\newline

For more details please see the documentation!\newline

\noindent Save per page about $200\unit{ml}$ water,
$2\unit{g}$ CO$_{2}$ and $2\unit{g}$ wood:\newline
Therefore please print only if this is really necessary.\newline

Here is some text.\footnote{And this is a text footnote.}\newline

Tables \ref{tab.symbol}, \ref{tab.normal}, \ref{tab.another} and
\ref{tab.floatH} show normal tables, table~\ref{tab.sideways} depicts
a sidewaystable. Table~\ref{tab.floatH} uses the float
specifier~\texttt{H} from the float package.\newline

\texttt{Hyperref} option \verb|pdfview=FitH| and \verb|FitBH| do
not work due to a bug in pdf\TeX{}, computing wrong
anchor coordinates (\textsc{Heiko Oberdiek}, 29.~October 2011).
Depending on used pdf-viewer, hyperlinks in sidewaystables lead
e.\,g.~at the end of the document, not at the footnote.
\verb|pdfview=Fit|, \verb|FitV|, \verb|FitR|, \verb|FitB|,
\verb|FitBV| work, print is OK for all those options.

\bigskip

\listoftables

\pagebreak

\renewcommand{\thefootnote}{\fnsymbol{footnote}}
\verb|\renewcommand{\thefootnote}{\fnsymbol{footnote}}|
causes footnote{-}symbol{-}footnotes, which are possible
(see Table~\ref{tab.symbol}).

\begin{table}
\centering
\begin{tabular}{ccc}
Another\tablefootnote{A table footnote.} & %
text\tablefootnote{Another table footnote.} & %
in a table\tablefootnote{A third table footnote.}
\end{tabular}
\caption[A footnotesymbol table]{%
A table with footnote-symbol-footnotes.\label{tab.symbol}}
\end{table}

Some text.\footnote{A text footnote.}

\renewcommand{\thefootnote}{\arabic{footnote}}
\verb|\renewcommand{\thefootnote}{\arabic{footnote}}|
switches back to normal footnote numbers again.

\pagebreak

\begin{table}
\centering
\begin{tabular}{ccc}
Some\tablefootnote{A table footnote.} & %
text\tablefootnote[99]{A table footnote with custom footnote number.} & %
in a table\tablefootnote{A third table footnote.}
\end{tabular}
\caption[A table]{A normal table.\label{tab.normal}}
\end{table}

Some text.\footnote{Another text footnote.}

\pagebreak

%    \end{macrocode}
%
% \pagebreak
%
%    \begin{macrocode}
More text.\footnote{And yet another text footnote.}

\begin{table}[t]
\centering
\begin{tabular}{|c|c|c|}
\hline
Another\tablefootnote{A $3^{rd}$ table footnote.} & %
text\tablefootnote{Another $3^{rd}$ table footnote.} & %
in a table\tablefootnote{A $3^{rd}$ third table footnote.}\\ \hline
\end{tabular}
\caption[Another table]{Another table (third one)\label{tab.another}}
\end{table}

Please note that Table~\ref{tab.another} floated to the top of the page.
While the footnotes are set and (when hyperref is used) hyperlinked,
they are not automatically adapted. Thus either do not use a footnote
at the same page before the table, or place the table in
\textquotedblleft here\textquotedblright\ or
\textquotedblleft bottom\textquotedblright\ position.
\verb|\clear(double)page|, \verb|h(!)|, \verb|H|~from the
\texttt{float} package, or \verb|\FloatBarrier| from the
\texttt{picins} package might help, too. (Or move the table in the
source code near the position where it floats to
or use the optional footnote marks.)

Table~\ref{tab.floatH} (page~\pageref{tab.floatH}) uses float specifier
\texttt{H} from the float package and does not float.

Some text.\footnote{This is just another text footnote.}

\pagebreak

\FloatBarrier

\begin{sidewaystable}
\centering%
\begin{tabular}{ccc}
Text\tablefootnote{Please rotate the view for testing the %
hyperlinks.} & %
in a\tablefootnote[98]{Another sidewaystable footnote %
with optional footnote mark.} & %
sidewaystable%
\tablefootnote{Sidewaystable-environment provided by %
\url{http://www.ctan.org/pkg/rotating} package.}
\end{tabular}
\caption[A sideways table]{A table in the \texttt{sideways} %
environment\label{tab.sideways}}%
\end{sidewaystable}%

\FloatBarrier

\pagebreak

%    \end{macrocode}
%
% \pagebreak
%
%    \begin{macrocode}
A last table, here with float specifier \texttt{H} from the
float\footnote{\url{http://www.ctan.org/pkg/float}} package.

\begin{table}[H]
\centering
\begin{tabular}{ccc}
Another\tablefootnote{A $5^{th}$ table footnote.} & %
text\tablefootnote{Another $5^{th}$ table footnote.} & %
in a table\tablefootnote{A $5^{th}$ third table footnote.}
\end{tabular}
\caption[A last table]{A very last table\label{tab.floatH}}
\end{table}

Some text.\footnote{This is just another text footnote.}

\pagebreak

End of the example for the
tablefootnote\footnote{\url{http://www.ctan.org/pkg/tablefootnote}}
package.

\end{document}
%</example>
%    \end{macrocode}
%
% \StopEventually{}
%
% \newpage
%
% \section{The implementation}
%
% We start off by checking that we are loading into \LaTeXe{} and
% announcing the name and version of this package.
%
%    \begin{macrocode}
%<*package>
%    \end{macrocode}
%
%    \begin{macrocode}
\NeedsTeXFormat{LaTeX2e}[2011/06/27]
\ProvidesPackage{tablefootnote}[2014/01/26 v1.1c
              Table foot notes (HMM)]
%    \end{macrocode}
%
% A short description of the \xpackage{tablefootnote} package:
%
%    \begin{macrocode}
%% Provides the \tablefootnote[...]{...} command
%% for footnotes in (sideways)tables.

%    \end{macrocode}
%
% We need the \xpackage{ltxcmds} and \xpackage{letltxmacro} packages
% by \textsc{Heiko Oberdiek}:
%
%    \begin{macrocode}
\RequirePackage{ltxcmds}[2011/11/09]%     v1.22
\RequirePackage{letltxmacro}[2010/09/02]% v1.4
\RequirePackage{xifthen}[2009/04/17]%     v1.3
%    \end{macrocode}
%
% A last information for the user:
%
%    \begin{macrocode}
%% tablefootnote may work with earlier versions of LaTeX and those
%% packages, but this was not tested. Please consider updating
%% your LaTeX and packages to the most recent version
%% (if they are not already the most recent version).

%    \end{macrocode}
%
% \indent See subsection~\ref{ss:Downloads} about how to get them.\\
%
% When the \xpackage{rotating} package (for |sidewaystable|s) is used,
% it must be loaded before the \xpackage{tablefootnote} package and it must be
% a recent version:
%
%    \begin{macrocode}
\@ifpackageloaded{rotating}{%
  \gdef\tfn@rotating{1}
  \@ifpackagelater{rotating}{2009/03/28}% v2.16a
   {% >= 2009/03/28, OK
   }{% else
     \PackageWarningNoLine{tablefootnote}{%
       It is required version\MessageBreak%
       2009/03/28 v2.16a (or later) of package rotating,\MessageBreak%
       but only version\MessageBreak%
      `\csname ver@rotating.sty\endcsname'\MessageBreak%
      is available}%
   }%
 }{% else
   \gdef\tfn@rotating{0}%
 }

%    \end{macrocode}
%
% There are no options to be processed.\\
%
% We need some definitions:
%
%    \begin{macrocode}
\def\tfn@footnotetablecount{0}
\def\tfn@footnotetableprint{0}
\def\tfn@footnotezero{0}
\def\tfn@fnt{0}
\def\tfn@swt{0}%       _s_ide_w_ays_t_able
\def\tfn@fmpw{\relax}% f_oot_m_isc_p_ara_w_arn-switch
\def\tfn@fnbr{0}%      f_oot_n_ote_b_ack_r_ef(erence package)

%    \end{macrocode}
%
% \begin{macro}{\tfn@hyperfootnotes}
% We need a wrapper for the used |\ifHy@hyperfootnotes|,
% which is not defined when the \xpackage{hyperref} package has not been loaded:
%
%    \begin{macrocode}
\DeclareRobustCommand{\tfn@hyperfootnotes}[2]{%
  \ifHy@hyperfootnotes%
    #1%
  \else%
    #2%
  \fi%
  }

%    \end{macrocode}
% \end{macro}
%
% \begin{macro}{\tablemakefntext}
% |\tablemakefntext| makes the footnotetext for a footnote in a table.
%
%    \begin{macrocode}
\ltx@ifpackageloaded{footnotebackref}{%
  \gdef\tfn@fnbr{1}%
  \ifx\FootnoteBackref@symbol\empty%
    \ifFootnoteBackref@numberlinked%
      \newcommand\tablemakefntext[1]{%
        \noindent\makebox[1.634em][r]{%
         \mbox{\textsuperscript{\normalfont%
           \hyperlink{tfn:m\theHfootnote}{\@thefnmark}%
        }}}%
        \ #1%
      }
    \else
      \newcommand\tablemakefntext[1]{%
        \noindent\makebox[1.8em][r]{%
          \mbox{\textsuperscript{\normalfont%
            \@thefnmark%
        }}}%
        #1%
      }
    \fi
  \else
    \newcommand\tablemakefntext[1]{%
      \noindent\makebox[1.8em][r]{%
        \mbox{\textsuperscript{\normalfont%
          \ifFootnoteBackref@numberlinked%
            \hyperlink{tfn:m\theHfootnote}{\@thefnmark}\ %
          \else%
            \@thefnmark%
          \fi%
      }}}%
      \hyperlink{tfn:m\theHfootnote}{\FootnoteBackref@symbol}%
      #1%
    }
  \fi
}{%
  \newcommand\tablemakefntext[1]{%
    \noindent\makebox[1.8em][r]{%
      \mbox{\textsuperscript{\normalfont%
        \@thefnmark%
    }}}%
    #1%
  }
}

%    \end{macrocode}
%
% \end{macro}
%
% \pagebreak
%
% \begin{macro}{\swtablemakefntext}
%    \begin{macrocode}
\newcommand\swtablemakefntext[1]{\noindent\ #1}

%    \end{macrocode}
% \end{macro}
%
% \begin{macro}{\tfn@footnotemarks}
% |\tfn@footnotemarks| uses the |\footnotemark| command to place a foot note mark
% and decreases the |\tfn@footnotetablecount| (i.\,e.~increases the negative value)
% of the number of footnote(marks) in the table. If the \xpackage{footnotebackref}
% package is used, it is also necessary to create an according |\hypertarget| for
% the back reference.
%
%    \begin{macrocode}
\newcommand{\tfn@footnotemarks}[1][]{%
  \hbox{%
    \ifthenelse{\isempty{#1}}%
      {\footnotemark%
       \ltx@ifpackageloaded{footnotebackref}%
         {\raisebox{\ht\strutbox}{\hypertarget{tfn:m\theHfootnote}{}}}%
         {% footnotebackref not loaded
         }%
      }%
      {\ltx@ifpackageloaded{hyperref}%
        {\tfn@hyperfootnotes{%
            \hyperlink{tfn:#1}{\footnotemark[#1]}%
          }{\footnotemark[#1]}}%
        {\footnotemark[#1]}%
       \ltx@ifpackageloaded{footnotebackref}%
        {\raisebox{\ht\strutbox}{\hypertarget{tfn:m#1}{}}}{}%
      }%
   }%
  \@tempcnta=\tfn@footnotetablecount\relax%
  \advance\@tempcnta by -1\relax%
  \xdef\tfn@footnotetablecount{\the\@tempcnta}%
}

%    \end{macrocode}
% \end{macro}
%
% \begin{macro}{\tfn@footmiscparawarn}
% When the \xpackage{footmisc} package is used with option |para|,
% we need to issue a warning. (Any idea how to resolve this issue?)\\
% It must be given inside of it's own command, because |\ifFN@para| is not
% defined when \xpackage{footmisc} has not been loaded, and then the
% |\fi| would be a |\fi| without |\if| (and \TeX{} does not like single
% |\fi|s).
%
%    \begin{macrocode}
\newcommand{\tfn@footmiscparawarn}{%
  \ifFN@para%
    \PackageWarning{tablefootnote}{%
      Package footmisc with option para used.\MessageBreak%
      Unfortunately package tablefootnote is not yet\MessageBreak%
      able to regard this in sidewaystables.\MessageBreak%
      Here the footnotes are printed one below\MessageBreak%
      the other. Sorry!\MessageBreak%
     }%
  \fi%
  }

%    \end{macrocode}
% \end{macro}
%
% \begin{macro}{\tfn@updateHyper}
% |\tfn@updateHyper| updates some \xpackage{hyperref} internals.
%
%    \begin{macrocode}
\newcommand{\tfn@updateHyper}{%
  % The code of this command is partly from the
  % hyperref package by Heiko Oberdiek, 2011/10/01, v6.82j.
  % Newer versions of that package are available at CTAN.org.
  \global\let\Hy@saved@currentHref\@currentHref%
  \tfn@hyperfootnotes{\hyper@makecurrent{Hfootnote}}{}%
  \global\let\Hy@footnote@currentHref\@currentHref%
  \global\let\@currentHref\Hy@saved@currentHref%
  }

%    \end{macrocode}
% \end{macro}
%
% \begin{macro}{\tfn@footnotetext}
% |\tfn@footnotetext| contains the |\footnotetext[...]{...}| command and handles
% the numbering. |\tfn@footnotetext| will be executed after the table.
% The foot note counter has been increased automatically with each |\footnotemark|,
% therefore it must be reset to its value before the table began.
%
%    \begin{macrocode}
\newcommand{\tfn@footnotetext}[2][]{%
  \ifx\tfn@footnotetableprint\tfn@footnotezero%
    % This is the first footnotetext after the (sideways)table.
    \addtocounter{footnote}{\tfn@footnotetablecount}%
%    \end{macrocode}
%
% When the \xpackage{hyperref} package is used, also its |Hfootnote| counter must
% be reset (when the footnotes are hyperlinked).
%
%    \begin{macrocode}
    \ltx@ifpackageloaded{hyperref}{%
      \tfn@hyperfootnotes{%
        \addtocounter{Hfootnote}{\tfn@footnotetablecount}}{}%
      \tfn@updateHyper%
     }{}%
  \fi%
%    \end{macrocode}
%
% For issuing the text for the next footnote the footnote-counter
% (and |Hfootnote| counter in case of links) are increased by one.
%
%    \begin{macrocode}
  \addtocounter{footnote}{+1}%
  \ltx@ifpackageloaded{hyperref}{%
    \tfn@hyperfootnotes{\addtocounter{Hfootnote}{+1}}{}%
%    \end{macrocode}
%
% The \xpackage{hyperref} package was not expecting this kind of change of the
% |(H)footnote| numbers, therefore some code-lines are missing here which can be
% found at other places in that same package and are called by this command.
%
%    \begin{macrocode}
    \tfn@updateHyper%
%    \end{macrocode}
%
% When \xpackage{hyperref} is not loaded, nothing is to be done here.
%
%    \begin{macrocode}
   }{}%
%    \end{macrocode}
%
% The |\footnotetext[...]{...}| places the foot note text as usual.
%
%    \begin{macrocode}
  \ifx\tfn@swt\tfn@footnotezero%
    \ifthenelse{\isempty{#1}}%
      {\footnotetext{#2}}%
      {\ltx@ifpackageloaded{hyperref}%
         {\tfn@hyperfootnotes{%
             \footnotetext[#1]{%
               \raisebox{\ht\strutbox}{\hypertarget{tfn:#1}{}}#2}%
           }{\footnotetext[#1]{#2}%
           }%
         }{\footnotetext[#1]{#2}}%
      }%
  \else%
%    \end{macrocode}
%
% If there is more than one tablefootnote in the |sidewaystable|, \\
% the |\tfn@footmiscparawarn| shall be given,
%
%    \begin{macrocode}
    \ifx\tfn@footnotetableprint\tfn@footnotezero%
      \gdef\tfn@fmpw{\tfn@footmiscparawarn}%
    \else%
      \ltx@ifpackageloaded{footmisc}{\tfn@fmpw}{\relax}%
%    \end{macrocode}
%
% but we do not want to repeat the message for a third(, forth,\ldots)
% tablefootnote in the |sidewaystable|.
%
%    \begin{macrocode}
      \gdef\tfn@fmpw{\relax}%
    \fi%
%    \end{macrocode}
%
% |\@makefnmark| is \\
% |\hbox {\@textsuperscript {\normalfont \@thefnmark }}|, \\
% but according to |\show\@thefnmark| here |\@thefnmark| is
% |{\protect \itshape}| instead of |\thefootnote| (Why?!).\\
% If the \xpackage{footnotebackref} package is used,
% back references must be created for the table footnotes, too.
%
%    \begin{macrocode}
    \ltx@ifpackageloaded{footnotebackref}{%
      \LetLtxMacro{\tfncurrentmft}{\@makefntext}%
      \LetLtxMacro{\@makefntext}{\swtablemakefntext}%
      \ifthenelse{\isempty{#1}}%
        {\footnotetext{%
           \raisebox{\ht\strutbox}{\hypertarget{tfn:\theHfootnote}{}}%
           \hbox{\@textsuperscript{\normalfont%
              \hyperlink{tfn:m\theHfootnote}{\thefootnote}}}\thinspace%
         \ifx\FootnoteBackref@symbol\empty%
         \else\hyperlink{tfn:m\theHfootnote}{\FootnoteBackref@symbol}%
         \fi%
         #2}%
       }{\edef\tfn@tfn{\arabic{footnote}}%
         \edef\tfn@thn{\arabic{Hfootnote}}%
         \setcounter{footnote}{#1}%
         \setcounter{Hfootnote}{#1}%
         \tfn@updateHyper%
         \footnotetext{%
           \raisebox{\ht\strutbox}{\hypertarget{tfn:#1}{}}%
           \hbox{\@textsuperscript{\normalfont%
                  \hyperlink{tfn:m#1}{#1}}}%
           \thinspace%
           \ifx\FootnoteBackref@symbol\empty%
           \else%
             \raisebox{\ht\strutbox}{%
               \hyperlink{tfn:m#1}{\FootnoteBackref@symbol}}%
           \fi%
           #2%
          }%
         \setcounter{footnote}{\tfn@tfn}%
         \setcounter{Hfootnote}{\tfn@thn}%
         \tfn@updateHyper%
       }%
      \LetLtxMacro{\@makefntext}{\tfncurrentmft}%
     }{% footnotebackref not loaded
       \ifthenelse{\isempty{#1}}%
         {\footnotetext{%
            \hbox{\@textsuperscript{\normalfont\thefootnote}}%
            \ltx@ifpackageloaded{hyperref}%
              {\tfn@hyperfootnotes{%
                 \raisebox{\ht\strutbox}{%
                   \hypertarget{tfn:\theHfootnote}{}}}{}%
              }{}%
            \thinspace%
            #2}%
        }{%
%    \end{macrocode}
%
% \pagebreak
%
%    \begin{macrocode}
          \edef\tfn@tfn{\arabic{footnote}}%
          \ltx@ifpackageloaded{hyperref}{%
            \tfn@hyperfootnotes{\xdef\tfn@thn{\arabic{Hfootnote}}}{}%
           }{}%
          \setcounter{footnote}{#1}
          \ltx@ifpackageloaded{hyperref}{%
            \tfn@hyperfootnotes{\setcounter{Hfootnote}{#1}}{}%
            \tfn@updateHyper%
           }{}%
          \footnotetext{%
            \hbox{\@textsuperscript{\normalfont #1}}%
            \ltx@ifpackageloaded{hyperref}{%
              \tfn@hyperfootnotes{\raisebox{\ht\strutbox}{%
                                         \hypertarget{tfn:#1}{}}}{}%
             }{}%
            \thinspace%
            #2}%
          \setcounter{footnote}{\tfn@tfn}%
          \ltx@ifpackageloaded{hyperref}{%
            \tfn@hyperfootnotes{\setcounter{Hfootnote}{\tfn@thn}}{}%
            \tfn@updateHyper%
           }{}%
        }%
     }%
  \fi%
%    \end{macrocode}
%
% For further footnotes in the table the \\
% |\ifx\tfn@footnotetableprint\tfn@footnotezero|\ldots|\fi| \\
% from the beginning of this |\tfn@footnotetext| macro must not be repeated,
% therefore we deactivate this by defining |\tfn@footnotetableprint| to
% something other than |0| (here |1| is used).
%
%    \begin{macrocode}
  \gdef\tfn@footnotetableprint{1}%
%    \end{macrocode}
%
% |\tfn@footnotetablecount| contains the negative number of footnote texts
% to be printed, therefore it is increase by one (i.\,e.~the negative number
% decreases, i.\,e.~$-3$ becomes~$-2$).
%
%    \begin{macrocode}
  \@tempcnta=\tfn@footnotetablecount\relax%
  \advance\@tempcnta by +1\relax%
  \xdef\tfn@footnotetablecount{\the\@tempcnta}%
%    \end{macrocode}
%
% When it is zero, all footnote texts have been placed, and\\
% |\tfn@footnotetableprint| can be reset to zero and\\
% |\tfn@tablefootnoteprintout| can be cleared.
%
%    \begin{macrocode}
  \ifx\tfn@footnotetablecount\tfn@footnotezero% all footnotes placed
    \gdef\tfn@footnotetableprint{0}%
    \global\let\tfn@tablefootnoteprintout\relax% brute force
 %\else there is at least one \footnotemark which needs \footnotetext
  \fi
}

%    \end{macrocode}
% \end{macro}
%
% \begin{macro}{\tfn@tablefootnoteprintout}
% When something shall be cleared or redefined, it is a good idea
% to define it. It~will be later filled with the foot note text to be printed.
%
%    \begin{macrocode}
\newcommand{\tfn@tablefootnoteprintout}{\relax}

%    \end{macrocode}
% \end{macro}
%
% \pagebreak
%
% \begin{macro}{\tablefootnote}
% |\tablefootnote[...]{...}| is the command to be used by the user instead of
% |\footnote[...]{...}| in the (sideways)table. It places a foot note mark via
% |\tfn@footnotemarks| and appends a |\tfn@footnotetext{#1}| to the\\
% |\tfn@tablefootnoteprintout| macro, and remembers that table foot notes
% have been used.
%
%    \begin{macrocode}
\newcommand{\tablefootnote}[2][]{%
  \tfn@footnotemarks[#1]%
  \ifthenelse{\isempty{#1}}{\relax}{%
    \ltx@GlobalPrependToMacro{\tfn@tablefootnoteprintout}{%
      \addtocounter{footnote}{+1}%
      \ltx@ifpackageloaded{hyperref}{%
        \tfn@hyperfootnotes{\addtocounter{Hfootnote}{+1}}{}%
        \tfn@updateHyper%
       }{}%
     }%
   }%
  \ltx@GlobalAppendToMacro{\tfn@tablefootnoteprintout}{%
    \tfn@footnotetext[#1]{#2}%
    \ifthenelse{\isempty{#1}}{\relax}{%
      \addtocounter{footnote}{-1}%
      \ltx@ifpackageloaded{hyperref}{%
        \tfn@hyperfootnotes{\addtocounter{Hfootnote}{-1}}{}%
        \tfn@updateHyper%
       }{}%
     }%
   }%
  \gdef\tfn@fnt{1}%
  }

%    \end{macrocode}
% \end{macro}
%
% \begin{macro}{\tfn@swtbox}
% For |sidewaystable|s the |\tfn@tablefootnoteprintout| is placed in a \\
% |\parbox|, which can be placed at the end of the |sidewaystable|:
%
%    \begin{macrocode}
\newcommand{\tfn@swtbox}{%
\ifvoid \footins%
\vskip 2\baselineskip%
\else%
\vskip \skip\footins%
\fi%
\parbox[b][][t]{\linewidth}{\raggedright\tfn@tablefootnoteprintout}%
}

%    \end{macrocode}
% \end{macro}
%
% |\tfn@tablefootnoteprintout| shall be executed after the end of the table environment.
% The \xpackage{etoolbox} package provides very easy syntax for this:
%
%    \begin{macrocode}
\IfFileExists{etoolbox.sty}{%
  \RequirePackage{etoolbox}[2011/01/03]%   v2.1
  \AfterEndEnvironment{table}{%
    \ltx@ifpackageloaded{footnotebackref}{%
      \LetLtxMacro{\tfncurrentmft}{\@makefntext}%
      \LetLtxMacro{\@makefntext}{\tablemakefntext}%
       \tfn@tablefootnoteprintout%
      \LetLtxMacro{\@makefntext}{\tfncurrentmft}%
     }{% footnotebackref not loaded
      \tfn@tablefootnoteprintout%
     }%
      \gdef\tfn@fnt{0}%
   }%
%    \end{macrocode}
%
% When the \xpackage{rotating} package has been loaded, we need to detect
% whether the |\tablefootnote| is placed inside of a |sidewaystable|.
%
%    \begin{macrocode}
  \ltx@ifpackageloaded{rotating}{%
    \AtBeginEnvironment{sidewaystable}{%
      \gdef\tfn@swt{1}%
    }
%    \end{macrocode}
%
% At the end of |sidewaystable| environments |\tfn@swtbox| shall be placed
% (if~there was any |\tablefootnote| in that |sidewaystable| at all).
% And |\tfn@swt| as well as |\tfn@fnt| must be reset to |0|.
%
%    \begin{macrocode}
    \AtEndEnvironment{sidewaystable}{%
      \ifx\tfn@fnt\tfn@footnotezero% \relax
      \else%
        \ltx@ifpackageloaded{footnotebackref}{%
          \LetLtxMacro{\tfncurrentmft}{\@makefntext}%
          \LetLtxMacro{\@makefntext}{\tablemakefntext}%
           \tfn@swtbox%
          \LetLtxMacro{\@makefntext}{\tfncurrentmft}%
         }{% footnotebackref not loaded
          \tfn@swtbox%
          }%
      \fi%
      \gdef\tfn@swt{0}%
      \gdef\tfn@fnt{0}%
     }%
   }{}% else \relax
}{%
%    \end{macrocode}
%
% When the \xpackage{etoolbox} package is not available (e.\,g. because of missing
% \eTeX{}) we~do a workaround:
%
%    \begin{macrocode}
  \PackageWarning{tablefootnote}{%
    Package etoolbox not available, therefore\MessageBreak%
    substituting commands \string\AtEndEnvironment\space %
    and\MessageBreak%
    \string\AfterEndEnvironment\space for tablefootnotes%
    \MessageBreak%
   }%
  \@ifpackagelater{letltxmacro}{2010/09/02}%
   {% >= 2010/09/02, OK
    \GlobalLetLtxMacro{\tfnorigtable}{\table}
    \GlobalLetLtxMacro{\tfnendorigtable}{\endtable}
%    \end{macrocode}
%
% |\GlobalLetLtxMacro| was introduced in version 2010/09/02 v1.4 of the
% \xpackage{letltxmacro} package.
%
%    \begin{macrocode}
   }{% else
     \PackageError{tablefootnote}{%
       Outdated version of letltxmacro package used%
      }{Version 2010/09/02 v1.4 or newer needed but not found!%
        \MessageBreak%
        Update the used letltxmacro package.}
        \LetLtxMacro{\tfnorigtable}{\table}
        \LetLtxMacro{\tfnendorigtable}{\endtable}
    }%
%    \end{macrocode}
%
% \pagebreak
%
%    \begin{macrocode}
  \ltx@ifpackageloaded{float}{%
    \@ifpackagelater{float}{2001/11/08}%   v1.3d
     {% >= 2001/11/08, OK
     }{% else
       \PackageWarningNoLine{tablefootnote}{%
         It is required version\MessageBreak
         2001/11/08 v1.3d (or later) of package float,\MessageBreak
         but only version\MessageBreak
        `\csname ver@float.sty\endcsname'\MessageBreak
        is available}%
     }%
   }{\relax}
  \renewenvironment{table}{%
    \ltx@ifpackageloaded{float}{%
%    \end{macrocode}
%
% When \xpackage{etoolbox.sty} was not found, and the \xpackage{float} package
% has been loaded, and the float specifier is \texttt{H}, then |\endtable| was
% let to |\float@endH| by the \xpackage{float} package (line~79 of
% \xpackage{float.sty}, 2001/11/08 v1.3d:\\
% |\expandafter\let\csname end#1\endcsname\float@endH|), i.\,e.~appending\\
% |\tfn@tablefootnoteprintout \gdef\tfn@fnt{0}| to |\endtable| is useless.
% Instead of this, it must be appended to |\float@endH|.\\
% (Why is this not necessary when \xpackage{etoolbox} is used?
% That is a good question, answers are welcome.)
%
%    \begin{macrocode}
      \ltx@LocalAppendToMacro{\float@endH}{%
        \ltx@ifpackageloaded{footnotebackref}{%
          \LetLtxMacro{\tfncurrentmft}{\@makefntext}%
          \LetLtxMacro{\@makefntext}{\tablemakefntext}%
           \tfn@tablefootnoteprintout%
          \LetLtxMacro{\@makefntext}{\tfncurrentmft}%
         }{% footnotebackref not loaded
          \tfn@tablefootnoteprintout%
         }%
          \gdef\tfn@fnt{0}%
       }%
     }{\relax}
    \tfnorigtable%
   }{%
     \tfnendorigtable%
     \ltx@ifpackageloaded{footnotebackref}{%
       \LetLtxMacro{\tfncurrentmft}{\@makefntext}%
       \LetLtxMacro{\@makefntext}{\tablemakefntext}%
        \tfn@tablefootnoteprintout%
       \LetLtxMacro{\@makefntext}{\tfncurrentmft}%
      }{% footnotebackref not loaded
       \tfn@tablefootnoteprintout%
      }%
     \gdef\tfn@fnt{0}%
    }%
%    \end{macrocode}
%
% \pagebreak
%
% The |sidewaystable|-environment must be treated separately:
%
%    \begin{macrocode}
  \ltx@ifpackageloaded{rotating}{%
    \let\tfnorigsidewaystable\sidewaystable
    \let\tfnendorigsidewaystable\endsidewaystable
    \renewenvironment{sidewaystable}{%
      \gdef\tfn@swt{1}%
      \tfnorigsidewaystable%
    }{%
      \ifx\tfn@fnt\tfn@footnotezero% \relax
      \else%
        \ltx@ifpackageloaded{footnotebackref}{%
          \LetLtxMacro{\tfncurrentmft}{\@makefntext}%
          \LetLtxMacro{\@makefntext}{\tablemakefntext}%
           \tfn@swtbox%
          \LetLtxMacro{\@makefntext}{\tfncurrentmft}%
         }{% footnotebackref not loaded
          \tfn@swtbox%
         }%
      \fi%
      \gdef\tfn@swt{0}%
      \gdef\tfn@fnt{0}%
      \tfnendorigsidewaystable%
     }%
   }{}%
}

%    \end{macrocode}
%
% \DescribeMacro{\AtBeginDocument}
% When the \xpackage{rotating} (for |sidewaystable|s) or the
% \xpackage{footnotebackref} (for back references for the footnotes)
% package is used, it must be loaded before the \xpackage{tablefootnote} package.
% Using both packages is also possible, in which case both must be loaded before
% the \xpackage{tablefootnote} package. If any one of those packages is loaded
% |\AtBeginDocument| after this check, the according error message will not be
% given, but packages really should be loaded before |\AtBeginDocument|.
%
%    \begin{macrocode}
\AtBeginDocument{%
  \ltx@ifpackageloaded{footnotebackref}{%
    \ifx\tfn@fnbr\tfn@footnotezero%
      \PackageError{tablefootnote}{%
        Package footnotebackref loaded after tablefootnote}{%
        When using both the footnotebackref and the tablefootnote %
        package,\MessageBreak%
        the footnotebackref package MUST be loaded before the %
        tablefootnote package!\MessageBreak%
       }%
    \fi%
   }{% else \relax
   }
  \ltx@ifpackageloaded{rotating}{%
    \ifx\tfn@rotating\tfn@footnotezero%
      \PackageError{tablefootnote}{%
        Package rotating loaded after tablefootnote}{%
        When using both the rotating and the tablefootnote package,%
        \MessageBreak%
        the rotating package MUST be loaded before the tablefootnote%
        \space package!\MessageBreak%
       }%
    \fi%
   }{% else \relax
   }
%    \end{macrocode}
%
% \pagebreak
%
% When the document is compiled with \LuaLaTeX{}, hyperlinks in rotated content
% will be misplaced, regardless of the use of the \xpackage{tablefootnote} package
% (or of tables or footnotes). The text to be printed will be OK.
% This problem of \LuaLaTeX{} (not \xpackage{tablefootnote}) was found and reported
% by \textsc{Arno Trautmann}, 2011/11/15 -- thanks!
%
%    \begin{macrocode}
  \ltx@ifpackageloaded{hyperref}{%
    \tfn@hyperfootnotes{
      \ifluatex%
        \PackageWarningNoLine{tablefootnote}{%
          LuaTeX and hyperref package used:\MessageBreak%
          Hyperreferences in rotated content\MessageBreak%
          will be misplaced\MessageBreak%
          regardless of tablefootnote package.\MessageBreak%
          The text to be printed will be OK%
         }%
      \fi%
     }{% else \relax
     }%
   }{% else \relax
   }%
%    \end{macrocode}
%
% When the \xpackage{footnotebackref} package is used but the footnotes are not
% hyperlinked, neither the back references nor the tablefootnotes will work.
%
%    \begin{macrocode}
  \ltx@ifpackageloaded{footnotebackref}{%
     \ltx@ifpackageloaded{hyperref}{%
        \tfn@hyperfootnotes{\relax}{% else
          \PackageError{tablefootnote}{%
             footnotebackref but not hyperfootnotes}{%
             The footnotebackref package is used\MessageBreak%
             but the hyperfootnotes option of the hyperref package%
             \MessageBreak%
             was set to false.\MessageBreak%
             Either enable hyperfootnotes or do not use footnotebackref.%
             \MessageBreak%
             Loading of the tablefootnote package will be aborted.%
           }%
          \endinput%
      }%
    }{\PackageError{tablefootnote}{footnotebackref but not hyperref}{%
        The footnotebackref package is used\MessageBreak%
        but not the hyperref package.\MessageBreak%
        Either load hyperref or do not use footnotebackref.%
        \MessageBreak%
        Loading of the tablefootnote package will be aborted.%
        }%
      \endinput%
    }%
  }{}
}

%    \end{macrocode}
%
%    \begin{macrocode}
%</package>
%    \end{macrocode}
%
% \pagebreak
%
% \section{Installation}
%
% \subsection{Downloads\label{ss:Downloads}}
%
% Everything is available on \CTAN{}, \url{http://www.ctan.org/},
% but may need additional packages themselves.\\
%
% \DescribeMacro{tablefootnote.dtx}
% For unpacking the |tablefootnote.dtx| file and constructing the documentation
% it is required:
% \begin{description}
% \item[-] \TeX{} Format \LaTeXe{}: \url{http://www.CTAN.org/}
%
% \item[-] document class \xpackage{ltxdoc}, 2007/11/11, v2.0u,
%           \url{http://www.ctan.org/pkg/ltxdoc}
%
% \item[-] package \xpackage{holtxdoc}, 2012/03/21, v0.24,
%           \url{http://www.ctan.org/pkg/holtxdoc}
%
% \item[-] package \xpackage{hypdoc}, 2011/08/19, v1.11,
%           \url{http://www.ctan.org/pkg/hypdoc}
% \end{description}
%
% \DescribeMacro{tablefootnote.sty}
% The |tablefootnote.sty| for \LaTeXe{} (i.\,e.~each document using
% the \xpackage{tablefootnote} package) requires:
% \begin{description}
% \item[-] \TeX{} Format \LaTeXe{}, \url{http://www.CTAN.org/}
%
% \item[-] package \xpackage{letltxmacro}, 2010/09/02, v1.4,
%           \url{http://www.ctan.org/pkg/letltxmacro}.
%
% \item[-] package \xpackage{ltxcmds}, 2011/11/09, v1.22,
%           \url{http://www.ctan.org/pkg/ltxcmds}
% \end{description}
%
% When the \xpackage{hyperref} package is used, also the \xpackage{ifluatex} package
% is needed, but it is already loaded automatically by the \xpackage{hyperref}
% package.\\
% When the \xpackage{etoolbox} package is available, it is used:
% \begin{description}
% \item[-] package \xpackage{etoolbox}, 2011/01/03, v2.1,
%           \url{http://www.ctan.org/pkg/etoolbox}.
% \end{description}
%
% \DescribeMacro{tablefootnote-example.tex}
% The |tablefootnote-example.tex| requires the same files as all
% documents using the \xpackage{tablefootnote} package and additionally:
% \begin{description}
% \item[-] class \xpackage{article}, 2007/10/19, v1.4h, from \xpackage{classes.dtx}:\\
%           \CTAN{macros/latex/base/classes.dtx}
%
% \item[-] package \xpackage{rotating}, 2009/03/28, v2.16a,
%           \url{http://www.ctan.org/pkg/rotating}
%
% \item[-] package \xpackage{float}, 2001/11/08, v1.3d,
%           \url{http://www.ctan.org/pkg/float}
%
% \item[-] package \xpackage{placeins}, 2005/04/18, v2.2,
%           \url{http://www.ctan.org/pkg/placeins}\\
%           for |\FloatBarrier|
%
% \item[-] package \xpackage{hyperref}, 2012/11/06, v6.83m,
%           \url{http://www.ctan.org/pkg/hyperref}
%
% \item[-] package \xpackage{tablefootnote}, 2014/01/26, v1.1c,\\
%           \url{http://www.ctan.org/pkg/tablefootnote}\\
%   (Well, it is the example file for this package, and because you
%    are reading the documentation for the \xpackage{tablefootnote} package,
%    it can be assumed that you already have some version of it~--
%    is it the current one?)
% \end{description}
%
% \DescribeMacro{Oberdiek}
% \DescribeMacro{holtxdoc}
% \DescribeMacro{ifluatex}
% \DescribeMacro{letltxmacro}
% \DescribeMacro{ltxcmds}
% All packages of \textsc{Heiko Oberdiek}'s bundle `oberdiek'
% (especially \xpackage{holtxdoc}, \xpackage{ifluatex}, \xpackage{letltxmacro},
% and \xpackage{ltxcmds}) are also available in a TDS compliant ZIP archive:\\
% \CTAN{install/macros/latex/contrib/oberdiek.tds.zip}.\\
% It is probably best to download and use this, because the packages in there
% are quite probably both recent and compatible among themselves.\\
%
% \pagebreak
%
% \DescribeMacro{hyperref}
% \noindent \xpackage{hyperref} is not included in that bundle and needs to be
% downloaded separately,\\
% \url{http://www.ctan.org/install/macros/latex/contrib/hyperref.tds.zip}.\\
%
% \DescribeMacro{M\"{u}nch}
% A hyperlinked list of my (other) packages can be found at
% \url{http://www.ctan.org/author/muench-hm}.\\
%
% \subsection{Package, unpacking TDS}
% \paragraph{Package.} This package is available on \CTAN{}
% \begin{description}
% \item[\CTAN{macros/latex/contrib/tablefootnote/tablefootnote.dtx}]\hspace*{0.1cm} \\
%       The source file.
% \item[\CTAN{macros/latex/contrib/tablefootnote/tablefootnote.pdf}]\hspace*{0.1cm} \\
%       The documentation.
% \item[\CTAN{macros/latex/contrib/tablefootnote/tablefootnote-example.pdf}]\hspace*{0.1cm} \\
%       The compiled example file, as it should look like.
% \item[\CTAN{macros/latex/contrib/tablefootnote/README}]\hspace*{0.1cm} \\
%       The README file.
% \end{description}
% There is also a tablefootnote.tds.zip available:
% \begin{description}
% \item[\CTAN{install/macros/latex/contrib/tablefootnote.tds.zip}]\hspace*{0.1cm} \\
%       Everything in \xfile{TDS} compliant, compiled format.
% \end{description}
% which additionally contains\\
% \begin{tabular}{ll}
% tablefootnote.ins & The installation file.\\
% tablefootnote.drv & The driver to generate the documentation.\\
% tablefootnote.sty & The \xext{sty}le file.\\
% tablefootnote-example.tex & The example file.
% \end{tabular}
%
% \bigskip
%
% \noindent For required other packages, please see the preceding subsection.
%
% \paragraph{Unpacking.} The \xfile{.dtx} file is a self-extracting
% \docstrip{} archive. The files are extracted by running the
% \xext{dtx} through \plainTeX{}:
% \begin{quote}
%   \verb|tex tablefootnote.dtx|
% \end{quote}
%
% \noindent About generating the documentation see paragraph~\ref{GenDoc} below.\\
%
% \paragraph{TDS.} Now the different files must be moved into
% the different directories in your installation TDS tree
% (also known as \xfile{texmf} tree):
% \begin{quote}
% \def\t{^^A
% \begin{tabular}{@{}>{\ttfamily}l@{ $\rightarrow$ }>{\ttfamily}l@{}}
%   tablefootnote.sty & tex/latex/tablefootnote/tablefootnote.sty\\
%   tablefootnote.pdf & doc/latex/tablefootnote/tablefootnote.pdf\\
%   tablefootnote-example.tex & doc/latex/tablefootnote/tablefootnote-example.tex\\
%   tablefootnote-example.pdf & doc/latex/tablefootnote/tablefootnote-example.pdf\\
%   tablefootnote.dtx & source/latex/tablefootnote/tablefootnote.dtx\\
% \end{tabular}^^A
% }^^A
% \sbox0{\t}^^A
% \ifdim\wd0>\linewidth
%   \begingroup
%     \advance\linewidth by\leftmargin
%     \advance\linewidth by\rightmargin
%   \edef\x{\endgroup
%     \def\noexpand\lw{\the\linewidth}^^A
%   }\x
%   \def\lwbox{^^A
%     \leavevmode
%     \hbox to \linewidth{^^A
%       \kern-\leftmargin\relax
%       \hss
%       \usebox0
%       \hss
%       \kern-\rightmargin\relax
%     }^^A
%   }^^A
%   \ifdim\wd0>\lw
%     \sbox0{\small\t}^^A
%     \ifdim\wd0>\linewidth
%       \ifdim\wd0>\lw
%         \sbox0{\footnotesize\t}^^A
%         \ifdim\wd0>\linewidth
%           \ifdim\wd0>\lw
%             \sbox0{\scriptsize\t}^^A
%             \ifdim\wd0>\linewidth
%               \ifdim\wd0>\lw
%                 \sbox0{\tiny\t}^^A
%                 \ifdim\wd0>\linewidth
%                   \lwbox
%                 \else
%                   \usebox0
%                 \fi
%               \else
%                 \lwbox
%               \fi
%             \else
%               \usebox0
%             \fi
%           \else
%             \lwbox
%           \fi
%         \else
%           \usebox0
%         \fi
%       \else
%         \lwbox
%       \fi
%     \else
%       \usebox0
%     \fi
%   \else
%     \lwbox
%   \fi
% \else
%   \usebox0
% \fi
% \end{quote}
% If you have a \xfile{docstrip.cfg} that configures and enables \docstrip{}'s
% \xfile{TDS} installing feature, then some files can already be in the right
% place, see the documentation of \docstrip{}.
%
% \subsection{Refresh file name databases}
%
% If your \TeX{}~distribution (\TeX live, \mikTeX{}, \dots{}) relies on file name
% databases, you must refresh these. For example, \teTeX{} users run
% \verb|texhash| or \verb|mktexlsr|.
%
% \subsection{Some details for the interested}
%
% \paragraph{Unpacking with \LaTeX{}.}
% The \xfile{.dtx} chooses its action depending on the format:
% \begin{description}
% \item[\plainTeX:] Run \docstrip{} and extract the files.
% \item[\LaTeX:] Generate the documentation.
% \end{description}
% If you insist on using \LaTeX{} for \docstrip{} (really,
% \docstrip{} does not need \LaTeX{}), then inform the autodetect routine
% about your intention:
% \begin{quote}
%   \verb|latex \let\install=y\input{tablefootnote.dtx}|
% \end{quote}
% Do not forget to quote the argument according to the demands
% of your shell.
%
% \paragraph{Generating the documentation.\label{GenDoc}}
% You can use both the \xfile{.dtx} or the \xfile{.drv} to generate
% the documentation. The process can be configured by a
% configuration file \xfile{ltxdoc.cfg}. For instance, put the following
% line into this file, if you want to have A4 as paper format:
% \begin{quote}
%   \verb|\PassOptionsToClass{a4paper}{article}|
% \end{quote}
%
% \noindent An example follows how to generate the
% documentation with \pdfLaTeX{}:
%
% \begin{quote}
%\begin{verbatim}
%pdflatex tablefootnote.dtx
%makeindex -s gind.ist tablefootnote.idx
%pdflatex tablefootnote.dtx
%makeindex -s gind.ist tablefootnote.idx
%pdflatex tablefootnote.dtx
%\end{verbatim}
% \end{quote}
%
% \subsection{Compiling the example}
%
% The example file, \textsf{tablefootnote-example.tex}, can be compiled via\\
% \indent |latex tablefootnote-example.tex|,\\
% \indent |lualatex tablefootnote-example.tex|,\\
% or (recommended)\\
% \indent |pdflatex thumbs-example.tex|\\
% and will need at least two compiler runs to get everything right.
%
% \bigskip
%
% \section{Acknowledgements}
%
% I would like to thank \textsc{Heiko Oberdiek} for providing
% the \xpackage{hyperref}, \xpackage{ifluatex}, \xpackage{letltxmacro},
% \xpackage{ltxcmds}, as well as a~lot~(!) of other useful packages
% (from which I also got everything I know about creating a file in
% \xext{dtx} format, OK, say it: copying),
% and for information about the \pdfTeX{}-bug,
% \textsc{Arno Trautmann} for reporting the bug of misplaced hyperreferences in
% |sidewaystable|s using \LuaTeX{}, independently of the \xpackage{tablefootnote}
% package or footnotes at all,
% \textsc{Thomas~V.} for reporting the bug of not handling the case of
% \xpackage{hyperref} option |hyperfootnotes=false|,
% \textsc{Sverre Stausland Johnsen} for reporting the bug in case of \emph{not}
% using \xpackage{hyperref},
% and the \Newsgroup{comp.text.tex} and \Newsgroup{de.comp.text.tex} newsgroups
% as well as everybody at \url{http://tex.stackexchange.com}
% for their help in all things \TeX{}.
%
% \pagebreak
%
% \phantomsection
% \begin{History}\label{History}
%   \begin{Version}{2011/10/26 v1.0a, tabfootn}
%     \item Upload to \CTAN{macros/latex/contrib/tablefootnote/}.
%   \end{Version}
%   \begin{Version}{2011/10/29 v1.0b}
%     \item Renamed to \xpackage{tablefootnote}.
%     \item Added support for the |sidewaystable|-environment of the
%             \xpackage{rotating} package.
%     \item Diverse changes in documentation and README.
%   \end{Version}
%   \begin{Version}{2011/11/06 v1.0c}
%     \item Replaced |^{\text{\thefootnote}}| by |{}^{\text{\textup{\thefootnote}}}|.
%     \item Bug fix: When \xpackage{etoolbox.sty} was not found, and the \xpackage{float}
%             package had been loaded, and the float specifier was \texttt{H},
%             then the |\tablefootnote|s were not printed for that table. Fixed.
%   \end{Version}
%   \begin{Version}{2011/11/19 v1.0d}
%     \item Replaced |{}^{\text{\textup{\thefootnote}}}| by
%             |\textsuperscript{\thefootnote}|, therefore the \xpackage{amstext}
%             package is no longer required by the \xpackage{tablefootnote} package.
%     \item \xpackage{lscape} and \xpackage{pdflscape} package now work with
%             \xpackage{tablefootnote} (except for using a \texttt{sidewaystable} on
%             a \texttt{landscape} page).
%     \item Added information about footnotes in \texttt{sidewaystable} being printed
%             one below the other, even if \xpackage{footmisc} package with option
%             \texttt{para} is used.
%     \item Redefined |\tfn@swtbox| from
%  \begin{verbatim}
%  \vspace{0.8cm}%
%  \begin{spacing}{0.1}%
%  \subcaptionbox*{}[\linewidth][l]{\tfn@tablefootnoteprintout}%
%  \end{spacing}%
%  \end{verbatim}
%             to
%  \begin{verbatim}
%  \vspace{2\baselineskip}%
%  \parbox[b][][t]{\linewidth}{\raggedright\tfn@tablefootnoteprintout}%
%  \end{verbatim}
%             , therefore the \xpackage{setspace} and \xpackage{subcaption} packages
%             are no longer required by the \xpackage{tablefootnote} package.
%     \item No longer uses |\numexpr|, therefore also usable without \eTeX{}.
%     \item Replaced |\textsuperscript{\thefootnote}| by
%             |\hbox {\@textsuperscript {\normalfont \thefootnote }}|,
%             which is next to the original definition of |\@makefnmark|
%             (|\hbox {\@textsuperscript {\normalfont \@thefnmark }}|,
%             but according to |\show\@thefnmark| here |\@thefnmark| is
%             |{\protect \itshape}| instead of |\thefootnote|).
%     \item Added an error message, when \LuaTeX{}, \xpackage{rotating} package, and
%            \xpackage{hyperref} package are used together. Hyperreferences in
%            |sidewaystable|s are misplaced, independently of the
%            \xpackage{tablefootnote} package or footnotes at all. (Bug reported
%            by \textsc{Arno Trautmann}, 2011/11/15 -- thanks!)
%     \item Added footnotesymbol-footnote-numbers in the example.
%   \end{Version}
%   \begin{Version}{2011/11/26 v1.0e}
%     \item Check for \xpackage{hyperref} is only done once.
%     \item Message regarding the misplacement of hyperlinks by \LuaLaTeX{}
%             changed form error to warning.
%     \item Replaced |\vspace{2\baselineskip}| by
%  \begin{verbatim}
% \ifvoid \footins%
% \vskip 2\baselineskip%
% \else%
% \vskip \skip\footins%
% \fi%
%  \end{verbatim}
%             between the |sidewaystable|s and their footnotes.
%     \item The warning about un-regarded |para|-option of \xpackage{footmoisc}
%             package is only given at the second |tablefootnote| in a |sidewaystable|,
%             not repeatedly for every |tablefootnote| in a |sidewaystable|.
%   \end{Version}
%   \begin{Version}{2012/01/01 v1.0f}
%     \item Bugfix: Wrong installation path given in the documentation, fixed.
%     \item Update of documentation, README, and \xfile{dtx} internals.
%   \end{Version}
%   \begin{Version}{2012/01/14 v1.0g}
%     \item Bugfix: There was a |relax| instead of a |\relax|, fixed.
%   \end{Version}
%   \begin{Version}{2012/07/29 v1.0h}
%     \item \xpackage{tablefootnote} is now compatible to the new
%             \xpackage{footnotebackref} package, 2012/07/01, v1.0,
%             \url{http://www.ctan.org/pkg/footnotebackref}.
%   \end{Version}
%   \begin{Version}{2014/01/08 v1.1a}
%     \item Tablefootnotes now regard optional footnote marks:
%             |\tablefootnote[| custom mark here! |]{...}|
%     \item Bug fix: the case of \xpackage{hyperref} option |hyperfootnotes=false|
%             was not handeled at all. (Bug reported by \textsc{Thomas~V.}, %
%             2014/01/02 -- thanks!)
%     \item A lot of details.
%   \end{Version}
%   \begin{Version}{2014/01/20 v1.1b}
%     \item Bug fix: Package was broken when \xpackage{hyperref} was \emph{not} used.
%            (Bug reported by \textsc{Sverre Stausland Johnsen} -- thanks!)
%   \end{Version}
%   \begin{Version}{2014/01/26 v1.1c}
%     \item The emergency bug fix of version 2014/01/20 v1.1b was replaced by using
%             the new robust macro |\tfn@hyperfootnotes|, which only \emph{internally}
%             uses |\ifHy@hyperfootnotes|. If the \xpackage{hyperref} package is not used,
%             \LaTeX{} now does not see any confusing
%             |\ifHy@hyperfootnotes|\ldots|\else|\ldots|\fi| at all.
%     \item URLs and README updated.
%   \end{Version}
% \end{History}
%
% \bigskip
%
% When you find a mistake or have a suggestion for an improvement of this package,
% please send an e-mail to the maintainer, thanks! (Please see BUG REPORTS in the README.)
%
% \bigskip
%
% \PrintIndex
%
% \Finale
\endinput
%        (quote the arguments according to the demands of your shell)
%
% Documentation:
%    (a) If tablefootnote.drv is present:
%           (pdf)latex tablefootnote.drv
%           makeindex -s gind.ist tablefootnote.idx
%           (pdf)latex tablefootnote.drv
%           makeindex -s gind.ist tablefootnote.idx
%           (pdf)latex tablefootnote.drv
%    (b) Without tablefootnote.drv:
%           (pdf)latex tablefootnote.dtx
%           makeindex -s gind.ist tablefootnote.idx
%           (pdf)latex tablefootnote.dtx
%           makeindex -s gind.ist tablefootnote.idx
%           (pdf)latex tablefootnote.dtx
%
%    The class ltxdoc loads the configuration file ltxdoc.cfg
%    if available. Here you can specify further options, e.g.
%    use DIN A4 as paper format:
%       \PassOptionsToClass{a4paper}{article}
%
% Installation:
%    TDS:tex/latex/tablefootnote/tablefootnote.sty
%    TDS:doc/latex/tablefootnote/tablefootnote.pdf
%    TDS:doc/latex/tablefootnote/tablefootnote-example.tex
%    TDS:doc/latex/tablefootnote/tablefootnote-example.pdf
%    TDS:source/latex/tablefootnote/tablefootnote.dtx
%
%<*ignore>
\begingroup
  \catcode123=1 %
  \catcode125=2 %
  \def\x{LaTeX2e}%
\expandafter\endgroup
\ifcase 0\ifx\install y1\fi\expandafter
         \ifx\csname processbatchFile\endcsname\relax\else1\fi
         \ifx\fmtname\x\else 1\fi\relax
\else\csname fi\endcsname
%</ignore>
%<*install>
\input docstrip.tex
\Msg{******************************************************************}
\Msg{* Installation                                                   *}
\Msg{* Package: tablefootnote 2014/01/26 v1.1c Table foot notes (HMM) *}
\Msg{******************************************************************}

\keepsilent
\askforoverwritefalse

\let\MetaPrefix\relax
\preamble

This is a generated file.

Project: tablefootnote
Version: 2014/01/26 v1.1c

Copyright (C) 2011 - 2014 by
    H.-Martin M"unch <Martin dot Muench at Uni-Bonn dot de>

The usual disclaimer applies:
If it doesn't work right that's your problem.
(Nevertheless, send an e-mail to the maintainer
 when you find an error in this package.)

This work may be distributed and/or modified under the
conditions of the LaTeX Project Public License, either
version 1.3c of this license or (at your option) any later
version. This version of this license is in
   http://www.latex-project.org/lppl/lppl-1-3c.txt
and the latest version of this license is in
   http://www.latex-project.org/lppl.txt
and version 1.3c or later is part of all distributions of
LaTeX version 2005/12/01 or later.

This work has the LPPL maintenance status "maintained".

The Current Maintainer of this work is H.-Martin Muench.

This work consists of the main source file tablefootnote.dtx,
the README, and the derived files
   tablefootnote.sty, tablefootnote.pdf,
   tablefootnote.ins, tablefootnote.drv,
   tablefootnote-example.tex, tablefootnote-example.pdf.

In memoriam Tommy Muench + 2014/01/02.

\endpreamble
\let\MetaPrefix\DoubleperCent

\generate{%
  \file{tablefootnote.ins}{\from{tablefootnote.dtx}{install}}%
  \file{tablefootnote.drv}{\from{tablefootnote.dtx}{driver}}%
  \usedir{tex/latex/tablefootnote}%
  \file{tablefootnote.sty}{\from{tablefootnote.dtx}{package}}%
  \usedir{doc/latex/tablefootnote}%
  \file{tablefootnote-example.tex}{\from{tablefootnote.dtx}{example}}%
}

\catcode32=13\relax% active space
\let =\space%
\Msg{************************************************************************}
\Msg{*}
\Msg{* To finish the installation you have to move the following}
\Msg{* file into a directory searched by TeX:}
\Msg{*}
\Msg{*     tablefootnote.sty}
\Msg{*}
\Msg{* To produce the documentation run the file `tablefootnote.drv'}
\Msg{* through (pdf)LaTeX, e.g.}
\Msg{*  pdflatex tablefootnote.drv}
\Msg{*  makeindex -s gind.ist tablefootnote.idx}
\Msg{*  pdflatex tablefootnote.drv}
\Msg{*  makeindex -s gind.ist tablefootnote.idx}
\Msg{*  pdflatex tablefootnote.drv}
\Msg{*}
\Msg{* At least three runs are necessary e.g. to get the}
\Msg{*  references right!}
\Msg{*}
\Msg{* Happy TeXing!}
\Msg{*}
\Msg{************************************************************************}

\endbatchfile
%</install>
%<*ignore>
\fi
%</ignore>
%
% \section{The documentation driver file}
%
% The next bit of code contains the documentation driver file for
% \TeX{}, i.\,e., the file that will produce the documentation you
% are currently reading. It will be extracted from this file by the
% \texttt{docstrip} programme. That is, run \LaTeX{} on \texttt{docstrip}
% and specify the \texttt{driver} option when \texttt{docstrip}
% asks for options.
%
%    \begin{macrocode}
%<*driver>
\NeedsTeXFormat{LaTeX2e}[2011/06/27]
\ProvidesFile{tablefootnote.drv}%
  [2014/01/26 v1.1c Table foot notes (HMM)]
\documentclass{ltxdoc}[2007/11/11]% v2.0u
\usepackage{holtxdoc}[2012/03/21]%  v0.24
%% tablefootnote may work with earlier versions of LaTeX2e and those
%% class and package, but this was not tested.
%% Please consider updating your LaTeX, class, and package
%% to the most recent version (if they are not already the most
%% recent version).
\hypersetup{%
 pdfsubject={Footnotes in Tables (HMM)},%
 pdfkeywords={LaTeX, tablefootnote, footnote, table, H.-Martin Muench},%
 pdfencoding=auto,%
 pdflang={en},%
 breaklinks=true,%
 linktoc=all,%
 pdfstartview=FitH,%
 pdfpagelayout=OneColumn,%
 bookmarksnumbered=true,%
 bookmarksopen=true,%
 bookmarksopenlevel=3,%
 pdfmenubar=true,%
 pdftoolbar=true,%
 pdfwindowui=true,%
 pdfnewwindow=true%
}
\CodelineIndex
\hyphenation{printing docu-ment}
\gdef\unit#1{\mathord{\thinspace\mathrm{#1}}}%
\makeatletter
\@ifundefined{eTeX}{\gdef\eTeX{$\m@th \varepsilon $-\TeX }}{% else \relax
}
\makeatother
\begin{document}
  \DocInput{tablefootnote.dtx}%
\end{document}
%</driver>
%    \end{macrocode}
%
% \fi
%
% \CheckSum{584}
%
% \CharacterTable
%  {Upper-case    \A\B\C\D\E\F\G\H\I\J\K\L\M\N\O\P\Q\R\S\T\U\V\W\X\Y\Z
%   Lower-case    \a\b\c\d\e\f\g\h\i\j\k\l\m\n\o\p\q\r\s\t\u\v\w\x\y\z
%   Digits        \0\1\2\3\4\5\6\7\8\9
%   Exclamation   \!     Double quote  \"     Hash (number) \#
%   Dollar        \$     Percent       \%     Ampersand     \&
%   Acute accent  \'     Left paren    \(     Right paren   \)
%   Asterisk      \*     Plus          \+     Comma         \,
%   Minus         \-     Point         \.     Solidus       \/
%   Colon         \:     Semicolon     \;     Less than     \<
%   Equals        \=     Greater than  \>     Question mark \?
%   Commercial at \@     Left bracket  \[     Backslash     \\
%   Right bracket \]     Circumflex    \^     Underscore    \_
%   Grave accent  \`     Left brace    \{     Vertical bar  \|
%   Right brace   \}     Tilde         \~}
%
% \GetFileInfo{tablefootnote.drv}
%
% \begingroup
%   \def\x{\#,\$,\^,\_,\~,\ ,\&,\{,\},\%}%
%   \makeatletter
%   \@onelevel@sanitize\x
% \expandafter\endgroup
% \expandafter\DoNotIndex\expandafter{\x}
% \expandafter\DoNotIndex\expandafter{\string\ }
% \begingroup
%   \makeatletter
%     \lccode`9=32\relax
%     \lowercase{%^^A
%       \edef\x{\noexpand\DoNotIndex{\@backslashchar9}}%^^A
%     }%^^A
%   \expandafter\endgroup\x
%
% \DoNotIndex{\\,\,}
% \DoNotIndex{\documentclass,\usepackage,\ProvidesPackage,\begin,\end}
% \DoNotIndex{\MessageBreak,\message}
% \DoNotIndex{\NeedsTeXFormat,\DoNotIndex,\verb}
% \DoNotIndex{\def,\edef,\gdef,\global}
% \DoNotIndex{\ifx,\listfiles,\mathord,\mathrm}
% \DoNotIndex{\kvoptions,\SetupKeyvalOptions,\ProcessKeyvalOptions}
% \DoNotIndex{\bigskip,\space,\thinspace,\ldots}
% \DoNotIndex{\indent,\noindent,\newline,\linebreak,\pagebreak}
% \DoNotIndex{\textbf,\textit,\textsf,\texttt,\textsc,\textquotedblleft,\textquotedblright}
% \DoNotIndex{\plainTeX,\TeX,\LaTeX,\pdfLaTeX}
% \DoNotIndex{\chapter,\section,\the,\csname,\endscname}
% \DoNotIndex{\value,\arabic,\hline,\text,\caption,\centering,\clear,\url}
%
% \title{The \xpackage{tablefootnote} package}
% \date{2014/01/26 v1.1c}
% \author{H.-Martin M\"{u}nch\\\xemail{Martin.Muench at Uni-Bonn.de}}
%
% \maketitle
%
% \begin{abstract}
% This \LaTeX{} package provides the command |\tablefootnote| to be used
% in a \texttt{table} or \texttt{sidewaystable} environment,
% where |\footnote| will not work and when using |\footnotemark| and
% |\footnotetext| and adjusting the counters (including \texttt{Hfootnote})
% manually is either too much work or would not even work (\texttt{sidewaystable}).
% \end{abstract}
%
% \bigskip
%
% \noindent Disclaimer for web links: The author is not responsible for any contents
% referred to in this work unless he has full knowledge of illegal contents.
% If any damage occurs by the use of information presented there, only the
% author of the respective pages might be liable, not the one who has referred
% to these pages.
%
% \bigskip
%
% \noindent {\color{green} Save per page about $200\unit{ml}$ water,
% $2\unit{g}$ CO$_{2}$ and $2\unit{g}$ wood:\\
% Therefore please print only if this is really necessary.}
%
% \pagebreak
%
% \tableofcontents
%
% \pagebreak
%
% \section{Introduction\label{sec:Introduction}}
%
% \indent This \LaTeX{} package provides the command |\tablefootnote{...}| to be
% used in a \texttt{table} or \texttt{sidewaystable} environment. In those
% environments |\footnote{...}| would not work and would need to be replaced
% by |\footnotemark| in the \texttt{(sideways)\allowbreak table} and
% |\footnotetext{...}| after the \texttt{(sideways)table} environment ended.
% Additionally the counters for \texttt{footnote} and
% (when the \xpackage{hyperref} package is used) \texttt{Hfootnote}
% would need to be adjusted manually. Then still the hyperlinks
% in a \texttt{(sideways)table} would not work.
% When the \xpackage{footnotebackref} package is used,
% |\footnotemark| and |\footnotetext{...}| would need to be redefined.
% Just using |\tablefootnote{...}| (with optional argument for custom foot note
% marks) in the (sideways)table does all this automatically.
% (Compatibility with the \xpackage{footnotebackref} package
% was achieved without redefining |\footnotemark| or |\footnotetext|.)
% Redefining |\footnote| was \emph{not} done as to not create conflicts with
% other packages.
%
% \section{Usage}
%
% \indent Just load the package placing
% \begin{quote}
%   |\usepackage{tablefootnote}|
% \end{quote}
% \noindent in the preamble of your \LaTeXe{} source file and use\newline
% |\tablefootnote{...}| instead of |\footnote{...}| and\newline
% |\tablefootnote[...]{...}| instead of |\footnote[...]{...}|
% in (sideways)\allowbreak tables.\\
% When the \xpackage{footnotebackref} package is used,
% it must be loaded before the \xpackage{tablefootnote} package
% and the \xpackage{hyperrefpackage} with option |hyperfootnotes=true|
% must be loaded, too. (Backreference{-}links without links do not make sense.)\\
% When the \xpackage{rotating} package (for |sidewaystable|s) is used,
% it must be loaded before the \xpackage{tablefootnote} package and it must be
% a recent version (v2.16a, 2009/03/28, or newer).\\
% If |\tablefootnote{...}| is used in a tabular environment, which is inside
% of a (sideways)table environment, everything is fine, but when the tabular
% environment is not inside of a (sideways)table, the \textbf{table}footnote
% will not work.\\
%
% Pages with tables with footnotes, which are rotated with the \xpackage{lscape}-
% or with the \xpackage{pdflscape}-package (|\begin{landscape}|, table, |\end{landscape}|),
% are possible. But when the \xpackage{footmisc}-package is used, \xpackage{(pdf)lscape}
% should be loaded before \xpackage{footmisc}.\\
%
% When the \xpackage{footmisc}-package is used with option \texttt{para},
% this is regarded with the exception of a \texttt{sidewaystable}. There the footnotes
% are printed one below the other. (Well, this is better than no footnote in a
% \texttt{sidewaystable} at all, isn't it?)\\
%
% When the (sideways)table floats, the footnotes are set and (when \xpackage{hyperref}
% is used) hyperlinked, but they are not automatically adapted when the table floats
% over/under another footnote. Thus either do not use a footnote between original and
% float position of the (sideways)table, or place the (sideways)table in
% \textquotedblleft here\textquotedblright\ position. |\clear(double)page|, |h(!)|,
% |H|~from the \xpackage{float} package (\url{http://www.ctan.org/pkg/float}),
% or |\FloatBarrier| from the \xpackage{picins} package
% (\url{http://www.ctan.org/pkg/picins}) might help, too.
% (Or after finishing the document, move the (sideways)table in the source code
% near the position where it floats to or use the optional footnote marks.)
%
% \pagebreak
%
% \section{Alternatives\label{sec:Alternatives}}
%
% \begin{description}
% \item[-] The \xpackage{longtable} package provides the \texttt{longtable} environment
%   as replacement for the combined \texttt{table} and \texttt{tabular} environments.
%   Footnotes are real footnotes (not just tablenotes), are continuously numbered and
%   hyperlinked (when using the \xpackage{hyperref} package), and the hyperlinks really
%   work. As drawback the appearance of the caption changes slightly (e.\,g.~distance
%   to the table, width of the caption), which can probably be changed back manually.
%   Furthermore, longtables are meaned to break over more than one page. If that is not
%   wished, it must be prevented by |\nopagebreak|-commands and by ending the longtable
%   lines with |\\*| instead of |\\|. longtables do not float. (Therefore using the
%   \xpackage{tablefootnote} package and |\FloatBarrier| from the \xpackage{picins} package
%   before and after the table environment is similar~- but \xpackage{tablefootnote} does not
%   change the table-caption!) \texttt{sidewaystable} does not work with it.\\
%   \url{http://www.ctan.org/pkg/longtable}
%
% \item[-] The \xpackage{supertabular} package provides the \texttt{mpsupertabular}
%   environment as replacement for the combined \texttt{table} and \texttt{tabular}
%   environments. Footnotes are just tablenotes (with working hyperlinks when using the
%   \xpackage{hyperref} package), i.\,e. numbered a, b, c and placed below the table
%   and not at the end of the page. Therefore there is no float problem (because the
%   tablenotes numbering is not included in the continuous numbering of the footnotes).
%   Placing the \texttt{supertabular} inside of a \texttt{sidewaystable} breaks the
%   hyperlinks to the tablenotes.\\
%   \url{http://www.ctan.org/pkg/supertabular}
%
% \item[-] The \xpackage{ctable} package has its very own notation for defining tables.
%   It can create tablenotes and sideways-tables. The tablenotes are not automatically
%   hyperlinked. The \texttt{ctable}s float. Because the tablenotes numbering is not
%   included in the continuous numbering of the footnotes there is no float problem.\\
%   \url{http://www.ctan.org/pkg/ctable}
%
% \item[-] The \xpackage{footnote} package provides |\makesavenoteenv{table}|. After
%   loading the package and using that command in the preamble, in tables |\footnote{...}|
%   can be used. Using |\makesavenoteenv{tabular}| and \\
%   |\makesavenoteenv{sidewaystable}|
%   is possible, but it neither solves the float problem, nor do the created hyperlinks
%   work (i.\,e.~they aim at wrong locations). The \xpackage{mdwtab} from the same bundle
%   is incompatible with other table-related packages (e.\,g. \xpackage{supertabular},
%   \xpackage{array}) and not $100\,\%$ compatible with the \texttt{tabular} environment.\\
%   \url{http://www.ctan.org/pkg/footnote}\\
%   \url{http://www.ctan.org/pkg/mdwtab}
%
% \item[-] The \xpackage{tabularx} package does produce footnotes for sidewaystables,
%   but uses a, b, c instead of 1, 2, 3. The hyperlinks to the footnotes do not work.
%   Because the footnotes numbering is not included in the continuous numbering of
%   the other footnotes there is no float problem.\\
%   \url{http://www.ctan.org/pkg/tabularx}
%
% \item[-] Placing a tabular inside a minipage inside a table produces tablenotes.
%   Therefore there is no float problem (because the footnotes are not continuously
%   numbered). The hyperlinks to the table notes indeed work.
%
% \pagebreak
%
% \item[-] The \xpackage{threeparttable} package creates tablenotes again. Therefore
%   there is no float problem (because the tablenotes are not continuously numbered with
%   the footnotes). There are no hyperlinks to the table notes (at least not automatically).
%   Using |sidewaystable| (with table notes) works.\\
%   \url{http://www.ctan.org/pkg/threeparttable}
%
% \item[-] The \xpackage{threeparttablex} package creates tablenotes again. Therefore there
%   is no float problem (because the tablenotes are not continuously numbered with the
%   footnotes). With option \texttt{referable} the tablenotes are hyperlinked. Use of
%   a |sidewaystable| (with table notes) did not work for me. When using the
%   \texttt{referable} option according to the example in the \xpackage{threeparttablex}
%   manual the \xpackage{longtable} package is used, therefore that package could be used
%   directly without \xpackage{threeparttablex} (see above).\\
%   \url{http://www.ctan.org/pkg/threeparttablex}
%
% \item[-] One can manually use |\footnotemark| in the table and |\footnotetext{...}|
%  after the table environment ended and manually change the |footnote| and
%  (when the \xpackage{hyperref} package is used) |Hfootnote| counters and needs
%  to insert
%  \begin{verbatim}
%     \makeatletter
%     \global\let\Hy@saved@currentHref\@currentHref
%     \hyper@makecurrent{Hfootnote}%
%     \global\let\Hy@footnote@currentHref\@currentHref
%     \global\let\@currentHref\Hy@saved@currentHref
%     \makeatother
%  \end{verbatim}
%  before each |\footnotetext{...}|, but with more than very few footnotes this can become
%  much work (and is prone to produce errors). And this does not even work with
%  sidewaystables. (And it does not work with the \xpackage{footnotebackref} package
%  without redefining |\footnotemark| and |\footnotetext{...}|.)
% \end{description}
%
% \bigskip
%
% \noindent (You programmed or found another alternative,
%  which is available at \CTAN{}?\\
%  OK, send an e-mail to me with the name, location at \CTAN{},
%  and a short notice, and I will probably include it in
%  the list above.)
%
% \newpage
%
% \section{Example}
%
%    \begin{macrocode}
%<*example>
\documentclass[british]{article}[2007/10/19]% v1.4h
%%%%%%%%%%%%%%%%%%%%%%%%%%%%%%%%%%%%%%%%%%%%%%%%%%%%%%%%%%%%%%%%%%%%%
\usepackage{float}[2001/11/08]%    v1.3d
\usepackage{placeins}[2005/04/18]% v2.2  ; for \FloatBarrier
\usepackage{rotating}[2009/03/28]% v2.16a; for sidewaystable-environment
\usepackage[%
 hyperfootnotes=true,%
 extension=pdf,%
 plainpages=false,%
 pdfpagelabels=true,%
 hyperindex=false,%
 pdflang={en},%
 pdftitle={tablefootnote package example},%
 pdfauthor={H.-Martin Muench},%
 pdfsubject={Example for the tablefootnote package},%
 pdfkeywords={LaTeX, tablefootnote, footnote, table, H.-Martin Muench},%
 % pdfview=FitH and FitBH do not work: hyperlinks in sidewaystables
 % do no lead to the footnotes, due to a bug in pdfTeX,
 % computing wrong anchor coordinates (Heiko Oberdiek, 29. October 2011)
 % pdfview=Fit, FitV, FitR, FitB, FitBV work
 % print is OK for all those options
 pdfstartview=FitH,%
 pdfpagelayout=OneColumn%
]{hyperref}[2012/11/06]% v6.83m
 % Due to the urls used in the example, either the hyperref or the url
 % package are needed (or the urls must be removed before compiling).

\usepackage{footnotebackref}[2012/07/01]% v1.0
\usepackage{tablefootnote}[2014/01/26]%   v1.1c

\gdef\unit#1{\mathord{\thinspace\mathrm{#1}}}%
\listfiles
\begin{document}
\pagenumbering{arabic}
\section*{Example for tablefootnote}

This example demonstrates the use of package\newline
\textsf{tablefootnote}, v1.1c as of 2014/01/26 (HMM).\newline
There were no options used. (The package provides no options.)\newline

\textbf{The \texttt{tablefootnote-example.tex} needs to be compiled
at least twice to get the references right!}\newline

If the etoolbox-package is found, it is automatically used.\newline

For more details please see the documentation!\newline

\noindent Save per page about $200\unit{ml}$ water,
$2\unit{g}$ CO$_{2}$ and $2\unit{g}$ wood:\newline
Therefore please print only if this is really necessary.\newline

Here is some text.\footnote{And this is a text footnote.}\newline

Tables \ref{tab.symbol}, \ref{tab.normal}, \ref{tab.another} and
\ref{tab.floatH} show normal tables, table~\ref{tab.sideways} depicts
a sidewaystable. Table~\ref{tab.floatH} uses the float
specifier~\texttt{H} from the float package.\newline

\texttt{Hyperref} option \verb|pdfview=FitH| and \verb|FitBH| do
not work due to a bug in pdf\TeX{}, computing wrong
anchor coordinates (\textsc{Heiko Oberdiek}, 29.~October 2011).
Depending on used pdf-viewer, hyperlinks in sidewaystables lead
e.\,g.~at the end of the document, not at the footnote.
\verb|pdfview=Fit|, \verb|FitV|, \verb|FitR|, \verb|FitB|,
\verb|FitBV| work, print is OK for all those options.

\bigskip

\listoftables

\pagebreak

\renewcommand{\thefootnote}{\fnsymbol{footnote}}
\verb|\renewcommand{\thefootnote}{\fnsymbol{footnote}}|
causes footnote{-}symbol{-}footnotes, which are possible
(see Table~\ref{tab.symbol}).

\begin{table}
\centering
\begin{tabular}{ccc}
Another\tablefootnote{A table footnote.} & %
text\tablefootnote{Another table footnote.} & %
in a table\tablefootnote{A third table footnote.}
\end{tabular}
\caption[A footnotesymbol table]{%
A table with footnote-symbol-footnotes.\label{tab.symbol}}
\end{table}

Some text.\footnote{A text footnote.}

\renewcommand{\thefootnote}{\arabic{footnote}}
\verb|\renewcommand{\thefootnote}{\arabic{footnote}}|
switches back to normal footnote numbers again.

\pagebreak

\begin{table}
\centering
\begin{tabular}{ccc}
Some\tablefootnote{A table footnote.} & %
text\tablefootnote[99]{A table footnote with custom footnote number.} & %
in a table\tablefootnote{A third table footnote.}
\end{tabular}
\caption[A table]{A normal table.\label{tab.normal}}
\end{table}

Some text.\footnote{Another text footnote.}

\pagebreak

%    \end{macrocode}
%
% \pagebreak
%
%    \begin{macrocode}
More text.\footnote{And yet another text footnote.}

\begin{table}[t]
\centering
\begin{tabular}{|c|c|c|}
\hline
Another\tablefootnote{A $3^{rd}$ table footnote.} & %
text\tablefootnote{Another $3^{rd}$ table footnote.} & %
in a table\tablefootnote{A $3^{rd}$ third table footnote.}\\ \hline
\end{tabular}
\caption[Another table]{Another table (third one)\label{tab.another}}
\end{table}

Please note that Table~\ref{tab.another} floated to the top of the page.
While the footnotes are set and (when hyperref is used) hyperlinked,
they are not automatically adapted. Thus either do not use a footnote
at the same page before the table, or place the table in
\textquotedblleft here\textquotedblright\ or
\textquotedblleft bottom\textquotedblright\ position.
\verb|\clear(double)page|, \verb|h(!)|, \verb|H|~from the
\texttt{float} package, or \verb|\FloatBarrier| from the
\texttt{picins} package might help, too. (Or move the table in the
source code near the position where it floats to
or use the optional footnote marks.)

Table~\ref{tab.floatH} (page~\pageref{tab.floatH}) uses float specifier
\texttt{H} from the float package and does not float.

Some text.\footnote{This is just another text footnote.}

\pagebreak

\FloatBarrier

\begin{sidewaystable}
\centering%
\begin{tabular}{ccc}
Text\tablefootnote{Please rotate the view for testing the %
hyperlinks.} & %
in a\tablefootnote[98]{Another sidewaystable footnote %
with optional footnote mark.} & %
sidewaystable%
\tablefootnote{Sidewaystable-environment provided by %
\url{http://www.ctan.org/pkg/rotating} package.}
\end{tabular}
\caption[A sideways table]{A table in the \texttt{sideways} %
environment\label{tab.sideways}}%
\end{sidewaystable}%

\FloatBarrier

\pagebreak

%    \end{macrocode}
%
% \pagebreak
%
%    \begin{macrocode}
A last table, here with float specifier \texttt{H} from the
float\footnote{\url{http://www.ctan.org/pkg/float}} package.

\begin{table}[H]
\centering
\begin{tabular}{ccc}
Another\tablefootnote{A $5^{th}$ table footnote.} & %
text\tablefootnote{Another $5^{th}$ table footnote.} & %
in a table\tablefootnote{A $5^{th}$ third table footnote.}
\end{tabular}
\caption[A last table]{A very last table\label{tab.floatH}}
\end{table}

Some text.\footnote{This is just another text footnote.}

\pagebreak

End of the example for the
tablefootnote\footnote{\url{http://www.ctan.org/pkg/tablefootnote}}
package.

\end{document}
%</example>
%    \end{macrocode}
%
% \StopEventually{}
%
% \newpage
%
% \section{The implementation}
%
% We start off by checking that we are loading into \LaTeXe{} and
% announcing the name and version of this package.
%
%    \begin{macrocode}
%<*package>
%    \end{macrocode}
%
%    \begin{macrocode}
\NeedsTeXFormat{LaTeX2e}[2011/06/27]
\ProvidesPackage{tablefootnote}[2014/01/26 v1.1c
              Table foot notes (HMM)]
%    \end{macrocode}
%
% A short description of the \xpackage{tablefootnote} package:
%
%    \begin{macrocode}
%% Provides the \tablefootnote[...]{...} command
%% for footnotes in (sideways)tables.

%    \end{macrocode}
%
% We need the \xpackage{ltxcmds} and \xpackage{letltxmacro} packages
% by \textsc{Heiko Oberdiek}:
%
%    \begin{macrocode}
\RequirePackage{ltxcmds}[2011/11/09]%     v1.22
\RequirePackage{letltxmacro}[2010/09/02]% v1.4
\RequirePackage{xifthen}[2009/04/17]%     v1.3
%    \end{macrocode}
%
% A last information for the user:
%
%    \begin{macrocode}
%% tablefootnote may work with earlier versions of LaTeX and those
%% packages, but this was not tested. Please consider updating
%% your LaTeX and packages to the most recent version
%% (if they are not already the most recent version).

%    \end{macrocode}
%
% \indent See subsection~\ref{ss:Downloads} about how to get them.\\
%
% When the \xpackage{rotating} package (for |sidewaystable|s) is used,
% it must be loaded before the \xpackage{tablefootnote} package and it must be
% a recent version:
%
%    \begin{macrocode}
\@ifpackageloaded{rotating}{%
  \gdef\tfn@rotating{1}
  \@ifpackagelater{rotating}{2009/03/28}% v2.16a
   {% >= 2009/03/28, OK
   }{% else
     \PackageWarningNoLine{tablefootnote}{%
       It is required version\MessageBreak%
       2009/03/28 v2.16a (or later) of package rotating,\MessageBreak%
       but only version\MessageBreak%
      `\csname ver@rotating.sty\endcsname'\MessageBreak%
      is available}%
   }%
 }{% else
   \gdef\tfn@rotating{0}%
 }

%    \end{macrocode}
%
% There are no options to be processed.\\
%
% We need some definitions:
%
%    \begin{macrocode}
\def\tfn@footnotetablecount{0}
\def\tfn@footnotetableprint{0}
\def\tfn@footnotezero{0}
\def\tfn@fnt{0}
\def\tfn@swt{0}%       _s_ide_w_ays_t_able
\def\tfn@fmpw{\relax}% f_oot_m_isc_p_ara_w_arn-switch
\def\tfn@fnbr{0}%      f_oot_n_ote_b_ack_r_ef(erence package)

%    \end{macrocode}
%
% \begin{macro}{\tfn@hyperfootnotes}
% We need a wrapper for the used |\ifHy@hyperfootnotes|,
% which is not defined when the \xpackage{hyperref} package has not been loaded:
%
%    \begin{macrocode}
\DeclareRobustCommand{\tfn@hyperfootnotes}[2]{%
  \ifHy@hyperfootnotes%
    #1%
  \else%
    #2%
  \fi%
  }

%    \end{macrocode}
% \end{macro}
%
% \begin{macro}{\tablemakefntext}
% |\tablemakefntext| makes the footnotetext for a footnote in a table.
%
%    \begin{macrocode}
\ltx@ifpackageloaded{footnotebackref}{%
  \gdef\tfn@fnbr{1}%
  \ifx\FootnoteBackref@symbol\empty%
    \ifFootnoteBackref@numberlinked%
      \newcommand\tablemakefntext[1]{%
        \noindent\makebox[1.634em][r]{%
         \mbox{\textsuperscript{\normalfont%
           \hyperlink{tfn:m\theHfootnote}{\@thefnmark}%
        }}}%
        \ #1%
      }
    \else
      \newcommand\tablemakefntext[1]{%
        \noindent\makebox[1.8em][r]{%
          \mbox{\textsuperscript{\normalfont%
            \@thefnmark%
        }}}%
        #1%
      }
    \fi
  \else
    \newcommand\tablemakefntext[1]{%
      \noindent\makebox[1.8em][r]{%
        \mbox{\textsuperscript{\normalfont%
          \ifFootnoteBackref@numberlinked%
            \hyperlink{tfn:m\theHfootnote}{\@thefnmark}\ %
          \else%
            \@thefnmark%
          \fi%
      }}}%
      \hyperlink{tfn:m\theHfootnote}{\FootnoteBackref@symbol}%
      #1%
    }
  \fi
}{%
  \newcommand\tablemakefntext[1]{%
    \noindent\makebox[1.8em][r]{%
      \mbox{\textsuperscript{\normalfont%
        \@thefnmark%
    }}}%
    #1%
  }
}

%    \end{macrocode}
%
% \end{macro}
%
% \pagebreak
%
% \begin{macro}{\swtablemakefntext}
%    \begin{macrocode}
\newcommand\swtablemakefntext[1]{\noindent\ #1}

%    \end{macrocode}
% \end{macro}
%
% \begin{macro}{\tfn@footnotemarks}
% |\tfn@footnotemarks| uses the |\footnotemark| command to place a foot note mark
% and decreases the |\tfn@footnotetablecount| (i.\,e.~increases the negative value)
% of the number of footnote(marks) in the table. If the \xpackage{footnotebackref}
% package is used, it is also necessary to create an according |\hypertarget| for
% the back reference.
%
%    \begin{macrocode}
\newcommand{\tfn@footnotemarks}[1][]{%
  \hbox{%
    \ifthenelse{\isempty{#1}}%
      {\footnotemark%
       \ltx@ifpackageloaded{footnotebackref}%
         {\raisebox{\ht\strutbox}{\hypertarget{tfn:m\theHfootnote}{}}}%
         {% footnotebackref not loaded
         }%
      }%
      {\ltx@ifpackageloaded{hyperref}%
        {\tfn@hyperfootnotes{%
            \hyperlink{tfn:#1}{\footnotemark[#1]}%
          }{\footnotemark[#1]}}%
        {\footnotemark[#1]}%
       \ltx@ifpackageloaded{footnotebackref}%
        {\raisebox{\ht\strutbox}{\hypertarget{tfn:m#1}{}}}{}%
      }%
   }%
  \@tempcnta=\tfn@footnotetablecount\relax%
  \advance\@tempcnta by -1\relax%
  \xdef\tfn@footnotetablecount{\the\@tempcnta}%
}

%    \end{macrocode}
% \end{macro}
%
% \begin{macro}{\tfn@footmiscparawarn}
% When the \xpackage{footmisc} package is used with option |para|,
% we need to issue a warning. (Any idea how to resolve this issue?)\\
% It must be given inside of it's own command, because |\ifFN@para| is not
% defined when \xpackage{footmisc} has not been loaded, and then the
% |\fi| would be a |\fi| without |\if| (and \TeX{} does not like single
% |\fi|s).
%
%    \begin{macrocode}
\newcommand{\tfn@footmiscparawarn}{%
  \ifFN@para%
    \PackageWarning{tablefootnote}{%
      Package footmisc with option para used.\MessageBreak%
      Unfortunately package tablefootnote is not yet\MessageBreak%
      able to regard this in sidewaystables.\MessageBreak%
      Here the footnotes are printed one below\MessageBreak%
      the other. Sorry!\MessageBreak%
     }%
  \fi%
  }

%    \end{macrocode}
% \end{macro}
%
% \begin{macro}{\tfn@updateHyper}
% |\tfn@updateHyper| updates some \xpackage{hyperref} internals.
%
%    \begin{macrocode}
\newcommand{\tfn@updateHyper}{%
  % The code of this command is partly from the
  % hyperref package by Heiko Oberdiek, 2011/10/01, v6.82j.
  % Newer versions of that package are available at CTAN.org.
  \global\let\Hy@saved@currentHref\@currentHref%
  \tfn@hyperfootnotes{\hyper@makecurrent{Hfootnote}}{}%
  \global\let\Hy@footnote@currentHref\@currentHref%
  \global\let\@currentHref\Hy@saved@currentHref%
  }

%    \end{macrocode}
% \end{macro}
%
% \begin{macro}{\tfn@footnotetext}
% |\tfn@footnotetext| contains the |\footnotetext[...]{...}| command and handles
% the numbering. |\tfn@footnotetext| will be executed after the table.
% The foot note counter has been increased automatically with each |\footnotemark|,
% therefore it must be reset to its value before the table began.
%
%    \begin{macrocode}
\newcommand{\tfn@footnotetext}[2][]{%
  \ifx\tfn@footnotetableprint\tfn@footnotezero%
    % This is the first footnotetext after the (sideways)table.
    \addtocounter{footnote}{\tfn@footnotetablecount}%
%    \end{macrocode}
%
% When the \xpackage{hyperref} package is used, also its |Hfootnote| counter must
% be reset (when the footnotes are hyperlinked).
%
%    \begin{macrocode}
    \ltx@ifpackageloaded{hyperref}{%
      \tfn@hyperfootnotes{%
        \addtocounter{Hfootnote}{\tfn@footnotetablecount}}{}%
      \tfn@updateHyper%
     }{}%
  \fi%
%    \end{macrocode}
%
% For issuing the text for the next footnote the footnote-counter
% (and |Hfootnote| counter in case of links) are increased by one.
%
%    \begin{macrocode}
  \addtocounter{footnote}{+1}%
  \ltx@ifpackageloaded{hyperref}{%
    \tfn@hyperfootnotes{\addtocounter{Hfootnote}{+1}}{}%
%    \end{macrocode}
%
% The \xpackage{hyperref} package was not expecting this kind of change of the
% |(H)footnote| numbers, therefore some code-lines are missing here which can be
% found at other places in that same package and are called by this command.
%
%    \begin{macrocode}
    \tfn@updateHyper%
%    \end{macrocode}
%
% When \xpackage{hyperref} is not loaded, nothing is to be done here.
%
%    \begin{macrocode}
   }{}%
%    \end{macrocode}
%
% The |\footnotetext[...]{...}| places the foot note text as usual.
%
%    \begin{macrocode}
  \ifx\tfn@swt\tfn@footnotezero%
    \ifthenelse{\isempty{#1}}%
      {\footnotetext{#2}}%
      {\ltx@ifpackageloaded{hyperref}%
         {\tfn@hyperfootnotes{%
             \footnotetext[#1]{%
               \raisebox{\ht\strutbox}{\hypertarget{tfn:#1}{}}#2}%
           }{\footnotetext[#1]{#2}%
           }%
         }{\footnotetext[#1]{#2}}%
      }%
  \else%
%    \end{macrocode}
%
% If there is more than one tablefootnote in the |sidewaystable|, \\
% the |\tfn@footmiscparawarn| shall be given,
%
%    \begin{macrocode}
    \ifx\tfn@footnotetableprint\tfn@footnotezero%
      \gdef\tfn@fmpw{\tfn@footmiscparawarn}%
    \else%
      \ltx@ifpackageloaded{footmisc}{\tfn@fmpw}{\relax}%
%    \end{macrocode}
%
% but we do not want to repeat the message for a third(, forth,\ldots)
% tablefootnote in the |sidewaystable|.
%
%    \begin{macrocode}
      \gdef\tfn@fmpw{\relax}%
    \fi%
%    \end{macrocode}
%
% |\@makefnmark| is \\
% |\hbox {\@textsuperscript {\normalfont \@thefnmark }}|, \\
% but according to |\show\@thefnmark| here |\@thefnmark| is
% |{\protect \itshape}| instead of |\thefootnote| (Why?!).\\
% If the \xpackage{footnotebackref} package is used,
% back references must be created for the table footnotes, too.
%
%    \begin{macrocode}
    \ltx@ifpackageloaded{footnotebackref}{%
      \LetLtxMacro{\tfncurrentmft}{\@makefntext}%
      \LetLtxMacro{\@makefntext}{\swtablemakefntext}%
      \ifthenelse{\isempty{#1}}%
        {\footnotetext{%
           \raisebox{\ht\strutbox}{\hypertarget{tfn:\theHfootnote}{}}%
           \hbox{\@textsuperscript{\normalfont%
              \hyperlink{tfn:m\theHfootnote}{\thefootnote}}}\thinspace%
         \ifx\FootnoteBackref@symbol\empty%
         \else\hyperlink{tfn:m\theHfootnote}{\FootnoteBackref@symbol}%
         \fi%
         #2}%
       }{\edef\tfn@tfn{\arabic{footnote}}%
         \edef\tfn@thn{\arabic{Hfootnote}}%
         \setcounter{footnote}{#1}%
         \setcounter{Hfootnote}{#1}%
         \tfn@updateHyper%
         \footnotetext{%
           \raisebox{\ht\strutbox}{\hypertarget{tfn:#1}{}}%
           \hbox{\@textsuperscript{\normalfont%
                  \hyperlink{tfn:m#1}{#1}}}%
           \thinspace%
           \ifx\FootnoteBackref@symbol\empty%
           \else%
             \raisebox{\ht\strutbox}{%
               \hyperlink{tfn:m#1}{\FootnoteBackref@symbol}}%
           \fi%
           #2%
          }%
         \setcounter{footnote}{\tfn@tfn}%
         \setcounter{Hfootnote}{\tfn@thn}%
         \tfn@updateHyper%
       }%
      \LetLtxMacro{\@makefntext}{\tfncurrentmft}%
     }{% footnotebackref not loaded
       \ifthenelse{\isempty{#1}}%
         {\footnotetext{%
            \hbox{\@textsuperscript{\normalfont\thefootnote}}%
            \ltx@ifpackageloaded{hyperref}%
              {\tfn@hyperfootnotes{%
                 \raisebox{\ht\strutbox}{%
                   \hypertarget{tfn:\theHfootnote}{}}}{}%
              }{}%
            \thinspace%
            #2}%
        }{%
%    \end{macrocode}
%
% \pagebreak
%
%    \begin{macrocode}
          \edef\tfn@tfn{\arabic{footnote}}%
          \ltx@ifpackageloaded{hyperref}{%
            \tfn@hyperfootnotes{\xdef\tfn@thn{\arabic{Hfootnote}}}{}%
           }{}%
          \setcounter{footnote}{#1}
          \ltx@ifpackageloaded{hyperref}{%
            \tfn@hyperfootnotes{\setcounter{Hfootnote}{#1}}{}%
            \tfn@updateHyper%
           }{}%
          \footnotetext{%
            \hbox{\@textsuperscript{\normalfont #1}}%
            \ltx@ifpackageloaded{hyperref}{%
              \tfn@hyperfootnotes{\raisebox{\ht\strutbox}{%
                                         \hypertarget{tfn:#1}{}}}{}%
             }{}%
            \thinspace%
            #2}%
          \setcounter{footnote}{\tfn@tfn}%
          \ltx@ifpackageloaded{hyperref}{%
            \tfn@hyperfootnotes{\setcounter{Hfootnote}{\tfn@thn}}{}%
            \tfn@updateHyper%
           }{}%
        }%
     }%
  \fi%
%    \end{macrocode}
%
% For further footnotes in the table the \\
% |\ifx\tfn@footnotetableprint\tfn@footnotezero|\ldots|\fi| \\
% from the beginning of this |\tfn@footnotetext| macro must not be repeated,
% therefore we deactivate this by defining |\tfn@footnotetableprint| to
% something other than |0| (here |1| is used).
%
%    \begin{macrocode}
  \gdef\tfn@footnotetableprint{1}%
%    \end{macrocode}
%
% |\tfn@footnotetablecount| contains the negative number of footnote texts
% to be printed, therefore it is increase by one (i.\,e.~the negative number
% decreases, i.\,e.~$-3$ becomes~$-2$).
%
%    \begin{macrocode}
  \@tempcnta=\tfn@footnotetablecount\relax%
  \advance\@tempcnta by +1\relax%
  \xdef\tfn@footnotetablecount{\the\@tempcnta}%
%    \end{macrocode}
%
% When it is zero, all footnote texts have been placed, and\\
% |\tfn@footnotetableprint| can be reset to zero and\\
% |\tfn@tablefootnoteprintout| can be cleared.
%
%    \begin{macrocode}
  \ifx\tfn@footnotetablecount\tfn@footnotezero% all footnotes placed
    \gdef\tfn@footnotetableprint{0}%
    \global\let\tfn@tablefootnoteprintout\relax% brute force
 %\else there is at least one \footnotemark which needs \footnotetext
  \fi
}

%    \end{macrocode}
% \end{macro}
%
% \begin{macro}{\tfn@tablefootnoteprintout}
% When something shall be cleared or redefined, it is a good idea
% to define it. It~will be later filled with the foot note text to be printed.
%
%    \begin{macrocode}
\newcommand{\tfn@tablefootnoteprintout}{\relax}

%    \end{macrocode}
% \end{macro}
%
% \pagebreak
%
% \begin{macro}{\tablefootnote}
% |\tablefootnote[...]{...}| is the command to be used by the user instead of
% |\footnote[...]{...}| in the (sideways)table. It places a foot note mark via
% |\tfn@footnotemarks| and appends a |\tfn@footnotetext{#1}| to the\\
% |\tfn@tablefootnoteprintout| macro, and remembers that table foot notes
% have been used.
%
%    \begin{macrocode}
\newcommand{\tablefootnote}[2][]{%
  \tfn@footnotemarks[#1]%
  \ifthenelse{\isempty{#1}}{\relax}{%
    \ltx@GlobalPrependToMacro{\tfn@tablefootnoteprintout}{%
      \addtocounter{footnote}{+1}%
      \ltx@ifpackageloaded{hyperref}{%
        \tfn@hyperfootnotes{\addtocounter{Hfootnote}{+1}}{}%
        \tfn@updateHyper%
       }{}%
     }%
   }%
  \ltx@GlobalAppendToMacro{\tfn@tablefootnoteprintout}{%
    \tfn@footnotetext[#1]{#2}%
    \ifthenelse{\isempty{#1}}{\relax}{%
      \addtocounter{footnote}{-1}%
      \ltx@ifpackageloaded{hyperref}{%
        \tfn@hyperfootnotes{\addtocounter{Hfootnote}{-1}}{}%
        \tfn@updateHyper%
       }{}%
     }%
   }%
  \gdef\tfn@fnt{1}%
  }

%    \end{macrocode}
% \end{macro}
%
% \begin{macro}{\tfn@swtbox}
% For |sidewaystable|s the |\tfn@tablefootnoteprintout| is placed in a \\
% |\parbox|, which can be placed at the end of the |sidewaystable|:
%
%    \begin{macrocode}
\newcommand{\tfn@swtbox}{%
\ifvoid \footins%
\vskip 2\baselineskip%
\else%
\vskip \skip\footins%
\fi%
\parbox[b][][t]{\linewidth}{\raggedright\tfn@tablefootnoteprintout}%
}

%    \end{macrocode}
% \end{macro}
%
% |\tfn@tablefootnoteprintout| shall be executed after the end of the table environment.
% The \xpackage{etoolbox} package provides very easy syntax for this:
%
%    \begin{macrocode}
\IfFileExists{etoolbox.sty}{%
  \RequirePackage{etoolbox}[2011/01/03]%   v2.1
  \AfterEndEnvironment{table}{%
    \ltx@ifpackageloaded{footnotebackref}{%
      \LetLtxMacro{\tfncurrentmft}{\@makefntext}%
      \LetLtxMacro{\@makefntext}{\tablemakefntext}%
       \tfn@tablefootnoteprintout%
      \LetLtxMacro{\@makefntext}{\tfncurrentmft}%
     }{% footnotebackref not loaded
      \tfn@tablefootnoteprintout%
     }%
      \gdef\tfn@fnt{0}%
   }%
%    \end{macrocode}
%
% When the \xpackage{rotating} package has been loaded, we need to detect
% whether the |\tablefootnote| is placed inside of a |sidewaystable|.
%
%    \begin{macrocode}
  \ltx@ifpackageloaded{rotating}{%
    \AtBeginEnvironment{sidewaystable}{%
      \gdef\tfn@swt{1}%
    }
%    \end{macrocode}
%
% At the end of |sidewaystable| environments |\tfn@swtbox| shall be placed
% (if~there was any |\tablefootnote| in that |sidewaystable| at all).
% And |\tfn@swt| as well as |\tfn@fnt| must be reset to |0|.
%
%    \begin{macrocode}
    \AtEndEnvironment{sidewaystable}{%
      \ifx\tfn@fnt\tfn@footnotezero% \relax
      \else%
        \ltx@ifpackageloaded{footnotebackref}{%
          \LetLtxMacro{\tfncurrentmft}{\@makefntext}%
          \LetLtxMacro{\@makefntext}{\tablemakefntext}%
           \tfn@swtbox%
          \LetLtxMacro{\@makefntext}{\tfncurrentmft}%
         }{% footnotebackref not loaded
          \tfn@swtbox%
          }%
      \fi%
      \gdef\tfn@swt{0}%
      \gdef\tfn@fnt{0}%
     }%
   }{}% else \relax
}{%
%    \end{macrocode}
%
% When the \xpackage{etoolbox} package is not available (e.\,g. because of missing
% \eTeX{}) we~do a workaround:
%
%    \begin{macrocode}
  \PackageWarning{tablefootnote}{%
    Package etoolbox not available, therefore\MessageBreak%
    substituting commands \string\AtEndEnvironment\space %
    and\MessageBreak%
    \string\AfterEndEnvironment\space for tablefootnotes%
    \MessageBreak%
   }%
  \@ifpackagelater{letltxmacro}{2010/09/02}%
   {% >= 2010/09/02, OK
    \GlobalLetLtxMacro{\tfnorigtable}{\table}
    \GlobalLetLtxMacro{\tfnendorigtable}{\endtable}
%    \end{macrocode}
%
% |\GlobalLetLtxMacro| was introduced in version 2010/09/02 v1.4 of the
% \xpackage{letltxmacro} package.
%
%    \begin{macrocode}
   }{% else
     \PackageError{tablefootnote}{%
       Outdated version of letltxmacro package used%
      }{Version 2010/09/02 v1.4 or newer needed but not found!%
        \MessageBreak%
        Update the used letltxmacro package.}
        \LetLtxMacro{\tfnorigtable}{\table}
        \LetLtxMacro{\tfnendorigtable}{\endtable}
    }%
%    \end{macrocode}
%
% \pagebreak
%
%    \begin{macrocode}
  \ltx@ifpackageloaded{float}{%
    \@ifpackagelater{float}{2001/11/08}%   v1.3d
     {% >= 2001/11/08, OK
     }{% else
       \PackageWarningNoLine{tablefootnote}{%
         It is required version\MessageBreak
         2001/11/08 v1.3d (or later) of package float,\MessageBreak
         but only version\MessageBreak
        `\csname ver@float.sty\endcsname'\MessageBreak
        is available}%
     }%
   }{\relax}
  \renewenvironment{table}{%
    \ltx@ifpackageloaded{float}{%
%    \end{macrocode}
%
% When \xpackage{etoolbox.sty} was not found, and the \xpackage{float} package
% has been loaded, and the float specifier is \texttt{H}, then |\endtable| was
% let to |\float@endH| by the \xpackage{float} package (line~79 of
% \xpackage{float.sty}, 2001/11/08 v1.3d:\\
% |\expandafter\let\csname end#1\endcsname\float@endH|), i.\,e.~appending\\
% |\tfn@tablefootnoteprintout \gdef\tfn@fnt{0}| to |\endtable| is useless.
% Instead of this, it must be appended to |\float@endH|.\\
% (Why is this not necessary when \xpackage{etoolbox} is used?
% That is a good question, answers are welcome.)
%
%    \begin{macrocode}
      \ltx@LocalAppendToMacro{\float@endH}{%
        \ltx@ifpackageloaded{footnotebackref}{%
          \LetLtxMacro{\tfncurrentmft}{\@makefntext}%
          \LetLtxMacro{\@makefntext}{\tablemakefntext}%
           \tfn@tablefootnoteprintout%
          \LetLtxMacro{\@makefntext}{\tfncurrentmft}%
         }{% footnotebackref not loaded
          \tfn@tablefootnoteprintout%
         }%
          \gdef\tfn@fnt{0}%
       }%
     }{\relax}
    \tfnorigtable%
   }{%
     \tfnendorigtable%
     \ltx@ifpackageloaded{footnotebackref}{%
       \LetLtxMacro{\tfncurrentmft}{\@makefntext}%
       \LetLtxMacro{\@makefntext}{\tablemakefntext}%
        \tfn@tablefootnoteprintout%
       \LetLtxMacro{\@makefntext}{\tfncurrentmft}%
      }{% footnotebackref not loaded
       \tfn@tablefootnoteprintout%
      }%
     \gdef\tfn@fnt{0}%
    }%
%    \end{macrocode}
%
% \pagebreak
%
% The |sidewaystable|-environment must be treated separately:
%
%    \begin{macrocode}
  \ltx@ifpackageloaded{rotating}{%
    \let\tfnorigsidewaystable\sidewaystable
    \let\tfnendorigsidewaystable\endsidewaystable
    \renewenvironment{sidewaystable}{%
      \gdef\tfn@swt{1}%
      \tfnorigsidewaystable%
    }{%
      \ifx\tfn@fnt\tfn@footnotezero% \relax
      \else%
        \ltx@ifpackageloaded{footnotebackref}{%
          \LetLtxMacro{\tfncurrentmft}{\@makefntext}%
          \LetLtxMacro{\@makefntext}{\tablemakefntext}%
           \tfn@swtbox%
          \LetLtxMacro{\@makefntext}{\tfncurrentmft}%
         }{% footnotebackref not loaded
          \tfn@swtbox%
         }%
      \fi%
      \gdef\tfn@swt{0}%
      \gdef\tfn@fnt{0}%
      \tfnendorigsidewaystable%
     }%
   }{}%
}

%    \end{macrocode}
%
% \DescribeMacro{\AtBeginDocument}
% When the \xpackage{rotating} (for |sidewaystable|s) or the
% \xpackage{footnotebackref} (for back references for the footnotes)
% package is used, it must be loaded before the \xpackage{tablefootnote} package.
% Using both packages is also possible, in which case both must be loaded before
% the \xpackage{tablefootnote} package. If any one of those packages is loaded
% |\AtBeginDocument| after this check, the according error message will not be
% given, but packages really should be loaded before |\AtBeginDocument|.
%
%    \begin{macrocode}
\AtBeginDocument{%
  \ltx@ifpackageloaded{footnotebackref}{%
    \ifx\tfn@fnbr\tfn@footnotezero%
      \PackageError{tablefootnote}{%
        Package footnotebackref loaded after tablefootnote}{%
        When using both the footnotebackref and the tablefootnote %
        package,\MessageBreak%
        the footnotebackref package MUST be loaded before the %
        tablefootnote package!\MessageBreak%
       }%
    \fi%
   }{% else \relax
   }
  \ltx@ifpackageloaded{rotating}{%
    \ifx\tfn@rotating\tfn@footnotezero%
      \PackageError{tablefootnote}{%
        Package rotating loaded after tablefootnote}{%
        When using both the rotating and the tablefootnote package,%
        \MessageBreak%
        the rotating package MUST be loaded before the tablefootnote%
        \space package!\MessageBreak%
       }%
    \fi%
   }{% else \relax
   }
%    \end{macrocode}
%
% \pagebreak
%
% When the document is compiled with \LuaLaTeX{}, hyperlinks in rotated content
% will be misplaced, regardless of the use of the \xpackage{tablefootnote} package
% (or of tables or footnotes). The text to be printed will be OK.
% This problem of \LuaLaTeX{} (not \xpackage{tablefootnote}) was found and reported
% by \textsc{Arno Trautmann}, 2011/11/15 -- thanks!
%
%    \begin{macrocode}
  \ltx@ifpackageloaded{hyperref}{%
    \tfn@hyperfootnotes{
      \ifluatex%
        \PackageWarningNoLine{tablefootnote}{%
          LuaTeX and hyperref package used:\MessageBreak%
          Hyperreferences in rotated content\MessageBreak%
          will be misplaced\MessageBreak%
          regardless of tablefootnote package.\MessageBreak%
          The text to be printed will be OK%
         }%
      \fi%
     }{% else \relax
     }%
   }{% else \relax
   }%
%    \end{macrocode}
%
% When the \xpackage{footnotebackref} package is used but the footnotes are not
% hyperlinked, neither the back references nor the tablefootnotes will work.
%
%    \begin{macrocode}
  \ltx@ifpackageloaded{footnotebackref}{%
     \ltx@ifpackageloaded{hyperref}{%
        \tfn@hyperfootnotes{\relax}{% else
          \PackageError{tablefootnote}{%
             footnotebackref but not hyperfootnotes}{%
             The footnotebackref package is used\MessageBreak%
             but the hyperfootnotes option of the hyperref package%
             \MessageBreak%
             was set to false.\MessageBreak%
             Either enable hyperfootnotes or do not use footnotebackref.%
             \MessageBreak%
             Loading of the tablefootnote package will be aborted.%
           }%
          \endinput%
      }%
    }{\PackageError{tablefootnote}{footnotebackref but not hyperref}{%
        The footnotebackref package is used\MessageBreak%
        but not the hyperref package.\MessageBreak%
        Either load hyperref or do not use footnotebackref.%
        \MessageBreak%
        Loading of the tablefootnote package will be aborted.%
        }%
      \endinput%
    }%
  }{}
}

%    \end{macrocode}
%
%    \begin{macrocode}
%</package>
%    \end{macrocode}
%
% \pagebreak
%
% \section{Installation}
%
% \subsection{Downloads\label{ss:Downloads}}
%
% Everything is available on \CTAN{}, \url{http://www.ctan.org/},
% but may need additional packages themselves.\\
%
% \DescribeMacro{tablefootnote.dtx}
% For unpacking the |tablefootnote.dtx| file and constructing the documentation
% it is required:
% \begin{description}
% \item[-] \TeX{} Format \LaTeXe{}: \url{http://www.CTAN.org/}
%
% \item[-] document class \xpackage{ltxdoc}, 2007/11/11, v2.0u,
%           \url{http://www.ctan.org/pkg/ltxdoc}
%
% \item[-] package \xpackage{holtxdoc}, 2012/03/21, v0.24,
%           \url{http://www.ctan.org/pkg/holtxdoc}
%
% \item[-] package \xpackage{hypdoc}, 2011/08/19, v1.11,
%           \url{http://www.ctan.org/pkg/hypdoc}
% \end{description}
%
% \DescribeMacro{tablefootnote.sty}
% The |tablefootnote.sty| for \LaTeXe{} (i.\,e.~each document using
% the \xpackage{tablefootnote} package) requires:
% \begin{description}
% \item[-] \TeX{} Format \LaTeXe{}, \url{http://www.CTAN.org/}
%
% \item[-] package \xpackage{letltxmacro}, 2010/09/02, v1.4,
%           \url{http://www.ctan.org/pkg/letltxmacro}.
%
% \item[-] package \xpackage{ltxcmds}, 2011/11/09, v1.22,
%           \url{http://www.ctan.org/pkg/ltxcmds}
% \end{description}
%
% When the \xpackage{hyperref} package is used, also the \xpackage{ifluatex} package
% is needed, but it is already loaded automatically by the \xpackage{hyperref}
% package.\\
% When the \xpackage{etoolbox} package is available, it is used:
% \begin{description}
% \item[-] package \xpackage{etoolbox}, 2011/01/03, v2.1,
%           \url{http://www.ctan.org/pkg/etoolbox}.
% \end{description}
%
% \DescribeMacro{tablefootnote-example.tex}
% The |tablefootnote-example.tex| requires the same files as all
% documents using the \xpackage{tablefootnote} package and additionally:
% \begin{description}
% \item[-] class \xpackage{article}, 2007/10/19, v1.4h, from \xpackage{classes.dtx}:\\
%           \CTAN{macros/latex/base/classes.dtx}
%
% \item[-] package \xpackage{rotating}, 2009/03/28, v2.16a,
%           \url{http://www.ctan.org/pkg/rotating}
%
% \item[-] package \xpackage{float}, 2001/11/08, v1.3d,
%           \url{http://www.ctan.org/pkg/float}
%
% \item[-] package \xpackage{placeins}, 2005/04/18, v2.2,
%           \url{http://www.ctan.org/pkg/placeins}\\
%           for |\FloatBarrier|
%
% \item[-] package \xpackage{hyperref}, 2012/11/06, v6.83m,
%           \url{http://www.ctan.org/pkg/hyperref}
%
% \item[-] package \xpackage{tablefootnote}, 2014/01/26, v1.1c,\\
%           \url{http://www.ctan.org/pkg/tablefootnote}\\
%   (Well, it is the example file for this package, and because you
%    are reading the documentation for the \xpackage{tablefootnote} package,
%    it can be assumed that you already have some version of it~--
%    is it the current one?)
% \end{description}
%
% \DescribeMacro{Oberdiek}
% \DescribeMacro{holtxdoc}
% \DescribeMacro{ifluatex}
% \DescribeMacro{letltxmacro}
% \DescribeMacro{ltxcmds}
% All packages of \textsc{Heiko Oberdiek}'s bundle `oberdiek'
% (especially \xpackage{holtxdoc}, \xpackage{ifluatex}, \xpackage{letltxmacro},
% and \xpackage{ltxcmds}) are also available in a TDS compliant ZIP archive:\\
% \CTAN{install/macros/latex/contrib/oberdiek.tds.zip}.\\
% It is probably best to download and use this, because the packages in there
% are quite probably both recent and compatible among themselves.\\
%
% \pagebreak
%
% \DescribeMacro{hyperref}
% \noindent \xpackage{hyperref} is not included in that bundle and needs to be
% downloaded separately,\\
% \url{http://www.ctan.org/install/macros/latex/contrib/hyperref.tds.zip}.\\
%
% \DescribeMacro{M\"{u}nch}
% A hyperlinked list of my (other) packages can be found at
% \url{http://www.ctan.org/author/muench-hm}.\\
%
% \subsection{Package, unpacking TDS}
% \paragraph{Package.} This package is available on \CTAN{}
% \begin{description}
% \item[\CTAN{macros/latex/contrib/tablefootnote/tablefootnote.dtx}]\hspace*{0.1cm} \\
%       The source file.
% \item[\CTAN{macros/latex/contrib/tablefootnote/tablefootnote.pdf}]\hspace*{0.1cm} \\
%       The documentation.
% \item[\CTAN{macros/latex/contrib/tablefootnote/tablefootnote-example.pdf}]\hspace*{0.1cm} \\
%       The compiled example file, as it should look like.
% \item[\CTAN{macros/latex/contrib/tablefootnote/README}]\hspace*{0.1cm} \\
%       The README file.
% \end{description}
% There is also a tablefootnote.tds.zip available:
% \begin{description}
% \item[\CTAN{install/macros/latex/contrib/tablefootnote.tds.zip}]\hspace*{0.1cm} \\
%       Everything in \xfile{TDS} compliant, compiled format.
% \end{description}
% which additionally contains\\
% \begin{tabular}{ll}
% tablefootnote.ins & The installation file.\\
% tablefootnote.drv & The driver to generate the documentation.\\
% tablefootnote.sty & The \xext{sty}le file.\\
% tablefootnote-example.tex & The example file.
% \end{tabular}
%
% \bigskip
%
% \noindent For required other packages, please see the preceding subsection.
%
% \paragraph{Unpacking.} The \xfile{.dtx} file is a self-extracting
% \docstrip{} archive. The files are extracted by running the
% \xext{dtx} through \plainTeX{}:
% \begin{quote}
%   \verb|tex tablefootnote.dtx|
% \end{quote}
%
% \noindent About generating the documentation see paragraph~\ref{GenDoc} below.\\
%
% \paragraph{TDS.} Now the different files must be moved into
% the different directories in your installation TDS tree
% (also known as \xfile{texmf} tree):
% \begin{quote}
% \def\t{^^A
% \begin{tabular}{@{}>{\ttfamily}l@{ $\rightarrow$ }>{\ttfamily}l@{}}
%   tablefootnote.sty & tex/latex/tablefootnote/tablefootnote.sty\\
%   tablefootnote.pdf & doc/latex/tablefootnote/tablefootnote.pdf\\
%   tablefootnote-example.tex & doc/latex/tablefootnote/tablefootnote-example.tex\\
%   tablefootnote-example.pdf & doc/latex/tablefootnote/tablefootnote-example.pdf\\
%   tablefootnote.dtx & source/latex/tablefootnote/tablefootnote.dtx\\
% \end{tabular}^^A
% }^^A
% \sbox0{\t}^^A
% \ifdim\wd0>\linewidth
%   \begingroup
%     \advance\linewidth by\leftmargin
%     \advance\linewidth by\rightmargin
%   \edef\x{\endgroup
%     \def\noexpand\lw{\the\linewidth}^^A
%   }\x
%   \def\lwbox{^^A
%     \leavevmode
%     \hbox to \linewidth{^^A
%       \kern-\leftmargin\relax
%       \hss
%       \usebox0
%       \hss
%       \kern-\rightmargin\relax
%     }^^A
%   }^^A
%   \ifdim\wd0>\lw
%     \sbox0{\small\t}^^A
%     \ifdim\wd0>\linewidth
%       \ifdim\wd0>\lw
%         \sbox0{\footnotesize\t}^^A
%         \ifdim\wd0>\linewidth
%           \ifdim\wd0>\lw
%             \sbox0{\scriptsize\t}^^A
%             \ifdim\wd0>\linewidth
%               \ifdim\wd0>\lw
%                 \sbox0{\tiny\t}^^A
%                 \ifdim\wd0>\linewidth
%                   \lwbox
%                 \else
%                   \usebox0
%                 \fi
%               \else
%                 \lwbox
%               \fi
%             \else
%               \usebox0
%             \fi
%           \else
%             \lwbox
%           \fi
%         \else
%           \usebox0
%         \fi
%       \else
%         \lwbox
%       \fi
%     \else
%       \usebox0
%     \fi
%   \else
%     \lwbox
%   \fi
% \else
%   \usebox0
% \fi
% \end{quote}
% If you have a \xfile{docstrip.cfg} that configures and enables \docstrip{}'s
% \xfile{TDS} installing feature, then some files can already be in the right
% place, see the documentation of \docstrip{}.
%
% \subsection{Refresh file name databases}
%
% If your \TeX{}~distribution (\TeX live, \mikTeX{}, \dots{}) relies on file name
% databases, you must refresh these. For example, \teTeX{} users run
% \verb|texhash| or \verb|mktexlsr|.
%
% \subsection{Some details for the interested}
%
% \paragraph{Unpacking with \LaTeX{}.}
% The \xfile{.dtx} chooses its action depending on the format:
% \begin{description}
% \item[\plainTeX:] Run \docstrip{} and extract the files.
% \item[\LaTeX:] Generate the documentation.
% \end{description}
% If you insist on using \LaTeX{} for \docstrip{} (really,
% \docstrip{} does not need \LaTeX{}), then inform the autodetect routine
% about your intention:
% \begin{quote}
%   \verb|latex \let\install=y% \iffalse meta-comment
%
% File: tablefootnote.dtx
% Version: 2014/01/26 v1.1c
%
% Copyright (C) 2011 - 2014 by
%    H.-Martin M"unch <Martin dot Muench at Uni-Bonn dot de>
%
% This work may be distributed and/or modified under the
% conditions of the LaTeX Project Public License, either
% version 1.3c of this license or (at your option) any later
% version. This version of this license is in
%    http://www.latex-project.org/lppl/lppl-1-3c.txt
% and the latest version of this license is in
%    http://www.latex-project.org/lppl.txt
% and version 1.3c or later is part of all distributions of
% LaTeX version 2005/12/01 or later.
%
% This work has the LPPL maintenance status "maintained".
%
% The Current Maintainer of this work is H.-Martin Muench.
%
% This work consists of the main source file tablefootnote.dtx,
% the README, and the derived files
%    tablefootnote.sty, tablefootnote.pdf,
%    tablefootnote.ins, tablefootnote.drv,
%    tablefootnote-example.tex, tablefootnote-example.pdf.
%
% Distribution:
%    http://mirror.ctan.org/macros/latex/contrib/tablefootnote/tablefootnote.dtx
%    http://mirror.ctan.org/macros/latex/contrib/tablefootnote/tablefootnote.pdf
%    http://mirror.ctan.org/install/macros/latex/contrib/tablefootnote.tds.zip
%
% See http://www.ctan.org/pkg/tablefootnote (Catalogue)
%
% Unpacking:
%    (a) If tablefootnote.ins is present:
%           tex tablefootnote.ins
%    (b) Without tablefootnote.ins:
%           tex tablefootnote.dtx
%    (c) If you insist on using LaTeX
%           latex \let\install=y\input{tablefootnote.dtx}
%        (quote the arguments according to the demands of your shell)
%
% Documentation:
%    (a) If tablefootnote.drv is present:
%           (pdf)latex tablefootnote.drv
%           makeindex -s gind.ist tablefootnote.idx
%           (pdf)latex tablefootnote.drv
%           makeindex -s gind.ist tablefootnote.idx
%           (pdf)latex tablefootnote.drv
%    (b) Without tablefootnote.drv:
%           (pdf)latex tablefootnote.dtx
%           makeindex -s gind.ist tablefootnote.idx
%           (pdf)latex tablefootnote.dtx
%           makeindex -s gind.ist tablefootnote.idx
%           (pdf)latex tablefootnote.dtx
%
%    The class ltxdoc loads the configuration file ltxdoc.cfg
%    if available. Here you can specify further options, e.g.
%    use DIN A4 as paper format:
%       \PassOptionsToClass{a4paper}{article}
%
% Installation:
%    TDS:tex/latex/tablefootnote/tablefootnote.sty
%    TDS:doc/latex/tablefootnote/tablefootnote.pdf
%    TDS:doc/latex/tablefootnote/tablefootnote-example.tex
%    TDS:doc/latex/tablefootnote/tablefootnote-example.pdf
%    TDS:source/latex/tablefootnote/tablefootnote.dtx
%
%<*ignore>
\begingroup
  \catcode123=1 %
  \catcode125=2 %
  \def\x{LaTeX2e}%
\expandafter\endgroup
\ifcase 0\ifx\install y1\fi\expandafter
         \ifx\csname processbatchFile\endcsname\relax\else1\fi
         \ifx\fmtname\x\else 1\fi\relax
\else\csname fi\endcsname
%</ignore>
%<*install>
\input docstrip.tex
\Msg{******************************************************************}
\Msg{* Installation                                                   *}
\Msg{* Package: tablefootnote 2014/01/26 v1.1c Table foot notes (HMM) *}
\Msg{******************************************************************}

\keepsilent
\askforoverwritefalse

\let\MetaPrefix\relax
\preamble

This is a generated file.

Project: tablefootnote
Version: 2014/01/26 v1.1c

Copyright (C) 2011 - 2014 by
    H.-Martin M"unch <Martin dot Muench at Uni-Bonn dot de>

The usual disclaimer applies:
If it doesn't work right that's your problem.
(Nevertheless, send an e-mail to the maintainer
 when you find an error in this package.)

This work may be distributed and/or modified under the
conditions of the LaTeX Project Public License, either
version 1.3c of this license or (at your option) any later
version. This version of this license is in
   http://www.latex-project.org/lppl/lppl-1-3c.txt
and the latest version of this license is in
   http://www.latex-project.org/lppl.txt
and version 1.3c or later is part of all distributions of
LaTeX version 2005/12/01 or later.

This work has the LPPL maintenance status "maintained".

The Current Maintainer of this work is H.-Martin Muench.

This work consists of the main source file tablefootnote.dtx,
the README, and the derived files
   tablefootnote.sty, tablefootnote.pdf,
   tablefootnote.ins, tablefootnote.drv,
   tablefootnote-example.tex, tablefootnote-example.pdf.

In memoriam Tommy Muench + 2014/01/02.

\endpreamble
\let\MetaPrefix\DoubleperCent

\generate{%
  \file{tablefootnote.ins}{\from{tablefootnote.dtx}{install}}%
  \file{tablefootnote.drv}{\from{tablefootnote.dtx}{driver}}%
  \usedir{tex/latex/tablefootnote}%
  \file{tablefootnote.sty}{\from{tablefootnote.dtx}{package}}%
  \usedir{doc/latex/tablefootnote}%
  \file{tablefootnote-example.tex}{\from{tablefootnote.dtx}{example}}%
}

\catcode32=13\relax% active space
\let =\space%
\Msg{************************************************************************}
\Msg{*}
\Msg{* To finish the installation you have to move the following}
\Msg{* file into a directory searched by TeX:}
\Msg{*}
\Msg{*     tablefootnote.sty}
\Msg{*}
\Msg{* To produce the documentation run the file `tablefootnote.drv'}
\Msg{* through (pdf)LaTeX, e.g.}
\Msg{*  pdflatex tablefootnote.drv}
\Msg{*  makeindex -s gind.ist tablefootnote.idx}
\Msg{*  pdflatex tablefootnote.drv}
\Msg{*  makeindex -s gind.ist tablefootnote.idx}
\Msg{*  pdflatex tablefootnote.drv}
\Msg{*}
\Msg{* At least three runs are necessary e.g. to get the}
\Msg{*  references right!}
\Msg{*}
\Msg{* Happy TeXing!}
\Msg{*}
\Msg{************************************************************************}

\endbatchfile
%</install>
%<*ignore>
\fi
%</ignore>
%
% \section{The documentation driver file}
%
% The next bit of code contains the documentation driver file for
% \TeX{}, i.\,e., the file that will produce the documentation you
% are currently reading. It will be extracted from this file by the
% \texttt{docstrip} programme. That is, run \LaTeX{} on \texttt{docstrip}
% and specify the \texttt{driver} option when \texttt{docstrip}
% asks for options.
%
%    \begin{macrocode}
%<*driver>
\NeedsTeXFormat{LaTeX2e}[2011/06/27]
\ProvidesFile{tablefootnote.drv}%
  [2014/01/26 v1.1c Table foot notes (HMM)]
\documentclass{ltxdoc}[2007/11/11]% v2.0u
\usepackage{holtxdoc}[2012/03/21]%  v0.24
%% tablefootnote may work with earlier versions of LaTeX2e and those
%% class and package, but this was not tested.
%% Please consider updating your LaTeX, class, and package
%% to the most recent version (if they are not already the most
%% recent version).
\hypersetup{%
 pdfsubject={Footnotes in Tables (HMM)},%
 pdfkeywords={LaTeX, tablefootnote, footnote, table, H.-Martin Muench},%
 pdfencoding=auto,%
 pdflang={en},%
 breaklinks=true,%
 linktoc=all,%
 pdfstartview=FitH,%
 pdfpagelayout=OneColumn,%
 bookmarksnumbered=true,%
 bookmarksopen=true,%
 bookmarksopenlevel=3,%
 pdfmenubar=true,%
 pdftoolbar=true,%
 pdfwindowui=true,%
 pdfnewwindow=true%
}
\CodelineIndex
\hyphenation{printing docu-ment}
\gdef\unit#1{\mathord{\thinspace\mathrm{#1}}}%
\makeatletter
\@ifundefined{eTeX}{\gdef\eTeX{$\m@th \varepsilon $-\TeX }}{% else \relax
}
\makeatother
\begin{document}
  \DocInput{tablefootnote.dtx}%
\end{document}
%</driver>
%    \end{macrocode}
%
% \fi
%
% \CheckSum{584}
%
% \CharacterTable
%  {Upper-case    \A\B\C\D\E\F\G\H\I\J\K\L\M\N\O\P\Q\R\S\T\U\V\W\X\Y\Z
%   Lower-case    \a\b\c\d\e\f\g\h\i\j\k\l\m\n\o\p\q\r\s\t\u\v\w\x\y\z
%   Digits        \0\1\2\3\4\5\6\7\8\9
%   Exclamation   \!     Double quote  \"     Hash (number) \#
%   Dollar        \$     Percent       \%     Ampersand     \&
%   Acute accent  \'     Left paren    \(     Right paren   \)
%   Asterisk      \*     Plus          \+     Comma         \,
%   Minus         \-     Point         \.     Solidus       \/
%   Colon         \:     Semicolon     \;     Less than     \<
%   Equals        \=     Greater than  \>     Question mark \?
%   Commercial at \@     Left bracket  \[     Backslash     \\
%   Right bracket \]     Circumflex    \^     Underscore    \_
%   Grave accent  \`     Left brace    \{     Vertical bar  \|
%   Right brace   \}     Tilde         \~}
%
% \GetFileInfo{tablefootnote.drv}
%
% \begingroup
%   \def\x{\#,\$,\^,\_,\~,\ ,\&,\{,\},\%}%
%   \makeatletter
%   \@onelevel@sanitize\x
% \expandafter\endgroup
% \expandafter\DoNotIndex\expandafter{\x}
% \expandafter\DoNotIndex\expandafter{\string\ }
% \begingroup
%   \makeatletter
%     \lccode`9=32\relax
%     \lowercase{%^^A
%       \edef\x{\noexpand\DoNotIndex{\@backslashchar9}}%^^A
%     }%^^A
%   \expandafter\endgroup\x
%
% \DoNotIndex{\\,\,}
% \DoNotIndex{\documentclass,\usepackage,\ProvidesPackage,\begin,\end}
% \DoNotIndex{\MessageBreak,\message}
% \DoNotIndex{\NeedsTeXFormat,\DoNotIndex,\verb}
% \DoNotIndex{\def,\edef,\gdef,\global}
% \DoNotIndex{\ifx,\listfiles,\mathord,\mathrm}
% \DoNotIndex{\kvoptions,\SetupKeyvalOptions,\ProcessKeyvalOptions}
% \DoNotIndex{\bigskip,\space,\thinspace,\ldots}
% \DoNotIndex{\indent,\noindent,\newline,\linebreak,\pagebreak}
% \DoNotIndex{\textbf,\textit,\textsf,\texttt,\textsc,\textquotedblleft,\textquotedblright}
% \DoNotIndex{\plainTeX,\TeX,\LaTeX,\pdfLaTeX}
% \DoNotIndex{\chapter,\section,\the,\csname,\endscname}
% \DoNotIndex{\value,\arabic,\hline,\text,\caption,\centering,\clear,\url}
%
% \title{The \xpackage{tablefootnote} package}
% \date{2014/01/26 v1.1c}
% \author{H.-Martin M\"{u}nch\\\xemail{Martin.Muench at Uni-Bonn.de}}
%
% \maketitle
%
% \begin{abstract}
% This \LaTeX{} package provides the command |\tablefootnote| to be used
% in a \texttt{table} or \texttt{sidewaystable} environment,
% where |\footnote| will not work and when using |\footnotemark| and
% |\footnotetext| and adjusting the counters (including \texttt{Hfootnote})
% manually is either too much work or would not even work (\texttt{sidewaystable}).
% \end{abstract}
%
% \bigskip
%
% \noindent Disclaimer for web links: The author is not responsible for any contents
% referred to in this work unless he has full knowledge of illegal contents.
% If any damage occurs by the use of information presented there, only the
% author of the respective pages might be liable, not the one who has referred
% to these pages.
%
% \bigskip
%
% \noindent {\color{green} Save per page about $200\unit{ml}$ water,
% $2\unit{g}$ CO$_{2}$ and $2\unit{g}$ wood:\\
% Therefore please print only if this is really necessary.}
%
% \pagebreak
%
% \tableofcontents
%
% \pagebreak
%
% \section{Introduction\label{sec:Introduction}}
%
% \indent This \LaTeX{} package provides the command |\tablefootnote{...}| to be
% used in a \texttt{table} or \texttt{sidewaystable} environment. In those
% environments |\footnote{...}| would not work and would need to be replaced
% by |\footnotemark| in the \texttt{(sideways)\allowbreak table} and
% |\footnotetext{...}| after the \texttt{(sideways)table} environment ended.
% Additionally the counters for \texttt{footnote} and
% (when the \xpackage{hyperref} package is used) \texttt{Hfootnote}
% would need to be adjusted manually. Then still the hyperlinks
% in a \texttt{(sideways)table} would not work.
% When the \xpackage{footnotebackref} package is used,
% |\footnotemark| and |\footnotetext{...}| would need to be redefined.
% Just using |\tablefootnote{...}| (with optional argument for custom foot note
% marks) in the (sideways)table does all this automatically.
% (Compatibility with the \xpackage{footnotebackref} package
% was achieved without redefining |\footnotemark| or |\footnotetext|.)
% Redefining |\footnote| was \emph{not} done as to not create conflicts with
% other packages.
%
% \section{Usage}
%
% \indent Just load the package placing
% \begin{quote}
%   |\usepackage{tablefootnote}|
% \end{quote}
% \noindent in the preamble of your \LaTeXe{} source file and use\newline
% |\tablefootnote{...}| instead of |\footnote{...}| and\newline
% |\tablefootnote[...]{...}| instead of |\footnote[...]{...}|
% in (sideways)\allowbreak tables.\\
% When the \xpackage{footnotebackref} package is used,
% it must be loaded before the \xpackage{tablefootnote} package
% and the \xpackage{hyperrefpackage} with option |hyperfootnotes=true|
% must be loaded, too. (Backreference{-}links without links do not make sense.)\\
% When the \xpackage{rotating} package (for |sidewaystable|s) is used,
% it must be loaded before the \xpackage{tablefootnote} package and it must be
% a recent version (v2.16a, 2009/03/28, or newer).\\
% If |\tablefootnote{...}| is used in a tabular environment, which is inside
% of a (sideways)table environment, everything is fine, but when the tabular
% environment is not inside of a (sideways)table, the \textbf{table}footnote
% will not work.\\
%
% Pages with tables with footnotes, which are rotated with the \xpackage{lscape}-
% or with the \xpackage{pdflscape}-package (|\begin{landscape}|, table, |\end{landscape}|),
% are possible. But when the \xpackage{footmisc}-package is used, \xpackage{(pdf)lscape}
% should be loaded before \xpackage{footmisc}.\\
%
% When the \xpackage{footmisc}-package is used with option \texttt{para},
% this is regarded with the exception of a \texttt{sidewaystable}. There the footnotes
% are printed one below the other. (Well, this is better than no footnote in a
% \texttt{sidewaystable} at all, isn't it?)\\
%
% When the (sideways)table floats, the footnotes are set and (when \xpackage{hyperref}
% is used) hyperlinked, but they are not automatically adapted when the table floats
% over/under another footnote. Thus either do not use a footnote between original and
% float position of the (sideways)table, or place the (sideways)table in
% \textquotedblleft here\textquotedblright\ position. |\clear(double)page|, |h(!)|,
% |H|~from the \xpackage{float} package (\url{http://www.ctan.org/pkg/float}),
% or |\FloatBarrier| from the \xpackage{picins} package
% (\url{http://www.ctan.org/pkg/picins}) might help, too.
% (Or after finishing the document, move the (sideways)table in the source code
% near the position where it floats to or use the optional footnote marks.)
%
% \pagebreak
%
% \section{Alternatives\label{sec:Alternatives}}
%
% \begin{description}
% \item[-] The \xpackage{longtable} package provides the \texttt{longtable} environment
%   as replacement for the combined \texttt{table} and \texttt{tabular} environments.
%   Footnotes are real footnotes (not just tablenotes), are continuously numbered and
%   hyperlinked (when using the \xpackage{hyperref} package), and the hyperlinks really
%   work. As drawback the appearance of the caption changes slightly (e.\,g.~distance
%   to the table, width of the caption), which can probably be changed back manually.
%   Furthermore, longtables are meaned to break over more than one page. If that is not
%   wished, it must be prevented by |\nopagebreak|-commands and by ending the longtable
%   lines with |\\*| instead of |\\|. longtables do not float. (Therefore using the
%   \xpackage{tablefootnote} package and |\FloatBarrier| from the \xpackage{picins} package
%   before and after the table environment is similar~- but \xpackage{tablefootnote} does not
%   change the table-caption!) \texttt{sidewaystable} does not work with it.\\
%   \url{http://www.ctan.org/pkg/longtable}
%
% \item[-] The \xpackage{supertabular} package provides the \texttt{mpsupertabular}
%   environment as replacement for the combined \texttt{table} and \texttt{tabular}
%   environments. Footnotes are just tablenotes (with working hyperlinks when using the
%   \xpackage{hyperref} package), i.\,e. numbered a, b, c and placed below the table
%   and not at the end of the page. Therefore there is no float problem (because the
%   tablenotes numbering is not included in the continuous numbering of the footnotes).
%   Placing the \texttt{supertabular} inside of a \texttt{sidewaystable} breaks the
%   hyperlinks to the tablenotes.\\
%   \url{http://www.ctan.org/pkg/supertabular}
%
% \item[-] The \xpackage{ctable} package has its very own notation for defining tables.
%   It can create tablenotes and sideways-tables. The tablenotes are not automatically
%   hyperlinked. The \texttt{ctable}s float. Because the tablenotes numbering is not
%   included in the continuous numbering of the footnotes there is no float problem.\\
%   \url{http://www.ctan.org/pkg/ctable}
%
% \item[-] The \xpackage{footnote} package provides |\makesavenoteenv{table}|. After
%   loading the package and using that command in the preamble, in tables |\footnote{...}|
%   can be used. Using |\makesavenoteenv{tabular}| and \\
%   |\makesavenoteenv{sidewaystable}|
%   is possible, but it neither solves the float problem, nor do the created hyperlinks
%   work (i.\,e.~they aim at wrong locations). The \xpackage{mdwtab} from the same bundle
%   is incompatible with other table-related packages (e.\,g. \xpackage{supertabular},
%   \xpackage{array}) and not $100\,\%$ compatible with the \texttt{tabular} environment.\\
%   \url{http://www.ctan.org/pkg/footnote}\\
%   \url{http://www.ctan.org/pkg/mdwtab}
%
% \item[-] The \xpackage{tabularx} package does produce footnotes for sidewaystables,
%   but uses a, b, c instead of 1, 2, 3. The hyperlinks to the footnotes do not work.
%   Because the footnotes numbering is not included in the continuous numbering of
%   the other footnotes there is no float problem.\\
%   \url{http://www.ctan.org/pkg/tabularx}
%
% \item[-] Placing a tabular inside a minipage inside a table produces tablenotes.
%   Therefore there is no float problem (because the footnotes are not continuously
%   numbered). The hyperlinks to the table notes indeed work.
%
% \pagebreak
%
% \item[-] The \xpackage{threeparttable} package creates tablenotes again. Therefore
%   there is no float problem (because the tablenotes are not continuously numbered with
%   the footnotes). There are no hyperlinks to the table notes (at least not automatically).
%   Using |sidewaystable| (with table notes) works.\\
%   \url{http://www.ctan.org/pkg/threeparttable}
%
% \item[-] The \xpackage{threeparttablex} package creates tablenotes again. Therefore there
%   is no float problem (because the tablenotes are not continuously numbered with the
%   footnotes). With option \texttt{referable} the tablenotes are hyperlinked. Use of
%   a |sidewaystable| (with table notes) did not work for me. When using the
%   \texttt{referable} option according to the example in the \xpackage{threeparttablex}
%   manual the \xpackage{longtable} package is used, therefore that package could be used
%   directly without \xpackage{threeparttablex} (see above).\\
%   \url{http://www.ctan.org/pkg/threeparttablex}
%
% \item[-] One can manually use |\footnotemark| in the table and |\footnotetext{...}|
%  after the table environment ended and manually change the |footnote| and
%  (when the \xpackage{hyperref} package is used) |Hfootnote| counters and needs
%  to insert
%  \begin{verbatim}
%     \makeatletter
%     \global\let\Hy@saved@currentHref\@currentHref
%     \hyper@makecurrent{Hfootnote}%
%     \global\let\Hy@footnote@currentHref\@currentHref
%     \global\let\@currentHref\Hy@saved@currentHref
%     \makeatother
%  \end{verbatim}
%  before each |\footnotetext{...}|, but with more than very few footnotes this can become
%  much work (and is prone to produce errors). And this does not even work with
%  sidewaystables. (And it does not work with the \xpackage{footnotebackref} package
%  without redefining |\footnotemark| and |\footnotetext{...}|.)
% \end{description}
%
% \bigskip
%
% \noindent (You programmed or found another alternative,
%  which is available at \CTAN{}?\\
%  OK, send an e-mail to me with the name, location at \CTAN{},
%  and a short notice, and I will probably include it in
%  the list above.)
%
% \newpage
%
% \section{Example}
%
%    \begin{macrocode}
%<*example>
\documentclass[british]{article}[2007/10/19]% v1.4h
%%%%%%%%%%%%%%%%%%%%%%%%%%%%%%%%%%%%%%%%%%%%%%%%%%%%%%%%%%%%%%%%%%%%%
\usepackage{float}[2001/11/08]%    v1.3d
\usepackage{placeins}[2005/04/18]% v2.2  ; for \FloatBarrier
\usepackage{rotating}[2009/03/28]% v2.16a; for sidewaystable-environment
\usepackage[%
 hyperfootnotes=true,%
 extension=pdf,%
 plainpages=false,%
 pdfpagelabels=true,%
 hyperindex=false,%
 pdflang={en},%
 pdftitle={tablefootnote package example},%
 pdfauthor={H.-Martin Muench},%
 pdfsubject={Example for the tablefootnote package},%
 pdfkeywords={LaTeX, tablefootnote, footnote, table, H.-Martin Muench},%
 % pdfview=FitH and FitBH do not work: hyperlinks in sidewaystables
 % do no lead to the footnotes, due to a bug in pdfTeX,
 % computing wrong anchor coordinates (Heiko Oberdiek, 29. October 2011)
 % pdfview=Fit, FitV, FitR, FitB, FitBV work
 % print is OK for all those options
 pdfstartview=FitH,%
 pdfpagelayout=OneColumn%
]{hyperref}[2012/11/06]% v6.83m
 % Due to the urls used in the example, either the hyperref or the url
 % package are needed (or the urls must be removed before compiling).

\usepackage{footnotebackref}[2012/07/01]% v1.0
\usepackage{tablefootnote}[2014/01/26]%   v1.1c

\gdef\unit#1{\mathord{\thinspace\mathrm{#1}}}%
\listfiles
\begin{document}
\pagenumbering{arabic}
\section*{Example for tablefootnote}

This example demonstrates the use of package\newline
\textsf{tablefootnote}, v1.1c as of 2014/01/26 (HMM).\newline
There were no options used. (The package provides no options.)\newline

\textbf{The \texttt{tablefootnote-example.tex} needs to be compiled
at least twice to get the references right!}\newline

If the etoolbox-package is found, it is automatically used.\newline

For more details please see the documentation!\newline

\noindent Save per page about $200\unit{ml}$ water,
$2\unit{g}$ CO$_{2}$ and $2\unit{g}$ wood:\newline
Therefore please print only if this is really necessary.\newline

Here is some text.\footnote{And this is a text footnote.}\newline

Tables \ref{tab.symbol}, \ref{tab.normal}, \ref{tab.another} and
\ref{tab.floatH} show normal tables, table~\ref{tab.sideways} depicts
a sidewaystable. Table~\ref{tab.floatH} uses the float
specifier~\texttt{H} from the float package.\newline

\texttt{Hyperref} option \verb|pdfview=FitH| and \verb|FitBH| do
not work due to a bug in pdf\TeX{}, computing wrong
anchor coordinates (\textsc{Heiko Oberdiek}, 29.~October 2011).
Depending on used pdf-viewer, hyperlinks in sidewaystables lead
e.\,g.~at the end of the document, not at the footnote.
\verb|pdfview=Fit|, \verb|FitV|, \verb|FitR|, \verb|FitB|,
\verb|FitBV| work, print is OK for all those options.

\bigskip

\listoftables

\pagebreak

\renewcommand{\thefootnote}{\fnsymbol{footnote}}
\verb|\renewcommand{\thefootnote}{\fnsymbol{footnote}}|
causes footnote{-}symbol{-}footnotes, which are possible
(see Table~\ref{tab.symbol}).

\begin{table}
\centering
\begin{tabular}{ccc}
Another\tablefootnote{A table footnote.} & %
text\tablefootnote{Another table footnote.} & %
in a table\tablefootnote{A third table footnote.}
\end{tabular}
\caption[A footnotesymbol table]{%
A table with footnote-symbol-footnotes.\label{tab.symbol}}
\end{table}

Some text.\footnote{A text footnote.}

\renewcommand{\thefootnote}{\arabic{footnote}}
\verb|\renewcommand{\thefootnote}{\arabic{footnote}}|
switches back to normal footnote numbers again.

\pagebreak

\begin{table}
\centering
\begin{tabular}{ccc}
Some\tablefootnote{A table footnote.} & %
text\tablefootnote[99]{A table footnote with custom footnote number.} & %
in a table\tablefootnote{A third table footnote.}
\end{tabular}
\caption[A table]{A normal table.\label{tab.normal}}
\end{table}

Some text.\footnote{Another text footnote.}

\pagebreak

%    \end{macrocode}
%
% \pagebreak
%
%    \begin{macrocode}
More text.\footnote{And yet another text footnote.}

\begin{table}[t]
\centering
\begin{tabular}{|c|c|c|}
\hline
Another\tablefootnote{A $3^{rd}$ table footnote.} & %
text\tablefootnote{Another $3^{rd}$ table footnote.} & %
in a table\tablefootnote{A $3^{rd}$ third table footnote.}\\ \hline
\end{tabular}
\caption[Another table]{Another table (third one)\label{tab.another}}
\end{table}

Please note that Table~\ref{tab.another} floated to the top of the page.
While the footnotes are set and (when hyperref is used) hyperlinked,
they are not automatically adapted. Thus either do not use a footnote
at the same page before the table, or place the table in
\textquotedblleft here\textquotedblright\ or
\textquotedblleft bottom\textquotedblright\ position.
\verb|\clear(double)page|, \verb|h(!)|, \verb|H|~from the
\texttt{float} package, or \verb|\FloatBarrier| from the
\texttt{picins} package might help, too. (Or move the table in the
source code near the position where it floats to
or use the optional footnote marks.)

Table~\ref{tab.floatH} (page~\pageref{tab.floatH}) uses float specifier
\texttt{H} from the float package and does not float.

Some text.\footnote{This is just another text footnote.}

\pagebreak

\FloatBarrier

\begin{sidewaystable}
\centering%
\begin{tabular}{ccc}
Text\tablefootnote{Please rotate the view for testing the %
hyperlinks.} & %
in a\tablefootnote[98]{Another sidewaystable footnote %
with optional footnote mark.} & %
sidewaystable%
\tablefootnote{Sidewaystable-environment provided by %
\url{http://www.ctan.org/pkg/rotating} package.}
\end{tabular}
\caption[A sideways table]{A table in the \texttt{sideways} %
environment\label{tab.sideways}}%
\end{sidewaystable}%

\FloatBarrier

\pagebreak

%    \end{macrocode}
%
% \pagebreak
%
%    \begin{macrocode}
A last table, here with float specifier \texttt{H} from the
float\footnote{\url{http://www.ctan.org/pkg/float}} package.

\begin{table}[H]
\centering
\begin{tabular}{ccc}
Another\tablefootnote{A $5^{th}$ table footnote.} & %
text\tablefootnote{Another $5^{th}$ table footnote.} & %
in a table\tablefootnote{A $5^{th}$ third table footnote.}
\end{tabular}
\caption[A last table]{A very last table\label{tab.floatH}}
\end{table}

Some text.\footnote{This is just another text footnote.}

\pagebreak

End of the example for the
tablefootnote\footnote{\url{http://www.ctan.org/pkg/tablefootnote}}
package.

\end{document}
%</example>
%    \end{macrocode}
%
% \StopEventually{}
%
% \newpage
%
% \section{The implementation}
%
% We start off by checking that we are loading into \LaTeXe{} and
% announcing the name and version of this package.
%
%    \begin{macrocode}
%<*package>
%    \end{macrocode}
%
%    \begin{macrocode}
\NeedsTeXFormat{LaTeX2e}[2011/06/27]
\ProvidesPackage{tablefootnote}[2014/01/26 v1.1c
              Table foot notes (HMM)]
%    \end{macrocode}
%
% A short description of the \xpackage{tablefootnote} package:
%
%    \begin{macrocode}
%% Provides the \tablefootnote[...]{...} command
%% for footnotes in (sideways)tables.

%    \end{macrocode}
%
% We need the \xpackage{ltxcmds} and \xpackage{letltxmacro} packages
% by \textsc{Heiko Oberdiek}:
%
%    \begin{macrocode}
\RequirePackage{ltxcmds}[2011/11/09]%     v1.22
\RequirePackage{letltxmacro}[2010/09/02]% v1.4
\RequirePackage{xifthen}[2009/04/17]%     v1.3
%    \end{macrocode}
%
% A last information for the user:
%
%    \begin{macrocode}
%% tablefootnote may work with earlier versions of LaTeX and those
%% packages, but this was not tested. Please consider updating
%% your LaTeX and packages to the most recent version
%% (if they are not already the most recent version).

%    \end{macrocode}
%
% \indent See subsection~\ref{ss:Downloads} about how to get them.\\
%
% When the \xpackage{rotating} package (for |sidewaystable|s) is used,
% it must be loaded before the \xpackage{tablefootnote} package and it must be
% a recent version:
%
%    \begin{macrocode}
\@ifpackageloaded{rotating}{%
  \gdef\tfn@rotating{1}
  \@ifpackagelater{rotating}{2009/03/28}% v2.16a
   {% >= 2009/03/28, OK
   }{% else
     \PackageWarningNoLine{tablefootnote}{%
       It is required version\MessageBreak%
       2009/03/28 v2.16a (or later) of package rotating,\MessageBreak%
       but only version\MessageBreak%
      `\csname ver@rotating.sty\endcsname'\MessageBreak%
      is available}%
   }%
 }{% else
   \gdef\tfn@rotating{0}%
 }

%    \end{macrocode}
%
% There are no options to be processed.\\
%
% We need some definitions:
%
%    \begin{macrocode}
\def\tfn@footnotetablecount{0}
\def\tfn@footnotetableprint{0}
\def\tfn@footnotezero{0}
\def\tfn@fnt{0}
\def\tfn@swt{0}%       _s_ide_w_ays_t_able
\def\tfn@fmpw{\relax}% f_oot_m_isc_p_ara_w_arn-switch
\def\tfn@fnbr{0}%      f_oot_n_ote_b_ack_r_ef(erence package)

%    \end{macrocode}
%
% \begin{macro}{\tfn@hyperfootnotes}
% We need a wrapper for the used |\ifHy@hyperfootnotes|,
% which is not defined when the \xpackage{hyperref} package has not been loaded:
%
%    \begin{macrocode}
\DeclareRobustCommand{\tfn@hyperfootnotes}[2]{%
  \ifHy@hyperfootnotes%
    #1%
  \else%
    #2%
  \fi%
  }

%    \end{macrocode}
% \end{macro}
%
% \begin{macro}{\tablemakefntext}
% |\tablemakefntext| makes the footnotetext for a footnote in a table.
%
%    \begin{macrocode}
\ltx@ifpackageloaded{footnotebackref}{%
  \gdef\tfn@fnbr{1}%
  \ifx\FootnoteBackref@symbol\empty%
    \ifFootnoteBackref@numberlinked%
      \newcommand\tablemakefntext[1]{%
        \noindent\makebox[1.634em][r]{%
         \mbox{\textsuperscript{\normalfont%
           \hyperlink{tfn:m\theHfootnote}{\@thefnmark}%
        }}}%
        \ #1%
      }
    \else
      \newcommand\tablemakefntext[1]{%
        \noindent\makebox[1.8em][r]{%
          \mbox{\textsuperscript{\normalfont%
            \@thefnmark%
        }}}%
        #1%
      }
    \fi
  \else
    \newcommand\tablemakefntext[1]{%
      \noindent\makebox[1.8em][r]{%
        \mbox{\textsuperscript{\normalfont%
          \ifFootnoteBackref@numberlinked%
            \hyperlink{tfn:m\theHfootnote}{\@thefnmark}\ %
          \else%
            \@thefnmark%
          \fi%
      }}}%
      \hyperlink{tfn:m\theHfootnote}{\FootnoteBackref@symbol}%
      #1%
    }
  \fi
}{%
  \newcommand\tablemakefntext[1]{%
    \noindent\makebox[1.8em][r]{%
      \mbox{\textsuperscript{\normalfont%
        \@thefnmark%
    }}}%
    #1%
  }
}

%    \end{macrocode}
%
% \end{macro}
%
% \pagebreak
%
% \begin{macro}{\swtablemakefntext}
%    \begin{macrocode}
\newcommand\swtablemakefntext[1]{\noindent\ #1}

%    \end{macrocode}
% \end{macro}
%
% \begin{macro}{\tfn@footnotemarks}
% |\tfn@footnotemarks| uses the |\footnotemark| command to place a foot note mark
% and decreases the |\tfn@footnotetablecount| (i.\,e.~increases the negative value)
% of the number of footnote(marks) in the table. If the \xpackage{footnotebackref}
% package is used, it is also necessary to create an according |\hypertarget| for
% the back reference.
%
%    \begin{macrocode}
\newcommand{\tfn@footnotemarks}[1][]{%
  \hbox{%
    \ifthenelse{\isempty{#1}}%
      {\footnotemark%
       \ltx@ifpackageloaded{footnotebackref}%
         {\raisebox{\ht\strutbox}{\hypertarget{tfn:m\theHfootnote}{}}}%
         {% footnotebackref not loaded
         }%
      }%
      {\ltx@ifpackageloaded{hyperref}%
        {\tfn@hyperfootnotes{%
            \hyperlink{tfn:#1}{\footnotemark[#1]}%
          }{\footnotemark[#1]}}%
        {\footnotemark[#1]}%
       \ltx@ifpackageloaded{footnotebackref}%
        {\raisebox{\ht\strutbox}{\hypertarget{tfn:m#1}{}}}{}%
      }%
   }%
  \@tempcnta=\tfn@footnotetablecount\relax%
  \advance\@tempcnta by -1\relax%
  \xdef\tfn@footnotetablecount{\the\@tempcnta}%
}

%    \end{macrocode}
% \end{macro}
%
% \begin{macro}{\tfn@footmiscparawarn}
% When the \xpackage{footmisc} package is used with option |para|,
% we need to issue a warning. (Any idea how to resolve this issue?)\\
% It must be given inside of it's own command, because |\ifFN@para| is not
% defined when \xpackage{footmisc} has not been loaded, and then the
% |\fi| would be a |\fi| without |\if| (and \TeX{} does not like single
% |\fi|s).
%
%    \begin{macrocode}
\newcommand{\tfn@footmiscparawarn}{%
  \ifFN@para%
    \PackageWarning{tablefootnote}{%
      Package footmisc with option para used.\MessageBreak%
      Unfortunately package tablefootnote is not yet\MessageBreak%
      able to regard this in sidewaystables.\MessageBreak%
      Here the footnotes are printed one below\MessageBreak%
      the other. Sorry!\MessageBreak%
     }%
  \fi%
  }

%    \end{macrocode}
% \end{macro}
%
% \begin{macro}{\tfn@updateHyper}
% |\tfn@updateHyper| updates some \xpackage{hyperref} internals.
%
%    \begin{macrocode}
\newcommand{\tfn@updateHyper}{%
  % The code of this command is partly from the
  % hyperref package by Heiko Oberdiek, 2011/10/01, v6.82j.
  % Newer versions of that package are available at CTAN.org.
  \global\let\Hy@saved@currentHref\@currentHref%
  \tfn@hyperfootnotes{\hyper@makecurrent{Hfootnote}}{}%
  \global\let\Hy@footnote@currentHref\@currentHref%
  \global\let\@currentHref\Hy@saved@currentHref%
  }

%    \end{macrocode}
% \end{macro}
%
% \begin{macro}{\tfn@footnotetext}
% |\tfn@footnotetext| contains the |\footnotetext[...]{...}| command and handles
% the numbering. |\tfn@footnotetext| will be executed after the table.
% The foot note counter has been increased automatically with each |\footnotemark|,
% therefore it must be reset to its value before the table began.
%
%    \begin{macrocode}
\newcommand{\tfn@footnotetext}[2][]{%
  \ifx\tfn@footnotetableprint\tfn@footnotezero%
    % This is the first footnotetext after the (sideways)table.
    \addtocounter{footnote}{\tfn@footnotetablecount}%
%    \end{macrocode}
%
% When the \xpackage{hyperref} package is used, also its |Hfootnote| counter must
% be reset (when the footnotes are hyperlinked).
%
%    \begin{macrocode}
    \ltx@ifpackageloaded{hyperref}{%
      \tfn@hyperfootnotes{%
        \addtocounter{Hfootnote}{\tfn@footnotetablecount}}{}%
      \tfn@updateHyper%
     }{}%
  \fi%
%    \end{macrocode}
%
% For issuing the text for the next footnote the footnote-counter
% (and |Hfootnote| counter in case of links) are increased by one.
%
%    \begin{macrocode}
  \addtocounter{footnote}{+1}%
  \ltx@ifpackageloaded{hyperref}{%
    \tfn@hyperfootnotes{\addtocounter{Hfootnote}{+1}}{}%
%    \end{macrocode}
%
% The \xpackage{hyperref} package was not expecting this kind of change of the
% |(H)footnote| numbers, therefore some code-lines are missing here which can be
% found at other places in that same package and are called by this command.
%
%    \begin{macrocode}
    \tfn@updateHyper%
%    \end{macrocode}
%
% When \xpackage{hyperref} is not loaded, nothing is to be done here.
%
%    \begin{macrocode}
   }{}%
%    \end{macrocode}
%
% The |\footnotetext[...]{...}| places the foot note text as usual.
%
%    \begin{macrocode}
  \ifx\tfn@swt\tfn@footnotezero%
    \ifthenelse{\isempty{#1}}%
      {\footnotetext{#2}}%
      {\ltx@ifpackageloaded{hyperref}%
         {\tfn@hyperfootnotes{%
             \footnotetext[#1]{%
               \raisebox{\ht\strutbox}{\hypertarget{tfn:#1}{}}#2}%
           }{\footnotetext[#1]{#2}%
           }%
         }{\footnotetext[#1]{#2}}%
      }%
  \else%
%    \end{macrocode}
%
% If there is more than one tablefootnote in the |sidewaystable|, \\
% the |\tfn@footmiscparawarn| shall be given,
%
%    \begin{macrocode}
    \ifx\tfn@footnotetableprint\tfn@footnotezero%
      \gdef\tfn@fmpw{\tfn@footmiscparawarn}%
    \else%
      \ltx@ifpackageloaded{footmisc}{\tfn@fmpw}{\relax}%
%    \end{macrocode}
%
% but we do not want to repeat the message for a third(, forth,\ldots)
% tablefootnote in the |sidewaystable|.
%
%    \begin{macrocode}
      \gdef\tfn@fmpw{\relax}%
    \fi%
%    \end{macrocode}
%
% |\@makefnmark| is \\
% |\hbox {\@textsuperscript {\normalfont \@thefnmark }}|, \\
% but according to |\show\@thefnmark| here |\@thefnmark| is
% |{\protect \itshape}| instead of |\thefootnote| (Why?!).\\
% If the \xpackage{footnotebackref} package is used,
% back references must be created for the table footnotes, too.
%
%    \begin{macrocode}
    \ltx@ifpackageloaded{footnotebackref}{%
      \LetLtxMacro{\tfncurrentmft}{\@makefntext}%
      \LetLtxMacro{\@makefntext}{\swtablemakefntext}%
      \ifthenelse{\isempty{#1}}%
        {\footnotetext{%
           \raisebox{\ht\strutbox}{\hypertarget{tfn:\theHfootnote}{}}%
           \hbox{\@textsuperscript{\normalfont%
              \hyperlink{tfn:m\theHfootnote}{\thefootnote}}}\thinspace%
         \ifx\FootnoteBackref@symbol\empty%
         \else\hyperlink{tfn:m\theHfootnote}{\FootnoteBackref@symbol}%
         \fi%
         #2}%
       }{\edef\tfn@tfn{\arabic{footnote}}%
         \edef\tfn@thn{\arabic{Hfootnote}}%
         \setcounter{footnote}{#1}%
         \setcounter{Hfootnote}{#1}%
         \tfn@updateHyper%
         \footnotetext{%
           \raisebox{\ht\strutbox}{\hypertarget{tfn:#1}{}}%
           \hbox{\@textsuperscript{\normalfont%
                  \hyperlink{tfn:m#1}{#1}}}%
           \thinspace%
           \ifx\FootnoteBackref@symbol\empty%
           \else%
             \raisebox{\ht\strutbox}{%
               \hyperlink{tfn:m#1}{\FootnoteBackref@symbol}}%
           \fi%
           #2%
          }%
         \setcounter{footnote}{\tfn@tfn}%
         \setcounter{Hfootnote}{\tfn@thn}%
         \tfn@updateHyper%
       }%
      \LetLtxMacro{\@makefntext}{\tfncurrentmft}%
     }{% footnotebackref not loaded
       \ifthenelse{\isempty{#1}}%
         {\footnotetext{%
            \hbox{\@textsuperscript{\normalfont\thefootnote}}%
            \ltx@ifpackageloaded{hyperref}%
              {\tfn@hyperfootnotes{%
                 \raisebox{\ht\strutbox}{%
                   \hypertarget{tfn:\theHfootnote}{}}}{}%
              }{}%
            \thinspace%
            #2}%
        }{%
%    \end{macrocode}
%
% \pagebreak
%
%    \begin{macrocode}
          \edef\tfn@tfn{\arabic{footnote}}%
          \ltx@ifpackageloaded{hyperref}{%
            \tfn@hyperfootnotes{\xdef\tfn@thn{\arabic{Hfootnote}}}{}%
           }{}%
          \setcounter{footnote}{#1}
          \ltx@ifpackageloaded{hyperref}{%
            \tfn@hyperfootnotes{\setcounter{Hfootnote}{#1}}{}%
            \tfn@updateHyper%
           }{}%
          \footnotetext{%
            \hbox{\@textsuperscript{\normalfont #1}}%
            \ltx@ifpackageloaded{hyperref}{%
              \tfn@hyperfootnotes{\raisebox{\ht\strutbox}{%
                                         \hypertarget{tfn:#1}{}}}{}%
             }{}%
            \thinspace%
            #2}%
          \setcounter{footnote}{\tfn@tfn}%
          \ltx@ifpackageloaded{hyperref}{%
            \tfn@hyperfootnotes{\setcounter{Hfootnote}{\tfn@thn}}{}%
            \tfn@updateHyper%
           }{}%
        }%
     }%
  \fi%
%    \end{macrocode}
%
% For further footnotes in the table the \\
% |\ifx\tfn@footnotetableprint\tfn@footnotezero|\ldots|\fi| \\
% from the beginning of this |\tfn@footnotetext| macro must not be repeated,
% therefore we deactivate this by defining |\tfn@footnotetableprint| to
% something other than |0| (here |1| is used).
%
%    \begin{macrocode}
  \gdef\tfn@footnotetableprint{1}%
%    \end{macrocode}
%
% |\tfn@footnotetablecount| contains the negative number of footnote texts
% to be printed, therefore it is increase by one (i.\,e.~the negative number
% decreases, i.\,e.~$-3$ becomes~$-2$).
%
%    \begin{macrocode}
  \@tempcnta=\tfn@footnotetablecount\relax%
  \advance\@tempcnta by +1\relax%
  \xdef\tfn@footnotetablecount{\the\@tempcnta}%
%    \end{macrocode}
%
% When it is zero, all footnote texts have been placed, and\\
% |\tfn@footnotetableprint| can be reset to zero and\\
% |\tfn@tablefootnoteprintout| can be cleared.
%
%    \begin{macrocode}
  \ifx\tfn@footnotetablecount\tfn@footnotezero% all footnotes placed
    \gdef\tfn@footnotetableprint{0}%
    \global\let\tfn@tablefootnoteprintout\relax% brute force
 %\else there is at least one \footnotemark which needs \footnotetext
  \fi
}

%    \end{macrocode}
% \end{macro}
%
% \begin{macro}{\tfn@tablefootnoteprintout}
% When something shall be cleared or redefined, it is a good idea
% to define it. It~will be later filled with the foot note text to be printed.
%
%    \begin{macrocode}
\newcommand{\tfn@tablefootnoteprintout}{\relax}

%    \end{macrocode}
% \end{macro}
%
% \pagebreak
%
% \begin{macro}{\tablefootnote}
% |\tablefootnote[...]{...}| is the command to be used by the user instead of
% |\footnote[...]{...}| in the (sideways)table. It places a foot note mark via
% |\tfn@footnotemarks| and appends a |\tfn@footnotetext{#1}| to the\\
% |\tfn@tablefootnoteprintout| macro, and remembers that table foot notes
% have been used.
%
%    \begin{macrocode}
\newcommand{\tablefootnote}[2][]{%
  \tfn@footnotemarks[#1]%
  \ifthenelse{\isempty{#1}}{\relax}{%
    \ltx@GlobalPrependToMacro{\tfn@tablefootnoteprintout}{%
      \addtocounter{footnote}{+1}%
      \ltx@ifpackageloaded{hyperref}{%
        \tfn@hyperfootnotes{\addtocounter{Hfootnote}{+1}}{}%
        \tfn@updateHyper%
       }{}%
     }%
   }%
  \ltx@GlobalAppendToMacro{\tfn@tablefootnoteprintout}{%
    \tfn@footnotetext[#1]{#2}%
    \ifthenelse{\isempty{#1}}{\relax}{%
      \addtocounter{footnote}{-1}%
      \ltx@ifpackageloaded{hyperref}{%
        \tfn@hyperfootnotes{\addtocounter{Hfootnote}{-1}}{}%
        \tfn@updateHyper%
       }{}%
     }%
   }%
  \gdef\tfn@fnt{1}%
  }

%    \end{macrocode}
% \end{macro}
%
% \begin{macro}{\tfn@swtbox}
% For |sidewaystable|s the |\tfn@tablefootnoteprintout| is placed in a \\
% |\parbox|, which can be placed at the end of the |sidewaystable|:
%
%    \begin{macrocode}
\newcommand{\tfn@swtbox}{%
\ifvoid \footins%
\vskip 2\baselineskip%
\else%
\vskip \skip\footins%
\fi%
\parbox[b][][t]{\linewidth}{\raggedright\tfn@tablefootnoteprintout}%
}

%    \end{macrocode}
% \end{macro}
%
% |\tfn@tablefootnoteprintout| shall be executed after the end of the table environment.
% The \xpackage{etoolbox} package provides very easy syntax for this:
%
%    \begin{macrocode}
\IfFileExists{etoolbox.sty}{%
  \RequirePackage{etoolbox}[2011/01/03]%   v2.1
  \AfterEndEnvironment{table}{%
    \ltx@ifpackageloaded{footnotebackref}{%
      \LetLtxMacro{\tfncurrentmft}{\@makefntext}%
      \LetLtxMacro{\@makefntext}{\tablemakefntext}%
       \tfn@tablefootnoteprintout%
      \LetLtxMacro{\@makefntext}{\tfncurrentmft}%
     }{% footnotebackref not loaded
      \tfn@tablefootnoteprintout%
     }%
      \gdef\tfn@fnt{0}%
   }%
%    \end{macrocode}
%
% When the \xpackage{rotating} package has been loaded, we need to detect
% whether the |\tablefootnote| is placed inside of a |sidewaystable|.
%
%    \begin{macrocode}
  \ltx@ifpackageloaded{rotating}{%
    \AtBeginEnvironment{sidewaystable}{%
      \gdef\tfn@swt{1}%
    }
%    \end{macrocode}
%
% At the end of |sidewaystable| environments |\tfn@swtbox| shall be placed
% (if~there was any |\tablefootnote| in that |sidewaystable| at all).
% And |\tfn@swt| as well as |\tfn@fnt| must be reset to |0|.
%
%    \begin{macrocode}
    \AtEndEnvironment{sidewaystable}{%
      \ifx\tfn@fnt\tfn@footnotezero% \relax
      \else%
        \ltx@ifpackageloaded{footnotebackref}{%
          \LetLtxMacro{\tfncurrentmft}{\@makefntext}%
          \LetLtxMacro{\@makefntext}{\tablemakefntext}%
           \tfn@swtbox%
          \LetLtxMacro{\@makefntext}{\tfncurrentmft}%
         }{% footnotebackref not loaded
          \tfn@swtbox%
          }%
      \fi%
      \gdef\tfn@swt{0}%
      \gdef\tfn@fnt{0}%
     }%
   }{}% else \relax
}{%
%    \end{macrocode}
%
% When the \xpackage{etoolbox} package is not available (e.\,g. because of missing
% \eTeX{}) we~do a workaround:
%
%    \begin{macrocode}
  \PackageWarning{tablefootnote}{%
    Package etoolbox not available, therefore\MessageBreak%
    substituting commands \string\AtEndEnvironment\space %
    and\MessageBreak%
    \string\AfterEndEnvironment\space for tablefootnotes%
    \MessageBreak%
   }%
  \@ifpackagelater{letltxmacro}{2010/09/02}%
   {% >= 2010/09/02, OK
    \GlobalLetLtxMacro{\tfnorigtable}{\table}
    \GlobalLetLtxMacro{\tfnendorigtable}{\endtable}
%    \end{macrocode}
%
% |\GlobalLetLtxMacro| was introduced in version 2010/09/02 v1.4 of the
% \xpackage{letltxmacro} package.
%
%    \begin{macrocode}
   }{% else
     \PackageError{tablefootnote}{%
       Outdated version of letltxmacro package used%
      }{Version 2010/09/02 v1.4 or newer needed but not found!%
        \MessageBreak%
        Update the used letltxmacro package.}
        \LetLtxMacro{\tfnorigtable}{\table}
        \LetLtxMacro{\tfnendorigtable}{\endtable}
    }%
%    \end{macrocode}
%
% \pagebreak
%
%    \begin{macrocode}
  \ltx@ifpackageloaded{float}{%
    \@ifpackagelater{float}{2001/11/08}%   v1.3d
     {% >= 2001/11/08, OK
     }{% else
       \PackageWarningNoLine{tablefootnote}{%
         It is required version\MessageBreak
         2001/11/08 v1.3d (or later) of package float,\MessageBreak
         but only version\MessageBreak
        `\csname ver@float.sty\endcsname'\MessageBreak
        is available}%
     }%
   }{\relax}
  \renewenvironment{table}{%
    \ltx@ifpackageloaded{float}{%
%    \end{macrocode}
%
% When \xpackage{etoolbox.sty} was not found, and the \xpackage{float} package
% has been loaded, and the float specifier is \texttt{H}, then |\endtable| was
% let to |\float@endH| by the \xpackage{float} package (line~79 of
% \xpackage{float.sty}, 2001/11/08 v1.3d:\\
% |\expandafter\let\csname end#1\endcsname\float@endH|), i.\,e.~appending\\
% |\tfn@tablefootnoteprintout \gdef\tfn@fnt{0}| to |\endtable| is useless.
% Instead of this, it must be appended to |\float@endH|.\\
% (Why is this not necessary when \xpackage{etoolbox} is used?
% That is a good question, answers are welcome.)
%
%    \begin{macrocode}
      \ltx@LocalAppendToMacro{\float@endH}{%
        \ltx@ifpackageloaded{footnotebackref}{%
          \LetLtxMacro{\tfncurrentmft}{\@makefntext}%
          \LetLtxMacro{\@makefntext}{\tablemakefntext}%
           \tfn@tablefootnoteprintout%
          \LetLtxMacro{\@makefntext}{\tfncurrentmft}%
         }{% footnotebackref not loaded
          \tfn@tablefootnoteprintout%
         }%
          \gdef\tfn@fnt{0}%
       }%
     }{\relax}
    \tfnorigtable%
   }{%
     \tfnendorigtable%
     \ltx@ifpackageloaded{footnotebackref}{%
       \LetLtxMacro{\tfncurrentmft}{\@makefntext}%
       \LetLtxMacro{\@makefntext}{\tablemakefntext}%
        \tfn@tablefootnoteprintout%
       \LetLtxMacro{\@makefntext}{\tfncurrentmft}%
      }{% footnotebackref not loaded
       \tfn@tablefootnoteprintout%
      }%
     \gdef\tfn@fnt{0}%
    }%
%    \end{macrocode}
%
% \pagebreak
%
% The |sidewaystable|-environment must be treated separately:
%
%    \begin{macrocode}
  \ltx@ifpackageloaded{rotating}{%
    \let\tfnorigsidewaystable\sidewaystable
    \let\tfnendorigsidewaystable\endsidewaystable
    \renewenvironment{sidewaystable}{%
      \gdef\tfn@swt{1}%
      \tfnorigsidewaystable%
    }{%
      \ifx\tfn@fnt\tfn@footnotezero% \relax
      \else%
        \ltx@ifpackageloaded{footnotebackref}{%
          \LetLtxMacro{\tfncurrentmft}{\@makefntext}%
          \LetLtxMacro{\@makefntext}{\tablemakefntext}%
           \tfn@swtbox%
          \LetLtxMacro{\@makefntext}{\tfncurrentmft}%
         }{% footnotebackref not loaded
          \tfn@swtbox%
         }%
      \fi%
      \gdef\tfn@swt{0}%
      \gdef\tfn@fnt{0}%
      \tfnendorigsidewaystable%
     }%
   }{}%
}

%    \end{macrocode}
%
% \DescribeMacro{\AtBeginDocument}
% When the \xpackage{rotating} (for |sidewaystable|s) or the
% \xpackage{footnotebackref} (for back references for the footnotes)
% package is used, it must be loaded before the \xpackage{tablefootnote} package.
% Using both packages is also possible, in which case both must be loaded before
% the \xpackage{tablefootnote} package. If any one of those packages is loaded
% |\AtBeginDocument| after this check, the according error message will not be
% given, but packages really should be loaded before |\AtBeginDocument|.
%
%    \begin{macrocode}
\AtBeginDocument{%
  \ltx@ifpackageloaded{footnotebackref}{%
    \ifx\tfn@fnbr\tfn@footnotezero%
      \PackageError{tablefootnote}{%
        Package footnotebackref loaded after tablefootnote}{%
        When using both the footnotebackref and the tablefootnote %
        package,\MessageBreak%
        the footnotebackref package MUST be loaded before the %
        tablefootnote package!\MessageBreak%
       }%
    \fi%
   }{% else \relax
   }
  \ltx@ifpackageloaded{rotating}{%
    \ifx\tfn@rotating\tfn@footnotezero%
      \PackageError{tablefootnote}{%
        Package rotating loaded after tablefootnote}{%
        When using both the rotating and the tablefootnote package,%
        \MessageBreak%
        the rotating package MUST be loaded before the tablefootnote%
        \space package!\MessageBreak%
       }%
    \fi%
   }{% else \relax
   }
%    \end{macrocode}
%
% \pagebreak
%
% When the document is compiled with \LuaLaTeX{}, hyperlinks in rotated content
% will be misplaced, regardless of the use of the \xpackage{tablefootnote} package
% (or of tables or footnotes). The text to be printed will be OK.
% This problem of \LuaLaTeX{} (not \xpackage{tablefootnote}) was found and reported
% by \textsc{Arno Trautmann}, 2011/11/15 -- thanks!
%
%    \begin{macrocode}
  \ltx@ifpackageloaded{hyperref}{%
    \tfn@hyperfootnotes{
      \ifluatex%
        \PackageWarningNoLine{tablefootnote}{%
          LuaTeX and hyperref package used:\MessageBreak%
          Hyperreferences in rotated content\MessageBreak%
          will be misplaced\MessageBreak%
          regardless of tablefootnote package.\MessageBreak%
          The text to be printed will be OK%
         }%
      \fi%
     }{% else \relax
     }%
   }{% else \relax
   }%
%    \end{macrocode}
%
% When the \xpackage{footnotebackref} package is used but the footnotes are not
% hyperlinked, neither the back references nor the tablefootnotes will work.
%
%    \begin{macrocode}
  \ltx@ifpackageloaded{footnotebackref}{%
     \ltx@ifpackageloaded{hyperref}{%
        \tfn@hyperfootnotes{\relax}{% else
          \PackageError{tablefootnote}{%
             footnotebackref but not hyperfootnotes}{%
             The footnotebackref package is used\MessageBreak%
             but the hyperfootnotes option of the hyperref package%
             \MessageBreak%
             was set to false.\MessageBreak%
             Either enable hyperfootnotes or do not use footnotebackref.%
             \MessageBreak%
             Loading of the tablefootnote package will be aborted.%
           }%
          \endinput%
      }%
    }{\PackageError{tablefootnote}{footnotebackref but not hyperref}{%
        The footnotebackref package is used\MessageBreak%
        but not the hyperref package.\MessageBreak%
        Either load hyperref or do not use footnotebackref.%
        \MessageBreak%
        Loading of the tablefootnote package will be aborted.%
        }%
      \endinput%
    }%
  }{}
}

%    \end{macrocode}
%
%    \begin{macrocode}
%</package>
%    \end{macrocode}
%
% \pagebreak
%
% \section{Installation}
%
% \subsection{Downloads\label{ss:Downloads}}
%
% Everything is available on \CTAN{}, \url{http://www.ctan.org/},
% but may need additional packages themselves.\\
%
% \DescribeMacro{tablefootnote.dtx}
% For unpacking the |tablefootnote.dtx| file and constructing the documentation
% it is required:
% \begin{description}
% \item[-] \TeX{} Format \LaTeXe{}: \url{http://www.CTAN.org/}
%
% \item[-] document class \xpackage{ltxdoc}, 2007/11/11, v2.0u,
%           \url{http://www.ctan.org/pkg/ltxdoc}
%
% \item[-] package \xpackage{holtxdoc}, 2012/03/21, v0.24,
%           \url{http://www.ctan.org/pkg/holtxdoc}
%
% \item[-] package \xpackage{hypdoc}, 2011/08/19, v1.11,
%           \url{http://www.ctan.org/pkg/hypdoc}
% \end{description}
%
% \DescribeMacro{tablefootnote.sty}
% The |tablefootnote.sty| for \LaTeXe{} (i.\,e.~each document using
% the \xpackage{tablefootnote} package) requires:
% \begin{description}
% \item[-] \TeX{} Format \LaTeXe{}, \url{http://www.CTAN.org/}
%
% \item[-] package \xpackage{letltxmacro}, 2010/09/02, v1.4,
%           \url{http://www.ctan.org/pkg/letltxmacro}.
%
% \item[-] package \xpackage{ltxcmds}, 2011/11/09, v1.22,
%           \url{http://www.ctan.org/pkg/ltxcmds}
% \end{description}
%
% When the \xpackage{hyperref} package is used, also the \xpackage{ifluatex} package
% is needed, but it is already loaded automatically by the \xpackage{hyperref}
% package.\\
% When the \xpackage{etoolbox} package is available, it is used:
% \begin{description}
% \item[-] package \xpackage{etoolbox}, 2011/01/03, v2.1,
%           \url{http://www.ctan.org/pkg/etoolbox}.
% \end{description}
%
% \DescribeMacro{tablefootnote-example.tex}
% The |tablefootnote-example.tex| requires the same files as all
% documents using the \xpackage{tablefootnote} package and additionally:
% \begin{description}
% \item[-] class \xpackage{article}, 2007/10/19, v1.4h, from \xpackage{classes.dtx}:\\
%           \CTAN{macros/latex/base/classes.dtx}
%
% \item[-] package \xpackage{rotating}, 2009/03/28, v2.16a,
%           \url{http://www.ctan.org/pkg/rotating}
%
% \item[-] package \xpackage{float}, 2001/11/08, v1.3d,
%           \url{http://www.ctan.org/pkg/float}
%
% \item[-] package \xpackage{placeins}, 2005/04/18, v2.2,
%           \url{http://www.ctan.org/pkg/placeins}\\
%           for |\FloatBarrier|
%
% \item[-] package \xpackage{hyperref}, 2012/11/06, v6.83m,
%           \url{http://www.ctan.org/pkg/hyperref}
%
% \item[-] package \xpackage{tablefootnote}, 2014/01/26, v1.1c,\\
%           \url{http://www.ctan.org/pkg/tablefootnote}\\
%   (Well, it is the example file for this package, and because you
%    are reading the documentation for the \xpackage{tablefootnote} package,
%    it can be assumed that you already have some version of it~--
%    is it the current one?)
% \end{description}
%
% \DescribeMacro{Oberdiek}
% \DescribeMacro{holtxdoc}
% \DescribeMacro{ifluatex}
% \DescribeMacro{letltxmacro}
% \DescribeMacro{ltxcmds}
% All packages of \textsc{Heiko Oberdiek}'s bundle `oberdiek'
% (especially \xpackage{holtxdoc}, \xpackage{ifluatex}, \xpackage{letltxmacro},
% and \xpackage{ltxcmds}) are also available in a TDS compliant ZIP archive:\\
% \CTAN{install/macros/latex/contrib/oberdiek.tds.zip}.\\
% It is probably best to download and use this, because the packages in there
% are quite probably both recent and compatible among themselves.\\
%
% \pagebreak
%
% \DescribeMacro{hyperref}
% \noindent \xpackage{hyperref} is not included in that bundle and needs to be
% downloaded separately,\\
% \url{http://www.ctan.org/install/macros/latex/contrib/hyperref.tds.zip}.\\
%
% \DescribeMacro{M\"{u}nch}
% A hyperlinked list of my (other) packages can be found at
% \url{http://www.ctan.org/author/muench-hm}.\\
%
% \subsection{Package, unpacking TDS}
% \paragraph{Package.} This package is available on \CTAN{}
% \begin{description}
% \item[\CTAN{macros/latex/contrib/tablefootnote/tablefootnote.dtx}]\hspace*{0.1cm} \\
%       The source file.
% \item[\CTAN{macros/latex/contrib/tablefootnote/tablefootnote.pdf}]\hspace*{0.1cm} \\
%       The documentation.
% \item[\CTAN{macros/latex/contrib/tablefootnote/tablefootnote-example.pdf}]\hspace*{0.1cm} \\
%       The compiled example file, as it should look like.
% \item[\CTAN{macros/latex/contrib/tablefootnote/README}]\hspace*{0.1cm} \\
%       The README file.
% \end{description}
% There is also a tablefootnote.tds.zip available:
% \begin{description}
% \item[\CTAN{install/macros/latex/contrib/tablefootnote.tds.zip}]\hspace*{0.1cm} \\
%       Everything in \xfile{TDS} compliant, compiled format.
% \end{description}
% which additionally contains\\
% \begin{tabular}{ll}
% tablefootnote.ins & The installation file.\\
% tablefootnote.drv & The driver to generate the documentation.\\
% tablefootnote.sty & The \xext{sty}le file.\\
% tablefootnote-example.tex & The example file.
% \end{tabular}
%
% \bigskip
%
% \noindent For required other packages, please see the preceding subsection.
%
% \paragraph{Unpacking.} The \xfile{.dtx} file is a self-extracting
% \docstrip{} archive. The files are extracted by running the
% \xext{dtx} through \plainTeX{}:
% \begin{quote}
%   \verb|tex tablefootnote.dtx|
% \end{quote}
%
% \noindent About generating the documentation see paragraph~\ref{GenDoc} below.\\
%
% \paragraph{TDS.} Now the different files must be moved into
% the different directories in your installation TDS tree
% (also known as \xfile{texmf} tree):
% \begin{quote}
% \def\t{^^A
% \begin{tabular}{@{}>{\ttfamily}l@{ $\rightarrow$ }>{\ttfamily}l@{}}
%   tablefootnote.sty & tex/latex/tablefootnote/tablefootnote.sty\\
%   tablefootnote.pdf & doc/latex/tablefootnote/tablefootnote.pdf\\
%   tablefootnote-example.tex & doc/latex/tablefootnote/tablefootnote-example.tex\\
%   tablefootnote-example.pdf & doc/latex/tablefootnote/tablefootnote-example.pdf\\
%   tablefootnote.dtx & source/latex/tablefootnote/tablefootnote.dtx\\
% \end{tabular}^^A
% }^^A
% \sbox0{\t}^^A
% \ifdim\wd0>\linewidth
%   \begingroup
%     \advance\linewidth by\leftmargin
%     \advance\linewidth by\rightmargin
%   \edef\x{\endgroup
%     \def\noexpand\lw{\the\linewidth}^^A
%   }\x
%   \def\lwbox{^^A
%     \leavevmode
%     \hbox to \linewidth{^^A
%       \kern-\leftmargin\relax
%       \hss
%       \usebox0
%       \hss
%       \kern-\rightmargin\relax
%     }^^A
%   }^^A
%   \ifdim\wd0>\lw
%     \sbox0{\small\t}^^A
%     \ifdim\wd0>\linewidth
%       \ifdim\wd0>\lw
%         \sbox0{\footnotesize\t}^^A
%         \ifdim\wd0>\linewidth
%           \ifdim\wd0>\lw
%             \sbox0{\scriptsize\t}^^A
%             \ifdim\wd0>\linewidth
%               \ifdim\wd0>\lw
%                 \sbox0{\tiny\t}^^A
%                 \ifdim\wd0>\linewidth
%                   \lwbox
%                 \else
%                   \usebox0
%                 \fi
%               \else
%                 \lwbox
%               \fi
%             \else
%               \usebox0
%             \fi
%           \else
%             \lwbox
%           \fi
%         \else
%           \usebox0
%         \fi
%       \else
%         \lwbox
%       \fi
%     \else
%       \usebox0
%     \fi
%   \else
%     \lwbox
%   \fi
% \else
%   \usebox0
% \fi
% \end{quote}
% If you have a \xfile{docstrip.cfg} that configures and enables \docstrip{}'s
% \xfile{TDS} installing feature, then some files can already be in the right
% place, see the documentation of \docstrip{}.
%
% \subsection{Refresh file name databases}
%
% If your \TeX{}~distribution (\TeX live, \mikTeX{}, \dots{}) relies on file name
% databases, you must refresh these. For example, \teTeX{} users run
% \verb|texhash| or \verb|mktexlsr|.
%
% \subsection{Some details for the interested}
%
% \paragraph{Unpacking with \LaTeX{}.}
% The \xfile{.dtx} chooses its action depending on the format:
% \begin{description}
% \item[\plainTeX:] Run \docstrip{} and extract the files.
% \item[\LaTeX:] Generate the documentation.
% \end{description}
% If you insist on using \LaTeX{} for \docstrip{} (really,
% \docstrip{} does not need \LaTeX{}), then inform the autodetect routine
% about your intention:
% \begin{quote}
%   \verb|latex \let\install=y\input{tablefootnote.dtx}|
% \end{quote}
% Do not forget to quote the argument according to the demands
% of your shell.
%
% \paragraph{Generating the documentation.\label{GenDoc}}
% You can use both the \xfile{.dtx} or the \xfile{.drv} to generate
% the documentation. The process can be configured by a
% configuration file \xfile{ltxdoc.cfg}. For instance, put the following
% line into this file, if you want to have A4 as paper format:
% \begin{quote}
%   \verb|\PassOptionsToClass{a4paper}{article}|
% \end{quote}
%
% \noindent An example follows how to generate the
% documentation with \pdfLaTeX{}:
%
% \begin{quote}
%\begin{verbatim}
%pdflatex tablefootnote.dtx
%makeindex -s gind.ist tablefootnote.idx
%pdflatex tablefootnote.dtx
%makeindex -s gind.ist tablefootnote.idx
%pdflatex tablefootnote.dtx
%\end{verbatim}
% \end{quote}
%
% \subsection{Compiling the example}
%
% The example file, \textsf{tablefootnote-example.tex}, can be compiled via\\
% \indent |latex tablefootnote-example.tex|,\\
% \indent |lualatex tablefootnote-example.tex|,\\
% or (recommended)\\
% \indent |pdflatex thumbs-example.tex|\\
% and will need at least two compiler runs to get everything right.
%
% \bigskip
%
% \section{Acknowledgements}
%
% I would like to thank \textsc{Heiko Oberdiek} for providing
% the \xpackage{hyperref}, \xpackage{ifluatex}, \xpackage{letltxmacro},
% \xpackage{ltxcmds}, as well as a~lot~(!) of other useful packages
% (from which I also got everything I know about creating a file in
% \xext{dtx} format, OK, say it: copying),
% and for information about the \pdfTeX{}-bug,
% \textsc{Arno Trautmann} for reporting the bug of misplaced hyperreferences in
% |sidewaystable|s using \LuaTeX{}, independently of the \xpackage{tablefootnote}
% package or footnotes at all,
% \textsc{Thomas~V.} for reporting the bug of not handling the case of
% \xpackage{hyperref} option |hyperfootnotes=false|,
% \textsc{Sverre Stausland Johnsen} for reporting the bug in case of \emph{not}
% using \xpackage{hyperref},
% and the \Newsgroup{comp.text.tex} and \Newsgroup{de.comp.text.tex} newsgroups
% as well as everybody at \url{http://tex.stackexchange.com}
% for their help in all things \TeX{}.
%
% \pagebreak
%
% \phantomsection
% \begin{History}\label{History}
%   \begin{Version}{2011/10/26 v1.0a, tabfootn}
%     \item Upload to \CTAN{macros/latex/contrib/tablefootnote/}.
%   \end{Version}
%   \begin{Version}{2011/10/29 v1.0b}
%     \item Renamed to \xpackage{tablefootnote}.
%     \item Added support for the |sidewaystable|-environment of the
%             \xpackage{rotating} package.
%     \item Diverse changes in documentation and README.
%   \end{Version}
%   \begin{Version}{2011/11/06 v1.0c}
%     \item Replaced |^{\text{\thefootnote}}| by |{}^{\text{\textup{\thefootnote}}}|.
%     \item Bug fix: When \xpackage{etoolbox.sty} was not found, and the \xpackage{float}
%             package had been loaded, and the float specifier was \texttt{H},
%             then the |\tablefootnote|s were not printed for that table. Fixed.
%   \end{Version}
%   \begin{Version}{2011/11/19 v1.0d}
%     \item Replaced |{}^{\text{\textup{\thefootnote}}}| by
%             |\textsuperscript{\thefootnote}|, therefore the \xpackage{amstext}
%             package is no longer required by the \xpackage{tablefootnote} package.
%     \item \xpackage{lscape} and \xpackage{pdflscape} package now work with
%             \xpackage{tablefootnote} (except for using a \texttt{sidewaystable} on
%             a \texttt{landscape} page).
%     \item Added information about footnotes in \texttt{sidewaystable} being printed
%             one below the other, even if \xpackage{footmisc} package with option
%             \texttt{para} is used.
%     \item Redefined |\tfn@swtbox| from
%  \begin{verbatim}
%  \vspace{0.8cm}%
%  \begin{spacing}{0.1}%
%  \subcaptionbox*{}[\linewidth][l]{\tfn@tablefootnoteprintout}%
%  \end{spacing}%
%  \end{verbatim}
%             to
%  \begin{verbatim}
%  \vspace{2\baselineskip}%
%  \parbox[b][][t]{\linewidth}{\raggedright\tfn@tablefootnoteprintout}%
%  \end{verbatim}
%             , therefore the \xpackage{setspace} and \xpackage{subcaption} packages
%             are no longer required by the \xpackage{tablefootnote} package.
%     \item No longer uses |\numexpr|, therefore also usable without \eTeX{}.
%     \item Replaced |\textsuperscript{\thefootnote}| by
%             |\hbox {\@textsuperscript {\normalfont \thefootnote }}|,
%             which is next to the original definition of |\@makefnmark|
%             (|\hbox {\@textsuperscript {\normalfont \@thefnmark }}|,
%             but according to |\show\@thefnmark| here |\@thefnmark| is
%             |{\protect \itshape}| instead of |\thefootnote|).
%     \item Added an error message, when \LuaTeX{}, \xpackage{rotating} package, and
%            \xpackage{hyperref} package are used together. Hyperreferences in
%            |sidewaystable|s are misplaced, independently of the
%            \xpackage{tablefootnote} package or footnotes at all. (Bug reported
%            by \textsc{Arno Trautmann}, 2011/11/15 -- thanks!)
%     \item Added footnotesymbol-footnote-numbers in the example.
%   \end{Version}
%   \begin{Version}{2011/11/26 v1.0e}
%     \item Check for \xpackage{hyperref} is only done once.
%     \item Message regarding the misplacement of hyperlinks by \LuaLaTeX{}
%             changed form error to warning.
%     \item Replaced |\vspace{2\baselineskip}| by
%  \begin{verbatim}
% \ifvoid \footins%
% \vskip 2\baselineskip%
% \else%
% \vskip \skip\footins%
% \fi%
%  \end{verbatim}
%             between the |sidewaystable|s and their footnotes.
%     \item The warning about un-regarded |para|-option of \xpackage{footmoisc}
%             package is only given at the second |tablefootnote| in a |sidewaystable|,
%             not repeatedly for every |tablefootnote| in a |sidewaystable|.
%   \end{Version}
%   \begin{Version}{2012/01/01 v1.0f}
%     \item Bugfix: Wrong installation path given in the documentation, fixed.
%     \item Update of documentation, README, and \xfile{dtx} internals.
%   \end{Version}
%   \begin{Version}{2012/01/14 v1.0g}
%     \item Bugfix: There was a |relax| instead of a |\relax|, fixed.
%   \end{Version}
%   \begin{Version}{2012/07/29 v1.0h}
%     \item \xpackage{tablefootnote} is now compatible to the new
%             \xpackage{footnotebackref} package, 2012/07/01, v1.0,
%             \url{http://www.ctan.org/pkg/footnotebackref}.
%   \end{Version}
%   \begin{Version}{2014/01/08 v1.1a}
%     \item Tablefootnotes now regard optional footnote marks:
%             |\tablefootnote[| custom mark here! |]{...}|
%     \item Bug fix: the case of \xpackage{hyperref} option |hyperfootnotes=false|
%             was not handeled at all. (Bug reported by \textsc{Thomas~V.}, %
%             2014/01/02 -- thanks!)
%     \item A lot of details.
%   \end{Version}
%   \begin{Version}{2014/01/20 v1.1b}
%     \item Bug fix: Package was broken when \xpackage{hyperref} was \emph{not} used.
%            (Bug reported by \textsc{Sverre Stausland Johnsen} -- thanks!)
%   \end{Version}
%   \begin{Version}{2014/01/26 v1.1c}
%     \item The emergency bug fix of version 2014/01/20 v1.1b was replaced by using
%             the new robust macro |\tfn@hyperfootnotes|, which only \emph{internally}
%             uses |\ifHy@hyperfootnotes|. If the \xpackage{hyperref} package is not used,
%             \LaTeX{} now does not see any confusing
%             |\ifHy@hyperfootnotes|\ldots|\else|\ldots|\fi| at all.
%     \item URLs and README updated.
%   \end{Version}
% \end{History}
%
% \bigskip
%
% When you find a mistake or have a suggestion for an improvement of this package,
% please send an e-mail to the maintainer, thanks! (Please see BUG REPORTS in the README.)
%
% \bigskip
%
% \PrintIndex
%
% \Finale
\endinput|
% \end{quote}
% Do not forget to quote the argument according to the demands
% of your shell.
%
% \paragraph{Generating the documentation.\label{GenDoc}}
% You can use both the \xfile{.dtx} or the \xfile{.drv} to generate
% the documentation. The process can be configured by a
% configuration file \xfile{ltxdoc.cfg}. For instance, put the following
% line into this file, if you want to have A4 as paper format:
% \begin{quote}
%   \verb|\PassOptionsToClass{a4paper}{article}|
% \end{quote}
%
% \noindent An example follows how to generate the
% documentation with \pdfLaTeX{}:
%
% \begin{quote}
%\begin{verbatim}
%pdflatex tablefootnote.dtx
%makeindex -s gind.ist tablefootnote.idx
%pdflatex tablefootnote.dtx
%makeindex -s gind.ist tablefootnote.idx
%pdflatex tablefootnote.dtx
%\end{verbatim}
% \end{quote}
%
% \subsection{Compiling the example}
%
% The example file, \textsf{tablefootnote-example.tex}, can be compiled via\\
% \indent |latex tablefootnote-example.tex|,\\
% \indent |lualatex tablefootnote-example.tex|,\\
% or (recommended)\\
% \indent |pdflatex thumbs-example.tex|\\
% and will need at least two compiler runs to get everything right.
%
% \bigskip
%
% \section{Acknowledgements}
%
% I would like to thank \textsc{Heiko Oberdiek} for providing
% the \xpackage{hyperref}, \xpackage{ifluatex}, \xpackage{letltxmacro},
% \xpackage{ltxcmds}, as well as a~lot~(!) of other useful packages
% (from which I also got everything I know about creating a file in
% \xext{dtx} format, OK, say it: copying),
% and for information about the \pdfTeX{}-bug,
% \textsc{Arno Trautmann} for reporting the bug of misplaced hyperreferences in
% |sidewaystable|s using \LuaTeX{}, independently of the \xpackage{tablefootnote}
% package or footnotes at all,
% \textsc{Thomas~V.} for reporting the bug of not handling the case of
% \xpackage{hyperref} option |hyperfootnotes=false|,
% \textsc{Sverre Stausland Johnsen} for reporting the bug in case of \emph{not}
% using \xpackage{hyperref},
% and the \Newsgroup{comp.text.tex} and \Newsgroup{de.comp.text.tex} newsgroups
% as well as everybody at \url{http://tex.stackexchange.com}
% for their help in all things \TeX{}.
%
% \pagebreak
%
% \phantomsection
% \begin{History}\label{History}
%   \begin{Version}{2011/10/26 v1.0a, tabfootn}
%     \item Upload to \CTAN{macros/latex/contrib/tablefootnote/}.
%   \end{Version}
%   \begin{Version}{2011/10/29 v1.0b}
%     \item Renamed to \xpackage{tablefootnote}.
%     \item Added support for the |sidewaystable|-environment of the
%             \xpackage{rotating} package.
%     \item Diverse changes in documentation and README.
%   \end{Version}
%   \begin{Version}{2011/11/06 v1.0c}
%     \item Replaced |^{\text{\thefootnote}}| by |{}^{\text{\textup{\thefootnote}}}|.
%     \item Bug fix: When \xpackage{etoolbox.sty} was not found, and the \xpackage{float}
%             package had been loaded, and the float specifier was \texttt{H},
%             then the |\tablefootnote|s were not printed for that table. Fixed.
%   \end{Version}
%   \begin{Version}{2011/11/19 v1.0d}
%     \item Replaced |{}^{\text{\textup{\thefootnote}}}| by
%             |\textsuperscript{\thefootnote}|, therefore the \xpackage{amstext}
%             package is no longer required by the \xpackage{tablefootnote} package.
%     \item \xpackage{lscape} and \xpackage{pdflscape} package now work with
%             \xpackage{tablefootnote} (except for using a \texttt{sidewaystable} on
%             a \texttt{landscape} page).
%     \item Added information about footnotes in \texttt{sidewaystable} being printed
%             one below the other, even if \xpackage{footmisc} package with option
%             \texttt{para} is used.
%     \item Redefined |\tfn@swtbox| from
%  \begin{verbatim}
%  \vspace{0.8cm}%
%  \begin{spacing}{0.1}%
%  \subcaptionbox*{}[\linewidth][l]{\tfn@tablefootnoteprintout}%
%  \end{spacing}%
%  \end{verbatim}
%             to
%  \begin{verbatim}
%  \vspace{2\baselineskip}%
%  \parbox[b][][t]{\linewidth}{\raggedright\tfn@tablefootnoteprintout}%
%  \end{verbatim}
%             , therefore the \xpackage{setspace} and \xpackage{subcaption} packages
%             are no longer required by the \xpackage{tablefootnote} package.
%     \item No longer uses |\numexpr|, therefore also usable without \eTeX{}.
%     \item Replaced |\textsuperscript{\thefootnote}| by
%             |\hbox {\@textsuperscript {\normalfont \thefootnote }}|,
%             which is next to the original definition of |\@makefnmark|
%             (|\hbox {\@textsuperscript {\normalfont \@thefnmark }}|,
%             but according to |\show\@thefnmark| here |\@thefnmark| is
%             |{\protect \itshape}| instead of |\thefootnote|).
%     \item Added an error message, when \LuaTeX{}, \xpackage{rotating} package, and
%            \xpackage{hyperref} package are used together. Hyperreferences in
%            |sidewaystable|s are misplaced, independently of the
%            \xpackage{tablefootnote} package or footnotes at all. (Bug reported
%            by \textsc{Arno Trautmann}, 2011/11/15 -- thanks!)
%     \item Added footnotesymbol-footnote-numbers in the example.
%   \end{Version}
%   \begin{Version}{2011/11/26 v1.0e}
%     \item Check for \xpackage{hyperref} is only done once.
%     \item Message regarding the misplacement of hyperlinks by \LuaLaTeX{}
%             changed form error to warning.
%     \item Replaced |\vspace{2\baselineskip}| by
%  \begin{verbatim}
% \ifvoid \footins%
% \vskip 2\baselineskip%
% \else%
% \vskip \skip\footins%
% \fi%
%  \end{verbatim}
%             between the |sidewaystable|s and their footnotes.
%     \item The warning about un-regarded |para|-option of \xpackage{footmoisc}
%             package is only given at the second |tablefootnote| in a |sidewaystable|,
%             not repeatedly for every |tablefootnote| in a |sidewaystable|.
%   \end{Version}
%   \begin{Version}{2012/01/01 v1.0f}
%     \item Bugfix: Wrong installation path given in the documentation, fixed.
%     \item Update of documentation, README, and \xfile{dtx} internals.
%   \end{Version}
%   \begin{Version}{2012/01/14 v1.0g}
%     \item Bugfix: There was a |relax| instead of a |\relax|, fixed.
%   \end{Version}
%   \begin{Version}{2012/07/29 v1.0h}
%     \item \xpackage{tablefootnote} is now compatible to the new
%             \xpackage{footnotebackref} package, 2012/07/01, v1.0,
%             \url{http://www.ctan.org/pkg/footnotebackref}.
%   \end{Version}
%   \begin{Version}{2014/01/08 v1.1a}
%     \item Tablefootnotes now regard optional footnote marks:
%             |\tablefootnote[| custom mark here! |]{...}|
%     \item Bug fix: the case of \xpackage{hyperref} option |hyperfootnotes=false|
%             was not handeled at all. (Bug reported by \textsc{Thomas~V.}, %
%             2014/01/02 -- thanks!)
%     \item A lot of details.
%   \end{Version}
%   \begin{Version}{2014/01/20 v1.1b}
%     \item Bug fix: Package was broken when \xpackage{hyperref} was \emph{not} used.
%            (Bug reported by \textsc{Sverre Stausland Johnsen} -- thanks!)
%   \end{Version}
%   \begin{Version}{2014/01/26 v1.1c}
%     \item The emergency bug fix of version 2014/01/20 v1.1b was replaced by using
%             the new robust macro |\tfn@hyperfootnotes|, which only \emph{internally}
%             uses |\ifHy@hyperfootnotes|. If the \xpackage{hyperref} package is not used,
%             \LaTeX{} now does not see any confusing
%             |\ifHy@hyperfootnotes|\ldots|\else|\ldots|\fi| at all.
%     \item URLs and README updated.
%   \end{Version}
% \end{History}
%
% \bigskip
%
% When you find a mistake or have a suggestion for an improvement of this package,
% please send an e-mail to the maintainer, thanks! (Please see BUG REPORTS in the README.)
%
% \bigskip
%
% \PrintIndex
%
% \Finale
\endinput
%        (quote the arguments according to the demands of your shell)
%
% Documentation:
%    (a) If tablefootnote.drv is present:
%           (pdf)latex tablefootnote.drv
%           makeindex -s gind.ist tablefootnote.idx
%           (pdf)latex tablefootnote.drv
%           makeindex -s gind.ist tablefootnote.idx
%           (pdf)latex tablefootnote.drv
%    (b) Without tablefootnote.drv:
%           (pdf)latex tablefootnote.dtx
%           makeindex -s gind.ist tablefootnote.idx
%           (pdf)latex tablefootnote.dtx
%           makeindex -s gind.ist tablefootnote.idx
%           (pdf)latex tablefootnote.dtx
%
%    The class ltxdoc loads the configuration file ltxdoc.cfg
%    if available. Here you can specify further options, e.g.
%    use DIN A4 as paper format:
%       \PassOptionsToClass{a4paper}{article}
%
% Installation:
%    TDS:tex/latex/tablefootnote/tablefootnote.sty
%    TDS:doc/latex/tablefootnote/tablefootnote.pdf
%    TDS:doc/latex/tablefootnote/tablefootnote-example.tex
%    TDS:doc/latex/tablefootnote/tablefootnote-example.pdf
%    TDS:source/latex/tablefootnote/tablefootnote.dtx
%
%<*ignore>
\begingroup
  \catcode123=1 %
  \catcode125=2 %
  \def\x{LaTeX2e}%
\expandafter\endgroup
\ifcase 0\ifx\install y1\fi\expandafter
         \ifx\csname processbatchFile\endcsname\relax\else1\fi
         \ifx\fmtname\x\else 1\fi\relax
\else\csname fi\endcsname
%</ignore>
%<*install>
\input docstrip.tex
\Msg{******************************************************************}
\Msg{* Installation                                                   *}
\Msg{* Package: tablefootnote 2014/01/26 v1.1c Table foot notes (HMM) *}
\Msg{******************************************************************}

\keepsilent
\askforoverwritefalse

\let\MetaPrefix\relax
\preamble

This is a generated file.

Project: tablefootnote
Version: 2014/01/26 v1.1c

Copyright (C) 2011 - 2014 by
    H.-Martin M"unch <Martin dot Muench at Uni-Bonn dot de>

The usual disclaimer applies:
If it doesn't work right that's your problem.
(Nevertheless, send an e-mail to the maintainer
 when you find an error in this package.)

This work may be distributed and/or modified under the
conditions of the LaTeX Project Public License, either
version 1.3c of this license or (at your option) any later
version. This version of this license is in
   http://www.latex-project.org/lppl/lppl-1-3c.txt
and the latest version of this license is in
   http://www.latex-project.org/lppl.txt
and version 1.3c or later is part of all distributions of
LaTeX version 2005/12/01 or later.

This work has the LPPL maintenance status "maintained".

The Current Maintainer of this work is H.-Martin Muench.

This work consists of the main source file tablefootnote.dtx,
the README, and the derived files
   tablefootnote.sty, tablefootnote.pdf,
   tablefootnote.ins, tablefootnote.drv,
   tablefootnote-example.tex, tablefootnote-example.pdf.

In memoriam Tommy Muench + 2014/01/02.

\endpreamble
\let\MetaPrefix\DoubleperCent

\generate{%
  \file{tablefootnote.ins}{\from{tablefootnote.dtx}{install}}%
  \file{tablefootnote.drv}{\from{tablefootnote.dtx}{driver}}%
  \usedir{tex/latex/tablefootnote}%
  \file{tablefootnote.sty}{\from{tablefootnote.dtx}{package}}%
  \usedir{doc/latex/tablefootnote}%
  \file{tablefootnote-example.tex}{\from{tablefootnote.dtx}{example}}%
}

\catcode32=13\relax% active space
\let =\space%
\Msg{************************************************************************}
\Msg{*}
\Msg{* To finish the installation you have to move the following}
\Msg{* file into a directory searched by TeX:}
\Msg{*}
\Msg{*     tablefootnote.sty}
\Msg{*}
\Msg{* To produce the documentation run the file `tablefootnote.drv'}
\Msg{* through (pdf)LaTeX, e.g.}
\Msg{*  pdflatex tablefootnote.drv}
\Msg{*  makeindex -s gind.ist tablefootnote.idx}
\Msg{*  pdflatex tablefootnote.drv}
\Msg{*  makeindex -s gind.ist tablefootnote.idx}
\Msg{*  pdflatex tablefootnote.drv}
\Msg{*}
\Msg{* At least three runs are necessary e.g. to get the}
\Msg{*  references right!}
\Msg{*}
\Msg{* Happy TeXing!}
\Msg{*}
\Msg{************************************************************************}

\endbatchfile
%</install>
%<*ignore>
\fi
%</ignore>
%
% \section{The documentation driver file}
%
% The next bit of code contains the documentation driver file for
% \TeX{}, i.\,e., the file that will produce the documentation you
% are currently reading. It will be extracted from this file by the
% \texttt{docstrip} programme. That is, run \LaTeX{} on \texttt{docstrip}
% and specify the \texttt{driver} option when \texttt{docstrip}
% asks for options.
%
%    \begin{macrocode}
%<*driver>
\NeedsTeXFormat{LaTeX2e}[2011/06/27]
\ProvidesFile{tablefootnote.drv}%
  [2014/01/26 v1.1c Table foot notes (HMM)]
\documentclass{ltxdoc}[2007/11/11]% v2.0u
\usepackage{holtxdoc}[2012/03/21]%  v0.24
%% tablefootnote may work with earlier versions of LaTeX2e and those
%% class and package, but this was not tested.
%% Please consider updating your LaTeX, class, and package
%% to the most recent version (if they are not already the most
%% recent version).
\hypersetup{%
 pdfsubject={Footnotes in Tables (HMM)},%
 pdfkeywords={LaTeX, tablefootnote, footnote, table, H.-Martin Muench},%
 pdfencoding=auto,%
 pdflang={en},%
 breaklinks=true,%
 linktoc=all,%
 pdfstartview=FitH,%
 pdfpagelayout=OneColumn,%
 bookmarksnumbered=true,%
 bookmarksopen=true,%
 bookmarksopenlevel=3,%
 pdfmenubar=true,%
 pdftoolbar=true,%
 pdfwindowui=true,%
 pdfnewwindow=true%
}
\CodelineIndex
\hyphenation{printing docu-ment}
\gdef\unit#1{\mathord{\thinspace\mathrm{#1}}}%
\makeatletter
\@ifundefined{eTeX}{\gdef\eTeX{$\m@th \varepsilon $-\TeX }}{% else \relax
}
\makeatother
\begin{document}
  \DocInput{tablefootnote.dtx}%
\end{document}
%</driver>
%    \end{macrocode}
%
% \fi
%
% \CheckSum{584}
%
% \CharacterTable
%  {Upper-case    \A\B\C\D\E\F\G\H\I\J\K\L\M\N\O\P\Q\R\S\T\U\V\W\X\Y\Z
%   Lower-case    \a\b\c\d\e\f\g\h\i\j\k\l\m\n\o\p\q\r\s\t\u\v\w\x\y\z
%   Digits        \0\1\2\3\4\5\6\7\8\9
%   Exclamation   \!     Double quote  \"     Hash (number) \#
%   Dollar        \$     Percent       \%     Ampersand     \&
%   Acute accent  \'     Left paren    \(     Right paren   \)
%   Asterisk      \*     Plus          \+     Comma         \,
%   Minus         \-     Point         \.     Solidus       \/
%   Colon         \:     Semicolon     \;     Less than     \<
%   Equals        \=     Greater than  \>     Question mark \?
%   Commercial at \@     Left bracket  \[     Backslash     \\
%   Right bracket \]     Circumflex    \^     Underscore    \_
%   Grave accent  \`     Left brace    \{     Vertical bar  \|
%   Right brace   \}     Tilde         \~}
%
% \GetFileInfo{tablefootnote.drv}
%
% \begingroup
%   \def\x{\#,\$,\^,\_,\~,\ ,\&,\{,\},\%}%
%   \makeatletter
%   \@onelevel@sanitize\x
% \expandafter\endgroup
% \expandafter\DoNotIndex\expandafter{\x}
% \expandafter\DoNotIndex\expandafter{\string\ }
% \begingroup
%   \makeatletter
%     \lccode`9=32\relax
%     \lowercase{%^^A
%       \edef\x{\noexpand\DoNotIndex{\@backslashchar9}}%^^A
%     }%^^A
%   \expandafter\endgroup\x
%
% \DoNotIndex{\\,\,}
% \DoNotIndex{\documentclass,\usepackage,\ProvidesPackage,\begin,\end}
% \DoNotIndex{\MessageBreak,\message}
% \DoNotIndex{\NeedsTeXFormat,\DoNotIndex,\verb}
% \DoNotIndex{\def,\edef,\gdef,\global}
% \DoNotIndex{\ifx,\listfiles,\mathord,\mathrm}
% \DoNotIndex{\kvoptions,\SetupKeyvalOptions,\ProcessKeyvalOptions}
% \DoNotIndex{\bigskip,\space,\thinspace,\ldots}
% \DoNotIndex{\indent,\noindent,\newline,\linebreak,\pagebreak}
% \DoNotIndex{\textbf,\textit,\textsf,\texttt,\textsc,\textquotedblleft,\textquotedblright}
% \DoNotIndex{\plainTeX,\TeX,\LaTeX,\pdfLaTeX}
% \DoNotIndex{\chapter,\section,\the,\csname,\endscname}
% \DoNotIndex{\value,\arabic,\hline,\text,\caption,\centering,\clear,\url}
%
% \title{The \xpackage{tablefootnote} package}
% \date{2014/01/26 v1.1c}
% \author{H.-Martin M\"{u}nch\\\xemail{Martin.Muench at Uni-Bonn.de}}
%
% \maketitle
%
% \begin{abstract}
% This \LaTeX{} package provides the command |\tablefootnote| to be used
% in a \texttt{table} or \texttt{sidewaystable} environment,
% where |\footnote| will not work and when using |\footnotemark| and
% |\footnotetext| and adjusting the counters (including \texttt{Hfootnote})
% manually is either too much work or would not even work (\texttt{sidewaystable}).
% \end{abstract}
%
% \bigskip
%
% \noindent Disclaimer for web links: The author is not responsible for any contents
% referred to in this work unless he has full knowledge of illegal contents.
% If any damage occurs by the use of information presented there, only the
% author of the respective pages might be liable, not the one who has referred
% to these pages.
%
% \bigskip
%
% \noindent {\color{green} Save per page about $200\unit{ml}$ water,
% $2\unit{g}$ CO$_{2}$ and $2\unit{g}$ wood:\\
% Therefore please print only if this is really necessary.}
%
% \pagebreak
%
% \tableofcontents
%
% \pagebreak
%
% \section{Introduction\label{sec:Introduction}}
%
% \indent This \LaTeX{} package provides the command |\tablefootnote{...}| to be
% used in a \texttt{table} or \texttt{sidewaystable} environment. In those
% environments |\footnote{...}| would not work and would need to be replaced
% by |\footnotemark| in the \texttt{(sideways)\allowbreak table} and
% |\footnotetext{...}| after the \texttt{(sideways)table} environment ended.
% Additionally the counters for \texttt{footnote} and
% (when the \xpackage{hyperref} package is used) \texttt{Hfootnote}
% would need to be adjusted manually. Then still the hyperlinks
% in a \texttt{(sideways)table} would not work.
% When the \xpackage{footnotebackref} package is used,
% |\footnotemark| and |\footnotetext{...}| would need to be redefined.
% Just using |\tablefootnote{...}| (with optional argument for custom foot note
% marks) in the (sideways)table does all this automatically.
% (Compatibility with the \xpackage{footnotebackref} package
% was achieved without redefining |\footnotemark| or |\footnotetext|.)
% Redefining |\footnote| was \emph{not} done as to not create conflicts with
% other packages.
%
% \section{Usage}
%
% \indent Just load the package placing
% \begin{quote}
%   |\usepackage{tablefootnote}|
% \end{quote}
% \noindent in the preamble of your \LaTeXe{} source file and use\newline
% |\tablefootnote{...}| instead of |\footnote{...}| and\newline
% |\tablefootnote[...]{...}| instead of |\footnote[...]{...}|
% in (sideways)\allowbreak tables.\\
% When the \xpackage{footnotebackref} package is used,
% it must be loaded before the \xpackage{tablefootnote} package
% and the \xpackage{hyperrefpackage} with option |hyperfootnotes=true|
% must be loaded, too. (Backreference{-}links without links do not make sense.)\\
% When the \xpackage{rotating} package (for |sidewaystable|s) is used,
% it must be loaded before the \xpackage{tablefootnote} package and it must be
% a recent version (v2.16a, 2009/03/28, or newer).\\
% If |\tablefootnote{...}| is used in a tabular environment, which is inside
% of a (sideways)table environment, everything is fine, but when the tabular
% environment is not inside of a (sideways)table, the \textbf{table}footnote
% will not work.\\
%
% Pages with tables with footnotes, which are rotated with the \xpackage{lscape}-
% or with the \xpackage{pdflscape}-package (|\begin{landscape}|, table, |\end{landscape}|),
% are possible. But when the \xpackage{footmisc}-package is used, \xpackage{(pdf)lscape}
% should be loaded before \xpackage{footmisc}.\\
%
% When the \xpackage{footmisc}-package is used with option \texttt{para},
% this is regarded with the exception of a \texttt{sidewaystable}. There the footnotes
% are printed one below the other. (Well, this is better than no footnote in a
% \texttt{sidewaystable} at all, isn't it?)\\
%
% When the (sideways)table floats, the footnotes are set and (when \xpackage{hyperref}
% is used) hyperlinked, but they are not automatically adapted when the table floats
% over/under another footnote. Thus either do not use a footnote between original and
% float position of the (sideways)table, or place the (sideways)table in
% \textquotedblleft here\textquotedblright\ position. |\clear(double)page|, |h(!)|,
% |H|~from the \xpackage{float} package (\url{http://www.ctan.org/pkg/float}),
% or |\FloatBarrier| from the \xpackage{picins} package
% (\url{http://www.ctan.org/pkg/picins}) might help, too.
% (Or after finishing the document, move the (sideways)table in the source code
% near the position where it floats to or use the optional footnote marks.)
%
% \pagebreak
%
% \section{Alternatives\label{sec:Alternatives}}
%
% \begin{description}
% \item[-] The \xpackage{longtable} package provides the \texttt{longtable} environment
%   as replacement for the combined \texttt{table} and \texttt{tabular} environments.
%   Footnotes are real footnotes (not just tablenotes), are continuously numbered and
%   hyperlinked (when using the \xpackage{hyperref} package), and the hyperlinks really
%   work. As drawback the appearance of the caption changes slightly (e.\,g.~distance
%   to the table, width of the caption), which can probably be changed back manually.
%   Furthermore, longtables are meaned to break over more than one page. If that is not
%   wished, it must be prevented by |\nopagebreak|-commands and by ending the longtable
%   lines with |\\*| instead of |\\|. longtables do not float. (Therefore using the
%   \xpackage{tablefootnote} package and |\FloatBarrier| from the \xpackage{picins} package
%   before and after the table environment is similar~- but \xpackage{tablefootnote} does not
%   change the table-caption!) \texttt{sidewaystable} does not work with it.\\
%   \url{http://www.ctan.org/pkg/longtable}
%
% \item[-] The \xpackage{supertabular} package provides the \texttt{mpsupertabular}
%   environment as replacement for the combined \texttt{table} and \texttt{tabular}
%   environments. Footnotes are just tablenotes (with working hyperlinks when using the
%   \xpackage{hyperref} package), i.\,e. numbered a, b, c and placed below the table
%   and not at the end of the page. Therefore there is no float problem (because the
%   tablenotes numbering is not included in the continuous numbering of the footnotes).
%   Placing the \texttt{supertabular} inside of a \texttt{sidewaystable} breaks the
%   hyperlinks to the tablenotes.\\
%   \url{http://www.ctan.org/pkg/supertabular}
%
% \item[-] The \xpackage{ctable} package has its very own notation for defining tables.
%   It can create tablenotes and sideways-tables. The tablenotes are not automatically
%   hyperlinked. The \texttt{ctable}s float. Because the tablenotes numbering is not
%   included in the continuous numbering of the footnotes there is no float problem.\\
%   \url{http://www.ctan.org/pkg/ctable}
%
% \item[-] The \xpackage{footnote} package provides |\makesavenoteenv{table}|. After
%   loading the package and using that command in the preamble, in tables |\footnote{...}|
%   can be used. Using |\makesavenoteenv{tabular}| and \\
%   |\makesavenoteenv{sidewaystable}|
%   is possible, but it neither solves the float problem, nor do the created hyperlinks
%   work (i.\,e.~they aim at wrong locations). The \xpackage{mdwtab} from the same bundle
%   is incompatible with other table-related packages (e.\,g. \xpackage{supertabular},
%   \xpackage{array}) and not $100\,\%$ compatible with the \texttt{tabular} environment.\\
%   \url{http://www.ctan.org/pkg/footnote}\\
%   \url{http://www.ctan.org/pkg/mdwtab}
%
% \item[-] The \xpackage{tabularx} package does produce footnotes for sidewaystables,
%   but uses a, b, c instead of 1, 2, 3. The hyperlinks to the footnotes do not work.
%   Because the footnotes numbering is not included in the continuous numbering of
%   the other footnotes there is no float problem.\\
%   \url{http://www.ctan.org/pkg/tabularx}
%
% \item[-] Placing a tabular inside a minipage inside a table produces tablenotes.
%   Therefore there is no float problem (because the footnotes are not continuously
%   numbered). The hyperlinks to the table notes indeed work.
%
% \pagebreak
%
% \item[-] The \xpackage{threeparttable} package creates tablenotes again. Therefore
%   there is no float problem (because the tablenotes are not continuously numbered with
%   the footnotes). There are no hyperlinks to the table notes (at least not automatically).
%   Using |sidewaystable| (with table notes) works.\\
%   \url{http://www.ctan.org/pkg/threeparttable}
%
% \item[-] The \xpackage{threeparttablex} package creates tablenotes again. Therefore there
%   is no float problem (because the tablenotes are not continuously numbered with the
%   footnotes). With option \texttt{referable} the tablenotes are hyperlinked. Use of
%   a |sidewaystable| (with table notes) did not work for me. When using the
%   \texttt{referable} option according to the example in the \xpackage{threeparttablex}
%   manual the \xpackage{longtable} package is used, therefore that package could be used
%   directly without \xpackage{threeparttablex} (see above).\\
%   \url{http://www.ctan.org/pkg/threeparttablex}
%
% \item[-] One can manually use |\footnotemark| in the table and |\footnotetext{...}|
%  after the table environment ended and manually change the |footnote| and
%  (when the \xpackage{hyperref} package is used) |Hfootnote| counters and needs
%  to insert
%  \begin{verbatim}
%     \makeatletter
%     \global\let\Hy@saved@currentHref\@currentHref
%     \hyper@makecurrent{Hfootnote}%
%     \global\let\Hy@footnote@currentHref\@currentHref
%     \global\let\@currentHref\Hy@saved@currentHref
%     \makeatother
%  \end{verbatim}
%  before each |\footnotetext{...}|, but with more than very few footnotes this can become
%  much work (and is prone to produce errors). And this does not even work with
%  sidewaystables. (And it does not work with the \xpackage{footnotebackref} package
%  without redefining |\footnotemark| and |\footnotetext{...}|.)
% \end{description}
%
% \bigskip
%
% \noindent (You programmed or found another alternative,
%  which is available at \CTAN{}?\\
%  OK, send an e-mail to me with the name, location at \CTAN{},
%  and a short notice, and I will probably include it in
%  the list above.)
%
% \newpage
%
% \section{Example}
%
%    \begin{macrocode}
%<*example>
\documentclass[british]{article}[2007/10/19]% v1.4h
%%%%%%%%%%%%%%%%%%%%%%%%%%%%%%%%%%%%%%%%%%%%%%%%%%%%%%%%%%%%%%%%%%%%%
\usepackage{float}[2001/11/08]%    v1.3d
\usepackage{placeins}[2005/04/18]% v2.2  ; for \FloatBarrier
\usepackage{rotating}[2009/03/28]% v2.16a; for sidewaystable-environment
\usepackage[%
 hyperfootnotes=true,%
 extension=pdf,%
 plainpages=false,%
 pdfpagelabels=true,%
 hyperindex=false,%
 pdflang={en},%
 pdftitle={tablefootnote package example},%
 pdfauthor={H.-Martin Muench},%
 pdfsubject={Example for the tablefootnote package},%
 pdfkeywords={LaTeX, tablefootnote, footnote, table, H.-Martin Muench},%
 % pdfview=FitH and FitBH do not work: hyperlinks in sidewaystables
 % do no lead to the footnotes, due to a bug in pdfTeX,
 % computing wrong anchor coordinates (Heiko Oberdiek, 29. October 2011)
 % pdfview=Fit, FitV, FitR, FitB, FitBV work
 % print is OK for all those options
 pdfstartview=FitH,%
 pdfpagelayout=OneColumn%
]{hyperref}[2012/11/06]% v6.83m
 % Due to the urls used in the example, either the hyperref or the url
 % package are needed (or the urls must be removed before compiling).

\usepackage{footnotebackref}[2012/07/01]% v1.0
\usepackage{tablefootnote}[2014/01/26]%   v1.1c

\gdef\unit#1{\mathord{\thinspace\mathrm{#1}}}%
\listfiles
\begin{document}
\pagenumbering{arabic}
\section*{Example for tablefootnote}

This example demonstrates the use of package\newline
\textsf{tablefootnote}, v1.1c as of 2014/01/26 (HMM).\newline
There were no options used. (The package provides no options.)\newline

\textbf{The \texttt{tablefootnote-example.tex} needs to be compiled
at least twice to get the references right!}\newline

If the etoolbox-package is found, it is automatically used.\newline

For more details please see the documentation!\newline

\noindent Save per page about $200\unit{ml}$ water,
$2\unit{g}$ CO$_{2}$ and $2\unit{g}$ wood:\newline
Therefore please print only if this is really necessary.\newline

Here is some text.\footnote{And this is a text footnote.}\newline

Tables \ref{tab.symbol}, \ref{tab.normal}, \ref{tab.another} and
\ref{tab.floatH} show normal tables, table~\ref{tab.sideways} depicts
a sidewaystable. Table~\ref{tab.floatH} uses the float
specifier~\texttt{H} from the float package.\newline

\texttt{Hyperref} option \verb|pdfview=FitH| and \verb|FitBH| do
not work due to a bug in pdf\TeX{}, computing wrong
anchor coordinates (\textsc{Heiko Oberdiek}, 29.~October 2011).
Depending on used pdf-viewer, hyperlinks in sidewaystables lead
e.\,g.~at the end of the document, not at the footnote.
\verb|pdfview=Fit|, \verb|FitV|, \verb|FitR|, \verb|FitB|,
\verb|FitBV| work, print is OK for all those options.

\bigskip

\listoftables

\pagebreak

\renewcommand{\thefootnote}{\fnsymbol{footnote}}
\verb|\renewcommand{\thefootnote}{\fnsymbol{footnote}}|
causes footnote{-}symbol{-}footnotes, which are possible
(see Table~\ref{tab.symbol}).

\begin{table}
\centering
\begin{tabular}{ccc}
Another\tablefootnote{A table footnote.} & %
text\tablefootnote{Another table footnote.} & %
in a table\tablefootnote{A third table footnote.}
\end{tabular}
\caption[A footnotesymbol table]{%
A table with footnote-symbol-footnotes.\label{tab.symbol}}
\end{table}

Some text.\footnote{A text footnote.}

\renewcommand{\thefootnote}{\arabic{footnote}}
\verb|\renewcommand{\thefootnote}{\arabic{footnote}}|
switches back to normal footnote numbers again.

\pagebreak

\begin{table}
\centering
\begin{tabular}{ccc}
Some\tablefootnote{A table footnote.} & %
text\tablefootnote[99]{A table footnote with custom footnote number.} & %
in a table\tablefootnote{A third table footnote.}
\end{tabular}
\caption[A table]{A normal table.\label{tab.normal}}
\end{table}

Some text.\footnote{Another text footnote.}

\pagebreak

%    \end{macrocode}
%
% \pagebreak
%
%    \begin{macrocode}
More text.\footnote{And yet another text footnote.}

\begin{table}[t]
\centering
\begin{tabular}{|c|c|c|}
\hline
Another\tablefootnote{A $3^{rd}$ table footnote.} & %
text\tablefootnote{Another $3^{rd}$ table footnote.} & %
in a table\tablefootnote{A $3^{rd}$ third table footnote.}\\ \hline
\end{tabular}
\caption[Another table]{Another table (third one)\label{tab.another}}
\end{table}

Please note that Table~\ref{tab.another} floated to the top of the page.
While the footnotes are set and (when hyperref is used) hyperlinked,
they are not automatically adapted. Thus either do not use a footnote
at the same page before the table, or place the table in
\textquotedblleft here\textquotedblright\ or
\textquotedblleft bottom\textquotedblright\ position.
\verb|\clear(double)page|, \verb|h(!)|, \verb|H|~from the
\texttt{float} package, or \verb|\FloatBarrier| from the
\texttt{picins} package might help, too. (Or move the table in the
source code near the position where it floats to
or use the optional footnote marks.)

Table~\ref{tab.floatH} (page~\pageref{tab.floatH}) uses float specifier
\texttt{H} from the float package and does not float.

Some text.\footnote{This is just another text footnote.}

\pagebreak

\FloatBarrier

\begin{sidewaystable}
\centering%
\begin{tabular}{ccc}
Text\tablefootnote{Please rotate the view for testing the %
hyperlinks.} & %
in a\tablefootnote[98]{Another sidewaystable footnote %
with optional footnote mark.} & %
sidewaystable%
\tablefootnote{Sidewaystable-environment provided by %
\url{http://www.ctan.org/pkg/rotating} package.}
\end{tabular}
\caption[A sideways table]{A table in the \texttt{sideways} %
environment\label{tab.sideways}}%
\end{sidewaystable}%

\FloatBarrier

\pagebreak

%    \end{macrocode}
%
% \pagebreak
%
%    \begin{macrocode}
A last table, here with float specifier \texttt{H} from the
float\footnote{\url{http://www.ctan.org/pkg/float}} package.

\begin{table}[H]
\centering
\begin{tabular}{ccc}
Another\tablefootnote{A $5^{th}$ table footnote.} & %
text\tablefootnote{Another $5^{th}$ table footnote.} & %
in a table\tablefootnote{A $5^{th}$ third table footnote.}
\end{tabular}
\caption[A last table]{A very last table\label{tab.floatH}}
\end{table}

Some text.\footnote{This is just another text footnote.}

\pagebreak

End of the example for the
tablefootnote\footnote{\url{http://www.ctan.org/pkg/tablefootnote}}
package.

\end{document}
%</example>
%    \end{macrocode}
%
% \StopEventually{}
%
% \newpage
%
% \section{The implementation}
%
% We start off by checking that we are loading into \LaTeXe{} and
% announcing the name and version of this package.
%
%    \begin{macrocode}
%<*package>
%    \end{macrocode}
%
%    \begin{macrocode}
\NeedsTeXFormat{LaTeX2e}[2011/06/27]
\ProvidesPackage{tablefootnote}[2014/01/26 v1.1c
              Table foot notes (HMM)]
%    \end{macrocode}
%
% A short description of the \xpackage{tablefootnote} package:
%
%    \begin{macrocode}
%% Provides the \tablefootnote[...]{...} command
%% for footnotes in (sideways)tables.

%    \end{macrocode}
%
% We need the \xpackage{ltxcmds} and \xpackage{letltxmacro} packages
% by \textsc{Heiko Oberdiek}:
%
%    \begin{macrocode}
\RequirePackage{ltxcmds}[2011/11/09]%     v1.22
\RequirePackage{letltxmacro}[2010/09/02]% v1.4
\RequirePackage{xifthen}[2009/04/17]%     v1.3
%    \end{macrocode}
%
% A last information for the user:
%
%    \begin{macrocode}
%% tablefootnote may work with earlier versions of LaTeX and those
%% packages, but this was not tested. Please consider updating
%% your LaTeX and packages to the most recent version
%% (if they are not already the most recent version).

%    \end{macrocode}
%
% \indent See subsection~\ref{ss:Downloads} about how to get them.\\
%
% When the \xpackage{rotating} package (for |sidewaystable|s) is used,
% it must be loaded before the \xpackage{tablefootnote} package and it must be
% a recent version:
%
%    \begin{macrocode}
\@ifpackageloaded{rotating}{%
  \gdef\tfn@rotating{1}
  \@ifpackagelater{rotating}{2009/03/28}% v2.16a
   {% >= 2009/03/28, OK
   }{% else
     \PackageWarningNoLine{tablefootnote}{%
       It is required version\MessageBreak%
       2009/03/28 v2.16a (or later) of package rotating,\MessageBreak%
       but only version\MessageBreak%
      `\csname ver@rotating.sty\endcsname'\MessageBreak%
      is available}%
   }%
 }{% else
   \gdef\tfn@rotating{0}%
 }

%    \end{macrocode}
%
% There are no options to be processed.\\
%
% We need some definitions:
%
%    \begin{macrocode}
\def\tfn@footnotetablecount{0}
\def\tfn@footnotetableprint{0}
\def\tfn@footnotezero{0}
\def\tfn@fnt{0}
\def\tfn@swt{0}%       _s_ide_w_ays_t_able
\def\tfn@fmpw{\relax}% f_oot_m_isc_p_ara_w_arn-switch
\def\tfn@fnbr{0}%      f_oot_n_ote_b_ack_r_ef(erence package)

%    \end{macrocode}
%
% \begin{macro}{\tfn@hyperfootnotes}
% We need a wrapper for the used |\ifHy@hyperfootnotes|,
% which is not defined when the \xpackage{hyperref} package has not been loaded:
%
%    \begin{macrocode}
\DeclareRobustCommand{\tfn@hyperfootnotes}[2]{%
  \ifHy@hyperfootnotes%
    #1%
  \else%
    #2%
  \fi%
  }

%    \end{macrocode}
% \end{macro}
%
% \begin{macro}{\tablemakefntext}
% |\tablemakefntext| makes the footnotetext for a footnote in a table.
%
%    \begin{macrocode}
\ltx@ifpackageloaded{footnotebackref}{%
  \gdef\tfn@fnbr{1}%
  \ifx\FootnoteBackref@symbol\empty%
    \ifFootnoteBackref@numberlinked%
      \newcommand\tablemakefntext[1]{%
        \noindent\makebox[1.634em][r]{%
         \mbox{\textsuperscript{\normalfont%
           \hyperlink{tfn:m\theHfootnote}{\@thefnmark}%
        }}}%
        \ #1%
      }
    \else
      \newcommand\tablemakefntext[1]{%
        \noindent\makebox[1.8em][r]{%
          \mbox{\textsuperscript{\normalfont%
            \@thefnmark%
        }}}%
        #1%
      }
    \fi
  \else
    \newcommand\tablemakefntext[1]{%
      \noindent\makebox[1.8em][r]{%
        \mbox{\textsuperscript{\normalfont%
          \ifFootnoteBackref@numberlinked%
            \hyperlink{tfn:m\theHfootnote}{\@thefnmark}\ %
          \else%
            \@thefnmark%
          \fi%
      }}}%
      \hyperlink{tfn:m\theHfootnote}{\FootnoteBackref@symbol}%
      #1%
    }
  \fi
}{%
  \newcommand\tablemakefntext[1]{%
    \noindent\makebox[1.8em][r]{%
      \mbox{\textsuperscript{\normalfont%
        \@thefnmark%
    }}}%
    #1%
  }
}

%    \end{macrocode}
%
% \end{macro}
%
% \pagebreak
%
% \begin{macro}{\swtablemakefntext}
%    \begin{macrocode}
\newcommand\swtablemakefntext[1]{\noindent\ #1}

%    \end{macrocode}
% \end{macro}
%
% \begin{macro}{\tfn@footnotemarks}
% |\tfn@footnotemarks| uses the |\footnotemark| command to place a foot note mark
% and decreases the |\tfn@footnotetablecount| (i.\,e.~increases the negative value)
% of the number of footnote(marks) in the table. If the \xpackage{footnotebackref}
% package is used, it is also necessary to create an according |\hypertarget| for
% the back reference.
%
%    \begin{macrocode}
\newcommand{\tfn@footnotemarks}[1][]{%
  \hbox{%
    \ifthenelse{\isempty{#1}}%
      {\footnotemark%
       \ltx@ifpackageloaded{footnotebackref}%
         {\raisebox{\ht\strutbox}{\hypertarget{tfn:m\theHfootnote}{}}}%
         {% footnotebackref not loaded
         }%
      }%
      {\ltx@ifpackageloaded{hyperref}%
        {\tfn@hyperfootnotes{%
            \hyperlink{tfn:#1}{\footnotemark[#1]}%
          }{\footnotemark[#1]}}%
        {\footnotemark[#1]}%
       \ltx@ifpackageloaded{footnotebackref}%
        {\raisebox{\ht\strutbox}{\hypertarget{tfn:m#1}{}}}{}%
      }%
   }%
  \@tempcnta=\tfn@footnotetablecount\relax%
  \advance\@tempcnta by -1\relax%
  \xdef\tfn@footnotetablecount{\the\@tempcnta}%
}

%    \end{macrocode}
% \end{macro}
%
% \begin{macro}{\tfn@footmiscparawarn}
% When the \xpackage{footmisc} package is used with option |para|,
% we need to issue a warning. (Any idea how to resolve this issue?)\\
% It must be given inside of it's own command, because |\ifFN@para| is not
% defined when \xpackage{footmisc} has not been loaded, and then the
% |\fi| would be a |\fi| without |\if| (and \TeX{} does not like single
% |\fi|s).
%
%    \begin{macrocode}
\newcommand{\tfn@footmiscparawarn}{%
  \ifFN@para%
    \PackageWarning{tablefootnote}{%
      Package footmisc with option para used.\MessageBreak%
      Unfortunately package tablefootnote is not yet\MessageBreak%
      able to regard this in sidewaystables.\MessageBreak%
      Here the footnotes are printed one below\MessageBreak%
      the other. Sorry!\MessageBreak%
     }%
  \fi%
  }

%    \end{macrocode}
% \end{macro}
%
% \begin{macro}{\tfn@updateHyper}
% |\tfn@updateHyper| updates some \xpackage{hyperref} internals.
%
%    \begin{macrocode}
\newcommand{\tfn@updateHyper}{%
  % The code of this command is partly from the
  % hyperref package by Heiko Oberdiek, 2011/10/01, v6.82j.
  % Newer versions of that package are available at CTAN.org.
  \global\let\Hy@saved@currentHref\@currentHref%
  \tfn@hyperfootnotes{\hyper@makecurrent{Hfootnote}}{}%
  \global\let\Hy@footnote@currentHref\@currentHref%
  \global\let\@currentHref\Hy@saved@currentHref%
  }

%    \end{macrocode}
% \end{macro}
%
% \begin{macro}{\tfn@footnotetext}
% |\tfn@footnotetext| contains the |\footnotetext[...]{...}| command and handles
% the numbering. |\tfn@footnotetext| will be executed after the table.
% The foot note counter has been increased automatically with each |\footnotemark|,
% therefore it must be reset to its value before the table began.
%
%    \begin{macrocode}
\newcommand{\tfn@footnotetext}[2][]{%
  \ifx\tfn@footnotetableprint\tfn@footnotezero%
    % This is the first footnotetext after the (sideways)table.
    \addtocounter{footnote}{\tfn@footnotetablecount}%
%    \end{macrocode}
%
% When the \xpackage{hyperref} package is used, also its |Hfootnote| counter must
% be reset (when the footnotes are hyperlinked).
%
%    \begin{macrocode}
    \ltx@ifpackageloaded{hyperref}{%
      \tfn@hyperfootnotes{%
        \addtocounter{Hfootnote}{\tfn@footnotetablecount}}{}%
      \tfn@updateHyper%
     }{}%
  \fi%
%    \end{macrocode}
%
% For issuing the text for the next footnote the footnote-counter
% (and |Hfootnote| counter in case of links) are increased by one.
%
%    \begin{macrocode}
  \addtocounter{footnote}{+1}%
  \ltx@ifpackageloaded{hyperref}{%
    \tfn@hyperfootnotes{\addtocounter{Hfootnote}{+1}}{}%
%    \end{macrocode}
%
% The \xpackage{hyperref} package was not expecting this kind of change of the
% |(H)footnote| numbers, therefore some code-lines are missing here which can be
% found at other places in that same package and are called by this command.
%
%    \begin{macrocode}
    \tfn@updateHyper%
%    \end{macrocode}
%
% When \xpackage{hyperref} is not loaded, nothing is to be done here.
%
%    \begin{macrocode}
   }{}%
%    \end{macrocode}
%
% The |\footnotetext[...]{...}| places the foot note text as usual.
%
%    \begin{macrocode}
  \ifx\tfn@swt\tfn@footnotezero%
    \ifthenelse{\isempty{#1}}%
      {\footnotetext{#2}}%
      {\ltx@ifpackageloaded{hyperref}%
         {\tfn@hyperfootnotes{%
             \footnotetext[#1]{%
               \raisebox{\ht\strutbox}{\hypertarget{tfn:#1}{}}#2}%
           }{\footnotetext[#1]{#2}%
           }%
         }{\footnotetext[#1]{#2}}%
      }%
  \else%
%    \end{macrocode}
%
% If there is more than one tablefootnote in the |sidewaystable|, \\
% the |\tfn@footmiscparawarn| shall be given,
%
%    \begin{macrocode}
    \ifx\tfn@footnotetableprint\tfn@footnotezero%
      \gdef\tfn@fmpw{\tfn@footmiscparawarn}%
    \else%
      \ltx@ifpackageloaded{footmisc}{\tfn@fmpw}{\relax}%
%    \end{macrocode}
%
% but we do not want to repeat the message for a third(, forth,\ldots)
% tablefootnote in the |sidewaystable|.
%
%    \begin{macrocode}
      \gdef\tfn@fmpw{\relax}%
    \fi%
%    \end{macrocode}
%
% |\@makefnmark| is \\
% |\hbox {\@textsuperscript {\normalfont \@thefnmark }}|, \\
% but according to |\show\@thefnmark| here |\@thefnmark| is
% |{\protect \itshape}| instead of |\thefootnote| (Why?!).\\
% If the \xpackage{footnotebackref} package is used,
% back references must be created for the table footnotes, too.
%
%    \begin{macrocode}
    \ltx@ifpackageloaded{footnotebackref}{%
      \LetLtxMacro{\tfncurrentmft}{\@makefntext}%
      \LetLtxMacro{\@makefntext}{\swtablemakefntext}%
      \ifthenelse{\isempty{#1}}%
        {\footnotetext{%
           \raisebox{\ht\strutbox}{\hypertarget{tfn:\theHfootnote}{}}%
           \hbox{\@textsuperscript{\normalfont%
              \hyperlink{tfn:m\theHfootnote}{\thefootnote}}}\thinspace%
         \ifx\FootnoteBackref@symbol\empty%
         \else\hyperlink{tfn:m\theHfootnote}{\FootnoteBackref@symbol}%
         \fi%
         #2}%
       }{\edef\tfn@tfn{\arabic{footnote}}%
         \edef\tfn@thn{\arabic{Hfootnote}}%
         \setcounter{footnote}{#1}%
         \setcounter{Hfootnote}{#1}%
         \tfn@updateHyper%
         \footnotetext{%
           \raisebox{\ht\strutbox}{\hypertarget{tfn:#1}{}}%
           \hbox{\@textsuperscript{\normalfont%
                  \hyperlink{tfn:m#1}{#1}}}%
           \thinspace%
           \ifx\FootnoteBackref@symbol\empty%
           \else%
             \raisebox{\ht\strutbox}{%
               \hyperlink{tfn:m#1}{\FootnoteBackref@symbol}}%
           \fi%
           #2%
          }%
         \setcounter{footnote}{\tfn@tfn}%
         \setcounter{Hfootnote}{\tfn@thn}%
         \tfn@updateHyper%
       }%
      \LetLtxMacro{\@makefntext}{\tfncurrentmft}%
     }{% footnotebackref not loaded
       \ifthenelse{\isempty{#1}}%
         {\footnotetext{%
            \hbox{\@textsuperscript{\normalfont\thefootnote}}%
            \ltx@ifpackageloaded{hyperref}%
              {\tfn@hyperfootnotes{%
                 \raisebox{\ht\strutbox}{%
                   \hypertarget{tfn:\theHfootnote}{}}}{}%
              }{}%
            \thinspace%
            #2}%
        }{%
%    \end{macrocode}
%
% \pagebreak
%
%    \begin{macrocode}
          \edef\tfn@tfn{\arabic{footnote}}%
          \ltx@ifpackageloaded{hyperref}{%
            \tfn@hyperfootnotes{\xdef\tfn@thn{\arabic{Hfootnote}}}{}%
           }{}%
          \setcounter{footnote}{#1}
          \ltx@ifpackageloaded{hyperref}{%
            \tfn@hyperfootnotes{\setcounter{Hfootnote}{#1}}{}%
            \tfn@updateHyper%
           }{}%
          \footnotetext{%
            \hbox{\@textsuperscript{\normalfont #1}}%
            \ltx@ifpackageloaded{hyperref}{%
              \tfn@hyperfootnotes{\raisebox{\ht\strutbox}{%
                                         \hypertarget{tfn:#1}{}}}{}%
             }{}%
            \thinspace%
            #2}%
          \setcounter{footnote}{\tfn@tfn}%
          \ltx@ifpackageloaded{hyperref}{%
            \tfn@hyperfootnotes{\setcounter{Hfootnote}{\tfn@thn}}{}%
            \tfn@updateHyper%
           }{}%
        }%
     }%
  \fi%
%    \end{macrocode}
%
% For further footnotes in the table the \\
% |\ifx\tfn@footnotetableprint\tfn@footnotezero|\ldots|\fi| \\
% from the beginning of this |\tfn@footnotetext| macro must not be repeated,
% therefore we deactivate this by defining |\tfn@footnotetableprint| to
% something other than |0| (here |1| is used).
%
%    \begin{macrocode}
  \gdef\tfn@footnotetableprint{1}%
%    \end{macrocode}
%
% |\tfn@footnotetablecount| contains the negative number of footnote texts
% to be printed, therefore it is increase by one (i.\,e.~the negative number
% decreases, i.\,e.~$-3$ becomes~$-2$).
%
%    \begin{macrocode}
  \@tempcnta=\tfn@footnotetablecount\relax%
  \advance\@tempcnta by +1\relax%
  \xdef\tfn@footnotetablecount{\the\@tempcnta}%
%    \end{macrocode}
%
% When it is zero, all footnote texts have been placed, and\\
% |\tfn@footnotetableprint| can be reset to zero and\\
% |\tfn@tablefootnoteprintout| can be cleared.
%
%    \begin{macrocode}
  \ifx\tfn@footnotetablecount\tfn@footnotezero% all footnotes placed
    \gdef\tfn@footnotetableprint{0}%
    \global\let\tfn@tablefootnoteprintout\relax% brute force
 %\else there is at least one \footnotemark which needs \footnotetext
  \fi
}

%    \end{macrocode}
% \end{macro}
%
% \begin{macro}{\tfn@tablefootnoteprintout}
% When something shall be cleared or redefined, it is a good idea
% to define it. It~will be later filled with the foot note text to be printed.
%
%    \begin{macrocode}
\newcommand{\tfn@tablefootnoteprintout}{\relax}

%    \end{macrocode}
% \end{macro}
%
% \pagebreak
%
% \begin{macro}{\tablefootnote}
% |\tablefootnote[...]{...}| is the command to be used by the user instead of
% |\footnote[...]{...}| in the (sideways)table. It places a foot note mark via
% |\tfn@footnotemarks| and appends a |\tfn@footnotetext{#1}| to the\\
% |\tfn@tablefootnoteprintout| macro, and remembers that table foot notes
% have been used.
%
%    \begin{macrocode}
\newcommand{\tablefootnote}[2][]{%
  \tfn@footnotemarks[#1]%
  \ifthenelse{\isempty{#1}}{\relax}{%
    \ltx@GlobalPrependToMacro{\tfn@tablefootnoteprintout}{%
      \addtocounter{footnote}{+1}%
      \ltx@ifpackageloaded{hyperref}{%
        \tfn@hyperfootnotes{\addtocounter{Hfootnote}{+1}}{}%
        \tfn@updateHyper%
       }{}%
     }%
   }%
  \ltx@GlobalAppendToMacro{\tfn@tablefootnoteprintout}{%
    \tfn@footnotetext[#1]{#2}%
    \ifthenelse{\isempty{#1}}{\relax}{%
      \addtocounter{footnote}{-1}%
      \ltx@ifpackageloaded{hyperref}{%
        \tfn@hyperfootnotes{\addtocounter{Hfootnote}{-1}}{}%
        \tfn@updateHyper%
       }{}%
     }%
   }%
  \gdef\tfn@fnt{1}%
  }

%    \end{macrocode}
% \end{macro}
%
% \begin{macro}{\tfn@swtbox}
% For |sidewaystable|s the |\tfn@tablefootnoteprintout| is placed in a \\
% |\parbox|, which can be placed at the end of the |sidewaystable|:
%
%    \begin{macrocode}
\newcommand{\tfn@swtbox}{%
\ifvoid \footins%
\vskip 2\baselineskip%
\else%
\vskip \skip\footins%
\fi%
\parbox[b][][t]{\linewidth}{\raggedright\tfn@tablefootnoteprintout}%
}

%    \end{macrocode}
% \end{macro}
%
% |\tfn@tablefootnoteprintout| shall be executed after the end of the table environment.
% The \xpackage{etoolbox} package provides very easy syntax for this:
%
%    \begin{macrocode}
\IfFileExists{etoolbox.sty}{%
  \RequirePackage{etoolbox}[2011/01/03]%   v2.1
  \AfterEndEnvironment{table}{%
    \ltx@ifpackageloaded{footnotebackref}{%
      \LetLtxMacro{\tfncurrentmft}{\@makefntext}%
      \LetLtxMacro{\@makefntext}{\tablemakefntext}%
       \tfn@tablefootnoteprintout%
      \LetLtxMacro{\@makefntext}{\tfncurrentmft}%
     }{% footnotebackref not loaded
      \tfn@tablefootnoteprintout%
     }%
      \gdef\tfn@fnt{0}%
   }%
%    \end{macrocode}
%
% When the \xpackage{rotating} package has been loaded, we need to detect
% whether the |\tablefootnote| is placed inside of a |sidewaystable|.
%
%    \begin{macrocode}
  \ltx@ifpackageloaded{rotating}{%
    \AtBeginEnvironment{sidewaystable}{%
      \gdef\tfn@swt{1}%
    }
%    \end{macrocode}
%
% At the end of |sidewaystable| environments |\tfn@swtbox| shall be placed
% (if~there was any |\tablefootnote| in that |sidewaystable| at all).
% And |\tfn@swt| as well as |\tfn@fnt| must be reset to |0|.
%
%    \begin{macrocode}
    \AtEndEnvironment{sidewaystable}{%
      \ifx\tfn@fnt\tfn@footnotezero% \relax
      \else%
        \ltx@ifpackageloaded{footnotebackref}{%
          \LetLtxMacro{\tfncurrentmft}{\@makefntext}%
          \LetLtxMacro{\@makefntext}{\tablemakefntext}%
           \tfn@swtbox%
          \LetLtxMacro{\@makefntext}{\tfncurrentmft}%
         }{% footnotebackref not loaded
          \tfn@swtbox%
          }%
      \fi%
      \gdef\tfn@swt{0}%
      \gdef\tfn@fnt{0}%
     }%
   }{}% else \relax
}{%
%    \end{macrocode}
%
% When the \xpackage{etoolbox} package is not available (e.\,g. because of missing
% \eTeX{}) we~do a workaround:
%
%    \begin{macrocode}
  \PackageWarning{tablefootnote}{%
    Package etoolbox not available, therefore\MessageBreak%
    substituting commands \string\AtEndEnvironment\space %
    and\MessageBreak%
    \string\AfterEndEnvironment\space for tablefootnotes%
    \MessageBreak%
   }%
  \@ifpackagelater{letltxmacro}{2010/09/02}%
   {% >= 2010/09/02, OK
    \GlobalLetLtxMacro{\tfnorigtable}{\table}
    \GlobalLetLtxMacro{\tfnendorigtable}{\endtable}
%    \end{macrocode}
%
% |\GlobalLetLtxMacro| was introduced in version 2010/09/02 v1.4 of the
% \xpackage{letltxmacro} package.
%
%    \begin{macrocode}
   }{% else
     \PackageError{tablefootnote}{%
       Outdated version of letltxmacro package used%
      }{Version 2010/09/02 v1.4 or newer needed but not found!%
        \MessageBreak%
        Update the used letltxmacro package.}
        \LetLtxMacro{\tfnorigtable}{\table}
        \LetLtxMacro{\tfnendorigtable}{\endtable}
    }%
%    \end{macrocode}
%
% \pagebreak
%
%    \begin{macrocode}
  \ltx@ifpackageloaded{float}{%
    \@ifpackagelater{float}{2001/11/08}%   v1.3d
     {% >= 2001/11/08, OK
     }{% else
       \PackageWarningNoLine{tablefootnote}{%
         It is required version\MessageBreak
         2001/11/08 v1.3d (or later) of package float,\MessageBreak
         but only version\MessageBreak
        `\csname ver@float.sty\endcsname'\MessageBreak
        is available}%
     }%
   }{\relax}
  \renewenvironment{table}{%
    \ltx@ifpackageloaded{float}{%
%    \end{macrocode}
%
% When \xpackage{etoolbox.sty} was not found, and the \xpackage{float} package
% has been loaded, and the float specifier is \texttt{H}, then |\endtable| was
% let to |\float@endH| by the \xpackage{float} package (line~79 of
% \xpackage{float.sty}, 2001/11/08 v1.3d:\\
% |\expandafter\let\csname end#1\endcsname\float@endH|), i.\,e.~appending\\
% |\tfn@tablefootnoteprintout \gdef\tfn@fnt{0}| to |\endtable| is useless.
% Instead of this, it must be appended to |\float@endH|.\\
% (Why is this not necessary when \xpackage{etoolbox} is used?
% That is a good question, answers are welcome.)
%
%    \begin{macrocode}
      \ltx@LocalAppendToMacro{\float@endH}{%
        \ltx@ifpackageloaded{footnotebackref}{%
          \LetLtxMacro{\tfncurrentmft}{\@makefntext}%
          \LetLtxMacro{\@makefntext}{\tablemakefntext}%
           \tfn@tablefootnoteprintout%
          \LetLtxMacro{\@makefntext}{\tfncurrentmft}%
         }{% footnotebackref not loaded
          \tfn@tablefootnoteprintout%
         }%
          \gdef\tfn@fnt{0}%
       }%
     }{\relax}
    \tfnorigtable%
   }{%
     \tfnendorigtable%
     \ltx@ifpackageloaded{footnotebackref}{%
       \LetLtxMacro{\tfncurrentmft}{\@makefntext}%
       \LetLtxMacro{\@makefntext}{\tablemakefntext}%
        \tfn@tablefootnoteprintout%
       \LetLtxMacro{\@makefntext}{\tfncurrentmft}%
      }{% footnotebackref not loaded
       \tfn@tablefootnoteprintout%
      }%
     \gdef\tfn@fnt{0}%
    }%
%    \end{macrocode}
%
% \pagebreak
%
% The |sidewaystable|-environment must be treated separately:
%
%    \begin{macrocode}
  \ltx@ifpackageloaded{rotating}{%
    \let\tfnorigsidewaystable\sidewaystable
    \let\tfnendorigsidewaystable\endsidewaystable
    \renewenvironment{sidewaystable}{%
      \gdef\tfn@swt{1}%
      \tfnorigsidewaystable%
    }{%
      \ifx\tfn@fnt\tfn@footnotezero% \relax
      \else%
        \ltx@ifpackageloaded{footnotebackref}{%
          \LetLtxMacro{\tfncurrentmft}{\@makefntext}%
          \LetLtxMacro{\@makefntext}{\tablemakefntext}%
           \tfn@swtbox%
          \LetLtxMacro{\@makefntext}{\tfncurrentmft}%
         }{% footnotebackref not loaded
          \tfn@swtbox%
         }%
      \fi%
      \gdef\tfn@swt{0}%
      \gdef\tfn@fnt{0}%
      \tfnendorigsidewaystable%
     }%
   }{}%
}

%    \end{macrocode}
%
% \DescribeMacro{\AtBeginDocument}
% When the \xpackage{rotating} (for |sidewaystable|s) or the
% \xpackage{footnotebackref} (for back references for the footnotes)
% package is used, it must be loaded before the \xpackage{tablefootnote} package.
% Using both packages is also possible, in which case both must be loaded before
% the \xpackage{tablefootnote} package. If any one of those packages is loaded
% |\AtBeginDocument| after this check, the according error message will not be
% given, but packages really should be loaded before |\AtBeginDocument|.
%
%    \begin{macrocode}
\AtBeginDocument{%
  \ltx@ifpackageloaded{footnotebackref}{%
    \ifx\tfn@fnbr\tfn@footnotezero%
      \PackageError{tablefootnote}{%
        Package footnotebackref loaded after tablefootnote}{%
        When using both the footnotebackref and the tablefootnote %
        package,\MessageBreak%
        the footnotebackref package MUST be loaded before the %
        tablefootnote package!\MessageBreak%
       }%
    \fi%
   }{% else \relax
   }
  \ltx@ifpackageloaded{rotating}{%
    \ifx\tfn@rotating\tfn@footnotezero%
      \PackageError{tablefootnote}{%
        Package rotating loaded after tablefootnote}{%
        When using both the rotating and the tablefootnote package,%
        \MessageBreak%
        the rotating package MUST be loaded before the tablefootnote%
        \space package!\MessageBreak%
       }%
    \fi%
   }{% else \relax
   }
%    \end{macrocode}
%
% \pagebreak
%
% When the document is compiled with \LuaLaTeX{}, hyperlinks in rotated content
% will be misplaced, regardless of the use of the \xpackage{tablefootnote} package
% (or of tables or footnotes). The text to be printed will be OK.
% This problem of \LuaLaTeX{} (not \xpackage{tablefootnote}) was found and reported
% by \textsc{Arno Trautmann}, 2011/11/15 -- thanks!
%
%    \begin{macrocode}
  \ltx@ifpackageloaded{hyperref}{%
    \tfn@hyperfootnotes{
      \ifluatex%
        \PackageWarningNoLine{tablefootnote}{%
          LuaTeX and hyperref package used:\MessageBreak%
          Hyperreferences in rotated content\MessageBreak%
          will be misplaced\MessageBreak%
          regardless of tablefootnote package.\MessageBreak%
          The text to be printed will be OK%
         }%
      \fi%
     }{% else \relax
     }%
   }{% else \relax
   }%
%    \end{macrocode}
%
% When the \xpackage{footnotebackref} package is used but the footnotes are not
% hyperlinked, neither the back references nor the tablefootnotes will work.
%
%    \begin{macrocode}
  \ltx@ifpackageloaded{footnotebackref}{%
     \ltx@ifpackageloaded{hyperref}{%
        \tfn@hyperfootnotes{\relax}{% else
          \PackageError{tablefootnote}{%
             footnotebackref but not hyperfootnotes}{%
             The footnotebackref package is used\MessageBreak%
             but the hyperfootnotes option of the hyperref package%
             \MessageBreak%
             was set to false.\MessageBreak%
             Either enable hyperfootnotes or do not use footnotebackref.%
             \MessageBreak%
             Loading of the tablefootnote package will be aborted.%
           }%
          \endinput%
      }%
    }{\PackageError{tablefootnote}{footnotebackref but not hyperref}{%
        The footnotebackref package is used\MessageBreak%
        but not the hyperref package.\MessageBreak%
        Either load hyperref or do not use footnotebackref.%
        \MessageBreak%
        Loading of the tablefootnote package will be aborted.%
        }%
      \endinput%
    }%
  }{}
}

%    \end{macrocode}
%
%    \begin{macrocode}
%</package>
%    \end{macrocode}
%
% \pagebreak
%
% \section{Installation}
%
% \subsection{Downloads\label{ss:Downloads}}
%
% Everything is available on \CTAN{}, \url{http://www.ctan.org/},
% but may need additional packages themselves.\\
%
% \DescribeMacro{tablefootnote.dtx}
% For unpacking the |tablefootnote.dtx| file and constructing the documentation
% it is required:
% \begin{description}
% \item[-] \TeX{} Format \LaTeXe{}: \url{http://www.CTAN.org/}
%
% \item[-] document class \xpackage{ltxdoc}, 2007/11/11, v2.0u,
%           \url{http://www.ctan.org/pkg/ltxdoc}
%
% \item[-] package \xpackage{holtxdoc}, 2012/03/21, v0.24,
%           \url{http://www.ctan.org/pkg/holtxdoc}
%
% \item[-] package \xpackage{hypdoc}, 2011/08/19, v1.11,
%           \url{http://www.ctan.org/pkg/hypdoc}
% \end{description}
%
% \DescribeMacro{tablefootnote.sty}
% The |tablefootnote.sty| for \LaTeXe{} (i.\,e.~each document using
% the \xpackage{tablefootnote} package) requires:
% \begin{description}
% \item[-] \TeX{} Format \LaTeXe{}, \url{http://www.CTAN.org/}
%
% \item[-] package \xpackage{letltxmacro}, 2010/09/02, v1.4,
%           \url{http://www.ctan.org/pkg/letltxmacro}.
%
% \item[-] package \xpackage{ltxcmds}, 2011/11/09, v1.22,
%           \url{http://www.ctan.org/pkg/ltxcmds}
% \end{description}
%
% When the \xpackage{hyperref} package is used, also the \xpackage{ifluatex} package
% is needed, but it is already loaded automatically by the \xpackage{hyperref}
% package.\\
% When the \xpackage{etoolbox} package is available, it is used:
% \begin{description}
% \item[-] package \xpackage{etoolbox}, 2011/01/03, v2.1,
%           \url{http://www.ctan.org/pkg/etoolbox}.
% \end{description}
%
% \DescribeMacro{tablefootnote-example.tex}
% The |tablefootnote-example.tex| requires the same files as all
% documents using the \xpackage{tablefootnote} package and additionally:
% \begin{description}
% \item[-] class \xpackage{article}, 2007/10/19, v1.4h, from \xpackage{classes.dtx}:\\
%           \CTAN{macros/latex/base/classes.dtx}
%
% \item[-] package \xpackage{rotating}, 2009/03/28, v2.16a,
%           \url{http://www.ctan.org/pkg/rotating}
%
% \item[-] package \xpackage{float}, 2001/11/08, v1.3d,
%           \url{http://www.ctan.org/pkg/float}
%
% \item[-] package \xpackage{placeins}, 2005/04/18, v2.2,
%           \url{http://www.ctan.org/pkg/placeins}\\
%           for |\FloatBarrier|
%
% \item[-] package \xpackage{hyperref}, 2012/11/06, v6.83m,
%           \url{http://www.ctan.org/pkg/hyperref}
%
% \item[-] package \xpackage{tablefootnote}, 2014/01/26, v1.1c,\\
%           \url{http://www.ctan.org/pkg/tablefootnote}\\
%   (Well, it is the example file for this package, and because you
%    are reading the documentation for the \xpackage{tablefootnote} package,
%    it can be assumed that you already have some version of it~--
%    is it the current one?)
% \end{description}
%
% \DescribeMacro{Oberdiek}
% \DescribeMacro{holtxdoc}
% \DescribeMacro{ifluatex}
% \DescribeMacro{letltxmacro}
% \DescribeMacro{ltxcmds}
% All packages of \textsc{Heiko Oberdiek}'s bundle `oberdiek'
% (especially \xpackage{holtxdoc}, \xpackage{ifluatex}, \xpackage{letltxmacro},
% and \xpackage{ltxcmds}) are also available in a TDS compliant ZIP archive:\\
% \CTAN{install/macros/latex/contrib/oberdiek.tds.zip}.\\
% It is probably best to download and use this, because the packages in there
% are quite probably both recent and compatible among themselves.\\
%
% \pagebreak
%
% \DescribeMacro{hyperref}
% \noindent \xpackage{hyperref} is not included in that bundle and needs to be
% downloaded separately,\\
% \url{http://www.ctan.org/install/macros/latex/contrib/hyperref.tds.zip}.\\
%
% \DescribeMacro{M\"{u}nch}
% A hyperlinked list of my (other) packages can be found at
% \url{http://www.ctan.org/author/muench-hm}.\\
%
% \subsection{Package, unpacking TDS}
% \paragraph{Package.} This package is available on \CTAN{}
% \begin{description}
% \item[\CTAN{macros/latex/contrib/tablefootnote/tablefootnote.dtx}]\hspace*{0.1cm} \\
%       The source file.
% \item[\CTAN{macros/latex/contrib/tablefootnote/tablefootnote.pdf}]\hspace*{0.1cm} \\
%       The documentation.
% \item[\CTAN{macros/latex/contrib/tablefootnote/tablefootnote-example.pdf}]\hspace*{0.1cm} \\
%       The compiled example file, as it should look like.
% \item[\CTAN{macros/latex/contrib/tablefootnote/README}]\hspace*{0.1cm} \\
%       The README file.
% \end{description}
% There is also a tablefootnote.tds.zip available:
% \begin{description}
% \item[\CTAN{install/macros/latex/contrib/tablefootnote.tds.zip}]\hspace*{0.1cm} \\
%       Everything in \xfile{TDS} compliant, compiled format.
% \end{description}
% which additionally contains\\
% \begin{tabular}{ll}
% tablefootnote.ins & The installation file.\\
% tablefootnote.drv & The driver to generate the documentation.\\
% tablefootnote.sty & The \xext{sty}le file.\\
% tablefootnote-example.tex & The example file.
% \end{tabular}
%
% \bigskip
%
% \noindent For required other packages, please see the preceding subsection.
%
% \paragraph{Unpacking.} The \xfile{.dtx} file is a self-extracting
% \docstrip{} archive. The files are extracted by running the
% \xext{dtx} through \plainTeX{}:
% \begin{quote}
%   \verb|tex tablefootnote.dtx|
% \end{quote}
%
% \noindent About generating the documentation see paragraph~\ref{GenDoc} below.\\
%
% \paragraph{TDS.} Now the different files must be moved into
% the different directories in your installation TDS tree
% (also known as \xfile{texmf} tree):
% \begin{quote}
% \def\t{^^A
% \begin{tabular}{@{}>{\ttfamily}l@{ $\rightarrow$ }>{\ttfamily}l@{}}
%   tablefootnote.sty & tex/latex/tablefootnote/tablefootnote.sty\\
%   tablefootnote.pdf & doc/latex/tablefootnote/tablefootnote.pdf\\
%   tablefootnote-example.tex & doc/latex/tablefootnote/tablefootnote-example.tex\\
%   tablefootnote-example.pdf & doc/latex/tablefootnote/tablefootnote-example.pdf\\
%   tablefootnote.dtx & source/latex/tablefootnote/tablefootnote.dtx\\
% \end{tabular}^^A
% }^^A
% \sbox0{\t}^^A
% \ifdim\wd0>\linewidth
%   \begingroup
%     \advance\linewidth by\leftmargin
%     \advance\linewidth by\rightmargin
%   \edef\x{\endgroup
%     \def\noexpand\lw{\the\linewidth}^^A
%   }\x
%   \def\lwbox{^^A
%     \leavevmode
%     \hbox to \linewidth{^^A
%       \kern-\leftmargin\relax
%       \hss
%       \usebox0
%       \hss
%       \kern-\rightmargin\relax
%     }^^A
%   }^^A
%   \ifdim\wd0>\lw
%     \sbox0{\small\t}^^A
%     \ifdim\wd0>\linewidth
%       \ifdim\wd0>\lw
%         \sbox0{\footnotesize\t}^^A
%         \ifdim\wd0>\linewidth
%           \ifdim\wd0>\lw
%             \sbox0{\scriptsize\t}^^A
%             \ifdim\wd0>\linewidth
%               \ifdim\wd0>\lw
%                 \sbox0{\tiny\t}^^A
%                 \ifdim\wd0>\linewidth
%                   \lwbox
%                 \else
%                   \usebox0
%                 \fi
%               \else
%                 \lwbox
%               \fi
%             \else
%               \usebox0
%             \fi
%           \else
%             \lwbox
%           \fi
%         \else
%           \usebox0
%         \fi
%       \else
%         \lwbox
%       \fi
%     \else
%       \usebox0
%     \fi
%   \else
%     \lwbox
%   \fi
% \else
%   \usebox0
% \fi
% \end{quote}
% If you have a \xfile{docstrip.cfg} that configures and enables \docstrip{}'s
% \xfile{TDS} installing feature, then some files can already be in the right
% place, see the documentation of \docstrip{}.
%
% \subsection{Refresh file name databases}
%
% If your \TeX{}~distribution (\TeX live, \mikTeX{}, \dots{}) relies on file name
% databases, you must refresh these. For example, \teTeX{} users run
% \verb|texhash| or \verb|mktexlsr|.
%
% \subsection{Some details for the interested}
%
% \paragraph{Unpacking with \LaTeX{}.}
% The \xfile{.dtx} chooses its action depending on the format:
% \begin{description}
% \item[\plainTeX:] Run \docstrip{} and extract the files.
% \item[\LaTeX:] Generate the documentation.
% \end{description}
% If you insist on using \LaTeX{} for \docstrip{} (really,
% \docstrip{} does not need \LaTeX{}), then inform the autodetect routine
% about your intention:
% \begin{quote}
%   \verb|latex \let\install=y% \iffalse meta-comment
%
% File: tablefootnote.dtx
% Version: 2014/01/26 v1.1c
%
% Copyright (C) 2011 - 2014 by
%    H.-Martin M"unch <Martin dot Muench at Uni-Bonn dot de>
%
% This work may be distributed and/or modified under the
% conditions of the LaTeX Project Public License, either
% version 1.3c of this license or (at your option) any later
% version. This version of this license is in
%    http://www.latex-project.org/lppl/lppl-1-3c.txt
% and the latest version of this license is in
%    http://www.latex-project.org/lppl.txt
% and version 1.3c or later is part of all distributions of
% LaTeX version 2005/12/01 or later.
%
% This work has the LPPL maintenance status "maintained".
%
% The Current Maintainer of this work is H.-Martin Muench.
%
% This work consists of the main source file tablefootnote.dtx,
% the README, and the derived files
%    tablefootnote.sty, tablefootnote.pdf,
%    tablefootnote.ins, tablefootnote.drv,
%    tablefootnote-example.tex, tablefootnote-example.pdf.
%
% Distribution:
%    http://mirror.ctan.org/macros/latex/contrib/tablefootnote/tablefootnote.dtx
%    http://mirror.ctan.org/macros/latex/contrib/tablefootnote/tablefootnote.pdf
%    http://mirror.ctan.org/install/macros/latex/contrib/tablefootnote.tds.zip
%
% See http://www.ctan.org/pkg/tablefootnote (Catalogue)
%
% Unpacking:
%    (a) If tablefootnote.ins is present:
%           tex tablefootnote.ins
%    (b) Without tablefootnote.ins:
%           tex tablefootnote.dtx
%    (c) If you insist on using LaTeX
%           latex \let\install=y% \iffalse meta-comment
%
% File: tablefootnote.dtx
% Version: 2014/01/26 v1.1c
%
% Copyright (C) 2011 - 2014 by
%    H.-Martin M"unch <Martin dot Muench at Uni-Bonn dot de>
%
% This work may be distributed and/or modified under the
% conditions of the LaTeX Project Public License, either
% version 1.3c of this license or (at your option) any later
% version. This version of this license is in
%    http://www.latex-project.org/lppl/lppl-1-3c.txt
% and the latest version of this license is in
%    http://www.latex-project.org/lppl.txt
% and version 1.3c or later is part of all distributions of
% LaTeX version 2005/12/01 or later.
%
% This work has the LPPL maintenance status "maintained".
%
% The Current Maintainer of this work is H.-Martin Muench.
%
% This work consists of the main source file tablefootnote.dtx,
% the README, and the derived files
%    tablefootnote.sty, tablefootnote.pdf,
%    tablefootnote.ins, tablefootnote.drv,
%    tablefootnote-example.tex, tablefootnote-example.pdf.
%
% Distribution:
%    http://mirror.ctan.org/macros/latex/contrib/tablefootnote/tablefootnote.dtx
%    http://mirror.ctan.org/macros/latex/contrib/tablefootnote/tablefootnote.pdf
%    http://mirror.ctan.org/install/macros/latex/contrib/tablefootnote.tds.zip
%
% See http://www.ctan.org/pkg/tablefootnote (Catalogue)
%
% Unpacking:
%    (a) If tablefootnote.ins is present:
%           tex tablefootnote.ins
%    (b) Without tablefootnote.ins:
%           tex tablefootnote.dtx
%    (c) If you insist on using LaTeX
%           latex \let\install=y\input{tablefootnote.dtx}
%        (quote the arguments according to the demands of your shell)
%
% Documentation:
%    (a) If tablefootnote.drv is present:
%           (pdf)latex tablefootnote.drv
%           makeindex -s gind.ist tablefootnote.idx
%           (pdf)latex tablefootnote.drv
%           makeindex -s gind.ist tablefootnote.idx
%           (pdf)latex tablefootnote.drv
%    (b) Without tablefootnote.drv:
%           (pdf)latex tablefootnote.dtx
%           makeindex -s gind.ist tablefootnote.idx
%           (pdf)latex tablefootnote.dtx
%           makeindex -s gind.ist tablefootnote.idx
%           (pdf)latex tablefootnote.dtx
%
%    The class ltxdoc loads the configuration file ltxdoc.cfg
%    if available. Here you can specify further options, e.g.
%    use DIN A4 as paper format:
%       \PassOptionsToClass{a4paper}{article}
%
% Installation:
%    TDS:tex/latex/tablefootnote/tablefootnote.sty
%    TDS:doc/latex/tablefootnote/tablefootnote.pdf
%    TDS:doc/latex/tablefootnote/tablefootnote-example.tex
%    TDS:doc/latex/tablefootnote/tablefootnote-example.pdf
%    TDS:source/latex/tablefootnote/tablefootnote.dtx
%
%<*ignore>
\begingroup
  \catcode123=1 %
  \catcode125=2 %
  \def\x{LaTeX2e}%
\expandafter\endgroup
\ifcase 0\ifx\install y1\fi\expandafter
         \ifx\csname processbatchFile\endcsname\relax\else1\fi
         \ifx\fmtname\x\else 1\fi\relax
\else\csname fi\endcsname
%</ignore>
%<*install>
\input docstrip.tex
\Msg{******************************************************************}
\Msg{* Installation                                                   *}
\Msg{* Package: tablefootnote 2014/01/26 v1.1c Table foot notes (HMM) *}
\Msg{******************************************************************}

\keepsilent
\askforoverwritefalse

\let\MetaPrefix\relax
\preamble

This is a generated file.

Project: tablefootnote
Version: 2014/01/26 v1.1c

Copyright (C) 2011 - 2014 by
    H.-Martin M"unch <Martin dot Muench at Uni-Bonn dot de>

The usual disclaimer applies:
If it doesn't work right that's your problem.
(Nevertheless, send an e-mail to the maintainer
 when you find an error in this package.)

This work may be distributed and/or modified under the
conditions of the LaTeX Project Public License, either
version 1.3c of this license or (at your option) any later
version. This version of this license is in
   http://www.latex-project.org/lppl/lppl-1-3c.txt
and the latest version of this license is in
   http://www.latex-project.org/lppl.txt
and version 1.3c or later is part of all distributions of
LaTeX version 2005/12/01 or later.

This work has the LPPL maintenance status "maintained".

The Current Maintainer of this work is H.-Martin Muench.

This work consists of the main source file tablefootnote.dtx,
the README, and the derived files
   tablefootnote.sty, tablefootnote.pdf,
   tablefootnote.ins, tablefootnote.drv,
   tablefootnote-example.tex, tablefootnote-example.pdf.

In memoriam Tommy Muench + 2014/01/02.

\endpreamble
\let\MetaPrefix\DoubleperCent

\generate{%
  \file{tablefootnote.ins}{\from{tablefootnote.dtx}{install}}%
  \file{tablefootnote.drv}{\from{tablefootnote.dtx}{driver}}%
  \usedir{tex/latex/tablefootnote}%
  \file{tablefootnote.sty}{\from{tablefootnote.dtx}{package}}%
  \usedir{doc/latex/tablefootnote}%
  \file{tablefootnote-example.tex}{\from{tablefootnote.dtx}{example}}%
}

\catcode32=13\relax% active space
\let =\space%
\Msg{************************************************************************}
\Msg{*}
\Msg{* To finish the installation you have to move the following}
\Msg{* file into a directory searched by TeX:}
\Msg{*}
\Msg{*     tablefootnote.sty}
\Msg{*}
\Msg{* To produce the documentation run the file `tablefootnote.drv'}
\Msg{* through (pdf)LaTeX, e.g.}
\Msg{*  pdflatex tablefootnote.drv}
\Msg{*  makeindex -s gind.ist tablefootnote.idx}
\Msg{*  pdflatex tablefootnote.drv}
\Msg{*  makeindex -s gind.ist tablefootnote.idx}
\Msg{*  pdflatex tablefootnote.drv}
\Msg{*}
\Msg{* At least three runs are necessary e.g. to get the}
\Msg{*  references right!}
\Msg{*}
\Msg{* Happy TeXing!}
\Msg{*}
\Msg{************************************************************************}

\endbatchfile
%</install>
%<*ignore>
\fi
%</ignore>
%
% \section{The documentation driver file}
%
% The next bit of code contains the documentation driver file for
% \TeX{}, i.\,e., the file that will produce the documentation you
% are currently reading. It will be extracted from this file by the
% \texttt{docstrip} programme. That is, run \LaTeX{} on \texttt{docstrip}
% and specify the \texttt{driver} option when \texttt{docstrip}
% asks for options.
%
%    \begin{macrocode}
%<*driver>
\NeedsTeXFormat{LaTeX2e}[2011/06/27]
\ProvidesFile{tablefootnote.drv}%
  [2014/01/26 v1.1c Table foot notes (HMM)]
\documentclass{ltxdoc}[2007/11/11]% v2.0u
\usepackage{holtxdoc}[2012/03/21]%  v0.24
%% tablefootnote may work with earlier versions of LaTeX2e and those
%% class and package, but this was not tested.
%% Please consider updating your LaTeX, class, and package
%% to the most recent version (if they are not already the most
%% recent version).
\hypersetup{%
 pdfsubject={Footnotes in Tables (HMM)},%
 pdfkeywords={LaTeX, tablefootnote, footnote, table, H.-Martin Muench},%
 pdfencoding=auto,%
 pdflang={en},%
 breaklinks=true,%
 linktoc=all,%
 pdfstartview=FitH,%
 pdfpagelayout=OneColumn,%
 bookmarksnumbered=true,%
 bookmarksopen=true,%
 bookmarksopenlevel=3,%
 pdfmenubar=true,%
 pdftoolbar=true,%
 pdfwindowui=true,%
 pdfnewwindow=true%
}
\CodelineIndex
\hyphenation{printing docu-ment}
\gdef\unit#1{\mathord{\thinspace\mathrm{#1}}}%
\makeatletter
\@ifundefined{eTeX}{\gdef\eTeX{$\m@th \varepsilon $-\TeX }}{% else \relax
}
\makeatother
\begin{document}
  \DocInput{tablefootnote.dtx}%
\end{document}
%</driver>
%    \end{macrocode}
%
% \fi
%
% \CheckSum{584}
%
% \CharacterTable
%  {Upper-case    \A\B\C\D\E\F\G\H\I\J\K\L\M\N\O\P\Q\R\S\T\U\V\W\X\Y\Z
%   Lower-case    \a\b\c\d\e\f\g\h\i\j\k\l\m\n\o\p\q\r\s\t\u\v\w\x\y\z
%   Digits        \0\1\2\3\4\5\6\7\8\9
%   Exclamation   \!     Double quote  \"     Hash (number) \#
%   Dollar        \$     Percent       \%     Ampersand     \&
%   Acute accent  \'     Left paren    \(     Right paren   \)
%   Asterisk      \*     Plus          \+     Comma         \,
%   Minus         \-     Point         \.     Solidus       \/
%   Colon         \:     Semicolon     \;     Less than     \<
%   Equals        \=     Greater than  \>     Question mark \?
%   Commercial at \@     Left bracket  \[     Backslash     \\
%   Right bracket \]     Circumflex    \^     Underscore    \_
%   Grave accent  \`     Left brace    \{     Vertical bar  \|
%   Right brace   \}     Tilde         \~}
%
% \GetFileInfo{tablefootnote.drv}
%
% \begingroup
%   \def\x{\#,\$,\^,\_,\~,\ ,\&,\{,\},\%}%
%   \makeatletter
%   \@onelevel@sanitize\x
% \expandafter\endgroup
% \expandafter\DoNotIndex\expandafter{\x}
% \expandafter\DoNotIndex\expandafter{\string\ }
% \begingroup
%   \makeatletter
%     \lccode`9=32\relax
%     \lowercase{%^^A
%       \edef\x{\noexpand\DoNotIndex{\@backslashchar9}}%^^A
%     }%^^A
%   \expandafter\endgroup\x
%
% \DoNotIndex{\\,\,}
% \DoNotIndex{\documentclass,\usepackage,\ProvidesPackage,\begin,\end}
% \DoNotIndex{\MessageBreak,\message}
% \DoNotIndex{\NeedsTeXFormat,\DoNotIndex,\verb}
% \DoNotIndex{\def,\edef,\gdef,\global}
% \DoNotIndex{\ifx,\listfiles,\mathord,\mathrm}
% \DoNotIndex{\kvoptions,\SetupKeyvalOptions,\ProcessKeyvalOptions}
% \DoNotIndex{\bigskip,\space,\thinspace,\ldots}
% \DoNotIndex{\indent,\noindent,\newline,\linebreak,\pagebreak}
% \DoNotIndex{\textbf,\textit,\textsf,\texttt,\textsc,\textquotedblleft,\textquotedblright}
% \DoNotIndex{\plainTeX,\TeX,\LaTeX,\pdfLaTeX}
% \DoNotIndex{\chapter,\section,\the,\csname,\endscname}
% \DoNotIndex{\value,\arabic,\hline,\text,\caption,\centering,\clear,\url}
%
% \title{The \xpackage{tablefootnote} package}
% \date{2014/01/26 v1.1c}
% \author{H.-Martin M\"{u}nch\\\xemail{Martin.Muench at Uni-Bonn.de}}
%
% \maketitle
%
% \begin{abstract}
% This \LaTeX{} package provides the command |\tablefootnote| to be used
% in a \texttt{table} or \texttt{sidewaystable} environment,
% where |\footnote| will not work and when using |\footnotemark| and
% |\footnotetext| and adjusting the counters (including \texttt{Hfootnote})
% manually is either too much work or would not even work (\texttt{sidewaystable}).
% \end{abstract}
%
% \bigskip
%
% \noindent Disclaimer for web links: The author is not responsible for any contents
% referred to in this work unless he has full knowledge of illegal contents.
% If any damage occurs by the use of information presented there, only the
% author of the respective pages might be liable, not the one who has referred
% to these pages.
%
% \bigskip
%
% \noindent {\color{green} Save per page about $200\unit{ml}$ water,
% $2\unit{g}$ CO$_{2}$ and $2\unit{g}$ wood:\\
% Therefore please print only if this is really necessary.}
%
% \pagebreak
%
% \tableofcontents
%
% \pagebreak
%
% \section{Introduction\label{sec:Introduction}}
%
% \indent This \LaTeX{} package provides the command |\tablefootnote{...}| to be
% used in a \texttt{table} or \texttt{sidewaystable} environment. In those
% environments |\footnote{...}| would not work and would need to be replaced
% by |\footnotemark| in the \texttt{(sideways)\allowbreak table} and
% |\footnotetext{...}| after the \texttt{(sideways)table} environment ended.
% Additionally the counters for \texttt{footnote} and
% (when the \xpackage{hyperref} package is used) \texttt{Hfootnote}
% would need to be adjusted manually. Then still the hyperlinks
% in a \texttt{(sideways)table} would not work.
% When the \xpackage{footnotebackref} package is used,
% |\footnotemark| and |\footnotetext{...}| would need to be redefined.
% Just using |\tablefootnote{...}| (with optional argument for custom foot note
% marks) in the (sideways)table does all this automatically.
% (Compatibility with the \xpackage{footnotebackref} package
% was achieved without redefining |\footnotemark| or |\footnotetext|.)
% Redefining |\footnote| was \emph{not} done as to not create conflicts with
% other packages.
%
% \section{Usage}
%
% \indent Just load the package placing
% \begin{quote}
%   |\usepackage{tablefootnote}|
% \end{quote}
% \noindent in the preamble of your \LaTeXe{} source file and use\newline
% |\tablefootnote{...}| instead of |\footnote{...}| and\newline
% |\tablefootnote[...]{...}| instead of |\footnote[...]{...}|
% in (sideways)\allowbreak tables.\\
% When the \xpackage{footnotebackref} package is used,
% it must be loaded before the \xpackage{tablefootnote} package
% and the \xpackage{hyperrefpackage} with option |hyperfootnotes=true|
% must be loaded, too. (Backreference{-}links without links do not make sense.)\\
% When the \xpackage{rotating} package (for |sidewaystable|s) is used,
% it must be loaded before the \xpackage{tablefootnote} package and it must be
% a recent version (v2.16a, 2009/03/28, or newer).\\
% If |\tablefootnote{...}| is used in a tabular environment, which is inside
% of a (sideways)table environment, everything is fine, but when the tabular
% environment is not inside of a (sideways)table, the \textbf{table}footnote
% will not work.\\
%
% Pages with tables with footnotes, which are rotated with the \xpackage{lscape}-
% or with the \xpackage{pdflscape}-package (|\begin{landscape}|, table, |\end{landscape}|),
% are possible. But when the \xpackage{footmisc}-package is used, \xpackage{(pdf)lscape}
% should be loaded before \xpackage{footmisc}.\\
%
% When the \xpackage{footmisc}-package is used with option \texttt{para},
% this is regarded with the exception of a \texttt{sidewaystable}. There the footnotes
% are printed one below the other. (Well, this is better than no footnote in a
% \texttt{sidewaystable} at all, isn't it?)\\
%
% When the (sideways)table floats, the footnotes are set and (when \xpackage{hyperref}
% is used) hyperlinked, but they are not automatically adapted when the table floats
% over/under another footnote. Thus either do not use a footnote between original and
% float position of the (sideways)table, or place the (sideways)table in
% \textquotedblleft here\textquotedblright\ position. |\clear(double)page|, |h(!)|,
% |H|~from the \xpackage{float} package (\url{http://www.ctan.org/pkg/float}),
% or |\FloatBarrier| from the \xpackage{picins} package
% (\url{http://www.ctan.org/pkg/picins}) might help, too.
% (Or after finishing the document, move the (sideways)table in the source code
% near the position where it floats to or use the optional footnote marks.)
%
% \pagebreak
%
% \section{Alternatives\label{sec:Alternatives}}
%
% \begin{description}
% \item[-] The \xpackage{longtable} package provides the \texttt{longtable} environment
%   as replacement for the combined \texttt{table} and \texttt{tabular} environments.
%   Footnotes are real footnotes (not just tablenotes), are continuously numbered and
%   hyperlinked (when using the \xpackage{hyperref} package), and the hyperlinks really
%   work. As drawback the appearance of the caption changes slightly (e.\,g.~distance
%   to the table, width of the caption), which can probably be changed back manually.
%   Furthermore, longtables are meaned to break over more than one page. If that is not
%   wished, it must be prevented by |\nopagebreak|-commands and by ending the longtable
%   lines with |\\*| instead of |\\|. longtables do not float. (Therefore using the
%   \xpackage{tablefootnote} package and |\FloatBarrier| from the \xpackage{picins} package
%   before and after the table environment is similar~- but \xpackage{tablefootnote} does not
%   change the table-caption!) \texttt{sidewaystable} does not work with it.\\
%   \url{http://www.ctan.org/pkg/longtable}
%
% \item[-] The \xpackage{supertabular} package provides the \texttt{mpsupertabular}
%   environment as replacement for the combined \texttt{table} and \texttt{tabular}
%   environments. Footnotes are just tablenotes (with working hyperlinks when using the
%   \xpackage{hyperref} package), i.\,e. numbered a, b, c and placed below the table
%   and not at the end of the page. Therefore there is no float problem (because the
%   tablenotes numbering is not included in the continuous numbering of the footnotes).
%   Placing the \texttt{supertabular} inside of a \texttt{sidewaystable} breaks the
%   hyperlinks to the tablenotes.\\
%   \url{http://www.ctan.org/pkg/supertabular}
%
% \item[-] The \xpackage{ctable} package has its very own notation for defining tables.
%   It can create tablenotes and sideways-tables. The tablenotes are not automatically
%   hyperlinked. The \texttt{ctable}s float. Because the tablenotes numbering is not
%   included in the continuous numbering of the footnotes there is no float problem.\\
%   \url{http://www.ctan.org/pkg/ctable}
%
% \item[-] The \xpackage{footnote} package provides |\makesavenoteenv{table}|. After
%   loading the package and using that command in the preamble, in tables |\footnote{...}|
%   can be used. Using |\makesavenoteenv{tabular}| and \\
%   |\makesavenoteenv{sidewaystable}|
%   is possible, but it neither solves the float problem, nor do the created hyperlinks
%   work (i.\,e.~they aim at wrong locations). The \xpackage{mdwtab} from the same bundle
%   is incompatible with other table-related packages (e.\,g. \xpackage{supertabular},
%   \xpackage{array}) and not $100\,\%$ compatible with the \texttt{tabular} environment.\\
%   \url{http://www.ctan.org/pkg/footnote}\\
%   \url{http://www.ctan.org/pkg/mdwtab}
%
% \item[-] The \xpackage{tabularx} package does produce footnotes for sidewaystables,
%   but uses a, b, c instead of 1, 2, 3. The hyperlinks to the footnotes do not work.
%   Because the footnotes numbering is not included in the continuous numbering of
%   the other footnotes there is no float problem.\\
%   \url{http://www.ctan.org/pkg/tabularx}
%
% \item[-] Placing a tabular inside a minipage inside a table produces tablenotes.
%   Therefore there is no float problem (because the footnotes are not continuously
%   numbered). The hyperlinks to the table notes indeed work.
%
% \pagebreak
%
% \item[-] The \xpackage{threeparttable} package creates tablenotes again. Therefore
%   there is no float problem (because the tablenotes are not continuously numbered with
%   the footnotes). There are no hyperlinks to the table notes (at least not automatically).
%   Using |sidewaystable| (with table notes) works.\\
%   \url{http://www.ctan.org/pkg/threeparttable}
%
% \item[-] The \xpackage{threeparttablex} package creates tablenotes again. Therefore there
%   is no float problem (because the tablenotes are not continuously numbered with the
%   footnotes). With option \texttt{referable} the tablenotes are hyperlinked. Use of
%   a |sidewaystable| (with table notes) did not work for me. When using the
%   \texttt{referable} option according to the example in the \xpackage{threeparttablex}
%   manual the \xpackage{longtable} package is used, therefore that package could be used
%   directly without \xpackage{threeparttablex} (see above).\\
%   \url{http://www.ctan.org/pkg/threeparttablex}
%
% \item[-] One can manually use |\footnotemark| in the table and |\footnotetext{...}|
%  after the table environment ended and manually change the |footnote| and
%  (when the \xpackage{hyperref} package is used) |Hfootnote| counters and needs
%  to insert
%  \begin{verbatim}
%     \makeatletter
%     \global\let\Hy@saved@currentHref\@currentHref
%     \hyper@makecurrent{Hfootnote}%
%     \global\let\Hy@footnote@currentHref\@currentHref
%     \global\let\@currentHref\Hy@saved@currentHref
%     \makeatother
%  \end{verbatim}
%  before each |\footnotetext{...}|, but with more than very few footnotes this can become
%  much work (and is prone to produce errors). And this does not even work with
%  sidewaystables. (And it does not work with the \xpackage{footnotebackref} package
%  without redefining |\footnotemark| and |\footnotetext{...}|.)
% \end{description}
%
% \bigskip
%
% \noindent (You programmed or found another alternative,
%  which is available at \CTAN{}?\\
%  OK, send an e-mail to me with the name, location at \CTAN{},
%  and a short notice, and I will probably include it in
%  the list above.)
%
% \newpage
%
% \section{Example}
%
%    \begin{macrocode}
%<*example>
\documentclass[british]{article}[2007/10/19]% v1.4h
%%%%%%%%%%%%%%%%%%%%%%%%%%%%%%%%%%%%%%%%%%%%%%%%%%%%%%%%%%%%%%%%%%%%%
\usepackage{float}[2001/11/08]%    v1.3d
\usepackage{placeins}[2005/04/18]% v2.2  ; for \FloatBarrier
\usepackage{rotating}[2009/03/28]% v2.16a; for sidewaystable-environment
\usepackage[%
 hyperfootnotes=true,%
 extension=pdf,%
 plainpages=false,%
 pdfpagelabels=true,%
 hyperindex=false,%
 pdflang={en},%
 pdftitle={tablefootnote package example},%
 pdfauthor={H.-Martin Muench},%
 pdfsubject={Example for the tablefootnote package},%
 pdfkeywords={LaTeX, tablefootnote, footnote, table, H.-Martin Muench},%
 % pdfview=FitH and FitBH do not work: hyperlinks in sidewaystables
 % do no lead to the footnotes, due to a bug in pdfTeX,
 % computing wrong anchor coordinates (Heiko Oberdiek, 29. October 2011)
 % pdfview=Fit, FitV, FitR, FitB, FitBV work
 % print is OK for all those options
 pdfstartview=FitH,%
 pdfpagelayout=OneColumn%
]{hyperref}[2012/11/06]% v6.83m
 % Due to the urls used in the example, either the hyperref or the url
 % package are needed (or the urls must be removed before compiling).

\usepackage{footnotebackref}[2012/07/01]% v1.0
\usepackage{tablefootnote}[2014/01/26]%   v1.1c

\gdef\unit#1{\mathord{\thinspace\mathrm{#1}}}%
\listfiles
\begin{document}
\pagenumbering{arabic}
\section*{Example for tablefootnote}

This example demonstrates the use of package\newline
\textsf{tablefootnote}, v1.1c as of 2014/01/26 (HMM).\newline
There were no options used. (The package provides no options.)\newline

\textbf{The \texttt{tablefootnote-example.tex} needs to be compiled
at least twice to get the references right!}\newline

If the etoolbox-package is found, it is automatically used.\newline

For more details please see the documentation!\newline

\noindent Save per page about $200\unit{ml}$ water,
$2\unit{g}$ CO$_{2}$ and $2\unit{g}$ wood:\newline
Therefore please print only if this is really necessary.\newline

Here is some text.\footnote{And this is a text footnote.}\newline

Tables \ref{tab.symbol}, \ref{tab.normal}, \ref{tab.another} and
\ref{tab.floatH} show normal tables, table~\ref{tab.sideways} depicts
a sidewaystable. Table~\ref{tab.floatH} uses the float
specifier~\texttt{H} from the float package.\newline

\texttt{Hyperref} option \verb|pdfview=FitH| and \verb|FitBH| do
not work due to a bug in pdf\TeX{}, computing wrong
anchor coordinates (\textsc{Heiko Oberdiek}, 29.~October 2011).
Depending on used pdf-viewer, hyperlinks in sidewaystables lead
e.\,g.~at the end of the document, not at the footnote.
\verb|pdfview=Fit|, \verb|FitV|, \verb|FitR|, \verb|FitB|,
\verb|FitBV| work, print is OK for all those options.

\bigskip

\listoftables

\pagebreak

\renewcommand{\thefootnote}{\fnsymbol{footnote}}
\verb|\renewcommand{\thefootnote}{\fnsymbol{footnote}}|
causes footnote{-}symbol{-}footnotes, which are possible
(see Table~\ref{tab.symbol}).

\begin{table}
\centering
\begin{tabular}{ccc}
Another\tablefootnote{A table footnote.} & %
text\tablefootnote{Another table footnote.} & %
in a table\tablefootnote{A third table footnote.}
\end{tabular}
\caption[A footnotesymbol table]{%
A table with footnote-symbol-footnotes.\label{tab.symbol}}
\end{table}

Some text.\footnote{A text footnote.}

\renewcommand{\thefootnote}{\arabic{footnote}}
\verb|\renewcommand{\thefootnote}{\arabic{footnote}}|
switches back to normal footnote numbers again.

\pagebreak

\begin{table}
\centering
\begin{tabular}{ccc}
Some\tablefootnote{A table footnote.} & %
text\tablefootnote[99]{A table footnote with custom footnote number.} & %
in a table\tablefootnote{A third table footnote.}
\end{tabular}
\caption[A table]{A normal table.\label{tab.normal}}
\end{table}

Some text.\footnote{Another text footnote.}

\pagebreak

%    \end{macrocode}
%
% \pagebreak
%
%    \begin{macrocode}
More text.\footnote{And yet another text footnote.}

\begin{table}[t]
\centering
\begin{tabular}{|c|c|c|}
\hline
Another\tablefootnote{A $3^{rd}$ table footnote.} & %
text\tablefootnote{Another $3^{rd}$ table footnote.} & %
in a table\tablefootnote{A $3^{rd}$ third table footnote.}\\ \hline
\end{tabular}
\caption[Another table]{Another table (third one)\label{tab.another}}
\end{table}

Please note that Table~\ref{tab.another} floated to the top of the page.
While the footnotes are set and (when hyperref is used) hyperlinked,
they are not automatically adapted. Thus either do not use a footnote
at the same page before the table, or place the table in
\textquotedblleft here\textquotedblright\ or
\textquotedblleft bottom\textquotedblright\ position.
\verb|\clear(double)page|, \verb|h(!)|, \verb|H|~from the
\texttt{float} package, or \verb|\FloatBarrier| from the
\texttt{picins} package might help, too. (Or move the table in the
source code near the position where it floats to
or use the optional footnote marks.)

Table~\ref{tab.floatH} (page~\pageref{tab.floatH}) uses float specifier
\texttt{H} from the float package and does not float.

Some text.\footnote{This is just another text footnote.}

\pagebreak

\FloatBarrier

\begin{sidewaystable}
\centering%
\begin{tabular}{ccc}
Text\tablefootnote{Please rotate the view for testing the %
hyperlinks.} & %
in a\tablefootnote[98]{Another sidewaystable footnote %
with optional footnote mark.} & %
sidewaystable%
\tablefootnote{Sidewaystable-environment provided by %
\url{http://www.ctan.org/pkg/rotating} package.}
\end{tabular}
\caption[A sideways table]{A table in the \texttt{sideways} %
environment\label{tab.sideways}}%
\end{sidewaystable}%

\FloatBarrier

\pagebreak

%    \end{macrocode}
%
% \pagebreak
%
%    \begin{macrocode}
A last table, here with float specifier \texttt{H} from the
float\footnote{\url{http://www.ctan.org/pkg/float}} package.

\begin{table}[H]
\centering
\begin{tabular}{ccc}
Another\tablefootnote{A $5^{th}$ table footnote.} & %
text\tablefootnote{Another $5^{th}$ table footnote.} & %
in a table\tablefootnote{A $5^{th}$ third table footnote.}
\end{tabular}
\caption[A last table]{A very last table\label{tab.floatH}}
\end{table}

Some text.\footnote{This is just another text footnote.}

\pagebreak

End of the example for the
tablefootnote\footnote{\url{http://www.ctan.org/pkg/tablefootnote}}
package.

\end{document}
%</example>
%    \end{macrocode}
%
% \StopEventually{}
%
% \newpage
%
% \section{The implementation}
%
% We start off by checking that we are loading into \LaTeXe{} and
% announcing the name and version of this package.
%
%    \begin{macrocode}
%<*package>
%    \end{macrocode}
%
%    \begin{macrocode}
\NeedsTeXFormat{LaTeX2e}[2011/06/27]
\ProvidesPackage{tablefootnote}[2014/01/26 v1.1c
              Table foot notes (HMM)]
%    \end{macrocode}
%
% A short description of the \xpackage{tablefootnote} package:
%
%    \begin{macrocode}
%% Provides the \tablefootnote[...]{...} command
%% for footnotes in (sideways)tables.

%    \end{macrocode}
%
% We need the \xpackage{ltxcmds} and \xpackage{letltxmacro} packages
% by \textsc{Heiko Oberdiek}:
%
%    \begin{macrocode}
\RequirePackage{ltxcmds}[2011/11/09]%     v1.22
\RequirePackage{letltxmacro}[2010/09/02]% v1.4
\RequirePackage{xifthen}[2009/04/17]%     v1.3
%    \end{macrocode}
%
% A last information for the user:
%
%    \begin{macrocode}
%% tablefootnote may work with earlier versions of LaTeX and those
%% packages, but this was not tested. Please consider updating
%% your LaTeX and packages to the most recent version
%% (if they are not already the most recent version).

%    \end{macrocode}
%
% \indent See subsection~\ref{ss:Downloads} about how to get them.\\
%
% When the \xpackage{rotating} package (for |sidewaystable|s) is used,
% it must be loaded before the \xpackage{tablefootnote} package and it must be
% a recent version:
%
%    \begin{macrocode}
\@ifpackageloaded{rotating}{%
  \gdef\tfn@rotating{1}
  \@ifpackagelater{rotating}{2009/03/28}% v2.16a
   {% >= 2009/03/28, OK
   }{% else
     \PackageWarningNoLine{tablefootnote}{%
       It is required version\MessageBreak%
       2009/03/28 v2.16a (or later) of package rotating,\MessageBreak%
       but only version\MessageBreak%
      `\csname ver@rotating.sty\endcsname'\MessageBreak%
      is available}%
   }%
 }{% else
   \gdef\tfn@rotating{0}%
 }

%    \end{macrocode}
%
% There are no options to be processed.\\
%
% We need some definitions:
%
%    \begin{macrocode}
\def\tfn@footnotetablecount{0}
\def\tfn@footnotetableprint{0}
\def\tfn@footnotezero{0}
\def\tfn@fnt{0}
\def\tfn@swt{0}%       _s_ide_w_ays_t_able
\def\tfn@fmpw{\relax}% f_oot_m_isc_p_ara_w_arn-switch
\def\tfn@fnbr{0}%      f_oot_n_ote_b_ack_r_ef(erence package)

%    \end{macrocode}
%
% \begin{macro}{\tfn@hyperfootnotes}
% We need a wrapper for the used |\ifHy@hyperfootnotes|,
% which is not defined when the \xpackage{hyperref} package has not been loaded:
%
%    \begin{macrocode}
\DeclareRobustCommand{\tfn@hyperfootnotes}[2]{%
  \ifHy@hyperfootnotes%
    #1%
  \else%
    #2%
  \fi%
  }

%    \end{macrocode}
% \end{macro}
%
% \begin{macro}{\tablemakefntext}
% |\tablemakefntext| makes the footnotetext for a footnote in a table.
%
%    \begin{macrocode}
\ltx@ifpackageloaded{footnotebackref}{%
  \gdef\tfn@fnbr{1}%
  \ifx\FootnoteBackref@symbol\empty%
    \ifFootnoteBackref@numberlinked%
      \newcommand\tablemakefntext[1]{%
        \noindent\makebox[1.634em][r]{%
         \mbox{\textsuperscript{\normalfont%
           \hyperlink{tfn:m\theHfootnote}{\@thefnmark}%
        }}}%
        \ #1%
      }
    \else
      \newcommand\tablemakefntext[1]{%
        \noindent\makebox[1.8em][r]{%
          \mbox{\textsuperscript{\normalfont%
            \@thefnmark%
        }}}%
        #1%
      }
    \fi
  \else
    \newcommand\tablemakefntext[1]{%
      \noindent\makebox[1.8em][r]{%
        \mbox{\textsuperscript{\normalfont%
          \ifFootnoteBackref@numberlinked%
            \hyperlink{tfn:m\theHfootnote}{\@thefnmark}\ %
          \else%
            \@thefnmark%
          \fi%
      }}}%
      \hyperlink{tfn:m\theHfootnote}{\FootnoteBackref@symbol}%
      #1%
    }
  \fi
}{%
  \newcommand\tablemakefntext[1]{%
    \noindent\makebox[1.8em][r]{%
      \mbox{\textsuperscript{\normalfont%
        \@thefnmark%
    }}}%
    #1%
  }
}

%    \end{macrocode}
%
% \end{macro}
%
% \pagebreak
%
% \begin{macro}{\swtablemakefntext}
%    \begin{macrocode}
\newcommand\swtablemakefntext[1]{\noindent\ #1}

%    \end{macrocode}
% \end{macro}
%
% \begin{macro}{\tfn@footnotemarks}
% |\tfn@footnotemarks| uses the |\footnotemark| command to place a foot note mark
% and decreases the |\tfn@footnotetablecount| (i.\,e.~increases the negative value)
% of the number of footnote(marks) in the table. If the \xpackage{footnotebackref}
% package is used, it is also necessary to create an according |\hypertarget| for
% the back reference.
%
%    \begin{macrocode}
\newcommand{\tfn@footnotemarks}[1][]{%
  \hbox{%
    \ifthenelse{\isempty{#1}}%
      {\footnotemark%
       \ltx@ifpackageloaded{footnotebackref}%
         {\raisebox{\ht\strutbox}{\hypertarget{tfn:m\theHfootnote}{}}}%
         {% footnotebackref not loaded
         }%
      }%
      {\ltx@ifpackageloaded{hyperref}%
        {\tfn@hyperfootnotes{%
            \hyperlink{tfn:#1}{\footnotemark[#1]}%
          }{\footnotemark[#1]}}%
        {\footnotemark[#1]}%
       \ltx@ifpackageloaded{footnotebackref}%
        {\raisebox{\ht\strutbox}{\hypertarget{tfn:m#1}{}}}{}%
      }%
   }%
  \@tempcnta=\tfn@footnotetablecount\relax%
  \advance\@tempcnta by -1\relax%
  \xdef\tfn@footnotetablecount{\the\@tempcnta}%
}

%    \end{macrocode}
% \end{macro}
%
% \begin{macro}{\tfn@footmiscparawarn}
% When the \xpackage{footmisc} package is used with option |para|,
% we need to issue a warning. (Any idea how to resolve this issue?)\\
% It must be given inside of it's own command, because |\ifFN@para| is not
% defined when \xpackage{footmisc} has not been loaded, and then the
% |\fi| would be a |\fi| without |\if| (and \TeX{} does not like single
% |\fi|s).
%
%    \begin{macrocode}
\newcommand{\tfn@footmiscparawarn}{%
  \ifFN@para%
    \PackageWarning{tablefootnote}{%
      Package footmisc with option para used.\MessageBreak%
      Unfortunately package tablefootnote is not yet\MessageBreak%
      able to regard this in sidewaystables.\MessageBreak%
      Here the footnotes are printed one below\MessageBreak%
      the other. Sorry!\MessageBreak%
     }%
  \fi%
  }

%    \end{macrocode}
% \end{macro}
%
% \begin{macro}{\tfn@updateHyper}
% |\tfn@updateHyper| updates some \xpackage{hyperref} internals.
%
%    \begin{macrocode}
\newcommand{\tfn@updateHyper}{%
  % The code of this command is partly from the
  % hyperref package by Heiko Oberdiek, 2011/10/01, v6.82j.
  % Newer versions of that package are available at CTAN.org.
  \global\let\Hy@saved@currentHref\@currentHref%
  \tfn@hyperfootnotes{\hyper@makecurrent{Hfootnote}}{}%
  \global\let\Hy@footnote@currentHref\@currentHref%
  \global\let\@currentHref\Hy@saved@currentHref%
  }

%    \end{macrocode}
% \end{macro}
%
% \begin{macro}{\tfn@footnotetext}
% |\tfn@footnotetext| contains the |\footnotetext[...]{...}| command and handles
% the numbering. |\tfn@footnotetext| will be executed after the table.
% The foot note counter has been increased automatically with each |\footnotemark|,
% therefore it must be reset to its value before the table began.
%
%    \begin{macrocode}
\newcommand{\tfn@footnotetext}[2][]{%
  \ifx\tfn@footnotetableprint\tfn@footnotezero%
    % This is the first footnotetext after the (sideways)table.
    \addtocounter{footnote}{\tfn@footnotetablecount}%
%    \end{macrocode}
%
% When the \xpackage{hyperref} package is used, also its |Hfootnote| counter must
% be reset (when the footnotes are hyperlinked).
%
%    \begin{macrocode}
    \ltx@ifpackageloaded{hyperref}{%
      \tfn@hyperfootnotes{%
        \addtocounter{Hfootnote}{\tfn@footnotetablecount}}{}%
      \tfn@updateHyper%
     }{}%
  \fi%
%    \end{macrocode}
%
% For issuing the text for the next footnote the footnote-counter
% (and |Hfootnote| counter in case of links) are increased by one.
%
%    \begin{macrocode}
  \addtocounter{footnote}{+1}%
  \ltx@ifpackageloaded{hyperref}{%
    \tfn@hyperfootnotes{\addtocounter{Hfootnote}{+1}}{}%
%    \end{macrocode}
%
% The \xpackage{hyperref} package was not expecting this kind of change of the
% |(H)footnote| numbers, therefore some code-lines are missing here which can be
% found at other places in that same package and are called by this command.
%
%    \begin{macrocode}
    \tfn@updateHyper%
%    \end{macrocode}
%
% When \xpackage{hyperref} is not loaded, nothing is to be done here.
%
%    \begin{macrocode}
   }{}%
%    \end{macrocode}
%
% The |\footnotetext[...]{...}| places the foot note text as usual.
%
%    \begin{macrocode}
  \ifx\tfn@swt\tfn@footnotezero%
    \ifthenelse{\isempty{#1}}%
      {\footnotetext{#2}}%
      {\ltx@ifpackageloaded{hyperref}%
         {\tfn@hyperfootnotes{%
             \footnotetext[#1]{%
               \raisebox{\ht\strutbox}{\hypertarget{tfn:#1}{}}#2}%
           }{\footnotetext[#1]{#2}%
           }%
         }{\footnotetext[#1]{#2}}%
      }%
  \else%
%    \end{macrocode}
%
% If there is more than one tablefootnote in the |sidewaystable|, \\
% the |\tfn@footmiscparawarn| shall be given,
%
%    \begin{macrocode}
    \ifx\tfn@footnotetableprint\tfn@footnotezero%
      \gdef\tfn@fmpw{\tfn@footmiscparawarn}%
    \else%
      \ltx@ifpackageloaded{footmisc}{\tfn@fmpw}{\relax}%
%    \end{macrocode}
%
% but we do not want to repeat the message for a third(, forth,\ldots)
% tablefootnote in the |sidewaystable|.
%
%    \begin{macrocode}
      \gdef\tfn@fmpw{\relax}%
    \fi%
%    \end{macrocode}
%
% |\@makefnmark| is \\
% |\hbox {\@textsuperscript {\normalfont \@thefnmark }}|, \\
% but according to |\show\@thefnmark| here |\@thefnmark| is
% |{\protect \itshape}| instead of |\thefootnote| (Why?!).\\
% If the \xpackage{footnotebackref} package is used,
% back references must be created for the table footnotes, too.
%
%    \begin{macrocode}
    \ltx@ifpackageloaded{footnotebackref}{%
      \LetLtxMacro{\tfncurrentmft}{\@makefntext}%
      \LetLtxMacro{\@makefntext}{\swtablemakefntext}%
      \ifthenelse{\isempty{#1}}%
        {\footnotetext{%
           \raisebox{\ht\strutbox}{\hypertarget{tfn:\theHfootnote}{}}%
           \hbox{\@textsuperscript{\normalfont%
              \hyperlink{tfn:m\theHfootnote}{\thefootnote}}}\thinspace%
         \ifx\FootnoteBackref@symbol\empty%
         \else\hyperlink{tfn:m\theHfootnote}{\FootnoteBackref@symbol}%
         \fi%
         #2}%
       }{\edef\tfn@tfn{\arabic{footnote}}%
         \edef\tfn@thn{\arabic{Hfootnote}}%
         \setcounter{footnote}{#1}%
         \setcounter{Hfootnote}{#1}%
         \tfn@updateHyper%
         \footnotetext{%
           \raisebox{\ht\strutbox}{\hypertarget{tfn:#1}{}}%
           \hbox{\@textsuperscript{\normalfont%
                  \hyperlink{tfn:m#1}{#1}}}%
           \thinspace%
           \ifx\FootnoteBackref@symbol\empty%
           \else%
             \raisebox{\ht\strutbox}{%
               \hyperlink{tfn:m#1}{\FootnoteBackref@symbol}}%
           \fi%
           #2%
          }%
         \setcounter{footnote}{\tfn@tfn}%
         \setcounter{Hfootnote}{\tfn@thn}%
         \tfn@updateHyper%
       }%
      \LetLtxMacro{\@makefntext}{\tfncurrentmft}%
     }{% footnotebackref not loaded
       \ifthenelse{\isempty{#1}}%
         {\footnotetext{%
            \hbox{\@textsuperscript{\normalfont\thefootnote}}%
            \ltx@ifpackageloaded{hyperref}%
              {\tfn@hyperfootnotes{%
                 \raisebox{\ht\strutbox}{%
                   \hypertarget{tfn:\theHfootnote}{}}}{}%
              }{}%
            \thinspace%
            #2}%
        }{%
%    \end{macrocode}
%
% \pagebreak
%
%    \begin{macrocode}
          \edef\tfn@tfn{\arabic{footnote}}%
          \ltx@ifpackageloaded{hyperref}{%
            \tfn@hyperfootnotes{\xdef\tfn@thn{\arabic{Hfootnote}}}{}%
           }{}%
          \setcounter{footnote}{#1}
          \ltx@ifpackageloaded{hyperref}{%
            \tfn@hyperfootnotes{\setcounter{Hfootnote}{#1}}{}%
            \tfn@updateHyper%
           }{}%
          \footnotetext{%
            \hbox{\@textsuperscript{\normalfont #1}}%
            \ltx@ifpackageloaded{hyperref}{%
              \tfn@hyperfootnotes{\raisebox{\ht\strutbox}{%
                                         \hypertarget{tfn:#1}{}}}{}%
             }{}%
            \thinspace%
            #2}%
          \setcounter{footnote}{\tfn@tfn}%
          \ltx@ifpackageloaded{hyperref}{%
            \tfn@hyperfootnotes{\setcounter{Hfootnote}{\tfn@thn}}{}%
            \tfn@updateHyper%
           }{}%
        }%
     }%
  \fi%
%    \end{macrocode}
%
% For further footnotes in the table the \\
% |\ifx\tfn@footnotetableprint\tfn@footnotezero|\ldots|\fi| \\
% from the beginning of this |\tfn@footnotetext| macro must not be repeated,
% therefore we deactivate this by defining |\tfn@footnotetableprint| to
% something other than |0| (here |1| is used).
%
%    \begin{macrocode}
  \gdef\tfn@footnotetableprint{1}%
%    \end{macrocode}
%
% |\tfn@footnotetablecount| contains the negative number of footnote texts
% to be printed, therefore it is increase by one (i.\,e.~the negative number
% decreases, i.\,e.~$-3$ becomes~$-2$).
%
%    \begin{macrocode}
  \@tempcnta=\tfn@footnotetablecount\relax%
  \advance\@tempcnta by +1\relax%
  \xdef\tfn@footnotetablecount{\the\@tempcnta}%
%    \end{macrocode}
%
% When it is zero, all footnote texts have been placed, and\\
% |\tfn@footnotetableprint| can be reset to zero and\\
% |\tfn@tablefootnoteprintout| can be cleared.
%
%    \begin{macrocode}
  \ifx\tfn@footnotetablecount\tfn@footnotezero% all footnotes placed
    \gdef\tfn@footnotetableprint{0}%
    \global\let\tfn@tablefootnoteprintout\relax% brute force
 %\else there is at least one \footnotemark which needs \footnotetext
  \fi
}

%    \end{macrocode}
% \end{macro}
%
% \begin{macro}{\tfn@tablefootnoteprintout}
% When something shall be cleared or redefined, it is a good idea
% to define it. It~will be later filled with the foot note text to be printed.
%
%    \begin{macrocode}
\newcommand{\tfn@tablefootnoteprintout}{\relax}

%    \end{macrocode}
% \end{macro}
%
% \pagebreak
%
% \begin{macro}{\tablefootnote}
% |\tablefootnote[...]{...}| is the command to be used by the user instead of
% |\footnote[...]{...}| in the (sideways)table. It places a foot note mark via
% |\tfn@footnotemarks| and appends a |\tfn@footnotetext{#1}| to the\\
% |\tfn@tablefootnoteprintout| macro, and remembers that table foot notes
% have been used.
%
%    \begin{macrocode}
\newcommand{\tablefootnote}[2][]{%
  \tfn@footnotemarks[#1]%
  \ifthenelse{\isempty{#1}}{\relax}{%
    \ltx@GlobalPrependToMacro{\tfn@tablefootnoteprintout}{%
      \addtocounter{footnote}{+1}%
      \ltx@ifpackageloaded{hyperref}{%
        \tfn@hyperfootnotes{\addtocounter{Hfootnote}{+1}}{}%
        \tfn@updateHyper%
       }{}%
     }%
   }%
  \ltx@GlobalAppendToMacro{\tfn@tablefootnoteprintout}{%
    \tfn@footnotetext[#1]{#2}%
    \ifthenelse{\isempty{#1}}{\relax}{%
      \addtocounter{footnote}{-1}%
      \ltx@ifpackageloaded{hyperref}{%
        \tfn@hyperfootnotes{\addtocounter{Hfootnote}{-1}}{}%
        \tfn@updateHyper%
       }{}%
     }%
   }%
  \gdef\tfn@fnt{1}%
  }

%    \end{macrocode}
% \end{macro}
%
% \begin{macro}{\tfn@swtbox}
% For |sidewaystable|s the |\tfn@tablefootnoteprintout| is placed in a \\
% |\parbox|, which can be placed at the end of the |sidewaystable|:
%
%    \begin{macrocode}
\newcommand{\tfn@swtbox}{%
\ifvoid \footins%
\vskip 2\baselineskip%
\else%
\vskip \skip\footins%
\fi%
\parbox[b][][t]{\linewidth}{\raggedright\tfn@tablefootnoteprintout}%
}

%    \end{macrocode}
% \end{macro}
%
% |\tfn@tablefootnoteprintout| shall be executed after the end of the table environment.
% The \xpackage{etoolbox} package provides very easy syntax for this:
%
%    \begin{macrocode}
\IfFileExists{etoolbox.sty}{%
  \RequirePackage{etoolbox}[2011/01/03]%   v2.1
  \AfterEndEnvironment{table}{%
    \ltx@ifpackageloaded{footnotebackref}{%
      \LetLtxMacro{\tfncurrentmft}{\@makefntext}%
      \LetLtxMacro{\@makefntext}{\tablemakefntext}%
       \tfn@tablefootnoteprintout%
      \LetLtxMacro{\@makefntext}{\tfncurrentmft}%
     }{% footnotebackref not loaded
      \tfn@tablefootnoteprintout%
     }%
      \gdef\tfn@fnt{0}%
   }%
%    \end{macrocode}
%
% When the \xpackage{rotating} package has been loaded, we need to detect
% whether the |\tablefootnote| is placed inside of a |sidewaystable|.
%
%    \begin{macrocode}
  \ltx@ifpackageloaded{rotating}{%
    \AtBeginEnvironment{sidewaystable}{%
      \gdef\tfn@swt{1}%
    }
%    \end{macrocode}
%
% At the end of |sidewaystable| environments |\tfn@swtbox| shall be placed
% (if~there was any |\tablefootnote| in that |sidewaystable| at all).
% And |\tfn@swt| as well as |\tfn@fnt| must be reset to |0|.
%
%    \begin{macrocode}
    \AtEndEnvironment{sidewaystable}{%
      \ifx\tfn@fnt\tfn@footnotezero% \relax
      \else%
        \ltx@ifpackageloaded{footnotebackref}{%
          \LetLtxMacro{\tfncurrentmft}{\@makefntext}%
          \LetLtxMacro{\@makefntext}{\tablemakefntext}%
           \tfn@swtbox%
          \LetLtxMacro{\@makefntext}{\tfncurrentmft}%
         }{% footnotebackref not loaded
          \tfn@swtbox%
          }%
      \fi%
      \gdef\tfn@swt{0}%
      \gdef\tfn@fnt{0}%
     }%
   }{}% else \relax
}{%
%    \end{macrocode}
%
% When the \xpackage{etoolbox} package is not available (e.\,g. because of missing
% \eTeX{}) we~do a workaround:
%
%    \begin{macrocode}
  \PackageWarning{tablefootnote}{%
    Package etoolbox not available, therefore\MessageBreak%
    substituting commands \string\AtEndEnvironment\space %
    and\MessageBreak%
    \string\AfterEndEnvironment\space for tablefootnotes%
    \MessageBreak%
   }%
  \@ifpackagelater{letltxmacro}{2010/09/02}%
   {% >= 2010/09/02, OK
    \GlobalLetLtxMacro{\tfnorigtable}{\table}
    \GlobalLetLtxMacro{\tfnendorigtable}{\endtable}
%    \end{macrocode}
%
% |\GlobalLetLtxMacro| was introduced in version 2010/09/02 v1.4 of the
% \xpackage{letltxmacro} package.
%
%    \begin{macrocode}
   }{% else
     \PackageError{tablefootnote}{%
       Outdated version of letltxmacro package used%
      }{Version 2010/09/02 v1.4 or newer needed but not found!%
        \MessageBreak%
        Update the used letltxmacro package.}
        \LetLtxMacro{\tfnorigtable}{\table}
        \LetLtxMacro{\tfnendorigtable}{\endtable}
    }%
%    \end{macrocode}
%
% \pagebreak
%
%    \begin{macrocode}
  \ltx@ifpackageloaded{float}{%
    \@ifpackagelater{float}{2001/11/08}%   v1.3d
     {% >= 2001/11/08, OK
     }{% else
       \PackageWarningNoLine{tablefootnote}{%
         It is required version\MessageBreak
         2001/11/08 v1.3d (or later) of package float,\MessageBreak
         but only version\MessageBreak
        `\csname ver@float.sty\endcsname'\MessageBreak
        is available}%
     }%
   }{\relax}
  \renewenvironment{table}{%
    \ltx@ifpackageloaded{float}{%
%    \end{macrocode}
%
% When \xpackage{etoolbox.sty} was not found, and the \xpackage{float} package
% has been loaded, and the float specifier is \texttt{H}, then |\endtable| was
% let to |\float@endH| by the \xpackage{float} package (line~79 of
% \xpackage{float.sty}, 2001/11/08 v1.3d:\\
% |\expandafter\let\csname end#1\endcsname\float@endH|), i.\,e.~appending\\
% |\tfn@tablefootnoteprintout \gdef\tfn@fnt{0}| to |\endtable| is useless.
% Instead of this, it must be appended to |\float@endH|.\\
% (Why is this not necessary when \xpackage{etoolbox} is used?
% That is a good question, answers are welcome.)
%
%    \begin{macrocode}
      \ltx@LocalAppendToMacro{\float@endH}{%
        \ltx@ifpackageloaded{footnotebackref}{%
          \LetLtxMacro{\tfncurrentmft}{\@makefntext}%
          \LetLtxMacro{\@makefntext}{\tablemakefntext}%
           \tfn@tablefootnoteprintout%
          \LetLtxMacro{\@makefntext}{\tfncurrentmft}%
         }{% footnotebackref not loaded
          \tfn@tablefootnoteprintout%
         }%
          \gdef\tfn@fnt{0}%
       }%
     }{\relax}
    \tfnorigtable%
   }{%
     \tfnendorigtable%
     \ltx@ifpackageloaded{footnotebackref}{%
       \LetLtxMacro{\tfncurrentmft}{\@makefntext}%
       \LetLtxMacro{\@makefntext}{\tablemakefntext}%
        \tfn@tablefootnoteprintout%
       \LetLtxMacro{\@makefntext}{\tfncurrentmft}%
      }{% footnotebackref not loaded
       \tfn@tablefootnoteprintout%
      }%
     \gdef\tfn@fnt{0}%
    }%
%    \end{macrocode}
%
% \pagebreak
%
% The |sidewaystable|-environment must be treated separately:
%
%    \begin{macrocode}
  \ltx@ifpackageloaded{rotating}{%
    \let\tfnorigsidewaystable\sidewaystable
    \let\tfnendorigsidewaystable\endsidewaystable
    \renewenvironment{sidewaystable}{%
      \gdef\tfn@swt{1}%
      \tfnorigsidewaystable%
    }{%
      \ifx\tfn@fnt\tfn@footnotezero% \relax
      \else%
        \ltx@ifpackageloaded{footnotebackref}{%
          \LetLtxMacro{\tfncurrentmft}{\@makefntext}%
          \LetLtxMacro{\@makefntext}{\tablemakefntext}%
           \tfn@swtbox%
          \LetLtxMacro{\@makefntext}{\tfncurrentmft}%
         }{% footnotebackref not loaded
          \tfn@swtbox%
         }%
      \fi%
      \gdef\tfn@swt{0}%
      \gdef\tfn@fnt{0}%
      \tfnendorigsidewaystable%
     }%
   }{}%
}

%    \end{macrocode}
%
% \DescribeMacro{\AtBeginDocument}
% When the \xpackage{rotating} (for |sidewaystable|s) or the
% \xpackage{footnotebackref} (for back references for the footnotes)
% package is used, it must be loaded before the \xpackage{tablefootnote} package.
% Using both packages is also possible, in which case both must be loaded before
% the \xpackage{tablefootnote} package. If any one of those packages is loaded
% |\AtBeginDocument| after this check, the according error message will not be
% given, but packages really should be loaded before |\AtBeginDocument|.
%
%    \begin{macrocode}
\AtBeginDocument{%
  \ltx@ifpackageloaded{footnotebackref}{%
    \ifx\tfn@fnbr\tfn@footnotezero%
      \PackageError{tablefootnote}{%
        Package footnotebackref loaded after tablefootnote}{%
        When using both the footnotebackref and the tablefootnote %
        package,\MessageBreak%
        the footnotebackref package MUST be loaded before the %
        tablefootnote package!\MessageBreak%
       }%
    \fi%
   }{% else \relax
   }
  \ltx@ifpackageloaded{rotating}{%
    \ifx\tfn@rotating\tfn@footnotezero%
      \PackageError{tablefootnote}{%
        Package rotating loaded after tablefootnote}{%
        When using both the rotating and the tablefootnote package,%
        \MessageBreak%
        the rotating package MUST be loaded before the tablefootnote%
        \space package!\MessageBreak%
       }%
    \fi%
   }{% else \relax
   }
%    \end{macrocode}
%
% \pagebreak
%
% When the document is compiled with \LuaLaTeX{}, hyperlinks in rotated content
% will be misplaced, regardless of the use of the \xpackage{tablefootnote} package
% (or of tables or footnotes). The text to be printed will be OK.
% This problem of \LuaLaTeX{} (not \xpackage{tablefootnote}) was found and reported
% by \textsc{Arno Trautmann}, 2011/11/15 -- thanks!
%
%    \begin{macrocode}
  \ltx@ifpackageloaded{hyperref}{%
    \tfn@hyperfootnotes{
      \ifluatex%
        \PackageWarningNoLine{tablefootnote}{%
          LuaTeX and hyperref package used:\MessageBreak%
          Hyperreferences in rotated content\MessageBreak%
          will be misplaced\MessageBreak%
          regardless of tablefootnote package.\MessageBreak%
          The text to be printed will be OK%
         }%
      \fi%
     }{% else \relax
     }%
   }{% else \relax
   }%
%    \end{macrocode}
%
% When the \xpackage{footnotebackref} package is used but the footnotes are not
% hyperlinked, neither the back references nor the tablefootnotes will work.
%
%    \begin{macrocode}
  \ltx@ifpackageloaded{footnotebackref}{%
     \ltx@ifpackageloaded{hyperref}{%
        \tfn@hyperfootnotes{\relax}{% else
          \PackageError{tablefootnote}{%
             footnotebackref but not hyperfootnotes}{%
             The footnotebackref package is used\MessageBreak%
             but the hyperfootnotes option of the hyperref package%
             \MessageBreak%
             was set to false.\MessageBreak%
             Either enable hyperfootnotes or do not use footnotebackref.%
             \MessageBreak%
             Loading of the tablefootnote package will be aborted.%
           }%
          \endinput%
      }%
    }{\PackageError{tablefootnote}{footnotebackref but not hyperref}{%
        The footnotebackref package is used\MessageBreak%
        but not the hyperref package.\MessageBreak%
        Either load hyperref or do not use footnotebackref.%
        \MessageBreak%
        Loading of the tablefootnote package will be aborted.%
        }%
      \endinput%
    }%
  }{}
}

%    \end{macrocode}
%
%    \begin{macrocode}
%</package>
%    \end{macrocode}
%
% \pagebreak
%
% \section{Installation}
%
% \subsection{Downloads\label{ss:Downloads}}
%
% Everything is available on \CTAN{}, \url{http://www.ctan.org/},
% but may need additional packages themselves.\\
%
% \DescribeMacro{tablefootnote.dtx}
% For unpacking the |tablefootnote.dtx| file and constructing the documentation
% it is required:
% \begin{description}
% \item[-] \TeX{} Format \LaTeXe{}: \url{http://www.CTAN.org/}
%
% \item[-] document class \xpackage{ltxdoc}, 2007/11/11, v2.0u,
%           \url{http://www.ctan.org/pkg/ltxdoc}
%
% \item[-] package \xpackage{holtxdoc}, 2012/03/21, v0.24,
%           \url{http://www.ctan.org/pkg/holtxdoc}
%
% \item[-] package \xpackage{hypdoc}, 2011/08/19, v1.11,
%           \url{http://www.ctan.org/pkg/hypdoc}
% \end{description}
%
% \DescribeMacro{tablefootnote.sty}
% The |tablefootnote.sty| for \LaTeXe{} (i.\,e.~each document using
% the \xpackage{tablefootnote} package) requires:
% \begin{description}
% \item[-] \TeX{} Format \LaTeXe{}, \url{http://www.CTAN.org/}
%
% \item[-] package \xpackage{letltxmacro}, 2010/09/02, v1.4,
%           \url{http://www.ctan.org/pkg/letltxmacro}.
%
% \item[-] package \xpackage{ltxcmds}, 2011/11/09, v1.22,
%           \url{http://www.ctan.org/pkg/ltxcmds}
% \end{description}
%
% When the \xpackage{hyperref} package is used, also the \xpackage{ifluatex} package
% is needed, but it is already loaded automatically by the \xpackage{hyperref}
% package.\\
% When the \xpackage{etoolbox} package is available, it is used:
% \begin{description}
% \item[-] package \xpackage{etoolbox}, 2011/01/03, v2.1,
%           \url{http://www.ctan.org/pkg/etoolbox}.
% \end{description}
%
% \DescribeMacro{tablefootnote-example.tex}
% The |tablefootnote-example.tex| requires the same files as all
% documents using the \xpackage{tablefootnote} package and additionally:
% \begin{description}
% \item[-] class \xpackage{article}, 2007/10/19, v1.4h, from \xpackage{classes.dtx}:\\
%           \CTAN{macros/latex/base/classes.dtx}
%
% \item[-] package \xpackage{rotating}, 2009/03/28, v2.16a,
%           \url{http://www.ctan.org/pkg/rotating}
%
% \item[-] package \xpackage{float}, 2001/11/08, v1.3d,
%           \url{http://www.ctan.org/pkg/float}
%
% \item[-] package \xpackage{placeins}, 2005/04/18, v2.2,
%           \url{http://www.ctan.org/pkg/placeins}\\
%           for |\FloatBarrier|
%
% \item[-] package \xpackage{hyperref}, 2012/11/06, v6.83m,
%           \url{http://www.ctan.org/pkg/hyperref}
%
% \item[-] package \xpackage{tablefootnote}, 2014/01/26, v1.1c,\\
%           \url{http://www.ctan.org/pkg/tablefootnote}\\
%   (Well, it is the example file for this package, and because you
%    are reading the documentation for the \xpackage{tablefootnote} package,
%    it can be assumed that you already have some version of it~--
%    is it the current one?)
% \end{description}
%
% \DescribeMacro{Oberdiek}
% \DescribeMacro{holtxdoc}
% \DescribeMacro{ifluatex}
% \DescribeMacro{letltxmacro}
% \DescribeMacro{ltxcmds}
% All packages of \textsc{Heiko Oberdiek}'s bundle `oberdiek'
% (especially \xpackage{holtxdoc}, \xpackage{ifluatex}, \xpackage{letltxmacro},
% and \xpackage{ltxcmds}) are also available in a TDS compliant ZIP archive:\\
% \CTAN{install/macros/latex/contrib/oberdiek.tds.zip}.\\
% It is probably best to download and use this, because the packages in there
% are quite probably both recent and compatible among themselves.\\
%
% \pagebreak
%
% \DescribeMacro{hyperref}
% \noindent \xpackage{hyperref} is not included in that bundle and needs to be
% downloaded separately,\\
% \url{http://www.ctan.org/install/macros/latex/contrib/hyperref.tds.zip}.\\
%
% \DescribeMacro{M\"{u}nch}
% A hyperlinked list of my (other) packages can be found at
% \url{http://www.ctan.org/author/muench-hm}.\\
%
% \subsection{Package, unpacking TDS}
% \paragraph{Package.} This package is available on \CTAN{}
% \begin{description}
% \item[\CTAN{macros/latex/contrib/tablefootnote/tablefootnote.dtx}]\hspace*{0.1cm} \\
%       The source file.
% \item[\CTAN{macros/latex/contrib/tablefootnote/tablefootnote.pdf}]\hspace*{0.1cm} \\
%       The documentation.
% \item[\CTAN{macros/latex/contrib/tablefootnote/tablefootnote-example.pdf}]\hspace*{0.1cm} \\
%       The compiled example file, as it should look like.
% \item[\CTAN{macros/latex/contrib/tablefootnote/README}]\hspace*{0.1cm} \\
%       The README file.
% \end{description}
% There is also a tablefootnote.tds.zip available:
% \begin{description}
% \item[\CTAN{install/macros/latex/contrib/tablefootnote.tds.zip}]\hspace*{0.1cm} \\
%       Everything in \xfile{TDS} compliant, compiled format.
% \end{description}
% which additionally contains\\
% \begin{tabular}{ll}
% tablefootnote.ins & The installation file.\\
% tablefootnote.drv & The driver to generate the documentation.\\
% tablefootnote.sty & The \xext{sty}le file.\\
% tablefootnote-example.tex & The example file.
% \end{tabular}
%
% \bigskip
%
% \noindent For required other packages, please see the preceding subsection.
%
% \paragraph{Unpacking.} The \xfile{.dtx} file is a self-extracting
% \docstrip{} archive. The files are extracted by running the
% \xext{dtx} through \plainTeX{}:
% \begin{quote}
%   \verb|tex tablefootnote.dtx|
% \end{quote}
%
% \noindent About generating the documentation see paragraph~\ref{GenDoc} below.\\
%
% \paragraph{TDS.} Now the different files must be moved into
% the different directories in your installation TDS tree
% (also known as \xfile{texmf} tree):
% \begin{quote}
% \def\t{^^A
% \begin{tabular}{@{}>{\ttfamily}l@{ $\rightarrow$ }>{\ttfamily}l@{}}
%   tablefootnote.sty & tex/latex/tablefootnote/tablefootnote.sty\\
%   tablefootnote.pdf & doc/latex/tablefootnote/tablefootnote.pdf\\
%   tablefootnote-example.tex & doc/latex/tablefootnote/tablefootnote-example.tex\\
%   tablefootnote-example.pdf & doc/latex/tablefootnote/tablefootnote-example.pdf\\
%   tablefootnote.dtx & source/latex/tablefootnote/tablefootnote.dtx\\
% \end{tabular}^^A
% }^^A
% \sbox0{\t}^^A
% \ifdim\wd0>\linewidth
%   \begingroup
%     \advance\linewidth by\leftmargin
%     \advance\linewidth by\rightmargin
%   \edef\x{\endgroup
%     \def\noexpand\lw{\the\linewidth}^^A
%   }\x
%   \def\lwbox{^^A
%     \leavevmode
%     \hbox to \linewidth{^^A
%       \kern-\leftmargin\relax
%       \hss
%       \usebox0
%       \hss
%       \kern-\rightmargin\relax
%     }^^A
%   }^^A
%   \ifdim\wd0>\lw
%     \sbox0{\small\t}^^A
%     \ifdim\wd0>\linewidth
%       \ifdim\wd0>\lw
%         \sbox0{\footnotesize\t}^^A
%         \ifdim\wd0>\linewidth
%           \ifdim\wd0>\lw
%             \sbox0{\scriptsize\t}^^A
%             \ifdim\wd0>\linewidth
%               \ifdim\wd0>\lw
%                 \sbox0{\tiny\t}^^A
%                 \ifdim\wd0>\linewidth
%                   \lwbox
%                 \else
%                   \usebox0
%                 \fi
%               \else
%                 \lwbox
%               \fi
%             \else
%               \usebox0
%             \fi
%           \else
%             \lwbox
%           \fi
%         \else
%           \usebox0
%         \fi
%       \else
%         \lwbox
%       \fi
%     \else
%       \usebox0
%     \fi
%   \else
%     \lwbox
%   \fi
% \else
%   \usebox0
% \fi
% \end{quote}
% If you have a \xfile{docstrip.cfg} that configures and enables \docstrip{}'s
% \xfile{TDS} installing feature, then some files can already be in the right
% place, see the documentation of \docstrip{}.
%
% \subsection{Refresh file name databases}
%
% If your \TeX{}~distribution (\TeX live, \mikTeX{}, \dots{}) relies on file name
% databases, you must refresh these. For example, \teTeX{} users run
% \verb|texhash| or \verb|mktexlsr|.
%
% \subsection{Some details for the interested}
%
% \paragraph{Unpacking with \LaTeX{}.}
% The \xfile{.dtx} chooses its action depending on the format:
% \begin{description}
% \item[\plainTeX:] Run \docstrip{} and extract the files.
% \item[\LaTeX:] Generate the documentation.
% \end{description}
% If you insist on using \LaTeX{} for \docstrip{} (really,
% \docstrip{} does not need \LaTeX{}), then inform the autodetect routine
% about your intention:
% \begin{quote}
%   \verb|latex \let\install=y\input{tablefootnote.dtx}|
% \end{quote}
% Do not forget to quote the argument according to the demands
% of your shell.
%
% \paragraph{Generating the documentation.\label{GenDoc}}
% You can use both the \xfile{.dtx} or the \xfile{.drv} to generate
% the documentation. The process can be configured by a
% configuration file \xfile{ltxdoc.cfg}. For instance, put the following
% line into this file, if you want to have A4 as paper format:
% \begin{quote}
%   \verb|\PassOptionsToClass{a4paper}{article}|
% \end{quote}
%
% \noindent An example follows how to generate the
% documentation with \pdfLaTeX{}:
%
% \begin{quote}
%\begin{verbatim}
%pdflatex tablefootnote.dtx
%makeindex -s gind.ist tablefootnote.idx
%pdflatex tablefootnote.dtx
%makeindex -s gind.ist tablefootnote.idx
%pdflatex tablefootnote.dtx
%\end{verbatim}
% \end{quote}
%
% \subsection{Compiling the example}
%
% The example file, \textsf{tablefootnote-example.tex}, can be compiled via\\
% \indent |latex tablefootnote-example.tex|,\\
% \indent |lualatex tablefootnote-example.tex|,\\
% or (recommended)\\
% \indent |pdflatex thumbs-example.tex|\\
% and will need at least two compiler runs to get everything right.
%
% \bigskip
%
% \section{Acknowledgements}
%
% I would like to thank \textsc{Heiko Oberdiek} for providing
% the \xpackage{hyperref}, \xpackage{ifluatex}, \xpackage{letltxmacro},
% \xpackage{ltxcmds}, as well as a~lot~(!) of other useful packages
% (from which I also got everything I know about creating a file in
% \xext{dtx} format, OK, say it: copying),
% and for information about the \pdfTeX{}-bug,
% \textsc{Arno Trautmann} for reporting the bug of misplaced hyperreferences in
% |sidewaystable|s using \LuaTeX{}, independently of the \xpackage{tablefootnote}
% package or footnotes at all,
% \textsc{Thomas~V.} for reporting the bug of not handling the case of
% \xpackage{hyperref} option |hyperfootnotes=false|,
% \textsc{Sverre Stausland Johnsen} for reporting the bug in case of \emph{not}
% using \xpackage{hyperref},
% and the \Newsgroup{comp.text.tex} and \Newsgroup{de.comp.text.tex} newsgroups
% as well as everybody at \url{http://tex.stackexchange.com}
% for their help in all things \TeX{}.
%
% \pagebreak
%
% \phantomsection
% \begin{History}\label{History}
%   \begin{Version}{2011/10/26 v1.0a, tabfootn}
%     \item Upload to \CTAN{macros/latex/contrib/tablefootnote/}.
%   \end{Version}
%   \begin{Version}{2011/10/29 v1.0b}
%     \item Renamed to \xpackage{tablefootnote}.
%     \item Added support for the |sidewaystable|-environment of the
%             \xpackage{rotating} package.
%     \item Diverse changes in documentation and README.
%   \end{Version}
%   \begin{Version}{2011/11/06 v1.0c}
%     \item Replaced |^{\text{\thefootnote}}| by |{}^{\text{\textup{\thefootnote}}}|.
%     \item Bug fix: When \xpackage{etoolbox.sty} was not found, and the \xpackage{float}
%             package had been loaded, and the float specifier was \texttt{H},
%             then the |\tablefootnote|s were not printed for that table. Fixed.
%   \end{Version}
%   \begin{Version}{2011/11/19 v1.0d}
%     \item Replaced |{}^{\text{\textup{\thefootnote}}}| by
%             |\textsuperscript{\thefootnote}|, therefore the \xpackage{amstext}
%             package is no longer required by the \xpackage{tablefootnote} package.
%     \item \xpackage{lscape} and \xpackage{pdflscape} package now work with
%             \xpackage{tablefootnote} (except for using a \texttt{sidewaystable} on
%             a \texttt{landscape} page).
%     \item Added information about footnotes in \texttt{sidewaystable} being printed
%             one below the other, even if \xpackage{footmisc} package with option
%             \texttt{para} is used.
%     \item Redefined |\tfn@swtbox| from
%  \begin{verbatim}
%  \vspace{0.8cm}%
%  \begin{spacing}{0.1}%
%  \subcaptionbox*{}[\linewidth][l]{\tfn@tablefootnoteprintout}%
%  \end{spacing}%
%  \end{verbatim}
%             to
%  \begin{verbatim}
%  \vspace{2\baselineskip}%
%  \parbox[b][][t]{\linewidth}{\raggedright\tfn@tablefootnoteprintout}%
%  \end{verbatim}
%             , therefore the \xpackage{setspace} and \xpackage{subcaption} packages
%             are no longer required by the \xpackage{tablefootnote} package.
%     \item No longer uses |\numexpr|, therefore also usable without \eTeX{}.
%     \item Replaced |\textsuperscript{\thefootnote}| by
%             |\hbox {\@textsuperscript {\normalfont \thefootnote }}|,
%             which is next to the original definition of |\@makefnmark|
%             (|\hbox {\@textsuperscript {\normalfont \@thefnmark }}|,
%             but according to |\show\@thefnmark| here |\@thefnmark| is
%             |{\protect \itshape}| instead of |\thefootnote|).
%     \item Added an error message, when \LuaTeX{}, \xpackage{rotating} package, and
%            \xpackage{hyperref} package are used together. Hyperreferences in
%            |sidewaystable|s are misplaced, independently of the
%            \xpackage{tablefootnote} package or footnotes at all. (Bug reported
%            by \textsc{Arno Trautmann}, 2011/11/15 -- thanks!)
%     \item Added footnotesymbol-footnote-numbers in the example.
%   \end{Version}
%   \begin{Version}{2011/11/26 v1.0e}
%     \item Check for \xpackage{hyperref} is only done once.
%     \item Message regarding the misplacement of hyperlinks by \LuaLaTeX{}
%             changed form error to warning.
%     \item Replaced |\vspace{2\baselineskip}| by
%  \begin{verbatim}
% \ifvoid \footins%
% \vskip 2\baselineskip%
% \else%
% \vskip \skip\footins%
% \fi%
%  \end{verbatim}
%             between the |sidewaystable|s and their footnotes.
%     \item The warning about un-regarded |para|-option of \xpackage{footmoisc}
%             package is only given at the second |tablefootnote| in a |sidewaystable|,
%             not repeatedly for every |tablefootnote| in a |sidewaystable|.
%   \end{Version}
%   \begin{Version}{2012/01/01 v1.0f}
%     \item Bugfix: Wrong installation path given in the documentation, fixed.
%     \item Update of documentation, README, and \xfile{dtx} internals.
%   \end{Version}
%   \begin{Version}{2012/01/14 v1.0g}
%     \item Bugfix: There was a |relax| instead of a |\relax|, fixed.
%   \end{Version}
%   \begin{Version}{2012/07/29 v1.0h}
%     \item \xpackage{tablefootnote} is now compatible to the new
%             \xpackage{footnotebackref} package, 2012/07/01, v1.0,
%             \url{http://www.ctan.org/pkg/footnotebackref}.
%   \end{Version}
%   \begin{Version}{2014/01/08 v1.1a}
%     \item Tablefootnotes now regard optional footnote marks:
%             |\tablefootnote[| custom mark here! |]{...}|
%     \item Bug fix: the case of \xpackage{hyperref} option |hyperfootnotes=false|
%             was not handeled at all. (Bug reported by \textsc{Thomas~V.}, %
%             2014/01/02 -- thanks!)
%     \item A lot of details.
%   \end{Version}
%   \begin{Version}{2014/01/20 v1.1b}
%     \item Bug fix: Package was broken when \xpackage{hyperref} was \emph{not} used.
%            (Bug reported by \textsc{Sverre Stausland Johnsen} -- thanks!)
%   \end{Version}
%   \begin{Version}{2014/01/26 v1.1c}
%     \item The emergency bug fix of version 2014/01/20 v1.1b was replaced by using
%             the new robust macro |\tfn@hyperfootnotes|, which only \emph{internally}
%             uses |\ifHy@hyperfootnotes|. If the \xpackage{hyperref} package is not used,
%             \LaTeX{} now does not see any confusing
%             |\ifHy@hyperfootnotes|\ldots|\else|\ldots|\fi| at all.
%     \item URLs and README updated.
%   \end{Version}
% \end{History}
%
% \bigskip
%
% When you find a mistake or have a suggestion for an improvement of this package,
% please send an e-mail to the maintainer, thanks! (Please see BUG REPORTS in the README.)
%
% \bigskip
%
% \PrintIndex
%
% \Finale
\endinput
%        (quote the arguments according to the demands of your shell)
%
% Documentation:
%    (a) If tablefootnote.drv is present:
%           (pdf)latex tablefootnote.drv
%           makeindex -s gind.ist tablefootnote.idx
%           (pdf)latex tablefootnote.drv
%           makeindex -s gind.ist tablefootnote.idx
%           (pdf)latex tablefootnote.drv
%    (b) Without tablefootnote.drv:
%           (pdf)latex tablefootnote.dtx
%           makeindex -s gind.ist tablefootnote.idx
%           (pdf)latex tablefootnote.dtx
%           makeindex -s gind.ist tablefootnote.idx
%           (pdf)latex tablefootnote.dtx
%
%    The class ltxdoc loads the configuration file ltxdoc.cfg
%    if available. Here you can specify further options, e.g.
%    use DIN A4 as paper format:
%       \PassOptionsToClass{a4paper}{article}
%
% Installation:
%    TDS:tex/latex/tablefootnote/tablefootnote.sty
%    TDS:doc/latex/tablefootnote/tablefootnote.pdf
%    TDS:doc/latex/tablefootnote/tablefootnote-example.tex
%    TDS:doc/latex/tablefootnote/tablefootnote-example.pdf
%    TDS:source/latex/tablefootnote/tablefootnote.dtx
%
%<*ignore>
\begingroup
  \catcode123=1 %
  \catcode125=2 %
  \def\x{LaTeX2e}%
\expandafter\endgroup
\ifcase 0\ifx\install y1\fi\expandafter
         \ifx\csname processbatchFile\endcsname\relax\else1\fi
         \ifx\fmtname\x\else 1\fi\relax
\else\csname fi\endcsname
%</ignore>
%<*install>
\input docstrip.tex
\Msg{******************************************************************}
\Msg{* Installation                                                   *}
\Msg{* Package: tablefootnote 2014/01/26 v1.1c Table foot notes (HMM) *}
\Msg{******************************************************************}

\keepsilent
\askforoverwritefalse

\let\MetaPrefix\relax
\preamble

This is a generated file.

Project: tablefootnote
Version: 2014/01/26 v1.1c

Copyright (C) 2011 - 2014 by
    H.-Martin M"unch <Martin dot Muench at Uni-Bonn dot de>

The usual disclaimer applies:
If it doesn't work right that's your problem.
(Nevertheless, send an e-mail to the maintainer
 when you find an error in this package.)

This work may be distributed and/or modified under the
conditions of the LaTeX Project Public License, either
version 1.3c of this license or (at your option) any later
version. This version of this license is in
   http://www.latex-project.org/lppl/lppl-1-3c.txt
and the latest version of this license is in
   http://www.latex-project.org/lppl.txt
and version 1.3c or later is part of all distributions of
LaTeX version 2005/12/01 or later.

This work has the LPPL maintenance status "maintained".

The Current Maintainer of this work is H.-Martin Muench.

This work consists of the main source file tablefootnote.dtx,
the README, and the derived files
   tablefootnote.sty, tablefootnote.pdf,
   tablefootnote.ins, tablefootnote.drv,
   tablefootnote-example.tex, tablefootnote-example.pdf.

In memoriam Tommy Muench + 2014/01/02.

\endpreamble
\let\MetaPrefix\DoubleperCent

\generate{%
  \file{tablefootnote.ins}{\from{tablefootnote.dtx}{install}}%
  \file{tablefootnote.drv}{\from{tablefootnote.dtx}{driver}}%
  \usedir{tex/latex/tablefootnote}%
  \file{tablefootnote.sty}{\from{tablefootnote.dtx}{package}}%
  \usedir{doc/latex/tablefootnote}%
  \file{tablefootnote-example.tex}{\from{tablefootnote.dtx}{example}}%
}

\catcode32=13\relax% active space
\let =\space%
\Msg{************************************************************************}
\Msg{*}
\Msg{* To finish the installation you have to move the following}
\Msg{* file into a directory searched by TeX:}
\Msg{*}
\Msg{*     tablefootnote.sty}
\Msg{*}
\Msg{* To produce the documentation run the file `tablefootnote.drv'}
\Msg{* through (pdf)LaTeX, e.g.}
\Msg{*  pdflatex tablefootnote.drv}
\Msg{*  makeindex -s gind.ist tablefootnote.idx}
\Msg{*  pdflatex tablefootnote.drv}
\Msg{*  makeindex -s gind.ist tablefootnote.idx}
\Msg{*  pdflatex tablefootnote.drv}
\Msg{*}
\Msg{* At least three runs are necessary e.g. to get the}
\Msg{*  references right!}
\Msg{*}
\Msg{* Happy TeXing!}
\Msg{*}
\Msg{************************************************************************}

\endbatchfile
%</install>
%<*ignore>
\fi
%</ignore>
%
% \section{The documentation driver file}
%
% The next bit of code contains the documentation driver file for
% \TeX{}, i.\,e., the file that will produce the documentation you
% are currently reading. It will be extracted from this file by the
% \texttt{docstrip} programme. That is, run \LaTeX{} on \texttt{docstrip}
% and specify the \texttt{driver} option when \texttt{docstrip}
% asks for options.
%
%    \begin{macrocode}
%<*driver>
\NeedsTeXFormat{LaTeX2e}[2011/06/27]
\ProvidesFile{tablefootnote.drv}%
  [2014/01/26 v1.1c Table foot notes (HMM)]
\documentclass{ltxdoc}[2007/11/11]% v2.0u
\usepackage{holtxdoc}[2012/03/21]%  v0.24
%% tablefootnote may work with earlier versions of LaTeX2e and those
%% class and package, but this was not tested.
%% Please consider updating your LaTeX, class, and package
%% to the most recent version (if they are not already the most
%% recent version).
\hypersetup{%
 pdfsubject={Footnotes in Tables (HMM)},%
 pdfkeywords={LaTeX, tablefootnote, footnote, table, H.-Martin Muench},%
 pdfencoding=auto,%
 pdflang={en},%
 breaklinks=true,%
 linktoc=all,%
 pdfstartview=FitH,%
 pdfpagelayout=OneColumn,%
 bookmarksnumbered=true,%
 bookmarksopen=true,%
 bookmarksopenlevel=3,%
 pdfmenubar=true,%
 pdftoolbar=true,%
 pdfwindowui=true,%
 pdfnewwindow=true%
}
\CodelineIndex
\hyphenation{printing docu-ment}
\gdef\unit#1{\mathord{\thinspace\mathrm{#1}}}%
\makeatletter
\@ifundefined{eTeX}{\gdef\eTeX{$\m@th \varepsilon $-\TeX }}{% else \relax
}
\makeatother
\begin{document}
  \DocInput{tablefootnote.dtx}%
\end{document}
%</driver>
%    \end{macrocode}
%
% \fi
%
% \CheckSum{584}
%
% \CharacterTable
%  {Upper-case    \A\B\C\D\E\F\G\H\I\J\K\L\M\N\O\P\Q\R\S\T\U\V\W\X\Y\Z
%   Lower-case    \a\b\c\d\e\f\g\h\i\j\k\l\m\n\o\p\q\r\s\t\u\v\w\x\y\z
%   Digits        \0\1\2\3\4\5\6\7\8\9
%   Exclamation   \!     Double quote  \"     Hash (number) \#
%   Dollar        \$     Percent       \%     Ampersand     \&
%   Acute accent  \'     Left paren    \(     Right paren   \)
%   Asterisk      \*     Plus          \+     Comma         \,
%   Minus         \-     Point         \.     Solidus       \/
%   Colon         \:     Semicolon     \;     Less than     \<
%   Equals        \=     Greater than  \>     Question mark \?
%   Commercial at \@     Left bracket  \[     Backslash     \\
%   Right bracket \]     Circumflex    \^     Underscore    \_
%   Grave accent  \`     Left brace    \{     Vertical bar  \|
%   Right brace   \}     Tilde         \~}
%
% \GetFileInfo{tablefootnote.drv}
%
% \begingroup
%   \def\x{\#,\$,\^,\_,\~,\ ,\&,\{,\},\%}%
%   \makeatletter
%   \@onelevel@sanitize\x
% \expandafter\endgroup
% \expandafter\DoNotIndex\expandafter{\x}
% \expandafter\DoNotIndex\expandafter{\string\ }
% \begingroup
%   \makeatletter
%     \lccode`9=32\relax
%     \lowercase{%^^A
%       \edef\x{\noexpand\DoNotIndex{\@backslashchar9}}%^^A
%     }%^^A
%   \expandafter\endgroup\x
%
% \DoNotIndex{\\,\,}
% \DoNotIndex{\documentclass,\usepackage,\ProvidesPackage,\begin,\end}
% \DoNotIndex{\MessageBreak,\message}
% \DoNotIndex{\NeedsTeXFormat,\DoNotIndex,\verb}
% \DoNotIndex{\def,\edef,\gdef,\global}
% \DoNotIndex{\ifx,\listfiles,\mathord,\mathrm}
% \DoNotIndex{\kvoptions,\SetupKeyvalOptions,\ProcessKeyvalOptions}
% \DoNotIndex{\bigskip,\space,\thinspace,\ldots}
% \DoNotIndex{\indent,\noindent,\newline,\linebreak,\pagebreak}
% \DoNotIndex{\textbf,\textit,\textsf,\texttt,\textsc,\textquotedblleft,\textquotedblright}
% \DoNotIndex{\plainTeX,\TeX,\LaTeX,\pdfLaTeX}
% \DoNotIndex{\chapter,\section,\the,\csname,\endscname}
% \DoNotIndex{\value,\arabic,\hline,\text,\caption,\centering,\clear,\url}
%
% \title{The \xpackage{tablefootnote} package}
% \date{2014/01/26 v1.1c}
% \author{H.-Martin M\"{u}nch\\\xemail{Martin.Muench at Uni-Bonn.de}}
%
% \maketitle
%
% \begin{abstract}
% This \LaTeX{} package provides the command |\tablefootnote| to be used
% in a \texttt{table} or \texttt{sidewaystable} environment,
% where |\footnote| will not work and when using |\footnotemark| and
% |\footnotetext| and adjusting the counters (including \texttt{Hfootnote})
% manually is either too much work or would not even work (\texttt{sidewaystable}).
% \end{abstract}
%
% \bigskip
%
% \noindent Disclaimer for web links: The author is not responsible for any contents
% referred to in this work unless he has full knowledge of illegal contents.
% If any damage occurs by the use of information presented there, only the
% author of the respective pages might be liable, not the one who has referred
% to these pages.
%
% \bigskip
%
% \noindent {\color{green} Save per page about $200\unit{ml}$ water,
% $2\unit{g}$ CO$_{2}$ and $2\unit{g}$ wood:\\
% Therefore please print only if this is really necessary.}
%
% \pagebreak
%
% \tableofcontents
%
% \pagebreak
%
% \section{Introduction\label{sec:Introduction}}
%
% \indent This \LaTeX{} package provides the command |\tablefootnote{...}| to be
% used in a \texttt{table} or \texttt{sidewaystable} environment. In those
% environments |\footnote{...}| would not work and would need to be replaced
% by |\footnotemark| in the \texttt{(sideways)\allowbreak table} and
% |\footnotetext{...}| after the \texttt{(sideways)table} environment ended.
% Additionally the counters for \texttt{footnote} and
% (when the \xpackage{hyperref} package is used) \texttt{Hfootnote}
% would need to be adjusted manually. Then still the hyperlinks
% in a \texttt{(sideways)table} would not work.
% When the \xpackage{footnotebackref} package is used,
% |\footnotemark| and |\footnotetext{...}| would need to be redefined.
% Just using |\tablefootnote{...}| (with optional argument for custom foot note
% marks) in the (sideways)table does all this automatically.
% (Compatibility with the \xpackage{footnotebackref} package
% was achieved without redefining |\footnotemark| or |\footnotetext|.)
% Redefining |\footnote| was \emph{not} done as to not create conflicts with
% other packages.
%
% \section{Usage}
%
% \indent Just load the package placing
% \begin{quote}
%   |\usepackage{tablefootnote}|
% \end{quote}
% \noindent in the preamble of your \LaTeXe{} source file and use\newline
% |\tablefootnote{...}| instead of |\footnote{...}| and\newline
% |\tablefootnote[...]{...}| instead of |\footnote[...]{...}|
% in (sideways)\allowbreak tables.\\
% When the \xpackage{footnotebackref} package is used,
% it must be loaded before the \xpackage{tablefootnote} package
% and the \xpackage{hyperrefpackage} with option |hyperfootnotes=true|
% must be loaded, too. (Backreference{-}links without links do not make sense.)\\
% When the \xpackage{rotating} package (for |sidewaystable|s) is used,
% it must be loaded before the \xpackage{tablefootnote} package and it must be
% a recent version (v2.16a, 2009/03/28, or newer).\\
% If |\tablefootnote{...}| is used in a tabular environment, which is inside
% of a (sideways)table environment, everything is fine, but when the tabular
% environment is not inside of a (sideways)table, the \textbf{table}footnote
% will not work.\\
%
% Pages with tables with footnotes, which are rotated with the \xpackage{lscape}-
% or with the \xpackage{pdflscape}-package (|\begin{landscape}|, table, |\end{landscape}|),
% are possible. But when the \xpackage{footmisc}-package is used, \xpackage{(pdf)lscape}
% should be loaded before \xpackage{footmisc}.\\
%
% When the \xpackage{footmisc}-package is used with option \texttt{para},
% this is regarded with the exception of a \texttt{sidewaystable}. There the footnotes
% are printed one below the other. (Well, this is better than no footnote in a
% \texttt{sidewaystable} at all, isn't it?)\\
%
% When the (sideways)table floats, the footnotes are set and (when \xpackage{hyperref}
% is used) hyperlinked, but they are not automatically adapted when the table floats
% over/under another footnote. Thus either do not use a footnote between original and
% float position of the (sideways)table, or place the (sideways)table in
% \textquotedblleft here\textquotedblright\ position. |\clear(double)page|, |h(!)|,
% |H|~from the \xpackage{float} package (\url{http://www.ctan.org/pkg/float}),
% or |\FloatBarrier| from the \xpackage{picins} package
% (\url{http://www.ctan.org/pkg/picins}) might help, too.
% (Or after finishing the document, move the (sideways)table in the source code
% near the position where it floats to or use the optional footnote marks.)
%
% \pagebreak
%
% \section{Alternatives\label{sec:Alternatives}}
%
% \begin{description}
% \item[-] The \xpackage{longtable} package provides the \texttt{longtable} environment
%   as replacement for the combined \texttt{table} and \texttt{tabular} environments.
%   Footnotes are real footnotes (not just tablenotes), are continuously numbered and
%   hyperlinked (when using the \xpackage{hyperref} package), and the hyperlinks really
%   work. As drawback the appearance of the caption changes slightly (e.\,g.~distance
%   to the table, width of the caption), which can probably be changed back manually.
%   Furthermore, longtables are meaned to break over more than one page. If that is not
%   wished, it must be prevented by |\nopagebreak|-commands and by ending the longtable
%   lines with |\\*| instead of |\\|. longtables do not float. (Therefore using the
%   \xpackage{tablefootnote} package and |\FloatBarrier| from the \xpackage{picins} package
%   before and after the table environment is similar~- but \xpackage{tablefootnote} does not
%   change the table-caption!) \texttt{sidewaystable} does not work with it.\\
%   \url{http://www.ctan.org/pkg/longtable}
%
% \item[-] The \xpackage{supertabular} package provides the \texttt{mpsupertabular}
%   environment as replacement for the combined \texttt{table} and \texttt{tabular}
%   environments. Footnotes are just tablenotes (with working hyperlinks when using the
%   \xpackage{hyperref} package), i.\,e. numbered a, b, c and placed below the table
%   and not at the end of the page. Therefore there is no float problem (because the
%   tablenotes numbering is not included in the continuous numbering of the footnotes).
%   Placing the \texttt{supertabular} inside of a \texttt{sidewaystable} breaks the
%   hyperlinks to the tablenotes.\\
%   \url{http://www.ctan.org/pkg/supertabular}
%
% \item[-] The \xpackage{ctable} package has its very own notation for defining tables.
%   It can create tablenotes and sideways-tables. The tablenotes are not automatically
%   hyperlinked. The \texttt{ctable}s float. Because the tablenotes numbering is not
%   included in the continuous numbering of the footnotes there is no float problem.\\
%   \url{http://www.ctan.org/pkg/ctable}
%
% \item[-] The \xpackage{footnote} package provides |\makesavenoteenv{table}|. After
%   loading the package and using that command in the preamble, in tables |\footnote{...}|
%   can be used. Using |\makesavenoteenv{tabular}| and \\
%   |\makesavenoteenv{sidewaystable}|
%   is possible, but it neither solves the float problem, nor do the created hyperlinks
%   work (i.\,e.~they aim at wrong locations). The \xpackage{mdwtab} from the same bundle
%   is incompatible with other table-related packages (e.\,g. \xpackage{supertabular},
%   \xpackage{array}) and not $100\,\%$ compatible with the \texttt{tabular} environment.\\
%   \url{http://www.ctan.org/pkg/footnote}\\
%   \url{http://www.ctan.org/pkg/mdwtab}
%
% \item[-] The \xpackage{tabularx} package does produce footnotes for sidewaystables,
%   but uses a, b, c instead of 1, 2, 3. The hyperlinks to the footnotes do not work.
%   Because the footnotes numbering is not included in the continuous numbering of
%   the other footnotes there is no float problem.\\
%   \url{http://www.ctan.org/pkg/tabularx}
%
% \item[-] Placing a tabular inside a minipage inside a table produces tablenotes.
%   Therefore there is no float problem (because the footnotes are not continuously
%   numbered). The hyperlinks to the table notes indeed work.
%
% \pagebreak
%
% \item[-] The \xpackage{threeparttable} package creates tablenotes again. Therefore
%   there is no float problem (because the tablenotes are not continuously numbered with
%   the footnotes). There are no hyperlinks to the table notes (at least not automatically).
%   Using |sidewaystable| (with table notes) works.\\
%   \url{http://www.ctan.org/pkg/threeparttable}
%
% \item[-] The \xpackage{threeparttablex} package creates tablenotes again. Therefore there
%   is no float problem (because the tablenotes are not continuously numbered with the
%   footnotes). With option \texttt{referable} the tablenotes are hyperlinked. Use of
%   a |sidewaystable| (with table notes) did not work for me. When using the
%   \texttt{referable} option according to the example in the \xpackage{threeparttablex}
%   manual the \xpackage{longtable} package is used, therefore that package could be used
%   directly without \xpackage{threeparttablex} (see above).\\
%   \url{http://www.ctan.org/pkg/threeparttablex}
%
% \item[-] One can manually use |\footnotemark| in the table and |\footnotetext{...}|
%  after the table environment ended and manually change the |footnote| and
%  (when the \xpackage{hyperref} package is used) |Hfootnote| counters and needs
%  to insert
%  \begin{verbatim}
%     \makeatletter
%     \global\let\Hy@saved@currentHref\@currentHref
%     \hyper@makecurrent{Hfootnote}%
%     \global\let\Hy@footnote@currentHref\@currentHref
%     \global\let\@currentHref\Hy@saved@currentHref
%     \makeatother
%  \end{verbatim}
%  before each |\footnotetext{...}|, but with more than very few footnotes this can become
%  much work (and is prone to produce errors). And this does not even work with
%  sidewaystables. (And it does not work with the \xpackage{footnotebackref} package
%  without redefining |\footnotemark| and |\footnotetext{...}|.)
% \end{description}
%
% \bigskip
%
% \noindent (You programmed or found another alternative,
%  which is available at \CTAN{}?\\
%  OK, send an e-mail to me with the name, location at \CTAN{},
%  and a short notice, and I will probably include it in
%  the list above.)
%
% \newpage
%
% \section{Example}
%
%    \begin{macrocode}
%<*example>
\documentclass[british]{article}[2007/10/19]% v1.4h
%%%%%%%%%%%%%%%%%%%%%%%%%%%%%%%%%%%%%%%%%%%%%%%%%%%%%%%%%%%%%%%%%%%%%
\usepackage{float}[2001/11/08]%    v1.3d
\usepackage{placeins}[2005/04/18]% v2.2  ; for \FloatBarrier
\usepackage{rotating}[2009/03/28]% v2.16a; for sidewaystable-environment
\usepackage[%
 hyperfootnotes=true,%
 extension=pdf,%
 plainpages=false,%
 pdfpagelabels=true,%
 hyperindex=false,%
 pdflang={en},%
 pdftitle={tablefootnote package example},%
 pdfauthor={H.-Martin Muench},%
 pdfsubject={Example for the tablefootnote package},%
 pdfkeywords={LaTeX, tablefootnote, footnote, table, H.-Martin Muench},%
 % pdfview=FitH and FitBH do not work: hyperlinks in sidewaystables
 % do no lead to the footnotes, due to a bug in pdfTeX,
 % computing wrong anchor coordinates (Heiko Oberdiek, 29. October 2011)
 % pdfview=Fit, FitV, FitR, FitB, FitBV work
 % print is OK for all those options
 pdfstartview=FitH,%
 pdfpagelayout=OneColumn%
]{hyperref}[2012/11/06]% v6.83m
 % Due to the urls used in the example, either the hyperref or the url
 % package are needed (or the urls must be removed before compiling).

\usepackage{footnotebackref}[2012/07/01]% v1.0
\usepackage{tablefootnote}[2014/01/26]%   v1.1c

\gdef\unit#1{\mathord{\thinspace\mathrm{#1}}}%
\listfiles
\begin{document}
\pagenumbering{arabic}
\section*{Example for tablefootnote}

This example demonstrates the use of package\newline
\textsf{tablefootnote}, v1.1c as of 2014/01/26 (HMM).\newline
There were no options used. (The package provides no options.)\newline

\textbf{The \texttt{tablefootnote-example.tex} needs to be compiled
at least twice to get the references right!}\newline

If the etoolbox-package is found, it is automatically used.\newline

For more details please see the documentation!\newline

\noindent Save per page about $200\unit{ml}$ water,
$2\unit{g}$ CO$_{2}$ and $2\unit{g}$ wood:\newline
Therefore please print only if this is really necessary.\newline

Here is some text.\footnote{And this is a text footnote.}\newline

Tables \ref{tab.symbol}, \ref{tab.normal}, \ref{tab.another} and
\ref{tab.floatH} show normal tables, table~\ref{tab.sideways} depicts
a sidewaystable. Table~\ref{tab.floatH} uses the float
specifier~\texttt{H} from the float package.\newline

\texttt{Hyperref} option \verb|pdfview=FitH| and \verb|FitBH| do
not work due to a bug in pdf\TeX{}, computing wrong
anchor coordinates (\textsc{Heiko Oberdiek}, 29.~October 2011).
Depending on used pdf-viewer, hyperlinks in sidewaystables lead
e.\,g.~at the end of the document, not at the footnote.
\verb|pdfview=Fit|, \verb|FitV|, \verb|FitR|, \verb|FitB|,
\verb|FitBV| work, print is OK for all those options.

\bigskip

\listoftables

\pagebreak

\renewcommand{\thefootnote}{\fnsymbol{footnote}}
\verb|\renewcommand{\thefootnote}{\fnsymbol{footnote}}|
causes footnote{-}symbol{-}footnotes, which are possible
(see Table~\ref{tab.symbol}).

\begin{table}
\centering
\begin{tabular}{ccc}
Another\tablefootnote{A table footnote.} & %
text\tablefootnote{Another table footnote.} & %
in a table\tablefootnote{A third table footnote.}
\end{tabular}
\caption[A footnotesymbol table]{%
A table with footnote-symbol-footnotes.\label{tab.symbol}}
\end{table}

Some text.\footnote{A text footnote.}

\renewcommand{\thefootnote}{\arabic{footnote}}
\verb|\renewcommand{\thefootnote}{\arabic{footnote}}|
switches back to normal footnote numbers again.

\pagebreak

\begin{table}
\centering
\begin{tabular}{ccc}
Some\tablefootnote{A table footnote.} & %
text\tablefootnote[99]{A table footnote with custom footnote number.} & %
in a table\tablefootnote{A third table footnote.}
\end{tabular}
\caption[A table]{A normal table.\label{tab.normal}}
\end{table}

Some text.\footnote{Another text footnote.}

\pagebreak

%    \end{macrocode}
%
% \pagebreak
%
%    \begin{macrocode}
More text.\footnote{And yet another text footnote.}

\begin{table}[t]
\centering
\begin{tabular}{|c|c|c|}
\hline
Another\tablefootnote{A $3^{rd}$ table footnote.} & %
text\tablefootnote{Another $3^{rd}$ table footnote.} & %
in a table\tablefootnote{A $3^{rd}$ third table footnote.}\\ \hline
\end{tabular}
\caption[Another table]{Another table (third one)\label{tab.another}}
\end{table}

Please note that Table~\ref{tab.another} floated to the top of the page.
While the footnotes are set and (when hyperref is used) hyperlinked,
they are not automatically adapted. Thus either do not use a footnote
at the same page before the table, or place the table in
\textquotedblleft here\textquotedblright\ or
\textquotedblleft bottom\textquotedblright\ position.
\verb|\clear(double)page|, \verb|h(!)|, \verb|H|~from the
\texttt{float} package, or \verb|\FloatBarrier| from the
\texttt{picins} package might help, too. (Or move the table in the
source code near the position where it floats to
or use the optional footnote marks.)

Table~\ref{tab.floatH} (page~\pageref{tab.floatH}) uses float specifier
\texttt{H} from the float package and does not float.

Some text.\footnote{This is just another text footnote.}

\pagebreak

\FloatBarrier

\begin{sidewaystable}
\centering%
\begin{tabular}{ccc}
Text\tablefootnote{Please rotate the view for testing the %
hyperlinks.} & %
in a\tablefootnote[98]{Another sidewaystable footnote %
with optional footnote mark.} & %
sidewaystable%
\tablefootnote{Sidewaystable-environment provided by %
\url{http://www.ctan.org/pkg/rotating} package.}
\end{tabular}
\caption[A sideways table]{A table in the \texttt{sideways} %
environment\label{tab.sideways}}%
\end{sidewaystable}%

\FloatBarrier

\pagebreak

%    \end{macrocode}
%
% \pagebreak
%
%    \begin{macrocode}
A last table, here with float specifier \texttt{H} from the
float\footnote{\url{http://www.ctan.org/pkg/float}} package.

\begin{table}[H]
\centering
\begin{tabular}{ccc}
Another\tablefootnote{A $5^{th}$ table footnote.} & %
text\tablefootnote{Another $5^{th}$ table footnote.} & %
in a table\tablefootnote{A $5^{th}$ third table footnote.}
\end{tabular}
\caption[A last table]{A very last table\label{tab.floatH}}
\end{table}

Some text.\footnote{This is just another text footnote.}

\pagebreak

End of the example for the
tablefootnote\footnote{\url{http://www.ctan.org/pkg/tablefootnote}}
package.

\end{document}
%</example>
%    \end{macrocode}
%
% \StopEventually{}
%
% \newpage
%
% \section{The implementation}
%
% We start off by checking that we are loading into \LaTeXe{} and
% announcing the name and version of this package.
%
%    \begin{macrocode}
%<*package>
%    \end{macrocode}
%
%    \begin{macrocode}
\NeedsTeXFormat{LaTeX2e}[2011/06/27]
\ProvidesPackage{tablefootnote}[2014/01/26 v1.1c
              Table foot notes (HMM)]
%    \end{macrocode}
%
% A short description of the \xpackage{tablefootnote} package:
%
%    \begin{macrocode}
%% Provides the \tablefootnote[...]{...} command
%% for footnotes in (sideways)tables.

%    \end{macrocode}
%
% We need the \xpackage{ltxcmds} and \xpackage{letltxmacro} packages
% by \textsc{Heiko Oberdiek}:
%
%    \begin{macrocode}
\RequirePackage{ltxcmds}[2011/11/09]%     v1.22
\RequirePackage{letltxmacro}[2010/09/02]% v1.4
\RequirePackage{xifthen}[2009/04/17]%     v1.3
%    \end{macrocode}
%
% A last information for the user:
%
%    \begin{macrocode}
%% tablefootnote may work with earlier versions of LaTeX and those
%% packages, but this was not tested. Please consider updating
%% your LaTeX and packages to the most recent version
%% (if they are not already the most recent version).

%    \end{macrocode}
%
% \indent See subsection~\ref{ss:Downloads} about how to get them.\\
%
% When the \xpackage{rotating} package (for |sidewaystable|s) is used,
% it must be loaded before the \xpackage{tablefootnote} package and it must be
% a recent version:
%
%    \begin{macrocode}
\@ifpackageloaded{rotating}{%
  \gdef\tfn@rotating{1}
  \@ifpackagelater{rotating}{2009/03/28}% v2.16a
   {% >= 2009/03/28, OK
   }{% else
     \PackageWarningNoLine{tablefootnote}{%
       It is required version\MessageBreak%
       2009/03/28 v2.16a (or later) of package rotating,\MessageBreak%
       but only version\MessageBreak%
      `\csname ver@rotating.sty\endcsname'\MessageBreak%
      is available}%
   }%
 }{% else
   \gdef\tfn@rotating{0}%
 }

%    \end{macrocode}
%
% There are no options to be processed.\\
%
% We need some definitions:
%
%    \begin{macrocode}
\def\tfn@footnotetablecount{0}
\def\tfn@footnotetableprint{0}
\def\tfn@footnotezero{0}
\def\tfn@fnt{0}
\def\tfn@swt{0}%       _s_ide_w_ays_t_able
\def\tfn@fmpw{\relax}% f_oot_m_isc_p_ara_w_arn-switch
\def\tfn@fnbr{0}%      f_oot_n_ote_b_ack_r_ef(erence package)

%    \end{macrocode}
%
% \begin{macro}{\tfn@hyperfootnotes}
% We need a wrapper for the used |\ifHy@hyperfootnotes|,
% which is not defined when the \xpackage{hyperref} package has not been loaded:
%
%    \begin{macrocode}
\DeclareRobustCommand{\tfn@hyperfootnotes}[2]{%
  \ifHy@hyperfootnotes%
    #1%
  \else%
    #2%
  \fi%
  }

%    \end{macrocode}
% \end{macro}
%
% \begin{macro}{\tablemakefntext}
% |\tablemakefntext| makes the footnotetext for a footnote in a table.
%
%    \begin{macrocode}
\ltx@ifpackageloaded{footnotebackref}{%
  \gdef\tfn@fnbr{1}%
  \ifx\FootnoteBackref@symbol\empty%
    \ifFootnoteBackref@numberlinked%
      \newcommand\tablemakefntext[1]{%
        \noindent\makebox[1.634em][r]{%
         \mbox{\textsuperscript{\normalfont%
           \hyperlink{tfn:m\theHfootnote}{\@thefnmark}%
        }}}%
        \ #1%
      }
    \else
      \newcommand\tablemakefntext[1]{%
        \noindent\makebox[1.8em][r]{%
          \mbox{\textsuperscript{\normalfont%
            \@thefnmark%
        }}}%
        #1%
      }
    \fi
  \else
    \newcommand\tablemakefntext[1]{%
      \noindent\makebox[1.8em][r]{%
        \mbox{\textsuperscript{\normalfont%
          \ifFootnoteBackref@numberlinked%
            \hyperlink{tfn:m\theHfootnote}{\@thefnmark}\ %
          \else%
            \@thefnmark%
          \fi%
      }}}%
      \hyperlink{tfn:m\theHfootnote}{\FootnoteBackref@symbol}%
      #1%
    }
  \fi
}{%
  \newcommand\tablemakefntext[1]{%
    \noindent\makebox[1.8em][r]{%
      \mbox{\textsuperscript{\normalfont%
        \@thefnmark%
    }}}%
    #1%
  }
}

%    \end{macrocode}
%
% \end{macro}
%
% \pagebreak
%
% \begin{macro}{\swtablemakefntext}
%    \begin{macrocode}
\newcommand\swtablemakefntext[1]{\noindent\ #1}

%    \end{macrocode}
% \end{macro}
%
% \begin{macro}{\tfn@footnotemarks}
% |\tfn@footnotemarks| uses the |\footnotemark| command to place a foot note mark
% and decreases the |\tfn@footnotetablecount| (i.\,e.~increases the negative value)
% of the number of footnote(marks) in the table. If the \xpackage{footnotebackref}
% package is used, it is also necessary to create an according |\hypertarget| for
% the back reference.
%
%    \begin{macrocode}
\newcommand{\tfn@footnotemarks}[1][]{%
  \hbox{%
    \ifthenelse{\isempty{#1}}%
      {\footnotemark%
       \ltx@ifpackageloaded{footnotebackref}%
         {\raisebox{\ht\strutbox}{\hypertarget{tfn:m\theHfootnote}{}}}%
         {% footnotebackref not loaded
         }%
      }%
      {\ltx@ifpackageloaded{hyperref}%
        {\tfn@hyperfootnotes{%
            \hyperlink{tfn:#1}{\footnotemark[#1]}%
          }{\footnotemark[#1]}}%
        {\footnotemark[#1]}%
       \ltx@ifpackageloaded{footnotebackref}%
        {\raisebox{\ht\strutbox}{\hypertarget{tfn:m#1}{}}}{}%
      }%
   }%
  \@tempcnta=\tfn@footnotetablecount\relax%
  \advance\@tempcnta by -1\relax%
  \xdef\tfn@footnotetablecount{\the\@tempcnta}%
}

%    \end{macrocode}
% \end{macro}
%
% \begin{macro}{\tfn@footmiscparawarn}
% When the \xpackage{footmisc} package is used with option |para|,
% we need to issue a warning. (Any idea how to resolve this issue?)\\
% It must be given inside of it's own command, because |\ifFN@para| is not
% defined when \xpackage{footmisc} has not been loaded, and then the
% |\fi| would be a |\fi| without |\if| (and \TeX{} does not like single
% |\fi|s).
%
%    \begin{macrocode}
\newcommand{\tfn@footmiscparawarn}{%
  \ifFN@para%
    \PackageWarning{tablefootnote}{%
      Package footmisc with option para used.\MessageBreak%
      Unfortunately package tablefootnote is not yet\MessageBreak%
      able to regard this in sidewaystables.\MessageBreak%
      Here the footnotes are printed one below\MessageBreak%
      the other. Sorry!\MessageBreak%
     }%
  \fi%
  }

%    \end{macrocode}
% \end{macro}
%
% \begin{macro}{\tfn@updateHyper}
% |\tfn@updateHyper| updates some \xpackage{hyperref} internals.
%
%    \begin{macrocode}
\newcommand{\tfn@updateHyper}{%
  % The code of this command is partly from the
  % hyperref package by Heiko Oberdiek, 2011/10/01, v6.82j.
  % Newer versions of that package are available at CTAN.org.
  \global\let\Hy@saved@currentHref\@currentHref%
  \tfn@hyperfootnotes{\hyper@makecurrent{Hfootnote}}{}%
  \global\let\Hy@footnote@currentHref\@currentHref%
  \global\let\@currentHref\Hy@saved@currentHref%
  }

%    \end{macrocode}
% \end{macro}
%
% \begin{macro}{\tfn@footnotetext}
% |\tfn@footnotetext| contains the |\footnotetext[...]{...}| command and handles
% the numbering. |\tfn@footnotetext| will be executed after the table.
% The foot note counter has been increased automatically with each |\footnotemark|,
% therefore it must be reset to its value before the table began.
%
%    \begin{macrocode}
\newcommand{\tfn@footnotetext}[2][]{%
  \ifx\tfn@footnotetableprint\tfn@footnotezero%
    % This is the first footnotetext after the (sideways)table.
    \addtocounter{footnote}{\tfn@footnotetablecount}%
%    \end{macrocode}
%
% When the \xpackage{hyperref} package is used, also its |Hfootnote| counter must
% be reset (when the footnotes are hyperlinked).
%
%    \begin{macrocode}
    \ltx@ifpackageloaded{hyperref}{%
      \tfn@hyperfootnotes{%
        \addtocounter{Hfootnote}{\tfn@footnotetablecount}}{}%
      \tfn@updateHyper%
     }{}%
  \fi%
%    \end{macrocode}
%
% For issuing the text for the next footnote the footnote-counter
% (and |Hfootnote| counter in case of links) are increased by one.
%
%    \begin{macrocode}
  \addtocounter{footnote}{+1}%
  \ltx@ifpackageloaded{hyperref}{%
    \tfn@hyperfootnotes{\addtocounter{Hfootnote}{+1}}{}%
%    \end{macrocode}
%
% The \xpackage{hyperref} package was not expecting this kind of change of the
% |(H)footnote| numbers, therefore some code-lines are missing here which can be
% found at other places in that same package and are called by this command.
%
%    \begin{macrocode}
    \tfn@updateHyper%
%    \end{macrocode}
%
% When \xpackage{hyperref} is not loaded, nothing is to be done here.
%
%    \begin{macrocode}
   }{}%
%    \end{macrocode}
%
% The |\footnotetext[...]{...}| places the foot note text as usual.
%
%    \begin{macrocode}
  \ifx\tfn@swt\tfn@footnotezero%
    \ifthenelse{\isempty{#1}}%
      {\footnotetext{#2}}%
      {\ltx@ifpackageloaded{hyperref}%
         {\tfn@hyperfootnotes{%
             \footnotetext[#1]{%
               \raisebox{\ht\strutbox}{\hypertarget{tfn:#1}{}}#2}%
           }{\footnotetext[#1]{#2}%
           }%
         }{\footnotetext[#1]{#2}}%
      }%
  \else%
%    \end{macrocode}
%
% If there is more than one tablefootnote in the |sidewaystable|, \\
% the |\tfn@footmiscparawarn| shall be given,
%
%    \begin{macrocode}
    \ifx\tfn@footnotetableprint\tfn@footnotezero%
      \gdef\tfn@fmpw{\tfn@footmiscparawarn}%
    \else%
      \ltx@ifpackageloaded{footmisc}{\tfn@fmpw}{\relax}%
%    \end{macrocode}
%
% but we do not want to repeat the message for a third(, forth,\ldots)
% tablefootnote in the |sidewaystable|.
%
%    \begin{macrocode}
      \gdef\tfn@fmpw{\relax}%
    \fi%
%    \end{macrocode}
%
% |\@makefnmark| is \\
% |\hbox {\@textsuperscript {\normalfont \@thefnmark }}|, \\
% but according to |\show\@thefnmark| here |\@thefnmark| is
% |{\protect \itshape}| instead of |\thefootnote| (Why?!).\\
% If the \xpackage{footnotebackref} package is used,
% back references must be created for the table footnotes, too.
%
%    \begin{macrocode}
    \ltx@ifpackageloaded{footnotebackref}{%
      \LetLtxMacro{\tfncurrentmft}{\@makefntext}%
      \LetLtxMacro{\@makefntext}{\swtablemakefntext}%
      \ifthenelse{\isempty{#1}}%
        {\footnotetext{%
           \raisebox{\ht\strutbox}{\hypertarget{tfn:\theHfootnote}{}}%
           \hbox{\@textsuperscript{\normalfont%
              \hyperlink{tfn:m\theHfootnote}{\thefootnote}}}\thinspace%
         \ifx\FootnoteBackref@symbol\empty%
         \else\hyperlink{tfn:m\theHfootnote}{\FootnoteBackref@symbol}%
         \fi%
         #2}%
       }{\edef\tfn@tfn{\arabic{footnote}}%
         \edef\tfn@thn{\arabic{Hfootnote}}%
         \setcounter{footnote}{#1}%
         \setcounter{Hfootnote}{#1}%
         \tfn@updateHyper%
         \footnotetext{%
           \raisebox{\ht\strutbox}{\hypertarget{tfn:#1}{}}%
           \hbox{\@textsuperscript{\normalfont%
                  \hyperlink{tfn:m#1}{#1}}}%
           \thinspace%
           \ifx\FootnoteBackref@symbol\empty%
           \else%
             \raisebox{\ht\strutbox}{%
               \hyperlink{tfn:m#1}{\FootnoteBackref@symbol}}%
           \fi%
           #2%
          }%
         \setcounter{footnote}{\tfn@tfn}%
         \setcounter{Hfootnote}{\tfn@thn}%
         \tfn@updateHyper%
       }%
      \LetLtxMacro{\@makefntext}{\tfncurrentmft}%
     }{% footnotebackref not loaded
       \ifthenelse{\isempty{#1}}%
         {\footnotetext{%
            \hbox{\@textsuperscript{\normalfont\thefootnote}}%
            \ltx@ifpackageloaded{hyperref}%
              {\tfn@hyperfootnotes{%
                 \raisebox{\ht\strutbox}{%
                   \hypertarget{tfn:\theHfootnote}{}}}{}%
              }{}%
            \thinspace%
            #2}%
        }{%
%    \end{macrocode}
%
% \pagebreak
%
%    \begin{macrocode}
          \edef\tfn@tfn{\arabic{footnote}}%
          \ltx@ifpackageloaded{hyperref}{%
            \tfn@hyperfootnotes{\xdef\tfn@thn{\arabic{Hfootnote}}}{}%
           }{}%
          \setcounter{footnote}{#1}
          \ltx@ifpackageloaded{hyperref}{%
            \tfn@hyperfootnotes{\setcounter{Hfootnote}{#1}}{}%
            \tfn@updateHyper%
           }{}%
          \footnotetext{%
            \hbox{\@textsuperscript{\normalfont #1}}%
            \ltx@ifpackageloaded{hyperref}{%
              \tfn@hyperfootnotes{\raisebox{\ht\strutbox}{%
                                         \hypertarget{tfn:#1}{}}}{}%
             }{}%
            \thinspace%
            #2}%
          \setcounter{footnote}{\tfn@tfn}%
          \ltx@ifpackageloaded{hyperref}{%
            \tfn@hyperfootnotes{\setcounter{Hfootnote}{\tfn@thn}}{}%
            \tfn@updateHyper%
           }{}%
        }%
     }%
  \fi%
%    \end{macrocode}
%
% For further footnotes in the table the \\
% |\ifx\tfn@footnotetableprint\tfn@footnotezero|\ldots|\fi| \\
% from the beginning of this |\tfn@footnotetext| macro must not be repeated,
% therefore we deactivate this by defining |\tfn@footnotetableprint| to
% something other than |0| (here |1| is used).
%
%    \begin{macrocode}
  \gdef\tfn@footnotetableprint{1}%
%    \end{macrocode}
%
% |\tfn@footnotetablecount| contains the negative number of footnote texts
% to be printed, therefore it is increase by one (i.\,e.~the negative number
% decreases, i.\,e.~$-3$ becomes~$-2$).
%
%    \begin{macrocode}
  \@tempcnta=\tfn@footnotetablecount\relax%
  \advance\@tempcnta by +1\relax%
  \xdef\tfn@footnotetablecount{\the\@tempcnta}%
%    \end{macrocode}
%
% When it is zero, all footnote texts have been placed, and\\
% |\tfn@footnotetableprint| can be reset to zero and\\
% |\tfn@tablefootnoteprintout| can be cleared.
%
%    \begin{macrocode}
  \ifx\tfn@footnotetablecount\tfn@footnotezero% all footnotes placed
    \gdef\tfn@footnotetableprint{0}%
    \global\let\tfn@tablefootnoteprintout\relax% brute force
 %\else there is at least one \footnotemark which needs \footnotetext
  \fi
}

%    \end{macrocode}
% \end{macro}
%
% \begin{macro}{\tfn@tablefootnoteprintout}
% When something shall be cleared or redefined, it is a good idea
% to define it. It~will be later filled with the foot note text to be printed.
%
%    \begin{macrocode}
\newcommand{\tfn@tablefootnoteprintout}{\relax}

%    \end{macrocode}
% \end{macro}
%
% \pagebreak
%
% \begin{macro}{\tablefootnote}
% |\tablefootnote[...]{...}| is the command to be used by the user instead of
% |\footnote[...]{...}| in the (sideways)table. It places a foot note mark via
% |\tfn@footnotemarks| and appends a |\tfn@footnotetext{#1}| to the\\
% |\tfn@tablefootnoteprintout| macro, and remembers that table foot notes
% have been used.
%
%    \begin{macrocode}
\newcommand{\tablefootnote}[2][]{%
  \tfn@footnotemarks[#1]%
  \ifthenelse{\isempty{#1}}{\relax}{%
    \ltx@GlobalPrependToMacro{\tfn@tablefootnoteprintout}{%
      \addtocounter{footnote}{+1}%
      \ltx@ifpackageloaded{hyperref}{%
        \tfn@hyperfootnotes{\addtocounter{Hfootnote}{+1}}{}%
        \tfn@updateHyper%
       }{}%
     }%
   }%
  \ltx@GlobalAppendToMacro{\tfn@tablefootnoteprintout}{%
    \tfn@footnotetext[#1]{#2}%
    \ifthenelse{\isempty{#1}}{\relax}{%
      \addtocounter{footnote}{-1}%
      \ltx@ifpackageloaded{hyperref}{%
        \tfn@hyperfootnotes{\addtocounter{Hfootnote}{-1}}{}%
        \tfn@updateHyper%
       }{}%
     }%
   }%
  \gdef\tfn@fnt{1}%
  }

%    \end{macrocode}
% \end{macro}
%
% \begin{macro}{\tfn@swtbox}
% For |sidewaystable|s the |\tfn@tablefootnoteprintout| is placed in a \\
% |\parbox|, which can be placed at the end of the |sidewaystable|:
%
%    \begin{macrocode}
\newcommand{\tfn@swtbox}{%
\ifvoid \footins%
\vskip 2\baselineskip%
\else%
\vskip \skip\footins%
\fi%
\parbox[b][][t]{\linewidth}{\raggedright\tfn@tablefootnoteprintout}%
}

%    \end{macrocode}
% \end{macro}
%
% |\tfn@tablefootnoteprintout| shall be executed after the end of the table environment.
% The \xpackage{etoolbox} package provides very easy syntax for this:
%
%    \begin{macrocode}
\IfFileExists{etoolbox.sty}{%
  \RequirePackage{etoolbox}[2011/01/03]%   v2.1
  \AfterEndEnvironment{table}{%
    \ltx@ifpackageloaded{footnotebackref}{%
      \LetLtxMacro{\tfncurrentmft}{\@makefntext}%
      \LetLtxMacro{\@makefntext}{\tablemakefntext}%
       \tfn@tablefootnoteprintout%
      \LetLtxMacro{\@makefntext}{\tfncurrentmft}%
     }{% footnotebackref not loaded
      \tfn@tablefootnoteprintout%
     }%
      \gdef\tfn@fnt{0}%
   }%
%    \end{macrocode}
%
% When the \xpackage{rotating} package has been loaded, we need to detect
% whether the |\tablefootnote| is placed inside of a |sidewaystable|.
%
%    \begin{macrocode}
  \ltx@ifpackageloaded{rotating}{%
    \AtBeginEnvironment{sidewaystable}{%
      \gdef\tfn@swt{1}%
    }
%    \end{macrocode}
%
% At the end of |sidewaystable| environments |\tfn@swtbox| shall be placed
% (if~there was any |\tablefootnote| in that |sidewaystable| at all).
% And |\tfn@swt| as well as |\tfn@fnt| must be reset to |0|.
%
%    \begin{macrocode}
    \AtEndEnvironment{sidewaystable}{%
      \ifx\tfn@fnt\tfn@footnotezero% \relax
      \else%
        \ltx@ifpackageloaded{footnotebackref}{%
          \LetLtxMacro{\tfncurrentmft}{\@makefntext}%
          \LetLtxMacro{\@makefntext}{\tablemakefntext}%
           \tfn@swtbox%
          \LetLtxMacro{\@makefntext}{\tfncurrentmft}%
         }{% footnotebackref not loaded
          \tfn@swtbox%
          }%
      \fi%
      \gdef\tfn@swt{0}%
      \gdef\tfn@fnt{0}%
     }%
   }{}% else \relax
}{%
%    \end{macrocode}
%
% When the \xpackage{etoolbox} package is not available (e.\,g. because of missing
% \eTeX{}) we~do a workaround:
%
%    \begin{macrocode}
  \PackageWarning{tablefootnote}{%
    Package etoolbox not available, therefore\MessageBreak%
    substituting commands \string\AtEndEnvironment\space %
    and\MessageBreak%
    \string\AfterEndEnvironment\space for tablefootnotes%
    \MessageBreak%
   }%
  \@ifpackagelater{letltxmacro}{2010/09/02}%
   {% >= 2010/09/02, OK
    \GlobalLetLtxMacro{\tfnorigtable}{\table}
    \GlobalLetLtxMacro{\tfnendorigtable}{\endtable}
%    \end{macrocode}
%
% |\GlobalLetLtxMacro| was introduced in version 2010/09/02 v1.4 of the
% \xpackage{letltxmacro} package.
%
%    \begin{macrocode}
   }{% else
     \PackageError{tablefootnote}{%
       Outdated version of letltxmacro package used%
      }{Version 2010/09/02 v1.4 or newer needed but not found!%
        \MessageBreak%
        Update the used letltxmacro package.}
        \LetLtxMacro{\tfnorigtable}{\table}
        \LetLtxMacro{\tfnendorigtable}{\endtable}
    }%
%    \end{macrocode}
%
% \pagebreak
%
%    \begin{macrocode}
  \ltx@ifpackageloaded{float}{%
    \@ifpackagelater{float}{2001/11/08}%   v1.3d
     {% >= 2001/11/08, OK
     }{% else
       \PackageWarningNoLine{tablefootnote}{%
         It is required version\MessageBreak
         2001/11/08 v1.3d (or later) of package float,\MessageBreak
         but only version\MessageBreak
        `\csname ver@float.sty\endcsname'\MessageBreak
        is available}%
     }%
   }{\relax}
  \renewenvironment{table}{%
    \ltx@ifpackageloaded{float}{%
%    \end{macrocode}
%
% When \xpackage{etoolbox.sty} was not found, and the \xpackage{float} package
% has been loaded, and the float specifier is \texttt{H}, then |\endtable| was
% let to |\float@endH| by the \xpackage{float} package (line~79 of
% \xpackage{float.sty}, 2001/11/08 v1.3d:\\
% |\expandafter\let\csname end#1\endcsname\float@endH|), i.\,e.~appending\\
% |\tfn@tablefootnoteprintout \gdef\tfn@fnt{0}| to |\endtable| is useless.
% Instead of this, it must be appended to |\float@endH|.\\
% (Why is this not necessary when \xpackage{etoolbox} is used?
% That is a good question, answers are welcome.)
%
%    \begin{macrocode}
      \ltx@LocalAppendToMacro{\float@endH}{%
        \ltx@ifpackageloaded{footnotebackref}{%
          \LetLtxMacro{\tfncurrentmft}{\@makefntext}%
          \LetLtxMacro{\@makefntext}{\tablemakefntext}%
           \tfn@tablefootnoteprintout%
          \LetLtxMacro{\@makefntext}{\tfncurrentmft}%
         }{% footnotebackref not loaded
          \tfn@tablefootnoteprintout%
         }%
          \gdef\tfn@fnt{0}%
       }%
     }{\relax}
    \tfnorigtable%
   }{%
     \tfnendorigtable%
     \ltx@ifpackageloaded{footnotebackref}{%
       \LetLtxMacro{\tfncurrentmft}{\@makefntext}%
       \LetLtxMacro{\@makefntext}{\tablemakefntext}%
        \tfn@tablefootnoteprintout%
       \LetLtxMacro{\@makefntext}{\tfncurrentmft}%
      }{% footnotebackref not loaded
       \tfn@tablefootnoteprintout%
      }%
     \gdef\tfn@fnt{0}%
    }%
%    \end{macrocode}
%
% \pagebreak
%
% The |sidewaystable|-environment must be treated separately:
%
%    \begin{macrocode}
  \ltx@ifpackageloaded{rotating}{%
    \let\tfnorigsidewaystable\sidewaystable
    \let\tfnendorigsidewaystable\endsidewaystable
    \renewenvironment{sidewaystable}{%
      \gdef\tfn@swt{1}%
      \tfnorigsidewaystable%
    }{%
      \ifx\tfn@fnt\tfn@footnotezero% \relax
      \else%
        \ltx@ifpackageloaded{footnotebackref}{%
          \LetLtxMacro{\tfncurrentmft}{\@makefntext}%
          \LetLtxMacro{\@makefntext}{\tablemakefntext}%
           \tfn@swtbox%
          \LetLtxMacro{\@makefntext}{\tfncurrentmft}%
         }{% footnotebackref not loaded
          \tfn@swtbox%
         }%
      \fi%
      \gdef\tfn@swt{0}%
      \gdef\tfn@fnt{0}%
      \tfnendorigsidewaystable%
     }%
   }{}%
}

%    \end{macrocode}
%
% \DescribeMacro{\AtBeginDocument}
% When the \xpackage{rotating} (for |sidewaystable|s) or the
% \xpackage{footnotebackref} (for back references for the footnotes)
% package is used, it must be loaded before the \xpackage{tablefootnote} package.
% Using both packages is also possible, in which case both must be loaded before
% the \xpackage{tablefootnote} package. If any one of those packages is loaded
% |\AtBeginDocument| after this check, the according error message will not be
% given, but packages really should be loaded before |\AtBeginDocument|.
%
%    \begin{macrocode}
\AtBeginDocument{%
  \ltx@ifpackageloaded{footnotebackref}{%
    \ifx\tfn@fnbr\tfn@footnotezero%
      \PackageError{tablefootnote}{%
        Package footnotebackref loaded after tablefootnote}{%
        When using both the footnotebackref and the tablefootnote %
        package,\MessageBreak%
        the footnotebackref package MUST be loaded before the %
        tablefootnote package!\MessageBreak%
       }%
    \fi%
   }{% else \relax
   }
  \ltx@ifpackageloaded{rotating}{%
    \ifx\tfn@rotating\tfn@footnotezero%
      \PackageError{tablefootnote}{%
        Package rotating loaded after tablefootnote}{%
        When using both the rotating and the tablefootnote package,%
        \MessageBreak%
        the rotating package MUST be loaded before the tablefootnote%
        \space package!\MessageBreak%
       }%
    \fi%
   }{% else \relax
   }
%    \end{macrocode}
%
% \pagebreak
%
% When the document is compiled with \LuaLaTeX{}, hyperlinks in rotated content
% will be misplaced, regardless of the use of the \xpackage{tablefootnote} package
% (or of tables or footnotes). The text to be printed will be OK.
% This problem of \LuaLaTeX{} (not \xpackage{tablefootnote}) was found and reported
% by \textsc{Arno Trautmann}, 2011/11/15 -- thanks!
%
%    \begin{macrocode}
  \ltx@ifpackageloaded{hyperref}{%
    \tfn@hyperfootnotes{
      \ifluatex%
        \PackageWarningNoLine{tablefootnote}{%
          LuaTeX and hyperref package used:\MessageBreak%
          Hyperreferences in rotated content\MessageBreak%
          will be misplaced\MessageBreak%
          regardless of tablefootnote package.\MessageBreak%
          The text to be printed will be OK%
         }%
      \fi%
     }{% else \relax
     }%
   }{% else \relax
   }%
%    \end{macrocode}
%
% When the \xpackage{footnotebackref} package is used but the footnotes are not
% hyperlinked, neither the back references nor the tablefootnotes will work.
%
%    \begin{macrocode}
  \ltx@ifpackageloaded{footnotebackref}{%
     \ltx@ifpackageloaded{hyperref}{%
        \tfn@hyperfootnotes{\relax}{% else
          \PackageError{tablefootnote}{%
             footnotebackref but not hyperfootnotes}{%
             The footnotebackref package is used\MessageBreak%
             but the hyperfootnotes option of the hyperref package%
             \MessageBreak%
             was set to false.\MessageBreak%
             Either enable hyperfootnotes or do not use footnotebackref.%
             \MessageBreak%
             Loading of the tablefootnote package will be aborted.%
           }%
          \endinput%
      }%
    }{\PackageError{tablefootnote}{footnotebackref but not hyperref}{%
        The footnotebackref package is used\MessageBreak%
        but not the hyperref package.\MessageBreak%
        Either load hyperref or do not use footnotebackref.%
        \MessageBreak%
        Loading of the tablefootnote package will be aborted.%
        }%
      \endinput%
    }%
  }{}
}

%    \end{macrocode}
%
%    \begin{macrocode}
%</package>
%    \end{macrocode}
%
% \pagebreak
%
% \section{Installation}
%
% \subsection{Downloads\label{ss:Downloads}}
%
% Everything is available on \CTAN{}, \url{http://www.ctan.org/},
% but may need additional packages themselves.\\
%
% \DescribeMacro{tablefootnote.dtx}
% For unpacking the |tablefootnote.dtx| file and constructing the documentation
% it is required:
% \begin{description}
% \item[-] \TeX{} Format \LaTeXe{}: \url{http://www.CTAN.org/}
%
% \item[-] document class \xpackage{ltxdoc}, 2007/11/11, v2.0u,
%           \url{http://www.ctan.org/pkg/ltxdoc}
%
% \item[-] package \xpackage{holtxdoc}, 2012/03/21, v0.24,
%           \url{http://www.ctan.org/pkg/holtxdoc}
%
% \item[-] package \xpackage{hypdoc}, 2011/08/19, v1.11,
%           \url{http://www.ctan.org/pkg/hypdoc}
% \end{description}
%
% \DescribeMacro{tablefootnote.sty}
% The |tablefootnote.sty| for \LaTeXe{} (i.\,e.~each document using
% the \xpackage{tablefootnote} package) requires:
% \begin{description}
% \item[-] \TeX{} Format \LaTeXe{}, \url{http://www.CTAN.org/}
%
% \item[-] package \xpackage{letltxmacro}, 2010/09/02, v1.4,
%           \url{http://www.ctan.org/pkg/letltxmacro}.
%
% \item[-] package \xpackage{ltxcmds}, 2011/11/09, v1.22,
%           \url{http://www.ctan.org/pkg/ltxcmds}
% \end{description}
%
% When the \xpackage{hyperref} package is used, also the \xpackage{ifluatex} package
% is needed, but it is already loaded automatically by the \xpackage{hyperref}
% package.\\
% When the \xpackage{etoolbox} package is available, it is used:
% \begin{description}
% \item[-] package \xpackage{etoolbox}, 2011/01/03, v2.1,
%           \url{http://www.ctan.org/pkg/etoolbox}.
% \end{description}
%
% \DescribeMacro{tablefootnote-example.tex}
% The |tablefootnote-example.tex| requires the same files as all
% documents using the \xpackage{tablefootnote} package and additionally:
% \begin{description}
% \item[-] class \xpackage{article}, 2007/10/19, v1.4h, from \xpackage{classes.dtx}:\\
%           \CTAN{macros/latex/base/classes.dtx}
%
% \item[-] package \xpackage{rotating}, 2009/03/28, v2.16a,
%           \url{http://www.ctan.org/pkg/rotating}
%
% \item[-] package \xpackage{float}, 2001/11/08, v1.3d,
%           \url{http://www.ctan.org/pkg/float}
%
% \item[-] package \xpackage{placeins}, 2005/04/18, v2.2,
%           \url{http://www.ctan.org/pkg/placeins}\\
%           for |\FloatBarrier|
%
% \item[-] package \xpackage{hyperref}, 2012/11/06, v6.83m,
%           \url{http://www.ctan.org/pkg/hyperref}
%
% \item[-] package \xpackage{tablefootnote}, 2014/01/26, v1.1c,\\
%           \url{http://www.ctan.org/pkg/tablefootnote}\\
%   (Well, it is the example file for this package, and because you
%    are reading the documentation for the \xpackage{tablefootnote} package,
%    it can be assumed that you already have some version of it~--
%    is it the current one?)
% \end{description}
%
% \DescribeMacro{Oberdiek}
% \DescribeMacro{holtxdoc}
% \DescribeMacro{ifluatex}
% \DescribeMacro{letltxmacro}
% \DescribeMacro{ltxcmds}
% All packages of \textsc{Heiko Oberdiek}'s bundle `oberdiek'
% (especially \xpackage{holtxdoc}, \xpackage{ifluatex}, \xpackage{letltxmacro},
% and \xpackage{ltxcmds}) are also available in a TDS compliant ZIP archive:\\
% \CTAN{install/macros/latex/contrib/oberdiek.tds.zip}.\\
% It is probably best to download and use this, because the packages in there
% are quite probably both recent and compatible among themselves.\\
%
% \pagebreak
%
% \DescribeMacro{hyperref}
% \noindent \xpackage{hyperref} is not included in that bundle and needs to be
% downloaded separately,\\
% \url{http://www.ctan.org/install/macros/latex/contrib/hyperref.tds.zip}.\\
%
% \DescribeMacro{M\"{u}nch}
% A hyperlinked list of my (other) packages can be found at
% \url{http://www.ctan.org/author/muench-hm}.\\
%
% \subsection{Package, unpacking TDS}
% \paragraph{Package.} This package is available on \CTAN{}
% \begin{description}
% \item[\CTAN{macros/latex/contrib/tablefootnote/tablefootnote.dtx}]\hspace*{0.1cm} \\
%       The source file.
% \item[\CTAN{macros/latex/contrib/tablefootnote/tablefootnote.pdf}]\hspace*{0.1cm} \\
%       The documentation.
% \item[\CTAN{macros/latex/contrib/tablefootnote/tablefootnote-example.pdf}]\hspace*{0.1cm} \\
%       The compiled example file, as it should look like.
% \item[\CTAN{macros/latex/contrib/tablefootnote/README}]\hspace*{0.1cm} \\
%       The README file.
% \end{description}
% There is also a tablefootnote.tds.zip available:
% \begin{description}
% \item[\CTAN{install/macros/latex/contrib/tablefootnote.tds.zip}]\hspace*{0.1cm} \\
%       Everything in \xfile{TDS} compliant, compiled format.
% \end{description}
% which additionally contains\\
% \begin{tabular}{ll}
% tablefootnote.ins & The installation file.\\
% tablefootnote.drv & The driver to generate the documentation.\\
% tablefootnote.sty & The \xext{sty}le file.\\
% tablefootnote-example.tex & The example file.
% \end{tabular}
%
% \bigskip
%
% \noindent For required other packages, please see the preceding subsection.
%
% \paragraph{Unpacking.} The \xfile{.dtx} file is a self-extracting
% \docstrip{} archive. The files are extracted by running the
% \xext{dtx} through \plainTeX{}:
% \begin{quote}
%   \verb|tex tablefootnote.dtx|
% \end{quote}
%
% \noindent About generating the documentation see paragraph~\ref{GenDoc} below.\\
%
% \paragraph{TDS.} Now the different files must be moved into
% the different directories in your installation TDS tree
% (also known as \xfile{texmf} tree):
% \begin{quote}
% \def\t{^^A
% \begin{tabular}{@{}>{\ttfamily}l@{ $\rightarrow$ }>{\ttfamily}l@{}}
%   tablefootnote.sty & tex/latex/tablefootnote/tablefootnote.sty\\
%   tablefootnote.pdf & doc/latex/tablefootnote/tablefootnote.pdf\\
%   tablefootnote-example.tex & doc/latex/tablefootnote/tablefootnote-example.tex\\
%   tablefootnote-example.pdf & doc/latex/tablefootnote/tablefootnote-example.pdf\\
%   tablefootnote.dtx & source/latex/tablefootnote/tablefootnote.dtx\\
% \end{tabular}^^A
% }^^A
% \sbox0{\t}^^A
% \ifdim\wd0>\linewidth
%   \begingroup
%     \advance\linewidth by\leftmargin
%     \advance\linewidth by\rightmargin
%   \edef\x{\endgroup
%     \def\noexpand\lw{\the\linewidth}^^A
%   }\x
%   \def\lwbox{^^A
%     \leavevmode
%     \hbox to \linewidth{^^A
%       \kern-\leftmargin\relax
%       \hss
%       \usebox0
%       \hss
%       \kern-\rightmargin\relax
%     }^^A
%   }^^A
%   \ifdim\wd0>\lw
%     \sbox0{\small\t}^^A
%     \ifdim\wd0>\linewidth
%       \ifdim\wd0>\lw
%         \sbox0{\footnotesize\t}^^A
%         \ifdim\wd0>\linewidth
%           \ifdim\wd0>\lw
%             \sbox0{\scriptsize\t}^^A
%             \ifdim\wd0>\linewidth
%               \ifdim\wd0>\lw
%                 \sbox0{\tiny\t}^^A
%                 \ifdim\wd0>\linewidth
%                   \lwbox
%                 \else
%                   \usebox0
%                 \fi
%               \else
%                 \lwbox
%               \fi
%             \else
%               \usebox0
%             \fi
%           \else
%             \lwbox
%           \fi
%         \else
%           \usebox0
%         \fi
%       \else
%         \lwbox
%       \fi
%     \else
%       \usebox0
%     \fi
%   \else
%     \lwbox
%   \fi
% \else
%   \usebox0
% \fi
% \end{quote}
% If you have a \xfile{docstrip.cfg} that configures and enables \docstrip{}'s
% \xfile{TDS} installing feature, then some files can already be in the right
% place, see the documentation of \docstrip{}.
%
% \subsection{Refresh file name databases}
%
% If your \TeX{}~distribution (\TeX live, \mikTeX{}, \dots{}) relies on file name
% databases, you must refresh these. For example, \teTeX{} users run
% \verb|texhash| or \verb|mktexlsr|.
%
% \subsection{Some details for the interested}
%
% \paragraph{Unpacking with \LaTeX{}.}
% The \xfile{.dtx} chooses its action depending on the format:
% \begin{description}
% \item[\plainTeX:] Run \docstrip{} and extract the files.
% \item[\LaTeX:] Generate the documentation.
% \end{description}
% If you insist on using \LaTeX{} for \docstrip{} (really,
% \docstrip{} does not need \LaTeX{}), then inform the autodetect routine
% about your intention:
% \begin{quote}
%   \verb|latex \let\install=y% \iffalse meta-comment
%
% File: tablefootnote.dtx
% Version: 2014/01/26 v1.1c
%
% Copyright (C) 2011 - 2014 by
%    H.-Martin M"unch <Martin dot Muench at Uni-Bonn dot de>
%
% This work may be distributed and/or modified under the
% conditions of the LaTeX Project Public License, either
% version 1.3c of this license or (at your option) any later
% version. This version of this license is in
%    http://www.latex-project.org/lppl/lppl-1-3c.txt
% and the latest version of this license is in
%    http://www.latex-project.org/lppl.txt
% and version 1.3c or later is part of all distributions of
% LaTeX version 2005/12/01 or later.
%
% This work has the LPPL maintenance status "maintained".
%
% The Current Maintainer of this work is H.-Martin Muench.
%
% This work consists of the main source file tablefootnote.dtx,
% the README, and the derived files
%    tablefootnote.sty, tablefootnote.pdf,
%    tablefootnote.ins, tablefootnote.drv,
%    tablefootnote-example.tex, tablefootnote-example.pdf.
%
% Distribution:
%    http://mirror.ctan.org/macros/latex/contrib/tablefootnote/tablefootnote.dtx
%    http://mirror.ctan.org/macros/latex/contrib/tablefootnote/tablefootnote.pdf
%    http://mirror.ctan.org/install/macros/latex/contrib/tablefootnote.tds.zip
%
% See http://www.ctan.org/pkg/tablefootnote (Catalogue)
%
% Unpacking:
%    (a) If tablefootnote.ins is present:
%           tex tablefootnote.ins
%    (b) Without tablefootnote.ins:
%           tex tablefootnote.dtx
%    (c) If you insist on using LaTeX
%           latex \let\install=y\input{tablefootnote.dtx}
%        (quote the arguments according to the demands of your shell)
%
% Documentation:
%    (a) If tablefootnote.drv is present:
%           (pdf)latex tablefootnote.drv
%           makeindex -s gind.ist tablefootnote.idx
%           (pdf)latex tablefootnote.drv
%           makeindex -s gind.ist tablefootnote.idx
%           (pdf)latex tablefootnote.drv
%    (b) Without tablefootnote.drv:
%           (pdf)latex tablefootnote.dtx
%           makeindex -s gind.ist tablefootnote.idx
%           (pdf)latex tablefootnote.dtx
%           makeindex -s gind.ist tablefootnote.idx
%           (pdf)latex tablefootnote.dtx
%
%    The class ltxdoc loads the configuration file ltxdoc.cfg
%    if available. Here you can specify further options, e.g.
%    use DIN A4 as paper format:
%       \PassOptionsToClass{a4paper}{article}
%
% Installation:
%    TDS:tex/latex/tablefootnote/tablefootnote.sty
%    TDS:doc/latex/tablefootnote/tablefootnote.pdf
%    TDS:doc/latex/tablefootnote/tablefootnote-example.tex
%    TDS:doc/latex/tablefootnote/tablefootnote-example.pdf
%    TDS:source/latex/tablefootnote/tablefootnote.dtx
%
%<*ignore>
\begingroup
  \catcode123=1 %
  \catcode125=2 %
  \def\x{LaTeX2e}%
\expandafter\endgroup
\ifcase 0\ifx\install y1\fi\expandafter
         \ifx\csname processbatchFile\endcsname\relax\else1\fi
         \ifx\fmtname\x\else 1\fi\relax
\else\csname fi\endcsname
%</ignore>
%<*install>
\input docstrip.tex
\Msg{******************************************************************}
\Msg{* Installation                                                   *}
\Msg{* Package: tablefootnote 2014/01/26 v1.1c Table foot notes (HMM) *}
\Msg{******************************************************************}

\keepsilent
\askforoverwritefalse

\let\MetaPrefix\relax
\preamble

This is a generated file.

Project: tablefootnote
Version: 2014/01/26 v1.1c

Copyright (C) 2011 - 2014 by
    H.-Martin M"unch <Martin dot Muench at Uni-Bonn dot de>

The usual disclaimer applies:
If it doesn't work right that's your problem.
(Nevertheless, send an e-mail to the maintainer
 when you find an error in this package.)

This work may be distributed and/or modified under the
conditions of the LaTeX Project Public License, either
version 1.3c of this license or (at your option) any later
version. This version of this license is in
   http://www.latex-project.org/lppl/lppl-1-3c.txt
and the latest version of this license is in
   http://www.latex-project.org/lppl.txt
and version 1.3c or later is part of all distributions of
LaTeX version 2005/12/01 or later.

This work has the LPPL maintenance status "maintained".

The Current Maintainer of this work is H.-Martin Muench.

This work consists of the main source file tablefootnote.dtx,
the README, and the derived files
   tablefootnote.sty, tablefootnote.pdf,
   tablefootnote.ins, tablefootnote.drv,
   tablefootnote-example.tex, tablefootnote-example.pdf.

In memoriam Tommy Muench + 2014/01/02.

\endpreamble
\let\MetaPrefix\DoubleperCent

\generate{%
  \file{tablefootnote.ins}{\from{tablefootnote.dtx}{install}}%
  \file{tablefootnote.drv}{\from{tablefootnote.dtx}{driver}}%
  \usedir{tex/latex/tablefootnote}%
  \file{tablefootnote.sty}{\from{tablefootnote.dtx}{package}}%
  \usedir{doc/latex/tablefootnote}%
  \file{tablefootnote-example.tex}{\from{tablefootnote.dtx}{example}}%
}

\catcode32=13\relax% active space
\let =\space%
\Msg{************************************************************************}
\Msg{*}
\Msg{* To finish the installation you have to move the following}
\Msg{* file into a directory searched by TeX:}
\Msg{*}
\Msg{*     tablefootnote.sty}
\Msg{*}
\Msg{* To produce the documentation run the file `tablefootnote.drv'}
\Msg{* through (pdf)LaTeX, e.g.}
\Msg{*  pdflatex tablefootnote.drv}
\Msg{*  makeindex -s gind.ist tablefootnote.idx}
\Msg{*  pdflatex tablefootnote.drv}
\Msg{*  makeindex -s gind.ist tablefootnote.idx}
\Msg{*  pdflatex tablefootnote.drv}
\Msg{*}
\Msg{* At least three runs are necessary e.g. to get the}
\Msg{*  references right!}
\Msg{*}
\Msg{* Happy TeXing!}
\Msg{*}
\Msg{************************************************************************}

\endbatchfile
%</install>
%<*ignore>
\fi
%</ignore>
%
% \section{The documentation driver file}
%
% The next bit of code contains the documentation driver file for
% \TeX{}, i.\,e., the file that will produce the documentation you
% are currently reading. It will be extracted from this file by the
% \texttt{docstrip} programme. That is, run \LaTeX{} on \texttt{docstrip}
% and specify the \texttt{driver} option when \texttt{docstrip}
% asks for options.
%
%    \begin{macrocode}
%<*driver>
\NeedsTeXFormat{LaTeX2e}[2011/06/27]
\ProvidesFile{tablefootnote.drv}%
  [2014/01/26 v1.1c Table foot notes (HMM)]
\documentclass{ltxdoc}[2007/11/11]% v2.0u
\usepackage{holtxdoc}[2012/03/21]%  v0.24
%% tablefootnote may work with earlier versions of LaTeX2e and those
%% class and package, but this was not tested.
%% Please consider updating your LaTeX, class, and package
%% to the most recent version (if they are not already the most
%% recent version).
\hypersetup{%
 pdfsubject={Footnotes in Tables (HMM)},%
 pdfkeywords={LaTeX, tablefootnote, footnote, table, H.-Martin Muench},%
 pdfencoding=auto,%
 pdflang={en},%
 breaklinks=true,%
 linktoc=all,%
 pdfstartview=FitH,%
 pdfpagelayout=OneColumn,%
 bookmarksnumbered=true,%
 bookmarksopen=true,%
 bookmarksopenlevel=3,%
 pdfmenubar=true,%
 pdftoolbar=true,%
 pdfwindowui=true,%
 pdfnewwindow=true%
}
\CodelineIndex
\hyphenation{printing docu-ment}
\gdef\unit#1{\mathord{\thinspace\mathrm{#1}}}%
\makeatletter
\@ifundefined{eTeX}{\gdef\eTeX{$\m@th \varepsilon $-\TeX }}{% else \relax
}
\makeatother
\begin{document}
  \DocInput{tablefootnote.dtx}%
\end{document}
%</driver>
%    \end{macrocode}
%
% \fi
%
% \CheckSum{584}
%
% \CharacterTable
%  {Upper-case    \A\B\C\D\E\F\G\H\I\J\K\L\M\N\O\P\Q\R\S\T\U\V\W\X\Y\Z
%   Lower-case    \a\b\c\d\e\f\g\h\i\j\k\l\m\n\o\p\q\r\s\t\u\v\w\x\y\z
%   Digits        \0\1\2\3\4\5\6\7\8\9
%   Exclamation   \!     Double quote  \"     Hash (number) \#
%   Dollar        \$     Percent       \%     Ampersand     \&
%   Acute accent  \'     Left paren    \(     Right paren   \)
%   Asterisk      \*     Plus          \+     Comma         \,
%   Minus         \-     Point         \.     Solidus       \/
%   Colon         \:     Semicolon     \;     Less than     \<
%   Equals        \=     Greater than  \>     Question mark \?
%   Commercial at \@     Left bracket  \[     Backslash     \\
%   Right bracket \]     Circumflex    \^     Underscore    \_
%   Grave accent  \`     Left brace    \{     Vertical bar  \|
%   Right brace   \}     Tilde         \~}
%
% \GetFileInfo{tablefootnote.drv}
%
% \begingroup
%   \def\x{\#,\$,\^,\_,\~,\ ,\&,\{,\},\%}%
%   \makeatletter
%   \@onelevel@sanitize\x
% \expandafter\endgroup
% \expandafter\DoNotIndex\expandafter{\x}
% \expandafter\DoNotIndex\expandafter{\string\ }
% \begingroup
%   \makeatletter
%     \lccode`9=32\relax
%     \lowercase{%^^A
%       \edef\x{\noexpand\DoNotIndex{\@backslashchar9}}%^^A
%     }%^^A
%   \expandafter\endgroup\x
%
% \DoNotIndex{\\,\,}
% \DoNotIndex{\documentclass,\usepackage,\ProvidesPackage,\begin,\end}
% \DoNotIndex{\MessageBreak,\message}
% \DoNotIndex{\NeedsTeXFormat,\DoNotIndex,\verb}
% \DoNotIndex{\def,\edef,\gdef,\global}
% \DoNotIndex{\ifx,\listfiles,\mathord,\mathrm}
% \DoNotIndex{\kvoptions,\SetupKeyvalOptions,\ProcessKeyvalOptions}
% \DoNotIndex{\bigskip,\space,\thinspace,\ldots}
% \DoNotIndex{\indent,\noindent,\newline,\linebreak,\pagebreak}
% \DoNotIndex{\textbf,\textit,\textsf,\texttt,\textsc,\textquotedblleft,\textquotedblright}
% \DoNotIndex{\plainTeX,\TeX,\LaTeX,\pdfLaTeX}
% \DoNotIndex{\chapter,\section,\the,\csname,\endscname}
% \DoNotIndex{\value,\arabic,\hline,\text,\caption,\centering,\clear,\url}
%
% \title{The \xpackage{tablefootnote} package}
% \date{2014/01/26 v1.1c}
% \author{H.-Martin M\"{u}nch\\\xemail{Martin.Muench at Uni-Bonn.de}}
%
% \maketitle
%
% \begin{abstract}
% This \LaTeX{} package provides the command |\tablefootnote| to be used
% in a \texttt{table} or \texttt{sidewaystable} environment,
% where |\footnote| will not work and when using |\footnotemark| and
% |\footnotetext| and adjusting the counters (including \texttt{Hfootnote})
% manually is either too much work or would not even work (\texttt{sidewaystable}).
% \end{abstract}
%
% \bigskip
%
% \noindent Disclaimer for web links: The author is not responsible for any contents
% referred to in this work unless he has full knowledge of illegal contents.
% If any damage occurs by the use of information presented there, only the
% author of the respective pages might be liable, not the one who has referred
% to these pages.
%
% \bigskip
%
% \noindent {\color{green} Save per page about $200\unit{ml}$ water,
% $2\unit{g}$ CO$_{2}$ and $2\unit{g}$ wood:\\
% Therefore please print only if this is really necessary.}
%
% \pagebreak
%
% \tableofcontents
%
% \pagebreak
%
% \section{Introduction\label{sec:Introduction}}
%
% \indent This \LaTeX{} package provides the command |\tablefootnote{...}| to be
% used in a \texttt{table} or \texttt{sidewaystable} environment. In those
% environments |\footnote{...}| would not work and would need to be replaced
% by |\footnotemark| in the \texttt{(sideways)\allowbreak table} and
% |\footnotetext{...}| after the \texttt{(sideways)table} environment ended.
% Additionally the counters for \texttt{footnote} and
% (when the \xpackage{hyperref} package is used) \texttt{Hfootnote}
% would need to be adjusted manually. Then still the hyperlinks
% in a \texttt{(sideways)table} would not work.
% When the \xpackage{footnotebackref} package is used,
% |\footnotemark| and |\footnotetext{...}| would need to be redefined.
% Just using |\tablefootnote{...}| (with optional argument for custom foot note
% marks) in the (sideways)table does all this automatically.
% (Compatibility with the \xpackage{footnotebackref} package
% was achieved without redefining |\footnotemark| or |\footnotetext|.)
% Redefining |\footnote| was \emph{not} done as to not create conflicts with
% other packages.
%
% \section{Usage}
%
% \indent Just load the package placing
% \begin{quote}
%   |\usepackage{tablefootnote}|
% \end{quote}
% \noindent in the preamble of your \LaTeXe{} source file and use\newline
% |\tablefootnote{...}| instead of |\footnote{...}| and\newline
% |\tablefootnote[...]{...}| instead of |\footnote[...]{...}|
% in (sideways)\allowbreak tables.\\
% When the \xpackage{footnotebackref} package is used,
% it must be loaded before the \xpackage{tablefootnote} package
% and the \xpackage{hyperrefpackage} with option |hyperfootnotes=true|
% must be loaded, too. (Backreference{-}links without links do not make sense.)\\
% When the \xpackage{rotating} package (for |sidewaystable|s) is used,
% it must be loaded before the \xpackage{tablefootnote} package and it must be
% a recent version (v2.16a, 2009/03/28, or newer).\\
% If |\tablefootnote{...}| is used in a tabular environment, which is inside
% of a (sideways)table environment, everything is fine, but when the tabular
% environment is not inside of a (sideways)table, the \textbf{table}footnote
% will not work.\\
%
% Pages with tables with footnotes, which are rotated with the \xpackage{lscape}-
% or with the \xpackage{pdflscape}-package (|\begin{landscape}|, table, |\end{landscape}|),
% are possible. But when the \xpackage{footmisc}-package is used, \xpackage{(pdf)lscape}
% should be loaded before \xpackage{footmisc}.\\
%
% When the \xpackage{footmisc}-package is used with option \texttt{para},
% this is regarded with the exception of a \texttt{sidewaystable}. There the footnotes
% are printed one below the other. (Well, this is better than no footnote in a
% \texttt{sidewaystable} at all, isn't it?)\\
%
% When the (sideways)table floats, the footnotes are set and (when \xpackage{hyperref}
% is used) hyperlinked, but they are not automatically adapted when the table floats
% over/under another footnote. Thus either do not use a footnote between original and
% float position of the (sideways)table, or place the (sideways)table in
% \textquotedblleft here\textquotedblright\ position. |\clear(double)page|, |h(!)|,
% |H|~from the \xpackage{float} package (\url{http://www.ctan.org/pkg/float}),
% or |\FloatBarrier| from the \xpackage{picins} package
% (\url{http://www.ctan.org/pkg/picins}) might help, too.
% (Or after finishing the document, move the (sideways)table in the source code
% near the position where it floats to or use the optional footnote marks.)
%
% \pagebreak
%
% \section{Alternatives\label{sec:Alternatives}}
%
% \begin{description}
% \item[-] The \xpackage{longtable} package provides the \texttt{longtable} environment
%   as replacement for the combined \texttt{table} and \texttt{tabular} environments.
%   Footnotes are real footnotes (not just tablenotes), are continuously numbered and
%   hyperlinked (when using the \xpackage{hyperref} package), and the hyperlinks really
%   work. As drawback the appearance of the caption changes slightly (e.\,g.~distance
%   to the table, width of the caption), which can probably be changed back manually.
%   Furthermore, longtables are meaned to break over more than one page. If that is not
%   wished, it must be prevented by |\nopagebreak|-commands and by ending the longtable
%   lines with |\\*| instead of |\\|. longtables do not float. (Therefore using the
%   \xpackage{tablefootnote} package and |\FloatBarrier| from the \xpackage{picins} package
%   before and after the table environment is similar~- but \xpackage{tablefootnote} does not
%   change the table-caption!) \texttt{sidewaystable} does not work with it.\\
%   \url{http://www.ctan.org/pkg/longtable}
%
% \item[-] The \xpackage{supertabular} package provides the \texttt{mpsupertabular}
%   environment as replacement for the combined \texttt{table} and \texttt{tabular}
%   environments. Footnotes are just tablenotes (with working hyperlinks when using the
%   \xpackage{hyperref} package), i.\,e. numbered a, b, c and placed below the table
%   and not at the end of the page. Therefore there is no float problem (because the
%   tablenotes numbering is not included in the continuous numbering of the footnotes).
%   Placing the \texttt{supertabular} inside of a \texttt{sidewaystable} breaks the
%   hyperlinks to the tablenotes.\\
%   \url{http://www.ctan.org/pkg/supertabular}
%
% \item[-] The \xpackage{ctable} package has its very own notation for defining tables.
%   It can create tablenotes and sideways-tables. The tablenotes are not automatically
%   hyperlinked. The \texttt{ctable}s float. Because the tablenotes numbering is not
%   included in the continuous numbering of the footnotes there is no float problem.\\
%   \url{http://www.ctan.org/pkg/ctable}
%
% \item[-] The \xpackage{footnote} package provides |\makesavenoteenv{table}|. After
%   loading the package and using that command in the preamble, in tables |\footnote{...}|
%   can be used. Using |\makesavenoteenv{tabular}| and \\
%   |\makesavenoteenv{sidewaystable}|
%   is possible, but it neither solves the float problem, nor do the created hyperlinks
%   work (i.\,e.~they aim at wrong locations). The \xpackage{mdwtab} from the same bundle
%   is incompatible with other table-related packages (e.\,g. \xpackage{supertabular},
%   \xpackage{array}) and not $100\,\%$ compatible with the \texttt{tabular} environment.\\
%   \url{http://www.ctan.org/pkg/footnote}\\
%   \url{http://www.ctan.org/pkg/mdwtab}
%
% \item[-] The \xpackage{tabularx} package does produce footnotes for sidewaystables,
%   but uses a, b, c instead of 1, 2, 3. The hyperlinks to the footnotes do not work.
%   Because the footnotes numbering is not included in the continuous numbering of
%   the other footnotes there is no float problem.\\
%   \url{http://www.ctan.org/pkg/tabularx}
%
% \item[-] Placing a tabular inside a minipage inside a table produces tablenotes.
%   Therefore there is no float problem (because the footnotes are not continuously
%   numbered). The hyperlinks to the table notes indeed work.
%
% \pagebreak
%
% \item[-] The \xpackage{threeparttable} package creates tablenotes again. Therefore
%   there is no float problem (because the tablenotes are not continuously numbered with
%   the footnotes). There are no hyperlinks to the table notes (at least not automatically).
%   Using |sidewaystable| (with table notes) works.\\
%   \url{http://www.ctan.org/pkg/threeparttable}
%
% \item[-] The \xpackage{threeparttablex} package creates tablenotes again. Therefore there
%   is no float problem (because the tablenotes are not continuously numbered with the
%   footnotes). With option \texttt{referable} the tablenotes are hyperlinked. Use of
%   a |sidewaystable| (with table notes) did not work for me. When using the
%   \texttt{referable} option according to the example in the \xpackage{threeparttablex}
%   manual the \xpackage{longtable} package is used, therefore that package could be used
%   directly without \xpackage{threeparttablex} (see above).\\
%   \url{http://www.ctan.org/pkg/threeparttablex}
%
% \item[-] One can manually use |\footnotemark| in the table and |\footnotetext{...}|
%  after the table environment ended and manually change the |footnote| and
%  (when the \xpackage{hyperref} package is used) |Hfootnote| counters and needs
%  to insert
%  \begin{verbatim}
%     \makeatletter
%     \global\let\Hy@saved@currentHref\@currentHref
%     \hyper@makecurrent{Hfootnote}%
%     \global\let\Hy@footnote@currentHref\@currentHref
%     \global\let\@currentHref\Hy@saved@currentHref
%     \makeatother
%  \end{verbatim}
%  before each |\footnotetext{...}|, but with more than very few footnotes this can become
%  much work (and is prone to produce errors). And this does not even work with
%  sidewaystables. (And it does not work with the \xpackage{footnotebackref} package
%  without redefining |\footnotemark| and |\footnotetext{...}|.)
% \end{description}
%
% \bigskip
%
% \noindent (You programmed or found another alternative,
%  which is available at \CTAN{}?\\
%  OK, send an e-mail to me with the name, location at \CTAN{},
%  and a short notice, and I will probably include it in
%  the list above.)
%
% \newpage
%
% \section{Example}
%
%    \begin{macrocode}
%<*example>
\documentclass[british]{article}[2007/10/19]% v1.4h
%%%%%%%%%%%%%%%%%%%%%%%%%%%%%%%%%%%%%%%%%%%%%%%%%%%%%%%%%%%%%%%%%%%%%
\usepackage{float}[2001/11/08]%    v1.3d
\usepackage{placeins}[2005/04/18]% v2.2  ; for \FloatBarrier
\usepackage{rotating}[2009/03/28]% v2.16a; for sidewaystable-environment
\usepackage[%
 hyperfootnotes=true,%
 extension=pdf,%
 plainpages=false,%
 pdfpagelabels=true,%
 hyperindex=false,%
 pdflang={en},%
 pdftitle={tablefootnote package example},%
 pdfauthor={H.-Martin Muench},%
 pdfsubject={Example for the tablefootnote package},%
 pdfkeywords={LaTeX, tablefootnote, footnote, table, H.-Martin Muench},%
 % pdfview=FitH and FitBH do not work: hyperlinks in sidewaystables
 % do no lead to the footnotes, due to a bug in pdfTeX,
 % computing wrong anchor coordinates (Heiko Oberdiek, 29. October 2011)
 % pdfview=Fit, FitV, FitR, FitB, FitBV work
 % print is OK for all those options
 pdfstartview=FitH,%
 pdfpagelayout=OneColumn%
]{hyperref}[2012/11/06]% v6.83m
 % Due to the urls used in the example, either the hyperref or the url
 % package are needed (or the urls must be removed before compiling).

\usepackage{footnotebackref}[2012/07/01]% v1.0
\usepackage{tablefootnote}[2014/01/26]%   v1.1c

\gdef\unit#1{\mathord{\thinspace\mathrm{#1}}}%
\listfiles
\begin{document}
\pagenumbering{arabic}
\section*{Example for tablefootnote}

This example demonstrates the use of package\newline
\textsf{tablefootnote}, v1.1c as of 2014/01/26 (HMM).\newline
There were no options used. (The package provides no options.)\newline

\textbf{The \texttt{tablefootnote-example.tex} needs to be compiled
at least twice to get the references right!}\newline

If the etoolbox-package is found, it is automatically used.\newline

For more details please see the documentation!\newline

\noindent Save per page about $200\unit{ml}$ water,
$2\unit{g}$ CO$_{2}$ and $2\unit{g}$ wood:\newline
Therefore please print only if this is really necessary.\newline

Here is some text.\footnote{And this is a text footnote.}\newline

Tables \ref{tab.symbol}, \ref{tab.normal}, \ref{tab.another} and
\ref{tab.floatH} show normal tables, table~\ref{tab.sideways} depicts
a sidewaystable. Table~\ref{tab.floatH} uses the float
specifier~\texttt{H} from the float package.\newline

\texttt{Hyperref} option \verb|pdfview=FitH| and \verb|FitBH| do
not work due to a bug in pdf\TeX{}, computing wrong
anchor coordinates (\textsc{Heiko Oberdiek}, 29.~October 2011).
Depending on used pdf-viewer, hyperlinks in sidewaystables lead
e.\,g.~at the end of the document, not at the footnote.
\verb|pdfview=Fit|, \verb|FitV|, \verb|FitR|, \verb|FitB|,
\verb|FitBV| work, print is OK for all those options.

\bigskip

\listoftables

\pagebreak

\renewcommand{\thefootnote}{\fnsymbol{footnote}}
\verb|\renewcommand{\thefootnote}{\fnsymbol{footnote}}|
causes footnote{-}symbol{-}footnotes, which are possible
(see Table~\ref{tab.symbol}).

\begin{table}
\centering
\begin{tabular}{ccc}
Another\tablefootnote{A table footnote.} & %
text\tablefootnote{Another table footnote.} & %
in a table\tablefootnote{A third table footnote.}
\end{tabular}
\caption[A footnotesymbol table]{%
A table with footnote-symbol-footnotes.\label{tab.symbol}}
\end{table}

Some text.\footnote{A text footnote.}

\renewcommand{\thefootnote}{\arabic{footnote}}
\verb|\renewcommand{\thefootnote}{\arabic{footnote}}|
switches back to normal footnote numbers again.

\pagebreak

\begin{table}
\centering
\begin{tabular}{ccc}
Some\tablefootnote{A table footnote.} & %
text\tablefootnote[99]{A table footnote with custom footnote number.} & %
in a table\tablefootnote{A third table footnote.}
\end{tabular}
\caption[A table]{A normal table.\label{tab.normal}}
\end{table}

Some text.\footnote{Another text footnote.}

\pagebreak

%    \end{macrocode}
%
% \pagebreak
%
%    \begin{macrocode}
More text.\footnote{And yet another text footnote.}

\begin{table}[t]
\centering
\begin{tabular}{|c|c|c|}
\hline
Another\tablefootnote{A $3^{rd}$ table footnote.} & %
text\tablefootnote{Another $3^{rd}$ table footnote.} & %
in a table\tablefootnote{A $3^{rd}$ third table footnote.}\\ \hline
\end{tabular}
\caption[Another table]{Another table (third one)\label{tab.another}}
\end{table}

Please note that Table~\ref{tab.another} floated to the top of the page.
While the footnotes are set and (when hyperref is used) hyperlinked,
they are not automatically adapted. Thus either do not use a footnote
at the same page before the table, or place the table in
\textquotedblleft here\textquotedblright\ or
\textquotedblleft bottom\textquotedblright\ position.
\verb|\clear(double)page|, \verb|h(!)|, \verb|H|~from the
\texttt{float} package, or \verb|\FloatBarrier| from the
\texttt{picins} package might help, too. (Or move the table in the
source code near the position where it floats to
or use the optional footnote marks.)

Table~\ref{tab.floatH} (page~\pageref{tab.floatH}) uses float specifier
\texttt{H} from the float package and does not float.

Some text.\footnote{This is just another text footnote.}

\pagebreak

\FloatBarrier

\begin{sidewaystable}
\centering%
\begin{tabular}{ccc}
Text\tablefootnote{Please rotate the view for testing the %
hyperlinks.} & %
in a\tablefootnote[98]{Another sidewaystable footnote %
with optional footnote mark.} & %
sidewaystable%
\tablefootnote{Sidewaystable-environment provided by %
\url{http://www.ctan.org/pkg/rotating} package.}
\end{tabular}
\caption[A sideways table]{A table in the \texttt{sideways} %
environment\label{tab.sideways}}%
\end{sidewaystable}%

\FloatBarrier

\pagebreak

%    \end{macrocode}
%
% \pagebreak
%
%    \begin{macrocode}
A last table, here with float specifier \texttt{H} from the
float\footnote{\url{http://www.ctan.org/pkg/float}} package.

\begin{table}[H]
\centering
\begin{tabular}{ccc}
Another\tablefootnote{A $5^{th}$ table footnote.} & %
text\tablefootnote{Another $5^{th}$ table footnote.} & %
in a table\tablefootnote{A $5^{th}$ third table footnote.}
\end{tabular}
\caption[A last table]{A very last table\label{tab.floatH}}
\end{table}

Some text.\footnote{This is just another text footnote.}

\pagebreak

End of the example for the
tablefootnote\footnote{\url{http://www.ctan.org/pkg/tablefootnote}}
package.

\end{document}
%</example>
%    \end{macrocode}
%
% \StopEventually{}
%
% \newpage
%
% \section{The implementation}
%
% We start off by checking that we are loading into \LaTeXe{} and
% announcing the name and version of this package.
%
%    \begin{macrocode}
%<*package>
%    \end{macrocode}
%
%    \begin{macrocode}
\NeedsTeXFormat{LaTeX2e}[2011/06/27]
\ProvidesPackage{tablefootnote}[2014/01/26 v1.1c
              Table foot notes (HMM)]
%    \end{macrocode}
%
% A short description of the \xpackage{tablefootnote} package:
%
%    \begin{macrocode}
%% Provides the \tablefootnote[...]{...} command
%% for footnotes in (sideways)tables.

%    \end{macrocode}
%
% We need the \xpackage{ltxcmds} and \xpackage{letltxmacro} packages
% by \textsc{Heiko Oberdiek}:
%
%    \begin{macrocode}
\RequirePackage{ltxcmds}[2011/11/09]%     v1.22
\RequirePackage{letltxmacro}[2010/09/02]% v1.4
\RequirePackage{xifthen}[2009/04/17]%     v1.3
%    \end{macrocode}
%
% A last information for the user:
%
%    \begin{macrocode}
%% tablefootnote may work with earlier versions of LaTeX and those
%% packages, but this was not tested. Please consider updating
%% your LaTeX and packages to the most recent version
%% (if they are not already the most recent version).

%    \end{macrocode}
%
% \indent See subsection~\ref{ss:Downloads} about how to get them.\\
%
% When the \xpackage{rotating} package (for |sidewaystable|s) is used,
% it must be loaded before the \xpackage{tablefootnote} package and it must be
% a recent version:
%
%    \begin{macrocode}
\@ifpackageloaded{rotating}{%
  \gdef\tfn@rotating{1}
  \@ifpackagelater{rotating}{2009/03/28}% v2.16a
   {% >= 2009/03/28, OK
   }{% else
     \PackageWarningNoLine{tablefootnote}{%
       It is required version\MessageBreak%
       2009/03/28 v2.16a (or later) of package rotating,\MessageBreak%
       but only version\MessageBreak%
      `\csname ver@rotating.sty\endcsname'\MessageBreak%
      is available}%
   }%
 }{% else
   \gdef\tfn@rotating{0}%
 }

%    \end{macrocode}
%
% There are no options to be processed.\\
%
% We need some definitions:
%
%    \begin{macrocode}
\def\tfn@footnotetablecount{0}
\def\tfn@footnotetableprint{0}
\def\tfn@footnotezero{0}
\def\tfn@fnt{0}
\def\tfn@swt{0}%       _s_ide_w_ays_t_able
\def\tfn@fmpw{\relax}% f_oot_m_isc_p_ara_w_arn-switch
\def\tfn@fnbr{0}%      f_oot_n_ote_b_ack_r_ef(erence package)

%    \end{macrocode}
%
% \begin{macro}{\tfn@hyperfootnotes}
% We need a wrapper for the used |\ifHy@hyperfootnotes|,
% which is not defined when the \xpackage{hyperref} package has not been loaded:
%
%    \begin{macrocode}
\DeclareRobustCommand{\tfn@hyperfootnotes}[2]{%
  \ifHy@hyperfootnotes%
    #1%
  \else%
    #2%
  \fi%
  }

%    \end{macrocode}
% \end{macro}
%
% \begin{macro}{\tablemakefntext}
% |\tablemakefntext| makes the footnotetext for a footnote in a table.
%
%    \begin{macrocode}
\ltx@ifpackageloaded{footnotebackref}{%
  \gdef\tfn@fnbr{1}%
  \ifx\FootnoteBackref@symbol\empty%
    \ifFootnoteBackref@numberlinked%
      \newcommand\tablemakefntext[1]{%
        \noindent\makebox[1.634em][r]{%
         \mbox{\textsuperscript{\normalfont%
           \hyperlink{tfn:m\theHfootnote}{\@thefnmark}%
        }}}%
        \ #1%
      }
    \else
      \newcommand\tablemakefntext[1]{%
        \noindent\makebox[1.8em][r]{%
          \mbox{\textsuperscript{\normalfont%
            \@thefnmark%
        }}}%
        #1%
      }
    \fi
  \else
    \newcommand\tablemakefntext[1]{%
      \noindent\makebox[1.8em][r]{%
        \mbox{\textsuperscript{\normalfont%
          \ifFootnoteBackref@numberlinked%
            \hyperlink{tfn:m\theHfootnote}{\@thefnmark}\ %
          \else%
            \@thefnmark%
          \fi%
      }}}%
      \hyperlink{tfn:m\theHfootnote}{\FootnoteBackref@symbol}%
      #1%
    }
  \fi
}{%
  \newcommand\tablemakefntext[1]{%
    \noindent\makebox[1.8em][r]{%
      \mbox{\textsuperscript{\normalfont%
        \@thefnmark%
    }}}%
    #1%
  }
}

%    \end{macrocode}
%
% \end{macro}
%
% \pagebreak
%
% \begin{macro}{\swtablemakefntext}
%    \begin{macrocode}
\newcommand\swtablemakefntext[1]{\noindent\ #1}

%    \end{macrocode}
% \end{macro}
%
% \begin{macro}{\tfn@footnotemarks}
% |\tfn@footnotemarks| uses the |\footnotemark| command to place a foot note mark
% and decreases the |\tfn@footnotetablecount| (i.\,e.~increases the negative value)
% of the number of footnote(marks) in the table. If the \xpackage{footnotebackref}
% package is used, it is also necessary to create an according |\hypertarget| for
% the back reference.
%
%    \begin{macrocode}
\newcommand{\tfn@footnotemarks}[1][]{%
  \hbox{%
    \ifthenelse{\isempty{#1}}%
      {\footnotemark%
       \ltx@ifpackageloaded{footnotebackref}%
         {\raisebox{\ht\strutbox}{\hypertarget{tfn:m\theHfootnote}{}}}%
         {% footnotebackref not loaded
         }%
      }%
      {\ltx@ifpackageloaded{hyperref}%
        {\tfn@hyperfootnotes{%
            \hyperlink{tfn:#1}{\footnotemark[#1]}%
          }{\footnotemark[#1]}}%
        {\footnotemark[#1]}%
       \ltx@ifpackageloaded{footnotebackref}%
        {\raisebox{\ht\strutbox}{\hypertarget{tfn:m#1}{}}}{}%
      }%
   }%
  \@tempcnta=\tfn@footnotetablecount\relax%
  \advance\@tempcnta by -1\relax%
  \xdef\tfn@footnotetablecount{\the\@tempcnta}%
}

%    \end{macrocode}
% \end{macro}
%
% \begin{macro}{\tfn@footmiscparawarn}
% When the \xpackage{footmisc} package is used with option |para|,
% we need to issue a warning. (Any idea how to resolve this issue?)\\
% It must be given inside of it's own command, because |\ifFN@para| is not
% defined when \xpackage{footmisc} has not been loaded, and then the
% |\fi| would be a |\fi| without |\if| (and \TeX{} does not like single
% |\fi|s).
%
%    \begin{macrocode}
\newcommand{\tfn@footmiscparawarn}{%
  \ifFN@para%
    \PackageWarning{tablefootnote}{%
      Package footmisc with option para used.\MessageBreak%
      Unfortunately package tablefootnote is not yet\MessageBreak%
      able to regard this in sidewaystables.\MessageBreak%
      Here the footnotes are printed one below\MessageBreak%
      the other. Sorry!\MessageBreak%
     }%
  \fi%
  }

%    \end{macrocode}
% \end{macro}
%
% \begin{macro}{\tfn@updateHyper}
% |\tfn@updateHyper| updates some \xpackage{hyperref} internals.
%
%    \begin{macrocode}
\newcommand{\tfn@updateHyper}{%
  % The code of this command is partly from the
  % hyperref package by Heiko Oberdiek, 2011/10/01, v6.82j.
  % Newer versions of that package are available at CTAN.org.
  \global\let\Hy@saved@currentHref\@currentHref%
  \tfn@hyperfootnotes{\hyper@makecurrent{Hfootnote}}{}%
  \global\let\Hy@footnote@currentHref\@currentHref%
  \global\let\@currentHref\Hy@saved@currentHref%
  }

%    \end{macrocode}
% \end{macro}
%
% \begin{macro}{\tfn@footnotetext}
% |\tfn@footnotetext| contains the |\footnotetext[...]{...}| command and handles
% the numbering. |\tfn@footnotetext| will be executed after the table.
% The foot note counter has been increased automatically with each |\footnotemark|,
% therefore it must be reset to its value before the table began.
%
%    \begin{macrocode}
\newcommand{\tfn@footnotetext}[2][]{%
  \ifx\tfn@footnotetableprint\tfn@footnotezero%
    % This is the first footnotetext after the (sideways)table.
    \addtocounter{footnote}{\tfn@footnotetablecount}%
%    \end{macrocode}
%
% When the \xpackage{hyperref} package is used, also its |Hfootnote| counter must
% be reset (when the footnotes are hyperlinked).
%
%    \begin{macrocode}
    \ltx@ifpackageloaded{hyperref}{%
      \tfn@hyperfootnotes{%
        \addtocounter{Hfootnote}{\tfn@footnotetablecount}}{}%
      \tfn@updateHyper%
     }{}%
  \fi%
%    \end{macrocode}
%
% For issuing the text for the next footnote the footnote-counter
% (and |Hfootnote| counter in case of links) are increased by one.
%
%    \begin{macrocode}
  \addtocounter{footnote}{+1}%
  \ltx@ifpackageloaded{hyperref}{%
    \tfn@hyperfootnotes{\addtocounter{Hfootnote}{+1}}{}%
%    \end{macrocode}
%
% The \xpackage{hyperref} package was not expecting this kind of change of the
% |(H)footnote| numbers, therefore some code-lines are missing here which can be
% found at other places in that same package and are called by this command.
%
%    \begin{macrocode}
    \tfn@updateHyper%
%    \end{macrocode}
%
% When \xpackage{hyperref} is not loaded, nothing is to be done here.
%
%    \begin{macrocode}
   }{}%
%    \end{macrocode}
%
% The |\footnotetext[...]{...}| places the foot note text as usual.
%
%    \begin{macrocode}
  \ifx\tfn@swt\tfn@footnotezero%
    \ifthenelse{\isempty{#1}}%
      {\footnotetext{#2}}%
      {\ltx@ifpackageloaded{hyperref}%
         {\tfn@hyperfootnotes{%
             \footnotetext[#1]{%
               \raisebox{\ht\strutbox}{\hypertarget{tfn:#1}{}}#2}%
           }{\footnotetext[#1]{#2}%
           }%
         }{\footnotetext[#1]{#2}}%
      }%
  \else%
%    \end{macrocode}
%
% If there is more than one tablefootnote in the |sidewaystable|, \\
% the |\tfn@footmiscparawarn| shall be given,
%
%    \begin{macrocode}
    \ifx\tfn@footnotetableprint\tfn@footnotezero%
      \gdef\tfn@fmpw{\tfn@footmiscparawarn}%
    \else%
      \ltx@ifpackageloaded{footmisc}{\tfn@fmpw}{\relax}%
%    \end{macrocode}
%
% but we do not want to repeat the message for a third(, forth,\ldots)
% tablefootnote in the |sidewaystable|.
%
%    \begin{macrocode}
      \gdef\tfn@fmpw{\relax}%
    \fi%
%    \end{macrocode}
%
% |\@makefnmark| is \\
% |\hbox {\@textsuperscript {\normalfont \@thefnmark }}|, \\
% but according to |\show\@thefnmark| here |\@thefnmark| is
% |{\protect \itshape}| instead of |\thefootnote| (Why?!).\\
% If the \xpackage{footnotebackref} package is used,
% back references must be created for the table footnotes, too.
%
%    \begin{macrocode}
    \ltx@ifpackageloaded{footnotebackref}{%
      \LetLtxMacro{\tfncurrentmft}{\@makefntext}%
      \LetLtxMacro{\@makefntext}{\swtablemakefntext}%
      \ifthenelse{\isempty{#1}}%
        {\footnotetext{%
           \raisebox{\ht\strutbox}{\hypertarget{tfn:\theHfootnote}{}}%
           \hbox{\@textsuperscript{\normalfont%
              \hyperlink{tfn:m\theHfootnote}{\thefootnote}}}\thinspace%
         \ifx\FootnoteBackref@symbol\empty%
         \else\hyperlink{tfn:m\theHfootnote}{\FootnoteBackref@symbol}%
         \fi%
         #2}%
       }{\edef\tfn@tfn{\arabic{footnote}}%
         \edef\tfn@thn{\arabic{Hfootnote}}%
         \setcounter{footnote}{#1}%
         \setcounter{Hfootnote}{#1}%
         \tfn@updateHyper%
         \footnotetext{%
           \raisebox{\ht\strutbox}{\hypertarget{tfn:#1}{}}%
           \hbox{\@textsuperscript{\normalfont%
                  \hyperlink{tfn:m#1}{#1}}}%
           \thinspace%
           \ifx\FootnoteBackref@symbol\empty%
           \else%
             \raisebox{\ht\strutbox}{%
               \hyperlink{tfn:m#1}{\FootnoteBackref@symbol}}%
           \fi%
           #2%
          }%
         \setcounter{footnote}{\tfn@tfn}%
         \setcounter{Hfootnote}{\tfn@thn}%
         \tfn@updateHyper%
       }%
      \LetLtxMacro{\@makefntext}{\tfncurrentmft}%
     }{% footnotebackref not loaded
       \ifthenelse{\isempty{#1}}%
         {\footnotetext{%
            \hbox{\@textsuperscript{\normalfont\thefootnote}}%
            \ltx@ifpackageloaded{hyperref}%
              {\tfn@hyperfootnotes{%
                 \raisebox{\ht\strutbox}{%
                   \hypertarget{tfn:\theHfootnote}{}}}{}%
              }{}%
            \thinspace%
            #2}%
        }{%
%    \end{macrocode}
%
% \pagebreak
%
%    \begin{macrocode}
          \edef\tfn@tfn{\arabic{footnote}}%
          \ltx@ifpackageloaded{hyperref}{%
            \tfn@hyperfootnotes{\xdef\tfn@thn{\arabic{Hfootnote}}}{}%
           }{}%
          \setcounter{footnote}{#1}
          \ltx@ifpackageloaded{hyperref}{%
            \tfn@hyperfootnotes{\setcounter{Hfootnote}{#1}}{}%
            \tfn@updateHyper%
           }{}%
          \footnotetext{%
            \hbox{\@textsuperscript{\normalfont #1}}%
            \ltx@ifpackageloaded{hyperref}{%
              \tfn@hyperfootnotes{\raisebox{\ht\strutbox}{%
                                         \hypertarget{tfn:#1}{}}}{}%
             }{}%
            \thinspace%
            #2}%
          \setcounter{footnote}{\tfn@tfn}%
          \ltx@ifpackageloaded{hyperref}{%
            \tfn@hyperfootnotes{\setcounter{Hfootnote}{\tfn@thn}}{}%
            \tfn@updateHyper%
           }{}%
        }%
     }%
  \fi%
%    \end{macrocode}
%
% For further footnotes in the table the \\
% |\ifx\tfn@footnotetableprint\tfn@footnotezero|\ldots|\fi| \\
% from the beginning of this |\tfn@footnotetext| macro must not be repeated,
% therefore we deactivate this by defining |\tfn@footnotetableprint| to
% something other than |0| (here |1| is used).
%
%    \begin{macrocode}
  \gdef\tfn@footnotetableprint{1}%
%    \end{macrocode}
%
% |\tfn@footnotetablecount| contains the negative number of footnote texts
% to be printed, therefore it is increase by one (i.\,e.~the negative number
% decreases, i.\,e.~$-3$ becomes~$-2$).
%
%    \begin{macrocode}
  \@tempcnta=\tfn@footnotetablecount\relax%
  \advance\@tempcnta by +1\relax%
  \xdef\tfn@footnotetablecount{\the\@tempcnta}%
%    \end{macrocode}
%
% When it is zero, all footnote texts have been placed, and\\
% |\tfn@footnotetableprint| can be reset to zero and\\
% |\tfn@tablefootnoteprintout| can be cleared.
%
%    \begin{macrocode}
  \ifx\tfn@footnotetablecount\tfn@footnotezero% all footnotes placed
    \gdef\tfn@footnotetableprint{0}%
    \global\let\tfn@tablefootnoteprintout\relax% brute force
 %\else there is at least one \footnotemark which needs \footnotetext
  \fi
}

%    \end{macrocode}
% \end{macro}
%
% \begin{macro}{\tfn@tablefootnoteprintout}
% When something shall be cleared or redefined, it is a good idea
% to define it. It~will be later filled with the foot note text to be printed.
%
%    \begin{macrocode}
\newcommand{\tfn@tablefootnoteprintout}{\relax}

%    \end{macrocode}
% \end{macro}
%
% \pagebreak
%
% \begin{macro}{\tablefootnote}
% |\tablefootnote[...]{...}| is the command to be used by the user instead of
% |\footnote[...]{...}| in the (sideways)table. It places a foot note mark via
% |\tfn@footnotemarks| and appends a |\tfn@footnotetext{#1}| to the\\
% |\tfn@tablefootnoteprintout| macro, and remembers that table foot notes
% have been used.
%
%    \begin{macrocode}
\newcommand{\tablefootnote}[2][]{%
  \tfn@footnotemarks[#1]%
  \ifthenelse{\isempty{#1}}{\relax}{%
    \ltx@GlobalPrependToMacro{\tfn@tablefootnoteprintout}{%
      \addtocounter{footnote}{+1}%
      \ltx@ifpackageloaded{hyperref}{%
        \tfn@hyperfootnotes{\addtocounter{Hfootnote}{+1}}{}%
        \tfn@updateHyper%
       }{}%
     }%
   }%
  \ltx@GlobalAppendToMacro{\tfn@tablefootnoteprintout}{%
    \tfn@footnotetext[#1]{#2}%
    \ifthenelse{\isempty{#1}}{\relax}{%
      \addtocounter{footnote}{-1}%
      \ltx@ifpackageloaded{hyperref}{%
        \tfn@hyperfootnotes{\addtocounter{Hfootnote}{-1}}{}%
        \tfn@updateHyper%
       }{}%
     }%
   }%
  \gdef\tfn@fnt{1}%
  }

%    \end{macrocode}
% \end{macro}
%
% \begin{macro}{\tfn@swtbox}
% For |sidewaystable|s the |\tfn@tablefootnoteprintout| is placed in a \\
% |\parbox|, which can be placed at the end of the |sidewaystable|:
%
%    \begin{macrocode}
\newcommand{\tfn@swtbox}{%
\ifvoid \footins%
\vskip 2\baselineskip%
\else%
\vskip \skip\footins%
\fi%
\parbox[b][][t]{\linewidth}{\raggedright\tfn@tablefootnoteprintout}%
}

%    \end{macrocode}
% \end{macro}
%
% |\tfn@tablefootnoteprintout| shall be executed after the end of the table environment.
% The \xpackage{etoolbox} package provides very easy syntax for this:
%
%    \begin{macrocode}
\IfFileExists{etoolbox.sty}{%
  \RequirePackage{etoolbox}[2011/01/03]%   v2.1
  \AfterEndEnvironment{table}{%
    \ltx@ifpackageloaded{footnotebackref}{%
      \LetLtxMacro{\tfncurrentmft}{\@makefntext}%
      \LetLtxMacro{\@makefntext}{\tablemakefntext}%
       \tfn@tablefootnoteprintout%
      \LetLtxMacro{\@makefntext}{\tfncurrentmft}%
     }{% footnotebackref not loaded
      \tfn@tablefootnoteprintout%
     }%
      \gdef\tfn@fnt{0}%
   }%
%    \end{macrocode}
%
% When the \xpackage{rotating} package has been loaded, we need to detect
% whether the |\tablefootnote| is placed inside of a |sidewaystable|.
%
%    \begin{macrocode}
  \ltx@ifpackageloaded{rotating}{%
    \AtBeginEnvironment{sidewaystable}{%
      \gdef\tfn@swt{1}%
    }
%    \end{macrocode}
%
% At the end of |sidewaystable| environments |\tfn@swtbox| shall be placed
% (if~there was any |\tablefootnote| in that |sidewaystable| at all).
% And |\tfn@swt| as well as |\tfn@fnt| must be reset to |0|.
%
%    \begin{macrocode}
    \AtEndEnvironment{sidewaystable}{%
      \ifx\tfn@fnt\tfn@footnotezero% \relax
      \else%
        \ltx@ifpackageloaded{footnotebackref}{%
          \LetLtxMacro{\tfncurrentmft}{\@makefntext}%
          \LetLtxMacro{\@makefntext}{\tablemakefntext}%
           \tfn@swtbox%
          \LetLtxMacro{\@makefntext}{\tfncurrentmft}%
         }{% footnotebackref not loaded
          \tfn@swtbox%
          }%
      \fi%
      \gdef\tfn@swt{0}%
      \gdef\tfn@fnt{0}%
     }%
   }{}% else \relax
}{%
%    \end{macrocode}
%
% When the \xpackage{etoolbox} package is not available (e.\,g. because of missing
% \eTeX{}) we~do a workaround:
%
%    \begin{macrocode}
  \PackageWarning{tablefootnote}{%
    Package etoolbox not available, therefore\MessageBreak%
    substituting commands \string\AtEndEnvironment\space %
    and\MessageBreak%
    \string\AfterEndEnvironment\space for tablefootnotes%
    \MessageBreak%
   }%
  \@ifpackagelater{letltxmacro}{2010/09/02}%
   {% >= 2010/09/02, OK
    \GlobalLetLtxMacro{\tfnorigtable}{\table}
    \GlobalLetLtxMacro{\tfnendorigtable}{\endtable}
%    \end{macrocode}
%
% |\GlobalLetLtxMacro| was introduced in version 2010/09/02 v1.4 of the
% \xpackage{letltxmacro} package.
%
%    \begin{macrocode}
   }{% else
     \PackageError{tablefootnote}{%
       Outdated version of letltxmacro package used%
      }{Version 2010/09/02 v1.4 or newer needed but not found!%
        \MessageBreak%
        Update the used letltxmacro package.}
        \LetLtxMacro{\tfnorigtable}{\table}
        \LetLtxMacro{\tfnendorigtable}{\endtable}
    }%
%    \end{macrocode}
%
% \pagebreak
%
%    \begin{macrocode}
  \ltx@ifpackageloaded{float}{%
    \@ifpackagelater{float}{2001/11/08}%   v1.3d
     {% >= 2001/11/08, OK
     }{% else
       \PackageWarningNoLine{tablefootnote}{%
         It is required version\MessageBreak
         2001/11/08 v1.3d (or later) of package float,\MessageBreak
         but only version\MessageBreak
        `\csname ver@float.sty\endcsname'\MessageBreak
        is available}%
     }%
   }{\relax}
  \renewenvironment{table}{%
    \ltx@ifpackageloaded{float}{%
%    \end{macrocode}
%
% When \xpackage{etoolbox.sty} was not found, and the \xpackage{float} package
% has been loaded, and the float specifier is \texttt{H}, then |\endtable| was
% let to |\float@endH| by the \xpackage{float} package (line~79 of
% \xpackage{float.sty}, 2001/11/08 v1.3d:\\
% |\expandafter\let\csname end#1\endcsname\float@endH|), i.\,e.~appending\\
% |\tfn@tablefootnoteprintout \gdef\tfn@fnt{0}| to |\endtable| is useless.
% Instead of this, it must be appended to |\float@endH|.\\
% (Why is this not necessary when \xpackage{etoolbox} is used?
% That is a good question, answers are welcome.)
%
%    \begin{macrocode}
      \ltx@LocalAppendToMacro{\float@endH}{%
        \ltx@ifpackageloaded{footnotebackref}{%
          \LetLtxMacro{\tfncurrentmft}{\@makefntext}%
          \LetLtxMacro{\@makefntext}{\tablemakefntext}%
           \tfn@tablefootnoteprintout%
          \LetLtxMacro{\@makefntext}{\tfncurrentmft}%
         }{% footnotebackref not loaded
          \tfn@tablefootnoteprintout%
         }%
          \gdef\tfn@fnt{0}%
       }%
     }{\relax}
    \tfnorigtable%
   }{%
     \tfnendorigtable%
     \ltx@ifpackageloaded{footnotebackref}{%
       \LetLtxMacro{\tfncurrentmft}{\@makefntext}%
       \LetLtxMacro{\@makefntext}{\tablemakefntext}%
        \tfn@tablefootnoteprintout%
       \LetLtxMacro{\@makefntext}{\tfncurrentmft}%
      }{% footnotebackref not loaded
       \tfn@tablefootnoteprintout%
      }%
     \gdef\tfn@fnt{0}%
    }%
%    \end{macrocode}
%
% \pagebreak
%
% The |sidewaystable|-environment must be treated separately:
%
%    \begin{macrocode}
  \ltx@ifpackageloaded{rotating}{%
    \let\tfnorigsidewaystable\sidewaystable
    \let\tfnendorigsidewaystable\endsidewaystable
    \renewenvironment{sidewaystable}{%
      \gdef\tfn@swt{1}%
      \tfnorigsidewaystable%
    }{%
      \ifx\tfn@fnt\tfn@footnotezero% \relax
      \else%
        \ltx@ifpackageloaded{footnotebackref}{%
          \LetLtxMacro{\tfncurrentmft}{\@makefntext}%
          \LetLtxMacro{\@makefntext}{\tablemakefntext}%
           \tfn@swtbox%
          \LetLtxMacro{\@makefntext}{\tfncurrentmft}%
         }{% footnotebackref not loaded
          \tfn@swtbox%
         }%
      \fi%
      \gdef\tfn@swt{0}%
      \gdef\tfn@fnt{0}%
      \tfnendorigsidewaystable%
     }%
   }{}%
}

%    \end{macrocode}
%
% \DescribeMacro{\AtBeginDocument}
% When the \xpackage{rotating} (for |sidewaystable|s) or the
% \xpackage{footnotebackref} (for back references for the footnotes)
% package is used, it must be loaded before the \xpackage{tablefootnote} package.
% Using both packages is also possible, in which case both must be loaded before
% the \xpackage{tablefootnote} package. If any one of those packages is loaded
% |\AtBeginDocument| after this check, the according error message will not be
% given, but packages really should be loaded before |\AtBeginDocument|.
%
%    \begin{macrocode}
\AtBeginDocument{%
  \ltx@ifpackageloaded{footnotebackref}{%
    \ifx\tfn@fnbr\tfn@footnotezero%
      \PackageError{tablefootnote}{%
        Package footnotebackref loaded after tablefootnote}{%
        When using both the footnotebackref and the tablefootnote %
        package,\MessageBreak%
        the footnotebackref package MUST be loaded before the %
        tablefootnote package!\MessageBreak%
       }%
    \fi%
   }{% else \relax
   }
  \ltx@ifpackageloaded{rotating}{%
    \ifx\tfn@rotating\tfn@footnotezero%
      \PackageError{tablefootnote}{%
        Package rotating loaded after tablefootnote}{%
        When using both the rotating and the tablefootnote package,%
        \MessageBreak%
        the rotating package MUST be loaded before the tablefootnote%
        \space package!\MessageBreak%
       }%
    \fi%
   }{% else \relax
   }
%    \end{macrocode}
%
% \pagebreak
%
% When the document is compiled with \LuaLaTeX{}, hyperlinks in rotated content
% will be misplaced, regardless of the use of the \xpackage{tablefootnote} package
% (or of tables or footnotes). The text to be printed will be OK.
% This problem of \LuaLaTeX{} (not \xpackage{tablefootnote}) was found and reported
% by \textsc{Arno Trautmann}, 2011/11/15 -- thanks!
%
%    \begin{macrocode}
  \ltx@ifpackageloaded{hyperref}{%
    \tfn@hyperfootnotes{
      \ifluatex%
        \PackageWarningNoLine{tablefootnote}{%
          LuaTeX and hyperref package used:\MessageBreak%
          Hyperreferences in rotated content\MessageBreak%
          will be misplaced\MessageBreak%
          regardless of tablefootnote package.\MessageBreak%
          The text to be printed will be OK%
         }%
      \fi%
     }{% else \relax
     }%
   }{% else \relax
   }%
%    \end{macrocode}
%
% When the \xpackage{footnotebackref} package is used but the footnotes are not
% hyperlinked, neither the back references nor the tablefootnotes will work.
%
%    \begin{macrocode}
  \ltx@ifpackageloaded{footnotebackref}{%
     \ltx@ifpackageloaded{hyperref}{%
        \tfn@hyperfootnotes{\relax}{% else
          \PackageError{tablefootnote}{%
             footnotebackref but not hyperfootnotes}{%
             The footnotebackref package is used\MessageBreak%
             but the hyperfootnotes option of the hyperref package%
             \MessageBreak%
             was set to false.\MessageBreak%
             Either enable hyperfootnotes or do not use footnotebackref.%
             \MessageBreak%
             Loading of the tablefootnote package will be aborted.%
           }%
          \endinput%
      }%
    }{\PackageError{tablefootnote}{footnotebackref but not hyperref}{%
        The footnotebackref package is used\MessageBreak%
        but not the hyperref package.\MessageBreak%
        Either load hyperref or do not use footnotebackref.%
        \MessageBreak%
        Loading of the tablefootnote package will be aborted.%
        }%
      \endinput%
    }%
  }{}
}

%    \end{macrocode}
%
%    \begin{macrocode}
%</package>
%    \end{macrocode}
%
% \pagebreak
%
% \section{Installation}
%
% \subsection{Downloads\label{ss:Downloads}}
%
% Everything is available on \CTAN{}, \url{http://www.ctan.org/},
% but may need additional packages themselves.\\
%
% \DescribeMacro{tablefootnote.dtx}
% For unpacking the |tablefootnote.dtx| file and constructing the documentation
% it is required:
% \begin{description}
% \item[-] \TeX{} Format \LaTeXe{}: \url{http://www.CTAN.org/}
%
% \item[-] document class \xpackage{ltxdoc}, 2007/11/11, v2.0u,
%           \url{http://www.ctan.org/pkg/ltxdoc}
%
% \item[-] package \xpackage{holtxdoc}, 2012/03/21, v0.24,
%           \url{http://www.ctan.org/pkg/holtxdoc}
%
% \item[-] package \xpackage{hypdoc}, 2011/08/19, v1.11,
%           \url{http://www.ctan.org/pkg/hypdoc}
% \end{description}
%
% \DescribeMacro{tablefootnote.sty}
% The |tablefootnote.sty| for \LaTeXe{} (i.\,e.~each document using
% the \xpackage{tablefootnote} package) requires:
% \begin{description}
% \item[-] \TeX{} Format \LaTeXe{}, \url{http://www.CTAN.org/}
%
% \item[-] package \xpackage{letltxmacro}, 2010/09/02, v1.4,
%           \url{http://www.ctan.org/pkg/letltxmacro}.
%
% \item[-] package \xpackage{ltxcmds}, 2011/11/09, v1.22,
%           \url{http://www.ctan.org/pkg/ltxcmds}
% \end{description}
%
% When the \xpackage{hyperref} package is used, also the \xpackage{ifluatex} package
% is needed, but it is already loaded automatically by the \xpackage{hyperref}
% package.\\
% When the \xpackage{etoolbox} package is available, it is used:
% \begin{description}
% \item[-] package \xpackage{etoolbox}, 2011/01/03, v2.1,
%           \url{http://www.ctan.org/pkg/etoolbox}.
% \end{description}
%
% \DescribeMacro{tablefootnote-example.tex}
% The |tablefootnote-example.tex| requires the same files as all
% documents using the \xpackage{tablefootnote} package and additionally:
% \begin{description}
% \item[-] class \xpackage{article}, 2007/10/19, v1.4h, from \xpackage{classes.dtx}:\\
%           \CTAN{macros/latex/base/classes.dtx}
%
% \item[-] package \xpackage{rotating}, 2009/03/28, v2.16a,
%           \url{http://www.ctan.org/pkg/rotating}
%
% \item[-] package \xpackage{float}, 2001/11/08, v1.3d,
%           \url{http://www.ctan.org/pkg/float}
%
% \item[-] package \xpackage{placeins}, 2005/04/18, v2.2,
%           \url{http://www.ctan.org/pkg/placeins}\\
%           for |\FloatBarrier|
%
% \item[-] package \xpackage{hyperref}, 2012/11/06, v6.83m,
%           \url{http://www.ctan.org/pkg/hyperref}
%
% \item[-] package \xpackage{tablefootnote}, 2014/01/26, v1.1c,\\
%           \url{http://www.ctan.org/pkg/tablefootnote}\\
%   (Well, it is the example file for this package, and because you
%    are reading the documentation for the \xpackage{tablefootnote} package,
%    it can be assumed that you already have some version of it~--
%    is it the current one?)
% \end{description}
%
% \DescribeMacro{Oberdiek}
% \DescribeMacro{holtxdoc}
% \DescribeMacro{ifluatex}
% \DescribeMacro{letltxmacro}
% \DescribeMacro{ltxcmds}
% All packages of \textsc{Heiko Oberdiek}'s bundle `oberdiek'
% (especially \xpackage{holtxdoc}, \xpackage{ifluatex}, \xpackage{letltxmacro},
% and \xpackage{ltxcmds}) are also available in a TDS compliant ZIP archive:\\
% \CTAN{install/macros/latex/contrib/oberdiek.tds.zip}.\\
% It is probably best to download and use this, because the packages in there
% are quite probably both recent and compatible among themselves.\\
%
% \pagebreak
%
% \DescribeMacro{hyperref}
% \noindent \xpackage{hyperref} is not included in that bundle and needs to be
% downloaded separately,\\
% \url{http://www.ctan.org/install/macros/latex/contrib/hyperref.tds.zip}.\\
%
% \DescribeMacro{M\"{u}nch}
% A hyperlinked list of my (other) packages can be found at
% \url{http://www.ctan.org/author/muench-hm}.\\
%
% \subsection{Package, unpacking TDS}
% \paragraph{Package.} This package is available on \CTAN{}
% \begin{description}
% \item[\CTAN{macros/latex/contrib/tablefootnote/tablefootnote.dtx}]\hspace*{0.1cm} \\
%       The source file.
% \item[\CTAN{macros/latex/contrib/tablefootnote/tablefootnote.pdf}]\hspace*{0.1cm} \\
%       The documentation.
% \item[\CTAN{macros/latex/contrib/tablefootnote/tablefootnote-example.pdf}]\hspace*{0.1cm} \\
%       The compiled example file, as it should look like.
% \item[\CTAN{macros/latex/contrib/tablefootnote/README}]\hspace*{0.1cm} \\
%       The README file.
% \end{description}
% There is also a tablefootnote.tds.zip available:
% \begin{description}
% \item[\CTAN{install/macros/latex/contrib/tablefootnote.tds.zip}]\hspace*{0.1cm} \\
%       Everything in \xfile{TDS} compliant, compiled format.
% \end{description}
% which additionally contains\\
% \begin{tabular}{ll}
% tablefootnote.ins & The installation file.\\
% tablefootnote.drv & The driver to generate the documentation.\\
% tablefootnote.sty & The \xext{sty}le file.\\
% tablefootnote-example.tex & The example file.
% \end{tabular}
%
% \bigskip
%
% \noindent For required other packages, please see the preceding subsection.
%
% \paragraph{Unpacking.} The \xfile{.dtx} file is a self-extracting
% \docstrip{} archive. The files are extracted by running the
% \xext{dtx} through \plainTeX{}:
% \begin{quote}
%   \verb|tex tablefootnote.dtx|
% \end{quote}
%
% \noindent About generating the documentation see paragraph~\ref{GenDoc} below.\\
%
% \paragraph{TDS.} Now the different files must be moved into
% the different directories in your installation TDS tree
% (also known as \xfile{texmf} tree):
% \begin{quote}
% \def\t{^^A
% \begin{tabular}{@{}>{\ttfamily}l@{ $\rightarrow$ }>{\ttfamily}l@{}}
%   tablefootnote.sty & tex/latex/tablefootnote/tablefootnote.sty\\
%   tablefootnote.pdf & doc/latex/tablefootnote/tablefootnote.pdf\\
%   tablefootnote-example.tex & doc/latex/tablefootnote/tablefootnote-example.tex\\
%   tablefootnote-example.pdf & doc/latex/tablefootnote/tablefootnote-example.pdf\\
%   tablefootnote.dtx & source/latex/tablefootnote/tablefootnote.dtx\\
% \end{tabular}^^A
% }^^A
% \sbox0{\t}^^A
% \ifdim\wd0>\linewidth
%   \begingroup
%     \advance\linewidth by\leftmargin
%     \advance\linewidth by\rightmargin
%   \edef\x{\endgroup
%     \def\noexpand\lw{\the\linewidth}^^A
%   }\x
%   \def\lwbox{^^A
%     \leavevmode
%     \hbox to \linewidth{^^A
%       \kern-\leftmargin\relax
%       \hss
%       \usebox0
%       \hss
%       \kern-\rightmargin\relax
%     }^^A
%   }^^A
%   \ifdim\wd0>\lw
%     \sbox0{\small\t}^^A
%     \ifdim\wd0>\linewidth
%       \ifdim\wd0>\lw
%         \sbox0{\footnotesize\t}^^A
%         \ifdim\wd0>\linewidth
%           \ifdim\wd0>\lw
%             \sbox0{\scriptsize\t}^^A
%             \ifdim\wd0>\linewidth
%               \ifdim\wd0>\lw
%                 \sbox0{\tiny\t}^^A
%                 \ifdim\wd0>\linewidth
%                   \lwbox
%                 \else
%                   \usebox0
%                 \fi
%               \else
%                 \lwbox
%               \fi
%             \else
%               \usebox0
%             \fi
%           \else
%             \lwbox
%           \fi
%         \else
%           \usebox0
%         \fi
%       \else
%         \lwbox
%       \fi
%     \else
%       \usebox0
%     \fi
%   \else
%     \lwbox
%   \fi
% \else
%   \usebox0
% \fi
% \end{quote}
% If you have a \xfile{docstrip.cfg} that configures and enables \docstrip{}'s
% \xfile{TDS} installing feature, then some files can already be in the right
% place, see the documentation of \docstrip{}.
%
% \subsection{Refresh file name databases}
%
% If your \TeX{}~distribution (\TeX live, \mikTeX{}, \dots{}) relies on file name
% databases, you must refresh these. For example, \teTeX{} users run
% \verb|texhash| or \verb|mktexlsr|.
%
% \subsection{Some details for the interested}
%
% \paragraph{Unpacking with \LaTeX{}.}
% The \xfile{.dtx} chooses its action depending on the format:
% \begin{description}
% \item[\plainTeX:] Run \docstrip{} and extract the files.
% \item[\LaTeX:] Generate the documentation.
% \end{description}
% If you insist on using \LaTeX{} for \docstrip{} (really,
% \docstrip{} does not need \LaTeX{}), then inform the autodetect routine
% about your intention:
% \begin{quote}
%   \verb|latex \let\install=y\input{tablefootnote.dtx}|
% \end{quote}
% Do not forget to quote the argument according to the demands
% of your shell.
%
% \paragraph{Generating the documentation.\label{GenDoc}}
% You can use both the \xfile{.dtx} or the \xfile{.drv} to generate
% the documentation. The process can be configured by a
% configuration file \xfile{ltxdoc.cfg}. For instance, put the following
% line into this file, if you want to have A4 as paper format:
% \begin{quote}
%   \verb|\PassOptionsToClass{a4paper}{article}|
% \end{quote}
%
% \noindent An example follows how to generate the
% documentation with \pdfLaTeX{}:
%
% \begin{quote}
%\begin{verbatim}
%pdflatex tablefootnote.dtx
%makeindex -s gind.ist tablefootnote.idx
%pdflatex tablefootnote.dtx
%makeindex -s gind.ist tablefootnote.idx
%pdflatex tablefootnote.dtx
%\end{verbatim}
% \end{quote}
%
% \subsection{Compiling the example}
%
% The example file, \textsf{tablefootnote-example.tex}, can be compiled via\\
% \indent |latex tablefootnote-example.tex|,\\
% \indent |lualatex tablefootnote-example.tex|,\\
% or (recommended)\\
% \indent |pdflatex thumbs-example.tex|\\
% and will need at least two compiler runs to get everything right.
%
% \bigskip
%
% \section{Acknowledgements}
%
% I would like to thank \textsc{Heiko Oberdiek} for providing
% the \xpackage{hyperref}, \xpackage{ifluatex}, \xpackage{letltxmacro},
% \xpackage{ltxcmds}, as well as a~lot~(!) of other useful packages
% (from which I also got everything I know about creating a file in
% \xext{dtx} format, OK, say it: copying),
% and for information about the \pdfTeX{}-bug,
% \textsc{Arno Trautmann} for reporting the bug of misplaced hyperreferences in
% |sidewaystable|s using \LuaTeX{}, independently of the \xpackage{tablefootnote}
% package or footnotes at all,
% \textsc{Thomas~V.} for reporting the bug of not handling the case of
% \xpackage{hyperref} option |hyperfootnotes=false|,
% \textsc{Sverre Stausland Johnsen} for reporting the bug in case of \emph{not}
% using \xpackage{hyperref},
% and the \Newsgroup{comp.text.tex} and \Newsgroup{de.comp.text.tex} newsgroups
% as well as everybody at \url{http://tex.stackexchange.com}
% for their help in all things \TeX{}.
%
% \pagebreak
%
% \phantomsection
% \begin{History}\label{History}
%   \begin{Version}{2011/10/26 v1.0a, tabfootn}
%     \item Upload to \CTAN{macros/latex/contrib/tablefootnote/}.
%   \end{Version}
%   \begin{Version}{2011/10/29 v1.0b}
%     \item Renamed to \xpackage{tablefootnote}.
%     \item Added support for the |sidewaystable|-environment of the
%             \xpackage{rotating} package.
%     \item Diverse changes in documentation and README.
%   \end{Version}
%   \begin{Version}{2011/11/06 v1.0c}
%     \item Replaced |^{\text{\thefootnote}}| by |{}^{\text{\textup{\thefootnote}}}|.
%     \item Bug fix: When \xpackage{etoolbox.sty} was not found, and the \xpackage{float}
%             package had been loaded, and the float specifier was \texttt{H},
%             then the |\tablefootnote|s were not printed for that table. Fixed.
%   \end{Version}
%   \begin{Version}{2011/11/19 v1.0d}
%     \item Replaced |{}^{\text{\textup{\thefootnote}}}| by
%             |\textsuperscript{\thefootnote}|, therefore the \xpackage{amstext}
%             package is no longer required by the \xpackage{tablefootnote} package.
%     \item \xpackage{lscape} and \xpackage{pdflscape} package now work with
%             \xpackage{tablefootnote} (except for using a \texttt{sidewaystable} on
%             a \texttt{landscape} page).
%     \item Added information about footnotes in \texttt{sidewaystable} being printed
%             one below the other, even if \xpackage{footmisc} package with option
%             \texttt{para} is used.
%     \item Redefined |\tfn@swtbox| from
%  \begin{verbatim}
%  \vspace{0.8cm}%
%  \begin{spacing}{0.1}%
%  \subcaptionbox*{}[\linewidth][l]{\tfn@tablefootnoteprintout}%
%  \end{spacing}%
%  \end{verbatim}
%             to
%  \begin{verbatim}
%  \vspace{2\baselineskip}%
%  \parbox[b][][t]{\linewidth}{\raggedright\tfn@tablefootnoteprintout}%
%  \end{verbatim}
%             , therefore the \xpackage{setspace} and \xpackage{subcaption} packages
%             are no longer required by the \xpackage{tablefootnote} package.
%     \item No longer uses |\numexpr|, therefore also usable without \eTeX{}.
%     \item Replaced |\textsuperscript{\thefootnote}| by
%             |\hbox {\@textsuperscript {\normalfont \thefootnote }}|,
%             which is next to the original definition of |\@makefnmark|
%             (|\hbox {\@textsuperscript {\normalfont \@thefnmark }}|,
%             but according to |\show\@thefnmark| here |\@thefnmark| is
%             |{\protect \itshape}| instead of |\thefootnote|).
%     \item Added an error message, when \LuaTeX{}, \xpackage{rotating} package, and
%            \xpackage{hyperref} package are used together. Hyperreferences in
%            |sidewaystable|s are misplaced, independently of the
%            \xpackage{tablefootnote} package or footnotes at all. (Bug reported
%            by \textsc{Arno Trautmann}, 2011/11/15 -- thanks!)
%     \item Added footnotesymbol-footnote-numbers in the example.
%   \end{Version}
%   \begin{Version}{2011/11/26 v1.0e}
%     \item Check for \xpackage{hyperref} is only done once.
%     \item Message regarding the misplacement of hyperlinks by \LuaLaTeX{}
%             changed form error to warning.
%     \item Replaced |\vspace{2\baselineskip}| by
%  \begin{verbatim}
% \ifvoid \footins%
% \vskip 2\baselineskip%
% \else%
% \vskip \skip\footins%
% \fi%
%  \end{verbatim}
%             between the |sidewaystable|s and their footnotes.
%     \item The warning about un-regarded |para|-option of \xpackage{footmoisc}
%             package is only given at the second |tablefootnote| in a |sidewaystable|,
%             not repeatedly for every |tablefootnote| in a |sidewaystable|.
%   \end{Version}
%   \begin{Version}{2012/01/01 v1.0f}
%     \item Bugfix: Wrong installation path given in the documentation, fixed.
%     \item Update of documentation, README, and \xfile{dtx} internals.
%   \end{Version}
%   \begin{Version}{2012/01/14 v1.0g}
%     \item Bugfix: There was a |relax| instead of a |\relax|, fixed.
%   \end{Version}
%   \begin{Version}{2012/07/29 v1.0h}
%     \item \xpackage{tablefootnote} is now compatible to the new
%             \xpackage{footnotebackref} package, 2012/07/01, v1.0,
%             \url{http://www.ctan.org/pkg/footnotebackref}.
%   \end{Version}
%   \begin{Version}{2014/01/08 v1.1a}
%     \item Tablefootnotes now regard optional footnote marks:
%             |\tablefootnote[| custom mark here! |]{...}|
%     \item Bug fix: the case of \xpackage{hyperref} option |hyperfootnotes=false|
%             was not handeled at all. (Bug reported by \textsc{Thomas~V.}, %
%             2014/01/02 -- thanks!)
%     \item A lot of details.
%   \end{Version}
%   \begin{Version}{2014/01/20 v1.1b}
%     \item Bug fix: Package was broken when \xpackage{hyperref} was \emph{not} used.
%            (Bug reported by \textsc{Sverre Stausland Johnsen} -- thanks!)
%   \end{Version}
%   \begin{Version}{2014/01/26 v1.1c}
%     \item The emergency bug fix of version 2014/01/20 v1.1b was replaced by using
%             the new robust macro |\tfn@hyperfootnotes|, which only \emph{internally}
%             uses |\ifHy@hyperfootnotes|. If the \xpackage{hyperref} package is not used,
%             \LaTeX{} now does not see any confusing
%             |\ifHy@hyperfootnotes|\ldots|\else|\ldots|\fi| at all.
%     \item URLs and README updated.
%   \end{Version}
% \end{History}
%
% \bigskip
%
% When you find a mistake or have a suggestion for an improvement of this package,
% please send an e-mail to the maintainer, thanks! (Please see BUG REPORTS in the README.)
%
% \bigskip
%
% \PrintIndex
%
% \Finale
\endinput|
% \end{quote}
% Do not forget to quote the argument according to the demands
% of your shell.
%
% \paragraph{Generating the documentation.\label{GenDoc}}
% You can use both the \xfile{.dtx} or the \xfile{.drv} to generate
% the documentation. The process can be configured by a
% configuration file \xfile{ltxdoc.cfg}. For instance, put the following
% line into this file, if you want to have A4 as paper format:
% \begin{quote}
%   \verb|\PassOptionsToClass{a4paper}{article}|
% \end{quote}
%
% \noindent An example follows how to generate the
% documentation with \pdfLaTeX{}:
%
% \begin{quote}
%\begin{verbatim}
%pdflatex tablefootnote.dtx
%makeindex -s gind.ist tablefootnote.idx
%pdflatex tablefootnote.dtx
%makeindex -s gind.ist tablefootnote.idx
%pdflatex tablefootnote.dtx
%\end{verbatim}
% \end{quote}
%
% \subsection{Compiling the example}
%
% The example file, \textsf{tablefootnote-example.tex}, can be compiled via\\
% \indent |latex tablefootnote-example.tex|,\\
% \indent |lualatex tablefootnote-example.tex|,\\
% or (recommended)\\
% \indent |pdflatex thumbs-example.tex|\\
% and will need at least two compiler runs to get everything right.
%
% \bigskip
%
% \section{Acknowledgements}
%
% I would like to thank \textsc{Heiko Oberdiek} for providing
% the \xpackage{hyperref}, \xpackage{ifluatex}, \xpackage{letltxmacro},
% \xpackage{ltxcmds}, as well as a~lot~(!) of other useful packages
% (from which I also got everything I know about creating a file in
% \xext{dtx} format, OK, say it: copying),
% and for information about the \pdfTeX{}-bug,
% \textsc{Arno Trautmann} for reporting the bug of misplaced hyperreferences in
% |sidewaystable|s using \LuaTeX{}, independently of the \xpackage{tablefootnote}
% package or footnotes at all,
% \textsc{Thomas~V.} for reporting the bug of not handling the case of
% \xpackage{hyperref} option |hyperfootnotes=false|,
% \textsc{Sverre Stausland Johnsen} for reporting the bug in case of \emph{not}
% using \xpackage{hyperref},
% and the \Newsgroup{comp.text.tex} and \Newsgroup{de.comp.text.tex} newsgroups
% as well as everybody at \url{http://tex.stackexchange.com}
% for their help in all things \TeX{}.
%
% \pagebreak
%
% \phantomsection
% \begin{History}\label{History}
%   \begin{Version}{2011/10/26 v1.0a, tabfootn}
%     \item Upload to \CTAN{macros/latex/contrib/tablefootnote/}.
%   \end{Version}
%   \begin{Version}{2011/10/29 v1.0b}
%     \item Renamed to \xpackage{tablefootnote}.
%     \item Added support for the |sidewaystable|-environment of the
%             \xpackage{rotating} package.
%     \item Diverse changes in documentation and README.
%   \end{Version}
%   \begin{Version}{2011/11/06 v1.0c}
%     \item Replaced |^{\text{\thefootnote}}| by |{}^{\text{\textup{\thefootnote}}}|.
%     \item Bug fix: When \xpackage{etoolbox.sty} was not found, and the \xpackage{float}
%             package had been loaded, and the float specifier was \texttt{H},
%             then the |\tablefootnote|s were not printed for that table. Fixed.
%   \end{Version}
%   \begin{Version}{2011/11/19 v1.0d}
%     \item Replaced |{}^{\text{\textup{\thefootnote}}}| by
%             |\textsuperscript{\thefootnote}|, therefore the \xpackage{amstext}
%             package is no longer required by the \xpackage{tablefootnote} package.
%     \item \xpackage{lscape} and \xpackage{pdflscape} package now work with
%             \xpackage{tablefootnote} (except for using a \texttt{sidewaystable} on
%             a \texttt{landscape} page).
%     \item Added information about footnotes in \texttt{sidewaystable} being printed
%             one below the other, even if \xpackage{footmisc} package with option
%             \texttt{para} is used.
%     \item Redefined |\tfn@swtbox| from
%  \begin{verbatim}
%  \vspace{0.8cm}%
%  \begin{spacing}{0.1}%
%  \subcaptionbox*{}[\linewidth][l]{\tfn@tablefootnoteprintout}%
%  \end{spacing}%
%  \end{verbatim}
%             to
%  \begin{verbatim}
%  \vspace{2\baselineskip}%
%  \parbox[b][][t]{\linewidth}{\raggedright\tfn@tablefootnoteprintout}%
%  \end{verbatim}
%             , therefore the \xpackage{setspace} and \xpackage{subcaption} packages
%             are no longer required by the \xpackage{tablefootnote} package.
%     \item No longer uses |\numexpr|, therefore also usable without \eTeX{}.
%     \item Replaced |\textsuperscript{\thefootnote}| by
%             |\hbox {\@textsuperscript {\normalfont \thefootnote }}|,
%             which is next to the original definition of |\@makefnmark|
%             (|\hbox {\@textsuperscript {\normalfont \@thefnmark }}|,
%             but according to |\show\@thefnmark| here |\@thefnmark| is
%             |{\protect \itshape}| instead of |\thefootnote|).
%     \item Added an error message, when \LuaTeX{}, \xpackage{rotating} package, and
%            \xpackage{hyperref} package are used together. Hyperreferences in
%            |sidewaystable|s are misplaced, independently of the
%            \xpackage{tablefootnote} package or footnotes at all. (Bug reported
%            by \textsc{Arno Trautmann}, 2011/11/15 -- thanks!)
%     \item Added footnotesymbol-footnote-numbers in the example.
%   \end{Version}
%   \begin{Version}{2011/11/26 v1.0e}
%     \item Check for \xpackage{hyperref} is only done once.
%     \item Message regarding the misplacement of hyperlinks by \LuaLaTeX{}
%             changed form error to warning.
%     \item Replaced |\vspace{2\baselineskip}| by
%  \begin{verbatim}
% \ifvoid \footins%
% \vskip 2\baselineskip%
% \else%
% \vskip \skip\footins%
% \fi%
%  \end{verbatim}
%             between the |sidewaystable|s and their footnotes.
%     \item The warning about un-regarded |para|-option of \xpackage{footmoisc}
%             package is only given at the second |tablefootnote| in a |sidewaystable|,
%             not repeatedly for every |tablefootnote| in a |sidewaystable|.
%   \end{Version}
%   \begin{Version}{2012/01/01 v1.0f}
%     \item Bugfix: Wrong installation path given in the documentation, fixed.
%     \item Update of documentation, README, and \xfile{dtx} internals.
%   \end{Version}
%   \begin{Version}{2012/01/14 v1.0g}
%     \item Bugfix: There was a |relax| instead of a |\relax|, fixed.
%   \end{Version}
%   \begin{Version}{2012/07/29 v1.0h}
%     \item \xpackage{tablefootnote} is now compatible to the new
%             \xpackage{footnotebackref} package, 2012/07/01, v1.0,
%             \url{http://www.ctan.org/pkg/footnotebackref}.
%   \end{Version}
%   \begin{Version}{2014/01/08 v1.1a}
%     \item Tablefootnotes now regard optional footnote marks:
%             |\tablefootnote[| custom mark here! |]{...}|
%     \item Bug fix: the case of \xpackage{hyperref} option |hyperfootnotes=false|
%             was not handeled at all. (Bug reported by \textsc{Thomas~V.}, %
%             2014/01/02 -- thanks!)
%     \item A lot of details.
%   \end{Version}
%   \begin{Version}{2014/01/20 v1.1b}
%     \item Bug fix: Package was broken when \xpackage{hyperref} was \emph{not} used.
%            (Bug reported by \textsc{Sverre Stausland Johnsen} -- thanks!)
%   \end{Version}
%   \begin{Version}{2014/01/26 v1.1c}
%     \item The emergency bug fix of version 2014/01/20 v1.1b was replaced by using
%             the new robust macro |\tfn@hyperfootnotes|, which only \emph{internally}
%             uses |\ifHy@hyperfootnotes|. If the \xpackage{hyperref} package is not used,
%             \LaTeX{} now does not see any confusing
%             |\ifHy@hyperfootnotes|\ldots|\else|\ldots|\fi| at all.
%     \item URLs and README updated.
%   \end{Version}
% \end{History}
%
% \bigskip
%
% When you find a mistake or have a suggestion for an improvement of this package,
% please send an e-mail to the maintainer, thanks! (Please see BUG REPORTS in the README.)
%
% \bigskip
%
% \PrintIndex
%
% \Finale
\endinput|
% \end{quote}
% Do not forget to quote the argument according to the demands
% of your shell.
%
% \paragraph{Generating the documentation.\label{GenDoc}}
% You can use both the \xfile{.dtx} or the \xfile{.drv} to generate
% the documentation. The process can be configured by a
% configuration file \xfile{ltxdoc.cfg}. For instance, put the following
% line into this file, if you want to have A4 as paper format:
% \begin{quote}
%   \verb|\PassOptionsToClass{a4paper}{article}|
% \end{quote}
%
% \noindent An example follows how to generate the
% documentation with \pdfLaTeX{}:
%
% \begin{quote}
%\begin{verbatim}
%pdflatex tablefootnote.dtx
%makeindex -s gind.ist tablefootnote.idx
%pdflatex tablefootnote.dtx
%makeindex -s gind.ist tablefootnote.idx
%pdflatex tablefootnote.dtx
%\end{verbatim}
% \end{quote}
%
% \subsection{Compiling the example}
%
% The example file, \textsf{tablefootnote-example.tex}, can be compiled via\\
% \indent |latex tablefootnote-example.tex|,\\
% \indent |lualatex tablefootnote-example.tex|,\\
% or (recommended)\\
% \indent |pdflatex thumbs-example.tex|\\
% and will need at least two compiler runs to get everything right.
%
% \bigskip
%
% \section{Acknowledgements}
%
% I would like to thank \textsc{Heiko Oberdiek} for providing
% the \xpackage{hyperref}, \xpackage{ifluatex}, \xpackage{letltxmacro},
% \xpackage{ltxcmds}, as well as a~lot~(!) of other useful packages
% (from which I also got everything I know about creating a file in
% \xext{dtx} format, OK, say it: copying),
% and for information about the \pdfTeX{}-bug,
% \textsc{Arno Trautmann} for reporting the bug of misplaced hyperreferences in
% |sidewaystable|s using \LuaTeX{}, independently of the \xpackage{tablefootnote}
% package or footnotes at all,
% \textsc{Thomas~V.} for reporting the bug of not handling the case of
% \xpackage{hyperref} option |hyperfootnotes=false|,
% \textsc{Sverre Stausland Johnsen} for reporting the bug in case of \emph{not}
% using \xpackage{hyperref},
% and the \Newsgroup{comp.text.tex} and \Newsgroup{de.comp.text.tex} newsgroups
% as well as everybody at \url{http://tex.stackexchange.com}
% for their help in all things \TeX{}.
%
% \pagebreak
%
% \phantomsection
% \begin{History}\label{History}
%   \begin{Version}{2011/10/26 v1.0a, tabfootn}
%     \item Upload to \CTAN{macros/latex/contrib/tablefootnote/}.
%   \end{Version}
%   \begin{Version}{2011/10/29 v1.0b}
%     \item Renamed to \xpackage{tablefootnote}.
%     \item Added support for the |sidewaystable|-environment of the
%             \xpackage{rotating} package.
%     \item Diverse changes in documentation and README.
%   \end{Version}
%   \begin{Version}{2011/11/06 v1.0c}
%     \item Replaced |^{\text{\thefootnote}}| by |{}^{\text{\textup{\thefootnote}}}|.
%     \item Bug fix: When \xpackage{etoolbox.sty} was not found, and the \xpackage{float}
%             package had been loaded, and the float specifier was \texttt{H},
%             then the |\tablefootnote|s were not printed for that table. Fixed.
%   \end{Version}
%   \begin{Version}{2011/11/19 v1.0d}
%     \item Replaced |{}^{\text{\textup{\thefootnote}}}| by
%             |\textsuperscript{\thefootnote}|, therefore the \xpackage{amstext}
%             package is no longer required by the \xpackage{tablefootnote} package.
%     \item \xpackage{lscape} and \xpackage{pdflscape} package now work with
%             \xpackage{tablefootnote} (except for using a \texttt{sidewaystable} on
%             a \texttt{landscape} page).
%     \item Added information about footnotes in \texttt{sidewaystable} being printed
%             one below the other, even if \xpackage{footmisc} package with option
%             \texttt{para} is used.
%     \item Redefined |\tfn@swtbox| from
%  \begin{verbatim}
%  \vspace{0.8cm}%
%  \begin{spacing}{0.1}%
%  \subcaptionbox*{}[\linewidth][l]{\tfn@tablefootnoteprintout}%
%  \end{spacing}%
%  \end{verbatim}
%             to
%  \begin{verbatim}
%  \vspace{2\baselineskip}%
%  \parbox[b][][t]{\linewidth}{\raggedright\tfn@tablefootnoteprintout}%
%  \end{verbatim}
%             , therefore the \xpackage{setspace} and \xpackage{subcaption} packages
%             are no longer required by the \xpackage{tablefootnote} package.
%     \item No longer uses |\numexpr|, therefore also usable without \eTeX{}.
%     \item Replaced |\textsuperscript{\thefootnote}| by
%             |\hbox {\@textsuperscript {\normalfont \thefootnote }}|,
%             which is next to the original definition of |\@makefnmark|
%             (|\hbox {\@textsuperscript {\normalfont \@thefnmark }}|,
%             but according to |\show\@thefnmark| here |\@thefnmark| is
%             |{\protect \itshape}| instead of |\thefootnote|).
%     \item Added an error message, when \LuaTeX{}, \xpackage{rotating} package, and
%            \xpackage{hyperref} package are used together. Hyperreferences in
%            |sidewaystable|s are misplaced, independently of the
%            \xpackage{tablefootnote} package or footnotes at all. (Bug reported
%            by \textsc{Arno Trautmann}, 2011/11/15 -- thanks!)
%     \item Added footnotesymbol-footnote-numbers in the example.
%   \end{Version}
%   \begin{Version}{2011/11/26 v1.0e}
%     \item Check for \xpackage{hyperref} is only done once.
%     \item Message regarding the misplacement of hyperlinks by \LuaLaTeX{}
%             changed form error to warning.
%     \item Replaced |\vspace{2\baselineskip}| by
%  \begin{verbatim}
% \ifvoid \footins%
% \vskip 2\baselineskip%
% \else%
% \vskip \skip\footins%
% \fi%
%  \end{verbatim}
%             between the |sidewaystable|s and their footnotes.
%     \item The warning about un-regarded |para|-option of \xpackage{footmoisc}
%             package is only given at the second |tablefootnote| in a |sidewaystable|,
%             not repeatedly for every |tablefootnote| in a |sidewaystable|.
%   \end{Version}
%   \begin{Version}{2012/01/01 v1.0f}
%     \item Bugfix: Wrong installation path given in the documentation, fixed.
%     \item Update of documentation, README, and \xfile{dtx} internals.
%   \end{Version}
%   \begin{Version}{2012/01/14 v1.0g}
%     \item Bugfix: There was a |relax| instead of a |\relax|, fixed.
%   \end{Version}
%   \begin{Version}{2012/07/29 v1.0h}
%     \item \xpackage{tablefootnote} is now compatible to the new
%             \xpackage{footnotebackref} package, 2012/07/01, v1.0,
%             \url{http://www.ctan.org/pkg/footnotebackref}.
%   \end{Version}
%   \begin{Version}{2014/01/08 v1.1a}
%     \item Tablefootnotes now regard optional footnote marks:
%             |\tablefootnote[| custom mark here! |]{...}|
%     \item Bug fix: the case of \xpackage{hyperref} option |hyperfootnotes=false|
%             was not handeled at all. (Bug reported by \textsc{Thomas~V.}, %
%             2014/01/02 -- thanks!)
%     \item A lot of details.
%   \end{Version}
%   \begin{Version}{2014/01/20 v1.1b}
%     \item Bug fix: Package was broken when \xpackage{hyperref} was \emph{not} used.
%            (Bug reported by \textsc{Sverre Stausland Johnsen} -- thanks!)
%   \end{Version}
%   \begin{Version}{2014/01/26 v1.1c}
%     \item The emergency bug fix of version 2014/01/20 v1.1b was replaced by using
%             the new robust macro |\tfn@hyperfootnotes|, which only \emph{internally}
%             uses |\ifHy@hyperfootnotes|. If the \xpackage{hyperref} package is not used,
%             \LaTeX{} now does not see any confusing
%             |\ifHy@hyperfootnotes|\ldots|\else|\ldots|\fi| at all.
%     \item URLs and README updated.
%   \end{Version}
% \end{History}
%
% \bigskip
%
% When you find a mistake or have a suggestion for an improvement of this package,
% please send an e-mail to the maintainer, thanks! (Please see BUG REPORTS in the README.)
%
% \bigskip
%
% \PrintIndex
%
% \Finale
\endinput|
% \end{quote}
% Do not forget to quote the argument according to the demands
% of your shell.
%
% \paragraph{Generating the documentation.\label{GenDoc}}
% You can use both the \xfile{.dtx} or the \xfile{.drv} to generate
% the documentation. The process can be configured by a
% configuration file \xfile{ltxdoc.cfg}. For instance, put the following
% line into this file, if you want to have A4 as paper format:
% \begin{quote}
%   \verb|\PassOptionsToClass{a4paper}{article}|
% \end{quote}
%
% \noindent An example follows how to generate the
% documentation with \pdfLaTeX{}:
%
% \begin{quote}
%\begin{verbatim}
%pdflatex tablefootnote.dtx
%makeindex -s gind.ist tablefootnote.idx
%pdflatex tablefootnote.dtx
%makeindex -s gind.ist tablefootnote.idx
%pdflatex tablefootnote.dtx
%\end{verbatim}
% \end{quote}
%
% \subsection{Compiling the example}
%
% The example file, \textsf{tablefootnote-example.tex}, can be compiled via\\
% \indent |latex tablefootnote-example.tex|,\\
% \indent |lualatex tablefootnote-example.tex|,\\
% or (recommended)\\
% \indent |pdflatex thumbs-example.tex|\\
% and will need at least two compiler runs to get everything right.
%
% \bigskip
%
% \section{Acknowledgements}
%
% I would like to thank \textsc{Heiko Oberdiek} for providing
% the \xpackage{hyperref}, \xpackage{ifluatex}, \xpackage{letltxmacro},
% \xpackage{ltxcmds}, as well as a~lot~(!) of other useful packages
% (from which I also got everything I know about creating a file in
% \xext{dtx} format, OK, say it: copying),
% and for information about the \pdfTeX{}-bug,
% \textsc{Arno Trautmann} for reporting the bug of misplaced hyperreferences in
% |sidewaystable|s using \LuaTeX{}, independently of the \xpackage{tablefootnote}
% package or footnotes at all,
% \textsc{Thomas~V.} for reporting the bug of not handling the case of
% \xpackage{hyperref} option |hyperfootnotes=false|,
% \textsc{Sverre Stausland Johnsen} for reporting the bug in case of \emph{not}
% using \xpackage{hyperref},
% and the \Newsgroup{comp.text.tex} and \Newsgroup{de.comp.text.tex} newsgroups
% as well as everybody at \url{http://tex.stackexchange.com}
% for their help in all things \TeX{}.
%
% \pagebreak
%
% \phantomsection
% \begin{History}\label{History}
%   \begin{Version}{2011/10/26 v1.0a, tabfootn}
%     \item Upload to \CTAN{macros/latex/contrib/tablefootnote/}.
%   \end{Version}
%   \begin{Version}{2011/10/29 v1.0b}
%     \item Renamed to \xpackage{tablefootnote}.
%     \item Added support for the |sidewaystable|-environment of the
%             \xpackage{rotating} package.
%     \item Diverse changes in documentation and README.
%   \end{Version}
%   \begin{Version}{2011/11/06 v1.0c}
%     \item Replaced |^{\text{\thefootnote}}| by |{}^{\text{\textup{\thefootnote}}}|.
%     \item Bug fix: When \xpackage{etoolbox.sty} was not found, and the \xpackage{float}
%             package had been loaded, and the float specifier was \texttt{H},
%             then the |\tablefootnote|s were not printed for that table. Fixed.
%   \end{Version}
%   \begin{Version}{2011/11/19 v1.0d}
%     \item Replaced |{}^{\text{\textup{\thefootnote}}}| by
%             |\textsuperscript{\thefootnote}|, therefore the \xpackage{amstext}
%             package is no longer required by the \xpackage{tablefootnote} package.
%     \item \xpackage{lscape} and \xpackage{pdflscape} package now work with
%             \xpackage{tablefootnote} (except for using a \texttt{sidewaystable} on
%             a \texttt{landscape} page).
%     \item Added information about footnotes in \texttt{sidewaystable} being printed
%             one below the other, even if \xpackage{footmisc} package with option
%             \texttt{para} is used.
%     \item Redefined |\tfn@swtbox| from
%  \begin{verbatim}
%  \vspace{0.8cm}%
%  \begin{spacing}{0.1}%
%  \subcaptionbox*{}[\linewidth][l]{\tfn@tablefootnoteprintout}%
%  \end{spacing}%
%  \end{verbatim}
%             to
%  \begin{verbatim}
%  \vspace{2\baselineskip}%
%  \parbox[b][][t]{\linewidth}{\raggedright\tfn@tablefootnoteprintout}%
%  \end{verbatim}
%             , therefore the \xpackage{setspace} and \xpackage{subcaption} packages
%             are no longer required by the \xpackage{tablefootnote} package.
%     \item No longer uses |\numexpr|, therefore also usable without \eTeX{}.
%     \item Replaced |\textsuperscript{\thefootnote}| by
%             |\hbox {\@textsuperscript {\normalfont \thefootnote }}|,
%             which is next to the original definition of |\@makefnmark|
%             (|\hbox {\@textsuperscript {\normalfont \@thefnmark }}|,
%             but according to |\show\@thefnmark| here |\@thefnmark| is
%             |{\protect \itshape}| instead of |\thefootnote|).
%     \item Added an error message, when \LuaTeX{}, \xpackage{rotating} package, and
%            \xpackage{hyperref} package are used together. Hyperreferences in
%            |sidewaystable|s are misplaced, independently of the
%            \xpackage{tablefootnote} package or footnotes at all. (Bug reported
%            by \textsc{Arno Trautmann}, 2011/11/15 -- thanks!)
%     \item Added footnotesymbol-footnote-numbers in the example.
%   \end{Version}
%   \begin{Version}{2011/11/26 v1.0e}
%     \item Check for \xpackage{hyperref} is only done once.
%     \item Message regarding the misplacement of hyperlinks by \LuaLaTeX{}
%             changed form error to warning.
%     \item Replaced |\vspace{2\baselineskip}| by
%  \begin{verbatim}
% \ifvoid \footins%
% \vskip 2\baselineskip%
% \else%
% \vskip \skip\footins%
% \fi%
%  \end{verbatim}
%             between the |sidewaystable|s and their footnotes.
%     \item The warning about un-regarded |para|-option of \xpackage{footmoisc}
%             package is only given at the second |tablefootnote| in a |sidewaystable|,
%             not repeatedly for every |tablefootnote| in a |sidewaystable|.
%   \end{Version}
%   \begin{Version}{2012/01/01 v1.0f}
%     \item Bugfix: Wrong installation path given in the documentation, fixed.
%     \item Update of documentation, README, and \xfile{dtx} internals.
%   \end{Version}
%   \begin{Version}{2012/01/14 v1.0g}
%     \item Bugfix: There was a |relax| instead of a |\relax|, fixed.
%   \end{Version}
%   \begin{Version}{2012/07/29 v1.0h}
%     \item \xpackage{tablefootnote} is now compatible to the new
%             \xpackage{footnotebackref} package, 2012/07/01, v1.0,
%             \url{http://www.ctan.org/pkg/footnotebackref}.
%   \end{Version}
%   \begin{Version}{2014/01/08 v1.1a}
%     \item Tablefootnotes now regard optional footnote marks:
%             |\tablefootnote[| custom mark here! |]{...}|
%     \item Bug fix: the case of \xpackage{hyperref} option |hyperfootnotes=false|
%             was not handeled at all. (Bug reported by \textsc{Thomas~V.}, %
%             2014/01/02 -- thanks!)
%     \item A lot of details.
%   \end{Version}
%   \begin{Version}{2014/01/20 v1.1b}
%     \item Bug fix: Package was broken when \xpackage{hyperref} was \emph{not} used.
%            (Bug reported by \textsc{Sverre Stausland Johnsen} -- thanks!)
%   \end{Version}
%   \begin{Version}{2014/01/26 v1.1c}
%     \item The emergency bug fix of version 2014/01/20 v1.1b was replaced by using
%             the new robust macro |\tfn@hyperfootnotes|, which only \emph{internally}
%             uses |\ifHy@hyperfootnotes|. If the \xpackage{hyperref} package is not used,
%             \LaTeX{} now does not see any confusing
%             |\ifHy@hyperfootnotes|\ldots|\else|\ldots|\fi| at all.
%     \item URLs and README updated.
%   \end{Version}
% \end{History}
%
% \bigskip
%
% When you find a mistake or have a suggestion for an improvement of this package,
% please send an e-mail to the maintainer, thanks! (Please see BUG REPORTS in the README.)
%
% \bigskip
%
% \PrintIndex
%
% \Finale
\endinput