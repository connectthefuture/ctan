\documentclass[11pt,a4paper,openany]{book}
\usepackage[T1]{fontenc}
\usepackage[utf8]{inputenc}
\usepackage{lmodern}
\usepackage[english]{babel}
\usepackage{amsmath, amssymb}
\usepackage{tabularx}
\usepackage{fancyhdr}
\usepackage{graphicx}
\usepackage{listings}
\usepackage{booktabs}
\usepackage{colortbl}
\usepackage{xcolor}
\usepackage{textcomp}
\lstset{language=[LaTeX]TeX,%
        basicstyle=\footnotesize\ttfamily}
\usepackage{engtlc}
\usepackage[pdfpagelabels,plainpages=false,colorlinks=true,hyperindex=true]{hyperref}

% Commands for the documentation
\newcommand*{\meta}[1]{{\normalfont\textlangle\textit{#1}\textrangle}}
\newcommand*{\marg}[1]{\texttt{\{\meta{#1}\}}}
\newcommand*{\cs}[1]{\texttt{\char92#1}}

\setlength{\headheight}{14pt}
\begin{document}
\frontmatter
\begin{titlepage}
\title{\begin{huge}\textbf{The \textsf{Engtlc} package}\end{huge}\\[.5cm]
       \begin{large}Version 3.2\end{large}}
\author{\begin{LARGE}Claudio Fiandrino \end{LARGE}}
\maketitle
\end{titlepage}
\hypersetup{urlcolor=blue}
\hypersetup{linkcolor=blue}
\pagestyle{fancy}
\renewcommand{\chaptermark}[1]{%
 \markboth{\MakeUppercase{%
 \chaptername}\ \thechapter.%
 \ #1}{}}
\renewcommand{\sectionmark}[1]{\markright{\thesection.\ #1}}
\fancyhead[RO,LE]{\thepage}
\fancyhead[RE]{\leftmark}
\fancyhead[LO]{\rightmark}
\fancyfoot{}
\tableofcontents

\mainmatter
\chapter{Introduction}
This package has been realized in four different periods: in the first one I wrote the macros for  measurement units and general symbols; in the second one I extended the units part and I added some other symbols such as the impedance and admittance ones.
In the third version, in which Alessio Sanna gave me precious suggestions, I inserted byte and chip units and the symbols for electrical and magnetic fields. I finished it on 18/12/2009 and on 13/01/2010 it had been published on CTAN. 

In that third version there were some mistakes due of my inexperience in \LaTeX\ programming.  Thanks to Enrico Gregorio, who send me a list of them, I decided to correct them in a new version and added some new symbol commands.

This fourth version, thanks to Claudio Beccari for the precious help, complies to ISO regulations and introduces \emph{alias} commands in english, as explained in section~\ref{sec:simboli}. Moreover some new commands have been added: probability symbols, commands to define signals in temporal, frequency, discrete and z-transform domain are the major novelties.

The purpose to create \textsf{engtlc} is very simple: it can be used by all those who work in electrical engineering or telecommunication area; indeed, \textsf{engtlc} is the acronym of \emph{Engineering Telecommunications}. 

Where does it help?
It helps in writing a document with \LaTeX, of course; in my personal experience I observed that \LaTeX\ code isn't very user friendly when you have to repeat over and over again some words where the code is a little bit different every time.

If you have to write the probability of $x$ you can write the \LaTeX\ code like:
\begin{tabbing}
\hspace{7cm}\=\kill
\begin{lstlisting}
$\mathcal{P}(x)$
\end{lstlisting}  \>   $\mathcal{P}(x)$
\end{tabbing} 
If, after some time, you have to write the probability of the event A:
\begin{tabbing}
\hspace{7cm}\=\kill
\begin{lstlisting}
$\mathcal{P}(\text{A})$ 
\end{lstlisting} \>   $\mathcal{P}(\text{A})$
\end{tabbing} 
or you can copy the previous code and change the argument.

With \textsf{engtlc} it is easier; you simply write:
\begin{flushleft}
\texttt{\$\cs{prob}\marg{argument}\$}
\end{flushleft}
where \meta{argument} is the argument you have to put like $x$ or A.

The following chapters are structured in this way:
\begin{itemize}
\item[$\star$] the procedures to install the package and its dependencies are explained in chapter~\ref{chap:installazione};
\item[$\star$] the specific new commands introduced by \textsf{engtlc} are shown in chapter~\ref{chap:comandi}.
\end{itemize}


\chapter{How to install \textsf{engtlc}}
\label{chap:installazione}
If it is not already intalled in your \TeX\ distribution, you can find the package in my website \url{http://claudiofiandrino.altervista.org} within the ``latex projects'' section; alternatively you can download it from the official \LaTeX\ packages website \url{http://tug.ctan.org/tex-archive/macros/latex/contrib/engtlc}.

After downloading the \emph{.zip} file, decompress it and move all the files into the \texttt{~/texmf/tex/latex/engtlc/} in your home directory (if you don't have a personal \texttt{texmf} tree in your home directory, create one). After moving the unzipped files in the aforementioned folder, rembeber to refresh the filename database if your \TeX\ distribution and operating system require it.

The home folder, shown with \texttt{\textasciitilde}, is a normal way of indicating it on Linux platforms. On Macs the \texttt{texmf} personal tree should be rooted in \texttt{\textasciitilde/Library}; on Windows platforms, from Vista and newer releases, the concept of ``home'' is now well known; for previous version you should root your personal \texttt{texmf} tree somewhere such as \texttt{C:\char92Documents and Settings\char92\meta{your username}}.

Remember: This package is dependent on some external packages such as \textsf{textcomp}, \textsf{amsmath}, \textsf{amssymb}. The package does not expressly load them in order to avoid conflicts among different sets of options the user might want to specify. Therefore the user is responsable for loading them with the options s/he likes best. Since these packages are generally loaded by the user for other purposes, this should not be an inconvenience. Anyway, if you do not load 
\textsf{amsmath} and \textsf{amssymb}, you will receive an error: this because some commands need this packages. On the contrary, if you do not load \textsf{textcomp}, you will see a warning.


\chapter{Commands introduced by \textsf{engtlc}}\label{chap:comandi}
Let's start to examine \textsf{engtlc} commands.

In the first section several commands for writing units of measurement are shown, while in the second section several new symbol commands are listed.

\section{Units of measurement}
I believe that the possibility of introducing units of measurement in a short and consistent way might be very useful. Of course there are other packages that allow to do the same, but possibly with longer constructs. For example with siunuitx  if you have to use ``kbit/s'' you have to type in:
\begin{tabbing}
\hspace{7cm}\= \kill
Code \> Visualization \\ 
\begin{lstlisting}
\si{\kibi\bit\per\second}
\end{lstlisting}  \> \kbits
\end{tabbing} 
So I decided to create shorter codes that can replace the longer ones shown above.

\textsf{Engtlc} units of measurement must used only within math environments \texttt{\$ \$}, \texttt{[ ]} or within any kind \emph{equation like} ones. 


Be careful: if in your text you use a unit command and forget that the space after a control sequence is absorbed by the control sequence itself, you  observe this mistake:
\begin{tabbing}
\hspace{4cm}\= \kill
Code \> Visualization\\
\begin{lstlisting}
8 \cm and 
\end{lstlisting} \> 8 \cm and
\end{tabbing} 

\subsubsection{Time units}
\begin{center}
\begin{tabular}{lll}
\toprule
Unit & Equivalent code & Visualization\\
\midrule
hours & \cs{ho} & \ho \\
seconds & \cs{s} & \s \\
milliseconds & \cs{ms} & $\ms$ \\
microseconds & \cs{us} & $\us$ \\
nanoseconds & \cs{ns} & $\ns$\\
picoseconds & \cs{ps} & $\ps$\\
\bottomrule
\end{tabular}
\end{center}


%%%%%%%%%%%%%%%%%%%%%%%%%%%%%%%%%%%%%%%%%%%%%
\subsubsection{Length units}
\begin{center}
\begin{tabular}{lll}
\toprule
Unit & Equivalent code & Visualization\\
\midrule
micrometres & \cs{um} & $\um$ \\
millimetres & \cs{mm} & $\mm$ \\
centimetres & \cs{cm} & $\cm$ \\
decimetres & \cs{dm} & $\dm$\\
metres & \cs{m} & $\m$\\
kilometres & \cs{km} & $\km$\\
\bottomrule
\end{tabular}
\end{center}



%%%%%%%%%%%%%%%%%%%%%%%%%%%%%%%%%%%%%%%%
\subsubsection{Current units}
\begin{center}
\begin{tabular}{lll}
\toprule
Unit & Equivalent code & Visualization\\
\midrule
microampere & \cs{uA} & $\uA$ \\
milliampere & \cs{mA} & $\mA$ \\
ampere & \cs{A} & $\A$ \\
\bottomrule
\end{tabular}
\end{center}
%%%%%%%%%%%%%%%%%%%%%%%%%%%%%%%%%%%%
\subsubsection{Voltage units}
\begin{center}
\begin{tabular}{lll}
\toprule
Unit & Equivalent code & Visualization\\
\midrule
microvolt & \cs{uV} & $\uV$ \\
millivolt & \cs{mV} & $\mV$ \\
volt & \cs{V} & $\V$\\
megavolt & \cs{MV} & $\MV$\\
\bottomrule
\end{tabular}
\end{center}
%%%%%%%%%%%%%%%%%%%%%%%%%%%%%%%%%%%%%%%%
\subsubsection{Resistance units}
\begin{center}
\begin{tabular}{lll}
\toprule
Unit & Equivalent code & Visualization\\
\midrule
milliohm & \cs{mohm} & $\mohm$ \\
ohm & \cs{ohm} & $\ohm$\\
kilohm & \cs{kohm} & $\kohm$\\
megaohm & \cs{Mohm} & $\Mohm$\\
\bottomrule
\end{tabular}
\end{center}
%%%%%%%%%%%%%%%%%%%%%%%%%%%%%%%%
\subsubsection{Conductance units}
\begin{center}
\begin{tabular}{lll}
\toprule
Unit & Equivalent code & Visualization\\
\midrule
picosiemens & \cs{pSi} & $\pSi$\\
nanosiemens & \cs{nSi} & $\nSi$ \\
microsiemens &\cs{uSi} & $\uSi$ \\
millisiemens & \cs{mSi} & $\mSi$ \\
siemens & \cs{Si} & $\Si$\\
kilosiemens & \cs{kSi} & $\kSi$\\
megasiemens & \cs{MSi} & $\MSi$\\
\bottomrule
\end{tabular}
\end{center}
%%%%%%%%%%%%%%%%%%%%%%%%%%%%%%%%
\subsubsection{Capacity units} 
\begin{center}
\begin{tabular}{lll}
\toprule
Unit & Equivalent code & Visualization\\
\midrule
femtofarad & \cs{fFa} & $\fFa$ \\
picofarad & \cs{pFa} & $\pFa$\\
nanofarad & \cs{nFa} & $\nFa$ \\
microfarad & \cs{uFa} & $\uFa$\\
millifarad & \cs{mFa} & $\mFa$ \\
farad & \cs{Fa} & $\Fa$\\
\bottomrule
\end{tabular}
\end{center}
%%%%%%%%%%%%%%%%%%%%%%%%%%%%%%%%
\subsubsection{Inductance units}
\begin{center}
\begin{tabular}{lll}
\toprule
Unit & Equivalent code & Visualization\\
\midrule
femtohenry & \cs{fHe} & $\fHe$ \\
picohenry & \cs{pHe} & $\pHe$\\
nanohenry & \cs{nHe} & $\nHe$ \\
microhenry & \cs{uHe} & $\uHe$\\
millihenry & \cs{mHe} & $\mHe$ \\
henry & \cs{He} & $\He$\\
\bottomrule
\end{tabular}
\end{center}
%%%%%%%%%%%%%%%%%%%%%%%%%%%%%%%%
\subsubsection{Level units}
\begin{center}
\begin{tabular}{lll}
\toprule
Unit & Equivalent code & Visualization\\
\midrule
dB & \cs{dB} & $\dB$ \\
dBm & \cs{dBm} & $\dBm$\\
\bottomrule
\end{tabular}
\end{center}
%%%%%%%%%%%%%%%%%%%%%%%%%%%%%%%%
\subsubsection{Power units}
\begin{center}
\begin{tabular}{lll}
\toprule
Unit & Equivalent code & Visualization\\
\midrule
microwatt & \cs{uW} & $\uW$\\
milliwatt & \cs{mW} & $\mW$ \\
watt & \cs{W} & $\W$\\
kilowatt & \cs{kW} & $\kW$\\
megawatt & \cs{MW} & $\MW$\\
\bottomrule
\end{tabular}
\end{center}
%%%%%%%%%%%%%%%%%%%%%%%%%%%%%%%%
\subsubsection{Frequency units}
\begin{center}
\begin{tabular}{lll}
\toprule
Unit & Equivalent code & Visualization\\
\midrule
hertz & \cs{Hz} & $\Hz$\\
kilohertz & \cs{kHz} & $\kHz$ \\
megahertz & \cs{MHz} & $\MHz$\\
gigaahertz& \cs{GHz} & $\GHz$\\
terahertz & \cs{THz} & $\THz$\\
\bottomrule
\end{tabular}
\end{center}
%%%%%%%%%%%%%%%%%%%%%%%%%%%%%%%%
\subsubsection{Bit, byte, and chip units according to the ISO regulations}
\begin{center}
\begin{tabular}{lll}
\toprule
Unit & Equivalent code & Visualization\\
\midrule
bit & \cs{bit} & $\bit$\\
kibibit & \cs{kbit} & $\kbit$ \\
mebibit & \cs{Mbit} & $\Mbit$\\
byte & \cs{Byte} & $\Byte$\\
kibibyte & \cs{kByte} & $\kByte$ \\
mebibyte & \cs{MByte} & $\MByte$\\
gibibyte & \cs{GByte}	&	$\GByte$	\\
tebibyte & \cs{TByte}  & $\TByte$ \\
bit per second & \cs{bits} & $\bits$\\
kibibit per second & \cs{kbits} & $\kbits$ \\
mebibit per second & \cs{Mbits} & $\Mbits$\\
byte per second & \cs{Bytes} & $\Bytes$\\
kibibyte per second & \cs{kBytes} & $\kBytes$ \\
mebibyte per second & \cs{MBytes} & $\MBytes$\\
gibibyte per second & \cs{GBytes}	& $\GBytes$\\
tebibyte per second & \cs{TBytes}	& $\TBytes$\\
chip per second & \cs{chips} & $\chips$\\
kibichip per second & \cs{kchips} & $\kchips$ \\
mebichip per second & \cs{Mchips} & $\Mchips$\\
chip su bit per second & \cs{chipsubit} & $\chipsubit$\\
\bottomrule
\end{tabular}
\end{center}
\section{Symbols}\label{sec:simboli}
Several  symbol macros are listed in this section. 
All these macros are supposed to be used only in math mode. Each macro has an English and an Italian contracted name and the examples will show both of them: in the first line there will be the English version and in the second line the Italian one; if only one line is shown, there is no difference between the English and the Italian names.

\subsection{General symbols}

\subsubsection{End of exercise}
The command of end exercise puts a black square flush to the right side of the text block.
\begin{center}
\begin{tabular}{cc}
\toprule
Code & Visualization\\
\midrule
\cs{exerend} & $\blacksquare$\\
\cs{finees} & $\blacksquare$\\
\bottomrule
\end{tabular}
\end{center}

\subsubsection{Spaced ``implies'' command}
This command  is very similar to \cs{implies}, but it puts before and after the symbol a space that can be chosen by the user by means of an optional argument; by default this space is 0.5\,cm; you can change the unit of measure for the spacing by assigning a value to the length register \cs{Implspace}, for example:
\begin{verbatim}
\setlength{\Implspace}{3mm}
\end{verbatim}
The syntax is therefore:
\begin{flushleft}
\cs{Spimplies}\texttt{[\meta{optional space in units of \normalfont\cs{Implspace}}]}\\
\cs{frecciadex}\texttt{[\meta{optional space in units of \normalfont\cs{Implspace}}]}
\end{flushleft}
In the following table, keeping in mind that \cs{Spimpiles} and \cs{frecciadex} are synonymous, in the first line the optional argument has been used, while in the second line the default spacing has been used, so as to evaluate the difference; in both cases the length register \cs{Implspace} was not reset and contains the value of 0.5\,cm.
\begin{center}
\begin{tabular}{cc}
\toprule
Code & Visualization\\
\midrule
\texttt{A\cs{Spimplies[0.3]} B} & $A\Spimplies[0.3]B$	\\
\texttt{A\cs{frecciadex} B} 		& $A\Spimplies B$		\\
\bottomrule
\end{tabular}
\end{center}
Notice that, by inserting in the optional argument the default length, you will obtain the same result:
\begin{center}
\begin{tabular}{cc}
\toprule
Codice & Visualizzazione\\
\midrule
\texttt{A\cs{Spimplies} B} & $A\Spimplies B$	\\
\texttt{A\cs{frecciadex[0.5]} B} 		& $A\frecciadex[0.5] B$		\\
\bottomrule
\end{tabular}
\end{center}

\subsubsection{Vertical ``implies'' command}
\begin{center}
\begin{tabular}{cc}
\toprule
Code & Visualization\\
\midrule
\cs{Downimplies} & $\Downarrow$\\
\cs{frecciadown} & $\Downarrow$\\
\bottomrule
\end{tabular}
\end{center}

\subsubsection{White noise command}
\begin{center}
\begin{tabular}{cc}
\toprule
Code & Visualization\\
\midrule
\cs{noisevar} & \noisevar\\[1ex]
\cs{varianzarumore} & \varianzarumore\\
\bottomrule
\end{tabular}
\end{center}

\subsubsection{Fourier Transform}
Command for the Fourier transform of $x$.
\begin{center}
\begin{tabular}{cc}
\toprule
Code & Visualization\\
\midrule
\cs{fourier\{x\}} & \fourier{x}\\
\bottomrule
\end{tabular}
\end{center}

\subsubsection{Inverse Fourier Transform}	
This command is very similar to the previous one.
\begin{center}
\begin{tabular}{cc}
\toprule
Code & Visualization\\
\midrule
\cs{invfourier\{x\}} & \invfourier{x}\\
\bottomrule
\end{tabular}
\end{center}

\subsubsection{Real part}
\begin{center}
\begin{tabular}{cc}
\toprule
Code & Visualization\\
\midrule
\cs{bfRe\{x\}} & \partereale{x}\\
\cs{partereale\{x\}} & \partereale{x}\\
\bottomrule
\end{tabular}
\end{center}

\subsubsection{Imaginary part}
\begin{center}
\begin{tabular}{cc}
\toprule
Code & Visualization\\
\midrule
\cs{bfIm\{x\}} & \parteimm{x}\\
\cs{parteimm\{x\}} & \parteimm{x}\\
\bottomrule
\end{tabular}
\end{center}

\subsubsection{Information element}
\begin{center}
\begin{tabular}{cc}
\toprule
Code & Visualization\\
\midrule
\cs{Info\{x\}} & \Info{x}\\
\bottomrule
\end{tabular}
\end{center}

\subsubsection{Signals in different domains}

The following four macros define signal functions, by using as a convention lower case letters for signals  in the time domain and for discrete sequences while signals in frequency and $z$-transform domains have automatically the upper case letter.
\begin{center}
\begin{tabular}{cc}
\toprule
Code & Visualization\\
\midrule
\cs{signt\{f\}} & \signt{f}\\
\cs{signf\{g\}} & \signf{g}\\
\cs{signn\{h\}} & \signn{h}\\
\cs{signz\{k\}} & \signz{k}\\
\bottomrule
\end{tabular}
\end{center}

\subsubsection{Analytic signal}
\begin{center}
\begin{tabular}{cc}
\toprule
Code & Visualization\\
\midrule
\cs{analytic\{x\}} & \analytic{x}\\
\cs{analitic\{x\}} & \analitic{x}\\
\bottomrule
\end{tabular}
\end{center}


\subsubsection{Unit vector}
\begin{center}
\begin{tabular}{cc}
\toprule
Code & Visualization\\
\midrule
\cs{unitvec\{x\}} & \versore{x}\\
\cs{versore\{x\}} & \versore{x}\\
\bottomrule
\end{tabular}
\end{center}

\subsubsection{Vector}
\begin{center}
\begin{tabular}{cc}
\toprule
Code & Visualization\\
\midrule
\cs{vector\{x\}}  & \vettore{x}\\
\cs{vettore\{x\}}  & \vettore{x}\\
\bottomrule
\end{tabular}
\end{center}

\subsubsection{Cosine wave of a given frequency}
\begin{center}
\begin{tabular}{cc}
\toprule
Code & Visualization\\
\midrule
\cs{cosine\{f\_0\}} & \coseno{f_0}\\
\cs{coseno\{f\_0\}} & \coseno{f_0}\\
\bottomrule
\end{tabular}
\end{center}

\subsubsection{Sine wave of a given frequency}
\begin{center}
\begin{tabular}{cc}
\toprule
Code & Visualization\\
\midrule
\cs{sine\{f\_0\}} & \seno{f_0}\\
\cs{seno\{f\_0\}} & \seno{f_0}\\
\bottomrule
\end{tabular}
\end{center}

\subsubsection{Energy}
\begin{center}
\begin{tabular}{cc}
\toprule
Code & Visualization\\
\midrule
\cs{energy\{m\}} & \energia{m}\\
\cs{energia\{m\}} & \energia{m}\\
\bottomrule
\end{tabular}
\end{center}

\subsubsection{Module}
\begin{center}
\begin{tabular}{cc}
\toprule
Code & Visualization\\
\midrule
\cs{Abs\{x\}} & \modulo{x}\\
\cs{modulo\{x\}} & \modulo{x}\\
\bottomrule
\end{tabular}
\end{center}

\subsubsection{Exponential with ISO compliant natural base}
The ISO regulations require that the base of natural logarithms and of the exponential function be set in roman type.
\begin{center}
\begin{tabular}{cc}
\toprule
Code & Visualization\\
\midrule
\cs{rmexp\{x\}} & \rmexp{x}\\
\cs{ex\{x\}} & \ex{x}\\
\bottomrule
\end{tabular}
\end{center}

\subsubsection{ISO compliant imaginary unit, engineering style}
Similarly the ISO regulations require the imaginary unit to be typeset in roman type; electrical and telecommunications engineers use the letter ``j'' in order to avoid any possible confusion with the standard current symbol $i$; the roman type avoids also any possible confusion with the standard symbols for the current density~$j$.
\begin{center}
\begin{tabular}{cc}
\toprule
Code & Visualization\\
\midrule
\cs{iu\cs{omega}} & $\iu\omega$\\
\bottomrule
\end{tabular}
\end{center}

\subsubsection{Module with exponent}
\begin{center}
\begin{tabular}{cc}
\toprule
Code & Visualization\\
\midrule
\cs{AbsPow\{x\}\{2\}} & \moduloexp{x}{2}\\
\cs{moduloexp\{x\}\{2\}} & \moduloexp{x}{2}\\
\bottomrule
\end{tabular}
\end{center}

\subsubsection{Function value for a certain value of the independent variable}
\begin{center}
\begin{tabular}{cc}
\toprule
Code & Visualization\\
\midrule
\cs{for\{f(x)\}\{x\_0\}} & \for{f(x)}{x_0}\\
\bottomrule
\end{tabular}
\end{center}


\subsubsection{A ratio in dB}
A special application of the previous command \cs{for}.
\begin{center}
\begin{tabular}{cc}
\toprule
Code & Visualization\\
\midrule
\cs{indB\{\cs{dfrac}\{C\}\{I\}\}}  & $\indB{\dfrac{C}{I}}$\\
\bottomrule
\end{tabular}
\end{center}

\subsubsection{Maximum}
\begin{center}
\begin{tabular}{cc}
\toprule
Code & Visualization\\
\midrule
\cs{Max\{x\}} & \massimo{x}\\
\cs{massimo\{x\}} & \massimo{x}\\
\bottomrule
\end{tabular}
\end{center}

\subsubsection{Minimum}
\begin{center}
\begin{tabular}{cc}
\toprule
Code & Visualization\\
\midrule
\cs{Min\{x\}} & \minimo{x}\\
\cs{minimo\{x\}} & \minimo{x}\\
\bottomrule
\end{tabular}
\end{center}

\subsubsection{Speed of light}
\begin{center}
\begin{tabular}{cc}
\toprule
Code & Visualization\\
\midrule
\cs{clight} & $\valc$\\
\cs{valc} & $\valc$\\
\bottomrule
\end{tabular}
\end{center}
\subsubsection{Logarithm with specified base}
\begin{center}
\begin{tabular}{cc}
\toprule
Code & Visualization\\
\midrule
\cs{Log\{2\}\{x\}} & \loga{2}{x}\\
\cs{loga\{2\}\{x\}} & \loga{2}{x}\\
\bottomrule
\end{tabular}
\end{center}

\subsubsection{Integral}
An integral on the whole real domain, from $-\infty$ to $+\infty$:
\begin{center}
\begin{tabular}{cc}
\toprule
Code & Visualization\\
\midrule
\cs{infint\{x\cs{diff} x\}} & \intinf{x\diff x}\\[1ex]
\cs{intinf\{x\cs{diff} x\}} & \intinf{x\diff x}\\
\bottomrule
\end{tabular}
\end{center}
Notice the use of the differential sign \cs{diff} that complies with the ISO regulations that require the differential sign be set in roman type with the proper spacings on the left and on the right.

\subsubsection{Dirac's Delta with independent variable}
\begin{center}
\begin{tabular}{cc}
\toprule
Code & Visualization\\
\midrule
\cs{deltain\{x\}} & \deltain{x}\\
\bottomrule
\end{tabular}
\end{center}

\subsection{Reflectances}
The reflectance symbols used in electrical and magnetic field theory courses are  described; you have two kind of commands, the generic ones, and the specific ones where you can specify the point of measure.

\subsubsection{Generic reflectances}
\begin{center}
\begin{tabular}{cc}
\toprule
Code & Visualization\\
\midrule
\cs{Vgamma} & \gammatens\\
\cs{gammatens} & \gammatens\\
\cs{Cgamma} & \gammacorr\\
\cs{gammacorr} & \gammacorr\\
\bottomrule
\end{tabular}
\end{center}

\subsubsection{Specific reflectances}
\begin{center}
\begin{tabular}{cc}
\toprule
Code & Visualization\\
\midrule
\cs{Vgammain\{A\}} & \gammatensin{A}\\
\cs{gammatensin\{A\}} & \gammatensin{A}\\
\cs{Cgammain\{A\}} & \gammacorrin{A}\\
\cs{gammacorrin\{A\}} & \gammacorrin{A}\\
\bottomrule
\end{tabular}
\end{center}
Since the voltage reflectance (gamma) is used more often than the current one, it may be input with a shorter command:
\begin{center}
\begin{tabular}{cc}
\toprule
Code & Visualization\\
\midrule
\cs{gammain\{A\}} & \gammain{A}\\
\bottomrule
\end{tabular}
\end{center}
\subsubsection{The Kurokawa reflectance}
\begin{center}
\begin{tabular}{cc}
\toprule
Code & Visualization\\
\midrule
\cs{gammak} & \gammak\\
\bottomrule
\end{tabular}
\end{center}

\subsection{Usage examples}
Here are some examples that clearly show the usefulness of \textsf{engtlc}:
\begin{enumerate}
\item \begin{tabbing}
\hspace{7cm}\=\kill
Code \> Visualization\\ 
\cs{AbsPow\{\cs{gammak}\}\{2\}} \> \AbsPow{\gammak}{2}
\end{tabbing}
\item \begin{tabbing}
\hspace{9cm}\=\kill
Code \> Visualization \\ 
\cs{bfRe\{\cs{fourier}\{\cs{AbsPow}\{x\}\{2\}\}\}} \> $\bfRe{\fourier{\AbsPow{x}{2}}}$
\end{tabbing}
The standard code to write the above expression is:
\begin{lstlisting}
$\textbf{Re}\left\lbrace \mathcal{F}\left\lbrace
\left\vert x \right\vert^{2} \right\rbrace \right\rbrace$
\end{lstlisting}

\end{enumerate}

\subsection{Wave length symbols}
There are three different types of symbols:

\subsubsection{In the open space}
\begin{center}
\begin{tabular}{cc}
\toprule
Code & Visualization\\
\midrule
\cs{lbvt} & $\lbvt$\\
\bottomrule
\end{tabular}
\end{center}

\subsubsection{In a guide filled with generic dielectric}
\begin{center}
\begin{tabular}{cc}
\toprule
Code & Visualization\\
\midrule
\cs{lbg} & $\lbg$\\
\bottomrule
\end{tabular}
\end{center}

\subsubsection{In an empty guide }
\begin{center}
\begin{tabular}{cc}
\toprule
Code & Visualization\\
\midrule
\cs{lbgvt} & $\lbgvt$\\
\bottomrule
\end{tabular}
\end{center}

\subsection{Impedance and admittance symbols}
With \textsf{engtlc} you can also write any kind of impedance and admittance.

\subsubsection{Generic impedance and admittance}
To input an impedance or admittance at a specific port or line section~A use:
\begin{center}
\begin{tabular}{cc}
\toprule
Code & Visualization\\
\midrule
\cs{z\{A\}} & \z{A}\\
\cs{y\{A\}} & \y{A}\\
\bottomrule
\end{tabular}
\end{center}
To input a normalized impedance or admittance at a specific port or line section~A use:
\begin{center}
\begin{tabular}{cc}
\toprule
Code & Visualization\\
\midrule
\cs{znorm\{A\}} & \znorm{A}\\
\cs{ynorm\{A\}} & \ynorm{A}\\
\bottomrule
\end{tabular}
\end{center}

\subsubsection{Characteristic impedance and admittance}
To input the symbol for characteristic impedance or admittance use:
\begin{center}
\begin{tabular}{cc}
\toprule
Code & Visualization\\
\midrule
\cs{zinf} & \zinf\\
\cs{yinf} & \yinf\\
\bottomrule
\end{tabular}
\end{center}
In a guide there are several mode dependent characteristic impendances: you may use another command in order to distinguish them; for example if you want to label with~2 a second impedance you may use:
\begin{center}
\begin{tabular}{cc}
\toprule
Code & Visualization\\
\midrule
\cs{zinf[2]}	& \zinf[2]\\
\cs{zinfn\{2\}} & \zinfn{2}\\
\cs{yinf[2]}	& \yinf[2]\\
\cs{yinfn\{2\}} & \yinfn{2}\\
\bottomrule
\end{tabular}
\end{center}
To avoid confusion pay attention when you input a numbered impedance and you use the \cs{zinf} or \cs{yinf} commands:  you must use either the optional argument command or the n-extended command:
\begin{tabbing}
\hspace{7cm}\=\kill
Impedance with optional argument 	\> Visualization	\\
\cs{zinf[2]}						\> \zinf[2]			\\
n-extended impedance Code 			\> Visualization 	\\ 
\cs{zinfn\{2\}} 					\>  \zinfn{2}		\\
Generic impedance Code 				\> Visualization 	\\ 
\cs{zinf\{2\}} 						\>  \zinf{2}
\end{tabbing}
The last code is not wrong under the \LaTeX\ point of view, but probably the output is not the expected one\dots

The open space impedance and admittance is input by means of:
\begin{center}
\begin{tabular}{cc}
\toprule
Code & Visualization\\
\midrule
\cs{zvt} & \zvt\\
\cs{yvt} & \yvt\\
\bottomrule
\end{tabular}
\end{center}

\subsection{Some more usage examples}

\noindent With \textsf{engtlc} commands the available power can be written with the code:
\begin{verbatim}
\[\availpow=\frac{\AbsPow{V}{2}}{\bfRe{4\cdot\z{G}}}\]
\end{verbatim}
that yields:
\[ \availpow=\frac{\AbsPow{V}{2}}{4\cdot \bfRe{\z{G}}} \]
Without the \textsf{engtlc} commands the code would be:
\begin{verbatim}
\[P_{\mathrm{disp}}=\frac{\left| V \right|^2}%
{ 4\cdot\textbf{Re}\left\{ Z_\mathrm{G}\right\}}\]
\end{verbatim}

\subsection{Power symbols}
Here we have the commands to input power symbols; the following commands may be used in both text and math modes.

\subsubsection{Power into a port}
To write the power entering into a specific port or line\slash guide section~A use:
\begin{center}
\begin{tabular}{cc}
\toprule
Code & Visualization\\
\midrule
\cs{powerin\{A\}} & \powerin{A}\\
\cs{potin\{A\}} & \potin{A}\\
\bottomrule
\end{tabular}
\end{center}

\subsubsection{Available Power}
\begin{center}
\begin{tabular}{cc}
\toprule
Code & Visualization\\
\midrule
\cs{availpow} & \potdisp\\
\cs{potdisp} & \potdisp\\
\bottomrule
\end{tabular}
\end{center}

\subsubsection{Power supply}
\begin{center}
\begin{tabular}{cc}
\toprule
Code & Visualization\\
\midrule
\cs{potDC} & \potDC\\
\cs{potCC} & \potCC\\
\bottomrule
\end{tabular}
\end{center}
For compatibility reasons, there is also the possibility of exploit \cs{potalim} the old name for the symbol.

\subsubsection{Radiation Power}
\begin{center}
\begin{tabular}{cc}
\toprule
Code & Visualization\\
\midrule
\cs{irrpow} & \potirr\\
\cs{potirr} & \potirr\\
\bottomrule
\end{tabular}
\end{center}

\subsubsection{Dissipated Power}
\begin{center}
\begin{tabular}{cc}
\toprule
Code & Visualization\\
\midrule
\cs{disspow} & \potdiss\\
\cs{potdiss} & \potdiss\\
\bottomrule
\end{tabular}
\end{center}

\subsubsection{Incident power}
\begin{center}
\begin{tabular}{cc}
\toprule
Code & Visualization\\
\midrule
\cs{incpow} & \potinc\\
\cs{potinc} & \potinc\\
\bottomrule
\end{tabular}
\end{center}

Pay attention that, in this version, all shortcuts describing power symbols, a part from the first one \cs{powerin\{A\}}, have an optional argument: it is used to indicate explicitly the point in which you can measure the power. To describe, for example, the dissipated power for a general transmitting device, you should use \cs{disspow[tx]} which give as a result $\disspow[tx]$. Be careful: if you put \{ \} instead of [ ] you obtain a different, undesired result. Indeed, \cs{irrpow\{rx\}} becomes $\irrpow{rx}$. 

\subsection{Electrical and magnetic field symbols}
With regards to electrical and magnetic fields, \textsf{engtlc} commands are not so well established, as they depend on every particular school, whether they are typeset or handwritten, etcetera. For informal handouts the underlined vector convention is often used. We distinguish:
\begin{itemize}
\item[$\star$] electric and magnetic field as a function of a vector position \vector{r} and time;
\item[$\star$] electric and magnetic field phasors as a function of a vector position~\vector{r}.
\end{itemize}
In several electric and magnetic fields courses in order to denote  $A$ as a vector, instead of using arrows ``accents'' or bold symbols, the convention $\underline{A}$ is used:
\begin{center}
\begin{tabular}{lc@{\qquad}lr}
\toprule
\multicolumn2{c@{\qquad}}{Time dependent fields}&
                             \multicolumn2c{Field phasors}		\\
%\midrule
Code 		& Visualization	& Code 				& \makebox[3em][r]{Visualization}	\\
\midrule
\cs{Efield} & \campoe		& \cs{phasorEfield}	& \campoefas	\\
\cs{campoe} & \campoe		& \cs{campoefas} 	& \campoefas	\\
\cs{Hfield} & \campoh		& \cs{phasorHfield}	& \campohfas	\\
\cs{campoh} & \campoh		& \cs{campohfas} 	& \campohfas	\\
\bottomrule
\end{tabular}
\end{center}

Pay attention: we distinguish the symbols of energy \cs{energy\marg{argument}} and electric field \cs{campoe} because the second one is a vector and it does not have subscript symbols.
\subsection{Probability symbols}

\subsubsection{Probability}
The probability of event A, or of variable $x$, is input as:
\begin{center}
\begin{tabular}{cc}
\toprule
Code & Visualization\\
\midrule
\cs{prob\{\cs{text}\{A\}\}} & \prob{\text{A}}\\
\cs{prob\{x\}} & \prob{x}\\
\bottomrule
\end{tabular}
\end{center}

\subsubsection{Expected value}
\begin{center}
\begin{tabular}{cc}
\toprule
Code & Visualization\\
\midrule
\cs{expval\{x\}} & \expval{x}\\
\cs{valatt\{x\}} & \valatt{x}\\
\bottomrule
\end{tabular}
\end{center}

\subsubsection{Variance}
\begin{center}
\begin{tabular}{cc}
\toprule
Code & Visualization\\
\midrule
\cs{var\{x\}} & \var{x}\\
\bottomrule
\end{tabular}
\end{center}

\subsubsection{Joint probability}
This macro inserts a well separated comma between two stochastic variables:
\begin{center}
\begin{tabular}{cc}
\toprule
Code & Visualization\\
\midrule
\cs{comma} & \prob{x\comma y}\\
\bottomrule
\end{tabular}
\end{center}

\subsubsection{Conditional probability}
These macros insert a vertical bar separator between two stochastic variables:
\begin{center}
\begin{tabular}{cc}
\toprule
Code & Visualization\\
\midrule
\cs{given} & \prob{x\given y}\\
\cs{dato} & \prob{x\dato y}\\
\bottomrule
\end{tabular}
\end{center}

\section*{Aknowledgements}
I wish to thank prof.~Enrico Gregorio, who very kindly sent me a complete list of errors in my third package edition and documentation.

I wish to thank also prof.~Claudio Beccari for many useful suggestions and the adaptation of several commands to the ISO regulations.


\chapter{LPPL Licence}
\textsf{engtlc} is distributed under LPPL Licence: \LaTeX\, Project Public Licence.
\begin{footnotesize}
\begin{verbatim}
%% engtlc.sty
%% Copyright 2010-2012 Claudio Fiandrino
%
%  This work may be distributed and/or modified under the
%  conditions of the LaTeX Project Public License, either version 1.3
%  of this license or (at your option) any later version.
%  The latest version of this license is in
%  http://www.latex-project.org/lppl.txt
%  and version 1.3 or later is part of all distributions of LaTeX
%  version 2005/12/01 or later.
%
%  This work has the LPPL maintenance status `maintained'.
% 
%  The Current Maintainer of this work is Fiandrino Claudio.
%
%  This work consists of the file engtlc.sty.
\end{verbatim}
\end{footnotesize}
\end{document}
