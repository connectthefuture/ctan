% \iffalse meta-comment
%
% Package lcd
% Copyright (c) 2004 Mike Kaufmann, all rights reserved
%
% This program is provided under the terms of the
% LaTeX Project Public License distributed from CTAN
% archives in directory macros/latex/base/lppl.txt.
%
% Author: Mike Kaufmann
%         Mike.Kaufmann@ei.fh-giessen.de
% \fi
%% \CharacterTable
%%  {Upper-case    \A\B\C\D\E\F\G\H\I\J\K\L\M\N\O\P\Q\R\S\T\U\V\W\X\Y\Z
%%   Lower-case    \a\b\c\d\e\f\g\h\i\j\k\l\m\n\o\p\q\r\s\t\u\v\w\x\y\z
%%   Digits        \0\1\2\3\4\5\6\7\8\9
%%   Exclamation   \!     Double quote  \"     Hash (number) \#
%%   Dollar        \$     Percent       \%     Ampersand     \&
%%   Acute accent  \'     Left paren    \(     Right paren   \)
%%   Asterisk      \*     Plus          \+     Comma         \,
%%   Minus         \-     Point         \.     Solidus       \/
%%   Colon         \:     Semicolon     \;     Less than     \<
%%   Equals        \=     Greater than  \>     Question mark \?
%%   Commercial at \@     Left bracket  \[     Backslash     \\
%%   Right bracket \]     Circumflex    \^     Underscore    \_
%%   Grave accent  \`     Left brace    \{     Vertical bar  \|
%%   Right brace   \}     Tilde         \~}
%%
% \CheckSum{469}
%
% \iffalse meta-comment
%
%<*package>
\def\fileversion{0.3}
\def\filedate{2004/01/05}
%</package>
%
%<*driver>
\documentclass{ltxdoc}
\usepackage[latin1]{inputenc}
% redefine � (\mu) to get it right in source code listing
\DeclareInputText{181}{\ensuremath{\mu}}
\usepackage{lcd}
\EnableCrossrefs
\CodelineIndex
\RecordChanges
%\OnlyDescription
\begin{document}
   \DocInput{lcd.dtx}
\end{document}
%</driver>
%
%<*package>
% \fi
%
% ^^A Don't ask me why, but the first two \DoNotIndex-line don't work
% \DoNotIndex{\",\#,\$,\%,\&,\^,\_,\~}
% \DoNotIndex{\^^b0,\^^b5,\^^b7,\^^df,\^^e4,\^^f6,\^^fc}
% \DoNotIndex{\@@end,\@empty,\@ifundefined}
% \DoNotIndex{\@tempa,\@tempcnta,\@tempcntb}
% \DoNotIndex{\AtBeginDocument,\advance,\addtocounter}
% \DoNotIndex{\begin,\begingroup}
% \DoNotIndex{\csname,\catcode,\color,\colorbox}
% \DoNotIndex{\def}
% \DoNotIndex{\else,\end,\endgroup,\endcsname,\expandafter}
% \DoNotIndex{\fboxsep,\framebox}
% \DoNotIndex{\global}
% \DoNotIndex{\fi,\filedate,\fileversion}
% \DoNotIndex{\hspace}
% \DoNotIndex{\if,\ifnum,\ifx}
% \DoNotIndex{\let,\line,\linethickness}
% \DoNotIndex{\makebox,\multiply}
% \DoNotIndex{\NeedsTeXFormat,\newcommand,\newcounter,\newlength,\newif}
% \DoNotIndex{\obeyspaces}
% \DoNotIndex{\ProvidesPackage,\put}
% \DoNotIndex{\relax}
% \DoNotIndex{\space,\setlength,\setcounter,\settoheight,\stepcounter}
% \DoNotIndex{\the}
% \DoNotIndex{\unitlength}
% \DoNotIndex{\z@}
%
% ^^A\changes{0.3}{2004/01/04}{initial release}
%
% ^^A\MakeShortVerb{\+}
% ^^A\DeleteShortVerb{\|}
%
% \newcommand*{\bs}{\char '134 }
% \newcommand*{\lb}{\char '173 }
% \newcommand*{\rb}{\char '175 }
% \newcommand*{\param}[1]{\texttt{\textit{#1}}}
% \newcommand*\lcd{\textLCD{3}|LCD|}
%
% \title{The \lcd\ Package\\alphanumerical LCDisplays with \LaTeX}
% \author{Mike Kaufmann\\|Mike.Kaufmann@ei.fh-giessen.de|}
% \date{\filedate~(v\fileversion)}
% ^^A--------------------------------------------------------------------------
% \maketitle
% \begin{abstract}
% The \lcd\ Package provides macros that display text which looks like on
% alphanumerical LCDisplays.
% \end{abstract}
%
% \tableofcontents
%
% ^^A--------------------------------------------------------------------------
% \newpage
% \section{Introduction}
% \subsection{Why this Package}\label{sec:why}
% Without this package you can show the contents of an alphanumerical LCDisplay
% with the typewriter font or you can take a picture and include it in your
% document.  The typewriter font doesn't provide all characters possible on an
% LCD and perhaps a picture will need postprocessing.
%
% \begin{center}
% \LCD{4}{40}+With the LCD package text appears like  +
%            +on an alphanumerical LCDisplay (colors  +
%            +are available) and you can easily define+
%            +your own symbols (5x7 dot matrix).      +
% \end{center}
%
% ^^A---------------------------------------------
% \subsection{Making the Stylefile}
% Because character with ASCII-codes greater then 127 are used in the
% macro code, your TeX must be able to handle 8-bit ASCII to get a correct
% stylefile!
%
% After generating the file lcd.sty search for the string ``|^^|'' (followed
% by a hex-number). If you find it, you are in trouble.
%
% If you are using teTeX the command
% \begin{verbatim}
%         tex --translate-file=cp8bit lcd.ins\end{verbatim}
% ^^A\end{verbatim}
% will work properly.
%
% Here a list of the hex-number--character pairs:\newline
% |^^b5| $\to \mu$, |^^b0| $\to \;^\circ$, |^^b7| $\to \cdot$,
% |^^e4| $\to$ \"a, |^^f6| $\to$ \"o, |^^fc| $\to$ \"u,
% |^^df| $\to$ \ss.\newline
% The list applies to Linux. There might be more trouble with other operating
% systems. But in worst case only these characters will not work.
%
% ^^A---------------------------------------------
% \subsection{Legal Stuff}
% This program is provided under the terms of the
% LaTeX Project Public License distributed from CTAN
% archives in directory macros/latex/base/lppl.txt.
%
% ^^A--------------------------------------------------------------------------
% \section{Using the Package}
% \subsection{The main \protect\lcd\ Macros}\label{sec:mac}
% \DescribeMacro\LCD
% To show the contents of an alphanumerical LCDisplay the macro
% \begin{center}
% |\LCD{|\param{LCDlines}|}{|\param{LCDcolumns}|}|\meta{delimchar}\param{LCDtext}\meta{delimchar}\\
% \ldots \meta{delimchar}\param{LCDtext}\meta{delimchar}
% \end{center}
% can be used.
%
% Here \param{LCDlines} is the number of lines you want to have. It has to be
% at least one, which is tested and corrected if necessary, but without any
% message or warning.
%
% The parameter \param{LCDcolumns} is the number of columns you want to have. It also
% has to be at least one, but there is no test, message or warning jet.
%
% The numbers of lines and columns are not limited by the macro. But of course
% there are limits by \TeX.
%
% The delimiter character \meta{delimchar} has to follow immediately after the
% \param{LCDcolumns} parameter. It can be any character with catcode 11 or 12,
% which means that active characters, a space and characters with a special
% meaning to \TeX\ are not allowed. Of course \meta{delimchar} must not be part
% of \param{LCDtext}.
%
% The opening \meta{delimchar} of one LCD-line can follow
% immediately after the closing \meta{delimchar} of the line before. And there
% can be spaces and/or newlines between these delimiters. Also text can be
% there, but since it will be ignored it wouldn't make any sense.
%
% The \param{LCDtext} can consist of nearly any characters you can type, expect
% of |\|, |{| and |}|. The braces are used to enclose so called multi-letter
% characters, which are used to show symbols you can't type (e.g.\ |{clock}| to
% get a clock symbol; see \ref{sec:pre} for more).
%
% The example in \ref{sec:why} was produced with
% \begin{verbatim}
% \LCD{4}{40}+With the LCD package text appears like  +
%            +on an alphanumerical LCDisplay (colors  +
%            +are available) and you can easily define+
%            +your own symbols (5x7 dot matrix).      +\end{verbatim}
% ^^A\end{verbatim}
%
% \DescribeMacro\textLCD
% To put \param{LCDtext} in normal text the macro
% \begin{center}
% |\textLCD[|\param{LCDcorr}|]{|\param{LCDcolumns}|}|\meta{delimchar}\param{LCDtext}\meta{delimchar}
% \end{center}
% can be used.
%
% For \param{LCDcolumns}, \meta{delimchar} and \param{LCDtext} applies the same as
% for these parameters in |\LCD|.
%
% The parameter \param{LCDcorr} is optional. It is used to reduce or enlarge
% the space between the \param{LCDtext} and the surrounding normal text. The
% default value ($-2$) fits for white backgrounded \param{LCDtext}. If you have
% a background color the correct value is 0. It has to be a decimal number.
%
% \DescribeMacro\textLCDcorr
% The default value is defined as the macro
% |\textLCDcorr|. If you want to change the default value globaly, you can
% redefine this macro in the preamble by typing
% |\renewcommand{\textLCDcorr}{|\param{newdefault}|}| where \param{newdefault}
% must be only a decimal number.
%
% ^^A---------------------------------------------
% \subsection{Restrictions}
% There are some restrictions you have to remember when using the \lcd\ package.
% Ignoring them will lead to strange behavior or strange error messages.
% And there are no package errors or package warnings jet. Here are the
% restrictions:
%
% \begin{itemize}
% \item There have to be exactly the number of LCD-lines given in the
%       \param{LCDlines}-Parameter. Lines that are too much will be printed as
%       normal text including the delimiter characters. If there are not enough
%       lines, you'll get a ``Runaway argument'' error or other strange things.
% \item After a multi-letter character there have to be at least one other
%       charater, otherwise you'll get the word instead of the symbol
%       (e.g.\ the word ``clock'' and not the clock symbol). If a multi-letter
%       character is the last one in the line, just type a space after it.
% \item The number of characters in a line is not tested. If there are too many
%       characters, they will be drawn over the right end.
% \item It is not possible to place comments between the LCD-lines, the \%-sign
%       will be ignored there and in \param{LCDtext} it is a normal character.
% \item There can't be a linebreak in \param{LCDtext} when using |\textLDC|.
% \item You can't use the |\LCD| and |\textLCD| macro in other macros or as
%       parameter to a macro. Well, if the LCD-text only consists of letters,
%       digits and some other characters it would work. But a space, active
%       characters and characters with a special meaning to \TeX\ (all
%       characters with other catcodes than 11 or 12) wouldn't work. If you want
%       to use |\textLCD| in sectioning commands or in captions, you have to
%       |\protect| it.
% \item You can't use the |\LCD| and |\textLCD| macro in boxes like |\parbox|,
%       |\raisebox| or |\makebox| (with the same exceptions as to the point
%       above). But you can use it in environments like |minipage| or |figure|.
% \end{itemize}
%
% ^^A---------------------------------------------
% \subsection{Size}\label{sec:size}
% \DescribeMacro\LCDunitlength
% To specify the absolte size of the LCD representation you have to set the length
% |\LCDunitlength|. It represents the width of a dot plus the gap between the
% dots. The default value is 0.5\,mm.
%
% For |\textLCD| the size is calculated automaticly so that the height of
% \param{LCDtext} and the surrounding text fits. This does not affect
% |\LCDunitlength|.
%
% ^^A---------------------------------------------
% \subsection{Frames}\label{sec:frame}
% \DescribeMacro\LCDnoframe
% \DescribeMacro\LCDframe
% By default a frame is drawn around the LCD representation. You can change this
% behavior with the macro
% |\LCDnoframe| in the preamble or befor the |\LCD| command. With
% |\LCDframe| you can switch back to the default. Note that with |\textLCD|
% there will never be a frame for typographical reasons.
%
% ^^A---------------------------------------------
% \subsection{Using Colors}\label{sec:color}
% \DescribeMacro\LCDcolors
% To use colors you need the color package and you have to define foreground,
% background and frame colors. The latter because the \lcd\ package only handles
% named colors. After that you can set the colors with
% \begin{center}
% |\LCDcolors[|\param{framecolor}|]{|\param{foregroundcolor}|}{|\param{backgroundcolor}|}|.
% \end{center}
% This can be done in the preamble or befor the |\LCD| or |\textLCD| command.
% The parameter \param{framecolor} is optional and the frame color will be set
% to black if it is not given. The default colors where set with
% |\LCDcolors{black}{white}|.
%
% ^^A---------------------------------------------
% \subsection{Predefined Characters}\label{sec:pre}
% Of course all letters (A--Z and a--z), digits (0--9) and the characters
% ! ' ( ) * + , - . / : ; $<$ = $>$ ? [ ] ` and $\vert$ are predefined.
% You can also type \$ \# $\mu$ $^\circ$ $\cdot$ \"a \"o \"u \ss\ " \% \&
% \^\ and \_ dircetly. The character \~\ will be shown as a space.
%
% Some characters are available as multi-letter characters.
% These are a clock symbol (|{clock}|), a right arrow (|{rarrow}|), a left arrow
% (|{larrow}|), $\Omega$ (|{Omega}|), $\Sigma$ (|{Sigma}|), $\pi$ (|{pi}|),
% a square root symbol (|{sqrt}|), a rectangle (|{rect}|),
% a full cursor (|{fcur}|), \{ (|{lb}|), \} (|{rb}|) and $\alpha$ (|{alpha}|).
%
% ^^A---------------------------------------------
% \subsection{Defining new Characters}\label{sec:def}
% \DescribeMacro\DefineLCDchar
% New characters can be defined with
% |\DefineLCDchar{|\param{char}|}{|\param{charmatrix}|}|.
%
% Here \param{char} can be a single character or multiple letters. Note
% that characters with catcodes other than 11 or 12 (characters that are
% control sequences or have a special meaning to \TeX) will cause troubble.
% Use mutiple letters instead.
%
% LCD characters are shown as a $5\times7$ dot matrix. Other matrix dimensions
% are not possible. So in \param{charmatrix} you have to type exactly 35
% times |0| or |1|, the ones are for visible dots. The first five ar for the
% first dot row, the next five are for the second dot row and so on.
%
% As example, lets define an Euro symbol:
%
% \begin{figure}[htbp]
% \centering
% \unitlength2mm
% \begin{picture}(54,10)
% \put(0,8){\makebox(0,0)[bl]{
%    \texttt{\bs DefineLCDchar\lb euro\rb\lb00111010001111101000111110100000111\rb}}}
% \put(5.5,6.5){\line(1,0){16.7}}\put(22.2,6.5){\line(0,1){1}}\put(20.2,7.5){\line(1,0){4}}
% \put(5.5,5.5){\line(1,0){21.4}}\put(26.9,5.5){\line(0,1){2}}\put(24.9,7.5){\line(1,0){4}}
% \put(5.5,4.5){\line(1,0){26.1}}\put(31.6,4.5){\line(0,1){3}}\put(29.5,7.5){\line(1,0){4}}
% \put(5.5,3.5){\line(1,0){30.8}}\put(36.3,3.5){\line(0,1){4}}\put(34.3,7.5){\line(1,0){4}}
% \put(5.5,2.5){\line(1,0){35.5}}\put(41.0,2.5){\line(0,1){5}}\put(39.0,7.5){\line(1,0){4}}
% \put(5.5,1.5){\line(1,0){40.2}}\put(45.7,1.5){\line(0,1){6}}\put(43.7,7.5){\line(1,0){4}}
% \put(5.5,0.5){\line(1,0){44.9}}\put(50.4,0.5){\line(0,1){7}}\put(48.4,7.5){\line(1,0){4}}
% \multiput(0,0)(1,0){6}{\line(0,1){7}}
% \multiput(0,0)(0,1){8}{\line(1,0){5}}
% \linethickness{0.7\unitlength}
% \put(2,6.35){\line(1,0){0.7}}
% \put(3,6.35){\line(1,0){0.7}}
% \put(4,6.35){\line(1,0){0.7}}
% \put(1,5.35){\line(1,0){0.7}}
% \put(0,4.35){\line(1,0){0.7}}
% \put(1,4.35){\line(1,0){0.7}}
% \put(2,4.35){\line(1,0){0.7}}
% \put(3,4.35){\line(1,0){0.7}}
% \put(4,4.35){\line(1,0){0.7}}
% \put(1,3.35){\line(1,0){0.7}}
% \put(0,2.35){\line(1,0){0.7}}
% \put(1,2.35){\line(1,0){0.7}}
% \put(2,2.35){\line(1,0){0.7}}
% \put(3,2.35){\line(1,0){0.7}}
% \put(4,2.35){\line(1,0){0.7}}
% \put(1,1.35){\line(1,0){0.7}}
% \put(2,0.35){\line(1,0){0.7}}
% \put(3,0.35){\line(1,0){0.7}}
% \put(4,0.35){\line(1,0){0.7}}
% \end{picture}
% \caption{Defining a new Character}
% \end{figure}
%
% ^^A---------------------------------------------
% \subsection{Package Options}
% Only for completeness: there are no package options in this version.
%
% ^^A--------------------------------------------------------------------------
% \StopEventually{\newpage\PrintIndex \PrintChanges}
%
% ^^A--------------------------------------------------------------------------
% \newpage
% \section{The Code}
% \subsection{The Usual}
% First the usual things.
%    \begin{macrocode}
\NeedsTeXFormat{LaTeX2e}[2001/06/01]
\ProvidesPackage{lcd}[\filedate\space
    v\fileversion\space drawing alphanumerical LCDisplays]
%    \end{macrocode}
%
% ^^A---------------------------------------------
% \subsection{Defining Characters and predefined Characters}
% \begin{macro}{\DefineLCDchar}
% The macro |\DefineLCDchar| defines a macro |\@LCD@|\param{char} for every
% character.
%    \begin{macrocode}
\newcommand*\DefineLCDchar[2]{%
    \global\expandafter\def\csname @LCD@#1\endcsname{#2}}
%    \end{macrocode}
% \end{macro}
%
% \noindent
% Here are the predefined characters.
%    \begin{macrocode}
\DefineLCDchar{A}{01110100011000110001111111000110001}
\DefineLCDchar{B}{11110100011000111110100011000111110}
\DefineLCDchar{C}{01110100011000010000100001000101110}
\DefineLCDchar{D}{11100100101000110001100011001011100}
\DefineLCDchar{E}{11111100001000011111100001000011111}
\DefineLCDchar{F}{11111100001000011111100001000010000}
\DefineLCDchar{G}{01110100011000010111100011000101110}
\DefineLCDchar{H}{10001100011000111111100011000110001}
\DefineLCDchar{I}{01110001000010000100001000010001110}
\DefineLCDchar{J}{00111000100001000010000101001001100}
\DefineLCDchar{K}{10001100101010011000101001001010001}
\DefineLCDchar{L}{10000100001000010000100001000011111}
\DefineLCDchar{M}{10001110111010110101100011000110001}
\DefineLCDchar{N}{10001100011100110101100111000110001}
\DefineLCDchar{O}{01110100011000110001100011000101110}
\DefineLCDchar{P}{11110100011000111110100001000010000}
\DefineLCDchar{Q}{01110100011000110001101011001001101}
\DefineLCDchar{R}{11110100011000111110101001001010001}
\DefineLCDchar{S}{01111100001000001110000010000111110}
\DefineLCDchar{T}{11111001000010000100001000010000100}
\DefineLCDchar{U}{10001100011000110001100011000101110}
\DefineLCDchar{V}{10001100011000110001100010101000100}
\DefineLCDchar{W}{10101101011010110101101011010101010}
\DefineLCDchar{X}{10001100010101000100010101000110001}
\DefineLCDchar{Y}{10001100011000101010001000010000100}
\DefineLCDchar{Z}{11111000010001000100010001000011111}
\DefineLCDchar{a}{00000000000111100001011111000101111}
\DefineLCDchar{b}{10000100001011011001100011000111110}
\DefineLCDchar{c}{00000000000111010000100001000101110}
\DefineLCDchar{d}{00001000010110110011100011000101111}
\DefineLCDchar{e}{00000000000111010001111111000001110}
\DefineLCDchar{f}{00110010010100011100010000100001000}
\DefineLCDchar{g}{00000011111000110001011110000101110}
\DefineLCDchar{h}{10000100001011011001100011000110001}
\DefineLCDchar{i}{00100000000110000100001000010001110}
\DefineLCDchar{j}{00010000000011000010000101001001100}
\DefineLCDchar{k}{10000100001001010100110001010010010}
\DefineLCDchar{l}{01100001000010000100001000010001110}
\DefineLCDchar{m}{00000000001101010101101011000110001}
\DefineLCDchar{n}{00000000001011011001100011000110001}
\DefineLCDchar{o}{00000000000111010001100011000101110}
\DefineLCDchar{p}{00000000001111010001111101000010000}
\DefineLCDchar{q}{00000000000110110011011110000100001}
\DefineLCDchar{r}{00000000001011011001100001000010000}
\DefineLCDchar{s}{00000000000111010000011100000111110}
\DefineLCDchar{t}{01000010001110001000010000100100110}
\DefineLCDchar{u}{00000000001000110001100011001101101}
\DefineLCDchar{v}{00000000001000110001100010111000100}
\DefineLCDchar{w}{00000000001010110101101011010101110}
\DefineLCDchar{x}{00000000001000101010001000101010001}
\DefineLCDchar{y}{00000000001000110001011110000101110}
\DefineLCDchar{z}{00000000001111100010001000100011111}
\DefineLCDchar{0}{01110100011001110101110011000101110}
\DefineLCDchar{1}{00100011000010000100001000010001110}
\DefineLCDchar{2}{01110100010000100010001000100011111}
\DefineLCDchar{3}{11111000100010000010000011000101110}
\DefineLCDchar{4}{00010001100101010010111110001000010}
\DefineLCDchar{5}{11111100001111000001000011000101110}
\DefineLCDchar{6}{00110010001000011110100011000101110}
\DefineLCDchar{7}{11111000010001000100010000100001000}
\DefineLCDchar{8}{01110100011000101110100011000101110}
\DefineLCDchar{9}{01110100011000101111000010001001100}
\DefineLCDchar{!}{00100001000010000100000000000000100}
\DefineLCDchar{'}{01100001000100000000000000000000000}
\DefineLCDchar{(}{00010001000100001000010000010000010}
\DefineLCDchar{)}{01000001000001000010000100010001000}
\DefineLCDchar{*}{00000001001010101110101010010000000}
\DefineLCDchar{+}{00000001000010011111001000010000000}
\DefineLCDchar{,}{00000000000000000000011000010001000}
\DefineLCDchar{-}{00000000000000011111000000000000000}
\DefineLCDchar{.}{00000000000000000000000000110001100}
\DefineLCDchar{/}{00000000010001000100010001000000000}
\DefineLCDchar{:}{00000011000110000000011000110000000}
\DefineLCDchar{;}{00000011000110000000011000010001000}
\DefineLCDchar{<}{00010001000100010000010000010000010}
\DefineLCDchar{=}{00000000001111100000111110000000000}
\DefineLCDchar{>}{10000010000010000010001000100010000}
\DefineLCDchar{?}{01110100010000100010001000000000100}
\DefineLCDchar{[}{01110010000100001000010000100001110}
\DefineLCDchar{]}{01110000100001000010000100001001110}
\DefineLCDchar{`}{01000001000001000000000000000000000}
\DefineLCDchar{|}{00100001000010000100001000010000100}
%    \end{macrocode}
%
% \noindent
% And the predefined multi-letter characters.
%    \begin{macrocode}
\DefineLCDchar{clock}{01110100011000111101101011010101110}
\DefineLCDchar{rarrow}{00000001000001011111000100010000000}
\DefineLCDchar{larrow}{00000001000100011111010000010000000}
\DefineLCDchar{Omega}{00000011101000110001010101101100000}
\DefineLCDchar{Sigma}{11111100000100000100010001000011111}
\DefineLCDchar{pi}{00000000001111101010010100101010011}
\DefineLCDchar{sqrt}{00000000000011100100001001010001000}
\DefineLCDchar{rect}{00000111111000110001100011000111111}
\DefineLCDchar{fcur}{11111111111111111111111111111111111}
\DefineLCDchar{lb}{00010001000010001000001000010000010}
\DefineLCDchar{rb}{01000001000010000010001000010001000}
\DefineLCDchar{alpha}{00000000000100110101100101001001101}
%    \end{macrocode}
%
% \noindent
% The next characters normaly can't be typed dirctly, so the catcodes are
% changed before defining them.
%    \begin{macrocode}
\begingroup
\catcode`\~=11 \catcode`\$=11 \catcode`\�=11 \catcode`\�=11
\catcode`\�=11 \catcode`\�=11 \catcode`\�=11 \catcode`\�=11
\catcode`\�=11 \catcode`\"=11 \catcode`\#=11 \catcode`\&=11
\catcode`\^=11 \catcode`\_=11
\DefineLCDchar{$}{00100011111010001110001011111000100}%$
\DefineLCDchar{#}{01010010101111101010111110101001010}
\DefineLCDchar{�}{00000100011000110001100111110110000}
\DefineLCDchar{�}{11100101001110000000000000000000000}
\DefineLCDchar{�}{00000000000000001100011000000000000}
\DefineLCDchar{�}{01010000000111000001011111000101111}
\DefineLCDchar{�}{01010000000111010001100011000101110}
\DefineLCDchar{�}{01010000001000110001100011001101101}
\DefineLCDchar{�}{00000011101000111110100011111010000}
\DefineLCDchar{"}{01010010100101000000000000000000000}
\DefineLCDchar{&}{01100100101010001000101011001011101}
\DefineLCDchar{^}{00100010101000100000000000000000000}
\DefineLCDchar{_}{00000000000000000000000000000011111}
\DefineLCDchar{~}{00000000000000000000000000000000000}
\catcode`\%=11
\DefineLCDchar{%}{11000110010001000100010001001100011}
\endgroup
%    \end{macrocode}
%
% ^^A---------------------------------------------
% \subsection{Registers needed}
% \begin{macro}{\LCDunitlength}
% The dimen |\LCDunitlength| is described in \ref{sec:size}.
%    \begin{macrocode}
\newlength\LCDunitlength
%    \end{macrocode}
% \end{macro}
%
% \begin{macro}{\c@@LCDdotx}
% The counter |\c@@LCDdotx| is used to calculate and hold the horizontal
% position of the actual dot.
%    \begin{macrocode}
\newcounter{@LCDdotx}
%    \end{macrocode}
% \end{macro}
%
% \begin{macro}{\c@@LCDdoty}
% The counter |\c@@LCDdoty| is used to calculate and hold the vertical
% position of the actual dot.
%    \begin{macrocode}
\newcounter{@LCDdoty}
%    \end{macrocode}
% \end{macro}
%
% \begin{macro}{\c@@LCDchrx}
% The counter |\c@@LCDchrx| holds the number of the actual character in a line,
% beginning with zero.
%    \begin{macrocode}
\newcounter{@LCDchrx}
%    \end{macrocode}
% \end{macro}
%
% \begin{macro}{\c@@LCDlines}
% The counter |\c@@LCDlines| is used to check the number of lines and after
% that the number of lines left to do is counted down there.
%    \begin{macrocode}
\newcounter{@LCDlines}
%    \end{macrocode}
% \end{macro}
%
% \begin{macro}{\c@@LCDlower}
% The counter |\c@@LCDlower| is used to lower the LCD contents in normal text.
% For |\LCD| it is set to zero, which means the contents isn't lowered. For
% |\textLCD| it is set to two, so the baselines of LCD contents and surrounding
% text will fit.
%    \begin{macrocode}
\newcounter{@LCDlower}
%    \end{macrocode}
% \end{macro}
%
% ^^A---------------------------------------------
% \subsection{Setups and Defaults}
% \begin{macro}{\LCDframe}
% \begin{macro}{\@LCDbox}
% The macro |\LCDframe| (see \ref{sec:frame}) sets the internal macro
% |\@LCDbox| to |\framebox|, so frames will be drawn.
%    \begin{macrocode}
\newcommand*\LCDframe{\let\@LCDbox\framebox}
%    \end{macrocode}
% \end{macro}
% \end{macro}
%
% \begin{macro}{\LCDnoframe}
% \begin{macro}{\@LCDbox}
% The macro |\LCDnoframe| (see \ref{sec:frame}) sets the internal macro
% |\@LCDbox| to |\makebox|, so frames will not be drawn.
%    \begin{macrocode}
\newcommand*\LCDnoframe{\let\@LCDbox\makebox}
%    \end{macrocode}
% \end{macro}
% \end{macro}
%
% \begin{macro}{\LCDcolors}
% \begin{macro}{\@LCDfr}
% \begin{macro}{\@LCDfg}
% \begin{macro}{\@LCDbg}
% The macro |\LCDcolors| (see \ref{sec:color}) defines three internal macros.
% |\@LCDfr| will be the framecolor, |\@LCDfg| will be the foregroundcolor and
% |\@LCDbg| will be the backgroundcolor.
%    \begin{macrocode}
\newcommand*\LCDcolors[3][black]{%
    \def\@LCDfr{#1}\def\@LCDfg{#2}\def\@LCDbg{#3}}
%    \end{macrocode}
% \end{macro}
% \end{macro}
% \end{macro}
% \end{macro}
%
% \noindent
% And now setting up the defaults.
%    \begin{macrocode}
\setlength\LCDunitlength{0.5mm}
\LCDframe
\LCDcolors{black}{white}
%    \end{macrocode}
%
% \noindent
% If the color package is not loaded the macros |\color| and |\colorbox| are
% defined, because they are used in |\@LCDstart|.
%    \begin{macrocode}
\AtBeginDocument{
    \@ifundefined{color}{\def\color#1{}}{}
    \@ifundefined{colorbox}{\def\colorbox#1#2{#2}}{}}
%    \end{macrocode}
%
% ^^A---------------------------------------------
% \subsection{Drawing}
% \begin{macro}{\@DrawLCDDot}
% The macro |\@DrawLCDDot| draws a single dot of a character.
%    \begin{macrocode}
\newcommand*\@DrawLCDDot{%
    \put(\the\c@@LCDdotx,\the\c@@LCDdoty.35){\line(1,0){0.7}}}
%    \end{macrocode}
% \end{macro}
%
% \begin{macro}{\@DrawLCDRow}
% The macro |\@DrawLCDRow| draws one dot line of a character. The parameters
% are five zeros or ones from the character matrix (see \ref{sec:def}).
%    \begin{macrocode}
\newcommand*\@DrawLCDRow[5]{%
    \def\@tempa{#1}\if\@tempa0\relax\else\@DrawLCDDot\fi\stepcounter{@LCDdotx}%
    \def\@tempa{#2}\if\@tempa0\relax\else\@DrawLCDDot\fi\stepcounter{@LCDdotx}%
    \def\@tempa{#3}\if\@tempa0\relax\else\@DrawLCDDot\fi\stepcounter{@LCDdotx}%
    \def\@tempa{#4}\if\@tempa0\relax\else\@DrawLCDDot\fi\stepcounter{@LCDdotx}%
    \def\@tempa{#5}\if\@tempa0\relax\else\@DrawLCDDot\fi
    \addtocounter{@LCDdotx}{-4}}
%    \end{macrocode}
% \end{macro}
%
% \begin{macro}{\@DrawLCDRows}
% The macro |\@DrawLCDRows| draws all seven dot rows of a character. It is
% first called with the complete character matrix (see \ref{sec:def}). The
% first five zeros or ones will be the parameters |#1| to |#5|, which are passed
% to |\@DrawLCDRow|. The parameter |#6| takes the rest of the character matrix.
% |\@DrawLCDRows| is called recursively until the whole matrix is processed.
%    \begin{macrocode}
\def\@DrawLCDRows#1#2#3#4#5#6\@@end{%
    \@DrawLCDRow#1#2#3#4#5\addtocounter{@LCDdoty}{-1}%
    \def\@tempa{#6}\ifx\@tempa\@empty\else\@DrawLCDRows#6\@@end\fi}
%    \end{macrocode}
% \end{macro}
%
% \begin{macro}{\@DrawLCDchar}
% At first in |\@DrawLCDchar| the position of the upper left dot of the
% character is calculated. Then |\@DrawLCDRows| is called with the already
% expanded character matrix. The parameters are the position of the character
% in x- (|#1|) and y-direction (|#2|) and the character to draw (|#3|). Note:
% the positions of the characters are beginning with 0,0 in the lower left
% corner.
%    \begin{macrocode}
\newcommand*\@DrawLCDchar[3]{%
    \setcounter{@LCDdotx}{#1}\multiply\c@@LCDdotx 6\addtocounter{@LCDdotx}{2}%
    \setcounter{@LCDdoty}{#2}\multiply\c@@LCDdoty 10\addtocounter{@LCDdoty}{8}%
    \expandafter\expandafter\expandafter
    \@DrawLCDRows\csname @LCD@#3\endcsname\@@end}
%    \end{macrocode}
% \end{macro}
%
% \begin{macro}{\@DrawLCDchars}
% The macro |\@DrawLCDchars| draws all characters of a LCD-line. First spaces
% are skiped. In |\c@@LCDchrx| the characters x-position is counted (beginning
% with zero). |\@DrawLCDchars| is first called with the line positon (zero for
% the lowest line) as parameter |#1| and the whole line text. Here the first
% character becomes parameter |#2| and the rest parameter |#3|.
% |\@DrawLCDchars| is called recursively with the line positon and the rest of
% the line text until all characters are processed.
%    \begin{macrocode}
\def\@DrawLCDchars#1#2#3\@@end{%
    \def\@tempa{#2}\if\@tempa\space\else\@DrawLCDchar{\the\c@@LCDchrx}{#1}{#2}\fi
    \stepcounter{@LCDchrx}%
    \def\@tempa{#3}\ifx\@tempa\@empty\else\@DrawLCDchars{#1}#3\@@end\fi}
%    \end{macrocode}
% \end{macro}
%
% \begin{macro}{\@DrawLCDLine}
% In |\@DrawLCDLine| |\c@@LCDchrx| is set to zero and the initial call of
% |\@DrawLCDchars| is done, but only if |#2| is not empty. The parameters
% are the line position (|#1|) and the
% complete line text (|#2|).
%    \begin{macrocode}
\newcommand*\@DrawLCDLine[2]{%
    \def\@tempa{#2}\ifx\@tempa\@empty\else
        \setcounter{@LCDchrx}{0}%
        \@DrawLCDchars{#1}#2\@@end\fi}
%    \end{macrocode}
% \end{macro}
%
% ^^A---------------------------------------------
% \subsection{Help Macros}
% \begin{macro}{\if@textLCD}
% \begin{macro}{\@textLCDtrue}
% \begin{macro}{\@textLCDfalse}
% In |\textLCD| the picture height is reduced by the height of the upper
% border, otherwise there would be extra space over the actual line. But
% the height of the box for colored background must not be reduced.
% The boolean |\if@textLCD| is needed to distinguish if the background box
% has to be corrected or not.
%    \begin{macrocode}
\newif\if@textLCD
%    \end{macrocode}
% \end{macro}
% \end{macro}
% \end{macro}
%
% \begin{macro}{\@LCDstart}
% The macro |\@LCDstart| starts the picture environment, sets the colors,
% draws the frame if wanted and fixes the height of the bachground box
% if necessary.
%    \begin{macrocode}
\newcommand*\@LCDstart{\unitlength\LCDunitlength
    \begin{picture}(\the\@tempcntb.7,\the\@tempcnta.7)(0,\the\c@@LCDlower)
    \color{\@LCDfr}
    \if@textLCD\advance\@tempcnta 2\fi
    \put(0,0){\fboxsep\z@\colorbox{\@LCDbg}{\@LCDbox(\the\@tempcntb.7,\the\@tempcnta.7){}}}
    \color{\@LCDfg}
    \linethickness{0.7\unitlength}}
%    \end{macrocode}
% \end{macro}
%
% \begin{macro}{\@LCDend}
% The macro |\@LCDend| only ends the picture environment.
%    \begin{macrocode}
\newcommand*\@LCDend{\end{picture}}
%    \end{macrocode}
% \end{macro}
%
% \begin{macro}{\do@LCDspecials}
% The macro |\do@LCDspecials| sets new catcodes for the characters that
% normaly can't be typed dirctly.
%    \begin{macrocode}
\newcommand*\do@LCDspecials{%
    \catcode`\~=11 \catcode`\$=11 \catcode`\�=11 \catcode`\�=11
    \catcode`\�=11 \catcode`\�=11 \catcode`\�=11 \catcode`\�=11
    \catcode`\�=11 \catcode`\"=11 \catcode`\&=11 \catcode`\#=11
    \catcode`\^=11 \catcode`\_=11 \catcode`\%=11 \obeyspaces}
%    \end{macrocode}
% \end{macro}
%
% \begin{macro}{\calc@LCDsize}
% In |\calc@LCDsize| only the size of the picture environment is calculated.
%    \begin{macrocode}
\newcommand*\calc@LCDsize[2]{\@tempcnta#1\multiply\@tempcnta10\relax
    \@tempcntb#2\multiply\@tempcntb6\advance\@tempcntb2\relax}
%    \end{macrocode}
% \end{macro}
%
% ^^A---------------------------------------------
% \subsection{The Main Macros}
% \begin{macro}{\textLCDcorr}
% The macro |\textLCDcorr| provides the ability to change the default of the
% first parameter of |\textLCD| to the user. Here it is defined to be the value
% needed for white background.
%    \begin{macrocode}
\newcommand*\textLCDcorr{-2}
%    \end{macrocode}
% \end{macro}
%
% \begin{macro}{\textLCD}
% Everything in |\textLCD| is done within a group (ended in |\@textLCD|), so
% changes to registers, macros and catcodes are local. The parameters |#1| and
% |#2| are described in \ref{sec:mac}. Parameter |#3| takes the first
% \meta{delimchar}, needed to define |\@textLCD|.
%
% First, some setups are done here and |\LCDunitlength| is calculated so that
% the height of \param{LCDtext} fits to the surrounding text.
%    \begin{macrocode}
\newcommand*\textLCD[3][\textLCDcorr]{\begingroup
    \LCDnoframe\settoheight{\LCDunitlength}{M}%
    \setlength{\LCDunitlength}{0.146342\LCDunitlength}%
    \setcounter{@LCDlower}{2}\setcounter{@LCDlines}{1}%
%    \end{macrocode}
% The first |\hspace| corrects the space before \param{LCDtext}, which will
% always be to large without it. The second and the one at the end of
% |\@textLCD| are needed to shrink the spaces around \param{LCDtext} if the
% background color is white.
%    \begin{macrocode}
    \hspace{-.25em}\hspace{#1\LCDunitlength}%
%    \end{macrocode}
% After calculating the picture size its height is reduced to prevent wrong
% linespacing.
%    \begin{macrocode}
    \calc@LCDsize{1}{#2}\advance\@tempcnta-2\@textLCDtrue\do@LCDspecials
%    \end{macrocode}
% \begin{macro}{\@textLCD}
% Now |\@textLCD| can be defined so that the \meta{delimchar} (|#3|) denotes
% the end of its parameter which is taken as \param{LCDtext}. The macro is
% called at the very end of |\textLCD|, so its parameter begins with the first
% character after the leading \meta{delimchar}.
%    \begin{macrocode}
    \def\@textLCD##1#3{\@LCDstart
        \@DrawLCDLine{0}{##1}\@LCDend\hspace{#1\LCDunitlength}\endgroup}
    \@textLCD}
%    \end{macrocode}
% \end{macro}
% \end{macro}
%
% \begin{macro}{\LCD}
% Also in |\LCD| everything is done within a group (ended in |\@LCDlast|) to
% keep changes to registers, macros and catcodes local, which are done for
% setup.
%
% The parameters |#1| and |#2| are described in \ref{sec:mac}. Parameter |#3|
% takes the first \meta{delimchar}, needed to define |\@LCD| and |\@LCDlast|.
%    \begin{macrocode}
\newcommand*\LCD[3]{\begingroup
    \setcounter{@LCDlower}{0}\setcounter{@LCDlines}{#1}%
%    \end{macrocode}
% Here the number of lines given is tested and corrected if necessary (and
% other setups are done).
%    \begin{macrocode}
    \ifnum\c@@LCDlines<1\setcounter{@LCDlines}{1}\fi
    \calc@LCDsize{\c@@LCDlines}{#2}\@textLCDfalse\do@LCDspecials
%    \end{macrocode}
% \begin{macro}{\@LCDlast}
% Now |\@LCDlast| is defined so that the \meta{delimchar} (|#3|) denotes the
% end of its parameter which is taken as \param{LCDtext}. It only handles the
% last line.
%    \begin{macrocode}
    \def\@LCDlast##1#3{\@DrawLCDLine{0}{##1}\@LCDend\endgroup}
%    \end{macrocode}
% \begin{macro}{\@LCD}
% The macro |\@LCD| handles all lines expect the last. The number of lines left
% to do are counted down here. Parameter |##1| is taken as \param{LCDtext},
% parameter |##2| takes everything between the ending \meta{delimchar} of an
% LCD-line and the leading one of the next. Because |##2| is not used, things
% between two LCD-lines are ignored. Note: due to the fact that |%| is catcoded
% to 11 when calling |\@LCD|, comments are not possible.
%    \begin{macrocode}
    \def\@LCD##1#3##2#3{%
        \addtocounter{@LCDlines}{-1}\@DrawLCDLine{\c@@LCDlines}{##1}
%    \end{macrocode}
% \begin{macro}{\@LCDnext}
% If there is more then one line to process |\@LCDnext| is set to |\@LCD|,
% else to |\@LCDlast|.
%    \begin{macrocode}
       \ifnum\c@@LCDlines>1\let\@LCDnext\@LCD\else
            \let\@LCDnext\@LCDlast\fi\@LCDnext}
%    \end{macrocode}
% \begin{macro}{\@LCDnext}
% Here |\@LCDnext| is set the first time. It's called at the very end of |\LCD|,
% so the (first) parameter of |\@LCD| or |\@LCDlast| begins with the first
% character after the leading \meta{delimchar}.
%    \begin{macrocode}
   \ifnum\c@@LCDlines>1\let\@LCDnext\@LCD\else\let\@LCDnext\@LCDlast\fi
    \@LCDstart\@LCDnext}
%    \end{macrocode}
% \end{macro}
% \end{macro}
% \end{macro}
% \end{macro}
% \end{macro}
%
% \Finale\endinput
%</package>
% ^^A Now the source of the example
%<*example>
\documentclass[a4paper]{article}
\usepackage[latin1]{inputenc}
\usepackage{color}
\usepackage{lcd}

\parindent0pt
\parskip1ex plus.3ex minus.2ex
\pagestyle{empty}

\newcommand\lcd{\textLCD{3}|LCD|}
\newcommand\bs{\char '134 }
\definecolor{lightgreen}{rgb}{0.05,0.97,0.55}
\definecolor{darkgreen}{rgb}{0.22,0.26,0.19}
\definecolor{lightblue}{rgb}{0.9,0.91,0.99}
\definecolor{darkblue}{rgb}{0.14,0.2,0.66}
\definecolor{lightred}{rgb}{1.0,0.27,0.37}
\definecolor{darkred}{rgb}{0.37,0.14,0.18}

\DefineLCDchar{euro}{00111010001111101000111110100000111}

\begin{document}
\centerline{\textbf{\LARGE Some Examples for the \lcd\ package.\footnote{The
source of this example file is part of \texttt{lcd.dtx}.}}}

As seen in the headline and here, the \lcd\ package calculates the size for
LCD-text in normal text (\verb|\textLCD|) automaticly. It works for all
fontsizes:

\begin{center}
{\tiny MM M \lcd\ M MM tiny}\hfill{\Huge Huge MM M \lcd\ M MM}

{\scriptsize MM M \lcd\ M MM scriptsize}\hfill{\huge huge MM M \lcd\ M MM}

{\footnotesize MM M \lcd\ M MM footnotesize}\hfill{\LARGE LARGE MM M \lcd\ M MM}

{\small MM M \lcd\ M MM small}\hfill{\Large Large MM M \lcd\ M MM}

{\normalsize MM M \lcd\ M MM normalsize}\hfill{\large large MM M \lcd\ M MM}
\end{center}

Now let's have some colored
\LCDcolors{darkgreen}{lightgreen}\textLCD[0]{8}|LCD-text|.
Here first the colors where set with
\verb|\LCDcolors{darkgreen}{lightgreen}|\footnote{The color names where
defined with \texttt{\bs definecolor} from the \textsf{color} package in the
preamble.}
and then the LCD-text where done with \verb+\textLCD[0]{8}|LCD-text|+.
To invert the LCD, just exchange the
\LCDcolors{lightgreen}{darkgreen}\textLCD[0]{6}|colors|
(\verb|\LCDcolors{lightgreen}{darkgreen}|).

\begin{minipage}[t]{.5\textwidth}
Now some seperate LCD representations.  But first let's  change the colors
to some not as ugly. The LCD was generated with
\begin{verbatim}
\LCD{4}{18}|LCD representation|
           |made with the LCD |
           |package for LaTeX |
           |04.01.2004 {clock} 18:23|
\end{verbatim}
\end{minipage}
\hspace{\fill}
\begin{minipage}[t]{.46\textwidth}
\mbox{}

\LCDcolors{darkblue}{lightblue}%
\LCD{4}{18}|LCD representation|
           |made with the LCD |
           |package for LaTeX |
           |04.01.2004 {clock} 18:23|
\end{minipage}

The \verb|{clock}| is a so called multi-letter character. It generates the
clock symbol.

As you can see, there is a black colored frame around it. The frame color
can be changed with the optional first argument of \verb|\LCDcolors|
(\verb|\LCDcolors[red]|\ldots; left part of figure 1). And
with \verb|\LCDnoframe| you can disable frames (reenabled with
\verb|\LCDframe|; right part of figure 1).
Of course \verb|\LCD| works within a figure environment.

\begin{figure}[h]
\LCDcolors[red]{darkblue}{lightblue}%
\LCD{4}{18}|LCD representation|
           |made with the LCD |
           |package for LaTeX |
           |04.01.2004 {clock} 18:45|
\hspace{\fill}\LCDnoframe
\LCD{4}{18}|LCD representation|
           |made with the LCD |
           |package for LaTeX |
           |04.01.2004 {clock} 18:47|
\caption{Example with red colored frame and without frame}
\end{figure}

\LCDcolors[lightgreen]{lightred}{darkred}\LCDframe
\LCD{2}{36}|For more information please refer to|
           |the documentation!                  |
\end{document}
%</example>
