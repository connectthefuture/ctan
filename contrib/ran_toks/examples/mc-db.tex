\documentclass{article}
\usepackage[allowrandomize,nosolutions,
    forpaper,pointsonleft,noparttotals]{eqexam}
\usepackage{ran_toks}

\useLastAsSeed
% To reproduce the same pseudo-random sequence, you need to supply a seed
%\useThisSeed{1125676795}

\examNum{1}
\title[T\nExam]{Test \nExam}
\author{D. P. Story}
\subject[MC]{My Course}
\date{Spring \the\year}
\keywords{Test~\nExam, Section 001}

\university
{%
      THE UNIVERSITY OF AKRON\\
    Theoretical and Applied Mathematics
}
\email{dpstory@uakron.edu}

% Input the DB files used in the demo file
\useTheseDBs{db1,db2,db3,db4}


\begin{document}
\maketitle

\begin{exam}{Part1}

\begin{instructions}
Solve each without error. Passing is 100\%.

\medskip\noindent
This part demonstrates how to not only randomly pull problems (two from each of the DB files) at random, but
to randomize the order they are listed in the exam.
\end{instructions}

\ranToks{myExam}{%
    {\useRanTok[DB1-]{1}}
    {\useRanTok[DB1-]{2}}
    {\useRanTok[DB2-]{1}}
    {\useRanTok[DB2-]{2}}
    {\useRanTok[DB3-]{1}}
    {\useRanTok[DB3-]{2}}
    {\useRanTok[DB4-]{1}}
    {\useRanTok[DB4-]{2}}
}
\displayListRandomly{myExam}

\end{exam}

\begin{exam}{Part2}

\begin{instructions}
In this part, we take the same two problems from each of the four DB files, but do not randomize
the order of the questions.
\end{instructions}

\useRanTok[DB1-]{1}
\useRanTok[DB1-]{2}
\useRanTok[DB2-]{1}
\useRanTok[DB2-]{2}
\useRanTok[DB3-]{1}
\useRanTok[DB3-]{2}
\useRanTok[DB4-]{1}
\useRanTok[DB4-]{2}

\end{exam}

\end{document}


