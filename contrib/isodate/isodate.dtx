% \iffalse meta comment
% File: isodate.dtx Copyright (C) 2000--2010 Harald Harders
% \fi
%
% \iffalse
%
%<*driver>
\documentclass[UKenglish]{ltxdoc}
\usepackage{babel}
\IfFileExists{isodate.sty}{\usepackage[english,iso]{isodate}
 \let\isodateSTYfound\active}{\GenericWarning{isodate.dtx}{Package
 file isodate.sty not found (Documentation will be messed up!^^J^^A
 Generate isodate.sty by (La)TeXing isodate.ins, process
 isodate.dtx again)^^J}\stop}
\usepackage{array}
\usepackage{url}
\usepackage{longtable}
\usepackage{morefloats}
\GetFileInfo{isodate.sty}
\title{The \texttt{isodate} package\thanks{This file has version 
    \fileversion\ last revised \printdateTeX{\filedate}.}}
\author{Harald Harders\\\texttt{h.harders@tu-bs.de}}
\date{File Date \printdateTeX{\filedate}, Printed \today}
\newlength{\tempdima}%
\makeatletter
\renewenvironment{table}[1][]{%
  \@float{table}[#1]%
  \centering%
  \setlength{\tempdima}{\abovecaptionskip}%
  \setlength{\abovecaptionskip}{\belowcaptionskip}%
  \setlength{\belowcaptionskip}{\tempdima}%
  }{%
\end@float
}
\makeatother
\EnableCrossrefs
\CodelineIndex
\DoNotIndex{\def,\edef,\let,\newcommand,\newenvironment,\newcounter,%
  \gdef,\@ifundefined,\@empty,\@firstoftwo,\@secondoftwo,\@nil,%
  \,,\number,\renewcommand,\setboolean,\whiledo,\ifthenelse}
\DoNotIndex{\setcounter,\space,\MessageBreak,\DeclareRobustCommand,\equal}
\DoNotIndex{\csname,\endcsname,\ifx,\else,\fi,\if,\fi,\@tempswafalse}
\DoNotIndex{\@tempswatrue,\undefined,\if,\or,\and,\fi}
\DoNotIndex{\begingroup,\endgroup,\begin,\end,\CurrentOption}
\CodelineNumbered
\RecordChanges
\CheckSum{2493}
\begin{document}
 \DocInput{isodate.dtx}
\end{document}
%</driver>
% \fi
%
% \changes{2.30}{2010/01/03}{Add a month-year format}%
% \changes{2.30}{2010/01/03}{Move defintion of language-independent
%   formats into the main style file}%
% \changes{2.26}{2005/03/10}{Support different input formats
%   containing slashes}%
% \changes{2.23}{2004/11/19}{Avoid to use the \texttt{calc} package
%   since it causes problems with many other packages}%
% \changes{2.22}{2004/02/12}{Path changed according to new CTAN
%   structure}%
% \changes{2.22}{2004/02/12}{Makefile adapted for \TeX Live}%
% \changes{2.21}{2003/12/06}{Fix some bugs in date ranges when both
%   month and year are equal (several language)}%
% \changes{2.21}{2003/12/06}{Support to print date without year (in
%   all language-dependent commands \cs{iso@printmonthday@...} and
%   \cs{iso@printdate@...})}%
% \changes{2.20}{2003/12/06}{Avoid usage of \cs{filedate} and
%   \cs{fileversion}}
% \changes{2.01}{2000/08/24}{For the case that none of the packages
%   babel, german, and ngerman is loaded there is a new macro
%   \texttt{\textbackslash iso@languagename} that contains the name of
%   the last loaded language. If one of the packages is loaded it
%   contains the current language.}
% \changes{2.00}{2000/08/21}{Total reimplementation of the
%   package. The old package has renamed to isodateo.}
%
% \maketitle
%
% \begin{abstract}
% \noindent This package provides commands to switch between different 
% date formats (standard, ISO, numeric, \LaTeX\ package).
% They are used by the \cs{today} command, by the \cs{printdate} and
% \cs{printdateTeX} commands that print any date, and by the
% \cs{daterange} command that prints a date range.
% At the moment, this package supports German (old and new
% orthography, Austrian), British, US, Australian as well as New
% Zealand English,\footnote{In order to use Australian or New Zealand,
% you need a version of babel that supports the used language. It
% should be available, soon.}
% French, Italian, Danish, Swedish, and Norwegian.
%
% The idea for this package was taken from the akletter class. 
% \end{abstract}
%
% \tableofcontents
% 
% \section*{Acknowledgements}
%
% First of all I have to thank Axel Kielhorn who wrote the package
% \verb|akletter| which inspired me to write \verb|isodate|. The help
% of Heiko Oberdiek was necessary to handle characters in substrings
% which resulted in the package \verb|substr|. David Sanderson found
% the bug which disabled \verb|isodate| to work without
% \verb|babel|. He also helped me to improve the documentation and
% sent me a link to the ISO 8601 norm \cite{iso8601a}.
% Svend Tollak Munkejord has added the Norwegian language, Christian
% Schlauer has added Swedish, Philip Ratcliffe has added Italian.
%
% \section*{Requirements}
%
% The package |isodate| needs the package |substr.sty| which can be
% obtained from |CTAN:macros/latex/contrib/substr/|.
%
% \section{Commands}
% 
% \subsection{Switching the date output format}
%
% \DescribeMacro{\isodate}%
% \DescribeMacro{\numdate}%
% \DescribeMacro{\shortdate}%
% \DescribeMacro{\TeXdate}%
% \DescribeMacro{\origdate}%
% \changes{2.05}{2001/05/19}{Added an original format with a two digit year.}%
% \DescribeMacro{\shortorigdate}%
% \DescribeMacro{\Romandate}%
% \DescribeMacro{\romandate}%
% This package provides five commands to switch the output format of
% all commands that print dates (described later):
% \DescribeMacro{\shortRomandate}%
% \DescribeMacro{\shortromandate}%
% \medskip 
%
% \begin{longtable}[l]{@{}ll}
% \verb|\isodate|&date format described in ISO 8601 and DIN 5008
% \cite{iso8601a}\\
% &(yyyy-mm-dd)\\
% \verb|\numdate|&numeric date format with four digits of the year\\
% \verb|\shortdate|&short numeric date format with two digits of the
% year\\
% \verb|\TeXdate|&date format used for version description of 
% packages\\
% &(yyyy/mm/dd)\\
% \verb|\origdate|&original \LaTeX\ format\\
% \verb|\shortorigdate|&original \LaTeX\ format with two instead of four
% digits of\\
% &the year\\
% \verb|\Romandate|& As \cs{numdate} but with uppercase Roman numerals
% \\
% & for the month\\
% \verb|\romandate|& As \cs{numdate} but with lowercase Roman numerals
% \\
% & for the month\\
% \verb|\shortRomandate|& As \cs{shortdate} but with uppercase Roman numerals
% \\
% & for the month\\
% \verb|\shortromandate|& As \cs{shortdate} but with lowercase Roman numerals
% \\
% & for the month\\
% \end{longtable}
% \medskip 
%
% These commands \emph{do not} print any dates and they don't take an
% argument.
% They just switch the format for later usage of the date-printing
% commands \cs{today}, \cs{printdate}, \cs{printdateTeX}, and
% \cs{daterange}.
%
% The numeric and short numeric as well as the Roman numbered formats
% change their behaviour depending on the current language:
%
% \noindent\begin{tabular}{@{}llll@{}}
% German, nGerman&\verb|dd.\,mm.~yyyy|&resp.&\verb|dd.\,mm.\,yy|\\
% US English&\verb|mm/dd/yyyy|&resp.&\verb|mm/dd/yy|\\ 
% other languages&\verb|dd/mm/yyyy|&resp.&\verb|dd/mm/yy|
% \end{tabular}
%
% This package supports German (old and new rules, Austrian), US
% English, French, Danish, Italian, Swedish, and Norwegian.
% Switching the language by using \verb|\selectlanguage| does \emph{not}
% switch back to the original date format. So the current date format
% stays active when changing the language.
%
% The change of the date format works locally. So it is possible to
% change it locally inside a group; e.g.,
% \begin{verbatim}
%\today, {\origdate\today}, \today\end{verbatim}
% leads to `\today, {\origdate\today}, \today'.
%
% \DescribeMacro{\printyearoff}%
% \DescribeMacro{\printyearon}%
% By default, all formats print the day, month, and year.
% Sometimes, you may want to print the date without the year. 
% This can be reached by using the command \cs{printyearoff}.
% You can switch back with \cs{printyearon} or by using
% \cs{printyearoff} inside a group (e.g., an environment).
% To switch globally, preceed the command by \cs{global}.
% An example:
% \begin{verbatim}
%\today, {\printyearoff\today}, \today\end{verbatim}
% leads to `{\origdate\today, {\printyearoff\today}, \today}'.
%
% \DescribeMacro{\printdayoff}%
% \DescribeMacro{\printdayon}%
% Likewise you can switch on or off printing the day using
% \cs{printdayon} and \cs{printdayoff}.
% Note that you still have to provide complete dates in the
% \cs{printdate} command, described in Section~\ref{sec:anydate} below.
%
% \subsection{Printing today's date}
%
% \DescribeMacro{\today}%
% As usual, the command \cs{today} prints the date of today. Its
% appearance is influenced by the current date format
%
% \subsection{Printing any date}
% \label{sec:anydate}
%
% \DescribeMacro{\printdate}%
% The command \verb|\printdate{#1}| prints any date in the current
% format. The argument may be a date in German, British English, or
% ISO format, e.g.,
% \begin{verbatim}
%\printdate{24.12.2000}
%\printdate{24/12/2000}
%\printdate{2000-12-24}\end{verbatim}
%
% \DescribeMacro{\printdateTeX}%
% The command \verb|\printdateTeX{#1}| prints any date in the actual
% format. The argument must be in the \LaTeX\ format
% \verb|yyyy/mm/dd|, e.g.,
% \begin{verbatim}
%\printdateTeX{2000/12/24}\end{verbatim}
% This command is useful for printing version information stored in a
% macro. For example the version of this package is stored in
% the macro \verb|\filedate| (`\filedate'). To print it with the actual
% date format you can use the command \verb|\printdateTeX{\filedate}|
% which leads to e.g., `\printdateTeX{\filedate}' or
% {\origdate`\printdateTeX{\filedate}'}.
% Another possibility is to switch the input format to |tex| using
% \cs{dateinputformat}, described below.
%
% \subsection{Printing date ranges}
%
% \DescribeMacro{\daterange}%
% The command \verb|\daterange{#1}{#2}| prints a date range in the current
% format. The arguments may be a date in German, British English, or
% ISO format (see above). But there is a limitation: Both arguments
% must have the same input format.
%
% Depending on the language and date format, this commands leaves out
% some of the data. The simplest way to understand it is to watch some
% examples:
% \par\medskip\noindent
% \verb|{\isodate|{\isodate\\
% \verb|\daterange{1999-05-03}{1999-05-31}| $\longrightarrow$
%   \daterange{1999-05-03}{1999-05-31}\\
% \verb|\daterange{1999-05-03}{1999-11-03}| $\longrightarrow$
%   \daterange{1999-05-03}{1999-11-03}\\
% \verb|\daterange{1999-05-03}{2000-04-07}| $\longrightarrow$
%   \daterange{1999-05-03}{2000-04-07}\\
% \verb|}|}\\
% \verb|{\origdate|{\origdate\\
% \verb|\daterange{1999-05-03}{1999-05-31}| $\longrightarrow$
%   \daterange{1999-05-03}{1999-05-31}\\
% \verb|\daterange{1999-05-03}{1999-11-03}| $\longrightarrow$
%   \daterange{1999-05-03}{1999-11-03}\\
% \verb|\daterange{1999-05-03}{2000-04-07}| $\longrightarrow$
%   \daterange{1999-05-03}{2000-04-07}\\
% \verb|}|}
%
% \subsection{Changing the ISO format}
%
% \DescribeMacro{\isodash}%
% The ISO norm says that the date format is `yyyy-mm-dd' or
% `yyyymmdd' \cite{iso8601a}.
% By default I use the hyphen `-' as separator. You can change this
% using the \verb|\isodash|\footnote{The name `isodash' is a little
% bit confusing and was chosen due to my limited knowledge in
% English. It should be named `isoseparator' or `isosep'. But for 
% compatiblity reasons I will not change it.} command, e.g.,
% \begin{verbatim}
%\printdate{24/12/2000},
%\isodash{--}%
%\printdate{24/12/2000},
%\isodash{}%
%\printdate{24/12/2000}\end{verbatim}
% leads to `2000-12-24, 2000--12--24, 20001224'. Or for example
% \begin{verbatim}
%\isodash{$\cdot$}
%\printdate{24/12/2000}\end{verbatim}
% leads to `2000$\cdot$12$\cdot$24'.
%
%
% \subsection{Changing the original and short original format}
%
% \DescribeMacro{\isospacebeforeday}%
% \DescribeMacro{\isospacebeforemonth}%
% \DescribeMacro{\isospacebeforeyear}%
%
% By default, the original and short original format prints
% unbreakable spaces between the parts of the dates, e.g.,
% `|19~May~2001|'.
% If you want to allow breakable spaces or other characters, you can
% redefine the spaces using \cs{isospacebeforeday},
% \cs{isospacebeforemonth}, and \cs{isospacebeforeyear}:
% \begin{verbatim}
%\isospacebeforeyear{\ }\end{verbatim}
% leads to `|19~May\ 2001|'. Notice that the space is written as
% \verb*|\ | to ensure that the space is not getting lost under all
% circumstances.
%
% As the names imply, the spaces before the specified part (day, month,
% or year) is changed. For most formats, only \cs{isospacebeforemonth}
% and \cs{isospacebeforeyear} are relevant, while for US English,
% \cs{isospacebeforeday} and \cs{isospacebeforeyear} are used.
%
% This only effects the \verb|orig| and \verb|shortorig| formats.
%
%
% \subsection{Changing the short original format}
%
% \DescribeMacro{\shortyearsign}%
%
% The short original format normally prints the year with two digits, e.g.,
% `19th May 01'.
% Some people want to add an additional sign before the year, e.g.,
% `19th May '01'.
% This can be achieved by using the command \verb|\shortyearsign|, e.g.,
% \begin{verbatim}
%\printdate{24/12/2000},
%\shortyearsign{'}%
%\printdate{24/12/2000}\end{verbatim}
% leads to `24 december 00, 24 december '00' in English.
%
% This only effects the \verb|shortorig| format. The \verb|short| numerical
% format stays unchanged.
%
%
% \subsection{Changing the German format}
%
% \changes{2.03}{2001/05/04}{Allow change of spaces for German language}%
% The spacings for the numerical formats in the German language 
% (24.\,12.~2000 resp. 24.\,12.\,00) were taken from the Duden
% \cite{duden1996a} and are the defaults when using one of the
% German derivatives.
% Some people want to use different spacings. Thus from version 2.03 on it 
% is possible to change them.
% \DescribeMacro{\daymonthsepgerman}%
% \DescribeMacro{\monthyearsepgerman}%
% \DescribeMacro{\monthyearsepnodaygerman}%
% You can change the spacing between the day and the month using the 
% command \cs{daymonthsepgerman}. Using the command
% \cs{monthyearsepgerman} you can change the spacing between the
% month and the year for the long and the short format, e.g.,
% \begin{verbatim}
%\daymonthsepgerman{\quad}%
%\monthyearsepgerman{\qquad}{\quad}%
%{\numdate\printdate{24.12.2000}}, {\shortdate\printdate{24.12.2000}}\end{verbatim}
% leads to `24.\quad12.\qquad2000, 24.\quad12.\quad00'.
%
% The default values are `\cs{,}' for the separator between day and
% month resp.\ `\cs{,}' between month and year in the short format
% and `|~|' in the long format.
%
% Likewise, \cs{monthyearsepnodaygerman} is used for defining the
% spacing between the month and the year when printing the day is
% switched off (using \cs{printdayoff}).
% The arguments are the same as for \cs{monthyearsepgerman}.
% Default is no space for long and short format.
%
%
% \subsection{Changing the English format}
%
% \changes{2.28}{2005/04/15}{Introduce option cleanlook for English
%   date format}%
% By default, the English date format looks like `24th December
% 2000'.
% During the last years, a change has occured in many documents
% towards `24 December 2000'.
% This new format is called `clean look'.
% \DescribeMacro{\cleanlookdateon}%
% \DescribeMacro{\cleanlookdateoff}%
% Isodate's behaviour can be changed towards it using
% \cs{cleanlookdateon} and \cs{cleanlookdateoff}.
% This can also be done globally using the |cleanlook| package option.
%
% At the moment, the `clean look' functionality only affects British
% English.
% If this trend also counts for different languages, please tell it me
% that I can add support for them.
%
%
%
% \subsection{User defined month formatting}
%
% Internally, the formats using Roman numerals for the month are just
% links to the \cs{numdate} and \cs{shortdate} formats with a changed
% format for printing the month.
% Therefore, the command \cs{Romandate} calls \cs{numdate} by
% following sequence:
% \begin{verbatim}
%\numdate[Roman]%
%\isotwodigitdayfalse\end{verbatim}
% This tells \cs{numdate} to format the month using the \cs{Roman}
% command and to typeset the day without a leading zero if it is
% less than ten.
%
% You may do similar things, e.g.,
% \begin{verbatim}
%\numdate[Alph]\end{verbatim}
% prints the months with the command \cs{Alph}, `A', `B', \ldots{}
% The day is printed with two digits since every call of \cs{numdate}
% or \cs{shortdate} calls \cs{isotwodigitdaytrue} which switches
% printing the day with two digits on.
% This does not make any sence but may serve as example.
% If you want to enable days with one digit, append
% \cs{isotwodigitdaytrue}:
% \begin{verbatim}
%\numdate[Alph]%
%\isotwodigitdaytrue\end{verbatim}
%
% You may declare any command that typesets a counter that is given as
% its mandatory argument (e.g., \cs{alph}, \cs{Alph}, \cs{arabic},
% \ldots) in the optional argument of the \cs{numdate},
% \cs{shortdate}, \cs{isodate}, and \cs{TeXdate} commands, without the
% leading backslash.
% You can, of course, define own commands that do it.
% For instance,
% \begin{verbatim}
%{\def\boldnum#1{\textbf{\twodigitarabic{#1}}}%
%\numdate[boldnum]%
%\printdate{24.3.2000}}\end{verbatim}
% leas to
% `{\def\boldnum#1{\textbf{\twodigitarabic{#1}}}\ignorespaces
% \numdate[boldnum]\ignorespaces
% \printdate{24.3.2000}}'.
% \DescribeMacro{\twodigitarabic}
% Here, the \cs{twodigitarabic} command has been used that prints a
% positive number with at least two digits.\footnote{This command is
%   also used for the numerical date formats.}
%
% If you, for example want a numerical date format with the day and
% month printed with the `natural' number of digits rather than with
% two digits, you may do it as follows:
% \begin{verbatim}
%{\numdate[arabic]\isotwodigitdayfalse
%\printdate{1.2.2000}}\end{verbatim}
% which leads to `{\numdate[arabic]\isotwodigitdayfalse
% \printdate{1.2.2000}}'.
%
% Using one of the other date formats reset the numerical format to
% its standard settings with arabic numerals (with two digits), e.g.,
% \begin{verbatim}
%{\numdate[Alph]\printdate{6.12.2000};
%\isodate\printdate{6.12.2000};
%\numdate\printdate{6.12.2000}}\end{verbatim}
% leads to `{\numdate[Alph]\isotwodigitdayfalse\printdate{6.12.2000};
% \isodate\printdate{6.12.2000};
% \numdate\printdate{6.12.2000}}'.
%
% \subsection{Switching the date input format}
%
% \DescribeMacro{\dateinputformat}%
% As described above, the date can be given in different formats.
% For the German format |dd.mm.yyyy| and the ISO format |yyyy-mm-dd|,
% the input format is definite.
% But when using slashes to seperate the day, month, and year,
% different formats exist.
% British people use |dd/mm/yyyy|, American people use |mm/dd/yyyy|,
% while \TeX\ uses |yyyy/mm/dd| which in fact is an ISO format with
% slashes instead of dashes.
%
% By default, the British format is used.
% If the user wants to give the American or \TeX\ format as argument
% of the \cs{printdate} or \cs{daterange} commands, the macro
% \cs{dateinputformat} can be used to change the behaviour.
% This macro takes the name of the input format as single parameter,
% e.g., \cs{dateinputformat\{american\}}, for switching to American
% behaviour, e.i., |mm/dd/yyyy|.
% For example,
% \begin{verbatim}
%\numdate
%\selectlanguage{UKenglish}%
%\dateinputformat{american}%
%\printdate{12/31/2004}\end{verbatim}
% gives
% \begingroup
% \numdate
% \selectlanguage{UKenglish}\dateinputformat{american}\printdate{12/31/2004}.
% \endgroup
% In this example, \emph{input} format is American while the
% \emph{output} format is English.
%
% Valid arguments for the \cs{dateinputformat} command are |english|,
% |UKenglish|, |british|, |american|, |USenglish|, |tex|, |latex|,
% |TeX|, |LaTeX|.
% The meaning of most possibilities should be clear; |english| means
% British English.
%
% Beware that the input format may only be changed for the date format
% using slashes.
% Thus, you don't have to and are not allowed to specify input formats
% other than these described above.
% For example, \cs{dateinputformat\{german\}} is not
% allowed (and not necessary).
%
% \section{Calling the package}
%
% The package is called using the \verb|\usepackage| command:\\
% \verb|\usepackage[|option\verb|]{isodate}|. 
%
% The possible package options can be seen in table~\ref{tab:options}.
% %
% \begin{table}[!tbp]
% \centering
% \caption{Package options}
% \label{tab:options}
% \begin{minipage}{\linewidth}
% \begin{tabular}{ll}\hline
% option&function\\ \hline
% \verb|iso|&start with ISO date format\\
% \verb|num|&start with numeric date format with 4 digits of the year\\
% \verb|short|&start with numeric date format with 2 digits of the year\\
% \verb|TeX|&start with \LaTeX\ numeric date format (yyyy/mm/dd)\\
% \verb|orig|&start with normal \LaTeX\ date format
% (default\footnote{The original format is used as default in order
% to avoid a different document output by just including the package.})\\
% \verb|shortorig|&start with short normal \LaTeX\ date format (2 digits)\\
% \verb|Roman|&start with numeric date format (month in uppercase\\
%   & Roman numerals)\\
% \verb|roman|&start with numeric date format (month in lowercase\\
%   & Roman numerals)\\
% \verb|shortRoman|&start with short Roman format\\
% \verb|shortroman|&start with short roman format\\
% \hline
% \verb|american|&support American English date format\\
% \verb|austrian|&support Austrian date format\\
% \verb|british|&support British English date format\\
% \verb|danish|&support Danish date format\\
% \verb|english|&support British English date format\\
% \verb|french|&support French date format\\
% \verb|german|&support German date format\\
% \verb|naustrian|&support new Austrian date format\\
% \verb|ngerman|&support new German date format\\
% \verb|italian|&support Italian date format\\
% \verb|norsk|&support Norwegian date format\\
% \verb|norwegian|&support Norwegian date format\\
% \verb|swedish|&support Swedish date format\\
% \verb|UKenglish|&support British English date format\\
% \verb|USenglish|&support American English date format\\
% \hline
% \verb|inputenglish|& English date input format (default)\\ 
% \verb|inputbritish|& English date input format (default)\\
% \verb|inputUKenglish|& English date input format (default)\\
% \verb|inputamerican|& American date input format\\
% \verb|inputUSenglish|& American date input format\\
% \verb|inputtex|& \TeX\ date input format\\
% \verb|inputTeX|& \TeX\ date input format\\
% \verb|inputlatex|& \TeX\ date input format\\
% \verb|inputLaTeX|& \TeX\ date input format\\
% \hline
% \verb|cleanlook|&use `clean look' for English dates\\
% \verb|nocleanlook|&don't use `clean look' for English dates (default)\\
% \hline
% \verb|printdayon|&print complete date including the day (default)\\
% \verb|printdayoff|&omit the day in the date\\
% \hline
% \end{tabular}
% \end{minipage}
% \end{table}
%
% \emph{Be aware that at least one language option must be set when calling
% isodate.} The last language in the option list is the default language.
%
% The package isodate works well together with \verb|babel.sty|,
% \verb|german.sty|, or \verb|ngerman.sty|.
% It does not matter if \verb|isodate| is loaded before or after the
% used language package.
%
% It is also possible to use isodate without one of the language
% packages. Then it is not possible to switch between languages using 
% the \verb|\selectlanguage| command.\footnote{Yes, there is a way to
%   change the date language, but it is a little bit tricky:\\
% \texttt{\textbackslash makeatletter\\
% \textbackslash def\textbackslash iso@languagename\{german\}\%\\
% \textbackslash dategerman\%\\
% \textbackslash makeatother}} 
% Then the default language is the last one in the option list. If an
% error occurs when using isodate without one of the packages
% \verb|babel.sty|, \verb|german.sty|, and \verb|ngerman.sty| please
% run \verb|tstlang.tex| through latex and send the file
% \verb|tstlang.log| to the address \verb|h.harders@tu-bs.de|.
%
% If using isodate
% together with babel it can be useful to put the language options as 
% global options into the optional parameters of the
% \verb|\documentclass| command.
% Then automatically the available languages are the same for the text
% and the dates, and the default language is also the same.
% For example:
% \begin{verbatim}
%\documentclass[english,german]{article}
%\usepackage{babel}
%\usepackage[num]{isodate}\end{verbatim}
%
% The input format options specify the input format that is used at
% the begin of the document. 
% You don't have to define multiple options if you want to change the
% input format in the document using \cs{dateinputformat}.
% For example,
% \begin{verbatim}
%\documentclass[american,german,british]{article}
%\usepackage{babel}
%\usepackage[iso,inputamerican]{isodate}
%\begin{document}
%D \printdate{28.2.2000}\par  
%ISO \printdate{2000-2-28}\par
%US \printdate{2/28/2000}\par 
%\dateinputformat{british}UK \printdate{28/2/2000}\par
%\dateinputformat{tex}\TeX\ \printdate{2000/2/28}
%\end{document}\end{verbatim}
% works as expected.
%
% Beware that only the mentioned input formats are defined.
% For example, |inputgerman| does not exist because it is not
% necessary.
% 
%
% \section{Add new languages to the package}
%
% The easiest way to add new languages to the package is to copy one
% of the simple language files \verb|danish.idf| or \verb|french.idf|
% to the new language name, e.g., \verb|plattdeutsch.idf|, and change it
% as necessary.
%
% This new file can be used without changing \verb|isodate.sty| if you 
% use its name explicitly in the optional parameter of the
% \verb|\usepackage| command. If you have added support for a new
% language please mail me.
%
% \changes{2.04}{2001/05/17}{Added section for solvable problems.}
% \changes{2.10}{2003/10/13}{Removed section about solvable problems
%   since it was wrong.}
%
% \appendix
% 
% \section{Licence} 
% 
% Copyright 2000--2010 Harald Harders
%
% This program can be redistributed and/or modified under the terms
% of the LaTeX Project Public License Distributed from CTAN
% archives in directory macros/latex/base/lppl.txt; either
% version 1 of the License, or any later version.
% 
% \section{Known errors}
%
% \begin{itemize} 
% \item The \cs{printdate} and \cs{printdateTeX} commands are not
%   very good in checking the argument for correct syntax.
% \item The language definition files \verb|french.idf| and
% \verb|german.idf| are not yet commented.
% \item Isodate and draftcopy do not work together.
% \item Documentation of the code is partly poor.
% \end{itemize}
%
% \section{Planned features and changes}
%
% \begin{itemize}
% \item Add other languages.
%   Please help me with this topic. I don't know the date formats in
%   other languages.
% \item Format short given years to four digits and calculate
%   reasonable first and second digits.
% \end{itemize}
%
% \begin{thebibliography}{1}
% \bibitem{iso8601a}
%   International Standard: ISO~8601.
%   \newblock \url{http://www.iso.ch/markete/8601.pdf}, 1988-06-15.
% \bibitem{duden1996a}
%   DUDEN Band 1.
%   \newblock Die deutsche Rechtschreibung.
%   \newblock 21. Auf\/lage, Dudenverlag, Mannheim, Germany, 1996.
% \end{thebibliography}
%                                
% \StopEventually{\PrintChanges \PrintIndex}
%
%
% \section{The implementation}
%
% \subsection{Package file isodate.sty}
%
% Heading of the files:
%    \begin{macrocode}
%<isodate>\NeedsTeXFormat{LaTeX2e}
%<isodate>\ProvidesPackage{isodate}
%<danish>\ProvidesFile{danish.idf}
%<english>\ProvidesFile{english.idf}
%<french>\ProvidesFile{french.idf}
%<german>\ProvidesFile{german.idf}
%<italian>\ProvidesFile{italian.idf}
%<norsk>\ProvidesFile{norsk.idf}
%<swedish>\ProvidesFile{swedish.idf}
%<isodate>  [2010/01/03  v2.30  Print dates with different formats (HH)]
%<language>  [2010/01/03  v2.30  Language definitions for isodate package (HH)]
%    \end{macrocode}
% The package:
%    \begin{macrocode}
%<*isodate>
\RequirePackage{ifthen}
\IfFileExists{substr.sty}{\RequirePackage{substr}%
 }{\PackageError{isodate.sty}{Package file substr.sty not found}
   {This version of isodate.sty needs the package substr.sty.^^J%
     You can download it from
     CTAN:/macros/latex/contrib/substr/^^J%
     e.g., one CTAN node is ftp.dante.de.
     Install substr.sty into your TeX tree.}}
%    \end{macrocode}
% Declare the options for the default date format.
% \changes{2.05}{2001/05/19}{Execute options at the end of the package instead
%   of at the end of the preamble.}
% \changes{2.10}{2003/10/13}{Add month in Roman numerals}%
%    \begin{macrocode}
\DeclareOption{iso}{\AtEndOfPackage{\isodate}}
\DeclareOption{num}{\AtEndOfPackage{\numdate}}
\DeclareOption{short}{\AtEndOfPackage{\shortdate}}
\DeclareOption{TeX}{\AtEndOfPackage{\TeXdate}}
\DeclareOption{orig}{\AtEndOfPackage{\origdate}}
\DeclareOption{shortorig}{\AtEndOfPackage{\shortorigdate}}
\DeclareOption{Roman}{\AtEndOfPackage{\Romandate}}
\DeclareOption{roman}{\AtEndOfPackage{\romandate}}
\DeclareOption{shortRoman}{\AtEndOfPackage{\shortRomandate}}
\DeclareOption{shortroman}{\AtEndOfPackage{\shortromandate}}
\DeclareOption{cleanlook}{\AtEndOfPackage{\cleanlookdateon}}
\DeclareOption{nocleanlook}{\AtEndOfPackage{\cleanlookdateoff}}
%    \end{macrocode}
% \changes{2.30}{2010/01/03}{Add a month-year format}%
% Declare the options which decide wheather day is printed.
%    \begin{macrocode}
\DeclareOption{printdayoff}{\AtEndOfPackage{\printdayoff}}
\DeclareOption{printdayon}{\AtEndOfPackage{\printdayon}}
%    \end{macrocode}
% \changes{2.26}{2005/03/10}{Support different input formats
%   containing slashes}%
% Declare the options for the default date input format.
%    \begin{macrocode}
\DeclareOption{inputenglish}{\AtEndOfPackage{\dateinputformat{english}}}
\DeclareOption{inputbritish}{\AtEndOfPackage{\dateinputformat{english}}}
\DeclareOption{inputUKenglish}{\AtEndOfPackage{\dateinputformat{english}}}
\DeclareOption{inputamerican}{\AtEndOfPackage{\dateinputformat{american}}}
\DeclareOption{inputUSenglish}{\AtEndOfPackage{\dateinputformat{american}}}
\DeclareOption{inputtex}{\AtEndOfPackage{\dateinputformat{tex}}}
\DeclareOption{inputTeX}{\AtEndOfPackage{\dateinputformat{tex}}}
\DeclareOption{inputlatex}{\AtEndOfPackage{\dateinputformat{tex}}}
\DeclareOption{inputLaTeX}{\AtEndOfPackage{\dateinputformat{tex}}}
%    \end{macrocode}
% Declare the options for language support.
% \changes{2.07}{2003/07/29}{Add Swedish language}%
% \changes{2.20}{2003/12/06}{Add Australian and New Zealand}%
% \changes{2.24}{2005/02/17}{Add option frenchb}%
% \changes{2.26}{2005/03/10}{Add option british}%
%    \begin{macrocode}
\DeclareOption{american}{% \iffalse meta-comment
%
% Copyright 1989-2005 Johannes L. Braams and any individual authors
% listed elsewhere in this file.  All rights reserved.
%    2013-2017 Javier Bezos, Johannes L. Braams
% This file is part of the Babel system.
% --------------------------------------
% 
% It may be distributed and/or modified under the
% conditions of the LaTeX Project Public License, either version 1.3
% of this license or (at your option) any later version.
% The latest version of this license is in
%   http://www.latex-project.org/lppl.txt
% and version 1.3 or later is part of all distributions of LaTeX
% version 2003/12/01 or later.
% 
% This work has the LPPL maintenance status "maintained".
% 
% The Current Maintainer of this work is Javier Bezos.
% 
% The list of all files belonging to the Babel system is
% given in the file `manifest.bbl. See also `legal.bbl' for additional
% information.
% 
% The list of derived (unpacked) files belonging to the distribution
% and covered by LPPL is defined by the unpacking scripts (with
% extension .ins) which are part of the distribution.
% \fi
% \iffalse
%    Tell the \LaTeX\ system who we are and write an entry on the
%    transcript.
%<*dtx>
\ProvidesFile{english.dtx}
%</dtx>
%<english>\ProvidesLanguage{english}
%<american>\ProvidesLanguage{american}
%<usenglish>\ProvidesLanguage{USenglish}
%<british>\ProvidesLanguage{british}
%<ukenglish>\ProvidesLanguage{UKenglish}
%<australian>\ProvidesLanguage{australian}
%<newzealand>\ProvidesLanguage{newzealand}
%<canadian>\ProvidesLanguage{canadian}
%\fi
%\ProvidesFile{english.dtx}
        [2017/06/06 v3.3r English support from the babel system]
%\iffalse
%% File 'english.dtx'
%% Babel package for LaTeX version 2e
%% Copyright (C) 1989 - 2005
%%           by Johannes Braams, TeXniek
%%           2013-2017 Javier Bezos, Johannes Braams
%
%
%    This file is part of the babel system, it provides the source
%    code for the English language definition file.
%<*filedriver>
\documentclass{ltxdoc}
\newcommand*\TeXhax{\TeX hax}
\newcommand*\babel{\textsf{babel}}
\newcommand*\langvar{$\langle \mathit lang \rangle$}
\newcommand*\note[1]{}
\newcommand*\Lopt[1]{\textsf{#1}}
\newcommand*\file[1]{\texttt{#1}}
\begin{document}
 \DocInput{english.dtx}
\end{document}
%</filedriver>
%\fi
% \GetFileInfo{english.dtx}
%
% \changes{english-2.0a}{1990/04/02}{Added checking of format}
% \changes{english-2.1}{1990/04/24}{Reflect changes in babel 2.1}
% \changes{english-2.1a}{1990/05/14}{Incorporated Nico's comments}
% \changes{english-2.1b}{1990/05/14}{merged \file{USenglish.sty} into
%    this file}
% \changes{english-2.1c}{1990/05/22}{fixed typo in definition for
%    american language found by Werenfried Spit (nspit@fys.ruu.nl)}
% \changes{english-2.1d}{1990/07/16}{Fixed some typos}
% \changes{english-3.0}{1991/04/23}{Modified for babel 3.0}
% \changes{english-3.0a}{1991/05/29}{Removed bug found by van der Meer}
% \changes{english-3.0c}{1991/07/15}{Renamed \file{babel.sty} in
%    \file{babel.com}}
% \changes{english-3.1}{1991/11/05}{Rewrote parts of the code to use
%    the new features of babel version 3.1}
% \changes{english-3.3}{1994/02/08}{Update or \LaTeXe}
% \changes{english-3.3c}{1994/06/26}{Removed the use of \cs{filedate}
%    and moved the identification after the loading of
%    \file{babel.def}}
% \changes{english-3.3g}{1996/07/10}{Replaced \cs{undefined} with
%    \cs{@undefined} and \cs{empty} with \cs{@empty} for consistency
%    with \LaTeX} 
% \changes{english-3.3h}{1996/10/10}{Moved the definition of
%    \cs{atcatcode} right to the beginning.} 
% \changes{english-3.3q}{2017/01/10}{Added the proxy files for the
%    dialects}
%
%  \section{The English language}
%
%    The file \file{\filename}\footnote{The file described in this
%    section has version number \fileversion\ and was last revised on
%    \filedate.} defines all the language definition macros for the
%    English language as well as for the American and Australian
%    version of this language. For the Australian version the British
%    hyphenation patterns will be used, if available, for the Canadian
%    variant the American patterns are selected.
%
%    For this language currently no special definitions are needed or
%    available.
%
% \StopEventually{}
%
%    The macro |\LdfInit| takes care of preventing that this file is
%    loaded more than once, checking the category code of the
%    \texttt{@} sign, etc.
% \changes{english-3.3h}{1996/11/02}{Now use \cs{LdfInit} to perform
%    initial checks} 
%    \begin{macrocode}
%<*code>
\LdfInit\CurrentOption{date\CurrentOption}
%    \end{macrocode}
%
%    When this file is read as an option, i.e. by the |\usepackage|
%    command, \texttt{english} could be an `unknown' language in which
%    case we have to make it known.  So we check for the existence of
%    |\l@english| to see whether we have to do something here.
%
% \changes{english-3.0}{1991/04/23}{Now use \cs{adddialect} if
%    language undefined}
% \changes{english-3.0d}{1991/10/22}{removed use of \cs{@ifundefined}}
% \changes{english-3.3c}{1994/06/26}{Now use \cs{@nopatterns} to
%    produce the warning}
% \changes{english-3.3g}{1996/07/10}{Allow british as the name of the
%    UK patterns}
% \changes{english-3.3j}{2000/01/21}{Also allow american english
%    hyphenation patterns to be used for `english'}
%    We allow for the british english patterns to be loaded as either
%    `british', or `UKenglish'. When neither of those is
%    known we try to define |\l@english| as an alias for |\l@american|
%    or |\l@USenglish|.
% \changes{english-3.3k}{2001/02/07}{Added support for canadian}
% \changes{english-3.3n}{2004/06/12}{Added support for australian and
%    newzealand} 
%    \begin{macrocode}
\ifx\l@english\@undefined
  \ifx\l@UKenglish\@undefined
    \ifx\l@british\@undefined
      \ifx\l@american\@undefined
        \ifx\l@USenglish\@undefined
          \ifx\l@canadian\@undefined
            \ifx\l@australian\@undefined
              \ifx\l@newzealand\@undefined
                \@nopatterns{English}
                \adddialect\l@english0
              \else
                \let\l@english\l@newzealand
              \fi
            \else
              \let\l@english\l@australian
            \fi
          \else
            \let\l@english\l@canadian
          \fi
        \else
          \let\l@english\l@USenglish
        \fi
      \else
        \let\l@english\l@american
      \fi
    \else
      \let\l@english\l@british
    \fi 
  \else
    \let\l@english\l@UKenglish
  \fi
\fi
%    \end{macrocode}
%    Because we allow `british' to be used as the babel option we need
%    to make sure that it will be recognised by |\selectlanguage|. In
%    the code above we have made sure that |\l@english| was defined.
%    Now we want to make sure that |\l@british| and |\l@UKenglish| are
%    defined as well. When either of them is we make them equal to
%    each other, when neither is we fall back to the default,
%    |\l@english|. 
% \changes{english-3.3o}{2004/06/14}{Make sure that british patterns
%    are used if they were loaded}
%    \begin{macrocode}
\ifx\l@british\@undefined
  \ifx\l@UKenglish\@undefined
    \adddialect\l@british\l@english
    \adddialect\l@UKenglish\l@english
  \else
    \let\l@british\l@UKenglish
  \fi
\else
  \let\l@UKenglish\l@british
\fi
%    \end{macrocode}
%    `American' is a version of `English' which can have its own
%    hyphenation patterns. The default english patterns are in fact
%    for american english. We allow for the patterns to be loaded as
%    `english' `american' or `USenglish'.
% \changes{english-3.0}{1990/04/23}{Now use \cs{adddialect} for
%    american}
% \changes{english-3.0b}{1991/06/06}{Removed \cs{global} definitions}
% \changes{english-3.3d}{1995/02/01}{Only define american as a
%    dialect when no separate patterns have been loaded}
% \changes{english-3.3g}{1996/07/10}{Allow USenglish as the name of
%    the american patterns} 
%    \begin{macrocode}
\ifx\l@american\@undefined
  \ifx\l@USenglish\@undefined
%    \end{macrocode}
%    When the patterns are not know as `american' or `USenglish' we
%    add a ``dialect''.
%    \begin{macrocode}
    \adddialect\l@american\l@english
  \else
    \let\l@american\l@USenglish
  \fi
\else
%    \end{macrocode}
%    Make sure that USenglish is known, even if the patterns were
%    loaded as `american'.
% \changes{english-3.3j}{2000/01/21}{Ensure that \cs{l@USenglish} is
%    alway defined}
% \changes{english-3.3l}{2001/04/15}{Added missing backslash}
%    \begin{macrocode}
  \ifx\l@USenglish\@undefined
    \let\l@USenglish\l@american
  \fi
\fi
%    \end{macrocode}
%
% \changes{english-3.3k}{2001/02/07}{Added support for canadian}
%    `Canadian' english spelling is a hybrid of British and American
%    spelling. Although so far no special `translations' have been
%    reported we allow this file to be loaded by the option
%    \Lopt{candian} as well.
%    \begin{macrocode}
\ifx\l@canadian\@undefined
  \adddialect\l@canadian\l@american
\fi
%    \end{macrocode}
%
% \changes{english-3.3n}{2004/06/12}{Added support for australian and
%   newzealand}
%    `Australian' and `New Zealand' english spelling seem to be the
%    same as British spelling. Although so far no special
%    `translations' have been reported we allow this file to be loaded
%    by the options \Lopt{australian} and \Lopt{newzealand} as well.
%    \begin{macrocode}
\ifx\l@australian\@undefined
  \adddialect\l@australian\l@british
\fi
\ifx\l@newzealand\@undefined
  \adddialect\l@newzealand\l@british
\fi
%    \end{macrocode}
%
 
%  \begin{macro}{\englishhyphenmins}
% \changes{english-3.3m}{2003/11/17}{Added default for setting of
%    hyphenmin parameters} 
%    This macro is used to store the correct values of the hyphenation
%    parameters |\lefthyphenmin| and |\righthyphenmin|.
%    \begin{macrocode}
\providehyphenmins{\CurrentOption}{\tw@\thr@@}
%    \end{macrocode}
%  \end{macro}
%
%    The next step consists of defining commands to switch to (and
%    from) the English language.
% \begin{macro}{\captionsenglish}
%    The macro |\captionsenglish| defines all strings used
%    in the four standard document classes provided with \LaTeX.
% \changes{english-3.0b}{1991/06/06}{Removed \cs{global} definitions}
% \changes{english-3.0b}{1991/06/06}{\cs{pagename} should be
%    \cs{headpagename}}
% \changes{english-3.1a}{1991/11/11}{added \cs{seename} and
%    \cs{alsoname}}
% \changes{english-3.1b}{1992/01/26}{added \cs{prefacename}}
% \changes{english-3.2}{1993/07/15}{\cs{headpagename} should be
%    \cs{pagename}}
% \changes{english-3.3e}{1995/07/04}{Added \cs{proofname} for
%    AMS-\LaTeX}
% \changes{english-3.3g}{1996/07/10}{Construct control sequence on the
%    fly} 
% \changes{english-3.3j}{2000/09/19}{Added \cs{glossaryname}}
%    \begin{macrocode}
\@namedef{captions\CurrentOption}{%
  \def\prefacename{Preface}%
  \def\refname{References}%
  \def\abstractname{Abstract}%
  \def\bibname{Bibliography}%
  \def\chaptername{Chapter}%
  \def\appendixname{Appendix}%
  \def\contentsname{Contents}%
  \def\listfigurename{List of Figures}%
  \def\listtablename{List of Tables}%
  \def\indexname{Index}%
  \def\figurename{Figure}%
  \def\tablename{Table}%
  \def\partname{Part}%
  \def\enclname{encl}%
  \def\ccname{cc}%
  \def\headtoname{To}%
  \def\pagename{Page}%
  \def\seename{see}%
  \def\alsoname{see also}%
  \def\proofname{Proof}%
  \def\glossaryname{Glossary}%
  }
%    \end{macrocode}
% \end{macro}
%
% \begin{macro}{\dateenglish}
%    In order to define |\today| correctly we need to know whether it
%    should be `english', `australian', or `american'. We can find
%    this out by checking the value of |\CurrentOption|.
% \changes{english-3.3j}{2000/01/21}{Make sure that the value of
%    \cs{today} is correct for both options `american' and
%    `USenglish'}
% \changes{english-3.3n}{2004/06/12}{Added support for `Australian'
%    and `Newzealand'}
% \changes{english-3.3o}{2004/06/14}{Explicitly choose the UK form of
%    date} 
% \changes{english-3.3p}{2012/11/07}{Warning if `english' is used with
%    other options} 
%    \begin{macrocode}
\def\bbl@tempa{british}
\ifx\CurrentOption\bbl@tempa\def\bbl@tempb{UK}\fi
\def\bbl@tempa{UKenglish}
\ifx\CurrentOption\bbl@tempa\def\bbl@tempb{UK}\fi
\def\bbl@tempa{american}
\ifx\CurrentOption\bbl@tempa\def\bbl@tempb{US}\fi
\def\bbl@tempa{USenglish}
\ifx\CurrentOption\bbl@tempa\def\bbl@tempb{US}\fi
\def\bbl@tempa{canadian}
\ifx\CurrentOption\bbl@tempa\def\bbl@tempb{US}\fi
\def\bbl@tempa{australian}
\ifx\CurrentOption\bbl@tempa\def\bbl@tempb{AU}\fi
\def\bbl@tempa{newzealand}
\ifx\CurrentOption\bbl@tempa\def\bbl@tempb{AU}\fi
\def\bbl@tempa{english}
\ifx\CurrentOption\bbl@tempa
  \AtEndOfPackage{\@nameuse{bbl@englishwarning}}
\else
  \edef\bbl@englishwarning{%
    \let\noexpand\bbl@englishwarning\relax
    \noexpand\PackageWarning{Babel}{%
      The package option `english' should not be used\noexpand\MessageBreak
      with a more specific one (like `\CurrentOption')}}
\fi
%    \end{macrocode}
%
%    The macro |\dateenglish| redefines the command |\today| to
%    produce English dates.
% \changes{english-3.0b}{1991/06/06}{Removed \cs{global} definitions}
% \changes{english-3.3g}{1996/07/10}{Construct control sequence on the
%    fly}
% \changes{english-3.3i}{1997/10/01}{Use \cs{edef} to define \cs{today}
%    to save memory}
% \changes{english-3.3i}{1998/03/28}{use \cs{def} instead of
%    \cs{edef}}
%    \begin{macrocode}
\def\bbl@tempa{UK}
\ifx\bbl@tempa\bbl@tempb
  \@namedef{date\CurrentOption}{%
    \def\today{\ifcase\day\or
      1st\or 2nd\or 3rd\or 4th\or 5th\or
      6th\or 7th\or 8th\or 9th\or 10th\or
      11th\or 12th\or 13th\or 14th\or 15th\or
      16th\or 17th\or 18th\or 19th\or 20th\or
      21st\or 22nd\or 23rd\or 24th\or 25th\or
      26th\or 27th\or 28th\or 29th\or 30th\or
      31st\fi~\ifcase\month\or
      January\or February\or March\or April\or May\or June\or
      July\or August\or September\or October\or November\or 
      December\fi\space \number\year}}
%    \end{macrocode}
% \end{macro}
%
% \begin{macro}{\dateaustralian}
%    Now, test for `australian' or `american'.
% \changes{english-3.3n}{2004/06/12}{Add australian date}
%    \begin{macrocode}
\else
%    \end{macrocode}
%
%    The macro |\dateaustralian| redefines the command |\today| to
%    produce Australian resp.\ New Zealand dates.
%    \begin{macrocode}
  \def\bbl@tempa{AU}
  \ifx\bbl@tempa\bbl@tempb
    \@namedef{date\CurrentOption}{%
      \def\today{\number\day~\ifcase\month\or
        January\or February\or March\or April\or May\or June\or
        July\or August\or September\or October\or November\or 
        December\fi\space \number\year}}
%    \end{macrocode}
% \end{macro}
%
% \begin{macro}{\dateamerican}
%    The macro |\dateamerican| redefines the command |\today| to
%    produce American dates.
% \changes{english-3.0b}{1991/06/06}{Removed \cs{global} definitions}
% \changes{english-3.3i}{1997/10/01}{Use \cs{edef} to define
%    \cs{today} to save memory}
% \changes{english-3.3i}{1998/03/28}{use \cs{def} instead of
%    \cs{edef}}
%    \begin{macrocode}
  \else
    \@namedef{date\CurrentOption}{%
      \def\today{\ifcase\month\or
        January\or February\or March\or April\or May\or June\or
        July\or August\or September\or October\or November\or
        December\fi \space\number\day, \number\year}}
  \fi
\fi
%    \end{macrocode}
% \end{macro}
%
% \begin{macro}{\extrasenglish}
% \begin{macro}{\noextrasenglish}
%    The macro |\extrasenglish| will perform all the extra definitions
%    needed for the English language. The macro |\noextrasenglish| is
%    used to cancel the actions of |\extrasenglish|.  For the moment
%    these macros are empty but they are defined for compatibility
%    with the other language definition files.
%
% \changes{english-3.3g}{1996/07/10}{Construct control sequences on
%    the fly} 
%    \begin{macrocode}
\@namedef{extras\CurrentOption}{}
\@namedef{noextras\CurrentOption}{}
%    \end{macrocode}
% \end{macro}
% \end{macro}
%
%    The macro |\ldf@finish| takes care of looking for a
%    configuration file, setting the main language to be switched on
%    at |\begin{document}| and resetting the category code of
%    \texttt{@} to its original value.
% \changes{english-3.3h}{1996/11/02}{Now use \cs{ldf@finish} to wrap
%    up} 
%    \begin{macrocode}
\ldf@finish\CurrentOption
%</code>
%    \end{macrocode}
%
% Finally, We create  a few proxy files, which just load english.ldf.
%
%    \begin{macrocode}
%<*american|usenglish|british|ukenglish|australian|newzealand|canadian>
\input english.ldf\relax
%</american|usenglish|british|ukenglish|australian|newzealand|canadian>
%    \end{macrocode}
%
% \Finale
%%
%% \CharacterTable
%%  {Upper-case    \A\B\C\D\E\F\G\H\I\J\K\L\M\N\O\P\Q\R\S\T\U\V\W\X\Y\Z
%%   Lower-case    \a\b\c\d\e\f\g\h\i\j\k\l\m\n\o\p\q\r\s\t\u\v\w\x\y\z
%%   Digits        \0\1\2\3\4\5\6\7\8\9
%%   Exclamation   \!     Double quote  \"     Hash (number) \#
%%   Dollar        \$     Percent       \%     Ampersand     \&
%%   Acute accent  \'     Left paren    \(     Right paren   \)
%%   Asterisk      \*     Plus          \+     Comma         \,
%%   Minus         \-     Point         \.     Solidus       \/
%%   Colon         \:     Semicolon     \;     Less than     \<
%%   Equals        \=     Greater than  \>     Question mark \?
%%   Commercial at \@     Left bracket  \[     Backslash     \\
%%   Right bracket \]     Circumflex    \^     Underscore    \_
%%   Grave accent  \`     Left brace    \{     Vertical bar  \|
%%   Right brace   \}     Tilde         \~}
%%
\endinput
}
\DeclareOption{australian}{% \iffalse meta-comment
%
% Copyright 1989-2005 Johannes L. Braams and any individual authors
% listed elsewhere in this file.  All rights reserved.
%    2013-2017 Javier Bezos, Johannes L. Braams
% This file is part of the Babel system.
% --------------------------------------
% 
% It may be distributed and/or modified under the
% conditions of the LaTeX Project Public License, either version 1.3
% of this license or (at your option) any later version.
% The latest version of this license is in
%   http://www.latex-project.org/lppl.txt
% and version 1.3 or later is part of all distributions of LaTeX
% version 2003/12/01 or later.
% 
% This work has the LPPL maintenance status "maintained".
% 
% The Current Maintainer of this work is Javier Bezos.
% 
% The list of all files belonging to the Babel system is
% given in the file `manifest.bbl. See also `legal.bbl' for additional
% information.
% 
% The list of derived (unpacked) files belonging to the distribution
% and covered by LPPL is defined by the unpacking scripts (with
% extension .ins) which are part of the distribution.
% \fi
% \iffalse
%    Tell the \LaTeX\ system who we are and write an entry on the
%    transcript.
%<*dtx>
\ProvidesFile{english.dtx}
%</dtx>
%<english>\ProvidesLanguage{english}
%<american>\ProvidesLanguage{american}
%<usenglish>\ProvidesLanguage{USenglish}
%<british>\ProvidesLanguage{british}
%<ukenglish>\ProvidesLanguage{UKenglish}
%<australian>\ProvidesLanguage{australian}
%<newzealand>\ProvidesLanguage{newzealand}
%<canadian>\ProvidesLanguage{canadian}
%\fi
%\ProvidesFile{english.dtx}
        [2017/06/06 v3.3r English support from the babel system]
%\iffalse
%% File 'english.dtx'
%% Babel package for LaTeX version 2e
%% Copyright (C) 1989 - 2005
%%           by Johannes Braams, TeXniek
%%           2013-2017 Javier Bezos, Johannes Braams
%
%
%    This file is part of the babel system, it provides the source
%    code for the English language definition file.
%<*filedriver>
\documentclass{ltxdoc}
\newcommand*\TeXhax{\TeX hax}
\newcommand*\babel{\textsf{babel}}
\newcommand*\langvar{$\langle \mathit lang \rangle$}
\newcommand*\note[1]{}
\newcommand*\Lopt[1]{\textsf{#1}}
\newcommand*\file[1]{\texttt{#1}}
\begin{document}
 \DocInput{english.dtx}
\end{document}
%</filedriver>
%\fi
% \GetFileInfo{english.dtx}
%
% \changes{english-2.0a}{1990/04/02}{Added checking of format}
% \changes{english-2.1}{1990/04/24}{Reflect changes in babel 2.1}
% \changes{english-2.1a}{1990/05/14}{Incorporated Nico's comments}
% \changes{english-2.1b}{1990/05/14}{merged \file{USenglish.sty} into
%    this file}
% \changes{english-2.1c}{1990/05/22}{fixed typo in definition for
%    american language found by Werenfried Spit (nspit@fys.ruu.nl)}
% \changes{english-2.1d}{1990/07/16}{Fixed some typos}
% \changes{english-3.0}{1991/04/23}{Modified for babel 3.0}
% \changes{english-3.0a}{1991/05/29}{Removed bug found by van der Meer}
% \changes{english-3.0c}{1991/07/15}{Renamed \file{babel.sty} in
%    \file{babel.com}}
% \changes{english-3.1}{1991/11/05}{Rewrote parts of the code to use
%    the new features of babel version 3.1}
% \changes{english-3.3}{1994/02/08}{Update or \LaTeXe}
% \changes{english-3.3c}{1994/06/26}{Removed the use of \cs{filedate}
%    and moved the identification after the loading of
%    \file{babel.def}}
% \changes{english-3.3g}{1996/07/10}{Replaced \cs{undefined} with
%    \cs{@undefined} and \cs{empty} with \cs{@empty} for consistency
%    with \LaTeX} 
% \changes{english-3.3h}{1996/10/10}{Moved the definition of
%    \cs{atcatcode} right to the beginning.} 
% \changes{english-3.3q}{2017/01/10}{Added the proxy files for the
%    dialects}
%
%  \section{The English language}
%
%    The file \file{\filename}\footnote{The file described in this
%    section has version number \fileversion\ and was last revised on
%    \filedate.} defines all the language definition macros for the
%    English language as well as for the American and Australian
%    version of this language. For the Australian version the British
%    hyphenation patterns will be used, if available, for the Canadian
%    variant the American patterns are selected.
%
%    For this language currently no special definitions are needed or
%    available.
%
% \StopEventually{}
%
%    The macro |\LdfInit| takes care of preventing that this file is
%    loaded more than once, checking the category code of the
%    \texttt{@} sign, etc.
% \changes{english-3.3h}{1996/11/02}{Now use \cs{LdfInit} to perform
%    initial checks} 
%    \begin{macrocode}
%<*code>
\LdfInit\CurrentOption{date\CurrentOption}
%    \end{macrocode}
%
%    When this file is read as an option, i.e. by the |\usepackage|
%    command, \texttt{english} could be an `unknown' language in which
%    case we have to make it known.  So we check for the existence of
%    |\l@english| to see whether we have to do something here.
%
% \changes{english-3.0}{1991/04/23}{Now use \cs{adddialect} if
%    language undefined}
% \changes{english-3.0d}{1991/10/22}{removed use of \cs{@ifundefined}}
% \changes{english-3.3c}{1994/06/26}{Now use \cs{@nopatterns} to
%    produce the warning}
% \changes{english-3.3g}{1996/07/10}{Allow british as the name of the
%    UK patterns}
% \changes{english-3.3j}{2000/01/21}{Also allow american english
%    hyphenation patterns to be used for `english'}
%    We allow for the british english patterns to be loaded as either
%    `british', or `UKenglish'. When neither of those is
%    known we try to define |\l@english| as an alias for |\l@american|
%    or |\l@USenglish|.
% \changes{english-3.3k}{2001/02/07}{Added support for canadian}
% \changes{english-3.3n}{2004/06/12}{Added support for australian and
%    newzealand} 
%    \begin{macrocode}
\ifx\l@english\@undefined
  \ifx\l@UKenglish\@undefined
    \ifx\l@british\@undefined
      \ifx\l@american\@undefined
        \ifx\l@USenglish\@undefined
          \ifx\l@canadian\@undefined
            \ifx\l@australian\@undefined
              \ifx\l@newzealand\@undefined
                \@nopatterns{English}
                \adddialect\l@english0
              \else
                \let\l@english\l@newzealand
              \fi
            \else
              \let\l@english\l@australian
            \fi
          \else
            \let\l@english\l@canadian
          \fi
        \else
          \let\l@english\l@USenglish
        \fi
      \else
        \let\l@english\l@american
      \fi
    \else
      \let\l@english\l@british
    \fi 
  \else
    \let\l@english\l@UKenglish
  \fi
\fi
%    \end{macrocode}
%    Because we allow `british' to be used as the babel option we need
%    to make sure that it will be recognised by |\selectlanguage|. In
%    the code above we have made sure that |\l@english| was defined.
%    Now we want to make sure that |\l@british| and |\l@UKenglish| are
%    defined as well. When either of them is we make them equal to
%    each other, when neither is we fall back to the default,
%    |\l@english|. 
% \changes{english-3.3o}{2004/06/14}{Make sure that british patterns
%    are used if they were loaded}
%    \begin{macrocode}
\ifx\l@british\@undefined
  \ifx\l@UKenglish\@undefined
    \adddialect\l@british\l@english
    \adddialect\l@UKenglish\l@english
  \else
    \let\l@british\l@UKenglish
  \fi
\else
  \let\l@UKenglish\l@british
\fi
%    \end{macrocode}
%    `American' is a version of `English' which can have its own
%    hyphenation patterns. The default english patterns are in fact
%    for american english. We allow for the patterns to be loaded as
%    `english' `american' or `USenglish'.
% \changes{english-3.0}{1990/04/23}{Now use \cs{adddialect} for
%    american}
% \changes{english-3.0b}{1991/06/06}{Removed \cs{global} definitions}
% \changes{english-3.3d}{1995/02/01}{Only define american as a
%    dialect when no separate patterns have been loaded}
% \changes{english-3.3g}{1996/07/10}{Allow USenglish as the name of
%    the american patterns} 
%    \begin{macrocode}
\ifx\l@american\@undefined
  \ifx\l@USenglish\@undefined
%    \end{macrocode}
%    When the patterns are not know as `american' or `USenglish' we
%    add a ``dialect''.
%    \begin{macrocode}
    \adddialect\l@american\l@english
  \else
    \let\l@american\l@USenglish
  \fi
\else
%    \end{macrocode}
%    Make sure that USenglish is known, even if the patterns were
%    loaded as `american'.
% \changes{english-3.3j}{2000/01/21}{Ensure that \cs{l@USenglish} is
%    alway defined}
% \changes{english-3.3l}{2001/04/15}{Added missing backslash}
%    \begin{macrocode}
  \ifx\l@USenglish\@undefined
    \let\l@USenglish\l@american
  \fi
\fi
%    \end{macrocode}
%
% \changes{english-3.3k}{2001/02/07}{Added support for canadian}
%    `Canadian' english spelling is a hybrid of British and American
%    spelling. Although so far no special `translations' have been
%    reported we allow this file to be loaded by the option
%    \Lopt{candian} as well.
%    \begin{macrocode}
\ifx\l@canadian\@undefined
  \adddialect\l@canadian\l@american
\fi
%    \end{macrocode}
%
% \changes{english-3.3n}{2004/06/12}{Added support for australian and
%   newzealand}
%    `Australian' and `New Zealand' english spelling seem to be the
%    same as British spelling. Although so far no special
%    `translations' have been reported we allow this file to be loaded
%    by the options \Lopt{australian} and \Lopt{newzealand} as well.
%    \begin{macrocode}
\ifx\l@australian\@undefined
  \adddialect\l@australian\l@british
\fi
\ifx\l@newzealand\@undefined
  \adddialect\l@newzealand\l@british
\fi
%    \end{macrocode}
%
 
%  \begin{macro}{\englishhyphenmins}
% \changes{english-3.3m}{2003/11/17}{Added default for setting of
%    hyphenmin parameters} 
%    This macro is used to store the correct values of the hyphenation
%    parameters |\lefthyphenmin| and |\righthyphenmin|.
%    \begin{macrocode}
\providehyphenmins{\CurrentOption}{\tw@\thr@@}
%    \end{macrocode}
%  \end{macro}
%
%    The next step consists of defining commands to switch to (and
%    from) the English language.
% \begin{macro}{\captionsenglish}
%    The macro |\captionsenglish| defines all strings used
%    in the four standard document classes provided with \LaTeX.
% \changes{english-3.0b}{1991/06/06}{Removed \cs{global} definitions}
% \changes{english-3.0b}{1991/06/06}{\cs{pagename} should be
%    \cs{headpagename}}
% \changes{english-3.1a}{1991/11/11}{added \cs{seename} and
%    \cs{alsoname}}
% \changes{english-3.1b}{1992/01/26}{added \cs{prefacename}}
% \changes{english-3.2}{1993/07/15}{\cs{headpagename} should be
%    \cs{pagename}}
% \changes{english-3.3e}{1995/07/04}{Added \cs{proofname} for
%    AMS-\LaTeX}
% \changes{english-3.3g}{1996/07/10}{Construct control sequence on the
%    fly} 
% \changes{english-3.3j}{2000/09/19}{Added \cs{glossaryname}}
%    \begin{macrocode}
\@namedef{captions\CurrentOption}{%
  \def\prefacename{Preface}%
  \def\refname{References}%
  \def\abstractname{Abstract}%
  \def\bibname{Bibliography}%
  \def\chaptername{Chapter}%
  \def\appendixname{Appendix}%
  \def\contentsname{Contents}%
  \def\listfigurename{List of Figures}%
  \def\listtablename{List of Tables}%
  \def\indexname{Index}%
  \def\figurename{Figure}%
  \def\tablename{Table}%
  \def\partname{Part}%
  \def\enclname{encl}%
  \def\ccname{cc}%
  \def\headtoname{To}%
  \def\pagename{Page}%
  \def\seename{see}%
  \def\alsoname{see also}%
  \def\proofname{Proof}%
  \def\glossaryname{Glossary}%
  }
%    \end{macrocode}
% \end{macro}
%
% \begin{macro}{\dateenglish}
%    In order to define |\today| correctly we need to know whether it
%    should be `english', `australian', or `american'. We can find
%    this out by checking the value of |\CurrentOption|.
% \changes{english-3.3j}{2000/01/21}{Make sure that the value of
%    \cs{today} is correct for both options `american' and
%    `USenglish'}
% \changes{english-3.3n}{2004/06/12}{Added support for `Australian'
%    and `Newzealand'}
% \changes{english-3.3o}{2004/06/14}{Explicitly choose the UK form of
%    date} 
% \changes{english-3.3p}{2012/11/07}{Warning if `english' is used with
%    other options} 
%    \begin{macrocode}
\def\bbl@tempa{british}
\ifx\CurrentOption\bbl@tempa\def\bbl@tempb{UK}\fi
\def\bbl@tempa{UKenglish}
\ifx\CurrentOption\bbl@tempa\def\bbl@tempb{UK}\fi
\def\bbl@tempa{american}
\ifx\CurrentOption\bbl@tempa\def\bbl@tempb{US}\fi
\def\bbl@tempa{USenglish}
\ifx\CurrentOption\bbl@tempa\def\bbl@tempb{US}\fi
\def\bbl@tempa{canadian}
\ifx\CurrentOption\bbl@tempa\def\bbl@tempb{US}\fi
\def\bbl@tempa{australian}
\ifx\CurrentOption\bbl@tempa\def\bbl@tempb{AU}\fi
\def\bbl@tempa{newzealand}
\ifx\CurrentOption\bbl@tempa\def\bbl@tempb{AU}\fi
\def\bbl@tempa{english}
\ifx\CurrentOption\bbl@tempa
  \AtEndOfPackage{\@nameuse{bbl@englishwarning}}
\else
  \edef\bbl@englishwarning{%
    \let\noexpand\bbl@englishwarning\relax
    \noexpand\PackageWarning{Babel}{%
      The package option `english' should not be used\noexpand\MessageBreak
      with a more specific one (like `\CurrentOption')}}
\fi
%    \end{macrocode}
%
%    The macro |\dateenglish| redefines the command |\today| to
%    produce English dates.
% \changes{english-3.0b}{1991/06/06}{Removed \cs{global} definitions}
% \changes{english-3.3g}{1996/07/10}{Construct control sequence on the
%    fly}
% \changes{english-3.3i}{1997/10/01}{Use \cs{edef} to define \cs{today}
%    to save memory}
% \changes{english-3.3i}{1998/03/28}{use \cs{def} instead of
%    \cs{edef}}
%    \begin{macrocode}
\def\bbl@tempa{UK}
\ifx\bbl@tempa\bbl@tempb
  \@namedef{date\CurrentOption}{%
    \def\today{\ifcase\day\or
      1st\or 2nd\or 3rd\or 4th\or 5th\or
      6th\or 7th\or 8th\or 9th\or 10th\or
      11th\or 12th\or 13th\or 14th\or 15th\or
      16th\or 17th\or 18th\or 19th\or 20th\or
      21st\or 22nd\or 23rd\or 24th\or 25th\or
      26th\or 27th\or 28th\or 29th\or 30th\or
      31st\fi~\ifcase\month\or
      January\or February\or March\or April\or May\or June\or
      July\or August\or September\or October\or November\or 
      December\fi\space \number\year}}
%    \end{macrocode}
% \end{macro}
%
% \begin{macro}{\dateaustralian}
%    Now, test for `australian' or `american'.
% \changes{english-3.3n}{2004/06/12}{Add australian date}
%    \begin{macrocode}
\else
%    \end{macrocode}
%
%    The macro |\dateaustralian| redefines the command |\today| to
%    produce Australian resp.\ New Zealand dates.
%    \begin{macrocode}
  \def\bbl@tempa{AU}
  \ifx\bbl@tempa\bbl@tempb
    \@namedef{date\CurrentOption}{%
      \def\today{\number\day~\ifcase\month\or
        January\or February\or March\or April\or May\or June\or
        July\or August\or September\or October\or November\or 
        December\fi\space \number\year}}
%    \end{macrocode}
% \end{macro}
%
% \begin{macro}{\dateamerican}
%    The macro |\dateamerican| redefines the command |\today| to
%    produce American dates.
% \changes{english-3.0b}{1991/06/06}{Removed \cs{global} definitions}
% \changes{english-3.3i}{1997/10/01}{Use \cs{edef} to define
%    \cs{today} to save memory}
% \changes{english-3.3i}{1998/03/28}{use \cs{def} instead of
%    \cs{edef}}
%    \begin{macrocode}
  \else
    \@namedef{date\CurrentOption}{%
      \def\today{\ifcase\month\or
        January\or February\or March\or April\or May\or June\or
        July\or August\or September\or October\or November\or
        December\fi \space\number\day, \number\year}}
  \fi
\fi
%    \end{macrocode}
% \end{macro}
%
% \begin{macro}{\extrasenglish}
% \begin{macro}{\noextrasenglish}
%    The macro |\extrasenglish| will perform all the extra definitions
%    needed for the English language. The macro |\noextrasenglish| is
%    used to cancel the actions of |\extrasenglish|.  For the moment
%    these macros are empty but they are defined for compatibility
%    with the other language definition files.
%
% \changes{english-3.3g}{1996/07/10}{Construct control sequences on
%    the fly} 
%    \begin{macrocode}
\@namedef{extras\CurrentOption}{}
\@namedef{noextras\CurrentOption}{}
%    \end{macrocode}
% \end{macro}
% \end{macro}
%
%    The macro |\ldf@finish| takes care of looking for a
%    configuration file, setting the main language to be switched on
%    at |\begin{document}| and resetting the category code of
%    \texttt{@} to its original value.
% \changes{english-3.3h}{1996/11/02}{Now use \cs{ldf@finish} to wrap
%    up} 
%    \begin{macrocode}
\ldf@finish\CurrentOption
%</code>
%    \end{macrocode}
%
% Finally, We create  a few proxy files, which just load english.ldf.
%
%    \begin{macrocode}
%<*american|usenglish|british|ukenglish|australian|newzealand|canadian>
\input english.ldf\relax
%</american|usenglish|british|ukenglish|australian|newzealand|canadian>
%    \end{macrocode}
%
% \Finale
%%
%% \CharacterTable
%%  {Upper-case    \A\B\C\D\E\F\G\H\I\J\K\L\M\N\O\P\Q\R\S\T\U\V\W\X\Y\Z
%%   Lower-case    \a\b\c\d\e\f\g\h\i\j\k\l\m\n\o\p\q\r\s\t\u\v\w\x\y\z
%%   Digits        \0\1\2\3\4\5\6\7\8\9
%%   Exclamation   \!     Double quote  \"     Hash (number) \#
%%   Dollar        \$     Percent       \%     Ampersand     \&
%%   Acute accent  \'     Left paren    \(     Right paren   \)
%%   Asterisk      \*     Plus          \+     Comma         \,
%%   Minus         \-     Point         \.     Solidus       \/
%%   Colon         \:     Semicolon     \;     Less than     \<
%%   Equals        \=     Greater than  \>     Question mark \?
%%   Commercial at \@     Left bracket  \[     Backslash     \\
%%   Right bracket \]     Circumflex    \^     Underscore    \_
%%   Grave accent  \`     Left brace    \{     Vertical bar  \|
%%   Right brace   \}     Tilde         \~}
%%
\endinput
}
\DeclareOption{austrian}{%%
%% This file will generate fast loadable files and documentation
%% driver files from the doc files in this package when run through
%% LaTeX or TeX.
%%
%% Copyright 1989--2016 Johannes L. Braams
%%                      Bernd Raichle
%%                      Walter Schmidt,
%%                      Juergen Spitzmueller
%% All rights reserved.
%% 
%% This file is part of the babel-german bundle,
%% an extension to the Babel system.
%% ----------------------------------------------
%% 
%% It may be distributed and/or modified under the
%% conditions of the LaTeX Project Public License, either version 1.3
%% of this license or (at your option) any later version.
%% The latest version of this license is in
%%   http://www.latex-project.org/lppl.txt
%% and version 1.3 or later is part of all distributions of LaTeX
%% version 2003/12/01 or later.
%% 
%% This work has the LPPL maintenance status "maintained".
%% 
%% The Current Maintainer of this work is Juergen Spitzmueller.
%%
%% --------------- start of docstrip commands ------------------
%%
\def\filedate{2016/11/01}

\input docstrip.tex

{\ifx\generate\undefined
\Msg{**********************************************}
\Msg{*}
\Msg{* This installation requires docstrip}
\Msg{* version 2.3c or later.}
\Msg{*}
\Msg{* An older version of docstrip has been input}
\Msg{*}
\Msg{**********************************************}
\errhelp{Move or rename old docstrip.tex.}
\errmessage{Old docstrip in input path}
\batchmode
\csname @@end\endcsname
\fi}

\preamble
This is a generated file.

Copyright 1989--2016 Johannes L. Braams
                     Bernd Raichle
                     Walter Schmidt,
                     Juergen Spitzmueller
All rights reserved.

This file is part of the babel-german bundle,
an extension to the Babel system.
----------------------------------------------

It may be distributed and/or modified under the
conditions of the LaTeX Project Public License, either version 1.3
of this license or (at your option) any later version.
The latest version of this license is in
  http://www.latex-project.org/lppl.txt
and version 1.3 or later is part of all distributions of LaTeX
version 2003/12/01 or later.

This work has the LPPL maintenance status "maintained".

The Current Maintainer of this work is Juergen Spitzmueller.

Please report errors to: Juergen Spitzmueller
                         juergen at spitzmueller dot org

\endpreamble

\keepsilent

\usedir{tex/generic/babel-german} 

\generate{\file{germanb.ldf}{\from{germanb.dtx}{germanb}}
          \file{german.ldf}{\from{germanb.dtx}{german}}
          \file{austrian.ldf}{\from{germanb.dtx}{austrian}}
          \file{swissgerman.ldf}{\from{germanb.dtx}{swiss}}
          \file{ngermanb.ldf}{\from{ngermanb.dtx}{germanb}}          
          \file{ngerman.ldf}{\from{ngermanb.dtx}{german}}
          \file{naustrian.ldf}{\from{ngermanb.dtx}{austrian}}
          \file{nswissgerman.ldf}{\from{ngermanb.dtx}{swiss}}
          }

\ifToplevel{
\Msg{***********************************************************}
\Msg{*}
\Msg{* To finish the installation you have to move the following}
\Msg{* files into a directory searched by TeX:}
\Msg{*}
\Msg{* \space\space austrian.ldf, german.ldf, germanb.ldf,}
\Msg{* \space\space naustrian.ldf, ngerman.ldf, ngermanb.ldf,}
\Msg{* \space\space nswissgerman.ldf and swissgerman.ldf}
\Msg{*}
\Msg{* To produce the documentation run the files }
\Msg{* germanb.dtx and ngermanb.dtx through LaTeX.}
\Msg{*}
\Msg{* Happy TeXing}
\Msg{***********************************************************}
}
 
\endbatchfile
}
\DeclareOption{danish}{% \iffalse meta-comment
%
% Copyright 1989-2009 Johannes L. Braams and any individual authors
% listed elsewhere in this file.  All rights reserved.
% 
% This file is part of the Babel system.
% --------------------------------------
% 
% It may be distributed and/or modified under the
% conditions of the LaTeX Project Public License, either version 1.3
% of this license or (at your option) any later version.
% The latest version of this license is in
%   http://www.latex-project.org/lppl.txt
% and version 1.3 or later is part of all distributions of LaTeX
% version 2003/12/01 or later.
% 
% This work has the LPPL maintenance status "maintained".
% 
% The Current Maintainer of this work is Johannes Braams.
% 
% The list of all files belonging to the Babel system is
% given in the file `manifest.bbl. See also `legal.bbl' for additional
% information.
% 
% The list of derived (unpacked) files belonging to the distribution
% and covered by LPPL is defined by the unpacking scripts (with
% extension .ins) which are part of the distribution.
% \fi
% \CheckSum{160}
% \iffalse
%    Tell the \LaTeX\ system who we are and write an entry on the
%    transcript.
%<*dtx>
\ProvidesFile{danish.dtx}
%</dtx>
%<code>\ProvidesLanguage{danish}
%\fi
%\ProvidesFile{danish.dtx}
        [2009/09/19 v1.3r Danish support from the babel system]
%\iffalse
%% File `danish.dtx'
%% Babel package for LaTeX version 2e
%% Copyright (C) 1989 - 2009
%%           by Johannes Braams, TeXniek
%
%% Please report errors to: J.L. Braams
%%                          babel at braams.xs4all.nl
%
%    This file is part of the babel system, it provides the source
%    code for the Danish language definition file.
%<*filedriver>
\documentclass{ltxdoc}
\newcommand*\TeXhax{\TeX hax}
\newcommand*\babel{\textsf{babel}}
\newcommand*\langvar{$\langle \it lang \rangle$}
\newcommand*\note[1]{}
\newcommand*\Lopt[1]{\textsf{#1}}
\newcommand*\file[1]{\texttt{#1}}
\begin{document}
 \DocInput{danish.dtx}
\end{document}
%</filedriver>
%    A contribution was made by Henning Larsen (larsen@cernvm.cern.ch)
%\fi
% \GetFileInfo{danish.dtx}
%
% \changes{danish-1.0a}{1991/07/15}{Renamed \file{babel.sty} in
%    \file{babel.com}}
% \changes{danish-1.1}{1992/02/15}{Brought up-to-date with babel 3.2a}
% \changes{danish-1.3}{1994/02/27}{Update for \LaTeXe}
% \changes{danish-1.3f}{1994/06/26}{Removed the use of \cs{filedate}
%    and moved identification after the loading of \file{babel.def}}
% \changes{danish-1.3g}{1995/06/08}{Added the active double quote
%    character as suggested by Peter Busk Laursen}
% \changes{danish-1.3j}{1996/10/10}{Replaced \cs{undefined} with
%    \cs{@undefined} and \cs{empty} with \cs{@empty} for consistency
%    with \LaTeX, moved the definition of \cs{atcatcode} right to the
%    beginning.}
%
%  \section{The Danish language}
%
%    The file \file{\filename}\footnote{The file described in this
%    section has version number \fileversion\ and was last revised on
%    \filedate.  A contribution was made by Henning Larsen
%    (\texttt{larsen@cernvm.cern.ch})} defines all the
%    language definition macros for the Danish language.
%
%    For this language the character |"| is made active. In
%    table~\ref{tab:danish-quote} an overview is given of its purpose.
%
%    \begin{table}[htb]
%     \centering
%     \begin{tabular}{lp{8cm}}
%       \verb="|= & disable ligature at this position.\\
%        |"-| & an explicit hyphen sign, allowing hyphenation
%               in the rest of the word.\\
%        |""| & like \verb="-=, but producing no hyphen sign (for
%              words that should break at some sign such as
%              ``entrada/salida.''\\
%        |"`| & lowered double left quotes (looks like ,,)\\
%        |"'| & normal double right quotes\\
%        |"<| & for French left double quotes (similar to $<<$).\\
%        |">| & for French right double quotes (similar to $>>$).\\
%        |\-| & like the old |\-|, but allowing hyphenation
%               in the rest of the word.
%     \end{tabular}
%     \caption{The extra definitions made by \file{danish.ldf}}
%     \label{tab:danish-quote}
%    \end{table}
%
% \StopEventually{}
%
%    The macro |\LdfInit| takes care of preventing that this file is
%    loaded more than once, checking the category code of the
%    \texttt{@} sign, etc.
% \changes{danish-1.3j}{1996/11/02}{Now use \cs{LdfInit} to perform
%    initial checks} 
%    \begin{macrocode}
%<*code>
\LdfInit{danish}\captionsdanish
%    \end{macrocode}
%
%    When this file is read as an option, i.e. by the |\usepackage|
%    command, \texttt{danish} will be an `unknown' language in which
%    case we have to make it known.  So we check for the existence of
%    |\l@danish| to see whether we have to do something here.
%
% \changes{danish-1.0b}{1991/10/27}{Removed use of \cs{@ifundefined}}
% \changes{danish-1.1}{1992/02/15}{Added a warning when no hyphenation
%    patterns were loaded.}
% \changes{danish-1.3f}{1994/06/26}{Now use \cs{@nopatterns} to
%    produce the warning}
%    \begin{macrocode}
\ifx\l@danish\@undefined
    \@nopatterns{Danish}
    \adddialect\l@danish0\fi
%    \end{macrocode}
%
%  \begin{macro}{\englishhyphenmins}
% \changes{danish-1.3p}{2003/11/17}{Added default for setting of
%    hyphenmin parameters}
%    This macro is used to store the correct values of the hyphenation
%    parameters |\lefthyphenmin| and |\righthyphenmin|.
% \changes{danish-1.3q}{2008/03/17}{Set lefthyphenmin to two}
%    \begin{macrocode}
\providehyphenmins{\CurrentOption}{\tw@\tw@}
%    \end{macrocode}
%  \end{macro}
%
%    The next step consists of defining commands to switch to (and
%    from) the Danish language.
%
% \begin{macro}{\captionsdanish}
%    The macro |\captionsdanish| defines all strings used in the four
%    standard documentclasses provided with \LaTeX.
% \changes{danish-1.1}{1992/02/15}{Added \cs{seename}, \cs{alsoname}
%    and \cs{prefacename}}
% \changes{danish-1.2}{1993/07/11}{\cs{headpagename} should be
%    \cs{pagename}}
% \changes{danish-1.2b}{1993/10/23}{Added a few translations}
% \changes{danish-1.3c}{1994/06/04}{Included some revisions from Peter
%    Busk Larsen}
% \changes{danish-1.3h}{1995/07/02}{Added \cs{proofname} for
%    AMS-\LaTeX}
% \changes{danish-1.3i}{1995/07/13}{Added translation of `Proof'}
% \changes{danish-1.3n}{2000/09/19}{Added \cs{glossaryname}}
% \changes{danish-1.3o}{2003/04/10}{Added translation of `Glossary'}
%    \begin{macrocode}
\addto\captionsdanish{%
  \def\prefacename{Forord}%
  \def\refname{Litteratur}%
  \def\abstractname{Resum\'e}%
  \def\bibname{Litteratur}%
  \def\chaptername{Kapitel}%
  \def\appendixname{Bilag}%
  \def\contentsname{Indhold}%
  \def\listfigurename{Figurer}%
  \def\listtablename{Tabeller}%
  \def\indexname{Indeks}%
  \def\figurename{Figur}%
  \def\tablename{Tabel}%
  \def\partname{Del}%
  \def\enclname{Vedlagt}%
  \def\ccname{Kopi til}%   or    Kopi sendt til
  \def\headtoname{Til}% in letter
  \def\pagename{Side}%
  \def\seename{Se}%
  \def\alsoname{Se ogs{\aa}}%
  \def\proofname{Bevis}%
  \def\glossaryname{Gloseliste}%
  }%
%    \end{macrocode}
% \end{macro}
%
% \begin{macro}{\datedanish}
%    The macro |\datedanish| redefines the command |\today| to produce
%    Danish dates.
% \changes{danish-1.3a}{1994/03/23}{Added `.' to definition of
%    \cs{today}}
% \changes{danish-1.3k}{1997/10/01}{Use \cs{edef} to define \cs{today}
%    to save memory}
% \changes{danish-1.3k}{1998/03/28}{use \cs{def} instead of \cs{edef}}
%    \begin{macrocode}
\def\datedanish{%
  \def\today{\number\day.~\ifcase\month\or
    januar\or februar\or marts\or april\or maj\or juni\or
    juli\or august\or september\or oktober\or november\or december\fi
    \space\number\year}}
%    \end{macrocode}
% \end{macro}
%
% \begin{macro}{\extrasdanish}
% \changes{danish-1.3h}{1995/07/02}{Added \cs{bbl@frenchspacing}}
% \begin{macro}{\noextrasdanish}
% \changes{danish-1.3h}{1995/07/02}{Added \cs{bbl@nonfrenchspacing}}
%    The macro |\extrasdanish| will perform all the extra definitions
%    needed for the Danish language. The macro |\noextrasdanish| is
%    used to cancel the actions of |\extrasdanish|.
%
%    Danish typesetting requires |\frenchspacing| to be in effect.
%    \begin{macrocode}
\addto\extrasdanish{\bbl@frenchspacing}
\addto\noextrasdanish{\bbl@nonfrenchspacing}
%    \end{macrocode}
% \end{macro}
%
%    For Danish the \texttt{"} character is made active. This is
%    done once, later on its definition may vary. Other languages in
%    the same document may also use the \texttt{"} character for
%    shorthands; we specify that the danish group of shorthands
%    should be used.
%
%    \begin{macrocode}
\initiate@active@char{"}
\addto\extrasdanish{\languageshorthands{danish}}
\addto\extrasdanish{\bbl@activate{"}}
%    \end{macrocode}
%    Don't forget to turn the shorthands off again.
% \changes{danish-1.3m}{1999/12/16}{Deactivate shorthands ouside of
%    Danish}
%    \begin{macrocode}
\addto\noextrasdanish{\bbl@deactivate{"}}
%    \end{macrocode}
%
%    First we define access to the low opening double quote and
%    guillemets for quotations,
% \changes{danish-1.3j}{1996/08/15}{Changed definition of \texttt{"'}
%    to print \texttt{``} instead of \texttt{''}} 
% \changes{danish-1.3k}{1997/04/03}{Removed empty groups after double
%    quote and guillemot characters}
%    \begin{macrocode}
\declare@shorthand{danish}{"`}{%
  \textormath{\quotedblbase}{\mbox{\quotedblbase}}}
\declare@shorthand{danish}{"'}{%
  \textormath{\textquotedblleft}{\mbox{\textquotedblleft}}}
\declare@shorthand{danish}{"<}{%
  \textormath{\guillemotleft}{\mbox{\guillemotleft}}}
\declare@shorthand{danish}{">}{%
  \textormath{\guillemotright}{\mbox{\guillemotright}}}
%    \end{macrocode}
%    then we define commands to be able to specify hyphenation
%    breakpoints that behave a little different from |\-|.
% \changes{danish-1.3j}{1996/08/15}{Added definition of
%    \texttt{"\char126} and \texttt{"\char61}} 
%    \begin{macrocode}
\declare@shorthand{danish}{"-}{\nobreak-\bbl@allowhyphens}
\declare@shorthand{danish}{""}{\hskip\z@skip}
\declare@shorthand{danish}{"~}{\textormath{\leavevmode\hbox{-}}{-}}
\declare@shorthand{danish}{"=}{\nobreak-\hskip\z@skip}
%    \end{macrocode}
%    And we want to have a shorthand for disabling a ligature.
%    \begin{macrocode}
\declare@shorthand{danish}{"|}{%
  \textormath{\discretionary{-}{}{\kern.03em}}{}}
%    \end{macrocode}
%
%    To enable hyphenation in two words, written together but
%    separated by a slash, as in `uitdrukking/opmerking' we define the
%    command |"/|.
% \changes{danish-1.3q}{2008/03/17}{Added definition of \texttt{"/}
%    from \texttt{dutch.ldf}} 
% \changes{danish-1.3r}{2009/09/19}{Made "/ a real Danish shorthand}
%    \begin{macrocode}
\declare@shorthand{danish}{"/}{\textormath
  {\bbl@allowhyphens\discretionary{/}{}{/}\bbl@allowhyphens}{}}
%    \end{macrocode}
%
%  \begin{macro}{\-}
% \changes{danish-1.3q}{2008/03/17}{Added redefinition of \cs{-} from
%    \texttt{dutch.ldf}}
%    All that is left now is the redefinition of |\-|. The new version
%    of |\-| should indicate an extra hyphenation position, while
%    allowing other hyphenation positions to be generated
%    automatically. The standard behaviour of \TeX\ in this respect is
%    very unfortunate for languages such as Dutch and German, where
%    long compound words are quite normal and all one needs is a means
%    to indicate an extra hyphenation position on top of the ones that
%    \TeX\ can generate from the hyphenation patterns.
%    \begin{macrocode}
\expandafter\addto\csname extras\CurrentOption\endcsname{%
  \babel@save\-}
\expandafter\addto\csname extras\CurrentOption\endcsname{%
  \def\-{\bbl@allowhyphens\discretionary{-}{}{}\bbl@allowhyphens}}
%    \end{macrocode}
%  \end{macro}
% \end{macro}
%
%    The macro |\ldf@finish| takes care of looking for a
%    configuration file, setting the main language to be switched on
%    at |\begin{document}| and resetting the category code of
%    \texttt{@} to its original value.
% \changes{danish-1.3j}{1996/11/02}{Now use \cs{ldf@finish} to wrap
%    up} 
%    \begin{macrocode}
\ldf@finish{danish}
%</code>
%    \end{macrocode}
%
% \Finale
%%
%% \CharacterTable
%%  {Upper-case    \A\B\C\D\E\F\G\H\I\J\K\L\M\N\O\P\Q\R\S\T\U\V\W\X\Y\Z
%%   Lower-case    \a\b\c\d\e\f\g\h\i\j\k\l\m\n\o\p\q\r\s\t\u\v\w\x\y\z
%%   Digits        \0\1\2\3\4\5\6\7\8\9
%%   Exclamation   \!     Double quote  \"     Hash (number) \#
%%   Dollar        \$     Percent       \%     Ampersand     \&
%%   Acute accent  \'     Left paren    \(     Right paren   \)
%%   Asterisk      \*     Plus          \+     Comma         \,
%%   Minus         \-     Point         \.     Solidus       \/
%%   Colon         \:     Semicolon     \;     Less than     \<
%%   Equals        \=     Greater than  \>     Question mark \?
%%   Commercial at \@     Left bracket  \[     Backslash     \\
%%   Right bracket \]     Circumflex    \^     Underscore    \_
%%   Grave accent  \`     Left brace    \{     Vertical bar  \|
%%   Right brace   \}     Tilde         \~}
%%
\endinput
}
\DeclareOption{english}{% \iffalse meta-comment
%
% Copyright 1989-2005 Johannes L. Braams and any individual authors
% listed elsewhere in this file.  All rights reserved.
%    2013-2017 Javier Bezos, Johannes L. Braams
% This file is part of the Babel system.
% --------------------------------------
% 
% It may be distributed and/or modified under the
% conditions of the LaTeX Project Public License, either version 1.3
% of this license or (at your option) any later version.
% The latest version of this license is in
%   http://www.latex-project.org/lppl.txt
% and version 1.3 or later is part of all distributions of LaTeX
% version 2003/12/01 or later.
% 
% This work has the LPPL maintenance status "maintained".
% 
% The Current Maintainer of this work is Javier Bezos.
% 
% The list of all files belonging to the Babel system is
% given in the file `manifest.bbl. See also `legal.bbl' for additional
% information.
% 
% The list of derived (unpacked) files belonging to the distribution
% and covered by LPPL is defined by the unpacking scripts (with
% extension .ins) which are part of the distribution.
% \fi
% \iffalse
%    Tell the \LaTeX\ system who we are and write an entry on the
%    transcript.
%<*dtx>
\ProvidesFile{english.dtx}
%</dtx>
%<english>\ProvidesLanguage{english}
%<american>\ProvidesLanguage{american}
%<usenglish>\ProvidesLanguage{USenglish}
%<british>\ProvidesLanguage{british}
%<ukenglish>\ProvidesLanguage{UKenglish}
%<australian>\ProvidesLanguage{australian}
%<newzealand>\ProvidesLanguage{newzealand}
%<canadian>\ProvidesLanguage{canadian}
%\fi
%\ProvidesFile{english.dtx}
        [2017/06/06 v3.3r English support from the babel system]
%\iffalse
%% File 'english.dtx'
%% Babel package for LaTeX version 2e
%% Copyright (C) 1989 - 2005
%%           by Johannes Braams, TeXniek
%%           2013-2017 Javier Bezos, Johannes Braams
%
%
%    This file is part of the babel system, it provides the source
%    code for the English language definition file.
%<*filedriver>
\documentclass{ltxdoc}
\newcommand*\TeXhax{\TeX hax}
\newcommand*\babel{\textsf{babel}}
\newcommand*\langvar{$\langle \mathit lang \rangle$}
\newcommand*\note[1]{}
\newcommand*\Lopt[1]{\textsf{#1}}
\newcommand*\file[1]{\texttt{#1}}
\begin{document}
 \DocInput{english.dtx}
\end{document}
%</filedriver>
%\fi
% \GetFileInfo{english.dtx}
%
% \changes{english-2.0a}{1990/04/02}{Added checking of format}
% \changes{english-2.1}{1990/04/24}{Reflect changes in babel 2.1}
% \changes{english-2.1a}{1990/05/14}{Incorporated Nico's comments}
% \changes{english-2.1b}{1990/05/14}{merged \file{USenglish.sty} into
%    this file}
% \changes{english-2.1c}{1990/05/22}{fixed typo in definition for
%    american language found by Werenfried Spit (nspit@fys.ruu.nl)}
% \changes{english-2.1d}{1990/07/16}{Fixed some typos}
% \changes{english-3.0}{1991/04/23}{Modified for babel 3.0}
% \changes{english-3.0a}{1991/05/29}{Removed bug found by van der Meer}
% \changes{english-3.0c}{1991/07/15}{Renamed \file{babel.sty} in
%    \file{babel.com}}
% \changes{english-3.1}{1991/11/05}{Rewrote parts of the code to use
%    the new features of babel version 3.1}
% \changes{english-3.3}{1994/02/08}{Update or \LaTeXe}
% \changes{english-3.3c}{1994/06/26}{Removed the use of \cs{filedate}
%    and moved the identification after the loading of
%    \file{babel.def}}
% \changes{english-3.3g}{1996/07/10}{Replaced \cs{undefined} with
%    \cs{@undefined} and \cs{empty} with \cs{@empty} for consistency
%    with \LaTeX} 
% \changes{english-3.3h}{1996/10/10}{Moved the definition of
%    \cs{atcatcode} right to the beginning.} 
% \changes{english-3.3q}{2017/01/10}{Added the proxy files for the
%    dialects}
%
%  \section{The English language}
%
%    The file \file{\filename}\footnote{The file described in this
%    section has version number \fileversion\ and was last revised on
%    \filedate.} defines all the language definition macros for the
%    English language as well as for the American and Australian
%    version of this language. For the Australian version the British
%    hyphenation patterns will be used, if available, for the Canadian
%    variant the American patterns are selected.
%
%    For this language currently no special definitions are needed or
%    available.
%
% \StopEventually{}
%
%    The macro |\LdfInit| takes care of preventing that this file is
%    loaded more than once, checking the category code of the
%    \texttt{@} sign, etc.
% \changes{english-3.3h}{1996/11/02}{Now use \cs{LdfInit} to perform
%    initial checks} 
%    \begin{macrocode}
%<*code>
\LdfInit\CurrentOption{date\CurrentOption}
%    \end{macrocode}
%
%    When this file is read as an option, i.e. by the |\usepackage|
%    command, \texttt{english} could be an `unknown' language in which
%    case we have to make it known.  So we check for the existence of
%    |\l@english| to see whether we have to do something here.
%
% \changes{english-3.0}{1991/04/23}{Now use \cs{adddialect} if
%    language undefined}
% \changes{english-3.0d}{1991/10/22}{removed use of \cs{@ifundefined}}
% \changes{english-3.3c}{1994/06/26}{Now use \cs{@nopatterns} to
%    produce the warning}
% \changes{english-3.3g}{1996/07/10}{Allow british as the name of the
%    UK patterns}
% \changes{english-3.3j}{2000/01/21}{Also allow american english
%    hyphenation patterns to be used for `english'}
%    We allow for the british english patterns to be loaded as either
%    `british', or `UKenglish'. When neither of those is
%    known we try to define |\l@english| as an alias for |\l@american|
%    or |\l@USenglish|.
% \changes{english-3.3k}{2001/02/07}{Added support for canadian}
% \changes{english-3.3n}{2004/06/12}{Added support for australian and
%    newzealand} 
%    \begin{macrocode}
\ifx\l@english\@undefined
  \ifx\l@UKenglish\@undefined
    \ifx\l@british\@undefined
      \ifx\l@american\@undefined
        \ifx\l@USenglish\@undefined
          \ifx\l@canadian\@undefined
            \ifx\l@australian\@undefined
              \ifx\l@newzealand\@undefined
                \@nopatterns{English}
                \adddialect\l@english0
              \else
                \let\l@english\l@newzealand
              \fi
            \else
              \let\l@english\l@australian
            \fi
          \else
            \let\l@english\l@canadian
          \fi
        \else
          \let\l@english\l@USenglish
        \fi
      \else
        \let\l@english\l@american
      \fi
    \else
      \let\l@english\l@british
    \fi 
  \else
    \let\l@english\l@UKenglish
  \fi
\fi
%    \end{macrocode}
%    Because we allow `british' to be used as the babel option we need
%    to make sure that it will be recognised by |\selectlanguage|. In
%    the code above we have made sure that |\l@english| was defined.
%    Now we want to make sure that |\l@british| and |\l@UKenglish| are
%    defined as well. When either of them is we make them equal to
%    each other, when neither is we fall back to the default,
%    |\l@english|. 
% \changes{english-3.3o}{2004/06/14}{Make sure that british patterns
%    are used if they were loaded}
%    \begin{macrocode}
\ifx\l@british\@undefined
  \ifx\l@UKenglish\@undefined
    \adddialect\l@british\l@english
    \adddialect\l@UKenglish\l@english
  \else
    \let\l@british\l@UKenglish
  \fi
\else
  \let\l@UKenglish\l@british
\fi
%    \end{macrocode}
%    `American' is a version of `English' which can have its own
%    hyphenation patterns. The default english patterns are in fact
%    for american english. We allow for the patterns to be loaded as
%    `english' `american' or `USenglish'.
% \changes{english-3.0}{1990/04/23}{Now use \cs{adddialect} for
%    american}
% \changes{english-3.0b}{1991/06/06}{Removed \cs{global} definitions}
% \changes{english-3.3d}{1995/02/01}{Only define american as a
%    dialect when no separate patterns have been loaded}
% \changes{english-3.3g}{1996/07/10}{Allow USenglish as the name of
%    the american patterns} 
%    \begin{macrocode}
\ifx\l@american\@undefined
  \ifx\l@USenglish\@undefined
%    \end{macrocode}
%    When the patterns are not know as `american' or `USenglish' we
%    add a ``dialect''.
%    \begin{macrocode}
    \adddialect\l@american\l@english
  \else
    \let\l@american\l@USenglish
  \fi
\else
%    \end{macrocode}
%    Make sure that USenglish is known, even if the patterns were
%    loaded as `american'.
% \changes{english-3.3j}{2000/01/21}{Ensure that \cs{l@USenglish} is
%    alway defined}
% \changes{english-3.3l}{2001/04/15}{Added missing backslash}
%    \begin{macrocode}
  \ifx\l@USenglish\@undefined
    \let\l@USenglish\l@american
  \fi
\fi
%    \end{macrocode}
%
% \changes{english-3.3k}{2001/02/07}{Added support for canadian}
%    `Canadian' english spelling is a hybrid of British and American
%    spelling. Although so far no special `translations' have been
%    reported we allow this file to be loaded by the option
%    \Lopt{candian} as well.
%    \begin{macrocode}
\ifx\l@canadian\@undefined
  \adddialect\l@canadian\l@american
\fi
%    \end{macrocode}
%
% \changes{english-3.3n}{2004/06/12}{Added support for australian and
%   newzealand}
%    `Australian' and `New Zealand' english spelling seem to be the
%    same as British spelling. Although so far no special
%    `translations' have been reported we allow this file to be loaded
%    by the options \Lopt{australian} and \Lopt{newzealand} as well.
%    \begin{macrocode}
\ifx\l@australian\@undefined
  \adddialect\l@australian\l@british
\fi
\ifx\l@newzealand\@undefined
  \adddialect\l@newzealand\l@british
\fi
%    \end{macrocode}
%
 
%  \begin{macro}{\englishhyphenmins}
% \changes{english-3.3m}{2003/11/17}{Added default for setting of
%    hyphenmin parameters} 
%    This macro is used to store the correct values of the hyphenation
%    parameters |\lefthyphenmin| and |\righthyphenmin|.
%    \begin{macrocode}
\providehyphenmins{\CurrentOption}{\tw@\thr@@}
%    \end{macrocode}
%  \end{macro}
%
%    The next step consists of defining commands to switch to (and
%    from) the English language.
% \begin{macro}{\captionsenglish}
%    The macro |\captionsenglish| defines all strings used
%    in the four standard document classes provided with \LaTeX.
% \changes{english-3.0b}{1991/06/06}{Removed \cs{global} definitions}
% \changes{english-3.0b}{1991/06/06}{\cs{pagename} should be
%    \cs{headpagename}}
% \changes{english-3.1a}{1991/11/11}{added \cs{seename} and
%    \cs{alsoname}}
% \changes{english-3.1b}{1992/01/26}{added \cs{prefacename}}
% \changes{english-3.2}{1993/07/15}{\cs{headpagename} should be
%    \cs{pagename}}
% \changes{english-3.3e}{1995/07/04}{Added \cs{proofname} for
%    AMS-\LaTeX}
% \changes{english-3.3g}{1996/07/10}{Construct control sequence on the
%    fly} 
% \changes{english-3.3j}{2000/09/19}{Added \cs{glossaryname}}
%    \begin{macrocode}
\@namedef{captions\CurrentOption}{%
  \def\prefacename{Preface}%
  \def\refname{References}%
  \def\abstractname{Abstract}%
  \def\bibname{Bibliography}%
  \def\chaptername{Chapter}%
  \def\appendixname{Appendix}%
  \def\contentsname{Contents}%
  \def\listfigurename{List of Figures}%
  \def\listtablename{List of Tables}%
  \def\indexname{Index}%
  \def\figurename{Figure}%
  \def\tablename{Table}%
  \def\partname{Part}%
  \def\enclname{encl}%
  \def\ccname{cc}%
  \def\headtoname{To}%
  \def\pagename{Page}%
  \def\seename{see}%
  \def\alsoname{see also}%
  \def\proofname{Proof}%
  \def\glossaryname{Glossary}%
  }
%    \end{macrocode}
% \end{macro}
%
% \begin{macro}{\dateenglish}
%    In order to define |\today| correctly we need to know whether it
%    should be `english', `australian', or `american'. We can find
%    this out by checking the value of |\CurrentOption|.
% \changes{english-3.3j}{2000/01/21}{Make sure that the value of
%    \cs{today} is correct for both options `american' and
%    `USenglish'}
% \changes{english-3.3n}{2004/06/12}{Added support for `Australian'
%    and `Newzealand'}
% \changes{english-3.3o}{2004/06/14}{Explicitly choose the UK form of
%    date} 
% \changes{english-3.3p}{2012/11/07}{Warning if `english' is used with
%    other options} 
%    \begin{macrocode}
\def\bbl@tempa{british}
\ifx\CurrentOption\bbl@tempa\def\bbl@tempb{UK}\fi
\def\bbl@tempa{UKenglish}
\ifx\CurrentOption\bbl@tempa\def\bbl@tempb{UK}\fi
\def\bbl@tempa{american}
\ifx\CurrentOption\bbl@tempa\def\bbl@tempb{US}\fi
\def\bbl@tempa{USenglish}
\ifx\CurrentOption\bbl@tempa\def\bbl@tempb{US}\fi
\def\bbl@tempa{canadian}
\ifx\CurrentOption\bbl@tempa\def\bbl@tempb{US}\fi
\def\bbl@tempa{australian}
\ifx\CurrentOption\bbl@tempa\def\bbl@tempb{AU}\fi
\def\bbl@tempa{newzealand}
\ifx\CurrentOption\bbl@tempa\def\bbl@tempb{AU}\fi
\def\bbl@tempa{english}
\ifx\CurrentOption\bbl@tempa
  \AtEndOfPackage{\@nameuse{bbl@englishwarning}}
\else
  \edef\bbl@englishwarning{%
    \let\noexpand\bbl@englishwarning\relax
    \noexpand\PackageWarning{Babel}{%
      The package option `english' should not be used\noexpand\MessageBreak
      with a more specific one (like `\CurrentOption')}}
\fi
%    \end{macrocode}
%
%    The macro |\dateenglish| redefines the command |\today| to
%    produce English dates.
% \changes{english-3.0b}{1991/06/06}{Removed \cs{global} definitions}
% \changes{english-3.3g}{1996/07/10}{Construct control sequence on the
%    fly}
% \changes{english-3.3i}{1997/10/01}{Use \cs{edef} to define \cs{today}
%    to save memory}
% \changes{english-3.3i}{1998/03/28}{use \cs{def} instead of
%    \cs{edef}}
%    \begin{macrocode}
\def\bbl@tempa{UK}
\ifx\bbl@tempa\bbl@tempb
  \@namedef{date\CurrentOption}{%
    \def\today{\ifcase\day\or
      1st\or 2nd\or 3rd\or 4th\or 5th\or
      6th\or 7th\or 8th\or 9th\or 10th\or
      11th\or 12th\or 13th\or 14th\or 15th\or
      16th\or 17th\or 18th\or 19th\or 20th\or
      21st\or 22nd\or 23rd\or 24th\or 25th\or
      26th\or 27th\or 28th\or 29th\or 30th\or
      31st\fi~\ifcase\month\or
      January\or February\or March\or April\or May\or June\or
      July\or August\or September\or October\or November\or 
      December\fi\space \number\year}}
%    \end{macrocode}
% \end{macro}
%
% \begin{macro}{\dateaustralian}
%    Now, test for `australian' or `american'.
% \changes{english-3.3n}{2004/06/12}{Add australian date}
%    \begin{macrocode}
\else
%    \end{macrocode}
%
%    The macro |\dateaustralian| redefines the command |\today| to
%    produce Australian resp.\ New Zealand dates.
%    \begin{macrocode}
  \def\bbl@tempa{AU}
  \ifx\bbl@tempa\bbl@tempb
    \@namedef{date\CurrentOption}{%
      \def\today{\number\day~\ifcase\month\or
        January\or February\or March\or April\or May\or June\or
        July\or August\or September\or October\or November\or 
        December\fi\space \number\year}}
%    \end{macrocode}
% \end{macro}
%
% \begin{macro}{\dateamerican}
%    The macro |\dateamerican| redefines the command |\today| to
%    produce American dates.
% \changes{english-3.0b}{1991/06/06}{Removed \cs{global} definitions}
% \changes{english-3.3i}{1997/10/01}{Use \cs{edef} to define
%    \cs{today} to save memory}
% \changes{english-3.3i}{1998/03/28}{use \cs{def} instead of
%    \cs{edef}}
%    \begin{macrocode}
  \else
    \@namedef{date\CurrentOption}{%
      \def\today{\ifcase\month\or
        January\or February\or March\or April\or May\or June\or
        July\or August\or September\or October\or November\or
        December\fi \space\number\day, \number\year}}
  \fi
\fi
%    \end{macrocode}
% \end{macro}
%
% \begin{macro}{\extrasenglish}
% \begin{macro}{\noextrasenglish}
%    The macro |\extrasenglish| will perform all the extra definitions
%    needed for the English language. The macro |\noextrasenglish| is
%    used to cancel the actions of |\extrasenglish|.  For the moment
%    these macros are empty but they are defined for compatibility
%    with the other language definition files.
%
% \changes{english-3.3g}{1996/07/10}{Construct control sequences on
%    the fly} 
%    \begin{macrocode}
\@namedef{extras\CurrentOption}{}
\@namedef{noextras\CurrentOption}{}
%    \end{macrocode}
% \end{macro}
% \end{macro}
%
%    The macro |\ldf@finish| takes care of looking for a
%    configuration file, setting the main language to be switched on
%    at |\begin{document}| and resetting the category code of
%    \texttt{@} to its original value.
% \changes{english-3.3h}{1996/11/02}{Now use \cs{ldf@finish} to wrap
%    up} 
%    \begin{macrocode}
\ldf@finish\CurrentOption
%</code>
%    \end{macrocode}
%
% Finally, We create  a few proxy files, which just load english.ldf.
%
%    \begin{macrocode}
%<*american|usenglish|british|ukenglish|australian|newzealand|canadian>
\input english.ldf\relax
%</american|usenglish|british|ukenglish|australian|newzealand|canadian>
%    \end{macrocode}
%
% \Finale
%%
%% \CharacterTable
%%  {Upper-case    \A\B\C\D\E\F\G\H\I\J\K\L\M\N\O\P\Q\R\S\T\U\V\W\X\Y\Z
%%   Lower-case    \a\b\c\d\e\f\g\h\i\j\k\l\m\n\o\p\q\r\s\t\u\v\w\x\y\z
%%   Digits        \0\1\2\3\4\5\6\7\8\9
%%   Exclamation   \!     Double quote  \"     Hash (number) \#
%%   Dollar        \$     Percent       \%     Ampersand     \&
%%   Acute accent  \'     Left paren    \(     Right paren   \)
%%   Asterisk      \*     Plus          \+     Comma         \,
%%   Minus         \-     Point         \.     Solidus       \/
%%   Colon         \:     Semicolon     \;     Less than     \<
%%   Equals        \=     Greater than  \>     Question mark \?
%%   Commercial at \@     Left bracket  \[     Backslash     \\
%%   Right bracket \]     Circumflex    \^     Underscore    \_
%%   Grave accent  \`     Left brace    \{     Vertical bar  \|
%%   Right brace   \}     Tilde         \~}
%%
\endinput
}
\DeclareOption{british}{% \iffalse meta-comment
%
% Copyright 1989-2005 Johannes L. Braams and any individual authors
% listed elsewhere in this file.  All rights reserved.
%    2013-2017 Javier Bezos, Johannes L. Braams
% This file is part of the Babel system.
% --------------------------------------
% 
% It may be distributed and/or modified under the
% conditions of the LaTeX Project Public License, either version 1.3
% of this license or (at your option) any later version.
% The latest version of this license is in
%   http://www.latex-project.org/lppl.txt
% and version 1.3 or later is part of all distributions of LaTeX
% version 2003/12/01 or later.
% 
% This work has the LPPL maintenance status "maintained".
% 
% The Current Maintainer of this work is Javier Bezos.
% 
% The list of all files belonging to the Babel system is
% given in the file `manifest.bbl. See also `legal.bbl' for additional
% information.
% 
% The list of derived (unpacked) files belonging to the distribution
% and covered by LPPL is defined by the unpacking scripts (with
% extension .ins) which are part of the distribution.
% \fi
% \iffalse
%    Tell the \LaTeX\ system who we are and write an entry on the
%    transcript.
%<*dtx>
\ProvidesFile{english.dtx}
%</dtx>
%<english>\ProvidesLanguage{english}
%<american>\ProvidesLanguage{american}
%<usenglish>\ProvidesLanguage{USenglish}
%<british>\ProvidesLanguage{british}
%<ukenglish>\ProvidesLanguage{UKenglish}
%<australian>\ProvidesLanguage{australian}
%<newzealand>\ProvidesLanguage{newzealand}
%<canadian>\ProvidesLanguage{canadian}
%\fi
%\ProvidesFile{english.dtx}
        [2017/06/06 v3.3r English support from the babel system]
%\iffalse
%% File 'english.dtx'
%% Babel package for LaTeX version 2e
%% Copyright (C) 1989 - 2005
%%           by Johannes Braams, TeXniek
%%           2013-2017 Javier Bezos, Johannes Braams
%
%
%    This file is part of the babel system, it provides the source
%    code for the English language definition file.
%<*filedriver>
\documentclass{ltxdoc}
\newcommand*\TeXhax{\TeX hax}
\newcommand*\babel{\textsf{babel}}
\newcommand*\langvar{$\langle \mathit lang \rangle$}
\newcommand*\note[1]{}
\newcommand*\Lopt[1]{\textsf{#1}}
\newcommand*\file[1]{\texttt{#1}}
\begin{document}
 \DocInput{english.dtx}
\end{document}
%</filedriver>
%\fi
% \GetFileInfo{english.dtx}
%
% \changes{english-2.0a}{1990/04/02}{Added checking of format}
% \changes{english-2.1}{1990/04/24}{Reflect changes in babel 2.1}
% \changes{english-2.1a}{1990/05/14}{Incorporated Nico's comments}
% \changes{english-2.1b}{1990/05/14}{merged \file{USenglish.sty} into
%    this file}
% \changes{english-2.1c}{1990/05/22}{fixed typo in definition for
%    american language found by Werenfried Spit (nspit@fys.ruu.nl)}
% \changes{english-2.1d}{1990/07/16}{Fixed some typos}
% \changes{english-3.0}{1991/04/23}{Modified for babel 3.0}
% \changes{english-3.0a}{1991/05/29}{Removed bug found by van der Meer}
% \changes{english-3.0c}{1991/07/15}{Renamed \file{babel.sty} in
%    \file{babel.com}}
% \changes{english-3.1}{1991/11/05}{Rewrote parts of the code to use
%    the new features of babel version 3.1}
% \changes{english-3.3}{1994/02/08}{Update or \LaTeXe}
% \changes{english-3.3c}{1994/06/26}{Removed the use of \cs{filedate}
%    and moved the identification after the loading of
%    \file{babel.def}}
% \changes{english-3.3g}{1996/07/10}{Replaced \cs{undefined} with
%    \cs{@undefined} and \cs{empty} with \cs{@empty} for consistency
%    with \LaTeX} 
% \changes{english-3.3h}{1996/10/10}{Moved the definition of
%    \cs{atcatcode} right to the beginning.} 
% \changes{english-3.3q}{2017/01/10}{Added the proxy files for the
%    dialects}
%
%  \section{The English language}
%
%    The file \file{\filename}\footnote{The file described in this
%    section has version number \fileversion\ and was last revised on
%    \filedate.} defines all the language definition macros for the
%    English language as well as for the American and Australian
%    version of this language. For the Australian version the British
%    hyphenation patterns will be used, if available, for the Canadian
%    variant the American patterns are selected.
%
%    For this language currently no special definitions are needed or
%    available.
%
% \StopEventually{}
%
%    The macro |\LdfInit| takes care of preventing that this file is
%    loaded more than once, checking the category code of the
%    \texttt{@} sign, etc.
% \changes{english-3.3h}{1996/11/02}{Now use \cs{LdfInit} to perform
%    initial checks} 
%    \begin{macrocode}
%<*code>
\LdfInit\CurrentOption{date\CurrentOption}
%    \end{macrocode}
%
%    When this file is read as an option, i.e. by the |\usepackage|
%    command, \texttt{english} could be an `unknown' language in which
%    case we have to make it known.  So we check for the existence of
%    |\l@english| to see whether we have to do something here.
%
% \changes{english-3.0}{1991/04/23}{Now use \cs{adddialect} if
%    language undefined}
% \changes{english-3.0d}{1991/10/22}{removed use of \cs{@ifundefined}}
% \changes{english-3.3c}{1994/06/26}{Now use \cs{@nopatterns} to
%    produce the warning}
% \changes{english-3.3g}{1996/07/10}{Allow british as the name of the
%    UK patterns}
% \changes{english-3.3j}{2000/01/21}{Also allow american english
%    hyphenation patterns to be used for `english'}
%    We allow for the british english patterns to be loaded as either
%    `british', or `UKenglish'. When neither of those is
%    known we try to define |\l@english| as an alias for |\l@american|
%    or |\l@USenglish|.
% \changes{english-3.3k}{2001/02/07}{Added support for canadian}
% \changes{english-3.3n}{2004/06/12}{Added support for australian and
%    newzealand} 
%    \begin{macrocode}
\ifx\l@english\@undefined
  \ifx\l@UKenglish\@undefined
    \ifx\l@british\@undefined
      \ifx\l@american\@undefined
        \ifx\l@USenglish\@undefined
          \ifx\l@canadian\@undefined
            \ifx\l@australian\@undefined
              \ifx\l@newzealand\@undefined
                \@nopatterns{English}
                \adddialect\l@english0
              \else
                \let\l@english\l@newzealand
              \fi
            \else
              \let\l@english\l@australian
            \fi
          \else
            \let\l@english\l@canadian
          \fi
        \else
          \let\l@english\l@USenglish
        \fi
      \else
        \let\l@english\l@american
      \fi
    \else
      \let\l@english\l@british
    \fi 
  \else
    \let\l@english\l@UKenglish
  \fi
\fi
%    \end{macrocode}
%    Because we allow `british' to be used as the babel option we need
%    to make sure that it will be recognised by |\selectlanguage|. In
%    the code above we have made sure that |\l@english| was defined.
%    Now we want to make sure that |\l@british| and |\l@UKenglish| are
%    defined as well. When either of them is we make them equal to
%    each other, when neither is we fall back to the default,
%    |\l@english|. 
% \changes{english-3.3o}{2004/06/14}{Make sure that british patterns
%    are used if they were loaded}
%    \begin{macrocode}
\ifx\l@british\@undefined
  \ifx\l@UKenglish\@undefined
    \adddialect\l@british\l@english
    \adddialect\l@UKenglish\l@english
  \else
    \let\l@british\l@UKenglish
  \fi
\else
  \let\l@UKenglish\l@british
\fi
%    \end{macrocode}
%    `American' is a version of `English' which can have its own
%    hyphenation patterns. The default english patterns are in fact
%    for american english. We allow for the patterns to be loaded as
%    `english' `american' or `USenglish'.
% \changes{english-3.0}{1990/04/23}{Now use \cs{adddialect} for
%    american}
% \changes{english-3.0b}{1991/06/06}{Removed \cs{global} definitions}
% \changes{english-3.3d}{1995/02/01}{Only define american as a
%    dialect when no separate patterns have been loaded}
% \changes{english-3.3g}{1996/07/10}{Allow USenglish as the name of
%    the american patterns} 
%    \begin{macrocode}
\ifx\l@american\@undefined
  \ifx\l@USenglish\@undefined
%    \end{macrocode}
%    When the patterns are not know as `american' or `USenglish' we
%    add a ``dialect''.
%    \begin{macrocode}
    \adddialect\l@american\l@english
  \else
    \let\l@american\l@USenglish
  \fi
\else
%    \end{macrocode}
%    Make sure that USenglish is known, even if the patterns were
%    loaded as `american'.
% \changes{english-3.3j}{2000/01/21}{Ensure that \cs{l@USenglish} is
%    alway defined}
% \changes{english-3.3l}{2001/04/15}{Added missing backslash}
%    \begin{macrocode}
  \ifx\l@USenglish\@undefined
    \let\l@USenglish\l@american
  \fi
\fi
%    \end{macrocode}
%
% \changes{english-3.3k}{2001/02/07}{Added support for canadian}
%    `Canadian' english spelling is a hybrid of British and American
%    spelling. Although so far no special `translations' have been
%    reported we allow this file to be loaded by the option
%    \Lopt{candian} as well.
%    \begin{macrocode}
\ifx\l@canadian\@undefined
  \adddialect\l@canadian\l@american
\fi
%    \end{macrocode}
%
% \changes{english-3.3n}{2004/06/12}{Added support for australian and
%   newzealand}
%    `Australian' and `New Zealand' english spelling seem to be the
%    same as British spelling. Although so far no special
%    `translations' have been reported we allow this file to be loaded
%    by the options \Lopt{australian} and \Lopt{newzealand} as well.
%    \begin{macrocode}
\ifx\l@australian\@undefined
  \adddialect\l@australian\l@british
\fi
\ifx\l@newzealand\@undefined
  \adddialect\l@newzealand\l@british
\fi
%    \end{macrocode}
%
 
%  \begin{macro}{\englishhyphenmins}
% \changes{english-3.3m}{2003/11/17}{Added default for setting of
%    hyphenmin parameters} 
%    This macro is used to store the correct values of the hyphenation
%    parameters |\lefthyphenmin| and |\righthyphenmin|.
%    \begin{macrocode}
\providehyphenmins{\CurrentOption}{\tw@\thr@@}
%    \end{macrocode}
%  \end{macro}
%
%    The next step consists of defining commands to switch to (and
%    from) the English language.
% \begin{macro}{\captionsenglish}
%    The macro |\captionsenglish| defines all strings used
%    in the four standard document classes provided with \LaTeX.
% \changes{english-3.0b}{1991/06/06}{Removed \cs{global} definitions}
% \changes{english-3.0b}{1991/06/06}{\cs{pagename} should be
%    \cs{headpagename}}
% \changes{english-3.1a}{1991/11/11}{added \cs{seename} and
%    \cs{alsoname}}
% \changes{english-3.1b}{1992/01/26}{added \cs{prefacename}}
% \changes{english-3.2}{1993/07/15}{\cs{headpagename} should be
%    \cs{pagename}}
% \changes{english-3.3e}{1995/07/04}{Added \cs{proofname} for
%    AMS-\LaTeX}
% \changes{english-3.3g}{1996/07/10}{Construct control sequence on the
%    fly} 
% \changes{english-3.3j}{2000/09/19}{Added \cs{glossaryname}}
%    \begin{macrocode}
\@namedef{captions\CurrentOption}{%
  \def\prefacename{Preface}%
  \def\refname{References}%
  \def\abstractname{Abstract}%
  \def\bibname{Bibliography}%
  \def\chaptername{Chapter}%
  \def\appendixname{Appendix}%
  \def\contentsname{Contents}%
  \def\listfigurename{List of Figures}%
  \def\listtablename{List of Tables}%
  \def\indexname{Index}%
  \def\figurename{Figure}%
  \def\tablename{Table}%
  \def\partname{Part}%
  \def\enclname{encl}%
  \def\ccname{cc}%
  \def\headtoname{To}%
  \def\pagename{Page}%
  \def\seename{see}%
  \def\alsoname{see also}%
  \def\proofname{Proof}%
  \def\glossaryname{Glossary}%
  }
%    \end{macrocode}
% \end{macro}
%
% \begin{macro}{\dateenglish}
%    In order to define |\today| correctly we need to know whether it
%    should be `english', `australian', or `american'. We can find
%    this out by checking the value of |\CurrentOption|.
% \changes{english-3.3j}{2000/01/21}{Make sure that the value of
%    \cs{today} is correct for both options `american' and
%    `USenglish'}
% \changes{english-3.3n}{2004/06/12}{Added support for `Australian'
%    and `Newzealand'}
% \changes{english-3.3o}{2004/06/14}{Explicitly choose the UK form of
%    date} 
% \changes{english-3.3p}{2012/11/07}{Warning if `english' is used with
%    other options} 
%    \begin{macrocode}
\def\bbl@tempa{british}
\ifx\CurrentOption\bbl@tempa\def\bbl@tempb{UK}\fi
\def\bbl@tempa{UKenglish}
\ifx\CurrentOption\bbl@tempa\def\bbl@tempb{UK}\fi
\def\bbl@tempa{american}
\ifx\CurrentOption\bbl@tempa\def\bbl@tempb{US}\fi
\def\bbl@tempa{USenglish}
\ifx\CurrentOption\bbl@tempa\def\bbl@tempb{US}\fi
\def\bbl@tempa{canadian}
\ifx\CurrentOption\bbl@tempa\def\bbl@tempb{US}\fi
\def\bbl@tempa{australian}
\ifx\CurrentOption\bbl@tempa\def\bbl@tempb{AU}\fi
\def\bbl@tempa{newzealand}
\ifx\CurrentOption\bbl@tempa\def\bbl@tempb{AU}\fi
\def\bbl@tempa{english}
\ifx\CurrentOption\bbl@tempa
  \AtEndOfPackage{\@nameuse{bbl@englishwarning}}
\else
  \edef\bbl@englishwarning{%
    \let\noexpand\bbl@englishwarning\relax
    \noexpand\PackageWarning{Babel}{%
      The package option `english' should not be used\noexpand\MessageBreak
      with a more specific one (like `\CurrentOption')}}
\fi
%    \end{macrocode}
%
%    The macro |\dateenglish| redefines the command |\today| to
%    produce English dates.
% \changes{english-3.0b}{1991/06/06}{Removed \cs{global} definitions}
% \changes{english-3.3g}{1996/07/10}{Construct control sequence on the
%    fly}
% \changes{english-3.3i}{1997/10/01}{Use \cs{edef} to define \cs{today}
%    to save memory}
% \changes{english-3.3i}{1998/03/28}{use \cs{def} instead of
%    \cs{edef}}
%    \begin{macrocode}
\def\bbl@tempa{UK}
\ifx\bbl@tempa\bbl@tempb
  \@namedef{date\CurrentOption}{%
    \def\today{\ifcase\day\or
      1st\or 2nd\or 3rd\or 4th\or 5th\or
      6th\or 7th\or 8th\or 9th\or 10th\or
      11th\or 12th\or 13th\or 14th\or 15th\or
      16th\or 17th\or 18th\or 19th\or 20th\or
      21st\or 22nd\or 23rd\or 24th\or 25th\or
      26th\or 27th\or 28th\or 29th\or 30th\or
      31st\fi~\ifcase\month\or
      January\or February\or March\or April\or May\or June\or
      July\or August\or September\or October\or November\or 
      December\fi\space \number\year}}
%    \end{macrocode}
% \end{macro}
%
% \begin{macro}{\dateaustralian}
%    Now, test for `australian' or `american'.
% \changes{english-3.3n}{2004/06/12}{Add australian date}
%    \begin{macrocode}
\else
%    \end{macrocode}
%
%    The macro |\dateaustralian| redefines the command |\today| to
%    produce Australian resp.\ New Zealand dates.
%    \begin{macrocode}
  \def\bbl@tempa{AU}
  \ifx\bbl@tempa\bbl@tempb
    \@namedef{date\CurrentOption}{%
      \def\today{\number\day~\ifcase\month\or
        January\or February\or March\or April\or May\or June\or
        July\or August\or September\or October\or November\or 
        December\fi\space \number\year}}
%    \end{macrocode}
% \end{macro}
%
% \begin{macro}{\dateamerican}
%    The macro |\dateamerican| redefines the command |\today| to
%    produce American dates.
% \changes{english-3.0b}{1991/06/06}{Removed \cs{global} definitions}
% \changes{english-3.3i}{1997/10/01}{Use \cs{edef} to define
%    \cs{today} to save memory}
% \changes{english-3.3i}{1998/03/28}{use \cs{def} instead of
%    \cs{edef}}
%    \begin{macrocode}
  \else
    \@namedef{date\CurrentOption}{%
      \def\today{\ifcase\month\or
        January\or February\or March\or April\or May\or June\or
        July\or August\or September\or October\or November\or
        December\fi \space\number\day, \number\year}}
  \fi
\fi
%    \end{macrocode}
% \end{macro}
%
% \begin{macro}{\extrasenglish}
% \begin{macro}{\noextrasenglish}
%    The macro |\extrasenglish| will perform all the extra definitions
%    needed for the English language. The macro |\noextrasenglish| is
%    used to cancel the actions of |\extrasenglish|.  For the moment
%    these macros are empty but they are defined for compatibility
%    with the other language definition files.
%
% \changes{english-3.3g}{1996/07/10}{Construct control sequences on
%    the fly} 
%    \begin{macrocode}
\@namedef{extras\CurrentOption}{}
\@namedef{noextras\CurrentOption}{}
%    \end{macrocode}
% \end{macro}
% \end{macro}
%
%    The macro |\ldf@finish| takes care of looking for a
%    configuration file, setting the main language to be switched on
%    at |\begin{document}| and resetting the category code of
%    \texttt{@} to its original value.
% \changes{english-3.3h}{1996/11/02}{Now use \cs{ldf@finish} to wrap
%    up} 
%    \begin{macrocode}
\ldf@finish\CurrentOption
%</code>
%    \end{macrocode}
%
% Finally, We create  a few proxy files, which just load english.ldf.
%
%    \begin{macrocode}
%<*american|usenglish|british|ukenglish|australian|newzealand|canadian>
\input english.ldf\relax
%</american|usenglish|british|ukenglish|australian|newzealand|canadian>
%    \end{macrocode}
%
% \Finale
%%
%% \CharacterTable
%%  {Upper-case    \A\B\C\D\E\F\G\H\I\J\K\L\M\N\O\P\Q\R\S\T\U\V\W\X\Y\Z
%%   Lower-case    \a\b\c\d\e\f\g\h\i\j\k\l\m\n\o\p\q\r\s\t\u\v\w\x\y\z
%%   Digits        \0\1\2\3\4\5\6\7\8\9
%%   Exclamation   \!     Double quote  \"     Hash (number) \#
%%   Dollar        \$     Percent       \%     Ampersand     \&
%%   Acute accent  \'     Left paren    \(     Right paren   \)
%%   Asterisk      \*     Plus          \+     Comma         \,
%%   Minus         \-     Point         \.     Solidus       \/
%%   Colon         \:     Semicolon     \;     Less than     \<
%%   Equals        \=     Greater than  \>     Question mark \?
%%   Commercial at \@     Left bracket  \[     Backslash     \\
%%   Right bracket \]     Circumflex    \^     Underscore    \_
%%   Grave accent  \`     Left brace    \{     Vertical bar  \|
%%   Right brace   \}     Tilde         \~}
%%
\endinput
}
\DeclareOption{french}{\input{french.idf}}
\DeclareOption{frenchb}{\input{french.idf}}
\DeclareOption{german}{%%
%% This file will generate fast loadable files and documentation
%% driver files from the doc files in this package when run through
%% LaTeX or TeX.
%%
%% Copyright 1989--2016 Johannes L. Braams
%%                      Bernd Raichle
%%                      Walter Schmidt,
%%                      Juergen Spitzmueller
%% All rights reserved.
%% 
%% This file is part of the babel-german bundle,
%% an extension to the Babel system.
%% ----------------------------------------------
%% 
%% It may be distributed and/or modified under the
%% conditions of the LaTeX Project Public License, either version 1.3
%% of this license or (at your option) any later version.
%% The latest version of this license is in
%%   http://www.latex-project.org/lppl.txt
%% and version 1.3 or later is part of all distributions of LaTeX
%% version 2003/12/01 or later.
%% 
%% This work has the LPPL maintenance status "maintained".
%% 
%% The Current Maintainer of this work is Juergen Spitzmueller.
%%
%% --------------- start of docstrip commands ------------------
%%
\def\filedate{2016/11/01}

\input docstrip.tex

{\ifx\generate\undefined
\Msg{**********************************************}
\Msg{*}
\Msg{* This installation requires docstrip}
\Msg{* version 2.3c or later.}
\Msg{*}
\Msg{* An older version of docstrip has been input}
\Msg{*}
\Msg{**********************************************}
\errhelp{Move or rename old docstrip.tex.}
\errmessage{Old docstrip in input path}
\batchmode
\csname @@end\endcsname
\fi}

\preamble
This is a generated file.

Copyright 1989--2016 Johannes L. Braams
                     Bernd Raichle
                     Walter Schmidt,
                     Juergen Spitzmueller
All rights reserved.

This file is part of the babel-german bundle,
an extension to the Babel system.
----------------------------------------------

It may be distributed and/or modified under the
conditions of the LaTeX Project Public License, either version 1.3
of this license or (at your option) any later version.
The latest version of this license is in
  http://www.latex-project.org/lppl.txt
and version 1.3 or later is part of all distributions of LaTeX
version 2003/12/01 or later.

This work has the LPPL maintenance status "maintained".

The Current Maintainer of this work is Juergen Spitzmueller.

Please report errors to: Juergen Spitzmueller
                         juergen at spitzmueller dot org

\endpreamble

\keepsilent

\usedir{tex/generic/babel-german} 

\generate{\file{germanb.ldf}{\from{germanb.dtx}{germanb}}
          \file{german.ldf}{\from{germanb.dtx}{german}}
          \file{austrian.ldf}{\from{germanb.dtx}{austrian}}
          \file{swissgerman.ldf}{\from{germanb.dtx}{swiss}}
          \file{ngermanb.ldf}{\from{ngermanb.dtx}{germanb}}          
          \file{ngerman.ldf}{\from{ngermanb.dtx}{german}}
          \file{naustrian.ldf}{\from{ngermanb.dtx}{austrian}}
          \file{nswissgerman.ldf}{\from{ngermanb.dtx}{swiss}}
          }

\ifToplevel{
\Msg{***********************************************************}
\Msg{*}
\Msg{* To finish the installation you have to move the following}
\Msg{* files into a directory searched by TeX:}
\Msg{*}
\Msg{* \space\space austrian.ldf, german.ldf, germanb.ldf,}
\Msg{* \space\space naustrian.ldf, ngerman.ldf, ngermanb.ldf,}
\Msg{* \space\space nswissgerman.ldf and swissgerman.ldf}
\Msg{*}
\Msg{* To produce the documentation run the files }
\Msg{* germanb.dtx and ngermanb.dtx through LaTeX.}
\Msg{*}
\Msg{* Happy TeXing}
\Msg{***********************************************************}
}
 
\endbatchfile
}
\DeclareOption{italian}{% \iffalse meta-comment
%
%
% It may be distributed and/or modified under the
% conditions of the LaTeX Project Public License, either version 1.3
% of this license or (at your option) any later version.
% The latest version of this license is in
%   http://www.latex-project.org/lppl.txt
% and version 1.3 or later is part of all distributions of LaTeX
% version 2003/12/01 or later.
%
% This work has the LPPL maintenance status "maintained".
%
% The Current Maintainer of the babel system is Javier Bezos
% The current maintainer of Italian language support is Claudio Beccari
%
% The list of all files belonging to the LaTeX base distribution is
% given in the file `manifest.bbl. See also `legal.bbl' for additional
% information.
%
% The list of derived (unpacked) files belonging to the distribution
% and covered by LPPL is defined by the unpacking scripts (with
% extension .ins) which are part of the distribution.
% \fi
% \CheckSum{740}
% \iffalse
%    Tell the \LaTeX\ system who we are and write an entry on the
%    transcript.
%\fi
%\iffalse
%<*filedriver>
\ProvidesFile{italian.dtx}
%</filedriver>
%<code>\ProvidesLanguage{italian}
%<*code>
        [2015/03/26 v1.3n Italian support from the babel system]
%</code>
%%
%% Please report errors to: claudio dot beccari at gmail dot com
%%
%
%    This file is part of the babel system, it provides the source
%    code for the Italian language definition file.
%    The original version of this file was written by Maurizio
%    Codogno, (mau@beatles.cselt.stet.it). 
%    The package was completely rewritten by Claudio Beccari, who added
%    several features.
%<*filedriver>
\documentclass{ltxdoc}
\usepackage[T1]{fontenc}
\usepackage{lmodern}
\usepackage{booktabs}
\newcommand*\TeXhax{\TeX hax}
\newcommand*\babel{\textsf{babel}}
\newcommand*\langvar{$\langle \it lang \rangle$}
\newcommand*\note[1]{}
\newcommand*\Lopt[1]{\textsf{#1}}
\newcommand*\file[1]{\texttt{#1}}
\begin{document}
 \DocInput{italian.dtx}
\end{document}
%</filedriver>
%\fi
% \GetFileInfo{italian.dtx}
%
% \changes{italian-0.99}{1990/07/11}{First version, from english.doc}
% \changes{italian-1.0}{1991/04/23}{Modified for babel 3.0}
% \changes{italian-1.0a}{1991/05/23}{removed typo}
% \changes{italian-1.0b}{1991/05/29}{Removed bug found by van der Meer}
% \changes{italian-1.0e}{1991/07/15}{Renamed \file{babel.sty} in
%    \file{babel.com}}
% \changes{italian-1.1}{1992/02/16}{Brought up-to-date with babel 3.2a}
% \changes{italian-1.2}{1994/02/09}{Update for\ LaTeXe}
% \changes{italian-1.2e}{1994/06/26}{Removed the use of \cs{filedate}
%    and moved identification after the loading of \file{babel.def}}
% \changes{italian-1.2f}{1995/05/28}{Updated for babel 3.5}
% \changes{italian-1.2i}{1996/10/10}{Replaced \cs{undefined} with
%    \cs{@undefined} and \cs{empty} with \cs{@empty} for consistency
%    with \LaTeX, moved the definition of \cs{atcatcode} right to the
%    beginning.}
% \changes{italian-1.2l}{1999/04/24}{Added \cs{unit}, \cs{ap}, and
%    \cs{ped} macros}
% \changes{italian-1.2m}{2000/01/05}{Added support for etymological
%    hyphenation}
% \changes{italian-1.2n}{2000/02/02}{Completely modified etymological
%    hyphenation facility}
% \changes{italian-1.2n}{2000/05/28}{Added several commands for the
%   caporali double quotes and for simplifying the accented vowel input}
% \changes{italian-1.2o}{2000/12/12}{Added \cs{glossaryname}}
% \changes{italian-1.2p}{2002/07/10}{Removed redefinition of
%    \cs{add@acc} since its functionality has been introduced into the
%    kernel of LaTeX 2001/06/01}
% \changes{italian-1.2q}{2005/02/05}{Added test for avoiding conflict
%    with package units.sty; adjusted caporali functionality, since
%    the previous one did not work with the standard (although obsolete)
%    slides class file}
% \changes{italian-1.3}{2013/09/29}{Completely changed the Italian
%   guillemet functionality in order to simplify their handling}
% \changes{italian-1.3}{2013/09/30}{Eliminated the \emph{traditional} attribute;
%   the same functionality is obtained with the declaration \cs{XXIletters}}
% \changes{italian-1.3}{2-13/09/30}{Command \cs{unit} is disabled if packages
%   units, SIunits, or siunitx have been loaded; this control is now deferred
%   at the end of the preamble, when all packages have already been loaded}
% \changes{italian-1.3a}{2013/10/02}{Eliminated conflict with the amsmath
%   package}
% \changes{italian-1.3-d}{2013/11/11}{Corrected the \string" behaviour for
%   etymological hyphenation}
% \changes{italian-1.3f}{2013/11/22}{Deleted extra \cs{selectlanguage}\{italian\}}
% \changes{italian-1.3g}{2014/02/16}{Corrected the \string"/ shorthand}
% \changes{italian-1.3g}{2014/02/16}{Italian shorthandss are now optional as
%   well as the intelligent comma facility}
% \changes{italian-i.3i}{2014/02/17}{Updated documentation}
% \changes{italian-1.3i}{2014/02/17}{Modified the \string"- shorthand}
% \changes{italian-1.3j}{2014/02/22}{\string\setactivedoublequote\ does work
%   but it must be delayed until \string\begin\{document\}; with babel 3.9g
%   the main language is set before this point, therefore it is necessary
%   to select again the Italian language if this is the main one}
% \changes{italian-1.3k}{2014/03/29}{Corrected discrepancy between 
%   \cs{SetISOcompliance} (wrong in  documentation) and \cs{setISOcompliance}
%   (correct in code)}
% \changes{italian-1.3l}{2014/05/25}{Corrections in the documentation}
% \changes{italian-1.3m}{2015/03/12}{More corrections to the documentations,
%  specifically on conflicts with the dcolumn package and its D column.
%  The whole \(No)IntelligentComma mechanism has been actually completely
%  rewritten; so the change does not deal only with documentation, but also
%  with code.}
%
%  \section{The Italian language}
%    \textbf{Important notice}: This language description file relies on
%    functionalities provided by a modern TeX system distribution with
%    pdfLaTeX working in extended mode (eTeX commands available); it
%    should perform correctly also with XeLaTeX and LuaLaTeX; tests have
%    been made also with the latter programs, but it was really tested in
%    depth with |babel| and pdfLaTeX.
%
%    \bigskip
%
%    The file \file{\filename}\footnote{The file described in this
%    section has version number \fileversion\ and was last revised on
%    \filedate. The original author is Maurizio Codogno.
%    It was initially revised
%    by Johannes Braams and then completely rewritten by Claudio Beccari}
%    defines all the required and some optional language-specific
%    macros for the Italian language.
%
%    \begin{table}[htb]\centering
%    \begin{tabular}{cp{90mm}}
%    \toprule
%    |"|    & inserts a compound word mark where hyphenation is legal;
%             it allows etymological hyphenation which is recommended
%             for technical terms, chemical names and the like; it
%             does not work if the next character is represented with
%             a control sequence or is an accented character.\\
%    \texttt{\string"\string|}
%           & the same as the above without the limitation on
%            characters represented with control sequences or accented
%            ones.\\
%    |""|   & inserts open quotes ``.\\ %^^A'' emacs matching
%    |"<|   & inserts open guillemets without trailing space.\\
%    |">|   & inserts closed guillemets without leading space.\\
%    |"/|   & allows hyphenation of both words connected with slash.\\
%    |"-|   & allows hyphenation of both words connected with a short dash
%                (\emph{trattino copulativo}, in Italian)\\
%    \bottomrule
%    \end{tabular}
%    \caption{shorthands for the Italian language. These shorthands are
%    available only if command \texttt{\string\setactivedoublequote} is given
%    after loading \babel\ and before \texttt{\string\begin\{document\}}.}
%    \label{t:itshrtct}
%    \end{table}
%
%    The features of this language definition file are the following:
%    \begin{enumerate}
%    \item The Italian hyphenation is invoked, provided that the Italian 
%      hyphenation pattern files were loaded when the specific format file
%      was built.
%    \item The language dependent infix words to be inserted by such
%      commands as |\chapter|, |\caption|, |\tableofcontents|,
%      etc. are redefined in accordance with the Italian
%      typographical practice.
%    \item Since Italian can be easily hyphenated and Italian practice
%      allows to break a word before the last two letters, hyphenation
%      parameters have been set accordingly, but a very high demerit
%      value has been set in order to avoid word breaks in the
%      penultimate line of a paragraph. Specifically the |\clubpenalty|,
%      and the |\widowpenalty| are set to rather high values and
%      |\finalhyphendemerits| is set to such a high value that
%      hyphenation is strongly discouraged between the last two lines
%      of a paragraph.
%^^A
%^^A  Qui ci si potrebbero mettere le indicazioni per gli attributi e modificatori
%^^A
%    \item Some language specific shorthands have been defined so as to
%      allow etymological hyphenation, specifically |"| inserts a
%      break point at any word boundary that the typesetter chooses,
%      provided it is not followed by an accented letter (very unlikely
%      in Italian, where compulsory accents fall only on the last and
%      ending vowel of a word, but it may take place with compound words
%      that include foreign roots), and \verb="|= when the desired break
%      point falls before an accented letter. As you can read in
%      table~\ref{t:itshrtct}, these shorthands are available only if
%      they get activated with |\setactivedoublequote| after
%      loading \babel\ but before the |\begin{docuemnt}| statement. This
%      is done in order to preserve the user from package conflicts:
%      if s/he wants to use these facilities s/he must remember that
%      conflicts may arise unless active characters are deactivated;
%      this can be done with the \babel\ command |\shorthadsoff{"}|
%      (and reactivated with |\shorthandson{"}|) when its wise to do
%      so; conflicts have been reported with package \file{xypic} and
%      with \texttt{TikZ}, but the latter has its own library to
%      deactivate all active characters, not just the double quotes,
%      the only Italian language possibly activated  character.
%    \item Some Italian compound words have a connecting short dash (a
%      hyphen sign) between them without any space between the component
%      words and the short dash; in this situation standard \LaTeX\ allows
%      a line break only just after the short dash; this may lead to
%      paragraphs with protruding lines or with ugly looking wide inter
%      word spaces. If a break point is desired in the second word, one
%      may use  a |"| sign just after the short dash; but if a line break
%      is required in the first word, them the |"-| shorthand comes in handy;
%      pay attention though, that if you use an en-dash or an em-dash (both
%      should not be used in Italian as compound words connectors, but\dots)
%      then the |"-| shorthand might impeach the |--| or |---| ligatures, thus
%      producing an unacceptable appearance.
%    \item The shorthand |""| introduces the raised (English) opening
%      double quotes; this shorthand proves its usefulness when one
%      reminds that the Italian keyboard misses the backtick key, and
%      the backtick on a Windows based platform may be obtained only by
%      pressing the \texttt{Alt} key while keying the numerical code
%      0096 in the numeric keypad; very, very annoying!
%    \item The shorthands |"<| and |">| insert the guillemets sometimes
%      used also in Italian typography; with the T1 font encoding
%      the ligatures |<<| and |>>| should insert such signs directly,
%      but not all the virtual fonts that claim to follow the T1 font
%      encoding actually contain the guillemets; with the OT1 encoding
%      the guillemets are not available and must be faked in some
%      way. By using the |"<| and |">| shorthands (even with the T1
%      encoding) the necessary tests are performed and in case the
%      guillemets are faked by means of the special LaTeX math symbols.
%      At the same time if OpenType fonts are being used with XeLaTeX
%      or LuaLaTeX, there are no problems with guillemets.
%    \item Three new specific commands |\unit|, |\ped|, and |\ap| are
%      introduced so as to  enable the correct composition of technical
%      mathematics according to the ISO~31/XI recommendations. 
%      The definition of |\unit| takes place only at ``begin document''
%      so that it is possible to verify if some other similar functionalities
%      have already been defined by other packages, such as |units.sty|
%      or |siunitx.sty|. In particular command |\unit| is deactivated by
%      default; the user can activate it by entering the command
%      |\setISOcompliance| after loading the \babel\ package and before the
%      |\begin{document}| statement. The above checks will enter into
%      action even if this ISO compliance is set, in order to avoid conflicts
%      with the above named packages. The |\ap| and |\ped| commands remain
%      available because up to now no specific conflicts have been reported. 
%    \item Since in all languages different from English the decimal separator
%      according to the ISO regulations \emph{must} be a comma\footnote{Actually
%      the Bureau International des Pois et M\'esures allows also the point
%      as a decimal separator without mentioning any language, but recommends
%      to follow the national typographical traditions}; since
%      no language description file nor the \babel\ package itself provides
%      for this functionality, a not so simple intelligent comma definition is
%      provided such that at least in mathematics it behaves correctly.
%      There are other packages that provide a similar functionality, for
%      example |icomma| and |ncccomma|; |icomma|, apparently is not in conflict
%      with |dcolomn|, but requires a space after the comma all the times it
%      plays the r\^ole of a punctuation mark; |ncccomma|, checks if the next
%      token is a digit, but it repeated ten tests every time it meets a comma,
%      irrespective from what it is followed by. I believe that my solution
%      is better than that provided by both those packages; but I assume that
%      if the user loads on of those packages, it prefers to use that
%      functionality; In case one of those pachages is loaded, this module
%      excludes its intelligent comma functionality.
%      By default this functionality is turned \emph{off}, therefore the user
%      should turn it on by means of the |\IntelligentComma| command; it can
%      turn it off by means of |\NoIntelligentComma|. Please, read
%      subsection~\ref{ssec:comma} to see the various situations where a
%      mathematical comma may be used and how to overcome the few cases when
%      the macros of this file don't behave as expected. The section describes
%      also some limitations when some cong=flicting packages are being loaded.
%    \item In Italian legal documents it is common to tag list-items
%      with the old fashioned 21-letter Italian alphabet, that differs from
%      the Latin one by the omission of the letters `j', `k', `w',`x', and
%     `y'. This applies for both upper and lower case tags.
%      This feature is obtained by using the commands |\XXIletters| and
%      |\XXVIletters| that allow to switch back and forth
%      between 21- and 26-letter tagging.
%    \end{enumerate}
%
%    For this language a few shorthands have been defined,
%    table~\ref{t:itshrtct}, some of which are introduced to overcome
%    certain limitations of the Italian keyboard; in
%    section~\ref{s:itkbd} there are other comments and hints in order
%    to overcome some other keyboard limitations.
%
% \subsection*{Acknowlegements}
% It is my pleasure to acknowledge the contributions of Giovanni Dore,
% Davide Liessi, Grazia Messineo, Giuseppe Toscano, who spotted some bugs
% or conflicts with other packages, mainly |amsmath| and |icomma|, and
% with digits hidden inside macros or control sequences representing
% implicit characters. Testing by real users and their feedback is
% essential with open software such as the uncountable contributions
% to the \TeX\ system. Thank you very much.
%
% \StopEventually{%
%    \begin{thebibliography}{1}
%    \bibitem{CBec} Beccari C., ``Computer Aided Hyphenation for
%    Italian and Modern Latin'', \textsf{TUGboat} vol.~13, n.~1,
%    pp.~23-33 (1992).
%    \bibitem{Becc2} Beccari C., ``Typesetting mathematics for science
%    and technology according to ISO\,31/XI'', \textsf{TUGboat}
%    vol.~18, n.~1, pp.~39-48 (1997).
%    \end{thebibliography}}
%
%\subsection{The commented code}
%    The macro |\LdfInit| takes care of preventing that this file is
%    loaded more than once, checking the category code of the
%    \texttt{@} sign, etc.
% \changes{italian-1.2i}{1996/11/03}{Now use \cs{LdfInit} to perform
%    initial checks}
% \changes{italian-1.2j}{1996/12/29}{Added braces around second arg of
%    \cs{LdfInit}}
%\iffalse
%<*code>
%\fi
%    \begin{macrocode}
\LdfInit{italian}{captionsitalian}%
%    \end{macrocode}
%
%    When this file is read as an option, i.e. by the |\usepackage|
%    command, \texttt{italian} will be an `unknown' language in which
%    case we have to make it known.  So we check for the existence of
%    |\l@italian| to see whether we have to do something here.
%
% \changes{italian-1.0}{1991/04/23}{Now use \cs{adddialect} if
%    language undefined}
% \changes{italian-1.0h}{1991/10/08}{Removed use of \cs{@ifundefined}}
% \changes{italian-1.1}{1992/02/16}{Added a warning when no
%    hyphenation patterns were loaded.}
% \changes{italian-1.2e}{1994/06/26}{Now use \cs{@nopatterns} to
%    produce the warning}
%    \begin{macrocode}
\ifx\l@italian\@undefined
    \@nopatterns{Italian}%
    \adddialect\l@italian0\fi
%    \end{macrocode}
%
%    The next step consists of defining commands to switch to (and
%    from) the Italian language.
%
% \begin{macro}{\captionsitalian}
%    The macro |\captionsitalian| defines all strings used
%    in the four standard document classes provided with \LaTeX.
% \changes{italian-1.0c}{1991/06/06}{Removed \cs{global} definitions}
% \changes{italian-1.0c}{1991/06/06}{\cs{pagename} should be
%    \cs{headpagename}}
% \changes{italian-1.0d}{1991/07/01}{`contiene' substitued by `Allegati'
%    as suggested by Marco Bozzo (\texttt{BOZZO@CERNVM}).}
% \changes{italian-1.1}{1992/02/16}{Added \cs{seename}, \cs{alsoname}
%    and \cs{prefacename}}
% \changes{italian-1.1}{1993/07/15}{\cs{headpagename} should be
%    \cs{pagename}}
% \changes{italian-1.2b}{1994/05/19}{Changed some of the infix words}
% \changes{italian-1.2g}{1995/07/04}{Added \cs{proofname} for
%    AMS-\LaTeX}
% \changes{italian-1.2h}{1995/07/27}{Added translation of `Proof'}
%    \begin{macrocode}
\addto\captionsitalian{%
  \def\prefacename{Prefazione}%
  \def\refname{Riferimenti bibliografici}%
  \def\abstractname{Sommario}%
  \def\bibname{Bibliografia}%
  \def\chaptername{Capitolo}%
  \def\appendixname{Appendice}%
  \def\contentsname{Indice}%
  \def\listfigurename{Elenco delle figure}%
  \def\listtablename{Elenco delle tabelle}%
  \def\indexname{Indice analitico}%
  \def\figurename{Figura}%
  \def\tablename{Tabella}%
  \def\partname{Parte}%
  \def\enclname{Allegati}%
  \def\ccname{e~p.~c.}%
  \def\headtoname{Per}%
  \def\pagename{Pag.}%   
  \def\seename{vedi}%
  \def\alsoname{vedi anche}%
  \def\proofname{Dimostrazione}%
  \def\glossaryname{Glossario}%
  }%
%    \end{macrocode}
% \end{macro}
%
% \begin{macro}{\dateitalian}
%    The macro |\dateitalian| redefines the command
%    |\today| to produce Italian dates.
% \changes{italian-1.0c}{1991/06/06}{Removed \cs{global} definitions}
%    \begin{macrocode}
\def\dateitalian{%
  \def\today{\number\day~\ifcase\month\or
    gennaio\or febbraio\or marzo\or aprile\or maggio\or giugno\or
    luglio\or agosto\or settembre\or ottobre\or novembre\or
    dicembre\fi\space \number\year}}%
%    \end{macrocode}
% \end{macro}
%
% \begin{macro}{\italianhyphenmins}
% \changes{italian-1.2b}{1994/05/19}{Added setting of left and
%    righthyphenmin}
%
%    The italian hyphenation patterns can be used with both
%    |\lefthyphenmin| and |\righthyphenmin| set to~2.
% \changes{italian-1.2m}{2000/09/22}{Now use \cs{providehyphenmins} to
%    provide a default value}
%    \begin{macrocode}
\providehyphenmins{\CurrentOption}{\tw@\tw@}
%    \end{macrocode}
% \end{macro}
%
% \begin{macro}{\extrasitalian}
% \begin{macro}{\noextrasitalian}
%
% \changes{italian-1.2b}{1994/05/19}{Added settings of club- and
%    widowpenalty}
%    Lower the chance that clubs or widows occur.
%    \begin{macrocode}
\addto\extrasitalian{%
  \babel@savevariable\clubpenalty
  \babel@savevariable\widowpenalty
  \babel@savevariable\@clubpenalty
  \clubpenalty3000\widowpenalty3000\@clubpenalty\clubpenalty}%
%    \end{macrocode}
%
% \changes{italian-1.2b}{1994/05/19}{Added setting of
%    finalhyphendemerits}
%
%    Never ever break a word between the last two lines of a paragraph
%    in Italian texts.
%    \begin{macrocode}
\addto\extrasitalian{%
  \babel@savevariable\finalhyphendemerits
  \finalhyphendemerits50000000}%
%    \end{macrocode}
%
% \changes{italian-1.2h}{1995/11/10}{Now give the apostrophe a
%    lowercase code}
% \changes{italian-1.2l}{1999/04/5}{Changed example ``begl'italiani''
%    (obsolete spelling) with another, ``nell'altezza'', that behaves
%    the same way}
%    In order to enable the hyphenation of words such as
%    ``nell'altezza'' we give the \texttt{'} a non-zero lower case
%    code. When we do that \TeX\ finds the following hyphenation
%    points |nel-l'al-tez-za| instead of none.
%    \begin{macrocode}
\addto\extrasitalian{%
  \lccode`'=`'}%
\addto\noextrasitalian{%
  \lccode`'=0}%
%    \end{macrocode}
% \end{macro}
% \end{macro}
%
% \subsection{Traditionally labelled enumerate environment}
% \changes{italian-1.2v}{2010/01/02}{Support for traditional Italian}
%    In some traditional texts, especially of legal nature, enumerations labelled
%    with lower or upper case letters use the reduced Latin alphabet that omits
%    the so called ``non Italian letters'': j, k, w, x, and y. 
%
% \changes{italian-1.2w}{2011/01/03}{Added switching mechanism between normal
%    and traditional enumeration labelling}
% \changes{italian-1.3}{2013/09/27}{The |traditional| language attribute has
%    been dropped while the commands to switch back and forth from 21- to
%    26-letter alphabet are retained}
% \begin{macro}{\XXIletters}
% \begin{macro}{\XXVIletters}
%    At the same time it is considered useful to have the possibility of
%    switching back and forth from the 21-letter tagging and the 26-letter one.
%    This requires a counter that keeps the switching status (0 for 21 letters
%    and 1 for 26 letters) and commands |\XXIletters| and |\XXVIletters| 
%    to set the switch. Default is 26 letter tagging.
%    \begin{macrocode}
\newcount\it@lettering \it@lettering=\@ne
\newcommand*\XXIletters{\it@lettering=\z@}
\newcommand*\XXVIletters{\it@lettering=\@ne}
\let\bbl@alph\@alph \let\bbl@Alph\@Alph
\addto\extrasitalian{\babel@savevariable\it@lettering
  \let\@alph\it@alph \let\@Alph\it@Alph}
\addto\noextrasitalian{\let\@alph\bbl@alph\let\@Alph\bbl@Alph}
%    \end{macrocode}
% \end{macro}
% \end{macro}
%    To make this feasible it's necessary to redefine the way the \LaTeX\ |\@alph|
%    and |\@Alph| work. Let's make the alternate definitions:
%    \begin{macrocode}
\def\it@alph#1{%
\ifcase\it@lettering
   \ifcase#1\or a\or b\or c\or d\or e\or f\or g\or h\or i\or
   l\or m\or n\or o\or p\or q\or r\or s\or t\or u\or v\or
   z\else\@ctrerr\fi
\or
   \ifcase#1\or a\or b\or c\or d\or e\or f\or g\or h\or i\or
   j\or k\or l\or m\or n\or o\or p\or q\or r\or s\or t\or u\or v\or
   w\or x\or y\or z\else\@ctrerr\fi
\fi}%
\def\it@Alph#1{%
\ifcase\it@lettering
   \ifcase#1\or A\or B\or C\or D\or E\or F\or G\or H\or I\or
   L\or M\or N\or O\or P\or Q\or R\or S\or T\or U\or V\or
   Z\else\@ctrerr\fi
\or
   \ifcase#1\or A\or B\or C\or D\or E\or F\or G\or H\or I\or
   J\or K\or L\or M\or N\or O\or P\or Q\or R\or S\or T\or U\or V\or
   W\or X\or Y\or Z\else\@ctrerr\fi
\fi}%
%    \end{macrocode}
%
% In order to have a complete description, the situation is as such:
%    \begin{enumerate}
%    \item If you want to always use the 21-letter item tagging, simply
%       use the |\XXIletters| declaration just after |\begin{document}|
%       and this setting remains global (provided, of course, that the
%       declaration is defined, i.e. that the Italian language is the
%       default one); in this way the setting is global while you use
%       the Italian language.
%    \item The |XXVIletter| command, issued outside any environment sets
%       the 26-letter item tagging in a global way; this setting is the default one.
%    \item If you specify |\XXIletters| just before entering an
%       environment that uses alphabetic tagging, this environment will
%       be tagged with the 21-letter alphabet, but this is a local setting,
%       because the letter tagging takes place only from the second level
%       of enumeration.
%    \item The declarations |\XXIletters| and |\XXVIletters| let you 
%      switch back and forth between the two kinds of tagging, But this
%      kind of tagging, the 21-letter one, is meaningful only in Italian
%      and when you change language, letter tagging reverts to the 26-letter
%      one. 
%    \end{enumerate}
%
% \changes{italian-1.2m}{2000/01/05}{Support for etymological
%    hyphenation}
%
%   \subsection{Support for etymological hyphenation}

%    In Italian etymological hyphenation is desirable with
%    technical terms, chemical names, and the like.

%    \subsubsection{Some history}
%    In his article on Italian hyphenation \cite{CBec} Beccari pointed
%    out that the Italian language gets hyphenated on a phonetic
%    basis, although etymological hyphenation is allowed; this is in
%    contrast with what happens in Latin, for example, where
%    etymological hyphenation is always used. Since the patterns for
%    both languages would become too complicated in order to cope with
%    etymological hyphenation, in his paper Beccari proposed the
%    definition of an active character `|_|' such that it could insert
%    a ``soft'' discretionary hyphen at the compound word
%    boundary. For several reasons that idea and the specific active
%    character proved to be unpractical and was abandoned.
%
%    This problem is so important with the majority of the European
%    languages, that \babel\ from the very beginning developed
%    the tradition of making the |"| character active so as to perform
%    several actions that turned useful with every language.
%    One of these actions consisted in defining the shorthand \verb="|=,
%    that was extensively used in German and in many other languages,
%    in order to insert a discretionary hyphen such that hyphenation
%    would not be precluded in the rest of the word as it happens with
%    the standard \TeX\ command |\-|.
%
%    Meanwhile the \texttt{ec} fonts with the double Cork encoding
%    (thus formerly called the \texttt{dc} fonts) have become more or
%    less standard and are widely used by virtually all Europeans that
%    write languages with many special national characters; by so
%    doing they avoid the use of the |\accent| primitive which would
%    be required with the standard OT1 encoded \texttt{cm} fonts;
%    with such OT1 encoded fonts the primitive command |\accent| is such
%    that hyphenation becomes almost impossible, in any case strongly
%    impeached.
%
%    The T1 encoded fonts contain a special character, named
%    ``compound word mark'', that occupies slot 23 (or |'27| or |"17|
%    in the font scheme and may be input with the sequence |^^W|.
%    Up to now, apparently, this special character has never been used
%    in a practical way for typesetting languages rich of compound
%    words; moreover it has never been inserted in the hyphenation pattern
%    files of any language. Beccari modified his pattern file
%    \file{ithyph.tex v4.8b} for Italian so as to contain five new
%    patterns that involve |^^W|, and he tried to give the
%    \babel\ active character |"| a new shorthand definition,
%    so as to allow the insertion of the ``compound word mark'' in the
%    proper place within any word where two semantic fragments join
%    up. With such facility for marking the compound word boundaries,
%    etymological hyphenation becomes possible even if the patterns
%    know nothing about etymology (but the typesetter hopefully
%    does!).
%
%    \subsubsection{The current solution}
%
%    Even this solution proved to be inconvenient on certain *NIX
%    platforms, so Beccari resorted to another approach that uses the
%    \babel\ active character |"| and relies on the category
%    code of the character that follows |"|.
%
% \changes{italian-1.2n}{2000/02/02}{Completely new etymological
%    hyphenation facility}
% \changes{italian-1.3g}{2014/01/22}{The active double straight
%    quote conflicts with other packages; set as an optional facility.}
%    Instead of a boolean switch we use a private counter so as to check
%    at |\begin{document}| if this facility has to be activated. The default
%    value is zero; anything different from zero means that the facility
%    has to be activated; this is done with command |\setactivedoublequote|
%    to be issued before |\begin{document}|
%
%    \begin{macrocode}
\newcount\it@doublequoteactive \it@doublequoteactive=\z@
\def\setactivedoublequote{\it@doublequoteactive=\@ne}
{\catcode`"=12 \global\let\it@doublequote"}
{\catcode`"=13 \global\let\it@@dqactive"}
\AtBeginDocument{%
  \unless\ifnum\it@doublequoteactive=\z@
  \initiate@active@char{"}%
  \addto\extrasitalian{\bbl@activate{"}\languageshorthands{italian}}%
%    \end{macrocode}
%    \begin{macro}{\it@cwm}
%    The active character |"| is now defined for language |italian| so
%    as to perform different actions in math mode compared to text
%    mode; specifically in math mode a double quote is inserted so as
%    to produce a double prime sign, while in text mode the temporary
%    macro |\it@next| is defined so as to defer any further action
%    until the next token category code has been tested.
%    \begin{macrocode}
  \declare@shorthand{italian}{"}{%
    \ifmmode
      \def\it@next{''}%
    \else
      \def\it@next{\futurelet\it@temp\it@cwm}%
    \fi
    \it@next
  }%
\fi
%    \end{macrocode}
% The following statement must be conditionally executed after the above
% modification of the |\extraasitalian| list; in facts at the ``begin
% document'' execution the main language has already been set without
% the above modifications; therefore nothing takes place unless the
% Italian main language is selected again with the explicit command
% |\selectlanguage| without this trick the active double quotes would
% remain inactive; of course |\languagename| contains the string |italian|
% if this language was the main one; by testing this string, the suitable
% command may be issued again with the new settings and the double quotes
% become really active. Thanks to Davide Liessi for reporting this bug.
%    \begin{macrocode}
\ifdefstring{\languagename}{italian}{\selectlanguage{italian}}{\relax}
}%
%    \end{macrocode}
%    \begin{macro}{\it@cwm}
%    The \cs{it@next} service control sequence is such that upon its
%    execution a temporary variable \cs{it@temp} is made equivalent to
%    the next token in the input list without actually removing it.
%    Such temporary token is then tested by the macro \cs{it@cwm}
%      and if it is found to be a letter token (cathode=11), then it
%      introduces a compound word separator control sequence
%      \cs{it@allowhyphens} whose expansion introduces a discretionary
%      hyphen and an unbreakable zero space;
%    otherwise the token is not a letter; then it is therefore
%      tested against \verb=|=$_{12}$: if so a compound word separator
%      is inserted and the \verb=|= token is removed;
%    otherwise two other tests are performed to see if guillemets
%      have to be inserted, and in case a suitable intelligent
%      guillemet macro is introduced that gobbles unwanted leading
%      or trailing spaces; 
%    otherwise a test is made to see if the next char is a slash
%      character, and in case a special discretionary break is inserted
%      such as to maintain the slash while allowing the hyphenation
%      of both words before and after the slash;
%    otherwise another test is performed to see if another
%      double quote sign follows: in this case a double open quote
%      mark is inserted;
%    otherwise another test is made to see if a connecting hyphen char
%      follows, and in this case the hyphen char is substituted with
%      a discretionary break that allows hyphenation of both words
%      before and after the hyphen char;
%    otherwise nothing is done.
%
%    The double quote shorthand for inserting a double open quote sign
%    is useful for people who are inputting Italian text by means of
%    an Italian keyboard which unfortunately misses the grave or
%    backtick key.
%    The shorthand |""| becomes equivalent to |``| for inserting
%    raised open high double quotes.
%    \begin{macrocode}
\def\it@@cwm{\bbl@allowhyphens\discretionary{-}{}{}\bbl@allowhyphens}%
\def\it@@slash{\bbl@allowhyphens\discretionary{/}{}{/}\bbl@allowhyphens}%
\def\it@@trattino{\bbl@allowhyphens\discretionary{-}{}{-}\bbl@allowhyphens}%
\def\it@@ocap#1{\it@ocap}\def\it@@ccap#1{\it@ccap}%
\DeclareRobustCommand*{\it@cwm}{\let\it@@next\it@doublequote
\ifcat\noexpand\it@temp a%
    \def\it@@next{\it@@cwm}%
\else
    \if\noexpand\it@temp \string|%
        \def\it@@next{\it@@cwm\@gobble}%
    \else
        \if\noexpand\it@temp \string<%
            \def\it@@next{\it@@ocap}%
        \else
            \if\noexpand\it@temp \string>%
                \def\it@@next{\it@@ccap}%
            \else
                \if\noexpand\it@temp\string/%
                    \def\it@@next{\it@@slash\@gobble}%
                \else
                    \ifcat\noexpand\it@temp\noexpand\it@@dqactive
                        \def\it@@next{``\@gobble}%
                    \else
                        \if\noexpand\it@temp\string-%
                            \def\it@@next{\it@@trattino\@gobble}%
                        \fi
                    \fi
                \fi
            \fi
        \fi
    \fi
\fi
\it@@next}%
%    \end{macrocode}
%    \end{macro}
%    \end{macro}
%
%
%   \begin{sloppypar}
%    By this definition of |"| if one types |macro"istruzione| the
%    possible break points become \textsf{ma-cro-istru-zio-ne}, while
%    without  the |"| mark they would be \textsf{ma-croi-stru-zio-ne},
%    according to the phonetic rules such that the |macro| prefix is
%    not taken as a unit.
%    A chemical name such as \texttt{des"clor"fenir"amina"cloridrato}
%    is breakable as \textsf{des-clor-fe-nir-ami-na-clo-ri-dra-to}
%    instead of \textsf{de-sclor-fe-ni-ra-mi-na-\dots}
%
%    In other language description files a shorthand is defined so as
%    to allow a break point without actually inserting any hyphen
%    sign; examples are given such as \mbox{entrada/salida}; actually
%    if one wants to allow a breakpoint after the slash, it is much
%    clearer to type |\slash| instead of |/| and \LaTeX\ does
%    everything by itself; here the shorthand |"/| was introduced to
%    stand for |\slash| so that one can type |input"/output| and allow
%    a line break after the slash.
%    This shorthand works only for the slash, since in Italian such
%    constructs are extremely rare.
%   \end{sloppypar}
%
%    Attention: the expansion of |"| takes place before the actual
%    expansion of OT1 or T1 accented sequences such as |\`{a}|;
%    therefore this etymological hyphenation facility works as it
%    should only when the semantic word fragments \textit{do not
%    start} with an accented letter; this in Italian is always
%    avoidable, because compulsory accents fall only on the last
%    vowel, but it may be necessary to mark a compound word where one
%    or more components come from a foreign language and contain
%    diacritical marks according to the spelling rules of that
%    language. In this case the special shorthand \verb!"|! may be
%    used that performs exactly as |"| normally does, except that the
%    \verb!|! sign is removed from the token input list:
%    \verb!kilo"|\"orsted! gets hyphenated as
%    \texttt{ki-lo-\"or-sted}; but also 
%    \texttt{kilo\string"\string|\"orsted} gets hyphenated correctly
%    as \texttt{ki-lo-\"or-sted} The \verb="|= macro is necessary
%    because, even with a suitable option specified to the |inputenc|
%    package, the letter `\"o' does not have category code 11, as the ASCII
%    letters do, because of the LICR (LaTeX Internal Character Representation),
%    i.e. the set of intermediate macros that have to be expanded
%    in order to fetch the proper glyph in the output font.
%
%    \changes{italian-1.2l}{1999/04/05}{Added useful macros for
%    fulfilling ISO 31/XI regulations}
%
%   \subsection{Facilities required by the ISO 31/XI regulations}
%    The ISO 31/XI regulations require that units of measure are
%    typeset in upright font in both math and text, and
%    that in text mode they are separated from the  numerical
%    indication of the measure with an unbreakable (thin) space.
%    The command |\unit| that was defined for achieving this
%    goal happened to conflict with the homonymous command defined by the
%    |units.sty| package; we therefore need to test if that package
%    has already been loaded so as to avoid conflicts; we assume that
%    if the user loads that package, s/he wants to use that package
%    facilities and command syntax.
%    
%    Actually there are around several packages that help to typeset 
%    units of measure in the proper way; besides |units.sty| there are also 
%    |SIunits| and |siunitx.sty|; the latter nowadays offers the best
%    performances in this domain. Therefore we keep controlling the
%    possibility that |units.sty| has been loaded just for backwards
%    compatibility, but we must do the same with |SIunits| and |siunitx.sty|.
%    In order to avoid the necessity o loading packages in a certain order,
%    we delay the test until |\begin{document}|.
%
%    The same ISO regulations require also that super and subscripts
%    (apices and pedices) are in upright font, \emph{not in math
%    italics}, when they represent ``adjectives'' or appositions to
%    mathematical or physical variables that do not represent
%    countable or measurable entities: for example,
%    $V_{\mathrm{max}}$ or $V_{\mathrm{rms}}$ for a maximum voltage or a root
%    mean square voltage, compared to $V_i$  or $V_T$ as the $i$-th
%    voltage in a set, or a voltage that depends on the thermodynamic
%    temperature $T$. See \cite{Becc2} for a complete description of
%    the ISO regulations in connection with typesetting.
%
%    More rarely it happens to use superscripts that are not
%    mathematical variables, such as the notation
%    $\mathbf{A}^{\!\mathrm{T}}$ to denote the transpose of matrix
%    $\mathbf{A}$; text superscripts are useful also as ordinals or
%    in old fashioned abbreviations in text mode; for example the
%    feminine ordinal $1^{\mathrm{a}}$ or the  old fashioned obsolete
%    abbreviation F$^{\mathrm{lli}}$ for \mbox{Fratelli} in company
%    names (compare with ``Bros.'' for \underline{Bro}ther\underline{s}
%    in American English); text subscripts are mostly used in logos.
%
%    \begin{macro}{\unit}
%    \begin{macro}{\ap}
%    \begin{macro}{\ped}
%    \begin{macro}{\setISOcompliance}
%    First we define the new (internal) commands |\bbl@unit|, |\bbl@ap|,
%    and |\bbl@ped| as robust ones.
% \changes{italian-1.3}{2013/09/27}{Added testing to avoid conflicts
%     with the units.sty and  siunitx.sty packages}
% \changes{italian-1.3g}{2014/01/22}{Besides controlling conflicts with
%    units, siunitx, this facility is deactivated by default, and activated
%    only if the user wants to}
%    This facility is deactivated by default according to the contents of
%    an internal counter and the setting of the activation command by the
%    user; commands for apices and pedices remain available in any case.
%    \begin{macrocode}
\newcount\it@ISOcompliance \it@ISOcompliance=\z@
\def\setISOcompliance{\it@ISOcompliance=\@ne}
\AtBeginDocument{\unless\ifnum\it@ISOcompliance=\z@%
\def\activate@it@unit{\DeclareRobustCommand*{\bbl@it@unit}[1]{%
    \textormath{\,\textup{##1}}{\,\mathrm{##1}}}}
\@ifpackageloaded{units}{}{\@ifpackageloaded{siunitx}{}{%
  \@ifpackageloaded{SIunits}{}{%
  \activate@it@unit\addto\extrasitalian{%
    \babel@save\unit\let\unit\bbl@it@unit}%
  }}}\ifcsstring{bbl@main@language}{italian}{\selectlanguage{italian}}{}%
\fi}
\DeclareRobustCommand*{\bbl@it@ap}[1]{%
  \textormath{\textsuperscript{#1}}{^{\mathrm{#1}}}}%
\DeclareRobustCommand*{\bbl@it@ped}[1]{%
  \textormath{$_{\mbox{\fontsize\sf@size\z@
        \selectfont#1}}$}{_\mathrm{#1}}}%
%    \end{macrocode}
%    Then we can use |\let| to define the user level commands, but in
%    case the macros already have a different meaning before entering
%    in Italian mode typesetting, we first save their meaning so
%    as to restore them on exit.
%    \begin{macrocode}
\addto\extrasitalian{%
  \babel@save\ap\let\ap\bbl@it@ap
  \babel@save\ped\let\ped\bbl@it@ped
  }%
%    \end{macrocode}
%    \end{macro}
%    \end{macro}
%    \end{macro}
%    \end{macro}
%
% \subsection{Intelligent comma}\label{ssec:comma}
%
%    We need to perform some tests that require some
%    smart control-sequence handling; therefore we call the |etoolbox|
%    package that allows us more testing functionality. There are no problems
%    with this package that can be invoked also by other ones before or
%    after |babel| is called; the |\RequirePackage| mechanism is sufficiently
%    smart to avoid reloading the same package more than once. But we
%    have to delay this call, because |italian.ldf| is being read while
%    processing the options passed to |babel|, and while options are being
%    scanned and processed it is forbidden to load packages; we
%    delay it at the end of processing the |babel| package itself.
%    \begin{macrocode}
\AtEndOfPackage{\RequirePackage{etoolbox}}
%    \end{macrocode}
%
%    \begin{macro}{\IntelligentComma}
%    \begin{macro}{\NoIntellgentComma}
%    This feature is optional, in the sense that it is necessary to issue
%    a specific command to activate it; actually this functionality is activated
%    or, respectively, deactivated with the self explanatory commands 
%    |\IntelligentComma| and |\NoIntelligentComma|. They operate by setting or
%    resetting  the comma sign as an active character in mathematics.
%    We defer the definition of the commands that turn on and off the intelligent
%    comma feature at the end of the preamble, so as to avoid possible conflicts
%    with other packages. It has already been pointed out that this procedure
%    for setting up the active comma to behave intelligently in math mode,
%    conflicts with the |dcolumn| package; therefore we assume these commands
%    are defined when the final user typesets a document, but they will be
%    possibly defined only at the end of the preamble when it will be known
%    if the |dcolumn| package has been loaded. We do the same if packages
%    |icomma| or |ncccomma| have been loaded, since that assumes that the user
%    wants to use their functionality, not the functionality of this package.
%
%    We need a command to set the comma as an active charter only in math mode;
%    the special |\mathcode| that classifies an active character in math is the
%    hexadecimal value |"8000|. By default we set the punctuation
%    type of comma, but we let |\IntelligentComma| and |\NoIntelligentComma|
%    to |\relax| so that their use is forbidden when one of the named packages
%    is loaded. In this way all known conflicts are avoided; should the user
%    find out other conflicts, s/he is kindly required to notify it to the
%    maintainer.
%    \begin{macrocode}
\AtEndOfPackage{\AtEndPreamble{%
\newcommand*\IntelligentComma{\mathcode`\,=\string"8000}% Active comma
\newcommand*\NoIntelligentComma{\mathcode`\,=\string"613B}% Punctuation comma
\@ifpackageloaded{icomma}{\let\IntelligentComma\relax
  \let\NoIntelligentComma\relax}{%
  \@ifpackageloaded{ncccomma}{\let\IntelligentComma\relax
  \let\NoIntelligentComma\relax}{%
    \@ifpackageloaded{dcolumn}{\let\IntelligentComma\relax
      \let\NoIntelligentComma\relax}{%
      \@ifpackageloaded{polyglossia}{%
        \ifcsstring{xpg@main@language}{english}{\relax}{%
          \mathcode`\,=\string"613B}
      }{%
        \ifcsstring{languagename}{english}{\relax}{%
          \mathcode`\,=\string"613B}
      }%
  }}}%
}}
%    \end{macrocode}
%    These commands are defined only in the |babel| support for the Italian
%    language (this file) and are not defined in the corresponding |polyglossia|
%    support for the language.
%    In order to have this functionality work properly with pdfLaTeX,
%    XeLaTeX, and LuaLaTeX, it is necessary to discover which engine
%    is being used, or better, which language handling package is
%    being used: |babel| or |polyglossia|? Let us remember that testing
%    the actual engine, as it would be possible with package |iftex|,
%    does not tell the whole truth, because LuaLaTeX and XeLaTeX  behave
%    in a similar way for what concerns language handling since they can
%    use both |babel| and |polyglossia| (obviously not at the same time);
%    so the use of |babel| or |polyglossia| is the real discriminant factor,
%    not the typesetting engine.
%
%    \begin{macro}{\virgola}
%    \begin{macro}{\virgoladecimale}
%    We need two kinds of comma, one that is a decimal separator, and 
%    a second one that is a punctuation mark.
%    \begin{macrocode}
 \DeclareMathSymbol{\virgola}{\mathpunct}{letters}{"3B}
 \DeclareMathSymbol{\virgoladecimale}{\mathord}{letters}{"3B}
%    \end{macrocode}
%    \end{macro}
%    \end{macro}
%
% \changes{italian-1.3a}{2013/10/02}{Corrected the bug of the intelligent
%     comma when used within the AMS alignment environments}
%
%    Math comma activation is done only after the preamble has
%    been completed, that is after the |\begin{document}| statement has been
%    completely executed. Now we must give a definition to the active comma:
%    probably it is not necessary to pass through an intermediate robust
%    command, but certainly it is not wrong to do it.
%    \begin{macrocode}
\DeclareRobustCommand*\it@comma@def{\futurelet\let@token\@@math@comma}%
{\catcode `,=\active \gdef,{\it@comma@def}}%
%    \end{macrocode}
%    The real work shall be performed by |\it@comma@def|. In facts the
%    above macro stores the token that immediately follows |\@@math@comma|
%    into a temporary control sequence that behaves as an implicit character
%    if that token is a single character, even a space, or behaves as
%    an alias of a control sequence otherwise. Actually at the end of the
%    preamble this macro shall be |\let| to be an alias for the real 
%   |\@math@comma|.
%
%    Is is important to remark that |\@math@comma| must be a command that
%    does not require arguments; this makes it robust when it is followed
%    by other characters that may play special r\^oles within the arguments
%    of other macros or environments. Matter of fact the first version 
%    of this file in version 1.3 did accept an argument; and the result was
%    that the active comma would ``eat up'' the |&| in vertical math
%    alignments  and very nasty errors took place, especially within the
%    |amsmath| defined ones. This macro |\@@math@comma| without arguments
%    does not do any harm to the AMS environments and the actual intelligent
%    comma work shall be executed by other macros.
%
%    At this point the situation may become complicated:  the comma character
%    in the input file may be followed by a real digit, by an analphabetic
%    character of category 12 (other character), by an implicit digit, by a
%    macro defined to be a digit, by a macro that is not defined to be a
%    digit, by a special character (for example a closed brace, an alignment
%    command, et cetera); therefore it is necessary to distinguish all these
%    situations; remember that an implicit digit cannot be used as a real
%    digit, and a macro gets expanded when used with any |\if| clause, unless
%    it is a |\ifx| one, or is prefixed with |noexpand|. The tests that are
%    going to be made are therefore of different kinds, according to this
%    scheme:
%    \begin{itemize}
%    \item 
%    the |\let@token| is tested against an asterisk to see it it is of
%    category 12; this is true if the token is a real digit, or an implicit
%    digit, or an analphabetic character;
%    \begin{itemize}
%    \item 
%    an implicit digit is represented by a control sequence; so we
%    first check this feature; 
%    \item 
%    if it is a control sequence, we have to test its nature of a
%    digit by testing if it represented one of the ten digits; 
%    \item otherwise it is an analphabetic character.
%    \end{itemize}
%    \item 
%    otherwise the |\let@token| is a special character or a
%    macro/command; 
%    \item 
%    a test is made to see if it is a macro; in this case we check
%    if has been defined to be a digit,
%    \item 
%    it is not a macro, it must be some other kind of token for
%    example a space or another special character.
%    \end{itemize}
%
%    Notice that if the token is a macro, we do not test if it is defined
%    to be a single digit or a string made up of more digits and/or other
%    charters. If the macro represents one digit the test is correct,
%    otherwise funny results may take place. For this reason it is always
%    better to prefix any macro with a space, whatever its definition might
%    be; if the macro represents a parameter defined to have a variable
%    value in the range 0--9, then it may represent the fractional part of
%    a (short) decimal value, and it is correct to avoid prefixing it with a
%    space; but the user is warned not to make use of numeric strings in
%    the definition of parameters, unless he knows what he is doing.
%    The user may rather use a balanced brace comma group |{,}| in the input file
%    so that the macro will not be considered by the expansion of the
%    active comma. For example if |\x| is defined to be the numerical
%    string |89|, the source input |$2{,}\x$| will be correctly typeset
%    as 2,89; the input \verb*!$2, \x$! will be typeset as 2,\,89 (with
%    an unbreakable thin space after the comma) while |$2,\x$| will be
%    typeset as 29,89, obviously wrong.
%
%    So first we test if the comma must act intelligently; if the counter
%    |\Virgola| contains zero, we assume that the comma must always perform
%    as a punctuation mark; but if we want to distinguish if it must behave
%    as a decimal separator, we have to perform more delicate tests; this
%    latter task is demanded to other macros with arguments |\@math@@comma|
%    and |\@@math@@comma|. In order to make the various tests robust we
%    have to resort to the usual trick of the auxiliary macros |\@firstoftwo|
%    and |\@secondoftwo| and various |\expandafter| commands so as to be
%    sure that every |\if| clause is correctly exited without leaving any
%    trace behind.
%    \begin{macrocode}
\DeclareRobustCommand*\@math@comma{%
 \ifcat\noexpand\let@token*%
    \expandafter\@firstoftwo
 \else
    \expandafter\@secondoftwo
 \fi{%  \let@token is of  category 12
      \@math@@comma
    }{% test if \let@token is a macro
       \ifcat\noexpand\let@token\noexpand\relax
          \expandafter\@firstoftwo
       \else
          \expandafter\@secondoftwo
       \fi{% it is a macro 
             \@@math@@comma
          }{% it is something else. 
             \virgola
          }
    }
}
%    \end{macrocode}
%
%    In particular this macro must test if the argument has category
%    code 12, that is ``other character'', not a letter, nor other
%    special signs, as |&| for example. In case the category code is
%    not 12, the comma must act as a punctuation mark; but if it is,
%    it might be a digit, or another character, an asterisk, for example;
%    so we have to test its digit nature; the simplest that was found
%    to test if a token is a digit, is to test its ASCII code against
%    the ASCII codes of `0' (zero) and `9'.
%    The typesetting engines give the backtick, |`|, the property that
%    when a number is required, it yields the ASCII code if the following
%    token in an explicit character or a macro argument; this is why we
%    can't use the temporary implicit token we just tested, but we must
%    examine the first non blank token that follows the |\@math@@comma|
%    macro. Only if the token is a digit, we use the decimal comma,
%    otherwise the punctuation mark. This is therefore the definition of
%    the |\@math@@comma| macro which is not that simple, although the testing
%    macros have clear meanings:
%    \begin{macrocode}
\DeclareRobustCommand*\@math@@comma[1]{% argument is certainly of category 12
    \ifcsundef{\expandafter\@gobble\string #1}{% test if it is a real digit
        \ifnumless{`#1}{`0}{\virgola}%
          {\ifnumgreater{`#1}{`9}{\virgola}%
             {\virgoladecimale}}%
    }{% it's an implicit character of category 12
      \let\@tempVirgola\virgola
      \@tfor\@tempCifra:=0123456789\do{%
        \expandafter\if\@tempCifra#1\let\@tempVirgola\virgoladecimale
        \@break@tfor\fi}\@tempVirgola
    }#1}
    
\DeclareRobustCommand*\@@math@@comma[1]{% argument is a macro
   \let\@tempVirgola\virgola
   \@tfor\@tempCifra:=0123456789\do{%
   \if\@tempCifra#1\let\@tempVirgola\virgoladecimale
   \@break@tfor\fi}\@tempVirgola#1
}
%    \end{macrocode}
%    The service macros |\ifcsundef|, |\ifnumless|, and |\ifnumgreater| are
%    provided by the |etoolbox| package, that shall be read at most at the
%    end of the |babel| package processing; therefore we must delay the code
%    at ``end preamble'' time, since only at that time
%    it will be known if the main language is English, or any other one.
%    This is why we have to perform such a baroque definition as the
%    following one:
%    \begin{macrocode}
\AtEndOfPackage{\AtEndPreamble{\let\@@math@comma\@math@comma}}
%    \end{macrocode}
%    This intelligent comma definition is pretty intelligent, but it requires
%    some kind of information from the context; this context does not
%    give enough bits of information to this `intelligence' in just one case:
%    when the comma plays the r\^ole of a serial separator in expressions
%    such as $i=1, 2, 3,\dots,\infty$, entered as 
%    \verb*?$i=1, 2, 3,\dots,\infty$?. Only in this case  the comma must
%    be followed by an explicit space; should this space
%    be absent the macro takes the following non blank token as a digit,
%    and since it actually is a digit, it would use the decimal comma, which
%    would be wrong. The control sequences |\dots| and |\infty| are
%    tested to see if they are undefined, and since they are defined and
%    do not represent digits, the macro inserts a punctuation mark,
%    instead of a decimal separator.
%
%    Notice that this macro may appear to be inconsistent with the contents
%    of a language description file. I don't agree: matter of facts even
%    math is part of typesetting a text in a certain language.
%    Does this set of macros influence other language description files?
%    May be, but I think that the clever use of macros |\IntelligentComma|
%    and |\NoIntellingentComma| may solve any interference; they allow to
%    use the proper mark even if the Italian language is not the main
%    language, the important point is to turn the switch on and/or off.
%    By default it is off, so there should not be any interference even
%    with legacy documents typeset in Italian.
%
%    Notice that there are other packages that contain facilities for
%    using the decimal comma as the correct decimal separator; for
%    example |SIunitx| defines a command |\num| that not only correctly
%    spaces the decimal separator, but also can change the input glyph
%    with another one, so that it is possible to copy and paste numbers
%    from texts in English (with the decimal point)  and paste them
%    into the argument of the |\num| macro in an Italian document where
%    the decimal point is changed automatically into a decimal comma. Of
%    course |SIunitx| does much more than that; if it's being loaded,
%    then the default |\NoIntelligentComma| declaration disables the
%    functionality defined in this language description file and the
%    user can do what he desires with the many functionalities of that
%    package.
%
%    Apparently a conflict with the active comma arises with the D column
%    defined by the |dcolomn| package. Disabling the ``Italian'' active comma
%    allows the D column operate correctly. Thanks to Giuseppe Toscano for
%    telling me about this conflict.
%    \end{macro}
%    \end{macro}
%
% \subsection*{Accents}\label{s:itkbd}
%    Most of the other language description files introduce a number
%    of shorthands for inserting accents and other language specific
%    diacritical marks in a more comfortable way compared with the
%    lengthy standard \TeX\ conventions. When an Italian keyboard is
%    being used on a Windows based platform, it exhibits such
%    limitations that to my best knowledge no convenient shorthands have been
%    developed; the reason lies in the fact that the Italian keyboard
%    lacks the grave accent (also known as ``backtick''), which is
%    compulsory on all accented vowels,  but, on the
%    opposite, it carries the keys with all the accented \emph{lowercase}
%    vowels \`a, \`e, \'e, \`i, \`o, \`u, bot no \emph{uppercase} accented
%    vowels are directly avalaible from the keyboard; the keyboard lacks also
%    the tie |~| (tilde) key, while the curly braces require pressing three
%    keys simultaneously.
%
%    The best solution Italians have found so far is to use a smart
%    editor that accepts shorthand definitions such that, for example,
%    by striking |"(| one gets directly |{| on the screen and the same
%    sign is saved into the \file{.tex} file; the same smart editor
%    should be capable of translating the accented characters into the
%    standard \TeX\ sequences when writing a file to disk (for the
%    sake of file portability), and to transform the standard \TeX\
%    sequences into the corresponding signs when loading a \file{.tex}
%    file from disk to memory. Such smart editors do exist and can be
%    downloaded from the \textsc{ctan} archives.
%
% \changes{italian-1.2p}{2002/07/10}{Removed redefinition of \cs{add@acc} since its
%    functionality has been introduced into the kernel of LaTeX 2001/06/01}
%
%    For what concerns the missing backtick key,
%    which is used also for inputting the open quotes, it must be
%    noticed that the shorthand |""| described above completely solves
%    the problem for \textit{double} raised open quotes; besides this,
%    a single open raised  quote may be input whit the little known
%    \LaTeX\ kernel command |\lq|; according to
%    the traditions of particular publishing houses, since there are
%    no  compulsory regulations on the matter, the guillemets
%    may be used; in this case the T1 font encoding solves the problem
%    by means of its built in ligatures |<<| and |>>|; such ligatures
%    are also available when using OpenType fonts with XeLaTeX and
%    LuaLaTeX, provided they are loaded with the option
%    \texttt{Ligatures = TeX}. But\dots
%
%    \subsection*{\emph{Caporali} or French double quotes}
%    Although the T1 font encoding ligatures solve the problem, there
%    are some circumstances where even the T1 font encoding cannot be
%    used, either because the author\slash typesetter wants to use the
%    OT1 encoding, or because s/he uses a font set that does
%    not comply completely with the T1 font encoding; some virtual
%    fonts, for example, are supposed to implement the double Cork
%    font encoding but actually miss some glyphs; one such virtual
%    font set is given by the \texttt{ae} virtual fonts, because they
%    are supposed to implement such double font encoding by using
%    virtual fonts that map the |CM| fonts to a T1 font scheme; the
%    type~1 PostScript version of the |CM| fonts do exist, therefore
%    one believes to be able of using them with pdfLaTeX; but since
%    the |CM| fonts do not contain the guillemets, neither the |AM|
%    ones do.
%    Since guillemets (in Italian \emph{caporali}) do not exist in
%    any OT1 encoded \texttt{cm} Latin font, their glyphs must be
%    substituted with something else that fakes them.
%
% \changes{italian-1.2q}{2005/02/05}{Redefined the caporali machinery
%     so as to avoid incompatibilities with the slides class, as there
%     are no Cyrillic slides fonts as there are for Latin script}
% \changes{italian-1.3}{2013/09/30}{The slide font fix up is not any
%     more necessary with the new caporali handling method.}
%
%    In the previous versions of this language description file the
%    absent guillemets were faked with other fonts, by taking example
%    from the solution the French had found for their language
%    description file; they would get suitable guillemets from the
%    cyrillic fonts; this solution was good in most cases, except when
%    the ``slides fonts'' were used, because there is no Cyryllic slide
%    font around.
% 
%    This might seem a negligible ``feature'' because the modern classes
%    or extension modules to produce slides mostly avoid the ``old'' fonts
%    for slides created by Leslie Lamport when he made available the macro
%    package LaTeX to the TeX community.
%
%    Since I designed renewed slide fonts extending those 
%    created by Leslie Lamport to the T1 encoding, the Text Companion
%    fonts, and the most frequent ``regular'' and AMS math fonts with the same
%    graphic style and excellent legibility (LXfonts), I thought that
%    this feature is not so negligible. It's true that nowadays nobody
%    should use the old OT1 encoding when typesetting in any language,
%    English included, because independently form the document main
%    language, it is very frequent to quote passages in other languages,
%    or to type foreign proper names of persons or places; nevertheless
%    having in mind a minimum of backwards compatibility and hoping
%    that the deliberate use of OT1 encoding (still necessary to typeset
%    mathematics) is being abandoned, I decided to simplify the previous
%    handling of guillemets.
%
%    Therefore here I will test at ``begin document'' only if the OT1
%    encoding is the default one, while if the T1 encoding is the default
%    one, that the font collection |AE| is not being used; should
%    it be the case, I will substitute the guillemets with the LaTeX
%    special symbols reduced to script size, and I will not try to fake
%    the guillemets with better solutions; evidently if OpenType fonts
%    are being used, nothing is done; so the tests that follow concern
%    only typesetting old documents or the lack of a wiser choice of fonts
%    and their encodings; an info message is issued and output to the
%    |.log| file. 
%
%    \begin{macro}{\LtxSymbCaporali}
%    \begin{macro}{\it@ocap}
%    \begin{macro}{\it@ccap}
%     First the macro |\LtxSymvCaporali| is defined so as to assign a
%     default definition of the faked guillemets: each one of these macro sets
%     actually redefines the control sequences |\it@ocap| and |\it@ccap| that
%     are the ones effectively activated by the shorthands |"<| and |">|. 
%     By default the caporali glyphs are taken from T1-encoded fonts; at
%     the end of the preamble some tests are performed to control if the
%     default fonts contain such glyphs, and in case a different font is chosen.
%    \begin{macrocode}
\def\LtxSymbCaporali{%
     \DeclareRobustCommand*{\it@ocap}{\mbox{%
        \fontencoding{U}\fontfamily{lasy}\selectfont(\kern-0.20em(}%
        \ignorespaces}%
     \DeclareRobustCommand*{\it@ccap}{\ifdim\lastskip>\z@\unskip\fi
     \mbox{%
        \fontencoding{U}\fontfamily{lasy}\selectfont)\kern-0.20em)}}%
}%
\def\T@unoCaporali{\DeclareRobustCommand*{\it@ocap}{<<\ignorespaces}%
     \DeclareRobustCommand*{\it@ccap}{\ifdim\lastskip>\z@\unskip\fi>>}}%
\T@unoCaporali
%    \end{macrocode}
%    Nevertheless a macro for choosing where to get glyphs for real guillemets
%    is offered; the syntax is the following:
%    \begin {flushleft}
%    |\CaporaliFrom|\marg{encoding}\marg{family}\marg{open guill. slot}%
%    \marg{close guill. slot}
%    \end{flushleft}
%    where \meta{encoding} and \meta{family} identify the font family
%    name of that particular encoding from which to get the missing
%    guillemets; \meta{open guill. slot} and \meta{close guill. slot}
%    are the (preferably) decimal slot addresses of the opening and
%    closing guillemets the user wants to use. For example if the 
%    T1-encoded Latin Modern fonts are desired, the specific command
%    should be
%     \begin {flushleft}
%     |\CaporaliFrom{T1}{lmr}{19}{20}|
%     \end{flushleft}
%     These user choices might be necessary for assuring the correct
%     typesetting
%     with fonts that contain the required glyphs and are available
%     also in PostScript form so as to use them directly with {pdfLaTeX},
%     for example.
%    \begin{macrocode}
\def\CaporaliFrom#1#2#3#4{%
  \DeclareFontEncoding{#1}{}{}%
  \DeclareTextCommand{\it@ocap}{T1}{%
    {\fontencoding{#1}\fontfamily{#2}\selectfont\char#3\ignorespaces}}%
  \DeclareTextCommand{\it@ccap}{T1}{\ifdim\lastskip>\z@\unskip\fi%
    {\fontencoding{#1}\fontfamily{#2}\selectfont\char#4}}%
  \DeclareTextCommand{\it@ocap}{OT1}{%
    {\fontencoding{#1}\fontfamily{#2}\selectfont\char#3\ignorespaces}}%
  \DeclareTextCommand{\it@ccap}{OT1}{\ifdim\lastskip>\z@\unskip\fi%
    {\fontencoding{#1}\fontfamily{#2}\selectfont\char#4}}}
%    \end{macrocode}
%    Notice that the above macro is strictly tied to the T1 encoding;
%    it won't do anything if the default encoding is not the T1 one.
%    Therefore if the |AE| font collection is being used it would be
%    good idea to issue the command shown above as an example in order
%    to get the proper guillemets\footnote{Actually the \texttt{AE}
%    fonts should not be used at all; the same results, more or less
%    are obtained by using the Latin Modern ones, that are not virtual
%    fonts and contain the whole T1 font scheme. Nevertheless the faked
%    glyphs are not so bad, so the solution I restored from old versions
%    of the language description file is acceptable}.

%     Then we set a boolean variable and test the default family;
%     if such family has a name that starts with the letters ``ae''
%     then we have no built in guillemets; of course if the AE font family is chosen
%     after the \babel\ package is loaded, the test does not perform as required.
%    \begin{macrocode}
\def\get@ae#1#2#3!{\def\bbl@ae{#1#2}}%
\def\@ifT@one@noCap{\expandafter\get@ae\f@family!%
\def\bbl@temp{ae}\ifx\bbl@ae\bbl@temp\expandafter\@firstoftwo\else
    \expandafter\@secondoftwo\fi}%
%    \end{macrocode}
%    Now we can set some real settings; first we start by testing
%    the encoding; if the encoding is OT1 we set the faked caporali
%    with LaTeX symbols and issue a warning; then we test if the font
%    family is the AE one we set again the faked caporali and issue
%    another warning\footnote{Notice the it is impossible to check
%    if the slots 19 and 20 of the AE fonts are defined by means of
%    the eTeX macro \texttt{\char92iffontchar}, because they are
%    actually defined as black squares!}; otherwise we set the commands
%    valid for the T1 encoding, that work well also with the TeX
%    Ligatures of the OpenType fonts.
%    \begin{macrocode}
\AtBeginDocument{\normalfont\def\bbl@temp{OT1}%
  \ifx\cf@encoding\bbl@temp
    \LtxSymbCaporali
    \GenericWarning{italian.ldf\space}{%
    File italian.ldf warning: \MessageBreak\space\space\space
    With OT1 encoding guillemets are poorly faked\MessageBreak
    \space\space\space
    Use T1 encoding\MessageBreak\space\space\space
    or specify a font with command \string\CaporaliFrom\MessageBreak
    \space\space\space
    See the documentation concerning the babel-italian typesetting
    \MessageBreak\space\space}%
  \else
    \ifx\cf@encoding\bbl@t@one
      \@ifT@one@noCap{%
        \LtxSymbCaporali
        \GenericWarning{italian.ldf\space}{%
        File italian.ldf warning: \MessageBreak\space\space\space
        The AE font collection does not contain the guillemets
        \MessageBreak\space\space\space
        Use the Latin Modern font collection instead
        \MessageBreak\space}
      }%
    {\T@unoCaporali}\fi
  \fi
}
%    \end{macrocode}
%    \end{macro}
%    \end{macro}
%    \end{macro}
%
%    \subsection*{Finishing commands}
%    The macro |\ldf@finish| takes care of looking for a
%    configuration file, setting the main language to be switched on
%    at |\begin{document}| and resetting the category code of
%    \texttt{@} to its original value.
% \changes{italian-1.2i}{1996/11/03}{Now use \cs{ldf@finish} to wrap
%    up}
%    \begin{macrocode}
\ldf@finish{italian}%
%    \end{macrocode}
%\iffalse
%</code>
%\fi
%
% \Finale
%%
%% \CharacterTable
%%  {Upper-case    \A\B\C\D\E\F\G\H\I\J\K\L\M\N\O\P\Q\R\S\T\U\V\W\X\Y\Z
%%   Lower-case    \a\b\c\d\e\f\g\h\i\j\k\l\m\n\o\p\q\r\s\t\u\v\w\x\y\z
%%   Digits        \0\1\2\3\4\5\6\7\8\9
%%   Exclamation   \!     Double quote  \"     Hash (number) \#
%%   Dollar        \$     Percent       \%     Ampersand     \&
%%   Acute accent  \'     Left paren    \(     Right paren   \)
%%   Asterisk      \*     Plus          \+     Comma         \,
%%   Minus         \-     Point         \.     Solidus       \/
%%   Colon         \:     Semicolon     \;     Less than     \<
%%   Equals        \=     Greater than  \>     Question mark \?
%%   Commercial at \@     Left bracket  \[     Backslash     \\
%%   Right bracket \]     Circumflex    \^     Underscore    \_
%%   Grave accent  \`     Left brace    \{     Vertical bar  \|
%%   Right brace   \}     Tilde         \~}
%%
\endinput
}
\DeclareOption{naustrian}{%%
%% This file will generate fast loadable files and documentation
%% driver files from the doc files in this package when run through
%% LaTeX or TeX.
%%
%% Copyright 1989--2016 Johannes L. Braams
%%                      Bernd Raichle
%%                      Walter Schmidt,
%%                      Juergen Spitzmueller
%% All rights reserved.
%% 
%% This file is part of the babel-german bundle,
%% an extension to the Babel system.
%% ----------------------------------------------
%% 
%% It may be distributed and/or modified under the
%% conditions of the LaTeX Project Public License, either version 1.3
%% of this license or (at your option) any later version.
%% The latest version of this license is in
%%   http://www.latex-project.org/lppl.txt
%% and version 1.3 or later is part of all distributions of LaTeX
%% version 2003/12/01 or later.
%% 
%% This work has the LPPL maintenance status "maintained".
%% 
%% The Current Maintainer of this work is Juergen Spitzmueller.
%%
%% --------------- start of docstrip commands ------------------
%%
\def\filedate{2016/11/01}

\input docstrip.tex

{\ifx\generate\undefined
\Msg{**********************************************}
\Msg{*}
\Msg{* This installation requires docstrip}
\Msg{* version 2.3c or later.}
\Msg{*}
\Msg{* An older version of docstrip has been input}
\Msg{*}
\Msg{**********************************************}
\errhelp{Move or rename old docstrip.tex.}
\errmessage{Old docstrip in input path}
\batchmode
\csname @@end\endcsname
\fi}

\preamble
This is a generated file.

Copyright 1989--2016 Johannes L. Braams
                     Bernd Raichle
                     Walter Schmidt,
                     Juergen Spitzmueller
All rights reserved.

This file is part of the babel-german bundle,
an extension to the Babel system.
----------------------------------------------

It may be distributed and/or modified under the
conditions of the LaTeX Project Public License, either version 1.3
of this license or (at your option) any later version.
The latest version of this license is in
  http://www.latex-project.org/lppl.txt
and version 1.3 or later is part of all distributions of LaTeX
version 2003/12/01 or later.

This work has the LPPL maintenance status "maintained".

The Current Maintainer of this work is Juergen Spitzmueller.

Please report errors to: Juergen Spitzmueller
                         juergen at spitzmueller dot org

\endpreamble

\keepsilent

\usedir{tex/generic/babel-german} 

\generate{\file{germanb.ldf}{\from{germanb.dtx}{germanb}}
          \file{german.ldf}{\from{germanb.dtx}{german}}
          \file{austrian.ldf}{\from{germanb.dtx}{austrian}}
          \file{swissgerman.ldf}{\from{germanb.dtx}{swiss}}
          \file{ngermanb.ldf}{\from{ngermanb.dtx}{germanb}}          
          \file{ngerman.ldf}{\from{ngermanb.dtx}{german}}
          \file{naustrian.ldf}{\from{ngermanb.dtx}{austrian}}
          \file{nswissgerman.ldf}{\from{ngermanb.dtx}{swiss}}
          }

\ifToplevel{
\Msg{***********************************************************}
\Msg{*}
\Msg{* To finish the installation you have to move the following}
\Msg{* files into a directory searched by TeX:}
\Msg{*}
\Msg{* \space\space austrian.ldf, german.ldf, germanb.ldf,}
\Msg{* \space\space naustrian.ldf, ngerman.ldf, ngermanb.ldf,}
\Msg{* \space\space nswissgerman.ldf and swissgerman.ldf}
\Msg{*}
\Msg{* To produce the documentation run the files }
\Msg{* germanb.dtx and ngermanb.dtx through LaTeX.}
\Msg{*}
\Msg{* Happy TeXing}
\Msg{***********************************************************}
}
 
\endbatchfile
}
\DeclareOption{newzealand}{% \iffalse meta-comment
%
% Copyright 1989-2005 Johannes L. Braams and any individual authors
% listed elsewhere in this file.  All rights reserved.
%    2013-2017 Javier Bezos, Johannes L. Braams
% This file is part of the Babel system.
% --------------------------------------
% 
% It may be distributed and/or modified under the
% conditions of the LaTeX Project Public License, either version 1.3
% of this license or (at your option) any later version.
% The latest version of this license is in
%   http://www.latex-project.org/lppl.txt
% and version 1.3 or later is part of all distributions of LaTeX
% version 2003/12/01 or later.
% 
% This work has the LPPL maintenance status "maintained".
% 
% The Current Maintainer of this work is Javier Bezos.
% 
% The list of all files belonging to the Babel system is
% given in the file `manifest.bbl. See also `legal.bbl' for additional
% information.
% 
% The list of derived (unpacked) files belonging to the distribution
% and covered by LPPL is defined by the unpacking scripts (with
% extension .ins) which are part of the distribution.
% \fi
% \iffalse
%    Tell the \LaTeX\ system who we are and write an entry on the
%    transcript.
%<*dtx>
\ProvidesFile{english.dtx}
%</dtx>
%<english>\ProvidesLanguage{english}
%<american>\ProvidesLanguage{american}
%<usenglish>\ProvidesLanguage{USenglish}
%<british>\ProvidesLanguage{british}
%<ukenglish>\ProvidesLanguage{UKenglish}
%<australian>\ProvidesLanguage{australian}
%<newzealand>\ProvidesLanguage{newzealand}
%<canadian>\ProvidesLanguage{canadian}
%\fi
%\ProvidesFile{english.dtx}
        [2017/06/06 v3.3r English support from the babel system]
%\iffalse
%% File 'english.dtx'
%% Babel package for LaTeX version 2e
%% Copyright (C) 1989 - 2005
%%           by Johannes Braams, TeXniek
%%           2013-2017 Javier Bezos, Johannes Braams
%
%
%    This file is part of the babel system, it provides the source
%    code for the English language definition file.
%<*filedriver>
\documentclass{ltxdoc}
\newcommand*\TeXhax{\TeX hax}
\newcommand*\babel{\textsf{babel}}
\newcommand*\langvar{$\langle \mathit lang \rangle$}
\newcommand*\note[1]{}
\newcommand*\Lopt[1]{\textsf{#1}}
\newcommand*\file[1]{\texttt{#1}}
\begin{document}
 \DocInput{english.dtx}
\end{document}
%</filedriver>
%\fi
% \GetFileInfo{english.dtx}
%
% \changes{english-2.0a}{1990/04/02}{Added checking of format}
% \changes{english-2.1}{1990/04/24}{Reflect changes in babel 2.1}
% \changes{english-2.1a}{1990/05/14}{Incorporated Nico's comments}
% \changes{english-2.1b}{1990/05/14}{merged \file{USenglish.sty} into
%    this file}
% \changes{english-2.1c}{1990/05/22}{fixed typo in definition for
%    american language found by Werenfried Spit (nspit@fys.ruu.nl)}
% \changes{english-2.1d}{1990/07/16}{Fixed some typos}
% \changes{english-3.0}{1991/04/23}{Modified for babel 3.0}
% \changes{english-3.0a}{1991/05/29}{Removed bug found by van der Meer}
% \changes{english-3.0c}{1991/07/15}{Renamed \file{babel.sty} in
%    \file{babel.com}}
% \changes{english-3.1}{1991/11/05}{Rewrote parts of the code to use
%    the new features of babel version 3.1}
% \changes{english-3.3}{1994/02/08}{Update or \LaTeXe}
% \changes{english-3.3c}{1994/06/26}{Removed the use of \cs{filedate}
%    and moved the identification after the loading of
%    \file{babel.def}}
% \changes{english-3.3g}{1996/07/10}{Replaced \cs{undefined} with
%    \cs{@undefined} and \cs{empty} with \cs{@empty} for consistency
%    with \LaTeX} 
% \changes{english-3.3h}{1996/10/10}{Moved the definition of
%    \cs{atcatcode} right to the beginning.} 
% \changes{english-3.3q}{2017/01/10}{Added the proxy files for the
%    dialects}
%
%  \section{The English language}
%
%    The file \file{\filename}\footnote{The file described in this
%    section has version number \fileversion\ and was last revised on
%    \filedate.} defines all the language definition macros for the
%    English language as well as for the American and Australian
%    version of this language. For the Australian version the British
%    hyphenation patterns will be used, if available, for the Canadian
%    variant the American patterns are selected.
%
%    For this language currently no special definitions are needed or
%    available.
%
% \StopEventually{}
%
%    The macro |\LdfInit| takes care of preventing that this file is
%    loaded more than once, checking the category code of the
%    \texttt{@} sign, etc.
% \changes{english-3.3h}{1996/11/02}{Now use \cs{LdfInit} to perform
%    initial checks} 
%    \begin{macrocode}
%<*code>
\LdfInit\CurrentOption{date\CurrentOption}
%    \end{macrocode}
%
%    When this file is read as an option, i.e. by the |\usepackage|
%    command, \texttt{english} could be an `unknown' language in which
%    case we have to make it known.  So we check for the existence of
%    |\l@english| to see whether we have to do something here.
%
% \changes{english-3.0}{1991/04/23}{Now use \cs{adddialect} if
%    language undefined}
% \changes{english-3.0d}{1991/10/22}{removed use of \cs{@ifundefined}}
% \changes{english-3.3c}{1994/06/26}{Now use \cs{@nopatterns} to
%    produce the warning}
% \changes{english-3.3g}{1996/07/10}{Allow british as the name of the
%    UK patterns}
% \changes{english-3.3j}{2000/01/21}{Also allow american english
%    hyphenation patterns to be used for `english'}
%    We allow for the british english patterns to be loaded as either
%    `british', or `UKenglish'. When neither of those is
%    known we try to define |\l@english| as an alias for |\l@american|
%    or |\l@USenglish|.
% \changes{english-3.3k}{2001/02/07}{Added support for canadian}
% \changes{english-3.3n}{2004/06/12}{Added support for australian and
%    newzealand} 
%    \begin{macrocode}
\ifx\l@english\@undefined
  \ifx\l@UKenglish\@undefined
    \ifx\l@british\@undefined
      \ifx\l@american\@undefined
        \ifx\l@USenglish\@undefined
          \ifx\l@canadian\@undefined
            \ifx\l@australian\@undefined
              \ifx\l@newzealand\@undefined
                \@nopatterns{English}
                \adddialect\l@english0
              \else
                \let\l@english\l@newzealand
              \fi
            \else
              \let\l@english\l@australian
            \fi
          \else
            \let\l@english\l@canadian
          \fi
        \else
          \let\l@english\l@USenglish
        \fi
      \else
        \let\l@english\l@american
      \fi
    \else
      \let\l@english\l@british
    \fi 
  \else
    \let\l@english\l@UKenglish
  \fi
\fi
%    \end{macrocode}
%    Because we allow `british' to be used as the babel option we need
%    to make sure that it will be recognised by |\selectlanguage|. In
%    the code above we have made sure that |\l@english| was defined.
%    Now we want to make sure that |\l@british| and |\l@UKenglish| are
%    defined as well. When either of them is we make them equal to
%    each other, when neither is we fall back to the default,
%    |\l@english|. 
% \changes{english-3.3o}{2004/06/14}{Make sure that british patterns
%    are used if they were loaded}
%    \begin{macrocode}
\ifx\l@british\@undefined
  \ifx\l@UKenglish\@undefined
    \adddialect\l@british\l@english
    \adddialect\l@UKenglish\l@english
  \else
    \let\l@british\l@UKenglish
  \fi
\else
  \let\l@UKenglish\l@british
\fi
%    \end{macrocode}
%    `American' is a version of `English' which can have its own
%    hyphenation patterns. The default english patterns are in fact
%    for american english. We allow for the patterns to be loaded as
%    `english' `american' or `USenglish'.
% \changes{english-3.0}{1990/04/23}{Now use \cs{adddialect} for
%    american}
% \changes{english-3.0b}{1991/06/06}{Removed \cs{global} definitions}
% \changes{english-3.3d}{1995/02/01}{Only define american as a
%    dialect when no separate patterns have been loaded}
% \changes{english-3.3g}{1996/07/10}{Allow USenglish as the name of
%    the american patterns} 
%    \begin{macrocode}
\ifx\l@american\@undefined
  \ifx\l@USenglish\@undefined
%    \end{macrocode}
%    When the patterns are not know as `american' or `USenglish' we
%    add a ``dialect''.
%    \begin{macrocode}
    \adddialect\l@american\l@english
  \else
    \let\l@american\l@USenglish
  \fi
\else
%    \end{macrocode}
%    Make sure that USenglish is known, even if the patterns were
%    loaded as `american'.
% \changes{english-3.3j}{2000/01/21}{Ensure that \cs{l@USenglish} is
%    alway defined}
% \changes{english-3.3l}{2001/04/15}{Added missing backslash}
%    \begin{macrocode}
  \ifx\l@USenglish\@undefined
    \let\l@USenglish\l@american
  \fi
\fi
%    \end{macrocode}
%
% \changes{english-3.3k}{2001/02/07}{Added support for canadian}
%    `Canadian' english spelling is a hybrid of British and American
%    spelling. Although so far no special `translations' have been
%    reported we allow this file to be loaded by the option
%    \Lopt{candian} as well.
%    \begin{macrocode}
\ifx\l@canadian\@undefined
  \adddialect\l@canadian\l@american
\fi
%    \end{macrocode}
%
% \changes{english-3.3n}{2004/06/12}{Added support for australian and
%   newzealand}
%    `Australian' and `New Zealand' english spelling seem to be the
%    same as British spelling. Although so far no special
%    `translations' have been reported we allow this file to be loaded
%    by the options \Lopt{australian} and \Lopt{newzealand} as well.
%    \begin{macrocode}
\ifx\l@australian\@undefined
  \adddialect\l@australian\l@british
\fi
\ifx\l@newzealand\@undefined
  \adddialect\l@newzealand\l@british
\fi
%    \end{macrocode}
%
 
%  \begin{macro}{\englishhyphenmins}
% \changes{english-3.3m}{2003/11/17}{Added default for setting of
%    hyphenmin parameters} 
%    This macro is used to store the correct values of the hyphenation
%    parameters |\lefthyphenmin| and |\righthyphenmin|.
%    \begin{macrocode}
\providehyphenmins{\CurrentOption}{\tw@\thr@@}
%    \end{macrocode}
%  \end{macro}
%
%    The next step consists of defining commands to switch to (and
%    from) the English language.
% \begin{macro}{\captionsenglish}
%    The macro |\captionsenglish| defines all strings used
%    in the four standard document classes provided with \LaTeX.
% \changes{english-3.0b}{1991/06/06}{Removed \cs{global} definitions}
% \changes{english-3.0b}{1991/06/06}{\cs{pagename} should be
%    \cs{headpagename}}
% \changes{english-3.1a}{1991/11/11}{added \cs{seename} and
%    \cs{alsoname}}
% \changes{english-3.1b}{1992/01/26}{added \cs{prefacename}}
% \changes{english-3.2}{1993/07/15}{\cs{headpagename} should be
%    \cs{pagename}}
% \changes{english-3.3e}{1995/07/04}{Added \cs{proofname} for
%    AMS-\LaTeX}
% \changes{english-3.3g}{1996/07/10}{Construct control sequence on the
%    fly} 
% \changes{english-3.3j}{2000/09/19}{Added \cs{glossaryname}}
%    \begin{macrocode}
\@namedef{captions\CurrentOption}{%
  \def\prefacename{Preface}%
  \def\refname{References}%
  \def\abstractname{Abstract}%
  \def\bibname{Bibliography}%
  \def\chaptername{Chapter}%
  \def\appendixname{Appendix}%
  \def\contentsname{Contents}%
  \def\listfigurename{List of Figures}%
  \def\listtablename{List of Tables}%
  \def\indexname{Index}%
  \def\figurename{Figure}%
  \def\tablename{Table}%
  \def\partname{Part}%
  \def\enclname{encl}%
  \def\ccname{cc}%
  \def\headtoname{To}%
  \def\pagename{Page}%
  \def\seename{see}%
  \def\alsoname{see also}%
  \def\proofname{Proof}%
  \def\glossaryname{Glossary}%
  }
%    \end{macrocode}
% \end{macro}
%
% \begin{macro}{\dateenglish}
%    In order to define |\today| correctly we need to know whether it
%    should be `english', `australian', or `american'. We can find
%    this out by checking the value of |\CurrentOption|.
% \changes{english-3.3j}{2000/01/21}{Make sure that the value of
%    \cs{today} is correct for both options `american' and
%    `USenglish'}
% \changes{english-3.3n}{2004/06/12}{Added support for `Australian'
%    and `Newzealand'}
% \changes{english-3.3o}{2004/06/14}{Explicitly choose the UK form of
%    date} 
% \changes{english-3.3p}{2012/11/07}{Warning if `english' is used with
%    other options} 
%    \begin{macrocode}
\def\bbl@tempa{british}
\ifx\CurrentOption\bbl@tempa\def\bbl@tempb{UK}\fi
\def\bbl@tempa{UKenglish}
\ifx\CurrentOption\bbl@tempa\def\bbl@tempb{UK}\fi
\def\bbl@tempa{american}
\ifx\CurrentOption\bbl@tempa\def\bbl@tempb{US}\fi
\def\bbl@tempa{USenglish}
\ifx\CurrentOption\bbl@tempa\def\bbl@tempb{US}\fi
\def\bbl@tempa{canadian}
\ifx\CurrentOption\bbl@tempa\def\bbl@tempb{US}\fi
\def\bbl@tempa{australian}
\ifx\CurrentOption\bbl@tempa\def\bbl@tempb{AU}\fi
\def\bbl@tempa{newzealand}
\ifx\CurrentOption\bbl@tempa\def\bbl@tempb{AU}\fi
\def\bbl@tempa{english}
\ifx\CurrentOption\bbl@tempa
  \AtEndOfPackage{\@nameuse{bbl@englishwarning}}
\else
  \edef\bbl@englishwarning{%
    \let\noexpand\bbl@englishwarning\relax
    \noexpand\PackageWarning{Babel}{%
      The package option `english' should not be used\noexpand\MessageBreak
      with a more specific one (like `\CurrentOption')}}
\fi
%    \end{macrocode}
%
%    The macro |\dateenglish| redefines the command |\today| to
%    produce English dates.
% \changes{english-3.0b}{1991/06/06}{Removed \cs{global} definitions}
% \changes{english-3.3g}{1996/07/10}{Construct control sequence on the
%    fly}
% \changes{english-3.3i}{1997/10/01}{Use \cs{edef} to define \cs{today}
%    to save memory}
% \changes{english-3.3i}{1998/03/28}{use \cs{def} instead of
%    \cs{edef}}
%    \begin{macrocode}
\def\bbl@tempa{UK}
\ifx\bbl@tempa\bbl@tempb
  \@namedef{date\CurrentOption}{%
    \def\today{\ifcase\day\or
      1st\or 2nd\or 3rd\or 4th\or 5th\or
      6th\or 7th\or 8th\or 9th\or 10th\or
      11th\or 12th\or 13th\or 14th\or 15th\or
      16th\or 17th\or 18th\or 19th\or 20th\or
      21st\or 22nd\or 23rd\or 24th\or 25th\or
      26th\or 27th\or 28th\or 29th\or 30th\or
      31st\fi~\ifcase\month\or
      January\or February\or March\or April\or May\or June\or
      July\or August\or September\or October\or November\or 
      December\fi\space \number\year}}
%    \end{macrocode}
% \end{macro}
%
% \begin{macro}{\dateaustralian}
%    Now, test for `australian' or `american'.
% \changes{english-3.3n}{2004/06/12}{Add australian date}
%    \begin{macrocode}
\else
%    \end{macrocode}
%
%    The macro |\dateaustralian| redefines the command |\today| to
%    produce Australian resp.\ New Zealand dates.
%    \begin{macrocode}
  \def\bbl@tempa{AU}
  \ifx\bbl@tempa\bbl@tempb
    \@namedef{date\CurrentOption}{%
      \def\today{\number\day~\ifcase\month\or
        January\or February\or March\or April\or May\or June\or
        July\or August\or September\or October\or November\or 
        December\fi\space \number\year}}
%    \end{macrocode}
% \end{macro}
%
% \begin{macro}{\dateamerican}
%    The macro |\dateamerican| redefines the command |\today| to
%    produce American dates.
% \changes{english-3.0b}{1991/06/06}{Removed \cs{global} definitions}
% \changes{english-3.3i}{1997/10/01}{Use \cs{edef} to define
%    \cs{today} to save memory}
% \changes{english-3.3i}{1998/03/28}{use \cs{def} instead of
%    \cs{edef}}
%    \begin{macrocode}
  \else
    \@namedef{date\CurrentOption}{%
      \def\today{\ifcase\month\or
        January\or February\or March\or April\or May\or June\or
        July\or August\or September\or October\or November\or
        December\fi \space\number\day, \number\year}}
  \fi
\fi
%    \end{macrocode}
% \end{macro}
%
% \begin{macro}{\extrasenglish}
% \begin{macro}{\noextrasenglish}
%    The macro |\extrasenglish| will perform all the extra definitions
%    needed for the English language. The macro |\noextrasenglish| is
%    used to cancel the actions of |\extrasenglish|.  For the moment
%    these macros are empty but they are defined for compatibility
%    with the other language definition files.
%
% \changes{english-3.3g}{1996/07/10}{Construct control sequences on
%    the fly} 
%    \begin{macrocode}
\@namedef{extras\CurrentOption}{}
\@namedef{noextras\CurrentOption}{}
%    \end{macrocode}
% \end{macro}
% \end{macro}
%
%    The macro |\ldf@finish| takes care of looking for a
%    configuration file, setting the main language to be switched on
%    at |\begin{document}| and resetting the category code of
%    \texttt{@} to its original value.
% \changes{english-3.3h}{1996/11/02}{Now use \cs{ldf@finish} to wrap
%    up} 
%    \begin{macrocode}
\ldf@finish\CurrentOption
%</code>
%    \end{macrocode}
%
% Finally, We create  a few proxy files, which just load english.ldf.
%
%    \begin{macrocode}
%<*american|usenglish|british|ukenglish|australian|newzealand|canadian>
\input english.ldf\relax
%</american|usenglish|british|ukenglish|australian|newzealand|canadian>
%    \end{macrocode}
%
% \Finale
%%
%% \CharacterTable
%%  {Upper-case    \A\B\C\D\E\F\G\H\I\J\K\L\M\N\O\P\Q\R\S\T\U\V\W\X\Y\Z
%%   Lower-case    \a\b\c\d\e\f\g\h\i\j\k\l\m\n\o\p\q\r\s\t\u\v\w\x\y\z
%%   Digits        \0\1\2\3\4\5\6\7\8\9
%%   Exclamation   \!     Double quote  \"     Hash (number) \#
%%   Dollar        \$     Percent       \%     Ampersand     \&
%%   Acute accent  \'     Left paren    \(     Right paren   \)
%%   Asterisk      \*     Plus          \+     Comma         \,
%%   Minus         \-     Point         \.     Solidus       \/
%%   Colon         \:     Semicolon     \;     Less than     \<
%%   Equals        \=     Greater than  \>     Question mark \?
%%   Commercial at \@     Left bracket  \[     Backslash     \\
%%   Right bracket \]     Circumflex    \^     Underscore    \_
%%   Grave accent  \`     Left brace    \{     Vertical bar  \|
%%   Right brace   \}     Tilde         \~}
%%
\endinput
}
\DeclareOption{ngerman}{%%
%% This file will generate fast loadable files and documentation
%% driver files from the doc files in this package when run through
%% LaTeX or TeX.
%%
%% Copyright 1989--2016 Johannes L. Braams
%%                      Bernd Raichle
%%                      Walter Schmidt,
%%                      Juergen Spitzmueller
%% All rights reserved.
%% 
%% This file is part of the babel-german bundle,
%% an extension to the Babel system.
%% ----------------------------------------------
%% 
%% It may be distributed and/or modified under the
%% conditions of the LaTeX Project Public License, either version 1.3
%% of this license or (at your option) any later version.
%% The latest version of this license is in
%%   http://www.latex-project.org/lppl.txt
%% and version 1.3 or later is part of all distributions of LaTeX
%% version 2003/12/01 or later.
%% 
%% This work has the LPPL maintenance status "maintained".
%% 
%% The Current Maintainer of this work is Juergen Spitzmueller.
%%
%% --------------- start of docstrip commands ------------------
%%
\def\filedate{2016/11/01}

\input docstrip.tex

{\ifx\generate\undefined
\Msg{**********************************************}
\Msg{*}
\Msg{* This installation requires docstrip}
\Msg{* version 2.3c or later.}
\Msg{*}
\Msg{* An older version of docstrip has been input}
\Msg{*}
\Msg{**********************************************}
\errhelp{Move or rename old docstrip.tex.}
\errmessage{Old docstrip in input path}
\batchmode
\csname @@end\endcsname
\fi}

\preamble
This is a generated file.

Copyright 1989--2016 Johannes L. Braams
                     Bernd Raichle
                     Walter Schmidt,
                     Juergen Spitzmueller
All rights reserved.

This file is part of the babel-german bundle,
an extension to the Babel system.
----------------------------------------------

It may be distributed and/or modified under the
conditions of the LaTeX Project Public License, either version 1.3
of this license or (at your option) any later version.
The latest version of this license is in
  http://www.latex-project.org/lppl.txt
and version 1.3 or later is part of all distributions of LaTeX
version 2003/12/01 or later.

This work has the LPPL maintenance status "maintained".

The Current Maintainer of this work is Juergen Spitzmueller.

Please report errors to: Juergen Spitzmueller
                         juergen at spitzmueller dot org

\endpreamble

\keepsilent

\usedir{tex/generic/babel-german} 

\generate{\file{germanb.ldf}{\from{germanb.dtx}{germanb}}
          \file{german.ldf}{\from{germanb.dtx}{german}}
          \file{austrian.ldf}{\from{germanb.dtx}{austrian}}
          \file{swissgerman.ldf}{\from{germanb.dtx}{swiss}}
          \file{ngermanb.ldf}{\from{ngermanb.dtx}{germanb}}          
          \file{ngerman.ldf}{\from{ngermanb.dtx}{german}}
          \file{naustrian.ldf}{\from{ngermanb.dtx}{austrian}}
          \file{nswissgerman.ldf}{\from{ngermanb.dtx}{swiss}}
          }

\ifToplevel{
\Msg{***********************************************************}
\Msg{*}
\Msg{* To finish the installation you have to move the following}
\Msg{* files into a directory searched by TeX:}
\Msg{*}
\Msg{* \space\space austrian.ldf, german.ldf, germanb.ldf,}
\Msg{* \space\space naustrian.ldf, ngerman.ldf, ngermanb.ldf,}
\Msg{* \space\space nswissgerman.ldf and swissgerman.ldf}
\Msg{*}
\Msg{* To produce the documentation run the files }
\Msg{* germanb.dtx and ngermanb.dtx through LaTeX.}
\Msg{*}
\Msg{* Happy TeXing}
\Msg{***********************************************************}
}
 
\endbatchfile
}
\DeclareOption{norsk}{% \iffalse meta-comment
%
% Copyright 1989-2005 Johannes L. Braams and any individual authors
% listed elsewhere in this file.  All rights reserved.
% 
% This file is part of the Babel system.
% --------------------------------------
% 
% It may be distributed and/or modified under the
% conditions of the LaTeX Project Public License, either version 1.3
% of this license or (at your option) any later version.
% The latest version of this license is in
%   http://www.latex-project.org/lppl.txt
% and version 1.3 or later is part of all distributions of LaTeX
% version 2003/12/01 or later.
% 
% This work has the LPPL maintenance status "maintained".
% 
% The Current Maintainer of this work is Johannes Braams.
% 
% The list of all files belonging to the Babel system is
% given in the file `manifest.bbl. See also `legal.bbl' for additional
% information.
% 
% The list of derived (unpacked) files belonging to the distribution
% and covered by LPPL is defined by the unpacking scripts (with
% extension .ins) which are part of the distribution.
% \fi
%\CheckSum{305}
% \iffalse
%    Tell the \LaTeX\ system who we are and write an entry on the
%    transcript.
%<*dtx>
\ProvidesFile{norsk.dtx}
%</dtx>
%<code>\ProvidesLanguage{norsk}
%\fi
%\ProvidesFile{norsk.dtx}
        [2012/08/06 v2.0i Norsk support from the babel system]
%\iffalse
%%File `norsk.dtx'
%% Babel package for LaTeX version 2e
%% Copyright (C) 1989 - 2005
%%           by Johannes Braams, TeXniek
%
%% Please report errors to: J.L. Braams
%%                          babel at braams.cistron.nl
%
%    This file is part of the babel system, it provides the source
%    code for the Norwegian language definition file.  Contributions
%    were made by Haavard Helstrup (HAAVARD@CERNVM) and Alv Kjetil
%    Holme (HOLMEA@CERNVM); the `nynorsk' variant has been supplied by
%    Per Steinar Iversen (iversen@vxcern.cern.ch) and Terje Engeset
%    Petterst (TERJEEP@VSFYS1.FI.UIB.NO)
%
%    Rune Kleveland (runekl at math.uio.no) added the shorthand
%    definitions 
%<*filedriver>
\documentclass{ltxdoc}
\newcommand*\TeXhax{\TeX hax}
\newcommand*\babel{\textsf{babel}}
\newcommand*\langvar{$\langle \it lang \rangle$}
\newcommand*\note[1]{}
\newcommand*\Lopt[1]{\textsf{#1}}
\newcommand*\file[1]{\texttt{#1}}
\begin{document}
 \DocInput{norsk.dtx}
\end{document}
%</filedriver>
%\fi
% \GetFileInfo{norsk.dtx}
%
% \changes{norsk-1.0a}{1991/07/15}{Renamed \file{babel.sty} in
%    \file{babel.com}}
% \changes{norsk-1.1a}{1992/02/16}{Brought up-to-date with babel 3.2a}
% \changes{norsk-1.1c}{1993/11/11}{Added a couple of translations
%    (from Per Norman Oma, TeX@itk.unit.no)}
% \changes{norsk-1.2a}{1994/02/27}{Update for \LaTeXe}
% \changes{norsk-1.2d}{1994/06/26}{Removed the use of \cs{filedate}
%    and moved identification after the loading of \file{babel.def}}
% \changes{norsk-1.2h}{1996/07/12}{Replaced \cs{undefined} with
%    \cs{@undefined} and \cs{empty} with \cs{@empty} for consistency
%    with \LaTeX} 
% \changes{norsk-1.2h}{1996/10/10}{Moved the definition of
%    \cs{atcatcode} right to the beginning.}
%
%
%  \section{The Norwegian language}
%
%    The file \file{\filename}\footnote{The file described in this
%    section has version number \fileversion\ and was last revised on
%    \filedate.  Contributions were made by Haavard Helstrup
%    (\texttt{HAAVARD@CERNVM)} and Alv Kjetil Holme
%    (\texttt{HOLMEA@CERNVM}); the `nynorsk' variant has been supplied
%    by Per Steinar Iversen \texttt{iversen@vxcern.cern.ch}) and Terje
%    Engeset Petterst (\texttt{TERJEEP@VSFYS1.FI.UIB.NO)}; the
%    shorthand definitions were provided by Rune Kleveland
%    (\texttt{runekl@math.uio.no}).} defines all the language definition
%    macros for the Norwegian language as well as for an alternative
%    variant `nynorsk' of this language. 
%
%    For this language the character |"| is made active. In
%    table~\ref{tab:norsk-quote} an overview is given of its purpose.
%    \begin{table}[htb]
%     \begin{center}
%     \begin{tabular}{lp{.7\textwidth}}
%      |"ff|& for |ff| to be hyphenated as |ff-f|,
%             this is also implemented for b, d, f, g, l, m, n,
%             p, r, s, and t. (|o"ppussing|)                        \\
%      |"ee|& Hyphenate |"ee| as |\'e-e|. (|komit"een|)             \\
%      |"-| & an explicit hyphen sign, allowing hyphenation in the
%             composing words. Use this for compound words when the
%             hyphenation patterns fail to hyphenate
%             properly. (|alpin"-anlegg|)                           \\
%      \verb="|= & Like |"-|, but inserts 0.03em space.  Use it if
%             the compound point is spanned by a ligature.
%             (\verb=hoff"|intriger=)                               \\
%      |""| & Like |"-|, but producing no hyphen sign.
%             (|i""g\aa{}r|)                                        \\
%      |"~| & Like |-|, but allows no hyphenation at all. (|E"~cup|)\\
%      |"=| & Like |-|, but allowing hyphenation in the composing
%             words. (|marksistisk"=leninistisk|)                   \\
%      |"<| & for French left double quotes (similar to $<<$).      \\
%      |">| & for French right double quotes (similar to $>>$).     \\
%     \end{tabular}
%     \caption{The extra definitions made
%              by \file{norsk.sty}}\label{tab:norsk-quote}
%     \end{center}
%    \end{table}
% \changes{norsk-2.0a}{1998/06/24}{Describe the use of double quote as
%    active character}
%
%    Rune Kleveland distributes a Norwegian dictionary for ispell
%    (570000 words). It can be found at
%    |http://www.uio.no/~runekl/dictionary.html|. 
%
%    This dictionary supports the spellings |spi"sslede| for
%    `spisslede' (hyphenated spiss-slede) and other such words, and
%    also suggest the spelling |spi"sslede| for `spisslede' and
%    `spissslede'.
%
% \StopEventually{}
%
%    The macro |\LdfInit| takes care of preventing that this file is
%    loaded more than once, checking the category code of the
%    \texttt{@} sign, etc.
% \changes{norsk-1.2h}{1996/11/03}{Now use \cs{LdfInit} to perform
%    initial checks} 
%    \begin{macrocode}
%<*code>
\LdfInit\CurrentOption{captions\CurrentOption}
%    \end{macrocode}
%
%    When this file is read as an option, i.e. by the |\usepackage|
%    command, \texttt{norsk} will be an `unknown' language in which
%    case we have to make it known.  So we check for the existence of
%    |\l@norsk| to see whether we have to do something here.
%
% \changes{norsk-1.0c}{1991/10/29}{Removed use of \cs{@ifundefined}}
% \changes{norsk-1.1a}{1992/02/16}{Added a warning when no hyphenation
%    patterns were loaded.}
% \changes{norsk-1.2d}{1994/06/26}{Now use \cs{@nopatterns} to produce
%    the warning}
%    \begin{macrocode}
\ifx\l@norsk\@undefined
    \@nopatterns{Norsk}
    \adddialect\l@norsk0\fi
%    \end{macrocode}
%
%  \begin{macro}{\norskhyphenmins}
%     Some sets of Norwegian hyphenation patterns can be used with
%     |\lefthyphenmin| set to~1 and |\righthyphenmin| set to~2, but
%     the most common set |nohyph.tex| can't.  So we use
%     |\lefthyphenmin=2| by default.
% \changes{norsk-1.2f}{1995/07/02}{Added setting of hyphenmin
%    parameters}
% \changes{norsk-2.0a}{1998/06/24}{Changed setting of hyphenmin
%    parameters to 2~2} 
% \changes{norsk-2.0e}{2000/09/22}{Now use \cs{providehyphenmins} to
%    provide a default value}
%    \begin{macrocode}
\providehyphenmins{\CurrentOption}{\tw@\tw@}
%    \end{macrocode}
%  \end{macro}
%
%    Now we have to decide which version of the captions should be
%    made available. This can be done by checking the contents of
%    |\CurrentOption|. 
%    \begin{macrocode}
\def\bbl@tempa{norsk}
\ifx\CurrentOption\bbl@tempa
%    \end{macrocode}
%
%    The next step consists of defining commands to switch to (and
%    from) the Norwegian language.
%
% \begin{macro}{\captionsnorsk}
%    The macro |\captionsnorsk| defines all strings used
%    in the four standard documentclasses provided with \LaTeX.
% \changes{norsk-1.1a}{1992/02/16}{Added \cs{seename}, \cs{alsoname} and
%    \cs{prefacename}}
% \changes{norsk-1.1b}{1993/07/15}{\cs{headpagename} should be
%    \cs{pagename}}
% \changes{norsk-1.2f}{1995/07/02}{Added \cs{proofname} for
%    AMS-\LaTeX}
% \changes{norsk-1.2g}{1996/04/01}{Replaced `Proof' by its
%    translation} 
% \changes{norsk-2.0e}{2000/09/20}{Added \cs{glossaryname}}
% \changes{norsk-2.0g}{1996/04/01}{Replaced `Glossary' by its
%    translation} 
%    \begin{macrocode}
  \def\captionsnorsk{%
    \def\prefacename{Forord}%
    \def\refname{Referanser}%
    \def\abstractname{Sammendrag}%
    \def\bibname{Bibliografi}%     or Litteraturoversikt
    %                              or Litteratur or Referanser
    \def\chaptername{Kapittel}%
    \def\appendixname{Tillegg}%    or Appendiks
    \def\contentsname{Innhold}%
    \def\listfigurename{Figurer}%  or Figurliste
    \def\listtablename{Tabeller}%  or Tabelliste
    \def\indexname{Register}%
    \def\figurename{Figur}%
    \def\tablename{Tabell}%
    \def\partname{Del}%
    \def\enclname{Vedlegg}%
    \def\ccname{Kopi sendt}%
    \def\headtoname{Til}% in letter
    \def\pagename{Side}%
    \def\seename{Se}%
    \def\alsoname{Se ogs\aa{}}%
    \def\proofname{Bevis}%
    \def\glossaryname{Ordliste}%
    }
\else
%    \end{macrocode}
% \end{macro}
%
%    For the `nynorsk' version of these definitions we just add a
%    ``dialect''.
%    \begin{macrocode}
  \adddialect\l@nynorsk\l@norsk
%    \end{macrocode}
%
% \begin{macro}{\captionsnynorsk}
%    The macro |\captionsnynorsk| defines all strings used in the four
%    standard documentclasses provided with \LaTeX, but using a
%    different spelling than in the command |\captionsnorsk|.
% \changes{norsk-1.1a}{1992/02/16}{Added \cs{seename}, \cs{alsoname} and
%    \cs{prefacename}}
% \changes{norsk-1.1b}{1993/07/15}{\cs{headpagename} should be
%    \cs{pagename}}
% \changes{norsk-1.2g}{1996/04/01}{Replaced `Proof' by its
%    translation} 
% \changes{norsk-2.0e}{2000/09/20}{Added \cs{glossaryname}}
% \changes{norsk-2.0g}{1996/04/01}{Replaced `Glossary' by its
%    translation} 
% \changes{norks-2.0h}{2001/01/12}{Changed \cs{ccname} and \cs{alsoname}}
%    \begin{macrocode}
  \def\captionsnynorsk{%
    \def\prefacename{Forord}%
    \def\refname{Referansar}%
    \def\abstractname{Samandrag}%
    \def\bibname{Litteratur}%     or Litteraturoversyn
     %                             or Referansar
    \def\chaptername{Kapittel}%
    \def\appendixname{Tillegg}%   or Appendiks
    \def\contentsname{Innhald}%
    \def\listfigurename{Figurar}% or Figurliste
    \def\listtablename{Tabellar}% or Tabelliste
    \def\indexname{Register}%
    \def\figurename{Figur}%
    \def\tablename{Tabell}%
    \def\partname{Del}%
    \def\enclname{Vedlegg}%
    \def\ccname{Kopi til}%
    \def\headtoname{Til}% in letter
    \def\pagename{Side}%
    \def\seename{Sj\aa{}}%
    \def\alsoname{Sj\aa{} \`{o}g}%
    \def\proofname{Bevis}%
    \def\glossaryname{Ordliste}%
    }
\fi
%    \end{macrocode}
% \end{macro}
%
% \begin{macro}{\datenorsk}
%    The macro |\datenorsk| redefines the command |\today| to produce
%    Norwegian dates.
% \changes{norsk-1.2i}{1997/10/01}{Use \cs{edef} to define
%    \cs{today} to save memory}
% \changes{norsk-1.2i}{1998/03/28}{use \cs{def} instead of \cs{edef}}
% \changes{norsk-2.0i}{2012/08/06}{Removed extra space after `desember'}
%    \begin{macrocode}
\@namedef{date\CurrentOption}{%
  \def\today{\number\day.~\ifcase\month\or
    januar\or februar\or mars\or april\or mai\or juni\or
    juli\or august\or september\or oktober\or november\or
    desember\fi
    \space\number\year}}
%    \end{macrocode}
% \end{macro}
%
% \begin{macro}{\extrasnorsk}
% \begin{macro}{\extrasnynorsk}
%    The macro |\extrasnorsk| will perform all the extra definitions
%    needed for the Norwegian language. The macro |\noextrasnorsk| is
%    used to cancel the actions of |\extrasnorsk|.  
%
%    Norwegian typesetting requires |\frencspacing| to be in effect.
%    \begin{macrocode}
\@namedef{extras\CurrentOption}{\bbl@frenchspacing}
\@namedef{noextras\CurrentOption}{\bbl@nonfrenchspacing}
%    \end{macrocode}
% \end{macro}
% \end{macro}
%
%    For Norsk the \texttt{"} character is made active. This is done
%    once, later on its definition may vary.
% \changes{norsk-2.0a}{1998/06/24}{Made double quote character active}
%    \begin{macrocode}
\initiate@active@char{"}
\expandafter\addto\csname extras\CurrentOption\endcsname{%
  \languageshorthands{norsk}}
\expandafter\addto\csname extras\CurrentOption\endcsname{%
  \bbl@activate{"}}
%    \end{macrocode}
%    Don't forget to turn the shorthands off again.
% \changes{norsk-2.0c}{1999/12/17}{Deactivate shorthands ouside of
%    Norsk}
%    \begin{macrocode}
\expandafter\addto\csname noextras\CurrentOption\endcsname{%
  \bbl@deactivate{"}}
%    \end{macrocode}
%
%    The code above is necessary because we need to define a number of
%    shorthand commands. These sharthand commands are then used as
%    indicated in table~\ref{tab:norsk-quote}.
%
%    To be able to define the function of |"|, we first define a
%    couple of `support' macros.
%
%  \begin{macro}{\dq}
%    We save the original double quote character in |\dq| to keep
%    it available, the math accent |\"| can now be typed as |"|.
%    \begin{macrocode}
\begingroup \catcode`\"12
\def\x{\endgroup
  \def\@SS{\mathchar"7019 }
  \def\dq{"}}
\x
%    \end{macrocode}
%  \end{macro}
%
%    Now we can define the discretionary shorthand commands.
%    The number of words where such hyphenation is required is for
%    each character
%    \begin{center}
%      \begin{tabular}{*{11}c}
%        b&d&f &g&k &l &n&p &r&s &t \\
%        4&4&15&3&43&30&8&12&1&33&35
%       \end{tabular}
%    \end{center}
%    taken from a list of 83000 ispell-roots.
%
% \changes{norsk-2.0d}{2000/02/29}{Shorthands are the same for both
%    spelling variants, no need to use \cs{CurrentOption}}
%    \begin{macrocode}
\declare@shorthand{norsk}{"b}{\textormath{\bbl@disc b{bb}}{b}}
\declare@shorthand{norsk}{"B}{\textormath{\bbl@disc B{BB}}{B}}
\declare@shorthand{norsk}{"d}{\textormath{\bbl@disc d{dd}}{d}}
\declare@shorthand{norsk}{"D}{\textormath{\bbl@disc D{DD}}{D}}
\declare@shorthand{norsk}{"e}{\textormath{\bbl@disc e{\'e}}{}}
\declare@shorthand{norsk}{"E}{\textormath{\bbl@disc E{\'E}}{}}
\declare@shorthand{norsk}{"F}{\textormath{\bbl@disc F{FF}}{F}}
\declare@shorthand{norsk}{"g}{\textormath{\bbl@disc g{gg}}{g}}
\declare@shorthand{norsk}{"G}{\textormath{\bbl@disc G{GG}}{G}}
\declare@shorthand{norsk}{"k}{\textormath{\bbl@disc k{kk}}{k}}
\declare@shorthand{norsk}{"K}{\textormath{\bbl@disc K{KK}}{K}}
\declare@shorthand{norsk}{"l}{\textormath{\bbl@disc l{ll}}{l}}
\declare@shorthand{norsk}{"L}{\textormath{\bbl@disc L{LL}}{L}}
\declare@shorthand{norsk}{"n}{\textormath{\bbl@disc n{nn}}{n}}
\declare@shorthand{norsk}{"N}{\textormath{\bbl@disc N{NN}}{N}}
\declare@shorthand{norsk}{"p}{\textormath{\bbl@disc p{pp}}{p}}
\declare@shorthand{norsk}{"P}{\textormath{\bbl@disc P{PP}}{P}}
\declare@shorthand{norsk}{"r}{\textormath{\bbl@disc r{rr}}{r}}
\declare@shorthand{norsk}{"R}{\textormath{\bbl@disc R{RR}}{R}}
\declare@shorthand{norsk}{"s}{\textormath{\bbl@disc s{ss}}{s}}
\declare@shorthand{norsk}{"S}{\textormath{\bbl@disc S{SS}}{S}}
\declare@shorthand{norsk}{"t}{\textormath{\bbl@disc t{tt}}{t}}
\declare@shorthand{norsk}{"T}{\textormath{\bbl@disc T{TT}}{T}}
%    \end{macrocode}
%    We need to treat |"f| a bit differently in order to preserve the
%    ff-ligature. 
% \changes{norsk-2.0b}{1999/11/19}{Copied the coding for \texttt{"f}
%    from germanb.dtx version 2.6g} 
%    \begin{macrocode}
\declare@shorthand{norsk}{"f}{\textormath{\bbl@discff}{f}}
\def\bbl@discff{\penalty\@M
  \afterassignment\bbl@insertff \let\bbl@nextff= }
\def\bbl@insertff{%
  \if f\bbl@nextff
    \expandafter\@firstoftwo\else\expandafter\@secondoftwo\fi
  {\relax\discretionary{ff-}{f}{ff}\allowhyphens}{f\bbl@nextff}}
\let\bbl@nextff=f
%    \end{macrocode}
%    We now  define the French double quotes and some commands 
%    concerning hyphenation:
% \changes{norsk-2.0b}{1999/11/22}{added the french double quotes}
% \changes{norsk-2.0d}{2000/01/28}{Use \cs{bbl@allowhyphens} in
%    \texttt{"-}}
%    \begin{macrocode}
\declare@shorthand{norsk}{"<}{\flqq}
\declare@shorthand{norsk}{">}{\frqq}
\declare@shorthand{norsk}{"-}{\penalty\@M\-\bbl@allowhyphens}
\declare@shorthand{norsk}{"|}{%
  \textormath{\penalty\@M\discretionary{-}{}{\kern.03em}%
              \allowhyphens}{}}
\declare@shorthand{norsk}{""}{\hskip\z@skip}
\declare@shorthand{norsk}{"~}{\textormath{\leavevmode\hbox{-}}{-}}
\declare@shorthand{norsk}{"=}{\penalty\@M-\hskip\z@skip}
%    \end{macrocode}
%
%    The macro |\ldf@finish| takes care of looking for a
%    configuration file, setting the main language to be switched on
%    at |\begin{document}| and resetting the category code of
%    \texttt{@} to its original value.
% \changes{norsk-1.2h}{1996/11/03}{Now use \cs{ldf@finish} to wrap up}
%    \begin{macrocode}
\ldf@finish\CurrentOption
%</code>
%    \end{macrocode}
%
% \Finale
%%
%% \CharacterTable
%%  {Upper-case    \A\B\C\D\E\F\G\H\I\J\K\L\M\N\O\P\Q\R\S\T\U\V\W\X\Y\Z
%%   Lower-case    \a\b\c\d\e\f\g\h\i\j\k\l\m\n\o\p\q\r\s\t\u\v\w\x\y\z
%%   Digits        \0\1\2\3\4\5\6\7\8\9
%%   Exclamation   \!     Double quote  \"     Hash (number) \#
%%   Dollar        \$     Percent       \%     Ampersand     \&
%%   Acute accent  \'     Left paren    \(     Right paren   \)
%%   Asterisk      \*     Plus          \+     Comma         \,
%%   Minus         \-     Point         \.     Solidus       \/
%%   Colon         \:     Semicolon     \;     Less than     \<
%%   Equals        \=     Greater than  \>     Question mark \?
%%   Commercial at \@     Left bracket  \[     Backslash     \\
%%   Right bracket \]     Circumflex    \^     Underscore    \_
%%   Grave accent  \`     Left brace    \{     Vertical bar  \|
%%   Right brace   \}     Tilde         \~}
%%
\endinput
}
\DeclareOption{norwegian}{% \iffalse meta-comment
%
% Copyright 1989-2005 Johannes L. Braams and any individual authors
% listed elsewhere in this file.  All rights reserved.
% 
% This file is part of the Babel system.
% --------------------------------------
% 
% It may be distributed and/or modified under the
% conditions of the LaTeX Project Public License, either version 1.3
% of this license or (at your option) any later version.
% The latest version of this license is in
%   http://www.latex-project.org/lppl.txt
% and version 1.3 or later is part of all distributions of LaTeX
% version 2003/12/01 or later.
% 
% This work has the LPPL maintenance status "maintained".
% 
% The Current Maintainer of this work is Johannes Braams.
% 
% The list of all files belonging to the Babel system is
% given in the file `manifest.bbl. See also `legal.bbl' for additional
% information.
% 
% The list of derived (unpacked) files belonging to the distribution
% and covered by LPPL is defined by the unpacking scripts (with
% extension .ins) which are part of the distribution.
% \fi
%\CheckSum{305}
% \iffalse
%    Tell the \LaTeX\ system who we are and write an entry on the
%    transcript.
%<*dtx>
\ProvidesFile{norsk.dtx}
%</dtx>
%<code>\ProvidesLanguage{norsk}
%\fi
%\ProvidesFile{norsk.dtx}
        [2012/08/06 v2.0i Norsk support from the babel system]
%\iffalse
%%File `norsk.dtx'
%% Babel package for LaTeX version 2e
%% Copyright (C) 1989 - 2005
%%           by Johannes Braams, TeXniek
%
%% Please report errors to: J.L. Braams
%%                          babel at braams.cistron.nl
%
%    This file is part of the babel system, it provides the source
%    code for the Norwegian language definition file.  Contributions
%    were made by Haavard Helstrup (HAAVARD@CERNVM) and Alv Kjetil
%    Holme (HOLMEA@CERNVM); the `nynorsk' variant has been supplied by
%    Per Steinar Iversen (iversen@vxcern.cern.ch) and Terje Engeset
%    Petterst (TERJEEP@VSFYS1.FI.UIB.NO)
%
%    Rune Kleveland (runekl at math.uio.no) added the shorthand
%    definitions 
%<*filedriver>
\documentclass{ltxdoc}
\newcommand*\TeXhax{\TeX hax}
\newcommand*\babel{\textsf{babel}}
\newcommand*\langvar{$\langle \it lang \rangle$}
\newcommand*\note[1]{}
\newcommand*\Lopt[1]{\textsf{#1}}
\newcommand*\file[1]{\texttt{#1}}
\begin{document}
 \DocInput{norsk.dtx}
\end{document}
%</filedriver>
%\fi
% \GetFileInfo{norsk.dtx}
%
% \changes{norsk-1.0a}{1991/07/15}{Renamed \file{babel.sty} in
%    \file{babel.com}}
% \changes{norsk-1.1a}{1992/02/16}{Brought up-to-date with babel 3.2a}
% \changes{norsk-1.1c}{1993/11/11}{Added a couple of translations
%    (from Per Norman Oma, TeX@itk.unit.no)}
% \changes{norsk-1.2a}{1994/02/27}{Update for \LaTeXe}
% \changes{norsk-1.2d}{1994/06/26}{Removed the use of \cs{filedate}
%    and moved identification after the loading of \file{babel.def}}
% \changes{norsk-1.2h}{1996/07/12}{Replaced \cs{undefined} with
%    \cs{@undefined} and \cs{empty} with \cs{@empty} for consistency
%    with \LaTeX} 
% \changes{norsk-1.2h}{1996/10/10}{Moved the definition of
%    \cs{atcatcode} right to the beginning.}
%
%
%  \section{The Norwegian language}
%
%    The file \file{\filename}\footnote{The file described in this
%    section has version number \fileversion\ and was last revised on
%    \filedate.  Contributions were made by Haavard Helstrup
%    (\texttt{HAAVARD@CERNVM)} and Alv Kjetil Holme
%    (\texttt{HOLMEA@CERNVM}); the `nynorsk' variant has been supplied
%    by Per Steinar Iversen \texttt{iversen@vxcern.cern.ch}) and Terje
%    Engeset Petterst (\texttt{TERJEEP@VSFYS1.FI.UIB.NO)}; the
%    shorthand definitions were provided by Rune Kleveland
%    (\texttt{runekl@math.uio.no}).} defines all the language definition
%    macros for the Norwegian language as well as for an alternative
%    variant `nynorsk' of this language. 
%
%    For this language the character |"| is made active. In
%    table~\ref{tab:norsk-quote} an overview is given of its purpose.
%    \begin{table}[htb]
%     \begin{center}
%     \begin{tabular}{lp{.7\textwidth}}
%      |"ff|& for |ff| to be hyphenated as |ff-f|,
%             this is also implemented for b, d, f, g, l, m, n,
%             p, r, s, and t. (|o"ppussing|)                        \\
%      |"ee|& Hyphenate |"ee| as |\'e-e|. (|komit"een|)             \\
%      |"-| & an explicit hyphen sign, allowing hyphenation in the
%             composing words. Use this for compound words when the
%             hyphenation patterns fail to hyphenate
%             properly. (|alpin"-anlegg|)                           \\
%      \verb="|= & Like |"-|, but inserts 0.03em space.  Use it if
%             the compound point is spanned by a ligature.
%             (\verb=hoff"|intriger=)                               \\
%      |""| & Like |"-|, but producing no hyphen sign.
%             (|i""g\aa{}r|)                                        \\
%      |"~| & Like |-|, but allows no hyphenation at all. (|E"~cup|)\\
%      |"=| & Like |-|, but allowing hyphenation in the composing
%             words. (|marksistisk"=leninistisk|)                   \\
%      |"<| & for French left double quotes (similar to $<<$).      \\
%      |">| & for French right double quotes (similar to $>>$).     \\
%     \end{tabular}
%     \caption{The extra definitions made
%              by \file{norsk.sty}}\label{tab:norsk-quote}
%     \end{center}
%    \end{table}
% \changes{norsk-2.0a}{1998/06/24}{Describe the use of double quote as
%    active character}
%
%    Rune Kleveland distributes a Norwegian dictionary for ispell
%    (570000 words). It can be found at
%    |http://www.uio.no/~runekl/dictionary.html|. 
%
%    This dictionary supports the spellings |spi"sslede| for
%    `spisslede' (hyphenated spiss-slede) and other such words, and
%    also suggest the spelling |spi"sslede| for `spisslede' and
%    `spissslede'.
%
% \StopEventually{}
%
%    The macro |\LdfInit| takes care of preventing that this file is
%    loaded more than once, checking the category code of the
%    \texttt{@} sign, etc.
% \changes{norsk-1.2h}{1996/11/03}{Now use \cs{LdfInit} to perform
%    initial checks} 
%    \begin{macrocode}
%<*code>
\LdfInit\CurrentOption{captions\CurrentOption}
%    \end{macrocode}
%
%    When this file is read as an option, i.e. by the |\usepackage|
%    command, \texttt{norsk} will be an `unknown' language in which
%    case we have to make it known.  So we check for the existence of
%    |\l@norsk| to see whether we have to do something here.
%
% \changes{norsk-1.0c}{1991/10/29}{Removed use of \cs{@ifundefined}}
% \changes{norsk-1.1a}{1992/02/16}{Added a warning when no hyphenation
%    patterns were loaded.}
% \changes{norsk-1.2d}{1994/06/26}{Now use \cs{@nopatterns} to produce
%    the warning}
%    \begin{macrocode}
\ifx\l@norsk\@undefined
    \@nopatterns{Norsk}
    \adddialect\l@norsk0\fi
%    \end{macrocode}
%
%  \begin{macro}{\norskhyphenmins}
%     Some sets of Norwegian hyphenation patterns can be used with
%     |\lefthyphenmin| set to~1 and |\righthyphenmin| set to~2, but
%     the most common set |nohyph.tex| can't.  So we use
%     |\lefthyphenmin=2| by default.
% \changes{norsk-1.2f}{1995/07/02}{Added setting of hyphenmin
%    parameters}
% \changes{norsk-2.0a}{1998/06/24}{Changed setting of hyphenmin
%    parameters to 2~2} 
% \changes{norsk-2.0e}{2000/09/22}{Now use \cs{providehyphenmins} to
%    provide a default value}
%    \begin{macrocode}
\providehyphenmins{\CurrentOption}{\tw@\tw@}
%    \end{macrocode}
%  \end{macro}
%
%    Now we have to decide which version of the captions should be
%    made available. This can be done by checking the contents of
%    |\CurrentOption|. 
%    \begin{macrocode}
\def\bbl@tempa{norsk}
\ifx\CurrentOption\bbl@tempa
%    \end{macrocode}
%
%    The next step consists of defining commands to switch to (and
%    from) the Norwegian language.
%
% \begin{macro}{\captionsnorsk}
%    The macro |\captionsnorsk| defines all strings used
%    in the four standard documentclasses provided with \LaTeX.
% \changes{norsk-1.1a}{1992/02/16}{Added \cs{seename}, \cs{alsoname} and
%    \cs{prefacename}}
% \changes{norsk-1.1b}{1993/07/15}{\cs{headpagename} should be
%    \cs{pagename}}
% \changes{norsk-1.2f}{1995/07/02}{Added \cs{proofname} for
%    AMS-\LaTeX}
% \changes{norsk-1.2g}{1996/04/01}{Replaced `Proof' by its
%    translation} 
% \changes{norsk-2.0e}{2000/09/20}{Added \cs{glossaryname}}
% \changes{norsk-2.0g}{1996/04/01}{Replaced `Glossary' by its
%    translation} 
%    \begin{macrocode}
  \def\captionsnorsk{%
    \def\prefacename{Forord}%
    \def\refname{Referanser}%
    \def\abstractname{Sammendrag}%
    \def\bibname{Bibliografi}%     or Litteraturoversikt
    %                              or Litteratur or Referanser
    \def\chaptername{Kapittel}%
    \def\appendixname{Tillegg}%    or Appendiks
    \def\contentsname{Innhold}%
    \def\listfigurename{Figurer}%  or Figurliste
    \def\listtablename{Tabeller}%  or Tabelliste
    \def\indexname{Register}%
    \def\figurename{Figur}%
    \def\tablename{Tabell}%
    \def\partname{Del}%
    \def\enclname{Vedlegg}%
    \def\ccname{Kopi sendt}%
    \def\headtoname{Til}% in letter
    \def\pagename{Side}%
    \def\seename{Se}%
    \def\alsoname{Se ogs\aa{}}%
    \def\proofname{Bevis}%
    \def\glossaryname{Ordliste}%
    }
\else
%    \end{macrocode}
% \end{macro}
%
%    For the `nynorsk' version of these definitions we just add a
%    ``dialect''.
%    \begin{macrocode}
  \adddialect\l@nynorsk\l@norsk
%    \end{macrocode}
%
% \begin{macro}{\captionsnynorsk}
%    The macro |\captionsnynorsk| defines all strings used in the four
%    standard documentclasses provided with \LaTeX, but using a
%    different spelling than in the command |\captionsnorsk|.
% \changes{norsk-1.1a}{1992/02/16}{Added \cs{seename}, \cs{alsoname} and
%    \cs{prefacename}}
% \changes{norsk-1.1b}{1993/07/15}{\cs{headpagename} should be
%    \cs{pagename}}
% \changes{norsk-1.2g}{1996/04/01}{Replaced `Proof' by its
%    translation} 
% \changes{norsk-2.0e}{2000/09/20}{Added \cs{glossaryname}}
% \changes{norsk-2.0g}{1996/04/01}{Replaced `Glossary' by its
%    translation} 
% \changes{norks-2.0h}{2001/01/12}{Changed \cs{ccname} and \cs{alsoname}}
%    \begin{macrocode}
  \def\captionsnynorsk{%
    \def\prefacename{Forord}%
    \def\refname{Referansar}%
    \def\abstractname{Samandrag}%
    \def\bibname{Litteratur}%     or Litteraturoversyn
     %                             or Referansar
    \def\chaptername{Kapittel}%
    \def\appendixname{Tillegg}%   or Appendiks
    \def\contentsname{Innhald}%
    \def\listfigurename{Figurar}% or Figurliste
    \def\listtablename{Tabellar}% or Tabelliste
    \def\indexname{Register}%
    \def\figurename{Figur}%
    \def\tablename{Tabell}%
    \def\partname{Del}%
    \def\enclname{Vedlegg}%
    \def\ccname{Kopi til}%
    \def\headtoname{Til}% in letter
    \def\pagename{Side}%
    \def\seename{Sj\aa{}}%
    \def\alsoname{Sj\aa{} \`{o}g}%
    \def\proofname{Bevis}%
    \def\glossaryname{Ordliste}%
    }
\fi
%    \end{macrocode}
% \end{macro}
%
% \begin{macro}{\datenorsk}
%    The macro |\datenorsk| redefines the command |\today| to produce
%    Norwegian dates.
% \changes{norsk-1.2i}{1997/10/01}{Use \cs{edef} to define
%    \cs{today} to save memory}
% \changes{norsk-1.2i}{1998/03/28}{use \cs{def} instead of \cs{edef}}
% \changes{norsk-2.0i}{2012/08/06}{Removed extra space after `desember'}
%    \begin{macrocode}
\@namedef{date\CurrentOption}{%
  \def\today{\number\day.~\ifcase\month\or
    januar\or februar\or mars\or april\or mai\or juni\or
    juli\or august\or september\or oktober\or november\or
    desember\fi
    \space\number\year}}
%    \end{macrocode}
% \end{macro}
%
% \begin{macro}{\extrasnorsk}
% \begin{macro}{\extrasnynorsk}
%    The macro |\extrasnorsk| will perform all the extra definitions
%    needed for the Norwegian language. The macro |\noextrasnorsk| is
%    used to cancel the actions of |\extrasnorsk|.  
%
%    Norwegian typesetting requires |\frencspacing| to be in effect.
%    \begin{macrocode}
\@namedef{extras\CurrentOption}{\bbl@frenchspacing}
\@namedef{noextras\CurrentOption}{\bbl@nonfrenchspacing}
%    \end{macrocode}
% \end{macro}
% \end{macro}
%
%    For Norsk the \texttt{"} character is made active. This is done
%    once, later on its definition may vary.
% \changes{norsk-2.0a}{1998/06/24}{Made double quote character active}
%    \begin{macrocode}
\initiate@active@char{"}
\expandafter\addto\csname extras\CurrentOption\endcsname{%
  \languageshorthands{norsk}}
\expandafter\addto\csname extras\CurrentOption\endcsname{%
  \bbl@activate{"}}
%    \end{macrocode}
%    Don't forget to turn the shorthands off again.
% \changes{norsk-2.0c}{1999/12/17}{Deactivate shorthands ouside of
%    Norsk}
%    \begin{macrocode}
\expandafter\addto\csname noextras\CurrentOption\endcsname{%
  \bbl@deactivate{"}}
%    \end{macrocode}
%
%    The code above is necessary because we need to define a number of
%    shorthand commands. These sharthand commands are then used as
%    indicated in table~\ref{tab:norsk-quote}.
%
%    To be able to define the function of |"|, we first define a
%    couple of `support' macros.
%
%  \begin{macro}{\dq}
%    We save the original double quote character in |\dq| to keep
%    it available, the math accent |\"| can now be typed as |"|.
%    \begin{macrocode}
\begingroup \catcode`\"12
\def\x{\endgroup
  \def\@SS{\mathchar"7019 }
  \def\dq{"}}
\x
%    \end{macrocode}
%  \end{macro}
%
%    Now we can define the discretionary shorthand commands.
%    The number of words where such hyphenation is required is for
%    each character
%    \begin{center}
%      \begin{tabular}{*{11}c}
%        b&d&f &g&k &l &n&p &r&s &t \\
%        4&4&15&3&43&30&8&12&1&33&35
%       \end{tabular}
%    \end{center}
%    taken from a list of 83000 ispell-roots.
%
% \changes{norsk-2.0d}{2000/02/29}{Shorthands are the same for both
%    spelling variants, no need to use \cs{CurrentOption}}
%    \begin{macrocode}
\declare@shorthand{norsk}{"b}{\textormath{\bbl@disc b{bb}}{b}}
\declare@shorthand{norsk}{"B}{\textormath{\bbl@disc B{BB}}{B}}
\declare@shorthand{norsk}{"d}{\textormath{\bbl@disc d{dd}}{d}}
\declare@shorthand{norsk}{"D}{\textormath{\bbl@disc D{DD}}{D}}
\declare@shorthand{norsk}{"e}{\textormath{\bbl@disc e{\'e}}{}}
\declare@shorthand{norsk}{"E}{\textormath{\bbl@disc E{\'E}}{}}
\declare@shorthand{norsk}{"F}{\textormath{\bbl@disc F{FF}}{F}}
\declare@shorthand{norsk}{"g}{\textormath{\bbl@disc g{gg}}{g}}
\declare@shorthand{norsk}{"G}{\textormath{\bbl@disc G{GG}}{G}}
\declare@shorthand{norsk}{"k}{\textormath{\bbl@disc k{kk}}{k}}
\declare@shorthand{norsk}{"K}{\textormath{\bbl@disc K{KK}}{K}}
\declare@shorthand{norsk}{"l}{\textormath{\bbl@disc l{ll}}{l}}
\declare@shorthand{norsk}{"L}{\textormath{\bbl@disc L{LL}}{L}}
\declare@shorthand{norsk}{"n}{\textormath{\bbl@disc n{nn}}{n}}
\declare@shorthand{norsk}{"N}{\textormath{\bbl@disc N{NN}}{N}}
\declare@shorthand{norsk}{"p}{\textormath{\bbl@disc p{pp}}{p}}
\declare@shorthand{norsk}{"P}{\textormath{\bbl@disc P{PP}}{P}}
\declare@shorthand{norsk}{"r}{\textormath{\bbl@disc r{rr}}{r}}
\declare@shorthand{norsk}{"R}{\textormath{\bbl@disc R{RR}}{R}}
\declare@shorthand{norsk}{"s}{\textormath{\bbl@disc s{ss}}{s}}
\declare@shorthand{norsk}{"S}{\textormath{\bbl@disc S{SS}}{S}}
\declare@shorthand{norsk}{"t}{\textormath{\bbl@disc t{tt}}{t}}
\declare@shorthand{norsk}{"T}{\textormath{\bbl@disc T{TT}}{T}}
%    \end{macrocode}
%    We need to treat |"f| a bit differently in order to preserve the
%    ff-ligature. 
% \changes{norsk-2.0b}{1999/11/19}{Copied the coding for \texttt{"f}
%    from germanb.dtx version 2.6g} 
%    \begin{macrocode}
\declare@shorthand{norsk}{"f}{\textormath{\bbl@discff}{f}}
\def\bbl@discff{\penalty\@M
  \afterassignment\bbl@insertff \let\bbl@nextff= }
\def\bbl@insertff{%
  \if f\bbl@nextff
    \expandafter\@firstoftwo\else\expandafter\@secondoftwo\fi
  {\relax\discretionary{ff-}{f}{ff}\allowhyphens}{f\bbl@nextff}}
\let\bbl@nextff=f
%    \end{macrocode}
%    We now  define the French double quotes and some commands 
%    concerning hyphenation:
% \changes{norsk-2.0b}{1999/11/22}{added the french double quotes}
% \changes{norsk-2.0d}{2000/01/28}{Use \cs{bbl@allowhyphens} in
%    \texttt{"-}}
%    \begin{macrocode}
\declare@shorthand{norsk}{"<}{\flqq}
\declare@shorthand{norsk}{">}{\frqq}
\declare@shorthand{norsk}{"-}{\penalty\@M\-\bbl@allowhyphens}
\declare@shorthand{norsk}{"|}{%
  \textormath{\penalty\@M\discretionary{-}{}{\kern.03em}%
              \allowhyphens}{}}
\declare@shorthand{norsk}{""}{\hskip\z@skip}
\declare@shorthand{norsk}{"~}{\textormath{\leavevmode\hbox{-}}{-}}
\declare@shorthand{norsk}{"=}{\penalty\@M-\hskip\z@skip}
%    \end{macrocode}
%
%    The macro |\ldf@finish| takes care of looking for a
%    configuration file, setting the main language to be switched on
%    at |\begin{document}| and resetting the category code of
%    \texttt{@} to its original value.
% \changes{norsk-1.2h}{1996/11/03}{Now use \cs{ldf@finish} to wrap up}
%    \begin{macrocode}
\ldf@finish\CurrentOption
%</code>
%    \end{macrocode}
%
% \Finale
%%
%% \CharacterTable
%%  {Upper-case    \A\B\C\D\E\F\G\H\I\J\K\L\M\N\O\P\Q\R\S\T\U\V\W\X\Y\Z
%%   Lower-case    \a\b\c\d\e\f\g\h\i\j\k\l\m\n\o\p\q\r\s\t\u\v\w\x\y\z
%%   Digits        \0\1\2\3\4\5\6\7\8\9
%%   Exclamation   \!     Double quote  \"     Hash (number) \#
%%   Dollar        \$     Percent       \%     Ampersand     \&
%%   Acute accent  \'     Left paren    \(     Right paren   \)
%%   Asterisk      \*     Plus          \+     Comma         \,
%%   Minus         \-     Point         \.     Solidus       \/
%%   Colon         \:     Semicolon     \;     Less than     \<
%%   Equals        \=     Greater than  \>     Question mark \?
%%   Commercial at \@     Left bracket  \[     Backslash     \\
%%   Right bracket \]     Circumflex    \^     Underscore    \_
%%   Grave accent  \`     Left brace    \{     Vertical bar  \|
%%   Right brace   \}     Tilde         \~}
%%
\endinput
}
\DeclareOption{swedish}{%%
%% This file will generate fast loadable files and documentation
%% driver files from the doc files in this package when run through
%% LaTeX or TeX.
%%
%% Copyright 1989-2005 Johannes L. Braams and any individual authors
%% listed elsewhere in this file.  All rights reserved.
%% 
%% This file is part of the Babel system.
%% --------------------------------------
%% 
%% It may be distributed and/or modified under the
%% conditions of the LaTeX Project Public License, either version 1.3
%% of this license or (at your option) any later version.
%% The latest version of this license is in
%%   http://www.latex-project.org/lppl.txt
%% and version 1.3 or later is part of all distributions of LaTeX
%% version 2003/12/01 or later.
%% 
%% This work has the LPPL maintenance status "maintained".
%% 
%% The Current Maintainer of this work is Johannes Braams.
%% 
%% The list of all files belonging to the LaTeX base distribution is
%% given in the file `manifest.bbl. See also `legal.bbl' for additional
%% information.
%% 
%% The list of derived (unpacked) files belonging to the distribution
%% and covered by LPPL is defined by the unpacking scripts (with
%% extension .ins) which are part of the distribution.
%%
%% --------------- start of docstrip commands ------------------
%%
\def\filedate{1999/04/11}
\def\batchfile{swedish.ins}
\input docstrip.tex

{\ifx\generate\undefined
\Msg{**********************************************}
\Msg{*}
\Msg{* This installation requires docstrip}
\Msg{* version 2.3c or later.}
\Msg{*}
\Msg{* An older version of docstrip has been input}
\Msg{*}
\Msg{**********************************************}
\errhelp{Move or rename old docstrip.tex.}
\errmessage{Old docstrip in input path}
\batchmode
\csname @@end\endcsname
\fi}

\declarepreamble\mainpreamble
This is a generated file.

Copyright 1989-2005 Johannes L. Braams and any individual authors
listed elsewhere in this file.  All rights reserved.

This file was generated from file(s) of the Babel system.
---------------------------------------------------------

It may be distributed and/or modified under the
conditions of the LaTeX Project Public License, either version 1.3
of this license or (at your option) any later version.
The latest version of this license is in
  http://www.latex-project.org/lppl.txt
and version 1.3 or later is part of all distributions of LaTeX
version 2003/12/01 or later.

This work has the LPPL maintenance status "maintained".

The Current Maintainer of this work is Johannes Braams.

This file may only be distributed together with a copy of the Babel
system. You may however distribute the Babel system without
such generated files.

The list of all files belonging to the Babel distribution is
given in the file `manifest.bbl'. See also `legal.bbl for additional
information.

The list of derived (unpacked) files belonging to the distribution
and covered by LPPL is defined by the unpacking scripts (with
extension .ins) which are part of the distribution.
\endpreamble

\declarepreamble\fdpreamble
This is a generated file.

Copyright 1989-2005 Johannes L. Braams and any individual authors
listed elsewhere in this file.  All rights reserved.

This file was generated from file(s) of the Babel system.
---------------------------------------------------------

It may be distributed and/or modified under the
conditions of the LaTeX Project Public License, either version 1.3
of this license or (at your option) any later version.
The latest version of this license is in
  http://www.latex-project.org/lppl.txt
and version 1.3 or later is part of all distributions of LaTeX
version 2003/12/01 or later.

This work has the LPPL maintenance status "maintained".

The Current Maintainer of this work is Johannes Braams.

This file may only be distributed together with a copy of the Babel
system. You may however distribute the Babel system without
such generated files.

The list of all files belonging to the Babel distribution is
given in the file `manifest.bbl'. See also `legal.bbl for additional
information.

In particular, permission is granted to customize the declarations in
this file to serve the needs of your installation.

However, NO PERMISSION is granted to distribute a modified version
of this file under its original name.

\endpreamble

\keepsilent

\usedir{tex/generic/babel} 

\usepreamble\mainpreamble
\generate{\file{swedish.ldf}{\from{swedish.dtx}{code}}
          }
\usepreamble\fdpreamble

\ifToplevel{
\Msg{***********************************************************}
\Msg{*}
\Msg{* To finish the installation you have to move the following}
\Msg{* files into a directory searched by TeX:}
\Msg{*}
\Msg{* \space\space All *.def, *.fd, *.ldf, *.sty}
\Msg{*}
\Msg{* To produce the documentation run the files ending with}
\Msg{* '.dtx' and `.fdd' through LaTeX.}
\Msg{*}
\Msg{* Happy TeXing}
\Msg{***********************************************************}
}
 
\endinput
}
\DeclareOption{UKenglish}{% \iffalse meta-comment
%
% Copyright 1989-2005 Johannes L. Braams and any individual authors
% listed elsewhere in this file.  All rights reserved.
%    2013-2017 Javier Bezos, Johannes L. Braams
% This file is part of the Babel system.
% --------------------------------------
% 
% It may be distributed and/or modified under the
% conditions of the LaTeX Project Public License, either version 1.3
% of this license or (at your option) any later version.
% The latest version of this license is in
%   http://www.latex-project.org/lppl.txt
% and version 1.3 or later is part of all distributions of LaTeX
% version 2003/12/01 or later.
% 
% This work has the LPPL maintenance status "maintained".
% 
% The Current Maintainer of this work is Javier Bezos.
% 
% The list of all files belonging to the Babel system is
% given in the file `manifest.bbl. See also `legal.bbl' for additional
% information.
% 
% The list of derived (unpacked) files belonging to the distribution
% and covered by LPPL is defined by the unpacking scripts (with
% extension .ins) which are part of the distribution.
% \fi
% \iffalse
%    Tell the \LaTeX\ system who we are and write an entry on the
%    transcript.
%<*dtx>
\ProvidesFile{english.dtx}
%</dtx>
%<english>\ProvidesLanguage{english}
%<american>\ProvidesLanguage{american}
%<usenglish>\ProvidesLanguage{USenglish}
%<british>\ProvidesLanguage{british}
%<ukenglish>\ProvidesLanguage{UKenglish}
%<australian>\ProvidesLanguage{australian}
%<newzealand>\ProvidesLanguage{newzealand}
%<canadian>\ProvidesLanguage{canadian}
%\fi
%\ProvidesFile{english.dtx}
        [2017/06/06 v3.3r English support from the babel system]
%\iffalse
%% File 'english.dtx'
%% Babel package for LaTeX version 2e
%% Copyright (C) 1989 - 2005
%%           by Johannes Braams, TeXniek
%%           2013-2017 Javier Bezos, Johannes Braams
%
%
%    This file is part of the babel system, it provides the source
%    code for the English language definition file.
%<*filedriver>
\documentclass{ltxdoc}
\newcommand*\TeXhax{\TeX hax}
\newcommand*\babel{\textsf{babel}}
\newcommand*\langvar{$\langle \mathit lang \rangle$}
\newcommand*\note[1]{}
\newcommand*\Lopt[1]{\textsf{#1}}
\newcommand*\file[1]{\texttt{#1}}
\begin{document}
 \DocInput{english.dtx}
\end{document}
%</filedriver>
%\fi
% \GetFileInfo{english.dtx}
%
% \changes{english-2.0a}{1990/04/02}{Added checking of format}
% \changes{english-2.1}{1990/04/24}{Reflect changes in babel 2.1}
% \changes{english-2.1a}{1990/05/14}{Incorporated Nico's comments}
% \changes{english-2.1b}{1990/05/14}{merged \file{USenglish.sty} into
%    this file}
% \changes{english-2.1c}{1990/05/22}{fixed typo in definition for
%    american language found by Werenfried Spit (nspit@fys.ruu.nl)}
% \changes{english-2.1d}{1990/07/16}{Fixed some typos}
% \changes{english-3.0}{1991/04/23}{Modified for babel 3.0}
% \changes{english-3.0a}{1991/05/29}{Removed bug found by van der Meer}
% \changes{english-3.0c}{1991/07/15}{Renamed \file{babel.sty} in
%    \file{babel.com}}
% \changes{english-3.1}{1991/11/05}{Rewrote parts of the code to use
%    the new features of babel version 3.1}
% \changes{english-3.3}{1994/02/08}{Update or \LaTeXe}
% \changes{english-3.3c}{1994/06/26}{Removed the use of \cs{filedate}
%    and moved the identification after the loading of
%    \file{babel.def}}
% \changes{english-3.3g}{1996/07/10}{Replaced \cs{undefined} with
%    \cs{@undefined} and \cs{empty} with \cs{@empty} for consistency
%    with \LaTeX} 
% \changes{english-3.3h}{1996/10/10}{Moved the definition of
%    \cs{atcatcode} right to the beginning.} 
% \changes{english-3.3q}{2017/01/10}{Added the proxy files for the
%    dialects}
%
%  \section{The English language}
%
%    The file \file{\filename}\footnote{The file described in this
%    section has version number \fileversion\ and was last revised on
%    \filedate.} defines all the language definition macros for the
%    English language as well as for the American and Australian
%    version of this language. For the Australian version the British
%    hyphenation patterns will be used, if available, for the Canadian
%    variant the American patterns are selected.
%
%    For this language currently no special definitions are needed or
%    available.
%
% \StopEventually{}
%
%    The macro |\LdfInit| takes care of preventing that this file is
%    loaded more than once, checking the category code of the
%    \texttt{@} sign, etc.
% \changes{english-3.3h}{1996/11/02}{Now use \cs{LdfInit} to perform
%    initial checks} 
%    \begin{macrocode}
%<*code>
\LdfInit\CurrentOption{date\CurrentOption}
%    \end{macrocode}
%
%    When this file is read as an option, i.e. by the |\usepackage|
%    command, \texttt{english} could be an `unknown' language in which
%    case we have to make it known.  So we check for the existence of
%    |\l@english| to see whether we have to do something here.
%
% \changes{english-3.0}{1991/04/23}{Now use \cs{adddialect} if
%    language undefined}
% \changes{english-3.0d}{1991/10/22}{removed use of \cs{@ifundefined}}
% \changes{english-3.3c}{1994/06/26}{Now use \cs{@nopatterns} to
%    produce the warning}
% \changes{english-3.3g}{1996/07/10}{Allow british as the name of the
%    UK patterns}
% \changes{english-3.3j}{2000/01/21}{Also allow american english
%    hyphenation patterns to be used for `english'}
%    We allow for the british english patterns to be loaded as either
%    `british', or `UKenglish'. When neither of those is
%    known we try to define |\l@english| as an alias for |\l@american|
%    or |\l@USenglish|.
% \changes{english-3.3k}{2001/02/07}{Added support for canadian}
% \changes{english-3.3n}{2004/06/12}{Added support for australian and
%    newzealand} 
%    \begin{macrocode}
\ifx\l@english\@undefined
  \ifx\l@UKenglish\@undefined
    \ifx\l@british\@undefined
      \ifx\l@american\@undefined
        \ifx\l@USenglish\@undefined
          \ifx\l@canadian\@undefined
            \ifx\l@australian\@undefined
              \ifx\l@newzealand\@undefined
                \@nopatterns{English}
                \adddialect\l@english0
              \else
                \let\l@english\l@newzealand
              \fi
            \else
              \let\l@english\l@australian
            \fi
          \else
            \let\l@english\l@canadian
          \fi
        \else
          \let\l@english\l@USenglish
        \fi
      \else
        \let\l@english\l@american
      \fi
    \else
      \let\l@english\l@british
    \fi 
  \else
    \let\l@english\l@UKenglish
  \fi
\fi
%    \end{macrocode}
%    Because we allow `british' to be used as the babel option we need
%    to make sure that it will be recognised by |\selectlanguage|. In
%    the code above we have made sure that |\l@english| was defined.
%    Now we want to make sure that |\l@british| and |\l@UKenglish| are
%    defined as well. When either of them is we make them equal to
%    each other, when neither is we fall back to the default,
%    |\l@english|. 
% \changes{english-3.3o}{2004/06/14}{Make sure that british patterns
%    are used if they were loaded}
%    \begin{macrocode}
\ifx\l@british\@undefined
  \ifx\l@UKenglish\@undefined
    \adddialect\l@british\l@english
    \adddialect\l@UKenglish\l@english
  \else
    \let\l@british\l@UKenglish
  \fi
\else
  \let\l@UKenglish\l@british
\fi
%    \end{macrocode}
%    `American' is a version of `English' which can have its own
%    hyphenation patterns. The default english patterns are in fact
%    for american english. We allow for the patterns to be loaded as
%    `english' `american' or `USenglish'.
% \changes{english-3.0}{1990/04/23}{Now use \cs{adddialect} for
%    american}
% \changes{english-3.0b}{1991/06/06}{Removed \cs{global} definitions}
% \changes{english-3.3d}{1995/02/01}{Only define american as a
%    dialect when no separate patterns have been loaded}
% \changes{english-3.3g}{1996/07/10}{Allow USenglish as the name of
%    the american patterns} 
%    \begin{macrocode}
\ifx\l@american\@undefined
  \ifx\l@USenglish\@undefined
%    \end{macrocode}
%    When the patterns are not know as `american' or `USenglish' we
%    add a ``dialect''.
%    \begin{macrocode}
    \adddialect\l@american\l@english
  \else
    \let\l@american\l@USenglish
  \fi
\else
%    \end{macrocode}
%    Make sure that USenglish is known, even if the patterns were
%    loaded as `american'.
% \changes{english-3.3j}{2000/01/21}{Ensure that \cs{l@USenglish} is
%    alway defined}
% \changes{english-3.3l}{2001/04/15}{Added missing backslash}
%    \begin{macrocode}
  \ifx\l@USenglish\@undefined
    \let\l@USenglish\l@american
  \fi
\fi
%    \end{macrocode}
%
% \changes{english-3.3k}{2001/02/07}{Added support for canadian}
%    `Canadian' english spelling is a hybrid of British and American
%    spelling. Although so far no special `translations' have been
%    reported we allow this file to be loaded by the option
%    \Lopt{candian} as well.
%    \begin{macrocode}
\ifx\l@canadian\@undefined
  \adddialect\l@canadian\l@american
\fi
%    \end{macrocode}
%
% \changes{english-3.3n}{2004/06/12}{Added support for australian and
%   newzealand}
%    `Australian' and `New Zealand' english spelling seem to be the
%    same as British spelling. Although so far no special
%    `translations' have been reported we allow this file to be loaded
%    by the options \Lopt{australian} and \Lopt{newzealand} as well.
%    \begin{macrocode}
\ifx\l@australian\@undefined
  \adddialect\l@australian\l@british
\fi
\ifx\l@newzealand\@undefined
  \adddialect\l@newzealand\l@british
\fi
%    \end{macrocode}
%
 
%  \begin{macro}{\englishhyphenmins}
% \changes{english-3.3m}{2003/11/17}{Added default for setting of
%    hyphenmin parameters} 
%    This macro is used to store the correct values of the hyphenation
%    parameters |\lefthyphenmin| and |\righthyphenmin|.
%    \begin{macrocode}
\providehyphenmins{\CurrentOption}{\tw@\thr@@}
%    \end{macrocode}
%  \end{macro}
%
%    The next step consists of defining commands to switch to (and
%    from) the English language.
% \begin{macro}{\captionsenglish}
%    The macro |\captionsenglish| defines all strings used
%    in the four standard document classes provided with \LaTeX.
% \changes{english-3.0b}{1991/06/06}{Removed \cs{global} definitions}
% \changes{english-3.0b}{1991/06/06}{\cs{pagename} should be
%    \cs{headpagename}}
% \changes{english-3.1a}{1991/11/11}{added \cs{seename} and
%    \cs{alsoname}}
% \changes{english-3.1b}{1992/01/26}{added \cs{prefacename}}
% \changes{english-3.2}{1993/07/15}{\cs{headpagename} should be
%    \cs{pagename}}
% \changes{english-3.3e}{1995/07/04}{Added \cs{proofname} for
%    AMS-\LaTeX}
% \changes{english-3.3g}{1996/07/10}{Construct control sequence on the
%    fly} 
% \changes{english-3.3j}{2000/09/19}{Added \cs{glossaryname}}
%    \begin{macrocode}
\@namedef{captions\CurrentOption}{%
  \def\prefacename{Preface}%
  \def\refname{References}%
  \def\abstractname{Abstract}%
  \def\bibname{Bibliography}%
  \def\chaptername{Chapter}%
  \def\appendixname{Appendix}%
  \def\contentsname{Contents}%
  \def\listfigurename{List of Figures}%
  \def\listtablename{List of Tables}%
  \def\indexname{Index}%
  \def\figurename{Figure}%
  \def\tablename{Table}%
  \def\partname{Part}%
  \def\enclname{encl}%
  \def\ccname{cc}%
  \def\headtoname{To}%
  \def\pagename{Page}%
  \def\seename{see}%
  \def\alsoname{see also}%
  \def\proofname{Proof}%
  \def\glossaryname{Glossary}%
  }
%    \end{macrocode}
% \end{macro}
%
% \begin{macro}{\dateenglish}
%    In order to define |\today| correctly we need to know whether it
%    should be `english', `australian', or `american'. We can find
%    this out by checking the value of |\CurrentOption|.
% \changes{english-3.3j}{2000/01/21}{Make sure that the value of
%    \cs{today} is correct for both options `american' and
%    `USenglish'}
% \changes{english-3.3n}{2004/06/12}{Added support for `Australian'
%    and `Newzealand'}
% \changes{english-3.3o}{2004/06/14}{Explicitly choose the UK form of
%    date} 
% \changes{english-3.3p}{2012/11/07}{Warning if `english' is used with
%    other options} 
%    \begin{macrocode}
\def\bbl@tempa{british}
\ifx\CurrentOption\bbl@tempa\def\bbl@tempb{UK}\fi
\def\bbl@tempa{UKenglish}
\ifx\CurrentOption\bbl@tempa\def\bbl@tempb{UK}\fi
\def\bbl@tempa{american}
\ifx\CurrentOption\bbl@tempa\def\bbl@tempb{US}\fi
\def\bbl@tempa{USenglish}
\ifx\CurrentOption\bbl@tempa\def\bbl@tempb{US}\fi
\def\bbl@tempa{canadian}
\ifx\CurrentOption\bbl@tempa\def\bbl@tempb{US}\fi
\def\bbl@tempa{australian}
\ifx\CurrentOption\bbl@tempa\def\bbl@tempb{AU}\fi
\def\bbl@tempa{newzealand}
\ifx\CurrentOption\bbl@tempa\def\bbl@tempb{AU}\fi
\def\bbl@tempa{english}
\ifx\CurrentOption\bbl@tempa
  \AtEndOfPackage{\@nameuse{bbl@englishwarning}}
\else
  \edef\bbl@englishwarning{%
    \let\noexpand\bbl@englishwarning\relax
    \noexpand\PackageWarning{Babel}{%
      The package option `english' should not be used\noexpand\MessageBreak
      with a more specific one (like `\CurrentOption')}}
\fi
%    \end{macrocode}
%
%    The macro |\dateenglish| redefines the command |\today| to
%    produce English dates.
% \changes{english-3.0b}{1991/06/06}{Removed \cs{global} definitions}
% \changes{english-3.3g}{1996/07/10}{Construct control sequence on the
%    fly}
% \changes{english-3.3i}{1997/10/01}{Use \cs{edef} to define \cs{today}
%    to save memory}
% \changes{english-3.3i}{1998/03/28}{use \cs{def} instead of
%    \cs{edef}}
%    \begin{macrocode}
\def\bbl@tempa{UK}
\ifx\bbl@tempa\bbl@tempb
  \@namedef{date\CurrentOption}{%
    \def\today{\ifcase\day\or
      1st\or 2nd\or 3rd\or 4th\or 5th\or
      6th\or 7th\or 8th\or 9th\or 10th\or
      11th\or 12th\or 13th\or 14th\or 15th\or
      16th\or 17th\or 18th\or 19th\or 20th\or
      21st\or 22nd\or 23rd\or 24th\or 25th\or
      26th\or 27th\or 28th\or 29th\or 30th\or
      31st\fi~\ifcase\month\or
      January\or February\or March\or April\or May\or June\or
      July\or August\or September\or October\or November\or 
      December\fi\space \number\year}}
%    \end{macrocode}
% \end{macro}
%
% \begin{macro}{\dateaustralian}
%    Now, test for `australian' or `american'.
% \changes{english-3.3n}{2004/06/12}{Add australian date}
%    \begin{macrocode}
\else
%    \end{macrocode}
%
%    The macro |\dateaustralian| redefines the command |\today| to
%    produce Australian resp.\ New Zealand dates.
%    \begin{macrocode}
  \def\bbl@tempa{AU}
  \ifx\bbl@tempa\bbl@tempb
    \@namedef{date\CurrentOption}{%
      \def\today{\number\day~\ifcase\month\or
        January\or February\or March\or April\or May\or June\or
        July\or August\or September\or October\or November\or 
        December\fi\space \number\year}}
%    \end{macrocode}
% \end{macro}
%
% \begin{macro}{\dateamerican}
%    The macro |\dateamerican| redefines the command |\today| to
%    produce American dates.
% \changes{english-3.0b}{1991/06/06}{Removed \cs{global} definitions}
% \changes{english-3.3i}{1997/10/01}{Use \cs{edef} to define
%    \cs{today} to save memory}
% \changes{english-3.3i}{1998/03/28}{use \cs{def} instead of
%    \cs{edef}}
%    \begin{macrocode}
  \else
    \@namedef{date\CurrentOption}{%
      \def\today{\ifcase\month\or
        January\or February\or March\or April\or May\or June\or
        July\or August\or September\or October\or November\or
        December\fi \space\number\day, \number\year}}
  \fi
\fi
%    \end{macrocode}
% \end{macro}
%
% \begin{macro}{\extrasenglish}
% \begin{macro}{\noextrasenglish}
%    The macro |\extrasenglish| will perform all the extra definitions
%    needed for the English language. The macro |\noextrasenglish| is
%    used to cancel the actions of |\extrasenglish|.  For the moment
%    these macros are empty but they are defined for compatibility
%    with the other language definition files.
%
% \changes{english-3.3g}{1996/07/10}{Construct control sequences on
%    the fly} 
%    \begin{macrocode}
\@namedef{extras\CurrentOption}{}
\@namedef{noextras\CurrentOption}{}
%    \end{macrocode}
% \end{macro}
% \end{macro}
%
%    The macro |\ldf@finish| takes care of looking for a
%    configuration file, setting the main language to be switched on
%    at |\begin{document}| and resetting the category code of
%    \texttt{@} to its original value.
% \changes{english-3.3h}{1996/11/02}{Now use \cs{ldf@finish} to wrap
%    up} 
%    \begin{macrocode}
\ldf@finish\CurrentOption
%</code>
%    \end{macrocode}
%
% Finally, We create  a few proxy files, which just load english.ldf.
%
%    \begin{macrocode}
%<*american|usenglish|british|ukenglish|australian|newzealand|canadian>
\input english.ldf\relax
%</american|usenglish|british|ukenglish|australian|newzealand|canadian>
%    \end{macrocode}
%
% \Finale
%%
%% \CharacterTable
%%  {Upper-case    \A\B\C\D\E\F\G\H\I\J\K\L\M\N\O\P\Q\R\S\T\U\V\W\X\Y\Z
%%   Lower-case    \a\b\c\d\e\f\g\h\i\j\k\l\m\n\o\p\q\r\s\t\u\v\w\x\y\z
%%   Digits        \0\1\2\3\4\5\6\7\8\9
%%   Exclamation   \!     Double quote  \"     Hash (number) \#
%%   Dollar        \$     Percent       \%     Ampersand     \&
%%   Acute accent  \'     Left paren    \(     Right paren   \)
%%   Asterisk      \*     Plus          \+     Comma         \,
%%   Minus         \-     Point         \.     Solidus       \/
%%   Colon         \:     Semicolon     \;     Less than     \<
%%   Equals        \=     Greater than  \>     Question mark \?
%%   Commercial at \@     Left bracket  \[     Backslash     \\
%%   Right bracket \]     Circumflex    \^     Underscore    \_
%%   Grave accent  \`     Left brace    \{     Vertical bar  \|
%%   Right brace   \}     Tilde         \~}
%%
\endinput
}
\DeclareOption{USenglish}{% \iffalse meta-comment
%
% Copyright 1989-2005 Johannes L. Braams and any individual authors
% listed elsewhere in this file.  All rights reserved.
%    2013-2017 Javier Bezos, Johannes L. Braams
% This file is part of the Babel system.
% --------------------------------------
% 
% It may be distributed and/or modified under the
% conditions of the LaTeX Project Public License, either version 1.3
% of this license or (at your option) any later version.
% The latest version of this license is in
%   http://www.latex-project.org/lppl.txt
% and version 1.3 or later is part of all distributions of LaTeX
% version 2003/12/01 or later.
% 
% This work has the LPPL maintenance status "maintained".
% 
% The Current Maintainer of this work is Javier Bezos.
% 
% The list of all files belonging to the Babel system is
% given in the file `manifest.bbl. See also `legal.bbl' for additional
% information.
% 
% The list of derived (unpacked) files belonging to the distribution
% and covered by LPPL is defined by the unpacking scripts (with
% extension .ins) which are part of the distribution.
% \fi
% \iffalse
%    Tell the \LaTeX\ system who we are and write an entry on the
%    transcript.
%<*dtx>
\ProvidesFile{english.dtx}
%</dtx>
%<english>\ProvidesLanguage{english}
%<american>\ProvidesLanguage{american}
%<usenglish>\ProvidesLanguage{USenglish}
%<british>\ProvidesLanguage{british}
%<ukenglish>\ProvidesLanguage{UKenglish}
%<australian>\ProvidesLanguage{australian}
%<newzealand>\ProvidesLanguage{newzealand}
%<canadian>\ProvidesLanguage{canadian}
%\fi
%\ProvidesFile{english.dtx}
        [2017/06/06 v3.3r English support from the babel system]
%\iffalse
%% File 'english.dtx'
%% Babel package for LaTeX version 2e
%% Copyright (C) 1989 - 2005
%%           by Johannes Braams, TeXniek
%%           2013-2017 Javier Bezos, Johannes Braams
%
%
%    This file is part of the babel system, it provides the source
%    code for the English language definition file.
%<*filedriver>
\documentclass{ltxdoc}
\newcommand*\TeXhax{\TeX hax}
\newcommand*\babel{\textsf{babel}}
\newcommand*\langvar{$\langle \mathit lang \rangle$}
\newcommand*\note[1]{}
\newcommand*\Lopt[1]{\textsf{#1}}
\newcommand*\file[1]{\texttt{#1}}
\begin{document}
 \DocInput{english.dtx}
\end{document}
%</filedriver>
%\fi
% \GetFileInfo{english.dtx}
%
% \changes{english-2.0a}{1990/04/02}{Added checking of format}
% \changes{english-2.1}{1990/04/24}{Reflect changes in babel 2.1}
% \changes{english-2.1a}{1990/05/14}{Incorporated Nico's comments}
% \changes{english-2.1b}{1990/05/14}{merged \file{USenglish.sty} into
%    this file}
% \changes{english-2.1c}{1990/05/22}{fixed typo in definition for
%    american language found by Werenfried Spit (nspit@fys.ruu.nl)}
% \changes{english-2.1d}{1990/07/16}{Fixed some typos}
% \changes{english-3.0}{1991/04/23}{Modified for babel 3.0}
% \changes{english-3.0a}{1991/05/29}{Removed bug found by van der Meer}
% \changes{english-3.0c}{1991/07/15}{Renamed \file{babel.sty} in
%    \file{babel.com}}
% \changes{english-3.1}{1991/11/05}{Rewrote parts of the code to use
%    the new features of babel version 3.1}
% \changes{english-3.3}{1994/02/08}{Update or \LaTeXe}
% \changes{english-3.3c}{1994/06/26}{Removed the use of \cs{filedate}
%    and moved the identification after the loading of
%    \file{babel.def}}
% \changes{english-3.3g}{1996/07/10}{Replaced \cs{undefined} with
%    \cs{@undefined} and \cs{empty} with \cs{@empty} for consistency
%    with \LaTeX} 
% \changes{english-3.3h}{1996/10/10}{Moved the definition of
%    \cs{atcatcode} right to the beginning.} 
% \changes{english-3.3q}{2017/01/10}{Added the proxy files for the
%    dialects}
%
%  \section{The English language}
%
%    The file \file{\filename}\footnote{The file described in this
%    section has version number \fileversion\ and was last revised on
%    \filedate.} defines all the language definition macros for the
%    English language as well as for the American and Australian
%    version of this language. For the Australian version the British
%    hyphenation patterns will be used, if available, for the Canadian
%    variant the American patterns are selected.
%
%    For this language currently no special definitions are needed or
%    available.
%
% \StopEventually{}
%
%    The macro |\LdfInit| takes care of preventing that this file is
%    loaded more than once, checking the category code of the
%    \texttt{@} sign, etc.
% \changes{english-3.3h}{1996/11/02}{Now use \cs{LdfInit} to perform
%    initial checks} 
%    \begin{macrocode}
%<*code>
\LdfInit\CurrentOption{date\CurrentOption}
%    \end{macrocode}
%
%    When this file is read as an option, i.e. by the |\usepackage|
%    command, \texttt{english} could be an `unknown' language in which
%    case we have to make it known.  So we check for the existence of
%    |\l@english| to see whether we have to do something here.
%
% \changes{english-3.0}{1991/04/23}{Now use \cs{adddialect} if
%    language undefined}
% \changes{english-3.0d}{1991/10/22}{removed use of \cs{@ifundefined}}
% \changes{english-3.3c}{1994/06/26}{Now use \cs{@nopatterns} to
%    produce the warning}
% \changes{english-3.3g}{1996/07/10}{Allow british as the name of the
%    UK patterns}
% \changes{english-3.3j}{2000/01/21}{Also allow american english
%    hyphenation patterns to be used for `english'}
%    We allow for the british english patterns to be loaded as either
%    `british', or `UKenglish'. When neither of those is
%    known we try to define |\l@english| as an alias for |\l@american|
%    or |\l@USenglish|.
% \changes{english-3.3k}{2001/02/07}{Added support for canadian}
% \changes{english-3.3n}{2004/06/12}{Added support for australian and
%    newzealand} 
%    \begin{macrocode}
\ifx\l@english\@undefined
  \ifx\l@UKenglish\@undefined
    \ifx\l@british\@undefined
      \ifx\l@american\@undefined
        \ifx\l@USenglish\@undefined
          \ifx\l@canadian\@undefined
            \ifx\l@australian\@undefined
              \ifx\l@newzealand\@undefined
                \@nopatterns{English}
                \adddialect\l@english0
              \else
                \let\l@english\l@newzealand
              \fi
            \else
              \let\l@english\l@australian
            \fi
          \else
            \let\l@english\l@canadian
          \fi
        \else
          \let\l@english\l@USenglish
        \fi
      \else
        \let\l@english\l@american
      \fi
    \else
      \let\l@english\l@british
    \fi 
  \else
    \let\l@english\l@UKenglish
  \fi
\fi
%    \end{macrocode}
%    Because we allow `british' to be used as the babel option we need
%    to make sure that it will be recognised by |\selectlanguage|. In
%    the code above we have made sure that |\l@english| was defined.
%    Now we want to make sure that |\l@british| and |\l@UKenglish| are
%    defined as well. When either of them is we make them equal to
%    each other, when neither is we fall back to the default,
%    |\l@english|. 
% \changes{english-3.3o}{2004/06/14}{Make sure that british patterns
%    are used if they were loaded}
%    \begin{macrocode}
\ifx\l@british\@undefined
  \ifx\l@UKenglish\@undefined
    \adddialect\l@british\l@english
    \adddialect\l@UKenglish\l@english
  \else
    \let\l@british\l@UKenglish
  \fi
\else
  \let\l@UKenglish\l@british
\fi
%    \end{macrocode}
%    `American' is a version of `English' which can have its own
%    hyphenation patterns. The default english patterns are in fact
%    for american english. We allow for the patterns to be loaded as
%    `english' `american' or `USenglish'.
% \changes{english-3.0}{1990/04/23}{Now use \cs{adddialect} for
%    american}
% \changes{english-3.0b}{1991/06/06}{Removed \cs{global} definitions}
% \changes{english-3.3d}{1995/02/01}{Only define american as a
%    dialect when no separate patterns have been loaded}
% \changes{english-3.3g}{1996/07/10}{Allow USenglish as the name of
%    the american patterns} 
%    \begin{macrocode}
\ifx\l@american\@undefined
  \ifx\l@USenglish\@undefined
%    \end{macrocode}
%    When the patterns are not know as `american' or `USenglish' we
%    add a ``dialect''.
%    \begin{macrocode}
    \adddialect\l@american\l@english
  \else
    \let\l@american\l@USenglish
  \fi
\else
%    \end{macrocode}
%    Make sure that USenglish is known, even if the patterns were
%    loaded as `american'.
% \changes{english-3.3j}{2000/01/21}{Ensure that \cs{l@USenglish} is
%    alway defined}
% \changes{english-3.3l}{2001/04/15}{Added missing backslash}
%    \begin{macrocode}
  \ifx\l@USenglish\@undefined
    \let\l@USenglish\l@american
  \fi
\fi
%    \end{macrocode}
%
% \changes{english-3.3k}{2001/02/07}{Added support for canadian}
%    `Canadian' english spelling is a hybrid of British and American
%    spelling. Although so far no special `translations' have been
%    reported we allow this file to be loaded by the option
%    \Lopt{candian} as well.
%    \begin{macrocode}
\ifx\l@canadian\@undefined
  \adddialect\l@canadian\l@american
\fi
%    \end{macrocode}
%
% \changes{english-3.3n}{2004/06/12}{Added support for australian and
%   newzealand}
%    `Australian' and `New Zealand' english spelling seem to be the
%    same as British spelling. Although so far no special
%    `translations' have been reported we allow this file to be loaded
%    by the options \Lopt{australian} and \Lopt{newzealand} as well.
%    \begin{macrocode}
\ifx\l@australian\@undefined
  \adddialect\l@australian\l@british
\fi
\ifx\l@newzealand\@undefined
  \adddialect\l@newzealand\l@british
\fi
%    \end{macrocode}
%
 
%  \begin{macro}{\englishhyphenmins}
% \changes{english-3.3m}{2003/11/17}{Added default for setting of
%    hyphenmin parameters} 
%    This macro is used to store the correct values of the hyphenation
%    parameters |\lefthyphenmin| and |\righthyphenmin|.
%    \begin{macrocode}
\providehyphenmins{\CurrentOption}{\tw@\thr@@}
%    \end{macrocode}
%  \end{macro}
%
%    The next step consists of defining commands to switch to (and
%    from) the English language.
% \begin{macro}{\captionsenglish}
%    The macro |\captionsenglish| defines all strings used
%    in the four standard document classes provided with \LaTeX.
% \changes{english-3.0b}{1991/06/06}{Removed \cs{global} definitions}
% \changes{english-3.0b}{1991/06/06}{\cs{pagename} should be
%    \cs{headpagename}}
% \changes{english-3.1a}{1991/11/11}{added \cs{seename} and
%    \cs{alsoname}}
% \changes{english-3.1b}{1992/01/26}{added \cs{prefacename}}
% \changes{english-3.2}{1993/07/15}{\cs{headpagename} should be
%    \cs{pagename}}
% \changes{english-3.3e}{1995/07/04}{Added \cs{proofname} for
%    AMS-\LaTeX}
% \changes{english-3.3g}{1996/07/10}{Construct control sequence on the
%    fly} 
% \changes{english-3.3j}{2000/09/19}{Added \cs{glossaryname}}
%    \begin{macrocode}
\@namedef{captions\CurrentOption}{%
  \def\prefacename{Preface}%
  \def\refname{References}%
  \def\abstractname{Abstract}%
  \def\bibname{Bibliography}%
  \def\chaptername{Chapter}%
  \def\appendixname{Appendix}%
  \def\contentsname{Contents}%
  \def\listfigurename{List of Figures}%
  \def\listtablename{List of Tables}%
  \def\indexname{Index}%
  \def\figurename{Figure}%
  \def\tablename{Table}%
  \def\partname{Part}%
  \def\enclname{encl}%
  \def\ccname{cc}%
  \def\headtoname{To}%
  \def\pagename{Page}%
  \def\seename{see}%
  \def\alsoname{see also}%
  \def\proofname{Proof}%
  \def\glossaryname{Glossary}%
  }
%    \end{macrocode}
% \end{macro}
%
% \begin{macro}{\dateenglish}
%    In order to define |\today| correctly we need to know whether it
%    should be `english', `australian', or `american'. We can find
%    this out by checking the value of |\CurrentOption|.
% \changes{english-3.3j}{2000/01/21}{Make sure that the value of
%    \cs{today} is correct for both options `american' and
%    `USenglish'}
% \changes{english-3.3n}{2004/06/12}{Added support for `Australian'
%    and `Newzealand'}
% \changes{english-3.3o}{2004/06/14}{Explicitly choose the UK form of
%    date} 
% \changes{english-3.3p}{2012/11/07}{Warning if `english' is used with
%    other options} 
%    \begin{macrocode}
\def\bbl@tempa{british}
\ifx\CurrentOption\bbl@tempa\def\bbl@tempb{UK}\fi
\def\bbl@tempa{UKenglish}
\ifx\CurrentOption\bbl@tempa\def\bbl@tempb{UK}\fi
\def\bbl@tempa{american}
\ifx\CurrentOption\bbl@tempa\def\bbl@tempb{US}\fi
\def\bbl@tempa{USenglish}
\ifx\CurrentOption\bbl@tempa\def\bbl@tempb{US}\fi
\def\bbl@tempa{canadian}
\ifx\CurrentOption\bbl@tempa\def\bbl@tempb{US}\fi
\def\bbl@tempa{australian}
\ifx\CurrentOption\bbl@tempa\def\bbl@tempb{AU}\fi
\def\bbl@tempa{newzealand}
\ifx\CurrentOption\bbl@tempa\def\bbl@tempb{AU}\fi
\def\bbl@tempa{english}
\ifx\CurrentOption\bbl@tempa
  \AtEndOfPackage{\@nameuse{bbl@englishwarning}}
\else
  \edef\bbl@englishwarning{%
    \let\noexpand\bbl@englishwarning\relax
    \noexpand\PackageWarning{Babel}{%
      The package option `english' should not be used\noexpand\MessageBreak
      with a more specific one (like `\CurrentOption')}}
\fi
%    \end{macrocode}
%
%    The macro |\dateenglish| redefines the command |\today| to
%    produce English dates.
% \changes{english-3.0b}{1991/06/06}{Removed \cs{global} definitions}
% \changes{english-3.3g}{1996/07/10}{Construct control sequence on the
%    fly}
% \changes{english-3.3i}{1997/10/01}{Use \cs{edef} to define \cs{today}
%    to save memory}
% \changes{english-3.3i}{1998/03/28}{use \cs{def} instead of
%    \cs{edef}}
%    \begin{macrocode}
\def\bbl@tempa{UK}
\ifx\bbl@tempa\bbl@tempb
  \@namedef{date\CurrentOption}{%
    \def\today{\ifcase\day\or
      1st\or 2nd\or 3rd\or 4th\or 5th\or
      6th\or 7th\or 8th\or 9th\or 10th\or
      11th\or 12th\or 13th\or 14th\or 15th\or
      16th\or 17th\or 18th\or 19th\or 20th\or
      21st\or 22nd\or 23rd\or 24th\or 25th\or
      26th\or 27th\or 28th\or 29th\or 30th\or
      31st\fi~\ifcase\month\or
      January\or February\or March\or April\or May\or June\or
      July\or August\or September\or October\or November\or 
      December\fi\space \number\year}}
%    \end{macrocode}
% \end{macro}
%
% \begin{macro}{\dateaustralian}
%    Now, test for `australian' or `american'.
% \changes{english-3.3n}{2004/06/12}{Add australian date}
%    \begin{macrocode}
\else
%    \end{macrocode}
%
%    The macro |\dateaustralian| redefines the command |\today| to
%    produce Australian resp.\ New Zealand dates.
%    \begin{macrocode}
  \def\bbl@tempa{AU}
  \ifx\bbl@tempa\bbl@tempb
    \@namedef{date\CurrentOption}{%
      \def\today{\number\day~\ifcase\month\or
        January\or February\or March\or April\or May\or June\or
        July\or August\or September\or October\or November\or 
        December\fi\space \number\year}}
%    \end{macrocode}
% \end{macro}
%
% \begin{macro}{\dateamerican}
%    The macro |\dateamerican| redefines the command |\today| to
%    produce American dates.
% \changes{english-3.0b}{1991/06/06}{Removed \cs{global} definitions}
% \changes{english-3.3i}{1997/10/01}{Use \cs{edef} to define
%    \cs{today} to save memory}
% \changes{english-3.3i}{1998/03/28}{use \cs{def} instead of
%    \cs{edef}}
%    \begin{macrocode}
  \else
    \@namedef{date\CurrentOption}{%
      \def\today{\ifcase\month\or
        January\or February\or March\or April\or May\or June\or
        July\or August\or September\or October\or November\or
        December\fi \space\number\day, \number\year}}
  \fi
\fi
%    \end{macrocode}
% \end{macro}
%
% \begin{macro}{\extrasenglish}
% \begin{macro}{\noextrasenglish}
%    The macro |\extrasenglish| will perform all the extra definitions
%    needed for the English language. The macro |\noextrasenglish| is
%    used to cancel the actions of |\extrasenglish|.  For the moment
%    these macros are empty but they are defined for compatibility
%    with the other language definition files.
%
% \changes{english-3.3g}{1996/07/10}{Construct control sequences on
%    the fly} 
%    \begin{macrocode}
\@namedef{extras\CurrentOption}{}
\@namedef{noextras\CurrentOption}{}
%    \end{macrocode}
% \end{macro}
% \end{macro}
%
%    The macro |\ldf@finish| takes care of looking for a
%    configuration file, setting the main language to be switched on
%    at |\begin{document}| and resetting the category code of
%    \texttt{@} to its original value.
% \changes{english-3.3h}{1996/11/02}{Now use \cs{ldf@finish} to wrap
%    up} 
%    \begin{macrocode}
\ldf@finish\CurrentOption
%</code>
%    \end{macrocode}
%
% Finally, We create  a few proxy files, which just load english.ldf.
%
%    \begin{macrocode}
%<*american|usenglish|british|ukenglish|australian|newzealand|canadian>
\input english.ldf\relax
%</american|usenglish|british|ukenglish|australian|newzealand|canadian>
%    \end{macrocode}
%
% \Finale
%%
%% \CharacterTable
%%  {Upper-case    \A\B\C\D\E\F\G\H\I\J\K\L\M\N\O\P\Q\R\S\T\U\V\W\X\Y\Z
%%   Lower-case    \a\b\c\d\e\f\g\h\i\j\k\l\m\n\o\p\q\r\s\t\u\v\w\x\y\z
%%   Digits        \0\1\2\3\4\5\6\7\8\9
%%   Exclamation   \!     Double quote  \"     Hash (number) \#
%%   Dollar        \$     Percent       \%     Ampersand     \&
%%   Acute accent  \'     Left paren    \(     Right paren   \)
%%   Asterisk      \*     Plus          \+     Comma         \,
%%   Minus         \-     Point         \.     Solidus       \/
%%   Colon         \:     Semicolon     \;     Less than     \<
%%   Equals        \=     Greater than  \>     Question mark \?
%%   Commercial at \@     Left bracket  \[     Backslash     \\
%%   Right bracket \]     Circumflex    \^     Underscore    \_
%%   Grave accent  \`     Left brace    \{     Vertical bar  \|
%%   Right brace   \}     Tilde         \~}
%%
\endinput
}
%    \end{macrocode}
% Make it possible to load language definition files that are not
% known by this package.
%    \begin{macrocode}
\DeclareOption*{%
  \InputIfFileExists{\CurrentOption.idf}{}{%
    \PackageError{isodate}{%
      Isodate definition file \CurrentOption.idf not found}{%
      Maybe you misspelled the language option?}}%
  }
%    \end{macrocode}
% Set default option to \verb|orig|.
%    \begin{macrocode}
\ExecuteOptions{orig,nocleanlook,printdayon}
%    \end{macrocode}
% Process the options.
%    \begin{macrocode}
\ProcessOptions*
%    \end{macrocode}
% Handle the case that no language was given. Throw an error message.
% Each language definition file \verb|*.idf| must contain a line
% \begin{verbatim}
%\let\iso@languageloaded\active\end{verbatim}
% that defines the command \verb|\iso@languageloaded|.
%    \begin{macrocode}
\ifx\iso@languageloaded\@undefined
  \PackageError{isodate}{%
    You haven't specified a language option}{%
    You need to specify a language, either as a global
    option\MessageBreak
    or as an optional argument to the \string\usepackage\space
    command.\MessageBreak
    If you have used the old isodate package (version <=1.06) you can
    change the\MessageBreak
    usepackage command to \protect\usepackage{isodate}.\MessageBreak
    You shouldn't try to proceed from here, type x to quit.}
\fi
%    \end{macrocode}
% \changes{2.10}{2003/10/13}{Add month in Roman numerals}%
% \begin{macro}{\iso@printday}
% \changes{2.14}{2003/10/26}{Control the number of digits for the day
%   by a boolean rather than by the command calls}%
% Prints a day.
%    \begin{macrocode}
\newcommand*\iso@printday[1]{%
  \ifisotwodigitday
    \ifthenelse{\number#1<10}{0}{}%
  \fi
  \number#1%
}%
%    \end{macrocode}
% \end{macro}
% \begin{macro}{\twodigitarabic}
% \changes{2.10}{2003/10/13}{Added \cs{twodigitarabic}}%
% Typesets the given counter with at least two digits.
% This command is very simple and does only work for positive numbers
% below 100.
%    \begin{macrocode}
\newcommand*\twodigitarabic[1]{%
  \ifthenelse{\number\arabic{#1}<10}{0}{}%
  \arabic{#1}%
}
%    \end{macrocode}
% \end{macro}
% \begin{macro}{\iso@printmonth}
% \changes{2.10}{2003/10/13}{Use \cs{twodigitarabic}}%
% Prints a month using \cs{theiso@tmpmonth} as output fourmat.
%    \begin{macrocode}
\newcommand*\iso@printmonth[1]{%
  \setcounter{iso@tmpmonth}{#1}%
  \theiso@tmpmonth%
}
%    \end{macrocode}
% Define the help counter that prints the month and initialize it to
% print arabic numbers.
%    \begin{macrocode}
\newcounter{iso@tmpmonth}
%\def\theiso@tmpmonth{\arabic{iso@tmpmonth}}
%    \end{macrocode}
% \end{macro}
% \begin{macro}{\iso@yeartwo}
% Prints the argument of the command with two
% digits. 
%
% Example: \verb|\iso@yeartwo{1873}| $\longrightarrow$
% \makeatletter\iso@yeartwo{1873}\makeatother.
%    \begin{macrocode}
\newcounter{iso@yeartwo}%
\newcommand*\iso@yeartwo[1]{%
  \setcounter{iso@yeartwo}{\number#1}%
  \whiledo{\theiso@yeartwo>99}{%
    \addtocounter{iso@yeartwo}{-100}}{}%
  \ifthenelse{\number\theiso@yeartwo<10}{0}{}\theiso@yeartwo
}
%    \end{macrocode}
% \end{macro}
% \begin{macro}{\iso@yearfour}
% \changes{2.26}{2005/03/10}{Force year in four digits for long formats}%
% Prints the argument of the command with four digits. 
%    \begin{macrocode}
\newcommand*\iso@yearfour[1]{%
  \ifthenelse{\number#1<1000}{0}{}%
  \ifthenelse{\number#1<100}{0}{}%
  \ifthenelse{\number#1<10}{0}{}%
  \number#1%
}%
%    \end{macrocode}
% \end{macro}
% \changes{2.14}{2003/10/26}{Control the number of digits for the day
%   by a boolean rather than by the command calls}%
% \begin{macro}{\ifisotwodigitday}
% Print day with two digits or natural number of digits?
%    \begin{macrocode}
\newif\ifisotwodigitday
%    \end{macrocode}
% \end{macro}
% \begin{macro}{\iso@dateformat}
% In this command, the current active date format ist stored. Possible
% values are: \verb|numeric|, \verb|short|, \verb|iso|, \verb|orig|,
% \verb|shortorig|, \verb|TeX|.
%    \begin{macrocode}
\def\iso@dateformat{numeric}
%    \end{macrocode}
% \end{macro}
% \begin{macro}{\iso@inputformat}
% \changes{2.26}{2005/03/10}{Support different input formats
%   containing slashes}%
% This macro stores which input format is used for dates given with
% slashes. Valid formats are |english| (dd/mm/yyyy), |american|
% (mm/dd/yyyy), and |tex| (yyyy/mm/dd). By default, English is used.
%    \begin{macrocode}
\DeclareRobustCommand*\dateinputformat[1]{%
  \ifthenelse{%
    \equal{#1}{english}\OR
    \equal{#1}{british}\OR
    \equal{#1}{UKenglish}}{%
    \def\iso@inputformat{english}%
  }{%
    \ifthenelse{%
      \equal{#1}{american}\OR
      \equal{#1}{USenglish}}{%
      \def\iso@inputformat{american}%
    }{%
      \ifthenelse{%
        \equal{#1}{tex}\OR
        \equal{#1}{TeX}\OR
        \equal{#1}{latex}\OR
        \equal{#1}{LaTeX}}{%
        \def\iso@inputformat{tex}%
      }{%
        \PackageError{isodate}{Invalid date input format}{%
          Maybe you misspelled the language option (english, american,
          tex)?}%
      }%
    }%
  }%
}
%    \end{macrocode}
% \end{macro}
% \begin{macro}{\iso@inputformat}
% \changes{2.26}{2005/03/10}{Support different input formats
%   containing slashes}%
% This macro stores which input format is used for dates given with
% slashes. Valid formats are |english| (dd/mm/yyyy), |american|
% (mm/dd/yyyy), and |tex| (yyyy/mm/dd). By default, English is used.
%    \begin{macrocode}
\dateinputformat{english}
%    \end{macrocode}
% \end{macro}
% \changes{2.10}{2003/10/13}{Add month in Roman numerals}%
% \begin{macro}{\numdate}
% Switches to long numerical date format.
%    \begin{macrocode}
\DeclareRobustCommand*\numdate[1][twodigitarabic]{%
  \def\iso@dateformat{numeric}%
  \isotwodigitdaytrue
  \def\theiso@tmpmonth{\csname #1\endcsname{iso@tmpmonth}}%
}
%    \end{macrocode}
% \end{macro}
% \changes{2.10}{2003/10/13}{Add month in Roman numerals}%
% \changes{2.14}{2003/10/26}{Don't print day with two digits when
%   Roman numerals are used for the month}%
% \begin{macro}{\shortdate}
% Switches to short numerical date format.
%    \begin{macrocode}
\DeclareRobustCommand*\shortdate[1][twodigitarabic]{%
  \def\iso@dateformat{short}%
  \isotwodigitdaytrue
  \def\theiso@tmpmonth{\csname #1\endcsname{iso@tmpmonth}}%
}
%    \end{macrocode}
% \end{macro}
% \changes{2.12}{2003/10/14}{Wrong one-digit months avoided}%
% \begin{macro}{\isodate}
% \changes{2.14}{2003/10/26}{Allow change in format for month}%
% Switches to ISO date format.
%    \begin{macrocode}
\DeclareRobustCommand*\isodate[1][twodigitarabic]{%
  \def\iso@dateformat{iso}%
  \isotwodigitdaytrue
  \def\theiso@tmpmonth{\csname #1\endcsname{iso@tmpmonth}}%
}
%    \end{macrocode}
% \end{macro}
% \begin{macro}{\origdate}
% Switches to the original date format.
%    \begin{macrocode}
\DeclareRobustCommand*\origdate{%
  \def\iso@dateformat{orig}%
  \isotwodigitdayfalse
  \def\theiso@tmpmonth{\twodigitarabic{iso@tmpmonth}}%
}
%    \end{macrocode}
% \end{macro}
% \begin{macro}{\shortorigdate}
% Switches to the short original date format.
%    \begin{macrocode}
\DeclareRobustCommand*\shortorigdate{%
  \def\iso@dateformat{shortorig}%
  \isotwodigitdayfalse
  \def\theiso@tmpmonth{\twodigitarabic{iso@tmpmonth}}%
}
%    \end{macrocode}
% \end{macro}q
% \begin{macro}{\TeXdate}
% \changes{2.14}{2003/10/26}{Allow change in format for month}%
% Switches to \LaTeX\ date format.
%    \begin{macrocode}
\DeclareRobustCommand*\TeXdate[1][twodigitarabic]{%
  \def\iso@dateformat{TeX}%
  \isotwodigitdaytrue
  \def\theiso@tmpmonth{\csname #1\endcsname{iso@tmpmonth}}%
}
%    \end{macrocode}
% \end{macro}
% \changes{2.10}{2003/10/13}{Add month in Roman numerals}%
% \begin{macro}{\Romandate}
% Switches to long numerical date format with month printed in
% uppercase Roman numerals.
%    \begin{macrocode}
\DeclareRobustCommand*\Romandate{%
  \numdate[Roman]%
  \isotwodigitdayfalse
}
%    \end{macrocode}
% \end{macro}
% \begin{macro}{\romandate}
% Switches to long numerical date format with month printed in
% lowercase Roman numerals.
%    \begin{macrocode}
\DeclareRobustCommand*\romandate{%
  \numdate[roman]%
  \isotwodigitdayfalse
}
%    \end{macrocode}
% \end{macro}
% \begin{macro}{\shortRomandate}
% Switches to short numerical date format with month printed in
% uppercase Roman numerals.
%    \begin{macrocode}
\DeclareRobustCommand*\shortRomandate{%
  \shortdate[Roman]%
  \isotwodigitdayfalse
}
%    \end{macrocode}
% \end{macro}
% \begin{macro}{\shortromandate}
% Switches to short numerical date format with month printed in
% lowercase Roman numerals.
%    \begin{macrocode}
\DeclareRobustCommand*\shortromandate{%
  \shortdate[roman]%
  \isotwodigitdayfalse
}
%    \end{macrocode}
% \end{macro}
% \begin{macro}{\isodash}
% Changes the dash in the ISO date format. The default is `-'.
%    \begin{macrocode}
\def\iso@isodash{-}%
\DeclareRobustCommand*\isodash[1]{\def\iso@isodash{#1}}%
%    \end{macrocode}
% \end{macro}
% Define the sign that is printed before a two digit year in the short
% original format. Default is nothing.
% \begin{macro}{\shortyearsign}
%    \begin{macrocode}
\def\iso@twodigitsign{}
\DeclareRobustCommand*\shortyearsign[1]{\def\iso@twodigitsign{#1}}%
%    \end{macrocode}
% \end{macro}
% \begin{macro}{\isorangesign}
% Defines the sign or word that is printed between the two dates in a
% date range. e.g., in English the default is `~to~'.
%    \begin{macrocode}
\def\iso@rangesign{\csname iso@rangesign@\iso@languagename\endcsname}%
\DeclareRobustCommand*\isorangesign[1]{\def\iso@rangesign{#1}}%
%    \end{macrocode}
% \end{macro}
% \begin{macro}{\printyearoff}
% \begin{macro}{\printyearon}
% \changes{2.21}{2003/12/06}{Switch on or off printing of year}%
% Switches printing of the year on or off.
% Default is to print the year.
%    \begin{macrocode}
\newif\ifiso@printyear
\DeclareRobustCommand*\printyearon{\iso@printyeartrue}
\DeclareRobustCommand*\printyearoff{\iso@printyearfalse}
\printyearon
%    \end{macrocode}
% \end{macro}
% \end{macro}
% \begin{macro}{\printdayoff}
% \begin{macro}{\printdayon}
%   \changes{2.30}{2010/01/03}{Add a month-year format}%
% Switch on or off suppressing the day in date output.
% Default is not print the day.
%    \begin{macrocode}
\newif\ifiso@doprintday
\DeclareRobustCommand*\printdayon{\iso@doprintdaytrue}
\DeclareRobustCommand*\printdayoff{\iso@doprintdayfalse}
\printdayon
%    \end{macrocode}
% \end{macro}
% \end{macro}
% \begin{macro}{\cleanlookdateoff}
% \begin{macro}{\cleanlookdateon}
% \changes{2.28}{2005/04/15}{Introduce option cleanlook for English
%   date format}%
% Switch on or off `clean look' for English dates.
% Default is not to use `clean look'.
%    \begin{macrocode}
\newif\ifiso@cleanlook
\DeclareRobustCommand*\cleanlookdateon{\iso@cleanlooktrue}
\DeclareRobustCommand*\cleanlookdateoff{\iso@cleanlookfalse}
\cleanlookdateoff
%    \end{macrocode}
% \end{macro}
% \end{macro}
% \begin{macro}{\isospacebeforeday}
% \begin{macro}{\isospacebeforemonth}
% \begin{macro}{\isospacebeforeyear}
% \changes{2.29}{2007/04/09}{Allow to change the unbreakable spaces in
%   the orig and shortorig format}%
% Change the spaces in the orig and short orig format.
% Default is |~| for all of them.
%    \begin{macrocode}
\newcommand*\iso@daysep{~}
\newcommand*\iso@monthsep{~}
\newcommand*\iso@yearsep{~}
\DeclareRobustCommand*\isospacebeforeday[1]{\def\iso@daysep{#1}}
\DeclareRobustCommand*\isospacebeforemonth[1]{\def\iso@monthsep{#1}}
\DeclareRobustCommand*\isospacebeforeyear[1]{\def\iso@yearsep{#1}}
%    \end{macrocode}
% \end{macro}
% \end{macro}
% \end{macro}
% \begin{macro}{\iso@printdate}
% Defines the command \verb|iso@printdate| which takes three arguments 
% (year, month, day) and prints the date by using the \cs{today} command.
%    \begin{macrocode}
\newcommand*\iso@printdate[3]{%
  \begingroup%
%    \end{macrocode}
% \changes{2.25}{2005/02/21}{Warning for unknown languages}%
% \changes{2.25}{2005/02/21}{Fall-back format for unknown languages}%
% Generate a warning if the active language is not known by |isodate|.
%    \begin{macrocode}
    \@ifundefined{iso@printdate@\iso@languagename}{%
      \PackageWarning{isodate}{Language \iso@languagename\space unknown
        to isodate.\MessageBreak
        Using default format}%
    }{}%
%    \end{macrocode}
% \changes{2.25}{2005/02/21}{Changed \cs{year}, \cs{month}, and
%   \cs{day} from macros to counters}%
% The counters \cs{year}, \cs{month}, and \cs{day} are preserved as
% counters instead of changed to macros (as it has been done until
% version 2.25) to avoid problems with languages that are not defined
% in |isodate.sty|.
%    \begin{macrocode}
    \year=#1 %
    \month=#2 %
    \day=#3 %
    \today%
  \endgroup%
}
%    \end{macrocode}
% \end{macro}
% \begin{macro}{\printdate}
% Prints a date that is given as one argument in one of these formats: 
% \verb|yyyy-mm-dd|, \verb|dd/mm/yyyy|, \verb|dd.mm.yyyy|.
%    \begin{macrocode}
\DeclareRobustCommand*\printdate[1]{%
%    \end{macrocode}
% Define \verb|\iso@date| command to expand the argument \verb|#1|.
%    \begin{macrocode}
  \edef\iso@date{#1}%
%    \end{macrocode}
% Count appearances of `/', `-', and `.' in the argument.
%    \begin{macrocode}
  \SubStringsToCounter{iso@slash}{/}{\iso@date}%
  \SubStringsToCounter{iso@minus}{-}{\iso@date}%
  \SubStringsToCounter{iso@dot}{.}{\iso@date}%
%    \end{macrocode}
% If number of `.' in the argument is equal to 2 then the German
% format \verb|dd.mm.yyyy| is used.
%    \begin{macrocode}
  \ifthenelse{\equal{\theiso@dot}{2}}{%
    \expandafter\iso@input@german\iso@date\@empty}{%
%    \end{macrocode}
% If number of `-' in the argument is equal to 2 then the ISO
% format \verb|yyyy-mm-dd| is used.
%    \begin{macrocode}
    \ifthenelse{\equal{\theiso@minus}{2}}{%
      \expandafter\iso@input@iso\iso@date\@empty}{%
%    \end{macrocode}
% If number of `/' in the argument is equal to 2 then the British English
% format \verb|dd/mm/yyyy| is used.
%    \begin{macrocode}
      \ifthenelse{\equal{\theiso@slash}{2}}{%
        \expandafter\iso@input@english\iso@date\@empty}{%
%    \end{macrocode}
% Else no of the formats above is used an thus an error message is thrown.
%    \begin{macrocode}
        ????\iso@isodash ??\iso@isodash ??%
        \PackageError{isodate}{unrecognized date format}{Use one of
          the following formats as macro argument:^^J%
          \space\space dd.mm.yyyy^^J%
          \space\space dd/mm/yyyy^^J%
          \space\space yyyy-mm-dd^^J%
          Don't use any spaces or commands like \protect\, or
          \protect~ inside the argument.}%
        }}}%
}
%    \end{macrocode}
% \end{macro}
% \begin{macro}{\iso@input@iso}
% Converts a string with the format \verb|yyyy-mm-dd| to three
% arguments \verb|{#1}{#2}{#3}| and calls \verb|\iso@printdate|.
%    \begin{macrocode}
\def\iso@input@iso#1-#2-#3\@empty{\iso@printdate{#1}{#2}{#3}}
%    \end{macrocode}
% \end{macro}
% \begin{macro}{\iso@input@german}
% Converts a string with the format \verb|dd.mm.yyyy| to three
% arguments \verb|{#3}{#2}{#1}| and calls \verb|\iso@printdate|.
%    \begin{macrocode}
\def\iso@input@german#1.#2.#3\@empty{\iso@printdate{#3}{#2}{#1}}
%    \end{macrocode}
% \end{macro}
% \begin{macro}{\iso@input@english}
% \changes{2.26}{2005/03/10}{Support different input formats
%   containing slashes}%
% Converts a string with the format \verb|dd/mm/yyyy| to three
% arguments \verb|{#3}{#2}{#1}| and calls \verb|\iso@printdate|.
%    \begin{macrocode}
\def\iso@input@english#1/#2/#3\@empty{%
  \ifthenelse{\equal{\iso@inputformat}{tex}}{%
    \iso@printdate{#1}{#2}{#3}%
  }{%
    \ifthenelse{\equal{\iso@inputformat}{american}}{%
      \iso@printdate{#3}{#1}{#2}%
    }{%
      \iso@printdate{#3}{#2}{#1}%
    }%
  }%
}
%    \end{macrocode}
% \end{macro}
% \begin{macro}{\printdateTeX}
% Prints a date that is given as one argument in the format
% \verb|yyyy/mm/dd|.
%    \begin{macrocode}
\DeclareRobustCommand*\printdateTeX[1]{%
%    \end{macrocode}
% Define \verb|\iso@date| command to expand the argument \verb|#1|.
%    \begin{macrocode}
  \edef\iso@date{#1}%
%    \end{macrocode}
% Count appearances of `/' in the argument.
%    \begin{macrocode}
  \SubStringsToCounter{iso@slash}{/}{\iso@date}%
%    \end{macrocode}
% If number of `/' in the argument is equal to 2 then the \LaTeX\
% format \verb|yyyy/mm/dd| is used.
%    \begin{macrocode}
  \ifthenelse{\equal{\theiso@slash}{2}}{%
    \expandafter\iso@input@TeX\iso@date\@empty}{%
%    \end{macrocode}
% Else no of the formats above is used an thus an error message is thrown.
%    \begin{macrocode}
    ????\iso@isodash ??\iso@isodash ??%
    \PackageError{isodate}{unrecognized date format}{Use one of
      the following formats as macro argument:^^J%
      \space\space dd.mm.yyyy^^J%
      \space\space dd/mm/yyyy^^J%
      \space\space yyyy-mm-dd^^J%
      Don't use any spaces or commands like \protect\, or
      \protect~ inside the argument.}%
    }}
%    \end{macrocode}
% \end{macro}
% \begin{macro}{\iso@input@TeX}
% Converts a string with the format \verb|yyyy/mm/dd| to three
% arguments \verb|{#1}{#2}{#3}| and calls \verb|\iso@printdate|.
%    \begin{macrocode}
\def\iso@input@TeX#1/#2/#3\@empty{\iso@printdate{#1}{#2}{#3}}
%    \end{macrocode}
% \end{macro}
% \begin{macro}{\iso@printmonthday@int}
% ??????
%    \begin{macrocode}
\def\iso@printmonthday@int#1#2{%
  \ifthenelse{\equal{\iso@dateformat}{iso}}{%
    \iso@printmonth{#1}%
    \ifiso@doprintday
      \iso@isodash\iso@printday{#2}%
    \fi
  }{%
    \ifthenelse{\equal{\iso@dateformat}{TeX}}{%
      \iso@printmonth{#1}%
      \ifiso@doprintday
        /\iso@printday{#2}%
      \fi
    }{%
      \PackageError{isodate.sty}{\csname iso@printmonthday\endcsname:
        Invalid date format `\iso@dateformat'}{Internal error. Please
        report to the package author.}
    }%
  }%
}
%    \end{macrocode}
% \end{macro}
% \begin{macro}{\iso@printdate@int}
% ??????
%    \begin{macrocode}
\def\iso@printdate@int#1#2#3{%
  \ifiso@printyear
    \ifthenelse{\equal{\iso@dateformat}{iso}}{%
      \iso@yearfour{\number#1}\iso@isodash%
    }{%
      \ifthenelse{\equal{\iso@dateformat}{TeX}}{%
        \iso@yearfour{\number#1}/%
      }{%
        \PackageError{isodate.sty}{\csname iso@printmonthday\endcsname:
          Invalid date format `\iso@dateformat'}{Internal error. Please
          report to the package author.}
      }%
    }%
  \fi
  \csname iso@printmonthday@int\endcsname{\number#2}{\number#3}%
}
%    \end{macrocode}
% \end{macro}
% \begin{macro}{\iso@daterange@int}
% ??????
%    \begin{macrocode}
\def\iso@daterange@int#1#2#3#4#5#6{%
  \ifthenelse{\equal{\iso@dateformat}{iso}\OR
              \equal{\iso@dateformat}{TeX}}{%
    \csname iso@printdate@\iso@languagename\endcsname{#1}{#2}{#3}%
    \iso@rangesign%
    \ifthenelse{\equal{\number#1}{\number#4}}{%
      \ifthenelse{\equal{\number#2}{\number#5}}{%
        \ifiso@doprintday
          \iso@printday{#6}%
        \else
          \iso@printmonthday@int{#5}{#6}%
        \fi
      }{%
        \iso@printmonthday@int{#5}{#6}%
      }%
    }{%
      \iso@printdate@int{#4}{#5}{#6}%
    }%
  }{%
    \PackageError{isodate.sty}{\csname iso@printmonthday\endcsname:
      Invalid date format `\iso@dateformat'}{Internal error. Please
      report to the package author.}
  }%
}
%    \end{macrocode}
% \end{macro}
% \begin{macro}{\daterange}
% Prints a date range.
%    \begin{macrocode}
\DeclareRobustCommand*\daterange[2]{%
%    \end{macrocode}
% Define \verb|\iso@date| and \verb|\iso@@date| commands to expand the
% argument \verb|#1| and \verb|#2|. Define \verb|\iso@@@date| which
% contains both arguments devided by a komma.
%    \begin{macrocode}
  \edef\iso@date{#1}%
  \edef\iso@@date{#2}%
  \edef\iso@@@date{\iso@date,\iso@@date}%
%    \end{macrocode}
% Count appearances of `/', `-', and `.' in the arguments.
%    \begin{macrocode}
  \SubStringsToCounter{iso@slash}{/}{\iso@date}%
  \SubStringsToCounter{iso@minus}{-}{\iso@date}%
  \SubStringsToCounter{iso@dot}{.}{\iso@date}%
  \SubStringsToCounter{iso@@slash}{/}{\iso@@date}%
  \SubStringsToCounter{iso@@minus}{-}{\iso@@date}%
  \SubStringsToCounter{iso@@dot}{.}{\iso@@date}%
%    \end{macrocode}
% If number of `.' in both arguments is equal to 2 then the German
% format \verb|dd.mm.yyyy| is used.
%    \begin{macrocode}
  \ifthenelse{\equal{\theiso@dot}{2}\AND\equal{\theiso@@dot}{2}}{%
    \expandafter\iso@range@input@german\iso@@@date\@empty}{%
%    \end{macrocode}
% If number of `-' in both arguments is equal to 2 then the ISO
% format \verb|yyyy-mm-dd| is used.
%    \begin{macrocode}
    \ifthenelse{\equal{\theiso@minus}{2}\AND\equal{\theiso@@minus}{2}}{%
      \expandafter\iso@range@input@iso\iso@@@date\@empty}{%
%    \end{macrocode}
% If number of `/' in both arguments is equal to 2 then the British English
% format \verb|dd/mm/yyyy| is used.
%    \begin{macrocode}
      \ifthenelse{\equal{\theiso@slash}{2}\AND
                  \equal{\theiso@@slash}{2}}{%
        \expandafter\iso@range@input@english\iso@@@date\@empty}{%
%    \end{macrocode}
% Else no of the formats above is used an thus an error message is thrown.
%    \begin{macrocode}
        ????\iso@isodash ??\iso@isodash ??%
        \PackageError{isodate}{unrecognized date format}{Use one of
          the following formats as macro argument:^^J%
          \space\space dd.mm.yyyy^^J%
          \space\space dd/mm/yyyy^^J%
          \space\space yyyy-mm-dd^^J%
          Don't use any spaces or commands like \protect\, or
          \protect~ inside the argument.^^J
          Use the same format for both arguments.}%
        }}}%
}
%    \end{macrocode}
% \end{macro}
% \begin{macro}{\iso@range@input@iso}
% \changes{2.21}{2003/12/06}{Support to print date without year}%
% Converts a string with the format \verb|yyyy-mm-dd,yyyy-mm-dd| to six
% arguments \verb|{#1}{#2}{#3}{#4}{#5}{#6}| and calls
% \verb|\iso@daterange@|language. 
%    \begin{macrocode}
\def\iso@range@input@iso#1-#2-#3,#4-#5-#6\@empty{%
  \begingroup
%    \end{macrocode}
% \changes{2.25}{2005/02/21}{Warning for unknown languages}%
% Generate a warning if the active language is not known by |isodate|.
%    \begin{macrocode}
    \@ifundefined{iso@daterange@\iso@languagename}{%
      \PackageWarning{isodate}{Language \iso@languagename\space unknown
        to isodate.\MessageBreak
        Using default date range\MessageBreak
        with range sign --}%
        \expandafter\def\csname iso@printdate@\iso@languagename\endcsname{}%
%    \end{macrocode}
% \changes{2.25}{2005/02/21}{Fall-back format for unknown languages}%
% Print date range in fall-back format.
%    \begin{macrocode}
      \iso@printdate{#1}{#2}{#3}--\iso@printdate{#4}{#5}{#6}%
    }{%
%    \end{macrocode}
% Print date range in the chosen |isodate| format.
%    \begin{macrocode}
      \ifthenelse{\equal{\number#1}{\number#4}}{}{\printyearon}%
      \csname iso@daterange@\iso@languagename\endcsname{%
        #1}{#2}{#3}{#4}{#5}{#6}%
    }%
  \endgroup
}
%    \end{macrocode}
% \end{macro}
% \begin{macro}{\iso@range@input@german}
% \changes{2.21}{2003/12/06}{Support to print date without year}%
% Converts a string with the format \verb|dd.mm.yyyy,dd.mm.yyyy| to six
% arguments \verb|{#3}{#2}{#1}{#6}{#5}{#4}| and calls
% \verb|\iso@daterange@|language. 
%    \begin{macrocode}
\def\iso@range@input@german#1.#2.#3,#4.#5.#6\@empty{%
  \begingroup
%    \end{macrocode}
% \changes{2.25}{2005/02/21}{Warning for unknown languages}%
% Generate a warning if the active language is not known by |isodate|.
%    \begin{macrocode}
    \@ifundefined{iso@daterange@\iso@languagename}{%
      \PackageWarning{isodate}{Language \iso@languagename\space unknown
        to isodate.\MessageBreak
        Using default date range\MessageBreak
        with range sign --}%
        \expandafter\def\csname iso@printdate@\iso@languagename\endcsname{}%
%    \end{macrocode}
% \changes{2.25}{2005/02/21}{Fall-back format for unknown languages}%
% Print date range in fall-back format.
%    \begin{macrocode}
      \iso@printdate{#3}{#2}{#1}--\iso@printdate{#6}{#5}{#4}%
    }{%
%    \end{macrocode}
% Print date range in the chosen |isodate| format.
%    \begin{macrocode}
      \ifthenelse{\equal{\number#3}{\number#6}}{}{\printyearon}%
      \csname iso@daterange@\iso@languagename\endcsname{%
        #3}{#2}{#1}{#6}{#5}{#4}%
    }%
  \endgroup
}
%    \end{macrocode}
% \end{macro}
% \begin{macro}{\iso@range@input@english}
% \changes{2.21}{2003/12/06}{Support to print date without year}%
% Converts a string with the format \verb|dd/mm/yyyy,dd/mm/yyyy| to six
% arguments \verb|{#3}{#2}{#1}{#6}{#5}{#4}| and calls
% \verb|\iso@daterange@|language. 
%    \begin{macrocode}
\def\iso@range@input@english#1/#2/#3,#4/#5/#6\@empty{%
  \begingroup
%    \end{macrocode}
% \changes{2.25}{2005/02/21}{Warning for unknown languages}%
% Generate a warning if the active language is not known by |isodate|.
%    \begin{macrocode}
    \@ifundefined{iso@daterange@\iso@languagename}{%
      \PackageWarning{isodate}{Language \iso@languagename\space unknown
        to isodate.\MessageBreak
        Using default date range\MessageBreak
        with range sign --}%
        \expandafter\def\csname iso@printdate@\iso@languagename\endcsname{}%
%    \end{macrocode}
% \changes{2.26}{2005/03/10}{Support different input formats
%   containing slashes}%
% \changes{2.25}{2005/02/21}{Fall-back format for unknown languages}%
% Print date range in fall-back format.
%    \begin{macrocode}
        \ifthenelse{\equal{\iso@inputformat}{tex}}{%
          \iso@printdate{#1}{#2}{#3}--\iso@printdate{#4}{#5}{#6}%
        }{%
          \ifthenelse{\equal{\iso@inputformat}{american}}{%
            \iso@printdate{#3}{#1}{#2}--\iso@printdate{#6}{#4}{#5}%
          }{%
            \iso@printdate{#3}{#2}{#1}--\iso@printdate{#6}{#5}{#4}%
          }%
        }%
    }{%
%    \end{macrocode}
% \changes{2.26}{2005/03/10}{Support different input formats
%   containing slashes}%
% Print date range in the chosen |isodate| format.
%    \begin{macrocode}
      \ifthenelse{\equal{\number#3}{\number#6}}{}{\printyearon}%
      \ifthenelse{\equal{\iso@inputformat}{tex}}{%
        \csname iso@daterange@\iso@languagename\endcsname{%
          #1}{#2}{#3}{#4}{#5}{#6}%
      }{%
        \ifthenelse{\equal{\iso@inputformat}{american}}{%
          \csname iso@daterange@\iso@languagename\endcsname{%
            #3}{#1}{#2}{#6}{#4}{#5}%
        }{%
          \csname iso@daterange@\iso@languagename\endcsname{%
            #3}{#2}{#1}{#6}{#5}{#4}%
        }%
      }%
    }%
  \endgroup
}
%    \end{macrocode}
% \end{macro}
% Define the counters for conting the appearances of `.', `-', and 
% '/' in the arguments.
%    \begin{macrocode}
\newcounter{iso@slash}
\newcounter{iso@minus}
\newcounter{iso@dot}
\newcounter{iso@@slash}
\newcounter{iso@@minus}
\newcounter{iso@@dot}
%    \end{macrocode}
% The command \verb|\iso@languagename| is defined to be able to use
% this package without loading one of the language packages
% babel.sty, german.sty, or ngerman.sty.
%
% If neither babel.sty nor german.sty nor ngerman.sty is loaded my
% computer returns `nohyphenation' when using \verb|\languagename|.
% So this is the indication that none of the above packages is loaded.
% \changes{2.01}{2000/08/24}{Handle case of not loaded language
%   package babel, german and ngerman}
% \changes{2.12}{2003/10/14}{Test for babel improved}%
% \changes{2.14}{2003/10/26}{Test on babel, german, and ngerman}%
%    \begin{macrocode}
\AtBeginDocument{%
  \@tempswafalse
  \@ifpackageloaded{babel}{%
    \@tempswatrue
    \typeout{isodate: babel.sty has been loaded}%
  }{}%
  \@ifpackageloaded{german}{%
    \@tempswatrue
    \typeout{isodate: german.sty has been loaded}%
  }{}%
  \@ifpackageloaded{ngerman}{%
    \@tempswatrue
    \typeout{isodate: ngerman.sty has been loaded}%
  }{}%
%    \end{macrocode}
% The language is not equal `nohyphenation'. So one of the language
% packages is loaded. Replace the internal language name
% \verb|\iso@languagename| by the global language name \verb|\languagename|.
%    \begin{macrocode}
  \if@tempswa
    \gdef\iso@languagename{\languagename}%
%    \end{macrocode}
% Reload language to surely switch to new date format. The
% languagename gets first expanded because of errors that would occur 
% otherwise.
%    \begin{macrocode}
    \edef\iso@tmplang{\languagename}%
    \expandafter\selectlanguage\expandafter{\iso@tmplang}%
  \else
%    \end{macrocode}
% At the end of the preamble still none of the language packages are loaded.
% So no language switching
% is possible. Set the date language manually to the last language
% that was loaded for isodate.
%    \begin{macrocode}
    \typeout{isodate: babel.sty, (n)german.sty have not been loaded}%
    \csname date\iso@languagename\endcsname%
  \fi
}
%</isodate>
%    \end{macrocode}
%
% \subsection{Language definition file danish.idf}
% \changes{2.26}{2005/03/10}{Force year in four digits for long formats}%
%
% \begin{macro}{\iso@languageloaded}
% Define the command \verb|\iso@languageloaded| in order to enable
% \verb|isodate.sty| to determine if at least one language is loaded.
%    \begin{macrocode}
%<*danish>
\let\iso@languageloaded\active
\typeout{Define commands for Danish date format}
%    \end{macrocode}
% \end{macro}
% \begin{macro}{\month@danish}
% Prints the name of today's month in the long form for the original
% date format.
%    \begin{macrocode}
\def\month@danish{\ifcase\month\or
    januar\or februar\or marts\or april\or maj\or juni\or
    juli\or august\or september\or oktober\or november\or december\fi}
%    \end{macrocode}
% \end{macro}
% \begin{macro}{\iso@printmonthday@danish}
% Prints the month and the day given as two arguments
% (\verb|{mm}{dd}|) in the current date format.
%    \begin{macrocode}
\def\iso@printmonthday@danish#1#2{%
  \ifthenelse{\equal{\iso@dateformat}{iso}\OR
              \equal{\iso@dateformat}{TeX}}{%
    \iso@printmonthday@int{#1}{#2}%
  }{%
%    \end{macrocode}
% Numeric and short date format: \verb|dd/mm/|
%    \begin{macrocode}
    \ifthenelse{\equal{\iso@dateformat}{numeric}\OR
                \equal{\iso@dateformat}{short}}{%
      \ifiso@doprintday
        \iso@printday{#2}/%
      \fi
      \iso@printmonth{#1}%
    }{%
%    \end{macrocode}
% Original date format: \verb|d. mmm|
%    \begin{macrocode}
      \ifthenelse{\equal{\iso@dateformat}{orig}\OR
                  \equal{\iso@dateformat}{shortorig}}{%
        \ifiso@doprintday
          \iso@printday{#2}.\iso@monthsep
        \fi
        \begingroup
        \edef\lmonth{#1}\def\month{\lmonth}%
        \month@danish%
        \endgroup
      }{}%
    }%
  }%
}
%    \end{macrocode}
% \end{macro}
% \begin{macro}{\iso@printdate@danish}
% Prints the date given as three arguments (\verb|{yyyy}{mm}{dd}|) in
% the actual date format
%    \begin{macrocode}
\def\iso@printdate@danish#1#2#3{%
%    \end{macrocode}
% ISO or \LaTeX date format: \verb|yyyy\iso@printmonthday@danish|
%    \begin{macrocode}
  \ifthenelse{\equal{\iso@dateformat}{iso}\OR
              \equal{\iso@dateformat}{TeX}}{%
    \iso@printdate@int{#1}{#2}{#3}%
  }{%
    \iso@printmonthday@danish{\number#2}{\number#3}%
    \ifiso@printyear
%    \end{macrocode}
% ?????
%    \begin{macrocode}
      \ifthenelse{\equal{\iso@dateformat}{orig}\OR
                  \equal{\iso@dateformat}{shortorig}}{%
      }{%
        /%
      }%
%    \end{macrocode}
% numeric date format: \verb|\iso@printmonthday@danish yyyy|
%    \begin{macrocode}
      \ifthenelse{\equal{\iso@dateformat}{numeric}}{%
        \iso@yearfour{\number#1}%
      }{%
%    \end{macrocode}
% original date format: \verb|\iso@printmonthday@danish~yyyy|
%    \begin{macrocode}
        \ifthenelse{\equal{\iso@dateformat}{orig}}{%
          \iso@yearsep\iso@yearfour{\number#1}%
        }{%
%    \end{macrocode}
% short original date format: \verb|\iso@printmonthday@danish~yy|
%    \begin{macrocode}
          \ifthenelse{\equal{\iso@dateformat}{shortorig}}{%
            \iso@yearsep\iso@twodigitsign\iso@yeartwo{\number#1}%
          }{%
%    \end{macrocode}
% short date format: \verb|\iso@printmonthday@danish yy|
%    \begin{macrocode}
            \ifthenelse{\equal{\iso@dateformat}{short}}{%
              \iso@yeartwo{\number#1}%
            }{}%
          }%
        }%
      }%
    \fi
  }%
}
%    \end{macrocode}
% \end{macro}
% \begin{macro}{\iso@datedanish}
% This command redefines the \cs{today} command to print in the
% actual date format.
%    \begin{macrocode}
\def\iso@datedanish{%
  \def\today{\iso@printdate@danish{\year}{\month}{\day}}}%
%    \end{macrocode}
% \end{macro}
% \begin{macro}{\iso@daterange@...}
% Define date-range commands for dialects.
%    \begin{macrocode}
\expandafter\def\csname iso@daterange@\CurrentOption\endcsname{%
  \iso@daterange@danish}%
%    \end{macrocode}
% \end{macro}
% \begin{macro}{\iso@daterange@danish}
% This command takes six arguments
% (\verb|{yyyy1}{mm1}{dd1}{yyyy2}{mm2}{dd2}|) and prints the corrosponding
% date range in the actual date format.
%    \begin{macrocode}
\def\iso@daterange@danish#1#2#3#4#5#6{%
%    \end{macrocode}
% ISO or \LaTeX\ date format.
%    \begin{macrocode}
  \ifthenelse{\equal{\iso@dateformat}{iso}\OR
              \equal{\iso@dateformat}{TeX}}{%
%    \end{macrocode}
% Call the appropriate international routine.
%    \begin{macrocode}
    \iso@daterange@int{#1}{#2}{#3}{#4}{#5}{#6}%
  }{%
%    \end{macrocode}
% Numeric, short, or original date format.
%
% If year and month are equal, only print the day of the start date. If
% only the year is equal, only print month and day of the start
% date. Otherwise print the whole start date.
%    \begin{macrocode}
    \ifthenelse{\equal{\number#1}{\number#4}}{%
      \ifthenelse{\equal{\number#2}{\number#5}}{%
        \ifiso@doprintday
          \ifthenelse{\equal{\iso@dateformat}{orig}\OR
                      \equal{\iso@dateformat}{shortorig}}{%
            \iso@printday{#3}.%
          }{%
            \iso@printday{#3}%
          }%
        \else
          \iso@printmonthday@danish{#2}{#3}%
        \fi
      }{%
        \iso@printmonthday@danish{#2}{#3}%
      }%
    }{%
      \csname iso@printdate@\iso@languagename\endcsname{#1}{#2}{#3}%
    }%
%    \end{macrocode}
% Print the end date.
%    \begin{macrocode}
    \iso@rangesign
    \csname iso@printdate@\iso@languagename\endcsname{#4}{#5}{#6}%
  }%
}
%    \end{macrocode}
% \end{macro}
% \begin{macro}{\iso@rangesign@danish}
% Sets the word between start and end date in a date range to `~til~'.
%    \begin{macrocode}
\expandafter\def\csname iso@rangesign@\CurrentOption\endcsname{~til~}
%    \end{macrocode}
% \end{macro}
% Define the language name that will the active language for isodate
% if none of the packages babel.sty, german.sty, and ngerman.sty is
% loaded and if this is the last language that is used for isodate.
% If one of the above packages is used this definition will be
% overridden by the command \verb|\languagename| that will always
% return the current used language.
%    \begin{macrocode}
\def\iso@languagename{danish}%
%    \end{macrocode}
% Redefine the command \verb|\datedanish| that is used by babel to
% switch to the original Danish date format to enable the use of
% different date formats.
% This has to be done after the preamble in order to ensure to overwrite
% the babel command.
%    \begin{macrocode}
\AtBeginDocument{%
  \ifx\undefined\iso@datedanish\else
    \def\datedanish{\iso@datedanish}%
  \fi
}
%</danish>
%    \end{macrocode}
%
% \subsection{Language definition file english.idf}
% \changes{2.26}{2005/03/10}{Force year in four digits for long formats}%
%
% \begin{macro}{\iso@languageloaded}
% Define the command \verb|\iso@languageloaded| in order to enable
% \verb|isodate.sty| to determine if at least one language is loaded.
%    \begin{macrocode}
%<*english>
\let\iso@languageloaded\active
%    \end{macrocode}
% \end{macro}
% \begin{macro}{\month@english}
% Prints the name of today's month in the long form for the original
% date format.
%    \begin{macrocode}
\def\month@english{\ifcase\month\or
    January\or February\or March\or April\or May\or June\or
    July\or August\or September\or October\or November\or December\fi}
%    \end{macrocode}
% \end{macro}
% British and American English dates are very different. So handle
% them seperately. It might have been easier to put them in different
% files but I wanted to organize my files analogous to babel.
%
% First handle British English.
%    \begin{macrocode}
\ifthenelse{\equal{\CurrentOption}{english}\OR
            \equal{\CurrentOption}{british}\OR
            \equal{\CurrentOption}{UKenglish}}{%
  \typeout{Define commands for English date format}
%    \end{macrocode}
% \begin{macro}{\day@english}
% \changes{2.28}{2005/04/15}{Introduce option cleanlook for English
%   date format}%
% Prints today's day for the original date format.
%    \begin{macrocode}
  \def\day@english{%
    \ifiso@cleanlook
      \day
    \else
      \ifcase\day\or
        1st\or 2nd\or 3rd\or 4th\or 5th\or
        6th\or 7th\or 8th\or 9th\or 10th\or
        11th\or 12th\or 13th\or 14th\or 15th\or
        16th\or 17th\or 18th\or 19th\or 20th\or
        21st\or 22nd\or 23rd\or 24th\or 25th\or
        26th\or 27th\or 28th\or 29th\or 30th\or
        31st%
      \fi
    \fi
  }
%    \end{macrocode}
% \end{macro}
% \begin{macro}{\iso@printmonthday@english}
% Prints the month and the day given as two arguments
% (\verb|{mm}{dd}|) in the current date format.
%    \begin{macrocode}
  \def\iso@printmonthday@english#1#2{%
%    \end{macrocode}
% Numeric and short date format: \verb|dd/mm/|
%    \begin{macrocode}
    \ifthenelse{\equal{\iso@dateformat}{iso}\OR
                \equal{\iso@dateformat}{TeX}}{%
      \iso@printmonthday@int{#1}{#2}%
    }{%
      \ifthenelse{\equal{\iso@dateformat}{numeric}\OR
                  \equal{\iso@dateformat}{short}}{%
        \ifiso@doprintday
          \iso@printday{#2}/%
        \fi
        \iso@printmonth{#1}%
      }{%
%    \end{macrocode}
% Original date format: \verb|ddd mmm|
%    \begin{macrocode}
        \ifthenelse{\equal{\iso@dateformat}{orig}\OR
                    \equal{\iso@dateformat}{shortorig}}{%
          \begingroup
          \edef\lday{#2}\def\day{\lday}%
          \edef\lmonth{#1}\def\month{\lmonth}%
          \ifiso@doprintday
            \day@english\iso@monthsep\@empty
          \fi
          \month@english
          \endgroup
        }{}%
      }%
    }%
  }
%    \end{macrocode}
% \end{macro}
% \begin{macro}{\iso@printdate@english}
% Prints the date given as three arguments (\verb|{yyyy}{mm}{dd}|) in
% the actual date format.
%    \begin{macrocode}
  \def\iso@printdate@english#1#2#3{%
    \ifthenelse{\equal{\iso@dateformat}{iso}\OR
                \equal{\iso@dateformat}{TeX}}{%
      \iso@printdate@int{#1}{#2}{#3}%
    }{%
%    \end{macrocode}
% ISO date format: \verb|yyyy-\iso@printmonthday@english|
%    \begin{macrocode}
      \iso@printmonthday@english{\number#2}{\number#3}%
%    \end{macrocode}
% Numeric date format: \verb|\iso@printmonthday@english yyyy|
%    \begin{macrocode}
      \ifiso@printyear
        \ifthenelse{\equal{\iso@dateformat}{orig}\OR
                    \equal{\iso@dateformat}{shortorig}}{%
        }{%
          /%
        }%
        \ifthenelse{\equal{\iso@dateformat}{numeric}}{%
          \iso@yearfour{\number#1}%
        }{%
%    \end{macrocode}
% Original date format: \verb|\iso@printmonthday@english~yyyy|
%    \begin{macrocode}
          \ifthenelse{\equal{\iso@dateformat}{orig}}{%
            \iso@yearsep\iso@yearfour{\number#1}%
          }{%
%    \end{macrocode}
% Short original date format: \verb|\iso@printmonthday@english~yy|
%    \begin{macrocode}
            \ifthenelse{\equal{\iso@dateformat}{shortorig}}{%
              \iso@yearsep\iso@twodigitsign\iso@yeartwo{\number#1}%
            }{%
%    \end{macrocode}
% Short date format: \verb|\iso@printmonthday@english yy|
%    \begin{macrocode}
              \ifthenelse{\equal{\iso@dateformat}{short}}{%
                \iso@yeartwo{\number#1}%
              }{}%
            }%
          }%
        }%
      \fi
    }%
  }
%    \end{macrocode}
% \end{macro}
% \begin{macro}{\iso@printdate@UKenglish}
% Just a second name for \verb|\iso@printdate@UKenglish|.
%    \begin{macrocode}
  \def\iso@printdate@UKenglish{\iso@printdate@english}
  \def\iso@printdate@british{\iso@printdate@english}
%    \end{macrocode}
% \end{macro}
% \begin{macro}{\iso@dateenglish}
% This command redefines the \cs{today} command to print in the
% actual date format.
%    \begin{macrocode}
  \def\iso@dateenglish{%
    \def\today{\iso@printdate@english{\year}{\month}{\day}}}%
%    \end{macrocode}
% \end{macro}
% \begin{macro}{\iso@daterange@...}
% Define date-range commands for dialects of English.
%    \begin{macrocode}
  \expandafter\def\csname iso@daterange@\CurrentOption\endcsname{%
    \iso@daterange@english}%
%    \end{macrocode}
% \end{macro}
% \begin{macro}{\iso@daterange@english}
% This command takes six arguments
% (\verb|{yyyy1}{mm1}{dd1}{yyyy2}{mm2}{dd2}|) and prints the corrosponding
% date range in the actual date format.
%    \begin{macrocode}
  \def\iso@daterange@english#1#2#3#4#5#6{%
%    \end{macrocode}
% ISO or \LaTeX\ date format.
%    \begin{macrocode}
    \ifthenelse{\equal{\iso@dateformat}{iso}\OR
                \equal{\iso@dateformat}{TeX}}{%
%    \end{macrocode}
% Print the start date.
%    \begin{macrocode}
      \iso@daterange@int{#1}{#2}{#3}{#4}{#5}{#6}%
    }{%
%    \end{macrocode}
% Numeric, short, or original date format.
%
% If year and month are equal, only print the day of the start date. If
% only the year is equal, only print month and day of the start
% date. Otherwise print the whole start date.
%    \begin{macrocode}
      \ifthenelse{\equal{\number#1}{\number#4}}{%
        \ifthenelse{\equal{\number#2}{\number#5}}{%
          \ifiso@doprintday
            \ifthenelse{\equal{\iso@dateformat}{orig}\OR
                        \equal{\iso@dateformat}{shortorig}}{%
              \begingroup
              \edef\lday{#3}\def\day{\lday}%
              \day@english
              \endgroup
            }{%
              \iso@printday{#3}%
            }%
          \else
            \csname iso@printmonthday@\iso@languagename\endcsname{#2}{#3}%
          \fi
        }{%
          \csname iso@printmonthday@\iso@languagename\endcsname{#2}{#3}%
        }%
      }{%
        \csname iso@printdate@\iso@languagename\endcsname{#1}{#2}{#3}%
      }%
%    \end{macrocode}
% Print the end date.
%    \begin{macrocode}
      \iso@rangesign
      \csname iso@printdate@\iso@languagename\endcsname{#4}{#5}{#6}%
    }%
  }
%    \end{macrocode}
% \end{macro}
% Define the language name that will the active language for isodate
% if none of the packages babel.sty, german.sty, and ngerman.sty is
% loaded and if this is the last language that is used for isodate.
% If one of the above packages is used this definition will be
% overridden by the command \verb|\languagename| that will always
% return the current used language.
%    \begin{macrocode}
  \def\iso@languagename{english}%
%    \end{macrocode}
% The end of the British section.
%
% \changes{2.20}{2003/12/06}{Add Australian and New Zealand}%
% Second handle Australian and New Zealand.
%    \begin{macrocode}
}{%
  \ifthenelse{\equal{\CurrentOption}{australian}\OR
              \equal{\CurrentOption}{newzealand}}{%
    \typeout{Define commands for Australian date format}
%    \end{macrocode}
% \begin{macro}{\iso@printmonthday@australian}
% Prints the month and the day given as two arguments
% (\verb|{mm}{dd}|) in the current date format.
%    \begin{macrocode}
    \def\iso@printmonthday@australian#1#2{%
      \ifthenelse{\equal{\iso@dateformat}{iso}\OR
                  \equal{\iso@dateformat}{TeX}}{%
        \iso@printmonthday@int{#1}{#2}%
      }{%
%    \end{macrocode}
% Numeric and short date format: \verb|dd/mm/|
%    \begin{macrocode}
        \ifthenelse{\equal{\iso@dateformat}{numeric}\OR
                    \equal{\iso@dateformat}{short}}{%
          \ifiso@doprintday
            \iso@printday{#2}/%
          \fi
          \iso@printmonth{#1}%
        }{%
%    \end{macrocode}
% Original date format: \verb|ddd mmm|
%    \begin{macrocode}
          \ifthenelse{\equal{\iso@dateformat}{orig}\OR
                      \equal{\iso@dateformat}{shortorig}}{%
            \begingroup
            \edef\lmonth{#1}\def\month{\lmonth}%
            \ifiso@doprintday
              \iso@printday{#2}\iso@monthsep\@empty
            \fi
            \month@english
            \endgroup
          }{}%
        }%
      }%
    }
%    \end{macrocode}
% \end{macro}
% \begin{macro}{\iso@printdate@australian}
% Prints the date given as three arguments (\verb|{yyyy}{mm}{dd}|) in
% the actual date format.
%    \begin{macrocode}
    \def\iso@printdate@australian#1#2#3{%
      \ifthenelse{\equal{\iso@dateformat}{iso}\OR
                  \equal{\iso@dateformat}{TeX}}{%
        \iso@printdate@int{#1}{#2}{#3}%
      }{%
        \iso@printmonthday@australian{\number#2}{\number#3}%
%    \end{macrocode}
% Numeric date format: \verb|\iso@printmonthday@australian yyyy|
%    \begin{macrocode}
        \ifiso@printyear
          \ifthenelse{\equal{\iso@dateformat}{orig}\OR
                      \equal{\iso@dateformat}{shortorig}}{%
          }{%
            /%
          }%
          \ifthenelse{\equal{\iso@dateformat}{numeric}}{%
            \iso@yearfour{\number#1}%
          }{%
%    \end{macrocode}
% Original date format: \verb|\iso@printmonthday@australian~yyyy|
%    \begin{macrocode}
            \ifthenelse{\equal{\iso@dateformat}{orig}}{%
              \iso@yearsep\iso@yearfour{\number#1}%
            }{%
%    \end{macrocode}
% Short original date format: \verb|\iso@printmonthday@australian~yy|
%    \begin{macrocode}
              \ifthenelse{\equal{\iso@dateformat}{shortorig}}{%
                \iso@yearsep\iso@twodigitsign\iso@yeartwo{\number#1}%
              }{%
%    \end{macrocode}
% Short date format: \verb|\iso@printmonthday@australian yy|
%    \begin{macrocode}
                \ifthenelse{\equal{\iso@dateformat}{short}}{%
                  \iso@yeartwo{\number#1}%
                }{}%
              }%
            }%
          }%
        \fi
      }%
    }
%    \end{macrocode}
% \end{macro}
% \begin{macro}{\iso@printdate@newzealand}
% Just a second name for \verb|\iso@printdate@UKenglish|.
%    \begin{macrocode}
    \def\iso@printdate@newzealand{\iso@printdate@australian}
%    \end{macrocode}
% \end{macro}
% \begin{macro}{\iso@dateaustralian}
% This command redefines the \cs{today} command to print in the
% actual date format.
%    \begin{macrocode}
    \def\iso@dateaustralian{%
      \def\today{\iso@printdate@australian{\year}{\month}{\day}}}%
%    \end{macrocode}
% \end{macro}
% \begin{macro}{\iso@daterange@...}
% Define date-range commands for dialects of Australian.
%    \begin{macrocode}
    \expandafter\def\csname iso@daterange@\CurrentOption\endcsname{%
      \iso@daterange@australian}%
%    \end{macrocode}
% \end{macro}
% \begin{macro}{\iso@daterange@australian}
% This command takes six arguments
% (\verb|{yyyy1}{mm1}{dd1}{yyyy2}{mm2}{dd2}|) and prints the corrosponding
% date range in the actual date format.
%    \begin{macrocode}
    \def\iso@daterange@australian#1#2#3#4#5#6{%
%    \end{macrocode}
% ISO or \LaTeX\ date format.
%    \begin{macrocode}
      \ifthenelse{\equal{\iso@dateformat}{iso}\OR
                  \equal{\iso@dateformat}{TeX}}{%
%    \end{macrocode}
% Print the start date.
%    \begin{macrocode}
        \iso@daterange@int{#1}{#2}{#3}{#4}{#5}{#6}%
      }{%
%    \end{macrocode}
% Numeric, short, or original date format.
%
% If year and month are equal, only print the day of the start date. If
% only the year is equal, only print month and day of the start
% date. Otherwise print the whole start date.
%    \begin{macrocode}
        \ifthenelse{\equal{\number#1}{\number#4}}{%
          \ifthenelse{\equal{\number#2}{\number#5}}{%
            \ifiso@doprintday
              \iso@printday{#3}%
            \else
              \csname iso@printmonthday@\iso@languagename\endcsname{#2}{#3}%
            \fi
          }{%
            \iso@printmonthday@australian{#2}{#3}%
          }%
        }{%
          \csname iso@printdate@\iso@languagename\endcsname{#1}{#2}{#3}%
        }%
%    \end{macrocode}
% Print the end date.
%    \begin{macrocode}
        \iso@rangesign
        \csname iso@printdate@\iso@languagename\endcsname{#4}{#5}{#6}%
      }%
    }
%    \end{macrocode}
% \end{macro}
% Define the language name that will the active language for isodate
% if none of the packages babel.sty, german.sty, and ngerman.sty is
% loaded and if this is the last language that is used for isodate.
% If one of the above packages is used this definition will be
% overridden by the command \verb|\languagename| that will always
% return the current used language.
%    \begin{macrocode}
    \def\iso@languagename{australian}%
%    \end{macrocode}
% The end of the Australian section.
%
% Third, handle American.
%    \begin{macrocode}
  }{%
    \typeout{Define commands for American date format}
%    \end{macrocode}
% \begin{macro}{\iso@printmonthday@american}
% Prints the month and the day given as two arguments
% (\verb|{mm}{dd}|) in the current date format.
%    \begin{macrocode}
    \def\iso@printmonthday@american#1#2{%
      \ifthenelse{\equal{\iso@dateformat}{iso}\OR
                  \equal{\iso@dateformat}{TeX}}{%
        \iso@printmonthday@int{#1}{#2}%
      }{%
%    \end{macrocode}
% Numeric and short date format: \verb|mm/dd/|
%    \begin{macrocode}
        \ifthenelse{\equal{\iso@dateformat}{numeric}\OR
                    \equal{\iso@dateformat}{short}}{%
          \iso@printmonth{#1}%
          \ifiso@doprintday
            /\iso@printday{#2}%
          \fi
        }{%
%    \end{macrocode}
% Original date format: \verb|mmm d|
%    \begin{macrocode}
          \ifthenelse{\equal{\iso@dateformat}{orig}\OR
                      \equal{\iso@dateformat}{shortorig}}{%
            \begingroup%
            \edef\lmonth{#1}%
            \def\month{\lmonth}%
            \month@english%
            \endgroup
            \ifiso@doprintday
              \iso@daysep\iso@printday{#2}%
            \fi
          }{}%
        }%
      }%
    }
%    \end{macrocode}
% \end{macro}
% \begin{macro}{\iso@printdate@american}
% Prints the date given as three arguments (\verb|{yyyy}{mm}{dd}|) in
% the actual date format.
%    \begin{macrocode}
    \def\iso@printdate@american#1#2#3{%
      \ifthenelse{\equal{\iso@dateformat}{iso}\OR
                  \equal{\iso@dateformat}{TeX}}{%
        \iso@printdate@int{#1}{#2}{#3}%
      }{%
        \iso@printmonthday@american{\number#2}{\number#3}%
%    \end{macrocode}
% Numeric date format: \verb|\iso@printmonthday@american yyyy|
%    \begin{macrocode}
        \ifiso@printyear
          \ifthenelse{\equal{\iso@dateformat}{orig}\OR
                      \equal{\iso@dateformat}{shortorig}}{%
          }{%
            /%
          }%
          \ifthenelse{\equal{\iso@dateformat}{numeric}}{%
            \iso@yearfour{\number#1}%
          }{%
%    \end{macrocode}
% Original date format: \verb|\iso@printmonthday@american,~yyyy|
%    \begin{macrocode}
            \ifthenelse{\equal{\iso@dateformat}{orig}}{%
              \ifiso@doprintday,\fi
              \iso@yearsep\iso@yearfour{\number#1}%
            }{%
%    \end{macrocode}
% Short original date format: \verb|\iso@printmonthday@american,~yyyy|
%    \begin{macrocode}
              \ifthenelse{\equal{\iso@dateformat}{shortorig}}{%
                \ifiso@doprintday,\fi
                \iso@yearsep\iso@twodigitsign\iso@yeartwo{\number#1}%
              }{%
%    \end{macrocode}
% Short date format: \verb|\iso@printmonthday@american yy|
%    \begin{macrocode}
                \ifthenelse{\equal{\iso@dateformat}{short}}{%
                  \iso@yeartwo{\number#1}%
                }{}%
              }%
            }%
          }%
        \fi
      }%
    }
%    \end{macrocode}
% \end{macro}
% \begin{macro}{\iso@printdate@USenglish}
% Just a second name for \verb|\iso@printdate@UKamerican|.
%    \begin{macrocode}
    \def\iso@printdate@USenglish{\iso@printdate@american}
%    \end{macrocode}
% \end{macro}
% \begin{macro}{\iso@dateamerican}
% This command redefines the \cs{today} command to print in the
% actual date format.
%    \begin{macrocode}
    \def\iso@dateamerican{%
      \def\today{\iso@printdate@american{\year}{\month}{\day}}}%
%    \end{macrocode}
% \end{macro}
% \begin{macro}{\iso@daterange@...}
% Define date-range commands for dialects of American.
%    \begin{macrocode}
    \expandafter\def\csname iso@daterange@\CurrentOption\endcsname{%
      \iso@daterange@american}%
%    \end{macrocode}
% \end{macro}
% \begin{macro}{\iso@daterange@american}
% This command takes six arguments
% (\verb|{yyyy1}{mm1}{dd1}{yyyy2}{mm2}{dd2}|) and prints the corrosponding
% date range in the actual date format.
%    \begin{macrocode}
    \def\iso@daterange@american#1#2#3#4#5#6{%
%    \end{macrocode}
% ISO or \LaTeX\ date format.
%    \begin{macrocode}
      \ifthenelse{\equal{\iso@dateformat}{iso}\OR
                  \equal{\iso@dateformat}{TeX}}{%
%    \end{macrocode}
% Print the start date.
%    \begin{macrocode}
        \iso@daterange@int{#1}{#2}{#3}{#4}{#5}{#6}%
      }{%
%    \end{macrocode}
% Original date format.
%
% If year and month are equal, print \verb|mmm d1 to d2, yyyy|. If
% only the year is equal, print \verb|mmm1 d1 to mmm2 d2, yyyy|. 
% Otherwise print the whole start and end date.
%    \begin{macrocode}
        \ifthenelse{\equal{\iso@dateformat}{orig}\OR
                    \equal{\iso@dateformat}{shortorig}}{%
          \ifthenelse{\equal{\number#1}{\number#4}}{%
            \ifthenelse{\equal{\number#2}{\number#5}}{%
              \iso@printmonthday@american{#2}{#3}%
              \iso@rangesign
              \ifiso@doprintday
                \iso@printday{#6},\iso@yearsep\@empty
              \else
                \iso@printmonthday@american{#5}{#6}\iso@yearsep\@empty
              \fi
              \ifthenelse{\equal{\iso@dateformat}{orig}}{%
                \iso@yearfour{\number#4}%
              }{%
                \iso@twodigitsign\iso@yeartwo{\number#4}%
              }%
            }{%
              \iso@printmonthday@american{#2}{#3}%
              \iso@rangesign
              \csname iso@printdate@\iso@languagename\endcsname{%
                #4}{#5}{#6}%
            }%
          }{%
            \csname iso@printdate@\iso@languagename\endcsname{#1}{#2}{#3}%
            \iso@rangesign%
            \csname iso@printdate@\iso@languagename\endcsname{#4}{#5}{#6}%
          }%
        }{%
%    \end{macrocode}
% Numeric or short date format.
%
% If year and month are equal, only print the day of the end date.
% Otherwise print the whole end date.
%    \begin{macrocode}
          \ifthenelse{\equal{\number#1}{\number#4}}{%
            \iso@printmonthday@american{#2}{#3}%
          }{%
            \csname iso@printdate@\iso@languagename\endcsname{#1}{#2}{#3}%
          }%
%    \end{macrocode}
% Print the end date.
%    \begin{macrocode}
          \iso@rangesign
          \csname iso@printdate@\iso@languagename\endcsname{#4}{#5}{#6}%
        }%
      }%
    }
%    \end{macrocode}
% \end{macro}
% Define the language name that will the active language for isodate
% if none of the packages babel.sty, german.sty, and ngerman.sty is
% loaded and if this is the last language that is used for isodate.
% If one of the above packages is used this definition will be
% overridden by the command \verb|\languagename| that will always
% return the current used language.
%    \begin{macrocode}
    \def\iso@languagename{american}%
%    \end{macrocode}
% The end of the American section.
%    \begin{macrocode}
  }
}
%    \end{macrocode}
% \begin{macro}{\iso@rangesign@...}
% Sets the word between start and end date in a date range to `~to'.
%    \begin{macrocode}
\expandafter\def\csname iso@rangesign@\CurrentOption\endcsname{~to~}
%    \end{macrocode}
% \end{macro}
% Redefine the command \verb|date|language that is used by babel.sty,
% german.sty, and ngerman.sty to switch to the original
% English/American date format to enable the use of different date
% formats.
% This has to be done after the preamble in order to ensure to overwrite
% the babel command.
%
% Do this only if \verb|\iso@date|language is defined.
%    \begin{macrocode}
\AtBeginDocument{%
  \ifx\undefined\iso@dateenglish\else
    \def\dateenglish{\iso@dateenglish}%
    \def\datebritish{\iso@dateenglish}%
    \def\dateUKenglish{\iso@dateenglish}%
  \fi
  \ifx\undefined\iso@dateaustralian\else
    \def\dateaustralian{\iso@dateaustralian}%
    \def\datenewzealand{\iso@dateaustralian}%
  \fi
  \ifx\undefined\iso@dateamerican\else
    \def\dateamerican{\iso@dateamerican}%
    \def\dateUSenglish{\iso@dateamerican}%
  \fi
}
%</english>
%    \end{macrocode}
%
% \subsection{Language definition file french.idf}
% \changes{2.26}{2005/03/10}{Force year in four digits for long formats}%
%
% \begin{macro}{\iso@languageloaded}
% Define the command \verb|\iso@languageloaded| in order to enable
% \verb|isodate.sty| to determine if at least one language is loaded.
%    \begin{macrocode}
%<*french>
\let\iso@languageloaded\active
\typeout{Define commands for French date format}
%    \end{macrocode}
% \end{macro}
%    \begin{macrocode}
\def\month@french{\ifcase\month\or
  janvier\or f\'evrier\or mars\or avril\or mai\or juin\or
  juillet\or ao\^ut\or septembre\or octobre\or novembre\or
  d\'ecembre\fi}
%    \end{macrocode}
%    \begin{macrocode}
\def\iso@printmonthday@french#1#2{%
  \ifthenelse{\equal{\iso@dateformat}{iso}\OR
              \equal{\iso@dateformat}{TeX}}{%
    \iso@printmonthday@int{#1}{#2}%
  }{%
    \ifthenelse{\equal{\iso@dateformat}{numeric}\OR
                \equal{\iso@dateformat}{short}}{%
      \ifiso@doprintday
        \iso@printday{#2}/%
      \fi
      \iso@printmonth{#1}%
    }{%
      \ifthenelse{\equal{\iso@dateformat}{orig}\OR
                  \equal{\iso@dateformat}{shortorig}}{%
        \begingroup
        \edef\lday{#2}\edef\day{\lday}%
        \edef\lmonth{#1}\def\month{\lmonth}%
        \ifiso@doprintday
          \number\day\ifnum1=\day \noexpand\ier\fi\iso@monthsep
        \fi
        \month@french
        \endgroup
      }{}%
    }%
  }%
}
%    \end{macrocode}
%    \begin{macrocode}
\def\iso@printdate@french#1#2#3{%
  \ifthenelse{\equal{\iso@dateformat}{iso}\OR
              \equal{\iso@dateformat}{TeX}}{%
    \iso@printdate@int{#1}{#2}{#3}%
  }{%
    \iso@printmonthday@french{\number#2}{\number#3}%
    \ifiso@printyear
      \ifthenelse{\equal{\iso@dateformat}{orig}\OR
                  \equal{\iso@dateformat}{shortorig}}{%
      }{%
        /%
      }%
      \ifthenelse{\equal{\iso@dateformat}{numeric}}{%
        \iso@yearfour{\number#1}%
      }{%
        \ifthenelse{\equal{\iso@dateformat}{orig}}{%
          \iso@yearsep\iso@yearfour{\number#1}%
        }{%
          \ifthenelse{\equal{\iso@dateformat}{shortorig}}{%
            \iso@yearsep\iso@twodigitsign\iso@yeartwo{\number#1}%
          }{%
            \ifthenelse{\equal{\iso@dateformat}{short}}{%
              \iso@yeartwo{\number#1}%
            }{}%
          }%
        }%
      }%
    \fi
  }%
}
%    \end{macrocode}
%    \begin{macrocode}
\def\iso@datefrench{%
  \def\today{\iso@printdate@french{\year}{\month}{\day}}}%
%    \end{macrocode}
% \begin{macro}{\iso@daterange@...}
% Define date-range commands for dialects.
%    \begin{macrocode}
\expandafter\def\csname iso@daterange@\CurrentOption\endcsname{%
  \iso@daterange@french}%
%    \end{macrocode}
% \end{macro}
%    \begin{macrocode}
\def\iso@daterange@french#1#2#3#4#5#6{%
  \ifthenelse{\equal{\iso@dateformat}{iso}\OR
              \equal{\iso@dateformat}{TeX}}{%
    \iso@daterange@int{#1}{#2}{#3}{#4}{#5}{#6}%
  }{%
%    \end{macrocode}
%    \begin{macrocode}
    \ifthenelse{\equal{\number#1}{\number#4}}{%
      \ifthenelse{\equal{\number#2}{\number#5}}{%
        \ifiso@doprintday
          \ifthenelse{\equal{\iso@dateformat}{orig}}{%
            \begingroup
            \edef\lday{#3}\edef\day{\lday}%
            \number\day\ifnum1=\day \noexpand\ier\fi
            \endgroup
          }{%
            \iso@printday{#3}%
          }%
        \else
          \csname iso@printmonthday@\iso@languagename\endcsname{#2}{#3}%
        \fi
      }{%
        \iso@printmonthday@french{#2}{#3}%
      }%
    }{%
      \csname iso@printdate@\iso@languagename\endcsname{#1}{#2}{#3}%
    }%
    \iso@rangesign
    \csname iso@printdate@\iso@languagename\endcsname{#4}{#5}{#6}%
  }%
}
%    \end{macrocode}
%    \begin{macrocode}
\expandafter\def\csname iso@rangesign@\CurrentOption\endcsname{~au~}
%    \end{macrocode}
% \changes{2.06}{2002/04/08}{Changed range sign for French language,
% thanks to Felix P\"utsch}
% Define the language name that will the active language for isodate
% if none of the packages babel.sty, german.sty, and ngerman.sty is
% loaded and if this is the last language that is used for isodate.
% If one of the above packages is used this definition will be
% overridden by the command \verb|\languagename| that will always
% return the current used language.
%    \begin{macrocode}
\def\iso@languagename{french}%
%    \end{macrocode}
% \changes{2.03}{2001/05/04}{Fixed a bug in the French language that caused
%   not to switch to it correctly on startup.}
% \verb|\datefrenchb| has to be defined additionally because babel starts
% with language frenchb instead of french.
%    \begin{macrocode}
\AtBeginDocument{%
  \ifx\undefined\iso@datefrench\else
    \def\datefrench{\iso@datefrench}%
    \def\datefrenchb{\iso@datefrench}%
  \fi
}
%</french>
%    \end{macrocode}
%
% \subsection{Language definition file german.idf}
% \changes{2.26}{2005/03/10}{Force year in four digits for long formats}%
%
% \begin{macro}{\iso@languageloaded}
% Define the command \verb|\iso@languageloaded| in order to enable
% \verb|isodate.sty| to determine if at least one language is loaded.
%    \begin{macrocode}
%<*german>
\let\iso@languageloaded\active
\typeout{Define commands for German date format (\CurrentOption)}
%    \end{macrocode}
% \end{macro}
% \changes{2.03}{2001/05/04}{Allow change of spaces for German language}%
% Define spaces between day and month resp. month and year. \verb|dm| 
% stands for day-month and \verb|my| for month-year. The defaults are taken
% from the Duden \cite{duden1996a}.
%    \begin{macrocode}
\def\iso@dmsepgerman{\,}%
\def\iso@mylongsepgerman{~}%
\def\iso@myshortsepgerman{\,}%
\def\iso@mylongsepnodaygerman{}%
\def\iso@myshortsepnodaygerman{}%
%    \end{macrocode}
% \begin{macro}{\daymonthsepgerman}
% Change space between day and month in numeric date formats for the
% German language. The only parameter is the new spacing.
%    \begin{macrocode}
\DeclareRobustCommand*\daymonthsepgerman[1]{\def\iso@dmsepgerman{#1}}
%    \begin{macrocode}
% \end{macro}
% \begin{macro}{\monthyearsepgerman}
% Change space between month and year in numeric date formats for the
% German language. The first parameter is the new spacing for the long
% format and the second for the short format.
%    \begin{macrocode}
\DeclareRobustCommand*\monthyearsepgerman[2]{%
  \def\iso@mylongsepgerman{#1}%
  \def\iso@myshortsepgerman{#2}}
\DeclareRobustCommand*\monthyearsepnodaygerman[2]{%
  \def\iso@mylongsepnodaygerman{#1}%
  \def\iso@myshortsepnodaygerman{#2}}
%    \end{macrocode}
% \end{macro}
% \changes{2.02}{2000/10/03}{Changed the umlauts to normal \TeX\ commands to
% be able to use German dates without german.sty or babel.sty.}
%    \begin{macrocode}
\def\month@german{\ifcase\month\or
  Januar\or Februar\or M\"arz\or April\or Mai\or Juni\or
  Juli\or August\or September\or Oktober\or November\or Dezember\fi}
\def\month@ngerman{\month@german}
\def\month@austrian{\ifnum1=\month
  J\"anner\else \month@german\fi}
\def\month@naustrian{\month@austrian}
%    \end{macrocode}
%    \begin{macrocode}
\@namedef{iso@printmonthday@\CurrentOption}#1#2{%
  \ifthenelse{\equal{\iso@dateformat}{iso}\OR
              \equal{\iso@dateformat}{TeX}}{%
    \iso@printmonthday@int{#1}{#2}%
  }{%
    \ifthenelse{\equal{\iso@dateformat}{numeric}\OR
                \equal{\iso@dateformat}{short}}{%
      \ifiso@doprintday
        \iso@printday{#2}.\iso@dmsepgerman
      \fi
      \iso@printmonth{#1}%
    }{%
      \ifthenelse{\equal{\iso@dateformat}{orig}\OR
                  \equal{\iso@dateformat}{shortorig}}{%
        \ifiso@doprintday
          \iso@printday{#2}.\iso@monthsep\@empty
        \fi
        \begingroup
        \edef\lmonth{#1}%
        \def\month{\lmonth}\csname month@\iso@languagename\endcsname%
        \endgroup
      }{}%
    }%
  }%
}
%    \end{macrocode}
%    \begin{macrocode}
\@namedef{iso@printdate@\CurrentOption}#1#2#3{%
  \ifthenelse{\equal{\iso@dateformat}{iso}\OR
              \equal{\iso@dateformat}{TeX}}{%
    \iso@printdate@int{#1}{#2}{#3}%
  }{%
    \csname iso@printmonthday@\iso@languagename\endcsname{%
      \number#2}{\number#3}%
    \ifiso@printyear
      \ifthenelse{\equal{\iso@dateformat}{orig}\OR
                  \equal{\iso@dateformat}{shortorig}}{%
      }{%
        \ifiso@doprintday.\else/\fi
      }%
      \ifthenelse{\equal{\iso@dateformat}{numeric}}{%
        \ifiso@doprintday
          \iso@mylongsepgerman\@empty
        \else
          \iso@mylongsepnodaygerman\@empty
        \fi
        \iso@yearfour{\number#1}%
      }{%
        \ifthenelse{\equal{\iso@dateformat}{orig}}{%
          \iso@yearsep\iso@yearfour{\number#1}%
        }{%
          \ifthenelse{\equal{\iso@dateformat}{shortorig}}{%
            \iso@yearsep\iso@twodigitsign\iso@yeartwo{\number#1}%
          }{%
            \ifthenelse{\equal{\iso@dateformat}{short}}{%
              \ifiso@doprintday
                \iso@myshortsepgerman\@empty
              \else
                \iso@myshortsepnodaygerman\@empty
              \fi
              \iso@yeartwo{\number#1}%
            }{}%
          }%
        }%
      }%
    \fi
  }%
}
%    \end{macrocode}
%    \begin{macrocode}
\@namedef{iso@daterange@\CurrentOption}#1#2#3#4#5#6{%
  \ifthenelse{\equal{\iso@dateformat}{iso}\OR
              \equal{\iso@dateformat}{TeX}}{%
    \iso@daterange@int{#1}{#2}{#3}{#4}{#5}{#6}%
  }{%
%    \end{macrocode}
%    \begin{macrocode}
    \ifthenelse{\equal{\number#1}{\number#4}}{%
      \ifthenelse{\equal{\number#2}{\number#5}}{%
        \ifiso@doprintday
          \iso@printday{#3}.%
        \else
          \csname iso@printmonthday@\iso@languagename\endcsname{#2}{#3}%
        \fi
      }{%
          \csname iso@printmonthday@\iso@languagename\endcsname{#2}{#3}%
        }%
    }{%
        \csname iso@printdate@\iso@languagename\endcsname{#1}{#2}{#3}%
      }%
    \iso@rangesign
    \csname iso@printdate@\iso@languagename\endcsname{#4}{#5}{#6}%
  }%
}
%    \end{macrocode}
%    \begin{macrocode}
\expandafter\def\csname iso@rangesign@\CurrentOption\endcsname{~bis~}
%    \end{macrocode}
%    \begin{macrocode}
\ifthenelse{\equal{\CurrentOption}{german}}{%
  \def\iso@dategerman{%
    \def\today{\iso@printdate@german{\year}{\month}{\day}}}%
%    \end{macrocode}
% Define the language name that will the active language for isodate
% if none of the packages babel.sty, german.sty, and ngerman.sty is
% loaded and if this is the last language that is used for isodate.
% If one of the above packages is used this definition will be
% overridden by the command \verb|\languagename| that will always
% return the current used language.
%    \begin{macrocode}
  \def\iso@languagename{german}%
}{%
%    \end{macrocode}
%    \begin{macrocode}
\ifthenelse{\equal{\CurrentOption}{ngerman}}{%
  \def\iso@datengerman{%
    \def\today{\iso@printdate@ngerman{\year}{\month}{\day}}}%
%    \end{macrocode}
% Define the language name that will the active language for isodate
% if none of the packages babel.sty, german.sty, and ngerman.sty is
% loaded and if this is the last language that is used for isodate.
% If one of the above packages is used this definition will be
% overridden by the command \verb|\languagename| that will always
% return the current used language.
%    \begin{macrocode}
  \def\iso@languagename{ngerman}%
}{%
%    \end{macrocode}
%    \begin{macrocode}
\ifthenelse{\equal{\CurrentOption}{austrian}}{%
  \def\iso@dateaustrian{%
    \def\today{\iso@printdate@austrian{\year}{\month}{\day}}}%
%    \end{macrocode}
% Define the language name that will the active language for isodate
% if none of the packages babel.sty, german.sty, and ngerman.sty is
% loaded and if this is the last language that is used for isodate.
% If one of the above packages is used this definition will be
% overridden by the command \verb|\languagename| that will always
% return the current used language.
%    \begin{macrocode}
  \def\iso@languagename{austrian}%
}{%
%    \end{macrocode}
%    \begin{macrocode}
\ifthenelse{\equal{\CurrentOption}{naustrian}}{%
  \def\iso@datenaustrian{%
    \def\today{\iso@printdate@naustrian{\year}{\month}{\day}}}%
%    \end{macrocode}
% Define the language name that will the active language for isodate
% if none of the packages babel.sty, german.sty, and ngerman.sty is
% loaded and if this is the last language that is used for isodate.
% If one of the above packages is used this definition will be
% overridden by the command \verb|\languagename| that will always
% return the current used language.
%    \begin{macrocode}
  \def\iso@languagename{naustrian}%
}{%
}}}}
%    \end{macrocode}
% Redefine the command \verb|date|language that is used by babel.sty,
% german.sty, and ngerman.sty to switch to the original
% German date format to enable the use of different date
% formats.
% This has to be done after the preamble in order to ensure to overwrite
% the babel command.
%
% Do this only if \verb|\iso@date|language is defined.
%    \begin{macrocode}
\AtBeginDocument{%
  \ifx\undefined\iso@dategerman\else
    \def\dategerman{\iso@dategerman}%
  \fi
  \ifx\undefined\iso@datengerman\else
    \def\datengerman{\iso@datengerman}%
  \fi
  \ifx\undefined\iso@dateaustrian\else
    \def\dateaustrian{\iso@dateaustrian}%
  \fi
  \ifx\undefined\iso@datenaustrian\else
    \def\datenaustrian{\iso@datenaustrian}%
  \fi
}
%</german>
%    \end{macrocode}
%
% \subsection{Language definition file italian.idf}
% \changes{2.28}{2005/04/15}{Add Italian language by Philip Ratcliffe}%
%
% \begin{macro}{\iso@languageloaded}
% Define the command \verb|\iso@languageloaded| in order to enable
% \verb|isodate.sty| to determine if at least one language is loaded.
%    \begin{macrocode}
%<*italian>
\let\iso@languageloaded\active
\typeout{Define commands for Italian date format}
%    \end{macrocode}
% \end{macro}
%    \begin{macrocode}
\def\month@italian{\ifcase\month\or
  gennaio\or febbraio\or marzo\or aprile\or maggio\or giugno\or
  luglio\or agosto\or settembre\or ottobre\or novembre\or
  dicembre\fi}
%    \end{macrocode}
%    \begin{macrocode}
\def\iso@printmonthday@italian#1#2{%
  \ifthenelse{\equal{\iso@dateformat}{iso}\OR
              \equal{\iso@dateformat}{TeX}}{%
    \iso@printmonthday@int{#1}{#2}%
  }{%
    \ifthenelse{\equal{\iso@dateformat}{numeric}\OR
                \equal{\iso@dateformat}{short}}{%
      \ifiso@doprintday
        \iso@printday{#2}/%
      \fi
      \iso@printmonth{#1}%
    }{%
      \ifthenelse{\equal{\iso@dateformat}{orig}\OR
                  \equal{\iso@dateformat}{shortorig}}{%
        \begingroup
        \edef\lday{#2}\edef\day{\lday}%
        \edef\lmonth{#1}\def\month{\lmonth}%
        \ifiso@doprintday
          \number\day\ifnum1=\day \noexpand\textordmasculine\fi
          \iso@monthsep
        \fi
        \month@italian
        \endgroup
      }{}%
    }%
  }%
}
%    \end{macrocode}
%    \begin{macrocode}
\def\iso@printdate@italian#1#2#3{%
  \ifthenelse{\equal{\iso@dateformat}{iso}\OR
    \equal{\iso@dateformat}{TeX}}{%
    \iso@printdate@int{#1}{#2}{#3}%
  }{%
    \iso@printmonthday@italian{\number#2}{\number#3}%
    \ifiso@printyear
      \ifthenelse{\equal{\iso@dateformat}{orig}\OR
                  \equal{\iso@dateformat}{shortorig}}{%
      }{%
        /%
      }%
      \ifthenelse{\equal{\iso@dateformat}{numeric}}{%
        \iso@yearfour{\number#1}%
      }{%
        \ifthenelse{\equal{\iso@dateformat}{orig}}{%
          \iso@yearsep\iso@yearfour{\number#1}%
        }{%
          \ifthenelse{\equal{\iso@dateformat}{shortorig}}{%
            \iso@yearsep\iso@twodigitsign\iso@yeartwo{\number#1}%
          }{%
            \ifthenelse{\equal{\iso@dateformat}{short}}{%
              \iso@yeartwo{\number#1}%
            }{}%
          }%
        }%
      }%
    \fi
  }%
}
%    \end{macrocode}
%    \begin{macrocode}
\def\iso@dateitalian{%
  \def\today{\iso@printdate@italian{\year}{\month}{\day}}}%
%    \end{macrocode}
% \begin{macro}{\iso@daterange@...}
% Define date-range commands for dialects.
%    \begin{macrocode}
\expandafter\def\csname iso@daterange@\CurrentOption\endcsname{%
  \iso@daterange@italian}%
%    \end{macrocode}
% \end{macro}
%    \begin{macrocode}
\def\iso@daterange@italian#1#2#3#4#5#6{%
  \ifthenelse{\equal{\iso@dateformat}{iso}\OR
              \equal{\iso@dateformat}{TeX}}{%
    \iso@daterange@int{#1}{#2}{#3}{#4}{#5}{#6}%
  }{%
%    \end{macrocode}
%    \begin{macrocode}
    \ifthenelse{\equal{\number#1}{\number#4}}{%
      \ifthenelse{\equal{\number#2}{\number#5}}{%
        \ifiso@doprintday
          \ifthenelse{\equal{\iso@dateformat}{orig}}{%
            \begingroup
            \edef\lday{#3}\edef\day{\lday}%
            \number\day\ifnum1=\day \noexpand\textordmasculine\fi
            \endgroup
          }{%
            \iso@printday{#3}%
          }%
        \else
          \iso@printmonthday@italian{#2}{#3}%
        \fi
      }{%
        \iso@printmonthday@italian{#2}{#3}%
      }%
    }{%
      \csname iso@printdate@\iso@languagename\endcsname{#1}{#2}{#3}%
    }%
    \iso@rangesign
    \csname iso@printdate@\iso@languagename\endcsname{#4}{#5}{#6}%
  }%
}
%    \end{macrocode}
%    \begin{macrocode}
\expandafter\def\csname iso@rangesign@\CurrentOption\endcsname{~al~}
%    \end{macrocode}
% Define the language name that will the active language for isodate
% if none of the packages babel.sty, german.sty, and ngerman.sty is
% loaded and if this is the last language that is used for isodate.
% If one of the above packages is used this definition will be
% overridden by the command \verb|\languagename| that will always
% return the current used language.
%    \begin{macrocode}
\def\iso@languagename{italian}%
%    \end{macrocode}
%    \begin{macrocode}
\AtBeginDocument{%
  \ifx\undefined\iso@dateitalian\else
    \def\dateitalian{\iso@dateitalian}%
  \fi
}
%</italian>
%    \end{macrocode}
%
% \changes{2.02}{2001/04/30}{Added Norwegian language by Svend Tollak
% Munkejord}
% \subsection{Language definition file norsk.idf}
% \changes{2.26}{2005/03/10}{Force year in four digits for long formats}%
% 
% This file was provided by Svend Tollak Munkejord
% (svend.t.munkejord@energy.sintef.no).
%
% \begin{macro}{\iso@languageloaded}
% Define the command \verb|\iso@languageloaded| in order to enable
% \verb|isodate.sty| to determine if at least one language is loaded.
%    \begin{macrocode}
%<*norsk>
\let\iso@languageloaded\active
\typeout{Define commands for Norwegian date format}
%    \end{macrocode}
% \end{macro}
% \begin{macro}{\month@norsk}
% Prints the name of today's month in the long form for the original
% date format.
%    \begin{macrocode}
\def\month@norsk{\ifcase\month\or
    januar\or februar\or mars\or april\or mai\or juni\or
    juli\or august\or september\or oktober\or november\or desember\fi}
%    \end{macrocode}
% \end{macro}
% \begin{macro}{\iso@printmonthday@norsk}
% Prints the month and the day given as two arguments
% (\verb|{mm}{dd}|) in the current date format.
%    \begin{macrocode}
\def\iso@printmonthday@norsk#1#2{%
  \ifthenelse{\equal{\iso@dateformat}{iso}\OR
              \equal{\iso@dateformat}{TeX}}{%
    \iso@printmonthday@int{#1}{#2}%
  }{%
%    \end{macrocode}
% Numeric and short date format: \verb|dd/mm/|
%    \begin{macrocode}
    \ifthenelse{\equal{\iso@dateformat}{numeric}\OR
                \equal{\iso@dateformat}{short}}{%
      \ifiso@doprintday
        \iso@printday{#2}/%
      \fi
      \iso@printmonth{#1}%
    }{%
%    \end{macrocode}
% Original date format: \verb|d. mmm|
%    \begin{macrocode}
      \ifthenelse{\equal{\iso@dateformat}{orig}\OR
                  \equal{\iso@dateformat}{shortorig}}{%
        \ifiso@doprintday
          \iso@printday{#2}.\iso@monthsep
        \fi
        \begingroup
        \edef\lmonth{#1}\def\month{\lmonth}%
        \month@norsk%
        \endgroup
      }{}%
    }%
  }%
}
%    \end{macrocode}
% \end{macro}
% \begin{macro}{\iso@printdate@norsk}
% Prints the date given as three arguments (\verb|{yyyy}{mm}{dd}|) in
% the actual date format
%    \begin{macrocode}
\def\iso@printdate@norsk#1#2#3{%
%    \end{macrocode}
% ISO or \LaTeX date format: \verb|yyyy\iso@printmonthday@norsk|
%    \begin{macrocode}
  \ifthenelse{\equal{\iso@dateformat}{iso}\OR
              \equal{\iso@dateformat}{TeX}}{%
    \iso@printdate@int{#1}{#2}{#3}%
  }{%
    \iso@printmonthday@norsk{\number#2}{\number#3}%
%    \end{macrocode}
% numeric date format: \verb|\iso@printmonthday@norsk yyyy|
%    \begin{macrocode}
    \ifiso@printyear
       \ifthenelse{\equal{\iso@dateformat}{orig}\OR
                  \equal{\iso@dateformat}{shortorig}}{%
      }{%
        /%
      }%
      \ifthenelse{\equal{\iso@dateformat}{numeric}}{%
        \iso@yearfour{\number#1}%
      }{%
%    \end{macrocode}
% original date format: \verb|\iso@printmonthday@norsk~yyyy|
%    \begin{macrocode}
        \ifthenelse{\equal{\iso@dateformat}{orig}}{%
          \iso@yearsep\iso@yearfour{\number#1}%
        }{%
%    \end{macrocode}
% short original date format: \verb|\iso@printmonthday@norsk~yyyy|
%    \begin{macrocode}
          \ifthenelse{\equal{\iso@dateformat}{shortorig}}{%
            \iso@yearsep\iso@twodigitsign\iso@yeartwo{\number#1}%
          }{%
%    \end{macrocode}
% short date format: \verb|\iso@printmonthday@norsk yy|
%    \begin{macrocode}
            \ifthenelse{\equal{\iso@dateformat}{short}}{%
              \iso@yeartwo{\number#1}%
            }{}%
          }%
        }%
      }%
    \fi
  }%
}
%    \end{macrocode}
% \end{macro}
% \begin{macro}{\iso@datenorsk}
% This command redefines the \cs{today} command to print in the
% actual date format.
%    \begin{macrocode}
\def\iso@datenorsk{%
  \def\today{\iso@printdate@norsk{\year}{\month}{\day}}}%
%    \end{macrocode}
% \end{macro}
% \begin{macro}{\iso@daterange@...}
% Define date-range commands for dialects.
%    \begin{macrocode}
\expandafter\def\csname iso@daterange@\CurrentOption\endcsname{%
  \iso@daterange@norsk}%
%    \end{macrocode}
% \end{macro}
% \begin{macro}{\iso@daterange@norsk}
% This command takes six arguments
% (\verb|{yyyy1}{mm1}{dd1}{yyyy2}{mm2}{dd2}|) and prints the corrosponding
% date range in the actual date format.
%    \begin{macrocode}
\def\iso@daterange@norsk#1#2#3#4#5#6{%
%    \end{macrocode}
% ISO or \LaTeX\ date format.
%    \begin{macrocode}
  \ifthenelse{\equal{\iso@dateformat}{iso}\OR
              \equal{\iso@dateformat}{TeX}}{%
    \iso@daterange@int{#1}{#2}{#3}{#4}{#5}{#6}%
  }{%
%    \end{macrocode}
% Numeric, short, or original date format.
%
% If year and month are equal, only print the day of the start date. If
% only the year is equal, only print month and day of the start
% date. Otherwise print the whole start date.
%    \begin{macrocode}
    \ifthenelse{\equal{\number#1}{\number#4}}{%
      \ifthenelse{\equal{\number#2}{\number#5}}{%
        \ifiso@doprintday
          \ifthenelse{\equal{\iso@dateformat}{orig}\OR
                      \equal{\iso@dateformat}{shortorig}}{%
            \iso@printday{#3}.%
          }{%
            \iso@printday{#3}%
          }%
        \else
          \iso@printmonthday@norsk{#2}{#3}%
        \fi
      }{%
        \iso@printmonthday@norsk{#2}{#3}%
      }%
    }{%
      \csname iso@printdate@\iso@languagename\endcsname{#1}{#2}{#3}%
    }%
%    \end{macrocode}
% Print the end date.
%    \begin{macrocode}
    \iso@rangesign
    \csname iso@printdate@\iso@languagename\endcsname{#4}{#5}{#6}%
  }%
}
%    \end{macrocode}
% \end{macro}
% \begin{macro}{\iso@rangesign@norsk}
% Sets the word between start and end date in a date range to `~til~'.
%    \begin{macrocode}
\expandafter\def\csname iso@rangesign@\CurrentOption\endcsname{~til~}
%    \end{macrocode}
% \end{macro}
% Define the language name that will the active language for isodate
% if none of the packages babel.sty, german.sty, and ngerman.sty is
% loaded and if this is the last language that is used for isodate.
% If one of the above packages is used this definition will be
% overridden by the command \verb|\languagename| that will always
% return the current used language.
%    \begin{macrocode}
\def\iso@languagename{norsk}%
%    \end{macrocode}
% Redefine the command \verb|\datenorsk| that is used by babel to
% switch to the original Norsk date format to enable the use of
% different date formats.
% This has to be done after the preamble in order to ensure to overwrite
% the babel command.
%    \begin{macrocode}
\AtBeginDocument{%
  \ifx\undefined\iso@datenorsk\else
    \def\datenorsk{\iso@datenorsk}%
  \fi
}
%</norsk>
%    \end{macrocode}
%
% \changes{2.07}{2003/07/29}{Add Swedish language by Christian
% Schlauer}
% \subsection{Language definition file swedish.idf}
% \changes{2.26}{2005/03/10}{Force year in four digits for long formats}%
% 
% This file was provided by Christian Schlauer
% (christian.schlauer@web.de).
%
% \begin{macro}{\iso@languageloaded}
% Define the command \verb|\iso@languageloaded| in order to enable
% \verb|isodate.sty| to determine if at least one language is loaded.
%    \begin{macrocode}
%<*swedish>
\let\iso@languageloaded\active
\typeout{Define commands for Swedish date format}
%    \end{macrocode}
% \end{macro}
% \begin{macro}{\month@swedish}
% Prints the name of today's month in the long form for the original
% date format.
%    \begin{macrocode}
\def\month@swedish{\ifcase\month\or
    januari\or februari\or mars\or april\or maj\or juni\or
    juli\or augusti\or september\or oktober\or november\or december\fi}
%    \end{macrocode}
% \end{macro}
% \begin{macro}{\iso@printmonthday@swedish}
% Prints the month and the day given as two arguments
% (\verb|{mm}{dd}|) in the current date format.
%    \begin{macrocode}
\def\iso@printmonthday@swedish#1#2{%
  \ifthenelse{\equal{\iso@dateformat}{iso}\OR
              \equal{\iso@dateformat}{TeX}}{%
    \iso@printmonthday@int{#1}{#2}%
  }{%
%    \end{macrocode}
% Numeric and short date format: \verb|dd/mm/|
%    \begin{macrocode}
    \ifthenelse{\equal{\iso@dateformat}{numeric}\OR
                \equal{\iso@dateformat}{short}}{%
      \ifiso@doprintday
        \iso@printday{#2}/%
      \fi
      \iso@printmonth{#1}%
    }{%
%    \end{macrocode}
% Original date format: \verb|d. mmm|
%    \begin{macrocode}
      \ifthenelse{\equal{\iso@dateformat}{orig}\OR
                  \equal{\iso@dateformat}{shortorig}}{%
        \ifiso@doprintday
          \iso@printday{#2}.\iso@monthsep
        \fi
        \begingroup
        \edef\lmonth{#1}\def\month{\lmonth}%
        \month@swedish%
        \endgroup
      }{}%
    }%
  }%
}
%    \end{macrocode}
% \end{macro}
% \begin{macro}{\iso@printdate@swedish}
% Prints the date given as three arguments (\verb|{yyyy}{mm}{dd}|) in
% the actual date format
%    \begin{macrocode}
\def\iso@printdate@swedish#1#2#3{%
%    \end{macrocode}
% ISO or \LaTeX date format: \verb|yyyy\iso@printmonthday@swedish|
%    \begin{macrocode}
  \ifthenelse{\equal{\iso@dateformat}{iso}\OR
              \equal{\iso@dateformat}{TeX}}{%
    \iso@printdate@int{#1}{#2}{#3}%
  }{%
    \iso@printmonthday@swedish{\number#2}{\number#3}%
%    \end{macrocode}
% numeric date format: \verb|\iso@printmonthday@swedish yyyy|
%    \begin{macrocode}
    \ifiso@printyear
      \ifthenelse{\equal{\iso@dateformat}{orig}\OR
                  \equal{\iso@dateformat}{shortorig}}{%
      }{%
        /%
      }%
      \ifthenelse{\equal{\iso@dateformat}{numeric}}{%
        \iso@yearfour{\number#1}%
      }{%
%    \end{macrocode}
% original date format: \verb|\iso@printmonthday@swedish~yyyy|
%    \begin{macrocode}
        \ifthenelse{\equal{\iso@dateformat}{orig}}{%
          \iso@yearsep\iso@yearfour{\number#1}%
        }{%
%    \end{macrocode}
% short original date format: \verb|\iso@printmonthday@swedish~yy|
%    \begin{macrocode}
          \ifthenelse{\equal{\iso@dateformat}{shortorig}}{%
            \iso@yearsep\iso@twodigitsign\iso@yeartwo{\number#1}%
          }{%
%    \end{macrocode}
% short date format: \verb|\iso@printmonthday@swedish yy|
%    \begin{macrocode}
            \ifthenelse{\equal{\iso@dateformat}{short}}{%
              \iso@yeartwo{\number#1}%
            }{}%
          }%
        }%
      }%
    \fi
  }%
}
%    \end{macrocode}
% \end{macro}
% \begin{macro}{\iso@dateswedish}
% This command redefines the \cs{today} command to print in the
% actual date format.
%    \begin{macrocode}
\def\iso@dateswedish{%
  \def\today{\iso@printdate@swedish{\year}{\month}{\day}}}%
%    \end{macrocode}
% \end{macro}
% \begin{macro}{\iso@daterange@...}
% Define date-range commands for dialects.
%    \begin{macrocode}
\expandafter\def\csname iso@daterange@\CurrentOption\endcsname{%
  \iso@daterange@swedish}%
%    \end{macrocode}
% \end{macro}
% \begin{macro}{\iso@daterange@swedish}
% This command takes six arguments
% (\verb|{yyyy1}{mm1}{dd1}{yyyy2}{mm2}{dd2}|) and prints the corrosponding
% date range in the actual date format.
%    \begin{macrocode}
\def\iso@daterange@swedish#1#2#3#4#5#6{%
%    \end{macrocode}
% ISO or \LaTeX\ date format.
%    \begin{macrocode}
  \ifthenelse{\equal{\iso@dateformat}{iso}\OR
              \equal{\iso@dateformat}{TeX}}{%
    \iso@daterange@int{#1}{#2}{#3}{#4}{#5}{#6}%
  }{%
%    \end{macrocode}
% Numeric, short, or original date format.
%
% If year and month are equal, only print the day of the start date. If
% only the year is equal, only print month and day of the start
% date. Otherwise print the whole start date.
%    \begin{macrocode}
    \ifthenelse{\equal{\number#1}{\number#4}}{%
      \ifthenelse{\equal{\number#2}{\number#5}}{%
        \ifiso@doprintday
          \ifthenelse{\equal{\iso@dateformat}{orig}\OR
                      \equal{\iso@dateformat}{shortorig}}{%
            \iso@printday{#3}.%
          }{%
            \iso@printday{#3}%
          }%
        \else
          \iso@printmonthday@swedish{#2}{#3}%
        \fi
      }{%
        \iso@printmonthday@swedish{#2}{#3}%
      }%
    }{%
      \csname iso@printdate@\iso@languagename\endcsname{#1}{#2}{#3}%
    }%
%    \end{macrocode}
% Print the end date.
%    \begin{macrocode}
    \iso@rangesign
    \csname iso@printdate@\iso@languagename\endcsname{#4}{#5}{#6}%
  }%
}
%    \end{macrocode}
% \end{macro}
% \begin{macro}{\iso@rangesign@swedish}
% Sets the word between start and end date in a date range to `~till~'.
%    \begin{macrocode}
\expandafter\def\csname iso@rangesign@\CurrentOption\endcsname{~till~}
%    \end{macrocode}
% \end{macro}
% Define the language name that will the active language for isodate
% if none of the packages babel.sty, german.sty, and ngerman.sty is
% loaded and if this is the last language that is used for isodate.
% If one of the above packages is used this definition will be
% overridden by the command \verb|\languagename| that will always
% return the current used language.
%    \begin{macrocode}
\def\iso@languagename{swedish}%
%    \end{macrocode}
% Redefine the command \verb|\dateswedish| that is used by babel to
% switch to the original Swedish date format to enable the use of
% different date formats.
% This has to be done after the preamble in order to ensure to overwrite
% the babel command.
%    \begin{macrocode}
\AtBeginDocument{%
  \ifx\undefined\iso@dateswedish\else
    \def\dateswedish{\iso@dateswedish}%
  \fi
}
%</swedish>
%    \end{macrocode}
%
% \Finale
