%	Filename: LongSample.tex
% Sample cookbook using RecipeBook class.  Demonstrates
% syntax for creating a title page, hyperlinked table of
% contents, and several recipe formats.

\documentclass{RecipeBook}

\begin{document}

\titlepage[Recipes]
	{images/French_dip.jpg}
	{images/Curried_butternut_squash_soup.jpg}
	{images/Grilled_chicken_with_capers.jpg}
	
\makecontents

\pagestyle{recipe}

\section{Beef}
\subsection{French dip sandwiches}
%====French dip sandwiches===================================================%
\author{Brad}
%----Title and info----------------------------------------------------------%
\title{French dip}
\subtitle{\hspace*{0.75in}sandwiches}
\leftbgpic{images/French_dip.jpg}

\begin{shrink}

\begin{info}
	Makes: & 6 - 8 sandwiches \\
	Roast prep: & 15 minutes \\
	Roast time: & 1 - 4 hours \\
	Remaining: & 45 minutes
\end{info}

%----Ingredients-------------------------------------------------------------%
\recipesection{Ingredients}
\begin{ingredients}
%	\group{For the roast}
	\item{4 lb. beef round or sirloin roast}
	\item{2 cups chopped vegetables (onion, carrot, celery, garlic)}
	\item{Several sprigs fresh thyme}
	\item{\nicefrac{1}{4} cup red wine}
	\item{Salt and pepper}
	\group{For the au jus}
	\item{1 tsp butter or coconut oil}
	\item{1 tsp flour}
	\item{2 cups beef broth}
	\group{For the sandwiches}
	\item{1 large onion, sliced}
	\item{\nicefrac{1}{2} block provolone cheese (4 oz.), thinly sliced}
	\item{Steak rolls, halved, for serving}
\end{ingredients}

%----Instructions------------------------------------------------------------%
\recipesection{Instructions}
\hlgroup{ For the roast }
\step{1.}{
	Preheat the oven to between 200\degree\ and 350\degree.  
	Pat the roast dry and season with salt and pepper.  Heat oil in a medium
	cast-iron skillet or Dutch oven over medium heat.
}
\step{2.}{
	Sear the roast over medium-high heat, several minutes per 
	side, to brown. Set aside and add the root vegetables and thyme, 
	stirring for several minutes to sweat.  Add the red wine and place the
	roast on top.
}
\step{3.}{
	Transfer to the preheated oven and cook for 1 hour at 300\degree,
	1 \nicefrac{1}{2} hours at 250\degree, 2 hours at 225\degree, or
	4 hours at 180\degree.  Remove from the oven when the internal temperature 
	reaches about 135\degree\ for medium-rare, and set the roast aside to rest.
}
\hlgroup{For the au jus}
\step{4.}{
	Add the beef broth to the Dutch oven and bring the contents 
	to a boil, using a wooden spoon to deglaze the bottom. Reduce heat and 
	simmer to fully soften the vegetables.
}
\step{5.}{
	Strain the liquid, pressing on the solids.  Make a light roux
	from 1 tsp oil and flour and slowly stir in the reserved liquid.
}
\hlgroup{For the sandwiches}
\step{6.}{
	Saut\'{e} the onion with some oil over medium-high heat, stirring
	occasionally, to soften.
}
\step{7.}{
	Spread or spray oil on the insides of the rolls and toast under the
	broiler set to low (don't preheat) for 3 - 4 minutes. }
\step{8.}{
	Slice the roast thinly.  Place the roast beef and cheese on 
	the bottom-halves of the buns, and place under the broiler briefly to 
	melt the cheese.  Top with the onions and top-halves of the buns and serve
	with the au jus, for dipping.
}

\recipesection{Goes well with}
roasted vegetables or sweet potato fries.

\recipesection{Notes}

\end{shrink}

\section{Pork}
\subsection{Italian calzones}
%====Italian calzones========================================================%
\author{Brad, dough recipe from Chef Rider @ allrecipes.com}
%----Title and info----------------------------------------------------------%
\title{Italian calzones}

\begin{info}
	Makes: & 4 - 6 servings \\
		& (8 calzones) \\
	Prep time: & 1 hour \\
	Cook time: & 25 minutes \\
\end{info}

%----Ingredients-------------------------------------------------------------%
\recipesection{Ingredients}
\begin{ingredients}
	\group{For the dough}
	\item{1 (\nicefrac{1}{4} oz.) package active dry yeast}
	\item{1 tsp sugar}
	\item{1 cup warm water}
	\item{2 \nicefrac{1}{2} cups bread or all-purpose flour}
	\item{1 tsp salt}
	\item{1 Tbs dried oregano}
	\group{For the filling}
	\item{1 lb ground sausage}
	\item{\nicefrac{1}{2} onion, diced}
	\item{\nicefrac{1}{2} bell pepper, seeded and diced}
	\item{1 tsp each dried basil, dried thyme, and red pepper flakes}
	\item{\nicefrac{1}{2} block Mozzarella cheese $\quad\quad$ (4 oz.), diced}
	\item{\nicefrac{1}{2} cup sliced olives}
	\item{\nicefrac{3}{4} cup vodka or marinara sauce}
\end{ingredients}

%----Instructions------------------------------------------------------------%
\recipesection{Instructions}

\step{1.}{
	Combine the sugar, yeast, and warm water in a large bowl
	and rest 10 minutes to bloom. Add the other ingredients for the dough
	and mix well.
}
\step{2.}{
	Cook the onions, peppers, sausage, and seasoning together
	in a medium skillet, breaking the sausage apart as it browns until 
	cooked through.
}
\step{3.}{
	Drain the grease and remove from heat. Preheat the oven to
	400\degree.
}
\step{4.}{
	Divide the dough into eighths and roll into thin disks on 
	a lightly floured surface, keeping extra flour handy to prevent 
	sticking. 
}
\step{5.}{
	Spread each disk with marinara sauce and divide the sausage
	mixture among them.  Fold the dough around the sausage mixture, pressing
	flat, and fold or pinch the edges closed. 
}
\step{6.}{
	Transfer the calzones to prepared baking pans and use your 
	fingers to spread a light coat of olive oil on the exposed dough. 
	Cook in the preheated oven for 20 - 25 minutes, or until the
	crust is lightly browned. 
}

\pic{images/Italian_calzones.jpg} 

\section{Chicken}
\subsection{Grilled chicken with capers}
%----Title and info----------------------------------------------------------%
\author{Brad, original by Sandra Lee @ www.foodnetwork.com}
\title{Grilled chicken}
\subtitle{\hspace*{0.4in}with capers}

\begin{info}
	Makes: & 2 servings \\
	Time: & 45 minutes \\
\end{info}

\recipesection{For the chicken}

\begin{ingredients*}
	\item{2 chicken breasts, butterflied}
	\item{Salt and pepper}
\end{ingredients*}

\step{1.}{
	Rub the chicken with salt and pepper. Add oil to a skillet and place over
	medium-high heat.  Add the chicken and cook for 3 - 4 minutes, or until the 
	edges have turned white, and flip.  Cook until the chicken firms to the touch
	and is cooked through.
}

\recipesection{For the pasta}
\begin{ingredients*}
	\item{2 servings* fettuccini or linguini pasta}
\end{ingredients*}

\step{2.}{
	Bring a pot of water to boil and add the pasta.  Boil for about 10 
	minutes or until soft.  Drain and rinse to halt cooking, and stir in
	some olive oil.
}

\pagebreak
\recipesection{For the caper sauce}
\begin{ingredients*}
	\item{1 shallot, diced}
	\item{2 cloves garlic, minced}
	\item{2 Tbs capers}
	\item{1 cup white wine or chicken stock}
	\item{1 Tbs Dijon mustard}
	\item{1 Tbs dried parsley}
	\item{\nicefrac{1}{4} cup plain Greek yogurt}
\end{ingredients*}

\step{3.}{ 
	Saut\'{e} the shallot and garlic with a little oil
	to soften. Add the capers and white wine and simmer, uncovered, to 
	reduce by about half.
}
\step{4.}{ 
	Mix in the mustard and parsley, and season with salt and
	pepper to taste.  Remove from heat and stir in the Greek yogurt.
}

\recipesection{Notes}

*One serving of pasta is about the diameter of a nickel.

\pic[t]{images/Grilled_chicken_with_capers.jpg} 

\section{Sides and Desserts}

\subsection{Curried butternut squash soup}
\subsection{Pumpkin pie soup}
\author{Brad}
% Background images for side-by-side recipes
\rightbgpic{images/Curried_butternut_squash_soup.jpg}
\leftbgpic{images/Pumpkin_pie_soup.jpg}

%----Title and info----------------------------------------------------------%
\title{Pumpkin pie soup}

\begin{info}
	Makes: & 6 - 8 servings \\
	Time: & 1 hour 15 minutes \\
\end{info}

%----Ingredients-------------------------------------------------------------%
\recipesection{Ingredients}
\begin{ingredients}
	\item{2 carnival squash}
	\item{2 cups coconut milk}
	\item{2 cups chicken stock}
	\item{2 shallots, diced}
	\item{2 Tbs maple syrup or honey}
	\item{Pinch salt, pepper, cinnamon, nutmeg, and ground cloves}
\end{ingredients}

%----Instructions------------------------------------------------------------%
\recipesection{Instructions}

\step{1.}{
	Preheat the oven to 300\degree.  Cut off the top of the squash
	with the stem and then cut in half.  Scrape the insides with a spoon and
	coat the cut-sides with olive oil. Place flesh-down on a baking pan and 
	roast for 45 minutes, or until soft.
}
\step{2.}{
	Cook the shallot with some oil in a large saucepan until
	translucent and reduce the heat.  Add the coconut milk, chicken stock, 
	maple syrup or honey, and spices.
}
\step{3.}{
	Scoop out the squash flesh and add to the saucepan. Use an 
	immersion blender to puree the soup.  Heat through and serve with a 
	sprinkle of nutmeg.
}

\pagebreak

%----Title and info----------------------------------------------------------%
\title{Curried butternut squash soup}

\begin{info}
	Makes: & 4 - 6 servings \\
	Time: & 45 minutes \\
\end{info}

%----Ingredients-------------------------------------------------------------%
\recipesection{Ingredients}
\begin{ingredients}
	\item{1 butternut squash, peeled, seeded, and chopped}
	\item{1 apple, cored and chopped}
	\item{1 onion, chopped}
	\item{2 cloves garlic, chopped}
	\item{2 cups coconut milk}
	\item{1 Tbs curry powder}
	\item{1 tsp mild chili powder}
	\item{\nicefrac{1}{2} tsp ground black pepper}
\end{ingredients}

%----Instructions------------------------------------------------------------%
\recipesection{Instructions}

\step{1.}{
	Place a medium saucepan or Dutch oven with some oil over medium
	heat.  Add the onions, garlic, apple, and spices and cook to soften, 
	about 5 minutes.
}
\step{2.}{
	Add the squash and coconut milk and bring to a boil.  Reduce the
	heat, cover, and simmer for 20 - 30 minutes to soften the squash.
}
\step{3.}{
	Puree the soup with an immersion blender.  Serve with a splash
	of heavy cream for garnish.
}

\recipesection{Notes}

Divide the leftover soup into containers and freeze for up to 3 months.

\clearpage

\end{document}

