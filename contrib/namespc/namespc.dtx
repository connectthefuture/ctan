% \iffalse meta-comment
%
% Copyright (C) 2003 by Alexander Dreyer <adreyer@web.de>
% --------------------------------------------------
% 
% This file may be distributed and/or modified under the
% conditions of the LaTeX Project Public License, either version 1.2
% of this license or (at your option) any later version.
% The latest version of this license is in:
%
%    http://www.latex-project.org/lppl.txt
%
% and version 1.2 or later is part of all distributions of LaTeX 
% version 1999/12/01 or later.
%
% \fi
%
% \iffalse
%<package>\NeedsTeXFormat{LaTeX2e}[1998/12/01]
%<package>\ProvidesPackage{namespc}
%<package>   [2003/10/03 namespace environment] 
%
%<*batchfile>
\begingroup
\input docstrip.tex
\keepsilent

\usedir{tex/latex/namespc}

\preamble

\iffalse meta-comment

This is a generated file.

Copyright (C) 2003 by  Alexander Dreyer <adreyer@web.de>

This file may be distributed and/or modified under the conditions of
the LaTeX Project Public License, either version 1.2 of this license
or (at your option) any later version.  The latest version of this
license is in:

   http://www.latex-project.org/lppl.txt

and version 1.2 or later is part of all distributions of LaTeX version
1999/12/01 or later.

\fi

\endpreamble
\askforoverwritefalse
\generate{\usepreamble\empty
          \usepostamble\empty
          \file{namespc.ind}{\from{namespc.dtx}{indexfile}}}
\generate{\file{namespc.sty}{\from{namespc.dtx}{package}}}

\obeyspaces
\Msg{*************************************************************}
\Msg{*                                                           *}
\Msg{* To finish the installation you have to move the following *}
\Msg{* file into a directory searched by TeX:                    *}
\Msg{*                                                           *}
\Msg{*     namespc.sty                                           *}
\Msg{*                                                           *}
\Msg{* To produce the documentation run the file namespc.dtx     *}
\Msg{* through LaTeX.                                            *}
\Msg{*                                                           *}
\Msg{* Happy TeXing!                                             *}
\Msg{*                                                           *}
\Msg{*************************************************************}
\endgroup
%</batchfile>
%<*driver>
\documentclass{ltxdoc}
\usepackage{namespc}
\usepackage{a4}
\WideMargins
\EnableCrossrefs         
\CodelineIndex
\RecordChanges
\begin{document}
  \DocInput{namespc.dtx}
\end{document}
%</driver>
%
%<*package>
% \fi
%
% \CheckSum{46}
%
% \CharacterTable
%  {Upper-case    \A\B\C\D\E\F\G\H\I\J\K\L\M\N\O\P\Q\R\S\T\U\V\W\X\Y\Z
%   Lower-case    \a\b\c\d\e\f\g\h\i\j\k\l\m\n\o\p\q\r\s\t\u\v\w\x\y\z
%   Digits        \0\1\2\3\4\5\6\7\8\9
%   Exclamation   \!     Double quote  \"     Hash (number) \#
%   Dollar        \$     Percent       \%     Ampersand     \&
%   Acute accent  \'     Left paren    \(     Right paren   \)
%   Asterisk      \*     Plus          \+     Comma         \,
%   Minus         \-     Point         \.     Solidus       \/
%   Colon         \:     Semicolon     \;     Less than     \<
%   Equals        \=     Greater than  \>     Question mark \?
%   Commercial at \@     Left bracket  \[     Backslash     \\
%   Right bracket \]     Circumflex    \^     Underscore    \_
%   Grave accent  \`     Left brace    \{     Vertical bar  \|
%   Right brace   \}     Tilde         \~}
%
% \changes{v1.0}{2003/10/03}{Initial version}
%
% \GetFileInfo{namespc.sty}
%
% \DoNotIndex{\newcommand,\newenvironment,\def,\csname,\endcsname,
% \endcsname,\expandafter,\long,\relax,\:}
% 
%
% \title{The \textsf{namespc} package\thanks{This document
%   corresponds to \textsf{namespc}~\fileversion, dated \filedate.}}
% \author{Alexander Dreyer \\ \texttt{adreyer@web.de}}
%
% \maketitle
%
% \begin{abstract}
%
% The \textsf{namespc} package adds rudimentary \emph{c++}-like namespace
% functionality to \LaTeX. It may be used to declare local \LaTeX{} commands,
% which can be made accessible in a later contexts without defining them 
% globally.
%
% \end{abstract}
%
% \section{Introduction}
% This package can be used to declare local \LaTeX{} commands,
% which may be reused in a later context without polluting the global
% namespace, as an equivalent to \emph{c++}-like namespaces.
%
% \section{Usage}
%
% \DescribeMacro{\namespace}
% This command is used to instantiate and reuse a certain namespace. 
% The |\namespace| command has the following syntax:
%
% |\namespace{|\emph{spcname}|}{|\emph{before}|}{|\emph{body}|}|
%\\
% The first parameter corresponds to the name of the namespace 
% to be started or used.
% The \emph{before} arguments of the current call of |\namespace| are appended 
% to the preamble of namespace \emph{spcname}, which is processed
% before the text in \emph{body} gets processed.
%
% \DescribeEnv{\emph{spcname}}
% With the first call of
% |\namespace{|\emph{spcname}|}{|\ldots|}{|\emph{\ldots}|}| the
% corresponding environment \emph{spcname} is defined which processes 
% all \emph{before} arguments within its preamble.
%
% \DescribeMacro{\namespace*}
% Essentially the same as the |\namespace| command but without generation 
% of the corresponding environment.
%
% \DescribeMacro{\usingnamespace}
% By using |\usingnamespace{|\emph{spcname}|}| one can force \LaTeX{} to 
% process the namespace preamble of \emph{spcname}.
%
% \StopEventually{\PrintChanges\PrintIndex}
%
% \section{Implementation}
%
% We first make |@| to a character in order to use and define internal 
% commands:
%
%    \begin{macrocode}
\makeatletter
%    \end{macrocode}
%
% \begin{macro}{\@defnamespace@ifundef}
%
% First we define an internal macro which will be used to instatiate the 
% (internal) namespace preamble macro |\@namespace@|\emph{spcname} 
%
%    \begin{macrocode}
\def\@defnamespace@ifundef#1{
  \@ifundefined{#1}{
    \long\expandafter\def\csname#1\endcsname{}
  }{}
}
%    \end{macrocode}
% \end{macro}
%
% \begin{macro}{\@n@mesp@cestar}
% Now we are able to define the internal version of the starred |\namespace|
% command.
%
%    \begin{macrocode}
\newcommand{\@n@mesp@cestar}[3]{
  \@defnamespace@ifundef{@namespace@#1}
  \expandafter\g@addto@macro\csname @namespace@#1\endcsname{#2}%
  {
    \expandafter\relax\csname @namespace@#1\endcsname
    #3
  }
}
%    \end{macrocode}
%    \end{macro}
%
% \begin{macro}{\usingnamespace}
% The user command |\usingnamespace| just forces initiatiation and processing 
% of the namespace preamble.
%
%    \begin{macrocode}
\newcommand{\usingnamespace}[1]{
  \@defnamespace@ifundef{@namespace@#1}
  \csname @namespace@#1\endcsname
}
%    \end{macrocode}
%    \end{macro}
%
% \begin{macro}{\@provide@namespace@env}
% This is an internal command which provides the corresponding environment.
%
%    \begin{macrocode}
\def\@provide@namespace@env#1{
  \@ifundefined{#1} {
   \newenvironment{#1}{\usingnamespace{#1}}{}  
  }{}
}
%    \end{macrocode}
%    \end{macro}
%
% \begin{macro}{\::}
% The double colon notation may be used to access commands defined within a
% namespace preamble once. Note that the command's arguments have to be
% surrounded by colons:
%    \begin{macrocode}
\def\::#1::#2:#3:{%
  {\csname @namespace@#1\endcsname
   \csname#2\endcsname#3}
}
%    \end{macrocode}
%    \end{macro}
%
% \begin{macro}{\@n@mesp@ce}
% Next we just add the definition of the respective environment to |\namespace*|
% in order to obtain the internal version of |\namespace|:
%
%    \begin{macrocode}
\newcommand{\@n@mesp@ce}[3]{
  \@provide@namespace@env{#1}
  \@n@mesp@cestar{#1}{#2}{#3}
}
%    \end{macrocode}
%    \end{macro}
%
% \begin{macro}{\namespace}
% Finally we use the |\@ifstar| macro to define the user commands
% |\namespace| and |\namespace*|:
%
%    \begin{macrocode}
\newcommand*{\namespace}{\@ifstar\@n@mesp@cestar\@n@mesp@ce}
%    \end{macrocode}
%    \end{macro}
% |@| is made special character again:
%    \begin{macrocode}
\makeatother
%    \end{macrocode}
%
% \section{Example}
%
% This is example code for the use of the \textsf{namespc}~package:
%
%\begin{verbatim}
%\namespace{spcname}
%{\newcommand{\spcwidecmd}{\emph{command}\xspace}}{
% Using \spcwidecmd for the first time.
%} % end of namespace
%
% % not defined here: \spcwidecmd
%
%\namespace{spcname}{}{
%  \spcwidecmd may be used again later.
%} % end of namespace
%
%\begin{spcname}
%  We can also use \spcwidecmd within the corresponding environment\ldots
%\end{spcname}
%
%  Alternatively we may use the double colon notation: \::spcname::spcwidecmd::
%
%\usingnamespace{spcname}
%\ldots{} or globally: \spcwidecmd.
%
%\end{verbatim}
%
% And here's the corresponding \LaTeX{} output:
% \\
%\namespace{spcname}
%{\newcommand{\spcwidecmd}{\emph{command}\ }}{
%
% Using \spcwidecmd for the first time.
%} 
%
%
%\namespace{spcname}{}{
%  \spcwidecmd may be used again later.
%} 
%
%\begin{spcname}
%  We can also use \spcwidecmd within the corresponding environment\ldots
%\end{spcname}
%
%  Alternatively we may use the double colon notation: \::spcname::spcwidecmd::
%
%\usingnamespace{spcname}
%\ldots{} or globally: \spcwidecmd
%
% \Finale
% 
% \endinput
%</package>
%
%<*indexfile>

 \begin{theindex} 
 \makeatletter\scan@allowedfalse
{\bfseries\hfil Symbols\hfil}\nopagebreak

  \item \verb*+\::+\pfill \main{24}
  \item \verb*+\@defnamespace@ifundef+\pfill \main{2}, 8, 16
  \item \verb*+\@ifstar+\pfill 32
  \item \verb*+\@ifundefined+\pfill 3, 20
  \item \verb*+\@n@mesp@ce+\pfill \main{28}, 32
  \item \verb*+\@n@mesp@cestar+\pfill \main{7}, 30, 32
  \item \verb*+\@provide@namespace@env+\pfill \main{19}, 29
  \item {\ttfamily \emph  {spcname}} (environment)\pfill \usage{1}

  \indexspace
{\bfseries\hfil E\hfil}\nopagebreak

  \item environments:\efill 
 \subitem {\ttfamily \emph  {spcname}}\pfill \usage{1}

  \indexspace
{\bfseries\hfil G\hfil}\nopagebreak

  \item \verb*+\g@addto@macro+\pfill 9

  \indexspace
{\bfseries\hfil M\hfil}\nopagebreak

  \item \verb*+\makeatletter+\pfill 1
  \item \verb*+\makeatother+\pfill 33

  \indexspace
{\bfseries\hfil N\hfil}\nopagebreak

  \item \verb*+\namespace+\pfill \usage{1}, \main{32}
  \item \verb*+\namespace*+\pfill \usage{1}

  \indexspace
{\bfseries\hfil U\hfil}\nopagebreak

  \item \verb*+\usingnamespace+\pfill \usage{1}, \main{15}, 21

 \end{theindex}


%</indexfile>
