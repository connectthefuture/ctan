%
% \CheckSum{27}
%
% \iffalse
% File: wnri.dtx
% Copyright (C) 1998, 2011 by Anshuman Pandey.
% Released under GPLv2+.
%
%<package>\NeedsTeXFormat{LaTeX2e}
%<package>\ProvidesPackage{wnri}
%<wnr>\ProvidesFile{ot1wnr.fd}
%<wnss>\ProvidesFile{ot1wnss.fd}
%<wntt>\ProvidesFile{ot1wntt.fd}
%<-driver> [2011/05/03 v1.0b
%<package>  WNRI style package]
%<wnr>      WNRI roman font definitions]
%<wnss>     WNRI san serif font definitions]
%<wntt>     WNRI typewriter font definitions]
%
%<*driver>
\documentclass{ltxdoc}
\usepackage{mflogo}
\providecommand\dst{\expandafter{\normalfont\scshape docstrip}}
\title{Washington Romanized Indic for \LaTeXe}
\author{Anshuman Pandey}
\date{19 February 1998}
\begin{document}
\maketitle
 \DocInput{wnri.dtx}
\end{document}
%</driver>
%
% \fi
% 
% \changes{v1.0b}{2011/05/03}{clarify license, downcase filenames}
% \changes{v1.0a}{1998/02/19}{Updated for use with \LaTeXe{}}
% \changes{v1.0}{1993/06/19}{Original fonts released}
%
%
% \section{Introduction}
%
% The Washington Romanized (WNRI) Indic package enables texts encoded 
% in the 8-bit Classical Sanskrit/Classical Sanskrit eXtended (CS/CSX) 
% encoding to be typeset in \TeX{} without modification of the input 
% scheme.
% 
% This package defines the font shape groups for the WNRI fonts and 
% adapts them for use with \LaTeXe{}. These fonts were designed by 
% Thomas Ridgeway in 1993 at the Humanities Academic Computing Center 
% (HACC), University of Washington, Seattle, WA. I took the liberty on 
% behalf of the successor to HACC, the Center for Advanced Research 
% Technology in the Arts and Humanities (CARTAH), to update the package.
%
% The Washington Romanized Indic family of fonts is based on the 
% Computer Modern Roman fonts. The fonts retain the CMR encoding in 
% positions 0 to 127. The `upper ASCII' (positions 128-255) contain 
% characters of the CS/CSX encoding for transliterated Indic languages. 
% CS/CSX is a system used by scholars of Indology to facilitate the 
% exchange of data via a stable medium. This convention is well on its 
% way to becoming a standard
%
% Although WNRI is based on the CS/CSX character set, these fonts were 
% developed to contain other characters of which all are not recognized 
% in the CS/CSX standard. Some of these are drawn from the IBM-PC 
% character set, other transliteration systems, and other languages which 
% might be encountered in an Indic context, and which, as Ridgeway 
% remarked, may be useful to someone working in ``east-of-Suez contexts.''
%
% However, as the International Standards Organization Working Group for 
% the Transliteration of Indic Scripts (ISO/TC46/SC2/WG12) is currently 
% developing a standard transliteration (which will seems like a further 
% extension of CS/CSX), most of the anomolous and unused characters in the
% inventory of WNRI will be replaced by attested and accepted `standardized' 
% counterparts. Therefore, please be advised that the current WNRI font
% encoding will change as a result of this standardization. The changes, 
% however, will definitely reflect current practice.
%
% Ridgeway originally made these fonts available on a `need-to-know' 
% basis; that those interested may obtain and use them to their needs. 
% However, it has been more than five years since these fonts appear to 
% be last touched. I feel that these fonts deserve a larger audience and 
% so have updated them for use with \LaTeXe{}. In keeping with Ridgeway's 
% original wishes for the font, you are welcome to circulate the fonts and 
% information about them to other individuals you feel might be able 
% to benefit from or contribute to the enterprise with their suggestions.
%
% Some of these have had little or no real world testing, so evaluate 
% before committing to their use, particularly any san serif and 
% typewriter faces. Also, the WNRI characters still do not have kerning 
% values applied to them. Additionally, as the WNRI fonts do not place
% the correct information about their heights and depths in the metric
% files, \TeX{}'s native accent operations will not work on these fonts.
%
% I don't know whether Thomas Ridgeway is still working on these fonts 
% or not. The {\sc WNRI} fonts were originally stored on the infamous, 
% but sadly, now defunct, Blackbox\footnote{{\tt blackbox.hacc.washington.edu}} 
% server.
%
% The original release of WNRI contained two other fonts called
% Washington Gerald Barnett Old English and Washington Puget Sound
% Salish. These have been removed from the package as support for
% them was non-existent. Any previous releases of WNRI are obsolete
% as of this release. Numerous files have been removed and the 
% structure of the fonts have been slightly rearranged. The
% Postscript and TrueType versions of WNRI are also obsolete as of
% this release due to the modification of certain glyphs.
%
% \section{The Fonts}
%
% \begin{center}
% \begin{tabular}{ll}
% \multicolumn{2}{c}{\it Washington Romanized Indic} \\
% \hline
% {\tt wnrib8.mf}   & bold 8pt \\
% {\tt wnrib10.mf}  & bold 10pt \\
% {\tt wnribi10.mf} & bold italic 10pt \\
% {\tt wnrii8.mf}   & italic 8pt \\
% {\tt wnrii10.mf}  & italic 10pt \\
% {\tt wnrir8.mf}   & roman 8pt \\
% {\tt wnrir10.mf}  & roman 10pt \\
% {\tt wnris8.mf}   & sans serif 8pt \\
% {\tt wnris10.mf}  & sans serif 10pt \\
% {\tt wnrit8.mf}   & typewriter 8pt \\
% {\tt wnrit10.mf}  & typewriter 10pt \\
% \hline
% \end{tabular}
% \end{center}
%
% \section{Modifications and Updates}
% The \MF{} files have been modified to account for minor changes
% in centering of accents and distance of accents from base character.
%
% The next update will be a revision of the characters in the font.
% The unused and obsolete glyphs will be replaced by commonly
% used characters which are not represented in {\sc WNRI}. Such
% characters are r-underring and l-underring.
%
% \section{Implementation}
% This update package consists simply of a style package which redefines 
% the |\rmfamily|, |\sffamily|, and |\ttfamily| fonts, and provides two 
% font definition files which setup the {\sc WNRI} Roman, San Serif, and 
% Typewriter fonts.
%
% To specify {\sc WNRI} as the primary font invoke \texttt{wnri} through
% the |\usepackage| command.
%
% \subsection{Style Code}
%
%    The style file specifies |OT1| as the default encoding and also
%    changes the substitution defaults for this encoding. If
%    |\familydefault| is not changed directly, then the change to
%    |\rmdefault| will automatically change the main font too.
%
%    Default for |\rmfamily| will be Washington Roman Indic Regular 
%    and for |\ttfamily| Washington Roman Indic Typewriter. Also,
%    assume that for \LaTeX{} the standard magnifications are 
%    available. 
%
%    \begin{macrocode}
%<*package>
\renewcommand{\encodingdefault}{OT1}
\DeclareFontSubstitution{OT1}{wnr}{m}{n}
\renewcommand{\rmdefault}{wnr}
\renewcommand{\sfdefault}{wnss}
\renewcommand{\ttdefault}{wntt}
%</package>
%    \end{macrocode}
%
%
%  \subsection{The Font-Definition Files}
%
%    The Washington Roman Indic family exists in medium, bold, italic,
%    sans serif, and typewriter series. All of the other shapes will be 
%    given substitution shapes.
%
%    \begin{macrocode}
%<*wnr>
\DeclareFontFamily{OT1}{wnr}{}
\DeclareFontShape{OT1}{wnr}{m}{n}{
   <5> <6> <7> wnrir8
   <8> <9> <10> <10.95> <12>
   <14.4> <17.28> <20.74> <24.88> wnrir10 }{}
\DeclareFontShape{OT1}{wnr}{bx}{n}{
   <5> <6> <7> wnrib8
   <8> <9> <10> <10.95> <12>
   <14.4> <17.28> <20.74> <24.88> wnrib10 }{}
\DeclareFontShape{OT1}{wnr}{bx}{it}{
   <5> <6> <7> <8> <9> <10> <10.95> <12>
   <14.4> <17.28> <20.74> <24.88> wnribi10 }{}
\DeclareFontShape{OT1}{wnr}{b}{n}{ <-> ssub * wnr/bx/n }{}
\DeclareFontShape{OT1}{wnr}{b}{it}{ <-> ssub * wnr/bx/it }{}
\DeclareFontShape{OT1}{wnr}{m}{it}{
   <5> <6> <7> wnrii8
   <8> <9> <10> <10.95> <12>
   <14.4> <17.28> <20.74> <24.88> wnrii10}{}
\DeclareFontShape{OT1}{wnr}{m}{sl}{ <-> ssub * wnr/m/it }{}
\DeclareFontShape{OT1}{wnr}{m}{sc}{ <-> ssub * wnr/m/n }{}
\DeclareFontShape{OT1}{wnr}{m}{sf}{ 
   <5> <6> <7> wnris8
   <8> <9> <10> <10.95> <12>
   <14.4> <17.28> <20.74> <24.88> wnris10 }{}
%</wnr>
%    \end{macrocode}
%
%    The Washington Roman Indic San Serif family has only the medium series.
%
%    \begin{macrocode}
%<*wnss>
\DeclareFontFamily{OT1}{wnss}{}
\DeclareFontShape{OT1}{wnss}{m}{n}{
   <5> <6> <7> <8> wnris8
   <9> <10> <10.95> <12>
   <14.4> <17.28> <20.74> <24.88> wnris10  }{}
\DeclareFontShape{OT1}{wnss}{b}{n}{ <-> ssub * wnss/m/n }{}
\DeclareFontShape{OT1}{wnss}{bx}{n}{ <-> ssub * wnss/m/n }{}
\DeclareFontShape{OT1}{wnss}{m}{sl}{ <-> ssub * wnss/m/n }{}
\DeclareFontShape{OT1}{wnss}{m}{it}{ <-> ssub * wnss/m/n }{}
%</wnss>
%    \end{macrocode}
%
%    The Washington Roman Indic Typewriter family has only the medium series.
%
%    \begin{macrocode}
%<*wntt>
\DeclareFontFamily{OT1}{wntt}{}
\DeclareFontShape{OT1}{wntt}{m}{n}{
      <5> <6> <7> <8> wnrit8
      <9> <10> <10.95> <12>
      <14.4> <17.28> <20.74> <24.88> wnrit10  }{}
\DeclareFontShape{OT1}{wntt}{m}{it}{ <-> ssub * wntt/m/n }{}
\DeclareFontShape{OT1}{wntt}{m}{sl}{ <-> ssub * wntt/m/n }{}
\DeclareFontShape{OT1}{wntt}{m}{sc}{ <-> ssub * wntt/m/n }{}
\DeclareFontShape{OT1}{wntt}{m}{ui}{ <-> ssub * wntt/m/n }{}
\DeclareFontShape{OT1}{wntt}{bx}{n}{ <-> ssub * wntt/m/n }{}
\DeclareFontShape{OT1}{wntt}{bx}{it}{ <-> ssub * wntt/m/n }{}
\DeclareFontShape{OT1}{wntt}{bx}{ui}{ <-> ssub * wntt/m/n }{}
%</wntt>
%    \end{macrocode}
%
%    \begin{macrocode}
\endinput
%    \end{macrocode}
%
% \Finale
%
%% \CharacterTable
%%  {Upper-case    \A\B\C\D\E\F\G\H\I\J\K\L\M\N\O\P\Q\R\S\T\U\V\W\X\Y\Z
%%   Lower-case    \a\b\c\d\e\f\g\h\i\j\k\l\m\n\o\p\q\r\s\t\u\v\w\x\y\z
%%   Digits        \0\1\2\3\4\5\6\7\8\9
%%   Exclamation   \!     Double quote  \"     Hash (number) \#
%%   Dollar        \$     Percent       \%     Ampersand     \&
%%   Acute accent  \'     Left paren    \(     Right paren   \)
%%   Asterisk      \*     Plus          \+     Comma         \,
%%   Minus         \-     Point         \.     Solidus       \/
%%   Colon         \:     Semicolon     \;     Less than     \<
%%   Equals        \=     Greater than  \>     Question mark \?
%%   Commercial at \@     Left bracket  \[     Backslash     \\
%%   Right bracket \]     Circumflex    \^     Underscore    \_
%%   Grave accent  \`     Left brace    \{     Vertical bar  \|
%%   Right brace   \}     Tilde         \~}
