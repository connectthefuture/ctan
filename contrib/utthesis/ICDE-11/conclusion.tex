\section{Conclusion and future work}
\label{sec:conclusion}

Current rule-based query optimizers do not provide a very intuitive and
conceptually streamlined framework to define rules and actions.  Our
experiences with the Volcano optimizer generator suggest that its model
of rules and the expression of these rules is much more complicated and
too low-level than it needs to be.  As a consequence, rule sets in
Volcano are fragile, hard to write, and debug.  Similar problems may
exist in other contemporary rule-based query optimizers.

We believe that rule-based query optimizers will be standard tools
of future database systems.  The pragmatic difficulties of using
existing rule-based optimizers led us to develop Prairie, an
extensible and structured algebraic framework for specifying rules.
Prairie is similar to existing optimizers in that it supports both
transformation rules and implementation rules.  However, Prairie
makes several improvements:
\begin{enumerate}
\item it offers a conceptually more streamlined model for rule specification;
\item rules are encapsulated, there are no ``hidden'' operators or
      ``hidden'' algorithms;
\item implementation hints (\eg enforcers) are deduced automatically;
\item and it has efficient implementations.
\end{enumerate}

We have explained how the first three points are important for
simplifying rule specifications and making rule sets less brittle for
extensibility.  A consequence is that Prairie rules are simpler and
more robust than rules of existing optimizers (\eg Volcano).  We
addressed the fourth point by building a P2V pre-processor which uses
sophisticated algorithms to compose and compact a Prairie rule set into
a Volcano rule set.  To demonstrate the scalability of our approach, we
rewrote the TI Open OODB rule set as a Prairie rule set, generated its
Volcano counterpart, and showed that the performance of the synthesized
Volcano rule set closely matches the hand-crafted Volcano rule set.

Our future work will concentrate on developing higher-level
abstractions using Prairie, including automatically generating Prairie
rule sets, and combining multiple Prairie rule sets to automatically
generate efficient optimizers.

\section*{Acknowledgments}
\label{sec:acknowledgments}

We wish to thank Texas Instruments, Inc.\ for making the Open OODB
source code available to us.  Comments by Jos\'e Blakeley, Anne Ngu,
Vivek Singhal, Thomas Woo and the anonymous referees greatly improved
the quality of the paper.
