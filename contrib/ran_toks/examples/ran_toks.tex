\documentclass{article}
\usepackage{ran_toks}
\usepackage{ifthen}

\useThisSeed{606574325} % comment out this line, and uncomment next line to get ...
%\useLastAsSeed         % a new pseudo-random sequence each time you compile.

% Other useful switches
%\ranToksOn             % the default
%\ranToksOff

\parindent0pt \parskip6pt
\def\cs#1{\texttt{\char`\\#1}}

\begin{document}
\begin{center}\bfseries
    Test file for \textsf{ran\_toks} Package\\[3pt]
        D. P. Story
\end{center}

Test of the \verb!\ranToks! command.

\ranToks{myPals}{%
    {Jim}{Richard}{Don}
    {Alex}{Tom}{J\"{u}rgen}
}
I have {\nToksFor{myPals}} pals, they are \useRanTok{1}, \useRanTok{2},
\useRanTok{3}, \useRanTok{4}, {\useRanTok{5}} and \useRanTok{6}. (Listed
in the order of best friend to least best friend.)

Test of the \verb!\bRTVToks!/\verb!\eRTVToks! pair of commands, which encloses
\texttt{rtVW} environments.

\bRTVToks{myThoughts}%
\begin{rtVW}
\begin{minipage}[t]{.67\linewidth}
Roses are red and violets are blue,
I've forgotten the rest, have you too?
\end{minipage}
\end{rtVW}
\begin{rtVW}
\begin{minipage}[t]{.67\linewidth}
I gave up saying bad things like
\verb!$#%%%^*%^&#$@#! when I was just a teenager.
\end{minipage}
\end{rtVW}
\begin{rtVW}
\begin{minipage}[t]{.67\linewidth}
I am a good guy, pass it on! The code for this last sentence is,
\begin{verbatim}
%#$% I am a good guy, pass it on! ^&*&^*
\end{verbatim}
How did that other stuff get in there?
\end{minipage}
\end{rtVW}
\eRTVToks
\begin{enumerate}
    \displayListRandomly[\item]{myThoughts}
\end{enumerate}
Use \verb!\useRTName! command when another list separates the current
position from the list you want to use. Here we want to use the list named
\texttt{myPals}, but since that definition, a new list named
\texttt{myThoughts} was declared.

\useRTName{myPals}
List of pals: \useRanTok{1}, \useRanTok{2}, \useRanTok{3},
\useRanTok{4}, \useRanTok{5}, and \useRanTok{6}.

For mixing lists. it might be easier to use the optional parameter:
{\useRanTok[myPals]{1}} and \useRanTok[myThoughts]{1}

The \cs{rtTokByNum} can retrieve an item from the list in its declared order;
eg, from the \texttt{myPals} list, the first and last are
{\rtTokByNum[myPals]{1}} and \rtTokByNum[myPals]{\nToksFor{myPals}}.

We demonstrate the command \cs{reorderRanToks} and the optional parameters of \cs{displayListRandomly}:
\begin{quote}\reorderRanToks{myPals}%
List of pals: \displayListRandomly[\ifnum\i=\last\space and \fi][\ifnum\i=\last.\else,\fi\space]{myPals}
\end{quote}
The reordering is global, so \displayListRandomly[\ifnum\i=\last\space and \fi][\ifnum\i=\last\else, \fi]{myPals}
are listed in the same order as above, and different from the original random order seen in the second paragraph
of this document.

Without the Oxford comma: My pals are
\displayListRandomly[\ifnum\i=\last and \fi][\ifnum\i=\last.\else\ifnum\i=\lessone\relax\space\else, \fi\fi]{myPals}

The following is the same logic but uses the syntax of the \textsf{ifthen} package: My pals are
\displayListRandomly[\ifthenelse{\i=\last}{and }{}][\ifthenelse{\i=\last}{.}{\ifthenelse{\i=\lessone}{\space}{, }}]{myPals}

Test the \cs{copyRanToks} command:
\copyRanToks{myPals}{myPals1}\displayListRandomly[\ifthenelse{\i=\last}{and }{}][\ifthenelse{\i=\last}{.}{\ifthenelse{\i=\lessone}{\space}{, }}]{myPals1}
This does not change the order of the \texttt{myPals} list, which is still
\displayListRandomly[\ifnum\i=\last\space and \fi][\ifnum\i=\last.\else,\fi\space]{myPals}

\end{document}
