% http://golatex.de/aufgabe-loesungs-umgebung-t18844.html
\documentclass[18pt,a4paper]{article}
\usepackage[utf8]{inputenc}
\usepackage[ngerman]{babel}

\usepackage{xsim,tcolorbox}
\usepackage{amsmath}
\xsimsetup{
  exercise/within = section ,
  exercise/the-counter = \thesection.\arabic{exercise} ,
  print-solutions/headings-template=none
}
\SetExerciseParameters{exercise}{
  exercise-template = mine ,
  solution-template = mine
}

\DeclareExerciseEnvironmentTemplate{mine}
  {%
    \tcolorbox[
      % colors:
      colback = white , colframe = black , coltitle = black ,
      % rules:
      boxrule = 0pt , toprule = 1pt , bottomrule = 1pt , arc = 0pt ,
      % spacing:
      boxsep = 0pt , left = 0pt , right = 0pt ,
      % title:
      detach title , before upper = \tcbtitle\par\noindent ,
      fonttitle = \bfseries ,
      title = \XSIMmixedcase{\GetExerciseName}~\GetExerciseProperty{counter}
    ]
  }
  {\endtcolorbox}

\DeclareExerciseTranslation{German}{exercise}{Aufgabe}
\DeclareExerciseTagging{difficulty}

\begin{document}

\section{Wellenausbreitung im Vakuum und in Materie}
\subsection{Maxwellsche Gleichungen}

\begin{align*}
  \nabla \cdot \vec{E}\left(\vec{r}, t \right) =
    \frac{\varrho\left(\vec{r}, t \right)}{\varepsilon_0}
\end{align*}

\begin{exercise}[difficulty=easy]
Something stupid
\end{exercise}
\begin{solution}
Here is a nonstupid solution for your problem
\end{solution}

\subsection{Empirischer Zugang zu Wellengleichungen}

\begin{exercise}[difficulty=hard]
Eine weitere Aufgabe
\end{exercise}
\begin{solution}
Solution Number 2
\end{solution}

\subsection*{Lösungen}
\printsolutions[section,difficulty=hard]

\section{Wellenausbreitung im Vakuum und in Materie}
\subsection{Maxwellsche Gleichungen}

\begin{align*}
  \nabla \cdot \vec{E}\left(\vec{r}, t \right) =
    \frac{\varrho\left(\vec{r}, t \right)}{\varepsilon_0}
\end{align*}

\begin{exercise}[difficulty=easy]
Something stupid
\end{exercise}
\begin{solution}
Here is a nonstupid solution for your problem
\end{solution}

\subsection{Empirischer Zugang zu Wellengleichungen}

\begin{exercise}[difficulty=hard]
Eine weitere Aufgabe
\end{exercise}
\begin{solution}
Solution Number 2
\end{solution}

\subsection*{Lösungen}
\printsolutions[section]

\end{document}
