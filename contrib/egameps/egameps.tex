%
% Changes relative to egame.sty:
%    1. No anomalous cases for \iib
%2. eginfods changed to eginfodotsep
%3. No small disks at ends of histories
%
%Allows any slope for branches, and any style.
%
\documentclass[12pt]{article}
\usepackage{mjoarti}
\usepackage{pstricks}
\usepackage{pstcol,pst-mjo,pst-3d}
\usepackage{url}
\usepackage{egameps}
\begin{document}

\renewcommand{\descriptionlabel}[1]{\hspace\labelsep
                                \normalfont #1\phantom{.}}

\begin{center}
{\bfseries Manual for {\tt egameps.sty}}\\
by\\
Martin J. Osborne\\
{\tt martin.osborne@utoronto.ca}\\
Version 1.1, June 2004\\
(based on \texttt{egame.sty}, November 1997)
\end{center}


\section{Introduction}
{\tt egameps.sty} is a \LaTeX2e\ style file for drawing extensive games.  It
is intended to have the capability of drawing any extensive game.  It is not
very fast (probably because it is not written very efficiently, but possibly
simply because it needs to make a lot of computations).  The latest version is
available at \url{http://www.economics.utoronto.ca/osborne/latex/}.

The style requires the PSTricks package (available on CTAN; documented in
\emph{The \LaTeX\ Graphics Companion}) and a means of printing a Postscript
file (e.g.\ a Postscript printer, or a non-Postscript printer and the program
Ghostscript).  Because the style uses the PSTricks macros, it is incompatible
with \verb+pdftex+.  You may produce a pdf file by first creating a dvi file
normally, then using \verb+dvips+ to create a Postscript file, and finally
using Ghostscript to create a pdf file from the Postscript file.  (If your
operating system is DOS or Windows, you can automate this process in a batch
file.  If your text editor is sufficiently configurable, you will be able to
initiate the process with a single keystroke.)  Alternatively, you use the
\verb+pdftricks+ package to finesse the limitation of \verb+pdftex+; see
Section~\ref{s:pdftricks}.
\section{Installation}
\begin{itemize}
   \item Put \texttt{egameps.sty} in a directory from which \TeX\ reads input
files.  (In MiKTeX the directory might be something like
\url{\miktex\localtexmf\tex\latex\} or a subdirectory thereof.)
   \item Let \TeX\ know that \texttt{egameps} has arrived.  (In MiKTeX,
``refresh the filename database''.)
   \item Check that you have PSTricks.  If you do not, get it from CTAN.
   \item To use the package in a document, put the lines\newline
\indent\verb+\usepackage{pstricks}+\newline
\indent\verb+\usepackage{pstcol}+\newline
\indent\verb+\usepackage{egameps}+\newline
in the preamble.  To use the macro \verb+\ctmbarc+ you need to include also
the line\newline
\indent\verb+\usepackage{pst-3d}+
\end{itemize}
\section{Using the style}
To draw a game using the style, first break the game into components, each
consisting of a node together with the branches that emanate from it, the
names with which the actions are labeled, and the players' payoffs if the
nodes at the end of the branches are terminal.  To draw each component, calls
to two macros are needed.  First \verb+\putbranch+ is called, which sets the
position of the node and the slope and length of the branch(es), then either
\verb+\ib+, \verb+\iib+, or \verb+\iiib+.  The macro \verb+\ib+ draws a single
branch, while \verb+\iib+ and \verb+\iiib+ draw two and three branches
respectively.  If the node has more than three branches, calls to a
combination of these macros are needed.

A game is begun by a call of the type\newline
\indent\verb+\begin{egame}400,500)+\newline
which starts a \verb+pspicture+, with dimensions $(400,500)$ and the
unitlength equal to the default of 0.1mm, or\newline
\indent\verb+\begin{egame}(400,500)[1mm]+\newline
which starts a \verb+pspicture+, with dimensions $(400,500)$ and sets the
unitlength at 1mm.

The next section gives examples that illustrate many of the features of the
package.  Precise descriptions of the macros are given in
Section~\ref{s:descr}.
\section{Examples}
The game in Figure~\ref{f:one} is produced by the following code.

%%%%%%%%%%%%%%
%% EXAMPLE 1 %
%%%%%%%%%%%%%%
\begin{verbatim}
\begin{figure}[htb]
\hspace*{\fill}
\begin{egame}(600,280)
%
% put the initial branch at (300,240), with (x,y) direction
% (2,1), and horizontal length 200
\putbranch(300,240)(2,1){200}
%
% give the branch two actions, label it for player 1,
% and label the actions $L$ and $R$
\iib{1}{$L$}{$R$}
%
% put a branch at (100,140), with (x,y) direction
% (1,1) and horizontal length 100
\putbranch(100,140)(1,1){100}
%
% give the branch two actions, omit a player label,
% label the actions $a$ and $b$, and assign the payoffs
% $1,0$ and $2,3$ to these actions
\iib{}{$a$}{$b$}[$1,0$][$2,3$]
%
% put a branch at (500,140), with (x,y) direction (1,1)
% and horizontal length 100
\putbranch(500,140)(1,1){100}
%
% give the branch two actions, omit a player label, 
% label the actions $c$ and $d$, and assign the payoffs
% $0,1$ and $-1,0$ to these actions
\iib{}{$c$}{$d$}[$0,1$][$-1,0$]
%
% draw an information set between the nodes at (100,140)
% and (500,140)
\infoset(100,140){400}{2}
%
\end{egame}
\hspace*{\fill}
\caption[]{An extensive game}\label{f:one}
\end{figure}
\end{verbatim}

\begin{figure}[htb]
\hspace*{\fill}
\begin{egame}(600,280)
\putbranch(300,240)(2,1){200}
\iib{1}{$L$}{$R$}
\putbranch(100,140)(1,1){100}
\iib{}{$a$}{$b$}[$1,0$][$2,3$]
\putbranch(500,140)(1,1){100}
\iib{}{$c$}{$d$}[$0,1$][$-1,0$]
\infoset(100,140){400}{2}
\end{egame}
\hspace*{\fill}
\caption[]{An extensive game}\label{f:one}
\end{figure}

Another example, illustrating more features, is produced by the following
code, and is shown in Figure~\ref{f:two}.

%%%%%%%%%%%%%%
%% EXAMPLE 2 %
%%%%%%%%%%%%%%

\begin{verbatim}
\begin{figure}[htb]
\hspace*{\fill}
%
% fill boxes containing payoffs with a solid red color
\renewcommand{\egpayboxfillstyle}{solid}
\renewcommand{\egpayboxfillcolor}{red}
% add a bit of separation around the payoffs
\egpayoffboxsep=1mm
%
% fill boxes containing action labels with a solid blue color
\renewcommand{\egalboxfillstyle}{solid}
\renewcommand{\egalboxfillcolor}{blue}
\egactionboxsep=1mm
%
% put player labels in green circles
\renewcommand{\egplbox}{c}
\renewcommand{\egplboxlinestyle}{solid}
\renewcommand{\egplboxlinecolor}{green}
\egplayerboxsep=1mm
%
\begin{egame}(500,280)
%
% put the initial branch at (300,240), with (x,y) direction
% (2,1), and horizontal length 200
\putbranch(300,240)(2,1){200}
%
% give the branch three actions, label it for player 1,
% label the actions $L$, $M$, and $R$, and make the middle
% and right actions terminal, with payoffs $1,2$ and $0,1$
\iiib{1}{$L$}{$M$}{$R$}[][$1,2$][$0,1$]
%
% put a branch at (100,140), with (x,y) direction
% (1,1) and horizontal length 100
\putbranch(100,140)(1,1){100}
%
% give the branch two actions, make the left-hand one dashed,
% label the branch for player 2, putting
% the player label to the top left of the node,
% label the actions $a$ and $b$, and assign the payoffs
% $1,0$ and $2,3$ to these actions
\iib[linestyle=dashed][]{2}[l]{$a$}{$b$}[$1,0$][$2,3$]
%
\end{egame}
\hspace*{\fill}
\caption[]{Another extensive game}\label{f:two}
\end{figure}
\end{verbatim}

\begin{figure}[htb]
\hspace*{\fill}
\renewcommand{\egpayboxfillstyle}{solid}
\renewcommand{\egpayboxfillcolor}{red}
\egpayoffboxsep=1mm
\renewcommand{\egalboxfillstyle}{solid}
\renewcommand{\egalboxfillcolor}{blue}
\egactionboxsep=1mm
\renewcommand{\egplbox}{c}
\renewcommand{\egplboxlinestyle}{solid}
\renewcommand{\egplboxlinecolor}{green}
\egplayerboxsep=1mm
\begin{egame}(500,280)
%
% put the initial branch at (300,240), with (x,y) direction
% (2,1), and horizontal length 200
\putbranch(300,240)(2,1){200}
%
% give the branch three actions, label it for player 1,
% label the actions $L$, $M$, and $R$, and make the middle
% and right actions terminal, with payoffs $1,2$ and $0,1$
\iiib{1}{$L$}{$M$}{$R$}[][$1,2$][$0,1$]
%
% put a branch at (100,140), with (x,y) direction
% (1,1) and horizontal length 100
\putbranch(100,140)(1,1){100}
%
% give the branch two actions, label it for player 2, putting
% the player label to the top left of the node,
% label the actions $a$ and $b$, and assign the payoffs
% $1,0$ and $2,3$ to these actions
\iib[linestyle=dashed][]{2}[l]{$a$}{$b$}[$1,0$][$2,3$]
%
\end{egame}
\hspace*{\fill}
\caption[]{Another extensive game}\label{f:two}
\end{figure}

Yet another example, illustrating still more features, is produced by the
following code, and is shown in Figure~\ref{f:three}.

\begin{verbatim}
\begin{figure}[htb]
\hspace*{\fill}
%
\begin{egame}(600,480)
%
% put the initial branch at (300,240), with (x,y) direction
% (1,0), and horizontal length 200
\putbranch(300,240)(1,0){200}
%
% give the branch two actions, label it for player $c$,
% and label the actions ${1\over 2}$ and ${1\over 2}$
\iib{$c$}{${1\over 2}$}{${1\over 2}$}
%
% put a branch at (100,240), with (x,y) direction
% (0,1), going right, and vertical length 100.  (Notes: If the
% branch were specified as going left, it would look the same,
% but the player label would be in the wrong place.  The
% third mandatory argument of \putbranch is the horizontal
% distance unless the branch is vertical, in which case it is
% the vertical distance.)
\putbranch(100,240)(0,1)[r]{100}
%
% give the branch two actions, label it for player 1,
% and label the actions $a$ and $b$
\iib{1}{$a$}{$b$}
%
% put a branch at (500,240), with (x,y) direction
% (0,1), going left, and vertical length 100.
\putbranch(500,240)(0,1)[l]{100}
%
% give the branch two actions, label it for player 1,
% and label the actions $b$ and $a$
\iib{1}{$b$}{$a$}
%
% put an information set at (100,340), of length 400,
% assigned to player 2
\infoset(100,340){400}{2}
%
% put an information set at (100,140), of length 400,
% assigned to player 2
\infoset(100,140){400}{2}
%
% put a branch at (100,340), with (x,y) direction
% (1,1), going up, and horizontal length 100.
\putbranch(100,340)(1,1)[u]{100}
%
% specify a positive horizontal shift for the action labels,
% which slides the ones on the right-hand branches to the
% right along the branches and the ones on the left to the
% left, improving the label positions
\egalhshift=20
%
% give the branch two actions, give it no player label,
% label the actions $L$ and $R$, and put payoffs of
% $-1,0$ and $0,-1$.
\iib{}{$L$}{$R$}[$-1,0$][$0,-1$]
%
% put a branch at (500,340), with (x,y) direction
% (1,1), going up, and horizontal length 100.
\putbranch(500,340)(1,1)[u]{100}
%
\egalhshift=20
%
% give the branch two actions, give it no player label,
% label the actions $L$ and $R$, and put payoffs of
% $1,0$ and $0,1$.
\iib{}{$L$}{$R$}[$1,0$][$0,1$]
%
% put a branch at (100,140), with (x,y) direction
% (1,1), going down, and horizontal length 100.
\putbranch(100,140)(1,1)[d]{100}
%
\egalhshift=20
%
% give the branch two actions, give it no player label,
% label the actions $L$ and $R$, and put payoffs of
% $2,0$ and $0,2$.
\iib{}{$L$}{$R$}[$1,0$][$0,1$]
%
% put a branch at (500,140), with (x,y) direction
% (1,1), going down, and horizontal length 100.
\putbranch(500,140)(1,1)[d]{100}
%
\egalhshift=20
%
% give the branch two actions, give it no player label,
% label the actions $L$ and $R$, and put payoffs of
% $3,0$ and $0,3$.
\iib{}{$L$}{$R$}[$1,0$][$0,1$]
%
\end{egame}
\hspace*{\fill}
\caption[]{Yet another extensive game}\label{f:three}
\end{figure}
\end{verbatim}

\begin{figure}[htb]
\hspace*{\fill}
\egpayoffboxsep=0mm
\begin{egame}(600,480)
\putbranch(300,240)(1,0){200}
\iib{$c$}{${1\over 2}$}{${1\over 2}$}
\putbranch(100,240)(0,1)[r]{100}
\iib{1}{$a$}{$b$}
\putbranch(500,240)(0,1)[l]{100}
\iib{1}{$b$}{$a$}
\infoset(100,340){400}{2}
\infoset(100,140){400}{2}
\putbranch(100,340)(1,1)[u]{100}
\egalhshift=20
\iib{}{$L$}{$R$}[$-1,0$][$0,-1$]
\putbranch(500,340)(1,1)[u]{100}
\egalhshift=20
\iib{}{$L$}{$R$}[$1,0$][$0,1$]
\putbranch(100,140)(1,1)[d]{100}
\egalhshift=20
\iib{}{$L$}{$R$}[$2,0$][$0,2$]
\putbranch(500,140)(1,1)[d]{100}
\egalhshift=20
\iib{}{$L$}{$R$}[$3,0$][$0,3$]
\end{egame}
\hspace*{\fill}
\caption[]{Yet another extensive game}\label{f:three}
\end{figure}

The next example, shown in Figure~\ref{f:four}, shows how to combine calls to
\verb+\iib+ to draw nodes followed by four branches.

\begin{verbatim}
\begin{figure}[htb]
\hspace*{\fill}
\begin{egame}(1200,380)
%
% put an initial node at (700,340), with (x,y) direction
% (3,1), and horizontal length 600
\putbranch(700,340)(3,1){600}
%
% give the branch two actions, label it for player $1$,
% label the actions $A$ and $D$, and make the right-
% hand node terminal, with payoffs $0,1$.
\iib[linecolor=red][]{$1$}{$A$}{$D$}[][$0,1$]
%
% to add two more branches to the initial node, force
% the branch to be initial
\initialtrue
% and specify the direction (1,1) and the horizontal length 200
\putbranch(700,340)(1,1){200}
%
% tighten the spacing between labels and branches, to improve
% appearance (given the other branches)
\egactionlabelsep=0.5mm
% give the branch two actions and label the actions $B$ and $C$
\iib{}{$B$}{$C$}
%
% reset default spacing
\egactionlabelsep=1mm
% put a branch at (100,140), with direction (1,1)
% and horizontal length 100.
\putbranch(100,140)(1,1){100}
%
% give the branch two actions, omit a player label,
% label the actions $a$ and $b$, and put payoffs
\iib[][linecolor=red]{}{$a$}{$b$}[$2,1$][$4,0$]
%
% put a branch at (500,140), with direction (1,1)
% and horizontal length 100.
\putbranch(500,140)(1,1){100}
%
% give the branch two actions, omit a player label,
% label the actions $a$ and $b$, and put payoffs
\iib[][linecolor=red]{}{$a$}{$b$}[$1,3$][$-1,0$]
%
% put a branch at (900,140), with direction (1,1)
% and horizontal length 100.
\putbranch(900,140)(1,1){100}
%
% give the branch two actions, omit a player label,
% label the actions $a$ and $b$, and put payoffs
\iib[][linecolor=red]{}{$a$}{$b$}[$-1,0$][$2,1$]
%
% put an information set at (100,140), of length 800,
% assigned to player 2
\infoset(100,140){800}{2}
%
\end{egame}
\hspace*{\fill}
\caption[]{Yet another extensive game}\label{f:four}
\end{figure}
\end{verbatim}

\begin{figure}[htb]
\hspace*{\fill}
\begin{egame}(1300,380)
%
% put an initial node at (700,340), with (x,y) direction
% (6,1), and horizontal length 600
\putbranch(700,340)(3,1){600}
%
% give the branch two actions, label it for player $1$,
% and label the actions $A$ and $D$, and make the right-
% hand node terminal, with payoffs $0,1$.
\iib[linecolor=red][]{$1$}{$A$}{$D$}[][$0,1$]
%
% to add two more branches to the initial node, force
% the branch to be initial
\initialtrue
% and specify the direction (2,1) and the horizontal length 200
\putbranch(700,340)(1,1){200}
%
% tighten the spacing between labels and branches, to improve
% appearance (given the other branches)
\egactionlabelsep=0.2mm
% give the branch two actions and label the actions $B$ and $C$
\iib{}{$B$}{$C$}
%
% reset default spacing
\egactionlabelsep=0.7mm
% put a branch at (100,140), with direction (1,1)
% and horizontal length 100.
\putbranch(100,140)(1,1){100}
%
% give the branch two actions, omit a player label,
% label the actions $a$ and $b$, and put payoffs
\iib[][linecolor=red]{}{$a$}{$b$}[$2,1$][$4,0$]
%
% put a branch at (500,140), with direction (1,1)
% and horizontal length 100.
\putbranch(500,140)(1,1){100}
%
% give the branch two actions, omit a player label,
% label the actions $a$ and $b$, and put payoffs
\iib[][linecolor=red]{}{$a$}{$b$}[$1,3$][$-1,0$]
%
% put a branch at (900,140), with direction (1,1)
% and horizontal length 100.
\putbranch(900,140)(1,1){100}
%
% give the branch two actions, omit a player label,
% label the actions $a$ and $b$, and put payoffs
\iib[][linecolor=red]{}{$a$}{$b$}[$-1,0$][$2,1$]
%
% put an information set at (100,140), of length 800,
% assigned to player 2
\infoset(100,140){800}{2}
%
\end{egame}
\hspace*{\fill}
\caption[]{Yet another extensive game}\label{f:four}
\end{figure}


\section{Description of macros\label{s:descr}}
\verb+\begin{egame}(+\emph{width}\verb+,+\emph{height}%
{\tt )[}\emph{unitlength}\verb+]+\newline
\noindent begins an extensive game of width \emph{width} and height
\emph{height} and optionally sets the unitlength to be \emph{unitlength}
(default 0.1mm).  In the game, all distances are given as integers, which are
interpreted as multiples of the unitlength.  These integers should (probably)
be divisible by two, so that the integer arithmetic employed by \TeX\ doesn't
lose accuracy when numbers are divided by two.  I have tested the macros
thoroughly only with the default unitlength of 0.1mm; unless there is a
compelling reason to use some different unitlength, I suggest sticking to
0.1mm.  (\verb+\begin{egame}(w,h)+ starts a \verb+pspicture+ environment
(defined by PSTricks) of width {\tt w} and height {\tt h}.)

Permissible values:
\begin{description}
\item[\emph{width}, \emph{height}] Any pair of integers.  (You need to
calculate these numbers.)
\item[\emph{unitlength}] Any dimension.  Default: 0.1mm.
\end{description}

\vspace{4mm}

\noindent\verb+\end{egame}+\newline
\noindent ends an extensive game.

\vspace{4mm}

\noindent\verb+\putbranch(+\emph{x-coord}\verb+,+\emph{y-coord}{\tt
)(}\emph{h-incr},\emph{v-incr}{\tt
)[}\emph{direction}\verb+]{+\emph{length}\verb+}+\newline
\noindent sets up the parameters for a branch at the point
$($\emph{x-coord}$,$\emph{y-coord}$)$, with direction parameter
$($\emph{h-incr}$,$\emph{v-incr}$)$, optional direction \emph{direction}, and
length \emph{length}.  Note that this macro merely sets up the parameters for
a call to \verb+\ib+ (one branch), \verb+\iib+ (two branches), \verb+\iiib+
(three branches), or \verb+\ctmb+ (continuum of branches); it does not draw
anything.  The way in which the direction parameter
$($\emph{h-incr}$,$\emph{v-incr}$)$ is interpreted depends on whether
\verb+\ib+, \verb+\iib+, or \verb+\iiib+ is used to draw the branches.  The
\emph{length} is the horizontal distance between the ends of one of the
branches to be drawn, unless \emph{h-incr} is 0, in which case \emph{length}
is the vertical distance between the ends of one of the branches.

Permissible values:
\begin{description}
\item[$($\emph{x-coord}$,$\emph{y-coord}$)$] Any pair of integers.
\item[$($\emph{h-incr}$,$\emph{v-incr}$)$] Any pair of integers except
\verb+(0,0)+.
\item[\emph{direction}] {\tt d} (down), {\tt u} (up), {\tt r} (right), or {\tt
l} (left); default {\tt d}.  (The direction can be changed also globally, by
specifying \verb+\egdirection{+\emph{direction}\verb+}+ before the call to
\verb+\putbranch+.)
\item[\emph{length}] Any positive integer.
\end{description}

\vspace{4mm}

\noindent\verb+\ib[+\emph{branchstyle}\verb+]{+\emph{player-name}\verb+}[+%
\emph{player-label-position}\verb+]{+\emph{action-label}\verb+}+\newline
\verb+[+\emph{action-label-position}\verb+][+\emph{payoffs}{\tt ]}\newline
\noindent puts a single branch with the parameters of the preceding call to
\verb+\putbranch+, optionally using the PSTricks style \emph{branchstyle},
assigns it the player name \emph{player-name}, optionally positions the player
label relative to the node according to \emph{player-label-position}, labels
the action \emph{action-label}, optionally positions the label according to
\emph{player-label-position}, and optionally adds the payoffs \emph{payoffs}. 
A single optional argument at the end is interpreted as follows: if its value
is \verb+o+ (outside), \verb+i+ (inside), or \verb+c+ (centered), it
determines the position of the action label; otherwise, it is a payoff label. 
If there are two optional arguments, the first must be \verb+o+ (outside),
\verb+i+ (inside), or \verb+c+ (centered), determining the position of the
action label, and the second is a payoff label.  (If you do not want an
optional positioning argument and you want your payoff label to be ``o'',
``i'', or ``c'', here's a workaround: put the label in an \verb+hbox+, as in
\verb+\hbox{o}+.  (Note that you do not need to do this if your payoff label
is \verb+$o$+, etc.---only if it is just ``o''.)

The first branch in any {\tt egame} is taken to be the initial branch of the
game; its beginning node is indicated by \verb+\eginode+, the default of which
is a small circle.  Subsequent branches are taken to be noninitial, and are
indicated by \verb+\egnode+, the default value of which is a small disk.  To
force a node to be initial, specify \verb+\initialtrue+ before it; to force a
node to be noninitial, specify \verb+\initialfalse+ before it.  To change the
appearance of nodes, see Section~\ref{s:params}.

Permissible values:
\begin{description}
\item[\emph{branchstyle}] Any PSTricks linestyle (e.g.\ \verb+linecolor=red+,
\verb+linestyle=dashed+, \verb+linewidth=2pt+, \verb+doubleline=true+).

Default: \verb+linecolor=black+, \verb+linestyle=solid+,
\verb+linewidth=0.8pt+.
\item[\emph{player-name}] Any character string.
\item[\emph{player-label-position}] Either \verb+o+ (centered over the node),
or
\begin{itemize}
   \item If the direction of the branch (as specified either by
\verb+\egdirection+ or by an optional argument of \verb+\putbranch+) is {\tt
d} or {\tt u}, either {\tt l} (above/below and to the left) and {\tt r}
(above/below and to the right).
   \item If the direction of the branch is {\tt r} or {\tt l}, either {\tt u}
(to the left/right and up) and {\tt d} (to the left/right and down). 
\end{itemize}
Default: is to center the label above for \verb+\egdirection+=\verb+d+, below
for \verb+\egdirection+=\verb+u+, to the left of the node for
\verb+\egdirection+=\verb+r+, or to the right of the node for
\verb+\egdirection+=\verb+l+.
\item[\emph{action-label}] Any character string.
\item[\emph{action-label-position}] \verb+o+ (outside), \verb+i+ (inside), or
\verb+c+ (centered on branch).

Default: \verb+o+.  (For more on action label positioning, see
Section~\ref{s:alp}.)
\item[\emph{payoffs}] Any character string.
\end{description}

Examples:
\begin{verbatim}
\putbranch(0,100)(2,-1){200}
\ib{1}{$x$}
\end{verbatim}
$$\begin{egame}(200,180)
\putbranch(0,140)(2,-1){200}
\ib{1}{$x$}
\psline{<->}(0,0)(200,0)
\rput*(100,0){200}
\end{egame}$$

Note that in the following example the \verb+1+ in the direction pair
\verb+(0,1)+ is interpreted as sending the branch down, given that the
\verb+\egdirection+ is \verb+d+.  (That is, once you specify the
\verb+\egdirection+ (or leave it at the default \verb+d+) you do not need to
worry about getting the sign of the \emph{v-incr} correct (if the direction is
\verb+d+ or \verb+u+) or the sign of the \emph{h-incr} correct (if the
direction is \verb+l+ or \verb+r+).

\begin{verbatim}
\putbranch(0,100)(0,1){100}
\ib[linecolor=gray]{1}[l]{$x$}[c]
\end{verbatim}
$$\begin{egame}(100,140)
\putbranch(0,100)(0,1){100}
\ib[linecolor=gray]{1}[l]{$x$}[c]
\psline{<->}(80,0)(80,100)
\rput*(80,50){100}
\end{egame}$$

\begin{verbatim}
\initialfalse
\putbranch(0,40)(1,1)[u]{100}
\ib{1}{$x$}[$0,2$]
\end{verbatim}
$$\begin{egame}(100,180)
\initialfalse
\putbranch(0,40)(1,1)[u]{100}
\ib{1}{$x$}[$0,2$]
\end{egame}$$

Notice that the optional direction specifier {\tt u} in \verb+\putbranch+
affects not only the direction of the branch (which is determined by the
argument {\tt (1,1)}), but also the placement of the player label relative to
the starting node.

\begin{verbatim}
\initialfalse
\putbranch(0,100)(1,1){100}
\ib{1}[l]{$x$}[i][$0,2$]
\end{verbatim}
$$\begin{egame}(100,180)
\putbranch(0,140)(1,-1){100}
\ib{1}[l]{$x$}[i][$0,2$]
\end{egame}$$

\vspace{4mm}

\noindent\verb+\iib[+\emph{style1}\verb+][+\emph{style2}\verb+]{+\emph{player-%
name}\verb+}[+%
\emph{player-label-position}\verb+]{+\emph{action-label1}\verb+}+\newline
\verb+{+\emph{action-label2}\verb+}[+\emph{action-label-position}\verb+][+%
\emph{payoffs1}%
\verb+][+\emph{payoffs2}{\tt ]}\newline
\noindent puts two branches with the parameters of the preceding call to
\verb+\putbranch+, optionally using the PSTricks style \emph{branchstyle1} for
the left or upper branch and the style \emph{branchstyle2} for the other
branch (or \emph{branchstyle1} for both branches, if there is only one
optional argument), assigns it the player name \emph{player-name}, optionally
positions the player label relative to the node according to
\emph{player-label-position}, labels the left/upper action with
\emph{action-label1} and the right/lower action with \emph{action-label2},
optionally using the position \emph{action-label-position}, and optionally
adds payoffs \emph{payoffs1} and \emph{payoffs2}.  The signs of the direction
parameters $($\emph{h-incr}$,$\emph{v-incr}$)$ in the preceding
\verb+\putbranch+ call are ignored; the directions of the branches is
determined by the direction of the branch (as specified either by
\verb+\egdirection+ or by the optional argument of \verb+\putbranch+).  If,
for example, the direction is {\tt d}, then one branch goes down and to the
left and the other goes down and to the right.

The first branch in any {\tt egame} is taken to be the initial branch of the
game; its beginning node is indicated by \verb+\eginode+.  Subsequent branches
are taken to be non-initial, and are indicated by \verb+\egnode+.  To force a
node to be initial, specify \verb+\initialtrue+ before it; to force a node to
be noninitial, specify \verb+\initialfalse+ before it.

Permissible values:
\begin{description}
\item[\emph{style1}, \emph{style2}] Any PSTricks line styles; see \verb+\ib+. 
If there is only one style present, it is applied to both branches.  If there
are two styles, the first is applied to the right hand branch and the second
to the left hand branch if the game direction is up or down, and the first is
applied to the lower branch and the second to the upper branch if the game
direction is left or right.
\item[\emph{player-name}] Any character string.
\item[\emph{player-label-position}] Either \verb+o+ (centered over the node),
or
\begin{itemize}
   \item If the direction of the branch (as specified either by
\verb+\egdirection+ or by an optional argument of \verb+\putbranch+) is {\tt
d} or {\tt u}, either {\tt l} (above/below and to the left) and {\tt r}
(above/below and to the right).
   \item If the direction of the branch is {\tt r} or {\tt l}, either {\tt u}
(to the left/right and up) and {\tt d} (to the left/right and down). 
\end{itemize}
Default: is to center the label above for \verb+\egdirection+=\verb+d+, below
for \verb+\egdirection+=\verb+u+, to the left of the node for
\verb+\egdirection+=\verb+r+, or to the right of the node for
\verb+\egdirection+=\verb+l+.
\item[\emph{action-label1} and \emph{action-label2}] Any character strings.
\item[\emph{action-label-position}] \verb+o+ (outside), \verb+i+ (inside), or
\verb+c+ (centered).  Default: \verb+o+.
\item[\emph{payoffs1} and \emph{payoffs2}] Any character strings.
\end{description}

Examples:

Action labels ``outside'' (default):
\begin{verbatim}
\putbranch(100,140)(1,1){100}
\iib[linestyle=dashed]{1}{$L$}{$R$}
\end{verbatim}
$$\begin{egame}(200,180)
\putbranch(100,140)(1,1){100}
\iib[linestyle=dashed]{1}{$L$}{$R$}
\psline{<->}(100,0)(200,0)
\rput*(150,0){100}
\end{egame}$$

Action labels ``inside'':
\begin{verbatim}
\putbranch(100,140)(1,1){100}
\iib[linestyle=dashed]{1}{$L$}{$R$}[i]
\end{verbatim}
$$\begin{egame}(200,140)
\putbranch(100,100)(1,1){100}
\iib[linestyle=dashed]{1}{$L$}{$R$}[i]
\end{egame}$$

Action labels ``centered'':
\begin{verbatim}
\renewcommand{\egarrowstyle}{e}
\putbranch(0,140)(2,1)[r]{200}
\iib{1}{$L$}{$R$}[c]
\end{verbatim}
$$\begin{egame}(100,280)
\renewcommand{\egarrowstyle}{e}
\putbranch(0,140)(2,1)[r]{200}
\iib{1}{$L$}{$R$}[c]
\psline{<->}(0,0)(200,0)
\rput*(100,0){200}
\end{egame}$$

\begin{verbatim}
\renewcommand{\egarrowstyle}{e}
\putbranch(0,100)(1,1)[r]{100}
\iib[linecolor=gray][doubleline=true]{1}{$L$}{$R$}[$1,2$][$0,-1$]
\end{verbatim}
$$\begin{egame}(100,250)
\renewcommand{\egarrowstyle}{e}
\putbranch(0,100)(1,1)[r]{100}
\iib[linecolor=gray][doubleline=true]{1}{$L$}{$R$}[c][$1,2$][$0,-1$]
\end{egame}$$

\begin{verbatim}
\renewcommand{\egarrowstyle}{m}
\psset{arrowscale=2}
\putbranch(200,40)(1,0)[d]{200}
\iib{1}{$L$}{$R$}[$1,2$][$0,-1$]
\end{verbatim}
$$\begin{egame}(400,80)
\renewcommand{\egarrowstyle}{m}
\psset{arrowscale=2}
\putbranch(200,40)(1,0)[d]{200}
\iib{1}{$L$}{$R$}[$1,2$][$0,-1$]
\end{egame}$$

\vspace{4mm}

\psset{arrowscale=1}

\noindent\verb+\iiib[+\emph{style1}\verb+]{+\emph{player-name}\verb+}[+\emph{p%
layer-label-position}%
\verb+]{+\emph{action-label1}\verb+}+\newline
\verb+{+\emph{action-label2}\verb+}{+\emph{action-label3}\verb+}[+\emph{action%
-label-position}\verb+][+%
\emph{payoffs1}\verb+][+\emph{payoffs2}\verb+]+\newline
\verb+[+\emph{payoffs3}{\tt ]}\newline
\noindent puts three branches with the parameters of the preceding call to
\verb+\putbranch+, optionally all in the PSTricks style \emph{style}, assigns
it the player name \emph{player-name}, optionally positions the player label
relative to the node according to \emph{player-label-position}, and labels the
left/top action with \emph{action-label1}, the middle action with
\emph{action-label3}, and the right/bottom action with \emph{action-label2},
optionally positioning the labels according to \emph{action-label-position}
and putting the \emph{payoffs1}, \emph{payoffs2}, and \emph{payoffs3}.  The
signs of the direction parameters $($\emph{h-incr}$,$\emph{v-incr}$)$ in the
preceding \verb+\putbranch+ call are ignored; the directions of the branches
are determined by the direction of the branch (as specified either by
\verb+\egdirection+ or by the optional argument of \verb+\putbranch+).  If,
for example, the direction is {\tt d}, then one branch goes down and to the
left, one goes straight down, and one goes down and to the right.

The first branch in any {\tt egame} is taken to be the initial branch of the
game; its beginning node is indicated by a small circle.  Subsequent branches
are taken to be non-initial, and are indicated by \verb+\egnode+, the default
value of which is a small disk.  To force a node to be initial, specify
\verb+\initialtrue+ before it; to force a node to be noninitial, specify
\verb+\initialfalse+ before it.

Permissible values:
\begin{description}
\item[\emph{style}] Any PSTricks line style.  [Note that only one style is
allowed, which applies to all three branches.  If you want the branches to
have different styles, you need to use a combination of \verb+\ib+ and
\verb+\iib+.]
\item[\emph{player-name}] Any character string.
\item[\emph{player-label-position}] Either \verb+o+ (centered over the node),
or
\begin{itemize}
   \item If the direction of the branch (as specified either by
\verb+\egdirection+ or by an optional argument of \verb+\putbranch+) is {\tt
d} or {\tt u}, either {\tt l} (above/below and to the left) and {\tt r}
(above/below and to the right).
   \item If the direction of the branch is {\tt r} or {\tt l}, either {\tt u}
(to the left/right and up) and {\tt d} (to the left/right and down). 
\end{itemize}
Default: is to center the label above for \verb+\egdirection+=\verb+d+, below
for \verb+\egdirection+=\verb+u+, to the left of the node for
\verb+\egdirection+=\verb+r+, or to the right of the node for
\verb+\egdirection+=\verb+l+.
\item[\emph{action-label1}, \emph{action-label2}, and \emph{action-label3}]
Any character strings.
\item[\emph{action-label-position}] \verb+o+ (outside), \verb+i+ (inside), or
\verb+c+ (centered).  Default: \verb+o+.
\item[\emph{payoffs1}, \emph{payoffs1}, and \emph{payoffs3}] Any character
strings.
\end{description}

Examples:
\begin{verbatim}
\putbranch(200,240)(1,1){200}
\iiib{1}{$L$}{$M$}{$R$}
\end{verbatim}
$$\begin{egame}(400,280)
\putbranch(200,240)(1,1){200}
\iiib{1}{$L$}{$M$}{$R$}
\psline{<->}(200,0)(400,0)
\rput*(300,0){200}
\end{egame}$$

\begin{verbatim}
\putbranch(30,210)(1,1)[r]{200}
\iiib[linecolor=gray]{1}{$B$}{$M$}{$T$}[c][$1,2$][$0,-1$][$7,1$]
\end{verbatim}
$$\begin{egame}(340,460)
\putbranch(30,250)(1,1)[r]{200}
\iiib[linecolor=gray]{1}{$B$}{$M$}{$T$}[c][$1,2$][$0,-1$][$7,1$]
\psline{<->}(30,0)(230,0)
\rput*(130,0){200}
\end{egame}$$

\vspace{4mm}

\noindent
\verb+\ctmb[+\emph{style}\verb+]{+\emph{player-name}\verb+}[+\emph{player-labe%
l-position}%
\verb+](+\emph{h-incr}\verb+,+\emph{v-incr}\verb+)+\newline
\verb+{+\emph{action-label}\verb+}[+\emph{action-label-position}\verb+][+\emph%
{payoffs}%
\verb+]+\newline
\noindent draws a continuum of branches starting with the parameters of the
previous call to \verb+\putbranch+, with a single branch drawn as a line with
PSTricks style \emph{style}, player \emph{player}, with the label optionally
positioned according to \emph{player-label-position}, and slope
\emph{h-incr,v-incr}, labeled with \emph{action-label}, which is optionally
positioned according to \emph{action-label-position}, and optionally with
payoffs \emph{payoffs}.

\sloppy
The color of the triangle representing the continuum of branches is
\verb+\ctmfillcolor+ (default \verb+verylightgray+ (defined in the style)).


\fussy
Permissible values:
\begin{description}
\item[\emph{style}] Any PSTricks line style.  [Note that this is the style of
the single branch, {\bfseries not} of the triangle.]
\item[\emph{player-name}] Any character string.
\item[\emph{player-label-position}] Either \verb+o+ (centered over the node),
or
\begin{itemize}
   \item If the direction of the branch (as specified either by
\verb+\egdirection+ or by an optional argument of \verb+\putbranch+) is {\tt
d} or {\tt u}, either {\tt l} (above/below and to the left) and {\tt r}
(above/below and to the right).
   \item If the direction of the branch is {\tt r} or {\tt l}, either {\tt u}
(to the left/right and up) and {\tt d} (to the left/right and down). 
\end{itemize}
Default: is to center the label above for \verb+\egdirection+=\verb+d+, below
for \verb+\egdirection+=\verb+u+, to the left of the node for
\verb+\egdirection+=\verb+r+, or to the right of the node for
\verb+\egdirection+=\verb+l+.
\item[$($\emph{h-incr}$,$\emph{v-incr}$)$] Any pair of integers.
\item[\emph{action-label}] Any character string.
\item[\emph{action-label-position}] \verb+o+ (outside), \verb+i+ (inside), or
\verb+c+ (centered).  Default: \verb+o+.
\item[\emph{payoffs}] Any character strings.
\end{description}

Examples:

\begin{verbatim}\putbranch(50,240)(2,-1)[d]{200}
\ctmb{1}(0,1){$A$}[i][$2,2$]:\end{verbatim}
$$\begin{egame}(100,280)
\putbranch(50,240)(2,-1)[d]{200}
\ctmb{1}(0,1){$A$}[i][$2,2$]
\end{egame}$$

\begin{verbatim}\renewcommand{\ctmfillcolor}{red}
\renewcommand{\egarrowstyle}{e}
\putbranch(50,240)(2,-1)[d]{200}
\ctmb[linecolor=blue]{1}(-1,1){$A$}[i][$2,2$]:\end{verbatim}
$$\begin{egame}(100,280)
\renewcommand{\ctmfillcolor}{red}
\renewcommand{\egarrowstyle}{e}
\putbranch(50,240)(2,-1)[d]{200}
\ctmb[linecolor=blue]{1}(-1,1){$A$}[i][$2,2$]
\end{egame}$$

\begin{verbatim}\putbranch(50,0)(2,1)[u]{200}
\ctmb{1}(1,1){$A$}[i][$2,2$]:\end{verbatim}
$$\begin{egame}(100,180)
\putbranch(50,0)(2,1)[u]{200}
\ctmb{1}(1,1){$A$}[i][$2,2$]
\end{egame}$$

\vspace{4mm}

\noindent
\verb+\ctmbarc[+\emph{style}\verb+]{+\emph{player-name}\verb+}[+\emph{player-l%
abel-position}%
\verb+](+\emph{h-incr}\verb+,+\emph{v-incr}\verb+)+\newline
\verb+{+\emph{action-label}\verb+}[+\emph{action-label-position}\verb+][+\emph%
{payoffs}%
\verb+]+\newline
\noindent represents a continuum of branches by an arc.  The branches start
with the parameters of the previous call to \verb+\putbranch+, with a single
branch and the arc drawn using the PSTricks style \emph{style}, player
\emph{player}, the label optionally positioned according to
\emph{player-label-position}, the line with slope \emph{h-incr,v-incr},
labeled with \emph{action-label} optionally positioned according to
\emph{action-label-position}, and optionally with payoffs \emph{payoffs}.

\emph{This macro requires the package} \verb+pst-3d+ \emph{to be loaded after}
\verb+pstricks+\emph{.  In the preamble to your document, include the line}
\verb+\include{pst-3d}+.  (\verb+\pstricks+ itself includes the needed macro,
but its definition appears to differ from the one in \verb+pst-3d+, which is 
the definition \verb+\ctmbarc+ needs.)

The line drawn by the macro is the intersection of (\emph{i})~a circle
centered at the start of the branch with radius equal to a half of the
distance to the following node and (\emph{ii})~the triangle defined by the
starting node and the two following nodes.  For some parameter values, this
line is not a full arc---see the last example.

The options are the same as those for \verb+\ctmb+.  The style of the arc is
controlled by the parameters \verb+\egarclinestyle+ (possible values
\verb+solid+, \verb+dashed+, \verb+dotted+, and \verb+none+; default
\verb+solid+), \verb+\egarclinewidth+ (any dimension; default \verb+0.6pt+),
\verb+\egarclinecolor+ (any color; default \verb+black+).

Examples:

\begin{verbatim}\putbranch(50,240)(2,2)[d]{200}
\ctmbarc{1}(1,-2){$A$}[i][$2,2$]:\end{verbatim}
$$\begin{egame}(100,280)
\putbranch(50,240)(2,2)[d]{200}
\egalhshift=-12
\ctmbarc{1}(1,-2){$A$}[i][$2,2$]
\end{egame}$$

\begin{verbatim}\def\egarclinestyle{dashed}
\def\egarclinewidth{2pt}
\def\egarclinecolor{blue}
\putbranch(0,150)(2,2)[r]{150}
\ctmbarc[linewidth=0.3pt,linecolor=red]{1}(1,0){$A$}[$2,2$]:\end{verbatim}
$$\begin{egame}(150,300)
\def\egarclinestyle{dashed}\def\egarclinewidth{2pt}\def\egarclinecolor{blue}
\putbranch(0,150)(2,2)[r]{150}
\ctmbarc[linewidth=0.3pt,linecolor=red]{1}(1,0){$A$}[$2,2$]
\end{egame}$$

In the next example part of the arc extends beyond the triangle defined by the
three nodes, and consequently does not appear.

\begin{verbatim}\putbranch(50,0)(2,1)[u]{200}
\ctmbarc{1}(1,1){$A$}[i][$2,2$]:\end{verbatim}
$$\begin{egame}(100,180)
\putbranch(50,0)(2,1)[u]{200}
\ctmbarc{1}(1,1){$A$}[i][$2,2$]
\end{egame}$$

\vspace{4mm}

\noindent\verb+\infoset(+\emph{x-coord}\verb+,+\emph{y-coord}%
\verb+)[+\emph{direction}\verb+]{+\emph{length}%
\verb+}{+\emph{player-name}\verb+}+\newline
\verb+[+\emph{player-label-position}\verb+]+\newline
\noindent draws an information set starting at
$($\emph{x-coord}$,$\emph{y-coord}$)$, optionally with direction
\emph{direction}, of length \emph{length}, with player label
\emph{player-name}.  If the direction of the game is either down or up, the
information set is horizontal; if the direction of the game is either right or
left, the information set is vertical.  The player label is positioned at the
middle of the information set, either above, below, to the left, or to the
right of it, depending on the direction of the game.  (Its position can be
adjusted by adding some space before or after the player name.  For
example,\newline \verb+\infoset(100,200){400}{1\hspace{10mm}}+\newline
centers the box containing \verb+1\hspace{10mm}+ relative to the information
set, thus moving the label by about 5mm.

The dot character used in the information set is given by \verb+\infosetdot+,
the default value of which is \verb+\pscircle*{2.5}+; the spacing between the
dots is set by \verb+\infosetdotsep+, the default value of which is 20.  (See
Section~\ref{s:params}.)

Permissible values:
\begin{description}
\item[$($\emph{x-coord}$,$\emph{y-coord}$)$] Any pair of integers.
\item[\emph{direction}] {\tt h} (horizontal) or {\tt v} (vertical).  If none
is specified, it is assumed to be \verb+h+ if the game direction is \verb+d+
or \verb+u+, and \verb+v+ if the game direction is \verb+r+ or \verb+l+.
\item[\emph{length}] Any nonnegative integer.
\item[\emph{player-name}] Any character string.
\item[\emph{player-label-position}] \verb+o+ (over), \verb+u+ (up), or
\verb+d+ (down) if direction is \verb+h+, \verb+o+ (over), \verb+l+ (left), or
\verb+r+ (right) if direction is \verb+v+.
\end{description}

Examples:
\begin{verbatim}
\egdirection{d}
\infoset(0,0){300}{1}
\end{verbatim}
$$\begin{egame}(300,4)
\egdirection{d}
\infoset(0,0){300}{1}
\end{egame}$$

\begin{verbatim}
\egdirection{d}
\infoset(0,0){300}{1}[o]
\end{verbatim}
$$\begin{egame}(300,4)
\egdirection{d}
\infoset(0,0){300}{1}[o]
\end{egame}$$

\begin{verbatim}
\egdirection{d}
\infoset(0,0)[u]{200}{1}
\end{verbatim}
$$\begin{egame}(0,200)
\egdirection{d}
\infoset(0,0)[v]{200}{1}
\end{egame}$$

\section{Positioning action labels}\label{s:alp}
The following algorithm is used to position an action label.  The ``reference
point'' for a branch is its midpoint.  An action label is surrounded by a box
(which may be colored and bordered by a line), from which it is separated by
\verb+\egactionboxsep+, then this box is surrounded by another box, from which
it is separated by \verb+\egactionlabelsep+.  Then, for example, for a label
positioned above a downward-sloping branch, the bottom left-hand corner of the
outer box is placed at the reference point, as in Figure~\ref{f:al}.  For a
label positioned above a horizontal branch, the bottom center of the outer box
is placed at the reference point.
\begin{figure}[htb]
\hspace*{\fill}
\begin{egame}(400,200)[0.2mm]
\putbranch(0,200)(1,1){200}
\ib{}{}
\pspolygon[linewidth=0.4pt](100,100)(100,200)(200,200)(200,100)
\pspolygon[linewidth=0.4pt](120,120)(120,180)(180,180)(180,120)
\pspolygon[linewidth=0.4pt](145,145)(145,155)(155,155)(155,145)
\rput(150,150){o}
\psline{<->}(150,100)(150,120)
\rput[l](160,110){$x$}
\psline{<->}(150,120)(150,145)
\rput[l](160,132.5){$y$}
\rput[l](220,160){$y=\setminus $egactionboxsep}
\rput[l](220,130){$x=\setminus $egactionlabelsep}
\end{egame}
\hspace*{\fill}
\caption[]{The position of an action label above a downward-sloping
branch.}\label{f:al}
\end{figure}

This algorithm is not perfect.  (Note that it is not continuous in the slope
of the branch.)  The positions of labels may need fine tuning, which may be
achieved by setting either \verb+\egalhshift+ or \verb+\egalvshift+ or both to
be nonzero (each is an integer, interpreted as a multiple of the unitlength).

\verb+\egalhshift+ and \verb+\egalvshift+ must be set between
\verb+\putbranch+ and \verb+\ib+, \verb+\iib+, \verb+\iiib+, or \verb+\ctmb+. 
(They are set to zero by \verb+\putbranch+.)

If \verb+\egalhshift+ is nonzero and \verb+\egalvshift+ is not, the reference
point for the action label on the single branch in \verb+\ib+, of the
\emph{right-hand} branch for directions \verb+d+ and \verb+u+ and of both
branches for directions \verb+r+ and \verb+l+ in \verb+\iib+, and for the
outer branches of \verb+\iiib+, is moved horizontally by this amount and is
moved vertically to maintain the same separation from the branch.  (That is,
the action label slides parallel to the branch; it moves horizontally by the
amount \verb+\egalhshift+.)  The reference point of the label on the symmetric
branch is moved symmetrically.  (Note that the label on the middle branch of
\verb+\iiib+ is not moved.  If you want to move it, you need to write separate
calls to \verb+\iib+ and \verb+\ib+.)

Similarly, if \verb+\egalvshift+ is nonzero and \verb+\egalhshift+ is not, the
reference point for the action label is moved vertically by this amount and is
moved horizontally to maintain the same separation from the branch.  After a
branch is drawn, \verb+\egalhshift+ and \verb+\egalvshift+ are reset to zero.

Note that the same effect can be acheived with either \verb+\egalhshift+ or
\verb+\egalvshift+.  The two methods are provided simply because in some cases
 a vertical shift might be more natural to specify, whereas in other cases a
horizontal shift might be more natural to specify.

If \emph{both} \verb+\egalhshift+ and \verb+\egalvshift+ are nonzero, the
reference point is moved horizontally by \verb+\egalhshift+ and vertically by
\verb+\egalvshift+.

Examples:

\verb+\egdirection{d}+
\verb+\egalhshift=40+
$$\begin{egame}(100,280)
\putbranch(50,240)(2,-1)[d]{200}
\egalhshift=40
\iib{1}{$A$}{$B$}[$1,1$][$2,2$]
\end{egame}$$

\verb+\egalhshift=-40+
$$\begin{egame}(100,280)
\putbranch(50,240)(2,-1)[d]{200}
\egalhshift=-40
\iib{1}{$A$}{$B$}[$1,1$][$2,2$]
\end{egame}$$

\verb+\egalhshift=40\egalvshift=40+
$$\begin{egame}(100,280)
\putbranch(50,240)(2,-1)[d]{200}
\egalhshift=40\egalvshift=40
\iib{1}{$A$}{$B$}[$1,1$][$2,2$]
\end{egame}$$

\verb+\egdirection{u}\egalhshift=60+
$$\begin{egame}(100,280)
\putbranch(40,140)(2,-1)[r]{200}
\egalhshift=60
\iib{1}{$A$}{$B$}[$1,1$][$2,2$]
\end{egame}$$
\section{Arrows}
Branches can have arrows either at the end or in the middle, by setting 

\verb+\renewcommand{\egarrowstyle}{e}+

\noindent (for arrows at the end), or

\verb+\renewcommand{\egarrowstyle}{m}+

\noindent (for arrows in the middle).  The positioning of arrows in the middle
is not ideal: the \emph{tip} of the arrow is placed in the middle of the
branch, while ideally the \emph{middle} of the arrow should be there.  I can't
see any easy way to improve the placement.

The style of the arrows can be controlled with PSTricks' various parameters
(as described, for example, in the \LaTeX\ Graphics Companion).
\section{Parameters}\label{s:params}
\noindent\verb+\egdirection+\newline
\noindent Direction of game.  Possible values: {\tt d} (down), {\tt u} (up),
{\tt r} (right), or {\tt l} (left).  Default: {\tt d}.  Example:
\verb+\egdirection{u}+.

\vspace{4mm}

\noindent\verb+\initialtrue+, \verb+\initialfalse+\newline
\noindent Force a node other than the first one in an \verb+egame+ to be
initial, or force the first node to be noninitial.

\vspace{4mm}

\sloppy
\noindent\verb+\eginode+\newline
\noindent Object used for initial node.  Possible values: any object. 
Default: \verb+\pscircle+ \verb+[linewidth=0.4pt]{5}+.  Example:
\verb+\renewcommand{\eginode}+ \verb+{\makebox(0,0){\rule{0.5mm}{0.5mm}}}+
(the \verb+\makebox+ causes the object to be positioned correctly).

\fussy
\vspace{4mm}

\noindent\verb+\egnode+\newline
\noindent Object used for nodes.  Possible values: any object.  Default:
\verb+\pscircle*{5}+.  Example:
\verb+\renewcommand{\egnode}{\makebox(0,0){\rule{0.5mm}{0.5mm}}}+ (the
\verb+\makebox+ causes the object to be positioned correctly).

\vspace{4mm}

\sloppy
\noindent\verb+\infosetdot+\newline
\noindent Object used for ``dots'' in information sets.  Possible values: any
object.  Default: \verb+\pscircle*{2.5}+.  Example:
\verb+\renewcommand{\infosetdot}+ \verb+{\pscircle*{5}}+.

\fussy
\vspace{4mm}

\noindent\verb+\infosetdotsep+\newline
\noindent Spacing between dots in information set, as a multiple of the unit
length.  Possible values: any positive integer.  Default: 20.  Example:
\verb+\renewcommand+ \verb+{\infosetdotsep}{40}+.

\vspace{4mm}

\noindent\verb+\egplayerlabelsep+\newline
\noindent Spacing used to position box containing player label relative to
center of initial node, and relative to center of information set.  Possible
values: any dimension.  Default: 1mm.  Example: \verb+\egplayerlabelsep=2mm+.

\vspace{4mm}

\noindent\verb+\egplayerboxsep+\newline
\noindent Distance between edge of box around player label and player label. 
Possible values: any dimension.  Default: 0mm.  Example:
\verb+\egplayerboxsep=1mm+.

\vspace{4mm}

\noindent\verb+\egactionlabelsep+\newline
\noindent Spacing used to position box containing action label relative to
branch.  Possible values: any dimension.  Default: 0.7mm.  Example:
\verb+\egactionlabelsep=+ \verb+1mm+.

\vspace{4mm}

\noindent\verb+\egactionboxsep+\newline
\noindent Distance between edge of box around action label and action label. 
Possible values: any dimension.  Default: 0mm.  Example:
\verb+\egactionboxsep=1mm+.

\vspace{4mm}

\sloppy
\noindent\verb+\egpayofflabelsep+\newline
\noindent Spacing used to position box containing action label relative to end
of branch.  Possible values: any dimension.  Default: 2mm.  Example:
\verb+\egpayofflabelsep=1mm+.

\fussy
\vspace{4mm}

\noindent\verb+\egpayoffboxsep+\newline
\noindent Distance between edge of box around payoffs and payoffs.  Possible
values: any dimension.  Default: 0mm.  Example: \verb+\egpayoffboxsep=1mm+.

\vspace{4mm}

\noindent\verb+\egalpos+\newline
\noindent Position of action labels.  Possible values: \verb+o+ (outside),
\verb+i+ (inside), \verb+c+ (center).  Default: \verb+o+.  Example:
\verb+\renewcommand{\egalpos}{c}+.

\vspace{4mm}

\noindent\verb+\egalbox+\newline
\noindent Type of box for action labels.  Possible values: \verb+f+ (frame),
\verb+c+ (circle).  Default: \verb+f+.  Example:
\verb+\renewcommand{\egalbox}{c}+.

\vspace{4mm}

\sloppy
\noindent\verb+\egalboxlinestyle+\newline
\noindent Style of lines around boxes containing action labels.  Possible
values: \verb+none+, \verb+solid+, \verb+dashed+, \verb+dotted+.  Default:
\verb+none+.  Example: \verb+\renewcommand+ \verb+{\egalboxlinestyle}{solid}+.

\fussy
\vspace{4mm}

\noindent\verb+\egalboxlinecolor+\newline
\noindent Color of lines around boxes containing action labels.  Possible
values: any defined color.  Default: \verb+black+.  Example:
\verb+\renewcommand{\egalboxlinecolor}+ \verb+{red}+.

\vspace{4mm}

\noindent\verb+\egalboxfillstyle+\newline
\noindent Style of fill of boxes containing action labels.  Possible values:
\verb+none+, \verb+solid+, \verb+vlines+, \verb+vlines*+, \verb+hlines+,
\verb+hlines*+, \verb+crosshatch+, \verb+crosshatch*+.  Default: \verb+none+. 
Example: \verb+\renewcommand{\egalboxfillstyle}{solid}+.

\vspace{4mm}

\sloppy
\noindent\verb+\egalboxfillcolor+\newline
\noindent Color of fill of boxes containing action labels.  Possible values:
any defined color.  Default: \verb+white+.  Example:
\verb+\renewcommand{\egalboxfillcolor}+ \verb+{blue}+.

\fussy
\vspace{4mm}

\noindent\verb+\egpaybox+\newline
\noindent Type of box for payoffs.  Possible values: \verb+f+ (frame),
\verb+c+ (circle).  Default: \verb+f+.  Example:
\verb+\renewcommand{\egpaybox}{c}+.

\vspace{4mm}

\sloppy
\noindent\verb+\egpayboxlinestyle+\newline
\noindent Style of lines around boxes containing payoffs.  Possible values:
\verb+none+, \verb+solid+, \verb+dashed+, \verb+dotted+.  Default:
\verb+none+.  Example: \verb+\renewcommand+
\verb+{\egpayboxlinestyle}{solid}+.

\fussy
\vspace{4mm}

\noindent\verb+\egpayboxlinecolor+\newline
\noindent Color of lines around boxes containing payoffs.  Possible values:
any defined color.  Default: \verb+black+.  Example:
\verb+\renewcommand{\egpayboxlinecolor}+ \verb+{gray}+.

\vspace{4mm}

\noindent\verb+\egpayboxfillstyle+\newline
\noindent Style of fill of boxes containing action payoffs.  Possible values:
\verb+none+, \verb+solid+, \verb+vlines+, \verb+vlines*+, \verb+hlines+,
\verb+hlines*+, \verb+crosshatch+, \verb+crosshatch*+.  Default: \verb+none+. 
Example: \verb+\renewcommand{\egpayboxfillstyle}{solid}+.

\vspace{4mm}

\noindent\verb+\egpayboxfillcolor+\newline
\noindent Color of fill of boxes containing payoffs.  Possible values: any
defined color.  Default: \verb+white+.  Example:
\verb+\renewcommand{\egpayboxfillcolor}{gray}+.

\vspace{4mm}

\noindent\verb+\egplbox+\newline
\noindent Type of box for player labels.  Possible values: \verb+f+ (frame),
\verb+c+ (circle).  Default: \verb+f+.  Example:
\verb+\renewcommand{\egplbox}{c}+.

\vspace{4mm}

\sloppy
\noindent\verb+\egplboxlinestyle+\newline
\noindent Style of lines around boxes containing player labels.  Possible
values: \verb+none+, \verb+solid+, \verb+dashed+, \verb+dotted+.  Default:
\verb+none+.  Example: \verb+\renewcommand+ \verb+{\egplboxlinestyle}{solid}+.

\fussy
\vspace{4mm}

\noindent\verb+\egplboxlinecolor+\newline
\noindent Color of lines around boxes containing player labels.  Possible
values: any defined color.  Default: \verb+black+.  Example:
\verb+\renewcommand{\egplboxlinecolor}+ \verb+{gray}+.

\vspace{4mm}

\noindent\verb+\egplboxfillstyle+\newline
\noindent Style of fill of boxes containing player labels.  Possible values:
\verb+none+, \verb+solid+, \verb+vlines+, \verb+vlines*+, \verb+hlines+,
\verb+hlines*+, \verb+crosshatch+, \verb+crosshatch*+.  Default: \verb+none+. 
Example: \verb+\renewcommand{\egplboxfillstyle}{solid}+.

\vspace{4mm}

\sloppy
\noindent\verb+\egplboxfillcolor+\newline
\noindent Color of fill of boxes containing player labels.  Possible values:
any defined color.  Default: \verb+white+.  Example:
\verb+\renewcommand{\egplboxfillcolor}+ \verb+{gray}+.

\fussy
\vspace{4mm}

\noindent\verb+\egarrowstyle+\newline
\noindent Style of arrows on branches.  Possible values: \verb+n+ (none),
\verb+e+ (end), \verb+m+ (mid).  Default: \verb+n+ (none).  Example:
\verb+\renewcommand{\egarrowstyle}{m}+ (mid).  The size of the arrows can be
adjusted by using any of the PSTricks parameters---e.g.\
\verb+\psset{arrowscale=1.5}+.

\vspace{4mm}

\sloppy
\noindent\verb+\ctmfillcolor+\newline
\noindent Color of triangle used to indicate continuum of branches.  Possible 
values: any defined color.  Default: \verb+verylightgray+.  Example:
\verb+\renewcommand+ \verb+{\ctmfillcolor}{red}+.

\fussy
\section{Suggestions}
It seems hard to proceed without first drawing the game on a piece of paper,
at least roughly.  Be sure to allow enough space under (in the case of a
downward-pointing game) the terminal nodes---put the nodes that precede the
terminal nodes high enough that the bottom of the payoffs will be at height 0.
 Similarly, specify the height of the game so that the label for the initial
player does not poke out the top.  If parts of your game poke out of the frame
defined by the size you specify for your \verb+egame+, \TeX\ will not warn
you, but the spacing above and/or below your game will not be right.  (If you
specify the height to be too small, for example, the top of your game may
overlap the text above it.)

You can float your game in a figure (as I have done in the examples above), or
you can put it in the text.  (For example, you may use
\verb+$$\begin{egame}...+ \verb+\end{egame}$$+).  Putting it in the text has
the disadvantage that if it's big and happens to start at the bottom of a page
then you may get a lot of white space if it doesn't fit on the page.
\section{Enhancements}
It would be nice to have a graphical interface, like that in \LaTeX CAD\@. 
I'm not capable of writing one.

It would be nice also if the macros could calculate the dimensions of the
whole game, so the the user does not have to specify them in the
\verb+\begin{game}+ call.  To make this change looks like a tough project to
me.

The label-positioning algorithm could be enhanced, as discussed in
Section~\ref{s:alp}.

\section{History}
\begin{description}
\item[Version 1.0] 2001-4-9 (based on EGAME.STY, 1997-10-16)
\item[Version 1.1] 2004-6-19 (\verb+\egctmarc+ added, following suggestion of
Paul Schweinzer).
\end{description}

\section{Appendix: using pdftricks with pdftex to create a pdf
file}\label{s:pdftricks}
If you want to create a pdf file from your \LaTeX\ file and do not want to
take the dvi $\rightarrow$ ps $\rightarrow$ pdf route, you may use
\verb+pdftricks+.  Eric Gartzke reports that the following input produces the
expected output.  (See his comments for the limitations of this method.)  If
you find a better solution, please let me know.

\begin{verbatim}\documentclass[12pt]{article}
\usepackage[shell]{pdftricks}

\begin{psinputs}
\usepackage{pstricks}
\usepackage{color}
\usepackage{pstcol}
\usepackage{pst-plot}
\usepackage{pst-tree}
\usepackage{pst-eps}
\usepackage{multido}
\usepackage{pst-node}
\usepackage{pst-eps}
\usepackage{egameps}
\end{psinputs}

\usepackage{amsfonts}
\usepackage{graphicx}
\usepackage{amsmath}
\usepackage{setspace}
\usepackage{fullpage}
\usepackage{harvard}
\usepackage{threeparttable}
\usepackage{amssymb}

\newtheorem{hypothesis}{Hypothesis}

Postscript commands need to be separated from other commands.
pdftricks literally stops the compiling of the TeX file to run
postscript/dvi in the background, generating separate small
TeX files and pdf files of each figure.  Notice that egameps 
is called last.  This seems to make a considerable
difference.  Here is an example of a simple egame tree.

\begin{pdfpic}
\begin{figure}[htb]
\hspace*{\fill}
\begin{egame}(600,1250)
\renewcommand{\egarrowstyle}{e}
\renewcommand{\egnode}{\pscircle*{8}}
\putbranch(200,1200)(0,1){150} \ib{$N$}[r]{$~k_{i},~k_{j}$}
\renewcommand{\egarrowstyle}{}
\putbranch(200,1050)(1,1){200} \egdirection{d}
\iib{$i$}[l]{$\begin{array}{c}Status\\Quo\end{array}$}{$Challenge$}[$b_{
i},~1+b_{j}$][]
\putbranch(400,850)(1,1){200}
\iib{$~j$}[r]{$Reject$}{$Accept$}[][$\begin{array}{c}1-
d+b_{i},\\d+b_{j}\end{array}$]
\putbranch(200,650)(1,1){200}
\iib{$i$}[l]{$\begin{array}{c}Back\\Down\end{array}$}{$Persist$}[$\begin
{array}{c}-a_{i},\\1\end{array}$][]
\putbranch(400,450)(1,1){200}
\iib{$j$}[r]{$Fight$}{$Acquiesce$}[$\begin{array}{c}p-k_{i},\\1-p-
k_{j}\end{array}$][$\begin{array}{c}1-d,\\d\end{array}$]
\end{egame}
\setcounter{figure}{1}
\hspace*{\fill} \caption[]{Selection Game
\label{figure:selection}}
\end{figure}
\end{pdfpic}

Again, the figure is isolated from the rest of the TeX file by
pdftricks commands.  Also, the regular LaTeX numbering of 
figures is interfered with, so I adjust with a \setcounter 
command.  Finally, the \label and \reference commands do not 
work, so one must simply number the figures in the text.
\end{verbatim}



\end{document}



