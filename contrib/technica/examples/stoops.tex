\documentclass{book}
\usepackage[pagestyles,outermarks,clearempty]{titlesec}[2005/01/22 v2.6]
\usepackage{titletoc}[2005/01/22 v1.5]
\usepackage[repeat]{drama}
\usepackage{poetry}
\usepackage{example}

\parindent=0pt
\hfuzz 1pt

\TextHeight {6in}
\TextWidth  {27pc}

\frenchspacing

%%%%%%%%%%%%%%%%%%%%%%%%%%%%%%%%%%%%%%%%%%%%%%%%%%%%%%%%%%%%%%%%%%%
%%%%%%%%%%%%%%%%%%%%%%%%%%%%%%%%%%%%%%%%%%%%%%%%%%%%%%%%%%%%%%%%%%%
%%%%%%%%%%%%%%%%%%%%%%%%%%%%%%%%%%%%%%%%%%%%%%%%%%%%%%%%%%%%%%%%%%%

\newpagestyle {MainMatterPage} {

  \sethead   [\oldstylenums{\thepage}]
             [\small She Stoops to Conquer]
             [\firstlinetitlemarks
              Act \MakeUppercase{\romannumeral \chaptertitle}]
             {\firstlinetitlemarks
              Act \MakeUppercase{\romannumeral \chaptertitle}}
             {\small She Stoops to Conquer}
             {\oldstylenums{\thepage}}

  \Capita    {chapter}{section}
}

%%%%%%%%%%%%%%%%%%%%%%%%%%%%%%%%%%%%%%%%%%%%%%%%%%%%%%%%%%%%%%%%%%%
%%%%%%%%%%%%%%%%%%%%%%%%%%%%%%%%%%%%%%%%%%%%%%%%%%%%%%%%%%%%%%%%%%%
%%%%%%%%%%%%%%%%%%%%%%%%%%%%%%%%%%%%%%%%%%%%%%%%%%%%%%%%%%%%%%%%%%%

\Facies \tituli {\textsc{#1}}
\SpatiumSupra   {2ex plus .5ex}
\SpatiumInfra   {1ex plus .25ex}

\Novus \titulus \Titulus
\Facies         {\ifthenelse {\value{page} > 3}{\newpage}{}
                 \spatium* {.15\textheight}\RelSize{+1}#1}
\SpatiumInfra   {2ex plus .5ex minus .5ex}

\Novus \numerus \Nact
\Facies         {\MakeUppercase{\romannumeral#1}}
\Caput          {\chapter}

\Novus \titulus \endAct
\Facies         {\ifthenelse {\value{Nact} = 5}
                             {\textsc{finis}}
                             {\textsc{end of the \ordinal{\value{Nact}}%
                                      \theordinal\ act}
                              \newpage
                              \thispagestyle{empty}
                              \spatium* {4\leading}
                              }}
\SpatiumSupra   {6ex plus 2ex\penalty 10000}
  
\Novus \titulus \Act
\Facies         {\ifthenelse {\value{Nact} > 0}
                             {\endAct\empty}
                             {\spatium {2ex plus 1ex}}%
                 ACT \Nact*{=+1}}
\SpatiumInfra   {1\leading plus 1\leading}

\Forma \personae {\hangafter 1 \hangindent 1em\relax}
\Facies          {\textsc{\MakeLowercase{#1}}.%
                  \hskip .5em plus .25em minus .125em%
                  \\
                  \textsc{\MakeLowercase{#1}}}
\SpatiumSupra    {0ex plus .2ex}

\Facies \[       {\ifthenelse {\isopt r}
                              {[\textit{#1}}
                              {(\textit{#1\/})}%
                 }
\SpatiumAnte     {.33em plus .11em minus .11em}
\SpatiumPost     {.33em plus .11em minus .11em}

\Forma \(        {\centered}
\Facies          {\itshape}
\SpatiumSupra    {.5\leading plus .5\leading minus .125\leading}
\SpatiumInfra    {.5\leading plus .5\leading minus .125\leading}

\Locus \numeri {\rightmargin + 2em\\\rightmargin + 2em}
\Facies        {\number #1}
\Progressio    {{10}}
\Modus         {\pagewise}

\begin{document}

\ExampleTitle {Oliver Goldsmith}{She Stoops to Conquer}
              {Poems and Plays\\[1ex]Dent, 1975}

\pagestyle {empty}

\Titulus {DRAMATIS PERSONAE}

\begin{versus}

 \Novus \textus \actor
 \Locus         {\leftmargin + .5\textwidth}
 \Facies        {#1}

\titulus{men}

Sir Charles Marlow,           \actor{Mr. Gardner}
Young Marlow (his Son),       \actor{Mr. Lee Lewes}
Hardcastle,                   \actor{Mr. Shuter}
Hastings,                     \actor{Mr. Dubellamy}
Tony Lumpkin,                 \actor{Mr. Quick}
Diggory,                      \actor{Mr. Saunders}

\titulus{women}

Mrs. Hardcastle,              \actor{Mrs. Green}
Miss Hardcastle,              \actor{Mrs. Bulkley}
Miss neville,                 \actor{Mrs. Kniveton}
Maid,                         \actor{Miss Williams}

\titulus {\textit{Landlord, Servants, etc. etc.}}
\end{versus}

\newpage
\Titulus{TO SAMUEL JOHNSON, LL.D.}

Dear Sir,
\spatium {.5ex}

By inscribing this slight performance to you, I do not mean
so much to compliment you as myself.  It may do me some honour to
inform the public, that I have lived many years in intimacy with you. 
It may serve the interests of mankind also to inform them, that the
greatest wit may be found in a character, without impairing the most
unaffected piety.

\quad I have, particularly, reason to thank you for your
partiality to this performance. The undertaking a comedy not merely
sentimental was very dangerous; and Mr. Colman, who saw this piece in
its various stages, always thought it so. However, I ventured to trust
it to the public; and, though it was necessarily delayed till late in
the season, I have every reason to be grateful.

\spatium {.5ex}

\begin{versus}
\Forma \strophae {345{1ex}6\\1.5em}
I am, Dear Sir,
Your most sincere friend,
And admirer,
\textsc{oliver goldsmith}
\end{versus}

\Titulus{PROLOGUE,}
\titulus{BY DAVID GARRICK, ESQ.}

\begin{drama}
\begin{versus}
\itshape
\(c)Enter \persona{Mr. Woodward},\\ Dressed in Black, and holding a Handkerchief
to his Eyes\)
\Locus \textus {+4em}
\00 Excuse me, sirs, I pray---I can't yet speak---
I'm crying now---and have been all the week.
\textup{'Tis not alone this mourning suit,} good masters:
\textup{I've that within}---for which there are no plasters!
Pray, would you know the reason why I'm crying?
The Comic Muse, long sick, is now a-dying!
And if she goes, my tears will never stop;
For as a player, I can't squeeze out one drop:
I am undone, that's all---shall lose my bread---
I'd rather, but that's nothing---lose my head.
When the sweet maid is laid upon the bier,
\textup{Shuter} and I shall be chief mourners here.
To \textup{her} a mawkish drab of spurious breed,
Who deals in \textup{sentimentals} will succeed!
Poor \textup{Ned} and \textup{I} are dead to all intents;
We can as soon speak \textup{Greek} as \textup{sentiments}!
Both nervous grown, to keep our spirits up.
We now and then take down a hearty cup.
What shall we do?  If Comedy forsake us,
\textup{They'll turn us out, and no one else will take us.}
But why can't I be moral?---Let me try---
My heart thus pressing---fixed my face and eye---
With a sententious look, that nothing means,
(Faces are blocks in sentimental scenes)
Thus I begin: \textup{All is not gold that glitters},
{\upshape
Pleasure seems sweet, but proves a glass of bitters.
When Ignorance enters, Folly is at hand:
Learning is better far than house and land.
Let not your virtue trip; who trips may stumble,
And virtue is not virtue, if she tumble.
}
I give it up---morals won't do for me;
To make you laugh, I must play tragedy.
One hope remains---hearing the maid was ill,
A \textup{doctor} comes this night to show his skill.
To cheer her heart, and give your muscles motion,
He, in \textup{five draughts} prepar'd, presents a potion:
A kind of magic charm---for be assur'd,
If you will \textup{swallow it}, the maid is cur'd:
But desperate the Doctor, and her case is,
If you reject the dose, and make wry faces!
This truth he boasts, will boast it while he lives,
No \textup{poisonous drugs} are mixed in what he gives.
Should he succeed, you'll give him his degree;
If not, within he will receive no fee!
The College \textup{you}, must his pretensions back,
Pronounce him \textup{regular}, or dub him \textup{quack}.
\end{versus}
\end{drama}

\Drama
\Prosa
\persona*[1]{Sir Charles}
\persona*[2]{Marlow}
\persona*[3]{Hardcastle}
\persona*[4]{Hastings}
\persona*[5]{Tony}
\persona*[6]{Diggory}

\persona*[7]{Mrs. Hardcastle}
\persona*[8]{Miss Hardcastle}
\persona*[9]{Miss Neville}
\persona*[10]{Maid}
\persona*[11]{Landlord}
\persona*[12]{Servant}

\persona*[01]{Scene}


        \Titulus {SHE STOOPS TO CONQUER\\[.5ex]
                  {\RelSize{-1}OR,}\\[.67ex]
                  THE MISTAKES OF A NIGHT}

\spatium* {1\leading}

\thispagestyle {empty}
\pagestyle {MainMatterPage}

\Act


\(\01, A Chamber in an old-fashioned House\)


\(Enter \7 and \persona{Mr. Hardcastle}\)

\numerus{1}

\7  I vow, Mr. Hardcastle, you're very particular.  Is
there a creature in the whole country but ourselves, that does not take
a trip to town now and then, to rub off the rust a little?  There's the
two Miss Hoggs, and our neighbour Mrs. Grigsby, go to take a month's
polishing every winter.

\3  Ay, and bring back vanity and affectation to last them the
whole year.  I wonder why London cannot keep its own fools at home!  In
my time, the follies of the town crept slowly among us, but now they
travel faster than a stage-coach.  Its fopperies come down not only as
inside passengers, but in the very basket.

\7  Ay, your times were fine times indeed; you have been
telling us of them for many a long year.  Here we live in an old
rumbling mansion, that looks for all the world like an inn, but that we
never see company.  Our best visitors are old Mrs. Oddfish, the
curate's wife, and little Cripplegate, the lame dancing-master; and all
our entertainment your old stories of Prince Eugene and the Duke of
Marlborough.  I hate such old-fashioned trumpery.

\3  And I love it.  I love everything that's old: old
friends, old times, old manners, old books, old wine; and I believe,
Dorothy, \[taking her hand\] you'll own I have been pretty fond of an old
wife.

\7  Lord, Mr. Hardcastle, you're for ever at your
Dorothys and your old wifes.  You may be a Darby, but I'll be no Joan,
I promise you.  I'm not so old as you'd make me, by more than one good
year.  Add twenty to twenty, and make money of that.

\3  Let me see; twenty added to twenty makes just fifty and
seven.

\7  It's false, Mr. Hardcastle; I was but twenty when I
was brought to bed of Tony, that I had by Mr. Lumpkin, my first
husband; and he's not come to years of discretion yet.

\3  Nor ever will, I dare answer for him.  Ay, you have
taught him finely.

\7  No matter.  Tony Lumpkin has a good fortune.  My son
is not to live by his learning.  I don't think a boy wants much
learning to spend fifteen hundred a year.

\3  Learning, quotha! a mere composition of tricks and
mischief.

\7  Humour, my dear; nothing but humour.  Come, Mr.
Hardcastle, you must allow the boy a little humour.

\3  I'd sooner allow him a horse-pond.  If burning the
footmen's shoes, frightening the maids, and worrying the kittens be
humour, he has it.  It was but yesterday he fastened my wig to the back
of my chair, and when I went to make a bow, I popt my bald head in Mrs.
Frizzle's face.

\7  And am I to blame?  The poor boy was always too
sickly to do any good.  A school would be his death.  When he comes to
be a little stronger, who knows what a year or two's Latin may do for
him?

\3  Latin for him!  A cat and fiddle.  No, no; the alehouse
and the stable are the only schools he'll ever go to.

\7  Well, we must not snub the poor boy now, for I
believe we shan't have him long among us.  Anybody that looks in his
face may see he's consumptive.

\3  Ay, if growing too fat be one of the symptoms.

\7  He coughs sometimes.

\3  Yes, when his liquor goes the wrong way.

\7  I'm actually afraid of his lungs.

\3  And truly so am I; for he sometimes whoops like a
speaking trumpet---\[ap]\5 hallooing behind the scenes\]---O, there he
goes---a very consumptive figure, truly.


\(Enter \5, crossing the stage.\)


\7  Tony, where are you going, my charmer?  Won't you
give papa and I a little of your company, lovee?

\5  I'm in haste, mother; I cannot stay.

\7  You shan't venture out this raw evening, my dear; you
look most shockingly.

\5  I can't stay, I tell you.  The Three Pigeons expects me down
every moment.  There's some fun going forward.

\3  Ay; the alehouse, the old place: I thought so.


\7  A low, paltry set of fellows.

\5  Not so low, neither.  There's Dick Muggins the exciseman, Jack
Slang the horse doctor, Little Aminadab that grinds the music box, and
Tom Twist that spins the pewter platter.

\7  Pray, my dear, disappoint them for one night at
least.

\5  As for disappointing them, I should not so much mind; but I
can't abide to disappoint myself.

\7  \[Detaining him\]  You shan't go.

\5  I will, I tell you.

\7  I say you shan't.

\5  We'll see which is strongest, you or I.  \[r]Exit, hauling her
out\]

\(\3, Solus\)

\3 Ay, there goes a pair that only spoil each
other.  But is not the whole age in a combination to drive sense and
discretion out of doors?  There's my pretty darling Kate! the fashions
of the times have almost infected her too.  By living a year or two in
town, she is as fond of gauze and French frippery as the best of them.


\(Enter \8\)


\3  Blessings on my pretty innocence! drest out as usual, my
Kate.  Goodness!  What a quantity of superfluous silk hast thou got
about thee, girl!  I could never teach the fools of this age, that the
indigent world could be clothed out of the trimmings of the vain.

\8  You know our agreement, sir.  You allow me the
morning to receive and pay visits, and to dress in my own manner; and
in the evening I put on my housewife's dress to please you.

\3  Well, remember, I insist on the terms of our agreement;
and, by the bye, I believe I shall have occasion to try your obedience
this very evening.

\8  I protest, sir, I don't comprehend your meaning.

\3  Then to be plain with you, Kate, I expect the young
gentleman I have chosen to be your husband from town this very day.  I
have his father's letter, in which he informs me his son is set out,
and that he intends to follow himself shortly after.

\8  Indeed!  I wish I had known something of this
before.  Bless me, how shall I behave?  It's a thousand to one I
shan't like him; our meeting will be so formal, and so like a thing of
business, that I shall find no room for friendship or esteem.

\3  Depend upon it, child, I'll never control your choice; but
Mr. Marlow, whom I have pitched upon, is the son of my old friend, Sir
Charles Marlow, of whom you have heard me talk so often.  The young
gentleman has been bred a scholar, and is designed for an employment in
the service of his country.  I am told he's a man of an excellent
understanding.

\8  Is he?

\3  Very generous.

\8  I believe I shall like him.

\3  Young and brave.

\8  I'm sure I shall like him.

\3  And very handsome.

\8  My dear papa, say no more \[kissing his hand\] he's
mine; I'll have him.

\3  And, to crown all, Kate, he's one of the most bashful and
reserved young fellows in all the world.

\8  Eh! you have frozen me to death again.  That word
\textit{reserved} has undone all the rest of his accomplishments.  A reserved
lover, it is said, always makes a suspicious husband.

\3  On the contrary, modesty seldom resides in a breast that
is not enriched with nobler virtues.  It was the very feature in his
character that first struck me.

\8  He must have more striking features to catch me, I
promise you.  However, if he be so young, so handsome, and so
everything as you mention, I believe he'll do still.  I think I'll have
him.

\3  Ay, Kate, but there is still an obstacle.  It's more than
an even wager he may not have you.

\8  My dear papa, why will you mortify one so?---Well, if
he refuses, instead of breaking my heart at his indifference, I'll only
break my glass for its flattery, set my cap to some newer fashion, and
look out for some less difficult admirer.

\3  Bravely resolved!  In the mean time I'll go prepare the
servants for his reception: as we seldom see company, they want as much
training as a company of recruits the first day's muster.  \[r]Exit\]

    \(\8, Sola\)

\8   Lud, this news of papa's puts me all in a
flutter.  Young, handsome: these he put last; but I put them foremost. 
Sensible, good-natured; I like all that.  But then reserved and
sheepish; that's much against him.  Yet can't he be cured of his
timidity, by being taught to be proud of his wife?  Yes, and can't
I---But I vow I'm disposing of the husband before I have secured the
lover.


\(Enter \9\)


\8  I'm glad you're come, Neville, my dear.  Tell me,
Constance, how do I look this evening?  Is there anything whimsical
about me?  Is it one of my well-looking days, child?  Am I in face
to-day?

\9  Perfectly, my dear.  Yet now I look again---bless
me!---sure no accident has happened among the canary birds or the gold
fishes.  Has your brother or the cat been meddling? or has the last
novel been too moving?

\8  No; nothing of all this.  I have been threatened---I
can scarce get it out---I have been threatened with a lover.

\9  And his name---

\8  Is Marlow.

\9  Indeed!

\8  The son of Sir Charles Marlow.

\9  As I live, the most intimate friend of Mr. Hastings, my
admirer.  They are never asunder.  I believe you must have seen him
when we lived in town.

\8  Never.

\9  He's a very singular character, I assure you.  Among
women of reputation and virtue he is the modestest man alive; but his
acquaintance give him a very different character among creatures of
another stamp: you understand me.

\8  An odd character indeed.  I shall never be able to
manage him.  What shall I do?  Pshaw, think no more of him, but trust
to occurrences for success.  But how goes on your own affair, my dear?
has my mother been courting you for my brother Tony as usual?

\9  I have just come from one of our agreeable
tete-a-tetes.  She has been saying a hundred tender things, and setting
off her pretty monster as the very pink of perfection.

\8  And her partiality is such, that she actually thinks
him so.  A fortune like yours is no small temptation.  Besides, as she
has the sole management of it, I'm not surprised to see her unwilling
to let it go out of the family.

\9  A fortune like mine, which chiefly consists in jewels,
is no such mighty temptation.  But at any rate, if my dear Hastings be
but constant, I make no doubt to be too hard for her at last.  However,
I let her suppose that I am in love with her son; and she never once
dreams that my affections are fixed upon another.

\8  My good brother holds out stoutly.  I could almost
love him for hating you so.

\9  It is a good-natured creature at bottom, and I'm sure
would wish to see me married to anybody but himself.  But my aunt's
bell rings for our afternoon's walk round the improvements.  Allons! 
Courage is necessary, as our affairs are critical.

\8 Would it were bed-time, and all were well. 

\[r]Exeunt\]

\(j)\01, An Alehouse Room.  Several shabby Fellows with punch and
tobacco.  \5 at the head of the table, a little higher than the
rest, a mallet in his hand.\)


\persona{Omnes}  Hurrea! hurrea! hurrea! bravo!

\persona{First Fellow}  Now, gentlemen, silence for a song.  The 'squire is
going to knock himself down for a song.

\persona{Omnes}  Ay, a song, a song!

\5  Then I'll sing you, gentlemen, a song I made upon this
alehouse, the Three Pigeons.

\SpatiumSupra \titulum {2ex plus 1ex\penalty -100}
\SpatiumInfra          {2ex plus 1ex\penalty 10000}
\titulus{\textsc{Song}}

\begin{versus}
\Forma \strophae {010101016}
\SpatiumInfra {1.5ex plus .5ex\penalty -100}
\Locus \textus {+4em}
\itshape

Let schoolmasters puzzle their brain
     With grammar, and nonsense, and learning,
Good liquor, I stoutly maintain,
     Gives \textup{genus} a better discerning.
Let them brag of their heathenish gods,
     Their Lethes, their Styxes, and Stygians,
Their Quis, and their Quaes, and their Quods,
     They're all but a parcel of Pigeons.
          \textup{Toroddle, toroddle, toroll.}

When methodist preachers come down,
     A-preaching that drinking is sinful,
I'll wager the rascals a crown,
     They always preach best with a skinful.
But when you come down with your pence,
     For a slice of their scurvy religion,
I'll leave it to all men of sense,
     But you, my good friend, are the Pigeon.
          \textup{Toroddle, toroddle, toroll.}

Then come, put the jorum about,
     And let us be merry and clever,
Our hearts and our liquors are stout,
     Here's the Three Jolly Pigeons for ever.
Let some cry up woodcock or hare,
     Your bustards, your ducks, and your widgeons;
But of all the \textup{gay} birds in the air,
     Here's a health to the Three Jolly Pigeons.
          \textup{Toroddle, toroddle, toroll.}

\end{versus}

\persona{Omnes}  Bravo, bravo!

\persona{First Fellow}  The 'squire has got spunk in him.

\persona{Second Fellow}  I loves to hear him sing, bekeays he never gives us
nothing that's low.

\persona{Third Fellow}  O damn anything that's low, I cannot bear it.

\persona{Fourth Fellow}  The genteel thing is the genteel thing any time: if so
be that a gentleman bees in a concatenation accordingly.

\persona{Third Fellow}  I likes the maxum of it, Master Muggins.  What, though I
am obligated to dance a bear, a man may be a gentleman for all that. 
May this be my poison, if my bear ever dances but to the very
genteelest of tunes; ``Water Parted,'' or ``The minuet in Ariadne.''

\persona{Second Fellow}  What a pity it is the 'squire is not come to his own. 
It would be well for all the publicans within ten miles round of him.

\5  Ecod, and so it would, Master Slang.  I'd then show what it was
to keep choice of company.

\persona{Second Fellow}  O he takes after his own father for that.  To be sure
old 'Squire Lumpkin was the finest gentleman I ever set my eyes on. 
For winding the straight horn, or beating a thicket for a hare, or a
wench, he never had his fellow.  It was a saying in the place, that he
kept the best horses, dogs, and girls, in the whole county.

\5  Ecod, and when I'm of age, I'll be no bastard, I promise you.  I
have been thinking of Bet Bouncer and the miller's grey mare to begin
with.  But come, my boys, drink about and be merry, for you pay no
reckoning.  Well, Stingo, what's the matter?


\(Enter \11.\)


\11  There be two gentlemen in a post-chaise at the door.  They
have lost their way upo' the forest; and they are talking something
about Mr. Hardcastle.

\5  As sure as can be, one of them must be the gentleman that's
coming down to court my sister.  Do they seem to be Londoners?

\11  I believe they may.  They look woundily like Frenchmen.

\5  Then desire them to step this way, and I'll set them right in a
twinkling. \[Exit \11\]
Gentlemen, as they mayn't be good enough
company for you, step down for a moment, and I'll be with you in the
squeezing of a lemon.  \[r]Exeunt mob\]

    \(\5 solus\)
\5    Father-in-law has been calling me whelp and hound this
half year.  Now, if I pleased, I could be so revenged upon the old
grumbletonian.  But then I'm afraid---afraid of what?  I shall soon be
worth fifteen hundred a year, and let him frighten me out of \textit{that}
if he can.


\(Enter \11, conducting \2 and \4\)


\2  What a tedious uncomfortable day have we had of it!  We were
told it was but forty miles across the country, and we have come above
threescore.

\4  And all, Marlow, from that unaccountable reserve of yours,
that would not let us inquire more frequently on the way.

\2  I own, Hastings, I am unwilling to lay myself under an
obligation to every one I meet, and often stand the chance of an
unmannerly answer.

\4  At present, however, we are not likely to receive any
answer.

\5  No offence, gentlemen.  But I'm told you have been inquiring for
one Mr. Hardcastle in these parts.  Do you know what part of the
country you are in?

\4  Not in the least, sir, but should thank you for
information.

\5  Nor the way you came?

\4  No, sir: but if you can inform us------

\5  Why, gentlemen, if you know neither the road you are going, nor
where you are, nor the road you came, the first thing I have to inform
you is, that---you have lost your way.

\2  We wanted no ghost to tell us that.

\5  Pray, gentlemen, may I be so bold so as to ask the place from
whence you came?

\2  That's not necessary towards directing us where we are to go.

\5  No offence; but question for question is all fair, you know. 
Pray, gentlemen, is not this same Hardcastle a cross-grained,
old-fashioned, whimsical fellow, with an ugly face, a daughter, and a
pretty son?

\4  We have not seen the gentleman; but he has the family you
mention.

\5  The daughter, a tall, trapesing, trolloping, talkative maypole;
the son, a pretty, well-bred, agreeable youth, that everybody is fond
of.

\2  Our information differs in this.  The daughter is said to be
well-bred and beautiful; the son an awkward booby, reared up and
spoiled at his mother's apron-string.

\5  He-he-hem!---Then, gentlemen, all I have to tell you is, that you
won't reach Mr. Hardcastle's house this night, I believe.

\4  Unfortunate!

\5  It's a damn'd long, dark, boggy, dirty, dangerous way.  Stingo,
tell the gentlemen the way to Mr. Hardcastle's!  \[winking upon the
\11\]  Mr. Hardcastle's, of Quagmire Marsh, you understand me.

\11 Master Hardcastle's!  Lock-a-daisy, my masters, you're come
a deadly deal wrong!  When you came to the bottom of the hill, you
should have crossed down Squash Lane.

\2  Cross down Squash Lane!

\11  Then you were to keep straight forward, till you came to
four roads.

\2  Come to where four roads meet?

\5  Ay; but you must be sure to take only one of them.

\2  O, sir, you're facetious.

\5  Then keeping to the right, you are to go sideways till you come
upon Crackskull Common: there you must look sharp for the track of the
wheel, and go forward till you come to farmer Murrain's barn.  Coming
to the farmer's barn, you are to turn to the right, and then to the
left, and then to the right about again, till you find out the old
mill---

\2  Zounds, man! we could as soon find out the longitude!

\4  What's to be done, Marlow?

\2  This house promises but a poor reception; though perhaps the
landlord can accommodate us.

\11  Alack, master, we have but one spare bed in the whole
house.

\5  And to my knowledge, that's taken up by three lodgers already. 
\[after a pause, in which the rest seem disconcerted\] I have hit it. 
Don't you think, Stingo, our landlady could accommodate the gentlemen
by the fire-side, with------three chairs and a bolster?

\4  I hate sleeping by the fire-side.

\2  And I detest your three chairs and a bolster.

\5  You do, do you? then, let me see---what if you go on a mile
further, to the Buck's Head; the old Buck's Head on the hill, one of
the best inns in the whole county?

\4  O ho! so we have escaped an adventure for this night,
however.

\11 \[apart to \5\]  Sure, you ben't sending them to your
father's as an inn, be you?

\5  Mum, you fool you.  Let \textit{them} find that out. \[to them\]  You
have only to keep on straight forward, till you come to a large old
house by the road side.  You'll see a pair of large horns over the
door.  That's the sign.  Drive up the yard, and call stoutly about you.

\4  Sir, we are obliged to you.  The servants can't miss the
way?

\5  No, no: but I tell you, though, the landlord is rich, and going
to leave off business; so he wants to be thought a gentleman, saving
your presence, he! he! he!  He'll be for giving you his company; and,
ecod, if you mind him, he'll persuade you that his mother was an
alderman, and his aunt a justice of peace.

\11  A troublesome old blade, to be sure; but a keeps as good
wines and beds as any in the whole country.

\2  Well, if he supplies us with these, we shall want no farther
connexion.  We are to turn to the right, did you say?

\5  No, no; straight forward.  I'll just step myself, and show you a
piece of the way.  \[to the \11\]  Mum!

\11 Ah, bless your heart, for a sweet, pleasant---damn'd
mischie\-vous  son of a whore. \[r]Exeunt\]



\Act


\(\01, An old-fashioned House.\)


\(Enter \3, followed by three or four awkward Servants.\)


\3  Well, I hope you are perfect in the table exercise I have
been teaching you these three days.  You all know your posts and your
places, and can show that you have been used to good company, without
ever stirring from home.

\persona{Omnes}  Ay, ay.

\3  When company comes you are not to pop out and stare, and
then run in again, like frightened rabbits in a warren.

\persona{Omnes}  No, no.

\3  You, Diggory, whom I have taken from the barn, are to make
a show at the side-table; and you, Roger, whom I have advanced from the
plough, are to place yourself behind my chair.  But you're not to stand
so, with your hands in your pockets.  Take your hands from your
pockets, Roger; and from your head, you blockhead you.  See how Diggory
carries his hands.  They're a little too stiff, indeed, but that's no
great matter.

\6  Ay, mind how I hold them.  I learned to hold my hands this
way when I was upon drill for the militia.  And so being upon drill------

\3  You must not be so talkative, Diggory.  You must be all
attention to the guests.  You must hear us talk, and not think of
talking; you must see us drink, and not think of drinking; you must see
us eat, and not think of eating.

\6  By the laws, your worship, that's parfectly unpossible. 
Whenever Diggory sees yeating going forward, ecod, he's always wishing
for a mouthful himself.

\3  Blockhead!  Is not a belly-full in the kitchen as good as
a belly-full in the parlour?  Stay your stomach with that reflection.

\6  Ecod, I thank your worship, I'll make a shift to stay my
stomach with a slice of cold beef in the pantry.

\3  Diggory, you are too talkative.---Then, if I happen to say
a good thing, or tell a good story at table, you must not all burst out
a-laughing, as if you made part of the company.

\6  Then ecod your worship must not tell the story of Ould
Grouse in the gun-room: I can't help laughing at that---he! he!
he!---for the soul of me.  We have laughed at that these twenty
years---ha! ha! ha!

\3  Ha! ha! ha!  The story is a good one.  Well, honest
Diggory, you may laugh at that---but still remember to be attentive. 
Suppose one of the company should call for a glass of wine, how will
you behave?  A glass of wine, sir, if you please \[to \6\].---Eh, why
don't you move?

\6  Ecod, your worship, I never have courage till I see the
eatables and drinkables brought upo' the table, and then I'm as bauld
as a lion.

\3  What, will nobody move?

\persona{First Servant}  I'm not to leave this pleace.

\persona{Second Servant}  I'm sure it's no pleace of mine.

\persona{Third Servant}  Nor mine, for sartain.

\6  Wauns, and I'm sure it canna be mine.

\3  You numskulls! and so while, like your betters, you are
quarrelling for places, the guests must be starved.  O you dunces!  I
find I must begin all over again------But don't I hear a coach drive into
the yard?  To your posts, you blockheads.  I'll go in the mean time and
give my old friend's son a hearty reception at the gate.

\[r]Exit \3\]

\6  By the elevens, my pleace is gone quite out of my head.

\persona{Roger}  I know that my pleace is to be everywhere.

\persona{First Servant}  Where the devil is mine?

\persona{Second Servant}  My pleace is to be nowhere at all; and so I'ze go
about my business.

\[r]Exeunt Servants, running about as if frightened, different ways\]


\(Enter \12 with candles, showing in \2 and \4\)


\12  Welcome, gentlemen, very welcome!  This way.

\4  After the disappointments of the day, welcome once more,
Charles, to the comforts of a clean room and a good fire.  Upon my
word, a very well-looking house; antique but creditable.

\2  The usual fate of a large mansion.  Having first ruined the
master by good housekeeping, it at last comes to levy contributions as
an inn.

\4  As you say, we passengers are to be taxed to pay all these
fineries.  I have often seen a good sideboard, or a marble
chimney-piece, though not actually put in the bill, inflame a
reckoning confoundedly.

\2  Travellers, George, must pay in all places: the only
difference is, that in good inns you pay dearly for luxuries; in bad
inns you are fleeced and starved.

\4  You have lived very much among them.  In truth, I have been
often surprised, that you who have seen so much of the world, with your
natural good sense, and your many opportunities, could never yet
acquire a requisite share of assurance.

\2  The Englishman's malady.  But tell me, George, where could I
have learned that assurance you talk of?  My life has been chiefly
spent in a college or an inn, in seclusion from that lovely part of the
creation that chiefly teach men confidence.  I don't know that I was
ever familiarly acquainted with a single modest woman---except my
mother---But among females of another class, you know------

\4  Ay, among them you are impudent enough of all conscience.

\2  They are of \textit{us}, you know.

\4  But in the company of women of reputation I never saw such
an idiot, such a trembler; you look for all the world as if you wanted
an opportunity of stealing out of the room.

\2  Why, man, that's because I do want to steal out of the room. 
Faith, I have often formed a resolution to break the ice, and rattle
away at any rate.  But I don't know how, a single glance from a pair of
fine eyes has totally overset my resolution.  An impudent fellow may
counterfeit modesty; but I'll be hanged if a modest man can ever
counterfeit impudence.

\4  If you could but say half the fine things to them that I
have heard you lavish upon the bar-maid of an inn, or even a college
bed-maker------

\2  Why, George, I can't say fine things to them; they freeze,
they petrify me.  They may talk of a comet, or a burning mountain, or
some such bagatelle; but, to me, a modest woman, drest out in all her
finery, is the most tremendous object of the whole creation.

\4  Ha! ha! ha!  At this rate, man, how can you ever expect to
marry?

\2  Never; unless, as among kings and princes, my bride were to be
courted by proxy.  If, indeed, like an Eastern bridegroom, one were to
be introduced to a wife he never saw before, it might be endured.  But
to go through all the terrors of a formal courtship, together with the
episode of aunts, grandmothers, and cousins, and at last to blurt out
the broad staring question of, Madam, will you marry me?  No, no,
that's a strain much above me, I assure you.

\4  I pity you.  But how do you intend behaving to the lady you
are come down to visit at the request of your father?

\2  As I behave to all other ladies.  Bow very low, answer yes or
no to all her demands---But for the rest, I don't think I shall venture
to look in her face till I see my father's again.

\4  I'm surprised that one who is so warm a friend can be so
cool a lover.

\2  To be explicit, my dear Hastings, my chief inducement down was
to be instrumental in forwarding your happiness, not my own.  Miss
Neville loves you, the family don't know you; as my friend you are sure
of a reception, and let honour do the rest.

\4  My dear Marlow!  But I'll suppress the emotion.  Were I a
wretch, meanly seeking to carry off a fortune, you should be the last
man in the world I would apply to for assistance.  But Miss Neville's
person is all I ask, and that is mine, both from her deceased father's
consent, and her own inclination.

\2  Happy man!  You have talents and art to captivate any woman. 
I'm doom'd to adore the sex, and yet to converse with the only part of
it I despise.  This stammer in my address, and this awkward
prepossessing visage of mine, can never permit me to soar above the
reach of a milliner's 'prentice, or one of the duchesses of Drury-lane. 
Pshaw! this fellow here to interrupt us.


\(Enter \3\)


\3  Gentlemen, once more you are heartily welcome.  Which is
Mr. Marlow?  Sir, you are heartily welcome.  It's not my way, you see,
to receive my friends with my back to the fire.  I like give them a
hearty reception in the old style at my gate.  I like to see their
horses and trunks taken care of.

\2  \[Aside\]  He has got our names from the servants already. \[To him\]
We approve your caution and hospitality, sir.  \[To \4\]  I
have been thinking, George, of changing our travelling dresses in the
morning.  I am grown confoundedly ashamed of mine.

\3  I beg, Mr. Marlow, you'll use no ceremony in this house.

\4  I fancy, Charles, you're right: the first blow is half the
battle.  I intend opening the campaign with the white and gold.

\3  Mr. Marlow---Mr. Hastings---gentlemen---pray be under no
constraint in this house.  This is Liberty-hall, gentlemen.  You may do
just as you please here.

\2  Yet, George, if we open the campaign too fiercely at first, we
may want ammunition before it is over.  I think to reserve the
embroidery to secure a retreat.

\3  Your talking of a retreat, Mr. Marlow, puts me in mind of
the Duke of Marlborough, when we went to besiege Denain.  He first
summoned the garrison------

\2  Don't you think the \textit{ventre d'or} waistcoat will do with the
plain brown?

\3  He first summoned the garrison, which might consist of
about five thousand men------

\4  I think not: brown and yellow mix but very poorly.

\3  I say, gentlemen, as I was telling you, be summoned the
garrison, which might consist of about five thousand men------

\2  The girls like finery.

\3  Which might consist of about five thousand men, well
appointed with stores, ammunition, and other implements of war.  Now,
says the Duke of Marlborough to George Brooks, that stood next to
him---you must have heard of George Brooks---I'll pawn my dukedom, says
he, but I take that garrison without spilling a drop of blood.  So------

\2  What, my good friend, if you gave us a glass of punch in the
mean time; it would help us to carry on the siege with vigour.

\3  Punch, sir!  \[Aside\]  This is the most unaccountable kind
of modesty I ever met with.

\2  Yes, sir, punch.  A glass of warm punch, after our journey,
will be comfortable.  This is Liberty-hall, you know.

\3  Here's a cup, sir.

\2  \[Aside\]  So this fellow, in his Liberty-hall, will only let
us have just what he pleases.

\3  \[Taking the cup\]  I hope you'll find it to your mind.  I
have prepared it with my own hands, and I believe you'll own the
ingredients are tolerable.  Will you be so good as to pledge me, sir? 
Here, Mr. Marlow, here is to our better acquaintance.  \[drinks\]

\2  \[Aside\]  A very impudent fellow this! but he's a character,
and I'll humour him a little.  Sir, my service to you.  \[drinks\]

\4  \[Aside\]  I see this fellow wants to give us his company,
and forgets that he's an innkeeper, before he has learned to be a
gentleman.

\2  From the excellence of your cup, my old friend, I suppose you
have a good deal of business in this part of the country.  Warm work,
now and then, at elections, I suppose.

\3  No, sir, I have long given that work over.  Since our
betters have hit upon the expedient of electing each other, there is no
business \textit{for us that sell ale}.

\4  So, then, you have no turn for politics, I find.

\3  Not in the least.  There was a time, indeed, I fretted
myself about the mistakes of government, like other people; but finding
myself every day grow more angry, and the government growing no better,
I left it to mend itself.  Since that, I no more trouble my head about
\textit{Hyder Ally}, or \textit{Ally Cawn}, than about \textit{Ally
Croker}.  Sir, my service to you.

\4  So that with eating above stairs, and drinking below, with
receiving your friends within, and amusing them without, you lead a
good pleasant bustling life of it.

\3  I do stir about a great deal, that's certain.  Half the
differences of the parish are adjusted in this very parlour.

\2  \[After drinking\]  And you have an argument in your cup, old
gentleman, better than any in Westminster-hall.

\3  Ay, young gentleman, that, and a little philosophy.

\2  \[Aside\]  Well, this is the first time I ever heard of an
innkeeper's philosophy.

\4  So then, like an experienced general, you attack them on
every quarter.  If you find their reason manageable, you attack it with
your philosophy; if you find they have no reason, you attack them with
this.  Here's your health, my philosopher.\[drinks\]

\3  Good, very good, thank you; ha! ha!  Your generalship puts
me in mind of Prince Eugene, when he fought the Turks at the battle of
Belgrade.  You shall hear.

\2  Instead of the battle of Belgrade, I believe it's almost time
to talk about supper.  What has your philosophy got in the house for
supper?

\3  For supper, sir!  \[Aside\]  Was ever such a request to a
man in his own house?

\2  Yes, sir, supper, sir; I begin to feel an appetite.  I shall
make devilish work to-night in the larder, I promise you.

\3  \[Aside\]  Such a brazen dog sure never my eyes beheld. 
\[To Him\]  Why, really, sir, as for supper I can't well tell.  My
Dorothy and the cook-maid settle these things between them.  I leave
these kind of things entirely to them.

\2  You do, do you?

\3  Entirely.  By the bye, I believe they are in actual
consultation upon what's for supper this moment in the kitchen.

\2  Then I beg they'll admit me as one of their privy council. 
It's a way I have got.  When I travel, I always chose to regulate my
own supper.  Let the cook be called.  No offence I hope, sir.

\3  O no, sir, none in the least; yet I don't know how; our
Bridget, the cook-maid, is not very communicative upon these
occasions.  Should we send for her, she might scold us all out of the
house.

\4  Let's see your list of the larder then.  I ask it as a
favour.  I always match my appetite to my bill of fare.

\2  \[To \3, who looks at them with surprise\]  Sir, he's
very right, and it's my way too.

\3  Sir, you have a right to command here.  Here, Roger,
bring us the bill of fare for to-night's supper: I believe it's drawn
out---Your manner, Mr. Hastings, puts me in mind of my uncle, Colonel
Wallop.  It was a saying of his, that no man was sure of his supper
till he had eaten it.

\4  \[Aside\]  All upon the high rope!  His uncle a colonel! we
shall soon hear of his mother being a justice of the peace.  But let's
hear the bill of fare.

\2  \[Perusing\]  What's here?  For the first course; for the
second course; for the dessert.  The devil, sir, do you think we have
brought down a whole Joiners' Company, or the corporation of Bedford,
to eat up such a supper?  Two or three little things, clean and
comfortable, will do.

\4  But let's hear it.

\2  \[Reading\]  For the first course, at the top, a pig and prune
sauce.

\4  Damn your pig, I say.

\2  And damn your prune sauce, say I.

\3  And yet, gentlemen, to men that are hungry, pig with
prune sauce is very good eating.

\2  At the bottom, a calf's tongue and brains.

\4  Let your brains be knocked out, my good sir, I don't like
them.

\2  Or you may clap them on a plate by themselves.  I do.

\3  \[Aside\]  Their impudence confounds me.  \[To them\] 
Gentlemen, you are my guests, make what alterations you please.  Is
there anything else you wish to retrench or alter, gentlemen?

\2  Item, a pork pie, a boiled rabbit and sausages, a Florentine,
a shaking pudding, and a dish of tiff---taff---taffety cream.

\4  Confound your made dishes; I shall be as much at a loss in
this house as at a green and yellow dinner at the French ambassador's
table.  I'm for plain eating.

\3  I'm sorry, gentlemen, that I have nothing you like, but if
there be anything you have a particular fancy to------

\2  Why, really, sir, your bill of fare is so exquisite, that any
one part of it is full as good as another.  Send us what you please. 
So much for supper.  And now to see that our beds are aired, and
properly taken care of.

\3  I entreat you'll leave that to me.  You shall not stir a
step.

\2  Leave that to you!  I protest, sir, you must excuse me, I
always look to these things myself.

\3  I must insist, sir, you'll make yourself easy on that
head.

\2  You see I'm resolved on it.  \[Aside\]  A very troublesome
fellow this, as I ever met with.

\3  Well, sir, I'm resolved at least to attend you.  \[Aside\] 
This may be modem modesty, but I never saw anything look so like
old-fashioned impudence.  \[r]Exeunt \2 and \3\]

\4  \[Alone\]  So I find this fellow's civilities begin to grow
troublesome.  But who can be angry at those assiduities which are meant
to please him?  Ha! what do I see?  Miss Neville, by all that's happy!


\(Enter \9\)


\9  My dear Hastings!  To what unexpected good fortune, to
what accident, am I to ascribe this happy meeting?

\4  Rather let me ask the same question, as I could never have
hoped to meet my dearest Constance at an inn.

\9  An inn! sure you mistake: my aunt, my guardian, lives
here.  What could induce you to think this house an inn?

\4  My friend, Mr. Marlow, with whom I came down, and I, have
been sent here as to an inn, I assure you.  A young fellow, whom we
accidentally met at a house hard by, directed us hither.

\9  Certainly it must be one of my hopeful cousin's tricks,
of whom you have heard me talk so often; ha! ha! ha!

\4  He whom your aunt intends for you? he of whom I have such
just apprehensions?

\9  You have nothing to fear from him, I assure you.  You'd
adore him, if you knew how heartily he despises me.  My aunt knows it
too, and has undertaken to court me for him, and actually begins to
think she has made a conquest.

\4  Thou dear dissembler!  You must know, my Constance, I have
just seized this happy opportunity of my friend's visit here to get
admittance into the family.  The horses that carried us down are now
fatigued with their journey, but they'll soon be refreshed; and then,
if my dearest girl will trust in her faithful Hastings, we shall soon
be landed in France, where even among slaves the laws of marriage are
respected.

\9  I have often told you, that though ready to obey you, I
yet should leave my little fortune behind with reluctance.  The
greatest part of it was left me by my uncle, the India director, and
chiefly consists in jewels.  I have been for some time persuading my
aunt to let me wear them.  I fancy I'm very near succeeding.  The
instant they are put into my possession, you shall find me ready to
make them and myself yours.

\4  Perish the baubles!  Your person is all I desire.  In the
mean time, my friend Marlow must not be let into his mistake.  I know
the strange reserve of his temper is such, that if abruptly informed of
it, he would instantly quit the house before our plan was ripe for
execution.

\9  But how shall we keep him in the deception?  Miss
Hardcastle is just returned from walking; what if we still continue to
deceive him?------This, this way------\[r]They confer\]


\(Enter \2\)


\2  The assiduities of these good people teaze me beyond bearing. 
My host seems to think it ill manners to leave me alone, and so he
claps not only himself, but his old-fashioned wife, on my back.  They
talk of coming to sup with us too; and then, I suppose, we are to run
the gantlet through all the rest of the family.---What have we got here?

\4  My dear Charles!  Let me congratulate you!---The most
fortunate accident!---Who do you think is just alighted?

\2  Cannot guess.

\4  Our mistresses, boy, Miss Hardcastle and Miss Neville. 
Give me leave to introduce Miss Constance Neville to your
acquaintance.  Happening to dine in the neighbourhood, they called on
their return to take fresh horses here.  Miss Hardcastle has just stept
into the next room, and will be back in an instant.  Wasn't it lucky? 
eh!

\2  \[Aside\]  I have been mortified enough of all conscience, and
here comes something to complete my embarrassment.

\4  Well, but wasn't it the most fortunate thing in the world?

\2  Oh! yes.  Very fortunate---a most joyful encounter---But our
dresses, George, you know are in disorder---What if we should postpone
the happiness till to-morrow?---To-morrow at her own house---It will be
every bit as convenient---and rather more re\-spec\-tful---To-morrow let it
be.  \[r]Offering to go\]

\9  By no means, sir.  Your ceremony will displease her. 
The disorder of your dress will show the ardour of your impatience. 
Besides, she knows you are in the house, and will permit you to see
her.

\2  O! the devil! how shall I support it?  Hem! hem!  Hastings,
you must not go.  You are to assist me, you know.  I shall be
confoundedly ridiculous.  Yet, hang it!  I'll take courage.  Hem!

\4  Pshaw, man! it's but the first plunge, and all's over. 
She's but a woman, you know.

\2  And, of all women, she that I dread most to encounter.


\(Enter \8, as returned from walking, a bonnet, etc.\)


\4  \[introducing them\]  Miss Hardcastle, Mr. Marlow.  I'm
proud of bringing two persons of such merit together, that only want to
know, to esteem each other.

\8  \[Aside\]  Now for meeting my modest gentleman with a
demure face, and quite in his own manner.  \[After a pause, in which he
appears very uneasy and disconcerted\]  I'm glad of your safe arrival,
sir.  I'm told you had some accidents by the way.

\2  Only a few, madam.  Yes, we had some.  Yes, madam, a good many
accidents, but should be sorry---madam---or rather glad of any
accidents---that are so agreeably concluded.  Hem!

\4  \[To him\]  You never spoke better in your whole life.  Keep
it up, and I'll insure you the victory.

\8  I'm afraid you flatter, sir.  You that have seen so
much of the finest company, can find little entertainment in an obscure
corner of the country.

\2  \[Gathering courage\]  I have lived, indeed, in the world,
madam; but I have kept very little company.  I have been but an
observer upon life, madam, while others were enjoying it.

\9  But that, I am told, is the way to enjoy it at last.

\4  \[To him\]  Cicero never spoke better.  Once more, and you
are confirmed in assurance for ever.

\2  \[To him\]  Hem!  Stand by me, then, and when I'm down, throw
in a word or two, to set me up again.

\8  An observer, like you, upon life were, I fear,
disagreeably employed, since you must have had much more to censure
than to approve.

\2  Pardon me, madam.  I was always willing to be amused.  The
folly of most people is rather an object of mirth than uneasiness.

\4  \[To him\]  Bravo, bravo.  Never spoke so well in your whole
life.  Well, Miss Hardcastle, I see that you and Mr. Marlow are going
to be very good company.  I believe our being here will but embarrass
the interview.

\2  Not in the least, Mr. Hastings.  We like your company of all
things.  \[To him\]  Zounds!  George, sure you won't go? how can you
leave us?

\4  Our presence will but spoil conversation, so we'll retire to
the next room.  \[To him\]  You don't consider, man, that we are to
manage a little tete-a-tete of our own.  \[r]Exeunt\]

\8  \[after a pause\].  But you have not been wholly an
observer, I presume, sir: the ladies, I should hope, have employed some
part of your addresses.

\2  \[Relapsing into timidity\]  Pardon me, madam, I---I---I---as yet
have studied---only---to---deserve them.

\8  And that, some say, is the very worst way to obtain them.

\2  Perhaps so, madam.  But I love to converse only with the more
grave and sensible part of the sex.  But I'm afraid I grow tiresome.

\8  Not at all, sir; there is nothing I like so much as
grave conversation myself; I could hear it for ever.  Indeed, I have
often been surprised how a man of sentiment could ever admire those
light airy pleasures, where nothing reaches the heart.

\2  It's------a disease------of the mind, madam.  In the variety of
tastes there must be some who, wanting a relish------for------um---a---um.

\8  I understand you, sir.  There must be some, who,
wanting a relish for refined pleasures, pretend to despise what they
are incapable of tasting.

\2  My meaning, madam, but infinitely better expressed.  And I
can't help observing------a------

\8  \[Aside\]  Who could ever suppose this fellow
impudent upon some occasions?  \[To him\]  You were going to observe,
sir------

\2  I was observing, madam---I protest, madam, I forget what I was
going to observe.

\8  \[Aside\]  I vow and so do I.  \[To him\]  You were
observing, sir, that in this age of hypocrisy---something about
hypocrisy, sir.

\2  Yes, madam.  In this age of hypocrisy there are few who upon
strict inquiry do not---a---a---a---

\8  I understand you perfectly, sir.

\2  \[Aside\]  Egad! and that's more than I do myself.

\8  You mean that in this hypocritical age there are few
that do not condemn in public what they practise in private, and think
they pay every debt to virtue when they praise it.

\2  True, madam; those who have most virtue in their mouths, have
least of it in their bosoms.  But I'm sure I tire you, madam.

\8  Not in the least, sir; there's something so
agreeable and spirited in your manner, such life and force---pray, sir,
go on.

\2  Yes, madam.  I was saying------that there are some occasions,
when a total want of courage, madam, destroys all the------and puts
us------upon a---a---a---

\8  I agree with you entirely; a want of courage upon
some occasions assumes the appearance of ignorance, and betrays us when
we most want to excel.  I beg you'll proceed.

\2  Yes, madam.  Morally speaking, madam---But I see Miss Neville
expecting us in the next room.  I would not intrude for the world.

\8  I protest, sir, I never was more agreeably
entertained in all my life.  Pray go on.

\2  Yes, madam, I was------But she beckons us to join her.  Madam,
shall I do myself the honour to attend you?

\8  Well, then, I'll follow.

\2  \[Aside\]  This pretty smooth dialogue has done for me. 
\[r]Exit\]

\8  \[Alone\]  Ha! ha! ha!  Was there ever such a sober,
sentimental interview?  I'm certain he scarce looked in my face the
whole time.  Yet the fellow, but for his unaccountable bashfulness, is
pretty well too.  He has good sense, but then so buried in his fears,
that it fatigues one more than ignorance.  If I could teach him a
little confidence, it would be doing somebody that I know of a piece of
service.  But who is that somebody?---That, faith, is a question I can
scarce answer.  \[r]Exit\]


\(Enter \5 and \9, followed by \7 and \4\)


\5  What do you follow me for, cousin Con?  I wonder you're not
ashamed to be so very engaging.

\9  I hope, cousin, one may speak to one's own relations,
and not be to blame.

\5  Ay, but I know what sort of a relation you want to make me,
though; but it won't do.  I tell you, cousin Con, it won't do; so I beg
you'll keep your distance, I want no nearer relationship.

\[r]She follows, coquetting him to the back scene\]

\7  Well!  I vow, Mr. Hastings, you are very
entertaining.  There's nothing in the world I love to talk of so much
as London, and the fashions, though I was never there myself.

\4  Never there!  You amaze me!  From your air and manner, I
concluded you had been bred all your life either at Ranelagh, St. 
James's, or Tower Wharf.

\7  O! sir, you're only pleased to say so.  We country
persons can have no manner at all.  I'm in love with the town, and that
serves to raise me above some of our neighbouring rustics; but who can
have a manner, that has never seen the Pantheon, the Grotto Gardens,
the Borough, and such places where the nobility chiefly resort?  All I
can do is to enjoy London at second-hand.  I take care to know every
tete-a-tete from the Scandalous Magazine, and have all the fashions, as
they come out, in a letter from the two Miss Rickets of Crooked Lane. 
Pray how do you like this head, Mr. Hastings?

\4  Extremely elegant and degagee, upon my word, madam.  Your
friseur is a Frenchman, I suppose?

\7  I protest, I dressed it myself from a print in the
Ladies' Memorandum-book for the last year.

\4  Indeed!  Such a head in a side-box at the play-house would
draw as many gazers as my Lady Mayoress at a City Ball.

\7  I vow, since inoculation began, there is no such
thing to be seen as a plain woman; so one must dress a little
particular, or one may escape in the crowd.

\4  But that can never be your case, madam, in any dress. 
\[Bowing\]

\7  Yet, what signifies my dressing when I have such a
piece of antiquity by my side as Mr. Hardcastle: all I can say will
never argue down a single button from his clothes.  I have often wanted
him to throw off his great flaxen wig, and where he was bald, to
plaster it over, like my Lord Pately, with powder.

\4  You are right, madam; for, as among the ladies there are
none ugly, so among the men there are none old.

\7  But what do you think his answer was?  Why, with his
usual Gothic vivacity, he said I only wanted him to throw off his wig,
to convert it into a tete for my own wearing.

\4  Intolerable!  At your age you may wear what you please, and
it must become you.

\7  Pray, Mr. Hastings, what do you take to be the most
fashionable age about town?

\4  Some time ago, forty was all the mode; but I'm told the
ladies intend to bring up fifty for the ensuing winter.

\7  Seriously.  Then I shall be too young for the
fashion.

\4  No lady begins now to put on jewels till she's past forty. 
For instance, Miss there, in a polite circle, would be considered as a
child, as a mere maker of samplers.

\7  And yet Mrs. Niece thinks herself as much a woman,
and is as fond of jewels, as the oldest of us all.

\4  Your niece, is she?  And that young gentleman, a brother of
yours, I should presume?

\7  My son, sir.  They are contracted to each other. 
Observe their little sports.  They fall in and out ten times a day, as
if they were man and wife already.  \[To them\]  Well, Tony, child, what
soft things are you saying to your cousin Constance this evening?

\5  I have been saying no soft things; but that it's very hard to be
followed about so.  Ecod! I've not a place in the house now that's left
to myself, but the stable.

\7  Never mind him, Con, my dear.  He's in another story
behind your back.

\9  There's something generous in my cousin's manner.  He
falls out before faces to be forgiven in private.

\5  That's a damned confounded---crack.

\7  Ah! he's a sly one.  Don't you think they are like
each other about the mouth, Mr. Hastings?  The Blenkinsop mouth to a T. 
They're of a size too.  Back to back, my pretties, that Mr. Hastings
may see you.  Come, Tony.

\5  You had as good not make me, I tell you. \[Measuring\]

\9  O lud! he has almost cracked my head.

\7  O, the monster!  For shame, Tony.  You a man, and behave so!

\5  If I'm a man, let me have my fortin.  Ecod! I'll not be made a
fool of no longer.

\7  Is this, ungrateful boy, all that I'm to get for the
pains I have taken in your education?  I that have rocked you in your
cradle, and fed that pretty mouth with a spoon!  Did not I work that
waistcoat to make you genteel?  Did not I prescribe for you every day,
and weep while the receipt was operating?

\5  Ecod! you had reason to weep, for you have been dosing me ever
since I was born.  I have gone through every receipt in the Complete
Huswife ten times over; and you have thoughts of coursing me through
Quincy next spring.  But, ecod! I tell you, I'll not be made a fool of
no longer.

\7  Wasn't it all for your good, viper?  Wasn't it all for your good?

\5  I wish you'd let me and my good alone, then.  Snubbing this way
when I'm in spirits.  If I'm to have any good, let it come of itself;
not to keep dinging it, dinging it into one so.

\7  That's false; I never see you when you're in
spirits.  No, Tony, you then go to the alehouse or kennel.  I'm never
to be delighted with your agreeable wild notes, unfeeling monster!

\5  Ecod! mamma, your own notes are the wildest of the two.

\7  Was ever the like?  But I see he wants to break my
heart, I see he does.

\4  Dear madam, permit me to lecture the young gentleman a
little.  I'm certain I can persuade him to his duty.

\7  Well, I must retire.  Come, Constance, my love.  You
see, Mr. Hastings, the wretchedness of my situation: was ever poor
woman so plagued with a dear sweet, pretty, provoking, undutiful
boy? \[r]Exeunt \7 and \9\]

\5  \[Singing\]  \textsl{There was a young man riding by, and fain would
have his will.  Rang do didlo dee}.------Don't mind her.  Let her cry. 
It's the comfort of her heart.  I have seen her and sister cry over a
book for an hour together; and they said they liked the book the better
the more it made them cry.

\4  Then you're no friend to the ladies, I find, my pretty
young gentleman?

\5  That's as I find 'um.

\4  Not to her of your mother's choosing, I dare answer?  And
yet she appears to me a pretty well-tempered girl.

\5  That's because you don't know her as well as I.  Ecod! I know
every inch about her; and there's not a more bitter cantankerous toad
in all Christendom.

\4  \[Aside\]  Pretty encouragement this for a lover!

\5  I have seen her since the height of that.  She has as many
tricks as a hare in a thicket, or a colt the first day's breaking.

\4  To me she appears sensible and silent.

\5  Ay, before company.  But when she's with her playmate, she's as
loud as a hog in a gate.

\4  But there is a meek modesty about her that charms me.

\5  Yes, but curb her never so little, she kicks up, and you're
flung in a ditch.

\4  Well, but you must allow her a little beauty.---Yes, you must
allow her some beauty.

\5  Bandbox!  She's all a made-up thing, mun.  Ah! could you but see
Bet Bouncer of these parts, you might then talk of beauty.  Ecod, she
has two eyes as black as sloes, and cheeks as broad and red as a pulpit
cushion.  She'd make two of she.

\4  Well, what say you to a friend that would take this bitter
bargain off your hands?

\5  Anon.

\4  Would you thank him that would take Miss Neville, and leave
you to happiness and your dear Betsy?

\5  Ay; but where is there such a friend, for who would take her?

\4  I am he.  If you but assist me, I'll engage to whip her off
to France, and you shall never hear more of her.

\5  Assist you!  Ecod I will, to the last drop of my blood.  I'll
clap a pair of horses to your chaise that shall trundle you off in a
twinkling, and may be get you a part of her fortin beside, in jewels,
that you little dream of.

\4  My dear 'squire, this looks like a lad of spirit.

\5  Come along, then, and you shall see more of my spirit before you
have done with me. \[singing\] \textsl{We are the boys
That fears no noise
Where the thundering cannons roar}.  \[r]Exeunt\]



\Act


\(Enter \3, alone.\)


\3  What could my old friend Sir Charles mean by recommending
his son as the modestest young man in town?  To me he appears the most
impudent piece of brass that ever spoke with a tongue.  He has taken
possession of the easy chair by the fire-side already.  He took off his
boots in the parlour, and desired me to see them taken care of.  I'm
desirous to know how his impudence affects my daughter.  She will
certainly be shocked at it.


\(Enter \8, plainly dressed.\)


\3  Well, my Kate, I see you have changed your dress, as I
bade you; and yet, I believe, there was no great occasion.

\8  I find such a pleasure, sir, in obeying your
commands, that I take care to observe them without ever debating their
propriety.

\3  And yet, Kate, I sometimes give you some cause,
particularly when I recommended my modest gentleman to you as a lover
to-day.

\8  You taught me to expect something extraordinary, and
I find the original exceeds the description.

\3  I was never so surprised in my life!  He has quite
confounded all my faculties!

\8  I never saw anything like it: and a man of the world
too!

\3  Ay, he learned it all abroad---what a fool was I, to think
a young man could learn modesty by travelling.  He might as soon learn
wit at a masquerade.

\8  It seems all natural to him.

\3  A good deal assisted by bad company and a French danc\-ing-master.

{\nonfrenchspacing
\8  Sure you mistake, papa!  A French dancing-master
could never have taught him that timid look,---that awkward
address,---that bashful manner---

}

\3  Whose look?  whose manner, child?

\8  Mr. Marlow's: his mauvaise honte, his timidity,
struck me at the first sight.

\3  Then your first sight deceived you; for I think him one of
the most brazen first sights that ever astonished my senses.

\8  Sure, sir, you rally!  I never saw any one so
modest.

\3  And can you be serious?  I never saw such a bouncing,
swaggering puppy since I was born.  Bully Dawson was but a fool to him.

\8  Surprising!  He met me with a respectful bow, a
stammering voice, and a look fixed on the ground.

\3  He met me with a loud voice, a lordly air, and a
familiarity that made my blood freeze again.

\8  He treated me with diffidence and respect; censured
the manners of the age; admired the prudence of girls that never
laughed; tired me with apologies for being tiresome; then left the room
with a bow, and ``Madam, I would not for the world detain you.''

\3  He spoke to me as if he knew me all his life before;
asked twenty questions, and never waited for an answer; interrupted my
best remarks with some silly pun; and when I was in my best story of
the Duke of Marlborough and Prince Eugene, he asked if I had not a good
hand at making punch.  Yes, Kate, he asked your father if he was a
maker of punch!

\8  One of us must certainly be mistaken.

\3  If he be what he has shown himself, I'm determined he
shall never have my consent.

\8  And if he be the sullen thing I take him, he shall
never have mine.

\3  In one thing then we are agreed---to reject him.

\8  Yes: but upon conditions.  For if you should find him
less impudent, and I more presuming---if you find him more respectful,
and I more importunate---I don't know---the fellow is well enough for a
man---Certainly, we don't meet many such at a horse-race in the country.

\3  If we should find him so------But that's impossible.  The
first appearance has done my business.  I'm seldom deceived in that.

\8  And yet there may be many good qualities under that
first appearance.

\3  Ay, when a girl finds a fellow's outside to her taste, she
then sets about guessing the rest of his furniture.  With her, a smooth
face stands for good sense, and a genteel figure for every virtue.

\8  I hope, sir, a conversation begun with a compliment
to my good sense, won't end with a sneer at my understanding?

\3  Pardon me, Kate.  But if young Mr. Brazen can find the art
of reconciling contradictions, he may please us both, perhaps.

\8  And as one of us must be mistaken, what if we go to
make further discoveries?

\3  Agreed.  But depend on't I'm in the right.

\8  And depend on't I'm not much in the wrong. 
\[r]Exeunt\]


\(Enter Tony, running in with a casket.\)


\5  Ecod! I have got them.  Here they are.  My cousin Con's
necklaces, bobs and all.  My mother shan't cheat the poor souls out of
their fortin neither.  O! my genus, is that you?


\(Enter \4\)


\4  My dear friend, how have you managed with your mother?  I
hope you have amused her with pretending love for your cousin, and that
you are willing to be reconciled at last?  Our horses will be refreshed
in a short time, and we shall soon be ready to set off.

\5  And here's something to bear your charges by the way \[giving the
casket\]; your sweetheart's jewels.  Keep them: and hang those, I say,
that would rob you of one of them.

\4  But how have you procured them from your mother?

\5  Ask me no questions, and I'll tell you no fibs.  I procured them
by the rule of thumb.  If I had not a key to every drawer in mother's
bureau, how could I go to the alehouse so often as I do?  An honest man
may rob himself of his own at any time.

\4  Thousands do it every day.  But to be plain with you; Miss
Neville is endeavouring to procure them from her aunt this very
instant.  If she succeeds, it will be the most delicate way at least of
obtaining them.

\5  Well, keep them, till you know how it will be.  But I know how
it will be well enough; she'd as soon part with the only sound tooth in
her head.

\4  But I dread the effects of her resentment, when she finds
she has lost them.

\5  Never you mind her resentment, leave \textit{me} to manage that.  I
don't value her resentment the bounce of a cracker.  Zounds! here they
are.  Morrice! Prance!  \[r]Exit \4\]


\(Enter \7 and \9\)


\7  Indeed, Constance, you amaze me.  Such a girl as you
want jewels!  It will be time enough for jewels, my dear, twenty years
hence, when your beauty begins to want repairs.

\9  But what will repair beauty at forty, will certainly
improve it at twenty, madam.

\7  Yours, my dear, can admit of none.  That natural
blush is beyond a thousand ornaments.  Besides, child, jewels are quite
out at present.  Don't you see half the ladies of our acquaintance, my
Lady Kill-daylight, and Mrs. Crump, and the rest of them, carry their
jewels to town, and bring nothing but paste and marcasites back.

\9  But who knows, madam, but somebody that shall be
nameless would like me best with all my little finery about me?

\7  Consult your glass, my dear, and then see if, with
such a pair of eyes, you want any better sparklers.  What do you think,
Tony, my dear? does your cousin Con. want any jewels in your eyes to
set off her beauty?

\5  That's as thereafter may be.

\9  My dear aunt, if you knew how it would oblige me.

\7  A parcel of old-fashioned rose and table-cut things. 
They would make you look like the court of King Solomon at a
puppet-show.  Besides, I believe, I can't readily come at them.  They
may be missing, for aught I know to the contrary.

\5  \[Apart to \7\]  Then why don't you tell her so at
once, as she's so longing for them?  Tell her they're lost.  It's the
only way to quiet her.  Say they're lost, and call me to bear witness.

\7  \[Apart to \5\]  You know, my dear, I'm only
keeping them for you.  So if I say they're gone, you'll bear me
witness, will you?  He! he! he!

\5  Never fear me.  Ecod! I'll say I saw them taken out with my own
eyes.

\9  I desire them but for a day, madam.  Just to be
permitted to show them as relics, and then they may be locked up
again.

\7  To be plain with you, my dear Constance, if I could
find them you should have them.  They're missing, I assure you.  Lost,
for aught I know; but we must have patience wherever they are.

\9  I'll not believe it! this is but a shallow pretence to
deny me.  I know they are too valuable to be so slightly kept, and as
you are to answer for the loss---

\7  Don't be alarmed, Constance.  If they be lost, I must
restore an equivalent.  But my son knows they are missing, and not to
be found.

\5  That I can bear witness to.  They are missing, and not to be
found; I'll take my oath on't.

\7  You must learn resignation, my dear; for though we
lose our fortune, yet we should not lose our patience.  See me, how
calm I am.

\9  Ay, people are generally calm at the misfortunes of
others.

\7  Now I wonder a girl of your good sense should waste a
thought upon such trumpery.  We shall soon find them; and in the mean
time you shall make use of my garnets till your jewels be found.

\9  I detest garnets.

\7  The most becoming things in the world to set off a
clear complexion.  You have often seen how well they look upon me.  You
\textit{shall} have them.  \[r]Exit\]

\9  I dislike them of all things.  You shan't stir.---Was
ever anything so provoking, to mislay my own jewels, and force me to
wear her trumpery?

\5  Don't be a fool.  If she gives you the garnets, take what you
can get.  The jewels are your own already.  I have stolen them out of
her bureau, and she does not know it.  Fly to your spark, he'll tell
you more of the matter.  Leave me to manage her.

\9  My dear cousin!

\5  Vanish.  She's here, and has missed them already.  \[r]Exit \9\]
Zounds! how she fidgets and spits about like a Catherine wheel.


\(Enter \7\)


\7  Confusion! thieves! robbers! we are cheated,
plundered, broke open, undone.

\5  What's the matter, what's the matter, mamma?  I hope nothing has
happened to any of the good family!

\7  We are robbed.  My bureau has been broken open, the
jewels taken out, and I'm undone.

\5  Oh! is that all?  Ha! ha! ha!  By the laws, I never saw it
acted better in my life.  Ecod, I thought you was ruined in earnest,
ha! ha! ha!

\7  Why, boy, I \textit{am} ruined in earnest.  My bureau has been
broken open, and all taken away.

\5  Stick to that: ha! ha! ha! stick to that.  I'll bear witness,
you know; call me to bear witness.

\7  I tell you, Tony, by all that's precious, the jewels
are gone, and I shall be ruined for ever.

\5  Sure I know they're gone, and I'm to say so.

\7  My dearest Tony, but hear me.  They're gone, I say.

\5  By the laws, mamma, you make me for to laugh, ha! ha!  I know
who took them well enough, ha! ha! ha!

\7  Was there ever such a blockhead, that can't tell the
difference between jest and earnest?  I tell you I'm not in jest,
booby.

\5  That's right, that's right; you must be in a bitter passion, and
then nobody will suspect either of us.  I'll bear witness that they are
gone.

\7  Was there ever such a cross-grained brute, that
won't hear me?  Can you bear witness that you're no better than a
fool?  Was ever poor woman so beset with fools on one hand, and
thieves on the other?

\5  I can bear witness to that.

\7  Bear witness again, you blockhead you, and I'll turn
you out of the room directly.  My poor niece, what will become of her? 
Do you laugh, you unfeeling brute, as if you enjoyed my distress?

\5  I can bear witness to that.

\7  Do you insult me, monster?  I'll teach you to vex
your mother, I will.

\5  I can bear witness to that.  [He runs off, she follows him.]


\(Enter \8 and \10.\)


\8  What an unaccountable creature is that brother of
mine, to send them to the house as an inn! ha! ha!  I don't wonder at
his impudence.

\10  But what is more, madam, the young gentleman, as you passed by
in your present dress, asked me if you were the bar-maid.  He mistook
you for the bar-maid, madam.

\8  Did he?  Then as I live, I'm resolved to keep up the
delusion.  Tell me, Pimple, how do you like my present dress?  Don't
you think I look something like Cherry in the Beaux Stratagem?

\10  It's the dress, madam, that every lady wears in the country, but
when she visits or receives company.

\8  And are you sure he does not remember my face or
person?

\10  Certain of it.

\8  I vow, I thought so; for, though we spoke for some
time together, yet his fears were such, that he never once looked up
during the interview.  Indeed, if he had, my bonnet would have kept him
from seeing me.

\10  But what do you hope from keeping him in his mistake?

\8  In the first place I shall be seen, and that is no
small advantage to a girl who brings her face to market.  Then I shall
perhaps make an acquaintance, and that's no small victory gained over
one who never addresses any but the wildest of her sex.  But my chief
aim is, to take my gentleman off his guard, and, like an invisible
champion of romance, examine the giant's force before I offer to
combat.

\10  But you are sure you can act your part, and disguise your voice
so that he may mistake that, as he has already mistaken your person?

\8  Never fear me.  I think I have got the true bar
cant---Did your honour call?---Attend the Lion there---Pipes and tobacco
for the Angel.---The Lamb has been outrageous this half-hour.

\10  It will do, madam.  But he's here.  \[r]Exit \10\]


\(Enter \2\)


\2  What a bawling in every part of the house!  I have scarce a
moment's repose.  If I go to the best room, there I find my host and
his story: if I fly to the gallery, there we have my hostess with her
curtsey down to the ground.  I have at last got a moment to myself, and
now for recollection.  [Walks and muses.]

\8  Did you call, sir?  Did your honour call?

\2  \[Musing\]  As for Miss Hardcastle, she's too grave and
sentimental for me.

\8  Did your honour call?  \[She still places herself
before him, he turning away\]

\2  No, child.  \[Musing\]  Besides, from the glimpse I had of her,
I think she squints.

\8  I'm sure, sir, I heard the bell ring.

\2  No, no.  \[Musing\]  I have pleased my father, however, by
coming down, and I'll to-morrow please myself by returning. \[Taking
out his tablets, and perusing\]

\8  Perhaps the other gentleman called, sir?

\2  I tell you, no.

\8  I should be glad to know, sir.  We have such a
parcel of servants!

\2  No, no, I tell you.  \[Looks full in her face\]  Yes, child, I
think I did call.  I wanted---I wanted---I vow, child, you are vastly
handsome.

\8  O la, sir, you'll make one ashamed.

\2  Never saw a more sprightly malicious eye.  Yes, yes, my dear,
I did call.  Have you got any of your---a---what d'ye call it in the
house?

\8  No, sir, we have been out of that these ten days.

\2  One may call in this house, I find, to very little purpose. 
Suppose I should call for a taste, just by way of a trial, of the
nectar of your lips; perhaps I might be disappointed in that too.

\8  Nectar! nectar!  That's a liquor there's no call for
in these parts.  French, I suppose.  We sell no French wines here, sir.

\2  Of true English growth, I assure you.

\8  Then it's odd I should not know it.  We brew all
sorts of wines in this house, and I have lived here these eighteen
years.

\2  Eighteen years!  Why, one would think, child, you kept the bar
before you were born.  How old are you?

\8  O! sir, I must not tell my age.  They say women and
music should never be dated.

\2  To guess at this distance, you can't be much above forty
\[approaching\].  Yet, nearer, I don't think so much \[approaching\].  By
coming close to some women they look younger still; but when we come
very close indeed---\[attempting to kiss her\].

\8  Pray, sir, keep your distance.  One would think you
wanted to know one's age, as they do horses, by mark of mouth.

\2  I protest, child, you use me extremely ill.  If you keep me at
this distance, how is it possible you and I can ever be acquainted?

\8  And who wants to be acquainted with you?  I want no
such acquaintance, not I.  I'm sure you did not treat Miss Hardcastle,
that was here awhile ago, in this obstropalous manner.  I'll warrant
me, before her you looked dashed, and kept bowing to the ground, and
talked, for all the world, as if you was before a justice of peace.

\2  \[Aside\]  Egad, she has hit it, sure enough!  \[To her\]  In
awe of her, child?  Ha! ha! ha!  A mere awkward squinting thing; no,
no.  I find you don't know me.  I laughed and rallied her a little; but
I was unwilling to be too severe.  No, I could not be too severe, curse
me!

\8  O! then, sir, you are a favourite, I find, among the
ladies?

\2  Yes, my dear, a great favourite.  And yet hang me, I don't see
what they find in me to follow.  At the Ladies' Club in town I'm called
their agreeable Rattle.  Rattle, child, is not my real name, but one
I'm known by.  My name is Solomons; Mr. Solomons, my dear, at your
service.  \[Offering to salute her\]

\8  Hold, sir; you are introducing me to your club, not
to yourself.  And you're so great a favourite there, you say?

\2  Yes, my dear.  There's Mrs. Mantrap, Lady Betty Blackleg, the
Countess of Sligo, Mrs. Langhorns, old Miss Biddy Buckskin, and your
humble servant, keep up the spirit of the place.

\8  Then it's a very merry place, I suppose?

\2  Yes, as merry as cards, supper, wine, and old women can make
us.

\8  And their agreeable Rattle, ha! ha! ha!

\2  \[Aside\]  Egad! I don't quite like this chit.  She looks
knowing, methinks.  You laugh, child?

\8  I can't but laugh, to think what time they all have
for minding their work or their family.

\2  \[Aside\]  All's well; she don't laugh at me.  \[To her\]  Do
you ever work, child?

\8  Ay, sure.  There's not a screen or quilt in the
whole house but what can bear witness to that.

\2  Odso! then you must show me your embroidery.  I embroider and
draw patterns myself a little.  If you want a judge of your work, you
must apply to me.  \[Seizing her hand\]

\8  Ay, but the colours do not look well by candlelight. 
You shall see all in the morning.  \[Struggling\]

\2  And why not now, my angel?  Such beauty fires beyond the
power of resistance.---Pshaw! the father here!  My old luck: I never
nicked seven that I did not throw ames ace three times following. 
\[r]Exit \2\]


\(Enter \3, who stands in surprise.\)


\3  So, madam.  So, I find \textit{this} is your \textit{modest} lover.  This is
your humble admirer, that kept his eyes fixed on the ground, and only
adored at humble distance.  Kate, Kate, art thou not ashamed to deceive
your father so?

\8  Never trust me, dear papa, but he's still the modest
man I first took him for; you'll be convinced of it as well as I.

\3  By the hand of my body, I believe his impudence is
infectious!  Didn't I see him seize your hand?  Didn't I see him haul
you about like a milkmaid?  And now you talk of his respect and his
modesty, forsooth!

\8  But if I shortly convince you of his modesty, that he
has only the faults that will pass off with time, and the virtues that
will improve with age, I hope you'll forgive him.

\3  The girl would actually make one run mad!  I tell you,
I'll not be convinced.  I am convinced.  He has scarce been three hours
in the house, and he has already encroached on all my prerogatives. 
You may like his impudence, and call it modesty; but my son-in-law,
madam, must have very different qualifications.

\8  Sir, I ask but this night to convince you.

\3  You shall not have half the time, for I have thoughts of
turning him out this very hour.

\8  Give me that hour then, and I hope to satisfy you.

\3  Well, an hour let it be then.  But I'll have no trifling
with your father.  All fair and open, do you mind me.

\8  I hope, sir, you have ever found that I considered
your commands as my pride; for your kindness is such, that my duty as
yet has been inclination.  \[r]Exeunt\]



\Act


\(Enter \4 and \9\)


\4  You surprise me; Sir Charles Marlow expected here this
night!  Where have you had your information?

\9  You may depend upon it.  I just saw his letter to Mr.
Hardcastle, in which he tells him he intends setting out a few hours
after his son.

\4  Then, my Constance, all must be completed before he
arrives.  He knows me; and should he find me here, would discover my
name, and perhaps my designs, to the rest of the family.

\9  The jewels, I hope, are safe?

\4  Yes, yes, I have sent them to Marlow, who keeps the keys of
our baggage.  In the mean time, I'll go to prepare matters for our
elopement.  I have had the 'squire's promise of a fresh pair of horses;
and if I should not see him again, will write him further directions. 
\[r]Exit\]

\9  Well! success attend you.  In the mean time I'll go and
amuse my aunt with the old pretence of a violent passion for my cousin. 
\[r]Exit\]


\(Enter \2, followed by a \12.\)


\2  I wonder what Hastings could mean by sending me so valuable a
thing as a casket to keep for him, when he knows the only place I have
is the seat of a post-coach at an inn-door.  Have you deposited the
casket with the landlady, as I ordered you?  Have you put it into her
own hands?

\12  Yes, your honour.

\2  She said she'd keep it safe, did she?

\12  Yes, she said she'd keep it safe enough; she asked me how I
came by it; and she said she had a great mind to make me give an
account of myself.  \[r]Exit \12\]

\2  Ha! ha! ha!  They're safe, however.  What an unaccountable set
of beings have we got amongst!  This little bar-maid though runs in my
head most strangely, and drives out the absurdities of all the rest of
the family.  She's mine, she must be mine, or I'm greatly mistaken.


\(Enter \4\)


\4  Bless me!  I quite forgot to tell her that I intended to
prepare at the bottom of the garden.  Marlow here, and in spirits too!

\2  Give me joy, George!  Crown me, shadow me with laurels! 
Well, George, after all, we modest fellows don't want for success
among the women.

\4  Some women, you mean.  But what success has your honour's
modesty been crowned with now, that it grows so insolent upon us?

\2  Didn't you see the tempting, brisk, lovely little thing, that
runs about the house with a bunch of keys to its girdle?

\4  Well, and what then?

\2  She's mine, you rogue you.  Such fire, such motion, such
eyes, such lips; but, egad! she would not let me kiss them though.

\4  But are you so sure, so very sure of her?

\2  Why, man, she talked of showing me her work above stairs, and
I am to improve the pattern.

\4  But how can you, Charles, go about to rob a woman of her
honour?

\2  Pshaw! pshaw!  We all know the honour of the bar-maid of an
inn.  I don't intend to rob her, take my word for it; there's nothing
in this house I shan't honestly pay for.

\4  I believe the girl has virtue.

\2  And if she has, I should be the last man in the world that
would attempt to corrupt it.

\4  You have taken care, I hope, of the casket I sent you to
lock up?  Is it in safety?

\2  Yes, yes.  It's safe enough.  I have taken care of it.  But
how could you think the seat of a post-coach at an inn-door a place of
safety?  Ah! numskull!  I have taken better precautions for you than
you did for yourself------I have------

\4  What?

\2  I have sent it to the landlady to keep for you.

\4  To the landlady!

\2  The landlady.

\4  You did?

\2  I did.  She's to be answerable for its forthcoming, you know.

\4  Yes, she'll bring it forth with a witness.

\2  Wasn't I right?  I believe you'll allow that I acted
prudently upon this occasion.

\4  \[Aside\]  He must not see my uneasiness.

\2  You seem a little disconcerted though, methinks.  Sure
nothing has happened?

\4  No, nothing.  Never was in better spirits in all my life. 
And so you left it with the landlady, who, no doubt, very readily
undertook the charge.

\2  Rather too readily.  For she not only kept the casket, but,
through her great precaution, was going to keep the messenger too.  Ha!
ha! ha!

\4  He! he! he!  They're safe, however.

\2  As a guinea in a miser's purse.

\4  \[Aside\]  So now all hopes of fortune are at an end, and we
must set off without it.  \[To him\]  Well, Charles, I'll leave you to
your meditations on the pretty bar-maid, and, he! he! he! may you be as
successful for yourself, as you have been for me!  \[r]Exit\]

\2  Thank ye, George: I ask no more.  Ha! ha! ha!


\(Enter \3\)


\3  I no longer know my own house.  It's turned all
topsy-turvy.  His servants have got drunk already.  I'll bear it no
longer; and yet, from my respect for his father, I'll be calm.  \[To
him\]  Mr. Marlow, your servant.  I'm your very humble servant. 
\[Bowing low\]

\2  Sir, your humble servant.  \[Aside\]  What's to be the wonder
now?

\3  I believe, sir, you must be sensible, sir, that no man
alive ought to be more welcome than your father's son, sir.  I hope you
think so?

\2  I do from my soul, sir.  I don't want much entreaty.  I
generally make my father's son welcome wherever he goes.

\3  I believe you do, from my soul, sir.  But though I say
nothing to your own conduct, that of your servants is insufferable. 
Their manner of drinking is setting a very bad example in this house, 
I assure you.

\2  I protest, my very good sir, that is no fault of mine.  If
they don't drink as they ought, they are to blame.  I ordered them not
to spare the cellar.  I did, I assure you.  \[To the side scene\]  Here,
let one of my servants come up.  \[To him\]  My positive directions
were, that as I did not drink myself, they should make up for my
deficiencies below.

\3  Then they had your orders for what they do?  I'm
satisfied!

\2  They had, I assure you.  You shall hear from one of
themselves.


\(Enter \12, drunk.\)


\2  You, Jeremy!  Come forward, sirrah!  What were my orders? 
Were you not told to drink freely, and call for what you thought fit,
for the good of the house?

\3  \[Aside\]  I begin to lose my patience.

\persona{Jeremy} Please your honour, liberty and Fleet-street for ever! 
Though I'm but a servant, I'm as good as another man.  I'll drink for
no man before supper, sir, damme!  Good liquor will sit upon a good
supper, but a good supper will not sit upon------hiccup------on my
conscience, sir.

\2  You see, my old friend, the fellow is as drunk as he can
possibly be.  I don't know what you'd have more, unless you'd have the
poor devil soused in a beer-barrel.

\3  Zounds! he'll drive me distracted, if I contain myself any
longer.  Mr. Marlow---Sir; I have submitted to your insolence for more
than four hours, and I see no likelihood of its coming to an end.  I'm
now resolved to be master here, sir; and I desire that you and your
drunken pack may leave my house directly.

\2  Leave your house!------Sure you jest, my good friend!  What?
when I'm doing what I can to please you.

\3  I tell you, sir, you don't please me; so I desire you'll
leave my house.

\2  Sure you cannot be serious?  At this time o' night, and such a
night?  You only mean to banter me.

\3  I tell you, sir, I'm serious! and now that my passions are
roused, I say this house is mine, sir; this house is mine, and I
command you to leave it directly.

\2  Ha! ha! ha!  A puddle in a storm.  I shan't stir a step, I
assure you.  \[In a serious tone\]  This your house, fellow!  It's my
house.  This is my house.  Mine, while I choose to stay.  What right
have you to bid me leave this house, sir?  I never met with such
impudence, curse me; never in my whole life before.

\3  Nor I, confound me if ever I did.  To come to my house, to
call for what he likes, to turn me out of my own chair, to insult the
family, to order his servants to get drunk, and then to tell me, ``This
house is mine, sir.''  By all that's impudent, it makes me laugh.  Ha!
ha! ha!  Pray, sir \[bantering\], as you take the house, what think you
of taking the rest of the furniture?  There's a pair of silver
candlesticks, and there's a fire-screen, and here's a pair of
brazen-nosed bellows; perhaps you may take a fancy to them?

\2  Bring me your bill, sir; bring me your bill, and let's make no
more words about it.

\3  There are a set of prints, too.  What think you of the
Rake's Progress, for your own apartment?

\2  Bring me your bill, I say; and I'll leave you and your
infernal house directly.

\3  Then there's a mahogany table that you may see your own
face in.

\2  My bill, I say.

\3  I had forgot the great chair for your own particular
slumbers, after a hearty meal.

\2  Zounds! bring me my bill, I say, and let's hear no more on't.

\3  Young man, young man, from your father's letter to me, I
was taught to expect a well-bred modest man as a visitor here, but now
I find him no better than a coxcomb and a bully; but he will be down
here presently, and shall hear more of it.  \[r]Exit\]

\2  How's this?  Sure I have not mistaken the house.  Everything
looks like an inn.  The servants cry, coming; the attendance is
awkward; the bar-maid, too, to attend us.  But she's here, and will
further inform me.  Whither so fast, child?  A word with you.


\(Enter \8\)


\8  Let it be short, then.  I'm in a hurry.  \[Aside\]  I
believe be begins to find out his mistake.  But it's too soon quite to
undeceive him.

\2  Pray, child, answer me one question.  What are you, and what
may your business in this house be?

\8  A relation of the family, sir.

\2  What, a poor relation.

\8  Yes, sir.  A poor relation, appointed to keep the
keys, and to see that the guests want nothing in my power to give them.

\2  That is, you act as the bar-maid of this inn.

\8  Inn!  O law------what brought that in your head?  One
of the best families in the country keep an inn---Ha! ha! ha! old Mr.
Hardcastle's house an inn!

\2  Mr. Hardcastle's house!  Is this Mr. Hardcastle's house,
child?

\8  Ay, sure!  Whose else should it be?

\2  So then, all's out, and I have been damnably imposed on.  O,
confound my stupid head, I shall be laughed at over the whole town.  I
shall be stuck up in caricatura in all the print-shops.  The
\textit{Dullissimo Maccaroni}.  To mistake this house of all others for an inn, and my
father's old friend for an innkeeper!  What a swaggering puppy must he
take me for!  What a silly puppy do I find myself!  There again, may I
be hanged, my dear, but I mistook you for the bar-maid.

\8  Dear me! dear me!  I'm sure there's nothing in my
\textit{behaviour} to put me on a level with one of that stamp.

\2  Nothing, my dear, nothing.  But I was in for a list of
blunders, and could not help making you a subscriber.  My stupidity saw
everything the wrong way.  I mistook your assiduity for assurance, and
your simplicity for allurement.  But it's over.  This house I no more
show \textit{my} face in.

\8  I hope, sir, I have done nothing to disoblige you. 
I'm sure I should be sorry to affront any gentleman who has been so
polite, and said so many civil things to me.  I'm sure I should be
sorry \[pretending to cry\] if he left the family upon my account.  I'm
sure I should be sorry if people said anything amiss, since I have no
fortune but my character.

\2  \[Aside\]  By Heaven! she weeps.  This is the first mark of
tenderness I ever had from a modest woman, and it touches me.  \[To
her\]  Excuse me, my lovely girl; you are the only part of the family I
leave with reluctance.  But to be plain with you, the difference of our
birth, fortune, and education, makes an honourable connexion
impossible; and I can never harbour a thought of seducing simplicity
that trusted in my honour, of bringing ruin upon one whose only fault
was being too lovely.

\8  \[Aside\]  Generous man!  I now begin to admire him. 
\[To him\]  But I am sure my family is as good as Miss Hardcastle's; and
though I'm poor, that's no great misfortune to a contented mind; and,
until this moment, I never thought that it was bad to want fortune.

\2  And why now, my pretty simplicity?

\8  Because it puts me at a distance from one that, if I
had a thousand pounds, I would give it all to.

\2  \[Aside\]  This simplicity bewitches me, so that if I stay, I'm
undone.  I must make one bold effort, and leave her.  \[To her\]  Your
partiality in my favour, my dear, touches me most sensibly: and were I
to live for myself alone, I could easily fix my choice.  But I owe too
much to the opinion of the world, too much to the authority of a
father; so that---I can scarcely speak it---it affects me.  Farewell. 
\[r]Exit\]

\8  I never knew half his merit till now.  He shall not
go, if I have power or art to detain him.  I'll still preserve the
character in which \textit{i stooped to conquer}; but will undeceive my papa,
who perhaps may laugh him out of his resolution.  \[r]Exit\]


\(Enter Tony and \9\)


\5  Ay, you may steal for yourselves the next time.  I have done my
duty.  She has got the jewels again, that's a sure thing; but she
believes it was all a mistake of the servants.

\9  But, my dear cousin, sure you won't forsake us in this
distress?  If she in the least suspects that I am going off, I shall
certainly be locked up, or sent to my aunt Pedigree's, which is ten
times worse.

\5  To be sure, aunts of all kinds are damned bad things.  But what
can I do?  I have got you a pair of horses that will fly like
Whistle-jacket; and I'm sure you can't say but I have courted you
nicely before her face.  Here she comes, we must court a bit or two
more, for fear she should suspect us.  [They retire, and seem to
fondle.]


\(Enter \7\)


\7  Well, I was greatly fluttered, to be sure.  But my
son tells me it was all a mistake of the servants.  I shan't be easy,
however, till they are fairly married, and then let her keep her own
fortune.  But what do I see? fondling together, as I'm alive.  I never
saw Tony so sprightly before.  Ah! have I caught you, my pretty doves? 
What, billing, exchanging stolen glances and broken murmurs?  Ah!

\5  As for murmurs, mother, we grumble a little now and then, to be
sure.  But there's no love lost between us.

\7  A mere sprinkling, Tony, upon the flame, only to make
it burn brighter.

\9  Cousin Tony promises to give us more of his company at
home.  Indeed, he shan't leave us any more.  It won't leave us, cousin
Tony, will it?

\5  O! it's a pretty creature.  No, I'd sooner leave my horse in a
pound, than leave you when you smile upon one so.  Your laugh makes you
so becoming.

\9  Agreeable cousin!  Who can help admiring that natural
humour, that pleasant, broad, red, thoughtless \[patting his cheek\]---ah!
it's a bold face.

\7  Pretty innocence!

\5  I'm sure I always loved cousin Con.'s hazle eyes, and her
pretty long fingers, that she twists this way and that over the
haspicholls, like a parcel of bobbins.

\7  Ah! he would charm the bird from the tree.  I was
never so happy before.  My boy takes after his father, poor Mr.
Lumpkin, exactly.  The jewels, my dear Con., shall be yours
incontinently.  You shall have them.  Isn't he a sweet boy, my dear? 
You shall be married to-morrow, and we'll put off the rest of his
education, like Dr. Drowsy's sermons, to a fitter opportunity.


\(Enter \6\)


\6  Where's the 'squire?  I have got a letter for your worship.

\5  Give it to my mamma.  She reads all my letters first.

\6  I had orders to deliver it into your own hands.

\5  Who does it come from?

\6  Your worship mun ask that o' the letter itself.

\5  I could wish to know though \[turning the letter, and gazing on
it\].

\9  \[Aside\]  Undone! undone!  A letter to him from
Hastings.  I know the hand.  If my aunt sees it, we are ruined for
ever.  I'll keep her employed a little if I can.  \[To \3\]  But I have not told you, madam, of my cousin's smart
answer just now to Mr. Marlow.  We so laughed.---You must know,
madam.---This way a little, for he must not hear us.  [They confer.]

\5  \[Still gazing\]  A damned cramp piece of penmanship, as ever I
saw in my life.  I can read your print hand very well.  But here are
such handles, and shanks, and dashes, that one can scarce tell the head
from the tail.---``To Anthony Lumpkin, Esquire.''  It's very odd, I can
read the outside of my letters, where my own name is, well enough; but
when I come to open it, it's all------buzz.  That's hard, very hard; for
the inside of the letter is always the cream of the correspondence.

\7  Ha! ha! ha!  Very well, very well.  And so my son was
too hard for the philosopher.

\9  Yes, madam; but you must hear the rest, madam.  A
little more this way, or he may hear us.  You'll hear how he puzzled
him again.

\7  He seems strangely puzzled now himself, methinks.

\5  \[Still gazing\]  A damned up and down hand, as if it was
disguised in liquor.---\[Reading\]  Dear Sir,---ay, that's that.  Then
there's an M, and a T, and an S, but whether the next be an izzard, or
an R, confound me, I cannot tell.

\7  What's that, my dear?  Can I give you any
assistance?

\9  Pray, aunt, let me read it.  Nobody reads a cramp hand
better than I. \[Twitching the letter from him\]  Do you know who it is
from?

\5  Can't tell, except from Dick Ginger, the feeder.

\9  Ay, so it is.  \[Pretending to read\]  Dear 'Squire,
hoping that you're in health, as I am at this present.  The gentlemen
of the Shake-bag club has cut the gentlemen of Goose-green quite out of
feather.  The odds---um---odd battle---um---long fighting---um---here, here,
it's all about cocks and fighting; it's of no consequence; here, put it
up, put it up.  \[Thrusting the crumpled letter upon him\]

\5  But I tell you, miss, it's of all the consequence in the world. 
I would not lose the rest of it for a guinea.  Here, mother, do you
make it out.  Of no consequence!  \[Giving \7 the letter\]

\7  How's this?---\[Reads\]  ``Dear 'Squire, I'm now
waiting for Miss Neville, with a post-chaise and pair, at the bottom of
the garden, but I find my horses yet unable to perform the journey.  I
expect you'll assist us with a pair of fresh horses, as you promised. 
Dispatch is necessary, as the \textit{hag} (ay, the hag), your mother, will
otherwise suspect us!  Yours, Hastings.''  Grant me patience.  I shall
run distracted!  My rage chokes me.

\9  I hope, madam, you'll suspend your resentment for a few
moments, and not impute to me any impertinence, or sinister design,
that belongs to another.

\7  \[Curtseying very low\]  Fine spoken, madam, you are
most miraculously polite and engaging, and quite the very pink of
courtesy and circumspection, madam.  \[Changing her tone\]  And you, you
great ill-fashioned oaf, with scarce sense enough to keep your mouth
shut: were you, too, joined against me?  But I'll defeat all your plots
in a moment.  As for you, madam, since you have got a pair of fresh
horses ready, it would be cruel to disappoint them.  So, if you please,
instead of running away with your spark, prepare, this very moment, to
run off with \textit{me}.  Your old aunt Pedigree will keep you secure, I'll
warrant me.  You too, sir, may mount your horse, and guard us upon the
way.  Here, Thomas, Roger, Diggory!  I'll show you, that I wish you
better than you do yourselves.  \[r]Exit\]

\9  So now I'm completely ruined.

\5  Ay, that's a sure thing.

\9  What better could be expected from being connected with
such a stupid fool,---and after all the nods and signs I made him?

\5  By the laws, miss, it was your own cleverness, and not my
stupidity, that did your business.  You were so nice and so busy with
your Shake-bags and Goose-greens, that I thought you could never be
making believe.


\(Enter \4\)


\4  So, sir, I find by my servant, that you have shown my
letter, and betrayed us.  Was this well done, young gentleman?

\5  Here's another.  Ask miss there, who betrayed you.  Ecod, it was
her doing, not mine.


\(Enter \2\)


\2  So I have been finely used here among you.  Rendered
contemptible, driven into ill manners, despised, insulted, laughed at.

\5  Here's another.  We shall have old Bedlam broke loose
presently.

\9  And there, sir, is the gentleman to whom we all owe
every obligation.

\2  What can I say to him, a mere boy, an idiot, whose ignorance
and age are a protection?

\4  A poor contemptible booby, that would but disgrace
correction.

\9  Yet with cunning and malice enough to make himself
merry with all our embarrassments.

\4  An insensible cub.

\2  Replete with tricks and mischief.

\5  Baw! damme, but I'll fight you both, one after the
other------with baskets.

\2  As for him, he's below resentment.  But your conduct, Mr.
Hastings, requires an explanation.  You knew of my mistakes, yet would
not undeceive me.

\4  Tortured as I am with my own disappointments, is this a time
for explanations?  It is not friendly, Mr. Marlow.

\2  But, sir------

\9  Mr. Marlow, we never kept on your mistake till it was
too late to undeceive you.


\(Enter \12.\)


\12  My mistress desires you'll get ready immediately, madam.  The
horses are putting to.  Your hat and things are in the next room.  We
are to go thirty miles before morning.  \[r]Exit \12\]

\9  Well, well: I'll come presently.

\2  \[To \4\]  Was it well done, sir, to assist in rendering
me ridiculous?  To hang me out for the scorn of all my acquaintance? 
Depend upon it, sir, I shall expect an explanation.

\4  Was it well done, sir, if you're upon that subject, to
deliver what I entrusted to yourself, to the care of another sir?

\9  Mr. Hastings!  Mr. Marlow!  Why will you increase my
distress by this groundless dispute?  I implore, I entreat you------


\(Enter \12.\)


\12  Your cloak, madam.  My mistress is impatient.  \[r]Exit \12\]

\9  I come.  Pray be pacified.  If I leave you thus, I
shall die with apprehension.


\(Enter \12.\)


\12  Your fan, muff, and gloves, madam.  The horses are waiting.

\9  O, Mr. Marlow! if you knew what a scene of constraint
and ill-nature lies before me, I'm sure it would convert your
resentment into pity.

\2  I'm so distracted with a variety of passions, that I don't
know what I do.  Forgive me, madam.  George, forgive me.  You know my
hasty temper, and should not exasperate it.

\4  The torture of my situation is my only excuse.

\9  Well, my dear Hastings, if you have that esteem for me
that I think, that I am sure you have, your constancy for three years
will but increase the happiness of our future connexion.  If------

\7  \[Within\]  Miss Neville.  Constance, why Constance, I
say.

\9  I'm coming.  Well, constancy, remember, constancy is the
word.  \[r]Exit\]

\4  My heart! how can I support this?  To be so near happiness,
and such happiness!

\2  \[To Tony\]  You see now, young gentleman, the effects of your
folly.  What might be amusement to you, is here disappointment, and
even distress.

\5  \[From a reverie\]  Ecod, I have hit it.  It's here.  Your
hands.  Yours and yours, my poor Sulky!---My boots there, ho!---Meet me
two hours hence at the bottom of the garden; and if you don't find Tony
Lumpkin a more good-natured fellow than you thought for, I'll give you
leave to take my best horse, and Bet Bouncer into the bargain.  Come
along.  My boots, ho!  \[r]Exeunt\]



\Act


\((\01 continued.)\)


\(Enter \4 and \12.\)


\4  You saw the old lady and Miss Neville drive off, you say?

\12  Yes, your honour.  They went off in a post-coach, and the
young 'squire went on horseback.  They're thirty miles off by this
time.

\4  Then all my hopes are over.

\12  Yes, sir.  Old Sir Charles has arrived.  He and the old
gentleman of the house have been laughing at Mr. Marlow's mistake this
half hour.  They are coming this way.

\4  Then I must not be seen.  So now to my fruitless
appointment at the bottom of the garden.  This is about the time. 
\[r]Exit\]


\(Enter \1 and \3\)


\3  Ha! ha! ha!  The peremptory tone in which he sent forth
his sublime commands!

\1  And the reserve with which I suppose he treated all your
advances.

\3  And yet he might have seen something in me above a common
innkeeper, too.

\1  Yes, Dick, but he mistook you for an uncommon innkeeper,
ha! ha! ha!

\3  Well, I'm in too good spirits to think of anything but
joy.  Yes, my dear friend, this union of our families will make our
personal friendships hereditary; and though my daughter's fortune is
but small---

\1  Why, Dick, will you talk of fortune to \textit{me}?  My son is
possessed of more than a competence already, and can want nothing but a
good and virtuous girl to share his happiness and increase it.  If they
like each other, as you say they do---

\3  \textit{If}, man!  I tell you they \textit{do} like each other.  My
daughter as good as told me so.

\1  But girls are apt to flatter themselves, you know.

\3  I saw him grasp her hand in the warmest manner myself; and
here he comes to put you out of your \textit{ifs}, I warrant him.


\(Enter \2\)


\2  I come, sir, once more, to ask pardon for my strange conduct. 
I can scarce reflect on my insolence without confusion.

\3  Tut, boy, a trifle!  You take it too gravely.  An hour or
two's laughing with my daughter will set all to rights again.  She'll
never like you the worse for it.

\2  Sir, I shall be always proud of her approbation.

\3  Approbation is but a cold word, Mr. Marlow; if I am not
deceived, you have something more than approbation thereabouts.  You
take me?

\2  Really, sir, I have not that happiness.

\3  Come, boy, I'm an old fellow, and know what's what as well
as you that are younger.  I know what has passed between you; but mum.

\2  Sure, sir, nothing has passed between us but the most
profound respect on my side, and the most distant reserve on hers.  You
don't think, sir, that my impudence has been passed upon all the rest
of the family.

\3  Impudence!  No, I don't say that---not quite
impudence---though girls like to be played with, and rumpled a little
too, sometimes.  But she has told no tales, I assure you.

\2  I never gave her the slightest cause.

\3  Well, well, I like modesty in its place well enough.  But
this is over-acting, young gentleman.  You may be open.  Your father
and I will like you all the better for it.

\2  May I die, sir, if I ever------

\3  I tell you, she don't dislike you; and as I'm sure you
like her------

\2  Dear sir---I protest, sir------

\3  I see no reason why you should not be joined as fast as
the parson can tie you.

\2  But hear me, sir---

\3  Your father approves the match, I admire it; every
moment's delay will be doing mischief.  So---

\2  But why won't you hear me?  By all that's just and true, I
never gave Miss Hardcastle the slightest mark of my attachment, or even
the most distant hint to suspect me of affection.  We had but one
interview, and that was formal, modest, and uninteresting.

\3  \[Aside\]  This fellow's formal modest impudence is be\-yond
bearing.

\1  And you never grasped her hand, or made any
protestations?

\2  As Heaven is my witness, I came down in obedience to your
commands.  I saw the lady without emotion, and parted without
reluctance.  I hope you'll exact no farther proofs of my duty, nor
prevent me from leaving a house in which I suffer so many
mortifications.  \[r]Exit\]

\1  I'm astonished at the air of sincerity with which he
parted.

\3  And I'm astonished at the deliberate intrepidity of his
assurance.

\1  I dare pledge my life and honour upon his truth.

\3  Here comes my daughter, and I would stake my happiness
upon her veracity.


\(Enter \8\)


\3  Kate, come hither, child.  Answer us sincerely and
without reserve: has Mr. Marlow made you any professions of love and
affection?

\8  The question is very abrupt, sir.  But since you
require unreserved sincerity, I think he has.

\3  \[To \1\]  You see.

\1  And pray, madam, have you and my son had more than one
interview?

\8  Yes, sir, several.

\3  \[To \1\]  You see.

\1  But did be profess any attachment?

\8  A lasting one.

\1  Did he talk of love?

\8  Much, sir.

\1  Amazing!  And all this formally?

\8  Formally.

\3  Now, my friend, I hope you are satisfied.

\1  And how did he behave, madam?

\8  As most profest admirers do: said some civil things
of my face, talked much of his want of merit, and the greatness of
mine; mentioned his heart, gave a short tragedy speech, and ended with
pretended rapture.

\1  Now I'm perfectly convinced, indeed.  I know his
conversation among women to be modest and submissive: this forward
canting ranting manner by no means describes him; and, I am confident,
he never sat for the picture.

\8  Then, what, sir, if I should convince you to your
face of my sincerity?  If you and my papa, in about half an hour, will
place yourselves behind that screen, you shall hear him declare his
passion to me in person.

\1  Agreed.  And if I find him what you describe, all my
happiness in him must have an end.  \[r]Exit\]

\8  And if you don't find him what I describe---I fear my
happiness must never have a beginning.  \[r]Exeunt\]


\(\01 changes to the back of the Garden.\)


\(Enter \4\)


\4  What an idiot am I, to wait here for a fellow who probably
takes a delight in mortifying me.  He never intended to be punctual,
and I'll wait no longer.  What do I see?  It is he! and perhaps with
news of my Constance.


\(Enter Tony, booted and spattered.\)


\4  My honest 'squire!  I now find you a man of your word. 
This looks like friendship.

\5  Ay, I'm your friend, and the best friend you have in the world,
if you knew but all.  This riding by night, by the bye, is cursedly
tiresome.  It has shook me worse than the basket of a stage-coach.

\4  But how? where did you leave your fellow-travellers?  Are
they in safety?  Are they housed?

\5  Five and twenty miles in two hours and a half is no such bad
driving.  The poor beasts have smoked for it: rabbit me, but I'd rather
ride forty miles after a fox than ten with such varment.

\4  Well, but where have you left the ladies?  I die with
impatience.

\5  Left them!  Why where should I leave them but where I found
them?

\4  This is a riddle.

\5  Riddle me this then.  What's that goes round the house, and
round the house, and never touches the house?

\4  I'm still astray.

\5  Why, that's it, mon.  I have led them astray.  By jingo,
there's not a pond or a slough within five miles of the place but they
can tell the taste of.

\4  Ha! ha! ha! I understand: you took them in a round, while
they supposed themselves going forward, and so you have at last brought
them home again.

\5  You shall hear.  I first took them down Feather-bed Lane, where
we stuck fast in the mud.  I then rattled them crack over the stones of
Up-and-down Hill.  I then introduced them to the gibbet on Heavy-tree
Heath; and from that, with a circumbendibus, I fairly lodged them in
the horse-pond at the bottom of the garden.

\4  But no accident, I hope?

\5  No, no.  Only mother is confoundedly frightened.  She thinks
herself forty miles off.  She's sick of the journey; and the cattle can
scarce crawl.  So if your own horses be ready, you may whip off with
cousin, and I'll be bound that no soul here can budge a foot to follow
you.

\4  My dear friend, how can I be grateful?

\5  Ay, now it's dear friend, noble 'squire.  Just now, it was all
idiot, cub, and run me through the guts.  Damn \textit{your} way of fighting, I
say.  After we take a knock in this part of the country, we kiss and be
friends.  But if you had run me through the guts, then I should be
dead, and you might go kiss the hangman.

\4  The rebuke is just.  But I must hasten to relieve Miss
Neville: if you keep the old lady employed, I promise to take care of
the young one.  \[r]Exit \4\]

\5  Never fear me.  Here she comes.  Vanish.  She's got from the
pond, and draggled up to the waist like a mermaid.


\(Enter \7\)


\7  Oh, Tony, I'm killed!  Shook!  Battered to death.  I
shall never survive it.  That last jolt, that laid us against the
quickset hedge, has done my business.

\5  Alack, mamma, it was all your own fault.  You would be for
running away by night, without knowing one inch of the way.

\7  I wish we were at home again.  I never met so many
accidents in so short a journey.  Drenched in the mud, overturned in a
ditch, stuck fast in a slough, jolted to a jelly, and at last to lose
our way.  Whereabouts do you think we are, Tony?

\5  By my guess we should come upon Crackskull Common, about forty
miles from home.

\7  O lud! O lud!  The most notorious spot in all the
country.  We only want a robbery to make a complete night on't.

\5  Don't be afraid, mamma, don't be afraid.  Two of the five that
kept here are hanged, and the other three may not find us.  Don't be
afraid.---Is that a man that's galloping behind us?  No; it's only a
tree.---Don't be afraid.

\7  The fright will certainly kill me.

\5  Do you see anything like a black hat moving behind the thicket?

\7  Oh, death!

\5  No; it's only a cow.  Don't be afraid, mamma; don't he afraid.

\7  As I'm alive, Tony, I see a man coming towards us. 
Ah!  I'm sure on't.  If he perceives us, we are undone.

\5  \[Aside\]  Father-in-law, by all that's unlucky, come to take one
of his night walks.  \[To her\]  Ah, it's a highwayman with pistols as
long as my arm.  A damned ill-looking fellow.

\7  Good Heaven defend us!  He approaches.

\5  Do you hide yourself in that thicket, and leave me to manage
him.  If there be any danger, I'll cough, and cry hem.  When I cough,
be sure to keep close.  \[\7 hides behind a tree in the back scene\]


\(Enter \3\)


\3  I'm mistaken, or I heard voices of people in want of
help.  Oh, Tony! is that you?  I did not expect you so soon back.  Are
your mother and her charge in safety?

\5  Very safe, sir, at my aunt Pedigree's.  Hem.

\7  \[From behind\]  Ah, death!  I find there's danger.

\3  Forty miles in three hours; sure that's too much, my
youngster.

\5  Stout horses and willing minds make short journeys, as they say. 
Hem.

\7  \[From behind\]  Sure he'll do the dear boy no harm.

\3  But I heard a voice here; I should be glad to know from
whence it came.

\5  It was I, sir, talking to myself, sir.  I was saying that forty
miles in four hours was very good going.  Hem.  As to be sure it was. 
Hem.  I have got a sort of cold by being out in the air.  We'll go in,
if you please.  Hem.

\3  But if you talked to yourself you did not answer
yourself.  I'm certain I heard two voices, and am resolved \[raising his
voice\] to find the other out.

\7  \[From behind\]  Oh! he's coming to find me out.  Oh!

\5  What need you go, sir, if I tell you?  Hem.  I'll lay down my
life for the truth---hem---I'll tell you all, sir.  \[Detaining him\]

\3  I tell you I will not be detained.  I insist on seeing. 
It's in vain to expect I'll believe you.

\7  \[Running forward from behind\]  O lud! he'll murder
my poor boy, my darling!  Here, good gentleman, whet your rage upon me. 
Take my money, my life, but spare that young gentleman; spare my child,
if you have any mercy.

\3  My wife, as I'm a Christian.  From whence can she come? or
what does she mean?

\7  \[Kneeling\]  Take compassion on us, good Mr.
Highwayman.  Take our money, our watches, all we have, but spare our
lives.  We will never bring you to justice; indeed we won't, good Mr.
Highwayman.

\3  I believe the woman's out of her senses.  What, Dorothy,
don't you know \textit{me}?

\7  Mr. Hardcastle, as I'm alive!  My fears blinded me. 
But who, my dear, could have expected to meet you here, in this
frightful place, so far from home?  What has brought you to follow us?

\3  Sure, Dorothy, you have not lost your wits?  So far from
home, when you are within forty yards of your own door!  \[To him\] 
This is one of your old tricks, you graceless rogue, you.  \[To her\] 
Don't you know the gate, and the mulberry-tree; and don't you remember
the horse-pond, my dear?

\7  Yes, I shall remember the horse-pond as long as I
live; I have caught my death in it.  \[To \5\]  And it is to you, you
graceless varlet, I owe all this?  I'll teach you to abuse your mother,
I will.

\5  Ecod, mother, all the parish says you have spoiled me, and so
you may take the fruits on't.

\7  I'll spoil you, I will.  [Follows him off the stage. 
\[r]Exit\]

\3  There's morality, however, in his reply.  \[r]Exit\]


\(Enter \4 and \9\)


\4  My dear Constance, why will you deliberate thus?  If we
delay a moment, all is lost for ever.  Pluck up a little resolution,
and we shall soon be out of the reach of her malignity.

\9  I find it impossible.  My spirits are so sunk with the
agitations I have suffered, that I am unable to face any new danger. 
Two or three years' patience will at last crown us with happiness.

\4  Such a tedious delay is worse than inconstancy.  Let us fly,
my charmer.  Let us date our happiness from this very moment.  Perish
fortune!  Love and content will increase what we possess beyond a
monarch's revenue.  Let me prevail!

\9  No, Mr. Hastings, no.  Prudence once more comes to my
relief, and I will obey its dictates.  In the moment of passion fortune
may be despised, but it ever produces a lasting repentance.  I'm
resolved to apply to Mr. Hardcastle's compassion and justice for
redress.

\4  But tho' he had the will, he has not the power to relieve
you.

\9  But he has influence, and upon that I am resolved to
rely.

\4  I have no hopes.  But since you persist, I must reluctantly
obey you.  \[r]Exeunt\]


\(\01 changes.\)


\(Enter \1 and \8\)


\1  What a situation am I in!  If what you say appears, I
shall then find a guilty son.  If what he says be true, I shall then
lose one that, of all others, I most wished for a daughter.

\8  I am proud of your approbation, and to show I merit
it, if you place yourselves as I directed, you shall hear his explicit
declaration.  But he comes.

\1  I'll to your father, and keep him to the appointment. 

\[r]Exit \1\]


\(Enter \2\)


\2  Though prepared for setting out, I come once more to take
leave; nor did I, till this moment, know the pain I feel in the
separation.

\8  \[In her own natural manner\]  I believe sufferings
cannot be very great, sir, which you can so easily remove.  A day or
two longer, perhaps, might lessen your uneasiness, by showing the
little value of what you now think proper to regret.

\2  \[Aside\]  This girl every moment improves upon me.  \[To her\] 
It must not be, madam.  I have already trifled too long with my heart. 
My very pride begins to submit to my passion.  The disparity of
education and fortune, the anger of a parent, and the contempt of my
equals, begin to lose their weight; and nothing can restore me to
myself but this painful effort of resolution.

\8  Then go, sir:  I'll urge nothing more to detain you. 
Though my family be as good as hers you came down to visit, and my
education, I hope, not inferior, what are these advantages without
equal affluence?  I must remain contented with the slight approbation
of imputed merit; I must have only the mockery of your addresses, while
all your serious aims are fixed on fortune.


\(Enter \3 and \1 from behind.\)


\1  Here, behind this screen.

\3  Ay, ay; make no noise.  I'll engage my Kate covers him
with confusion at last.

\2  By heavens, madam! fortune was ever my smallest
consideration.  Your beauty at first caught my eye; for who could see
that without emotion?  But every moment that I converse with you steals
in some new grace, heightens the picture, and gives it stronger
expression.  What at first seemed rustic plainness, now appears refined
simplicity.  What seemed forward assurance, now strikes me as the
result of courageous innocence and conscious virtue.

\1  What can it mean?  He amazes me!

\3  I told you how it would be.  Hush!

\2  I am now determined to stay, madam; and I have too good an
opinion of my father's discernment, when he sees you, to doubt his
approbation.

\8  No, Mr. Marlow, I will not, cannot detain you.  Do
you think I could suffer a connexion in which there is the smallest
room for repentance?  Do you think I would take the mean advantage of a
transient passion, to load you with confusion?  Do you think I could
ever relish that happiness which was acquired by lessening yours?

\2  By all that's good, I can have no happiness but what's in your
power to grant me!  Nor shall I ever feel repentance but in not having
seen your merits before.  I will stay even contrary to your wishes; and
though you should persist to shun me, I will make my respectful
assiduities atone for the levity of my past conduct.

\8  Sir, I must entreat you'll desist.  As our
acquaintance began, so let it end, in indifference.  I might have
given an hour or two to levity; but seriously, Mr. Marlow, do you
think I could ever submit to a connexion where I must appear
mercenary, and you imprudent?  Do you think I could ever catch at the
confident addresses of a secure admirer?

\2  \[Kneeling\]  Does this look like security?  Does this look
like confidence?  No, madam, every moment that shows me your merit,
only serves to increase my diffidence and confusion.  Here let me
continue------

\1  I can hold it no longer.  Charles, Charles, how hast thou
deceived me!  Is this your indifference, your uninteresting
conversation?

\3  Your cold contempt; your formal interview!  What have you
to say now?

\2  That I'm all amazement!  What can it mean?

\3  It means that you can say and unsay things at pleasure:
that you can address a lady in private, and deny it in public: that you
have one story for us, and another for my daughter.

\2  Daughter!---This lady your daughter?

\3  Yes, sir, my only daughter; my Kate; whose else should she
be?

\2  Oh, the devil!

\8  Yes, sir, that very identical tall squinting lady you
were pleased to take me for \[courtseying\]; she that you addressed as
the mild, modest, sentimental man of gravity, and the bold, forward,
agreeable Rattle of the Ladies' Club.  Ha! ha! ha!

\2  Zounds! there's no bearing this; it's worse than death!

\8  In which of your characters, sir, will you give us
leave to address you?  As the faltering gentleman, with looks on the
ground, that speaks just to be heard, and hates hypocrisy; or the loud
confident creature, that keeps it up with Mrs. Mantrap, and old Miss
Biddy Buckskin, till three in the morning?  Ha! ha! ha!

\2  O, curse on my noisy head.  I never attempted to be impudent
yet, that I was not taken down.  I must be gone.

\3  By the hand of my body, but you shall not.  I see it was
all a mistake, and I am rejoiced to find it.  You shall not, sir, I
tell you.  I know she'll forgive you.  Won't you forgive him, Kate? 
We'll all forgive you.  Take courage, man.  \[They retire, she
tormenting him, to the back scene\]


\(Enter \7 and Tony.\)


\7  So, so, they're gone off.  Let them go, I care not.

\3  Who gone?

\7  My dutiful niece and her gentleman, Mr. Hastings,
from town.  He who came down with our modest visitor here.

\1  Who, my honest George Hastings?  As worthy a fellow as
lives, and the girl could not have made a more prudent choice.

\3  Then, by the hand of my body, I'm proud of the connexion.

\7  Well, if he has taken away the lady, he has not
taken her fortune; that remains in this family to console us for her
loss.

\3  Sure, Dorothy, you would not be so mercenary?

\7  Ay, that's my affair, not yours.

\3  But you know if your son, when of age, refuses to marry
his cousin, her whole fortune is then at her own disposal.

\7  Ay, but he's not of age, and she has not thought
proper to wait for his refusal.


\(Enter \4 and \9\)


\7  \[Aside\]  What, returned so soon!  I begin not to
like it.

\4  \[To \3\]  For my late attempt to fly off with your
niece let my present confusion be my punishment.  We are now come back,
to appeal from your justice to your humanity.  By her father's consent,
I first paid her my addresses, and our passions were first founded in
duty.

\9  Since his death, I have been obliged to stoop to
dissimulation to avoid oppression.  In an hour of levity, I was ready
to give up my fortune to secure my choice.  But I am now recovered from
the delusion, and hope from your tenderness what is denied me from a
nearer connexion.

\7  Pshaw, pshaw! this is all but the whining end of a
modern novel.

\3  Be it what it will, I'm glad they're come back to reclaim
their due.  Come hither, Tony, boy.  Do you refuse this lady's hand
whom I now offer you?

\5  What signifies my refusing?  You know I can't refuse her till
I'm of age, father.

\3  While I thought concealing your age, boy, was likely to
conduce to your improvement, I concurred with your mother's desire to
keep it secret.  But since I find she turns it to a wrong use, I must
now declare you have been of age these three months.

\5  Of age!  Am I of age, father?

\3  Above three months.

\5  Then you'll see the first use I'll make of my liberty.  \[Taking
\9's hand\]  Witness all men by these presents, that I,
Anthony Lumpkin, Esquire, of \textsc{blank} place, refuse you, Constantia
Neville, spinster, of no place at all, for my true and lawful wife.  So
Constance Neville may marry whom she pleases, and Tony Lumpkin is his
own man again.

\1  O brave 'squire!

\4  My worthy friend!

\7  My undutiful offspring!

\2  Joy, my dear George!  I give you joy sincerely.  And could I
prevail upon my little tyrant here to be less arbitrary, I should be
the happiest man alive, if you would return me the favour.

\4  \[To \8\]  Come, madam, you are now driven to
the very last scene of all your contrivances.  I know you like him, I'm
sure he loves you, and you must and shall have him.

\3  \[Joining their hands\]  And I say so too.  And, Mr.
Marlow, if she makes as good a wife as she has a daughter, I don't
believe you'll ever repent your bargain.  So now to supper.  To-morrow
we shall gather all the poor of the parish about us, and the mistakes
of the night shall be crowned with a merry morning.  So, boy, take her;
and as you have been mistaken in the mistress, my wish is, that you may
never be mistaken in the wife.  \[r]Exeunt Omnes\]

\spatium* {6ex plus 2ex\penalty 10000}
\Facies \titulus {\textsc{#1}}
\titulus{finis}%

\endProsa
\endDrama
\end{document}

