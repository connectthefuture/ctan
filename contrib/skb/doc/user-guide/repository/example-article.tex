\opt{text}{\skbheading{A Simple Article}}

Take the article that describes the state of the art in protocols. Remember, we 
have all the contents for that in our \skbem[code]{repository} directory. 
We go the directory that has the published articles 
\skbem[code]{published/articles} and create a new file say 
\skbem[code]{protocols.tex} \opt{note}{as shown on this slide and the following two slides}.

\opt{text}{\lstinputlisting[style=generic,language=TeX]{\skbfileroot{examples/example}}}

The article uses the class \skbem[code]{skbarticle}. That class will load the \skbacft{A3DS:SKB} package and
the memoir class and do all settings we need. It prepares the title page and 
prints the table of contents like any other \LaTeX~article. It uses \cmd{\skbinput}
to load files from the repository. The first one is loaded without requesting a level.
In other words, there is some text right at the beginning of our article, without 
any special heading, like an abstract.

Then we do start the section 'Introduction' and collect a few files with their heading 
categorised as sub-sections. Reading the directory and file names, we can already guess
what the introduction will be doing: it introduces general protocol concepts with regard
to data encoding, protocol message formats, protocols themselves and protocol services. 
The last block loads four files with headings categorised as sections.
Using the directory names, we see that the remaining article 
describes the protocols \ac{OMG:GIOP} defined by the \ac{organisation:OMG}, \ac{IETF:SNMP} by the
\ac{organisation:IETF}, \ac{ITU:CMIP} by the \ac{organisation:ITU} and finally \ac{W3C:HTTP} by the 
\ac{organisation:W3C}.

Finally, we load acronyms and bebliography and finishing the article.
This example will create a table of contents similar to this:

\lstinputlisting[style=generic]{\skbfileroot{examples/example-toc}}

Job done. Now we can use \LaTeX~or PDF-\LaTeX~to compile our article. 
