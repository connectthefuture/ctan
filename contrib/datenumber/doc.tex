%%
%% This is file `docgerman.tex'.
%% 
%% This file is part of the `datenumber' package,
%% 
%% Copyright (C) 2001 J"org-Michael Schr"oder.
%% 
%% IMPORTANT NOTICE:
%% 
%% This program may be distributed and/or modified under the
%% conditions of the LaTeX Project Public License, either version 1.2
%% of this license or (at your option) any later version.
%% The latest version of this license is in
%% http://www.latex-project.org/lppl.txt
%% and version 1.2 or later is part of all distributions of LaTeX
%% version 1999/12/01 or later.
%% 
%% This program consists of all files listed in README.txt.
%% 
%% For error reports, or offers to help make this a more powerful
%% tool, please contact me on schrj020@mail.uni-mainz.de
%% 

\documentclass[draft]{article}
\usepackage{datenumber}

\oddsidemargin 0.25in
\textwidth 5.875 in
\addtolength{\textheight}{8\baselineskip}
\addtolength{\topmargin}{-5\baselineskip}

\title{The {\tt datenumber.sty} package \fileversion}

\author{J\"org-Michael Schr\"oder\\{\small\texttt{schrj020@mail.uni-mainz.de}}}

\date{\filedate}

\newcounter{dateone}\newcounter{datetwo}%
\newcommand{\daydifftoday}[3]{%
  \setmydatenumber{dateone}{\the\year}{\the\month}{\the\day}%
  \setmydatenumber{datetwo}{#1}{#2}{#3}%
  \addtocounter{datetwo}{-\thedateone}%
  \thedatetwo
}

\newcommand{\sd}{%
\ifcase\thedatedayname \or
    Mon\or Tue\or Wed\or Thu\or
    Fri\or Sat\or Sun\fi
}%

\newcommand{\pnext}{%
\thedateyear/%
\ifnum\value{datemonth}<10 0\fi
\thedatemonth/%
\ifnum\value{dateday}<10 0\fi
\thedateday%
\nextdate
}

\begin{document}
\maketitle
\begin{abstract}

This package provides commands to convert a date into a number. Turned around a date can be calculated also by a number. Additionally there are commands for incrementing and decrementing a date. Leap years and the Gregorian calendar reform are considered.
\end{abstract}

\tableofcontents
\section{Start year}
The start of the counting is determined with \verb+\setstartyear{year}+ (standard 1800). The first day of the start year gets the number 1. The value of \texttt{startyear} must be greater 0. It may not be larger than the year of a date to be calculated. If the difference of date and \texttt{startyear} is large, the calculation can last for a long time. The correct setting of the weekdays is guaranteed only if the value of \texttt{startyear} is 1800, 1900 or 2000.
\section{Counters}
There are five counters defined
\begin{description}
\item[\texttt{datenumber}:] number of the day
\item[\texttt{dateyear}:] year
\item[\texttt{datemonth}:] month
\item[\texttt{dateday}:] day
\item[\texttt{datedayname}:] weekday: 1--7 (Monday--Sunday)
\end{description}
\section{Macros}
\subsection{Macros which operate with defined counters\label{macro}}
All counters specified above are updated by these macros. \verb+\datedayname+ and \verb+\datemonthname+ are also updated.
\begin{description}
\item[\texttt{\textbackslash setdatenumber\{year\}\{month\}\{day\}}:] Sets the counter \texttt{datenumber} to a value, which corresponds to the date.
\item[\texttt{\textbackslash setdatebynumber\{number\}}:] Sets the counters \texttt{dateyear}, \texttt{datemonth}, and \texttt{dateday} to values, which corresponds to the number.
\item[\texttt{\textbackslash nextdate}:] Sets the counters \texttt{dateyear}, \texttt{datemonth}, and \texttt{dateday} to the next date.
\item[\texttt{\textbackslash prevdate}:] Sets the counters \texttt{dateyear}, \texttt{datemonth}, and \texttt{dateday} to the previous date.
\item[\texttt{\textbackslash setdate\{year\}\{month\}\{day\}}:] Sets the counters \texttt{dateyear}, \texttt{datemonth}, and \texttt{dateday} to \texttt{year}, \texttt{month}, and \texttt{day}.
\item[\texttt{\textbackslash setdatetoday}:] Sets the counters \texttt{dateyear}, \texttt{datemonth}, and \texttt{dateday} to the current date.
\item[\texttt{\textbackslash datemonthname}:] typesets the month (see section \ref{monthname}).
\item[\texttt{\textbackslash datedayname}:] typesets the weekday (see section \ref{dayname}).
\item[\texttt{\textbackslash datedate}:] typesets the date, corresponding to the counters \texttt{dateyear}, \texttt{datemonth}, \texttt{dateday}.
\end{description}
\subsection{Macros which operate with your own counters}
Only the counters you specified are updated by these macros. \verb+\datedayname+ and \verb+\datemonthname+ are not updated.
\begin{description}\sloppypar
\item[\texttt{\textbackslash setmydatenumber\{numbercount\}\{year\}\{month\}\{day\}}:] Sets the counter \texttt{numbercount} to a value, which corresponds to the date.
\item[\texttt{\textbackslash setmydatebynumber\{number\}\{yearcount\}\{monthcount\}\{daycount\}}:] Sets the counters \texttt{yearcount}, \texttt{monthcount}, and \texttt{daycount} to values, which corresponds to the number.
\item[\texttt{\textbackslash mynextdate\{yearcount\}\{monthcount\}\{daycount\}}:] Sets the counters \texttt{yearcount}, \texttt{monthcount}, and \texttt{daycount} to the next date.
\item[\texttt{\textbackslash mynextdate\{yearcount\}\{monthcount\}\{daycount\}}:]Sets the counters \texttt{yearcount}, \texttt{monthcount}, and \texttt{daycount} to the previous date.
\end{description}

\subsection{Month\label{monthname}}
The command \verb+\datemonthname+ typesets the month. It is updated by macros described in section \ref{macro}. You can do this by your own saying \verb+\setmonthname{number}+.

\subsection{Weekday\label{dayname}}
To typeset the weekday say \verb+\datedayname+. This command is updated by macros described in section \ref{macro}.
You can do this by your own saying \verb+\setmonthname{number}+ (1 for Monday and 7 for Sunday). You can also write \verb+\setdaynamebynumber{number}+, were \verb+number+ is the number of a date. If \texttt{startyear} is set to 1800, 1900 or 2000 the calculation of the weekday will work.

\section{Language}\sloppypar
The language options \texttt{english}, \texttt{USenglish} (standard), \texttt{french}, \texttt{spanish}, \texttt{german}, and \texttt{ngerman} are supported. Say \verb+\dateselectlanguage{language}+ to select a language. For other languages: Create a file \texttt{datenumbermylanguage.ldf}. Copy the text from \texttt{datenumberdummy.ldf}. Replace every ``dummy'' with ``mylanguage'' and change the months and weekdays. Say \verb+\usepackage{datenumber}+ \verb+\input{datenumbermylanguage.ldf}+ in your document.

\section{Examples}
\begin{quote}
\begin{verbatim}
\setdate{2002}{1}{1}%
\thedatenumber
\end{verbatim}
\setdate{2002}{1}{1}%
Result: \thedatenumber
\end{quote}

\begin{quote}
\begin{verbatim}
\setdatetoday
\addtocounter{datenumber}{10}%
\setdatebynumber{\thedatenumber}%
In 10 days is \datedate
\end{verbatim}
\setdatetoday
\addtocounter{datenumber}{10}%
\setdatebynumber{\thedatenumber}%
Result: In 10 days is \datedate
\end{quote}

\begin{quote}
\begin{verbatim}
\newcounter{dateone}\newcounter{datetwo}%

\newcommand{\daydifftoday}[3]{%
  \setmydatenumber{dateone}{\the\year}{\the\month}{\the\day}%
  \setmydatenumber{datetwo}{#1}{#2}{#3}%
  \addtocounter{datetwo}{-\thedateone}%
  \thedatetwo
}

There is still \daydifftoday{\the\year}{12}{25} days to Christmas.
\end{verbatim}
Result: There is still \daydifftoday{\the\year}{12}{25} days to Christmas.
\end{quote}

\clearpage
\begin{quote}
\begin{verbatim}
\newcommand{\sd}{%
\ifcase\thedatedayname \or
    Mon\or Tue\or Wed\or Thu\or
    Fri\or Sat\or Sun\fi
}%

\newcommand{\pnext}{%
\thedateyear/%
\ifnum\value{datemonth}<10 0\fi
\thedatemonth/%
\ifnum\value{dateday}<10 0\fi
\thedateday%
\nextdate
}

\setdate{2001}{9}{29}%
\[\begin{tabular}{lll}
\sd & \pnext & Abc\\
\sd & \pnext & Def\\
\sd & \pnext & Ghi\\
\sd & \pnext & Jkl\\
\end{tabular}\]
\end{verbatim}
Result: \setdate{2001}{9}{29}%
\[\begin{tabular}{lll}
\sd & \pnext & Abc\\
\sd & \pnext & Def\\
\sd & \pnext & Ghi\\
\sd & \pnext & Jkl\\
\end{tabular}\]
\end{quote}

\section{Other}
\begin{itemize}
\item leap year test
\begin{quote}
\begin{verbatim}
The \the\year\ is
\ifleapyear{\the\year} a \else no \fi leap year.
\end{verbatim}
Result: The \the\year\ is \ifleapyear{\the\year} a \else no \fi leap year.
\end{quote}
\item date test
\begin{quote}
\begin{verbatim}
The 29.2.1900 is
\ifvaliddate{1900}{2}{29} a \else no \fi valid date.
\end{verbatim}
Result: The 29.2.1900 is \ifvaliddate{1900}{2}{29} a \else no \fi valid date.%
\footnote{There are programs, which have another opinion about that. Search for ''Gregorian calendar'' in order to get more information about leap years and October 5, 1582}
\end{quote}
\end{itemize}

\end{document}
\endinput
%%
%% End of file `doc.tex'.
