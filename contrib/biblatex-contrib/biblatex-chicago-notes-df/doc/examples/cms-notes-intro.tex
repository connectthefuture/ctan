\documentclass[a4paper,12pt]{article}
\usepackage[T1]{fontenc}
\usepackage{textcomp}
\usepackage{endnotes}
\usepackage[latin1]{inputenc}
\usepackage[american]{babel}
\usepackage[autostyle]{csquotes}
%\usepackage[document]{ragged2e}
\usepackage[notes,strict,backend=biber,autolang=other,%
bibencoding=latin1,booklongxref=false,annotation]{biblatex-chicago}
\usepackage{lmodern}
\usepackage{gentium}
%\renewcommand*{\rmdefault}{fgn}% The font (gentium) used for pdf
\usepackage{ifthen}
\usepackage{setspace}
\usepackage{vmargin} \setpapersize{A4}
\setmarginsrb{1in}{20pt}{1in}{.5in}{1pt}{2pt}{0pt}{6mm}
\makeatletter 
\renewcommand{\@makeenmark}{\textcolor{DarkSlateGrey}{\textsf{\@theenmark}}}
\makeatother
\usepackage{url}
\urlstyle{rm}
\appto\bibsetup{\sloppy}
\hyphenation{tech-re-port}
\protected\def\onethird{{\mbox{\scriptsize\raisebox{.7ex}{1}%
    \hspace{-0.1em}\raisebox{.2ex}{/}\hspace{-0.03em}3}}}
\setlength{\parindent}{0pt}
%\setlength{\parskip}{5pt}
\setlength{\dimen\footins}{9.5in}
\newcommand{\mycolor}{}%[1]{\textcolor[HTML]{228B22}{#1}}
\usepackage{xr-hyper}
\externaldocument[cms-]{../../Docs/biblatex-chicago}%
\usepackage[pdftex,hyperref,svgnames]{xcolor}
\usepackage[pdftex,colorlinks,filecolor=Teal,citecolor=black,
plainpages=false,breaklinks=true,urlcolor=DarkSlateBlue,
linkcolor=DarkSlateBlue,baseurl=biblatex-chicago.pdf\#]{hyperref}
\usepackage{cmsdocs}
\newcommand{\cmd}[1]{\texttt{\textbackslash #1}}
\newcommand{\mylittlespace}{\vspace{5pt}}%.5\baselineskip}}
\bibliography{notes-test}
%%\onehalfspacing
%\tracingstats=2
\begin{document}

\section*{The Chicago Notes \&\ Bibliography Specification}
\label{sec:spec}
  
This file is intended as a brief introduction to the Chicago notes \&\
bibliography specification (16th ed.)\autocite{chicago:manual}\ as
implemented by \textsf{biblatex-chicago}, and falls somewhere in
between the \enquote{Quickstart} section of
\textsf{biblatex-chicago.pdf} and the full documentation as presented
in section~4 \cmssecref{cms-sec:Spec} of that same document.  I've
attempted to design this introduction for ease of cross-reference, so
clicking on long-note citations should bring you to the bibliography
entry, whence clicking on the entry key in the annotation should
present you with the entry as it appears in the .bib file, where
clicking on the entry type should return you to the long note.  If you
have questions beyond the scope of this introduction, then the full
documentation is the place to look next --- marginal notes here refer
to section or page numbers there, and if you've installed the package
using the standard \TeX\ Live method then clicking on these marginal
notes should take you to the other document.  If you can't find
answers there, please write to me at the email address in
\textsf{biblatex-chicago.pdf}.

\subsection*{Standard entry types}
\label{sec:standard}

\begin{refsection}
  These \cmssecref{cms-sec:entrytypes} should pose no particular
  issues to those who have used \textsc{Bib}\TeX\ or \textsf{biblatex}
  before, but here is an example of each of the following standard
  entry types:
  \endnote[\value{Article}]{\cite{garaud:gatine}.},
  \endnote[\value{Book}]{\cite{mchugh:wake}.},
  \endnote[\value{Booklet}]{\cite{clark:mesopot}.},
  \endnote[\value{InBook}]{\cite{ashbrook:brain}.},
  \endnote[\value{InCollection}]{\cite{contrib:contrib}.},
  \endnote[\value{InProceedings}]{\cite{frede:inproc}.},
  \endnote[\value{Manual}]{\cite{dyna:browser}.},
  \endnote[\value{MastersThesis}]{\cite{ross:thesis}.},
  \endnote[\value{TechReport}]{\cite{herwign:office}.}, and
  \endnote[\value{Unpublished}]{\cite{nass:address}.}.

  {\renewcommand{\notesname}{\normalsize Long-Note Style} \theendnotes}
  \printbibliography[title=\normalsize Bibliography Style (with
  annotations)]
\end{refsection}

\subsection*{Other entry types}
\label{sec:other}

\begin{refsection}

  These \cmssecref{cms-sec:entrytypes} entry types are
  \textsf{biblatex} innovations, designed to cater for as large a
  range of reference needs as possible.  The list here is by no means
  exhaustive, but rather tries to exemplify some of the more
  complicated or (possibly) unfamiliar entry types, including:
  \endnote[\value{Artwork}]{\cite{leo:madonna}.},
  \endnote[\value{Audio}]{\cite{schubert:muellerin}.},
  \endnote[\value{BookInBook}]{\cite{euripides:orestes}.},
  \endnote[\value{InReference}]{\cite{wikiped:bibtex}.},
  \endnote[\value{Letter}]{\cite{jackson:paulina:letter}.},
  \endnote[\value{Music}]{\cite{holiday:fool}.},
  \endnote[\value{Review}]{\cite{ratliff:review}.},
  \endnote[\value{SuppBook}]{\cite{polakow:afterw}.}, and
  \endnote[\value{Video}]{\cite{friends:leia}.}.

  {\renewcommand{\notesname}{\normalsize Long-Note Style} \theendnotes}
  \printbibliography[title=\normalsize Bibliography Style]
\end{refsection}

\subsection*{Short notes}
\label{sec:short}

\begin{refsection}

  The \cmssecref[shortauthor]{cms-sec:shortauthor} note forms we've
  seen so far are intended to appear on the first citation of a given
  work, while subsequent citations use a shorter form, usually merely
  \textsf{Author}, \textsf{Title}.  Both of these fields, of course,
  have a \textsf{short} form to allow space-saving abridgements of
  names and titles.  You can also use the option \texttt{short} when
  you load \textsf{biblatex-chicago} and you'll get the short form
  from the start, something only recommended by the \emph{CMS} when
  you have a full bibliography to clarify all the abbreviated
  references.  The following are the short forms of all the works
  cited in long notes in previous sections:
  \endnote[\value{Article}]{\shortcite{garaud:gatine}.},
  \endnote[\value{Artwork}]{\shortcite{leo:madonna}.},
  \endnote[\value{Audio}]{\shortcite{schubert:muellerin}.},
  \endnote[\value{Book}]{\shortcite{mchugh:wake}.},
  \endnote[\value{BookInBook}]{\shortcite{euripides:orestes}.},
  \endnote[\value{Booklet}]{\shortcite{clark:mesopot}.},
  \endnote[\value{InBook}]{\shortcite{ashbrook:brain}.},
  \endnote[\value{InCollection}]{\shortcite{contrib:contrib}.},
  \endnote[\value{InProceedings}]{\shortcite{frede:inproc}.},
  \endnote[\value{InReference}]{\shortcite[Aristotle]{wikiped:bibtex}.},
  \endnote[\value{Letter}]{\shortcite{jackson:paulina:letter}.},
  \endnote[\value{Manual}]{\shortcite{dyna:browser}.},
  \endnote[\value{MastersThesis}]{\shortcite{ross:thesis}.},
  \endnote[\value{Music}]{\shortcite{holiday:fool}.},
  \endnote[\value{Review}]{\shortcite{ratliff:review}.},
  \endnote[\value{SuppBook}]{\shortcite{polakow:afterw}.},
  \endnote[\value{TechReport}]{\shortcite{herwign:office}.},
  \endnote[\value{Unpublished}]{\shortcite{nass:address}.}, and
  \endnote[\value{Video}]{\shortcite{friends:leia}.}.

  {\renewcommand{\notesname}{\normalsize Short-Note Style} \theendnotes}
\end{refsection}

\subsection*{The \textsf{entrysubtype} field}
\label{sec:subtype}

\begin{refsection}
  The \cmssecref[entrysubtype]{cms-sec:entrysub} Chicago notes \&\
  bibliography style covers a wide variety of source materials, so it
  is perhaps no surprise that even the range of entry types offered by
  \textsf{biblatex} isn't quite sufficient.  In many cases, the
  \textsf{entrysubtype} field can further expand the repertoire
  available to users.  Such cases include, in particular, the
  periodical types, where the \emph{CMS} differentiates between
  articles and reviews in scholarly journals and those in magazines
  and newspapers aimed at a more general readership.  For the latter
  two sorts of source, you place the string \texttt{magazine} in the
  \textsf{entrysubtype} field, and the citation style changes
  accordingly:
  \endnote[\value{Article}]{\cite{lakeforester:pushcarts}.} and
  \endnote[\value{Review}]{\cite{bundy:macneil}.}.

  \mylittlespace The \textsf{Misc} type provides a second
  differentiating function for the \textsf{entrysubtype} field.
  Without such a field, \textsf{Misc} entries function as they do in
  standard \textsf{biblatex} and in \textsc{Bib}\TeX, that is, as
  hold-alls for sources that won't easily fit into other categories.
  (Ideally, such entries will be very rare when using
  \textsf{biblatex-chicago}.)  With an \textsf{entrysubtype},
  \textsf{Misc} entries will present their source as part of an
  unpublished archive, to be distinguished from \textsf{Unpublished}
  entries, which usually will have a specific title and won't come
  from a named archive:
  \endnote[\value{Misc}]{\headlesscite{creel:house}.}.

  \mylittlespace The \textsf{entrysubtype} field is, finally, also
  useful for presenting pre-Renaissance works by their traditional
  divisions into books, sections, lines, etc., divisions which are
  presumed to be the same across all editions.  For such citations,
  you put the string \texttt{classical} into the \textsf{entrysubtype}
  field, and though this has no effect on long notes or in the
  bibliography, it changes the punctuation in short notes, as below:
  \endnote[\value{BookInBook}]{\shortcite[360e--361b]{plato:republic:gr}.}.
  (Were you citing such a work by the pages in a modern edition, the
  \textsf{entrysubtype} would be unnecessary --- see the Euripides
  citation above.)

  {\renewcommand{\notesname}{\normalsize Note Style} \theendnotes}
  \printbibliography[title=\normalsize Bibliography Style]
\end{refsection}

\subsection*{Abbreviated references }
\label{sec:abbrev}

\begin{refsection}

  The \cmssecref[crossref]{cms-sec:crossref} \emph{CMS} suggests, as a
  space-saving measure, that when multiple parts of a single
  collection are present in a reference apparatus, then references
  may, following certain rules, abbreviate the portion that refers to
  the collection as a whole.  \textsf{Biblatex-chicago} implements
  this recommendation using a combination of package options (both
  entry and preamble) and the \textsf{crossref} and \textsf{xref}
  fields.  In \textsf{InBook}, \textsf{InCollection},
  \textsf{InProceedings}, and \textsf{Letter} entries, the option is
  \texttt{longcrossref}, set to \texttt{false} by default, so if more
  than one such entry cross-references the same parent entry, then the
  abbreviated notes and bibliography entries will automatically
  appear.  The first full note citing such a source is not
  abbreviated, but all subsequent notes, and all bibliography entries,
  are:
  \endnote[\value{InCollection}]{\cite{ellet:galena}.},
  \endnote[\value{InCollection}]{\cite{keating:dearborn}.},
  \endnote[\value{Collection}]{\cite{prairie:state}.}.

\mylittlespace The possible settings for the \texttt{longcrossref}
option are \texttt{true} (no abbreviated references); \texttt{false}
(abbreviated references in notes and bibliography); \texttt{notes}
(abbreviated references only in the bibliography); \texttt{bib}
(abbreviated references only in notes); and \texttt{none} (abbreviated
references everywhere, including in the four entry types controlled by
the \texttt{booklongxref} option).

\mylittlespace The four entry types subject to the
\texttt{booklongxref} option are \texttt{Book}, \texttt{BookInBook},
\texttt{Collec\-tion}, and \texttt{Proceedings}.  The option has the
same four settings as \texttt{longcrossref}, excluding the
\texttt{none} switch, but it is set to \texttt{true} by default,
because the \emph{CMS} isn't as explicit in condoning abbreviated
references in such entry types, so you have to turn them on yourself,
as I have in this document using \texttt{booklongxref=false} in the
preamble when loading \texttt{biblatex-chi\-ca\-go}:
  \endnote[\value{Collection}]{\cite{harley:ancient:cart}.},
  \endnote[\value{Collection}]{\cite{harley:cartography}.},
  \endnote[\value{MVCollection}]{\cite{harley:hoc}.}.

  {\renewcommand{\notesname}{\normalsize Note Style} \theendnotes}
  \printbibliography[title=\normalsize Bibliography Style]
\end{refsection}

\subsection*{Online materials}
\label{sec:online}

\begin{refsection}

  With \cmssecref[online]{cms-sec:online} online sources, the
  \emph{CMS} emphasizes the nature of the source rather more than the
  place where that source is found.  This means, for example, that an
  online edition of a book calls for a
  \endnote[\value{Book}]{\cite{james:ambassadors}.} entry.  Even an
  intrinsically online source, if it is structured more or less like a
  conventional printed periodical, may demand an
  \endnote[\value{Article}]{\cite{stenger:privacy}.} or
  \textsf{Review} entry rather than an \textsf{Online} one.  Blogs
  lend themselves well to the
  \endnote[\value{Article}]{\cite{ellis:blog}.} type, while a comment
  on a blog becomes a
  \endnote[\value{Review}]{\cite{ac:comment}.}.  For things like
  mailing lists or less journalistic web pages, the
  \endnote[\value{Online}]{\cite{powell:email}.} type works well, as
  it does for short online videos
  (\endnote[\value{Online}]{\cite{pollan:plant}.}) and for short
  online audio pieces, too:
  \endnote[\value{Online}]{\cite{coolidge:speech}.}.  (Some of these
  latter also work well in a \textsf{Misc} entry with an
  \textsf{entrysubtype}.)

  {\renewcommand{\notesname}{\normalsize Note Style} \theendnotes}
  \printbibliography[title=\normalsize Bibliography Style]
\end{refsection}

\subsection*{Related entries}
\label{sec:related}

\begin{refsection}

  \textsf{Biblatex} provides \cmssecref{cms-sec:related} a powerful
  mechanism, using the \textsf{related} field, for grouping two (or
  more) works together in a single entry in the bibliography and/or in
  long notes, while \textsf{biblatex-chicago} offers both this
  functionality and some Chicago-specific variants which employ
  different means.  You can find a full discussion of this in
  \textsf{biblatex-chicago.pdf}, but one of the entries already cited
  in the previous section (\cmslink{coolidge:speech}) presents the two
  entries together in both notes and bibliography, whereas a second
  example places a text and its translation together, but only in the
  bibliography: \endnote[\value{Book}]{\cite{furet:related}.}.

  {\renewcommand{\notesname}{\normalsize Note Style} \theendnotes}
  \printbibliography[title=\normalsize Bibliography Style]
\end{refsection}

\subsection*{Citation commands}
\label{sec:citation}

\begin{refsection}

  Although \cmssecref{cms-sec:citecommands} \cmd{autocite} will no
  doubt be the most commonly used citation command,
  \textsf{biblatex-chicago}, following \textsf{biblatex}, does provide
  a range of other commands for more specialized usages.  We have
  already seen
  \cmd{headlesscite},\footnote[1]{\headlesscite{creel:house}.} which
  allows you to avoid, in notes, repetition of an \textsf{author's}
  name when it appears in the \textsf{title} as well.  We have also
  seen \cmd{full\-cite},\footnote{\fullcite{loc:leaders}.} to
  guarantee a long note, and
  \cmd{shortcite},\footnote{\shortcite{coolidge:speech}.} to guarantee
  a short one.  (You can also use \cmd{footfullcite} to get a
  \cmd{fullcite} in a footnote.)  There are a few others that may
  occasionally be useful:
  \cmd{surnamecite},\footnote{\surnamecite{harley:hoc}.} for when a
  note follows a discussion where the presence of the
  \textsf{authors'} (or \textsf{editors'}, etc.)\ full names makes
  their full repetition in the note unnecessary;
  \cmd{citejournal},\footnote{\citejournal{lakeforester:pushcarts}.}
  which provides an alternative short form when citing
  \textsf{Articles}; and the standard \cmd{textcite}, which inserts
  the name of an author or other \textcite{contrib:contrib} into the
  flow of text, with a footnote below.

%  {\renewcommand{\notesname}{\large Note Style} \theendnotes}
%  \printbibliography[title=\large Bibliography Style]
\end{refsection}

\subsection*{In conclusion}
\label{sec:conclude}

Allow me, finally, to emphasize just how multifarious are the sources
illustrated in the \emph{CMS}, only a small selection of which have
appeared in this introduction.  You will find significantly fuller
guidance in \textsf{biblatex.pdf} and \textsf{biblatex-chicago.pdf},
but the \emph{CMS} itself defines the specification and shall
arbitrate all disputes.  If you see something in
\textsf{biblatex-chicago} that looks wrong to you, or if the
documentation has left you perplexed, please let me know.

\printbibliography[title=\large References]

\twocolumn[\Large \texttt{The Database File}]
\vspace*{-6pt}
\begin{lstlisting}[language=BibTeX,label=prologue]
%% Database entries used to produce
%% citations in this file, taken
%% from briefintro.bib. I have
%% removed the annotations to save
%% room -- they can be viewed in the
%% main text above. You can click on
%% the entry type to return to the
%% long-note formats, and you can
%% click on text with a grey back-
%% ground to switch to that entry
%% within this .bib listing.

@String{uchp = {University of Chicago Press}}
\end{lstlisting}
\begin{lstlisting}[language=BiBTeX,label=ac:comment]
*\lnbackref{Review}{6}*{ac:comment,
  entrysubtype = {magazine},
  date = 	 {2008-06-30},
  author = 	 {AC},
  eventdate = 	 {2008-07-01},
  nameaddon = 	 {(10:18 a.m.)},
  crossref = 	 *\hyperlink{\getrefbykeydefault{ellis:blog}{anchor}{}}{\{\colorbox{Gainsboro}{ellis:blog}\}}*,
  title = 	 {comment on Rhian Ellis, \mkbibquote{Squatters' Rights}}
}
\end{lstlisting}
\begin{lstlisting}[language=BibTeX,label=ashbrook:brain]
*\lnbackref{InBook}{1}*{ashbrook:brain,
  author = 	 {Ashbrook, James~B. and Albright, Carol Rausch},
  title = 	 {The Frontal Lobes, Intending, and a Purposeful God},
  booktitle = 	 {The Humanizing Brain},
  publisher = {Pilgrim Press},
  year = 	 1997,
  chapter = 	 7,
  location =  {Cleveland, OH},
  shorttitle = {The Frontal Lobes}
}
\end{lstlisting}
\begin{lstlisting}[language=BibTeX,label=bundy:macneil]
*\lnbackref{Review}{4}*{bundy:macneil,
  journaltitle = {MacNeil/Lehrer NewsHour},
  usera = 	 {PBS},
  entrysubtype = {magazine},
  date = 	 {1990-02-07},
  author = 	 {Bundy, McGeorge},
  title = 	 {interview by Robert MacNeil},
  shorttitle = {interview}
}
\end{lstlisting}
\begin{lstlisting}[language=BibTeX,label=chicago:manual]
*\hyperlink{Hfootnote.1}{\color{DarkBlue}@Book}*{chicago:manual,
  title = 	 {The Chicago Manual of Style},
  year = 	 2010,
  author = 	 {{University of Chicago Press}},
  publisher = uchp,
  edition = 	 16,
  location =  {Chicago}
}
\end{lstlisting}
\begin{lstlisting}[language=BibTeX,label=clark:mesopot]
*\lnbackref{Booklet}{1}*{clark:mesopot,
  title = 	 {Mesopotamia},
  subtitle = 	 {Between Two Rivers},
  author = 	 {Hazel V. Clark},
  howpublished = {End of the Commons General Store},
  year = 	 {\mkbibbrackets{1957?}},
  location =     {Mesopotamia, OH}
}
\end{lstlisting}
\begin{lstlisting}[language=BibTeX,label=contrib:contrib]
*\lnbackref{InCollection}{1}*{contrib:contrib,
  author = 	 {Contributor, Anna},
  title = 	 {Contribution},
  booktitle = 	 {Edited Volume},
  publisher = {Publisher},
  year = 	 {forthcoming},
  editor = 	 {Editor, Ellen},
  location =  {Place}
}
\end{lstlisting}
\begin{lstlisting}[language=BibTeX,label=coolidge:speech]
*\lnbackref{Online}{6}*{coolidge:speech,
  author = 	 {Coolidge, Calvin},
  title = 	 {Equal Rights},
  titleaddon =   {(speech)},
  note = 	 {copy of an undated 78 rpm disc},
  related = 	 {*\hyperlink{\getrefbykeydefault%
{loc:leaders}{anchor}{}}{\colorbox{Gainsboro}{loc:leaders}}*},
  options =      {related=true,ptitleaddon=
        space,ctitleaddon=space},
  year = 	 {ca.\,1920}
}
\end{lstlisting}
\begin{lstlisting}[language=BibTeX,label=creel:house]
*\lnbackref{Misc}{4}*{creel:house,
  author = 	 {Creel, George},
  entrysubtype = {letter},
  title = 	 {George Creel to Colonel House},
  origdate = 	 {1918-09-25},
  note = 	 {Edward~M. House Papers},
  organization =  {Yale University Library}
}
\end{lstlisting}
\begin{lstlisting}[language=BibTeX,label=dyna:browser]
*\lnbackref{Manual}{1}*{dyna:browser,
  title = 	 {Dynatext, Electronic Book Indexer/Browser},
  organization = {Electronic Book Technology Inc.},
  address = 	 {Providence, RI},
  year = 	 1991,
  shorttitle = {Dynatext}
}
\end{lstlisting}
\begin{lstlisting}[language=BibTeX,label=ellet:galena]
*\lnbackref{InCollection}{5}*{ellet:galena,
  author = 	 {Ellet, Elizabeth~F.~L.},
  title = 	 {By Rail and Stage to Galena},
  crossref =  *\hyperlink{\getrefbykeydefault{prairie:state}%
{anchor}{}}{\{\colorbox{Gainsboro}{prairie:state}\}}*,
  pages = 	 {271--79}
}
\end{lstlisting}
\begin{lstlisting}[language=BibTeX,label=ellis:blog]
*\lnbackref{Article}{6}*{ellis:blog,
  author = 	 {Ellis, Rhian},
  title = 	 {Squatters' Rights},
  journaltitle = {Ward Six},
  location = 	 {blog},
  date = 	 {2008-06-30},
  url = 	 {http://wardsix.blogspot.com/ 2008/06/sqatters-rights.html},
  entrysubtype = {magazine}
}
\end{lstlisting}
\begin{lstlisting}[language=BibTeX,label=euripides:orestes]
*\lnbackref{BookInBook}{2}*{euripides:orestes,
  title = 	 {Orestes},
  year = 	 1958,
  booktitle = 	 {Euripides},
  maintitle = 	 {The Complete Greek Tragedies},
  nameb = 	 {Arrowsmith, William},
  volume = 	 4,
  author = 	 {Euripides},
  editor = 	 {Grene, David and Lattimore, Richmond},
  publisher = uchp,
  pages = 	 {185--288},
  location =  {Chicago}
}
\end{lstlisting}
\begin{lstlisting}[language=BibTeX,label=frede:inproc]
*\lnbackref{InProceedings}{1}*{frede:inproc,
  author = {Dorothea Frede}, 
  title = {\mkbibemph{Nicomachean Ethics} VII. 11--12},
  subtitle = {Pleasure},
  booktitle = {Aristotle},
  booksubtitle = {\mkbibemph{Nicomachean Ethics}, Book VII},
  series = {Symposium Aristotelicum},
  editor = {Carlo Natali},
  publisher = {Oxford University Press},
  address  = {Oxford},
  year = {2009},
  pages = {183-207}
}
\end{lstlisting}
\begin{lstlisting}[language=BibTeX,label=friends:leia]
*\lnbackref{Video}{2}*{friends:leia,
  title = 	 {The One with the Princess Leia Fantasy},
  date = 	 2003,
  booktitle = 	 {Friends},
  booktitleaddon = 	 {season~3, episode~1},
  author = 	 {Curtis, Michael and Malins, Gregory~S.},
  eventdate = 	 {1996-09-19},
  editor = 	 {Mancuso, Gail},
  editortype = 	 {director},
  publisher = {Warner Home Video},
  type = 	 {DVD},
  address = 	 {Burbank, CA}
}
\end{lstlisting}
\begin{lstlisting}[language=BibTeX,label=furet:passing:eng]
*\lnbackref{Book}{7}*{furet:passing:eng,
  title = 	 {The Passing of an Illusion},
  year = 	 1999,
  author = 	 {Furet, Fran�ois},
  userf = 	 {furet:passing:fr},
  translator = 	 {Furet, Deborah},
  publisher = uchp,
  location =  {Chicago}
}
\end{lstlisting}%\enlargethispage{-\baselineskip}
\begin{lstlisting}[language=BibTeX,label=furet:related]
*\lnbackref{Book}{7}*{furet:related,
  title = 	 {Le pass� d'une illusion},
  year = 	 1995,
  related = 	 {*\hyperlink{\getrefbykeydefault%
{furet:passing:eng}{anchor}{}}{\colorbox{Gainsboro}{furet:passing:eng}}*},
  relatedtype =  {bytranslator},
  author = 	 {Furet, Fran�ois},
  publisher = {�ditions Robert Laffont},
  location =  {Paris}
}
\end{lstlisting}
\begin{lstlisting}[language=BibTeX,label=garaud:gatine]
*\lnbackref{Article}{1}*{garaud:gatine,
  author =	 {Garaud, Marcel},
  title =	 {Recherches sur les d�frichements dans la G�tine poitevine aux XIe et XIIe si�cles},
  journaltitle = {Bulletin de la Soci�t� des antiquaires de l'Ouest},
  year =	 1967,
  volume =	 9,
  series =	 4,
  pages =	 {11--27},
  shorttitle =	 {Recherches sur les d�frichements}
}
\end{lstlisting}
\begin{lstlisting}[language=BibTeX,label=harley:ancient:cart]
*\lnbackref{Collection}{5}*{harley:ancient:cart,
  title = 	 {Cartography in Prehistoric, Ancient, and Medieval Europe and the Mediterranean},
  crossref = 	 *\hyperlink{\getrefbykeydefault{harley:hoc}%
{anchor}{}}{\{\colorbox{Gainsboro}{harley:hoc}\}}*,
  date = 	 {1987},
  volume = 	 1
}
\end{lstlisting}
\begin{lstlisting}[language=BibTeX,label=harley:cartography]
*\lnbackref{Collection}{5}*{harley:cartography,
  title =	 {Cartography in the Traditional East and Southeast Asian Societies},
  year =	 1994,
  crossref = 	 *\hyperlink{\getrefbykeydefault{harley:hoc}%
{anchor}{}}{\{\colorbox{Gainsboro}{harley:hoc}\}}*,
  volume =	 {2},
  part = 	 {2},
  shorttitle =	 {Cartography in East and Southeast Asia}
}
\end{lstlisting}
\begin{lstlisting}[language=BibTeX,label=harley:hoc]
*\lnbackref{MVCollection}{5}*{harley:hoc,
  title = 	 {The History of Cartography},
  date = 	 {1987/},
  editor = 	 {Harley, J.~B. and Woodward, David},
  volumes = 	 {3},
  publisher = uchp,
  location =  {Chicago}
}
\end{lstlisting}
\begin{lstlisting}[language=BibTeX,label=herwign:office]
*\lnbackref{TechReport}{1}*{herwign:office,
  options = 	 {useprefix=true},
  author = 	 {{van} Herwijnen, Eric},
  title = 	 {Future Office Systems Requirements},
  institution =  {CERN DD internal note},
  year = 	 1988,
  month = 	 11
}
\end{lstlisting}\enlargethispage{-\baselineskip}
\begin{lstlisting}[language=BibTeX,label=holiday:fool]
*\lnbackref{Music}{2}*{holiday:fool,
  title = 	 {I'm a Fool to Want You},
  eventdate = 	 {1958-02-20},
  date = 	 {1960},
  booktitle = 	 {Lady in Satin},
  author = 	 {Herron, Joel and Sinatra, Frank and Wolf, Jack},
  editor = 	 {Holiday, Billie},
  editortype = 	 {none},
  number = 	 {CL 1157},
  publisher = {Columbia},
  type = 	 {33\onethird~rpm},
  note = 	 {with Ray Ellis},
  options = 	 {useauthor=false}
}
\end{lstlisting}
\begin{lstlisting}[language=BibTeX,label=jackson:paulina:letter]
*\lnbackref{Letter}{2}*{jackson:paulina:letter,
  author =	 {Jackson, Paulina},
  title =        {Paulina Jackson to John Pepys Junior},
  booktitle =    {The Letters of Samuel Pepys and His Family Circle},
  origdate = 	 {1676-10-03},
  publisher =	 {Clarendon Press},
  year =	 1955,
  editor =	 {Heath, Helen Truesdell},
  shorttitle =   {to John Pepys Junior},
  pages =	 {\bibstring{number} 42},
  location =	 {Oxford}
}
\end{lstlisting}
\begin{lstlisting}[language=BibTeX,label=james:ambassadors]
*\lnbackref{Book}{6}*{james:ambassadors,
  title = 	 {The Ambassadors},
  year = 	 1996,
  origdate = 	 1909,
  author = 	 {James, Henry},
  publisher = {Project Gutenberg},
  url = 	 {ftp://ibiblio.org/pub/docs/ books/gutenberg/etext96/ ambas10.txt}
}
\end{lstlisting}
\begin{lstlisting}[language=BibTeX,label=keating:dearborn]
*\lnbackref{InCollection}{5}*{keating:dearborn,
  author = 	 {Keating, William~H.},
  title = 	 {Fort Dearborn and Chicago},
  crossref =  *\hyperlink{\getrefbykeydefault{prairie:state}%
{anchor}{}}{\{\colorbox{Gainsboro}{prairie:state}\}}*,
  pages = 	 {84--87}
}
\end{lstlisting}
\begin{lstlisting}[language=BibTeX,label=lakeforester:pushcarts]
*\lnbackref{Article}{4}*{lakeforester:pushcarts,
  journaltitle = {Lake Forester},
  date = 	 {2000-03-23},
  entrysubtype = {magazine},
  title = 	 {Pushcarts Evolve to Trendy Kiosks},
  location =  {Lake Forest, IL},
  shorttitle = {Pushcarts Evolve}
}
\end{lstlisting}
\begin{lstlisting}[language=BibTeX,label=leo:madonna]
*\lnbackref{Artwork}{2}*{leo:madonna,
  author = 	 {{Leonardo da Vinci}},
  shortauthor =  {Leonardo},
  title = 	 {Madonna of the Rocks},
  type = 	 {oil on canvas},
  institution =  {Louvre},
  year = 	 {1480s},
  note = 	 {78 x 48.5 in\adddot},
  location =  {Paris}
}
\end{lstlisting}
\begin{lstlisting}[language=BibTeX,label=loc:leaders]
*\hyperlink{Hendnote.623}{\color{DarkBlue}@Online}*{loc:leaders,
  author = 	 {Library of Congress},
  title = 	 {American Leaders Speak},
  subtitle = 	 {Recordings from World War I and the 1920 Election, 1918--1920},
  url = 	 {http://memory.loc.gov/ammem/ nfhtml/nforSpeakers01.html},
  note = 	 {RealAudio and WAV formats},
  options = 	 {skipbib}
}
\end{lstlisting}
\begin{lstlisting}[language=BibTeX,label=mchugh:wake]
*\lnbackref{Book}{1}*{mchugh:wake,
  title = 	 {Annotations to \mkbibquote{Finnegans Wake}},
  year = 	 1980,
  author = 	 {McHugh, Roland},
  publisher = {Johns Hopkins University Press},
  location =  {Baltimore}
}
\end{lstlisting}
\begin{lstlisting}[language=BibTeX,label=nass:address]
*\lnbackref{Unpublished}{1}*{nass:address,
  author = 	 {Nass, Clifford},
  title = 	 {Why Researchers Treat On-Line Journals Like Real People},
  note =  {keynote address, annual meeting of the Council of Science Editors},
  location = 	 {San Antonio, TX},
  date = 	 {2000-05-06/2000-05-09}
}
\end{lstlisting}
\begin{lstlisting}[language=BibTeX,label=plato:republic:gr]
*\lnbackref{BookInBook}{4}*{plato:republic:gr,
  title = 	 {Republic},
  entrysubtype = {classical},
  year = 	 1902,
  volume = 	 4,
  author = 	 {Plato},
  editor = 	 {Burnet, J.},
  booktitle = 	 {Clitopho, Res Publica, Timaeus, Critias},
  maintitle = 	 {Opera},
  publisher = {Clarendon Press},
  series = {Oxford Classical Texts},
  pages = 	 {327--621},
  location =  {Oxford}
}
\end{lstlisting}
\begin{lstlisting}[language=BibTeX,label=polakow:afterw]
*\lnbackref{SuppBook}{2}*{polakow:afterw,
  author =	 {Polakow, Valerie},
  title =	 {Lives on the Edge},
  subtitle = {Single Mothers and Their Children in the Other America},
  pages = 	 {175--184},
  afterword =	 {yes},
  year =	 1993,
  publisher =	 uchp,
  location =	 {Chicago}
}
\end{lstlisting}
\begin{lstlisting}[language=BibTeX,label=pollan:plant]
*\lnbackref{Online}{6}*{pollan:plant,
  author = 	 {Pollan, Michael},
  title = {Michael Pollan Gives a Plant's-Eye View},
  organization = {TED video, 17:31},
  url = 	 {http://www.ted.com/index.php/
        talks/michael_pollan_gives_a_
        plant_s_eye_view.html},
  urldate = 	 {2008-02},
  date = 	 {2007-03},
  userd = 	 {posted}
}
\end{lstlisting}
\begin{lstlisting}[language=BibTeX,label=powell:email]
*\lnbackref{Online}{6}*{powell:email,
  author = 	 {Powell, John},
  date = 	 {1998-04-23},
  titleaddon = {e-mail to Grapevine mailing list},
  url =        {http://www.electriceditors.net/ grapevine/issues/83.txt},
  shorttitle =  {\autocap{e}-mail to Grapevine mailing list}
}
\end{lstlisting}
\begin{lstlisting}[language=BibTeX,label=prairie:state]
*\lnbackref{Collection}{5}*{prairie:state,
  title = {Prairie State},
  subtitle = {Impressions of Illinois, 1673--1967, by Travelers and Other Observers},
  year = 1968,
  editor = {Angle, Paul~M.},
  publisher = uchp,
  location = {Chicago}
}
\end{lstlisting}
\begin{lstlisting}[language=BibTeX,label=ratliff:review]
*\lnbackref{Review}{2}*{ratliff:review,
  author =	 {Ratliff, Ben},
  title =	 {\bibstring{reviewof} \mkbibemph{The Mystery of
         Samba: Popular Music and
         National Identity in Brazil},
         \bibstring{by} Hermano Vianna,
         \parteditandtrans John Charles
         Chasteen},
  journaltitle = {Lingua Franca},
  date = 	 {1999-04},
  volume =	 9,
  pages =	 {B13--B14},
  shorttitle =	 {\bibstring{reviewof} \mkbibemph{The Mystery of Samba}}
}
\end{lstlisting}
\begin{lstlisting}[language=BibTeX,label=ross:thesis]
*\lnbackref{MastersThesis}{1}*{ross:thesis,
  author = 	 {Ross, Dorothy},
  title = 	 {The Irish-Catholic Immigrant, 1880--1900},
  subtitle = 	 {A Study in Social Mobility},
  school = 	 {Columbia University},
  year = 	 {\bibstring{nodate}}
}
\end{lstlisting}
\begin{lstlisting}[language=BibTeX,label=schubert:muellerin]
*\lnbackref{Audio}{2}*{schubert:muellerin,
  title = 	 {Das Wandern (Wandering)},
  date = 	 1895,
  shorttitle = 	 {Das Wandern},
  booktitle = 	 {Die sch\"one M\"ullerin (The Maid of the Mill)},
  maintitle = 	 {First Vocal Album \mkbibemph{(for high voice)}},
  author = 	 {Schubert, Franz},
  publisher = {G.~Schirmer},
  address = 	 {New York}
}
\end{lstlisting}
\begin{lstlisting}[language=BibTeX,label=stenger:privacy]
*\lnbackref{Article}{6}*{stenger:privacy,
  journaltitle = {CNN.com},
  entrysubtype = {magazine},
  date = 	 {1999-12-20},
  author = 	 {Stenger, Richard},
  title = 	 {Tiny Human-Borne Monitoring Device Sparks Privacy Fears},
  url = 	 {http://www.cnn.com/1999/TECH/ ptech/12/20/implant.device/},
  shorttitle = {Tiny Human-Borne Monitoring Device}
}
\end{lstlisting}
\begin{lstlisting}[language=BibTeX,label=wikiped:bibtex]
*\lnbackref{InReference}{2}*{wikiped:bibtex,
  title = 	 {Wikipedia},
  lista = {BibTeX},
  url = 	 {http://en.wikipedia.org/ wiki/BibTeX},
  urldate = 	 {2011-09-03},
  userd = 	 {last modified}
}
\end{lstlisting}

%% \printshorthands

\end{document}
%%% Local Variables: 
%%% mode: latex
%%% TeX-master: t
%%% TeX-trailer-start: "^[^%\n]*\\\\printshorthands"
%%% End: 
