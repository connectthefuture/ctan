%
% Copyright (C) 2007 by Gustavo MEZZETTI <gustavo.mezzetti@istruzione.it>.
%
% This file is part of a work named "bullcntr package".
%
% The bullcntr package may be distributed and/or modified under the
% conditions of the LaTeX Project Public License, either version 1.3 of
% this license or (at your option) any later version.
% The latest version of this license is in
%   http://www.latex-project.org/lppl.txt
% and version 1.3 or later is part of all distributions of LaTeX
% version 2005/12/01 or later.
%
% The bullcntr package has the LPPL maintenance status
%   "author-maintained".
%
% The file `manifest.txt' that comes along with this file specifies what
% the bullcntr package consists of; more precisely, it explains how the
% locutions "Work" and "Compiled Work", used in the LaTeX Project Public
% License, are to be interpreted in the case of this work.
%
% Note that the distribution of this file alone is "distribution of only
% part of the Work" in the sense of the LaTeX Project Public License,
% and should be avoided (see the LaTeX Project Public Licence for
% details).
%
\documentclass[a4paper]{article}
\usepackage{bullenum}
\usepackage[bookmarksnumbered]{hyperref}

\title{Overview\\of \tbull
	\thanks{This document describes version \bullvers\ (\bulldate).
		Copyright \copyright~2007 by G.~Mezzetti
		(see page~\pageref{S:Copyright} for details).}}
\author{G.~Mezzetti\\{\ttfamily\normalsize
	gustavo.mezzetti@istruzione.it}}
\date{October~10, 2008}

\newcommand*{\bullvers}{0.04}
\newcommand*{\bulldate}{2007/04/02}
\newcommand*{\filename}{bullcntr-man.tex}

\DeclareRobustCommand*{\command}[1]{%
	{\texorpdfstring{\normalfont\ttfamily \char\escapechar}{\pdfbslash}#1}%
}
\DeclareRobustCommand*{\packlass}[1]{%
	{\texorpdfstring{\normalfont \sffamily}{}#1}%
}
\DeclareRobustCommand*{\meta}[1]{\mbox{\(\langle\textit{#1}\rangle\)}}
\DeclareRobustCommand*{\filedir}[1]{{\normalfont\ttfamily #1}}
\DeclareRobustCommand*{\env}[1]{\texttt{#1}}
\DeclareRobustCommand*{\cnt}[1]{\texttt{#1}}
\DeclareRobustCommand*{\opt}[1]{{\normalfont\ttfamily #1}}
\newcommand*{\Bullcntr}{bullcntr}
\newcommand*{\bull}{\packlass{\Bullcntr}}
\newcommand*{\tbull}{the \bull\ package}
\newcommand*{\Bullenum}{bullenum}
\newcommand*{\benu}{\packlass{\Bullenum}}
\newcommand*{\tbenu}{the \benu\ package}
\newcommand*{\hreff}[1]{\href{#1}{\texttt{#1}}}
\newcommand*{\Enumerat}{enumerate}
\newcommand*{\beenv}{\env{\Bullenum}}
\newcommand*{\enenv}{\env{\Enumerat}}

\newcommand*{\pdfbslash}{}
{\catcode`\|=0 |catcode`|\=12 |gdef|pdfbslash{\\}}

\makeatletter
\newcommand*\hyref[1]{% actually two arguments, <label> and <text>
  \@ifundefined{r@#1}{% #1 is <label>
    \G@refundefinedtrue % \protect seems superfluous in our context
    \@latex@warning{Reference `#1' on page \thepage \space undefined}%
    \@firstofone % returns <text> (removing braces)
  }{%
    \expandafter\@hyref\expandafter{\csname r@#1\endcsname}%
  }%
}
\@ifdefinable\@hyref{\def\@hyref#1{%
  \expandafter\@@hyref#1%
}}
\@ifdefinable\@@hyref{\def\@@hyref#1#2#3#4#5{%
  \hyperlink{#4}% <text> follows and becomes the second argument
}}
\makeatother

\hfuzz = 0pt
\vfuzz = 0pt



\begin{document}

\maketitle

\begin{abstract}
	We describe a \LaTeX\ package which defines a new command, similar
	to \verb|\fnsymbol|, for displaying the value of a counter: this
	command typesets a number of bullets equal to the value of the
	counter, in a nice arrangement.  The value of the counter must lie
	between 1 and~9, inclusive.
	
	We also present an ancillary package which defines a new
	environment, similar to \enenv, for creating lists of items
	numbered in this way.
\end{abstract}



\tableofcontents



\begingroup

\setlength{\parskip}{\medskipamount}
\setlength{\parindent}{0pt}

\section*{Copyright notice}
\label{S:Copyright}
\vspace{-\parskip}

Copyright \copyright~2007 by Gustavo \textsc{Mezzetti}.  All rights
are reserved, except as noted below.

This document has been produced by feeding to a \LaTeX\ typesetting
engine the file \filedir{\filename}.  This file is part of a work
named ``\bull\ package'', and may be distributed and/or modified only
as a part thereof.  Subsection~\ref{SS:License} on
page~\pageref{SS:License} explains the conditions under which \tbull\
may be distributed and/or modified.

Permission is granted to make printed copies of this document, by any
means (\emph{e.g.}, photocopying, direct printing from the electronic
version, and so on).

Note, though, that distribution of this document alone is
``distribution of only part of the Work'' in the sense of the
\href{http://www.latex-project.org/lppl.txt}{\LaTeX\ Project Public
License}, and should be avoided: please distribute, in addition, the
source file \filedir{\filename} together with all other files listed
in \filedir{manifest.txt}, as explained as well in
Subsection~\ref{SS:License} and in the file \filedir{manifest.txt}
itself.

\endgroup



\clearpage
\setcounter{secnumdepth}{0}

\section{Introduction}
\label{S:Intro}

I~can't remember exactly where the idea of this package arose from,
but I~am pretty sure that the first time I~saw a list of items
numbered with an increasing number of bullets was in certain
handwritten notes about sheaves cohomology, may be a decade or so ago.
Since then, I~had repeatedly thought of writing a \LaTeX\ package to
implement this kind of numbering, but I~had never resolved to actually
do so, until very recently, when I~suddenly decided to sit down and
% ( paren match
write the code!~:-)

Regrettably, I~had to discover a major bug in the macros I~had dashed
off in such a haste, which required extensive (and more careful!)\
redesign of the code.  I~took this opportunity to thoroughly
reorganize the package, making it easier, for the user, to switch from
one style of bullets to another, and also adding an ancillary package
that simplifies the creation of lists numbered with bullets.

This document describes \tbull\ version \textbf{\bullvers}
(\textbf{\bulldate}).



\setcounter{secnumdepth}{3}

\section{Installing \tbull}

As usual for \LaTeX\ distributions, \tbull\ is distributed in the form
of a \filedir{.dtx} file, namely \filedir{\Bullcntr.dtx}, and an
accompanying \filedir{.ins} file, namely \filedir{\Bullcntr.ins}.



\subsection{Installation}

To install the package, run \LaTeX\ (or Plain \TeX) once on the file
\filedir{\Bullcntr.ins}.  This will generate the following \LaTeX\
input files:
%
\begin{quote}
	\ttfamily
	\Bullcntr.sty\\
	\Bullenum.sty
\end{quote}
%
To finish the installation, move them into a \LaTeX\ input directory.
The above listing of the files you need to move is also displayed on
the terminal at the end of the run of the file
\filedir{\Bullcntr.ins}.  The documentation of your \TeX\ installation
should tell you how to find the \LaTeX\ input directory/ies, and
probably also how to create new \LaTeX\ input directories reserved to
contain your private classes and packages.  If your \TeX\ installation
offers you the chance of defining your private \LaTeX\ input
directories, I~recommend you exploit this possibility and place the
generated files into such a directory.

Please note that before using \tbull\ you must read the license (see
Subsection~\ref{SS:License}) to see whether its terms are acceptable
for you, especially for what concerns the lack of any warranty; if
they are not, don't use it.



\subsection{Documentation}

To produce the documentation, run \LaTeX\ three times (for the table
of contents to be correct) on the file \filedir{\Bullcntr.dtx}.  This
won't produce the index and the change history, however: for those,
you have to run \emph{MakeIndex} on the files \filedir{\Bullcntr.idx}
and \filedir{\Bullcntr.glo}, produced during the last of the aforesaid
three \LaTeX\ runs, with the styles \filedir{gind.ist} and
\filedir{gglo.ist}, respectively (these styles are part of the
standard \LaTeX\ distribution); then pass \filedir{\Bullcntr.dtx}
through \LaTeX\ twice more.



\subsection{License}
\label{SS:License}

The \bull\ package is \emph{not} in the public domain: its author,
Gustavo \textsc{Mezzetti}, owns the copyright, and in general retains
all the rights therein; but as a special exception, the author grants
you the permissions indicated below.



\subsubsection{Distribution and/or modification}

The \bull\ package may be distributed and/or modified under the
conditions of the \href{http://www.latex-project.org/lppl.txt}{\LaTeX\
Project Public License}, either version~1.3 of this license or (at
your option) any later version.  The latest version of this license is
in
%
\begin{quote}
	\hreff{http://www.latex-project.org/lppl.txt}
\end{quote}
%
and version~1.3 or later is part of all distributions of \LaTeX\
version 2005/12/01 or later.

The \bull\ package has the LPPL maintenance status
``author-maintained''.
% Switching to the ``maintained'' status (which
% is the status recommended by the \TeX\ community) is currently being
% considered, but for now I~still prefer to keep my tiny unfledged
% % ( paren match
% creation under my protecting wing\ldots~:-)\spacefactor=\sfcode`.
% \space Leaving jokes apart, I'm waiting to see if other ``disastrous''
% mistakes are still lurking in the code, and whether there is an
% interest of the \LaTeX\ community into this silly package that
% justifies maintaining it in a more efficient way.

The file \filedir{manifest.txt} included in \tbull\ specifies what
this package consists of; more precisely, it explains how the
locutions ``Work'' and ``Compiled Work'', used in the \LaTeX\ Project
Public License, are to be interpreted in the case of this work.



\subsubsection{Use}

The use of \tbull\ is unrestricted, provided that you accept the terms
and conditions of the
\href{http://www.latex-project.org/lppl.txt}{\LaTeX\ Project Public
License} and of the \hyref{SSS:NoWarranty}{following subsection} for
what concerns the absence of any warranty.



\subsubsection{No warranty}
\label{SSS:NoWarranty}

There is absolutely no warranty for \tbull.  The Copyright Holder
provides \tbull\ ``as is'', without warranty of any kind, either
expressed or implied, including, but not limited to, the implied
warranties of merchantability and fitness for a particular purpose.
The entire risk as to the quality and performance of \tbull\ is with
you.  Should \tbull\ prove defective, you assume the cost of all
necessary servicing, repair, or correction.

In no event will The Copyright Holder, or any other party who may
distribute and/or modify \tbull\ as permitted by the \LaTeX\ Project
Public License, be liable to you for damages, including any general,
special, incidental or consequential damages arising out of any use of
\tbull\ or out of inability to use it (including, but not limited to,
loss of data, data being rendered inaccurate, or losses sustained by
anyone as a result of any failure of \tbull\ to operate with any other
programs), even if The Copyright Holder or said other party has been
advised of the possibility of such damages.



\subsection{Your comments}

If you have comments, suggestions, etc.\ about \tbull, please let me
know them: I'll be happy to read what you think.  My e-mail address is
indicated under the title of this document.  Please include the
\emph{exact} phrase \verb*|bullcntr package| in the \emph{subject} of
your message; otherwise, it could be thrown away by a mail filter.



\section{Using \tbull}

Version \bullvers\ of \tbull\ is a bit more elaborate than its
predecessor: in addition to the
\hyref{SS:bullcntr}{\command{\Bullcntr} main command} and to the
\hyref{SS:Hooks}{hooks for the bullets}, it now features
\hyref{SS:CntrInvo}{options} and \hyref{SS:BullStyl}{predefined bullet
styles}.



\subsection{Package invocation and options}
\label{SS:CntrInvo}

You load \tbull\ by means of the usual \verb|\usepackage| declaration:
%
\begin{flushleft}
	\verb|\usepackage|\meta{options}\verb|{bullcntr}|
\end{flushleft}
%
You may specify one of the following, mutually exclusive options,
which affect the size (or the type) of the bullets used by
\verb|\bullcntr| command:
%
\begin{description}
%
\item[\opt{largectrbull}]
(Relatively) ``large'' bullets are used for all values from 1 to~9.
This has exactly the same effect as a \verb|\largectrbull| global
declaration (see Subsection~\ref{SS:BullStyl}).
%
\item[\opt{smallctrbull}]
The opposite: ``small'' bullets are used for all values from 1 to~9.
This has exactly the same effect as a \verb|\smallctrbull| global
declaration (see Subsection~\ref{SS:BullStyl}).
%
\item[\opt{smartctrbull}]
(Relatively) ``large'' bullets are used for values from 1 through~5
(inclusive), while ``small'' ones are used for those greater than or
equal to~6.  This has exactly the same effect as a
\verb|\smartctrbull| global declaration (see
Subsection~\ref{SS:BullStyl}).
%
\item[\opt{heartctrbull}]
Hearts are used (instead of bullets) for all values from 1 to~9.  This
has exactly the same effect as a \verb|\heartctrbull| global
declaration (see Subsection~\ref{SS:BullStyl}).
%
\end{description}
%
If you don't specify any, the default option is \opt{smartctrbull}.




\subsection{The \command{\Bullcntr} command}
\label{SS:bullcntr}

This is a command analogous to the \verb|\fnsymbol| command (see the
\textsl{\LaTeX book} \cite[Subsection~C.8.4]{LaTeXbook}): you may use
it directly in the text, as in
%
\begin{flushleft}
	\ldots\texttt{the counter value displays as}\verb|~\bullcntr{|\meta{ctr}\verb|}|\ldots
\end{flushleft}
%
(where \meta{ctr} is a counter name, defined with \verb|\newcounter|),
or you may redefine the \verb|\the|\meta{ctr} command, as in
%
\begin{flushleft}
	\verb|\renewcommand*{\the|\meta{ctr}\verb|}{\bullcntr{|\meta{ctr}\verb|}}|
\end{flushleft}
%
By way of example, in the case of a counter named \cnt{myctr}, you
would write
%
\begin{verbatim}
The counter value displays as~\bullcntr{myctr}.
\end{verbatim}
%
directly in the text, or the following to redefine the
\verb|\themyctr| command:
%
\begin{verbatim}
\renewcommand*{\themyctr}{\bullcntr{myctr}}
\end{verbatim}
%

When \( \verb|\value{|\meta{ctr}\verb|}|=0 \), the command
\verb|\bullcntr{|\meta{ctr}\verb|}| silently typesets nothing (not
even a space); but when \( \verb|\value{|\meta{ctr}\verb|}|<0 \) or \(
\verb|\value{|\meta{ctr}\verb|}|>9 \), an error is generated.



\subsection{Hooks for the bullets}
\label{SS:Hooks}

The main task of \tbull\ is to calculate the position of the
``bullets'' used to display the value of the given counter; but you
are free to decide what characters are actually drawn in these
positions: they need not necessarily be ``bullets''.  Indeed, two
``hooks'' are provided in the package to define the actual characters
employed, and the package simply invokes these hooks.

If the value of the counter being displayed is less than~6 (less
than~6 bullets need to be drawn), the \verb|\bullcntr| command invokes
the \verb|\counterlargebullet| macro to draw each of them.  If you
want, you should redefine this macro (using \verb|\renewcommand|) so
that it generates a symbol suitable for arrangements in which the
``bullets'' are relatively spaced out.  For example,
%
\begin{verbatim}
\renewcommand*{\counterlargebullet}{\textbullet}
\end{verbatim}
%

You don't \emph{need} to redefine this command: a convenient default
definition is already set by \tbull\ itself.

When, on the other hand, the value of the counter being displayed is
greater than or equal to~6 (6~or more bullets need to be drawn), the
\verb|\bullcntr| command calls the \verb|\countersmallbullet| macro.
It is intended that this macro typeset a character suitable for
situations in which the ``bullets'' have to be grouped in a compact
cluster; typically, this means that a smaller ``bullet'' is used than
the one typeset by \verb|\counterlargebullet|, although it is
perfectly in order to use the same character in both situations.  If
you want (but you don't \emph{need} to do this, since \tbull\ already
defines this macro) you may change the definition with
\verb|\renewcommand|, \emph{e.g.}:
%
\begin{verbatim}
\renewcommand*{\countersmallbullet}{\textperiodcentered}
\end{verbatim}
%

For both hooks, it is best to \verb|\protect| fragile commands that
appear in the code used in their (re)definition.

``Macho \TeX\ programmers'' may also use \verb|\def| instead of
\verb|\renewcommand|; for example:
%
\begin{verbatim}
\def\counterlargebullet{\textbullet}
\def\countersmallbullet{\textperiodcentered}
\end{verbatim}
%
(Actually, in a case like this, even \verb|\let| could be used.)



\subsection{Predefined bullet styles}
\label{SS:BullStyl}

To spare you the nuisance of redefining the hooks mentioned in
Subsection~\ref{SS:Hooks}, at least in normal situations, four
``bullet styles'' are already predefined by \tbull, covering what the
author believes are the most common cases.  You can switch to any of
these predefined styles by means of the following four declarations.

\begin{description}

\item[\command{largectrbull}]
This declaration causes (relatively) ``large'' bullets to be used for
all values from 1 to~9.

\item[\command{smallctrbull}]
This declaration causes ``small'' bullets to be used for all values
from 1 to~9.

\item[\command{smartctrbull}]
This declaration causes (relatively) ``large'' bullets to be used for
values from 1 to~5 (inclusive), and ``small'' bullets for values equal
to or above~6.

\item[\command{heartctrbull}]
This declaration causes little hearts (instead of bullets) to be used
for all values from 1 to~9.

\end{description}

These are in all respects normal declarations, obeying to ordinary
scoping rules.  The corresponding options described in
Subsection~\ref{SS:CntrInvo} simply set the style at the outer
(global, or document-wide) level.



\subsection{Cross-references}
\label{SS:CrossRef}

Cross-references now work fine with the \verb|\bullcntr| command.  You
may use \verb|\label| (preceded by \verb|\refstepcounter| if
necessary) in the usual way to label, for example, an item in a list
numbered via \verb|\bullcntr| (recall that \verb|\item| calls
\verb|\refstepcounter| by itself), and elsewhere cross-reference that
item with \verb|\ref|: the reference will be typeset correctly, using
the same bullet style that was in effect at the spot where
\verb|\label| (and not \verb|\ref|) was used.

Examples of such cross-references are shown later, in
Subsection~\ref{SS:CrossXmp}.



\section{Using \tbenu}

The \bull\ package was conceived for the purpose of imitating in print
an unusual way of numbering items in a list, that the author saw in
certain handwritten notes (see the \hyref{S:Intro}{Introduction}).
With the \verb|\bullcntr| command at hand, this could be achieved
simply by saying something like
%
\begin{verbatim}
\begin{enumerate}
    \renewcommand*{\theenumi}{\bullcntr{enumi}}
    \renewcommand*{\labelenumi}{\theenumi}
\end{verbatim}
\begin{flushleft}
	\texttt{\ \ \ \ }\meta{items}
\end{flushleft}
\begin{verbatim}
\end{enumerate}
\end{verbatim}
%
This, nonetheless, is not perfect, since it breaks if the \enenv\
environment is moved inside another \enenv, because of the explicit
mention of \verb|\theenumi| and of \verb|\labelenumi|.

For this reason, and also to facilitate the use of the
\verb|\bullcntr| command for numbering list items, a new package has
been added to \tbull; this helper package, called \benu, simply
defines the \beenv\ environment, a variation of \enenv\ that numbers
its items using the \verb|\bullcntr| command.  Indeed,
%
\begin{verbatim}
\begin{bullenum}
\end{verbatim}
\begin{flushleft}
	\texttt{\ \ \ \ }\meta{items}
\end{flushleft}
\begin{verbatim}
\end{bullenum}
\end{verbatim}
%
is equivalent to
\begin{flushleft}
	\verb|\begin{enumerate}|\\
	\verb|    \renewcommand*{\theenum|\meta{lvl}\verb|}{\bullcntr{enum|\meta{lvl}\verb|}}|\\
	\verb|    \renewcommand*{\labelenum|\meta{lvl}\verb|}{\theenum|\meta{lvl}\verb|}|
\end{flushleft}
\begin{flushleft}
	\texttt{\ \ \ \ }\meta{items}\\[\topsep]
\end{flushleft}
\begin{verbatim}
\end{enumerate}
\end{verbatim}
%
where \meta{lvl} is \texttt{i}, \texttt{ii}, \texttt{iii},
or~\texttt{iv}, depending on the current level of the \enenv\
environment.

In other words, \beenv\ can be used in every place where \enenv\ can,
and numbers its items with an increasing number of bullets, regardless
of the definition of \verb|\theenum|\meta{lvl} and
\verb|\labelenum|\meta{lvl}.  Of course, the environment must contain
no more than 9~items.  Moreover, if you use \verb|\label| to set up a
reference, subsequent \verb|\ref|'s will typeset the reference as a
``bulletted'' number, using, for the ``bullets'', the character that
was current at the time the \verb|\label| was encountered (in brief,
they will do the right thing).

Notice that you should \emph{not} load \tbull\ if you invoke \benu,
because the latter calls the former.  As explained below, you may pass
to \tbenu\ the options that you would require from \bull: the outer
package will forward them to the inner one.  Actually, most users will
just use \tbenu, and never call \bull\ directly.



\subsection{Package invocation and options}
\label{SS:EnumInvo}

You invoke \tbenu\ with the usual \verb|\usepackage| declaration put
in the preamble of your document.  In this declaration you may specify
exactly the same options that you would pass to \tbull, with the same
meaning (see Subsection~\ref{SS:CntrInvo}).  The \benu\ package then
calls \bull\ for you, just passing on to it the options, if any, you
asked for.

% For example, if, when preparing a valentine, you want to use the
% \beenv\ environment to create list numbered with hearts, you might
% say
% %
% \begin{verbatim}
% \usepackage[heartctrbull]{bullenum}
% \end{verbatim}
% %
% at the beginning of your preamble.



\subsection{The \beenv\ environment}
\label{SS:EnumEnvi}

This environment is syntactically analogous to \enenv, but numbers its
items with an increasing number of bullets (using the \verb|\bullcntr|
command).  You must include no more than 9~items in this environment.
The style of the bullets is governed by the declarations described in
Subsection~\ref{SS:BullStyl}, according to ordinary scoping rules.  A
change local to the environment being employed can be obtained by
enclosing the relevant declaration inside the environment itself; for
example,
%
\begin{verbatim}
\begin{bullenum}
    \smallctrbull
\end{verbatim}
\begin{flushleft}
	\texttt{\ \ \ \ }\meta{items}
\end{flushleft}
\begin{verbatim}
\end{bullenum}
\end{verbatim}
%
will use ``small'' bullets regardless of whatever style might have
been in force outside the \beenv\ environment.

\pagebreak[2]

If you want finer control over the appearance of the ``bullets'', just
redefine the hooks of Subsection~\ref{SS:Hooks}; doing this inside the
environment itself will keep changes to the hooks local.  For example,
you might say:
%
\begin{verbatim}
\begin{bullenum}
    \renewcommand*{\counterlargebullet}
        {$\scriptscriptstyle \diamondsuit$}
    \renewcommand*{\countersmallbullet}{\counterlargebullet}
\end{verbatim}
\begin{flushleft}
	\texttt{\ \ \ \ }\meta{items}
\end{flushleft}
\begin{verbatim}
\end{bullenum}
\end{verbatim}
%
(yes, ``macho \TeX\ programmers'' may also use \verb|\def|, or
\verb|\let| where appropriate).

Of course, this environment respects the definitions of
\verb|\theenum|\meta{lvl} and of \verb|\labelenum|\meta{lvl} in force
outside the environment itself, for \( \meta{lvl} =
\texttt{i},\ldots,\texttt{iv} \).



\subsection{Nesting \beenv\ environments}
\label{SS:Nesting}

Because a \beenv\ environment is nothing more than a modified \enenv,
it can be nested in the same way, up to four levels deep.

For what concerns the syntax, it is perfectly in order to nest one
\beenv\ environment inside another; it is from the viewpoint of the
resulting visual appearance that this is not recommendable, unless
quite different bullet styles are used for the two enumerations, so as
to sharply mark the visual distinction between their respective
labels.  Of course, this problem does not arise if you nest a \beenv\
environment inside an \enenv, or viceversa.

Examples of such nested environments will be presented in
Subsection~\ref{SS:NestXmp}.



\section{Examples}

A few examples follow, illustrating the usage of the \bull\slash\benu\
packages.



\subsection{A sample list}

Here we illustrate the use of the \beenv\ environment to produce lists
whose items are numbered with the \verb|\bullcntr| command.

The lists in the various examples are produced by the following source
code:
%
\begin{verbatim}
\begin{bullenum}
\end{verbatim}
\begin{flushleft}
\texttt{\ \ \ \ }\meta{items}
\end{flushleft}
\begin{verbatim}
\end{bullenum}
\end{verbatim}
%

We shall now show a few variations of a list of the above kind,
changing the definition of the commands \verb|\counterlargebullet| and
\verb|\countersmallbullet| among the various examples, so as to
illustrate the use of several bullet sizes.  In three examples out of
five, however, this will be actually done by means of the declarations
described in Subsection~\ref{SS:BullStyl}, thus falling back on one of
the predefined bullet styles.



\subsubsection{The standard sizes (\command{smartctrbull})}
\label{SSS:Xmp-Standard}

In our first example, actually, we do not alter the default
definitions of the two hooks that typeset the bullets (see
Subsection~\ref{SS:Hooks}).  The following is the output the
\bull\slash\benu\ packages will produce by themselves, if you invoke
either of them without options and do not alter the default bullet
style; in any case, it corresponds to the style selected by
\verb|\smartctrbull|.

\begin{bullenum}
	\item
		This is the first item of the list, that is, item~\#1.  It is
		therefore numbered with a single bullet.  Poor bullet!  How
		lonely it must feel!

	\item
		This is the second item of the list, that is, item~\#2.  It is
		therefore numbered with two bullets, one on the left and one
		on the right.  If you are a romantic nature, you might even
		think of them as Romeo and Juliet\ldots

	\item\label{Standard-3}
		This is the third item of the list, that is---guess
		what?---item~\#3.  Not surprisingly, it is numbered with three
		bullets neatly arranged in the shape of an equilateral
		triangle.

	\item
		This is the fourth item, item~\#4.  A square has exactly four
		vertices, so it seems quite appropriate to lay out the four
		bullets that must be used to number this item exactly in this
		form, as you can see in the label.

	\item
		If you have not lost the track, you will agree that this is
		the fifth item of the list.  A regular pentagon would probably
		have been the optimal choice for displaying this value of the
		counter, but for some reason the arrangement shown in the
		label of this item has been preferred by the author of \tbull.

	\item\label{Standard-6}
		Sixth item!  Item~\#6!  A regular hexagon is used here!  Note,
		however, that the bullets have shrunk to a smaller size,
		because the size used so far would have made the bullet
		cluster look a bit too crowded.

	\item
		In the opinion of the author, the shape used to display the
		value~7 is even more pleasant than the regular hexagon used
		for the value~6; indeed, this is a regular hexagon too, but
		with the center marked by an additional bullet, which makes
		the figure look more symmetric.

	\item
		If 7~was the perfect number of bullets, here we face the
		disruption of this perfection.  The solution adopted for
		arranging 8~bullets, which you can see here, is, in the
		author's opinion, the less inelegant among all possibilities.

	\item
		Item~\#9, the last value that the \verb|\bullcntr| command can
		display.  The bullets are just stacked in a $3\times3$ square
		pattern---the obvious choice!
\end{bullenum}



\subsubsection{Big bullets}
\label{SSS:Xmp-Big}

You might think that the \verb|\textbullet| character is fitter for
labelling list items; this is probably true for item numbers that do
not exceede three, but begins to become questionable for higher
values.  In any case, here is an extreme example: \verb|\textbullets|
is used for all values from 1 to~9.  To achieve this, the following
source code should be used:
%
\begin{verbatim}
\begin{bullenum}
    \renewcommand*{\counterlargebullet}{\textbullet}
    \renewcommand*{\countersmallbullet}{\textbullet}
\end{verbatim}
\begin{flushleft}
	\texttt{\ \ \ \ }\meta{items}
\end{flushleft}
\begin{verbatim}
\end{bullenum}
\end{verbatim}
%
This is what we write below.

\begin{bullenum}
	\renewcommand*{\counterlargebullet}{\textbullet}
	\renewcommand*{\countersmallbullet}{\textbullet}

	\item
		This is the first item of the list, that is, item~\#1.  It is
		therefore numbered with a single bullet.  Poor bullet!  How
		lonely it must feel!

	\item\label{Big-2}
		This is the second item of the list, that is, item~\#2.  It is
		therefore numbered with two bullets, one on the left and one
		on the right.  If you are a romantic nature, you might even
		think of them as Romeo and Juliet\ldots

	\item\label{Big-3}
		This is the third item of the list, that is---guess
		what?---item~\#3.  Not surprisingly, it is numbered with three
		bullets neatly arranged in the shape of an equilateral
		triangle.

	\item
		This is the fourth item, item~\#4.  A square has exactly four
		vertices, so it seems quite appropriate to lay out the four
		bullets that must be used to number this item exactly in this
		form, as you can see in the label.

	\item
		If you have not lost the track, you will agree that this is
		the fifth item of the list.  A regular pentagon would probably
		have been the optimal choice for displaying this value of the
		counter, but for some reason the arrangement shown in the
		label of this item has been preferred by the author of \tbull.

	\item\label{Big-6}
		Sixth item!  Item~\#6!  A regular hexagon is used here!  Note,
		this time, that the bullets have \emph{not} shrunk to a
		smaller size, and that the resulting picture looks very
		crowded---pretty awful, I~would say!

	\item
		In the opinion of the author, the shape used to display the
		value~7 is even more pleasant than the regular hexagon used
		for the value~6; indeed, this is a regular hexagon too, but
		with the center marked by an additional bullet, which makes
		the figure look more symmetric.

	\item
		If 7~was the perfect number of bullets, here we face the
		disruption of this perfection.  The solution adopted for
		arranging 8~bullets, which you can see here, is, in the
		author's opinion, the less inelegant among all possibilities.

	\item\label{Big-9}
		Item~\#9, the last value that the \verb|\bullcntr| command can
		display.  The bullets are just stacked in a $3\times3$ square
		pattern---the obvious choice!
\end{bullenum}

A cross-reference to an item run into the main text may look really
bad with bullets thus large: for instance, referencing the second item
produces~\ref{Big-2} (quite acceptable), referencing the third one
gives~\ref{Big-3} (just about acceptable), but a reference to the
sixth item is typeset as~\ref{Big-6}, and the ninth one even
yields~\ref{Big-9}!  Note, however, that lines are not spaced out any
further than usual.



\subsubsection{Moderately large bullets (\command{largectrbull})}
\label{SSS:Xmp-Large}

Our next example illustrates the output produced by the style selected
with a \verb|\largectrbull| declaration; this style continues to use,
for the values over~5, the same bullet size adopted for the first five
values.  So, we say
%
\begin{verbatim}
\begin{bullenum}
    \largectrbull
\end{verbatim}
\begin{flushleft}
	\texttt{\ \ \ \ }\meta{items}
\end{flushleft}
\begin{verbatim}
\end{bullenum}
\end{verbatim}
%
to produce the desired list.

\begin{bullenum}
	\largectrbull

	\item
		This is the first item of the list, that is, item~\#1.  It is
		therefore numbered with a single bullet.  Poor bullet!  How
		lonely it must feel!

	\item
		This is the second item of the list, that is, item~\#2.  It is
		therefore numbered with two bullets, one on the left and one
		on the right.  If you are a romantic nature, you might even
		think of them as Romeo and Juliet\ldots

	\item\label{Large-3}
		This is the third item of the list, that is---guess
		what?---item~\#3.  Not surprisingly, it is numbered with three
		bullets neatly arranged in the shape of an equilateral
		triangle.

	\item
		This is the fourth item, item~\#4.  A square has exactly four
		vertices, so it seems quite appropriate to lay out the four
		bullets that must be used to number this item exactly in this
		form, as you can see in the label.

	\item
		If you have not lost the track, you will agree that this is
		the fifth item of the list.  A regular pentagon would probably
		have been the optimal choice for displaying this value of the
		counter, but for some reason the arrangement shown in the
		label of this item has been preferred by the author of \tbull.

	\item\label{Large-6}
		Sixth item!  Item~\#6!  A regular hexagon is used here!  Note,
		once again, that the bullets have \emph{not} shrunk to a
		smaller size, so that the resulting picture looks pretty
		crowded, although not intolerably bad-looking.  Compare the
		label of this item with a reference to the corresponding
		item~\ref{Standard-6} under~\ref{SSS:Xmp-Standard}, and with a
		reference to the corresponding item~\ref{Big-6}
		under~\ref{SSS:Xmp-Big}.

	\item
		In the opinion of the author, the shape used to display the
		value~7 is even more pleasant than the regular hexagon used
		for the value~6; indeed, this is a regular hexagon too, but
		with the center marked by an additional bullet, which makes
		the figure look more symmetric.

	\item
		If 7~was the perfect number of bullets, here we face the
		disruption of this perfection.  The solution adopted for
		arranging 8~bullets, which you can see here, is, in the
		author's opinion, the less inelegant among all possibilities.

	\item
		Item~\#9, the last value that the \verb|\bullcntr| command can
		display.  The bullets are just stacked in a $3\times3$ square
		pattern---the obvious choice!
\end{bullenum}



\subsubsection{Small bullets (\command{smallctrbull})}
\label{SSS:Xmp-Small}

The reciprocal example is to use small bullets throughout, from 1 to~9.
Thus, we use the following source code:
%
\begin{verbatim}
\begin{bullenum}
    \smallctrbull
\end{verbatim}
\begin{flushleft}
	\texttt{\ \ \ \ }\meta{items}
\end{flushleft}
\begin{verbatim}
\end{bullenum}
\end{verbatim}
%
And here is the output it yields.

\begin{bullenum}
	\smallctrbull

	\item
		This is the first item of the list, that is, item~\#1.  It is
		therefore numbered with a single bullet.  Poor bullet!  How
		lonely it must feel!

	\item
		This is the second item of the list, that is, item~\#2.  It is
		therefore numbered with two bullets, one on the left and one
		on the right.  If you are a romantic nature, you might even
		think of them as Romeo and Juliet\ldots

	\item\label{Small-3}
		This is the third item of the list, that is---guess
		what?---item~\#3.  Not surprisingly, it is numbered with three
		bullets neatly arranged in the shape of an equilateral
		triangle.

	\item
		This is the fourth item, item~\#4.  A square has exactly four
		vertices, so it seems quite appropriate to lay out the four
		bullets that must be used to number this item exactly in this
		form, as you can see in the label.

	\item
		If you have not lost the track, you will agree that this is
		the fifth item of the list.  A regular pentagon would probably
		have been the optimal choice for displaying this value of the
		counter, but for some reason the arrangement shown in the
		label of this item has been preferred by the author of \tbull.

	\item\label{Small-6}
		Sixth item!  Item~\#6!  A regular hexagon is used here!  Note,
		once more, that the bullets have \emph{not} shrunk to a
		smaller size, because they were \emph{already} small in the
		items shown thus far.

	\item
		In the opinion of the author, the shape used to display the
		value~7 is even more pleasant than the regular hexagon used
		for the value~6; indeed, this is a regular hexagon too, but
		with the center marked by an additional bullet, which makes
		the figure look more symmetric.

	\item
		If 7~was the perfect number of bullets, here we face the
		disruption of this perfection.  The solution adopted for
		arranging 8~bullets, which you can see here, is, in the
		author's opinion, the less inelegant among all possibilities.

	\item
		Item~\#9, the last value that the \verb|\bullcntr| command can
		display.  The bullets are just stacked in a $3\times3$ square
		pattern---the obvious choice!
\end{bullenum}

Again, compare the references to items \#3 and~\#6 from different
examples: standard bullets (\ref{Standard-3} and~\ref{Standard-6}),
big bullets (\ref{Big-3} and~\ref{Big-6}), moderately large bullets
(\ref{Large-3} and~\ref{Large-6}), small bullets (\ref{Small-3}
and~\ref{Small-6}).



\subsubsection{Tiny bullets}
\label{SSS:Xmp-Tiny}

We come now to another extreme example: the use of
\verb|\textperiodcentered|.  In other words,
%
\begin{verbatim}
\begin{bullenum}
    \renewcommand*{\counterlargebullet}{\textperiodcentered}
    \renewcommand*{\countersmallbullet}{\textperiodcentered}
\end{verbatim}
\begin{flushleft}
	\texttt{\ \ \ \ }\meta{items}
\end{flushleft}
\begin{verbatim}
\end{bullenum}
\end{verbatim}
%
is asserted in order to produce the list.

\begin{bullenum}
	\renewcommand*{\counterlargebullet}{\textperiodcentered}
	\renewcommand*{\countersmallbullet}{\textperiodcentered}

	\item
		This is the first item of the list, that is, item~\#1.  It is
		therefore numbered with a single bullet.  Poor bullet!  How
		lonely it must feel!

	\item
		This is the second item of the list, that is, item~\#2.  It is
		therefore numbered with two bullets, one on the left and one
		on the right.  If you are a romantic nature, you might even
		think of them as Romeo and Juliet\ldots

	\item
		This is the third item of the list, that is---guess
		what?---item~\#3.  Not surprisingly, it is numbered with three
		bullets neatly arranged in the shape of an equilateral
		triangle.

	\item
		This is the fourth item, item~\#4.  A square has exactly four
		vertices, so it seems quite appropriate to lay out the four
		bullets that must be used to number this item exactly in this
		form, as you can see in the label.

	\item
		If you have not lost the track, you will agree that this is
		the fifth item of the list.  A regular pentagon would probably
		have been the optimal choice for displaying this value of the
		counter, but for some reason the arrangement shown in the
		label of this item has been preferred by the author of \tbull.

	\item
		Sixth item!  Item~\#6!  A regular hexagon is used here!  Note
		that it was really impossible, this time, for the bullets to
		shrunk to a still smaller size than the one that has been
		adopted in the above items!

	\item
		In the opinion of the author, the shape used to display the
		value~7 is even more pleasant than the regular hexagon used
		for the value~6; indeed, this is a regular hexagon too, but
		with the center marked by an additional bullet, which makes
		the figure look more symmetric.

	\item
		If 7~was the perfect number of bullets, here we face the
		disruption of this perfection.  The solution adopted for
		arranging 8~bullets, which you can see here, is, in the
		author's opinion, the less inelegant among all possibilities.

	\item
		Item~\#9, the last value that the \verb|\bullcntr| command can
		display.  The bullets are just stacked in a $3\times3$ square
		pattern---the obvious choice!
\end{bullenum}

This bullet style might be employed in a \beenv\ environment nested
inside another one.



\subsubsection{St.\texorpdfstring{\,}{ }Valentine (\command{heartctrbull})}
\label{SSS:Xmp-Hearts}

To end this inventory of variations, let us illustrate the use of a
fancier ``bullet'': a little heart.  We obtain this by saying
%
\begin{verbatim}
\begin{bullenum}
    \heartctrbull
\end{verbatim}
\begin{flushleft}
	\texttt{\ \ \ \ }\meta{items}
\end{flushleft}
\begin{verbatim}
\end{bullenum}
\end{verbatim}
%
% ( paren match
This could be used in valentines to enumerate to your beloved the
terms of your passion\ldots~:-)

\begin{bullenum}
	\heartctrbull

	\item\label{Hearts-1}
		This is the first item of the list, that is, item~\#1.  It is
		therefore numbered with a single heart.  Poor heart!  It is a
		true lonely heart!

	\item
		This is the second item of the list, that is, item~\#2.  It is
		therefore numbered with two hearts, one on the left and one on
		the right.  Thinking of Romeo and Juliet is absolutely natural
		in this context\ldots

	\item\label{Hearts-3}
		This is the third item of the list, that is---guess
		what?---item~\#3.  Not surprisingly, it is numbered with three
		hearts neatly arranged in the shape of an equilateral
		triangle.

	\item\label{Hearts-4}
		This is the fourth item, item~\#4.  A square has exactly four
		vertices, so it seems quite appropriate to lay out the four
		hearts that must be used to number this item exactly in this
		form, as you can see in the label.

	\item\label{Hearts-5}
		If you have not lost the track, you will agree that this is
		the fifth item of the list.  A regular pentagon would probably
		have been the optimal choice for displaying this value of the
		counter, but for some reason the arrangement shown in the
		label of this item has been preferred by the author of \tbull.

	\item
		Sixth item!  Item~\#6!  A regular hexagon is used here,
		adorned with hearts!  Of course, the hearts do not shrink, as
		doesn't yours, and as doesn't your ardent passion!

	\item\label{Hearts-7}
		In the opinion of the author, the shape used to display the
		value~7 is even more pleasant than the regular hexagon used
		for the value~6; indeed, this is a regular hexagon too, but
		with the center marked by an additional heart, which makes the
		figure look more symmetric.

	\item\label{Hearts-8}
		If 7~was the perfect number of hearts, here we face the
		disruption of this perfection.  The solution adopted for
		arranging 8~hearts, which you can see here, is, in the
		author's opinion, the less inelegant among all possibilities.

	\item
		Item~\#9, the last value that the \verb|\bullcntr| command can
		display.  The hearts are just stacked in a $3\times3$ square
		pattern---the obvious choice!
\end{bullenum}

Cross-references are a bit funny with this style; they look like this:
\ref{Hearts-1}, \ref{Hearts-3}, \ref{Hearts-4}, \ref{Hearts-5},
\ref{Hearts-7}, and~\ref{Hearts-8}.



\subsection{Nested enumerations}
\label{SS:NestXmp}

Try not to overdo this: carrying it to excess could lead to confused
output.



\subsubsection{Nesting one \beenv\ inside another}
\label{SSS:NestBullBull}

We use ``large'' bullets for the outer environment, and ``tiny''
bullets for the inner one.

\begin{bullenum}
	\largectrbull

	\item
		First item: let this one alone, just to have a term of
		comparison.

	\item\label{NestBull-2}
		Second item.  Here we nest another \beenv\ environment:
		\begin{bullenum}
			\renewcommand*{\counterlargebullet}{\textperiodcentered}
			\renewcommand*{\countersmallbullet}{\textperiodcentered}
			\item\label{NestBull-2.1}
				First sub-item.
			\item\label{NestBull-2.2}
				Second sub-item.
			\item\label{NestBull-2.3}
				Third sub-item.
			\item\label{NestBull-2.4}
				Fourth sub-item.  OK, this should suffice.
		\end{bullenum}
		Don't forget, we are still inside item~\ref{NestBull-2} of
		the outer enumeration.

	\item\label{NestBull-3}
		Third item (of the outer enumeration).  Let us repeat exactly
		the same nested enumeration as in item~\ref{NestBull-2}.
		\begin{bullenum}
			\renewcommand*{\counterlargebullet}{\textperiodcentered}
			\renewcommand*{\countersmallbullet}{\textperiodcentered}
			\item\label{NestBull-3.1}
				First sub-item.
			\item\label{NestBull-3.2}
				Second sub-item.
			\item\label{NestBull-3.3}
				Third sub-item.
			\item\label{NestBull-3.4}
				Fourth sub-item.  OK, this should suffice.
		\end{bullenum}
		Back to item~\ref{NestBull-3} of the outer enumeration.
\end{bullenum}

Now the bad news: you might want to refer to, say, item
\ref{NestBull-3.4} under item~\ref{NestBull-3}.  As you can see, the
reference comes out wrong, since the label from the outer enumeration
% ( paren match
is typeset with the style of the inner one.~:-(\spacefactor=\sfcode`.
\space This is a known bug: I~chose not to correct it, since doing so
would require to introduce the concept of bullet styles distinguished
by level, which I~deem intolerably burdensome.  Be aware of this
limitation when you nest \beenv\ environments inside one another (this
does not apply, however, to \beenv's nested within \enenv's, as we'll
see in the following examples).



\subsubsection{Nesting \beenv\ inside \enenv}
\label{SSS:NestBullEnum}

Let us repeat \hyref{SSS:NestBullBull}{the same example shown above},
but changing the outer environment to \enenv, and suppressing style
variations.

\begin{enumerate}
	\item
		First item: let this one alone, just to have a term of
		comparison.

	\item\label{NestBuEn-2}
		Second item.  Here we nest a \beenv\ environment:
		\begin{bullenum}
			\item\label{NestBuEn-2.1}
				First sub-item.
			\item\label{NestBuEn-2.2}
				Second sub-item.
			\item\label{NestBuEn-2.3}
				Third sub-item.
			\item\label{NestBuEn-2.4}
				Fourth sub-item.  OK, this should suffice.
		\end{bullenum}
		Don't forget, we are still inside item~\ref{NestBuEn-2} of
		the outer enumeration.

	\item\label{NestBuEn-3}
		Third item (of the outer enumeration).  Let us repeat exactly
		the same nested enumeration as in item~\ref{NestBuEn-2}.
		\begin{bullenum}
			\item\label{NestBuEn-3.1}
				First sub-item.
			\item\label{NestBuEn-3.2}
				Second sub-item.
			\item\label{NestBuEn-3.3}
				Third sub-item.
			\item\label{NestBuEn-3.4}
				Fourth sub-item.  OK, this should suffice.
		\end{bullenum}
		Back to item~\ref{NestBuEn-3} of the outer enumeration.
\end{enumerate}

See Subsection~\ref{SS:CrossXmp} for examples of cross-references to
the items of these lists.



\subsubsection{Nesting \enenv\ inside \beenv}
\label{SSS:NestEnumBull}

As \hyref{SSS:NestBullEnum}{before}, but the other way around: \beenv\
is now the outer environment, and \enenv\ the inner one.

\begin{bullenum}
	\item
		First item: let this one alone, just to have a term of
		comparison.

	\item\label{NestEnBu-2}
		Second item.  Here we nest an \enenv\ environment:
		\begin{enumerate}
			\item\label{NestEnBu-2.1}
				First sub-item.
			\item\label{NestEnBu-2.2}
				Second sub-item.
			\item\label{NestEnBu-2.3}
				Third sub-item.
			\item\label{NestEnBu-2.4}
				Fourth sub-item.  OK, this should suffice.
		\end{enumerate}
		Don't forget, we are still inside item~\ref{NestEnBu-2} of
		the outer enumeration.

	\item\label{NestEnBu-3}
		Third item (of the outer enumeration).  Let us repeat exactly
		the same nested enumeration as in item~\ref{NestEnBu-2}.
		\begin{enumerate}
			\item\label{NestEnBu-3.1}
				First sub-item.
			\item\label{NestEnBu-3.2}
				Second sub-item.
			\item\label{NestEnBu-3.3}
				Third sub-item.
			\item\label{NestEnBu-3.4}
				Fourth sub-item.  OK, this should suffice.
		\end{enumerate}
		Back to item~\ref{NestEnBu-3} of the outer enumeration.
\end{bullenum}

Again, see Subsection~\ref{SS:CrossXmp} for examples of
cross-references to the items of the above lists.



\subsubsection{Deep nesting}
\label{SSS:DeepNest}

Just to show what the output looks like, here is an example of \enenv\
and \beenv\ environments nested down to the fourth level.  We avoid,
however, nesting \beenv\ environments one inside the other, because of
the problem with cross-references outlined
under~\ref{SSS:NestBullBull}.

\begin{enumerate}
	\item\label{Invertebrates}
		Invertebrates are of many kinds.

	\item\label{Vertebrates}
		We detail the vertebrates.
		\begin{enumerate}
			\item\label{Fishes}
				Fishes.  They include, for instance:
				\begin{bullenum}
					\item\label{Barracuda}  barracuda;
					\item\label{Carp}  carp;
					\item\label{Swordfish}  swordfish.
				\end{bullenum}
			\item\label{Amphibians}
				Amphibians.  They include, for instance:
				\begin{bullenum}
					\item\label{Frogs}  frogs, among which are
						\begin{enumerate}
							\item\label{BarkingFrog}  the barking frog,
							\item\label{FlyingFrog}  the flying frog,
							\item\label{GreenFrog}  the green frog,
							\item\label{TreeFrog}  the tree frog,
						\end{enumerate}
						and may others;
					\item\label{Toads}  toads, that comprise
						\begin{enumerate}
							\item\label{CaneToad}  the cane toad,
							\item\label{GiantToad}  the giant toad,
						\end{enumerate}
						just to name a few;
					\item\label{Salamanders}  salamanders, \emph{e.g.},
						\begin{enumerate}
							\item\label{GreenSal}  the green salamander,
							\item\label{SlimySal}  the slimy salamander,
							\item\label{TigerSal}  the tiger salamander.
						\end{enumerate}
				\end{bullenum}
			\item\label{Reptiles}
				Reptiles. Among these we find:
				\begin{bullenum}
					\item\label{Snakes}  snakes, for instance
						\begin{enumerate}
							\item\label{Cobra}  cobra,
							\item\label{Python}  python,
							\item\label{Anaconda}  anaconda,
							\item\label{Viper}  viper,
							\item\label{Rattlesnake}  rattlesnake;
						\end{enumerate}
					\item\label{Turtles}  turtles, of which the following
						\begin{enumerate}
							\item\label{GreenTurtle}  green turtle,
							\item\label{PaintedTurtle}  painted turtle,
							\item\label{SnappingTurtle}  snapping turtle
						\end{enumerate}
						are just a few examples.
				\end{bullenum}
			\item\label{Birds}
				Birds.  (I'm a bit bored, now.)
			\item\label{Mammals}
				Mammals.
		\end{enumerate}
\end{enumerate}



\subsection{An example of use in the middle of text}

\newcounter{mytest}
\setcounter{mytest}{3}
\newlength{\W}
\setlength{\W}{\bigskipamount}

We have just defined, in the source file of this document, a new
counter named \cnt{mytest}, and we have set the value of the
\cnt{mytest} counter to~\arabic{mytest}.  With the default bullet
sizes, this value is displayed by the command \verb|\bullcntr{mytest}|
as~\bullcntr{mytest}.  Other examples:

\begin{tabbing}
	\indent\verb|\footnotesize \bullcntr{mytest}|\hspace{2\tabbingsep}\=\kill
	\> \verb|\Huge \bullcntr{mytest}|
		\' {\Huge produces~\bullcntr{mytest}}  \\[\W]
	\> \verb|\huge \bullcntr{mytest}|
		\' {\huge produces~\bullcntr{mytest}}  \\[\W]
	\> \verb|\LARGE \bullcntr{mytest}|
		\' {\LARGE produces~\bullcntr{mytest}}  \\[\W]
	\> \verb|\Large \bullcntr{mytest}|
		\' {\Large produces~\bullcntr{mytest}}  \\[\W]
	\> \verb|\large \bullcntr{mytest}|
		\' {\large produces~\bullcntr{mytest}}  \\[\W]
	\> \verb|\normalsize \bullcntr{mytest}|
		\' {\normalsize produces~\bullcntr{mytest}}  \\[\W]
	\> \verb|\small \bullcntr{mytest}|
		\' {\small produces~\bullcntr{mytest}}  \\[\W]
	\> \verb|\footnotesize \bullcntr{mytest}|
		\' {\footnotesize produces~\bullcntr{mytest}}  \\[\W]
	\> \verb|\scriptsize \bullcntr{mytest}|
		\' {\scriptsize produces~\bullcntr{mytest}}  \\[\W]
	\> \verb|\tiny \bullcntr{mytest}|
		\' {\tiny produces~\bullcntr{mytest}}
\end{tabbing}

\setcounter{mytest}{7}

Let us repeat this example after assigning to the \cnt{mytest} counter
the value~\arabic{mytest}, which \verb|\bullcntr{mytest}| displays
as~\bullcntr{mytest}:

\begin{tabbing}
	\indent\verb|\footnotesize \bullcntr{mytest}|\hspace{2\tabbingsep}\=\kill
	\> \verb|\Huge \bullcntr{mytest}|
		\' {\Huge produces~\bullcntr{mytest}}  \\[\W]
	\> \verb|\huge \bullcntr{mytest}|
		\' {\huge produces~\bullcntr{mytest}}  \\[\W]
	\> \verb|\LARGE \bullcntr{mytest}|
		\' {\LARGE produces~\bullcntr{mytest}}  \\[\W]
	\> \verb|\Large \bullcntr{mytest}|
		\' {\Large produces~\bullcntr{mytest}}  \\[\W]
	\> \verb|\large \bullcntr{mytest}|
		\' {\large produces~\bullcntr{mytest}}  \\[\W]
	\> \verb|\normalsize \bullcntr{mytest}|
		\' {\normalsize produces~\bullcntr{mytest}}  \\[\W]
	\> \verb|\small \bullcntr{mytest}|
		\' {\small produces~\bullcntr{mytest}}  \\[\W]
	\> \verb|\footnotesize \bullcntr{mytest}|
		\' {\footnotesize produces~\bullcntr{mytest}}  \\[\W]
	\> \verb|\scriptsize \bullcntr{mytest}|
		\' {\scriptsize produces~\bullcntr{mytest}}  \\[\W]
	\> \verb|\tiny \bullcntr{mytest}|
		\' {\tiny produces~\bullcntr{mytest} (Ugh!)}
\end{tabbing}

But look at the output of
%
\begin{verbatim}
\begin{quote}
    \renewcommand*{\countersmallbullet}{\textperiodcentered}
    \tiny
    The value is~\bullcntr{mytest}.
\end{quote}
\end{verbatim}
%
Here it is:
%
\begin{quote}
    \renewcommand*{\countersmallbullet}{\textperiodcentered}
    \tiny
    The value is~\bullcntr{mytest}.
\end{quote}
%
% ( paren match
How cute it looks!~:-)



\subsection{Examples of cross-references}
\label{SS:CrossXmp}

Cross-references are printed using the bullet style that was in
effect when the \verb|\label| command was given.  The following
examples illustrate this.

In~\ref{SSS:Xmp-Standard} we had references to~\ref{Standard-3} and
to~\ref{Standard-6}.  In~\ref{SSS:Xmp-Hearts}, on the other hand, the
references were to~\ref{Hearts-1}, to~\ref{Hearts-3},
to~\ref{Hearts-4}, to~\ref{Hearts-5}, to~\ref{Hearts-7}, and
to~\ref{Hearts-8}.

We turn now to nested lists.  We let alone the example
of~\ref{SSS:NestBullBull}, we already know that references come out
wrong in this case.  In~\ref{SSS:NestBullEnum} we had items
\ref{NestBuEn-2.1}, \ref{NestBuEn-2.2}, \ref{NestBuEn-2.3}, and
\ref{NestBuEn-2.4}, all listed under item~\ref{NestBuEn-2}, and
similarly items \ref{NestBuEn-3.1}, \ref{NestBuEn-3.2},
\ref{NestBuEn-3.3}, and \ref{NestBuEn-3.4}, as part of
item~\ref{NestBuEn-3}.  In~\ref{SSS:NestEnumBull}, on the other hand,
we had items \ref{NestEnBu-2.1}, \ref{NestEnBu-2.2},
\ref{NestEnBu-2.3}, and \ref{NestEnBu-2.4}, all listed under
item~\ref{NestEnBu-2}: as you can see, the bullets now precede the
reference to the item of \enenv; similarly, items \ref{NestEnBu-3.1},
\ref{NestEnBu-3.2}, \ref{NestEnBu-3.3}, and \ref{NestEnBu-3.4} are now
a part of the ``bulletted'' item~\ref{NestEnBu-3}.

Coming to~\ref{SSS:DeepNest} (the animal kingdom), we see that, for
instance, the green frog is \ref{GreenFrog}, an example of frog (under
\ref{Frogs}), which, in turn, is an example of amphibian
(see~\ref{Amphibians}).  On the other hand, the tiger salamander is a
kind of salamander (cf.~\ref{Salamanders}) listed under
\ref{TigerSal}.

And so on\ldots



\setcounter{secnumdepth}{0}

\begin{thebibliography}{9}
	\providecommand*{\bysame}{\leavevmode\hbox to3em{\hrulefill}\thinspace}
	\addcontentsline{toc}{section}{\refname}

	\bibitem{LaTeXbook}
		L.~Lamport, \emph{\LaTeX---A Document Preparation
		System---User's Guide and Reference Manual}, 2nd~edition,
		Addison-Wesley, Reading, 1994.

\end{thebibliography}

\end{document}
