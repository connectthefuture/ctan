\opt{text}{\skbheading{Emphasising Text}}

\DescribeMacro{\skbem}
Highlighting or emphasising text is an important aspect of many technical documents. One can use
\LaTeX~macros directly to set text in italic or bold. This has the disadvantage that there is no
meaningful information given as on why that text is treated in a special way. Furthermore, when
the editor requires to change certain highlights, it will be very difficult to go through a large
document and figure out which text is to be changed.

To prevent that from happening, one can use \LaTeX~macros to actually distignguish between different
highlighted text. A simple start is provided by the \skbacft{A3DS:SKB}. It is simply because, at the moment, it 
only supports three different ways and no furhter meaningful information. But it is a start.

The macro \cmd{\skbem} comes with three different options. The option \skbem[code]{bold} will set the 
text given in the argument in bold face. Similar, the option \skbem[code]{italic} will set it italic.
Last not least, the option \skbem[code]{code} will use another \skbacft{A3DS:SKB} macro (\cmd{\skbcode}) for typesetting
the argument text. \opt{note}{This slide shows a few examples along with the resulting type setting}%
\opt{text}{The following code shows some examples for the macro: \lstinputlisting[style=generic,language=TeX]{\skbfileroot{examples/skbem}}}%
\opt{text}{And here the tinal type setting of that example:

\skbinput{examples/skbem}}%

\opt{note}{\skbheading{skbcode}}
\DescribeMacro{\skbcode}
This macro \cmd{\skbcode} is a facade for calling the macro \cmd{lstinline} from the listing package with a basic style that 
uses type writer font (ttfamily).