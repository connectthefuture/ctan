\documentclass[%
doctor,      % тип документа
%natbib,      % использовать пакет natbib для "сжатия" цитирований
subf,        % использовать пакет subcaption для вложенной нумерации рисунков
href,        % использовать пакет hyperref для создания гиперссылок
colorlinks,  % цветные гиперссылки
%fixint,     % включить прямые знаки интегралов
%classified, % гриф секретности
%facsimile,  % отображать факсимиле диссертанта
]{disser}

\usepackage[
  a4paper, mag=1000,
  left=2.5cm, right=1cm, top=2cm, bottom=2cm, headsep=0.7cm, footskip=1cm
]{geometry}

\usepackage[intlimits]{amsmath}
\usepackage{amssymb,amsfonts}

\usepackage[T2A]{fontenc}
\usepackage[utf8]{inputenc}
\usepackage[english,russian]{babel}
\ifpdf\usepackage{epstopdf}\fi
\usepackage[autostyle]{csquotes}

% Список сокращений и условных обозначений
\usepackage[intoc,nocfg,russian]{nomencl}
\newcommand{\nomencl}[2]{#1 --- #2\nomenclature{#1}{#2}}
\setlength{\nomlabelwidth}{3em}
\setlength{\nomitemsep}{-\parsep}
\renewcommand{\nomlabel}[1]{#1 ---}
\makenomenclature

% Шрифт Times в тексте как основной
%\usepackage{tempora}
% альтернативный пакет из дистрибутива TeX Live
%\usepackage{cyrtimes}

% Шрифт Times в формулах как основной
%\usepackage[varg,cmbraces,cmintegrals]{newtxmath}
% альтернативный пакет
%\usepackage[subscriptcorrection,nofontinfo]{mtpro2}

\usepackage[style=gost-numeric,
  backend=biber,
  language=auto,
  hyperref=auto,
  autolang=other,
  sorting=none
]{biblatex}

\addbibresource{thesis.bib}

% Номера страниц снизу и по центру
%\pagestyle{footcenter}
%\chapterpagestyle{footcenter}

% Точка с запятой в качестве разделителя между номерами цитирований
%\setcitestyle{semicolon}

% Ссылки на работы соискателя включаются в общий список литературы
\let\citeown=\cite

% Использовать полужирное начертание для векторов
\let\vec=\mathbf

% Путь к файлам с иллюстрациями
\graphicspath{{fig/}}

\begin{document}

% Переопределение стандартных заголовков
%\def\contentsname{Содержание}
%\def\conclusionname{Выводы}
%\def\bibname{Литература}

% Включение файла с общим текстом диссертации и автореферата
% (текст титульного листа и характеристика работы).
% Общие поля титульного листа диссертации и автореферата
\institution{Название организации}

\topic{Тема диссертации}

\author{ФИО автора}

\specnum{01.04.05}
\spec{Оптика}
%\specsndnum{01.04.07}
%\specsnd{Физика конденсированного состояния}

\scon{ФИО консультанта}
\sconstatus{д.~ф.-м.~н., проф.}
%\sconsnd{ФИО второго консультанта}
%\sconsndstatus{д.~ф.-м.~н., проф.}

\city{Санкт-Петербург}
\date{\number\year}

% Общие разделы автореферата и диссертации
\mkcommonsect{actuality}{Актуальность темы исследования.}{%
Текст об актуальности. Ссылка~\cite{Yoffe_1993_AP_42_173}.
}

\mkcommonsect{development}{Степень разработанности темы исследования.}{
Текст о степени разработанности темы.
}

\mkcommonsect{objective}{Цели и задачи диссертационной работы:}{%
Список целей.

Для достижения поставленных целей были решены следующие задачи:
}

\mkcommonsect{novelty}{Научная новизна.}{%
Текст о новизне.
}

\mkcommonsect{value}{Теоретическая и практическая значимость.}{%
Результаты, изложенные в диссертации, могут быть использованы для ...
}

\mkcommonsect{methods}{Методология и методы исследования.}{%
Текст о методах исследования.
}

\mkcommonsect{results}{Положения, выносимые на защиту:}{%
Текст о положениях и результатах.
}

\mkcommonsect{approbation}{Степень достоверности и апробация результатов.}{%
Основные результаты диссертации докладывались на следующих конференциях:
}

\mkcommonsect{pub}{Публикации.}{%
Материалы диссертации опубликованы в $N$ печатных работах, из них $n_1$
статей в рецензируемых журналах~\cite{Ivanov_1999_Journal_17_173,
Petrov_2001_Journal_23_12321,Sidorov_2002_Journal_32_1531}, $n_2$ статей в
сборниках трудов конференций и $n_3$ тезисов докладов.
}

\mkcommonsect{contrib}{Личный вклад автора.}{%
Содержание диссертации и основные положения, выносимые на защиту, отражают персональный вклад автора в опубликованные работы.
Подготовка к публикации полученных результатов проводилась совместно с соавторами, причем вклад диссертанта был определяющим. Все представленные в диссертации результаты получены лично автором.
}

\mkcommonsect{struct}{Структура и объем диссертации.}{%
Диссертация состоит из введения, обзора литературы, $n$ глав, заключения и библиографии.
Общий объем диссертации $P$ страниц, из них $p_1$ страницы текста, включая $f$ рисунков.
Библиография включает $B$ наименований на $p_2$ страницах.
}


% номер копии для грифа секретности
%\copynum{1}
% класс доступа
%\classlabel{Для служебного пользования}

% номер УДК
\libcatnum{12345}

\title{ДИССЕРТАЦИЯ\\
на соискание ученой степени\\
доктора физико-математических наук}

\maketitle

%%
%% Titlepage in English
%%
%
%\institution{Name of Organization}
%
%\title{Doctoral Dissertation}
%
%% Topic
%\topic{Dummy Title}
%
%% Author
%\author{Author's Name}
%
%\specnum{01.04.05}
%\spec{Optics}
%
%%\specsndnum{01.04.07}
%%\specsnd{Condensed matter physics}
%
%% Scientific consultants
%\scon{B.\,B.~Baranov}
%\sconstatus{Professor}
%%\sconsnd{P.\,P.~Petrov}
%%\sconsndstatus{Professor}
%
%% City & Year
%\city{Saint Petersburg}
%\date{\number\year}
%
%\maketitle[en]

% Содержание
\tableofcontents

% Введение
\intro

%
% Используемые далее команды определяются в файле common.tex.
%

% Актуальность работы
\actualitysection
\actualitytext

% Степень разработанности темы исследования
\developmentsection
\developmenttext

% Цели и задачи диссертационной работы
\objectivesection
\objectivetext

% Научная новизна
\noveltysection
\noveltytext

% Теоретическая и практическая значимость
\valuesection
\valuetext

% Методология и методы исследования
\methodssection
\methodstext

% Результаты и положения, выносимые на защиту
\resultssection
\resultstext

% Степень достоверности и апробация результатов
\approbationsection
\approbationtext

% Публикации
\pubsection
\pubtext

% Личный вклад автора
\contribsection
\contribtext

% Структура и объем диссертации
\structsection
\structtext


% Обзор литературы
%\review


% Основная часть
%% Глава 1
\chapter{Название главы}
\section{Название секции}

Внутритекстовая формула $\frac{1}{\epsilon^*}=\frac{1}{\epsilon_\infty}-\frac{1}{\epsilon_0}$.
\nomenclature{$\epsilon_\infty$}{высокочастотная диэлектрическая проницаемость}
\nomenclature{$\epsilon_0$}{статическая диэлектрическая проницаемость}
Внутритекстовая формула в стиле выделенной $\dfrac{1}{\epsilon_\infty}$.
Ссылки на литературу~\cite{Yoffe_1993_AP_42_173,Efros_1982_FTP_16_7_1209,%
Anselm_1978,Segall_1968,Agranovich_1983,InP,Mishchenko_1996,Skvortsov_2008,%
Perelman_2003_math:0307245,Nielsen_2010_1006.2735,patent1,patent2}.
Ссылка на формулу~\eqref{e:Coulomb}
\begin{equation}\label{e:Coulomb}
  \frac{1}{|\vec r_1 - \vec r_2|} =
  4\pi \int \frac{d^3 q}{(2\pi)^3}\,
  \frac{e^{i\vec q(\vec r_1 - \vec r_2)}}{q^2},
\end{equation}
где \nomencl{$\vec r_i$}{координата $i$-й частицы}.

\section{Выводы к первой главе}
%% Глава 2
%\input{2}

% Заключение
\conclusion


% Список сокращений и условных обозначений
\printnomenclature

% Словарь терминов
%% Словарь терминов
\dict

\textbf{Термин} "--- определение.


% Список литературы
\printbibliography[heading=bibintoc]

% Список иллюстративного материала
%\listoffigures

% Приложения
%\appendix
%\chapter{Название приложения}


\end{document}
